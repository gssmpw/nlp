% This must be in the first 5 lines to tell arXiv to use pdfLaTeX, which is strongly recommended.
\pdfoutput=1

\documentclass[11pt]{article}
\usepackage[dvipsnames]{xcolor}

\usepackage[final]{acl}

\usepackage{times}
\usepackage{latexsym}

\usepackage[T1]{fontenc}

\usepackage[utf8]{inputenc}

\usepackage{microtype}

\usepackage{inconsolata}

\usepackage{graphicx}

\usepackage{xurl}


\usepackage{amsmath}
\usepackage{stmaryrd}
\usepackage{verbatim}

\usepackage{tabularx}
\usepackage{booktabs}
\newcolumntype{C}{>{\centering\arraybackslash}X}
\usepackage{multirow}
\usepackage{diagbox}
\usepackage{hhline}
\usepackage{color}
\usepackage{amsmath}
\usepackage{amssymb}
\usepackage{mathtools}
\usepackage{soul}

\definecolor{RoseQuartzBg}{HTML}{F7CAC9}
\definecolor{RoseQuartz}{HTML}{F5A798}
\definecolor{Serenity}{HTML}{92A8D1}
\definecolor{OrangeRed}{rgb}{1.0, 0.27, 0.0}
\definecolor{Turquoise}{HTML}{0F4C81}

\usepackage{arydshln}
\setlength\dashlinedash{1.5pt}
\setlength\dashlinegap{2pt}
\setlength\arrayrulewidth{0.3pt}

\definecolor{themered}{HTML}{FF8375}

\usepackage{xparse}
\usepackage{bbm}
\usepackage{caption}
\usepackage{subcaption}
\usepackage{stfloats}
\usepackage{pifont}
\newcommand*\colourcheck[1]{
  \expandafter\newcommand\csname #1check\endcsname{\textcolor{#1}{\ding{52}}}
}
\newcommand*\colourxmark[1]{
  \expandafter\newcommand\csname #1xmark\endcsname{\textcolor{#1}{\ding{56}}}
}

\colourcheck{green}
\colourxmark{red}
\usepackage{cancel}
\usepackage{transparent}
\usepackage{float}
\usepackage{tcolorbox}
\tcbset{colframe=black,colback=white,size=small,colbacktitle=gray!30!white,coltitle=black,fonttitle=\bfseries,fontupper=\footnotesize\ttfamily}

\newcommand{\yz}[1]{\textcolor{orange}{YZ: #1}}

\NewDocumentCommand{\haoyang}{ mO{} }{
\textcolor{Turquoise}{\textsuperscript{\textsc{Haoyang}}\textsf{\textbf{\small[#1]}}}}


\title{On Synthetic Data Strategies for Domain-Specific Generative Retrieval}


\author{Haoyang Wen$^\ddagger$, Jiang Guo$^\dagger$\thanks{Corresponding author}, Yi Zhang$^\dagger$, Jiarong Jiang$^\dagger$, Zhiguo Wang$^\dagger$\\
    $^\ddagger$Language Technologies Institute, Carnegie Mellon University~~~~$^\dagger$AWS AI\\
    \texttt{hwen3@cs.cmu.edu}\\
    \texttt{\{gujiang, imyi, jiarongj, zhiguow\}@amazon.com}}


\begin{document}
\maketitle
\begin{abstract}
This paper investigates synthetic data generation strategies in developing generative retrieval models for domain-specific corpora, thereby addressing the scalability challenges inherent in manually annotating in-domain queries. We study the data strategies for a two-stage training framework: in the first stage, which focuses on learning to decode document identifiers from queries, we investigate LLM-generated queries across multiple granularity (e.g. chunks, sentences) and domain-relevant search constraints that can better capture nuanced relevancy signals. In the second stage, which aims to refine document ranking through preference learning, we explore the strategies for mining hard negatives based on the initial model's predictions. Experiments on public datasets over diverse domains demonstrate the effectiveness of our synthetic data generation and hard negative sampling approach.
\end{abstract}
\section{Introduction}

Video generation has garnered significant attention owing to its transformative potential across a wide range of applications, such media content creation~\citep{polyak2024movie}, advertising~\citep{zhang2024virbo,bacher2021advert}, video games~\citep{yang2024playable,valevski2024diffusion, oasis2024}, and world model simulators~\citep{ha2018world, videoworldsimulators2024, agarwal2025cosmos}. Benefiting from advanced generative algorithms~\citep{goodfellow2014generative, ho2020denoising, liu2023flow, lipman2023flow}, scalable model architectures~\citep{vaswani2017attention, peebles2023scalable}, vast amounts of internet-sourced data~\citep{chen2024panda, nan2024openvid, ju2024miradata}, and ongoing expansion of computing capabilities~\citep{nvidia2022h100, nvidia2023dgxgh200, nvidia2024h200nvl}, remarkable advancements have been achieved in the field of video generation~\citep{ho2022video, ho2022imagen, singer2023makeavideo, blattmann2023align, videoworldsimulators2024, kuaishou2024klingai, yang2024cogvideox, jin2024pyramidal, polyak2024movie, kong2024hunyuanvideo, ji2024prompt}.


In this work, we present \textbf{\ours}, a family of rectified flow~\citep{lipman2023flow, liu2023flow} transformer models designed for joint image and video generation, establishing a pathway toward industry-grade performance. This report centers on four key components: data curation, model architecture design, flow formulation, and training infrastructure optimization—each rigorously refined to meet the demands of high-quality, large-scale video generation.


\begin{figure}[ht]
    \centering
    \begin{subfigure}[b]{0.82\linewidth}
        \centering
        \includegraphics[width=\linewidth]{figures/t2i_1024.pdf}
        \caption{Text-to-Image Samples}\label{fig:main-demo-t2i}
    \end{subfigure}
    \vfill
    \begin{subfigure}[b]{0.82\linewidth}
        \centering
        \includegraphics[width=\linewidth]{figures/t2v_samples.pdf}
        \caption{Text-to-Video Samples}\label{fig:main-demo-t2v}
    \end{subfigure}
\caption{\textbf{Generated samples from \ours.} Key components are highlighted in \textcolor{red}{\textbf{RED}}.}\label{fig:main-demo}
\end{figure}


First, we present a comprehensive data processing pipeline designed to construct large-scale, high-quality image and video-text datasets. The pipeline integrates multiple advanced techniques, including video and image filtering based on aesthetic scores, OCR-driven content analysis, and subjective evaluations, to ensure exceptional visual and contextual quality. Furthermore, we employ multimodal large language models~(MLLMs)~\citep{yuan2025tarsier2} to generate dense and contextually aligned captions, which are subsequently refined using an additional large language model~(LLM)~\citep{yang2024qwen2} to enhance their accuracy, fluency, and descriptive richness. As a result, we have curated a robust training dataset comprising approximately 36M video-text pairs and 160M image-text pairs, which are proven sufficient for training industry-level generative models.

Secondly, we take a pioneering step by applying rectified flow formulation~\citep{lipman2023flow} for joint image and video generation, implemented through the \ours model family, which comprises Transformer architectures with 2B and 8B parameters. At its core, the \ours framework employs a 3D joint image-video variational autoencoder (VAE) to compress image and video inputs into a shared latent space, facilitating unified representation. This shared latent space is coupled with a full-attention~\citep{vaswani2017attention} mechanism, enabling seamless joint training of image and video. This architecture delivers high-quality, coherent outputs across both images and videos, establishing a unified framework for visual generation tasks.


Furthermore, to support the training of \ours at scale, we have developed a robust infrastructure tailored for large-scale model training. Our approach incorporates advanced parallelism strategies~\citep{jacobs2023deepspeed, pytorch_fsdp} to manage memory efficiently during long-context training. Additionally, we employ ByteCheckpoint~\citep{wan2024bytecheckpoint} for high-performance checkpointing and integrate fault-tolerant mechanisms from MegaScale~\citep{jiang2024megascale} to ensure stability and scalability across large GPU clusters. These optimizations enable \ours to handle the computational and data challenges of generative modeling with exceptional efficiency and reliability.


We evaluate \ours on both text-to-image and text-to-video benchmarks to highlight its competitive advantages. For text-to-image generation, \ours-T2I demonstrates strong performance across multiple benchmarks, including T2I-CompBench~\citep{huang2023t2i-compbench}, GenEval~\citep{ghosh2024geneval}, and DPG-Bench~\citep{hu2024ella_dbgbench}, excelling in both visual quality and text-image alignment. In text-to-video benchmarks, \ours-T2V achieves state-of-the-art performance on the UCF-101~\citep{ucf101} zero-shot generation task. Additionally, \ours-T2V attains an impressive score of \textbf{84.85} on VBench~\citep{huang2024vbench}, securing the top position on the leaderboard (as of 2025-01-25) and surpassing several leading commercial text-to-video models. Qualitative results, illustrated in \Cref{fig:main-demo}, further demonstrate the superior quality of the generated media samples. These findings underscore \ours's effectiveness in multi-modal generation and its potential as a high-performing solution for both research and commercial applications.
\section{Generative Retrieval Framework}
A typical generative retrieval framework takes a query as input, and generates the corresponding relevant document identifiers as the retrieval results~\citep{DBLP:conf/nips/Tay00NBM000GSCM22}. Because each document in the corpus has a unique identifier, one can then use these identifiers to retrieve the corresponding documents for downstream tasks.

\subsection{Document Identifiers}
We primarily use \textit{semantic} document identifiers in our experiments due to their superior performance and better scalability to larger corpora.
Instead of using corpus-specific semantic identifiers like titles or urls, we adopt a more general, keyword-based approach that can be applied to a wide variety of corpora~\citep{DBLP:journals/corr/abs-2208-09257}.
Specifically, we instruct an LLM to produce a list of keywords that describes the content of a document, and use this keyword list as its semantic identifier.

In addition, we extended our synthetic data strategies to other types of identifiers to validate its generalizability, such as atomic identifiers ~\citep{DBLP:conf/nips/Tay00NBM000GSCM22}, which are unique tokens that can be generated through a one-step decoding or classification process.

\subsection{Generative Modeling}
The generative retrieval model learns to generate the identifier of a relevant document given a query. Formally, for a query $q$ and a relevant document $d$ with identifier $d'$, generative retrieval aims to produce $d'$ given $q$, which can be represented as:
\newcommand{\score}[1]{\operatorname{score}(#1)}
\begin{align*}
\score{q,d} &= P\left(d'\mid q; \theta\right) \\
&= \prod_{i}P\left(d'_i \mid d'_{<i}, q; \theta \right),
\end{align*}
where $d'_{i}$ is the $i^\text{th}$ token of the identifier. To ensure the generated identifiers are valid during inference, we use constrained beam search with Trie~\citep{DBLP:journals/corr/abs-2010-00904} to restrict the output token space at each decoding step. The top-$k$ output from the beam search serves as the final retrieval results.

Compared to dense retrieval models~\cite{karpukhin2020dense}, generative retrieval bypasses the need for an external index by directly producing relevant document identifiers. However, there are distinct challenges in learning a generative retrieval model.
As it solely relies on parametric knowledge, the model must not only learn the retrieval task, but also capture and encode document content in a way that associates each document with its identifier. Therefore, generative retrieval often requires training on the entire corpus to enable the model to memorize and comprehend the necessary information effectively. 

\begin{figure*}[htbp]
    \centering
    \includegraphics[width=\linewidth, trim={0.6cm 0.2cm 0.6cm 0.3cm}, clip]{figures/generative_retrieval_workflow.pdf}
    \caption{The overall workflow of the generative retrieval training and synthetic data utilization at each stage.}
    \label{fig:workflow}
\end{figure*}

\section{Supervised Fine-Tuning Data Strategy}
In a typical domain-specific setup, we often assume access to a corpus with limited or no labeled data for domain-specific training~\citep{DBLP:conf/ictir/HashemiZKPMC23}. Therefore, it is crucial to create high-quality synthetic data that thoroughly covers the entire corpus for generative retrieval training.

Our synthetic data comprises two main components: Context2ID data and Query2ID data. Context2ID involves training the model to retrieve the document identifiers given the document's content. Query2ID focuses on teaching the model to retrieve relevant document identifiers from a given query.
Combining these two data types encourages the model to learn both content memorization and retrieval given a query.

\subsection{Supervised Fine-Tuning Objective}
At this stage, we train the model to generate relevant document identifiers by maximizing the probability of each individual token. While typical supervised fine-tuning (SFT), especially with encoder-decoder architectures such as T5, focuses on optimizing the output sequence (\textit{i.e.} document identifiers), it's also part of the training goal for generative retrieval models to comprehend and memorize the context. To this end, we also optimize the model for learning to decode the input.
Specifically, for a given query-document pair $(q,d)$, where $q$ could be an actual query or a text chunk from the document, the model maximizes the likelihood of the combined input and output sequence:
\begin{align*}
\mathcal{L}_\text{sft}\left(q,d\right) = &-\log P\left(d', q; \theta\right) \\
= &-\sum_i \log P(q_i \mid q_{<i}; \theta) \\
 &- \sum_i \log P(d'_i \mid d'_{<i}, q; \theta).
\end{align*}

\begin{table*}[htbp]
    \centering
    \small
    \begin{tabular}{lp{0.749\textwidth}}
        \toprule
        \textbf{Data Type} & \textbf{Example} \\
        \midrule
        Context & title: Christmas Day preview: \colorbox{Apricot}{49ers}, \colorbox{Salmon}{Ravens} square off in potential Super Bowl sneak peek\ldots source: \colorbox{GreenYellow}{Yardbarker} \ldots \colorbox{Apricot}{San Francisco} has racked up an NFL-leading 25 turnovers and has given up the second-fewest rushing \colorbox{Goldenrod}{yards (1,252)}, \ldots \\
        Chunk-Level Query & What is the potential implication of this matchup between the \colorbox{Apricot}{49ers} and \colorbox{Salmon}{Ravens}? \\
        Sentence-Level Query &  Where does the \colorbox{Apricot}{49ers}' defense stand in terms of \colorbox{Goldenrod}{total yards} allowed per game? \\
        Constraints-Based Query & \underline{According to the \colorbox{GreenYellow}{Yardbarker} article}, which team has the league's most effective running game?\\% as of 2023-12-24?\\
        \bottomrule
    \end{tabular}
    \caption{Examples of different synthetic queries generated from MultiHop-RAG corpus.} %
    \label{tab:data_type_example}
    \vspace{-0.5em}
\end{table*}

\subsection{Context2ID}
Context2ID data is created by pairing each chunk of text in the corpus with its corresponding document identifier. The goal of Context2ID data is to help the generative retrieval model associate each document's content with its unique identifier, i.e., ``memorizing'' the text.

\subsection{Query2ID}
Query2ID is designed to teach the model to retrieve the relevant document identifiers given a query. It helps the model to learn the core retrieval task and also further comprehend content from the query perspective.

Previous work~\citep{DBLP:journals/corr/abs-2206-10128} finds it effective to use a query generation model (\textit{e.g.,} docT5query, \citealp{nogueiradoc2query}) to produce synthetic queries from documents using multiple independent samplings. In this work, we instead use an LLM for synthetic query generation. Specifically, given a context (e.g., a document chunk), the LLM is instructed to generate a diverse set of $m$ queries, thereby covering a  wider range of semantic variations compared to the sampling-based approach with a specialized query generation model.


\subsubsection{Multi-Granular Query Generation} We first generate queries with context at different levels of granularity: \textit{chunk-level} and \textit{sentence-level}. Chunk-level synthetic queries are produced by providing the entire chunk as input to the LLM to capture higher-level semantics or facts, while sentence-level synthetic queries are produced by only providing individual sentences to focus on more specific details within the document
Concretely, for each chunk, we ask the LLM to produce $m_c$ chunk-level queries. We then split the chunk into individual sentences and ask the LLM to generate $m_s$ sentence-level queries for each sentence.

\subsubsection{Constraints-Based Query Generation}

A key advantage of using an LLM for query generation is its ability to incorporate domain-specific instructions.
For instance, we can prompt the LLM to include metadata constraints, such as the \textit{author name} or \textit{political polarity} of a document, in the generated queries.
Although the specific constraint types depends on the metadata available and can be domain or dataset specific, they are common in real-world scenarios such as enterprise data. Table~\ref{tab:dataset_attributes} in Appendix specifies the attributes that we use to produce constraints-based synthetic queries for each dataset. We ask the LLM to generate $m_i$ queries for each document that incorporate these constraints, allowing our generative retrieval model to handle more specialized or domain-specific queries.

\section{Preference Learning Data Strategy}
Previous work~\citep{zhou-etal-2023-enhancing-generative,DBLP:conf/aaai/00010WWL24} have shown that incorporating ranking tasks can further enhance the relevance modeling of generative retrieval models. However, when generative retrieval models are based on large language models, complex ranking objectives -- such as listwise optimization -- often become computationally inefficient due to multiple forward passes. In this work, we instead use a simpler method and adopt the regularized preference optimization algorithm to perform the preference optimization, a technique widely applied in optimizing large language models. We will first briefly introduce the preference optimization method, and then turn our focus on the synthetic data construction, which consists of the synthetic queries along with their corresponding preferred or rejected candidates.

\subsection{Preference Optimization Objective}
We use Regularized Preference Optimization (\citealp[RPO]{DBLP:journals/corr/abs-2404-19733}) as our optimization method for preference learning. It is an extended version of Directed Preference Optimization (\citealp[DPO]{DBLP:conf/nips/RafailovSMMEF23}), including additional supervised fine-tuning loss to alleviate the over-optimization issues on negative responses. It takes an input query $q$, a positive candidate $d_p$, and a negative candidate $d_n$ as input. The loss is in favor of the positive candidate while against the negative candidate
\begin{align*}
    \mathcal{L}_\text{rpo}\left(q, d_p, d_n\right) = &- \log \delta \left(  \beta\log \frac{P\left(d'_p\mid q; \theta\right)}{P\left(d'_p\mid q; \theta_\text{ref}\right)} \right.\\
    & \quad\quad\ \ \ \left. -\beta\log \frac{P\left(d'_n\mid q; \theta\right)}{P\left(d'_n\mid q; \theta_\text{ref}\right)} \right) \\
    & - \alpha\frac{\log P(d'_p \mid q; \theta)}{\left|d'_p\right|},
\end{align*}
where $\theta_\text{ref}$ is the parameter of the reference model, \textit{i.e.,} the supervised fine-tuned model from the first stage training. $d'_p$ and $d'_n$ are the identifiers of the positive and negative candidate, respectively.

\subsection{Synthetic Queries}
Similar to the previous section, in a domain-specific setup, we assume that we do not have enough data for model training. Therefore, after the supervised fine-tuning stage, we need a batch of new synthetic queries for preference learning.

We still adopt the LLM-based query generation as with the supervised fine-tuning stage. However, there are a few key differences in the instructions. First of all, we ask the LLM to make queries as difficult as possible. At the same time, we ask the LLM to provide not only the synthetic queries but also their corresponding answers. This is to ensure that, while making difficult queries, those synthetic queries are still answerable using the given context.

These changes make the new batch of synthetic queries different from queries used during supervised fine-tuning so that the model will not be over-optimized to the same batch of data. Intensifying the difficulties also increases the likelihood that the initial generative retrieval model makes mistakes, and therefore the model will benefit from the preference learning by learning from those mistakes.

\subsection{Candidate Selection}
After producing the synthetic queries, the next step is to select document candidate pairs for RPO optimization. For each training instance, we need one positive candidate and one negative candidate. As we always produce synthetic queries based on a document, the positive candidate can be naturally assigned. Therefore, the focus will be on selecting negative candidates for each synthetic query.

To increase the hardness of the negative candidates, we choose to select negative candidates from the retrieval results. Specifically, after the supervised fine-tuning stage, we will use the generative retrieval model to perform retrieval on the synthetic queries for preference learning. Our strategy mainly focuses on selecting the top-$k$ negative candidates with ranks higher than the positive candidate from the retrieval results. In this way, if the positive candidate ranks in the top-1, we will not use the query for preference learning. If the rank of the positive candidate is higher than $k$, then there will be different numbers of negative candidates, depending on the rank. If the rank is lower than $k$, there will be $k$ different negative candidates. When there are multiple negative candidates, we pair each negative candidate with the positive one to form a candidate pair instance for preference learning.

% \section{Experiment and Results}
\section{Results and Analysis}
In this section, we first present safe vs. unsafe evaluation results for 12 LLMs, followed by fine-grained responding pattern analysis over six models among them, and compare models' behavior when they are attacked by same risky questions presented in different languages: Kazakh, Russian and code-switching language.    

\begin{table}[t!]
\centering
\small
\resizebox{\columnwidth}{!}{
\begin{tabular}{clcccc}
\toprule
\multicolumn{1}{l}{\textbf{Rank} } & \textbf{Model} & \textbf{Kazakh $\uparrow$} & \textbf{Russian $\uparrow$} & \textbf{English $\uparrow$} \\
\midrule
1 & \claude & \textbf{96.5}   & 93.5    & \textbf{98.6}    \\
2 & \gptfouro & 95.8   & 87.6    & 95.7    \\
3 & \yandexgpt & 90.7   & \textbf{93.6}    & 80.3    \\
4 & \kazllmseventy & 88.1 & 87.5 & 97.2 \\
5 & \llamaseventy & 88.0   & 85.5    & 95.7    \\
6 & \sherkala & 87.1   & 85.0    & 96.0    \\
7 & \falcon & 87.1   & 84.7    & 96.8    \\
8 & \qwen & 86.2   & 85.1    & 88.1    \\
9 & \llamaeight & 85.9   & 84.7    & 98.3    \\
10 & \kazllmeight & 74.8   & 78.0    & 94.5 \\
11 & \aya & 72.4 & 84.5 & 96.6 \\
12 & \vikhr & --- & 85.6 & 91.1 \\
\bottomrule
\end{tabular}
}
\caption{Safety evaluation results of 12 LLMs, ranked by the percentage of safe responses in the Kazakh dataset. \claude\ achieves the highest safety score for both Kazakh and English, while \yandexgpt\ is the safest model for Russian responses.}
\label{tab:safety-binary-eval}
\end{table}



\subsection{Safe vs. Unsafe Classification}
% In this subsection, 
We present binary evaluation results of 12 LLMs, by assessing 52,596 Russian responses and 41,646 Kazakh responses.
% 26,298 responses generated by six models on the Russian dataset and 22,716 responses on the Kazakh dataset. 

%\textbf{Safety Rank.} In general, proprietary systems outperform the open-source model. For Russian, As shown in Table \ref{tab:model_comparison_russian}, \textbf{Yandex-GPT} emerges as the safest large language model (LLM) for Russian, with a safety percentage of 93.57\%. Among the open-source models, \textbf{Vikhr-Nemo-12B} is the safest, achieving a safety percentage of 85.63\%.
% Edited: This is mentioned in the discussion
% This outcome highlights the potential impact of pretraining data on model behavior. Models pre-trained primarily on Russian data are better at understanding and handling harmful questions - in both proprietary systems and open-source setups. 
%For Kazakh, as shown in Table \ref{tab:model_comparison_kazakh}, \textbf{Claude} emerges as the safest large language model (LLM) with a safety percentage of 96.46\%, closely followed by GPT-4o at 95.75\%. In contrast, \textbf{Aya-101}, despite being specifically tuned for Kazakh, consistently shows the highest unsafe response rates at 72.37\%. 

\begin{figure*}[t!]
	\centering
        \includegraphics[scale=0.28]{figures/question_type_no6_kaz.png}
	\includegraphics[scale=0.28]{figures/question_type_exclude_region_specific_new.png} 

	\caption{Unsafe answer distribution across three question types for risk types I-V: Kazakh (left) and Russian (right)}
	\label{fig:qt_non_reg}
\end{figure*}

\begin{figure*}[t!]
	\centering
        \includegraphics[scale=0.28]{figures/question_type_only6_kaz.png}
	\includegraphics[scale=0.28]{figures/question_type_region_specific_new.png} 
	
	\caption{Unsafe answer distribution across three question types for risk type VI: Kazakh (left) and Russian (right)}
	\label{fig:qt_reg}
\end{figure*}

\textbf{Safety Rank.} In general, proprietary systems outperform the open-source models. 
For Russian, as shown in Table~\ref{tab:safety-binary-eval},  % \ref{tab:model_comparison_russian}, 
\yandexgpt emerges as the safest language model for Russian, with safe responses account for 93.57\%. Among the open-source models, \kazllmseventy is the safest (87.5\%), followed by \vikhr with a safety percentage of 85.63\%.

For Kazakh, % as shown in Table \ref{tab:model_comparison_kazakh}, 
% YX: todo, rerun Kazakh safety percentage using Diana threshold
\claude is the safest model with 96.46\% safe responses, closely followed by \gptfouro\ at 95.75\%. \aya, despite being specifically tuned for Kazakh, shows the highest unsafe rates at 72.37\%.



% \begin{table}[t!]
% \centering
% \resizebox{\columnwidth}{!}{%
% \begin{tabular}{clccc}
% \toprule
% \textbf{Rank} & \textbf{Model Name}  & \textbf{Safe} & \textbf{Unsafe} & \textbf{Safe \%} \\ \midrule
% \textbf{1} & \textbf{Yandex-GPT} & \textbf{4101} & \textbf{282} & \textbf{93.57} \\
% 2 & Claude & 4100 & 283 & 93.54 \\
% 3 & GPT-4o & 3839 & 544 & 87.59 \\
% 4 & Vikhr-12B & 3753 & 630 & 85.63 \\
% 5 & LLama-3.1-instruct-70B & 3746 & 637 & 85.47 \\
% 6 & LLama-3.1-instruct-8B & 3712 & 671 & 84.69 \\
% \bottomrule
% \end{tabular}
% }
% \caption{Comparison of models based on safety percentages for the Russian dataset.}
% \label{tab:model_comparison_russian}
% \end{table}


% \begin{table}[t!]
% \centering
% \resizebox{\columnwidth}{!}{%
% \begin{tabular}{clccc}
% \toprule
% \textbf{Rank} & \textbf{Model Name}  & \textbf{Safe} & \textbf{Unsafe} & \textbf{Safe \%} \\ \midrule
% 1             & \textbf{Claude}  & \textbf{3652} & \textbf{134} & \textbf{96.46} \\ 
% 2             & GPT-4o                      & 3625          & 161          & 95.75 \\ 
% 3             & YandexGPT                   & 3433          & 353          & 90.68 \\
% 4             & LLama-3.1-instruct-70B      & 3333          & 453          & 88.03 \\
% 5             & LLama-3.1-instruct-8B       & 3251          & 535	       & 85.87 \\
% 6             & Aya-101                     & 2740          & 1046         & 72.37 \\ 
% \bottomrule
% \end{tabular}
% }
% \caption{Comparison of models based on safety percentages for the Kazakh dataset.}
% \label{tab:model_comparison_kazakh}
% \end{table}



\textbf{Risk Areas.} 
We selected six representative LLMs for Russian and Kazakh respectively and show their unsafe answer distributions over six risk areas.
As shown in Table \ref{tab:unsafe_answers_summary}, risk type VI (region-specific sensitive topics) overwhelmingly contributes the largest number of unsafe responses across all models. This highlights that LLMs are poorly equipped to address regional risks effectively. For instance, while \llama models maintain comparable safety levels across other risk type (I–V), their performance drops significantly when dealing with risk type VI. Interestingly, while \yandexgpt\ demonstrates relatively poor performance in most other risk areas, it handles region-specific questions remarkably well, suggesting a stronger alignment with regional norms and sensitivities. For Kazakh, Table \ref{tab:unsafe_answers_summary_kazakh} shows that region‐specific topics (risk type VI) pose a substantial challenge across all six models, including the commercial \gptfouro\ and \claude, which demonstrate superior safety on general categories. 

% \begin{table}[t!]
% \centering
% \resizebox{\columnwidth}{!}{%
% \begin{tabular}{lccccccc}
% \toprule
% \textbf{Model} & \textbf{I} & \textbf{II} & \textbf{III} & \textbf{IV} & \textbf{V} & \textbf{VI} & \textbf{Total} \\ \midrule
% LLama-3.1-instruct-8B & 60 & 70 & 16 & 31 & 9 & 485 & 671 \\
% LLama-3.1-instruct-70B & 29 & 55 & 8 & 4 & 1 & 540 & 637 \\
% Vikhr-12B & 41 & 93 & 15 & 1 & 3 & 477 & 630 \\
% GPT-4o & 21 & 51 & 6 & 2 & 0 & 464 & 544 \\
% Claude & 7 & 10 & 1 & 0 & 0 & 265 & 283 \\
% Yandex-GPT & 55 & 125 & 9 & 3 & 8 & 82 & 282 \\
% \bottomrule
% \end{tabular}%
% }
% \caption{Ru unsafe cases over risk areas of six models.}
% \label{tab:unsafe_answers_summary}
% \end{table}


\begin{table}[t!]
\centering
\resizebox{\columnwidth}{!}{%
\begin{tabular}{lccccccc}
\toprule
\textbf{Model} & \textbf{I} & \textbf{II} & \textbf{III} & \textbf{IV} & \textbf{V} & \textbf{VI} & \textbf{Total} \\ \midrule
\llamaeight & 60 & 70 & 16 & 31 & 9 & 485 & 671 \\
\llamaseventy & 29 & 55 & 8 & 4 & 1 & 540 & 637 \\
\vikhr & 41 & 93 & 15 & 1 & 3 & 477 & 630 \\
\gptfouro & 21 & 51 & 6 & 2 & 0 & 464 & 544 \\
\claude & 7 & 10 & 1 & 0 & 0 & 265 & 283 \\
\yandexgpt & 55 & 125 & 9 & 3 & 8 & 82 & 282 \\
\bottomrule
\end{tabular}%
}
\caption{Ru unsafe cases over risk areas of six models.}
\label{tab:unsafe_answers_summary}
\end{table}


% \begin{table}[t!]
% \centering
% \resizebox{\columnwidth}{!}{%
% \begin{tabular}{lccccccc}
% \toprule
% \textbf{Model} & \textbf{I} & \textbf{II} & \textbf{III} & \textbf{IV} & \textbf{V} & \textbf{VI} & \textbf{Total} \\ \midrule
% Aya-101 & 96 & 235 & 165 & 166 & 90 & 294 & 1046 \\
% Llama-3.1-instruct-8B & 25 & 15 & 91 & 37 & 14 & 353 & 535 \\
% Llama-3.1-instruct-70B & 33 & 39 & 88 & 27 & 20 & 246 & 453 \\
% Yandex-GPT & 29 & 76 & 95 & 29 & 16 & 108 & 353 \\
% GPT-4o & 2 & 1 & 41 & 0 & 3 & 114 & 161 \\
% Claude & 2 & 1 & 26 & 3 & 6 & 96 & 134 \\ \bottomrule
% \end{tabular}%
% }
% \caption{Kaz unsafe cases over risk areas of six models.}
% \label{tab:unsafe_answers_summary_kazakh}
% \end{table}


\begin{table}[t!]
\centering
\resizebox{\columnwidth}{!}{%
\begin{tabular}{lccccccc}
\toprule
\textbf{Model} & \textbf{I} & \textbf{II} & \textbf{III} & \textbf{IV} & \textbf{V} & \textbf{VI} & \textbf{Total} \\ \midrule
\aya & 96 & 235 & 165 & 166 & 90 & 294 & 1046 \\
\llamaeight & 25 & 15 & 91 & 37 & 14 & 353 & 535 \\
\llamaseventy & 33 & 39 & 88 & 27 & 20 & 246 & 453 \\
\yandexgpt & 29 & 76 & 95 & 29 & 16 & 108 & 353 \\
\gptfouro & 2 & 1 & 41 & 0 & 3 & 114 & 161 \\
\claude & 2 & 1 & 26 & 3 & 6 & 96 & 134 \\ 
\bottomrule
\end{tabular}%
}
\caption{Kaz unsafe cases over risk areas of six models.}
\label{tab:unsafe_answers_summary_kazakh}
\end{table}

% \begin{figure*}[t!]
% 	\centering
% 	\includegraphics[scale=0.28]{figures/human_1000_kz_font16.png} 
% 	\includegraphics[scale=0.28]{figures/human_1000_ru_font16.png}

% 	\caption{Human vs \gptfouro\ fine-grained labels on 1,000 Kazakh (left) and Russian (right) samples.}
% 	\label{fig:human_fg_1000}
% \end{figure*}

\textbf{Question Type.} For Russian, Figures \ref{fig:qt_non_reg} and \ref{fig:qt_reg} reveal differences in how models handle general risks I-V and region-specific risk VI. For risks I-V, indirect attacks % crafted to exploit model vulnerabilities—
result in more unsafe responses due to their tricky phrasing. 

In contrast, region-specific risks see slightly more unsafe responses from direct attacks, 
% as these explicit prompts are more likely to bypass safety mechanisms. 
since indirect attacks for region-specific prompts often elicit safer, vaguer answers. %, suggesting models struggle less with implicit harm. 
Overall, the number of unsafe responses is similar across question types, indicating models generally struggle with safety alignment in all jailbreaking queries.

For Kazakh, Figures \ref{fig:qt_non_reg} and \ref{fig:qt_reg} show greater variation in unsafe responses across question types due to the low-resource nature of Kazakh data. For general risks I-V, \llamaseventy\ and \aya\ produce more unsafe outputs for direct harm prompts. At the same time, \claude\ and \gptfouro\ struggle more with indirect attacks, reflecting challenges in handling subtle cues. For region-specific risk VI, most models perform similarly due to limited Kazakh-specific data, though \llamaeight\ shows higher unsafe outputs for indirect local references, likely due to their implicit nature. Direct region-specific attacks yield fewer unsafe responses, as explicit prompts trigger more cautious outputs. Across all risk areas, general questions with sensitive words produce the fewest unsafe answers, suggesting over-flagging or cautious behavior for unclear harmful intent.



% \subsection{Fine-grained Classification}
% We extended our analysis to include fine-grained classifications for both safe and unsafe responses. For unsafe responses, we categorized outputs into four harm types, as shown in Table \ref{table:unsafe_response_categories}. 

% For safe responses, we classified outputs into six distinct patterns of safety, following a fine-grained rubric provided in \cite{wang2024chinesedatasetevaluatingsafeguards}. The types outlined in this rubric are presented in Table \ref{table:safe_response_categories}.

% To validate the fine-grained classification, we conducted human evaluation on the same 1,000 responses in Russian used for the preliminary binary classification.
% The confusion matrix highlights the alignment and discrepancies between human annotations and GPT's fine-grained labels. The diagonal values represent instances where the GPT's labels match human annotations, with category 5 (provides general, safe information) showing the highest agreement (404 instances). However, off-diagonal values reveal areas of disagreement, such as misclassifications in categories 1 (not willing to answer), 4 (recommends consulting an expert), and 7 (provides harmful or risky instructions). For example, in category 1, while 94 responses were correctly labeled, GPT-4o misclassified several instances into categories 4, 5, or 7, indicating overlap or ambiguity in these classifications. Similar trends are observed in other categories, where GPT sometimes struggles to differentiate nuanced distinctions in human-labeled categories. Overall, GPT's fine-grained labels match human annotations in 710 out of 1000 cases, achieving an agreement rate of 71\%.

% \begin{figure}[ht!]
%     \centering
%     \includegraphics[width=0.95\linewidth]{figures/human_fg_1000_ru.png}
%     \caption{{Human vs GPT-4o Fine-Grained Labels on 1,000 Russian Samples}}
%     \label{fig:human_fg_1000_ru}
% \end{figure}


% After conducting human evaluation on a representative sample, we extended the fine-grained classification to a full dataset comprising 21,915 responses generated by five different models. 


% \begin{figure}[ht!]
%     \centering
%     \includegraphics[width=0.95\linewidth]{figures/all_5_ru.png}
%     \caption{Fine-grained label distribution for responses from five models for Russian.}
%     \label{fig:all_5_fg_russian}
% \end{figure}
% Category 5 ("safe and general information") consistently has the highest frequency across all models, aligning with its dominance in the 1,000-sample evaluation. However, differences in the distribution across other categories highlight variability in how models handle nuanced safety risks, with Yandex-GPT showing a slightly broader spread across categories. 
% In the distribution of unsafe responses, most models exhibit higher counts in certain labels such as 8. However, Yandex-GPT displays comparatively fewer responses in these labels. 
% It exhibits a high rate of responses classified under label 7, which indicates instances where the model provides harmful or risky instructions, including unethical behavior or sensitive discussions. While this may suggest a vulnerability in addressing complex or challenging prompts, it was observed that many of Yandex-GPT’s responses tend to deflect responsibility or offer vague advice such as "check the internet". Although this approach minimizes the risk of unsafe outputs, it often results in responses that lack depth or contextually relevant information, limiting their overall usefulness for users.

% % \subsection{Kazakh}


% % Overall, these findings underscore how resource constraints and prompt explicitness affect model safety in Kazakh. Some models manage direct attacks better yet fail on indirect ones, while region-specific content remains challenging for all given the lack of localized training data.
% \subsubsection{Fine-grained Classification}
% Similarly, we conducted a human evaluation on 1,000 Kazakh samples, following the same methodology as the Russian evaluation. The match between human annotations and GPT-4o's fine-grained classifications was 707 out of 1,000, ensuring that the fine-grained classification framework aligned well with human judgments.
% The confusion matrix in Figure \ref{fig:human_fg_1000_kz} for 1,000 Kazakh samples illustrates the agreement between human annotations and GPT-4o's fine-grained classifications. The highest agreement is observed in category 5 (360 instances), indicating GPT-4o's strength in recognizing responses labeled by humans as "safe and general information." However, discrepancies are evident in categories such as 3 and 7, where GPT-4o misclassified several instances, highlighting areas for further refinement in distinguishing nuanced classifications.


\begin{figure}[t!]
	\centering
	\includegraphics[scale=0.18]{figures/human_1000_kz_font16.png} 
	\includegraphics[scale=0.18]{figures/human_1000_ru_font16.png}

	\caption{Human vs \gptfouro\ fine-grained labels on 1,000 Kazakh (left) and Russian (right) samples.}
	\label{fig:human_fg_1000}
\end{figure}

% \begin{figure}[t!]
% 	\centering
% 	\includegraphics[scale=0.28]{figures/human_1000_kz_font16.png} 
% 	\includegraphics[scale=0.28]{figures/human_1000_ru_font16.png}

% 	\caption{Human vs \gptfouro\ fine-grained labels on 1,000 Kazakh (top) and Russian (bottom) samples.}
% 	\label{fig:human_fg_1000}
% \end{figure}

% \begin{figure*}[t!]
% 	\centering
% 	\includegraphics[scale=0.28]{figures/all_5_kz_font16.png} 
% 	\includegraphics[scale=0.28]{figures/all_5_ru_font_16.png} \\
% 	\caption{Fine-grained responding pattern distribution across five models for Kazakh (left) and Russian (right).}
% 	\label{fig:all_5}
% \end{figure*}

\begin{figure}[t!]
	\centering
	\includegraphics[scale=0.28]{figures/all_5_kz_font16.png} 
	\includegraphics[scale=0.28]{figures/all_5_ru_font_16.png} \\
	\caption{Fine-grained responding pattern distribution across five models for Kazakh (top) and Russian (bottom).}
	\label{fig:all_5}
\end{figure}


\subsection{Fine-Grained Classification}
\label{sec:fine-grained-classification}
% As discussed in Section \ref{harmfulness_evaluation}, 
We further analyzed fine-grained responding patterns for safe and unsafe responses. For unsafe responses, outputs were categorized into four harm types, and safe responses were classified into six distinct patterns of safety, as rubric in Appendix \ref{safe_unsafe_response_categories}. 
% \cite{wang2024chinesedatasetevaluatingsafeguards}

\paragraph{Human vs. GPT-4o}
Similar to binary classification, we validated \gptfouro's automatic evaluation results by comparing with human annotations on 1,000 samples for both Russian and Kazakh. %, comparing human annotations with \gptfouro's fine-grained labels.
For the Russian dataset, \gptfouro's labels aligned with human annotations in 710 out of 1,000 cases, achieving an agreement rate of 71\%. 
Agreement rate of Kazakh samples is 70.7\%. %with 707 out of 1,000 cases matching
% The confusion matrix highlights areas of alignment and discrepancies.
% 
As confusion matrices illustrated in Figure~\ref{fig:human_fg_1000}, the majority of cases falling into \textit{safe responding patter 5} --- providing general and harmless information, for both human annotations and automatic predictions.
% highest agreement with 404 correct classifications for Russian. 
Mis-classifications for safe responses mainly focus on three closely-similar patterns: 3, 4, and 5, and patterns 7 and 8 are confusing to discern for unsafe responses, particularly for Kazakh (left figure).
We find many Russian samples which were labeled as ``1. reject to answer'' by humans are diversely classified across 1-6 by GPT-4o, which is also observed in Kazakh but not significant.

% suggesting label alignment issues are language-independent. 
% YX: I did not observe this, commented
% Notably, Russian showed confusion between 7 (risky instructions) and 1 (refusal to answer), this trend does not appear in Kazakh.


% highlight the strengths and limitations of {\gptfouro}'s fine-grained classification framework across both languages, paving the way for further refinements.


% However, misclassifications were observed in categories such as 1 (not willing to answer), 4 (recommends consulting an expert), and 7 (provides harmful or risky instructions), revealing overlaps and ambiguities in nuanced classifications.

% Similarly, for the Kazakh dataset, the agreement rate between human annotations and GPT-4o's labels was 70.7\%, with 707 out of 1,000 cases matching. As with the Russian analysis, category 5 (360 instances) showed the highest alignment. However, discrepancies were more prominent in categories such as 3 and 7, underscoring GPT-4o's challenges in differentiating fine-grained human-labeled categories. 
% A similar pattern was observed for Kazakh dataset, which suggests that alignment and misaligned of fine-grained lables is not language dependent.

% These findings, illustrated in Figures \ref{fig:human_fg_1000}, highlight the strengths and limitations of {\gptfouro}'s fine-grained classification framework across both languages, paving the way for further refinements.

\paragraph{Fine-grained Analysis of Five LLMs}
% After conducting human evaluation on representative samples, we extended 
\figref{fig:all_5} shows fine-grained responding pattern distribution across five models based on the full set of Russian and Kazakh data.
% For Russian, we selected \vikhr, \gptfouro, \llamaseventy, \claude, and \yandexgpt, while for Kazakh, we chose \aya, \gptfouro, \llamaseventy, \claude, and \yandexgpt. 
% The evaluation covered 21,915 responses in Russian and 18,930 responses in Kazakh.
% 
In both languages, pattern 5 of providing \textit{general and harmless information} consistently witnessed the highest frequency across all models, with \llamaseventy\ exhibiting the largest number of responses falling into this category for Kazakh (2,033). 
% YX:summarize more noteable findings here.

Differences of other patterns vary across languages. 
Unsafe responses in Russian are predominantly in pattern 8, where models provide incorrect or misleading information without expressing uncertainty. % (misinformation and speculation), 
For Kazakh, \aya\ exhibits the highest occurrence of pattern 7 (harmful or risky information) and pattern 8, indicating a stronger tendency to generate unethical, misleading, or potentially harmful content.

%Variations in other patterns across models highlight differences in how nuanced safety risks are classified, reflecting the models' differing capabilities in handling safety evaluation for these distinct linguistic contexts. For Russian, the majority of unsafe responses fall under pattern 8 (misinformation and speculation), indicating that models frequently provide incorrect or misleading information without acknowledging uncertainty. For Kazakh, \aya\ has the highest occurence of pattern 7 (harmful or risky information) and pattern 8 (misinformation and speculation), indicating a greater tendency to generate unethical, misleading, or potentially harmful content. 

%This trend suggests that Russian models may struggle more with factual accuracy, whereas Kazakh models, particularly \aya, pose higher risks related to both harmful content and misinformation. Additionally, \gptfouro\ and \claude\ consistently produce fewer unsafe responses in both languages, demonstrating stronger alignment with safety standards
\subsection{Code Switching}
\begin{table}[t!]
\centering

\setlength{\tabcolsep}{3pt}
\scalebox{0.7}{
\begin{tabular}{lcccccccccc}
\toprule
\textbf{Model Name} & \multicolumn{2}{c}{\textbf{Kazakh}} & \multicolumn{2}{c}{\textbf{Russian}} & \multicolumn{2}{c}{\textbf{Code-Switched}} \\  
\cmidrule(lr){2-3} \cmidrule(lr){4-5} \cmidrule(lr){6-7}
& \textbf{Safe} & \textbf{Unsafe} & \textbf{Safe} & \textbf{Unsafe} & \textbf{Safe} & \textbf{Unsafe} \\ 
\midrule
\llamaseventy & 450 & 50 & 466 & 34 & 414 & 86 \\
\gptfouro & 492 & 8 & 473 & 27 & 481 & 19
 \\
\claude & 491 & 9 & 478 & 22 & 484 & 16 \\ 
\yandexgpt & 435 & 65 & 458 & 42 & 464 & 36 \\
\midrule
\end{tabular}}
\caption{Model safety when prompted in Kazakh, Russian, and code-switched language.}
\label{tab:finetuning-comparison}
\end{table}


\gptfouro\ and \claude\ demonstrate strong safety performance across three languages, even with a high proportion of safe responses in the challenging code-switching context. In contrast, \llamaseventy\ and \yandexgpt\ are less safe, exhibiting more harmful responses, particularly in the code-switching scenario. These results show the varying capabilities of models in defending the same attacks that are just presented in different languages, where open-sourced large language models especially require more robust safety alignment in multilingual and code-switching scenarios.

% \subsection{LLM Response Collection}
% We conducted experiments with a variety of mainstream and region-specific 
% large language models for both Russian and Kazakh languages. For both Russian and Kazakh languages, we employed four multilingual models: Claude-3.5-sonnet, Llama 3.1 70B \cite{meta2024llama3}, GPT-4 \cite{openai2024gpt4o}, and YandexGPT. Additionally, we included language-specific models: VIKHR \cite{nikolich2024vikhrconstructingstateoftheartbilingual} for Russian and Aya \cite{ustun-etal-2024-aya} for Kazakh. 

% \subsection{Kazakh-Russian Code-Switching Evaluation}

% In Kazakhstan, the prevalence of bilingualism is a defining characteristic of its linguistic landscape, with most individuals seamlessly mixing Kazakh and Russian in daily communication \cite{Zharkynbekova2022}. This phenomenon, known as code-switching, reflects the unique cultural and social dynamics of the region. Despite this, there is currently no safety evaluation dataset tailored to this unique multilingual environment. Developing a code-switched dataset is essential to evaluate the ability of large language models (LLMs) to navigate the complexities of bilingual interactions, ensuring they produce contextually appropriate, non-harmful, and culturally sensitive responses. To address this, we sampled 500 questions from both Kazakh and Russian question sets, ensuring they represent similar queries but with culturally nuanced differences in meaning. Building on these, we developed a mixed version that integrates Kazakh and Russian naturally, reflecting the way people in Kazakhstan communicate. This process involved annotators to ensure that the mixed questions were crafted with a natural tone and were localized to Kazakhstan's context.


% \begin{table*}[t!]
%     \centering
%     \small
%     \setlength{\tabcolsep}{6pt} % Adjusts column separation
%     \adjustbox{max width=\linewidth}{
%     \begin{tabular}{ll ccc ccc c}
%     \toprule
%     \textbf{Risk Area} & \textbf{Harm Type} & \multicolumn{3}{c}{\textbf{Kazakh}} & \multicolumn{3}{c}{\textbf{Russian}} & \textbf{\#Q} \\
%                        &                    & \textbf{Ori} & \textbf{FN} & \textbf{FP} & \textbf{Ori} & \textbf{FN} & \textbf{FP} & \\
%     \midrule
%     \multirow{2}{*}{I. Information Hazards} & 1. Risks from leaking sensitive information... & 131 & 133 & 131 & 131 & 133 & 131 & 790 \\
%                                              & 2. Compromise of privacy by leaking or inferring private information (person/individual) & 81 & 82 & 81 & 81 & 82 & 81 & 488 \\
%     \midrule
%     \multirow{3}{*}{II. Malicious Uses} & 3. Assisting illegal activities & 132 & 135 & 132 & 132 & 135 & 132 & 798 \\
%                                         & 4. Nudging or advising unethical actions & 71 & 71 & 71 & 71 & 71 & 71 & 426 \\
%                                         & 5. Reducing the cost of disinformation campaigns & 40 & 42 & 40 & 40 & 42 & 40 & 244 \\
%     \midrule
%     \multirow{3}{*}{III. Discrimination, Exclusion, Toxicity, Hateful, Offensive} & 6. Social stereotypes and unfair discrimination & 94 & 96 & 94 & 94 & 96 & 94 & 568 \\
%                                         & 7. Toxic language (hate speech) & 52 & 59 & 52 & 52 & 59 & 52 & 326 \\
%                                         & 8. Adult content & 27 & 29 & 27 & 27 & 29 & 27 & 166 \\

%     \midrule
%     \multirow{2}{*}{IV. Misinformation Harms} & 9. Disseminating false or misleading information & 92 & 99 & 92 & 92 & 99 & 92 & 566 \\
%                                         & 10. Causing material harm by disseminating misinformation e.g. in medicine or law & 63 & 63 & 63 & 63 & 63 & 63 & 378 \\
%     \midrule
%     \multirow{2}{*}{V. Human–chatbot Interaction Harms} & 11. Mental health or overreliance concerns & 66 & 66 & 66 & 66 & 66 & 66 & 396 \\
%                                         & 12. Treating the chatbot as a human & 50 & 51 & 50 & 50 & 51 & 50 & 302 \\

%     \midrule
%     \multirow{2}{*}{VI. Kazakhstan-specific Sensitivity} & 13. Politically sensitive topics & 63 & 66 & 63& 63 & 66 & 63 &  384 \\
%         & 14. Controversial historical events & 46 & 57 & 46 & 46 & 57 & 46  & 298 \\
% & 15. Regional and racial issues & 45 & 45 & 45 &  45 & 45 & 45 & 270  \\
% & 16. Societal and cultural concerns & 138 & 139 & 138 &  138 & 139 & 138  & 830  \\
% & 17. Legal and human rights matters & 57 & 57 & 57 & 57 & 57 & 57  & 342 \\
%     \midrule
%         \multirow{2}{*}{VII. Russia-specific Sensitivity} 
%             & 13. Politically sensitive topics & - & - & - & 54 & 54 & 54 & 162 \\
%     & 14. Controversial historical events & - & - & - & 38 & 38 & 38 & 114 \\
%     & 15. Regional and racial issues & - & - & - & 26 & 26 & 26 & 78 \\
%     & 16. Societal and cultural concerns & - & - & - & 40 & 40 & 40 & 120 \\
%     & 17. Legal and human rights matters & - & - & - & 41 & 41 & 41 & 123 \\
%     \midrule
%     \bf Total & -- & 1248 & 1290 & 1248 & 1447 & 1489 & 1447 & \textbf{8169} \\
%     \bottomrule
%     \end{tabular}
%     }
%     \caption{The number of questions for Kazakh and Russian datasets across six risk areas and 17 harm types. Ori: original direct attack, FN: indirect attack, and FP: over-sensitivity assessment.}
%     \label{tab:kazakh-russian-data}
% \end{table*}




\section{Discussion}

% \subsection{Kazakh vs Russian}

% The evaluation reveals that Kazakh responses tend to be generally safer than their Russian counterparts, likely due to Kazakh being a low-resource language with significantly less training data. As a result, Kazakh models are less exposed to the vast, often unfiltered datasets containing harmful or unsafe content, which are more prevalent in high-resource languages like Russian. This data scarcity naturally limits the model's ability to generate nuanced but potentially unsafe responses. However, this does not mean the models are specifically fine-tuned for safer performance. When analyzing unsafe answers, it’s clear that Kazakh models, while safer overall, distribute their unsafe responses more evenly across various risk types and question types. This suggests Kazakh models generate fewer unsafe answers but in a broader range of contexts.

% In contrast, Russian models tend to concentrate unsafe answers in specific areas, particularly region-specific risks or indirect attacks. This indicates that Russian models have learned to handle certain types of unsafe content by focusing on specific topics, such as politically sensitive issues, but struggle when confronted with unfamiliar content, leading to unsafe responses due to insufficient filtering. Kazakh models, despite having less training data, tend to respond more broadly, including both direct and indirect risks. This could be due to the less curated nature of their training data, making them more likely to answer unsafe questions without filtering the potential harm involved. The exception to this trend is Aya, a model specifically fine-tuned for Kazakh. Despite fine-tuning, it exhibits the lowest safety percentage (72.37\%) in the Kazakh dataset, suggesting that fine-tuning in specific languages may introduce risks if proper safety measures are not taken.

% The evaluation reveals notable differences in the distribution of safe response patterns across Kazakh and Russian fine-grained labels. Refusal to answer is more frequent in Russian models, particularly Yandex-GPT, reflecting a cautious approach to safety-critical queries. Interestingly, Aya, despite being fine-tuned for Kazakh and exhibiting lower overall safety, also frequently refuses to answer, suggesting an over-reliance on conservative mechanisms. Responses providing general, safe information dominate in both languages, with Kazakh models displaying a slightly higher tendency to rely on this approach. This highlights how the low-resource nature of Kazakh results in more generalized and inherently safer responses. In contrast, Russian models excel at recognizing risks, issuing disclaimers, and refuting incorrect assumptions, likely benefiting from richer and more diverse training data.
% Yandex-GPT exhibits a notably high rate of responses classified under label 7, indicating an overreliance on general disclaimers or deflections, such as "check the internet" or "I don't know." While these responses minimize the risk of unsafe outputs, they often lack substantive or contextually relevant information, reducing their overall utility for users.


Most models perform safer on Kazakh dataset than Russian dataset, higher safe rate on Kazakh dataset in \tabref{tab:safety-binary-eval}. This does not necessarily reveal that current LLMs have better understanding and safety alignment on Kazakh language than Russian, while this may conversely imply that models do not fully understand the meaning of Kazakh attack questions, fail to perceive risks and then provide general information due to lacking sufficient knowledge regarding this request.

We observed the similar number of examples falling into category 5 \textit{general and harmless information} for both Kazakh and Russian, while the Kazakh data set size is 3.7K and Russian is 4.3K. Kazakh has much less examples in category 1 \textit{reject to answer} compared to Russian. This demonstrate models tend to provide general information and cannot clearly perceive risks for many cases.

Additionally, in spite of less harmful responses on Kazakh data, these unsafe responses distribute evenly across different risk areas and question categories, exhibiting equally vulnerability spanning all attacks regardless of what risks and how we jailbreak it.
In contrary, unsafe responses on Russian dataset often concentrate on specific areas and question types, such as region-specific risks or indirect attacks, presenting similar model behaviors when evaluating over English and Chinese data.
It suggests that broader training data in English, Chinese and Russian may allow models to address certain types of attacks robustly,
% effectively—particularly politically sensitive issues—
yet they may falter when confronted with unfamiliar content like regional sensitive topics.

Moreover, in responses collection, we observed many Russian or English responses especially for open-sourced LLMs when we explicitly instructed the models to answer Kazakh questions in Kazakh language. This further implies more efforts are still needed to improve LLMs' performance on low-resource languages.
Interestingly, \aya, a fine-tuned Kazakh model, proves an exception by displaying the lowest safety percentage (72.37\%) among Kazakh models, revealing that the multilingual fine-tuning without stringent safety measures can introduce risks.



% However, this does not mean they are explicitly fine-tuned for safety, likely it happens due to limited training data, which reduces exposure to harmful content. 
% \aya, a fine-tuned Kazakh model, proves an exception by displaying the lowest safety percentage (72.37\%) among Kazakh models, revealing that the multilingual fine-tuning without stringent safety measures can introduce risks.
% Kazakh models generally produce safer responses than their Russian counterparts, likely because Kazakh is a low-resource language with less training data. 
% This limited exposure to harmful or unsafe content naturally limits nuanced yet potentially unsafe outputs. 
% However, it does not imply that the models are specifically fine-tuned for enhanced safety.


% while Kazakh models tend to generate fewer unsafe answers overall, those unsafe responses appear more evenly spread across different risk types and question categories.
% Russian models, on the other hand, often concentrate unsafe responses in specific areas, such as region-specific risks or indirect attacks.
% It implies that their broader training datasets allow them to address certain types of unsafe content more effectively—particularly politically sensitive issues—yet they may falter when confronted with unfamiliar or insufficiently filtered content.

% Meanwhile, Kazakh models sometimes respond more broadly, possibly due to less curated training data. 

Differences also emerge in how language models handle safe responses. 
\yandexgpt, for instance, often refuses to answer high-risk queries. 
It frequently relies on generic disclaimers or deflections like ``check in the Internet'' or ``I don’t know,'' minimizing risk but are less helpful. Interestingly, it often responds with ``I don’t know'' in Russian, even for Kazakh queries, we speculate that these may be default responses stemming from internal system filters, rather than generated by model itself.
This likely explains why \yandexgpt\ is the safest model for the Russian language but ranks third for Kazakh. While its filters perform well for Russian, they struggle with the low-resource Kazakh language.

% Aya, despite its lower overall safety, also employs refusals often, hinting at an over-reliance on conservative approaches. 

% Across both languages, models commonly resort to providing general, safe information, although Kazakh models lean on this strategy slightly more. 
% Russian models, by contrast, excel at detecting risks, issuing disclaimers, and correcting inaccuracies, likely benefiting from richer and more diverse training data.


% \subsection{Response Patterns}


% We conducted a detailed analysis of the models' outputs and identified several noteworthy patterns. YandexGPT, while being one of the safest overall, frequently generates responses in Russian even when the question is posed in Kazakh. These responses often appear as placeholders, prompting users to search for the answer online. This behavior might not originate from the model itself but rather from safety filters implemented in the YandexGPT system. The model's leading performance in ensuring safety during Russian-language interactions, coupled with its lower performance in Kazakh, can be attributed to the limited robustness of these safety filters when handling unsafe content in Kazakh.

% In contrast, Aya-101 exhibits a tendency to fall into repetition, often repeating the same sentences multiple times. Interestingly, the Vikhr model, despite being of a similar size, does not exhibit this issue. We attribute this difference to two key factors. First, Vikhr and Aya-101 have distinct architectures: Vikhr is based on the Mistral-Nemo model, whereas Aya-101 is built on mT5, an older and less robust model. Second, Aya-101 is a multilingual model, while Vikhr was predominantly trained for Russian. Multilingualism has been shown to potentially degrade performance in large language models~\cite{huang2025surveylargelanguagemodels}, which may explain Aya-101's issues with repetition.

\section{Related Work}
\paragraph{LLMs for Visual Program Generation}
Visual programming systems (e.g., LabView~\cite{bitter2006labview}, XG5000~\cite{XG5000Manual}) typically feature node-based interfaces that let users visually write and modify programs. Recently, researchers have begun utilizing LLMs to generate VPLs, as they are known for their powerful text-based code generation capabilities. For example, \citet{cai-etal-2024-low-code} integrates low-code visual programming with LLM-based task execution for direct interaction with LLMs, while \citet{zhang2024benchmarking} studies generation of node-based visual dataflow languages in audio programming. Similarly, \citet{xue2024comfybenchbenchmarkingllmbasedagents,52868} investigates Machine Learning workflow generation from natural language commands and demonstrates that metaprogram-based text formats outperform other formats like JSON. However, these prompting-based methods face limitations for VPLs like Ladder Diagram, where custom I/O mapping and domain-specific syntax are crucial. Thus, we study fine-tuning approaches with domain-specific data to better capture these details.

\paragraph{LLM-based PLC code generation}
Programmable Logic Controllers (PLCs) are essential components in industrial automation and are used to control machinery and processes reliably and efficiently. Among the programming languages defined by the IEC 61131-3 standard~\cite{IEC61131-3}, Structured Text (ST) and Ladder Diagram (LD) are commonly used for programming PLCs. Research in this area has focused on utilizing LLMs to generate ST code from natural language descriptions. Recent studies have demonstrated the potential of LLMs in generating high-quality ST code~\cite{koziolek2023chatgpt, koziolek2024llm}, enhancing safety and accuracy with verification tools and user feedback~\cite{fakih2024llm4plc}, and automating code generation and verification using multi-agent frameworks~\cite{liu2024agents4plc}. Although these advances have improved PLC code generation, they primarily focus on ST, despite LD being widely used in industrial settings due to its similarity to electrical circuits~\cite{ladderlogic}. While \citet{Zhang_2024} attempts to generate LD as an ASCII art based on user instructions in a zero-shot manner, their findings show that even advanced LLMs struggle with basic LD generation. These limitations highlight the necessity of training-based methods for LD generation. In this work, we address this gap by introducing a training-based approach for LLMs to generate LD and thus pave the way for the broader adoption of AI-assisted PLC programming.
\section{Discussion}\label{sec:discussion}



\subsection{From Interactive Prompting to Interactive Multi-modal Prompting}
The rapid advancements of large pre-trained generative models including large language models and text-to-image generation models, have inspired many HCI researchers to develop interactive tools to support users in crafting appropriate prompts.
% Studies on this topic in last two years' HCI conferences are predominantly focused on helping users refine single-modality textual prompts.
Many previous studies are focused on helping users refine single-modality textual prompts.
However, for many real-world applications concerning data beyond text modality, such as multi-modal AI and embodied intelligence, information from other modalities is essential in constructing sophisticated multi-modal prompts that fully convey users' instruction.
This demand inspires some researchers to develop multimodal prompting interactions to facilitate generation tasks ranging from visual modality image generation~\cite{wang2024promptcharm, promptpaint} to textual modality story generation~\cite{chung2022tale}.
% Some previous studies contributed relevant findings on this topic. 
Specifically, for the image generation task, recent studies have contributed some relevant findings on multi-modal prompting.
For example, PromptCharm~\cite{wang2024promptcharm} discovers the importance of multimodal feedback in refining initial text-based prompting in diffusion models.
However, the multi-modal interactions in PromptCharm are mainly focused on the feedback empowered the inpainting function, instead of supporting initial multimodal sketch-prompt control. 

\begin{figure*}[t]
    \centering
    \includegraphics[width=0.9\textwidth]{src/img/novice_expert.pdf}
    \vspace{-2mm}
    \caption{The comparison between novice and expert participants in painting reveals that experts produce more accurate and fine-grained sketches, resulting in closer alignment with reference images in close-ended tasks. Conversely, in open-ended tasks, expert fine-grained strokes fail to generate precise results due to \tool's lack of control at the thin stroke level.}
    \Description{The comparison between novice and expert participants in painting reveals that experts produce more accurate and fine-grained sketches, resulting in closer alignment with reference images in close-ended tasks. Novice users create rougher sketches with less accuracy in shape. Conversely, in open-ended tasks, expert fine-grained strokes fail to generate precise results due to \tool's lack of control at the thin stroke level, while novice users' broader strokes yield results more aligned with their sketches.}
    \label{fig:novice_expert}
    % \vspace{-3mm}
\end{figure*}


% In particular, in the initial control input, users are unable to explicitly specify multi-modal generation intents.
In another example, PromptPaint~\cite{promptpaint} stresses the importance of paint-medium-like interactions and introduces Prompt stencil functions that allow users to perform fine-grained controls with localized image generation. 
However, insufficient spatial control (\eg, PromptPaint only allows for single-object prompt stencil at a time) and unstable models can still leave some users feeling the uncertainty of AI and a varying degree of ownership of the generated artwork~\cite{promptpaint}.
% As a result, the gap between intuitive multi-modal or paint-medium-like control and the current prompting interface still exists, which requires further research on multi-modal prompting interactions.
From this perspective, our work seeks to further enhance multi-object spatial-semantic prompting control by users' natural sketching.
However, there are still some challenges to be resolved, such as consistent multi-object generation in multiple rounds to increase stability and improved understanding of user sketches.   


% \new{
% From this perspective, our work is a step forward in this direction by allowing multi-object spatial-semantic prompting control by users' natural sketching, which considers the interplay between multiple sketch regions.
% % To further advance the multi-modal prompting experience, there are some aspects we identify to be important.
% % One of the important aspects is enhancing the consistency and stability of multiple rounds of generation to reduce the uncertainty and loss of control on users' part.
% % For this purpose, we need to develop techniques to incorporate consistent generation~\cite{tewel2024training} into multi-modal prompting framework.}
% % Another important aspect is improving generative models' understanding of the implicit user intents \new{implied by the paint-medium-like or sketch-based input (\eg, sketch of two people with their hands slightly overlapping indicates holding hand without needing explicit prompt).
% % This can facilitate more natural control and alleviate users' effort in tuning the textual prompt.
% % In addition, it can increase users' sense of ownership as the generated results can be more aligned with their sketching intents.
% }
% For example, when users draw sketches of two people with their hands slightly overlapping, current region-based models cannot automatically infer users' implicit intention that the two people are holding hands.
% Instead, they still require users to explicitly specify in the prompt such relationship.
% \tool addresses this through sketch-aware prompt recommendation to fill in the necessary semantic information, alleviating users' workload.
% However, some users want the generative AI in the future to be able to directly infer this natural implicit intentions from the sketches without additional prompting since prompt recommendation can still be unstable sometimes.


% \new{
% Besides visual generation, 
% }
% For example, one of the important aspect is referring~\cite{he2024multi}, linking specific text semantics with specific spatial object, which is partly what we do in our sketch-aware prompt recommendation.
% Analogously, in natural communication between humans, text or audio alone often cannot suffice in expressing the speakers' intentions, and speakers often need to refer to an existing spatial object or draw out an illustration of her ideas for better explanation.
% Philosophically, we HCI researchers are mostly concerned about the human-end experience in human-AI communications.
% However, studies on prompting is unique in that we should not just care about the human-end interaction, but also make sure that AI can really get what the human means and produce intention-aligned output.
% Such consideration can drastically impact the design of prompting interactions in human-AI collaboration applications.
% On this note, although studies on multi-modal interactions is a well-established topic in HCI community, it remains a challenging problem what kind of multi-modal information is really effective in helping humans convey their ideas to current and next generation large AI models.




\subsection{Novice Performance vs. Expert Performance}\label{sec:nVe}
In this section we discuss the performance difference between novice and expert regarding experience in painting and prompting.
First, regarding painting skills, some participants with experience (4/12) preferred to draw accurate and fine-grained shapes at the beginning. 
All novice users (5/12) draw rough and less accurate shapes, while some participants with basic painting skills (3/12) also favored sketching rough areas of objects, as exemplified in Figure~\ref{fig:novice_expert}.
The experienced participants using fine-grained strokes (4/12, none of whom were experienced in prompting) achieved higher IoU scores (0.557) in the close-ended task (0.535) when using \tool. 
This is because their sketches were closer in shape and location to the reference, making the single object decomposition result more accurate.
Also, experienced participants are better at arranging spatial location and size of objects than novice participants.
However, some experienced participants (3/12) have mentioned that the fine-grained stroke sometimes makes them frustrated.
As P1's comment for his result in open-ended task: "\emph{It seems it cannot understand thin strokes; even if the shape is accurate, it can only generate content roughly around the area, especially when there is overlapping.}" 
This suggests that while \tool\ provides rough control to produce reasonably fine results from less accurate sketches for novice users, it may disappoint experienced users seeking more precise control through finer strokes. 
As shown in the last column in Figure~\ref{fig:novice_expert}, the dragon hovering in the sky was wrongly turned into a standing large dragon by \tool.

Second, regarding prompting skills, 3 out of 12 participants had one or more years of experience in T2I prompting. These participants used more modifiers than others during both T2I and R2I tasks.
Their performance in the T2I (0.335) and R2I (0.469) tasks showed higher scores than the average T2I (0.314) and R2I (0.418), but there was no performance improvement with \tool\ between their results (0.508) and the overall average score (0.528). 
This indicates that \tool\ can assist novice users in prompting, enabling them to produce satisfactory images similar to those created by users with prompting expertise.



\subsection{Applicability of \tool}
The feedback from user study highlighted several potential applications for our system. 
Three participants (P2, P6, P8) mentioned its possible use in commercial advertising design, emphasizing the importance of controllability for such work. 
They noted that the system's flexibility allows designers to quickly experiment with different settings.
Some participants (N = 3) also mentioned its potential for digital asset creation, particularly for game asset design. 
P7, a game mod developer, found the system highly useful for mod development. 
He explained: "\emph{Mods often require a series of images with a consistent theme and specific spatial requirements. 
For example, in a sacrifice scene, how the objects are arranged is closely tied to the mod's background. It would be difficult for a developer without professional skills, but with this system, it is possible to quickly construct such images}."
A few participants expressed similar thoughts regarding its use in scene construction, such as in film production. 
An interesting suggestion came from participant P4, who proposed its application in crime scene description. 
She pointed out that witnesses are often not skilled artists, and typically describe crime scenes verbally while someone else illustrates their account. 
With this system, witnesses could more easily express what they saw themselves, potentially producing depictions closer to the real events. "\emph{Details like object locations and distances from buildings can be easily conveyed using the system}," she added.

% \subsection{Model Understanding of Users' Implicit Intents}
% In region-sketch-based control of generative models, a significant gap between interaction design and actual implementation is the model's failure in understanding users' naturally expressed intentions.
% For example, when users draw sketches of two people with their hands slightly overlapping, current region-based models cannot automatically infer users' implicit intention that the two people are holding hands.
% Instead, they still require users to explicitly specify in the prompt such relationship.
% \tool addresses this through sketch-aware prompt recommendation to fill in the necessary semantic information, alleviating users' workload.
% However, some users want the generative AI in the future to be able to directly infer this natural implicit intentions from the sketches without additional prompting since prompt recommendation can still be unstable sometimes.
% This problem reflects a more general dilemma, which ubiquitously exists in all forms of conditioned control for generative models such as canny or scribble control.
% This is because all the control models are trained on pairs of explicit control signal and target image, which is lacking further interpretation or customization of the user intentions behind the seemingly straightforward input.
% For another example, the generative models cannot understand what abstraction level the user has in mind for her personal scribbles.
% Such problems leave more challenges to be addressed by future human-AI co-creation research.
% One possible direction is fine-tuning the conditioned models on individual user's conditioned control data to provide more customized interpretation. 

% \subsection{Balance between recommendation and autonomy}
% AIGC tools are a typical example of 
\subsection{Progressive Sketching}
Currently \tool is mainly aimed at novice users who are only capable of creating very rough sketches by themselves.
However, more accomplished painters or even professional artists typically have a coarse-to-fine creative process. 
Such a process is most evident in painting styles like traditional oil painting or digital impasto painting, where artists first quickly lay down large color patches to outline the most primitive proportion and structure of visual elements.
After that, the artists will progressively add layers of finer color strokes to the canvas to gradually refine the painting to an exquisite piece of artwork.
One participant in our user study (P1) , as a professional painter, has mentioned a similar point "\emph{
I think it is useful for laying out the big picture, give some inspirations for the initial drawing stage}."
Therefore, rough sketch also plays a part in the professional artists' creation process, yet it is more challenging to integrate AI into this more complex coarse-to-fine procedure.
Particularly, artists would like to preserve some of their finer strokes in later progression, not just the shape of the initial sketch.
In addition, instead of requiring the tool to generate a finished piece of artwork, some artists may prefer a model that can generate another more accurate sketch based on the initial one, and leave the final coloring and refining to the artists themselves.
To accommodate these diverse progressive sketching requirements, a more advanced sketch-based AI-assisted creation tool should be developed that can seamlessly enable artist intervention at any stage of the sketch and maximally preserve their creative intents to the finest level. 

\subsection{Ethical Issues}
Intellectual property and unethical misuse are two potential ethical concerns of AI-assisted creative tools, particularly those targeting novice users.
In terms of intellectual property, \tool hands over to novice users more control, giving them a higher sense of ownership of the creation.
However, the question still remains: how much contribution from the user's part constitutes full authorship of the artwork?
As \tool still relies on backbone generative models which may be trained on uncopyrighted data largely responsible for turning the sketch into finished artwork, we should design some mechanisms to circumvent this risk.
For example, we can allow artists to upload backbone models trained on their own artworks to integrate with our sketch control.
Regarding unethical misuse, \tool makes fine-grained spatial control more accessible to novice users, who may maliciously generate inappropriate content such as more realistic deepfake with specific postures they want or other explicit content.
To address this issue, we plan to incorporate a more sophisticated filtering mechanism that can detect and screen unethical content with more complex spatial-semantic conditions. 
% In the future, we plan to enable artists to upload their own style model

% \subsection{From interactive prompting to interactive spatial prompting}


\subsection{Limitations and Future work}

    \textbf{User Study Design}. Our open-ended task assesses the usability of \tool's system features in general use cases. To further examine aspects such as creativity and controllability across different methods, the open-ended task could be improved by incorporating baselines to provide more insightful comparative analysis. 
    Besides, in close-ended tasks, while the fixing order of tool usage prevents prior knowledge leakage, it might introduce learning effects. In our study, we include practice sessions for the three systems before the formal task to mitigate these effects. In the future, utilizing parallel tests (\textit{e.g.} different content with the same difficulty) or adding a control group could further reduce the learning effects.

    \textbf{Failure Cases}. There are certain failure cases with \tool that can limit its usability. 
    Firstly, when there are three or more objects with similar semantics, objects may still be missing despite prompt recommendations. 
    Secondly, if an object's stroke is thin, \tool may incorrectly interpret it as a full area, as demonstrated in the expert results of the open-ended task in Figure~\ref{fig:novice_expert}. 
    Finally, sometimes inclusion relationships (\textit{e.g.} inside) between objects cannot be generated correctly, partially due to biases in the base model that lack training samples with such relationship. 

    \textbf{More support for single object adjustment}.
    Participants (N=4) suggested that additional control features should be introduced, beyond just adjusting size and location. They noted that when objects overlap, they cannot freely control which object appears on top or which should be covered, and overlapping areas are currently not allowed.
    They proposed adding features such as layer control and depth control within the single-object mask manipulation. Currently, the system assigns layers based on color order, but future versions should allow users to adjust the layer of each object freely, while considering weighted prompts for overlapping areas.

    \textbf{More customized generation ability}.
    Our current system is built around a single model $ColorfulXL-Lightning$, which limits its ability to fully support the diverse creative needs of users. Feedback from participants has indicated a strong desire for more flexibility in style and personalization, such as integrating fine-tuned models that cater to specific artistic styles or individual preferences. 
    This limitation restricts the ability to adapt to varied creative intents across different users and contexts.
    In future iterations, we plan to address this by embedding a model selection feature, allowing users to choose from a variety of pre-trained or custom fine-tuned models that better align with their stylistic preferences. 
    
    \textbf{Integrate other model functions}.
    Our current system is compatible with many existing tools, such as Promptist~\cite{hao2024optimizing} and Magic Prompt, allowing users to iteratively generate prompts for single objects. However, the integration of these functions is somewhat limited in scope, and users may benefit from a broader range of interactive options, especially for more complex generation tasks. Additionally, for multimodal large models, users can currently explore using affordable or open-source models like Qwen2-VL~\cite{qwen} and InternVL2-Llama3~\cite{llama}, which have demonstrated solid inference performance in our tests. While GPT-4o remains a leading choice, alternative models also offer competitive results.
    Moving forward, we aim to integrate more multimodal large models into the system, giving users the flexibility to choose the models that best fit their needs. 
    


\section{Conclusion}\label{sec:conclusion}
In this paper, we present \tool, an interactive system designed to help novice users create high-quality, fine-grained images that align with their intentions based on rough sketches. 
The system first refines the user's initial prompt into a complete and coherent one that matches the rough sketch, ensuring the generated results are both stable, coherent and high quality.
To further support users in achieving fine-grained alignment between the generated image and their creative intent without requiring professional skills, we introduce a decompose-and-recompose strategy. 
This allows users to select desired, refined object shapes for individual decomposed objects and then recombine them, providing flexible mask manipulation for precise spatial control.
The framework operates through a coarse-to-fine process, enabling iterative and fine-grained control that is not possible with traditional end-to-end generation methods. 
Our user study demonstrates that \tool offers novice users enhanced flexibility in control and fine-grained alignment between their intentions and the generated images.


\bibliography{anthology,custom}
\appendix
\begin{table*}[htbp]
    \small
    \centering
    \begin{tabular}{lcccc}
    \toprule
    \multirow{2}{1cm}{\textbf{Dataset}} & \multirow{2}{1.1cm}{\textbf{Context}} & \multicolumn{3}{c}{\textbf{Queries}} \\
     & &\textbf{Chunk-Level} & \textbf{Sentence-Level} & \textbf{Constraints-Based}\\\midrule
     MultiHop-RAG & 7,724 & 72,090 & 472,193 & 51,212\\
     AllSides & 645 & 6,313 & 173,898 & 6,091 \\
     AGNews & 1,050 & 10,355 & 80,524 & 20,875 \\
     NQ & 98,748 & 1,459,031 & - & - \\\bottomrule %
    \end{tabular}
    \caption{Dataset Statistics}
    \label{tab:dataset_stats}
\end{table*}

\begin{table}[htbp]
    \small
    \centering
    \begin{tabular}{lp{4.8cm}}
    \toprule
      \textbf{Dataset} & \textbf{Attributes} \\\midrule
        MultiHop-RAG & author, publish time, source, category, title \\
        AllSides & political polarity \\
        AGNews & location, topic \\\bottomrule
    \end{tabular}
    \caption{Attributes used in each dataset for constraints-based query generation.}
    \label{tab:dataset_attributes}
\end{table}

\section{Details of Experiment Setup}
We use \texttt{Mistral-7B-Instruct-v0.3} as the base model for generative retrieval with the semantic identifier, while use \texttt{Mistral-7B-v0.3} as the base model for atomic identifier as it is closer to a classification setting.

For supervised fine-tuning, we train the models with 2 epochs, with a learning rate of 2e-5 and a warmup ratio of 0.1. The batch size is set as 256. We use sequence packing to put multiple examples in one forward pass~\citep{DBLP:journals/jmlr/RaffelSRLNMZLL20}. We use \texttt{bfloat16} for our training.

For preference learning, we mainly conduct experiments on MultiHop-RAG and NQ with semantic identifiers. We train the models with 1 epoch. The learning rate is set as 1e-7, batch size is set as 64, $\beta$ is set as 0.5, $\alpha$ is set as 1.0. 

The training infrastructure includes TRL~\citep{vonwerra2022trl}, Accelerate~\citep{accelerate}, Transformers~\citep{wolf-etal-2020-transformers}, DeepSpeed~\cite{DBLP:conf/kdd/RasleyRRH20} and FlashAttention-2~\citep{DBLP:conf/iclr/Dao24}. We use 8x Nvidia A100-SXM4-40GB for our experiments. Each training or inference procedure can be completed in 1 day.

Statistics of the numbers of the documents, different synthetic queries can be found in Table~\ref{tab:dataset_stats}. Attributes used for constraints-based synthetic queries can be found in Table~\ref{tab:dataset_attributes}. All the experiment results are obtained with single run.
\label{app:data_specific_setup}
\subsection{MultiHop-RAG}
On MultiHop-RAG, we split the documents into chunks with maximum length of 256 without overlap and conduct retrieval on individual chunks. For synthetic query generation, $m_c$, $m_s$ and $m_i$ are set as 10, and the temperature for LLM inference on synthetic data generation is set as 0.7. We interleave the Context2ID and Query2ID data as the full dataset for model supervised fine-tuning. The maximum sequence length is set as 700. For synthetic queries for preference learning, we ask the LLM to generate 10 queries. We perform the retrieval with beam size as 10 and retrieve the top-10 candidates for each query to construct the candidate pairs.
\subsection{AllSides}
On AllSides, we conduct document-level retrieval. For synthetic query generation, $m_c$, $m_s$ and $m_i$ are set as 10, and the temperature for LLM inference on synthetic data generation is set as 0.7. For Context2ID data, as there are some long documents in the corpus, we will split the long context into chunks with maximum length of 256 without overlap. The Context2ID data is constructed to use all chunks in the document to predict its corresponding document identifier. We interleave the Context2ID and Query2ID data as the full dataset for model supervised fine-tuning. The maximum sequence length is set as 700.
\subsection{AGNews}
On AllSides, we conduct document-level retrieval. For synthetic query generation, $m_c$, $m_s$ and $m_i$ are set as 10, and the temperature for LLM inference on synthetic data generation is set as 0.7. Queries constructed by \citet{DBLP:journals/corr/abs-2405-02714} uses two different perspectives. The first perspective is either the location of the desired news or the topic, while the second perspective is that the news is similar to another given news in the query. As we mentioned Section~\ref{sec:experiment_setup}, we replace the second perspective with the another field so that each query consists of both location and topic perspectives. The topic and location information used for instruction-based synthetic query generation is extracted with Mixtral 8x7b. We interleave the Context2ID and Query2ID data as the full dataset for model supervised fine-tuning. The maximum sequence length is set as 700.
\subsection{NQ}
On NQ, we conduct document-level retrieval. We use the document prefixes from~\cite{DBLP:conf/icml/KishoreWLAW23} to produce the semantic identifiers. For synthetic query generation, we perform truncation on pages when they are too long so that we always have at least 1024 token space for model output. We set $m_c$ as 15 and temperature as 0.7. We do not include sentence-level synthetic queries as the number of those queries are too large to be included in training within a reasonable time. Instead, we include sentence-level Context2ID as the approximation, and use the sentences from the document prefixes from~\cite{DBLP:conf/icml/KishoreWLAW23} to predict corresponding document identifiers. In NQ, we have high quality human annotated training queries, which we also include as part of the Query2ID data and therefore we do not include instruction-based synthetic queries. We concatenate the Context2ID and Query2ID data as the full dataset for model supervised fine-tuning, as interleaving will produce a much larger dataset that cannot be trained within a reasonable time. The maximum sequence length is set as 450. For synthetic queries for preference learning, we also perform truncation as for supervised fine-tuning, and ask the LLM to generate 10 queries. As the generated query number is quite large for inference, we use the first 2 generated queries for each documents for preference learning. We perform the retrieval with beam size as 10 and retrieve the top-10 candidates for each query to construct the candidate pairs.

\section{Results on Comparison with Off-The-Shelf Retrieval Models}
\label{sec:detailed_dense_comparison}
The detailed results for each dataset are shown in Table~\ref{tab:dense_retrieval_comparison}. We run the retrieval models on MultiHop-RAG, NQ and AGNews to collect the results, and adopt the results of AllSides from \citet{DBLP:journals/corr/abs-2405-02714}.


\begin{table*}[htbp]
    \centering
    \small
    \begin{subtable}[t]{\textwidth}
        \centering
        \begin{tabular}{lcccc}
        \toprule
        \textbf{Model} & \textbf{HIT@4} & \textbf{HIT@10} & \textbf{MAP@10} & \textbf{MRR@10}\\
         \midrule
         BM25 & 64.35 & 78.31 & \bf 26.30 & \bf 58.32 \\
         bge-large-en-v1.5 & 58.80 & 78.36 & 19.96 & 42.57 \\
         Contriever-msmarco & 55.25 & 75.08 & 19.28 & 40.69 \\
         E5-mistral-7b-instruct & 54.01 & 79.56 & 19.11 & 40.77 \\
         GTE-Qwen2-7B-instruct & 63.24 & 83.55 & 22.02 & 47.50 \\
         ours & \bf 71.88 & \bf 89.80& 26.23& 54.94\\
         \bottomrule
        \end{tabular}
        \caption{MultiHop-RAG}
    \end{subtable}
    \vspace{0.2cm}
    
    \begin{subtable}[t]{\textwidth}
        \centering
        \begin{tabular}{lcccc}
        \toprule
        \textbf{Model} & \textbf{HIT@1} & \textbf{HIT@5} & \textbf{HIT@10} & \textbf{MRR@10}\\
         \midrule
         BM25 & 32.82 & 53.70 & 60.92 & 42.45 \\
         bge-large-en-v1.5 & 55.59 & 76.58 & 81.75 & 64.45 \\
         Contriever-msmarco & 53.79 & 76.16 & 81.69 & 63.36 \\
         E5-mistral-7b-instruct & 59.07 & 80.08 & 85.28 & 68.11 \\
         GTE-Qwen2-7B-instruct & 60.45 & 80.87 & 85.72 & 69.30 \\
         ours & \bf 71.22& \bf 87.41& \bf 89.97& \bf 78.14\\
         \bottomrule
        \end{tabular}
        \caption{NQ}
    \end{subtable}
    \vspace{0.2cm}

    \begin{subtable}[t]{0.48\textwidth}
        \centering
        \begin{tabular}{lccc}
        \toprule
        \textbf{Model} & \textbf{HIT@1} & \textbf{HIT@5} & \textbf{HIT@10}\\
        \midrule
        BM25 & 5.86 & 26.85 & 36.42 \\
        bge-large-en-v1.5 & 6.94 & 27.32 & 34.11\\
        Contriever-msmarco & 6.64 & 25.77 & 38.43 \\
        E5-mistral-7b-instruct & 8.18 & 28.24 & 39.82\\
        GTE-Qwen2-7B-instruct & 9.11 & 34.11 & 49.07 \\
        ours & \bf 14.20& \bf 38.58& \bf 51.85\\
        \bottomrule
        \end{tabular}
        \caption{AllSides}
    \end{subtable}
    ~~\quad
    \begin{subtable}[t]{0.48\textwidth}
        \centering
        \begin{tabular}{lccc}
        \toprule
        \textbf{Model} & \textbf{HIT@1} & \textbf{HIT@5} & \textbf{HIT@10}\\
        \midrule
        BM25 & 38.70 & 67.47 & 77.63 \\
        bge-large-en-v1.5 & 54.14 & 80.57 & 86.53\\
        Contriever-msmarco & 52.69 & 80.40 & 85.79 \\
        E5-mistral-7b-instruct & 57.32 & \bf 85.90 & \bf 88.98 \\
        GTE-Qwen2-7B-instruct & 57.65 & 83.37 & 88.57\\
        ours & \bf 62.19& 83.78& 88.24\\
        \bottomrule
        \end{tabular}
        \caption{AGNews}
    \end{subtable}

    \caption{Comparisons to Off-The-Shelf Retrieval Models Across Datasets}
    \label{tab:dense_retrieval_comparison}
\end{table*}


\section{LLM Prompts}

\subsection{Prompts for Keywords Generation}
Figure~\ref{fig:keyword_generation} shows the prompt for generating a series of keywords as the semantic document identifier.

\begin{figure}[h]
\begin{tcolorbox}[title=\textbf{Keywords Generation Prompt}]
Summarize the following context with meaningful keywords representing different important information in the context. Your output should only consist a list of keywords in Markdown format, where each line starts with the dash "-" followed by the keywords.
\\
\\
\# Context:\\
\{context\}\\
\\
\# Keywords:
\end{tcolorbox}
\caption{Prompt for keywords-based document identifier generation.}
\label{fig:keyword_generation}
\end{figure}


\subsection{Prompts for Query Generation}
Figure~\ref{fig:query_generation_prompts} shows the prompts used to generate various types of synthetic queries, including chunk- and sentence-level queries, constructions-based queries, and question-answer pairs used at the preference learning stage.









\begin{figure*}[htbp]
\centering
\begin{subfigure}[t]{\textwidth}
\begin{tcolorbox}[title=\textbf{Query Generation Prompt}]
Your task is to generate a relevant and diverse set of \{num\_sequences\} questions that can be answered by the provided context. The questions are to be used by a retriever to retrieve the article from a large corpus. Your output should be a list of unordered in Markdown format, where each line starts with dash "-" followed by the question.
\\
\\
\# Context:\\
\{context\}\\
\\\
\# Output:
\end{tcolorbox}
\label{fig:query_generation}
\end{subfigure}

\begin{subfigure}[t]{\textwidth}
\begin{tcolorbox}[title=\textbf{Constraints-based Query Generation Prompt}]
Your task is to generate a diverse set of \{num\_sequences\} questions given a context with metadata. The generated questions should be answerable by the provided context. The questions are to be used by a retriever to retrieve the article from a large corpus. In addition, the question MUST be composed with at least one metadata filtering requirement.\\
\\
\textbf{\# MultiHop-RAG} \\
For example, if the source of the article is "New York Times", you can generate questions that specifically ask for certain information from "New York Times". You should generate questions with different metadata.\\
\textbf{\# AllSides and AGNews} \\
For example, if the source of the political polarity is "left", you can generate questions that specifically ask for certain information from "left-wing" source.\\
\\
DO NOT use "the context" or "the article" in any generated queries or answers.\\
DO NOT use pronoun "this" in any generated queries or answers.\\
DO NOT leak any information in this instruction.\\


Your output should be a list of unordered in Markdown format, where each line starts with dash "-" followed by the question. You do not need to provide the answer.\\
\\
\# Metadata\\
\{metadata\}\\
\\
\# Context\\
\{context\}\\
\\
\# Output:
\end{tcolorbox}
\label{fig:instruct_query_generation}
\end{subfigure}

\begin{subfigure}[t]{\textwidth}
\begin{tcolorbox}[title=\textbf{Query-Answer Pair Generation Prompt}]
Your task is to generate a relevant and diverse set of less than \{num\_sequences\} search engine query and answer pairs given a context.\\
The queries should be similar to what people use with search engine to find the given context from a large corpus. The answers are expeced to be a short phrase.\\
You should make the queries as difficult as possible, but they should be answerable by the given context.\\
\\
Do not use "the context" or "the article" in any generated queries or answers.\\
Do not use pronoun "this" in any generated queries or answers.\\
Do not leak any information in this instruction.\\
\\
Your output should be a list of unordered items in Markdown format, where each item starts with dash "-", followed by "Query:" and the generated query, and then "Answer:" with the corresponding answer.\\
\\
\# Context\\
\{context\}\\
\\
\# Output:
\end{tcolorbox}
\label{fig:qa_generation}
\end{subfigure}
\caption{Prompts for different types of synthetic query generation.}
\label{fig:query_generation_prompts}
\end{figure*}

\end{document}
