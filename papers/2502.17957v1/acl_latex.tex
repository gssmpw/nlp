% This must be in the first 5 lines to tell arXiv to use pdfLaTeX, which is strongly recommended.
\pdfoutput=1

\documentclass[11pt]{article}
\usepackage[dvipsnames]{xcolor}

\usepackage[final]{acl}

\usepackage{times}
\usepackage{latexsym}

\usepackage[T1]{fontenc}

\usepackage[utf8]{inputenc}

\usepackage{microtype}

\usepackage{inconsolata}

\usepackage{graphicx}

\usepackage{xurl}


\usepackage{amsmath}
\usepackage{stmaryrd}
\usepackage{verbatim}

\usepackage{tabularx}
\usepackage{booktabs}
\newcolumntype{C}{>{\centering\arraybackslash}X}
\usepackage{multirow}
\usepackage{diagbox}
\usepackage{hhline}
\usepackage{color}
\usepackage{amsmath}
\usepackage{amssymb}
\usepackage{mathtools}
\usepackage{soul}

\definecolor{RoseQuartzBg}{HTML}{F7CAC9}
\definecolor{RoseQuartz}{HTML}{F5A798}
\definecolor{Serenity}{HTML}{92A8D1}
\definecolor{OrangeRed}{rgb}{1.0, 0.27, 0.0}
\definecolor{Turquoise}{HTML}{0F4C81}

\usepackage{arydshln}
\setlength\dashlinedash{1.5pt}
\setlength\dashlinegap{2pt}
\setlength\arrayrulewidth{0.3pt}

\definecolor{themered}{HTML}{FF8375}

\usepackage{xparse}
\usepackage{bbm}
\usepackage{caption}
\usepackage{subcaption}
\usepackage{stfloats}
\usepackage{pifont}
\newcommand*\colourcheck[1]{
  \expandafter\newcommand\csname #1check\endcsname{\textcolor{#1}{\ding{52}}}
}
\newcommand*\colourxmark[1]{
  \expandafter\newcommand\csname #1xmark\endcsname{\textcolor{#1}{\ding{56}}}
}

\colourcheck{green}
\colourxmark{red}
\usepackage{cancel}
\usepackage{transparent}
\usepackage{float}
\usepackage{tcolorbox}
\tcbset{colframe=black,colback=white,size=small,colbacktitle=gray!30!white,coltitle=black,fonttitle=\bfseries,fontupper=\footnotesize\ttfamily}

\newcommand{\yz}[1]{\textcolor{orange}{YZ: #1}}

\NewDocumentCommand{\haoyang}{ mO{} }{
\textcolor{Turquoise}{\textsuperscript{\textsc{Haoyang}}\textsf{\textbf{\small[#1]}}}}


\title{On Synthetic Data Strategies for Domain-Specific Generative Retrieval}


\author{Haoyang Wen$^\ddagger$, Jiang Guo$^\dagger$\thanks{Corresponding author}, Yi Zhang$^\dagger$, Jiarong Jiang$^\dagger$, Zhiguo Wang$^\dagger$\\
    $^\ddagger$Language Technologies Institute, Carnegie Mellon University~~~~$^\dagger$AWS AI\\
    \texttt{hwen3@cs.cmu.edu}\\
    \texttt{\{gujiang, imyi, jiarongj, zhiguow\}@amazon.com}}


\begin{document}
\maketitle
\begin{abstract}
This paper investigates synthetic data generation strategies in developing generative retrieval models for domain-specific corpora, thereby addressing the scalability challenges inherent in manually annotating in-domain queries. We study the data strategies for a two-stage training framework: in the first stage, which focuses on learning to decode document identifiers from queries, we investigate LLM-generated queries across multiple granularity (e.g. chunks, sentences) and domain-relevant search constraints that can better capture nuanced relevancy signals. In the second stage, which aims to refine document ranking through preference learning, we explore the strategies for mining hard negatives based on the initial model's predictions. Experiments on public datasets over diverse domains demonstrate the effectiveness of our synthetic data generation and hard negative sampling approach.
\end{abstract}
\section{Introduction}\label{sec:intro}

In computational finance, Monte Carlo simulations are used extensively to estimate the expected value of financial payoffs based on the solution of stochastic differential equations (SDEs) which model the evolution of stock prices, interest rates, exchange rates and other quantities \cite{glasserman04}.  Monte Carlo methods are very general and flexible, but for high accuracy it requires generating a large number of costly SDE path approximations, which has motivated research into a number of variance reduction or, equivalently, cost reduction techniques. One such method is
Multilevel Monte Carlo (MLMC), which was proposed in \cite{GILES2008} and was adapted for various applications that are summarised in \cite{Giles_overview17} and successfully combined with other methods such as quasi-Monte Carlo methods. The main idea of MLMC is to approximate the payoff using different time stepping resolutions when numerically solving the underlying SDE and to generate an optimal number of samples on each level, such that the overall computational cost is minimised subject to the desired bound on the variance. %, such that the total computational cost is minimised. 
The computational savings come from the fact that most samples are computed on the coarser levels and hence are less expensive while only a few samples from the finest levels are required \cite{GILES2008}.


Among the directions in which the computational cost 
of MLMC methods could further be reduced, an important avenue is the use of lower precision calculations, especially for the first Monte Carlo levels where the targeted accuracy is relatively low. 
 An overview of the research on mixed precision for the standard Monte Carlo (MC) framework is provided in \cite{ChowMixedPrecisionStandardMC} but only a few references study the potential of low precision computation in the MLMC framework \cite{Rounding_error_oliver}. To the best of our knowledge, the only MLMC framework with customised precision in the literature is \cite{brugger2014mixed}, but they use a uniform precision for all operations on each Monte Carlo level instead of optimising 
 the precision of each intermediary variable to reduce as much as possible the cost of path generation.
 
An important motivation for an MLMC framework with variable precision would be performing the low precision computations on reconfigurable hardware devices such as Field Programmable Gate Arrays (FPGAs). FPGAs contain customizable logic blocks and connectors that make it easy to adapt the digital circuit architecture for a specific application, leading to a highly parallel and optimised implementation. Therefore they are successfully exploited in applications that require high speed and have high computational workload, such as signal processing \cite{woods2008fpga}, and real time applications like high frequency trading \cite{HFT1,HFT2}. That is why a number of previous works in hardware architecture design implemented the MLMC algorithm to price financial options using FPGAs as accelerators, which resulted in improved speed and power efficiency compared to full CPU architectures \cite{Schryver2013AMM}. The paper \cite{lindsey2016domain} also proposed 
a Domain Specific Language to automate the configuration of FPGAs for this specific application. However, only \cite{brugger2014mixed} proposed a heuristic to reduce the precision in calculations.

In addition, all aforementioned works considered that the random number generation (RNG) is performed in single or double precision. Yet in most cases an important portion of the workload in the overall MLMC simulation comes from the RNG and in \cite{brugger2014mixed} this limited the total computational savings.
To reduce the cost of MLMC simulations in particular those based on the Geometric Brownian Motion (GBM), \cite{approximateICDF_Oliver, NestedOliver} have proposed to use approximate random numbers that are generated by applying an approximation of the inverse CDF to uniform random numbers. In \cite{NestedOliver}, the authors proposed a way to integrate these lower precision random variables into a \textit{nested} MLMC framework and completed a numerical analysis to bound the resulting error at each MC level by a product of the time step and the error in the random number approximation. The same authors show in \cite{approximateICDF_Oliver} that using approximate random variables reduces the cost of path generation by a factor 7.


In this paper we propose a nested MLMC framework that combines the use of approximate random normal variables and lower precision calculations to reduce the computational cost of MLMC even further than \cite{brugger2014mixed,NestedOliver}. We illustrate the efficiency of our framework in Matlab, after making several assumptions on the cost of operations and size of the errors that we carefully justify. We focus on the case of GBM and use the approximate RNG methods presented in \cite{approximateICDF_Oliver} as well as a new slightly modified method that combines CDF inversion and the central limit theorem. To choose the precision of the variables in the low precision path generation, we introduce a novel method to optimise the bit-widths. This optimisation is performed before the main path generation loop is executed and is based on a linear model of the payoff error  
due to rounding when computing in low precision. The error model relies on algorithmic differentiation in a similar manner to \cite{unifying-bwoptim,bitwidth-AD,ADAPT}. The bit-width optimisation procedure can be performed off-line, so this stage can be excluded from the on-line time complexity of our framework. The user specified desired accuracy is then enforced by calculating on-line the number of samples that need to be generated.

In terms of hardware design, we suggest implementing the low precision path generation on FPGAs and the full-precision ones on a CPU or GPU. 
The FPGA offers enough flexibility to define a separate bit-width for every variable in the low precision path generation, and can be reconfigured periodically to update the bit-widths when the market parameters have changed considerably. 


The paper is organized as follows : \Cref{sec:MLMC} introduces MLMC and nested MLMC to make clear the estimator that is implemented in our framework. Then in \Cref{sec:RNG} we detail the methods that could be used to obtain approximate random normally distributed numbers very cheaply for the low precision path generation. In \Cref{sec:error_model} and \Cref{sec:costModel} we propose an error model and a cost model (resp.) that we then use to formulate the optimisation problem that is solved to obtain the optimal bit-widths of fixed point variables in \Cref{sec:optimisation}. Finally we summarise our results and future directions in \Cref{sec:conclusion}.



\section{Generative Retrieval Framework}
A typical generative retrieval framework takes a query as input, and generates the corresponding relevant document identifiers as the retrieval results~\citep{DBLP:conf/nips/Tay00NBM000GSCM22}. Because each document in the corpus has a unique identifier, one can then use these identifiers to retrieve the corresponding documents for downstream tasks.

\subsection{Document Identifiers}
We primarily use \textit{semantic} document identifiers in our experiments due to their superior performance and better scalability to larger corpora.
Instead of using corpus-specific semantic identifiers like titles or urls, we adopt a more general, keyword-based approach that can be applied to a wide variety of corpora~\citep{DBLP:journals/corr/abs-2208-09257}.
Specifically, we instruct an LLM to produce a list of keywords that describes the content of a document, and use this keyword list as its semantic identifier.

In addition, we extended our synthetic data strategies to other types of identifiers to validate its generalizability, such as atomic identifiers ~\citep{DBLP:conf/nips/Tay00NBM000GSCM22}, which are unique tokens that can be generated through a one-step decoding or classification process.

\subsection{Generative Modeling}
The generative retrieval model learns to generate the identifier of a relevant document given a query. Formally, for a query $q$ and a relevant document $d$ with identifier $d'$, generative retrieval aims to produce $d'$ given $q$, which can be represented as:
\newcommand{\score}[1]{\operatorname{score}(#1)}
\begin{align*}
\score{q,d} &= P\left(d'\mid q; \theta\right) \\
&= \prod_{i}P\left(d'_i \mid d'_{<i}, q; \theta \right),
\end{align*}
where $d'_{i}$ is the $i^\text{th}$ token of the identifier. To ensure the generated identifiers are valid during inference, we use constrained beam search with Trie~\citep{DBLP:journals/corr/abs-2010-00904} to restrict the output token space at each decoding step. The top-$k$ output from the beam search serves as the final retrieval results.

Compared to dense retrieval models~\cite{karpukhin2020dense}, generative retrieval bypasses the need for an external index by directly producing relevant document identifiers. However, there are distinct challenges in learning a generative retrieval model.
As it solely relies on parametric knowledge, the model must not only learn the retrieval task, but also capture and encode document content in a way that associates each document with its identifier. Therefore, generative retrieval often requires training on the entire corpus to enable the model to memorize and comprehend the necessary information effectively. 

\begin{figure*}[htbp]
    \centering
    \includegraphics[width=\linewidth, trim={0.6cm 0.2cm 0.6cm 0.3cm}, clip]{figures/generative_retrieval_workflow.pdf}
    \caption{The overall workflow of the generative retrieval training and synthetic data utilization at each stage.}
    \label{fig:workflow}
\end{figure*}

\section{Supervised Fine-Tuning Data Strategy}
In a typical domain-specific setup, we often assume access to a corpus with limited or no labeled data for domain-specific training~\citep{DBLP:conf/ictir/HashemiZKPMC23}. Therefore, it is crucial to create high-quality synthetic data that thoroughly covers the entire corpus for generative retrieval training.

Our synthetic data comprises two main components: Context2ID data and Query2ID data. Context2ID involves training the model to retrieve the document identifiers given the document's content. Query2ID focuses on teaching the model to retrieve relevant document identifiers from a given query.
Combining these two data types encourages the model to learn both content memorization and retrieval given a query.

\subsection{Supervised Fine-Tuning Objective}
At this stage, we train the model to generate relevant document identifiers by maximizing the probability of each individual token. While typical supervised fine-tuning (SFT), especially with encoder-decoder architectures such as T5, focuses on optimizing the output sequence (\textit{i.e.} document identifiers), it's also part of the training goal for generative retrieval models to comprehend and memorize the context. To this end, we also optimize the model for learning to decode the input.
Specifically, for a given query-document pair $(q,d)$, where $q$ could be an actual query or a text chunk from the document, the model maximizes the likelihood of the combined input and output sequence:
\begin{align*}
\mathcal{L}_\text{sft}\left(q,d\right) = &-\log P\left(d', q; \theta\right) \\
= &-\sum_i \log P(q_i \mid q_{<i}; \theta) \\
 &- \sum_i \log P(d'_i \mid d'_{<i}, q; \theta).
\end{align*}

\begin{table*}[htbp]
    \centering
    \small
    \begin{tabular}{lp{0.749\textwidth}}
        \toprule
        \textbf{Data Type} & \textbf{Example} \\
        \midrule
        Context & title: Christmas Day preview: \colorbox{Apricot}{49ers}, \colorbox{Salmon}{Ravens} square off in potential Super Bowl sneak peek\ldots source: \colorbox{GreenYellow}{Yardbarker} \ldots \colorbox{Apricot}{San Francisco} has racked up an NFL-leading 25 turnovers and has given up the second-fewest rushing \colorbox{Goldenrod}{yards (1,252)}, \ldots \\
        Chunk-Level Query & What is the potential implication of this matchup between the \colorbox{Apricot}{49ers} and \colorbox{Salmon}{Ravens}? \\
        Sentence-Level Query &  Where does the \colorbox{Apricot}{49ers}' defense stand in terms of \colorbox{Goldenrod}{total yards} allowed per game? \\
        Constraints-Based Query & \underline{According to the \colorbox{GreenYellow}{Yardbarker} article}, which team has the league's most effective running game?\\% as of 2023-12-24?\\
        \bottomrule
    \end{tabular}
    \caption{Examples of different synthetic queries generated from MultiHop-RAG corpus.} %
    \label{tab:data_type_example}
    \vspace{-0.5em}
\end{table*}

\subsection{Context2ID}
Context2ID data is created by pairing each chunk of text in the corpus with its corresponding document identifier. The goal of Context2ID data is to help the generative retrieval model associate each document's content with its unique identifier, i.e., ``memorizing'' the text.

\subsection{Query2ID}
Query2ID is designed to teach the model to retrieve the relevant document identifiers given a query. It helps the model to learn the core retrieval task and also further comprehend content from the query perspective.

Previous work~\citep{DBLP:journals/corr/abs-2206-10128} finds it effective to use a query generation model (\textit{e.g.,} docT5query, \citealp{nogueiradoc2query}) to produce synthetic queries from documents using multiple independent samplings. In this work, we instead use an LLM for synthetic query generation. Specifically, given a context (e.g., a document chunk), the LLM is instructed to generate a diverse set of $m$ queries, thereby covering a  wider range of semantic variations compared to the sampling-based approach with a specialized query generation model.


\subsubsection{Multi-Granular Query Generation} We first generate queries with context at different levels of granularity: \textit{chunk-level} and \textit{sentence-level}. Chunk-level synthetic queries are produced by providing the entire chunk as input to the LLM to capture higher-level semantics or facts, while sentence-level synthetic queries are produced by only providing individual sentences to focus on more specific details within the document
Concretely, for each chunk, we ask the LLM to produce $m_c$ chunk-level queries. We then split the chunk into individual sentences and ask the LLM to generate $m_s$ sentence-level queries for each sentence.

\subsubsection{Constraints-Based Query Generation}

A key advantage of using an LLM for query generation is its ability to incorporate domain-specific instructions.
For instance, we can prompt the LLM to include metadata constraints, such as the \textit{author name} or \textit{political polarity} of a document, in the generated queries.
Although the specific constraint types depends on the metadata available and can be domain or dataset specific, they are common in real-world scenarios such as enterprise data. Table~\ref{tab:dataset_attributes} in Appendix specifies the attributes that we use to produce constraints-based synthetic queries for each dataset. We ask the LLM to generate $m_i$ queries for each document that incorporate these constraints, allowing our generative retrieval model to handle more specialized or domain-specific queries.

\section{Preference Learning Data Strategy}
Previous work~\citep{zhou-etal-2023-enhancing-generative,DBLP:conf/aaai/00010WWL24} have shown that incorporating ranking tasks can further enhance the relevance modeling of generative retrieval models. However, when generative retrieval models are based on large language models, complex ranking objectives -- such as listwise optimization -- often become computationally inefficient due to multiple forward passes. In this work, we instead use a simpler method and adopt the regularized preference optimization algorithm to perform the preference optimization, a technique widely applied in optimizing large language models. We will first briefly introduce the preference optimization method, and then turn our focus on the synthetic data construction, which consists of the synthetic queries along with their corresponding preferred or rejected candidates.

\subsection{Preference Optimization Objective}
We use Regularized Preference Optimization (\citealp[RPO]{DBLP:journals/corr/abs-2404-19733}) as our optimization method for preference learning. It is an extended version of Directed Preference Optimization (\citealp[DPO]{DBLP:conf/nips/RafailovSMMEF23}), including additional supervised fine-tuning loss to alleviate the over-optimization issues on negative responses. It takes an input query $q$, a positive candidate $d_p$, and a negative candidate $d_n$ as input. The loss is in favor of the positive candidate while against the negative candidate
\begin{align*}
    \mathcal{L}_\text{rpo}\left(q, d_p, d_n\right) = &- \log \delta \left(  \beta\log \frac{P\left(d'_p\mid q; \theta\right)}{P\left(d'_p\mid q; \theta_\text{ref}\right)} \right.\\
    & \quad\quad\ \ \ \left. -\beta\log \frac{P\left(d'_n\mid q; \theta\right)}{P\left(d'_n\mid q; \theta_\text{ref}\right)} \right) \\
    & - \alpha\frac{\log P(d'_p \mid q; \theta)}{\left|d'_p\right|},
\end{align*}
where $\theta_\text{ref}$ is the parameter of the reference model, \textit{i.e.,} the supervised fine-tuned model from the first stage training. $d'_p$ and $d'_n$ are the identifiers of the positive and negative candidate, respectively.

\subsection{Synthetic Queries}
Similar to the previous section, in a domain-specific setup, we assume that we do not have enough data for model training. Therefore, after the supervised fine-tuning stage, we need a batch of new synthetic queries for preference learning.

We still adopt the LLM-based query generation as with the supervised fine-tuning stage. However, there are a few key differences in the instructions. First of all, we ask the LLM to make queries as difficult as possible. At the same time, we ask the LLM to provide not only the synthetic queries but also their corresponding answers. This is to ensure that, while making difficult queries, those synthetic queries are still answerable using the given context.

These changes make the new batch of synthetic queries different from queries used during supervised fine-tuning so that the model will not be over-optimized to the same batch of data. Intensifying the difficulties also increases the likelihood that the initial generative retrieval model makes mistakes, and therefore the model will benefit from the preference learning by learning from those mistakes.

\subsection{Candidate Selection}
After producing the synthetic queries, the next step is to select document candidate pairs for RPO optimization. For each training instance, we need one positive candidate and one negative candidate. As we always produce synthetic queries based on a document, the positive candidate can be naturally assigned. Therefore, the focus will be on selecting negative candidates for each synthetic query.

To increase the hardness of the negative candidates, we choose to select negative candidates from the retrieval results. Specifically, after the supervised fine-tuning stage, we will use the generative retrieval model to perform retrieval on the synthetic queries for preference learning. Our strategy mainly focuses on selecting the top-$k$ negative candidates with ranks higher than the positive candidate from the retrieval results. In this way, if the positive candidate ranks in the top-1, we will not use the query for preference learning. If the rank of the positive candidate is higher than $k$, then there will be different numbers of negative candidates, depending on the rank. If the rank is lower than $k$, there will be $k$ different negative candidates. When there are multiple negative candidates, we pair each negative candidate with the positive one to form a candidate pair instance for preference learning.



\section{Experimental Results}
\begin{table*}[t]
\centering
\caption{Total Variation Distance on CIFAR-10-LT ($N_l = 500$, $M_l = 4000$) with different class imbalance ratios $\gamma_l$ and $\gamma_u$ under five different unlabeled class distributions.}
\label{tab:cifar10-tv}
\resizebox{\textwidth}{!}{
\begin{tabular}{lccccccccccc}
\toprule
& & \multicolumn{2}{c}{consistent} & \multicolumn{2}{c}{uniform} & \multicolumn{2}{c}{reversed} & \multicolumn{2}{c}{middle} & \multicolumn{2}{c}{head-tail} \\
\cmidrule(lr){3-4} \cmidrule(lr){5-6} \cmidrule(lr){7-8} \cmidrule(lr){9-10} \cmidrule(lr){11-12}
& & $\gamma_l = 150$ & $\gamma_l = 100$ & $\gamma_l = 150$ & $\gamma_l = 100$ & $\gamma_l = 150$ & $\gamma_l = 100$ & $\gamma_l = 150$ & $\gamma_l = 100$ & $\gamma_l = 150$ & $\gamma_l = 100$ \\
Model & Estimator & $\gamma_u = 150$ & $\gamma_u = 100$ & $\gamma_u = 1$ & $\gamma_u = 1$ & $\gamma_u = 1/150$ & $\gamma_u = 1/100$ & $\gamma_u = 150$ & $\gamma_u = 100$ & $\gamma_u = 150$ & $\gamma_u = 100$ \\
\midrule
Supervised & MLLS & 0.269 ± 0.252 & 0.038 ± 0.006 & 0.251 ± 0.046 & 0.255 ± 0.060 & 0.429 ± 0.028 & 0.493 ± 0.050 & 0.333 ± 0.042 & 0.320 ± 0.009 & 0.457 ± 0.034 & 0.444 ± 0.043 \\
Supervised & RLLS & 0.043 ± 0.001 & 0.044 ± 0.010 & 0.348 ± 0.034 & 0.305 ± 0.068 & 0.769 ± 0.016 & 0.678 ± 0.028 & 0.430 ± 0.008 & 0.368 ± 0.013 & 0.539 ± 0.018 & 0.503 ± 0.020 \\
\midrule
MLE & IPW & 0.027 ± 0.001 & 0.027 ± 0.000 & 0.319 ± 0.072 & 0.243 ± 0.010 & 0.674 ± 0.020 & 0.646 ± 0.041 & 0.438 ± 0.020 & 0.454 ± 0.026 & 0.547 ± 0.049 & 0.491 ± 0.059 \\
MLE & OR & 0.045 ± 0.004 & 0.042 ± 0.000 & 0.215 ± 0.026 & 0.203 ± 0.032 & 0.433 ± 0.017 & 0.395 ± 0.033 & 0.193 ± 0.006 & 0.209 ± 0.037 & 0.307 ± 0.147 & 0.249 ± 0.130 \\
MLE & DR & 0.090 ± 0.002 & 0.079 ± 0.000 & 0.407 ± 0.027 & 0.360 ± 0.007 & 0.425 ± 0.007 & 0.421 ± 0.029 & 0.256 ± 0.001 & 0.286 ± 0.031 & 0.435 ± 0.136 & 0.362 ± 0.122 \\
\midrule
EM & IPW & 0.035 ± 0.002 & 0.040 ± 0.001 & 0.021 ± 0.001 & 0.029 ± 0.015 & 0.303 ± 0.187 & 0.091 ± 0.010 & 0.119 ± 0.011 & 0.105 ± 0.022 & 0.104 ± 0.026 & 0.104 ± 0.051 \\
EM & OR & 0.037 ± 0.003 & 0.042 ± 0.002 & 0.016 ± 0.001 & 0.024 ± 0.012 & 0.269 ± 0.183 & 0.090 ± 0.008 & 0.122 ± 0.012 & 0.103 ± 0.022 & 0.072 ± 0.012 & 0.073 ± 0.024 \\
EM & DR & 0.034 ± 0.004 & 0.037 ± 0.001 & 0.014 ± 0.001 & 0.027 ± 0.020 & 0.264 ± 0.191 & 0.092 ± 0.005 & 0.111 ± 0.019 & 0.097 ± 0.026 & 0.077 ± 0.016 & 0.073 ± 0.028 \\
\midrule
SimPro & IPW & 0.070 ± 0.011 & 0.058 ± 0.000 & 0.046 ± 0.001 & 0.049 ± 0.005 & 0.254 ± 0.074 & 0.223 ± 0.098 & 0.097 ± 0.025 & 0.067 ± 0.002 & 0.105 ± 0.066 & 0.110 ± 0.079 \\
SimPro & OR & 0.071 ± 0.012 & 0.058 ± 0.000 & 0.045 ± 0.001 & 0.049 ± 0.006 & 0.040 ± 0.003 & 0.059 ± 0.017 & 0.074 ± 0.006 & 0.075 ± 0.002 & 0.033 ± 0.003 & 0.033 ± 0.003 \\
SimPro & DR & 0.017 ± 0.004 & 0.026 ± 0.001 & 0.019 ± 0.002 & 0.018 ± 0.003 & 0.039 ± 0.003 & 0.058 ± 0.025 & 0.091 ± 0.007 & 0.031 ± 0.001 & 0.015 ± 0.003 & 0.019 ± 0.007 \\
\bottomrule
\end{tabular}
}
\end{table*}


\begin{table*}[t]
\centering
\caption{Total Variation Distance on CIFAR-100-LT ($N_l = 50$, $M_l = 400$) with different class imbalance ratios $\gamma_l$ and $\gamma_u$ under five different unlabeled class distributions.}
\label{tab:cifar100-tv}
\resizebox{\textwidth}{!}{
\begin{tabular}{lccccccccccc}
\toprule
& & \multicolumn{2}{c}{consistent} & \multicolumn{2}{c}{uniform} & \multicolumn{2}{c}{reversed} & \multicolumn{2}{c}{middle} & \multicolumn{2}{c}{head-tail} \\
\cmidrule(lr){3-4} \cmidrule(lr){5-6} \cmidrule(lr){7-8} \cmidrule(lr){9-10} \cmidrule(lr){11-12}
& & $\gamma_l = 20$ & $\gamma_l = 10$ & $\gamma_l = 20$ & $\gamma_l = 10$ & $\gamma_l = 20$ & $\gamma_l = 10$ & $\gamma_l = 20$ & $\gamma_l = 10$ & $\gamma_l = 20$ & $\gamma_l = 10$ \\
Model & Estimator & $\gamma_u = 20$ & $\gamma_u = 10$ & $\gamma_u = 1$ & $\gamma_u = 1$ & $\gamma_u = 1/20$ & $\gamma_u = 1/10$ & $\gamma_u = 20$ & $\gamma_u = 10$ & $\gamma_u = 20$ & $\gamma_u = 10$ \\
\midrule
Supervised & MLLS & 0.707 ± 0.016 & 0.313 ± 0.100 & 0.445 ± 0.172 & 0.309 ± 0.119 & 0.383 ± 0.075 & 0.397 ± 0.006 & 0.570 ± 0.001 & 0.373 ± 0.107 & 0.543 ± 0.009 & 0.231 ± 0.057 \\
Supervised & RLLS & 0.520 ± 0.007 & 0.133 ± 0.003 & 0.337 ± 0.125 & 0.253 ± 0.082 & 0.424 ± 0.060 & 0.463 ± 0.003 & 0.454 ± 0.021 & 0.306 ± 0.074 & 0.460 ± 0.028 & 0.241 ± 0.040 \\
\midrule
MLE & IPW & 0.075 ± 0.000 & 0.071 ± 0.001 & 0.229 ± 0.001 & 0.167 ± 0.002 & 0.565 ± 0.005 & 0.443 ± 0.007 & 0.415 ± 0.000 & 0.311 ± 0.005 & 0.343 ± 0.000 & 0.280 ± 0.001 \\
MLE & OR & 0.065 ± 0.002 & 0.061 ± 0.001 & 0.200 ± 0.007 & 0.143 ± 0.001 & 0.526 ± 0.011 & 0.399 ± 0.023 & 0.360 ± 0.003 & 0.256 ± 0.012 & 0.328 ± 0.003 & 0.266 ± 0.005 \\
MLE & DR & 0.149 ± 0.019 & 0.145 ± 0.010 & 0.243 ± 0.004 & 0.214 ± 0.019 & 0.568 ± 0.005 & 0.464 ± 0.014 & 0.403 ± 0.014 & 0.309 ± 0.012 & 0.365 ± 0.007 & 0.320 ± 0.004 \\
\midrule
EM & IPW & 0.097 ± 0.008 & 0.092 ± 0.004 & 0.239 ± 0.007 & 0.179 ± 0.003 & 0.478 ± 0.012 & 0.329 ± 0.020 & 0.262 ± 0.016 & 0.202 ± 0.003 & 0.312 ± 0.002 & 0.227 ± 0.001 \\
EM & OR & 0.121 ± 0.007 & 0.108 ± 0.005 & 0.261 ± 0.007 & 0.189 ± 0.004 & 0.489 ± 0.013 & 0.335 ± 0.020 & 0.274 ± 0.016 & 0.211 ± 0.004 & 0.336 ± 0.003 & 0.235 ± 0.001 \\
EM & DR & 0.125 ± 0.005 & 0.111 ± 0.004 & 0.269 ± 0.007 & 0.194 ± 0.005 & 0.497 ± 0.010 & 0.336 ± 0.024 & 0.281 ± 0.019 & 0.219 ± 0.008 & 0.336 ± 0.007 & 0.233 ± 0.004 \\
\midrule
SimPro & IPW & 0.125 ± 0.001 & 0.100 ± 0.005 & 0.166 ± 0.007 & 0.141 ± 0.009 & 0.353 ± 0.023 & 0.261 ± 0.008 & 0.202 ± 0.003 & 0.158 ± 0.005 & 0.277 ± 0.009 & 0.197 ± 0.003 \\
SimPro & OR & 0.133 ± 0.005 & 0.100 ± 0.004 & 0.160 ± 0.007 & 0.138 ± 0.010 & 0.322 ± 0.014 & 0.253 ± 0.008 & 0.202 ± 0.003 & 0.156 ± 0.005 & 0.269 ± 0.006 & 0.191 ± 0.004 \\
SimPro & DR & 0.122 ± 0.003 & 0.106 ± 0.006 & 0.188 ± 0.009 & 0.149 ± 0.006 & 0.343 ± 0.023 & 0.257 ± 0.007 & 0.219 ± 0.010 & 0.172 ± 0.002 & 0.279 ± 0.007 & 0.198 ± 0.004 \\
\bottomrule
\end{tabular}
}
\end{table*}
\begin{table*}[t]
\centering
\caption{Top-1 accuracy (\%) on CIFAR-10-LT ($N_l = 500$, $M_l = 4000$) with different class imbalance ratios $\gamma_l$ and $\gamma_u$ under five different unlabeled class distributions. In most settings, our two stage algorithm improves SimPro (9 / 10) and BOAT (8 / 10). We use {\green green} to indicate when our plug-in improves and {\red red} when it degrades the base model.}
\label{tab:cifar10-acc}
\resizebox{\textwidth}{!}{
\begin{tabular}{lcccccccccc}
\toprule

& \multicolumn{2}{c}{consistent} & \multicolumn{2}{c}{uniform} & \multicolumn{2}{c}{reversed} & \multicolumn{2}{c}{middle} & \multicolumn{2}{c}{head-tail} \\
\cmidrule(lr){2-3} \cmidrule(lr){4-5} \cmidrule(lr){6-7} \cmidrule(lr){8-9} \cmidrule(lr){10-11}

& $\gamma_l = 150$ & $\gamma_l = 100$ & $\gamma_l = 150$ & $\gamma_l = 100$ & $\gamma_l = 150$ & $\gamma_l = 100$ & $\gamma_l = 150$ & $\gamma_l = 100$ & $\gamma_l = 150$ & $\gamma_l = 100$ \\
& $\gamma_u = 150$ & $\gamma_u = 100$ & $\gamma_u = 1$ & $\gamma_u = 1$ & $\gamma_u = 1/150$ & $\gamma_u = 1/100$ & $\gamma_u = 150$ & $\gamma_u = 100$ & $\gamma_u = 150$ & $\gamma_u = 100$ \\

\midrule

FixMatch & 62.9 $\pm$ 0.36 & 67.8 $\pm$ 1.13 & 67.6 $\pm$ 2.56 & 73.0 $\pm$ 3.81 & 59.9 $\pm$ 0.82 & 62.5 $\pm$ 0.94 & 64.3 $\pm$ 0.63 & 71.7 $\pm$ 0.46 & 58.3 $\pm$ 1.46 & 66.6 $\pm$ 0.87 \\
CReST+ & 67.5 $\pm$ 0.45 & 76.3 $\pm$ 0.86 & 74.9 $\pm$ 0.90 & 82.2 $\pm$ 1.53 & 62.0 $\pm$ 1.18 & 62.9 $\pm$ 1.39 & 58.5 $\pm$ 0.68 & 71.4 $\pm$ 0.60 & 59.3 $\pm$ 0.72 & 67.2 $\pm$ 0.48 \\
DASO & 70.1 $\pm$ 1.81 & 76.0 $\pm$ 0.37 & 83.1 $\pm$ 0.47 & 86.6 $\pm$ 0.84 & 64.0 $\pm$ 0.11 & 71.0 $\pm$ 0.95 & 69.0 $\pm$ 0.31 & 73.1 $\pm$ 0.68 & 70.5 $\pm$ 0.59 & 71.1 $\pm$ 0.32 \\
% w/ ACR$\dagger$ (Wei \& Gan, 2023) & 70.9 $\pm$ 0.37 & 76.1 $\pm$ 0.42 & 91.9 $\pm$ 0.02 & 92.5 $\pm$ 0.19 & 83.2 $\pm$ 0.39 & 85.2 $\pm$ 0.12 & 77.6 $\pm$ 0.20 & 79.3 $\pm$ 0.30 & 73.8 $\pm$ 0.83 & 79.3 $\pm$ 0.48 \\
% w/ SimPro & 74.2 $\pm$ 0.90 & 80.7 $\pm$ 0.30 & 93.6 $\pm$ 0.08 & 93.8 $\pm$ 0.10 & 83.5 $\pm$ 0.95 & 85.8 $\pm$ 0.48 & 82.6 $\pm$ 0.38 & 84.8 $\pm$ 0.54 & 81.0 $\pm$ 0.27 & 83.0 $\pm$ 0.36 \\
Supervised & 63.2 $\pm$ 0.14 & 66.0 $\pm$ 0.27 & 63.3 $\pm$ 0.28 & 65.8 $\pm$ 0.19 & 63.1 $\pm$ 0.19 & 65.9 $\pm$ 0.51 & 63.5 $\pm$ 0.22 & 65.8 $\pm$ 0.03 & 63.0 $\pm$ 0.18 & 66.4 $\pm$ 0.07 \\
\midrule
EM & 69.1 $\pm$ 1.29 & 73.8 $\pm$ 0.71 & 94.0 $\pm$ 0.08 & 93.2 $\pm$ 0.94 & 76.6 $\pm$ 2.72 & 82.2 $\pm$ 0.24 & 79.5 $\pm$ 0.35 & 81.6 $\pm$ 0.58 & 79.2 $\pm$ 0.50 & 79.8 $\pm$ 0.17 \\
\midrule
SimPro & 74.4 $\pm$ 0.71 & 79.7 $\pm$ 0.45 & 93.3 $\pm$ 0.10 & 93.3 $\pm$ 0.47 & 83.8 $\pm$ 0.80 & 84.1 $\pm$ 0.24 & 78.7 $\pm$ 0.30 & 84.2 $\pm$ 0.26 & 81.2 $\pm$ 0.20 & 82.0 $\pm$ 1.07 \\
% \midrule
SimPro+ & \green 77.8 $\pm$ 1.50 & \green 81.2 $\pm$ 0.39 & \green 93.7 $\pm$ 0.07 & \green 93.7 $\pm$ 0.24 & \red 83.3 $\pm$ 0.38 & \green 84.7 $\pm$ 0.78 & \green 79.2 $\pm$ 0.70 & \green 85.4 $\pm$ 0.66 & \green 81.3 $\pm$ 0.27 & \green 82.5 $\pm$ 0.56 \\
\midrule
BOAT & 80.5 $\pm$ 0.39 & 83.3 $\pm$ 0.27 & 93.9 $\pm$ 0.03 & 94.1 $\pm$ 0.10 & 79.7 $\pm$ 0.25 & 81.1 $\pm$ 0.15 & 79.7 $\pm$ 1.15 & 81.6 $\pm$ 0.09 & 79.4 $\pm$ 0.44 & 80.9 $\pm$ 0.16 \\
% \midrule
BOAT+ & \green 81.6 $\pm$ 0.15 & \green 83.8 $\pm$ 0.04 & \red 93.7 $\pm$ 0.23 & 94.1 $\pm$ 0.17 & \green 80.4 $\pm$ 0.71 & \green 81.7 $\pm$ 0.38 & \green 80.3 $\pm$ 0.28 & \green 83.1 $\pm$ 0.45 & \green 79.7 $\pm$ 0.29 & \green 81.0 $\pm$ 0.36 \\
\bottomrule
\end{tabular}
}
\end{table*}

\begin{table*}[t]
\centering
\caption{Top-1 accuracy (\%) on CIFAR-100-LT ($N_l = 50$, $M_l = 400$) with different class imbalance ratios $\gamma_l$ and $\gamma_u$ under five different unlabeled class distributions. Despite poor estimation in stage 1, our approach does not degrade the accuracy for most of the settings. We use {\green green} to indicate when our plug-in improves and {\red red} when it degrades the base method.}
\label{tab:cifar100-acc}
\resizebox{\textwidth}{!}{
\begin{tabular}{lccccccccccc}
\toprule

& \multicolumn{2}{c}{consistent} & \multicolumn{2}{c}{uniform} & \multicolumn{2}{c}{reversed} & \multicolumn{2}{c}{middle} & \multicolumn{2}{c}{head-tail} \\
\cmidrule(lr){2-3} \cmidrule(lr){4-5} \cmidrule(lr){6-7} \cmidrule(lr){8-9} \cmidrule(lr){10-11}

& $\gamma_l = 20$ & $\gamma_l = 10$ & $\gamma_l = 20$ & $\gamma_l = 10$ & $\gamma_l = 20$ & $\gamma_l = 10$ & $\gamma_l = 20$ & $\gamma_l = 10$ & $\gamma_l = 20$ & $\gamma_l = 10$ \\
& $\gamma_u = 20$ & $\gamma_u = 10$ & $\gamma_u = 1$ & $\gamma_u = 1$ & $\gamma_u = 1/20$ & $\gamma_u = 1/10$ & $\gamma_u = 20$ & $\gamma_u = 10$ & $\gamma_u = 20$ & $\gamma_u = 10$ \\

\midrule
% FixMatch & 40.0 $\pm$ 0.96 & 45.2 $\pm$ 0.55 & 39.6 $\pm$ 1.16 & \\
% CReST+ & 40.1 $\pm$ 1.28 & 44.5 $\pm$ 0.94 & 37.6 $\pm$ 0.88 & \\
% DASO & 43.0 $\pm$ 0.15 & 49.8 $\pm$ 0.24 & 49.4 $\pm$ 0.93 & \\
Supervised & 32.4 $\pm$ 0.40 & 38.4 $\pm$ 0.18 & 32.7 $\pm$ 0.25 & 38.0 $\pm$ 0.22 & 32.5 $\pm$ 0.51 & 38.4 $\pm$ 0.43 & 32.3 $\pm$ 0.08 & 37.9 $\pm$ 0.43 & 32.1 $\pm$ 0.33 & 38.2 $\pm$ 0.38 \\
% \midrule
EM & 42.4 $\pm$ 0.43 & 49.6 $\pm$ 0.30 & 50.9 $\pm$ 0.27 & 58.0 $\pm$ 0.35 & 42.1 $\pm$ 0.16 & 49.8 $\pm$ 0.47 & 42.8 $\pm$ 0.41 & 49.6 $\pm$ 0.36 & 41.5 $\pm$ 1.26 & 49.5 $\pm$ 0.18 \\
\midrule
SimPro & 42.5 $\pm$ 0.58 & 49.6 $\pm$ 0.22 & 51.7 $\pm$ 0.22 & 58.1 $\pm$ 0.53 & 44.9 $\pm$ 0.21 & 51.8 $\pm$ 0.42 & 42.7 $\pm$ 0.06 & 49.8 $\pm$ 0.45 & 43.3 $\pm$ 0.76 & 50.9 $\pm$ 0.19 \\
% \midrule
SimPro+ & \green 42.8 $\pm$ 0.49 & \green 50.1 $\pm$ 0.33 & \red 51.6 $\pm$ 0.63 & \red 57.8 $\pm$ 0.38 & \red 44.7 $\pm$ 0.51 & \red 51.4 $\pm$ 0.88 & \green 43.4 $\pm$ 0.58 & \green 50.4 $\pm$ 0.28 & \green 43.8 $\pm$ 0.50 & \red 50.7 $\pm$ 0.76 \\
\midrule
BOAT & 43.7 $\pm$ 0.16 & 51.4 $\pm$ 0.32 & 55.1 $\pm$ 0.95 & 60.5 $\pm$ 0.15 & 43.1 $\pm$ 0.49 & 52.7 $\pm$ 0.23 & 43.6 $\pm$ 0.19 & 51.4 $\pm$ 0.39 & 43.9 $\pm$ 0.42 & 51.4 $\pm$ 0.14 \\
% \midrule
BOAT+ & \green 44.8 $\pm$ 0.13 & 51.4 $\pm$ 0.51 & \red 53.8 $\pm$ 0.32 & 60.5 $\pm$ 0.69 & \green 43.4 $\pm$ 0.56 & \red 52.4 $\pm$ 0.36 & \green 43.9 $\pm$ 0.59 & \red 50.8 $\pm$ 0.09 & \red 43.6 $\pm$ 0.50 & \green 51.9 $\pm$ 0.49 \\
\bottomrule
\end{tabular}
}
\end{table*}

We perform experiments for each stage of our algorithm. In the first stage, we compare among various methods to estimate the unlabeled class distribution $P(Y|A=0)$, showing that SimPro + DR performs well. In the second stage, we freeze the unlabeled class distribution, using our best estimator  SimPro + DR, and plug it into 2 SOTA semi-supervised learning algorithms, SimPro and BOAT~\cite{boat}. We show that this simple procedure improves the existing methods, and is even capable of improving the original SimPro when used for both stages.


% \textbf{Datasets} We adopt 4 standard benchmarks used frequently in other semi-supervised learning work: CIFAR-10, CIFAR-100~\cite{cifar}, STL-10~\cite{stl10} and Imagenet-127~\cite{cossl}. To match our RTSSL setting, we create long-tailed labeled and unlabeled sets from CIFAR-10 and CIFAR-100. Specifically, we use $\gamma_l$ and $n_1$ to denote the imbalance ratio and the head class's number of samples of the labeled data, the remaining class's size is computed as $n_c = n_1 \times \gamma_l^{-\frac{c-1}{C-1}}$ and likewise, $\gamma_u$ and $m_1$ of the unlabeled data. For CIFAR-10, we fix $n_1=500$ and $m_1=4000$. We test 2 different configurations $\gamma_l=\gamma_c=150$ and $\gamma_l=\gamma_c=100$. We further permute classes the unlabeled sets in 5 ways: consistent, uniform, reversed, middle and headtail, similar to \cite{simpro} and visualized in figure~\ref{fig:distribution}, which results in 10 different datasets in total. Similarly for CIFAR-100, we fix $n_1=500$ and $m_1=4000$, use 2 configurations $\gamma_l=\gamma_c=20$ and $\gamma_l=\gamma_c=10$, and permute the classes in 5 ways, resulting in 10 datasets as well. For STL-10, the unlabeled set has no ground truth labels, therefore we use all samples in the head class and set the imbalance ratio $\gamma_l$ to $10$ or $20$. Imagenet-127 is a naturally long-tailed dataset with 127 classes. We train on 32x32 and 64x64 image resolutions following ~\cite{cossl}.


\textbf{Datasets} We evaluate our method on four standard semi-supervised learning benchmarks: CIFAR-10, CIFAR-100~\cite{cifar}, STL-10~\cite{stl10}, and Imagenet-127~\cite{cossl}. To simulate RTSSL, we construct long-tailed labeled and unlabeled sets for CIFAR-10 and CIFAR-100. The labeled data follows an imbalance ratio $\gamma_l$ with head class size $n_1$, while the remaining class sizes are computed as $n_c = n_1 \times \gamma_l^{-\frac{c-1}{C-1}}$. The unlabeled data follows a similar setup with $\gamma_u$ and $m_1$.  

For CIFAR-10, we set $n_1 = 500$, $m_1 = 4000$, and test two configurations: $\gamma_l = \gamma_u = 150$ and $\gamma_l = \gamma_u = 100$. We generate 10 datasets by permuting the unlabeled class distributions in five ways: \textit{consistent, uniform, reversed, middle}, and \textit{head-tail}, as in~\cite{simpro}. CIFAR-100 follows the same setup with $n_1 = 50$, $m_1 = 400$, and $\gamma_l, \gamma_u$ values of 20 and 10.  

For STL-10, where unlabeled data lacks ground-truth labels, we use all head-class samples and set $\gamma_l$ to 10 or 20. Imagenet-127 is naturally long-tailed with 127 classes, and we train on 32$\times$32 and 64$\times$64 resolutions as in~\cite{cossl}.


\paragraph{Training.} We follow the implementation and hyperparameter settings of \cite{simpro}. We defer these details in \cref{subsec:training-setting}. One important exception is that for Imagenet-127, we use the smaller Wide ResNet-28-2 in stage 1 and the larger ResNet-50 for stage 2, to demonstrate that a smaller model is sufficient for distribution estimation.


\begin{table}[t]
\small
\centering
\caption{Top-1 Accuracy (\%) on STL-10. Our two-stage algorithms improves both SimPro and BOAT for both settings.}
\label{tab:stl10-acc}
% \resizebox{\linewidth}{!}{
\begin{tabular}{lcc}
\toprule
Method & $\gamma_l=10$ & $\gamma_l=20$ \\ \hline
Supervised & 73.9 $\pm$ 0.57 & 70.4 $\pm$ 0.95 \\
\midrule
MLE & 67.6 $\pm$ 0.57 & 58.9 $\pm$ 4.05 \\
\midrule
EM & 84.9 $\pm$ 0.14 & 83.6 $\pm$ 0.25 \\
\midrule
SimPro & 82.4 $\pm$ 1.57 & 80.5 $\pm$ 0.96 \\
SimPro+ & \green 83.9 $\pm$ 0.76 & \green 82.7 $\pm$ 0.86 \\
\midrule
BOAT & 83.8 $\pm$ 0.20 & 82.0 $\pm$ 0.34 \\
BOAT+ & \green 84.1 $\pm$ 0.38 & \green 82.4 $\pm$ 0.10 \\
\bottomrule
\end{tabular}
\end{table}















\begin{table}[t]
% \setlength{\tabcolsep}{3.5pt}
\small
\centering
\caption{Top-1 Accuracy (\%) on Imagenet-127. Our two-stage approach improves both SimPro and BOAT for both resolutions.}
\label{tab:imagenet-127-acc}
% \resizebox{\linewidth}{!}{
\begin{tabular}{lcc}
\toprule
Method & $32 \times 32$ & $64 \times 64$ \\ \hline
SimPro & 54.8 & 63.7 \\
SimPro+ & \green 55.1 & \green 64.2 \\
\midrule
BOAT & 51.6 & 58.7 \\
BOAT+ & \green 52.0 & \green 59.2 \\

\bottomrule
\end{tabular}
% }
\end{table}


\begin{table}[t]
% \setlength{\tabcolsep}{3.5pt}
\small\centering
\caption{Total Variation Distance on Imagenet-127}
\label{tab:imagenet-127-tv}
% \resizebox{\linewidth}{!}{
\begin{tabular}{cccc}
\toprule
Method & Estimator & $32 \times 32$ & $64 \times 64$ \\ \hline
MLE & IPW  & 0.103 $\pm$ 0.034 & 0.051 $\pm$ 0.000 \\
MLE & OR  & 0.153 $\pm$ 0.052 & 0.041 $\pm$ 0.000 \\
MLE & DR  & \green 0.100 $\pm$ 0.029 & \green 0.075 $\pm$ 0.003 \\
\midrule
EM & IPW  & 0.141 $\pm$ 0.006 & 0.163 $\pm$ 0.010 \\
EM & OR  & 0.205 $\pm$ 0.006 & 0.236 $\pm$ 0.011 \\
EM & DR  & \green 0.024 $\pm$ 0.001 & \green 0.042 $\pm$ 0.004 \\
\midrule
SimPro & IPW  & 0.041 $\pm$ 0.012 & 0.224 $\pm$ 0.040 \\
SimPro & OR  & 0.036 $\pm$ 0.014 & 0.291 $\pm$ 0.079 \\
SimPro & DR  & \green 0.017 $\pm$ 0.000 & \green 0.037 $\pm$ 0.004 \\
\bottomrule
\end{tabular}
% }
\end{table}

\subsection{Better results on label distribution} 
\label{subsec:label}
We have mentioned various ways throughout the papers to estimate the unlabeled class distribution. In what follows, method consists of a model, which is how the learning is done, and an estimator, which is how the final distribution is estimated using parameters learned from the model.

%\begin{enumerate}
%\item 
\noindent
\textbf{Supervised}. The model is trained on the labeled set only and used to estimate the unlabeled class distribution \cite{unifiedlabelshift}. 2 successful estimators for this setting are \textbf{RLLS} \cite{rlls} and \textbf{MLLS} \cite{mlls}. 

%\item 
\noindent\textbf{MLE}. The model is trained by directly maximizing the likelihood \cref{eq:likelihood}. We also use the decomposition $P(Y|X)$ and $P(A|Y)$, and write the unlabeled term as $P(A=0, X) = \sum_{c} P(Y=c|X) P(A=0|Y=c)$, which enables gradient descent training on these parameters. This is also the MLE method to estimate $P(A|Y)$ in \cite{arelabelsinformative}.

%\item 
\noindent\textbf{EM}. We further test the EM algorithm in \cref{subsec:em}. In particular we also use strong and weak augmentations similar to FixMatch, but not the pseudo labeling operator. Confidence thresholding removes the soft predictions of the non-dominant classes, which may be better to keep since our target of the first stage is the global class statistics. We also try 3 estimators on this model.

%\item 
\noindent\textbf{SimPro} \cite{simpro} can be seen as our previous EM but also with FixMatch's confidence thresholding and logit adjustment loss in \cref{subsec:simpro}. Confidence thresholding is a powerful regularization technique that encodes the assumption that classes are well separated \cite{entropyminimization}, but can introduce bias to the estimation, which justifies the use of DR.
%\end{enumerate}

% For semi-supervised methods MLE, EM and SimPro, as we now have additional information on the missingness mechanism, we can use 3 estimators OR, IPW and DR presented in \cref{subsec:2-stage}


Results on \cref{tab:cifar10-tv} presents the performance of various models and estimators on CIFAR-10. We can see that SimPro + DR performs best. In contrast, SimPro + OR, SimPro's original way of estimating $P(Y|A=0)$, and SimPro + IPW tend to underperform EM on the consistent and uniform datasets. The consistent setting is worth noting, since it arises when data is sampled uniformly at random for labeling,  representative of a large number of real world situations. EM is competitive to SimPro as well even without pseudo labeling, but overall we found this regularization to offer significant gains in the reversed, middle and head-tail settings. Finally, Supervised with either MLLS or RLLS estimators performs much worse than the semi-supervise methods.

\cref{tab:imagenet-127-tv} aligns with the observations  made in \cref{tab:cifar10-tv}. In particular, SimPro + DR is the best method for class distribution estimation of the much larger Imagenet-127. We also found that a small neural network and a small image resolution is sufficient for the distribution estimation of the much larger dataset Imagenet-127. This matches our intuition that the finite-dimensional parameter is easier to learn.

\cref{tab:cifar100-tv} shows that most methods understandably struggle to estimate the class distributions in CIFAR-100. This is because there are few samples in each class, the head class has 10 times less samples while the number of classes multiplies 10 times compared to CIFAR-10. We see here that SimPro + DR is not the best method, but the relative gap between estimators are small.

% Among the models, the supervised baseline do not perform well even in the consistent setting, showing that when unlabeled data is available during training, learning from them can be valuable for class distribution estimation, especially in the cases with little labeled data like ours. Both the MLE and supervised models perform badly on the reversed, middle and head-tail settings

% Among the estimators, we see that DR boosts the performance of SimPro and EM in CIFAR-10, and of all semi-supervised models in Imagenet-127. It does not improve MLE on CIFAR-10, and it does not improve on CIFAR-100. However, for most of the time, the decrease is not much. In constrast, IPW estimators can be significantly worse, for example in the reversed setting of CIFAR-10, where the distance is $0.254$ for $\gamma_l=150$ and $0.233$ for $\gamma_l=100$, compared to OR's 0.040 and 0.059. 

% Both the MLE and supervised models perform badly on the reversed, middle and head-tail settings. EM does a decent job, though not as well as SimPro, on all 5 distribution settings of CIFAR-10. However, on Imagenet-127, EM without DR performs worse than MLE.

% We note that the performance on DR is similar to OR in these cases, showing that DR has a double robustness property. While IPW only relies on the finite-dimensional $P(A|Y)$, which intuitively is easy to estimate, we found that the inverse probability weight can nevertheless be unstable when some probabilities are small, and this is where DR shows its strength by combining both IPW and OR.



\subsection{Two-stage algorithm improves accuracy}

In the second stage of our algorithm, we freeze our estimation and plug it in SimPro and BOAT. We denote SimPro+ and BOAT+ for algorithms that use our first stage estimate.



\cref{tab:cifar10-acc} shows that for CIFAR-10 SimPro+ and BOAT+ improve over their original versions across most settings, leading to large improvements in both the consistent and middle class distribution settings. In particular, our two-stage approach improves SimPro in 9 / 10 settings and BOAT in 8 / 10 settings.
We also observe consistent improvements ove both base algorithms, SimPro and BOAT, for several other datasets. \cref{tab:stl10-acc} demonstrates improvements for 2 / 2 class imbalance ratios in STL-10 and \cref{tab:imagenet-127-acc} for 2 / 2  different resolutions of ImageNet-127. 


We also evaluate on CIFAR-100 for multiple unlabeled  class distribution settings and with mediocre class label distribution estimates in stage 1, demonstrate no degradation in accuracy in stage 2. As shown in \cref{tab:cifar100-acc}, the two stage algorithm with a mediocre stage 1 estimation leads to parity with the baseline. Stage 2 provides small improvements in 5 / 10 settings for SimPro and in 4 / 10 (with 2 ties) for BOAT.


\subsection{Ablation Study: Alternative implementations.}
\label{subsec:ablation-1}
In this section, we ablate on our 2-stage choice. Specifically, we consider 2 alternative implementations:
\paragraph{\textbf{Doubly-robust risk}}  
This approach is \cite{arelabelsinformative, onnonrandommissinglabels}, as discussed in \cref{sec:background}. we consider the doubly-robust risk as our training loss. We use the missingness mechanism estimation from stage-1 of SimPro+ for fair comparison. \cref{eq:dr-risk} is used for training in which the pseudo-labeling operators can be applied straightforwardly. More detail in \cref{subsec:dr-risk}
\paragraph{\textbf{Batch-update doubly-robust $P(Y|A)$}} Different from SimPro+, here we remove the first stage and instead update our doubly robust estimation of the unlabeled class distribution using a moving average of the batch statistics.

\cref{tab:cifar10-ablation-1} shows that the batch-update version of SimPro+ is significantly worse on the consistent and uniform settings, while the doubly-robust risk is worst overall, especially in the reversed setting where $P(A|Y)$ is very small for the labeled tail classes, causing instability issues during training. In conclusion, our 2-stage approach is the best.


\begin{table}[t]
\small
\centering
\caption{Top-1 Accuracy (\%) on CIFAR-10. We compare our 2-stage SimPro+ with 1) an 1-stage alternative that updates and uses the doubly-robust estimation on-the-fly and 2) SimPro with doubly-robust risk. We use $\gamma_l=150$. {\green green} color indicates that our method performs best.}
\label{tab:cifar10-ablation-1}
\resizebox{\linewidth}{!}{
\begin{tabular}{lccccc}
\toprule
Method & consistent & uniform & reversed & middle & headtail\\ \hline
SimPro+ & \green 77.8 & \green 93.7 & \green 83.3 & \green 79.2 & \green 81.3 \\
batch-update & 71.9 & 91.4 & 82.6 & 78.6 & 81.2 \\
DR-risk & 72.1 & 89.8 & 67.1 & 75.6 & 79.5 \\
\bottomrule
\end{tabular}
}
\end{table}
\section{Related Work}

\paragraph{Commonsense Reasoning Evaluation} 
There are numerous benchmarks and datasets for commonsense reasoning, most of which are in English. 
%Some work focused on evaluating general commonsense knowledge, such as HellaSwag \cite{zellers2019hellaswag}, CommonsenseQA \cite{talmor2019commonsenseqa}, OpenBookQA \cite{OpenBookQA2018}, and WSC \cite{levesque2012winograd}. 
Some studies focus on evaluating general commonsense knowledge \cite{zellers2019hellaswag,talmor2019commonsenseqa,OpenBookQA2018}. 
%Others target specific aspects of commonsense reasoning, including temporal commonsense with MCTACO \cite{zhou2019going}, physical commonsense with PIQA \cite{bisk2020piqa}, social commonsense with SocialIQA \cite{sap2019socialiqa}, numerical commonsense with NumerSense \cite{lin2020birds}, and scientific commonsense with ARC \cite{clark2018think} and QASC \cite{khot2020qasc}. Notably, most of these datasets are in English. 
Others target specific aspects of commonsense reasoning\cite{zhou2019going,bisk2020piqa,sap2019socialiqa,lin2020birds,clark2018think,khot2020qasc}.
There are some Chinese datasets for commonsense reasoning \cite{sun2024benchmarking,shi2024corecode}. 
For instance, CHARM \cite{sun2024benchmarking} distinguishes between global commonsense and Chinese-specific commonsense but includes only a limited number of everyday commonsense cases. 
However, evaluations aimed at assessing the robustness of commonsense reasoning are still understudied. 

\paragraph{Datasets on Different Reasoning Forms}
There are several datasets relevant to our variant design. For reverse reasoning, ART \cite{DBLP:conf/iclr/BhagavatulaBMSH20}, $\delta$-NLI \cite{DBLP:conf/emnlp/RudingerSHBFBSC20}, and CLUTRR \cite{DBLP:conf/emnlp/SinhaSDPH19} explore different reasoning directions. FCR \cite{DBLP:journals/corr/abs-2204-07408} and NatQuest \cite{ceraolo2024analyzinghumanquestioningbehavior} evaluate causal reasoning, while TimeTravel \cite{DBLP:conf/emnlp/QinBHBCC19} focuses on counterfactual scenario refinement. Additionally, PoE \cite{balepur2024s} assesses reasoning involving negation. 
However, not all these datasets focus on commonsense reasoning, nor are they structured by original questions and their variants. Furthermore, they typically target limited reasoning types. Lastly, our dataset is large-scale and covers diverse commonsense knowledge. 

\paragraph{Robustness and Consistency in LLMs} 
Early work focuses on adversarial attacks, with developing evaluation methods for reading comprehension systems \cite{jia2017adversarial}, followed by universal adversarial triggers \cite{wallace2019universal}. The field then expands to examine spurious correlations, with revealing how models often exploit superficial patterns rather than engaging in genuine reasoning \cite{branco2021shortcutted,geirhos2020shortcut}. And \citealp{ross2022does} investigates whether self-explanation can mitigate these spurious correlations. Coherence and consistency evaluation advances through classifier assessment methods \cite{storks2021beyond} and analysis of accuracy-consistency trade-offs \cite{johnson2023much}. While these studies primarily address model robustness against adversarial attacks or spurious correlations, our work takes a novel approach by examining robustness in reasoning forms.
%, specifically focusing on how models maintain consistent reasoning when presented with different reasoning forms of the same commonsense knowledge.
%\paragraph{Dataset Construction by LLM} 
%Research indicates that when LLMs are utilized for dataset generation, the resulting datasets are more accurate and fluent \cite{lu2022fantastically, min-etal-2022-rethinking} than those created by crowd-sourced annotators. Furthermore, generating datasets with LLMs is significantly more cost-effective than using crowd-sourced annotations \cite{liu2022wanli, wiegreffe2022reframing, west2022symbolic}. Hence, we generate our benchmark by LLM in-context learning.

% \paragraph{In-Context Learning} 
% As LLMs become more widely used, in-context learning (\citealp{brown2020language}; \citealp{ouyang2022training}; \citealp{min-etal-2022-rethinking}) has emerged as the primary approach for executing various tasks. This method involves supplying LLMs with textual instructions and examples and removes the necessity for parameter modifications. Research indicates that when LLMs are utilized for dataset generation, the resulting datasets are more accurate and fluent (\citealp{lu2022fantastically}; \citealp{min-etal-2022-rethinking}) than those created by crowd-sourced annotators. Furthermore, generating datasets with LLMs is significantly more cost-effective than using crowd-sourced annotations (\citealp{liu2022wanli}; \citealp{wiegreffe2022reframing}; \citealp{west2022symbolic}). Hence, we have decided to construct our benchmark by over-generating data using in-context learning and employing human annotators for filtering to ensure high efficiency and high quality.
% Moxin: 这部分应该改成用LLM 生成dataset的工作?
% 

\section{Conclusion and future directions} \label{sec:conclusion}

In this paper we proposed a nested MLMC framework that offers important computational savings by performing most calculations in low precision and exploiting approximate random normal variables for the low precision path calculations. The low precision calculations could be performed in fixed precision on an FPGA for greater efficiency, and we suggested a procedure to optimise the bit-widths of every variable at each Monte Carlo level. This is an important improvement over previous mixed precision MLMC frameworks which held the lower precision fixed \cite{Rounding_error_oliver} or defined uniform bit-width at every level heuristically \cite{brugger2014mixed}. Our numerical results suggest that for the first levels our procedure reduces the cost at these levels by a factor 5 or 7. Hence the overall savings are significant since most paths are calculated on the first levels. Our approach would be even more efficient for the Milstein scheme because its higher order strong convergence leads to a greater proportion of the computational costs being on the coarsest levels.

The next stage of the research project will be to implement the RNG methods and the nested framework on FPGAs to determine the hardware requirements and confirm the extent of the computational savings. It would also be good to compare the performance benefits to using half-precision floating point arithmetic on GPUs or CPUs for the low-accuracy computations.




\bibliography{anthology,custom}
\appendix
\begin{table*}[htbp]
    \small
    \centering
    \begin{tabular}{lcccc}
    \toprule
    \multirow{2}{1cm}{\textbf{Dataset}} & \multirow{2}{1.1cm}{\textbf{Context}} & \multicolumn{3}{c}{\textbf{Queries}} \\
     & &\textbf{Chunk-Level} & \textbf{Sentence-Level} & \textbf{Constraints-Based}\\\midrule
     MultiHop-RAG & 7,724 & 72,090 & 472,193 & 51,212\\
     AllSides & 645 & 6,313 & 173,898 & 6,091 \\
     AGNews & 1,050 & 10,355 & 80,524 & 20,875 \\
     NQ & 98,748 & 1,459,031 & - & - \\\bottomrule %
    \end{tabular}
    \caption{Dataset Statistics}
    \label{tab:dataset_stats}
\end{table*}

\begin{table}[htbp]
    \small
    \centering
    \begin{tabular}{lp{4.8cm}}
    \toprule
      \textbf{Dataset} & \textbf{Attributes} \\\midrule
        MultiHop-RAG & author, publish time, source, category, title \\
        AllSides & political polarity \\
        AGNews & location, topic \\\bottomrule
    \end{tabular}
    \caption{Attributes used in each dataset for constraints-based query generation.}
    \label{tab:dataset_attributes}
\end{table}

\section{Details of Experiment Setup}
We use \texttt{Mistral-7B-Instruct-v0.3} as the base model for generative retrieval with the semantic identifier, while use \texttt{Mistral-7B-v0.3} as the base model for atomic identifier as it is closer to a classification setting.

For supervised fine-tuning, we train the models with 2 epochs, with a learning rate of 2e-5 and a warmup ratio of 0.1. The batch size is set as 256. We use sequence packing to put multiple examples in one forward pass~\citep{DBLP:journals/jmlr/RaffelSRLNMZLL20}. We use \texttt{bfloat16} for our training.

For preference learning, we mainly conduct experiments on MultiHop-RAG and NQ with semantic identifiers. We train the models with 1 epoch. The learning rate is set as 1e-7, batch size is set as 64, $\beta$ is set as 0.5, $\alpha$ is set as 1.0. 

The training infrastructure includes TRL~\citep{vonwerra2022trl}, Accelerate~\citep{accelerate}, Transformers~\citep{wolf-etal-2020-transformers}, DeepSpeed~\cite{DBLP:conf/kdd/RasleyRRH20} and FlashAttention-2~\citep{DBLP:conf/iclr/Dao24}. We use 8x Nvidia A100-SXM4-40GB for our experiments. Each training or inference procedure can be completed in 1 day.

Statistics of the numbers of the documents, different synthetic queries can be found in Table~\ref{tab:dataset_stats}. Attributes used for constraints-based synthetic queries can be found in Table~\ref{tab:dataset_attributes}. All the experiment results are obtained with single run.
\label{app:data_specific_setup}
\subsection{MultiHop-RAG}
On MultiHop-RAG, we split the documents into chunks with maximum length of 256 without overlap and conduct retrieval on individual chunks. For synthetic query generation, $m_c$, $m_s$ and $m_i$ are set as 10, and the temperature for LLM inference on synthetic data generation is set as 0.7. We interleave the Context2ID and Query2ID data as the full dataset for model supervised fine-tuning. The maximum sequence length is set as 700. For synthetic queries for preference learning, we ask the LLM to generate 10 queries. We perform the retrieval with beam size as 10 and retrieve the top-10 candidates for each query to construct the candidate pairs.
\subsection{AllSides}
On AllSides, we conduct document-level retrieval. For synthetic query generation, $m_c$, $m_s$ and $m_i$ are set as 10, and the temperature for LLM inference on synthetic data generation is set as 0.7. For Context2ID data, as there are some long documents in the corpus, we will split the long context into chunks with maximum length of 256 without overlap. The Context2ID data is constructed to use all chunks in the document to predict its corresponding document identifier. We interleave the Context2ID and Query2ID data as the full dataset for model supervised fine-tuning. The maximum sequence length is set as 700.
\subsection{AGNews}
On AllSides, we conduct document-level retrieval. For synthetic query generation, $m_c$, $m_s$ and $m_i$ are set as 10, and the temperature for LLM inference on synthetic data generation is set as 0.7. Queries constructed by \citet{DBLP:journals/corr/abs-2405-02714} uses two different perspectives. The first perspective is either the location of the desired news or the topic, while the second perspective is that the news is similar to another given news in the query. As we mentioned Section~\ref{sec:experiment_setup}, we replace the second perspective with the another field so that each query consists of both location and topic perspectives. The topic and location information used for instruction-based synthetic query generation is extracted with Mixtral 8x7b. We interleave the Context2ID and Query2ID data as the full dataset for model supervised fine-tuning. The maximum sequence length is set as 700.
\subsection{NQ}
On NQ, we conduct document-level retrieval. We use the document prefixes from~\cite{DBLP:conf/icml/KishoreWLAW23} to produce the semantic identifiers. For synthetic query generation, we perform truncation on pages when they are too long so that we always have at least 1024 token space for model output. We set $m_c$ as 15 and temperature as 0.7. We do not include sentence-level synthetic queries as the number of those queries are too large to be included in training within a reasonable time. Instead, we include sentence-level Context2ID as the approximation, and use the sentences from the document prefixes from~\cite{DBLP:conf/icml/KishoreWLAW23} to predict corresponding document identifiers. In NQ, we have high quality human annotated training queries, which we also include as part of the Query2ID data and therefore we do not include instruction-based synthetic queries. We concatenate the Context2ID and Query2ID data as the full dataset for model supervised fine-tuning, as interleaving will produce a much larger dataset that cannot be trained within a reasonable time. The maximum sequence length is set as 450. For synthetic queries for preference learning, we also perform truncation as for supervised fine-tuning, and ask the LLM to generate 10 queries. As the generated query number is quite large for inference, we use the first 2 generated queries for each documents for preference learning. We perform the retrieval with beam size as 10 and retrieve the top-10 candidates for each query to construct the candidate pairs.

\section{Results on Comparison with Off-The-Shelf Retrieval Models}
\label{sec:detailed_dense_comparison}
The detailed results for each dataset are shown in Table~\ref{tab:dense_retrieval_comparison}. We run the retrieval models on MultiHop-RAG, NQ and AGNews to collect the results, and adopt the results of AllSides from \citet{DBLP:journals/corr/abs-2405-02714}.


\begin{table*}[htbp]
    \centering
    \small
    \begin{subtable}[t]{\textwidth}
        \centering
        \begin{tabular}{lcccc}
        \toprule
        \textbf{Model} & \textbf{HIT@4} & \textbf{HIT@10} & \textbf{MAP@10} & \textbf{MRR@10}\\
         \midrule
         BM25 & 64.35 & 78.31 & \bf 26.30 & \bf 58.32 \\
         bge-large-en-v1.5 & 58.80 & 78.36 & 19.96 & 42.57 \\
         Contriever-msmarco & 55.25 & 75.08 & 19.28 & 40.69 \\
         E5-mistral-7b-instruct & 54.01 & 79.56 & 19.11 & 40.77 \\
         GTE-Qwen2-7B-instruct & 63.24 & 83.55 & 22.02 & 47.50 \\
         ours & \bf 71.88 & \bf 89.80& 26.23& 54.94\\
         \bottomrule
        \end{tabular}
        \caption{MultiHop-RAG}
    \end{subtable}
    \vspace{0.2cm}
    
    \begin{subtable}[t]{\textwidth}
        \centering
        \begin{tabular}{lcccc}
        \toprule
        \textbf{Model} & \textbf{HIT@1} & \textbf{HIT@5} & \textbf{HIT@10} & \textbf{MRR@10}\\
         \midrule
         BM25 & 32.82 & 53.70 & 60.92 & 42.45 \\
         bge-large-en-v1.5 & 55.59 & 76.58 & 81.75 & 64.45 \\
         Contriever-msmarco & 53.79 & 76.16 & 81.69 & 63.36 \\
         E5-mistral-7b-instruct & 59.07 & 80.08 & 85.28 & 68.11 \\
         GTE-Qwen2-7B-instruct & 60.45 & 80.87 & 85.72 & 69.30 \\
         ours & \bf 71.22& \bf 87.41& \bf 89.97& \bf 78.14\\
         \bottomrule
        \end{tabular}
        \caption{NQ}
    \end{subtable}
    \vspace{0.2cm}

    \begin{subtable}[t]{0.48\textwidth}
        \centering
        \begin{tabular}{lccc}
        \toprule
        \textbf{Model} & \textbf{HIT@1} & \textbf{HIT@5} & \textbf{HIT@10}\\
        \midrule
        BM25 & 5.86 & 26.85 & 36.42 \\
        bge-large-en-v1.5 & 6.94 & 27.32 & 34.11\\
        Contriever-msmarco & 6.64 & 25.77 & 38.43 \\
        E5-mistral-7b-instruct & 8.18 & 28.24 & 39.82\\
        GTE-Qwen2-7B-instruct & 9.11 & 34.11 & 49.07 \\
        ours & \bf 14.20& \bf 38.58& \bf 51.85\\
        \bottomrule
        \end{tabular}
        \caption{AllSides}
    \end{subtable}
    ~~\quad
    \begin{subtable}[t]{0.48\textwidth}
        \centering
        \begin{tabular}{lccc}
        \toprule
        \textbf{Model} & \textbf{HIT@1} & \textbf{HIT@5} & \textbf{HIT@10}\\
        \midrule
        BM25 & 38.70 & 67.47 & 77.63 \\
        bge-large-en-v1.5 & 54.14 & 80.57 & 86.53\\
        Contriever-msmarco & 52.69 & 80.40 & 85.79 \\
        E5-mistral-7b-instruct & 57.32 & \bf 85.90 & \bf 88.98 \\
        GTE-Qwen2-7B-instruct & 57.65 & 83.37 & 88.57\\
        ours & \bf 62.19& 83.78& 88.24\\
        \bottomrule
        \end{tabular}
        \caption{AGNews}
    \end{subtable}

    \caption{Comparisons to Off-The-Shelf Retrieval Models Across Datasets}
    \label{tab:dense_retrieval_comparison}
\end{table*}


\section{LLM Prompts}

\subsection{Prompts for Keywords Generation}
Figure~\ref{fig:keyword_generation} shows the prompt for generating a series of keywords as the semantic document identifier.

\begin{figure}[h]
\begin{tcolorbox}[title=\textbf{Keywords Generation Prompt}]
Summarize the following context with meaningful keywords representing different important information in the context. Your output should only consist a list of keywords in Markdown format, where each line starts with the dash "-" followed by the keywords.
\\
\\
\# Context:\\
\{context\}\\
\\
\# Keywords:
\end{tcolorbox}
\caption{Prompt for keywords-based document identifier generation.}
\label{fig:keyword_generation}
\end{figure}


\subsection{Prompts for Query Generation}
Figure~\ref{fig:query_generation_prompts} shows the prompts used to generate various types of synthetic queries, including chunk- and sentence-level queries, constructions-based queries, and question-answer pairs used at the preference learning stage.









\begin{figure*}[htbp]
\centering
\begin{subfigure}[t]{\textwidth}
\begin{tcolorbox}[title=\textbf{Query Generation Prompt}]
Your task is to generate a relevant and diverse set of \{num\_sequences\} questions that can be answered by the provided context. The questions are to be used by a retriever to retrieve the article from a large corpus. Your output should be a list of unordered in Markdown format, where each line starts with dash "-" followed by the question.
\\
\\
\# Context:\\
\{context\}\\
\\\
\# Output:
\end{tcolorbox}
\label{fig:query_generation}
\end{subfigure}

\begin{subfigure}[t]{\textwidth}
\begin{tcolorbox}[title=\textbf{Constraints-based Query Generation Prompt}]
Your task is to generate a diverse set of \{num\_sequences\} questions given a context with metadata. The generated questions should be answerable by the provided context. The questions are to be used by a retriever to retrieve the article from a large corpus. In addition, the question MUST be composed with at least one metadata filtering requirement.\\
\\
\textbf{\# MultiHop-RAG} \\
For example, if the source of the article is "New York Times", you can generate questions that specifically ask for certain information from "New York Times". You should generate questions with different metadata.\\
\textbf{\# AllSides and AGNews} \\
For example, if the source of the political polarity is "left", you can generate questions that specifically ask for certain information from "left-wing" source.\\
\\
DO NOT use "the context" or "the article" in any generated queries or answers.\\
DO NOT use pronoun "this" in any generated queries or answers.\\
DO NOT leak any information in this instruction.\\


Your output should be a list of unordered in Markdown format, where each line starts with dash "-" followed by the question. You do not need to provide the answer.\\
\\
\# Metadata\\
\{metadata\}\\
\\
\# Context\\
\{context\}\\
\\
\# Output:
\end{tcolorbox}
\label{fig:instruct_query_generation}
\end{subfigure}

\begin{subfigure}[t]{\textwidth}
\begin{tcolorbox}[title=\textbf{Query-Answer Pair Generation Prompt}]
Your task is to generate a relevant and diverse set of less than \{num\_sequences\} search engine query and answer pairs given a context.\\
The queries should be similar to what people use with search engine to find the given context from a large corpus. The answers are expeced to be a short phrase.\\
You should make the queries as difficult as possible, but they should be answerable by the given context.\\
\\
Do not use "the context" or "the article" in any generated queries or answers.\\
Do not use pronoun "this" in any generated queries or answers.\\
Do not leak any information in this instruction.\\
\\
Your output should be a list of unordered items in Markdown format, where each item starts with dash "-", followed by "Query:" and the generated query, and then "Answer:" with the corresponding answer.\\
\\
\# Context\\
\{context\}\\
\\
\# Output:
\end{tcolorbox}
\label{fig:qa_generation}
\end{subfigure}
\caption{Prompts for different types of synthetic query generation.}
\label{fig:query_generation_prompts}
\end{figure*}

\end{document}
