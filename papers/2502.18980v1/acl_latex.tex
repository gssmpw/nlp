% This must be in the first 5 lines to tell arXiv to use pdfLaTeX, which is strongly recommended.
\pdfoutput=1
% In particular, the hyperref package requires pdfLaTeX in order to break URLs across lines.

\documentclass[11pt]{article}

% Change "review" to "final" to generate the final (sometimes called camera-ready) version.
% Change to "preprint" to generate a non-anonymous version with page numbers.
% \usepackage[review]{acl}
\usepackage[preprint]{acl}

% Standard package includes
\usepackage{times}
\usepackage{latexsym}

% For proper rendering and hyphenation of words containing Latin characters (including in bib files)
\usepackage[T1]{fontenc}
% For Vietnamese characters
% \usepackage[T5]{fontenc}
% See https://www.latex-project.org/help/documentation/encguide.pdf for other character sets

% This assumes your files are encoded as UTF8
\usepackage[utf8]{inputenc}

% This is not strictly necessary, and may be commented out,
% but it will improve the layout of the manuscript,
% and will typically save some space.
\usepackage{microtype}

% This is also not strictly necessary, and may be commented out.
% However, it will improve the aesthetics of text in
% the typewriter font.
\usepackage{inconsolata}

%Including images in your LaTeX document requires adding
%additional package(s)
\usepackage{graphicx}
\usepackage{booktabs}
\usepackage{multirow}
\usepackage{makecell}
\usepackage{amsmath}
\usepackage{amssymb}
\usepackage[most]{tcolorbox}
% \usepackage{xcolor}


% If the title and author information does not fit in the area allocated, uncomment the following
%
%\setlength\titlebox{<dim>}
%
% and set <dim> to something 5cm or larger.

\usepackage{xspace}
\newcommand{\benchmark}{PEToolBench\xspace} % 加空格
\newcommand{\framework}{PEToolLLaMA\xspace}
% \newcommand{\benchmark}{LaMP}


\title{PEToolLLM: Towards Personalized Tool Learning in \\Large Language Models}

% Author information can be set in various styles:
% For several authors from the same institution:
% \author{Author 1 \and ... \and Author n \\
%         Address line \\ ... \\ Address line}
% if the names do not fit well on one line use
%         Author 1 \\ {\bf Author 2} \\ ... \\ {\bf Author n} \\
% For authors from different institutions:
% \author{Author 1 \\ Address line \\  ... \\ Address line
%         \And  ... \And
%         Author n \\ Address line \\ ... \\ Address line}
% To start a separate ``row'' of authors use \AND, as in
% \author{Author 1 \\ Address line \\  ... \\ Address line
%         \AND
%         Author 2 \\ Address line \\ ... \\ Address line \And
%         Author 3 \\ Address line \\ ... \\ Address line}

\author{Qiancheng Xu$^{1}$, Yongqi Li$^{1\dagger}$, Heming Xia$^{1}$, Fan Liu$^{2}$, Min Yang$^{3}$, Wenjie Li$^{1}$ \\
% \thanks{Corresponding author.}
$^{1}$ The Hong Kong Polytechnic University \quad
$^{2}$ National University of Singapore \\
$^{3}$ Shenzhen Institutes of Advanced Technology, Chinese Academy of Sciences \\
\texttt{\{qiancheng.xu, he-ming.xia\}@connect.polyu.hk} \\
\texttt{liyongqi0@gmail.com} \quad
\texttt{cswjli@comp.polyu.edu.hk}
}

% \author{First Author \\
%   Affiliation / Address line 1 \\
%   Affiliation / Address line 2 \\
%   Affiliation / Address line 3 \\
%   \texttt{email@domain} \\\And
%   Second Author \\
%   Affiliation / Address line 1 \\
%   Affiliation / Address line 2 \\
%   Affiliation / Address line 3 \\
%   \texttt{email@domain} \\}

%\author{
%  \textbf{First Author\textsuperscript{1}},
%  \textbf{Second Author\textsuperscript{1,2}},
%  \textbf{Third T. Author\textsuperscript{1}},
%  \textbf{Fourth Author\textsuperscript{1}},
%\\
%  \textbf{Fifth Author\textsuperscript{1,2}},
%  \textbf{Sixth Author\textsuperscript{1}},
%  \textbf{Seventh Author\textsuperscript{1}},
%  \textbf{Eighth Author \textsuperscript{1,2,3,4}},
%\\
%  \textbf{Ninth Author\textsuperscript{1}},
%  \textbf{Tenth Author\textsuperscript{1}},
%  \textbf{Eleventh E. Author\textsuperscript{1,2,3,4,5}},
%  \textbf{Twelfth Author\textsuperscript{1}},
%\\
%  \textbf{Thirteenth Author\textsuperscript{3}},
%  \textbf{Fourteenth F. Author\textsuperscript{2,4}},
%  \textbf{Fifteenth Author\textsuperscript{1}},
%  \textbf{Sixteenth Author\textsuperscript{1}},
%\\
%  \textbf{Seventeenth S. Author\textsuperscript{4,5}},
%  \textbf{Eighteenth Author\textsuperscript{3,4}},
%  \textbf{Nineteenth N. Author\textsuperscript{2,5}},
%  \textbf{Twentieth Author\textsuperscript{1}}
%\\
%\\
%  \textsuperscript{1}Affiliation 1,
%  \textsuperscript{2}Affiliation 2,
%  \textsuperscript{3}Affiliation 3,
%  \textsuperscript{4}Affiliation 4,
%  \textsuperscript{5}Affiliation 5
%\\
%  \small{
%    \textbf{Correspondence:} \href{mailto:email@domain}{email@domain}
%  }
%}

\begin{document}
\maketitle
\begingroup\def\thefootnote{$\dagger$}\footnotetext{Corresponding author.}\endgroup

\begin{abstract}
Tool learning has emerged as a promising direction by extending Large Language Models' (LLMs) capabilities with external tools. Existing tool learning studies primarily focus on the general-purpose tool-use capability, which addresses explicit user requirements in instructions. 
However, they overlook the importance of personalized tool-use capability, leading to an inability to handle implicit user preferences.
% meet personalized user needs.
To address the limitation, we first formulate the task of personalized tool learning, which integrates user's interaction history towards personalized tool usage. 
To fill the gap of missing benchmarks, we construct \benchmark, featuring diverse user preferences reflected in interaction history under three distinct personalized settings, and encompassing a wide range of tool-use scenarios.
Moreover, we propose a framework \framework to adapt LLMs to the personalized tool learning task, which is trained through supervised fine-tuning and direct preference optimization.
Extensive experiments on \benchmark demonstrate the superiority of \framework over existing LLMs. 
We release our code and data at
% for review at \href{https://anonymous.4open.science/r/PEToolBench-952F/}{https://anonymous.4open.science/
% r/PEToolBench-952F/}.
\href{https://github.com/travis-xu/PEToolBench}{https://github.com/travis-xu/PEToolBench}.

\end{abstract}

\section{Introduction}
% Large Language Models (LLMs) have increasingly been regarded as a potential pathway toward developing general AI assistants for human users. 
Large Language Models (LLMs) possess extensive knowledge and have powerful instruction-following abilities, making them effective AI assistants for tasks such as text rewriting, question answering, and code writing~\cite{zhao2023survey}.
However, they often struggle in addressing user needs in scenarios such as checking weather and booking flights. 
To address this, tool learning~\cite{10.1145/3704435,qu2024tool} has emerged as a promising solution by enabling LLMs to utilize external tools, such as real-time weather APIs and booking systems.
% To bridge these gaps, tool learning has emerged as a promising approach, equipping LLMs with the ability to leverage external tools like real-time weather APIs and booking systems. 
% Tool learning not only overcomes current limitations of LLMs, but also extends their 
In this way, tool learning has extended LLMs' capabilities to tackle more complex tasks, enabling them to fulfill a wide range of user needs.

Current tool learning procedure typically begins with a user instruction, and then LLMs are required to use tools with appropriate functionalities for satisfying users' needs.
% use tools with appropriate functionalities to fulfill user’s requirements in the instruction. 
Existing tool learning methods can be categorized into in-context learning~\cite{wu-etal-2024-toolplanner,liu2025toolplanner} and fine-tuning approaches~\cite{NEURIPS2023_d842425e,wang2025toolgen}. The former approach allows LLMs to use tools by directly providing tool documentation in input but the performance is constrained by the input length.
% or demonstrations  % the limited input length 
The latter approach trains LLMs to internalize tool knowledge but struggles with tool generalization.
% specially
% Existing tool learning studies primarily focus on the \textit{general-purpose} tool-use capability, where
% % , i.e., utilizing various tools to address a wide range of user needs.
% % Specifically, given user instructions and tool documentation as inputs, 
% LLMs are required to understand two main aspects: 1) \textbf{explicit user requirements}, expressed in user instructions, and 2) \textbf{functionalities of tools}, derived from tool documentation. By understanding both aspects, LLMs can accurately utilize tools with appropriate functionalities to address user needs.


\begin{figure*}[!t]
    \centering
    % \centerline{\includegraphics[width=2.0\columnwidth]{MainFigure.pdf}}
    \includegraphics[width=1.0\textwidth]{fig_intro.pdf}
    \caption{Comparison between (a) tool learning and (b) personalized tool learning. Personalized tool learning facilitates implicit preference comprehension and customized tool usage for individual users.}
    \label{fig:intro}
\vspace{-1em}
\end{figure*}


Despite the advancement, existing tool learning methods primarily focus on the general-purpose tool-use capability but overlook the critical role of personalization. 
In tool learning, more personalized user needs are expected to be derived from the user's previous 
% interaction 
tool usage 
history as a supplement to user instructions, which can help LLMs provide more customized tool-usage assistance to enhance the user experience.
% This helps the LLM provide more customized tool assistance, thereby improving the user experience.
% requires LLMs to offer more customized tool-usage assistance for specific users. 
% uncover
As illustrated in Figure~\ref{fig:intro}, personalized tool learning is non-trivial due to the following aspects. 
% As tool-use LLMs are designed to serve human users, and considering that each user has unique needs, LLMs must use specific tools to meet user-specific needs, referred to as the \textit{personalized} tool-use capability.
% As user serving AI assistants, considering that each user has unique needs, LLMs must be capable of meeting these distinct requirements.
% Since tool learning aims at enabling LLMs to better serve human users, each of whom has unique needs.
% Unfortunately, they exhibit significant limitations due to the neglect of two critical aspects:
% catering for unique user needs,
% role of \textit{personalized} tool-use capabilities in LLMs, including two additional aspects
% 1) \textbf{implicit user preferences}, which are not explicitly stated in user instructions but can be inferred from personal user data (e.g., interaction history). 
1) \textbf{Implicit user preferences}. 
User preferences for tool usage are often implicitly conveyed through the user's history rather than explicitly stated in user instructions, making them difficult to understand.
% as they are implicitly concealed within the user's history instead of explicitly stated in user instructions.
For instance, when a user requests a search for articles, their preference for academic-related content needs to be inferred from previous interactions with academic tools like Google Scholar.
% For instance, when requesting a search for articles, a user may prefer academic-related results if their interaction history includes academic tools like Google Scholar. 
% may indicate a preference for academic-related search results.
% and are revealed in the user's interaction history.
% comprehension, which are not explicitly stated in user instructions but can be inferred from user's interaction history.
% For instance, when requesting a search for articles, a user may prefer academic-related results if their interaction history includes academic tools like Google Scholar. 
% For example, when requesting a search, a user's interaction history involving academic tools (e.g., Google Scholar) is likely to prefer academic-related search results;
% academically supportive
2) \textbf{Non-functional tool attributes}.
% In real-world scenarios, many tools have the same functionalities, 
% reflect the differences in 
Since many tools have the same functionalities, user preferences cannot be effectively distinguished based solely on tool functionalities. 
% which is challenging for LLMs to differentiate and align with different user preferences.
% To address this, it is crucial 
This underscores the need to consider non-functional tool attributes, such as usability, integrability, and accessibility, which can better reflect user preferences. 
% Due to the existence of multiple same-functionality tools, LLMs cannot determine the user's preferred tools based solely on their functionality. 
% In this paper, we refer to these as the \textit{non-functional attributes} of tools.
% to address specific personalized needs.
% discern which tool a user prefers based solely on its functionality.
% allowing LLMs to use the user's preferred tool from multiple tools with the same functionality.
% customization most
% This is possible because tools often have unique \textit{non-functional attributes} (e.g., usability, integrability, accessibility, etc.) that are preferred by different users.
% distinguish between same-functionality tools and use the most suitable one for individual user.
% beyond functionalities, which can significantly impact user preferences.
% \textbf{non-functional attributes of tools}, referring to attributes beyond functionalities (e.g., usability, integrability, etc.) that significantly influence user preferences.
% can differentiate same-functionality tools to align with different user preferences. 
As shown in Figure~\ref{fig:intro}, Google Search can be distinguished from other search tools by its integration into Google’s ecosystem with Google Scholar, making it more suitable for users with academic needs.
% These non-functional attributes are crucial for LLMs to call the most preferred tool for a specific user, especially when there exist multiple same-functionality tools. 
% Therefore, these aspects are essential for LLMs to form a holistic understanding of tools and user needs, facilitating customized tool-usage assistance.
% form a more comprehensive and deeper understanding of tools and user needs, advancing them to be a more personalized and user-centric tool-use assistant.

% To this end
To address the above issues, we formulate the task of personalized tool learning in LLMs, aiming at personalized tool usage for individual users.
Formally, given user instructions along with user's 
% user data (e.g. interaction history)
interaction history, LLMs are required to answer user instructions with tools by considering both explicit user requirements in instructions and implicit user preferences behind interaction history.
% must satisfy user needs by using tools which not only have appropriate functionalities to meet explicit user requirements, 
% % as well as with 
% but also possess suitable non-functional attributes to align with implicit user preferences.
% understand both the user's explicit requirements and implicit preferences, as well as functionalities and non-functional attributes of tools, then select and call the appropriate tool with corresponding parameters.
% , then select and call the appropriate tool with corresponding parameters to meet the user's needs.

% which takes personal user data and tool characteristics into consideration
% aiming for personalized user needs comprehension and tool utilization.
% , aiming to align with explicit user instructions and implicit user preferences derived from personalized data.
% by taking personalized user data and tool characteristics into consideration, aiming to facilitate more tailored tool usage that satisfies both the users' functional requirements and their individual preferences. 


Since there is no benchmark for this task currently, we fill this gap by introducing the first personalized tool learning benchmark (\benchmark). 
Specifically, the benchmark is created through three following steps.
1) Tool Preparation.
We collect a bunch of high-quality tools from RapidAPI and then
% as our seed tools. 
% And then we 
leverage LLM to understand the functionality and non-functional attributes of each tool. 
2) Preference Construction. 
% We leverage the obtained tool attributes to construct the user preferences on tools.
% We classify tools with the same functionalities into groups. 
Among same-functionality tools, we construct the user's tool preferences by assigning tools with distinct non-functional attributes to different users.
% construct tool preferences for users by assigning the user's preferred tool and non-preferred tools based on non-functional attributes.
% we identify the preferred and non-preferred tools of a specific user within the functionality group.
% by randomly assign a tool with distinct non-functional attributes as the user's preferred tool and other tools as non-preferred tools.
% for a specific user. 
% To enrich the user preference, we use the preferred tool to retrieve more tools with similar non-functional attributes.
% The preferred tool will also be used to retrieve more tools with similar non-functional attributes to be the user's preferred tool .
% In this way, we can obtain a bunch of preferred and non-preferred tools representing different user preferences.
% After multiple functionality groups, we can obtain a number of preferred and non-preferred tools representing .
% all serving as preferred tools for the specific user.  
3) Data Creation. 
Based on tool preference, we synthesize the user's interaction history into a sequence of tool-use interactions, each consisting of a user instruction and an LLM's tool call.
% The user's interaction history is generated into a sequence of tool-use interactions, each containing a user instruction and an LLM's tool call. 
% both preferred and non-preferred tools with binary ratings reflecting the user's preference; 
% Finally, we select tools 
We design three personalized tool-usage settings by generating the interaction history in three types, i.e., preferred-only, rating-integrated, and chronological. 
% representing different forms of user preferences.
% in  ways, we
And then we use tools not included in the interaction history to synthesize user instructions.
% as ground truth tools to generate user instructions for each data instance. 
% Each instruction is combined with the interaction history into a data instance. construct data instance 
% We generate interaction history into three types, i.e., preferred-only, rating-integrated, chronological, to represent different forms of user preferences.
% 1) \textit{preferred-only} history, which only includes the user's preferred tools ; 2) \textit{rating-integrated} history, integrating user's binary ratings on each tool call according to user preferences; and 3) \textit{chronological} history, where interactions are arranged in time order to capture changes in user preferences over time. 
After rigorous filtering, we obtain 12,000 user instructions with interaction histories reflecting diverse user preferences and cover a wide range of tool-use scenarios by encompassing 7454 tools across 46 categories.
 % of 1,703 users 

% .Overall, our benchmark sets itself apart through several distinctive features:
% structured through three primary phases:
% \begin{enumerate}
%     \item User's Preferred Tools Selection: 
%     We adopt real-world tools from ToolBench dataset, and identify their user-centric attributes beyond functionalities. Then we classify and select tools with similar user-centric attributes as the user's preferred tools. 
%     \item Interaction History Construction: 
%     We sample a bunch of user's preferred tools to synthesize a sequence of tool-use cases as the user's interaction history, constructed in three forms (i.e., preferred-only, rating-integrated and chronological), representing different manifestations of user preferences.
%     \item User Instructions Creation:
%     We utilize users' preferred tools not included in the interaction history as ground truth tools. Then we use them to synthesize user instructions and corresponding tool calling parameters.
% \end{enumerate}

% To equip LLMs with personalized tool-use capability, we propose \framework. Specifically, we train LLM on \benchmark through two stages: 
Based on the \benchmark dataset, we propose the personalized tool learning framework (\framework) to equip LLMs with personalized tool-use capability. The training process consists of two stages:
1) the supervised fine-tuning (SFT) stage, which equips LLM with foundational tool-use capability to address user needs; 2) the direct preference optimization (DPO) stage, which samples the user's preferred and non-preferred tool calls for pair-wise optimization to better align with user preferences.
 % (Dubey et al., 2024)
We evaluate 6 distinct open-source and closed-source LLMs including the latest GPT-4o on \benchmark. 
Experimental results demonstrate that our \framework significantly outperforms the best-performing LLM across all settings with improvements even more than 50\%, showcasing its superior personalized tool-use capabilities.
% to provide user-centric and customized assistance.
% Experimental results demonstrate that \framework consistently and significantly outperforms existing baselines, showcasing the potential of personalized tool-use LLMs to provide more user-centric and customized tool-usage assistance.


In summary, our contributions are as follows.
\begin{itemize}
\item 
We are the first to formulate the task of personalized tool learning in LLMs, 
which incorporates user's interaction history to achieve personalized tool-usage assistance.
% for individual users. 
% tool selection and calling, bridging users with customized tool-usage assistance.

\item 
We construct the first benchmark for personalized tool learning in LLMs, \benchmark,
featuring user instructions integrated with interaction history reflecting diverse user preferences and encompassing various tools.
% featuring diverse user instructions with interaction history across three types.
\item 
% We propose the first personalized tool-use LLM, \framework.
We propose a novel personalized tool learning framework \framework. 
% Extensive experiments show that \framework consistently exceeds existing baselines, effectively fulfilling both user requirements and preferences.
Extensive experiments demonstrate that \framework significantly surpass the best-performing LLM by more than 50\%, exibiting exceptional personalized tool-use capabilities.
\end{itemize}

\section{Related Work}
\subsection{Tool Learning in LLMs}
Tool learning aims at extending the capabilities of LLMs by equipping them with external tools to solve tasks like weather inquiry, car navigation, and restaurant reservation. Existing benchmarks primarily focus on evaluating the tool learning proficiency of LLMs in addressing user instructions, from aspects such as 
% tool usage awareness~\cite{huang2024metatool}, 
tool selection and calling accuracy~\cite{xu-etal-2024-enhancing-tool,NEURIPS2024_8a75ee6d,ye-etal-2024-rotbench,wang2025mtubench}, tool planning ability~\cite{basu-etal-2024-api,wang-etal-2024-appbench,NEURIPS2024_085185ea,liu2025toolace}, and complex workflow creation~\cite{shen2025shortcutsbench,qiao2025benchmarking,fan2025workflowllm}. To improve tool-use capabilities, various strategies have been introduced, including in-context learning which enables LLMs to use tools via documentation~\cite{yuan2024easytool,shi-etal-2024-learning,qu2025from},
% (Hsieh et al., 2023), 
and fine-tuning which trains LLMs on specialized tool-use datasets~\cite{zhuang2024toolchain,chen2024advancing,chen2025learning}. However, prior studies neglect the crucial role of personalized tool usage in LLMs. This paper addresses this gap by introducing personalized tool learning, developing a comprehensive benchmark for evaluation, and proposing an optimization strategy to enhance personalized tool-use capabilities in LLMs.

\subsection{Personalization in LLMs}
The goal of personalization in LLMs is to leverage personal user data, such as historical behaviors and background information, to generate outputs that better align with the user preferences~\cite{tseng-etal-2024-two}.
% (Chen et al., 2023e; Deshpande et al., 2024). 
Approaches such as fine-tuning~\cite{cai2025large} and prompt engineering~\cite{yuan-etal-2025-personalized}
% , and user-specific embeddings 
have been explored to adapt LLMs to individual or domain-specific tasks. These approaches have been applied across various fields, including recommendation systems~\cite{lyu-etal-2024-llm}, search engines~\cite{10.1145/3589334.3645482}, education~\cite{liu2024socraticlm}, 
% healthcare, 
and dialogue generation~\cite{wang-etal-2023-target}. However, previous research has not investigated LLMs' personalization in the area of tool learning. In this work, we bridge this gap by incorporating user's interaction history to assess and enhance the LLMs' capability in providing personalized tool-usage assistance for specific users.
% generating tool calls tailored to user-specific needs.



\begin{figure*}[!t]
    \centering
    \includegraphics[width=1.0\textwidth]{fig_benchmark.pdf}
    % \caption{Illustration of our \benchmark.}
    \caption{Illustration of the process for constructing our \benchmark.}
    \label{fig:benchmark}
\vspace{-1em}
\end{figure*}

% Due to the lack of real personal data on tool-usage, we adopt a tool-driven approach to simulate user data by leveraging tool attributes to construct user preferences and interaction history. 
% real-world tools to simulate user data, distinguishing user preferences by tool attributes. 
% (1) Tool Preparation.
% We collect a bunch of high-quality tools from RapidAPI.
% % as our seed tools. 
% Then we leverage LLM to understand the functionality and non-functional attributes of each tool. 
% (2) Preference Construction. 
% % We leverage the obtained tool attributes to construct the user preferences on tools.
% We classify tools with the same functionalities into groups. 
% Within each functionality group, we construct the user preferences by
% % we identify the preferred and non-preferred tools of a specific user within the functionality group.
% randomly assign a tool with distinct non-functional attributes as the user's preferred tool and other tools as non-preferred tools.
% % for a specific user. 
% To enrich the user preference, we use the preferred tool to retrieve more tools with similar non-functional attributes.
% % The preferred tool will also be used to retrieve more tools with similar non-functional attributes to be the user's preferred tool .
% In this way, we can obtain a bunch of preferred and non-preferred tools representing different user preferences.
% % After multiple functionality groups, we can obtain a number of preferred and non-preferred tools representing .
% % all serving as preferred tools for the specific user.  
% (3) Data Creation. 
% Based on the user preference, we generate the user's interaction history as a sequence of tool-use interactions, each containing a user instruction and a corresponding tool call by LLM. The interaction history is constructed in three types: 1) \textit{preferred-only} history, which only includes the user's preferred tools ; 2) \textit{rating-integrated} history, integrating user's binary ratings on each tool call according to user preferences; and 3) \textit{chronological} history, where interactions are arranged in time order to capture changes in user preferences over time. 
% % both preferred and non-preferred tools with binary ratings reflecting the user's preference; 
% % Finally, we select tools 
% Finally, we generate user instructions based on the user's preferred tools not included in the interaction history.
% % as ground truth tools to generate user instructions for each data instance. 
% % Each instruction is combined with the interaction history into a data instance. construct data instance 


\section{Task and Benchmark}
\subsection{Task Formulation}
\paragraph{Tool Learning}
Given an instruction $q_u$ of the user $u$, tool learning aims to generate an appropriate tool call, including the selected tool and its corresponding parameters, from a set of candidate tools. Formally, let the candidate tool set be $\mathcal{T}=\{d(t_1), d(t_2), ..., d(t_N)\}$, where $d(t_i)$ represents the documentation of tool $t_i$ and $N$ is the total number of candidate tools. The LLM is then tasked with generating a tool call $c = (t, p)$, where \(t \in \mathcal{T}\) and \(p\) denotes its parameters: 
%The LLM $\rho_\theta$ needs to generate a tool call $c$ which consists of an appropriate tool \(t\) from the candidate tool set $\mathcal{D}$ and corresponding parameters \(p\), denoted as: 
\begin{equation}
% c = \rho_\theta(t,p|q_u,\mathcal{T})
(t,p) = LLM(q_u,\mathcal{T}).
% \leftarrow LLM()
\end{equation}

\paragraph{Personalized Tool Learning}
In personalized tool learning, we incorporate the users' interaction history alongside their instructions, enabling the LLM to generate tool calls that satisfy both the users' explicit requirements and implicit preferences.
For a user u, we define the interaction history as $\mathcal{H}_u = \{h_u^1, h_u^2, ..., h_u^M\}$, where each $h_u^i$ consists of a past user instruction $q_{u}^i$ and the corresponding tool call $c_u^i=(t_u^i,p_u^i)$, with $t_u^i$ representing the selected tool and $p_u^i$ denoting its associated parameters. Let $c_u = (t_u,p_u)$ represent the personalized tool call for user u, the personalized tool learning task can then be formulated as:
% The LLM $\rho_\theta$ needs to generate a tool call $c$ 
\begin{equation}
(t_u,p_u) = LLM(q_u,\mathcal{T},\mathcal{H}_u),
% c = \rho_\theta(t,p|q_u,\mathcal{T},\mathcal{H}_u)
\end{equation}

% requirement and preference. 
% Formally, for a user $u$, we define the user's instruction as $q_u$ and interaction history as $\mathcal{H_u}$ represented by a sequence $\{h_u^1, h_u^2, ..., h_u^M\}$. Each $h_u^i$ denotes an interaction, i.e., the past user instruction $q_{u}^i$ and the LLM's tool call $c^i=(t_i,p_i)$, containing the selected tool $t_i$ and corresponding parameters $p_i$.
% The candidate tool set is denoted by $\mathcal{T}=\{d(t_1), d(t_2), ..., d(t_N)\}$, where $d(t_i)$ represents the documentation of each candidate tool $t_i$ and $N$ is the total number of candidate tools.
% Based on the user's requirement from instruction $q_u$ and the user's preference from interaction history $\mathcal{H_u}$, the personalized tool-use LLM needs to generate a tool call $c$ which consists of an appropriate tool \(t\) from the candidate tool set $\mathcal{T}$ and corresponding parameters \(p\). 
% select a tool $t$ from the candidate tool set $D$ and call it with appropriate parameters $p$ to solve the task.
% and the toolset as $T=\{T_1, T_2, ..., T_n\}$,
% Based on $P_u$ and $q$, the model is expected to select a user-specific tool set $\hat{D}_u=\{d_i^u\}_{i=1}^k$ from $D$, and then output a user-specific response $r^u$ based on a tool-use trajectory
% $T=[t_j(d^u_i,c^u_i)|d^u_i\in \hat{D}_u]_{j=1}^S$, containing $S$ tool-use steps which sequentially uses a selected user-specific tool $d^u_i$ with corresponding user-specific tool calling parameters $c^u_i$.

\subsection{Benchmark Construction}
Due to the lack of real user interaction histories on tool-usage, we adopt a tool-driven approach to simulate interactions based on pre-constructed user's tool preferences. The whole process for constructing \benchmark, illustrated in Figure~\ref{fig:benchmark}, consists of three steps: tool preparation, preference construction, and data creation.
\subsubsection{Tool Preparation}

\paragraph{Tool Collection}
% We curate xx high-quality tools selected from real-world APIs in ToolBench dataset.
Following ToolBench~\cite{qin2024toolllm}, we adopt the tools from RapidAPI for our benchmark, since it offers a large-scale and diverse collection of real-world tools that can potentially address a wide range of user needs.
% Despite the numerous tools collected in ToolBench, the quality of tools is not guaranteed.
To ensure the quality of the collected tools, we perform strict filtering by removing: 1) outdated tools, which are marked as deprecated in RapidAPI; 2) tools with insufficient information, such as inadequate or missing tool documentation; and 3) duplicate tools, which have repeated tool names, descriptions, or category names.
% After filtering, we obtain xx high-quality tools.
% redundancies
% Complete Documentation. Despite the numerous tools collected in ToolBench (Qin et al., 2023b) from RapidAPI, the documentation quality is not guaranteed. To reduce the failure of tool calling cases caused by inadequate tool descriptions, which focus the evaluation attention on pure LLM abilities, we manually generate high-quality and detailed tool documentation for each tool.

\paragraph{Tool Understanding}
Since tool documentation often contains redundant and irrelevant information, directly extracting tool attributes from it can be challenging.
To address this, we 
% design a tool understanding process that leverages LLMs to comprehend both the functionality and non-functional attributes of each tool. Specifically, we 
first provide the documentation of each tool to LLM and prompt it to generate a tool-use example, including a simulated user instruction and parameters for calling the tool. 
Next, based on the tool documentation and tool-use example, the LLM is instructed to generate descriptions of the tool's functionality and non-functional attributes separately.
Besides, we include demonstrations in the prompt to help the LLM distinguish between these two attribute types.
By leveraging specific tool-use examples and demonstrations, the LLM can develop a more comprehensive understanding of each tool’s functionality and non-functional characteristics.
% plausible scenario for using the tool, relevant to the API
% We also add demonstrations into the instruction to enhance the instruction-following of LLMs in parsing tool documentation.
% tool documentation usually includes plenty of irrelevant information that makes it difficult to understand practical usage for LLMs
% To ensure a holistic understanding of tool attributes to align with diverse users' requirements and preferences, we leverage LLM to comprehend both the functionalities and non-functional attributes of each tool. Specifically, 
% we provide the tool documentation of each tool, and prompt LLM to understand 

\subsubsection{Preference Construction}

\paragraph{Tool Classification}
% that meet three main principles:
% tools with the same functionalities 
To identify potential tool-usage scenarios for users, we classify tools with the same functionalities into groups.
Specifically, we first employ the Ada Embedding model~\footnote{\url{https://platform.openai.com/docs/guides/embeddings/embedding-models}.} to compute embeddings for the functionality descriptions of all tools.
Then, we apply the DBSCAN algorithm~\cite{schubert2017dbscan} to cluster these tools into multiple groups based on the similarity of their embeddings.
Within each group, the tools share the same functionality and can be applied to a specific tool-usage scenario.
% Consequently, all tools within a group share the same functionality and can be applied to a specific tool-usage scenario.
To further ensure that tools within each group exhibit uniform functionality, we conduct rigorous filtering and only retain groups where tools 1) have the same input-output formats (i.e., required/optional parameters and response schema) and 2) belong to the same category (e.g., sports, music, finance).
% perform rigorous filtering 

% embeddings based on the semantic similarities between task descriptions and retain one instance for each cluster
% text embeddings for these explanations
% Due to the diversity in both the intent detection and slot-filling strategies, as well as the creation of specific tools based on LLM, the synthesized tool set can be highly redundant. 
% For example, tools like “search_movie” and “find_movie” may have different names but essentially perform the same function. 
% we cluster them based on the semantic similarities between task descriptions

\paragraph{Tool Preference Construction}
We leverage non-functional tool attributes to construct the user's tool preference. 
First, we randomly sample a functionality group for a user, representing a potential tool-usage scenario for interaction. Within this group, we choose a tool with specific non-functional attributes as the user's preferred tool, while the others are considered non-preferred. Using the preferred tool as a reference, we retrieve the top-$5$ tools with the most similar non-functional attributes. Similarity is computed based on the embeddings of the tools' non-functional descriptions, which are generated in the Tool Understanding phase. Through multiple iterations of sampling and retrieving, we obtain a diverse set of preferred and non-preferred tools that represent user preferences. After each iteration, we check for functionality overlap between newly retrieved tools and previously selected ones. If an overlap is detected, the tools are discarded, and the sampling process is restarted. This ensures that each tool-usage scenario is associated with only one preferred tool per user.
By following this approach, we construct diverse tool sets that align with different user preferences.
% Through this process, 
% Then, we utilize the Ada Embedding model to compute xx embeddings of all tools, based on descriptions of their non-functional attributes generated in the Tool Understanding phase.
% Then we use the chosen tool to retrieve top-$K$ tools based on the similarity scores between their embeddings.
% This ensures that each user has a unique preference for a single tool in each tool-usage scenario.
% from the whole tool collection 

\subsubsection{Data Creation}

\paragraph{Interaction History Generation}
Based on tool preference, we leverage the LLM to construct the user's interaction history.
Specifically, for each user, we provide LLM with the user's preferred and non-preferred tools, including tool attributes and tool-use examples generated in the Tool Understanding phase.
The LLM will generate a sequence of simulated user-LLM interactions, each consisting of a user instruction and an LLM's tool call, as the user's interaction history.
% , each consisting of a user instruction and an LLM's tool call.
% LLM is then prompted to leverage these tools to 
% between the user and the tool-usage LLM, as the user's interaction history.
% realistic and coherent user's interaction history, containing a sequence of interactions between the user and the tool-usage LLM. 
% Each interaction consists of a user instruction and a corresponding tool call, which can be reused from the provided tool-use examples or newly generated based on the history context.
% Next, we use the preferred tools to synthesize a sequence of interactions as the user's interaction history, each containing a user instruction and a corresponding tool call. 


We design three personalized tool-use settings by generating the interaction history in three types (illustrated in Figure~\ref{fig:history}):
1) \textit{preferred-only} history, where the tools involved in the interactions are all preferred by the user; 
2) \textit{rating-integrated} history, including both the user's preferred and non-preferred tools, with a user's binary rating for each tool-usage interaction representing the user preference, i.e., ``liked'' if the tool aligns with the user preferences, and ``disliked'' otherwise.
% i.e., the rating is set to 1 if the tool is preferred by the user, and 0 otherwise; 
3) \textit{chronological} history, which organizes interactions in time order to reflect changes in user preferences over time, i.e., the more recent tool-usage interactions are more preferred by the user, while earlier interactions are less preferred.
In this way, we can present different forms of user preferences.
% maintain the context length

\paragraph{Instruction Generation}
Next, we use LLM to generate user instructions based on the user's preferred tools that are not included in the user's interaction history. 
% do not appear
% We adopt tools that share the same functionality with other tools as ground truth tools for instruction generation.
We instruct the LLM to avoid directly generating the name of the tool in the instruction, ensuring that the user preference for the tool can only be inferred from the user's interaction history.
% to focus on the tool functionality and do not directly including the tool name in the instruction.
% identifying the preferred tool among xx based on user preferences.
% When generating the user instruction, we ask LLM to focus on the tool functionality and avoid including the tool name in the user instruction directly, in order to xx.
Each user instruction is combined with the user’s interaction history into a data instance. 
% where 

Finally, we obtain 12,000 data instances
% , simulating interaction history and 
encompassing 7,454 tools across 46 categories. We split all data into two parts: a training set comprising 9,000 instances for three personalized settings and a test set containing the rest instances.

\subsection{Benchmark Analysis}
% \paragraph{Statistic Analysis}
% We present the statistical information of our \benchmark in Figure \ref{statistics_length}, Table~\ref{dataset_statistics}. 
We present the statistical information of our \benchmark in Table~\ref{fig:statistics_instances_length}, including the statistics of data instances in three settings and under varying interaction history lengths.
We also present the distribution of tool categories in Figure~\ref{fig:statistics_category}. 
Statistical information demonstrates the diversity and complexity of our dataset.

% % \begin{figure*}[htbp]
%     % 左侧图片
%     \begin{minipage}{0.77\linewidth}  % 调整宽度
%         \centering
%         \includegraphics[width=\linewidth]{images/benchmark_construction.pdf}
%     \end{minipage}%
%     % 间隔
%     \hfill
%     % 右侧表格
%     \begin{minipage}{0.23\linewidth}  % 调整宽度
%         \centering
%         \resizebox{\linewidth}{!}{  % 调整表格至合适的宽度
%             \begin{tabular}{lcc}
%                 \toprule
%                 \textbf{Statistic} & \textbf{Number} \\
%                 \midrule
%                 \rowcolor[HTML]{F2F2F2} 
%                 \textit{Domain Count} &  \\
%                 \midrule
%                 Domain & 103 \\
%                 Requirement & 8 \\
%                 \midrule
%                 \rowcolor[HTML]{F2F2F2} 
%                 \textit{Token Count} &  \\
%                 \midrule
%                 Description & 851.6 $\pm$ 515.2 \\
%                 - Min/Max & [159, 2814] \\
%                 Domain & 1187.2 $\pm$ 1212.1 \\
%                 - Min/Max & [85, 7514] \\
%                 \midrule
%                 \rowcolor[HTML]{F2F2F2} 
%                 \textit{Line Count} &  \\
%                 \midrule
%                 Domain & 75.4 $\pm$ 62.9 \\
%                 - Min/Max & [9, 394] \\
%                 \midrule
%                 \rowcolor[HTML]{F2F2F2} 
%                 \textit{Component Count} &  \\
%                 \midrule
%                 Actions & 4.5 $\pm$ 2.8 \\
%                 - Min/Max & [1, 16] \\
%                 Predicates & 8.1 $\pm$ 4.8 \\
%                 - Min/Max & [1, 25] \\
%                 Types & 1.1 $\pm$ 1.3 \\
%                 - Min/Max & [1, 8] \\
%                 \bottomrule
%             \end{tabular}
%         }
%     \end{minipage}
%     % 公共标题
%     \caption{Dataset construction process (left) and key statistics (right) of the \texttt{\benchmark} dataset.     Dataset construction process including: (a) \textit{Data Acquisition} (\S\ref{sec:data_acquisition}); (b) \textit{Data Filtering and Manual Selection} (\S\ref{sec:data_filtering}); (c) \textit{Data Annotation and Quality Assurance}(\S\ref{sec:data_annotation} and \S\ref{sec:quality_assurance}). Tokens are counted by GPT-2~\cite{openai2019gpt2} tokenizer.}
%     \label{fig:combined}
% \end{figure*}


\begin{figure}[!t]
    \centering
    \includegraphics[width=1.0\linewidth]{statistics_instances_length.pdf}
    \caption{Statistics of data instances in three personalized settings (in the left figure) and distributions of interaction history length (in the right figure).}
    \label{fig:statistics_instances_length}%文中引用该图片代号
% \vspace{-1em}
\end{figure}

\begin{figure}[tbp]
    \centering
    \includegraphics[width=1.0\linewidth]{statistics_category.pdf}
    \caption{Distributions of tool categories.}
    \label{fig:statistics_category}%文中引用该图片代号
\vspace{-1em}
\end{figure}



% \begin{figure}[htbp]
% \centering
% \begin{subfigure}{0.45\linewidth}
%     \centering
%     \includegraphics[width=0.9\linewidth]{statistics_category.pdf}
%     \caption{(a) Distributions of tool categories}
%     \label{statistics_category}%文中引用该图片代号
% \end{subfigure}
% \centering
% \begin{subfigure}{0.45\linewidth}
%     \centering
%     \includegraphics[width=0.9\linewidth]{statistics_length.pdf}
%     \caption{(b) Distribution of History Length}
%     \label{statistics_length}%文中引用该图片代号
% \end{subfigure}
% \end{figure}


% \paragraph{Consistency Analysis}
% % Since the user's interaction history are generated in our benchmark, 
% To verify the reliability of PersonalToolBench, we analyze the consistency of all three types of user’s interaction history in the data instance.  
% % by evaluating how well the users’ interaction history align with the ground truth tool call . 
% Specifically, we randomly sample xx data instances from PersonalToolBench, including all three types of user’s interaction history, and then ask LLM/human annotators to evaluate how well the users’ interaction history align with the ground truth tool call.
% % invite
% For each sampled instance, they should answer two yes/no questions: 1) whether the interaction history reflects the user preference, and 2) whether the ground-truth tool call matches that user preference.
% The results in Figure xx show that our constructed exhibit a xx consistency with the labeled tool call and xx alignment with user preference.


\subsection{Evaluation Metrics}
Given the user's instruction and interaction history, LLM is expected to select the appropriate tool from a candidate tool set, and then call the selected tool with corresponding parameters. Therefore, we define two metrics as follows.
\begin{itemize}
    \item Tool Accuracy (Tool Acc): The metric assesses the ability of LLM to select the appropriate tool to call. If the tool is correctly selected, the score is 1; otherwise, the score is 0.
    \item Parameter Accuracy (Param Acc): The metric assesses the ability of LLM to generate correct parameters for the tool call. If the input parameters are correctly generated, the score is 1; otherwise, the score is 0.
\end{itemize}

\documentclass{MITstyle}

%\usepackage[table]{xcolor}
\usepackage{chngcntr}
\usepackage{hyperref}
\usepackage{microtype}

\title{A Lightweight and Extensible Cell Segmentation and Classification Model for Whole Slide Images}

\author{Nikita Shvetsov~$^{1, }$\footnote{Correspondence e-mail: nikita.shvetsov@uit.no}, Thomas K. Kilvaer~$^{2, 3}$, Masoud Tafavvoghi~$^{4}$, Anders Sildnes~$^{1}$, \\ Kajsa Møllersen~$^{4}$, Lill-Tove Rasmussen Busund~$^{5, 6}$, Lars Ailo Bongo~$^{1}$ \\
%
\vspace{1em} % Space between authors and afilliations
%
\normalfont{\small $^{1}$Department of Computer Science, UiT The Arctic University of Norway}\\
\normalfont{\small $^{2}$Department of Oncology, University Hospital of North Norway}\\
\normalfont{\small $^{3}$Department of Clinical Medicine, UiT The Arctic University of Norway}\\
\normalfont{\small $^{4}$Department of Community Medicine, UiT The Arctic University of Norway}\\
\normalfont{\small $^{5}$Department of Medical Biology, UiT The Arctic University of Norway} \\
\normalfont{\small $^{6}$Department of Clinical Pathology, University Hospital of North Norway} %\vspace{2em}
}

\begin{document}
\maketitle

\section*{Abstract}

% \begin{abstract}
% Developing clinically useful cell-level analysis tools in digital pathology remains challenging due to limitations in dataset granularity, inconsistent annotations, computational demands of advanced models, and difficulties in integrating new technologies into clinical workflows. To address these challenges, we propose a multi-faceted solution that enhances data quality, model performance, and usability to create a lightweight and extensible cell segmentation and classification model.

% First, we update data labels by employing a cross-relabeling process that refines the labels of two existing datasets, PanNuke and MoNuSAC, to create a new unified dataset with enhanced granularity, encompassing seven distinct cell types. Second, we leverage the H-Optimus foundation model as a fixed encoder to improve feature representation for simultaneous cell segmentation and classification tasks. Third, to address the computational demands of foundation models, we employ knowledge distillation to reduce model size and complexity while maintaining comparable performance. Finally, to facilitate integration into clinical workflows, we integrate the distilled model into the QuPath software, a widely used open-source platform in digital pathology.

% Our results demonstrate improvements in cell segmentation and classification performance using the H‑Optimus-based model compared to a CNN-based model. Specifically, the average $R^2$ improved from 0.575 to 0.871, and the average $PQ$ score improved from 0.450 to 0.492, indicating better alignment with actual cell counts and enhanced segmentation and classification quality. Furthermore, the distilled student model maintains performance comparable to the larger foundation model while reducing the parameter count by a factor of 48.
% Overall, by reducing computational complexity and integrating it into existing workflows, the proposed approach may significantly impact diagnostic processes, reduce the workload of pathologists, and contribute to improved patient outcomes. Though our approach shows potential enhancements in efficiency and usability of cell segmentation and classification models in digital pathology, extensive validation is needed to deploy these models in clinical practice.
% \end{abstract}

%%% shortened abstract
\begin{abstract}
Developing clinically useful cell-level analysis tools in digital pathology remains challenging due to limitations in dataset granularity, inconsistent annotations, high computational demands, and difficulties integrating new technologies into workflows. To address these issues, we propose a solution that enhances data quality, model performance, and usability by creating a lightweight, extensible cell segmentation and classification model. 

First, we update data labels through cross-relabeling to refine annotations of PanNuke and MoNuSAC, producing a unified dataset with seven distinct cell types. Second, we leverage the H-Optimus foundation model as a fixed encoder to improve feature representation for simultaneous segmentation and classification tasks. Third, to address foundation models' computational demands, we distill knowledge to reduce model size and complexity while maintaining comparable performance. Finally, we integrate the distilled model into QuPath, a widely used open-source digital pathology platform. 

Results demonstrate improved segmentation and classification performance using the H-Optimus-based model compared to a CNN-based model. Specifically, average $R^2$ improved from 0.575 to 0.871, and average $PQ$ score improved from 0.450 to 0.492, indicating better alignment with actual cell counts and enhanced segmentation quality. The distilled model maintains comparable performance while reducing parameter count by a factor of 48. By reducing computational complexity and integrating into workflows, this approach may significantly impact diagnostics, reduce pathologist workload, and improve outcomes. Although the method shows promise, extensive validation is necessary prior to clinical deployment.
\end{abstract}
\clearpage

\section{Introduction}
In digital pathology, accurate segmentation and classification of cells are crucial for many diagnostic, prognostic, and predictive analyses \cite{Jaber_Beziaeva_etal._2019,Lin_Pan_etal._2022,Park_Ock_etal._2022,Shen_Choi_etal._2024}. Nowadays, developments in computational pathology offer multiple solutions \cite{H._Qu_P._Wu_etal._2020,Javed_Mahmood_etal._2020} to utilize cell-level datasets to train machine learning models that solve these problems. The quality and specificity of training datasets are critical for robust and accurate models. Adhering to the principle of "garbage in, garbage out", it is essential to ensure that these datasets are extensively and accurately labeled with distinct classes that reflect the diverse biological characteristics of different cell types. Unfortunately, the number of open-source datasets comprising such high-quality annotations is limited. Existing cell segmentation datasets \cite{Gamper_Koohbanani_etal._2019,Graham_Vu_etal._2019,Verma_Kumar_etal._2021} may offer extensive annotations for certain cell types while providing more general labels for others. For example, in PanNuke, which is one of the largest open-source datasets comprising labeled cells, various types of morphologically and functionally different inflammatory cells like macrophages and lymphocytes are clustered in a broad "inflammatory" class. Consequently, these classes are frequently omitted from analyses or aggregated into broader meta-classes \cite{Gamper_Koohbanani_etal._2020} and likely interfere with other cell classes included in the dataset. This and similar inconsistencies in annotation granularity limit the ability of machine learning models to learn the comprehensive and nuanced features necessary for accurate cell segmentation and classification. To address these challenges, methods for refining and standardizing dataset annotations are essential to enhance the quality of training data.

A complementary approach to mitigate the absence of high-quality training data is the use of foundation models. Foundation models as encoders are defined as large-scale, versatile networks pre-trained on vast, diverse datasets using self-supervised learning, contrasting with convolutional neural network (CNN) pre-trained encoders that rely on supervised learning with labeled data. In practice, foundation models leverage enormous amounts of weakly or unlabeled data from millions of whole slide images (WSIs) and employ self-attention mechanisms to capture long-range dependencies and global context \cite{Chen_Ding_etal._2024,Saillard_Jenatton_etal._2024,Vorontsov_Bozkurt_etal._2024,Xu_Usuyama_etal._2024}. As a consequence, foundation models are able to produce transferable feature representations across different cell types and tissue environments. The feature representations can be leveraged by decoder networks to produce segmentation masks and pixel-level classifications. Because foundation models have comprehensive feature representations, they can be effectively fine-tuned using much smaller amounts of cell-level data compared to the large datasets needed to train models from scratch. Furthermore, foundation models incorporate adversarial training elements or contrastive learning \cite{Chen_Ding_etal._2024,Xu_Usuyama_etal._2024}, enhancing their resilience and adaptability by exposing them to challenging and varied scenarios during training. This may result in more generalizable models, often making them well-suited for diverse and complex tasks in digital pathology.

Despite the inherent advantages of foundation models, their deployment for practical use faces its own obstacles. In particular, they require substantial computational power, financial investments and rigorous testing to ensure reliability and efficacy for a given task \cite{Akkus_Dangott_etal._2022,Dragomir_Cocuz_etal._2022,Go_2022,Jafri_Farooqui_etal._2024}. Moreover, while foundation models enhance feature representation and performance, they depend on the quality of available annotations for decoder fine-tuning and, like any other model, cannot resolve existing inconsistencies or ambiguities in data labels. Therefore, there remains a critical need for solutions that address both data quality and practical deployment considerations.
Further, integrating new technologies into existing clinical workflows often encounters resistance, as it necessitates adjustments to established diagnostic processes. So, there is a need to develop solutions that could be integrated into current practices, minimizing the burden on medical professionals to adopt new tools \cite{King_Williams_etal._2023}.

Existing solutions \cite{Goldsborough_Philps_etal._2024,Hörst_Rempe_etal._2024}, while addressing some aspects of these challenges, fall short in providing a comprehensive approach. To address the data quality and clinical deployment issues, we propose a multi-faceted solution that encompasses data refinement, model optimization, and integration with existing pathology tools (\hyperref[fig:fig1]{Figure 1}). The outcome is a lightweight cell segmentation and classification model that can be integrated into digital pathology workflows for practical clinical use.

\begin{figure}[h!]
    \centering
    \includegraphics[width=\textwidth, height=0.82\textheight, keepaspectratio]{images/Figure_1.pdf}
    \caption{Overview of the proposed solution, including 1) Data refinement using cross-relabeling, 2) Teacher model development and fine tuning, 3) Student model optimization with knowledge distillation and 4) Student model and QuPath integration}
    \label{fig:fig1}
\end{figure}
\clearpage

Our approach begins with preparing the data for the fine-tuning and training of the machine learning models. We create a refined dataset, acquired via cross-relabeling two cell-level datasets, enhancing annotation specificity and consistency of the labeled data. Subsequently, we create a cell segmentation and classification model based on the foundation model. We leverage the foundation model as a fixed encoder and fine-tune a decoder using the refined dataset to improve generalization across diverse tissue- and cell types.
To ensure that the model remains lightweight and deployable in a possibly resource-constrained environment, we employ knowledge distillation to approximate the functionality of the foundation model. Finally, to facilitate the practical application of our model in digital pathology workflows, we integrate it with the QuPath \cite{Bankhead_Loughrey_etal._2017} application. Each methodological component contributes to the overarching goal of enhancing model performance, generalizability, and usability in clinical settings.

The primary contributions of this paper are:
\begin{enumerate}
    \item \textit{Data labels refinement through cross-relabeling:}
    
    We propose a new method for refining labels of cell-level datasets through cross-relabeling. This method employs classification models to re-label broad and ambiguous instances, resulting in a more diverse dataset. Our evaluation demonstrates that these classification models achieve high accuracy on test subsets, indicating the reliability of the method for label refinement.

    \item \textit{Enhanced model performance via foundation models:}
    
    We employ a foundation model as a feature extractor for the cell segmentation and classification task. In comparison with training a CNN model from scratch, the foundation model backbone only needs fine-tuning, which significantly reduces training time, computational resources and data requirements. We show that using a foundation model encoder leads to better performance in cell segmentation and classification networks than using a CNN-based encoder. This improvement may enable the model to generalize more effectively across various tissue types and imaging methods.
    
    \item \textit{Model optimization through knowledge distillation:}
    
    We show that a smaller student model trained using knowledge distillation on the refined dataset obtained via our cross-relabeling approach from a foundation model achieves comparable performance in cell segmentation and quantification tasks. As a result, this model is more suitable for deployment in environments without high-performance computing resources.
    
    \item \textit{Integration with QuPath:}
    
    We integrate the distilled cell segmentation and classification model into QuPath, a widely used open-source digital pathology platform, to accelerate clinical adaptation by enabling pathologists to more easily incorporate advanced computational tools into their existing workflows.
\end{enumerate}

Through these methodological steps, we aim to bridge the gap between advanced machine learning techniques and practical clinical applications, making accurate and efficient digital pathology accessible in a broader range of healthcare settings.

\section{Refining Existing Datasets Using Cross-Relabeling}
To address the limitations of sparse and ambiguous labeling of cell-level datasets, we propose a generalizable cross-relabeling strategy that can be applied to any dataset containing broadly categorized or imprecisely labeled cell types. This approach involves training and subsequently leveraging classification models to refine broad categories into more specific or biologically relevant classes.
When applied to cell-level data, the methodology includes extracting individual cell images from the dataset patches, preprocessing these images to standardize the size and accommodate partial cells, and then training deep learning classifiers capable of distinguishing between the finer cell subtypes within the coarser categories. 
To illustrate our approach, we focus on the PanNuke \cite{Gamper_Koohbanani_etal._2020, Gamper_Koohbanani_etal._2019} and MoNuSAC \cite{Verma_Kumar_etal._2021} datasets that we have used to train models for cell quantification in our previous works \cite{Shvetsov_Grønnesby_etal._2022,Shvetsov_Sildnes_etal._2024}. We find that for better cell differentiation we have to introduce more granular labels. PanNuke includes a broad classification of "inflammatory" cells, encompassing lymphocytes, macrophages, and neutrophils. Each cell type differs significantly in structure, function, and clinical relevance. Conversely, MoNuSAC uses the label "epithelial" for a class that comprises both benign epithelial cells and malignant neoplastic cells. This practice makes it challenging to differentiate between benign and malignant epithelial cells in the dataset, which is a critical distinction when identifying tumor areas within tissue samples. To address these issues, we implement a cross-relabeling strategy as shown in \hyperref[fig:fig2]{Figure 2}. The key components are two classification models: one is trained on singular cell images from PanNuke data to classify the epithelial meta-class into epithelial and neoplastic classes. The other is trained on MoNuSAC to refine the inflammatory class into lymphocytes, neutrophils, and macrophages.

\begin{figure}[h!]
    \centering
    \includegraphics[width=\textwidth]{images/Figure_2.pdf}
    \caption{Refined dataset generation via cross relabeling}
    \label{fig:fig2}
\end{figure}

The refining approach consists of three consecutive steps. The first is the preprocessing step, in which we extract individual cells from both datasets (\hyperref[fig:fig3]{Figure 3}). The specifics of PanNuke and MoNuSAC patch preparation before cell preprocessing are provided in \hyperref[chap:S1]{Appendix S1}.

\begin{figure}[h!]
    \centering
    \includegraphics[width=\textwidth]{images/Figure_3.pdf}
    \caption{Cell instances preprocessing including (1) cell map extraction, (2) bounding box delineation, (3) adjusting cell boxes and (4) cropping and resizing of cell images}
    \label{fig:fig3}
\end{figure}

During preprocessing, we extract cell type maps from the ground truth label mask and calculate bounding boxes around each cell instance. To accommodate partial cells at patch borders, a common issue in cropped patch images, we employ mirror padding and extend the field of view of the cell label by 15 pixels to capture adjacent cells. We then crop and resize the identified regions to $64 \times 64$ pixels using bicubic interpolation.

The preprocessed PanNuke dataset comprises 68,031 neoplastic and 23,207 epithelial cell images, while MoNuSAC comprises  33,104 lymphocytes, 1,252 neutrophils, and 1,695 macrophages, which we subsequently use in training cell classification models and classifying the cell image data \hyperref[fig:S2]{Appendix Figure S2 (1)}. 

The next step is to train two distinct ResNet50-based classifiers tailored to address the specific labeling challenges inherent in each dataset. We use ResNet50 for classification models due to its proven effectiveness for image classification tasks in histopathology \cite{pan2022reviewmachinelearningapproaches}, and its compatibility with small images. For the PanNuke dataset, we design the classifier, trained on MoNuSAC data, to disaggregate the heterogeneous "inflammatory" cell category into distinct subtypes: lymphocytes, macrophages, and neutrophils. Similarly, for the MoNuSAC dataset, the classifier is trained on PanNuke data and distinguishes between benign and malignant epithelial cells within the overarching "epithelial" label. By applying these targeted classifiers to their respective datasets, we assign more specific labels to individual cell instances, thus enabling us to create a unified dataset.
To ensure a balanced representation of classes, we train both models on datasets that had been equalized to match the size of the least represented class. Thus, we obtain datasets comprising 23,207 samples per class for PanNuke and 1,252 samples per class for MoNuSAC data. Next, we partition both of them into training (70\%), validation (20\%), and testing (10\%) subsets. To mitigate the risk of overfitting, we use a single dropout layer with a rate of p=0.5 in both models and data augmentation using randomized color perturbations, rotation, and horizontal and vertical flipping. We employ AdamW optimizer and the cross-entropy loss function for the training criterion.

To evaluate the two trained models, we measure the classification accuracy on the respective test subsets. The accuracies on the test subset for both classifiers are presented in \hyperref[tab:1]{Table 1}. The PanNuke model achieves an average accuracy of 93.57\%, with higher accuracy for neoplastic cells (96.06\%) compared to epithelial cells (86.26\%). The confusion matrix in Figure A3.1 shows that the model predominantly distinguishes accurately between epithelial and neoplastic tissues, with a substantial number of correct classifications and relatively few misclassifications. The MoNuSAC model demonstrates an average accuracy of 98.92\%, excelling in classifying lymphocytes (99.67\%) and macrophages (94.12\%), with lower performance for neutrophils (85.71\%). The confusion matrix in Figure A3.2 shows that the model identifies lymphocytes and performs reasonably well with macrophages and neutrophils.

\begin{table}[h!]
\renewcommand{\arraystretch}{1.5}
  \centering
  \caption{Cell classification results for PanNuke and MoNuSAC trained models (CI 95\%).}
  \label{tab:1}
  \begin{tabular}{|l|c|c|}
   \hline
   %\rowcolor{gray!30}
    Accuracy               & PanNuke model              & MoNuSAC model              \\
    \hline
    Average      & 0.936 (0.931--0.941)         & 0.989 (0.986--0.993)        \\
    \hline
    Neoplastic   & 0.961 (0.956--0.965)         & -                          \\
    \hline
    Epithelial   & 0.863 (0.849--0.877)         & -                          \\
    \hline
    Lymphocytes  & -                          & 0.997 (0.995--0.999)        \\
    \hline
    Neutrophils  & -                          & 0.857 (0.796--0.918)        \\
    \hline
    Macrophages  & -                          & 0.941 (0.906--0.976)        \\
    \hline
  \end{tabular}
\end{table}

Finally, during the last step, we use the model trained on PanNuke data for epithelial cells in MoNuSAC and the model trained on MoNuSAC for the inflammatory cells class in PanNuke. Specifically, we use classifier models to relabel epithelial cells in MoNuSAC and inflammatory cells in PanNuke data. Then we combine cells with refined labels and the rest of the cells in both datasets to create a refined dataset (\hyperref[fig:S2]{Appendix Figure S2 (2)}). The process of relabeling cells and visualizing them on a patch is shown in \hyperref[fig:fig4]{Figure 4}. The cell counts in the refined dataset are provided in \hyperref[tab:S4]{Appendix Table S4}.

\begin{figure}[h!]
    \centering
    \includegraphics[width=\textwidth, height=0.42\textheight, keepaspectratio]{images/Figure_4.pdf}
    \caption{Cell relabeling procedure for epithelial and inflammatory cell classes}
    \label{fig:fig4}
\end{figure}

%\hfill

Relabeling and combining datasets have been explored in a prior study \cite{Parulekar_Kanwat_etal._2023}, where consecutive fine-tuning on multiple datasets was employed to account for hierarchical class label structures. While the method presented in \cite{Parulekar_Kanwat_etal._2023} is intuitive, it often lacks consistency and requires multiple fine-tuning runs, which can be cumbersome and time-consuming. 
In contrast, cross-relabeling simplifies this process by using specialized classification models tailored to each dataset's specific labeling challenges. This approach provides better transparency and produces a unified dataset encompassing seven distinct cell types across multiple tissue samples, enhancing data diversity for further model training or fine-tuning.

Despite these improvements, cross-relabeling does not entirely resolve issues related to poor labeling quality or the amount of labeled data. Specifically, our results show lower accuracies persist for underrepresented classes, such as macrophages, which may stem from a limited sample availability and intrinsic challenges in distinguishing these cells based solely on H\&E staining. Furthermore, while our method enhances label specificity, it relies on the initial quality of the broad labels; thus, any fundamental inaccuracies in the original annotations can propagate through the relabeling process. Addressing the overall problem of limited data labels may require integrating additional data sources or utilizing complementary immunohistochemical staining methods.
Although the reported performance metrics are obtained from evaluations on the native test sets of each dataset, it is important to note that the primary application of these classifiers is to perform cross-relabeling, where a model trained on one dataset (e.g., PanNuke) is applied to another (e.g., MoNuSAC) and vice versa. We acknowledge that a more systematic evaluation of cross-dataset generalization is needed and could be performed in future work.

Overall, the refined dataset produced by our approach can enhance the supervised training or fine-tuning of cell segmentation and classification models, especially those that utilize pre-trained foundation models to improve feature extraction robustness. In addition, these models can detect nuanced classes that enable researchers to conduct more detailed analyses of biological processes in computational pathology.

\section{Foundation models for robust cell segmentation and classification}

Accurate cell segmentation and classification in digital pathology are hindered by limited labeled data and the fact that conventional CNNs are unable to capture global contextual information due to their local receptive field constraints \cite{Gheflati_Rivaz_2022,Yang_Marcus_etal.}. Traditional approaches in cell quantification have predominantly relied on CNN encoders, such as ResNet50, given their proven effectiveness in semantic segmentation tasks \cite{Deshmane_2023,Graham_Vu_etal._2019,Mukasheva_Koishiyeva_etal._2024,Stringer_Wang_etal._2021}. However, approaches that include fine-tuning of pretrained CNNs, data augmentation, and stain normalization to partially increase data variability and address staining differences often fail to achieve the necessary generalization and robustness across diverse tissue types and staining conditions \cite{G._Wang_W._Li_etal._2018,Gao_Bagci_etal._2018,Karim_El_Khoury_Martin_Fockedey_etal._2021}.

To overcome these challenges, we leverage an encoder-decoder network that uses a foundation model as the encoder and a CNN upsampling decoder (\hyperref[fig:fig5]{Figure 5}) for simultaneous cell segmentation and classification in 2D patches extracted from WSIs. Foundation models with transformer-based architectures are viable alternatives to CNN-based encoders \cite{Shamshad_Khan_etal._2023,Sourget_2023}. They enable the creation of more advanced architectures that can decode or transform learned features more effectively \cite{Chen_Duan_etal._2023,Cheng_Misra_etal._2022,Xie_Wang_etal._2021}.

\begin{figure}[h!]
    \centering
    \includegraphics[width=\textwidth]{images/Figure_5.pdf}
    \caption{UNETR-like model with foundational model as backbone}
    \label{fig:fig5}
\end{figure}

By utilizing a transformer-based encoder, we incorporate global contextual information into the feature extraction process, which is a key advantage of such architectures \cite{Chen_Lu_etal._2021}. This foundation model integration facilitates accurate pixel-wise segmentation and classification without the need for extensive encoder training, thereby potentially improving generalization across varied cellular structures and tissue types.
In our implementation, we employ a modified UNETR \cite{Hatamizadeh_Tang_etal._2021} architecture that combines a vision transformer (ViT) \cite{Dosovitskiy_Beyer_etal._2021} encoder with a CNN-based decoder. The encoder utilizes the pretrained H-Optimus foundation model, which contains 1.1 billion parameters and is trained on over 500,000 H\&E stained WSIs \cite{Saillard_Jenatton_etal._2024}. We extract outputs from four evenly spaced transformer blocks $Z_i$, where $i \in [1, 14, 26, 38]$, to serve as residual connections for the CNN decoder. We select these blocks based on our observation that features from non-adjacent levels of the encoder lead to better overall performance on the test subset.

The CNN decoder upsamples the feature representations, acquired from the transformer blocks, to generate an intermediate vector that is handled by two task-specific layers that generate cell segmentation and classification masks. The first task-specific layer is the ‘Cellpose head’,  which is used to delineate cell instances. The layer generates horizontal and vertical gradient maps to form vector fields that are refined through gradient tracking in a post-processing step using the Cellpose algorithm \cite{Stringer_Wang_etal._2021}, known for its efficacy in cell segmentation tasks and generalizability across multiple domains \cite{Pachitariu_Stringer_2022,Stringer_Pachitariu_2024}. The second task-specific layer is the "Cell type head", which assigns labels to individual pixels. In the post-processing step, we determine the output classification label of each segmented cell instance by majority voting over the labeled pixels that comprise the cell in the segmentation map.

To evaluate model performance and measure the impact of adding a foundation model as backbone, we compare it to a ResNet50-based model. ResNet50 is a widely used solution for encoders in segmentation architectures in the medical domain \cite{Deshmane_2023,Graham_Vu_etal._2019,Mukasheva_Koishiyeva_etal._2024,Stringer_Wang_etal._2021}. For the H-Optimus-based model, we utilize frozen weights for the encoder and only fine-tune the decoder to take advantage of the extensive pre-training of the foundation model. For the ResNet50-based model we start with ImageNet \cite{Deng_Dong_etal.} weights and train both encoder and decoder parts. Hyperparameters for the training step are set to be identical, where possible, for comparable evaluation. 
For this evaluation, we deliberately use the PanNuke dataset to provide a standardized and controlled comparison between the H‑Optimus and ResNet50-based models (\hyperref[fig:S2]{Appendix Figure S2 (3)}). Specifically, we use two of the default PanNuke dataset splits (66\%) for training and validation, and reserve the third split (33\%) for testing.

To address the challenge of cell class imbalance in the PanNuke dataset, which is a common characteristic in most cell-level H\&E patch datasets, both models’ training processes employ a weighted loss function comprising cross-entropy and focal loss \cite{Lin_Goyal_etal._2018}. The focal loss component is adjusted with coefficients derived from each cell class' instance frequency, emphasizing learning from underrepresented classes and enhancing the model's sensitivity to rare but significant cellular patterns. The cross-entropy loss is augmented with spectral decoupling regularization \cite{Pezeshki_Kaba_etal._2021,Pohjonen_Stürenberg_etal._2022} and spatially varying label smoothing \cite{Islam_Glocker_2021}, which potentially stabilizes training and improves generalization in case of complex tissue morphologies. For optimization, we employ the \textit{AdamW} \cite{Loshchilov_Hutter_2019} to counter unbalanced class scenarios, with cosine annealing learning rate scheduler.

We utilize the scikit-learn library \cite{Van_der_Walt_Schönberger_etal._2014} and HoVer-Net \cite{Graham_Vu_etal._2019} implementations of $R^2$ (the coefficient of determination) and $PQ$ (panoptic quality) to evaluate our experiments. Complete mathematical formulations and detailed explanations of these metrics are provided in \hyperref[chap:S5]{Appendix S5}. To compute confidence intervals, we use nonparametric bootstrapping, where after calculating the metric on the full sample, we generated 1000 bootstrap replicates by resampling with replacement and then determined the 95\% confidence intervals as the 2.5th and 97.5th percentiles of the resulting empirical distribution.

%\hfill

The model comparisons are summarized in \hyperref[tab:2]{Table 2}. The H‑Optimus-based model achieves higher $R^2$ across all cell classes compared to the ResNet50-based model, which means that its predictions are more closely aligned with the PanNuke cell counts, indicating a stronger correlation with the observed data. Notably, the improvement of $R^2_{dead}$ may be an indicator of better global contextual representations provided by the foundation model backbone. In terms of segmentation and classification quality combined, measured by the PQ score, the H‑Optimus-based model demonstrates notable improvements across most cell classes. Overall, the average $R^2$ improved from 0.575 to 0.871, while the average $PQ$ score improved from 0.450 to 0.492, demonstrating better performance of the H-Optimus-based model.

\begin{table}[h!]
\renewcommand{\arraystretch}{1.5}
  \centering
  \caption{Cell quantification metrics for baseline and proposed models (CI 95\%).}
  \label{tab:2}
  \begin{tabular}{|l|c|c|}
    \hline
    %\rowcolor{gray!30}
    Metric             & Resnet50-based            & H-optimus-based              \\
    \hline
    $R^2_{neoplastic}$    & 0.681 (0.576--0.769)       & \textbf{0.941 (0.917--0.960)} \\
    \hline
    $R^2_{inflammatory}$  & 0.863 (0.778--0.903)       & \textbf{0.949 (0.918--0.966)} \\
    \hline
    $R^2_{connective}$    & 0.600 (0.488--0.698)       & 0.609 (0.436--0.772)          \\
    \hline
    $R^2_{dead}$          & 0.097 (-11.389--0.669)     & 0.925 (0.404--0.982)          \\
    \hline
    $R^2_{epithelial}$    & 0.635 (0.490--0.747)       & \textbf{0.930 (0.886--0.964)} \\
    \hline
    $PQ_{neoplastic}$       & 0.517 (0.499--0.535)       & \textbf{0.589 (0.575--0.604)} \\
    \hline
    $PQ_{inflammatory}$     & 0.455 (0.429--0.482)       & \textbf{0.528 (0.507--0.549)} \\
    \hline
    $PQ_{connective}$       & 0.416 (0.400--0.431)       & \textbf{0.451 (0.436--0.465)} \\
    \hline
    $PQ_{dead}$             & 0.374 (0.342--0.408)       & 0.292 (0.209--0.365)          \\
    \hline
    $PQ_{epithelial}$       & 0.488 (0.460--0.519)       & \textbf{0.599 (0.579--0.618)} \\
    \hline
  \end{tabular}
\end{table}

Our results  show that integrating the H‑Optimus foundation model within the UNETR architecture enhances the model's ability to segment and classify cells across diverse tissues from PanNuke data. The pretrained transformer encoder provides robust feature representations, resulting in higher average $R^2$ and $PQ$ scores compared to the CNN-based model. This leads to more reliable cell quantification and more accurate downstream analysis. Additionally, the streamlined fine-tuning process reduces computational overhead and training time, making the model more adaptable for new data.

Despite these advancements, the foundation model-based approach does not fully resolve all challenges related to cell segmentation and classification. We observe lower metric scores for underrepresented classes in the training data. Furthermore, foundation models typically encompass billions of parameters, resulting in substantial computational and memory requirements. It therefore poses challenges for deployment in resource-constrained environments, limiting their practical applicability in certain clinical settings.

\section{Model optimization via Knowledge Distillation}

To address the limitations posed by the extensive size of foundation models, we implement knowledge distillation — a model compression technique that leverages the teacher-student paradigm \cite{Hinton_Vinyals_etal._2015}. By training a smaller, more efficient student model to replicate the output of a larger, pre-trained teacher model, we retain performance while significantly reducing the model's complexity and resource requirements (\hyperref[fig:fig6]{Figure 6}).

\begin{figure}[h!]
    \centering
    \includegraphics[width=\textwidth, height=0.45\textheight, keepaspectratio]{images/Figure_6.pdf}
    \caption{Knowledge distillation framework for training a student model using a pre-trained teacher}
    \label{fig:fig6}
\end{figure}

We employ knowledge distillation to compress the H‑Optimus-based teacher model into a more efficient student model. The teacher model is the modified UNETR architecture with the H‑Optimus foundation model described in the previous chapter. The student model is based on a UNet architecture augmented with residual connections and incorporates a smaller ViT encoder with 9 million parameters \cite{Steiner_Kolesnikov_etal._2022,Wightman_2019}. 

First, we fine-tune the teacher model using the refined dataset from the cross-relabeling procedure (Section 2). Initially we train the decoder of the teacher model while keeping the encoder weights frozen. We split the refined dataset into train (70\%), validation (20\%) and test (10\%) subsets (\hyperref[fig:S2]{Appendix Figure S2 (4)}). During fine-tuning, we use the train and validation subsets, while leaving the test subset for model evaluation. We set the training procedure and model hyperparameters to be identical to those that were used to demonstrate the utility of foundation models for the simultaneous cell segmentation and classification task.

Next, we perform knowledge distillation from teacher to student using the refined dataset used to fine-tune the teacher model. The student model is trained to replicate the teacher model's outputs. We utilize a specialized loss function that aligns the student's predicted probability distribution with the teacher's, incorporating the teacher's class probability distribution derived from the output. Following the methodology of Hinton et al. \cite{Hinton_Vinyals_etal._2015}, we experiment with various hyperparameter settings for the temperature ($T$) and the balancing coefficients ($\alpha$ and $\beta$) in the loss function. We vary $T$ from 1 to 20 and adjust $\alpha$ and $\beta$ to balance the distillation and student losses. Through iterative tuning and evaluation, we identify that setting $T=14$, $\alpha=0.3$, and $\beta=0.7$ yields a configuration that converges and closely approximates the teacher model's performance during training.

Finally, we assess the performance of both models using the $R^2$ and $PQ$ (defined in \hyperref[chap:S5]{Appendix S5}) on the test set of the refined dataset (\hyperref[tab:3]{Table 3}). We observe that the 95\% confidence intervals overlap for most cell types, so we cannot claim statistically significant performance differences between the teacher and student models. One exception appears in the neoplastic class. The teacher model produces an $R^2$ of 0.919, while the student model shows an $R^2$ of 0.852. In addition, the student model achieves higher $PQ$ values for the neoplastic and connective classes, though the confidence intervals show overlap.

\begin{table}[h!]
\renewcommand{\arraystretch}{1.5}
  \centering
  \caption{Cell quantification metrics for teacher and distilled student models (CI 95\%).}
  \label{tab:3}
  \begin{tabular}{|l|c|c|}
    \hline
    %\rowcolor{gray!30}
    Metric & Teacher & Student \\
    \hline
    $R^2_{neoplastic}$    & \textbf{0.919} (0.898--0.939) & 0.852 (0.800--0.891) \\
    \hline
    $R^2_{lymphocyte}$    & 0.969 (0.956--0.977)         & 0.969 (0.956--0.978) \\
    \hline
    $R^2_{connective}$    & 0.694 (0.548--0.809)         & 0.618 (0.469--0.741) \\
    \hline
    $R^2_{dead}$          & 0.755 (0.400--0.908)         & 0.424 (0.100--0.731) \\
    \hline
    $R^2_{epithelial}$    & 0.922 (0.870--0.958)         & 0.843 (0.738--0.917) \\
    \hline
    $R^2_{macrophage}$    & 0.384 (-0.369--0.724)        & 0.704 (0.352--0.859) \\
    \hline
    $R^2_{neutrofil}$     & 0.854 (0.578--0.929)         & 0.833 (0.502--0.925) \\
    \hline
    $PQ_{neoplastic}$       & 0.581 (0.569--0.593)         & 0.601 (0.588--0.613) \\
    \hline
    $PQ_{lymphocyte}$       & 0.536 (0.520--0.553)         & 0.563 (0.544--0.579) \\
    \hline
    $PQ_{connective}$       & 0.436 (0.421--0.451)         & 0.457 (0.441--0.474) \\
    \hline
    $PQ_{dead}$             & 0.272 (0.235--0.315)         & 0.279 (0.201--0.369) \\
    \hline
    $PQ_{epithelial}$       & 0.522 (0.500--0.545)         & 0.530 (0.506--0.555) \\
    \hline
    $PQ_{macrophage}$       & 0.524 (0.459--0.588)         & 0.474 (0.405--0.543) \\
    \hline
    $PQ_{neutrofil}$        & 0.541 (0.490--0.592)         & 0.565 (0.522--0.607) \\
    \hline
  \end{tabular}
\end{table}


We further decompose the $PQ$ metric into its $SQ$ and $DQ$ components (\hyperref[tab:S6]{Appendix Table S6}). Both models produce nearly identical $SQ$ values, which indicates that they predict instance boundaries with similar precision. Although the student model shows some improvement in $DQ$ scores for certain classes, the confidence intervals overlap and do not confirm a statistically significant difference.

We observe that the student and teacher models yield comparable detection performance despite the student model using a much smaller and simpler architecture. A model with fewer parameters reduces the risk of overfitting when training data are scarce relative to the model’s complexity \cite{Farias_Ludermir_etal._2022}. The knowledge distillation process also encourages the student model to focus on the most generalizable detection features learned from the teacher. These factors enable the student model to achieve similar detection performance across different cell types.

Additionally, considering the model sizes reported in \hyperref[tab:4]{Table 4}, the distilled model achieves a significant reduction compared to the teacher model, with a 48-fold decrease in parameter count and a 5.5-fold reduction in on-disk size. In inference mode, the teacher model requires 16 GB of VRAM for a batch size of 32, while the distilled model only needs 3 GB of VRAM for the same batch size. These reductions make the distilled model significantly more practical for fine-tuning and deployment in resource-constrained environments.

\begin{table}[h!]
\renewcommand{\arraystretch}{1.5}
  \centering
  \caption{Parameter counts and size of teacher and distilled model}
  \label{tab:4}
  \adjustbox{max width=\textwidth}{%
  \begin{tabular}{|l|c|c|c|}
    \hline
    %\rowcolor{gray!30}
    Metric & H-optimus-based (Teacher) & mobileViT-based (Student) & Magnitude of difference \\
    \hline
    Parameters count       & 1,158,917,906   & \textbf{24,093,393}   & \textbf{48x}  \\
    \hline
    Estimated Total Size (MB) & 87,912       & \textbf{15,935}    & \textbf{5.5x} \\
    \hline
  \end{tabular}%
}
\end{table}

%\hfill

With recent advancements in complex network architectures and the use of pretrained encoders to achieve state-of-the-art performance \cite{Baumann_Dislich_etal._2024,Hörst_Rempe_etal._2024} in cell segmentation and classification tasks, model size, computational complexity, and processing times have increased. This limits the scalability and accessibility of these models. As we demonstrate, this may be mitigated using knowledge distillation. Studies in the field of natural language processing have demonstrated the efficacy of knowledge distillation in retaining the capabilities of the teacher model while achieving significant reductions in size and complexity \cite{Huangpu_Gao_2024,Sun_Yu_etal.}. 

We demonstrate the feasibility of knowledge distillation in digital pathology, specifically for cell segmentation and classification tasks. Moreover, we achieve this performance while also significantly reducing the parameter count. In addressing the challenge of knowledge transfer, we found that distillation from a transformer-based model to a smaller transformer is more straightforward than attempting to map transformer features to CNN blocks. In our experiments, using a CNN-based network as a student results in worse cell quantification performance due to the structural constraints of CNN feature space dimensions. 

Although our primary approach relies on a transformer-based student model that performs well, it can be further optimized to incorporate advantages from CNN architectures. For example, employing alternative techniques such as using ViT adapters \cite{Chen_Duan_etal._2023} or $1 \times 1$ convolutions to adjust feature map sizes may be beneficial for harnessing CNN advantages like enhanced local feature extraction. Moreover, if additional performance improvements are desired, the process can be further enhanced by applying supplementary knowledge distillation techniques, such as self-distillation \cite{Zhang_Song_etal._2019} or online distillation \cite{Houyon_Cioppa_etal._2023}.

Despite these promising results, further validation on independent datasets is necessary to fully understand the model's limitations. Underrepresented classes may pose challenges when addressing complex cases. Pathologists need to validate these models to adopt them in clinical settings. While the distilled models are smaller and more deployable, a technological gap persists because pathologists traditionally rely on established methods for inspecting WSIs and diagnosing diseases. Addressing the complexities involved in deploying models for inference and supporting pathologists in adopting new tools is essential for integrating these models into clinical workflows.

\section{Model integration with QuPath}
Digital pathology tools with graphical user interfaces are essential for visualizing and analyzing WSIs. To make our student model useful in clinical pathology workflows, it needs to be integrated into a tool that enables inspecting regions, creating annotations, and providing quantitative analyses of biomarkers. Therefore, we integrate the trained student model from the previous chapter into the QuPath open‑source platform \cite{Bankhead_Loughrey_etal._2017}. QuPath provides the required annotation, visualization, and analysis tools to interpret complex histological data, including workflows for cell segmentation, classification, and quantification (\hyperref[fig:fig7]{Figure 7}). 

\begin{figure}[h!]
    \centering
    \includegraphics[width=\textwidth]{images/Figure_7.pdf}
    \caption{Visualization of model-generated cell quantification annotations (left) and the corresponding unannotated slide (right) in QuPath}
    \label{fig:fig7}
\end{figure}

To identify the regions in a WSI critical for prognosticating tumor development, such as specific tumor areas or border regions without overlapping healthy tissue, the pathologist uses QuPath to outline these regions. Then, the pathologist initiates a cell segmentation and classification script through the QuPath interface for the selected regions. The resulting annotations and quantified cell information are then directly overlaid onto the WSI in the QuPath interface. Additional design and implementation details are in \hyperref[chap:S7]{Appendix S7}. 

Two common approaches for integrating deep learning models into QuPath are Java‑based native QuPath extensions \cite{Goldsborough_Philps_etal._2024} and the execution of RESTful API requests to a model server coupled with handling the response via an extension, as demonstrated in the application of cell segmentation models applied to immunofluorescence images \cite{Sugawara_2023}. While the community is actively working on these integration strategies, there is currently no universal solution that fully addresses all integration and performance requirements.

Extensions may offer better integration with QuPath, allowing slightly improved performance and more widespread usage of the built-in QuPath models, but they lack the flexibility to customize models and modify their behavior. For example, the newest version of QuPath includes models such as StarDist \cite{Weigert_Schmidt} and InstanSeg \cite{Goldsborough_Philps_etal._2024} that can perform cell segmentation. Both models pose limitations when applied to simultaneous cell segmentation and classification. StarDist performs well only on convex, round shapes by design, whereas some neoplastic, inflammatory, and connective cells exhibit complex and non-convex shapes. InstanSeg provides only semantic segmentation without assigning classes to the segmented cells.

%\hfill

In contrast, our approach offers an alternative integration strategy. It utilizes the paquo library to directly interact with QuPath’s internal application programming interface from within Python. This enables data exchange and processing without the need for intermediate conversion steps and provides greater control over model customization, retraining, and the incorporation of custom processing steps.

The integration of our custom model with QuPath underscores its potential to significantly enhance the diagnostic process by reducing the time burden on pathologists and enabling them to focus on more complex interpretative tasks using familiar software. Leveraging a tool that is already well-established among pathologists increases the likelihood of its adoption into daily clinical workflows. The quantitative data generated through the automated workflow is critical for both clinical decision-making and research, facilitating more accurate biomarker analysis, enabling robust statistical evaluations, and supporting hypothesis generation and testing. Additionally, by streamlining cell segmentation and classification, the tool enhances the scalability and reproducibility of pathological assessments, ultimately contributing to improved diagnostic accuracy and patient outcomes.

\section{Conclusion and future work}

In this study, we address critical challenges in digital pathology and tackle the usability and deployment issues of the developed models in standard computing environments without the need for high-performance computing systems. Our multi-faceted approach encompasses data refinement through cross-relabeling, leveraging foundation models for robust cell segmentation and classification, optimizing model performance via knowledge distillation, and integrating the optimized model into the QuPath software for practical application. This approach is used to construct a capable, versatile, and adjustable model for cell segmentation and classification, with enhanced performance and usability.

\begin{sloppypar}
While our approach shows potential in the field of computational pathology, certain limitations persist. 
For example, our implementation currently exhibits lower performance in detecting macrophages. 
This serves as an instance of the broader challenge of accurately identifying complex cell types. In order to address this issue, extending our approach to incorporate additional data sources, exploring alternative modeling approaches, and integrating other imaging modalities such as immunohistochemical staining may help improve detection accuracy. Moreover, although the distilled model reduces computational demands, integrating advanced deep learning models into clinical practice requires addressing technological gaps and potential resistance to adopting new tools within established diagnostic processes.
\end{sloppypar}

Future work could focus on several key areas to refine the proposed approach and facilitate its adoption in clinical environments. Enhancing the cell-relabeling process with additional datasets \cite{Graham_Jahanifar_etal._2021} could improve the representation of underrepresented cell types and enhance overall model performance. Also, incorporating additional data sources, such as multi-modal imaging or complementary staining methods, may address limitations related to cell type differentiation and class imbalance. Exploring other foundation models \cite{Vorontsov_Bozkurt_etal._2024,Zimmermann_Vorontsov_etal._2024} or introducing additional modalities \cite{Ding_Wagner_etal._2024,Vaidya_Zhang_etal._2025} may provide alternative architectures better suited to specific tasks or offer improved efficiency. Implementing more complex knowledge distillation techniques \cite{Houyon_Cioppa_etal._2023,Zhang_Song_etal._2019} could further optimize the model's performance and adaptability. Additionally, deeper integration with QuPath or other digital pathology software could provide pathologists more control over cell quantification analysis directly within the QuPath interface, thereby increasing accessibility and usability. Such enhancements would not only refine model performance but also ensure greater adaptability and scalability within various clinical environments. Finally, extensive validation of the model by pathologists and benchmarking against independent datasets are essential steps toward establishing the model's reliability and fostering confidence in its clinical utility.

\section*{Acknowledgments} 
This work was funded in part by the Research Council of Norway grant no. 309439 SFI Visual Intelligence, and the North Norwegian Health Authority grant no. HNF1521-20.

\bibliographystyle{IEEEtran}
\begin{sloppypar}
\begin{thebibliography}{99}

\bibitem{chaplot2020neural} Chaplot, Devendra Singh, et al. "Neural topological slam for visual navigation." Proceedings of the IEEE/CVF conference on computer vision and pattern recognition. 2020.

\bibitem{maksymets2021thda} Maksymets, Oleksandr, et al. "Thda: Treasure hunt data augmentation for semantic navigation." Proceedings of the IEEE/CVF International Conference on Computer Vision. 2021.

\bibitem{mezghan2022memory} Mezghan, Lina, et al. "Memory-augmented reinforcement learning for image-goal navigation." 2022 IEEE/RSJ International Conference on Intelligent Robots and Systems (IROS). IEEE, 2022.

\bibitem{al2022zero} Al-Halah, Ziad, Santhosh Kumar Ramakrishnan, and Kristen Grauman. "Zero experience required: Plug \& play modular transfer learning for semantic visual navigation." Proceedings of the IEEE/CVF Conference on Computer Vision and Pattern Recognition. 2022.

\bibitem{ye2021auxiliary} Ye, Joel, et al. "Auxiliary tasks and exploration enable objectgoal navigation." Proceedings of the IEEE/CVF international conference on computer vision. 2021.

\bibitem{chaplot2020object} Chaplot, Devendra Singh, et al. "Object goal navigation using goal-oriented semantic exploration." Advances in Neural Information Processing Systems 33 (2020)

\bibitem{ramakrishnan2022poni} Ramakrishnan, Santhosh Kumar, et al. "Poni: Potential functions for objectgoal navigation with interaction-free learning." Proceedings of the IEEE/CVF Conference on Computer Vision and Pattern Recognition. 2022.

\bibitem{ramrakhya2022habitat} Ramrakhya, Ram, et al. "Habitat-web: Learning embodied object-search strategies from human demonstrations at scale." Proceedings of the IEEE/CVF Conference on Computer Vision and Pattern Recognition. 2022.

\bibitem{mousavian2019visual} Mousavian, Arsalan, et al. "Visual representations for semantic target driven navigation." 2019 International Conference on Robotics and Automation (ICRA). IEEE, 2019.

\bibitem{dhariwal2021diffusion} Dhariwal, Prafulla, and Alexander Nichol. "Diffusion models beat gans on image synthesis." Advances in neural information processing systems 34 (2021)

\bibitem{ho2022classifier} Ho, Jonathan, and Tim Salimans. "Classifier-free diffusion guidance." arXiv preprint arXiv:2207.12598 (2022).

\bibitem{nichol2021glide} Nichol, Alex, et al. "Glide: Towards photorealistic image generation and editing with text-guided diffusion models." arXiv preprint arXiv:2112.10741 (2021)

\bibitem{brooks2023instructpix2pix} Brooks, Tim, Aleksander Holynski, and Alexei A. Efros. "Instructpix2pix: Learning to follow image editing instructions." Proceedings of the IEEE/CVF Conference on Computer Vision and Pattern Recognition. 2023.

\bibitem{fu2023guiding} Fu, Tsu-Jui, et al. "Guiding instruction-based image editing via multimodal large language models." arXiv preprint arXiv:2309.17102 (2023).

\bibitem{geng2024instructdiffusion} Geng, Zigang, et al. "Instructdiffusion: A generalist modeling interface for vision tasks." Proceedings of the IEEE/CVF Conference on Computer Vision and Pattern Recognition. 2024.

\bibitem{zhou2024minedreamer} Zhou, Enshen, et al. "Minedreamer: Learning to follow instructions via chain-of-imagination for simulated-world control." arXiv preprint arXiv:2403.12037 (2024).

\bibitem{zhou2023esc} Zhou, Kaiwen, et al. "Esc: Exploration with soft commonsense constraints for zero-shot object navigation." International Conference on Machine Learning. PMLR, 2023.

\bibitem{yu2023l3mvn} Yu, Bangguo, Hamidreza Kasaei, and Ming Cao. "L3mvn: Leveraging large language models for visual target navigation." 2023 IEEE/RSJ International Conference on Intelligent Robots and Systems (IROS). IEEE, 2023.

\bibitem{gadre2023cows} Gadre, Samir Yitzhak, et al. "Cows on pasture: Baselines and benchmarks for language-driven zero-shot object navigation." Proceedings of the IEEE/CVF Conference on Computer Vision and Pattern Recognition. 2023.

\bibitem{shah2023navigation} Shah, Dhruv, et al. "Navigation with large language models: Semantic guesswork as a heuristic for planning." Conference on Robot Learning. PMLR, 2023.

\bibitem{cai2024bridging} Cai, Wenzhe, et al. "Bridging zero-shot object navigation and foundation models through pixel-guided navigation skill." 2024 IEEE International Conference on Robotics and Automation (ICRA). IEEE, 2024.

\bibitem{yu2023co} Yu, Bangguo, Hamidreza Kasaei, and Ming Cao. "Co-NavGPT: Multi-robot cooperative visual semantic navigation using large language models." arXiv preprint arXiv:2310.07937 (2023).

\bibitem{wu2024voronav} Wu, Pengying, et al. "Voronav: Voronoi-based zero-shot object navigation with large language model." arXiv preprint arXiv:2401.02695 (2024).

\bibitem{qin2023mp5} Qin, Yiran, et al. "Mp5: A multi-modal open-ended embodied system in minecraft via active perception." arXiv preprint arXiv:2312.07472 (2023).

\bibitem{du2024learning} Du, Yilun, et al. "Learning universal policies via text-guided video generation." Advances in Neural Information Processing Systems 36 (2024).

\bibitem{ajay2024compositional} Ajay, Anurag, et al. "Compositional foundation models for hierarchical planning." Advances in Neural Information Processing Systems 36 (2024).

\bibitem{liang2024skilldiffuser} Liang, Zhixuan, et al. "Skilldiffuser: Interpretable hierarchical planning via skill abstractions in diffusion-based task execution." Proceedings of the IEEE/CVF Conference on Computer Vision and Pattern Recognition. 2024.

\bibitem{heusel2017gans} Heusel, Martin, et al. "Gans trained by a two time-scale update rule converge to a local nash equilibrium." Advances in neural information processing systems 30 (2017).

\bibitem{zhang2018unreasonable} Zhang, Richard, et al. "The unreasonable effectiveness of deep features as a perceptual metric." Proceedings of the IEEE conference on computer vision and pattern recognition. 2018.

\bibitem{brown2020language} Brown, Tom B. "Language models are few-shot learners." arXiv preprint arXiv:2005.14165 (2020).

\bibitem{podell2023sdxl} Podell, Dustin, et al. "Sdxl: Improving latent diffusion models for high-resolution image synthesis." arXiv preprint arXiv:2307.01952 (2023).

\bibitem{brohan2022rt} Brohan, Anthony, et al. "Rt-1: Robotics transformer for real-world control at scale." arXiv preprint arXiv:2212.06817 (2022).

\bibitem{brohan2023rt} Brohan, Anthony, et al. "Rt-2: Vision-language-action models transfer web knowledge to robotic control." arXiv preprint arXiv:2307.15818 (2023).

\bibitem{li2024manipllm} Li, Xiaoqi, et al. "Manipllm: Embodied multimodal large language model for object-centric robotic manipulation." Proceedings of the IEEE/CVF Conference on Computer Vision and Pattern Recognition. 2024.

\bibitem{shah2023vint} Shah, Dhruv, et al. "ViNT: A foundation model for visual navigation." arXiv preprint arXiv:2306.14846 (2023).

\bibitem{liu2024visual} Liu, Haotian, et al. "Visual instruction tuning." Advances in neural information processing systems 36 (2024).

\bibitem{hu2021lora} Hu, Edward J., et al. "Lora: Low-rank adaptation of large language models." arXiv preprint arXiv:2106.09685 (2021).

\bibitem{qin2023supfusion} Qin, Yiran, et al. "SupFusion: Supervised LiDAR-camera fusion for 3D object detection." Proceedings of the IEEE/CVF International Conference on Computer Vision. 2023.

\bibitem{qin2024worldsimbench} Qin, Yiran, et al. "Worldsimbench: Towards video generation models as world simulators." arXiv preprint arXiv:2410.18072 (2024).

\bibitem{yu2025gamefactory} Yu, Jiwen, et al. "GameFactory: Creating New Games with Generative Interactive Videos." arXiv preprint arXiv:2501.08325 (2025).

\bibitem{zhou2024code} Zhou, Enshen, et al. "Code-as-Monitor: Constraint-aware Visual Programming for Reactive and Proactive Robotic Failure Detection." arXiv preprint arXiv:2412.04455 (2024).

\bibitem{zhang2024ad} Zhang, Zaibin, et al. "AD-H: Autonomous Driving with Hierarchical Agents." arXiv preprint arXiv:2406.03474 (2024).

\bibitem{wang2024toward} Wang, Chaoqun, et al. "Toward Accurate Camera-based 3D Object Detection via Cascade Depth Estimation and Calibration." arXiv preprint arXiv:2402.04883 (2024).

\bibitem{huang2024story3d} Huang, Yuzhou, et al. "Story3d-agent: Exploring 3d storytelling visualization with large language models." arXiv preprint arXiv:2408.11801 (2024).

\bibitem{savinov2018semi} Savinov, Nikolay, Alexey Dosovitskiy, and Vladlen Koltun. "Semi-parametric topological memory for navigation." arXiv preprint arXiv:1803.00653 (2018).

\bibitem{majumdar2022zson} Majumdar, Arjun, et al. "Zson: Zero-shot object-goal navigation using multimodal goal embeddings." Advances in Neural Information Processing Systems 35 (2022): 32340-32352.

\bibitem{yadav2023offline} Yadav, Karmesh, et al. "Offline visual representation learning for embodied navigation." Workshop on Reincarnating Reinforcement Learning at ICLR 2023. 2023.

\bibitem{yadav2023ovrl} Yadav, Karmesh, et al. "Ovrl-v2: A simple state-of-art baseline for imagenav and objectnav." arXiv preprint arXiv:2303.07798 (2023).

\bibitem{sun2024fgprompt} Sun, Xinyu, et al. "FGPrompt: fine-grained goal prompting for image-goal navigation." Advances in Neural Information Processing Systems 36 (2024).

\bibitem{zhu2017target} Zhu, Yuke, et al. "Target-driven visual navigation in indoor scenes using deep reinforcement learning." 2017 IEEE international conference on robotics and automation (ICRA). IEEE, 2017.

\bibitem{koh2024generating} Koh, Jing Yu, Daniel Fried, and Russ R. Salakhutdinov. "Generating images with multimodal language models." Advances in Neural Information Processing Systems 36 (2024).

\bibitem{krantz2022instance} Krantz, Jacob, et al. "Instance-specific image goal navigation: Training embodied agents to find object instances." arXiv preprint arXiv:2211.15876 (2022).

\bibitem{schulman2017proximal} Schulman, John, et al. "Proximal policy optimization algorithms." arXiv preprint arXiv:1707.06347 (2017).

\bibitem{anderson2018evaluation} Anderson, Peter, et al. "On evaluation of embodied navigation agents." arXiv preprint arXiv:1807.06757 (2018).

\bibitem{lin2024navcot} Lin, Bingqian, et al. "NavCoT: Boosting LLM-Based Vision-and-Language Navigation via Learning Disentangled Reasoning." arXiv preprint arXiv:2403.07376 (2024).

\bibitem{NavGPT} Zhou, Gengze, Yicong Hong, and Qi Wu. "Navgpt: Explicit reasoning in vision-and-language navigation with large language models." Proceedings of the AAAI Conference on Artificial Intelligence.

\bibitem{hahn2021no} Hahn, Meera, et al. "No rl, no simulation: Learning to navigate without navigating." Advances in Neural Information Processing Systems 34 (2021): 26661-26673.

\bibitem{li2025t2isafety} Li, Lijun, et al. "T2ISafety: Benchmark for Assessing Fairness, Toxicity, and Privacy in Image Generation." arXiv preprint arXiv:2501.12612 (2025).

\bibitem{an2024agfsync} An, Jingkun, et al. "AGFSync: Leveraging AI-Generated Feedback for Preference Optimization in Text-to-Image Generation." arXiv preprint arXiv:2403.13352 (2024).


\end{thebibliography}
\end{sloppypar}

\clearpage
\beginsupplement
\section*{Appendix}
\renewcommand{\thesubsection}{S\arabic{subsection}}

\subsection{\label{chap:S1}PanNuke and MoNuSAC preprocessing}
The PanNuke dataset comprises a set of 7,901 RGB patches, each with dimensions of $256 \times 256$ pixels, which we set as the standard patch size for our analysis. In contrast, the MoNuSAC dataset encompasses 294 images of heterogeneous dimensions. To standardize the MoNuSAC images with our experiments, we implement a standardization protocol. Specifically, for images exceeding the dimensions of $256 \times 256$ pixels, we segment them into equal-sized patches and apply mirror padding to the remaining portions to avoid information loss at the peripherals. Patches with dimensions less than $128 \times 128$ pixels are excluded from the dataset due to the insufficient resolution to capture relevant cellular details. For patches where either dimension falls between 128 and 256 pixels, we employ upsampling to achieve the standard patch size. As a result, we obtain a total of 2,823 RGB patches derived from the MoNuSAC dataset for subsequent analysis. For additional details on the MoNuSAC data preparation process, refer to the source code \cite{Shvetsov_2025a}.
\clearpage

\subsection{\label{chap:S2}Data usage for the methodology}

\counterwithin{figure}{subsection}
\renewcommand{\thefigure}{S\arabic{subsection}}

\begin{figure}[h!]
    \centering
    \includegraphics[width=\textwidth, height=0.85\textheight, keepaspectratio]{images/A2.pdf}
    \caption{Overview of the methodology for cross-labeling, dataset refinement, and model comparison. (1) Cross-relabeling - training and testing cell classification models, (2) Cross-relabeling - using cell classification models to create refined dataset, (3) Fine-tuning and training models for comparison, (4) Student knowledge distillation with refined dataset}
    \label{fig:S2}
\end{figure}
\clearpage

\subsection{\label{chap:S3}Confusion matrices for classification models}
\counterwithin{figure}{subsection}
\renewcommand{\thefigure}{S\arabic{subsection}.\arabic{figure}}

\begin{figure}[h!]
    \centering
    \includegraphics[width=\textwidth, height=0.4\textheight, keepaspectratio]{images/A3_1.pdf}
    \caption{Confusion matrix for PanNuke trained model}
    \label{fig:S3.1}
\end{figure}

\begin{figure}[h!]
    \centering
    \includegraphics[width=\textwidth, height=0.4\textheight, keepaspectratio]{images/A3_2.pdf}
    \caption{Confusion matrix for MoNuSAC trained model}
    \label{fig:S3.2}
\end{figure}

\clearpage

\subsection{\label{chap:S4}Datasets cell counts}

\counterwithin{table}{subsection}
\renewcommand{\thetable}{S\arabic{subsection}}

\begin{table}[h!]
\renewcommand{\arraystretch}{2.0}
\centering
\caption{\label{tab:S4}Cell counts for PanNuke, MoNuSAC and refined datasets. Numbers in parentheses indicate preprocessed cell counts for cell classifier models training and testing.}
%\adjustbox{max width=\textwidth}{%
\begin{tabular}{|l|c|c|c|}
\hline
%\rowcolor{gray!30}
Cell type & PanNuke & MoNuSAC & Refined \\
\hline
Neoplastic & 77,403 (68,031) & - & 105,451 \\
\hline
Epithelial & 26,572 (23,207) & - & 29,926 \\
\hline
Epithelial (benign and malignant) & - & 31,402 & - \\
\hline
Inflammatory & 32,276 & - & - \\
\hline
Lymphocytes & - & 37,045 (33,104) & 65,275 \\
\hline
Neutrophils & - & 1,355 (1,252) & 3,833 \\
\hline
Macrophage & - & 1,842 (1,695) & 3,410 \\
\hline
Dead & 2,908 & - & 2,908 \\
\hline
Connective & 50,585 & - & 50,585 \\
\hline
\end{tabular}
%
%}
\end{table}



\clearpage

\subsection{\label{chap:S5}Definition of validation metrics}
\counterwithin{equation}{subsection}
\renewcommand{\theequation}{\arabic{equation}}

\subsubsection{\label{chap:S5.1}R\textsuperscript{2}}
The coefficient of determination, denoted as $R^2$, is a statistical measure that represents the proportion of variance in the dependent variable that is predictable from the independent variables. In the context of cell quantification in pathology, $R^2$ is used to assess how well the predicted quantities of different cell types in a patch align with the actual quantities observed in the ground truth data, with higher values representing more accurate quantification. $R^2$ is defined as
\begin{equation*}
R^2 = 1 - \frac{\sum_{i=1}^n (y_i - \hat{y}_i)^2}{\sum_{i=1}^n (y_i - \bar{y})^2},
\end{equation*}
where $y_i$ represents the actual number of cells of a specific type in the $i$-th image, $\hat{y}_i$ represents the predicted number of cells of that type in the $i$-th image, $\bar{y}$ is the mean of the actual numbers across all images, and $n$ is the total number of images in the dataset.

The $R^2$ metric has a range of $(-\infty, 1]$. An $R^2$ of 1 indicates perfect prediction, where all predicted values exactly match the actual values. An $R^2$ of 0 suggests that the model explains none of the variability of the response data around its mean. If $R^2$ is negative, it indicates that the model performs worse than a model that simply predicts the mean of the actual values for all observations.

\subsubsection{\label{chap:S5.2}PQ}
Panoptic Quality ($PQ$) is a comprehensive metric used to evaluate the performance of segmentation models in tasks that require both instance segmentation and classification. $PQ$ provides a single score that encapsulates both the detection accuracy (i.e., how many objects were correctly identified) and the segmentation quality (i.e., how accurately the objects' boundaries were delineated). This metric is particularly useful in multiclass scenarios where each pixel is classified into distinct categories, such as different cell types in pathology images.

$PQ$ is calculated as the product of two terms: Detection Quality ($DQ$) and Segmentation Quality ($SQ$). It can be expressed as
\begin{equation*}
PQ = DQ \cdot SQ,
\end{equation*}
where
\begin{equation*}
DQ = \frac{TP}{TP + 0.5\, FP + 0.5\, FN},
\end{equation*}
\begin{equation*}
SQ = \frac{\sum_{(p, g) \in \mathcal{M}} IoU(p, g)}{TP}.
\end{equation*}
In these formulas, $TP$ denotes the number of correctly matched instances between ground truth and prediction, $FP$ denotes the predicted instances that have no corresponding ground truth, $FN$ denotes the ground truth instances that were not detected, $IoU(p, g)$ is the Intersection over Union for a pair of matched instances $p$ (prediction) and $g$ (ground truth), and $\mathcal{M}$ is the set of matched pairs.

The $PQ$ metric is calculated for each class and is averaged across classes to provide a global performance measure.

The $PQ$ score has a range of $[0, 1.0]$, where a higher score indicates better performance in both detecting and segmenting the instances correctly. A $PQ$ of 1 signifies perfect identification and segmentation of all instances, whereas a $PQ$ of 0 indicates that no instances were correctly identified and segmented.

\clearpage

\subsection{\label{chap:S6}Segmentation and Detection quality metrics for teacher and student models}

\begin{table}[h!]
\renewcommand{\arraystretch}{2.0}
\centering
\caption{Segmentation and detection quality for student and teacher models (CI 95\%)}
\label{tab:S6}
%\adjustbox{max width=\textwidth}{%
\begin{tabular}{|l|c|c|}
\hline
%\rowcolor{gray!30}
Metric & Teacher & Student \\
\hline
$SQ_{neoplastic}$ & 0.819 (0.815--0.823) & 0.824 (0.819--0.828) \\
\hline
$SQ_{lymphocyte}$ & 0.795 (0.788--0.802) & 0.790 (0.783--0.796) \\
\hline
$SQ_{connective}$ & 0.770 (0.762--0.776) & 0.780 (0.772--0.786) \\
\hline
$SQ_{dead}$ & 0.659 (0.623--0.688) & 0.657 (0.624--0.695) \\
\hline
$SQ_{epithelial}$ & 0.780 (0.770--0.790) & 0.788 (0.779--0.797) \\
\hline
$SQ_{macrophage}$ & 0.788 (0.760--0.810) & 0.757 (0.730--0.783) \\
\hline
$SQ_{neutrofil}$ & 0.782 (0.761--0.801) & 0.775 (0.759--0.792) \\
\hline
$DQ_{neoplastic}$ & 0.706 (0.692--0.719) & 0.727 (0.712--0.741) \\
\hline
$DQ_{lymphocyte}$ & 0.675 (0.656--0.698) & 0.713 (0.691--0.734) \\
\hline
$DQ_{connective}$ & 0.566 (0.546--0.584) & 0.583 (0.565--0.602) \\
\hline
$DQ_{dead}$ & 0.410 (0.361--0.465) & 0.435 (0.306--0.561) \\
\hline
$DQ_{epithelial}$ & 0.668 (0.639--0.694) & 0.673 (0.644--0.702) \\
\hline
$DQ_{macrophage}$ & 0.657 (0.583--0.727) & 0.615 (0.531--0.703) \\
\hline
$DQ_{neutrofil}$ & 0.691 (0.625--0.753) & 0.729 (0.679--0.778) \\
\hline
\end{tabular}
%
%}
\end{table}

\clearpage

\subsection{\label{chap:S7}QuPath integration method}
We adopt an integration strategy leveraging the paquo \cite{Bayer_AG} library, a Python package that enables direct interaction with QuPath’s internal API, thereby facilitating seamless data exchange without intermediate conversion steps. The data processing pipeline (\hyperref[fig:S7]{Appendix Figure S7}) begins with the acquisition of WSIs and their associated annotations from QuPath, which are represented as Shapely \cite{Gillies_Wel_etal._2024} polygons. Utilizing paquo, we directly read, create, and modify these annotations and detections within a QuPath project in the Python environment. Images are then cropped using these polygons and processed by cell segmentation and classification models employing standard vision processing toolkits such as OpenCV, pyvips, and PyTorch. Additionally, QuPath employs Groovy scripts to initiate a Python process that starts the entire pipeline from QuPath graphical interface: fetching polygons, extracting images from them, and running deep learning model inference on the cropped images. 
The results are returned to QuPath, leveraging paquo's Python bindings to manipulate QuPath data while minimizing the computational overhead typically associated with cross-environment communication.

\counterwithin{figure}{subsection}
\renewcommand{\thefigure}{S\arabic{subsection}}

\begin{figure}[h!]
    \centering
    \includegraphics[width=\textwidth]{images/A7.pdf}
    \caption{QuPath integration workflow using Python environment}
    \label{fig:S7}
\end{figure}

Compared to traditional workflows that involve exporting annotations as GeoJSON, classifying them in Python, and reimporting them into QuPath, our approach offers several advantages. We eliminate the need to switch between programming languages, providing a cohesive and streamlined development process entirely within QuPath software and removing the necessity to use other tools. Meanwhile, we avoid storing annotations as intermediate JSON files unless required for external use or archiving. By conducting the entire inference and post-processing workflow within the Python environment, we leverage the power and flexibility of Python libraries for image processing and machine learning. This approach also enables adjustments to any set of labels and models, thereby improving its applicability.

%\hfill

The distilled model and QuPath integration code are packaged into a Docker container, enabling streamlined execution with the Docker engine. Detailed integration code and deployment instructions can be found in the GitHub repository \cite{Shvetsov_2025b}.

Despite these benefits, we acknowledge that the paquo library is a proof‑of‑concept project in its early development stage and has not been tested across all versions of QuPath.

\clearpage

\subsection{\label{chap:S8}Data and code availability statement}
All datasets, models, and code used in this study are publicly available and can be obtained from the repositories listed below. 
The PanNuke \cite{Gamper_Koohbanani_etal._2019} and MoNuSAC \cite{Verma_Kumar_etal._2021} datasets are publicly accessible, and download information along with detailed descriptions can be found in their respective articles. Preprocessing scripts for PanNuke and MoNuSAC data, as well as individual cell extraction scripts, are available on GitHub \cite{Shvetsov_2025a}. The H-Optimus foundation model used in our experiments can be downloaded from the HuggingFace repository \cite{hoptimus2024}, and model information is available on GitHub \cite{Saillard_Jenatton_etal._2024}. In addition, the integration code for QuPath and the distilled model packaged in a Docker container are provided in the repository \cite{Shvetsov_2025b}, and paquo Python library is available from the authors GitHub repository \cite{Bayer_AG}.
\clearpage

\end{document}



\section{Method: \framework}
To equip LLM with personalized tool-use capability, we conduct a two-stage training process: 1) personalized SFT, where LLM is fine-tuned on \benchmark to acquire fundamental proficiency in personalized tool usage, and 2) personalized DPO, where LLM is optimized on a preference dataset for better alignment with user preferences.

\paragraph{Personalized SFT.}
The first stage in our approach is Supervised Fine-Tuning (SFT), where we directly fine-tune LLM on the training set of \benchmark. Given the user's instruction $q_u$, interaction history $\mathcal{H}_u$, and the candidate tool set $\mathcal{T}$ as inputs, LLM is trained to generate the ground truth tool call \(c\). $\mathcal{H}_u$ uniformly covers all three types of user interactions to capture diverse user preferences. In this way, LLM can obtain basic personalized tool-usage experiences by understanding both the user needs and preferences.

\paragraph{Personalized DPO.}
In the second stage, we further enhance the LLM's performance through direct preference optimization (DPO)~\cite{NEURIPS2023_a85b405e}. 
Our goal is to guide the LLM to call the user's preferred tools instead of non-preferred ones.
Specifically, for each user instruction $q_u$, we collect multiple tool calls generated by LLM after the SFT stage. 
Then we select the user's preferred and non-preferred tool calls \(c_w\) and \(c_l\) based on the user's tool preference constructed in \benchmark.
\(c_w\) and \(c_l\) will be used to construct the preference dataset \(\mathcal{D}_{\text{DPO}} = \{ (x, c_w, c_l) \}\), where \(x\) denotes the input, including the user instruction $q_u$, interaction history $\mathcal{H}_u$, and the candidate tool set $\mathcal{T}$.
We then apply DPO to optimize the LLM by guiding it to generate the desired tool call \(c_w\) while avoid generating \(c_l\).
% generation direction, and a “rejected” pair to train the LLM to avoid specific outputs.
% prioritize
% encouraging it to generate function parameters similar to \(p_{i}^{\text{b}}\) and discouraging it from generating function parameters similar to \(p_{i}^{\text{w}}\). 
The loss function can be defined as:
\begin{equation} \label{dpo_loss}
\small
{
\mathcal{L} = -\mathbb{E} \left[ \log \sigma \left( \beta \log \frac{\pi_{\theta}(c_w \mid x)}{\pi_{\text{ref}}(c_w \mid x)} - \beta \log \frac{\pi_{\theta}(c_l \mid x)}{\pi_{\text{ref}}(c_l \mid x)} \right) \right],
}
\end{equation}
where \(\sigma\) is the logistic function and \(\beta\) is a weighting parameter that controls the deviation of the policy model $\pi_{\theta}$ (i.e., the LLM we need to optimize) from the reference model $\pi_{\text{ref}}$ (i.e., the LLM after SFT stage).
% the sensitivity of the model's preference to the log-ratio difference between the policy model \(\pi_{\theta}\) for optimization and reference model \(\pi_{\text{ref}}\) derived from the SFT stage.
In this way, LLM can focus on generating tool calls that are more aligned with individual user preferences.
% By directly optimizing for user preferences

\section{Experiments}
\subsection{Setup}
\paragraph{Baselines.} 
% following (Huang et al., 2024a; Zhuang et al., 2023). 
We adopt multiple LLMs from both closed-source and open-source models to ensure a comprehensive evaluation.
For closed-source LLMs, we select two representative models: GPT-4o and GPT-4o-mini from OpenAI.
For open-source LLMs, we include a wide spectrum of models, i.e., LLaMA-3.1-8B~\cite{dubey2024llama}, QWen-2.5-7B~\citep{yang2024qwen2}, Vicuna-7B-v1.5~\cite{chiang2023vicuna} and Mistral-7B-v0.3~\cite{jiang2023mistral}.

\paragraph{Implementation details.} 
% \footnote{\url{https://openai.com/index/introducing-chatgpt-and-whisper-apis/}.} 
% since it its the instruction following .
% The number of candidate tools $N$ is set to $10$, which 
In \benchmark construction, we employ gpt-4o-mini
as the LLM for tool understanding and generation of user instructions and interaction history. 
The candidate tool set consists of three parts: the ground-truth tool along with all other tools sharing the same functionality, five tools retrieved using ToolRetriever~\cite{qin2024toolllm}, and the remaining tools that were randomly sampled.
% The tools retrieved We adopt ToolRetriever~\cite{qin2024toolllm} as the dense retriever which is specifically finetuned on tool retrieval datasets. 


% highlighting several key insights.
\subsection{Main Results}
The detailed experimental results are shown in Table~\ref{main_results}. 
From the results, we can obtain the following key findings.
1) It can be observed that the performance of LLMs is generally unsatisfactory, particularly in tool accuracy with the majority failing to exceed 50\%. This indicates that current LLMs are severely limited in personalized tool-use capabilities. Additionally, the lower tool accuracy compared to parameter accuracy further suggests that personalized tool selection is more challenging than parameter configuration. This is because LLMs must account for both implicit user preferences and explicit user requirements when determining which tool to use.
2) Most LLMs perform worse in the rating-integrated and chronological settings. 
This is likely due to the inclusion of non-preferred interactions in the interaction history, which confuses LLMs and hinders their ability to accurately recognize user preferences. Notably, the chronological setting yields the lowest scores, suggesting that capturing evolving user preferences over time is even more challenging than interpreting explicit user ratings.
3) Our proposed \benchmark significantly outperforms all closed-source and open-source LLMs, demonstrating both effectiveness and robustness. It maintains strong performance, even in the two more challenging settings, by enabling the LLM to better understand diverse manifestations of user preferences and facilitate personalized tool usage.
% with superior
% It can be observed that existing open-source LLMs lay behind closed-source LLMs, i.e., GPT-4o and GPT-4o-mini, across all three scenarios. This is reasonable since GPT-4o is well known for its superior instruction-following and comprehension abilities compared with most open-source LLMs. 

\subsection{Ablation Study}
We conduct ablation studies to investigate the efficacy of the two-stage training process in our \framework.
First, we remove the second training stage (i.e., personalized DPO) to assess its contribution.
Then, we examine the impact of the SFT stage by directly conducting DPO training on the initial LLaMA3-8B model.
Table~\ref{main_results} reports the performance on the test set of \benchmark in all three settings.
The results indicate that the SFT stage is crucial for personalized tool learning performance, as it endows the model with fundamental tool usage and personalization capabilities. Removing the DPO stage results in a slight performance drop, suggesting that it can further refine the tool usage alignment with user preferences.

\subsection{In-depth Analysis}

\begin{figure}[tbp]
    \centering
    \includegraphics[width=1.0\linewidth]{scores_wo_history.pdf}
    \caption{Performance comparison of tool accuracy when provided with and without interaction history.}
    \label{fig:scores_wo_history}
\vspace{-1em}
\end{figure}

\begin{figure}[!t]
    \centering
    \includegraphics[width=1.0\linewidth]{scores_length.pdf}
    \caption{Performance comparison of tool accuracy on different interaction history length in the preferred-only setting.}
    \label{fig:scores_length}
\vspace{-1em}
\end{figure}



\begin{table*}[t]
\centering
\caption{The percentage (\%) of different error types in LLMs on the test set of \benchmark under preferred-only and chronological settings. IF, TH, TFM, TPM, PNM, PVM stand for Invalid Format, Tool Hallucination, Tool Functionality Mismatch, Tool Preference Mismatch, Parameter Name Mismatch and Parameter Value Mismatch errors, respectively. 
% \% improve represents the relative improvement achieved by our method over the previously best tool retrieval method.
} 
% \small
\resizebox{0.87\linewidth}{!}
{
\begin{tabular}{@{}l|c|ccc|cc|c|ccc|cc@{}}
\toprule
\multirow{2}{*}{\textbf{Models}} & \multicolumn{6}{c|}{\textbf{\textsc{Preferred-only}}} & \multicolumn{6}{c}{\textbf{\textsc{Chronological}}} \\ 
\cmidrule(lr){2-13}
&IF  &TH  &TFM &TPM  &PNM  &PVM &IF  &TH  &TFM &TPM  &PNM  &PVM \\ 
 % &IF (\%)  &TH (\%)  &TFM (\%) &TPM (\%) &PNM (\%) &PVM (\%) &IF (\%)  &TH (\%)  &TFM (\%) &TPM (\%) &PNM (\%) &PVM (\%) \\ 
 \midrule
Qwen2.5-7B &$10.9$  &$3.6$  &$19.6$  &$25.5$  &$10.4$  &$14.9$ &$11.2$  &$2.3$  &$7.1$  &$54.6$  &$5.4$  &$13.2$  \\ 
LLaMA3-8B &$2.5$  &$5.3$  &$19.3$  &$24.8$  &$11.4$  &$15.0$ &$2.5$  &$3.7$  &$2.8$  &$64.4$  &$6.8$  &$12.6$ \\
GPT-4o-mini &\bf 0.1  &$3.5$  &$20.1$  &$24.4$  &$10.6$  &$16.4$ &\bf 0.0  &$1.5$  &$6.9$  &$60.2$  &$6.2$  &$13.3$ \\ 
GPT-4o &$0.5$  &\bf 1.5 &$20.0$  &$30.3$  &$7.9$  &$14.0$ &$1.3$  &\bf 1.1 &$6.6$  &$57.3$  &\bf 4.7  &$12.2$\\ 
% \midrule
\textbf{\framework} &$0.6$  &$3.4$  &\bf 9.3  &\bf 12.0  &\bf 7.6  &\bf 3.9 &$0.5$  &$1.8$  &\bf 6.5  &\bf 10.4  &$5.1$  &\bf 3.1 \\ 
\bottomrule
\end{tabular}} 
\label{error_results}
\end{table*}




% \begin{table*}[t]
% \centering
% \caption{In-domain evaluation on TR-bench in terms of NDCG@$m$ under scenarios including Invalid Format, Tool Hallucination, Tool Functionality Mismatch, Tool Preference Mismatch, Parameter Name Mismatch, Parameter Value Mismatch. \% improve represents the
% relative improvement achieved by our method over the previously best tool retrieval method.} 
% % \small
% \resizebox{0.5\linewidth}{!}{
% \begin{tabular}{@{}l|c|ccc|cc@{}}
% \toprule
% % \multirow{2}{*}{\textbf{Models}} & \multicolumn{2}{c|}{\textbf{{Preferred-only}}} & \multicolumn{2}{c|}{\textbf{{Rating-integrated}}} & \multicolumn{2}{c|}{\textbf{{Chronological}}} \\ 
% % \cmidrule(lr){2-7}
% \textbf{Models} & scenarios &IF  &TH  &TFM &TPM  &PNM  &PVM  \\ \midrule
% Qwen2.5-7B &$10.9/11.2$  &$3.6/2.3$  &$19.6/7.1$  &$25.5/54.6$  &$10.4/5.4$  &$14.9/13.2$  \\ 
% LLaMA3-8B &$2.5/2.5$  &$5.3/3.7$  &$19.3/2.8$  &$24.8/64.4$  &$11.4/6.8$  &$15.0/12.6$  \\
% GPT-4o-mini &$0.1/0.0$  &$3.5/1.5$  &$20.1/6.9$  &$24.4/60.2$  &$10.6/6.2$  &$16.4/13.3$ \\ 
% GPT-4o &$0.5/1.3$  &$1.5/1.1$  &$20.0/6.6$  &$30.3/57.3$  &$7.9/4.7$  &$14.0/12.2$ \\ 
% \midrule
% \textbf{Ours} &\bf 0.6/0.5  &\bf 3.4/1.8  &\bf 9.3/6.5  &\bf 12.0/10.4  &\bf 7.6/5.1  &\bf 3.9/3.1 \\ 
% \bottomrule
% \end{tabular}} 
% \label{error_results}
% \end{table*}

\paragraph{Analysis on the impact of interaction history.}
% Since interaction history plays a critical role
To investigate the impact of interaction history on LLM performance, we remove the interaction history from the inputs and provide only the user instructions with candidate tools set to conduct our experiments. The results are presented in Figure~\ref{fig:scores_wo_history}.
From the results, we can observe that both closed-source and open-source LLMs experience varying degrees of performance degradation without interaction history, compared to when provided with preferred-only history. This suggests that interaction history only containing the user's preferred tools can help the LLM effectively infer user preferences. On the other hand, we find that LLMs perform better in the absence of interaction history than with chronological history. This indicates that including both preferred and non-preferred tools can interfere with the LLM's understanding of user preferences, thus hindering its personalization capabilities.
In contrast, our \framework consistently improves performance across all three types of interaction history compared to the no-history setting. This demonstrates that our method enables LLM to effectively recognize different forms of user preferences from the interaction history.


\paragraph{Analysis on interaction history length.}
To evaluate the performance of LLMs under varying interaction history lengths, we break down the tool accuracy scores of LLMs based on the number of interactions in the history under the preferred-only setting.
As shown in Figure~\ref{fig:scores_length}, the performance of both closed-source and open-source LLMs deteriorates as interaction history length increases.
This is because a longer interaction history makes it more challenging for the LLM to identify the historical preferences relevant to identify relevant historical preferences in relation to the user’s current context.
In contrast, our \framework significantly outperforms all LLMs and maintains strong, consistent performance even as interaction history grows. This demonstrates that our method enables LLMs to effectively extract and utilize user preferences from complex historical data.


\subsection{Error Analysis}
We further conduct an error analysis to investigate the issues leading to incorrect tool calls in two personalized settings. 
We categorize the errors into six types:
1) Invalid Format. The tool call generated by the LLMs does not follow the expected JSON format. 
2) Tool Hallucination. The LLM generates a tool that does not exist in the given candidate tool set, which is a common hallucination issue in LLMs.
% ~\cite{huang2024survey}.
3) Tool Functionality Mismatch. The selected tool lacks the necessary functionality to fulfill the user’s requirements.
4) Tool Preference Mismatch. The selected tool has the correct functionality but is not preferred by the user.
5) Parameter Name Mismatch. The tool call contains missing or incorrect parameter names.
6) Parameter Value Mismatch. The parameter names are correctly generated, but the parameter values do not match the ground truth.

From the results in Table~\ref{error_results}, we observe that most LLMs perform worst in Tool Preference Mismatch, particularly in the chronological setting, where the error rate exceeds 50\%. This suggests that identifying user preferences from the interaction history is highly challenging, especially when preferences change over time, leading to significant model misinterpretation. In contrast, our \framework significantly reduces the error rate in Tool Preference Mismatch, demonstrating its effectiveness in capturing implicit user preferences. Additionally, the reduction in Tool Functionality Mismatch and Parameter Value Mismatch errors suggests that our method enhances LLMs' fundamental tool-usage ability, improving their handling of explicit user requirements.
Furthermore, \framework achieves low error rates in Invalid Format and Tool Hallucination, comparable to closed-source LLMs, highlighting its strong instruction-following capabilities.

\section{Conclusion and Future Work}
In this paper, we advanced general-purpose tool-use LLMs into personalized tool-use LLMs, aiming to provide users with customized tool-usage assistance. We formulate the task of personalized tool learning and identify the goal of leveraging user's interaction history to achieve implicit preference understanding and personalized tool calling. 
For training and evaluation, we construct the first \benchmark benchmark, featuring diverse users’ interaction history in three types. 
We also propose a novel personalized framework \framework conducted under a two-stage training process to endow LLMs with personalized tool-use capabilities.
Extensive experiments on \benchmark demonstrate that \framework consistently surpasses existing baselines, effectively meeting user requirements and preferences.
We believe that the task, benchmark, and framework for personalized tool learning will broaden the research scope, introduce new challenges and inspire novel methods. 

In the future, we aim to enhance this work from the following dimensions.
1) We plan to explore more heterogeneous personal user data beyond interaction history, such as user profiles or personas. This will allow us to reflect user preferences from multiple dimensions, providing a more comprehensive evaluation on the personalized tool-use capabilities of LLMs.
2) Currently, our work is limited to tool-usage scenarios involving a single tool. In the future, we intend to expand to more complex personalized tool-usage, such as multi-tool scenarios. These scenarios will require LLMs to perform personalized tool planning and engage in multi-round tool calling to address user needs effectively.

\section*{Limitations}
1) Due to the lack of real user interaction histories on tool usage, we utilize LLM to synthesize such data. However, this approach may compromise the authenticity and reliability of the data, which is a common challenge in data synthesis methods.
To mitigate this issue, we incorporate pre-constructed user preference information into the data generation process. This strategy helps guide LLM in generating contextually relevant outputs, thereby improving the quality and consistency of the synthesized data.
2) In real-world scenarios, tools have multiple dimensions of attributes. However, due to the limited information contained in tool documentation, it is difficult to identify and fully exploit all possible tool attributes. Fortunately, the attributes we have obtained are sufficient to differentiate between tools, enabling us to effectively construct user preferences.

\section*{Ethics Statement}
The dataset used in our work is derived from publicly available sources and generated through interactions with LLMs in English. Since the user interaction histories in our study are entirely simulated, user privacy is fully protected, and no real personal information is included in the dataset. Furthermore, all scientific artifacts used in this research are publicly accessible for academic purposes under permissive licenses, and their use in this paper complies with their intended purposes. Given these considerations, we believe our research adheres to the ethical standards of the conference.

% Taking a query $q$ and a user profile $U$ as input, the model is expected to select a user-specific tool set $D_u=\{d_i\}_{i=1}^k$ from 
% Formally, we design a number of user profiles $\{u_1, u_2, ..., u_m\}$, each of which represents a specific user.
% associate the query $q$ with $u_i$, where $u_i$ is a user-specific profile from a number of users.
% $\{u_1, u_2, ..., d_N\}$
% % ${u_i|i\in (1,M)}$ 
% where $M$ is the number of users.

% Given a user's instruction, tool learning aims to select a small number of tools, which could aid the LLM in answering the instruction, from a large-scale tool set and then .

% Given an input query $q$, tool learning model first selects from a large-scale tool set $D=\{d_1, d_2, ..., d_N\}$ a small number of tools $D(q)=\{d_1, d_2, ..., d_K\}$ which could help solve the query $q$, and then output a final response $r$ based on a tool-use trajectory $T=\{t(d_i)|d_i\in D(q)\}$ containing the tool execution results after calling each selected tools with parameters.


% In contrast, personalized tool learning aims to solve the query given by different users.
% It can be formulated as conditioning the tool learning process on $(q,u)$, where $u$ is a user-specific profile.
%  the model's output on a user

% Personalized tool learning can be formulated as conditioning the model's output on a user- $u$, represented by a user profile.



% In tool learning, a typical data entry consists of three components: an input query $q$, a tool set $D=\{d_1, d_2, ..., d_N\}$ containing $N$ ground-truth tools solving the query $q$,  
% % where $d_i$ represents the description of each tool and $N$ is the total number of tools, 
% and a tool-use trajectory $Traj=$.


% sequence $x$ that serves as the model's input, a target output $y$ that the model is expected to produce, and a profile $P_u$ that encapsulates any auxiliary information that can be used to personalize the model for the user.


% Given a user's query $q$, tool learning aims to select a small number of tools, which could aid the LLM in answering the instruction, from a large-scale tool set.
% Formally, we define the user instruction as $q$ and the tool set as $D=\{d_1, d_2, ..., d_N\}$, where $d_i$ represents the description of each tool and $N$ is the total number of tools.
% The retriever model $R$ needs to measure the relevance $R(q, d_i)$ 
% between the instruction $q$ and each tool description $d_i$, and return $K$ tools, denoted as $D=\{d_1, d_2, ..., d_K\}$.


% Generative language models often take an input $x$ and predict the most probable sequence tokens $y$ that follows $x$. 
% Tool learning can be formulated as xxx
% Personalized tool learning can be formulated as conditioning the model's output on a user $u$, represented by a user profile.



% \section{Introduction}
% These instructions are for authors submitting papers to *ACL conferences using \LaTeX. They are not self-contained. All authors must follow the general instructions for *ACL proceedings,\footnote{\url{http://acl-org.github.io/ACLPUB/formatting.html}} and this document contains additional instructions for the \LaTeX{} style files.

% The templates include the \LaTeX{} source of this document (\texttt{acl\_latex.tex}),
% the \LaTeX{} style file used to format it (\texttt{acl.sty}),
% an ACL bibliography style (\texttt{acl\_natbib.bst}),
% an example bibliography (\texttt{custom.bib}),
% and the bibliography for the ACL Anthology (\texttt{anthology.bib}).

% \section{Engines}

% To produce a PDF file, pdf\LaTeX{} is strongly recommended (over original \LaTeX{} plus dvips+ps2pdf or dvipdf). Xe\LaTeX{} also produces PDF files, and is especially suitable for text in non-Latin scripts.

% \section{Preamble}

% The first line of the file must be
% \begin{quote}
% \begin{verbatim}
% \documentclass[11pt]{article}
% \end{verbatim}
% \end{quote}

% To load the style file in the review version:
% \begin{quote}
% \begin{verbatim}
% \usepackage[review]{acl}
% \end{verbatim}
% \end{quote}
% For the final version, omit the \verb|review| option:
% \begin{quote}
% \begin{verbatim}
% \usepackage{acl}
% \end{verbatim}
% \end{quote}

% To use Times Roman, put the following in the preamble:
% \begin{quote}
% \begin{verbatim}
% \usepackage{times}
% \end{verbatim}
% \end{quote}
% (Alternatives like txfonts or newtx are also acceptable.)

% Please see the \LaTeX{} source of this document for comments on other packages that may be useful.

% Set the title and author using \verb|\title| and \verb|\author|. Within the author list, format multiple authors using \verb|\and| and \verb|\And| and \verb|\AND|; please see the \LaTeX{} source for examples.

% By default, the box containing the title and author names is set to the minimum of 5 cm. If you need more space, include the following in the preamble:
% \begin{quote}
% \begin{verbatim}
% \setlength\titlebox{<dim>}
% \end{verbatim}
% \end{quote}
% where \verb|<dim>| is replaced with a length. Do not set this length smaller than 5 cm.

% \section{Document Body}

% \subsection{Footnotes}

% Footnotes are inserted with the \verb|\footnote| command.\footnote{This is a footnote.}

% \subsection{Tables and figures}

% See Table~\ref{tab:accents} for an example of a table and its caption.
% \textbf{Do not override the default caption sizes.}

% \begin{table}
%   \centering
%   \begin{tabular}{lc}
%     \hline
%     \textbf{Command} & \textbf{Output} \\
%     \hline
%     \verb|{\"a}|     & {\"a}           \\
%     \verb|{\^e}|     & {\^e}           \\
%     \verb|{\`i}|     & {\`i}           \\
%     \verb|{\.I}|     & {\.I}           \\
%     \verb|{\o}|      & {\o}            \\
%     \verb|{\'u}|     & {\'u}           \\
%     \verb|{\aa}|     & {\aa}           \\\hline
%   \end{tabular}
%   \begin{tabular}{lc}
%     \hline
%     \textbf{Command} & \textbf{Output} \\
%     \hline
%     \verb|{\c c}|    & {\c c}          \\
%     \verb|{\u g}|    & {\u g}          \\
%     \verb|{\l}|      & {\l}            \\
%     \verb|{\~n}|     & {\~n}           \\
%     \verb|{\H o}|    & {\H o}          \\
%     \verb|{\v r}|    & {\v r}          \\
%     \verb|{\ss}|     & {\ss}           \\
%     \hline
%   \end{tabular}
%   \caption{Example commands for accented characters, to be used in, \emph{e.g.}, Bib\TeX{} entries.}
%   \label{tab:accents}
% \end{table}

% As much as possible, fonts in figures should conform
% to the document fonts. See Figure~\ref{fig:experiments} for an example of a figure and its caption.

% Using the \verb|graphicx| package graphics files can be included within figure
% environment at an appropriate point within the text.
% The \verb|graphicx| package supports various optional arguments to control the
% appearance of the figure.
% You must include it explicitly in the \LaTeX{} preamble (after the
% \verb|\documentclass| declaration and before \verb|\begin{document}|) using
% \verb|\usepackage{graphicx}|.

% \begin{figure}[t]
%   \includegraphics[width=\columnwidth]{example-image-golden}
%   \caption{A figure with a caption that runs for more than one line.
%     Example image is usually available through the \texttt{mwe} package
%     without even mentioning it in the preamble.}
%   \label{fig:experiments}
% \end{figure}

% \begin{figure*}[t]
%   \includegraphics[width=0.48\linewidth]{example-image-a} \hfill
%   \includegraphics[width=0.48\linewidth]{example-image-b}
%   \caption {A minimal working example to demonstrate how to place
%     two images side-by-side.}
% \end{figure*}

% \subsection{Hyperlinks}

% Users of older versions of \LaTeX{} may encounter the following error during compilation:
% \begin{quote}
% \verb|\pdfendlink| ended up in different nesting level than \verb|\pdfstartlink|.
% \end{quote}
% This happens when pdf\LaTeX{} is used and a citation splits across a page boundary. The best way to fix this is to upgrade \LaTeX{} to 2018-12-01 or later.

% \subsection{Citations}

% \begin{table*}
%   \centering
%   \begin{tabular}{lll}
%     \hline
%     \textbf{Output}           & \textbf{natbib command} & \textbf{ACL only command} \\
%     \hline
%     \citep{Gusfield:97}       & \verb|\citep|           &                           \\
%     \citealp{Gusfield:97}     & \verb|\citealp|         &                           \\
%     \citet{Gusfield:97}       & \verb|\citet|           &                           \\
%     \citeyearpar{Gusfield:97} & \verb|\citeyearpar|     &                           \\
%     \citeposs{Gusfield:97}    &                         & \verb|\citeposs|          \\
%     \hline
%   \end{tabular}
%   \caption{\label{citation-guide}
%     Citation commands supported by the style file.
%     The style is based on the natbib package and supports all natbib citation commands.
%     It also supports commands defined in previous ACL style files for compatibility.
%   }
% \end{table*}

% Table~\ref{citation-guide} shows the syntax supported by the style files.
% We encourage you to use the natbib styles.
% You can use the command \verb|\citet| (cite in text) to get ``author (year)'' citations, like this citation to a paper by \citet{Gusfield:97}.
% You can use the command \verb|\citep| (cite in parentheses) to get ``(author, year)'' citations \citep{Gusfield:97}.
% You can use the command \verb|\citealp| (alternative cite without parentheses) to get ``author, year'' citations, which is useful for using citations within parentheses (e.g. \citealp{Gusfield:97}).

% A possessive citation can be made with the command \verb|\citeposs|.
% This is not a standard natbib command, so it is generally not compatible
% with other style files.

% \subsection{References}

% \nocite{Ando2005,andrew2007scalable,rasooli-tetrault-2015}

% The \LaTeX{} and Bib\TeX{} style files provided roughly follow the American Psychological Association format.
% If your own bib file is named \texttt{custom.bib}, then placing the following before any appendices in your \LaTeX{} file will generate the references section for you:
% \begin{quote}
% \begin{verbatim}
% \bibliography{custom}
% \end{verbatim}
% \end{quote}

% You can obtain the complete ACL Anthology as a Bib\TeX{} file from \url{https://aclweb.org/anthology/anthology.bib.gz}.
% To include both the Anthology and your own .bib file, use the following instead of the above.
% \begin{quote}
% \begin{verbatim}
% \bibliography{anthology,custom}
% \end{verbatim}
% \end{quote}

% Please see Section~\ref{sec:bibtex} for information on preparing Bib\TeX{} files.

% \subsection{Equations}

% An example equation is shown below:
% \begin{equation}
%   \label{eq:example}
%   A = \pi r^2
% \end{equation}

% Labels for equation numbers, sections, subsections, figures and tables
% are all defined with the \verb|\label{label}| command and cross references
% to them are made with the \verb|\ref{label}| command.

% This an example cross-reference to Equation~\ref{eq:example}.

% \subsection{Appendices}

% Use \verb|\appendix| before any appendix section to switch the section numbering over to letters. See Appendix~\ref{sec:appendix} for an example.

% \section{Bib\TeX{} Files}
% \label{sec:bibtex}

% Unicode cannot be used in Bib\TeX{} entries, and some ways of typing special characters can disrupt Bib\TeX's alphabetization. The recommended way of typing special characters is shown in Table~\ref{tab:accents}.

% Please ensure that Bib\TeX{} records contain DOIs or URLs when possible, and for all the ACL materials that you reference.
% Use the \verb|doi| field for DOIs and the \verb|url| field for URLs.
% If a Bib\TeX{} entry has a URL or DOI field, the paper title in the references section will appear as a hyperlink to the paper, using the hyperref \LaTeX{} package.

% \section*{Acknowledgments}

% This document has been adapted
% by Steven Bethard, Ryan Cotterell and Rui Yan
% from the instructions for earlier ACL and NAACL proceedings, including those for
% ACL 2019 by Douwe Kiela and Ivan Vuli\'{c},
% NAACL 2019 by Stephanie Lukin and Alla Roskovskaya,
% ACL 2018 by Shay Cohen, Kevin Gimpel, and Wei Lu,
% NAACL 2018 by Margaret Mitchell and Stephanie Lukin,
% Bib\TeX{} suggestions for (NA)ACL 2017/2018 from Jason Eisner,
% ACL 2017 by Dan Gildea and Min-Yen Kan,
% NAACL 2017 by Margaret Mitchell,
% ACL 2012 by Maggie Li and Michael White,
% ACL 2010 by Jing-Shin Chang and Philipp Koehn,
% ACL 2008 by Johanna D. Moore, Simone Teufel, James Allan, and Sadaoki Furui,
% ACL 2005 by Hwee Tou Ng and Kemal Oflazer,
% ACL 2002 by Eugene Charniak and Dekang Lin,
% and earlier ACL and EACL formats written by several people, including
% John Chen, Henry S. Thompson and Donald Walker.
% Additional elements were taken from the formatting instructions of the \emph{International Joint Conference on Artificial Intelligence} and the \emph{Conference on Computer Vision and Pattern Recognition}.

% Bibliography entries for the entire Anthology, followed by custom entries
% \bibliography{anthology,custom}
% Custom bibliography entries only
\bibliography{acl_latex}

\appendix

\section{Details of Benchmark Construction}
We provide the illustration of three types of the user's interaction history in Figure~\ref{fig:history}.
\begin{figure}[htbp]
    \centering
    \includegraphics[width=1.0\columnwidth]{fig_history.pdf}
    \caption{Illustration of three types of the user's interaction history.}
    \label{fig:history}
\vspace{-1em}
\end{figure}

\section{Implementation details}
To train \framework, we fine-tune the LLaMA-3.1-8B model with LoRA
% () 
and a warm-up ratio of $0.1$ in the SFT stage. 
The learning rate is set to $1e{-4}$ with a batch size of $16$ per GPU. 
In the DPO stage, the learning rate is set to $1e{-6}$ and the balancing factor $\beta$ is set to $0.1$ with a batch size of $32$.
We have trained the model several times to ensure the improvement is not randomly achieved and present the mid one. 
For evaluation, we set the number of candidate tools $N$ to $10$ and the temperature to 0.1 to reduce randomness. 
Since the maximum context length varies in different LLMs, we constrain the context window to 4000 tokens. The experiments on closed-source LLMs are fulfilled by APIs of OpenAI and those on open-source LLMs are conducted on NVIDIA A6000 GPUs with 48 GB of memory. 

\definecolor{lightgray}{RGB}{240, 240, 240}
% \definecolor{lightgray}{gray}{0.95}
\lstdefinestyle{prompt}{
    basicstyle=\ttfamily\fontsize{7pt}{8pt}\selectfont,
    frame=none,
    breaklines=true,
    backgroundcolor=\color{lightgray},
    breakatwhitespace=true,
    breakindent=0pt,
    escapeinside={(*@}{@*)},
    numbers=none,
    numbersep=5pt,
    xleftmargin=5pt,
}
\tcbset{
  aibox/.style={
    % width=220pt,
    % top=10pt,
    % colback=lightgray,
    % colframe=black,
    % colbacktitle=black,
    % enhanced,
    % center,
    % breakable,
    % attach boxed title to top left={yshift=-0.1in,xshift=0.15in},
    % boxed title style={boxrule=0pt,colframe=white,},
  }
}
\newtcolorbox{AIbox}[2][]{aibox, title=#2,#1}


\section{Prompt Details}
The prompt templates in for tool-use example generation and tool attributes understanding are shown in Figure~\ref{fig:prompt_example} and Figure~\ref{fig:prompt_attributes}. The prompt templates for interaction history generation across three types are shown in Figure~\ref{fig:prompt_history(p)}, Figure~\ref{fig:prompt_history(r)}, and Figure~\ref{fig:prompt_history(c)}.
The prompt template for instruction generation is shown in Figure~\ref{fig:prompt_instruction}.

\begin{figure*}[!ht] 
\vspace{-5mm}
\begin{AIbox}{Prompt for Tool-use Example Generation}
{\bf Prompt:} \\
{
Given a tool documentation as input, your task is to output an example for using this tool, including a simulated user instruction and parameters for calling the tool. The output example should be in JSON format: \{``instruction'': xx, ``parameters'': xx\}
\clearpage
Here is a demonstration:

Input:
\begin{lstlisting}[style=prompt]
{
    "tool_name": "<Text_Analysis>.<Spellout>.<Languages>",
    "tool_desciption": "List ISO 639 languages",
    "required_parameters": [],
    "optional_parameters": [
        {
            "name": "nameFilter",
            "type": "STRING",
            "description": "Filter as \"contains\" by language name",
            "default": ""
        }
    ]
}
\end{lstlisting}
Output:
\begin{lstlisting}[style=prompt]
{
    "instruction": "I want to filter the list of languages by English",
    "parameters": {
        "nameFilter": "English"
    }
}
\end{lstlisting}
Now you will be given the tool documentation, please generate the tool-use example. 

Begin!
}
\end{AIbox} 
\caption{The prompt for tool-use example generation.}
\label{fig:prompt_example}
\vspace{-5mm}
\end{figure*}



\begin{figure*}[!ht] 
\vspace{-5mm}
\begin{AIbox}{Prompt for Tool Attributes Understanding}
{\bf Prompt:} \\
{
Given a tool documentation and the corresponding tool-use example as input, your task is to understand the tool attributes thoroughly. Then generate two descriptions about the functionality and non-functional attributes of the tool respectively. 
% The functionality refers to the core function that the tool performs in order to fulfill its purpose. 
% Non-functional attributes refer to additional attributes beyond functionality that can reflect different characteristics of the tool and result in different user experience, such as usability, integrability, accessibility, and security.
\clearpage
Here is a demonstration:

Input:
\begin{lstlisting}[style=prompt]
Tool documentation:
{
    "tool_name": "<Commerce>.<Face Compare>.<GET Call>",
    "tool_desciption": "Used to fetch results using the request id received in responses.",
    "required_parameters": [
        {
            "name": "request_id",
            "type": "STRING",
            "description": "",
            "default": "76d1c748-51ed-435b-bcd8-3d9c9d3eb68a"
        }
    ],
Tool-use example:
{
    "instruction": "I want to use the request id '76d1c748-51ed-435b-bcd8-3d9c9d3eb68a' to fetch the result",
    "parameters": {
        "request_id": "76d1c748-51ed-435b-bcd8-3d9c9d3eb68a"
    }
}
\end{lstlisting}
Output:
\begin{lstlisting}[style=prompt]
Functionality: Fetches API results based on the request ID received in previous responses.
Non-functional attributes: Designed for commerce applications, used in face comparison scenarios.
\end{lstlisting}
Now you will be given the tool documentation and the tool-use example, generate two short phrases to describe the two types of attributes. 

Begin!
}
\end{AIbox} 
\caption{The prompt for tool attributes understanding.}
\label{fig:prompt_attributes}
\vspace{-5mm}
\end{figure*}


\begin{figure*}[!ht] 
\vspace{-5mm}
\begin{AIbox}{Prompt for Interaction History (Preferred-only) Generation}
{\bf Prompt:} \\
{
Given a list of tools preferred by a user as input, your task is to simulate the user's interaction history based on these tools. You should output a sequence of tool-usage interactions, each consisting of a simulated user instruction and a tool call to fulfill that instruction. The interaction sequence should be a list in JSON format: 
% \clearpage
% Here is a demonstration:
% Input:
\begin{lstlisting}[style=prompt]
[
    {
        "instruction": xx,
        "tool_call": {
            "tool_name": xx, 
            "parameters": xx
        }
    }, ...
]
\end{lstlisting}
Now you will be given the tools, please generate the interaction sequence. 

Begin!
}
\end{AIbox} 
\caption{The prompt for interaction history (preferred-only) generation.}
\label{fig:prompt_history(p)}
\vspace{-5mm}
\end{figure*}


\begin{figure*}[!ht] 
\vspace{-5mm}
\begin{AIbox}{Prompt for Interaction History (Rating-integrated) Generation}
{\bf Prompt:} \\
{
Given a list of tools preferred by a user and a list of tools not preferred as input, your task is to simulate the user's interaction history based on these two lists. You should output a sequence of tool-usage interactions, each consisting of a simulated user instruction, a tool call to fulfill that instruction, and a binary rating reflecting the user's satisfaction with the tool call. The interaction sequence should be a list in JSON format: 
\begin{lstlisting}[style=prompt]
[
    {
        "instruction": xx,
        "tool_call": {
            "tool_name": xx, 
            "parameters": xx
        },
        "rating": 1 or 0,
    }, ...
]
\end{lstlisting}
Now you will be given the two lists of tools, please generate the interaction sequence. 

Begin!
}
\end{AIbox} 
\caption{The prompt for interaction history (rating-integrated) generation.}
\label{fig:prompt_history(r)}
\vspace{-5mm}
\end{figure*}


\begin{figure*}[!ht] 
\vspace{-5mm}
\begin{AIbox}{Prompt for Interaction History (Chronological) Generation}
{\bf Prompt:} \\
{
Given a list of tools preferred by a user and a list of tools not preferred as input, your task is to simulate the user's interaction history based on these two lists. You should output a sequence of tool-usage interactions, each consisting of a simulated user instruction, a tool call to fulfill that instruction. The interactions should be organized in time order to reflect changes in user preferences over time, i.e., the more recent tool-usage interactions are more preferred by the user, while earlier interactions are less preferred. The interaction sequence should be a list in JSON format: 
\begin{lstlisting}[style=prompt]
[
    {
        "instruction": xx,
        "tool_call": {
            "tool_name": xx, 
            "parameters": xx
        }
    }, ...
]
\end{lstlisting}
Now you will be given the two lists of tools, please generate the interaction sequence. 

Begin!
}
\end{AIbox} 
\caption{The prompt for interaction history (chronological) generation.}
\label{fig:prompt_history(c)}
\vspace{-5mm}
\end{figure*}


\begin{figure*}[!ht] 
% \vspace{-2mm}
\begin{AIbox}{Prompt for Instruction Generation}
{\bf Prompt:} \\
{
Given a user's interaction history and a tool documentation as input, your task is to generate a simulated user instruction which can be fulfilled by calling the tool with parameters. The generated output should be in JSON format: 
\begin{lstlisting}[style=prompt]
{
    "instruction": xx,
    "parameters": xx
}

\end{lstlisting}
Remember, tool name is strictly prohibited from appearing in the generated instruction. 
Now you will be given the user's interaction history and tool documentation, please generate the output. 

Begin!
}
\end{AIbox} 
\caption{The prompt for instruction generation.}
\label{fig:prompt_instruction}
% \vspace{-5mm}
\end{figure*}


\label{sec:appendix}

% This is an appendix.

\end{document}
