% This must be in the first 5 lines to tell arXiv to use pdfLaTeX, which is strongly recommended.
\pdfoutput=1
% In particular, the hyperref package requires pdfLaTeX in order to break URLs across lines.

\documentclass[11pt]{article}

% Change "review" to "final" to generate the final (sometimes called camera-ready) version.
% Change to "preprint" to generate a non-anonymous version with page numbers.
% \usepackage[review]{acl}
\usepackage[preprint]{acl}

% Standard package includes
\usepackage{times}
\usepackage{latexsym}

% For proper rendering and hyphenation of words containing Latin characters (including in bib files)
\usepackage[T1]{fontenc}
% For Vietnamese characters
% \usepackage[T5]{fontenc}
% See https://www.latex-project.org/help/documentation/encguide.pdf for other character sets

% This assumes your files are encoded as UTF8
\usepackage[utf8]{inputenc}

% This is not strictly necessary, and may be commented out,
% but it will improve the layout of the manuscript,
% and will typically save some space.
\usepackage{microtype}

% This is also not strictly necessary, and may be commented out.
% However, it will improve the aesthetics of text in
% the typewriter font.
\usepackage{inconsolata}

%Including images in your LaTeX document requires adding
%additional package(s)
\usepackage{graphicx}
\usepackage{booktabs}
\usepackage{multirow}
\usepackage{makecell}
\usepackage{amsmath}
\usepackage{amssymb}
\usepackage[most]{tcolorbox}
% \usepackage{xcolor}


% If the title and author information does not fit in the area allocated, uncomment the following
%
%\setlength\titlebox{<dim>}
%
% and set <dim> to something 5cm or larger.

\usepackage{xspace}
\newcommand{\benchmark}{PEToolBench\xspace} % 加空格
\newcommand{\framework}{PEToolLLaMA\xspace}
% \newcommand{\benchmark}{LaMP}


\title{PEToolLLM: Towards Personalized Tool Learning in \\Large Language Models}

% Author information can be set in various styles:
% For several authors from the same institution:
% \author{Author 1 \and ... \and Author n \\
%         Address line \\ ... \\ Address line}
% if the names do not fit well on one line use
%         Author 1 \\ {\bf Author 2} \\ ... \\ {\bf Author n} \\
% For authors from different institutions:
% \author{Author 1 \\ Address line \\  ... \\ Address line
%         \And  ... \And
%         Author n \\ Address line \\ ... \\ Address line}
% To start a separate ``row'' of authors use \AND, as in
% \author{Author 1 \\ Address line \\  ... \\ Address line
%         \AND
%         Author 2 \\ Address line \\ ... \\ Address line \And
%         Author 3 \\ Address line \\ ... \\ Address line}

\author{Qiancheng Xu$^{1}$, Yongqi Li$^{1\dagger}$, Heming Xia$^{1}$, Fan Liu$^{2}$, Min Yang$^{3}$, Wenjie Li$^{1}$ \\
% \thanks{Corresponding author.}
$^{1}$ The Hong Kong Polytechnic University \quad
$^{2}$ National University of Singapore \\
$^{3}$ Shenzhen Institutes of Advanced Technology, Chinese Academy of Sciences \\
\texttt{\{qiancheng.xu, he-ming.xia\}@connect.polyu.hk} \\
\texttt{liyongqi0@gmail.com} \quad
\texttt{cswjli@comp.polyu.edu.hk}
}

% \author{First Author \\
%   Affiliation / Address line 1 \\
%   Affiliation / Address line 2 \\
%   Affiliation / Address line 3 \\
%   \texttt{email@domain} \\\And
%   Second Author \\
%   Affiliation / Address line 1 \\
%   Affiliation / Address line 2 \\
%   Affiliation / Address line 3 \\
%   \texttt{email@domain} \\}

%\author{
%  \textbf{First Author\textsuperscript{1}},
%  \textbf{Second Author\textsuperscript{1,2}},
%  \textbf{Third T. Author\textsuperscript{1}},
%  \textbf{Fourth Author\textsuperscript{1}},
%\\
%  \textbf{Fifth Author\textsuperscript{1,2}},
%  \textbf{Sixth Author\textsuperscript{1}},
%  \textbf{Seventh Author\textsuperscript{1}},
%  \textbf{Eighth Author \textsuperscript{1,2,3,4}},
%\\
%  \textbf{Ninth Author\textsuperscript{1}},
%  \textbf{Tenth Author\textsuperscript{1}},
%  \textbf{Eleventh E. Author\textsuperscript{1,2,3,4,5}},
%  \textbf{Twelfth Author\textsuperscript{1}},
%\\
%  \textbf{Thirteenth Author\textsuperscript{3}},
%  \textbf{Fourteenth F. Author\textsuperscript{2,4}},
%  \textbf{Fifteenth Author\textsuperscript{1}},
%  \textbf{Sixteenth Author\textsuperscript{1}},
%\\
%  \textbf{Seventeenth S. Author\textsuperscript{4,5}},
%  \textbf{Eighteenth Author\textsuperscript{3,4}},
%  \textbf{Nineteenth N. Author\textsuperscript{2,5}},
%  \textbf{Twentieth Author\textsuperscript{1}}
%\\
%\\
%  \textsuperscript{1}Affiliation 1,
%  \textsuperscript{2}Affiliation 2,
%  \textsuperscript{3}Affiliation 3,
%  \textsuperscript{4}Affiliation 4,
%  \textsuperscript{5}Affiliation 5
%\\
%  \small{
%    \textbf{Correspondence:} \href{mailto:email@domain}{email@domain}
%  }
%}

\begin{document}
\maketitle
\begingroup\def\thefootnote{$\dagger$}\footnotetext{Corresponding author.}\endgroup

\begin{abstract}
Tool learning has emerged as a promising direction by extending Large Language Models' (LLMs) capabilities with external tools. Existing tool learning studies primarily focus on the general-purpose tool-use capability, which addresses explicit user requirements in instructions. 
However, they overlook the importance of personalized tool-use capability, leading to an inability to handle implicit user preferences.
% meet personalized user needs.
To address the limitation, we first formulate the task of personalized tool learning, which integrates user's interaction history towards personalized tool usage. 
To fill the gap of missing benchmarks, we construct \benchmark, featuring diverse user preferences reflected in interaction history under three distinct personalized settings, and encompassing a wide range of tool-use scenarios.
Moreover, we propose a framework \framework to adapt LLMs to the personalized tool learning task, which is trained through supervised fine-tuning and direct preference optimization.
Extensive experiments on \benchmark demonstrate the superiority of \framework over existing LLMs. 
We release our code and data at
% for review at \href{https://anonymous.4open.science/r/PEToolBench-952F/}{https://anonymous.4open.science/
% r/PEToolBench-952F/}.
\href{https://github.com/travis-xu/PEToolBench}{https://github.com/travis-xu/PEToolBench}.

\end{abstract}

\section{Introduction}
% Large Language Models (LLMs) have increasingly been regarded as a potential pathway toward developing general AI assistants for human users. 
Large Language Models (LLMs) possess extensive knowledge and have powerful instruction-following abilities, making them effective AI assistants for tasks such as text rewriting, question answering, and code writing~\cite{zhao2023survey}.
However, they often struggle in addressing user needs in scenarios such as checking weather and booking flights. 
To address this, tool learning~\cite{10.1145/3704435,qu2024tool} has emerged as a promising solution by enabling LLMs to utilize external tools, such as real-time weather APIs and booking systems.
% To bridge these gaps, tool learning has emerged as a promising approach, equipping LLMs with the ability to leverage external tools like real-time weather APIs and booking systems. 
% Tool learning not only overcomes current limitations of LLMs, but also extends their 
In this way, tool learning has extended LLMs' capabilities to tackle more complex tasks, enabling them to fulfill a wide range of user needs.

Current tool learning procedure typically begins with a user instruction, and then LLMs are required to use tools with appropriate functionalities for satisfying users' needs.
% use tools with appropriate functionalities to fulfill user’s requirements in the instruction. 
Existing tool learning methods can be categorized into in-context learning~\cite{wu-etal-2024-toolplanner,liu2025toolplanner} and fine-tuning approaches~\cite{NEURIPS2023_d842425e,wang2025toolgen}. The former approach allows LLMs to use tools by directly providing tool documentation in input but the performance is constrained by the input length.
% or demonstrations  % the limited input length 
The latter approach trains LLMs to internalize tool knowledge but struggles with tool generalization.
% specially
% Existing tool learning studies primarily focus on the \textit{general-purpose} tool-use capability, where
% % , i.e., utilizing various tools to address a wide range of user needs.
% % Specifically, given user instructions and tool documentation as inputs, 
% LLMs are required to understand two main aspects: 1) \textbf{explicit user requirements}, expressed in user instructions, and 2) \textbf{functionalities of tools}, derived from tool documentation. By understanding both aspects, LLMs can accurately utilize tools with appropriate functionalities to address user needs.


\begin{figure*}[!t]
    \centering
    % \centerline{\includegraphics[width=2.0\columnwidth]{MainFigure.pdf}}
    \includegraphics[width=1.0\textwidth]{fig_intro.pdf}
    \caption{Comparison between (a) tool learning and (b) personalized tool learning. Personalized tool learning facilitates implicit preference comprehension and customized tool usage for individual users.}
    \label{fig:intro}
\vspace{-1em}
\end{figure*}


Despite the advancement, existing tool learning methods primarily focus on the general-purpose tool-use capability but overlook the critical role of personalization. 
In tool learning, more personalized user needs are expected to be derived from the user's previous 
% interaction 
tool usage 
history as a supplement to user instructions, which can help LLMs provide more customized tool-usage assistance to enhance the user experience.
% This helps the LLM provide more customized tool assistance, thereby improving the user experience.
% requires LLMs to offer more customized tool-usage assistance for specific users. 
% uncover
As illustrated in Figure~\ref{fig:intro}, personalized tool learning is non-trivial due to the following aspects. 
% As tool-use LLMs are designed to serve human users, and considering that each user has unique needs, LLMs must use specific tools to meet user-specific needs, referred to as the \textit{personalized} tool-use capability.
% As user serving AI assistants, considering that each user has unique needs, LLMs must be capable of meeting these distinct requirements.
% Since tool learning aims at enabling LLMs to better serve human users, each of whom has unique needs.
% Unfortunately, they exhibit significant limitations due to the neglect of two critical aspects:
% catering for unique user needs,
% role of \textit{personalized} tool-use capabilities in LLMs, including two additional aspects
% 1) \textbf{implicit user preferences}, which are not explicitly stated in user instructions but can be inferred from personal user data (e.g., interaction history). 
1) \textbf{Implicit user preferences}. 
User preferences for tool usage are often implicitly conveyed through the user's history rather than explicitly stated in user instructions, making them difficult to understand.
% as they are implicitly concealed within the user's history instead of explicitly stated in user instructions.
For instance, when a user requests a search for articles, their preference for academic-related content needs to be inferred from previous interactions with academic tools like Google Scholar.
% For instance, when requesting a search for articles, a user may prefer academic-related results if their interaction history includes academic tools like Google Scholar. 
% may indicate a preference for academic-related search results.
% and are revealed in the user's interaction history.
% comprehension, which are not explicitly stated in user instructions but can be inferred from user's interaction history.
% For instance, when requesting a search for articles, a user may prefer academic-related results if their interaction history includes academic tools like Google Scholar. 
% For example, when requesting a search, a user's interaction history involving academic tools (e.g., Google Scholar) is likely to prefer academic-related search results;
% academically supportive
2) \textbf{Non-functional tool attributes}.
% In real-world scenarios, many tools have the same functionalities, 
% reflect the differences in 
Since many tools have the same functionalities, user preferences cannot be effectively distinguished based solely on tool functionalities. 
% which is challenging for LLMs to differentiate and align with different user preferences.
% To address this, it is crucial 
This underscores the need to consider non-functional tool attributes, such as usability, integrability, and accessibility, which can better reflect user preferences. 
% Due to the existence of multiple same-functionality tools, LLMs cannot determine the user's preferred tools based solely on their functionality. 
% In this paper, we refer to these as the \textit{non-functional attributes} of tools.
% to address specific personalized needs.
% discern which tool a user prefers based solely on its functionality.
% allowing LLMs to use the user's preferred tool from multiple tools with the same functionality.
% customization most
% This is possible because tools often have unique \textit{non-functional attributes} (e.g., usability, integrability, accessibility, etc.) that are preferred by different users.
% distinguish between same-functionality tools and use the most suitable one for individual user.
% beyond functionalities, which can significantly impact user preferences.
% \textbf{non-functional attributes of tools}, referring to attributes beyond functionalities (e.g., usability, integrability, etc.) that significantly influence user preferences.
% can differentiate same-functionality tools to align with different user preferences. 
As shown in Figure~\ref{fig:intro}, Google Search can be distinguished from other search tools by its integration into Google’s ecosystem with Google Scholar, making it more suitable for users with academic needs.
% These non-functional attributes are crucial for LLMs to call the most preferred tool for a specific user, especially when there exist multiple same-functionality tools. 
% Therefore, these aspects are essential for LLMs to form a holistic understanding of tools and user needs, facilitating customized tool-usage assistance.
% form a more comprehensive and deeper understanding of tools and user needs, advancing them to be a more personalized and user-centric tool-use assistant.

% To this end
To address the above issues, we formulate the task of personalized tool learning in LLMs, aiming at personalized tool usage for individual users.
Formally, given user instructions along with user's 
% user data (e.g. interaction history)
interaction history, LLMs are required to answer user instructions with tools by considering both explicit user requirements in instructions and implicit user preferences behind interaction history.
% must satisfy user needs by using tools which not only have appropriate functionalities to meet explicit user requirements, 
% % as well as with 
% but also possess suitable non-functional attributes to align with implicit user preferences.
% understand both the user's explicit requirements and implicit preferences, as well as functionalities and non-functional attributes of tools, then select and call the appropriate tool with corresponding parameters.
% , then select and call the appropriate tool with corresponding parameters to meet the user's needs.

% which takes personal user data and tool characteristics into consideration
% aiming for personalized user needs comprehension and tool utilization.
% , aiming to align with explicit user instructions and implicit user preferences derived from personalized data.
% by taking personalized user data and tool characteristics into consideration, aiming to facilitate more tailored tool usage that satisfies both the users' functional requirements and their individual preferences. 


Since there is no benchmark for this task currently, we fill this gap by introducing the first personalized tool learning benchmark (\benchmark). 
Specifically, the benchmark is created through three following steps.
1) Tool Preparation.
We collect a bunch of high-quality tools from RapidAPI and then
% as our seed tools. 
% And then we 
leverage LLM to understand the functionality and non-functional attributes of each tool. 
2) Preference Construction. 
% We leverage the obtained tool attributes to construct the user preferences on tools.
% We classify tools with the same functionalities into groups. 
Among same-functionality tools, we construct the user's tool preferences by assigning tools with distinct non-functional attributes to different users.
% construct tool preferences for users by assigning the user's preferred tool and non-preferred tools based on non-functional attributes.
% we identify the preferred and non-preferred tools of a specific user within the functionality group.
% by randomly assign a tool with distinct non-functional attributes as the user's preferred tool and other tools as non-preferred tools.
% for a specific user. 
% To enrich the user preference, we use the preferred tool to retrieve more tools with similar non-functional attributes.
% The preferred tool will also be used to retrieve more tools with similar non-functional attributes to be the user's preferred tool .
% In this way, we can obtain a bunch of preferred and non-preferred tools representing different user preferences.
% After multiple functionality groups, we can obtain a number of preferred and non-preferred tools representing .
% all serving as preferred tools for the specific user.  
3) Data Creation. 
Based on tool preference, we synthesize the user's interaction history into a sequence of tool-use interactions, each consisting of a user instruction and an LLM's tool call.
% The user's interaction history is generated into a sequence of tool-use interactions, each containing a user instruction and an LLM's tool call. 
% both preferred and non-preferred tools with binary ratings reflecting the user's preference; 
% Finally, we select tools 
We design three personalized tool-usage settings by generating the interaction history in three types, i.e., preferred-only, rating-integrated, and chronological. 
% representing different forms of user preferences.
% in  ways, we
And then we use tools not included in the interaction history to synthesize user instructions.
% as ground truth tools to generate user instructions for each data instance. 
% Each instruction is combined with the interaction history into a data instance. construct data instance 
% We generate interaction history into three types, i.e., preferred-only, rating-integrated, chronological, to represent different forms of user preferences.
% 1) \textit{preferred-only} history, which only includes the user's preferred tools ; 2) \textit{rating-integrated} history, integrating user's binary ratings on each tool call according to user preferences; and 3) \textit{chronological} history, where interactions are arranged in time order to capture changes in user preferences over time. 
After rigorous filtering, we obtain 12,000 user instructions with interaction histories reflecting diverse user preferences and cover a wide range of tool-use scenarios by encompassing 7454 tools across 46 categories.
 % of 1,703 users 

% .Overall, our benchmark sets itself apart through several distinctive features:
% structured through three primary phases:
% \begin{enumerate}
%     \item User's Preferred Tools Selection: 
%     We adopt real-world tools from ToolBench dataset, and identify their user-centric attributes beyond functionalities. Then we classify and select tools with similar user-centric attributes as the user's preferred tools. 
%     \item Interaction History Construction: 
%     We sample a bunch of user's preferred tools to synthesize a sequence of tool-use cases as the user's interaction history, constructed in three forms (i.e., preferred-only, rating-integrated and chronological), representing different manifestations of user preferences.
%     \item User Instructions Creation:
%     We utilize users' preferred tools not included in the interaction history as ground truth tools. Then we use them to synthesize user instructions and corresponding tool calling parameters.
% \end{enumerate}

% To equip LLMs with personalized tool-use capability, we propose \framework. Specifically, we train LLM on \benchmark through two stages: 
Based on the \benchmark dataset, we propose the personalized tool learning framework (\framework) to equip LLMs with personalized tool-use capability. The training process consists of two stages:
1) the supervised fine-tuning (SFT) stage, which equips LLM with foundational tool-use capability to address user needs; 2) the direct preference optimization (DPO) stage, which samples the user's preferred and non-preferred tool calls for pair-wise optimization to better align with user preferences.
 % (Dubey et al., 2024)
We evaluate 6 distinct open-source and closed-source LLMs including the latest GPT-4o on \benchmark. 
Experimental results demonstrate that our \framework significantly outperforms the best-performing LLM across all settings with improvements even more than 50\%, showcasing its superior personalized tool-use capabilities.
% to provide user-centric and customized assistance.
% Experimental results demonstrate that \framework consistently and significantly outperforms existing baselines, showcasing the potential of personalized tool-use LLMs to provide more user-centric and customized tool-usage assistance.


In summary, our contributions are as follows.
\begin{itemize}
\item 
We are the first to formulate the task of personalized tool learning in LLMs, 
which incorporates user's interaction history to achieve personalized tool-usage assistance.
% for individual users. 
% tool selection and calling, bridging users with customized tool-usage assistance.

\item 
We construct the first benchmark for personalized tool learning in LLMs, \benchmark,
featuring user instructions integrated with interaction history reflecting diverse user preferences and encompassing various tools.
% featuring diverse user instructions with interaction history across three types.
\item 
% We propose the first personalized tool-use LLM, \framework.
We propose a novel personalized tool learning framework \framework. 
% Extensive experiments show that \framework consistently exceeds existing baselines, effectively fulfilling both user requirements and preferences.
Extensive experiments demonstrate that \framework significantly surpass the best-performing LLM by more than 50\%, exibiting exceptional personalized tool-use capabilities.
\end{itemize}

\section{Related Work}
\subsection{Tool Learning in LLMs}
Tool learning aims at extending the capabilities of LLMs by equipping them with external tools to solve tasks like weather inquiry, car navigation, and restaurant reservation. Existing benchmarks primarily focus on evaluating the tool learning proficiency of LLMs in addressing user instructions, from aspects such as 
% tool usage awareness~\cite{huang2024metatool}, 
tool selection and calling accuracy~\cite{xu-etal-2024-enhancing-tool,NEURIPS2024_8a75ee6d,ye-etal-2024-rotbench,wang2025mtubench}, tool planning ability~\cite{basu-etal-2024-api,wang-etal-2024-appbench,NEURIPS2024_085185ea,liu2025toolace}, and complex workflow creation~\cite{shen2025shortcutsbench,qiao2025benchmarking,fan2025workflowllm}. To improve tool-use capabilities, various strategies have been introduced, including in-context learning which enables LLMs to use tools via documentation~\cite{yuan2024easytool,shi-etal-2024-learning,qu2025from},
% (Hsieh et al., 2023), 
and fine-tuning which trains LLMs on specialized tool-use datasets~\cite{zhuang2024toolchain,chen2024advancing,chen2025learning}. However, prior studies neglect the crucial role of personalized tool usage in LLMs. This paper addresses this gap by introducing personalized tool learning, developing a comprehensive benchmark for evaluation, and proposing an optimization strategy to enhance personalized tool-use capabilities in LLMs.

\subsection{Personalization in LLMs}
The goal of personalization in LLMs is to leverage personal user data, such as historical behaviors and background information, to generate outputs that better align with the user preferences~\cite{tseng-etal-2024-two}.
% (Chen et al., 2023e; Deshpande et al., 2024). 
Approaches such as fine-tuning~\cite{cai2025large} and prompt engineering~\cite{yuan-etal-2025-personalized}
% , and user-specific embeddings 
have been explored to adapt LLMs to individual or domain-specific tasks. These approaches have been applied across various fields, including recommendation systems~\cite{lyu-etal-2024-llm}, search engines~\cite{10.1145/3589334.3645482}, education~\cite{liu2024socraticlm}, 
% healthcare, 
and dialogue generation~\cite{wang-etal-2023-target}. However, previous research has not investigated LLMs' personalization in the area of tool learning. In this work, we bridge this gap by incorporating user's interaction history to assess and enhance the LLMs' capability in providing personalized tool-usage assistance for specific users.
% generating tool calls tailored to user-specific needs.



\begin{figure*}[!t]
    \centering
    \includegraphics[width=1.0\textwidth]{fig_benchmark.pdf}
    % \caption{Illustration of our \benchmark.}
    \caption{Illustration of the process for constructing our \benchmark.}
    \label{fig:benchmark}
\vspace{-1em}
\end{figure*}

% Due to the lack of real personal data on tool-usage, we adopt a tool-driven approach to simulate user data by leveraging tool attributes to construct user preferences and interaction history. 
% real-world tools to simulate user data, distinguishing user preferences by tool attributes. 
% (1) Tool Preparation.
% We collect a bunch of high-quality tools from RapidAPI.
% % as our seed tools. 
% Then we leverage LLM to understand the functionality and non-functional attributes of each tool. 
% (2) Preference Construction. 
% % We leverage the obtained tool attributes to construct the user preferences on tools.
% We classify tools with the same functionalities into groups. 
% Within each functionality group, we construct the user preferences by
% % we identify the preferred and non-preferred tools of a specific user within the functionality group.
% randomly assign a tool with distinct non-functional attributes as the user's preferred tool and other tools as non-preferred tools.
% % for a specific user. 
% To enrich the user preference, we use the preferred tool to retrieve more tools with similar non-functional attributes.
% % The preferred tool will also be used to retrieve more tools with similar non-functional attributes to be the user's preferred tool .
% In this way, we can obtain a bunch of preferred and non-preferred tools representing different user preferences.
% % After multiple functionality groups, we can obtain a number of preferred and non-preferred tools representing .
% % all serving as preferred tools for the specific user.  
% (3) Data Creation. 
% Based on the user preference, we generate the user's interaction history as a sequence of tool-use interactions, each containing a user instruction and a corresponding tool call by LLM. The interaction history is constructed in three types: 1) \textit{preferred-only} history, which only includes the user's preferred tools ; 2) \textit{rating-integrated} history, integrating user's binary ratings on each tool call according to user preferences; and 3) \textit{chronological} history, where interactions are arranged in time order to capture changes in user preferences over time. 
% % both preferred and non-preferred tools with binary ratings reflecting the user's preference; 
% % Finally, we select tools 
% Finally, we generate user instructions based on the user's preferred tools not included in the interaction history.
% % as ground truth tools to generate user instructions for each data instance. 
% % Each instruction is combined with the interaction history into a data instance. construct data instance 


\section{Task and Benchmark}
\subsection{Task Formulation}
\paragraph{Tool Learning}
Given an instruction $q_u$ of the user $u$, tool learning aims to generate an appropriate tool call, including the selected tool and its corresponding parameters, from a set of candidate tools. Formally, let the candidate tool set be $\mathcal{T}=\{d(t_1), d(t_2), ..., d(t_N)\}$, where $d(t_i)$ represents the documentation of tool $t_i$ and $N$ is the total number of candidate tools. The LLM is then tasked with generating a tool call $c = (t, p)$, where \(t \in \mathcal{T}\) and \(p\) denotes its parameters: 
%The LLM $\rho_\theta$ needs to generate a tool call $c$ which consists of an appropriate tool \(t\) from the candidate tool set $\mathcal{D}$ and corresponding parameters \(p\), denoted as: 
\begin{equation}
% c = \rho_\theta(t,p|q_u,\mathcal{T})
(t,p) = LLM(q_u,\mathcal{T}).
% \leftarrow LLM()
\end{equation}

\paragraph{Personalized Tool Learning}
In personalized tool learning, we incorporate the users' interaction history alongside their instructions, enabling the LLM to generate tool calls that satisfy both the users' explicit requirements and implicit preferences.
For a user u, we define the interaction history as $\mathcal{H}_u = \{h_u^1, h_u^2, ..., h_u^M\}$, where each $h_u^i$ consists of a past user instruction $q_{u}^i$ and the corresponding tool call $c_u^i=(t_u^i,p_u^i)$, with $t_u^i$ representing the selected tool and $p_u^i$ denoting its associated parameters. Let $c_u = (t_u,p_u)$ represent the personalized tool call for user u, the personalized tool learning task can then be formulated as:
% The LLM $\rho_\theta$ needs to generate a tool call $c$ 
\begin{equation}
(t_u,p_u) = LLM(q_u,\mathcal{T},\mathcal{H}_u),
% c = \rho_\theta(t,p|q_u,\mathcal{T},\mathcal{H}_u)
\end{equation}

% requirement and preference. 
% Formally, for a user $u$, we define the user's instruction as $q_u$ and interaction history as $\mathcal{H_u}$ represented by a sequence $\{h_u^1, h_u^2, ..., h_u^M\}$. Each $h_u^i$ denotes an interaction, i.e., the past user instruction $q_{u}^i$ and the LLM's tool call $c^i=(t_i,p_i)$, containing the selected tool $t_i$ and corresponding parameters $p_i$.
% The candidate tool set is denoted by $\mathcal{T}=\{d(t_1), d(t_2), ..., d(t_N)\}$, where $d(t_i)$ represents the documentation of each candidate tool $t_i$ and $N$ is the total number of candidate tools.
% Based on the user's requirement from instruction $q_u$ and the user's preference from interaction history $\mathcal{H_u}$, the personalized tool-use LLM needs to generate a tool call $c$ which consists of an appropriate tool \(t\) from the candidate tool set $\mathcal{T}$ and corresponding parameters \(p\). 
% select a tool $t$ from the candidate tool set $D$ and call it with appropriate parameters $p$ to solve the task.
% and the toolset as $T=\{T_1, T_2, ..., T_n\}$,
% Based on $P_u$ and $q$, the model is expected to select a user-specific tool set $\hat{D}_u=\{d_i^u\}_{i=1}^k$ from $D$, and then output a user-specific response $r^u$ based on a tool-use trajectory
% $T=[t_j(d^u_i,c^u_i)|d^u_i\in \hat{D}_u]_{j=1}^S$, containing $S$ tool-use steps which sequentially uses a selected user-specific tool $d^u_i$ with corresponding user-specific tool calling parameters $c^u_i$.

\subsection{Benchmark Construction}
Due to the lack of real user interaction histories on tool-usage, we adopt a tool-driven approach to simulate interactions based on pre-constructed user's tool preferences. The whole process for constructing \benchmark, illustrated in Figure~\ref{fig:benchmark}, consists of three steps: tool preparation, preference construction, and data creation.
\subsubsection{Tool Preparation}

\paragraph{Tool Collection}
% We curate xx high-quality tools selected from real-world APIs in ToolBench dataset.
Following ToolBench~\cite{qin2024toolllm}, we adopt the tools from RapidAPI for our benchmark, since it offers a large-scale and diverse collection of real-world tools that can potentially address a wide range of user needs.
% Despite the numerous tools collected in ToolBench, the quality of tools is not guaranteed.
To ensure the quality of the collected tools, we perform strict filtering by removing: 1) outdated tools, which are marked as deprecated in RapidAPI; 2) tools with insufficient information, such as inadequate or missing tool documentation; and 3) duplicate tools, which have repeated tool names, descriptions, or category names.
% After filtering, we obtain xx high-quality tools.
% redundancies
% Complete Documentation. Despite the numerous tools collected in ToolBench (Qin et al., 2023b) from RapidAPI, the documentation quality is not guaranteed. To reduce the failure of tool calling cases caused by inadequate tool descriptions, which focus the evaluation attention on pure LLM abilities, we manually generate high-quality and detailed tool documentation for each tool.

\paragraph{Tool Understanding}
Since tool documentation often contains redundant and irrelevant information, directly extracting tool attributes from it can be challenging.
To address this, we 
% design a tool understanding process that leverages LLMs to comprehend both the functionality and non-functional attributes of each tool. Specifically, we 
first provide the documentation of each tool to LLM and prompt it to generate a tool-use example, including a simulated user instruction and parameters for calling the tool. 
Next, based on the tool documentation and tool-use example, the LLM is instructed to generate descriptions of the tool's functionality and non-functional attributes separately.
Besides, we include demonstrations in the prompt to help the LLM distinguish between these two attribute types.
By leveraging specific tool-use examples and demonstrations, the LLM can develop a more comprehensive understanding of each tool’s functionality and non-functional characteristics.
% plausible scenario for using the tool, relevant to the API
% We also add demonstrations into the instruction to enhance the instruction-following of LLMs in parsing tool documentation.
% tool documentation usually includes plenty of irrelevant information that makes it difficult to understand practical usage for LLMs
% To ensure a holistic understanding of tool attributes to align with diverse users' requirements and preferences, we leverage LLM to comprehend both the functionalities and non-functional attributes of each tool. Specifically, 
% we provide the tool documentation of each tool, and prompt LLM to understand 

\subsubsection{Preference Construction}

\paragraph{Tool Classification}
% that meet three main principles:
% tools with the same functionalities 
To identify potential tool-usage scenarios for users, we classify tools with the same functionalities into groups.
Specifically, we first employ the Ada Embedding model~\footnote{\url{https://platform.openai.com/docs/guides/embeddings/embedding-models}.} to compute embeddings for the functionality descriptions of all tools.
Then, we apply the DBSCAN algorithm~\cite{schubert2017dbscan} to cluster these tools into multiple groups based on the similarity of their embeddings.
Within each group, the tools share the same functionality and can be applied to a specific tool-usage scenario.
% Consequently, all tools within a group share the same functionality and can be applied to a specific tool-usage scenario.
To further ensure that tools within each group exhibit uniform functionality, we conduct rigorous filtering and only retain groups where tools 1) have the same input-output formats (i.e., required/optional parameters and response schema) and 2) belong to the same category (e.g., sports, music, finance).
% perform rigorous filtering 

% embeddings based on the semantic similarities between task descriptions and retain one instance for each cluster
% text embeddings for these explanations
% Due to the diversity in both the intent detection and slot-filling strategies, as well as the creation of specific tools based on LLM, the synthesized tool set can be highly redundant. 
% For example, tools like “search_movie” and “find_movie” may have different names but essentially perform the same function. 
% we cluster them based on the semantic similarities between task descriptions

\paragraph{Tool Preference Construction}
We leverage non-functional tool attributes to construct the user's tool preference. 
First, we randomly sample a functionality group for a user, representing a potential tool-usage scenario for interaction. Within this group, we choose a tool with specific non-functional attributes as the user's preferred tool, while the others are considered non-preferred. Using the preferred tool as a reference, we retrieve the top-$5$ tools with the most similar non-functional attributes. Similarity is computed based on the embeddings of the tools' non-functional descriptions, which are generated in the Tool Understanding phase. Through multiple iterations of sampling and retrieving, we obtain a diverse set of preferred and non-preferred tools that represent user preferences. After each iteration, we check for functionality overlap between newly retrieved tools and previously selected ones. If an overlap is detected, the tools are discarded, and the sampling process is restarted. This ensures that each tool-usage scenario is associated with only one preferred tool per user.
By following this approach, we construct diverse tool sets that align with different user preferences.
% Through this process, 
% Then, we utilize the Ada Embedding model to compute xx embeddings of all tools, based on descriptions of their non-functional attributes generated in the Tool Understanding phase.
% Then we use the chosen tool to retrieve top-$K$ tools based on the similarity scores between their embeddings.
% This ensures that each user has a unique preference for a single tool in each tool-usage scenario.
% from the whole tool collection 

\subsubsection{Data Creation}

\paragraph{Interaction History Generation}
Based on tool preference, we leverage the LLM to construct the user's interaction history.
Specifically, for each user, we provide LLM with the user's preferred and non-preferred tools, including tool attributes and tool-use examples generated in the Tool Understanding phase.
The LLM will generate a sequence of simulated user-LLM interactions, each consisting of a user instruction and an LLM's tool call, as the user's interaction history.
% , each consisting of a user instruction and an LLM's tool call.
% LLM is then prompted to leverage these tools to 
% between the user and the tool-usage LLM, as the user's interaction history.
% realistic and coherent user's interaction history, containing a sequence of interactions between the user and the tool-usage LLM. 
% Each interaction consists of a user instruction and a corresponding tool call, which can be reused from the provided tool-use examples or newly generated based on the history context.
% Next, we use the preferred tools to synthesize a sequence of interactions as the user's interaction history, each containing a user instruction and a corresponding tool call. 


We design three personalized tool-use settings by generating the interaction history in three types (illustrated in Figure~\ref{fig:history}):
1) \textit{preferred-only} history, where the tools involved in the interactions are all preferred by the user; 
2) \textit{rating-integrated} history, including both the user's preferred and non-preferred tools, with a user's binary rating for each tool-usage interaction representing the user preference, i.e., ``liked'' if the tool aligns with the user preferences, and ``disliked'' otherwise.
% i.e., the rating is set to 1 if the tool is preferred by the user, and 0 otherwise; 
3) \textit{chronological} history, which organizes interactions in time order to reflect changes in user preferences over time, i.e., the more recent tool-usage interactions are more preferred by the user, while earlier interactions are less preferred.
In this way, we can present different forms of user preferences.
% maintain the context length

\paragraph{Instruction Generation}
Next, we use LLM to generate user instructions based on the user's preferred tools that are not included in the user's interaction history. 
% do not appear
% We adopt tools that share the same functionality with other tools as ground truth tools for instruction generation.
We instruct the LLM to avoid directly generating the name of the tool in the instruction, ensuring that the user preference for the tool can only be inferred from the user's interaction history.
% to focus on the tool functionality and do not directly including the tool name in the instruction.
% identifying the preferred tool among xx based on user preferences.
% When generating the user instruction, we ask LLM to focus on the tool functionality and avoid including the tool name in the user instruction directly, in order to xx.
Each user instruction is combined with the user’s interaction history into a data instance. 
% where 

Finally, we obtain 12,000 data instances
% , simulating interaction history and 
encompassing 7,454 tools across 46 categories. We split all data into two parts: a training set comprising 9,000 instances for three personalized settings and a test set containing the rest instances.

\subsection{Benchmark Analysis}
% \paragraph{Statistic Analysis}
% We present the statistical information of our \benchmark in Figure \ref{statistics_length}, Table~\ref{dataset_statistics}. 
We present the statistical information of our \benchmark in Table~\ref{fig:statistics_instances_length}, including the statistics of data instances in three settings and under varying interaction history lengths.
We also present the distribution of tool categories in Figure~\ref{fig:statistics_category}. 
Statistical information demonstrates the diversity and complexity of our dataset.

% % \begin{figure*}[htbp]
%     % 左侧图片
%     \begin{minipage}{0.77\linewidth}  % 调整宽度
%         \centering
%         \includegraphics[width=\linewidth]{images/benchmark_construction.pdf}
%     \end{minipage}%
%     % 间隔
%     \hfill
%     % 右侧表格
%     \begin{minipage}{0.23\linewidth}  % 调整宽度
%         \centering
%         \resizebox{\linewidth}{!}{  % 调整表格至合适的宽度
%             \begin{tabular}{lcc}
%                 \toprule
%                 \textbf{Statistic} & \textbf{Number} \\
%                 \midrule
%                 \rowcolor[HTML]{F2F2F2} 
%                 \textit{Domain Count} &  \\
%                 \midrule
%                 Domain & 103 \\
%                 Requirement & 8 \\
%                 \midrule
%                 \rowcolor[HTML]{F2F2F2} 
%                 \textit{Token Count} &  \\
%                 \midrule
%                 Description & 851.6 $\pm$ 515.2 \\
%                 - Min/Max & [159, 2814] \\
%                 Domain & 1187.2 $\pm$ 1212.1 \\
%                 - Min/Max & [85, 7514] \\
%                 \midrule
%                 \rowcolor[HTML]{F2F2F2} 
%                 \textit{Line Count} &  \\
%                 \midrule
%                 Domain & 75.4 $\pm$ 62.9 \\
%                 - Min/Max & [9, 394] \\
%                 \midrule
%                 \rowcolor[HTML]{F2F2F2} 
%                 \textit{Component Count} &  \\
%                 \midrule
%                 Actions & 4.5 $\pm$ 2.8 \\
%                 - Min/Max & [1, 16] \\
%                 Predicates & 8.1 $\pm$ 4.8 \\
%                 - Min/Max & [1, 25] \\
%                 Types & 1.1 $\pm$ 1.3 \\
%                 - Min/Max & [1, 8] \\
%                 \bottomrule
%             \end{tabular}
%         }
%     \end{minipage}
%     % 公共标题
%     \caption{Dataset construction process (left) and key statistics (right) of the \texttt{\benchmark} dataset.     Dataset construction process including: (a) \textit{Data Acquisition} (\S\ref{sec:data_acquisition}); (b) \textit{Data Filtering and Manual Selection} (\S\ref{sec:data_filtering}); (c) \textit{Data Annotation and Quality Assurance}(\S\ref{sec:data_annotation} and \S\ref{sec:quality_assurance}). Tokens are counted by GPT-2~\cite{openai2019gpt2} tokenizer.}
%     \label{fig:combined}
% \end{figure*}


\begin{figure}[!t]
    \centering
    \includegraphics[width=1.0\linewidth]{statistics_instances_length.pdf}
    \caption{Statistics of data instances in three personalized settings (in the left figure) and distributions of interaction history length (in the right figure).}
    \label{fig:statistics_instances_length}%文中引用该图片代号
% \vspace{-1em}
\end{figure}

\begin{figure}[tbp]
    \centering
    \includegraphics[width=1.0\linewidth]{statistics_category.pdf}
    \caption{Distributions of tool categories.}
    \label{fig:statistics_category}%文中引用该图片代号
\vspace{-1em}
\end{figure}



% \begin{figure}[htbp]
% \centering
% \begin{subfigure}{0.45\linewidth}
%     \centering
%     \includegraphics[width=0.9\linewidth]{statistics_category.pdf}
%     \caption{(a) Distributions of tool categories}
%     \label{statistics_category}%文中引用该图片代号
% \end{subfigure}
% \centering
% \begin{subfigure}{0.45\linewidth}
%     \centering
%     \includegraphics[width=0.9\linewidth]{statistics_length.pdf}
%     \caption{(b) Distribution of History Length}
%     \label{statistics_length}%文中引用该图片代号
% \end{subfigure}
% \end{figure}


% \paragraph{Consistency Analysis}
% % Since the user's interaction history are generated in our benchmark, 
% To verify the reliability of PersonalToolBench, we analyze the consistency of all three types of user’s interaction history in the data instance.  
% % by evaluating how well the users’ interaction history align with the ground truth tool call . 
% Specifically, we randomly sample xx data instances from PersonalToolBench, including all three types of user’s interaction history, and then ask LLM/human annotators to evaluate how well the users’ interaction history align with the ground truth tool call.
% % invite
% For each sampled instance, they should answer two yes/no questions: 1) whether the interaction history reflects the user preference, and 2) whether the ground-truth tool call matches that user preference.
% The results in Figure xx show that our constructed exhibit a xx consistency with the labeled tool call and xx alignment with user preference.


\subsection{Evaluation Metrics}
Given the user's instruction and interaction history, LLM is expected to select the appropriate tool from a candidate tool set, and then call the selected tool with corresponding parameters. Therefore, we define two metrics as follows.
\begin{itemize}
    \item Tool Accuracy (Tool Acc): The metric assesses the ability of LLM to select the appropriate tool to call. If the tool is correctly selected, the score is 1; otherwise, the score is 0.
    \item Parameter Accuracy (Param Acc): The metric assesses the ability of LLM to generate correct parameters for the tool call. If the input parameters are correctly generated, the score is 1; otherwise, the score is 0.
\end{itemize}

% This must be in the first 5 lines to tell arXiv to use pdfLaTeX, which is strongly recommended.
\pdfoutput=1
% In particular, the hyperref package requires pdfLaTeX in order to break URLs across lines.

\documentclass[11pt]{article}

% Change "review" to "final" to generate the final (sometimes called camera-ready) version.
% Change to "preprint" to generate a non-anonymous version with page numbers.
\usepackage{acl}

% Standard package includes
\usepackage{times}
\usepackage{latexsym}

% Draw tables
\usepackage{booktabs}
\usepackage{multirow}
\usepackage{xcolor}
\usepackage{colortbl}
\usepackage{array} 
\usepackage{amsmath}

\newcolumntype{C}{>{\centering\arraybackslash}p{0.07\textwidth}}
% For proper rendering and hyphenation of words containing Latin characters (including in bib files)
\usepackage[T1]{fontenc}
% For Vietnamese characters
% \usepackage[T5]{fontenc}
% See https://www.latex-project.org/help/documentation/encguide.pdf for other character sets
% This assumes your files are encoded as UTF8
\usepackage[utf8]{inputenc}

% This is not strictly necessary, and may be commented out,
% but it will improve the layout of the manuscript,
% and will typically save some space.
\usepackage{microtype}
\DeclareMathOperator*{\argmax}{arg\,max}
% This is also not strictly necessary, and may be commented out.
% However, it will improve the aesthetics of text in
% the typewriter font.
\usepackage{inconsolata}

%Including images in your LaTeX document requires adding
%additional package(s)
\usepackage{graphicx}
% If the title and author information does not fit in the area allocated, uncomment the following
%
%\setlength\titlebox{<dim>}
%
% and set <dim> to something 5cm or larger.

\title{Wi-Chat: Large Language Model Powered Wi-Fi Sensing}

% Author information can be set in various styles:
% For several authors from the same institution:
% \author{Author 1 \and ... \and Author n \\
%         Address line \\ ... \\ Address line}
% if the names do not fit well on one line use
%         Author 1 \\ {\bf Author 2} \\ ... \\ {\bf Author n} \\
% For authors from different institutions:
% \author{Author 1 \\ Address line \\  ... \\ Address line
%         \And  ... \And
%         Author n \\ Address line \\ ... \\ Address line}
% To start a separate ``row'' of authors use \AND, as in
% \author{Author 1 \\ Address line \\  ... \\ Address line
%         \AND
%         Author 2 \\ Address line \\ ... \\ Address line \And
%         Author 3 \\ Address line \\ ... \\ Address line}

% \author{First Author \\
%   Affiliation / Address line 1 \\
%   Affiliation / Address line 2 \\
%   Affiliation / Address line 3 \\
%   \texttt{email@domain} \\\And
%   Second Author \\
%   Affiliation / Address line 1 \\
%   Affiliation / Address line 2 \\
%   Affiliation / Address line 3 \\
%   \texttt{email@domain} \\}
% \author{Haohan Yuan \qquad Haopeng Zhang\thanks{corresponding author} \\ 
%   ALOHA Lab, University of Hawaii at Manoa \\
%   % Affiliation / Address line 2 \\
%   % Affiliation / Address line 3 \\
%   \texttt{\{haohany,haopengz\}@hawaii.edu}}
  
\author{
{Haopeng Zhang$\dag$\thanks{These authors contributed equally to this work.}, Yili Ren$\ddagger$\footnotemark[1], Haohan Yuan$\dag$, Jingzhe Zhang$\ddagger$, Yitong Shen$\ddagger$} \\
ALOHA Lab, University of Hawaii at Manoa$\dag$, University of South Florida$\ddagger$ \\
\{haopengz, haohany\}@hawaii.edu\\
\{yiliren, jingzhe, shen202\}@usf.edu\\}



  
%\author{
%  \textbf{First Author\textsuperscript{1}},
%  \textbf{Second Author\textsuperscript{1,2}},
%  \textbf{Third T. Author\textsuperscript{1}},
%  \textbf{Fourth Author\textsuperscript{1}},
%\\
%  \textbf{Fifth Author\textsuperscript{1,2}},
%  \textbf{Sixth Author\textsuperscript{1}},
%  \textbf{Seventh Author\textsuperscript{1}},
%  \textbf{Eighth Author \textsuperscript{1,2,3,4}},
%\\
%  \textbf{Ninth Author\textsuperscript{1}},
%  \textbf{Tenth Author\textsuperscript{1}},
%  \textbf{Eleventh E. Author\textsuperscript{1,2,3,4,5}},
%  \textbf{Twelfth Author\textsuperscript{1}},
%\\
%  \textbf{Thirteenth Author\textsuperscript{3}},
%  \textbf{Fourteenth F. Author\textsuperscript{2,4}},
%  \textbf{Fifteenth Author\textsuperscript{1}},
%  \textbf{Sixteenth Author\textsuperscript{1}},
%\\
%  \textbf{Seventeenth S. Author\textsuperscript{4,5}},
%  \textbf{Eighteenth Author\textsuperscript{3,4}},
%  \textbf{Nineteenth N. Author\textsuperscript{2,5}},
%  \textbf{Twentieth Author\textsuperscript{1}}
%\\
%\\
%  \textsuperscript{1}Affiliation 1,
%  \textsuperscript{2}Affiliation 2,
%  \textsuperscript{3}Affiliation 3,
%  \textsuperscript{4}Affiliation 4,
%  \textsuperscript{5}Affiliation 5
%\\
%  \small{
%    \textbf{Correspondence:} \href{mailto:email@domain}{email@domain}
%  }
%}

\begin{document}
\maketitle
\begin{abstract}
Recent advancements in Large Language Models (LLMs) have demonstrated remarkable capabilities across diverse tasks. However, their potential to integrate physical model knowledge for real-world signal interpretation remains largely unexplored. In this work, we introduce Wi-Chat, the first LLM-powered Wi-Fi-based human activity recognition system. We demonstrate that LLMs can process raw Wi-Fi signals and infer human activities by incorporating Wi-Fi sensing principles into prompts. Our approach leverages physical model insights to guide LLMs in interpreting Channel State Information (CSI) data without traditional signal processing techniques. Through experiments on real-world Wi-Fi datasets, we show that LLMs exhibit strong reasoning capabilities, achieving zero-shot activity recognition. These findings highlight a new paradigm for Wi-Fi sensing, expanding LLM applications beyond conventional language tasks and enhancing the accessibility of wireless sensing for real-world deployments.
\end{abstract}

\section{Introduction}

In today’s rapidly evolving digital landscape, the transformative power of web technologies has redefined not only how services are delivered but also how complex tasks are approached. Web-based systems have become increasingly prevalent in risk control across various domains. This widespread adoption is due their accessibility, scalability, and ability to remotely connect various types of users. For example, these systems are used for process safety management in industry~\cite{kannan2016web}, safety risk early warning in urban construction~\cite{ding2013development}, and safe monitoring of infrastructural systems~\cite{repetto2018web}. Within these web-based risk management systems, the source search problem presents a huge challenge. Source search refers to the task of identifying the origin of a risky event, such as a gas leak and the emission point of toxic substances. This source search capability is crucial for effective risk management and decision-making.

Traditional approaches to implementing source search capabilities into the web systems often rely on solely algorithmic solutions~\cite{ristic2016study}. These methods, while relatively straightforward to implement, often struggle to achieve acceptable performances due to algorithmic local optima and complex unknown environments~\cite{zhao2020searching}. More recently, web crowdsourcing has emerged as a promising alternative for tackling the source search problem by incorporating human efforts in these web systems on-the-fly~\cite{zhao2024user}. This approach outsources the task of addressing issues encountered during the source search process to human workers, leveraging their capabilities to enhance system performance.

These solutions often employ a human-AI collaborative way~\cite{zhao2023leveraging} where algorithms handle exploration-exploitation and report the encountered problems while human workers resolve complex decision-making bottlenecks to help the algorithms getting rid of local deadlocks~\cite{zhao2022crowd}. Although effective, this paradigm suffers from two inherent limitations: increased operational costs from continuous human intervention, and slow response times of human workers due to sequential decision-making. These challenges motivate our investigation into developing autonomous systems that preserve human-like reasoning capabilities while reducing dependency on massive crowdsourced labor.

Furthermore, recent advancements in large language models (LLMs)~\cite{chang2024survey} and multi-modal LLMs (MLLMs)~\cite{huang2023chatgpt} have unveiled promising avenues for addressing these challenges. One clear opportunity involves the seamless integration of visual understanding and linguistic reasoning for robust decision-making in search tasks. However, whether large models-assisted source search is really effective and efficient for improving the current source search algorithms~\cite{ji2022source} remains unknown. \textit{To address the research gap, we are particularly interested in answering the following two research questions in this work:}

\textbf{\textit{RQ1: }}How can source search capabilities be integrated into web-based systems to support decision-making in time-sensitive risk management scenarios? 
% \sq{I mention ``time-sensitive'' here because I feel like we shall say something about the response time -- LLM has to be faster than humans}

\textbf{\textit{RQ2: }}How can MLLMs and LLMs enhance the effectiveness and efficiency of existing source search algorithms? 

% \textit{\textbf{RQ2:}} To what extent does the performance of large models-assisted search align with or approach the effectiveness of human-AI collaborative search? 

To answer the research questions, we propose a novel framework called Auto-\
S$^2$earch (\textbf{Auto}nomous \textbf{S}ource \textbf{Search}) and implement a prototype system that leverages advanced web technologies to simulate real-world conditions for zero-shot source search. Unlike traditional methods that rely on pre-defined heuristics or extensive human intervention, AutoS$^2$earch employs a carefully designed prompt that encapsulates human rationales, thereby guiding the MLLM to generate coherent and accurate scene descriptions from visual inputs about four directional choices. Based on these language-based descriptions, the LLM is enabled to determine the optimal directional choice through chain-of-thought (CoT) reasoning. Comprehensive empirical validation demonstrates that AutoS$^2$-\ 
earch achieves a success rate of 95–98\%, closely approaching the performance of human-AI collaborative search across 20 benchmark scenarios~\cite{zhao2023leveraging}. 

Our work indicates that the role of humans in future web crowdsourcing tasks may evolve from executors to validators or supervisors. Furthermore, incorporating explanations of LLM decisions into web-based system interfaces has the potential to help humans enhance task performance in risk control.






\section{Related Work}
\label{sec:relatedworks}

% \begin{table*}[t]
% \centering 
% \renewcommand\arraystretch{0.98}
% \fontsize{8}{10}\selectfont \setlength{\tabcolsep}{0.4em}
% \begin{tabular}{@{}lc|cc|cc|cc@{}}
% \toprule
% \textbf{Methods}           & \begin{tabular}[c]{@{}c@{}}\textbf{Training}\\ \textbf{Paradigm}\end{tabular} & \begin{tabular}[c]{@{}c@{}}\textbf{$\#$ PT Data}\\ \textbf{(Tokens)}\end{tabular} & \begin{tabular}[c]{@{}c@{}}\textbf{$\#$ IFT Data}\\ \textbf{(Samples)}\end{tabular} & \textbf{Code}  & \begin{tabular}[c]{@{}c@{}}\textbf{Natural}\\ \textbf{Language}\end{tabular} & \begin{tabular}[c]{@{}c@{}}\textbf{Action}\\ \textbf{Trajectories}\end{tabular} & \begin{tabular}[c]{@{}c@{}}\textbf{API}\\ \textbf{Documentation}\end{tabular}\\ \midrule 
% NexusRaven~\citep{srinivasan2023nexusraven} & IFT & - & - & \textcolor{green}{\CheckmarkBold} & \textcolor{green}{\CheckmarkBold} &\textcolor{red}{\XSolidBrush}&\textcolor{red}{\XSolidBrush}\\
% AgentInstruct~\citep{zeng2023agenttuning} & IFT & - & 2k & \textcolor{green}{\CheckmarkBold} & \textcolor{green}{\CheckmarkBold} &\textcolor{red}{\XSolidBrush}&\textcolor{red}{\XSolidBrush} \\
% AgentEvol~\citep{xi2024agentgym} & IFT & - & 14.5k & \textcolor{green}{\CheckmarkBold} & \textcolor{green}{\CheckmarkBold} &\textcolor{green}{\CheckmarkBold}&\textcolor{red}{\XSolidBrush} \\
% Gorilla~\citep{patil2023gorilla}& IFT & - & 16k & \textcolor{green}{\CheckmarkBold} & \textcolor{green}{\CheckmarkBold} &\textcolor{red}{\XSolidBrush}&\textcolor{green}{\CheckmarkBold}\\
% OpenFunctions-v2~\citep{patil2023gorilla} & IFT & - & 65k & \textcolor{green}{\CheckmarkBold} & \textcolor{green}{\CheckmarkBold} &\textcolor{red}{\XSolidBrush}&\textcolor{green}{\CheckmarkBold}\\
% LAM~\citep{zhang2024agentohana} & IFT & - & 42.6k & \textcolor{green}{\CheckmarkBold} & \textcolor{green}{\CheckmarkBold} &\textcolor{green}{\CheckmarkBold}&\textcolor{red}{\XSolidBrush} \\
% xLAM~\citep{liu2024apigen} & IFT & - & 60k & \textcolor{green}{\CheckmarkBold} & \textcolor{green}{\CheckmarkBold} &\textcolor{green}{\CheckmarkBold}&\textcolor{red}{\XSolidBrush} \\\midrule
% LEMUR~\citep{xu2024lemur} & PT & 90B & 300k & \textcolor{green}{\CheckmarkBold} & \textcolor{green}{\CheckmarkBold} &\textcolor{green}{\CheckmarkBold}&\textcolor{red}{\XSolidBrush}\\
% \rowcolor{teal!12} \method & PT & 103B & 95k & \textcolor{green}{\CheckmarkBold} & \textcolor{green}{\CheckmarkBold} & \textcolor{green}{\CheckmarkBold} & \textcolor{green}{\CheckmarkBold} \\
% \bottomrule
% \end{tabular}
% \caption{Summary of existing tuning- and pretraining-based LLM agents with their training sample sizes. "PT" and "IFT" denote "Pre-Training" and "Instruction Fine-Tuning", respectively. }
% \label{tab:related}
% \end{table*}

\begin{table*}[ht]
\begin{threeparttable}
\centering 
\renewcommand\arraystretch{0.98}
\fontsize{7}{9}\selectfont \setlength{\tabcolsep}{0.2em}
\begin{tabular}{@{}l|c|c|ccc|cc|cc|cccc@{}}
\toprule
\textbf{Methods} & \textbf{Datasets}           & \begin{tabular}[c]{@{}c@{}}\textbf{Training}\\ \textbf{Paradigm}\end{tabular} & \begin{tabular}[c]{@{}c@{}}\textbf{\# PT Data}\\ \textbf{(Tokens)}\end{tabular} & \begin{tabular}[c]{@{}c@{}}\textbf{\# IFT Data}\\ \textbf{(Samples)}\end{tabular} & \textbf{\# APIs} & \textbf{Code}  & \begin{tabular}[c]{@{}c@{}}\textbf{Nat.}\\ \textbf{Lang.}\end{tabular} & \begin{tabular}[c]{@{}c@{}}\textbf{Action}\\ \textbf{Traj.}\end{tabular} & \begin{tabular}[c]{@{}c@{}}\textbf{API}\\ \textbf{Doc.}\end{tabular} & \begin{tabular}[c]{@{}c@{}}\textbf{Func.}\\ \textbf{Call}\end{tabular} & \begin{tabular}[c]{@{}c@{}}\textbf{Multi.}\\ \textbf{Step}\end{tabular}  & \begin{tabular}[c]{@{}c@{}}\textbf{Plan}\\ \textbf{Refine}\end{tabular}  & \begin{tabular}[c]{@{}c@{}}\textbf{Multi.}\\ \textbf{Turn}\end{tabular}\\ \midrule 
\multicolumn{13}{l}{\emph{Instruction Finetuning-based LLM Agents for Intrinsic Reasoning}}  \\ \midrule
FireAct~\cite{chen2023fireact} & FireAct & IFT & - & 2.1K & 10 & \textcolor{red}{\XSolidBrush} &\textcolor{green}{\CheckmarkBold} &\textcolor{green}{\CheckmarkBold}  & \textcolor{red}{\XSolidBrush} &\textcolor{green}{\CheckmarkBold} & \textcolor{red}{\XSolidBrush} &\textcolor{green}{\CheckmarkBold} & \textcolor{red}{\XSolidBrush} \\
ToolAlpaca~\cite{tang2023toolalpaca} & ToolAlpaca & IFT & - & 4.0K & 400 & \textcolor{red}{\XSolidBrush} &\textcolor{green}{\CheckmarkBold} &\textcolor{green}{\CheckmarkBold} & \textcolor{red}{\XSolidBrush} &\textcolor{green}{\CheckmarkBold} & \textcolor{red}{\XSolidBrush}  &\textcolor{green}{\CheckmarkBold} & \textcolor{red}{\XSolidBrush}  \\
ToolLLaMA~\cite{qin2023toolllm} & ToolBench & IFT & - & 12.7K & 16,464 & \textcolor{red}{\XSolidBrush} &\textcolor{green}{\CheckmarkBold} &\textcolor{green}{\CheckmarkBold} &\textcolor{red}{\XSolidBrush} &\textcolor{green}{\CheckmarkBold}&\textcolor{green}{\CheckmarkBold}&\textcolor{green}{\CheckmarkBold} &\textcolor{green}{\CheckmarkBold}\\
AgentEvol~\citep{xi2024agentgym} & AgentTraj-L & IFT & - & 14.5K & 24 &\textcolor{red}{\XSolidBrush} & \textcolor{green}{\CheckmarkBold} &\textcolor{green}{\CheckmarkBold}&\textcolor{red}{\XSolidBrush} &\textcolor{green}{\CheckmarkBold}&\textcolor{red}{\XSolidBrush} &\textcolor{red}{\XSolidBrush} &\textcolor{green}{\CheckmarkBold}\\
Lumos~\cite{yin2024agent} & Lumos & IFT  & - & 20.0K & 16 &\textcolor{red}{\XSolidBrush} & \textcolor{green}{\CheckmarkBold} & \textcolor{green}{\CheckmarkBold} &\textcolor{red}{\XSolidBrush} & \textcolor{green}{\CheckmarkBold} & \textcolor{green}{\CheckmarkBold} &\textcolor{red}{\XSolidBrush} & \textcolor{green}{\CheckmarkBold}\\
Agent-FLAN~\cite{chen2024agent} & Agent-FLAN & IFT & - & 24.7K & 20 &\textcolor{red}{\XSolidBrush} & \textcolor{green}{\CheckmarkBold} & \textcolor{green}{\CheckmarkBold} &\textcolor{red}{\XSolidBrush} & \textcolor{green}{\CheckmarkBold}& \textcolor{green}{\CheckmarkBold}&\textcolor{red}{\XSolidBrush} & \textcolor{green}{\CheckmarkBold}\\
AgentTuning~\citep{zeng2023agenttuning} & AgentInstruct & IFT & - & 35.0K & - &\textcolor{red}{\XSolidBrush} & \textcolor{green}{\CheckmarkBold} & \textcolor{green}{\CheckmarkBold} &\textcolor{red}{\XSolidBrush} & \textcolor{green}{\CheckmarkBold} &\textcolor{red}{\XSolidBrush} &\textcolor{red}{\XSolidBrush} & \textcolor{green}{\CheckmarkBold}\\\midrule
\multicolumn{13}{l}{\emph{Instruction Finetuning-based LLM Agents for Function Calling}} \\\midrule
NexusRaven~\citep{srinivasan2023nexusraven} & NexusRaven & IFT & - & - & 116 & \textcolor{green}{\CheckmarkBold} & \textcolor{green}{\CheckmarkBold}  & \textcolor{green}{\CheckmarkBold} &\textcolor{red}{\XSolidBrush} & \textcolor{green}{\CheckmarkBold} &\textcolor{red}{\XSolidBrush} &\textcolor{red}{\XSolidBrush}&\textcolor{red}{\XSolidBrush}\\
Gorilla~\citep{patil2023gorilla} & Gorilla & IFT & - & 16.0K & 1,645 & \textcolor{green}{\CheckmarkBold} &\textcolor{red}{\XSolidBrush} &\textcolor{red}{\XSolidBrush}&\textcolor{green}{\CheckmarkBold} &\textcolor{green}{\CheckmarkBold} &\textcolor{red}{\XSolidBrush} &\textcolor{red}{\XSolidBrush} &\textcolor{red}{\XSolidBrush}\\
OpenFunctions-v2~\citep{patil2023gorilla} & OpenFunctions-v2 & IFT & - & 65.0K & - & \textcolor{green}{\CheckmarkBold} & \textcolor{green}{\CheckmarkBold} &\textcolor{red}{\XSolidBrush} &\textcolor{green}{\CheckmarkBold} &\textcolor{green}{\CheckmarkBold} &\textcolor{red}{\XSolidBrush} &\textcolor{red}{\XSolidBrush} &\textcolor{red}{\XSolidBrush}\\
API Pack~\cite{guo2024api} & API Pack & IFT & - & 1.1M & 11,213 &\textcolor{green}{\CheckmarkBold} &\textcolor{red}{\XSolidBrush} &\textcolor{green}{\CheckmarkBold} &\textcolor{red}{\XSolidBrush} &\textcolor{green}{\CheckmarkBold} &\textcolor{red}{\XSolidBrush}&\textcolor{red}{\XSolidBrush}&\textcolor{red}{\XSolidBrush}\\ 
LAM~\citep{zhang2024agentohana} & AgentOhana & IFT & - & 42.6K & - & \textcolor{green}{\CheckmarkBold} & \textcolor{green}{\CheckmarkBold} &\textcolor{green}{\CheckmarkBold}&\textcolor{red}{\XSolidBrush} &\textcolor{green}{\CheckmarkBold}&\textcolor{red}{\XSolidBrush}&\textcolor{green}{\CheckmarkBold}&\textcolor{green}{\CheckmarkBold}\\
xLAM~\citep{liu2024apigen} & APIGen & IFT & - & 60.0K & 3,673 & \textcolor{green}{\CheckmarkBold} & \textcolor{green}{\CheckmarkBold} &\textcolor{green}{\CheckmarkBold}&\textcolor{red}{\XSolidBrush} &\textcolor{green}{\CheckmarkBold}&\textcolor{red}{\XSolidBrush}&\textcolor{green}{\CheckmarkBold}&\textcolor{green}{\CheckmarkBold}\\\midrule
\multicolumn{13}{l}{\emph{Pretraining-based LLM Agents}}  \\\midrule
% LEMUR~\citep{xu2024lemur} & PT & 90B & 300.0K & - & \textcolor{green}{\CheckmarkBold} & \textcolor{green}{\CheckmarkBold} &\textcolor{green}{\CheckmarkBold}&\textcolor{red}{\XSolidBrush} & \textcolor{red}{\XSolidBrush} &\textcolor{green}{\CheckmarkBold} &\textcolor{red}{\XSolidBrush}&\textcolor{red}{\XSolidBrush}\\
\rowcolor{teal!12} \method & \dataset & PT & 103B & 95.0K  & 76,537  & \textcolor{green}{\CheckmarkBold} & \textcolor{green}{\CheckmarkBold} & \textcolor{green}{\CheckmarkBold} & \textcolor{green}{\CheckmarkBold} & \textcolor{green}{\CheckmarkBold} & \textcolor{green}{\CheckmarkBold} & \textcolor{green}{\CheckmarkBold} & \textcolor{green}{\CheckmarkBold}\\
\bottomrule
\end{tabular}
% \begin{tablenotes}
%     \item $^*$ In addition, the StarCoder-API can offer 4.77M more APIs.
% \end{tablenotes}
\caption{Summary of existing instruction finetuning-based LLM agents for intrinsic reasoning and function calling, along with their training resources and sample sizes. "PT" and "IFT" denote "Pre-Training" and "Instruction Fine-Tuning", respectively.}
\vspace{-2ex}
\label{tab:related}
\end{threeparttable}
\end{table*}

\noindent \textbf{Prompting-based LLM Agents.} Due to the lack of agent-specific pre-training corpus, existing LLM agents rely on either prompt engineering~\cite{hsieh2023tool,lu2024chameleon,yao2022react,wang2023voyager} or instruction fine-tuning~\cite{chen2023fireact,zeng2023agenttuning} to understand human instructions, decompose high-level tasks, generate grounded plans, and execute multi-step actions. 
However, prompting-based methods mainly depend on the capabilities of backbone LLMs (usually commercial LLMs), failing to introduce new knowledge and struggling to generalize to unseen tasks~\cite{sun2024adaplanner,zhuang2023toolchain}. 

\noindent \textbf{Instruction Finetuning-based LLM Agents.} Considering the extensive diversity of APIs and the complexity of multi-tool instructions, tool learning inherently presents greater challenges than natural language tasks, such as text generation~\cite{qin2023toolllm}.
Post-training techniques focus more on instruction following and aligning output with specific formats~\cite{patil2023gorilla,hao2024toolkengpt,qin2023toolllm,schick2024toolformer}, rather than fundamentally improving model knowledge or capabilities. 
Moreover, heavy fine-tuning can hinder generalization or even degrade performance in non-agent use cases, potentially suppressing the original base model capabilities~\cite{ghosh2024a}.

\noindent \textbf{Pretraining-based LLM Agents.} While pre-training serves as an essential alternative, prior works~\cite{nijkamp2023codegen,roziere2023code,xu2024lemur,patil2023gorilla} have primarily focused on improving task-specific capabilities (\eg, code generation) instead of general-domain LLM agents, due to single-source, uni-type, small-scale, and poor-quality pre-training data. 
Existing tool documentation data for agent training either lacks diverse real-world APIs~\cite{patil2023gorilla, tang2023toolalpaca} or is constrained to single-tool or single-round tool execution. 
Furthermore, trajectory data mostly imitate expert behavior or follow function-calling rules with inferior planning and reasoning, failing to fully elicit LLMs' capabilities and handle complex instructions~\cite{qin2023toolllm}. 
Given a wide range of candidate API functions, each comprising various function names and parameters available at every planning step, identifying globally optimal solutions and generalizing across tasks remains highly challenging.



\section{Preliminaries}
\label{Preliminaries}
\begin{figure*}[t]
    \centering
    \includegraphics[width=0.95\linewidth]{fig/HealthGPT_Framework.png}
    \caption{The \ourmethod{} architecture integrates hierarchical visual perception and H-LoRA, employing a task-specific hard router to select visual features and H-LoRA plugins, ultimately generating outputs with an autoregressive manner.}
    \label{fig:architecture}
\end{figure*}
\noindent\textbf{Large Vision-Language Models.} 
The input to a LVLM typically consists of an image $x^{\text{img}}$ and a discrete text sequence $x^{\text{txt}}$. The visual encoder $\mathcal{E}^{\text{img}}$ converts the input image $x^{\text{img}}$ into a sequence of visual tokens $\mathcal{V} = [v_i]_{i=1}^{N_v}$, while the text sequence $x^{\text{txt}}$ is mapped into a sequence of text tokens $\mathcal{T} = [t_i]_{i=1}^{N_t}$ using an embedding function $\mathcal{E}^{\text{txt}}$. The LLM $\mathcal{M_\text{LLM}}(\cdot|\theta)$ models the joint probability of the token sequence $\mathcal{U} = \{\mathcal{V},\mathcal{T}\}$, which is expressed as:
\begin{equation}
    P_\theta(R | \mathcal{U}) = \prod_{i=1}^{N_r} P_\theta(r_i | \{\mathcal{U}, r_{<i}\}),
\end{equation}
where $R = [r_i]_{i=1}^{N_r}$ is the text response sequence. The LVLM iteratively generates the next token $r_i$ based on $r_{<i}$. The optimization objective is to minimize the cross-entropy loss of the response $\mathcal{R}$.
% \begin{equation}
%     \mathcal{L}_{\text{VLM}} = \mathbb{E}_{R|\mathcal{U}}\left[-\log P_\theta(R | \mathcal{U})\right]
% \end{equation}
It is worth noting that most LVLMs adopt a design paradigm based on ViT, alignment adapters, and pre-trained LLMs\cite{liu2023llava,liu2024improved}, enabling quick adaptation to downstream tasks.


\noindent\textbf{VQGAN.}
VQGAN~\cite{esser2021taming} employs latent space compression and indexing mechanisms to effectively learn a complete discrete representation of images. VQGAN first maps the input image $x^{\text{img}}$ to a latent representation $z = \mathcal{E}(x)$ through a encoder $\mathcal{E}$. Then, the latent representation is quantized using a codebook $\mathcal{Z} = \{z_k\}_{k=1}^K$, generating a discrete index sequence $\mathcal{I} = [i_m]_{m=1}^N$, where $i_m \in \mathcal{Z}$ represents the quantized code index:
\begin{equation}
    \mathcal{I} = \text{Quantize}(z|\mathcal{Z}) = \arg\min_{z_k \in \mathcal{Z}} \| z - z_k \|_2.
\end{equation}
In our approach, the discrete index sequence $\mathcal{I}$ serves as a supervisory signal for the generation task, enabling the model to predict the index sequence $\hat{\mathcal{I}}$ from input conditions such as text or other modality signals.  
Finally, the predicted index sequence $\hat{\mathcal{I}}$ is upsampled by the VQGAN decoder $G$, generating the high-quality image $\hat{x}^\text{img} = G(\hat{\mathcal{I}})$.



\noindent\textbf{Low Rank Adaptation.} 
LoRA\cite{hu2021lora} effectively captures the characteristics of downstream tasks by introducing low-rank adapters. The core idea is to decompose the bypass weight matrix $\Delta W\in\mathbb{R}^{d^{\text{in}} \times d^{\text{out}}}$ into two low-rank matrices $ \{A \in \mathbb{R}^{d^{\text{in}} \times r}, B \in \mathbb{R}^{r \times d^{\text{out}}} \}$, where $ r \ll \min\{d^{\text{in}}, d^{\text{out}}\} $, significantly reducing learnable parameters. The output with the LoRA adapter for the input $x$ is then given by:
\begin{equation}
    h = x W_0 + \alpha x \Delta W/r = x W_0 + \alpha xAB/r,
\end{equation}
where matrix $ A $ is initialized with a Gaussian distribution, while the matrix $ B $ is initialized as a zero matrix. The scaling factor $ \alpha/r $ controls the impact of $ \Delta W $ on the model.

\section{HealthGPT}
\label{Method}


\subsection{Unified Autoregressive Generation.}  
% As shown in Figure~\ref{fig:architecture}, 
\ourmethod{} (Figure~\ref{fig:architecture}) utilizes a discrete token representation that covers both text and visual outputs, unifying visual comprehension and generation as an autoregressive task. 
For comprehension, $\mathcal{M}_\text{llm}$ receives the input joint sequence $\mathcal{U}$ and outputs a series of text token $\mathcal{R} = [r_1, r_2, \dots, r_{N_r}]$, where $r_i \in \mathcal{V}_{\text{txt}}$, and $\mathcal{V}_{\text{txt}}$ represents the LLM's vocabulary:
\begin{equation}
    P_\theta(\mathcal{R} \mid \mathcal{U}) = \prod_{i=1}^{N_r} P_\theta(r_i \mid \mathcal{U}, r_{<i}).
\end{equation}
For generation, $\mathcal{M}_\text{llm}$ first receives a special start token $\langle \text{START\_IMG} \rangle$, then generates a series of tokens corresponding to the VQGAN indices $\mathcal{I} = [i_1, i_2, \dots, i_{N_i}]$, where $i_j \in \mathcal{V}_{\text{vq}}$, and $\mathcal{V}_{\text{vq}}$ represents the index range of VQGAN. Upon completion of generation, the LLM outputs an end token $\langle \text{END\_IMG} \rangle$:
\begin{equation}
    P_\theta(\mathcal{I} \mid \mathcal{U}) = \prod_{j=1}^{N_i} P_\theta(i_j \mid \mathcal{U}, i_{<j}).
\end{equation}
Finally, the generated index sequence $\mathcal{I}$ is fed into the decoder $G$, which reconstructs the target image $\hat{x}^{\text{img}} = G(\mathcal{I})$.

\subsection{Hierarchical Visual Perception}  
Given the differences in visual perception between comprehension and generation tasks—where the former focuses on abstract semantics and the latter emphasizes complete semantics—we employ ViT to compress the image into discrete visual tokens at multiple hierarchical levels.
Specifically, the image is converted into a series of features $\{f_1, f_2, \dots, f_L\}$ as it passes through $L$ ViT blocks.

To address the needs of various tasks, the hidden states are divided into two types: (i) \textit{Concrete-grained features} $\mathcal{F}^{\text{Con}} = \{f_1, f_2, \dots, f_k\}, k < L$, derived from the shallower layers of ViT, containing sufficient global features, suitable for generation tasks; 
(ii) \textit{Abstract-grained features} $\mathcal{F}^{\text{Abs}} = \{f_{k+1}, f_{k+2}, \dots, f_L\}$, derived from the deeper layers of ViT, which contain abstract semantic information closer to the text space, suitable for comprehension tasks.

The task type $T$ (comprehension or generation) determines which set of features is selected as the input for the downstream large language model:
\begin{equation}
    \mathcal{F}^{\text{img}}_T =
    \begin{cases}
        \mathcal{F}^{\text{Con}}, & \text{if } T = \text{generation task} \\
        \mathcal{F}^{\text{Abs}}, & \text{if } T = \text{comprehension task}
    \end{cases}
\end{equation}
We integrate the image features $\mathcal{F}^{\text{img}}_T$ and text features $\mathcal{T}$ into a joint sequence through simple concatenation, which is then fed into the LLM $\mathcal{M}_{\text{llm}}$ for autoregressive generation.
% :
% \begin{equation}
%     \mathcal{R} = \mathcal{M}_{\text{llm}}(\mathcal{U}|\theta), \quad \mathcal{U} = [\mathcal{F}^{\text{img}}_T; \mathcal{T}]
% \end{equation}
\subsection{Heterogeneous Knowledge Adaptation}
We devise H-LoRA, which stores heterogeneous knowledge from comprehension and generation tasks in separate modules and dynamically routes to extract task-relevant knowledge from these modules. 
At the task level, for each task type $ T $, we dynamically assign a dedicated H-LoRA submodule $ \theta^T $, which is expressed as:
\begin{equation}
    \mathcal{R} = \mathcal{M}_\text{LLM}(\mathcal{U}|\theta, \theta^T), \quad \theta^T = \{A^T, B^T, \mathcal{R}^T_\text{outer}\}.
\end{equation}
At the feature level for a single task, H-LoRA integrates the idea of Mixture of Experts (MoE)~\cite{masoudnia2014mixture} and designs an efficient matrix merging and routing weight allocation mechanism, thus avoiding the significant computational delay introduced by matrix splitting in existing MoELoRA~\cite{luo2024moelora}. Specifically, we first merge the low-rank matrices (rank = r) of $ k $ LoRA experts into a unified matrix:
\begin{equation}
    \mathbf{A}^{\text{merged}}, \mathbf{B}^{\text{merged}} = \text{Concat}(\{A_i\}_1^k), \text{Concat}(\{B_i\}_1^k),
\end{equation}
where $ \mathbf{A}^{\text{merged}} \in \mathbb{R}^{d^\text{in} \times rk} $ and $ \mathbf{B}^{\text{merged}} \in \mathbb{R}^{rk \times d^\text{out}} $. The $k$-dimension routing layer generates expert weights $ \mathcal{W} \in \mathbb{R}^{\text{token\_num} \times k} $ based on the input hidden state $ x $, and these are expanded to $ \mathbb{R}^{\text{token\_num} \times rk} $ as follows:
\begin{equation}
    \mathcal{W}^\text{expanded} = \alpha k \mathcal{W} / r \otimes \mathbf{1}_r,
\end{equation}
where $ \otimes $ denotes the replication operation.
The overall output of H-LoRA is computed as:
\begin{equation}
    \mathcal{O}^\text{H-LoRA} = (x \mathbf{A}^{\text{merged}} \odot \mathcal{W}^\text{expanded}) \mathbf{B}^{\text{merged}},
\end{equation}
where $ \odot $ represents element-wise multiplication. Finally, the output of H-LoRA is added to the frozen pre-trained weights to produce the final output:
\begin{equation}
    \mathcal{O} = x W_0 + \mathcal{O}^\text{H-LoRA}.
\end{equation}
% In summary, H-LoRA is a task-based dynamic PEFT method that achieves high efficiency in single-task fine-tuning.

\subsection{Training Pipeline}

\begin{figure}[t]
    \centering
    \hspace{-4mm}
    \includegraphics[width=0.94\linewidth]{fig/data.pdf}
    \caption{Data statistics of \texttt{VL-Health}. }
    \label{fig:data}
\end{figure}
\noindent \textbf{1st Stage: Multi-modal Alignment.} 
In the first stage, we design separate visual adapters and H-LoRA submodules for medical unified tasks. For the medical comprehension task, we train abstract-grained visual adapters using high-quality image-text pairs to align visual embeddings with textual embeddings, thereby enabling the model to accurately describe medical visual content. During this process, the pre-trained LLM and its corresponding H-LoRA submodules remain frozen. In contrast, the medical generation task requires training concrete-grained adapters and H-LoRA submodules while keeping the LLM frozen. Meanwhile, we extend the textual vocabulary to include multimodal tokens, enabling the support of additional VQGAN vector quantization indices. The model trains on image-VQ pairs, endowing the pre-trained LLM with the capability for image reconstruction. This design ensures pixel-level consistency of pre- and post-LVLM. The processes establish the initial alignment between the LLM’s outputs and the visual inputs.

\noindent \textbf{2nd Stage: Heterogeneous H-LoRA Plugin Adaptation.}  
The submodules of H-LoRA share the word embedding layer and output head but may encounter issues such as bias and scale inconsistencies during training across different tasks. To ensure that the multiple H-LoRA plugins seamlessly interface with the LLMs and form a unified base, we fine-tune the word embedding layer and output head using a small amount of mixed data to maintain consistency in the model weights. Specifically, during this stage, all H-LoRA submodules for different tasks are kept frozen, with only the word embedding layer and output head being optimized. Through this stage, the model accumulates foundational knowledge for unified tasks by adapting H-LoRA plugins.

\begin{table*}[!t]
\centering
\caption{Comparison of \ourmethod{} with other LVLMs and unified multi-modal models on medical visual comprehension tasks. \textbf{Bold} and \underline{underlined} text indicates the best performance and second-best performance, respectively.}
\resizebox{\textwidth}{!}{
\begin{tabular}{c|lcc|cccccccc|c}
\toprule
\rowcolor[HTML]{E9F3FE} &  &  &  & \multicolumn{2}{c}{\textbf{VQA-RAD \textuparrow}} & \multicolumn{2}{c}{\textbf{SLAKE \textuparrow}} & \multicolumn{2}{c}{\textbf{PathVQA \textuparrow}} &  &  &  \\ 
\cline{5-10}
\rowcolor[HTML]{E9F3FE}\multirow{-2}{*}{\textbf{Type}} & \multirow{-2}{*}{\textbf{Model}} & \multirow{-2}{*}{\textbf{\# Params}} & \multirow{-2}{*}{\makecell{\textbf{Medical} \\ \textbf{LVLM}}} & \textbf{close} & \textbf{all} & \textbf{close} & \textbf{all} & \textbf{close} & \textbf{all} & \multirow{-2}{*}{\makecell{\textbf{MMMU} \\ \textbf{-Med}}\textuparrow} & \multirow{-2}{*}{\textbf{OMVQA}\textuparrow} & \multirow{-2}{*}{\textbf{Avg. \textuparrow}} \\ 
\midrule \midrule
\multirow{9}{*}{\textbf{Comp. Only}} 
& Med-Flamingo & 8.3B & \Large \ding{51} & 58.6 & 43.0 & 47.0 & 25.5 & 61.9 & 31.3 & 28.7 & 34.9 & 41.4 \\
& LLaVA-Med & 7B & \Large \ding{51} & 60.2 & 48.1 & 58.4 & 44.8 & 62.3 & 35.7 & 30.0 & 41.3 & 47.6 \\
& HuatuoGPT-Vision & 7B & \Large \ding{51} & 66.9 & 53.0 & 59.8 & 49.1 & 52.9 & 32.0 & 42.0 & 50.0 & 50.7 \\
& BLIP-2 & 6.7B & \Large \ding{55} & 43.4 & 36.8 & 41.6 & 35.3 & 48.5 & 28.8 & 27.3 & 26.9 & 36.1 \\
& LLaVA-v1.5 & 7B & \Large \ding{55} & 51.8 & 42.8 & 37.1 & 37.7 & 53.5 & 31.4 & 32.7 & 44.7 & 41.5 \\
& InstructBLIP & 7B & \Large \ding{55} & 61.0 & 44.8 & 66.8 & 43.3 & 56.0 & 32.3 & 25.3 & 29.0 & 44.8 \\
& Yi-VL & 6B & \Large \ding{55} & 52.6 & 42.1 & 52.4 & 38.4 & 54.9 & 30.9 & 38.0 & 50.2 & 44.9 \\
& InternVL2 & 8B & \Large \ding{55} & 64.9 & 49.0 & 66.6 & 50.1 & 60.0 & 31.9 & \underline{43.3} & 54.5 & 52.5\\
& Llama-3.2 & 11B & \Large \ding{55} & 68.9 & 45.5 & 72.4 & 52.1 & 62.8 & 33.6 & 39.3 & 63.2 & 54.7 \\
\midrule
\multirow{5}{*}{\textbf{Comp. \& Gen.}} 
& Show-o & 1.3B & \Large \ding{55} & 50.6 & 33.9 & 31.5 & 17.9 & 52.9 & 28.2 & 22.7 & 45.7 & 42.6 \\
& Unified-IO 2 & 7B & \Large \ding{55} & 46.2 & 32.6 & 35.9 & 21.9 & 52.5 & 27.0 & 25.3 & 33.0 & 33.8 \\
& Janus & 1.3B & \Large \ding{55} & 70.9 & 52.8 & 34.7 & 26.9 & 51.9 & 27.9 & 30.0 & 26.8 & 33.5 \\
& \cellcolor[HTML]{DAE0FB}HealthGPT-M3 & \cellcolor[HTML]{DAE0FB}3.8B & \cellcolor[HTML]{DAE0FB}\Large \ding{51} & \cellcolor[HTML]{DAE0FB}\underline{73.7} & \cellcolor[HTML]{DAE0FB}\underline{55.9} & \cellcolor[HTML]{DAE0FB}\underline{74.6} & \cellcolor[HTML]{DAE0FB}\underline{56.4} & \cellcolor[HTML]{DAE0FB}\underline{78.7} & \cellcolor[HTML]{DAE0FB}\underline{39.7} & \cellcolor[HTML]{DAE0FB}\underline{43.3} & \cellcolor[HTML]{DAE0FB}\underline{68.5} & \cellcolor[HTML]{DAE0FB}\underline{61.3} \\
& \cellcolor[HTML]{DAE0FB}HealthGPT-L14 & \cellcolor[HTML]{DAE0FB}14B & \cellcolor[HTML]{DAE0FB}\Large \ding{51} & \cellcolor[HTML]{DAE0FB}\textbf{77.7} & \cellcolor[HTML]{DAE0FB}\textbf{58.3} & \cellcolor[HTML]{DAE0FB}\textbf{76.4} & \cellcolor[HTML]{DAE0FB}\textbf{64.5} & \cellcolor[HTML]{DAE0FB}\textbf{85.9} & \cellcolor[HTML]{DAE0FB}\textbf{44.4} & \cellcolor[HTML]{DAE0FB}\textbf{49.2} & \cellcolor[HTML]{DAE0FB}\textbf{74.4} & \cellcolor[HTML]{DAE0FB}\textbf{66.4} \\
\bottomrule
\end{tabular}
}
\label{tab:results}
\end{table*}
\begin{table*}[ht]
    \centering
    \caption{The experimental results for the four modality conversion tasks.}
    \resizebox{\textwidth}{!}{
    \begin{tabular}{l|ccc|ccc|ccc|ccc}
        \toprule
        \rowcolor[HTML]{E9F3FE} & \multicolumn{3}{c}{\textbf{CT to MRI (Brain)}} & \multicolumn{3}{c}{\textbf{CT to MRI (Pelvis)}} & \multicolumn{3}{c}{\textbf{MRI to CT (Brain)}} & \multicolumn{3}{c}{\textbf{MRI to CT (Pelvis)}} \\
        \cline{2-13}
        \rowcolor[HTML]{E9F3FE}\multirow{-2}{*}{\textbf{Model}}& \textbf{SSIM $\uparrow$} & \textbf{PSNR $\uparrow$} & \textbf{MSE $\downarrow$} & \textbf{SSIM $\uparrow$} & \textbf{PSNR $\uparrow$} & \textbf{MSE $\downarrow$} & \textbf{SSIM $\uparrow$} & \textbf{PSNR $\uparrow$} & \textbf{MSE $\downarrow$} & \textbf{SSIM $\uparrow$} & \textbf{PSNR $\uparrow$} & \textbf{MSE $\downarrow$} \\
        \midrule \midrule
        pix2pix & 71.09 & 32.65 & 36.85 & 59.17 & 31.02 & 51.91 & 78.79 & 33.85 & 28.33 & 72.31 & 32.98 & 36.19 \\
        CycleGAN & 54.76 & 32.23 & 40.56 & 54.54 & 30.77 & 55.00 & 63.75 & 31.02 & 52.78 & 50.54 & 29.89 & 67.78 \\
        BBDM & {71.69} & {32.91} & {34.44} & 57.37 & 31.37 & 48.06 & \textbf{86.40} & 34.12 & 26.61 & {79.26} & 33.15 & 33.60 \\
        Vmanba & 69.54 & 32.67 & 36.42 & {63.01} & {31.47} & {46.99} & 79.63 & 34.12 & 26.49 & 77.45 & 33.53 & 31.85 \\
        DiffMa & 71.47 & 32.74 & 35.77 & 62.56 & 31.43 & 47.38 & 79.00 & {34.13} & {26.45} & 78.53 & {33.68} & {30.51} \\
        \rowcolor[HTML]{DAE0FB}HealthGPT-M3 & \underline{79.38} & \underline{33.03} & \underline{33.48} & \underline{71.81} & \underline{31.83} & \underline{43.45} & {85.06} & \textbf{34.40} & \textbf{25.49} & \underline{84.23} & \textbf{34.29} & \textbf{27.99} \\
        \rowcolor[HTML]{DAE0FB}HealthGPT-L14 & \textbf{79.73} & \textbf{33.10} & \textbf{32.96} & \textbf{71.92} & \textbf{31.87} & \textbf{43.09} & \underline{85.31} & \underline{34.29} & \underline{26.20} & \textbf{84.96} & \underline{34.14} & \underline{28.13} \\
        \bottomrule
    \end{tabular}
    }
    \label{tab:conversion}
\end{table*}

\noindent \textbf{3rd Stage: Visual Instruction Fine-Tuning.}  
In the third stage, we introduce additional task-specific data to further optimize the model and enhance its adaptability to downstream tasks such as medical visual comprehension (e.g., medical QA, medical dialogues, and report generation) or generation tasks (e.g., super-resolution, denoising, and modality conversion). Notably, by this stage, the word embedding layer and output head have been fine-tuned, only the H-LoRA modules and adapter modules need to be trained. This strategy significantly improves the model's adaptability and flexibility across different tasks.


\section{Experiment}
\label{s:experiment}

\subsection{Data Description}
We evaluate our method on FI~\cite{you2016building}, Twitter\_LDL~\cite{yang2017learning} and Artphoto~\cite{machajdik2010affective}.
FI is a public dataset built from Flickr and Instagram, with 23,308 images and eight emotion categories, namely \textit{amusement}, \textit{anger}, \textit{awe},  \textit{contentment}, \textit{disgust}, \textit{excitement},  \textit{fear}, and \textit{sadness}. 
% Since images in FI are all copyrighted by law, some images are corrupted now, so we remove these samples and retain 21,828 images.
% T4SA contains images from Twitter, which are classified into three categories: \textit{positive}, \textit{neutral}, and \textit{negative}. In this paper, we adopt the base version of B-T4SA, which contains 470,586 images and provides text descriptions of the corresponding tweets.
Twitter\_LDL contains 10,045 images from Twitter, with the same eight categories as the FI dataset.
% 。
For these two datasets, they are randomly split into 80\%
training and 20\% testing set.
Artphoto contains 806 artistic photos from the DeviantArt website, which we use to further evaluate the zero-shot capability of our model.
% on the small-scale dataset.
% We construct and publicly release the first image sentiment analysis dataset containing metadata.
% 。

% Based on these datasets, we are the first to construct and publicly release metadata-enhanced image sentiment analysis datasets. These datasets include scenes, tags, descriptions, and corresponding confidence scores, and are available at this link for future research purposes.


% 
\begin{table}[t]
\centering
% \begin{center}
\caption{Overall performance of different models on FI and Twitter\_LDL datasets.}
\label{tab:cap1}
% \resizebox{\linewidth}{!}
{
\begin{tabular}{l|c|c|c|c}
\hline
\multirow{2}{*}{\textbf{Model}} & \multicolumn{2}{c|}{\textbf{FI}}  & \multicolumn{2}{c}{\textbf{Twitter\_LDL}} \\ \cline{2-5} 
  & \textbf{Accuracy} & \textbf{F1} & \textbf{Accuracy} & \textbf{F1}  \\ \hline
% (\rownumber)~AlexNet~\cite{krizhevsky2017imagenet}  & 58.13\% & 56.35\%  & 56.24\%& 55.02\%  \\ 
% (\rownumber)~VGG16~\cite{simonyan2014very}  & 63.75\%& 63.08\%  & 59.34\%& 59.02\%  \\ 
(\rownumber)~ResNet101~\cite{he2016deep} & 66.16\%& 65.56\%  & 62.02\% & 61.34\%  \\ 
(\rownumber)~CDA~\cite{han2023boosting} & 66.71\%& 65.37\%  & 64.14\% & 62.85\%  \\ 
(\rownumber)~CECCN~\cite{ruan2024color} & 67.96\%& 66.74\%  & 64.59\%& 64.72\% \\ 
(\rownumber)~EmoVIT~\cite{xie2024emovit} & 68.09\%& 67.45\%  & 63.12\% & 61.97\%  \\ 
(\rownumber)~ComLDL~\cite{zhang2022compound} & 68.83\%& 67.28\%  & 65.29\% & 63.12\%  \\ 
(\rownumber)~WSDEN~\cite{li2023weakly} & 69.78\%& 69.61\%  & 67.04\% & 65.49\% \\ 
(\rownumber)~ECWA~\cite{deng2021emotion} & 70.87\%& 69.08\%  & 67.81\% & 66.87\%  \\ 
(\rownumber)~EECon~\cite{yang2023exploiting} & 71.13\%& 68.34\%  & 64.27\%& 63.16\%  \\ 
(\rownumber)~MAM~\cite{zhang2024affective} & 71.44\%  & 70.83\% & 67.18\%  & 65.01\%\\ 
(\rownumber)~TGCA-PVT~\cite{chen2024tgca}   & 73.05\%  & 71.46\% & 69.87\%  & 68.32\% \\ 
(\rownumber)~OEAN~\cite{zhang2024object}   & 73.40\%  & 72.63\% & 70.52\%  & 69.47\% \\ \hline
(\rownumber)~\shortname  & \textbf{79.48\%} & \textbf{79.22\%} & \textbf{74.12\%} & \textbf{73.09\%} \\ \hline
\end{tabular}
}
\vspace{-6mm}
% \end{center}
\end{table}
% 

\subsection{Experiment Setting}
% \subsubsection{Model Setting.}
% 
\textbf{Model Setting:}
For feature representation, we set $k=10$ to select object tags, and adopt clip-vit-base-patch32 as the pre-trained model for unified feature representation.
Moreover, we empirically set $(d_e, d_h, d_k, d_s) = (512, 128, 16, 64)$, and set the classification class $L$ to 8.

% 

\textbf{Training Setting:}
To initialize the model, we set all weights such as $\boldsymbol{W}$ following the truncated normal distribution, and use AdamW optimizer with the learning rate of $1 \times 10^{-4}$.
% warmup scheduler of cosine, warmup steps of 2000.
Furthermore, we set the batch size to 32 and the epoch of the training process to 200.
During the implementation, we utilize \textit{PyTorch} to build our entire model.
% , and our project codes are publicly available at https://github.com/zzmyrep/MESN.
% Our project codes as well as data are all publicly available on GitHub\footnote{https://github.com/zzmyrep/KBCEN}.
% Code is available at \href{https://github.com/zzmyrep/KBCEN}{https://github.com/zzmyrep/KBCEN}.

\textbf{Evaluation Metrics:}
Following~\cite{zhang2024affective, chen2024tgca, zhang2024object}, we adopt \textit{accuracy} and \textit{F1} as our evaluation metrics to measure the performance of different methods for image sentiment analysis. 



\subsection{Experiment Result}
% We compare our model against the following baselines: AlexNet~\cite{krizhevsky2017imagenet}, VGG16~\cite{simonyan2014very}, ResNet101~\cite{he2016deep}, CECCN~\cite{ruan2024color}, EmoVIT~\cite{xie2024emovit}, WSCNet~\cite{yang2018weakly}, ECWA~\cite{deng2021emotion}, EECon~\cite{yang2023exploiting}, MAM~\cite{zhang2024affective} and TGCA-PVT~\cite{chen2024tgca}, and the overall results are summarized in Table~\ref{tab:cap1}.
We compare our model against several baselines, and the overall results are summarized in Table~\ref{tab:cap1}.
We observe that our model achieves the best performance in both accuracy and F1 metrics, significantly outperforming the previous models. 
This superior performance is mainly attributed to our effective utilization of metadata to enhance image sentiment analysis, as well as the exceptional capability of the unified sentiment transformer framework we developed. These results strongly demonstrate that our proposed method can bring encouraging performance for image sentiment analysis.

\setcounter{magicrownumbers}{0} 
\begin{table}[t]
\begin{center}
\caption{Ablation study of~\shortname~on FI dataset.} 
% \vspace{1mm}
\label{tab:cap2}
\resizebox{.9\linewidth}{!}
{
\begin{tabular}{lcc}
  \hline
  \textbf{Model} & \textbf{Accuracy} & \textbf{F1} \\
  \hline
  (\rownumber)~Ours (w/o vision) & 65.72\% & 64.54\% \\
  (\rownumber)~Ours (w/o text description) & 74.05\% & 72.58\% \\
  (\rownumber)~Ours (w/o object tag) & 77.45\% & 76.84\% \\
  (\rownumber)~Ours (w/o scene tag) & 78.47\% & 78.21\% \\
  \hline
  (\rownumber)~Ours (w/o unified embedding) & 76.41\% & 76.23\% \\
  (\rownumber)~Ours (w/o adaptive learning) & 76.83\% & 76.56\% \\
  (\rownumber)~Ours (w/o cross-modal fusion) & 76.85\% & 76.49\% \\
  \hline
  (\rownumber)~Ours  & \textbf{79.48\%} & \textbf{79.22\%} \\
  \hline
\end{tabular}
}
\end{center}
\vspace{-5mm}
\end{table}


\begin{figure}[t]
\centering
% \vspace{-2mm}
\includegraphics[width=0.42\textwidth]{fig/2dvisual-linux4-paper2.pdf}
\caption{Visualization of feature distribution on eight categories before (left) and after (right) model processing.}
% 
\label{fig:visualization}
\vspace{-5mm}
\end{figure}

\subsection{Ablation Performance}
In this subsection, we conduct an ablation study to examine which component is really important for performance improvement. The results are reported in Table~\ref{tab:cap2}.

For information utilization, we observe a significant decline in model performance when visual features are removed. Additionally, the performance of \shortname~decreases when different metadata are removed separately, which means that text description, object tag, and scene tag are all critical for image sentiment analysis.
Recalling the model architecture, we separately remove transformer layers of the unified representation module, the adaptive learning module, and the cross-modal fusion module, replacing them with MLPs of the same parameter scale.
In this way, we can observe varying degrees of decline in model performance, indicating that these modules are indispensable for our model to achieve better performance.

\subsection{Visualization}
% 


% % 开始使用minipage进行左右排列
% \begin{minipage}[t]{0.45\textwidth}  % 子图1宽度为45%
%     \centering
%     \includegraphics[width=\textwidth]{2dvisual.pdf}  % 插入图片
%     \captionof{figure}{Visualization of feature distribution.}  % 使用captionof添加图片标题
%     \label{fig:visualization}
% \end{minipage}


% \begin{figure}[t]
% \centering
% \vspace{-2mm}
% \includegraphics[width=0.45\textwidth]{fig/2dvisual.pdf}
% \caption{Visualization of feature distribution.}
% \label{fig:visualization}
% % \vspace{-4mm}
% \end{figure}

% \begin{figure}[t]
% \centering
% \vspace{-2mm}
% \includegraphics[width=0.45\textwidth]{fig/2dvisual-linux3-paper.pdf}
% \caption{Visualization of feature distribution.}
% \label{fig:visualization}
% % \vspace{-4mm}
% \end{figure}



\begin{figure}[tbp]   
\vspace{-4mm}
  \centering            
  \subfloat[Depth of adaptive learning layers]   
  {
    \label{fig:subfig1}\includegraphics[width=0.22\textwidth]{fig/fig_sensitivity-a5}
  }
  \subfloat[Depth of fusion layers]
  {
    % \label{fig:subfig2}\includegraphics[width=0.22\textwidth]{fig/fig_sensitivity-b2}
    \label{fig:subfig2}\includegraphics[width=0.22\textwidth]{fig/fig_sensitivity-b2-num.pdf}
  }
  \caption{Sensitivity study of \shortname~on different depth. }   
  \label{fig:fig_sensitivity}  
\vspace{-2mm}
\end{figure}

% \begin{figure}[htbp]
% \centerline{\includegraphics{2dvisual.pdf}}
% \caption{Visualization of feature distribution.}
% \label{fig:visualization}
% \end{figure}

% In Fig.~\ref{fig:visualization}, we use t-SNE~\cite{van2008visualizing} to reduce the dimension of data features for visualization, Figure in left represents the metadata features before model processing, the features are obtained by embedding through the CLIP model, and figure in right shows the features of the data after model processing, it can be observed that after the model processing, the data with different label categories fall in different regions in the space, therefore, we can conclude that the Therefore, we can conclude that the model can effectively utilize the information contained in the metadata and use it to guide the model for classification.

In Fig.~\ref{fig:visualization}, we use t-SNE~\cite{van2008visualizing} to reduce the dimension of data features for visualization.
The left figure shows metadata features before being processed by our model (\textit{i.e.}, embedded by CLIP), while the right shows the distribution of features after being processed by our model.
We can observe that after the model processing, data with the same label are closer to each other, while others are farther away.
Therefore, it shows that the model can effectively utilize the information contained in the metadata and use it to guide the classification process.

\subsection{Sensitivity Analysis}
% 
In this subsection, we conduct a sensitivity analysis to figure out the effect of different depth settings of adaptive learning layers and fusion layers. 
% In this subsection, we conduct a sensitivity analysis to figure out the effect of different depth settings on the model. 
% Fig.~\ref{fig:fig_sensitivity} presents the effect of different depth settings of adaptive learning layers and fusion layers. 
Taking Fig.~\ref{fig:fig_sensitivity} (a) as an example, the model performance improves with increasing depth, reaching the best performance at a depth of 4.
% Taking Fig.~\ref{fig:fig_sensitivity} (a) as an example, the performance of \shortname~improves with the increase of depth at first, reaching the best performance at a depth of 4.
When the depth continues to increase, the accuracy decreases to varying degrees.
Similar results can be observed in Fig.~\ref{fig:fig_sensitivity} (b).
Therefore, we set their depths to 4 and 6 respectively to achieve the best results.

% Through our experiments, we can observe that the effect of modifying these hyperparameters on the results of the experiments is very weak, and the surface model is not sensitive to the hyperparameters.


\subsection{Zero-shot Capability}
% 

% (1)~GCH~\cite{2010Analyzing} & 21.78\% & (5)~RA-DLNet~\cite{2020A} & 34.01\% \\ \hline
% (2)~WSCNet~\cite{2019WSCNet}  & 30.25\% & (6)~CECCN~\cite{ruan2024color} & 43.83\% \\ \hline
% (3)~PCNN~\cite{2015Robust} & 31.68\%  & (7)~EmoVIT~\cite{xie2024emovit} & 44.90\% \\ \hline
% (4)~AR~\cite{2018Visual} & 32.67\% & (8)~Ours (Zero-shot) & 47.83\% \\ \hline


\begin{table}[t]
\centering
\caption{Zero-shot capability of \shortname.}
\label{tab:cap3}
\resizebox{1\linewidth}{!}
{
\begin{tabular}{lc|lc}
\hline
\textbf{Model} & \textbf{Accuracy} & \textbf{Model} & \textbf{Accuracy} \\ \hline
(1)~WSCNet~\cite{2019WSCNet}  & 30.25\% & (5)~MAM~\cite{zhang2024affective} & 39.56\%  \\ \hline
(2)~AR~\cite{2018Visual} & 32.67\% & (6)~CECCN~\cite{ruan2024color} & 43.83\% \\ \hline
(3)~RA-DLNet~\cite{2020A} & 34.01\%  & (7)~EmoVIT~\cite{xie2024emovit} & 44.90\% \\ \hline
(4)~CDA~\cite{han2023boosting} & 38.64\% & (8)~Ours (Zero-shot) & 47.83\% \\ \hline
\end{tabular}
}
\vspace{-5mm}
\end{table}

% We use the model trained on the FI dataset to test on the artphoto dataset to verify the model's generalization ability as well as robustness to other distributed datasets.
% We can observe that the MESN model shows strong competitiveness in terms of accuracy when compared to other trained models, which suggests that the model has a good generalization ability in the OOD task.

To validate the model's generalization ability and robustness to other distributed datasets, we directly test the model trained on the FI dataset, without training on Artphoto. 
% As observed in Table 3, compared to other models trained on Artphoto, we achieve highly competitive zero-shot performance, indicating that the model has good generalization ability in out-of-distribution tasks.
From Table~\ref{tab:cap3}, we can observe that compared with other models trained on Artphoto, we achieve competitive zero-shot performance, which shows that the model has good generalization ability in out-of-distribution tasks.


%%%%%%%%%%%%
%  E2E     %
%%%%%%%%%%%%


\section{Conclusion}
In this paper, we introduced Wi-Chat, the first LLM-powered Wi-Fi-based human activity recognition system that integrates the reasoning capabilities of large language models with the sensing potential of wireless signals. Our experimental results on a self-collected Wi-Fi CSI dataset demonstrate the promising potential of LLMs in enabling zero-shot Wi-Fi sensing. These findings suggest a new paradigm for human activity recognition that does not rely on extensive labeled data. We hope future research will build upon this direction, further exploring the applications of LLMs in signal processing domains such as IoT, mobile sensing, and radar-based systems.

\section*{Limitations}
While our work represents the first attempt to leverage LLMs for processing Wi-Fi signals, it is a preliminary study focused on a relatively simple task: Wi-Fi-based human activity recognition. This choice allows us to explore the feasibility of LLMs in wireless sensing but also comes with certain limitations.

Our approach primarily evaluates zero-shot performance, which, while promising, may still lag behind traditional supervised learning methods in highly complex or fine-grained recognition tasks. Besides, our study is limited to a controlled environment with a self-collected dataset, and the generalizability of LLMs to diverse real-world scenarios with varying Wi-Fi conditions, environmental interference, and device heterogeneity remains an open question.

Additionally, we have yet to explore the full potential of LLMs in more advanced Wi-Fi sensing applications, such as fine-grained gesture recognition, occupancy detection, and passive health monitoring. Future work should investigate the scalability of LLM-based approaches, their robustness to domain shifts, and their integration with multimodal sensing techniques in broader IoT applications.


% Bibliography entries for the entire Anthology, followed by custom entries
%\bibliography{anthology,custom}
% Custom bibliography entries only
\bibliography{main}
\newpage
\appendix

\section{Experiment prompts}
\label{sec:prompt}
The prompts used in the LLM experiments are shown in the following Table~\ref{tab:prompts}.

\definecolor{titlecolor}{rgb}{0.9, 0.5, 0.1}
\definecolor{anscolor}{rgb}{0.2, 0.5, 0.8}
\definecolor{labelcolor}{HTML}{48a07e}
\begin{table*}[h]
	\centering
	
 % \vspace{-0.2cm}
	
	\begin{center}
		\begin{tikzpicture}[
				chatbox_inner/.style={rectangle, rounded corners, opacity=0, text opacity=1, font=\sffamily\scriptsize, text width=5in, text height=9pt, inner xsep=6pt, inner ysep=6pt},
				chatbox_prompt_inner/.style={chatbox_inner, align=flush left, xshift=0pt, text height=11pt},
				chatbox_user_inner/.style={chatbox_inner, align=flush left, xshift=0pt},
				chatbox_gpt_inner/.style={chatbox_inner, align=flush left, xshift=0pt},
				chatbox/.style={chatbox_inner, draw=black!25, fill=gray!7, opacity=1, text opacity=0},
				chatbox_prompt/.style={chatbox, align=flush left, fill=gray!1.5, draw=black!30, text height=10pt},
				chatbox_user/.style={chatbox, align=flush left},
				chatbox_gpt/.style={chatbox, align=flush left},
				chatbox2/.style={chatbox_gpt, fill=green!25},
				chatbox3/.style={chatbox_gpt, fill=red!20, draw=black!20},
				chatbox4/.style={chatbox_gpt, fill=yellow!30},
				labelbox/.style={rectangle, rounded corners, draw=black!50, font=\sffamily\scriptsize\bfseries, fill=gray!5, inner sep=3pt},
			]
											
			\node[chatbox_user] (q1) {
				\textbf{System prompt}
				\newline
				\newline
				You are a helpful and precise assistant for segmenting and labeling sentences. We would like to request your help on curating a dataset for entity-level hallucination detection.
				\newline \newline
                We will give you a machine generated biography and a list of checked facts about the biography. Each fact consists of a sentence and a label (True/False). Please do the following process. First, breaking down the biography into words. Second, by referring to the provided list of facts, merging some broken down words in the previous step to form meaningful entities. For example, ``strategic thinking'' should be one entity instead of two. Third, according to the labels in the list of facts, labeling each entity as True or False. Specifically, for facts that share a similar sentence structure (\eg, \textit{``He was born on Mach 9, 1941.''} (\texttt{True}) and \textit{``He was born in Ramos Mejia.''} (\texttt{False})), please first assign labels to entities that differ across atomic facts. For example, first labeling ``Mach 9, 1941'' (\texttt{True}) and ``Ramos Mejia'' (\texttt{False}) in the above case. For those entities that are the same across atomic facts (\eg, ``was born'') or are neutral (\eg, ``he,'' ``in,'' and ``on''), please label them as \texttt{True}. For the cases that there is no atomic fact that shares the same sentence structure, please identify the most informative entities in the sentence and label them with the same label as the atomic fact while treating the rest of the entities as \texttt{True}. In the end, output the entities and labels in the following format:
                \begin{itemize}[nosep]
                    \item Entity 1 (Label 1)
                    \item Entity 2 (Label 2)
                    \item ...
                    \item Entity N (Label N)
                \end{itemize}
                % \newline \newline
                Here are two examples:
                \newline\newline
                \textbf{[Example 1]}
                \newline
                [The start of the biography]
                \newline
                \textcolor{titlecolor}{Marianne McAndrew is an American actress and singer, born on November 21, 1942, in Cleveland, Ohio. She began her acting career in the late 1960s, appearing in various television shows and films.}
                \newline
                [The end of the biography]
                \newline \newline
                [The start of the list of checked facts]
                \newline
                \textcolor{anscolor}{[Marianne McAndrew is an American. (False); Marianne McAndrew is an actress. (True); Marianne McAndrew is a singer. (False); Marianne McAndrew was born on November 21, 1942. (False); Marianne McAndrew was born in Cleveland, Ohio. (False); She began her acting career in the late 1960s. (True); She has appeared in various television shows. (True); She has appeared in various films. (True)]}
                \newline
                [The end of the list of checked facts]
                \newline \newline
                [The start of the ideal output]
                \newline
                \textcolor{labelcolor}{[Marianne McAndrew (True); is (True); an (True); American (False); actress (True); and (True); singer (False); , (True); born (True); on (True); November 21, 1942 (False); , (True); in (True); Cleveland, Ohio (False); . (True); She (True); began (True); her (True); acting career (True); in (True); the late 1960s (True); , (True); appearing (True); in (True); various (True); television shows (True); and (True); films (True); . (True)]}
                \newline
                [The end of the ideal output]
				\newline \newline
                \textbf{[Example 2]}
                \newline
                [The start of the biography]
                \newline
                \textcolor{titlecolor}{Doug Sheehan is an American actor who was born on April 27, 1949, in Santa Monica, California. He is best known for his roles in soap operas, including his portrayal of Joe Kelly on ``General Hospital'' and Ben Gibson on ``Knots Landing.''}
                \newline
                [The end of the biography]
                \newline \newline
                [The start of the list of checked facts]
                \newline
                \textcolor{anscolor}{[Doug Sheehan is an American. (True); Doug Sheehan is an actor. (True); Doug Sheehan was born on April 27, 1949. (True); Doug Sheehan was born in Santa Monica, California. (False); He is best known for his roles in soap operas. (True); He portrayed Joe Kelly. (True); Joe Kelly was in General Hospital. (True); General Hospital is a soap opera. (True); He portrayed Ben Gibson. (True); Ben Gibson was in Knots Landing. (True); Knots Landing is a soap opera. (True)]}
                \newline
                [The end of the list of checked facts]
                \newline \newline
                [The start of the ideal output]
                \newline
                \textcolor{labelcolor}{[Doug Sheehan (True); is (True); an (True); American (True); actor (True); who (True); was born (True); on (True); April 27, 1949 (True); in (True); Santa Monica, California (False); . (True); He (True); is (True); best known (True); for (True); his roles in soap operas (True); , (True); including (True); in (True); his portrayal (True); of (True); Joe Kelly (True); on (True); ``General Hospital'' (True); and (True); Ben Gibson (True); on (True); ``Knots Landing.'' (True)]}
                \newline
                [The end of the ideal output]
				\newline \newline
				\textbf{User prompt}
				\newline
				\newline
				[The start of the biography]
				\newline
				\textcolor{magenta}{\texttt{\{BIOGRAPHY\}}}
				\newline
				[The ebd of the biography]
				\newline \newline
				[The start of the list of checked facts]
				\newline
				\textcolor{magenta}{\texttt{\{LIST OF CHECKED FACTS\}}}
				\newline
				[The end of the list of checked facts]
			};
			\node[chatbox_user_inner] (q1_text) at (q1) {
				\textbf{System prompt}
				\newline
				\newline
				You are a helpful and precise assistant for segmenting and labeling sentences. We would like to request your help on curating a dataset for entity-level hallucination detection.
				\newline \newline
                We will give you a machine generated biography and a list of checked facts about the biography. Each fact consists of a sentence and a label (True/False). Please do the following process. First, breaking down the biography into words. Second, by referring to the provided list of facts, merging some broken down words in the previous step to form meaningful entities. For example, ``strategic thinking'' should be one entity instead of two. Third, according to the labels in the list of facts, labeling each entity as True or False. Specifically, for facts that share a similar sentence structure (\eg, \textit{``He was born on Mach 9, 1941.''} (\texttt{True}) and \textit{``He was born in Ramos Mejia.''} (\texttt{False})), please first assign labels to entities that differ across atomic facts. For example, first labeling ``Mach 9, 1941'' (\texttt{True}) and ``Ramos Mejia'' (\texttt{False}) in the above case. For those entities that are the same across atomic facts (\eg, ``was born'') or are neutral (\eg, ``he,'' ``in,'' and ``on''), please label them as \texttt{True}. For the cases that there is no atomic fact that shares the same sentence structure, please identify the most informative entities in the sentence and label them with the same label as the atomic fact while treating the rest of the entities as \texttt{True}. In the end, output the entities and labels in the following format:
                \begin{itemize}[nosep]
                    \item Entity 1 (Label 1)
                    \item Entity 2 (Label 2)
                    \item ...
                    \item Entity N (Label N)
                \end{itemize}
                % \newline \newline
                Here are two examples:
                \newline\newline
                \textbf{[Example 1]}
                \newline
                [The start of the biography]
                \newline
                \textcolor{titlecolor}{Marianne McAndrew is an American actress and singer, born on November 21, 1942, in Cleveland, Ohio. She began her acting career in the late 1960s, appearing in various television shows and films.}
                \newline
                [The end of the biography]
                \newline \newline
                [The start of the list of checked facts]
                \newline
                \textcolor{anscolor}{[Marianne McAndrew is an American. (False); Marianne McAndrew is an actress. (True); Marianne McAndrew is a singer. (False); Marianne McAndrew was born on November 21, 1942. (False); Marianne McAndrew was born in Cleveland, Ohio. (False); She began her acting career in the late 1960s. (True); She has appeared in various television shows. (True); She has appeared in various films. (True)]}
                \newline
                [The end of the list of checked facts]
                \newline \newline
                [The start of the ideal output]
                \newline
                \textcolor{labelcolor}{[Marianne McAndrew (True); is (True); an (True); American (False); actress (True); and (True); singer (False); , (True); born (True); on (True); November 21, 1942 (False); , (True); in (True); Cleveland, Ohio (False); . (True); She (True); began (True); her (True); acting career (True); in (True); the late 1960s (True); , (True); appearing (True); in (True); various (True); television shows (True); and (True); films (True); . (True)]}
                \newline
                [The end of the ideal output]
				\newline \newline
                \textbf{[Example 2]}
                \newline
                [The start of the biography]
                \newline
                \textcolor{titlecolor}{Doug Sheehan is an American actor who was born on April 27, 1949, in Santa Monica, California. He is best known for his roles in soap operas, including his portrayal of Joe Kelly on ``General Hospital'' and Ben Gibson on ``Knots Landing.''}
                \newline
                [The end of the biography]
                \newline \newline
                [The start of the list of checked facts]
                \newline
                \textcolor{anscolor}{[Doug Sheehan is an American. (True); Doug Sheehan is an actor. (True); Doug Sheehan was born on April 27, 1949. (True); Doug Sheehan was born in Santa Monica, California. (False); He is best known for his roles in soap operas. (True); He portrayed Joe Kelly. (True); Joe Kelly was in General Hospital. (True); General Hospital is a soap opera. (True); He portrayed Ben Gibson. (True); Ben Gibson was in Knots Landing. (True); Knots Landing is a soap opera. (True)]}
                \newline
                [The end of the list of checked facts]
                \newline \newline
                [The start of the ideal output]
                \newline
                \textcolor{labelcolor}{[Doug Sheehan (True); is (True); an (True); American (True); actor (True); who (True); was born (True); on (True); April 27, 1949 (True); in (True); Santa Monica, California (False); . (True); He (True); is (True); best known (True); for (True); his roles in soap operas (True); , (True); including (True); in (True); his portrayal (True); of (True); Joe Kelly (True); on (True); ``General Hospital'' (True); and (True); Ben Gibson (True); on (True); ``Knots Landing.'' (True)]}
                \newline
                [The end of the ideal output]
				\newline \newline
				\textbf{User prompt}
				\newline
				\newline
				[The start of the biography]
				\newline
				\textcolor{magenta}{\texttt{\{BIOGRAPHY\}}}
				\newline
				[The ebd of the biography]
				\newline \newline
				[The start of the list of checked facts]
				\newline
				\textcolor{magenta}{\texttt{\{LIST OF CHECKED FACTS\}}}
				\newline
				[The end of the list of checked facts]
			};
		\end{tikzpicture}
        \caption{GPT-4o prompt for labeling hallucinated entities.}\label{tb:gpt-4-prompt}
	\end{center}
\vspace{-0cm}
\end{table*}
% \section{Full Experiment Results}
% \begin{table*}[th]
    \centering
    \small
    \caption{Classification Results}
    \begin{tabular}{lcccc}
        \toprule
        \textbf{Method} & \textbf{Accuracy} & \textbf{Precision} & \textbf{Recall} & \textbf{F1-score} \\
        \midrule
        \multicolumn{5}{c}{\textbf{Zero Shot}} \\
                Zero-shot E-eyes & 0.26 & 0.26 & 0.27 & 0.26 \\
        Zero-shot CARM & 0.24 & 0.24 & 0.24 & 0.24 \\
                Zero-shot SVM & 0.27 & 0.28 & 0.28 & 0.27 \\
        Zero-shot CNN & 0.23 & 0.24 & 0.23 & 0.23 \\
        Zero-shot RNN & 0.26 & 0.26 & 0.26 & 0.26 \\
DeepSeek-0shot & 0.54 & 0.61 & 0.54 & 0.52 \\
DeepSeek-0shot-COT & 0.33 & 0.24 & 0.33 & 0.23 \\
DeepSeek-0shot-Knowledge & 0.45 & 0.46 & 0.45 & 0.44 \\
Gemma2-0shot & 0.35 & 0.22 & 0.38 & 0.27 \\
Gemma2-0shot-COT & 0.36 & 0.22 & 0.36 & 0.27 \\
Gemma2-0shot-Knowledge & 0.32 & 0.18 & 0.34 & 0.20 \\
GPT-4o-mini-0shot & 0.48 & 0.53 & 0.48 & 0.41 \\
GPT-4o-mini-0shot-COT & 0.33 & 0.50 & 0.33 & 0.38 \\
GPT-4o-mini-0shot-Knowledge & 0.49 & 0.31 & 0.49 & 0.36 \\
GPT-4o-0shot & 0.62 & 0.62 & 0.47 & 0.42 \\
GPT-4o-0shot-COT & 0.29 & 0.45 & 0.29 & 0.21 \\
GPT-4o-0shot-Knowledge & 0.44 & 0.52 & 0.44 & 0.39 \\
LLaMA-0shot & 0.32 & 0.25 & 0.32 & 0.24 \\
LLaMA-0shot-COT & 0.12 & 0.25 & 0.12 & 0.09 \\
LLaMA-0shot-Knowledge & 0.32 & 0.25 & 0.32 & 0.28 \\
Mistral-0shot & 0.19 & 0.23 & 0.19 & 0.10 \\
Mistral-0shot-Knowledge & 0.21 & 0.40 & 0.21 & 0.11 \\
        \midrule
        \multicolumn{5}{c}{\textbf{4 Shot}} \\
GPT-4o-mini-4shot & 0.58 & 0.59 & 0.58 & 0.53 \\
GPT-4o-mini-4shot-COT & 0.57 & 0.53 & 0.57 & 0.50 \\
GPT-4o-mini-4shot-Knowledge & 0.56 & 0.51 & 0.56 & 0.47 \\
GPT-4o-4shot & 0.77 & 0.84 & 0.77 & 0.73 \\
GPT-4o-4shot-COT & 0.63 & 0.76 & 0.63 & 0.53 \\
GPT-4o-4shot-Knowledge & 0.72 & 0.82 & 0.71 & 0.66 \\
LLaMA-4shot & 0.29 & 0.24 & 0.29 & 0.21 \\
LLaMA-4shot-COT & 0.20 & 0.30 & 0.20 & 0.13 \\
LLaMA-4shot-Knowledge & 0.15 & 0.23 & 0.13 & 0.13 \\
Mistral-4shot & 0.02 & 0.02 & 0.02 & 0.02 \\
Mistral-4shot-Knowledge & 0.21 & 0.27 & 0.21 & 0.20 \\
        \midrule
        
        \multicolumn{5}{c}{\textbf{Suprevised}} \\
        SVM & 0.94 & 0.92 & 0.91 & 0.91 \\
        CNN & 0.98 & 0.98 & 0.97 & 0.97 \\
        RNN & 0.99 & 0.99 & 0.99 & 0.99 \\
        % \midrule
        % \multicolumn{5}{c}{\textbf{Conventional Wi-Fi-based Human Activity Recognition Systems}} \\
        E-eyes & 1.00 & 1.00 & 1.00 & 1.00 \\
        CARM & 0.98 & 0.98 & 0.98 & 0.98 \\
\midrule
 \multicolumn{5}{c}{\textbf{Vision Models}} \\
           Zero-shot SVM & 0.26 & 0.25 & 0.25 & 0.25 \\
        Zero-shot CNN & 0.26 & 0.25 & 0.26 & 0.26 \\
        Zero-shot RNN & 0.28 & 0.28 & 0.29 & 0.28 \\
        SVM & 0.99 & 0.99 & 0.99 & 0.99 \\
        CNN & 0.98 & 0.99 & 0.98 & 0.98 \\
        RNN & 0.98 & 0.99 & 0.98 & 0.98 \\
GPT-4o-mini-Vision & 0.84 & 0.85 & 0.84 & 0.84 \\
GPT-4o-mini-Vision-COT & 0.90 & 0.91 & 0.90 & 0.90 \\
GPT-4o-Vision & 0.74 & 0.82 & 0.74 & 0.73 \\
GPT-4o-Vision-COT & 0.70 & 0.83 & 0.70 & 0.68 \\
LLaMA-Vision & 0.20 & 0.23 & 0.20 & 0.09 \\
LLaMA-Vision-Knowledge & 0.22 & 0.05 & 0.22 & 0.08 \\

        \bottomrule
    \end{tabular}
    \label{full}
\end{table*}




\end{document}



\section{Method: \framework}
To equip LLM with personalized tool-use capability, we conduct a two-stage training process: 1) personalized SFT, where LLM is fine-tuned on \benchmark to acquire fundamental proficiency in personalized tool usage, and 2) personalized DPO, where LLM is optimized on a preference dataset for better alignment with user preferences.

\paragraph{Personalized SFT.}
The first stage in our approach is Supervised Fine-Tuning (SFT), where we directly fine-tune LLM on the training set of \benchmark. Given the user's instruction $q_u$, interaction history $\mathcal{H}_u$, and the candidate tool set $\mathcal{T}$ as inputs, LLM is trained to generate the ground truth tool call \(c\). $\mathcal{H}_u$ uniformly covers all three types of user interactions to capture diverse user preferences. In this way, LLM can obtain basic personalized tool-usage experiences by understanding both the user needs and preferences.

\paragraph{Personalized DPO.}
In the second stage, we further enhance the LLM's performance through direct preference optimization (DPO)~\cite{NEURIPS2023_a85b405e}. 
Our goal is to guide the LLM to call the user's preferred tools instead of non-preferred ones.
Specifically, for each user instruction $q_u$, we collect multiple tool calls generated by LLM after the SFT stage. 
Then we select the user's preferred and non-preferred tool calls \(c_w\) and \(c_l\) based on the user's tool preference constructed in \benchmark.
\(c_w\) and \(c_l\) will be used to construct the preference dataset \(\mathcal{D}_{\text{DPO}} = \{ (x, c_w, c_l) \}\), where \(x\) denotes the input, including the user instruction $q_u$, interaction history $\mathcal{H}_u$, and the candidate tool set $\mathcal{T}$.
We then apply DPO to optimize the LLM by guiding it to generate the desired tool call \(c_w\) while avoid generating \(c_l\).
% generation direction, and a “rejected” pair to train the LLM to avoid specific outputs.
% prioritize
% encouraging it to generate function parameters similar to \(p_{i}^{\text{b}}\) and discouraging it from generating function parameters similar to \(p_{i}^{\text{w}}\). 
The loss function can be defined as:
\begin{equation} \label{dpo_loss}
\small
{
\mathcal{L} = -\mathbb{E} \left[ \log \sigma \left( \beta \log \frac{\pi_{\theta}(c_w \mid x)}{\pi_{\text{ref}}(c_w \mid x)} - \beta \log \frac{\pi_{\theta}(c_l \mid x)}{\pi_{\text{ref}}(c_l \mid x)} \right) \right],
}
\end{equation}
where \(\sigma\) is the logistic function and \(\beta\) is a weighting parameter that controls the deviation of the policy model $\pi_{\theta}$ (i.e., the LLM we need to optimize) from the reference model $\pi_{\text{ref}}$ (i.e., the LLM after SFT stage).
% the sensitivity of the model's preference to the log-ratio difference between the policy model \(\pi_{\theta}\) for optimization and reference model \(\pi_{\text{ref}}\) derived from the SFT stage.
In this way, LLM can focus on generating tool calls that are more aligned with individual user preferences.
% By directly optimizing for user preferences

\section{Experiments}
\subsection{Setup}
\paragraph{Baselines.} 
% following (Huang et al., 2024a; Zhuang et al., 2023). 
We adopt multiple LLMs from both closed-source and open-source models to ensure a comprehensive evaluation.
For closed-source LLMs, we select two representative models: GPT-4o and GPT-4o-mini from OpenAI.
For open-source LLMs, we include a wide spectrum of models, i.e., LLaMA-3.1-8B~\cite{dubey2024llama}, QWen-2.5-7B~\citep{yang2024qwen2}, Vicuna-7B-v1.5~\cite{chiang2023vicuna} and Mistral-7B-v0.3~\cite{jiang2023mistral}.

\paragraph{Implementation details.} 
% \footnote{\url{https://openai.com/index/introducing-chatgpt-and-whisper-apis/}.} 
% since it its the instruction following .
% The number of candidate tools $N$ is set to $10$, which 
In \benchmark construction, we employ gpt-4o-mini
as the LLM for tool understanding and generation of user instructions and interaction history. 
The candidate tool set consists of three parts: the ground-truth tool along with all other tools sharing the same functionality, five tools retrieved using ToolRetriever~\cite{qin2024toolllm}, and the remaining tools that were randomly sampled.
% The tools retrieved We adopt ToolRetriever~\cite{qin2024toolllm} as the dense retriever which is specifically finetuned on tool retrieval datasets. 


% highlighting several key insights.
\subsection{Main Results}
The detailed experimental results are shown in Table~\ref{main_results}. 
From the results, we can obtain the following key findings.
1) It can be observed that the performance of LLMs is generally unsatisfactory, particularly in tool accuracy with the majority failing to exceed 50\%. This indicates that current LLMs are severely limited in personalized tool-use capabilities. Additionally, the lower tool accuracy compared to parameter accuracy further suggests that personalized tool selection is more challenging than parameter configuration. This is because LLMs must account for both implicit user preferences and explicit user requirements when determining which tool to use.
2) Most LLMs perform worse in the rating-integrated and chronological settings. 
This is likely due to the inclusion of non-preferred interactions in the interaction history, which confuses LLMs and hinders their ability to accurately recognize user preferences. Notably, the chronological setting yields the lowest scores, suggesting that capturing evolving user preferences over time is even more challenging than interpreting explicit user ratings.
3) Our proposed \benchmark significantly outperforms all closed-source and open-source LLMs, demonstrating both effectiveness and robustness. It maintains strong performance, even in the two more challenging settings, by enabling the LLM to better understand diverse manifestations of user preferences and facilitate personalized tool usage.
% with superior
% It can be observed that existing open-source LLMs lay behind closed-source LLMs, i.e., GPT-4o and GPT-4o-mini, across all three scenarios. This is reasonable since GPT-4o is well known for its superior instruction-following and comprehension abilities compared with most open-source LLMs. 

\subsection{Ablation Study}
We conduct ablation studies to investigate the efficacy of the two-stage training process in our \framework.
First, we remove the second training stage (i.e., personalized DPO) to assess its contribution.
Then, we examine the impact of the SFT stage by directly conducting DPO training on the initial LLaMA3-8B model.
Table~\ref{main_results} reports the performance on the test set of \benchmark in all three settings.
The results indicate that the SFT stage is crucial for personalized tool learning performance, as it endows the model with fundamental tool usage and personalization capabilities. Removing the DPO stage results in a slight performance drop, suggesting that it can further refine the tool usage alignment with user preferences.

\subsection{In-depth Analysis}

\begin{figure}[tbp]
    \centering
    \includegraphics[width=1.0\linewidth]{scores_wo_history.pdf}
    \caption{Performance comparison of tool accuracy when provided with and without interaction history.}
    \label{fig:scores_wo_history}
\vspace{-1em}
\end{figure}

\begin{figure}[!t]
    \centering
    \includegraphics[width=1.0\linewidth]{scores_length.pdf}
    \caption{Performance comparison of tool accuracy on different interaction history length in the preferred-only setting.}
    \label{fig:scores_length}
\vspace{-1em}
\end{figure}




\begin{lemma}\label{Lemma:multi1} 
   Fixing the number of data contributor $i$ collects $n_i$, and others' strategies $\strategy_{-i}$, $\hat{\mu}\left(X_i\right)$ is the minimax estimator for the Normal distribution class $\Normaldistrib := \left\{\mathcal{N}(\mu,\sigma^2) \;\middle|\; \mu \in \mathbb{R}\right\}$,
    \begin{align*}
       \hat{\mu}(X_i)  = \underset{\hat{\mu}}{\arg\min} \sbr{\sup _\mu \mathbb{E}\left[(\hat{\mu}( Y_i)- \hat{\mu}( Y_{-i}) )^2 \;\middle|\;  \mu \right] }
    \end{align*} 
     
\end{lemma}


\begin{proof}

\begin{align*}
    & \ \mathbb{E}\left[ \left( \hat{\mu}\left( Y_i \right)-\hat{\mu}\left( Y_{-i} \right)  \right)^2 \right] \\ =  & \ \mathbb{E}\left[ \left( (\hat{\mu}\left(  Y_i \right)-\mu) -(\hat{\mu}\left(  Y_{-i} \right) -\mu) \right)^2   \right] \\ =  & \ A_0 + \mathbb{E}\left[ (\hat{\mu}\left(  Y_i \right)-\mu)^2  \right]
\end{align*}
where $A_0$ is a positive coefficient.

Thus the maximum risk can be written as:

\begin{align*}
    \sup _\mu \mathbb{E}\left[A_0 + \left(\hat{\mu}\left( Y_i\right)-\mu\right)^{2} \;\middle|\;  \mu \right]
\end{align*}


We construct a lower bound on the maximum risk using a sequence of Bayesian risks. Let $\Lambda_{\ell}:=\mathcal{N}\left(0, \ell^2\right), \ell=1,2, \ldots$ be a sequence of prior for $\mu$. For fixed $\ell$, the posterior distribution is:
$$
\begin{aligned}
p\left(\mu \;\middle|\;  X_i\right) & \propto p\left(X_i \;\middle|\;  \mu\right) p(\mu) \\ & \propto \exp \left(-\frac{1}{2 \sigma^2} \sum_{x \in X_i}(x-\mu)^2\right) \exp \left(-\frac{1}{2 \ell^2} \mu^2\right) \\
& \propto \exp \left(-\frac{1}{2}\left(\frac{n_i}{\sigma^2}+\frac{1}{\ell^2}\right) \mu^2+\frac{1}{2} 2 \frac{\sum_{x \in X_i} x}{\sigma^2} \mu\right) .
\end{aligned}
$$

This means the posterior of $\mu$ given $X_i$ is Gaussian with:

\begin{align*}
    \mu \lvert\, X_i & \sim \mathcal{N}\left(\frac{n_i \hat{\mu}\left(X_i\right) / \sigma^2}{n_i / \sigma^2+1 / \ell^2}, \frac{1}{n_i / \sigma^2+1 / \ell^2}\right) 
    \\ & =: \mathcal{N}\left(\mu_{\ell}, \sigma_{\ell}^2\right).
\end{align*}



Therefore, the posterior risk is: 
$$
\begin{aligned}
&   \mathbb{E}\left[A_0 + \left(\hat{\mu}\left( Y_i\right)-\mu\right)^{2}  \;\middle|\;  X_i\right] \\ = &  \mathbb{E}\left[A_0 +  \left(\left(\hat{\mu}\left( Y_i\right)-\mu_{\ell}\right)-\left(\mu-\mu_{\ell}\right)\right)^{2 j} \;\middle|\;  X_i\right] \\ =
& A_0+\int_{-\infty}^{\infty} \underbrace{\left(e-\left(\hat{\mu}\left( Y_i\right)-\mu_{\ell}\right)\right)^2}_{=: F_1\left(e-\left(\hat{\mu}\left(Y_i\right)-\mu_{\ell}\right)\right)} \underbrace{\frac{1}{\sigma_{\ell} \sqrt{2 \pi}} \exp \left(-\frac{e^2}{2 \sigma_{\ell}^2}\right)}_{=: F_2(e)} d e
\end{aligned}
$$

Because:
\begin{itemize}
    \item $F_1(\cdot)$ is even function and increases on $[0, \infty)$;
    \item $F_2(\cdot)$ is even function and decreases on $\left[0, \infty \right)$, and $\int_{\mathbb{R}} F_2(e) de<\infty$
    \item For any $a \in \mathbb{R}, \int_{\mathbb{R}} F_1(e-a) F_2(e) de<\infty$
\end{itemize}

By the corollary of Hardy-Littlewood inequality in Lemma \ref{lemmaHardy},
$$
\int_{\mathbb{R}} F_1(e-a) F_2(e) d e \geq \int_{\mathbb{R}} F_1(e) F_2(e) d e
$$
which means the posterior risk is minimized when $\hat{\mu}\left(Y_i\right)=\mu_{\ell}$. We then write the Bayes risk as, the Bayes risk is minimized by the posterior mean $\mu_{\ell}$:

\begin{align*}    
R_{\ell}:= & \mathbb{E}\left[ A_0+\mathbb{E}\left[\left(\mu-\mu_{\ell}\right)^{2 } \;\middle|\;  X_i\right]\right] \\ = & A_0 + \sigma_{\ell}^{2}
\end{align*}

and the limit of Bayesian risk as $\ell \rightarrow \infty$ is
$$
R_{\infty}:= A_0 + \frac{\sigma^{2}}{n_i}.
$$

When $\hat{\mu}\left(Y_i\right)=\hat{\mu}\left(X_i\right)$, i.e, the contributor submit a set of size $n_i$ with each element equal to $ \hat{\mu}\left(X_i\right)$, the maximum risk is:

\begin{align*}
& \sup _\mu \mathbb{E}\left[A_0+\left(\mu- 
\hat{\mu}\left(Y_i\right) \right)^{2 } \;\middle|\;  \mu \right] \\
= & \sup _\mu \mathbb{E}\left[A_0+\left(\mu- 
\hat{\mu}\left(X_i\right) \right)^{2 } \;\middle|\;  \mu \right]  \\
= & A_0+ \sigma^{2 } n_i^{-1}  \\
= &  R_{\infty}.
\end{align*}

This implies that,
\begin{align*}
    & \underset{\mu}{\sup}\; \mathbb{E} \sbr{ \rbr{\hat{\mu}\left( Y_i \right)-\hat{\mu}\left( Y_{-i} \right)  }^2 \;\middle|\;  \mu }  \\ \geq & \; R_{\infty} =  \sup _\mu \; \mathbb{E}\left[A_0+\left(\mu- 
\hat{\mu}\left(X_i\right) \right)^{2 } \;\middle|\;  \mu \right]
\end{align*}

Therefore, the recommended strategy $\hat{\mu}(Y_i) =\hat{\mu}( X_i)$ has a smaller maximum risk than other strategies. 

\end{proof}




1. The payment from the buyer a constant $v(n^{\star})$.


2. If the payment for every seller is a fixed constant, then sellers can fabricate data without actually collecting data.\\


%$p_1 = b/2 +(\hat{\mu}(Y_1)- \hat{\mu}(Y_2))^2$, $p_2 = b/2 -(\hat{\mu}(Y_1)- \hat{\mu}(Y_2))^2$, seller 1 can choose ${\mu}' = u + \epsilon$, expected payment for seller 1 is larger than $b/2$. %NIC for seller 1: $g({\mu}',\mu ) < b/2$ for all ${\mu}' \neq \mu$. NIC for seller 2: $g( \mu, {\mu}') >  b/2 $ for all ${\mu}' \neq \mu$.

To demonstrate that no truthful mechanism (NIC) satisfies all desired properties in a two-seller setting, we use proof by contradiction.  

Suppose that there is a NIC mechanism $M$ satisfying property 1-5. Under this mechanism, the best strategy for each seller is to collect $N_i^{\star}$ amount of data and submit truthfully, where $N_1^{\star}+N_2^{\star} = n^{\star} $. Since $M$ is NIC for strategy space $\left\{ (f_i,N_i)\right\}_{i=1,2}$, it must be NIC for the sub strategy space $\left\{ (f_i, N_i^{\star})\right\}_{i=1,2}$. 


Consider the case in which everyone collects $N_i^{\star}$ data point and submits $N_i^{\star}$ data point. Assume that the true mean is $\mu$, seller $1$ submit $N({\mu}', \sigma^2),\ {\mu}' = f(\mu) $ while seller 2 submit $N({\mu}, \sigma^2) $. We denote seller 1's expected payment as  $\mathbb{E}\left[ p_1(M,\strategy) \right] = g({\mu}', \mu)$.
Seller 1's utility is then:
\[ u_1(M,f) = \underset{\mu}{\inf} \ g({\mu}', \mu) -c\times N_1^{\star}\] where $c$ is the cost for collecting one data point.


The total payment from the buyer is $v(n^{\star})$, hence by budget balance, \[p_2 (M,f)  = v(n^{\star}) -  p_1(M,f), \ \mathbb{E}\left[ p_2(M,f) \right] = v(n^{\star}) - g({\mu}', \mu) \]

By NIC, we have,
\[ \underset{\mu}{\inf} \ g({\mu}', \mu) -c\times N_1^{\star} \leq \underset{\mu}{\inf} \ g({\mu}, \mu) -c\times N_1^{\star} \] \[ \underset{\mu}{\inf} \ (v(n^{\star})- g( \mu, {\mu}')) -c\times N_2^{\star} \leq \underset{\mu}{\inf} \ (v(n^{\star})-g( \mu, {\mu})) -c\times N_2^{\star}  \]

%Using the fact that $\underset{\mu}{\inf} \ g({\mu}, \mu) = \underset{\mu}{\sup} \ g({\mu}, \mu) = {v(n^{\star})}/2 $.
We obtain that for any ${\mu}'$ and $\mu$,
\[  \underset{\mu}{\inf} \ g({\mu}', \mu) \leq \underset{\mu}{\inf} \ g({\mu}, \mu)   \] \[  \underset{\mu}{\sup} \  g( \mu, {\mu})  \leq \underset{\mu}{\sup } \ g( \mu, {\mu}')  \]

We next show that the inequalities are strict. Assume, for contradiction there exists ${\mu}'$, for any $\mu$, $g({\mu}', 
\mu) \geq \underset{\mu}{\inf} \ g({\mu}, \mu)$. It then follows that $ \underset{\mu}{\inf} \ g({\mu}', \mu) \geq \underset{\mu}{\inf} \ g({\mu}, \mu)$. Under this assumption, seller 1 could fabricate data by submitting $N({\mu}', \sigma^2)$ without collecting any actual data. This contradicts with the fact that $(f_1 = I, N_1 = N_1^{\star})$ is the best strategy for seller 1. Hence, for any ${\mu}'$, there exists some $\mu$ such that $g({\mu}', 
\mu) < \underset{\mu}{\inf} \ g({\mu}, \mu)$. Therefore, for any ${\mu}'$, \[ \underset{\mu}{\inf} \ g({\mu}', \mu) < \underset{\mu}{\inf} \ g({\mu}, \mu). \]Similarly, we also have \[ \underset{\mu}{\sup} \  g( \mu, {\mu})  < \underset{\mu}{\sup } \ g( \mu, {\mu}').  \] 


For any $ {\mu}'$, let $f({\mu}') =  \underset{\mu}{\arg\sup}\, g(\mu, {\mu}')$, then we have for any ${\mu}'$,  $f({\mu}') \neq {\mu}'$ and $ g(f({\mu}'), {\mu}') > \underset{\mu}{\sup} \  g( \mu, {\mu}) \geq  \underset{\mu}{\inf} \  g( \mu, {\mu})$. This implies that seller 1 could fabricate data based on function $f$, this contradicts with the fact that the mechanism is NIC.


pay the seller $v(n^{\star})/2 - \beta (\hat{\mu}(Y_1)-\hat{\mu}(Y_2))^2$, charge buyer $v(n^{\star}) - 2\beta (\hat{\mu}(Y_1)-\hat{\mu}(Y_2))^2$


Buyer utility: $v(n^*)$-payment
Seller utility: payment - $cn^*$.

Sellers utility is positive?

Seller payment $(v(n^*)/2)-\beta (\hat{\mu}(Y_1)-\hat{\mu}(Y_2))^2 $, $\beta = (v(n^*)-cn^*)c(n^*)^2 / 4\sigma^2$, 



\section{Multiple buyers}
\subsection{}
Question 1: Do we fix the amount of data for sale ahead of time?


Assume we fix $N$, the amount of data for sale. The goal of mechanism is to maximize the sellers' revenue. According to previous paper, there exists at least one type who purchase at the amount $N$. Suppose that in offline setting, i.e., when the mechanism knows the buyer valuation and type distribution, the optimal revenue is $\text{OPT} $. 


We ask $d$ sellers to collect $N$ data points, and split $\text{OPT} $ revenue among sellers. (data can be duplicated). 

\[ p_i(M,s) =\mathbb{I}\left( \left| Y_i \right| = \frac{N}{d} \right) \rbr{\frac{\text{OPT}}{d}+d_i \frac{\sigma^2}{N_{-i}^{\star}} +d_i \frac{\sigma^2}{N_i^{\star}} }- d_i \rbr{\hat{\mu}(Y_i)-\hat{\mu}(Y_{-i}) }^2  \]

Buyer's expected utility is non negative. Next, we discuss sellers' expected utility $T\mathbb{E}[p_i]- cn_i$ (over $T$ roundsm\, maybe $T$ is fixed). Let $N_i^{\star} = \frac{N}{d}$.

\[ u_i(M,s) = \mathbb{I}\left( \left| Y_i \right| = \frac{N}{d} \right) \rbr{\frac{\text{OPT}}{d}+d_i \frac{\sigma^2}{N_{-i}^{\star}} +d_i \frac{\sigma^2}{N_i^{\star}} }T- Td_i \mathbb{E}\rbr{\hat{\mu}(Y_i)-\hat{\mu}(Y_{-i}) }^2 
 -  cn_i \]
Choose $d_i = \frac{c(N_i^{\star})^2}{T\sigma^2}$.


If we do not fix \( T \) in advance, let \( T_0 \) represent the time at which the cumulative utility over at least \( T_0 \) rounds is non-negative. We can select \( d_i \) such that \( d_i \geq \frac{c(N/d)^2}{T_0 \sigma^2} \). This ensures that the seller will never choose to collect less than \( N/d \) amount of data.








\section{Single buyer} \label{section: singlebuyer}


Each contributor \( i \) incurs a cost \( c_i \) to collect data,  without loss of generality, we assume \( c_1 \leq c_2 \leq \dots \leq c_d \). The broker is assumed to have full knowledge of the buyer's valuation curve \( \val(n) \), as well as the contributors’ costs $ c_{ i \in \contributors}$  for collecting each data point.

The maximum total profit for the contributors, assuming no constraints on truthful submissions, is given by:
\[
\mathrm{profit}^\star = \underset{\datanum_1, \dots, \datanum_d}{\max} \left( \val \left(\sum_{i=1}^{d} \datanum_i\right) - \sum_{i=1}^{d} c_i \datanum_i \right),
\]

where \( \datanum_i \) represents the number of data points collected by contributor \( i \). In this unconstrained scenario, since contributor 1 has the lowest collection cost, the optimal strategy is for contributor 1 to collect all the required data points while other contributors collect none. This approach maximizes total profit without considering the incentive for truthful submissions.

However, when truthful submission is taken into account, at least two contributors are needed because we need to use one contributor's data to verify the other's. We demonstrate that the maximum profit achievable under Nash Equilibrium is:
\[
\mathrm{profit}^\star + (c_1 - c_2),
\]
where \( c_1 - c_2 \) represents the additional cost differential caused by enforcing truthful behavior among contributors. 

\begin{algorithm}[H]
    \caption{Process of mechanism.}
    \begin{algorithmic}
        \STATE {\bfseries Input:} A population of buyers $\buyers$.
        \STATE The broker chooses the optimal data allocation to maximize contributors' profit:
        $$
        \{ \datanum_i^{\star} \}_{i=1}^d = \underset{\datanum_1,\dots,\datanum_d}{\arg\max}\  \rbr{v\rbr{\sum_i \datanum_i}-\sum_i \cost_i \datanum_i  }
      $$
       
        \STATE The broker recommend a strategy to each contributor: $\strategy_i^{\star} = (\datanum_i^{\star}, \mathbf{I})$.
        \STATE Each contributor selects a strategy $\strati = (\datanum_i, f_i)$, collects $\datanum_i$ data points $X_i$, and submits $Y_i = f_i(X_i)$.
        \STATE The mechanism generates an estimator $\hat{\mu}(M,\strategy)$ for the buyer, and charge her $\price_{j \in \buyers}$. \COMMENT{See (\ref{eq:buyer_pay}) }
        \STATE Each contributor is paid $\payi$.    \COMMENT{See (\ref{eq:seller_pay}) }
    \end{algorithmic}   
\end{algorithm}




\begin{theorem}
    there exists NIC mechanism satisfying the following properties (1) $\strategy^{\star}$ is Nash equilibrium. (2) The mechanism is individually rational at $\strategy^{\star}$ for both buyers and sellers. (3) Budget balance. (4) Under strategy $\strategy^{\star}$, the expected profit of buyers approximates the optimal profit $ \mathrm{profit}^{\star}$ within an additive error $\cost_2 - \cost_1$. 
\end{theorem}


Let $n^{\star}$ denote the optimal total number of data to be collected, $\datanum_1^{\star}=n^{\star}-1$, and $\datanum_2^{\star}=1$. Let $w=\val(n^{\star})-cn^{\star}$ denote the social welfare. One option for payment function is

\begin{align*}
    &\; \pay_i(M,\strategy^{\star}) \\  
    = & \;\mathbb{I}\left( \left| Y_i \right| = \datanum_i^{\star} \right) \rbr{\frac{\datanum_1^{\star}}{n^{\star}}\val(n^{\star})+d_i \frac{\sigma^2}{\datanum_{-i}^{\star}} +d_i \frac{\sigma^2}{\datanum_i^{\star}} } \\ & - d_i \rbr{\hat{\mu}(Y_i)-\hat{\mu}(Y_{-i}) }^2, \\[20pt] % Adds vertical space between equations
    &\; \price(M,\strategy^{\star}) \\  
    = & \; \sum_{i=1}^{2}\mathbb{I}\left(  \left| Y_i \right| = \datanum_i^{\star} \right) \rbr{\frac{\datanum_1^{\star}}{n^{\star}}\val(n^{\star}) +d_i \frac{\sigma^2}{\datanum_{-i}^{\star}} +d_i \frac{\sigma^2}{\datanum_i^{\star}} } \\ 
    & - \sum_{i=1}^{2} d_i \rbr{\hat{\mu}(Y_i)-\hat{\mu}(Y_{-i}) }^2, \\[20pt] % Adds vertical space between equations
    &\;  \utilityb (M,\strategy^{\star}) \\ 
   = & \; v(\datanum^{\star}) -\mathbb{E}[\price(M,\strategy^{\star})] \\ 
    = & \; - \sum_{i=1}^{d} \rbr{d_i \frac{\sigma^2}{\datanum_{-i}^{\star}} +d_i \frac{\sigma^2}{\datanum_i^{\star}}  } + \sum_{i=1}^{d}d_i \mathbb{E} \rbr{\hat{\mu}(Y_i)-\hat{\mu}(Y_{-i}) }^2 \\ 
    = & \; 0.
\end{align*}



Then contributors i's expected ptofit under strategy $\strategy^{\star}$ is 

\begin{align*}
& \; \utilci \rbr{\mechspace, \strategy^{\star} } \\ = &  \; \mathbb{E}\sbr{\pay_i(M,\strategy^{\star})} - \cost_i n_i^{\star} \\ = &  \; \mathbb{I}\left(\left| Y_i \right| = \datanum_i^{\star} \right) \rbr{\frac{\datanum_1^{\star}}{n^{\star}}\val(n^{\star})  +d_i \frac{\sigma^2}{\datanum_{-i}^{\star}} +d_i \frac{\sigma^2}{\datanum_i^{\star}} }\\  & -  d_i \mathbb{E}\rbr{\hat{\mu}(Y_i)-\hat{\mu}(Y_{-i}) }^2  -\cost n_i^{\star} \\ = &  \;  \frac{w}{\numcontributors}
\end{align*}
where $d_i = c(\datanum_i^*/d)^2 $, 

%\textcolor{red}{Buyer payment $\pi(M,s)$ can ve negative? Can it be interpreted as when the quality of data is bad, the mechanism pays money to the buyer as compensate, the contributor pays money to the mechanism as a penalty. }\textcolor{red}{Buyer pays $v(n^*)$, $p_i = v(n^*) \frac{\rbr{\hat{\mu}(Y_i)-\hat{\mu}(Y_{-i}) }^{-2}}{\sum{\rbr{\hat{\mu}(Y_i)-\hat{\mu}(Y_{-i}) }^{-2}}}$ }


We prove the NIC in three steps.


\textbf{First step} \textcolor{red}{to be fixed}: Giving others submitting truthfully, we know that when fixing $n_i$, submitting $\left| Y_i \right| = \frac{n^{\star}}{d}$ is the best strategy, otherwise, $\pay_i <0$ when $\left| Y_i \right| \neq \frac{n^{\star}}{d}$. Therefore, for any $n_i$ and $f_i$, we have for any $\mu$,
\begin{align*}
& u_i\rbr{\mechspace, (n_i,f_i, \left| Y_i \right| =\datanum_i^{\star}),\strategy_{-i}^{\star} } \\  \geq \  & u_i\rbr{\mechspace, \strategy_{-i}^{\star}} 
\end{align*}

\textbf{Second step}: Fixing $n_i$ and $\left| Y_i \right|$, sample mean $\hat{\mu}(X_i)$ is minimax estimator of $\mathbb{E}\rbr{\rbr{\hat{\mu}(Y_i)-\hat{\mu}(Y_{-i})}^2 \;\middle|\; P} $, i.e., \[ \hat{\mu}(X_i) = \underset{\hat{\mu}}{\inf} \  \underset{\distrifamily}{\sup}\ \mathbb{E}\sbr{\rbr{\hat{\mu}(Y_i)-\hat{\mu}(Y_{-i})}^2  \;\middle|\; \distri} \]Therefore we have 
\begin{align*}
     & \underset{\distrifamily}{\inf}\;u_i\rbr{\mechspace, (n_i,\hat{\mu}(Y_i)=\hat{\mu}(X_i), \left| Y_i \right|),\strategy_{-i}^{\star} } \\  \geq \  & \underset{\distrifamily}{\inf}\;u_i\rbr{\mechspace, (n_i,\hat{\mu}(Y_i), \left| Y_i \right|),\strategy_{-i}^{\star} } 
\end{align*}


\textbf{Third step}: By setting constant $d_i= c\rbr{\frac{\datanum_i^{\star}}{\sigma}}^2$, when fixing $\hat{\mu}(Y_i)=\hat{\mu}(X_i)$ and $\left| Y_i \right| = \datanum_i^{\star}$, collecting $n_i = \datanum_i^{\star}$ amount of data maximize the contributor utility 
\begin{align*}
    \underset{\distrifamily}{\inf}\;u_i\rbr{\mechspace, (n_i= \datanum_i^{\star},\hat{\mu}(Y_i)=\hat{\mu}(X_i), \left| Y_i \right|=\strategy_{-i}^{\star} } \\ \geq \underset{\distrifamily}{\inf}\;u_i\rbr{\mechspace, (n_i,\hat{\mu}(Y_i)=\hat{\mu}(X_i), \left| Y_i \right|=s_{-i}^{\star}}  
\end{align*}
 

\begin{align*}
    & u_i\rbr{\mechspace, (n_i,\hat{\mu}(Y_i)=\hat{\mu}(X_i), \left| Y_i \right|=\datanum_i^{\star}),\strategy_{-i}^{\star} }  \\ = & \rbr{\frac{w}{\numcontributors}+ c \datanum_i^{\star} +d_i \frac{\sigma^2}{\datanum_{-i}^{\star}} +d_i \frac{\sigma^2}{\datanum_i^{\star}} } - d_i \rbr{ \frac{\sigma^2}{\datanum_{-i}^{\star}} +\frac{\sigma^2}{\datanum_i^{\star}} } - cn_i
\end{align*}

Therefore, we have 
\begin{align*}
    &\underset{\distrifamily}{\inf}\;u_i \rbr{\mechspace, (n_i= \datanum_i^{\star},\hat{\mu}(Y_i)=\hat{\mu}(X_i), \left| Y_i \right|=\datanum_i^{\star}),\strategy_{-i}^{\star} } \\  = & \underset{\distrifamily}{\inf}\;u_i\rbr{\mechspace, (\datanum_i^{\star},f_i^{\star}),\strategy_{-i}^{\star} } \\ \geq &   \underset{\distrifamily}{\inf}\; u_i\rbr{\mechspace, (n_i,f_i ),\strategy_{-i}^{\star}} 
\end{align*}

When following the best strategy, properties 1-5 are all satisfied.




\paragraph{Analysis on the impact of interaction history.}
% Since interaction history plays a critical role
To investigate the impact of interaction history on LLM performance, we remove the interaction history from the inputs and provide only the user instructions with candidate tools set to conduct our experiments. The results are presented in Figure~\ref{fig:scores_wo_history}.
From the results, we can observe that both closed-source and open-source LLMs experience varying degrees of performance degradation without interaction history, compared to when provided with preferred-only history. This suggests that interaction history only containing the user's preferred tools can help the LLM effectively infer user preferences. On the other hand, we find that LLMs perform better in the absence of interaction history than with chronological history. This indicates that including both preferred and non-preferred tools can interfere with the LLM's understanding of user preferences, thus hindering its personalization capabilities.
In contrast, our \framework consistently improves performance across all three types of interaction history compared to the no-history setting. This demonstrates that our method enables LLM to effectively recognize different forms of user preferences from the interaction history.


\paragraph{Analysis on interaction history length.}
To evaluate the performance of LLMs under varying interaction history lengths, we break down the tool accuracy scores of LLMs based on the number of interactions in the history under the preferred-only setting.
As shown in Figure~\ref{fig:scores_length}, the performance of both closed-source and open-source LLMs deteriorates as interaction history length increases.
This is because a longer interaction history makes it more challenging for the LLM to identify the historical preferences relevant to identify relevant historical preferences in relation to the user’s current context.
In contrast, our \framework significantly outperforms all LLMs and maintains strong, consistent performance even as interaction history grows. This demonstrates that our method enables LLMs to effectively extract and utilize user preferences from complex historical data.


\subsection{Error Analysis}
We further conduct an error analysis to investigate the issues leading to incorrect tool calls in two personalized settings. 
We categorize the errors into six types:
1) Invalid Format. The tool call generated by the LLMs does not follow the expected JSON format. 
2) Tool Hallucination. The LLM generates a tool that does not exist in the given candidate tool set, which is a common hallucination issue in LLMs.
% ~\cite{huang2024survey}.
3) Tool Functionality Mismatch. The selected tool lacks the necessary functionality to fulfill the user’s requirements.
4) Tool Preference Mismatch. The selected tool has the correct functionality but is not preferred by the user.
5) Parameter Name Mismatch. The tool call contains missing or incorrect parameter names.
6) Parameter Value Mismatch. The parameter names are correctly generated, but the parameter values do not match the ground truth.

From the results in Table~\ref{error_results}, we observe that most LLMs perform worst in Tool Preference Mismatch, particularly in the chronological setting, where the error rate exceeds 50\%. This suggests that identifying user preferences from the interaction history is highly challenging, especially when preferences change over time, leading to significant model misinterpretation. In contrast, our \framework significantly reduces the error rate in Tool Preference Mismatch, demonstrating its effectiveness in capturing implicit user preferences. Additionally, the reduction in Tool Functionality Mismatch and Parameter Value Mismatch errors suggests that our method enhances LLMs' fundamental tool-usage ability, improving their handling of explicit user requirements.
Furthermore, \framework achieves low error rates in Invalid Format and Tool Hallucination, comparable to closed-source LLMs, highlighting its strong instruction-following capabilities.

\section{Conclusion and Future Work}
In this paper, we advanced general-purpose tool-use LLMs into personalized tool-use LLMs, aiming to provide users with customized tool-usage assistance. We formulate the task of personalized tool learning and identify the goal of leveraging user's interaction history to achieve implicit preference understanding and personalized tool calling. 
For training and evaluation, we construct the first \benchmark benchmark, featuring diverse users’ interaction history in three types. 
We also propose a novel personalized framework \framework conducted under a two-stage training process to endow LLMs with personalized tool-use capabilities.
Extensive experiments on \benchmark demonstrate that \framework consistently surpasses existing baselines, effectively meeting user requirements and preferences.
We believe that the task, benchmark, and framework for personalized tool learning will broaden the research scope, introduce new challenges and inspire novel methods. 

In the future, we aim to enhance this work from the following dimensions.
1) We plan to explore more heterogeneous personal user data beyond interaction history, such as user profiles or personas. This will allow us to reflect user preferences from multiple dimensions, providing a more comprehensive evaluation on the personalized tool-use capabilities of LLMs.
2) Currently, our work is limited to tool-usage scenarios involving a single tool. In the future, we intend to expand to more complex personalized tool-usage, such as multi-tool scenarios. These scenarios will require LLMs to perform personalized tool planning and engage in multi-round tool calling to address user needs effectively.

\section*{Limitations}
1) Due to the lack of real user interaction histories on tool usage, we utilize LLM to synthesize such data. However, this approach may compromise the authenticity and reliability of the data, which is a common challenge in data synthesis methods.
To mitigate this issue, we incorporate pre-constructed user preference information into the data generation process. This strategy helps guide LLM in generating contextually relevant outputs, thereby improving the quality and consistency of the synthesized data.
2) In real-world scenarios, tools have multiple dimensions of attributes. However, due to the limited information contained in tool documentation, it is difficult to identify and fully exploit all possible tool attributes. Fortunately, the attributes we have obtained are sufficient to differentiate between tools, enabling us to effectively construct user preferences.

\section*{Ethics Statement}
The dataset used in our work is derived from publicly available sources and generated through interactions with LLMs in English. Since the user interaction histories in our study are entirely simulated, user privacy is fully protected, and no real personal information is included in the dataset. Furthermore, all scientific artifacts used in this research are publicly accessible for academic purposes under permissive licenses, and their use in this paper complies with their intended purposes. Given these considerations, we believe our research adheres to the ethical standards of the conference.

% Taking a query $q$ and a user profile $U$ as input, the model is expected to select a user-specific tool set $D_u=\{d_i\}_{i=1}^k$ from 
% Formally, we design a number of user profiles $\{u_1, u_2, ..., u_m\}$, each of which represents a specific user.
% associate the query $q$ with $u_i$, where $u_i$ is a user-specific profile from a number of users.
% $\{u_1, u_2, ..., d_N\}$
% % ${u_i|i\in (1,M)}$ 
% where $M$ is the number of users.

% Given a user's instruction, tool learning aims to select a small number of tools, which could aid the LLM in answering the instruction, from a large-scale tool set and then .

% Given an input query $q$, tool learning model first selects from a large-scale tool set $D=\{d_1, d_2, ..., d_N\}$ a small number of tools $D(q)=\{d_1, d_2, ..., d_K\}$ which could help solve the query $q$, and then output a final response $r$ based on a tool-use trajectory $T=\{t(d_i)|d_i\in D(q)\}$ containing the tool execution results after calling each selected tools with parameters.


% In contrast, personalized tool learning aims to solve the query given by different users.
% It can be formulated as conditioning the tool learning process on $(q,u)$, where $u$ is a user-specific profile.
%  the model's output on a user

% Personalized tool learning can be formulated as conditioning the model's output on a user- $u$, represented by a user profile.



% In tool learning, a typical data entry consists of three components: an input query $q$, a tool set $D=\{d_1, d_2, ..., d_N\}$ containing $N$ ground-truth tools solving the query $q$,  
% % where $d_i$ represents the description of each tool and $N$ is the total number of tools, 
% and a tool-use trajectory $Traj=$.


% sequence $x$ that serves as the model's input, a target output $y$ that the model is expected to produce, and a profile $P_u$ that encapsulates any auxiliary information that can be used to personalize the model for the user.


% Given a user's query $q$, tool learning aims to select a small number of tools, which could aid the LLM in answering the instruction, from a large-scale tool set.
% Formally, we define the user instruction as $q$ and the tool set as $D=\{d_1, d_2, ..., d_N\}$, where $d_i$ represents the description of each tool and $N$ is the total number of tools.
% The retriever model $R$ needs to measure the relevance $R(q, d_i)$ 
% between the instruction $q$ and each tool description $d_i$, and return $K$ tools, denoted as $D=\{d_1, d_2, ..., d_K\}$.


% Generative language models often take an input $x$ and predict the most probable sequence tokens $y$ that follows $x$. 
% Tool learning can be formulated as xxx
% Personalized tool learning can be formulated as conditioning the model's output on a user $u$, represented by a user profile.



% \section{Introduction}
% These instructions are for authors submitting papers to *ACL conferences using \LaTeX. They are not self-contained. All authors must follow the general instructions for *ACL proceedings,\footnote{\url{http://acl-org.github.io/ACLPUB/formatting.html}} and this document contains additional instructions for the \LaTeX{} style files.

% The templates include the \LaTeX{} source of this document (\texttt{acl\_latex.tex}),
% the \LaTeX{} style file used to format it (\texttt{acl.sty}),
% an ACL bibliography style (\texttt{acl\_natbib.bst}),
% an example bibliography (\texttt{custom.bib}),
% and the bibliography for the ACL Anthology (\texttt{anthology.bib}).

% \section{Engines}

% To produce a PDF file, pdf\LaTeX{} is strongly recommended (over original \LaTeX{} plus dvips+ps2pdf or dvipdf). Xe\LaTeX{} also produces PDF files, and is especially suitable for text in non-Latin scripts.

% \section{Preamble}

% The first line of the file must be
% \begin{quote}
% \begin{verbatim}
% \documentclass[11pt]{article}
% \end{verbatim}
% \end{quote}

% To load the style file in the review version:
% \begin{quote}
% \begin{verbatim}
% \usepackage[review]{acl}
% \end{verbatim}
% \end{quote}
% For the final version, omit the \verb|review| option:
% \begin{quote}
% \begin{verbatim}
% \usepackage{acl}
% \end{verbatim}
% \end{quote}

% To use Times Roman, put the following in the preamble:
% \begin{quote}
% \begin{verbatim}
% \usepackage{times}
% \end{verbatim}
% \end{quote}
% (Alternatives like txfonts or newtx are also acceptable.)

% Please see the \LaTeX{} source of this document for comments on other packages that may be useful.

% Set the title and author using \verb|\title| and \verb|\author|. Within the author list, format multiple authors using \verb|\and| and \verb|\And| and \verb|\AND|; please see the \LaTeX{} source for examples.

% By default, the box containing the title and author names is set to the minimum of 5 cm. If you need more space, include the following in the preamble:
% \begin{quote}
% \begin{verbatim}
% \setlength\titlebox{<dim>}
% \end{verbatim}
% \end{quote}
% where \verb|<dim>| is replaced with a length. Do not set this length smaller than 5 cm.

% \section{Document Body}

% \subsection{Footnotes}

% Footnotes are inserted with the \verb|\footnote| command.\footnote{This is a footnote.}

% \subsection{Tables and figures}

% See Table~\ref{tab:accents} for an example of a table and its caption.
% \textbf{Do not override the default caption sizes.}

% \begin{table}
%   \centering
%   \begin{tabular}{lc}
%     \hline
%     \textbf{Command} & \textbf{Output} \\
%     \hline
%     \verb|{\"a}|     & {\"a}           \\
%     \verb|{\^e}|     & {\^e}           \\
%     \verb|{\`i}|     & {\`i}           \\
%     \verb|{\.I}|     & {\.I}           \\
%     \verb|{\o}|      & {\o}            \\
%     \verb|{\'u}|     & {\'u}           \\
%     \verb|{\aa}|     & {\aa}           \\\hline
%   \end{tabular}
%   \begin{tabular}{lc}
%     \hline
%     \textbf{Command} & \textbf{Output} \\
%     \hline
%     \verb|{\c c}|    & {\c c}          \\
%     \verb|{\u g}|    & {\u g}          \\
%     \verb|{\l}|      & {\l}            \\
%     \verb|{\~n}|     & {\~n}           \\
%     \verb|{\H o}|    & {\H o}          \\
%     \verb|{\v r}|    & {\v r}          \\
%     \verb|{\ss}|     & {\ss}           \\
%     \hline
%   \end{tabular}
%   \caption{Example commands for accented characters, to be used in, \emph{e.g.}, Bib\TeX{} entries.}
%   \label{tab:accents}
% \end{table}

% As much as possible, fonts in figures should conform
% to the document fonts. See Figure~\ref{fig:experiments} for an example of a figure and its caption.

% Using the \verb|graphicx| package graphics files can be included within figure
% environment at an appropriate point within the text.
% The \verb|graphicx| package supports various optional arguments to control the
% appearance of the figure.
% You must include it explicitly in the \LaTeX{} preamble (after the
% \verb|\documentclass| declaration and before \verb|\begin{document}|) using
% \verb|\usepackage{graphicx}|.

% \begin{figure}[t]
%   \includegraphics[width=\columnwidth]{example-image-golden}
%   \caption{A figure with a caption that runs for more than one line.
%     Example image is usually available through the \texttt{mwe} package
%     without even mentioning it in the preamble.}
%   \label{fig:experiments}
% \end{figure}

% \begin{figure*}[t]
%   \includegraphics[width=0.48\linewidth]{example-image-a} \hfill
%   \includegraphics[width=0.48\linewidth]{example-image-b}
%   \caption {A minimal working example to demonstrate how to place
%     two images side-by-side.}
% \end{figure*}

% \subsection{Hyperlinks}

% Users of older versions of \LaTeX{} may encounter the following error during compilation:
% \begin{quote}
% \verb|\pdfendlink| ended up in different nesting level than \verb|\pdfstartlink|.
% \end{quote}
% This happens when pdf\LaTeX{} is used and a citation splits across a page boundary. The best way to fix this is to upgrade \LaTeX{} to 2018-12-01 or later.

% \subsection{Citations}

% \begin{table*}
%   \centering
%   \begin{tabular}{lll}
%     \hline
%     \textbf{Output}           & \textbf{natbib command} & \textbf{ACL only command} \\
%     \hline
%     \citep{Gusfield:97}       & \verb|\citep|           &                           \\
%     \citealp{Gusfield:97}     & \verb|\citealp|         &                           \\
%     \citet{Gusfield:97}       & \verb|\citet|           &                           \\
%     \citeyearpar{Gusfield:97} & \verb|\citeyearpar|     &                           \\
%     \citeposs{Gusfield:97}    &                         & \verb|\citeposs|          \\
%     \hline
%   \end{tabular}
%   \caption{\label{citation-guide}
%     Citation commands supported by the style file.
%     The style is based on the natbib package and supports all natbib citation commands.
%     It also supports commands defined in previous ACL style files for compatibility.
%   }
% \end{table*}

% Table~\ref{citation-guide} shows the syntax supported by the style files.
% We encourage you to use the natbib styles.
% You can use the command \verb|\citet| (cite in text) to get ``author (year)'' citations, like this citation to a paper by \citet{Gusfield:97}.
% You can use the command \verb|\citep| (cite in parentheses) to get ``(author, year)'' citations \citep{Gusfield:97}.
% You can use the command \verb|\citealp| (alternative cite without parentheses) to get ``author, year'' citations, which is useful for using citations within parentheses (e.g. \citealp{Gusfield:97}).

% A possessive citation can be made with the command \verb|\citeposs|.
% This is not a standard natbib command, so it is generally not compatible
% with other style files.

% \subsection{References}

% \nocite{Ando2005,andrew2007scalable,rasooli-tetrault-2015}

% The \LaTeX{} and Bib\TeX{} style files provided roughly follow the American Psychological Association format.
% If your own bib file is named \texttt{custom.bib}, then placing the following before any appendices in your \LaTeX{} file will generate the references section for you:
% \begin{quote}
% \begin{verbatim}
% \bibliography{custom}
% \end{verbatim}
% \end{quote}

% You can obtain the complete ACL Anthology as a Bib\TeX{} file from \url{https://aclweb.org/anthology/anthology.bib.gz}.
% To include both the Anthology and your own .bib file, use the following instead of the above.
% \begin{quote}
% \begin{verbatim}
% \bibliography{anthology,custom}
% \end{verbatim}
% \end{quote}

% Please see Section~\ref{sec:bibtex} for information on preparing Bib\TeX{} files.

% \subsection{Equations}

% An example equation is shown below:
% \begin{equation}
%   \label{eq:example}
%   A = \pi r^2
% \end{equation}

% Labels for equation numbers, sections, subsections, figures and tables
% are all defined with the \verb|\label{label}| command and cross references
% to them are made with the \verb|\ref{label}| command.

% This an example cross-reference to Equation~\ref{eq:example}.

% \subsection{Appendices}

% Use \verb|\appendix| before any appendix section to switch the section numbering over to letters. See Appendix~\ref{sec:appendix} for an example.

% \section{Bib\TeX{} Files}
% \label{sec:bibtex}

% Unicode cannot be used in Bib\TeX{} entries, and some ways of typing special characters can disrupt Bib\TeX's alphabetization. The recommended way of typing special characters is shown in Table~\ref{tab:accents}.

% Please ensure that Bib\TeX{} records contain DOIs or URLs when possible, and for all the ACL materials that you reference.
% Use the \verb|doi| field for DOIs and the \verb|url| field for URLs.
% If a Bib\TeX{} entry has a URL or DOI field, the paper title in the references section will appear as a hyperlink to the paper, using the hyperref \LaTeX{} package.

% \section*{Acknowledgments}

% This document has been adapted
% by Steven Bethard, Ryan Cotterell and Rui Yan
% from the instructions for earlier ACL and NAACL proceedings, including those for
% ACL 2019 by Douwe Kiela and Ivan Vuli\'{c},
% NAACL 2019 by Stephanie Lukin and Alla Roskovskaya,
% ACL 2018 by Shay Cohen, Kevin Gimpel, and Wei Lu,
% NAACL 2018 by Margaret Mitchell and Stephanie Lukin,
% Bib\TeX{} suggestions for (NA)ACL 2017/2018 from Jason Eisner,
% ACL 2017 by Dan Gildea and Min-Yen Kan,
% NAACL 2017 by Margaret Mitchell,
% ACL 2012 by Maggie Li and Michael White,
% ACL 2010 by Jing-Shin Chang and Philipp Koehn,
% ACL 2008 by Johanna D. Moore, Simone Teufel, James Allan, and Sadaoki Furui,
% ACL 2005 by Hwee Tou Ng and Kemal Oflazer,
% ACL 2002 by Eugene Charniak and Dekang Lin,
% and earlier ACL and EACL formats written by several people, including
% John Chen, Henry S. Thompson and Donald Walker.
% Additional elements were taken from the formatting instructions of the \emph{International Joint Conference on Artificial Intelligence} and the \emph{Conference on Computer Vision and Pattern Recognition}.

% Bibliography entries for the entire Anthology, followed by custom entries
% \bibliography{anthology,custom}
% Custom bibliography entries only
\bibliography{acl_latex}

\appendix

\section{Details of Benchmark Construction}
We provide the illustration of three types of the user's interaction history in Figure~\ref{fig:history}.
\begin{figure}[htbp]
    \centering
    \includegraphics[width=1.0\columnwidth]{fig_history.pdf}
    \caption{Illustration of three types of the user's interaction history.}
    \label{fig:history}
\vspace{-1em}
\end{figure}

\section{Implementation details}
To train \framework, we fine-tune the LLaMA-3.1-8B model with LoRA
% () 
and a warm-up ratio of $0.1$ in the SFT stage. 
The learning rate is set to $1e{-4}$ with a batch size of $16$ per GPU. 
In the DPO stage, the learning rate is set to $1e{-6}$ and the balancing factor $\beta$ is set to $0.1$ with a batch size of $32$.
We have trained the model several times to ensure the improvement is not randomly achieved and present the mid one. 
For evaluation, we set the number of candidate tools $N$ to $10$ and the temperature to 0.1 to reduce randomness. 
Since the maximum context length varies in different LLMs, we constrain the context window to 4000 tokens. The experiments on closed-source LLMs are fulfilled by APIs of OpenAI and those on open-source LLMs are conducted on NVIDIA A6000 GPUs with 48 GB of memory. 

\definecolor{lightgray}{RGB}{240, 240, 240}
% \definecolor{lightgray}{gray}{0.95}
\lstdefinestyle{prompt}{
    basicstyle=\ttfamily\fontsize{7pt}{8pt}\selectfont,
    frame=none,
    breaklines=true,
    backgroundcolor=\color{lightgray},
    breakatwhitespace=true,
    breakindent=0pt,
    escapeinside={(*@}{@*)},
    numbers=none,
    numbersep=5pt,
    xleftmargin=5pt,
}
\tcbset{
  aibox/.style={
    % width=220pt,
    % top=10pt,
    % colback=lightgray,
    % colframe=black,
    % colbacktitle=black,
    % enhanced,
    % center,
    % breakable,
    % attach boxed title to top left={yshift=-0.1in,xshift=0.15in},
    % boxed title style={boxrule=0pt,colframe=white,},
  }
}
\newtcolorbox{AIbox}[2][]{aibox, title=#2,#1}


\section{Prompt Details}
The prompt templates in for tool-use example generation and tool attributes understanding are shown in Figure~\ref{fig:prompt_example} and Figure~\ref{fig:prompt_attributes}. The prompt templates for interaction history generation across three types are shown in Figure~\ref{fig:prompt_history(p)}, Figure~\ref{fig:prompt_history(r)}, and Figure~\ref{fig:prompt_history(c)}.
The prompt template for instruction generation is shown in Figure~\ref{fig:prompt_instruction}.

\begin{figure*}[!ht] 
\vspace{-5mm}
\begin{AIbox}{Prompt for Tool-use Example Generation}
{\bf Prompt:} \\
{
Given a tool documentation as input, your task is to output an example for using this tool, including a simulated user instruction and parameters for calling the tool. The output example should be in JSON format: \{``instruction'': xx, ``parameters'': xx\}
\clearpage
Here is a demonstration:

Input:
\begin{lstlisting}[style=prompt]
{
    "tool_name": "<Text_Analysis>.<Spellout>.<Languages>",
    "tool_desciption": "List ISO 639 languages",
    "required_parameters": [],
    "optional_parameters": [
        {
            "name": "nameFilter",
            "type": "STRING",
            "description": "Filter as \"contains\" by language name",
            "default": ""
        }
    ]
}
\end{lstlisting}
Output:
\begin{lstlisting}[style=prompt]
{
    "instruction": "I want to filter the list of languages by English",
    "parameters": {
        "nameFilter": "English"
    }
}
\end{lstlisting}
Now you will be given the tool documentation, please generate the tool-use example. 

Begin!
}
\end{AIbox} 
\caption{The prompt for tool-use example generation.}
\label{fig:prompt_example}
\vspace{-5mm}
\end{figure*}



\begin{figure*}[!ht] 
\vspace{-5mm}
\begin{AIbox}{Prompt for Tool Attributes Understanding}
{\bf Prompt:} \\
{
Given a tool documentation and the corresponding tool-use example as input, your task is to understand the tool attributes thoroughly. Then generate two descriptions about the functionality and non-functional attributes of the tool respectively. 
% The functionality refers to the core function that the tool performs in order to fulfill its purpose. 
% Non-functional attributes refer to additional attributes beyond functionality that can reflect different characteristics of the tool and result in different user experience, such as usability, integrability, accessibility, and security.
\clearpage
Here is a demonstration:

Input:
\begin{lstlisting}[style=prompt]
Tool documentation:
{
    "tool_name": "<Commerce>.<Face Compare>.<GET Call>",
    "tool_desciption": "Used to fetch results using the request id received in responses.",
    "required_parameters": [
        {
            "name": "request_id",
            "type": "STRING",
            "description": "",
            "default": "76d1c748-51ed-435b-bcd8-3d9c9d3eb68a"
        }
    ],
Tool-use example:
{
    "instruction": "I want to use the request id '76d1c748-51ed-435b-bcd8-3d9c9d3eb68a' to fetch the result",
    "parameters": {
        "request_id": "76d1c748-51ed-435b-bcd8-3d9c9d3eb68a"
    }
}
\end{lstlisting}
Output:
\begin{lstlisting}[style=prompt]
Functionality: Fetches API results based on the request ID received in previous responses.
Non-functional attributes: Designed for commerce applications, used in face comparison scenarios.
\end{lstlisting}
Now you will be given the tool documentation and the tool-use example, generate two short phrases to describe the two types of attributes. 

Begin!
}
\end{AIbox} 
\caption{The prompt for tool attributes understanding.}
\label{fig:prompt_attributes}
\vspace{-5mm}
\end{figure*}


\begin{figure*}[!ht] 
\vspace{-5mm}
\begin{AIbox}{Prompt for Interaction History (Preferred-only) Generation}
{\bf Prompt:} \\
{
Given a list of tools preferred by a user as input, your task is to simulate the user's interaction history based on these tools. You should output a sequence of tool-usage interactions, each consisting of a simulated user instruction and a tool call to fulfill that instruction. The interaction sequence should be a list in JSON format: 
% \clearpage
% Here is a demonstration:
% Input:
\begin{lstlisting}[style=prompt]
[
    {
        "instruction": xx,
        "tool_call": {
            "tool_name": xx, 
            "parameters": xx
        }
    }, ...
]
\end{lstlisting}
Now you will be given the tools, please generate the interaction sequence. 

Begin!
}
\end{AIbox} 
\caption{The prompt for interaction history (preferred-only) generation.}
\label{fig:prompt_history(p)}
\vspace{-5mm}
\end{figure*}


\begin{figure*}[!ht] 
\vspace{-5mm}
\begin{AIbox}{Prompt for Interaction History (Rating-integrated) Generation}
{\bf Prompt:} \\
{
Given a list of tools preferred by a user and a list of tools not preferred as input, your task is to simulate the user's interaction history based on these two lists. You should output a sequence of tool-usage interactions, each consisting of a simulated user instruction, a tool call to fulfill that instruction, and a binary rating reflecting the user's satisfaction with the tool call. The interaction sequence should be a list in JSON format: 
\begin{lstlisting}[style=prompt]
[
    {
        "instruction": xx,
        "tool_call": {
            "tool_name": xx, 
            "parameters": xx
        },
        "rating": 1 or 0,
    }, ...
]
\end{lstlisting}
Now you will be given the two lists of tools, please generate the interaction sequence. 

Begin!
}
\end{AIbox} 
\caption{The prompt for interaction history (rating-integrated) generation.}
\label{fig:prompt_history(r)}
\vspace{-5mm}
\end{figure*}


\begin{figure*}[!ht] 
\vspace{-5mm}
\begin{AIbox}{Prompt for Interaction History (Chronological) Generation}
{\bf Prompt:} \\
{
Given a list of tools preferred by a user and a list of tools not preferred as input, your task is to simulate the user's interaction history based on these two lists. You should output a sequence of tool-usage interactions, each consisting of a simulated user instruction, a tool call to fulfill that instruction. The interactions should be organized in time order to reflect changes in user preferences over time, i.e., the more recent tool-usage interactions are more preferred by the user, while earlier interactions are less preferred. The interaction sequence should be a list in JSON format: 
\begin{lstlisting}[style=prompt]
[
    {
        "instruction": xx,
        "tool_call": {
            "tool_name": xx, 
            "parameters": xx
        }
    }, ...
]
\end{lstlisting}
Now you will be given the two lists of tools, please generate the interaction sequence. 

Begin!
}
\end{AIbox} 
\caption{The prompt for interaction history (chronological) generation.}
\label{fig:prompt_history(c)}
\vspace{-5mm}
\end{figure*}


\begin{figure*}[!ht] 
% \vspace{-2mm}
\begin{AIbox}{Prompt for Instruction Generation}
{\bf Prompt:} \\
{
Given a user's interaction history and a tool documentation as input, your task is to generate a simulated user instruction which can be fulfilled by calling the tool with parameters. The generated output should be in JSON format: 
\begin{lstlisting}[style=prompt]
{
    "instruction": xx,
    "parameters": xx
}

\end{lstlisting}
Remember, tool name is strictly prohibited from appearing in the generated instruction. 
Now you will be given the user's interaction history and tool documentation, please generate the output. 

Begin!
}
\end{AIbox} 
\caption{The prompt for instruction generation.}
\label{fig:prompt_instruction}
% \vspace{-5mm}
\end{figure*}


\label{sec:appendix}

% This is an appendix.

\end{document}
