\section{Related Work}
Conventional image captioning evaluation has relied on reference-based assessments, where machine-generated captions are compared against human-generated ones (i.e., the references). 
Frequently used metrics such as BLEU or CIDEr____  rely on lexical matches, and hence may fail to capture fine nuances and semantic overlaps in rich captions. A recent shift in the evaluation paradigm involves the use of learned vision-and-language models to enable evaluation through reference-free metrics.

The CLIPScore metric____ was one of the first proposals for evaluating image captions that departed from traditional metrics. Grounded in a vision-and-language encoder, specifically the original Contrastive Language-Image Pretraining (CLIP) model____, this strategy employs a modified cosine similarity between representations for the input image and the caption under evaluation. CLIPScore exhibits a high correlation with human judgments across various datasets, and despite being a reference-free metric, it even surpasses established reference-based metrics like BLEU and CIDEr. Today, CLIPScore-based metrics are widely used for evaluating image captions, and have inspired the development of numerous new evaluation metrics that build on CLIP____, including reference-based variants.

Despite the advancements, recent studies have noted that CLIPScore can lack rating granularity, emphasising the need for better benchmark datasets to evaluate image captioning metrics____. Datasets like VICR____ or Polaris____ have employed more rigorous methods for collecting human ratings, but remain limited to English. Alongside these datasets, researchers have shown that models specifically trained for image captioning evaluation can slightly outperform CLIPScore. We corroborate these findings by proposing a finetuned version of CLIPScore that surpasses previous variants.

Recent studies emphasize the importance of evaluating visio-linguistic grounding capabilities with more complex linguistic constructs ____. Additionally, the absence of multilingual and multicultural benchmarks has been a key topic of discussion. For instance, ____ proposed a multilingual image captioning evaluation method. They finetuned the CLIP text encoder using a language-agnostic approach to distinguish between original and perturbed text. Moreover, the authors introduced a new dataset aimed at evaluating multilingual captioning metrics, although it has not yet been made publicly available.

The lack of linguistic and cultural diversity in vision-language datasets has been further criticized by____, who argued that relying on English-based lexical databases and image queries introduces a North American and Western European bias, since existing vision-linguistic datasets often fail to capture concepts relevant to non-English languages and cultures outside of Europe and North America. To mitigate this limitation, the authors built MaRVL, which proposes a new protocol for constructing an ImageNet-style class hierarchy that better reflects diverse languages and cultures, together with textual descriptions reflecting these classes. They benchmarked several multilingual multimodal models, but these often performed just above chance, highlighting the challenges of handling out-of-distribution concepts and images. The results call for a reassessment of model robustness and accuracy, and for the development of more inclusive systems.

Similarly, ____ highlighted the importance of multilingual benchmarks, noting that vision-and-language research has primarily focused on English tasks. To address this, the authors proposed a multilingual dataset for evaluating vision-language inference, where models predict entailment relationships between a textual hypothesis and an image premise, expanding the scope of evaluation to include diverse languages.