\section{Theoretical Justifications}\label{sec:proofs}
\begin{prop}
    A data matrix $\D$ admits a flag decomposition of type $(n_1,n_2, \cdots, n_k; n)$ if and only if $\cA_1 \subset \cA_2 \subset \cdots \subset \cA_k$ is a column hierarchy for $\D$.
\end{prop}

\begin{proof}
    We first tackle the forward direction. Suppose $\D = \Q \bR \bP^\top$. Define $\B = [\B_1| \B_2 | \cdots | \B_k] = \D \bP^{\top} = \Q\bR$. Then (for $i=1,2,\dots,k$)
    \begin{equation}
        \B_i = \sum_{j=1}^i \Q_j \bR_{j,i}.
    \end{equation}
    So each $\B_i$ is in the span of the columns of $\Q_1,\Q_2,\dots,\Q_i$ for $i=1,2,\dots,k$. This implies that $[\B_1,\B_2,\dots,\B_i] = [\Q_1, \Q_2, \dots,\Q_i]$. Since $\Q_{i-1}^T\Q_{i} = \bm{0}$, 
    \begin{equation}
        \mathrm{dim}([\Q_1, \Q_2, \dots,\Q_{i-1}]) < \mathrm{dim}([\Q_1, \Q_2, \dots,\Q_i])
    \end{equation} 
    we have 
    \begin{equation}
        \mathrm{dim}([\B_1, \B_2, \dots,\B_{i-1}]) < \mathrm{dim}([\B_1, \B_2, \dots,\B_i]).
    \end{equation}
    Thus $\mathrm{dim}([\D_{\cA_{i-1}}]) < \mathrm{dim}([\D_{\cA_i}])$ proving $\cA_1 \subset \cA_2 \subset \cdots \subset \cA_k$ is a column hierarchy for $\D$.
    
    The backward direction is proved in~\cref{prop:stiefel_coords_app,prop:proj_prop_and_flags_app}.


\end{proof}


% \begin{prop}\label{prop:projection_property}
%     Suppose $\cA_1 \subset \cA_2 \subset \cdots \subset \cA_k$ is a column hierarchy for $\D$. Then there exists a $\Q= [\Q_1\,|\,\Q_2\,|\, \cdots\,|\,\Q_k] \in St(n_k,n)$ that satisfies (for $i=1,2\dots,k$) $[\Q_i] = [\bP_{\Q_{i-1}^\perp}\cdots \bP_{\Q_1^\perp}\B_i]$ and the \textbf{projection property}: 
%     \begin{equation}\label{eq:projection_property}
%     \bP_{\Q_i^\perp}\bP_{\Q_{i-1}^\perp}\cdots \bP_{\Q_1^\perp} \B_i = 0.
%     \end{equation}
% \end{prop}
\begin{prop}\label{prop:stiefel_coords_app}
    Suppose $\cA_1 \subset \cA_2 \subset \cdots \subset \cA_k$ is a column hierarchy for $\D$. Then there exists $\Q= [\Q_1\,|\,\Q_2\,|\, \cdots\,|\,\Q_k]$ that are coordinates for the flag $[\![\Q]\!]\in\flag(n_1,n_2,\dots,n_k;n)$ where $n_i = \mathrm{rank}(\D_{\mathcal{A}_i})$ that satisfies $[\Q_i] = [\bP_{\Q_{i-1}^\perp}\cdots \bP_{\Q_1^\perp}\B_i]$ and the \textbf{projection property} (for $i=1,2\dots,k$): 
    \begin{equation}\label{eq:projection_property_app}
    \bP_{\Q_i^\perp}\bP_{\Q_{i-1}^\perp}\cdots \bP_{\Q_1^\perp} \B_i = 0.
    \end{equation}
\end{prop}
\begin{proof}
    For $i=1$ define $\C_1 = \B_1$ and $\Q_1 \in \R^{n \times m_1}$ whose columns are an orthonormal basis for the column space of $\C_1$, specifically $[\Q_1] = [\C_1]$ and $\Q_1^\top \Q_1 = \I$. By construction, $m_1 = \mathrm{rank}(\B_1)$. Using the projector onto the null space of $[\Q_i]$, denoted $\bP_{\Q_i^\perp} = \I - \Q_i \Q_i^\top$ define $\C_i$ (for $i=2,\dots,k$) through 
    \begin{equation}
        \C_i = \bP_{\Q_{i-1}^\perp}\cdots \bP_{\Q_1^\perp}\B_i
    \end{equation} 
    and $\Q_i \in \R^{n \times m_i}$ so that $[\Q_i] = [\C_i]$ and $\Q_i^\top \Q_i = \I$. By construction, $m_i = \mathrm{rank}(\C_i)$.

    % Now we show $\C_i \neq 0$ for $i=1,\dots,k$. For $i=1$ this is clear because $\C_1 = \B_1 \neq 0$. To prove this for all $i=2,\dots,k$, we define the recursion~\footnote{In the manuscript we call $\T_i = \bP^\prime_{\Q_i^\perp}$. We switch to using $\T_i$ for notational simplicity.}
    % \begin{equation}
    %     \T_i = \Q_i \Q_i^\top + \bP_{\Q_i^\perp} \T_{i-1}
    % \end{equation}
    % with $\T_0 = \mathbf{0}$.

    % We prove via mathematical induction that (for $i=1,2,\dots,k$)
    % \begin{equation}\label{eq: proj_rec_identity}
    %     \I -  \T_i = \mathbf{P}_{\Q_i^\perp} \cdots \mathbf{P}_{\Q_1^\perp}.
    % \end{equation}
    % For $i=1$ 
    % \begin{align}
    %      \I -  \T_{1} &= \I - (\Q_1 \Q_1^T + \mathbf{P}_{\Q_1^\perp} \T_{0}),\\
    %        &= \I - \Q_1 \Q_1^T,\\
    %        &= \mathbf{P}_{\Q_1^\perp}.
    % \end{align}
    % Fix some $j < k$ and suppose~\cref{eq: proj_rec_identity} holds for or all $i < j$. Then
    % \begin{align}
    %     \I -  \T_j &= \I - \Q_j \Q_j^T - \mathbf{P}_{\Q_j^\perp} \T_{j-1},\\
    %      &=\mathbf{P}_{\Q_j^\perp} - \mathbf{P}_{\Q_j^\perp} \T_{j-1},\\
    %     &=\mathbf{P}_{\Q_j^\perp}(\I - \T_{j-1}),\\
    %     &= \mathbf{P}_{\Q_j^\perp} \cdots \mathbf{P}_{\Q_1^\perp}.
    % \end{align}
    
    % Suppose, by way of contradiction that $\C_2 = 0$.
    % Then $ \mathbf{P}_{\Q_{1}^\perp}\B_2 = \bm{0} $. Then $[\B_2] \subseteq [\Q_1] = [\B_1]$ so $\mathrm{dim}([\B_1,\B_2]) = \mathrm{dim}([\B_1])$ which contradicts the assumption  of a column hierarchy for $\D$ that implies $\mathrm{dim}([\B_1,\B_2]) > \mathrm{dim}([\B_1])$.

    % We proceed, once again, by mathematical induction on $i$. Fix some $j < k$ and suppose $\C_i \neq 0$ holds for or all $i < j$. Suppose by way of contradiction that $\C_j = 0$. Then $\mathbf{P}_{\Q_{j-1}^\perp} \cdots \mathbf{P}_{\Q_1^\perp}\B_j$. This means that, for at least one $j\ast \in [j-1]$ we have $\bP_{\Q^\perp_{j^\ast-1}}\cdots \bP_{\Q^\perp_{1}}\B_j$ is in the null-space of $\bP_{\Q^\perp_{j^\ast-1}}\cdots \bP_{\Q^\perp_{1}}\B_j$
    
   %  \begin{align}
   %      \bm{0} &= \mathbf{P}_{\Q_{j-1}^\perp} \cdots \mathbf{P}_{\Q_1^\perp}\B_j,\\
   %      \bm{0} &= (\I - \T_{j-1})\B_j\\
   %      \B_j   &= \T_{j-1}\B_j\\
   %      \B_j   &= \Q_{j-1} \Q_{j-1}^\top\B_j + \bP_{\Q_{j-1}^\perp} \T_{j-2}\B_j\\
   %      \bP_{\Q_{j-1}^\perp}\B_j   
   %      &= \bP_{\Q_{j-1}^\perp}\T_{j-2}\B_j\\
   %      \bm{0}  &= \bP_{\Q_{j-1}^\perp}(\I-\T_{j-2})\B_j.
   %  \end{align}
   %  Either $(\I-\T_{j-2})\B_j =\bm{0}$ or 
    
   %  This means that either $\T_{j-2}\B_j = \B_j$ or $[\B_j] = [\Q_{j-1}]$. The latter cannot be true. If the prior is true, then 
   %  \begin{align*}
   %      \B_i   &= \mathbf{P}^\prime_{\Q_{i-2}^\perp}\B_i\\
   %      \bm{0} &= (\I - \mathbf{P}^\prime_{\Q_{i-2}^\perp}) \B_i\\
   %      \bm{0} &= \mathbf{P}_{\Q_{i-2}^\perp} \cdots \mathbf{P}_{\Q_1^\perp}\B_i.
   %  \end{align*}
   % We can repeat this process and arrive at $\mathbf{P}_{\Q_1^\perp}\B_i = \bm{0}$. But not all of $[\B_i]$ is in $[\Q_1] = [\B_1]$.

    % \paragraph{projection property} by definition

    % \paragraph{steifel coordinates} 
        Let $\Q = [\Q_1 |\Q_2 | \cdots | \Q_k]$. We must verify $\Q^T \Q = \I$. This is equivalent to $\Q_i^T\Q_i = \I$ and $\Q_i^T\Q_j = \mathbf{0}$ for all $i\neq j$. First, $\Q_i^T\Q_i = \I$ by construction.    

        To show $\Q_i^T\Q_j = \mathbf{0}$ for all $i\neq j$, it suffices to fix an arbitrary $j$ and show $\Q_{j+\ell}^T\Q_j$ for all $\ell > 1$ and $\ell < k-j$. 
        
        Fix $j \in \{1,2,\dots,k-1\}$. First we will show $\Q_{j+\ell}^\top\Q_j = \bm{0}$ is equivalent to $\C_{j+\ell}^\top\C_{j}= \mathbf{0}$ for all $\ell > 1$ and $\ell < k-j$.  Take $\M_i \in O(m_i)$. For each $i$, $\Q_i \M_i = \U_i$ where $\U_i \bm{\Sigma}_i \V_i = \C_i$ is the SVD (with $\U_i \in \R^{n \times m_i}$, $\bm{\Sigma}_i \in \R^{m_i \times m_i}$, and $\V_i \in \R^{m_i \times m_i}$) because $[\U_i] = [\C_i] = [\Q_i]$. Notice that $\U_i^\top \U_i = \bm{\Sigma}_i^{-1}\bm{\Sigma}_i= \bm{\Sigma}_i\bm{\Sigma}_i^{-1} = \V_i^\top \V_i = \I$ for $i=j$ and $i=j+\ell$. Thus
        \begin{align*}
            \mathbf{0} &= \C_{j+\ell}^\top\C_{j}, \\
            &= (\U_{j+\ell} \mathbf{\Sigma}_{j+\ell} \V_{j+\ell}^T)^\top\U_{j} \mathbf{\Sigma}_{j} \V_{j}^\top, \\
            &= \mathbf{\Sigma}_{j+\ell}^{-1} \V_{j+\ell}^\top \V_{j+\ell} \mathbf{\Sigma}_{j+\ell} \U_{j+\ell}^\top \U_{j} \mathbf{\Sigma}_j \V_j^\top \V_j \mathbf{\Sigma}_j^{-1},\\
            &= \U_{j+\ell}^\top \U_j, \\
            &= (\Q_{j+\ell}\M_i)^\top (\Q_j\M_i),\\
            &= \M_i^\top \Q_{j+\ell}^\top \Q_j\M_i,\\
            &= \Q_{j+\ell}^\top \Q_j.
        \end{align*}
        For any $j$, recall $\bP_{\Q_j^\perp} = (\I - \Q_j \Q_j^\top)$ projects into the null space of $\Q_j$. Since the span of the columns of $\Q_j$ is the same as the span of the columns of $\C_j$, $\bP_{\Q_j^\perp}$ projects into the null space of $\C_j$. So, $\bP_{\Q_j^\perp} \C_j= \mathbf{0}$ meaning that $\C_j$ satisfies the projection property for $j=1,2,\dots,k$.
            
        We will proceed by induction on $\ell$ to show $\C_{j + \ell}^\top\C_j = \mathbf{0}$. For the base case take $\ell = 1$. Then
        \begin{equation}
            \C_{j + 1}^\top\C_j = \B_{j + 1}^\top\bP_{\Q_1^\perp} \cdots \bP_{\Q_{j-1}^\perp}\underbrace{(\bP_{\Q_j^\perp}\C_j)}_{\mathbf{0}} = \mathbf{0}.
        \end{equation}
        
        Suppose $\C_{j + s}^\top\C_j = \mathbf{0}$ for all $0 < s < \ell$. Since $[\C_i] = [\Q_i]$ for $i=1,2,\dots,k$, their null spaces are also equal. Also $[\C_j]$ is in the space perpendicular to $[\C_{j + s}]$ for all $0 < s < \ell$. This implies, for all $0 < s < \ell$, projecting onto $[\C_{j+s}] = [\Q_{j+s}]$ fixes $\C_j$, namely $\bP_{\Q_{j+s}^\perp}\C_j = \C_j$. Then
        \begin{align}
            \C_{j + \ell}^\top\C_j &= \B_{j + \ell}^\top\bP_{\Q_1^\perp} \cdots \bP_{\Q_j^\perp} \underbrace{(\bP_{\Q_{j+1}^\perp} \cdots \bP_{\Q_{j+\ell-1}^\perp}\C_j)}_{\C_j}\\
            &= \B_{j + \ell}^\top\bP_{\Q_1^\perp} \cdots \bP_{\Q_{j-1}^\perp}\underbrace{(\bP_{\Q_j^\perp} \C_j)}_{\mathbf{0}} = \mathbf{0}
        \end{align}
        because $\mathbf{P}_{j+\ell -1}$ projects onto the null space of $\C_{j+\ell - 1}$. So we have shown $\bm{0} = \C_{j + \ell}^\top\C_j = \Q_{j + \ell}^\top\Q_j$. So $\Q_i^\top \Q_j = \bm{0}$ for all $i \neq j$.


        Thus $\Q^\top \Q = \I$, and $\Q\in \R^{n \times n_k}$ is in Stiefel coordinates where $n_k = \sum_{i=1}^k m_k$. 

        This means that 
        \begin{align*}
            \bP_{\Q_i^\perp} \bP_{\Q_j^\perp} &= (\I - \Q_i \Q_i^\top)(\I - \Q_j \Q_j^\top) \\
                                            &= \I - \Q_i \Q_i^\top - \Q_j \Q_j^\top +  \Q_i \underbrace{\Q_i^\top\Q_j}_{\bm{0}} \Q_j^\top\\
                                            &= \I - \Q_i \Q_i^\top - \Q_j \Q_j^\top\\
        \end{align*}
        In general,
        \begin{align*}
            \bP_{\Q_i^\perp} \cdots \bP_{\Q_1^\perp} &= \I - \sum_{j=1}^i \Q_j \Q_j^\top \\
            &= \I - [\Q_1|\cdots|\Q_i] [\Q_1|\cdots|\Q_i]^\top \\
            &= \bP_{[\Q_1|\cdots|\Q_i]^\perp}.
        \end{align*}
        Furthermore
        $\C_i = \bP_{\Q_{i-1}^\perp} \cdots \bP_{\Q_1}^\perp\B_i = \bP_{[\Q_1|\cdots|\Q_{i-1}]^\perp}\B_i$.

        Suppose, $[\Q_1,\dots,\Q_i] = [\B_1,\dots, \B_i]$ for all $i<j$. Then
        $[\Q_j] = [\C_j] = [\bP_{[\Q_1|\cdots|\Q_{i-1}]^\perp}\B_i]$ which is orthogonal to $[\Q_1,\dots,\Q_{j-1}] = [\B_1,\dots, \B_{j-1}]$. Thus $[\Q_1,\dots,\Q_j] = [\B_1,\dots, \B_j]$. So $[\![\Q]\!]$ is hierarchy preserving.

        By construction $[\Q_1|\cdots|\Q_i] \in \R^{n \times n_k}$ where $n_k = \mathrm{dim}([\B_1,\dots, \B_i]) = \mathrm{dim}(\D_{\mathcal{A}_i})$.

        Since $\mathrm{dim}([\D_{\mathcal{A}_i}]) < \mathrm{dim}([\D_{\mathcal{A}_{i+1}}])$ we have $\C_i = \bP_{[\Q_1|\cdots|\Q_{i-1}]^\perp}\B_i \neq \bm{0}$ for all $i$.
        
        % This means that $\{\bP_{\Q_1^\perp}, \bP_{\Q_2^\perp}, \dots , \bP_{\Q_k^\perp}\}$ is a set of pairwise orthogonal projection matrices. Thus $\bP_{\Q_i^\perp}\bP_{\Q_j^\perp} = \bP_{\Q_j^\perp}\bP_{\Q_i^\perp}$ for all $i \neq j$. We use this to show that, each $\C_i \neq 0$ for each $i=1,,2,\dots, k$. Fix $i \in \{1,2,\dots,k\}$. Suppose, by way of contradiction that $\C_i = \bm{0}$. Then, for any $j=1,2,\dots,i-1$, we can write
        % \begin{equation}
        %     \bm{0} = \C_i =   \bP_{\Q_{i-1}^\perp} \cdots \bP_{\Q_1^\perp}\C_i = \bP_{\Q_j^\perp} \cdots \C_i.
        % \end{equation}
        % This means that $\C_i$ is in the nullspace of each $\Q_j$ for $j < i$. The intersection of all these nullspaces is $[\Q_1,\dots,\Q_{i-1}]$.

    % \paragraph{hierarchy preserving}
    % $[\Q_1] = [\B_1]$. Suppose that $[\Q_1,\dots,\Q_{i-1}] = [\B_1,\dots,\B_{i-1}]$. Suppose BWOC $[\Q_1,\dots,\Q_i] \neq [\B_1,\dots,\B_i]$. 
    % First, we find a hierarchy-preserving flag. Let $n_i = \mathrm{dim}(\D_{\cA_i})$ for $i=1,\dots,k$. Since $\cA_1 \subset \cA_2 \subset \cdots \subset \cA_k$ is a column hierarchy for $\D$, we have $\mathrm{dim}([\D_{\cA_i}]) < \mathrm{dim}([\D_{\cA_{i+1}}])$. Thus $n_1<n_2<\cdots<n_k$. This also implies $\mathrm{dim}([\B_1,\cdots,\B_i]) < \mathrm{dim}([\B_1,\cdots,\B_{i+1}])$ for all $i=1,2,\dots,k-1$. So, for $i=1,2,\dots,k-1$, there exists some subspace $[\Q_{i+1}] \neq \emptyset$ that fills this `gap between subspaces.' Specifically, $[\Q_{i+1}]$ is orthogonal to $[\B_1,\cdots,\B_{i}]$ and $[\B_1,\cdots,\B_{i},\Q_{i+1}]=[\B_1,\cdots,\B_i,\B_{i+1}]$. Thus, taking $[\Q_1] = [\B_1]$, we have $[\Q_1,\cdots,\Q_i]=[\B_1,\cdots,\B_i]$ for $i=2,\dots,k$.
\end{proof}
    % 

    % Now we show $[\X_i]$ is orthogonal to $[\X_{i+1}]$ for all $i=1,2,\dots,k-1$. By construction, for $i=2,\dots,k$, $[\X_i]$ is orthogonal to $[\B_1,\cdots,\B_{i-1}] = [\B_1,\cdots,\B_{i-2},\X_{i-1}]$. Since $[\X_{i-1}] \subset [\B_1,\cdots,\X_{i-1}]$, $[\X_i]$ must also be orthogonal to $[\X_{i-1}]$.

    % Fix $\ell < k-i$ and suppose $[\X_{i+j}]$ is orthogonal to $[\X_i]$ for all $j < \ell$. Thus the following subspace splits into a union of subspaces $[\X_1,\dots,\X_{i+\ell-1}] = [\X_1]\cup \cdots \cup [\X_{i+\ell-1}]$. By construction, $[\X_{i+\ell}]$ is orthogonal to $[\X_1,\dots,\X_{i+\ell-1}]$ so it must be orthogonal to each of $[\X_i]$.

    % Therefore $\X_i^\top \X_j = \bm{0}$ and the matrix $[\X_1|\X_2|\cdots|\X_k]$ is in Stiefel coordinates.

    % Define the projector onto the null space of $[\X_i]$, namely $\bP_{\X_i^\perp} = \I - \X_i \X_i^\top$.
    
    % For $i=2$ we have
    % $[\bP_{\Q_1^\perp}\B_2]$ is the part of $[\B_2]$ that is orthogonal to $[\Q_1] = [\B_1]$. This is exactly $[\X_2] \neq \emptyset$. Thus $[\Q_1,\Q_2] = [\B_1,\B_2]$.

    % Now fix $2<i<k$ and suppose for all $j < i$ we have
    % \begin{align*}
    %     [\C_j] &\neq \emptyset, \\
    %     [\Q_1,\dots,\Q_j] &= [\B_1,\dots,\B_j].
    % \end{align*}
    % Then, $\bP_{\Q_1^\perp}\B_i$ maps $\B_i$ onto $[\Q_1]^\perp$
    
    % %$\C_i = \bP_{\Q_{i-1}^\perp}\cdots \bP_{\Q_1^\perp}\B_i$ maps $\B_i$ onto a subspace of $[\B_i] \cap [\Q_1,\dots,\Q_{i-1}]^\perp = [\B_i] \cap [\B_1,\dots,\B_{i-1}]^\perp = [\X_i] \neq \emptyset$.
    
    % Fix $i > 1$. Notice $\bP_{\Q_{i-1}^\perp}\cdots \bP_{\Q_1^\perp}$ projects into a space that is a subset of the orthogonal complement of $[\Q_1,\dots,\Q_{i-1}]$.
    

    % \vspace{1cm}
    
    
    
    % By inducion- Fix some $1<i<k$ and suppose $\C_j \neq 0$ for all $j<i$. Then 
    % $\C_i = $


    
    
    % Then we show that the projection property is satisfied, which results in $\C_i^\top\C_j = \bm{0}$that implies $\Q_i^\top\Q_j = 0$

    \paragraph{Induction with i=2}
    We begin with the base case of $i=2$.

    First we show $\C_2 \neq 0$. By way of contradiction, assume $\C_2 = 0$. Then $\bP_{\Q_1^\perp}\B_2=0$ which means that $[\B_2] \subseteq [\Q_1] = [\B_1]$. This also implies that $\mathrm{rank}(\D_{\cA_2}) = \mathrm{dim}([\B_1,\B_2]) = \mathrm{dim}([\B_1]) = \mathrm{rank}(\D_{\cA_1})$. This contradicts the assumption of a column hierarchy, namely $\mathrm{rank}(\D_{\cA_2}) > \mathrm{rank}(\D_{\cA_1})$. Therefore $\C_2 \neq 0$.

    Now we show $[\Q_1|\Q_2]$ is in Stiefel coordinates. By construction, $[\Q_2] = [\C_2] = [\bP_{\Q_1^\perp}\B_2]$. So $[\Q_2]$ is orthogonal to $[\Q_1]$. Thus $\Q_2^\top \Q_1 = 0$.

    Now we show $[\Q_1,\Q_2] = [\B_1,\B_2]$. Since $[\Q_2] = [\bP_{\Q_1^\perp}\B_2]$, we satisfy the projection property for $k=2$. So
    \begin{align*}
        \bm{0} &= \bP_{\Q_2^{\perp}}\bP_{\Q_1^{\perp}}\B_2,\\
               &= (\I - \Q_1 \Q_1^\top - \Q_2 \Q_2^\top + 2 \Q_2 \underbrace{\Q_2^\top \Q_1}_{\bm{0}} \Q_1^\top)\B_2,\\
               &= (\I - \Q_1 \Q_1^\top - \Q_2 \Q_2^\top)\B_2,\\
        \B_2   &= [\Q_1|\Q_2][\Q_1|\Q_2]^\top \B_2.
    \end{align*}
    Thus $[\B_2] \subseteq [\Q_1,\Q_2]$. Since $[\B_1] = [\Q_1]$, we have $[\B_1,\B_2] \subseteq [\Q_1,\Q_2]$.
    
    In contrast, $[\B_2] \supseteq [\bP_{\Q_1^\perp}\B_2] = [\C_2] = [\Q_2]$. This and $[\B_1] = [\Q_1]$ implies $[\B_1,\B_2] \supseteq [\Q_1,\Q_2]$. Therefore $[\B_1,\B_2] = [\Q_1,\Q_2]$.
    
    Finally, we specify the dimensions of $\Q_2$. Call $m_1 = n_1$, so $\Q_1 \in \R^{n \times m_1}$. Let $n_2 = \mathrm{rank}(\D_{\cA_2}) = \mathrm{dim}([\B_1,\B_2]) = \mathrm{dim}([\Q_1,\Q_2])$, thus $[\Q_1|\Q_2] \in \R^{n \times n_2}$ and $\Q_2 \in \R^{n \times m_2}$ with $m_2 = n_2 - n_1$.




\begin{prop}\label{prop:proj_prop_and_flags_app}
 Suppose $\cA_1 \subset \cA_2 \subset \cdots \subset \cA_k$ is a column hierarchy for $\D$. Then there exists some hierarchy-preserving 
 $[\![\Q]\!] \in \flag(n_1,n_2,\dots, n_k;n)$ (with $n_i = \mathrm{rank}(\D_{\mathcal{A}_i})$) 
 that satisfies the projection property of $\D$ and can be used for a flag decomposition of $\D$ with
    \begin{align}\label{eq:R_and_P}
        \bR_{i,j} &= 
        \begin{cases}
            \Q_i^\top\bP_{\Q_{i-1}^\perp}\cdots \bP_{\Q_1^\perp} \B_i, &i=j\\
            \Q_i^\top\bP_{\Q_{i-1}^\perp}\cdots \bP_{\Q_1^\perp} \B_j, &i < j
        \end{cases},\\
        \bP_i &= \left[ \,\mathbf{e}_{b_{i,1}}\,|\, \mathbf{e}_{b_{i,2}}\,|\, \cdots\,|\, \mathbf{e}_{b_{i,|\cB_i|}} \right]
    \end{align}
    where $\{b_{i,j}\}_{j=1}^{|\cB_i|} = \cB_i$ and $\mathbf{e}_{b}$ is the $b_{i,j}$$^{\mathrm{th}}$ standard basis vector.
\end{prop}
\begin{proof}
    We can find $[\![\Q]\!]$ using the previous proposition.

    We can easily find the permutation matrix $\bP = [\bP_1|\bP_2|\cdots |\bP_k]$ so that $\D\bP = \B$ and $\D = \B \bP^\top$. We assign the non-zero values in each column of $\bP_i$ to be the index of each element in $\mathcal{B}_i$. In summary, $\bP_i$ is defined so that $\D\bP= [\B_1|\B_2|\cdots|\B_k]$ so $\D = \B = [\B_1|\B_2|\cdots|\B_k]\bP^\top$.

    We aim to find $\R$ so that $\B = \Q \bR$.

    Define the recursion~\footnote{In the manuscript we call $\T_i = \bP^\prime_{\Q_i^\perp}$. We switch to using $\T_i$ for notational simplicity.}
    \begin{equation}
        \T_i = \Q_i \Q_i^\top + \bP_{\Q_i^\perp} \T_{i-1}
    \end{equation}
    with $\T_0 = \mathbf{0}$.

    We prove via mathematical induction that (for $i=1,2,\dots,k$)
    \begin{equation}\label{eq: proj_rec_identity}
        \I -  \T_i = \mathbf{P}_{\Q_i^\perp} \cdots \mathbf{P}_{\Q_1^\perp}.
    \end{equation}
    For $i=1$ 
    \begin{align}
         \I -  \T_{1} &= \I - (\Q_1 \Q_1^T + \mathbf{P}_{\Q_1^\perp} \T_{0}),\\
           &= \I - \Q_1 \Q_1^T,\\
           &= \mathbf{P}_{\Q_1^\perp}.
    \end{align}
    Fix some $j < k$ and suppose~\cref{eq: proj_rec_identity} holds for or all $i < j$. Then
    \begin{align}
        \I -  \T_j &= \I - \Q_j \Q_j^T - \mathbf{P}_{\Q_j^\perp} \T_{j-1},\\
         &=\mathbf{P}_{\Q_j^\perp} - \mathbf{P}_{\Q_j^\perp} \T_{j-1},\\
        &=\mathbf{P}_{\Q_j^\perp}(\I - \T_{j-1}),\\
        &= \mathbf{P}_{\Q_j^\perp} \cdots \mathbf{P}_{\Q_1^\perp}.
    \end{align}
    
    For any $i$, we have
    \begin{align}
        \T_i &= (\Q_i \Q_i^T + (\I - \Q_i \Q_i^T) \T_{i-1}), \\
        &= \Q_i \Q_i^T(\I - \T_{i-1}) + \T_{i-1}, \\
        &= \Q_i \Q_i^T\mathbf{P}_{\Q_{i-1}^\perp}\cdots \mathbf{P}_{\Q_1^\perp} + \T_{i-1}.
    \end{align}
    Replacing every $\T_j$ for each $j < i$ with 
    \begin{equation}
        \Q_j \Q_j^T\mathbf{P}_{\Q_{j-1}^\perp}\cdots \mathbf{P}_{\Q_1^\perp}+ \T_{j-1},
    \end{equation}
    we have
    \begin{align}
    \begin{aligned}
        \T_i &= \Q_i \Q_i^T\mathbf{P}_{\Q_{i-1}^\perp}\cdots \mathbf{P}_{\Q_{1}^\perp} \\
        &+ \Q_{i-1} \Q_{i-1}^T\mathbf{P}_{\Q_{i-2}^\perp}\cdots \mathbf{P}_{\Q_{1}^\perp} \\
        &+ \cdots +\Q_1 \Q_1^T \mathbf{P}_{\Q_{1}^\perp}
    \end{aligned}
    \end{align}
    Then, we have 
    \begin{align}
    \begin{aligned}
        \B_i & = \T_i \B_i,\\
        &= (\Q_i \Q_i^T\mathbf{P}_{i-1}\cdots \mathbf{P}_1 \\
        &+ \Q_{i-1} \Q_{i-1}^T\mathbf{P}_{i-2}\cdots \mathbf{P}_1 \\
        &+ \cdots +\Q_1 \Q_1^T \mathbf{P}_1)\B_i,\\
        &= \Q_i \bR_{i,i} + \Q_{i-1} \bR_{i-1,i} + \cdots +\Q_1 \bR_{1,i}.
    \end{aligned}
    \end{align}
    Stacking this into matrices for each $i$ gives us the desired result.

    % Now we want to show that this decomposition is hierarchy preserving, \eg, $[\B_1,\dots,\B_i] = [\Q_1,\dots,\Q_i]$ for each $i=1,\dots,k$. From our decomposition we see that $[\B_i] \subseteq [\Q_1,\dots,\Q_i]$ for each $i$. Thus $[\B_1,\dots,\B_i] \subseteq [\Q_1,\dots,\Q_i]$. Showing 
    % \begin{equation}\label{eq:dim_ineq}
    %     \mathrm{dim}([\Q_1,\dots,\Q_i]) \leq \mathrm{dim}([\B_1,\dots,\B_i])
    % \end{equation}
    % for $i=1,\dots,k$ will give us $[\Q_1,\dots,\Q_i] \subseteq [\B_1,\dots,\B_i]$ which implies $[\B_1,\dots,\B_i] = [\Q_1,\dots,\Q_i]$. We show~\cref{eq:dim_ineq} via mathematical induction.

    
    % $\Q\bR = \B$ is proved using the recursion 
    % \begin{equation}
    %     \mathbf{P}_{\Q_i^\perp}^\prime = \X_i \X_i^\top + \bP_{\Q_i^\perp} \mathbf{P}^\prime_{\Q_{i-1}^\perp}
    % \end{equation}
    % with $\mathbf{P}^\prime_{\Q_0^\perp} = \mathbf{0}$.
    % We show that it is hierarchy-preserving after the recursion
\end{proof}


\begin{prop}[Block rotational ambiguity]
    Given the FD $\D = \Q \bR \bP^\top$, any other Stiefel coordinates for the flag $[\![\Q]\!]$ produce an FD of $\D$ (via~\cref{prop:proj_prop_and_flags}). Furthermore, different Stiefel coordinates for $[\![\Q]\!]$ produce the same objective function values in~\cref{eq:general_opt} and~\cref{eq:iterative_opt} (for $i=1,\cdots,k$).
\end{prop}
\begin{proof}
    The flag manifold $\flag(n_1,n_2,\dots,n_k;n)$ is diffeomorphic to $St(n_k,n)/(O(m_1)\times \cdots \times O(m_k))$ where $m_i = n_i - n_{i-1}$. Suppose $\D = \Q \bR \bP^\top$ is a flag decomposition. Consider $\Q \mathbf{M} \in St(n_k,n)$ with $\mathbf{M} = \mathrm{diag}([\mathbf{M}_1|\mathbf{M}_2|\cdots | \mathbf{M}_k]) \in O(m_1)\times \cdots \times O(m_k)$. Notice $\Q$ and $\Q \mathbf{M}$ are Steifel coordinates for the same flag, $[\![\Q]\!]=[\![\Q\M]\!]$. Indeed, both satisfy the projection property relative to $\D$ because (for $i=1,2,\dots,k$)
    \begin{align}\label{eq:proj_mat_equiv}
    \begin{aligned}%
        \bP_{\Q_i^\perp} &= \I - \Q_i \Q_i^\top\\
        &= \I - \Q_i \mathbf{M}_i \mathbf{M}_i^\top\Q_i^\top\\
        &= \I - \Q_i \mathbf{M}_i (\mathbf{M}_i \Q_i)^\top\\
        &= \bP_{(\Q_i\mathbf{M}_i)^\perp}
    \end{aligned}
    \end{align}
    which implies that the objective function values in~\cref{eq:general_opt} and~\cref{eq:iterative_opt} (for $i=1,2,\dots,k$) are the same for $\Q$ and $\Q\mathbf{M}$. Additionally,~\cref{eq:proj_mat_equiv} implies
    \begin{align}
        \bm{0} &=\bP_{\Q_i^\perp}\bP_{\Q_{i-1}^\perp}\cdots \bP_{\Q_1^\perp} \B_i \\
        &= \bP_{(\Q_i\mathbf{M}_i)^\perp}\bP_{(\Q_{i-1}\mathbf{M}_{i-1})^\perp}\cdots \bP_{(\Q_1\mathbf{M}_1)^\perp} \B_i.
    \end{align}
    Define $\bR^{(\mathbf{M})}$ with blocks $\bR^{(\mathbf{M})}_{i,j}$
    \begin{equation}
        \begin{cases}
            (\Q_i\mathbf{M}_i)^\top\bP_{(\Q_{i-1}\mathbf{M}_{i-1})^\perp}\cdots \bP_{(\Q_1\mathbf{M}_1)^\perp} \B_i, &i=j\\
            (\Q_i\mathbf{M}_i)^\top\bP_{(\Q_{i-1}\mathbf{M}_{i-1})^\perp}\cdots \bP_{(\Q_1\mathbf{M}_1)^\perp} \B_j, &i < j.
        \end{cases}        
    \end{equation}
    Notice 
    \begin{align}
        \B_i &= \sum_{j=1}^i \Q_j \bR_{j,i}\\
             &= \sum_{j=1}^i \Q_j \Q_j^\top \bP_{\Q_{j-1}^\perp}\cdots \bP_{\Q_1^\perp} \B_i\\
             &= \sum_{j=1}^i \Q_j (\M_j \M_j^\top) \Q_j^\top \bP_{\Q_{j-1}^\perp}\cdots \bP_{\Q_1^\perp} \B_i\\
             &= \sum_{j=1}^i \Q_j \M_j (\M_j \Q_j)^\top \bP_{\Q_{j-1}^\perp}\cdots \bP_{\Q_1^\perp} \B_i\\
             % &= \sum_{j=1}^i (\Q_j \M_j) (\M_j \Q_j)^\top \bP_{(\Q_{j-1}\mathbf{M}_{j-1})^\perp}\cdots \bP_{(\Q_1\mathbf{M}_1)^\perp} \B_i\\
             &= \sum_{j=1}^i (\Q_j \M_j)\bR^{(\mathbf{M})}_{j,i}.
    \end{align}

    Thus $\D = (\Q\mathbf{M})\bR^{(\mathbf{M})} \bP^\top$ is a flag decomposition.
\end{proof}
