\begin{table*}[t]
\centering
\newcommand{\nar}[1]{\makebox[1em][c]{\scalebox{.85}[1.0]{#1}}}
\setlength{\tabcolsep}{4.5pt}
\resizebox{\textwidth}{!}{%
\begin{tabular}{llcccccccccccccccccc}
\toprule
&
    & \multicolumn{3}{c}{\nar{Single Document QA}} 
    & \multicolumn{3}{c}{\nar{Multi Document QA}} 
    & \multicolumn{3}{c}{Summarization} 
    & \multicolumn{3}{c}{Few-shot Learning} 
    & \multicolumn{2}{c}{Synthetic} 
    & \multicolumn{2}{c}{Code} 
    & \multirow{2.25}{*}{\makecell{Avg.\\Abs.}}
    & \multirow{2.25}{*}{\makecell{Avg.\\Rel.\nar{{\small\ (\%)}}}} \\
\cmidrule(lr){3-5}
\cmidrule(lr){6-8}
\cmidrule(lr){9-11}
\cmidrule(lr){12-14}
\cmidrule(lr){15-16}
\cmidrule(lr){17-18}
&
    & {\nar{NQA}}
    & {\nar{Qasper}}
    & {\nar{MFQA}}
    & {\nar{HQA}}
    & {\nar{2WMQ}}
    & {\nar{MSQ}}
    & {\nar{GR}}
    & {\nar{QMS}}
    & {\nar{MN}}
    & {\nar{TREC}}
    & {\nar{TQA}}
    & {\nar{SAMS}}
    & {\nar{PC}}
    & {\nar{PR}}
    & {\nar{RBP}}
    & {\nar{LCC}} & \\
\midrule
Methods & Window & \multicolumn{17}{@{}c@{}}{Llama 3 (8B)} \\
\midrule
FA2 &8K 
    &19.9 &42.4 &41.0 &47.4 &39.2 &23.0 &29.9 &21.4 &27.5 &74.0 &90.5 &42.3 &\textbf{8.5} &62.5 &49.1 &60.8 
    &\cellcolor[HTML]{ff8370}42.47 
    &\cellcolor[HTML]{ff8370}87.69 \\
Infinite &8K 
    &19.4 &42.8 &40.4 &43.8 &37.9 &18.3 &29.3 &21.4 &\textbf{27.6} &74.0 &90.1 &41.7 &4.5 &50.0  &48.6 &60.1 
    &\cellcolor[HTML]{ff8370}40.62 
    &\cellcolor[HTML]{ff8370}83.23 \\
Streaming &8K 
    &20.1 &42.5 &39.5 &43.7 &37.9 &19.7 &29.2 &21.3 &\textbf{27.6} &73.5 &90.1 &41.5 &5.0 &49.0 &49.0 &60.4 
    &\cellcolor[HTML]{ff8370}40.61
    &\cellcolor[HTML]{ff8370}83.21 \\
InfLLM &8K 
    &22.6 &\textbf{43.7} &49.0 &49.0 &35.6 &26.1 &30.8 &22.7 &\textbf{27.6} &73.5 &\textbf{90.9} &42.4 &7.2 &84.0 &46.5 &59.9 
    &\cellcolor[HTML]{ffae5a}44.47
    &\cellcolor[HTML]{ffaa5d}92.83 \\
\cdashline{1-19}[0.5pt/1pt]\rule{0pt}{2.6ex}%
\textbf{InfiniteHiP} &\textbf{3K} 
    &\textbf{26.6} &43.2 &\textbf{50.3} &\textbf{51.9} &\textbf{41.0} &\textbf{30.9} &\textbf{31.7} &\textbf{23.3} &26.9 &\textbf{75.5} &90.3 &\textbf{43.0} &7.5 &\textbf{93.5} &\textbf{64.8} &\textbf{63.1} 
    &\cellcolor[HTML]{00b1b0}\textbf{47.72}
    &\cellcolor[HTML]{00b1b0}\textbf{100.00} \\
\midrule
Methods & Window & \multicolumn{17}{@{}c@{}}{Mistral 0.2 (7B)} \\
\midrule
FA2 &32K 
    &22.1 &29.2 &47.6 &37.5 &22.0 &19.0 &31.1 &\textbf{23.9} &26.6 &\textbf{71.0} &86.0 &42.3 &\textbf{4.0} &86.9 &54.1 &57.4 
    &\cellcolor[HTML]{a9c16e}41.29
    &\cellcolor[HTML]{b5c269}96.44 \\
Infinite &6K 
    &18.4 &30.0 &39.0 &32.0 &22.3 &15.8 &29.7 &21.9 &26.6 &70.0 &85.2 &41.6 &2.1 &42.8 &53.4 &57.1 
    &\cellcolor[HTML]{ff8370}36.76
    &\cellcolor[HTML]{ff8370}83.49 \\
Streaming &6K 
    &17.9 &\textbf{30.1} &39.1 &32.2 &21.8 &14.7 &29.8 &21.9 &26.6 &70.0 &85.6 &41.3 &2.5 &42.2 &51.5 &55.4 
    &\cellcolor[HTML]{ff8370}36.41 
    &\cellcolor[HTML]{ff8370}82.63 \\
InfLLM &6K 
    &22.1 &29.3 &47.4 &36.6 &22.3 &17.7 &31.0 &23.5 &\textbf{26.7} &69.0 &86.7 &42.5 &2.9 &64.0 &53.0 &56.7 
    &\cellcolor[HTML]{ffa360}39.46
    &\cellcolor[HTML]{ff9368}91.23 \\
InfLLM &12K 
    &23.0 &29.5 &47.6 &39.5 &\textbf{23.6} &18.9 &31.4 &23.8 &\textbf{26.7} &\textbf{71.0} &87.3 &41.8 &3.0 &\textbf{87.4} &52.1 &56.7 
    &\cellcolor[HTML]{95bf76}41.46
    &\cellcolor[HTML]{99bf74}96.99 \\
\cdashline{1-19}[0.5pt/1pt]\rule{0pt}{2.6ex}%
\textbf{InfiniteHiP} &\textbf{3K} 
    &\textbf{24.1} &28.7 &\textbf{48.6} &\textbf{40.4} &23.2 &\textbf{22.1} &\textbf{31.6} &23.8 &26.5 &70.5 &\textbf{88.8} &\textbf{42.7} &3.5 &86.6 &\textbf{62.1} &\textbf{60.4} 
    &\cellcolor[HTML]{00b1b0}\textbf{42.71}
    &\cellcolor[HTML]{08b2ad}\textbf{99.85} \\
\bottomrule
\end{tabular}%
}
\caption{\textbf{LongBench Results.} \textit{FA2} refers to truncated FlashAttention2, \textit{Infinite} refers to LM-Infinite, and \textit{Streaming} refers to StreamingLLM. The `Avg.\,Rel.' column shows the average of the \textit{relative score} of each subset. The relative score is computed by dividing the original score by the highest score in its column. We believe that the relative score better represents the differences in performance because the variance is normalized per subset. The best values in each column are shown in bold font.
}
\label{tab:longbench}
\end{table*}
