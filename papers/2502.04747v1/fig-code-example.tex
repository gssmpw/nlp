\begin{figure}
    \centering
     \includegraphics[width=0.95\textwidth]{figs/code-example-1.pdf}
     % \vspace{-10pt}
    \caption{Examples of the action code generated by different LLMs.}
    % \vspace{-10pt}
    \label{fig:code-example}
\end{figure}

\begin{lstlisting}[label=code:multi-iterations,caption=An example of how multi-round interactions between execution sandbox and code agent helps task accomplishment. Task: ``increase the volume slightly'' on music player; model: o3-mini.]
// 1st Iteraction (failed): "TypeError: Cannot read property 'volume' of undefined."
let currentVolume = vueInstance.player.volume;
let newVolume = Math.min(currentVolume + 0.1, 1);
vueInstance.player.volume = newVolume;
console.log('Volume increased to', newVolume);

// 2nd Iteraction (failed): "Error: Player component not found."
let player = (vueInstance.$refs && vueInstance.$refs.player) ? vueInstance.$refs.player : vueInstance.player;
if (!player) {
    throw new Error('Player component not found');
}
let currentVolume = player.volume;
let newVolume = Math.min(currentVolume + 0.1, 1);
player.volume = newVolume;
console.log('Volume increased to', newVolume);

// 3rd Interaction. Task successfully accomplished.
let player = vueInstance.player || (window.yesplaymusic && window.yesplaymusic.player) || (vueInstance.$store && vueInstance.$store.state && vueInstance.$store.state.player);
if (!player) {
    throw new Error('Player component not found');
}
let currentVolume = player.volume;
let newVolume = Math.min(currentVolume + 0.1, 1);
player.volume = newVolume;
console.log('Volume increased to', newVolume);
\end{lstlisting}
