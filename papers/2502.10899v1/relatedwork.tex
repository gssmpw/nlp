\section{Related Work}
Previous work focused on detecting and segmenting leukemia cells from microscopic blood smear images \cite{dhal2020acute,genovese2021acute,das2021transfer,mohapatra2012lymphocyte}, while some worked on classifying a specific type of leukemia against healthy cells~\cite{jothi2019rough, shah2019classification, negm2018decision, rawat2017computer}. However, at the time of writing, the authors are only aware of two papers that work on classifying multiple types against healthy cells, and both target the ALL, AML, CLL, and CML subtypes.

The authors in~\cite{ahmed2019identification} propose using a small custom CNN, using the ALL-IDB~\cite{6115881} dataset as well as some images from the American Society of Hematology (ASH) Image Bank~\cite{imagebank}. The authors obtained 903 images and implemented several augmentations to reduce overfitting. Despite this, their results show a large difference between the training and validation accuracies, indicating that their model is greatly overfitting. This can be attributed to the small dataset and overly simplified model. In the most recent paper~\cite{9425264}, the authors propose using a pre-trained GoogleNet model, which is fine-tuned on 1,200 images from the ASH Image Bank. The authors do not provide detailed information on the hyperparameters used, making it difficult to study their impact on the model.
% determine the impact of these parameters on the performance of the proposed method.

In summary, both papers lacked sufficient detail regarding the techniques used to attain their results. %Additionally, images from the ASH Image Bank are prohibited for publication and the% 
CNNs utilized were either small or obsolete, having been surpassed in performance by newer models such as ConvNeXt \cite{liu2022convnet} or Vision Transformers (ViTs) \cite{dosovitskiy2020image}.

%%%%%%%%%%%%%%%%%%%%%%%%%%%%%%%%%%%%%%%%%%%%%%%%%%%%%%%%%%%%%%%%%%%%%%%%%%%%%%%%%%%%%%%%%%%%%%%%%%%%%%%%%%%%%%%%%%%%%