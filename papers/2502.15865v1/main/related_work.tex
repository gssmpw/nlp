

\section{Related Work}
\subsection{Language Agents and Applications}

Recent advanced progress of Large Language Models has significantly boosted the development of language agents \citep{sun2024genesis,durante2024agent,feng2024far,wu2024copilot} that could interact with environments to perform complex tasks. Methods such as prompt engineering \citep{yao2022react}, tool or code use \citep{ma2024sciagent,sun2024survey}, self-improvement \citep{hu2023language, cheng2024vision}, multi-model collaboration \citep{wu2023autogen, sun2023corex, jiang2024multimodal,li2023modelscope, jin2024mmtom} or finetuning with trajectories \citep{cheng2024seeclick,wu2024atlas,xie2024osworld,chen2024guicourse} has enabled impressive performances on a wide range of real-world tasks that are involved with web \citep{yao2022webshop,deng2024mind2web}, desktop \citep{kapoor2025omniact, niu2024screenagent} and mobile platforms \citep{li2024effects, wang2024mobile}. With these success in general tasks, language agents are also recently applied into financial domain such as financial question answering \citep{Fatemi_2024}, financial decision making \citep{ding2024largelanguagemodelagent,yang2024finrobotopensourceaiagent,li2024investorbenchbenchmarkfinancialdecisionmaking,yu2024finconsynthesizedllmmultiagent}, and financial simulation \citep{gao2024simulatingfinancialmarketlarge}. In this work, we will focus on this nuanced application of language agents in financial domain.

\begin{figure*}[h!]
    \centering
    \includegraphics[width=0.9\linewidth]{figures/problem.pdf}
    \caption{The limitations of traditional benchmarks for LLM agents in the financial domain. While accuracy-based metrics and investment performance metrics yield high scores in controlled tests, they fail to capture critical safety risks hiding beneath the surface. These hidden risks can lead to unsafe development and real-world failures.}
    \label{fig:limitations}
\end{figure*}
\vspace{-3pt}

\subsection{Evaluation Metrics in Financial Benchmarks}
Existing financial benchmarks primarily focus on task performance, such as accuracy and performance~\citep{xie2024pixiu,yuan-etal-2024-r,islam2023financebench}, which may not be sufficient to capture the real-world financial risks.
The primary evaluation metrics can be categorized into two groups: (1) accuracy-based metrics, and (2) investment performance metrics. The accuracy-based metrics include F1 score, precision, recall, BLEU, ROUGE, METEOR, MSE and MAE~\citep{zhang2024dolares,hirano-2024-construction,quan-liu-2024-econlogicqa,li2024alphafin}. 
The investment performance metrics include Annualized Rate of Return (ARR), Annualized Excess Rate of Return (AERR), Annualized Volatility (ANVOL), Sharpe Ratio (SR), Maximum Drawdown (MD), Calmar Ratio (CR), Maximum Drawdown Duration (MDD), Annualized Volatility (AV)~\citep{li2024investorbenchbenchmarkfinancialdecisionmaking,yu2024finconsynthesizedllmmultiagent,li2024alphafin}. 
While these metrics are useful for basic performance assessment, they fail to account for higher-order risks concerns, which are crucial in real-world financial applications. We summarize the existing financial benchmarks and their potential risks in Table~\ref{tab:financial_benchmarks}.





