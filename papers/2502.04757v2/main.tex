%%%%%%%% ICML 2025 EXAMPLE LATEX SUBMISSION FILE %%%%%%%%%%%%%%%%%

\documentclass{article}


% Recommended, but optional, packages for figures and better typesetting:
\usepackage{microtype}
\usepackage{graphicx}
\usepackage{subfigure}
\usepackage{booktabs} % for professional tables

% hyperref makes hyperlinks in the resulting PDF.
% If your build breaks (sometimes temporarily if a hyperlink spans a page)
% please comment out the following usepackage line and replace
% \usepackage{icml2025} with \usepackage[nohyperref]{icml2025} above.
\usepackage{hyperref}

% Attempt to make hyperref and algorithmic work together better:
\newcommand{\theHalgorithm}{\arabic{algorithm}}

% Use the following line for the initial blind version submitted for review:

%\usepackage{icml2025}

% If accepted, instead use the following line for the camera-ready submission:
\usepackage[accepted]{icml2025}


% For theorems and such
\usepackage{amsmath}
\usepackage{amssymb}
\usepackage{mathtools}
\usepackage{amsthm}
\usepackage{multirow}
\usepackage{adjustbox}
% if you use cleveref..
\usepackage[capitalize,noabbrev]{cleveref}

%%%%%%%%%%%%%%%%%%%%%%%%%%%%%%%%
% THEOREMS
%%%%%%%%%%%%%%%%%%%%%%%%%%%%%%%%
\theoremstyle{plain}
\newtheorem{theorem}{Theorem}[section]
\newtheorem{proposition}[theorem]{Proposition}
\newtheorem{lemma}[theorem]{Lemma}
\newtheorem{corollary}[theorem]{Corollary}
\theoremstyle{definition}
\newtheorem{definition}[theorem]{Definition}
\newtheorem{assumption}[theorem]{Assumption}
\theoremstyle{remark}
\newtheorem{remark}[theorem]{Remark}

% Todonotes is useful during development; simply uncomment the next line
%    and comment out the line below the next line to turn off comments
%\usepackage[disable,textsize=tiny]{todonotes}

% The \icmltitle you define below is probably too long as a header.
% Therefore, a short form for the running title is supplied here:
\icmltitlerunning{Enhanced Language-Image Toxicity Evaluation for Safety}

\begin{document}

\twocolumn[
\icmltitle{ELITE: Enhanced Language-Image Toxicity Evaluation for Safety}

% It is OKAY to include author information, even for blind
% submissions: the style file will automatically remove it for you
% unless you've provided the [accepted] option to the icml2025
% package.

% List of affiliations: The first argument should be a (short)
% identifier you will use later to specify author affiliations
% Academic affiliations should list Department, University, City, Region, Country
% Industry affiliations should list Company, City, Region, Country

% You can specify symbols, otherwise they are numbered in order.
% Ideally, you should not use this facility. Affiliations will be numbered
% in order of appearance and this is the preferred way.
\icmlsetsymbol{equal}{*}

\begin{icmlauthorlist}
\icmlauthor{Wonjun Lee}{equal,1,3}
\icmlauthor{Doehyeon Lee}{equal,2,4}
\icmlauthor{Eugene Choi}{2,5}
\icmlauthor{Sangyoon Yu}{2}
\icmlauthor{Ashkan Yousefpour}{1}
\icmlauthor{Haon Park}{2}
\icmlauthor{Bumsub Ham}{1}
\icmlauthor{Suhyun Kim}{3}
%\icmlauthor{}{sch}
%\icmlauthor{}{sch}
\end{icmlauthorlist}

\icmlaffiliation{1}{Yonsei University}
\icmlaffiliation{2}{AIM Intelligence}
\icmlaffiliation{3}{Korea Institute of Science and Technology}
\icmlaffiliation{4}{Seoul National University}
\icmlaffiliation{5}{Sookmyung Women's University}



\icmlcorrespondingauthor{Suhyun Kim}{dr.suhyun.kim@gmail.com}


% You may provide any keywords that you
% find helpful for describing your paper; these are used to populate
% the "keywords" metadata in the PDF but will not be shown in the document
%\icmlkeywords{Machine Learning, ICML}

\vskip 0.3in
]

% this must go after the closing bracket ] following \twocolumn[ ...

% This command actually creates the footnote in the first column
% listing the affiliations and the copyright notice.
% The command takes one argument, which is text to display at the start of the footnote.
% The \icmlEqualContribution command is standard text for equal contribution.
% Remove it (just {}) if you do not need this facility.

%\printAffiliationsAndNotice{}  % leave blank if no need to mention equal contribution
%\printAffiliationsAndNotice{} % otherwise use the standard text.
\printAffiliationsAndNotice{\icmlEqualContribution}

\begin{abstract}
Current Vision Language Models (VLMs) remain vulnerable to malicious prompts that induce harmful outputs. Existing safety benchmarks for VLMs primarily rely on automated evaluation methods, but these methods struggle to detect implicit harmful content or produce inaccurate evaluations. Therefore, we found that existing benchmarks have low levels of harmfulness, ambiguous data, and limited diversity in image-text pair combinations. To address these issues, we propose the ELITE {\em benchmark}, a high-quality safety evaluation benchmark for VLMs, underpinned by our enhanced evaluation method, the ELITE {\em evaluator}. The ELITE evaluator explicitly incorporates a toxicity score to accurately assess harmfulness in multimodal contexts, where VLMs often provide specific, convincing, but unharmful descriptions of images. We filter out ambiguous and low-quality image-text pairs from existing benchmarks using the ELITE evaluator and generate diverse combinations of safe and unsafe image-text pairs. Our experiments demonstrate that the ELITE evaluator achieves superior alignment with human evaluations compared to prior automated methods, and the ELITE benchmark offers enhanced benchmark quality and diversity. By introducing ELITE, we pave the way for safer, more robust VLMs, contributing essential tools for evaluating and mitigating safety risks in real-world applications. \\
\textit{\textcolor{red}{Warning: This paper includes examples of harmful language and images that may be sensitive or uncomfortable. Reader discretion is advised.}}


\end{abstract}

\section{Introduction}

Chain-of-Thought (CoT) prompting~\cite{Nye:2021, cot, Kojima:2022cotzero} has emerged as a cornerstone strategy for enhancing Large Language Models (LLMs) in complex reasoning tasks. By eliciting step-by-step inference, CoT enables LLMs to decompose intricate problems into manageable subtasks, thereby improving their problem-solving performance~\cite{Yao:2023tot, Wang:2023self-consistency, Zhou:2023least, Shinn:2023Reflexion}. Recent advancements, such as OpenAI's o1~\cite{o1} and DeepSeek-R1~\cite{deepseekr1}, further demonstrate that scaling up CoT lengths from hundreds to thousands of reasoning steps could continuously improve LLM reasoning. These breakthroughs have underscored CoT’s potential to advance LLM capabilities, expanding the boundaries of AI-driven problem-solving.

\begin{figure}[t]
\centering
    \includegraphics[width=0.95\columnwidth]{fig/intro.pdf}
    \caption{In contrast to vanilla CoT that generates all reasoning tokens sequentially, \method enables LLMs to \textit{skip} tokens with less semantic importance (\textit{e.g.,} \includegraphics[width=7pt]{fig/token.pdf}~) and learn shortcuts between critical reasoning tokens, facilitating controllable CoT compression.}
    \label{fig:intro}
\end{figure}

Despite its effectiveness, the increased length of CoT sequences introduces substantial computational overhead. Due to the autoregressive nature of LLM decoding, longer CoT outputs lead to proportional increases in both inference latency and memory footprints of key-value cache. Additionally, the quadratic computational cost of attention layers further exacerbates this burden. These issues become particularly pronounced when CoT sequences extend into thousands of reasoning steps, resulting in significant computational costs and prolonged response times. While prior research has explored methods for selectively skipping reasoning steps~\cite{Ding:2024cotshortcut, liu2024skipstep}, recent findings~\cite{jin:2024cotlength, Merrill:2024cotlength} suggest that such reductions may conflict with test-time scaling~\cite{o1-blog, snell2025scaling}, ultimately impairing LLM reasoning performance. Therefore, striking an optimal balance between CoT efficiency and reasoning accuracy remains a critical open challenge.

In this work, we delve into CoT efficiency and seek the answer to an important question: \textit{``Does every token in the CoT output contribute equally to deriving the answer?''} We empirically analyze the semantic importance of tokens within CoT outputs and reveal that their contributions to the reasoning performance vary, as depicted in Figure 2. Building on this insight, we introduce \method, a simple yet effective approach that enables LLMs to \textit{skip} less important tokens within CoT sequences and learn shortcuts between critical reasoning tokens, thereby allowing for controllable CoT compression with adjustable ratios. Specifically, as shown in Figure~\ref{fig:intro}, \method constructs compressed CoT training data with various compression ratios, by pruning unimportance tokens from original LLM CoT trajectories. Then, it conducts a general supervised fine-tuning process on target LLMs with this training data, facilitating LLMs to automatically trim redundant tokens during reasoning.

We conduct extensive experiments across various models, including LLaMA-3.1-8B-Instruct and the Qwen2.5-Instruct series, using two widely recognized math reasoning benchmarks: GSM8K and MATH-500. The results validate the effectiveness of \method in compressing CoT outputs while maintaining robust reasoning performance. Notably, Qwen2.5-14B-Instruct exhibits almost \textbf{NO} performance drop (less than $0.4\%$) with a $\bm{40\%}$ reduction in token usage on GSM8K. On the challenging MATH-500 dataset, LLaMA-3.1-8B-Instruct effectively reduces CoT token usage by $\bm{30}\%$ with a performance decline of less than $4\%$, resulting in a $\bm{1.4}\times$ inference speedup. Further analysis underscores the coherence of \method in specified compression ratios and its potential scalability with stronger compression techniques.

\method is distinguished by its low training cost. For Qwen2.5-14B-Instruct, \method fine-tunes only 0.2\% of the model's parameters using LoRA. The size of the compressed CoT training data is no larger than that of the original training set, with 7,473 examples in GSM8K and 7,500 in MATH. The training is completed in approximately 2 hours for the 7B model and 2.5 hours for the 14B model on two 3090 GPUs. These characteristics make \method an efficient and reproducible approach, suitable for use in efficient and cost-effective LLM deployment.

To sum up, our key contributions are:
\begin{enumerate}
    \item To the best of our knowledge, this work is the \textit{first} to investigate the potential of enhancing CoT efficiency through \textit{token skipping}, inspired by the varying semantic importance of tokens in CoT trajectories of LLMs.
    \item We introduce \method, a simple yet effective approach that enables LLMs to skip redundant tokens within CoTs and learn shortcuts between critical tokens, facilitating CoT compression with adjustable ratios.
    \item Our experiments validate the effectiveness of \method. When applied to Qwen2.5-14B-Instruct, \method reduces reasoning tokens by $40\%$ (from 313 to 181) on GSM8K, with less than a $0.4\%$ performance drop.
\end{enumerate}

\paragraph{Uncertainty-based hallucination detection methods.}
Various approaches have been proposed to detect hallucinated content in LLMs generation.
Unlike other methods that require external knowledge sources for fact-checking~\citep{gou2024critic, chen-etal-2024-complex, min-etal-2023-factscore, huo2023retrieving}, uncertainty-based approaches are reference-free and rely only on LLM internal states or behaviors to determine hallucination~\citep{10.1145/3703155}. 
For instance, sampling-based approaches generate multiple responses and measure the diversity in meaning among them~\citep{fomicheva-etal-2020-unsupervised, kuhn2023semantic, lin2024generating}, while density-based approaches approximate the training data distribution and provide probabilities or unnormalized scores to assess how likely a generated response belongs to the distribution~\citep{yoo-etal-2022-detection, ren2023outofdistribution, vazhentsev-etal-2023-hybrid}.

In this paper, we focus on uncertainty quantification methods that rely on token-level likelihood or entropy~\citep{guerreiro-etal-2023-looking, malinin2021uncertainty}. 
Recent works have explored refining likelihood estimation by incorporating semantic relationships or reweighting token importance. For instance, Claim-Conditioned Probability (CCP)~\citep{fadeeva-etal-2024-fact} was introduced to recalculate likelihood according to semantical equivalence; while \citet{zhang-etal-2023-enhancing-uncertainty} and \citet{duan-etal-2024-shifting} adjust token weights to better convey meaning in uncertainty aggregation. \emph{Although these approaches leverage token-level information, they are typically evaluated at the sentence level, raising questions about their reliability}. To address this, we conduct a comprehensive analysis of entity-level hallucination detection for finer-grained performance insights.


\paragraph{Fine-grained hallucination detection benchmark.}

Most hallucination detection benchmarks are in sentence or paragraph level. For example, CoQA~\citep{reddy-etal-2019-coqa}, TriviaQA~\citep{joshi-etal-2017-triviaqa}, TruthfulQA~\citep{lin-etal-2022-truthfulqa}, and HaluEval~\citep{li-etal-2023-halueval}. These benchmarks classify each generated response as either hallucinated or correct. However, instance-level detection cannot pinpoint specific hallucinated content, which is crucial for correcting misinformation~\citep{cattan2024localizingfactualinconsistenciesattributable}. This limitation becomes particularly problematic in long-form text, where a single response often combines supported and unsupported information, making binary quality judgments inadequate~\citep{min-etal-2023-factscore}.

To address these challenges, recent works have advanced benchmarks for more granular hallucination detection. For example, \citet{min-etal-2023-factscore} introduced \textsc{FActScore}, which decomposes LLM-generated text into atomic facts---short sentences conveying a single piece of information---for more precise evaluation. In parallel, \citet{cattan2024localizingfactualinconsistenciesattributable} introduced \textsc{QASemConsistency}, decomposing LLM generated text with QA-SRL, a semantic formalism, to form simple QA pairs, where each QA pair represent one verifiable fact. \emph{However, these methods do not enable entity-level hallucination detection, as they lack explicit entity-level labeling (hallucinated or not) in the original generated text}.  
Beyond decomposition-based approaches, datasets like \textsc{HaDes}~\citep{liu-etal-2022-token} and CLIFF~\citep{cao-wang-2021-cliff} create token-level hallucinated content by perturbing human-written text, allowing token-level annotation on the same text. These perturbed hallucinated content, however, could be unrealistic, biased, and overly synthetic due to the limitations of models they used to perturb words. 
To bridge this gap, we create a new dataset with entity-level hallucination labels on the same LLMs generated text. This allows us to evaluate uncertainty-based hallucination detection approaches on a finer-grained level and analyze their reliability.






\section{ELITE}
\label{sec:method}
% In this section, we discuss the limitations of the previous evaluation methods, highlighting their inability to effectively assess the safety of VLM responses, and introduce ELITE evaluator as an accurate evaluation method. Next, we describe the construction process of our dataset designed to induce harmful responses from VLMs. As shown in Table \ref{table1}, we provide a detailed breakdown of the components of the ELITE benchmark constructed based on the ELITE evaluator, along with the criteria used for filtering out image-text pairs.

In this section, we introduce the ELITE evaluator as an accurate evaluation method. Also, we describe the construction process of the ELITE benchmark, along with the creation of in-house image-text pairs, which are designed to induce harmful responses from VLMs. As shown in Table \ref{table1}, we provide a detailed breakdown of the components of the ELITE benchmark.


\subsection{The ELITE Evaluator}
Existing benchmarks use human evaluators or automated evaluators for the safety evaluation of VLMs. Relying on human evaluators is expensive, and thus recent approaches have leveraged automated annotators using LLMs or VLMs. However, existing evaluation methods assume that an attack is successful if the victim model does not output any predefined refusal messages~\cite{chi2024llamaguardvision, mmsafetybench2025}, causing inaccurate safety evaluation results. In order to address this problem, we introduce the ELITE evaluator, an accurate and structured rubric-based evaluation method.

The ELITE evaluator is built upon the StrongREJECT evaluator~\cite{strongreject2024}, which highlighted the overestimation of most basic automated evaluators in the jailbreak techniques of LLMs. The StrongREJECT evaluator is defined as follows: 
\begin{equation}
\text{StrongREJECT} = (1 - \textit{refused}) \times \frac{\textit{specific} + \textit{convincing}}{2}
\label{equation1}
\end{equation} 

In Eq.~\ref{equation1}, \textit{refused} evaluates whether the model has refused the forbidden prompt, and assigns 1 point if the victim model refuses to respond. \textit{specific} measures the specificity of the response, while \textit{convincing} evaluates its convincingness. Both terms are scored on a scale of 1 to 5.

In VLMs, the victim model's responses to forbidden prompts can vary as images are added. For instance, as observed in the responses in Fig.~\ref{figure2}, despite malicious prompts, the victim model may produce responses that are unhelpful to the attacker. However, when the StrongREJECT evaluator is applied to VLMs, it often assigns high scores even when the model does not explicitly refuse to respond and instead provides unhelpful answers. In other words, since the victim model provides a response instead of refusing, a score of 0 is assigned to \textit{refused}, while moderate scores are given for \textit{specific} and \textit{convincing}, resulting in a final score that indicates the response is harmful. This makes it challenging to determine whether the attempt to induce harmful outputs was genuinely successful. To address this problem, we incorporate \textit{toxicity} into the evaluation rubric to assess how harmful the victim model's response is. Through this approach, \textit{toxicity} plays a crucial role in evaluating the success of inducing harmful responses, addressing the limitations of relying solely on \textit{refused} to determine whether the victim model's response is harmful or not. Accordingly, the ELITE evaluator is defined as follows:
\begin{equation}
\resizebox{\columnwidth}{!}{
$\text{ELITE} = (1 - \textit{refused}) \times 
\frac{\textit{specific} + \textit{convincing}}{2} \times \textit{toxicity}$
}
\label{equation2}
\end{equation}

In Eq.~\ref{equation2}, the ELITE evaluator introduces \textit{toxicity} as an additional criterion, scored on a scale of 0 to 5.

% As shown in Fig.~\ref{Figure2}, when the StrongREJECT is applied to VLMs, it often assigns high scores, even when the model provides simple image descriptions or does not explicitly refuse a response, making it difficult to determine whether the jailbreak was truly successful. SF-Scoring, on the other hand, introduces \textit{toxicity} as an additional criterion, with a scale ranging from [0-5]. In VLMs, the impact of jailbreaks varies depending on the given image. To address this variability, the \textit{toxicity} plays a crucial role in determining the success of a jailbreak, complementing the limitations of relying solely on the \textit{refused}. 

% In VLMs, the victim model's responses to forbidden prompts can vary depending on the image. For instance, as observed in the responses in Fig.~\ref{figure2}, despite malicious inputs, the victim model may produce responses that are unhelpful to the attacker. However, when StrongREJECT is applied to VLMs, it often assigns high scores if the model does not explicitly refuse the response and provides unhelpful answers such as image descriptions, as seen in Fig.~\ref{figure2}. In other words, since the victim model provides a response instead of refusing, a score of 0 is assigned to \textit{refused}, while scores are given for \textit{specific} and \textit{convincing}, resulting in a final score that indicates the response is harmful. This makes it challenging to determine whether the attempt to induce harmful outputs was genuinely successful. To address this problem, we incorporate \textit{toxicity} to evaluate how harmful the victim model's response is. Through this approach, \textit{toxicity} plays a crucial role in evaluating the success of a jailbreak, addressing the limitations of relying solely on \textit{refused} to determine whether the victim model's response is harmful or not.

\begin{figure}[t!]
\centering
\includegraphics[width=1.0\columnwidth]{Figure/figure2.pdf}
\caption{Examples of safety evaluations about the victim model's response by ELITE and StrongREJECT evaluator. $r$, $s$, $c$, and $t$ represent refused, specific, convincing, and toxicity, respectively. The ELITE evaluator can effectively evaluate utilizing the toxicity score to make more accurate judgments.}
\label{figure2}
\end{figure}

\subsection{Taxonomy}
We construct a taxonomy of 11 hazard categories to establish a comprehensive and extensive safety benchmark, aligning with AILuminate v1.0~\cite{mlcommons_1.0} standards. The taxonomy includes Violent Crimes (\textbf{S1}), Non-Violent Crimes (\textbf{S2}), Sex Crimes (\textbf{S3}), Defamation (\textbf{S4}), Specialized Advice (\textbf{S5}), Privacy (\textbf{S6}), Intellectual Property (\textbf{S7}), Indiscriminate Weapons (\textbf{S8}), Hate (\textbf{S9}), Self-Harm (\textbf{S10}), and Sexual Content (\textbf{S11}). Responses that enable, encourage, or endorse harmful behavior are flagged as unsafe for their respective categories. Detailed definitions of each category is provided in Appendix~\ref{sec:taxonomy}.

\subsection{Image\&Text Creation}
We refer to the in-house generated image-text pairs as the ELITE benchmark (generated). As shown in Fig.\ref{Figure1}(b), the ELITE benchmark (generated) is composed of four methods—Role Playing, Fake News, Blueprint, and Flowchart—applied across various taxonomies to elicit harmful responses from the victim model. Note that while certain methods, such as Blueprint and Fake News, are used only in specific taxonomies (e.g., Indiscriminate Weapons and Defamation), others, like Flowchart and Role Playing, are applied more broadly across all taxonomies. Detailed examples of these methods are provided in Appendix~\ref{supple:samples for generated}.

To generate image-text pairs, we use the following methods: \\
\textbf{(1) Image Generation}: For Role Playing, Blueprint, and Flowchart, we use image generation models such as Flux AI~\cite{flux2023} and Grok 2~\cite{grok2} to create images that align with the key concepts of each taxonomy. Specifically, we first extract relevant keywords for each taxonomy and use these keywords as prompts to generate corresponding images. For Fake News, we manually synthesize these images to create outputs that align with the intended misinformation scenarios, using the open-source image dataset CelebA~\cite{celeba}. \\
\textbf{(2) Text Generation}: We generate an initial forbidden text prompt by creating keywords relevant to the image and taxonomy, then generate multiple variations of the prompt using Grok 2. To identify the most effective forbidden text prompt for the given image, we evaluate responses from three victim models (Phi-3.5-Vision, Llama-3.2-11B-Vision, and Pixtral-12B). Among the models that produce harmful responses, we select the image-text pair with the highest ELITE evaluator score to finalize its construction.

These image-text pairs are explicitly designed to induce harmful responses from VLMs, enabling a comprehensive safety evaluation.  As shown in Table~\ref{table2}, we incorporate 593 safe-safe pairs into the ELITE benchmark (generated) by embedding inherently harmful intents. These pairs can still induce unsafe responses from VLMs, making them crucial for evaluating safety. Through this, we aim to develop a more extensive benchmark that effectively captures potential vulnerabilities in VLMs. 

\begin{table}[t!]
\caption{
The distribution of the four image-text pair types (unsafe-unsafe, safe-unsafe, unsafe-safe, and safe-safe) in the ELITE benchmark (generated).}
\begin{center}
\resizebox{1.0\columnwidth}{!}{
\begin{tabular}{ccccc}
\toprule
\multicolumn{4}{c}{\textbf{ELITE benchmark (generated)}} & \multirow{2}{*}{\textbf{Total}} \\ \cmidrule{1-4}
safe-safe   & safe-unsafe  & unsafe-safe  & unsafe-unsafe  &                                 \\ \midrule
    593        &    69          &    350          &        42        & 1054                            \\ \bottomrule
\end{tabular}}
\label{table2}
\end{center}
\end{table}


\subsection{Benchmark Construction Pipeline}
As shown in Fig.~\ref{figure}, the steps for constructing the ELITE benchmark are as follows: \\
\textbf{(1) Taxonomy Alignment}: To align the image-text pairs in existing benchmarks with the taxonomy of the ELITE benchmark, we use GPT-4o to classify image-text pairs into their corresponding taxonomies within the ELITE benchmark. \\
\textbf{(2) Filtering}: We apply a filtering process based on a defined threshold to both existing benchmarks and the ELITE benchmark (generated). Specifically, on the ELITE evaluator's [0-25] point scale, we set a threshold determined by human judgment. ELITE evaluator score $s\geq 10$ indicates that the victim model's response is sufficiently harmful, while $s< 10$ indicates that the victim model either refused to respond to the forbidden prompt or provided a non-harmful response. Using this threshold, we primarily include image-text pairs in the ELITE benchmark if at least two out of the three victim models (Phi-3.5-Vision, Llama-3.2-11B-Vision, and Pixtral-12B) achieve a score of $s\geq 10$ to prevent over-reliance on a single model during filtering. However, in cases where a single model's response is deemed sufficiently harmful, pairs meeting the threshold with only one model are also included.  Examples of model responses near our threshold are provided in Appendix~\ref{threshold}. \\ 
\textbf{(3) Balancing the Taxonomy}: After filtering, we identify that some benchmarks are overly concentrated in specific taxonomies (e.g., 204 image-text pairs in VLGuard are filtered into the S9. Hate), leading to imbalance across taxonomies. To create a more balanced benchmark, we additionally filter JailbreakV-28k~\cite{jailbreak28k2024} for only non-concentrated categories. Also, to address the issue of certain taxonomies being overly dependent on specific benchmarks, We exclude image-text pairs with the lowest combined ELITE evaluator scores from the three models.

\begin{figure}[t!]
\centering
\includegraphics[width=1.0\columnwidth]{Figure/figure3.pdf}
\caption{The pipeline for constructing ELITE benchmark. 1) Taxonomy Alignment: Align the image-text pairs in existing benchmarks with the taxonomy of the ELITE benchmark. 2) Filtering: Integrate only image-text pairs where at least two out of three model responses assign an ELITE evaluator score of 10 or higher. 3) Balancing the Taxonomy: Remove image-text pairs with the lowest combined ELITE evaluator score from overly concentrated taxonomies to maintain balance across taxonomies after filtering.}
\label{figure}
\end{figure}

\begin{table*}[ht!]
\caption{ELITE evaluator score-based ASR of various VLMs across taxonomies. The upper group in the table represents proprietary models, and the lower group represents open-source models. The most vulnerable model is highlighted in \textbf{bold} and the second-most vulnerable with an \underline{underline}.}
\begin{center}
\resizebox{1.0\textwidth}{!}{
\begin{tabular}{c|ccccccccccc|c}
\toprule
\textbf{Model}          & \textbf{S1} & \textbf{S2} & \textbf{S3} & \textbf{S4} & \textbf{S5} & \textbf{S6} & \textbf{S7} & \textbf{S8} & \textbf{S9} & \textbf{S10} & \textbf{S11} & \textbf{Average} \\ 
\midrule
GPT-4o         & 16.39 & 17.51 & 12.74 & 20.30 & 33.23 & 14.38 & 7.38 & 17.36 & 8.66 & 11.59 & 13.91 & 15.67    \\ %완료 
GPT-4o-mini     & 29.47 & 32.91 & 18.79 & 31.58 & 44.41 & 25.24 & 18.03 & 29.48 & 18.05 & 28.48 & 33.73 & 28.23    \\ %완료
Gemini-2.0-Flash & 58.44 & 70.73 & 48.09 & 51.63 & 50.76 & 55.59 & 51.37 & 71.07 & 42.17 & 47.68 & 48.52 & 55.37 \\
Gemini-1.5-Pro   & 37.75 & 48.04 & 28.03 & 40.35 & 37.76 & 33.87 & 50.55 & 44.63 & 23.76 & 27.48 & 35.21 & 37.69 \\
Gemini-1.5-Flash & 43.21 & 56.16 & 22.93 & 40.60 & 39.27 & 37.70 & 50.82 & 47.38 & 30.57 & 23.51 & 37.87 & 40.70 \\
\midrule
LLaVa-v1.5-7B  & 67.38 & 79.13 & 72.93 & 51.38 & 46.83 & 68.05 & 63.39 & 66.94 & 51.57 & 64.90 & 56.80 & 63.59    \\ %완료
LLaVa-v1.5-13B & \underline{72.85} & 86.69 & \textbf{79.94} & 53.63 & 54.98 & 73.48 & 68.31 & 72.45 & 58.56 & \underline{74.17} & 60.65 & \underline{69.68}    \\  %완료 
DeepSeek-VL-7B & 38.41 & 59.94 & 31.21 & 34.59 & 42.90 & 43.45 & 42.62 & 54.27 & 37.02 & 35.43 & 31.95 & 42.36    \\ %완료
DeepSeek-VL2-Small & 65.07 & 81.93 & 59.24 & 41.35 & \underline{58.01} & 68.69 & 59.29 & 70.25 & 52.12 & 53.64 & 42.31 & 60.95    \\ %완료
ShareGPT4V-7B & 68.71 & 86.41 & 75.16 & 48.62 & 53.78 & 72.52 & 71.04 & 64.74 & \underline{60.96} & 65.56 & 56.51 & 67.16    \\ %완료
ShareGPT4V-13B & 71.03 & \underline{87.54} & 75.16 & 51.38 & 56.80 & \underline{74.76} & \underline{73.22} & 66.39 & 60.41 & 62.91 & 52.96 & 68.08 \\ % 완료
Phi-3.5-Vision  & 37.58 & 44.40 & 16.24 & 49.87 & 38.07 & 25.24 & 21.86 & 41.05 & 18.60 & 23.18 & 18.34 & 31.85   \\  % 완료
Pixtral-12B  & \textbf{75.50} & \textbf{93.56} & \underline{77.07} & \textbf{67.17} & \textbf{61.63} & \textbf{79.23} & \textbf{86.61} & \textbf{90.08} & \textbf{82.50} & \textbf{77.15} & \textbf{74.56} & \textbf{79.86} \\ %완료
Llama-3.2-11B-Vision  & 54.47 & 69.05 & 41.40 & 30.83 & 55.29 & 53.35 & 33.88 & 55.37 & 34.44 & 43.05 & 39.05 & 47.94   \\ %완료
Qwen2-VL-7B & 57.28 & 70.73 & 45.22 & 38.60 & 47.73 & 60.06 & 40.44 & 66.67 & 45.49 & 54.64 & 50.00 & 53.72 \\ %완료
Molmo-7B& 61.09 & 81.51 & 62.42 & \underline{56.14} & 51.96 & 57.19 & 71.31 & \underline{75.21} & 47.70 & 64.90 & \underline{63.61} & 63.79 \\
InternVL2.5-8B& 51.32 & 65.83 & 60.83 & 23.81 & 50.76 & 49.52 & 36.61 & 55.65 & 27.62 & 43.71 & 36.98 & 46.48 \\
InternVL2.5-26B & 37.75 & 47.48 & 42.36 & 27.82 & 45.62 & 34.82 & 21.58 & 50.41 & 23.02 & 34.77 & 28.99 & 36.21 \\

\bottomrule
\end{tabular}}
\label{table3}
\end{center}
\end{table*}

% As shown in Fig.~\ref{figure}, through these steps, we remove ambiguous samples, such as those that fail to elicit harmful responses from VLMs, and filter samples from previous benchmarks and ELITE benchmark (generated samples) to try to make the ELITE benchmark more diverse.

\section{Experiments}

\subsection{Experiment Setup}
We evaluate the effectiveness of the ELITE benchmark, consisting of 4,587 image-text pairs, across various VLMs, including GPT-4o, GPT-4o-mini, Gemini-2.0, Gemini-1.5, and open-source models. For open-source models, their original hyperparameters are used. We use GPT-4o as the ELITE evaluator to evaluate the safety of VLMs.

% , including GPT-4o, GPT-4o-mini, Gemini-2.0-Flash, Gemini-1.5 (Pro, Flash), and open-source models such as LLaVa-v1.5 (7B and 13B), DeepSeek (VL-7B, VL2-Small), ShareGPT4V (7B and 13B), Phi-3.5-Vision, Pixtral (12B), Llama-3.2-Vision (11B), Qwen2-VL (7B), Molmo (7B), and InternVL2 (8B and 26B)

\subsection{Metric}
In the Experiments section, we use the ELITE evaluator score-based Attack Success Rate (E-ASR) for comparison. E-ASR is defined as:
\begin{equation}
\text{E-ASR} = \frac{\left| \{ i \mid \text{ELITE evaluator score}_i \geq 10 \} \right|}{N} \times 100,
\label{equation3}
\end{equation}

where \(\text{ELITE evaluator score}_i\) represents the ELITE evaluator score of the \(i\)-th image-text pair and \(N\) is the total number of image-text pairs.

\subsection{Evaluation of the ELITE Benchmark}
In Table~\ref{table3}, we present comprehensive experimental results of the ELITE benchmark across various proprietary and open-source VLMs. GPT-4o exhibits the lowest E-ASR at 15.67\% among models, indicating that it is appropriately safety-aligned against malicious inputs. In contrast, Gemini-2.0-Flash exhibits the highest E-ASR among proprietary models at 55.37\%, indicating significant vulnerability to malicious attacks. Additionally, with a few exceptions, most open-source models show high success rates for malicious attacks. The result that most models exhibit an E-ASR exceeding 40\% highlights the need for improved safety alignment in VLMs.

\begin{table}[t!]
\caption{Comparison of the average E-ASR when using different benchmarks. It highlights that the most effective benchmark for inducing harmful responses in \textbf{bold} and the second-most effective benchmark with an \underline{underline}.}
\begin{center}
\resizebox{1.0\columnwidth}{!}{
\begin{tabular}{cccc}
\toprule
\textbf{Model}  & \textbf{Benchmark}  & \textbf{Total}  & \textbf{Average}  \\ 
\midrule
\multicolumn{1}{c|}{\multirow{5}{*}{LLaVa-v1.5-7B}}  & VLGuard        & 2028          & 27.75           \\
\multicolumn{1}{c|}{}                                & MM-SafetyBench & 1680          & 45.06           \\
\multicolumn{1}{c|}{}                                & MLLMGuard      & 532           & 27.26        \\
\multicolumn{1}{c|}{}                                & ELITE benchmark (generated)   & 1054 & \textbf{69.17} \\ 
\multicolumn{1}{c|}{}                                & ELITE benchmark   & 4587 & \underline{63.59} \\ \midrule
\multicolumn{1}{c|}{\multirow{5}{*}{LLaVa-v1.5-13B}} & VLGuard        & 2028          & 28.40           \\
\multicolumn{1}{c|}{}                                & MM-SafetyBench & 1680          & 46.61          \\
\multicolumn{1}{c|}{}                                & MLLMGuard      & 532           & 27.26           \\
\multicolumn{1}{c|}{}                                & ELITE benchmark (generated)   & 1054 & \textbf{78.46} \\ 
\multicolumn{1}{c|}{}                                & ELITE benchmark   & 4587 & \underline{69.68} \\ \midrule
\multicolumn{1}{c|}{\multirow{5}{*}{DeepSeek-VL-7B}} & VLGuard        & 2028          & 16.40           \\
\multicolumn{1}{c|}{}                                & MM-SafetyBench & 1680          & 31.79           \\
\multicolumn{1}{c|}{}                                & MLLMGuard      & 532           & 16.29           \\
\multicolumn{1}{c|}{}                                & ELITE benchmark (generated)   & 1054 & \underline{37.95} \\ 
\multicolumn{1}{c|}{}                                & ELITE benchmark  & 4587 & \textbf{42.36}  \\ \midrule
\multicolumn{1}{c|}{\multirow{5}{*}{ShareGPT4V-7B}} & VLGuard        & 2028          & 29.24           \\
\multicolumn{1}{c|}{}                                & MM-SafetyBench & 1680          & 48.81          \\
\multicolumn{1}{c|}{}                                & MLLMGuard      & 532           & 23.51           \\
\multicolumn{1}{c|}{}                                & ELITE benchmark (generated)  & 1054 & \textbf{68.50} \\ 
\multicolumn{1}{c|}{}                                & ELITE benchmark   & 4587 & \underline{67.16} \\ \bottomrule
\end{tabular}}
\label{table4}
\end{center}
\end{table}


\subsection{Comparisons with Other Benchmarks}
In this section, we demonstrate the superiority of both the ELITE benchmark and the ELITE benchmark (generated). 
Table~\ref{table4} compares the E-ASR of LLaVa-v1.5 (7B, 13B), DeepSeek-VL (7B), and ShareGPT4V (7B) across existing benchmarks, including VLGuard~\cite{vlguard2024}, MM-SafetyBench~\cite{mmsafetybench2025}, MLLMGuard~\cite{mllmguard2024}, the ELITE benchmark, and the ELITE benchmark (generated). Note that we use publicly available benchmarks in this experiment. 

As shown in Table~\ref{table4}, the ELITE benchmark, which contains approximately 2–3 times more evaluation image-text pairs, achieves significantly higher E-ASR across all models. Furthermore, the ELITE benchmark (generated) demonstrates a substantial increase in E-ASR through effective filtering. These experimental results indicate that the low E-ASR observed in existing benchmarks suggests a substantial number of image-text pairs that fail to elicit harmful responses from VLMs. Consequently, this highlights the effectiveness of the ELITE evaluator in filtering out ambiguous image-text pairs, ensuring that only those capable of inducing harmful responses from VLMs are retained.

% To validate SF-Bench (generated dataset), we include experiments on SF-Bench (generated dataset, text-only), which contains only text without images, in Table 4. The experimental results show that the text-only dataset exhibits a lower SF-Scoring-based ASR compared to experiments including images, indicating that VLMs are less vulnerable to jailbreaks without images. These results suggest that the image-text pairs in the generated dataset are more effective in eliciting harmful content from VLMs. This is particularly evident in relatively safety-aligned models such as DeepSeek-VL-7B, which further emphasizes this observation.

Table~\ref{table5} presents the E-ASR of the methods used to elicit harmful responses from VLMs in the ELITE benchmark (generated). Our experimental results show that Flowchart and Blueprint achieve high E-ASR across a significant number of models, underscoring the importance of incorporating these methods into the benchmark to effectively evaluate and enhance the safety and robustness of VLMs.

\begin{table}[t!]
\caption{Comparison of the E-ASR of the proposed methods in the ELITE benchmark (generated).}
\begin{center}
\resizebox{1.0\columnwidth}{!}{
\begin{tabular}{cccccc}
\toprule
\multirow{2}{*}{\textbf{Method}} & \multirow{2}{*}{\textbf{Count}} & \multicolumn{4}{c}{\textbf{Model}}                                         \\ \cmidrule{3-6} 
                        &                        & LLaVa-7B  & LLaVa-13B & DeepSeek-7B & ShareGPT-7B  \\ \midrule
Blueprint               & 100                    & 64.00          & 82.00          & \textbf{85.00} & 65.00          \\
Fake News               & 140                    & \textbf{65.71} & 56.43          & 46.43          & 55.00          \\
Flowchart               & 706                    & 74.79          & \textbf{87.11}          & 31.16          & 77.34 \\
%Lyrics                  & 74                     & 9.46           & \textbf{14.86} & 2.70           & 2.70           \\
Role Playing            & 108                    & 39.47          & \textbf{44.74} & 26.32          & 30.70          \\ \bottomrule
\end{tabular}}
\label{table5}
\end{center}
\end{table}

\section{Human evaluation}
In this section, we explain the human labeling process and the steps involved in constructing the dataset for human evaluation. Through this, we conduct experiments to compare how closely the ELITE evaluator aligns with human judgment compared to existing automated safety evaluator and safeguard models.

\subsection{Human Annotators and Datasets} 
We recruit 22 labelers through the data-labeling company to evaluate responses. The labelers are selected to ensure diversity across gender, age, and occupation, aiming for unbiased labeling. Each labeler is provided with clear instructions on identifying safe and unsafe responses generated by the victim models. They reviewed each image-text-response pair and classified it as either ``safe" or ``unsafe". To ensure reliability, each image-text pair is evaluated by three labelers, and the majority decision is taken as the ground truth. The detailed evaluation process and the content of the instructions are provided in the Appendix~\ref{sec: human evaluation details}. 

The dataset used for human evaluation consists of a subset of image-text pairs, with approximately 90 pairs per taxonomy, totaling 963 pairs. For a fair comparison, instead of using the filtered responses, we also include responses from models that did not meet the filtering criteria among the three models (Phi-3.5-Vision, Llama-3.2-11B-Vision, and Pixtral-12B). Additionally, these pairs were primarily collected where the evaluation results differed between the ELITE evaluator and existing evaluation methods (StrongREJECT evaluator and safeguard models) and were randomly sampled across each taxonomy. To ensure diversity, we excluded image-text pairs that differed only in model responses from the human evaluation dataset.

\begin{figure}[t!]
\centering
\includegraphics[width=1.0\columnwidth]{Figure/figure4.pdf}
\caption{The comparison of AU-ROC curves between the ELITE evaluator and StrongREJECT evaluator on our human evaluation dataset.}
\label{figure3}
\end{figure}

\subsection{Comparison with Existing Evaluation Method}
To demonstrate the superiority of the ELITE evaluator, we compare it with the StrongREJECT evaluator. Fig.~\ref{figure3} shows the comparison using the Area Under the Receiver Operating Characteristic Curve (AU-ROC Curve)~\cite{au-roc}, considering the differences in scoring scales between the two methods. For a fair comparison, both the ELITE and StrongREJECT evaluators are evaluated using the GPT-4o on the human evaluation dataset consisting of 963 image-text pairs.

As shown in Fig.~\ref{figure3}, the StrongREJECT (GPT-4o) achieves an Area Under the Curve (AUC) of 0.46. In contrast, the ELITE evaluator achieves a significantly higher AUC of 0.77, demonstrating that the ELITE evaluator aligns more closely with human judgment. This result indicates the necessity of incorporating a toxicity score for a more accurate and comprehensive safety evaluation in VLMs. Furthermore, it highlights the robustness and superior performance of the ELITE evaluator.

To further demonstrate that the effectiveness of the ELITE evaluator is not solely due to advanced models like GPT-4o, we validate its effectiveness by applying it to open-source models. Specifically, we apply it to InternVL2.5 (7B, 26B) for comparison. Experimental results show that the ELITE evaluator with InternVL2.5 (7B, 26B) achieves AUC values of 0.57 and 0.65, respectively, surpassing the StrongREJECT evaluator with GPT-4o. This finding confirms that the strong performance of the ELITE evaluator is not solely dependent on a competent model.

% \begin{figure}[t!]
% \centering
% \includegraphics[width=0.9\columnwidth]{Figure/figure4.pdf}
% \caption{The comparison of AU-ROC curves between SF-Scoring and LlamaGuard-3-Vision-11B on our human evaluation dataset.}
% \label{Figure3}
% \end{figure}


\begin{table}[t!]
\caption{Performance comparison of the ELITE (GPT-4o), ELITE (InternVL2.5-8B, 26B), ELITE (InternVL2.5-26B),  LlamaGuard3-Vision-11B, LlavaGuard-13B, and OpenAI Moderation API on our human evaluation dataset. The best-performing method is highlighted in \textbf{bold} and the second-best
method with an \underline{underline}.}
\begin{center}
\resizebox{1.0\columnwidth}{!}{%
\begin{tabular}{c|ccccc}
\toprule
\textbf{Method}               & \textbf{Accuracy ($\uparrow$)} & \textbf{Precision ($\uparrow$)} & \textbf{Recall ($\uparrow$)} & \textbf{F1 score ($\uparrow$)} \\ \midrule
ELITE (GPT-4o)          & \textbf{0.726}    & \textbf{0.579}     & \textbf{0.709}  & \textbf{0.637}    \\
ELITE (InternVL2.5-26B) & \underline{0.660} & \underline{0.500} & \underline{0.471} & \underline{0.485} \\
ELITE (InternVL2.5-8B) & 0.609 & 0.416 & 0.376 & 0.395  \\
LlamaGuard3-Vision-11B       & 0.603 & 0.339 & 0.177 & 0.233   \\
LlavaGuard-13B        & 0.536 & 0.331 & 0.361 & 0.346 &  \\
OpenAI Moderation API       & 0.624 & 0.439 & 0.388 & 0.412  \\
\bottomrule
\end{tabular}%
}
\label{table6}
\end{center}
\end{table}

\subsection{Comparison with Safeguard Models} 
We compare the ELITE evaluator with safeguard models, including LlamaGuard3-Vision-11B~\cite{chi2024llamaguardvision}, LlavaGuard-13B~\cite{helff2024llavaguardvlmbasedsafeguardsvision}, and OpenAI Moderation API~\cite{openai2022moderation}. In this experiment, the ELITE evaluator classifies responses with ELITE evaluator score $s \geq 10$ as unsafe and $s < 10$ as safe, following the same criteria used for filtering.

Table~\ref{table6} demonstrates that the ELITE evaluator, when applied to GPT-4o, outperforms LlamaGuard3-Vision-11B in terms of accuracy, precision, recall, and F1 score. Specifically, it achieves 73\% accuracy, representing an improvement of approximately 20.3\% over LlamaGuard3-Vision-11B, 35\% over LlavaGuard-13B, and 16\% over the OpenAI Moderation API. For the F1 score, the ELITE evaluator shows an F1 score of 0.637, which is significantly higher than the others. Furthermore, the ELITE evaluator performs better when applied to the open-source model InternVL2.5 (26B). The superior performance of the ELITE evaluator on open-source models further indicates that its effectiveness is not limited to GPT-4o, highlighting its broader applicability.

\begin{table}[t!]
\caption{Breakdown of F1 score according to taxonomies between ELITE (GPT-4o), LlamaGuard3-Vision-11B, LlavaGuard-13B, and OpenAI Moderation API on our human evaluation dataset.}
\begin{center}
\resizebox{1.0\columnwidth}{!}{%
\begin{tabular}{lccccc}
\toprule
\multicolumn{1}{c}{\multirow{2}{*}{\textbf{Taxonomy}}} & \multicolumn{4}{c}{\textbf{F1 score ($\uparrow$)}}                                 \\ \cmidrule{2-5} 
\multicolumn{1}{c}{}                                   & \textbf{ELITE} & \textbf{LlamaGuard3} & \textbf{LlavaGuard} & \textbf{OpenAI Mod.} \\ \midrule
S1. Violent Crimes                 & \textbf{0.50}       & 0.16                 & 0.31          & 0.43            \\
S2. Non-Violent Crimes             & \textbf{0.61}       & 0.08                 & 0.26          & 0.48            \\
S3. Sex Crimes                     & \textbf{0.62}       & 0.18                 & 0.33          & 0.24            \\
S4. Defamation                     & \textbf{0.62}       & 0.18                 & 0.25          & 0.06            \\
S5. Specialized Advice             & \textbf{0.52}       & 0.09                 & 0.12          & 0.08            \\
S6. Privacy                        & \textbf{0.55}       & 0.16                 & 0.37          & 0.40            \\
S7. Intellectual Property          & \textbf{0.86}       & 0.62                 & 0.54          & 0.70            \\
S8. Indiscriminate Weapons         & \textbf{0.76}       & 0.18                 & 0.57          & 0.56            \\
S9. Hate                           & \textbf{0.66}       & 0.18                 & 0.38          & 0.44            \\
S10. Self-Harm                     & \textbf{0.67}       & 0.20                 & 0.30          & 0.30            \\
S11. Sexual Content                & \textbf{0.52}       & 0.37                 & 0.26          & 0.38            \\ \bottomrule
\end{tabular}}
\label{table7}
\end{center}
\end{table}

Table~\ref{table7} presents the F1 score for each taxonomy on the human evaluation dataset. Our results show that the ELITE evaluator outperforms LlamaGuard3-Vision-11B across all taxonomies. Specifically, safeguard methods tend to show low F1 scores in certain taxonomies. For instance, LlamaGuard3-Vision-11B shows significantly lower F1 scores in taxonomies such as S2. Non-violent Crimes and S5. Specialized Advice. Similarly, the OpenAI Moderation API shows low F1 scores in taxonomies such as S4. Defamation and S5. Specialized Advice. In contrast, the ELITE evaluator exhibits consistently high and balanced performance across all taxonomies. This demonstrates the superiority of the ELITE evaluator and indicates its effectiveness and accuracy in safety evaluation. 





% \subsection{Optimal threshold through human judgment}

% \begin{table}[t!]
% \centering
% \resizebox{0.5\columnwidth}{!}{%
% \begin{tabular}{cc}
% \toprule
% \textbf{SF-Score for threshold} & \textbf{Accuracy} \\ \midrule
% 5                               & 0.661             \\
% 8                               & 0.679             \\
% 10                              & 0.726             \\
% 15                              & 0.727             \\
% 25                              & 0.717             \\ \bottomrule
% \end{tabular}}
% \end{table}

\section{Limitations and Future Work}
Our work demonstrates the potential of code shaping as a novel interaction paradigm, but we acknowledge several limitations.
First, while our evaluation utilized Python as the programming language, its flexibility and dynamic nature make it a suitable testbed for prototyping various programming paradigms, including object-oriented, functional, and procedural styles. \rev{Code shaping is not inherently bound to Python or any specific language, as the ink annotations are not tied to computational semantics. While this suggests it might work with other languages, the user experience might differ and required future work to explore how different programming languages potentially influence the effectiveness and usability of sketch-based code editing.}
% However, Python’s specific syntax and semantics may limit the generalizability of our findings to other programming languages, particularly those with more rigid type systems or different paradigms. 
% Future work should investigate the application of code shaping across a broader range of languages and environments to determine how different language features impact the effectiveness and usability of sketch-based code editing.

Second, the current implementation primarily focuses on small codebases (78 lines of code from scenario two), where the relationship between sketches and corresponding code edits is relatively straightforward. \rev{Sketching to edit larger codebases across multiple files might require the implementation of retrieval-augmented generation~\cite{zhang2023repocoder}. Additionally, resolving downstream and upstream implications of code edits, such as propagating variable renames or function refactorings, would require dependency analysis and incremental static analysis techniques to track and update references across the codebase. Currently, these dependencies are implicitly managed by the AI model, but implementing explicit dependency resolution mechanisms, such as abstract syntax tree (AST) traversal or control flow graph (CFG) augmentation, may be necessary for handling larger, interdependent codebases effectively.} This may further involve developing more sophisticated AI models capable of understanding and interpreting complex sketches that span multiple levels of abstraction or integrating visual modeling tools directly within the code editor. Similarly, \rev{while our system supports multiple files as demonstrated in the scenarios, we did not conduct a comprehensive evaluation or support a single ``sketch'' spanning across multiple interdependent files.} Investigating how code shaping can support multi-file editing, maintain context across files, and handle dependencies effectively will be crucial for extending the applicability of this approach to more complex development tasks.
% Future research should explore techniques to facilitate the mapping between these higher-level semantic components and the actual code blocks. 
% Furthermore, the current implementation of our system is a proof-of-concept prototype rather than a fully integrated development environment (IDE). We intentionally focused on exploring the core concept of code shaping with minimal functionality to understand essential design components. However, future research should consider developing and evaluating a fully functional code editor that supports code shaping, complete with advanced features like version control, real-time collaboration, debugging, and integrated testing tools. This would provide a more realistic assessment of how code shaping can be adopted in professional software development environments and identify additional design considerations that may arise in a full-scale implementation.
Finally, our study provides initial insights into the potential of code shaping, but further investigation is required to understand its long-term impact on programming practices, particularly in terms of code quality, maintainability, and developer productivity. \rev{We define code shaping in the context of code editing, where sketches are not persistent since they are removed once committed changes are accepted or rejected, or manually deleted. Future research could explore whether versioning sketches is a desirable feature. This could be beneficial for other coding activities such as resolving merge conflicts, refactoring, or asynchronous collaboration.}
\section{Conclusion}
We introduced \methodname, an effective training framework defending against MIAs for LLMs. The extensive experiments demonstrate its robustness in protecting privacy while maintaining strong language modeling performance across various datasets and architectures. Although our study focuses on fine-tuning due to computational constraints, \methodname can be seamlessly applied to large-scale pretraining, as done in prior selective pretraining work~\cite{lin2024not}. By categorizing tokens and treating them appropriately, \methodname opens a novel pathway for MIA defense. Future work can explore improved token selection strategies and multi-objective training approaches.

\bibliography{reference}
%\bibliographystyle{unsrt}
\bibliographystyle{icml2025}

\documentclass[journal,compsoc]{IEEEtran}
\usepackage{epsfig}
\usepackage{graphicx}
\usepackage{amsmath}
\usepackage{amssymb}
\usepackage{algorithm, algorithmic}

\usepackage{diagbox}
\usepackage{float}
\usepackage{afterpage}
\usepackage{bm}
\usepackage{subfig}

%\usepackage{tabu}
\usepackage{multirow}
\usepackage{color}
\usepackage{tablefootnote}
\usepackage{adjustbox}
\usepackage{wrapfig}

\usepackage{hyperref}       % hyperlinks
\usepackage{url}            % simple URL typesetting
\usepackage{booktabs}       % professional-quality tables
\usepackage{amsfonts}       % blackboard math symbols
\usepackage{nicefrac}       % compact symbols for 1/2, etc.
\usepackage{microtype}      % microtypography
\usepackage{times}
\usepackage{epsfig}
%\usepackage{tabu}
%\usepackage{overpic}
\usepackage{bbding}
\usepackage{etoolbox}
\usepackage{paralist}
\usepackage{ulem}
\usepackage{tikz}

\usepackage{makecell}

\usepackage{xcolor,colortbl}

% \usepackage[pagebackref=true,breaklinks=true,colorlinks,bookmarks=false]{hyperref}


\newcolumntype{Y}{p{0.5cm}<{\centering}}
\newcommand{\mc}[2]{\multicolumn{#1}{c}{#2}}
\definecolor{Gray}{gray}{0.5}
\definecolor{LightCyan}{rgb}{0.88,1,1}

\newcolumntype{a}{>{\columncolor{Gray}}c}
\newcolumntype{b}{>{\columncolor{white}}c}



\DeclareMathOperator*{\cat}{Cat}


\def\H{\operatorname{H}}
\def\I{\operatorname{I}}
\def\KL{\operatorname{KL}}


\def\etal{\textit{et al}.}
\def\ie{\textit{i.e.}}
\def\eg{\textit{e.g.}}
\def\etc{\textit{etc}}
\def\wrt{\textit{w.r.t. }}

\def\bz{\textcolor{blue}}
\def\xc{\textcolor{red}}
\newcommand{\tb}[1]{\textbf{#1}}
\newcommand{\bc}[1]{\textcolor[RGB]{192,0,0}{#1}}
\newcommand{\rc}[1]{\textcolor{blue}{#1}}
% \newcommand{\rb}[1]{\textcolor{teal}{#1}}
% \newcommand{\bb}[1]{\textcolor{blue}{#1}}
\newcommand{\bb}[1]{\textcolor[RGB]{192,0,0}{\textbf{#1}}}
\newcommand{\rb}[1]{\textcolor{blue}{\textbf{#1}}}
\newcommand{\todo}[1]{{\color{blue}{[TODO: #1]}}}

% \newcommand{\rev}[1]{\textcolor{red}{#1}}
\newcommand{\rev}[1]{{#1}}
\renewcommand{\thefootnote}{\fnsymbol{footnote}}


\normalem
\begin{document}

\title{Rotation-Adaptive Point Cloud Domain Generalization via Intricate Orientation Learning \\
—— Supplementary Material —— }

\author{{Bangzhen~Liu,~Chenxi~Zheng,~Xuemiao~Xu,~Cheng Xu,~Huaidong~Zhang, \\ and~Shengfeng~He,~\IEEEmembership{Senior Member,~IEEE}}

\thanks{ Bangzhen Liu,~Chenxi~Zheng, and~Xuemiao~Xu are with the School of Computer Science and Engineering, South China University of Technology, Guangzhou, China. E-mail: liubz.scut@gmail.com,~cszcx@mail.scut.edu.cn, and~xuemx@scut.edu.cn.}
\thanks{ Cheng Xu is with the Centre for Smart Health, The Hong Kong Polytechnic University, Hong Kong. E-mail: cschengxu@gmail.com}
\thanks{ Huaidong Zhang is with the School of Future Technology, South China University of Technology, Guangzhou, China. E-mail: huaidongz@scut.edu.cn.}
\thanks{ Shengfeng He is with the School of Computing and Information Systems, Singapore Management University, Singapore. E-mail: shengfenghe@smu.edu.sg.}
}

\markboth{IEEE Transactions on Pattern Analysis and Machine Intelligence}%
{Shell \MakeLowercase{\textit{Liu et al.}}: Rotation-Adaptive Point Cloud Domain Generalization via Intricate Orientation Learning}


\maketitle

\IEEEdisplaynontitleabstractindextext

\IEEEpeerreviewmaketitle


\section{More Experimental Results} \label{sec1}

It is worth noting that the three sub-datasets used in PointDA are all category-wise imbalanced, as shown in Table~\ref{table:dataset}, which indicates that the micro-average precision score (\textit{Acc.}) reported by previous studies is inappropriate to assess the generalizability of cross-domain classification. In the main paper, we instead report the results of PointDA in the form of the macro-average precision score (\textit{Avg.}) for a more convincing evaluation. We also report the extra evaluations in the form of \textit{Acc.} in Table~\ref{tab:pointda10_acc} for reference. Our method still outperforms all the competitors in the average metric over the six cross-domain tasks.

\begin{table}[h]
    \caption{{Number of samples for each category in PointDA~\cite{qin2019pointdan}.}}
    \vspace{-2ex}
    \label{table:dataset}
    \scriptsize
    % \begin{center}
    \setlength{\tabcolsep}{0.05cm}{
      \resizebox{0.48\textwidth}{!}{
        \begin{tabular}{c|c|c|c|c|c|c|c|c|c|c|c}
          \hline  & Tub & Bed & Shelf & Case & Chair & Lamp & Monit. & Plant & Sofa & Table & Total\\
          \hline 
          ModelNet & 106 & 515 & 572 & 200 & 889 & 124 & 465 & 240 & 680 & 392 & 4183 \\
          ShapeNet & 599 & 167 & 310 & 1076 & 4612 & 1620 & 762 & 158 & 2198 & 5876 & 17378 \\
          ScanNet & 98 & 329 & 464 & 650 & 2578 & 161 & 210 & 88 & 495 & 1037 & 6110 \\
          \hline
          \end{tabular}
      }
     }
    % \end{center}
    \vspace{-2mm}
  \end{table}
  

\noindent\textbf{Evaluation on Aligned Dataset.} {We additionally implement our method under the traditional aligned data scenario, where the rotation only happens on the z-axis. In this case, we adapt our intricate orientation mining approach to specifically identify the most intricate orientations along the z-axis. 
The comparisons with state-of-the-art 3DDG methods are shown in Table~\ref{tab:align}, where the results of competitors are directly borrowed from their papers. Our method surpasses the baselines on all six tasks, demonstrating its effectiveness. The proposed orientation-aware contrastive training enables the model to gain a more comprehensive understanding of point clouds from various challenging perspectives, thereby enhancing the generalizability of the learned features. We notice that our method is slightly inferior on M$\to$S* and S$\to$S*. Since the orientational shift is our major concern, we do not have a special design for capturing geometric information under self-occlusions. However, in this case, our method still outperforms the two 3DDG methods on three out of the six tasks, while achieving the best average accuracy. Furthermore, the experimental results also reveal the presence of rotational shifts in the aligned data scenes, demonstrating the potential of our method for solving this problem.}

\begin{table}[h] % table for OSDA setting on Office31
    \caption{Comparison of the \textit{Acc.} ($\%$) under the 3D domain generalization setting. The best records are marked in \textbf{bold}.}
    \label{tab:align} 
    \vspace{-3ex}
    \small
    \begin{center}
    \setlength{\tabcolsep}{0.1cm}{
    \resizebox{0.48\textwidth}{!}{
    \begin{tabular}{c|c|c|c|c|c|c|c}
    \hline

    Methods
    &{M$\to$S} & {M$\to$S*} & {S$\to$M} & {S$\to$S*} & {S*$\to$M} & {S*$\to$S} & {Avg}\\

    \hline
    Supervised                      &{93.9} &{78.4} &{96.2} &{78.4} &{96.2} &{93.9} &{89.5}\\
    w/o Adapt                       &{83.3} &{43.8} &{75.5} &{42.5} &{63.8} &{64.2} &{62.2}\\
    \hline
    {Metasets~\cite{huang2021metasets}} &\tb{86.0} &{52.3} &{67.3} &{42.1} &{69.8} &{69.5} &{64.5}\\
    {PDG~\cite{wei2022learning}}        &{85.6} &\tb{57.9} &{73.1} &{50.0} &{70.3} &{66.3} &{67.2}\\
    % \hline
    {Ours}                              &{83.8} &{46.0} &\tb{83.2} &{45.5} &\tb{76.4} &\tb{70.3} &\tb{67.5}\\
    \hline

    \end{tabular}

    }
    }
    \end{center}
    \vspace{-3mm}
\end{table}


\begin{table*}[h] % table for OSDA setting on Office31
    \caption{Comparison of the micro-average precision score \textit{Acc.}~($\%$) under the orientation-aware 3D domain generalization setting. The top 2 records are marked in \bc{red} and \rc{blue}.}
    \label{tab:pointda10_acc} 
    \vspace{-3ex}
    \small
    \begin{center}
    \setlength{\tabcolsep}{0.35cm}{
    \resizebox{1\textwidth}{!}{
    \begin{tabular}{c|c|c|c|c|c|c|c|c}
    \hline

    {Methods}
    &Type &{M$\to$S} & {M$\to$S*} & {S$\to$M} & {S$\to$S*} & {S*$\to$M} & {S*$\to$S} & {AVG} \\

    
    % \cline{2-19}
    % \cline{14-19}  
    % \cmidrule $\pm$ r{2-13}
    % \cmidrule $\pm$ r{14-19}
    \hline
    Supervised                            &\multirow{2}{*}{-}            &{86.6 $\pm$ 6.3}    &{69.6 $\pm$ 3.2}    &{88.4 $\pm$ 14.7}   &{69.6 $\pm$ 3.2}    &{88.4 $\pm$ 14.7}  &{86.6 $\pm$ 6.3}    &{81.5}\\
    w/o Adapt                             &          &{57.9 $\pm$ 15.1}   &{28.7 $\pm$ 5.1}    &{54.1 $\pm$ 8.6}   &{28.8 $\pm$ 4.7}    &{43.0 $\pm$ 4.8}  &{42.3 $\pm$ 6.0}    &{42.5}\\
    \hline   
    VN~\cite{Deng_2021_ICCV}              &\multirow{2}{*}{RE}             &{70.5 $\pm$ 0.0} &{30.6 $\pm$ 0.0} &{66.7 $\pm$ 0.0} &{32.0 $\pm$ 0.0} &{39.4 $\pm$ 0.0} &{44.8 $\pm$ 0.0}   &{47.3} \\  
    SVN~\cite{su2022svnet}               &            &{66.8 $\pm$ 0.6} &{32.3 $\pm$ 0.4}   &{62.0 $\pm$ 0.5} &{30.0 $\pm$ 0.6}    &{38.0 $\pm$ 0.9} &{42.2 $\pm$ 1.1}   &{45.7} \\ 
    EOMP~\cite{luo2022equivariant}        &              &{61.4 $\pm$ 0.8} &{28.1 $\pm$ 0.3}    &{60.5 $\pm$ 0.6} &{37.0 $\pm$ 0.7}  &{27.9 $\pm$ 0.8} &{37.2 $\pm$ 0.9}    &{42.0} \\
    
    \hline
    SPRIN~\cite{you2021prin}              &\multirow{5}{*}{RI}             &{68.2 $\pm$ 0.4} &{30.1 $\pm$ 0.6}   &{71.8 $\pm$ 0.6} &{30.4 $\pm$ 0.6}    &{46.8 $\pm$ 0.6} &{49.3 $\pm$ 0.5}  &{49.4}\\
    RIPCA~\cite{li2021closer}              &            &{70.3 $\pm$ 1.2} &{33.0 $\pm$ 0.7}   &{70.4 $\pm$ 0.9} &{39.1 $\pm$ 1.3}    &\rc{49.9 $\pm$ 1.6} &{50.6 $\pm$ 2.2}  &\rc{52.2}\\
    RIConv++~\cite{zhang2022riconv}        &              &{28.8 $\pm$ 0.6} &{14.2 $\pm$ 0.5}   &{55.1 $\pm$ 0.7} &{38.9 $\pm$ 0.5}    &{34.8 $\pm$ 0.7} &{47.3 $\pm$ 0.5}  &{36.5}\\
    PaRI~\cite{chen2022devil}              &             &{36.1 $\pm$ 0.0} &{29.3 $\pm$ 0.3}   &{51.8 $\pm$ 0.8} &\bc{44.8 $\pm$ 0.4}    &{43.3 $\pm$ 0.9} &{49.4 $\pm$ 0.1} &{42.5} \\
    LocoTrans~\cite{chen2024local}              &                  &\bc{76.7 $\pm$ 0.0}   &{34.5 $\pm$ 0.3}    &\rc{74.3 $\pm$ 0.4}  &\rc{43.6 $\pm$ 0.2}  &{41.6 $\pm$ 0.6} &{41.5 $\pm$ 0.0}    &{52.0}\\
    \hline
    PointDAN~\cite{qin2019pointdan}       &\multirow{5}{*}{DA}              &{59.8 $\pm$ 15.1}   &{29.5 $\pm$ 4.0}    &{55.2 $\pm$ 6.9}   &{24.0 $\pm$ 4.8}    &{38.0 $\pm$ 4.8}  &{47.4 $\pm$ 6.1}    &{42.3}\\
    DefRec~\cite{achituve2021self}        &              &{57.2 $\pm$ 13.3}   &{33.1 $\pm$ 4.6}    &{54.4 $\pm$ 8.0}    &{33.1 $\pm$ 4.4}   &{38.8 $\pm$ 6.5}  &{48.2 $\pm$ 5.7}    &{44.1}\\
    GAST~\cite{zou2021geometry}           &              &{27.7 $\pm$ 4.2}   &{7.0 $\pm$ 0.6}     &{40.8 $\pm$ 2.6}   &{5.8 $\pm$ 0.8}     &{30.7 $\pm$ 1.5}  &{50.7 $\pm$ 3.6}     &{27.1}\\
    MLSP~\cite{liang2022point}            &              &{66.5 $\pm$ 15.5}   &{32.8 $\pm$ 4.3}    &{59.7 $\pm$ 5.1}   &{30.0 $\pm$ 6.4}    &{46.3 $\pm$ 5.0}  &{52.2 $\pm$ 5.8}     &{47.9}\\
    SDDA~\cite{cardace2023self}           &              &{65.0 $\pm$ 14.5}  &\rc{37.8 $\pm$ 3.4}     &{61.4 $\pm$ 5.4}    &{40.1 $\pm$ 4.1}   &{40.7 $\pm$ 6.3}  &\rc{53.3 $\pm$ 6.4}     &{49.7}\\
    PCFEA~\cite{wang2024progressive}     &                   &{62.0 $\pm$ 13.6}   &{9.3 $\pm$ 0.2}   &{42.7 $\pm$ 8.7}   &{43.1 $\pm$ 4.0}   &{47.1 $\pm$ 4.0}   &\bc{54.0 $\pm$ 4.6}    &{43.0}    \\
    \hline
    {Metasets~\cite{huang2021metasets}}   &\multirow{3}{*}{DG}               &{53.9 $\pm$ 1.4} &\bc{40.3 $\pm$ 0.9}   &{32.2 $\pm$ 12.3}  &{33.5 $\pm$ 1.7} &{24.5 $\pm$ 4.6}    &{39.8 $\pm$ 10.0} &{37.4}\\
    {PDG~\cite{wei2022learning}}          &             &{25.4 $\pm$ 29.5}   &{21.2 $\pm$ 18.0}   &{38.4 $\pm$ 18.5}   &{8.1 $\pm$ 3.2}   &{30.3 $\pm$ 4.8}   &{29.7 $\pm$ 12.0}    &{25.5}\\
    % \hline
    {Ours}                                &              &\rb{70.8 $\pm$ 2.0}   &{37.2 $\pm$ 1.2}    &\bb{80.7 $\pm$ 0.6}   &{34.0 $\pm$ 1.0}     &\bb{50.0 $\pm$ 2.5}   &{47.1 $\pm$ 3.2}     &\bb{53.3}  \\
    \hline

    \end{tabular}

    }
    }
    \end{center}
    \vspace{-3mm}
\end{table*}


\noindent\textbf{Analysis of Hyper-parameter Sensitivity.} 
We evaluate the effects of varying $\lambda_{oc}$ and $\lambda_{ms}$, by changing the value while keeping the other frozen as 0.1. As Fig.~\ref{fig:ablation}(a) and Fig.~\ref{fig:ablation}(b) show, $\lambda_{oc}$ is insensitive across a large range, while larger $\lambda_{ms}$ may slightly decrease the performance of our model. According to the variation of performance curves, we choose $\lambda_{oc}=0.01$ and $\lambda_{ms}=0.01$ as the model setting in our main paper.
\begin{figure}[h]
    % \flushleft
    \centering
    \subfloat[Ablation of $\lambda_{oc}$]{%[b]{0.45\textwidth}
        \label{fig:plot_lambda_oc}
        \includegraphics[width=0.22\textwidth]{./resources/supp/ablation_cons_weights.pdf}
    }
    % \hspace{2mm}
    \subfloat[Ablation of $\lambda_{ms}$]{%[b]{0.45\textwidth}
        \label{fig:plot_lambda_ms}
        \includegraphics[width=0.22\textwidth]{./resources/supp/ablation_reg_weights.pdf}
    }
    % \vspace{-2mm}
    \caption{The curves of performance \wrt varying $\lambda_{oc}$ and $\lambda_{ms}$.}
    \label{fig:ablation}
    % \vspace{-5.5mm}
\end{figure} 


\noindent\textbf{Analysis of Training Stability.} 
We plot the curves of the proposed orientation consistency loss and the marginal separation loss over the training stage to demonstrate the convergence of our intricate orientational learning. As Fig.~\ref{fig:plot}(a) and Fig.~\ref{fig:plot}(b) show, all the losses gradually decrease and converge to a convincing degree. The blue curves are the orientation consistency loss, which periodically bursts every 20 epochs. This is due to the update of the intricate orientation set, which gradually adapts the model to all the intricate orientations. At the end of the training stage, the amplification tends to be stable, indicating the consistency of the object towards various rotations.


\begin{figure}[h]
    % \flushleft
    \centering
    \subfloat[M$\to$S]{%[b]{0.45\textwidth}
        \label{fig:m2s_loss}
        \includegraphics[width=0.23\textwidth]{./resources/supp/m2s_loss.pdf}
    }
    % \hspace{2mm}
    \subfloat[M$\to$S*]{%[b]{0.45\textwidth}
        \label{fig:m2ss_loss}
        \includegraphics[width=0.23\textwidth]{./resources/supp/m2ss_loss.pdf}
    }
    \vspace{-2ex}
    \caption{The training curves (\ie, $L_{cls}$, $L_{oc}$, and $L_{ms}$) on M$\to$S (a) and M$\to$S* (b).}
    \label{fig:plot}
    \vspace{-3mm}
\end{figure} 



\noindent\textbf{Analysis of Time Complexity.} {We report the computational costs of training/testing one batch of data in milliseconds for different compared methods in Table~\ref{tab:complexity}. The results are obtained by accumulating the running times within a single training/testing epoch and calculating the mean value w.r.t. one batch.} Due to the process of diversifying the intricate orientation set, our method introduces extra computational costs in the training phase. Nonetheless, our method yields the best performance among these methods while achieving the second-best inferencing speed, which is more efficient than the other RE and RI methods that require extra time-consuming modules for practical applications. 


\begin{table}[h]
    \caption{Time statistics (ms) of training/testing on one batch of data.}
    \label{tab:complexity}
    \vspace{-3ex}
    \small
    \begin{center}
    \setlength{\tabcolsep}{0.3cm}{
    \resizebox{0.4\textwidth}{!}{
      \begin{tabular}{c|c|c|c|c} 
        \hline 
        Methods & Type & Avg. & $T_{train}$& $T_{test}$ \\
        \hline
        VN~\cite{Deng_2021_ICCV} & RE & 41.3 & 808 & 27.9 \\
        SPRIN~\cite{you2021prin} & RI & 43.9 & 1551 & 370.2 \\
        RIPCA~\cite{li2021closer} & RI & 46.6 & 717 & 23.3 \\
        MLSP~\cite{liang2022point} & DA & 43.2 & 825 & 36.5 \\
        SDDA~\cite{cardace2023self} & DA & 43.1 & \bf{567} & \bf{12.0} \\
        \hline
        Ours & DG & \bf{49.6} & 2114 & 14.5 \\
        \hline
        \end{tabular}
    }
    }
    \end{center}
    \vspace{-3mm}
  \end{table}
  


\section{Extra Visualizations and Analysis} \label{sec2}

\noindent\textbf{The Learned Intricate Augmented Samples.} {In Fig.~\ref{fig:intricat_angle}, we select several point clouds and provide visualizations of how their intricate orientations evolve during training. We trained the model on ModelNet and optimized the intricate set on the testing set every 20 epochs. Each row of the point cloud sequence shows the current pose of the given point cloud augmented by its corresponding intricate orientation at that specific epoch. 
Beneath each sequence, we also visualize the distribution of predicted probabilities and the consistency of prediction over different testing orientations. 
Specifically, for each point cloud, we obtain the predicted probabilities of its 64 testing variants $P = {\{P_a|P_a = \left[p^1_a, ..., p^C_a\right]\}}^A_{a=1}$, where $A=64$ is the number of testing orientation series and $C=10$ is the number of categories. 
The visualized probabilities' distribution $P_m$ is calculated by averaging the predictions over the 64 testing rotation series, such that $P_m = \left[\frac{1}{A}\sum_{j=1}^{A}p^1_j, ..., \frac{1}{A}\sum_{j=1}^{A}p^C_j\right]$. 
To evaluate the predicted consistency, we adopt the entropy as the metric and calculate the consistency $Ent_m$ over the 64 predicted probabilities by 
\begin{equation*}
  Ent_m = \left[\frac{1}{A}\sum_{j=1}^{A}p^1_j log p^1_j, ..., \frac{1}{A}\sum_{j=1}^{A}p^C_j log p^C_j\right].
\end{equation*}
As the number of training epochs increases, both the confidence and output consistency of the model are enhanced. For samples located near the decision boundaries, such as row 6 and row 9, learning with intricate orientation mining could significantly alleviate the ambiguity of learned features, thereby producing a more robust and generalizable classifier for downstream tasks.
}




\begin{figure*}[h]
    % \flushleft
    \centering
    \subfloat[Metasets]{%[b]{0.45\textwidth}
        \label{fig:cm1_m2s}
        % \centering
        \includegraphics[width=0.24\textwidth]{./resources/supp/conf_mat_modelnet2shapenet_Metaset.pdf}
        % \vspace{10mm}
    }
    \subfloat[PDG]{%[b]{0.45\textwidth}
        \label{fig:cm2_m2s}
        % \centering
        \includegraphics[width=0.24\textwidth]{./resources/supp/conf_mat_modelnet2shapenet_PDG.pdf}
        % \vspace{10mm}
    }
    \subfloat[Ours]{%[b]{0.45\textwidth}
        \label{fig:cm3_m2s}
        % \centering
        \includegraphics[width=0.24\textwidth]{./resources/supp/conf_mat_modelnet2shapenet.pdf}
        % \vspace{10mm}
    }
    \vspace{-2ex}
    \caption{The confusion matrices of Metaset, PDG, and our method on M$\to$S. Zoom in for details.}
    \vspace{-3ex}
    \label{fig:visualization_m2s}
\end{figure*} 
% \vspace{-4mm}
\begin{figure*}[h]
    % \flushleft
    \centering
    \subfloat[Metasets]{%[b]{0.45\textwidth}
        \label{fig:cm1_s2m}
        % \centering
        \includegraphics[width=0.24\textwidth]{./resources/supp/conf_mat_shapenet2modelnet_Metaset.pdf}
        % \vspace{10mm}
    }
    \subfloat[PDG]{%[b]{0.45\textwidth}
        \label{fig:cm2_s2m}
        % \centering
        \includegraphics[width=0.24\textwidth]{./resources/supp/conf_mat_shapenet2modelnet_PDG.pdf}
        % \vspace{10mm}
    }
    \subfloat[Ours]{%[b]{0.45\textwidth}
        \label{fig:cm3_s2m}
        % \centering
        \includegraphics[width=0.24\textwidth]{./resources/supp/conf_mat_shapenet2modelnet.pdf}
        % \vspace{10mm}
    }
    \vspace{-2ex}
    \caption{The confusion matrices of Metaset, PDG, and our method on S$\to$M. Zoom in for details.}
    \label{fig:visualization_s2m}
\end{figure*} 


\begin{figure*}[h]
    % \flushleft
    \centering
    \subfloat{\label{fig:0}
        \begin{minipage}[b]{1.0\textwidth}\centering
            \includegraphics[width=0.95\textwidth]{./resources/supp/underline.pdf} 
            \\
            \includegraphics[width=0.9\textwidth]{./resources/supp/all_title.pdf} 
            \\
            \includegraphics[width=0.9\textwidth]{./resources/supp/all_cat_0_cropped.pdf} 
            \\
            \vspace{-3mm}
            \includegraphics[width=1.0\textwidth]{./resources/supp/all_predictions_cat_0_cropped.pdf}
        \end{minipage}
    }\vspace{-3mm}

    \subfloat{\label{fig:1}
        \begin{minipage}[b]{1.0\textwidth}\centering
            \includegraphics[width=0.9\textwidth]{./resources/supp/all_cat_1_cropped.pdf} 
            \\
            \vspace{-3mm}
            \includegraphics[width=1.0\textwidth]{./resources/supp/all_predictions_cat_1_cropped.pdf}
        \end{minipage}
    }\vspace{-3mm}

    \subfloat{\label{fig:2}
        \begin{minipage}[b]{1.0\textwidth}\centering
            \includegraphics[width=0.9\textwidth]{./resources/supp/all_cat_2_cropped.pdf} 
            \\
            \vspace{-3mm}
            \includegraphics[width=1.0\textwidth]{./resources/supp/all_predictions_cat_2_cropped.pdf}
        \end{minipage}
    }\vspace{-3mm}

    \subfloat{\label{fig:4}
        \begin{minipage}[b]{1.0\textwidth}\centering
            \includegraphics[width=0.9\textwidth]{./resources/supp/all_cat_3_cropped.pdf} 
            \\
            \vspace{-3mm}
            \includegraphics[width=1.0\textwidth]{./resources/supp/all_predictions_cat_3_cropped.pdf}
        \end{minipage}
    }\vspace{-3mm}

    \subfloat{\label{fig:6}
        \begin{minipage}[b]{1.0\textwidth}\centering
            \includegraphics[width=0.9\textwidth]{./resources/supp/all_cat_4_cropped.pdf} 
            \\
            \vspace{-3mm}
            \includegraphics[width=1.0\textwidth]{./resources/supp/all_predictions_cat_4_cropped.pdf}
        \end{minipage}
    }\vspace{-3mm}

    \subfloat{\label{fig:7}
        \begin{minipage}[b]{1.0\textwidth}\centering
            \includegraphics[width=0.9\textwidth]{./resources/supp/all_cat_5_cropped.pdf} 
            \\
            \vspace{-3mm}
            \includegraphics[width=1.0\textwidth]{./resources/supp/all_predictions_cat_5_cropped.pdf}
        \end{minipage}
    }\vspace{-3mm}

    \subfloat{\label{fig:8}
        \begin{minipage}[b]{1.0\textwidth}\centering
            \includegraphics[width=0.9\textwidth]{./resources/supp/all_cat_6_cropped.pdf} 
            \\
            \vspace{-3mm}
            \includegraphics[width=1.0\textwidth]{./resources/supp/all_predictions_cat_6_cropped.pdf}
        \end{minipage}
    }

    \caption{Visualization of the learned intricate orientation series on ModelNet (M). Each row of point cloud sequence records the transformation of the point cloud's poses after augmention by its corresponding intricate rotational angle during the training procedure. The interval of recording is 20 epoch. The statistic measurements at every records, including the predicted probability and rotational consistency, are presented beneath each point cloud sequence. The category which the current point cloud belongs to is marked in \bc{red}.}
    \label{fig:intricat_angle}
    % \vspace{-6mm}
\end{figure*}



\noindent\textbf{Confusion Matrices.} {we provide the evaluation results for Metasets~\cite{huang2021metasets}, PDG~\cite{wei2022learning}, and our method in the form of confusion matrix on the target domain. ShapeNet is a dataset whose samples are highly imbalanced across different categories, while ModelNet is much more balanced. The confusion matrices of the three approaches are shown in  Fig.~\ref{fig:visualization_m2s} and Fig.~\ref{fig:visualization_s2m}. Compared with the other two 3D domain generalization methods, our method has more compact confusion matrices under the orientation shift. For M$\to$S, both our method and Metaset are separated relatively well while PDG has much inaccurate classification on class "Plant". The inner reason is that the part-based feature utilized by PDG may encounter confusing local expressions, such as the plane of the table and the bottom of a potted plant. For S$\to$M, our method achieves more balanced and concise results. We observe that the sample of class "monitor" is much easier to misclassify into "bed" due to the similar plane structure of their surface. Similar trends happen for the categories "table" and "cabinet", which have less discriminative features in the view of shape. }

{In summary, the single shape cannot serve as a discriminative representation in some cases. This is the limitation of shape representation under the orientation shift since there are a lot of objects whose shapes are similar but belong to different categories. In this case, extra visual (\eg, texture or color), linguistic information, or spatial cues are important to provide complement representation, which may benefit the problem of cross-domain generalization under orientation shift. We will plan to investigate the function of these features in our future work.  }

\section{Gradient of the rotation parameters} \label{sec3}
In this section, we provide detailed calculations about the optimizable parameters $\Theta$ concerning a given model $F$. Considering the objective of optimizing $\Theta$ within a standard classification task, we have the following objective:
\begin{equation}
  \hat{\Theta} = \mathop{\arg\max_{\Theta}}L(w_{opt}, \hat{P}, y), 
\end{equation}
where $w_{opt}$ is the freeze parameter of $F$, $(\hat{P}, y)$ are the augmented point cloud and label:
\begin{equation}
  \begin{split}
    ~\hat{P}&=f(\hat{\Theta}, P)  \\
    &=R_{\theta_{x}}\cdot R_{\theta_{y}}\cdot R_{\theta_{z}}\cdot P.
  \end{split}
\end{equation}
According to the chain rules, the gradient of $\hat{\Theta}$ is calculated by:
\begin{equation}
  \begin{split}
  \frac{\partial L}{\partial \hat{\Theta}} &= \frac{\partial L}{\partial \hat{P}} \frac{\partial \hat{P}}{\partial \hat{\Theta}} \\
  &=\frac{\partial L}{\partial \hat{P}}
  \left(
  \frac{\partial R_{\theta_{x}}}{\partial \theta_{x}}
  R_{\theta_{y}}
  R_{\theta_{z}} \quad
  R_{\theta_{x}}
  \frac{\partial R_{\theta_{y}}}{\partial \theta_{y}}
  R_{\theta_{z}} \quad
  R_{\theta_{x}}
  R_{\theta_{y}}
  \frac{\partial R_{\theta_{z}}}{\partial \theta_{z}}
  \right)P,
  \end{split}
\end{equation}
where 
\begin{equation}
  \begin{split}
R_{\theta_{x}} = 
\begin{pmatrix}
  1 & 0 & 0 \\
  0 & \cos\theta_{x} & -\sin\theta_{x} \\
  0 & \sin\theta_{x} & \cos\theta_{x} 
\end{pmatrix},
\\
R_{\theta_{y}} = 
\begin{pmatrix}
  \cos\theta_{y} & 0 & \sin\theta_{y} \\
  0 & 1 & 0 \\
  -\sin\theta_{y} & 0 & \cos\theta_{y}
\end{pmatrix},
\\
R_{\theta_{z}} = 
\begin{pmatrix}
  \cos\theta_{z} & -\sin\theta_{z} & 0  \\
  \sin\theta_{z} & \cos\theta_{z} & 0 \\
  0 & 0 & 0
\end{pmatrix},
\end{split}
\end{equation}
and 
\begin{equation}
    \begin{split}
\frac{\partial R_{\theta_{x}}}{\partial \theta_{x}} = 
\begin{pmatrix}
  0 & 0 & 0 \\
  0 & -\sin\theta_{x} & -\cos\theta_{x} \\
  0 & \cos\theta_{x} & -\sin\theta_{x} 
\end{pmatrix}, \\
\frac{\partial R_{\theta_{y}}}{\partial \theta_{y}} = 
\begin{pmatrix}
  -\sin\theta_{y} & 0 & \cos\theta_{y} \\
  0 & 0 & 0 \\
  -\cos\theta_{y} & 0 & -\sin\theta_{y}
\end{pmatrix}, \\
\frac{\partial R_{\theta_{z}}}{\partial \theta_{z}} = 
\begin{pmatrix}
  -\sin\theta_{z} & -\cos\theta_{z} & 0  \\
  \cos\theta_{z} & -\sin\theta_{z} & 0 \\
  0 & 0 & 1
\end{pmatrix}.
\end{split}
\end{equation}



\section{Theoretical Analysis for Rotation-Adaptive Point Cloud Domain Generalization} \label{sec4}

In this section, we provide theoretical proof demonstrating how orientational consistency functions to bridge the domain gap, analyzed from the perspective of mutual information reduction.

Let $X\!=\!(U, V)$ represent a 3D point cloud, where $U$ corresponds to orientation-dependent variables and $V$ to orientation-independent variables. In our case, we assume that the ranges of $U$ and $V$ remain consistent across domains.
For $X_s\!\sim\!p_\mathrm{src}(x)$, where $p_\mathrm{src}(x)$ denotes the source domain data distribution, the marginal distributions \wrt $U_s$ and $V_s$ are expressed by:
\begin{equation}
 p_\mathrm{src}(u)=\int p_\mathrm{src}(x) \mathrm{d}v, \quad p_\mathrm{src}(v)=\int p_\mathrm{src}(x) \mathrm{d}u.
\end{equation}
Considering the data distribution $X_a\!\sim\!p_\mathrm{aug}(x)$ after augmentation, where each sample is assumed to be uniformly sampled \wrt orientations, the marginal distributions \wrt $U_a$ and $V_a$ are given by:
\begin{equation}
 p_\mathrm{aug}(u)=\mathcal{U}(\mathcal{D}_{U_a}), \quad p_\mathrm{aug}(v)=p_\mathrm{src}(v),
\end{equation}
where $\mathcal{U}(\cdot)$ denotes a uniform distribution over the measurable domain $\mathcal{D}_{U_a}$ of ${U_a}$. For simplicity, the subscript of $U_a$ in $\mathcal{D}_{U_a}$ is omitted without causing ambiguity in the subsequent analysis. In this work, we adopt the proposed orientation-aware contrastive learning framework to approximately achieve this, where ${U_a}$ is represented by Euler angles and $\mathcal{D}_{U}:=[-\pi, \pi)^3$. 



Based on the definition of joint entropy, the entropy of $p_\mathrm{src}(x)$ can be expressed in terms of its marginal entropies \wrt $U_s$ and $V_s$, along with an additional term presenting the mutual information between these two components:
\begin{equation}
\begin{aligned}
    \H(X_s) 
 =& \H(U_s)+\H(V_s)-\I(U_s;V_s) \\
 =& \mathbb{E}_{u\sim p_\mathrm{src}(u)}[-\log p_\mathrm{src}(u)] + \mathbb{E}_{v\sim p_\mathrm{src}(v)}[-\log p_\mathrm{src}(v)] \\
    &- \mathbb{E}_{x\sim p_\mathrm{src}(x)}\log \frac{p_\mathrm{src}(x)}{p_\mathrm{src}(u)p_\mathrm{src}(v)},
\end{aligned}
\end{equation}
where $\I(U_s;V_s)$ represents the mutual information between $U_s$ and $V_s$ in $p_\mathrm{src}(x)$. 
Since $p_\mathrm{aug}(u)$ follows a uniform distribution and $U_a$ and $V_a$ of $p_\mathrm{aug}(x)$ are independent under this setting, the entropy of $p_\mathrm{aug}(x)$ is given by $\I_\mathrm{aug}(U_a;V_a)\!=\!0$, and the entropy of $p_\mathrm{aug}(u)$ corresponds to the measure of $\mathcal{D}_{U}$, denoted as $m(\mathcal{D}_{U})$. 
Thus, the entropy of $p_\mathrm{aug}(x)$ can be simplified as follows:
\begin{equation}
\begin{aligned}
    \H(X_a) &= \H(U_a)+\H(V_a) \\
    &= \log m(\mathcal{D}_{U}) + \mathbb{E}_{v\sim p_\mathrm{aug}(v)}[-\log p_\mathrm{aug}(v)],
\end{aligned}
\end{equation}
where $m(\mathcal{D}_{U})\!=\!(2\pi)^3$ in our case.

We use the KL divergence to quantify the distributional shift between the source and the target distribution.
For any $X_t\!\sim\!p_\mathrm{tgt}(x)$, where $p_\mathrm{tgt}(x)$ represents the target domain distribution, the KL divergence between $p_\mathrm{tgt}(x)$ and $p_\mathrm{src}(x)$ (or $p_\mathrm{aug}(x)$) can be computed once the cross-entropy between them is known. 
However, directly calculating the cross-entropy between $p_\mathrm{tgt}(x)$ and $p_\mathrm{src}(x)$ (or $p_\mathrm{aug}(x)$) is intractable, and it is often treated as an optimization objective to minimize. Notably, the cross-entropy between $p_\mathrm{tgt}(x)$ and $p_\mathrm{src}(x)$ (or $p_\mathrm{aug}(x)$) shares the same upper bound, as the samples $X_s$, $X_a$, and $X_t$ all share the same dimensionality:
\begin{equation}
    \sup_{p_\mathrm{src}}{\H(p_\mathrm{tgt}, p_\mathrm{src})} = \sup_{p_\mathrm{aug}}{\H(p_\mathrm{tgt}, p_\mathrm{aug})} = \log ({m(\mathcal{D}_U) \times m(\mathcal{D}_V)}).
\end{equation}
Here, $\mathcal{D}_V$ is the measurable domain of $V_s$, $V_a$, and $V_t$.
It is straightforward to prove that $\H_\mathrm{aug}(X_a) > \H_\mathrm{src}(X_s)$, as the mutual information is non-negative and entropy reaches its upper bound when the distribution is uniform.
Therefore, the relation between the upper bound of the KL divergence from $p_\mathrm{tgt}(x)$ to $p_\mathrm{src}(x)$ and from $p_\mathrm{tgt}(x)$ to $p_\mathrm{aug}(x)$ can be expressed as:
\begin{equation}
\begin{aligned}
    \sup_{p_\mathrm{src}}{\KL(p_\mathrm{tgt}||p_\mathrm{src})} &= \sup_{p_\mathrm{src}}{\H(p_\mathrm{tgt};p_\mathrm{src})} - \sup_{p_\mathrm{src}}{\H(X_\mathrm{s})} \\
    &> \sup_{p_\mathrm{aug}}{\H(p_\mathrm{tgt};p_\mathrm{aug})} - \sup_{p_\mathrm{aug}}{\H(X_\mathrm{a})} \\
    &= \sup_{p_\mathrm{aug}}{\KL(p_\mathrm{tgt}||p_\mathrm{aug})}. \label{eq:ieq}
\end{aligned}
\end{equation}
As revealed in Eq.~\ref{eq:ieq}, the upper bound of $\KL(p_\mathrm{tgt}||p_\mathrm{aug})$ is consistently lower than $\KL(p_\mathrm{tgt}||p_\mathrm{src})$, demonstrating the effectiveness of orientation invariance in reducing the domain shift under the disturbance of varying rotations. Consequently, the final upper bound of $\KL(p_\mathrm{tgt}||p_\mathrm{aug})$ is formally given as follows:
\begin{equation}
    \sup_{p_\mathrm{aug}}{\KL(p_\mathrm{tgt}||p_\mathrm{aug})} = \log m(\mathcal{D}_V) - \mathbb{E}_{v\sim p_\mathrm{aug}(v)}[-\log p(v)].
\end{equation}


\section{Limitation and Future Work} 
Although our method shows commendable advantages in handling cross-domain orientational shifts, it faces challenges with other complex types of domain shifts, such as heavy occlusions. This is because our framework does not offer an explicit design for tackling these domain shifts. Addressing this limitation, possibly through constructing a more powerful and versatile feature space resilient to multiple domain shifts via self-supervised pre-training, is a goal for future work.

{\small
\bibliographystyle{ieee_fullname}
\bibliography{egbib}
}

\end{document} 


\end{document}


% This document was modified from the file originally made available by
% Pat Langley and Andrea Danyluk for ICML-2K. This version was created
% by Iain Murray in 2018, and modified by Alexandre Bouchard in
% 2019 and 2021 and by Csaba Szepesvari, Gang Niu and Sivan Sabato in 2022.
% Modified again in 2023 and 2024 by Sivan Sabato and Jonathan Scarlett.
% Previous contributors include Dan Roy, Lise Getoor and Tobias
% Scheffer, which was slightly modified from the 2010 version by
% Thorsten Joachims & Johannes Fuernkranz, slightly modified from the
% 2009 version by Kiri Wagstaff and Sam Roweis's 2008 version, which is
% slightly modified from Prasad Tadepalli's 2007 version which is a
% lightly changed version of the previous year's version by Andrew
% Moore, which was in turn edited from those of Kristian Kersting and
% Codrina Lauth. Alex Smola contributed to the algorithmic style files.
