%%%%%%%% ICML 2025 EXAMPLE LATEX SUBMISSION FILE %%%%%%%%%%%%%%%%%

\documentclass{article}


% Recommended, but optional, packages for figures and better typesetting:
\usepackage{microtype}
\usepackage{graphicx}
\usepackage{subfigure}
\usepackage{booktabs} % for professional tables

% hyperref makes hyperlinks in the resulting PDF.
% If your build breaks (sometimes temporarily if a hyperlink spans a page)
% please comment out the following usepackage line and replace
% \usepackage{icml2025} with \usepackage[nohyperref]{icml2025} above.
\usepackage{hyperref}

% Attempt to make hyperref and algorithmic work together better:
\newcommand{\theHalgorithm}{\arabic{algorithm}}

% Use the following line for the initial blind version submitted for review:

%\usepackage{icml2025}

% If accepted, instead use the following line for the camera-ready submission:
\usepackage[accepted]{icml2025}


% For theorems and such
\usepackage{amsmath}
\usepackage{amssymb}
\usepackage{mathtools}
\usepackage{amsthm}
\usepackage{multirow}
\usepackage{adjustbox}
% if you use cleveref..
\usepackage[capitalize,noabbrev]{cleveref}

%%%%%%%%%%%%%%%%%%%%%%%%%%%%%%%%
% THEOREMS
%%%%%%%%%%%%%%%%%%%%%%%%%%%%%%%%
\theoremstyle{plain}
\newtheorem{theorem}{Theorem}[section]
\newtheorem{proposition}[theorem]{Proposition}
\newtheorem{lemma}[theorem]{Lemma}
\newtheorem{corollary}[theorem]{Corollary}
\theoremstyle{definition}
\newtheorem{definition}[theorem]{Definition}
\newtheorem{assumption}[theorem]{Assumption}
\theoremstyle{remark}
\newtheorem{remark}[theorem]{Remark}

% Todonotes is useful during development; simply uncomment the next line
%    and comment out the line below the next line to turn off comments
%\usepackage[disable,textsize=tiny]{todonotes}

% The \icmltitle you define below is probably too long as a header.
% Therefore, a short form for the running title is supplied here:
\icmltitlerunning{Submission and Formatting Instructions for ICML 2025}

\begin{document}

\twocolumn[
\icmltitle{ELITE: Enhanced Language-Image Toxicity Evaluation for Safety}

% It is OKAY to include author information, even for blind
% submissions: the style file will automatically remove it for you
% unless you've provided the [accepted] option to the icml2025
% package.

% List of affiliations: The first argument should be a (short)
% identifier you will use later to specify author affiliations
% Academic affiliations should list Department, University, City, Region, Country
% Industry affiliations should list Company, City, Region, Country

% You can specify symbols, otherwise they are numbered in order.
% Ideally, you should not use this facility. Affiliations will be numbered
% in order of appearance and this is the preferred way.
\icmlsetsymbol{equal}{*}

\begin{icmlauthorlist}
\icmlauthor{Wonjun Lee}{equal,1,3}
\icmlauthor{Doehyeon Lee}{equal,2}
\icmlauthor{Eugene Choi}{2}
\icmlauthor{Sangyoon Yu}{2}
\icmlauthor{Ashkan Yousefpour}{2}
\icmlauthor{Haon Park}{2}
\icmlauthor{Bumsub Ham}{1}
\icmlauthor{Suhyun Kim}{3}
%\icmlauthor{}{sch}
%\icmlauthor{}{sch}
\end{icmlauthorlist}

\icmlaffiliation{1}{Yonsei University, Republic of Korea}
\icmlaffiliation{2}{AIM Intelligence}
\icmlaffiliation{3}{Korea Institute of Science and Technology, Republic of Korea}

\icmlcorrespondingauthor{Suhyun Kim}{dr.suhyun.kim@gmail.com}


% You may provide any keywords that you
% find helpful for describing your paper; these are used to populate
% the "keywords" metadata in the PDF but will not be shown in the document
\icmlkeywords{Machine Learning, ICML}

\vskip 0.3in
]

% this must go after the closing bracket ] following \twocolumn[ ...

% This command actually creates the footnote in the first column
% listing the affiliations and the copyright notice.
% The command takes one argument, which is text to display at the start of the footnote.
% The \icmlEqualContribution command is standard text for equal contribution.
% Remove it (just {}) if you do not need this facility.

%\printAffiliationsAndNotice{}  % leave blank if no need to mention equal contribution
%\printAffiliationsAndNotice{} % otherwise use the standard text.
\printAffiliationsAndNotice{\icmlEqualContribution}

\begin{abstract}
Current Vision Language Models (VLMs) remain vulnerable to malicious prompts that induce harmful outputs. Existing safety benchmarks for VLMs primarily rely on automated evaluation methods, but these methods struggle to detect implicit harmful content or produce inaccurate evaluations. Therefore, we found that existing benchmarks have low levels of harmfulness, ambiguous data, and limited diversity in image-text pair combinations. To address these issues, we propose the ELITE {\em benchmark}, a high-quality safety evaluation benchmark for VLMs, underpinned by our enhanced evaluation method, the ELITE {\em evaluator}. The ELITE evaluator explicitly incorporates a toxicity score to accurately assess harmfulness in multimodal contexts, where VLMs often provide specific, convincing, but unharmful descriptions of images. We filter out ambiguous and low-quality image-text pairs from existing benchmarks using the ELITE evaluator and generate diverse combinations of safe and unsafe image-text pairs. Our experiments demonstrate that the ELITE evaluator achieves superior alignment with human evaluations compared to prior automated methods, and the ELITE benchmark offers enhanced benchmark quality and diversity. By introducing ELITE, we pave the way for safer, more robust VLMs, contributing essential tools for evaluating and mitigating safety risks in real-world applications. \\
\textit{\textcolor{red}{Warning: This paper includes examples of harmful language and images that may be sensitive or uncomfortable. Reader discretion is advised.}}


\end{abstract}

\section{Introduction}

Deep Reinforcement Learning (DRL) has emerged as a transformative paradigm for solving complex sequential decision-making problems. By enabling autonomous agents to interact with an environment, receive feedback in the form of rewards, and iteratively refine their policies, DRL has demonstrated remarkable success across a diverse range of domains including games (\eg Atari~\citep{mnih2013playing,kaiser2020model}, Go~\citep{silver2018general,silver2017mastering}, and StarCraft II~\citep{vinyals2019grandmaster,vinyals2017starcraft}), robotics~\citep{kalashnikov2018scalable}, communication networks~\citep{feriani2021single}, and finance~\citep{liu2024dynamic}. These successes underscore DRL's capability to surpass traditional rule-based systems, particularly in high-dimensional and dynamically evolving environments.

Despite these advances, a fundamental challenge remains: DRL agents typically rely on deep neural networks, which operate as black-box models, obscuring the rationale behind their decision-making processes. This opacity poses significant barriers to adoption in safety-critical and high-stakes applications, where interpretability is crucial for trust, compliance, and debugging. The lack of transparency in DRL can lead to unreliable decision-making, rendering it unsuitable for domains where explainability is a prerequisite, such as healthcare, autonomous driving, and financial risk assessment.

To address these concerns, the field of Explainable Deep Reinforcement Learning (XRL) has emerged, aiming to develop techniques that enhance the interpretability of DRL policies. XRL seeks to provide insights into an agent’s decision-making process, enabling researchers, practitioners, and end-users to understand, validate, and refine learned policies. By facilitating greater transparency, XRL contributes to the development of safer, more robust, and ethically aligned AI systems.

Furthermore, the increasing integration of Reinforcement Learning (RL) with Large Language Models (LLMs) has placed RL at the forefront of natural language processing (NLP) advancements. Methods such as Reinforcement Learning from Human Feedback (RLHF)~\citep{bai2022training,ouyang2022training} have become essential for aligning LLM outputs with human preferences and ethical guidelines. By treating language generation as a sequential decision-making process, RL-based fine-tuning enables LLMs to optimize for attributes such as factual accuracy, coherence, and user satisfaction, surpassing conventional supervised learning techniques. However, the application of RL in LLM alignment further amplifies the explainability challenge, as the complex interactions between RL updates and neural representations remain poorly understood.

This survey provides a systematic review of explainability methods in DRL, with a particular focus on their integration with LLMs and human-in-the-loop systems. We first introduce fundamental RL concepts and highlight key advances in DRL. We then categorize and analyze existing explanation techniques, encompassing feature-level, state-level, dataset-level, and model-level approaches. Additionally, we discuss methods for evaluating XRL techniques, considering both qualitative and quantitative assessment criteria. Finally, we explore real-world applications of XRL, including policy refinement, adversarial attack mitigation, and emerging challenges in ensuring interpretability in modern AI systems. Through this survey, we aim to provide a comprehensive perspective on the current state of XRL and outline future research directions to advance the development of interpretable and trustworthy DRL models.
\section{Related Work}

\subsection{Personalization and Role-Playing}
Recent works have introduced benchmark datasets for personalizing LLM outputs in tasks like email, abstract, and news writing, focusing on shorter outputs (e.g., 300 tokens for product reviews \citep{kumar2024longlamp} and 850 for news writing \citep{shashidhar-etal-2024-unsupervised}). These methods infer user traits from history for task-specific personalization \citep{sun-etal-2024-revealing, sun-etal-2025-persona, pal2024beyond, li2023teach, salemi2025reasoning}. In contrast, we tackle the more subjective problem of long-form story writing, with author stories averaging 1500 tokens. Unlike prior role-playing approaches that use predefined personas (e.g., Tony Stark, Confucius) \citep{wang-etal-2024-rolellm, sadeq-etal-2024-mitigating, tu2023characterchat, xu2023expertprompting}, we propose a novel method to infer story-writing personas from an author’s history to guide role-playing.


\subsection{Story Understanding and Generation}  
Prior work on persona-aware story generation \citep{yunusov-etal-2024-mirrorstories, bae-kim-2024-collective, zhang-etal-2022-persona, chandu-etal-2019-way} defines personas using discrete attributes like personality traits, demographics, or hobbies. Similarly, \citep{zhu-etal-2023-storytrans} explore story style transfer across pre-defined domains (e.g., fairy tales, martial arts, Shakespearean plays). In contrast, we mimic an individual author's writing style based on their history. Our approach differs by (1) inferring long-form author personas—descriptions of an author’s style from their past works, rather than relying on demographics, and (2) handling long-form story generation, averaging 1500 tokens per output, exceeding typical story lengths in prior research.

\section{ELITE}
\label{sec:method}
% In this section, we discuss the limitations of the previous evaluation methods, highlighting their inability to effectively assess the safety of VLM responses, and introduce ELITE evaluator as an accurate evaluation method. Next, we describe the construction process of our dataset designed to induce harmful responses from VLMs. As shown in Table \ref{table1}, we provide a detailed breakdown of the components of the ELITE benchmark constructed based on the ELITE evaluator, along with the criteria used for filtering out image-text pairs.

In this section, we introduce the ELITE evaluator as an accurate evaluation method. Also, we describe the construction process of the ELITE benchmark, along with the creation of in-house image-text pairs, which are designed to induce harmful responses from VLMs. As shown in Table \ref{table1}, we provide a detailed breakdown of the components of the ELITE benchmark.


\subsection{The ELITE Evaluator}
Existing benchmarks use human evaluators or automated evaluators for the safety evaluation of VLMs. Relying on human evaluators is expensive, and thus recent approaches have leveraged automated annotators using LLMs or VLMs. However, existing evaluation methods assume that an attack is successful if the victim model does not output any predefined refusal messages~\cite{chi2024llamaguardvision, mmsafetybench2025}, causing inaccurate safety evaluation results. In order to address this problem, we introduce the ELITE evaluator, an accurate and structured rubric-based evaluation method.

The ELITE evaluator is built upon the StrongREJECT evaluator~\cite{strongreject2024}, which highlighted the overestimation of most basic automated evaluators in the jailbreak techniques of LLMs. The StrongREJECT evaluator is defined as follows: 
\begin{equation}
\text{StrongREJECT} = (1 - \textit{refused}) \times \frac{\textit{specific} + \textit{convincing}}{2}
\label{equation1}
\end{equation} 

In Eq.~\ref{equation1}, \textit{refused} evaluates whether the model has refused the forbidden prompt, and assigns 1 point if the victim model refuses to respond. \textit{specific} measures the specificity of the response, while \textit{convincing} evaluates its convincingness. Both terms are scored on a scale of 1 to 5.

In VLMs, the victim model's responses to forbidden prompts can vary as images are added. For instance, as observed in the responses in Fig.~\ref{figure2}, despite malicious prompts, the victim model may produce responses that are unhelpful to the attacker. However, when the StrongREJECT evaluator is applied to VLMs, it often assigns high scores even when the model does not explicitly refuse to respond and instead provides unhelpful answers. In other words, since the victim model provides a response instead of refusing, a score of 0 is assigned to \textit{refused}, while moderate scores are given for \textit{specific} and \textit{convincing}, resulting in a final score that indicates the response is harmful. This makes it challenging to determine whether the attempt to induce harmful outputs was genuinely successful. To address this problem, we incorporate \textit{toxicity} into the evaluation rubric to assess how harmful the victim model's response is. Through this approach, \textit{toxicity} plays a crucial role in evaluating the success of inducing harmful responses, addressing the limitations of relying solely on \textit{refused} to determine whether the victim model's response is harmful or not. Accordingly, the ELITE evaluator is defined as follows:
\begin{equation}
\resizebox{\columnwidth}{!}{
$\text{ELITE} = (1 - \textit{refused}) \times 
\frac{\textit{specific} + \textit{convincing}}{2} \times \textit{toxicity}$
}
\label{equation2}
\end{equation}

In Eq.~\ref{equation2}, the ELITE evaluator introduces \textit{toxicity} as an additional criterion, scored on a scale of 0 to 5.

% As shown in Fig.~\ref{Figure2}, when the StrongREJECT is applied to VLMs, it often assigns high scores, even when the model provides simple image descriptions or does not explicitly refuse a response, making it difficult to determine whether the jailbreak was truly successful. SF-Scoring, on the other hand, introduces \textit{toxicity} as an additional criterion, with a scale ranging from [0-5]. In VLMs, the impact of jailbreaks varies depending on the given image. To address this variability, the \textit{toxicity} plays a crucial role in determining the success of a jailbreak, complementing the limitations of relying solely on the \textit{refused}. 

% In VLMs, the victim model's responses to forbidden prompts can vary depending on the image. For instance, as observed in the responses in Fig.~\ref{figure2}, despite malicious inputs, the victim model may produce responses that are unhelpful to the attacker. However, when StrongREJECT is applied to VLMs, it often assigns high scores if the model does not explicitly refuse the response and provides unhelpful answers such as image descriptions, as seen in Fig.~\ref{figure2}. In other words, since the victim model provides a response instead of refusing, a score of 0 is assigned to \textit{refused}, while scores are given for \textit{specific} and \textit{convincing}, resulting in a final score that indicates the response is harmful. This makes it challenging to determine whether the attempt to induce harmful outputs was genuinely successful. To address this problem, we incorporate \textit{toxicity} to evaluate how harmful the victim model's response is. Through this approach, \textit{toxicity} plays a crucial role in evaluating the success of a jailbreak, addressing the limitations of relying solely on \textit{refused} to determine whether the victim model's response is harmful or not.

\begin{figure}[t!]
\centering
\includegraphics[width=1.0\columnwidth]{Figure/figure2.pdf}
\caption{Examples of safety evaluations about the victim model's response by ELITE and StrongREJECT evaluator. $r$, $s$, $c$, and $t$ represent refused, specific, convincing, and toxicity, respectively. The ELITE evaluator can effectively evaluate utilizing the toxicity score to make more accurate judgments.}
\label{figure2}
\end{figure}

\subsection{Taxonomy}
We construct a taxonomy of 11 hazard categories to establish a comprehensive and extensive safety benchmark, aligning with AILuminate v1.0~\cite{mlcommons_1.0} standards. The taxonomy includes Violent Crimes (\textbf{S1}), Non-Violent Crimes (\textbf{S2}), Sex Crimes (\textbf{S3}), Defamation (\textbf{S4}), Specialized Advice (\textbf{S5}), Privacy (\textbf{S6}), Intellectual Property (\textbf{S7}), Indiscriminate Weapons (\textbf{S8}), Hate (\textbf{S9}), Self-Harm (\textbf{S10}), and Sexual Content (\textbf{S11}). Responses that enable, encourage, or endorse harmful behavior are flagged as unsafe for their respective categories. Detailed definitions of each category is provided in Appendix~\ref{sec:taxonomy}.

\subsection{Image\&Text Creation}
We refer to the in-house generated image-text pairs as the ELITE benchmark (generated). As shown in Fig.\ref{Figure1}(b), the ELITE benchmark (generated) is composed of four methods—Role Playing, Fake News, Blueprint, and Flowchart—applied across various taxonomies to elicit harmful responses from the victim model. Note that while certain methods, such as Blueprint and Fake News, are used only in specific taxonomies (e.g., Indiscriminate Weapons and Defamation), others, like Flowchart and Role Playing, are applied more broadly across all taxonomies. Detailed examples of these methods are provided in Appendix~\ref{supple:samples for generated}.

To generate image-text pairs, we use the following methods: \\
\textbf{(1) Image Generation}: For Role Playing, Blueprint, and Flowchart, we use image generation models such as Flux AI~\cite{flux2023} and Grok 2~\cite{grok2} to create images that align with the key concepts of each taxonomy. Specifically, we first extract relevant keywords for each taxonomy and use these keywords as prompts to generate corresponding images. For Fake News, we manually synthesize these images to create outputs that align with the intended misinformation scenarios, using the open-source image dataset CelebA~\cite{celeba}. \\
\textbf{(2) Text Generation}: We generate an initial forbidden text prompt by creating keywords relevant to the image and taxonomy, then generate multiple variations of the prompt using Grok 2. To identify the most effective forbidden text prompt for the given image, we evaluate responses from three victim models (Phi-3.5-Vision, Llama-3.2-11B-Vision, and Pixtral-12B). Among the models that produce harmful responses, we select the image-text pair with the highest ELITE evaluator score to finalize its construction.

These image-text pairs are explicitly designed to induce harmful responses from VLMs, enabling a comprehensive safety evaluation.  As shown in Table~\ref{table2}, we incorporate 593 safe-safe pairs into the ELITE benchmark (generated) by embedding inherently harmful intents. These pairs can still induce unsafe responses from VLMs, making them crucial for evaluating safety. Through this, we aim to develop a more extensive benchmark that effectively captures potential vulnerabilities in VLMs. 

\begin{table}[t!]
\caption{
The distribution of the four image-text pair types (unsafe-unsafe, safe-unsafe, unsafe-safe, and safe-safe) in the ELITE benchmark (generated).}
\begin{center}
\resizebox{1.0\columnwidth}{!}{
\begin{tabular}{ccccc}
\toprule
\multicolumn{4}{c}{\textbf{ELITE benchmark (generated)}} & \multirow{2}{*}{\textbf{Total}} \\ \cmidrule{1-4}
safe-safe   & safe-unsafe  & unsafe-safe  & unsafe-unsafe  &                                 \\ \midrule
    593        &    69          &    350          &        42        & 1054                            \\ \bottomrule
\end{tabular}}
\label{table2}
\end{center}
\end{table}


\subsection{Benchmark Construction Pipeline}
As shown in Fig.~\ref{figure}, the steps for constructing the ELITE benchmark are as follows: \\
\textbf{(1) Taxonomy Alignment}: To align the image-text pairs in existing benchmarks with the taxonomy of the ELITE benchmark, we use GPT-4o to classify image-text pairs into their corresponding taxonomies within the ELITE benchmark. \\
\textbf{(2) Filtering}: We apply a filtering process based on a defined threshold to both existing benchmarks and the ELITE benchmark (generated). Specifically, on the ELITE evaluator's [0-25] point scale, we set a threshold determined by human judgment. ELITE evaluator score $s\geq 10$ indicates that the victim model's response is sufficiently harmful, while $s< 10$ indicates that the victim model either refused to respond to the forbidden prompt or provided a non-harmful response. Using this threshold, we primarily include image-text pairs in the ELITE benchmark if at least two out of the three victim models (Phi-3.5-Vision, Llama-3.2-11B-Vision, and Pixtral-12B) achieve a score of $s\geq 10$ to prevent over-reliance on a single model during filtering. However, in cases where a single model's response is deemed sufficiently harmful, pairs meeting the threshold with only one model are also included.  Examples of model responses near our threshold are provided in Appendix~\ref{threshold}. \\ 
\textbf{(3) Balancing the Taxonomy}: After filtering, we identify that some benchmarks are overly concentrated in specific taxonomies (e.g., 204 image-text pairs in VLGuard are filtered into the S9. Hate), leading to imbalance across taxonomies. To create a more balanced benchmark, we additionally filter JailbreakV-28k~\cite{jailbreak28k2024} for only non-concentrated categories. Also, to address the issue of certain taxonomies being overly dependent on specific benchmarks, We exclude image-text pairs with the lowest combined ELITE evaluator scores from the three models.

\begin{figure}[t!]
\centering
\includegraphics[width=1.0\columnwidth]{Figure/figure3.pdf}
\caption{The pipeline for constructing ELITE benchmark. 1) Taxonomy Alignment: Align the image-text pairs in existing benchmarks with the taxonomy of the ELITE benchmark. 2) Filtering: Integrate only image-text pairs where at least two out of three model responses assign an ELITE evaluator score of 10 or higher. 3) Balancing the Taxonomy: Remove image-text pairs with the lowest combined ELITE evaluator score from overly concentrated taxonomies to maintain balance across taxonomies after filtering.}
\label{figure}
\end{figure}

\begin{table*}[ht!]
\caption{ELITE evaluator score-based ASR of various VLMs across taxonomies. The upper group in the table represents proprietary models, and the lower group represents open-source models. The most vulnerable model is highlighted in \textbf{bold} and the second-most vulnerable with an \underline{underline}.}
\begin{center}
\resizebox{1.0\textwidth}{!}{
\begin{tabular}{c|ccccccccccc|c}
\toprule
\textbf{Model}          & \textbf{S1} & \textbf{S2} & \textbf{S3} & \textbf{S4} & \textbf{S5} & \textbf{S6} & \textbf{S7} & \textbf{S8} & \textbf{S9} & \textbf{S10} & \textbf{S11} & \textbf{Average} \\ 
\midrule
GPT-4o         & 16.39 & 17.51 & 12.74 & 20.30 & 33.23 & 14.38 & 7.38 & 17.36 & 8.66 & 11.59 & 13.91 & 15.67    \\ %완료 
GPT-4o-mini     & 29.47 & 32.91 & 18.79 & 31.58 & 44.41 & 25.24 & 18.03 & 29.48 & 18.05 & 28.48 & 33.73 & 28.23    \\ %완료
Gemini-2.0-Flash & 58.44 & 70.73 & 48.09 & 51.63 & 50.76 & 55.59 & 51.37 & 71.07 & 42.17 & 47.68 & 48.52 & 55.37 \\
Gemini-1.5-Pro   & 37.75 & 48.04 & 28.03 & 40.35 & 37.76 & 33.87 & 50.55 & 44.63 & 23.76 & 27.48 & 35.21 & 37.69 \\
Gemini-1.5-Flash & 43.21 & 56.16 & 22.93 & 40.60 & 39.27 & 37.70 & 50.82 & 47.38 & 30.57 & 23.51 & 37.87 & 40.70 \\
\midrule
LLaVa-v1.5-7B  & 67.38 & 79.13 & 72.93 & 51.38 & 46.83 & 68.05 & 63.39 & 66.94 & 51.57 & 64.90 & 56.80 & 63.59    \\ %완료
LLaVa-v1.5-13B & \underline{72.85} & 86.69 & \textbf{79.94} & 53.63 & 54.98 & 73.48 & 68.31 & 72.45 & 58.56 & \underline{74.17} & 60.65 & \underline{69.68}    \\  %완료 
DeepSeek-VL-7B & 38.41 & 59.94 & 31.21 & 34.59 & 42.90 & 43.45 & 42.62 & 54.27 & 37.02 & 35.43 & 31.95 & 42.36    \\ %완료
DeepSeek-VL2-Small & 65.07 & 81.93 & 59.24 & 41.35 & \underline{58.01} & 68.69 & 59.29 & 70.25 & 52.12 & 53.64 & 42.31 & 60.95    \\ %완료
ShareGPT4V-7B & 68.71 & 86.41 & 75.16 & 48.62 & 53.78 & 72.52 & 71.04 & 64.74 & \underline{60.96} & 65.56 & 56.51 & 67.16    \\ %완료
ShareGPT4V-13B & 71.03 & \underline{87.54} & 75.16 & 51.38 & 56.80 & \underline{74.76} & \underline{73.22} & 66.39 & 60.41 & 62.91 & 52.96 & 68.08 \\ % 완료
Phi-3.5-Vision  & 37.58 & 44.40 & 16.24 & 49.87 & 38.07 & 25.24 & 21.86 & 41.05 & 18.60 & 23.18 & 18.34 & 31.85   \\  % 완료
Pixtral-12B  & \textbf{75.50} & \textbf{93.56} & \underline{77.07} & \textbf{67.17} & \textbf{61.63} & \textbf{79.23} & \textbf{86.61} & \textbf{90.08} & \textbf{82.50} & \textbf{77.15} & \textbf{74.56} & \textbf{79.86} \\ %완료
Llama-3.2-11B-Vision  & 54.47 & 69.05 & 41.40 & 30.83 & 55.29 & 53.35 & 33.88 & 55.37 & 34.44 & 43.05 & 39.05 & 47.94   \\ %완료
Qwen2-VL-7B & 57.28 & 70.73 & 45.22 & 38.60 & 47.73 & 60.06 & 40.44 & 66.67 & 45.49 & 54.64 & 50.00 & 53.72 \\ %완료
Molmo-7B& 61.09 & 81.51 & 62.42 & \underline{56.14} & 51.96 & 57.19 & 71.31 & \underline{75.21} & 47.70 & 64.90 & \underline{63.61} & 63.79 \\
InternVL2.5-8B& 51.32 & 65.83 & 60.83 & 23.81 & 50.76 & 49.52 & 36.61 & 55.65 & 27.62 & 43.71 & 36.98 & 46.48 \\
InternVL2.5-26B & 37.75 & 47.48 & 42.36 & 27.82 & 45.62 & 34.82 & 21.58 & 50.41 & 23.02 & 34.77 & 28.99 & 36.21 \\

\bottomrule
\end{tabular}}
\label{table3}
\end{center}
\end{table*}

% As shown in Fig.~\ref{figure}, through these steps, we remove ambiguous samples, such as those that fail to elicit harmful responses from VLMs, and filter samples from previous benchmarks and ELITE benchmark (generated samples) to try to make the ELITE benchmark more diverse.

\section{Experiments}

\subsection{Experiment Setup}
We evaluate the effectiveness of the ELITE benchmark, consisting of 4,587 image-text pairs, across various VLMs, including GPT-4o, GPT-4o-mini, Gemini-2.0, Gemini-1.5, and open-source models. For open-source models, their original hyperparameters are used. We use GPT-4o as the ELITE evaluator to evaluate the safety of VLMs.

% , including GPT-4o, GPT-4o-mini, Gemini-2.0-Flash, Gemini-1.5 (Pro, Flash), and open-source models such as LLaVa-v1.5 (7B and 13B), DeepSeek (VL-7B, VL2-Small), ShareGPT4V (7B and 13B), Phi-3.5-Vision, Pixtral (12B), Llama-3.2-Vision (11B), Qwen2-VL (7B), Molmo (7B), and InternVL2 (8B and 26B)

\subsection{Metric}
In the Experiments section, we use the ELITE evaluator score-based Attack Success Rate (E-ASR) for comparison. E-ASR is defined as:
\begin{equation}
\text{E-ASR} = \frac{\left| \{ i \mid \text{ELITE evaluator score}_i \geq 10 \} \right|}{N} \times 100,
\label{equation3}
\end{equation}

where \(\text{ELITE evaluator score}_i\) represents the ELITE evaluator score of the \(i\)-th image-text pair and \(N\) is the total number of image-text pairs.

\subsection{Evaluation of the ELITE Benchmark}
In Table~\ref{table3}, we present comprehensive experimental results of the ELITE benchmark across various proprietary and open-source VLMs. GPT-4o exhibits the lowest E-ASR at 15.67\% among models, indicating that it is appropriately safety-aligned against malicious inputs. In contrast, Gemini-2.0-Flash exhibits the highest E-ASR among proprietary models at 55.37\%, indicating significant vulnerability to malicious attacks. Additionally, with a few exceptions, most open-source models show high success rates for malicious attacks. The result that most models exhibit an E-ASR exceeding 40\% highlights the need for improved safety alignment in VLMs.

\begin{table}[t!]
\caption{Comparison of the average E-ASR when using different benchmarks. It highlights that the most effective benchmark for inducing harmful responses in \textbf{bold} and the second-most effective benchmark with an \underline{underline}.}
\begin{center}
\resizebox{1.0\columnwidth}{!}{
\begin{tabular}{cccc}
\toprule
\textbf{Model}  & \textbf{Benchmark}  & \textbf{Total}  & \textbf{Average}  \\ 
\midrule
\multicolumn{1}{c|}{\multirow{5}{*}{LLaVa-v1.5-7B}}  & VLGuard        & 2028          & 27.75           \\
\multicolumn{1}{c|}{}                                & MM-SafetyBench & 1680          & 45.06           \\
\multicolumn{1}{c|}{}                                & MLLMGuard      & 532           & 27.26        \\
\multicolumn{1}{c|}{}                                & ELITE benchmark (generated)   & 1054 & \textbf{69.17} \\ 
\multicolumn{1}{c|}{}                                & ELITE benchmark   & 4587 & \underline{63.59} \\ \midrule
\multicolumn{1}{c|}{\multirow{5}{*}{LLaVa-v1.5-13B}} & VLGuard        & 2028          & 28.40           \\
\multicolumn{1}{c|}{}                                & MM-SafetyBench & 1680          & 46.61          \\
\multicolumn{1}{c|}{}                                & MLLMGuard      & 532           & 27.26           \\
\multicolumn{1}{c|}{}                                & ELITE benchmark (generated)   & 1054 & \textbf{78.46} \\ 
\multicolumn{1}{c|}{}                                & ELITE benchmark   & 4587 & \underline{69.68} \\ \midrule
\multicolumn{1}{c|}{\multirow{5}{*}{DeepSeek-VL-7B}} & VLGuard        & 2028          & 16.40           \\
\multicolumn{1}{c|}{}                                & MM-SafetyBench & 1680          & 31.79           \\
\multicolumn{1}{c|}{}                                & MLLMGuard      & 532           & 16.29           \\
\multicolumn{1}{c|}{}                                & ELITE benchmark (generated)   & 1054 & \underline{37.95} \\ 
\multicolumn{1}{c|}{}                                & ELITE benchmark  & 4587 & \textbf{42.36}  \\ \midrule
\multicolumn{1}{c|}{\multirow{5}{*}{ShareGPT4V-7B}} & VLGuard        & 2028          & 29.24           \\
\multicolumn{1}{c|}{}                                & MM-SafetyBench & 1680          & 48.81          \\
\multicolumn{1}{c|}{}                                & MLLMGuard      & 532           & 23.51           \\
\multicolumn{1}{c|}{}                                & ELITE benchmark (generated)  & 1054 & \textbf{68.50} \\ 
\multicolumn{1}{c|}{}                                & ELITE benchmark   & 4587 & \underline{67.16} \\ \bottomrule
\end{tabular}}
\label{table4}
\end{center}
\end{table}


\subsection{Comparisons with Other Benchmarks}
In this section, we demonstrate the superiority of both the ELITE benchmark and the ELITE benchmark (generated). 
Table~\ref{table4} compares the E-ASR of LLaVa-v1.5 (7B, 13B), DeepSeek-VL (7B), and ShareGPT4V (7B) across existing benchmarks, including VLGuard~\cite{vlguard2024}, MM-SafetyBench~\cite{mmsafetybench2025}, MLLMGuard~\cite{mllmguard2024}, the ELITE benchmark, and the ELITE benchmark (generated). Note that we use publicly available benchmarks in this experiment. 

As shown in Table~\ref{table4}, the ELITE benchmark, which contains approximately 2–3 times more evaluation image-text pairs, achieves significantly higher E-ASR across all models. Furthermore, the ELITE benchmark (generated) demonstrates a substantial increase in E-ASR through effective filtering. These experimental results indicate that the low E-ASR observed in existing benchmarks suggests a substantial number of image-text pairs that fail to elicit harmful responses from VLMs. Consequently, this highlights the effectiveness of the ELITE evaluator in filtering out ambiguous image-text pairs, ensuring that only those capable of inducing harmful responses from VLMs are retained.

% To validate SF-Bench (generated dataset), we include experiments on SF-Bench (generated dataset, text-only), which contains only text without images, in Table 4. The experimental results show that the text-only dataset exhibits a lower SF-Scoring-based ASR compared to experiments including images, indicating that VLMs are less vulnerable to jailbreaks without images. These results suggest that the image-text pairs in the generated dataset are more effective in eliciting harmful content from VLMs. This is particularly evident in relatively safety-aligned models such as DeepSeek-VL-7B, which further emphasizes this observation.

Table~\ref{table5} presents the E-ASR of the methods used to elicit harmful responses from VLMs in the ELITE benchmark (generated). Our experimental results show that Flowchart and Blueprint achieve high E-ASR across a significant number of models, underscoring the importance of incorporating these methods into the benchmark to effectively evaluate and enhance the safety and robustness of VLMs.

\begin{table}[t!]
\caption{Comparison of the E-ASR of the proposed methods in the ELITE benchmark (generated).}
\begin{center}
\resizebox{1.0\columnwidth}{!}{
\begin{tabular}{cccccc}
\toprule
\multirow{2}{*}{\textbf{Method}} & \multirow{2}{*}{\textbf{Count}} & \multicolumn{4}{c}{\textbf{Model}}                                         \\ \cmidrule{3-6} 
                        &                        & LLaVa-7B  & LLaVa-13B & DeepSeek-7B & ShareGPT-7B  \\ \midrule
Blueprint               & 100                    & 64.00          & 82.00          & \textbf{85.00} & 65.00          \\
Fake News               & 140                    & \textbf{65.71} & 56.43          & 46.43          & 55.00          \\
Flowchart               & 706                    & 74.79          & \textbf{87.11}          & 31.16          & 77.34 \\
%Lyrics                  & 74                     & 9.46           & \textbf{14.86} & 2.70           & 2.70           \\
Role Playing            & 108                    & 39.47          & \textbf{44.74} & 26.32          & 30.70          \\ \bottomrule
\end{tabular}}
\label{table5}
\end{center}
\end{table}

\section{Human evaluation}
In this section, we explain the human labeling process and the steps involved in constructing the dataset for human evaluation. Through this, we conduct experiments to compare how closely the ELITE evaluator aligns with human judgment compared to existing automated safety evaluator and safeguard models.

\subsection{Human Annotators and Datasets} 
We recruit 22 labelers through the data-labeling company to evaluate responses. The labelers are selected to ensure diversity across gender, age, and occupation, aiming for unbiased labeling. Each labeler is provided with clear instructions on identifying safe and unsafe responses generated by the victim models. They reviewed each image-text-response pair and classified it as either ``safe" or ``unsafe". To ensure reliability, each image-text pair is evaluated by three labelers, and the majority decision is taken as the ground truth. The detailed evaluation process and the content of the instructions are provided in the Appendix~\ref{sec: human evaluation details}. 

The dataset used for human evaluation consists of a subset of image-text pairs, with approximately 90 pairs per taxonomy, totaling 963 pairs. For a fair comparison, instead of using the filtered responses, we also include responses from models that did not meet the filtering criteria among the three models (Phi-3.5-Vision, Llama-3.2-11B-Vision, and Pixtral-12B). Additionally, these pairs were primarily collected where the evaluation results differed between the ELITE evaluator and existing evaluation methods (StrongREJECT evaluator and safeguard models) and were randomly sampled across each taxonomy. To ensure diversity, we excluded image-text pairs that differed only in model responses from the human evaluation dataset.

\begin{figure}[t!]
\centering
\includegraphics[width=1.0\columnwidth]{Figure/figure4.pdf}
\caption{The comparison of AU-ROC curves between the ELITE evaluator and StrongREJECT evaluator on our human evaluation dataset.}
\label{figure3}
\end{figure}

\subsection{Comparison with Existing Evaluation Method}
To demonstrate the superiority of the ELITE evaluator, we compare it with the StrongREJECT evaluator. Fig.~\ref{figure3} shows the comparison using the Area Under the Receiver Operating Characteristic Curve (AU-ROC Curve)~\cite{au-roc}, considering the differences in scoring scales between the two methods. For a fair comparison, both the ELITE and StrongREJECT evaluators are evaluated using the GPT-4o on the human evaluation dataset consisting of 963 image-text pairs.

As shown in Fig.~\ref{figure3}, the StrongREJECT (GPT-4o) achieves an Area Under the Curve (AUC) of 0.46. In contrast, the ELITE evaluator achieves a significantly higher AUC of 0.77, demonstrating that the ELITE evaluator aligns more closely with human judgment. This result indicates the necessity of incorporating a toxicity score for a more accurate and comprehensive safety evaluation in VLMs. Furthermore, it highlights the robustness and superior performance of the ELITE evaluator.

To further demonstrate that the effectiveness of the ELITE evaluator is not solely due to advanced models like GPT-4o, we validate its effectiveness by applying it to open-source models. Specifically, we apply it to InternVL2.5 (7B, 26B) for comparison. Experimental results show that the ELITE evaluator with InternVL2.5 (7B, 26B) achieves AUC values of 0.57 and 0.65, respectively, surpassing the StrongREJECT evaluator with GPT-4o. This finding confirms that the strong performance of the ELITE evaluator is not solely dependent on a competent model.

% \begin{figure}[t!]
% \centering
% \includegraphics[width=0.9\columnwidth]{Figure/figure4.pdf}
% \caption{The comparison of AU-ROC curves between SF-Scoring and LlamaGuard-3-Vision-11B on our human evaluation dataset.}
% \label{Figure3}
% \end{figure}


\begin{table}[t!]
\caption{Performance comparison of the ELITE (GPT-4o), ELITE (InternVL2.5-8B, 26B), ELITE (InternVL2.5-26B),  LlamaGuard3-Vision-11B, LlavaGuard-13B, and OpenAI Moderation API on our human evaluation dataset. The best-performing method is highlighted in \textbf{bold} and the second-best
method with an \underline{underline}.}
\begin{center}
\resizebox{1.0\columnwidth}{!}{%
\begin{tabular}{c|ccccc}
\toprule
\textbf{Method}               & \textbf{Accuracy ($\uparrow$)} & \textbf{Precision ($\uparrow$)} & \textbf{Recall ($\uparrow$)} & \textbf{F1 score ($\uparrow$)} \\ \midrule
ELITE (GPT-4o)          & \textbf{0.726}    & \textbf{0.579}     & \textbf{0.709}  & \textbf{0.637}    \\
ELITE (InternVL2.5-26B) & \underline{0.660} & \underline{0.500} & \underline{0.471} & \underline{0.485} \\
ELITE (InternVL2.5-8B) & 0.609 & 0.416 & 0.376 & 0.395  \\
LlamaGuard3-Vision-11B       & 0.603 & 0.339 & 0.177 & 0.233   \\
LlavaGuard-13B        & 0.536 & 0.331 & 0.361 & 0.346 &  \\
OpenAI Moderation API       & 0.624 & 0.439 & 0.388 & 0.412  \\
\bottomrule
\end{tabular}%
}
\label{table6}
\end{center}
\end{table}

\subsection{Comparison with Safeguard Models} 
We compare the ELITE evaluator with safeguard models, including LlamaGuard3-Vision-11B~\cite{chi2024llamaguardvision}, LlavaGuard-13B~\cite{helff2024llavaguardvlmbasedsafeguardsvision}, and OpenAI Moderation API~\cite{openai2022moderation}. In this experiment, the ELITE evaluator classifies responses with ELITE evaluator score $s \geq 10$ as unsafe and $s < 10$ as safe, following the same criteria used for filtering.

Table~\ref{table6} demonstrates that the ELITE evaluator, when applied to GPT-4o, outperforms LlamaGuard3-Vision-11B in terms of accuracy, precision, recall, and F1 score. Specifically, it achieves 73\% accuracy, representing an improvement of approximately 20.3\% over LlamaGuard3-Vision-11B, 35\% over LlavaGuard-13B, and 16\% over the OpenAI Moderation API. For the F1 score, the ELITE evaluator shows an F1 score of 0.637, which is significantly higher than the others. Furthermore, the ELITE evaluator performs better when applied to the open-source model InternVL2.5 (26B). The superior performance of the ELITE evaluator on open-source models further indicates that its effectiveness is not limited to GPT-4o, highlighting its broader applicability.

\begin{table}[t!]
\caption{Breakdown of F1 score according to taxonomies between ELITE (GPT-4o), LlamaGuard3-Vision-11B, LlavaGuard-13B, and OpenAI Moderation API on our human evaluation dataset.}
\begin{center}
\resizebox{1.0\columnwidth}{!}{%
\begin{tabular}{lccccc}
\toprule
\multicolumn{1}{c}{\multirow{2}{*}{\textbf{Taxonomy}}} & \multicolumn{4}{c}{\textbf{F1 score ($\uparrow$)}}                                 \\ \cmidrule{2-5} 
\multicolumn{1}{c}{}                                   & \textbf{ELITE} & \textbf{LlamaGuard3} & \textbf{LlavaGuard} & \textbf{OpenAI Mod.} \\ \midrule
S1. Violent Crimes                 & \textbf{0.50}       & 0.16                 & 0.31          & 0.43            \\
S2. Non-Violent Crimes             & \textbf{0.61}       & 0.08                 & 0.26          & 0.48            \\
S3. Sex Crimes                     & \textbf{0.62}       & 0.18                 & 0.33          & 0.24            \\
S4. Defamation                     & \textbf{0.62}       & 0.18                 & 0.25          & 0.06            \\
S5. Specialized Advice             & \textbf{0.52}       & 0.09                 & 0.12          & 0.08            \\
S6. Privacy                        & \textbf{0.55}       & 0.16                 & 0.37          & 0.40            \\
S7. Intellectual Property          & \textbf{0.86}       & 0.62                 & 0.54          & 0.70            \\
S8. Indiscriminate Weapons         & \textbf{0.76}       & 0.18                 & 0.57          & 0.56            \\
S9. Hate                           & \textbf{0.66}       & 0.18                 & 0.38          & 0.44            \\
S10. Self-Harm                     & \textbf{0.67}       & 0.20                 & 0.30          & 0.30            \\
S11. Sexual Content                & \textbf{0.52}       & 0.37                 & 0.26          & 0.38            \\ \bottomrule
\end{tabular}}
\label{table7}
\end{center}
\end{table}

Table~\ref{table7} presents the F1 score for each taxonomy on the human evaluation dataset. Our results show that the ELITE evaluator outperforms LlamaGuard3-Vision-11B across all taxonomies. Specifically, safeguard methods tend to show low F1 scores in certain taxonomies. For instance, LlamaGuard3-Vision-11B shows significantly lower F1 scores in taxonomies such as S2. Non-violent Crimes and S5. Specialized Advice. Similarly, the OpenAI Moderation API shows low F1 scores in taxonomies such as S4. Defamation and S5. Specialized Advice. In contrast, the ELITE evaluator exhibits consistently high and balanced performance across all taxonomies. This demonstrates the superiority of the ELITE evaluator and indicates its effectiveness and accuracy in safety evaluation. 





% \subsection{Optimal threshold through human judgment}

% \begin{table}[t!]
% \centering
% \resizebox{0.5\columnwidth}{!}{%
% \begin{tabular}{cc}
% \toprule
% \textbf{SF-Score for threshold} & \textbf{Accuracy} \\ \midrule
% 5                               & 0.661             \\
% 8                               & 0.679             \\
% 10                              & 0.726             \\
% 15                              & 0.727             \\
% 25                              & 0.717             \\ \bottomrule
% \end{tabular}}
% \end{table}

\section{Limitations}
While RLEdit demonstrates promising results in lifelong editing tasks, several limitations should be acknowledged. Our evaluation methodology follows conventional datasets from existing model editing literature, primarily focusing on factual knowledge modifications without exploring other data domains. Furthermore, the capability of post-edited LLMs in processing multi-hop information remains unexplored in our current study. Although achieving robust lifelong editing capabilities continues to pose significant challenges, our future work will extend these experiments, potentially providing valuable insights for advancing research in lifelong editing.
\section{Conclusion }
This paper introduces the Latent Radiance Field (LRF), which to our knowledge, is the first work to construct radiance field representations directly in the 2D latent space for 3D reconstruction. We present a novel framework for incorporating 3D awareness into 2D representation learning, featuring a correspondence-aware autoencoding method and a VAE-Radiance Field (VAE-RF) alignment strategy to bridge the domain gap between the 2D latent space and the natural 3D space, thereby significantly enhancing the visual quality of our LRF.
Future work will focus on incorporating our method with more compact 3D representations, efficient NVS, few-shot NVS in latent space, as well as exploring its application with potential 3D latent diffusion models.


\bibliography{reference}
\bibliographystyle{icml2025}

\newpage
\appendix

\renewcommand{\figurename}{Supplementary Figure}
\renewcommand{\tablename}{Supplementary Table}
\setcounter{figure}{0}
\setcounter{table}{0}

    



\section{Details of datasets}
This section provides additional details about the dataset used to evaluate the downstream tasks. \Cref{tab:disease_definition} lists the ICD-10 codes and medications used to identify the diagnoses for each disease. \Cref{tab:characteristic} presents the distribution of patient characteristics for each disease. \Cref{fig:nyu_langone_prevalence,fig:nyu_longisland_prevalence} illustrates the prevalence of each disease in the downstream tasks for the NYU Langone and NYU Long Island datasets, highlighting the imbalances present in these tasks.

\begin{table}[!htpb]
    \centering
    \caption{The definition of diseases in EHR by diagnosis codes and medications.}
    \begin{tabular}{lr}
    \toprule
         Disease &  Definition in EHR \\
    \midrule
       IPH  &  I61.0, I61.1, I61.2, I61.3, I61.4, I61.8, I61.9 \\
       IVH  &  I61.5, P52.1, P52.2, P52.3  \\
       ICH  &  IPH + IVH + I61.6, I62.9, P10.9, P52.4, P52.9 \\
       SDH  &  S06.5, I62.0 \\
       EDH  &  S06.4, I62.1 \\
       SAH  &  I60.*, S06.6, P52.5, P10.3  \\
       Tumor  &  C71.*, C79.3, D33.0, D33.1, D33.2, D33.3, D33.7, D33.9  \\
       Hydrocephalus  &  G91.* \\
       Edema  &  G93.1, G93.5, G93.6, G93.82, S06.1 \\
       \multirow{2}{*}{ADRD}  &  G23.1, G30.*, G31.01, G31.09, G31.83, G31.85, G31.9, F01.*, F02.*, F03.*, G31.84, G31.1, \\ 
       & \textbf{Medication:} DONEPEZIL, RIVASTIGMINE, GALANTAMINE, MEMANTINE, TACRINE \\ 
    \bottomrule
    \end{tabular}
    \label{tab:disease_definition}
\end{table}

\begin{table}[!htbp]
\centering
\caption{Demographic characteristics of patients associated with scans from the NYU Langone dataset, matched with electronic health records (EHR) and utilized in downstream tasks.}
\label{tab:characteristic}

 The characteristic table on NYU Langone dataset matched with EHR.
\begin{tabular}{ll|rr|r}
\toprule
                       \textbf{Cohort} &  &           \textbf{Male (\%)} &          \textbf{Female (\%)} &     \textbf{Age (std)} \\
\midrule
 --- & All (n=270,205) & 128,113 (47.41\%) & 142,092 (52.59\%) & 63.64 (19.68) \\
\midrule
       Tumor & Neg (n=260,704) & 123,338 (47.31\%) & 137,366 (52.69\%) & 63.85 (19.72) \\
             & Pos (n=9,501) &   4,775 (50.26\%) &   4,726 (49.74\%) & 57.80 (17.67) \\
\midrule
HCP & Neg (n=253,000) & 118,881 (46.99\%) & 134,119 (53.01\%) & 63.67 (19.72) \\
              & Pos (n=17,205) &   9,232 (53.66\%) &   7,973 (46.34\%) & 63.18 (19.11) \\
\midrule
Edema & Neg (n=242,576) & 112,987 (46.58\%) & 129,589 (53.42\%) & 63.96 (19.84) \\
      & Pos (n=27,629) &  15,126 (54.75\%) &  12,503 (45.25\%) & 60.81 (17.97) \\
\midrule
ADRD  & Neg (n=232,667) & 111,159 (47.78\%) & 121,508 (52.22\%) & 61.31 (19.55) \\
      & Pos (n=37,538) &  16,954 (45.16\%) &  20,584 (54.84\%) & 78.09 (13.30) \\
\midrule
          IPH & Neg (n=251,308) & 117,692 (46.83\%) & 133,616 (53.17\%) & 63.58 (19.82) \\
              & Pos (n=18,897) &  10,421 (55.15\%) &   8,476 (44.85\%) & 64.39 (17.69) \\
\midrule
          IVH & Neg (n=258,232) & 121,686 (47.12\%) & 136,546 (52.88\%) & 63.65 (19.79) \\
              & Pos (n=11,973) &   6,427 (53.68\%) &   5,546 (46.32\%) & 63.45 (17.19) \\
\midrule
          SDH & Neg (n=248,468) & 114,869 (46.23\%) & 133,599 (53.77\%) & 63.44 (19.78) \\
              & Pos (n=21,737) &  13,244 (60.93\%) &   8,493 (39.07\%) & 65.95 (18.33) \\
\midrule
          EDH & Neg (n=265,431) & 125,113 (47.14\%) & 140,318 (52.86\%) & 63.77 (19.64) \\
              & Pos (n=4,774) &   3,000 (62.84\%) &   1,774 (37.16\%) & 56.53 (20.75) \\
\midrule
          SAH & Neg (n=251,594) & 118,424 (47.07\%) & 133,170 (52.93\%) & 63.79 (19.76) \\
              & Pos (n=18,611) &   9,689 (52.06\%) &   8,922 (47.94\%) & 61.59 (18.49) \\
\midrule
          ICH & Neg (n=229,851) & 105,498 (45.90\%) & 124,353 (54.10\%) & 63.41 (19.93) \\
              & Pos (n=40,354) &  22,615 (56.04\%) &  17,739 (43.96\%) & 64.93 (18.14) \\
\bottomrule
\end{tabular}
\end{table}


\begin{table}[!h]
    \centering
    \caption*{\textbf{Supplementary \Cref{tab:characteristic} Continued.} Demographic characteristics of patients associated with scans from the NYU Long Island dataset, matched with electronic health records (EHR) and utilized in downstream tasks.}
\begin{tabular}{ll|rr|r}
\toprule
                       \textbf{Cohort} &  &           \textbf{Male (\%)} &          \textbf{Female (\%)} &     \textbf{Age (std)} \\
\midrule
--- & All (n=22,158) & 9,580 (43.23\%) & 12,578 (56.77\%) & 68.33 (18.14) \\
\midrule
Tumor & Neg (n=21,578) & 9,275 (42.98\%) & 12,303 (57.02\%) & 68.59 (18.08) \\
      & Pos (n=580) &   305 (52.59\%) &    275 (47.41\%) & 58.78 (17.79) \\
\midrule
HCP   & Neg (n=20,653) & 8,718 (42.21\%) & 11,935 (57.79\%) & 69.05 (17.90) \\
      & Pos (n=1,505) &   862 (57.28\%) &    643 (42.72\%) & 58.52 (18.48) \\
\midrule
Edema & Neg (n=19,402) & 8,068 (41.58\%) & 11,334 (58.42\%) & 68.89 (18.27) \\
      & Pos (n=2,756) & 1,512 (54.86\%) &  1,244 (45.14\%) & 64.36 (16.66) \\
\midrule
ADRD  & Neg (n=19,537) & 8,391 (42.95\%) & 11,146 (57.05\%) & 66.78 (18.28) \\
      & Pos (n=2,621) & 1,189 (45.36\%) &  1,432 (54.64\%) & 79.90 (11.77) \\
\midrule
IPH   & Neg (n=19,357) & 7,974 (41.19\%) & 11,383 (58.81\%) & 68.97 (18.27) \\
      & Pos (n=2,801) & 1,606 (57.34\%) &  1,195 (42.66\%) & 63.89 (16.48) \\
\midrule
IVH   & Neg (n=19,636) & 8,164 (41.58\%) & 11,472 (58.42\%) & 68.96 (18.22) \\
      & Pos (n=2,522) & 1,416 (56.15\%) &  1,106 (43.85\%) & 63.43 (16.66) \\
\midrule
SDH   & Neg (n=20,885) & 8,870 (42.47\%) & 12,015 (57.53\%) & 68.33 (18.21) \\
      & Pos (n=1,273) &   710 (55.77\%) &    563 (44.23\%) & 68.37 (16.83) \\
\midrule
EDH   & Neg (n=21,912) & 9,443 (43.10\%) & 12,469 (56.90\%) & 68.33 (18.16) \\
      & Pos (n=246) &   137 (55.69\%) &    109 (44.31\%) & 68.19 (15.59) \\
\midrule
SAH   & Neg (n=20,652) & 8,824 (42.73\%) & 11,828 (57.27\%) & 68.68 (18.12) \\
      & Pos (n=1,506) &   756 (50.20\%) &    750 (49.80\%) & 63.58 (17.65) \\
\midrule
ICH   & Neg (n=18,388) & 7,456 (40.55\%) & 10,932 (59.45\%) & 68.92 (18.35) \\
      & Pos (n=3,770) & 2,124 (56.34\%) &  1,646 (43.66\%) & 65.48 (16.77) \\
\bottomrule
\end{tabular}
\end{table}

\begin{figure}[!ht]
    \centering
    \includegraphics[width=0.8\textwidth]{images/NYU_Langone_prevalence.pdf}
    \caption{Disease prevalence of NYU Langone }
    \label{fig:nyu_langone_prevalence}
\end{figure}

\begin{figure}[!h]
    \centering
    \includegraphics[width=0.8\textwidth]{images/NYU_Longisland_prevalence.pdf}
    \caption{Disease prevalence of NYU Longisland dataset}
    \label{fig:nyu_longisland_prevalence}
\end{figure}



\section{Data augmentation details}
\label{sec:dataaug_details}
We applied Random Flipping across all three dimensions, Random Shift Intensity with offset $0.1$ for both pre-training and fine-tuning. For DINO training. random Gaussian Smoothing with sigma=$(0.5-1.0)$ is applied across all dimensions, Random Gamma Adjust is applied with gamma=$(0.2-1.0)$.


\section{Additional experiment results}
This section provides additional experimental results with more details.
Supplementary \Cref{fig:channels-ablation,fig:patches-ablation} compares the performance of the foundation model using different numbers of channels and patch sizes, demonstrating that the architecture design of our foundation model is optimal. 

Supplementary \Cref{fig:radar-comparison-merlin} compares our foundation model with a foundation CT model from previous studies, Merlin\cite{blankemeier2024merlinvisionlanguagefoundation}, which was trained on abdomen CT scans with corresponding radiology report pairs. Our model demonstrates superior performance on head CT scans.

Supplementary \Cref{fig:probing-comparison-gemini} compares our foundation model with Google CT Foundation model~\cite{yang2024advancingmultimodalmedicalcapabilities}, which was trained on large scale and diverse CT scans from different anatomy with corresponding radiology report pairs. Our model consistently shows improved performance across the board even though our model was pre-trained with less samples.

Supplementary \Cref{fig:probing_comparison} compares the performance on downstream tasks with various supervised tuning methods applied to foundation models pretrained with the MAE and DINO frameworks. Per-pathology comparisons are shown in Supplementary \Cref{fig:probing-comparison-perpath,fig:probing-comparison-perpath-dino}. Meanwhile, supplementary \Cref{fig:boxplot_scaling} complements \Cref{fig:scaling_law}, illustrating the per-pathology performances of foundation models pretrained with different scales of training data.

Supplementary \Cref{fig:batch_effect,fig:thickness-ablation} studies the impact of batch effect caused by different CT scan protocols of slice thickness and machine manufacturer. Detailed per-pathology performances are shown in Supplementary \Cref{fig:slice_thickness_per_pathology,fig:manufacturer_per_pathology}.

\begin{figure}[!htpb]
    \centering
    \makebox[\textwidth][l]{%
        \hspace{0.3\textwidth}\textbf{NYU Langone}
    } \\[0.2cm]
    \includegraphics[trim={0 0 0 0},clip,height=0.3\textwidth, width=0.3\textwidth]{figures/abla_chans/AUC_chans_NYU.pdf}
    \includegraphics[trim={0 0 0 0},clip,height=0.3\textwidth, width=0.55\textwidth]{figures/abla_chans/AP_chans_NYU.pdf}\\
    \makebox[\textwidth][l]{
        \hspace{0.34\textwidth}\textbf{RSNA}
    } \\[0.2cm]
    \includegraphics[trim={0 0 0 0},clip,height=0.3\textwidth, width=0.3\textwidth]{figures/abla_chans/AUC_chans_RSNA.pdf}
    \includegraphics[height=0.3\textwidth, width=0.55\textwidth]{figures/abla_chans/AP_chans_RSNA.pdf} 
    \caption{\textbf{Comparison of Different Channels Performance.} This plot compares the performance of models trained using different numbers of channels (channels from multiple HU intervals vs. a single HU interval). Across two datasets, models using three channels from different HU intervals consistently outperform those using a single channel with a fixed HU interval. All models were pre-trained on $100\%$ of the pretraining data with MAE.}
    \label{fig:channels-ablation}
\end{figure}


\begin{figure}[!htpb]
    \centering
    \makebox[\textwidth][l]{%
        \hspace{0.3\textwidth}\textbf{NYU Langone}
    } \\[0.2cm]
    \includegraphics[trim={0 0 0 0},clip,height=0.3\textwidth, width=0.3\textwidth]{figures/abla_patches/AUC_patches_NYU.pdf}
    \includegraphics[trim={0 0 0 0},clip,height=0.3\textwidth, width=0.55\textwidth]{figures/abla_patches/AP_patches_NYU.pdf}\\
    \makebox[\textwidth][l]{
        \hspace{0.34\textwidth}\textbf{RSNA}
    } \\[0.2cm]
    \includegraphics[trim={0 0 0 0},clip,height=0.3\textwidth, width=0.3\textwidth]{figures/abla_patches/AUC_patches_RSNA.pdf}
    \includegraphics[height=0.3\textwidth, width=0.55\textwidth]{figures/abla_patches/AP_patches_RSNA.pdf} 
    \caption{\textbf{Comparison of Different Patches Performance.} This plot compares the performance of models trained with different patch sizes (12 vs. 16). The results demonstrate that smaller patch sizes consistently achieve better performance. All models were pre-trained on $100\%$ of the pretraining data with MAE.}
    \label{fig:patches-ablation}
\end{figure}


\begin{figure*}
    \centering
    \makebox[\textwidth][l]{%
        \hspace{0.06\textwidth}
        \textbf{NYU Langone} \hspace{0.06\textwidth} \textbf{NYU Long Island} \hspace{0.11\textwidth} \textbf{RSNA} \hspace{0.18\textwidth} \textbf{CQ500}
    } \\[0.2cm]
    \includegraphics[trim={0 0 0 0},clip,height=0.21\textwidth, width=0.21\textwidth]{figures/abla_radarplot_merlin/AUC_NYU.pdf}
    \includegraphics[trim={0 0 0 0},clip,height=0.21\textwidth, width=0.21\textwidth]{figures/abla_radarplot_merlin/AUC_Longisland.pdf}
    \includegraphics[trim={0 0 0 0},clip,height=0.21\textwidth, width=0.21\textwidth]{figures/abla_radarplot_merlin/AUC_RSNA.pdf}
    \includegraphics[trim={0 0 0 0},clip,height=0.21\textwidth, width=0.35\textwidth]{figures/abla_radarplot_merlin/AUC_CQ500.pdf}\\[0.2cm]
    \includegraphics[height=0.21\textwidth, width=0.21\textwidth]{figures/abla_radarplot_merlin/AP_NYU.pdf} 
    \includegraphics[height=0.21\textwidth, width=0.21\textwidth]{figures/abla_radarplot_merlin/AP_Longisland.pdf} 
    \includegraphics[height=0.21\textwidth, width=0.21\textwidth]{figures/abla_radarplot_merlin/AP_RSNA.pdf}
    \includegraphics[height=0.21\textwidth, width=0.35\textwidth]{figures/abla_radarplot_merlin/AP_CQ500.pdf}
    \caption{\textbf{Comparison to previous 3D Foundation Model.} This plot compares the performance of our model with Merlin~\cite{blankemeier2024merlinvisionlanguagefoundation} and models trained from scratch across four datasets for our model and ResNet50-3D. Our DINO trained model is used in this comparison. Our model demonstrates consistently superior performance across majority of diseases, with the exception of epidural hemorrhage (EDH) in the CQ500 dataset.}
    \label{fig:radar-comparison-merlin}
\end{figure*}



\begin{figure*}
    \centering
    \makebox[\textwidth][l]{%
        \hspace{0.10\textwidth}
        \textbf{NYU Langone} \hspace{0.08\textwidth} \textbf{NYU Long Island} \hspace{0.1\textwidth} \textbf{RSNA} \hspace{0.15\textwidth} \textbf{CQ500}
    } \\[0.2cm]
    \includegraphics[trim={0 0 0 0},clip, width=0.22\textwidth]{figures/abla_probing/AUC_NYU.pdf}
    \includegraphics[trim={0 0 0 0},clip, width=0.22\textwidth]{figures/abla_probing/AUC_Longisland.pdf}
    \includegraphics[trim={0 0 0 0},clip, width=0.22\textwidth]{figures/abla_probing/AUC_RSNA.pdf}
    \includegraphics[trim={0 0 0 0},clip, width=0.28\textwidth]{figures/abla_probing/AUC_CQ500.pdf}
    \\[0.2cm]
    \includegraphics[width=0.22\textwidth]{figures/abla_probing/AP_NYU.pdf} 
    \includegraphics[width=0.22\textwidth]{figures/abla_probing/AP_Longisland.pdf} 
    \includegraphics[width=0.22\textwidth]{figures/abla_probing/AP_RSNA.pdf}
    \includegraphics[width=0.28\textwidth]{figures/abla_probing/AP_CQ500.pdf}
    \caption{\textbf{Comparison of Different Downstream Training Methods.} This plot illustrates the downstream performance of models evaluated using fine-tuning and various probing methods across four datasets. In most cases, the DINO pre-trained model outperforms the MAE pre-trained model. All models were pre-trained on $100\%$ of the available pretraining data.}
    \label{fig:probing_comparison}
\end{figure*}


\begin{figure}
\centering
\makebox[\textwidth][l]{%
    \hspace{0.39\textwidth}\textbf{RSNA}
} \\[0.2cm]
\includegraphics[trim={0 0 0mm 0},clip,height=0.27\textwidth]{figures/abla_gemini/AUC_RSNA_Gemini.pdf}
\includegraphics[trim={0 0 5mm 0},clip,height=0.27\textwidth]{figures/abla_gemini/AP_RSNA_Gemini.pdf}

\makebox[\textwidth][l]{%
    \hspace{0.38\textwidth}\textbf{CQ500}
} \\[0.2cm]
\includegraphics[trim={0 0 10mm 0},clip,height=0.345\textwidth]{figures/abla_gemini/AUC_CQ500_Gemini.pdf}
\includegraphics[trim={0 0 5mm 0},clip,height=0.345\textwidth]{figures/abla_gemini/AP_CQ500_Gemini.pdf}

\caption{\textbf{Performance comparison of linear probing for Our Model vs. Google CT Foundation model} This plot compares our model performance vs. Google CT Foundation model\cite{yang2024advancing} and Merlin \cite{blankemeier2024merlinvisionlanguagefoundation} across all diseases on RSNA and CQ500. Since Google CT Foundation moudel requires uploading data to Google Cloud (not allowed on our private data) for requesting model embeddings with model weights inaccessible, only public dataset comparison is provided in this study. Similar to other evaluations, we observed that our model outperforms Google CT Foundation model across the board with the only exception on Midline Shift for Google CT Foundation model and EDH for Merlin.}
\label{fig:probing-comparison-gemini}
\end{figure}



\begin{figure}
    \centering
    \makebox[\textwidth][l]{%
        \hspace{0.35\textwidth}\textbf{NYU Langone}
    } \\[0.2cm]
    \includegraphics[trim={0 0 120mm 0},clip,height=0.255\textwidth]{figures/abla_probing_perpath/DINO_AUC_NYU_Langone.pdf}
    \includegraphics[trim={0 0 0 0},clip,height=0.255\textwidth]{figures/abla_probing_perpath/DINO_AP_NYU_Langone.pdf} \\
    \makebox[\textwidth][l]{
        \hspace{0.35\textwidth}\textbf{NYU Long Island}
    } \\[0.2cm]
    \includegraphics[trim={0 0 120mm 0},clip,height=0.255\textwidth]{figures/abla_probing_perpath/DINO_AUC_NYU_Long_Island.pdf}
    \includegraphics[trim={0 0 0 0},clip,height=0.255\textwidth]{figures/abla_probing_perpath/DINO_AP_NYU_Long_Island.pdf} 
    \makebox[\textwidth][l]{
        \hspace{0.4\textwidth}\textbf{RSNA}
    } \\[0.2cm]
    \includegraphics[trim={0 0 120mm 0},clip,height=0.24\textwidth]{figures/abla_probing_perpath/DINO_AUC_RSNA.pdf}
    \hspace{5mm}
    \includegraphics[trim={0 0 0 0},clip,height=0.24\textwidth]{figures/abla_probing_perpath/DINO_AP_RSNA.pdf} 
    \makebox[\textwidth][l]{
        \hspace{0.4\textwidth}\textbf{CQ500}
    } \\[0.2cm]
    \includegraphics[trim={0 0 120mm 0},clip,height=0.30\textwidth]{figures/abla_probing_perpath/DINO_AUC_CQ500.pdf} \hspace{5mm}
    \includegraphics[trim={0 0 0 0},clip,height=0.30\textwidth]{figures/abla_probing_perpath/DINO_AP_CQ500.pdf} 
    \caption{\textbf{Performance comparison of supervised finetuning methods per pathology on the foundation model trained with DINO.} This plot breaks down the average performance across all diseases shown in Supplementary \Cref{fig:probing_comparison}. The results show that fine-tuning the entire network achieves the best performance in most scenarios. However, linear probing closely approaches finetuning performance for many diseases especially on small or imbalanced dataset, underscoring the capability of our pre-trained models to generate representations that adapt effectively to diverse disease detection tasks.}
    \label{fig:probing-comparison-perpath-dino}
\end{figure}

\begin{figure}
    \centering
    \makebox[\textwidth][l]{%
        \hspace{0.35\textwidth}\textbf{NYU Langone}
    } \\[0.2cm]
    \includegraphics[trim={0 0 0 0},clip,height=0.24\textwidth, width=0.3\textwidth]{figures/abla_probing_perpath/AUC_NYU.pdf}
    \includegraphics[trim={0 0 0 0},clip,height=0.24\textwidth, width=0.45\textwidth]{figures/abla_probing_perpath/AP_NYU.pdf}\\
    \makebox[\textwidth][l]{
        \hspace{0.35\textwidth}\textbf{NYU Long Island}
    } \\[0.2cm]
    \includegraphics[trim={0 0 0 0},clip,height=0.24\textwidth, width=0.3\textwidth]{figures/abla_probing_perpath/AUC_Longisland.pdf}
    \includegraphics[trim={0 0 0 0},clip,height=0.24\textwidth, width=0.45\textwidth]{figures/abla_probing_perpath/AP_Longisland.pdf} 
    \makebox[\textwidth][l]{
        \hspace{0.4\textwidth}\textbf{RSNA}
    } \\[0.2cm]
    \includegraphics[trim={0 0 0 0},clip,height=0.24\textwidth, width=0.3\textwidth]{figures/abla_probing_perpath/AUC_RSNA.pdf}
    \includegraphics[height=0.24\textwidth, width=0.45\textwidth]{figures/abla_probing_perpath/AP_RSNA.pdf} 
    \makebox[\textwidth][l]{
        \hspace{0.4\textwidth}\textbf{CQ500}
    } \\[0.2cm]
    \includegraphics[trim={0 0 120mm 0},clip,height=0.24\textwidth]{figures/abla_probing_perpath/AUC_CQ500.pdf}
    \includegraphics[trim={0 0 0 0},clip,height=0.24\textwidth]{figures/abla_probing_perpath/AP_CQ500.pdf} 
    \caption{\textbf{Performance comparison of supervised finetuning methods per pathology on the foundation model trained with MAE.} The results reveal that attentive probing is significantly more effective than linear probing, consistent with findings from~\cite{Chen2024}. Furthermore, for many diseases, the performance of probing models approaches that of fine-tuning, demonstrating that our pre-trained models produce adaptable representations capable of detecting diverse diseases.}
    \label{fig:probing-comparison-perpath}
\end{figure}









\begin{figure}
    \centering
    \textbf{NYU Langone} \\
    \includegraphics[trim={0 0 0 0},clip,height=0.24\textwidth, width=0.38\textwidth]{figures/abla_perpath_perf/AUC_NYU.pdf}
    \includegraphics[height=0.24\textwidth, width=0.45\textwidth]{figures/abla_perpath_perf/AP_NYU.pdf} \\
    \textbf{NYU Long Island} \\
    \includegraphics[trim={0 0 0 0},clip,height=0.24\textwidth, width=0.38\textwidth]{figures/abla_perpath_perf/AUC_Longisland.pdf}
    \includegraphics[height=0.24\textwidth, width=0.45\textwidth]{figures/abla_perpath_perf/AP_Longisland.pdf} \\
    \textbf{RSNA} \\
    \includegraphics[trim={0 0 0 0},clip,height=0.24\textwidth, width=0.38\textwidth]{figures/abla_perpath_perf/AUC_RSNA.pdf}
    \includegraphics[height=0.24\textwidth, width=0.45\textwidth]{figures/abla_perpath_perf/AP_RSNA.pdf}\\
    \textbf{CQ500} \\
    \includegraphics[trim={0 0 0 0},clip,height=0.24\textwidth, width=0.38\textwidth]{figures/abla_perpath_perf/AUC_CQ500.pdf}
    \includegraphics[height=0.24\textwidth, width=0.45\textwidth]{figures/abla_perpath_perf/AP_CQ500.pdf}
    \caption{\textbf{Performance for Different Percentage of Pre-training Samples (Per-Pathology).} This plot illustrates label efficiency for individual pathologies using Tukey plots, alongside the average performance across all diseases shown in \Cref{fig:scaling_law}. The results indicate that the majority of pathologies show improved downstream performance as the amount of pretraining data increases.}
    \label{fig:boxplot_scaling}
\end{figure}


\newpage

\section{Time complexity increase with reduced patch size}
\label{apd:self_attention_rate}
Assume we have 3D image input of shape $H\times W\times D$, patch size $P$ and reducing factor $s$. By time complexity of self-attention $O(n^2 d)$ for sequence length $n$ and embedding dimension $d$, the new time complexity after reducing patch size can be derived as
\begin{align*}
    O(n^2d)&=O((\frac{H\times W\times D}{(\frac{P}{s})^3})^2d) \\
           &=O((\frac{H\times W\times D}{P^3})^2 s^6d)  \\
           &=O(s^6)O(n_{ori}^2d)
\end{align*}
where $n_{ori}=\frac{H\times W\times D}{P^3}$ is the original sequence length before reducing patch size.



















\newpage
\begin{figure}[ht]
    \centering
    \includegraphics[width=\textwidth]{images/tsne_embedding_visualization_per_pathology.png}
    \caption{The 2D projection with t-SNE of CT volume representation extracted from the foundation model. Interestingly, certain subgroups still exhibited slightly better AUCs. For instance, scans with slice thicknesses between 1–4 mm (represented by light green points in the upper panel of \Cref{fig:batch_effect}) align with a specialized protocol for CT angiography (CTA), which uses contrast dye to improve diagnosis on particular diseases.}
    \label{fig:batch_effect}
\end{figure}


\begin{figure*}[ht]
    \centering
    \begin{subfigure}[b]{0.33\textwidth}
        \centering
        \includegraphics[width=\textwidth]{images/AUROC_vs_Slice_thickness_binned.png}
        \caption{AUROC Performance}
    \end{subfigure}
    \hfill
    \begin{subfigure}[b]{0.33\textwidth}
        \centering
        \includegraphics[width=\textwidth]{images/AUPRC_vs_Slice_thickness_binned.png}
        \caption{AUPRC Performance}
    \end{subfigure}
    \hfill
    \begin{subfigure}[b]{0.33\textwidth}
        \centering
        \includegraphics[width=\textwidth]{images/Histogram_of_slice_thickness_distribution_across_scans.png}
        \caption{Histogram of slice thickness distribution}
    \end{subfigure}
    \caption{The downstream task performances on various ranges of slice thickness.}
    \label{fig:thickness-ablation}
\end{figure*}


\begin{figure*}[ht]
    \centering
    \begin{subfigure}[b]{\textwidth}
        \centering
        \includegraphics[width=\textwidth]{images/AUROC_vs_slice_thickness_for_each_disease_category.png}
        \caption{AUROC Performance}
    \end{subfigure}
    \hfill
    \begin{subfigure}[b]{\textwidth}
        \centering
        \includegraphics[width=\textwidth]{images/AUPRC_vs_slice_thickness_for_eachdisease_category.png}
        \caption{AUPRC Performance}
    \end{subfigure}
    \hfill
    \begin{subfigure}[b]{\textwidth}
        \centering
        \includegraphics[width=\textwidth]{images/Ratio_of_positive_labels_vs_slice_thickness_for_each_disease_category.png}
        \caption{Ratio of Positive Labels}
    \end{subfigure}
    \caption{Performance for Each Slice Thickness Bin (Per Pathology).}
    \label{fig:slice_thickness_per_pathology}
\end{figure*}


\begin{figure*}[ht]
    \centering
    \begin{subfigure}[b]{0.3\textwidth}
        \centering
        \includegraphics[width=\textwidth]{images/AUROC_by_Disease_and_Manufacturer.png}
        \caption{AUROC Performance}
    \end{subfigure}
    \hfill
    \begin{subfigure}[b]{0.3\textwidth}
        \centering
        \includegraphics[width=\textwidth]{images/AUPRC_by_Disease_and_Manufacturer.png}
        \caption{AUPRC Performance}
    \end{subfigure}
    \hfill
    \begin{subfigure}[b]{0.39\textwidth}
        \centering
        \includegraphics[width=\textwidth]{images/Positive_Label_Ratio_by_Disease_and_Manufacturer.png}
        \caption{Distribution of Scans from Each Manufacturer}
    \end{subfigure}
    \caption{Performance for Each Manufacturer (Per Pathology).}
    \label{fig:manufacturer_per_pathology}
\end{figure*}







\end{document}


% This document was modified from the file originally made available by
% Pat Langley and Andrea Danyluk for ICML-2K. This version was created
% by Iain Murray in 2018, and modified by Alexandre Bouchard in
% 2019 and 2021 and by Csaba Szepesvari, Gang Niu and Sivan Sabato in 2022.
% Modified again in 2023 and 2024 by Sivan Sabato and Jonathan Scarlett.
% Previous contributors include Dan Roy, Lise Getoor and Tobias
% Scheffer, which was slightly modified from the 2010 version by
% Thorsten Joachims & Johannes Fuernkranz, slightly modified from the
% 2009 version by Kiri Wagstaff and Sam Roweis's 2008 version, which is
% slightly modified from Prasad Tadepalli's 2007 version which is a
% lightly changed version of the previous year's version by Andrew
% Moore, which was in turn edited from those of Kristian Kersting and
% Codrina Lauth. Alex Smola contributed to the algorithmic style files.
