\begin{table}[t]
    \centering
    \small
    \resizebox{0.49\textwidth}{!}{
    \begin{tabular}{p{1.5cm}| p{3cm} |p{3.5cm}}
        \toprule
        \textbf{D/S-IAT Cognitive Partterns} & \textbf{Language Patterns Equivalent} & \textbf{Psychological Rationale} \\
        \midrule
        \textbf{Self-Death} & {\texttt{Death-Me}: Generates subtle expressions of self-identification with death.} & {Individuals with stronger ``Me = Death'' associations tend to have higher suicide risk \cite{method}.} \\
        \midrule
        \textbf{Others-Life / Self-Life} & {\texttt{Life-Not Me, Life-Me}: Generates narratives of feeling detached from life’s joys.} & {Weakened ``Me = Life'' associations are strong signal of suicidal behavior \cite{method5}.} \\
        \midrule
        \textbf{Others-Death} & {\texttt{Death-Not Me}: Generates reflections on others' deaths with subtle personal resonance.} & {Third-person narratives about death can indicate cognitive distancing or passive suicidal ideation \cite{method6}.} \\
        \bottomrule
    \end{tabular}
     }
    \caption{Mapping between D/S-IAT cognitive associations and equivalent language patterns, along with their psychological rationale.}
    \label{tab:ds_iat_mapping}
\end{table}
