\section{Human Suicidal Ideation \textit{Cognitive Partterns} and \textit{Thinking Partterns}}
\label{sec:partterns}

In this section, we describe two fundamental theories that assess human Suicidal Ideation through their Cognitive Patterns and Thinking Patterns: Death/Suicide Implicit Association Test (D/S-IAT) and Automatic Negative Thoughts (ANT).

\subsection{Death/Suicide Implicit Association Test (D/S-IAT)}
D/S-IAT is a validated psychological assessment tool \cite{test4,method} that measures implicit suicidal ideation through reaction-time-based tasks. Unlike traditional self-report questionnaires that rely on conscious reflection~\cite{Deming2021,TERRILL2021113737} D/S-IAT reveals unconscious cognitive patterns by measuring how quickly participants associate self-related terms (e.g., Me, Mine) with life-related (e.g., Alive, Breathing) versus death-related words (e.g., Dead, Lifeless). Research shows that individuals with suicidal tendencies demonstrate weaker ``self-life'' associations and stronger ``self-death'' associations \cite{method2}. Although D/S-IAT traditionally uses reaction times, research \cite{suicide1, method3, method4} has shown that implicit cognition of suicide can be extent to verbalizations in which individuals show more self-focus, feelings of detachment, and emotional compartmentalization. Based on the cognitive mechanisms identified in D/S-IAT, LLMs' demonstrated ability to understand and generate human-like \textit{language patterns}, including subtle linguistic markers and emotional undertones~\cite{NEURIPS2024_b0049c3f}, makes them suitable for simulating these implicit suicide-related \textit{congnitive patterns}. We illustrate this correspondence in Table~\ref{tab:ds_iat_mapping}.
\begin{table}[t]
    \centering
    \small
    \resizebox{0.49\textwidth}{!}{
    \begin{tabular}{p{1.5cm}| p{3cm} |p{3.5cm}}
        \toprule
        \textbf{D/S-IAT Cognitive Partterns} & \textbf{Language Patterns Equivalent} & \textbf{Psychological Rationale} \\
        \midrule
        \textbf{Self-Death} & {\texttt{Death-Me}: Generates subtle expressions of self-identification with death.} & {Individuals with stronger ``Me = Death'' associations tend to have higher suicide risk \cite{method}.} \\
        \midrule
        \textbf{Others-Life / Self-Life} & {\texttt{Life-Not Me, Life-Me}: Generates narratives of feeling detached from life’s joys.} & {Weakened ``Me = Life'' associations are strong signal of suicidal behavior \cite{method5}.} \\
        \midrule
        \textbf{Others-Death} & {\texttt{Death-Not Me}: Generates reflections on others' deaths with subtle personal resonance.} & {Third-person narratives about death can indicate cognitive distancing or passive suicidal ideation \cite{method6}.} \\
        \bottomrule
    \end{tabular}
     }
    \caption{Mapping between D/S-IAT cognitive associations and equivalent language patterns, along with their psychological rationale.}
    \label{tab:ds_iat_mapping}
\end{table}


\subsection{Automatic Negative Thoughts(ANT)}


The D/S-IAT-based implicit association helps identify conscious associations related to suicide, yet suicidal ideation also manifests through Automatic Negative Thoughts (ANT)—involuntary and repetitive negative thoughts that occur without conscious control~\cite{method9}. These Negative thoughts play a critical role in shaping an individual's emotional and cognitive state, often manifesting in specific thinking patterns, for example: \textit{All-or-Nothing thinking} refers to perceiving things in black-and-white categories. If a performance falls short of perfection, a person may see themselves as a total failure. \textit{Overgeneralization} means perceiving a single negative event as part of a never-ending pattern of defeat. We list all the Automatic Negative Thoughts and description in Appendix~\ref{app:ant} Research has shown that the frequency and intensity of these automatic negative thoughts strongly correlate with both depression severity and motivation to suicidal behavior~\cite{CAUDLE2024115787}. The characteristic patterns of ANT manifest in specific linguistic markers and mental filtering, labeling catastrophizing, etc. These cognitive distortions create distinct patterns that can be identified and analyzed in natural language expressions.
