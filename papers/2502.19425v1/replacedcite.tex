\section{Related work}
In this section, we review several related works on AI technology forecasting and logistic growth process respectively. 

\subsection{AI Technology Forecasting}
Although AI technology has developed so rapidly in the past decade, there is a relative lack of work on forecasting future AI technology trends. Currently, the relevant work on this topic can be roughly divided into three categories: literature mining-based, expert survey-based, and statistical modeling-based. 
The literature mining-based methods generally utilize various machine learning methods to capture key topics and keywords from a period of literature, thereby analyzing and forecasting the dynamics of different AI research fields. Based on the AMiner dataset, Shao et al____ combined traditional literature review and bibliometric methods to further summarize the evolution of AI in the past decade from the development of connectionism and discuss their trends in the next decade. Dwivedi et al____ employed structural topic modeling (STM) to extract and visualize latent topics from AI research literature and analyzed their future trends.
The expert survey-based methods collect various opinions through questionnaires of community experts and conduct statistical analysis to forecast technology trends. Baum et al____ presented an assessment of expert opinions regarding human-level AI research and analyzed when the Artificial General Intelligence (AGI) era will arrive. Gruetzemacher et al____ used quantitative expert survey data to reveal that conference attendance has a statistically significant impact on all predictions. Halal et al____ proposed TechCast, an online system that pools background trends and the judgment of experts around the world, to forecast breakthroughs in frontier technology fields including AI technology.
The statistical modeling-based methods are to directly collect industry-related trend data and build models for fitting and prediction. Villalobos et al____ established a growth model to analyze and  predict when the amount of data available for deep learning training will be exhausted. Modis____ modeled 28 historical milestones by the logistic model and estimated the emergence time of future milestones. 
In general, the methods mentioned above have predicted and discussed the future of AI technology from a certain perspective, but there is still a lack of quantitative modeling and prediction work based on the global history of AI development.

\subsection{Logistic Growth Process}
The logistic growth process is a continuously differentiable function with an approximate 'S' shape, which was first proposed in population research filed____. 
Over the past half-century, the logistic function has been increasingly used in various academic fields except demography. In ecology, the growth of bacteria____ and crop yield____ can be modeled by logistic function with corresponding growth factor. In the medical field, logistic functions have been widely used in tumor growth modeling and epidemic modeling. Laird____ proposed to employ Gompertz function____, a variant of logistic function, to fit the data of growth of tumors. Richards growth curve____, a flexible form of logistic function, has been successfully adopted to model the early phase of the COVID-19 outbreak worldwide____. Recently, Levene____ also proposed a skew logistic distribution for modeling COVID-19 waves.  
In sociology, logistic functions are often adopted to characterize socioeconomic dynamics and technological innovation. Carlota Perez____ employed logistic curves to model the Kondratiev cycles of economic dynamics. 
Arnulf Gruebler____ has studied the diffusion of infrastructure such as canals, railways, motorways, and airlines in-depth and found evidence that their spread patterns follow the logistic growth process. Harris et al____ used the logistic growth curve to model and forecast the production and consumption of US Energy.  Burg et al____ revisited Moore’s Law by establishing a logistic model of Intel chip density. 
In general, the logistic functions are applicable to model the groups with more complex components, which may have the capabilities to characterize the general evolution laws of many complex systems.