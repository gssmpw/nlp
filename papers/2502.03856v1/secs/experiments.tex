\section{Experiments}
\subsection{Experiment setup}
\textbf{Datasets.} We evaluated INOVA on two SGG benchmarks: 1) \textbf{VG}~\cite{krishna2017visual} contains annotations for 150 object categories and 50 relation categories across 108,777 images. Following standard setup~\cite{xu2017scene}, 70\% of the images are used for training, 5,000 for validation, and the remaining for testing. For a fair comparison, we excluded images overlapping with the pre-training dataset of Grounding DINO~\cite{liu2023grounding}, retaining 14,700 test images as in~\cite{zhang2023learning}. $\!$2) \textbf{GQA}~\cite{hudson2019gqa} uses the GQA200 split~\cite{dong2022stacked, sudhakaran2023vision}, including 200 object categories and 100 predicate categories. We randomly sampled 70\% of the object and predicate categories as the base, and more details can be found in the Appendix A.

\textbf{Metrics.} We conducted experiments under the challenging Scene Graph Detection (\textbf{SGDET}) protocol~\cite{xu2017scene,krishna2017visual}, which requires detecting objects and identifying relationships between object pairs without GT object labels or bounding boxes. We reported: 1) \textbf{Recall@K} (\textbf{R@K}): The proportion of ground-truth triplets correctly predicted within the top-K confident predictions.
2) \textbf{Mean R@K} (\textbf{mR@K}): The average R@K across all categories.
\begin{table*}[!t]
    \small
    \centering
    \caption{Experimental results of OvR-SGG setting on VG~\cite{krishna2017visual} test set.}   
    \setlength\tabcolsep{7.5pt}
    \scalebox{0.94}{
    \begin{tabular}{|rl||c|ccc|ccc|} 
        \hline
        \thickhline
        \rowcolor{mygray}
        & & & \multicolumn{3}{c|}{Base+Novel (Relation)} & 
        \multicolumn{3}{c|}{Novel (Relation)}  \\
        \rowcolor{mygray}
        \multicolumn{2}{|c||}{\multirow{-2}[0]{*}{Method}} & \multirow{-2}[0]{*}{Backbone} & R@20  & R@50  & R@100  & R@20 & R@50 & R@100  \\ 
        \hline
        \hline
        IMP~\cite{xu2017scene}&$_\text{CVPR'17}$ & - & - & 12.56    &  14.65 & - & 0.00 & 0.00 \\
        MOTIFS~\cite{zellers2018neural}&$_\text{CVPR'18}$& - & - & 15.41 & 16.96 & - & 0.00& 0.00 \\
        VCTREE~\cite{tang2019learning}&$_\text{CVPR'19}$& - & - & 15.61 & 17.26 & - & 0.00 & 0.00  \\
        TDE~\cite{tang2020unbiased}&$_\text{CVPR'20}$ & - & - & 15.50 & 17.37 & - &0.00 & 0.00 \\ 
        \hline
        $\text{VS}^3$~\cite{zhang2023learning}&$_\text{CVPR'23}$& \multirow{4}[0]{*}{Swin-T} & - &15.60 & 17.30 & - & 0.00 & 0.00 \\ 
        OvSGTR~\cite{chen2024expanding}  &$_\text{ECCV'24}$& & - & 20.46 & 23.86 & - & 13.45 & 16.19 \\ 
        RAHP~\cite{liu2025relation}  &$_\text{AAAI'25}$& & - & 20.50   & 25.74 & - & 15.59  & 19.92  \\ 

        \textbf{INOVA} (\textbf{Ours}) & &  & \textbf{17.49} & \textbf{23.22} & \textbf{27.40} & \textbf{12.90} & \textbf{17.89} & \textbf{21.70} \\
        \hline  OvSGTR~\cite{chen2024expanding}  &$_\text{ECCV'24}$& \multirow{2}[0]{*}{Swin-B} & - & 22.89   & 26.65 & - & 16.39  & 19.72  \\ 
        \textbf{INOVA} (\textbf{Ours}) & &  &
        \textbf{18.77} & \textbf{24.81} &
        \textbf{29.28} & \textbf{14.72} & \textbf{20.04} & \textbf{24.66}\\
        \thickhline
    \end{tabular}
    }
    \label{tab:ovr}
    \vspace{-1.5em}
\end{table*}

\begin{table*}[!t]
    \small
    \centering
    \caption{Experimental results of OvD+R-SGG setting on VG~\cite{krishna2017visual} test set.}    
    \setlength\tabcolsep{2.5pt}
    \scalebox{0.94}{
    \begin{tabular}{|rl||c|ccc|ccc|ccc|} 
        \hline
        \thickhline
        \rowcolor{mygray}
        & & & \multicolumn{3}{c|}{Joint Base+Novel} & 
        \multicolumn{3}{c|}{Novel (Obj)} & 
        \multicolumn{3}{c|}{Novel (Rel)} \\
        \rowcolor{mygray}
        \multicolumn{2}{|c||}{\multirow{-2}[0]{*}{Method}}  & \multirow{-2}[0]{*}{Backbone} & R@20  & R@50  & R@100  & R@20 & R@50 & R@100  & R@20 & R@50 & R@100 \\ 
        \hline
        \hline
        IMP~\cite{xu2017scene}&$_\text{CVPR'17}$ & - & - &   0.77  &  0.94 & - & 0.00 & 0.00 & - & 0.00 & 0.00 \\
        MOTIFS~\cite{zellers2018neural}&$_\text{CVPR'18}$ & - & - & 1.00 & 1.12 & - & 0.00 & 0.00 & - & 0.00 & 0.00 \\
        VCTREE~\cite{tang2019learning}&$_\text{CVPR'19}$ & - & - & 1.04 & 1.17 & - & 0.00 & 0.00 & - & 0.00 & 0.00 \\
        TDE~\cite{tang2020unbiased}&$_\text{CVPR'20}$ & - & - &  1.00 & 1.15 & - &0.00 & 0.00 & - & 0.00 & 0.00 \\ 
        \hline
        $\text{VS}^3$~\cite{zhang2023learning}&$_\text{CVPR'23}$& \multirow{3}[0]{*}{Swin-T} & - &
          5.88 & 7.20 & - & 0.00 & 0.00 & - & 0.00 & 0.00 \\ 
        \text{OvSGTR}~\cite{chen2024expanding}&$_\text{ECCV'24}$& & 10.02 & 13.50 & 16.37 & 10.56 & 14.32 & 17.48 & 7.09 & 9.19 & 11.18 \\ 
        \textbf{INOVA} (\textbf{Ours}) & & & \textbf{12.61}  & \textbf{17.43} & \textbf{21.27} & \textbf{12.48} & \textbf{17.16} & \textbf{21.10} & \textbf{11.38} & \textbf{15.90} & \textbf{19.46} \\  
        \hline\text{OvSGTR}~\cite{chen2024expanding} &$_\text{ECCV'24}$&\multirow{2}[0]{*}{Swin-B} & 12.37 & 17.14   & 21.03 & 12.63 & 17.58  & 21.70  & 10.56 & 14.62 & 18.22 \\ 
        
        \textbf{INOVA} (\textbf{Ours}) & & &  \textbf{13.50} & \textbf{18.88} & \textbf{23.19} & \textbf{13.46} & \textbf{18.84} & \textbf{23.29} & \textbf{12.37} & \textbf{17.50} & \textbf{21.73} \\
        \thickhline
    \end{tabular}
    }
    \label{tab:ovdr}
    \vspace{-1em}
\end{table*}


\textbf{Implementation Details.}
Due to space constraints, detailed implementation is provided in the Appendix A.
\subsection{Comparison with State-of-the-Art Methods}
\textbf{Setting.} Following~\cite{chen2024expanding}, we compared our INOVA with existing SOTA methods, \ie, \textbf{VS}~\cite{zhang2023learning}, \textbf{OvSGTR}~\cite{chen2024expanding}, and \textbf{RAHP}~\cite{liu2025relation} under two OVSGG settings: 1) \textbf{OvR-SGG}: Evaluates generalization to unseen relations while retaining original object categories. Fifteen of 50 relation categories in VG150 are removed during training, with performance measured on ``Base+Novel (Relation)'' and ``Novel (Relation)''.
2) \textbf{OvD+R-SGG}: Assesses handling of unseen objects and relations simultaneously. Both novel objects and relations are excluded during training, evaluated on ``Joint Base+Novel'', ``Novel (Object)'', and ``Novel (Relation)''. 

\textbf{Results.}
We conducted quantitative experiments on the VG dataset~\cite{krishna2017visual} in both the OvR-SGG and OvD+R-SGG setups, with results presented in Table~\ref{tab:ovr} and Table~\ref{tab:ovdr}, respectively. Notably, INOVA consistently outperforms the latest state-of-the-art methods across all metrics. In the OvR-SGG setup, INOVA surpasses the RAHP (Swin-T) by \textbf{+1.78}\% R@100 within the novel relation categories, demonstrating superior generalization and reduced overfitting. With the Swin-B backbone, INOVA achieves R@100 over OvSGTR across both base and novel relations, and \textbf{+4.94}\% R@100 in novel relations alone, further emphasizing its robustness. In the more challenging OvD+R-SGG scenario, INOVA continues to outperform the competition. Specifically, on the joint base and novel classes, INOVA gains \textbf{+4.90}\% and \textbf{+2.16}\% R@100 over OvSGTR with the Swin-T and Swin-B backbones, respectively. These results validate INOVA's superior performance and robust generalization across both relation and object domains.

\subsection{Diagnostic Experiment}



\begin{table*}[!t]
    \small
    \centering
    \caption{Analysis of key components on OvD+R-SGG setting of VG150~\cite{krishna2017visual} test set. \textbf{ITG}, \textbf{IQS}, and \textbf{RRD} stand for Interaction-aware Target Generation, Interaction-guided Query Selection, and Relative-interaction Retention Distillation in interaction-consistent knowledge distillation, respectively. The general OVSGG pipeline with visual-concept retention distillation as the baseline.}
    \setlength\tabcolsep{7.5pt}
    \scalebox{0.94}{
    \begin{tabular}{|ccc||ccc|ccc|ccc|} 
        \hline
        \thickhline
        \rowcolor{mygray}
        \multicolumn{3}{|c||}{Components}  & \multicolumn{3}{c|}{Joint Base+Novel} & 
        \multicolumn{3}{c|}{Novel (Obj)} & 
        \multicolumn{3}{c|}{Novel (Rel)} \\
        \rowcolor{mygray}
         ITG & IQS & RRD & R@20  & R@50  & R@100  & R@20 & R@50 & R@100  & R@20 & R@50 & R@100 \\ 
        \hline
        \hline
        &  &  & 10.02 & 13.50 & 16.37 & 10.56 & 14.32 & 17.48 & 7.09 & 9.19 & 11.18 \\
        & & \usym{1F5F8} & 11.43  & 15.67 & 19.20 & 11.57 & 15.65 & 19.32 & 10.07 & 14.00 & 17.32 \\
        & \usym{1F5F8} & & 11.37 & 15.71  & 19.37  & 11.43 & 15.80 & 19.61 & 9.84 & 13.92 & 17.38  \\
        \usym{1F5F8} &  &  &  11.92 & 16.67 & 20.31 & 11.75 & 16.51 & 20.16 & 10.72 & 15.10 & 18.52 \\
        & \usym{1F5F8} & \usym{1F5F8} &11.84  & 16.17 & 19.55 & 11.36 & 16.09 & 19.65 & 10.73 & 14.40 & 17.83  \\
        \usym{1F5F8} & & \usym{1F5F8} & 12.27  & 17.11 & 20.81 & 12.16 & 17.03 & 20.80 & 11.04 & 15.60 & 19.01 \\
        \usym{1F5F8} & \usym{1F5F8} &  & 12.42  & 17.22 & 21.10 & 12.29 & 17.08 & 20.99 & 11.16 & 15.51 & 19.16 \\
        \usym{1F5F8} & \usym{1F5F8} & \usym{1F5F8} &\textbf{12.61}  & \textbf{17.43} & \textbf{21.27} & \textbf{12.48} & \textbf{17.16} & \textbf{21.10} & \textbf{11.38}& \textbf{15.90} & \textbf{19.46} \\
        \thickhline
    \end{tabular}
    }
    \vspace{-1.5em}
    \label{tab:abla}
\end{table*}
To ensure a comprehensive evaluation, we performed a series of ablation studies on the VG dataset~\cite{krishna2017visual} in the challenging OvD+R-SGG scenario.


\textbf{Key Components Analysis.} 
The results are summarized in Table~\ref{tab:abla}, with the first row representing the baseline OVSGG pipeline with \textit{Visual-concept Retention Distillation} proposed in~\cite{chen2024expanding}. From this analysis, four key conclusions can be drawn: \textbf{First}, incorporating \textit{Interaction-aware Target Generation} (ITG) leads to consistent improvements across all metrics, including a \textbf{3.94}\% R@100 gain on the joint base and novel classes compared to the baseline. This demonstrates that ITG effectively improves performance by considering interaction contexts in supervision generation. \textbf{Second}, introducing \textit{Interaction-guided Query Selection} (IQS) further refines the query selection process. By prioritizing interacting objects and minimizing mismatched assignments, IQS achieves notable improvements, such as \textbf{3.00}\% R@100 gains, highlighting its ability to enhance precision by focusing on interacting object pairs. \textbf{Third}, leveraging \textit{Relative-interaction Retention Distillation} (RRD) ensures relational consistency during training, resulting in significant performance boosts. RRD contributes \textbf{2.83}\% R@100 gains, improving the model's ability to handle novel classes effectively. \textbf{Fourth}, the integration of all three components (\ie, ITG, IQS, and RRD) yields the best overall performance, with \textbf{1.92}\%$\sim$\textbf{8.28}\% improvements across all evaluation metrics. However, the improvement is less pronounced than expected, since each strategy prioritizes interacting objects, which may lead to diminishing returns by progressively reducing non-interacting objects. Despite this, the combined results still demonstrates enhanced relational understanding and serve as a valuable tool for improving performance in complex scenarios.


\begin{figure}[!t]
    \centering
    \includegraphics[width=1\linewidth]{figures/itg.pdf}
    \vspace{-2.0em}
    \caption{Interaction-aware target generation.}
    \label{fig:itg}
    \vspace{-1.5em}
\end{figure}



\begin{table}[h!]
    \centering
    \small
    \setlength{\tabcolsep}{5pt}
    \begin{tabular}{llccccc}
        \toprule
         {\#} & \textbf{Method (Stage @ Resolution)} & FID $\downarrow$ \\
        \midrule
         {1.} & \ourmodel (\pre@512) & \textbf{51.164} \\
        \midrule
         {2.} & Human-centric Filtering (\pre@128)  & \textbf{75.639} \\
         {3.} & No Filtering (\pre@128)  & 86.838 \\
         \midrule
         {4.} & Image-conditioned (\pre@128)  & \textbf{75.639} \\
         {5.} & Text-conditioned (\pre@128) & 109.720 \\
        \bottomrule
    \end{tabular}
    \caption{
        \textbf{Pretraining and Data Filtering.} We report results of the full pretrained model \pre@512, and compare
        several variants of \pre@128 models. We report FID on \mobile dataset (1k samples).
    }
    \label{tab:pretrain}
    \vspace{-10pt}
\end{table}


\begin{figure}[!t]
    \centering
    \vspace{-0.3em}
    \includegraphics[width=1\linewidth]{figures/query.pdf}
    \vspace{-2.3em}
    \caption{Interaction-guided query selection.}
    \label{fig:iqs}
    \vspace{-1.2em}
\end{figure}



\textbf{Supervision Analysis.}
We investigated ITG's impact in the pre-training process (\cf Table~\ref{tab:pretrain}). As seen, models pre-trained on COCO~\cite{chen2015microsoft} captions with INOVA variants consistently outperform others, achieving \textbf{13.31}\% R@100 with Swin-T and \textbf{14.22}\% R@100 with Swin-B. These results demonstrate the effectiveness of incorporating ITG in the VLM pre-training process.

In addition, we visualized the object detection results from ITG and the original methods that solely use object categories for detection. As displayed in Figure~\ref{fig:framework}, the original method produces redundant objects, complicating the identification of subject-object interactions. For instance, given the ``$\langle$\texttt{people}, \texttt{ride}, \texttt{bike}$\rangle$'' triplet, the baseline detects multiple instances of ``\texttt{people}'' and ``\texttt{bike}'', obscuring the interaction. In contrast, ITG leverages bidirectional interaction prompts and attention mechanisms to accurately localize the interaction-relevant objects. A similar enhancement is observed in the ``$\langle$\texttt{bikes}, \texttt{on}, \texttt{boat}$\rangle$'' triplet, where ITG focuses on interaction-relevant entities. 

\textbf{Query Visualization.} To demonstrate the effectiveness of IQS, we visualized the top-50 selected queries in Figure~\ref{fig:iqs}. As seen, the original approach makes no distinction between instances within the same category, such as ``\texttt{man}'' or ``\texttt{zebra}'', resulting in both interacting and non-interacting instances receiving a similar number of queries. This indiscriminate query generation increases the likelihood of incorrect matches during bipartite graph matching, as irrelevant regions compete with interaction-relevant instances. Conversely, IQS prioritizes queries for interacting instances (``man holding'' or ``zebra laying on'' in Figure~\ref{fig:iqs}), increasing discrimination among objects with the same categories. 
