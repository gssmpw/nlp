\section{Introduction}
% \begin{figure}
%     \centering
%     \includegraphics[width=\linewidth]{ssf_visual.eps}
%     \caption{Visualizations of the spatiotemporal tensor (ocean sound speed field) across different modes. It reveals that the signal exhibits significantly greater fluctuations in the time mode than in the other modes.}
%     \label{fig:ssf_visual}
% \end{figure}

Tensor is a ubiquitous data structure for organizing multi-dimensional data. For example, a four-mode tensor \textit{(longitude, latitude, depth, time)} can serve as a unified representation of spatiotemporal signals in the ocean, such as temperature or flow speed. Tensor decomposition is a prevailing framework for multiway data analysis that estimates latent factors to reconstruct the unobserved entries. Methods like CANDECOMP/PARAFAC (CP)\citep{HarshmanCP} and Tucker decomposition\citep{sidiropoulos2017tensor} are widely applied across fields, including climate science, oceanography, and social science.

An emerging trend in tensor community is to leverage the continuous timestamp of observed entries and build temporal tensor models, as the real-world tensor data is often irregularly collected in time, e.g., physical signals, accompanied with rich and complex time-varying patterns. The temporal tensor methods expand the classical tensor framework by using polynomial splines~\citep{zhang2021dynamic}, Gaussian processes~\citep{bctt,SFTL}, ODE~\citep{thisode} and energy-based models~\citep{tao2023undirected} to estimate the continuous temporal dynamics in latent space, instead of discretizing the time mode and setting a fixed number of factors.

Despite the successes of current temporal tensor methods, they inherit a fundamental limitation from traditional tensor models: they assume tensor data at each timestep must conform to a Cartesian grid structure with discrete indexes and finite-dimensional modes. This assumption poorly aligns with many real-world scenarios where modes are naturally continuous, such as spatial coordinates like \textit{(longitude, latitude, depth)}. To fit current models, we still need to discretize continuous indexes, which inevitably leads to a loss of fine-grained information encoded in these indexes. From a high-level perspective, while current temporal tensor methods have taken a crucial step forward by modeling continuous characteristics in the temporal mode compared to classical approaches, they still fail to fully utilize the rich complex patterns inherent in other continuous-indexed modes. 

Another crucial problem lies in determining the optimal rank for temporal tensor decomposition. As a fundamental hyperparameter in tensor modeling, the rank directly influences interpretability, sparsity, and model expressiveness. While classical tensor literature offers extensive theoretical analysis and learning-based solutions~\citep{zhao2015bayesianCP,cheng2022towards,pmlr-v32-rai14}, this topic has been largely overlooked in emerging temporal tensor methods. The introduction of dynamical patterns significantly complicates the latent landscape, and the lack of investigation into rank selection makes temporal tensor models more susceptible to hyperparameter choices and noise.

To fill these gaps, we propose 
\textsc{GreT}, a complexity-adaptive method for modeling temporal tensor data with continuous indexes across all modes.
Our method models general temporal tensor data with continuous indexes not only in the time mode but also in other modes. Specifically, \MODEL organizes the continuous indexes from non-temporal modes  into a Fourier-feature format, encodes them as the initial state of latent dynamics, and utilizes the neural ODE~\citep{chen2018neural} to model the factor trajectories.
To reveal the rank of the temporal tensor, \MODEL extends the classical rank selection framework~\citep{zhao2015bayesianCP} and assigns Gaussian-Gamma priors over factor trajectories to promote sparsity. For efficient inference, we propose a novel variational inference algorithm with an analytical evidence lower bound, enabling sampling-free inference  of the model parameters and latent dynamics.
For evaluation, we conducted experiments on both simulated and real-world tasks, demonstrating that \MODEL not only reveals the underlying ranks of temporal tensors but also significantly outperforms existing methods in prediction performance and robustness against noise.