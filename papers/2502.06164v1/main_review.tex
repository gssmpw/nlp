%%%%%%%% ICML 2025 EXAMPLE LATEX SUBMISSION FILE %%%%%%%%%%%%%%%%%

\documentclass{article}

% Recommended, but optional, packages for figures and better typesetting:
\usepackage{microtype}
\usepackage{graphicx}
\usepackage{subfigure}
\usepackage{booktabs} % for professional tables

% hyperref makes hyperlinks in the resulting PDF.
% If your build breaks (sometimes temporarily if a hyperlink spans a page)
% please comment out the following usepackage line and replace
% \usepackage{icml2025} with \usepackage[nohyperref]{icml2025} above.
\usepackage{hyperref}


% Attempt to make hyperref and algorithmic work together better:
\newcommand{\theHalgorithm}{\arabic{algorithm}}

% Use the following line for the initial blind version submitted for review:
%\usepackage{icml2025}

% If accepted, instead use the following line for the camera-ready submission:
\usepackage[accepted]{icml2025}

% For theorems and such
\usepackage{amsmath}
\usepackage{amssymb}
\usepackage{mathtools}
\usepackage{amsthm}
\usepackage{caption}
\usepackage{subcaption}
\usepackage{hyperref}
\usepackage{url}
\usepackage{array}
\usepackage{booktabs}
\usepackage{graphicx} 

%\let\listofalgorithms\relax
%\let\algorithm\relax
%\let\algorithm*\relax

%\usepackage{algorithm2e}
\usepackage{epsfig}
\usepackage{epstopdf} 

% if you use cleveref..
\usepackage[capitalize,noabbrev]{cleveref}

%%%%%%%%%%%%%%%%%%%%%%%%%%%%%%%%
% THEOREMS
%%%%%%%%%%%%%%%%%%%%%%%%%%%%%%%%
\theoremstyle{plain}
\newtheorem{theorem}{Theorem}[section]
\newtheorem{proposition}[theorem]{Proposition}
\newtheorem{lemma}[theorem]{Lemma}
\newtheorem{corollary}[theorem]{Corollary}
\theoremstyle{definition}
\newtheorem{definition}[theorem]{Definition}
\newtheorem{assumption}[theorem]{Assumption}
\theoremstyle{remark}
\newtheorem{remark}[theorem]{Remark}

% Todonotes is useful during development; simply uncomment the next line
%    and comment out the line below the next line to turn off comments
%\usepackage[disable,textsize=tiny]{todonotes}
\usepackage[textsize=tiny]{todonotes}


\newcommand{\llbracket}{[\![}
\newcommand{\rrbracket}{]\!]}
\newcommand{\bc}[1]{\mbox{\boldmath $\mathcal{#1}$}}
\newcommand{\bs}[1]{\boldsymbol{#1}}
\newcommand{\mf}[1]{\mathbf{#1}}
\newcommand{\mb}[1]{\mathbb{#1}}
\newcommand{\F}{\mathrm{F}}
\newcommand{\HH}{\mathrm{H}}
\newcommand{\T}{\mathrm{T}}
\newcommand{\MODEL}{\textsc{GreT }}


% The \icmltitle you define below is probably too long as a header.
% Therefore, a short form for the running title is supplied here:
\icmltitlerunning{Generalized Temporal Tensor Decomposition  with Rank-revealing Latent-ODE}

\begin{document}

\twocolumn[
\icmltitle{Generalized Temporal Tensor Decomposition  with Rank-revealing Latent-ODE}

% It is OKAY to include author information, even for blind
% submissions: the style file will automatically remove it for you
% unless you've provided the [accepted] option to the icml2025
% package.

% List of affiliations: The first argument should be a (short)
% identifier you will use later to specify author affiliations
% Academic affiliations should list Department, University, City, Region, Country
% Industry affiliations should list Company, City, Region, Country

% You can specify symbols, otherwise they are numbered in order.
% Ideally, you should not use this facility. Affiliations will be numbered
% in order of appearance and this is the preferred way.
\icmlsetsymbol{equal}{*}

\begin{icmlauthorlist}
	\icmlauthor{Panqi Chen}{ISEE}
	\icmlauthor{Lei Cheng}{ISEE}
	\icmlauthor{Jianlong Li}{ISEE}
	\icmlauthor{Weichang Li}{ISEE}
	\icmlauthor{Weiqing Liu}{MS}
	\icmlauthor{Jiang Bian}{MS}
	\icmlauthor{Shikai Fang}{MS}
\end{icmlauthorlist}

\icmlaffiliation{ISEE}{College of Information Science and Electronic Engineering, Zhejiang University, Hangzhou, China}
\icmlaffiliation{MS}{Microsoft Research Asia}


\icmlcorrespondingauthor{Lei Cheng, Shikai Fang}{  lei{\_}cheng@zju.edu.cn,fangshikai@microsoft.com}

% You may provide any keywords that you
% find helpful for describing your paper; these are used to populate
% the "keywords" metadata in the PDF but will not be shown in the document
%\icmlkeywords{Machine Learning, ICML}

\vskip 0.3in
]

% this must go after the closing bracket ] following \twocolumn[ ...

% This command actually creates the footnote in the first column
% listing the affiliations and the copyright notice.
% The command takes one argument, which is text to display at the start of the footnote.
% The \icmlEqualContribution command is standard text for equal contribution.
% Remove it (just {}) if you do not need this facility.

%\printAffiliationsAndNotice{}  % leave blank if no need to mention equal contribution
\printAffiliationsAndNotice{} % otherwise use the standard text.
\begin{abstract}

Tensor decomposition is a fundamental tool for analyzing multi-dimensional data by learning low-rank factors to represent high-order interactions. While recent works on temporal tensor decomposition have made significant progress by incorporating continuous timestamps in latent factors, they still struggle with general tensor data with continuous indexes not only in the temporal mode but also in other modes, such as spatial coordinates in climate data. Additionally, the problem of determining the tensor rank remains largely unexplored in temporal tensor models. To address these limitations, we propose \underline{G}eneralized temporal tensor decomposition with \underline{R}ank-r\underline{E}vealing laten\underline{T}-ODE (\textsc{GreT}). 
 Our approach encodes continuous spatial indexes as learnable Fourier features and employs neural ODEs in latent space to learn the temporal trajectories of factors. To automatically reveal the rank of temporal tensors, we introduce a rank-revealing Gaussian-Gamma prior over the factor trajectories. We develop an efficient variational inference scheme with an analytical evidence lower bound, enabling sampling-free optimization. Through extensive experiments on both synthetic and real-world datasets, we demonstrate that \MODEL not only reveals the underlying ranks of temporal tensors but also significantly outperforms existing methods in prediction performance and robustness against noise.


    
    % Temporal tensor decomposition, which often incorporates continuous timestamps, is a powerful tool for handling multiway temporal data. Existing methods primarily focus on enc\MODEL  g these timestamps into the low-rank representations of tensors, which are associated with finite objects in each mode,  corresponding to discrete indexes. While effective for certain cases, these approaches overlook the intrinsic property of real-world dynamic tensor data like spatiotemporal tensors, where entries are characterized by both continuous mode indexes and timestamps.
    % %Although several methods based on functional tensors have been proposed recently to model continuous-indexed data, these approaches do not explicitly model temporal dynamics. 
    % To address the limitation, we propose  Ode-aided Deep varIational decompositioN for Continuous-indexed Temporal Tensor (\MODEL  ).
    % We treat the continuous-indexed temporal data as the interaction between  a group of latent functions,  under a sparse Bayesian framework with automatic rank (i.e., model complexity) determination.
    % We  model the posterior mean of the latent functions  with  the continuous-indexed latent ODE, which is efficient and scalable for large tensor data. Furthermore, we derive an analytical variational lower bound for the marginal data likelihood, allowing for efficient optimization via gradient descent without relying on reparameterization tricks.
    % Our method facilitates the learning of off-grid relationships and complex temporal dynamics while allowing for self-adaptation of model complexity.
    %  Extensive experiments on synthetic and real-world datasets demonstrate the superior performance of the proposed approach. 
    \end{abstract}
    
\section{Introduction}
% \begin{figure}
%     \centering
%     \includegraphics[width=\linewidth]{ssf_visual.eps}
%     \caption{Visualizations of the spatiotemporal tensor (ocean sound speed field) across different modes. It reveals that the signal exhibits significantly greater fluctuations in the time mode than in the other modes.}
%     \label{fig:ssf_visual}
% \end{figure}

Tensor is a ubiquitous data structure for organizing multi-dimensional data. For example, a four-mode tensor \textit{(longitude, latitude, depth, time)} can serve as a unified representation of spatiotemporal signals in the ocean, such as temperature or flow speed. Tensor decomposition is a prevailing framework for multiway data analysis that estimates latent factors to reconstruct the unobserved entries. Methods like CANDECOMP/PARAFAC (CP)\citep{HarshmanCP} and Tucker decomposition\citep{sidiropoulos2017tensor} are widely applied across fields, including climate science, oceanography, and social science.

An emerging trend in tensor community is to leverage the continuous timestamp of observed entries and build temporal tensor models, as the real-world tensor data is often irregularly collected in time, e.g., physical signals, accompanied with rich and complex time-varying patterns. The temporal tensor methods expand the classical tensor framework by using polynomial splines~\citep{zhang2021dynamic}, Gaussian processes~\citep{bctt,SFTL}, ODE~\citep{thisode} and energy-based models~\citep{tao2023undirected} to estimate the continuous temporal dynamics in latent space, instead of discretizing the time mode and setting a fixed number of factors.

Despite the successes of current temporal tensor methods, they inherit a fundamental limitation from traditional tensor models: they assume tensor data at each timestep must conform to a Cartesian grid structure with discrete indexes and finite-dimensional modes. This assumption poorly aligns with many real-world scenarios where modes are naturally continuous, such as spatial coordinates like \textit{(longitude, latitude, depth)}. To fit current models, we still need to discretize continuous indexes, which inevitably leads to a loss of fine-grained information encoded in these indexes. From a high-level perspective, while current temporal tensor methods have taken a crucial step forward by modeling continuous characteristics in the temporal mode compared to classical approaches, they still fail to fully utilize the rich complex patterns inherent in other continuous-indexed modes. 

Another crucial problem lies in determining the optimal rank for temporal tensor decomposition. As a fundamental hyperparameter in tensor modeling, the rank directly influences interpretability, sparsity, and model expressiveness. While classical tensor literature offers extensive theoretical analysis and learning-based solutions~\citep{zhao2015bayesianCP,cheng2022towards,pmlr-v32-rai14}, this topic has been largely overlooked in emerging temporal tensor methods. The introduction of dynamical patterns significantly complicates the latent landscape, and the lack of investigation into rank selection makes temporal tensor models more susceptible to hyperparameter choices and noise.

To fill these gaps, we propose 
\textsc{GreT}, a complexity-adaptive method for modeling temporal tensor data with continuous indexes across all modes.
Our method models general temporal tensor data with continuous indexes not only in the time mode but also in other modes. Specifically, \MODEL organizes the continuous indexes from non-temporal modes  into a Fourier-feature format, encodes them as the initial state of latent dynamics, and utilizes the neural ODE~\citep{chen2018neural} to model the factor trajectories.
To reveal the rank of the temporal tensor, \MODEL extends the classical rank selection framework~\citep{zhao2015bayesianCP} and assigns Gaussian-Gamma priors over factor trajectories to promote sparsity. For efficient inference, we propose a novel variational inference algorithm with an analytical evidence lower bound, enabling sampling-free inference  of the model parameters and latent dynamics.
For evaluation, we conducted experiments on both simulated and real-world tasks, demonstrating that \MODEL not only reveals the underlying ranks of temporal tensors but also significantly outperforms existing methods in prediction performance and robustness against noise.
\section{Preliminary}
\subsection{Tensor Decomposition}
Tensor decomposition   represents multi-dimensional arrays  by decomposing them into  lower-dimensional components, thus revealing underlying patterns in high-dimensional data. We denote a $K$-mode tensor as $\bc{Y} \in \mb{R}^{I_1\times \cdots \times I_k  \times \cdots \times I_K}$, where the $k$-$th$ mode  consists of $I_k$ dimensions. Each entry of $\bc{Y}$, termed $y_{\mf{i}}$, is indexed by a $K$-tuple $\mf{i}=(i_1,\cdots,i_k, \cdots, i_K)$, where $i_k$ denotes the index of the node along the mode $k$ ($1\le k \le K$). 
For tensor decomposition,  a set of  factor matrices $\{\mf{U}^{k}\}_{k=1}^{K}$ are introduced to represent the nodes in each mode. Specifically, the $k$-$th$ factor matrix $\mf{U}^{k}$ is composed of $I_k$ latent factors, i.e., $\mf{U}^{k}=[\mf{u}^{k}_{1}, \cdots, \mf{u}^{k}_{i_k}, \cdots,\mf{u}^{k}_{I_k}]^{\T} \in \mb{R}^{I_k \times R_k}$ and $\mf{u}^{k}_{i_k} = [u_{i_k,1}^{k}, \cdots, u_{i_k,r_k}^{k}, \cdots, u_{i_k,R_k}^{k}]^{\T} \in \mb{R}^{R_k}$, where $R_k$ denotes  the  rank of mode-$k$.
The classic CANDECOMP/PARAFAC (CP) decomposition ~\citep{HarshmanCP} aims to decompose a tensor into a sum of rank-one tensors. It sets $R_1 = \cdots = R_k =\cdots = R_K = R$ and represents each entry using 
\vspace{-2mm}
\begin{equation}
    y_{\mf{i}} \approx \boldsymbol{1}^{\T}[\underset{k}{\circledast} 
\mf{u}_{i_k}^k]=\sum_{r=1}^{R}\prod_{k=1}^{K}u^{k}_{i_k, r},
 \label{eq:CP}
\end{equation}
 where $\boldsymbol{1} \in \mb{R}^{R}$ is the all-one vector and 
 $\underset{k}{\circledast} $ is   Hadamard product of a set of vectors  defined as  $\underset{k}{\circledast}\mf{u}_{i_k}^k = (\mf{u}^{1}_{i_1} \circledast \cdots \circledast \mf{u}^{k}_{i_k} \circledast  \cdots \circledast \mf{u}^{K}_{i_K})$. Here,  $\circledast$ is  Hadamard product.
Another popular model is  Tucker decomposition~\citep{sidiropoulos2017tensor}, which approximates each entry with the interactions between a core tensor and $K$ latent factors. Tucker model will degenerate into  CP model when all modes' ranks are set to the same and the core tensor is diagonal.

% with
% \begin{equation}
% \begin{split}
%         y_{\mf{i}} &\approx \text{vec}(\bc{W})\underset{k}{\otimes} 
%  \mf{u}_{i_k}^k=\sum_{r=1}^{R_1}\cdots \sum_{r=K}^{R_K}[w_{r_1,\cdots,r_K}\prod_{k=1}^{K}u_{i_k, r_k}^{k} ], 
% \end{split}
% \end{equation}

% where $\bc{W}\in \mathbb{R}^{R_1 \times \cdots \times R_k \times \cdots \times R_K}$ denotes the core tensor, 
% $\text{vec}(\cdot)$ is the vectorization operator. Similarly,   $\underset{k}{\otimes} 
%  \mf{u}_{i_k}^k = (\mf{u}^{1}_{i_1} \otimes \cdots \otimes \mf{u}^{k}_{i_k} \otimes \cdots  \otimes \mf{u}^{K}_{i_K})$ where $\otimes$ is the Kronecker product. Tucker model is able to capture more interactions in data than the CP model and it degenerate into CP model when all modes' ranks are set to the same and $\bc{W}$ is diagonal.

\subsection{ Automatic Tensor Rank Determination}
% \fang{
% 1.pls add content on the NP-hardness of the tensor model's rank determination.}
The tensor rank $R$ 
determines the complexity of the tensor model. An improper choice of the rank can lead to overfitting  or underfitting to the signal sources, potentially compromising the model interpretability. However, the optimal determination of the tensor rank is known to be NP-hard ~\citep{cheng2022towards, kolda2009tensor(NPhard), haastad1989tensor_np}.
Rather than exhaustively searching for the optimal tensor rank via trial and error experiments,  Bayesian methods have been introduced to facilitate Tucker/CP decomposition with automatic tensor rank learning~\citep{morup2009automatic_ARD, zhao2015bayesianCP, cheng2022towards, pmlr-v32-rai14}. These methods impose sparsity-promoting priors (e.g., the Gaussian-Gamma prior and Laplacian prior)  on the latent factors.

For example, Bayesian CP decomposition with  Gaussian-Gamma priors models the mean and precision of all latent factors with  zero elements and a set of latent variables $\boldsymbol{\lambda}=[\lambda_1, \cdots, \lambda_r, \cdots, \lambda_R]^{\T} \in \mb{R}^{R}$, respectively:  
\begin{align}
        p(\mf{u}_{i_k}^{k}|\boldsymbol{\lambda}) = \mathcal{N}(\mf{u}_{i_k}^{k}|\boldsymbol{0}, \boldsymbol{\Lambda}^{-1}), \forall k,\label{eq:u}\\
    p(\boldsymbol{\lambda}) = \prod_{r=1}^{R} \text{Gamma}(\lambda_r|a_r^0, b_r^0),
    \label{eq:lambda}
\end{align}
where $\boldsymbol{\Lambda} = \text{diag}(\boldsymbol{\lambda})$ is the inverse covariance matrix  shared by all latent factors  over  $K$ modes.   Note that $R$ components of $\mf{u}_{i_k}^{k}$ are assumed to be statistically independent and the distribution of the $r$-$th$ component is controlled by $\lambda_r$. For example, if $\lambda_r$ is large, then the density function peaks at mean zero, so the $r$-$th$ component is concentrated at zero. Otherwise, if $\lambda_r$ is small (which leads to heavy tails), it allows the component to spread out to wider range of values. The conjugated Gamma priors are assigned to $\boldsymbol{\lambda}$. Here $\text{Gamma}(x|a,b)=\frac{b^ax^{a-1}e^{-bx}}{\Gamma(a)}$ for $x \ge 0$, which represents the Gamma distribution for $\boldsymbol{\lambda}$. In this context, $a$ and $b$ denote the shape and rate parameters respectively, and $\Gamma(\cdot)$ denotes the Gamma function. $\{a_r^0, b_r^0\}_{r=1}^R$ are pre-determined hyperparameters. 
The tensor rank will be automatically determined by the inferred posteriors of $\boldsymbol{\lambda}$.

\subsection{Generalized Tensor with Continuous Modes}
Real-world tensor data often contains continuous modes, prompting increased studies on generalized tensor data with continuous indexes. Existing approaches can be broadly classified into two categories:



\textit{1.Temporal tensor model with continuous timestamps:} Recent studies encode the representations of continuous timestamps into the latent factor of  CP model~\citep{SFTL, NONFAT} or the tensor core of  Tucker model~\citep{bctt}. 
Taking CP as an example, the temporal tensor model with continuous timestamps can be written as:
\vspace{-2mm}
\begin{align}
    y_{\mf{i}}(t) \approx \boldsymbol{1}^{\T}[\underset{k}{\circledast} 
    \mf{u}_{i_k}^k(t)], \label{eq:temporal_CP}
\end{align}
where $\mf{i}$ is the tensor index, $t$ is the continuous timestamp, $\mf{u}_{i_k}^k(t)$ is the factor trajectory of the $i_k$-th node in the $k$-th mode. Although this modeling approach effectively captures complex temporal dynamics, it is inadequate for generalizing to data with continuous indexes over the entire domain, such as spatiotemporal data which has continuous coordinates on both spatial and temporal modes.~\citep{hamdi2022spatiotemporal}. 

\textit{2.Functional tensor model:} 
Another popular model to handle  continuous-indexed modes is the functional tensor~\citep{schmidt2009function_tensor, luo2023lowrank, Ballester-Ripoll_Paredes_Pajarola_2019_tt}, which assumes that the continuous-indexed tensor can be factorized as a set of mode-wise functions and the continuous timestamp is simply modeled as an extra mode. Still taking CP as an example, the functional tensor model can be written as:
\begin{align}
    y_{\mf{i}}(t) \approx  \boldsymbol{1}^{\T}[\underset{k}{\circledast} 
    \mf{u}^k(i_k){\circledast} \mf{u}^{\text{Temporal}}(t) ],\label{eq:function_CP}
\end{align} 
where  $\mf{u}^k(i_k): \mathbb{R}_{+} \to \mathbb{R}^{R}$ is the latent vector-valued function of the $k$-th mode, which takes the continuous index $i_k$ as input and outputs the  latent factor. $\mf{u}^{\text{Temporal}}(t) : \mathbb{R}_{+} \to \mathbb{R}^{R}$ is the latent function of the temporal mode. 
The fully-factorized form of \eqref{eq:function_CP} models each mode equally and independently. It often overlooks the complex dynamics of the temporal mode, which requires special treatment~\citep{hamdi2022spatiotemporal}.




\section{Methodology}
\begin{figure*}
    \centering
    \includegraphics[width=\linewidth]{figure/flowchart.pdf}
    \caption{Graphical illustration of the proposed \textsc{Gret} (the case of $K=3$).}
    \label{fig:flowchart}
    \vspace{-0.05in}
\end{figure*}
Despite recent advances in modeling temporal tensors, most of these methods are still unsuitable for generalized tensor data with continuous indexes across all domains. 
While functional tensor methods offer greater flexibility, they simply treat temporal dynamics as an independent mode, which tends to underfit the inherent complexity of the temporal dynamics.  Furthermore, rank determination remains a less explored issue in temporal tensor models.
To address these issues, we propose \textsc{GreT}, a novel temporal tensor model that integrates the continuous-indexed  features into a latent ODE  model with rank-revealing prior. 

Without loss of generality, we consider a $K$-mode generalized temporal tensor  with continuous indexes over all domains, and it actually corresponds to a function $\mf{F}(i_1, \cdots, i_K, t):\mb{R}_{+}^{K+1}\rightarrow \mb{R}^{1}$ to map the continuous indexes and timestamp to the tensor entry, denoted as $y_{\mf{i}}(t) = \mf{F}(i_1, \cdots, i_K, t)$. We assume the function $\mf{F}(i_1, \cdots, i_K, t)$ can be factorized into $K$  factor trajectories following the CP format with rank $R$, i.e., 
\vspace{-2mm}
\begin{equation}    
    y_{\mf{i}}(t) = \mf{F}(i_1, \cdots, i_K, t) \approx \boldsymbol{1}^{\T}[\underset{k}{\circledast}  \mf{u}^k(i_k, t)], \label{eq:gen_CP}
\end{equation}
where $\mf{u}^k(i_k, t):\mb{R}_{+}^{2}\rightarrow \mb{R}^{R}$  is the trajectory of latent factor, which is a $R$-size vector-valued function mapping the continuous index $i_k$ of mode-$k$ and timestamp $t$ to a $R$-dimensional latent factor. We claim that the proposed model \eqref{eq:gen_CP} is a generalization of existing temporal tensor methods \eqref{eq:temporal_CP} via modeling continuous-indexed patterns not only in the temporal mode but in all modes. If we restrict $i_k$ to finite and discrete, \eqref{eq:gen_CP} will degrade to \eqref{eq:temporal_CP}. Compared to the fully-factorized functional tensor \eqref{eq:function_CP}, the proposed method \eqref{eq:gen_CP} explicitly models the time-varying factor trajectories of all modes, known as dynamic factor learning~\citep{SFTL,NONFAT}. Given the fact that temporal mode always dominates and interacts with other modes, the proposed method is expected to improve the model's capability by learning time-varying representations in dynamical data.

\subsection{Continuous-indexed Latent-ODE}\label{sec:Latent-ODE-Flow}
To allow flexible modeling of the factor trajectory and continuous-indexed information, 
 we propose a temporal function $\mf{g}^k(i_k,t):\mb{R}^{2}_{+}\to \mb{R}^{R}$ based on encoder-decoder structure and neural ODE~\citep{chen2018neural} to approximate the factor trajectory $\mf{u}^k(i_k,t)$ of mode-$k$.  Specifically, we have:
 \vspace{-2mm}
\begin{align}
        \mf{z}^{k}(i_k,0) = & \text{Encoder}\big([\cos(2\pi\mf{b}_ki_k); \sin(2\pi\mf{b}_ki_k)]\big), \label{eq:latent-ode1}\\
        \mf{z}^{k}(i_k,t) = & \mf{z}^{k}(i_k,0) + \int_0^{t} h_{\boldsymbol{\theta}_k}(\mf{z}^{k}(i_k,s), s)ds,\label{eq:latent-ode2}\\
        \mf{g}^k(i_k,t) = & \text{Decoder}\big(\mf{z}^{k}(i_k,t)).\label{eq:latent-ode3}
\end{align}
 Eq.~\eqref{eq:latent-ode1} shows that how we obtain $\mf{z}(i,0) \in \mb{R}^{J}$, the initial state of the latent dynamics by encoding the continuous index $i_k$. In particular, the input coordinate $i_k$ is firstly expanded into a set of Fourier features $[\cos(2\pi\mf{b}_ki_k); \sin(2\pi\mf{b}_ki_k)] \in \mb{R}^{2M}$, where $\mf{b}_k\in \mb{R}^{M}$ is a learnable vector that scales  $i_k$  by $M$ different frequencies. This effectively expands the input space with high-frequency components~\citep{tancik2020fourier}, helping to capture fine-grained index information. The Fourier features are then fed into an encoder
$\text{Encoder}(\cdot):\mb{R}^{2M}\to \mb{R}^{J}$ to get $\mf{z}^{k}(i_k,0)$. Give the initial state, we then apply a neural network $ h_{\boldsymbol{\theta}_k}(\mf{z}^{k}(i_k,s), s):\mb{R}^{J}\to \mb{R}^{J}$ to model the state transition of the dynamics at each timestamp, which is parameterized by $\boldsymbol{\theta}_k$, and the state value can be calculated through integrations as shown in \eqref{eq:latent-ode2}. Finally, we will pass the output of the latent dynamics through a decoder $\text{Decoder}(\cdot):\mb{R}^{J}\to \mb{R}^{R}$ to obtain $\mf{g}^k(i_k,t)$ as the approximation  of the factor trajectory, as described in \eqref{eq:latent-ode3}. We  simply use the multilayer perceptrons (MLPs) to parameterize the encoder and the decoder.

 Note that \eqref{eq:latent-ode2} actually represents the neural ODE model~\citep{Tenenbaum_Pollard_ode, chen2018neural}, and we  follow \citet{chen2018neural} to track the gradient of $\boldsymbol{\theta}_k$ efficiently, when we handle the integration to obtain $\mf{z}^k(i_k,t)$ in \eqref{eq:latent-ode2} at arbitrary $t$ by using numerical ODE solvers:
 \vspace{-2mm}
 \begin{equation}
    \mf{z}^k(i_k,t) =  \text{ODESolve}(\mf{z}^k(i_k,0),h_{\boldsymbol{\theta}_k}).
\end{equation}
For computing efficiency, we  concatenate the initial states of multiple indexes together, and construct a matrix-valued trajectory, where each row corresponds to the initial state of an unique index. Then, we only need to call the ODE solver once to obtain the factor trajectory of observed indexes. For simplicity, we denote the all learnable parameters in \eqref{eq:latent-ode1}-\eqref{eq:latent-ode3} as $\boldsymbol{\omega}_k$ for mode-$k$, which includes the frequency-scale vectors $\mf{b}_k$ as well as the parameters of neural ODE $\boldsymbol{\theta}_k$ and the encoder-decoder.


\subsection{Rank-revealing Prior over Factor Trajectories}
To automatically determine the underlying rank in the dynamical scenario, we apply the Bayesian sparsity-promoting priors. Specifically, we extend the classical framework on automatic rank determination~\citep{zhao2015bayesianCP}, stated in \eqref{eq:u}\eqref{eq:lambda}, and assign  a dimension-wise Gaussian-Gamma prior to the factor trajectory,
\vspace{-2mm}
\begin{equation}
    p(\mf{u}^{k}(i_k,t)|\boldsymbol{\lambda}) = \mathcal{N}(\mf{u}^{k}(i_k,t)|\boldsymbol{0}, \boldsymbol{\Lambda}^{-1}), \forall k,
\end{equation}
where $\boldsymbol{\Lambda} = \text{diag}(\boldsymbol{\lambda})$ and $\boldsymbol{\lambda}=[\lambda_1, \cdots, \lambda_r, \cdots, \lambda_R]^{\T} \in \mb{R}^{R}$. We assign Gamma priors to $\boldsymbol{\lambda}$: $p(\boldsymbol{\lambda}) = \prod_{r=1}^{R} \text{Gamma}(\lambda_r|a_r^0, b_r^0)$, identical to \eqref{eq:lambda}. Then, the rank-revealing prior over all factor trajectories is:
\vspace{-2mm}
\begin{equation}
    p(\mathcal{U}, \boldsymbol{\lambda}) = p(\boldsymbol{\lambda}) \prod_{k=1}^K p(\mf{u}^{k}(i_k,t)|\boldsymbol{\lambda}), \label{eq:ODE_prior} 
\end{equation}
where $\mathcal{U}$ denotes the set of  factor trajectories $\{\mf{u}^{k}(\cdot,\cdot)\}_{k=1}^{K}$.
It is worth noting that the proposed prior is assigned over a group of latent functions, but not a set of static factors~\citep{zhao2015bayesianCP}. With proper inference, the informative components of each factor trajectory, i.e., the rank of the generalized temporal tensor, can be automatically revealed, and redundant components can be pruned.  

With finite observed entries $\mathcal{D}=\{y_n, \mf{i}_n, t_n\}_{n=1}^{N}$, where $y_{n}$ denotes the $n$-$th$ entry observed at continuous index tuple $\mf{i}_n = (i_1^n, \cdots, i_K^n)$ and timestamp $t_n$. Our goal is to learn a factorized function as described in \eqref{eq:gen_CP}  to construct a direct mapping from $(\mf{i}_n, t_n)$ to $y_{n}$. 
Therefore, for each observed entry $\{y_n, \mf{i}_n, t_n\}$, we model the Gaussian likelihood  as:
\vspace{-1mm}
\begin{equation}
\begin{split}
        p(y_{n}&|\mathcal{U},\tau)=\mathcal{N}(y_{n}|\boldsymbol{1}^{\T}[\underset{k}{\circledast} 
 \mf{u}^k(i_k^n,t_n)], \tau^{-1}),
        \label{eq:likelihood}
\end{split}
\end{equation}
where  $\tau$ is the inverse of the observation noise. We further assign a Gamma prior, $p(\tau) = \text{Gamma}(\tau|c^0, d^0)$, and the joint probability can be written as:
\vspace{-2mm}
\begin{equation}
\begin{split}
        &p(\mathcal{U}, \tau, \mathcal{D}) =p(\mathcal{U}, \boldsymbol{\lambda})p(\tau) \prod_{n=1}^{N}p(y_{n}|\mathcal{U}, \tau).
\end{split}
    \label{eq:joint}
\end{equation}
We illustrate \textsc{GreT} with the case of $K=3$ in Figure~\ref{fig:flowchart}.
\section{Model Inference}
\subsection{Factorized Posterior and Analytical Evidence Lower Bound}
It is intractable to compute the full posterior of latent variables in \eqref{eq:joint} due to the high-dimensional integral and complex form of likelihood. We take a workaround to construct a variational distribution $q(\mathcal{U}, \boldsymbol{\lambda}, \tau)$ to approximate the exact posterior $p(\mathcal{U}, \boldsymbol{\lambda}, \tau|\mathcal{D})$. 
Similar to the widely-used mean-field assumption, we  design the approximate posterior in a fully factorized form: $q(\mathcal{U}, \boldsymbol{\lambda}, \tau) = q(\mathcal{U})q(\boldsymbol{\lambda})q(\tau)$.

Specifically, the conditional conjugate property of Gaussian-Gamma distribution motivates us to formulate the corresponding variational posteriors as follows:
\vspace{-2mm}
\begin{equation}
    \begin{split}
        &q(\mathcal{U}) = \prod_{n=1}^{N}\prod_{k=1}^{K} \mathcal{N}(\mf{u}^k(i_k^n,t_n)|\mf{g}^k(i_k^n,t_n), \sigma^2\mf{I}),
        \label{eq:q_u} 
    \end{split}
\end{equation}
where $\mf{g}^k(\cdot, \cdot)$  is the mode-wise latent temporal representations parameterized by $\boldsymbol{\omega}_k$, as we mentioned in Section \ref{sec:Latent-ODE-Flow}, and $\sigma$ is the variational variance shared by all $\mf{u}^k$.

Similarly, we formulate $q(\boldsymbol{\lambda}), q(\tau)$ as:
\vspace{-2mm}
\begin{align}
        &q(\boldsymbol{\lambda}) =  \prod_{r=1}^{R} \text{Gamma}(\lambda_r|\alpha_r, \beta_r),\label{eq:q_lambda}\\
        &q(\tau) = \text{Gamma}(\tau|\rho, \iota),\label{eq:q_tau}
\end{align}
where $\{\alpha_r, \beta_r\}_{r=1}^{R}, \rho, \iota$ are the variational parameters to characterize the approximated posteriors. 

Our goal is to estimate the latent ODE parameters $\boldsymbol{\omega}_k$ and variational parameters $\{ \{\alpha_r, \beta_r\}_{r=1}^{R}, \sigma, \rho, \iota \}$ in \eqref{eq:q_u} \eqref{eq:q_lambda} \eqref{eq:q_tau} to make the approximated posterior $q(\mathcal{U}, \boldsymbol{\lambda}, \tau)$ as close as possible to the true posterior $p(\mathcal{U}, \boldsymbol{\lambda}, \tau|\mathcal{D})$. To do so, we follow the variational inference framework~\citep{variational_inference} and construct the following objective function by minimizing the Kullback-Leibler (KL) divergence between the approximated posterior and the true posterior $\text{KL}(q(\mathcal{U}, \boldsymbol{\lambda}, \tau)\|p(\mathcal{U}, \boldsymbol{\lambda}, \tau|\mathcal{D}))$, which  leads to the maximization of the evidence lower bound (ELBO): 
\vspace{-2mm}
\begin{align}
        &\text{ELBO} = \mb{E}_{q(\mathcal{U}, \boldsymbol{\lambda}, \tau)}[\ln p(\mathcal{D}|\mathcal{U}, \boldsymbol{\lambda}, \tau)] +\mb{E}_{q(\mathcal{U}, \boldsymbol{\lambda})}[\ln 
        \frac{p(\mathcal{U}|\boldsymbol{\lambda})}{q(\mathcal{U})}] \nonumber \\
        &\qquad - \text{KL}(q(\boldsymbol{\lambda})\|p(\boldsymbol{\lambda})) - \text{KL}(q(\tau)\|p(\tau)) \label{eq:elbo}.
\end{align}
The ELBO is consist of four terms. The first term is posterior expectation of log-likelihood  while the last three are KL terms. Usually, the first term is intractable if the likelihood model is complicated and requires the costly  sampling-based approximation to handle the integrals in the expectation ~\citep{doersch2016tutorialVAE, NONFAT}. Fortunately, by leveraging the well-designed conjugate priors and factorized structure of the posterior, we make an endeavor to derive its analytical expression:
\vspace{-2mm}
\begin{align}
        &\mb{E}_{q(\mathcal{U}, \boldsymbol{\lambda}, \tau)}[\ln p(\mathcal{D}|\mathcal{U}, \boldsymbol{\lambda}, \tau)] =-\frac{N}{2}\ln(2\pi) + \frac{N}{2}(\psi(\rho)-\ln\iota) \nonumber\\
        &-\frac{1}{2}\sum_{n=1}^{N}\frac{\rho}{\iota}\big\{ y_n^2 -2y_n\{\boldsymbol{1}^{\T}[\underset{k}{\circledast} \mf{g}^k(i_k^n,t_n)]\}\nonumber \\
&  +\boldsymbol{1}^{\T}[\underset{k}{\circledast} \text{vec}(\mf{g}^{k}(i_k^{n},t_n)\mf{g}^{k}(i_k^{n},t_n)^{\T}+\sigma^2\mf{I})]\big\},\label{term1}
\end{align}
where $\mf{g}^k_r(i_k^n, t_n)$ is the $r$-th element of the $k$-th mode's latent temporal representation $\mf{g}^k(i_k^n, t_n)$. We refer to Appendix \ref{ap:A.2} for the detailed derivation. 
%We note that the first term computes the posterior expectation of the log-likelihood of the noisy observation.
%, which approximates  the MAE criterion.
The second term of \eqref{eq:elbo} computes the KL divergence between prior and posterior of factor trajectories, which is also with a closed from:
\vspace{-2mm}
\begin{align}
    &\mb{E}_{q(\mathcal{U}, \boldsymbol{\lambda})}[\ln 
    \frac{p(\mathcal{U}|\boldsymbol{\lambda})}{q(\mathcal{U})}]= -\text{KL}(q(\mathcal{U}) \| p(\mathcal{U}|\boldsymbol{\lambda}=\mb{E}_q(\boldsymbol{\lambda})))=\nonumber \\
    &-\sum_{n=1}^N \sum_{k=1}^K\sum_{r=1}^R \frac{1}{2}\big\{\ln(\frac{\beta_r}{\alpha_r\sigma^2})+ \frac{\alpha_r}{\beta_r}\{\sigma^2+[\mf{g}^k_r(i_k^n, t_n)]^2\}-1\big\}, \label{term2}
\end{align}
where $\mb{E}_q(\boldsymbol{\lambda})=[\mb{E}_q({\lambda_1}),\ldots,\mb{E}_q({\lambda_R})]^{\T}=[\frac{\alpha_1}{\beta_1},\ldots,\frac{\alpha_R}{\beta_R}]^{\T}$.
This term encourages rank reduction,  as it drives the 
posterior mean of $\lambda_r$  to be large, thereby forcing the corresponding $r$-th component of $K$ factor trajectories $\{\mf{g}^{k}_r(\cdot, \cdot)\}_{k=1}^K$ to be zero. The combination of the above two terms enables an automatic rank determination mechanism by striking a balance between capacity of representation and model complexity. 
As the prior and posterior of $\boldsymbol{\lambda}$ and $\tau$ are both Gamma distribution, the last two KL terms in the ELBO are analytically computable, as shown in \eqref{eq:kl1}\eqref{eq:kl2} in Appendix \ref{ap:A.2}.




We highlight that all terms in \eqref{eq:elbo} are with analytical forms, so we don't need sampling-based approximation to compute the ELBO. 
This offers a significant advantage during training, as we can directly compute the gradient of the ELBO with respect to the variational parameters, enabling the use of standard gradient-based optimization methods to optimize both the variational and latent ODE parameters:
\begin{equation} 
    \text{argmax}_{\{\boldsymbol{\omega}_k\}_{k=1}^K, \{\alpha_r, \beta_r\}_{r=1}^{R}, \sigma, \rho, \iota} \text{ELBO}.
    \label{eq:loss}
\end{equation}
We summarize the inference algorithm in Algorithm 1 in Appendix~\ref{ap:algorithm}. 
When $N$ is large, we can use the mini-batch gradient descent method to accelerate the optimization process.

\subsection{Closed Form of Predictive Distribution}
After obtaining the variational posteriors of the latent variables, we can further derive the predictive distribution of the new data with arbitrary indexes. Given the index set $\{i^p_1,\cdots, i^p_k, t_p\}$ for prediction, we can obtain the variational predictive posterior distribution, which follows a Student's t-distribution (See Appendix \ref{pred_distr} for more details):
\begin{equation}
    \begin{split}
        &p(y_p|\mathcal{D}) \sim \mathcal{T}(y_p|\mu_p, s_p, \nu_p),\\
       &\mu_p =  \boldsymbol{1}^{\T}[\underset{k}{\circledast} 
 \mf{g}^k(i_k^p, t_p)], \quad \nu_p = 2\rho,\\
        &s_p =\big\{\frac{\iota}{\rho}+\sigma^2\sum_{j=1}^K[\underset{k\ne j}{\circledast} 
 \mf{g}^k(i_k^p, t_p)]^{\T}[\underset{k\ne j}{\circledast} 
 \mf{g}^k(i_k^p, t_p)]\big\}^{-1},
    \end{split}
\end{equation}
where $\mu_p$, $s_p$, $\mu_p$ is the mean, scale parameter and degree of freedom of the Student's t-distribution, respectively. The closed-form  predictive distribution is a great advantage for the prediction process, as it allows us to do the probabilistic reconstruction and prediction with uncertainty quantification over the arbitrary continuous indexes.


% % Algorithm
% \begin{algorithm}[H]
% \SetAlgoLined
% \caption{Training process of \MODEL}
% \KwIn{Training data $\mathcal{D}=\{y_n, \mf{i}_n, t_n\}_{n=1}^{N}$ }
% Collect all possible $I_k$ indexes for $K$ modes and construct $K$ initial time embedding tables $\{\mf{Z}_0^k\}_{k=1}^{K}$ with Encoder.
% \BlankLine
% % Steps
% \REPEAT{
% \For{$i = 1, 2, \cdots$}{
%     $\{\mf{Z}^k(t)\}_{k=1}^{K}$ = \text{ODESolve}($\{\mf{Z}^k_0\}_{k=1}^{K}$, 
%     $\{h_{\boldsymbol{\theta}}\}_{k=1}^{K}$,$t_{i-1},t_i)$
%     }
%     }
% Return $Y$\;
% \end{algorithm}




%\textit{Interpretation of Evidence Lower Bound:}


\section{RELATED WORK}
\label{sec:relatedwork}
In this section, we describe the previous works related to our proposal, which are divided into two parts. In Section~\ref{sec:relatedwork_exoplanet}, we present a review of approaches based on machine learning techniques for the detection of planetary transit signals. Section~\ref{sec:relatedwork_attention} provides an account of the approaches based on attention mechanisms applied in Astronomy.\par

\subsection{Exoplanet detection}
\label{sec:relatedwork_exoplanet}
Machine learning methods have achieved great performance for the automatic selection of exoplanet transit signals. One of the earliest applications of machine learning is a model named Autovetter \citep{MCcauliff}, which is a random forest (RF) model based on characteristics derived from Kepler pipeline statistics to classify exoplanet and false positive signals. Then, other studies emerged that also used supervised learning. \cite{mislis2016sidra} also used a RF, but unlike the work by \citet{MCcauliff}, they used simulated light curves and a box least square \citep[BLS;][]{kovacs2002box}-based periodogram to search for transiting exoplanets. \citet{thompson2015machine} proposed a k-nearest neighbors model for Kepler data to determine if a given signal has similarity to known transits. Unsupervised learning techniques were also applied, such as self-organizing maps (SOM), proposed \citet{armstrong2016transit}; which implements an architecture to segment similar light curves. In the same way, \citet{armstrong2018automatic} developed a combination of supervised and unsupervised learning, including RF and SOM models. In general, these approaches require a previous phase of feature engineering for each light curve. \par

%DL is a modern data-driven technology that automatically extracts characteristics, and that has been successful in classification problems from a variety of application domains. The architecture relies on several layers of NNs of simple interconnected units and uses layers to build increasingly complex and useful features by means of linear and non-linear transformation. This family of models is capable of generating increasingly high-level representations \citep{lecun2015deep}.

The application of DL for exoplanetary signal detection has evolved rapidly in recent years and has become very popular in planetary science.  \citet{pearson2018} and \citet{zucker2018shallow} developed CNN-based algorithms that learn from synthetic data to search for exoplanets. Perhaps one of the most successful applications of the DL models in transit detection was that of \citet{Shallue_2018}; who, in collaboration with Google, proposed a CNN named AstroNet that recognizes exoplanet signals in real data from Kepler. AstroNet uses the training set of labelled TCEs from the Autovetter planet candidate catalog of Q1–Q17 data release 24 (DR24) of the Kepler mission \citep{catanzarite2015autovetter}. AstroNet analyses the data in two views: a ``global view'', and ``local view'' \citep{Shallue_2018}. \par


% The global view shows the characteristics of the light curve over an orbital period, and a local view shows the moment at occurring the transit in detail

%different = space-based

Based on AstroNet, researchers have modified the original AstroNet model to rank candidates from different surveys, specifically for Kepler and TESS missions. \citet{ansdell2018scientific} developed a CNN trained on Kepler data, and included for the first time the information on the centroids, showing that the model improves performance considerably. Then, \citet{osborn2020rapid} and \citet{yu2019identifying} also included the centroids information, but in addition, \citet{osborn2020rapid} included information of the stellar and transit parameters. Finally, \citet{rao2021nigraha} proposed a pipeline that includes a new ``half-phase'' view of the transit signal. This half-phase view represents a transit view with a different time and phase. The purpose of this view is to recover any possible secondary eclipse (the object hiding behind the disk of the primary star).


%last pipeline applies a procedure after the prediction of the model to obtain new candidates, this process is carried out through a series of steps that include the evaluation with Discovery and Validation of Exoplanets (DAVE) \citet{kostov2019discovery} that was adapted for the TESS telescope.\par
%



\subsection{Attention mechanisms in astronomy}
\label{sec:relatedwork_attention}
Despite the remarkable success of attention mechanisms in sequential data, few papers have exploited their advantages in astronomy. In particular, there are no models based on attention mechanisms for detecting planets. Below we present a summary of the main applications of this modeling approach to astronomy, based on two points of view; performance and interpretability of the model.\par
%Attention mechanisms have not yet been explored in all sub-areas of astronomy. However, recent works show a successful application of the mechanism.
%performance

The application of attention mechanisms has shown improvements in the performance of some regression and classification tasks compared to previous approaches. One of the first implementations of the attention mechanism was to find gravitational lenses proposed by \citet{thuruthipilly2021finding}. They designed 21 self-attention-based encoder models, where each model was trained separately with 18,000 simulated images, demonstrating that the model based on the Transformer has a better performance and uses fewer trainable parameters compared to CNN. A novel application was proposed by \citet{lin2021galaxy} for the morphological classification of galaxies, who used an architecture derived from the Transformer, named Vision Transformer (VIT) \citep{dosovitskiy2020image}. \citet{lin2021galaxy} demonstrated competitive results compared to CNNs. Another application with successful results was proposed by \citet{zerveas2021transformer}; which first proposed a transformer-based framework for learning unsupervised representations of multivariate time series. Their methodology takes advantage of unlabeled data to train an encoder and extract dense vector representations of time series. Subsequently, they evaluate the model for regression and classification tasks, demonstrating better performance than other state-of-the-art supervised methods, even with data sets with limited samples.

%interpretation
Regarding the interpretability of the model, a recent contribution that analyses the attention maps was presented by \citet{bowles20212}, which explored the use of group-equivariant self-attention for radio astronomy classification. Compared to other approaches, this model analysed the attention maps of the predictions and showed that the mechanism extracts the brightest spots and jets of the radio source more clearly. This indicates that attention maps for prediction interpretation could help experts see patterns that the human eye often misses. \par

In the field of variable stars, \citet{allam2021paying} employed the mechanism for classifying multivariate time series in variable stars. And additionally, \citet{allam2021paying} showed that the activation weights are accommodated according to the variation in brightness of the star, achieving a more interpretable model. And finally, related to the TESS telescope, \citet{morvan2022don} proposed a model that removes the noise from the light curves through the distribution of attention weights. \citet{morvan2022don} showed that the use of the attention mechanism is excellent for removing noise and outliers in time series datasets compared with other approaches. In addition, the use of attention maps allowed them to show the representations learned from the model. \par

Recent attention mechanism approaches in astronomy demonstrate comparable results with earlier approaches, such as CNNs. At the same time, they offer interpretability of their results, which allows a post-prediction analysis. \par


\section{Experiments}
\label{sec:experiments}
The experiments are designed to address two key research questions.
First, \textbf{RQ1} evaluates whether the average $L_2$-norm of the counterfactual perturbation vectors ($\overline{||\perturb||}$) decreases as the model overfits the data, thereby providing further empirical validation for our hypothesis.
Second, \textbf{RQ2} evaluates the ability of the proposed counterfactual regularized loss, as defined in (\ref{eq:regularized_loss2}), to mitigate overfitting when compared to existing regularization techniques.

% The experiments are designed to address three key research questions. First, \textbf{RQ1} investigates whether the mean perturbation vector norm decreases as the model overfits the data, aiming to further validate our intuition. Second, \textbf{RQ2} explores whether the mean perturbation vector norm can be effectively leveraged as a regularization term during training, offering insights into its potential role in mitigating overfitting. Finally, \textbf{RQ3} examines whether our counterfactual regularizer enables the model to achieve superior performance compared to existing regularization methods, thus highlighting its practical advantage.

\subsection{Experimental Setup}
\textbf{\textit{Datasets, Models, and Tasks.}}
The experiments are conducted on three datasets: \textit{Water Potability}~\cite{kadiwal2020waterpotability}, \textit{Phomene}~\cite{phomene}, and \textit{CIFAR-10}~\cite{krizhevsky2009learning}. For \textit{Water Potability} and \textit{Phomene}, we randomly select $80\%$ of the samples for the training set, and the remaining $20\%$ for the test set, \textit{CIFAR-10} comes already split. Furthermore, we consider the following models: Logistic Regression, Multi-Layer Perceptron (MLP) with 100 and 30 neurons on each hidden layer, and PreactResNet-18~\cite{he2016cvecvv} as a Convolutional Neural Network (CNN) architecture.
We focus on binary classification tasks and leave the extension to multiclass scenarios for future work. However, for datasets that are inherently multiclass, we transform the problem into a binary classification task by selecting two classes, aligning with our assumption.

\smallskip
\noindent\textbf{\textit{Evaluation Measures.}} To characterize the degree of overfitting, we use the test loss, as it serves as a reliable indicator of the model's generalization capability to unseen data. Additionally, we evaluate the predictive performance of each model using the test accuracy.

\smallskip
\noindent\textbf{\textit{Baselines.}} We compare CF-Reg with the following regularization techniques: L1 (``Lasso''), L2 (``Ridge''), and Dropout.

\smallskip
\noindent\textbf{\textit{Configurations.}}
For each model, we adopt specific configurations as follows.
\begin{itemize}
\item \textit{Logistic Regression:} To induce overfitting in the model, we artificially increase the dimensionality of the data beyond the number of training samples by applying a polynomial feature expansion. This approach ensures that the model has enough capacity to overfit the training data, allowing us to analyze the impact of our counterfactual regularizer. The degree of the polynomial is chosen as the smallest degree that makes the number of features greater than the number of data.
\item \textit{Neural Networks (MLP and CNN):} To take advantage of the closed-form solution for computing the optimal perturbation vector as defined in (\ref{eq:opt-delta}), we use a local linear approximation of the neural network models. Hence, given an instance $\inst_i$, we consider the (optimal) counterfactual not with respect to $\model$ but with respect to:
\begin{equation}
\label{eq:taylor}
    \model^{lin}(\inst) = \model(\inst_i) + \nabla_{\inst}\model(\inst_i)(\inst - \inst_i),
\end{equation}
where $\model^{lin}$ represents the first-order Taylor approximation of $\model$ at $\inst_i$.
Note that this step is unnecessary for Logistic Regression, as it is inherently a linear model.
\end{itemize}

\smallskip
\noindent \textbf{\textit{Implementation Details.}} We run all experiments on a machine equipped with an AMD Ryzen 9 7900 12-Core Processor and an NVIDIA GeForce RTX 4090 GPU. Our implementation is based on the PyTorch Lightning framework. We use stochastic gradient descent as the optimizer with a learning rate of $\eta = 0.001$ and no weight decay. We use a batch size of $128$. The training and test steps are conducted for $6000$ epochs on the \textit{Water Potability} and \textit{Phoneme} datasets, while for the \textit{CIFAR-10} dataset, they are performed for $200$ epochs.
Finally, the contribution $w_i^{\varepsilon}$ of each training point $\inst_i$ is uniformly set as $w_i^{\varepsilon} = 1~\forall i\in \{1,\ldots,m\}$.

The source code implementation for our experiments is available at the following GitHub repository: \url{https://anonymous.4open.science/r/COCE-80B4/README.md} 

\subsection{RQ1: Counterfactual Perturbation vs. Overfitting}
To address \textbf{RQ1}, we analyze the relationship between the test loss and the average $L_2$-norm of the counterfactual perturbation vectors ($\overline{||\perturb||}$) over training epochs.

In particular, Figure~\ref{fig:delta_loss_epochs} depicts the evolution of $\overline{||\perturb||}$ alongside the test loss for an MLP trained \textit{without} regularization on the \textit{Water Potability} dataset. 
\begin{figure}[ht]
    \centering
    \includegraphics[width=0.85\linewidth]{img/delta_loss_epochs.png}
    \caption{The average counterfactual perturbation vector $\overline{||\perturb||}$ (left $y$-axis) and the cross-entropy test loss (right $y$-axis) over training epochs ($x$-axis) for an MLP trained on the \textit{Water Potability} dataset \textit{without} regularization.}
    \label{fig:delta_loss_epochs}
\end{figure}

The plot shows a clear trend as the model starts to overfit the data (evidenced by an increase in test loss). 
Notably, $\overline{||\perturb||}$ begins to decrease, which aligns with the hypothesis that the average distance to the optimal counterfactual example gets smaller as the model's decision boundary becomes increasingly adherent to the training data.

It is worth noting that this trend is heavily influenced by the choice of the counterfactual generator model. In particular, the relationship between $\overline{||\perturb||}$ and the degree of overfitting may become even more pronounced when leveraging more accurate counterfactual generators. However, these models often come at the cost of higher computational complexity, and their exploration is left to future work.

Nonetheless, we expect that $\overline{||\perturb||}$ will eventually stabilize at a plateau, as the average $L_2$-norm of the optimal counterfactual perturbations cannot vanish to zero.

% Additionally, the choice of employing the score-based counterfactual explanation framework to generate counterfactuals was driven to promote computational efficiency.

% Future enhancements to the framework may involve adopting models capable of generating more precise counterfactuals. While such approaches may yield to performance improvements, they are likely to come at the cost of increased computational complexity.


\subsection{RQ2: Counterfactual Regularization Performance}
To answer \textbf{RQ2}, we evaluate the effectiveness of the proposed counterfactual regularization (CF-Reg) by comparing its performance against existing baselines: unregularized training loss (No-Reg), L1 regularization (L1-Reg), L2 regularization (L2-Reg), and Dropout.
Specifically, for each model and dataset combination, Table~\ref{tab:regularization_comparison} presents the mean value and standard deviation of test accuracy achieved by each method across 5 random initialization. 

The table illustrates that our regularization technique consistently delivers better results than existing methods across all evaluated scenarios, except for one case -- i.e., Logistic Regression on the \textit{Phomene} dataset. 
However, this setting exhibits an unusual pattern, as the highest model accuracy is achieved without any regularization. Even in this case, CF-Reg still surpasses other regularization baselines.

From the results above, we derive the following key insights. First, CF-Reg proves to be effective across various model types, ranging from simple linear models (Logistic Regression) to deep architectures like MLPs and CNNs, and across diverse datasets, including both tabular and image data. 
Second, CF-Reg's strong performance on the \textit{Water} dataset with Logistic Regression suggests that its benefits may be more pronounced when applied to simpler models. However, the unexpected outcome on the \textit{Phoneme} dataset calls for further investigation into this phenomenon.


\begin{table*}[h!]
    \centering
    \caption{Mean value and standard deviation of test accuracy across 5 random initializations for different model, dataset, and regularization method. The best results are highlighted in \textbf{bold}.}
    \label{tab:regularization_comparison}
    \begin{tabular}{|c|c|c|c|c|c|c|}
        \hline
        \textbf{Model} & \textbf{Dataset} & \textbf{No-Reg} & \textbf{L1-Reg} & \textbf{L2-Reg} & \textbf{Dropout} & \textbf{CF-Reg (ours)} \\ \hline
        Logistic Regression   & \textit{Water}   & $0.6595 \pm 0.0038$   & $0.6729 \pm 0.0056$   & $0.6756 \pm 0.0046$  & N/A    & $\mathbf{0.6918 \pm 0.0036}$                     \\ \hline
        MLP   & \textit{Water}   & $0.6756 \pm 0.0042$   & $0.6790 \pm 0.0058$   & $0.6790 \pm 0.0023$  & $0.6750 \pm 0.0036$    & $\mathbf{0.6802 \pm 0.0046}$                    \\ \hline
%        MLP   & \textit{Adult}   & $0.8404 \pm 0.0010$   & $\mathbf{0.8495 \pm 0.0007}$   & $0.8489 \pm 0.0014$  & $\mathbf{0.8495 \pm 0.0016}$     & $0.8449 \pm 0.0019$                    \\ \hline
        Logistic Regression   & \textit{Phomene}   & $\mathbf{0.8148 \pm 0.0020}$   & $0.8041 \pm 0.0028$   & $0.7835 \pm 0.0176$  & N/A    & $0.8098 \pm 0.0055$                     \\ \hline
        MLP   & \textit{Phomene}   & $0.8677 \pm 0.0033$   & $0.8374 \pm 0.0080$   & $0.8673 \pm 0.0045$  & $0.8672 \pm 0.0042$     & $\mathbf{0.8718 \pm 0.0040}$                    \\ \hline
        CNN   & \textit{CIFAR-10} & $0.6670 \pm 0.0233$   & $0.6229 \pm 0.0850$   & $0.7348 \pm 0.0365$   & N/A    & $\mathbf{0.7427 \pm 0.0571}$                     \\ \hline
    \end{tabular}
\end{table*}

\begin{table*}[htb!]
    \centering
    \caption{Hyperparameter configurations utilized for the generation of Table \ref{tab:regularization_comparison}. For our regularization the hyperparameters are reported as $\mathbf{\alpha/\beta}$.}
    \label{tab:performance_parameters}
    \begin{tabular}{|c|c|c|c|c|c|c|}
        \hline
        \textbf{Model} & \textbf{Dataset} & \textbf{No-Reg} & \textbf{L1-Reg} & \textbf{L2-Reg} & \textbf{Dropout} & \textbf{CF-Reg (ours)} \\ \hline
        Logistic Regression   & \textit{Water}   & N/A   & $0.0093$   & $0.6927$  & N/A    & $0.3791/1.0355$                     \\ \hline
        MLP   & \textit{Water}   & N/A   & $0.0007$   & $0.0022$  & $0.0002$    & $0.2567/1.9775$                    \\ \hline
        Logistic Regression   &
        \textit{Phomene}   & N/A   & $0.0097$   & $0.7979$  & N/A    & $0.0571/1.8516$                     \\ \hline
        MLP   & \textit{Phomene}   & N/A   & $0.0007$   & $4.24\cdot10^{-5}$  & $0.0015$    & $0.0516/2.2700$                    \\ \hline
       % MLP   & \textit{Adult}   & N/A   & $0.0018$   & $0.0018$  & $0.0601$     & $0.0764/2.2068$                    \\ \hline
        CNN   & \textit{CIFAR-10} & N/A   & $0.0050$   & $0.0864$ & N/A    & $0.3018/
        2.1502$                     \\ \hline
    \end{tabular}
\end{table*}

\begin{table*}[htb!]
    \centering
    \caption{Mean value and standard deviation of training time across 5 different runs. The reported time (in seconds) corresponds to the generation of each entry in Table \ref{tab:regularization_comparison}. Times are }
    \label{tab:times}
    \begin{tabular}{|c|c|c|c|c|c|c|}
        \hline
        \textbf{Model} & \textbf{Dataset} & \textbf{No-Reg} & \textbf{L1-Reg} & \textbf{L2-Reg} & \textbf{Dropout} & \textbf{CF-Reg (ours)} \\ \hline
        Logistic Regression   & \textit{Water}   & $222.98 \pm 1.07$   & $239.94 \pm 2.59$   & $241.60 \pm 1.88$  & N/A    & $251.50 \pm 1.93$                     \\ \hline
        MLP   & \textit{Water}   & $225.71 \pm 3.85$   & $250.13 \pm 4.44$   & $255.78 \pm 2.38$  & $237.83 \pm 3.45$    & $266.48 \pm 3.46$                    \\ \hline
        Logistic Regression   & \textit{Phomene}   & $266.39 \pm 0.82$ & $367.52 \pm 6.85$   & $361.69 \pm 4.04$  & N/A   & $310.48 \pm 0.76$                    \\ \hline
        MLP   &
        \textit{Phomene} & $335.62 \pm 1.77$   & $390.86 \pm 2.11$   & $393.96 \pm 1.95$ & $363.51 \pm 5.07$    & $403.14 \pm 1.92$                     \\ \hline
       % MLP   & \textit{Adult}   & N/A   & $0.0018$   & $0.0018$  & $0.0601$     & $0.0764/2.2068$                    \\ \hline
        CNN   & \textit{CIFAR-10} & $370.09 \pm 0.18$   & $395.71 \pm 0.55$   & $401.38 \pm 0.16$ & N/A    & $1287.8 \pm 0.26$                     \\ \hline
    \end{tabular}
\end{table*}

\subsection{Feasibility of our Method}
A crucial requirement for any regularization technique is that it should impose minimal impact on the overall training process.
In this respect, CF-Reg introduces an overhead that depends on the time required to find the optimal counterfactual example for each training instance. 
As such, the more sophisticated the counterfactual generator model probed during training the higher would be the time required. However, a more advanced counterfactual generator might provide a more effective regularization. We discuss this trade-off in more details in Section~\ref{sec:discussion}.

Table~\ref{tab:times} presents the average training time ($\pm$ standard deviation) for each model and dataset combination listed in Table~\ref{tab:regularization_comparison}.
We can observe that the higher accuracy achieved by CF-Reg using the score-based counterfactual generator comes with only minimal overhead. However, when applied to deep neural networks with many hidden layers, such as \textit{PreactResNet-18}, the forward derivative computation required for the linearization of the network introduces a more noticeable computational cost, explaining the longer training times in the table.

\subsection{Hyperparameter Sensitivity Analysis}
The proposed counterfactual regularization technique relies on two key hyperparameters: $\alpha$ and $\beta$. The former is intrinsic to the loss formulation defined in (\ref{eq:cf-train}), while the latter is closely tied to the choice of the score-based counterfactual explanation method used.

Figure~\ref{fig:test_alpha_beta} illustrates how the test accuracy of an MLP trained on the \textit{Water Potability} dataset changes for different combinations of $\alpha$ and $\beta$.

\begin{figure}[ht]
    \centering
    \includegraphics[width=0.85\linewidth]{img/test_acc_alpha_beta.png}
    \caption{The test accuracy of an MLP trained on the \textit{Water Potability} dataset, evaluated while varying the weight of our counterfactual regularizer ($\alpha$) for different values of $\beta$.}
    \label{fig:test_alpha_beta}
\end{figure}

We observe that, for a fixed $\beta$, increasing the weight of our counterfactual regularizer ($\alpha$) can slightly improve test accuracy until a sudden drop is noticed for $\alpha > 0.1$.
This behavior was expected, as the impact of our penalty, like any regularization term, can be disruptive if not properly controlled.

Moreover, this finding further demonstrates that our regularization method, CF-Reg, is inherently data-driven. Therefore, it requires specific fine-tuning based on the combination of the model and dataset at hand.
\section{Conclusion}
In this work, we propose a simple yet effective approach, called SMILE, for graph few-shot learning with fewer tasks. Specifically, we introduce a novel dual-level mixup strategy, including within-task and across-task mixup, for enriching the diversity of nodes within each task and the diversity of tasks. Also, we incorporate the degree-based prior information to learn expressive node embeddings. Theoretically, we prove that SMILE effectively enhances the model's generalization performance. Empirically, we conduct extensive experiments on multiple benchmarks and the results suggest that SMILE significantly outperforms other baselines, including both in-domain and cross-domain few-shot settings.
\newpage
\section*{Impact Statement}
\system advances cost-efficient AI by demonstrating how small on-device language models can collaborate with powerful cloud-hosted models to perform data-intensive reasoning. By reducing reliance on expensive remote inference, \system makes advanced AI more accessible and sustainable. This has broad societal implications, including lowering barriers to AI adoption and enhancing data privacy by keeping more computations local. However, careful consideration must be given to potential biases in small models and the security risks of local code execution. 
\bibliography{references}
\bibliographystyle{icml2025}
\subsection{Lloyd-Max Algorithm}
\label{subsec:Lloyd-Max}
For a given quantization bitwidth $B$ and an operand $\bm{X}$, the Lloyd-Max algorithm finds $2^B$ quantization levels $\{\hat{x}_i\}_{i=1}^{2^B}$ such that quantizing $\bm{X}$ by rounding each scalar in $\bm{X}$ to the nearest quantization level minimizes the quantization MSE. 

The algorithm starts with an initial guess of quantization levels and then iteratively computes quantization thresholds $\{\tau_i\}_{i=1}^{2^B-1}$ and updates quantization levels $\{\hat{x}_i\}_{i=1}^{2^B}$. Specifically, at iteration $n$, thresholds are set to the midpoints of the previous iteration's levels:
\begin{align*}
    \tau_i^{(n)}=\frac{\hat{x}_i^{(n-1)}+\hat{x}_{i+1}^{(n-1)}}2 \text{ for } i=1\ldots 2^B-1
\end{align*}
Subsequently, the quantization levels are re-computed as conditional means of the data regions defined by the new thresholds:
\begin{align*}
    \hat{x}_i^{(n)}=\mathbb{E}\left[ \bm{X} \big| \bm{X}\in [\tau_{i-1}^{(n)},\tau_i^{(n)}] \right] \text{ for } i=1\ldots 2^B
\end{align*}
where to satisfy boundary conditions we have $\tau_0=-\infty$ and $\tau_{2^B}=\infty$. The algorithm iterates the above steps until convergence.

Figure \ref{fig:lm_quant} compares the quantization levels of a $7$-bit floating point (E3M3) quantizer (left) to a $7$-bit Lloyd-Max quantizer (right) when quantizing a layer of weights from the GPT3-126M model at a per-tensor granularity. As shown, the Lloyd-Max quantizer achieves substantially lower quantization MSE. Further, Table \ref{tab:FP7_vs_LM7} shows the superior perplexity achieved by Lloyd-Max quantizers for bitwidths of $7$, $6$ and $5$. The difference between the quantizers is clear at 5 bits, where per-tensor FP quantization incurs a drastic and unacceptable increase in perplexity, while Lloyd-Max quantization incurs a much smaller increase. Nevertheless, we note that even the optimal Lloyd-Max quantizer incurs a notable ($\sim 1.5$) increase in perplexity due to the coarse granularity of quantization. 

\begin{figure}[h]
  \centering
  \includegraphics[width=0.7\linewidth]{sections/figures/LM7_FP7.pdf}
  \caption{\small Quantization levels and the corresponding quantization MSE of Floating Point (left) vs Lloyd-Max (right) Quantizers for a layer of weights in the GPT3-126M model.}
  \label{fig:lm_quant}
\end{figure}

\begin{table}[h]\scriptsize
\begin{center}
\caption{\label{tab:FP7_vs_LM7} \small Comparing perplexity (lower is better) achieved by floating point quantizers and Lloyd-Max quantizers on a GPT3-126M model for the Wikitext-103 dataset.}
\begin{tabular}{c|cc|c}
\hline
 \multirow{2}{*}{\textbf{Bitwidth}} & \multicolumn{2}{|c|}{\textbf{Floating-Point Quantizer}} & \textbf{Lloyd-Max Quantizer} \\
 & Best Format & Wikitext-103 Perplexity & Wikitext-103 Perplexity \\
\hline
7 & E3M3 & 18.32 & 18.27 \\
6 & E3M2 & 19.07 & 18.51 \\
5 & E4M0 & 43.89 & 19.71 \\
\hline
\end{tabular}
\end{center}
\end{table}

\subsection{Proof of Local Optimality of LO-BCQ}
\label{subsec:lobcq_opt_proof}
For a given block $\bm{b}_j$, the quantization MSE during LO-BCQ can be empirically evaluated as $\frac{1}{L_b}\lVert \bm{b}_j- \bm{\hat{b}}_j\rVert^2_2$ where $\bm{\hat{b}}_j$ is computed from equation (\ref{eq:clustered_quantization_definition}) as $C_{f(\bm{b}_j)}(\bm{b}_j)$. Further, for a given block cluster $\mathcal{B}_i$, we compute the quantization MSE as $\frac{1}{|\mathcal{B}_{i}|}\sum_{\bm{b} \in \mathcal{B}_{i}} \frac{1}{L_b}\lVert \bm{b}- C_i^{(n)}(\bm{b})\rVert^2_2$. Therefore, at the end of iteration $n$, we evaluate the overall quantization MSE $J^{(n)}$ for a given operand $\bm{X}$ composed of $N_c$ block clusters as:
\begin{align*}
    \label{eq:mse_iter_n}
    J^{(n)} = \frac{1}{N_c} \sum_{i=1}^{N_c} \frac{1}{|\mathcal{B}_{i}^{(n)}|}\sum_{\bm{v} \in \mathcal{B}_{i}^{(n)}} \frac{1}{L_b}\lVert \bm{b}- B_i^{(n)}(\bm{b})\rVert^2_2
\end{align*}

At the end of iteration $n$, the codebooks are updated from $\mathcal{C}^{(n-1)}$ to $\mathcal{C}^{(n)}$. However, the mapping of a given vector $\bm{b}_j$ to quantizers $\mathcal{C}^{(n)}$ remains as  $f^{(n)}(\bm{b}_j)$. At the next iteration, during the vector clustering step, $f^{(n+1)}(\bm{b}_j)$ finds new mapping of $\bm{b}_j$ to updated codebooks $\mathcal{C}^{(n)}$ such that the quantization MSE over the candidate codebooks is minimized. Therefore, we obtain the following result for $\bm{b}_j$:
\begin{align*}
\frac{1}{L_b}\lVert \bm{b}_j - C_{f^{(n+1)}(\bm{b}_j)}^{(n)}(\bm{b}_j)\rVert^2_2 \le \frac{1}{L_b}\lVert \bm{b}_j - C_{f^{(n)}(\bm{b}_j)}^{(n)}(\bm{b}_j)\rVert^2_2
\end{align*}

That is, quantizing $\bm{b}_j$ at the end of the block clustering step of iteration $n+1$ results in lower quantization MSE compared to quantizing at the end of iteration $n$. Since this is true for all $\bm{b} \in \bm{X}$, we assert the following:
\begin{equation}
\begin{split}
\label{eq:mse_ineq_1}
    \tilde{J}^{(n+1)} &= \frac{1}{N_c} \sum_{i=1}^{N_c} \frac{1}{|\mathcal{B}_{i}^{(n+1)}|}\sum_{\bm{b} \in \mathcal{B}_{i}^{(n+1)}} \frac{1}{L_b}\lVert \bm{b} - C_i^{(n)}(b)\rVert^2_2 \le J^{(n)}
\end{split}
\end{equation}
where $\tilde{J}^{(n+1)}$ is the the quantization MSE after the vector clustering step at iteration $n+1$.

Next, during the codebook update step (\ref{eq:quantizers_update}) at iteration $n+1$, the per-cluster codebooks $\mathcal{C}^{(n)}$ are updated to $\mathcal{C}^{(n+1)}$ by invoking the Lloyd-Max algorithm \citep{Lloyd}. We know that for any given value distribution, the Lloyd-Max algorithm minimizes the quantization MSE. Therefore, for a given vector cluster $\mathcal{B}_i$ we obtain the following result:

\begin{equation}
    \frac{1}{|\mathcal{B}_{i}^{(n+1)}|}\sum_{\bm{b} \in \mathcal{B}_{i}^{(n+1)}} \frac{1}{L_b}\lVert \bm{b}- C_i^{(n+1)}(\bm{b})\rVert^2_2 \le \frac{1}{|\mathcal{B}_{i}^{(n+1)}|}\sum_{\bm{b} \in \mathcal{B}_{i}^{(n+1)}} \frac{1}{L_b}\lVert \bm{b}- C_i^{(n)}(\bm{b})\rVert^2_2
\end{equation}

The above equation states that quantizing the given block cluster $\mathcal{B}_i$ after updating the associated codebook from $C_i^{(n)}$ to $C_i^{(n+1)}$ results in lower quantization MSE. Since this is true for all the block clusters, we derive the following result: 
\begin{equation}
\begin{split}
\label{eq:mse_ineq_2}
     J^{(n+1)} &= \frac{1}{N_c} \sum_{i=1}^{N_c} \frac{1}{|\mathcal{B}_{i}^{(n+1)}|}\sum_{\bm{b} \in \mathcal{B}_{i}^{(n+1)}} \frac{1}{L_b}\lVert \bm{b}- C_i^{(n+1)}(\bm{b})\rVert^2_2  \le \tilde{J}^{(n+1)}   
\end{split}
\end{equation}

Following (\ref{eq:mse_ineq_1}) and (\ref{eq:mse_ineq_2}), we find that the quantization MSE is non-increasing for each iteration, that is, $J^{(1)} \ge J^{(2)} \ge J^{(3)} \ge \ldots \ge J^{(M)}$ where $M$ is the maximum number of iterations. 
%Therefore, we can say that if the algorithm converges, then it must be that it has converged to a local minimum. 
\hfill $\blacksquare$


\begin{figure}
    \begin{center}
    \includegraphics[width=0.5\textwidth]{sections//figures/mse_vs_iter.pdf}
    \end{center}
    \caption{\small NMSE vs iterations during LO-BCQ compared to other block quantization proposals}
    \label{fig:nmse_vs_iter}
\end{figure}

Figure \ref{fig:nmse_vs_iter} shows the empirical convergence of LO-BCQ across several block lengths and number of codebooks. Also, the MSE achieved by LO-BCQ is compared to baselines such as MXFP and VSQ. As shown, LO-BCQ converges to a lower MSE than the baselines. Further, we achieve better convergence for larger number of codebooks ($N_c$) and for a smaller block length ($L_b$), both of which increase the bitwidth of BCQ (see Eq \ref{eq:bitwidth_bcq}).


\subsection{Additional Accuracy Results}
%Table \ref{tab:lobcq_config} lists the various LOBCQ configurations and their corresponding bitwidths.
\begin{table}
\setlength{\tabcolsep}{4.75pt}
\begin{center}
\caption{\label{tab:lobcq_config} Various LO-BCQ configurations and their bitwidths.}
\begin{tabular}{|c||c|c|c|c||c|c||c|} 
\hline
 & \multicolumn{4}{|c||}{$L_b=8$} & \multicolumn{2}{|c||}{$L_b=4$} & $L_b=2$ \\
 \hline
 \backslashbox{$L_A$\kern-1em}{\kern-1em$N_c$} & 2 & 4 & 8 & 16 & 2 & 4 & 2 \\
 \hline
 64 & 4.25 & 4.375 & 4.5 & 4.625 & 4.375 & 4.625 & 4.625\\
 \hline
 32 & 4.375 & 4.5 & 4.625& 4.75 & 4.5 & 4.75 & 4.75 \\
 \hline
 16 & 4.625 & 4.75& 4.875 & 5 & 4.75 & 5 & 5 \\
 \hline
\end{tabular}
\end{center}
\end{table}

%\subsection{Perplexity achieved by various LO-BCQ configurations on Wikitext-103 dataset}

\begin{table} \centering
\begin{tabular}{|c||c|c|c|c||c|c||c|} 
\hline
 $L_b \rightarrow$& \multicolumn{4}{c||}{8} & \multicolumn{2}{c||}{4} & 2\\
 \hline
 \backslashbox{$L_A$\kern-1em}{\kern-1em$N_c$} & 2 & 4 & 8 & 16 & 2 & 4 & 2  \\
 %$N_c \rightarrow$ & 2 & 4 & 8 & 16 & 2 & 4 & 2 \\
 \hline
 \hline
 \multicolumn{8}{c}{GPT3-1.3B (FP32 PPL = 9.98)} \\ 
 \hline
 \hline
 64 & 10.40 & 10.23 & 10.17 & 10.15 &  10.28 & 10.18 & 10.19 \\
 \hline
 32 & 10.25 & 10.20 & 10.15 & 10.12 &  10.23 & 10.17 & 10.17 \\
 \hline
 16 & 10.22 & 10.16 & 10.10 & 10.09 &  10.21 & 10.14 & 10.16 \\
 \hline
  \hline
 \multicolumn{8}{c}{GPT3-8B (FP32 PPL = 7.38)} \\ 
 \hline
 \hline
 64 & 7.61 & 7.52 & 7.48 &  7.47 &  7.55 &  7.49 & 7.50 \\
 \hline
 32 & 7.52 & 7.50 & 7.46 &  7.45 &  7.52 &  7.48 & 7.48  \\
 \hline
 16 & 7.51 & 7.48 & 7.44 &  7.44 &  7.51 &  7.49 & 7.47  \\
 \hline
\end{tabular}
\caption{\label{tab:ppl_gpt3_abalation} Wikitext-103 perplexity across GPT3-1.3B and 8B models.}
\end{table}

\begin{table} \centering
\begin{tabular}{|c||c|c|c|c||} 
\hline
 $L_b \rightarrow$& \multicolumn{4}{c||}{8}\\
 \hline
 \backslashbox{$L_A$\kern-1em}{\kern-1em$N_c$} & 2 & 4 & 8 & 16 \\
 %$N_c \rightarrow$ & 2 & 4 & 8 & 16 & 2 & 4 & 2 \\
 \hline
 \hline
 \multicolumn{5}{|c|}{Llama2-7B (FP32 PPL = 5.06)} \\ 
 \hline
 \hline
 64 & 5.31 & 5.26 & 5.19 & 5.18  \\
 \hline
 32 & 5.23 & 5.25 & 5.18 & 5.15  \\
 \hline
 16 & 5.23 & 5.19 & 5.16 & 5.14  \\
 \hline
 \multicolumn{5}{|c|}{Nemotron4-15B (FP32 PPL = 5.87)} \\ 
 \hline
 \hline
 64  & 6.3 & 6.20 & 6.13 & 6.08  \\
 \hline
 32  & 6.24 & 6.12 & 6.07 & 6.03  \\
 \hline
 16  & 6.12 & 6.14 & 6.04 & 6.02  \\
 \hline
 \multicolumn{5}{|c|}{Nemotron4-340B (FP32 PPL = 3.48)} \\ 
 \hline
 \hline
 64 & 3.67 & 3.62 & 3.60 & 3.59 \\
 \hline
 32 & 3.63 & 3.61 & 3.59 & 3.56 \\
 \hline
 16 & 3.61 & 3.58 & 3.57 & 3.55 \\
 \hline
\end{tabular}
\caption{\label{tab:ppl_llama7B_nemo15B} Wikitext-103 perplexity compared to FP32 baseline in Llama2-7B and Nemotron4-15B, 340B models}
\end{table}

%\subsection{Perplexity achieved by various LO-BCQ configurations on MMLU dataset}


\begin{table} \centering
\begin{tabular}{|c||c|c|c|c||c|c|c|c|} 
\hline
 $L_b \rightarrow$& \multicolumn{4}{c||}{8} & \multicolumn{4}{c||}{8}\\
 \hline
 \backslashbox{$L_A$\kern-1em}{\kern-1em$N_c$} & 2 & 4 & 8 & 16 & 2 & 4 & 8 & 16  \\
 %$N_c \rightarrow$ & 2 & 4 & 8 & 16 & 2 & 4 & 2 \\
 \hline
 \hline
 \multicolumn{5}{|c|}{Llama2-7B (FP32 Accuracy = 45.8\%)} & \multicolumn{4}{|c|}{Llama2-70B (FP32 Accuracy = 69.12\%)} \\ 
 \hline
 \hline
 64 & 43.9 & 43.4 & 43.9 & 44.9 & 68.07 & 68.27 & 68.17 & 68.75 \\
 \hline
 32 & 44.5 & 43.8 & 44.9 & 44.5 & 68.37 & 68.51 & 68.35 & 68.27  \\
 \hline
 16 & 43.9 & 42.7 & 44.9 & 45 & 68.12 & 68.77 & 68.31 & 68.59  \\
 \hline
 \hline
 \multicolumn{5}{|c|}{GPT3-22B (FP32 Accuracy = 38.75\%)} & \multicolumn{4}{|c|}{Nemotron4-15B (FP32 Accuracy = 64.3\%)} \\ 
 \hline
 \hline
 64 & 36.71 & 38.85 & 38.13 & 38.92 & 63.17 & 62.36 & 63.72 & 64.09 \\
 \hline
 32 & 37.95 & 38.69 & 39.45 & 38.34 & 64.05 & 62.30 & 63.8 & 64.33  \\
 \hline
 16 & 38.88 & 38.80 & 38.31 & 38.92 & 63.22 & 63.51 & 63.93 & 64.43  \\
 \hline
\end{tabular}
\caption{\label{tab:mmlu_abalation} Accuracy on MMLU dataset across GPT3-22B, Llama2-7B, 70B and Nemotron4-15B models.}
\end{table}


%\subsection{Perplexity achieved by various LO-BCQ configurations on LM evaluation harness}

\begin{table} \centering
\begin{tabular}{|c||c|c|c|c||c|c|c|c|} 
\hline
 $L_b \rightarrow$& \multicolumn{4}{c||}{8} & \multicolumn{4}{c||}{8}\\
 \hline
 \backslashbox{$L_A$\kern-1em}{\kern-1em$N_c$} & 2 & 4 & 8 & 16 & 2 & 4 & 8 & 16  \\
 %$N_c \rightarrow$ & 2 & 4 & 8 & 16 & 2 & 4 & 2 \\
 \hline
 \hline
 \multicolumn{5}{|c|}{Race (FP32 Accuracy = 37.51\%)} & \multicolumn{4}{|c|}{Boolq (FP32 Accuracy = 64.62\%)} \\ 
 \hline
 \hline
 64 & 36.94 & 37.13 & 36.27 & 37.13 & 63.73 & 62.26 & 63.49 & 63.36 \\
 \hline
 32 & 37.03 & 36.36 & 36.08 & 37.03 & 62.54 & 63.51 & 63.49 & 63.55  \\
 \hline
 16 & 37.03 & 37.03 & 36.46 & 37.03 & 61.1 & 63.79 & 63.58 & 63.33  \\
 \hline
 \hline
 \multicolumn{5}{|c|}{Winogrande (FP32 Accuracy = 58.01\%)} & \multicolumn{4}{|c|}{Piqa (FP32 Accuracy = 74.21\%)} \\ 
 \hline
 \hline
 64 & 58.17 & 57.22 & 57.85 & 58.33 & 73.01 & 73.07 & 73.07 & 72.80 \\
 \hline
 32 & 59.12 & 58.09 & 57.85 & 58.41 & 73.01 & 73.94 & 72.74 & 73.18  \\
 \hline
 16 & 57.93 & 58.88 & 57.93 & 58.56 & 73.94 & 72.80 & 73.01 & 73.94  \\
 \hline
\end{tabular}
\caption{\label{tab:mmlu_abalation} Accuracy on LM evaluation harness tasks on GPT3-1.3B model.}
\end{table}

\begin{table} \centering
\begin{tabular}{|c||c|c|c|c||c|c|c|c|} 
\hline
 $L_b \rightarrow$& \multicolumn{4}{c||}{8} & \multicolumn{4}{c||}{8}\\
 \hline
 \backslashbox{$L_A$\kern-1em}{\kern-1em$N_c$} & 2 & 4 & 8 & 16 & 2 & 4 & 8 & 16  \\
 %$N_c \rightarrow$ & 2 & 4 & 8 & 16 & 2 & 4 & 2 \\
 \hline
 \hline
 \multicolumn{5}{|c|}{Race (FP32 Accuracy = 41.34\%)} & \multicolumn{4}{|c|}{Boolq (FP32 Accuracy = 68.32\%)} \\ 
 \hline
 \hline
 64 & 40.48 & 40.10 & 39.43 & 39.90 & 69.20 & 68.41 & 69.45 & 68.56 \\
 \hline
 32 & 39.52 & 39.52 & 40.77 & 39.62 & 68.32 & 67.43 & 68.17 & 69.30  \\
 \hline
 16 & 39.81 & 39.71 & 39.90 & 40.38 & 68.10 & 66.33 & 69.51 & 69.42  \\
 \hline
 \hline
 \multicolumn{5}{|c|}{Winogrande (FP32 Accuracy = 67.88\%)} & \multicolumn{4}{|c|}{Piqa (FP32 Accuracy = 78.78\%)} \\ 
 \hline
 \hline
 64 & 66.85 & 66.61 & 67.72 & 67.88 & 77.31 & 77.42 & 77.75 & 77.64 \\
 \hline
 32 & 67.25 & 67.72 & 67.72 & 67.00 & 77.31 & 77.04 & 77.80 & 77.37  \\
 \hline
 16 & 68.11 & 68.90 & 67.88 & 67.48 & 77.37 & 78.13 & 78.13 & 77.69  \\
 \hline
\end{tabular}
\caption{\label{tab:mmlu_abalation} Accuracy on LM evaluation harness tasks on GPT3-8B model.}
\end{table}

\begin{table} \centering
\begin{tabular}{|c||c|c|c|c||c|c|c|c|} 
\hline
 $L_b \rightarrow$& \multicolumn{4}{c||}{8} & \multicolumn{4}{c||}{8}\\
 \hline
 \backslashbox{$L_A$\kern-1em}{\kern-1em$N_c$} & 2 & 4 & 8 & 16 & 2 & 4 & 8 & 16  \\
 %$N_c \rightarrow$ & 2 & 4 & 8 & 16 & 2 & 4 & 2 \\
 \hline
 \hline
 \multicolumn{5}{|c|}{Race (FP32 Accuracy = 40.67\%)} & \multicolumn{4}{|c|}{Boolq (FP32 Accuracy = 76.54\%)} \\ 
 \hline
 \hline
 64 & 40.48 & 40.10 & 39.43 & 39.90 & 75.41 & 75.11 & 77.09 & 75.66 \\
 \hline
 32 & 39.52 & 39.52 & 40.77 & 39.62 & 76.02 & 76.02 & 75.96 & 75.35  \\
 \hline
 16 & 39.81 & 39.71 & 39.90 & 40.38 & 75.05 & 73.82 & 75.72 & 76.09  \\
 \hline
 \hline
 \multicolumn{5}{|c|}{Winogrande (FP32 Accuracy = 70.64\%)} & \multicolumn{4}{|c|}{Piqa (FP32 Accuracy = 79.16\%)} \\ 
 \hline
 \hline
 64 & 69.14 & 70.17 & 70.17 & 70.56 & 78.24 & 79.00 & 78.62 & 78.73 \\
 \hline
 32 & 70.96 & 69.69 & 71.27 & 69.30 & 78.56 & 79.49 & 79.16 & 78.89  \\
 \hline
 16 & 71.03 & 69.53 & 69.69 & 70.40 & 78.13 & 79.16 & 79.00 & 79.00  \\
 \hline
\end{tabular}
\caption{\label{tab:mmlu_abalation} Accuracy on LM evaluation harness tasks on GPT3-22B model.}
\end{table}

\begin{table} \centering
\begin{tabular}{|c||c|c|c|c||c|c|c|c|} 
\hline
 $L_b \rightarrow$& \multicolumn{4}{c||}{8} & \multicolumn{4}{c||}{8}\\
 \hline
 \backslashbox{$L_A$\kern-1em}{\kern-1em$N_c$} & 2 & 4 & 8 & 16 & 2 & 4 & 8 & 16  \\
 %$N_c \rightarrow$ & 2 & 4 & 8 & 16 & 2 & 4 & 2 \\
 \hline
 \hline
 \multicolumn{5}{|c|}{Race (FP32 Accuracy = 44.4\%)} & \multicolumn{4}{|c|}{Boolq (FP32 Accuracy = 79.29\%)} \\ 
 \hline
 \hline
 64 & 42.49 & 42.51 & 42.58 & 43.45 & 77.58 & 77.37 & 77.43 & 78.1 \\
 \hline
 32 & 43.35 & 42.49 & 43.64 & 43.73 & 77.86 & 75.32 & 77.28 & 77.86  \\
 \hline
 16 & 44.21 & 44.21 & 43.64 & 42.97 & 78.65 & 77 & 76.94 & 77.98  \\
 \hline
 \hline
 \multicolumn{5}{|c|}{Winogrande (FP32 Accuracy = 69.38\%)} & \multicolumn{4}{|c|}{Piqa (FP32 Accuracy = 78.07\%)} \\ 
 \hline
 \hline
 64 & 68.9 & 68.43 & 69.77 & 68.19 & 77.09 & 76.82 & 77.09 & 77.86 \\
 \hline
 32 & 69.38 & 68.51 & 68.82 & 68.90 & 78.07 & 76.71 & 78.07 & 77.86  \\
 \hline
 16 & 69.53 & 67.09 & 69.38 & 68.90 & 77.37 & 77.8 & 77.91 & 77.69  \\
 \hline
\end{tabular}
\caption{\label{tab:mmlu_abalation} Accuracy on LM evaluation harness tasks on Llama2-7B model.}
\end{table}

\begin{table} \centering
\begin{tabular}{|c||c|c|c|c||c|c|c|c|} 
\hline
 $L_b \rightarrow$& \multicolumn{4}{c||}{8} & \multicolumn{4}{c||}{8}\\
 \hline
 \backslashbox{$L_A$\kern-1em}{\kern-1em$N_c$} & 2 & 4 & 8 & 16 & 2 & 4 & 8 & 16  \\
 %$N_c \rightarrow$ & 2 & 4 & 8 & 16 & 2 & 4 & 2 \\
 \hline
 \hline
 \multicolumn{5}{|c|}{Race (FP32 Accuracy = 48.8\%)} & \multicolumn{4}{|c|}{Boolq (FP32 Accuracy = 85.23\%)} \\ 
 \hline
 \hline
 64 & 49.00 & 49.00 & 49.28 & 48.71 & 82.82 & 84.28 & 84.03 & 84.25 \\
 \hline
 32 & 49.57 & 48.52 & 48.33 & 49.28 & 83.85 & 84.46 & 84.31 & 84.93  \\
 \hline
 16 & 49.85 & 49.09 & 49.28 & 48.99 & 85.11 & 84.46 & 84.61 & 83.94  \\
 \hline
 \hline
 \multicolumn{5}{|c|}{Winogrande (FP32 Accuracy = 79.95\%)} & \multicolumn{4}{|c|}{Piqa (FP32 Accuracy = 81.56\%)} \\ 
 \hline
 \hline
 64 & 78.77 & 78.45 & 78.37 & 79.16 & 81.45 & 80.69 & 81.45 & 81.5 \\
 \hline
 32 & 78.45 & 79.01 & 78.69 & 80.66 & 81.56 & 80.58 & 81.18 & 81.34  \\
 \hline
 16 & 79.95 & 79.56 & 79.79 & 79.72 & 81.28 & 81.66 & 81.28 & 80.96  \\
 \hline
\end{tabular}
\caption{\label{tab:mmlu_abalation} Accuracy on LM evaluation harness tasks on Llama2-70B model.}
\end{table}

%\section{MSE Studies}
%\textcolor{red}{TODO}


\subsection{Number Formats and Quantization Method}
\label{subsec:numFormats_quantMethod}
\subsubsection{Integer Format}
An $n$-bit signed integer (INT) is typically represented with a 2s-complement format \citep{yao2022zeroquant,xiao2023smoothquant,dai2021vsq}, where the most significant bit denotes the sign.

\subsubsection{Floating Point Format}
An $n$-bit signed floating point (FP) number $x$ comprises of a 1-bit sign ($x_{\mathrm{sign}}$), $B_m$-bit mantissa ($x_{\mathrm{mant}}$) and $B_e$-bit exponent ($x_{\mathrm{exp}}$) such that $B_m+B_e=n-1$. The associated constant exponent bias ($E_{\mathrm{bias}}$) is computed as $(2^{{B_e}-1}-1)$. We denote this format as $E_{B_e}M_{B_m}$.  

\subsubsection{Quantization Scheme}
\label{subsec:quant_method}
A quantization scheme dictates how a given unquantized tensor is converted to its quantized representation. We consider FP formats for the purpose of illustration. Given an unquantized tensor $\bm{X}$ and an FP format $E_{B_e}M_{B_m}$, we first, we compute the quantization scale factor $s_X$ that maps the maximum absolute value of $\bm{X}$ to the maximum quantization level of the $E_{B_e}M_{B_m}$ format as follows:
\begin{align}
\label{eq:sf}
    s_X = \frac{\mathrm{max}(|\bm{X}|)}{\mathrm{max}(E_{B_e}M_{B_m})}
\end{align}
In the above equation, $|\cdot|$ denotes the absolute value function.

Next, we scale $\bm{X}$ by $s_X$ and quantize it to $\hat{\bm{X}}$ by rounding it to the nearest quantization level of $E_{B_e}M_{B_m}$ as:

\begin{align}
\label{eq:tensor_quant}
    \hat{\bm{X}} = \text{round-to-nearest}\left(\frac{\bm{X}}{s_X}, E_{B_e}M_{B_m}\right)
\end{align}

We perform dynamic max-scaled quantization \citep{wu2020integer}, where the scale factor $s$ for activations is dynamically computed during runtime.

\subsection{Vector Scaled Quantization}
\begin{wrapfigure}{r}{0.35\linewidth}
  \centering
  \includegraphics[width=\linewidth]{sections/figures/vsquant.jpg}
  \caption{\small Vectorwise decomposition for per-vector scaled quantization (VSQ \citep{dai2021vsq}).}
  \label{fig:vsquant}
\end{wrapfigure}
During VSQ \citep{dai2021vsq}, the operand tensors are decomposed into 1D vectors in a hardware friendly manner as shown in Figure \ref{fig:vsquant}. Since the decomposed tensors are used as operands in matrix multiplications during inference, it is beneficial to perform this decomposition along the reduction dimension of the multiplication. The vectorwise quantization is performed similar to tensorwise quantization described in Equations \ref{eq:sf} and \ref{eq:tensor_quant}, where a scale factor $s_v$ is required for each vector $\bm{v}$ that maps the maximum absolute value of that vector to the maximum quantization level. While smaller vector lengths can lead to larger accuracy gains, the associated memory and computational overheads due to the per-vector scale factors increases. To alleviate these overheads, VSQ \citep{dai2021vsq} proposed a second level quantization of the per-vector scale factors to unsigned integers, while MX \citep{rouhani2023shared} quantizes them to integer powers of 2 (denoted as $2^{INT}$).

\subsubsection{MX Format}
The MX format proposed in \citep{rouhani2023microscaling} introduces the concept of sub-block shifting. For every two scalar elements of $b$-bits each, there is a shared exponent bit. The value of this exponent bit is determined through an empirical analysis that targets minimizing quantization MSE. We note that the FP format $E_{1}M_{b}$ is strictly better than MX from an accuracy perspective since it allocates a dedicated exponent bit to each scalar as opposed to sharing it across two scalars. Therefore, we conservatively bound the accuracy of a $b+2$-bit signed MX format with that of a $E_{1}M_{b}$ format in our comparisons. For instance, we use E1M2 format as a proxy for MX4.

\begin{figure}
    \centering
    \includegraphics[width=1\linewidth]{sections//figures/BlockFormats.pdf}
    \caption{\small Comparing LO-BCQ to MX format.}
    \label{fig:block_formats}
\end{figure}

Figure \ref{fig:block_formats} compares our $4$-bit LO-BCQ block format to MX \citep{rouhani2023microscaling}. As shown, both LO-BCQ and MX decompose a given operand tensor into block arrays and each block array into blocks. Similar to MX, we find that per-block quantization ($L_b < L_A$) leads to better accuracy due to increased flexibility. While MX achieves this through per-block $1$-bit micro-scales, we associate a dedicated codebook to each block through a per-block codebook selector. Further, MX quantizes the per-block array scale-factor to E8M0 format without per-tensor scaling. In contrast during LO-BCQ, we find that per-tensor scaling combined with quantization of per-block array scale-factor to E4M3 format results in superior inference accuracy across models. 

\end{document}


% This document was modified from the file originally made available by
% Pat Langley and Andrea Danyluk for ICML-2K. This version was created
% by Iain Murray in 2018, and modified by Alexandre Bouchard in
% 2019 and 2021 and by Csaba Szepesvari, Gang Niu and Sivan Sabato in 2022.
% Modified again in 2023 and 2024 by Sivan Sabato and Jonathan Scarlett.
% Previous contributors include Dan Roy, Lise Getoor and Tobias
% Scheffer, which was slightly modified from the 2010 version by
% Thorsten Joachims & Johannes Fuernkranz, slightly modified from the
% 2009 version by Kiri Wagstaff and Sam Roweis's 2008 version, which is
% slightly modified from Prasad Tadepalli's 2007 version which is a
% lightly changed version of the previous year's version by Andrew
% Moore, which was in turn edited from those of Kristian Kersting and
% Codrina Lauth. Alex Smola contributed to the algorithmic style files.
