\section{Related Work}
In this section we talk only about prior-art in the field  related specifically to adoption of LLM in University settings ( same for the history of  LLMS can be found in Appendix \ref{related}). Currently after the AI boom,  triggered and accelerated by introduction of ChatGPT ____,  several universities, government agencies and private institutions have established collaborations and consortia to reduce the barrier of entry to LLMs for learning groups. Arizona State University (ASU) recently partnered with OpenAI\footnote{\url{https://www.insidehighered.com/news/tech-innovation/artificial-intelligence/2024/05/21/unpacking-asus-openai-partnership-and}} to allow groups to apply for access to ChatGPT for research purposes. Purdue University's Anvil ____ provides API access to open-source LLMs through NSF's ACCESS. 


Unlike other platforms like Arizona State University’s partnership with OpenAI or Purdue’s Anvil, AI-VERDE offers several additional features, including automated user onboarding and API key management. While on-premises hosting in AI-VERDE mitigates privacy risks, it also provides proxy access to commercial models like OpenAI's GPT. In contrast, Anvil lacks extensive access management, whereas AI-VERDE enhances this by offering tools tailored for instructors to manage user access and allocate budgets to groups and classes.




%