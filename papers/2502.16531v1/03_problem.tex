% !TEX root = template.tex

\section{System Setup} \label{sec:problem}
Let a group of heterogeneous agents $\mathcal{N} = \{a_i \mid i=1,2,...,N\}$ operate within a partially known workspace. Each agent can execute primitive actions that may require assistance from others. The agents are connected via a shared network. Within this framework, agents can directly exchange messages with any other agent in the workspace. %Next, we define the agents' models and their assigned tasks.
\subsection{System description}
\subsubsection{Motion Transition System} \label{subsec:motion-ts}
Agent $a_i$'s motion within the workspace is modeled as an FTS. Our approach focuses on a set of ROIs while a low-level controller handles obstacle avoidance and inter-region movement. This approach significantly reduces computational complexity compared to a fully partitioned workspace but sacrifices the ability to track, at any time, the agent's exact location within the FTS. Each agent $a_i$ is aware only of the set of $M^{a_i}$ ROIs, denoted by $\Pi^{a_i}_{\mathcal{M}}=\{\pi^{a_i}_1,\pi^{a_i}_2,...,\pi^{a_i}_{M^{a_i}}\}$. The FTS assigned agent $a_i$ is:
\begin{equation} \label{eq:motion-fts} \mathcal{T}^{a_i}_{\mathcal{M}}\triangleq\left(\Pi^{a_i}_{\mathcal{M}}, \Pi^{a_i}_{\mathcal{M},0}, \Psi^{a_i}_{\mathcal{M}}, \Sigma^{a_i}_{\mathcal{M}}, \longrightarrow^{a_i}_{\mathcal{M}}, \mathrm{L}^{a_i}_{\mathcal{M}}, \mathrm{T}^{a_i}_{\mathcal{M}}\right),  
\end{equation}
where $\Pi^{a_i}_{\mathcal{M},0}\in\Pi^{a_i}_{\mathcal{M}}$ is the initial ROI, $\Psi^{a_i}_{\mathcal{M}}$ is the set of atomic propositions describing the properties of the workspace, $\Sigma^{a_i}_{\mathcal{M}}$ is the set of movement actions, $\longrightarrow^{a_i}_{\mathcal{M}}\subseteq \Pi^{a_i}_{\mathcal{M}}\times\Sigma^{a_i}_{\mathcal{M}}\times\Pi^{a_i}_{\mathcal{M}}$ is the transition relation, $\mathrm{L}^{a_i}_{\mathcal{M}}:\Pi^{a_i}\rightarrow2^{\Psi^{a_i}_{\mathcal{M}}}$ is the labeling function, indicating the properties held by each ROI, and $\mathrm{T} ^{a_i}_{\mathcal{M}}:\longrightarrow^{a_i}_{\mathcal{M}}\rightarrow\mathbb{R}^+$ is the transition time function, representing the estimated time necessary for each transition.

\subsubsection{Action Model}\label{subsec:action-model}
In addition to its movement actions, agent $a_i$ can perform actions  $\Sigma^{a_i}_{\mathscr{A}} \triangleq \Sigma^{a_i}_l \cup \Sigma^{a_i}_c \cup \Sigma^{a_i}_h$, where $\Sigma^{a_i}_l$ are \textit{local} actions performed independently, $\Sigma^{a_i}_c$ are \textit{collaborative} actions requiring assistance from other agents, and $\Sigma^{a_i}_h$ are \textit{assisting} actions carried out to help others. Lastly, $\sigma_0 = \mathit{None}\in \Sigma^{a_i}_l$ indicates that $a_i$ remains idle. The action model for agent $a_i$ is defined as the tuple:
\begin{equation}\label{eq:action-model}
\mathscr{A}^{a_i} \triangleq \left(\Sigma^{a_i}_{\mathscr{A}}, \Psi^{a_i}_{\mathscr{A}}, \mathrm{L}^{a_i}_{\mathscr{A}}, \mathrm{Cond}^{a_i}, \mathrm{Dura}^{a_i}, \mathrm{Depd}^{a_i}\right),
\end{equation}
where $\Psi^{a_i}_{\mathscr{A}}$ is the set of atomic propositions, $\mathrm{L}^{a_i}_{\mathscr{A}}$ is the labeling function as in \cite{meng_paper}, $\mathrm{Cond}^{a_i}$ is the region properties required to execute an action, $\mathrm{Dura}^{a_i}$ is the action duration, with $\mathrm{Dura}^{a_i}(\sigma_s) = T_s > 0$, and 
$\mathrm{Depd}^{a_i}: \Sigma^{a_i}_{\mathscr{A}} \rightarrow 2^{\Sigma^{\sim a_i}_h} \times 2^{\Pi^{\mathcal{N}}}$ denotes the dependence function, where  $\Sigma^{\sim a_i}_h$ is the set of \textit{external} assisting actions that agent $a_i$ depends on, and $\Pi^{\mathcal{N}}=\cup_{a_i\in\mathcal{N}}\Pi^{a_i}_{\mathcal{M}}$. Given $\sigma_c\in \Sigma^{a_i}_c $, we define the set of actions involved in a collaboration as:
\begin{equation}\label{eq:collaboration}
    \mathcal{C}(\sigma_c)=\{\sigma_c\}\cup\mathrm{Depd}^{a_i}(\sigma_c).
\end{equation}
\begin{definition}\label{def:succesful-collab}
    A collaboration is considered successful if all actions involved are synchronized; i.e. to complete $\sigma_c \in \Sigma_c^{a_i}$, it is necessary that all actions in $\mathcal{C}(\sigma_c)$ start simultaneously.
\end{definition}
\subsubsection{Agent Transition System} \label{subsec:agent-ts}

The planner in Sec. \ref{subsec:planning} requires to define agent $a_i$'s FTS by combining \eqref{eq:motion-fts} and \eqref{eq:action-model}.
\begin{definition}
 Given $\mathcal{T}^{a_i}_{\mathcal{M}}$ and $\mathscr{A}^{a_i}$, a valid FTS for agent $a_i$, according to \cite{model-checking}, can be constructed as follows:
    \begin{equation}\label{eq:agent-model}    \mathcal{T}^{a_i}_{\mathcal{G}}\triangleq\left(\Pi^{a_i}_{\mathcal{G}}, \Pi^{a_i}_{\mathcal{G},0}, \Psi^{a_i}_{\mathcal{G}}, \Sigma^{a_i}_{\mathcal{G}}, \longrightarrow^{a_i}_{\mathcal{G}}, \mathrm{L}^{a_i}_{\mathcal{G}}, \mathrm{T}^{a_i}_{\mathcal{G}}\right), 
    \end{equation}
where $\Pi^{a_i}_{\mathcal{G}} = \Pi^{a_i}_{\mathcal{M}}\times \Sigma^{a_i}_{\mathscr{A}}$ is the set states,
$\Pi^{a_i}_{\mathcal{G},0}=\langle\Pi^{a_i}_{\mathcal{M},0} , \mathit{None}\rangle$ is the initial state,
$\Psi^{a_i}_{\mathcal{G}}$ is the set of atomic propositions,
$\Sigma^{a_i}_{\mathcal{G}}=\Sigma^{a_i}_{\mathcal{M}}\bigcup\Sigma^{a_i}_{\mathscr{A}}$, with $\Sigma^{a_i}_{\mathcal{G}, l}=\Sigma^{a_i}_{\mathcal{M}}\bigcup\Sigma^{a_i}_l$, 
$\longrightarrow^{a_i}_{\mathcal{G}}$ is the transition relation,
$\mathrm{L}^{a_i}_{\mathcal{G}}$ is the labeling function, and 
$\mathrm{T}^{a_i}_{\mathcal{G}}$ is the transition estimated duration \cite{meng_paper}.
\end{definition}
As in \cite{meng_paper},  the path is denoted by  $\tau^{a_{i}}=\pi^{a_i}_{\mathcal{G}, 0} \pi^{a_i}_{\mathcal{G}, 1} \ldots$ its trace by $\mathit{trace}(\tau^{a_{i}})=L_{\mathcal{G}}^{a_{i}}(\pi^{a_i}_{\mathcal{G}, 0}) L_{\mathcal{G}}^{a_{i}}(\pi^{a_i}_{\mathcal{G}, 1}) \ldots$ and, the associated sequence of actions by $\rho^{a_i}=\sigma^{a_i}_0, \sigma^{a_i}_1,\ldots$, i.e., the actions that allow transition between the states of $\tau^{a_{i}}$. 

\subsubsection{Task Specification}\label{subsec:task}
%In \cite{meng_paper} sc-LTL was considered but o
Our focus is on implementing recurring tasks i.e., tasks that repeat infinitely often. We will consider the following syntax $\varphi' ::=\top \mid a \mid \neg a\mid \varphi'_1\wedge\varphi'_2 \mid \lozenge\varphi'$, and for agent $a_i$ we define the recurring task as
\begin{equation}\label{eq:recurringLTL}
\varphi^{a_i}_r=\varphi'_1\wedge\square\lozenge\varphi'_2.
\end{equation}
Note that $\varphi'_2$ cannot start with $\lozenge$ to guarantee the validity of the LTL formula.
Given any satisfying word of $\varphi^{a_i}_r$, inserting a detour i.e., a finite sequence of states, between two consecutive states results in a satisfying word.

%\subsection{Problem Statement}
%We can summarize the problem as follows:
\begin{problem}\label{problem:task}
Given $\mathcal{T}_{\mathcal{G}}^{a_i}$ and the locally assigned task $\varphi^{a_i}_r$, design a distributed coordination and synchronization scheme such that $\varphi^{a_i}_r$ is satisfied for all $a_i \in \mathcal{N}$.%, and 2)   %The algorithm must also compensate for delays induced by the experimental scenario and 
 %all joint actions involved in a collaboration in  \eqref{eq:collaboration} start simultaneously.
\end{problem}
%\begin{remark}
   %Synchronizing actions that require precise timing, such as loading boxes, is critical for successful collaboration. Otherwise, timing discrepancies could lead to failure. 
%\end{remark}
