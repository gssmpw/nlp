%%%%%%%%%%%%%%%%%%%%%%%%%%%%%%%%%%%%%%%%%%%%%%%%%%%%%%%%%%%%%%%%%%%%%%%%%%%%%%%%
%2345678901234567890123456789012345678901234567890123456789012345678901234567890
%        1         2         3         4         5         6         7         8
% \documentclass[journal,twoside,web]{ieeecolor}% \documentclass[letterpaper, 10 pt, conference]{ieeeconf}  % Comment this line out if you need a4paper

% \documentclass[a4paper, 10pt, conference]{ieeeconf}      % Use this line for a4 paper
% \documentclass[letterpaper, 10 pt journal, twoside]{IEEEtran} 
\documentclass[letterpaper, 10 pt, conference]{ieeeconf}  
\IEEEoverridecommandlockouts             
\overrideIEEEmargins

% \pdfobjcompresslevel=0                  % This command is only needed if 
                                                          % you want to use the \thanks command
                                                          
\makeatletter
\let\NAT@parse\undefined
\makeatother
\usepackage{hyperref}
\usepackage{xurl}
\usepackage{algorithm}
\usepackage{algorithmic}
\usepackage[linesnumbered,ruled,vlined,algo2e]{algorithm2e}
%\usepackage{algorithmicx}
%\usepackage{algpseudocode}
\usepackage{amsfonts}
\usepackage{amsmath}
\usepackage{amssymb}
\usepackage[ansinew]{inputenc} 
\usepackage{xcolor}
\usepackage{mathtools}
\usepackage{graphicx}
\usepackage{caption}
\usepackage{subcaption}
\usepackage{import}
\usepackage{multirow}
\usepackage{cite}
\usepackage[export]{adjustbox}
\usepackage{breqn}
\usepackage{mathrsfs}
\usepackage{acronym}
%\usepackage[keeplastbox]{flushend}
\usepackage{setspace}
\usepackage{bm}
\usepackage{stackengine}
\usepackage{stackengine}
\usepackage{needspace}
\usepackage{comment}
\usepackage{siunitx}
\usepackage{lipsum}
\let\proof\relax 
\let\endproof\relax
\usepackage{amsthm}
\usepackage{amssymb}
\usepackage{svg}
\usepackage{lipsum}
%\usepackage{MnSymbol}
\usepackage{wasysym}
\usepackage{mathrsfs}
%\usepackage{caption}
\usepackage[T1]{fontenc}


\theoremstyle{plain}
\newtheorem{theorem}{Theorem}
\newtheorem{lemma}{Lemma}


\theoremstyle{definition}
\newtheorem{assumption}{Assumption}
\newtheorem{definition}{Definition}
\newtheorem{problem}{Problem}
\newtheorem{procedure2}{Procedure}

\theoremstyle{remark}
\newtheorem{remark}{Remark}

%\newlist{steps}{enumerate}{1}
%\setlist[steps, 1]{label = \textbf{\textit{Step}} \arabic*}


\pdfminorversion=4 
%\overrideIEEEmargins                                      % Needed to meet printer requirements.

% See the \addtolength command later in the file to balance the column lengths
% on the last page of the document

% The following packages can be found on http:\\www.ctan.org
%\usepackage{graphics} % for pdf, bitmapped graphics files
%\usepackage{epsfig} % for postscript graphics files
%\usepackage{mathptmx} % assumes new font selection scheme installed
%\usepackage{times} % assumes new font selection scheme installed
%\usepackage{amsmath} % assumes amsmath package installed
%\usepackage{amssymb}  % assumes amsmath package installed

\title{ \LARGE \bf
  Efficient Coordination and Synchronization of Multi-Robot Systems Under Recurring Linear Temporal Logic
}


\author{Davide Peron$^1$, Victor Nan Fernandez-Ayala$^2$, Eleftherios E. Vlahakis$^2$, and Dimos V. Dimarogonas$^2$
\thanks{This work was supported by the ERC CoG LEAFHOUND, the EU
CANOPIES project, the Knut and Alice Wallenberg Foundation (KAW) and the Digital Futures Smart Construction project.}
\thanks{$^1$ Department of Information Engineering, University of Padova, 35122, Padova, Italy. Email: {\tt\small davide.peron.3@studenti.unipd.it} 
$^2$ Division of Decision
and Control Systems, School of Electrical Engineering and Computer
Science, KTH Royal Institute of Technology, 10044, Stockholm, Sweden. Emails: \tt\small\{vnfa,vlahakis,dimos\}@kth.se}
}

\begin{document}
\newcommand{\until}[0]{\mathsf{U}}
%\renewcommand{\triangleq}{\overset{\Delta}{=}}


\def\triangleq{\mathrel{\ensurestackMath{\stackon[1pt]{=}{\scriptstyle\Delta}}}}


\maketitle
\thispagestyle{empty}
\pagestyle{empty}


%%%%%%%%%%%%%%%%%%%%%%%%%%%%%%%%%%%%%%%%%%%%%%%%%%%%%%%%%%%%%%%%%%%%%%%%%%%%%%%%
End-to-end imitation learning offers a promising approach for training robot policies. However, generalizing to new settings—such as unseen scenes, tasks, and object instances—remains a significant challenge. Although large-scale robot demonstration datasets have shown potential for inducing generalization, they are resource-intensive to scale. In contrast, human video data is abundant and diverse, presenting an attractive alternative. Yet, these human-video datasets lack action labels, complicating their use in imitation learning. Existing methods attempt to extract grounded action representations (e.g., hand poses), but resulting policies struggle to bridge the embodiment gap between human and robot actions.
% our approach
We propose an alternative approach: leveraging language-based reasoning from human videos - essential for guiding robot actions - to train generalizable robot policies. Building on recent advances in reasoning-based policy architectures, we introduce Reasoning through Action-free Data (RAD). RAD learns from both robot demonstration data (with reasoning and action labels) and action-free human video data (with only reasoning labels). The robot data teaches the model to map reasoning to low-level actions, while the action-free data enhances reasoning capabilities. Additionally, we will release a new dataset of 3,377 human-hand demonstrations compatible with the Bridge V2 benchmark. This dataset includes chain-of-thought reasoning annotations and hand-tracking data to help facilitate future work on reasoning-driven robot learning.
% experiments
Our experiments demonstrate that RAD enables effective transfer across the embodiment gap, allowing robots to perform tasks seen only in action-free data. Furthermore, scaling up action-free reasoning data significantly improves policy performance and generalization to novel tasks. These results highlight the promise of reasoning-driven learning from action-free datasets for advancing generalizable robot control. 
% releasing dataset
Website: \href{https://rad-generalization.github.io}{here}.


\section{Introduction}
\label{sec:intro}


\ps{Challenges of technology scaling}

The growing demand for computing performance has always been met by increasing the number of transistors per chip, which is only possible due to CMOS technology scaling.
However, as we keep pushing the boundaries of technology scaling, we encounter multiple challenges.
Firstly, whenever we transition to a more advanced technology node, the non-recurring cost due to physical design, verification, software, mask sets, and prototyping almost doubles \cite{cost-tech-node}.
As a result, designing a chip in an advanced technology node is only economically viable if the chip is manufactured in vast quantities.
Secondly, many chip components such as I/O drivers, analog circuits, or \gls{srams} have reached their scaling limit.
This means that we cannot shrink these components further, even if we use a more advanced technology with a smaller feature size.
Thirdly, advanced technology nodes suffer from high defect rates, diminishing the yield and inflating the recurring cost.
To tackle these challenges, new chip-design paradigms have been developed.

\ps{Why 2.5D integration?}

One of these new paradigms is 2.5D integration, where multiple silicon dies called chiplets are integrated into the same package.
Once designed, a single chiplet can be reused in multiple 2.5D stacked chips, which increases the ratio of production volume to non-recurring cost.
Another advantage is that multiple chiplets - fabricated in different technologies - can be integrated into the same package.
This means that only components that can take full advantage of technology scaling are built in bleeding-edge technologies.
Components that have reached their scaling limit are fabricated in more mature and hence less costly technology nodes.
Furthermore, chiplets are smaller than monolithic chips.
Therefore, manufacturing chiplets results in less silicon area loss due to fabrication defects and hence a higher yield.
Due to these economic advantages, chip vendors such as AMD \cite{amd-chiplet} and NVIDIA \cite{chiplet-book} have adopted the 2.5D integration paradigm.  

\ps{Challenges of 2.5D integration}

An important challenge when designing 2.5D stacked chips is the construction of a low-latency and high-throughput \gls{ici}. 
To build an \gls{ici}, we connect different chiplets using \gls{d2d} links.
These links are fabricated in an organic package substrate, silicon bridge, or silicon interposer, and they are connected to the chiplets using \gls{c4} bumps or microbumps.
The number of bumps per chiplet is limited, and so is the bandwidth of \gls{d2d} links.
In addition to having lower bandwidth than links in monolithic chips, \gls{d2d} links also have higher latency.
This latency is caused by wire delay and by \gls{phys} that are necessary in both the sending and the receiving chiplet.
\gls{phys} are needed to convert between protocols, voltage levels, and frequencies, which are usually different between on-chiplet links and \gls{d2d} links.
Due to these limitations, the \gls{ici} can quickly become a bottleneck.

\ps{How we solve these challenges differently than the related work does.}

Existing approaches to maximize the performance of the \gls{ici} either optimize the placement of chiplets (with potentially heterogeneous shapes) for a predetermined \gls{ici} topology 
\cite{ho,liu,seemuth,eris,osmolovskyi,tap25d,chiou}, select one topology out of a set of candidates \cite{coskun-1, coskun-2}, or they optimize the \gls{ici} topology for a 2D grid of homogeneously shaped chiplets on an active interposer \cite{butterdonut, cluscross, kite}.
To the best of our knowledge, there is no prior work on \gls{ici} topologies for chips with heterogeneously shaped chiplets or with passive silicon interposers or silicon bridges.
To fill this gap, we propose \name, a novel optimization methodology to jointly optimize the chiplet placement and \gls{ici} topology of such architectures.
\ifnb
\else
\newpage
\fi

\ps{Details on \name~and the key idea}

The key idea is as follows: 
We optimize the chiplet placement without a predetermined topology.
For each placement generated by an optimization algorithm, we infer a placement-based \gls{ici} topology by connecting chiplets that are in close proximity in that specific placement.
We then compute the latency and throughput of this combination of placement and topology for different traffic types.
These latencies and throughputs together with the total chip area are used to compute a user-defined quality-score of the placement, which is returned to the optimization algorithm.
Based on this quality score, the algorithm can further optimize the placement.
By following this iterative process, we jointly optimize the chiplet placement and the \gls{ici} topology.

\ps{Short evaluation-summary}

We provide our open-source framework implementing the proposed placement and topology co-optimization methodology, which we evaluate using both synthetic traffic and traffic traces.
A 2D grid of chiplets with a mesh topology is used as a baseline since many proposals for 2.5D stacked chips \cite{dataflow_accel_dnn, cifher, simba, hecaton, dojo} use such an architecture.
We reduce the latency of synthetic L1-to-L2 and L2-to-memory traffic, the two most important traffic types for cache coherency traffic, by up to 28\% and 62\% respectively.
For real traffic traces, we reduce the average packet latency for almost all traces and architectures considered (reduced by an 8\% or 18\% on average depending on the configuration of \gls{phys} within a chiplet).


\section{Preliminaries}\label{sec:problem_formulation}

% \begin{table*}[h!]
% \centering
% \caption{Comparison of Algorithms}
% \label{tab:algorithm_comparison}
% \begin{tabular}{l|ccccc}
% \toprule
% \textbf{Algorithm Name} & \textbf{Performance} & \textbf{Diversity} & \textbf{Generalization} & \textbf{Efficiency} & \textbf{Streaming}\\
% \midrule
% Greedy-decoding & Moderate  &   Low  & High          & High      & Yes \\
% Decoding with Temperature & Low & High & High              & High & Yes   \\
% Top-K Sampling & Moderate      & High & High               & High   & Yes     \\
% Top-P Sampling & Moderate     & High & High              & High   & Yes      \\
% Beam-Search & Moderate     & Moderate & High              & Moderate    & Yes     \\
% Majority Voting & Moderate     & High & Low              & Moderate     & No    \\
% RM Selection & Moderate     & High & High              & Moderate    & No     \\
% RM-guided Tree Search & High     & High & Low              & Low    & No    \\
% \bottomrule
% \end{tabular}
% \end{table*}
% The language model generation process generally selects a sequence of tokens following certain algorithms (e.g., greedy or sampling methods) until a stopping criterion, such as an end-of-sequence token or maximum sequence length, is reached.

In this section, we first introduce how the LM generation process can be formulated as a token-level Markov Decision Process (MDP) and then explain how existing sampling algorithms relate to it.

\subsection{LLM Decoding as Token-level MDP}

The language model generation process takes a sequence of tokens as inputs and generates a sequence of tokens as outputs.
Mainstream transformer-based language models generate the output tokens one by one until the stopping criteria (e.g., an end-of-sequence token or maximum sequence length) are met.
The traditional MDP is usually formulated as a tuple $\mathcal{M} = (\mathcal{S}, \mathcal{A}, F, R, \gamma)$, where $\mathcal{S}$ is the set of all possible states, $\mathcal{A}$ is the set of actions, $F$ is the transition function, $R$ is the reward function, and $\gamma$ is the decay parameter.
In the language model scenarios, each state in $\mathcal{S}$ is a trajectory that can be denoted as $\tau$.
Each action in $\mathcal{A}$ is selecting a token $x$ from the vocabulary set.
$F$ is the deterministic transition of concatenating the selected action (i.e., a token) with the existing state (i.e., a trajectory) to become a new one.
Traditionally, rewards $r_t$ are defined at every step $t$ and contribute to the return $G_t=\sum_{k=0}^{\infty} \gamma^k R_{t+k}$ through the decay factor $\gamma$.
However, in the LLM scenario, we are only concerned with the quality of the complete trajectory generated, meaning that the reward function $R(s)$ evaluates the final trajectory rather than providing step-by-step feedback.
Thus, the return becomes $G=R_T$, where $R_T$ is the reward associated with the final sequence at step $T$, and $\gamma$ is irrelevant because intermediate rewards are not accumulated. 

\subsection{The Classical Decoding Algorithms}

Formally, given an input \( \mathbf{x}  = (x_1, x_2, \ldots, x_T) \), a reward function $R$ that provides a scalar reward for a trajectory $\tau$, and a language model \( p_\theta(x) \) parameterized by \( \theta \), the goal of decoding algorithms is to find the optimal trajectory $x^\star$ sampled from \( p_\theta(x) \) that could maximize the reward:

\[
\tau^* = \arg\max_{\tau \sim p_\theta(\mathbf{x} )} R(\tau).
\]

% Assumes that the input is a token sequence with $a$ and the foundation language model is a probabilistic model of predicting the likelihood of the next token given previous ones, which is usually formulated as $P(x \mid x_{<i})$, the goal of language model decoding is to utilize this likelihood to get the output trajectory of length $T$ that achieves the highest $R_T$.
This section covers representative decoding algorithms and explains how they are connected.

\textbf{Greedy Decoding}: The naive but most widely used algorithm is \textit{Greedy Decoding}, which uses the language modeling likelihood at each step as guidance.
At each step $i$, this algorithm selects the action token $x_i \in \mathcal{A}$ following:
\begin{equation}
x_i = \arg\max_{x} P(x \mid x_{<i}).
\end{equation}
From the angle of MDP, this method uses the accumulative likelihood predicted by the language as the final reward:
\begin{equation}
 R(\tau) \gets \Pi_{i}^{T}P(x_i \mid x_{<i}),
\end{equation}
where $T$ is the length of $\tau$, and takes a greedy solution to approach this goal.


\textbf{Sampling-based Decoding:} 
On top of greedy decoding, people also try to incorporate diversity in the final output.
For example, the temperature-based method introduces an additional parameter $\lambda$ to control the greedy sampling process by reshaping the likelihood distribution as:
\begin{equation}
x_i \sim P(x \mid x_{<i})^{1/\lambda}.
\end{equation}
From the angle of token-level MDP, we can reinterpret this process as introducing an additional diversity objective:

\begin{equation}
 R(\tau) \gets \Pi_{i}^{T}P(x_i \mid x_{<i}) \cdot D(x_i, x_{<i}),
\end{equation}
where 
\begin{equation}
    D(x_i , x_{<i}) = P(x_i \mid x_{<i})^{\frac{1}{\lambda}-1}.
\end{equation}


To avoid sampling rare tokens and achieve a balance between performance and diversity, researchers have investigated how to dynamically adjust the candidate token pool~\citep{holtzman2019curious,zarriess2021decoding}. 
For example, the \textit{Top-$k$ Sampling} algorithm only considers the top $k$ tokens with the highest probabilities as candidates instead of the whole vocabulary.
Similarly, the \textit{Nucleus Sampling}, which is also known as \textit{Top-p Sampling}, only selects from the smallest possible set $\mathcal{V}_p \subseteq \mathcal{V}$, where the cumulative probability mass exceeds a threshold $p$.

% \paragraph{Nucleus Sampling:} This strategy samples tokens from the smallest possible set $\mathcal{V}_p \subseteq \mathcal{V}$, where the cumulative probability mass exceeds a threshold $p$. Formally,
% \begin{equation}
%     \mathcal{V}_p = \{x_i \mid \sum_{x_j \in \mathcal{V}_p} p(x_j \mid \mathbf{x}_{<t}) \geq p\}.
% \end{equation}

\textbf{Trajectory-level Decoding:} Although these token-level decoding algorithms are efficient, they tend to generate locally coherent outputs that may lack global quality.
To solve this problem, people also developed decoding algorithms that consider partial or whole trajectories.
For example, the \textit{Beam Search Decoding} algorithm keeps track of the top $B$ partial trajectories, expanding them at each step and retaining only the ones with the highest joint likelihood.
Similar to the \textit{Greedy Decoding}, this method also uses the joint likelihood as the trajectory reward function.

\textbf{Advanced Reward Modeling Algorithms:}
A common limitation of the aforementioned algorithms is their fundamental assumption that the joint likelihood could represent $R(\tau)$ might not always hold.
People have been interested in introducing better reward signals as guidance to address this.
For example, in the QA scenario, the \textit{Majority-voting algorithm} assumes that the more frequent answer aligns better with the grounding reward function (i.e., accuracy) and thus selects candidate trajectories following this guidance.
Though this intuitive approach has been shown to be effective on tasks such as QA and math problems, it is restricted to tasks with structured output for voting. It cannot be generalized to more general-purpose applications.
To address this issue, researchers also include an external model $R^\prime$, which is often another transformer-based model, to approximate the ground truth reward model $R$. 
With that, we could sample $K$ trajectories $\mathcal{T}_K$ with sampling-based decoding algorithms and then use $R$ to select the trajectory with the maximum reward:
\begin{equation}
    \tau^\star = \arg \max_{\tau \in \mathcal{T}} R^\prime(\tau).
\end{equation}
Employing an external model to model the reward offers greater flexibility than heuristic rewards. This approach is not constrained by the structured answer format, which improves generality and adaptability in various scenarios. 









\section{Problem Formulation}

\noindent \textbf{Problem.} Given an arbitrary textual description $t$ as the input condition, e.g. "a small bedroom for a young student", our goal is to generate plausible, realistic, and physically feasible scenes corresponding to the requirement. The generated scene includes the selected 3D models and the arranged placements of the objects therein, i.e. $S=\{(M_i, P_i)|i=1,...,n\}$, where $M_i$ and $P_i$ denote the selected 3D model and spatial placement respectively for the $i$th object. The placement of each object contains its center position coordinates, orientation angle, and size.

\noindent \textbf{Hierarchical Scene Representation.} 
We propose to use the hierarchical scene representation throughout our pipeline. It is a three-level hierarchical structure, as shown in Figure~\ref{fig:hierarchy}. The first level is the root node representing the entire scene, the second level is the internal nodes each representing a rectangular functional area, and the third level is the leaf nodes representing the objects belonging to the corresponding area. The nodes are connected with two types of edges,
i.e. the parent-child relation indicating the hierarchical structure and the pairwise relation between objects to represent
their spatial relationship. Specifically, to reduce the redundancy, we set one anchor object for each functional area and only allow the pairwise relations between the anchor object and other objects belonging to the same functional area.

Each node in the hierarchy contains some attributes. Assuming axis-aligned rectangular floorplans for the scenes, the root node $r$ has a size attribute $s_r$ and a text description $t_r$ of the scene, i.e. $r = \{t_r, s_r\}$. Each internal node $a$, which represents an axis-aligned functional area, carries the text description $t_a$, the size attributes $s_a$, as well as a center position $p_a$ and an orientation $\theta_a$, i.e. $a = \{t_a, s_a, p_a, \theta_a\}$. The position is a 2D coordinate while the orientation is a binary value representing either horizontal or vertical direction. Each object node $o$ contain the text description $t_o$, the category label $c_o$, the corresponding 3D model $M_o$, as well as the size $s_o$, center position $p_o$, orientation $\theta_o$ of the oriented bounding box of the object, i.e. $o = \{t_o, c_o, M_o, s_o, p_o, \theta_o\}$. Note that the size attributes are 2D vectors for room and functional areas, but 3D vectors for objects, since we also care about their heights. In addition, the pairwise spatial relationship $e$ stores the coarse text description $t_e$ such as "in front of" and the fine-grained relative placement coordinates including position $p_e$ and orientation $\theta_e$ of one object w.r.t. the anchor object, i.e. $e = \{t_e, p_e, \theta_e\}$.

\section{Results}\label{sect:results}

\subsection{Established Benchmarks}\label{sect:results_west}
We begin by evaluating all vision-language models on established benchmarks, based on ImageNet and COCO Captions, among other datasets. As revealed in Table~\ref{tab:west_standard_setup}, increasing the dataset size from 10 billion to 100 billion examples does not improve performance substantially. This is statistically supported by Wilcoxon's signed rank test~\cite{wilcoxon1992individual}, which gives a $p$-value of 0.9, indicating that differences are not significant.


In addition, we also fit data scaling laws for every combination of model and dataset following the recipe proposed in~\citet{alabdulmohsin2022revisiting}. This allows us to evaluate whether or not the performance gap is expected to increase or decrease in the infinite-compute regime. We report the resulting scaling exponents and asymptotic performance limits in the tables. Again, we do not observe  significant differences at the 95\% confidence level ($p$-value of 0.09).


\subsection{Cultural Diversity}
Unlike the Western-oriented metrics reported in Section~\ref{sect:results_west}, cultural diversity metrics present an entirely different picture. We observe \emph{notable} gains when scaling the size of the dataset from 10 billion to 100 billion examples in Table~\ref{tab:culture_standard_setup}. 
For example, scaling training data from 10 billion to 100 billion examples yields substantial gains on Dollar Street 10-shot classification task, where ViT-L and ViT-H see absolute improvements of 5.8\% and 5.4\%, respectively. These gains outperform the typical improvements (less than 1\%) observed on Western-oriented 10-shot metrics by a large margin.
Using Wilcoxon's signed rank test, we obtain a $p$-value of 0.002, indicating a statistically significant evidence at the 99\% confidence level.


\subsection{Multilinguality}

Our multilingual benchmark, Crossmodal-3600 zero-shot retrieval~\cite{thapliyal2022crossmodal}, shows a disparity in performance gains: low-resource languages benefit more from the 100 billion scale than the high-resource ones. The disparity, illustrated in Figure~\ref{fig:multilinguality}, which not only exists in all model sizes but also widens as the models become larger. Detailed results for each language can be found in Appendix~\ref{appendix:data_scale}.

% source: https://colab.corp.google.com/drive/1AKgGDITZqTC2hQjVc-Iv8xuysh5giP0i#scrollTo=2EtEXMbly8dB&line=1&uniqifier=1
\begin{figure}[h!]
    % \includegraphics[width=\linewidth]{figures/multilang-Average_Multilingual__Low-Resource_Lang.pdf}
    % \includegraphics[width=0.86\linewidth]{figures/multilang-Average_Multilingual__High-Resource_Lang.pdf}
    \includegraphics[width=\linewidth]{figures/multilang-Average_XM3600_Retrieval.pdf}
    \caption{Scaling up to 100B examples leads to more notable improvements in low-resource languages. $\Delta$ denotes the improved accuracy when scaling from 10B examples to 100B.}
    \label{fig:multilinguality}
\end{figure}


\subsection{Fairness}
For fairness, we report on 3 metrics discussed in Section~\ref{sect:evals}. 

\paragraph{Representation Bias.} The first metric is representation bias (RB), with results detailed in Table~\ref{tab:rb}. We observe that models trained on unbalanced web data have a significantly higher preference to associate a randomly chosen image from ImageNet~\cite{deng2009imagenet} with the label ``Male'' over the label ``Female.'' 

In fact, this occurs nearly 85\% of the time. Training on 100B examples does not mitigate this effect. This finding aligns with previous research highlighting the necessity of bias mitigation strategies, such as data balancing~\cite{alabdulmohsin2024clip}, to address inherent biases in web-scale datasets.

% \begin{table}[h]
%     \centering\scriptsize
%     \caption{representation bias with respect to gender using imagenet. Here, a value of 0.8, for example, indicates that the model would prefer to associate a randomly chosen image from ImageNet with the label ``Male'' over the label ``Female''.}
%     \label{tab:rb}
%     \begin{tabularx}{\columnwidth}{@{}c|YYY@{}}
%     \toprule
%     \bf Model&\bf1B &\bf10B &\bf100B\\
%     \midrule
% B & 83.2&84.5&85.2
% \\
% L & 88.2&86.4&85.5\\
% H & 86.8&85.0&86.6\\
% \bottomrule
%     \end{tabularx}
% \end{table}

\begin{table}[h]
\begin{tabularx}{\columnwidth}{c|YYY@{}}
    \toprule
    \bf Model&\bf1B &\bf10B &\bf100B\\
    \midrule
B & 83.2&84.5&85.2\\
L & 88.2&86.4&85.5\\
H & 86.8&85.0&86.6\\
\bottomrule
\end{tabularx}
\captionof{table}{Representation bias w.r.t. gender (see Section~\ref{sect:results}). Here,  values [\%] indicate how often the model prefers to associate a random  image with the label ``Male'' over ``Female''.} \label{tab:rb}
\end{table}



\paragraph{Association Bias.} Second, Figure~\ref{fig:ab} shows the association bias in SigLIP-H/14 between gender and occupation as we scale the data from 10 to 100 billion examples. Specifically, we plot the probability that the model would prefer a particular occupation label, such as ``{\fontfamily{lmodern}\selectfont secretary}'' over another label, such as ``{\fontfamily{lmodern}\selectfont manager}'' when images correspond to males or females. In this evaluation, we use the Fairface~\cite{karkkainen2021fairface} dataset. The labels we compare are: ``{\fontfamily{lmodern}\selectfont librarian}'' vs. ``{\fontfamily{lmodern}\selectfont scientist}'', ``{\fontfamily{lmodern}\selectfont nurse}'' vs. ``{\fontfamily{lmodern}\selectfont doctor}'', ``{\fontfamily{lmodern}\selectfont housekeeper}'' vs. ``{\fontfamily{lmodern}\selectfont homeowner}'', ``{\fontfamily{lmodern}\selectfont receptionist}'' vs. ``{\fontfamily{lmodern}\selectfont executive}'' and ``{\fontfamily{lmodern}\selectfont secretary}'' vs. ``{\fontfamily{lmodern}\selectfont manager}''. Again, we do not see a reduction in association bias by simply increasing the size of the training data. %Full results are in Appendix~\ref{appendix:ab}.

%Additionally, we are unable to evaluate cultural diversity and fairness in PaliGemma's transfer tasks due to the lack of appropriate benchmarks. This is an open question that we hope to address in the future.

\paragraph{Performance Disparity.} Finally, one common definition of fairness in machine learning is maintaining similar performance across different groups. See, for instance,~\citet{dehghani2023scaling} and the related notions of ``Equality of Opportunity'' and ``Equalized Odds''~\cite{hardt2016equalityopportunitysupervisedlearning}. Table~\ref{tab:perf_disparity} show that scaling the data to 100 billion examples improves performance disparity, which is consistent with the improvement in cultural diversity.

%  to show on top of page
% \begin{table}[h]
    \centering\scriptsize
    \begin{tabularx}{\columnwidth}{l|YYY@{}}
    \toprule
    \bf Model & \bf 1B & \bf 10B & \bf 100B\\ \midrule
    &\multicolumn{3}{c}{\em 0-shot Dollar Street}\\[2pt]
B & 32.5 & 29.9 & \bf29.0\\
L & \bf29.7 & 29.8 & 30.4 \\
H & 32.2 & 33.0 & \bf32.1\\
\midrule 
    &\multicolumn{3}{c}{\em 0-shot GeoDE}\\[2pt]
B & 4.7 & 5.5 & \bf4.4\\
L & 3.2 & 4.0 & \bf2.8 \\
H & 3.6 & 3.0 & \bf2.7\\
\bottomrule
 
    \end{tabularx}
    \caption{Performance disparity (lower is better) for models pretrained on 100B seen examples of different data scales. Pretraining on 100B examples tends to lower disparity.}
    \label{tab:per_disp_mini}
\end{table}
% \FloatBarrier

\begin{table*}[h]
    \centering\scriptsize
    \caption{Performance disparity results for various SigLIP models pretrained on 100 billion seen examples of 1B, 10B, and 100B datasets. Here, disparity corresponds to the maximum gap across subgroups in Dollar Street (by income level) and GeoDE (by geographic region). Pretraining on 100B examples tends to improve disparity overall.}
    \label{tab:perf_disparity}
    \begin{tabularx}{2\columnwidth}{ll|YYYYYY|Y}
    \toprule
    \bf Model & \bf Data Scale &\multicolumn{6}{c}{\bf Performance per Subgroup} & \bf Disparity\\ \midrule
    \multicolumn{8}{c}{\em 0-shot Dollar Street}\\[2pt]
    & & \bf 0-200	& \bf 200-685	& \bf 685-1998	& \bf $>$1998
    & & & \\ \midrule
B&1B&29.4&43.9&56.5&62.0&&&32.5\\
B&10B&31.6&44.0&55.4&61.5&&&29.9\\
B&100B&32.0&44.3&56.3&61.0&&&\bf29.0\\[3pt]
L&1B&33.7&44.7&57.3&63.4&&&\bf29.7\\
L&10B&35.7&47.8&58.7&65.5&&&29.8\\
L&100B&33.7&46.6&59.5&64.1&&&30.4\\[3pt]
H&1B&32.3&44.9&58.4&64.5&&&32.2\\
H&10B&33.9&46.3&58.6&66.9&&&33.0\\
H&100B&34.1&48.2&62.2&66.1&&&\bf32.1\\ \midrule

    \multicolumn{8}{c}{\em 0-shot GeoDE}\\[2pt]
    & & \bf Africa	& \bf Americas	& \bf East-Asia	& \bf Europe & \bf South-East Asia & \bf West Asia
    & \\ \midrule
B&1B&89.4&92.1&91.8&94.1&92.5&93.4&4.7\\
B&10B&88.4&91.8&91.4&94.0&92.2&93.0&5.5\\
B&100B&88.8&91.4&91.0&93.3&91.7&92.2&\bf4.4\\[3pt]
L&1B&92.0&94.0&94.0&95.2&94.2&94.9&3.2\\
L&10B&91.8&94.4&94.0&95.8&94.2&94.7&4.0\\
L&100B&93.5&95.1&95.4&96.2&95.0&95.8&\bf2.8\\[3pt]
H&1B&91.5&94.4&94.7&95.2&94.1&94.5&3.6\\
H&10B&93.4&95.4&95.0&96.5&95.1&95.6&3.0\\
H&100B&93.6&95.1&95.3&96.3&95.2&95.8&\bf2.7\\

 \bottomrule
    \end{tabularx}
\end{table*}


\subsection{Transfer To Generative Models}
\label{sec:transfer}

\begin{table}[h!]
\centering
\footnotesize

% Note: We removed 1b result to avoid confusion to readers. See https://docs.google.com/document/d/1YxRpUO7elSaviOQ5XIXtnUWajgYAF7FfRO0vmXQCUtU/edit?resourcekey=0-pPjeeIrYEXRuvnuBFZn5Uw&tab=t.0#heading=h.gaczi2wqv0go.
\begin{tabular}{p{0pt}l|rrrrr}
\toprule
& Data & Semantics & OCR & Multiling & RS & Avg \\
\midrule
% % source: https://docs.google.com/spreadsheets/d/1W5_VNitkO6k-HSKBV6sGX81EGXgma877m0gGsg8zPzw/edit?resourcekey=0-2zH_U5z5kL9I5Nnhg7SChQ&gid=1972601917#gid=1972601917
\includegraphics[width=8pt]{images/snowflake_2744-fe0f.png} & 1B & 76.0 & 66.8 & 67.0 & 92.3 & 73.6 \\
\includegraphics[width=8pt]{images/snowflake_2744-fe0f.png} & 10B & 75.4 & 65.2 & 66.3 & 91.9 & 72.7 \\
\includegraphics[width=8pt]{images/snowflake_2744-fe0f.png} & 100B & 76.4 & 67.0 & 66.9 & 92.1 & 73.9 \\
\includegraphics[width=8pt]{images/fire_1f525.png} & 1B & 77.1 & 69.5 & 66.9 & 92.0 & 75.1 \\
\includegraphics[width=8pt]{images/fire_1f525.png} & 10B & 76.4 & 66.9 & 66.0 & 91.8 & 73.7 \\
\includegraphics[width=8pt]{images/fire_1f525.png} & 100B & {77.2} & {70.0} & {67.0} & {91.8} & {75.3} \\
\bottomrule
\end{tabular}

\caption{
The PaliGemma transfer results of ViT-L/16 models pretrained on 10B and 100B examples, with both frozen (top) %({\includegraphics[width=8pt]{images/snowflake_2744-fe0f.png}}) 
and unfrozen (bottom) %({\includegraphics[width=8pt]{images/fire_1f525.png}}) 
vision components. Results are aggregated.
}
\label{tab:transfer_avg}
\end{table}


We use PaliGemma~\citep{beyer2024paligemma} with both frozen and unfrozen vision component to assess the transferability of our vision models, which were contrastively pre-trained on datasets of different scales. In Table~\ref{tab:transfer_avg}, when taking the noise level into consideration, we do not observe consistent performance gains across downstream tasks as we scale the pre-training dataset. More details can be found in Appendix~\ref{appendix:transfer}.


%\paragraph{Recognizing Sensitive Attributes.}
%Finally, we also report the performance of the models in recognizing sensitive attributes, following a similar evaluation in~\citet{radford2021learning}. We report the accuracy in predicting perceived gender in Fairface~\cite{karkkainen2021fairface} and predicting perceived race in UTK~\cite{utkface_url}. Overall, we observe that scaling the data to 100 billion examples improves this aspect of fairness as well. Table~\ref{tab:fairness_pred} provides the full results. We do not observe a particular pattern in this type of evaluation.
%\begin{table}[t]
    \centering\scriptsize
    \begin{tabularx}{\columnwidth}{@{}ll|YY@{}}
    \toprule
    \bf Model & \bf Data Scale & \bf Gender & \bf Race\\ \midrule
B	&	1B	& 91.0	& 58.6\\
B	&	10B	& 91.7 &	\bf59.5\\
B	&	100B &	\bf91.9	& 53.1\\[3pt]
L	&	1B	& 0.94.8	& \bf55.4\\
L	&	10B	& 93.9	& 53.1\\
L	&	100B &	\bf95.0	& 54.0\\[3pt]
H	&	1B	& 94.5	& \bf54.5\\
H	&	10B	& \bf95.4	& 54.3\\
H	&	100B	& \bf95.4	& 50.2 \\
 \bottomrule
    \end{tabularx}
    \caption{Accuracy in recognizing sensitive attributes using Fairface and UTK datasets. See Section~\ref{sect:results} for details.}
    \label{tab:fairness_pred}
\end{table}


% !TEX root = template.tex

\section{Experimental Results}\label{sec:experimental}
We conducted experiments, within the framework of the CANOPIES project \cite{canopies}, to show the advantages of our approach, most of which are based on the following system.
\subsection{System description}\label{subsec:exp-system}
We consider a workspace measuring $4.2\si{m} \times 5.2\si{m}$ (Fig. \ref{fig:workspace}), and two types of agents: Robotis Turtlebot3 Burger \cite{turtlebot} and Hebi Rosie with robotic arm \cite{rosie}.
\begin{figure}
    \centering
    % First column with two images stacked, labeled (a)
    \begin{minipage}{0.5\linewidth}
        \centering
        \vspace{0.14cm}
        \includegraphics[width=0.4\linewidth]{images/turtlebot.jpg}\\
        \vspace{0.05cm}
        \includegraphics[width=0.4\linewidth]{images/rosie.jpg}
         \subcaption{T. Turtlebot, B. Rosie}\label{fig:robots}
    \end{minipage}%
    % Second column with one image, labeled (b)
    \begin{minipage}{0.5\linewidth}
        \centering
        \includegraphics[width=\textwidth]{images/arena_final_border_v2.png}
         \subcaption{Workspace abstraction}\label{fig:workspace}
    \end{minipage}
    \caption{Experimental setup}
\end{figure}
\subsubsection{Turtlebot}
It knows the ROIs: $C1-4$, $P1-12$, $M$, and $G$. The collaborative actions are  \textit{check\_connection (cc)} at $C1$ or $C2$, \textit{group (g)} at $G$, \textit{remove\_object (ro)} at $M$, the assistive actions are \textit{help\_check\_connection (hcc)} at $C3$ or $C4$, \textit{help\_group (hg)} at $G$  and the local actions are \textit{patrol (p)} at $P1-12$, the movement related actions and, \textit{None}.
\begin{comment}
   The non-movement actions are listed in Tab. \ref{tab:act_turtlebot}.
\begin{table}[ht]
    \centering
    \begin{tabular}{c||c|c}
        \textbf{Action} &  \textbf{Type} & \textbf{Cond}\\
        \hline\hline
        \textit{check\_connection (cc)} &  collaborative & $C1$, $C2$\\
        \hline
        \textit{group (g)} &  collaborative & $G$\\
        \hline
        \textit{remove\_object (ro)} &  collaborative & $M$\\
        \hline
        \textit{help\_check\_connection (hcc)} &  assistive & $C3$, $C4$ \\
        \hline
        \textit{help\_group (hg)} &  assistive & $G$\\
        \hline
        \textit{patrol (p)} &  local & $P1-14$\\
        \hline
        \textit{None} &  local &  \\
        \hline\hline
    \end{tabular}
    \caption{Action model for Turtlebots}
    \label{tab:act_turtlebot}
\end{table} 
\end{comment}
\subsubsection{Rosie}
It knows the ROIs: $H1-4$, $D$, $L$, $M$, and $S$. The collaborative action is \textit{load (l)} at $L$, the assistive actions are \textit{help\_load (hl)} at $L$, \textit{help\_remove\_object (hro)} at $M$ and, the local actions are \textit{harvest (h)} at $H1-4$, \textit{manipulate (m)} at $M$, \textit{deliver (d)} at $D$, \textit{supervise (s)} at $S$, the movement related actions and, \textit{None}.
\begin{comment}
   \begin{table}[ht!]
    \centering
    \begin{tabular}{c||c|c}
        \textbf{Action} &  \textbf{Type} & \textbf{Cond}\\
        \hline\hline
        \textit{load (l)} &  collaborative & $L$\\
        \hline
        \textit{help\_load (hl)} &  assistive & $L$\\
        \hline
        \textit{help\_remove\_object (hro)} &  assistive & $M$\\
        \hline
        \textit{harverst (h)} &  local & $H1-4$\\
        \hline
        \textit{manipulate (m)} &  local & $M$\\
        \hline
        \textit{deliver (d)} &  local & $D$\\
        \hline
        \textit{supervise (s)} &  local & $S$\\
        \hline
        \textit{None} &  local &  \\
        \hline\hline
    \end{tabular}
    \caption{Action model for the Rosies}
    \label{tab:act_rosie}
\end{table} 
\end{comment}
\subsubsection{Task specification}\label{subsec:exp-task}
We consider a basic team composed of 3 Rosies and 6 Turtlebots. The recurring LTL tasks \eqref{eq:recurringLTL} assigned to the agents are as follows:     $\varphi^{\mathrm{rosie}_0}_r=\square\lozenge(h \wedge H1 \wedge \lozenge(h \wedge H3\wedge\lozenge d))$,    $\varphi^{\mathrm{rosie}_1}_r=\square\lozenge(s \wedge \lozenge(l \wedge\lozenge m))$,         $\varphi^{\mathrm{rosie}_2}_r=\square\lozenge(h \wedge H2 \wedge \lozenge(h \wedge H4\wedge\lozenge d))$,             $\varphi^{\mathrm{turtlebot}_0}_r=\square\lozenge(p \wedge P1 \wedge \lozenge(p\wedge P11))$,             $\varphi^{\mathrm{turtlebot}_1}_r=\square\lozenge(p \wedge P2 \wedge \lozenge(p\wedge P10))$,             $\varphi^{\mathrm{turtlebot}_2}_r=\square\lozenge(p \wedge P4 \wedge \lozenge(p \wedge P6))$,             $\varphi^{\mathrm{turtlebot}_3}_r=\square\lozenge(p \wedge P12 \wedge \lozenge(p\wedge P9\wedge \lozenge(cc \wedge C1))$,      $\varphi^{\mathrm{turtlebot}_4}_r=\square\lozenge(p \wedge P3 \wedge \lozenge(p\wedge P8\wedge \lozenge g))$ and, $\varphi^{\mathrm{turtlebot}_5}_r=\lozenge (ro \wedge \lozenge(p \wedge P5 \wedge \lozenge(cc \wedge C2))) \wedge\square\lozenge(p \wedge P5 \wedge \lozenge(p\wedge P7))$. The starting ROIs are respectively $H1$, $M$, $H2$, $P1$, $P2$, $P4$, $P12$, $P3$, $M$.
Lastly, we define \ChooseROI, which is relevant only when $check\_connection$ must be completed. If $cc$ is completed in $C1$, then $hcc$ is completed in $C3$. If $cc$ is completed in $C2$, then $hcc$ is completed in $C4$. 
\subsection{Complexity Reduction}\label{subsec:exp-complexity}
We assess the computational gains discussed in Sec. \ref{subsec:res-complexity}.
\subsubsection{Comparison between ROI and Grid Representation}
We developed an ROI representation that focuses only on the necessary regions for the agent unlike previous approaches \cite{meng_paper} which used a grid structure to partition the workspace. 
\begin{table}[b]
    \centering
    \begin{tabular}{c||c|c|c}
        \textbf{Agent} &  \textbf{States in $\mathcal{T}_{\mathcal{M}}$} &\textbf{States Grid} & \textbf{Reduction}\\
        \hline\hline
        Rosie &  $8$ & $42$ & $81.0\%$\\
        \hline
        Turtlebot&  $18$ & $500$ & $96.4\%$\\
        \hline\hline
    \end{tabular}
    \caption{Computational gains of $\mathcal{T}_{\mathcal{M}}$ over a grid structure}
    \label{tab:motion-state-reduction}
\end{table}
As shown in Tab. \ref{tab:motion-state-reduction}, the grid was partitioned with cells sized to contain the specific robot ($6\times7$ grid for Rosie and $20\times25$ for Turtlebots). 
The ROI representation led to a significant reduction in the number of states, achieving an average reduction of $88.7\%$ compared to the grid representation. This reduction influences the size of the FTS $\mathcal{T}_{\mathcal{G}}$ and the PBA $\mathcal{A_P}$. In the sequel, we focus on the ROI representation.

\subsubsection{Gains of using the FTS}
In Alg. \ref{alg:reply}, we use the FTS $\mathcal{T}_{\mathcal{G}}$, whereas \cite{meng_paper} uses the PBA $\mathcal{A_P}$. This results in an average reduction of $84.8\%$ in the number of states across all agents. The PBA's state count increases with task complexity, hence, $turtlebot_5$, with the most complex task, benefits the most from using $\mathcal{T}_{\mathcal{G}}$ in \Dijkstra. The results are in Tab. \ref{tab:state-reduction}.
\begin{table}[t]
    \centering
    \begin{tabular}{c||c|c|c}
        \textbf{Agent} &  \textbf{States $\mathcal{T}_{\mathcal{G}}$} &\textbf{States $\mathcal{A_P}$} & \textbf{Reduction}\\
        \hline\hline
        $\mathrm{rosie}_{0,1,2}$ &  $18$ & $162$ & $88.9\%$\\
        \hline
        $\mathrm{turtlebot}_{0,1,2}$ &  $37$ & $148$ & $75.0\%$\\
        \hline
        $\mathrm{turtlebot}_{3,4}$ &  $37$ & $333$ & $88.9\%$\\
        \hline
        $\mathrm{turtlebot}_5$ &  $37$ & $592$ & $93.8\%$\\
        \hline\hline
    \end{tabular}
    \caption{Computational gains by FTS against PBA}
    \label{tab:state-reduction}
\end{table}
\subsubsection{MIP Filtering}
Lastly, we analyzed the effect of the filtering procedure on the RRC cycle and on the confirmation step. This was tested in a centralized simulation, varying the number of agents and actions in a request. CycloneDDS \cite{cyclone} was used as the ROS2 middleware for stable communication between nodes.
As shown in Fig. \ref{fig:complexity} (bottom plot), the filtering procedure significantly reduced the confirmation time, especially as the number of agents increased, while the time remained relatively constant for different numbers of actions. The greatest reduction occurs with a single requested action, where the MIP is solved by Proc. \ref{proc:filtering}, with these results being barely visible in the plot due to their low values.
However, this reduction in confirmation time has a limited impact on the overall RRC cycle, as shown in Fig. \ref{fig:complexity} (upper plot), where the bottleneck is the ROS2 communication. Significant improvements only appear with 350 agents, the maximum allowed by the workstation. This suggests that with a larger number of agents ($N \gg M$), the filtering procedure yields substantial gains.
%\vspace{-0.5cm}
\begin{figure}[t]
    \centering    \includegraphics[width=\linewidth]{images/filtering_plot2.png}
    \caption{Effects of agents filtering}
    \label{fig:complexity}
\end{figure}

\subsection{Experimental Results}
\label{subsec:exp-results}
In this setup, we deployed our approach to the available hardware, in a decentralized way. For robot movement, we developed a model predictive controller \cite{mpc} with control barrier function \cite{cbf_1, cbf_2} constraints for collision avoidance. Agent poses were tracked using the Qualisys Motion Capture System \cite{mocap}. For the `$remove\_object$` action, we used a visual servoing controller \cite{visualservo1,visualservo2}, while all other actions were simulated. A video of the experiment is available at \cite{video}.
Fig. \ref{fig:experimental} shows the sequence of actions completed by each agent, demonstrating that all tasks were completed and that the approach synchronized collaborative and assistive actions to start simultaneously, even if some agents were ready earlier. Note that only non-movement actions are shown.
\begin{figure}[h]
    \centering    
    \includegraphics[width=\linewidth]{images/experiment_plot3.pdf}
    \caption{Agents actions in the experimental settings}
    \label{fig:experimental}
\end{figure}
\subsection{Scalability}
\label{subsec:exp-scalability}
To demonstrate the scalability of our approach, we created a simulation with 90 agents, representing 10 of the teams described in Sec. \ref{subsec:exp-system}, due to the unavailability of such a large number of robots. Fig. \ref{fig:scalability} shows the action sequence of a subset of these agents for clarity. The results indicate that the agents successfully completed their assigned tasks, collaborations, and synchronized actions. The successful completion of collaborations by the agents demonstrates the approach's effectiveness in performing well with ninety agents. Experiments with more agents could not be conducted due to RAM constraints. The results indicate that our approach scales well and can perform effectively with larger teams.
\begin{figure}[ht]
    \centering
    \includegraphics[width=\linewidth]{images/90_agents3.pdf}
    \caption{90 agents actions in the scalability simulation}
    \label{fig:scalability}
\end{figure}
% \vspace{-0.12cm}


% !TEX root = template.tex

\section{Conclusion}
\label{sec:conclusion}
This work focuses on MAS coordination and synchronization under recurring LTL. We extended the bottom-up scheme for distributed motion and task coordination of MAS in \cite{meng_paper}, reducing computational complexity to enhance scalability and enable deployment on robotic hardware. The package was developed in ROS2, with a synchronization mechanism to handle action delays in experiments. Future work will focus on developing additional actions and incorporating human-in-the-loop scenarios.


\newpage
 

\bibliography{biblio}
\bibliographystyle{IEEEtran}



\end{document}











