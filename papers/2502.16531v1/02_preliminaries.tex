% !TEX root = template.tex

\section{Preliminaries} \label{sec:preliminaries}

\subsection{LTL and B\"{u}chi Automaton} \label{subsec:LTL}
Atomic propositions are Boolean variables that can either be true or false. An LTL formula is defined over a set of atomic propositions ($\Psi$), the Boolean connectors: negation ($\neg$), conjunction ($\wedge$), and the temporal operators, \textit{next} ($\Circle$) and \textit{until} ($\until$). It is specified according to the following syntax \cite{model-checking}:
$ \varphi::= \top\mid a\mid\neg\varphi\mid\varphi_1\wedge\varphi_2\mid\Circle\varphi\mid\varphi_1\until\varphi_2$ where $a\in\Psi$, and $\top\triangleq True$. Operators \textit{always} ($\square$), \textit{eventually} ($\lozenge$) can be derived from the syntax above \cite[Ch. 5]{model-checking}. The satisfaction of an LTL formula $\varphi$ is achieved over words and the language of such words can be captured through a nondeterministic B\"{u}chi automaton (NBA) \cite{buchi_book}, defined as $\mathcal{B}=\left(S, S_0,  2^{\Psi}, \delta, \mathcal{F}\right)$, where $S$ is a finite set of states, $S_{0} \subseteq S$ is the set of initial states, $2^{\Psi}$ is the set of all alphabets, $\delta: S \times 2^{\Psi} \rightarrow 2^{S}$ is the transition function, $\mathcal{F} \subseteq S$ is the set of accepting states.

\subsection{LTL Planning}\label{subsec:planning}
We briefly highlight the key points of the initial planning (implemented offline), which was developed in \cite{ltl_planner}. The planner node takes as input an LTL
task $\varphi$, and an FTS $\mathcal{T}_w$. First, $\varphi$ is used to generate the NBA $\mathcal{B}_{\varphi}$ via the LTL2BA software \cite{LTL2BA}. Building upon the work of \cite{ltl_planner_2}, a PBA $\mathcal{A_P}$ \cite{ltl_planner} is built, then through model checking techniques \cite{ltl_planner_3} the optimal run that satisfies the given LTL task and the corresponding sequence of actions are found. For all the details we refer to \cite{ltl_planner}, \cite{ltl_planner_2}, \cite{ltl_planner_3}, \cite{meng_paper}.