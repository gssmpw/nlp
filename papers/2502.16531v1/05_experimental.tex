% !TEX root = template.tex

\section{Experimental Results}\label{sec:experimental}
We conducted experiments, within the framework of the CANOPIES project \cite{canopies}, to show the advantages of our approach, most of which are based on the following system.
\subsection{System description}\label{subsec:exp-system}
We consider a workspace measuring $4.2\si{m} \times 5.2\si{m}$ (Fig. \ref{fig:workspace}), and two types of agents: Robotis Turtlebot3 Burger \cite{turtlebot} and Hebi Rosie with robotic arm \cite{rosie}.
\begin{figure}
    \centering
    % First column with two images stacked, labeled (a)
    \begin{minipage}{0.5\linewidth}
        \centering
        \vspace{0.14cm}
        \includegraphics[width=0.4\linewidth]{images/turtlebot.jpg}\\
        \vspace{0.05cm}
        \includegraphics[width=0.4\linewidth]{images/rosie.jpg}
         \subcaption{T. Turtlebot, B. Rosie}\label{fig:robots}
    \end{minipage}%
    % Second column with one image, labeled (b)
    \begin{minipage}{0.5\linewidth}
        \centering
        \includegraphics[width=\textwidth]{images/arena_final_border_v2.png}
         \subcaption{Workspace abstraction}\label{fig:workspace}
    \end{minipage}
    \caption{Experimental setup}
\end{figure}
\subsubsection{Turtlebot}
It knows the ROIs: $C1-4$, $P1-12$, $M$, and $G$. The collaborative actions are  \textit{check\_connection (cc)} at $C1$ or $C2$, \textit{group (g)} at $G$, \textit{remove\_object (ro)} at $M$, the assistive actions are \textit{help\_check\_connection (hcc)} at $C3$ or $C4$, \textit{help\_group (hg)} at $G$  and the local actions are \textit{patrol (p)} at $P1-12$, the movement related actions and, \textit{None}.
\begin{comment}
   The non-movement actions are listed in Tab. \ref{tab:act_turtlebot}.
\begin{table}[ht]
    \centering
    \begin{tabular}{c||c|c}
        \textbf{Action} &  \textbf{Type} & \textbf{Cond}\\
        \hline\hline
        \textit{check\_connection (cc)} &  collaborative & $C1$, $C2$\\
        \hline
        \textit{group (g)} &  collaborative & $G$\\
        \hline
        \textit{remove\_object (ro)} &  collaborative & $M$\\
        \hline
        \textit{help\_check\_connection (hcc)} &  assistive & $C3$, $C4$ \\
        \hline
        \textit{help\_group (hg)} &  assistive & $G$\\
        \hline
        \textit{patrol (p)} &  local & $P1-14$\\
        \hline
        \textit{None} &  local &  \\
        \hline\hline
    \end{tabular}
    \caption{Action model for Turtlebots}
    \label{tab:act_turtlebot}
\end{table} 
\end{comment}
\subsubsection{Rosie}
It knows the ROIs: $H1-4$, $D$, $L$, $M$, and $S$. The collaborative action is \textit{load (l)} at $L$, the assistive actions are \textit{help\_load (hl)} at $L$, \textit{help\_remove\_object (hro)} at $M$ and, the local actions are \textit{harvest (h)} at $H1-4$, \textit{manipulate (m)} at $M$, \textit{deliver (d)} at $D$, \textit{supervise (s)} at $S$, the movement related actions and, \textit{None}.
\begin{comment}
   \begin{table}[ht!]
    \centering
    \begin{tabular}{c||c|c}
        \textbf{Action} &  \textbf{Type} & \textbf{Cond}\\
        \hline\hline
        \textit{load (l)} &  collaborative & $L$\\
        \hline
        \textit{help\_load (hl)} &  assistive & $L$\\
        \hline
        \textit{help\_remove\_object (hro)} &  assistive & $M$\\
        \hline
        \textit{harverst (h)} &  local & $H1-4$\\
        \hline
        \textit{manipulate (m)} &  local & $M$\\
        \hline
        \textit{deliver (d)} &  local & $D$\\
        \hline
        \textit{supervise (s)} &  local & $S$\\
        \hline
        \textit{None} &  local &  \\
        \hline\hline
    \end{tabular}
    \caption{Action model for the Rosies}
    \label{tab:act_rosie}
\end{table} 
\end{comment}
\subsubsection{Task specification}\label{subsec:exp-task}
We consider a basic team composed of 3 Rosies and 6 Turtlebots. The recurring LTL tasks \eqref{eq:recurringLTL} assigned to the agents are as follows:     $\varphi^{\mathrm{rosie}_0}_r=\square\lozenge(h \wedge H1 \wedge \lozenge(h \wedge H3\wedge\lozenge d))$,    $\varphi^{\mathrm{rosie}_1}_r=\square\lozenge(s \wedge \lozenge(l \wedge\lozenge m))$,         $\varphi^{\mathrm{rosie}_2}_r=\square\lozenge(h \wedge H2 \wedge \lozenge(h \wedge H4\wedge\lozenge d))$,             $\varphi^{\mathrm{turtlebot}_0}_r=\square\lozenge(p \wedge P1 \wedge \lozenge(p\wedge P11))$,             $\varphi^{\mathrm{turtlebot}_1}_r=\square\lozenge(p \wedge P2 \wedge \lozenge(p\wedge P10))$,             $\varphi^{\mathrm{turtlebot}_2}_r=\square\lozenge(p \wedge P4 \wedge \lozenge(p \wedge P6))$,             $\varphi^{\mathrm{turtlebot}_3}_r=\square\lozenge(p \wedge P12 \wedge \lozenge(p\wedge P9\wedge \lozenge(cc \wedge C1))$,      $\varphi^{\mathrm{turtlebot}_4}_r=\square\lozenge(p \wedge P3 \wedge \lozenge(p\wedge P8\wedge \lozenge g))$ and, $\varphi^{\mathrm{turtlebot}_5}_r=\lozenge (ro \wedge \lozenge(p \wedge P5 \wedge \lozenge(cc \wedge C2))) \wedge\square\lozenge(p \wedge P5 \wedge \lozenge(p\wedge P7))$. The starting ROIs are respectively $H1$, $M$, $H2$, $P1$, $P2$, $P4$, $P12$, $P3$, $M$.
Lastly, we define \ChooseROI, which is relevant only when $check\_connection$ must be completed. If $cc$ is completed in $C1$, then $hcc$ is completed in $C3$. If $cc$ is completed in $C2$, then $hcc$ is completed in $C4$. 
\subsection{Complexity Reduction}\label{subsec:exp-complexity}
We assess the computational gains discussed in Sec. \ref{subsec:res-complexity}.
\subsubsection{Comparison between ROI and Grid Representation}
We developed an ROI representation that focuses only on the necessary regions for the agent unlike previous approaches \cite{meng_paper} which used a grid structure to partition the workspace. 
\begin{table}[b]
    \centering
    \begin{tabular}{c||c|c|c}
        \textbf{Agent} &  \textbf{States in $\mathcal{T}_{\mathcal{M}}$} &\textbf{States Grid} & \textbf{Reduction}\\
        \hline\hline
        Rosie &  $8$ & $42$ & $81.0\%$\\
        \hline
        Turtlebot&  $18$ & $500$ & $96.4\%$\\
        \hline\hline
    \end{tabular}
    \caption{Computational gains of $\mathcal{T}_{\mathcal{M}}$ over a grid structure}
    \label{tab:motion-state-reduction}
\end{table}
As shown in Tab. \ref{tab:motion-state-reduction}, the grid was partitioned with cells sized to contain the specific robot ($6\times7$ grid for Rosie and $20\times25$ for Turtlebots). 
The ROI representation led to a significant reduction in the number of states, achieving an average reduction of $88.7\%$ compared to the grid representation. This reduction influences the size of the FTS $\mathcal{T}_{\mathcal{G}}$ and the PBA $\mathcal{A_P}$. In the sequel, we focus on the ROI representation.

\subsubsection{Gains of using the FTS}
In Alg. \ref{alg:reply}, we use the FTS $\mathcal{T}_{\mathcal{G}}$, whereas \cite{meng_paper} uses the PBA $\mathcal{A_P}$. This results in an average reduction of $84.8\%$ in the number of states across all agents. The PBA's state count increases with task complexity, hence, $turtlebot_5$, with the most complex task, benefits the most from using $\mathcal{T}_{\mathcal{G}}$ in \Dijkstra. The results are in Tab. \ref{tab:state-reduction}.
\begin{table}[t]
    \centering
    \begin{tabular}{c||c|c|c}
        \textbf{Agent} &  \textbf{States $\mathcal{T}_{\mathcal{G}}$} &\textbf{States $\mathcal{A_P}$} & \textbf{Reduction}\\
        \hline\hline
        $\mathrm{rosie}_{0,1,2}$ &  $18$ & $162$ & $88.9\%$\\
        \hline
        $\mathrm{turtlebot}_{0,1,2}$ &  $37$ & $148$ & $75.0\%$\\
        \hline
        $\mathrm{turtlebot}_{3,4}$ &  $37$ & $333$ & $88.9\%$\\
        \hline
        $\mathrm{turtlebot}_5$ &  $37$ & $592$ & $93.8\%$\\
        \hline\hline
    \end{tabular}
    \caption{Computational gains by FTS against PBA}
    \label{tab:state-reduction}
\end{table}
\subsubsection{MIP Filtering}
Lastly, we analyzed the effect of the filtering procedure on the RRC cycle and on the confirmation step. This was tested in a centralized simulation, varying the number of agents and actions in a request. CycloneDDS \cite{cyclone} was used as the ROS2 middleware for stable communication between nodes.
As shown in Fig. \ref{fig:complexity} (bottom plot), the filtering procedure significantly reduced the confirmation time, especially as the number of agents increased, while the time remained relatively constant for different numbers of actions. The greatest reduction occurs with a single requested action, where the MIP is solved by Proc. \ref{proc:filtering}, with these results being barely visible in the plot due to their low values.
However, this reduction in confirmation time has a limited impact on the overall RRC cycle, as shown in Fig. \ref{fig:complexity} (upper plot), where the bottleneck is the ROS2 communication. Significant improvements only appear with 350 agents, the maximum allowed by the workstation. This suggests that with a larger number of agents ($N \gg M$), the filtering procedure yields substantial gains.
%\vspace{-0.5cm}
\begin{figure}[t]
    \centering    \includegraphics[width=\linewidth]{images/filtering_plot2.png}
    \caption{Effects of agents filtering}
    \label{fig:complexity}
\end{figure}

\subsection{Experimental Results}
\label{subsec:exp-results}
In this setup, we deployed our approach to the available hardware, in a decentralized way. For robot movement, we developed a model predictive controller \cite{mpc} with control barrier function \cite{cbf_1, cbf_2} constraints for collision avoidance. Agent poses were tracked using the Qualisys Motion Capture System \cite{mocap}. For the `$remove\_object$` action, we used a visual servoing controller \cite{visualservo1,visualservo2}, while all other actions were simulated. A video of the experiment is available at \cite{video}.
Fig. \ref{fig:experimental} shows the sequence of actions completed by each agent, demonstrating that all tasks were completed and that the approach synchronized collaborative and assistive actions to start simultaneously, even if some agents were ready earlier. Note that only non-movement actions are shown.
\begin{figure}[h]
    \centering    
    \includegraphics[width=\linewidth]{images/experiment_plot3.pdf}
    \caption{Agents actions in the experimental settings}
    \label{fig:experimental}
\end{figure}
\subsection{Scalability}
\label{subsec:exp-scalability}
To demonstrate the scalability of our approach, we created a simulation with 90 agents, representing 10 of the teams described in Sec. \ref{subsec:exp-system}, due to the unavailability of such a large number of robots. Fig. \ref{fig:scalability} shows the action sequence of a subset of these agents for clarity. The results indicate that the agents successfully completed their assigned tasks, collaborations, and synchronized actions. The successful completion of collaborations by the agents demonstrates the approach's effectiveness in performing well with ninety agents. Experiments with more agents could not be conducted due to RAM constraints. The results indicate that our approach scales well and can perform effectively with larger teams.
\begin{figure}[ht]
    \centering
    \includegraphics[width=\linewidth]{images/90_agents3.pdf}
    \caption{90 agents actions in the scalability simulation}
    \label{fig:scalability}
\end{figure}
% \vspace{-0.12cm}
