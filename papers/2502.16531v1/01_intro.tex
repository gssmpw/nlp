% !TEX root = template.tex

\section{Introduction} \label{sec:introduction}
Successful coordination and synchronization in multi-agent systems (MAS) is essential for completing complex, interdependent tasks that individual agents cannot handle alone,  making it one of the key challenges in robotic applications such as in logistics \cite{logistics} and precision agriculture \cite{agriculture}. As the number of agents and the complexity of tasks increase, ensuring scalability and conflict-free operation becomes more challenging, especially when tasks are expressed through temporal logic constraints. Therefore, finding an efficient and scalable solution is necessary to manage computational complexity and maintain robust performances. 
\par Linear Temporal Logic (LTL) is a logical formalism suitable for defining linear-time properties, widely used in formal verification, particularly in computer systems, and increasingly in robotics for specifying tasks in MAS \cite{ltl_planner_3}, \cite{multiagentltl1}. The introduction of Product B\"{u}chi Automata (PBA) has significantly advanced LTL compliance in MAS by systematically generating control strategies \cite{model-checking}. However, scalability remains a challenge as task complexity and the number of agents increase. Many recent approaches address this computational limitation by avoiding the direct construction of the PBA and relying on alternative control techniques, such as the sampling-based synthesis in \cite{large-scale, stylus} or reinforcement learning (RL) techniques in \cite{RL}. Although efficient, these techniques introduce a stochastic element to planning, whereas we aim to maintain the deterministic guarantees of graph search methods \cite{model-checking}. Other approaches focus on task assignment to individual agents to limit the exponential growth of the state space, rather than using centralized planning \cite{decentralized}. These methods often target specific robotic functions, such as motion \cite{motion2}, which hinders their applicability to more complex tasks, given the specificity of their context. Recent efforts, such as \cite{scratches}, introduce a task grammar that enables flexible, high-level task specification for heterogeneous agents, alleviating the need for explicit task assignments while maintaining correctness guarantees. Furthermore, synchronization mechanisms are becoming increasingly critical to ensure coordinated actions in MAS \cite{synchronization}.
\par In \cite{meng_paper} local tasks defined by syntactically cosafe LTL (sc-LTL) formulas \cite{scLTL} are considered, with agents navigating a grid-structured workspace and performing local, collaborative, and assistive actions. An offline planner generates initial plans for each agent, which are then adapted through a Request, Reply, and Confirmation cycle. The latter utilizes the PBA to incorporate assistive actions into the agents' plans. While robust, this approach is limited to finite-time tasks defined by sc-LTL formulas. In this work, we adapt \cite{meng_paper} by addressing its limitations and enhancing computational efficiency. Unlike the earlier study, which relied solely on simulations and did not fully capture edge cases, our work integrates both experiments and more exhaustive simulations. Specifically, we introduce a region of interest (ROI) based representation to reduce the size of the finite transition system (FTS) \cite{model-checking} (Sec. \ref{subsec:motion-ts}), and subsequently define a new subclass of LTL called recurring LTL to specify tasks that repeat infinitely often (Sec. \ref{subsec:task}). In Alg. \ref{alg:reply}  we eliminate the use of the PBA for plan adaptation in favor of the simpler FTS. Additionally, we reduce the complexity of selecting collaborative agents by using filtering Proc. \ref{proc:filtering} to warm-start the underlying mixed integer program (MIP) \cite{mip}. Lastly, we present a robust synchronization mechanism within the ROS2 \cite{ros2} framework that further enables effective collaboration among agents (Sec \ref{subsec:res-synchro}). Our approach is computationally efficient for real-world usage as proven with a team composed of nine commercially available robotic platforms (Sec. \ref{subsec:exp-results}) and suitable for large-scale robotic deployments as shown by a simulation with ninety robots (see Sec. \ref{subsec:exp-scalability}). The code is available at \cite{repo}.


