\section{Experimental Setups and Additional Empirical Results}\label{apx:exp}
In the section, we introduce the settings in Section~\ref{sec:exp} further, and discuss more experimental results on \nmon and \mon.

\subsection{Applications}\label{apx:app}
\textbf{Maxcut.}
In the context of the maxcut application, we start with a graph $G=(V, E)$ 
where each edge $ij \in E$ has a weight $w_{ij}$.
The objective is to find a cut that maximizes the total weight of edges crossing the cut.
The cut function $f: 2^V \to \reals$ is defined as follows,
\[f(S) = \sum_{i \in S} \sum_{j \in V\setminus S}w_{ij}, \forall S\subseteq V.\]
This is a non-monotone submodular function.
In our implementation, for simplicity, all edges have a weight of $1$.

\textbf{Revmax.}
In our revenue maximization application,
we adopt the revenue maximization model introduced in \citep{DBLP:conf/www/HartlineMS08}, which we will briefly outline here.
Consider a social network $G=(V, E)$,
where $V$ denotes the buyers.
Each buyer $i$'s value for a good depends on the set of buyers $S$ that already own it, 
which is formulated by 
\[v_i(S)=f_i\left(\sum_{j \in S} w_{ij}\right),\]
where $f_i: \reals \to \reals$ is a non-negative, monotone, concave function, and $w_{ij}$ is drawn independently from a distribution.
The total revenue generated from selling goods to the buyers $S$ is
\[f(S) = \sum_{i \in V\setminus S} f_i\left(\sum_{j \in S} w_{ij}\right).\]
This is a non-monotone submodular function.
In our implementation, 
we randomly choose each $w_{ij}\in (0,1)$,
and $f_i(x) = x^{\alpha_i}$, where $\alpha_i \in (0,1)$ is chosen uniformly randomly.

% \textbf{Imgsum.}
% We follow the setting of Personalized Image Summarization application in~\citet{mirzasoleiman2016fast} with the following objective function
% \[f(S) = \sum_{i \in \uni} \max_{j \in S}s_{ij} - \frac{1}{n}\sum_{i \in S}\sum_{j \in S}s_{ij},\]
% where $s_{ij}$ determines the cosine similarity of image $i$ to image $j$
% with pixel vectors.
% The first term tries to ensure that the set $S$ is a good summary of the dataset,
% while the second promotes diversity within the summary itself. 
% This is a non-monotone, submodular objective function.

\subsection{Datasets}\label{apx:data}
\textbf{er} is a synthetic random graph generated by Erd{\"{o}}s-R{\'{e}}nyi model~\citep{erdds1959random} by setting number of nodes $n=100,000$ and edge probability $p=\frac{5}{n}$.

\textbf{web-Google}~\citep{DBLP:journals/im/LeskovecLDM09}  is a web
graph of $n=875,713$ web pages as nodes and $5,105,039$ hyperlinks
as edges.

\textbf{musae-github}~\citep{rozemberczki2019multiscale} is a social network of GitHub developers with $n=37,700$ developers and $289,003$ edges,
where edges are mutual follower relationships between them.

\textbf{twitch-gamers}~\citep{rozemberczki2021twitch} is a social network of $n=168,114$ Twitch users with $6,797,557$ edges, 
where edges are mutual follower relationships between them.


% \textbf{CIFAR-10}~\citep{krizhevsky2009learning} dataset consists of $50,000$ training images and $10,000$ test images
% where each image 
% is represented by a pixel vector of length 3,072:
% $32 \times 32$ pixels with red, green, and blue channels.
% In this paper, we randomly choose $3,000$ images from the training dataset.

\subsection{Additional Results}\label{apx:nmon}
Fig.~\ref{fig:apx} provides additional results on musae-github dataset with $n=37,700$
and web-Google dataset with $n=875,713$.
It shows that as $n$ and $k$ increase, our algorithms achieve superior on objective values.
The results of query complexity and adaptivity align closely with those discussed in Section~\ref{sec:exp}.
Notably, the number of adaptive round of \ptgtwoshort exceeds $k$ on musae-github,
which may be attributed to the dataset's relatively small size.
\begin{figure}[ht]
    \centering
    \subfigure[musae-github, solution value]{\label{fig:git-val}
    \includegraphics[width=0.31\linewidth]{fig/epsi1/git-val.pdf}}
    \subfigure[musae-github, query]{\label{fig:git-query}
    \includegraphics[width=0.31\linewidth]{fig/epsi1/git-query.pdf}}
    \subfigure[musae-github, round]{\label{fig:git-round}
    \includegraphics[width=0.31\linewidth]{fig/epsi1/git-round.pdf}}
    \subfigure[web-Google, solution value]{\label{fig:google-val}
    \includegraphics[width=0.31\linewidth]{fig/epsi1/google-val.pdf}}
    \subfigure[web-Google, query]{\label{fig:google-query}
    \includegraphics[width=0.31\linewidth]{fig/epsi1/google-query.pdf}}
    \subfigure[web-Google, round]{\label{fig:google-round}
    \includegraphics[width=0.31\linewidth]{fig/epsi1/google-round.pdf}}
    \caption{Results for \revmax on musae-github with $n=37,700$,
    and \maxcut on web-Google with $n=875,713$.}
    \label{fig:apx}
\end{figure}

















