%%%%%%%% ICML 2025 EXAMPLE LATEX SUBMISSION FILE %%%%%%%%%%%%%%%%%

\documentclass{article}
\usepackage[letterpaper,top=2cm,bottom=2cm,left=3cm,right=3cm,marginparwidth=1.75cm]{geometry}

% % Recommended, but optional, packages for figures and better typesetting:
% \usepackage{microtype}
% \usepackage{graphicx}
% \usepackage{subfigure}
% \usepackage{booktabs} % for professional tables

% % hyperref makes hyperlinks in the resulting PDF.
% % If your build breaks (sometimes temporarily if a hyperlink spans a page)
% % please comment out the following usepackage line and replace
% % \usepackage{icml2025} with \usepackage[nohyperref]{icml2025} above.
% \usepackage[final]{hyperref}


% % Attempt to make hyperref and algorithmic work together better:
% \newcommand{\theHalgorithm}{\arabic{algorithm}}

% % Use the following line for the initial blind version submitted for review:
% % \usepackage{icml2025}

% % If accepted, instead use the following line for the camera-ready submission:
% % \usepackage[accepted]{icml2025}

% % For theorems and such
% \usepackage{amsmath}
% \usepackage{amssymb}
% \usepackage{mathtools}
% \usepackage{amsthm}

% % if you use cleveref..
% \usepackage[capitalize,noabbrev]{cleveref}

% %%%%%%%%%%%%%%%%%%%%%%%%%%%%%%%%
% % THEOREMS
% %%%%%%%%%%%%%%%%%%%%%%%%%%%%%%%%
% % \theoremstyle{plain}
% % \newtheorem{theorem}{Theorem}[section]
% % \newtheorem{proposition}[theorem]{Proposition}
% % \newtheorem{lemma}[theorem]{Lemma}
% % \newtheorem{corollary}[theorem]{Corollary}
% % \theoremstyle{definition}
% % \newtheorem{definition}[theorem]{Definition}
% % \newtheorem{assumption}[theorem]{Assumption}
% % \theoremstyle{remark}
% % \newtheorem{remark}[theorem]{Remark}

% % Todonotes is useful during development; simply uncomment the next line
% %    and comment out the line below the next line to turn off comments
% %\usepackage[disable,textsize=tiny]{todonotes}
\usepackage[textsize=tiny]{todonotes}
\setlength{\marginparwidth}{1.5cm}
\usepackage{natbib}
\usepackage{url}            % simple URL typesetting
\usepackage{booktabs}       % professional-quality tables
\usepackage[final]{hyperref}       % hyperlinks
\usepackage{amsfonts}       % blackboard math symbols
\usepackage{nicefrac}       % compact symbols for 1/2, etc.
\usepackage{microtype}      % microtypography
\setlength{\textfloatsep}{4mm}
\usepackage{bm}
\usepackage{changepage}
\usepackage{wrapfig}
\usepackage{array, makecell} 
\usepackage{tabularx}
\usepackage{amsmath}
\let\proof\relax \let\endproof\relax
\usepackage{amsthm}
% \usepackage{algorithm}
% \PassOptionsToPackage{notext, nolist}{algorithm}
%\usepackage[noend]{algorithmic}
% \usepackage{algorithmicx}
% \usepackage[noend]{algpseudocode}
% \usepackage{verbatim}
\usepackage{graphicx}
\usepackage[space]{grffile}
% \usepackage{subcaption} % provide subfigure
\usepackage{caption}
% % Redefine the caption format
% \DeclareCaptionLabelFormat{captionless}{}
% \captionsetup[figure]{labelformat=captionless}
\usepackage{tikz}
\usepackage{subfigure}
\usepackage{mathrsfs}
\usepackage{amssymb}
\usepackage{xspace}
\usepackage{thmtools}
\usepackage{thm-restate}
\usetikzlibrary{arrows}
\usepackage{xcolor}
\usepackage{multirow}
\usepackage{threeparttable}
\usepackage{footmisc}
\usepackage{tablefootnote}
\allowdisplaybreaks

\usepackage[ruled,lined,noend,linesnumbered]{algorithm2e}
\DontPrintSemicolon
\SetAlgoProcName{Paradigm}{anautorefname}
\SetKwInOut{Init}{Initialize}
\SetKwProg{Fn}{Procedure}{:}{}

\newcommand\numberthis{\addtocounter{equation}{1}\tag{\theequation}}

% ---- symbols ----
\newcommand{\uni}{\mathcal U}
\newcommand{\reals}{\mathbb{R}_{\ge 0}}
% \newcommand{\todo}[1]{\textcolor{red}{TODO: #1}}
\newcommand{\fix}[1]{\textcolor{red}{#1}}
\newcommand{\etal}{\textit{et al.}\xspace}
\newcommand{\ie}{\textit{i.e.}\xspace}
\newcommand{\eg}{\textit{e.g.}\xspace}
\newcommand{\st}{\textit{s.t.}\xspace}
\newcommand{\func}[2]{ #1 \left( #2 \right) }
\newcommand{\ff}[1]{ f \left( #1 \right) }
\newcommand{\ffsub}[2]{ f_{#1} \left( #2 \right) }
\newcommand{\marge}[2]{\Delta \left( #1 \, \middle| \, #2 \right) }
\newcommand{\margesub}[3]{\Delta_{#1} \left( #2 \, \middle| \, #3 \right) }
\newcommand{\ex}[1]{\mathbb{E}\left[ #1 \right]}
\newcommand{\exs}[2]{ \mathbb{E}_{ #1 } \left[ #2 \right] }
\newcommand{\exc}[2]{ \mathbb{E}\left[\left. #1 \, \right| \; #2 \right] }
\newcommand{\oh}[1]{ \mathcal O \left( #1 \right) }
\newcommand{\epsi}[0]{ \varepsilon }
\newcommand{\opt}{\text{OPT}}
\newcommand{\prob}[1]{ \mathbb{P} \left[ #1 \right] }
\newcommand{\probs}[2]{ \mathbb{P}_{ #1 } \left[ #2 \right] }
\newcommand{\probc}[2]{ \mathbb{P}\left[ #1 \, | \; #2\right] }
\newcommand{\sm}{\textsc{SMCC}\xspace}
\renewcommand{\restriction}{\mathord{\upharpoonright}}
\newcommand{\oht}[1]{\tilde{\mathcal{O}}\left( #1 \right)}

\DeclareMathOperator*{\argmax}{arg\,max}
\DeclareMathOperator*{\argmin}{arg\,min}

\newcommand{\nmon}{\textsc{SM-Gen}\xspace}
\newcommand{\mon}{\textsc{SM-Mon}\xspace}

\newcommand{\maxcut}{$\texttt{maxcut}$\xspace}
\newcommand{\revmax}{$\texttt{revmax}$\xspace}
\newcommand{\imgsum}{$\texttt{imgsum}$\xspace}


% ---- theorem env ----
\usepackage{thmtools,thm-restate}
\declaretheorem[style=definition,numberwithin=section]{theorem}
\declaretheorem[style=definition,sibling=theorem]{lemma}
\declaretheorem[style=definition,numberwithin=section]{proposition}
\declaretheorem[style=definition,numberwithin=section]{definition}
\declaretheorem[style=definition,numberwithin=section]{example}
\declaretheorem[style=definition,numberwithin=section]{remark}
\declaretheorem[style=definition,numberwithin=section]{claim}
\declaretheorem[style=definition,numberwithin=section]{corollary}
\declaretheorem[style=definition]{property}

% ---- algorithms ----
\newcommand{\rg}{\textsc{RandomGreedy}}
\newcommand{\greedy}{\textsc{Greedy}}
\newcommand{\linearseq}{\textsc{LinearSeq}\xspace}
\newcommand{\linearcnst}{\textsc{LinearConstant}\xspace}
\newcommand{\randomset}{\textsc{RandomSet}\xspace}
\newcommand{\ig}{\textsc{InterlaceGreedy}\xspace}
\newcommand{\itg}{\textsc{InterpolatedGreedy}\xspace}
\newcommand{\ptgone}{\textsc{ParallelInterlaceGreedy}\xspace}
\newcommand{\ptgtwo}{\textsc{ParallelInterpolatedGreedy}\xspace}
\newcommand{\ptgoneshort}{\textsc{PIG}\xspace}
\newcommand{\ptgtwoshort}{\textsc{PItG}\xspace}

\newcommand{\sts}{\textsc{SubsampledThreshSeq}\xspace}
\newcommand{\tssg}{\textsc{SubsampledTG}\xspace}
\newcommand{\ptg}{\textsc{ParallelTG}\xspace}
\newcommand{\ts}{\textsc{ThreshSeq}\xspace}
\newcommand{\thresh}{\textsc{Threshold}\xspace}
\newcommand{\tsmod}{\textsc{ThreshSeq-Mod}\xspace}
\newcommand{\tssub}{\textsc{ThreshSeq-Sub}\xspace}
\newcommand{\dist}{\textsc{Distribute}\xspace}
\newcommand{\prefix}{\textsc{Prefix-Selection}\xspace}
\newcommand{\update}{\textsc{Update}\xspace}

\newcommand{\randomgreedy}{\textsc{RandomGreedy}\xspace}
\newcommand{\ltl}{\textsc{LazierThanLazyGreedy}\xspace}
\newcommand{\ltlshort}{\textsc{LTLG}\xspace}
\newcommand{\fast}{\textsc{Fast}\xspace}
\newcommand{\lspgb}{\textsc{LS+PGB}\xspace}
\newcommand{\parskp}{\textsc{ParSKP}\xspace}
\newcommand{\parssp}{\textsc{ParSSP}\xspace}

\newcommand{\frg}{\textsc{FastRandomGreedy}\xspace}
\newcommand{\anm}{\textsc{AdaptiveNonmonotoneMax}\xspace}
\newcommand{\unc}{\textsc{UnconstrainedMax}\xspace}




% The \icmltitle you define below is probably too long as a header.
% Therefore, a short form for the running title is supplied here:
% \icmltitlerunning{Sublinear Adaptive Algorithm for Submodular Maximization}
\title{Breaking Barriers: Combinatorial Algorithms for Non-monotone Submodular Maximization with Sublinear Adaptivity and $1/e$ Approximation}
\date{}
\author{
  Yixin Chen, Wenjing Chen, Alan Kuhnle \\
  Department of Computer Science \& Engineering \\
  Texas A\&M University \\
  Colloge Station, TX\\
  \texttt{\{chen777, jj9754@tamu.edu, kuhnle\}@tamu.edu} \\
}

\begin{document}
\maketitle

% this must go after the closing bracket ] following \twocolumn[ ...

% This command actually creates the footnote in the first column
% listing the affiliations and the copyright notice.
% The command takes one argument, which is text to display at the start of the footnote.
% The \icmlEqualContribution command is standard text for equal contribution.
% Remove it (just {}) if you do not need this facility.

%\printAffiliationsAndNotice{}  % leave blank if no need to mention equal contribution
% \printAffiliationsAndNotice{\icmlEqualContribution} % otherwise use the standard text.

\begin{abstract}
% With the explosion of data in modern applications, practical algorithms have garnered increasing attention under various large-scale data settings.
% This work focuses on developing parallel combinatorial approximation algorithms for maximizing a non-monotone submodular function subject to a size constraint $k$
% with a ground set of size $n$.
% The current state-of-the-art approximation ratio for this problem is $1/e$, achieved by
% a continuous algorithm~\citep{Ene2020a} with adaptivity $\oh{\log(n)}$.
% We propose two parallel combinatorial algorithms, both
% achieving $\oh{\log(n)\log(k)}$ adaptivity and 
% $\oh{n\log(n)\log(k)}$ query complexity.
% These algorithms improve the best-known deterministic approximation ratio from $0.125-\epsi$ to $0.25-\epsi$
% and the best-known randomized approximation ratio from $0.25-\epsi$ to $1/e-\epsi\approx 0.367-\epsi$,
% breaking the barrier between continuous and combinatorial approaches.
% Empirical evaluations demonstrate the effectiveness of our methods, 
% achieving competitive objective values, 
% with the first algorithm excelling in query efficiency.
With the rapid growth of data in modern applications, parallel combinatorial algorithms 
for maximizing non-monotone submodular functions have gained significant attention.
The state-of-the-art approximation ratio of $1/e$ is 
currently achieved only by a continuous algorithm~\citep{Ene2020a} with adaptivity $\oh{\log(n)}$.
In this work, we focus on size constraints 
and propose a $(1/4-\epsi)$-approximation algorithm 
with high probability for this problem, 
as well as the first randomized parallel combinatorial algorithm
achieving a $1/e-\epsi$ approximation ratio,
which bridges the gap between continuous and combinatorial approaches.
Both algorithms achieve $\oh{\log(n)\log(k)}$ adaptivity and 
$\oh{n\log(n)\log(k)}$ query complexity.
Empirical results show our algorithms achieve competitive objective values, 
with the first algorithm particularly efficient in queries.
\end{abstract}

\section{Introduction}
Backdoor attacks pose a concealed yet profound security risk to machine learning (ML) models, for which the adversaries can inject a stealth backdoor into the model during training, enabling them to illicitly control the model's output upon encountering predefined inputs. These attacks can even occur without the knowledge of developers or end-users, thereby undermining the trust in ML systems. As ML becomes more deeply embedded in critical sectors like finance, healthcare, and autonomous driving \citep{he2016deep, liu2020computing, tournier2019mrtrix3, adjabi2020past}, the potential damage from backdoor attacks grows, underscoring the emergency for developing robust defense mechanisms against backdoor attacks.

To address the threat of backdoor attacks, researchers have developed a variety of strategies \cite{liu2018fine,wu2021adversarial,wang2019neural,zeng2022adversarial,zhu2023neural,Zhu_2023_ICCV, wei2024shared,wei2024d3}, aimed at purifying backdoors within victim models. These methods are designed to integrate with current deployment workflows seamlessly and have demonstrated significant success in mitigating the effects of backdoor triggers \cite{wubackdoorbench, wu2023defenses, wu2024backdoorbench,dunnett2024countering}.  However, most state-of-the-art (SOTA) backdoor purification methods operate under the assumption that a small clean dataset, often referred to as \textbf{auxiliary dataset}, is available for purification. Such an assumption poses practical challenges, especially in scenarios where data is scarce. To tackle this challenge, efforts have been made to reduce the size of the required auxiliary dataset~\cite{chai2022oneshot,li2023reconstructive, Zhu_2023_ICCV} and even explore dataset-free purification techniques~\cite{zheng2022data,hong2023revisiting,lin2024fusing}. Although these approaches offer some improvements, recent evaluations \cite{dunnett2024countering, wu2024backdoorbench} continue to highlight the importance of sufficient auxiliary data for achieving robust defenses against backdoor attacks.

While significant progress has been made in reducing the size of auxiliary datasets, an equally critical yet underexplored question remains: \emph{how does the nature of the auxiliary dataset affect purification effectiveness?} In  real-world  applications, auxiliary datasets can vary widely, encompassing in-distribution data, synthetic data, or external data from different sources. Understanding how each type of auxiliary dataset influences the purification effectiveness is vital for selecting or constructing the most suitable auxiliary dataset and the corresponding technique. For instance, when multiple datasets are available, understanding how different datasets contribute to purification can guide defenders in selecting or crafting the most appropriate dataset. Conversely, when only limited auxiliary data is accessible, knowing which purification technique works best under those constraints is critical. Therefore, there is an urgent need for a thorough investigation into the impact of auxiliary datasets on purification effectiveness to guide defenders in  enhancing the security of ML systems. 

In this paper, we systematically investigate the critical role of auxiliary datasets in backdoor purification, aiming to bridge the gap between idealized and practical purification scenarios.  Specifically, we first construct a diverse set of auxiliary datasets to emulate real-world conditions, as summarized in Table~\ref{overall}. These datasets include in-distribution data, synthetic data, and external data from other sources. Through an evaluation of SOTA backdoor purification methods across these datasets, we uncover several critical insights: \textbf{1)} In-distribution datasets, particularly those carefully filtered from the original training data of the victim model, effectively preserve the model’s utility for its intended tasks but may fall short in eliminating backdoors. \textbf{2)} Incorporating OOD datasets can help the model forget backdoors but also bring the risk of forgetting critical learned knowledge, significantly degrading its overall performance. Building on these findings, we propose Guided Input Calibration (GIC), a novel technique that enhances backdoor purification by adaptively transforming auxiliary data to better align with the victim model’s learned representations. By leveraging the victim model itself to guide this transformation, GIC optimizes the purification process, striking a balance between preserving model utility and mitigating backdoor threats. Extensive experiments demonstrate that GIC significantly improves the effectiveness of backdoor purification across diverse auxiliary datasets, providing a practical and robust defense solution.

Our main contributions are threefold:
\textbf{1) Impact analysis of auxiliary datasets:} We take the \textbf{first step}  in systematically investigating how different types of auxiliary datasets influence backdoor purification effectiveness. Our findings provide novel insights and serve as a foundation for future research on optimizing dataset selection and construction for enhanced backdoor defense.
%
\textbf{2) Compilation and evaluation of diverse auxiliary datasets:}  We have compiled and rigorously evaluated a diverse set of auxiliary datasets using SOTA purification methods, making our datasets and code publicly available to facilitate and support future research on practical backdoor defense strategies.
%
\textbf{3) Introduction of GIC:} We introduce GIC, the \textbf{first} dedicated solution designed to align auxiliary datasets with the model’s learned representations, significantly enhancing backdoor mitigation across various dataset types. Our approach sets a new benchmark for practical and effective backdoor defense.



\section{Preliminary}
\textbf{Notation.}
We denote the marginal gain of adding $A$ to $B$ by $\marge{A}{B} = \ff{A\cup B} - \ff{B}$.
For every set $S\subseteq U$ and an element $x\in \uni$,
we denote $S\cup \{x\}$ by $S+x$ and $S\setminus \{x\}$ by $S-x$.

\textbf{Submodularity.}
A set function $f:2^\uni\to \reals$ is submodular, 
if $\marge{x}{S}\ge \marge{x}{T}$ for all $S\subseteq T\subseteq \uni$
and $x\in \uni\setminus T$,
or equivalently, for all $A, B\subseteq \uni$,
it holds that $\ff{A} + \ff{B}\ge \ff{A\cup B} + \ff{A\cap B}$.
With a size constraint $k$,
let $O =\argmax_{S\subseteq \uni, |S| \le k} \ff{S}$.

In Appendix~\ref{apx:prop}, we provide several key propositions
derived from submodularity that streamline the analysis.

\textbf{Organization.}
In Section~\ref{sec:gd}, the blending technique is introduced
and applied to \ig and \itg,
with detailed analysis provided in Appendix~\ref{apx:greedy-1/4}
and~\ref{apx:greedy-1/e}.
Section 4 discusses their fast versions with
pseudocodes and comprehensive analysis in
Appendix~\ref{apx:tg}.
Subsequently, Section~\ref{sec:ptg} delves into our sublinear adaptive algorithms, 
with further technical details and proofs available in Appendix~\ref{apx:ptg}.
Finally, we provide the empirical evaluation in Section~\ref{sec:exp},
with its detailed setups and additional results
in Appendix~\ref{apx:exp}.


\section{The Blending Technique for Greedy}\label{sec:gd}
% \section{A Fresh Take on Interlaced Greedy Based Algorithms}
% In this section, we present a novel analysis of 
% \ig~\citep{DBLP:conf/nips/Kuhnle19} and \itg~\citep{DBLP:conf/kdd/ChenK23}.
% This analysis eliminates the need for the guessing step in both algorithms,
% simplifying their implementation while preserving their theoretical guarantees.
In this section, we present two practical greedy variants
that simplify \ig~\citep{DBLP:conf/nips/Kuhnle19} and \itg~\citep{DBLP:conf/kdd/ChenK23}.
These variants are developed using a novel analysis technique 
called the blended marginal gains strategy. 
Notably, the proposed algorithms are not only simpler than \ig and \itg 
but also retain their theoretical guarantees, 
making them both efficient and theoretically sound. 
This section provides a detailed exposition of these algorithms and their underlying principles.

\subsection{Deterministic Greedy Variant with $1/4$ Approximation Ratio}
\label{sec:greedy-1/4}
\textbf{Analysis of \ig.}
\ig (Alg.~\ref{alg:ig} in Appendix~\ref{apx:pseudocode}) operates as follows.
% Initially, it guesses whether the max singleton $a_0$ is in the optimal solution $O$.
First, two empty solution sets, $A$ and $B$, are maintained
and constructed by interlacing two greedy procedures.
Second, two additional solution sets, $D$ and $E$, are initialized with the maximum singleton $a_0$,
and are constructed using the same interlaced greedy procedure.
% and they are constructed using an interlaced greedy for loop.
% (same as Lines~\ref{line:gdone-for-begin}-\ref{line:gdone-for-end} of Alg.~\ref{alg:gdone}).
% Conversely, if $a_0\in O$, $A$ and $B$ are initialized with $\{a_0\}$,
% and the interlaced greedy step is repeated.
Finally, the best solution among these sets are returned.

The core idea of the algorithm is to greedily construct two disjoint solutions,
$A$ and $B$, in an alternating manner,
ensuring the following inequalities,
\begin{align}
    &\ff{O\cup A} + \ff{O\cup B} \ge \ff{O}, \label{inq:gdone-opt}\\
    &2\ff{A} \ge \ff{O\cup A},\label{inq:gdone-A}\\
    &2\ff{B} \ge \ff{O\cup B},\label{inq:gdone-B}
\end{align}
Note that, the element $a_0$ is treated separately.
% 1) if $a_0\notin O$, then above inequalities holds for $\{S, T\} = \{A, B\}$;
% 2) otherwise, $\{S, T\} = \{D, E\}$.
The second round of the interlaced greedy procedure is specifically designed to
handle the case where $a_0\in O$.
The above inequalities are then guaranteed to hold with solution sets $\{D, E\}$.
However, intuitively, adding an element from $O$ to the solution should not negatively impact it.
This leads to a natural question: 
\textit{Can the second interlaced greedy step be eliminated?}    


% To bound $\marge{O}{A}$, consider the elements in $O\setminus A$.
% Since the interlaced greedy for loop begins by adding elements to $A$
% in both cases,
% $O\setminus A$ can be ordered as $\{o_1, o_2, \ldots\}$,
% where each $o_i$ is a candidate for greedy selection
% when the $i$-th element is added to $A$.
% By the greedy selection rule,
% $\marge{o_i}{A} \le \ff{A_i} -\ff{A_{i-1}}$ for each $o_i\in O\setminus A$,
% where $A_i$ represents the first $i$ elements added to $A$.
% Then, by the first property in Proposition~\ref{prop:sum-marge}, it follows that

% \vspace*{-1em}
% {\small\begin{equation*}
% \marge{O}{A} \le \ff{A} \Rightarrow \ff{O\cup A} \le 2\ff{A}.
% \end{equation*}}

% Bounding $\marge{O}{B}$ requires a slightly different approach.
% In general, if $a_0 \notin O\setminus B$,
% the above analysis can be directly applied to bound $\marge{O}{B}$.
% Next, consider the two cases of $a_0$ discussed earlier.
% In the first case, where $a_0 \notin O$,
% it naturally follows that $a_0 \notin O\setminus B$.
% In the second case, where $a_0 \in O$,
% since $a_0$ is also included in $B$,
% it again follows that $a_0 \notin O\setminus B$.
% Therefore, a similar conclusion holds,

% \vspace*{-1em}
% {\small\begin{equation*}
% \marge{O}{B} \le \ff{B} \Rightarrow \ff{O\cup B} \le 2\ff{B}.
% \end{equation*}}

% However, this result does not directly extend to $B$,
% as the first element added to $A$, say $a_0$, might be in $O$.
% This prevents us from bounding 
% $\marge{a_0}{B}$ by $\ff{B_1}- \ff{\emptyset}$.
% To address this issue, $a_0$ is also added to $B$
% and solutions are then built with the same interlaced greedy step.
% Then, we can reorder $O\setminus B$ as $\{o_1, o_2, \ldots\}$,
% satisfying $\marge{o_i}{B} \le \ff{B_i} - \ff{B_{i-1}}$.
% Furthermore, the shared element $a_0$ is in $O$ which does not break Inequality~\eqref{inq:gdone-opt}.

% In summary, the second interlaced greedy step is introduced
% to address the case where $a_0\in O$.
% However, intuitively, adding an element from $O$ to the solution should not compromise it.
% This raises the question: 

% \textit{Can we eliminate the second interlaced greedy step?}

\textbf{A New Insight on Analyzing \ig: Blended Marginal Gains Strategy.}
The answer to the above question is YES.
In the original analysis of \ig, the upper bound of $\ff{O\cup A}$
only relies on $\ff{A}$ (Inequality~\eqref{inq:gdone-A}). 
Same as $\ff{O\cup B}$ (Inequality~\eqref{inq:gdone-B}).
In what follows,
we introduce a blended approach to analyze \ig with only
a single interlaced greedy step (Alg.~\ref{alg:gdone}).
Specifically, we utilize a mixture of $\ff{A}$ and $\ff{B}$,
or more precisely, a combination of the marginal gains when adding elements to each,
to establish tighter bounds for $\ff{O\cup A}$ or $\ff{O\cup B}$.
\begin{algorithm}[ht]
    \KwIn{evaluation oracle $f:2^{\uni} 
    \to \reals$, constraint $k$}
    \Init{$A\gets B\gets \emptyset$, add $2k$ dummy elements to the ground set}
    \For{$i\gets 1$ to $k$ \label{line:gdone-for-begin}}{
        $a\gets \argmax_{x\in \uni\setminus \left(A\cup B\right)} \marge{x}{A}$\; \label{line:gdone-greedy-A}
        $A\gets A+a$\;
        % \tcc*[r]{If $\marge{a_i}{A_{i-1}} < 0$, a dummy element is added instead.}
        $b\gets \argmax_{x\in \uni\setminus \left(A\cup B\right)} \marge{x}{B}$\; \label{line:gdone-greedy-B}
        $B\gets B+b$\;\label{line:gdone-for-end}
        % \tcc*[r]{If $\marge{b_i}{B_{i-1}} < 0$, a dummy element is added instead.}
    }
    \Return{$S\gets \argmax\{\ff{A}, \ff{B}\}$}
    \caption{A deterministic $1/4$-approximation algorithm with $\oh{ nk }$ queries.}
    \label{alg:gdone}
\end{algorithm}

We summarize our blending technique as follows.
Rather than relying solely on the greedy selection rule to 
bound $\marge{o}{A}$ or $\marge{o}{B}$ for each $o\in O$,
we split $O$ into two parts,
relaxing the marginal gain of one part solely through submodularity.
% To bound $\marge{O}{A}$, 
% split $O$ into two parts. 
% One is the part overlapping with the prefixes of $B$,
% and we use portion of $\ff{B}$ to bound it by submodularity.
% For the remaining element, we follow the original analysis
% of \ig and bound it with portion of $\ff{A}$.
Below, we provide a general proposition that captures the key
insight achieved by the blending technique.
\begin{proposition}\label{prop:blend}
For any submodular function $f:2^{\uni}\to \reals$ and $A, B, O \in \uni$,
Let $A_i$ be a prefix of $A$ with size $i$
such that $A_i \subseteq O$.
Similarly, define $B_j$.
It satisfies that,
\begin{align}
    \marge{O}{B} &\le \ff{A_i} + \marge{O\setminus A_i}{B},\label{inq:gdone-B2}\\
    \marge{O}{A} &\le \ff{B_j} + \marge{O\setminus B_j}{A}, \label{inq:gdone-A2}
\end{align}
\end{proposition}


By summing up Inequalities~\eqref{inq:gdone-B2} and~\eqref{inq:gdone-A2} and carefully
selecting the values of $i$ and $j$ under different cases illustrated in Fig.~\ref{fig:gdone},
the $1/4$ approximation ratio holds for Alg.~\ref{alg:gdone}.
\begin{figure}[ht]
\centering
\subfigure[$i^* \le j^*$]{\label{fig:gdone-1}\includegraphics[width=0.2\textwidth]{fig/ig-1.pdf}}
\subfigure[$i^* > j^*$]{\label{fig:gdone-2}\includegraphics[width=0.2\textwidth]{fig/ig-2.pdf}}
    \caption{This figure depicts the components of solution sets $A$ and $B$ in Alg.~\ref{alg:gdone}.
    The black rectangle highlights a sequence of consecutive elements from $O$
    that were added to the solution at the initial.
    Red circles with a cross mark signifies the first element in $A$ or $B$ that is outside $O$.
    }
\label{fig:gdone}
\end{figure}

% Let $a_i$ be the $i$-th element added to $A$,
% and $A_i$ be the set containing the first $i$ elements of $A$.
% Similarly, define $b_i$ and $B_i$ for the solution $B$.

% Following the original analysis of \ig, 
% by the greedy selection rule, the following inequalities hold,

% \vspace*{-1em}
% {\small\begin{align}
%     &\marge{o}{A_{i-1}} \le \marge{a_i}{A_{i-1}}, 
%     \forall o\in O\setminus (A_{i-1} \cup B_{i-1}) \label{inq:gdone-blend-1}\\
%     &\marge{o}{B_{i-1}} \le \marge{b_i}{B_{i-1}}, 
%     \forall o\in O\setminus (A_{i} \cup B_{i-1})\label{inq:gdone-blend-2}
% \end{align}}
% To derive a new bound, we track the longest prefix of $A$ and $B$ 
% that lies within the optimal solution $O$.
% Define $i^* = \argmax\{i \in [k]: A_i \subseteq O\}$
% and $j^* = \argmax\{i \in [k]: B_i \subseteq O\}$.
% Refer to Fig.~\ref{fig:gdone} as an illustration.

% For any $i \le i^*$, let $o_i = a_i$ and $O_i = A_i$.
% Then, by submodularity,

% \vspace*{-1em}
% {\small\begin{align}\label{inq:gdone-blend-3}
%     \marge{o_i}{B\cup O_{i-1}} \le \marge{a_i}{A_{i-1}}, \forall i \le i^*.
% \end{align}}
% Thus, by ordering $O\setminus B$,
% we can bound $\marge{O}{B}$ by blending Inequalities~\eqref{inq:gdone-blend-2}
% and~\eqref{inq:gdone-blend-3}.

% Similarly, for any $i \le j^*$, let $o_i = b_i$ and $O_i = B_i$.
% By submodularity,

% \vspace*{-1em}
% {\small\begin{align}\label{inq:gdone-blend-4}
%     \marge{o_i}{A\cup O_{i-1}} \le \marge{b_i}{B_{i-1}}, \forall i \le j^*.
% \end{align}}
% By ordering $O\setminus A$,
% we can bound $\marge{O}{A}$ by blending Inequalities~\eqref{inq:gdone-blend-1}
% and~\eqref{inq:gdone-blend-4}.

% By choosing the best blending strategy under different cases
% (Fig.~\ref{fig:gdone-1} and~\ref{fig:gdone-2}).
% It holds that

% \vspace*{-1em}
% {\small\begin{align*}
%     \marge{O}{A} + \marge{O}{B} \le \ff{A} + \ff{B}.
% \end{align*}}

% Next, order $O\setminus A$ as $\{o_1, o_2, \ldots\}$ such that
% $o_i = b_i$ if $o_i\in B$.
% Let $O_i$ be the first $i$ elements with this order.
% Then, if $i \le j^*$, we know that $o_i = b_i$ and $O_{i-1} = B_{i-1}$.
% By submodularity, it holds that,
% \begin{equation}\label{inq:gdone-blend-1}
% \marge{o_i}{A\cup O_{i-1}} \le \marge{b_i}{B_{i-1}}, \forall i \le j^*.
% \end{equation}
% Moreover, since $o_i \notin B_{i-1}$ for each $o_i \in O\setminus A$,
% $o_i$ is a candidate when $a_i$ is added to $A$.
% Thus, the following inequality holds by submodularity and the greedy selection rule,
% \begin{equation}\label{inq:gdone-blend-2}
% \marge{o_i}{A\cup O_{i-1}} \le \marge{a_i}{A_{i-1}}, \forall i \ge 1.
% \end{equation}

% Similarly, we can order $O\setminus B$ as $\{o_1, o_2, \ldots\}$ such that
% $o_i = a_i$ if $o_i\in A$,
% and let $O_i$ be the first $i$ elements with this order.
% Then, if $i \le i^*$, we know that $o_i = a_i$ and $O_{i-1} = A_{i-1}$.
% By submodularity, it holds that,
% \begin{equation}\label{inq:gdone-blend-3}
% \marge{o_i}{B\cup O_{i-1}} \le \marge{a_i}{A_{i-1}}, \forall i \le i^*.
% \end{equation}
% Also, since $o_{i} \notin A_{i-1}$,
% $o_i$ is a candidate when $b_{i-1}$ is added to $B$ for each $i \ge 2$.
% Thus,
% \begin{equation}\label{inq:gdone-blend-4}
% \marge{o_i}{B\cup O_{i-1}} \le \marge{b_{i-1}}{B_{i-2}}, \forall i \ge 2.
% \end{equation}
We provide the theoretical guarantee of Alg.~\ref{alg:gdone} below.
The detailed analysis can be found in Appendix~\ref{apx:greedy-1/4}.
\begin{restatable}{theorem}{thmgdone}\label{thm:gdone}
With input instance $(f, k)$, Alg.~\ref{alg:gdone} returns a set $S$ with $\oh{kn}$ queries
such that $\ff{S} \ge 1/4 \ff{O}$.
\end{restatable}


\begin{algorithm}[ht]
	\KwIn{evaluation oracle $f:2^{\uni} \to \reals$, constraint $k$, size of solution $\ell$, error $\epsi$}
    \Init{$G\gets \emptyset$, $V\gets \uni$, $m \gets \left\lfloor\frac{k}{\ell}\right\rfloor$, add $2k$ dummy elements to the ground set.}
    \For{$i\gets 1$ to $\ell$}{
    	$A_{l}\gets G, \forall l \in [\ell]$\;
    	\For{$j\gets 1$ to $m$}{\label{line:gdtwo-for-2-start}
    		\For{$l\gets 1$ to $\ell$}{
    			$a \gets \argmax_{x\in V}\marge{x}{A_{l}}$\;
                % $\delta_{l, j} \gets \marge{a_{l, j}}{A_l}$\;
    			$A_{l}\gets A_{l}+a$, $V\gets V-a$ \hspace*{-0.7em}\;
                % \tcc*[r]{If $\marge{a_{l, j}}{A_{l}} < 0$, a dummy element is added instead.}
    		}
    	}\label{line:gdtwo-for-2-end}
        % $A_l'\gets \left\{a_{l, j}: j\in \argmax\limits_{I\subseteq [m], |I| = m-1} \sum\limits_{j\in I}\delta_{l, j}\right\}$\;
    	$G\gets$ a random set in $\{A_l\}_{l\in [\ell]}$\;
    }
    \Return{$G$}
    \caption{A randomized $1/e$-approximation algorithm with $\oh{ nk\ell }$ queries. }
    \label{alg:gdtwo}
\end{algorithm}
\subsection{Randomized Greedy Variant with $1/e$ Approximation Ratio}
\label{sec:greedy-1/e}
In this section, we extend the blending technique introduced in the previous section 
to \itg~\citep{DBLP:conf/kdd/ChenK23} 
and propose a simplified algorithm (Alg.~\ref{alg:gdtwo}).
Alg.~\ref{alg:gdtwo} avoids the guessing step in \itg
improving the success probability from $(\ell+1)^{-\ell}$ to $1$.
The intuition behind the algorithm is outlined in the following.
\begin{restatable}{theorem}{thmgdtwo}\label{thm:gdtwo}
With input instance $(f, k, \ell, \epsi)$
such that $\ell =\oh{\epsi^{-1}} \ge \frac{2}{e\epsi}$ and $k \ge \frac{2(e\ell-2)}{e\epsi-\frac{2}{\ell}}$,
Alg.~\ref{alg:gdtwo} returns a set $G$ with $\oh{kn/\epsi}$ queries
such that $\ex{\ff{G}} \ge \left(1/e-\epsi\right) \ff{O}$.
\end{restatable}
In the following discussion, we focus on the case where $k\, \text{mod}\,\ell = 0$.
For the scenario where $k \, \text{mod}\,\ell > 0$, please refer to Appendix~\ref{apx:greedy-1/e}.

% \itg can be seen as an interpolation between 
% the standard greedy algorithm~\citep{DBLP:journals/mp/NemhauserWF78}
% and the \rg~\cite{DBLP:conf/soda/BuchbinderFNS14}.
% The algorithm runs a for loop, where in each iteration,
% built upon the solutions returned from the previous iteration,
% it creates $\ell+1$ pools, each containing $\ell$ candidate solutions,
% by guessing the element in the optimal solution that provides the largest marginal gain.
% To find the right pool, the algorithm must search through all pools.
% In the following, we offer a different perspective on analyzing the algorithm
% and show how the guessing step can be eliminated.

\textbf{From Interlacing $2$ Greedy to $\ell$ Greedy.}
% Let $G$ be the intermediate solution at the beginning of an iteration in \itg,
% and $\{A_l: l \in [\ell]\}$ be the solution sets at the end of this iteration.
% In iteration $m$ of the outer for loop in \itg (Alg.~\ref{alg:itg} in Appendix~\ref{apx:pseudocode}),
% $G_{m-1}$ represents the intermediate solution at the beginning of this iteration,
% and $\{a_1, \ldots, a_\ell\}$ are the top $\ell$ elements in $\uni\setminus G_{m-1}$
% with the largest marginal gains on $G_{m-1}$.
% Similar to \ig, \itg makes $\ell+1$ guesses to locate the element
% $o_{\max} = \argmax_{x \in O\setminus G_{m-1}} \marge{x}{G_{m-1}}$.
% Based on these guesses, $\ell+1$ sets of solutions are initialized,
% each constructed using the interlaced greedy procedure.
% At the end of iteration $m$, the algorithm finalizes the sets
% $\{A_{u,i}: 1\le i\le \ell, 0\le u\le \ell\}$.
% With a probability of $(\ell+1)^{-1}$, $G_m$ is chosen from
% the set of solutions $\{A_{u,i}: 1\le i\le \ell\}$ corresponding to the correct guess.
% Under this selection, the following inequality holds,
%
% \vspace*{-1em}
% {\small\[\marge{O}{A_{u, i}} \le \ell\marge{A_{u, i}}{G}, \forall i \in [\ell].\]}
%
% Recall the improvement we made on \ig, introduced in Section~\ref{sec:greedy-1/4}.
% This raises a direct question:
% \textit{Can we apply the blending technique to bound $\sum_{l=1}^\ell\marge{O}{A_l}$ without relying on the guessing step?}
By interlacing $\ell$ greedy procedures, \itg 
(Alg.~\ref{alg:itg} in Appendix~\ref{apx:pseudocode}) constructs 
$\ell+1$ pools of candidates, each containing $\ell$ nearly pairwise disjoint sets.
Among these pools, only $1$ is the \textit{right} pool that ensures the following guarantee.
\begin{align}
    \marge{O}{A_{u, i}} \le \ell\marge{A_{u, i}}{G}, \forall i \in [\ell],\label{inq:gdtwo-itg}
\end{align}
where $G$ is the intermediate solution at the start of this iteration,
and $A_{u,i}$ is the $i$-th solution in the $u$-th pool.
At each iteration, \itg randomly selects a set from all candidate pools,
resulting in a success probability of $(\ell+1)^{-\ell}$,
where the algorithm consistently identifies the correct pool.
In the following, we introduce how to incorporate the blending technique
into the analysis to eliminate the guessing step in \itg.

\textbf{Blended Marginal Gains for Each Pair of Solutions.}
% The only difference between the interlaced greedy procedures
% in Alg.~\ref{alg:gdone} and~\ref{alg:gdtwo}
% is the number of solutions built.
Unlike Alg.~\ref{alg:gdone}, which constructs solutions each of size $k$,
Alg.~\ref{alg:gdtwo} builds solutions with a total size of $k$.
% Different from the interlaced greedy procedure in Alg.~\ref{alg:gdone},
% the procedure in Alg.~\ref{alg:gdtwo} constructs
% $\ell$ solutions, each containing $k/\ell$ elements, rather than $k$.
To align with this structure, we partition $O$ into $\ell$ subsets,
as outlined in Claim~\ref{claim:par-A},
and pair each subset with one of the solutions.
\begin{restatable}{claim}{claimParA}\label{claim:par-A}
At an iteration $i$ of the outer for loop in Alg.~\ref{alg:gdtwo},
let $G_{i-1}$ be $G$ at the start of this iteration,
and $A_{l}$ be the set at the end of this iteration,
for each $l\in [\ell]$.
% Add dummy elements to $O\setminus G_{i-1}$ until its size equals $k$.
The set $O\setminus G_{i-1}$ can then be split into $\ell$ pairwise disjoint sets $\{O_1, \ldots, O_\ell\}$
such that $|O_l| \le\frac{k}{\ell}$ and $\left(O\setminus G_{i-1}\right) \cap \left(A_{l}\setminus G_{i-1}\right) \subseteq O_l$, for all $l \in [\ell]$.
\end{restatable}
Moreover, based on this claim,
we partition the marginal gain of adding $O$ to each $A_l$ as follows.
\begin{align*}
&\sum_{l\in [\ell]}\marge{O}{A_{l}} \le \sum_{l\in [\ell]} \sum_{i\in [\ell]} \marge{O_i}{A_{l}} \tag{Proposition~\ref{prop:sum-marge}}\\
&= \sum_{1\le l_1 < l_2 \le \ell} \left(\marge{O_{l_1}}{A_{l_2}}+\marge{O_{l_2}}{A_{l_1}}\right)+\sum_{l\in [\ell]} \marge{O_l}{A_{l}}. \numberthis \label{inq:gdtwo-par}
\end{align*}
According to Claim~\ref{claim:par-A}, any element in $O_l\setminus A_l$ 
is not sufficiently beneficial to be added to any solution set.
This ensures that $\marge{O_l}{A_{l}}\le \marge{A_l}{G_{i-1}}$.
As for the term $\marge{O_{l_1}}{A_{l_2}}+\marge{O_{l_2}}{A_{l_1}}$,
since elements are added to the solutions in an alternating manner,
Proposition~\ref{prop:blend} for the blending technique can be applied
to bound this term with the greedy selection rule.
These insights are formalized in the following lemma.
\begin{restatable}{lemma}{lemmaparA}\label{lemma:par-A}
Fix on $G_{i-1}$ for an iteration $i$ of the outer for loop in Alg.~\ref{alg:gdtwo}.
Following the definition in Claim~\ref{claim:par-A}, it holds that
\begin{align*}
\text{1) }&\marge{A_{l}}{G_{i-1}} \ge \marge{O_{l}}{A_{l}}, \forall 1\le l \le \ell,\\
\text{2) }&\left(1+\frac{1}{m}\right)\left(\marge{A_{l_1}}{G_{i-1}} + \marge{A_{l_2}}{G_{i-1}}\right)\ge \marge{O_{l_2}}{A_{l_1}} + \marge{O_{l_1}}{A_{l_2}}, \forall 1\le l_1< l_2 \le \ell.
\end{align*}
\end{restatable}
By applying Inequality~\eqref{inq:gdtwo-par} and Lemma~\ref{lemma:par-A}, 
we derive a result analogous to Inequality~\eqref{inq:gdtwo-itg} achieved by \itg,
\begin{align}
    \sum_{l\in [\ell]}\marge{O}{A_{u, i}} \le \ell\left(1+\frac{1}{m}\right)\sum_{l\in [\ell]}\marge{A_{u, i}}{G}.
\end{align} 
This forms the key property necessary to establish the $1/e-\epsi$ approximation ratio.
The detailed analysis of the approximation ratio is provided in Appendix~\ref{apx:greedy-1/e}.

\section{Preliminary Warm-Up of Parallel Approaches: Nearly-Linear Time Algorithms}\label{sec:tg}
In this section, we introduce the fast versions of Alg.~\ref{alg:gdone}
and~\ref{alg:gdtwo},
which substitute the standard greedy procedures
with descending threshold greedy procedure~\citep{DBLP:conf/soda/BadanidiyuruV14}
to achieve a query complexity of $\oh{n\log(k)}$.
Pseudocodes are provided in Appendix~\ref{apx:tg}
as Alg.~\ref{alg:tgone} and~\ref{alg:tgtwo}.
These fast algorithms serve as building blocks for the parallel algorithms introduced later in this work.
Below, we discuss the intuition behind Alg.~\ref{alg:tgone},
while Alg.~\ref{alg:tgtwo} operates in a similar fashion.

Like the greedy variants discussed earlier,
Alg.~\ref{alg:tgone} constructs the solution sets also in
an alternating manner.
Following the blending analysis technique introduced 
in Section~\ref{sec:greedy-1/e}-particularly Proposition~\ref{prop:blend}-the
primary challenge lies in bounding $\marge{O\setminus A_i}{B}$
and $\marge{O\setminus B_i}{A}$
under the threshold greedy framework.

By the alternating addition property,
the threshold value $\tau_i$ decreases to $\tau_i/(1-\epsi)$
if and only if any element outside $A\cup B$ has marginal gain
less than $\tau_i$.
% Furthermore, if the threshold value for solution $A$
% decreases from $\tau_1$ to $\tau_1/(1-\epsi)$ at some point,
% we know that any element outside $A\cup B$ has margianl gain less than $\tau_1$.
% Same for solution $B$.
Define $a_i$ as the $i$-th element added to $A$,
$A_i$ as the first $i$ elements added to $A$,
and $\tau_1^{a_i}$ as the threshold value when adding $a_i$ to $A$.
Similarly, define $b_i$, $B_i$, and $\tau_2^{b_i}$.
The following inequalities hold with upper bounds increased by a factor of
$1/(1-\epsi)$ compared to the standard greedy variants:
% Then, different from the properties 
% (Inequalities~\eqref{inq:gdone-blend-1} and~\eqref{inq:gdone-blend-2}) guaranteed by the standard greedy procedure,
% the descending threshold greedy procedure ensures
% the following inequalities,
% % for each $a_i$ with the corresponding threshold value $\tau_{a_i}$
% % when $a_i$ is added to $A$,
% % any element outside $A_{i-1}\cup B_{i-1}$ has marginal gain
% % less than $\tau_{a_i} /(1-\epsi)$.
% % Similarly, any element outside $A_{i}\cup B_{i-1}$ has marginal gain
% % less than $\tau_{b_i} /(1-\epsi)$.
% % The following inequalities are guaranteed by Alg.~\ref{alg:tgone}.
\begin{align*}
    &\marge{o}{A_{i-1}} \le \tau_1^{a_i} /(1-\epsi) \le \marge{a_i}{A_{i-1}}/(1-\epsi), \forall o\in O\setminus (A_{i-1} \cup B_{i-1}) \\
    &\marge{o}{B_{i-1}} \le \tau_2^{b_i} /(1-\epsi) \le \marge{b_i}{B_{i-1}}/(1-\epsi),\forall o\in O\setminus (A_{i} \cup B_{i-1})
\end{align*}
% However, unlike Alg.~\ref{alg:gdone},
% Alg.~\ref{alg:tgone} might return a set with size less than $k$.
% If final solution $A$ has size less than $k$,
% then, for any $o\in O\setminus (A\cup B)$,
% it holds that 
%
% \vspace*{-1em}
% {\small\begin{equation}\label{inq:tgone-blend-3}
%     \marge{o}{A} < \frac{\epsi M}{(1-\epsi)k} \le \frac{\epsi}{(1-\epsi)k}\ff{O}, \text{ if } |A| < k.
% \end{equation}}
% We can get a similar result for $B$ as follows,
%
% \vspace*{-1em}
% {\small\begin{equation}\label{inq:tgone-blend-4}
%     \marge{o}{B} < \frac{\epsi M}{(1-\epsi)k} \le \frac{\epsi}{(1-\epsi)k}\ff{O}, \text{ if } |B| < k.
% \end{equation}}
%
% Moreover, Inequalities~\eqref{inq:gdone-blend-3} and~\eqref{inq:gdone-blend-4} are also ensured in this case.
%
% Therefore, by blending those Inequalities, we can prove that
% Alg.~\ref{alg:tgone} achieve $1/4-\epsi$ approximation ratio.
Overall, Alg.~\ref{alg:tgone} sacrifices a constant $\epsi$ in approximation ratio
compared to Alg.~\ref{alg:gdone},
but achieves a significantly improved query complexity of $\oh{n\log (k)}$.

In the following, we provide the theoretical guarantees for
Alg.~\ref{alg:tgone} and~\ref{alg:tgtwo}
and left their detailed analysis in Appendix~\ref{apx:tg}.
\begin{restatable}{theorem}{thmtgone}\label{thm:tgone}
With input instance $(f, k, \epsi)$, Alg.~\ref{alg:tgone} returns a set $S$ with $\oh{n\log (k)/\epsi}$ queries
such that $\ff{S} \ge \left(\frac{1}{4}-\epsi\right) \ff{O}$.
\end{restatable}

\begin{restatable}{theorem}{thmtgtwo}\label{thm:tgtwo}
With input instance $(f, k, \epsi)$
such that $\ell = \oh{\epsi^{-1}}\ge \frac{4}{e\epsi}$
and $k \ge \frac{2(2-\epsi)\ell^2}{e\epsi\ell-4}$,
Alg.~\ref{alg:gdtwo} (Alg.~\ref{alg:tgtwo}) returns a set $G_\ell$ with $\oh{n\log(k)/\epsi^2}$ queries
such that $\ff{G_\ell} \ge \left(1/e-\epsi\right) \ff{O}$.
\end{restatable}

\begin{algorithm*}[ht]
\Fn{\ptgone($f, m, \ell, \tau_{\min}, \epsi$)}{
    \KwIn{evaluation oracle $f:2^{\uni} \to \reals$, 
    constraint $m$, constant $\ell$,
    minimum threshold value $\tau_{\min}$, error $\epsi$}
    \Init{$M\gets \max_{x\in \uni} \marge{\{x\}}{\emptyset}$, $I = [\ell]$, $m_0 \gets m$, $A_j\gets A_j'\gets \emptyset$, 
    $\tau_j\gets M$, $V_j \gets \uni, \forall j \in [\ell]$ }
    % \tcp*[h]{$V_j$ contains all good elements outside solutions}
    \While{$I \neq \emptyset$ and $m_0 > 0$}{
        \For(\textcolor{blue}{\tcc*[h]{Update candidate sets with high-quality elements}}){$j\in I$ in parallel\label{line:tgone-update-for-begin}}{
            $\{V_j, \tau_j\} \gets \update(f_{A_j}\restriction_{\uni\setminus\left(\bigcup_{l\in [\ell] A_l}\right)}, V_j, \tau_j, \epsi)$\label{line:tgone-update}\;
            \lIf{$\tau_j < \tau_{\min}$}{ $I\gets I-j$}\label{line:tgone-update-for-end}
        }
        \If(\textcolor{blue}{\tcc*[h]{Add 1 element to each solution alternately}}){$\exists i\in I$ \st $|V_i|< 2\ell$}{
            \For{$j\in I$ in sequence\label{line:tgone-for-begin}\label{line:pig-if-start}}{
                \lIf{$|V_j| = 0$}{
                    $\{V_j, \tau_j\} \gets \update(f_{A_j}\restriction_{\uni\setminus\left(\bigcup_{l\in [\ell] A_l}\right)}, V_j, \tau_j, \epsi)$\label{line:tgone-update-2}
                }
                \lIf{$\tau_j < \tau_{\min}$}{ $I\gets I-j$}
                \Else{
                    $x_j\gets $ randomly select one element from $V_j$ \label{line:tgone-select}\;
                    $A_j\gets A_j+x_j, A_j'\gets A_j'+x_j$\label{line:tgone-update-A}\;
                    $V_l\gets V_l-x_j, \forall l\in [\ell]$\;
                }
            }\label{line:tgone-for-end}
            $m_0\gets m_0-1$\;\label{line:pig-if-end}
        }
        \Else(\textcolor{blue}{\tcc*[h]{Add an equal number of elements to each solution}}){
            $\{\mathcal V_l: l\in I\} \gets \dist(\{V_l: l\in I\})$\label{line:tgone-dist}\label{line:pig-else-start} \tcp*[h]{Create pairwise disjoint candidate sets}\;
            $s \gets \min \{m_0, \min\{|\mathcal V_l|: l\in I\}\}$\;
            \lFor{$j\in I$ in parallel}{
                $i^*_j, B_j \gets \prefix(f_{A_j}, \mathcal V_j, s, \tau_j, \epsi)$\label{line:tgone-prefix}
            }
            $i^*\gets \min\{i^*_1, \ldots, i^*_\ell\}$ \label{line:tgone-index}\;
            \For(\textcolor{blue}{\tcc*[h]{Add $i^*$ high-quality elements to each set}}){$j\gets 1$ to $\ell$ in parallel \label{line:tgone-add-begin}}{
                $S_j\gets$ select $i^*$ elements from $\mathcal V_j[1:i^*_l]$ in three passes, prioritizing $B_j[i] = \textbf{true}$, then $B_j[i] = \textbf{none}$, and finally $B_j[i] = \textbf{false}$
                \label{line:tgone-subset}\;
                $S_j'\gets S_j\cap \left\{v_i \in \mathcal V_j: B_j[i]\neq \textbf{false}\right\}$\label{line:tgone-subset-2}\;
                $A_j\gets A_j\cup S_j, A_j'\gets A_j'\cup S_j'$\label{line:tgone-update-A-2}\;
            }
            $m_0\gets m_0-i*$ \label{line:tgone-update-size}\;\label{line:pig-else-end}
        }
    }
    \Return{$\{A_l': l\in [\ell]\}$}
}
\caption{A highly parallelized algorithm with $\oh{\ell^2\epsi^{-2} \log(n)\log\left(\frac{M}{\tau_{\min}}\right)}$ adaptivity
and $\oh{\ell^3\epsi^{-2} n \log(n)\log\left(\frac{M}{\tau_{\min}}\right)}$ query complexity. Subroutines \update, \dist and \prefix are provided in Appendix~\ref{apx:subroutine}.}
\label{alg:ptgone}
\end{algorithm*}

\section{Sublinear Adaptive Algorithms}\label{sec:ptg}
In this section, we present the main subroutine for our parallel algorithms, 
\ptgone (\ptgoneshort, Alg.~\ref{alg:ptgone}).
A single execution of \ptgoneshort achieves an 
approximation ratio of $1/4-\epsi$ with high probability,
while repeatedly running \ptgoneshort, as in \ptgtwo (\ptgtwoshort, Alg.~\ref{alg:ptgtwo} in 
Appendix~\ref{apx:ptgtwo}), 
guarantees a randomized approximation ratio of $1/e-\epsi$.
Below, we outline the theoretical guarantees,
with the detailed analysis provided in Appendix~\ref{apx:ptg}.
The remainder of this section is dedicated to 
explaining the intuition behind the algorithm.
\begin{restatable}{theorem}{thmptgone}\label{thm:ptgone}
With input $(f, k, 2, \frac{\epsi M}{k}, \epsi)$,
where $M = \max_{x\in \uni} \ff{x}$,
\ptgoneshort (Alg.~\ref{alg:ptgone}) returns $\{A_1', A_2'\}$
with $\oh{\epsi^{-4}\log(n)\log(k)}$ adaptive rounds and $\oh{\epsi^{-5}n\log(n)\log(k)}$ queries with a probability of $1-1/n$.
It satisfies that $\max\{\ff{A_1'}, \ff{A_2'}\}\ge (1/4-\epsi)\ff{O}$.
\end{restatable}

\begin{restatable}{theorem}{thmptgtwo}\label{thm:ptgtwo}
With input $(f, k, \epsi)$ such that
$\ell = \oh{\epsi^{-1}}\ge \frac{4}{e\epsi}$
and $k \ge \frac{(2-\epsi)^2\ell}{e\epsi\ell-4}$,
\ptgtwoshort (Alg.~\ref{alg:ptgtwo}) returns $G$
such that $\ex{\ff{G}} \ge (1/e-\epsi)\ff{O}$
with $\oh{\epsi^{-5}\log(n)\log(k)}$ adaptive rounds and $\oh{\epsi^{-6}n\log(n)\log(k)}$ queries with a probability of $1-\oh{1/(\epsi n)}$.
\end{restatable}
\ptgoneshort begins by initializing $\ell$ empty solutions $\{A_j: j\in [\ell]\}$,
with corresponding threshold values set to 
$M = \max_{x\in \uni}\marge{x}{\emptyset}$. 
Candidate sets $\{V_j: j\in [\ell]\}$ are maintained
and updated using \update
(Alg.~\ref{alg:update} in Appendix~\ref{apx:subroutine})
to filter out elements with marginal gain
less than $\tau_j$ for each solution $A_j$ 
at the start of every iteration 
in Lines~\ref{line:tgone-update-for-begin}-\ref{line:tgone-update-for-end}.
Two cases are then considered.
If any candidate set satisfies $|V_j| < 2\ell$,
$1$ element is added to each set alternately.
Otherwise, if all solutions have sufficient candidates,
pairwise disjoint candidate sets are then constructed in Line~\ref{line:tgone-dist} using \dist (Alg.~\ref{alg:dist} in Appendix~\ref{apx:subroutine}).
By executing a threshold sampling procedure \prefix (Alg.~\ref{alg:prefix} in Appendix~\ref{apx:subroutine}) on Line~\ref{line:tgone-prefix},
a block of elements of equal size is carefully selected and added to each solution.


\ptgoneshort is built upon the descending threshold greedy-based algorithms described in Section~\ref{sec:tg} to ensure nearly linear query complexity.
Additionally, it incorporates the threshold sampling algorithm, \ts~\citep{Chen2024},
to achieve sublinear adaptivity.
% It works as follows.
% For $j$-th solution, a pair of subsets $\{A_j, A_j'\}$ are maintained.
% The first set $A_j$ is responsible for filtering \textit{bad} elements that
% has marginal gain less than the current threshold value $\tau_j$,
% and the second set $A_j'$ is the actual solution such that
% $A_j'\subseteq A_j$ and $\ff{A_j'} \ge \ff{A_j}$.
% Furthermore, a candidate set $V_j$ is also maintained for each solution,
% where it contains all \textit{good} elements with marginal gain larger than $\tau_j$.
%
% To parallelize those algorithms, the following key properties must be maintained.
% \begin{enumerate}
%     \item Elements are added in an alternating manner.
%     \item Multiple elements are added to the solutions within constant adaptive round to achieve sublinear adaptivity.
%     \item Most of the elements added should contribute enough to the solutions.
% \end{enumerate}
% Next, we will introduce how \ptgoneshort achieve those goals.
To accomplish these goals,
several critical properties must be preserved throughout the process.

\subsection{Maintaining Alternating Additions during Parallel Algorithms}
\label{sec:ptg-alter}
This property is crucial to interlaced greedy variants 
introduced in prior sections. 
Below, we demonstrate that \ptgoneshort preserves this property.

During an iteration of the while loop in Alg.~\ref{alg:ptgone},
after updating the candidate sets in Lines~\ref{line:tgone-update-for-begin}-\ref{line:tgone-update-for-end},
two scenarios arise. 
In the first scenario, there exists a candidate set satisfies $|V_j| < 2\ell$,
Lines~\ref{line:pig-if-start}-\ref{line:pig-if-end} are executed,
and elements are appended to solutions one at a time in turn.
In this case, the alternating property is maintained immediately.

In the second scenario, Lines~\ref{line:pig-else-start}-\ref{line:pig-else-end} are executed.
Here, a block of elements with average marginal gain approximately exceeding $\tau_j$
is added to each solution $A_j$.
These blocks $S_j$ are of the same size $i^*$ (Line~\ref{line:tgone-subset})
selected from $\mathcal V_j$,
and guaranteed to be pairwise disjoint by 
Lemma~\ref{lemma:dist} (for \dist, Alg.~\ref{alg:dist} in Appendix~\ref{apx:subroutine}).
Crucially, threshold values $\tau_j$ 
remain unchanged during this step.
While a small fraction of elements in the blocks may have marginal gains below $\tau_j$,
the process retains the alternating property at a structural level: 
the uniform block sizes, and disjoint selection mimic the alternating addition of elements, even when processing multiple elements in parallel.

\subsection{Ensuring Sublinear Adaptivity Through Threshold Sampling}
The core mechanism for achieving sublinear adaptivity 
lies in iteratively reducing the pool of high-quality candidate elements 
(those with marginal gains above the threshold) 
by a constant factor within a constant number of adaptive rounds. 
This progressive reduction ensures efficient convergence.

At every iteration of the while loop in Alg.~\ref{alg:ptgone}, 
after updating the candidate sets,
if there exists $V_j$ such that $|V_j| < 2\ell$,
the following occurs after the for loop
(Lines~\ref{line:tgone-for-begin}-\ref{line:tgone-for-end}):
If the threshold $\tau_j$ remains unchanged,
one element from $V_j$ is added to the solution.
If $\tau_j$ is reduced,
$V_j$ is repopulated with high-quality elements.
This implies that a $1/(2\ell)$-fraction of $V_j$ is filtered out
after per iteration,
or even further, it becomes empty and
the threshold value is updated.

In the second case, where Lines~\ref{line:pig-else-start}-\ref{line:pig-else-end} are executed,
the algorithm employs \prefix (Alg.~\ref{alg:prefix} in Appendix~\ref{apx:subroutine}) in Line~\ref{line:tgone-prefix},
inspired by \ts~\citep{Chen2024}.
% At each iteration of \ts, a \textit{good prefix} is selected
% from the candidate set,
% consisting of elements outside the current solution
% that have marginal gains greater than
% the threshold value.
% After each addition to the solution, a constant fraction of elements in the candidate set
% is filtered out with constant probability, 
% based on the given threshold value,
% thus ensuring sublinear adaptivity.
% 
% In Alg.~\ref{alg:ptgone},
% we apply the same prefix selection step from \ts as
% \prefix (Alg.~\ref{alg:prefix} in Appendix~\ref{apx:subroutine})
% in Line~\ref{line:tgone-prefix}.
% If the prefix sizes returned for each solution are different,
% their minimum value $i^*$ are chosen and
% subsets of size $i^*$ are added to each solution 
% (Line~\ref{line:tgone-index}-\ref{line:tgone-update-A-2}).
Then, the smallest prefix size $i^*$ is selected in Line~\ref{line:tgone-index}.
For the solution where its corresponding call to \prefix returns $i^*$,
the entire prefix with size $i^*$ is added to it.
This ensures that a constant fraction of elements in $\mathcal V_j$
can be filtered out by Lemma~\ref{lemma:prefix-prob} in Appendix~\ref{apx:subroutine} with probability at least $1/2$.
Moreover, Lemma~\ref{lemma:dist} in Appendix~\ref{apx:subroutine}
guarantees that $|\mathcal V_j| \ge \frac{1}{\ell}|V_j|$
for each candidate set.
As a result, with constant probability,
at least one candidate set will filter out a constant fraction
of the elements.

\subsection{Ensuring Most Added Elements Significantly Contribute to the Solutions}
In \ts, the selection of a \textit{good prefix} inherently ensures this property immediately.
However, when interlacing $\ell$ threshold sampling processes,
prefix sizes selected in Line~\ref{line:tgone-prefix} by each solution may vary.
To preserve the alternating addition property introduced in Section~\ref{sec:ptg-alter},
subsets of equal size are selected instead of variable-length good prefixes.
This raises the question: \textit{How can a good subset be derived from a good prefix?}
The solution lies in Line~\ref{line:tgone-subset} of Alg.~\ref{alg:ptgone}.

For any $j \in I$, if $i_j^* = i^*$, $S_j$ is directly the good prefix $\mathcal V_j[1:i_j^*]$.
Otherwise, if $i_j^* \ge i^*$,
$i^*$ elements are selected from $\mathcal V_j[1:i_j^*]$ in three sequential passes until the size limit is reached:

\textbf{First pass}: Iterate through the prefix,
selecting those with marginal gains strictly greater than
$\tau_j$ (marked as \text{true} in $B_j$).

\textbf{Second pass}: 
From the remaining elements in the prefix, select those 
with marginal gains between $0$ and $\tau_j$
(marked as \text{none} in $B_j$).

\textbf{Third pass}: Fill any remaining slots with remaining elements from the prefix (marked as \text{false} in $B_j$).

This approach, combined with submodularity, 
ensures that any element marked as \textbf{true} in the selected subset has a marginal gain greater than $\tau_j$.
By prioritizing the addition of these \textbf{true} elements,
the selected subset remain high-quality while adhering to the alternating
addition framework.











\section{Experiments}\label{sec_exp}
\vspace{-0.2cm}
Our experiments investigate three key research questions:

\noindent\emph{Q1: Method Effectiveness.} How does our approach enhance performance across both in-domain and out-of-domain mathematical benchmarks compared to existing math LLMs?

\noindent\emph{Q2: Baseline Comparisons.} How does our method compare to standard RL and SFT baselines in terms of training efficiency and exploration patterns?

\noindent\emph{Q3: AutoCode Analysis.} What strategies does the model learn for code integration, and how do these strategies contribute to performance gains?

\noindent\textbf{Datasets and Benchmarks.} Our method only requires a query set for training. We collect public available queries from MATH~\citep{math} and Numina~\cite{numina}, and sample \(7K\) queries based on difficulties. We upload the collected data to the annonymous repo. For evaluation, we employ: GSM8k~\citep{gsm8k}, MATH500~\citep{math}, GaokaoMath2023~\citep{mario}, OlympiadBench~\citep{olympiad}, the American Invitational Mathematics Examination (AIME24), and the American
Mathematics Competitions (AMC23). This benchmark suite spans elementary to Olympiad-level mathematics. We adopt Pass@1 accuracy~\citep{pass1, dsr1} as our primary metric, using evaluation scripts from DeepseekMath~\citep{dsmath} and Qwen2Math~\citep{yang2024qwen2}. For competition-level benchmarks (AIME/AMC), we use 64 samples with temperature 0.6 following Deepseek R1 protocols.

% Please add the following required packages to your document preamble:

% Beamer presentation requires \usepackage{colortbl} instead of \usepackage[table,xcdraw]{xcolor}
\begin{table*}[t]
\centering
\caption{Main Results. Eurus-2-7B-PRIME demonstrates the best reasoning ability.}
\label{tab:main_results}
\resizebox{\textwidth}{!}{
\begin{tabular}{lcccccc}
\toprule
\textbf{Model}                     & \textbf{AIME 2024}                           & \textbf{MATH-500} & \textbf{AMC}          & \textbf{Minerva Math} & \textbf{OlympiadBench} & \textbf{Avg.}          \\ \midrule
\textbf{GPT-4o}                    & 9.3                                          & 76.4              & 45.8                  & 36.8                  & \textbf{43.3}          & 43.3                   \\
\textbf{Llama-3.1-70B-Instruct}    & 16.7                                         & 64.6              & 30.1                  & 35.3                  & 31.9                   & 35.7                   \\
\textbf{Qwen-2.5-Math-7B-Instruct} & 13.3                                         & \textbf{79.8}     & 50.6                  & 34.6                  & 40.7                   & 43.8                   \\
\textbf{Eurus-2-7B-SFT}            & 3.3                                          & 65.1              & 30.1                  & 32.7                  & 29.8                   & 32.2                   \\
\textbf{Eurus-2-7B-PRIME}          & \textbf{26.7 {\color[HTML]{009901} (+23.3)}} & 79.2 {\color[HTML]{009901}(+14.1)}      & \textbf{57.8 {\color[HTML]{009901}(+27.7)}} & \textbf{38.6 {\color[HTML]{009901}(+5.9)}}  & 42.1 {\color[HTML]{009901}(+12.3) }          & \textbf{48.9 {\color[HTML]{009901}(+ 16.7)}} \\ \bottomrule
\end{tabular}
}
\end{table*}
\noindent\textbf{Baselines and Implementation.} 
We compare against three model categories: \begin{itemize}[leftmargin=0.5cm,itemsep=0pt,parsep=0pt]
\item Proprietary models: o1~\cite{o1}, GPT-4~\citep{gpt4} and Claude~\citep{claude}
\item Recent math-specialized LMs: NuminaMath~\citep{numina}, Mathstral~\citep{mathstral}, Mammoth~\citep{mammoth}, ToRA~\citep{tora}, DartMath~\cite{tong2024dartmath}. We do not compare with models that rely on test-time scaling, such as MCTS or long CoT. 
\item Foundation models enhanced with our method: Qwen2Math~\citep{yang2024qwen2}, DeepseekMath~\citep{dsmath} and Qwen-2.5~\cite{qwen25}.
\end{itemize}

Our implementation uses \( K = 8 \) rollouts per query (temperature=1.0, top-p=0.9). Training completes in about 10 hours on \(8\times\) A100 (80GB) GPUs across three epochs of 7K queries. We release code, models and data via an \href{https://anonymous.4open.science/r/AnnonySubmission-0C62}{anonymous repository}.

% \vspace{-0.1cm}
\subsection{Main Results}\label{sec_main}
Notably, we observe a minimum performance gain of 11\% on the MATH500 benchmark, escalating to an impressive 9.4\% absolute improvement on the highly challenging AIME benchmark.  Across in-domain benchmarks, our method yields an average improvement of 8.9\%, and for out-of-domain benchmarks, we achieve a substantial average gain of 6.98\%. These results  validate the effectiveness of our approach across model families and problem difficulty levels.  

\subsection{Ablation Study}\label{sec_ablation}
We conduct three primary analyses: (a) comparison with standard RL and SFT baselines to validate our method's effectiveness in facilitating exploration, (b) visualization of exploration patterns to reveal limitations in the standard RL paradigam, and (c) behavioral analysis of code integration strategies. These analyses collectively demonstrate our method's benefits in facilitating guided exploration and explains how it improves performance.

\begin{figure*}[t]
    \centering
    \includegraphics[width=0.95\linewidth]{figs/rl_curves.pdf}
    \caption{ \small \textbf{Training Efficiency and Convergence.} We benchmark the learning dynamics of our approach against three two training paradigms: supervised fine-tuning and reinforcement learning (RL). The Pass@1 accuracy is evaluated on an held-out dev-set. We use Qwen-2.5-Base as the base model. SFT is conducted using collected public data~\cite{openmath, mammoth}. The dashed lines indicate asymptotic performance. }\label{fig_training_efficiency}
    % \begin{minipage}{0.47\textwidth}
    %     \centering
    %     \begin{subfigure}[b]{1.0\textwidth}
    %         \centering
    %         \includegraphics[width=0.95\linewidth]{figs/abl_qwen_curve.pdf}
    %         % \caption{Top right image}
    %     \end{subfigure}
    %     % \vskip -0.3\baselineskip % Add vertical space between subfigures
    %     \begin{subfigure}[b]{1.0\textwidth}
    %         \centering
    %         \includegraphics[width=1.\linewidth]{figs/abl_deepseek_curve.pdf}
    %         % \caption{Bottom right image}
    %     \end{subfigure}
    %     \caption{ \small \textbf{Performance Convergence. } Experiments are conducted based on Qwen2Math (Top) and DeepseekMath (Bottom). AutoCode achieves higher accuracy with sustained improvement, while standard RL converge to sub-optimal solutions. }\label{fig_training_effici}
    % \end{minipage}
    % \hfill
    % \begin{minipage}{0.47\textwidth}
    %     \centering
    %     % \vspace{-0.3cm}
    %     % \hspace{-1.6cm}
    %     \includegraphics[width=1.\linewidth]{figs/abl_strategies.pdf}
    %     \caption{\small \textbf{Analysis of the Learned Strategies.} Correct Responses are classified based on their alignment to the oracle selection, namely, \emph{StrictAlign}, \emph{AllowCode} and \emph{MisAlign}. We show how different categories of alignment contribute to the accuracy in the stacked bars, and include the overall StrictAlign rate in the separate orange bar.} \label{fig_learned_strategies}
    % \end{minipage}
% \vspace{-0.3cm}
\end{figure*}
\noindent\textbf{Training Efficiency.} We evaluated the learning dynamics of our approach in direct comparison to three established training paradigms:
\begin{itemize}[leftmargin=0.5cm,itemsep=0pt,parsep=0pt]
\item \emph{Base+RL}:  On-policy Reinforcement Learning (RL) initialized from a base model without Supervised Fine-Tuning (SFT). This follows the methodology of DeepSeek R1, designed to isolate and assess the pure effects of RL training.
\item \emph{SFT}: Supervised Fine-Tuning, the prevailing training paradigm widely adopted in current tool-integrated math Language Models (LMs).
\item \emph{SFT+RL}: Standard RL applied after SFT, serving as a conventional baseline for evaluating our EM-based RL method.
\end{itemize}

From the figure, we make the following key observations: 


\begin{itemize}[leftmargin=0.5cm,itemsep=0pt,parsep=0pt]
   \item  While Reinforcement Learning directly from the base model (\emph{Base+RL}) exhibits consistent performance improvement, its training efficiency is lower than training paradigms incorporating SFT.  In addition, the model rarely explores code-integrated solutions, with the code invocation rate below 5\%. This strongly suggest that \emph{reinforcement learning tool-usage behavior from scratch is inherently inefficient}.
    \item SFT effectively provides a strong initialization point, but \emph{SFT alone exhibits limited asymptotic performance}. This suggests that SFT lacks the capacity to adapt and optimize beyond the scope of the expert demonstrations, thereby limiting further improvement. 
    \item Standard RL applied after SFT shows initial further improvement but subsequently plateaus, \emph{even after an extended training stage}.  This suggests \emph{the exploration-exploitation dilemma when applying RL for LLM post-training}: standard RL with vanilla rollout exploration tends to exploit local optima and insufficiently explores the combinatorial code-integrated trajectories.
\end{itemize}

To further substantiate the exploration limitations inherent in the conventional \emph{SFT+RL} paradigm, we present a visualization of the exploration patterns. We partitioned the model-generated responses during self-exploration into three distinct training phases and analyzed the statistical distribution of code invocation rates across queries as the model's policy evolved throughout training. As depicted in Figure~\ref{fig_visualize_explore}, the distribution of code invocation progressively concentrates towards the extremes – either minimal or maximal code use – indicating the model's growing tendency to exploit its local policy neighborhood. This exploitation manifests as a focus on refining established code-triggering decisions, rather than engaging in broader exploration of alternative approaches.



\begin{figure}[t]
    \centering % Center the figure
    \resizebox{1.\linewidth}{!}{\includegraphics[width=\linewidth]{figs/visualize_explore.pdf} }% Include the figure
    \vspace{-0.2cm}
    \caption{\small \textbf{Visualization of Exploration in the SFT+RL paradigm.} \small The distribution of code invocation rates \emph{across queries} to visualize policy's exploration of code-integrated trajectories. Without external guidance, LLM tends to exploit its local policy neighborhood, concentrating code usage toward extremes as training phase evolves. } \vspace{-0.2cm}
    \label{fig_visualize_explore} 
\end{figure}
These empirical observations lend strong support to our assertion that standard RL methods are susceptible to premature exploitation of the local policy space when learning AutoCode strategies. In sharp contrast, our proposed EM method facilitates a more guided exploration by sub-sampling trajectories according to the reference strategy (Sec.~\ref{sec_impl}). This enables continuous performance (evidenced in Sec.~\ref{sec_main}) and mitigating the risk of converging to suboptimal local optima (Fig.~\ref{fig_training_efficiency}).



\begin{figure}[t]
    \centering % Center the figure
    \resizebox{1.\linewidth}{!}{\includegraphics[width=\linewidth]{figs/learned_behavior.pdf}} % Include the figure
    \caption{\small \textbf{Analysis of AutoCode Strategies. }\small We compare AutoCode performance against scenarios where models explicitly prompted to utilize code or CoT, and consider the union of solved queries as the bound for AutoCode performance. Existing models show inferior AutoCode performance than explicit instructed, with their AutoCode strategies close to random (50\%). Our approach consistently improves AutoCode performance, with AutoCode selection accuracy near 90\%.  } 
    \label{fig_learned_behavior} 
\end{figure}
\noindent\textbf{Analysis on Code Integration Behaviors.}
We investigated the properties of the learned code integration strategies to gain deeper insights into the mechanisms behind our method's performance gains. Our central hypothesis posits that optimal code integration unlocks synergistic performance benefits by effectively combining the strengths of CoT and code executions.  This synergy presents a "free lunch" scenario: a well-learned metacognitive tool-usage strategy can elevate overall performance, provided the model demonstrates competence in solving \emph{distinct} subsets of queries using either CoT or code execution.

To empirically validate this "free lunch" principle and demonstrate the superiority of our approach in realizing it, we benchmarked our model against baselines that inherently support both code execution and Chain-of-Thought (CoT) reasoning: GPT-4, Mammoth-70B, and DeepseekMath-Instruct-7B. Our analysis evaluated the model's autonomous decision to invoke code when not explicitly instructed on which strategy to employ. We compared this "AutoCode" performance against scenarios where models were explicitly prompted to utilize either code or CoT reasoning. We also considered the theoretical "free lunch" upper bound – the accuracy achieved by combining the successful predictions from either strategy (i.e., taking the union of queries solved by CoT or code).

As visually presented in Figure~\ref{fig_learned_behavior}, existing baseline models exhibit inferior performance in AutoCode mode compared to scenarios where code invocation is explicitly prompted, e.g., DeepseekMath-Instruct-7B shows a degradation of 11.54\% in AutoCode mode. This suggests that their AutoCode strategies are often suboptimal, performing closer to random selection between CoT and code (selection accuracy near 50\%), resulting in AutoCode falling between the performance of explicitly triggered CoT and code. In contrast, our models learn more effective code integration strategies.  AutoCode4Math-Qwen2.5, for example, improves upon explicitly code-triggered performance by 7\%, indicating a true synergistic integration of reasoning and code execution.


To quantify the effectiveness of these learned "AutoCode" strategies, we calculated the CoT/code selection accuracy. We used the outcome of explicit instruction (i.e., performance when explicitly prompted for CoT or code) as a proxy for the ground-truth optimal method selection.  Our model achieves a selection accuracy of 89.53\%, showcasing the high efficacy of the learned code integration strategy.
\section{Conclusion}
The state-of-the-art $1/e$ approximation ratio for sublinear adaptive algorithms 
is achieved by a continuous algorithm~\citep{Ene2020a}.
For combinatorial algorithms with sublinear adaptivity,
the best-known result is a randomized $1/4$ approximation ratio~\citep{Cui2023}.
In this work, we present a sublinear adaptive approximation algorithm 
achieving $1/4-\epsi$ approximation ratio with high probability,
and further improve this ratio achieved to $1/e$, 
breaking the barrier between continuous and combinatorial algorithms. 
These advancements are made by a novel blending analysis technique,
which offers a fresh perspective for analyzing greedy-based algorithms.

\bibliography{main}
\bibliographystyle{icml2025}

\newpage
\appendix
\onecolumn
\section{Technical Lemmata}\label{apx:tech}
% \begin{lemma}[\citep{feige2011maximizing}]\label{lemma:OneRandomSet}
% Let $f:2^{\mathcal{N}} \to \reals$ be submodular. 
% Denote by $A(p)$ a random subset of $A$ where each element
% appears with probability $p$ (not necessarily independently).
% Then
% \[\ex{f(A(p))} \ge (1-p)\cdot f(\emptyset) + p\cdot f(A).\]
% \end{lemma}

\begin{lemma}\label{lemma:val-inq}
    \begin{align*}
        & 1-\frac{1}{x}\le \log(x) \le x-1, &\forall x>0\\
        & 1-\frac{1}{x+1}\ge e^{-\frac{1}{x}} , &\forall x\in \mathbb{R}\\
        & (1-x)^{y-1}\ge e^{-xy}, &\forall xy \le 1
    \end{align*}
\end{lemma}

\begin{lemma}[Chernoff bounds \citep{mitzenmacher2017probability}]\label{lemma:chernoff}
    Suppose $X_1$, ... , $X_n$ are independent binary random variables such that 
    $\prob{X_i = 1} = p_i$. Let $\mu = \sum_{i=1}^n p_i$, and 
    $X = \sum_{i=1}^n X_i$. Then for any $\delta \geq 0$, we have
    \begin{align}
        \prob{X \ge (1+\delta)\mu} \le e^{-\frac{\delta^2 \mu}{2+\delta}}.
    \end{align}
    Moreover, for any $0 \leq \delta \leq 1$, we have
    \begin{align}
        \prob{X \le (1-\delta)\mu} \le e^{-\frac{\delta^2 \mu}{2}}.
    \end{align}
\end{lemma}
\begin{lemma}[\citet{Chen2021}] \label{lemma:indep}
    Suppose there is a sequence of $n$ Bernoulli trials:
    $X_1, X_2, \ldots, X_n,$
    where the success probability of $X_i$
    depends on the results of
    the preceding trials $X_1, \ldots, X_{i-1}$.
    Suppose it holds that $$\prob{X_i = 1 | X_1 = x_1, X_2 = x_2, \ldots, X_{i-1} = x_{i-1} } \ge \eta,$$ where $\eta > 0$ is a constant and $x_1,\ldots,x_{i-1}$ are arbitrary.
  
    Then, if $Y_1,\ldots, Y_n$ are independent Bernoulli trials, each with probability $\eta$ of
    success, then $$\prob {\sum_{i = 1}^n X_i \le b } \le \prob{\sum_{i = 1}^n Y_i \le b }, $$
    where $b$ is an arbitrary integer.
  
    Moreover, let $A$ be the first occurrence of success in sequence $X_i$.
    Then, $$\ex{A} \le 1/\eta.$$
\end{lemma}

\section{Propositions on Submodularity} \label{apx:prop}

\begin{proposition}\label{prop:sum-marge}
Let $\{A_1, A_2, \ldots, A_m\}$ be $m$ pairwise disjoint subsets of $\uni$,
and $B\in \uni$.
For any submodular function $f: 2^\uni \to \reals$,
it holds that
\begin{align*}
\text{1) }&\sum\limits_{i\in [m]} \marge{A_i}{B}\ge \marge{\bigcup\limits_{i\in [m]}A_i}{B},\\
\text{2) }&\sum_{i\in [m]}\ff{B\cup A_i}\ge (m-1)\ff{B}.
\end{align*}
\end{proposition}

\begin{proposition}\label{prop:subset}
Let $A=\{a_1, \ldots, a_m\}$ and $A_i = \{a_1, \ldots, a_i\}$ for all $i\in [m]$.
For any submodular function $f: 2^\uni \to \reals$,
let $B = \argmax\limits_{B\subseteq A, |B| = m-1}\sum\limits_{a_i \in B}\marge{a_i}{A_{i-1}}$.
It holds that
$\ff{B} \ge \left(1-\frac{1}{m}\right)\ff{A}$.
\end{proposition}

\begin{proposition}\label{prop:dif-opt}
For any submodular function $f: 2^\uni \to \reals$,
let $O_1 = \argmax_{S\subseteq \uni, |S|\le k_1} \ff{S}$ and 
$O_2 = \argmax_{S\subseteq \uni, |S|\le k_2} \ff{S}$.
It holds that 
\[\ff{O_1} \ge \frac{k_1}{k_2}\ff{O_2}.\]
\end{proposition}
\section{Pseudocode and Theoretical Guarantees of \ig~\citep{DBLP:conf/nips/Kuhnle19} and \itg~\citep{DBLP:conf/kdd/ChenK23}}\label{apx:pseudocode}
In this section, we provide the original greedy version of 
\ig~\citep{DBLP:conf/nips/Kuhnle19} and \itg~\citep{DBLP:conf/kdd/ChenK23}
with their theoretical guarantees.
\begin{algorithm}[ht]
    \KwIn{evaluation oracle $f:2^{\uni} \to \reals$, constraint $k$}
    \KwOut{$C\subseteq \uni$, such that $|C| \le k$}
    $A_0\gets B_0 \gets \emptyset$\;
    \For{$i\gets 0$ to $k-1$}{
        $a_i\gets \argmax_{x\in \uni\setminus \left(A_i\cup B_i\right)} \marge{x}{A_i}$\;
        $A_{i+1}\gets A_i+a_i$\;
        $b_i\gets \argmax_{x\in \uni\setminus \left(A_{i+1}\cup B_i\right)} \marge{x}{B_i}$\; 
        $B_{i+1}\gets B_i+b_i$\;
    }
    $D_1 \gets E_1 \gets \{a_0\}$\;
    \For{$i\gets 1$ to $k-1$}{
        $d_i\gets \argmax_{x\in \uni\setminus \left(D_i\cup E_i\right)} \marge{x}{D_i}$\;
        $D_{i+1}\gets D_i+d_i$\;
        $e_i\gets \argmax_{x\in \uni\setminus \left(D_{i+1}\cup E_i\right)} \marge{x}{E_i}$\; 
        $E_{i+1}\gets E_i+e_i$\;
    }
    \Return{$C\gets \argmax\{\ff{A_i}, \ff{B_i}, \ff{D_i}, \ff{E_i} : i\in [k+1]\}$}
    \caption{$\ig(f,k)$: The \ig Algorithm~\citep{DBLP:conf/nips/Kuhnle19}}
    \label{alg:ig}
\end{algorithm}
\begin{theorem}
Let $f:2^{\uni} \to \reals$ be submodular, let $k\in \uni$,
let $O = \argmax_{|S|\le k} \ff{S}$,
and let $C = \ig(f, k)$. Then
\[\ff{C} \ge \ff{O}/4,\]
and \ig makes $\oh{kn}$ queries to $f$.
\end{theorem}

\begin{algorithm}[ht]
	\KwIn{oracle $f:2^{\uni} \to \reals$, constraint $k$, error $\epsi$}
    \Init{$\ell \gets \frac{2e}{\epsi}+1$, $G_0\gets \emptyset$}
    \For{$m\gets 1$ to $\ell$}{
    	$\{a_1, \ldots, a_\ell\}\gets$ top $\ell$ elements in $\uni\setminus G_{m-1}$
		with respect to marginal gains on $G_{m-1}$\;
		\For{$u\gets 0$ to $\ell$ in parallel}{
			\lIf{$u = 0$}{$A_{u, l}\gets G\cup \{a_l\}$, for all $1\le l\le \ell$}
			\lElse{$A_{u, l}\gets G\cup \{a_u\}$, for all $1\le l\le \ell$}
			\For{$j\gets 1$ to $k/\ell-1$}{
				\For{$i\gets 1$ to $\ell$}{
					$x_{j, i} \gets \argmax_{x\in \uni\setminus\left(\bigcup_{l=1}^{\ell}A_{u, l}\right)}\marge{x}{A_{u, i}}$\;
					$A_{u, i}\gets A_{u, i}\cup \{x_{j, i}\}$\;
				}
			}
		}
		$G_m\gets$ a random set in $\{A_{u, i}:1\le i\le \ell, 0\le u\le \ell\}$
    }
    \Return{$G_\ell$}\;
    \caption{$\itg(f,k,\epsi)$: An $1/(e+\epsi)$-approximation algorithm for \sm}
    \label{alg:itg}
\end{algorithm}
\begin{theorem}
Let $\epsi \ge 0$, and $(f, k)$ be an instance of \sm, 
with optimal solution value \opt.
Algorithm \itg outputs a set $G_\ell$ with $\oh{\epsi^{-2}kn}$ queries
such that $\opt \le (e+\epsi)\ex{\ff{G_\ell}}$ with 
probability $(\ell+1)^{-\ell}$, where $\ell = \frac{2e}{\epsi}+1$.
\end{theorem}


\section{Analysis of Alg.~\ref{alg:gdone} in Section~\ref{sec:greedy-1/4}}
\label{apx:greedy-1/4}
In this section, we provide the detailed analysis of approximation ratio
for Alg.~\ref{alg:gdone}.
\thmgdone*
\begin{proof}[Proof of Theorem~\ref{thm:gdone}]
% Consider adding dummy elements to the ground set.
% If $|O|$ is less than $k$, add dummy elements to $O$ until $|O| = k$.
% During each iteration of the for-loop,
% if no element in $\uni\setminus (A\cup B)$ has marginal gain greater than 0,
% a dummy element will be added to $A$,
% and similarly for $B$.
% Thus, after the for loop ends, we ensure that $|A| = |B| = k$.
% Moreover, let $A_0 = B_0 = \emptyset$,
% and $A_i$, $B_i$ be $A$, $B$ after $i$-th element is added, respectively.

\textbf{Notation.}
Let $a_i$ be the $i$-th element added to $A$,
and $A_i$ be the set containing the first $i$ elements of $A$.
Similarly, define $b_i$ and $B_i$ for the solution $B$.

Since the two solutions $A_k$ and $B_k$ are disjoint, by submodularity and non-negativity,
\begin{equation*}
\ff{O} \le \ff{O\cup A_k} + \ff{O\cup B_k}.
\end{equation*}
Let $i^* = \max\{i \in [k]: A_i\subseteq O\}$
and $j^* = \max\{j \in [k]: B_j\subseteq O\}$.
If either $i^* = k$ or $j^* = k$,
then $\ff{S} = \ff{O}$.
In the following, we consider $i^* < k$ and $j^* < k$
and discuss two cases of the relationship between $i^*$ and $j^*$ (Fig.~\ref{fig:gdone}).



\textbf{Case 1: $0\le i^*\le j^* < k$; Fig.~\ref{fig:gdone-1}.}
First, we bound $\ff{O\cup A_k}$. 
Consider the set $\tilde{O} = O\setminus \left(A_k \cup B_{i^*}\right)$.
Obviously, it holds that $|\tilde{O}|\le k-i^*$.
Then, order $\tilde{O}$ as $\{o_1, o_2, \ldots\}$ such that $o_i \not \in B_{i+i^*-1}$,
for all $1\le i \le |\tilde{O}|$.
Thus, by the greedy selection step in Line~\ref{line:gdone-greedy-A},
it holds that $\marge{a_{i+i^*}}{A_{i+i^*-1}}\ge \marge{o_i}{A_{i+i^*-1}}$
for all $1\le i \le |\tilde{O}|$.
Then,
\begin{align*}
\ff{O\cup A_k} - \ff{A_k} &\le \marge{B_{i^*}}{A_k} + \marge{\tilde{O}}{A_k}\\
&\le \ff{B_{i^*}} + \sum\limits_{i=1}^{|\tilde{O}|} \marge{o_i}{A_k}\\
&\le \ff{B_{i^*}} + \sum\limits_{i=1}^{|\tilde{O}|} \marge{o_i}{A_{i+i^*-1}}\\
&\le \ff{B_{i^*}} + \sum\limits_{i=i^*+1}^{k} \marge{a_i}{A_{i-1}}
= \ff{B_{i^*}} + \ff{A_k} - \ff{A_{i^*}},
\end{align*}
where the first three inequalities follow from submodularity;
and the last inequality follows from 
$\marge{a_{i+i^*}}{A_{i+i^*-1}}\ge \marge{o_i}{A_{i+i^*-1}}$
for all $1\le i \le |\tilde{O}|$,
and $\marge{a_i}{A_{i-1}}\ge 0$ for all $i \in [k]$.

Next, we bound $\ff{O\cup B_k}$.
Consider the set $\tilde{O} = O\setminus \left(A_{i^*} \cup B_{k}\right)$.
Obviously, it holds that $|\tilde{O}| \le k-i^*$.
Since $i^* = \max\{i \in [k]: A_i\subseteq O\}$,
we know that $a_{i^*+1} \not \in O$.
Thus, we can order $|\tilde{O}|$ as $\{o_1, o_2, \ldots\}$
such that $o_i \not \in A_{i+i^*}$ for all $1\le i \le |\tilde{O}|$.
Then, by the greedy selection step in Line~\ref{line:gdone-greedy-B},
it holds that $\marge{b_{i+i^*}}{B_{i+i^*-1}}\ge \marge{o_i}{B_{i+i^*-1}}$
for all $1\le i \le |\tilde{O}|$.
Following the analysis for $\ff{O\cup A}$,
we get
\begin{align*}
\ff{O\cup B_k} - \ff{B_k} &\le \marge{A_{i^*}}{B_k} + \marge{\tilde{O}}{B_k}\\
&\le \ff{A_{i^*}} + \sum\limits_{i=1}^{|\tilde{O}|} \marge{o_i}{B_k}\\
&\le \ff{A_{i^*}} + \sum\limits_{i=1}^{|\tilde{O}|} \marge{o_i}{B_{i+i^*-1}}\\
&\le \ff{A_{i^*}} + \sum\limits_{i=i^*+1}^{k} \marge{b_i}{B_{i-1}}
= \ff{A_{i^*}} + \ff{B_k} - \ff{B_{i^*}}.
\end{align*}

\textbf{Case 2: $0\le j^* < i^* < k$; Fig.~\ref{fig:gdone-2}.}
First, we bound $\ff{O\cup A_k}$.
Consider the set $\tilde{O} = O\setminus \left(A_{k}\cup B_{j^*}\right)$,
where $|\tilde{O}| \le k-j^*-1$.
By the definition of $j^*$,
we know that $b_{j^*+1}\not\in O$.
Thus, we can order $\tilde{O}$ as $\{o_1, o_2, \ldots\}$
such that $o_i\not \in B_{i+j^*}$ for all $1\le i\le |\tilde{O}|$.
Then, by the greedy selection step in Line~\ref{line:gdone-greedy-A},
it holds that $\marge{a_{i+j^*+1}}{A_{i+j^*}}\ge \marge{o_i}{A_{i+j^*}}$
for all $1\le i\le |\tilde{O}|$.
Following the above analysis, we get
\begin{align*}
\ff{O\cup A_k}-\ff{A_k} &\le \marge{B_{j^*}}{A_k}+\marge{\tilde{O}}{A_k}\\
&\le \ff{B_{j^*}} + \sum\limits_{i=1}^{|\tilde{O}|} \marge{o_i}{A_k}\\
&\le \ff{B_{j^*}} + \sum\limits_{i=1}^{|\tilde{O}|} \marge{o_i}{A_{i+j^*}}\\
&\le \ff{B_{j^*}} + \sum\limits_{i=j^*+2}^{k} \marge{a_i}{A_{i-1}}
= \ff{B_{j^*}} + \ff{A_k}-\ff{A_{j^*+1}}.
\end{align*}

Next, we bound $\ff{O\cup B_k}$.
Consider the set $\tilde{O} = O\setminus \left(A_{j^*+1} \cup B_k\right)$,
where $|\tilde{O}| \le k-j^*-1$.
Then, order $\tilde{O}$ as $\{o_1, o_2, \ldots\}$
such that $o_i\not \in A_{i+j^*}$ for all $1\le i\le |\tilde{O}|$.
By the greedy selection step in Line~\ref{line:gdone-greedy-B},
it holds that $\marge{b_{i+j^*}}{B_{i+j^*-1}}\ge \marge{o_i}{B_{i+j^*-1}}$.
Then,
\begin{align*}
\ff{O\cup B_k} - \ff{B_k}&\le \marge{A_{j^*+1}}{B_k} + \marge{\tilde{O}}{B_k}\\
&\le \ff{A_{j^*+1}}+\sum\limits_{i=1}^{|\tilde{O}|} \marge{o_i}{B_k}\\
&\le \ff{A_{j^*+1}}+\sum\limits_{i=1}^{|\tilde{O}|} \marge{o_i}{B_{i+j^*-1}}\\
&\le \ff{A_{j^*+1}}+\sum\limits_{i=j^*+1}^{k} \marge{b_i}{B_{i-1}} 
= \ff{A_{j^*+1}} + \ff{B_k}-\ff{B_{j^*}}.
\end{align*}

Therefore, in both cases, it holds that
\[\ff{O}\le \ff{O\cup A_k}+\ff{O\cup B_k}\le 2\left(\ff{A_k}+\ff{B_k}\right)\le 4\ff{S}.\]

\end{proof}

\section{Analysis of Alg.~\ref{alg:gdtwo} in Section~\ref{sec:greedy-1/e}}
\label{apx:greedy-1/e}
In what follows, we address the scenario where $k\, \text{mod}\,\ell > 0$
and Alg.~\ref{alg:gdtwo} returns a solution with size smaller than $k$
in Appendix~\ref{apx:gdtwo-k}.
We then provide proofs for the relevant Lemmata in Appendix~\ref{apx:gdtwo-lemma},
and conclude with an analysis of approximation ratio in Appendix~\ref{apx:gdtwo-approx}.
\subsection{Scenario where $k\, \text{mod}\,\ell > 0$.}\label{apx:gdtwo-k}
If $k\, \text{mod}\,\ell = 0$, 
the algorithm returns an approximation solution for a size constraint of 
$\ell\cdot\left\lfloor\frac{k}{\ell}\right\rfloor$.
By Proposition~\ref{prop:dif-opt}, it holds that 
\begin{equation}\label{inq:dif-opt}
\ff{O'}\ge \ell\cdot\left\lfloor\frac{k}{\ell}\right\rfloor / k \ff{O}
\ge \left(1-\frac{\ell}{k}\right)\ff{O},
O' = \argmax\limits_{S\subseteq \uni, |S|\le \ell\cdot\left\lfloor\frac{k}{\ell}\right\rfloor} \ff{S}.
\end{equation}

\subsection{Proofs of Lemmata for Theorem~\ref{thm:gdtwo}} \label{apx:gdtwo-lemma}
\textbf{Notation.} 
Let $G_{i-1}$ be $G$ at the start of $i$-th iteration in Alg.~\ref{alg:gdtwo},
$A_l$ be the set at the end of this iteration,
and $a_{l, j}$ be the $j$-th element added to $A_l$ during this iteration.

\lemmaparA*
\begin{proof}[Proof of Lemma~\ref{lemma:par-A}]
\begin{figure}
\centering
\includegraphics[width=0.45\linewidth]{fig/ITG.pdf}
\caption{This figure depicts the components of the solution sets $A_{l_1}$ and $A_{l_2}$.
A blue circle with a check mark represents an element in $O$,
while a red circle with a cross mark represents an element outside of $O$.
The grey rectangles indicate a sequence of consecutive elements in $O$.
The pink rectangles indicate the corresponding elements used to bound $\marge{O_{l_2}}{A_{l_1}}$ or $\marge{O_{l_1}}{A_{l_2}}$.
It is illustrated that $\marge{O_{l_1}}{A_{l_2}} + \marge{O_{l_2}}{A_{l_1}}\le \marge{A_{l_1}}{G_{i-1}} +  \marge{A_{l_2}}{G_{i-1}}$ under both cases.}
\label{fig:gdtwo}
\end{figure}
Recall that $A_{l, j}$ is $A_l$ after $j$-th element is added to $A_l$ at iteration $i$ of the outer for loop,
and $c_l^* = \max\left\{c\in [m]: A_{l, c}\setminus G_{i-1}\subseteq O_l \right\}$.

First, we prove that the first inequality holds.
For each $l\in [\ell]$, order the elements in $O_l$ as $\{o_1, o_2, \ldots\}$
such that $o_j \not \in A_{l, j-1}$ for any $1\le j \le |O_l|$.
Since each $o_j$ is either in $A_l$ or not in any solution set,
it remains in the candidate pool when $a_{l, j}$ is considered to be added to the solution.
Therefore, it holds that
\begin{equation}\label{inq:itg-1}
\marge{a_{l, j}}{A_{l, j-1}}\ge \marge{o_j}{A_{l, j-1}}.
\end{equation}
Then,
\begin{align*}
\marge{O_{l}}{A_{l}} \le \sum_{o_j\in O_l} \marge{o_j}{A_{l}} \le \sum_{o_j\in O_l} \marge{o_j}{A_{l, j-1}}\le \sum_{j=1}^{m}\marge{a_{l, j}}{A_{l, j-1}} = \marge{A_{l}}{G_{j-1}},
\end{align*}
where the first inequality follows from Proposition~\ref{prop:sum-marge},
the second inequality follows from submodularity,
and the last inequality follows from Inequality~\eqref{inq:itg-1}.

In the following, we prove that the second inequality holds.
For any $1\le l_1\le l_2\le \ell$,
we analyze two cases of the relationship between $c_{l_1}^* $ and $ c_{l_2}^*$ in the following.

% \textbf{Case 1: $c_{l_1}^* = c_{l_2}^* = m$.}
% Then, $O_{l_1} = A_{l_1}\setminus G_{i-1}$ and $O_{l_2} = A_{l_2}\setminus G_{i-1}$.
% By submodularity,
% \[\marge{O_{l_2}}{A_{l_1}} + \marge{O_{l_1}}{A_{l_2}} \le \marge{O_{l_2}}{G_{i-1}} + \marge{O_{l_1}}{G_{i-1}} = \marge{A_{l_1}}{G_{i-1}} + \marge{A_{l_2}}{G_{i-1}}.\]
% Therefore, the lemma holds in this case.

\textbf{Case 1: $c_{l_1}^* \le c_{l_2}^*$; left half part in Fig.~\ref{fig:gdtwo}.}

First, we bound $\marge{O_{l_1}}{A_{l_2}}$.
Since $c_{l_1}^* \le m$, we know that the $(c_{l_1}^*+1)$-th element in $A_{l_1}\setminus G_{i-1}$ is not in $O$.
So, we can order the elements in $O_{l_1}\setminus A_{l_1, c_{l_1}^*}$ as $\{o_1, o_2, \ldots\}$ such that $o_j \not \in A_{l_1, c_{l_1}^*+j+1}$.
(Refer to the gray block with a dotted edge in the top left corner of Fig.~\ref{fig:gdtwo} for $O_{l_1}$.)
Since each $o_j$ is either added to $A_{l_1}$ or not in any solution set,
it remains in the candidate pool when $a_{l_2, c_{l_1}^*+j}$ is considered to be added to $A_{l_2}$.
Therefore, it holds that 
\begin{equation}\label{inq:itg-case2-1}
\marge{a_{l_2, c_{l_1}^*+j}}{A_{l_2, c_{l_1}^*+j-1}} \ge \marge{o_j}{A_{l_2, c_{l_1}^*+j-1}}, \forall 1\le j\le m-c_{l_1}^*.
\end{equation}
Then,
\begin{align*}
\marge{O_{l_1}}{A_{l_2}} &\le \marge{A_{l_1, c_{l_1}^*}}{A_{l_2}}  + \sum_{o_j \in O_{l_1}\setminus A_{l_1, c_{l_1}^*}}\marge{o_j}{A_{l_2}} \tag{Proposition~\ref{prop:sum-marge}}\\
&\le \marge{A_{l_1, c_{l_1}^*}}{G_{i-1}} + \sum_{o_j \in O_{l_1}\setminus A_{l_1, c_{l_1}^*}}\marge{o_j}{A_{l_2, , c_{l_1}^*+j-1}} \tag{submodularity}\\
&\le \ff{A_{l_1, c_{l_1}^*}}-\ff{G_{i-1}} + \sum_{j = 1}^{m-c_{l_1}^*}\marge{a_{l_2, c_{l_1}^*+j}}{A_{l_2, , c_{l_1}^*+j-1}} \tag{Inequality~\eqref{inq:itg-case2-1}}\\
& \le \ff{A_{l_1, c_{l_1}^*}}-\ff{G_{i-1}} + \ff{A_{l_2}} - \ff{A_{l_2, c_{l_1}^*}} 
\end{align*}

Similarly, we bound $\marge{O_{l_2}}{A_{l_1}}$ below.
Order the elements in $O_{l_2}\setminus A_{l_2, c_{l_1}^*}$ as $\{o_1, o_2, \ldots\}$
such that $o_j \not \in A_{l_2, c_{l_1}^* + j}$.
(See the gray block with a dotted edge in the bottom left corner of Fig.~\ref{fig:gdtwo} for $O_{l_2}$.)
Since each $o_j$ is either added to $A_{l_2}$ or not in any solution set,
it remains in the candidate pool when $a_{l_1, c_{l_1}^*+j}$ is considered to be added to $A_{l_2}$.
Therefore, it holds that
\begin{equation}\label{inq:itg-case2-2}
\marge{a_{l_1, c_{l_1}^*+j}}{A_{l_1, c_{l_1}^*+j-1}} \ge \marge{o_j}{A_{l_1, c_{l_1}^*+j-1}}, \forall 1\le j\le m-c_{l_1}^*.
\end{equation}
Then,
\begin{align*}
\marge{O_{l_2}}{A_{l_1}} &\le \marge{A_{l_2, c_{l_1}^*}}{A_{l_1}}  + \sum_{o_j \in O_{l_2}\setminus A_{l_2, c_{l_1}^*}}\marge{o_j}{A_{l_1}} \tag{Proposition~\ref{prop:sum-marge}}\\
&\le \marge{A_{l_2, c_{l_1}^*}}{G_{i-1}} + \sum_{o_j \in O_{l_2}\setminus A_{l_2, c_{l_1}^*}}\marge{o_j}{A_{l_1, , c_{l_1}^*+j-1}} \tag{submodularity}\\
&\le \ff{A_{l_2, c_{l_1}^*}}-\ff{G_{i-1}} + \sum_{j = 1}^{m-c_{l_1}^*}\marge{a_{l_1, c_{l_1}^*+j}}{A_{l_1, , c_{l_1}^*+j-1}} \tag{Inequality~\eqref{inq:itg-case2-1}}\\
& \le \ff{A_{l_1, c_{l_1}^*}}-\ff{G_{i-1}} + \ff{A_{l_1}} - \ff{A_{l_1, c_{l_1}^*}} 
\end{align*}
Thus, the lemma holds in this case.

\textbf{Case 2: $c_{l_1}^* > c_{l_2}^*$; right half part in Fig.~\ref{fig:gdtwo}.}
First, we bound $\marge{O_{l_1}}{A_{l_2}}$.
Order the elements in $O_{l_1}\setminus A_{l_1, c_{l_2}^*+1}$ as $\{o_1, o_2, \ldots\}$ such that $o_j \not \in A_{l_1, c_{l_2}^*+j}$.
(Refer to the gray block with a dotted edge in the top right corner of Fig.~\ref{fig:gdtwo} for $O_{l_1}$.)
Since each $o_j$ is either in $A_{l_1}$ or not in any solution set,
it remains in the candidate pool when $a_{l_2, c_{l_2}^*+j}$ is considered to be added to $A_{l_2}$.
Therefore, it holds that 
\begin{equation}\label{inq:itg-case3-1}
\marge{a_{l_2, c_{l_2}^*+j}}{A_{l_2, c_{l_2}^*+j-1}} \ge \marge{o_j}{A_{l_2, c_{l_2}^*+j-1}}, \forall 1\le j\le m-c_{l_2}^*-1.
\end{equation}
Then,
\begin{align*}
\marge{O_{l_1}}{A_{l_2}} &\le \marge{A_{l_1, c_{l_2}^*+1}}{A_{l_2}}  + \sum_{o_j \in O_{l_1}\setminus A_{l_1, c_{l_2}^*+1}}\marge{o_j}{A_{l_2}} \tag{Proposition~\ref{prop:sum-marge}}\\
&\le \marge{A_{l_1, c_{l_2}^*+1}}{G_{i-1}} + \sum_{o_j \in O_{l_1}\setminus A_{l_1, c_{l_2}^*+1}}\marge{o_j}{A_{l_2, , c_{l_2}^*+j-1}} \tag{submodularity}\\
&\le \ff{A_{l_1, c_{l_2}^*+1}}-\ff{G_{i-1}} + \sum_{j = 1}^{m-c_{l_2}^*-1}\marge{a_{l_2, c_{l_2}^*+j}}{A_{l_2, , c_{l_2}^*+j-1}} \tag{Inequality~\eqref{inq:itg-case3-1}}\\
& \le \ff{A_{l_1, c_{l_2}^*+1}}-\ff{G_{i-1}} + \ff{A_{l_2}} - \ff{A_{l_2, c_{l_2}^*}} 
\end{align*}

Similarly, we bound $\marge{O_{l_2}}{A_{l_1}}$ below.
Since $c_{l_2}^* < c_{l_2}^*$,
we know that the $(c_{l_2}^*+1)$-th element in $A_{l_2}\setminus G_{i-1}$
is not in $O$, which implies that $|O_{l_2}| \le m$.
So, we can order the elements in $O_{l_2}\setminus A_{l_2, c_{l_2}^*}$ as $\{o_1, o_2, \ldots\}$
such that $o_j \not \in A_{l_2, c_{l_2}^* + j}$ for each $1\le j\le m-c_{l_2}^*$.
(See the gray block with a dotted edge in the bottom right corner of Fig.~\ref{fig:gdtwo} for $O_{l_2}$.)

When $1\le j< m-c_{l_2}^*$,
since each $o_j$ is either in $A_{l_2}$ or not in any solution set,
it remains in the candidate pool when $a_{l_1, c_{l_2}^*+i+1}$ is considered to be added to $A_{l_1}$.
Therefore, it holds that
\begin{equation}
\marge{a_{l_1, c_{l_2}^*+j+1}}{A_{l_1, c_{l_2}^*+j}} \ge \marge{o_j}{A_{l_1, c_{l_2}^*+j}}, \forall 1\le j< m-c_{l_2}^*.
\end{equation}
As for the last element $o_{m-c_{l_2}^*}$ in $O_{l_2}\setminus A_{l_2, c_{l_2}^*}$,
we know that $o_{m-c_{l_2}^*}$ is not added to any solution set.
So,
\begin{equation}
\marge{o_{m-c_{l_2}^*}}{A_{l_1}} \le \frac{1}{m}\sum_{j = 1}^m \marge{a_{l_1, j}}{A_{l_1,j-1}}
 = \frac{1}{m} \marge{A_{l_1}}{G_{i-1}}
\end{equation}

Then,
\begin{align*}
\marge{O_{l_2}}{A_{l_1}} &\le \marge{A_{l_2, c_{l_2}^*}}{A_{l_1}}  + \sum_{o_j \in O_{l_2}\setminus A_{l_2, c_{l_2}^*}}\marge{o_j}{A_{l_1}} \tag{Proposition~\ref{prop:sum-marge}}\\
&\le \marge{A_{l_2, c_{l_2}^*}}{G_{i-1}} + \sum_{o_j \in O_{l_2}\setminus A_{l_2, c_{l_2}^*}}\marge{o_j}{A_{l_1, , c_{l_2}^*+j}} \tag{submodularity}\\
&\le \ff{A_{l_2, c_{l_2}^*}}-\ff{G_{i-1}} + \sum_{j = 1}^{m-c_{l_2}^*-1}\marge{a_{l_1, c_{l_2}^*+j+1}}{A_{l_1, , c_{l_2}^*+j}} + \frac{1}{m} \marge{A_{l_1}}{G_{i-1}} \tag{Inequality~\eqref{inq:itg-case3-1}}\\
& \le \ff{A_{l_1, c_{l_2}^*}}-\ff{G_{i-1}} + \ff{A_{l_1}} - \ff{A_{l_1, c_{l_2}^*+1}}+ \frac{1}{m} \marge{A_{l_1}}{G_{i-1}} \tag{$|O_{l_2}|\le m$}
\end{align*}
Thus, the lemma holds in this case.
\end{proof}

\begin{restatable}{lemma}{lemmagdtworec}\label{lemma:gdtwo-rec}
For any iteration $i$ of the outer for loop in Alg.~\ref{alg:gdtwo},
it holds that 

\vspace*{-1em}
{\small\begin{align*}
&\ex{\ff{G_i} - \ff{G_{i-1}}}\ge \frac{1}{\ell+1}\left(1-\frac{1}{m+1}\right) \\
&\cdot \left(\left(1-\frac{1}{\ell}\right)\ex{\ff{O\cup G_{i-1}}} - \ex{\ff{G_{i-1}}}\right)
% \left(1-\frac{1}{m+1}\right)\left((\ell-1)\ff{O\cup G_{i-1}}-\ell\ff{G_{i-1}}\right) \le (\ell+1)\sum_{l\in [\ell]} \marge{A_l}{G_{i-1}}.
\end{align*}}
\end{restatable}
\begin{proof}[Proof of Lemma~\ref{lemma:gdtwo-rec}]
Fix on $G_{i-1}$ for an iteration $i$ of the outer for loop in Alg.~\ref{alg:gdtwo}.
Let $A_l$ be the set after for loop in Lines~\ref{line:gdtwo-for-2-start}-\ref{line:gdtwo-for-2-end} ends (with $m$ iterations).
Then,
\begin{align*}
&\sum_{l\in [\ell]}\marge{O}{A_{l}} 
\le \sum_{l\in [\ell]} \marge{O_l}{A_{l}} + \sum_{1\le l_1 < l_2 \le \ell} \left(\marge{O_{l_1}}{A_{l_2}}+\marge{O_{l_2}}{A_{l_1}}\right) \tag{Inequality~\ref{inq:gdtwo-par}}\\
&\le \sum_{l\in [\ell]} \marge{A_{l}}{G_{i-1}} + \sum_{1\le l_1 < l_2 \le \ell}\left(1+\frac{1}{m}\right) \left(\marge{A_{l_1}}{G_{i-1}}+\marge{A_{l_2}}{G_{i-1}}\right)\tag{Lemma~\ref{lemma:par-A}}\\
&\le \ell\left(1+\frac{1}{m}\right) \sum_{l\in [\ell]} \marge{A_{l}}{G_{i-1}}\\
\Rightarrow& \left(\ell+1\right)\left(1+\frac{1}{m}\right)\sum_{l\in [\ell]}\marge{A_l}{G_{i-1}} \ge \sum_{l\in [\ell]}\ff{O\cup A_l} - \ell \ff{G_{i-1}}\\
& \hspace*{15em} \ge \left(\ell-1\right)\ff{O\cup G_{i-1}} - \ell \ff{G_{i-1}},\numberthis \label{inq:itg-rec-1}
\end{align*}
where the last inequality follows from Proposition~\ref{prop:sum-marge}.
Then, it holds that
\begin{align*}
&\exc{\ff{G_i} - \ff{G_{i-1}}}{G_{i-1}}  = \frac{1}{\ell}\sum_{l \in [\ell]}\marge{A_{l}}{G_{i-1}}\\
&\ge \frac{1}{\ell+1}\cdot\frac{m}{m+1}\cdot\left(\left(1-\frac{1}{\ell}\right)\ff{O\cup G_{i-1}} - \ff{G_{i-1}}\right) \tag{Inequality~\eqref{inq:itg-rec-1}}
\end{align*}
By unfixing $G_{i-1}$, the lemma holds.
\end{proof}

\begin{restatable}{lemma}{lemmagdtwodeg}\label{lemma:gdtwo-deg}
For any iteration $i$ of the outer for loop in Alg.~\ref{alg:gdtwo},
it holds that

\vspace*{-1em}
{\small\begin{align*}
\ex{\ff{O\cup G_i}} \ge \left(1-\frac{1}{\ell}\right) \ex{\ff{O\cup G_{i-1}}}.
\end{align*}}
\end{restatable}
\begin{proof}[Proof of Lemma~\ref{lemma:gdtwo-deg}]
Fix on $G_{i-1}$ at the beginning of this iteration.
Since $\left\{A_l\setminus G_{i-1}\right\}_{l\in [\ell]}$ 
are pairwise disjoint sets at the end of this iteration,
by Proposition~\ref{prop:sum-marge},
it holds that
\[\exc{\ff{O\cup G_i}}{G_{i-1}} = \frac{1}{\ell}\sum_{l\in [\ell]}\ff{O\cup A_l} \ge \left(1-\frac{1}{\ell}\right)\ff{O\cup G_{i-1}}.\]
Then, by unfixing $G_{i-1}$, the lemma holds.
\end{proof}

\subsection{Proof of Theorem~\ref{thm:gdtwo}}\label{apx:gdtwo-approx}
\thmgdtwo*
\begin{proof}
By Lemma~\ref{lemma:gdtwo-rec} and~\ref{lemma:gdtwo-deg},
the recurrence of $\ex{\ff{G_i}}$ can be expressed as follows,
\begin{align*}
\ex{\ff{G_i}} &\ge \left(1-\frac{1}{\ell+1}\left(1-\frac{1}{m+1}\right)\right)\ex{\ff{G_{i-1}}} + \frac{1}{\ell+1}\left(1-\frac{1}{m+1}\right)\left(1-\frac{1}{\ell}\right)^i\ff{O}\\
&\ge \left(1-\frac{1}{\ell}\right)\ex{\ff{G_{i-1}}} + \frac{1}{\ell+1}\left(1-\frac{1}{m+1}\right)\left(1-\frac{1}{\ell}\right)^i\ff{O}.
\end{align*}
By solving the above recurrence,
\begin{align*}
\ex{\ff{G_{\ell}}} &\ge \frac{\ell}{\ell+1}\left(1-\frac{1}{m+1}\right)\left(1-\frac{1}{\ell}\right)^\ell\ff{O}\\
&\ge \frac{\ell-1}{\ell+1}\left(1-\frac{1}{m+1}\right)e^{-1}\ff{O}\tag{Lemma~\ref{lemma:val-inq}}\\
&\ge \left(1-\frac{2}{\ell}\right)\left(1-\frac{\ell}{k}\right)e^{-1}\ff{O} \tag{$m = \left\lfloor \frac{k}{\ell} \right\rfloor$}\\
&\ge \frac{1}{1-\frac{\ell}{k}}\left(1-\frac{2}{\ell}-\frac{2\ell}{k}+\frac{4}{k}\right)e^{-1}\ff{O}\\
&\ge \frac{1}{1-\frac{\ell}{k}}\left(e^{-1}-\epsi\right)\ff{O}. \tag{$\ell \ge \frac{2}{e\epsi}, k\ge \frac{2(\ell-2)}{e\epsi-\frac{2}{\ell}}$}
\end{align*}
By Inequality~\eqref{inq:dif-opt}, the approximation ratio of Alg.~\ref{alg:gdtwo} is $e^{-1}-\epsi$.
\end{proof}

\section{Pseudocodes and Analysis of Algorithms in Section~\ref{sec:tg}}
\label{apx:tg}
In this section, we provide the pseudocodes and analysis of
the simplified fast \ig~\citep{DBLP:conf/nips/Kuhnle19} 
and \itg~\citep{DBLP:conf/kdd/ChenK23},
implemented as Alg.~\ref{alg:tgone} and~\ref{alg:tgtwo}, respectively.
The analysis of these algorithms employ a blending technique
to eliminate the guessing step in their original version.
\subsection{Simplified Fast \ig with $1/4-\epsi$ Approximation Ratio (Alg.~\ref{alg:tgone})}
\begin{algorithm}[ht]
    \KwIn{evaluation oracle $f:2^{\uni} 
    \to \reals$, constraint $k$, error $\epsi$}
    \Init{$A\gets \emptyset$, $B\gets \emptyset$, $M\gets \max_{x \in \uni}\ff{\{x\}}$,
    $\tau_1\gets M$, $\tau_2\gets M$}
    \For{$i\gets 1$ to $k$}{
        \While{$\tau_1 \ge \frac{\epsi M}{k}$ and $|A| < k$}{
            \If{$\exists a \in \uni\setminus\left(A\cup B\right)$ \st $\marge{a}{A} \ge \tau_1$}{
            $A\gets A+ a$\;
            \textbf{break}\;}
            \lElse{$\tau_1 \gets (1-\epsi)\tau_1$}
        }
        \While{$\tau_2 \ge \frac{\epsi M}{k}$ and $|B| < k$}{
            \If{$\exists b \in \uni\setminus\left(A\cup B\right)$ \st $\marge{b}{B} \ge \tau_2$}{
            $B\gets B+ b$\;
            \textbf{break}\;}
            \lElse{$\tau_2 \gets (1-\epsi)\tau_2$}
        }
    }
    \Return{$S\gets \argmax\{\ff{A}, \ff{B}\}$}
    \caption{A nearly-linear time, $(1/4-\epsi)$-approximation algorithm.}
    \label{alg:tgone}
\end{algorithm}
\thmtgone*
\begin{proof}
\textbf{Query Complexity.}
Without loss of generality, we analyze the number queries related to set $A$.
For each threshold value $\tau_1$, at most $n$ queries are made to the value oracle.
Since $\tau_1$ is initialized with value $M$, decreases by a factor of $1-\epsi$,
and cannot exceed $\frac{\epsi M}{k}$,
there are at most $\log_{1-\epsi}\left(\frac{\epsi}{k}\right)+1$ possible values of $\tau_1$.
Therefore, the total number of queries is bounded as follows,
\begin{align*}
\#\text{Queries} \le 2\cdot n\cdot \left(\log_{1-\epsi}\left(\frac{\epsi}{k}\right)+1\right)
\le \oh{n\log(k)/\epsi},
\end{align*}
where the last inequality follows from the first inequality in Lemma~\ref{lemma:val-inq}.

\textbf{Approximation Ratio.}
Since $A$ and $B$ are disjoint, by submodularity and non-negativity,
\begin{equation}\label{inq:tgone-1}
\ff{O} \le \ff{O\cup A} + \ff{O\cup B}.
\end{equation}

Let $a_i$ be the $i$-th element added to $A$,
$A_i$ be the first $i$ elements added to $A$,
and $\tau_1^{a_i}$ be the threshold value when $a_i$ is added to $A$.
Similarly, define $b_i$, $B_i$, and $\tau_2^{b_i}$.
Let $i^* = \max\{i \le |A|: A_i \subseteq O\}$
and $j^* = \max\{i \le |B|: B_i \subseteq O\}$.
If either $i^*= k$ or $j^* = k$,
then $\ff{S}= \ff{O}$.
Next, we follow the analysis of Alg.~\ref{alg:gdone} in Section~\ref{sec:greedy-1/4}
to analyze the approximation ratio of Alg.~\ref{alg:tgone}.

\textbf{Case 1: $0\le i^*\le j^* < k$; Fig.~\ref{fig:gdone-1}.}
First, we bound $\ff{O\cup A}$. 
Since $B_{i^*} \subseteq O$, by submodularity
\begin{equation}\label{inq:tgone-3}
\ff{O\cup A} - \ff{A} \le \marge{B_{i^*}}{A} + \marge{O\setminus B_{i^*}}{A}
\le \ff{B_{i^*}}+ \sum_{o\in O\setminus \left(A\cup B_{i^*}\right)}\marge{o}{A}.
\end{equation}
Next, we bound $\marge{o}{A}$ for each $o\in O\setminus \left(A\cup B_{i^*}\right)$.

Let $\tilde{O} = O\setminus \left(A \cup B_{i^*}\right)$.
Obviously, it holds that $|\tilde{O}|\le k-i^*$.
Then, order $\tilde{O}$ as $\{o_1, o_2, \ldots\}$ such that $o_i \not \in B_{i+i^*-1}$,
for all $1\le i \le |\tilde{O}|$.
If $|A| < k$, the algorithm terminates with $\tau_1 < \frac{\epsi M}{k}$.
Thus, it follows that
\begin{equation}\label{inq:tgone-4}
\marge{o_i}{A} < \frac{\epsi M}{k(1-\epsi)}, \forall |A|-i^* < i \le |\tilde{O}|.
\end{equation}

Next, consider tuple $(o_i, a_{i + i^*}, A_{i+i^*-1})$,
for any $1\le i \le \min\{|\tilde{O}|, |A|-i^*\}$.
Since $\tau_1^{a_{i + i^*}}$ is the threshold value when $a_{i + i^*}$ is added,
it holds that 
\begin{equation}\label{inq:tgone-2}
\marge{a_{i + i^*}}{A_{i+i^*-1}} \ge \tau_1^{a_{i + i^*}},
\forall 1\le i\le |A| - i^*.
\end{equation}
Then, we show that $\marge{o_i}{A_{i+i^*-1}} < \tau_1^{a_{i + i^*}}/(1-\epsi)$ always holds
for any $1\le i \le \min\{|\tilde{O}|, |A|-i^*\}$.

Since $M = \max_{x\in \uni}\ff{\{x\}}$,
if $\tau_1^{a_{i+i^*}} \ge M$,
it always holds that $\marge{o_i}{A_{i+i^*-1}} < M/(1-\epsi)\le \tau_1^{a_{i+i^*}}/(1-\epsi)$.
If $\tau_1^{a_{i+i^*}} < M$, 
since $o_i \not \in B_{i+i^*-1}$,
$o_i$ is not considered to be added to $A$ with threshold value $\tau_1^{a_{i+i^*}}/(1-\epsi)$.
Then, by submodularity,
$\marge{o_i}{A_{i+i^*-1}} < \tau_1^{a_{i+i^*}}/(1-\epsi)$.
Therefore, by submodularity and Inequality~\eqref{inq:tgone-2},
it holds that 
\begin{equation}\label{inq:tgone-5}
\marge{o_i}{A} \le \marge{o_i}{A_{i+i^*-1}} < \marge{a_{i + i^*}}{A_{i+i^*-1}}/(1-\epsi), \forall 1\le i \le \min\{|\tilde{O}|, |A|-i^*\}.
\end{equation}

Then,
\begin{align*}
\ff{O\cup A} - \ff{A} &\le \ff{B_{i^*}} + \sum_{o\in O\setminus \left(A\cup B_{i^*}\right)}\marge{o}{A}\\
&\le \ff{B_{i^*}} + \sum_{i = 1}^{\min\{|\tilde{O}|, |A|\}-i^*}\marge{a_{i + i^*}}{A_{i+i^*-1}}/(1-\epsi) + \frac{\epsi M}{1-\epsi}\\
&\le \frac{1}{1-\epsi}\left(\ff{B_{i^*}} + \ff{A}-\ff{A_{i^*}} + \epsi \ff{O}\right),\numberthis \label{inq:tgone-10}
\end{align*}
where the first inequality follows from Inequality~\eqref{inq:tgone-3};
the second inequality follows from Inequalities~\eqref{inq:tgone-4} and~\eqref{inq:tgone-5};
and the last inequality follows from $M\le \ff{O}$.

Second, we bound $\ff{O\cup B}$.
Since $A_{i^*} \subseteq O$, by submodularity
\begin{equation}\label{inq:tgone-6}
\ff{O\cup B} - \ff{B} \le \marge{A_{i^*}}{B} + \marge{O\setminus A_{i^*}}{B}
\le \ff{A_{i^*}}+ \sum_{o\in O\setminus \left(B\cup A_{i^*}\right)}\marge{o}{B}.
\end{equation}
Next, we bound $\marge{o}{B}$ for each $o\in O\setminus \left(B\cup A_{i^*}\right)$.

Let $\tilde{O} = O\setminus \left(B\cup A_{i^*}\right)$.
Obviously, it holds that $|\tilde{O}|\le k-i^*$.
Then, since $a_{i^*+1} \not\in O$,
we can order $\tilde{O}$ as $\{o_1, o_2, \ldots\}$ such that $o_i \not \in A_{i+i^*}$,
for all $1\le i \le |\tilde{O}|$.
If $|B| < k$, the algorithm terminates with $\tau_2 < \frac{\epsi M}{k}$.
Thus, it follows that
\begin{equation}\label{inq:tgone-7}
\marge{o_i}{B} < \frac{\epsi M}{k(1-\epsi)}, \forall |B|-i^* < i \le |\tilde{O}|.
\end{equation}

Next, consider tuple $(o_i, b_{i + i^*}, B_{i+i^*-1})$,
for any $1\le i \le \min\{|\tilde{O}|, |B|-i^*\}$.
Since $\tau_2^{b_{i + i^*}}$ is the threshold value when $b_{i + i^*}$ is added,
it holds that 
\begin{equation}\label{inq:tgone-8}
\marge{b_{i + i^*}}{B_{i+i^*-1}} \ge \tau_2^{b_{i + i^*}},
\forall 1\le i\le |B| - i^*.
\end{equation}
Then, we show that $\marge{o_i}{B_{i+i^*-1}} < \tau_2^{b_{i + i^*}}/(1-\epsi)$ always holds
for any $1\le i \le \min\{|\tilde{O}|, |B|-i^*\}$.

Since $M = \max_{x\in \uni}\ff{\{x\}}$,
if $\tau_2^{b_{i+i^*}} \ge M$,
it always holds that $\marge{o_i}{B_{i+i^*-1}} < M/(1-\epsi)\le \tau_2^{b_{i+i^*}}/(1-\epsi)$.
If $\tau_2^{b_{i+i^*}} < M$, 
since $o_i \not \in A_{i+i^*}$,
$o_i$ is not considered to be added to $B$ with threshold value $\tau_2^{b_{i+i^*}}/(1-\epsi)$.
Then, by submodularity,
$\marge{o_i}{B_{i+i^*-1}} < \tau_2^{b_{i+i^*}}/(1-\epsi)$.
Therefore, by submodularity and Inequality~\eqref{inq:tgone-8},
it holds that 
\begin{equation}\label{inq:tgone-9}
\marge{o_i}{B} \le \marge{o_i}{B_{i+i^*-1}} < \marge{b_{i + i^*}}{B_{i+i^*-1}}/(1-\epsi), \forall 1\le i \le \min\{|\tilde{O}|, |B|-i^*\}.
\end{equation}

Then,
\begin{align*}
\ff{O\cup B} - \ff{B} &\le \ff{A_{i^*}} + \sum_{o\in O\setminus \left(B\cup A_{i^*}\right)}\marge{o}{B}\\
&\le \ff{A_{i^*}} + \sum_{i = 1}^{\min\{|\tilde{O}|, |B|\}-i^*}\marge{b_{i + i^*}}{B_{i+i^*-1}}/(1-\epsi) + \frac{\epsi M}{1-\epsi}\\
&\le \frac{1}{1-\epsi}\left(\ff{A_{i^*}} + \ff{B}-\ff{B_{i^*}} + \epsi \ff{O}\right),\numberthis \label{inq:tgone-11}
\end{align*}
where the first inequality follows from Inequality~\eqref{inq:tgone-6};
the second inequality follows from Inequalities~\eqref{inq:tgone-7} and~\eqref{inq:tgone-9};
and the last inequality follows from $M\le \ff{O}$.

By Inequalities~\eqref{inq:tgone-1},~\eqref{inq:tgone-10} and~\eqref{inq:tgone-11},
it holds that
\begin{align*}
&\ff{O} \le \frac{2-\epsi}{1-\epsi}\left(\ff{A} + \ff{B}\right) + \frac{2\epsi}{1-\epsi}\ff{O}\\
\Rightarrow &\ff{S} \ge \left(\frac{1}{4} - \frac{5}{2(4-2\epsi)}\epsi\right)\ff{O}
\ge \left(\frac{1}{4}-\epsi\right)\ff{O}\tag{$\epsi < 1/2$}
\end{align*}

\textbf{Case 2: $0\le j^* < i^* < k$; Fig.~\ref{fig:gdone-2}.}

First, we bound $\ff{O\cup A}$. 
Since $B_{j^*} \subseteq O$, by submodularity
\begin{equation}\label{inq:tgone-20}
\ff{O\cup A} - \ff{A} \le \marge{B_{j^*}}{A} + \marge{O\setminus B_{j^*}}{A}
\le \ff{B_{j^*}}+ \sum_{o\in O\setminus \left(A\cup B_{j^*}\right)}\marge{o}{A}.
\end{equation}
Next, we bound $\marge{o}{A}$ for each $o\in O\setminus \left(A\cup B_{j^*}\right)$.

Let $\tilde{O} = O\setminus \left(A \cup B_{j^*}\right)$.
Since $i^* > j^*\ge 0$, 
it holds that $|\tilde{O}|\le k-j^*-1$.
Since $b_{j^*+1}\not \in O$,
we can order $\tilde{O}$ as $\{o_1, o_2, \ldots\}$ such that $o_i \not \in B_{i+j^*}$,
for all $1\le i \le |\tilde{O}|$.
If $|A| < k$, the algorithm terminates with $\tau_1 < \frac{\epsi M}{k}$.
Thus, it follows that
\begin{equation}\label{inq:tgone-21}
\marge{o_i}{A} < \frac{\epsi M}{k(1-\epsi)}, \forall |A|-j^*-1 < i \le |\tilde{O}|.
\end{equation}

Next, consider tuple $(o_i, a_{i + j^*+1}, A_{i+j^*})$,
for any $1\le i \le \min\{|\tilde{O}|, |A|-j^*-1\}$.
Since $\tau_1^{a_{i + j^*+1}}$ is the threshold value when $a_{i + j^*+1}$ is added,
it holds that 
\begin{equation}\label{inq:tgone-22}
\marge{a_{i + j^*+1}}{A_{i+j^*}} \ge \tau_1^{a_{i + j^*+1}},
\forall 1\le i\le |A| - j^*-1.
\end{equation}
Then, we show that $\marge{o_i}{A_{i+j^*}} < \tau_1^{a_{i + j^*+1}}/(1-\epsi)$ always holds
for any $1\le i \le \min\{|\tilde{O}|, |A|-j^*-1\}$.

Since $M = \max_{x\in \uni}\ff{\{x\}}$,
if $\tau_1^{a_{i + j^*+1}} \ge M$,
it always holds that $\marge{o_i}{A_{i+j^*}} < M/(1-\epsi)\le \tau_1^{a_{i + j^*+1}}/(1-\epsi)$.
If $\tau_1^{a_{i + j^*+1}} < M$, 
since $o_i \not \in B_{i+j^*}$,
$o_i$ is not considered to be added to $A$ with threshold value $\tau_1^{a_{i + j^*+1}}/(1-\epsi)$.
Then, by submodularity,
$\marge{o_i}{A_{i+j^*}} < \tau_1^{a_{i + j^*+1}}/(1-\epsi)$.
Therefore, by submodularity and Inequality~\eqref{inq:tgone-22},
it holds that 
\begin{equation}\label{inq:tgone-23}
\marge{o_i}{A} \le \marge{o_i}{A_{i+j^*}} < \marge{a_{i + j^*+1}}{A_{i+j^*}}/(1-\epsi), \forall 1\le i \le \min\{|\tilde{O}|, |A|-j^*-1\}.
\end{equation}

Then,
\begin{align*}
\ff{O\cup A} - \ff{A} &\le \ff{B_{j^*}} + \sum_{o\in O\setminus \left(A\cup B_{j^*}\right)}\marge{o}{A}\\
&\le \ff{B_{j^*}} + \sum_{i = 1}^{\min\{|\tilde{O}|, |A|-j^*-1\}}\marge{a_{i + j^*+1}}{A_{i+j^*}}/(1-\epsi) + \frac{\epsi M}{1-\epsi}\\
&\le \frac{1}{1-\epsi}\left(\ff{B_{j^*}} + \ff{A}-\ff{A_{j^*+1}} + \epsi \ff{O}\right),\numberthis \label{inq:tgone-24}
\end{align*}
where the first inequality follows from Inequality~\eqref{inq:tgone-20};
the second inequality follows from Inequalities~\eqref{inq:tgone-21} and~\eqref{inq:tgone-23};
and the last inequality follows from $M\le \ff{O}$.

Second, we bound $\ff{O\cup B}$.
Since $A_{j^*+1} \subseteq O$, by submodularity
\begin{equation}\label{inq:tgone-25}
\ff{O\cup B} - \ff{B} \le \marge{A_{j^*+1}}{B} + \marge{O\setminus A_{j^*+1}}{B}
\le \ff{A_{j^*+1}}+ \sum_{o\in O\setminus \left(B\cup A_{j^*+1}\right)}\marge{o}{B}.
\end{equation}
Next, we bound $\marge{o}{B}$ for each $o\in O\setminus \left(B\cup A_{j^*+1}\right)$.

Let $\tilde{O} = O\setminus \left(B\cup A_{j^*+1}\right)$.
Obviously, it holds that $|\tilde{O}|\le k-j^*-1$.
Then, order $\tilde{O}$ as $\{o_1, o_2, \ldots\}$ such that $o_i \not \in A_{i+j^*}$,
for all $1\le i \le |\tilde{O}|$.
If $|B| < k$, the algorithm terminates with $\tau_2 < \frac{\epsi M}{k}$.
Thus, it follows that
\begin{equation}\label{inq:tgone-26}
\marge{o_i}{B} < \frac{\epsi M}{k(1-\epsi)}, \forall |B|-j^*-1 < i \le |\tilde{O}|.
\end{equation}

Next, consider tuple $(o_i, b_{i + j^*}, B_{i+j^*-1})$,
for any $1\le i \le \min\{|\tilde{O}|, |B|-j^*-1\}$.
Since $\tau_2^{b_{i + j^*}}$ is the threshold value when $b_{i + j^*}$ is added,
it holds that 
\begin{equation}\label{inq:tgone-27}
\marge{b_{i + j^*}}{B_{i+j^*-1}} \ge \tau_2^{b_{i + j^*}},
\forall 1\le i\le |B| - j^*-1.
\end{equation}
Then, we show that $\marge{o_i}{B_{i+j^*-1}} < \tau_2^{b_{i + j^*}}/(1-\epsi)$ always holds
for any $1\le i \le \min\{|\tilde{O}|, |B|-j^*-1\}$.

Since $M = \max_{x\in \uni}\ff{\{x\}}$,
if $\tau_2^{b_{i+j^*}} \ge M$,
it always holds that $\marge{o_i}{B_{i+j^*-1}} < M/(1-\epsi)\le \tau_2^{b_{i+j^*}}/(1-\epsi)$.
If $\tau_2^{b_{i+j^*}} < M$, 
since $o_i \not \in A_{i+j^*}$,
$o_i$ is not considered to be added to $B$ with threshold value $\tau_2^{b_{i+j^*}}/(1-\epsi)$.
Then, by submodularity,
$\marge{o_i}{B_{i+j^*-1}} < \tau_2^{b_{i+j^*}}/(1-\epsi)$.
Therefore, by submodularity and Inequality~\eqref{inq:tgone-27},
it holds that 
\begin{equation}\label{inq:tgone-28}
\marge{o_i}{B} \le \marge{o_i}{B_{i+j^*-1}} < \marge{b_{i + j^*}}{B_{i+j^*-1}}/(1-\epsi), \forall 1\le i \le \min\{|\tilde{O}|, |B|-j^*-1\}.
\end{equation}

Then,
\begin{align*}
\ff{O\cup B} - \ff{B} &\le \ff{A_{j^*+1}} + \sum_{o\in O\setminus \left(B\cup A_{j^*+1}\right)}\marge{o}{B}\\
&\le \ff{A_{j^*+1}} + \sum_{i = 1}^{\min\{|\tilde{O}|, |B|-j^*-1\}}\marge{b_{i + j^*}}{B_{i+j^*-1}}/(1-\epsi) + \frac{\epsi M}{1-\epsi}\\
&\le \frac{1}{1-\epsi}\left(\ff{A_{j^*+1}} + \ff{B}-\ff{B_{i^*}} + \epsi \ff{O}\right),\numberthis \label{inq:tgone-29}
\end{align*}
where the first inequality follows from Inequality~\eqref{inq:tgone-25};
the second inequality follows from Inequalities~\eqref{inq:tgone-26} and~\eqref{inq:tgone-28};
and the last inequality follows from $M\le \ff{O}$.

By Inequalities~\eqref{inq:tgone-1},~\eqref{inq:tgone-24} and~\eqref{inq:tgone-29},
it holds that
\begin{align*}
&\ff{O} \le \frac{2-\epsi}{1-\epsi}\left(\ff{A} + \ff{B}\right) + \frac{2\epsi}{1-\epsi}\ff{O}\\
\Rightarrow &\ff{S} \ge \left(\frac{1}{4} - \frac{5}{2(4-2\epsi)}\epsi\right)\ff{O}
\ge \left(\frac{1}{4}-\epsi\right)\ff{O}\tag{$\epsi < 1/2$}
\end{align*}

Therefore, in both cases, it holds that
\[\ff{S} \ge \left(\frac{1}{4}-\epsi\right)\ff{O} .\]
\end{proof}


\subsection{Simplified Fast \itg with $1/e-\epsi$ Approximation Ratio (Alg.~\ref{alg:tgtwo})}
\begin{algorithm}[ht]
    \KwIn{evaluation oracle $f:2^{\uni} \to \reals$, constraint $k$, error $\epsi$}
    \Init{$G_0\gets \emptyset$, $\epsi'\gets \frac{\epsi}{2}$, $m \gets \left\lfloor\frac{k}{\ell}\right\rfloor$, $\ell\gets \left \lceil\frac{4}{e\epsi'}\right \rceil$,
    $M\gets \max_{x\in \uni} \ff{\{x\}}$}
    \For{$i\gets 1$ to $\ell$}{
        $\tau_l \gets M, \forall l\in [\ell]$\;
        $A_{l}\gets G_{i-1}, \forall l \in [\ell]$\;
        \For{$j\gets 1$ to $m$}{
            \For{$l\gets 1$ to $\ell$}{
                \While{$\tau_l \ge \frac{\epsi' M}{k}$ and $|A_l\setminus G_{i-1}| < m$}{
                    \If{$\exists x \in \uni\setminus\left(\bigcup_{r\in [\ell]} A_r\right)$ \st $\marge{a}{A_l} \ge \tau_l$}{
                    $A_l\gets A_l+ x$\;
                    \textbf{break}\;}
                    \lElse{$\tau_l \gets (1-\epsi')\tau_l$}
                }
            }
        }
        $G_i\gets$ a random set in $\{A_l\}_{l\in [\ell]}$\;
    }
    \Return{$G_\ell$}
    \caption{A nearly-linear time, $(1/e-\epsi)$-approximation algorithm.}
    \label{alg:tgtwo}
\end{algorithm}
\thmtgtwo*
\begin{proof}
When $k\,\text{mod}\,\ell > 0$, the algorithm returns an approximation with a size constraint of 
$\ell\cdot\left\lfloor \frac{k}{\ell}\right\rfloor$, where by Proposition~\ref{prop:dif-opt},
\begin{equation}\label{inq:tgtwo-dif-opt}
\ff{O'} \ge \left(1-\frac{\ell}{k}\right)\ff{O}, 
O' = \argmax\limits_{S\subseteq \uni, |S|\le \ell\cdot \left\lfloor \frac{k}{\ell} \right\rfloor}\ff{S}.
\end{equation}
In the following, we only consider the case where $k\,\text{mod}\,\ell = 0$.

At every iteration of the outer for loop,
$\ell$ solutions are constructed, with each solution being augmented
by at most $k/\ell$ elements.
To bound the marginal gain of the optimal set $O$ on each solution set $A_l$,
we consider partitioning $O$ into $\ell$ subsets.
We formalize this partition in the following claim,
which yields a result analogous to Claim~\ref{claim:par-A} presented in 
Section~\ref{sec:greedy-1/e}.
Specifically, the claim states that the optimal set $O$ can be evenly 
divided into $\ell$ subsets,
where each subset only overlaps with only one solution set.
\begin{claim}
At an iteration $i$ of the outer for loop in Alg.~\ref{alg:tgtwo},
let $G_{i-1}$ be $G$ at the start of this iteration,
and $A_{l}$ be the set at the end of this iteration,
for each $l\in [\ell]$.
% Add dummy elements to $O\setminus G_{i-1}$ until its size equals $k$.
The set $O\setminus G_{i-1}$ can then be split into $\ell$ pairwise disjoint sets $\{O_1, \ldots, O_\ell\}$
such that $|O_l| \le\frac{k}{\ell}$ and $\left(O\setminus G_{i-1}\right) \cap \left(A_{l}\setminus G_{i-1}\right) \subseteq O_l$, for all $l \in [\ell]$.
\end{claim}
Next, based on such partition, we introduce the following lemma, 
which provides a bound on the marginal gain of any subset $O_{l_1}$ 
with respect to any solution set $A_{l_2}$,
where $1\le l_1, l_2 \le \ell$.
\begin{lemma}\label{lemma:tg-par-A}
Fix on $G_{i-1}$ for an iteration $i$ of the outer for loop in Alg.~\ref{alg:tgtwo}.
Following the definition in Claim~\ref{claim:par-A}, it holds that
\begin{align*}
\text{1) }&\marge{O_{l}}{A_{l}}\le \frac{\marge{A_{l}}{G_{j-1}}}{1-\epsi'}+\frac{\epsi' M}{(1-\epsi')\ell}, \forall 1\le l \le \ell,\\
\text{2) }&\marge{O_{l_2}}{A_{l_1}} + \marge{O_{l_1}}{A_{l_2}} \le \frac{1}{1-\epsi'}\left(1+\frac{1}{m}\right)\left(\marge{A_{l_1}}{G_{i-1}}+\marge{A_{l_2}}{G_{i-1}}\right)
+\frac{2\epsi' M}{(1-\epsi')\ell}, \forall 1\le l_1 < l_2 \le \ell.
\end{align*}
\end{lemma}
Followed by the above lemma, 
we provide the recurrence of $\ex{\ff{G_i}}$ and $\ex{\ff{O\cup G_i}}$.
\begin{lemma}\label{lemma:tg-recur}
For any iteration $i$ of the outer for loop in Alg.~\ref{alg:tgtwo},
it holds that
\begin{align*}
\text{1) } & \ex{\ff{O\cup G_i}}\ge \left(1-\frac{1}{\ell}\right) \ex{\ff{O\cup G_{i-1}}}\\
\text{2) } & \ex{\ff{G_i} - \ff{G_{i-1}}}
\ge\frac{1}{1+\frac{\ell}{1-\epsi'}}\left(1-\frac{1}{m+1}\right)\left(\left(1-\frac{1}{\ell}\right)  \ex{\ff{O\cup G_{i-1}}} - \ex{\ff{G_{i-1}}} - \frac{\epsi'}{1-\epsi'}\ff{O}\right).
\end{align*}
\end{lemma}
By solving the recurrence in Lemma~\ref{lemma:tg-recur},
we calculate the approximation ratio of the algorithm as follows,
\begin{align*}
&\ex{\ff{G_{i}}}  \ge \left(1-\frac{1}{\ell}\right) \ex{\ff{G_{i-1}}}
+ \frac{1}{1+\frac{\ell}{1-\epsi'}}\left(1-\frac{1}{m+1}\right)\left(\left(1-\frac{1}{\ell}\right)^i - \frac{\epsi'}{1-\epsi'}\right)\ff{O}\\
\Rightarrow& \ex{\ff{G_\ell}} \ge \frac{\ell}{1+\frac{\ell}{1-\epsi'}}\left(1-\frac{1}{m+1}\right)\left(\left(1-\frac{1}{\ell}\right)^\ell - \frac{\epsi'}{1-\epsi'}\left(1-\left(1-\frac{1}{\ell}\right)^\ell\right)\right)\ff{O}\\
&\hspace*{4em} \ge \frac{\ell-1}{1+\frac{\ell}{1-\epsi'}}\left(1-\frac{1}{m+1}\right)\left(e^{-1} - \frac{\epsi'}{1-\epsi'}\left(1-e^{-1}\right)\right)\ff{O}\\
&\hspace*{4em} \ge \frac{1}{1-\frac{\ell}{k}}\left(1-\epsi' - \frac{2}{\ell}\right)\left(1-\frac{\ell}{k}\right)^2\left(e^{-1} - \frac{\epsi'}{1-\epsi'}\left(1-e^{-1}\right)\right) \ff{O}\\
% &\hspace*{4em} \ge \frac{1}{1-\frac{\ell}{k}}\left(1-\epsi' - \frac{2}{\ell}\right)\left(1-\frac{2\ell}{k}\right)\left(e^{-1} - \frac{\epsi'}{1-\epsi'}\left(1-e^{-1}\right)\right) \ff{O}\\
&\hspace*{4em} \ge \frac{1}{1-\frac{\ell}{k}}\left(1-\epsi' - \frac{2}{\ell}-\frac{2(1-\epsi')\ell}{k}\right)\left(e^{-1} - \frac{\epsi'}{1-\epsi'}\left(1-e^{-1}\right)\right) \ff{O}\\
&\hspace*{4em} \ge \frac{1}{1-\frac{\ell}{k}} \left(1-(e+1)\epsi'\right)\left(e^{-1} - \frac{\epsi'}{1-\epsi'}\left(1-e^{-1}\right)\right) \ff{O}\tag{$\ell \ge \frac{2}{e\epsi'}, k \ge \frac{2(1-\epsi')\ell}{e\epsi'-\frac{2}{\ell}}$}\\
&\hspace*{4em} \ge \frac{1}{1-\frac{\ell}{k}} \left(e^{-1}-\epsi\right)\ff{O}\tag{$\epsi' = \frac{\epsi}{2}$}.
\end{align*}
By Inequality~\ref{inq:tgtwo-dif-opt},
the approximation ratio of Alg.~\ref{alg:tgtwo} is $e^{-1}-\epsi$.
\end{proof}

In the rest of this section, we provide the proofs for 
Lemma~\ref{lemma:tg-par-A} and~\ref{lemma:tg-recur}.
\begin{proof}[Proof of Lemma~\ref{lemma:tg-par-A}]
At iteration $i$ of the outer for loop,
let $A_l$ be the set at the end of iteration $i$,
$a_{l, j}$ be the $j$-th element added to $A_l$,
$\tau_l^j$ be the threshold value of $\tau_l$ when $a_{l, j}$ is added to $A_l$,
and $A_{l, j}$ be $A_l$ after $a_{l, j}$ is added to $A_l$.
Let $c_l^* = \max\{c\in [m]:A_{l, c}\setminus G_{i-1}\subseteq O_l\}$.

First, we prove that the first inequality holds.
For each $l\in [\ell]$, order the elements in $O_l$ as $\{o_1, o_2, \ldots\}$
such that $o_j \not \in A_{l, j-1}$ for any $1\le j \le |A_l\setminus G_{i-1}|$,
and $o_j\not \in A_l$ for any $|A_l\setminus G_{i-1}| < j \le m$.

When $1\le j \le |A_l\setminus G_{i-1}|$, by Claim~\ref{claim:par-A},
each $o_j$ is either added to $A_l$ or not in any solution set.
Since $\tau_l$ is initialized with the maximum marginal gain $M$,
$o_j$ is not considered to be added to $A_l$ with threshold value 
$\tau_l^j/(1-\epsi')$.
Therefore, by submodularity it holds that
\begin{equation}\label{inq:tgtwo-1}
\marge{o_j}{A_{l, j-1}} < \tau_l^j/(1-\epsi')\le \marge{a_{l,j}}{A_{l,j-1}}/(1-\epsi'),
\forall 1\le j \le |A_l\setminus G_{i-1}|.
\end{equation}

When $|A_l\setminus G_{i-1}| <  m$,
the minimum value of $\tau_l$ is less than $\frac{\epsi' M}{k}$.
Then, for any $|A_l\setminus G_{i-1}| < j \le m$,
$o_j$ is not considered to be added to $A_l$ with threshold value less than $\frac{\epsi' M}{(1-\epsi')k}$.
It follows that 
\begin{equation}\label{inq:tgtwo-2}
\marge{o_j}{A_l} \le \frac{\epsi' M}{(1-\epsi')k},
\forall |A_l\setminus G_{i-1}| < j \le m.
\end{equation}

Then,
\begin{align*}
\marge{O_{l}}{A_{l}} &\le \sum_{o_j\in O_l} \marge{o_j}{A_{l}} \tag{Proposition~\ref{prop:sum-marge}}\\
&\le \sum_{j=1}^{|A_l\setminus G_{i-1}|} \marge{o_j}{A_{l, j-1}} + 
\sum_{j=|A_l\setminus G_{i-1}|+1}^m \marge{o_j}{A_{l}}\tag{Submodularity}\\
&\le \sum_{j=1}^{|A_l\setminus G_{i-1}|}\frac{\marge{a_{l,j}}{A_{l,j-1}}}{1-\epsi'}+\frac{\epsi' M}{(1-\epsi')\ell}
\tag{Inequalities~\eqref{inq:tgtwo-1} and~\eqref{inq:tgtwo-2}}\\
&= \frac{\marge{A_{l}}{G_{j-1}}}{1-\epsi'}+\frac{\epsi' M}{(1-\epsi')\ell}.
\end{align*}
The first inequality holds.

In the following, we prove that the second inequality holds.
For any $1\le l_1\le l_2\le \ell$,
we analyze two cases of the relationship between $c_{l_1}^* $ and $ c_{l_2}^*$ in the following.

% \textbf{Case 1: $c_{l_1}^* = c_{l_2}^* = m$.}
% Then, $O_{l_1} = A_{l_1}\setminus G_{i-1}$ and $O_{l_2} = A_{l_2}\setminus G_{i-1}$.
% By submodularity,
% \[\marge{O_{l_2}}{A_{l_1}} + \marge{O_{l_1}}{A_{l_2}} \le \marge{O_{l_2}}{G_{i-1}} + \marge{O_{l_1}}{G_{i-1}} = \marge{A_{l_1}}{G_{i-1}} + \marge{A_{l_2}}{G_{i-1}}.\]
% Therefore, the lemma holds in this case.

\textbf{Case 1: $c_{l_1}^* \le c_{l_2}^*$; left half part in Fig.~\ref{fig:gdtwo}.}

First, we bound $\marge{O_{l_1}}{A_{l_2}}$.
Order the elements in $O_{l_1}\setminus A_{l_1, c_{l_1}^*}$ as $\{o_1, o_2, \ldots\}$ such that $o_j \not \in A_{l_1, c_{l_1}^*+j}$.
(Refer to the gray block with a dotted edge in the top left corner of Fig.~\ref{fig:gdtwo} for $O_{l_1}$.
If $c_{l_1}^*+j$ is greater than the number of elements added to $A_{l_1}$,
$A_{l_1, c_{l_1}^*+j}$ refers to $A_{l_1}$.)
Note that, since $A_{l_1, c_{l_1}^*} \subseteq O_{l_1}$,
it follows that $|O_{l_1}\setminus A_{l_1, c_{l_1}^*}| \le m - c_{l_1}^*$.

When $1 \le j \le |A_{l_2}\setminus G_{i-1}| - c_{l_1}^*$,
since each $o_j$ is either added to $A_{l_1}$ or not in any solution set by Claim~\ref{claim:par-A}
and $\tau_{l_2}$ is initialized with the maximum marginal gain $M$,
$o_j$ is not considered to be added to $A_{l_2}$ with threshold value $\tau_{l_2}^{c_{l_1}^* + j}/(1-\epsi')$.
Therefore, it holds that 
\begin{equation}\label{inq:tgtwo-case2-1}
\marge{o_j}{A_{l_2, c_{l_1}^*+j-1}} < \frac{\tau_{l_2}^{c_{l_1}^* + j}}{1-\epsi'} \le \frac{\marge{a_{l_2, c_{l_1}^* + j}}{A_{l_2, c_{l_1}^* + j-1}}}{1-\epsi'}, \forall 1\le j\le |A_{l_2}\setminus G_{i-1}|-c_{l_1}^*.
\end{equation}

When $|A_{l_2}\setminus G_{i-1}| < m$ and $|A_{l_2}\setminus G_{i-1}|-c_{l_1}^* < j\le m-c_{l_1}^*$,
this iteration ends with $\tau_{l_2} < \frac{\epsi' M}{k}$ and
$o_j$ is never considered to be added to $A_{l_2}$.
Thus, it holds that
\begin{equation}\label{inq:tgtwo-case2-3}
\marge{o_j}{A_{l_2}} < \frac{\epsi' M}{(1-\epsi')k}, 
\forall |A_{l_2}\setminus G_{i-1}|-c_{l_1}^* < j \le m-c_{l_1}^*.
\end{equation}

Then,
\begin{align*}
\marge{O_{l_1}}{A_{l_2}} &\le \marge{A_{l_1, c_{l_1}^*}}{A_{l_2}}  + \sum_{o_j \in O_{l_1}\setminus A_{l_1, c_{l_1}^*}}\marge{o_j}{A_{l_2}} \tag{Proposition~\ref{prop:sum-marge}}\\
&\le \marge{A_{l_1, c_{l_1}^*}}{G_{i-1}} + \sum_{j = 1}^{|A_{l_2}\setminus G_{i-1}|-c_{l_1}^*}\marge{o_j}{A_{l_2, , c_{l_1}^*+j-1}} + \sum_{j=|A_{l_2}\setminus G_{i-1}|-c_{l_1}^*+1}^{m-c_{l_1}^*} \marge{o_j}{A_{l_2}} \tag{submodularity}\\
&\le \ff{A_{l_1, c_{l_1}^*}}-\ff{G_{i-1}} + \sum_{j = 1}^{|A_{l_2}\setminus G_{i-1}|-c_{l_1}^*}\frac{\marge{a_{l_2, c_{l_1}^*+j}}{A_{l_2, , c_{l_1}^*+j-1}}}{1-\epsi'} + \frac{\epsi' M}{(1-\epsi')\ell} \tag{Inequality~\eqref{inq:tgtwo-case2-1} and~\eqref{inq:tgtwo-case2-3}}\\
& \le \ff{A_{l_1, c_{l_1}^*}}-\ff{G_{i-1}} + \frac{\ff{A_{l_2}} - \ff{A_{l_2, c_{l_1}^*}}}{1-\epsi'} + \frac{\epsi' M}{(1-\epsi')\ell} \numberthis \label{inq:tgtwo-case2-6}
\end{align*}

Similarly, we bound $\marge{O_{l_2}}{A_{l_1}}$ below.
Order the elements in $O_{l_2}\setminus A_{l_2, c_{l_1}^*}$ as $\{o_1, o_2, \ldots\}$ such that $o_j \not \in A_{l_2, c_{l_1}^*+j-1}$.
(See the gray block with a dotted edge in the bottom left corner of Fig.~\ref{fig:gdtwo} for $O_{l_2}$.
If $c_{l_1}^*+j-1$ is greater than the number of elements added to $A_{l_2}$,
$A_{l_2, c_{l_1}^*+j-1}$ refers to $A_{l_2}$.)
Note that, since $A_{l_2, c_{l_1}^*} \subseteq O_{l_2}$,
it follows that $|O_{l_2}\setminus A_{l_2, c_{l_1}^*}| \le m - c_{l_1}^*$.

When $1 \le j \le |A_{l_1}\setminus G_{i-1}|-c_{l_1}^*$,
since each $o_j$ is either added to $A_{l_2}$ or not in any solution set by Claim~\ref{claim:par-A}
and $\tau_{l_1}$ is initialized with the maximum marginal gain $M$,
$o_j$ is not considered to be added to $A_{l_1}$ with threshold value $\tau_{l_1}^{c_{l_1}^* + j}/(1-\epsi')$.
Therefore, it holds that 
\begin{equation}\label{inq:tgtwo-case2-4}
\marge{o_j}{A_{l_1, c_{l_1}^*+j-1}} < \frac{\tau_{l_1}^{c_{l_1}^* + j}}{1-\epsi'} \le \frac{\marge{a_{l_1, c_{l_1}^* + j}}{A_{l_1, c_{l_1}^* + j-1}}}{1-\epsi'}, \forall 1\le j\le |A_{l_2}\setminus G_{i-1}|-c_{l_1}^*.
\end{equation}

When $|A_{l_1}\setminus G_{i-1}| < m$ and $|A_{l_1}\setminus G_{i-1}|-c_{l_1}^* < j\le m-c_{l_1}^*$,
this iteration ends with $\tau_{l_1} < \frac{\epsi' M}{k}$
and $o_j$ is never considered to be added to $A_{l_1}$.
Thus, it holds that
\begin{equation}\label{inq:tgtwo-case2-5}
\marge{o_j}{A_{l_1}} < \frac{\epsi' M}{(1-\epsi')k}, \forall |A_{l_1}\setminus G_{i-1}|-c_{l_1}^* < j \le m-c_{l_1}^*.
\end{equation}

Then,
\begin{align*}
\marge{O_{l_2}}{A_{l_1}} &\le \marge{A_{l_2, c_{l_1}^*}}{A_{l_1}}  + \sum_{o_j \in O_{l_2}\setminus A_{l_2, c_{l_1}^*}}\marge{o_j}{A_{l_1}} \tag{Proposition~\ref{prop:sum-marge}}\\
&\le \marge{A_{l_2, c_{l_1}^*}}{G_{i-1}} + \sum_{j = 1}^{|A_{l_1}\setminus G_{i-1}|-c_{l_1}^*}\marge{o_j}{A_{l_1, c_{l_1}^*+j-1}} + \sum_{j=|A_{l_1}\setminus G_{i-1}|-c_{l_1}^*+1}^{m-c_{l_1}^*} \marge{o_j}{A_{l_1}} \tag{submodularity}\\
&\le \ff{A_{l_2, c_{l_1}^*}}-\ff{G_{i-1}} + \sum_{j = 1}^{|A_{l_1}\setminus G_{i-1}|-c_{l_1}^*}\frac{\marge{a_{l_1, c_{l_1}^*+j}}{A_{l_1, , c_{l_1}^*+j-1}}}{1-\epsi'} + \frac{\epsi' M}{(1-\epsi')\ell} \tag{Inequality~\eqref{inq:tgtwo-case2-4} and~\eqref{inq:tgtwo-case2-5}}\\
& \le \ff{A_{l_2, c_{l_1}^*}}-\ff{G_{i-1}} + \frac{\ff{A_{l_1}} - \ff{A_{l_1, c_{l_1}^*}}}{1-\epsi'} + \frac{\epsi' M}{(1-\epsi')\ell} \numberthis \label{inq:tgtwo-case2-7}
\end{align*}

By Inequalities~\eqref{inq:tgtwo-case2-6} and~\eqref{inq:tgtwo-case2-7},
\begin{align*}
\marge{O_{l_1}}{A_{l_2}}+\marge{O_{l_2}}{A_{l_1}}
\le \frac{1}{1-\epsi'}\left(\marge{A_{l_1}}{G_{i-1}}+\marge{A_{l_2}}{G_{i-1}}\right)
+\frac{2\epsi' M}{(1-\epsi')\ell}
\end{align*}

Thus, the lemma holds in this case.

\textbf{Case 2: $c_{l_1}^* > c_{l_2}^*$; right half part in Fig.~\ref{fig:gdtwo}.}

First, we bound $\marge{O_{l_1}}{A_{l_2}}$.
Order the elements in $O_{l_1}\setminus A_{l_1, c_{l_2}^*+1}$ as $\{o_1, o_2, \ldots\}$ such that $o_j \not \in A_{l_1, c_{l_1}^*+j}$.
(Refer to the gray block with a dotted edge in the top right corner of Fig.~\ref{fig:gdtwo} for $O_{l_1}$.
If $c_{l_1}^*+j$ is greater than the number of elements added to $A_{l_1}$,
$A_{l_1, c_{l_1}^*+j}$ refers to $A_{l_1}$.)
Note that, since $A_{l_1, c_{l_2}^*+1} \subseteq O_{l_1}$,
it follows that $|O_{l_1}\setminus A_{l_1, c_{l_2}^*+1}| \le m - c_{l_2}^*-1$.

When $1 \le j \le |A_{l_2}\setminus G_{i-1}| - c_{l_2}^* - 1$,
since each $o_j$ is either added to $A_{l_1}$ or not in any solution set by Claim~\ref{claim:par-A}
and $\tau_{l_2}$ is initialized with the maximum marginal gain $M$,
$o_j$ is not considered to be added to $A_{l_2}$ with threshold value $\tau_{l_2}^{c_{l_2}^* + j}/(1-\epsi')$.
Therefore, it holds that 
\begin{equation}\label{inq:tgtwo-case3-1}
\marge{o_j}{A_{l_2, c_{l_2}^*+j-1}} < \frac{\tau_{l_2}^{c_{l_2}^* + j}}{1-\epsi'} \le \frac{\marge{a_{l_2, c_{l_2}^* + j}}{A_{l_2, c_{l_2}^* + j-1}}}{1-\epsi'}, \forall 1\le j\le |A_{l_2}\setminus G_{i-1}| - c_{l_2}^* - 1.
\end{equation}

When $|A_{l_2}\setminus G_{i-1}| < m$ and $|A_{l_2}\setminus G_{i-1}|- c_{l_2}^* - 1 < j\le m- c_{l_2}^* - 1$,
this iteration ends with $\tau_{l_2} < \frac{\epsi' M}{k}$ and
$o_j$ is never considered to be added to $A_{l_2}$.
Thus, it holds that
\begin{equation}\label{inq:tgtwo-case3-3}
\marge{o_j}{A_{l_2}} < \frac{\epsi' M}{(1-\epsi')k}, 
\forall |A_{l_2}\setminus G_{i-1}|- c_{l_2}^* - 1 < j \le m- c_{l_2}^* - 1.
\end{equation}

Then,
\begin{align*}
\marge{O_{l_1}}{A_{l_2}} &\le \marge{A_{l_1, c_{l_2}^*}}{A_{l_2}}  + \sum_{o_j \in O_{l_1}\setminus A_{l_1, c_{l_2}^*+1}}\marge{o_j}{A_{l_2}} \tag{Proposition~\ref{prop:sum-marge}}\\
&\le \marge{A_{l_1, c_{l_2}^*}}{G_{i-1}} + \sum_{j = 1}^{|A_{l_2}\setminus G_{i-1}|- c_{l_2}^* - 1}\marge{o_j}{A_{l_2, , c_{l_2}^*+j-1}} + \sum_{j=|A_{l_2}\setminus G_{i-1}|- c_{l_2}^*}^{m- c_{l_2}^* - 1} \marge{o_j}{A_{l_2}} \tag{submodularity}\\
&\le \ff{A_{l_1, c_{l_2}^*}}-\ff{G_{i-1}} + \sum_{j = 1}^{|A_{l_2}\setminus G_{i-1}|- c_{l_2}^* - 1}\frac{\marge{a_{l_2, c_{l_2}^*+j}}{A_{l_2, , c_{l_2}^*+j-1}}}{1-\epsi'} + \frac{\epsi' M}{(1-\epsi')\ell} \tag{Inequality~\eqref{inq:tgtwo-case3-1} and~\eqref{inq:tgtwo-case3-3}}\\
& \le \ff{A_{l_1, c_{l_2}^*}}-\ff{G_{i-1}} + \frac{\ff{A_{l_2}} - \ff{A_{l_2, c_{l_2}^*}}}{1-\epsi'} + \frac{\epsi' M}{(1-\epsi')\ell} \numberthis \label{inq:tgtwo-case3-6}
\end{align*}

Similarly, we bound $\marge{O_{l_2}}{A_{l_1}}$ below.
Order the elements in $O_{l_2}\setminus A_{l_2, c_{l_2}^*}$ as $\{o_1, o_2, \ldots\}$ such that $o_j \not \in A_{l_2, c_{l_2}^*+j}$.
(See the gray block with a dotted edge in the bottom right corner of Fig.~\ref{fig:gdtwo} for $O_{l_2}$.
If $c_{l_2}^*+j$ is greater than the number of elements added to $A_{l_2}$,
$A_{l_2, c_{l_2}^*+j}$ refers to $A_{l_2}$.)
Note that, since $A_{l_2, c_{l_2}^*} \subseteq O_{l_2}$,
it follows that $|O_{l_2}\setminus A_{l_2, c_{l_2}^*}| \le m - c_{l_2}^*$.

When $1 \le j \le |A_{l_1}\setminus G_{i-1}|- c_{l_2}^* - 1$, 
since each $o_j$ is either added to $A_{l_2}$ or not in any solution set by Claim~\ref{claim:par-A}
and $\tau_{l_1}$ is initialized with the maximum marginal gain $M$,
$o_j$ is not considered to be added to $A_{l_1}$ with threshold value $\tau_{l_1}^{c_{l_2}^* + j+1}/(1-\epsi')$.
Therefore, it holds that 
\begin{equation}\label{inq:tgtwo-case3-4}
\marge{o_j}{A_{l_1, c_{l_2}^*+j}} < \frac{\tau_{l_1}^{c_{l_2}^* + j+1}}{1-\epsi'} \le \frac{\marge{a_{l_1, c_{l_2}^* + j+1}}{A_{l_1, c_{l_2}^* + j}}}{1-\epsi'}, \forall 1\le j\le |A_{l_2}\setminus G_{i-1}|- c_{l_2}^* - 1.
\end{equation}
If $|A_{l_1}\setminus G_{i-1}| = m$,
consider the last element $o_{m-c_{l_2}^*}$ in $O_{l_2}\setminus A_{l_2, c_{l_2}^*}$.
Since $o_{m-c_{l_2}^*} \not\in A_{l_2}$ and $o_{m-c_{l_2}^*} \not\in A_{l_1}$, $o_{m-c_{l_2}^*}$ is not considered to be added to 
$A_{l_1}$ with threshold value $\tau_{l_1}^j/(1-\epsi')$ for any $j \in [m]$.
Then,
\begin{equation}\label{inq:tgtwo-case3-2}
\marge{o_{m-c_{l_2}^*}}{A_{l_1}} < \frac{\sum_{j=1}^m \tau_{l_1}^j}{(1-\epsi')m}
\le \frac{\sum_{j=1}^m \marge{a_{l_1, j}}{A_{l_1, j-1}}}{(1-\epsi')m}
 = \frac{\marge{A_{l_1}}{G_{i-1}}}{(1-\epsi')m}.
\end{equation}

When $|A_{l_1}\setminus G_{i-1}| < m$ and $|A_{l_1}\setminus G_{i-1}|- c_{l_2}^* - 1 < j\le m- c_{l_2}^*$,
this iteration ends with $\tau_{l_1} < \frac{\epsi' M}{k}$
and $o_j$ is never considered to be added to $A_{l_1}$.
Thus, it holds that
\begin{equation}\label{inq:tgtwo-case3-5}
\marge{o_j}{A_{l_1}} < \frac{\epsi' M}{(1-\epsi')k}, 
\forall |A_{l_1}\setminus G_{i-1}|- c_{l_2}^* - 1 < j \le m- c_{l_2}^*.
\end{equation}

Then,
\begin{align*}
\marge{O_{l_2}}{A_{l_1}} &\le \marge{A_{l_2, c_{l_2}^*}}{A_{l_1}}  + \sum_{o_j \in O_{l_2}\setminus A_{l_2, c_{l_2}^*}}\marge{o_j}{A_{l_1}} \tag{Proposition~\ref{prop:sum-marge}}\\
&\le \marge{A_{l_2, c_{l_2}^*}}{G_{i-1}} + \sum_{j = 1}^{|A_{l_1}\setminus G_{i-1}|- c_{l_2}^* - 1}\marge{o_j}{A_{l_1, c_{l_2}^*+j-1}} + \sum_{j=|A_{l_1}\setminus G_{i-1}|- c_{l_2}^*}^{m} \marge{o_j}{A_{l_1}} \tag{submodularity}\\
&\le \ff{A_{l_2, c_{l_2}^*}}-\ff{G_{i-1}} + \sum_{j = 1}^{|A_{l_1}\setminus G_{i-1}|- c_{l_2}^* - 1}\frac{\marge{a_{l_1, c_{l_2}^*+j}}{A_{l_1, , c_{l_2}^*+j-1}}}{1-\epsi'}
+ \frac{\marge{A_{l_1}}{G_{i-1}}}{(1-\epsi')m}
+\frac{\epsi' M}{(1-\epsi')\ell} \tag{Inequalities~\eqref{inq:tgtwo-case3-4}-\eqref{inq:tgtwo-case3-5}}\\
& \le \ff{A_{l_2, c_{l_2}^*}}-\ff{G_{i-1}} + \frac{\ff{A_{l_1}} - \ff{A_{l_1, c_{l_2}^*}}}{1-\epsi'} + \frac{\marge{A_{l_1}}{G_{i-1}}}{(1-\epsi')m} + \frac{\epsi' M}{(1-\epsi')\ell} \numberthis \label{inq:tgtwo-case3-7}
\end{align*}

By Inequalities~\eqref{inq:tgtwo-case3-6} and~\eqref{inq:tgtwo-case3-7},
\begin{align*}
\marge{O_{l_1}}{A_{l_2}}+\marge{O_{l_2}}{A_{l_1}}
\le \frac{1}{1-\epsi'}\left(1+\frac{1}{m}\right)\left(\marge{A_{l_1}}{G_{i-1}}+\marge{A_{l_2}}{G_{i-1}}\right)
+\frac{2\epsi' M}{(1-\epsi')\ell}
\end{align*}

Thus, the lemma holds in this case.
\end{proof}

\begin{proof}[Proof of Lemma~\ref{lemma:tg-recur}]
Fix on $G_{i-1}$ at the beginning of this iteration,
Since $\{A_l\setminus G_{i-1}: l\in [\ell]\}$ are pairwise disjoint sets,
by Proposition~\ref{prop:sum-marge}, it holds that
\[\exc{\ff{O\cup G_i}}{G_{i-1}} = \frac{1}{\ell}\sum_{l\in [\ell]}\ff{O\cup A_l} \ge \left(1-\frac{1}{\ell}\right)\ff{O\cup G_{i-1}}.\]
Then, by unfixing $G_{i-1}$, the first inequality holds.

To prove the second inequality, also consider fix on $G_{i-1}$ at the beginning of iteration $i$.
Then,
\begin{align*}
\sum_{l\in [\ell]}\marge{O}{A_l} &\le \sum_{l_1\in [\ell]}\sum_{l_2\in [\ell]}\marge{O_{l_1}}{A_{l_2}}\tag{Proposition~\ref{prop:sum-marge}}\\
& = \sum_{l \in [\ell]}\marge{O_{l}}{A_{l}} + \sum_{1\le l_1< l_2 \le \ell} \left(\marge{O_{l_1}}{A_{l_2}} +\marge{O_{l_2}}{A_{l_1}}\right) \tag{Lemma~\ref{lemma:tg-par-A}}\\
& \le \sum_{l \in [\ell]}\left(\frac{\marge{A_{l}}{G_{i-1}}}{1-\epsi'}+\frac{\epsi' M}{(1-\epsi')\ell}\right)\\
&\hspace*{2em}+\sum_{1\le l_1< l_2 \le \ell} \left(\frac{1}{1-\epsi'}\left(1+\frac{1}{m}\right)\left(\marge{A_{l_1}}{G_{i-1}}
+\marge{A_{l_2}}{G_{i-1}}\right)
+\frac{2\epsi' M}{(1-\epsi')\ell}\right)\tag{Lemma~\ref{lemma:tg-par-A}}\\
&\le \frac{\ell}{1-\epsi'}\left(1+\frac{1}{m}\right)\sum_{l \in [\ell]}\marge{A_{l}}{G_{i-1}} + \frac{\epsi' \ell}{1-\epsi'}\ff{O}\tag{$M \le \ff{O}$}
\end{align*}
\begin{align*}
\Rightarrow \left(1+\frac{\ell}{1-\epsi'}\right)\left(1+\frac{1}{m}\right) \sum_{l\in [\ell]}\marge{A_l}{G_{i-1}} &\ge \sum_{l\in [\ell]}\ff{O\cup A_l} -\ell\ff{G_{i-1}} - \frac{\epsi' \ell}{1-\epsi'}\ff{O}\\
&\ge \left(\ell-1\right)\ff{O\cup G_{i-1}}-\ell\ff{G_{i-1}} - \frac{\epsi' \ell}{1-\epsi'}\ff{O}
\end{align*}
Thus,
\begin{align*}
&\exc{\ff{G_i} - \ff{G_{i-1}}}{G_{i-1}}  = \frac{1}{\ell}\sum_{l \in [\ell]}\marge{A_{l}}{G_{i-1}}\\
&\ge \frac{1}{1+\frac{\ell}{1-\epsi'}} \frac{m}{m+1}\left(\left(1-\frac{1}{\ell}\right)  \ff{O\cup G_{i-1}} - \ff{G_{i-1}} - \frac{\epsi'}{1-\epsi'}\ff{O}\right)\tag{Proposition~\ref{prop:sum-marge}}
\end{align*}
By unfixing $G_{i-1}$, the second inequality holds.
\end{proof}

\section{Analysis of Section~\ref{sec:ptg}} % (fold)
\label{apx:ptg}
In this section, we provide the analysis of our parallel algorithms
introduced in Section~\ref{sec:ptg}.
First, we provide the subroutines used for \ptgoneshort in Appendix~\ref{apx:subroutine}.
Then, we analyze \ptgoneshort, the main parallel procedure,
in Appendix~\ref{apx:ptgone}.
At last, we provide that analysis of $1/4$ and $1/e$ approximation algorithms
in Appendix~\ref{apx:ptgone-guarantee} and Appendix~\ref{apx:ptgtwo},
respectively.
\subsection{Subroutines}\label{apx:subroutine}
\begin{algorithm}[ht]
\Fn{\dist($\{V_l\}_{l\in [\ell]}$)}{
	\KwIn{$V_1, V_2, \ldots, V_\ell \subseteq \uni$}
	\Init{$\mathcal V_1, \mathcal V_2, \ldots, \mathcal V_\ell \gets \emptyset$, $I\gets [\ell]$}
	\For{$i\gets 1$ to $\ell$}{
		$j\gets\argmin_{j\in I}|V_j|$\label{line:dis-index}\;
		$\mathcal V_j \gets$ randomly select $\left\lfloor\frac{|V_j|}{\ell}\right\rfloor$ elements in $V_j\setminus \left(\bigcup_{l \in [\ell]}\mathcal V_j\right)$ \label{line:dis-select}\;
		$I\gets I-j$\;
	}
	\Return{$\left\{\mathcal V_l\right\}_{l\in [\ell]}$}
	}
\caption{Return $\ell$ pairwise disjoint subsets where $|\mathcal V_j| \ge \frac{|V_j|}{2\ell}$ for any $j \in [\ell]$ if $|V_j|\ge 2\ell$}
\label{alg:dist}
\end{algorithm}
\begin{lemma}\label{lemma:dist}
With input $\{V_l\}_{l\in [\ell]}$, where $|V_l|\ge 2\ell$ for each $l\in [\ell]$,
\dist returns $\ell$ pairwise disjoint sets $\{\mathcal V_l\}_{l\in [\ell]}$ \st
$\mathcal V_l\subseteq V_l$ and $|\mathcal V_j| \ge \frac{|V_j|}{2\ell}$.
\end{lemma}


\begin{algorithm}[ht]
\Fn{\prefix($f, \mathcal V, s, \tau, \epsi$)}{
	\KwIn{evaluation oracle $f:2^{\uni} \to \reals$, maximum size $s$, threshold $\tau$, error $\epsi$, candidate pool $\mathcal V$ where $\marge{x}{\emptyset} \ge \tau $ for any $ x\in \mathcal V$}
	\Init{$B[1:s]\gets [\textbf{none}, \ldots, \textbf{none}]$}
	$\mathcal V \gets\left\{v_1, v_2, \ldots\right\} \gets \textbf{random-permutation}(\mathcal V)$\label{line:prefix-permute}\;
	\For{$i\gets 1$ to $s$ in parallel}{
		$T_{i-1} \gets \left\{v_1, \ldots, v_{i-1}\right\}$\;
		\lIf{$\marge{v_i}{T_{i-1}} \ge \tau$}{$B[i]\gets \textbf{true}$}\label{line:prefix-B-true}
		\lElseIf{$\marge{v_i}{T_{i-1}} < 0$}{$B[i]\gets \textbf{false}$}\label{line:prefix-B-false}
	}
	$i^*\gets \max \{i: \#\text{\textbf{true}s in }B[1:i] \ge (1-\epsi) i\}$\label{line:prefix-istar}\;
	\Return{$i^*$, $B$}
}
\caption{Select a prefix of $\mathcal V$ \st its average marginal gain is greater than $(1-\epsi)\tau$, and with a probability of $1/2$, more than an $\epsi/2$-fraction of $\mathcal V$ has a marginal gain less than $\tau$ relative to the prefix.}
\label{alg:prefix}
\end{algorithm}
Since the procedure \prefix is identical to Lines 8-15 in \ts~\citep{Chen2024},
the following two lemmata hold in a manner similar to Lemma 4 and 5 in \citet{Chen2024}.
\begin{lemma}\label{lemma:prefix-filter}
In \prefix, given $\mathcal V$ after \textbf{random-permutation} in Line~\ref{line:prefix-permute},
let $D_i = \left\{x\in \mathcal V: \marge{x}{T_i} < \tau\right\}$.
It holds that $|D_0|=0$, $|D_{|\mathcal V|}| = |\mathcal V|$, and $|D_{i-1}|\le |D_i|$.
\end{lemma}
\begin{lemma}\label{lemma:prefix-prob}
In \prefix, following the definition of $D_i$ in Lemma~\ref{lemma:prefix-filter},
let $t = \min\{i: |D_i| \ge \epsi |\mathcal V|/2\}$.
It holds that $\prob{i^* < \min\{s,t\}} \le 1/2$.
\end{lemma}
As defined in Lemma~\ref{lemma:prefix-filter},
$D_i$ contains the elements in $\mathcal V$
which can be filtered out by threshold value $\tau$
regarding the prefix $T_i$.
Therefore, Lemma~\ref{lemma:prefix-prob} indicates that,
with a probability of at least $1/2$,
$i^* = s$ or
more than $\epsi/2$-fraction of $\mathcal V$
can be filtered out if prefix $T_{i^*}$ is added to the solution.

\begin{algorithm}[ht]
\Fn{\update($f, V, \tau, \epsi$)}{
\KwIn{evaluation oracle $f:2^{\uni} \to \reals$, candidate set $V$, threshold value $\tau$, error $\epsi$}
	\For(\tcp*[f]{Update candidate sets with threshold values}){$j\gets 1$ to $\ell$ in parallel}{
		$V \gets \left\{x\in V : \marge{x}{\emptyset} \ge \tau\right\}$ \label{line:update-filter}\;
		\While{$|V| = 0$}{
			$\tau \gets (1-\epsi)\tau$\;
			$V \gets \left\{x\in \uni : \marge{x}{\emptyset} \ge \tau\right\}$\;
		}
	}
	\Return{$V, \tau$}
}
\caption{Update candidate set $V$ with threshold value $\tau$}
\label{alg:update}
\end{algorithm}
\subsection{Analysis of Alg.~\ref{alg:ptgone}}\label{apx:ptgone}
We provide the guarantees achieved by \ptgoneshort as follows,
\begin{lemma}\label{lemma:ptgone}
With input $(f, m, \ell, \tau_{\min}, \epsi)$, \ptgone (Alg.~\ref{alg:ptgone})
runs in $\oh{\ell^2\epsi^{-2}\log(n)\log\left(\frac{M}{\tau_{\min}}\right)}$ adaptive rounds and $\oh{\ell^3 \epsi^{-2}n\log(n)\log\left(\frac{M}{\tau_{\min}}\right)}$ queries with a probability of $1-1/n$,
and terminates with $\{(A_l, A_l'): l\in [\ell]\}$ \st
{\small
\begin{enumerate}
\item $A_l'\subseteq A_l$, $\marge{A_l'}{\emptyset} \ge \marge{A_l}{\emptyset}, \forall 1\le l \le \ell$, and $\{A_l: l\in [\ell]\}$ are pairwise disjoint sets,
\item $\marge{O_{l}}{A_{l}}\le \frac{\marge{A_{l}'}{\emptyset}}{(1-\epsi)^2}+\frac{m\cdot\tau_{\min}}{1-\epsi}, \forall 1\le l \le \ell$,
\item $\marge{O_{l_2}}{A_{l_1}} + \marge{O_{l_1}}{A_{l_2}} \le 
\frac{1+\frac{1}{m}}{(1-\epsi)^2}\left(\marge{A_{l_1}'}{\emptyset}+\marge{A_{l_2}'}{\emptyset}\right) + \frac{2m\cdot \tau_{\min}}{1-\epsi}, \text{ if } O_{l_1} = O_{l_2}, \forall 1\le l_1 < l_2 \le \ell$,
\end{enumerate}}
where $O_l\subseteq \uni$, $|O_l| \le m$, and $O_l \cap A_j = \emptyset$ for each $j \neq l$.

Especially, when $\ell = 2$,
{\small
\begin{itemize}
	\item[4.] $\marge{S}{A_{1}} + \marge{S}{A_{2}} \le \frac{1}{(1-\epsi)^2}\left(\marge{A_{l_1}'}{\emptyset}+\marge{A_{l_2}'}{\emptyset}\right) + \frac{2m\cdot \tau_{\min}}{1-\epsi}, \forall S\subseteq \uni, |S| \le m$. 
\end{itemize}}
\end{lemma}
Before proving Lemma~\ref{lemma:ptgone}, we provide the following lemma regarding each iteration of \ptgone.
\begin{lemma}\label{lemma:tgone-iteration}
For any iteration of the while loop in \ptgone (Alg.~\ref{alg:ptgone}),
let $A_{l, 0}$, $A_{l, 0}'$, $V_{l, 0}$, $\tau_{l, 0}$ be the set and threshold value at the beginning,
and $A_l$, $A_l'$, $V_l$, $\tau_l$ be those at the end.
The following properties hold. 
\begin{enumerate}
\item With a probability of at least $1/2$,
there exists $l \in [\ell]$ \st $\tau_l < \tau_{l, 0}$
or $m_0 = 0$ or
$|V_l| \le \left(1-\frac{\epsi}{4\ell}\right)|V_{l, 0}|$.
\item $\{A_l: l\in [\ell]\}$ have the same size and are pairwise disjoint.  
% \item $V_l = \left\{x\in V\setminus\left(\bigcup_{i\in [\ell]} A_l\right) : \marge{x}{A_l} \ge \tau_l \right\}$  for all $l\in [\ell]$.
% \item For each $l \in [\ell]$ \st $\tau_l < \tau_{l,0}$,
% it holds that $\marge{x}{A_l} < \frac{\tau_l}{1-\epsi}$ for all $x\in V\setminus \left(\bigcup_{i\in [\ell]} A_l\right)$.
\item For each $x\in A_l\setminus A_{l,0}$, let $\tau_l^{(x)}$ be the threshold value when $x$ is added to the solution,
$A_{l, (x)}$ be the largest prefix of $A_l$ that do not include $x$,
and for any $j\in [\ell]$ and $j\neq l$,
$A_{j, (x)}$ be the prefix of $A_j$ with $|A_{l, (x)}|$ elements if $j < l$,
or with $|A_{l, (x)}|-1$ elements if $j > l$.
Then, for any $l\in [\ell]$, $x\in A_l\setminus A_{l,0}$,
and $y\in \uni\setminus \left(\bigcup_{j\in [\ell]} A_{j, (x)}\right)$,
it holds that $\marge{y}{A_{l, (x)}} < \frac{\tau_l^{(x)}}{1-\epsi}$.
\item $A_l'\subseteq A_l$, $\marge{A_l'}{A_{l, 0}'} \ge \marge{A_l}{A_{l, 0}}$,
and $\marge{A_l'}{A_{l, 0}'}\ge (1-\epsi)\sum_{x \in A_l\setminus A_{l, 0}}\tau_l^{(x)}$ for all $l\in [\ell]$.
\end{enumerate}
\end{lemma}
\begin{proof}[Proof of Lemma~\ref{lemma:tgone-iteration}]
\textbf{Proof of Property 1.}
At the beginning of the iteration, if there exists $l\in I$ \st $|V_{l, 0}| < 2\ell$,
then either $\tau_{l, 0}$ is decreased to $\tau_l$ and $V_l$ is updated accordingly, 
or an element $x_l$ from $V_{l, 0}$ is added to $A_j$ and $A_j'$
and subsequently removed from $V_{l, 0}$. 
This implies that
\[|V_l| \le |V_{l,0}| -1 < \left(1-\frac{1}{2\ell}\right)|V_{l,0}|.\]
Property 1 holds in this case.

Otherwise, for all $l\in I$, it holds that $|V_{l, 0}| \ge 2\ell$,
and the algorithm proceeds to execute Lines~\ref{line:tgone-dist}-\ref{line:tgone-update-size}.
By Lemma~\ref{lemma:dist}, in Line~\ref{line:tgone-dist}, 
$|\mathcal V_l| \ge \frac{|V_{l, 0}|}{2\ell}$ for each $l\in I$ .
Consider the index $j\in I$ where $i_j^* = i^*$.
Then, $O_j$ consists of the first $i^*$ elements in $\mathcal V_j$
by Line~\ref{line:tgone-subset}.
By Lemma~\ref{lemma:prefix-prob},
with probability greater than $1/2$,
either $i^* = m_0$ or at least an $\frac{\epsi}{2}$-fraction
of elements $x\in \mathcal V_j$ satisfy $\marge{x}{A_j}< \tau_{j,0}$.
Consequently, either $m_0 = 0$ after Line~\ref{line:tgone-update-size},
or, after the \update procedure in Line~\ref{line:tgone-update},
one of the following holds:
$|V_l| \le \left(1-\frac{\epsi}{4\ell}\right)|V_{l, 0}|$,
or $\tau_{j} < \tau_{j, 0}$.
Therefore, Property 1 holds in this case. 

\textbf{Proof of Property 2.}
At any iteration of the while,
either $|I|$ different elements or $|I|$ pairwise disjoint sets with same size $i^*$
are added to solution sets $\{A_l: l\in I\}$.
Therefore, Property 2 holds.

\textbf{Proof of Property 3.}
At any iteration, if $\tau_l$ is not updated on Line~\ref{line:tgone-update-2},
then prior to this iteration, all the elements outside of the solutions
have marginal gain less than $\frac{\tau_l^{(x)}}{1-\epsi}$.
Thus, for any $x \in A_l\setminus A_{l, 0}$, $y\in \uni\setminus \left(\bigcup_{j\in [\ell]} A_{j, 0}\right)$,
it holds that $\marge{y}{A_{l, (x)}} < \frac{\tau_l^{(x)}}{1-\epsi}$
by submodularity. Property 3 holds in this case.

Otherwise, if $\tau_l$ is updated on Line~\ref{line:tgone-update-2},
only one element is added to each solution set during this iteration.
Let $x = A_l\setminus A_{l, 0}$.
For any $j\in [\ell]$ and $j\neq l$,
it holds that $A_{j, (x)} = A_j$ if $j < l$,
or $A_{j, (x)} = A_{j, 0}$ if $j > l$.
Since elements are added to each pair of solutions in sequence within the for loop
in Lines~\ref{line:tgone-for-begin}-\ref{line:tgone-for-end},
by the \update procedure,
for any $y \in \uni \setminus \left(\bigcup_{j\in [\ell]} A_{j, (x)}\right)$,
it holds that $\marge{y}{A_{l, (x)}} < \frac{\tau_l^{(x)}}{1-\epsi}$.
Therefore, Property 3 also holds in this case.

\textbf{Proof of Property 4.}
First, we prove $A_l'\subseteq A_l$ by induction.
At the beginning of the algorithm,
$A_l'$ and $A_l$ are initialized as empty sets.
Clearly, the property holds in the base case.
Then, suppose that $A_{l,0}'\subseteq A_{l,0}$.
There are three possible cases of updating $A_{l,0}'$ and $A_{l,0}$ at any iteration:
1) $A_l'=A_{l,0}'$ and $A_l = A_{l,0}$,
2) $A_l' = A_{l,0}' + x_l$ and $A_l = A_{l,0} + x_l$ in Line~\ref{line:tgone-update-A},
or 3) $A_l' = A_{l,0}' \cup S_l'$ and $A_l = A_{l,0} \cup S_l$ in Line~\ref{line:tgone-update-A-2}.
Clearly, $A_l'\subseteq A_l$ holds in all cases.

Next, we prove the rest of Property 4.

If $A_l'=A_{l,0}'$ and $A_l = A_{l,0}$, 
then $\marge{A_l'}{A_{l, 0}'} = \marge{A_l}{A_{l, 0}}=0$.
Property 4 holds.

If $A_{l,0}'$ and $A_{l,0}$ are updated in Line~\ref{line:tgone-update-A},
by submodularity, $\marge{A_l'}{A_{l, 0}'} = \marge{x_l}{A_{l, 0}'} \ge \marge{x_l}{A_{l, 0}}=\marge{A_l}{A_{l, 0}} \ge \tau_l^{(x_{l})}$.
Therefore, Property 4 also holds.

If $A_{l,0}'$ and $A_{l,0}$ are updated in Line~\ref{line:tgone-update-A-2},
we know that $A_l' = A_{l,0}' \cup S_l'$ and $A_l = A_{l,0} \cup S_l$.
Suppose the elements in $S_l$ and $S_l'$ retain their original order within $\mathcal V_l$.
For each $x\in S_l$, let $S_{l,(x)}$, $\mathcal V_{l, (x)}$ and $A_{l, (x)}$
be the largest prefixes of $S_l$, $\mathcal V_l$ and $A_l$ that do not include $x$, respectively.
Moreover, let $S_{l, (x)}' = S_{l, (x)}\cap S_l'$ and $A_{l, (x)}' = A_{l, (x)}\cap A_l'$.
Say an element $x\in S_l$ \textbf{true} if $B_l[(x)] = \textbf{true}$,
where $B_l[(x)]$ is the $i$-th element in $B_l$ if $x$ is the $i$-th element in $\mathcal V_l$.
Similarly, say an element $x\in S_l$ \textbf{false} if $B_l[(x)] = \textbf{false}$,
and \textbf{none} otherwise.

Following the above definitions, for any \textbf{true} or \textbf{none} element $x\in S_l$,
by Line~\ref{line:tgone-subset}, it holds that $S_{l, (x)} \subseteq \mathcal V_{l, (x)}$.
Then, by Line~\ref{line:prefix-B-true} and submodularity,
\begin{equation*}
\marge{x}{A_{l, (x)}} = \marge{x}{A_{l, 0} \cup S_{l, (x)}}
\ge \marge{x}{A_{l, 0} \cup \mathcal V_{l, (x)}} \ge \left\{
\begin{aligned}
&\tau_l^{(x)}, \text{ if } x \text{ is \textbf{true} element}\\
&0, \text{ if } x \text{ is \textbf{none} element}
\end{aligned}\right.
\end{equation*}
Since \textbf{true} elements are selected at first and $i_j^*\ge i^*$,
there are more than $(1-\epsi)i^*$ \textbf{true} elements in $S_l$.
Therefore,
\begin{align*}
\marge{A_l'}{A_{l, 0}'} &= \sum_{x\in A_l'\setminus A_{l, 0}', x\text{ is \textbf{true} element}} \marge{x}{A_{l, (x)}'}
+\sum_{x\in A_l'\setminus A_{l, 0}', x\text{ is \textbf{none} element}} \marge{x}{A_{l, (x)}'}\\
&\ge \sum_{x\in A_l'\setminus A_{l, 0}', x\text{ is \textbf{true} element}} \marge{x}{A_{l, (x)}}
+\sum_{x\in A_l'\setminus A_{l, 0}', x\text{ is \textbf{none} element}} \marge{x}{A_{l, (x)}}\\
&\ge (1-\epsi) |A_l\setminus A_{l, 0}| \tau_l^{(x)}, \text{for any } x\in A_l\setminus A_{l, 0}\\
&= (1-\epsi)\sum_{x\in A_l\setminus A_{l, 0}} \tau_l^{(x)}.
\end{align*}
The third part of Property 4 holds.

To prove the second part of Property 4, consider any \textbf{false} element $x\in S_l$.
By Line~\ref{line:tgone-subset}, it holds that $\mathcal V_{l, (x)} = S_{l, (x)}$.
Then, by Line~\ref{line:prefix-B-false}
\begin{equation}\label{ineq:tgone-false}
\marge{x}{A_{l, (x)}} = \marge{x}{A_{l, 0} \cup S_{l, (x)}}
= \marge{x}{A_{l, 0} \cup \mathcal V_{l, (x)}} < 0.
\end{equation}
By Line~\ref{line:tgone-subset-2}, all the elements in $S_l\setminus S_l'$ are
\textbf{false} elements.
Then,
\begin{align*}
\ff{A_l} - \ff{A_{l, 0}} &= \sum_{x\in S_l'}\marge{x}{A_{l, (x)}} + \sum_{x\in S_l\setminus S_l'}\marge{x}{A_{l, (x)}}\\
&< \sum_{x\in S_l'}\marge{x}{A_{l, (x)}} \tag{Inequality~\ref{ineq:tgone-false}}\\
&\le \sum_{x\in S_l'}\marge{x}{A_{l, (x)}'} \tag{Submodularity}\\
& = \ff{A_l'} - \ff{A_{l, 0}'}.
\end{align*}
\end{proof}

By Lemma~\ref{lemma:tgone-iteration},
we are ready to prove Lemma~\ref{lemma:ptgone}.
\begin{proof}[Proof of Lemma~\ref{lemma:ptgone}]
\textbf{Proof of Property 1.}
By Property 2 and 3 in Lemma~\ref{lemma:tgone-iteration},
this property holds immediately.

\textbf{Proof of Property 2.}
For any $l\in [\ell]$,
since $O_l\cap A_j  = \emptyset$ for each $j\neq l$,
$O_l\setminus A_l$ is outside of any solution set.
If $|A_l| = m$, by Property 4 of Lemma~\ref{lemma:tgone-iteration},
\begin{align*}
\marge{O_l}{A_l} &\le \sum_{y \in O_l\setminus A_l}\marge{y}{A_l}\\
&\le \sum_{x \in A_l}\tau_l^{(x)}/(1-\epsi)\tag{Property 3 in Lemma~\ref{lemma:tgone-iteration}}\\
& \le \frac{\marge{A_l'}{\emptyset}}{(1-\epsi)^2}.\tag{Property 5 in Lemma~\ref{lemma:tgone-iteration}}
\end{align*}
If $|A_l| < m$, then the threshold value for solution $A_l$ has been updated to be less than $\tau_{\min}$.
Therefore, for any $y\in O_l\setminus A_l$,
it holds that $\marge{y}{A_l} < \frac{\tau_{\min}}{1-\epsi}$.
Then,
\begin{align*}
\marge{O_l}{A_l} \le \sum_{y \in O_l\setminus A_l}\marge{y}{A_l}
\le \frac{m\tau_{\min}}{1-\epsi}.
\end{align*}
Therefore, Property 2 holds by summing the above two inequalities.

\textbf{Proof of Property 3 and 4.}
Let $a_{l, j}$ be the $j$-th element added to $A_l$,
$\tau_l^j$ be the threshold value of $\tau_l$ when $a_{l, j}$ is added to $A_l$,
and $A_{l, j}$ be $A_l$ after $a_{l, j}$ is added to $A_l$.
Let $c_l^* = \max\{c\in [m]:A_{l, c}\subseteq O_l\}$.

In the following, we analyze these properties together under two cases,
similar to the analysis of Alg.~\ref{alg:ptgtwo}.
For the case where $\ell = 2$, 
let $O_{1} = S\setminus A_2$, and $O_2 = S\setminus A_1$,
unifying the notations used in Property 3 and 4.
Note that, the only difference between the two analyses is that,
a small portion (no more than $\epsi$ fraction) of elements in the solution returned by Alg.~\ref{alg:ptgone}
do not have marginal gain greater than the threshold value.

\textbf{Case 1: $c_{l_1}^*\le c_{l_2}^*$; left half part in Fig.~\ref{fig:gdtwo}.}

First, we bound $\marge{O_{l_1}}{A_{l_2}}$.
Consider elements in $A_{l_1, c_{l_1}^*} \subseteq O_{l_1}$.
Let $A_{l_1, c_{l_1}^*} = \{o_1, \ldots, o_{c_{l_1}^*}\}$.
For each $1\le j \le c_{l_1}^*$, 
since $o_j$ is added to $A_{l_1}$ with threshold value $\tau_{l_1}^{j}$
and the threshold value starts from the maximum marginal gain $M$,
clearly, $o_j$ has been filtered out with threshold value $\tau_{l_1}^{j}/(1-\epsi)$.
Then, by submodularity,
\begin{equation}\label{inq:ptgone-case1-1}
\marge{A_{l_1, c_{l_1}^*} }{A_{l_2}} \le \marge{A_{l_1, c_{l_1}^*} }{\emptyset}
 = \sum_{j=1}^{c_{l_1}^*}\marge{o_j}{A_{l_1, j-1}}
 \le \sum_{j=1}^{c_{l_1}^*} \tau_{l_1}^{j}/(1-\epsi).
\end{equation}

Next, consider the elements in $O_{l_1}\setminus A_{l_1, c_{l_1}^*}$.
Order the elements in $O_{l_1}\setminus A_{l_1, c_{l_1}^*}$ as $\{o_1, o_2, \ldots\}$ such that $o_j \not \in A_{l_1, c_{l_1}^*+j}$.
(Refer to the gray block with a dotted edge in the top left corner of Fig.~\ref{fig:gdtwo} for $O_{l_1}$.
If $c_{l_1}^*+j$ is greater than $|A_{l_1}|$,
$A_{l_1, c_{l_1}^*+j}$ refers to $A_{l_1}$.)
Note that, since $A_{l_1, c_{l_1}^*} \subseteq O_{l_1}$,
it follows that $|O_{l_1}\setminus A_{l_1, c_{l_1}^*}| \le m - c_{l_1}^*$.

When $1 \le j \le |A_{l_2}| - c_{l_1}^*$,
since each $o_j$ is either added to $A_{l_1}$ or not in any solution set
and $\tau_{l_2}$ is initialized with the maximum marginal gain $M$,
$o_j$ is not considered to be added to $A_{l_2}$ with threshold value $\tau_{l_2}^{c_{l_1}^* + j}/(1-\epsi)$
by Property 3 of Lemma~\ref{lemma:tgone-iteration}.
Therefore, it holds that 
\begin{equation}\label{inq:ptgone-case1-2}
\marge{o_j}{A_{l_2, c_{l_1}^*+j-1}} < \frac{\tau_{l_2}^{c_{l_1}^* + j}}{1-\epsi} , \forall 1\le j\le |A_{l_2}|-c_{l_1}^*.
\end{equation}

When $|A_{l_2}| < m$ and $|A_{l_2}|-c_{l_1}^* < j\le m-c_{l_1}^*$,
the algorithm ends with $\tau_{l_2} < \tau_{\min}$ and
$o_j$ is never considered to be added to $A_{l_2}$.
Thus, it holds that
\begin{equation}\label{inq:ptgone-case1-3}
\marge{o_j}{A_{l_2}} < \frac{\tau_{\min}}{1-\epsi}, 
\forall |A_{l_2}|-c_{l_1}^* < j \le m-c_{l_1}^*.
\end{equation}

Then,
\begin{align*}
\marge{O_{l_1}}{A_{l_2}} &\le \marge{A_{l_1, c_{l_1}^*}}{A_{l_2}}  + \sum_{o_j \in O_{l_1}\setminus A_{l_1, c_{l_1}^*}}\marge{o_j}{A_{l_2}} \tag{Proposition~\ref{prop:sum-marge}}\\
&\le \marge{A_{l_1, c_{l_1}^*}}{\emptyset} + \sum_{j = 1}^{|A_{l_2}|-c_{l_1}^*}\marge{o_j}{A_{l_2, , c_{l_1}^*+j-1}} + \sum_{j=|A_{l_2}|-c_{l_1}^*+1}^{m-c_{l_1}^*} \marge{o_j}{A_{l_2}} \tag{submodularity}\\
&\le \sum_{j=1}^{c_{l_1}^*} \frac{\tau_{l_1}^{j}}{1-\epsi} + \sum_{j=c_{l_1}^*+1}^{|A_{l_2}|} \frac{\tau_{l_2}^{j}}{1-\epsi} + \frac{m \cdot \tau_{\min}}{1-\epsi} \numberthis \label{inq:ptgone-case1-4}
\end{align*}
where the last inequality follows from 
Inequalities~\eqref{inq:ptgone-case1-1}-\eqref{inq:ptgone-case1-3}.

Similarly, we bound $\marge{O_{l_2}}{A_{l_1}}$ below.
Consider elements in $A_{l_2, c_{l_1}^*} \subseteq O_{l_2}$.
Let $A_{l_2, c_{l_1}^*} = \{o_1, \ldots, o_{c_{l_1}^*}\}$.
For each $1\le j \le c_{l_1}^*$, 
since $o_j$ is added to $A_{l_2}$ with threshold value $\tau_{l_2}^{j}$
and the threshold value starts from the maximum marginal gain $M$,
clearly, $o_j$ has been filtered out with threshold value $\tau_{l_2}^{j}/(1-\epsi)$.
Then, by submodularity,
\begin{equation}\label{inq:ptgone-case1-5}
\marge{A_{l_2, c_{l_1}^*} }{A_{l_1}} \le \marge{A_{l_2, c_{l_1}^*} }{\emptyset}
 = \sum_{j=1}^{c_{l_1}^*}\marge{o_j}{A_{l_2, j-1}}
 \le \sum_{j=1}^{c_{l_1}^*} \tau_{l_2}^{j}/(1-\epsi).
\end{equation}

Next, consider the elements in $O_{l_2}\setminus A_{l_2, c_{l_1}^*}$.
Order the elements in $O_{l_2}\setminus A_{l_2, c_{l_1}^*}$ as $\{o_1, o_2, \ldots\}$ such that $o_j \not \in A_{l_2, c_{l_1}^*+j-1}$.
(See the gray block with a dotted edge in the bottom left corner of Fig.~\ref{fig:gdtwo} for $O_{l_2}$.
If $c_{l_1}^*+j-1$ is greater than $|A_{l_2}|$,
$A_{l_2, c_{l_1}^*+j-1}$ refers to $A_{l_2}$.)
Note that, since $A_{l_2, c_{l_1}^*} \subseteq O_{l_2}$,
it follows that $|O_{l_2}\setminus A_{l_2, c_{l_1}^*}| \le m - c_{l_1}^*$.

When $1 \le j \le |A_{l_1}|-c_{l_1}^*$,
since each $o_j$ is either added to $A_{l_2}$ or not in any solution set,
and $\tau_{l_1}$ is initialized with the maximum marginal gain $M$,
$o_j$ is not considered to be added to $A_{l_1}$ with threshold value $\tau_{l_1}^{c_{l_1}^* + j}/(1-\epsi)$
by Property 3 of Lemma~\ref{lemma:tgone-iteration}.
Therefore, it holds that 
\begin{equation}\label{inq:ptgone-case1-6}
\marge{o_j}{A_{l_1, c_{l_1}^*+j-1}} < \frac{\tau_{l_1}^{c_{l_1}^* + j}}{1-\epsi} , \forall 1\le j\le |A_{l_2}|-c_{l_1}^*.
\end{equation}

When $|A_{l_1}| < m$ and $|A_{l_1}|-c_{l_1}^* < j\le m-c_{l_1}^*$,
this iteration ends with $\tau_{l_1} < \tau_{\min}$
and $o_j$ is never considered to be added to $A_{l_1}$.
Thus, it holds that
\begin{equation}\label{inq:ptgone-case1-7}
\marge{o_j}{A_{l_1}} < \frac{\tau_{\min}}{1-\epsi}, \forall |A_{l_1}|-c_{l_1}^* < j \le m-c_{l_1}^*.
\end{equation}

Then,
\begin{align*}
\marge{O_{l_2}}{A_{l_1}} &\le \marge{A_{l_2, c_{l_1}^*}}{A_{l_1}}  + \sum_{o_j \in O_{l_2}\setminus A_{l_2, c_{l_1}^*}}\marge{o_j}{A_{l_1}} \tag{Proposition~\ref{prop:sum-marge}}\\
&\le \marge{A_{l_2, c_{l_1}^*}}{\emptyset} + \sum_{j = 1}^{|A_{l_1}|-c_{l_1}^*}\marge{o_j}{A_{l_1, c_{l_1}^*+j-1}} + \sum_{j=|A_{l_1}|-c_{l_1}^*+1}^{m-c_{l_1}^*} \marge{o_j}{A_{l_1}} \tag{submodularity}\\
&\le \sum_{j=1}^{c_{l_1}^*} \frac{\tau_{l_2}^{j}}{1-\epsi} + \sum_{j=c_{l_1}^*+1}^{|A_{l_1}|} \frac{\tau_{l_1}^{j}}{1-\epsi} + \frac{m\cdot \tau_{\min}}{1-\epsi}  \numberthis \label{inq:ptgone-case1-8}
\end{align*}
where the last inequality follows from Inequalities~\ref{inq:ptgone-case1-5}-\ref{inq:ptgone-case1-7}.

By Inequalities~\eqref{inq:ptgone-case1-4} and~\eqref{inq:ptgone-case1-8},
\begin{equation}\label{inq:ptgone-case1-final}
\marge{O_{l_1}}{A_{l_2}}+\marge{O_{l_2}}{A_{l_1}}
\le \sum_{j=1}^{|A_{l_1}|} \frac{\tau_{l_1}^{j}}{1-\epsi} + \sum_{j=1}^{|A_{l_2}|} \frac{\tau_{l_2}^{j}}{1-\epsi} + \frac{2m\cdot \tau_{\min}}{1-\epsi}
\end{equation}

\textbf{Case 2: $c_{l_1}^* > c_{l_2}^*$; right half part in Fig.~\ref{fig:gdtwo}.}

First, we bound $\marge{O_{l_1}}{A_{l_2}}$.
Consider elements in $A_{l_1, c_{l_2}^*+1} \subseteq O_{l_1}$.
Let $A_{l_1, c_{l_2}^*+1} = \{o_1, \ldots, o_{c_{l_2}^*+1}\}$.
For each $1\le j \le c_{l_2}^*+1$, 
since $o_j$ is added to $A_{l_1}$ with threshold value $\tau_{l_1}^{j}$
and the threshold value starts from the maximum marginal gain $M$,
clearly, $o_j$ has been filtered out with threshold value $\tau_{l_1}^{j}/(1-\epsi)$.
Then, by submodularity,
\begin{equation}\label{inq:ptgone-case2-1}
\marge{A_{l_1, c_{l_2}^*+1} }{A_{l_2}} \le \marge{A_{l_1, c_{l_2}^*+1} }{\emptyset}
 = \sum_{j=1}^{c_{l_2}^*+1}\marge{o_j}{A_{l_1, j-1}}
 \le \sum_{j=1}^{c_{l_2}^*+1} \tau_{l_1}^{j}/(1-\epsi).
\end{equation}

Next, consider the elements in $O_{l_1}\setminus A_{l_1, c_{l_2}^*+1}$.
Order the elements in $O_{l_1}\setminus A_{l_1, c_{l_2}^*+1}$ as $\{o_1, o_2, \ldots\}$ such that $o_j \not \in A_{l_1, c_{l_1}^*+j}$.
(Refer to the gray block with a dotted edge in the top right corner of Fig.~\ref{fig:gdtwo} for $O_{l_1}$.
If $c_{l_1}^*+j$ is greater than $|A_{l_1}|$,
$A_{l_1, c_{l_1}^*+j}$ refers to $A_{l_1}$.)
Note that, since $A_{l_1, c_{l_2}^*+1} \subseteq O_{l_1}$,
it follows that $|O_{l_1}\setminus A_{l_1, c_{l_2}^*+1}| \le m - c_{l_2}^*-1$.

When $1 \le j \le |A_{l_2}| - c_{l_2}^* - 1$,
since each $o_j$ is either added to $A_{l_1}$ or not in any solution set
and $\tau_{l_2}$ is initialized with the maximum marginal gain $M$,
$o_j$ is not considered to be added to $A_{l_2}$ with threshold value $\tau_{l_2}^{c_{l_2}^* + j}/(1-\epsi)$.
Therefore, it holds that 
\begin{equation}\label{inq:ptgone-case2-2}
\marge{o_j}{A_{l_2, c_{l_2}^*+j-1}} < \frac{\tau_{l_2}^{c_{l_2}^* + j}}{1-\epsi}, \forall 1\le j\le |A_{l_2}\setminus G_{i-1}| - c_{l_2}^* - 1.
\end{equation}

When $|A_{l_2}| < m$ and $|A_{l_2}|- c_{l_2}^* - 1 < j\le m- c_{l_2}^* - 1$,
this iteration ends with $\tau_{l_2} < \tau_{\min}$ and
$o_j$ is never considered to be added to $A_{l_2}$.
Thus, it holds that
\begin{equation}\label{inq:ptgone-case2-3}
\marge{o_j}{A_{l_2}} < \frac{\tau_{\min}}{1-\epsi}, 
\forall |A_{l_2}|- c_{l_2}^* - 1 < j \le m- c_{l_2}^* - 1.
\end{equation}

Then,
\begin{align*}
\marge{O_{l_1}}{A_{l_2}} &\le \marge{A_{l_1, c_{l_2}^*}}{A_{l_2}}  + \sum_{o_j \in O_{l_1}\setminus A_{l_1, c_{l_2}^*+1}}\marge{o_j}{A_{l_2}} \tag{Proposition~\ref{prop:sum-marge}}\\
&\le \marge{A_{l_1, c_{l_2}^*}}{\emptyset} + \sum_{j = 1}^{|A_{l_2}|- c_{l_2}^* - 1}\marge{o_j}{A_{l_2, c_{l_2}^*+j-1}} + \sum_{j=|A_{l_2}|- c_{l_2}^*}^{m- c_{l_2}^* - 1} \marge{o_j}{A_{l_2}} \tag{submodularity}\\
&\le \sum_{j=1}^{c_{l_2}^*+1} \tau_{l_1}^{j}/(1-\epsi)
+ \sum_{j = c_{l_2}^*+1}^{|A_{l_2}|} \tau_{l_2}^{j}/(1-\epsi) + \frac{m \cdot \tau_{\min}}{1-\epsi}
 \numberthis \label{inq:ptgone-case2-4}
\end{align*}
where the last inequality follows from 
Inequalities~\eqref{inq:ptgone-case2-1}-\eqref{inq:ptgone-case2-3}.

Similarly, we bound $\marge{O_{l_2}}{A_{l_1}}$ below.
Consider elements in $A_{l_1, c_{l_2}^*} \subseteq O_{l_2}$.
Let $A_{l_2, c_{l_2}^*} = \{o_1, \ldots, o_{c_{l_2}^*}\}$.
For each $1\le j \le c_{l_2}^*$, 
since $o_j$ is added to $A_{l_2}$ with threshold value $\tau_{l_2}^{j}$
and the threshold value starts from the maximum marginal gain $M$,
clearly, $o_j$ has been filtered out with threshold value $\tau_{l_2}^{j}/(1-\epsi)$.
Then, by submodularity,
\begin{equation}\label{inq:ptgone-case2-5}
\marge{A_{l_2, c_{l_2}^*} }{A_{l_1}} \le \marge{A_{l_2, c_{l_2}^*} }{\emptyset}
 = \sum_{j=1}^{c_{l_2}^*}\marge{o_j}{A_{l_2, j-1}}
 \le \sum_{j=1}^{c_{l_2}^*} \tau_{l_2}^{j}/(1-\epsi).
\end{equation}

Next, consider the elements in $O_{l_2}\setminus A_{l_2, c_{l_2}^*}$.
Order these elements as $\{o_1, o_2, \ldots\}$ such that $o_j \not \in A_{l_2, c_{l_2}^*+j}$.
(See the gray block with a dotted edge in the bottom right corner of Fig.~\ref{fig:gdtwo} for $O_{l_2}$.
If $c_{l_2}^*+j$ is greater than the number of elements added to $A_{l_2}$,
$A_{l_2, c_{l_2}^*+j}$ refers to $A_{l_2}$.)
Note that, since $A_{l_2, c_{l_2}^*} \subseteq O_{l_2}$,
it follows that $|O_{l_2}\setminus A_{l_2, c_{l_2}^*}| \le m - c_{l_2}^*$.

Furthermore, for the case where $\ell  = 2$, as considered in Property 4,
we have $O_1 = S\setminus A_2$ and $O_2 = S\setminus A_1$ for a given $S\subseteq \uni$
where $|S| \le m$.
Since $c_{l_1}^* > c_{l_2}^* \ge 0$,
it follows that $c_{l_1}^*\ge 1$, which implies $|O_2| = |S\setminus A_1|\le m-1$.
In this case, it holds that $|O_{l_2}\setminus A_{l_2, c_{l_2}^*}| \le m - c_{l_2}^*-1$.

When $1 \le j \le |A_{l_1}|- c_{l_2}^* - 1$, 
since each $o_j$ is either added to $A_{l_2}$ or not in any solution set by Claim~\ref{claim:par-A}
and $\tau_{l_1}$ is initialized with the maximum marginal gain $M$,
$o_j$ is not considered to be added to $A_{l_1}$ with threshold value $\tau_{l_1}^{c_{l_2}^* + j+1}/(1-\epsi)$.
Therefore, it holds that 
\begin{equation}\label{inq:ptgone-case2-6}
\marge{o_j}{A_{l_1, c_{l_2}^*+j}} < \frac{\tau_{l_1}^{c_{l_2}^* + j+1}}{1-\epsi}, \forall 1\le j\le |A_{l_2}|- c_{l_2}^* - 1.
\end{equation}

If $|A_{l_1}| = m$,
consider the last element $o_{m-c_{l_2}^*}$ in $O_{l_2}\setminus A_{l_2, c_{l_2}^*}$.
Since $o_{m-c_{l_2}^*} \not\in A_{l_2}$ and $o_{m-c_{l_2}^*} \not\in A_{l_1}$, $o_{m-c_{l_2}^*}$ is not considered to be added to 
$A_{l_1}$ with threshold value $\tau_{l_1}^j/(1-\epsi)$ for any $j \in [m]$.
Then,
\begin{equation}\label{inq:ptgone-case2-7}
\marge{o_{m-c_{l_2}^*}}{A_{l_1}} < \frac{\sum_{j=1}^m \tau_{l_1}^j}{(1-\epsi)m}.
\end{equation}
Else, $|A_{l_1}| < m$ and 
this iteration ends with $\tau_{l_1} < \frac{\epsi M}{k}$.
For any $|A_{l_1}|- c_{l_2}^* - 1 < j\le m- c_{l_2}^*$,
$o_j$ is never considered to be added to $A_{l_1}$.
Thus, it holds that
\begin{equation}\label{inq:ptgone-case2-8}
\marge{o_j}{A_{l_1}} < \frac{\tau_{\min}}{1-\epsi}, 
\forall |A_{l_1}|- c_{l_2}^* - 1 < j \le m- c_{l_2}^*.
\end{equation}

Then,
\begin{align*}
&\marge{O_{l_2}}{A_{l_1}} \le \marge{A_{l_2, c_{l_2}^*}}{A_{l_1}}  + \sum_{o_j \in O_{l_2}\setminus A_{l_2, c_{l_2}^*}}\marge{o_j}{A_{l_1}} \tag{Proposition~\ref{prop:sum-marge}}\\
&\le \left\{
\begin{aligned}
&\marge{A_{l_2, c_{l_2}^*}}{\emptyset} + \sum_{j = 1}^{|A_{l_1}\setminus G_{i-1}|- c_{l_2}^* - 1}\marge{o_j}{A_{l_1, c_{l_2}^*+j-1}} + \sum_{j=|A_{l_1}\setminus G_{i-1}|- c_{l_2}^*}^{m-c_{l_2}^*} \marge{o_j}{A_{l_1}}, &&\text{ if } |O_{l_2}| = m\\
&\marge{A_{l_2, c_{l_2}^*}}{\emptyset} + \sum_{j = 1}^{|A_{l_1}\setminus G_{i-1}|- c_{l_2}^* - 1}\marge{o_j}{A_{l_1, c_{l_2}^*+j-1}} + \sum_{j=|A_{l_1}\setminus G_{i-1}|- c_{l_2}^*}^{m-c_{l_2}^*-1} \marge{o_j}{A_{l_1}}, &&\text{otherwise}
\end{aligned}
\right. \tag{submodularity}\\
&\le\left\{
\begin{aligned}
	&\sum_{j=1}^{c_{l_2}^*} \frac{\tau_{l_2}^j}{1-\epsi} + \sum_{j=c_{l_2}^*+2}^{|A_{l_2}|} \left(1+\frac{1}{m}\right) \frac{\tau_{l_1}^j}{1-\epsi} + \frac{m\cdot \tau_{\min}}{1-\epsi}, &&\text{ if } |O_{l_2}| = m\\
	&\sum_{j=1}^{c_{l_2}^*} \frac{\tau_{l_2}^j}{1-\epsi} + \sum_{j=c_{l_2}^*+2}^{|A_{l_2}|} \frac{\tau_{l_1}^j}{1-\epsi} + \frac{m\cdot \tau_{\min}}{1-\epsi}, &&\text{otherwise}
\end{aligned}
\right. \numberthis \label{inq:ptgone-case2-9}
\end{align*}
where the last inequality follows from Inequalities~\eqref{inq:ptgone-case2-5}-\eqref{inq:ptgone-case2-8}.

By Inequalities~\eqref{inq:ptgone-case2-4} and~\eqref{inq:ptgone-case2-9},
\begin{equation}\label{inq:ptgone-case2-final}
\marge{O_{l_1}}{A_{l_2}}+\marge{O_{l_2}}{A_{l_1}}
\le \left\{
\begin{aligned}
	& \left(1+\frac{1}{m}\right) \frac{1}{1-\epsi}\left(\sum_{j=1}^{|A_{l_1}|} \tau_{l_1}^{j} + \sum_{j=1}^{|A_{l_2}|} \tau_{l_2}^{j}\right) + \frac{2m\cdot \tau_{\min}}{1-\epsi}, &&\text{ if } |O_{l_2}| = m \\
	&  \frac{1}{1-\epsi}\left(\sum_{j=1}^{|A_{l_1}|} \tau_{l_1}^{j} + \sum_{j=1}^{|A_{l_2}|} \tau_{l_2}^{j}\right) + \frac{2m\cdot \tau_{\min}}{1-\epsi}, &&\text{ otherwise }
\end{aligned}
\right.
\end{equation}

Overall, in both cases, if $|O_{l_2}| = m$,
\begin{align*}
\marge{O_{l_1}}{A_{l_2}}+\marge{O_{l_2}}{A_{l_1}}
&\le \left(1+\frac{1}{m}\right) \frac{1}{1-\epsi}\left(\sum_{j=1}^{|A_{l_1}|} \tau_{l_1}^{j} + \sum_{j=1}^{|A_{l_2}|} \tau_{l_2}^{j}\right) + \frac{2m\cdot \tau_{\min}}{1-\epsi} \tag{Inequalities~\eqref{inq:ptgone-case1-final} and~\eqref{inq:ptgone-case2-final}}\\
&\le \left(1+\frac{1}{m}\right)\frac{1}{(1-\epsi)^2}\left(\marge{A_{l_1}'}{\emptyset}+\marge{A_{l_2}'}{\emptyset}\right) + \frac{2m\cdot \tau_{\min}}{1-\epsi} \tag{Property 4 of Lemma~\ref{lemma:tgone-iteration}}
\end{align*}
Otherwise, if $|O_{l_2}| < m$,
\begin{align*}
\marge{O_{l_1}}{A_{l_2}}+\marge{O_{l_2}}{A_{l_1}}
&\le \frac{1}{1-\epsi}\left(\sum_{j=1}^{|A_{l_1}|} \tau_{l_1}^{j} + \sum_{j=1}^{|A_{l_2}|} \tau_{l_2}^{j}\right) + \frac{2m\cdot \tau_{\min}}{1-\epsi} \tag{Inequalities~\eqref{inq:ptgone-case1-final} and~\eqref{inq:ptgone-case2-final}}\\
&\le \frac{1}{(1-\epsi)^2}\left(\marge{A_{l_1}'}{\emptyset}+\marge{A_{l_2}'}{\emptyset}\right) + \frac{2m\cdot \tau_{\min}}{1-\epsi} \tag{Property 4 of Lemma~\ref{lemma:tgone-iteration}}
\end{align*}
Property (3) and (4) hold.

\textbf{Proof of Adaptivity and Query Complexity.}
Note that, at the beginning of every iteration,
for any $j\in I$, $V_j$ contains all the elements outside of all solutions that has marginal gain greater than $\tau_j$ with respect to solution $A_j$.
Say an iteration \textit{successful} if either
1) algorithm terminates after this iteration because of $m_0=0$,
2) all the elements in $V_j$ can be filtered out at the end of this iteration
and the value of $\tau_j$ decreases,
or 3) the size of $V_j$ decreases by a factor of $1-\frac{\epsi}{4\ell}$.
Then, by Property 1 of Lemma~\ref{lemma:tgone-iteration},
with a probability of at least $1/2$,
the iteration is successful.
Furthermore, if $\tau_j$ is less than $\tau_{\min}$,
$j$ will be removed from $I$ and 
solutions $A_j$ and $A_j'$ won't be updated anymore.

For each $j \in [\ell]$,
there are at most $\log_{1-\epsi}\left(\frac{\tau_{\min}}{M}\right) \le \epsi^{-1}\log\left(\frac{M}{\tau_{\min}}\right)$ possible threshold values.
And, for each threshold value, with at most 
$\log_{1-\frac{\epsi}{4\ell}}\left(\frac{1}{n}\right) \le 4\ell\epsi^{-1}\log(n)$
successful iterations regarding solution $A_j$,
the threshold value $\tau_j$ will decrease
or the algorithm terminates because of $m_0=0$.
Overall, with at most $4\ell^2\epsi^{-2}\log(n)\log\left(\frac{M}{\tau_{\min}}\right)$
successful iterations,
the algorithm terminates because of $m_0=0$ or $I=\emptyset$.

Next, we prove that, after $N=4\left(\log(n)+ 4\ell^2\epsi^{-2}\log(n)\log\left(\frac{M}{\tau_{\min}}\right)\right)$ iterations,
with a probability of $1-\frac{1}{n}$,
there exists at least $4\ell^2\epsi^{-2}\log(n)\log\left(\frac{M}{\tau_{\min}}\right)$
successful iterations,
or equivalently, the algorithm terminates.
Let $X$ be the number of successful iterations.
Then, $X$ can be regarded as a sum of $N$ dependent Bernoulli trails,
where the success probability is larger than $1/2$.
Let $Y$ be a sum of $N$ independent Bernoulli trials,
where the success probability is equal to $1/2$.
Then, the probability that the algorithm terminates with at most $N$ iterations can be bounded as follows,
\begin{align*}
\prob{\#\text{iterations} > N} &\le \prob{X \le 4\ell^2\epsi^{-2}\log(n)\log\left(\frac{M}{\tau_{\min}}\right)} \\
& \overset{(a)}{\le} \prob{Y \le 4\ell^2\epsi^{-2}\log(n)\log\left(\frac{M}{\tau_{\min}}\right)}\tag{Lemma~\ref{lemma:indep}}\\
&\le e^{- \frac{N}{4}\left(1-\frac{8\ell^2\epsi^{-2}\log(n)\log\left(\frac{M}{\tau_{\min}}\right)}{N}\right)^2} \tag{Lemma~\ref{lemma:chernoff}}\\
&= e^{-\frac{\left(4\log(n)+ 8\ell^2\epsi^{-2}\log(n)\log\left(\frac{M}{\tau_{\min}}\right)\right)^2}{16\left(\log(n)+ 4\ell^2\epsi^{-2}\log(n)\log\left(\frac{M}{\tau_{\min}}\right)\right)}} \le \frac{1}{n}.
\end{align*}
Therefore, with a probability of $1-\frac{1}{n}$,
the algorithm terminates with $\oh{\ell^2\epsi^{-2}\log(n)\log\left(\frac{M}{\tau_{\min}}\right)}$ iterations of the while loop.

In Alg.~\ref{alg:ptgone}, oracle queries occur during calls
to \update and \prefix 
on Line~\ref{line:tgone-update-2},~\ref{line:tgone-prefix} and~\ref{line:tgone-update}.
The \prefix algorithm,
with input $(f, \mathcal V, s, \tau, \epsi)$,
operates with $1$ adaptive rounds
and at most $|\mathcal V|$ queries.
The \update algorithm,
with input $(f, V_0, \tau_0, \epsi)$, 
outputs $(V, \tau)$
with $1+\log_{1-\epsi}\left(\frac{\tau}{\tau_0}\right)$ adaptive rounds
and at most $|V| + n\log_{1-\epsi}\left(\frac{\tau}{\tau_0}\right)$ queries.
Here, $\log_{1-\epsi}\left(\frac{\tau}{\tau_0}\right)$ equals the number of iterations in the while loop within \update.
Notably, every iteration is successful,
as the threshold value is updated.
Consequently, we can regard an iteration of the while loop in \update
as a separate iteration of the while loop in Alg.~\ref{alg:ptgone},
where such iteration only update one threshold value $\tau_j$ and its corresponding candidate set $V_j$.
So, each redefined iteration has no more than $2$ adaptive rounds,
and then the adaptivity of the algorithm should be no more than 
the number of successful iterations, which is
$\oh{\ell^2\epsi^{-2}\log(n)\log\left(\frac{M}{\tau_{\min}}\right)}$.
Since there are at most $\ell n$ queries at each adaptive rounds,
the query complexity is bounded by $\oh{\ell^3\epsi^{-2}n\log(n)\log\left(\frac{M}{\tau_{\min}}\right)}$.

% Next, we consider the query complexity of the algorithm. 
% Let $V_{j, i}$ be the set $V_j$ at the beginning of
% $i$-th redefined iteration of the while loop.
% Note that a redefined iteration of the while loop in Alg.~\ref{alg:ptgone} corresponds either to an iteration
% where none of the threshold values are updated
% or to an iteration of the while loop in \update.
% During every redefined iteration, 
% there are at most $2|V_{j, i}|$ queries if $\tau_j$ is not updated regrading solution $A_j$,
% or $|V_{j, i}| = n$ queries if $\tau_j$ is updated.

% In the worst case, at every successful iteration,
% only the set $V_j$ with minimum size decreases by a factor of $1-\frac{\epsi}{4\ell}$.
% Thus, the worst case scenario follows the following steps:
% 1) each $V_j$ starts from $\uni$
% and only one of $V_j$ decreases with a factor of $1-\frac{\epsi}{4\ell}$
% until $\tau_j$ is updated;
% 2) each $V_j$ becomes $\uni$ again and step (1) is repeated until
% all $\tau_j$ are below $\tau_{\min}$.
% Recall that, with at most $4\ell\epsi^{-1}\log(n)$ successful iterations
% regarding solution $A_j$, the threshold value $\tau_j$ will decrease
% or the algorithm terminates because of $m_0=0$.
% Moreover, for each $j\in [\ell]$,
% there are at most $\epsi^{-1}\log\left(\frac{M}{\tau_{\min}}\right)$
% possible threshold values
% resulting in at most $\ell\epsi^{-1}\log\left(\frac{M}{\tau_{\min}}\right)$ repeats of step (1).
% Let $Y_i$ be the number of iterations between 
% the $(i-1)$-th success and $i$-th success.
% By Lemma~\ref{lemma:indep}, since an iteration success with 
% a probability of $1/2$, it holds that $\ex{Y_i} \le 2$.
% Then, the expected number of iterations after $4\ell\epsi^{-1}\log(n)$ successful iterations can be bounded as follows,
% \begin{align*}
% 	\ex{\sum_{i=1}^{4\ell\epsi^{-1}\log(n)} Y_i} \le 8\ell\epsi^{-1}\log(n).
% \end{align*}
% Therefore, the query complexity of the algorithm can be bounded as follows,
% \begin{align*}
% 	\ex{\text{Queries}} & \le \sum_{i =1}^{N} \sum_{j = 1}^\ell 2|V_{j, i}|\\
% 	& \le \ell\epsi^{-1}\log\left(\frac{M}{\tau_{\min}}\right) \cdot
% 	\ex{\sum_{i=1}^{4\ell\epsi^{-1}\log(n)} Y_i\cdot 2\ell n}\\
% 	&\le 16 \ell^3 \epsi^{-1}n\log(n)\log\left(\frac{M}{\tau_{\min}}\right).
% \end{align*}
\end{proof}

\subsection{Analysis of Theorem~\ref{thm:ptgone} in Section~\ref{sec:ptg}}
\label{apx:ptgone-guarantee}
In this section, we provide the analysis of the parallel $1/4-\epsi$ approximation algorithm.
\thmptgone*
\begin{proof}[Proof of Theorem~\ref{thm:ptgone}]
The adaptivity and query complextiy are quite straightforward.
In the following, we will analyze the approximation ratio.

Let $S = O$ in Lemma~\ref{lemma:ptgone},
it holds that
\begin{align}
	& \ff{A_l'} \ge \ff{A_l}, \forall l = 1,2 \label{inq:ptg-1}\\
	& A_1 \cap A_2 = \emptyset \label{inq:ptg-2}\\
	& \marge{O}{A_1} + \marge{O}{A_2} \le \frac{1}{(1-\epsi)^2}\left(\ff{A_1'} + \ff{A_2'} \right) + \frac{2\epsi M}{1-\epsi}\label{inq:ptg-3}
\end{align}
Then,
\begin{align*}
	\ff{O} &\le \ff{O\cup A_1} + \ff{O\cup A_2} \tag{Submodularity, Nonnegativity, Inequality~\eqref{inq:ptg-2}} \\
	&\le \ff{A_1} + \ff{A_2} + \frac{1}{(1-\epsi)^2}\left(\ff{A_1'} + \ff{A_2'} \right) + \frac{2\epsi M}{1-\epsi} \tag{Inequality~\eqref{inq:ptg-3}}\\
	&\le 2\left(1+\frac{1}{(1-\epsi)^2}\right)\ff{G} + \frac{2\epsi}{1-\epsi}\ff{O} \tag{Inequality~\eqref{inq:ptg-1} and $G = \argmax\{\ff{A_1'}, \ff{A_2'}\}$}\\
	\Rightarrow \ff{G} &\ge \frac{(1-3\epsi)(1-\epsi)}{2\left((1-\epsi)^2 + 1+\frac{1}{k}\right)}\ff{O}\ge \left(\frac{1}{4}-\epsi\right)\ff{O} 
\end{align*}
\end{proof}

\subsection{Pseudocode and Analysis of Theorem~\ref{thm:ptgtwo} in Section~\ref{sec:ptg}}
\label{apx:ptgtwo}
\begin{algorithm}[ht]
\Fn{\ptgtwo($f, k, \epsi$)}{
	\KwIn{evaluation oracle $f:2^{\uni} \to \reals$, 
        constraint $k$, constant $\ell$, error $\epsi$}
	\Init{$G\gets \emptyset, \epsi' \gets \frac{\epsi}{2}, m\gets \left\lfloor \frac{k}{\ell} \right\rfloor, M\gets \max_{x\in\uni}\ff{\{x\}}, \tau_{\min}\gets \frac{\epsi'M}{k}$}
	\For{$i\gets 1$ to $\ell$}{
		$\{A_l': l\in [\ell]\} \gets \ptgone(f_{G}, m, \ell, \tau_{\min}, \epsi')$\;
		$G\gets$ a random set in $\{G\cup A_l': l\in [\ell]\}$\;
	}
	\Return{$G$}
	}
\caption{A randomized $(1/e-\epsi)$-approximation algorithm with $\oh{\ell^{3}\epsi^{-2}\log(n)\log(k)}$ adaptivity and $\oh{\ell^4\epsi^{-2}n\log(n)\log(k)}$ query complexity}\label{alg:ptg}
\label{alg:ptgtwo}
\end{algorithm}
In this section, we provide the pseudocode of the parallel $1/e-\epsi$ approximation algorithm with its analysis.

First, we provide the following lemma which provides a lower bound
on the gains achieved after every iteration in Alg.~\ref{alg:ptgtwo}.
\begin{lemma}\label{lemma:ptgtwo-recur}
For any iteration $i$ of the outer for loop in Alg.~\ref{alg:ptgtwo},
it holds that
\begin{align*}
\text{1) } & \ex{\ff{O\cup G_i}}\ge \left(1-\frac{1}{\ell}\right) \ex{\ff{O\cup G_{i-1}}}\\
\text{2) } & \ex{\ff{G_i} - \ff{G_{i-1}}}
\ge\frac{1}{1+\frac{\ell}{(1-\epsi')^2}}\left(1-\frac{1}{m+1}\right)\left(\left(1-\frac{1}{\ell}\right)  \ex{\ff{O\cup G_{i-1}}} - \ex{\ff{G_{i-1}}} - \frac{\epsi'}{1-\epsi'}\ff{O}\right).
\end{align*}
\end{lemma}
\begin{proof}[Proof of Lemma~\ref{lemma:ptgtwo-recur}]
Fix on $G_{i-1}$ at the beginning of this iteration,
Since $\{A_l: l\in [\ell]\}$ are pairwise disjoint sets,
by Proposition~\ref{prop:sum-marge}, it holds that
\[\exc{\ff{O\cup G_i}}{G_{i-1}} = \frac{1}{\ell}\sum_{l\in [\ell]}\ff{O\cup G_{i-1}\cup A_l} \ge \left(1-\frac{1}{\ell}\right)\ff{O\cup G_{i-1}}.\]
Then, by unfixing $G_{i-1}$, the first inequality holds.

To prove the second inequality, also consider fix on $G_{i-1}$ at the beginning of iteration $i$.
By Lemma~\ref{lemma:ptgone},
$\{A_l: l\in [\ell]\}$ are paiewise disjoint sets,
and the following inequalities hold,
\begin{align}
&A_l'\subseteq A_l, \marge{A_l'}{\emptyset} \ge \marge{A_l}{\emptyset}, \forall 1\le l \le \ell \label{inq:ptgtwo-1}\\
&\marge{O_{l}}{A_{l}}\le \frac{\marge{A_{l}'}{\emptyset}}{(1-\epsi')^2}+\frac{\epsi' M}{(1-\epsi')\ell}, \forall 1\le l \le \ell \label{inq:ptgtwo-2}\\
&\marge{O_{l_2}}{A_{l_1}} + \marge{O_{l_1}}{A_{l_2}} \le \frac{1+\frac{1}{m}}{(1-\epsi')^2}\left(\marge{A_{l_1}'}{\emptyset}+\marge{A_{l_2}'}{\emptyset}\right) + \frac{2\epsi' M}{(1-\epsi')\ell}, \forall 1\le l_1 < l_2 \le \ell \label{inq:ptgtwo-3}
\end{align}
Then,
\begin{align*}
\sum_{l\in [\ell]}\marge{O}{A_l\cup G_{i-1}} &\le \sum_{l_1\in [\ell]}\sum_{l_2\in [\ell]}\marge{O_{l_1}}{A_{l_2}\cup G_{i-1}}\tag{Proposition~\ref{prop:sum-marge}}\\
& = \sum_{l \in [\ell]}\marge{O_{l}}{A_{l}\cup G_{i-1}} + \sum_{1\le l_1< l_2 \le \ell} \left(\marge{O_{l_1}}{A_{l_2}\cup G_{i-1}} +\marge{O_{l_2}}{A_{l_1}\cup G_{i-1}}\right) \tag{Lemma~\ref{lemma:tg-par-A}}\\
& \le \sum_{l \in [\ell]}\left(\frac{\marge{A_{l}'}{G_{i-1}}}{(1-\epsi')^2}+\frac{\epsi' M}{(1-\epsi')\ell}\right)\\
&\hspace*{2em}+\sum_{1\le l_1< l_2 \le \ell} \left(\frac{\left(1+\frac{1}{m}\right)}{(1-\epsi')^2}\left(\marge{A_{l_1}'}{G_{i-1}}
+\marge{A_{l_2}'}{G_{i-1}}\right)
+\frac{2\epsi' M}{(1-\epsi')\ell}\right)\tag{Inequalities~\eqref{inq:ptgtwo-2} and~\eqref{inq:ptgtwo-3}}\\
&\le \frac{\ell}{(1-\epsi')^2}\left(1+\frac{1}{m}\right)\sum_{l \in [\ell]}\marge{A_{l}'}{G_{i-1}} + \frac{\epsi' \ell}{1-\epsi'}\ff{O}\tag{$M \le \ff{O}$}
\end{align*}
\begin{align*}
\Rightarrow \left(1+\frac{\ell}{(1-\epsi')^2}\right)\left(1+\frac{1}{m}\right) \sum_{l\in [\ell]}\marge{A_l'}{G_{i-1}} &\ge \sum_{l\in [\ell]}\ff{O\cup A_l\cup G_{i-1}} -\ell\ff{G_{i-1}} - \frac{\epsi' \ell}{1-\epsi'}\ff{O} \tag{Inequality~\eqref{inq:ptgtwo-1}}\\
&\ge \left(\ell-1\right)\ff{O\cup G_{i-1}}-\ell\ff{G_{i-1}} - \frac{\epsi' \ell}{1-\epsi'}\ff{O}
\end{align*}
Thus,
\begin{align*}
&\exc{\ff{G_i} - \ff{G_{i-1}}}{G_{i-1}}  = \frac{1}{\ell}\sum_{l \in [\ell]}\marge{A_{l}'}{G_{i-1}}\\
&\ge \frac{1}{1+\frac{\ell}{(1-\epsi')^2}} \frac{m}{m+1}\left(\left(1-\frac{1}{\ell}\right)  \ff{O\cup G_{i-1}} - \ff{G_{i-1}} - \frac{\epsi'}{1-\epsi'}\ff{O}\right)\tag{Proposition~\ref{prop:sum-marge}}
\end{align*}
By unfixing $G_{i-1}$, the second inequality holds.
\end{proof}
\thmptgtwo*
\begin{proof}[Proof of Theorem~\ref{thm:ptgtwo}]
Since the algorithm contains a for loop
which runs \ptgone $\ell = \oh{1/\epsi}$ times,
by Lemma~\ref{lemma:ptgone},
the adaptivity, query complexity and success probability holds immediately.

Next, we provide the analysis of approximation ratio.
By solving the recurrence in Lemma~\ref{lemma:ptgtwo-recur},
we calculate the approximation ratio of the algorithm as follows,
\begin{align*}
&\ex{\ff{G_{i}}}  \ge \left(1-\frac{1}{\ell}\right) \ex{\ff{G_{i-1}}}
+ \frac{1}{1+\frac{\ell}{(1-\epsi')^2}}\left(1-\frac{1}{m+1}\right)\left(\left(1-\frac{1}{\ell}\right)^i - \frac{\epsi'}{1-\epsi'}\right)\ff{O}\\
\Rightarrow& \ex{\ff{G_\ell}} \ge \frac{\ell}{1+\frac{\ell}{(1-\epsi')^2}}\left(1-\frac{1}{m+1}\right)\left(\left(1-\frac{1}{\ell}\right)^\ell - \frac{\epsi'}{1-\epsi'}\left(1-\left(1-\frac{1}{\ell}\right)^\ell\right)\right)\ff{O}\\
&\hspace*{4em} \ge \frac{\ell-1}{1+\frac{\ell}{(1-\epsi')^2}}\left(1-\frac{1}{m+1}\right)\left(e^{-1} - \frac{\epsi'}{1-\epsi'}\left(1-e^{-1}\right)\right)\ff{O}\\
&\hspace*{4em} \ge \frac{1}{1-\frac{\ell}{k}}\left((1-\epsi')^2 - \frac{2}{\ell}\right)\left(1-\frac{\ell}{k}\right)^2\left(e^{-1} - \frac{\epsi'}{1-\epsi'}\left(1-e^{-1}\right)\right) \ff{O}\\
% &\hspace*{4em} \ge \frac{1}{1-\frac{\ell}{k}}\left(1-\epsi' - \frac{2}{\ell}\right)\left(1-\frac{2\ell}{k}\right)\left(e^{-1} - \frac{\epsi'}{1-\epsi'}\left(1-e^{-1}\right)\right) \ff{O}\\
&\hspace*{4em} \ge \frac{1}{1-\frac{\ell}{k}}\left((1-\epsi')^2 - \frac{2}{\ell}-\frac{2(1-\epsi')^2 \ell}{k}\right)\left(e^{-1} - \frac{\epsi'}{1-\epsi'}\left(1-e^{-1}\right)\right) \ff{O}\\
&\hspace*{4em} \ge \frac{1}{1-\frac{\ell}{k}} \left(1-(e+1)\epsi'\right)\left(e^{-1} - \frac{\epsi'}{1-\epsi'}\left(1-e^{-1}\right)\right) \ff{O}\tag{$\ell\ge \frac{2}{e\epsi'}, k\ge \frac{2(1-\epsi')^2\ell}{e\epsi'-\frac{2}{\ell}}$}\\
&\hspace*{4em} \ge \frac{1}{1-\frac{\ell}{k}} \left(e^{-1}-\epsi\right)\ff{O}\tag{$\epsi' = \frac{\epsi}{2}$}.
\end{align*}
By Inequality~\ref{inq:tgtwo-dif-opt},
the approximation ratio of Alg.~\ref{alg:tgtwo} is $e^{-1}-\epsi$.
\end{proof}















\section{Experimental Setups and Additional Empirical Results}\label{apx:exp}
In the section, we introduce the settings in Section~\ref{sec:exp} further, and discuss more experimental results on \nmon and \mon.

\subsection{Applications}\label{apx:app}
\textbf{Maxcut.}
In the context of the maxcut application, we start with a graph $G=(V, E)$ 
where each edge $ij \in E$ has a weight $w_{ij}$.
The objective is to find a cut that maximizes the total weight of edges crossing the cut.
The cut function $f: 2^V \to \reals$ is defined as follows,
\[f(S) = \sum_{i \in S} \sum_{j \in V\setminus S}w_{ij}, \forall S\subseteq V.\]
This is a non-monotone submodular function.
In our implementation, for simplicity, all edges have a weight of $1$.

\textbf{Revmax.}
In our revenue maximization application,
we adopt the revenue maximization model introduced in \citep{DBLP:conf/www/HartlineMS08}, which we will briefly outline here.
Consider a social network $G=(V, E)$,
where $V$ denotes the buyers.
Each buyer $i$'s value for a good depends on the set of buyers $S$ that already own it, 
which is formulated by 
\[v_i(S)=f_i\left(\sum_{j \in S} w_{ij}\right),\]
where $f_i: \reals \to \reals$ is a non-negative, monotone, concave function, and $w_{ij}$ is drawn independently from a distribution.
The total revenue generated from selling goods to the buyers $S$ is
\[f(S) = \sum_{i \in V\setminus S} f_i\left(\sum_{j \in S} w_{ij}\right).\]
This is a non-monotone submodular function.
In our implementation, 
we randomly choose each $w_{ij}\in (0,1)$,
and $f_i(x) = x^{\alpha_i}$, where $\alpha_i \in (0,1)$ is chosen uniformly randomly.

% \textbf{Imgsum.}
% We follow the setting of Personalized Image Summarization application in~\citet{mirzasoleiman2016fast} with the following objective function
% \[f(S) = \sum_{i \in \uni} \max_{j \in S}s_{ij} - \frac{1}{n}\sum_{i \in S}\sum_{j \in S}s_{ij},\]
% where $s_{ij}$ determines the cosine similarity of image $i$ to image $j$
% with pixel vectors.
% The first term tries to ensure that the set $S$ is a good summary of the dataset,
% while the second promotes diversity within the summary itself. 
% This is a non-monotone, submodular objective function.

\subsection{Datasets}\label{apx:data}
\textbf{er} is a synthetic random graph generated by Erd{\"{o}}s-R{\'{e}}nyi model~\citep{erdds1959random} by setting number of nodes $n=100,000$ and edge probability $p=\frac{5}{n}$.

\textbf{web-Google}~\citep{DBLP:journals/im/LeskovecLDM09}  is a web
graph of $n=875,713$ web pages as nodes and $5,105,039$ hyperlinks
as edges.

\textbf{musae-github}~\citep{rozemberczki2019multiscale} is a social network of GitHub developers with $n=37,700$ developers and $289,003$ edges,
where edges are mutual follower relationships between them.

\textbf{twitch-gamers}~\citep{rozemberczki2021twitch} is a social network of $n=168,114$ Twitch users with $6,797,557$ edges, 
where edges are mutual follower relationships between them.


% \textbf{CIFAR-10}~\citep{krizhevsky2009learning} dataset consists of $50,000$ training images and $10,000$ test images
% where each image 
% is represented by a pixel vector of length 3,072:
% $32 \times 32$ pixels with red, green, and blue channels.
% In this paper, we randomly choose $3,000$ images from the training dataset.

\subsection{Additional Results}\label{apx:nmon}
Fig.~\ref{fig:apx} provides additional results on musae-github dataset with $n=37,700$
and web-Google dataset with $n=875,713$.
It shows that as $n$ and $k$ increase, our algorithms achieve superior on objective values.
The results of query complexity and adaptivity align closely with those discussed in Section~\ref{sec:exp}.
Notably, the number of adaptive round of \ptgtwoshort exceeds $k$ on musae-github,
which may be attributed to the dataset's relatively small size.
\begin{figure}[ht]
    \centering
    \subfigure[musae-github, solution value]{\label{fig:git-val}
    \includegraphics[width=0.31\linewidth]{fig/epsi1/git-val.pdf}}
    \subfigure[musae-github, query]{\label{fig:git-query}
    \includegraphics[width=0.31\linewidth]{fig/epsi1/git-query.pdf}}
    \subfigure[musae-github, round]{\label{fig:git-round}
    \includegraphics[width=0.31\linewidth]{fig/epsi1/git-round.pdf}}
    \subfigure[web-Google, solution value]{\label{fig:google-val}
    \includegraphics[width=0.31\linewidth]{fig/epsi1/google-val.pdf}}
    \subfigure[web-Google, query]{\label{fig:google-query}
    \includegraphics[width=0.31\linewidth]{fig/epsi1/google-query.pdf}}
    \subfigure[web-Google, round]{\label{fig:google-round}
    \includegraphics[width=0.31\linewidth]{fig/epsi1/google-round.pdf}}
    \caption{Results for \revmax on musae-github with $n=37,700$,
    and \maxcut on web-Google with $n=875,713$.}
    \label{fig:apx}
\end{figure}




















\end{document}


% This document was modified from the file originally made available by
% Pat Langley and Andrea Danyluk for ICML-2K. This version was created
% by Iain Murray in 2018, and modified by Alexandre Bouchard in
% 2019 and 2021 and by Csaba Szepesvari, Gang Niu and Sivan Sabato in 2022.
% Modified again in 2023 and 2024 by Sivan Sabato and Jonathan Scarlett.
% Previous contributors include Dan Roy, Lise Getoor and Tobias
% Scheffer, which was slightly modified from the 2010 version by
% Thorsten Joachims & Johannes Fuernkranz, slightly modified from the
% 2009 version by Kiri Wagstaff and Sam Roweis's 2008 version, which is
% slightly modified from Prasad Tadepalli's 2007 version which is a
% lightly changed version of the previous year's version by Andrew
% Moore, which was in turn edited from those of Kristian Kersting and
% Codrina Lauth. Alex Smola contributed to the algorithmic style files.
