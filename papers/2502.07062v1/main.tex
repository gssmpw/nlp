%%%%%%%% ICML 2025 EXAMPLE LATEX SUBMISSION FILE %%%%%%%%%%%%%%%%%

\documentclass{article}
\usepackage[letterpaper,top=2cm,bottom=2cm,left=3cm,right=3cm,marginparwidth=1.75cm]{geometry}

% % Recommended, but optional, packages for figures and better typesetting:
% \usepackage{microtype}
% \usepackage{graphicx}
% \usepackage{subfigure}
% \usepackage{booktabs} % for professional tables

% % hyperref makes hyperlinks in the resulting PDF.
% % If your build breaks (sometimes temporarily if a hyperlink spans a page)
% % please comment out the following usepackage line and replace
% % \usepackage{icml2025} with \usepackage[nohyperref]{icml2025} above.
% \usepackage[final]{hyperref}


% % Attempt to make hyperref and algorithmic work together better:
% \newcommand{\theHalgorithm}{\arabic{algorithm}}

% % Use the following line for the initial blind version submitted for review:
% % \usepackage{icml2025}

% % If accepted, instead use the following line for the camera-ready submission:
% % \usepackage[accepted]{icml2025}

% % For theorems and such
% \usepackage{amsmath}
% \usepackage{amssymb}
% \usepackage{mathtools}
% \usepackage{amsthm}

% % if you use cleveref..
% \usepackage[capitalize,noabbrev]{cleveref}

% %%%%%%%%%%%%%%%%%%%%%%%%%%%%%%%%
% % THEOREMS
% %%%%%%%%%%%%%%%%%%%%%%%%%%%%%%%%
% % \theoremstyle{plain}
% % \newtheorem{theorem}{Theorem}[section]
% % \newtheorem{proposition}[theorem]{Proposition}
% % \newtheorem{lemma}[theorem]{Lemma}
% % \newtheorem{corollary}[theorem]{Corollary}
% % \theoremstyle{definition}
% % \newtheorem{definition}[theorem]{Definition}
% % \newtheorem{assumption}[theorem]{Assumption}
% % \theoremstyle{remark}
% % \newtheorem{remark}[theorem]{Remark}

% % Todonotes is useful during development; simply uncomment the next line
% %    and comment out the line below the next line to turn off comments
% %\usepackage[disable,textsize=tiny]{todonotes}
\usepackage[textsize=tiny]{todonotes}
\setlength{\marginparwidth}{1.5cm}
\usepackage{xspace}
\newcommand{\eg}{e.g.,\xspace}
\newcommand{\ie}{i.e.,\xspace}
\newcommand{\etal}{et~al.\xspace}

\newcommand{\statsum}[3]{{$M$~=~#1,~$SD$~=~#2}}

\newcommand{\todo}[1]{{\color{red}[\textbf{TODO:} #1]}}

\newcommand{\david}[1]{{\color{blue}[\textit{David:} #1]}}
\newcommand{\customtilde}{{\raise.17ex\hbox{$\scriptstyle\sim$}}}


% The \icmltitle you define below is probably too long as a header.
% Therefore, a short form for the running title is supplied here:
% \icmltitlerunning{Sublinear Adaptive Algorithm for Submodular Maximization}
\title{Breaking Barriers: Combinatorial Algorithms for Non-monotone Submodular Maximization with Sublinear Adaptivity and $1/e$ Approximation}
\date{}
\author{
  Yixin Chen, Wenjing Chen, Alan Kuhnle \\
  Department of Computer Science \& Engineering \\
  Texas A\&M University \\
  Colloge Station, TX\\
  \texttt{\{chen777, jj9754@tamu.edu, kuhnle\}@tamu.edu} \\
}

\begin{document}
\maketitle

% this must go after the closing bracket ] following \twocolumn[ ...

% This command actually creates the footnote in the first column
% listing the affiliations and the copyright notice.
% The command takes one argument, which is text to display at the start of the footnote.
% The \icmlEqualContribution command is standard text for equal contribution.
% Remove it (just {}) if you do not need this facility.

%\printAffiliationsAndNotice{}  % leave blank if no need to mention equal contribution
% \printAffiliationsAndNotice{\icmlEqualContribution} % otherwise use the standard text.

\begin{abstract}
% With the explosion of data in modern applications, practical algorithms have garnered increasing attention under various large-scale data settings.
% This work focuses on developing parallel combinatorial approximation algorithms for maximizing a non-monotone submodular function subject to a size constraint $k$
% with a ground set of size $n$.
% The current state-of-the-art approximation ratio for this problem is $1/e$, achieved by
% a continuous algorithm~\citep{Ene2020a} with adaptivity $\oh{\log(n)}$.
% We propose two parallel combinatorial algorithms, both
% achieving $\oh{\log(n)\log(k)}$ adaptivity and 
% $\oh{n\log(n)\log(k)}$ query complexity.
% These algorithms improve the best-known deterministic approximation ratio from $0.125-\epsi$ to $0.25-\epsi$
% and the best-known randomized approximation ratio from $0.25-\epsi$ to $1/e-\epsi\approx 0.367-\epsi$,
% breaking the barrier between continuous and combinatorial approaches.
% Empirical evaluations demonstrate the effectiveness of our methods, 
% achieving competitive objective values, 
% with the first algorithm excelling in query efficiency.
With the rapid growth of data in modern applications, parallel combinatorial algorithms 
for maximizing non-monotone submodular functions have gained significant attention.
The state-of-the-art approximation ratio of $1/e$ is 
currently achieved only by a continuous algorithm~\citep{Ene2020a} with adaptivity $\oh{\log(n)}$.
In this work, we focus on size constraints 
and propose a $(1/4-\epsi)$-approximation algorithm 
with high probability for this problem, 
as well as the first randomized parallel combinatorial algorithm
achieving a $1/e-\epsi$ approximation ratio,
which bridges the gap between continuous and combinatorial approaches.
Both algorithms achieve $\oh{\log(n)\log(k)}$ adaptivity and 
$\oh{n\log(n)\log(k)}$ query complexity.
Empirical results show our algorithms achieve competitive objective values, 
with the first algorithm particularly efficient in queries.
\end{abstract}

\section{Introduction}
\label{sec:intro}
% Image editing methods in diffusion models depend on user-defined control directions - users can unlock their creativity using these methods by specifying the desired manipulation through prompts~\cite{gandikota2023concept}, reference images~\cite{ruiz2022dreambooth, kumari2022customdiffusion, gal2022image, chen2024trainingfreeregionalpromptingdiffusion}, or attribute vectors~\cite{parmar2023zero,hertz2022prompt}. In this work, we ask a fundamentally different question: \emph{Can we automatically discover the underlying visual structure of a concept within diffusion model's knowledge?} %Rather than requiring user-specified controls, we aim to decompose the model's internal knowledge into meaningful directions.

% This question touches on a fundamental limitation in how we interact with diffusion models. Current control methods ~\cite{zhang2023addingconditionalcontroltexttoimage, gandikota2023concept, ye2023ipadaptertextcompatibleimage,ye2023ipadaptertextcompatibleimage, hertz2024stylealignedimagegeneration, li2023photomaker, shi2024instantbooth, chen2024trainingfreeregionalpromptingdiffusion} require users to specify their desired manipulations in advance, limiting interactive creativity. This contrasts with natural human artistic workflows, where creators dynamically explore creative ideas while jointly refining them toward meaningful artistic outcomes~\cite{hoffmann2016modeling}. This synergy between specification and exploration is not new to generative models. Early GAN architectures naturally developed disentangled latent spaces that enabled continuous\cite{harkonen2020ganspace,radford2015unsupervised, wu2021stylespace, shen2020interfacegan}, compositional control over generated images. Users could explore these spaces to discover interesting variations that would be difficult to describe in words~\cite{wu2021stylespace}, then combine them to achieve their creative goals~\cite{grabe2022towards}. 


% While diffusion models have largely superseded GANs in conditional image synthesis~\cite{dhariwal2021diffusion},  their underlying structure remains less understood. Diffusion models achieve remarkable diversity through high-dimensional latents, unlike GANs' compact latent spaces.  With a single prompt, diffusion models can generate radically different variations through different random initializations of input noise. We ask - Is it possible to discover interpretable structure within this vast space of variations?

Text-to-image diffusion models are capable of generating remarkable visual variations from a single prompt through different random initializations. However, this vast creative potential remains largely opaque to users---while we can generate diverse images, we lack understanding of the underlying structure of these variations. This presents a fundamental challenge: how can we discover and expose the latent visual capabilities encoded within these models?

\let\thefootnote\relax \footnote{$^{*}$Correspondence to \texttt{gandikota.ro@northeastern.edu}}

The challenge touches on a key limitation in how we interact with diffusion models today. Current control methods require users to explicitly specify their desired edits in advance through prompts~\cite{gandikota2023concept}, reference images~\cite{zhang2023addingconditionalcontroltexttoimage, chen2024trainingfreeregionalpromptingdiffusion, ruiz2022dreambooth,kumari2022customdiffusion, Ryu_lora, hu2021lora}, or attribute vectors~\cite{ye2023ipadaptertextcompatibleimage, hertz2024stylealignedimagegeneration, li2023photomaker, shi2024instantbooth,parmar2023zero,hertz2022prompt}. That contrasts sharply with natural human creative workflows, where artists dynamically explore creative ideas and jointly refine them toward meaningful artistic outcomes~\cite{hoffmann2016modeling}. The need for pre-specified controls creates a barrier between users and the full creative potential of these models.

Interestingly, earlier generative models like GANs~\cite{gans,karras2019style,brock2018large} naturally developed more interpretable internal structures. Their compact latent spaces often exhibited emergent disentanglement~\cite{harkonen2020ganspace,radford2015unsupervised, wu2021stylespace, shen2020interfacegan}, enabling continuous and compositional control over generated images. Users could explore these spaces to discover interesting variations that would be difficult to describe in words~\cite{wu2021stylespace}, then combine them to achieve their creative goals~\cite{grabe2022towards}.

Diffusion models have largely superseded GANs in conditional image synthesis~\cite{dhariwal2021diffusion}, achieving greater diversity through much higher-dimensional latents. And yet an understanding of the underlying structure of these larger latent spaces has remained elusive. In this work, we ask a fundamental question: \emph{Can we automatically discover the visual structure within a diffusion model's knowledge of a concept?} Rather than requiring user-specified controls, we aim to decompose the model's internal representations into expressive directions that users can explore and combine.

To address these needs, we present \textbf{SliderSpace}, a framework that brings systematic explorability to diffusion models. Given just a text prompt, SliderSpace discovers a canonical set of meaningful, diverse, and controllable directions within the model's knowledge of that concept. Each direction is implemented as a low-rank adapter~\cite{hu2021lora} that can be scaled and composed with others, allowing users to explore and smoothly combine different aspects of variation, as shown in Figure~\ref{fig:intro}.

We ground SliderSpace discovery in three key requirements for meaningful decomposition of a diffusion model's visual manifold: 
\begin{enumerate}
    \item \textbf{Unsupervised Discovery:} The decomposition process should emerge from the intrinsic structure of the model's learned representation, rather than being guided by predefined attributes. This ensures we capture the true topology of the model's knowledge space rather than projecting our assumptions onto it.
    
    \item \textbf{Semantic Orthogonality:} Each discovered control must represent a distinct semantic direction. This is enforced in a semantic feature space, like CLIP, where every slider has an orthogonal effect in embeddings. This prevents discovering multiple controls that create similar semantic effects, making the system more efficient and easier.
    
    \item \textbf{Distribution Consistency:} Directions must induce consistent transformations across both random seeds and prompt variations. 
\end{enumerate}

These requirements naturally lead to our proposed framework, which we formalize in Section~\ref{sec:method}. As we show in our experiments, SliderSpace is architecture-agnostic, working with both conventional U-Net based models like Stable Diffusion~\cite{rombach2022high, rombach2022sd20, podell2023sdxl, turbo, dmd} and recent transformer-based architectures like Flux~\cite{flux}.

We demonstrate the expressiveness of SliderSpace through three applications: First, we show how SliderSpace can decompose high-level concepts into diverse and expressive components, revealing the natural axes of variation in the model's understanding. Second, we explore artistic style variation, where SliderSpace discovers directions that match or exceed the diversity of manually curated artist lists while being judged more useful by human evaluators. Finally, we show how SliderSpace can help reverse the mode collapse commonly observed in distilled diffusion models, restoring diversity while maintaining generation speed.

Beyond providing practical creative control, SliderSpace opens new avenues for understanding and utilizing the latent capabilities of diffusion models. By mapping these models' visual potential into intuitive, composable directions, we take a step toward making their creative possibilities more accessible and interpretable to users.

% Image editing methods in diffusion models unlock the creativity of users. In this work we ask an alternate question: \emph{Can we organize and expose what of the diffusion model is already capable of?}.
% Existing methods for controlling image generation typically require users to manually specify edit directions for desired changes. This process is time-consuming, requires technical expertise, and limits the spontaneity of the creative process. For instance, if a user wants to adjust the smile of a generated person, they must explicitly request this edit, often through imprecise prompt engineering or model fine-tuning. This approach of predefined controls or manual specifications restricts users from fully exploring the latent capabilities of the model. There may be interesting stylistic variations or attributes that the model can generate, but users have no easy way to discover or utilize these.

% Natural visual disentanglement was an emergent property in the latent space of Generative Adversarial Models (GANs) \cite{harkonen2020ganspace,radford2015unsupervised, wu2021stylespace, shen2020interfacegan}. In particular, it has been observed that StyleGAN~\cite{karras2019style} stylespace neurons offer detailed control over many meaningful aspects of images that would be difficult to describe in words~\cite{wu2021stylespace}. However, diffusion models do not share such a compact latent space~\cite{park2023unsupervised}; and efforts to uncover such a space in the semantic embeddings of the text conditioning have met with limited success \nik{Nick - is there a specific citation you were thinking about?}.

% In this work we introduce \textbf{SliderSpace}, which takes a step towards uncovering an analogous low dimensional representation of diffusion models' visual breadth; in essence treating the diffusion model as many generators sharing parameters, where a particular generator is defined by a specific prompt. For a given prompt we sample many random seeds (and optionally prompt expansions using an LLM), generate the corresponding images, and apply an off the shelf feature extractor (in this work CLIP, but our method can be applied to any differentiable feature extractor). We use PCA to analyze these features, and for each of the leading $k$ principal components we train a LoRA \cite{} which causes the diffusion model to produces images which increase the feature magnitude along that component when passed back through the same feature extractor. This leads to a 'Slider' for each principal component, because each LoRA can be scaled and applied to the original diffusion model, continuously varying those visual features in the generated results (as measured, in our case, by CLIP).

% There are many other works that enhance the controllability of diffusion models. One common approach is enabling users to add spatial constraints to a generation either manually, or via a reference image \cite{zhang2023addingconditionalcontroltexttoimage, chen2024trainingfreeregionalpromptingdiffusion}, a second is leveraging more abstract embeddings (e.g. identity, style) extracted from a reference image \cite{ye2023ipadaptertextcompatibleimage, hertz2024stylealignedimagegeneration, li2023photomaker, shi2024instantbooth}, a third is finetuning a foundation model to better generate a concept important to the user \cite{ruiz2022dreambooth, kumari2022customdiffusion, Ryu_lora, hu2021lora}, and a fourth (most relevant to this work) is finding low-rank adaptors of the model based on a prompt or small training set which can be scaled to provide continous control over one aspect of generated image (e.g. night vs day, basic vs luxury, etc.) \cite{gandikota2023concept}. SliderSpace is complementary to all of these methods and offers something distinct. All of the other methods we are aware require the user (and / or model designer) to know in advance what type of control they want. In contrast SliderSpace assists users in discovering and controlling hidden capabilities present in the diffusion model's distribution of possible generations.

%We propose that truly intuitive creative control in a text-to-image model should meet three key criteria: \emph{discoverability}, \emph{intuitiveness}, and \emph{specificity}. The model should reveal controllable attributes that may not be immediately obvious, offer controls that are easy to understand and manipulate, and ensure each control affects a distinct attribute of the generated image.

% We demonstrate the utility and power of SliderSpace using three applications built on top of SDXL-DMD \cite{dmd}, because its fast generation speed lends itself well to the continuous control offered by SliderSpace.

% First, we study concept decomposition (Section \ref{sec:concept_exp}), where we learn sliders for a specific concept (e.g. 'monster', 'waterfall', 'car'). Through quantitative metrics of diversity and text alignment we demonstrate that the learned sliders dramatically boost the diversity of generations when randomly applied without harming text alignment; we also ask humans to qualitatively judge these results in a user study where they find the SliderSpace results to be more 'Diverse', 'Useful', and 'Creative' than our baselines.

% Second, we attempt to compare the automatic discoveries of SliderSpace to a large scale manual study of artistic styles (Section \ref{sec:art_exp}), open-sourced by ParrotZone \cite{parrotzone}. In this study SDXL was prompted with over 4300 artist names,  and based on visual inspection the cases of successful stylistic mimicry recorded. Quantitatively SliderSpace more closely matches the distribution of artistic variation discovered by ParrotZone than other baselines, and in our user studies was judged to be significantly more 'Diverse' and 'Useful' than the baselines. To our surprise humans even judged SliderSpace results to be slightly more 'Diverse' than the results generated by the manually discovered artist names of \cite{parrotzone}.

% Third, we attempt to use SliderSpace to reverse the mode collapse commonly observed in distilled few-step diffusion models relative to the original teacher model (Section \ref{sec:diverse_exp}). We quantitatively demonstrate that applying SliderSpace to SDXL-DMD leads to more closely matching the distribution of images by the original teacher, SDXL.

%Through extensive experiments on various state-of-the-art text-to-image models, we demonstrate that SliderSpace significantly enhances user control and creative expression in AI-assisted image generation tasks. Our method enables a range of applications, including concept decomposition and control, diversity improvement in generated images, customization dissection and edits, and the exploration of artistic styles inherent in the model.

% SliderSpace goes beyond providing a practical tool for enhanced creative control. By mapping the visual potential of diffusion models it can open new avenues for generative creativity and deepens our understanding of each model's hidden potential.
\section{Preliminary}
\textbf{Notation.}
We denote the marginal gain of adding $A$ to $B$ by $\marge{A}{B} = \ff{A\cup B} - \ff{B}$.
For every set $S\subseteq U$ and an element $x\in \uni$,
we denote $S\cup \{x\}$ by $S+x$ and $S\setminus \{x\}$ by $S-x$.

\textbf{Submodularity.}
A set function $f:2^\uni\to \reals$ is submodular, 
if $\marge{x}{S}\ge \marge{x}{T}$ for all $S\subseteq T\subseteq \uni$
and $x\in \uni\setminus T$,
or equivalently, for all $A, B\subseteq \uni$,
it holds that $\ff{A} + \ff{B}\ge \ff{A\cup B} + \ff{A\cap B}$.
With a size constraint $k$,
let $O =\argmax_{S\subseteq \uni, |S| \le k} \ff{S}$.

In Appendix~\ref{apx:prop}, we provide several key propositions
derived from submodularity that streamline the analysis.

\textbf{Organization.}
In Section~\ref{sec:gd}, the blending technique is introduced
and applied to \ig and \itg,
with detailed analysis provided in Appendix~\ref{apx:greedy-1/4}
and~\ref{apx:greedy-1/e}.
Section 4 discusses their fast versions with
pseudocodes and comprehensive analysis in
Appendix~\ref{apx:tg}.
Subsequently, Section~\ref{sec:ptg} delves into our sublinear adaptive algorithms, 
with further technical details and proofs available in Appendix~\ref{apx:ptg}.
Finally, we provide the empirical evaluation in Section~\ref{sec:exp},
with its detailed setups and additional results
in Appendix~\ref{apx:exp}.


\section{The Blending Technique for Greedy}\label{sec:gd}
% \section{A Fresh Take on Interlaced Greedy Based Algorithms}
% In this section, we present a novel analysis of 
% \ig~\citep{DBLP:conf/nips/Kuhnle19} and \itg~\citep{DBLP:conf/kdd/ChenK23}.
% This analysis eliminates the need for the guessing step in both algorithms,
% simplifying their implementation while preserving their theoretical guarantees.
In this section, we present two practical greedy variants
that simplify \ig~\citep{DBLP:conf/nips/Kuhnle19} and \itg~\citep{DBLP:conf/kdd/ChenK23}.
These variants are developed using a novel analysis technique 
called the blended marginal gains strategy. 
Notably, the proposed algorithms are not only simpler than \ig and \itg 
but also retain their theoretical guarantees, 
making them both efficient and theoretically sound. 
This section provides a detailed exposition of these algorithms and their underlying principles.

\subsection{Deterministic Greedy Variant with $1/4$ Approximation Ratio}
\label{sec:greedy-1/4}
\textbf{Analysis of \ig.}
\ig (Alg.~\ref{alg:ig} in Appendix~\ref{apx:pseudocode}) operates as follows.
% Initially, it guesses whether the max singleton $a_0$ is in the optimal solution $O$.
First, two empty solution sets, $A$ and $B$, are maintained
and constructed by interlacing two greedy procedures.
Second, two additional solution sets, $D$ and $E$, are initialized with the maximum singleton $a_0$,
and are constructed using the same interlaced greedy procedure.
% and they are constructed using an interlaced greedy for loop.
% (same as Lines~\ref{line:gdone-for-begin}-\ref{line:gdone-for-end} of Alg.~\ref{alg:gdone}).
% Conversely, if $a_0\in O$, $A$ and $B$ are initialized with $\{a_0\}$,
% and the interlaced greedy step is repeated.
Finally, the best solution among these sets are returned.

The core idea of the algorithm is to greedily construct two disjoint solutions,
$A$ and $B$, in an alternating manner,
ensuring the following inequalities,
\begin{align}
    &\ff{O\cup A} + \ff{O\cup B} \ge \ff{O}, \label{inq:gdone-opt}\\
    &2\ff{A} \ge \ff{O\cup A},\label{inq:gdone-A}\\
    &2\ff{B} \ge \ff{O\cup B},\label{inq:gdone-B}
\end{align}
Note that, the element $a_0$ is treated separately.
% 1) if $a_0\notin O$, then above inequalities holds for $\{S, T\} = \{A, B\}$;
% 2) otherwise, $\{S, T\} = \{D, E\}$.
The second round of the interlaced greedy procedure is specifically designed to
handle the case where $a_0\in O$.
The above inequalities are then guaranteed to hold with solution sets $\{D, E\}$.
However, intuitively, adding an element from $O$ to the solution should not negatively impact it.
This leads to a natural question: 
\textit{Can the second interlaced greedy step be eliminated?}    


% To bound $\marge{O}{A}$, consider the elements in $O\setminus A$.
% Since the interlaced greedy for loop begins by adding elements to $A$
% in both cases,
% $O\setminus A$ can be ordered as $\{o_1, o_2, \ldots\}$,
% where each $o_i$ is a candidate for greedy selection
% when the $i$-th element is added to $A$.
% By the greedy selection rule,
% $\marge{o_i}{A} \le \ff{A_i} -\ff{A_{i-1}}$ for each $o_i\in O\setminus A$,
% where $A_i$ represents the first $i$ elements added to $A$.
% Then, by the first property in Proposition~\ref{prop:sum-marge}, it follows that

% \vspace*{-1em}
% {\small\begin{equation*}
% \marge{O}{A} \le \ff{A} \Rightarrow \ff{O\cup A} \le 2\ff{A}.
% \end{equation*}}

% Bounding $\marge{O}{B}$ requires a slightly different approach.
% In general, if $a_0 \notin O\setminus B$,
% the above analysis can be directly applied to bound $\marge{O}{B}$.
% Next, consider the two cases of $a_0$ discussed earlier.
% In the first case, where $a_0 \notin O$,
% it naturally follows that $a_0 \notin O\setminus B$.
% In the second case, where $a_0 \in O$,
% since $a_0$ is also included in $B$,
% it again follows that $a_0 \notin O\setminus B$.
% Therefore, a similar conclusion holds,

% \vspace*{-1em}
% {\small\begin{equation*}
% \marge{O}{B} \le \ff{B} \Rightarrow \ff{O\cup B} \le 2\ff{B}.
% \end{equation*}}

% However, this result does not directly extend to $B$,
% as the first element added to $A$, say $a_0$, might be in $O$.
% This prevents us from bounding 
% $\marge{a_0}{B}$ by $\ff{B_1}- \ff{\emptyset}$.
% To address this issue, $a_0$ is also added to $B$
% and solutions are then built with the same interlaced greedy step.
% Then, we can reorder $O\setminus B$ as $\{o_1, o_2, \ldots\}$,
% satisfying $\marge{o_i}{B} \le \ff{B_i} - \ff{B_{i-1}}$.
% Furthermore, the shared element $a_0$ is in $O$ which does not break Inequality~\eqref{inq:gdone-opt}.

% In summary, the second interlaced greedy step is introduced
% to address the case where $a_0\in O$.
% However, intuitively, adding an element from $O$ to the solution should not compromise it.
% This raises the question: 

% \textit{Can we eliminate the second interlaced greedy step?}

\textbf{A New Insight on Analyzing \ig: Blended Marginal Gains Strategy.}
The answer to the above question is YES.
In the original analysis of \ig, the upper bound of $\ff{O\cup A}$
only relies on $\ff{A}$ (Inequality~\eqref{inq:gdone-A}). 
Same as $\ff{O\cup B}$ (Inequality~\eqref{inq:gdone-B}).
In what follows,
we introduce a blended approach to analyze \ig with only
a single interlaced greedy step (Alg.~\ref{alg:gdone}).
Specifically, we utilize a mixture of $\ff{A}$ and $\ff{B}$,
or more precisely, a combination of the marginal gains when adding elements to each,
to establish tighter bounds for $\ff{O\cup A}$ or $\ff{O\cup B}$.
\begin{algorithm}[ht]
    \KwIn{evaluation oracle $f:2^{\uni} 
    \to \reals$, constraint $k$}
    \Init{$A\gets B\gets \emptyset$, add $2k$ dummy elements to the ground set}
    \For{$i\gets 1$ to $k$ \label{line:gdone-for-begin}}{
        $a\gets \argmax_{x\in \uni\setminus \left(A\cup B\right)} \marge{x}{A}$\; \label{line:gdone-greedy-A}
        $A\gets A+a$\;
        % \tcc*[r]{If $\marge{a_i}{A_{i-1}} < 0$, a dummy element is added instead.}
        $b\gets \argmax_{x\in \uni\setminus \left(A\cup B\right)} \marge{x}{B}$\; \label{line:gdone-greedy-B}
        $B\gets B+b$\;\label{line:gdone-for-end}
        % \tcc*[r]{If $\marge{b_i}{B_{i-1}} < 0$, a dummy element is added instead.}
    }
    \Return{$S\gets \argmax\{\ff{A}, \ff{B}\}$}
    \caption{A deterministic $1/4$-approximation algorithm with $\oh{ nk }$ queries.}
    \label{alg:gdone}
\end{algorithm}

We summarize our blending technique as follows.
Rather than relying solely on the greedy selection rule to 
bound $\marge{o}{A}$ or $\marge{o}{B}$ for each $o\in O$,
we split $O$ into two parts,
relaxing the marginal gain of one part solely through submodularity.
% To bound $\marge{O}{A}$, 
% split $O$ into two parts. 
% One is the part overlapping with the prefixes of $B$,
% and we use portion of $\ff{B}$ to bound it by submodularity.
% For the remaining element, we follow the original analysis
% of \ig and bound it with portion of $\ff{A}$.
Below, we provide a general proposition that captures the key
insight achieved by the blending technique.
\begin{proposition}\label{prop:blend}
For any submodular function $f:2^{\uni}\to \reals$ and $A, B, O \in \uni$,
Let $A_i$ be a prefix of $A$ with size $i$
such that $A_i \subseteq O$.
Similarly, define $B_j$.
It satisfies that,
\begin{align}
    \marge{O}{B} &\le \ff{A_i} + \marge{O\setminus A_i}{B},\label{inq:gdone-B2}\\
    \marge{O}{A} &\le \ff{B_j} + \marge{O\setminus B_j}{A}, \label{inq:gdone-A2}
\end{align}
\end{proposition}


By summing up Inequalities~\eqref{inq:gdone-B2} and~\eqref{inq:gdone-A2} and carefully
selecting the values of $i$ and $j$ under different cases illustrated in Fig.~\ref{fig:gdone},
the $1/4$ approximation ratio holds for Alg.~\ref{alg:gdone}.
\begin{figure}[ht]
\centering
\subfigure[$i^* \le j^*$]{\label{fig:gdone-1}\includegraphics[width=0.2\textwidth]{fig/ig-1.pdf}}
\subfigure[$i^* > j^*$]{\label{fig:gdone-2}\includegraphics[width=0.2\textwidth]{fig/ig-2.pdf}}
    \caption{This figure depicts the components of solution sets $A$ and $B$ in Alg.~\ref{alg:gdone}.
    The black rectangle highlights a sequence of consecutive elements from $O$
    that were added to the solution at the initial.
    Red circles with a cross mark signifies the first element in $A$ or $B$ that is outside $O$.
    }
\label{fig:gdone}
\end{figure}

% Let $a_i$ be the $i$-th element added to $A$,
% and $A_i$ be the set containing the first $i$ elements of $A$.
% Similarly, define $b_i$ and $B_i$ for the solution $B$.

% Following the original analysis of \ig, 
% by the greedy selection rule, the following inequalities hold,

% \vspace*{-1em}
% {\small\begin{align}
%     &\marge{o}{A_{i-1}} \le \marge{a_i}{A_{i-1}}, 
%     \forall o\in O\setminus (A_{i-1} \cup B_{i-1}) \label{inq:gdone-blend-1}\\
%     &\marge{o}{B_{i-1}} \le \marge{b_i}{B_{i-1}}, 
%     \forall o\in O\setminus (A_{i} \cup B_{i-1})\label{inq:gdone-blend-2}
% \end{align}}
% To derive a new bound, we track the longest prefix of $A$ and $B$ 
% that lies within the optimal solution $O$.
% Define $i^* = \argmax\{i \in [k]: A_i \subseteq O\}$
% and $j^* = \argmax\{i \in [k]: B_i \subseteq O\}$.
% Refer to Fig.~\ref{fig:gdone} as an illustration.

% For any $i \le i^*$, let $o_i = a_i$ and $O_i = A_i$.
% Then, by submodularity,

% \vspace*{-1em}
% {\small\begin{align}\label{inq:gdone-blend-3}
%     \marge{o_i}{B\cup O_{i-1}} \le \marge{a_i}{A_{i-1}}, \forall i \le i^*.
% \end{align}}
% Thus, by ordering $O\setminus B$,
% we can bound $\marge{O}{B}$ by blending Inequalities~\eqref{inq:gdone-blend-2}
% and~\eqref{inq:gdone-blend-3}.

% Similarly, for any $i \le j^*$, let $o_i = b_i$ and $O_i = B_i$.
% By submodularity,

% \vspace*{-1em}
% {\small\begin{align}\label{inq:gdone-blend-4}
%     \marge{o_i}{A\cup O_{i-1}} \le \marge{b_i}{B_{i-1}}, \forall i \le j^*.
% \end{align}}
% By ordering $O\setminus A$,
% we can bound $\marge{O}{A}$ by blending Inequalities~\eqref{inq:gdone-blend-1}
% and~\eqref{inq:gdone-blend-4}.

% By choosing the best blending strategy under different cases
% (Fig.~\ref{fig:gdone-1} and~\ref{fig:gdone-2}).
% It holds that

% \vspace*{-1em}
% {\small\begin{align*}
%     \marge{O}{A} + \marge{O}{B} \le \ff{A} + \ff{B}.
% \end{align*}}

% Next, order $O\setminus A$ as $\{o_1, o_2, \ldots\}$ such that
% $o_i = b_i$ if $o_i\in B$.
% Let $O_i$ be the first $i$ elements with this order.
% Then, if $i \le j^*$, we know that $o_i = b_i$ and $O_{i-1} = B_{i-1}$.
% By submodularity, it holds that,
% \begin{equation}\label{inq:gdone-blend-1}
% \marge{o_i}{A\cup O_{i-1}} \le \marge{b_i}{B_{i-1}}, \forall i \le j^*.
% \end{equation}
% Moreover, since $o_i \notin B_{i-1}$ for each $o_i \in O\setminus A$,
% $o_i$ is a candidate when $a_i$ is added to $A$.
% Thus, the following inequality holds by submodularity and the greedy selection rule,
% \begin{equation}\label{inq:gdone-blend-2}
% \marge{o_i}{A\cup O_{i-1}} \le \marge{a_i}{A_{i-1}}, \forall i \ge 1.
% \end{equation}

% Similarly, we can order $O\setminus B$ as $\{o_1, o_2, \ldots\}$ such that
% $o_i = a_i$ if $o_i\in A$,
% and let $O_i$ be the first $i$ elements with this order.
% Then, if $i \le i^*$, we know that $o_i = a_i$ and $O_{i-1} = A_{i-1}$.
% By submodularity, it holds that,
% \begin{equation}\label{inq:gdone-blend-3}
% \marge{o_i}{B\cup O_{i-1}} \le \marge{a_i}{A_{i-1}}, \forall i \le i^*.
% \end{equation}
% Also, since $o_{i} \notin A_{i-1}$,
% $o_i$ is a candidate when $b_{i-1}$ is added to $B$ for each $i \ge 2$.
% Thus,
% \begin{equation}\label{inq:gdone-blend-4}
% \marge{o_i}{B\cup O_{i-1}} \le \marge{b_{i-1}}{B_{i-2}}, \forall i \ge 2.
% \end{equation}
We provide the theoretical guarantee of Alg.~\ref{alg:gdone} below.
The detailed analysis can be found in Appendix~\ref{apx:greedy-1/4}.
\begin{restatable}{theorem}{thmgdone}\label{thm:gdone}
With input instance $(f, k)$, Alg.~\ref{alg:gdone} returns a set $S$ with $\oh{kn}$ queries
such that $\ff{S} \ge 1/4 \ff{O}$.
\end{restatable}


\begin{algorithm}[ht]
	\KwIn{evaluation oracle $f:2^{\uni} \to \reals$, constraint $k$, size of solution $\ell$, error $\epsi$}
    \Init{$G\gets \emptyset$, $V\gets \uni$, $m \gets \left\lfloor\frac{k}{\ell}\right\rfloor$, add $2k$ dummy elements to the ground set.}
    \For{$i\gets 1$ to $\ell$}{
    	$A_{l}\gets G, \forall l \in [\ell]$\;
    	\For{$j\gets 1$ to $m$}{\label{line:gdtwo-for-2-start}
    		\For{$l\gets 1$ to $\ell$}{
    			$a \gets \argmax_{x\in V}\marge{x}{A_{l}}$\;
                % $\delta_{l, j} \gets \marge{a_{l, j}}{A_l}$\;
    			$A_{l}\gets A_{l}+a$, $V\gets V-a$ \hspace*{-0.7em}\;
                % \tcc*[r]{If $\marge{a_{l, j}}{A_{l}} < 0$, a dummy element is added instead.}
    		}
    	}\label{line:gdtwo-for-2-end}
        % $A_l'\gets \left\{a_{l, j}: j\in \argmax\limits_{I\subseteq [m], |I| = m-1} \sum\limits_{j\in I}\delta_{l, j}\right\}$\;
    	$G\gets$ a random set in $\{A_l\}_{l\in [\ell]}$\;
    }
    \Return{$G$}
    \caption{A randomized $1/e$-approximation algorithm with $\oh{ nk\ell }$ queries. }
    \label{alg:gdtwo}
\end{algorithm}
\subsection{Randomized Greedy Variant with $1/e$ Approximation Ratio}
\label{sec:greedy-1/e}
In this section, we extend the blending technique introduced in the previous section 
to \itg~\citep{DBLP:conf/kdd/ChenK23} 
and propose a simplified algorithm (Alg.~\ref{alg:gdtwo}).
Alg.~\ref{alg:gdtwo} avoids the guessing step in \itg
improving the success probability from $(\ell+1)^{-\ell}$ to $1$.
The intuition behind the algorithm is outlined in the following.
\begin{restatable}{theorem}{thmgdtwo}\label{thm:gdtwo}
With input instance $(f, k, \ell, \epsi)$
such that $\ell =\oh{\epsi^{-1}} \ge \frac{2}{e\epsi}$ and $k \ge \frac{2(e\ell-2)}{e\epsi-\frac{2}{\ell}}$,
Alg.~\ref{alg:gdtwo} returns a set $G$ with $\oh{kn/\epsi}$ queries
such that $\ex{\ff{G}} \ge \left(1/e-\epsi\right) \ff{O}$.
\end{restatable}
In the following discussion, we focus on the case where $k\, \text{mod}\,\ell = 0$.
For the scenario where $k \, \text{mod}\,\ell > 0$, please refer to Appendix~\ref{apx:greedy-1/e}.

% \itg can be seen as an interpolation between 
% the standard greedy algorithm~\citep{DBLP:journals/mp/NemhauserWF78}
% and the \rg~\cite{DBLP:conf/soda/BuchbinderFNS14}.
% The algorithm runs a for loop, where in each iteration,
% built upon the solutions returned from the previous iteration,
% it creates $\ell+1$ pools, each containing $\ell$ candidate solutions,
% by guessing the element in the optimal solution that provides the largest marginal gain.
% To find the right pool, the algorithm must search through all pools.
% In the following, we offer a different perspective on analyzing the algorithm
% and show how the guessing step can be eliminated.

\textbf{From Interlacing $2$ Greedy to $\ell$ Greedy.}
% Let $G$ be the intermediate solution at the beginning of an iteration in \itg,
% and $\{A_l: l \in [\ell]\}$ be the solution sets at the end of this iteration.
% In iteration $m$ of the outer for loop in \itg (Alg.~\ref{alg:itg} in Appendix~\ref{apx:pseudocode}),
% $G_{m-1}$ represents the intermediate solution at the beginning of this iteration,
% and $\{a_1, \ldots, a_\ell\}$ are the top $\ell$ elements in $\uni\setminus G_{m-1}$
% with the largest marginal gains on $G_{m-1}$.
% Similar to \ig, \itg makes $\ell+1$ guesses to locate the element
% $o_{\max} = \argmax_{x \in O\setminus G_{m-1}} \marge{x}{G_{m-1}}$.
% Based on these guesses, $\ell+1$ sets of solutions are initialized,
% each constructed using the interlaced greedy procedure.
% At the end of iteration $m$, the algorithm finalizes the sets
% $\{A_{u,i}: 1\le i\le \ell, 0\le u\le \ell\}$.
% With a probability of $(\ell+1)^{-1}$, $G_m$ is chosen from
% the set of solutions $\{A_{u,i}: 1\le i\le \ell\}$ corresponding to the correct guess.
% Under this selection, the following inequality holds,
%
% \vspace*{-1em}
% {\small\[\marge{O}{A_{u, i}} \le \ell\marge{A_{u, i}}{G}, \forall i \in [\ell].\]}
%
% Recall the improvement we made on \ig, introduced in Section~\ref{sec:greedy-1/4}.
% This raises a direct question:
% \textit{Can we apply the blending technique to bound $\sum_{l=1}^\ell\marge{O}{A_l}$ without relying on the guessing step?}
By interlacing $\ell$ greedy procedures, \itg 
(Alg.~\ref{alg:itg} in Appendix~\ref{apx:pseudocode}) constructs 
$\ell+1$ pools of candidates, each containing $\ell$ nearly pairwise disjoint sets.
Among these pools, only $1$ is the \textit{right} pool that ensures the following guarantee.
\begin{align}
    \marge{O}{A_{u, i}} \le \ell\marge{A_{u, i}}{G}, \forall i \in [\ell],\label{inq:gdtwo-itg}
\end{align}
where $G$ is the intermediate solution at the start of this iteration,
and $A_{u,i}$ is the $i$-th solution in the $u$-th pool.
At each iteration, \itg randomly selects a set from all candidate pools,
resulting in a success probability of $(\ell+1)^{-\ell}$,
where the algorithm consistently identifies the correct pool.
In the following, we introduce how to incorporate the blending technique
into the analysis to eliminate the guessing step in \itg.

\textbf{Blended Marginal Gains for Each Pair of Solutions.}
% The only difference between the interlaced greedy procedures
% in Alg.~\ref{alg:gdone} and~\ref{alg:gdtwo}
% is the number of solutions built.
Unlike Alg.~\ref{alg:gdone}, which constructs solutions each of size $k$,
Alg.~\ref{alg:gdtwo} builds solutions with a total size of $k$.
% Different from the interlaced greedy procedure in Alg.~\ref{alg:gdone},
% the procedure in Alg.~\ref{alg:gdtwo} constructs
% $\ell$ solutions, each containing $k/\ell$ elements, rather than $k$.
To align with this structure, we partition $O$ into $\ell$ subsets,
as outlined in Claim~\ref{claim:par-A},
and pair each subset with one of the solutions.
\begin{restatable}{claim}{claimParA}\label{claim:par-A}
At an iteration $i$ of the outer for loop in Alg.~\ref{alg:gdtwo},
let $G_{i-1}$ be $G$ at the start of this iteration,
and $A_{l}$ be the set at the end of this iteration,
for each $l\in [\ell]$.
% Add dummy elements to $O\setminus G_{i-1}$ until its size equals $k$.
The set $O\setminus G_{i-1}$ can then be split into $\ell$ pairwise disjoint sets $\{O_1, \ldots, O_\ell\}$
such that $|O_l| \le\frac{k}{\ell}$ and $\left(O\setminus G_{i-1}\right) \cap \left(A_{l}\setminus G_{i-1}\right) \subseteq O_l$, for all $l \in [\ell]$.
\end{restatable}
Moreover, based on this claim,
we partition the marginal gain of adding $O$ to each $A_l$ as follows.
\begin{align*}
&\sum_{l\in [\ell]}\marge{O}{A_{l}} \le \sum_{l\in [\ell]} \sum_{i\in [\ell]} \marge{O_i}{A_{l}} \tag{Proposition~\ref{prop:sum-marge}}\\
&= \sum_{1\le l_1 < l_2 \le \ell} \left(\marge{O_{l_1}}{A_{l_2}}+\marge{O_{l_2}}{A_{l_1}}\right)+\sum_{l\in [\ell]} \marge{O_l}{A_{l}}. \numberthis \label{inq:gdtwo-par}
\end{align*}
According to Claim~\ref{claim:par-A}, any element in $O_l\setminus A_l$ 
is not sufficiently beneficial to be added to any solution set.
This ensures that $\marge{O_l}{A_{l}}\le \marge{A_l}{G_{i-1}}$.
As for the term $\marge{O_{l_1}}{A_{l_2}}+\marge{O_{l_2}}{A_{l_1}}$,
since elements are added to the solutions in an alternating manner,
Proposition~\ref{prop:blend} for the blending technique can be applied
to bound this term with the greedy selection rule.
These insights are formalized in the following lemma.
\begin{restatable}{lemma}{lemmaparA}\label{lemma:par-A}
Fix on $G_{i-1}$ for an iteration $i$ of the outer for loop in Alg.~\ref{alg:gdtwo}.
Following the definition in Claim~\ref{claim:par-A}, it holds that
\begin{align*}
\text{1) }&\marge{A_{l}}{G_{i-1}} \ge \marge{O_{l}}{A_{l}}, \forall 1\le l \le \ell,\\
\text{2) }&\left(1+\frac{1}{m}\right)\left(\marge{A_{l_1}}{G_{i-1}} + \marge{A_{l_2}}{G_{i-1}}\right)\ge \marge{O_{l_2}}{A_{l_1}} + \marge{O_{l_1}}{A_{l_2}}, \forall 1\le l_1< l_2 \le \ell.
\end{align*}
\end{restatable}
By applying Inequality~\eqref{inq:gdtwo-par} and Lemma~\ref{lemma:par-A}, 
we derive a result analogous to Inequality~\eqref{inq:gdtwo-itg} achieved by \itg,
\begin{align}
    \sum_{l\in [\ell]}\marge{O}{A_{u, i}} \le \ell\left(1+\frac{1}{m}\right)\sum_{l\in [\ell]}\marge{A_{u, i}}{G}.
\end{align} 
This forms the key property necessary to establish the $1/e-\epsi$ approximation ratio.
The detailed analysis of the approximation ratio is provided in Appendix~\ref{apx:greedy-1/e}.

\section{Preliminary Warm-Up of Parallel Approaches: Nearly-Linear Time Algorithms}\label{sec:tg}
In this section, we introduce the fast versions of Alg.~\ref{alg:gdone}
and~\ref{alg:gdtwo},
which substitute the standard greedy procedures
with descending threshold greedy procedure~\citep{DBLP:conf/soda/BadanidiyuruV14}
to achieve a query complexity of $\oh{n\log(k)}$.
Pseudocodes are provided in Appendix~\ref{apx:tg}
as Alg.~\ref{alg:tgone} and~\ref{alg:tgtwo}.
These fast algorithms serve as building blocks for the parallel algorithms introduced later in this work.
Below, we discuss the intuition behind Alg.~\ref{alg:tgone},
while Alg.~\ref{alg:tgtwo} operates in a similar fashion.

Like the greedy variants discussed earlier,
Alg.~\ref{alg:tgone} constructs the solution sets also in
an alternating manner.
Following the blending analysis technique introduced 
in Section~\ref{sec:greedy-1/e}-particularly Proposition~\ref{prop:blend}-the
primary challenge lies in bounding $\marge{O\setminus A_i}{B}$
and $\marge{O\setminus B_i}{A}$
under the threshold greedy framework.

By the alternating addition property,
the threshold value $\tau_i$ decreases to $\tau_i/(1-\epsi)$
if and only if any element outside $A\cup B$ has marginal gain
less than $\tau_i$.
% Furthermore, if the threshold value for solution $A$
% decreases from $\tau_1$ to $\tau_1/(1-\epsi)$ at some point,
% we know that any element outside $A\cup B$ has margianl gain less than $\tau_1$.
% Same for solution $B$.
Define $a_i$ as the $i$-th element added to $A$,
$A_i$ as the first $i$ elements added to $A$,
and $\tau_1^{a_i}$ as the threshold value when adding $a_i$ to $A$.
Similarly, define $b_i$, $B_i$, and $\tau_2^{b_i}$.
The following inequalities hold with upper bounds increased by a factor of
$1/(1-\epsi)$ compared to the standard greedy variants:
% Then, different from the properties 
% (Inequalities~\eqref{inq:gdone-blend-1} and~\eqref{inq:gdone-blend-2}) guaranteed by the standard greedy procedure,
% the descending threshold greedy procedure ensures
% the following inequalities,
% % for each $a_i$ with the corresponding threshold value $\tau_{a_i}$
% % when $a_i$ is added to $A$,
% % any element outside $A_{i-1}\cup B_{i-1}$ has marginal gain
% % less than $\tau_{a_i} /(1-\epsi)$.
% % Similarly, any element outside $A_{i}\cup B_{i-1}$ has marginal gain
% % less than $\tau_{b_i} /(1-\epsi)$.
% % The following inequalities are guaranteed by Alg.~\ref{alg:tgone}.
\begin{align*}
    &\marge{o}{A_{i-1}} \le \tau_1^{a_i} /(1-\epsi) \le \marge{a_i}{A_{i-1}}/(1-\epsi), \forall o\in O\setminus (A_{i-1} \cup B_{i-1}) \\
    &\marge{o}{B_{i-1}} \le \tau_2^{b_i} /(1-\epsi) \le \marge{b_i}{B_{i-1}}/(1-\epsi),\forall o\in O\setminus (A_{i} \cup B_{i-1})
\end{align*}
% However, unlike Alg.~\ref{alg:gdone},
% Alg.~\ref{alg:tgone} might return a set with size less than $k$.
% If final solution $A$ has size less than $k$,
% then, for any $o\in O\setminus (A\cup B)$,
% it holds that 
%
% \vspace*{-1em}
% {\small\begin{equation}\label{inq:tgone-blend-3}
%     \marge{o}{A} < \frac{\epsi M}{(1-\epsi)k} \le \frac{\epsi}{(1-\epsi)k}\ff{O}, \text{ if } |A| < k.
% \end{equation}}
% We can get a similar result for $B$ as follows,
%
% \vspace*{-1em}
% {\small\begin{equation}\label{inq:tgone-blend-4}
%     \marge{o}{B} < \frac{\epsi M}{(1-\epsi)k} \le \frac{\epsi}{(1-\epsi)k}\ff{O}, \text{ if } |B| < k.
% \end{equation}}
%
% Moreover, Inequalities~\eqref{inq:gdone-blend-3} and~\eqref{inq:gdone-blend-4} are also ensured in this case.
%
% Therefore, by blending those Inequalities, we can prove that
% Alg.~\ref{alg:tgone} achieve $1/4-\epsi$ approximation ratio.
Overall, Alg.~\ref{alg:tgone} sacrifices a constant $\epsi$ in approximation ratio
compared to Alg.~\ref{alg:gdone},
but achieves a significantly improved query complexity of $\oh{n\log (k)}$.

In the following, we provide the theoretical guarantees for
Alg.~\ref{alg:tgone} and~\ref{alg:tgtwo}
and left their detailed analysis in Appendix~\ref{apx:tg}.
\begin{restatable}{theorem}{thmtgone}\label{thm:tgone}
With input instance $(f, k, \epsi)$, Alg.~\ref{alg:tgone} returns a set $S$ with $\oh{n\log (k)/\epsi}$ queries
such that $\ff{S} \ge \left(\frac{1}{4}-\epsi\right) \ff{O}$.
\end{restatable}

\begin{restatable}{theorem}{thmtgtwo}\label{thm:tgtwo}
With input instance $(f, k, \epsi)$
such that $\ell = \oh{\epsi^{-1}}\ge \frac{4}{e\epsi}$
and $k \ge \frac{2(2-\epsi)\ell^2}{e\epsi\ell-4}$,
Alg.~\ref{alg:gdtwo} (Alg.~\ref{alg:tgtwo}) returns a set $G_\ell$ with $\oh{n\log(k)/\epsi^2}$ queries
such that $\ff{G_\ell} \ge \left(1/e-\epsi\right) \ff{O}$.
\end{restatable}

\begin{algorithm*}[ht]
\Fn{\ptgone($f, m, \ell, \tau_{\min}, \epsi$)}{
    \KwIn{evaluation oracle $f:2^{\uni} \to \reals$, 
    constraint $m$, constant $\ell$,
    minimum threshold value $\tau_{\min}$, error $\epsi$}
    \Init{$M\gets \max_{x\in \uni} \marge{\{x\}}{\emptyset}$, $I = [\ell]$, $m_0 \gets m$, $A_j\gets A_j'\gets \emptyset$, 
    $\tau_j\gets M$, $V_j \gets \uni, \forall j \in [\ell]$ }
    % \tcp*[h]{$V_j$ contains all good elements outside solutions}
    \While{$I \neq \emptyset$ and $m_0 > 0$}{
        \For(\textcolor{blue}{\tcc*[h]{Update candidate sets with high-quality elements}}){$j\in I$ in parallel\label{line:tgone-update-for-begin}}{
            $\{V_j, \tau_j\} \gets \update(f_{A_j}\restriction_{\uni\setminus\left(\bigcup_{l\in [\ell] A_l}\right)}, V_j, \tau_j, \epsi)$\label{line:tgone-update}\;
            \lIf{$\tau_j < \tau_{\min}$}{ $I\gets I-j$}\label{line:tgone-update-for-end}
        }
        \If(\textcolor{blue}{\tcc*[h]{Add 1 element to each solution alternately}}){$\exists i\in I$ \st $|V_i|< 2\ell$}{
            \For{$j\in I$ in sequence\label{line:tgone-for-begin}\label{line:pig-if-start}}{
                \lIf{$|V_j| = 0$}{
                    $\{V_j, \tau_j\} \gets \update(f_{A_j}\restriction_{\uni\setminus\left(\bigcup_{l\in [\ell] A_l}\right)}, V_j, \tau_j, \epsi)$\label{line:tgone-update-2}
                }
                \lIf{$\tau_j < \tau_{\min}$}{ $I\gets I-j$}
                \Else{
                    $x_j\gets $ randomly select one element from $V_j$ \label{line:tgone-select}\;
                    $A_j\gets A_j+x_j, A_j'\gets A_j'+x_j$\label{line:tgone-update-A}\;
                    $V_l\gets V_l-x_j, \forall l\in [\ell]$\;
                }
            }\label{line:tgone-for-end}
            $m_0\gets m_0-1$\;\label{line:pig-if-end}
        }
        \Else(\textcolor{blue}{\tcc*[h]{Add an equal number of elements to each solution}}){
            $\{\mathcal V_l: l\in I\} \gets \dist(\{V_l: l\in I\})$\label{line:tgone-dist}\label{line:pig-else-start} \tcp*[h]{Create pairwise disjoint candidate sets}\;
            $s \gets \min \{m_0, \min\{|\mathcal V_l|: l\in I\}\}$\;
            \lFor{$j\in I$ in parallel}{
                $i^*_j, B_j \gets \prefix(f_{A_j}, \mathcal V_j, s, \tau_j, \epsi)$\label{line:tgone-prefix}
            }
            $i^*\gets \min\{i^*_1, \ldots, i^*_\ell\}$ \label{line:tgone-index}\;
            \For(\textcolor{blue}{\tcc*[h]{Add $i^*$ high-quality elements to each set}}){$j\gets 1$ to $\ell$ in parallel \label{line:tgone-add-begin}}{
                $S_j\gets$ select $i^*$ elements from $\mathcal V_j[1:i^*_l]$ in three passes, prioritizing $B_j[i] = \textbf{true}$, then $B_j[i] = \textbf{none}$, and finally $B_j[i] = \textbf{false}$
                \label{line:tgone-subset}\;
                $S_j'\gets S_j\cap \left\{v_i \in \mathcal V_j: B_j[i]\neq \textbf{false}\right\}$\label{line:tgone-subset-2}\;
                $A_j\gets A_j\cup S_j, A_j'\gets A_j'\cup S_j'$\label{line:tgone-update-A-2}\;
            }
            $m_0\gets m_0-i*$ \label{line:tgone-update-size}\;\label{line:pig-else-end}
        }
    }
    \Return{$\{A_l': l\in [\ell]\}$}
}
\caption{A highly parallelized algorithm with $\oh{\ell^2\epsi^{-2} \log(n)\log\left(\frac{M}{\tau_{\min}}\right)}$ adaptivity
and $\oh{\ell^3\epsi^{-2} n \log(n)\log\left(\frac{M}{\tau_{\min}}\right)}$ query complexity. Subroutines \update, \dist and \prefix are provided in Appendix~\ref{apx:subroutine}.}
\label{alg:ptgone}
\end{algorithm*}

\section{Sublinear Adaptive Algorithms}\label{sec:ptg}
In this section, we present the main subroutine for our parallel algorithms, 
\ptgone (\ptgoneshort, Alg.~\ref{alg:ptgone}).
A single execution of \ptgoneshort achieves an 
approximation ratio of $1/4-\epsi$ with high probability,
while repeatedly running \ptgoneshort, as in \ptgtwo (\ptgtwoshort, Alg.~\ref{alg:ptgtwo} in 
Appendix~\ref{apx:ptgtwo}), 
guarantees a randomized approximation ratio of $1/e-\epsi$.
Below, we outline the theoretical guarantees,
with the detailed analysis provided in Appendix~\ref{apx:ptg}.
The remainder of this section is dedicated to 
explaining the intuition behind the algorithm.
\begin{restatable}{theorem}{thmptgone}\label{thm:ptgone}
With input $(f, k, 2, \frac{\epsi M}{k}, \epsi)$,
where $M = \max_{x\in \uni} \ff{x}$,
\ptgoneshort (Alg.~\ref{alg:ptgone}) returns $\{A_1', A_2'\}$
with $\oh{\epsi^{-4}\log(n)\log(k)}$ adaptive rounds and $\oh{\epsi^{-5}n\log(n)\log(k)}$ queries with a probability of $1-1/n$.
It satisfies that $\max\{\ff{A_1'}, \ff{A_2'}\}\ge (1/4-\epsi)\ff{O}$.
\end{restatable}

\begin{restatable}{theorem}{thmptgtwo}\label{thm:ptgtwo}
With input $(f, k, \epsi)$ such that
$\ell = \oh{\epsi^{-1}}\ge \frac{4}{e\epsi}$
and $k \ge \frac{(2-\epsi)^2\ell}{e\epsi\ell-4}$,
\ptgtwoshort (Alg.~\ref{alg:ptgtwo}) returns $G$
such that $\ex{\ff{G}} \ge (1/e-\epsi)\ff{O}$
with $\oh{\epsi^{-5}\log(n)\log(k)}$ adaptive rounds and $\oh{\epsi^{-6}n\log(n)\log(k)}$ queries with a probability of $1-\oh{1/(\epsi n)}$.
\end{restatable}
\ptgoneshort begins by initializing $\ell$ empty solutions $\{A_j: j\in [\ell]\}$,
with corresponding threshold values set to 
$M = \max_{x\in \uni}\marge{x}{\emptyset}$. 
Candidate sets $\{V_j: j\in [\ell]\}$ are maintained
and updated using \update
(Alg.~\ref{alg:update} in Appendix~\ref{apx:subroutine})
to filter out elements with marginal gain
less than $\tau_j$ for each solution $A_j$ 
at the start of every iteration 
in Lines~\ref{line:tgone-update-for-begin}-\ref{line:tgone-update-for-end}.
Two cases are then considered.
If any candidate set satisfies $|V_j| < 2\ell$,
$1$ element is added to each set alternately.
Otherwise, if all solutions have sufficient candidates,
pairwise disjoint candidate sets are then constructed in Line~\ref{line:tgone-dist} using \dist (Alg.~\ref{alg:dist} in Appendix~\ref{apx:subroutine}).
By executing a threshold sampling procedure \prefix (Alg.~\ref{alg:prefix} in Appendix~\ref{apx:subroutine}) on Line~\ref{line:tgone-prefix},
a block of elements of equal size is carefully selected and added to each solution.


\ptgoneshort is built upon the descending threshold greedy-based algorithms described in Section~\ref{sec:tg} to ensure nearly linear query complexity.
Additionally, it incorporates the threshold sampling algorithm, \ts~\citep{Chen2024},
to achieve sublinear adaptivity.
% It works as follows.
% For $j$-th solution, a pair of subsets $\{A_j, A_j'\}$ are maintained.
% The first set $A_j$ is responsible for filtering \textit{bad} elements that
% has marginal gain less than the current threshold value $\tau_j$,
% and the second set $A_j'$ is the actual solution such that
% $A_j'\subseteq A_j$ and $\ff{A_j'} \ge \ff{A_j}$.
% Furthermore, a candidate set $V_j$ is also maintained for each solution,
% where it contains all \textit{good} elements with marginal gain larger than $\tau_j$.
%
% To parallelize those algorithms, the following key properties must be maintained.
% \begin{enumerate}
%     \item Elements are added in an alternating manner.
%     \item Multiple elements are added to the solutions within constant adaptive round to achieve sublinear adaptivity.
%     \item Most of the elements added should contribute enough to the solutions.
% \end{enumerate}
% Next, we will introduce how \ptgoneshort achieve those goals.
To accomplish these goals,
several critical properties must be preserved throughout the process.

\subsection{Maintaining Alternating Additions during Parallel Algorithms}
\label{sec:ptg-alter}
This property is crucial to interlaced greedy variants 
introduced in prior sections. 
Below, we demonstrate that \ptgoneshort preserves this property.

During an iteration of the while loop in Alg.~\ref{alg:ptgone},
after updating the candidate sets in Lines~\ref{line:tgone-update-for-begin}-\ref{line:tgone-update-for-end},
two scenarios arise. 
In the first scenario, there exists a candidate set satisfies $|V_j| < 2\ell$,
Lines~\ref{line:pig-if-start}-\ref{line:pig-if-end} are executed,
and elements are appended to solutions one at a time in turn.
In this case, the alternating property is maintained immediately.

In the second scenario, Lines~\ref{line:pig-else-start}-\ref{line:pig-else-end} are executed.
Here, a block of elements with average marginal gain approximately exceeding $\tau_j$
is added to each solution $A_j$.
These blocks $S_j$ are of the same size $i^*$ (Line~\ref{line:tgone-subset})
selected from $\mathcal V_j$,
and guaranteed to be pairwise disjoint by 
Lemma~\ref{lemma:dist} (for \dist, Alg.~\ref{alg:dist} in Appendix~\ref{apx:subroutine}).
Crucially, threshold values $\tau_j$ 
remain unchanged during this step.
While a small fraction of elements in the blocks may have marginal gains below $\tau_j$,
the process retains the alternating property at a structural level: 
the uniform block sizes, and disjoint selection mimic the alternating addition of elements, even when processing multiple elements in parallel.

\subsection{Ensuring Sublinear Adaptivity Through Threshold Sampling}
The core mechanism for achieving sublinear adaptivity 
lies in iteratively reducing the pool of high-quality candidate elements 
(those with marginal gains above the threshold) 
by a constant factor within a constant number of adaptive rounds. 
This progressive reduction ensures efficient convergence.

At every iteration of the while loop in Alg.~\ref{alg:ptgone}, 
after updating the candidate sets,
if there exists $V_j$ such that $|V_j| < 2\ell$,
the following occurs after the for loop
(Lines~\ref{line:tgone-for-begin}-\ref{line:tgone-for-end}):
If the threshold $\tau_j$ remains unchanged,
one element from $V_j$ is added to the solution.
If $\tau_j$ is reduced,
$V_j$ is repopulated with high-quality elements.
This implies that a $1/(2\ell)$-fraction of $V_j$ is filtered out
after per iteration,
or even further, it becomes empty and
the threshold value is updated.

In the second case, where Lines~\ref{line:pig-else-start}-\ref{line:pig-else-end} are executed,
the algorithm employs \prefix (Alg.~\ref{alg:prefix} in Appendix~\ref{apx:subroutine}) in Line~\ref{line:tgone-prefix},
inspired by \ts~\citep{Chen2024}.
% At each iteration of \ts, a \textit{good prefix} is selected
% from the candidate set,
% consisting of elements outside the current solution
% that have marginal gains greater than
% the threshold value.
% After each addition to the solution, a constant fraction of elements in the candidate set
% is filtered out with constant probability, 
% based on the given threshold value,
% thus ensuring sublinear adaptivity.
% 
% In Alg.~\ref{alg:ptgone},
% we apply the same prefix selection step from \ts as
% \prefix (Alg.~\ref{alg:prefix} in Appendix~\ref{apx:subroutine})
% in Line~\ref{line:tgone-prefix}.
% If the prefix sizes returned for each solution are different,
% their minimum value $i^*$ are chosen and
% subsets of size $i^*$ are added to each solution 
% (Line~\ref{line:tgone-index}-\ref{line:tgone-update-A-2}).
Then, the smallest prefix size $i^*$ is selected in Line~\ref{line:tgone-index}.
For the solution where its corresponding call to \prefix returns $i^*$,
the entire prefix with size $i^*$ is added to it.
This ensures that a constant fraction of elements in $\mathcal V_j$
can be filtered out by Lemma~\ref{lemma:prefix-prob} in Appendix~\ref{apx:subroutine} with probability at least $1/2$.
Moreover, Lemma~\ref{lemma:dist} in Appendix~\ref{apx:subroutine}
guarantees that $|\mathcal V_j| \ge \frac{1}{\ell}|V_j|$
for each candidate set.
As a result, with constant probability,
at least one candidate set will filter out a constant fraction
of the elements.

\subsection{Ensuring Most Added Elements Significantly Contribute to the Solutions}
In \ts, the selection of a \textit{good prefix} inherently ensures this property immediately.
However, when interlacing $\ell$ threshold sampling processes,
prefix sizes selected in Line~\ref{line:tgone-prefix} by each solution may vary.
To preserve the alternating addition property introduced in Section~\ref{sec:ptg-alter},
subsets of equal size are selected instead of variable-length good prefixes.
This raises the question: \textit{How can a good subset be derived from a good prefix?}
The solution lies in Line~\ref{line:tgone-subset} of Alg.~\ref{alg:ptgone}.

For any $j \in I$, if $i_j^* = i^*$, $S_j$ is directly the good prefix $\mathcal V_j[1:i_j^*]$.
Otherwise, if $i_j^* \ge i^*$,
$i^*$ elements are selected from $\mathcal V_j[1:i_j^*]$ in three sequential passes until the size limit is reached:

\textbf{First pass}: Iterate through the prefix,
selecting those with marginal gains strictly greater than
$\tau_j$ (marked as \text{true} in $B_j$).

\textbf{Second pass}: 
From the remaining elements in the prefix, select those 
with marginal gains between $0$ and $\tau_j$
(marked as \text{none} in $B_j$).

\textbf{Third pass}: Fill any remaining slots with remaining elements from the prefix (marked as \text{false} in $B_j$).

This approach, combined with submodularity, 
ensures that any element marked as \textbf{true} in the selected subset has a marginal gain greater than $\tau_j$.
By prioritizing the addition of these \textbf{true} elements,
the selected subset remain high-quality while adhering to the alternating
addition framework.










\section{Experiment}\label{sec:experiment}
We carry out extensive experiments on four real-world datasets to answer the following research questions: 
\begin{itemize}[leftmargin=*]
    % \item \textbf{RQ1:} How does our proposed DEALRec perform compared to the coreset selection baselines for LLM-based recommendation and the models trained with full data? 
    \item \textbf{RQ1:} How does our proposed SETRec perform compared to different identifier baselines on different architectures of LLMs? 
    % \item \textbf{RQ2:} How do the different components of DEALRec (\ie influence score, gap regularization, and stratified sampling) affect the performance, and is DEALRec generalizable to different surrogate models? 
    \item \textbf{RQ2:} How do the different components of SETRec (\ie CF embeddings, semantic embeddings, query vectors, and sparse attention) affect the performance?
    \item \textbf{RQ3:} How does SETRec perform when scaling up the model size and how does SETRec improve the overall performance? 
    \item \textbf{RQ4:} How does SETRec perform with different number of semantic embeddings, tokenizer training strength, and semantic strength for inference? 
\end{itemize}
\subsection{Experimental Settings}
\subsubsection{\textbf{Datasets}}
We conduct experiments on four real-world datasets across various domains. 
From Amazon review datasets\footnote{\url{https://jmcauley.ucsd.edu/data/amazon/}.}, we adopt three widely used benchmarks 
1)\textbf{Toys}, 2) \textbf{Beauty}, and 3) \textbf{Sports}. 
The three Amazon datasets contain rich user interactions over a specific category of e-commerce products, where each item is associated with rich textual meta information such as title, description, category, and brand. 
In addition, we use a video games dataset 4) \textbf{Steam}\footnote{\url{https://github.com/kang205/SASRec}.} proposed in~\cite{kang2018self}, which contains substantial user interactions on video games with abundant textual semantic information. 
For all datasets, we follow previous work~\cite{wang2023causal} to sort user interactions chronologically according to the timestamps and divide them into training, validation, and testing sets with a ratio of 8:1:1. 
In addition, we divide the items into warm and cold items\footnote{We denote warm- and cold-start items as warm and cold items for brevity.}, where the items that appear in the training set are warm items, otherwise cold items. 


\noindent$\bullet\quad$\textbf{Evaluation.} 
We adopt the widely used metrics Recall@$K$ and NDCG@$K$, where $K=5$ and $10$ to evaluate all methods. 
Additionally, 
we introduce three different settings that evaluate over 1) all items, 2) warm items only, and 3) cold items only, respectively.  
% todo: 这里数据集可能要解释一下xxx为了保证cold数量能多一点,切割的比例是xxx


% Please add the following required packages to your document preamble:
% \usepackage{multirow}
% \usepackage[normalem]{ulem}
% \useunder{\uline}{\ul}{}
\begin{table*}[t]
\setlength{\abovecaptionskip}{0.05cm}
\setlength{\belowcaptionskip}{0.2cm}
\caption{Overall performance of baselines and SETRec instantiated on T5. The best results are in bold and the second-best results are underlined. $*$ implies the improvements over the second-best results are statistically significant ($p$-value < 0.01) under one-sample t-tests. ``Inf. Time'' denotes the inference time over all test users tested on a single NVIDIA RTX A5000 GPU.}
\setlength{\tabcolsep}{2mm}{
\resizebox{\textwidth}{!}{
\begin{tabular}{l|l|cccc|cccc|cccc|c}
\toprule
 &  & \multicolumn{4}{c|}{\textbf{All}} & \multicolumn{4}{c|}{\textbf{Warm}} & \multicolumn{4}{c|}{\textbf{Cold}} & \multicolumn{1}{l}{\textbf{Inf. Time (s)}} \\ \hline
\textbf{Dataset} & \textbf{Method} & \textbf{R@5} & \textbf{R@10} & \textbf{N@5} & \textbf{N@10} & \textbf{R@5} & \textbf{R@10} & \textbf{N@5} & \textbf{N@10} & \textbf{R@5} & \textbf{R@10} & \textbf{N@5} & \textbf{N@10} & \textbf{All Users} \\ \midrule
\multirow{9}{*}{\textbf{Toys}} & \textbf{DreamRec} & 0.0020 & 0.0027 & 0.0015 & 0.0018 & 0.0027 & 0.0039 & 0.0020 & 0.0024 & 0.0066 & 0.0168 & 0.0045 & 0.0082 & 912 \\
 & \textbf{E4SRec} & 0.0061 & 0.0098 & 0.0051 & 0.0064 & 0.0081 & 0.0128 & 0.0065 & 0.0082 & 0.0065 & 0.0122 & 0.0056 & 0.0078 & \textbf{55} \\ \cmidrule{2-15}
 & \textbf{BIGRec} & 0.0008 & 0.0013 & 0.0007 & 0.0009 & 0.0014 & 0.0019 & 0.0011 & 0.0013 & 0.0278 & 0.0360 & 0.0196 & 0.0223 & 2,079 \\
 & \textbf{IDGenRec} & 0.0063 & 0.0110 & 0.0052 & 0.0069 & 0.0109 & {\ul 0.0161} & 0.0081 & {0.0102} & {\ul 0.0318} & {\ul 0.0589} & {\ul 0.0236} & {\ul 0.0335} & 658 \\
 & \textbf{CID} & 0.0044 & 0.0082 & 0.0040 & 0.0053 & 0.0065 & 0.0128 & 0.0049 & 0.0071 & 0.0059 & 0.0111 & 0.0047 & 0.0066 & 810 \\
 & \textbf{SemID} & 0.0071 & 0.0108 & 0.0061 & 0.0074 & 0.0086 & 0.0153 & 0.0075 & 0.0100 & 0.0307 & 0.0507 & 0.0220 & 0.0292 & 1,215 \\
 & \textbf{TIGER} & 0.0064 & 0.0106 & 0.0060 & 0.0076 & 0.0091 & 0.0147 & 0.0080 & {\ul 0.0102} & 0.0315 & 0.0555 & 0.0228 & 0.0314 & 448 \\
 & \textbf{LETTER} & {\ul 0.0081} & {\ul 0.0117} & {\ul 0.0064} & {\ul 0.0077} & {\ul 0.0109} & 0.0155 & {\ul 0.0083} & 0.0101 & 0.0183 & 0.0395 & 0.0115 & 0.0190 & 448 \\  \cmidrule{2-15}
 & \cellcolor{gray!16}\textbf{SETRec} & \cellcolor{gray!16}\textbf{0.0110*} & \cellcolor{gray!16}\textbf{0.0189*} & \cellcolor{gray!16}\textbf{0.0089*} & \cellcolor{gray!16}\textbf{0.0118*} & \cellcolor{gray!16}\textbf{0.0139*} & \cellcolor{gray!16}\textbf{0.0236*} & \cellcolor{gray!16}\textbf{0.0112*} & \cellcolor{gray!16}\textbf{0.0147*} & \cellcolor{gray!16}\textbf{0.0443*} & \cellcolor{gray!16}\textbf{0.0812*} & \cellcolor{gray!16}\textbf{0.0310*} & \cellcolor{gray!16}\textbf{0.0445*} & \cellcolor{gray!16}{\ul 60} \\ \midrule\midrule
\multirow{9}{*}{\textbf{Beauty}} & \textbf{DreamRec} & 0.0012 & 0.0025 & 0.0013 & 0.0017 & 0.0016 & 0.0028 & 0.0016 & 0.0019 & 0.0078 & 0.0161 & 0.0065 & 0.0094 & 1,102 \\
 & \textbf{E4SRec} & 0.0061 & 0.0092 & 0.0052 & 0.0063 & 0.0080 & 0.0121 & 0.0067 & 0.0082 & 0.0072 & 0.0118 & 0.0065 & 0.0077 & \textbf{120} \\ \cmidrule{2-15}
 & \textbf{BIGRec} & 0.0054 & 0.0064 & 0.0051 & 0.0054 & 0.0008 & 0.0009 & 0.0006 & 0.0008 & 0.0106 & 0.0251 & 0.0095 & 0.0151 & 4,544 \\
 & \textbf{IDGenRec} & {\ul 0.0080} & 0.0115 & {\ul 0.0066} & {0.0078} & {\ul 0.0106} & 0.0165 & 0.0078 & 0.0099 & 0.0187 & 0.0350 & 0.0186 & 0.0224 & 840 \\
 & \textbf{CID} & 0.0071 & 0.0125 & 0.0060 & {\ul 0.0080} & 0.0098 & {0.0166} & 0.0077 & 0.0101 & 0.0087 & 0.0183 & 0.0071 & 0.0104 & 815 \\
 & \textbf{SemID} & 0.0071 & {\ul 0.0131} & 0.0056 & {0.0078} & 0.0098 & {\ul 0.0174} & 0.0074 & {\ul 0.0103} & {\ul 0.0260} & {\ul 0.0465} & 0.0178 & 0.0255 & 1,310 \\
 & \textbf{TIGER} & 0.0063 & 0.0098 & 0.0050 & 0.0062 & 0.0086 & 0.0131 & 0.0065 & 0.0082 & 0.0190 & 0.0325 & 0.0130 & 0.0178 & 430 \\ 
 & \textbf{LETTER} & 0.0071 & 0.0103 & 0.0061 & 0.0070 & 0.0094 & 0.0135 & {\ul 0.0079} & 0.0091 & 0.0251 & 0.0410 & {\ul 0.0241} & {\ul 0.0285} & 430 \\ \cmidrule{2-15}
 & \cellcolor{gray!16}\textbf{SETRec} & \cellcolor{gray!16}\textbf{0.0106*} & \cellcolor{gray!16}\textbf{0.0161*} & \cellcolor{gray!16}\textbf{0.0083*} & \cellcolor{gray!16}\textbf{0.0103*} & \cellcolor{gray!16}\textbf{0.0139*} & \cellcolor{gray!16}\textbf{0.0212*} & \cellcolor{gray!16}\textbf{0.0108*} & \cellcolor{gray!16}\textbf{0.0134*} & \cellcolor{gray!16}\textbf{0.0384*} & \cellcolor{gray!16}\textbf{0.0761*} & \cellcolor{gray!16}\textbf{0.0280*} & \cellcolor{gray!16}\textbf{0.0413*} & \cellcolor{gray!16}{\ul 126} \\ \midrule\midrule
\multirow{9}{*}{\textbf{Sports}} & \textbf{DreamRec} & 0.0027 & 0.0044 & 0.0025 & 0.0031 & 0.0032 & 0.0052 & 0.0028 & 0.0035 & 0.0045 & 0.0108 & 0.0026 & 0.0049 & 2,100 \\ 
 & \textbf{E4SRec} & 0.0079 & 0.0131 & 0.0075 & 0.0094 & 0.0092 & 0.0154 & 0.0085 & 0.0107 & 0.0031 & 0.0093 & 0.0019 & 0.0039 & \textbf{117} \\ \cmidrule{2-15}
 & \textbf{BIGRec} & 0.0033 & 0.0042 & 0.0030 & 0.0033 & 0.0001 & 0.0002 & 0.0001 & 0.0001 & 0.0059 & 0.0104 & 0.0043 & 0.0061 & 7,822 \\
 & \textbf{IDGenRec} & 0.0087 & 0.0127 & 0.0079 & 0.0092 & 0.0101 & 0.0149 & 0.0091 & 0.0107 & 0.0181 & 0.0302 & 0.0134 & 0.0179 & 1,724 \\
 & \textbf{CID} & 0.0077 & 0.0131 & 0.0073 & 0.0092 & 0.0074 & 0.0119 & 0.0045 & 0.0061 & 0.0082 & 0.0149 & 0.0075 & 0.0099 & 2,135 \\
 & \textbf{SemID} & {\ul 0.0094} & {\ul 0.0167} & {\ul 0.0088} & {\ul 0.0114} & {\ul 0.0119} & {\ul 0.0201} & {\ul 0.0104} & {\ul 0.0135} & {\ul 0.0254} & {\ul 0.0495} & {\ul 0.0175} & {\ul 0.0256} & 2,367 \\
 & \textbf{TIGER} & 0.0085 & 0.0129 & 0.0080 & 0.0095 & 0.0100 & 0.0151 & 0.0091 & 0.0109 & 0.0190 & 0.0310 & 0.0120 & 0.0159 & 481 \\
 & \textbf{LETTER} & 0.0077 & 0.0131 & 0.0073 & 0.0092 & 0.0074 & 0.0119 & 0.0045 & 0.0061 & 0.0082 & 0.0149 & 0.0075 & 0.0099 & 481 \\ \cmidrule{2-15}
 & \cellcolor{gray!16}\textbf{SETRec} & \cellcolor{gray!16}\textbf{0.0114*} & \cellcolor{gray!16}\textbf{0.0185*} & \cellcolor{gray!16}\textbf{0.0101*} & \cellcolor{gray!16}\textbf{0.0126*} & \cellcolor{gray!16}\textbf{0.0134*} & \cellcolor{gray!16}\textbf{0.0216*} & \cellcolor{gray!16}\textbf{0.0115*} & \cellcolor{gray!16}\textbf{0.0144*} & \cellcolor{gray!16}\textbf{0.0341*} & \cellcolor{gray!16}\textbf{0.0595*} & \cellcolor{gray!16}\textbf{0.0233*} & \cellcolor{gray!16}\textbf{0.0323*} & \cellcolor{gray!16}{\ul 136} \\ \midrule\midrule
\multirow{9}{*}{\textbf{Steam}} & \textbf{DreamRec} & 0.0029 & 0.0057 & 0.0037 & 0.0046 & 0.0042 & 0.0080 & 0.0045 & 0.0059 & 0.0017 & 0.0029 & 0.0013 & 0.0018 & 4,620 \\
 & \textbf{E4SRec} & 0.0194 & 0.0351 & 0.0220 & 0.0270 & 0.0312 & 0.0558 & 0.0283 & 0.0370 & 0.0006 & 0.0010 & 0.0006 & 0.0007 & \textbf{328} \\ \cmidrule{2-15}
 & \textbf{BIGRec} & 0.0030 & 0.0049 & 0.0046 & 0.0049 & 0.0048 & 0.0053 & 0.0061 & 0.0053 & 0.0099 & 0.0107 & {\ul 0.0129} & 0.0127 & 5,167 \\
 & \textbf{IDGenRec} & 0.0199 & 0.0307 & 0.0241 & 0.0265 & 0.0309 & 0.0479 & 0.0311 & 0.0363 & 0.0047 & 0.0151 & 0.0039 & 0.0078 & 2,846 \\
 & \textbf{CID} & 0.0200 & {\ul 0.0360} & {\ul 0.0249} & {\ul 0.0295} & 0.0314 & {\ul 0.0566} & {\ul 0.0315} & {\ul 0.0400} & 0.0008 & 0.0021 & 0.0006 & 0.0011 & 3,194 \\
 & \textbf{SemID} & 0.0155 & 0.0278 & 0.0192 & 0.0229 & 0.0248 & 0.0443 & 0.0246 & 0.0313 & 0.0017 & 0.0027 & 0.0015 & 0.0018 & 3,605 \\
 & \textbf{TIGER} & {\ul 0.0202} & 0.0348 & 0.0244 & 0.0287 & {\ul 0.0320} & 0.0552 & 0.0314 & 0.0393 & 0.0060 & {0.0152} & 0.0044 & 0.0078 & 1,747 \\
 & \textbf{LETTER} & 0.0164 & 0.0312 & 0.0195 & 0.0244 & 0.0268 & 0.0500 & 0.0253 & 0.0336 & {\ul 0.0115} & {\ul 0.0317} & {0.0077} & {\ul 0.0157} & 1,747 \\ \cmidrule{2-15}
 & \cellcolor{gray!16}\textbf{SETRec} & \cellcolor{gray!16}\textbf{0.0216*} & \cellcolor{gray!16}\textbf{0.0383*} & \cellcolor{gray!16}\textbf{0.0254*} & \cellcolor{gray!16}\textbf{0.0308*} & \cellcolor{gray!16}\textbf{0.0339*} & \cellcolor{gray!16}\textbf{0.0591*} & \cellcolor{gray!16}\textbf{0.0326*} & \cellcolor{gray!16}\textbf{0.0414*} & \cellcolor{gray!16}\textbf{0.0313*} & \cellcolor{gray!16}\textbf{0.0572*} & \cellcolor{gray!16}\textbf{0.0248*} & \cellcolor{gray!16}\textbf{0.0342*} & \cellcolor{gray!16}{\ul 347} \\ \hline
\end{tabular}
}}
\label{tab:overall_performance}
\end{table*}


\subsubsection{\textbf{Baselines}}
We compare SETRec with competitive baselines, including single-token identifiers (DreamRec, E4SRec) and token-sequence identifiers (BIGRec, IDGenRec, CID, SemID, TIGER, LETTER). 
1) \textbf{DreamRec}~\cite{yang2024generate} is a closely related method that leverages ID embedding to represent each item and adopts a diffusion model to refine the generated ID embedding from LLMs.  
2) \textbf{E4SRec}~\cite{li2023e4srec} utilizes a pre-trained CF model to obtain ID embedding, and uses a linear projection layer to obtain the item scores efficiently. 
3) \textbf{BIGRec}~\cite{bao2023bi} adopts item titles as identifiers, where the tokens are from human vocabulary. 
4) \textbf{IDGenRec}~\cite{tan2024idgenrec} is a learnable ID generator, which aims to generate concise but informative tags from human vocabulary to represent each item. 
5) \textbf{CID}~\cite{hua2023index} leverages hierarchical clustering to obtain token sequence, which utilizes item co-occurrence matrix to obtain identifiers to ensure items with similar interactions share similar tokens. 
6) \textbf{SemID}~\cite{hua2023index} also represents items with external token sequence, which is obtained based on the hierarchical item category. 
7) \textbf{TIGER}~\cite{rajput2023recommender} leverages RQ-VAE with codebooks to quantize item semantic information into token sequence with external tokens. The identifier sequentially contains coarse-grained to fine-grained information. 
8) \textbf{LETTER}~\cite{wang2024learnable} is one of the SOTA item tokenization methods, which incorporates both semantic and CF information into the training of RQ-VAE, achieving identifiers with multi-dimensional information and improved diversity. 

\subsubsection{\textbf{Implementation Details}} 
% 我们把所有的identifier方法都instantiate到了两个不同的LLMs上,T5-small 和 Qwen上,其中我们用1.5B来测试overall performance,然后还扩展到3B和7B上去验证scalability。
% 针对tokenizer的训练,对于使用到AE的方法(TIGER, LETTER,还有我们的方法),我们统一了隐藏层在"512,256,128". 
% 对于LLM的训练,我们为所有方法设置一样的prompt as "xxx"
% 对于T5-small模型,我们是全量微调。对于Qwen模型,我们采用parameter-efficeint tuning technique LoRA~\cite{}. 并且所有实验在4块A5000上跑。
% 针对我们的方法,N的数量在{1,2,3,4,5,6}里面选,alpha在0.1,0.3,0.5,0.7,0.9里选。而inference阶段的beta则是从0-1选。
We instantiate all methods on two LLMs with different architectures, \ie T5-small~\cite{raffel2020exploring} (encoder-decoder) and Qwen2.5~\cite{yang2024qwen2} (decoder-only). 
Specifically, we adopt Qwen\footnote{We denote T5-small and Qwen2.5 as T5 and Qwen for brevity.} with different sizes, including 1.5B, 3B, and 7B, for a comprehensive evaluation. 
To ensure a fair comparison, we set the hidden layer dimensions at 512, 256, and 128 with ReLU activation for methods that adopt AE in tokenizer training, including TIGER, LETTER, and our proposed SETRec. 
For LLM training, 
we adopt the same prompt for all methods as ``What would the user be likely to purchase next after buying items {history}?;'' for a fair comparison. 
We fully fine-tune the T5 model and perform parameter-efficient fine-tuning technique LoRA~\cite{hu2021lora} for Qwen. 
All experiments are conducted on four NVIDIA RTX A5000 GPUs. 
% For SETRec, 
% we use SASRec~\cite{kang2018self} as pre-trained CF model, and utilize 
% SentenceT5 and Qwen as semantic extractors for T5 and Qwen backend LLMs, respectively. 
For SETRec, 
we select $N$, $\alpha$, and $\beta$ from $\{1,2,3,4,5,6\}$, $\{0.1,0.3,0.5,0.7,0.9\}$, and $\{0, 0.1, 0.2, 0.3, 0.4, 0.5, 0.6, 0.7, 0.8, 0.9,1.0\}$, respectively. 


\begin{table*}[t]
\setlength{\abovecaptionskip}{0.05cm}
\setlength{\belowcaptionskip}{0.2cm}
\caption{Overall performance on Qwen-1.5B over Toys and Beauty. The best results are in bold and the second-best results are underlined. ``Inf. Time'' denotes the inference time over all test users tested on a single NVIDIA RTX A5000 GPU.}
\setlength{\tabcolsep}{2mm}{
\resizebox{\textwidth}{!}{
\begin{tabular}{l|l|cccc|cccc|cccc|c}
\toprule
 &  & \multicolumn{4}{c}{\textbf{All}} & \multicolumn{4}{c}{\textbf{Warm}} & \multicolumn{4}{c}{\textbf{Cold}} & \textbf{Inf. Time(s)} \\ \hline
\textbf{Dataset} & \textbf{Method} & \textbf{R@5} & \textbf{R@10} & \textbf{N@5} & \textbf{N@10} & \textbf{R@5} & \textbf{R@10} & \textbf{N@5} & \textbf{N@10} & \textbf{R@5} & \textbf{R@10} & \textbf{N@5} & \textbf{N@10} & \textbf{All Users} \\ \midrule
\multirow{9}{*}{\textbf{Toys}} & \textbf{DreamRec} & 0.0006 & 0.0013 & 0.0005 & 0.0008 & 0.0008 & 0.0019 & 0.0007 & 0.0012 & 0.0076 & 0.0137 & 0.0052 & 0.0074 & 1,093 \\
 & \textbf{E4SRec} & 0.0065 & 0.0108 & {\ul 0.0056} & 0.0072 & 0.0089 & 0.0144 & {\ul 0.0075} & {\ul 0.0096} & 0.0084 & 0.0235 & 0.0055 & 0.0111 & \textbf{905} \\ \cmidrule{2-15} 
 & \textbf{BIGRec} & 0.0009 & 0.0016 & 0.0009 & 0.0012 & 0.0011 & 0.0013 & 0.0010 & 0.0011 & 0.0194 & 0.0311 & 0.0147 & 0.0191 & 43,304 \\
 & \textbf{IDGenRec} & 0.0030 & 0.0053 & 0.0022 & 0.0031 & 0.0043 & 0.0086 & 0.0032 & 0.0048 & 0.0189 & 0.0364 & 0.0161 & 0.0224 & 30,720 \\
 & \textbf{CID} & 0.0027 & 0.0047 & 0.0025 & 0.0033 & 0.0055 & 0.0084 & 0.0044 & 0.0056 & 0.0055 & 0.0156 & 0.0044 & 0.0081 & {27,248} \\
 & \textbf{SemID} & 0.0024 & 0.0042 & 0.0018 & 0.0024 & 0.0034 & 0.0055 & 0.0026 & 0.0034 & 0.0140 & 0.0275 & 0.0095 & 0.0143 & 32,288 \\
 & \textbf{TIGER} & {\ul 0.0068} & {\ul 0.0117} & 0.0054 & {\ul 0.0072} & {\ul 0.0094} & {\ul 0.0159} & 0.0070 & 0.0095 & {\ul 0.0384} & {\ul 0.0715} & {\ul 0.0291} & {\ul 0.0408} & {13,800} \\
 & \textbf{LETTER} & 0.0057 & 0.0093 & 0.0050 & 0.0064 & 0.0080 & 0.0126 & 0.0066 & 0.0085 & 0.0217 & 0.0416 & 0.0170 & 0.0239 & 13,800 \\ \cmidrule{2-15} 
 & \cellcolor[HTML]{ECF4FF}\textbf{SETRec} & \cellcolor[HTML]{ECF4FF}\textbf{0.0116*} & \cellcolor[HTML]{ECF4FF}\textbf{0.0188*} & \cellcolor[HTML]{ECF4FF}\textbf{0.0095*} & \cellcolor[HTML]{ECF4FF}\textbf{0.0120*} & \cellcolor[HTML]{ECF4FF}\textbf{0.0144*} & \cellcolor[HTML]{ECF4FF}\textbf{0.0236*} & \cellcolor[HTML]{ECF4FF}\textbf{0.0118*} & \cellcolor[HTML]{ECF4FF}\textbf{0.0151*} & \cellcolor[HTML]{ECF4FF}\textbf{0.0531*} & \cellcolor[HTML]{ECF4FF}\textbf{0.0883*} & \cellcolor[HTML]{ECF4FF}\textbf{0.0382*} & \cellcolor[HTML]{ECF4FF}\textbf{0.0507*} & \cellcolor[HTML]{ECF4FF}{\ul 926} \\ \midrule\midrule
\multirow{9}{*}{\textbf{Beauty}} & \textbf{DreamRec} & 0.0007 & 0.0009 & 0.0005 & 0.0005 & 0.0010 & 0.0011 & 0.0007 & 0.0007 & 0.0090 & 0.0167 & 0.0075 & 0.0103 & 1,326 \\
 & \textbf{E4SRec} & {\ul 0.0067} & {\ul 0.0109} & {\ul 0.0056} & {\ul 0.0072} & {\ul 0.0088} & {\ul 0.0146} & {\ul 0.0072} & {\ul 0.0094} & 0.0017 & 0.0071 & 0.0010 & 0.0029 & \textbf{910} \\ \cmidrule{2-15} 
 & \textbf{BIGRec} & 0.0006 & 0.0010 & 0.0006 & 0.0007 & 0.0010 & 0.0010 & 0.0008 & 0.0008 & 0.0141 & 0.0246 & 0.0094 & 0.0135 & 29,500 \\
 & \textbf{IDGenRec}  & 0.0042 & 0.0078 & 0.0030 & 0.0043 & 0.0045 & 0.0104 & 0.0033 & 0.0054 & {\ul 0.0254} & {\ul 0.0471} & {\ul 0.0207} & {\ul 0.0292} & 35,040 \\
 & \textbf{CID} & 0.0046 & 0.0077 & 0.0040 & 0.0052 & 0.0059 & 0.0107 & 0.0051 & 0.0068 & 0.0075 & 0.0155 & 0.0071 & 0.0096 & {27,792} \\
 & \textbf{SemID} & 0.0030 & 0.0045 & 0.0027 & 0.0033 & 0.0050 & 0.0076 & 0.0042 & 0.0052 & 0.0159 & 0.0227 & 0.0116 & 0.0159 & 45,160 \\
 & \textbf{TIGER} & 0.0041 & 0.0065 & 0.0032 & 0.0041 & 0.0054 & 0.0085 & 0.0042 & 0.0054 & 0.0083 & 0.0167 & 0.0064 & 0.0091 & {12,600} \\
 & \textbf{LETTER} & 0.0040 & 0.0069 & 0.0031 & 0.0042 & 0.0051 & 0.0088 & 0.0039 & 0.0054 & 0.0043 & 0.0129 & 0.0043 & 0.0071 & 12,600 \\ \cmidrule{2-15} 
 & \cellcolor[HTML]{ECF4FF}\textbf{SETRec} & \cellcolor[HTML]{ECF4FF}\textbf{0.0104*} & \cellcolor[HTML]{ECF4FF}\textbf{0.0167*} & \cellcolor[HTML]{ECF4FF}\textbf{0.0085*} & \cellcolor[HTML]{ECF4FF}\textbf{0.0108*} & \cellcolor[HTML]{ECF4FF}\textbf{0.0140*} & \cellcolor[HTML]{ECF4FF}\textbf{0.0221*} & \cellcolor[HTML]{ECF4FF}\textbf{0.0109*} & \cellcolor[HTML]{ECF4FF}\textbf{0.0141*} & \cellcolor[HTML]{ECF4FF}\textbf{0.0477*} & \cellcolor[HTML]{ECF4FF}\textbf{0.0748*} & \cellcolor[HTML]{ECF4FF}\textbf{0.0370*} & \cellcolor[HTML]{ECF4FF}\textbf{0.0464*} & \cellcolor[HTML]{ECF4FF}{\ul 1,050} \\ \bottomrule
\end{tabular}
}}
\label{tab:Overall_performance_on_Qwen}
\end{table*}



\subsection{Overall Performance (RQ1)}\label{sec:overall_performance}


\subsubsection{\textbf{Performance on T5.}} 
The performance comparison between baselines and SETRec instantiated on T5 are shown in Table~\ref{tab:overall_performance}, from which we have the following observations: 
\begin{itemize}[leftmargin=*]
    % 1. token-seq-based 整体会比单一的embedding表示好。这是因为他们利用了多token来表示丰富的item信息。针对用human vocab来表示的方法,他们能够利用上语言模型内部的pre-training知识;针对那些词表的方法,他们将信息压缩到了多个token里,让item的表示更加具有层次化。
    \item Token-sequence identifier (BIGRec, IDGenRec, CID, SemID, TIGER, LETTER) generally performs better than single-token identifier under ``all'', ``warm'', and ``cold'' settings. This is reasonable because token-sequence identifier represent each item with multiple tokens, which explicitly encode rich item information into different dimensions.   
    % 2. token-seq-based中,用codebook的比用human vocab的大部分情况要好一些。这主要是因为他们利用了hierarchy的信息,从粗粒度到细粒度,一定程度上缓解了local optima的问题。
    \item Among the token-sequence identifiers, methods with external tokens (CID, SemID, TIGER, LETTER) generally outperform those relying on human vocabulary (\eg BIGRec) under ``all'' and ``warm'' settings. 
    This is attributed to their hierarchically structured identifier, where the initial tokens represent coarse-grained semantics while subsequent tokens contain fine-grained semantics. 
    This aligns better with the autoregressive generation process, potentially alleviating the local optima issue~\cite{wang2024learnable}. 
    % 3. 分析一下在cold场景下哪些更好:对于只用cf的方法(dreamrec, e4srec, cid),他们在cold上面效果不行。而那些利用了寓意信息的方法,在cold上表现就比较优秀。但是codebook的大多数情况仍然不如human vocab的那些方法。
    \item When recommending cold items\footnote{The higher values on cold performance are due to the limited number of cold items.}, methods that merely utilize CF information (DreamRec, E4SRec, and CID) fail to give satisfying results. 
    This is not surprising since CF information depends heavily on substantial interactions for training, thereby struggling with cold items. 
    In contrast, methods that integrate semantics into identifiers (BIGRec, IDGenRec, SemID, TIGER, and LETTER) generalize better on cold-start scenarios (superior performance under ``cold'' setting). 
    Specifically, BIGRec and IDGenRec tend to have competitive performance. 
    This is reasonable because they utilize readable human vocabulary to represent each item, which better leverages rich world knowledge encoded in LLMs. 

    % 4. 我们的方法significantly/constantly超过了其他方法。在accuracy上,我们在all,warm,和cold上都显著超越。我们利用了cf的信息,让那些拥有丰富交互的warm item能够被准确的推荐;此外我们利用了多维度的semantic信息,这让我们的模型能够泛化到cold item上面去。
    \item SETRec significantly outperform all baselines under ``all'', ``warm'', and ``cold'' settings across all four datasets. 
    The superior performance is attributed to 
    % 1) the incorporation of both CF and semantic information, which ensures the items with similar interactions have similar identifiers, thus recommending warm items accurately; 
    % 2) representation of rich semantics into multiple embeddings, which encourages the identifier to contain semantics of different dimensions, thus strengthening the cold-start generalization. 
    1) the incorporation of both CF and semantic information into a set of tokens, which ensures accurate warm item recommendation and strong generalization on cold items; 
    2) order agnosticism of identifier, which removes the possibly inaccurate dependencies across different tokens associated with an identifier. 
    
    \item From the perspective of efficiency, SETRec significantly reduces the inference time costs compared to the token-sequence identifiers. 
    SETRec achieves an average 15$\times$, 11$\times$, 18$\times$, and 8$\times$ speedup on Toys, Beauty, Sports, and Steam, respectively, compared to token-sequence identifiers. 
    The high efficiency is attributed to the simultaneous generation, which generates multiple tokens at a single LLM call, unlocking the real-world deployment of LLM-based generative recommendation. 
    % from perspective of efficiency, 我们的方法显著的超越了seq-based的这些方法,实现用一个single step就能够生成
\end{itemize}


% Overall performance on Qwen
% Please add the following required packages to your document preamble:
% \usepackage{multirow}

% \noindent$\bullet\quad$\textbf{Performance on Qwen-1.5B.} 




\subsubsection{\textbf{Performance on Qwen-1.5B}}
To evaluate SETRec on decoder-only LLMs, we instantiate SETRec and all baselines on Qwen-1.5B. We present the results on Toys and Beauty\footnote{We omit the results with similar observations on other datasets to save space.} in Table~\ref{tab:Overall_performance_on_Qwen}, from which we summarize several key different observations from performance on T5 as follows: 
% observations

\begin{itemize}[leftmargin=*]
    % 1. 在qwen上和t5不同的地方是,seq-based失去了它显著的效果,我们猜测这主要是因为qwen的参数量更大,他拥有更强的预训练知识。因此难以在数据有限的情况下很快的adpat到推荐任务上。相反的,E4SRec大部分情况能有非常competitive 的performance。我们猜测这主要是因为它把之前的vocabulary head换成了新的logits,这样利于大语言模型从原来的pre-training任务上adapt到推荐任务上,通过后面那个head高效调整。  
    \item Token-sequence identifiers show limited competitiveness compared to the counterparts on T5. 
    % We suspect that this might be caused by the magnified knowledge gap between the pre-training data and the recommendation data. 
    A possible reason is that Qwen-1.5B probably contains richer knowledge within its parameters, which amplifies the knowledge gap between the pre-training and recommendation tasks,  thereby hindering its adaptation to recommendation tasks with limited interaction data.  
    Conversely, E4SRec yields competitive performance in most cases. 
    This makes sense because E4SRec removes the original vocabulary head and replaces it with an item projection head, thus facilitating effective adaption to the recommendation tasks. 
    % 2. 和t5相比,这些用human vocab的在cold上面会有比较好的performance。-> 这个符合直觉。但是词表这种反而下降了,这个也符合直觉,因为需要更多的interaction来adapt,否则生成概率会偏低
    \item BIGRec and IDGenRec outperform their T5 counterparts on cold items on Beauty. Because they represent items with human vocabulary, which can leverage the rich world knowledge within Qwen-1.5B for better generalization. 
    On the contrary, identifiers with external tokens have inferior cold performance compared to their T5 counterparts. 
    This is also reasonable since it requires extensive interaction data to train external tokens. Otherwise, it is difficult for it to generalize to cold items accurately due to the low generation probability of these external tokens. 
    % 3. 我们的方法仍然能稳定的超过baseline,. 并且稳定的比t5要好。尤其是cold上面的performance。-> 在qwen上比较好的表现验证了我们方法在不同模型架构上的泛化能力。 
    \item SETRec constantly outperforms baselines, which is consistent with the observations on T5. 
    Notably, SETRec instantiated on Qwen-1.5B steadily surpasses SETRec on T5, especially under the ``cold'' setting. 
    This validates the strong generalization ability of SETRec on different architectures of LLMs. 
    Moreover, as the LLM size increases, the efficiency improvements over the token-sequence identifiers are more significant, resulting in an average of 20$\times$ speedup across the two datasets. 
    
\end{itemize}




\subsection{In-depth Analysis}

\subsubsection{\textbf{Ablation Study (RQ2)}} 
To study the effectiveness of each component of SETRec, we separately remove semantic tokens (``w/o Sem''), 
CF token (``w/o CF'').  
In addition, we replace learnable query vectors with random frozen vectors (``w/o Query'') and 
use the original attention mask (``w/o SA''), to evaluate the effect of query vectors and the sparse attention mask, respectively. 
The results of different ablation variants on T5 and Qwen-1.5B on Toys are presented in Figure~\ref{fig:ablation} and we omit the results on other datasets with similar observations to save space. 

From the figures, we can find similar observations on T5 and Qwen that 
% t5和qwen都有的现象:
% 1. 单独移除每一个元素在all,warm, cold 上performance都下降了。这验证了每个component的有效性。
1) removing each component causes performance drops under ``all'', ``warm'', and ``cold'' settings, which validates the effectiveness of each component of SETRec. 
% 2. 一处semantic对cold的影响非常大。这也说明了semantic对于cold start item的重要性。验证了引入semantic是必要的。
2) Discarding semantic tokens drastically degrades the recommendation accuracy under ``cold'' settings. 
This demonstrates the necessity of incorporating semantics into identifiers. 
% 3. 相比于移除cf embedding,移除semantic反而会让performance下降更多。这个interesting现象我们猜测是源于我们使用了多个embedding。这个现象也和XXX里的观测一致。我们补充了只有一个semantic的实验结果在appendix)
Interestingly, 
3) removing semantic tokens leads to worse performance compared to removing CF token. 
The possible reason for this is the utilization of multiple semantic tokens to represent each item, which highlights the significance of leveraging multi-dimensional semantic information. 
This observation is also consistent with the results in~\cite{lin2024bridging}. 
% t5和qwen不一样的现象:主要是在cold上,去掉cf有时候反而有更好的cold start performance。这个可能的原因是参数量大的模型能比参数量小的模型拥有更好的语义理解。更detialed analysis of the balance between cf and semantics are provided in XXX
Nonetheless, 
4) while removing CF tokens for T5 leads to inferior performance on cold items, using CF tokens for Qwen might negatively impact on cold items. 
A possible reason is that the larger-size Qwen is better at understanding semantics due to its stronger knowledge base encoded in the parameters, making the contribution of CF less significant. 



% % ablation figures on Toys
% \begin{figure}[t]
% % \vspace{-0.2cm}
% \setlength{\abovecaptionskip}{-0.15cm}
% \setlength{\belowcaptionskip}{-0cm}
%   \centering 
%   % \hspace{-0.7in}
%   \subfigure{
%     \includegraphics[height=1.35in]{figures/ablation-toys-t5-all.pdf}} 
%   % \hspace{-0.105in}
%   \subfigure{
%     \includegraphics[height=1.35in]{figures/ablation-toys-qwen-all.pdf}} 
%   % \hspace{-0.105in}
%   \subfigure{
%     \includegraphics[height=1.35in]{figures/ablation-toys-t5-warm.pdf}} 
%   % \hspace{-0.105in}
%   \subfigure{
%     \includegraphics[height=1.35in]{figures/ablation-toys-qwen-warm.pdf}} 
%   % \hspace{-0.105in}
%   \subfigure{
%     \includegraphics[height=1.35in]{figures/ablation-toys-t5-cold.pdf}} 
%   % \hspace{-0.105in}
%   \subfigure{
%     \includegraphics[height=1.35in]{figures/ablation-toys-qwen-cold.pdf}} 
%   % \hspace{-0.105in}
% \caption{Ablation study on Toys.}
%   \label{fig:ablation}
%   % \vspace{-0.3cm}
% \end{figure}



\begin{figure}[t]
% \vspace{-0.2cm}
\setlength{\abovecaptionskip}{0.02cm}
\setlength{\belowcaptionskip}{-0.3cm}
\centering
\includegraphics[scale=1.2]{figures/ablation.pdf}
\caption{Ablation study on Toys.}
\label{fig:ablation}
\end{figure}

\subsubsection{\textbf{Item Group Analysis (RQ3)}}
To understand how SETRec improves performance, we evaluate it over items with different popularity. 
% item group是怎么划分的 - 我们根据item popularity 排序,然后分成5组到Group1-group5, (从最popular到最不popular)
We divide the items into 5 groups according to their frequencies and test the models over each group respectively. 
The performance comparison between SETRec and two competitive baselines from token-sequence identifiers (LETTER) and single-token identifiers (E4SRec) are reported in Figure~\ref{fig:group_analysis}. 
We can observe that 
% 1. 从most popular 到least popular, item group performance 是在逐渐下降的,这符合预期。因为交互越少的item,llm能够decode出来的概率就会更低,欠拟合
1) the performance gradually drops from G1 to G5. 
This makes sense since the less popular items have fewer interactions for LLMs to learn, thus leading to worse generation probabilities. 
% 2. e4srec在第一组比letter要强很多,但是随着item的popularity降低,letter慢慢超过e4srec。这也符合我们的直觉。e4srec是纯靠cf信息的,非常依赖于大量的交互来学习cf info。而letter同时利用了语义信息,会在sparse的item上面有更好的表现
Besides, 
2) E4SRec outperforms LETTER on most popular items (G1) but usually yields inferior performance on unpopular items (G2-G5). 
This is due to that E4SRec only uses CF information, which relies on substantial interactions and therefore struggle on unpopular items. 
In contrast, LETTER additionally incorporates semantics into identifiers, thus achieving better generalization on sparse items. 
% 3. 每一组里面我们的方法都稳定的超过了competitive baselines。除此之外提升的百分比是在unpopular的item上面有更强的优势。这也部分说明了我们方法的泛化能力很强。
3) SETRec consistently excels both E4SRec and LETTER over all groups. 
Notably, the improvements over sparse items are more significant, which partially explains the superiority of SETRec regarding overall performance.  



% group analysis figure
\begin{figure}[t]
\vspace{-0.2cm}
\setlength{\abovecaptionskip}{-0.15cm}
\setlength{\belowcaptionskip}{-0cm}
  \centering 
  % \hspace{-0.7in}
  \subfigure{
    \includegraphics[height=1.65in]{figures/group_analysis_R10.pdf}} 
  % \hspace{-0.105in}
  \subfigure{
    \includegraphics[height=1.65in]{figures/group_analysis_N10.pdf}} 
\caption{Performance of SETRec, LETTER, and E4SRec (T5) on item groups with different popularity on Toys.}
  \label{fig:group_analysis}
  % \vspace{-0.3cm}
\end{figure}

\subsubsection{\textbf{Scalability on Model Parameters (RQ3)}}
To investigate whether SETRec can bring continuous performance when expanding the model parameters, we test SETRec on Qwen with different model sizes (1.5B, 3B, and 7B). 
Performance comparisons between SETRec, E4SRec, and LETTER on Toys are shown in Table~\ref{tab:scaling_performance}. 
% and the results on other datasets with similar observations are omitted to save space.
From the results, we can find that 
% 1. SETRec在cold上有比较明显的scaling,这因为模型对语义理解的能力更强。这展现了在cold start上比较promising的scaling的能力
1) SETRec clearly shows continued improvements over cold-start items when the model size scales from 1.5B to 7B, demonstrating promising scalability on cold items. 
We attribute this to the continued improvements of better semantic understanding by expanding the model parameters. 
% 2. SETRec在warm上可能已经到达瓶颈了,随着参数量的提升,对cf信息的接受没有进一步的提升。这个在e4srec的结果上也可以看得出来
Nonetheless, 
2) the performance on the warm items fails to continuously improve, indicating a relatively limited scalability over warm items. 
This shows that the larger models do not necessarily lead to better CF information understanding, which can also be indicated by the limited improvements of E4SRec under ``warm'' setting. 
% 3. 对于LETTER这种利用语意的identifier方法,也面临瓶颈。主要是因为扩展词表,其实和模型内部的语义没有很好的align,继续scale模型对于cold的提升其实作用并不明显
Besides, 
3) LETTER shows weak scalability over the three settings. 
This is mainly due to the utilization of external tokens, which do not necessarily align with the pre-trained knowledge in LLMs, thus showing limited improvements by expanding the model parameters. 


% Please add the following required packages to your document preamble:
% \usepackage{multirow}
% \begin{table*}[t]
% \setlength{\abovecaptionskip}{0.05cm}
% \setlength{\belowcaptionskip}{0.2cm}
% \caption{Performance comparison between SETRec and competitive baselines with different LLM sizes on Qwen. }
% \setlength{\tabcolsep}{2.5mm}{
% \resizebox{\textwidth}{!}{
% \begin{tabular}{clccccccccccccl}
% \toprule
% \multicolumn{15}{c}{\textbf{Toys}} \\ \hline
% \multicolumn{1}{l|}{} & \multicolumn{1}{l|}{} & \multicolumn{4}{c|}{\textbf{All}} & \multicolumn{4}{c|}{\textbf{Warm}} & \multicolumn{4}{c|}{\textbf{Cold}} & \multicolumn{1}{c}{\textbf{Inference Time (s)}} \\ \hline
% \multicolumn{1}{l|}{\textbf{Model Size}} & \multicolumn{1}{l|}{\textbf{Method}} & \textbf{R@5} & \textbf{R@10} & \textbf{N@5} & \multicolumn{1}{c|}{\textbf{N@10}} & \textbf{R@5} & \textbf{R@10} & \textbf{N@5} & \multicolumn{1}{c|}{\textbf{N@10}} & \textbf{R@5} & \textbf{R@10} & \textbf{N@5} & \multicolumn{1}{c|}{\textbf{N@10}} & \multicolumn{1}{c}{\textbf{All Users}} \\ \midrule
% \multicolumn{1}{c|}{\multirow{3}{*}{\textbf{1.5B}}} & \multicolumn{1}{l|}{\textbf{LETTER}} & 0.0057 & 0.0093 & 0.005 & \multicolumn{1}{c|}{0.0064} & 0.008 & 0.0126 & 0.0066 & \multicolumn{1}{c|}{0.0085} & 0.0217 & 0.0416 & 0.017 & \multicolumn{1}{c|}{0.0239} &  \\
% \multicolumn{1}{c|}{} & \multicolumn{1}{l|}{\textbf{E4SRec}} & 0.0065 & 0.0108 & 0.0056 & \multicolumn{1}{c|}{0.0072} & 0.0089 & 0.0144 & 0.0075 & \multicolumn{1}{c|}{0.0096} & 0.0084 & 0.0235 & 0.0055 & \multicolumn{1}{c|}{0.0111} &  \\
% \multicolumn{1}{c|}{} & \multicolumn{1}{l|}{\cellcolor{gray!16}\textbf{SETRec}} & \cellcolor{gray!16}\textbf{0.0116} & \cellcolor{gray!16}\textbf{0.0188} & \cellcolor{gray!16}\textbf{0.0095} & \multicolumn{1}{c|}{\cellcolor{gray!16}\textbf{0.012}} & \cellcolor{gray!16}\textbf{0.0144} & \cellcolor{gray!16}\textbf{0.0236} & \cellcolor{gray!16}\textbf{0.0118} & \multicolumn{1}{c|}{\cellcolor{gray!16}\textbf{0.0151}} & \cellcolor{gray!16}\textbf{0.0531} & \cellcolor{gray!16}\textbf{0.0883} & \cellcolor{gray!16}\textbf{0.0382} & \multicolumn{1}{c|}{\cellcolor{gray!16}\textbf{0.0507}} &  \\ \midrule
% \multicolumn{1}{c|}{\multirow{3}{*}{\textbf{3B}}} & \multicolumn{1}{l|}{\textbf{LETTER}} & 0.0057 & 0.0109 & 0.0053 & \multicolumn{1}{c|}{0.0072} & 0.0078 & 0.0151 & 0.0069 & \multicolumn{1}{c|}{0.0097} & 0.0254 & 0.0471 & 0.0162 & \multicolumn{1}{c|}{0.0236} &  \\
% \multicolumn{1}{c|}{} & \multicolumn{1}{l|}{\textbf{E4SRec}} & 0.0062 & 0.0096 & 0.0048 & \multicolumn{1}{c|}{0.0061} & 0.0082 & 0.0129 & 0.0062 & \multicolumn{1}{c|}{0.0081} & 0.0084 & 0.0218 & 0.0053 & \multicolumn{1}{c|}{0.0103} &  \\
% \multicolumn{1}{c|}{} & \multicolumn{1}{l|}{SETRec} & \textbf{0.0118} & \textbf{0.0195} & \textbf{0.0095} & \multicolumn{1}{c|}{\textbf{0.0123}} & \textbf{0.015} & \textbf{0.0258} & \textbf{0.0119} & \multicolumn{1}{c|}{\textbf{0.0159}} & \textbf{0.065} & \textbf{0.0964} & \textbf{0.0462} & \multicolumn{1}{c|}{\textbf{0.0571}} &  \\ \midrule
% \multicolumn{1}{c|}{\multirow{3}{*}{\textbf{7B}}} & \multicolumn{1}{l|}{\textbf{LETTER}} & 0.0057 & 0.0099 & 0.0044 & \multicolumn{1}{c|}{0.0061} & 0.0078 & 0.0137 & 0.0057 & \multicolumn{1}{c|}{0.0081} & 0.0215 & 0.0406 & 0.0144 & \multicolumn{1}{c|}{0.0216} &  \\
% \multicolumn{1}{c|}{} & \multicolumn{1}{l|}{\textbf{E4SRec}} & 0.0048 & 0.0088 & 0.0041 & \multicolumn{1}{c|}{0.0057} & 0.0062 & 0.0114 & 0.0053 & \multicolumn{1}{c|}{0.0072} & 0.0064 & 0.0133 & 0.0037 & \multicolumn{1}{c|}{0.0065} &  \\
% \multicolumn{1}{c|}{} & \multicolumn{1}{l|}{\cellcolor{gray!16}\textbf{SETRec}} & \cellcolor{gray!16}\textbf{0.0107} & \cellcolor{gray!16}\textbf{0.0194} & \cellcolor{gray!16}\textbf{0.0083} & \multicolumn{1}{c|}{\cellcolor{gray!16}\textbf{0.0115}} & \cellcolor{gray!16}\textbf{0.0127} & \cellcolor{gray!16}\textbf{0.0239} & \cellcolor{gray!16}\textbf{0.01} & \multicolumn{1}{c|}{\cellcolor{gray!16}\textbf{0.014}} & \cellcolor{gray!16}\textbf{0.0632} & \cellcolor{gray!16}\textbf{0.1016} & \cellcolor{gray!16}\textbf{0.0482} & \multicolumn{1}{c|}{\cellcolor{gray!16}\textbf{0.0613}} &  \\ \bottomrule
% \end{tabular}
% }}
% \label{tab:scaling_performance}
% \end{table*}

\begin{table}[t]
\setlength{\abovecaptionskip}{0.05cm}
\setlength{\belowcaptionskip}{0.2cm}
\caption{Performance comparison between SETRec and competitive baselines with different LLM sizes on Qwen. }
\setlength{\tabcolsep}{2.2mm}{
\resizebox{0.46\textwidth}{!}{
\begin{tabular}{clcccccc}
\toprule
% \multicolumn{8}{c}{\textbf{Toys}} \\ \midrule
\multicolumn{1}{l|}{} & \multicolumn{1}{l|}{} & \multicolumn{2}{c}{\textbf{All}} & \multicolumn{2}{c}{\textbf{Warm}} & \multicolumn{2}{c}{\textbf{Cold}} \\
\multicolumn{1}{l|}{} & \multicolumn{1}{l|}{} & \textbf{R@10} & \textbf{N@10} & \textbf{R@10} & \textbf{N@10} & \textbf{R@10} & \textbf{N@10} \\ \midrule\midrule
\multicolumn{1}{c|}{\multirow{3}{*}{\textbf{1.5B}}} & \multicolumn{1}{l|}{\textbf{LETTER}} & 0.0093 & 0.0064 & 0.0126 & 0.0085 & 0.0416 & 0.0239 \\
\multicolumn{1}{c|}{} & \multicolumn{1}{l|}{\textbf{E4SRec}} & 0.0108 & 0.0072 & 0.0144 & 0.0096 & 0.0235 & 0.0111 \\
\multicolumn{1}{c|}{} & \multicolumn{1}{l|}{\cellcolor{gray!16}\textbf{SETRec}} & \cellcolor{gray!16}\textbf{0.0188} & \cellcolor{gray!16}\textbf{0.0120} & \cellcolor{gray!16}\textbf{0.0236} & \cellcolor{gray!16}\textbf{0.0151} & \cellcolor{gray!16}\textbf{0.0883} & \cellcolor{gray!16}\textbf{0.0507} \\ \midrule
\multicolumn{1}{c|}{\multirow{3}{*}{\textbf{3B}}} & \multicolumn{1}{l|}{\textbf{LETTER}} & 0.0109 & 0.0072 & 0.0151 & 0.0097 & 0.0471 & 0.0236 \\
\multicolumn{1}{c|}{} & \multicolumn{1}{l|}{\textbf{E4SRec}} & 0.0096 & 0.0061 & 0.0129 & 0.0081 & 0.0218 & 0.0103 \\
\multicolumn{1}{c|}{} & \multicolumn{1}{l|}{\cellcolor{gray!16}\textbf{SETRec}} & \cellcolor{gray!16}\textbf{0.0195} & \cellcolor{gray!16}\textbf{0.0123} & \cellcolor{gray!16}\textbf{0.0258} & \cellcolor{gray!16}\textbf{0.0159} & \cellcolor{gray!16}\textbf{0.0964} & \cellcolor{gray!16}\textbf{0.0571} \\ \midrule
\multicolumn{1}{c|}{\multirow{3}{*}{\textbf{7B}}} & \multicolumn{1}{l|}{\textbf{LETTER}} & 0.0099 & 0.0061 & 0.0137 & 0.0081 & 0.0406 & 0.0216 \\
\multicolumn{1}{c|}{} & \multicolumn{1}{l|}{\textbf{E4SRec}} & 0.0088 & 0.0057 & 0.0114 & 0.0072 & 0.0133 & 0.0065 \\
\multicolumn{1}{c|}{} & \multicolumn{1}{l|}{\cellcolor{gray!16}\textbf{SETRec}} & \cellcolor{gray!16}\textbf{0.0194} & \cellcolor{gray!16}\textbf{0.0115} & \cellcolor{gray!16}\textbf{0.0239} & \cellcolor{gray!16}\textbf{0.0140} & \cellcolor{gray!16}\textbf{0.1016} & \cellcolor{gray!16}\textbf{0.0613} \\ \bottomrule
\end{tabular}
}}
\label{tab:scaling_performance}
\end{table}

\begin{figure*}[t]
% \vspace{-0.2cm}
\setlength{\abovecaptionskip}{-0.15cm}
\setlength{\belowcaptionskip}{-0cm}
  \centering 
  \hspace{-0.105in}
  \subfigure{
    \includegraphics[height=1.4in]{figures/hyper_alpha_R_10.pdf}} 
  % \hspace{-0.105in}
  \subfigure{
    \includegraphics[height=1.4in]{figures/hyper_alpha_N_10.pdf}} 
  \subfigure{
    \includegraphics[height=1.4in]{figures/hyper_N_R_10.pdf}}
  \subfigure{
    \includegraphics[height=1.4in]{figures/hyper_N_N_10.pdf}}
\caption{Performance of SETRec (T5) with different strength of AE loss $\alpha$ and different numbers of semantic tokens $N$.}
  \label{fig:hp}
  % \vspace{-0.3cm}
\end{figure*}

\subsubsection{\textbf{Effect of Semantic Strength $\bm{\beta}$ (RQ4)}}

\begin{figure}[t]
\vspace{-0.2cm}
\setlength{\abovecaptionskip}{-0.15cm}
\setlength{\belowcaptionskip}{-0.15cm}
  \centering 
  \hspace{-0.105in}
  \subfigure{
  \includegraphics[height=1.4in]{figures/hp_beta_warm.pdf}} 
  \hspace{-0.105in}
  \subfigure{    
  \includegraphics[height=1.4in]{figures/hp_beta_cold.pdf}} 
\caption{Performance of SETRec (T5) with different strength of semantics $\beta$ for inference.}
  \label{fig:hp_beta}
  % \vspace{-0.3cm}
\end{figure}

% 1. 如果只用cf的话,performance不行(significant inferior performance of beta=0 than beta>0)。这说明当同时利用cf和sem来encode user embedding的时候,decode也需要semantic的帮助。并且在cold上的提升要明显比warm上的提升大,这也说明了对cold start item推荐时semantic引入的必要性。
% 2. 持续增大到只用semantics时在warm和cold上仍然能取得不错的performance。说明了item rich semantic信息对于warm item的推荐也是有帮助的。这可能是因为在训练的时候semantic和cf之间achieve implicit alignment,无脑全入semantic也不会让performance掉太多。蕾丝的现象也在ablation里有观察到。
To investigate how semantic information contributes to the performance during inference, we vary $\beta$ from $0$ to $1$, where $\beta=0$ indicates that only CF score is used for ranking, and $\beta=1$ ranks items based solely on semantic scores (Eq. (\ref{eqn:single_logits})).  
From the results reported in Figure~\ref{fig:hp_beta}. 
we can find that 
1) Incorporating semantic information during inference is necessary (inferior performance of $\beta=0$ than $\beta>0$, which facilitates
global ranking over multi-dimensional information and lead to strong generalization ability. 
Notably, 
2) incorporating semantic scores brings more significant improvements on cold items, underscoring the critical role of semantic information for zero-shot scenarios.
Moreover, 
3) Gradually increase $\beta$ to rely solely on semantics ($\beta=1$), SETRec maintains competitive performance on warm items, which is probably attributed to the implicit alignment between CF and semantic tokens during training. 



\subsubsection{\textbf{Hyper-parameter Sensitivity (RQ4)}}\label{sec:exp_hyper_param}
We further study the hyper-parameter sensitivity to facilitate SETRec application.

\noindent$\bullet\quad$\textbf{Effect of $\bm{\alpha}$.} 
We vary the strength of AE loss $\alpha$ for SETRec training and present the results on Toys in Figure~\ref{fig:hp}(a-b). 
We can observe that 
% 1. alpha 从0慢慢增大,performance提升。这合理因为tokenizer肯定是需要随着llm一起优化的。
1) the performance is overall improved when $\alpha$ is increased from $0$ to $0.7$, which validates the effectiveness of reconstruction loss that encourages AE to preserve useful information in the latent space. 
% 2. 但是tokenizer的权重不能太高,和llm的loss一起做multi-task training的话,可能会导致模型偏向tokenizer更多,而这可能反过来影响llm,使他推荐能力学的差。(这个理由再想想,现在不太行)
Nonetheless, 
2) while continuously increasing $\alpha$ generally gives better performance on cold-start items, it might hurt the performance under ``warm'' setting. 
Based on the empirical results, we recommend setting $\alpha$ ranging from $0.5$ to $0.7$. 






\noindent$\bullet\quad$\textbf{Effect of $\bm{N}$.} 
We change the number of semantic tokens from $1$ to $6$ to investigate how $N$ affects the performance. 
From the results shown in Figure~\ref{fig:hp}(c-d), we can find that 
% 1. N提高,有提升。说明多侧面语义信息是有用的。
1) gradually increasing semantic tokens generally improves the performance, which validates the effectiveness of incorporating multiple tokens to mitigate the potential information conflicts~\cite{wang2024learnable} and embedding collapse issue~\cite{guoembedding}. 
% 2. 但是继续提升,反而会有所下降。这主要是因为XXX?
However, 
2) blindly increasing the number of semantic tokens might hurt the performance (decreased performance from $N=4$ to $N=6$). 
This is reasonable since it is non-trivial to recover the category-level preference aligning well with the real-world scenarios. 
Similar observations are also seen in~\cite{lin2024disentangled} and~\cite{lin2024temporally}. 





% \begin{figure}[t]
% % \vspace{-0.2cm}
% \setlength{\abovecaptionskip}{-0.15cm}
% \setlength{\belowcaptionskip}{-0cm}
%   \centering 
%   \hspace{-0.105in}
%   \subfigure{
%     \includegraphics[height=1.4in]{figures/hyper_N_R@10.pdf}} 
%   % \hspace{-0.105in}
%   \subfigure{
%     \includegraphics[height=1.4in]{figures/hyper_N_N@10.pdf}} 
% \caption{Performance of SETRec (T5) with different number of semantic embeddings on Toys.}
%   \label{fig:hp_N}
%   % \vspace{-0.3cm}
% \end{figure}


\section{Conclusion}
The state-of-the-art $1/e$ approximation ratio for sublinear adaptive algorithms 
is achieved by a continuous algorithm~\citep{Ene2020a}.
For combinatorial algorithms with sublinear adaptivity,
the best-known result is a randomized $1/4$ approximation ratio~\citep{Cui2023}.
In this work, we present a sublinear adaptive approximation algorithm 
achieving $1/4-\epsi$ approximation ratio with high probability,
and further improve this ratio achieved to $1/e$, 
breaking the barrier between continuous and combinatorial algorithms. 
These advancements are made by a novel blending analysis technique,
which offers a fresh perspective for analyzing greedy-based algorithms.

\bibliography{main}
\bibliographystyle{icml2025}

\newpage
\appendix
\onecolumn
\section{Technical Lemmata}\label{apx:tech}
% \begin{lemma}[\citep{feige2011maximizing}]\label{lemma:OneRandomSet}
% Let $f:2^{\mathcal{N}} \to \reals$ be submodular. 
% Denote by $A(p)$ a random subset of $A$ where each element
% appears with probability $p$ (not necessarily independently).
% Then
% \[\ex{f(A(p))} \ge (1-p)\cdot f(\emptyset) + p\cdot f(A).\]
% \end{lemma}

\begin{lemma}\label{lemma:val-inq}
    \begin{align*}
        & 1-\frac{1}{x}\le \log(x) \le x-1, &\forall x>0\\
        & 1-\frac{1}{x+1}\ge e^{-\frac{1}{x}} , &\forall x\in \mathbb{R}\\
        & (1-x)^{y-1}\ge e^{-xy}, &\forall xy \le 1
    \end{align*}
\end{lemma}

\begin{lemma}[Chernoff bounds \citep{mitzenmacher2017probability}]\label{lemma:chernoff}
    Suppose $X_1$, ... , $X_n$ are independent binary random variables such that 
    $\prob{X_i = 1} = p_i$. Let $\mu = \sum_{i=1}^n p_i$, and 
    $X = \sum_{i=1}^n X_i$. Then for any $\delta \geq 0$, we have
    \begin{align}
        \prob{X \ge (1+\delta)\mu} \le e^{-\frac{\delta^2 \mu}{2+\delta}}.
    \end{align}
    Moreover, for any $0 \leq \delta \leq 1$, we have
    \begin{align}
        \prob{X \le (1-\delta)\mu} \le e^{-\frac{\delta^2 \mu}{2}}.
    \end{align}
\end{lemma}
\begin{lemma}[\citet{Chen2021}] \label{lemma:indep}
    Suppose there is a sequence of $n$ Bernoulli trials:
    $X_1, X_2, \ldots, X_n,$
    where the success probability of $X_i$
    depends on the results of
    the preceding trials $X_1, \ldots, X_{i-1}$.
    Suppose it holds that $$\prob{X_i = 1 | X_1 = x_1, X_2 = x_2, \ldots, X_{i-1} = x_{i-1} } \ge \eta,$$ where $\eta > 0$ is a constant and $x_1,\ldots,x_{i-1}$ are arbitrary.
  
    Then, if $Y_1,\ldots, Y_n$ are independent Bernoulli trials, each with probability $\eta$ of
    success, then $$\prob {\sum_{i = 1}^n X_i \le b } \le \prob{\sum_{i = 1}^n Y_i \le b }, $$
    where $b$ is an arbitrary integer.
  
    Moreover, let $A$ be the first occurrence of success in sequence $X_i$.
    Then, $$\ex{A} \le 1/\eta.$$
\end{lemma}

\section{Propositions on Submodularity} \label{apx:prop}

\begin{proposition}\label{prop:sum-marge}
Let $\{A_1, A_2, \ldots, A_m\}$ be $m$ pairwise disjoint subsets of $\uni$,
and $B\in \uni$.
For any submodular function $f: 2^\uni \to \reals$,
it holds that
\begin{align*}
\text{1) }&\sum\limits_{i\in [m]} \marge{A_i}{B}\ge \marge{\bigcup\limits_{i\in [m]}A_i}{B},\\
\text{2) }&\sum_{i\in [m]}\ff{B\cup A_i}\ge (m-1)\ff{B}.
\end{align*}
\end{proposition}

\begin{proposition}\label{prop:subset}
Let $A=\{a_1, \ldots, a_m\}$ and $A_i = \{a_1, \ldots, a_i\}$ for all $i\in [m]$.
For any submodular function $f: 2^\uni \to \reals$,
let $B = \argmax\limits_{B\subseteq A, |B| = m-1}\sum\limits_{a_i \in B}\marge{a_i}{A_{i-1}}$.
It holds that
$\ff{B} \ge \left(1-\frac{1}{m}\right)\ff{A}$.
\end{proposition}

\begin{proposition}\label{prop:dif-opt}
For any submodular function $f: 2^\uni \to \reals$,
let $O_1 = \argmax_{S\subseteq \uni, |S|\le k_1} \ff{S}$ and 
$O_2 = \argmax_{S\subseteq \uni, |S|\le k_2} \ff{S}$.
It holds that 
\[\ff{O_1} \ge \frac{k_1}{k_2}\ff{O_2}.\]
\end{proposition}
\section{Pseudocode and Theoretical Guarantees of \ig~\citep{DBLP:conf/nips/Kuhnle19} and \itg~\citep{DBLP:conf/kdd/ChenK23}}\label{apx:pseudocode}
In this section, we provide the original greedy version of 
\ig~\citep{DBLP:conf/nips/Kuhnle19} and \itg~\citep{DBLP:conf/kdd/ChenK23}
with their theoretical guarantees.
\begin{algorithm}[ht]
    \KwIn{evaluation oracle $f:2^{\uni} \to \reals$, constraint $k$}
    \KwOut{$C\subseteq \uni$, such that $|C| \le k$}
    $A_0\gets B_0 \gets \emptyset$\;
    \For{$i\gets 0$ to $k-1$}{
        $a_i\gets \argmax_{x\in \uni\setminus \left(A_i\cup B_i\right)} \marge{x}{A_i}$\;
        $A_{i+1}\gets A_i+a_i$\;
        $b_i\gets \argmax_{x\in \uni\setminus \left(A_{i+1}\cup B_i\right)} \marge{x}{B_i}$\; 
        $B_{i+1}\gets B_i+b_i$\;
    }
    $D_1 \gets E_1 \gets \{a_0\}$\;
    \For{$i\gets 1$ to $k-1$}{
        $d_i\gets \argmax_{x\in \uni\setminus \left(D_i\cup E_i\right)} \marge{x}{D_i}$\;
        $D_{i+1}\gets D_i+d_i$\;
        $e_i\gets \argmax_{x\in \uni\setminus \left(D_{i+1}\cup E_i\right)} \marge{x}{E_i}$\; 
        $E_{i+1}\gets E_i+e_i$\;
    }
    \Return{$C\gets \argmax\{\ff{A_i}, \ff{B_i}, \ff{D_i}, \ff{E_i} : i\in [k+1]\}$}
    \caption{$\ig(f,k)$: The \ig Algorithm~\citep{DBLP:conf/nips/Kuhnle19}}
    \label{alg:ig}
\end{algorithm}
\begin{theorem}
Let $f:2^{\uni} \to \reals$ be submodular, let $k\in \uni$,
let $O = \argmax_{|S|\le k} \ff{S}$,
and let $C = \ig(f, k)$. Then
\[\ff{C} \ge \ff{O}/4,\]
and \ig makes $\oh{kn}$ queries to $f$.
\end{theorem}

\begin{algorithm}[ht]
	\KwIn{oracle $f:2^{\uni} \to \reals$, constraint $k$, error $\epsi$}
    \Init{$\ell \gets \frac{2e}{\epsi}+1$, $G_0\gets \emptyset$}
    \For{$m\gets 1$ to $\ell$}{
    	$\{a_1, \ldots, a_\ell\}\gets$ top $\ell$ elements in $\uni\setminus G_{m-1}$
		with respect to marginal gains on $G_{m-1}$\;
		\For{$u\gets 0$ to $\ell$ in parallel}{
			\lIf{$u = 0$}{$A_{u, l}\gets G\cup \{a_l\}$, for all $1\le l\le \ell$}
			\lElse{$A_{u, l}\gets G\cup \{a_u\}$, for all $1\le l\le \ell$}
			\For{$j\gets 1$ to $k/\ell-1$}{
				\For{$i\gets 1$ to $\ell$}{
					$x_{j, i} \gets \argmax_{x\in \uni\setminus\left(\bigcup_{l=1}^{\ell}A_{u, l}\right)}\marge{x}{A_{u, i}}$\;
					$A_{u, i}\gets A_{u, i}\cup \{x_{j, i}\}$\;
				}
			}
		}
		$G_m\gets$ a random set in $\{A_{u, i}:1\le i\le \ell, 0\le u\le \ell\}$
    }
    \Return{$G_\ell$}\;
    \caption{$\itg(f,k,\epsi)$: An $1/(e+\epsi)$-approximation algorithm for \sm}
    \label{alg:itg}
\end{algorithm}
\begin{theorem}
Let $\epsi \ge 0$, and $(f, k)$ be an instance of \sm, 
with optimal solution value \opt.
Algorithm \itg outputs a set $G_\ell$ with $\oh{\epsi^{-2}kn}$ queries
such that $\opt \le (e+\epsi)\ex{\ff{G_\ell}}$ with 
probability $(\ell+1)^{-\ell}$, where $\ell = \frac{2e}{\epsi}+1$.
\end{theorem}


\section{Analysis of Alg.~\ref{alg:gdone} in Section~\ref{sec:greedy-1/4}}
\label{apx:greedy-1/4}
In this section, we provide the detailed analysis of approximation ratio
for Alg.~\ref{alg:gdone}.
\thmgdone*
\begin{proof}[Proof of Theorem~\ref{thm:gdone}]
% Consider adding dummy elements to the ground set.
% If $|O|$ is less than $k$, add dummy elements to $O$ until $|O| = k$.
% During each iteration of the for-loop,
% if no element in $\uni\setminus (A\cup B)$ has marginal gain greater than 0,
% a dummy element will be added to $A$,
% and similarly for $B$.
% Thus, after the for loop ends, we ensure that $|A| = |B| = k$.
% Moreover, let $A_0 = B_0 = \emptyset$,
% and $A_i$, $B_i$ be $A$, $B$ after $i$-th element is added, respectively.

\textbf{Notation.}
Let $a_i$ be the $i$-th element added to $A$,
and $A_i$ be the set containing the first $i$ elements of $A$.
Similarly, define $b_i$ and $B_i$ for the solution $B$.

Since the two solutions $A_k$ and $B_k$ are disjoint, by submodularity and non-negativity,
\begin{equation*}
\ff{O} \le \ff{O\cup A_k} + \ff{O\cup B_k}.
\end{equation*}
Let $i^* = \max\{i \in [k]: A_i\subseteq O\}$
and $j^* = \max\{j \in [k]: B_j\subseteq O\}$.
If either $i^* = k$ or $j^* = k$,
then $\ff{S} = \ff{O}$.
In the following, we consider $i^* < k$ and $j^* < k$
and discuss two cases of the relationship between $i^*$ and $j^*$ (Fig.~\ref{fig:gdone}).



\textbf{Case 1: $0\le i^*\le j^* < k$; Fig.~\ref{fig:gdone-1}.}
First, we bound $\ff{O\cup A_k}$. 
Consider the set $\tilde{O} = O\setminus \left(A_k \cup B_{i^*}\right)$.
Obviously, it holds that $|\tilde{O}|\le k-i^*$.
Then, order $\tilde{O}$ as $\{o_1, o_2, \ldots\}$ such that $o_i \not \in B_{i+i^*-1}$,
for all $1\le i \le |\tilde{O}|$.
Thus, by the greedy selection step in Line~\ref{line:gdone-greedy-A},
it holds that $\marge{a_{i+i^*}}{A_{i+i^*-1}}\ge \marge{o_i}{A_{i+i^*-1}}$
for all $1\le i \le |\tilde{O}|$.
Then,
\begin{align*}
\ff{O\cup A_k} - \ff{A_k} &\le \marge{B_{i^*}}{A_k} + \marge{\tilde{O}}{A_k}\\
&\le \ff{B_{i^*}} + \sum\limits_{i=1}^{|\tilde{O}|} \marge{o_i}{A_k}\\
&\le \ff{B_{i^*}} + \sum\limits_{i=1}^{|\tilde{O}|} \marge{o_i}{A_{i+i^*-1}}\\
&\le \ff{B_{i^*}} + \sum\limits_{i=i^*+1}^{k} \marge{a_i}{A_{i-1}}
= \ff{B_{i^*}} + \ff{A_k} - \ff{A_{i^*}},
\end{align*}
where the first three inequalities follow from submodularity;
and the last inequality follows from 
$\marge{a_{i+i^*}}{A_{i+i^*-1}}\ge \marge{o_i}{A_{i+i^*-1}}$
for all $1\le i \le |\tilde{O}|$,
and $\marge{a_i}{A_{i-1}}\ge 0$ for all $i \in [k]$.

Next, we bound $\ff{O\cup B_k}$.
Consider the set $\tilde{O} = O\setminus \left(A_{i^*} \cup B_{k}\right)$.
Obviously, it holds that $|\tilde{O}| \le k-i^*$.
Since $i^* = \max\{i \in [k]: A_i\subseteq O\}$,
we know that $a_{i^*+1} \not \in O$.
Thus, we can order $|\tilde{O}|$ as $\{o_1, o_2, \ldots\}$
such that $o_i \not \in A_{i+i^*}$ for all $1\le i \le |\tilde{O}|$.
Then, by the greedy selection step in Line~\ref{line:gdone-greedy-B},
it holds that $\marge{b_{i+i^*}}{B_{i+i^*-1}}\ge \marge{o_i}{B_{i+i^*-1}}$
for all $1\le i \le |\tilde{O}|$.
Following the analysis for $\ff{O\cup A}$,
we get
\begin{align*}
\ff{O\cup B_k} - \ff{B_k} &\le \marge{A_{i^*}}{B_k} + \marge{\tilde{O}}{B_k}\\
&\le \ff{A_{i^*}} + \sum\limits_{i=1}^{|\tilde{O}|} \marge{o_i}{B_k}\\
&\le \ff{A_{i^*}} + \sum\limits_{i=1}^{|\tilde{O}|} \marge{o_i}{B_{i+i^*-1}}\\
&\le \ff{A_{i^*}} + \sum\limits_{i=i^*+1}^{k} \marge{b_i}{B_{i-1}}
= \ff{A_{i^*}} + \ff{B_k} - \ff{B_{i^*}}.
\end{align*}

\textbf{Case 2: $0\le j^* < i^* < k$; Fig.~\ref{fig:gdone-2}.}
First, we bound $\ff{O\cup A_k}$.
Consider the set $\tilde{O} = O\setminus \left(A_{k}\cup B_{j^*}\right)$,
where $|\tilde{O}| \le k-j^*-1$.
By the definition of $j^*$,
we know that $b_{j^*+1}\not\in O$.
Thus, we can order $\tilde{O}$ as $\{o_1, o_2, \ldots\}$
such that $o_i\not \in B_{i+j^*}$ for all $1\le i\le |\tilde{O}|$.
Then, by the greedy selection step in Line~\ref{line:gdone-greedy-A},
it holds that $\marge{a_{i+j^*+1}}{A_{i+j^*}}\ge \marge{o_i}{A_{i+j^*}}$
for all $1\le i\le |\tilde{O}|$.
Following the above analysis, we get
\begin{align*}
\ff{O\cup A_k}-\ff{A_k} &\le \marge{B_{j^*}}{A_k}+\marge{\tilde{O}}{A_k}\\
&\le \ff{B_{j^*}} + \sum\limits_{i=1}^{|\tilde{O}|} \marge{o_i}{A_k}\\
&\le \ff{B_{j^*}} + \sum\limits_{i=1}^{|\tilde{O}|} \marge{o_i}{A_{i+j^*}}\\
&\le \ff{B_{j^*}} + \sum\limits_{i=j^*+2}^{k} \marge{a_i}{A_{i-1}}
= \ff{B_{j^*}} + \ff{A_k}-\ff{A_{j^*+1}}.
\end{align*}

Next, we bound $\ff{O\cup B_k}$.
Consider the set $\tilde{O} = O\setminus \left(A_{j^*+1} \cup B_k\right)$,
where $|\tilde{O}| \le k-j^*-1$.
Then, order $\tilde{O}$ as $\{o_1, o_2, \ldots\}$
such that $o_i\not \in A_{i+j^*}$ for all $1\le i\le |\tilde{O}|$.
By the greedy selection step in Line~\ref{line:gdone-greedy-B},
it holds that $\marge{b_{i+j^*}}{B_{i+j^*-1}}\ge \marge{o_i}{B_{i+j^*-1}}$.
Then,
\begin{align*}
\ff{O\cup B_k} - \ff{B_k}&\le \marge{A_{j^*+1}}{B_k} + \marge{\tilde{O}}{B_k}\\
&\le \ff{A_{j^*+1}}+\sum\limits_{i=1}^{|\tilde{O}|} \marge{o_i}{B_k}\\
&\le \ff{A_{j^*+1}}+\sum\limits_{i=1}^{|\tilde{O}|} \marge{o_i}{B_{i+j^*-1}}\\
&\le \ff{A_{j^*+1}}+\sum\limits_{i=j^*+1}^{k} \marge{b_i}{B_{i-1}} 
= \ff{A_{j^*+1}} + \ff{B_k}-\ff{B_{j^*}}.
\end{align*}

Therefore, in both cases, it holds that
\[\ff{O}\le \ff{O\cup A_k}+\ff{O\cup B_k}\le 2\left(\ff{A_k}+\ff{B_k}\right)\le 4\ff{S}.\]

\end{proof}

\section{Analysis of Alg.~\ref{alg:gdtwo} in Section~\ref{sec:greedy-1/e}}
\label{apx:greedy-1/e}
In what follows, we address the scenario where $k\, \text{mod}\,\ell > 0$
and Alg.~\ref{alg:gdtwo} returns a solution with size smaller than $k$
in Appendix~\ref{apx:gdtwo-k}.
We then provide proofs for the relevant Lemmata in Appendix~\ref{apx:gdtwo-lemma},
and conclude with an analysis of approximation ratio in Appendix~\ref{apx:gdtwo-approx}.
\subsection{Scenario where $k\, \text{mod}\,\ell > 0$.}\label{apx:gdtwo-k}
If $k\, \text{mod}\,\ell = 0$, 
the algorithm returns an approximation solution for a size constraint of 
$\ell\cdot\left\lfloor\frac{k}{\ell}\right\rfloor$.
By Proposition~\ref{prop:dif-opt}, it holds that 
\begin{equation}\label{inq:dif-opt}
\ff{O'}\ge \ell\cdot\left\lfloor\frac{k}{\ell}\right\rfloor / k \ff{O}
\ge \left(1-\frac{\ell}{k}\right)\ff{O},
O' = \argmax\limits_{S\subseteq \uni, |S|\le \ell\cdot\left\lfloor\frac{k}{\ell}\right\rfloor} \ff{S}.
\end{equation}

\subsection{Proofs of Lemmata for Theorem~\ref{thm:gdtwo}} \label{apx:gdtwo-lemma}
\textbf{Notation.} 
Let $G_{i-1}$ be $G$ at the start of $i$-th iteration in Alg.~\ref{alg:gdtwo},
$A_l$ be the set at the end of this iteration,
and $a_{l, j}$ be the $j$-th element added to $A_l$ during this iteration.

\lemmaparA*
\begin{proof}[Proof of Lemma~\ref{lemma:par-A}]
\begin{figure}
\centering
\includegraphics[width=0.45\linewidth]{fig/ITG.pdf}
\caption{This figure depicts the components of the solution sets $A_{l_1}$ and $A_{l_2}$.
A blue circle with a check mark represents an element in $O$,
while a red circle with a cross mark represents an element outside of $O$.
The grey rectangles indicate a sequence of consecutive elements in $O$.
The pink rectangles indicate the corresponding elements used to bound $\marge{O_{l_2}}{A_{l_1}}$ or $\marge{O_{l_1}}{A_{l_2}}$.
It is illustrated that $\marge{O_{l_1}}{A_{l_2}} + \marge{O_{l_2}}{A_{l_1}}\le \marge{A_{l_1}}{G_{i-1}} +  \marge{A_{l_2}}{G_{i-1}}$ under both cases.}
\label{fig:gdtwo}
\end{figure}
Recall that $A_{l, j}$ is $A_l$ after $j$-th element is added to $A_l$ at iteration $i$ of the outer for loop,
and $c_l^* = \max\left\{c\in [m]: A_{l, c}\setminus G_{i-1}\subseteq O_l \right\}$.

First, we prove that the first inequality holds.
For each $l\in [\ell]$, order the elements in $O_l$ as $\{o_1, o_2, \ldots\}$
such that $o_j \not \in A_{l, j-1}$ for any $1\le j \le |O_l|$.
Since each $o_j$ is either in $A_l$ or not in any solution set,
it remains in the candidate pool when $a_{l, j}$ is considered to be added to the solution.
Therefore, it holds that
\begin{equation}\label{inq:itg-1}
\marge{a_{l, j}}{A_{l, j-1}}\ge \marge{o_j}{A_{l, j-1}}.
\end{equation}
Then,
\begin{align*}
\marge{O_{l}}{A_{l}} \le \sum_{o_j\in O_l} \marge{o_j}{A_{l}} \le \sum_{o_j\in O_l} \marge{o_j}{A_{l, j-1}}\le \sum_{j=1}^{m}\marge{a_{l, j}}{A_{l, j-1}} = \marge{A_{l}}{G_{j-1}},
\end{align*}
where the first inequality follows from Proposition~\ref{prop:sum-marge},
the second inequality follows from submodularity,
and the last inequality follows from Inequality~\eqref{inq:itg-1}.

In the following, we prove that the second inequality holds.
For any $1\le l_1\le l_2\le \ell$,
we analyze two cases of the relationship between $c_{l_1}^* $ and $ c_{l_2}^*$ in the following.

% \textbf{Case 1: $c_{l_1}^* = c_{l_2}^* = m$.}
% Then, $O_{l_1} = A_{l_1}\setminus G_{i-1}$ and $O_{l_2} = A_{l_2}\setminus G_{i-1}$.
% By submodularity,
% \[\marge{O_{l_2}}{A_{l_1}} + \marge{O_{l_1}}{A_{l_2}} \le \marge{O_{l_2}}{G_{i-1}} + \marge{O_{l_1}}{G_{i-1}} = \marge{A_{l_1}}{G_{i-1}} + \marge{A_{l_2}}{G_{i-1}}.\]
% Therefore, the lemma holds in this case.

\textbf{Case 1: $c_{l_1}^* \le c_{l_2}^*$; left half part in Fig.~\ref{fig:gdtwo}.}

First, we bound $\marge{O_{l_1}}{A_{l_2}}$.
Since $c_{l_1}^* \le m$, we know that the $(c_{l_1}^*+1)$-th element in $A_{l_1}\setminus G_{i-1}$ is not in $O$.
So, we can order the elements in $O_{l_1}\setminus A_{l_1, c_{l_1}^*}$ as $\{o_1, o_2, \ldots\}$ such that $o_j \not \in A_{l_1, c_{l_1}^*+j+1}$.
(Refer to the gray block with a dotted edge in the top left corner of Fig.~\ref{fig:gdtwo} for $O_{l_1}$.)
Since each $o_j$ is either added to $A_{l_1}$ or not in any solution set,
it remains in the candidate pool when $a_{l_2, c_{l_1}^*+j}$ is considered to be added to $A_{l_2}$.
Therefore, it holds that 
\begin{equation}\label{inq:itg-case2-1}
\marge{a_{l_2, c_{l_1}^*+j}}{A_{l_2, c_{l_1}^*+j-1}} \ge \marge{o_j}{A_{l_2, c_{l_1}^*+j-1}}, \forall 1\le j\le m-c_{l_1}^*.
\end{equation}
Then,
\begin{align*}
\marge{O_{l_1}}{A_{l_2}} &\le \marge{A_{l_1, c_{l_1}^*}}{A_{l_2}}  + \sum_{o_j \in O_{l_1}\setminus A_{l_1, c_{l_1}^*}}\marge{o_j}{A_{l_2}} \tag{Proposition~\ref{prop:sum-marge}}\\
&\le \marge{A_{l_1, c_{l_1}^*}}{G_{i-1}} + \sum_{o_j \in O_{l_1}\setminus A_{l_1, c_{l_1}^*}}\marge{o_j}{A_{l_2, , c_{l_1}^*+j-1}} \tag{submodularity}\\
&\le \ff{A_{l_1, c_{l_1}^*}}-\ff{G_{i-1}} + \sum_{j = 1}^{m-c_{l_1}^*}\marge{a_{l_2, c_{l_1}^*+j}}{A_{l_2, , c_{l_1}^*+j-1}} \tag{Inequality~\eqref{inq:itg-case2-1}}\\
& \le \ff{A_{l_1, c_{l_1}^*}}-\ff{G_{i-1}} + \ff{A_{l_2}} - \ff{A_{l_2, c_{l_1}^*}} 
\end{align*}

Similarly, we bound $\marge{O_{l_2}}{A_{l_1}}$ below.
Order the elements in $O_{l_2}\setminus A_{l_2, c_{l_1}^*}$ as $\{o_1, o_2, \ldots\}$
such that $o_j \not \in A_{l_2, c_{l_1}^* + j}$.
(See the gray block with a dotted edge in the bottom left corner of Fig.~\ref{fig:gdtwo} for $O_{l_2}$.)
Since each $o_j$ is either added to $A_{l_2}$ or not in any solution set,
it remains in the candidate pool when $a_{l_1, c_{l_1}^*+j}$ is considered to be added to $A_{l_2}$.
Therefore, it holds that
\begin{equation}\label{inq:itg-case2-2}
\marge{a_{l_1, c_{l_1}^*+j}}{A_{l_1, c_{l_1}^*+j-1}} \ge \marge{o_j}{A_{l_1, c_{l_1}^*+j-1}}, \forall 1\le j\le m-c_{l_1}^*.
\end{equation}
Then,
\begin{align*}
\marge{O_{l_2}}{A_{l_1}} &\le \marge{A_{l_2, c_{l_1}^*}}{A_{l_1}}  + \sum_{o_j \in O_{l_2}\setminus A_{l_2, c_{l_1}^*}}\marge{o_j}{A_{l_1}} \tag{Proposition~\ref{prop:sum-marge}}\\
&\le \marge{A_{l_2, c_{l_1}^*}}{G_{i-1}} + \sum_{o_j \in O_{l_2}\setminus A_{l_2, c_{l_1}^*}}\marge{o_j}{A_{l_1, , c_{l_1}^*+j-1}} \tag{submodularity}\\
&\le \ff{A_{l_2, c_{l_1}^*}}-\ff{G_{i-1}} + \sum_{j = 1}^{m-c_{l_1}^*}\marge{a_{l_1, c_{l_1}^*+j}}{A_{l_1, , c_{l_1}^*+j-1}} \tag{Inequality~\eqref{inq:itg-case2-1}}\\
& \le \ff{A_{l_1, c_{l_1}^*}}-\ff{G_{i-1}} + \ff{A_{l_1}} - \ff{A_{l_1, c_{l_1}^*}} 
\end{align*}
Thus, the lemma holds in this case.

\textbf{Case 2: $c_{l_1}^* > c_{l_2}^*$; right half part in Fig.~\ref{fig:gdtwo}.}
First, we bound $\marge{O_{l_1}}{A_{l_2}}$.
Order the elements in $O_{l_1}\setminus A_{l_1, c_{l_2}^*+1}$ as $\{o_1, o_2, \ldots\}$ such that $o_j \not \in A_{l_1, c_{l_2}^*+j}$.
(Refer to the gray block with a dotted edge in the top right corner of Fig.~\ref{fig:gdtwo} for $O_{l_1}$.)
Since each $o_j$ is either in $A_{l_1}$ or not in any solution set,
it remains in the candidate pool when $a_{l_2, c_{l_2}^*+j}$ is considered to be added to $A_{l_2}$.
Therefore, it holds that 
\begin{equation}\label{inq:itg-case3-1}
\marge{a_{l_2, c_{l_2}^*+j}}{A_{l_2, c_{l_2}^*+j-1}} \ge \marge{o_j}{A_{l_2, c_{l_2}^*+j-1}}, \forall 1\le j\le m-c_{l_2}^*-1.
\end{equation}
Then,
\begin{align*}
\marge{O_{l_1}}{A_{l_2}} &\le \marge{A_{l_1, c_{l_2}^*+1}}{A_{l_2}}  + \sum_{o_j \in O_{l_1}\setminus A_{l_1, c_{l_2}^*+1}}\marge{o_j}{A_{l_2}} \tag{Proposition~\ref{prop:sum-marge}}\\
&\le \marge{A_{l_1, c_{l_2}^*+1}}{G_{i-1}} + \sum_{o_j \in O_{l_1}\setminus A_{l_1, c_{l_2}^*+1}}\marge{o_j}{A_{l_2, , c_{l_2}^*+j-1}} \tag{submodularity}\\
&\le \ff{A_{l_1, c_{l_2}^*+1}}-\ff{G_{i-1}} + \sum_{j = 1}^{m-c_{l_2}^*-1}\marge{a_{l_2, c_{l_2}^*+j}}{A_{l_2, , c_{l_2}^*+j-1}} \tag{Inequality~\eqref{inq:itg-case3-1}}\\
& \le \ff{A_{l_1, c_{l_2}^*+1}}-\ff{G_{i-1}} + \ff{A_{l_2}} - \ff{A_{l_2, c_{l_2}^*}} 
\end{align*}

Similarly, we bound $\marge{O_{l_2}}{A_{l_1}}$ below.
Since $c_{l_2}^* < c_{l_2}^*$,
we know that the $(c_{l_2}^*+1)$-th element in $A_{l_2}\setminus G_{i-1}$
is not in $O$, which implies that $|O_{l_2}| \le m$.
So, we can order the elements in $O_{l_2}\setminus A_{l_2, c_{l_2}^*}$ as $\{o_1, o_2, \ldots\}$
such that $o_j \not \in A_{l_2, c_{l_2}^* + j}$ for each $1\le j\le m-c_{l_2}^*$.
(See the gray block with a dotted edge in the bottom right corner of Fig.~\ref{fig:gdtwo} for $O_{l_2}$.)

When $1\le j< m-c_{l_2}^*$,
since each $o_j$ is either in $A_{l_2}$ or not in any solution set,
it remains in the candidate pool when $a_{l_1, c_{l_2}^*+i+1}$ is considered to be added to $A_{l_1}$.
Therefore, it holds that
\begin{equation}
\marge{a_{l_1, c_{l_2}^*+j+1}}{A_{l_1, c_{l_2}^*+j}} \ge \marge{o_j}{A_{l_1, c_{l_2}^*+j}}, \forall 1\le j< m-c_{l_2}^*.
\end{equation}
As for the last element $o_{m-c_{l_2}^*}$ in $O_{l_2}\setminus A_{l_2, c_{l_2}^*}$,
we know that $o_{m-c_{l_2}^*}$ is not added to any solution set.
So,
\begin{equation}
\marge{o_{m-c_{l_2}^*}}{A_{l_1}} \le \frac{1}{m}\sum_{j = 1}^m \marge{a_{l_1, j}}{A_{l_1,j-1}}
 = \frac{1}{m} \marge{A_{l_1}}{G_{i-1}}
\end{equation}

Then,
\begin{align*}
\marge{O_{l_2}}{A_{l_1}} &\le \marge{A_{l_2, c_{l_2}^*}}{A_{l_1}}  + \sum_{o_j \in O_{l_2}\setminus A_{l_2, c_{l_2}^*}}\marge{o_j}{A_{l_1}} \tag{Proposition~\ref{prop:sum-marge}}\\
&\le \marge{A_{l_2, c_{l_2}^*}}{G_{i-1}} + \sum_{o_j \in O_{l_2}\setminus A_{l_2, c_{l_2}^*}}\marge{o_j}{A_{l_1, , c_{l_2}^*+j}} \tag{submodularity}\\
&\le \ff{A_{l_2, c_{l_2}^*}}-\ff{G_{i-1}} + \sum_{j = 1}^{m-c_{l_2}^*-1}\marge{a_{l_1, c_{l_2}^*+j+1}}{A_{l_1, , c_{l_2}^*+j}} + \frac{1}{m} \marge{A_{l_1}}{G_{i-1}} \tag{Inequality~\eqref{inq:itg-case3-1}}\\
& \le \ff{A_{l_1, c_{l_2}^*}}-\ff{G_{i-1}} + \ff{A_{l_1}} - \ff{A_{l_1, c_{l_2}^*+1}}+ \frac{1}{m} \marge{A_{l_1}}{G_{i-1}} \tag{$|O_{l_2}|\le m$}
\end{align*}
Thus, the lemma holds in this case.
\end{proof}

\begin{restatable}{lemma}{lemmagdtworec}\label{lemma:gdtwo-rec}
For any iteration $i$ of the outer for loop in Alg.~\ref{alg:gdtwo},
it holds that 

\vspace*{-1em}
{\small\begin{align*}
&\ex{\ff{G_i} - \ff{G_{i-1}}}\ge \frac{1}{\ell+1}\left(1-\frac{1}{m+1}\right) \\
&\cdot \left(\left(1-\frac{1}{\ell}\right)\ex{\ff{O\cup G_{i-1}}} - \ex{\ff{G_{i-1}}}\right)
% \left(1-\frac{1}{m+1}\right)\left((\ell-1)\ff{O\cup G_{i-1}}-\ell\ff{G_{i-1}}\right) \le (\ell+1)\sum_{l\in [\ell]} \marge{A_l}{G_{i-1}}.
\end{align*}}
\end{restatable}
\begin{proof}[Proof of Lemma~\ref{lemma:gdtwo-rec}]
Fix on $G_{i-1}$ for an iteration $i$ of the outer for loop in Alg.~\ref{alg:gdtwo}.
Let $A_l$ be the set after for loop in Lines~\ref{line:gdtwo-for-2-start}-\ref{line:gdtwo-for-2-end} ends (with $m$ iterations).
Then,
\begin{align*}
&\sum_{l\in [\ell]}\marge{O}{A_{l}} 
\le \sum_{l\in [\ell]} \marge{O_l}{A_{l}} + \sum_{1\le l_1 < l_2 \le \ell} \left(\marge{O_{l_1}}{A_{l_2}}+\marge{O_{l_2}}{A_{l_1}}\right) \tag{Inequality~\ref{inq:gdtwo-par}}\\
&\le \sum_{l\in [\ell]} \marge{A_{l}}{G_{i-1}} + \sum_{1\le l_1 < l_2 \le \ell}\left(1+\frac{1}{m}\right) \left(\marge{A_{l_1}}{G_{i-1}}+\marge{A_{l_2}}{G_{i-1}}\right)\tag{Lemma~\ref{lemma:par-A}}\\
&\le \ell\left(1+\frac{1}{m}\right) \sum_{l\in [\ell]} \marge{A_{l}}{G_{i-1}}\\
\Rightarrow& \left(\ell+1\right)\left(1+\frac{1}{m}\right)\sum_{l\in [\ell]}\marge{A_l}{G_{i-1}} \ge \sum_{l\in [\ell]}\ff{O\cup A_l} - \ell \ff{G_{i-1}}\\
& \hspace*{15em} \ge \left(\ell-1\right)\ff{O\cup G_{i-1}} - \ell \ff{G_{i-1}},\numberthis \label{inq:itg-rec-1}
\end{align*}
where the last inequality follows from Proposition~\ref{prop:sum-marge}.
Then, it holds that
\begin{align*}
&\exc{\ff{G_i} - \ff{G_{i-1}}}{G_{i-1}}  = \frac{1}{\ell}\sum_{l \in [\ell]}\marge{A_{l}}{G_{i-1}}\\
&\ge \frac{1}{\ell+1}\cdot\frac{m}{m+1}\cdot\left(\left(1-\frac{1}{\ell}\right)\ff{O\cup G_{i-1}} - \ff{G_{i-1}}\right) \tag{Inequality~\eqref{inq:itg-rec-1}}
\end{align*}
By unfixing $G_{i-1}$, the lemma holds.
\end{proof}

\begin{restatable}{lemma}{lemmagdtwodeg}\label{lemma:gdtwo-deg}
For any iteration $i$ of the outer for loop in Alg.~\ref{alg:gdtwo},
it holds that

\vspace*{-1em}
{\small\begin{align*}
\ex{\ff{O\cup G_i}} \ge \left(1-\frac{1}{\ell}\right) \ex{\ff{O\cup G_{i-1}}}.
\end{align*}}
\end{restatable}
\begin{proof}[Proof of Lemma~\ref{lemma:gdtwo-deg}]
Fix on $G_{i-1}$ at the beginning of this iteration.
Since $\left\{A_l\setminus G_{i-1}\right\}_{l\in [\ell]}$ 
are pairwise disjoint sets at the end of this iteration,
by Proposition~\ref{prop:sum-marge},
it holds that
\[\exc{\ff{O\cup G_i}}{G_{i-1}} = \frac{1}{\ell}\sum_{l\in [\ell]}\ff{O\cup A_l} \ge \left(1-\frac{1}{\ell}\right)\ff{O\cup G_{i-1}}.\]
Then, by unfixing $G_{i-1}$, the lemma holds.
\end{proof}

\subsection{Proof of Theorem~\ref{thm:gdtwo}}\label{apx:gdtwo-approx}
\thmgdtwo*
\begin{proof}
By Lemma~\ref{lemma:gdtwo-rec} and~\ref{lemma:gdtwo-deg},
the recurrence of $\ex{\ff{G_i}}$ can be expressed as follows,
\begin{align*}
\ex{\ff{G_i}} &\ge \left(1-\frac{1}{\ell+1}\left(1-\frac{1}{m+1}\right)\right)\ex{\ff{G_{i-1}}} + \frac{1}{\ell+1}\left(1-\frac{1}{m+1}\right)\left(1-\frac{1}{\ell}\right)^i\ff{O}\\
&\ge \left(1-\frac{1}{\ell}\right)\ex{\ff{G_{i-1}}} + \frac{1}{\ell+1}\left(1-\frac{1}{m+1}\right)\left(1-\frac{1}{\ell}\right)^i\ff{O}.
\end{align*}
By solving the above recurrence,
\begin{align*}
\ex{\ff{G_{\ell}}} &\ge \frac{\ell}{\ell+1}\left(1-\frac{1}{m+1}\right)\left(1-\frac{1}{\ell}\right)^\ell\ff{O}\\
&\ge \frac{\ell-1}{\ell+1}\left(1-\frac{1}{m+1}\right)e^{-1}\ff{O}\tag{Lemma~\ref{lemma:val-inq}}\\
&\ge \left(1-\frac{2}{\ell}\right)\left(1-\frac{\ell}{k}\right)e^{-1}\ff{O} \tag{$m = \left\lfloor \frac{k}{\ell} \right\rfloor$}\\
&\ge \frac{1}{1-\frac{\ell}{k}}\left(1-\frac{2}{\ell}-\frac{2\ell}{k}+\frac{4}{k}\right)e^{-1}\ff{O}\\
&\ge \frac{1}{1-\frac{\ell}{k}}\left(e^{-1}-\epsi\right)\ff{O}. \tag{$\ell \ge \frac{2}{e\epsi}, k\ge \frac{2(\ell-2)}{e\epsi-\frac{2}{\ell}}$}
\end{align*}
By Inequality~\eqref{inq:dif-opt}, the approximation ratio of Alg.~\ref{alg:gdtwo} is $e^{-1}-\epsi$.
\end{proof}

\section{Pseudocodes and Analysis of Algorithms in Section~\ref{sec:tg}}
\label{apx:tg}
In this section, we provide the pseudocodes and analysis of
the simplified fast \ig~\citep{DBLP:conf/nips/Kuhnle19} 
and \itg~\citep{DBLP:conf/kdd/ChenK23},
implemented as Alg.~\ref{alg:tgone} and~\ref{alg:tgtwo}, respectively.
The analysis of these algorithms employ a blending technique
to eliminate the guessing step in their original version.
\subsection{Simplified Fast \ig with $1/4-\epsi$ Approximation Ratio (Alg.~\ref{alg:tgone})}
\begin{algorithm}[ht]
    \KwIn{evaluation oracle $f:2^{\uni} 
    \to \reals$, constraint $k$, error $\epsi$}
    \Init{$A\gets \emptyset$, $B\gets \emptyset$, $M\gets \max_{x \in \uni}\ff{\{x\}}$,
    $\tau_1\gets M$, $\tau_2\gets M$}
    \For{$i\gets 1$ to $k$}{
        \While{$\tau_1 \ge \frac{\epsi M}{k}$ and $|A| < k$}{
            \If{$\exists a \in \uni\setminus\left(A\cup B\right)$ \st $\marge{a}{A} \ge \tau_1$}{
            $A\gets A+ a$\;
            \textbf{break}\;}
            \lElse{$\tau_1 \gets (1-\epsi)\tau_1$}
        }
        \While{$\tau_2 \ge \frac{\epsi M}{k}$ and $|B| < k$}{
            \If{$\exists b \in \uni\setminus\left(A\cup B\right)$ \st $\marge{b}{B} \ge \tau_2$}{
            $B\gets B+ b$\;
            \textbf{break}\;}
            \lElse{$\tau_2 \gets (1-\epsi)\tau_2$}
        }
    }
    \Return{$S\gets \argmax\{\ff{A}, \ff{B}\}$}
    \caption{A nearly-linear time, $(1/4-\epsi)$-approximation algorithm.}
    \label{alg:tgone}
\end{algorithm}
\thmtgone*
\begin{proof}
\textbf{Query Complexity.}
Without loss of generality, we analyze the number queries related to set $A$.
For each threshold value $\tau_1$, at most $n$ queries are made to the value oracle.
Since $\tau_1$ is initialized with value $M$, decreases by a factor of $1-\epsi$,
and cannot exceed $\frac{\epsi M}{k}$,
there are at most $\log_{1-\epsi}\left(\frac{\epsi}{k}\right)+1$ possible values of $\tau_1$.
Therefore, the total number of queries is bounded as follows,
\begin{align*}
\#\text{Queries} \le 2\cdot n\cdot \left(\log_{1-\epsi}\left(\frac{\epsi}{k}\right)+1\right)
\le \oh{n\log(k)/\epsi},
\end{align*}
where the last inequality follows from the first inequality in Lemma~\ref{lemma:val-inq}.

\textbf{Approximation Ratio.}
Since $A$ and $B$ are disjoint, by submodularity and non-negativity,
\begin{equation}\label{inq:tgone-1}
\ff{O} \le \ff{O\cup A} + \ff{O\cup B}.
\end{equation}

Let $a_i$ be the $i$-th element added to $A$,
$A_i$ be the first $i$ elements added to $A$,
and $\tau_1^{a_i}$ be the threshold value when $a_i$ is added to $A$.
Similarly, define $b_i$, $B_i$, and $\tau_2^{b_i}$.
Let $i^* = \max\{i \le |A|: A_i \subseteq O\}$
and $j^* = \max\{i \le |B|: B_i \subseteq O\}$.
If either $i^*= k$ or $j^* = k$,
then $\ff{S}= \ff{O}$.
Next, we follow the analysis of Alg.~\ref{alg:gdone} in Section~\ref{sec:greedy-1/4}
to analyze the approximation ratio of Alg.~\ref{alg:tgone}.

\textbf{Case 1: $0\le i^*\le j^* < k$; Fig.~\ref{fig:gdone-1}.}
First, we bound $\ff{O\cup A}$. 
Since $B_{i^*} \subseteq O$, by submodularity
\begin{equation}\label{inq:tgone-3}
\ff{O\cup A} - \ff{A} \le \marge{B_{i^*}}{A} + \marge{O\setminus B_{i^*}}{A}
\le \ff{B_{i^*}}+ \sum_{o\in O\setminus \left(A\cup B_{i^*}\right)}\marge{o}{A}.
\end{equation}
Next, we bound $\marge{o}{A}$ for each $o\in O\setminus \left(A\cup B_{i^*}\right)$.

Let $\tilde{O} = O\setminus \left(A \cup B_{i^*}\right)$.
Obviously, it holds that $|\tilde{O}|\le k-i^*$.
Then, order $\tilde{O}$ as $\{o_1, o_2, \ldots\}$ such that $o_i \not \in B_{i+i^*-1}$,
for all $1\le i \le |\tilde{O}|$.
If $|A| < k$, the algorithm terminates with $\tau_1 < \frac{\epsi M}{k}$.
Thus, it follows that
\begin{equation}\label{inq:tgone-4}
\marge{o_i}{A} < \frac{\epsi M}{k(1-\epsi)}, \forall |A|-i^* < i \le |\tilde{O}|.
\end{equation}

Next, consider tuple $(o_i, a_{i + i^*}, A_{i+i^*-1})$,
for any $1\le i \le \min\{|\tilde{O}|, |A|-i^*\}$.
Since $\tau_1^{a_{i + i^*}}$ is the threshold value when $a_{i + i^*}$ is added,
it holds that 
\begin{equation}\label{inq:tgone-2}
\marge{a_{i + i^*}}{A_{i+i^*-1}} \ge \tau_1^{a_{i + i^*}},
\forall 1\le i\le |A| - i^*.
\end{equation}
Then, we show that $\marge{o_i}{A_{i+i^*-1}} < \tau_1^{a_{i + i^*}}/(1-\epsi)$ always holds
for any $1\le i \le \min\{|\tilde{O}|, |A|-i^*\}$.

Since $M = \max_{x\in \uni}\ff{\{x\}}$,
if $\tau_1^{a_{i+i^*}} \ge M$,
it always holds that $\marge{o_i}{A_{i+i^*-1}} < M/(1-\epsi)\le \tau_1^{a_{i+i^*}}/(1-\epsi)$.
If $\tau_1^{a_{i+i^*}} < M$, 
since $o_i \not \in B_{i+i^*-1}$,
$o_i$ is not considered to be added to $A$ with threshold value $\tau_1^{a_{i+i^*}}/(1-\epsi)$.
Then, by submodularity,
$\marge{o_i}{A_{i+i^*-1}} < \tau_1^{a_{i+i^*}}/(1-\epsi)$.
Therefore, by submodularity and Inequality~\eqref{inq:tgone-2},
it holds that 
\begin{equation}\label{inq:tgone-5}
\marge{o_i}{A} \le \marge{o_i}{A_{i+i^*-1}} < \marge{a_{i + i^*}}{A_{i+i^*-1}}/(1-\epsi), \forall 1\le i \le \min\{|\tilde{O}|, |A|-i^*\}.
\end{equation}

Then,
\begin{align*}
\ff{O\cup A} - \ff{A} &\le \ff{B_{i^*}} + \sum_{o\in O\setminus \left(A\cup B_{i^*}\right)}\marge{o}{A}\\
&\le \ff{B_{i^*}} + \sum_{i = 1}^{\min\{|\tilde{O}|, |A|\}-i^*}\marge{a_{i + i^*}}{A_{i+i^*-1}}/(1-\epsi) + \frac{\epsi M}{1-\epsi}\\
&\le \frac{1}{1-\epsi}\left(\ff{B_{i^*}} + \ff{A}-\ff{A_{i^*}} + \epsi \ff{O}\right),\numberthis \label{inq:tgone-10}
\end{align*}
where the first inequality follows from Inequality~\eqref{inq:tgone-3};
the second inequality follows from Inequalities~\eqref{inq:tgone-4} and~\eqref{inq:tgone-5};
and the last inequality follows from $M\le \ff{O}$.

Second, we bound $\ff{O\cup B}$.
Since $A_{i^*} \subseteq O$, by submodularity
\begin{equation}\label{inq:tgone-6}
\ff{O\cup B} - \ff{B} \le \marge{A_{i^*}}{B} + \marge{O\setminus A_{i^*}}{B}
\le \ff{A_{i^*}}+ \sum_{o\in O\setminus \left(B\cup A_{i^*}\right)}\marge{o}{B}.
\end{equation}
Next, we bound $\marge{o}{B}$ for each $o\in O\setminus \left(B\cup A_{i^*}\right)$.

Let $\tilde{O} = O\setminus \left(B\cup A_{i^*}\right)$.
Obviously, it holds that $|\tilde{O}|\le k-i^*$.
Then, since $a_{i^*+1} \not\in O$,
we can order $\tilde{O}$ as $\{o_1, o_2, \ldots\}$ such that $o_i \not \in A_{i+i^*}$,
for all $1\le i \le |\tilde{O}|$.
If $|B| < k$, the algorithm terminates with $\tau_2 < \frac{\epsi M}{k}$.
Thus, it follows that
\begin{equation}\label{inq:tgone-7}
\marge{o_i}{B} < \frac{\epsi M}{k(1-\epsi)}, \forall |B|-i^* < i \le |\tilde{O}|.
\end{equation}

Next, consider tuple $(o_i, b_{i + i^*}, B_{i+i^*-1})$,
for any $1\le i \le \min\{|\tilde{O}|, |B|-i^*\}$.
Since $\tau_2^{b_{i + i^*}}$ is the threshold value when $b_{i + i^*}$ is added,
it holds that 
\begin{equation}\label{inq:tgone-8}
\marge{b_{i + i^*}}{B_{i+i^*-1}} \ge \tau_2^{b_{i + i^*}},
\forall 1\le i\le |B| - i^*.
\end{equation}
Then, we show that $\marge{o_i}{B_{i+i^*-1}} < \tau_2^{b_{i + i^*}}/(1-\epsi)$ always holds
for any $1\le i \le \min\{|\tilde{O}|, |B|-i^*\}$.

Since $M = \max_{x\in \uni}\ff{\{x\}}$,
if $\tau_2^{b_{i+i^*}} \ge M$,
it always holds that $\marge{o_i}{B_{i+i^*-1}} < M/(1-\epsi)\le \tau_2^{b_{i+i^*}}/(1-\epsi)$.
If $\tau_2^{b_{i+i^*}} < M$, 
since $o_i \not \in A_{i+i^*}$,
$o_i$ is not considered to be added to $B$ with threshold value $\tau_2^{b_{i+i^*}}/(1-\epsi)$.
Then, by submodularity,
$\marge{o_i}{B_{i+i^*-1}} < \tau_2^{b_{i+i^*}}/(1-\epsi)$.
Therefore, by submodularity and Inequality~\eqref{inq:tgone-8},
it holds that 
\begin{equation}\label{inq:tgone-9}
\marge{o_i}{B} \le \marge{o_i}{B_{i+i^*-1}} < \marge{b_{i + i^*}}{B_{i+i^*-1}}/(1-\epsi), \forall 1\le i \le \min\{|\tilde{O}|, |B|-i^*\}.
\end{equation}

Then,
\begin{align*}
\ff{O\cup B} - \ff{B} &\le \ff{A_{i^*}} + \sum_{o\in O\setminus \left(B\cup A_{i^*}\right)}\marge{o}{B}\\
&\le \ff{A_{i^*}} + \sum_{i = 1}^{\min\{|\tilde{O}|, |B|\}-i^*}\marge{b_{i + i^*}}{B_{i+i^*-1}}/(1-\epsi) + \frac{\epsi M}{1-\epsi}\\
&\le \frac{1}{1-\epsi}\left(\ff{A_{i^*}} + \ff{B}-\ff{B_{i^*}} + \epsi \ff{O}\right),\numberthis \label{inq:tgone-11}
\end{align*}
where the first inequality follows from Inequality~\eqref{inq:tgone-6};
the second inequality follows from Inequalities~\eqref{inq:tgone-7} and~\eqref{inq:tgone-9};
and the last inequality follows from $M\le \ff{O}$.

By Inequalities~\eqref{inq:tgone-1},~\eqref{inq:tgone-10} and~\eqref{inq:tgone-11},
it holds that
\begin{align*}
&\ff{O} \le \frac{2-\epsi}{1-\epsi}\left(\ff{A} + \ff{B}\right) + \frac{2\epsi}{1-\epsi}\ff{O}\\
\Rightarrow &\ff{S} \ge \left(\frac{1}{4} - \frac{5}{2(4-2\epsi)}\epsi\right)\ff{O}
\ge \left(\frac{1}{4}-\epsi\right)\ff{O}\tag{$\epsi < 1/2$}
\end{align*}

\textbf{Case 2: $0\le j^* < i^* < k$; Fig.~\ref{fig:gdone-2}.}

First, we bound $\ff{O\cup A}$. 
Since $B_{j^*} \subseteq O$, by submodularity
\begin{equation}\label{inq:tgone-20}
\ff{O\cup A} - \ff{A} \le \marge{B_{j^*}}{A} + \marge{O\setminus B_{j^*}}{A}
\le \ff{B_{j^*}}+ \sum_{o\in O\setminus \left(A\cup B_{j^*}\right)}\marge{o}{A}.
\end{equation}
Next, we bound $\marge{o}{A}$ for each $o\in O\setminus \left(A\cup B_{j^*}\right)$.

Let $\tilde{O} = O\setminus \left(A \cup B_{j^*}\right)$.
Since $i^* > j^*\ge 0$, 
it holds that $|\tilde{O}|\le k-j^*-1$.
Since $b_{j^*+1}\not \in O$,
we can order $\tilde{O}$ as $\{o_1, o_2, \ldots\}$ such that $o_i \not \in B_{i+j^*}$,
for all $1\le i \le |\tilde{O}|$.
If $|A| < k$, the algorithm terminates with $\tau_1 < \frac{\epsi M}{k}$.
Thus, it follows that
\begin{equation}\label{inq:tgone-21}
\marge{o_i}{A} < \frac{\epsi M}{k(1-\epsi)}, \forall |A|-j^*-1 < i \le |\tilde{O}|.
\end{equation}

Next, consider tuple $(o_i, a_{i + j^*+1}, A_{i+j^*})$,
for any $1\le i \le \min\{|\tilde{O}|, |A|-j^*-1\}$.
Since $\tau_1^{a_{i + j^*+1}}$ is the threshold value when $a_{i + j^*+1}$ is added,
it holds that 
\begin{equation}\label{inq:tgone-22}
\marge{a_{i + j^*+1}}{A_{i+j^*}} \ge \tau_1^{a_{i + j^*+1}},
\forall 1\le i\le |A| - j^*-1.
\end{equation}
Then, we show that $\marge{o_i}{A_{i+j^*}} < \tau_1^{a_{i + j^*+1}}/(1-\epsi)$ always holds
for any $1\le i \le \min\{|\tilde{O}|, |A|-j^*-1\}$.

Since $M = \max_{x\in \uni}\ff{\{x\}}$,
if $\tau_1^{a_{i + j^*+1}} \ge M$,
it always holds that $\marge{o_i}{A_{i+j^*}} < M/(1-\epsi)\le \tau_1^{a_{i + j^*+1}}/(1-\epsi)$.
If $\tau_1^{a_{i + j^*+1}} < M$, 
since $o_i \not \in B_{i+j^*}$,
$o_i$ is not considered to be added to $A$ with threshold value $\tau_1^{a_{i + j^*+1}}/(1-\epsi)$.
Then, by submodularity,
$\marge{o_i}{A_{i+j^*}} < \tau_1^{a_{i + j^*+1}}/(1-\epsi)$.
Therefore, by submodularity and Inequality~\eqref{inq:tgone-22},
it holds that 
\begin{equation}\label{inq:tgone-23}
\marge{o_i}{A} \le \marge{o_i}{A_{i+j^*}} < \marge{a_{i + j^*+1}}{A_{i+j^*}}/(1-\epsi), \forall 1\le i \le \min\{|\tilde{O}|, |A|-j^*-1\}.
\end{equation}

Then,
\begin{align*}
\ff{O\cup A} - \ff{A} &\le \ff{B_{j^*}} + \sum_{o\in O\setminus \left(A\cup B_{j^*}\right)}\marge{o}{A}\\
&\le \ff{B_{j^*}} + \sum_{i = 1}^{\min\{|\tilde{O}|, |A|-j^*-1\}}\marge{a_{i + j^*+1}}{A_{i+j^*}}/(1-\epsi) + \frac{\epsi M}{1-\epsi}\\
&\le \frac{1}{1-\epsi}\left(\ff{B_{j^*}} + \ff{A}-\ff{A_{j^*+1}} + \epsi \ff{O}\right),\numberthis \label{inq:tgone-24}
\end{align*}
where the first inequality follows from Inequality~\eqref{inq:tgone-20};
the second inequality follows from Inequalities~\eqref{inq:tgone-21} and~\eqref{inq:tgone-23};
and the last inequality follows from $M\le \ff{O}$.

Second, we bound $\ff{O\cup B}$.
Since $A_{j^*+1} \subseteq O$, by submodularity
\begin{equation}\label{inq:tgone-25}
\ff{O\cup B} - \ff{B} \le \marge{A_{j^*+1}}{B} + \marge{O\setminus A_{j^*+1}}{B}
\le \ff{A_{j^*+1}}+ \sum_{o\in O\setminus \left(B\cup A_{j^*+1}\right)}\marge{o}{B}.
\end{equation}
Next, we bound $\marge{o}{B}$ for each $o\in O\setminus \left(B\cup A_{j^*+1}\right)$.

Let $\tilde{O} = O\setminus \left(B\cup A_{j^*+1}\right)$.
Obviously, it holds that $|\tilde{O}|\le k-j^*-1$.
Then, order $\tilde{O}$ as $\{o_1, o_2, \ldots\}$ such that $o_i \not \in A_{i+j^*}$,
for all $1\le i \le |\tilde{O}|$.
If $|B| < k$, the algorithm terminates with $\tau_2 < \frac{\epsi M}{k}$.
Thus, it follows that
\begin{equation}\label{inq:tgone-26}
\marge{o_i}{B} < \frac{\epsi M}{k(1-\epsi)}, \forall |B|-j^*-1 < i \le |\tilde{O}|.
\end{equation}

Next, consider tuple $(o_i, b_{i + j^*}, B_{i+j^*-1})$,
for any $1\le i \le \min\{|\tilde{O}|, |B|-j^*-1\}$.
Since $\tau_2^{b_{i + j^*}}$ is the threshold value when $b_{i + j^*}$ is added,
it holds that 
\begin{equation}\label{inq:tgone-27}
\marge{b_{i + j^*}}{B_{i+j^*-1}} \ge \tau_2^{b_{i + j^*}},
\forall 1\le i\le |B| - j^*-1.
\end{equation}
Then, we show that $\marge{o_i}{B_{i+j^*-1}} < \tau_2^{b_{i + j^*}}/(1-\epsi)$ always holds
for any $1\le i \le \min\{|\tilde{O}|, |B|-j^*-1\}$.

Since $M = \max_{x\in \uni}\ff{\{x\}}$,
if $\tau_2^{b_{i+j^*}} \ge M$,
it always holds that $\marge{o_i}{B_{i+j^*-1}} < M/(1-\epsi)\le \tau_2^{b_{i+j^*}}/(1-\epsi)$.
If $\tau_2^{b_{i+j^*}} < M$, 
since $o_i \not \in A_{i+j^*}$,
$o_i$ is not considered to be added to $B$ with threshold value $\tau_2^{b_{i+j^*}}/(1-\epsi)$.
Then, by submodularity,
$\marge{o_i}{B_{i+j^*-1}} < \tau_2^{b_{i+j^*}}/(1-\epsi)$.
Therefore, by submodularity and Inequality~\eqref{inq:tgone-27},
it holds that 
\begin{equation}\label{inq:tgone-28}
\marge{o_i}{B} \le \marge{o_i}{B_{i+j^*-1}} < \marge{b_{i + j^*}}{B_{i+j^*-1}}/(1-\epsi), \forall 1\le i \le \min\{|\tilde{O}|, |B|-j^*-1\}.
\end{equation}

Then,
\begin{align*}
\ff{O\cup B} - \ff{B} &\le \ff{A_{j^*+1}} + \sum_{o\in O\setminus \left(B\cup A_{j^*+1}\right)}\marge{o}{B}\\
&\le \ff{A_{j^*+1}} + \sum_{i = 1}^{\min\{|\tilde{O}|, |B|-j^*-1\}}\marge{b_{i + j^*}}{B_{i+j^*-1}}/(1-\epsi) + \frac{\epsi M}{1-\epsi}\\
&\le \frac{1}{1-\epsi}\left(\ff{A_{j^*+1}} + \ff{B}-\ff{B_{i^*}} + \epsi \ff{O}\right),\numberthis \label{inq:tgone-29}
\end{align*}
where the first inequality follows from Inequality~\eqref{inq:tgone-25};
the second inequality follows from Inequalities~\eqref{inq:tgone-26} and~\eqref{inq:tgone-28};
and the last inequality follows from $M\le \ff{O}$.

By Inequalities~\eqref{inq:tgone-1},~\eqref{inq:tgone-24} and~\eqref{inq:tgone-29},
it holds that
\begin{align*}
&\ff{O} \le \frac{2-\epsi}{1-\epsi}\left(\ff{A} + \ff{B}\right) + \frac{2\epsi}{1-\epsi}\ff{O}\\
\Rightarrow &\ff{S} \ge \left(\frac{1}{4} - \frac{5}{2(4-2\epsi)}\epsi\right)\ff{O}
\ge \left(\frac{1}{4}-\epsi\right)\ff{O}\tag{$\epsi < 1/2$}
\end{align*}

Therefore, in both cases, it holds that
\[\ff{S} \ge \left(\frac{1}{4}-\epsi\right)\ff{O} .\]
\end{proof}


\subsection{Simplified Fast \itg with $1/e-\epsi$ Approximation Ratio (Alg.~\ref{alg:tgtwo})}
\begin{algorithm}[ht]
    \KwIn{evaluation oracle $f:2^{\uni} \to \reals$, constraint $k$, error $\epsi$}
    \Init{$G_0\gets \emptyset$, $\epsi'\gets \frac{\epsi}{2}$, $m \gets \left\lfloor\frac{k}{\ell}\right\rfloor$, $\ell\gets \left \lceil\frac{4}{e\epsi'}\right \rceil$,
    $M\gets \max_{x\in \uni} \ff{\{x\}}$}
    \For{$i\gets 1$ to $\ell$}{
        $\tau_l \gets M, \forall l\in [\ell]$\;
        $A_{l}\gets G_{i-1}, \forall l \in [\ell]$\;
        \For{$j\gets 1$ to $m$}{
            \For{$l\gets 1$ to $\ell$}{
                \While{$\tau_l \ge \frac{\epsi' M}{k}$ and $|A_l\setminus G_{i-1}| < m$}{
                    \If{$\exists x \in \uni\setminus\left(\bigcup_{r\in [\ell]} A_r\right)$ \st $\marge{a}{A_l} \ge \tau_l$}{
                    $A_l\gets A_l+ x$\;
                    \textbf{break}\;}
                    \lElse{$\tau_l \gets (1-\epsi')\tau_l$}
                }
            }
        }
        $G_i\gets$ a random set in $\{A_l\}_{l\in [\ell]}$\;
    }
    \Return{$G_\ell$}
    \caption{A nearly-linear time, $(1/e-\epsi)$-approximation algorithm.}
    \label{alg:tgtwo}
\end{algorithm}
\thmtgtwo*
\begin{proof}
When $k\,\text{mod}\,\ell > 0$, the algorithm returns an approximation with a size constraint of 
$\ell\cdot\left\lfloor \frac{k}{\ell}\right\rfloor$, where by Proposition~\ref{prop:dif-opt},
\begin{equation}\label{inq:tgtwo-dif-opt}
\ff{O'} \ge \left(1-\frac{\ell}{k}\right)\ff{O}, 
O' = \argmax\limits_{S\subseteq \uni, |S|\le \ell\cdot \left\lfloor \frac{k}{\ell} \right\rfloor}\ff{S}.
\end{equation}
In the following, we only consider the case where $k\,\text{mod}\,\ell = 0$.

At every iteration of the outer for loop,
$\ell$ solutions are constructed, with each solution being augmented
by at most $k/\ell$ elements.
To bound the marginal gain of the optimal set $O$ on each solution set $A_l$,
we consider partitioning $O$ into $\ell$ subsets.
We formalize this partition in the following claim,
which yields a result analogous to Claim~\ref{claim:par-A} presented in 
Section~\ref{sec:greedy-1/e}.
Specifically, the claim states that the optimal set $O$ can be evenly 
divided into $\ell$ subsets,
where each subset only overlaps with only one solution set.
\begin{claim}
At an iteration $i$ of the outer for loop in Alg.~\ref{alg:tgtwo},
let $G_{i-1}$ be $G$ at the start of this iteration,
and $A_{l}$ be the set at the end of this iteration,
for each $l\in [\ell]$.
% Add dummy elements to $O\setminus G_{i-1}$ until its size equals $k$.
The set $O\setminus G_{i-1}$ can then be split into $\ell$ pairwise disjoint sets $\{O_1, \ldots, O_\ell\}$
such that $|O_l| \le\frac{k}{\ell}$ and $\left(O\setminus G_{i-1}\right) \cap \left(A_{l}\setminus G_{i-1}\right) \subseteq O_l$, for all $l \in [\ell]$.
\end{claim}
Next, based on such partition, we introduce the following lemma, 
which provides a bound on the marginal gain of any subset $O_{l_1}$ 
with respect to any solution set $A_{l_2}$,
where $1\le l_1, l_2 \le \ell$.
\begin{lemma}\label{lemma:tg-par-A}
Fix on $G_{i-1}$ for an iteration $i$ of the outer for loop in Alg.~\ref{alg:tgtwo}.
Following the definition in Claim~\ref{claim:par-A}, it holds that
\begin{align*}
\text{1) }&\marge{O_{l}}{A_{l}}\le \frac{\marge{A_{l}}{G_{j-1}}}{1-\epsi'}+\frac{\epsi' M}{(1-\epsi')\ell}, \forall 1\le l \le \ell,\\
\text{2) }&\marge{O_{l_2}}{A_{l_1}} + \marge{O_{l_1}}{A_{l_2}} \le \frac{1}{1-\epsi'}\left(1+\frac{1}{m}\right)\left(\marge{A_{l_1}}{G_{i-1}}+\marge{A_{l_2}}{G_{i-1}}\right)
+\frac{2\epsi' M}{(1-\epsi')\ell}, \forall 1\le l_1 < l_2 \le \ell.
\end{align*}
\end{lemma}
Followed by the above lemma, 
we provide the recurrence of $\ex{\ff{G_i}}$ and $\ex{\ff{O\cup G_i}}$.
\begin{lemma}\label{lemma:tg-recur}
For any iteration $i$ of the outer for loop in Alg.~\ref{alg:tgtwo},
it holds that
\begin{align*}
\text{1) } & \ex{\ff{O\cup G_i}}\ge \left(1-\frac{1}{\ell}\right) \ex{\ff{O\cup G_{i-1}}}\\
\text{2) } & \ex{\ff{G_i} - \ff{G_{i-1}}}
\ge\frac{1}{1+\frac{\ell}{1-\epsi'}}\left(1-\frac{1}{m+1}\right)\left(\left(1-\frac{1}{\ell}\right)  \ex{\ff{O\cup G_{i-1}}} - \ex{\ff{G_{i-1}}} - \frac{\epsi'}{1-\epsi'}\ff{O}\right).
\end{align*}
\end{lemma}
By solving the recurrence in Lemma~\ref{lemma:tg-recur},
we calculate the approximation ratio of the algorithm as follows,
\begin{align*}
&\ex{\ff{G_{i}}}  \ge \left(1-\frac{1}{\ell}\right) \ex{\ff{G_{i-1}}}
+ \frac{1}{1+\frac{\ell}{1-\epsi'}}\left(1-\frac{1}{m+1}\right)\left(\left(1-\frac{1}{\ell}\right)^i - \frac{\epsi'}{1-\epsi'}\right)\ff{O}\\
\Rightarrow& \ex{\ff{G_\ell}} \ge \frac{\ell}{1+\frac{\ell}{1-\epsi'}}\left(1-\frac{1}{m+1}\right)\left(\left(1-\frac{1}{\ell}\right)^\ell - \frac{\epsi'}{1-\epsi'}\left(1-\left(1-\frac{1}{\ell}\right)^\ell\right)\right)\ff{O}\\
&\hspace*{4em} \ge \frac{\ell-1}{1+\frac{\ell}{1-\epsi'}}\left(1-\frac{1}{m+1}\right)\left(e^{-1} - \frac{\epsi'}{1-\epsi'}\left(1-e^{-1}\right)\right)\ff{O}\\
&\hspace*{4em} \ge \frac{1}{1-\frac{\ell}{k}}\left(1-\epsi' - \frac{2}{\ell}\right)\left(1-\frac{\ell}{k}\right)^2\left(e^{-1} - \frac{\epsi'}{1-\epsi'}\left(1-e^{-1}\right)\right) \ff{O}\\
% &\hspace*{4em} \ge \frac{1}{1-\frac{\ell}{k}}\left(1-\epsi' - \frac{2}{\ell}\right)\left(1-\frac{2\ell}{k}\right)\left(e^{-1} - \frac{\epsi'}{1-\epsi'}\left(1-e^{-1}\right)\right) \ff{O}\\
&\hspace*{4em} \ge \frac{1}{1-\frac{\ell}{k}}\left(1-\epsi' - \frac{2}{\ell}-\frac{2(1-\epsi')\ell}{k}\right)\left(e^{-1} - \frac{\epsi'}{1-\epsi'}\left(1-e^{-1}\right)\right) \ff{O}\\
&\hspace*{4em} \ge \frac{1}{1-\frac{\ell}{k}} \left(1-(e+1)\epsi'\right)\left(e^{-1} - \frac{\epsi'}{1-\epsi'}\left(1-e^{-1}\right)\right) \ff{O}\tag{$\ell \ge \frac{2}{e\epsi'}, k \ge \frac{2(1-\epsi')\ell}{e\epsi'-\frac{2}{\ell}}$}\\
&\hspace*{4em} \ge \frac{1}{1-\frac{\ell}{k}} \left(e^{-1}-\epsi\right)\ff{O}\tag{$\epsi' = \frac{\epsi}{2}$}.
\end{align*}
By Inequality~\ref{inq:tgtwo-dif-opt},
the approximation ratio of Alg.~\ref{alg:tgtwo} is $e^{-1}-\epsi$.
\end{proof}

In the rest of this section, we provide the proofs for 
Lemma~\ref{lemma:tg-par-A} and~\ref{lemma:tg-recur}.
\begin{proof}[Proof of Lemma~\ref{lemma:tg-par-A}]
At iteration $i$ of the outer for loop,
let $A_l$ be the set at the end of iteration $i$,
$a_{l, j}$ be the $j$-th element added to $A_l$,
$\tau_l^j$ be the threshold value of $\tau_l$ when $a_{l, j}$ is added to $A_l$,
and $A_{l, j}$ be $A_l$ after $a_{l, j}$ is added to $A_l$.
Let $c_l^* = \max\{c\in [m]:A_{l, c}\setminus G_{i-1}\subseteq O_l\}$.

First, we prove that the first inequality holds.
For each $l\in [\ell]$, order the elements in $O_l$ as $\{o_1, o_2, \ldots\}$
such that $o_j \not \in A_{l, j-1}$ for any $1\le j \le |A_l\setminus G_{i-1}|$,
and $o_j\not \in A_l$ for any $|A_l\setminus G_{i-1}| < j \le m$.

When $1\le j \le |A_l\setminus G_{i-1}|$, by Claim~\ref{claim:par-A},
each $o_j$ is either added to $A_l$ or not in any solution set.
Since $\tau_l$ is initialized with the maximum marginal gain $M$,
$o_j$ is not considered to be added to $A_l$ with threshold value 
$\tau_l^j/(1-\epsi')$.
Therefore, by submodularity it holds that
\begin{equation}\label{inq:tgtwo-1}
\marge{o_j}{A_{l, j-1}} < \tau_l^j/(1-\epsi')\le \marge{a_{l,j}}{A_{l,j-1}}/(1-\epsi'),
\forall 1\le j \le |A_l\setminus G_{i-1}|.
\end{equation}

When $|A_l\setminus G_{i-1}| <  m$,
the minimum value of $\tau_l$ is less than $\frac{\epsi' M}{k}$.
Then, for any $|A_l\setminus G_{i-1}| < j \le m$,
$o_j$ is not considered to be added to $A_l$ with threshold value less than $\frac{\epsi' M}{(1-\epsi')k}$.
It follows that 
\begin{equation}\label{inq:tgtwo-2}
\marge{o_j}{A_l} \le \frac{\epsi' M}{(1-\epsi')k},
\forall |A_l\setminus G_{i-1}| < j \le m.
\end{equation}

Then,
\begin{align*}
\marge{O_{l}}{A_{l}} &\le \sum_{o_j\in O_l} \marge{o_j}{A_{l}} \tag{Proposition~\ref{prop:sum-marge}}\\
&\le \sum_{j=1}^{|A_l\setminus G_{i-1}|} \marge{o_j}{A_{l, j-1}} + 
\sum_{j=|A_l\setminus G_{i-1}|+1}^m \marge{o_j}{A_{l}}\tag{Submodularity}\\
&\le \sum_{j=1}^{|A_l\setminus G_{i-1}|}\frac{\marge{a_{l,j}}{A_{l,j-1}}}{1-\epsi'}+\frac{\epsi' M}{(1-\epsi')\ell}
\tag{Inequalities~\eqref{inq:tgtwo-1} and~\eqref{inq:tgtwo-2}}\\
&= \frac{\marge{A_{l}}{G_{j-1}}}{1-\epsi'}+\frac{\epsi' M}{(1-\epsi')\ell}.
\end{align*}
The first inequality holds.

In the following, we prove that the second inequality holds.
For any $1\le l_1\le l_2\le \ell$,
we analyze two cases of the relationship between $c_{l_1}^* $ and $ c_{l_2}^*$ in the following.

% \textbf{Case 1: $c_{l_1}^* = c_{l_2}^* = m$.}
% Then, $O_{l_1} = A_{l_1}\setminus G_{i-1}$ and $O_{l_2} = A_{l_2}\setminus G_{i-1}$.
% By submodularity,
% \[\marge{O_{l_2}}{A_{l_1}} + \marge{O_{l_1}}{A_{l_2}} \le \marge{O_{l_2}}{G_{i-1}} + \marge{O_{l_1}}{G_{i-1}} = \marge{A_{l_1}}{G_{i-1}} + \marge{A_{l_2}}{G_{i-1}}.\]
% Therefore, the lemma holds in this case.

\textbf{Case 1: $c_{l_1}^* \le c_{l_2}^*$; left half part in Fig.~\ref{fig:gdtwo}.}

First, we bound $\marge{O_{l_1}}{A_{l_2}}$.
Order the elements in $O_{l_1}\setminus A_{l_1, c_{l_1}^*}$ as $\{o_1, o_2, \ldots\}$ such that $o_j \not \in A_{l_1, c_{l_1}^*+j}$.
(Refer to the gray block with a dotted edge in the top left corner of Fig.~\ref{fig:gdtwo} for $O_{l_1}$.
If $c_{l_1}^*+j$ is greater than the number of elements added to $A_{l_1}$,
$A_{l_1, c_{l_1}^*+j}$ refers to $A_{l_1}$.)
Note that, since $A_{l_1, c_{l_1}^*} \subseteq O_{l_1}$,
it follows that $|O_{l_1}\setminus A_{l_1, c_{l_1}^*}| \le m - c_{l_1}^*$.

When $1 \le j \le |A_{l_2}\setminus G_{i-1}| - c_{l_1}^*$,
since each $o_j$ is either added to $A_{l_1}$ or not in any solution set by Claim~\ref{claim:par-A}
and $\tau_{l_2}$ is initialized with the maximum marginal gain $M$,
$o_j$ is not considered to be added to $A_{l_2}$ with threshold value $\tau_{l_2}^{c_{l_1}^* + j}/(1-\epsi')$.
Therefore, it holds that 
\begin{equation}\label{inq:tgtwo-case2-1}
\marge{o_j}{A_{l_2, c_{l_1}^*+j-1}} < \frac{\tau_{l_2}^{c_{l_1}^* + j}}{1-\epsi'} \le \frac{\marge{a_{l_2, c_{l_1}^* + j}}{A_{l_2, c_{l_1}^* + j-1}}}{1-\epsi'}, \forall 1\le j\le |A_{l_2}\setminus G_{i-1}|-c_{l_1}^*.
\end{equation}

When $|A_{l_2}\setminus G_{i-1}| < m$ and $|A_{l_2}\setminus G_{i-1}|-c_{l_1}^* < j\le m-c_{l_1}^*$,
this iteration ends with $\tau_{l_2} < \frac{\epsi' M}{k}$ and
$o_j$ is never considered to be added to $A_{l_2}$.
Thus, it holds that
\begin{equation}\label{inq:tgtwo-case2-3}
\marge{o_j}{A_{l_2}} < \frac{\epsi' M}{(1-\epsi')k}, 
\forall |A_{l_2}\setminus G_{i-1}|-c_{l_1}^* < j \le m-c_{l_1}^*.
\end{equation}

Then,
\begin{align*}
\marge{O_{l_1}}{A_{l_2}} &\le \marge{A_{l_1, c_{l_1}^*}}{A_{l_2}}  + \sum_{o_j \in O_{l_1}\setminus A_{l_1, c_{l_1}^*}}\marge{o_j}{A_{l_2}} \tag{Proposition~\ref{prop:sum-marge}}\\
&\le \marge{A_{l_1, c_{l_1}^*}}{G_{i-1}} + \sum_{j = 1}^{|A_{l_2}\setminus G_{i-1}|-c_{l_1}^*}\marge{o_j}{A_{l_2, , c_{l_1}^*+j-1}} + \sum_{j=|A_{l_2}\setminus G_{i-1}|-c_{l_1}^*+1}^{m-c_{l_1}^*} \marge{o_j}{A_{l_2}} \tag{submodularity}\\
&\le \ff{A_{l_1, c_{l_1}^*}}-\ff{G_{i-1}} + \sum_{j = 1}^{|A_{l_2}\setminus G_{i-1}|-c_{l_1}^*}\frac{\marge{a_{l_2, c_{l_1}^*+j}}{A_{l_2, , c_{l_1}^*+j-1}}}{1-\epsi'} + \frac{\epsi' M}{(1-\epsi')\ell} \tag{Inequality~\eqref{inq:tgtwo-case2-1} and~\eqref{inq:tgtwo-case2-3}}\\
& \le \ff{A_{l_1, c_{l_1}^*}}-\ff{G_{i-1}} + \frac{\ff{A_{l_2}} - \ff{A_{l_2, c_{l_1}^*}}}{1-\epsi'} + \frac{\epsi' M}{(1-\epsi')\ell} \numberthis \label{inq:tgtwo-case2-6}
\end{align*}

Similarly, we bound $\marge{O_{l_2}}{A_{l_1}}$ below.
Order the elements in $O_{l_2}\setminus A_{l_2, c_{l_1}^*}$ as $\{o_1, o_2, \ldots\}$ such that $o_j \not \in A_{l_2, c_{l_1}^*+j-1}$.
(See the gray block with a dotted edge in the bottom left corner of Fig.~\ref{fig:gdtwo} for $O_{l_2}$.
If $c_{l_1}^*+j-1$ is greater than the number of elements added to $A_{l_2}$,
$A_{l_2, c_{l_1}^*+j-1}$ refers to $A_{l_2}$.)
Note that, since $A_{l_2, c_{l_1}^*} \subseteq O_{l_2}$,
it follows that $|O_{l_2}\setminus A_{l_2, c_{l_1}^*}| \le m - c_{l_1}^*$.

When $1 \le j \le |A_{l_1}\setminus G_{i-1}|-c_{l_1}^*$,
since each $o_j$ is either added to $A_{l_2}$ or not in any solution set by Claim~\ref{claim:par-A}
and $\tau_{l_1}$ is initialized with the maximum marginal gain $M$,
$o_j$ is not considered to be added to $A_{l_1}$ with threshold value $\tau_{l_1}^{c_{l_1}^* + j}/(1-\epsi')$.
Therefore, it holds that 
\begin{equation}\label{inq:tgtwo-case2-4}
\marge{o_j}{A_{l_1, c_{l_1}^*+j-1}} < \frac{\tau_{l_1}^{c_{l_1}^* + j}}{1-\epsi'} \le \frac{\marge{a_{l_1, c_{l_1}^* + j}}{A_{l_1, c_{l_1}^* + j-1}}}{1-\epsi'}, \forall 1\le j\le |A_{l_2}\setminus G_{i-1}|-c_{l_1}^*.
\end{equation}

When $|A_{l_1}\setminus G_{i-1}| < m$ and $|A_{l_1}\setminus G_{i-1}|-c_{l_1}^* < j\le m-c_{l_1}^*$,
this iteration ends with $\tau_{l_1} < \frac{\epsi' M}{k}$
and $o_j$ is never considered to be added to $A_{l_1}$.
Thus, it holds that
\begin{equation}\label{inq:tgtwo-case2-5}
\marge{o_j}{A_{l_1}} < \frac{\epsi' M}{(1-\epsi')k}, \forall |A_{l_1}\setminus G_{i-1}|-c_{l_1}^* < j \le m-c_{l_1}^*.
\end{equation}

Then,
\begin{align*}
\marge{O_{l_2}}{A_{l_1}} &\le \marge{A_{l_2, c_{l_1}^*}}{A_{l_1}}  + \sum_{o_j \in O_{l_2}\setminus A_{l_2, c_{l_1}^*}}\marge{o_j}{A_{l_1}} \tag{Proposition~\ref{prop:sum-marge}}\\
&\le \marge{A_{l_2, c_{l_1}^*}}{G_{i-1}} + \sum_{j = 1}^{|A_{l_1}\setminus G_{i-1}|-c_{l_1}^*}\marge{o_j}{A_{l_1, c_{l_1}^*+j-1}} + \sum_{j=|A_{l_1}\setminus G_{i-1}|-c_{l_1}^*+1}^{m-c_{l_1}^*} \marge{o_j}{A_{l_1}} \tag{submodularity}\\
&\le \ff{A_{l_2, c_{l_1}^*}}-\ff{G_{i-1}} + \sum_{j = 1}^{|A_{l_1}\setminus G_{i-1}|-c_{l_1}^*}\frac{\marge{a_{l_1, c_{l_1}^*+j}}{A_{l_1, , c_{l_1}^*+j-1}}}{1-\epsi'} + \frac{\epsi' M}{(1-\epsi')\ell} \tag{Inequality~\eqref{inq:tgtwo-case2-4} and~\eqref{inq:tgtwo-case2-5}}\\
& \le \ff{A_{l_2, c_{l_1}^*}}-\ff{G_{i-1}} + \frac{\ff{A_{l_1}} - \ff{A_{l_1, c_{l_1}^*}}}{1-\epsi'} + \frac{\epsi' M}{(1-\epsi')\ell} \numberthis \label{inq:tgtwo-case2-7}
\end{align*}

By Inequalities~\eqref{inq:tgtwo-case2-6} and~\eqref{inq:tgtwo-case2-7},
\begin{align*}
\marge{O_{l_1}}{A_{l_2}}+\marge{O_{l_2}}{A_{l_1}}
\le \frac{1}{1-\epsi'}\left(\marge{A_{l_1}}{G_{i-1}}+\marge{A_{l_2}}{G_{i-1}}\right)
+\frac{2\epsi' M}{(1-\epsi')\ell}
\end{align*}

Thus, the lemma holds in this case.

\textbf{Case 2: $c_{l_1}^* > c_{l_2}^*$; right half part in Fig.~\ref{fig:gdtwo}.}

First, we bound $\marge{O_{l_1}}{A_{l_2}}$.
Order the elements in $O_{l_1}\setminus A_{l_1, c_{l_2}^*+1}$ as $\{o_1, o_2, \ldots\}$ such that $o_j \not \in A_{l_1, c_{l_1}^*+j}$.
(Refer to the gray block with a dotted edge in the top right corner of Fig.~\ref{fig:gdtwo} for $O_{l_1}$.
If $c_{l_1}^*+j$ is greater than the number of elements added to $A_{l_1}$,
$A_{l_1, c_{l_1}^*+j}$ refers to $A_{l_1}$.)
Note that, since $A_{l_1, c_{l_2}^*+1} \subseteq O_{l_1}$,
it follows that $|O_{l_1}\setminus A_{l_1, c_{l_2}^*+1}| \le m - c_{l_2}^*-1$.

When $1 \le j \le |A_{l_2}\setminus G_{i-1}| - c_{l_2}^* - 1$,
since each $o_j$ is either added to $A_{l_1}$ or not in any solution set by Claim~\ref{claim:par-A}
and $\tau_{l_2}$ is initialized with the maximum marginal gain $M$,
$o_j$ is not considered to be added to $A_{l_2}$ with threshold value $\tau_{l_2}^{c_{l_2}^* + j}/(1-\epsi')$.
Therefore, it holds that 
\begin{equation}\label{inq:tgtwo-case3-1}
\marge{o_j}{A_{l_2, c_{l_2}^*+j-1}} < \frac{\tau_{l_2}^{c_{l_2}^* + j}}{1-\epsi'} \le \frac{\marge{a_{l_2, c_{l_2}^* + j}}{A_{l_2, c_{l_2}^* + j-1}}}{1-\epsi'}, \forall 1\le j\le |A_{l_2}\setminus G_{i-1}| - c_{l_2}^* - 1.
\end{equation}

When $|A_{l_2}\setminus G_{i-1}| < m$ and $|A_{l_2}\setminus G_{i-1}|- c_{l_2}^* - 1 < j\le m- c_{l_2}^* - 1$,
this iteration ends with $\tau_{l_2} < \frac{\epsi' M}{k}$ and
$o_j$ is never considered to be added to $A_{l_2}$.
Thus, it holds that
\begin{equation}\label{inq:tgtwo-case3-3}
\marge{o_j}{A_{l_2}} < \frac{\epsi' M}{(1-\epsi')k}, 
\forall |A_{l_2}\setminus G_{i-1}|- c_{l_2}^* - 1 < j \le m- c_{l_2}^* - 1.
\end{equation}

Then,
\begin{align*}
\marge{O_{l_1}}{A_{l_2}} &\le \marge{A_{l_1, c_{l_2}^*}}{A_{l_2}}  + \sum_{o_j \in O_{l_1}\setminus A_{l_1, c_{l_2}^*+1}}\marge{o_j}{A_{l_2}} \tag{Proposition~\ref{prop:sum-marge}}\\
&\le \marge{A_{l_1, c_{l_2}^*}}{G_{i-1}} + \sum_{j = 1}^{|A_{l_2}\setminus G_{i-1}|- c_{l_2}^* - 1}\marge{o_j}{A_{l_2, , c_{l_2}^*+j-1}} + \sum_{j=|A_{l_2}\setminus G_{i-1}|- c_{l_2}^*}^{m- c_{l_2}^* - 1} \marge{o_j}{A_{l_2}} \tag{submodularity}\\
&\le \ff{A_{l_1, c_{l_2}^*}}-\ff{G_{i-1}} + \sum_{j = 1}^{|A_{l_2}\setminus G_{i-1}|- c_{l_2}^* - 1}\frac{\marge{a_{l_2, c_{l_2}^*+j}}{A_{l_2, , c_{l_2}^*+j-1}}}{1-\epsi'} + \frac{\epsi' M}{(1-\epsi')\ell} \tag{Inequality~\eqref{inq:tgtwo-case3-1} and~\eqref{inq:tgtwo-case3-3}}\\
& \le \ff{A_{l_1, c_{l_2}^*}}-\ff{G_{i-1}} + \frac{\ff{A_{l_2}} - \ff{A_{l_2, c_{l_2}^*}}}{1-\epsi'} + \frac{\epsi' M}{(1-\epsi')\ell} \numberthis \label{inq:tgtwo-case3-6}
\end{align*}

Similarly, we bound $\marge{O_{l_2}}{A_{l_1}}$ below.
Order the elements in $O_{l_2}\setminus A_{l_2, c_{l_2}^*}$ as $\{o_1, o_2, \ldots\}$ such that $o_j \not \in A_{l_2, c_{l_2}^*+j}$.
(See the gray block with a dotted edge in the bottom right corner of Fig.~\ref{fig:gdtwo} for $O_{l_2}$.
If $c_{l_2}^*+j$ is greater than the number of elements added to $A_{l_2}$,
$A_{l_2, c_{l_2}^*+j}$ refers to $A_{l_2}$.)
Note that, since $A_{l_2, c_{l_2}^*} \subseteq O_{l_2}$,
it follows that $|O_{l_2}\setminus A_{l_2, c_{l_2}^*}| \le m - c_{l_2}^*$.

When $1 \le j \le |A_{l_1}\setminus G_{i-1}|- c_{l_2}^* - 1$, 
since each $o_j$ is either added to $A_{l_2}$ or not in any solution set by Claim~\ref{claim:par-A}
and $\tau_{l_1}$ is initialized with the maximum marginal gain $M$,
$o_j$ is not considered to be added to $A_{l_1}$ with threshold value $\tau_{l_1}^{c_{l_2}^* + j+1}/(1-\epsi')$.
Therefore, it holds that 
\begin{equation}\label{inq:tgtwo-case3-4}
\marge{o_j}{A_{l_1, c_{l_2}^*+j}} < \frac{\tau_{l_1}^{c_{l_2}^* + j+1}}{1-\epsi'} \le \frac{\marge{a_{l_1, c_{l_2}^* + j+1}}{A_{l_1, c_{l_2}^* + j}}}{1-\epsi'}, \forall 1\le j\le |A_{l_2}\setminus G_{i-1}|- c_{l_2}^* - 1.
\end{equation}
If $|A_{l_1}\setminus G_{i-1}| = m$,
consider the last element $o_{m-c_{l_2}^*}$ in $O_{l_2}\setminus A_{l_2, c_{l_2}^*}$.
Since $o_{m-c_{l_2}^*} \not\in A_{l_2}$ and $o_{m-c_{l_2}^*} \not\in A_{l_1}$, $o_{m-c_{l_2}^*}$ is not considered to be added to 
$A_{l_1}$ with threshold value $\tau_{l_1}^j/(1-\epsi')$ for any $j \in [m]$.
Then,
\begin{equation}\label{inq:tgtwo-case3-2}
\marge{o_{m-c_{l_2}^*}}{A_{l_1}} < \frac{\sum_{j=1}^m \tau_{l_1}^j}{(1-\epsi')m}
\le \frac{\sum_{j=1}^m \marge{a_{l_1, j}}{A_{l_1, j-1}}}{(1-\epsi')m}
 = \frac{\marge{A_{l_1}}{G_{i-1}}}{(1-\epsi')m}.
\end{equation}

When $|A_{l_1}\setminus G_{i-1}| < m$ and $|A_{l_1}\setminus G_{i-1}|- c_{l_2}^* - 1 < j\le m- c_{l_2}^*$,
this iteration ends with $\tau_{l_1} < \frac{\epsi' M}{k}$
and $o_j$ is never considered to be added to $A_{l_1}$.
Thus, it holds that
\begin{equation}\label{inq:tgtwo-case3-5}
\marge{o_j}{A_{l_1}} < \frac{\epsi' M}{(1-\epsi')k}, 
\forall |A_{l_1}\setminus G_{i-1}|- c_{l_2}^* - 1 < j \le m- c_{l_2}^*.
\end{equation}

Then,
\begin{align*}
\marge{O_{l_2}}{A_{l_1}} &\le \marge{A_{l_2, c_{l_2}^*}}{A_{l_1}}  + \sum_{o_j \in O_{l_2}\setminus A_{l_2, c_{l_2}^*}}\marge{o_j}{A_{l_1}} \tag{Proposition~\ref{prop:sum-marge}}\\
&\le \marge{A_{l_2, c_{l_2}^*}}{G_{i-1}} + \sum_{j = 1}^{|A_{l_1}\setminus G_{i-1}|- c_{l_2}^* - 1}\marge{o_j}{A_{l_1, c_{l_2}^*+j-1}} + \sum_{j=|A_{l_1}\setminus G_{i-1}|- c_{l_2}^*}^{m} \marge{o_j}{A_{l_1}} \tag{submodularity}\\
&\le \ff{A_{l_2, c_{l_2}^*}}-\ff{G_{i-1}} + \sum_{j = 1}^{|A_{l_1}\setminus G_{i-1}|- c_{l_2}^* - 1}\frac{\marge{a_{l_1, c_{l_2}^*+j}}{A_{l_1, , c_{l_2}^*+j-1}}}{1-\epsi'}
+ \frac{\marge{A_{l_1}}{G_{i-1}}}{(1-\epsi')m}
+\frac{\epsi' M}{(1-\epsi')\ell} \tag{Inequalities~\eqref{inq:tgtwo-case3-4}-\eqref{inq:tgtwo-case3-5}}\\
& \le \ff{A_{l_2, c_{l_2}^*}}-\ff{G_{i-1}} + \frac{\ff{A_{l_1}} - \ff{A_{l_1, c_{l_2}^*}}}{1-\epsi'} + \frac{\marge{A_{l_1}}{G_{i-1}}}{(1-\epsi')m} + \frac{\epsi' M}{(1-\epsi')\ell} \numberthis \label{inq:tgtwo-case3-7}
\end{align*}

By Inequalities~\eqref{inq:tgtwo-case3-6} and~\eqref{inq:tgtwo-case3-7},
\begin{align*}
\marge{O_{l_1}}{A_{l_2}}+\marge{O_{l_2}}{A_{l_1}}
\le \frac{1}{1-\epsi'}\left(1+\frac{1}{m}\right)\left(\marge{A_{l_1}}{G_{i-1}}+\marge{A_{l_2}}{G_{i-1}}\right)
+\frac{2\epsi' M}{(1-\epsi')\ell}
\end{align*}

Thus, the lemma holds in this case.
\end{proof}

\begin{proof}[Proof of Lemma~\ref{lemma:tg-recur}]
Fix on $G_{i-1}$ at the beginning of this iteration,
Since $\{A_l\setminus G_{i-1}: l\in [\ell]\}$ are pairwise disjoint sets,
by Proposition~\ref{prop:sum-marge}, it holds that
\[\exc{\ff{O\cup G_i}}{G_{i-1}} = \frac{1}{\ell}\sum_{l\in [\ell]}\ff{O\cup A_l} \ge \left(1-\frac{1}{\ell}\right)\ff{O\cup G_{i-1}}.\]
Then, by unfixing $G_{i-1}$, the first inequality holds.

To prove the second inequality, also consider fix on $G_{i-1}$ at the beginning of iteration $i$.
Then,
\begin{align*}
\sum_{l\in [\ell]}\marge{O}{A_l} &\le \sum_{l_1\in [\ell]}\sum_{l_2\in [\ell]}\marge{O_{l_1}}{A_{l_2}}\tag{Proposition~\ref{prop:sum-marge}}\\
& = \sum_{l \in [\ell]}\marge{O_{l}}{A_{l}} + \sum_{1\le l_1< l_2 \le \ell} \left(\marge{O_{l_1}}{A_{l_2}} +\marge{O_{l_2}}{A_{l_1}}\right) \tag{Lemma~\ref{lemma:tg-par-A}}\\
& \le \sum_{l \in [\ell]}\left(\frac{\marge{A_{l}}{G_{i-1}}}{1-\epsi'}+\frac{\epsi' M}{(1-\epsi')\ell}\right)\\
&\hspace*{2em}+\sum_{1\le l_1< l_2 \le \ell} \left(\frac{1}{1-\epsi'}\left(1+\frac{1}{m}\right)\left(\marge{A_{l_1}}{G_{i-1}}
+\marge{A_{l_2}}{G_{i-1}}\right)
+\frac{2\epsi' M}{(1-\epsi')\ell}\right)\tag{Lemma~\ref{lemma:tg-par-A}}\\
&\le \frac{\ell}{1-\epsi'}\left(1+\frac{1}{m}\right)\sum_{l \in [\ell]}\marge{A_{l}}{G_{i-1}} + \frac{\epsi' \ell}{1-\epsi'}\ff{O}\tag{$M \le \ff{O}$}
\end{align*}
\begin{align*}
\Rightarrow \left(1+\frac{\ell}{1-\epsi'}\right)\left(1+\frac{1}{m}\right) \sum_{l\in [\ell]}\marge{A_l}{G_{i-1}} &\ge \sum_{l\in [\ell]}\ff{O\cup A_l} -\ell\ff{G_{i-1}} - \frac{\epsi' \ell}{1-\epsi'}\ff{O}\\
&\ge \left(\ell-1\right)\ff{O\cup G_{i-1}}-\ell\ff{G_{i-1}} - \frac{\epsi' \ell}{1-\epsi'}\ff{O}
\end{align*}
Thus,
\begin{align*}
&\exc{\ff{G_i} - \ff{G_{i-1}}}{G_{i-1}}  = \frac{1}{\ell}\sum_{l \in [\ell]}\marge{A_{l}}{G_{i-1}}\\
&\ge \frac{1}{1+\frac{\ell}{1-\epsi'}} \frac{m}{m+1}\left(\left(1-\frac{1}{\ell}\right)  \ff{O\cup G_{i-1}} - \ff{G_{i-1}} - \frac{\epsi'}{1-\epsi'}\ff{O}\right)\tag{Proposition~\ref{prop:sum-marge}}
\end{align*}
By unfixing $G_{i-1}$, the second inequality holds.
\end{proof}

\section{Analysis of Section~\ref{sec:ptg}} % (fold)
\label{apx:ptg}
In this section, we provide the analysis of our parallel algorithms
introduced in Section~\ref{sec:ptg}.
First, we provide the subroutines used for \ptgoneshort in Appendix~\ref{apx:subroutine}.
Then, we analyze \ptgoneshort, the main parallel procedure,
in Appendix~\ref{apx:ptgone}.
At last, we provide that analysis of $1/4$ and $1/e$ approximation algorithms
in Appendix~\ref{apx:ptgone-guarantee} and Appendix~\ref{apx:ptgtwo},
respectively.
\subsection{Subroutines}\label{apx:subroutine}
\begin{algorithm}[ht]
\Fn{\dist($\{V_l\}_{l\in [\ell]}$)}{
	\KwIn{$V_1, V_2, \ldots, V_\ell \subseteq \uni$}
	\Init{$\mathcal V_1, \mathcal V_2, \ldots, \mathcal V_\ell \gets \emptyset$, $I\gets [\ell]$}
	\For{$i\gets 1$ to $\ell$}{
		$j\gets\argmin_{j\in I}|V_j|$\label{line:dis-index}\;
		$\mathcal V_j \gets$ randomly select $\left\lfloor\frac{|V_j|}{\ell}\right\rfloor$ elements in $V_j\setminus \left(\bigcup_{l \in [\ell]}\mathcal V_j\right)$ \label{line:dis-select}\;
		$I\gets I-j$\;
	}
	\Return{$\left\{\mathcal V_l\right\}_{l\in [\ell]}$}
	}
\caption{Return $\ell$ pairwise disjoint subsets where $|\mathcal V_j| \ge \frac{|V_j|}{2\ell}$ for any $j \in [\ell]$ if $|V_j|\ge 2\ell$}
\label{alg:dist}
\end{algorithm}
\begin{lemma}\label{lemma:dist}
With input $\{V_l\}_{l\in [\ell]}$, where $|V_l|\ge 2\ell$ for each $l\in [\ell]$,
\dist returns $\ell$ pairwise disjoint sets $\{\mathcal V_l\}_{l\in [\ell]}$ \st
$\mathcal V_l\subseteq V_l$ and $|\mathcal V_j| \ge \frac{|V_j|}{2\ell}$.
\end{lemma}


\begin{algorithm}[ht]
\Fn{\prefix($f, \mathcal V, s, \tau, \epsi$)}{
	\KwIn{evaluation oracle $f:2^{\uni} \to \reals$, maximum size $s$, threshold $\tau$, error $\epsi$, candidate pool $\mathcal V$ where $\marge{x}{\emptyset} \ge \tau $ for any $ x\in \mathcal V$}
	\Init{$B[1:s]\gets [\textbf{none}, \ldots, \textbf{none}]$}
	$\mathcal V \gets\left\{v_1, v_2, \ldots\right\} \gets \textbf{random-permutation}(\mathcal V)$\label{line:prefix-permute}\;
	\For{$i\gets 1$ to $s$ in parallel}{
		$T_{i-1} \gets \left\{v_1, \ldots, v_{i-1}\right\}$\;
		\lIf{$\marge{v_i}{T_{i-1}} \ge \tau$}{$B[i]\gets \textbf{true}$}\label{line:prefix-B-true}
		\lElseIf{$\marge{v_i}{T_{i-1}} < 0$}{$B[i]\gets \textbf{false}$}\label{line:prefix-B-false}
	}
	$i^*\gets \max \{i: \#\text{\textbf{true}s in }B[1:i] \ge (1-\epsi) i\}$\label{line:prefix-istar}\;
	\Return{$i^*$, $B$}
}
\caption{Select a prefix of $\mathcal V$ \st its average marginal gain is greater than $(1-\epsi)\tau$, and with a probability of $1/2$, more than an $\epsi/2$-fraction of $\mathcal V$ has a marginal gain less than $\tau$ relative to the prefix.}
\label{alg:prefix}
\end{algorithm}
Since the procedure \prefix is identical to Lines 8-15 in \ts~\citep{Chen2024},
the following two lemmata hold in a manner similar to Lemma 4 and 5 in \citet{Chen2024}.
\begin{lemma}\label{lemma:prefix-filter}
In \prefix, given $\mathcal V$ after \textbf{random-permutation} in Line~\ref{line:prefix-permute},
let $D_i = \left\{x\in \mathcal V: \marge{x}{T_i} < \tau\right\}$.
It holds that $|D_0|=0$, $|D_{|\mathcal V|}| = |\mathcal V|$, and $|D_{i-1}|\le |D_i|$.
\end{lemma}
\begin{lemma}\label{lemma:prefix-prob}
In \prefix, following the definition of $D_i$ in Lemma~\ref{lemma:prefix-filter},
let $t = \min\{i: |D_i| \ge \epsi |\mathcal V|/2\}$.
It holds that $\prob{i^* < \min\{s,t\}} \le 1/2$.
\end{lemma}
As defined in Lemma~\ref{lemma:prefix-filter},
$D_i$ contains the elements in $\mathcal V$
which can be filtered out by threshold value $\tau$
regarding the prefix $T_i$.
Therefore, Lemma~\ref{lemma:prefix-prob} indicates that,
with a probability of at least $1/2$,
$i^* = s$ or
more than $\epsi/2$-fraction of $\mathcal V$
can be filtered out if prefix $T_{i^*}$ is added to the solution.

\begin{algorithm}[ht]
\Fn{\update($f, V, \tau, \epsi$)}{
\KwIn{evaluation oracle $f:2^{\uni} \to \reals$, candidate set $V$, threshold value $\tau$, error $\epsi$}
	\For(\tcp*[f]{Update candidate sets with threshold values}){$j\gets 1$ to $\ell$ in parallel}{
		$V \gets \left\{x\in V : \marge{x}{\emptyset} \ge \tau\right\}$ \label{line:update-filter}\;
		\While{$|V| = 0$}{
			$\tau \gets (1-\epsi)\tau$\;
			$V \gets \left\{x\in \uni : \marge{x}{\emptyset} \ge \tau\right\}$\;
		}
	}
	\Return{$V, \tau$}
}
\caption{Update candidate set $V$ with threshold value $\tau$}
\label{alg:update}
\end{algorithm}
\subsection{Analysis of Alg.~\ref{alg:ptgone}}\label{apx:ptgone}
We provide the guarantees achieved by \ptgoneshort as follows,
\begin{lemma}\label{lemma:ptgone}
With input $(f, m, \ell, \tau_{\min}, \epsi)$, \ptgone (Alg.~\ref{alg:ptgone})
runs in $\oh{\ell^2\epsi^{-2}\log(n)\log\left(\frac{M}{\tau_{\min}}\right)}$ adaptive rounds and $\oh{\ell^3 \epsi^{-2}n\log(n)\log\left(\frac{M}{\tau_{\min}}\right)}$ queries with a probability of $1-1/n$,
and terminates with $\{(A_l, A_l'): l\in [\ell]\}$ \st
{\small
\begin{enumerate}
\item $A_l'\subseteq A_l$, $\marge{A_l'}{\emptyset} \ge \marge{A_l}{\emptyset}, \forall 1\le l \le \ell$, and $\{A_l: l\in [\ell]\}$ are pairwise disjoint sets,
\item $\marge{O_{l}}{A_{l}}\le \frac{\marge{A_{l}'}{\emptyset}}{(1-\epsi)^2}+\frac{m\cdot\tau_{\min}}{1-\epsi}, \forall 1\le l \le \ell$,
\item $\marge{O_{l_2}}{A_{l_1}} + \marge{O_{l_1}}{A_{l_2}} \le 
\frac{1+\frac{1}{m}}{(1-\epsi)^2}\left(\marge{A_{l_1}'}{\emptyset}+\marge{A_{l_2}'}{\emptyset}\right) + \frac{2m\cdot \tau_{\min}}{1-\epsi}, \text{ if } O_{l_1} = O_{l_2}, \forall 1\le l_1 < l_2 \le \ell$,
\end{enumerate}}
where $O_l\subseteq \uni$, $|O_l| \le m$, and $O_l \cap A_j = \emptyset$ for each $j \neq l$.

Especially, when $\ell = 2$,
{\small
\begin{itemize}
	\item[4.] $\marge{S}{A_{1}} + \marge{S}{A_{2}} \le \frac{1}{(1-\epsi)^2}\left(\marge{A_{l_1}'}{\emptyset}+\marge{A_{l_2}'}{\emptyset}\right) + \frac{2m\cdot \tau_{\min}}{1-\epsi}, \forall S\subseteq \uni, |S| \le m$. 
\end{itemize}}
\end{lemma}
Before proving Lemma~\ref{lemma:ptgone}, we provide the following lemma regarding each iteration of \ptgone.
\begin{lemma}\label{lemma:tgone-iteration}
For any iteration of the while loop in \ptgone (Alg.~\ref{alg:ptgone}),
let $A_{l, 0}$, $A_{l, 0}'$, $V_{l, 0}$, $\tau_{l, 0}$ be the set and threshold value at the beginning,
and $A_l$, $A_l'$, $V_l$, $\tau_l$ be those at the end.
The following properties hold. 
\begin{enumerate}
\item With a probability of at least $1/2$,
there exists $l \in [\ell]$ \st $\tau_l < \tau_{l, 0}$
or $m_0 = 0$ or
$|V_l| \le \left(1-\frac{\epsi}{4\ell}\right)|V_{l, 0}|$.
\item $\{A_l: l\in [\ell]\}$ have the same size and are pairwise disjoint.  
% \item $V_l = \left\{x\in V\setminus\left(\bigcup_{i\in [\ell]} A_l\right) : \marge{x}{A_l} \ge \tau_l \right\}$  for all $l\in [\ell]$.
% \item For each $l \in [\ell]$ \st $\tau_l < \tau_{l,0}$,
% it holds that $\marge{x}{A_l} < \frac{\tau_l}{1-\epsi}$ for all $x\in V\setminus \left(\bigcup_{i\in [\ell]} A_l\right)$.
\item For each $x\in A_l\setminus A_{l,0}$, let $\tau_l^{(x)}$ be the threshold value when $x$ is added to the solution,
$A_{l, (x)}$ be the largest prefix of $A_l$ that do not include $x$,
and for any $j\in [\ell]$ and $j\neq l$,
$A_{j, (x)}$ be the prefix of $A_j$ with $|A_{l, (x)}|$ elements if $j < l$,
or with $|A_{l, (x)}|-1$ elements if $j > l$.
Then, for any $l\in [\ell]$, $x\in A_l\setminus A_{l,0}$,
and $y\in \uni\setminus \left(\bigcup_{j\in [\ell]} A_{j, (x)}\right)$,
it holds that $\marge{y}{A_{l, (x)}} < \frac{\tau_l^{(x)}}{1-\epsi}$.
\item $A_l'\subseteq A_l$, $\marge{A_l'}{A_{l, 0}'} \ge \marge{A_l}{A_{l, 0}}$,
and $\marge{A_l'}{A_{l, 0}'}\ge (1-\epsi)\sum_{x \in A_l\setminus A_{l, 0}}\tau_l^{(x)}$ for all $l\in [\ell]$.
\end{enumerate}
\end{lemma}
\begin{proof}[Proof of Lemma~\ref{lemma:tgone-iteration}]
\textbf{Proof of Property 1.}
At the beginning of the iteration, if there exists $l\in I$ \st $|V_{l, 0}| < 2\ell$,
then either $\tau_{l, 0}$ is decreased to $\tau_l$ and $V_l$ is updated accordingly, 
or an element $x_l$ from $V_{l, 0}$ is added to $A_j$ and $A_j'$
and subsequently removed from $V_{l, 0}$. 
This implies that
\[|V_l| \le |V_{l,0}| -1 < \left(1-\frac{1}{2\ell}\right)|V_{l,0}|.\]
Property 1 holds in this case.

Otherwise, for all $l\in I$, it holds that $|V_{l, 0}| \ge 2\ell$,
and the algorithm proceeds to execute Lines~\ref{line:tgone-dist}-\ref{line:tgone-update-size}.
By Lemma~\ref{lemma:dist}, in Line~\ref{line:tgone-dist}, 
$|\mathcal V_l| \ge \frac{|V_{l, 0}|}{2\ell}$ for each $l\in I$ .
Consider the index $j\in I$ where $i_j^* = i^*$.
Then, $O_j$ consists of the first $i^*$ elements in $\mathcal V_j$
by Line~\ref{line:tgone-subset}.
By Lemma~\ref{lemma:prefix-prob},
with probability greater than $1/2$,
either $i^* = m_0$ or at least an $\frac{\epsi}{2}$-fraction
of elements $x\in \mathcal V_j$ satisfy $\marge{x}{A_j}< \tau_{j,0}$.
Consequently, either $m_0 = 0$ after Line~\ref{line:tgone-update-size},
or, after the \update procedure in Line~\ref{line:tgone-update},
one of the following holds:
$|V_l| \le \left(1-\frac{\epsi}{4\ell}\right)|V_{l, 0}|$,
or $\tau_{j} < \tau_{j, 0}$.
Therefore, Property 1 holds in this case. 

\textbf{Proof of Property 2.}
At any iteration of the while,
either $|I|$ different elements or $|I|$ pairwise disjoint sets with same size $i^*$
are added to solution sets $\{A_l: l\in I\}$.
Therefore, Property 2 holds.

\textbf{Proof of Property 3.}
At any iteration, if $\tau_l$ is not updated on Line~\ref{line:tgone-update-2},
then prior to this iteration, all the elements outside of the solutions
have marginal gain less than $\frac{\tau_l^{(x)}}{1-\epsi}$.
Thus, for any $x \in A_l\setminus A_{l, 0}$, $y\in \uni\setminus \left(\bigcup_{j\in [\ell]} A_{j, 0}\right)$,
it holds that $\marge{y}{A_{l, (x)}} < \frac{\tau_l^{(x)}}{1-\epsi}$
by submodularity. Property 3 holds in this case.

Otherwise, if $\tau_l$ is updated on Line~\ref{line:tgone-update-2},
only one element is added to each solution set during this iteration.
Let $x = A_l\setminus A_{l, 0}$.
For any $j\in [\ell]$ and $j\neq l$,
it holds that $A_{j, (x)} = A_j$ if $j < l$,
or $A_{j, (x)} = A_{j, 0}$ if $j > l$.
Since elements are added to each pair of solutions in sequence within the for loop
in Lines~\ref{line:tgone-for-begin}-\ref{line:tgone-for-end},
by the \update procedure,
for any $y \in \uni \setminus \left(\bigcup_{j\in [\ell]} A_{j, (x)}\right)$,
it holds that $\marge{y}{A_{l, (x)}} < \frac{\tau_l^{(x)}}{1-\epsi}$.
Therefore, Property 3 also holds in this case.

\textbf{Proof of Property 4.}
First, we prove $A_l'\subseteq A_l$ by induction.
At the beginning of the algorithm,
$A_l'$ and $A_l$ are initialized as empty sets.
Clearly, the property holds in the base case.
Then, suppose that $A_{l,0}'\subseteq A_{l,0}$.
There are three possible cases of updating $A_{l,0}'$ and $A_{l,0}$ at any iteration:
1) $A_l'=A_{l,0}'$ and $A_l = A_{l,0}$,
2) $A_l' = A_{l,0}' + x_l$ and $A_l = A_{l,0} + x_l$ in Line~\ref{line:tgone-update-A},
or 3) $A_l' = A_{l,0}' \cup S_l'$ and $A_l = A_{l,0} \cup S_l$ in Line~\ref{line:tgone-update-A-2}.
Clearly, $A_l'\subseteq A_l$ holds in all cases.

Next, we prove the rest of Property 4.

If $A_l'=A_{l,0}'$ and $A_l = A_{l,0}$, 
then $\marge{A_l'}{A_{l, 0}'} = \marge{A_l}{A_{l, 0}}=0$.
Property 4 holds.

If $A_{l,0}'$ and $A_{l,0}$ are updated in Line~\ref{line:tgone-update-A},
by submodularity, $\marge{A_l'}{A_{l, 0}'} = \marge{x_l}{A_{l, 0}'} \ge \marge{x_l}{A_{l, 0}}=\marge{A_l}{A_{l, 0}} \ge \tau_l^{(x_{l})}$.
Therefore, Property 4 also holds.

If $A_{l,0}'$ and $A_{l,0}$ are updated in Line~\ref{line:tgone-update-A-2},
we know that $A_l' = A_{l,0}' \cup S_l'$ and $A_l = A_{l,0} \cup S_l$.
Suppose the elements in $S_l$ and $S_l'$ retain their original order within $\mathcal V_l$.
For each $x\in S_l$, let $S_{l,(x)}$, $\mathcal V_{l, (x)}$ and $A_{l, (x)}$
be the largest prefixes of $S_l$, $\mathcal V_l$ and $A_l$ that do not include $x$, respectively.
Moreover, let $S_{l, (x)}' = S_{l, (x)}\cap S_l'$ and $A_{l, (x)}' = A_{l, (x)}\cap A_l'$.
Say an element $x\in S_l$ \textbf{true} if $B_l[(x)] = \textbf{true}$,
where $B_l[(x)]$ is the $i$-th element in $B_l$ if $x$ is the $i$-th element in $\mathcal V_l$.
Similarly, say an element $x\in S_l$ \textbf{false} if $B_l[(x)] = \textbf{false}$,
and \textbf{none} otherwise.

Following the above definitions, for any \textbf{true} or \textbf{none} element $x\in S_l$,
by Line~\ref{line:tgone-subset}, it holds that $S_{l, (x)} \subseteq \mathcal V_{l, (x)}$.
Then, by Line~\ref{line:prefix-B-true} and submodularity,
\begin{equation*}
\marge{x}{A_{l, (x)}} = \marge{x}{A_{l, 0} \cup S_{l, (x)}}
\ge \marge{x}{A_{l, 0} \cup \mathcal V_{l, (x)}} \ge \left\{
\begin{aligned}
&\tau_l^{(x)}, \text{ if } x \text{ is \textbf{true} element}\\
&0, \text{ if } x \text{ is \textbf{none} element}
\end{aligned}\right.
\end{equation*}
Since \textbf{true} elements are selected at first and $i_j^*\ge i^*$,
there are more than $(1-\epsi)i^*$ \textbf{true} elements in $S_l$.
Therefore,
\begin{align*}
\marge{A_l'}{A_{l, 0}'} &= \sum_{x\in A_l'\setminus A_{l, 0}', x\text{ is \textbf{true} element}} \marge{x}{A_{l, (x)}'}
+\sum_{x\in A_l'\setminus A_{l, 0}', x\text{ is \textbf{none} element}} \marge{x}{A_{l, (x)}'}\\
&\ge \sum_{x\in A_l'\setminus A_{l, 0}', x\text{ is \textbf{true} element}} \marge{x}{A_{l, (x)}}
+\sum_{x\in A_l'\setminus A_{l, 0}', x\text{ is \textbf{none} element}} \marge{x}{A_{l, (x)}}\\
&\ge (1-\epsi) |A_l\setminus A_{l, 0}| \tau_l^{(x)}, \text{for any } x\in A_l\setminus A_{l, 0}\\
&= (1-\epsi)\sum_{x\in A_l\setminus A_{l, 0}} \tau_l^{(x)}.
\end{align*}
The third part of Property 4 holds.

To prove the second part of Property 4, consider any \textbf{false} element $x\in S_l$.
By Line~\ref{line:tgone-subset}, it holds that $\mathcal V_{l, (x)} = S_{l, (x)}$.
Then, by Line~\ref{line:prefix-B-false}
\begin{equation}\label{ineq:tgone-false}
\marge{x}{A_{l, (x)}} = \marge{x}{A_{l, 0} \cup S_{l, (x)}}
= \marge{x}{A_{l, 0} \cup \mathcal V_{l, (x)}} < 0.
\end{equation}
By Line~\ref{line:tgone-subset-2}, all the elements in $S_l\setminus S_l'$ are
\textbf{false} elements.
Then,
\begin{align*}
\ff{A_l} - \ff{A_{l, 0}} &= \sum_{x\in S_l'}\marge{x}{A_{l, (x)}} + \sum_{x\in S_l\setminus S_l'}\marge{x}{A_{l, (x)}}\\
&< \sum_{x\in S_l'}\marge{x}{A_{l, (x)}} \tag{Inequality~\ref{ineq:tgone-false}}\\
&\le \sum_{x\in S_l'}\marge{x}{A_{l, (x)}'} \tag{Submodularity}\\
& = \ff{A_l'} - \ff{A_{l, 0}'}.
\end{align*}
\end{proof}

By Lemma~\ref{lemma:tgone-iteration},
we are ready to prove Lemma~\ref{lemma:ptgone}.
\begin{proof}[Proof of Lemma~\ref{lemma:ptgone}]
\textbf{Proof of Property 1.}
By Property 2 and 3 in Lemma~\ref{lemma:tgone-iteration},
this property holds immediately.

\textbf{Proof of Property 2.}
For any $l\in [\ell]$,
since $O_l\cap A_j  = \emptyset$ for each $j\neq l$,
$O_l\setminus A_l$ is outside of any solution set.
If $|A_l| = m$, by Property 4 of Lemma~\ref{lemma:tgone-iteration},
\begin{align*}
\marge{O_l}{A_l} &\le \sum_{y \in O_l\setminus A_l}\marge{y}{A_l}\\
&\le \sum_{x \in A_l}\tau_l^{(x)}/(1-\epsi)\tag{Property 3 in Lemma~\ref{lemma:tgone-iteration}}\\
& \le \frac{\marge{A_l'}{\emptyset}}{(1-\epsi)^2}.\tag{Property 5 in Lemma~\ref{lemma:tgone-iteration}}
\end{align*}
If $|A_l| < m$, then the threshold value for solution $A_l$ has been updated to be less than $\tau_{\min}$.
Therefore, for any $y\in O_l\setminus A_l$,
it holds that $\marge{y}{A_l} < \frac{\tau_{\min}}{1-\epsi}$.
Then,
\begin{align*}
\marge{O_l}{A_l} \le \sum_{y \in O_l\setminus A_l}\marge{y}{A_l}
\le \frac{m\tau_{\min}}{1-\epsi}.
\end{align*}
Therefore, Property 2 holds by summing the above two inequalities.

\textbf{Proof of Property 3 and 4.}
Let $a_{l, j}$ be the $j$-th element added to $A_l$,
$\tau_l^j$ be the threshold value of $\tau_l$ when $a_{l, j}$ is added to $A_l$,
and $A_{l, j}$ be $A_l$ after $a_{l, j}$ is added to $A_l$.
Let $c_l^* = \max\{c\in [m]:A_{l, c}\subseteq O_l\}$.

In the following, we analyze these properties together under two cases,
similar to the analysis of Alg.~\ref{alg:ptgtwo}.
For the case where $\ell = 2$, 
let $O_{1} = S\setminus A_2$, and $O_2 = S\setminus A_1$,
unifying the notations used in Property 3 and 4.
Note that, the only difference between the two analyses is that,
a small portion (no more than $\epsi$ fraction) of elements in the solution returned by Alg.~\ref{alg:ptgone}
do not have marginal gain greater than the threshold value.

\textbf{Case 1: $c_{l_1}^*\le c_{l_2}^*$; left half part in Fig.~\ref{fig:gdtwo}.}

First, we bound $\marge{O_{l_1}}{A_{l_2}}$.
Consider elements in $A_{l_1, c_{l_1}^*} \subseteq O_{l_1}$.
Let $A_{l_1, c_{l_1}^*} = \{o_1, \ldots, o_{c_{l_1}^*}\}$.
For each $1\le j \le c_{l_1}^*$, 
since $o_j$ is added to $A_{l_1}$ with threshold value $\tau_{l_1}^{j}$
and the threshold value starts from the maximum marginal gain $M$,
clearly, $o_j$ has been filtered out with threshold value $\tau_{l_1}^{j}/(1-\epsi)$.
Then, by submodularity,
\begin{equation}\label{inq:ptgone-case1-1}
\marge{A_{l_1, c_{l_1}^*} }{A_{l_2}} \le \marge{A_{l_1, c_{l_1}^*} }{\emptyset}
 = \sum_{j=1}^{c_{l_1}^*}\marge{o_j}{A_{l_1, j-1}}
 \le \sum_{j=1}^{c_{l_1}^*} \tau_{l_1}^{j}/(1-\epsi).
\end{equation}

Next, consider the elements in $O_{l_1}\setminus A_{l_1, c_{l_1}^*}$.
Order the elements in $O_{l_1}\setminus A_{l_1, c_{l_1}^*}$ as $\{o_1, o_2, \ldots\}$ such that $o_j \not \in A_{l_1, c_{l_1}^*+j}$.
(Refer to the gray block with a dotted edge in the top left corner of Fig.~\ref{fig:gdtwo} for $O_{l_1}$.
If $c_{l_1}^*+j$ is greater than $|A_{l_1}|$,
$A_{l_1, c_{l_1}^*+j}$ refers to $A_{l_1}$.)
Note that, since $A_{l_1, c_{l_1}^*} \subseteq O_{l_1}$,
it follows that $|O_{l_1}\setminus A_{l_1, c_{l_1}^*}| \le m - c_{l_1}^*$.

When $1 \le j \le |A_{l_2}| - c_{l_1}^*$,
since each $o_j$ is either added to $A_{l_1}$ or not in any solution set
and $\tau_{l_2}$ is initialized with the maximum marginal gain $M$,
$o_j$ is not considered to be added to $A_{l_2}$ with threshold value $\tau_{l_2}^{c_{l_1}^* + j}/(1-\epsi)$
by Property 3 of Lemma~\ref{lemma:tgone-iteration}.
Therefore, it holds that 
\begin{equation}\label{inq:ptgone-case1-2}
\marge{o_j}{A_{l_2, c_{l_1}^*+j-1}} < \frac{\tau_{l_2}^{c_{l_1}^* + j}}{1-\epsi} , \forall 1\le j\le |A_{l_2}|-c_{l_1}^*.
\end{equation}

When $|A_{l_2}| < m$ and $|A_{l_2}|-c_{l_1}^* < j\le m-c_{l_1}^*$,
the algorithm ends with $\tau_{l_2} < \tau_{\min}$ and
$o_j$ is never considered to be added to $A_{l_2}$.
Thus, it holds that
\begin{equation}\label{inq:ptgone-case1-3}
\marge{o_j}{A_{l_2}} < \frac{\tau_{\min}}{1-\epsi}, 
\forall |A_{l_2}|-c_{l_1}^* < j \le m-c_{l_1}^*.
\end{equation}

Then,
\begin{align*}
\marge{O_{l_1}}{A_{l_2}} &\le \marge{A_{l_1, c_{l_1}^*}}{A_{l_2}}  + \sum_{o_j \in O_{l_1}\setminus A_{l_1, c_{l_1}^*}}\marge{o_j}{A_{l_2}} \tag{Proposition~\ref{prop:sum-marge}}\\
&\le \marge{A_{l_1, c_{l_1}^*}}{\emptyset} + \sum_{j = 1}^{|A_{l_2}|-c_{l_1}^*}\marge{o_j}{A_{l_2, , c_{l_1}^*+j-1}} + \sum_{j=|A_{l_2}|-c_{l_1}^*+1}^{m-c_{l_1}^*} \marge{o_j}{A_{l_2}} \tag{submodularity}\\
&\le \sum_{j=1}^{c_{l_1}^*} \frac{\tau_{l_1}^{j}}{1-\epsi} + \sum_{j=c_{l_1}^*+1}^{|A_{l_2}|} \frac{\tau_{l_2}^{j}}{1-\epsi} + \frac{m \cdot \tau_{\min}}{1-\epsi} \numberthis \label{inq:ptgone-case1-4}
\end{align*}
where the last inequality follows from 
Inequalities~\eqref{inq:ptgone-case1-1}-\eqref{inq:ptgone-case1-3}.

Similarly, we bound $\marge{O_{l_2}}{A_{l_1}}$ below.
Consider elements in $A_{l_2, c_{l_1}^*} \subseteq O_{l_2}$.
Let $A_{l_2, c_{l_1}^*} = \{o_1, \ldots, o_{c_{l_1}^*}\}$.
For each $1\le j \le c_{l_1}^*$, 
since $o_j$ is added to $A_{l_2}$ with threshold value $\tau_{l_2}^{j}$
and the threshold value starts from the maximum marginal gain $M$,
clearly, $o_j$ has been filtered out with threshold value $\tau_{l_2}^{j}/(1-\epsi)$.
Then, by submodularity,
\begin{equation}\label{inq:ptgone-case1-5}
\marge{A_{l_2, c_{l_1}^*} }{A_{l_1}} \le \marge{A_{l_2, c_{l_1}^*} }{\emptyset}
 = \sum_{j=1}^{c_{l_1}^*}\marge{o_j}{A_{l_2, j-1}}
 \le \sum_{j=1}^{c_{l_1}^*} \tau_{l_2}^{j}/(1-\epsi).
\end{equation}

Next, consider the elements in $O_{l_2}\setminus A_{l_2, c_{l_1}^*}$.
Order the elements in $O_{l_2}\setminus A_{l_2, c_{l_1}^*}$ as $\{o_1, o_2, \ldots\}$ such that $o_j \not \in A_{l_2, c_{l_1}^*+j-1}$.
(See the gray block with a dotted edge in the bottom left corner of Fig.~\ref{fig:gdtwo} for $O_{l_2}$.
If $c_{l_1}^*+j-1$ is greater than $|A_{l_2}|$,
$A_{l_2, c_{l_1}^*+j-1}$ refers to $A_{l_2}$.)
Note that, since $A_{l_2, c_{l_1}^*} \subseteq O_{l_2}$,
it follows that $|O_{l_2}\setminus A_{l_2, c_{l_1}^*}| \le m - c_{l_1}^*$.

When $1 \le j \le |A_{l_1}|-c_{l_1}^*$,
since each $o_j$ is either added to $A_{l_2}$ or not in any solution set,
and $\tau_{l_1}$ is initialized with the maximum marginal gain $M$,
$o_j$ is not considered to be added to $A_{l_1}$ with threshold value $\tau_{l_1}^{c_{l_1}^* + j}/(1-\epsi)$
by Property 3 of Lemma~\ref{lemma:tgone-iteration}.
Therefore, it holds that 
\begin{equation}\label{inq:ptgone-case1-6}
\marge{o_j}{A_{l_1, c_{l_1}^*+j-1}} < \frac{\tau_{l_1}^{c_{l_1}^* + j}}{1-\epsi} , \forall 1\le j\le |A_{l_2}|-c_{l_1}^*.
\end{equation}

When $|A_{l_1}| < m$ and $|A_{l_1}|-c_{l_1}^* < j\le m-c_{l_1}^*$,
this iteration ends with $\tau_{l_1} < \tau_{\min}$
and $o_j$ is never considered to be added to $A_{l_1}$.
Thus, it holds that
\begin{equation}\label{inq:ptgone-case1-7}
\marge{o_j}{A_{l_1}} < \frac{\tau_{\min}}{1-\epsi}, \forall |A_{l_1}|-c_{l_1}^* < j \le m-c_{l_1}^*.
\end{equation}

Then,
\begin{align*}
\marge{O_{l_2}}{A_{l_1}} &\le \marge{A_{l_2, c_{l_1}^*}}{A_{l_1}}  + \sum_{o_j \in O_{l_2}\setminus A_{l_2, c_{l_1}^*}}\marge{o_j}{A_{l_1}} \tag{Proposition~\ref{prop:sum-marge}}\\
&\le \marge{A_{l_2, c_{l_1}^*}}{\emptyset} + \sum_{j = 1}^{|A_{l_1}|-c_{l_1}^*}\marge{o_j}{A_{l_1, c_{l_1}^*+j-1}} + \sum_{j=|A_{l_1}|-c_{l_1}^*+1}^{m-c_{l_1}^*} \marge{o_j}{A_{l_1}} \tag{submodularity}\\
&\le \sum_{j=1}^{c_{l_1}^*} \frac{\tau_{l_2}^{j}}{1-\epsi} + \sum_{j=c_{l_1}^*+1}^{|A_{l_1}|} \frac{\tau_{l_1}^{j}}{1-\epsi} + \frac{m\cdot \tau_{\min}}{1-\epsi}  \numberthis \label{inq:ptgone-case1-8}
\end{align*}
where the last inequality follows from Inequalities~\ref{inq:ptgone-case1-5}-\ref{inq:ptgone-case1-7}.

By Inequalities~\eqref{inq:ptgone-case1-4} and~\eqref{inq:ptgone-case1-8},
\begin{equation}\label{inq:ptgone-case1-final}
\marge{O_{l_1}}{A_{l_2}}+\marge{O_{l_2}}{A_{l_1}}
\le \sum_{j=1}^{|A_{l_1}|} \frac{\tau_{l_1}^{j}}{1-\epsi} + \sum_{j=1}^{|A_{l_2}|} \frac{\tau_{l_2}^{j}}{1-\epsi} + \frac{2m\cdot \tau_{\min}}{1-\epsi}
\end{equation}

\textbf{Case 2: $c_{l_1}^* > c_{l_2}^*$; right half part in Fig.~\ref{fig:gdtwo}.}

First, we bound $\marge{O_{l_1}}{A_{l_2}}$.
Consider elements in $A_{l_1, c_{l_2}^*+1} \subseteq O_{l_1}$.
Let $A_{l_1, c_{l_2}^*+1} = \{o_1, \ldots, o_{c_{l_2}^*+1}\}$.
For each $1\le j \le c_{l_2}^*+1$, 
since $o_j$ is added to $A_{l_1}$ with threshold value $\tau_{l_1}^{j}$
and the threshold value starts from the maximum marginal gain $M$,
clearly, $o_j$ has been filtered out with threshold value $\tau_{l_1}^{j}/(1-\epsi)$.
Then, by submodularity,
\begin{equation}\label{inq:ptgone-case2-1}
\marge{A_{l_1, c_{l_2}^*+1} }{A_{l_2}} \le \marge{A_{l_1, c_{l_2}^*+1} }{\emptyset}
 = \sum_{j=1}^{c_{l_2}^*+1}\marge{o_j}{A_{l_1, j-1}}
 \le \sum_{j=1}^{c_{l_2}^*+1} \tau_{l_1}^{j}/(1-\epsi).
\end{equation}

Next, consider the elements in $O_{l_1}\setminus A_{l_1, c_{l_2}^*+1}$.
Order the elements in $O_{l_1}\setminus A_{l_1, c_{l_2}^*+1}$ as $\{o_1, o_2, \ldots\}$ such that $o_j \not \in A_{l_1, c_{l_1}^*+j}$.
(Refer to the gray block with a dotted edge in the top right corner of Fig.~\ref{fig:gdtwo} for $O_{l_1}$.
If $c_{l_1}^*+j$ is greater than $|A_{l_1}|$,
$A_{l_1, c_{l_1}^*+j}$ refers to $A_{l_1}$.)
Note that, since $A_{l_1, c_{l_2}^*+1} \subseteq O_{l_1}$,
it follows that $|O_{l_1}\setminus A_{l_1, c_{l_2}^*+1}| \le m - c_{l_2}^*-1$.

When $1 \le j \le |A_{l_2}| - c_{l_2}^* - 1$,
since each $o_j$ is either added to $A_{l_1}$ or not in any solution set
and $\tau_{l_2}$ is initialized with the maximum marginal gain $M$,
$o_j$ is not considered to be added to $A_{l_2}$ with threshold value $\tau_{l_2}^{c_{l_2}^* + j}/(1-\epsi)$.
Therefore, it holds that 
\begin{equation}\label{inq:ptgone-case2-2}
\marge{o_j}{A_{l_2, c_{l_2}^*+j-1}} < \frac{\tau_{l_2}^{c_{l_2}^* + j}}{1-\epsi}, \forall 1\le j\le |A_{l_2}\setminus G_{i-1}| - c_{l_2}^* - 1.
\end{equation}

When $|A_{l_2}| < m$ and $|A_{l_2}|- c_{l_2}^* - 1 < j\le m- c_{l_2}^* - 1$,
this iteration ends with $\tau_{l_2} < \tau_{\min}$ and
$o_j$ is never considered to be added to $A_{l_2}$.
Thus, it holds that
\begin{equation}\label{inq:ptgone-case2-3}
\marge{o_j}{A_{l_2}} < \frac{\tau_{\min}}{1-\epsi}, 
\forall |A_{l_2}|- c_{l_2}^* - 1 < j \le m- c_{l_2}^* - 1.
\end{equation}

Then,
\begin{align*}
\marge{O_{l_1}}{A_{l_2}} &\le \marge{A_{l_1, c_{l_2}^*}}{A_{l_2}}  + \sum_{o_j \in O_{l_1}\setminus A_{l_1, c_{l_2}^*+1}}\marge{o_j}{A_{l_2}} \tag{Proposition~\ref{prop:sum-marge}}\\
&\le \marge{A_{l_1, c_{l_2}^*}}{\emptyset} + \sum_{j = 1}^{|A_{l_2}|- c_{l_2}^* - 1}\marge{o_j}{A_{l_2, c_{l_2}^*+j-1}} + \sum_{j=|A_{l_2}|- c_{l_2}^*}^{m- c_{l_2}^* - 1} \marge{o_j}{A_{l_2}} \tag{submodularity}\\
&\le \sum_{j=1}^{c_{l_2}^*+1} \tau_{l_1}^{j}/(1-\epsi)
+ \sum_{j = c_{l_2}^*+1}^{|A_{l_2}|} \tau_{l_2}^{j}/(1-\epsi) + \frac{m \cdot \tau_{\min}}{1-\epsi}
 \numberthis \label{inq:ptgone-case2-4}
\end{align*}
where the last inequality follows from 
Inequalities~\eqref{inq:ptgone-case2-1}-\eqref{inq:ptgone-case2-3}.

Similarly, we bound $\marge{O_{l_2}}{A_{l_1}}$ below.
Consider elements in $A_{l_1, c_{l_2}^*} \subseteq O_{l_2}$.
Let $A_{l_2, c_{l_2}^*} = \{o_1, \ldots, o_{c_{l_2}^*}\}$.
For each $1\le j \le c_{l_2}^*$, 
since $o_j$ is added to $A_{l_2}$ with threshold value $\tau_{l_2}^{j}$
and the threshold value starts from the maximum marginal gain $M$,
clearly, $o_j$ has been filtered out with threshold value $\tau_{l_2}^{j}/(1-\epsi)$.
Then, by submodularity,
\begin{equation}\label{inq:ptgone-case2-5}
\marge{A_{l_2, c_{l_2}^*} }{A_{l_1}} \le \marge{A_{l_2, c_{l_2}^*} }{\emptyset}
 = \sum_{j=1}^{c_{l_2}^*}\marge{o_j}{A_{l_2, j-1}}
 \le \sum_{j=1}^{c_{l_2}^*} \tau_{l_2}^{j}/(1-\epsi).
\end{equation}

Next, consider the elements in $O_{l_2}\setminus A_{l_2, c_{l_2}^*}$.
Order these elements as $\{o_1, o_2, \ldots\}$ such that $o_j \not \in A_{l_2, c_{l_2}^*+j}$.
(See the gray block with a dotted edge in the bottom right corner of Fig.~\ref{fig:gdtwo} for $O_{l_2}$.
If $c_{l_2}^*+j$ is greater than the number of elements added to $A_{l_2}$,
$A_{l_2, c_{l_2}^*+j}$ refers to $A_{l_2}$.)
Note that, since $A_{l_2, c_{l_2}^*} \subseteq O_{l_2}$,
it follows that $|O_{l_2}\setminus A_{l_2, c_{l_2}^*}| \le m - c_{l_2}^*$.

Furthermore, for the case where $\ell  = 2$, as considered in Property 4,
we have $O_1 = S\setminus A_2$ and $O_2 = S\setminus A_1$ for a given $S\subseteq \uni$
where $|S| \le m$.
Since $c_{l_1}^* > c_{l_2}^* \ge 0$,
it follows that $c_{l_1}^*\ge 1$, which implies $|O_2| = |S\setminus A_1|\le m-1$.
In this case, it holds that $|O_{l_2}\setminus A_{l_2, c_{l_2}^*}| \le m - c_{l_2}^*-1$.

When $1 \le j \le |A_{l_1}|- c_{l_2}^* - 1$, 
since each $o_j$ is either added to $A_{l_2}$ or not in any solution set by Claim~\ref{claim:par-A}
and $\tau_{l_1}$ is initialized with the maximum marginal gain $M$,
$o_j$ is not considered to be added to $A_{l_1}$ with threshold value $\tau_{l_1}^{c_{l_2}^* + j+1}/(1-\epsi)$.
Therefore, it holds that 
\begin{equation}\label{inq:ptgone-case2-6}
\marge{o_j}{A_{l_1, c_{l_2}^*+j}} < \frac{\tau_{l_1}^{c_{l_2}^* + j+1}}{1-\epsi}, \forall 1\le j\le |A_{l_2}|- c_{l_2}^* - 1.
\end{equation}

If $|A_{l_1}| = m$,
consider the last element $o_{m-c_{l_2}^*}$ in $O_{l_2}\setminus A_{l_2, c_{l_2}^*}$.
Since $o_{m-c_{l_2}^*} \not\in A_{l_2}$ and $o_{m-c_{l_2}^*} \not\in A_{l_1}$, $o_{m-c_{l_2}^*}$ is not considered to be added to 
$A_{l_1}$ with threshold value $\tau_{l_1}^j/(1-\epsi)$ for any $j \in [m]$.
Then,
\begin{equation}\label{inq:ptgone-case2-7}
\marge{o_{m-c_{l_2}^*}}{A_{l_1}} < \frac{\sum_{j=1}^m \tau_{l_1}^j}{(1-\epsi)m}.
\end{equation}
Else, $|A_{l_1}| < m$ and 
this iteration ends with $\tau_{l_1} < \frac{\epsi M}{k}$.
For any $|A_{l_1}|- c_{l_2}^* - 1 < j\le m- c_{l_2}^*$,
$o_j$ is never considered to be added to $A_{l_1}$.
Thus, it holds that
\begin{equation}\label{inq:ptgone-case2-8}
\marge{o_j}{A_{l_1}} < \frac{\tau_{\min}}{1-\epsi}, 
\forall |A_{l_1}|- c_{l_2}^* - 1 < j \le m- c_{l_2}^*.
\end{equation}

Then,
\begin{align*}
&\marge{O_{l_2}}{A_{l_1}} \le \marge{A_{l_2, c_{l_2}^*}}{A_{l_1}}  + \sum_{o_j \in O_{l_2}\setminus A_{l_2, c_{l_2}^*}}\marge{o_j}{A_{l_1}} \tag{Proposition~\ref{prop:sum-marge}}\\
&\le \left\{
\begin{aligned}
&\marge{A_{l_2, c_{l_2}^*}}{\emptyset} + \sum_{j = 1}^{|A_{l_1}\setminus G_{i-1}|- c_{l_2}^* - 1}\marge{o_j}{A_{l_1, c_{l_2}^*+j-1}} + \sum_{j=|A_{l_1}\setminus G_{i-1}|- c_{l_2}^*}^{m-c_{l_2}^*} \marge{o_j}{A_{l_1}}, &&\text{ if } |O_{l_2}| = m\\
&\marge{A_{l_2, c_{l_2}^*}}{\emptyset} + \sum_{j = 1}^{|A_{l_1}\setminus G_{i-1}|- c_{l_2}^* - 1}\marge{o_j}{A_{l_1, c_{l_2}^*+j-1}} + \sum_{j=|A_{l_1}\setminus G_{i-1}|- c_{l_2}^*}^{m-c_{l_2}^*-1} \marge{o_j}{A_{l_1}}, &&\text{otherwise}
\end{aligned}
\right. \tag{submodularity}\\
&\le\left\{
\begin{aligned}
	&\sum_{j=1}^{c_{l_2}^*} \frac{\tau_{l_2}^j}{1-\epsi} + \sum_{j=c_{l_2}^*+2}^{|A_{l_2}|} \left(1+\frac{1}{m}\right) \frac{\tau_{l_1}^j}{1-\epsi} + \frac{m\cdot \tau_{\min}}{1-\epsi}, &&\text{ if } |O_{l_2}| = m\\
	&\sum_{j=1}^{c_{l_2}^*} \frac{\tau_{l_2}^j}{1-\epsi} + \sum_{j=c_{l_2}^*+2}^{|A_{l_2}|} \frac{\tau_{l_1}^j}{1-\epsi} + \frac{m\cdot \tau_{\min}}{1-\epsi}, &&\text{otherwise}
\end{aligned}
\right. \numberthis \label{inq:ptgone-case2-9}
\end{align*}
where the last inequality follows from Inequalities~\eqref{inq:ptgone-case2-5}-\eqref{inq:ptgone-case2-8}.

By Inequalities~\eqref{inq:ptgone-case2-4} and~\eqref{inq:ptgone-case2-9},
\begin{equation}\label{inq:ptgone-case2-final}
\marge{O_{l_1}}{A_{l_2}}+\marge{O_{l_2}}{A_{l_1}}
\le \left\{
\begin{aligned}
	& \left(1+\frac{1}{m}\right) \frac{1}{1-\epsi}\left(\sum_{j=1}^{|A_{l_1}|} \tau_{l_1}^{j} + \sum_{j=1}^{|A_{l_2}|} \tau_{l_2}^{j}\right) + \frac{2m\cdot \tau_{\min}}{1-\epsi}, &&\text{ if } |O_{l_2}| = m \\
	&  \frac{1}{1-\epsi}\left(\sum_{j=1}^{|A_{l_1}|} \tau_{l_1}^{j} + \sum_{j=1}^{|A_{l_2}|} \tau_{l_2}^{j}\right) + \frac{2m\cdot \tau_{\min}}{1-\epsi}, &&\text{ otherwise }
\end{aligned}
\right.
\end{equation}

Overall, in both cases, if $|O_{l_2}| = m$,
\begin{align*}
\marge{O_{l_1}}{A_{l_2}}+\marge{O_{l_2}}{A_{l_1}}
&\le \left(1+\frac{1}{m}\right) \frac{1}{1-\epsi}\left(\sum_{j=1}^{|A_{l_1}|} \tau_{l_1}^{j} + \sum_{j=1}^{|A_{l_2}|} \tau_{l_2}^{j}\right) + \frac{2m\cdot \tau_{\min}}{1-\epsi} \tag{Inequalities~\eqref{inq:ptgone-case1-final} and~\eqref{inq:ptgone-case2-final}}\\
&\le \left(1+\frac{1}{m}\right)\frac{1}{(1-\epsi)^2}\left(\marge{A_{l_1}'}{\emptyset}+\marge{A_{l_2}'}{\emptyset}\right) + \frac{2m\cdot \tau_{\min}}{1-\epsi} \tag{Property 4 of Lemma~\ref{lemma:tgone-iteration}}
\end{align*}
Otherwise, if $|O_{l_2}| < m$,
\begin{align*}
\marge{O_{l_1}}{A_{l_2}}+\marge{O_{l_2}}{A_{l_1}}
&\le \frac{1}{1-\epsi}\left(\sum_{j=1}^{|A_{l_1}|} \tau_{l_1}^{j} + \sum_{j=1}^{|A_{l_2}|} \tau_{l_2}^{j}\right) + \frac{2m\cdot \tau_{\min}}{1-\epsi} \tag{Inequalities~\eqref{inq:ptgone-case1-final} and~\eqref{inq:ptgone-case2-final}}\\
&\le \frac{1}{(1-\epsi)^2}\left(\marge{A_{l_1}'}{\emptyset}+\marge{A_{l_2}'}{\emptyset}\right) + \frac{2m\cdot \tau_{\min}}{1-\epsi} \tag{Property 4 of Lemma~\ref{lemma:tgone-iteration}}
\end{align*}
Property (3) and (4) hold.

\textbf{Proof of Adaptivity and Query Complexity.}
Note that, at the beginning of every iteration,
for any $j\in I$, $V_j$ contains all the elements outside of all solutions that has marginal gain greater than $\tau_j$ with respect to solution $A_j$.
Say an iteration \textit{successful} if either
1) algorithm terminates after this iteration because of $m_0=0$,
2) all the elements in $V_j$ can be filtered out at the end of this iteration
and the value of $\tau_j$ decreases,
or 3) the size of $V_j$ decreases by a factor of $1-\frac{\epsi}{4\ell}$.
Then, by Property 1 of Lemma~\ref{lemma:tgone-iteration},
with a probability of at least $1/2$,
the iteration is successful.
Furthermore, if $\tau_j$ is less than $\tau_{\min}$,
$j$ will be removed from $I$ and 
solutions $A_j$ and $A_j'$ won't be updated anymore.

For each $j \in [\ell]$,
there are at most $\log_{1-\epsi}\left(\frac{\tau_{\min}}{M}\right) \le \epsi^{-1}\log\left(\frac{M}{\tau_{\min}}\right)$ possible threshold values.
And, for each threshold value, with at most 
$\log_{1-\frac{\epsi}{4\ell}}\left(\frac{1}{n}\right) \le 4\ell\epsi^{-1}\log(n)$
successful iterations regarding solution $A_j$,
the threshold value $\tau_j$ will decrease
or the algorithm terminates because of $m_0=0$.
Overall, with at most $4\ell^2\epsi^{-2}\log(n)\log\left(\frac{M}{\tau_{\min}}\right)$
successful iterations,
the algorithm terminates because of $m_0=0$ or $I=\emptyset$.

Next, we prove that, after $N=4\left(\log(n)+ 4\ell^2\epsi^{-2}\log(n)\log\left(\frac{M}{\tau_{\min}}\right)\right)$ iterations,
with a probability of $1-\frac{1}{n}$,
there exists at least $4\ell^2\epsi^{-2}\log(n)\log\left(\frac{M}{\tau_{\min}}\right)$
successful iterations,
or equivalently, the algorithm terminates.
Let $X$ be the number of successful iterations.
Then, $X$ can be regarded as a sum of $N$ dependent Bernoulli trails,
where the success probability is larger than $1/2$.
Let $Y$ be a sum of $N$ independent Bernoulli trials,
where the success probability is equal to $1/2$.
Then, the probability that the algorithm terminates with at most $N$ iterations can be bounded as follows,
\begin{align*}
\prob{\#\text{iterations} > N} &\le \prob{X \le 4\ell^2\epsi^{-2}\log(n)\log\left(\frac{M}{\tau_{\min}}\right)} \\
& \overset{(a)}{\le} \prob{Y \le 4\ell^2\epsi^{-2}\log(n)\log\left(\frac{M}{\tau_{\min}}\right)}\tag{Lemma~\ref{lemma:indep}}\\
&\le e^{- \frac{N}{4}\left(1-\frac{8\ell^2\epsi^{-2}\log(n)\log\left(\frac{M}{\tau_{\min}}\right)}{N}\right)^2} \tag{Lemma~\ref{lemma:chernoff}}\\
&= e^{-\frac{\left(4\log(n)+ 8\ell^2\epsi^{-2}\log(n)\log\left(\frac{M}{\tau_{\min}}\right)\right)^2}{16\left(\log(n)+ 4\ell^2\epsi^{-2}\log(n)\log\left(\frac{M}{\tau_{\min}}\right)\right)}} \le \frac{1}{n}.
\end{align*}
Therefore, with a probability of $1-\frac{1}{n}$,
the algorithm terminates with $\oh{\ell^2\epsi^{-2}\log(n)\log\left(\frac{M}{\tau_{\min}}\right)}$ iterations of the while loop.

In Alg.~\ref{alg:ptgone}, oracle queries occur during calls
to \update and \prefix 
on Line~\ref{line:tgone-update-2},~\ref{line:tgone-prefix} and~\ref{line:tgone-update}.
The \prefix algorithm,
with input $(f, \mathcal V, s, \tau, \epsi)$,
operates with $1$ adaptive rounds
and at most $|\mathcal V|$ queries.
The \update algorithm,
with input $(f, V_0, \tau_0, \epsi)$, 
outputs $(V, \tau)$
with $1+\log_{1-\epsi}\left(\frac{\tau}{\tau_0}\right)$ adaptive rounds
and at most $|V| + n\log_{1-\epsi}\left(\frac{\tau}{\tau_0}\right)$ queries.
Here, $\log_{1-\epsi}\left(\frac{\tau}{\tau_0}\right)$ equals the number of iterations in the while loop within \update.
Notably, every iteration is successful,
as the threshold value is updated.
Consequently, we can regard an iteration of the while loop in \update
as a separate iteration of the while loop in Alg.~\ref{alg:ptgone},
where such iteration only update one threshold value $\tau_j$ and its corresponding candidate set $V_j$.
So, each redefined iteration has no more than $2$ adaptive rounds,
and then the adaptivity of the algorithm should be no more than 
the number of successful iterations, which is
$\oh{\ell^2\epsi^{-2}\log(n)\log\left(\frac{M}{\tau_{\min}}\right)}$.
Since there are at most $\ell n$ queries at each adaptive rounds,
the query complexity is bounded by $\oh{\ell^3\epsi^{-2}n\log(n)\log\left(\frac{M}{\tau_{\min}}\right)}$.

% Next, we consider the query complexity of the algorithm. 
% Let $V_{j, i}$ be the set $V_j$ at the beginning of
% $i$-th redefined iteration of the while loop.
% Note that a redefined iteration of the while loop in Alg.~\ref{alg:ptgone} corresponds either to an iteration
% where none of the threshold values are updated
% or to an iteration of the while loop in \update.
% During every redefined iteration, 
% there are at most $2|V_{j, i}|$ queries if $\tau_j$ is not updated regrading solution $A_j$,
% or $|V_{j, i}| = n$ queries if $\tau_j$ is updated.

% In the worst case, at every successful iteration,
% only the set $V_j$ with minimum size decreases by a factor of $1-\frac{\epsi}{4\ell}$.
% Thus, the worst case scenario follows the following steps:
% 1) each $V_j$ starts from $\uni$
% and only one of $V_j$ decreases with a factor of $1-\frac{\epsi}{4\ell}$
% until $\tau_j$ is updated;
% 2) each $V_j$ becomes $\uni$ again and step (1) is repeated until
% all $\tau_j$ are below $\tau_{\min}$.
% Recall that, with at most $4\ell\epsi^{-1}\log(n)$ successful iterations
% regarding solution $A_j$, the threshold value $\tau_j$ will decrease
% or the algorithm terminates because of $m_0=0$.
% Moreover, for each $j\in [\ell]$,
% there are at most $\epsi^{-1}\log\left(\frac{M}{\tau_{\min}}\right)$
% possible threshold values
% resulting in at most $\ell\epsi^{-1}\log\left(\frac{M}{\tau_{\min}}\right)$ repeats of step (1).
% Let $Y_i$ be the number of iterations between 
% the $(i-1)$-th success and $i$-th success.
% By Lemma~\ref{lemma:indep}, since an iteration success with 
% a probability of $1/2$, it holds that $\ex{Y_i} \le 2$.
% Then, the expected number of iterations after $4\ell\epsi^{-1}\log(n)$ successful iterations can be bounded as follows,
% \begin{align*}
% 	\ex{\sum_{i=1}^{4\ell\epsi^{-1}\log(n)} Y_i} \le 8\ell\epsi^{-1}\log(n).
% \end{align*}
% Therefore, the query complexity of the algorithm can be bounded as follows,
% \begin{align*}
% 	\ex{\text{Queries}} & \le \sum_{i =1}^{N} \sum_{j = 1}^\ell 2|V_{j, i}|\\
% 	& \le \ell\epsi^{-1}\log\left(\frac{M}{\tau_{\min}}\right) \cdot
% 	\ex{\sum_{i=1}^{4\ell\epsi^{-1}\log(n)} Y_i\cdot 2\ell n}\\
% 	&\le 16 \ell^3 \epsi^{-1}n\log(n)\log\left(\frac{M}{\tau_{\min}}\right).
% \end{align*}
\end{proof}

\subsection{Analysis of Theorem~\ref{thm:ptgone} in Section~\ref{sec:ptg}}
\label{apx:ptgone-guarantee}
In this section, we provide the analysis of the parallel $1/4-\epsi$ approximation algorithm.
\thmptgone*
\begin{proof}[Proof of Theorem~\ref{thm:ptgone}]
The adaptivity and query complextiy are quite straightforward.
In the following, we will analyze the approximation ratio.

Let $S = O$ in Lemma~\ref{lemma:ptgone},
it holds that
\begin{align}
	& \ff{A_l'} \ge \ff{A_l}, \forall l = 1,2 \label{inq:ptg-1}\\
	& A_1 \cap A_2 = \emptyset \label{inq:ptg-2}\\
	& \marge{O}{A_1} + \marge{O}{A_2} \le \frac{1}{(1-\epsi)^2}\left(\ff{A_1'} + \ff{A_2'} \right) + \frac{2\epsi M}{1-\epsi}\label{inq:ptg-3}
\end{align}
Then,
\begin{align*}
	\ff{O} &\le \ff{O\cup A_1} + \ff{O\cup A_2} \tag{Submodularity, Nonnegativity, Inequality~\eqref{inq:ptg-2}} \\
	&\le \ff{A_1} + \ff{A_2} + \frac{1}{(1-\epsi)^2}\left(\ff{A_1'} + \ff{A_2'} \right) + \frac{2\epsi M}{1-\epsi} \tag{Inequality~\eqref{inq:ptg-3}}\\
	&\le 2\left(1+\frac{1}{(1-\epsi)^2}\right)\ff{G} + \frac{2\epsi}{1-\epsi}\ff{O} \tag{Inequality~\eqref{inq:ptg-1} and $G = \argmax\{\ff{A_1'}, \ff{A_2'}\}$}\\
	\Rightarrow \ff{G} &\ge \frac{(1-3\epsi)(1-\epsi)}{2\left((1-\epsi)^2 + 1+\frac{1}{k}\right)}\ff{O}\ge \left(\frac{1}{4}-\epsi\right)\ff{O} 
\end{align*}
\end{proof}

\subsection{Pseudocode and Analysis of Theorem~\ref{thm:ptgtwo} in Section~\ref{sec:ptg}}
\label{apx:ptgtwo}
\begin{algorithm}[ht]
\Fn{\ptgtwo($f, k, \epsi$)}{
	\KwIn{evaluation oracle $f:2^{\uni} \to \reals$, 
        constraint $k$, constant $\ell$, error $\epsi$}
	\Init{$G\gets \emptyset, \epsi' \gets \frac{\epsi}{2}, m\gets \left\lfloor \frac{k}{\ell} \right\rfloor, M\gets \max_{x\in\uni}\ff{\{x\}}, \tau_{\min}\gets \frac{\epsi'M}{k}$}
	\For{$i\gets 1$ to $\ell$}{
		$\{A_l': l\in [\ell]\} \gets \ptgone(f_{G}, m, \ell, \tau_{\min}, \epsi')$\;
		$G\gets$ a random set in $\{G\cup A_l': l\in [\ell]\}$\;
	}
	\Return{$G$}
	}
\caption{A randomized $(1/e-\epsi)$-approximation algorithm with $\oh{\ell^{3}\epsi^{-2}\log(n)\log(k)}$ adaptivity and $\oh{\ell^4\epsi^{-2}n\log(n)\log(k)}$ query complexity}\label{alg:ptg}
\label{alg:ptgtwo}
\end{algorithm}
In this section, we provide the pseudocode of the parallel $1/e-\epsi$ approximation algorithm with its analysis.

First, we provide the following lemma which provides a lower bound
on the gains achieved after every iteration in Alg.~\ref{alg:ptgtwo}.
\begin{lemma}\label{lemma:ptgtwo-recur}
For any iteration $i$ of the outer for loop in Alg.~\ref{alg:ptgtwo},
it holds that
\begin{align*}
\text{1) } & \ex{\ff{O\cup G_i}}\ge \left(1-\frac{1}{\ell}\right) \ex{\ff{O\cup G_{i-1}}}\\
\text{2) } & \ex{\ff{G_i} - \ff{G_{i-1}}}
\ge\frac{1}{1+\frac{\ell}{(1-\epsi')^2}}\left(1-\frac{1}{m+1}\right)\left(\left(1-\frac{1}{\ell}\right)  \ex{\ff{O\cup G_{i-1}}} - \ex{\ff{G_{i-1}}} - \frac{\epsi'}{1-\epsi'}\ff{O}\right).
\end{align*}
\end{lemma}
\begin{proof}[Proof of Lemma~\ref{lemma:ptgtwo-recur}]
Fix on $G_{i-1}$ at the beginning of this iteration,
Since $\{A_l: l\in [\ell]\}$ are pairwise disjoint sets,
by Proposition~\ref{prop:sum-marge}, it holds that
\[\exc{\ff{O\cup G_i}}{G_{i-1}} = \frac{1}{\ell}\sum_{l\in [\ell]}\ff{O\cup G_{i-1}\cup A_l} \ge \left(1-\frac{1}{\ell}\right)\ff{O\cup G_{i-1}}.\]
Then, by unfixing $G_{i-1}$, the first inequality holds.

To prove the second inequality, also consider fix on $G_{i-1}$ at the beginning of iteration $i$.
By Lemma~\ref{lemma:ptgone},
$\{A_l: l\in [\ell]\}$ are paiewise disjoint sets,
and the following inequalities hold,
\begin{align}
&A_l'\subseteq A_l, \marge{A_l'}{\emptyset} \ge \marge{A_l}{\emptyset}, \forall 1\le l \le \ell \label{inq:ptgtwo-1}\\
&\marge{O_{l}}{A_{l}}\le \frac{\marge{A_{l}'}{\emptyset}}{(1-\epsi')^2}+\frac{\epsi' M}{(1-\epsi')\ell}, \forall 1\le l \le \ell \label{inq:ptgtwo-2}\\
&\marge{O_{l_2}}{A_{l_1}} + \marge{O_{l_1}}{A_{l_2}} \le \frac{1+\frac{1}{m}}{(1-\epsi')^2}\left(\marge{A_{l_1}'}{\emptyset}+\marge{A_{l_2}'}{\emptyset}\right) + \frac{2\epsi' M}{(1-\epsi')\ell}, \forall 1\le l_1 < l_2 \le \ell \label{inq:ptgtwo-3}
\end{align}
Then,
\begin{align*}
\sum_{l\in [\ell]}\marge{O}{A_l\cup G_{i-1}} &\le \sum_{l_1\in [\ell]}\sum_{l_2\in [\ell]}\marge{O_{l_1}}{A_{l_2}\cup G_{i-1}}\tag{Proposition~\ref{prop:sum-marge}}\\
& = \sum_{l \in [\ell]}\marge{O_{l}}{A_{l}\cup G_{i-1}} + \sum_{1\le l_1< l_2 \le \ell} \left(\marge{O_{l_1}}{A_{l_2}\cup G_{i-1}} +\marge{O_{l_2}}{A_{l_1}\cup G_{i-1}}\right) \tag{Lemma~\ref{lemma:tg-par-A}}\\
& \le \sum_{l \in [\ell]}\left(\frac{\marge{A_{l}'}{G_{i-1}}}{(1-\epsi')^2}+\frac{\epsi' M}{(1-\epsi')\ell}\right)\\
&\hspace*{2em}+\sum_{1\le l_1< l_2 \le \ell} \left(\frac{\left(1+\frac{1}{m}\right)}{(1-\epsi')^2}\left(\marge{A_{l_1}'}{G_{i-1}}
+\marge{A_{l_2}'}{G_{i-1}}\right)
+\frac{2\epsi' M}{(1-\epsi')\ell}\right)\tag{Inequalities~\eqref{inq:ptgtwo-2} and~\eqref{inq:ptgtwo-3}}\\
&\le \frac{\ell}{(1-\epsi')^2}\left(1+\frac{1}{m}\right)\sum_{l \in [\ell]}\marge{A_{l}'}{G_{i-1}} + \frac{\epsi' \ell}{1-\epsi'}\ff{O}\tag{$M \le \ff{O}$}
\end{align*}
\begin{align*}
\Rightarrow \left(1+\frac{\ell}{(1-\epsi')^2}\right)\left(1+\frac{1}{m}\right) \sum_{l\in [\ell]}\marge{A_l'}{G_{i-1}} &\ge \sum_{l\in [\ell]}\ff{O\cup A_l\cup G_{i-1}} -\ell\ff{G_{i-1}} - \frac{\epsi' \ell}{1-\epsi'}\ff{O} \tag{Inequality~\eqref{inq:ptgtwo-1}}\\
&\ge \left(\ell-1\right)\ff{O\cup G_{i-1}}-\ell\ff{G_{i-1}} - \frac{\epsi' \ell}{1-\epsi'}\ff{O}
\end{align*}
Thus,
\begin{align*}
&\exc{\ff{G_i} - \ff{G_{i-1}}}{G_{i-1}}  = \frac{1}{\ell}\sum_{l \in [\ell]}\marge{A_{l}'}{G_{i-1}}\\
&\ge \frac{1}{1+\frac{\ell}{(1-\epsi')^2}} \frac{m}{m+1}\left(\left(1-\frac{1}{\ell}\right)  \ff{O\cup G_{i-1}} - \ff{G_{i-1}} - \frac{\epsi'}{1-\epsi'}\ff{O}\right)\tag{Proposition~\ref{prop:sum-marge}}
\end{align*}
By unfixing $G_{i-1}$, the second inequality holds.
\end{proof}
\thmptgtwo*
\begin{proof}[Proof of Theorem~\ref{thm:ptgtwo}]
Since the algorithm contains a for loop
which runs \ptgone $\ell = \oh{1/\epsi}$ times,
by Lemma~\ref{lemma:ptgone},
the adaptivity, query complexity and success probability holds immediately.

Next, we provide the analysis of approximation ratio.
By solving the recurrence in Lemma~\ref{lemma:ptgtwo-recur},
we calculate the approximation ratio of the algorithm as follows,
\begin{align*}
&\ex{\ff{G_{i}}}  \ge \left(1-\frac{1}{\ell}\right) \ex{\ff{G_{i-1}}}
+ \frac{1}{1+\frac{\ell}{(1-\epsi')^2}}\left(1-\frac{1}{m+1}\right)\left(\left(1-\frac{1}{\ell}\right)^i - \frac{\epsi'}{1-\epsi'}\right)\ff{O}\\
\Rightarrow& \ex{\ff{G_\ell}} \ge \frac{\ell}{1+\frac{\ell}{(1-\epsi')^2}}\left(1-\frac{1}{m+1}\right)\left(\left(1-\frac{1}{\ell}\right)^\ell - \frac{\epsi'}{1-\epsi'}\left(1-\left(1-\frac{1}{\ell}\right)^\ell\right)\right)\ff{O}\\
&\hspace*{4em} \ge \frac{\ell-1}{1+\frac{\ell}{(1-\epsi')^2}}\left(1-\frac{1}{m+1}\right)\left(e^{-1} - \frac{\epsi'}{1-\epsi'}\left(1-e^{-1}\right)\right)\ff{O}\\
&\hspace*{4em} \ge \frac{1}{1-\frac{\ell}{k}}\left((1-\epsi')^2 - \frac{2}{\ell}\right)\left(1-\frac{\ell}{k}\right)^2\left(e^{-1} - \frac{\epsi'}{1-\epsi'}\left(1-e^{-1}\right)\right) \ff{O}\\
% &\hspace*{4em} \ge \frac{1}{1-\frac{\ell}{k}}\left(1-\epsi' - \frac{2}{\ell}\right)\left(1-\frac{2\ell}{k}\right)\left(e^{-1} - \frac{\epsi'}{1-\epsi'}\left(1-e^{-1}\right)\right) \ff{O}\\
&\hspace*{4em} \ge \frac{1}{1-\frac{\ell}{k}}\left((1-\epsi')^2 - \frac{2}{\ell}-\frac{2(1-\epsi')^2 \ell}{k}\right)\left(e^{-1} - \frac{\epsi'}{1-\epsi'}\left(1-e^{-1}\right)\right) \ff{O}\\
&\hspace*{4em} \ge \frac{1}{1-\frac{\ell}{k}} \left(1-(e+1)\epsi'\right)\left(e^{-1} - \frac{\epsi'}{1-\epsi'}\left(1-e^{-1}\right)\right) \ff{O}\tag{$\ell\ge \frac{2}{e\epsi'}, k\ge \frac{2(1-\epsi')^2\ell}{e\epsi'-\frac{2}{\ell}}$}\\
&\hspace*{4em} \ge \frac{1}{1-\frac{\ell}{k}} \left(e^{-1}-\epsi\right)\ff{O}\tag{$\epsi' = \frac{\epsi}{2}$}.
\end{align*}
By Inequality~\ref{inq:tgtwo-dif-opt},
the approximation ratio of Alg.~\ref{alg:tgtwo} is $e^{-1}-\epsi$.
\end{proof}















\section{Experimental Setups and Additional Empirical Results}\label{apx:exp}
In the section, we introduce the settings in Section~\ref{sec:exp} further, and discuss more experimental results on \nmon and \mon.

\subsection{Applications}\label{apx:app}
\textbf{Maxcut.}
In the context of the maxcut application, we start with a graph $G=(V, E)$ 
where each edge $ij \in E$ has a weight $w_{ij}$.
The objective is to find a cut that maximizes the total weight of edges crossing the cut.
The cut function $f: 2^V \to \reals$ is defined as follows,
\[f(S) = \sum_{i \in S} \sum_{j \in V\setminus S}w_{ij}, \forall S\subseteq V.\]
This is a non-monotone submodular function.
In our implementation, for simplicity, all edges have a weight of $1$.

\textbf{Revmax.}
In our revenue maximization application,
we adopt the revenue maximization model introduced in \citep{DBLP:conf/www/HartlineMS08}, which we will briefly outline here.
Consider a social network $G=(V, E)$,
where $V$ denotes the buyers.
Each buyer $i$'s value for a good depends on the set of buyers $S$ that already own it, 
which is formulated by 
\[v_i(S)=f_i\left(\sum_{j \in S} w_{ij}\right),\]
where $f_i: \reals \to \reals$ is a non-negative, monotone, concave function, and $w_{ij}$ is drawn independently from a distribution.
The total revenue generated from selling goods to the buyers $S$ is
\[f(S) = \sum_{i \in V\setminus S} f_i\left(\sum_{j \in S} w_{ij}\right).\]
This is a non-monotone submodular function.
In our implementation, 
we randomly choose each $w_{ij}\in (0,1)$,
and $f_i(x) = x^{\alpha_i}$, where $\alpha_i \in (0,1)$ is chosen uniformly randomly.

% \textbf{Imgsum.}
% We follow the setting of Personalized Image Summarization application in~\citet{mirzasoleiman2016fast} with the following objective function
% \[f(S) = \sum_{i \in \uni} \max_{j \in S}s_{ij} - \frac{1}{n}\sum_{i \in S}\sum_{j \in S}s_{ij},\]
% where $s_{ij}$ determines the cosine similarity of image $i$ to image $j$
% with pixel vectors.
% The first term tries to ensure that the set $S$ is a good summary of the dataset,
% while the second promotes diversity within the summary itself. 
% This is a non-monotone, submodular objective function.

\subsection{Datasets}\label{apx:data}
\textbf{er} is a synthetic random graph generated by Erd{\"{o}}s-R{\'{e}}nyi model~\citep{erdds1959random} by setting number of nodes $n=100,000$ and edge probability $p=\frac{5}{n}$.

\textbf{web-Google}~\citep{DBLP:journals/im/LeskovecLDM09}  is a web
graph of $n=875,713$ web pages as nodes and $5,105,039$ hyperlinks
as edges.

\textbf{musae-github}~\citep{rozemberczki2019multiscale} is a social network of GitHub developers with $n=37,700$ developers and $289,003$ edges,
where edges are mutual follower relationships between them.

\textbf{twitch-gamers}~\citep{rozemberczki2021twitch} is a social network of $n=168,114$ Twitch users with $6,797,557$ edges, 
where edges are mutual follower relationships between them.


% \textbf{CIFAR-10}~\citep{krizhevsky2009learning} dataset consists of $50,000$ training images and $10,000$ test images
% where each image 
% is represented by a pixel vector of length 3,072:
% $32 \times 32$ pixels with red, green, and blue channels.
% In this paper, we randomly choose $3,000$ images from the training dataset.

\subsection{Additional Results}\label{apx:nmon}
Fig.~\ref{fig:apx} provides additional results on musae-github dataset with $n=37,700$
and web-Google dataset with $n=875,713$.
It shows that as $n$ and $k$ increase, our algorithms achieve superior on objective values.
The results of query complexity and adaptivity align closely with those discussed in Section~\ref{sec:exp}.
Notably, the number of adaptive round of \ptgtwoshort exceeds $k$ on musae-github,
which may be attributed to the dataset's relatively small size.
\begin{figure}[ht]
    \centering
    \subfigure[musae-github, solution value]{\label{fig:git-val}
    \includegraphics[width=0.31\linewidth]{fig/epsi1/git-val.pdf}}
    \subfigure[musae-github, query]{\label{fig:git-query}
    \includegraphics[width=0.31\linewidth]{fig/epsi1/git-query.pdf}}
    \subfigure[musae-github, round]{\label{fig:git-round}
    \includegraphics[width=0.31\linewidth]{fig/epsi1/git-round.pdf}}
    \subfigure[web-Google, solution value]{\label{fig:google-val}
    \includegraphics[width=0.31\linewidth]{fig/epsi1/google-val.pdf}}
    \subfigure[web-Google, query]{\label{fig:google-query}
    \includegraphics[width=0.31\linewidth]{fig/epsi1/google-query.pdf}}
    \subfigure[web-Google, round]{\label{fig:google-round}
    \includegraphics[width=0.31\linewidth]{fig/epsi1/google-round.pdf}}
    \caption{Results for \revmax on musae-github with $n=37,700$,
    and \maxcut on web-Google with $n=875,713$.}
    \label{fig:apx}
\end{figure}




















\end{document}


% This document was modified from the file originally made available by
% Pat Langley and Andrea Danyluk for ICML-2K. This version was created
% by Iain Murray in 2018, and modified by Alexandre Bouchard in
% 2019 and 2021 and by Csaba Szepesvari, Gang Niu and Sivan Sabato in 2022.
% Modified again in 2023 and 2024 by Sivan Sabato and Jonathan Scarlett.
% Previous contributors include Dan Roy, Lise Getoor and Tobias
% Scheffer, which was slightly modified from the 2010 version by
% Thorsten Joachims & Johannes Fuernkranz, slightly modified from the
% 2009 version by Kiri Wagstaff and Sam Roweis's 2008 version, which is
% slightly modified from Prasad Tadepalli's 2007 version which is a
% lightly changed version of the previous year's version by Andrew
% Moore, which was in turn edited from those of Kristian Kersting and
% Codrina Lauth. Alex Smola contributed to the algorithmic style files.
