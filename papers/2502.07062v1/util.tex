\usepackage{natbib}
\usepackage{url}            % simple URL typesetting
\usepackage{booktabs}       % professional-quality tables
\usepackage[final]{hyperref}       % hyperlinks
\usepackage{amsfonts}       % blackboard math symbols
\usepackage{nicefrac}       % compact symbols for 1/2, etc.
\usepackage{microtype}      % microtypography
\setlength{\textfloatsep}{4mm}
\usepackage{bm}
\usepackage{changepage}
\usepackage{wrapfig}
\usepackage{array, makecell} 
\usepackage{tabularx}
\usepackage{amsmath}
\let\proof\relax \let\endproof\relax
\usepackage{amsthm}
% \usepackage{algorithm}
% \PassOptionsToPackage{notext, nolist}{algorithm}
%\usepackage[noend]{algorithmic}
% \usepackage{algorithmicx}
% \usepackage[noend]{algpseudocode}
% \usepackage{verbatim}
\usepackage{graphicx}
\usepackage[space]{grffile}
% \usepackage{subcaption} % provide subfigure
\usepackage{caption}
% % Redefine the caption format
% \DeclareCaptionLabelFormat{captionless}{}
% \captionsetup[figure]{labelformat=captionless}
\usepackage{tikz}
\usepackage{subfigure}
\usepackage{mathrsfs}
\usepackage{amssymb}
\usepackage{xspace}
\usepackage{thmtools}
\usepackage{thm-restate}
\usetikzlibrary{arrows}
\usepackage{xcolor}
\usepackage{multirow}
\usepackage{threeparttable}
\usepackage{footmisc}
\usepackage{tablefootnote}
\allowdisplaybreaks

\usepackage[ruled,lined,noend,linesnumbered]{algorithm2e}
\DontPrintSemicolon
\SetAlgoProcName{Paradigm}{anautorefname}
\SetKwInOut{Init}{Initialize}
\SetKwProg{Fn}{Procedure}{:}{}

\newcommand\numberthis{\addtocounter{equation}{1}\tag{\theequation}}

% ---- symbols ----
\newcommand{\uni}{\mathcal U}
\newcommand{\reals}{\mathbb{R}_{\ge 0}}
% \newcommand{\todo}[1]{\textcolor{red}{TODO: #1}}
\newcommand{\fix}[1]{\textcolor{red}{#1}}
\newcommand{\etal}{\textit{et al.}\xspace}
\newcommand{\ie}{\textit{i.e.}\xspace}
\newcommand{\eg}{\textit{e.g.}\xspace}
\newcommand{\st}{\textit{s.t.}\xspace}
\newcommand{\func}[2]{ #1 \left( #2 \right) }
\newcommand{\ff}[1]{ f \left( #1 \right) }
\newcommand{\ffsub}[2]{ f_{#1} \left( #2 \right) }
\newcommand{\marge}[2]{\Delta \left( #1 \, \middle| \, #2 \right) }
\newcommand{\margesub}[3]{\Delta_{#1} \left( #2 \, \middle| \, #3 \right) }
\newcommand{\ex}[1]{\mathbb{E}\left[ #1 \right]}
\newcommand{\exs}[2]{ \mathbb{E}_{ #1 } \left[ #2 \right] }
\newcommand{\exc}[2]{ \mathbb{E}\left[\left. #1 \, \right| \; #2 \right] }
\newcommand{\oh}[1]{ \mathcal O \left( #1 \right) }
\newcommand{\epsi}[0]{ \varepsilon }
\newcommand{\opt}{\text{OPT}}
\newcommand{\prob}[1]{ \mathbb{P} \left[ #1 \right] }
\newcommand{\probs}[2]{ \mathbb{P}_{ #1 } \left[ #2 \right] }
\newcommand{\probc}[2]{ \mathbb{P}\left[ #1 \, | \; #2\right] }
\newcommand{\sm}{\textsc{SMCC}\xspace}
\renewcommand{\restriction}{\mathord{\upharpoonright}}
\newcommand{\oht}[1]{\tilde{\mathcal{O}}\left( #1 \right)}

\DeclareMathOperator*{\argmax}{arg\,max}
\DeclareMathOperator*{\argmin}{arg\,min}

\newcommand{\nmon}{\textsc{SM-Gen}\xspace}
\newcommand{\mon}{\textsc{SM-Mon}\xspace}

\newcommand{\maxcut}{$\texttt{maxcut}$\xspace}
\newcommand{\revmax}{$\texttt{revmax}$\xspace}
\newcommand{\imgsum}{$\texttt{imgsum}$\xspace}


% ---- theorem env ----
\usepackage{thmtools,thm-restate}
\declaretheorem[style=definition,numberwithin=section]{theorem}
\declaretheorem[style=definition,sibling=theorem]{lemma}
\declaretheorem[style=definition,numberwithin=section]{proposition}
\declaretheorem[style=definition,numberwithin=section]{definition}
\declaretheorem[style=definition,numberwithin=section]{example}
\declaretheorem[style=definition,numberwithin=section]{remark}
\declaretheorem[style=definition,numberwithin=section]{claim}
\declaretheorem[style=definition,numberwithin=section]{corollary}
\declaretheorem[style=definition]{property}

% ---- algorithms ----
\newcommand{\rg}{\textsc{RandomGreedy}}
\newcommand{\greedy}{\textsc{Greedy}}
\newcommand{\linearseq}{\textsc{LinearSeq}\xspace}
\newcommand{\linearcnst}{\textsc{LinearConstant}\xspace}
\newcommand{\randomset}{\textsc{RandomSet}\xspace}
\newcommand{\ig}{\textsc{InterlaceGreedy}\xspace}
\newcommand{\itg}{\textsc{InterpolatedGreedy}\xspace}
\newcommand{\ptgone}{\textsc{ParallelInterlaceGreedy}\xspace}
\newcommand{\ptgtwo}{\textsc{ParallelInterpolatedGreedy}\xspace}
\newcommand{\ptgoneshort}{\textsc{PIG}\xspace}
\newcommand{\ptgtwoshort}{\textsc{PItG}\xspace}

\newcommand{\sts}{\textsc{SubsampledThreshSeq}\xspace}
\newcommand{\tssg}{\textsc{SubsampledTG}\xspace}
\newcommand{\ptg}{\textsc{ParallelTG}\xspace}
\newcommand{\ts}{\textsc{ThreshSeq}\xspace}
\newcommand{\thresh}{\textsc{Threshold}\xspace}
\newcommand{\tsmod}{\textsc{ThreshSeq-Mod}\xspace}
\newcommand{\tssub}{\textsc{ThreshSeq-Sub}\xspace}
\newcommand{\dist}{\textsc{Distribute}\xspace}
\newcommand{\prefix}{\textsc{Prefix-Selection}\xspace}
\newcommand{\update}{\textsc{Update}\xspace}

\newcommand{\randomgreedy}{\textsc{RandomGreedy}\xspace}
\newcommand{\ltl}{\textsc{LazierThanLazyGreedy}\xspace}
\newcommand{\ltlshort}{\textsc{LTLG}\xspace}
\newcommand{\fast}{\textsc{Fast}\xspace}
\newcommand{\lspgb}{\textsc{LS+PGB}\xspace}
\newcommand{\parskp}{\textsc{ParSKP}\xspace}
\newcommand{\parssp}{\textsc{ParSSP}\xspace}

\newcommand{\frg}{\textsc{FastRandomGreedy}\xspace}
\newcommand{\anm}{\textsc{AdaptiveNonmonotoneMax}\xspace}
\newcommand{\unc}{\textsc{UnconstrainedMax}\xspace}

