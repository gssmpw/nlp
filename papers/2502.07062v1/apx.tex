\section{Pseudocode and Theoretical Guarantees of \ig~\citep{DBLP:conf/nips/Kuhnle19} and \itg~\citep{DBLP:conf/kdd/ChenK23}}\label{apx:pseudocode}
In this section, we provide the original greedy version of 
\ig~\citep{DBLP:conf/nips/Kuhnle19} and \itg~\citep{DBLP:conf/kdd/ChenK23}
with their theoretical guarantees.
\begin{algorithm}[ht]
    \KwIn{evaluation oracle $f:2^{\uni} \to \reals$, constraint $k$}
    \KwOut{$C\subseteq \uni$, such that $|C| \le k$}
    $A_0\gets B_0 \gets \emptyset$\;
    \For{$i\gets 0$ to $k-1$}{
        $a_i\gets \argmax_{x\in \uni\setminus \left(A_i\cup B_i\right)} \marge{x}{A_i}$\;
        $A_{i+1}\gets A_i+a_i$\;
        $b_i\gets \argmax_{x\in \uni\setminus \left(A_{i+1}\cup B_i\right)} \marge{x}{B_i}$\; 
        $B_{i+1}\gets B_i+b_i$\;
    }
    $D_1 \gets E_1 \gets \{a_0\}$\;
    \For{$i\gets 1$ to $k-1$}{
        $d_i\gets \argmax_{x\in \uni\setminus \left(D_i\cup E_i\right)} \marge{x}{D_i}$\;
        $D_{i+1}\gets D_i+d_i$\;
        $e_i\gets \argmax_{x\in \uni\setminus \left(D_{i+1}\cup E_i\right)} \marge{x}{E_i}$\; 
        $E_{i+1}\gets E_i+e_i$\;
    }
    \Return{$C\gets \argmax\{\ff{A_i}, \ff{B_i}, \ff{D_i}, \ff{E_i} : i\in [k+1]\}$}
    \caption{$\ig(f,k)$: The \ig Algorithm~\citep{DBLP:conf/nips/Kuhnle19}}
    \label{alg:ig}
\end{algorithm}
\begin{theorem}
Let $f:2^{\uni} \to \reals$ be submodular, let $k\in \uni$,
let $O = \argmax_{|S|\le k} \ff{S}$,
and let $C = \ig(f, k)$. Then
\[\ff{C} \ge \ff{O}/4,\]
and \ig makes $\oh{kn}$ queries to $f$.
\end{theorem}

\begin{algorithm}[ht]
	\KwIn{oracle $f:2^{\uni} \to \reals$, constraint $k$, error $\epsi$}
    \Init{$\ell \gets \frac{2e}{\epsi}+1$, $G_0\gets \emptyset$}
    \For{$m\gets 1$ to $\ell$}{
    	$\{a_1, \ldots, a_\ell\}\gets$ top $\ell$ elements in $\uni\setminus G_{m-1}$
		with respect to marginal gains on $G_{m-1}$\;
		\For{$u\gets 0$ to $\ell$ in parallel}{
			\lIf{$u = 0$}{$A_{u, l}\gets G\cup \{a_l\}$, for all $1\le l\le \ell$}
			\lElse{$A_{u, l}\gets G\cup \{a_u\}$, for all $1\le l\le \ell$}
			\For{$j\gets 1$ to $k/\ell-1$}{
				\For{$i\gets 1$ to $\ell$}{
					$x_{j, i} \gets \argmax_{x\in \uni\setminus\left(\bigcup_{l=1}^{\ell}A_{u, l}\right)}\marge{x}{A_{u, i}}$\;
					$A_{u, i}\gets A_{u, i}\cup \{x_{j, i}\}$\;
				}
			}
		}
		$G_m\gets$ a random set in $\{A_{u, i}:1\le i\le \ell, 0\le u\le \ell\}$
    }
    \Return{$G_\ell$}\;
    \caption{$\itg(f,k,\epsi)$: An $1/(e+\epsi)$-approximation algorithm for \sm}
    \label{alg:itg}
\end{algorithm}
\begin{theorem}
Let $\epsi \ge 0$, and $(f, k)$ be an instance of \sm, 
with optimal solution value \opt.
Algorithm \itg outputs a set $G_\ell$ with $\oh{\epsi^{-2}kn}$ queries
such that $\opt \le (e+\epsi)\ex{\ff{G_\ell}}$ with 
probability $(\ell+1)^{-\ell}$, where $\ell = \frac{2e}{\epsi}+1$.
\end{theorem}


\section{Analysis of Alg.~\ref{alg:gdone} in Section~\ref{sec:greedy-1/4}}
\label{apx:greedy-1/4}
In this section, we provide the detailed analysis of approximation ratio
for Alg.~\ref{alg:gdone}.
\thmgdone*
\begin{proof}[Proof of Theorem~\ref{thm:gdone}]
% Consider adding dummy elements to the ground set.
% If $|O|$ is less than $k$, add dummy elements to $O$ until $|O| = k$.
% During each iteration of the for-loop,
% if no element in $\uni\setminus (A\cup B)$ has marginal gain greater than 0,
% a dummy element will be added to $A$,
% and similarly for $B$.
% Thus, after the for loop ends, we ensure that $|A| = |B| = k$.
% Moreover, let $A_0 = B_0 = \emptyset$,
% and $A_i$, $B_i$ be $A$, $B$ after $i$-th element is added, respectively.

\textbf{Notation.}
Let $a_i$ be the $i$-th element added to $A$,
and $A_i$ be the set containing the first $i$ elements of $A$.
Similarly, define $b_i$ and $B_i$ for the solution $B$.

Since the two solutions $A_k$ and $B_k$ are disjoint, by submodularity and non-negativity,
\begin{equation*}
\ff{O} \le \ff{O\cup A_k} + \ff{O\cup B_k}.
\end{equation*}
Let $i^* = \max\{i \in [k]: A_i\subseteq O\}$
and $j^* = \max\{j \in [k]: B_j\subseteq O\}$.
If either $i^* = k$ or $j^* = k$,
then $\ff{S} = \ff{O}$.
In the following, we consider $i^* < k$ and $j^* < k$
and discuss two cases of the relationship between $i^*$ and $j^*$ (Fig.~\ref{fig:gdone}).



\textbf{Case 1: $0\le i^*\le j^* < k$; Fig.~\ref{fig:gdone-1}.}
First, we bound $\ff{O\cup A_k}$. 
Consider the set $\tilde{O} = O\setminus \left(A_k \cup B_{i^*}\right)$.
Obviously, it holds that $|\tilde{O}|\le k-i^*$.
Then, order $\tilde{O}$ as $\{o_1, o_2, \ldots\}$ such that $o_i \not \in B_{i+i^*-1}$,
for all $1\le i \le |\tilde{O}|$.
Thus, by the greedy selection step in Line~\ref{line:gdone-greedy-A},
it holds that $\marge{a_{i+i^*}}{A_{i+i^*-1}}\ge \marge{o_i}{A_{i+i^*-1}}$
for all $1\le i \le |\tilde{O}|$.
Then,
\begin{align*}
\ff{O\cup A_k} - \ff{A_k} &\le \marge{B_{i^*}}{A_k} + \marge{\tilde{O}}{A_k}\\
&\le \ff{B_{i^*}} + \sum\limits_{i=1}^{|\tilde{O}|} \marge{o_i}{A_k}\\
&\le \ff{B_{i^*}} + \sum\limits_{i=1}^{|\tilde{O}|} \marge{o_i}{A_{i+i^*-1}}\\
&\le \ff{B_{i^*}} + \sum\limits_{i=i^*+1}^{k} \marge{a_i}{A_{i-1}}
= \ff{B_{i^*}} + \ff{A_k} - \ff{A_{i^*}},
\end{align*}
where the first three inequalities follow from submodularity;
and the last inequality follows from 
$\marge{a_{i+i^*}}{A_{i+i^*-1}}\ge \marge{o_i}{A_{i+i^*-1}}$
for all $1\le i \le |\tilde{O}|$,
and $\marge{a_i}{A_{i-1}}\ge 0$ for all $i \in [k]$.

Next, we bound $\ff{O\cup B_k}$.
Consider the set $\tilde{O} = O\setminus \left(A_{i^*} \cup B_{k}\right)$.
Obviously, it holds that $|\tilde{O}| \le k-i^*$.
Since $i^* = \max\{i \in [k]: A_i\subseteq O\}$,
we know that $a_{i^*+1} \not \in O$.
Thus, we can order $|\tilde{O}|$ as $\{o_1, o_2, \ldots\}$
such that $o_i \not \in A_{i+i^*}$ for all $1\le i \le |\tilde{O}|$.
Then, by the greedy selection step in Line~\ref{line:gdone-greedy-B},
it holds that $\marge{b_{i+i^*}}{B_{i+i^*-1}}\ge \marge{o_i}{B_{i+i^*-1}}$
for all $1\le i \le |\tilde{O}|$.
Following the analysis for $\ff{O\cup A}$,
we get
\begin{align*}
\ff{O\cup B_k} - \ff{B_k} &\le \marge{A_{i^*}}{B_k} + \marge{\tilde{O}}{B_k}\\
&\le \ff{A_{i^*}} + \sum\limits_{i=1}^{|\tilde{O}|} \marge{o_i}{B_k}\\
&\le \ff{A_{i^*}} + \sum\limits_{i=1}^{|\tilde{O}|} \marge{o_i}{B_{i+i^*-1}}\\
&\le \ff{A_{i^*}} + \sum\limits_{i=i^*+1}^{k} \marge{b_i}{B_{i-1}}
= \ff{A_{i^*}} + \ff{B_k} - \ff{B_{i^*}}.
\end{align*}

\textbf{Case 2: $0\le j^* < i^* < k$; Fig.~\ref{fig:gdone-2}.}
First, we bound $\ff{O\cup A_k}$.
Consider the set $\tilde{O} = O\setminus \left(A_{k}\cup B_{j^*}\right)$,
where $|\tilde{O}| \le k-j^*-1$.
By the definition of $j^*$,
we know that $b_{j^*+1}\not\in O$.
Thus, we can order $\tilde{O}$ as $\{o_1, o_2, \ldots\}$
such that $o_i\not \in B_{i+j^*}$ for all $1\le i\le |\tilde{O}|$.
Then, by the greedy selection step in Line~\ref{line:gdone-greedy-A},
it holds that $\marge{a_{i+j^*+1}}{A_{i+j^*}}\ge \marge{o_i}{A_{i+j^*}}$
for all $1\le i\le |\tilde{O}|$.
Following the above analysis, we get
\begin{align*}
\ff{O\cup A_k}-\ff{A_k} &\le \marge{B_{j^*}}{A_k}+\marge{\tilde{O}}{A_k}\\
&\le \ff{B_{j^*}} + \sum\limits_{i=1}^{|\tilde{O}|} \marge{o_i}{A_k}\\
&\le \ff{B_{j^*}} + \sum\limits_{i=1}^{|\tilde{O}|} \marge{o_i}{A_{i+j^*}}\\
&\le \ff{B_{j^*}} + \sum\limits_{i=j^*+2}^{k} \marge{a_i}{A_{i-1}}
= \ff{B_{j^*}} + \ff{A_k}-\ff{A_{j^*+1}}.
\end{align*}

Next, we bound $\ff{O\cup B_k}$.
Consider the set $\tilde{O} = O\setminus \left(A_{j^*+1} \cup B_k\right)$,
where $|\tilde{O}| \le k-j^*-1$.
Then, order $\tilde{O}$ as $\{o_1, o_2, \ldots\}$
such that $o_i\not \in A_{i+j^*}$ for all $1\le i\le |\tilde{O}|$.
By the greedy selection step in Line~\ref{line:gdone-greedy-B},
it holds that $\marge{b_{i+j^*}}{B_{i+j^*-1}}\ge \marge{o_i}{B_{i+j^*-1}}$.
Then,
\begin{align*}
\ff{O\cup B_k} - \ff{B_k}&\le \marge{A_{j^*+1}}{B_k} + \marge{\tilde{O}}{B_k}\\
&\le \ff{A_{j^*+1}}+\sum\limits_{i=1}^{|\tilde{O}|} \marge{o_i}{B_k}\\
&\le \ff{A_{j^*+1}}+\sum\limits_{i=1}^{|\tilde{O}|} \marge{o_i}{B_{i+j^*-1}}\\
&\le \ff{A_{j^*+1}}+\sum\limits_{i=j^*+1}^{k} \marge{b_i}{B_{i-1}} 
= \ff{A_{j^*+1}} + \ff{B_k}-\ff{B_{j^*}}.
\end{align*}

Therefore, in both cases, it holds that
\[\ff{O}\le \ff{O\cup A_k}+\ff{O\cup B_k}\le 2\left(\ff{A_k}+\ff{B_k}\right)\le 4\ff{S}.\]

\end{proof}

\section{Analysis of Alg.~\ref{alg:gdtwo} in Section~\ref{sec:greedy-1/e}}
\label{apx:greedy-1/e}
In what follows, we address the scenario where $k\, \text{mod}\,\ell > 0$
and Alg.~\ref{alg:gdtwo} returns a solution with size smaller than $k$
in Appendix~\ref{apx:gdtwo-k}.
We then provide proofs for the relevant Lemmata in Appendix~\ref{apx:gdtwo-lemma},
and conclude with an analysis of approximation ratio in Appendix~\ref{apx:gdtwo-approx}.
\subsection{Scenario where $k\, \text{mod}\,\ell > 0$.}\label{apx:gdtwo-k}
If $k\, \text{mod}\,\ell = 0$, 
the algorithm returns an approximation solution for a size constraint of 
$\ell\cdot\left\lfloor\frac{k}{\ell}\right\rfloor$.
By Proposition~\ref{prop:dif-opt}, it holds that 
\begin{equation}\label{inq:dif-opt}
\ff{O'}\ge \ell\cdot\left\lfloor\frac{k}{\ell}\right\rfloor / k \ff{O}
\ge \left(1-\frac{\ell}{k}\right)\ff{O},
O' = \argmax\limits_{S\subseteq \uni, |S|\le \ell\cdot\left\lfloor\frac{k}{\ell}\right\rfloor} \ff{S}.
\end{equation}

\subsection{Proofs of Lemmata for Theorem~\ref{thm:gdtwo}} \label{apx:gdtwo-lemma}
\textbf{Notation.} 
Let $G_{i-1}$ be $G$ at the start of $i$-th iteration in Alg.~\ref{alg:gdtwo},
$A_l$ be the set at the end of this iteration,
and $a_{l, j}$ be the $j$-th element added to $A_l$ during this iteration.

\lemmaparA*
\begin{proof}[Proof of Lemma~\ref{lemma:par-A}]
\begin{figure}
\centering
\includegraphics[width=0.45\linewidth]{fig/ITG.pdf}
\caption{This figure depicts the components of the solution sets $A_{l_1}$ and $A_{l_2}$.
A blue circle with a check mark represents an element in $O$,
while a red circle with a cross mark represents an element outside of $O$.
The grey rectangles indicate a sequence of consecutive elements in $O$.
The pink rectangles indicate the corresponding elements used to bound $\marge{O_{l_2}}{A_{l_1}}$ or $\marge{O_{l_1}}{A_{l_2}}$.
It is illustrated that $\marge{O_{l_1}}{A_{l_2}} + \marge{O_{l_2}}{A_{l_1}}\le \marge{A_{l_1}}{G_{i-1}} +  \marge{A_{l_2}}{G_{i-1}}$ under both cases.}
\label{fig:gdtwo}
\end{figure}
Recall that $A_{l, j}$ is $A_l$ after $j$-th element is added to $A_l$ at iteration $i$ of the outer for loop,
and $c_l^* = \max\left\{c\in [m]: A_{l, c}\setminus G_{i-1}\subseteq O_l \right\}$.

First, we prove that the first inequality holds.
For each $l\in [\ell]$, order the elements in $O_l$ as $\{o_1, o_2, \ldots\}$
such that $o_j \not \in A_{l, j-1}$ for any $1\le j \le |O_l|$.
Since each $o_j$ is either in $A_l$ or not in any solution set,
it remains in the candidate pool when $a_{l, j}$ is considered to be added to the solution.
Therefore, it holds that
\begin{equation}\label{inq:itg-1}
\marge{a_{l, j}}{A_{l, j-1}}\ge \marge{o_j}{A_{l, j-1}}.
\end{equation}
Then,
\begin{align*}
\marge{O_{l}}{A_{l}} \le \sum_{o_j\in O_l} \marge{o_j}{A_{l}} \le \sum_{o_j\in O_l} \marge{o_j}{A_{l, j-1}}\le \sum_{j=1}^{m}\marge{a_{l, j}}{A_{l, j-1}} = \marge{A_{l}}{G_{j-1}},
\end{align*}
where the first inequality follows from Proposition~\ref{prop:sum-marge},
the second inequality follows from submodularity,
and the last inequality follows from Inequality~\eqref{inq:itg-1}.

In the following, we prove that the second inequality holds.
For any $1\le l_1\le l_2\le \ell$,
we analyze two cases of the relationship between $c_{l_1}^* $ and $ c_{l_2}^*$ in the following.

% \textbf{Case 1: $c_{l_1}^* = c_{l_2}^* = m$.}
% Then, $O_{l_1} = A_{l_1}\setminus G_{i-1}$ and $O_{l_2} = A_{l_2}\setminus G_{i-1}$.
% By submodularity,
% \[\marge{O_{l_2}}{A_{l_1}} + \marge{O_{l_1}}{A_{l_2}} \le \marge{O_{l_2}}{G_{i-1}} + \marge{O_{l_1}}{G_{i-1}} = \marge{A_{l_1}}{G_{i-1}} + \marge{A_{l_2}}{G_{i-1}}.\]
% Therefore, the lemma holds in this case.

\textbf{Case 1: $c_{l_1}^* \le c_{l_2}^*$; left half part in Fig.~\ref{fig:gdtwo}.}

First, we bound $\marge{O_{l_1}}{A_{l_2}}$.
Since $c_{l_1}^* \le m$, we know that the $(c_{l_1}^*+1)$-th element in $A_{l_1}\setminus G_{i-1}$ is not in $O$.
So, we can order the elements in $O_{l_1}\setminus A_{l_1, c_{l_1}^*}$ as $\{o_1, o_2, \ldots\}$ such that $o_j \not \in A_{l_1, c_{l_1}^*+j+1}$.
(Refer to the gray block with a dotted edge in the top left corner of Fig.~\ref{fig:gdtwo} for $O_{l_1}$.)
Since each $o_j$ is either added to $A_{l_1}$ or not in any solution set,
it remains in the candidate pool when $a_{l_2, c_{l_1}^*+j}$ is considered to be added to $A_{l_2}$.
Therefore, it holds that 
\begin{equation}\label{inq:itg-case2-1}
\marge{a_{l_2, c_{l_1}^*+j}}{A_{l_2, c_{l_1}^*+j-1}} \ge \marge{o_j}{A_{l_2, c_{l_1}^*+j-1}}, \forall 1\le j\le m-c_{l_1}^*.
\end{equation}
Then,
\begin{align*}
\marge{O_{l_1}}{A_{l_2}} &\le \marge{A_{l_1, c_{l_1}^*}}{A_{l_2}}  + \sum_{o_j \in O_{l_1}\setminus A_{l_1, c_{l_1}^*}}\marge{o_j}{A_{l_2}} \tag{Proposition~\ref{prop:sum-marge}}\\
&\le \marge{A_{l_1, c_{l_1}^*}}{G_{i-1}} + \sum_{o_j \in O_{l_1}\setminus A_{l_1, c_{l_1}^*}}\marge{o_j}{A_{l_2, , c_{l_1}^*+j-1}} \tag{submodularity}\\
&\le \ff{A_{l_1, c_{l_1}^*}}-\ff{G_{i-1}} + \sum_{j = 1}^{m-c_{l_1}^*}\marge{a_{l_2, c_{l_1}^*+j}}{A_{l_2, , c_{l_1}^*+j-1}} \tag{Inequality~\eqref{inq:itg-case2-1}}\\
& \le \ff{A_{l_1, c_{l_1}^*}}-\ff{G_{i-1}} + \ff{A_{l_2}} - \ff{A_{l_2, c_{l_1}^*}} 
\end{align*}

Similarly, we bound $\marge{O_{l_2}}{A_{l_1}}$ below.
Order the elements in $O_{l_2}\setminus A_{l_2, c_{l_1}^*}$ as $\{o_1, o_2, \ldots\}$
such that $o_j \not \in A_{l_2, c_{l_1}^* + j}$.
(See the gray block with a dotted edge in the bottom left corner of Fig.~\ref{fig:gdtwo} for $O_{l_2}$.)
Since each $o_j$ is either added to $A_{l_2}$ or not in any solution set,
it remains in the candidate pool when $a_{l_1, c_{l_1}^*+j}$ is considered to be added to $A_{l_2}$.
Therefore, it holds that
\begin{equation}\label{inq:itg-case2-2}
\marge{a_{l_1, c_{l_1}^*+j}}{A_{l_1, c_{l_1}^*+j-1}} \ge \marge{o_j}{A_{l_1, c_{l_1}^*+j-1}}, \forall 1\le j\le m-c_{l_1}^*.
\end{equation}
Then,
\begin{align*}
\marge{O_{l_2}}{A_{l_1}} &\le \marge{A_{l_2, c_{l_1}^*}}{A_{l_1}}  + \sum_{o_j \in O_{l_2}\setminus A_{l_2, c_{l_1}^*}}\marge{o_j}{A_{l_1}} \tag{Proposition~\ref{prop:sum-marge}}\\
&\le \marge{A_{l_2, c_{l_1}^*}}{G_{i-1}} + \sum_{o_j \in O_{l_2}\setminus A_{l_2, c_{l_1}^*}}\marge{o_j}{A_{l_1, , c_{l_1}^*+j-1}} \tag{submodularity}\\
&\le \ff{A_{l_2, c_{l_1}^*}}-\ff{G_{i-1}} + \sum_{j = 1}^{m-c_{l_1}^*}\marge{a_{l_1, c_{l_1}^*+j}}{A_{l_1, , c_{l_1}^*+j-1}} \tag{Inequality~\eqref{inq:itg-case2-1}}\\
& \le \ff{A_{l_1, c_{l_1}^*}}-\ff{G_{i-1}} + \ff{A_{l_1}} - \ff{A_{l_1, c_{l_1}^*}} 
\end{align*}
Thus, the lemma holds in this case.

\textbf{Case 2: $c_{l_1}^* > c_{l_2}^*$; right half part in Fig.~\ref{fig:gdtwo}.}
First, we bound $\marge{O_{l_1}}{A_{l_2}}$.
Order the elements in $O_{l_1}\setminus A_{l_1, c_{l_2}^*+1}$ as $\{o_1, o_2, \ldots\}$ such that $o_j \not \in A_{l_1, c_{l_2}^*+j}$.
(Refer to the gray block with a dotted edge in the top right corner of Fig.~\ref{fig:gdtwo} for $O_{l_1}$.)
Since each $o_j$ is either in $A_{l_1}$ or not in any solution set,
it remains in the candidate pool when $a_{l_2, c_{l_2}^*+j}$ is considered to be added to $A_{l_2}$.
Therefore, it holds that 
\begin{equation}\label{inq:itg-case3-1}
\marge{a_{l_2, c_{l_2}^*+j}}{A_{l_2, c_{l_2}^*+j-1}} \ge \marge{o_j}{A_{l_2, c_{l_2}^*+j-1}}, \forall 1\le j\le m-c_{l_2}^*-1.
\end{equation}
Then,
\begin{align*}
\marge{O_{l_1}}{A_{l_2}} &\le \marge{A_{l_1, c_{l_2}^*+1}}{A_{l_2}}  + \sum_{o_j \in O_{l_1}\setminus A_{l_1, c_{l_2}^*+1}}\marge{o_j}{A_{l_2}} \tag{Proposition~\ref{prop:sum-marge}}\\
&\le \marge{A_{l_1, c_{l_2}^*+1}}{G_{i-1}} + \sum_{o_j \in O_{l_1}\setminus A_{l_1, c_{l_2}^*+1}}\marge{o_j}{A_{l_2, , c_{l_2}^*+j-1}} \tag{submodularity}\\
&\le \ff{A_{l_1, c_{l_2}^*+1}}-\ff{G_{i-1}} + \sum_{j = 1}^{m-c_{l_2}^*-1}\marge{a_{l_2, c_{l_2}^*+j}}{A_{l_2, , c_{l_2}^*+j-1}} \tag{Inequality~\eqref{inq:itg-case3-1}}\\
& \le \ff{A_{l_1, c_{l_2}^*+1}}-\ff{G_{i-1}} + \ff{A_{l_2}} - \ff{A_{l_2, c_{l_2}^*}} 
\end{align*}

Similarly, we bound $\marge{O_{l_2}}{A_{l_1}}$ below.
Since $c_{l_2}^* < c_{l_2}^*$,
we know that the $(c_{l_2}^*+1)$-th element in $A_{l_2}\setminus G_{i-1}$
is not in $O$, which implies that $|O_{l_2}| \le m$.
So, we can order the elements in $O_{l_2}\setminus A_{l_2, c_{l_2}^*}$ as $\{o_1, o_2, \ldots\}$
such that $o_j \not \in A_{l_2, c_{l_2}^* + j}$ for each $1\le j\le m-c_{l_2}^*$.
(See the gray block with a dotted edge in the bottom right corner of Fig.~\ref{fig:gdtwo} for $O_{l_2}$.)

When $1\le j< m-c_{l_2}^*$,
since each $o_j$ is either in $A_{l_2}$ or not in any solution set,
it remains in the candidate pool when $a_{l_1, c_{l_2}^*+i+1}$ is considered to be added to $A_{l_1}$.
Therefore, it holds that
\begin{equation}
\marge{a_{l_1, c_{l_2}^*+j+1}}{A_{l_1, c_{l_2}^*+j}} \ge \marge{o_j}{A_{l_1, c_{l_2}^*+j}}, \forall 1\le j< m-c_{l_2}^*.
\end{equation}
As for the last element $o_{m-c_{l_2}^*}$ in $O_{l_2}\setminus A_{l_2, c_{l_2}^*}$,
we know that $o_{m-c_{l_2}^*}$ is not added to any solution set.
So,
\begin{equation}
\marge{o_{m-c_{l_2}^*}}{A_{l_1}} \le \frac{1}{m}\sum_{j = 1}^m \marge{a_{l_1, j}}{A_{l_1,j-1}}
 = \frac{1}{m} \marge{A_{l_1}}{G_{i-1}}
\end{equation}

Then,
\begin{align*}
\marge{O_{l_2}}{A_{l_1}} &\le \marge{A_{l_2, c_{l_2}^*}}{A_{l_1}}  + \sum_{o_j \in O_{l_2}\setminus A_{l_2, c_{l_2}^*}}\marge{o_j}{A_{l_1}} \tag{Proposition~\ref{prop:sum-marge}}\\
&\le \marge{A_{l_2, c_{l_2}^*}}{G_{i-1}} + \sum_{o_j \in O_{l_2}\setminus A_{l_2, c_{l_2}^*}}\marge{o_j}{A_{l_1, , c_{l_2}^*+j}} \tag{submodularity}\\
&\le \ff{A_{l_2, c_{l_2}^*}}-\ff{G_{i-1}} + \sum_{j = 1}^{m-c_{l_2}^*-1}\marge{a_{l_1, c_{l_2}^*+j+1}}{A_{l_1, , c_{l_2}^*+j}} + \frac{1}{m} \marge{A_{l_1}}{G_{i-1}} \tag{Inequality~\eqref{inq:itg-case3-1}}\\
& \le \ff{A_{l_1, c_{l_2}^*}}-\ff{G_{i-1}} + \ff{A_{l_1}} - \ff{A_{l_1, c_{l_2}^*+1}}+ \frac{1}{m} \marge{A_{l_1}}{G_{i-1}} \tag{$|O_{l_2}|\le m$}
\end{align*}
Thus, the lemma holds in this case.
\end{proof}

\begin{restatable}{lemma}{lemmagdtworec}\label{lemma:gdtwo-rec}
For any iteration $i$ of the outer for loop in Alg.~\ref{alg:gdtwo},
it holds that 

\vspace*{-1em}
{\small\begin{align*}
&\ex{\ff{G_i} - \ff{G_{i-1}}}\ge \frac{1}{\ell+1}\left(1-\frac{1}{m+1}\right) \\
&\cdot \left(\left(1-\frac{1}{\ell}\right)\ex{\ff{O\cup G_{i-1}}} - \ex{\ff{G_{i-1}}}\right)
% \left(1-\frac{1}{m+1}\right)\left((\ell-1)\ff{O\cup G_{i-1}}-\ell\ff{G_{i-1}}\right) \le (\ell+1)\sum_{l\in [\ell]} \marge{A_l}{G_{i-1}}.
\end{align*}}
\end{restatable}
\begin{proof}[Proof of Lemma~\ref{lemma:gdtwo-rec}]
Fix on $G_{i-1}$ for an iteration $i$ of the outer for loop in Alg.~\ref{alg:gdtwo}.
Let $A_l$ be the set after for loop in Lines~\ref{line:gdtwo-for-2-start}-\ref{line:gdtwo-for-2-end} ends (with $m$ iterations).
Then,
\begin{align*}
&\sum_{l\in [\ell]}\marge{O}{A_{l}} 
\le \sum_{l\in [\ell]} \marge{O_l}{A_{l}} + \sum_{1\le l_1 < l_2 \le \ell} \left(\marge{O_{l_1}}{A_{l_2}}+\marge{O_{l_2}}{A_{l_1}}\right) \tag{Inequality~\ref{inq:gdtwo-par}}\\
&\le \sum_{l\in [\ell]} \marge{A_{l}}{G_{i-1}} + \sum_{1\le l_1 < l_2 \le \ell}\left(1+\frac{1}{m}\right) \left(\marge{A_{l_1}}{G_{i-1}}+\marge{A_{l_2}}{G_{i-1}}\right)\tag{Lemma~\ref{lemma:par-A}}\\
&\le \ell\left(1+\frac{1}{m}\right) \sum_{l\in [\ell]} \marge{A_{l}}{G_{i-1}}\\
\Rightarrow& \left(\ell+1\right)\left(1+\frac{1}{m}\right)\sum_{l\in [\ell]}\marge{A_l}{G_{i-1}} \ge \sum_{l\in [\ell]}\ff{O\cup A_l} - \ell \ff{G_{i-1}}\\
& \hspace*{15em} \ge \left(\ell-1\right)\ff{O\cup G_{i-1}} - \ell \ff{G_{i-1}},\numberthis \label{inq:itg-rec-1}
\end{align*}
where the last inequality follows from Proposition~\ref{prop:sum-marge}.
Then, it holds that
\begin{align*}
&\exc{\ff{G_i} - \ff{G_{i-1}}}{G_{i-1}}  = \frac{1}{\ell}\sum_{l \in [\ell]}\marge{A_{l}}{G_{i-1}}\\
&\ge \frac{1}{\ell+1}\cdot\frac{m}{m+1}\cdot\left(\left(1-\frac{1}{\ell}\right)\ff{O\cup G_{i-1}} - \ff{G_{i-1}}\right) \tag{Inequality~\eqref{inq:itg-rec-1}}
\end{align*}
By unfixing $G_{i-1}$, the lemma holds.
\end{proof}

\begin{restatable}{lemma}{lemmagdtwodeg}\label{lemma:gdtwo-deg}
For any iteration $i$ of the outer for loop in Alg.~\ref{alg:gdtwo},
it holds that

\vspace*{-1em}
{\small\begin{align*}
\ex{\ff{O\cup G_i}} \ge \left(1-\frac{1}{\ell}\right) \ex{\ff{O\cup G_{i-1}}}.
\end{align*}}
\end{restatable}
\begin{proof}[Proof of Lemma~\ref{lemma:gdtwo-deg}]
Fix on $G_{i-1}$ at the beginning of this iteration.
Since $\left\{A_l\setminus G_{i-1}\right\}_{l\in [\ell]}$ 
are pairwise disjoint sets at the end of this iteration,
by Proposition~\ref{prop:sum-marge},
it holds that
\[\exc{\ff{O\cup G_i}}{G_{i-1}} = \frac{1}{\ell}\sum_{l\in [\ell]}\ff{O\cup A_l} \ge \left(1-\frac{1}{\ell}\right)\ff{O\cup G_{i-1}}.\]
Then, by unfixing $G_{i-1}$, the lemma holds.
\end{proof}

\subsection{Proof of Theorem~\ref{thm:gdtwo}}\label{apx:gdtwo-approx}
\thmgdtwo*
\begin{proof}
By Lemma~\ref{lemma:gdtwo-rec} and~\ref{lemma:gdtwo-deg},
the recurrence of $\ex{\ff{G_i}}$ can be expressed as follows,
\begin{align*}
\ex{\ff{G_i}} &\ge \left(1-\frac{1}{\ell+1}\left(1-\frac{1}{m+1}\right)\right)\ex{\ff{G_{i-1}}} + \frac{1}{\ell+1}\left(1-\frac{1}{m+1}\right)\left(1-\frac{1}{\ell}\right)^i\ff{O}\\
&\ge \left(1-\frac{1}{\ell}\right)\ex{\ff{G_{i-1}}} + \frac{1}{\ell+1}\left(1-\frac{1}{m+1}\right)\left(1-\frac{1}{\ell}\right)^i\ff{O}.
\end{align*}
By solving the above recurrence,
\begin{align*}
\ex{\ff{G_{\ell}}} &\ge \frac{\ell}{\ell+1}\left(1-\frac{1}{m+1}\right)\left(1-\frac{1}{\ell}\right)^\ell\ff{O}\\
&\ge \frac{\ell-1}{\ell+1}\left(1-\frac{1}{m+1}\right)e^{-1}\ff{O}\tag{Lemma~\ref{lemma:val-inq}}\\
&\ge \left(1-\frac{2}{\ell}\right)\left(1-\frac{\ell}{k}\right)e^{-1}\ff{O} \tag{$m = \left\lfloor \frac{k}{\ell} \right\rfloor$}\\
&\ge \frac{1}{1-\frac{\ell}{k}}\left(1-\frac{2}{\ell}-\frac{2\ell}{k}+\frac{4}{k}\right)e^{-1}\ff{O}\\
&\ge \frac{1}{1-\frac{\ell}{k}}\left(e^{-1}-\epsi\right)\ff{O}. \tag{$\ell \ge \frac{2}{e\epsi}, k\ge \frac{2(\ell-2)}{e\epsi-\frac{2}{\ell}}$}
\end{align*}
By Inequality~\eqref{inq:dif-opt}, the approximation ratio of Alg.~\ref{alg:gdtwo} is $e^{-1}-\epsi$.
\end{proof}

\section{Pseudocodes and Analysis of Algorithms in Section~\ref{sec:tg}}
\label{apx:tg}
In this section, we provide the pseudocodes and analysis of
the simplified fast \ig~\citep{DBLP:conf/nips/Kuhnle19} 
and \itg~\citep{DBLP:conf/kdd/ChenK23},
implemented as Alg.~\ref{alg:tgone} and~\ref{alg:tgtwo}, respectively.
The analysis of these algorithms employ a blending technique
to eliminate the guessing step in their original version.
\subsection{Simplified Fast \ig with $1/4-\epsi$ Approximation Ratio (Alg.~\ref{alg:tgone})}
\begin{algorithm}[ht]
    \KwIn{evaluation oracle $f:2^{\uni} 
    \to \reals$, constraint $k$, error $\epsi$}
    \Init{$A\gets \emptyset$, $B\gets \emptyset$, $M\gets \max_{x \in \uni}\ff{\{x\}}$,
    $\tau_1\gets M$, $\tau_2\gets M$}
    \For{$i\gets 1$ to $k$}{
        \While{$\tau_1 \ge \frac{\epsi M}{k}$ and $|A| < k$}{
            \If{$\exists a \in \uni\setminus\left(A\cup B\right)$ \st $\marge{a}{A} \ge \tau_1$}{
            $A\gets A+ a$\;
            \textbf{break}\;}
            \lElse{$\tau_1 \gets (1-\epsi)\tau_1$}
        }
        \While{$\tau_2 \ge \frac{\epsi M}{k}$ and $|B| < k$}{
            \If{$\exists b \in \uni\setminus\left(A\cup B\right)$ \st $\marge{b}{B} \ge \tau_2$}{
            $B\gets B+ b$\;
            \textbf{break}\;}
            \lElse{$\tau_2 \gets (1-\epsi)\tau_2$}
        }
    }
    \Return{$S\gets \argmax\{\ff{A}, \ff{B}\}$}
    \caption{A nearly-linear time, $(1/4-\epsi)$-approximation algorithm.}
    \label{alg:tgone}
\end{algorithm}
\thmtgone*
\begin{proof}
\textbf{Query Complexity.}
Without loss of generality, we analyze the number queries related to set $A$.
For each threshold value $\tau_1$, at most $n$ queries are made to the value oracle.
Since $\tau_1$ is initialized with value $M$, decreases by a factor of $1-\epsi$,
and cannot exceed $\frac{\epsi M}{k}$,
there are at most $\log_{1-\epsi}\left(\frac{\epsi}{k}\right)+1$ possible values of $\tau_1$.
Therefore, the total number of queries is bounded as follows,
\begin{align*}
\#\text{Queries} \le 2\cdot n\cdot \left(\log_{1-\epsi}\left(\frac{\epsi}{k}\right)+1\right)
\le \oh{n\log(k)/\epsi},
\end{align*}
where the last inequality follows from the first inequality in Lemma~\ref{lemma:val-inq}.

\textbf{Approximation Ratio.}
Since $A$ and $B$ are disjoint, by submodularity and non-negativity,
\begin{equation}\label{inq:tgone-1}
\ff{O} \le \ff{O\cup A} + \ff{O\cup B}.
\end{equation}

Let $a_i$ be the $i$-th element added to $A$,
$A_i$ be the first $i$ elements added to $A$,
and $\tau_1^{a_i}$ be the threshold value when $a_i$ is added to $A$.
Similarly, define $b_i$, $B_i$, and $\tau_2^{b_i}$.
Let $i^* = \max\{i \le |A|: A_i \subseteq O\}$
and $j^* = \max\{i \le |B|: B_i \subseteq O\}$.
If either $i^*= k$ or $j^* = k$,
then $\ff{S}= \ff{O}$.
Next, we follow the analysis of Alg.~\ref{alg:gdone} in Section~\ref{sec:greedy-1/4}
to analyze the approximation ratio of Alg.~\ref{alg:tgone}.

\textbf{Case 1: $0\le i^*\le j^* < k$; Fig.~\ref{fig:gdone-1}.}
First, we bound $\ff{O\cup A}$. 
Since $B_{i^*} \subseteq O$, by submodularity
\begin{equation}\label{inq:tgone-3}
\ff{O\cup A} - \ff{A} \le \marge{B_{i^*}}{A} + \marge{O\setminus B_{i^*}}{A}
\le \ff{B_{i^*}}+ \sum_{o\in O\setminus \left(A\cup B_{i^*}\right)}\marge{o}{A}.
\end{equation}
Next, we bound $\marge{o}{A}$ for each $o\in O\setminus \left(A\cup B_{i^*}\right)$.

Let $\tilde{O} = O\setminus \left(A \cup B_{i^*}\right)$.
Obviously, it holds that $|\tilde{O}|\le k-i^*$.
Then, order $\tilde{O}$ as $\{o_1, o_2, \ldots\}$ such that $o_i \not \in B_{i+i^*-1}$,
for all $1\le i \le |\tilde{O}|$.
If $|A| < k$, the algorithm terminates with $\tau_1 < \frac{\epsi M}{k}$.
Thus, it follows that
\begin{equation}\label{inq:tgone-4}
\marge{o_i}{A} < \frac{\epsi M}{k(1-\epsi)}, \forall |A|-i^* < i \le |\tilde{O}|.
\end{equation}

Next, consider tuple $(o_i, a_{i + i^*}, A_{i+i^*-1})$,
for any $1\le i \le \min\{|\tilde{O}|, |A|-i^*\}$.
Since $\tau_1^{a_{i + i^*}}$ is the threshold value when $a_{i + i^*}$ is added,
it holds that 
\begin{equation}\label{inq:tgone-2}
\marge{a_{i + i^*}}{A_{i+i^*-1}} \ge \tau_1^{a_{i + i^*}},
\forall 1\le i\le |A| - i^*.
\end{equation}
Then, we show that $\marge{o_i}{A_{i+i^*-1}} < \tau_1^{a_{i + i^*}}/(1-\epsi)$ always holds
for any $1\le i \le \min\{|\tilde{O}|, |A|-i^*\}$.

Since $M = \max_{x\in \uni}\ff{\{x\}}$,
if $\tau_1^{a_{i+i^*}} \ge M$,
it always holds that $\marge{o_i}{A_{i+i^*-1}} < M/(1-\epsi)\le \tau_1^{a_{i+i^*}}/(1-\epsi)$.
If $\tau_1^{a_{i+i^*}} < M$, 
since $o_i \not \in B_{i+i^*-1}$,
$o_i$ is not considered to be added to $A$ with threshold value $\tau_1^{a_{i+i^*}}/(1-\epsi)$.
Then, by submodularity,
$\marge{o_i}{A_{i+i^*-1}} < \tau_1^{a_{i+i^*}}/(1-\epsi)$.
Therefore, by submodularity and Inequality~\eqref{inq:tgone-2},
it holds that 
\begin{equation}\label{inq:tgone-5}
\marge{o_i}{A} \le \marge{o_i}{A_{i+i^*-1}} < \marge{a_{i + i^*}}{A_{i+i^*-1}}/(1-\epsi), \forall 1\le i \le \min\{|\tilde{O}|, |A|-i^*\}.
\end{equation}

Then,
\begin{align*}
\ff{O\cup A} - \ff{A} &\le \ff{B_{i^*}} + \sum_{o\in O\setminus \left(A\cup B_{i^*}\right)}\marge{o}{A}\\
&\le \ff{B_{i^*}} + \sum_{i = 1}^{\min\{|\tilde{O}|, |A|\}-i^*}\marge{a_{i + i^*}}{A_{i+i^*-1}}/(1-\epsi) + \frac{\epsi M}{1-\epsi}\\
&\le \frac{1}{1-\epsi}\left(\ff{B_{i^*}} + \ff{A}-\ff{A_{i^*}} + \epsi \ff{O}\right),\numberthis \label{inq:tgone-10}
\end{align*}
where the first inequality follows from Inequality~\eqref{inq:tgone-3};
the second inequality follows from Inequalities~\eqref{inq:tgone-4} and~\eqref{inq:tgone-5};
and the last inequality follows from $M\le \ff{O}$.

Second, we bound $\ff{O\cup B}$.
Since $A_{i^*} \subseteq O$, by submodularity
\begin{equation}\label{inq:tgone-6}
\ff{O\cup B} - \ff{B} \le \marge{A_{i^*}}{B} + \marge{O\setminus A_{i^*}}{B}
\le \ff{A_{i^*}}+ \sum_{o\in O\setminus \left(B\cup A_{i^*}\right)}\marge{o}{B}.
\end{equation}
Next, we bound $\marge{o}{B}$ for each $o\in O\setminus \left(B\cup A_{i^*}\right)$.

Let $\tilde{O} = O\setminus \left(B\cup A_{i^*}\right)$.
Obviously, it holds that $|\tilde{O}|\le k-i^*$.
Then, since $a_{i^*+1} \not\in O$,
we can order $\tilde{O}$ as $\{o_1, o_2, \ldots\}$ such that $o_i \not \in A_{i+i^*}$,
for all $1\le i \le |\tilde{O}|$.
If $|B| < k$, the algorithm terminates with $\tau_2 < \frac{\epsi M}{k}$.
Thus, it follows that
\begin{equation}\label{inq:tgone-7}
\marge{o_i}{B} < \frac{\epsi M}{k(1-\epsi)}, \forall |B|-i^* < i \le |\tilde{O}|.
\end{equation}

Next, consider tuple $(o_i, b_{i + i^*}, B_{i+i^*-1})$,
for any $1\le i \le \min\{|\tilde{O}|, |B|-i^*\}$.
Since $\tau_2^{b_{i + i^*}}$ is the threshold value when $b_{i + i^*}$ is added,
it holds that 
\begin{equation}\label{inq:tgone-8}
\marge{b_{i + i^*}}{B_{i+i^*-1}} \ge \tau_2^{b_{i + i^*}},
\forall 1\le i\le |B| - i^*.
\end{equation}
Then, we show that $\marge{o_i}{B_{i+i^*-1}} < \tau_2^{b_{i + i^*}}/(1-\epsi)$ always holds
for any $1\le i \le \min\{|\tilde{O}|, |B|-i^*\}$.

Since $M = \max_{x\in \uni}\ff{\{x\}}$,
if $\tau_2^{b_{i+i^*}} \ge M$,
it always holds that $\marge{o_i}{B_{i+i^*-1}} < M/(1-\epsi)\le \tau_2^{b_{i+i^*}}/(1-\epsi)$.
If $\tau_2^{b_{i+i^*}} < M$, 
since $o_i \not \in A_{i+i^*}$,
$o_i$ is not considered to be added to $B$ with threshold value $\tau_2^{b_{i+i^*}}/(1-\epsi)$.
Then, by submodularity,
$\marge{o_i}{B_{i+i^*-1}} < \tau_2^{b_{i+i^*}}/(1-\epsi)$.
Therefore, by submodularity and Inequality~\eqref{inq:tgone-8},
it holds that 
\begin{equation}\label{inq:tgone-9}
\marge{o_i}{B} \le \marge{o_i}{B_{i+i^*-1}} < \marge{b_{i + i^*}}{B_{i+i^*-1}}/(1-\epsi), \forall 1\le i \le \min\{|\tilde{O}|, |B|-i^*\}.
\end{equation}

Then,
\begin{align*}
\ff{O\cup B} - \ff{B} &\le \ff{A_{i^*}} + \sum_{o\in O\setminus \left(B\cup A_{i^*}\right)}\marge{o}{B}\\
&\le \ff{A_{i^*}} + \sum_{i = 1}^{\min\{|\tilde{O}|, |B|\}-i^*}\marge{b_{i + i^*}}{B_{i+i^*-1}}/(1-\epsi) + \frac{\epsi M}{1-\epsi}\\
&\le \frac{1}{1-\epsi}\left(\ff{A_{i^*}} + \ff{B}-\ff{B_{i^*}} + \epsi \ff{O}\right),\numberthis \label{inq:tgone-11}
\end{align*}
where the first inequality follows from Inequality~\eqref{inq:tgone-6};
the second inequality follows from Inequalities~\eqref{inq:tgone-7} and~\eqref{inq:tgone-9};
and the last inequality follows from $M\le \ff{O}$.

By Inequalities~\eqref{inq:tgone-1},~\eqref{inq:tgone-10} and~\eqref{inq:tgone-11},
it holds that
\begin{align*}
&\ff{O} \le \frac{2-\epsi}{1-\epsi}\left(\ff{A} + \ff{B}\right) + \frac{2\epsi}{1-\epsi}\ff{O}\\
\Rightarrow &\ff{S} \ge \left(\frac{1}{4} - \frac{5}{2(4-2\epsi)}\epsi\right)\ff{O}
\ge \left(\frac{1}{4}-\epsi\right)\ff{O}\tag{$\epsi < 1/2$}
\end{align*}

\textbf{Case 2: $0\le j^* < i^* < k$; Fig.~\ref{fig:gdone-2}.}

First, we bound $\ff{O\cup A}$. 
Since $B_{j^*} \subseteq O$, by submodularity
\begin{equation}\label{inq:tgone-20}
\ff{O\cup A} - \ff{A} \le \marge{B_{j^*}}{A} + \marge{O\setminus B_{j^*}}{A}
\le \ff{B_{j^*}}+ \sum_{o\in O\setminus \left(A\cup B_{j^*}\right)}\marge{o}{A}.
\end{equation}
Next, we bound $\marge{o}{A}$ for each $o\in O\setminus \left(A\cup B_{j^*}\right)$.

Let $\tilde{O} = O\setminus \left(A \cup B_{j^*}\right)$.
Since $i^* > j^*\ge 0$, 
it holds that $|\tilde{O}|\le k-j^*-1$.
Since $b_{j^*+1}\not \in O$,
we can order $\tilde{O}$ as $\{o_1, o_2, \ldots\}$ such that $o_i \not \in B_{i+j^*}$,
for all $1\le i \le |\tilde{O}|$.
If $|A| < k$, the algorithm terminates with $\tau_1 < \frac{\epsi M}{k}$.
Thus, it follows that
\begin{equation}\label{inq:tgone-21}
\marge{o_i}{A} < \frac{\epsi M}{k(1-\epsi)}, \forall |A|-j^*-1 < i \le |\tilde{O}|.
\end{equation}

Next, consider tuple $(o_i, a_{i + j^*+1}, A_{i+j^*})$,
for any $1\le i \le \min\{|\tilde{O}|, |A|-j^*-1\}$.
Since $\tau_1^{a_{i + j^*+1}}$ is the threshold value when $a_{i + j^*+1}$ is added,
it holds that 
\begin{equation}\label{inq:tgone-22}
\marge{a_{i + j^*+1}}{A_{i+j^*}} \ge \tau_1^{a_{i + j^*+1}},
\forall 1\le i\le |A| - j^*-1.
\end{equation}
Then, we show that $\marge{o_i}{A_{i+j^*}} < \tau_1^{a_{i + j^*+1}}/(1-\epsi)$ always holds
for any $1\le i \le \min\{|\tilde{O}|, |A|-j^*-1\}$.

Since $M = \max_{x\in \uni}\ff{\{x\}}$,
if $\tau_1^{a_{i + j^*+1}} \ge M$,
it always holds that $\marge{o_i}{A_{i+j^*}} < M/(1-\epsi)\le \tau_1^{a_{i + j^*+1}}/(1-\epsi)$.
If $\tau_1^{a_{i + j^*+1}} < M$, 
since $o_i \not \in B_{i+j^*}$,
$o_i$ is not considered to be added to $A$ with threshold value $\tau_1^{a_{i + j^*+1}}/(1-\epsi)$.
Then, by submodularity,
$\marge{o_i}{A_{i+j^*}} < \tau_1^{a_{i + j^*+1}}/(1-\epsi)$.
Therefore, by submodularity and Inequality~\eqref{inq:tgone-22},
it holds that 
\begin{equation}\label{inq:tgone-23}
\marge{o_i}{A} \le \marge{o_i}{A_{i+j^*}} < \marge{a_{i + j^*+1}}{A_{i+j^*}}/(1-\epsi), \forall 1\le i \le \min\{|\tilde{O}|, |A|-j^*-1\}.
\end{equation}

Then,
\begin{align*}
\ff{O\cup A} - \ff{A} &\le \ff{B_{j^*}} + \sum_{o\in O\setminus \left(A\cup B_{j^*}\right)}\marge{o}{A}\\
&\le \ff{B_{j^*}} + \sum_{i = 1}^{\min\{|\tilde{O}|, |A|-j^*-1\}}\marge{a_{i + j^*+1}}{A_{i+j^*}}/(1-\epsi) + \frac{\epsi M}{1-\epsi}\\
&\le \frac{1}{1-\epsi}\left(\ff{B_{j^*}} + \ff{A}-\ff{A_{j^*+1}} + \epsi \ff{O}\right),\numberthis \label{inq:tgone-24}
\end{align*}
where the first inequality follows from Inequality~\eqref{inq:tgone-20};
the second inequality follows from Inequalities~\eqref{inq:tgone-21} and~\eqref{inq:tgone-23};
and the last inequality follows from $M\le \ff{O}$.

Second, we bound $\ff{O\cup B}$.
Since $A_{j^*+1} \subseteq O$, by submodularity
\begin{equation}\label{inq:tgone-25}
\ff{O\cup B} - \ff{B} \le \marge{A_{j^*+1}}{B} + \marge{O\setminus A_{j^*+1}}{B}
\le \ff{A_{j^*+1}}+ \sum_{o\in O\setminus \left(B\cup A_{j^*+1}\right)}\marge{o}{B}.
\end{equation}
Next, we bound $\marge{o}{B}$ for each $o\in O\setminus \left(B\cup A_{j^*+1}\right)$.

Let $\tilde{O} = O\setminus \left(B\cup A_{j^*+1}\right)$.
Obviously, it holds that $|\tilde{O}|\le k-j^*-1$.
Then, order $\tilde{O}$ as $\{o_1, o_2, \ldots\}$ such that $o_i \not \in A_{i+j^*}$,
for all $1\le i \le |\tilde{O}|$.
If $|B| < k$, the algorithm terminates with $\tau_2 < \frac{\epsi M}{k}$.
Thus, it follows that
\begin{equation}\label{inq:tgone-26}
\marge{o_i}{B} < \frac{\epsi M}{k(1-\epsi)}, \forall |B|-j^*-1 < i \le |\tilde{O}|.
\end{equation}

Next, consider tuple $(o_i, b_{i + j^*}, B_{i+j^*-1})$,
for any $1\le i \le \min\{|\tilde{O}|, |B|-j^*-1\}$.
Since $\tau_2^{b_{i + j^*}}$ is the threshold value when $b_{i + j^*}$ is added,
it holds that 
\begin{equation}\label{inq:tgone-27}
\marge{b_{i + j^*}}{B_{i+j^*-1}} \ge \tau_2^{b_{i + j^*}},
\forall 1\le i\le |B| - j^*-1.
\end{equation}
Then, we show that $\marge{o_i}{B_{i+j^*-1}} < \tau_2^{b_{i + j^*}}/(1-\epsi)$ always holds
for any $1\le i \le \min\{|\tilde{O}|, |B|-j^*-1\}$.

Since $M = \max_{x\in \uni}\ff{\{x\}}$,
if $\tau_2^{b_{i+j^*}} \ge M$,
it always holds that $\marge{o_i}{B_{i+j^*-1}} < M/(1-\epsi)\le \tau_2^{b_{i+j^*}}/(1-\epsi)$.
If $\tau_2^{b_{i+j^*}} < M$, 
since $o_i \not \in A_{i+j^*}$,
$o_i$ is not considered to be added to $B$ with threshold value $\tau_2^{b_{i+j^*}}/(1-\epsi)$.
Then, by submodularity,
$\marge{o_i}{B_{i+j^*-1}} < \tau_2^{b_{i+j^*}}/(1-\epsi)$.
Therefore, by submodularity and Inequality~\eqref{inq:tgone-27},
it holds that 
\begin{equation}\label{inq:tgone-28}
\marge{o_i}{B} \le \marge{o_i}{B_{i+j^*-1}} < \marge{b_{i + j^*}}{B_{i+j^*-1}}/(1-\epsi), \forall 1\le i \le \min\{|\tilde{O}|, |B|-j^*-1\}.
\end{equation}

Then,
\begin{align*}
\ff{O\cup B} - \ff{B} &\le \ff{A_{j^*+1}} + \sum_{o\in O\setminus \left(B\cup A_{j^*+1}\right)}\marge{o}{B}\\
&\le \ff{A_{j^*+1}} + \sum_{i = 1}^{\min\{|\tilde{O}|, |B|-j^*-1\}}\marge{b_{i + j^*}}{B_{i+j^*-1}}/(1-\epsi) + \frac{\epsi M}{1-\epsi}\\
&\le \frac{1}{1-\epsi}\left(\ff{A_{j^*+1}} + \ff{B}-\ff{B_{i^*}} + \epsi \ff{O}\right),\numberthis \label{inq:tgone-29}
\end{align*}
where the first inequality follows from Inequality~\eqref{inq:tgone-25};
the second inequality follows from Inequalities~\eqref{inq:tgone-26} and~\eqref{inq:tgone-28};
and the last inequality follows from $M\le \ff{O}$.

By Inequalities~\eqref{inq:tgone-1},~\eqref{inq:tgone-24} and~\eqref{inq:tgone-29},
it holds that
\begin{align*}
&\ff{O} \le \frac{2-\epsi}{1-\epsi}\left(\ff{A} + \ff{B}\right) + \frac{2\epsi}{1-\epsi}\ff{O}\\
\Rightarrow &\ff{S} \ge \left(\frac{1}{4} - \frac{5}{2(4-2\epsi)}\epsi\right)\ff{O}
\ge \left(\frac{1}{4}-\epsi\right)\ff{O}\tag{$\epsi < 1/2$}
\end{align*}

Therefore, in both cases, it holds that
\[\ff{S} \ge \left(\frac{1}{4}-\epsi\right)\ff{O} .\]
\end{proof}


\subsection{Simplified Fast \itg with $1/e-\epsi$ Approximation Ratio (Alg.~\ref{alg:tgtwo})}
\begin{algorithm}[ht]
    \KwIn{evaluation oracle $f:2^{\uni} \to \reals$, constraint $k$, error $\epsi$}
    \Init{$G_0\gets \emptyset$, $\epsi'\gets \frac{\epsi}{2}$, $m \gets \left\lfloor\frac{k}{\ell}\right\rfloor$, $\ell\gets \left \lceil\frac{4}{e\epsi'}\right \rceil$,
    $M\gets \max_{x\in \uni} \ff{\{x\}}$}
    \For{$i\gets 1$ to $\ell$}{
        $\tau_l \gets M, \forall l\in [\ell]$\;
        $A_{l}\gets G_{i-1}, \forall l \in [\ell]$\;
        \For{$j\gets 1$ to $m$}{
            \For{$l\gets 1$ to $\ell$}{
                \While{$\tau_l \ge \frac{\epsi' M}{k}$ and $|A_l\setminus G_{i-1}| < m$}{
                    \If{$\exists x \in \uni\setminus\left(\bigcup_{r\in [\ell]} A_r\right)$ \st $\marge{a}{A_l} \ge \tau_l$}{
                    $A_l\gets A_l+ x$\;
                    \textbf{break}\;}
                    \lElse{$\tau_l \gets (1-\epsi')\tau_l$}
                }
            }
        }
        $G_i\gets$ a random set in $\{A_l\}_{l\in [\ell]}$\;
    }
    \Return{$G_\ell$}
    \caption{A nearly-linear time, $(1/e-\epsi)$-approximation algorithm.}
    \label{alg:tgtwo}
\end{algorithm}
\thmtgtwo*
\begin{proof}
When $k\,\text{mod}\,\ell > 0$, the algorithm returns an approximation with a size constraint of 
$\ell\cdot\left\lfloor \frac{k}{\ell}\right\rfloor$, where by Proposition~\ref{prop:dif-opt},
\begin{equation}\label{inq:tgtwo-dif-opt}
\ff{O'} \ge \left(1-\frac{\ell}{k}\right)\ff{O}, 
O' = \argmax\limits_{S\subseteq \uni, |S|\le \ell\cdot \left\lfloor \frac{k}{\ell} \right\rfloor}\ff{S}.
\end{equation}
In the following, we only consider the case where $k\,\text{mod}\,\ell = 0$.

At every iteration of the outer for loop,
$\ell$ solutions are constructed, with each solution being augmented
by at most $k/\ell$ elements.
To bound the marginal gain of the optimal set $O$ on each solution set $A_l$,
we consider partitioning $O$ into $\ell$ subsets.
We formalize this partition in the following claim,
which yields a result analogous to Claim~\ref{claim:par-A} presented in 
Section~\ref{sec:greedy-1/e}.
Specifically, the claim states that the optimal set $O$ can be evenly 
divided into $\ell$ subsets,
where each subset only overlaps with only one solution set.
\begin{claim}
At an iteration $i$ of the outer for loop in Alg.~\ref{alg:tgtwo},
let $G_{i-1}$ be $G$ at the start of this iteration,
and $A_{l}$ be the set at the end of this iteration,
for each $l\in [\ell]$.
% Add dummy elements to $O\setminus G_{i-1}$ until its size equals $k$.
The set $O\setminus G_{i-1}$ can then be split into $\ell$ pairwise disjoint sets $\{O_1, \ldots, O_\ell\}$
such that $|O_l| \le\frac{k}{\ell}$ and $\left(O\setminus G_{i-1}\right) \cap \left(A_{l}\setminus G_{i-1}\right) \subseteq O_l$, for all $l \in [\ell]$.
\end{claim}
Next, based on such partition, we introduce the following lemma, 
which provides a bound on the marginal gain of any subset $O_{l_1}$ 
with respect to any solution set $A_{l_2}$,
where $1\le l_1, l_2 \le \ell$.
\begin{lemma}\label{lemma:tg-par-A}
Fix on $G_{i-1}$ for an iteration $i$ of the outer for loop in Alg.~\ref{alg:tgtwo}.
Following the definition in Claim~\ref{claim:par-A}, it holds that
\begin{align*}
\text{1) }&\marge{O_{l}}{A_{l}}\le \frac{\marge{A_{l}}{G_{j-1}}}{1-\epsi'}+\frac{\epsi' M}{(1-\epsi')\ell}, \forall 1\le l \le \ell,\\
\text{2) }&\marge{O_{l_2}}{A_{l_1}} + \marge{O_{l_1}}{A_{l_2}} \le \frac{1}{1-\epsi'}\left(1+\frac{1}{m}\right)\left(\marge{A_{l_1}}{G_{i-1}}+\marge{A_{l_2}}{G_{i-1}}\right)
+\frac{2\epsi' M}{(1-\epsi')\ell}, \forall 1\le l_1 < l_2 \le \ell.
\end{align*}
\end{lemma}
Followed by the above lemma, 
we provide the recurrence of $\ex{\ff{G_i}}$ and $\ex{\ff{O\cup G_i}}$.
\begin{lemma}\label{lemma:tg-recur}
For any iteration $i$ of the outer for loop in Alg.~\ref{alg:tgtwo},
it holds that
\begin{align*}
\text{1) } & \ex{\ff{O\cup G_i}}\ge \left(1-\frac{1}{\ell}\right) \ex{\ff{O\cup G_{i-1}}}\\
\text{2) } & \ex{\ff{G_i} - \ff{G_{i-1}}}
\ge\frac{1}{1+\frac{\ell}{1-\epsi'}}\left(1-\frac{1}{m+1}\right)\left(\left(1-\frac{1}{\ell}\right)  \ex{\ff{O\cup G_{i-1}}} - \ex{\ff{G_{i-1}}} - \frac{\epsi'}{1-\epsi'}\ff{O}\right).
\end{align*}
\end{lemma}
By solving the recurrence in Lemma~\ref{lemma:tg-recur},
we calculate the approximation ratio of the algorithm as follows,
\begin{align*}
&\ex{\ff{G_{i}}}  \ge \left(1-\frac{1}{\ell}\right) \ex{\ff{G_{i-1}}}
+ \frac{1}{1+\frac{\ell}{1-\epsi'}}\left(1-\frac{1}{m+1}\right)\left(\left(1-\frac{1}{\ell}\right)^i - \frac{\epsi'}{1-\epsi'}\right)\ff{O}\\
\Rightarrow& \ex{\ff{G_\ell}} \ge \frac{\ell}{1+\frac{\ell}{1-\epsi'}}\left(1-\frac{1}{m+1}\right)\left(\left(1-\frac{1}{\ell}\right)^\ell - \frac{\epsi'}{1-\epsi'}\left(1-\left(1-\frac{1}{\ell}\right)^\ell\right)\right)\ff{O}\\
&\hspace*{4em} \ge \frac{\ell-1}{1+\frac{\ell}{1-\epsi'}}\left(1-\frac{1}{m+1}\right)\left(e^{-1} - \frac{\epsi'}{1-\epsi'}\left(1-e^{-1}\right)\right)\ff{O}\\
&\hspace*{4em} \ge \frac{1}{1-\frac{\ell}{k}}\left(1-\epsi' - \frac{2}{\ell}\right)\left(1-\frac{\ell}{k}\right)^2\left(e^{-1} - \frac{\epsi'}{1-\epsi'}\left(1-e^{-1}\right)\right) \ff{O}\\
% &\hspace*{4em} \ge \frac{1}{1-\frac{\ell}{k}}\left(1-\epsi' - \frac{2}{\ell}\right)\left(1-\frac{2\ell}{k}\right)\left(e^{-1} - \frac{\epsi'}{1-\epsi'}\left(1-e^{-1}\right)\right) \ff{O}\\
&\hspace*{4em} \ge \frac{1}{1-\frac{\ell}{k}}\left(1-\epsi' - \frac{2}{\ell}-\frac{2(1-\epsi')\ell}{k}\right)\left(e^{-1} - \frac{\epsi'}{1-\epsi'}\left(1-e^{-1}\right)\right) \ff{O}\\
&\hspace*{4em} \ge \frac{1}{1-\frac{\ell}{k}} \left(1-(e+1)\epsi'\right)\left(e^{-1} - \frac{\epsi'}{1-\epsi'}\left(1-e^{-1}\right)\right) \ff{O}\tag{$\ell \ge \frac{2}{e\epsi'}, k \ge \frac{2(1-\epsi')\ell}{e\epsi'-\frac{2}{\ell}}$}\\
&\hspace*{4em} \ge \frac{1}{1-\frac{\ell}{k}} \left(e^{-1}-\epsi\right)\ff{O}\tag{$\epsi' = \frac{\epsi}{2}$}.
\end{align*}
By Inequality~\ref{inq:tgtwo-dif-opt},
the approximation ratio of Alg.~\ref{alg:tgtwo} is $e^{-1}-\epsi$.
\end{proof}

In the rest of this section, we provide the proofs for 
Lemma~\ref{lemma:tg-par-A} and~\ref{lemma:tg-recur}.
\begin{proof}[Proof of Lemma~\ref{lemma:tg-par-A}]
At iteration $i$ of the outer for loop,
let $A_l$ be the set at the end of iteration $i$,
$a_{l, j}$ be the $j$-th element added to $A_l$,
$\tau_l^j$ be the threshold value of $\tau_l$ when $a_{l, j}$ is added to $A_l$,
and $A_{l, j}$ be $A_l$ after $a_{l, j}$ is added to $A_l$.
Let $c_l^* = \max\{c\in [m]:A_{l, c}\setminus G_{i-1}\subseteq O_l\}$.

First, we prove that the first inequality holds.
For each $l\in [\ell]$, order the elements in $O_l$ as $\{o_1, o_2, \ldots\}$
such that $o_j \not \in A_{l, j-1}$ for any $1\le j \le |A_l\setminus G_{i-1}|$,
and $o_j\not \in A_l$ for any $|A_l\setminus G_{i-1}| < j \le m$.

When $1\le j \le |A_l\setminus G_{i-1}|$, by Claim~\ref{claim:par-A},
each $o_j$ is either added to $A_l$ or not in any solution set.
Since $\tau_l$ is initialized with the maximum marginal gain $M$,
$o_j$ is not considered to be added to $A_l$ with threshold value 
$\tau_l^j/(1-\epsi')$.
Therefore, by submodularity it holds that
\begin{equation}\label{inq:tgtwo-1}
\marge{o_j}{A_{l, j-1}} < \tau_l^j/(1-\epsi')\le \marge{a_{l,j}}{A_{l,j-1}}/(1-\epsi'),
\forall 1\le j \le |A_l\setminus G_{i-1}|.
\end{equation}

When $|A_l\setminus G_{i-1}| <  m$,
the minimum value of $\tau_l$ is less than $\frac{\epsi' M}{k}$.
Then, for any $|A_l\setminus G_{i-1}| < j \le m$,
$o_j$ is not considered to be added to $A_l$ with threshold value less than $\frac{\epsi' M}{(1-\epsi')k}$.
It follows that 
\begin{equation}\label{inq:tgtwo-2}
\marge{o_j}{A_l} \le \frac{\epsi' M}{(1-\epsi')k},
\forall |A_l\setminus G_{i-1}| < j \le m.
\end{equation}

Then,
\begin{align*}
\marge{O_{l}}{A_{l}} &\le \sum_{o_j\in O_l} \marge{o_j}{A_{l}} \tag{Proposition~\ref{prop:sum-marge}}\\
&\le \sum_{j=1}^{|A_l\setminus G_{i-1}|} \marge{o_j}{A_{l, j-1}} + 
\sum_{j=|A_l\setminus G_{i-1}|+1}^m \marge{o_j}{A_{l}}\tag{Submodularity}\\
&\le \sum_{j=1}^{|A_l\setminus G_{i-1}|}\frac{\marge{a_{l,j}}{A_{l,j-1}}}{1-\epsi'}+\frac{\epsi' M}{(1-\epsi')\ell}
\tag{Inequalities~\eqref{inq:tgtwo-1} and~\eqref{inq:tgtwo-2}}\\
&= \frac{\marge{A_{l}}{G_{j-1}}}{1-\epsi'}+\frac{\epsi' M}{(1-\epsi')\ell}.
\end{align*}
The first inequality holds.

In the following, we prove that the second inequality holds.
For any $1\le l_1\le l_2\le \ell$,
we analyze two cases of the relationship between $c_{l_1}^* $ and $ c_{l_2}^*$ in the following.

% \textbf{Case 1: $c_{l_1}^* = c_{l_2}^* = m$.}
% Then, $O_{l_1} = A_{l_1}\setminus G_{i-1}$ and $O_{l_2} = A_{l_2}\setminus G_{i-1}$.
% By submodularity,
% \[\marge{O_{l_2}}{A_{l_1}} + \marge{O_{l_1}}{A_{l_2}} \le \marge{O_{l_2}}{G_{i-1}} + \marge{O_{l_1}}{G_{i-1}} = \marge{A_{l_1}}{G_{i-1}} + \marge{A_{l_2}}{G_{i-1}}.\]
% Therefore, the lemma holds in this case.

\textbf{Case 1: $c_{l_1}^* \le c_{l_2}^*$; left half part in Fig.~\ref{fig:gdtwo}.}

First, we bound $\marge{O_{l_1}}{A_{l_2}}$.
Order the elements in $O_{l_1}\setminus A_{l_1, c_{l_1}^*}$ as $\{o_1, o_2, \ldots\}$ such that $o_j \not \in A_{l_1, c_{l_1}^*+j}$.
(Refer to the gray block with a dotted edge in the top left corner of Fig.~\ref{fig:gdtwo} for $O_{l_1}$.
If $c_{l_1}^*+j$ is greater than the number of elements added to $A_{l_1}$,
$A_{l_1, c_{l_1}^*+j}$ refers to $A_{l_1}$.)
Note that, since $A_{l_1, c_{l_1}^*} \subseteq O_{l_1}$,
it follows that $|O_{l_1}\setminus A_{l_1, c_{l_1}^*}| \le m - c_{l_1}^*$.

When $1 \le j \le |A_{l_2}\setminus G_{i-1}| - c_{l_1}^*$,
since each $o_j$ is either added to $A_{l_1}$ or not in any solution set by Claim~\ref{claim:par-A}
and $\tau_{l_2}$ is initialized with the maximum marginal gain $M$,
$o_j$ is not considered to be added to $A_{l_2}$ with threshold value $\tau_{l_2}^{c_{l_1}^* + j}/(1-\epsi')$.
Therefore, it holds that 
\begin{equation}\label{inq:tgtwo-case2-1}
\marge{o_j}{A_{l_2, c_{l_1}^*+j-1}} < \frac{\tau_{l_2}^{c_{l_1}^* + j}}{1-\epsi'} \le \frac{\marge{a_{l_2, c_{l_1}^* + j}}{A_{l_2, c_{l_1}^* + j-1}}}{1-\epsi'}, \forall 1\le j\le |A_{l_2}\setminus G_{i-1}|-c_{l_1}^*.
\end{equation}

When $|A_{l_2}\setminus G_{i-1}| < m$ and $|A_{l_2}\setminus G_{i-1}|-c_{l_1}^* < j\le m-c_{l_1}^*$,
this iteration ends with $\tau_{l_2} < \frac{\epsi' M}{k}$ and
$o_j$ is never considered to be added to $A_{l_2}$.
Thus, it holds that
\begin{equation}\label{inq:tgtwo-case2-3}
\marge{o_j}{A_{l_2}} < \frac{\epsi' M}{(1-\epsi')k}, 
\forall |A_{l_2}\setminus G_{i-1}|-c_{l_1}^* < j \le m-c_{l_1}^*.
\end{equation}

Then,
\begin{align*}
\marge{O_{l_1}}{A_{l_2}} &\le \marge{A_{l_1, c_{l_1}^*}}{A_{l_2}}  + \sum_{o_j \in O_{l_1}\setminus A_{l_1, c_{l_1}^*}}\marge{o_j}{A_{l_2}} \tag{Proposition~\ref{prop:sum-marge}}\\
&\le \marge{A_{l_1, c_{l_1}^*}}{G_{i-1}} + \sum_{j = 1}^{|A_{l_2}\setminus G_{i-1}|-c_{l_1}^*}\marge{o_j}{A_{l_2, , c_{l_1}^*+j-1}} + \sum_{j=|A_{l_2}\setminus G_{i-1}|-c_{l_1}^*+1}^{m-c_{l_1}^*} \marge{o_j}{A_{l_2}} \tag{submodularity}\\
&\le \ff{A_{l_1, c_{l_1}^*}}-\ff{G_{i-1}} + \sum_{j = 1}^{|A_{l_2}\setminus G_{i-1}|-c_{l_1}^*}\frac{\marge{a_{l_2, c_{l_1}^*+j}}{A_{l_2, , c_{l_1}^*+j-1}}}{1-\epsi'} + \frac{\epsi' M}{(1-\epsi')\ell} \tag{Inequality~\eqref{inq:tgtwo-case2-1} and~\eqref{inq:tgtwo-case2-3}}\\
& \le \ff{A_{l_1, c_{l_1}^*}}-\ff{G_{i-1}} + \frac{\ff{A_{l_2}} - \ff{A_{l_2, c_{l_1}^*}}}{1-\epsi'} + \frac{\epsi' M}{(1-\epsi')\ell} \numberthis \label{inq:tgtwo-case2-6}
\end{align*}

Similarly, we bound $\marge{O_{l_2}}{A_{l_1}}$ below.
Order the elements in $O_{l_2}\setminus A_{l_2, c_{l_1}^*}$ as $\{o_1, o_2, \ldots\}$ such that $o_j \not \in A_{l_2, c_{l_1}^*+j-1}$.
(See the gray block with a dotted edge in the bottom left corner of Fig.~\ref{fig:gdtwo} for $O_{l_2}$.
If $c_{l_1}^*+j-1$ is greater than the number of elements added to $A_{l_2}$,
$A_{l_2, c_{l_1}^*+j-1}$ refers to $A_{l_2}$.)
Note that, since $A_{l_2, c_{l_1}^*} \subseteq O_{l_2}$,
it follows that $|O_{l_2}\setminus A_{l_2, c_{l_1}^*}| \le m - c_{l_1}^*$.

When $1 \le j \le |A_{l_1}\setminus G_{i-1}|-c_{l_1}^*$,
since each $o_j$ is either added to $A_{l_2}$ or not in any solution set by Claim~\ref{claim:par-A}
and $\tau_{l_1}$ is initialized with the maximum marginal gain $M$,
$o_j$ is not considered to be added to $A_{l_1}$ with threshold value $\tau_{l_1}^{c_{l_1}^* + j}/(1-\epsi')$.
Therefore, it holds that 
\begin{equation}\label{inq:tgtwo-case2-4}
\marge{o_j}{A_{l_1, c_{l_1}^*+j-1}} < \frac{\tau_{l_1}^{c_{l_1}^* + j}}{1-\epsi'} \le \frac{\marge{a_{l_1, c_{l_1}^* + j}}{A_{l_1, c_{l_1}^* + j-1}}}{1-\epsi'}, \forall 1\le j\le |A_{l_2}\setminus G_{i-1}|-c_{l_1}^*.
\end{equation}

When $|A_{l_1}\setminus G_{i-1}| < m$ and $|A_{l_1}\setminus G_{i-1}|-c_{l_1}^* < j\le m-c_{l_1}^*$,
this iteration ends with $\tau_{l_1} < \frac{\epsi' M}{k}$
and $o_j$ is never considered to be added to $A_{l_1}$.
Thus, it holds that
\begin{equation}\label{inq:tgtwo-case2-5}
\marge{o_j}{A_{l_1}} < \frac{\epsi' M}{(1-\epsi')k}, \forall |A_{l_1}\setminus G_{i-1}|-c_{l_1}^* < j \le m-c_{l_1}^*.
\end{equation}

Then,
\begin{align*}
\marge{O_{l_2}}{A_{l_1}} &\le \marge{A_{l_2, c_{l_1}^*}}{A_{l_1}}  + \sum_{o_j \in O_{l_2}\setminus A_{l_2, c_{l_1}^*}}\marge{o_j}{A_{l_1}} \tag{Proposition~\ref{prop:sum-marge}}\\
&\le \marge{A_{l_2, c_{l_1}^*}}{G_{i-1}} + \sum_{j = 1}^{|A_{l_1}\setminus G_{i-1}|-c_{l_1}^*}\marge{o_j}{A_{l_1, c_{l_1}^*+j-1}} + \sum_{j=|A_{l_1}\setminus G_{i-1}|-c_{l_1}^*+1}^{m-c_{l_1}^*} \marge{o_j}{A_{l_1}} \tag{submodularity}\\
&\le \ff{A_{l_2, c_{l_1}^*}}-\ff{G_{i-1}} + \sum_{j = 1}^{|A_{l_1}\setminus G_{i-1}|-c_{l_1}^*}\frac{\marge{a_{l_1, c_{l_1}^*+j}}{A_{l_1, , c_{l_1}^*+j-1}}}{1-\epsi'} + \frac{\epsi' M}{(1-\epsi')\ell} \tag{Inequality~\eqref{inq:tgtwo-case2-4} and~\eqref{inq:tgtwo-case2-5}}\\
& \le \ff{A_{l_2, c_{l_1}^*}}-\ff{G_{i-1}} + \frac{\ff{A_{l_1}} - \ff{A_{l_1, c_{l_1}^*}}}{1-\epsi'} + \frac{\epsi' M}{(1-\epsi')\ell} \numberthis \label{inq:tgtwo-case2-7}
\end{align*}

By Inequalities~\eqref{inq:tgtwo-case2-6} and~\eqref{inq:tgtwo-case2-7},
\begin{align*}
\marge{O_{l_1}}{A_{l_2}}+\marge{O_{l_2}}{A_{l_1}}
\le \frac{1}{1-\epsi'}\left(\marge{A_{l_1}}{G_{i-1}}+\marge{A_{l_2}}{G_{i-1}}\right)
+\frac{2\epsi' M}{(1-\epsi')\ell}
\end{align*}

Thus, the lemma holds in this case.

\textbf{Case 2: $c_{l_1}^* > c_{l_2}^*$; right half part in Fig.~\ref{fig:gdtwo}.}

First, we bound $\marge{O_{l_1}}{A_{l_2}}$.
Order the elements in $O_{l_1}\setminus A_{l_1, c_{l_2}^*+1}$ as $\{o_1, o_2, \ldots\}$ such that $o_j \not \in A_{l_1, c_{l_1}^*+j}$.
(Refer to the gray block with a dotted edge in the top right corner of Fig.~\ref{fig:gdtwo} for $O_{l_1}$.
If $c_{l_1}^*+j$ is greater than the number of elements added to $A_{l_1}$,
$A_{l_1, c_{l_1}^*+j}$ refers to $A_{l_1}$.)
Note that, since $A_{l_1, c_{l_2}^*+1} \subseteq O_{l_1}$,
it follows that $|O_{l_1}\setminus A_{l_1, c_{l_2}^*+1}| \le m - c_{l_2}^*-1$.

When $1 \le j \le |A_{l_2}\setminus G_{i-1}| - c_{l_2}^* - 1$,
since each $o_j$ is either added to $A_{l_1}$ or not in any solution set by Claim~\ref{claim:par-A}
and $\tau_{l_2}$ is initialized with the maximum marginal gain $M$,
$o_j$ is not considered to be added to $A_{l_2}$ with threshold value $\tau_{l_2}^{c_{l_2}^* + j}/(1-\epsi')$.
Therefore, it holds that 
\begin{equation}\label{inq:tgtwo-case3-1}
\marge{o_j}{A_{l_2, c_{l_2}^*+j-1}} < \frac{\tau_{l_2}^{c_{l_2}^* + j}}{1-\epsi'} \le \frac{\marge{a_{l_2, c_{l_2}^* + j}}{A_{l_2, c_{l_2}^* + j-1}}}{1-\epsi'}, \forall 1\le j\le |A_{l_2}\setminus G_{i-1}| - c_{l_2}^* - 1.
\end{equation}

When $|A_{l_2}\setminus G_{i-1}| < m$ and $|A_{l_2}\setminus G_{i-1}|- c_{l_2}^* - 1 < j\le m- c_{l_2}^* - 1$,
this iteration ends with $\tau_{l_2} < \frac{\epsi' M}{k}$ and
$o_j$ is never considered to be added to $A_{l_2}$.
Thus, it holds that
\begin{equation}\label{inq:tgtwo-case3-3}
\marge{o_j}{A_{l_2}} < \frac{\epsi' M}{(1-\epsi')k}, 
\forall |A_{l_2}\setminus G_{i-1}|- c_{l_2}^* - 1 < j \le m- c_{l_2}^* - 1.
\end{equation}

Then,
\begin{align*}
\marge{O_{l_1}}{A_{l_2}} &\le \marge{A_{l_1, c_{l_2}^*}}{A_{l_2}}  + \sum_{o_j \in O_{l_1}\setminus A_{l_1, c_{l_2}^*+1}}\marge{o_j}{A_{l_2}} \tag{Proposition~\ref{prop:sum-marge}}\\
&\le \marge{A_{l_1, c_{l_2}^*}}{G_{i-1}} + \sum_{j = 1}^{|A_{l_2}\setminus G_{i-1}|- c_{l_2}^* - 1}\marge{o_j}{A_{l_2, , c_{l_2}^*+j-1}} + \sum_{j=|A_{l_2}\setminus G_{i-1}|- c_{l_2}^*}^{m- c_{l_2}^* - 1} \marge{o_j}{A_{l_2}} \tag{submodularity}\\
&\le \ff{A_{l_1, c_{l_2}^*}}-\ff{G_{i-1}} + \sum_{j = 1}^{|A_{l_2}\setminus G_{i-1}|- c_{l_2}^* - 1}\frac{\marge{a_{l_2, c_{l_2}^*+j}}{A_{l_2, , c_{l_2}^*+j-1}}}{1-\epsi'} + \frac{\epsi' M}{(1-\epsi')\ell} \tag{Inequality~\eqref{inq:tgtwo-case3-1} and~\eqref{inq:tgtwo-case3-3}}\\
& \le \ff{A_{l_1, c_{l_2}^*}}-\ff{G_{i-1}} + \frac{\ff{A_{l_2}} - \ff{A_{l_2, c_{l_2}^*}}}{1-\epsi'} + \frac{\epsi' M}{(1-\epsi')\ell} \numberthis \label{inq:tgtwo-case3-6}
\end{align*}

Similarly, we bound $\marge{O_{l_2}}{A_{l_1}}$ below.
Order the elements in $O_{l_2}\setminus A_{l_2, c_{l_2}^*}$ as $\{o_1, o_2, \ldots\}$ such that $o_j \not \in A_{l_2, c_{l_2}^*+j}$.
(See the gray block with a dotted edge in the bottom right corner of Fig.~\ref{fig:gdtwo} for $O_{l_2}$.
If $c_{l_2}^*+j$ is greater than the number of elements added to $A_{l_2}$,
$A_{l_2, c_{l_2}^*+j}$ refers to $A_{l_2}$.)
Note that, since $A_{l_2, c_{l_2}^*} \subseteq O_{l_2}$,
it follows that $|O_{l_2}\setminus A_{l_2, c_{l_2}^*}| \le m - c_{l_2}^*$.

When $1 \le j \le |A_{l_1}\setminus G_{i-1}|- c_{l_2}^* - 1$, 
since each $o_j$ is either added to $A_{l_2}$ or not in any solution set by Claim~\ref{claim:par-A}
and $\tau_{l_1}$ is initialized with the maximum marginal gain $M$,
$o_j$ is not considered to be added to $A_{l_1}$ with threshold value $\tau_{l_1}^{c_{l_2}^* + j+1}/(1-\epsi')$.
Therefore, it holds that 
\begin{equation}\label{inq:tgtwo-case3-4}
\marge{o_j}{A_{l_1, c_{l_2}^*+j}} < \frac{\tau_{l_1}^{c_{l_2}^* + j+1}}{1-\epsi'} \le \frac{\marge{a_{l_1, c_{l_2}^* + j+1}}{A_{l_1, c_{l_2}^* + j}}}{1-\epsi'}, \forall 1\le j\le |A_{l_2}\setminus G_{i-1}|- c_{l_2}^* - 1.
\end{equation}
If $|A_{l_1}\setminus G_{i-1}| = m$,
consider the last element $o_{m-c_{l_2}^*}$ in $O_{l_2}\setminus A_{l_2, c_{l_2}^*}$.
Since $o_{m-c_{l_2}^*} \not\in A_{l_2}$ and $o_{m-c_{l_2}^*} \not\in A_{l_1}$, $o_{m-c_{l_2}^*}$ is not considered to be added to 
$A_{l_1}$ with threshold value $\tau_{l_1}^j/(1-\epsi')$ for any $j \in [m]$.
Then,
\begin{equation}\label{inq:tgtwo-case3-2}
\marge{o_{m-c_{l_2}^*}}{A_{l_1}} < \frac{\sum_{j=1}^m \tau_{l_1}^j}{(1-\epsi')m}
\le \frac{\sum_{j=1}^m \marge{a_{l_1, j}}{A_{l_1, j-1}}}{(1-\epsi')m}
 = \frac{\marge{A_{l_1}}{G_{i-1}}}{(1-\epsi')m}.
\end{equation}

When $|A_{l_1}\setminus G_{i-1}| < m$ and $|A_{l_1}\setminus G_{i-1}|- c_{l_2}^* - 1 < j\le m- c_{l_2}^*$,
this iteration ends with $\tau_{l_1} < \frac{\epsi' M}{k}$
and $o_j$ is never considered to be added to $A_{l_1}$.
Thus, it holds that
\begin{equation}\label{inq:tgtwo-case3-5}
\marge{o_j}{A_{l_1}} < \frac{\epsi' M}{(1-\epsi')k}, 
\forall |A_{l_1}\setminus G_{i-1}|- c_{l_2}^* - 1 < j \le m- c_{l_2}^*.
\end{equation}

Then,
\begin{align*}
\marge{O_{l_2}}{A_{l_1}} &\le \marge{A_{l_2, c_{l_2}^*}}{A_{l_1}}  + \sum_{o_j \in O_{l_2}\setminus A_{l_2, c_{l_2}^*}}\marge{o_j}{A_{l_1}} \tag{Proposition~\ref{prop:sum-marge}}\\
&\le \marge{A_{l_2, c_{l_2}^*}}{G_{i-1}} + \sum_{j = 1}^{|A_{l_1}\setminus G_{i-1}|- c_{l_2}^* - 1}\marge{o_j}{A_{l_1, c_{l_2}^*+j-1}} + \sum_{j=|A_{l_1}\setminus G_{i-1}|- c_{l_2}^*}^{m} \marge{o_j}{A_{l_1}} \tag{submodularity}\\
&\le \ff{A_{l_2, c_{l_2}^*}}-\ff{G_{i-1}} + \sum_{j = 1}^{|A_{l_1}\setminus G_{i-1}|- c_{l_2}^* - 1}\frac{\marge{a_{l_1, c_{l_2}^*+j}}{A_{l_1, , c_{l_2}^*+j-1}}}{1-\epsi'}
+ \frac{\marge{A_{l_1}}{G_{i-1}}}{(1-\epsi')m}
+\frac{\epsi' M}{(1-\epsi')\ell} \tag{Inequalities~\eqref{inq:tgtwo-case3-4}-\eqref{inq:tgtwo-case3-5}}\\
& \le \ff{A_{l_2, c_{l_2}^*}}-\ff{G_{i-1}} + \frac{\ff{A_{l_1}} - \ff{A_{l_1, c_{l_2}^*}}}{1-\epsi'} + \frac{\marge{A_{l_1}}{G_{i-1}}}{(1-\epsi')m} + \frac{\epsi' M}{(1-\epsi')\ell} \numberthis \label{inq:tgtwo-case3-7}
\end{align*}

By Inequalities~\eqref{inq:tgtwo-case3-6} and~\eqref{inq:tgtwo-case3-7},
\begin{align*}
\marge{O_{l_1}}{A_{l_2}}+\marge{O_{l_2}}{A_{l_1}}
\le \frac{1}{1-\epsi'}\left(1+\frac{1}{m}\right)\left(\marge{A_{l_1}}{G_{i-1}}+\marge{A_{l_2}}{G_{i-1}}\right)
+\frac{2\epsi' M}{(1-\epsi')\ell}
\end{align*}

Thus, the lemma holds in this case.
\end{proof}

\begin{proof}[Proof of Lemma~\ref{lemma:tg-recur}]
Fix on $G_{i-1}$ at the beginning of this iteration,
Since $\{A_l\setminus G_{i-1}: l\in [\ell]\}$ are pairwise disjoint sets,
by Proposition~\ref{prop:sum-marge}, it holds that
\[\exc{\ff{O\cup G_i}}{G_{i-1}} = \frac{1}{\ell}\sum_{l\in [\ell]}\ff{O\cup A_l} \ge \left(1-\frac{1}{\ell}\right)\ff{O\cup G_{i-1}}.\]
Then, by unfixing $G_{i-1}$, the first inequality holds.

To prove the second inequality, also consider fix on $G_{i-1}$ at the beginning of iteration $i$.
Then,
\begin{align*}
\sum_{l\in [\ell]}\marge{O}{A_l} &\le \sum_{l_1\in [\ell]}\sum_{l_2\in [\ell]}\marge{O_{l_1}}{A_{l_2}}\tag{Proposition~\ref{prop:sum-marge}}\\
& = \sum_{l \in [\ell]}\marge{O_{l}}{A_{l}} + \sum_{1\le l_1< l_2 \le \ell} \left(\marge{O_{l_1}}{A_{l_2}} +\marge{O_{l_2}}{A_{l_1}}\right) \tag{Lemma~\ref{lemma:tg-par-A}}\\
& \le \sum_{l \in [\ell]}\left(\frac{\marge{A_{l}}{G_{i-1}}}{1-\epsi'}+\frac{\epsi' M}{(1-\epsi')\ell}\right)\\
&\hspace*{2em}+\sum_{1\le l_1< l_2 \le \ell} \left(\frac{1}{1-\epsi'}\left(1+\frac{1}{m}\right)\left(\marge{A_{l_1}}{G_{i-1}}
+\marge{A_{l_2}}{G_{i-1}}\right)
+\frac{2\epsi' M}{(1-\epsi')\ell}\right)\tag{Lemma~\ref{lemma:tg-par-A}}\\
&\le \frac{\ell}{1-\epsi'}\left(1+\frac{1}{m}\right)\sum_{l \in [\ell]}\marge{A_{l}}{G_{i-1}} + \frac{\epsi' \ell}{1-\epsi'}\ff{O}\tag{$M \le \ff{O}$}
\end{align*}
\begin{align*}
\Rightarrow \left(1+\frac{\ell}{1-\epsi'}\right)\left(1+\frac{1}{m}\right) \sum_{l\in [\ell]}\marge{A_l}{G_{i-1}} &\ge \sum_{l\in [\ell]}\ff{O\cup A_l} -\ell\ff{G_{i-1}} - \frac{\epsi' \ell}{1-\epsi'}\ff{O}\\
&\ge \left(\ell-1\right)\ff{O\cup G_{i-1}}-\ell\ff{G_{i-1}} - \frac{\epsi' \ell}{1-\epsi'}\ff{O}
\end{align*}
Thus,
\begin{align*}
&\exc{\ff{G_i} - \ff{G_{i-1}}}{G_{i-1}}  = \frac{1}{\ell}\sum_{l \in [\ell]}\marge{A_{l}}{G_{i-1}}\\
&\ge \frac{1}{1+\frac{\ell}{1-\epsi'}} \frac{m}{m+1}\left(\left(1-\frac{1}{\ell}\right)  \ff{O\cup G_{i-1}} - \ff{G_{i-1}} - \frac{\epsi'}{1-\epsi'}\ff{O}\right)\tag{Proposition~\ref{prop:sum-marge}}
\end{align*}
By unfixing $G_{i-1}$, the second inequality holds.
\end{proof}

\section{Analysis of Section~\ref{sec:ptg}} % (fold)
\label{apx:ptg}
In this section, we provide the analysis of our parallel algorithms
introduced in Section~\ref{sec:ptg}.
First, we provide the subroutines used for \ptgoneshort in Appendix~\ref{apx:subroutine}.
Then, we analyze \ptgoneshort, the main parallel procedure,
in Appendix~\ref{apx:ptgone}.
At last, we provide that analysis of $1/4$ and $1/e$ approximation algorithms
in Appendix~\ref{apx:ptgone-guarantee} and Appendix~\ref{apx:ptgtwo},
respectively.
\subsection{Subroutines}\label{apx:subroutine}
\begin{algorithm}[ht]
\Fn{\dist($\{V_l\}_{l\in [\ell]}$)}{
	\KwIn{$V_1, V_2, \ldots, V_\ell \subseteq \uni$}
	\Init{$\mathcal V_1, \mathcal V_2, \ldots, \mathcal V_\ell \gets \emptyset$, $I\gets [\ell]$}
	\For{$i\gets 1$ to $\ell$}{
		$j\gets\argmin_{j\in I}|V_j|$\label{line:dis-index}\;
		$\mathcal V_j \gets$ randomly select $\left\lfloor\frac{|V_j|}{\ell}\right\rfloor$ elements in $V_j\setminus \left(\bigcup_{l \in [\ell]}\mathcal V_j\right)$ \label{line:dis-select}\;
		$I\gets I-j$\;
	}
	\Return{$\left\{\mathcal V_l\right\}_{l\in [\ell]}$}
	}
\caption{Return $\ell$ pairwise disjoint subsets where $|\mathcal V_j| \ge \frac{|V_j|}{2\ell}$ for any $j \in [\ell]$ if $|V_j|\ge 2\ell$}
\label{alg:dist}
\end{algorithm}
\begin{lemma}\label{lemma:dist}
With input $\{V_l\}_{l\in [\ell]}$, where $|V_l|\ge 2\ell$ for each $l\in [\ell]$,
\dist returns $\ell$ pairwise disjoint sets $\{\mathcal V_l\}_{l\in [\ell]}$ \st
$\mathcal V_l\subseteq V_l$ and $|\mathcal V_j| \ge \frac{|V_j|}{2\ell}$.
\end{lemma}


\begin{algorithm}[ht]
\Fn{\prefix($f, \mathcal V, s, \tau, \epsi$)}{
	\KwIn{evaluation oracle $f:2^{\uni} \to \reals$, maximum size $s$, threshold $\tau$, error $\epsi$, candidate pool $\mathcal V$ where $\marge{x}{\emptyset} \ge \tau $ for any $ x\in \mathcal V$}
	\Init{$B[1:s]\gets [\textbf{none}, \ldots, \textbf{none}]$}
	$\mathcal V \gets\left\{v_1, v_2, \ldots\right\} \gets \textbf{random-permutation}(\mathcal V)$\label{line:prefix-permute}\;
	\For{$i\gets 1$ to $s$ in parallel}{
		$T_{i-1} \gets \left\{v_1, \ldots, v_{i-1}\right\}$\;
		\lIf{$\marge{v_i}{T_{i-1}} \ge \tau$}{$B[i]\gets \textbf{true}$}\label{line:prefix-B-true}
		\lElseIf{$\marge{v_i}{T_{i-1}} < 0$}{$B[i]\gets \textbf{false}$}\label{line:prefix-B-false}
	}
	$i^*\gets \max \{i: \#\text{\textbf{true}s in }B[1:i] \ge (1-\epsi) i\}$\label{line:prefix-istar}\;
	\Return{$i^*$, $B$}
}
\caption{Select a prefix of $\mathcal V$ \st its average marginal gain is greater than $(1-\epsi)\tau$, and with a probability of $1/2$, more than an $\epsi/2$-fraction of $\mathcal V$ has a marginal gain less than $\tau$ relative to the prefix.}
\label{alg:prefix}
\end{algorithm}
Since the procedure \prefix is identical to Lines 8-15 in \ts~\citep{Chen2024},
the following two lemmata hold in a manner similar to Lemma 4 and 5 in \citet{Chen2024}.
\begin{lemma}\label{lemma:prefix-filter}
In \prefix, given $\mathcal V$ after \textbf{random-permutation} in Line~\ref{line:prefix-permute},
let $D_i = \left\{x\in \mathcal V: \marge{x}{T_i} < \tau\right\}$.
It holds that $|D_0|=0$, $|D_{|\mathcal V|}| = |\mathcal V|$, and $|D_{i-1}|\le |D_i|$.
\end{lemma}
\begin{lemma}\label{lemma:prefix-prob}
In \prefix, following the definition of $D_i$ in Lemma~\ref{lemma:prefix-filter},
let $t = \min\{i: |D_i| \ge \epsi |\mathcal V|/2\}$.
It holds that $\prob{i^* < \min\{s,t\}} \le 1/2$.
\end{lemma}
As defined in Lemma~\ref{lemma:prefix-filter},
$D_i$ contains the elements in $\mathcal V$
which can be filtered out by threshold value $\tau$
regarding the prefix $T_i$.
Therefore, Lemma~\ref{lemma:prefix-prob} indicates that,
with a probability of at least $1/2$,
$i^* = s$ or
more than $\epsi/2$-fraction of $\mathcal V$
can be filtered out if prefix $T_{i^*}$ is added to the solution.

\begin{algorithm}[ht]
\Fn{\update($f, V, \tau, \epsi$)}{
\KwIn{evaluation oracle $f:2^{\uni} \to \reals$, candidate set $V$, threshold value $\tau$, error $\epsi$}
	\For(\tcp*[f]{Update candidate sets with threshold values}){$j\gets 1$ to $\ell$ in parallel}{
		$V \gets \left\{x\in V : \marge{x}{\emptyset} \ge \tau\right\}$ \label{line:update-filter}\;
		\While{$|V| = 0$}{
			$\tau \gets (1-\epsi)\tau$\;
			$V \gets \left\{x\in \uni : \marge{x}{\emptyset} \ge \tau\right\}$\;
		}
	}
	\Return{$V, \tau$}
}
\caption{Update candidate set $V$ with threshold value $\tau$}
\label{alg:update}
\end{algorithm}
\subsection{Analysis of Alg.~\ref{alg:ptgone}}\label{apx:ptgone}
We provide the guarantees achieved by \ptgoneshort as follows,
\begin{lemma}\label{lemma:ptgone}
With input $(f, m, \ell, \tau_{\min}, \epsi)$, \ptgone (Alg.~\ref{alg:ptgone})
runs in $\oh{\ell^2\epsi^{-2}\log(n)\log\left(\frac{M}{\tau_{\min}}\right)}$ adaptive rounds and $\oh{\ell^3 \epsi^{-2}n\log(n)\log\left(\frac{M}{\tau_{\min}}\right)}$ queries with a probability of $1-1/n$,
and terminates with $\{(A_l, A_l'): l\in [\ell]\}$ \st
{\small
\begin{enumerate}
\item $A_l'\subseteq A_l$, $\marge{A_l'}{\emptyset} \ge \marge{A_l}{\emptyset}, \forall 1\le l \le \ell$, and $\{A_l: l\in [\ell]\}$ are pairwise disjoint sets,
\item $\marge{O_{l}}{A_{l}}\le \frac{\marge{A_{l}'}{\emptyset}}{(1-\epsi)^2}+\frac{m\cdot\tau_{\min}}{1-\epsi}, \forall 1\le l \le \ell$,
\item $\marge{O_{l_2}}{A_{l_1}} + \marge{O_{l_1}}{A_{l_2}} \le 
\frac{1+\frac{1}{m}}{(1-\epsi)^2}\left(\marge{A_{l_1}'}{\emptyset}+\marge{A_{l_2}'}{\emptyset}\right) + \frac{2m\cdot \tau_{\min}}{1-\epsi}, \text{ if } O_{l_1} = O_{l_2}, \forall 1\le l_1 < l_2 \le \ell$,
\end{enumerate}}
where $O_l\subseteq \uni$, $|O_l| \le m$, and $O_l \cap A_j = \emptyset$ for each $j \neq l$.

Especially, when $\ell = 2$,
{\small
\begin{itemize}
	\item[4.] $\marge{S}{A_{1}} + \marge{S}{A_{2}} \le \frac{1}{(1-\epsi)^2}\left(\marge{A_{l_1}'}{\emptyset}+\marge{A_{l_2}'}{\emptyset}\right) + \frac{2m\cdot \tau_{\min}}{1-\epsi}, \forall S\subseteq \uni, |S| \le m$. 
\end{itemize}}
\end{lemma}
Before proving Lemma~\ref{lemma:ptgone}, we provide the following lemma regarding each iteration of \ptgone.
\begin{lemma}\label{lemma:tgone-iteration}
For any iteration of the while loop in \ptgone (Alg.~\ref{alg:ptgone}),
let $A_{l, 0}$, $A_{l, 0}'$, $V_{l, 0}$, $\tau_{l, 0}$ be the set and threshold value at the beginning,
and $A_l$, $A_l'$, $V_l$, $\tau_l$ be those at the end.
The following properties hold. 
\begin{enumerate}
\item With a probability of at least $1/2$,
there exists $l \in [\ell]$ \st $\tau_l < \tau_{l, 0}$
or $m_0 = 0$ or
$|V_l| \le \left(1-\frac{\epsi}{4\ell}\right)|V_{l, 0}|$.
\item $\{A_l: l\in [\ell]\}$ have the same size and are pairwise disjoint.  
% \item $V_l = \left\{x\in V\setminus\left(\bigcup_{i\in [\ell]} A_l\right) : \marge{x}{A_l} \ge \tau_l \right\}$  for all $l\in [\ell]$.
% \item For each $l \in [\ell]$ \st $\tau_l < \tau_{l,0}$,
% it holds that $\marge{x}{A_l} < \frac{\tau_l}{1-\epsi}$ for all $x\in V\setminus \left(\bigcup_{i\in [\ell]} A_l\right)$.
\item For each $x\in A_l\setminus A_{l,0}$, let $\tau_l^{(x)}$ be the threshold value when $x$ is added to the solution,
$A_{l, (x)}$ be the largest prefix of $A_l$ that do not include $x$,
and for any $j\in [\ell]$ and $j\neq l$,
$A_{j, (x)}$ be the prefix of $A_j$ with $|A_{l, (x)}|$ elements if $j < l$,
or with $|A_{l, (x)}|-1$ elements if $j > l$.
Then, for any $l\in [\ell]$, $x\in A_l\setminus A_{l,0}$,
and $y\in \uni\setminus \left(\bigcup_{j\in [\ell]} A_{j, (x)}\right)$,
it holds that $\marge{y}{A_{l, (x)}} < \frac{\tau_l^{(x)}}{1-\epsi}$.
\item $A_l'\subseteq A_l$, $\marge{A_l'}{A_{l, 0}'} \ge \marge{A_l}{A_{l, 0}}$,
and $\marge{A_l'}{A_{l, 0}'}\ge (1-\epsi)\sum_{x \in A_l\setminus A_{l, 0}}\tau_l^{(x)}$ for all $l\in [\ell]$.
\end{enumerate}
\end{lemma}
\begin{proof}[Proof of Lemma~\ref{lemma:tgone-iteration}]
\textbf{Proof of Property 1.}
At the beginning of the iteration, if there exists $l\in I$ \st $|V_{l, 0}| < 2\ell$,
then either $\tau_{l, 0}$ is decreased to $\tau_l$ and $V_l$ is updated accordingly, 
or an element $x_l$ from $V_{l, 0}$ is added to $A_j$ and $A_j'$
and subsequently removed from $V_{l, 0}$. 
This implies that
\[|V_l| \le |V_{l,0}| -1 < \left(1-\frac{1}{2\ell}\right)|V_{l,0}|.\]
Property 1 holds in this case.

Otherwise, for all $l\in I$, it holds that $|V_{l, 0}| \ge 2\ell$,
and the algorithm proceeds to execute Lines~\ref{line:tgone-dist}-\ref{line:tgone-update-size}.
By Lemma~\ref{lemma:dist}, in Line~\ref{line:tgone-dist}, 
$|\mathcal V_l| \ge \frac{|V_{l, 0}|}{2\ell}$ for each $l\in I$ .
Consider the index $j\in I$ where $i_j^* = i^*$.
Then, $O_j$ consists of the first $i^*$ elements in $\mathcal V_j$
by Line~\ref{line:tgone-subset}.
By Lemma~\ref{lemma:prefix-prob},
with probability greater than $1/2$,
either $i^* = m_0$ or at least an $\frac{\epsi}{2}$-fraction
of elements $x\in \mathcal V_j$ satisfy $\marge{x}{A_j}< \tau_{j,0}$.
Consequently, either $m_0 = 0$ after Line~\ref{line:tgone-update-size},
or, after the \update procedure in Line~\ref{line:tgone-update},
one of the following holds:
$|V_l| \le \left(1-\frac{\epsi}{4\ell}\right)|V_{l, 0}|$,
or $\tau_{j} < \tau_{j, 0}$.
Therefore, Property 1 holds in this case. 

\textbf{Proof of Property 2.}
At any iteration of the while,
either $|I|$ different elements or $|I|$ pairwise disjoint sets with same size $i^*$
are added to solution sets $\{A_l: l\in I\}$.
Therefore, Property 2 holds.

\textbf{Proof of Property 3.}
At any iteration, if $\tau_l$ is not updated on Line~\ref{line:tgone-update-2},
then prior to this iteration, all the elements outside of the solutions
have marginal gain less than $\frac{\tau_l^{(x)}}{1-\epsi}$.
Thus, for any $x \in A_l\setminus A_{l, 0}$, $y\in \uni\setminus \left(\bigcup_{j\in [\ell]} A_{j, 0}\right)$,
it holds that $\marge{y}{A_{l, (x)}} < \frac{\tau_l^{(x)}}{1-\epsi}$
by submodularity. Property 3 holds in this case.

Otherwise, if $\tau_l$ is updated on Line~\ref{line:tgone-update-2},
only one element is added to each solution set during this iteration.
Let $x = A_l\setminus A_{l, 0}$.
For any $j\in [\ell]$ and $j\neq l$,
it holds that $A_{j, (x)} = A_j$ if $j < l$,
or $A_{j, (x)} = A_{j, 0}$ if $j > l$.
Since elements are added to each pair of solutions in sequence within the for loop
in Lines~\ref{line:tgone-for-begin}-\ref{line:tgone-for-end},
by the \update procedure,
for any $y \in \uni \setminus \left(\bigcup_{j\in [\ell]} A_{j, (x)}\right)$,
it holds that $\marge{y}{A_{l, (x)}} < \frac{\tau_l^{(x)}}{1-\epsi}$.
Therefore, Property 3 also holds in this case.

\textbf{Proof of Property 4.}
First, we prove $A_l'\subseteq A_l$ by induction.
At the beginning of the algorithm,
$A_l'$ and $A_l$ are initialized as empty sets.
Clearly, the property holds in the base case.
Then, suppose that $A_{l,0}'\subseteq A_{l,0}$.
There are three possible cases of updating $A_{l,0}'$ and $A_{l,0}$ at any iteration:
1) $A_l'=A_{l,0}'$ and $A_l = A_{l,0}$,
2) $A_l' = A_{l,0}' + x_l$ and $A_l = A_{l,0} + x_l$ in Line~\ref{line:tgone-update-A},
or 3) $A_l' = A_{l,0}' \cup S_l'$ and $A_l = A_{l,0} \cup S_l$ in Line~\ref{line:tgone-update-A-2}.
Clearly, $A_l'\subseteq A_l$ holds in all cases.

Next, we prove the rest of Property 4.

If $A_l'=A_{l,0}'$ and $A_l = A_{l,0}$, 
then $\marge{A_l'}{A_{l, 0}'} = \marge{A_l}{A_{l, 0}}=0$.
Property 4 holds.

If $A_{l,0}'$ and $A_{l,0}$ are updated in Line~\ref{line:tgone-update-A},
by submodularity, $\marge{A_l'}{A_{l, 0}'} = \marge{x_l}{A_{l, 0}'} \ge \marge{x_l}{A_{l, 0}}=\marge{A_l}{A_{l, 0}} \ge \tau_l^{(x_{l})}$.
Therefore, Property 4 also holds.

If $A_{l,0}'$ and $A_{l,0}$ are updated in Line~\ref{line:tgone-update-A-2},
we know that $A_l' = A_{l,0}' \cup S_l'$ and $A_l = A_{l,0} \cup S_l$.
Suppose the elements in $S_l$ and $S_l'$ retain their original order within $\mathcal V_l$.
For each $x\in S_l$, let $S_{l,(x)}$, $\mathcal V_{l, (x)}$ and $A_{l, (x)}$
be the largest prefixes of $S_l$, $\mathcal V_l$ and $A_l$ that do not include $x$, respectively.
Moreover, let $S_{l, (x)}' = S_{l, (x)}\cap S_l'$ and $A_{l, (x)}' = A_{l, (x)}\cap A_l'$.
Say an element $x\in S_l$ \textbf{true} if $B_l[(x)] = \textbf{true}$,
where $B_l[(x)]$ is the $i$-th element in $B_l$ if $x$ is the $i$-th element in $\mathcal V_l$.
Similarly, say an element $x\in S_l$ \textbf{false} if $B_l[(x)] = \textbf{false}$,
and \textbf{none} otherwise.

Following the above definitions, for any \textbf{true} or \textbf{none} element $x\in S_l$,
by Line~\ref{line:tgone-subset}, it holds that $S_{l, (x)} \subseteq \mathcal V_{l, (x)}$.
Then, by Line~\ref{line:prefix-B-true} and submodularity,
\begin{equation*}
\marge{x}{A_{l, (x)}} = \marge{x}{A_{l, 0} \cup S_{l, (x)}}
\ge \marge{x}{A_{l, 0} \cup \mathcal V_{l, (x)}} \ge \left\{
\begin{aligned}
&\tau_l^{(x)}, \text{ if } x \text{ is \textbf{true} element}\\
&0, \text{ if } x \text{ is \textbf{none} element}
\end{aligned}\right.
\end{equation*}
Since \textbf{true} elements are selected at first and $i_j^*\ge i^*$,
there are more than $(1-\epsi)i^*$ \textbf{true} elements in $S_l$.
Therefore,
\begin{align*}
\marge{A_l'}{A_{l, 0}'} &= \sum_{x\in A_l'\setminus A_{l, 0}', x\text{ is \textbf{true} element}} \marge{x}{A_{l, (x)}'}
+\sum_{x\in A_l'\setminus A_{l, 0}', x\text{ is \textbf{none} element}} \marge{x}{A_{l, (x)}'}\\
&\ge \sum_{x\in A_l'\setminus A_{l, 0}', x\text{ is \textbf{true} element}} \marge{x}{A_{l, (x)}}
+\sum_{x\in A_l'\setminus A_{l, 0}', x\text{ is \textbf{none} element}} \marge{x}{A_{l, (x)}}\\
&\ge (1-\epsi) |A_l\setminus A_{l, 0}| \tau_l^{(x)}, \text{for any } x\in A_l\setminus A_{l, 0}\\
&= (1-\epsi)\sum_{x\in A_l\setminus A_{l, 0}} \tau_l^{(x)}.
\end{align*}
The third part of Property 4 holds.

To prove the second part of Property 4, consider any \textbf{false} element $x\in S_l$.
By Line~\ref{line:tgone-subset}, it holds that $\mathcal V_{l, (x)} = S_{l, (x)}$.
Then, by Line~\ref{line:prefix-B-false}
\begin{equation}\label{ineq:tgone-false}
\marge{x}{A_{l, (x)}} = \marge{x}{A_{l, 0} \cup S_{l, (x)}}
= \marge{x}{A_{l, 0} \cup \mathcal V_{l, (x)}} < 0.
\end{equation}
By Line~\ref{line:tgone-subset-2}, all the elements in $S_l\setminus S_l'$ are
\textbf{false} elements.
Then,
\begin{align*}
\ff{A_l} - \ff{A_{l, 0}} &= \sum_{x\in S_l'}\marge{x}{A_{l, (x)}} + \sum_{x\in S_l\setminus S_l'}\marge{x}{A_{l, (x)}}\\
&< \sum_{x\in S_l'}\marge{x}{A_{l, (x)}} \tag{Inequality~\ref{ineq:tgone-false}}\\
&\le \sum_{x\in S_l'}\marge{x}{A_{l, (x)}'} \tag{Submodularity}\\
& = \ff{A_l'} - \ff{A_{l, 0}'}.
\end{align*}
\end{proof}

By Lemma~\ref{lemma:tgone-iteration},
we are ready to prove Lemma~\ref{lemma:ptgone}.
\begin{proof}[Proof of Lemma~\ref{lemma:ptgone}]
\textbf{Proof of Property 1.}
By Property 2 and 3 in Lemma~\ref{lemma:tgone-iteration},
this property holds immediately.

\textbf{Proof of Property 2.}
For any $l\in [\ell]$,
since $O_l\cap A_j  = \emptyset$ for each $j\neq l$,
$O_l\setminus A_l$ is outside of any solution set.
If $|A_l| = m$, by Property 4 of Lemma~\ref{lemma:tgone-iteration},
\begin{align*}
\marge{O_l}{A_l} &\le \sum_{y \in O_l\setminus A_l}\marge{y}{A_l}\\
&\le \sum_{x \in A_l}\tau_l^{(x)}/(1-\epsi)\tag{Property 3 in Lemma~\ref{lemma:tgone-iteration}}\\
& \le \frac{\marge{A_l'}{\emptyset}}{(1-\epsi)^2}.\tag{Property 5 in Lemma~\ref{lemma:tgone-iteration}}
\end{align*}
If $|A_l| < m$, then the threshold value for solution $A_l$ has been updated to be less than $\tau_{\min}$.
Therefore, for any $y\in O_l\setminus A_l$,
it holds that $\marge{y}{A_l} < \frac{\tau_{\min}}{1-\epsi}$.
Then,
\begin{align*}
\marge{O_l}{A_l} \le \sum_{y \in O_l\setminus A_l}\marge{y}{A_l}
\le \frac{m\tau_{\min}}{1-\epsi}.
\end{align*}
Therefore, Property 2 holds by summing the above two inequalities.

\textbf{Proof of Property 3 and 4.}
Let $a_{l, j}$ be the $j$-th element added to $A_l$,
$\tau_l^j$ be the threshold value of $\tau_l$ when $a_{l, j}$ is added to $A_l$,
and $A_{l, j}$ be $A_l$ after $a_{l, j}$ is added to $A_l$.
Let $c_l^* = \max\{c\in [m]:A_{l, c}\subseteq O_l\}$.

In the following, we analyze these properties together under two cases,
similar to the analysis of Alg.~\ref{alg:ptgtwo}.
For the case where $\ell = 2$, 
let $O_{1} = S\setminus A_2$, and $O_2 = S\setminus A_1$,
unifying the notations used in Property 3 and 4.
Note that, the only difference between the two analyses is that,
a small portion (no more than $\epsi$ fraction) of elements in the solution returned by Alg.~\ref{alg:ptgone}
do not have marginal gain greater than the threshold value.

\textbf{Case 1: $c_{l_1}^*\le c_{l_2}^*$; left half part in Fig.~\ref{fig:gdtwo}.}

First, we bound $\marge{O_{l_1}}{A_{l_2}}$.
Consider elements in $A_{l_1, c_{l_1}^*} \subseteq O_{l_1}$.
Let $A_{l_1, c_{l_1}^*} = \{o_1, \ldots, o_{c_{l_1}^*}\}$.
For each $1\le j \le c_{l_1}^*$, 
since $o_j$ is added to $A_{l_1}$ with threshold value $\tau_{l_1}^{j}$
and the threshold value starts from the maximum marginal gain $M$,
clearly, $o_j$ has been filtered out with threshold value $\tau_{l_1}^{j}/(1-\epsi)$.
Then, by submodularity,
\begin{equation}\label{inq:ptgone-case1-1}
\marge{A_{l_1, c_{l_1}^*} }{A_{l_2}} \le \marge{A_{l_1, c_{l_1}^*} }{\emptyset}
 = \sum_{j=1}^{c_{l_1}^*}\marge{o_j}{A_{l_1, j-1}}
 \le \sum_{j=1}^{c_{l_1}^*} \tau_{l_1}^{j}/(1-\epsi).
\end{equation}

Next, consider the elements in $O_{l_1}\setminus A_{l_1, c_{l_1}^*}$.
Order the elements in $O_{l_1}\setminus A_{l_1, c_{l_1}^*}$ as $\{o_1, o_2, \ldots\}$ such that $o_j \not \in A_{l_1, c_{l_1}^*+j}$.
(Refer to the gray block with a dotted edge in the top left corner of Fig.~\ref{fig:gdtwo} for $O_{l_1}$.
If $c_{l_1}^*+j$ is greater than $|A_{l_1}|$,
$A_{l_1, c_{l_1}^*+j}$ refers to $A_{l_1}$.)
Note that, since $A_{l_1, c_{l_1}^*} \subseteq O_{l_1}$,
it follows that $|O_{l_1}\setminus A_{l_1, c_{l_1}^*}| \le m - c_{l_1}^*$.

When $1 \le j \le |A_{l_2}| - c_{l_1}^*$,
since each $o_j$ is either added to $A_{l_1}$ or not in any solution set
and $\tau_{l_2}$ is initialized with the maximum marginal gain $M$,
$o_j$ is not considered to be added to $A_{l_2}$ with threshold value $\tau_{l_2}^{c_{l_1}^* + j}/(1-\epsi)$
by Property 3 of Lemma~\ref{lemma:tgone-iteration}.
Therefore, it holds that 
\begin{equation}\label{inq:ptgone-case1-2}
\marge{o_j}{A_{l_2, c_{l_1}^*+j-1}} < \frac{\tau_{l_2}^{c_{l_1}^* + j}}{1-\epsi} , \forall 1\le j\le |A_{l_2}|-c_{l_1}^*.
\end{equation}

When $|A_{l_2}| < m$ and $|A_{l_2}|-c_{l_1}^* < j\le m-c_{l_1}^*$,
the algorithm ends with $\tau_{l_2} < \tau_{\min}$ and
$o_j$ is never considered to be added to $A_{l_2}$.
Thus, it holds that
\begin{equation}\label{inq:ptgone-case1-3}
\marge{o_j}{A_{l_2}} < \frac{\tau_{\min}}{1-\epsi}, 
\forall |A_{l_2}|-c_{l_1}^* < j \le m-c_{l_1}^*.
\end{equation}

Then,
\begin{align*}
\marge{O_{l_1}}{A_{l_2}} &\le \marge{A_{l_1, c_{l_1}^*}}{A_{l_2}}  + \sum_{o_j \in O_{l_1}\setminus A_{l_1, c_{l_1}^*}}\marge{o_j}{A_{l_2}} \tag{Proposition~\ref{prop:sum-marge}}\\
&\le \marge{A_{l_1, c_{l_1}^*}}{\emptyset} + \sum_{j = 1}^{|A_{l_2}|-c_{l_1}^*}\marge{o_j}{A_{l_2, , c_{l_1}^*+j-1}} + \sum_{j=|A_{l_2}|-c_{l_1}^*+1}^{m-c_{l_1}^*} \marge{o_j}{A_{l_2}} \tag{submodularity}\\
&\le \sum_{j=1}^{c_{l_1}^*} \frac{\tau_{l_1}^{j}}{1-\epsi} + \sum_{j=c_{l_1}^*+1}^{|A_{l_2}|} \frac{\tau_{l_2}^{j}}{1-\epsi} + \frac{m \cdot \tau_{\min}}{1-\epsi} \numberthis \label{inq:ptgone-case1-4}
\end{align*}
where the last inequality follows from 
Inequalities~\eqref{inq:ptgone-case1-1}-\eqref{inq:ptgone-case1-3}.

Similarly, we bound $\marge{O_{l_2}}{A_{l_1}}$ below.
Consider elements in $A_{l_2, c_{l_1}^*} \subseteq O_{l_2}$.
Let $A_{l_2, c_{l_1}^*} = \{o_1, \ldots, o_{c_{l_1}^*}\}$.
For each $1\le j \le c_{l_1}^*$, 
since $o_j$ is added to $A_{l_2}$ with threshold value $\tau_{l_2}^{j}$
and the threshold value starts from the maximum marginal gain $M$,
clearly, $o_j$ has been filtered out with threshold value $\tau_{l_2}^{j}/(1-\epsi)$.
Then, by submodularity,
\begin{equation}\label{inq:ptgone-case1-5}
\marge{A_{l_2, c_{l_1}^*} }{A_{l_1}} \le \marge{A_{l_2, c_{l_1}^*} }{\emptyset}
 = \sum_{j=1}^{c_{l_1}^*}\marge{o_j}{A_{l_2, j-1}}
 \le \sum_{j=1}^{c_{l_1}^*} \tau_{l_2}^{j}/(1-\epsi).
\end{equation}

Next, consider the elements in $O_{l_2}\setminus A_{l_2, c_{l_1}^*}$.
Order the elements in $O_{l_2}\setminus A_{l_2, c_{l_1}^*}$ as $\{o_1, o_2, \ldots\}$ such that $o_j \not \in A_{l_2, c_{l_1}^*+j-1}$.
(See the gray block with a dotted edge in the bottom left corner of Fig.~\ref{fig:gdtwo} for $O_{l_2}$.
If $c_{l_1}^*+j-1$ is greater than $|A_{l_2}|$,
$A_{l_2, c_{l_1}^*+j-1}$ refers to $A_{l_2}$.)
Note that, since $A_{l_2, c_{l_1}^*} \subseteq O_{l_2}$,
it follows that $|O_{l_2}\setminus A_{l_2, c_{l_1}^*}| \le m - c_{l_1}^*$.

When $1 \le j \le |A_{l_1}|-c_{l_1}^*$,
since each $o_j$ is either added to $A_{l_2}$ or not in any solution set,
and $\tau_{l_1}$ is initialized with the maximum marginal gain $M$,
$o_j$ is not considered to be added to $A_{l_1}$ with threshold value $\tau_{l_1}^{c_{l_1}^* + j}/(1-\epsi)$
by Property 3 of Lemma~\ref{lemma:tgone-iteration}.
Therefore, it holds that 
\begin{equation}\label{inq:ptgone-case1-6}
\marge{o_j}{A_{l_1, c_{l_1}^*+j-1}} < \frac{\tau_{l_1}^{c_{l_1}^* + j}}{1-\epsi} , \forall 1\le j\le |A_{l_2}|-c_{l_1}^*.
\end{equation}

When $|A_{l_1}| < m$ and $|A_{l_1}|-c_{l_1}^* < j\le m-c_{l_1}^*$,
this iteration ends with $\tau_{l_1} < \tau_{\min}$
and $o_j$ is never considered to be added to $A_{l_1}$.
Thus, it holds that
\begin{equation}\label{inq:ptgone-case1-7}
\marge{o_j}{A_{l_1}} < \frac{\tau_{\min}}{1-\epsi}, \forall |A_{l_1}|-c_{l_1}^* < j \le m-c_{l_1}^*.
\end{equation}

Then,
\begin{align*}
\marge{O_{l_2}}{A_{l_1}} &\le \marge{A_{l_2, c_{l_1}^*}}{A_{l_1}}  + \sum_{o_j \in O_{l_2}\setminus A_{l_2, c_{l_1}^*}}\marge{o_j}{A_{l_1}} \tag{Proposition~\ref{prop:sum-marge}}\\
&\le \marge{A_{l_2, c_{l_1}^*}}{\emptyset} + \sum_{j = 1}^{|A_{l_1}|-c_{l_1}^*}\marge{o_j}{A_{l_1, c_{l_1}^*+j-1}} + \sum_{j=|A_{l_1}|-c_{l_1}^*+1}^{m-c_{l_1}^*} \marge{o_j}{A_{l_1}} \tag{submodularity}\\
&\le \sum_{j=1}^{c_{l_1}^*} \frac{\tau_{l_2}^{j}}{1-\epsi} + \sum_{j=c_{l_1}^*+1}^{|A_{l_1}|} \frac{\tau_{l_1}^{j}}{1-\epsi} + \frac{m\cdot \tau_{\min}}{1-\epsi}  \numberthis \label{inq:ptgone-case1-8}
\end{align*}
where the last inequality follows from Inequalities~\ref{inq:ptgone-case1-5}-\ref{inq:ptgone-case1-7}.

By Inequalities~\eqref{inq:ptgone-case1-4} and~\eqref{inq:ptgone-case1-8},
\begin{equation}\label{inq:ptgone-case1-final}
\marge{O_{l_1}}{A_{l_2}}+\marge{O_{l_2}}{A_{l_1}}
\le \sum_{j=1}^{|A_{l_1}|} \frac{\tau_{l_1}^{j}}{1-\epsi} + \sum_{j=1}^{|A_{l_2}|} \frac{\tau_{l_2}^{j}}{1-\epsi} + \frac{2m\cdot \tau_{\min}}{1-\epsi}
\end{equation}

\textbf{Case 2: $c_{l_1}^* > c_{l_2}^*$; right half part in Fig.~\ref{fig:gdtwo}.}

First, we bound $\marge{O_{l_1}}{A_{l_2}}$.
Consider elements in $A_{l_1, c_{l_2}^*+1} \subseteq O_{l_1}$.
Let $A_{l_1, c_{l_2}^*+1} = \{o_1, \ldots, o_{c_{l_2}^*+1}\}$.
For each $1\le j \le c_{l_2}^*+1$, 
since $o_j$ is added to $A_{l_1}$ with threshold value $\tau_{l_1}^{j}$
and the threshold value starts from the maximum marginal gain $M$,
clearly, $o_j$ has been filtered out with threshold value $\tau_{l_1}^{j}/(1-\epsi)$.
Then, by submodularity,
\begin{equation}\label{inq:ptgone-case2-1}
\marge{A_{l_1, c_{l_2}^*+1} }{A_{l_2}} \le \marge{A_{l_1, c_{l_2}^*+1} }{\emptyset}
 = \sum_{j=1}^{c_{l_2}^*+1}\marge{o_j}{A_{l_1, j-1}}
 \le \sum_{j=1}^{c_{l_2}^*+1} \tau_{l_1}^{j}/(1-\epsi).
\end{equation}

Next, consider the elements in $O_{l_1}\setminus A_{l_1, c_{l_2}^*+1}$.
Order the elements in $O_{l_1}\setminus A_{l_1, c_{l_2}^*+1}$ as $\{o_1, o_2, \ldots\}$ such that $o_j \not \in A_{l_1, c_{l_1}^*+j}$.
(Refer to the gray block with a dotted edge in the top right corner of Fig.~\ref{fig:gdtwo} for $O_{l_1}$.
If $c_{l_1}^*+j$ is greater than $|A_{l_1}|$,
$A_{l_1, c_{l_1}^*+j}$ refers to $A_{l_1}$.)
Note that, since $A_{l_1, c_{l_2}^*+1} \subseteq O_{l_1}$,
it follows that $|O_{l_1}\setminus A_{l_1, c_{l_2}^*+1}| \le m - c_{l_2}^*-1$.

When $1 \le j \le |A_{l_2}| - c_{l_2}^* - 1$,
since each $o_j$ is either added to $A_{l_1}$ or not in any solution set
and $\tau_{l_2}$ is initialized with the maximum marginal gain $M$,
$o_j$ is not considered to be added to $A_{l_2}$ with threshold value $\tau_{l_2}^{c_{l_2}^* + j}/(1-\epsi)$.
Therefore, it holds that 
\begin{equation}\label{inq:ptgone-case2-2}
\marge{o_j}{A_{l_2, c_{l_2}^*+j-1}} < \frac{\tau_{l_2}^{c_{l_2}^* + j}}{1-\epsi}, \forall 1\le j\le |A_{l_2}\setminus G_{i-1}| - c_{l_2}^* - 1.
\end{equation}

When $|A_{l_2}| < m$ and $|A_{l_2}|- c_{l_2}^* - 1 < j\le m- c_{l_2}^* - 1$,
this iteration ends with $\tau_{l_2} < \tau_{\min}$ and
$o_j$ is never considered to be added to $A_{l_2}$.
Thus, it holds that
\begin{equation}\label{inq:ptgone-case2-3}
\marge{o_j}{A_{l_2}} < \frac{\tau_{\min}}{1-\epsi}, 
\forall |A_{l_2}|- c_{l_2}^* - 1 < j \le m- c_{l_2}^* - 1.
\end{equation}

Then,
\begin{align*}
\marge{O_{l_1}}{A_{l_2}} &\le \marge{A_{l_1, c_{l_2}^*}}{A_{l_2}}  + \sum_{o_j \in O_{l_1}\setminus A_{l_1, c_{l_2}^*+1}}\marge{o_j}{A_{l_2}} \tag{Proposition~\ref{prop:sum-marge}}\\
&\le \marge{A_{l_1, c_{l_2}^*}}{\emptyset} + \sum_{j = 1}^{|A_{l_2}|- c_{l_2}^* - 1}\marge{o_j}{A_{l_2, c_{l_2}^*+j-1}} + \sum_{j=|A_{l_2}|- c_{l_2}^*}^{m- c_{l_2}^* - 1} \marge{o_j}{A_{l_2}} \tag{submodularity}\\
&\le \sum_{j=1}^{c_{l_2}^*+1} \tau_{l_1}^{j}/(1-\epsi)
+ \sum_{j = c_{l_2}^*+1}^{|A_{l_2}|} \tau_{l_2}^{j}/(1-\epsi) + \frac{m \cdot \tau_{\min}}{1-\epsi}
 \numberthis \label{inq:ptgone-case2-4}
\end{align*}
where the last inequality follows from 
Inequalities~\eqref{inq:ptgone-case2-1}-\eqref{inq:ptgone-case2-3}.

Similarly, we bound $\marge{O_{l_2}}{A_{l_1}}$ below.
Consider elements in $A_{l_1, c_{l_2}^*} \subseteq O_{l_2}$.
Let $A_{l_2, c_{l_2}^*} = \{o_1, \ldots, o_{c_{l_2}^*}\}$.
For each $1\le j \le c_{l_2}^*$, 
since $o_j$ is added to $A_{l_2}$ with threshold value $\tau_{l_2}^{j}$
and the threshold value starts from the maximum marginal gain $M$,
clearly, $o_j$ has been filtered out with threshold value $\tau_{l_2}^{j}/(1-\epsi)$.
Then, by submodularity,
\begin{equation}\label{inq:ptgone-case2-5}
\marge{A_{l_2, c_{l_2}^*} }{A_{l_1}} \le \marge{A_{l_2, c_{l_2}^*} }{\emptyset}
 = \sum_{j=1}^{c_{l_2}^*}\marge{o_j}{A_{l_2, j-1}}
 \le \sum_{j=1}^{c_{l_2}^*} \tau_{l_2}^{j}/(1-\epsi).
\end{equation}

Next, consider the elements in $O_{l_2}\setminus A_{l_2, c_{l_2}^*}$.
Order these elements as $\{o_1, o_2, \ldots\}$ such that $o_j \not \in A_{l_2, c_{l_2}^*+j}$.
(See the gray block with a dotted edge in the bottom right corner of Fig.~\ref{fig:gdtwo} for $O_{l_2}$.
If $c_{l_2}^*+j$ is greater than the number of elements added to $A_{l_2}$,
$A_{l_2, c_{l_2}^*+j}$ refers to $A_{l_2}$.)
Note that, since $A_{l_2, c_{l_2}^*} \subseteq O_{l_2}$,
it follows that $|O_{l_2}\setminus A_{l_2, c_{l_2}^*}| \le m - c_{l_2}^*$.

Furthermore, for the case where $\ell  = 2$, as considered in Property 4,
we have $O_1 = S\setminus A_2$ and $O_2 = S\setminus A_1$ for a given $S\subseteq \uni$
where $|S| \le m$.
Since $c_{l_1}^* > c_{l_2}^* \ge 0$,
it follows that $c_{l_1}^*\ge 1$, which implies $|O_2| = |S\setminus A_1|\le m-1$.
In this case, it holds that $|O_{l_2}\setminus A_{l_2, c_{l_2}^*}| \le m - c_{l_2}^*-1$.

When $1 \le j \le |A_{l_1}|- c_{l_2}^* - 1$, 
since each $o_j$ is either added to $A_{l_2}$ or not in any solution set by Claim~\ref{claim:par-A}
and $\tau_{l_1}$ is initialized with the maximum marginal gain $M$,
$o_j$ is not considered to be added to $A_{l_1}$ with threshold value $\tau_{l_1}^{c_{l_2}^* + j+1}/(1-\epsi)$.
Therefore, it holds that 
\begin{equation}\label{inq:ptgone-case2-6}
\marge{o_j}{A_{l_1, c_{l_2}^*+j}} < \frac{\tau_{l_1}^{c_{l_2}^* + j+1}}{1-\epsi}, \forall 1\le j\le |A_{l_2}|- c_{l_2}^* - 1.
\end{equation}

If $|A_{l_1}| = m$,
consider the last element $o_{m-c_{l_2}^*}$ in $O_{l_2}\setminus A_{l_2, c_{l_2}^*}$.
Since $o_{m-c_{l_2}^*} \not\in A_{l_2}$ and $o_{m-c_{l_2}^*} \not\in A_{l_1}$, $o_{m-c_{l_2}^*}$ is not considered to be added to 
$A_{l_1}$ with threshold value $\tau_{l_1}^j/(1-\epsi)$ for any $j \in [m]$.
Then,
\begin{equation}\label{inq:ptgone-case2-7}
\marge{o_{m-c_{l_2}^*}}{A_{l_1}} < \frac{\sum_{j=1}^m \tau_{l_1}^j}{(1-\epsi)m}.
\end{equation}
Else, $|A_{l_1}| < m$ and 
this iteration ends with $\tau_{l_1} < \frac{\epsi M}{k}$.
For any $|A_{l_1}|- c_{l_2}^* - 1 < j\le m- c_{l_2}^*$,
$o_j$ is never considered to be added to $A_{l_1}$.
Thus, it holds that
\begin{equation}\label{inq:ptgone-case2-8}
\marge{o_j}{A_{l_1}} < \frac{\tau_{\min}}{1-\epsi}, 
\forall |A_{l_1}|- c_{l_2}^* - 1 < j \le m- c_{l_2}^*.
\end{equation}

Then,
\begin{align*}
&\marge{O_{l_2}}{A_{l_1}} \le \marge{A_{l_2, c_{l_2}^*}}{A_{l_1}}  + \sum_{o_j \in O_{l_2}\setminus A_{l_2, c_{l_2}^*}}\marge{o_j}{A_{l_1}} \tag{Proposition~\ref{prop:sum-marge}}\\
&\le \left\{
\begin{aligned}
&\marge{A_{l_2, c_{l_2}^*}}{\emptyset} + \sum_{j = 1}^{|A_{l_1}\setminus G_{i-1}|- c_{l_2}^* - 1}\marge{o_j}{A_{l_1, c_{l_2}^*+j-1}} + \sum_{j=|A_{l_1}\setminus G_{i-1}|- c_{l_2}^*}^{m-c_{l_2}^*} \marge{o_j}{A_{l_1}}, &&\text{ if } |O_{l_2}| = m\\
&\marge{A_{l_2, c_{l_2}^*}}{\emptyset} + \sum_{j = 1}^{|A_{l_1}\setminus G_{i-1}|- c_{l_2}^* - 1}\marge{o_j}{A_{l_1, c_{l_2}^*+j-1}} + \sum_{j=|A_{l_1}\setminus G_{i-1}|- c_{l_2}^*}^{m-c_{l_2}^*-1} \marge{o_j}{A_{l_1}}, &&\text{otherwise}
\end{aligned}
\right. \tag{submodularity}\\
&\le\left\{
\begin{aligned}
	&\sum_{j=1}^{c_{l_2}^*} \frac{\tau_{l_2}^j}{1-\epsi} + \sum_{j=c_{l_2}^*+2}^{|A_{l_2}|} \left(1+\frac{1}{m}\right) \frac{\tau_{l_1}^j}{1-\epsi} + \frac{m\cdot \tau_{\min}}{1-\epsi}, &&\text{ if } |O_{l_2}| = m\\
	&\sum_{j=1}^{c_{l_2}^*} \frac{\tau_{l_2}^j}{1-\epsi} + \sum_{j=c_{l_2}^*+2}^{|A_{l_2}|} \frac{\tau_{l_1}^j}{1-\epsi} + \frac{m\cdot \tau_{\min}}{1-\epsi}, &&\text{otherwise}
\end{aligned}
\right. \numberthis \label{inq:ptgone-case2-9}
\end{align*}
where the last inequality follows from Inequalities~\eqref{inq:ptgone-case2-5}-\eqref{inq:ptgone-case2-8}.

By Inequalities~\eqref{inq:ptgone-case2-4} and~\eqref{inq:ptgone-case2-9},
\begin{equation}\label{inq:ptgone-case2-final}
\marge{O_{l_1}}{A_{l_2}}+\marge{O_{l_2}}{A_{l_1}}
\le \left\{
\begin{aligned}
	& \left(1+\frac{1}{m}\right) \frac{1}{1-\epsi}\left(\sum_{j=1}^{|A_{l_1}|} \tau_{l_1}^{j} + \sum_{j=1}^{|A_{l_2}|} \tau_{l_2}^{j}\right) + \frac{2m\cdot \tau_{\min}}{1-\epsi}, &&\text{ if } |O_{l_2}| = m \\
	&  \frac{1}{1-\epsi}\left(\sum_{j=1}^{|A_{l_1}|} \tau_{l_1}^{j} + \sum_{j=1}^{|A_{l_2}|} \tau_{l_2}^{j}\right) + \frac{2m\cdot \tau_{\min}}{1-\epsi}, &&\text{ otherwise }
\end{aligned}
\right.
\end{equation}

Overall, in both cases, if $|O_{l_2}| = m$,
\begin{align*}
\marge{O_{l_1}}{A_{l_2}}+\marge{O_{l_2}}{A_{l_1}}
&\le \left(1+\frac{1}{m}\right) \frac{1}{1-\epsi}\left(\sum_{j=1}^{|A_{l_1}|} \tau_{l_1}^{j} + \sum_{j=1}^{|A_{l_2}|} \tau_{l_2}^{j}\right) + \frac{2m\cdot \tau_{\min}}{1-\epsi} \tag{Inequalities~\eqref{inq:ptgone-case1-final} and~\eqref{inq:ptgone-case2-final}}\\
&\le \left(1+\frac{1}{m}\right)\frac{1}{(1-\epsi)^2}\left(\marge{A_{l_1}'}{\emptyset}+\marge{A_{l_2}'}{\emptyset}\right) + \frac{2m\cdot \tau_{\min}}{1-\epsi} \tag{Property 4 of Lemma~\ref{lemma:tgone-iteration}}
\end{align*}
Otherwise, if $|O_{l_2}| < m$,
\begin{align*}
\marge{O_{l_1}}{A_{l_2}}+\marge{O_{l_2}}{A_{l_1}}
&\le \frac{1}{1-\epsi}\left(\sum_{j=1}^{|A_{l_1}|} \tau_{l_1}^{j} + \sum_{j=1}^{|A_{l_2}|} \tau_{l_2}^{j}\right) + \frac{2m\cdot \tau_{\min}}{1-\epsi} \tag{Inequalities~\eqref{inq:ptgone-case1-final} and~\eqref{inq:ptgone-case2-final}}\\
&\le \frac{1}{(1-\epsi)^2}\left(\marge{A_{l_1}'}{\emptyset}+\marge{A_{l_2}'}{\emptyset}\right) + \frac{2m\cdot \tau_{\min}}{1-\epsi} \tag{Property 4 of Lemma~\ref{lemma:tgone-iteration}}
\end{align*}
Property (3) and (4) hold.

\textbf{Proof of Adaptivity and Query Complexity.}
Note that, at the beginning of every iteration,
for any $j\in I$, $V_j$ contains all the elements outside of all solutions that has marginal gain greater than $\tau_j$ with respect to solution $A_j$.
Say an iteration \textit{successful} if either
1) algorithm terminates after this iteration because of $m_0=0$,
2) all the elements in $V_j$ can be filtered out at the end of this iteration
and the value of $\tau_j$ decreases,
or 3) the size of $V_j$ decreases by a factor of $1-\frac{\epsi}{4\ell}$.
Then, by Property 1 of Lemma~\ref{lemma:tgone-iteration},
with a probability of at least $1/2$,
the iteration is successful.
Furthermore, if $\tau_j$ is less than $\tau_{\min}$,
$j$ will be removed from $I$ and 
solutions $A_j$ and $A_j'$ won't be updated anymore.

For each $j \in [\ell]$,
there are at most $\log_{1-\epsi}\left(\frac{\tau_{\min}}{M}\right) \le \epsi^{-1}\log\left(\frac{M}{\tau_{\min}}\right)$ possible threshold values.
And, for each threshold value, with at most 
$\log_{1-\frac{\epsi}{4\ell}}\left(\frac{1}{n}\right) \le 4\ell\epsi^{-1}\log(n)$
successful iterations regarding solution $A_j$,
the threshold value $\tau_j$ will decrease
or the algorithm terminates because of $m_0=0$.
Overall, with at most $4\ell^2\epsi^{-2}\log(n)\log\left(\frac{M}{\tau_{\min}}\right)$
successful iterations,
the algorithm terminates because of $m_0=0$ or $I=\emptyset$.

Next, we prove that, after $N=4\left(\log(n)+ 4\ell^2\epsi^{-2}\log(n)\log\left(\frac{M}{\tau_{\min}}\right)\right)$ iterations,
with a probability of $1-\frac{1}{n}$,
there exists at least $4\ell^2\epsi^{-2}\log(n)\log\left(\frac{M}{\tau_{\min}}\right)$
successful iterations,
or equivalently, the algorithm terminates.
Let $X$ be the number of successful iterations.
Then, $X$ can be regarded as a sum of $N$ dependent Bernoulli trails,
where the success probability is larger than $1/2$.
Let $Y$ be a sum of $N$ independent Bernoulli trials,
where the success probability is equal to $1/2$.
Then, the probability that the algorithm terminates with at most $N$ iterations can be bounded as follows,
\begin{align*}
\prob{\#\text{iterations} > N} &\le \prob{X \le 4\ell^2\epsi^{-2}\log(n)\log\left(\frac{M}{\tau_{\min}}\right)} \\
& \overset{(a)}{\le} \prob{Y \le 4\ell^2\epsi^{-2}\log(n)\log\left(\frac{M}{\tau_{\min}}\right)}\tag{Lemma~\ref{lemma:indep}}\\
&\le e^{- \frac{N}{4}\left(1-\frac{8\ell^2\epsi^{-2}\log(n)\log\left(\frac{M}{\tau_{\min}}\right)}{N}\right)^2} \tag{Lemma~\ref{lemma:chernoff}}\\
&= e^{-\frac{\left(4\log(n)+ 8\ell^2\epsi^{-2}\log(n)\log\left(\frac{M}{\tau_{\min}}\right)\right)^2}{16\left(\log(n)+ 4\ell^2\epsi^{-2}\log(n)\log\left(\frac{M}{\tau_{\min}}\right)\right)}} \le \frac{1}{n}.
\end{align*}
Therefore, with a probability of $1-\frac{1}{n}$,
the algorithm terminates with $\oh{\ell^2\epsi^{-2}\log(n)\log\left(\frac{M}{\tau_{\min}}\right)}$ iterations of the while loop.

In Alg.~\ref{alg:ptgone}, oracle queries occur during calls
to \update and \prefix 
on Line~\ref{line:tgone-update-2},~\ref{line:tgone-prefix} and~\ref{line:tgone-update}.
The \prefix algorithm,
with input $(f, \mathcal V, s, \tau, \epsi)$,
operates with $1$ adaptive rounds
and at most $|\mathcal V|$ queries.
The \update algorithm,
with input $(f, V_0, \tau_0, \epsi)$, 
outputs $(V, \tau)$
with $1+\log_{1-\epsi}\left(\frac{\tau}{\tau_0}\right)$ adaptive rounds
and at most $|V| + n\log_{1-\epsi}\left(\frac{\tau}{\tau_0}\right)$ queries.
Here, $\log_{1-\epsi}\left(\frac{\tau}{\tau_0}\right)$ equals the number of iterations in the while loop within \update.
Notably, every iteration is successful,
as the threshold value is updated.
Consequently, we can regard an iteration of the while loop in \update
as a separate iteration of the while loop in Alg.~\ref{alg:ptgone},
where such iteration only update one threshold value $\tau_j$ and its corresponding candidate set $V_j$.
So, each redefined iteration has no more than $2$ adaptive rounds,
and then the adaptivity of the algorithm should be no more than 
the number of successful iterations, which is
$\oh{\ell^2\epsi^{-2}\log(n)\log\left(\frac{M}{\tau_{\min}}\right)}$.
Since there are at most $\ell n$ queries at each adaptive rounds,
the query complexity is bounded by $\oh{\ell^3\epsi^{-2}n\log(n)\log\left(\frac{M}{\tau_{\min}}\right)}$.

% Next, we consider the query complexity of the algorithm. 
% Let $V_{j, i}$ be the set $V_j$ at the beginning of
% $i$-th redefined iteration of the while loop.
% Note that a redefined iteration of the while loop in Alg.~\ref{alg:ptgone} corresponds either to an iteration
% where none of the threshold values are updated
% or to an iteration of the while loop in \update.
% During every redefined iteration, 
% there are at most $2|V_{j, i}|$ queries if $\tau_j$ is not updated regrading solution $A_j$,
% or $|V_{j, i}| = n$ queries if $\tau_j$ is updated.

% In the worst case, at every successful iteration,
% only the set $V_j$ with minimum size decreases by a factor of $1-\frac{\epsi}{4\ell}$.
% Thus, the worst case scenario follows the following steps:
% 1) each $V_j$ starts from $\uni$
% and only one of $V_j$ decreases with a factor of $1-\frac{\epsi}{4\ell}$
% until $\tau_j$ is updated;
% 2) each $V_j$ becomes $\uni$ again and step (1) is repeated until
% all $\tau_j$ are below $\tau_{\min}$.
% Recall that, with at most $4\ell\epsi^{-1}\log(n)$ successful iterations
% regarding solution $A_j$, the threshold value $\tau_j$ will decrease
% or the algorithm terminates because of $m_0=0$.
% Moreover, for each $j\in [\ell]$,
% there are at most $\epsi^{-1}\log\left(\frac{M}{\tau_{\min}}\right)$
% possible threshold values
% resulting in at most $\ell\epsi^{-1}\log\left(\frac{M}{\tau_{\min}}\right)$ repeats of step (1).
% Let $Y_i$ be the number of iterations between 
% the $(i-1)$-th success and $i$-th success.
% By Lemma~\ref{lemma:indep}, since an iteration success with 
% a probability of $1/2$, it holds that $\ex{Y_i} \le 2$.
% Then, the expected number of iterations after $4\ell\epsi^{-1}\log(n)$ successful iterations can be bounded as follows,
% \begin{align*}
% 	\ex{\sum_{i=1}^{4\ell\epsi^{-1}\log(n)} Y_i} \le 8\ell\epsi^{-1}\log(n).
% \end{align*}
% Therefore, the query complexity of the algorithm can be bounded as follows,
% \begin{align*}
% 	\ex{\text{Queries}} & \le \sum_{i =1}^{N} \sum_{j = 1}^\ell 2|V_{j, i}|\\
% 	& \le \ell\epsi^{-1}\log\left(\frac{M}{\tau_{\min}}\right) \cdot
% 	\ex{\sum_{i=1}^{4\ell\epsi^{-1}\log(n)} Y_i\cdot 2\ell n}\\
% 	&\le 16 \ell^3 \epsi^{-1}n\log(n)\log\left(\frac{M}{\tau_{\min}}\right).
% \end{align*}
\end{proof}

\subsection{Analysis of Theorem~\ref{thm:ptgone} in Section~\ref{sec:ptg}}
\label{apx:ptgone-guarantee}
In this section, we provide the analysis of the parallel $1/4-\epsi$ approximation algorithm.
\thmptgone*
\begin{proof}[Proof of Theorem~\ref{thm:ptgone}]
The adaptivity and query complextiy are quite straightforward.
In the following, we will analyze the approximation ratio.

Let $S = O$ in Lemma~\ref{lemma:ptgone},
it holds that
\begin{align}
	& \ff{A_l'} \ge \ff{A_l}, \forall l = 1,2 \label{inq:ptg-1}\\
	& A_1 \cap A_2 = \emptyset \label{inq:ptg-2}\\
	& \marge{O}{A_1} + \marge{O}{A_2} \le \frac{1}{(1-\epsi)^2}\left(\ff{A_1'} + \ff{A_2'} \right) + \frac{2\epsi M}{1-\epsi}\label{inq:ptg-3}
\end{align}
Then,
\begin{align*}
	\ff{O} &\le \ff{O\cup A_1} + \ff{O\cup A_2} \tag{Submodularity, Nonnegativity, Inequality~\eqref{inq:ptg-2}} \\
	&\le \ff{A_1} + \ff{A_2} + \frac{1}{(1-\epsi)^2}\left(\ff{A_1'} + \ff{A_2'} \right) + \frac{2\epsi M}{1-\epsi} \tag{Inequality~\eqref{inq:ptg-3}}\\
	&\le 2\left(1+\frac{1}{(1-\epsi)^2}\right)\ff{G} + \frac{2\epsi}{1-\epsi}\ff{O} \tag{Inequality~\eqref{inq:ptg-1} and $G = \argmax\{\ff{A_1'}, \ff{A_2'}\}$}\\
	\Rightarrow \ff{G} &\ge \frac{(1-3\epsi)(1-\epsi)}{2\left((1-\epsi)^2 + 1+\frac{1}{k}\right)}\ff{O}\ge \left(\frac{1}{4}-\epsi\right)\ff{O} 
\end{align*}
\end{proof}

\subsection{Pseudocode and Analysis of Theorem~\ref{thm:ptgtwo} in Section~\ref{sec:ptg}}
\label{apx:ptgtwo}
\begin{algorithm}[ht]
\Fn{\ptgtwo($f, k, \epsi$)}{
	\KwIn{evaluation oracle $f:2^{\uni} \to \reals$, 
        constraint $k$, constant $\ell$, error $\epsi$}
	\Init{$G\gets \emptyset, \epsi' \gets \frac{\epsi}{2}, m\gets \left\lfloor \frac{k}{\ell} \right\rfloor, M\gets \max_{x\in\uni}\ff{\{x\}}, \tau_{\min}\gets \frac{\epsi'M}{k}$}
	\For{$i\gets 1$ to $\ell$}{
		$\{A_l': l\in [\ell]\} \gets \ptgone(f_{G}, m, \ell, \tau_{\min}, \epsi')$\;
		$G\gets$ a random set in $\{G\cup A_l': l\in [\ell]\}$\;
	}
	\Return{$G$}
	}
\caption{A randomized $(1/e-\epsi)$-approximation algorithm with $\oh{\ell^{3}\epsi^{-2}\log(n)\log(k)}$ adaptivity and $\oh{\ell^4\epsi^{-2}n\log(n)\log(k)}$ query complexity}\label{alg:ptg}
\label{alg:ptgtwo}
\end{algorithm}
In this section, we provide the pseudocode of the parallel $1/e-\epsi$ approximation algorithm with its analysis.

First, we provide the following lemma which provides a lower bound
on the gains achieved after every iteration in Alg.~\ref{alg:ptgtwo}.
\begin{lemma}\label{lemma:ptgtwo-recur}
For any iteration $i$ of the outer for loop in Alg.~\ref{alg:ptgtwo},
it holds that
\begin{align*}
\text{1) } & \ex{\ff{O\cup G_i}}\ge \left(1-\frac{1}{\ell}\right) \ex{\ff{O\cup G_{i-1}}}\\
\text{2) } & \ex{\ff{G_i} - \ff{G_{i-1}}}
\ge\frac{1}{1+\frac{\ell}{(1-\epsi')^2}}\left(1-\frac{1}{m+1}\right)\left(\left(1-\frac{1}{\ell}\right)  \ex{\ff{O\cup G_{i-1}}} - \ex{\ff{G_{i-1}}} - \frac{\epsi'}{1-\epsi'}\ff{O}\right).
\end{align*}
\end{lemma}
\begin{proof}[Proof of Lemma~\ref{lemma:ptgtwo-recur}]
Fix on $G_{i-1}$ at the beginning of this iteration,
Since $\{A_l: l\in [\ell]\}$ are pairwise disjoint sets,
by Proposition~\ref{prop:sum-marge}, it holds that
\[\exc{\ff{O\cup G_i}}{G_{i-1}} = \frac{1}{\ell}\sum_{l\in [\ell]}\ff{O\cup G_{i-1}\cup A_l} \ge \left(1-\frac{1}{\ell}\right)\ff{O\cup G_{i-1}}.\]
Then, by unfixing $G_{i-1}$, the first inequality holds.

To prove the second inequality, also consider fix on $G_{i-1}$ at the beginning of iteration $i$.
By Lemma~\ref{lemma:ptgone},
$\{A_l: l\in [\ell]\}$ are paiewise disjoint sets,
and the following inequalities hold,
\begin{align}
&A_l'\subseteq A_l, \marge{A_l'}{\emptyset} \ge \marge{A_l}{\emptyset}, \forall 1\le l \le \ell \label{inq:ptgtwo-1}\\
&\marge{O_{l}}{A_{l}}\le \frac{\marge{A_{l}'}{\emptyset}}{(1-\epsi')^2}+\frac{\epsi' M}{(1-\epsi')\ell}, \forall 1\le l \le \ell \label{inq:ptgtwo-2}\\
&\marge{O_{l_2}}{A_{l_1}} + \marge{O_{l_1}}{A_{l_2}} \le \frac{1+\frac{1}{m}}{(1-\epsi')^2}\left(\marge{A_{l_1}'}{\emptyset}+\marge{A_{l_2}'}{\emptyset}\right) + \frac{2\epsi' M}{(1-\epsi')\ell}, \forall 1\le l_1 < l_2 \le \ell \label{inq:ptgtwo-3}
\end{align}
Then,
\begin{align*}
\sum_{l\in [\ell]}\marge{O}{A_l\cup G_{i-1}} &\le \sum_{l_1\in [\ell]}\sum_{l_2\in [\ell]}\marge{O_{l_1}}{A_{l_2}\cup G_{i-1}}\tag{Proposition~\ref{prop:sum-marge}}\\
& = \sum_{l \in [\ell]}\marge{O_{l}}{A_{l}\cup G_{i-1}} + \sum_{1\le l_1< l_2 \le \ell} \left(\marge{O_{l_1}}{A_{l_2}\cup G_{i-1}} +\marge{O_{l_2}}{A_{l_1}\cup G_{i-1}}\right) \tag{Lemma~\ref{lemma:tg-par-A}}\\
& \le \sum_{l \in [\ell]}\left(\frac{\marge{A_{l}'}{G_{i-1}}}{(1-\epsi')^2}+\frac{\epsi' M}{(1-\epsi')\ell}\right)\\
&\hspace*{2em}+\sum_{1\le l_1< l_2 \le \ell} \left(\frac{\left(1+\frac{1}{m}\right)}{(1-\epsi')^2}\left(\marge{A_{l_1}'}{G_{i-1}}
+\marge{A_{l_2}'}{G_{i-1}}\right)
+\frac{2\epsi' M}{(1-\epsi')\ell}\right)\tag{Inequalities~\eqref{inq:ptgtwo-2} and~\eqref{inq:ptgtwo-3}}\\
&\le \frac{\ell}{(1-\epsi')^2}\left(1+\frac{1}{m}\right)\sum_{l \in [\ell]}\marge{A_{l}'}{G_{i-1}} + \frac{\epsi' \ell}{1-\epsi'}\ff{O}\tag{$M \le \ff{O}$}
\end{align*}
\begin{align*}
\Rightarrow \left(1+\frac{\ell}{(1-\epsi')^2}\right)\left(1+\frac{1}{m}\right) \sum_{l\in [\ell]}\marge{A_l'}{G_{i-1}} &\ge \sum_{l\in [\ell]}\ff{O\cup A_l\cup G_{i-1}} -\ell\ff{G_{i-1}} - \frac{\epsi' \ell}{1-\epsi'}\ff{O} \tag{Inequality~\eqref{inq:ptgtwo-1}}\\
&\ge \left(\ell-1\right)\ff{O\cup G_{i-1}}-\ell\ff{G_{i-1}} - \frac{\epsi' \ell}{1-\epsi'}\ff{O}
\end{align*}
Thus,
\begin{align*}
&\exc{\ff{G_i} - \ff{G_{i-1}}}{G_{i-1}}  = \frac{1}{\ell}\sum_{l \in [\ell]}\marge{A_{l}'}{G_{i-1}}\\
&\ge \frac{1}{1+\frac{\ell}{(1-\epsi')^2}} \frac{m}{m+1}\left(\left(1-\frac{1}{\ell}\right)  \ff{O\cup G_{i-1}} - \ff{G_{i-1}} - \frac{\epsi'}{1-\epsi'}\ff{O}\right)\tag{Proposition~\ref{prop:sum-marge}}
\end{align*}
By unfixing $G_{i-1}$, the second inequality holds.
\end{proof}
\thmptgtwo*
\begin{proof}[Proof of Theorem~\ref{thm:ptgtwo}]
Since the algorithm contains a for loop
which runs \ptgone $\ell = \oh{1/\epsi}$ times,
by Lemma~\ref{lemma:ptgone},
the adaptivity, query complexity and success probability holds immediately.

Next, we provide the analysis of approximation ratio.
By solving the recurrence in Lemma~\ref{lemma:ptgtwo-recur},
we calculate the approximation ratio of the algorithm as follows,
\begin{align*}
&\ex{\ff{G_{i}}}  \ge \left(1-\frac{1}{\ell}\right) \ex{\ff{G_{i-1}}}
+ \frac{1}{1+\frac{\ell}{(1-\epsi')^2}}\left(1-\frac{1}{m+1}\right)\left(\left(1-\frac{1}{\ell}\right)^i - \frac{\epsi'}{1-\epsi'}\right)\ff{O}\\
\Rightarrow& \ex{\ff{G_\ell}} \ge \frac{\ell}{1+\frac{\ell}{(1-\epsi')^2}}\left(1-\frac{1}{m+1}\right)\left(\left(1-\frac{1}{\ell}\right)^\ell - \frac{\epsi'}{1-\epsi'}\left(1-\left(1-\frac{1}{\ell}\right)^\ell\right)\right)\ff{O}\\
&\hspace*{4em} \ge \frac{\ell-1}{1+\frac{\ell}{(1-\epsi')^2}}\left(1-\frac{1}{m+1}\right)\left(e^{-1} - \frac{\epsi'}{1-\epsi'}\left(1-e^{-1}\right)\right)\ff{O}\\
&\hspace*{4em} \ge \frac{1}{1-\frac{\ell}{k}}\left((1-\epsi')^2 - \frac{2}{\ell}\right)\left(1-\frac{\ell}{k}\right)^2\left(e^{-1} - \frac{\epsi'}{1-\epsi'}\left(1-e^{-1}\right)\right) \ff{O}\\
% &\hspace*{4em} \ge \frac{1}{1-\frac{\ell}{k}}\left(1-\epsi' - \frac{2}{\ell}\right)\left(1-\frac{2\ell}{k}\right)\left(e^{-1} - \frac{\epsi'}{1-\epsi'}\left(1-e^{-1}\right)\right) \ff{O}\\
&\hspace*{4em} \ge \frac{1}{1-\frac{\ell}{k}}\left((1-\epsi')^2 - \frac{2}{\ell}-\frac{2(1-\epsi')^2 \ell}{k}\right)\left(e^{-1} - \frac{\epsi'}{1-\epsi'}\left(1-e^{-1}\right)\right) \ff{O}\\
&\hspace*{4em} \ge \frac{1}{1-\frac{\ell}{k}} \left(1-(e+1)\epsi'\right)\left(e^{-1} - \frac{\epsi'}{1-\epsi'}\left(1-e^{-1}\right)\right) \ff{O}\tag{$\ell\ge \frac{2}{e\epsi'}, k\ge \frac{2(1-\epsi')^2\ell}{e\epsi'-\frac{2}{\ell}}$}\\
&\hspace*{4em} \ge \frac{1}{1-\frac{\ell}{k}} \left(e^{-1}-\epsi\right)\ff{O}\tag{$\epsi' = \frac{\epsi}{2}$}.
\end{align*}
By Inequality~\ref{inq:tgtwo-dif-opt},
the approximation ratio of Alg.~\ref{alg:tgtwo} is $e^{-1}-\epsi$.
\end{proof}














