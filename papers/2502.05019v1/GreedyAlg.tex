\documentclass{article}
%\documentclass[anon,12pt]{colt2025}
\usepackage{graphicx} % Required for inserting images
 \usepackage{tikz}
 \usepackage{booktabs}
 \usepackage{natbib}
 \usepackage{pgfplots}
\usetikzlibrary{3d}
\usetikzlibrary{backgrounds}
\pgfdeclarelayer{bg}
\pgfsetlayers{bg,main}
\usepackage[left=1.5cm, right=2cm]{geometry}

 \newgeometry{left=3cm,bottom=4cm}

\tikzset{
  background/.style={%
    execute at begin node={\begin{pgfonlayer}{bg}},
    execute at end node={\end{pgfonlayer}}
  }
}

 %\usepackage{algorithm}
  %\usepackage{algorithmic}
 % \usepackage[square,comma,sort&compress]{natbib}
  \usepackage{algorithm,algpseudocode}
  \usepackage{tikz}
  \usetikzlibrary{calc}
  \usepackage{tkz-euclide}
  \usepackage{xcolor}
\usetikzlibrary{arrows}
\usetikzlibrary{shapes.geometric, arrows.meta}
\usepackage{amsmath}
  
%\algnewcommand{\algorithmicforeach}{\textbf{for each}}
%\algdef{SE}[FOR]{ForEach}{EndForEach}[1]
  %{\algorithmicforeach\ #1\ \algorithmicdo}% \ForEach{#1}
  %{\algorithmicend\ \algorithmicforeach}% \EndForEach
 \usetikzlibrary{math}
\title{$O(\sqrt{T})$ Static Regret and Instance Dependent Constraint Violation for Constrained Online Convex Optimization}
\author{%
  Rahul Vaze, Abhishek Sinha,  \\
 School of Technology and Computer Science \\
  Tata Institute of Fundamental Research \\
  Mumbai 400005, India \\
  \texttt{rahul.vaze@gmail.com},
  \texttt{abhishek.sinha@tifr.res.in}
}
%\author{Rahul vaze, Abhishek Sinha}
%\date{October 2024}

\begin{table}[tb]
\small
\centering
\setlength{\tabcolsep}{1.3mm}
\begin{tabular}{ll|cccc}
\toprule
& \textbf{Setting} & \textbf{Mistral} & \textbf{Llama-3} & \textbf{GPT-3.5} & \textbf{GPT-4o} \\
\midrule[0.5pt]
\multirow{2}{*}{\textbf{Move}}& Ori & 2.00 & 1.00 & 4.00 & 13.00 \\
& Small & 12.00& 12.00 & 20.00 & 28.00\\
\midrule
\multirow{2}{*}{\textbf{Copy}}&Ori & 2.00 & 4.00 & 4.00 & 15.00 \\
&Small &12.00 &9.00 &14.00 & 34.00 \\
\bottomrule
\end{tabular}
\caption{Acc (in percentage) of LLMs with different input sizes. See~\tref{tab:large matrix_plus} for the Not M\% scores.}
\vspace{-0.1in}
\label{tab:large matrix}
\end{table}

\iffalse
\begin{table}[tb]
\small
\centering
\setlength{\tabcolsep}{1mm}
\begin{tabular}{ll|cccccc}
\toprule
& \multirow{2}{*}{\textbf{Setting}} & \multicolumn{2}{c}{\textbf{Mistral}} & \multicolumn{2}{c}{\textbf{Llama-3}} & \multicolumn{2}{c}{\textbf{GPT}} \\
\cmidrule(lr){3-4} \cmidrule(lr){5-6} \cmidrule(lr){7-8}
&~ & \textbf{7B} & \textbf{8*7B} & \textbf{8B} & \textbf{70B} &\textbf{3.5} &\textbf{4o}\\
\midrule[0.5pt]
\multirow{2}{*}{\textbf{Move}} & Ori & 2.00 && 1.00 && 4.00 & 13.00 \\
& Small & 12.00 & & 12.00 && 20.00 & 28.00 \\
\midrule
\multirow{2}{*}{\textbf{Copy}} & Ori & 2.00 && 4.00 && 4.00 & 15.00 \\
& Small & 12.00 && 9.00 && 14.00 & 34.00 \\
\bottomrule
\end{tabular}
\caption{Acc (in percentage) of LLMs with different input sizes. See~\tref{tab:large matrix_plus} for the Not M\% scores.}
\vspace{-0.1in}
\label{tab:large matrix}
\end{table}
\fi

\begin{table}[tb]
\small
\centering
\setlength{\tabcolsep}{2mm}
\begin{tabular}{lcc|cc}
\toprule
\textbf{LLM} &\textbf{Move} & \textbf{Copy} & \textbf{Comp} %& \textbf{Rank}
\\
\midrule
$\text{Mistral-FT}_{\text{Move}}$ & 19.00 & 4.00& 0.00  \\
$\text{Mistral-FT}_{\text{Copy}}$ &11.00 &32.00 & 2.00\\
$\text{Mistral-FT}_{\text{Move+Copy}}$ & 25.00& 32.00&4.00\\
\midrule
$\text{Llama-3-FT}_{\text{Move}}$ &21.00 &4.00 &0.00\\
$\text{Llama-3-FT}_{\text{Copy}}$ &12.00 &33.00 & 3.00\\
$\text{Llama-3-FT}_{\text{Move+Copy}}$ &26.00 & 27.00&5.00\\
\midrule
GPT-3.5 &4.00 &4.00 & 0.00\\
%GPT-4 &14.00 &13.00 & 4.00\\
GPT-4o &13.00 &15.00 & 2.00\\
\bottomrule
\end{tabular}
\caption{Acc (in percentage) on tasks composing Move and Copy (Comp).
%fine-tuned on single and multiple tasks. 
%Comp refers to tasks composed of Move and Copy. 
See~\tref{tab:composition_plus} for the Not M\% scores.}
\vspace{-0.2in}
\label{tab:composition}
\end{table}

\begin{table*}[tb]
\renewcommand\arraystretch{1.1}
\centering
\setlength{\tabcolsep}{2.5mm}
\small
\begin{tabular}{lcccccc|c}
\toprule[1pt]
\multirow{2}*{LLM} & \multicolumn{6}{c}{\textbf{Individual Atomic Operation}} & \multicolumn{1}{c}{\textbf{ Composition}} \\
\cmidrule{2-8}
& \multicolumn{1}{c}{\textbf{Move}} & \multicolumn{1}{c}{\textbf{Change Color}} & \multicolumn{1}{c}{\textbf{Copy}} & \multicolumn{1}{c}{\textbf{Mirror}} & \multicolumn{1}{c}{\textbf{Fill Internal}} & \multicolumn{1}{c}{\textbf{Scale}} & \multicolumn{1}{c}{\textbf{ARC}}  \\
% & Acc$\uparrow$ & Acc$\uparrow$ & Acc$\uparrow$ & Acc$\uparrow$ & Acc$\uparrow$ & Acc$\uparrow$ & Acc$\uparrow$ \\
\midrule[0.5pt]
%Mistral & 2.00 & 15.00 & 2.00 & 1.00 & 9.00 & 0.00 & 2.00 \\
%$\text{Mistral-FT-atomic}$ & 12.00 & 100.00 & 20.00 & 26.00 & 97.00 & 89.00 & 1.00 \\
%$\text{Mistral-FT-atomic-arc}$ &14.00 &99.00&14.00&24.00&97.00&87.00&6.00\\
%\midrule
Llama-3 & 1.00 & 39.00 & 4.00 & 2.00 & 63.00 & 1.00 & 5.00 \\
$\text{Llama-3-FT-arc}$ &2.00 &73.00 &5.00 &3.00 &88.00 &0.00 &9.00 \\
$\text{Llama-3-FT-atomic}$ & 13.00 & 98.00 & 14.00 & 27.00 & 97.00 & 78.00 & 2.00 \\
$\text{Llama-3-FT-atomic-arc}$ & 12.00 &97.00&17.00&28.00&98.00&79.00&6.00 \\
\midrule
%GPT-3.5 &4.00&48.00&4.00&6.00&58.00&1.00 &6.00 \\
%GPT-4 &14.00 &97.00&13.00&14.00&100.00&3.00&17.00 \\
GPT-4o & 13.00&98.00&15.00&12.00&96.00&2.00&19.00 \\
\bottomrule[1pt]
\end{tabular}
\caption{Results of LLMs on individual and composition of atomic operations. %FT refers to fine-tune on the atomic operation data. S
See~\tref{tab:fine-tune arc performance_plus} for the Not M\% scores.}
\vspace{-0.2in}
\label{tab:fine-tune arc performance}
\end{table*}

\begin{table}[tb]
\small
\centering
\setlength{\tabcolsep}{3mm}
\begin{tabular}{lcccc}
\toprule
\textbf{LLM} &\textbf{Size} & \textbf{Location} & \textbf{Transpose} %& \textbf{Rank}
\\
\midrule
Mistral &0.32 &0.00 &0.02 %&0.29 
\\
Llama-3 & 0.63&0.04 &0.04 %&0.86
\\
%Mistral 8*7B & & & %&0.29 
%\\
%Llama-3 70B & & & %&0.86
%\\
\midrule
GPT-3.5 &0.93 &0.43 &0.34 %&0.30
\\
%GPT-4o &\textbf{1.00} &\textbf{0.77} &\textbf{0.83} \\%s&0.95\\
GPT-4o &\textbf{1.00} &\textbf{0.91} &\textbf{0.91} \\
\bottomrule
\end{tabular}
\caption{LLMs' accuracy on matrix-related questions. The best results under each column are \textbf{boldfaced}.}
\vspace{-0.1in}
\label{tab:understand matrix}
\end{table}

\section{Challenge on Input Format}
\label{sec:matrix}



Since LLMs cannot process visual inputs, we follow~\citet{wang2023hypothesis} to convert the 2D visual input-output grids in ARAOC tasks into matrix-format before feeding them to the LLMs (\sref{sec:arc setting}). However, it remains uncertain that whether this conversion affects LLMs' performances on ARAOC, since LLMs are mostly trained on natural language data, and may not understand such matrix-format inputs well. In this section, we first try to answer this question (\sref{sec:understand matrix}), then investigate a strategy to remedy its potential challenges (\sref{sec:natural language}).


\subsection{Matrix-format Understanding}
\label{sec:understand matrix}
%We analyze this problem from two perspectives. First, 
We first investigate whether LLMs understand the input matrices well. Specifically, we select the testing input matrices from the 100 ARAOC Move tasks, and ask LLMs to output the size, transpose, and subgrid's corner elements' locations of each matrix (see the input prompt in~\fref{fig:matrix property prompt}). Our intuition is that if LLMs correctly answer these questions, they should have understood the matrix-format input. % and this should not affect their performances on ARAOC tasks. 
Results are shown in~\tref{tab:understand matrix}, where GPT-4o answers these questions with high accuracy, indicating that it comprehends such matrix-format inputs well. However, other LLMs perform poorly on these tasks, which may further affect their results on ARAOC. 



To further investigate the impact of matrix-format input, we re-evaluate $\text{Llama-3-FT}_{\text{Move+Copy}}$ from~\tref{tab:composition} and GPT-4o on the Move and Copy tasks without using the location information of subgrids, as detailed in Appendix~\ref{appendix:banning}. The results in Appendix~\ref{appendix:banning} show that prohibiting the use of location information do reduce LLMs' performances on both tasks, indicating that a fundamental understanding of matrices is crucial for completing ARAOC and ARC tasks. However, as the combined results from~\tref{tab:araoc results} and~\tref{tab:understand matrix} suggest, possessing matrix understanding alone does not guarantee good performance on these tasks.



\begin{table}[tb]
\small
\centering
\setlength{\tabcolsep}{0.5mm}
\begin{tabular}{ll|cccc}
\toprule
& \textbf{Method}& \textbf{Mistral}& \textbf{Llama-3} & \textbf{GPT-3.5}%& \textbf{GPT-4}
&\textbf{GPT-4o}\\
\midrule[0.5pt]
\multirow{2}{*}{\textbf{Move}}& w/o NL &2.00 &1.00 &4.00 & 13.00\\%&14.00 \\
& NL & 5.00 &12.00 &23.00 &53.00 \\%49.00 \\
\midrule[0.5pt]
\multirow{2}{*}{\textbf{Color}}& w/o NL &15.00 &39.00 &48.00 & 98.00\\%97.00 \\
& NL &3.00 &83.00 &59.00 &99.00 \\%92.00 \\
\midrule[0.5pt]
\multirow{2}{*}{\textbf{Copy}}& w/o NL &2.00 &4.00 &4.00 & 15.00\\%13.00 \\
& NL &2.00 &6.00 &14.00 &40.00 \\%45.00 \\
\midrule[0.5pt]
\multirow{2}{*}{\textbf{Mirror}}& w/o NL &1.00 &2.00 &6.00 & 12.00\\%14.00 \\
& NL &2.00 &8.00 &21.00 & 30.00\\%29.00 \\
\midrule[0.5pt]
\multirow{2}{*}{\textbf{Fill Internal}}& w/o NL &9.00 &63.00 &58.00 &96.00 \\%100.00 \\
& NL &0.00 &10.00 &35.00 & 72.00\\%85.00 \\
\midrule[0.5pt]
\multirow{2}{*}{\textbf{Scale}}& w/o NL &0.00 &1.00 &1.00 &2.00 \\%3.00 \\
& NL &0.00 &2.00 &0.00 &4.00 \\%6.00 \\
\bottomrule
\end{tabular}
\caption{Acc (in percentage) of LLMs with natural language inputs (NL). %w/o NL refers to the results in~\tref{tab:araoc results}. 
See Not M \% scores in~\tref{tab:natural language input_plus}.}
\vspace{-0.2in}
\label{tab:natural language input}
\end{table}


\iffalse
\begin{table}[tb]
\small
\centering
\setlength{\tabcolsep}{0.5mm}
\begin{tabular}{ll|cccccc}
\toprule
& \multirow{2}{*}{\textbf{Setting}} & \multicolumn{2}{c}{\textbf{Mistral}} & \multicolumn{2}{c}{\textbf{Llama-3}} & \multicolumn{2}{c}{\textbf{GPT}} \\
\cmidrule(lr){3-4} \cmidrule(lr){5-6} \cmidrule(lr){7-8}
&~ & \textbf{7B} & \textbf{8*7B} & \textbf{8B} & \textbf{70B} &\textbf{3.5} &\textbf{4o}\\
\midrule[0.5pt]
\multirow{2}{*}{\textbf{Move}}& w/o NL &2.00 &&1.00& &4.00 & 13.00\\%&14.00 \\
& NL & 5.00& &12.00& &23.00 &53.00 \\%49.00 \\
\midrule[0.5pt]
\multirow{2}{*}{\textbf{Color}}& w/o NL &15.00& &39.00& &48.00 & 98.00\\%97.00 \\
& NL &3.00& &83.00& &59.00 &99.00 \\%92.00 \\
\midrule[0.5pt]
\multirow{2}{*}{\textbf{Copy}}& w/o NL &2.00& &4.00& &4.00 & 15.00\\%13.00 \\
& NL &2.00& &6.00& &14.00 &40.00 \\%45.00 \\
\midrule[0.5pt]
\multirow{2}{*}{\textbf{Mirror}}& w/o NL &1.00& &2.00& &6.00 & 12.00\\%14.00 \\
& NL &2.00& &8.00& &21.00 & 30.00\\%29.00 \\
\midrule[0.5pt]
\multirow{2}{*}{\textbf{Fill Internal}}& w/o NL &9.00& &63.00& &58.00 &96.00 \\%100.00 \\
& NL &0.00& &10.00& &35.00 & 72.00\\%85.00 \\
\midrule[0.5pt]
\multirow{2}{*}{\textbf{Scale}}& w/o NL &0.00& &1.00& &1.00 &2.00 \\%3.00 \\
& NL &0.00& &2.00& &0.00 &4.00 \\%6.00 \\
\bottomrule
\end{tabular}
\caption{Acc (in percentage) of LLMs with natural language inputs (NL). %w/o NL refers to the results in~\tref{tab:araoc results}. 
See Not M \% scores in~\tref{tab:natural language input_plus}.}
\vspace{-0.2in}
\label{tab:natural language input}
\end{table}
\fi


%\subsection{Fine-tuning on Matrix Data}
%\label{sec:fine tune on matrix}
%Given that both open-sourced LLMs have difficulties understanding matrix-format inputs, we first investigate whether fine-tuning LLMs on matrix property-related questions could improve their performances on tasks in ARAOC. Specifically, we generate 3000 extra input grids of the Move task and calculate the size, transpose, and locations of the subgrid's corner elements for these matrices as ground truths. Furthermore, since correctly recognizing the location of the subgrid may contribute more to finish the Move and Copy tasks compared to other properties, we create additional ground truths only with the gold locations of the subgrid's corner elements. We fine-tune Mistral and Llama-3 on these two sets of matrix property data using the same strategy and configuration described in~\sref{sec:evaluate on original arc}, and evaluate them on the Move and Copy tasks in ARAOC, respectively.

%As shown in~\tref{tab:fine-tune on matrix}, fine-tuning solely with the locations of subgrids provides more benefits than fine-tuning with all three matrix properties. This indicates that different atomic operations require a specific understanding of matrices, and acquiring this understanding can better enhance performance on specific atomic operations. However, compared to the results in~\tref{tab:araoc results}, all matrix-property data appear to decrease LLMs' performance on ARAOC tasks. This demonstrates that simply fine-tuning LLMs on matrix property data may not be an effective solution for improving their inductive reasoning abilities on ARAOC.


\subsection{Switching Matrix into Natural Language}
\label{sec:natural language}
%Finally, we propose an approach to relieve the challenge on input understanding, which is inspired by our previous finding that GPT-4 struggles to derive correct transformation rules from ARAOC's input matrices (\tref{tab:araoc results}) despite its excellent understanding of these matrices (\tref{tab:understand matrix}). 
%From previous sections we conclude that the challenges of ARAOC and ARC tasks is not due to LLMs' understanding of matrix-format inputs.
%Therefore, a new type of inputs for representing such tasks is necessary. 
Since LLMs are predominantly trained on natural language rather than matrix-format data, we further propose to convert the matrix-format input-output grids into natural language with the aid of a coordinate system-based prompt (listed in~\fref{fig:natural language prompt}). We evaluate LLMs using this new prompt on ARAOC, and the results are presented in~\tref{tab:natural language input}.

Notably, we find that on tasks that LLMs originally cannot answer well (Move, Copy, Mirror, and Scale), using natural language inputs can largely boost their performances. As for tasks that are relatively easy for LLMs, converting matrix-format input to natural language still keep the good performances. %These results illustrate the effectiveness of our proposed approach. 
One exception appears to be the Mistral model, whose performance decreases with the natural language prompt. This is probably because this model is not strong enough to encode the natural language input that can be handled by other LLMs, which makes its results not indicative.

\textbf{Overall, we conclude that LLMs' failure on fluid intelligence tests is not mainly due to their understanding of the specific matrix-format inputs, but their limitations on encoding such inputs for obtaining global representations of the input tasks.}










\begin{document}


\maketitle
\begin{abstract} The constrained version of the standard online convex optimization (OCO) framework, called COCO is considered, where on every round, a convex cost function and a convex constraint function are revealed to the learner after it chooses the action for that round.
The objective is to simultaneously minimize the static regret and cumulative constraint violation (CCV). 
An algorithm is proposed that guarantees a static regret of $O(\sqrt{T})$ and a CCV of $\min\{\cV, O(\sqrt{T}\log T) \}$, where $\cV$ depends on the distance between the consecutively revealed constraint sets, the shape of constraint sets, dimension of action space and the diameter of the action space. For special cases of constraint sets, $\cV=O(1)$. Compared to the state of the art results, static regret of $O(\sqrt{T})$ and CCV of $O(\sqrt{T}\log T)$, that were universal, the new result on CCV is instance dependent, which is derived by exploiting the geometric properties of the constraint sets.
\end{abstract}
% 
% 
The widespread integration of communication networks and smart devices in modern control systems has increased the vulnerability of industrial systems to online cyber-attacks, e.g., Industroyer, Blackenergy, etc \citep{osti_1505628}.
% Modern control systems have seen a large push to include communication networks and smart devices to increase performance, made possible by improvements in communication device cost and energy consumption. This trend has been coupled with the usage of open-standard communication protocols among industrial control systems, making them vulnerable to online cyber-attacks such as Industroyer, Blackenergy, etc \citep{osti_1505628}. 
To counter this, methods have been developed to improve security by achieving attack detection, mitigation, and monitoring, among others \citep{sandberg2022secure}. This paper focuses on active attack diagnosis to mitigate stealthy attacks. 
%
%\subsection{Literature review}

Active diagnosis techniques rely on the inclusion of additional moduli to control systems
% inclusion within the control system of additional moduli 
to alter the behavior of the system compared to information known by the attacker. 
For instance, the concept of additive watermarking was introduced in \cite{mo2015physical}, where noise signals of known mean and variance are added at the plant and compensated for it at the controller. 
This compensation, however, is not exact, causing some performance degradation. Thus, trade-offs between performance and detectability  are necessary \citep{zhu2023detection}.
% A later work \citep{zhu2023detection} designs the watermark signal by trading performance for detection. Thus, although additive watermarking serves as a good detection scheme, they endure performance losses even in the nominal case. 

In encrypted control \citep{darup2021encrypted}, the sensor data is encrypted, sent to the controller, and then operated on directly. Encrypted input signals are sent back to the plant for decryption. Although encryption is widespread in IT security, in control systems it presents some concerns, such as the introduction of time delays \citep{stabile2024verifiable}, while it may present inherent weaknesses \citep{alisic2023model}.
% they are not preferred as they introduce time delays \citep{stabile2024verifiable} which can cause instability, and some encryption schemes can be very weak  \citep{alisic2023model}. 

In moving target defense \citep{griffioen2020moving}, the plant is augmented with fictitious dynamics, known to the controller. The plant output is transmitted to the controller along with the fictitious states over a network under attack. 
The additional measurements then aide in the detection of attacks. 
This comes at the cost of higher communication bandwidth needs, which increases rapidly with the dimension of the augmented systems.
% Since the dynamics of the fictitious dynamics are exactly known to the controller, the attack is detected easily. However, when the scale of the system increases, the communication bandwidth used by moving the target defense approach increases rapidly. 

Other recently proposed works include two-way coding \citep{fang2019two}, a weak encryuption technique, and dynamic masking \citep{abdalmoaty2023privacy}, which enhances privacy as well as security, have been shown to be effective against zero-dynamics attacks.
% Two-way coding \citep{fang2019two} and dynamic masking \citep{abdalmoaty2023privacy} are other recently proposed approaches. Two-way coding is another form of weak encryption technique whilst dynamic masking proposes an architecture that enhances both privacy and security. These schemes are shown to be effective against zero dynamics attacks but remain to be studied for other classes of attacks. 
% Recent extensions include \citep{mukherjee2021secure,ramos2024privacy}.
% Some other works which are related are \citep{mukherjee2021secure}, an extension of \cite{fang2019two}. The work \citep{ramos2024privacy} is an extension of moving target defense for multi-agent systems. 
Furthermore, filtering techniques for attack detection are proposed by \cite{murguia2020security,hashemi2022codesign,escudero2023safety}, while not focusing on stealthy attacks.
% The works \citep{murguia2020security,hashemi2022codesign,escudero2023safety} develop filtering techniques to guarantee safety, without being focused on stealthy covert attacks.

Multiplicative watermarking (mWM) has been proposed by the authors as a diagnosis technique \citep{ferrari2020switching}. mWM consists of a pair of filters on each communication channel between the plant and its controller; the scheme is affine to weak encryption, whereby ``encoding'' and ``decoding'' are done by changing signals' dynamic characteristics through inverse pairs of filters. This enables original signals to be recovered exactly, and thus does not lead to performance degradation.
% A multiplicative watermark is an affine to a weak encryption technique, through which the signal is ``encoded'' by a filter, changing its dynamic behavior. The use of inverse pairs means that the original signal can be recovered, through ``decoding'' via an inverse filter. As such, differently to techniques based on additive watermarking, no performance is lost due to the injection of noise, and there are no bandwidth limitations.

%\subsection{Contributions}
One of the critical features of multiplicative watermarking is that to detect stealthy attacks, the mWM filter parameters must be switched over time. In this paper, an algorithm to optimally design the mWM parameters after a switching event is presented, enhancing detection performance, without changing the switching time.
% This is done without changing the switching time, which is taken as given.

\textcolor{black}{
To formalize the filter design problem, we suppose the defender is interested in optimal performance against adversaries injecting covert attacks with matched system parameters \citep{smith2015covert}, including the mWM parameters prior to the switch. This scenario represents a worst case where malicious agents can take full control of the system while remaining undetected.
Thus, the attack strategy is explicitly included within the formulation of the closed-loop system, and the mWM filters are chosen by solving an optimization problem minimizing the attack-energy-constrained output-to-output gain (AEC-OOG) \citep{anand2023risk}, a variation of the output-to-output gain proposed in  \cite{teixeira2015strategic}.
}
The main contributions of this paper are:
% We consider an adversary injecting a covert attack with matched system parameters \citep{smith2015covert}, i.e., an attacker with full knowledge of the control system parameters, including those of the mWM filters before the switch. This scenario is taken as a worst case, as it has been shown that this class of attacks can be made stealthy. To quantitatively define a cost, the output-to-output gain (OOG) \citep{teixeira2015strategic} is leveraged,
% a metric introduced to evaluate the impact of an additive attack in a control system. %Specifically, OOG evaluates the worst-case performance loss that an attacker injecting an undetectable attack can obtain. 
% Here, the maximum performance loss caused by a stealthy adversary with limited energy is taken, the attack-energy-constrained OOG (AEC-OOG) \citep{anand2023risk}. The main contributions of this paper are:
\begin{enumerate}
%[label=\alph*.]
\item The problem of optimally designing the switching mWM filters is formulated as an optimization problem, with the AEC-OOG is taken as the objective;%where the AEC-OOG is taken as the impact metric; 
\item The worst-case scenario of a covert attack with exact knowledge of plant and mWM filter parameters is embedded within the design problem;
% The optimization problem is defined to incorporate the worst-case scenario of a covert attack with exact knowledge of plant and mWM filter parameters;
\item The feasibility of the optimization problem is shown to be dependent only on stability conditions; 
\item A solution scheme is proposed to promote randomization of the mWM filter parameters such that an eavesdropping adversary cannot remain stealthy.
\end{enumerate} 

This builds on the results of \cite{ferrari2020switching}, where the focus was on the design of the switching protocols, rather than the parameters themselves.
Compared to previous work \citep{gallo2021design}, this paper introduces an optimization problem which is always feasible (thanks to the use of AEC-OOG in the objective), while also considering a more sophisticated class of covert attacks, where the presence of watermark is known to the adversary. 
Moreover, this paper poses a different objective than \citep{zhang2023hybrid}; indeed, while \citep{zhang2023hybrid} provided a design strategy to ensure certain privacy properties, in this paper we address the problem of optimal parameter design following a switching event.


%\subsection{Organization}
The rest of the paper is organized as follows. 
After formulating the problem in Section~\ref{sec:PF}, we propose our design algorithm in Section~\ref{sec:main}, and analyze its properties. It is then evaluated through a numerical example in Section~\ref{sec:NE}, and concluding remarks are given Section~\ref{sec:Con}.
% We provide the problem background in Section~\ref{sec:PF}. We formulate the design problem in Section~\ref{sec:main}, together with an analysis of its properties. The proposed algorithm is evaluated through a numerical example in Section \ref{sec:NE}. Concluding remarks are offered in Section \ref{sec:Con}.
\begin{table*}[h]
    \centering
    % \resizebox{\textwidth}{!}{%
    \addtolength{\tabcolsep}{-0.2pt}
    \begin{tabular}{lccccccccccccc}
        \toprule
        Models & Backbone & Epochs & AP & $AP_{50}$ & $AP_{75}$ & $AP_S$ & $AP_M$ & $AP_L$ & GFLOPs & Params \\
        \midrule
        \multirow{4}{*}{DINO} & ResNet-50(T) & 12 & 49.0 & 66.4 & 53.3 & 31.4 & 52.2 & 64.0 & 249 & 47M\\
         & ResNet-18(S) & 12 & 43.5 & 60.7 & 47.1 & 25.3 & 46.1 & 57.6 & 204 & 31M\\
         % & KD-DETR & 12 & 46.1 & 63.1 & 50.1 & 29.2 & 48.7 & 61.1 & 204 & 31M \\
         & Ours & 12 & 46.4 & 63.3 & 50.4 & 28.6 & 48.7 & 61.4 &204 & 31M \\
         & Gains & - & \textbf{+2.9} & \textbf{+2.6} & \textbf{+3.3} & \textbf{+3.3} & \textbf{+2.6} & \textbf{+3.8} & - & -
         \\
        \midrule
        \multirow{4}{*}{DAB-DETR} & ResNet-50(T) & 50 & 42.3 & 63.0 & 45.1 & 21.6 & 46.0 & 61.3 & 92 & 44M\\
         & ResNet-18(S) & 50 & 33.5 & 54.1 & 34.5 & 13.7 & 35.9 & 52.5 & 50 & 31M\\
         % & KD-DETR & 50 & 35.2 & 54.9 & 36.8 & 14.5 & 38.2 & 53.9 &50 & 31M \\
         & Ours & 50 & 36.3 & 56.2 & 37.9 & 15.5 & 39.6 & 55.3 & 50 & 31M \\
         & Gains & - & \textbf{+2.8} & \textbf{+2.1} & \textbf{+3.4} & \textbf{+1.8} & \textbf{+3.7} & \textbf{+2.8} & - & -
         \\
        \midrule
        \multirow{4}{*}{Deformable-DETR} & ResNet-50(T) & 50 & 44.3 & 63.2 & 48.6 & 26.8 & 47.7 & 58.8 & 174 & 40M\\
         & ResNet-18(S) & 50 & 37.8 & 56.7 & 40.8 & 19.4 & 40.5 & 51.7 & 129 & 24M \\
         % & KD-DETR & 50 & 39.0 & 56.4 & 41.8 & 20.9 & 41.6 & 54.0 & 129 & 24M \\
         & Ours & 50 & 40.0 & 57.8 & 43.4 & 21.9 & 42.6 & 54.1 & 129 & 24M \\
         & Gains & - & \textbf{+2.2} & \textbf{+1.1} & \textbf{+2.6} & \textbf{+2.5} & \textbf{+2.1} & \textbf{+2.4} & - & -
         \\
        \bottomrule
    \end{tabular}
    \caption{Distillation results of our CLoCKDistill method compared to the baseline across different DETR detectors on the COCO dataset. T: Teacher, S: Student. Input image dimensions: (1064, 800). All models converged within the indicated epochs.}
    \label{tab:COCO}
    % }
\end{table*}

\section{Algorithm from \cite{Sinha2024}}
The best known algorithm (Algorithm \ref{coco_sinha}) to solve COCO \cite{Sinha2024} was shown to have the following guarantee. 
\begin{theorem}\label{thm:sinha2024}[\cite{Sinha2024}]
Algorithm \ref{coco_sinha}'s $\textrm{Regret}_{[1:T]} = O(\sqrt{T})$ and  $\textrm{CCV}_{[1:T]} = O(\sqrt{T}\log T)$ when $f_t,g_t$ are convex.
\end{theorem}
We next show that in fact the analysis of \cite{Sinha2024} is tight for the CCV even when $d=1$ and $f_t(x)=f(x)$ and $g_t(x) =g(x)$ for all $t$. 
With finite diameter $D$ and the fact that any $x^\star \in \cX^\star$ belongs to all nested convex bodies $S_t$'s, when $d=1$, one expects that the CCV for any algorithm in this case will be $O(D)$. However, we as we show next,  Algorithm \ref{coco_sinha} does not effectively make use of geometric constraints imposed by nested convex bodies $S_t$'s.

\begin{algorithm}[tb]
   \caption{Online Algorithm from  \cite{Sinha2024}}
   \label{coco_sinha}
\begin{algorithmic}[1]
   \State {\bfseries Input:} Sequence of convex cost functions $\{f_t\}_{t=1}^T$ and constraint functions $\{g_t\}_{t=1}^T,$ $G=$ a common Lipschitz constant, $T=$ Horizon length,
   %an upper bound $G$ to the Euclidean norm of their (sub)-gradients, 
    $D=$ Euclidean diameter of the admissible set $\mathcal{X},$ $\mathcal{P}_\mathcal{X}(\cdot)=$ Euclidean projection oracle on the set $\mathcal{X}$ 
     \State {\bfseries Parameter settings:} 
     \begin{enumerate}
     	\item \textbf{Convex cost functions:} $\beta = (2GD)^{-1}, V=1, \lambda = \frac{1}{2\sqrt{T}}, \Phi(x)= \exp(\lambda x)-1.$
     
    \item \textbf{$\alpha$-Strongly convex cost functions:} $\beta =1, V=\frac{8G^2 \ln(Te)}{\alpha}, \Phi(x)= x^2.$
    \end{enumerate}
     %$ \alpha=\frac{1}{2GD}, n=\max(2, \lceil \ln T \rceil), V=(n-1)^{n-1}T^{\frac{n-1}{2}}, \Phi(x)=x^n.$ 
%   \REPEAT
  \State {\bfseries Initialization:} Set $ x_1={\bf 0}, \text{CCV}(0)=0$.
   \State {\bf For} $t=1:T$
   %\For{$t=1:T$}
   \State \quad Play $x_t,$ observe $f_t, g_t,$ incur a cost of $f_t(x_t)$ and constraint violation of $(g_t(x_t))^+$
   \State \quad $\tilde{f}_t \gets \beta f_t, \tilde{g}_t \gets \beta \max(0,g_t).$
   \State \quad $\text{CCV}(t)=\text{CCV}(t-1)+\tilde{g}_t(x_t).$
   \State \quad Compute $\nabla_t = \nabla \hat{f}_t(x_t),$ where $\hat{f}_t(x):= V\tilde{f}_t(x)+ \Phi'(\text{CCV}(t)) \tilde{g}_t(x), ~~ t \geq 1$.
   \State \quad $x_{t+1} = \mathcal{P}_\mathcal{X}(x_t - \eta_t \nabla_t)$, where 
   \quad \begin{eqnarray*}
   \eta_t =\begin{cases}
   	\frac{\sqrt{2}D}{2\sqrt{\sum_{\tau=1}^{t} ||\nabla_\tau||_2^2}}, ~&~\textrm{for convex costs} \\
   	\frac{1}{\sum_{s=1}^t H_s}, ~ &~ \textrm{for strongly convex costs } (H_t \textrm{ is the strong convexity parameter of } f_t). 
   	\end{cases}
   	\end{eqnarray*}
   	
%   \IF{$x_i > x_{i+1}$}
%   \STATE Swap $x_i$ and $x_{i+1}$
%   \STATE $noChange = false$
%   \ENDIF
   %\EndFor
   \State {\bf EndFor}
%   \UNTIL{$noChange$ is $true$}
\end{algorithmic}
\end{algorithm}




\begin{lemma}\label{lem:algwc}
Even when $d=1$ and $f_t(x)=f(x)$ and $g_t(x) =g(x)$ for all $t$, for Algorithm \ref{coco_sinha}, its $\textrm{CCV}_{[1:T]}  = \Omega(\sqrt{T} \log T)$.
\end{lemma}
\begin{proof}
{\bf Input:} Consider $d=1$, and let $\cX=[1, a], a>2$. Moreover, let $f_t(x)=f(x)$ and $g_t(x) =g(x)$ for all $t$. Let $f(x) = c x^2$ for some (large) $c>0$ and $g(x)$ be such that $G=\{x: g(x)\le 0\} \subseteq [a/2, a]$ and let $|\nabla g(x)|\le1$ for all $x$.

Let $1< x_1 < a/2$. Note that $\text{CCV}(t)$ (defined in Algorithm \ref{coco_sinha}) is a non-decreasing function, and let $t^\star$ be the earliest time $t$ such that $\Phi'(\text{CCV}(t)) \nabla g(x) <- c $. 
For $f(x) = c x^2$, $\nabla f(x) \ge c$ for all $x>1$. 
Thus, using Algorithm \ref{coco_sinha}'s definition, it follows that for all $t\le t^\star$, $x_t < a/2$, since the derivative of $f$ dominates the derivative of $\Phi'(\text{CCV}(t))  g(x)$ until then.


Since  $\Phi(x)= \exp(\lambda x)-1$ with $\lambda = \frac{1}{2\sqrt{T}}$, and by definition $|\nabla g(x)| \le 1$ for all $x$, thus, we have that by time $t^\star$, 
$\textrm{CCV}_{[1:t^\star]} = \Omega(\sqrt{T} \log T)$. Therefore, $\textrm{CCV}_{[1:T]} =\Omega(\sqrt{T} \log T)$.

 %Moreover, Algorithm \ref{coco_sinha}'s actions $x_t$'s keeps oscillating around $a/2$ after  time $t^\star$ leading to $\Omega(\sqrt{T})$ regret.
 \end{proof}
 
Essentially, Algorithm \ref{coco_sinha} is treating minimizing the $\text{CCV}$ problem as regret minimization for function $g$ similar to function $f$ and this leads to its CCV of $\Omega(\sqrt{T}\log T)$.
For any given input instance with $d=1$, an alternate algorithm that  chooses its actions following online gradient descent (OGD) projected on to the most recently revealed feasible set $S_t$ achieves $O(\sqrt{T})$ regret (irrespective of the starting action $x_1$) and $O(D)$ $\text{CCV}$ (since any $x^\star \in S_t$ for all $t$).  We extend this intuition in the next section, and present an algorithm that tries to exploit the geometry of the nested convex sets $S_t$ for general $d$.






%\input{NCBC}
\section{New Algorithm for solving COCO}
In this section, we present a simple algorithm (Algorithm \ref{coco_alg_1}) for solving COCO.
\begin{algorithm}[tb]
   \caption{Online Algorithm for COCO}
   \label{coco_alg_1}
\begin{algorithmic}[1]
   \State {\bfseries Input:} Sequence of convex cost functions $\{f_t\}_{t=1}^T$ and constraint functions $\{g_t\}_{t=1}^T,$ $G=$ a common Lipschitz constant,  $d$ dimension  of the admissible set $\mathcal{X},$ step size $\eta_t = \frac{D}{G \sqrt{t}}$. 
   %an upper bound $G$ to the Euclidean norm of their (sub)gradients, 
    $D=$ Euclidean diameter of the admissible set $\mathcal{X},$ $\mathcal{P}_\mathcal{X}(\cdot)=$ Euclidean projection operator on the set $\mathcal{X}$,      \State {\bfseries Initialization:} Set $ x_1 \in \mathcal{X}$ arbitrarily, $\text{CCV}(0)=0$.
   \State {\bf For} \ {$t=1:T$}
   \State \quad Play $x_t,$ observe $f_t, g_t,$ incur a cost of $f_t(x_t)$ and constraint violation of $(g_t(x_t))^+$
   %\State Update constraint violation $\text{CCV}(t)=\text{CCV}(t-1)+\tilde{g}_t(x_t).$
   \State \quad Set $S_t$ as defined in \eqref{defn:S}
    \State \quad $y_{t} =  \mathcal{P}_{S_{t-1}}\left(x_t - \eta_t \nabla f_t(x_t)\right)$
   \State \quad $x_{t+1} =  \mathcal{P}_{S_t}\left(y_t\right)$
   \State {\bf EndFor}
\end{algorithmic}
\end{algorithm}
Algorithm \ref{coco_alg_1} is essentially an online projected gradient algorithm (OGD), 
which first takes an OGD step from the previous action $x_{t-1}$ with respect to the most recently revealed loss function $f_{t-1}$ with appropriate step-size which is then projected onto $S_{t-2}$ to reach $y_{t-1}$, and then projects $y_{t-1}$ onto  the most recently revealed set $S_{t-1}$ to get $x_t$,  the action to be played at time $t$.
\eqref{defn:S}. 

\begin{rem} Step 6 of Algorithm \ref{coco_alg_1} might appear unnecessary, however, its useful for proving Theorem \ref{thm:tvmonotone}.
\end{rem}

Since Algorithm \ref{coco_alg_1} is essentially an online projected gradient algorithm, similar to classical result on OGD, next, we show that the regret of Algorithm \ref{coco_alg_1} is $O(\sqrt{T})$.
\begin{lemma}\label{lem:regretbound}
The $\textrm{Regret}_{[1:T]}$ for Algorithm \ref{coco_alg_1} is $O(\sqrt{T})$.
\end{lemma}
Extension of Lemma \ref{lem:regretbound} when $f_t$'s are strongly convex which results in $\textrm{Regret}_{[1:T]}=O(\log{T})$ for Algorithm \ref{coco_alg_1} follows standard arguments \cite{Hazan} and is omitted.




The real challenge is to bound the total $\text{CCV}$ for Algorithm \ref{coco_alg_1}. 
Let $x_t$ be the action played by Algorithm \ref{coco_alg_1}. Then by definition, $x_t \in S_{t-1}$. Moreover, from \eqref{eq:distviolationrelation}, the constraint violation at time $t$, $\text{CCV}(t) \le G \text{dist}(x_{t}, S_t)$.
The next action $x_{t+1}$ chosen by Algorithm \ref{coco_alg_1} belongs to $S_t$, however, it is obtained by first taking an OGD step from $x_t$ to reach $y_t$ and then projects $y_t$ onto $S_t$. Since $f_t$'s are arbitrary, the OGD step could be towards any direction, and thus, there is no direct relationship between $x_{t+1}$ and $x_t$. Informally, $(x_1, x_2, \dots, x_T)$ is not a connected curve with any useful property. Thus, we take recourse in upper bounding the CCV via upper bounding the total movement cost $M$ (defined below) between nested convex sets using projections.

  The total constraint violation for Algorithm \ref{coco_alg_1} is
\begin{align}\nn
\text{CCV}_{[1:t]} & \le G\sum_{\tau=1}^t \text{dist}(x_{\tau}, S_{\tau}), \\ \label{defn:genconvxmovement}
&\stackrel{(a)} \le G  \sum_{\tau=1}^t ||x_{\tau}-  b_\tau||, \\
&\stackrel{(b)} = G M_t,
\end{align}
where in $(a)$ $b_t$ is the projection of $x_t$ onto $S_{t}$, i.e., $b_t=\cP_{S_{t}}(x_t)$ and in $(b)$
\begin{equation} \label{defn:totalmovementcost1}
M_t= \sum_{\tau=1}^t ||x_{\tau}-  b_\tau||
\end{equation} is defined to be the  total movement cost  on the instance $S_1, \dots, S_t$. 
%The upper bound in \eqref{defn:genconvxmovement} corresponds to the maximum length between any point on $S_{t-1}$ and its projection onto $S_t$. 
The object of interest is $M_T$.

%In the next section, we will upper bound $M_T$. Instead of bounding 
%Note that in \eqref{defn:genconvxmovement}, if we fix $a_t=x_t$ which in fact will give the correct CCV, then we will get upper bound on $M_T$ that will be algorithm dependent and not just instance dependent which can potentially be lower than that we are going to derive next that will be only instance dependent.
 %\begin{algorithm}[tb]
%   \caption{Policy $\mathrm{Switch}$ for COCO}
%   \label{coco_alg}
%\begin{algorithmic}[1]
%   \State {\bfseries Input:} Sequence of convex cost functions $\{f_t\}_{t=1}^T$ and constraint functions $\{g_t\}_{t=1}^T,$ $G=$ a common Lipschitz constant,  $d$ dimension  of the admissible set $\mathcal{X},$
%   %an upper bound $G$ to the Euclidean norm of their (sub)gradients, 
%    $D=$ Euclidean diameter of the admissible set $\mathcal{X},$ $\mathcal{P}_\mathcal{X}(\cdot)=$ Euclidean projection operator on the set $\mathcal{X}$, $z(d) = (D d \log d)$
%     %\State {\bfseries Parameter settings:} 
%     %\begin{enumerate}
%     	%\item \textbf{Convex cost functions:} $\beta = (2GD)^{-1}, V=1, \lambda = \frac{1}{2\sqrt{T}}, \Phi(x)= \exp(\lambda x)-1.$
%     
%    %\item \textbf{$\alpha$-strongly convex cost functions:} $\beta =1, V=\frac{8G^2 \ln(Te)}{\alpha}, \Phi(x)= x^2.$
%    %\end{enumerate}
%     %$ \alpha=\frac{1}{2GD}, n=\max(2, \lceil \ln T \rceil), V=(n-1)^{n-1}T^{\frac{n-1}{2}}, \Phi(x)=x^n.$ 
%%   \REPEAT
%  \State {\bfseries Initialization:} Set $ x_1 \in \mathcal{X}$ arbitrarily, $\text{CCV}(0)=0$.
%   \ForEach{$t=1:T$}
%   \State Play $x_t,$ observe $f_t, g_t,$ incur a cost of $f_t(x_t)$ and constraint violation of $\tilde{g}_t(x_t)=(g_t(x_t))^+$
%   \State Update constraint violation $\text{CCV}(t)=\text{CCV}(t-1)+\tilde{g}_t(x_t).$
%   \State Set $S_t$ as defined in \eqref{defn:S}
%   \If{$\text{CCV}(t) < z(d)$}
%   \State $\eta_t= \frac{1}{\sqrt{t}}$
%    \State $x_{t+1} =  \mathcal{P}_{\mathcal{X}}\left(x_t - \eta_t \frac{\nabla f_t(x_t)}{||\nabla f_t(x_t)||}\right)$
%    \Else
%    \If{$x_t\in S_t$}
%    \State $\eta_t= \frac{1}{\sqrt{t}}$
%    \State $x_{t+1} = \mathcal{P}_{S_t}\left(x_t - \eta_t \frac{\nabla f_t(x_t)}{||\nabla f_t(x_t)||}\right)$
%    \Else
%    \State  $x_{t+1} = \mathsf{Centroid}(S_t)$
%    \EndIf
%    \EndIf
%  
%   	
%%   \IF{$x_i > x_{i+1}$}
%%   \STATE Swap $x_i$ and $x_{i+1}$
%%   \STATE $noChange = false$
%%   \ENDIF
%   \EndForEach
%%   \UNTIL{$noChange$ is $true$}
%\end{algorithmic}
%\end{algorithm}


%using the $\mathsf{Centroid}$ algorithm described in Section \ref{sec:NCBC}. The pseudo code 
%of the algorithm is given in Algorithm \ref{coco_alg}, where the basic idea is as follows. Our target for $\text{CCV}_{[1:T]}$ is $z(d) = (Dd \log d)D$ that is independent of $T$. 
%Thus, Algorithm \ref{coco_alg} in phase 1 tries to optimize just the regret (with respect to $f_t$'s) by employing online gradient descent (OGD) algorithm while disregarding the constraint violation as long as the CCV is at most $z(d)$. If CCV never exceeds $z(d)$, we are done, since OGD achieves the optimal $O(\sqrt{T})$ regret following \cite{Hazan}. 
%
%Therefore, the real case of interest is that at some time $t< T$ (defined as $t_{\min}$), $\text{CCV}_{[1:t_{\min]}}$ is greater than $z(d)$. 
%From time $t_{\min}$ onwards, whenever, the action $x_t \notin S_t$,  the $\mathsf{Centroid}$ algorithm is used to select the next action. It is important to note that in the pseudo code of Algorithm \ref{coco_alg} $\mathsf{Centroid}(S_t)$ means the output of the $\mathsf{Centroid}$ algorithm which is not necessarily the centroid of the `full' set $S_t$.
%
%In the other case when $x_t \in S_t$, Algorithm \ref{coco_alg} tries to optimize just the regret (with respect to $f_t$'s) by employing OGD algorithm without considering constraint violation, similar to phase 1.
%
%
%As we show next, the regret of Algorithm \ref{coco_alg} is $O(\sqrt{T})$ while the CCV is $O(z(d))$.
%
%\begin{theorem}\label{thm:main}
%For algorithm \ref{coco_alg}, the regret \eqref{intro-regret-def} 
%	$$\textrm{Regret}_{[1:T]} = O(\sqrt{T}),$$
%	while the CCV \eqref{intro-gen-oco-goal}
% 	$$\textrm{CCV}_{[1:T]} = O(D d \log d).$$
%\end{theorem}
%To show this result, we essentially bound the regret \eqref{intro-regret-def}  by the sum of the total movement cost incurred by the $\mathsf{Centroid}$ algorithm and an $O(\sqrt{T})$ term, and then show that the total movement cost incurred by the $\mathsf{Centroid}$ algorithm is $O(D d \log d)$. Moreover, since the $\textrm{CCV}_{[1:T]}$ is also upper bounded by the total movement cost incurred by the $\mathsf{Centroid}$ algorithm, we get the result.
%%\begin{algorithm}[tb]
%   \caption{Online Policy for COCO}
%   \label{coco_alg}
%\begin{algorithmic}[1]
%   \State {\bfseries Input:} Sequence of convex cost functions $\{f_t\}_{t=1}^T$ and constraint functions $\{g_t\}_{t=1}^T,$ $G=$ a common Lipschitz constant,  $d$ dimension  of the admissible set $\mathcal{X},$
%   %an upper bound $G$ to the Euclidean norm of their (sub)gradients, 
%    $D=$ Euclidean diameter of the admissible set $\mathcal{X},$ $\mathcal{P}_\mathcal{X}(\cdot)=$ Euclidean projection operator on the set $\mathcal{X}$, $z(d) = (D d \log d)$
%     %\State {\bfseries Parameter settings:} 
%     %\begin{enumerate}
%     	%\item \textbf{Convex cost functions:} $\beta = (2GD)^{-1}, V=1, \lambda = \frac{1}{2\sqrt{T}}, \Phi(x)= \exp(\lambda x)-1.$
%     
%    %\item \textbf{$\alpha$-strongly convex cost functions:} $\beta =1, V=\frac{8G^2 \ln(Te)}{\alpha}, \Phi(x)= x^2.$
%    %\end{enumerate}
%     %$ \alpha=\frac{1}{2GD}, n=\max(2, \lceil \ln T \rceil), V=(n-1)^{n-1}T^{\frac{n-1}{2}}, \Phi(x)=x^n.$ 
%%   \REPEAT
%  \State {\bfseries Initialization:} Set $ x_1 \in \mathcal{X}$ arbitrarily, $\text{CCV}(0)=0$.
%   \ForEach{$t=1:T$}
%   \State Play $x_t,$ observe $f_t, g_t,$ incur a cost of $f_t(x_t)$ and constraint violation of $\tilde{g}_t(x_t)=(g_t(x_t))^+$
%   \State Update constraint violation $\text{CCV}(t)=\text{CCV}(t-1)+\tilde{g}_t(x_t).$
%   \State Set $S_t$ as defined in \eqref{defn:S}
%   \If{$\text{CCV}(t) < z(d)$}
%   \State $\eta_t= \frac{1}{\sqrt{t}}$
%    \State $x_{t+1} =  \mathcal{P}_{\mathcal{X}}\left(x_t - \eta_t \frac{\nabla f_t(x_t)}{||\nabla f_t(x_t)||}\right)$
%    \Else
%    \If{$x_t\in S_t$}
%    \State $\eta_t= \frac{1}{\sqrt{t}}$
%    \State $x_{t+1} = \mathcal{P}_{S_t}\left(x_t - \eta_t \frac{\nabla f_t(x_t)}{||\nabla f_t(x_t)||}\right)$
%    \Else
%    \State  $x_{t+1} = \mathsf{Centroid}(S_t)$
%    \EndIf
%    \EndIf
%  
%   	
%%   \IF{$x_i > x_{i+1}$}
%%   \STATE Swap $x_i$ and $x_{i+1}$
%%   \STATE $noChange = false$
%%   \ENDIF
%   \EndForEach
%%   \UNTIL{$noChange$ is $true$}
%\end{algorithmic}
%\end{algorithm}


\section{Bounding the Total Movement Cost $M_T$ \eqref{defn:totalmovementcost1}}



%The movement cost $M$ is a general distance measure that counts the maximum total distance 
%that can be traversed when projections are taken from arbitrary points between nested convex sets, and is typically expected to be much larger than the actual violation $\sum_{\tau=1}^t \text{dist}(x_{\tau}, S_{\tau})$ that Algorithm \ref{coco_alg_1} will incur. 
%In Theorem \ref{thm:tvmonotone}, we saw that if the projections satisfy the monotonicity property, then the total CCV can be bounded independent of $T$ only in terms of $d$ and $D$. In general, however, Algorithm \ref{coco_alg_1} can have arbitrary projections. Thus, we upper bound $M$ itself, which when multiplied by $G$ is an upper bound on the CCV of Algorithm \ref{coco_alg_1}.

We start by considering two simple cases where bounding $M_T$ is easy.
 \begin{lemma}\label{lem:spheres}
If all nested convex bodies $S_1\supseteq S_2  \supseteq \dots \supseteq S_T$ are spheres then $M_T\le d^{3/2}D$.
\end{lemma}

\begin{proof}
Recall the definition that $x_t\in \partial S_{t-1}, b_t=\cP_{S_{t}}(x_t)\in S_t$ from \eqref{defn:genconvxmovement}.
Let $||x_t-b_t||=r$, then since all $S_t$'s are spheres, at least along one of the $d$-orthogonal canonical basis vectors, $\text{diameter}(S_{t})\le \text{diameter}(S_{t-1}) - \frac{r}{\sqrt{d}}$. Since the diameter along any of the $d$-axis is $D$, we get the answer.
\end{proof}

%Similar results can be shown for nice convex bodies, such as parallel cuboids. 
 \begin{lemma}\label{lem:square}
If all nested convex bodies $S_1\supseteq S_2  \supseteq \dots \supseteq S_T$ are cuboids that are axis parallel to each other, then $M\le d^{3/2}D$.
\end{lemma}
Proof is identical to Lemma \ref{lem:spheres}.
Note that similar results can be obtained when $S_t$'s are regular polygons that are axis parallel with each other.


 
 After exhausting the universal results for an upper bound on $M_T$ for `nice' nested convex bodies, we next give a general bound on $M_T$ for any sequence of nested convex bodies which depends on the geometry of the nested convex bodies (instance dependent).
 To state the result we need the following preliminaries.
 
 Following \eqref{defn:genconvxmovement}, $b_t=\cP_{S_t}(x_t)$ where $x_t\in \partial S_{t-1}$. Without loss of generality, 
 $x_t\notin S_{t}$ since otherwise the distance $||x_t-b_t||=0$.
 Let $m_t$ be the mid-point of $x_t$ and $b_t$, i.e. $m_t = \frac{x_t+b_t}{2}$.
 \begin{definition}\label{defn:anglewidth}
Let the convex hull of $m_t \cup S_{t}$ be $\cC_t$.
 Let $w_t$ be a unit vector such that there exists $c_t>0$ such that the cone 
 $$C_{w_t}(c_t) = \left\{z\in \bbR^d: -w_t^T\frac{(z-m_t)}{||(z-m_t)||} \ge c_t\right\}$$ contains $\cC_t$. Since $S_{t}$ is convex, such $w_t, c_t>0$ exist. For example, $w_t=b_t-x_t$ is one such choice for which $c_t>0$ since $m_t \notin S_t$.
See Fig. \ref{fig:anglewidthmain} for a pictorial representation.
 
 Let $c^\star_{w_t,t} = \arg \min_{c_t} C_{w_t}(c_t)$,
 $c^\star_t = \min_{w_t} c^\star_{w_t,t}$, and $w_t^\star= \arg \min_{w_t} c^\star_{w_t,t}$.
 Moreover, let $c^\star = \min_t  c^\star_t$, where by definition, $c^\star <1$. 
 \end{definition}
 
% \begin{figure}[]
%  \begin{center}
%%\begin{tikzpicture}
%\begin{tikzpicture}[scale=1.5, dot/.style={circle,inner sep=1pt,fill,label={#1},name=#1},
%  extended line/.style={shorten >=-#1,shorten <=-#1},
%  extended line/.default=1cm]
%% Define coordinates
%\coordinate (w_t) at (0,0);
%\coordinate (z_t) at (0,-.5);
%\coordinate (x_t) at (-1,-1);
%\coordinate (y_t) at (1,0);
%\coordinate (C_top) at (1,2.15);
%\coordinate (C_bottom) at (3.55,-0.55);
%
%
%% Draw object (shaded region)
%%\draw[thick, gray, fill=pink!50] (C_top) to[out=-.5,in=-15] (C_bottom) -- cycle;
%
%
%
%
%
%% Draw projected image (S_t+1)
%\draw[thick, cyan, fill=cyan!50,opacity=0.75] (1,0) -- (1.5,1.5) -- (3,1) -- (3.2,0.15)  -- cycle;
%\draw[thick, blue!10, fill=blue!10,opacity=0.5] (z_t) --(1.5,1.495) --  (y_t)   -- cycle;
%\draw[thick, blue!10, fill=blue!10,opacity=0.5] (z_t) --  (y_t)  -- (3.2,0.15) -- cycle;
%
%\filldraw[blue] (y_t) circle (2pt) node[below right] {$b_t$};
%\filldraw[black] (z_t) circle (2pt) node[below right] {$m_t$};
%% Draw camera and viewing ray
%%\begin{scope}[on background layer]
%\draw[thick,->] (x_t) -- (y_t);
%%\end{scope}
%\draw[dashed,->] (.65,0) -- (z_t) -- (-0.65,-1)  node[below right] {$w_t$};
%\draw[dashed,->] (.65,0) -- (z_t) -- (-0.65,-1)  node[below right] {$w_t$};;
%%\draw[dashed,->] (.65,0) -- (x_t) -- (-0.65,-1)  node[below right] {$w_t$};;
%
%\coordinate (u_1) at (0,.5);
%\coordinate (u_2) at (0,-2);
%
%%\draw [ dashed, <-] (u_1) -- (z_t) -- (u_2) node[above left] {$H_u$};
%%\draw [ dashed, ->] (1.3,-.7) -- (z_t) -- (-2.8,.2) node[above left] {$u$};
%
%%\draw [extended line, ->] ($(u_1)!(z_t)!(u_2)$) -- (-.5,-0.05) node[above left ] {$u$};
%
%%\draw [extended line,<-] ($(-0.25,.5)!(z_t)!(.25,-2)$) node[above left] {$u$};
%
%%\tkzDefLine[orthogonal=through (z_t)](x_t,y_t);
%
%
%%\filldraw[black] (w_t) circle (2pt) node[above left] {$w_t$};
%\filldraw[black] (x_t) circle (2pt) node[below left] {$a_t$};
%
%%\begin{scope}[on background layer]
%\draw[thick, gray, fill=gray!50,opacity=0.76] (C_top) to[out=.5,in=25] (C_bottom) -- cycle;
%%\end{scope}
%\draw[thick, gray, fill=gray!50,opacity=0.76] (C_bottom) to[out=-155,in=-165] (C_top) -- cycle;
%\node at (2.25,0.75) {$S_{t}$};
%
%% Draw lines connecting camera to object and projected image
%\draw[dotted] (z_t) -- (1.5,1.5);
%\draw[dotted] (z_t) -- (3.2,0.15);
%%\draw[dotted] (w_t) -- (3,1);
%%\draw[dotted] (w_t) -- (2.5,0.5);
%%\draw[dotted] (w_t) -- (1.5,0.5);
%%\begin{scope}[on background layer]
%\draw[gray!30,  thick,fill=gray!50,opacity=0.6] (z_t) -- (C_top) -- (C_bottom) -- (z_t);
%%\end{scope}
%%\draw[dotted] (z_t) -- (C_bottom);
%
%% Add labels for object and camera
%\node[above right] at (C_top) {$C_{w_t}(c_t)$};
%%\begin{axis}[
%%  axis lines=center,
%%  axis on top,
%%  %xlabel={$x$}, ylabel={$y$}, zlabel={$t$},
%%  domain=1:1,
%%  y domain=0:2*pi,
%%  xmin=-1.5, xmax=1.5,
%%  ymin=-1.5, ymax=1.5, zmin=0.0,
%%        %every axis x label/.style={at={(rel axis cs:0,0.5,0)},anchor=south},
%%        %every axis y label/.style={at={(rel axis cs:0.5,0,0)},anchor=north},
%%        %every axis z label/.style={at={(rel axis cs:0.5,0.5,0.9)},anchor=west},
%%  samples=30]
%%  \addplot3 [surf, colormap/blackwhite, shader=flat] ({x*cos(deg(y))},{x*sin(deg(y))},{x});
%%    
%%    %\addplot3[surf, samples=20, domain=-1:1, y domain=-1:1] {x^2 + y^2}; % Example surface plot
%%
%%\end{axis}
%\end{tikzpicture}
%\caption{Figure representing the cone $C_{w_t}(c_t)$ that contains the convex hull of $m_t$ and $S_{t}$ with unit vector $w_t$.} 
%%$u$ is a unit vector perpendicular to $H_u$ an hyperplane that is a supporting hyperplane $C_t$ at $m_t$ such that $\cC_t \cap H_u = \{z_t\}$ and 
%%$u^T (a_t-m_t)\ge 0$ }
%\label{fig:anglewidthmain}
%\end{center}
%\end{figure}

\begin{figure*}
\begin{center}
\includegraphics[width=10cm,keepaspectratio,angle=0]{Fig-ConeDef.png}
\caption{Figure representing the cone $C_{w_t}(c_t)$ that contains the convex hull of $m_t$ and $S_{t}$ with unit vector $w_t$.} 
\label{fig:anglewidthmain}
\end{center}
\end{figure*}


 
Essentially, $2\cos^{-1}(c^\star_t)$ is the angle width of $\cC_t$ with respect to $w_t^\star$, i.e. each element of $\cC_t$ makes an  angle of at most $ \cos^{-1}(c^\star_t)$ with $w_t^\star$.  
 
 

%\begin{rem} Instead of 
%\end{rem}
 
% \begin{rem} Theorem \ref{thm:manselli} obtains a universal result on $M$ by exploiting the self-expanded property that implies that $c^\star_t \ge \frac{1}{\sqrt{d}}$ (VI \cite{Manselli}) independent of the 
% shape of the convex bodies. Even when $x_t$ and $x_{t+1}$ are arbitrary points from $S_{t-1}$ and $S_{t}$ as is the considered case, it is still true that each sub-curve $(x_t, y_t)$ has the self-expanded property with respect to all the sub-curves $(x_\tau, y_\tau), \ \tau\ge t$ but not the whole curve $x_1,y_1, x_2,y_2, \dots, x_t,y_t,x_{t+1},y_{t+1}, \dots$.
% \end{rem}

\begin{rem}\label{rem:cbound}
Note that $c_t^\star$ is only a function of the distance $ ||x_t-b_t||$   and the shape of $S_t$'s, in particular, the maximum width of $S_t$ 
 along the directions perpendicular to vector $x_t-b_t$ $\forall \ t$ which can be at most the diameter $D$. 
 $c_t^\star$ decreases (increasing the ``width" of cone $C_{w_t^\star}(c_t^\star)$) as $||x_t-b_t||$ decreases, but small $x_t-b_t$ also implies small  violation at time $t$ from \eqref{defn:genconvxmovement}.
Across time slots, $d_{\min} = \min_t ||x_t-b_t||$ and shape of $S_t$'s control $c^\star$, where $d_{\min} > 0$ is inherent from the definition of $c^\star$ since a bound on $||x_t-b_t||$ is only needed for the case when $x_t\ne b_t$. 
\end{rem} 
 %Interestingly, if $d_{\min}$ is small then $c^\star$ but then $\text{CCV}_{[1:t]}$ in \eqref{defn:genconvx

\begin{rem}  Projecting $x_t\in \partial S_{t-1}$ onto $S_t$ to get $b_t=\cP_{S_t}(x_t)$, the diameter of $S_t$ is at most diameter of $S_{t-1} - ||x_t-b_t||$, however, only along the direction $b_t-x_t$. Since the shape of $S_{t}$ is arbitrary, as a result, the diameter of $S_t$ need not be smaller than the diameter of $S_{t-1}$ along any pre-specified direction, which was the main idea used to derive Lemma \ref{lem:spheres}.  Thus, to relate the distance $||x_t-b_t||$ with the decrease in the diameter of the convex bodies $S_t$'s, we use the concept of {\bf mean width} of a convex body that is defined as the expected width of the convex body along all the directions that are chosen uniformly randomly (formal definition is provided in Definition \ref{defn:avgwidth}).
\end{rem}

Next, we upper bound $M_T$ by connecting the distance $||x_t-b_t||$ to the decrease in mean width (to be defined ) of convex bodies $S_{t-1}$ and $S_t$'s.
 
 \begin{lemma}\label{lem:movementcost}
 The total movement cost 
$M_T$ in \eqref{defn:totalmovementcost1} is at most $$\frac{2V_d(d-1)}{V_{d-1}} \left(\frac{1}{c^\star}\right)^{d}D,$$ where 
$V_d$ is the $(d-1)$-dimensional Lebesgue measure of  the unit sphere in $d$ dimensions. 
\end{lemma}
%By definition, $c^\star <1$. Thus as $c^\star$ decreases, $M_T$ increases exponentially with exponent being $d$. 

Note that $V_d/V_{d-1} = O(1/\sqrt{d})$.
 Thus, we get the following {\bf main result} of the paper for Algorithm \ref{coco_alg_1} combining Lemma \ref{lem:regretbound} and Lemma \ref{lem:movementcost}. 
 \begin{theorem}\label{thm:main1}
For solving COCO, Algorithm \ref{coco_alg_1} has  $$\textrm{Regret}_{[1:T]} = O(\sqrt{T}), \ \text{and} \ \text{CCV}_{[1:T]}= O\left(\sqrt{d} \left(\frac{1}{c^\star}\right)^{d}D\right).$$
\end{theorem}

%\begin{rem} Specializing $a_t=x_t$ in \eqref{defn:genconvxmovement}, where $x_t$ is the action chosen by Algorithm \ref{coco_alg_1}, we will get a better bound on the $\text{CCV}_{[1:T]}$ via improved bound for $c^\star$, however, that will be both algorithm and instance dependent.
%\end{rem}
Compared to all prior results on COCO, that were universal (instance independent), where the best known one \cite{Sinha2024} has $\textrm{Regret}_{[1:T]} = O(\sqrt{T})$, and $\text{CCV}_{[1:T]}=O(\sqrt{T}\log T)$, Theorem \ref{thm:main1} is an instance 
dependent result for the CCV. In particular, it exploits the geometric structure of the nested convex sets $S_t$'s and 
derives an upper bound on the CCV that only depends on the `shape' of $S_t$'s. It can be the case that the instance is `badly' behaved and $c^\star$ is very small or dependent on $T$. If that is the case, in Section \ref{sec:algswitch} we show how to limit the CCV to $O(\sqrt{T}\log T)$. However, when $S_t$'s are `nice', e.g., $c^\star$ is independent of $T$ (Remark \ref{rem:cbound}) or $S_t$'s are spheres or axis parallel cuboids (Lemma \ref{lem:spheres} and \ref{lem:square}), the $\text{CCV}$ of Algorithm \ref{coco_alg_1} is independent of $T$, which is a fundamentally improved result compared to large body of prior work. In fact, in prior work this was largely assumed to be not possible. In particular, before the result of \cite{Sinha2024}, achieving simultaneous $\textrm{Regret}_{[1:T]} = O(\sqrt{T})$, and $\text{CCV}_{[1:T]}=O(\sqrt{T})$ itself was the final goal.




\section{Algorithm $\mathrm{Switch}$}\label{sec:algswitch}

Since Theorem \ref{thm:main1} provides an instance dependent bound on the CCV, that is a function of $c^\star$ which can be small, it can be the case that its CCV is larger than $O(\sqrt{T}\log T)$, thus providing a result that is inferior to that of Algorithm \ref{coco_sinha} \cite{Sinha2024}. Thus, next, we marry the two algorithms, Algorithm \ref{coco_sinha} and Algorithm \ref{coco_alg_1}, in Algorithm \ref{alg:switch} to provide a best of both results as follows. 


\begin{algorithm}[tb]
   \caption{$\mathrm{Switch}$}
   \label{alg:switch}
\begin{algorithmic}[1]
   \State {\bfseries Input:} Sequence of convex cost functions $\{f_t\}_{t=1}^T$ and constraint functions $\{g_t\}_{t=1}^T,$ $G=$ a common Lipschitz constant,  $d$ dimension  of the admissible set $\mathcal{X},$
   %an upper bound $G$ to the Euclidean norm of their (sub)gradients, 
    $D=$ Euclidean diameter of the admissible set $\mathcal{X},$ $\mathcal{P}_\mathcal{X}(\cdot)=$ Euclidean projection operator on the set $\mathcal{X}$,      \State {\bfseries Initialization:} Set $ x_1 \in \mathcal{X}$ arbitrarily, $\text{CCV}(0)=0$.
   \State {\bf For} \ {$t=1:T$}
   \State \quad {\bf If} {$\text{CCV}(t-1) \le \sqrt{T}\log T$}
   \State \quad \quad Follow Algorithm \ref{coco_alg_1}
   \State  \quad  \quad $\text{CCV}(t)=\text{CCV}(t-1)+\max\{g_t(x_t),0\}.$
   \State \quad {\bf Else} 
   \State \quad \quad Follow Algorithm \ref{coco_sinha} with resetting $\text{CCV}(t-1)=0$
   \State \quad {\bf EndIf} 
   \State {\bf EndFor}
\end{algorithmic}
\end{algorithm}

 \begin{theorem}\label{thm:main2}
$\mathrm{Switch}$ (Algorithm \ref{alg:switch}) has regret $\textrm{Regret}_{[1:T]} =O(\sqrt{T})$, while    $$\text{CCV}_{[1:T]}=
\min\left\{O\left(\sqrt{d} \left(\frac{1}{c^\star}\right)^{d}D\right), O(\sqrt{T}\log T)\right\}.$$
\end{theorem}
 
Algorithm $\mathrm{Switch}$ should be understood as the best of two worlds algorithm, where the two worlds corresponds to one having nice convex sets $S_t$'s that have CCV independent of $T$ or $o(\sqrt{T})$ for 
Algorithm \ref{coco_alg_1}, while in the other, CCV of Algorithm \ref{coco_alg_1} is large on its own, and the overall CCV is controlled by discontinuing the use of Algorithm \ref{coco_alg_1} once its CCV reaches $\sqrt{T}\log T$ and switching to Algorithm \ref{coco_sinha} thereafter that has universal guarantee of $O(\sqrt{T}\log T)$ on its CCV.
 
 After exhausting the general results on the CCV of Algorithm \ref{coco_alg_1}, we next consider the special case of $d=2$ and when the sets $S_t$ have a special structure defined by their projection hyperplanes. 
 Note that it is highly non-trivial to bound the CCV of Algorithm \ref{coco_alg_1} even when $d=2$.
 
  
 %\begin{remark}
 %\end{remark}

\section{Special case of $d=2$}
In this section, we show that if $d=2$ (all convex sets $S_t$'s lie in a plane) and the projections satisfy a monotonicity property depending on the problem instance, then we can bound the total CCV for Algorithm \ref{coco_alg_1} independent of the time horizon $T$ and consequently getting a $O(1)$ CCV. 




Recall from the definition of Algorithm \ref{coco_alg_1}, $y_t = \cP_{S_{t-1}}(x_t - \eta_t \nabla f_t(x_t))$ and 
$x_{t+1} = \cP_{S_t}(y_t)$.


\begin{definition}\label{defn:projhyperplane}
Let the hyperplane perpendicular to line segment $(y_t, x_{t+1})$ passing through $x_{t+1}$ be 
$F_t$. Without loss of generality, we let $y_t \notin S_t$, since then the projection is trivial. Essentially $F_t$ is the projection hyperplane at time $t$. %Let $N_t$ be the normal to $F_t$, then we
Let $\cH_t^+$ denote the positive half plane corresponding to $F_t$, i.e., 
$\cH_t^+ = \{z: z^T (y_t-x_{t+1})\ge 0\}$. 
Refer to Fig. \ref{fig:defF}.
Let the angle between $F_1$ and $F_t$ be $\theta_t$. 
\end{definition}
\begin{figure}


\includegraphics[width=10cm,keepaspectratio,angle=0]{FigDefF.pdf}


\caption{Definition of $F_t$'s.}
\label{fig:defF}
\end{figure}


\begin{definition}\label{defn:anglemonotone}
The instance $S_1 \supseteq S_2 \supseteq \dots \supseteq S_T$ is defined to be monotonic 
if $\theta_2 \le \theta_3 \le \dots \le \theta_T$.
\end{definition}




\begin{theorem}\label{thm:tvmonotone}
For $d=2$ when the instance is monotonic, $\text{CCV}_{[1:T]}$ for Algorithm \ref{coco_alg_1} is at most $O(GD)$.
\end{theorem}

Theorem \ref{thm:tvmonotone} provides a universal guarantee on the CCV of  Algorithm \ref{coco_alg_1} that is independent of the problem instance (as long as it is monotonic) unlike Lemma \ref{lem:movementcost}, even though it applies only for $d=2$. The proof is derived by using basic convex geometry results from \cite{Manselli} in combination with exploiting the definition of Algorithm \ref{coco_alg_1} and the monotonicity condition. It is worth noting that even under the monotonicity assumption it is non-trivial to upper bound the CCV since the successive angles made by $F_t$ with $F_1$ can increase arbitrarily slowly, making it difficult 
to control the total CCV. 
%It shows that the COCO problem when has structure can be exploited to get $O(1)$ CCV 
\section{OCS Problem}
In \cite{Sinha2024}, a special case of COCO, called the OCS problem, was introduced where $f_t\equiv0$ for all $t$. Essentially, with OCS, only constraint satisfaction is the objective.  In \cite{Sinha2024}, Algorithm \ref{coco_sinha} was shown to have CCV of $O(\sqrt{T}\log T)$. Next, we show that Algorithm \ref{coco_alg_1} has CCV of $O(1)$ for the OCS, a remarkable improvement.

 \begin{theorem}\label{thm:ocs}
For solving OCS, Algorithm \ref{coco_alg_1} has $\text{CCV}_{[1:T]}= O\left(d^{d/2} D\right)$.
\end{theorem}

As discussed in \cite{Sinha2024}, there are important applications of OCS, and it is important to find tight bounds on its CCV. Theorem \ref{thm:ocs} achieves this by showing that CCV of $O(1)$ can be achieved, where the constant depends only on the dimension of the action space and the diameter. This is a fundamental improvement compared to the CCV bound of $O(\sqrt{T}\log T)$ from \cite{Sinha2024}. Theorem  \ref{thm:ocs} is derived by using the connection between the curve obtained by successive projections on nested convex sets and self-expanded curves (Definition \ref{defn:se-curve}) and then using a classical result on self-expanded curves from \cite{Manselli}.

 
\section{Conclusions \pglen{0.25}}
\label{sec:conclude}

We present \sys, a holistic system for serving LLM inference requests with a wide range of SLAs, which maintains better GPU utilization, reduces resource fragmentation that occurs in silos, and increases utility by donating surplus instances to Spot instances. 
\sys achieves this through its unique elements, namely, a holistic deployment stack for requests of varying SLAs, its async feed module, and long-term aware proactive scaler logics that capitalize on the underutilized instances of another model in the same region by inter-model redeployment.

Future work includes extending \sys to accomodate workloads with a continuum of SLAs and conducting extensive studies on the benefits of the proposed approach with deployments across heterogeneous hardware types. We plan to open-source our trace data and simulator.


% \input{sections/new_data}

% conference papers do not normally have an appendix
% The Computer Society usually uses the plural form
% \section*{Acknowledgments}
% \ysnote{Thank all your colleagues who helped with the paper. It is good form.}



%\bibliographystyle{IEEEtran}
%\bibliography{../IEEEabrv,../Research}
\bibliography{OCO.bib} 
\newpage
\section{Proof of Lemma \ref{lem:regretbound}}
\begin{proof}
From the convexity of $f_t$'s, for $x^\star$ satisfying Assumption \eqref{feas-constr}, we have 
$$f_t(x_t) - f_t(x^\star) \le \nabla f_t^T (x_t-x^\star).$$
From the choice of Algorithm \ref{coco_alg_1} for $x_{t+1}$, we have 
\begin{align*} ||x_{t+1}-x^\star||^2 & = || \cP_{S_{t}}(y_t) - x^\star||^2 \\
& \stackrel{(a)}\le || y_t - x^\star||^2, \\
& = ||\cP_{S_{t-1}}\left(x_t - \eta_t \nabla f_t(x_t)\right)-x^\star||^2, \\
& \stackrel{(n)}\le || (x_t-\eta_t\nabla f_t^T(x_t)) - x^\star||^2,
\end{align*}
where inequalities $(a)$ and $(b)$ follow since $x^\star \in S_t$ for all $t$.
Hence
\begin{align*}
 ||x_{t+1}-x^\star||^2 & \le  ||x_t-x^\star||^2 + \eta_t^2||\nabla f_t(x_t)||^2 - 2\eta_t \nabla f_t^T(x_t)(x_t-x^\star), \\
 \nabla f_t^T(x_t)(x_t-x^\star) & \le \frac{||x_t-x^\star||^2-||x_{t+1}-x^\star||^2 }{\eta_t} + \eta_t G^2.
\end{align*}
Summing this over $t=1$ to $T$, we get 
\begin{align*}
2\sum_{t=1}^T (f_t(x_t) - f_t(x^\star)) & \le \sum_{t=1}^T\nabla f_t^T (x_t-x^\star), \\
& \le \sum_{t=1}^T  \frac{||x_t-x^\star||^2-||x_{t+1}-x^\star||^2 }{\eta_t} + \sum_{t=1}^T\eta_t G^2, \\
& \le D^2 \frac{1}{\eta_T} + G^2 \sum_{t=1}^T\eta_t,\\
& \le O( DG \sqrt{T}),
\end{align*}
where the final inequality follows by choosing $\eta_t = \frac{D}{G\sqrt{t}}$.
\end{proof}
\input{App-ProofAvgWidth}
\section{Proof of Theorem \ref{thm:main2}}
 \begin{proof}
 Since $\text{CCV}(t)$ is a monotone non-decreasing function, let $t_{\min}$ be the largest time until which Algorithm \ref{coco_alg_1} is followed by $\mathrm{Switch}$.
 The regret guarantee is easy to prove.  From Theorem \ref{thm:main1},
 regret until time $t_{\min}$ is at most $O(\sqrt{t_{\min}})$. Moreover, starting from time $t_{\min}$ till $T$, from Theorem \ref{thm:sinha2024}, the regret of Algorithm \ref{coco_sinha} is at most $O(\sqrt{T-t_{\min}})$. Thus, the overall regret for $\mathrm{Switch}$ is at most $O(\sqrt{T})$.
 
 For the CCV, with $\mathrm{Switch}$, until time $t_{\min}$, $\text{CCV}(t_{\min})\le \sqrt{T}\log T$. At 
 time $t_{\min}$, $\mathrm{Switch}$ starts to use Algorithm \ref{coco_sinha} which has the following appealing property from (8) \cite{Sinha2024} that for any $t\ge t_{\min}$ where at time $t_{\min}$  Algorithm \ref{coco_sinha} was started to be used with resetting $\text{CCV}(t_{\min})=0$. 
 For any $t\ge t_{\min}$
 \begin{eqnarray} \label{gen-fn-ineq}
		\Phi(\text{CCV}(t)) +\textrm{Regret}_t(x^\star) \leq \sqrt{\sum_{\tau=t_{\min}}^t \big(\Phi'(\text{CCV}(\tau))\big)^2} + \sqrt{t-t_{\min}}.
\end{eqnarray}
where $\beta = (2GD)^{-1}, V=1, \lambda = \frac{1}{2\sqrt{T}}, \Phi(x)= \exp(\lambda x)-1, $ and $\lambda=\frac{1}{2\sqrt{T}}$.
We trivially have $\textrm{Regret}_t(x^\star)\geq -\frac{Dt}{2D} \geq -\frac{t}{2}.$ Hence, from \eqref{gen-fn-ineq}, we have that for any $\lambda = \frac{1}{2\sqrt{T}}$ and any $t \ge t_{\min}$
$$\text{CCV}_{[t_{\min},T]} \leq 4GD\ln(2\big(1+2T)\big)\sqrt{T}.$$
Since as argued before, with $\mathrm{Switch}$,  $\text{CCV}(t_{\min})\le \sqrt{T}\log T$, we get that  $\text{CCV}_{[1:T]}\le O(\sqrt{T}\log T)$.
 \end{proof}
%\section{Proof of Theorem \ref{thm:tvmonotone}}
\section{Preliminaries for Bounding the CCV in Theorem \ref{thm:ocs} and Theorem \ref{thm:tvmonotone}}
% 
% \begin{definition}\label{defn:avgwidth}
%Let $K$ be a non-empty convex bounded set in $\bbR^d$. Let $u$ be a unit vector, and $\ell_u$ a line through the origin parallel to $u$. 
%Let $K_u$ be the orthogonal projection of $K$ onto $\ell_u$, with length $|K_u|$. The mean width of $K$ is defined as 
%\begin{equation}\label{eq:projlength}
%W(K) = \frac{1}{V_d} \int_{S_1^d} |K_u| du,
%\end{equation}
%where $S_1^d$ is the unit sphere in $d$ dimensions and $V_d$ its $(d-1)$-dimensional Lebesgue measure.
%\end{definition}
%
%
%The following is immediate. 
%\begin{equation}\label{eq:WBound1}
%0\le W(K) \le \text{diameter}(K).
%\end{equation}
%
%
%\begin{lemma}\label{lem:width2D1}
%For $d=2$, $$W(K)=\frac{\text{Perimeter}(K)}{\pi}.$$
%\end{lemma}

Let $K_1, \dots, K_T$ be nested (i.e., $K_1 \supseteq K_2 \supset K_3 \supseteq \dots \supseteq K_T$) bounded convex subsets of $\bbR^d$. 

%Let the minimum distance between $K_i$ and $K_{i+1}$ be $d_{i,i+1}$ and 
%\begin{equation}
%d_{\min} = \min_i d_{i,i+1}.
%\end{equation}
\begin{definition}\label{defn:projectioncurve}
If $\sigma_1\in K_1$, and $\sigma_{t+1} = \cP_{K_{t+1}}(\sigma_t)$, for $t=1, \dots, T$. Then the curve 
$${\underline \sigma}= \{(\sigma_1,\sigma_2), (\sigma_2,\sigma_3), \dots, (\sigma_{T-1},\sigma_T)\}$$ is called the projection curve on $K_1, \dots, K_T$.
\end{definition}
%Let $x_t \in \partial K_t$ and $y_t$ be the projection of $x_t$ on set $K_{t+1}$.  


We are interested in upper bounding the quantity 
\begin{equation}\label{eq:totalDistance}
\Sigma = \max_{{\underline \sigma}} \sum_{t=1}^{T-1} ||\sigma_t - \sigma_{t+1}||.
\end{equation}
%If $x_t\in K_{t+1}$ then $||x_t - y_t||=0$.

 \begin{lemma}\label{lem:projection}
For a projection curve ${\underline \sigma}$, $\Sigma \le d^{d/2} \text{diameter}(K_1)$.
\end{lemma}


To prove the result we need the following definition.

\begin{definition}\label{defn:se-curve} A curve $\gamma: I \rightarrow \bbR^d$  is called self-expanded, if for every $t$ where 
$\gamma'(t)$ exists, we have 
$$< \gamma'(t), \gamma(t)-\gamma(u)> \ \ge 0$$ for all $u\in I$ with $u \le t$, where $<.,.>$ represents the inner product. 
In words, what this means is that $\gamma$ starting in a point $x_0$ is self expanded, if for every $x\in \gamma$ for which there exists the tangent line $\sfT$, the arc (sub-curve) $(x_0, x)$ is
contained in one of the two half-spaces, bounded by the hyperplane through
$x$ and orthogonal to $\sfT$. 
\end{definition}
For self-expanded curves the following classical result is known.
\begin{theorem}\label{thm:manselli}\cite{Manselli}
For any self-expanded curve $\gamma$ belonging to a closed bounded convex set of $\bbR^d$ with diameter $D$, its total length is at most $O(d^{d/2} D)$.
\end{theorem}
\begin{proof}[Proof of Lemma \ref{lem:projection}]
From Definition \ref{defn:projectioncurve}, the projection curve is 
$${\underline \sigma}=\{(\sigma_1,\sigma_2), (\sigma_2,\sigma_3), \dots, (\sigma_{T-1},\sigma_T)\}.$$ Let the reverse curve be ${\underline r} = \{r_t\}_{t=0, \dots, T-2}$, where $r_t = (\sigma_{T-t}, \sigma_{T-t-1})$. Thus we are reading ${\underline \sigma}$ backwards and calling it ${\underline r}$. Note that since $\sigma_{t}$ is the projection of $\sigma_{t-1}$ on $K_t$, each piece-wise linear segment $(\sigma_t, \sigma_{t+1})$ is a straight line and hence differentiable except at the end points. Moreover, since each $\sigma_t$ is obtained by projecting $\sigma_{t-1}$ onto $K_t$ and $K_{t+1}\subseteq K_t$, we have that the projection hyperplane 
$F_t$ that passes through $\sigma_t=\cP_{K_t}(\sigma_{t-1})$ and is perpendicular to $\sigma_t - \sigma_{t-1}$ separates the two sub curves $\{(\sigma_1,\sigma_2), (\sigma_2,\sigma_3), \dots, (\sigma_{t-1},\sigma_t)\}$ and $\{(\sigma_t,\sigma_{t+1}), (\sigma_{t+1},\sigma_{t+2}), \dots, (\sigma_{T-1},\sigma_T)\}$.


Thus, we have that 
for each segment $r_\tau$, at each point where it is differentiable, the curve $r_1, \dots r_{\tau-1}$ lies on one side of the hyperplane that passes through the point and is perpendicular to $ r_{\tau+1}$. Thus, we conclude that curve ${\underline r}$ is self-expanded.

% apply the result from \cite{Manselli} that bounds the total distance of a self-expanded curve belonging to a closed bounded convex set, to get the result.  

As a result, Theorem \ref{defn:se-curve} implies that the length of ${\underline r}$ is at most $O(d^{d/2} \text{diameter}(K_1))$, and the result follows since the length of ${\underline r}$ is same as that of ${\underline \sigma}$ which is $\Sigma$. 
\end{proof}

\section{Proof of Theorem \ref{thm:ocs}}
Clearly, with $f_t\equiv0$ for all $t$, with Algorithm \ref{coco_alg_1}, $y_t=x_t$ and the successive $x_t$'s are such that $x_{t+1} = \cP_{S_t}(x_{t})$. Thus, essentially, the curve ${\underline x} = (x_1, x_2), (x_2,x_3), \dots, (x_{T-1}, x_{T})$ formed by Algorithm \ref{coco_alg_1} for OCS is a projection curve (Definition \ref{defn:projectioncurve}) on $S_1\supseteq, \dots, \supseteq S_T$ and the result follows from Lemma \ref{lem:projection} and the fact that $\text{diameter}(S_1)\le D$.

%\begin{lemma}\label{lem:meanwidthfactorize}
%Let $K_1, K_2 \subseteq K$ be such that Lebesgue measure of $K_1\cap K_2=0$ and $K_1,K_2$ are convex.
%Then 
%\begin{equation}\label{}
%W(K_1)+W(K_2) \le W(K).
%\end{equation}
%\end{lemma}
\section{Proof of Theorem \ref{thm:tvmonotone}} 

%\begin{figure}
%
%\includegraphics[width=10cm,keepaspectratio,angle=0]{FigPhaseAnalysis.pdf}
%
%\caption{Depiction of the definition of phases and related quantities for a monotonic instance, where phase $1$ has $t^\star(1)=4$, thus curve till $F_4$ has been explored in phase $1$.  The next phase, phase $2$ is empty, since the angle between $F_4$ and $F_5$ is more than $\pi/4$. Phase $3$ begins from $F_5$ as its first hyperplane by setting $s(3)=5$.}
%\label{fig:monotone}
%\end{figure}





\begin{proof}
Recall that $d=2$, and the definition of $F_t$ from Definition \ref{defn:projhyperplane}. Let the center be $\sfc=\cP_{S_1}(x_1)$.  Let $t_{\text{orth}}$ be the earliest $t$ for which $\angle (F_t, F_1) = \pi$.

Initialize $\kappa=1$, $s(1)=1$, $\tau(1) =1$. 



{\bf BeginProcedure}
Step 1:Definition of Phase $\kappa$.
Consider $$\tau(\kappa) = \arg \max_{s(\kappa)< t \le t_{\text{orth}}, \angle(F_{s(\kappa)}, F_t) \le \pi/4} t.$$

{\bf If there is no such $\tau(\kappa)$}, 

\quad Phase $\kappa$ ends, define Phase $\kappa$ as {\bf Empty},  $s(\kappa+1) =  \tau(\kappa)+1$.


{\bf Else If} 

\quad $\angle(F_{\tau(\kappa)}, F_1)=\pi$ Exit

{\bf Else If} 

\quad $s(\kappa+1)=\tau(\kappa)$

{\bf End If}

Increment $\kappa=\kappa+1$,  and Go to Step 1.

{\bf EndProcedure}

\begin{example}\label{exm:phasedef} To better understand the definition of phases, consider Fig. \ref{fig:phases}, where the largest $t$ for which the angle between $F_t$ and $F_1$ is at most $\pi/4$ is $3$. Thus, $\tau(1)=3$, i.e., phase $1$ explores till time $t=3$ and phase $1$ ends. The starting hyperplane to consider in phase $2$ is $s(2)=3$ 
and given that angle between $F_3$ and and the next hyperplane $F_4$ is more than $\pi/4$, phase $2$ is empty and phase $2$ ends by exploring till $t=4$. The starting hyperplane to consider in phase $3$ is $s(3)=4$ and the process goes on. The first time $t$ such that the angle between $F_1$ and $F_t$ is $\pi$ is $t=6$, and thus $t_{\text{orth}}=6$, and the process stops at time $t=6$. 
This also implies that $S_6 \subset F_1$. 
Since $S_t$'s are nested, for all $t\ge 6$, $S_t\subset F_1$. Hence the total CCV after $t\ge t_{\text{orth}}$ is at most $GD$.
\end{example}

%Essentially, $\tau(1)$ is the largest time $t$ by which the angle between $F_1$ and $F_t$ is at most $\pi/4$. If there is no such $t$, then the angular region making angle of $\pi/4$ with $F_1$ is defined to the Empty. Thus, in phase $1$ hyperplanes till $F_{\tau(1)}$ are explored. The next phase begins by resetting $F_1$ as $F_{\tau(1)}$ by incrementing $s(\kappa) = \tau(\kappa)$. 
The main idea with defining phases, is to partition the whole space into empty and non-empty regions, where in each non-empty region, the starting and ending hyperplanes have an angle to at most $\pi/4$, while in an empty phase the starting and ending hyperplanes have an angle of at least $\pi/4$. Thus, we get the following simple result.

\begin{figure}


\includegraphics[width=15cm,keepaspectratio,angle=0]{Fig-phase.png}
\caption{Figure corresponding to Example \ref{exm:phasedef}.}
\label{fig:phases}
\end{figure}



\begin{lemma}\label{lem:nrphases} For $d=2$, there can be at most $4$ non-empty and $4$ empty phases.  
%that the number of phases (counting both non-empty and empty) till time $t_{\text{orth}}$ is at most $4$. 
\end{lemma}
Proof is immediate from the definition of the phases, since any consecutively occurring non-empty and empty phase exhausts an angle of at least $\pi/4$.

\begin{rem}\label{rem:aftertorth}
Since we are in $d=2$ dimensions, for all $t\ge t_{\text{orth}}$, the movement is along the hyperplane $F_1$ and thus the resulting constraint violation after time $t\ge t_{\text{orth}}$ is at most $GD$. Thus, in the phase definition above, we have only considered time till $t_{\text{orth}}$ and we only need to upper bound the CCV till time $t_{\text{orth}}$. \end{rem}





We next define the following required quantities.


\begin{definition}\label{defn:tstar}
With respect to the quantities defined for Algorithm \ref{coco_alg_1}, let for a non-empty phase $\kappa$ 
$$r_{\max}(\kappa)= \max_{s(\kappa) < t\le \tau(\kappa)} || y_t - \sfc||\ \text{and} \ t^\star(\kappa) = \arg \max_{s(\kappa) < t\le \tau(\kappa)}^T || y_t- \sfc||.$$
%Thus, a non-empty phase $\kappa$ consists of time slots $\cT(\kappa) = [t^\star(\kappa) - t^\star(\kappa-1), \tau(\kappa)]$ and $t^\star(\kappa) \in \cT(\kappa)$, and the angle $\angle(F_{t_1}, F_{t_2}) \le \pi/4$ when $t_1,t_2\in \cT(\kappa)$.
%
%If phase $\kappa$ is empty then $t^\star(\kappa)=t^\star(\kappa-1)+1$ %Consider the ball $\cB_1$ 
%with center $c$ as the projection of $x_1$ onto $S_1$ and radius $r_{\max}$. 
\end{definition}
$t^\star(\kappa)$ is the time index belonging to phase $\kappa$ for which $y_t$ is the farthest.

\begin{definition} 
A non-empty phase $\kappa$ consists of time slots $\cT(\kappa) = [\tau(\kappa-1), \tau(\kappa)]$ and the angle $\angle(F_{t_1}, F_{t_2}) \le \pi/4$ for all $t_1,t_2\in \cT(\kappa)$. Using Definition \ref{defn:tstar}, we partition $\cT(\kappa)$ as $\cT(\kappa) = \cT^-(\kappa) \cup \cT^+(\kappa)$, where $\cT^-(\kappa) = [\tau(\kappa-1)+1, t^\star(\kappa)+1]$ and $\cT^+(\kappa) = [ t^\star(\kappa)+2, \tau(\kappa)]$.
\end{definition}

Thus, $\cT(\kappa)$ and $\cT(\kappa+1)$ have one common time slot.

\begin{figure}
\includegraphics[width=15cm,keepaspectratio,angle=0]{Fig-newviolation.png}
\caption{Illustration of definition of $z_t(\kappa)$ for $t\in \cT(\kappa)$. In this example, for phase $1$, $t^\star(1)=3$ since the distance of $y_3$ from $\sfc$ is the farthest for phase $1$ that consists of time slots 
$\cT(1) = \{2,3\}$. Hence $z_{t^\star(1)+1}(1)=x_4$. For $t \in \cT(1) \backslash  t^\star(1)+1$,  $z_{t}(1)$ are such 
$z_{t+1}(1)$ is a projection of $z_{t}(1)$ onto $F_t$.}
\label{fig:MaxR}
\end{figure}
\begin{definition}\label{}
[Definition of $z_t(\kappa)$ \  for $t\in \cT^-(\kappa)$]. Let 
$z_{t^\star(\kappa)+1} = x_{t^\star(\kappa)+1}$.
For $t \in \cT^-(\kappa) \backslash t^\star(\kappa)+1$, define $z_t(\kappa)$ inductively as follows. 
$z_t(\kappa)$ is the pre-image of $z_{t+1}(\kappa)$ on $F_{t-1}$ such that the projection of $z_t(\kappa)$ on $F_t$ 
is $z_{t+1}(\kappa)$. 
%Consider the hyperplane $F_t$ as defined before. Extend this line till it intersects with ball $\cB_1 \cap \chi$, and call it $L_i'$. $S_i'$ is convex hull of $L_i' \cup S_{i+1}$. 
\end{definition}

\begin{definition}\label{}
[Definition of $z_t(\kappa)$ \  for \ $t\in \cT^+(\kappa)$]. 
For $t\in \cT^+(\kappa)$, define $z_t(\kappa)$ inductively as follows. 
$z_t(\kappa)$ is the projection of $z_{t-1}(\kappa)$ on $F_{t-1}$. 
%Consider the hyperplane $F_t$ as defined before. Extend this line till it intersects with ball $\cB_1 \cap \chi$, and call it $L_i'$. $S_i'$ is convex hull of $L_i' \cup S_{i+1}$. 
\end{definition}

See Fig. \ref{fig:MaxR} for a visual illustration of $t^\star(\kappa)$ and $z_t(\kappa)$.

The main idea behind defining $z_t(\kappa)$'s  is as follows. For each non-empty phase, we will construct a projection curve (Definition \ref{defn:projectioncurve}) using points $z_k$ such that the length of the projection curve upper bounds the CCV of Algorithm \ref{coco_alg_1} (shown in Lemma \ref{lem:violationub}), and then use Lemma \ref{lem:projection} to upper bound the length of the projection curve. %For an empty phase, the CCV of Algorithm \ref{coco_alg_1} is at most $D$, and there are at most $4$ empty phases. Thus, we will get the required bound.

\begin{definition}\label{}
[Definition of $S_t'$ for a non-empty phase $\kappa$:]  $S_{t^\star(\kappa)+1}' = S_{t^\star(\kappa)+1}$.
For $t \in \cT^-(\kappa) \backslash t^\star(\kappa)+1$, 
$S_t'$ is the convex hull of $z_{t+1}(\kappa) \cup S_t \cup S'_{t+1}(\kappa)$. For $t\in \cT^+(\kappa)$, 
$S_t' =S_t$.
See Fig. \ref{fig:defSprime}.
\end{definition}

\begin{lemma}\label{lem:nestedprime} For a non-empty phase $\kappa$, for any $t \in \cT(\kappa)$, $S_{t+1}' \subseteq S_t' $, i.e. they are nested.
\end{lemma}
\begin{figure}
\includegraphics[width=10cm,keepaspectratio,angle=0]{FigSprime.pdf}
\caption{Definition of $S_t$'s where $U_t$ are the extra regions that are added to $S_t$ to get $S_t'$.}
\label{fig:defSprime}
\end{figure}

\begin{definition} For a non-empty phase, 
 $\chi(\kappa) = S_{\tau(\kappa-1)}'  \cap \cH_{\tau(\kappa)}^+$, where $\cH_{\tau(\kappa)}^+$ has been defined in Definition \ref{defn:projhyperplane}.

\end{definition}

\begin{definition}\label{}
[New Violations for  $t\in \cT(\kappa)$:] 
For a non-empty phase $\kappa$, for $t\in \cT(\kappa) \backslash \tau(\kappa-1)$, let 
$$v_t(\kappa) = ||z_t(\kappa)-z_{t-1}(\kappa)||.$$
\end{definition}
%Phase $\kappa$ ends and increment $\kappa=\kappa+1$, reset $s(\kappa) =  t^\star(\kappa)$ $c=x_{t^\star(\kappa)}$ and Go to Step 1. 

%
%\begin{definition}
%For an empty phase $\kappa$, $\chi(\kappa) = S_{t^\star(\kappa-1)}' \cap \cH_{t^\star(\kappa-1)}^+$.
%\end{definition}
%Refer to Fig. \ref{fig:monotone} for a pictorial description of a two-dimensional monotonic instance.


\begin{lemma}\label{lem:membership} For each non-empty phase $\kappa$, all $z_{t}(\kappa)$'s for $t\in \cT(\kappa)$ belongs to $\cB(\sfc, \sqrt{2}D)$, where $\cB(c,r)$ is a ball with radius $r$ centered at $c$. In other words, $\chi(\kappa) \subseteq \cB(\sfc, \sqrt{2}D)$.
\end{lemma}
\begin{proof}
Recall that for a non-empty phase $\kappa$,  $\cT(\kappa) = \cT^-(\kappa) \cup  \cT^+(\kappa).$ We first argue about $t\in \cT^-(\kappa)$.
By definition, $z_{t^\star(\kappa)+1} = x_{t^\star(\kappa)+1}$ and $x_{t^\star(\kappa)+1}\in S_{t^\star(\kappa)}$. Thus, $z_{t^\star(\kappa)+1} \in \cB(\sfc, \sqrt{2}D)$.
Next we argue for $t \in \cT^-(\kappa) \backslash t^\star(\kappa)+1$.
Recall that the diameter of $\cX$ is $D$, and the fact that $y_t \in S_{t-1}$ from Algorithm \ref{coco_alg_1}. Thus, for any non-empty phase $\kappa$, the distance from $\sfc$ to the farthest $y_t$ belonging to the phase $\kappa$ is at most $D$, i.e., $r_{\max}(\kappa)\le D$. 
Let the pre-image of $z_{t^\star(\kappa)+1}(\kappa)$ onto $F_{s(\kappa)}$ (the base hyperplane with respect to which all hyperplanes have an angle of at most $\pi/4$ in phase $\kappa$) be $p(\kappa)$ such that projection of $p(\kappa)$ onto $F_{s(\kappa)}$ is $z_{t^\star(\kappa)+1}(\kappa)$. 
From the definition of any non-empty phase, the angle between $F_{s(\kappa)}$ and $F_{t}$ for $t\in \cT(\kappa)$ is at most $\pi/4$. 
Thus, the distance of $p(\kappa)$ from $\sfc$ is at most $\sqrt{2}D$. 

%Thus, if $t^\star(\kappa)+1-s(\kappa)=1$, we are done. 



%When $t^\star(\kappa)+1-s(\kappa)>1$, we proceed as follows. 
Consider the `triangle' $\Pi(\kappa)$ that is the convex hull of $\sfc, z_{t^\star(\kappa)+1}(\kappa)$ and $p(\kappa)$.
Given that the angle between $F_{t^\star(\kappa)}$ and $F_{t^\star(\kappa)-1}$ is at most $\pi/4$, the argument above implies that 
$z_t(\kappa) \in \Pi(\kappa)$ for $t=t^\star(\kappa)$. For $t= t^\star(\kappa)-1$, $z_t(\kappa) \in F_{t-1}$ is the projection of  $z_{t-1}(\kappa)$ onto $S_{t-1}'$. This implies that the distance of $z_t(\kappa)$ (for $t=t^\star(\kappa)-1$) from $\sfc$ is at most 
$$\frac{D}{\cos(\alpha_{t, t^\star(\kappa)}) \cos(\alpha_{t^\star(\kappa), t^\star(\kappa)+1})},$$ where 
$\alpha_{t_1,t_2}$ is the angle between $F_{t_1}$ and $F_{t_2}$.
From the monotonicity of angles $\theta_t$ (Definition \ref{defn:anglemonotone}), and the definition of a non-empty phase, we have that $\alpha_{t, t^\star(\kappa)}+\alpha_{t^\star(\kappa), t^\star(\kappa)+1} \le \pi/4$ and $\alpha_{t, t^\star(\kappa)}\ge 0, \alpha_{t^\star(\kappa), t^\star(\kappa)+1}\ge 0$.
Next, we appeal to the identity
\begin{equation}\label{eq:cosidentity}
\cos(A+B) \le \cos(A)\cos(B)
\end{equation} where $A+B\le \pi/4$, to claim that $z_t(\kappa) \in \Pi(\kappa)$ for $t=t^\star(\kappa)-1$. 

Iteratively using this argument while invoking the identity \eqref{eq:cosidentity} gives the result that for any $t \in \cT^-(\kappa)$, we have that $z_{t}(\kappa)$  belongs to $\Pi(\kappa)$. Since $\Pi(\kappa) \subseteq \cB(\sfc, \sqrt{2}D)$, we have the claim for all $t\in \cT^-(\kappa)$. 
%$t^\star(\kappa) - t^\star(\kappa-1)\le t \le t^\star(\kappa)$.


%
%
%Recall that for $t^\star(\kappa) - t^\star(\kappa-1)\le t\le t^\star(\kappa)$ 
%we have that $z_t(\kappa) \in F_{t-1}$ and is the projection of  $z_{t-1}(\kappa)$ onto $S_t'$.  Moreover, angle between any $F_t$ and $F_{t^\star(\kappa)}$ is at most $\pi/4$. Thus, as above $z_{t}(\kappa)$ for $t=t^\star(\kappa)-1$ 
%Thus, following the identity that 
%
%
%the monotonicity property of angles $\theta_t$ (Definition \ref{defn:anglemonotone}) and the non-expansive property of convex projections\footnote{Distance from $z_{t}(\kappa)$ to $z_{\tau(\kappa)}(\kappa)$  decreases as $t$ increases from $t^\star(\kappa) - t^\star(\kappa-1)\le t$ to $\tau(\kappa)$}, we have that $z_{t}(\kappa)$  belongs to $\Pi(\kappa)$. Since $\Pi(\kappa) \subseteq \cB(c, \sqrt{2}D)$, we have the claim for $t^\star(\kappa) - t^\star(\kappa-1)\le t \le t^\star(\kappa)$.

By definition $z_{t}(\kappa)$ for $t\in \cT^+(\kappa)$ belong to $S_{t-1}\subseteq S_1$. Thus, their distance from $\sfc$ is at most $D$. 
\end{proof}



\begin{lemma}\label{lem:violationub} For each non-empty phase $\kappa$, and for $t\in \cT(\kappa)$ the violation $v_t(\kappa)\ge  \text{dist}(x_t, S_t)$, where $\text{dist}(x_t, S_t)$ is the original violation.
\end{lemma}
\begin{proof}
By construction of any non-empty phase $\kappa$, for $t\in \cT(\kappa)$ both $x_t(\kappa)$ and $z_t(\kappa)$ belong to $F_{t-1}$. Moreover, by construction, the distance of $z_t(\kappa)$ from $\sfc$ is at least as much as the distance of $x_t$ from $\sfc$. Thus, using  the monotonicity property of angles $\theta_t$ (Definition \ref{defn:anglemonotone}) we get the result. See Fig. \ref{fig:MaxR} for a visual illustration.
\end{proof}



For each non-empty phase $\kappa$, by definition, the curve defined by sequence $z_{t}(\kappa)$ for $t\in\cT(\kappa)$ is a projection curve (Definition \ref{defn:projectioncurve}) on sets $S'_t(\kappa)$ (note that $S'_t(\kappa)$'s  are nested from Lemma \ref{lem:nestedprime}). Moreover, for all $t\in\cT(\kappa)$, set $S'_t(\kappa) \subset \chi(\kappa)$ which is a bounded convex set. 
Thus, for $d=2$ from Lemma \ref{lem:projection} the length of curve ${\underline z}(\kappa) = \{(z_{t}(\kappa), z_{t+1}(\kappa))\}_{t\in \cT(\kappa)}$
\begin{equation}\label{eq:totallengthprojection}
\sum_{t\in \cT(\kappa)} v_t(\kappa) \le 2
\text{diameter}(\chi(\kappa)).
\end{equation}


 

%Next, we make use of Lemma \ref{lem:width2D} that exactly characterizes the average width in $d=2$-dimensions. 
By definition, the number of non-empty phases till time $t_{\text{orth}}$ is at most $4$. Moreover, in each non-empty phase $\chi(\kappa) \subseteq \cB(\sfc, \sqrt{2}D)$ from Lemma \ref{lem:membership} . 

Thus, from  \eqref{eq:totallengthprojection}, we have that \begin{align}\nn
\sum_{\text{Phase} \ \kappa \ \text{is non-empty}} \ \ \ \sum_{t\in \cT(\kappa)} v_t(\kappa) &\le \sum_{\text{Phase} \ \kappa \ \text{is non-empty}} 2\ \text{diameter}(\chi(\kappa)) \\ \label{eq:summeanwidth}
&\le 8 \ \text{diameter}(\cB(\sfc, \sqrt{2}D))\le O(D).
\end{align}

Using Lemma \ref{lem:violationub}, we get 
\begin{align}\label{eq:finalviolation}
\sum_{\text{Phase} \ \kappa \ \text{is non-empty}} \ \ \ \sum_{t\in \cT(\kappa)}\text{dist}(x_t, S_t) \le O(D).
\end{align}

For any empty phase, the constraint violation is the length of line segment $(x_t,\cP_{S_t}(x_{t}))$ (Algorithm \ref{coco_alg_1}) crossing it is a straight line whose length is at most $O(D)$. 
%The length $||(y_t - x_{t+1})||$ of the curve is also an upper bound on the CCV incurred by Algorithm \ref{coco_alg_1} when crossing the two hyperplanes that define the empty phase. 
 Moreover, the total number of empty phases (Lemma \ref{lem:nrphases}) is a constant.
 %, so the total length of the violation curve (Algorithm \ref{coco_alg_1}) crossing all empty phases is $O(D)$. 
 Thus, the length of the curve $(x_t,\cP_{S_t}(x_{t}))$ for Algorithm \ref{coco_alg_1} corresponding to all empty phases is at $O(D)$.
%
%From Lemma \ref{lem:membership} it follows that  that $\cup_{\kappa} \chi_k \subseteq \cB(c, \sqrt{2}D)$. Moreover, $\chi_k \cap \chi_{k+1} = F_{t^\star(\kappa)}$ and $\chi_j \cap \chi_k = \phi$ for $|j- k|>1$. Since $F_{t^\star(\kappa)}$ is 
%a hyperplane that contributes zero mass to the integral in \eqref{eq:projlength}, we have that 
%
%\begin{equation}\label{eq:summeanwidth}
%\cup_{\kappa} W(\chi(\kappa)) \le W(\cB(c, \sqrt{2}D)) \stackrel{(a)}\le O(\sqrt{2}D).
%\end{equation}
%where $(a)$ follows from \eqref{eq:WBound1}.

Recall from \eqref{eq:distviolationrelation} that the CCV is at most $G$ times $\text{dist}(x_t, S_t)$.
Thus, from \eqref{eq:finalviolation} we get that the total violation incurred by Algorithm \ref{coco_alg_1} corresponding to non-empty phases is at most $O(GD)$, while corresponding to empty phases is at $O(GD)$.
Finally, accounting for the very first violation $\text{dist}(x_1, S_1)\le D$ and the fact that the CCV after time $t\ge t_{\text{orth}}$ (Remark \ref{rem:aftertorth}) is at most $GD$, we get that the total constraint violation $\text{CCV}_{[1:T]}$ for Algorithm \ref{coco_alg_1} is at most $O(G D)$. 

\end{proof}


%\bibliographystyle{IEEE}

%  \def\rvara{2}
%\begin{tikzpicture}
%    % Define the center point and radii
%    \coordinate (center) at (0,0);
%  \def\avar{1}
%    \def\bvar{2}
%    \def\cvar{3}
%    \def\dvar{4}
%    \def\evar{5}
%
%    % Draw the concentric circles
%    \draw[thick] (center) circle (\avar);
%    \draw[thick] (center) circle (\bvar);
%    \draw[thick] (center) circle (\cvar);
%    \draw[thick] (center) circle (\dvar);
%    \draw[thick] (center) circle (\evar);
%
%    % Place the nodes
%    
%    
%    
%    
%    \node[circle,fill=black,inner sep=2pt] at (center) {};
%    \node[below] at (center) {$x_n$};
%
%    \node[circle,fill=black,inner sep=2pt] at (\avar,0) {};
%    \node[below] at (\avar,0) {$x_{n-1}$};
%
%    \node[circle,fill=black,inner sep=2pt] at (1,3.872) {};
%    \node[above] at (1,3.872) {$x_{\ell+2}$};
%
%    \node[circle,fill=black,inner sep=2pt] at (0,\evar) {};
%    \node[below] at (0,\evar) {$x_{\ell+1}$};
%    
%    \node[circle,fill=black,inner sep=2pt] at (-6,6) {};
%    \node[below] at (-6,6) {$x_{0}$};
%    
%    \node[circle,fill=black,inner sep=2pt] at (-5.5,5.5) {};
%    \node[below] at (-5.5,5.5) {$x_{1}$};
%    
%    \node[circle,fill=black,inner sep=2pt] at (-4, 4) {};
%    \node[below] at (-4, 4) {$x_{\ell}$};
%    
%    \draw (-6.5,6.5) -- (0,0)  -- (5,-5);
%    
%     \draw[dashed] (-4, 4) -- (0,\evar);
%     
%     \draw (0,\evar+1) -- (0,0)  -- (0,-6);
%    
%    \draw[dashed] (0,0)  -- (1,3.872);
%    
%    \draw (0.35,1.414) arc (60:75:1.5);
%    
%    \node[above] at (.35, 1.414) {$\alpha_{\ell+1}$};
%    
%      \draw (0,0) -- node[below] {$r_{\ell+1}$}  ++(-5,0);
%
%      \draw (0,0) -- node[below] {$r_{\ell+2}$}  ++(-3.75,1.39);
%
%    
%\end{tikzpicture}
%
%
%
%Consider the 2-D case alone for analyzing the greedy algorithm for Nested Convex Body Chasing. Let the starting point be $x_0$. Let at time $t=1, \dots, n$, a convex body $S_t \subseteq S_{t-1}$ is shown and the point chosen by Greedy by $x_t \in S_n$.
%
%Then $x_0, x_1, x_2, \dots, x_n$ is the sequence of points generated by the Greedy algorithm and let curve $C$ denote the line segments joining $x_0, x_1, x_2, \dots, x_n$.
%
%
%Let $x_1, x_2, \dots, x_\ell$ be such that they lie on the line joining $x_0$ and $x_n$. If there is no such $\ell$, then let $\ell =0$. This implies that the total length of $C$ from $x_0$ till $x_\ell$ is at most $D$, i.e., 
%\begin{equation}\label{eq:prefix}
%\sum_{i=0}^{\ell-1}d(x_i,x_{i+1}) \le D.
%\end{equation}
%
%
% 
%Let the $y$-axis be oriented along the line segment $x_{\ell+1}, x_n$ and $x$-axis be perpendicular to the $y$-axis. 
%
%{\bf Property 1:} First thing to note is that all convex bodies $S_{\ell+1}, S_{\ell+2}, \dots, S_n$ lie on one side of the $y$-axis, since otherwise it will contradict the convexity of  $S_t, t\ge \ell+1$.
%
%
% Let $r_i = d(x_i, x_n)$ be the distance between point $x_i$ and the final point $x_n$. As we discussed, because of successive projections, $r_i\le r_{i-1}$. 
%
%Consider $x_n$ as the center and draw $n-{\ell+1}$ concentric circles with radius $r_i, i=\ell+1, 2,\dots, n-1$. 
%Let $\alpha_i$ be the angle between line segments $(x_i,x_n)$ and $(x_{i+1},x_n)$. 
%\begin{lemma}\label{lem:angle}
%$\alpha_i\ge0$ for all $i=\ell+1, 2,\dots, n-1$.
%\end{lemma}
%Proof follows similar to Property 1.
%
%Then from triangle inequality we get 
%$d(x_i,x_{i+1}) \le r_{i} - r_{i+1} + \alpha_i d(x_i, x_n)$. 
%Therefore, the total distance of the curve $C$ from $x_{\ell+1}$ till $x_n$
%$$\sum_{i=\ell+1}^{n-1}d(x_i,x_{i+1}) \le \sum_{i=1}^{n-1} r_{i} - r_{i+1}+ \alpha_i d(x_i, x_n) \le D+  D \sum_{i=1}^{n-1} \alpha_i.$$ 
%From Lemma \ref{lem:angle} and Property 1, $\sum_{i=1}^{n-1} \alpha_i\le \pi$, 
%hence 
%$\sum_{i=\ell+1}^{n-1}d(x_i,x_{i+1}) \le D+D\pi$.
%Since the distance between $(x_{\ell},x_{\ell+1})$ is at most $D$, combining this with \eqref{eq:prefix}, we get that the total length of the curve $C$ is $3D + \pi D$.

%\input{NCBCtoConstraintViolation}
 
%\section{Introduction}
%Consider the 2-D case for the moment, so the first convex body $K_1 \subset \bbR^2$.
%Let the initial position of the Greedy algorithm be $x_1$, and $x_i$ be its location once $K_i \subset K_{i-1}$ has been revealed and the projection has been taken from $x_{i-1}$ onto $K_i$.
%Let the convex hull of the $x_1, \dots, x_i$ be $S_i$. 
%
%
%\section{Preliminaries}
%    Let a curve be called Lipschitz if the map $\gamma:I \rightarrow \bbR^n$ is Lipschitz.
%
%\begin{definition}\label{defn:se-curve} A Lipschitz curve $\gamma: I \rightarrow \bbR^n$  is called self-expanded, if for every $t$ where 
%$\gamma'(t)$ exists, we have 
%$$< \gamma'(t), \gamma(t)-\gamma(u)> \ \ge 0$$ for all $u\in I$ with $u \le t$. 
%In words, what this means is that if $\gamma$ starting in a point $x_0$ is self expanded, if for every $x\in \gamma$ for which there exists the tangent line $T$, the arc (sub-curve) $(x_0, x)$ is
%contained in one of the two half-spaces, bounded by the hyperplane through
%$x$ and orthogonal to $T$. 
%\end{definition}
%
%
%
%\begin{definition}
%Let $x=x(s)$ be the representation
%of $\gamma$ as a function of the arc length; let us define, $\gamma(s)= \{x\in \bbR^2, x=x(\sigma), 0 \le \sigma\le s\}$.
%$\gamma\in \Gamma$ is defined to 
%satisfy to be self-expanding if $\gamma'(s)$ exists for any $s$, then the 
%\end{definition}
%
%
%Consider $\bbR^2$, where in addition to the two orthogonal basis $x$ and $y$ axes, consider the rotation of $x,y$ axes by angle $\pi/4$, denoted by   $x'$ and $y'$, respectively, as another orthogonal basis pair.
%The set of total axes is $A = \{x,y, x', y'\}$.
%
%\begin{definition} Consider any convex body $B \subset \bbR^2$.
%For $a\in A$, let $W_a(i)$ be the minimum distance between any two supporting hyperplanes of $B$ that are perpendicular to $a$-axis.
%\end{definition}
%
%Clearly, for any $a\in A$, $W_a(B)$ is a non-increasing quantity as a function of $B$.  
%We will prove the following result for  self-expanding curves in $\bbR^2$.
% 
%\begin{theorem}\label{thm:mainresult}
%For any self-expanded curve in $\bbR^2$, let $\gamma^+$ be its convex hull. Then 
%$$\max_a\{W'_a(\gamma^+)\} \ge \frac{1}{\sqrt{2}}.$$
%\end{theorem}
% 
%The curved produced by the Greedy algorithm satisfies the self-contracting property that is defined as follows.
%
%  \begin{definition} A curve $\gamma: I \rightarrow \bbR^n$ is self-contracted if for every $t_1 \le t_2 \le t_3$ in $I$, 
%  $$d(\gamma(t_2), \gamma(t_3)) \le d(\gamma(t_1), \gamma(t_3)).$$
%  \end{definition}
%  
%  \begin{lemma} Let $\gamma$ be a self-contracted curve and let $\gamma$ be differentiable at point $t$, then 
%  $$< \gamma'(t), \gamma(u)-\gamma(t)> \ \ge 0$$ for all $u\in I$ with $u>t$. 
%\end{lemma}
%
%\begin{prop}\label{prop:scGreedy} The affine extension $C$ of the sequence of points  $\{x_1, x_2, \dots, x_m\}$ produced by the Greedy algorithm for the CBC problem with sets $\{K_1, K_2, \dots, K_m\}$ is a self-contracting curve because of the orthogonality principle of projections on convex bodies.
%\end{prop}
%Next, we connect the self-contracted and self-expanded curves as follows. 
%
%  \begin{definition} Given a curve $\gamma:I \rightarrow \bbR^n$, denote by $I^- = \{-t |  t\in I\}$ and define the reverse curve 
%  $\gamma^-:I^- \rightarrow \bbR^n$ as 
%  $$\gamma^-(t) = \gamma(-t), \ \text{for} \ t\in I^-.$$
%    \end{definition}
%    
%    Essentially, $\gamma^-$ is $\gamma$ traced backwards or in reverse order. See Fig. \ref{fig:cont-expand} for an example.
%    
%    \begin{lemma}\label{lem:se-scconnection}
%Let $\gamma:I \rightarrow \bbR^n$ be a Lipschitz curve. Then $\gamma$ is self-contracted if and only if $\gamma^-$ is self-expanded curve.
%\end{lemma}
%
%Lemma \ref{lem:se-scconnection} implies that the length of self-expanded and self-contracted curves is the same. 
%Thus, from Proposition \ref{prop:scGreedy}, to bound the length of curve $C$ produced by the Greedy algorithm, it is sufficient to bound the length of the self-expanded curve corresponding to $C$.  
%  
%Let  the diameter of $K_1$, the first convex body, be $D$. Then, Theorem \eqref{thm:mainresult} implies that the length of curve $C$ produced by the Greedy algorithm is at most $4\sqrt{2} D$.
%
 

%\begin{figure}[h]
%\centering
%\includegraphics[width=90mm]{greedyCBC.jpg}
%\caption{}
%\label{fig:greedy}
%\end{figure}
%Refer to Fig. \ref{fig:greedy}, where once the algorithm reaches $x_2$, in next step, either both $x$ and $y$ coordinates increase or only one of them does, i.e. $x_3$ belongs to either sector ABE or ACD. 
%Lets consider ABE. Because of the projection step to get to $x_2$ from $x_1$, we know that the angle $x_1x_2B$ is at least $\pi/2$. Therefore, angle $\theta_1\le \pi/4$, and thus the projection $\Delta_y$ of $\ell_1$ ( the distance moved by Greedy in the next step to get to $x_3$ from $x_2$) is at least $\ell_1/\sqrt{2}$. Even though the x-coordinate is decreasing between $x_3$ and $x_2$, we have $$\max\{W'_x(i), W'_y(i)\} \ge \frac{1}{\sqrt{2}}.$$
%
%The same idea works when $x_3$ belongs to sector ACD with roles reversed between $W_x$ and $W_y$.  Moreover, when $x_3$ belongs to the Good Area, both $x$ and $y$ coordinates increase with at least one of $W_x$ and $W_y$ having gradient at least as much as $1/\sqrt{2}$. Thus, we have 
%$$\max\{W'_x(i), W'_y(i)\} \ge \frac{1}{\sqrt{2}}$$ in each step.
%
%Now I want to claim that the total length $L$ of the curve followed by the greedy algorithm is at most $2\times \sqrt{2} \times \text{diameter}$. 
%
%Essentially, because of $$\max\{W'_x(i), W'_y(i)\} \ge \frac{1}{\sqrt{2}},$$ if $L > 2\times \sqrt{2} \times \text{diameter}$ then the curves has to leaves the body in at least one dimension. 
%
%\section{$d$-dimensions}
%{\bf Algorithm:} On arrival of convex $S_j \subseteq S_{j-1}$, choose $x_j$ as the projection of $x_{j-1}$ on to $S_t$.
%
%Let $x_j$ be the point chosen by the algorithm at time step $j$ on arrival of convex set $S_j$.
%Let the convex hull of the $x_1, \dots, x_j$ be $C_j$.
%At time step $j$, for $i=1,\dots, d$, let $W_i(j)$ be the minimum distance between any two hyperplanes parallel to the $i$-axis that contain $C_j$. $W_i(j)$ is essentially the $i$-axis diameter of $C_j$. %Similarly define  $W_y(i)$ to be the minimum distance between any two hyperplanes parallel to the x-axis that contain $C_i$. 
%
%\begin{lemma}
%Either $x_{j+1}=x_{j}$ or at each time step $j+1$, $\exists \ i$ such that $W_i(j+1)-W_i(j) > 0$ for $i=1,\dots, d$.
%\end{lemma}
%\begin{proof}
%Note that the angle between $x_{j+1} -x_t$ and $x_{t} -x_{t-1}$ is more than $\pi/4$ for any $t\le j$, since each point $x_t$ is found by projecting $x_{t-1}$ on to the convex set $S_t$. Moreover, $S_{t+1} \subseteq S_t$ $\forall \ t$. This implies that either $x_{j+1}=x_{j}$ or at each time step $j+1$, $||x_t-x_{j+1}||  >  ||x_j-x_{j+1}||$ for all $t < j$.
%Thus,  either $x_{j+1}=x_{j}$,
%or $\exists \ i$ such that $W_i(j+1)-W_i(j) > 0$ for $i=1,\dots, d$.
% 
%\end{proof}
%
%\begin{lemma}
%Among the indices $i$ for which $W_i(j+1)-W_i(j) > 0$ for $i=1,\dots, d$, $\exists \  i'$ such that $W_{i'}(j+1)-W_{i'}(j) \ge \frac{1}{\sqrt{d}} ||x_{j+1}-x_j||$.
%\end{lemma}
%
%\begin{proof}
%Consider a hyperplane $H$ passing through the origin. Let any vector $v$ lie in one of the half space $G$ corresponding to $H$. Let the projection of $v$ onto the $i^{th}$-axis be $P_i(v) = \frac{v_i}{||v||}$, and its length be $|P_i|$. Then clearly the  $\max_i\{|P_i(v)|\}\ge \frac{1}{\sqrt{d}}$.
%
%Moreover, if $v \in G'$ where $G'\subset G$ with dimension $d' < d$, then 
%\begin{equation}\label{eq:growth}
%\max_{i=1, \dots, d'}\{|P_i(v)|\}\ge \frac{1}{\sqrt{d'}}.
%\end{equation}
%
%Let the set of indices for which $W_i(j+1)-W_i(j) > 0$ be $\cM$ with cardinality $|\cM|$. Consider the supporting hyperplane $H_j$ of $S_j$ at point $x_j$ that is normal to $x_{j}-x_{j-1}$. Let $G_j$ be the half space of $H_j$ that contains $S_{j+1}$. Consider the restriction of $G_j$ onto the indices of $\cM$, and consider point $x_j$ as the origin. 
%Then from \eqref{eq:growth} we get that  
%$W_{i'}(j+1)-W_{i'}(j) \ge \frac{1}{\sqrt{|\cM|}} ||x_{j+1}-x_j||$.
%Since $|\cM|\le d$, we get the result.
%
%
%\end{proof} 
%
%%Extending the above argument to $d$ dimensions. Let $x_j$ be the most recent point found by the algorithm when convex body $C_j$ was revealed. Consider $H_j$ the supporting hyperplane of $C_j$ at $x_j$ that is normal to $x_j-x_i$. Let the halfspace of $H_j$ containing $x_j-x_i$ be $G_j$. 
%%
%%Given that $x_j$ was found by projection onto $C_j$, $C_{j+1} \cap G_j = \phi$. Thus, the vector $x_{j+1}-x_j$ lies in  $G_j^c$, and hence $W_i(j+1)$ increases by amount $P_i(x_{j+1}-x_j)$ $\forall \ i=1, \dots, d$. Given that the projection  satisfies
%%$\max_i\{|P_i(x_{j+1}-x_j)|\}\ge \frac{1}{\sqrt{d}}$, we get that 
%%$$\max_{i, i=1,\dots, d} \{W'_i(t)\} \ge \frac{1}{\sqrt{d}}$$ in each step $t$.
%%
%%Given that there are a total of $d$ dimensions, we will get $L \le  d \times \sqrt{d} \times \text{diameter}$.
%
 \end{document}