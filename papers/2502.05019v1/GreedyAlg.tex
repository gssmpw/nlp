\documentclass{article}
%\documentclass[anon,12pt]{colt2025}
\usepackage{graphicx} % Required for inserting images
 \usepackage{tikz}
 \usepackage{booktabs}
 \usepackage{natbib}
 \usepackage{pgfplots}
\usetikzlibrary{3d}
\usetikzlibrary{backgrounds}
\pgfdeclarelayer{bg}
\pgfsetlayers{bg,main}
\usepackage[left=1.5cm, right=2cm]{geometry}

 \newgeometry{left=3cm,bottom=4cm}

\tikzset{
  background/.style={%
    execute at begin node={\begin{pgfonlayer}{bg}},
    execute at end node={\end{pgfonlayer}}
  }
}

 %\usepackage{algorithm}
  %\usepackage{algorithmic}
 % \usepackage[square,comma,sort&compress]{natbib}
  \usepackage{algorithm,algpseudocode}
  \usepackage{tikz}
  \usetikzlibrary{calc}
  \usepackage{tkz-euclide}
  \usepackage{xcolor}
\usetikzlibrary{arrows}
\usetikzlibrary{shapes.geometric, arrows.meta}
\usepackage{amsmath}
  
%\algnewcommand{\algorithmicforeach}{\textbf{for each}}
%\algdef{SE}[FOR]{ForEach}{EndForEach}[1]
  %{\algorithmicforeach\ #1\ \algorithmicdo}% \ForEach{#1}
  %{\algorithmicend\ \algorithmicforeach}% \EndForEach
 \usetikzlibrary{math}
\title{$O(\sqrt{T})$ Static Regret and Instance Dependent Constraint Violation for Constrained Online Convex Optimization}
\author{%
  Rahul Vaze, Abhishek Sinha,  \\
 School of Technology and Computer Science \\
  Tata Institute of Fundamental Research \\
  Mumbai 400005, India \\
  \texttt{rahul.vaze@gmail.com},
  \texttt{abhishek.sinha@tifr.res.in}
}
%\author{Rahul vaze, Abhishek Sinha}
%\date{October 2024}
%\usepackage{fancyhdr}
%\pagestyle{fancy}

\usepackage{amsfonts}
\usepackage{times}
%\usepackage[pdftex]{graphicx}
%\DeclareGraphicsExtensions{.jpg}
%\usepackage[dvips]{graphicx}
%\DeclareGraphicsExtensions{.eps}
\usepackage{latexsym}
\usepackage{amssymb}
\usepackage{amsmath}
%\usepackage{cite}
\usepackage{verbatim}
\newtheorem{theorem}{Theorem}

\newtheorem{acknowledgement}[theorem]{Acknowledgement}
%\newtheorem{algorithm}[theorem]{Algorithm}
\newtheorem{assumption}[theorem]{Assumption}
\newtheorem{axiom}[theorem]{Axiom}
\newtheorem{case}[theorem]{Case}
\newtheorem{claim}[theorem]{Claim}
\newtheorem{conclusion}[theorem]{Conclusion}
\newtheorem{condition}[theorem]{Condition}
\newtheorem{conjecture}[theorem]{Conjecture}
\newtheorem{corollary}[theorem]{Corollary}
\newtheorem{rem}{Remark}
\newtheorem{definition}[theorem]{Definition}
\newtheorem{example}[theorem]{Example}
\newtheorem{exercise}[theorem]{Exercise}
\newtheorem{fact}[theorem]{Fact}
\newtheorem{lemma}[theorem]{Lemma}
%\newtheorem{notation}[theorem]{Notation}
%\newtheorem{problem}[theorem]{Problem}
%\newtheorem{proposition}[theorem]{Proposition}
%\newtheorem{solution}[theorem]{Solution}
%\newtheorem{summary}[theorem]{Summary}
%\newtheorem{remark}[theorem]{Remark}
\newenvironment{proof}{ \textbf{Proof:} }{ \hfill $\Box$}
%\newtheorem{definition}{Definition}
%\newtheorem{prop}{Proposition}
%\newtheorem{lemma}{Lemma}
%\newtheorem{corr}{Corrolary}
%\newtheorem{theorem}{Theorem}
%\newtheorem{conjecture}{Conjecture}

\newcommand{\R}{\mathbb{R}}
\newcommand{\Z}{\mathbb{Z}}
\newcommand{\ra}{\rightarrow} 
\newcommand{\ua}{\uparrow}
\newcommand{\prob}[1]{P\left(#1\right)} 
\newcommand{\imp}{\Rightarrow}
\newcommand{\re}{\mathbb{R}} 
\newcommand{\Exp}[1]{\mathbb{E}\left[#1\right]} %Expectation
\newcommand{\eqdist}{\stackrel{d}{=}}
\newcommand{\std}{\leq_{\mathrm{s.t.}}}
\newcommand{\indicator}[1]{\boldsymbol{1}_{\{#1\}}}
\newcommand{\ceil}[1]{\left\lceil #1 \right\rceil}
\newcommand{\floor}[1]{\left\lfloor #1 \right\rfloor}
%\DeclareMathOperator*{\argmax}{arg\,max}
%\DeclareMathOperator*{\argmin}{arg\,min}
\newcommand{\figref}[1]{{Fig.}~\ref{#1}}
\newcommand{\tabref}[1]{{Table}~\ref{#1}}
\newcommand{\bookemph}[1]{ {\em #1}}
\newcommand{\Ns}{N_s}
\newcommand{\Ut}{U_t}
\newcommand{\fig}[1]{Fig.\ \ref{#1}}
\def\onehalf{\frac{1}{2}}
\def\etal{et.\/ al.\/}
\newcommand{\bydef}{\triangleq}
\newcommand{\tr}{{\it{tr}}}
\def\SNR{{\textsf{SNR}}}
\def\Pe{{P_e}}
\def\SINR{{\mathsf{SINR}}}
\def\SIR{{\mathsf{SIR}}}
\def\MI{{\mathsf{MI}}}
\def\greedy{{\mathsf{GREEDY}}}

% blackboard lowercase
\def\bydef{:=}
\def\bba{{\mathbb{a}}}
\def\bbb{{\mathbb{b}}}
\def\bbc{{\mathbb{c}}}
\def\bbd{{\mathbb{d}}}
\def\bbee{{\mathbb{e}}}
\def\bbff{{\mathbb{f}}}
\def\bbg{{\mathbb{g}}}
\def\bbh{{\mathbb{h}}}
\def\bbi{{\mathbb{i}}}
\def\bbj{{\mathbb{j}}}
\def\bbk{{\mathbb{k}}}
\def\bbl{{\mathbb{l}}}
\def\bbm{{\mathbb{m}}}
\def\bbn{{\mathbb{n}}}
\def\bbo{{\mathbb{o}}}
\def\bbp{{\mathbb{p}}}
\def\bbq{{\mathbb{q}}}
\def\bbr{{\mathbb{r}}}
\def\bbs{{\mathbb{s}}}
\def\bbt{{\mathbb{t}}}
\def\bbu{{\mathbb{u}}}
\def\bbv{{\mathbb{v}}}
\def\bbw{{\mathbb{w}}}
\def\bbx{{\mathbb{x}}}
\def\bby{{\mathbb{y}}}
\def\bbz{{\mathbb{z}}}
\def\bb0{{\mathbb{0}}}

% Bold lowercase
\def\bydef{:=}
\def\ba{{\mathbf{a}}}
\def\bb{{\mathbf{b}}}
\def\bc{{\mathbf{c}}}
\def\bd{{\mathbf{d}}}
\def\bee{{\mathbf{e}}}
\def\bff{{\mathbf{f}}}
\def\bg{{\mathbf{g}}}
\def\bh{{\mathbf{h}}}
\def\bi{{\mathbf{i}}}
\def\bj{{\mathbf{j}}}
\def\bk{{\mathbf{k}}}
\def\bl{{\mathbf{l}}}
\def\bm{{\mathbf{m}}}
\def\bn{{\mathbf{n}}}
\def\bo{{\mathbf{o}}}
\def\bp{{\mathbf{p}}}
\def\bq{{\mathbf{q}}}
\def\br{{\mathbf{r}}}
\def\bs{{\mathbf{s}}}
\def\bt{{\mathbf{t}}}
\def\bu{{\mathbf{u}}}
\def\bv{{\mathbf{v}}}
\def\bw{{\mathbf{w}}}
\def\bx{{\mathbf{x}}}
\def\by{{\mathbf{y}}}
\def\bz{{\mathbf{z}}}
\def\b0{{\mathbf{0}}}
\def\opt{\mathsf{OPT}}
\def\off{\mathsf{OFF}}
% Bold capital letters
\def\bA{{\mathbf{A}}}
\def\bB{{\mathbf{B}}}
\def\bC{{\mathbf{C}}}
\def\bD{{\mathbf{D}}}
\def\bE{{\mathbf{E}}}
\def\bF{{\mathbf{F}}}
\def\bG{{\mathbf{G}}}
\def\bH{{\mathbf{H}}}
\def\bI{{\mathbf{I}}}
\def\bJ{{\mathbf{J}}}
\def\bK{{\mathbf{K}}}
\def\bL{{\mathbf{L}}}
\def\bM{{\mathbf{M}}}
\def\bN{{\mathbf{N}}}
\def\bO{{\mathbf{O}}}
\def\bP{{\mathbf{P}}}
\def\bQ{{\mathbf{Q}}}
\def\bR{{\mathbf{R}}}
\def\bS{{\mathbf{S}}}
\def\bT{{\mathbf{T}}}
\def\bU{{\mathbf{U}}}
\def\bV{{\mathbf{V}}}
\def\bW{{\mathbf{W}}}
\def\bX{{\mathbf{X}}}
\def\bY{{\mathbf{Y}}}
\def\bZ{{\mathbf{Z}}}
\def\b1{{\mathbf{1}}}


% Blackboard capital letters
\def\bbA{{\mathbb{A}}}
\def\bbB{{\mathbb{B}}}
\def\bbC{{\mathbb{C}}}
\def\bbD{{\mathbb{D}}}
\def\bbE{{\mathbb{E}}}
\def\bbF{{\mathbb{F}}}
\def\bbG{{\mathbb{G}}}
\def\bbH{{\mathbb{H}}}
\def\bbI{{\mathbb{I}}}
\def\bbJ{{\mathbb{J}}}
\def\bbK{{\mathbb{K}}}
\def\bbL{{\mathbb{L}}}
\def\bbM{{\mathbb{M}}}
\def\bbN{{\mathbb{N}}}
\def\bbO{{\mathbb{O}}}
\def\bbP{{\mathbb{P}}}
\def\bbQ{{\mathbb{Q}}}
\def\bbR{{\mathbb{R}}}
\def\bbS{{\mathbb{S}}}
\def\bbT{{\mathbb{T}}}
\def\bbU{{\mathbb{U}}}
\def\bbV{{\mathbb{V}}}
\def\bbW{{\mathbb{W}}}
\def\bbX{{\mathbb{X}}}
\def\bbY{{\mathbb{Y}}}
\def\bbZ{{\mathbb{Z}}}

% Caligraphic capital letters
\def\cA{\mathcal{A}}
\def\cB{\mathcal{B}}
\def\cC{\mathcal{C}}
\def\cD{\mathcal{D}}
\def\cE{\mathcal{E}}
\def\cF{\mathcal{F}}
\def\cG{\mathcal{G}}
\def\cH{\mathcal{H}}
\def\cI{\mathcal{I}}
\def\cJ{\mathcal{J}}
\def\cK{\mathcal{K}}
\def\cL{\mathcal{L}}
\def\cM{\mathcal{M}}
\def\cN{\mathcal{N}}
\def\cO{\mathcal{O}}
\def\cP{\mathcal{P}}
\def\cQ{\mathcal{Q}}
\def\cR{\mathcal{R}}
\def\cS{\mathcal{S}}
\def\cT{\mathcal{T}}
\def\cU{\mathcal{U}}
\def\cV{\mathcal{V}}
\def\cW{\mathcal{W}}
\def\cX{\mathcal{X}}
\def\cY{\mathcal{Y}}
\def\cZ{\mathcal{Z}}

% Sans serif capital letters
\def\sfA{\mathsf{A}}
\def\sfB{\mathsf{B}}
\def\sfC{\mathsf{C}}
\def\sfD{\mathsf{D}}
\def\sfE{\mathsf{E}}
\def\sfF{\mathsf{F}}
\def\sfG{\mathsf{G}}
\def\sfH{\mathsf{H}}
\def\sfI{\mathsf{I}}
\def\sfJ{\mathsf{J}}
\def\sfK{\mathsf{K}}
\def\sfL{\mathsf{L}}
\def\sfM{\mathsf{M}}
\def\sfN{\mathsf{N}}
\def\sfO{\mathsf{O}}
\def\sfP{\mathsf{P}}
\def\sfQ{\mathsf{Q}}
\def\sfR{\mathsf{R}}
\def\sfS{\mathsf{S}}
\def\sfT{\mathsf{T}}
\def\sfU{\mathsf{U}}
\def\sfV{\mathsf{V}}
\def\sfW{\mathsf{W}}
\def\sfX{\mathsf{X}}
\def\sfY{\mathsf{Y}}
\def\sfZ{\mathsf{Z}}


% sans serif lowercase
\def\bydef{:=}
\def\sfa{{\mathsf{a}}}
\def\sfb{{\mathsf{b}}}
\def\sfc{{\mathsf{c}}}
\def\sfd{{\mathsf{d}}}
\def\sfee{{\mathsf{e}}}
\def\sfff{{\mathsf{f}}}
\def\sfg{{\mathsf{g}}}
\def\sfh{{\mathsf{h}}}
\def\sfi{{\mathsf{i}}}
\def\sfj{{\mathsf{j}}}
\def\sfk{{\mathsf{k}}}
\def\sfl{{\mathsf{l}}}
\def\sfm{{\mathsf{m}}}
\def\sfn{{\mathsf{n}}}
\def\sfo{{\mathsf{o}}}
\def\sfp{{\mathsf{p}}}
\def\sfq{{\mathsf{q}}}
\def\sfr{{\mathsf{r}}}
\def\sfs{{\mathsf{s}}}
\def\sft{{\mathsf{t}}}
\def\sfu{{\mathsf{u}}}
\def\sfv{{\mathsf{v}}}
\def\sfw{{\mathsf{w}}}
\def\sfx{{\mathsf{x}}}
\def\sfy{{\mathsf{y}}}
\def\sfz{{\mathsf{z}}}
\def\sf0{{\mathsf{0}}}

\def\Nt{{N_t}}
\def\Nr{{N_r}}
\def\Ne{{N_e}}
\def\Ns{{N_s}}
\def\Es{{E_s}}
\def\No{{N_o}}
\def\sinc{\mathrm{sinc}}
\def\dmin{d^2_{\mathrm{min}}}
\def\vec{\mathrm{vec}~}
\def\kron{\otimes}
\def\Pe{{P_e}}
\newcommand{\expeq}{\stackrel{.}{=}}
\newcommand{\expg}{\stackrel{.}{\ge}}
\newcommand{\expl}{\stackrel{.}{\le}}
\def\SIR{{\mathsf{SIR}}}

% Added by Takao
\def\nn{\nonumber}




\begin{document}


\maketitle
\begin{abstract} The constrained version of the standard online convex optimization (OCO) framework, called COCO is considered, where on every round, a convex cost function and a convex constraint function are revealed to the learner after it chooses the action for that round.
The objective is to simultaneously minimize the static regret and cumulative constraint violation (CCV). 
An algorithm is proposed that guarantees a static regret of $O(\sqrt{T})$ and a CCV of $\min\{\cV, O(\sqrt{T}\log T) \}$, where $\cV$ depends on the distance between the consecutively revealed constraint sets, the shape of constraint sets, dimension of action space and the diameter of the action space. For special cases of constraint sets, $\cV=O(1)$. Compared to the state of the art results, static regret of $O(\sqrt{T})$ and CCV of $O(\sqrt{T}\log T)$, that were universal, the new result on CCV is instance dependent, which is derived by exploiting the geometric properties of the constraint sets.
\end{abstract}
\section{Introduction}

Large language models (LLMs) have achieved remarkable success in automated math problem solving, particularly through code-generation capabilities integrated with proof assistants~\citep{lean,isabelle,POT,autoformalization,MATH}. Although LLMs excel at generating solution steps and correct answers in algebra and calculus~\citep{math_solving}, their unimodal nature limits performance in plane geometry, where solution depends on both diagram and text~\citep{math_solving}. 

Specialized vision-language models (VLMs) have accordingly been developed for plane geometry problem solving (PGPS)~\citep{geoqa,unigeo,intergps,pgps,GOLD,LANS,geox}. Yet, it remains unclear whether these models genuinely leverage diagrams or rely almost exclusively on textual features. This ambiguity arises because existing PGPS datasets typically embed sufficient geometric details within problem statements, potentially making the vision encoder unnecessary~\citep{GOLD}. \cref{fig:pgps_examples} illustrates example questions from GeoQA and PGPS9K, where solutions can be derived without referencing the diagrams.

\begin{figure}
    \centering
    \begin{subfigure}[t]{.49\linewidth}
        \centering
        \includegraphics[width=\linewidth]{latex/figures/images/geoqa_example.pdf}
        \caption{GeoQA}
        \label{fig:geoqa_example}
    \end{subfigure}
    \begin{subfigure}[t]{.48\linewidth}
        \centering
        \includegraphics[width=\linewidth]{latex/figures/images/pgps_example.pdf}
        \caption{PGPS9K}
        \label{fig:pgps9k_example}
    \end{subfigure}
    \caption{
    Examples of diagram-caption pairs and their solution steps written in formal languages from GeoQA and PGPS9k datasets. In the problem description, the visual geometric premises and numerical variables are highlighted in green and red, respectively. A significant difference in the style of the diagram and formal language can be observable. %, along with the differences in formal languages supported by the corresponding datasets.
    \label{fig:pgps_examples}
    }
\end{figure}



We propose a new benchmark created via a synthetic data engine, which systematically evaluates the ability of VLM vision encoders to recognize geometric premises. Our empirical findings reveal that previously suggested self-supervised learning (SSL) approaches, e.g., vector quantized variataional auto-encoder (VQ-VAE)~\citep{unimath} and masked auto-encoder (MAE)~\citep{scagps,geox}, and widely adopted encoders, e.g., OpenCLIP~\citep{clip} and DinoV2~\citep{dinov2}, struggle to detect geometric features such as perpendicularity and degrees. 

To this end, we propose \geoclip{}, a model pre-trained on a large corpus of synthetic diagram–caption pairs. By varying diagram styles (e.g., color, font size, resolution, line width), \geoclip{} learns robust geometric representations and outperforms prior SSL-based methods on our benchmark. Building on \geoclip{}, we introduce a few-shot domain adaptation technique that efficiently transfers the recognition ability to real-world diagrams. We further combine this domain-adapted GeoCLIP with an LLM, forming a domain-agnostic VLM for solving PGPS tasks in MathVerse~\citep{mathverse}. 
%To accommodate diverse diagram styles and solution formats, we unify the solution program languages across multiple PGPS datasets, ensuring comprehensive evaluation. 

In our experiments on MathVerse~\citep{mathverse}, which encompasses diverse plane geometry tasks and diagram styles, our VLM with a domain-adapted \geoclip{} consistently outperforms both task-specific PGPS models and generalist VLMs. 
% In particular, it achieves higher accuracy on tasks requiring geometric-feature recognition, even when critical numerical measurements are moved from text to diagrams. 
Ablation studies confirm the effectiveness of our domain adaptation strategy, showing improvements in optical character recognition (OCR)-based tasks and robust diagram embeddings across different styles. 
% By unifying the solution program languages of existing datasets and incorporating OCR capability, we enable a single VLM, named \geovlm{}, to handle a broad class of plane geometry problems.

% Contributions
We summarize the contributions as follows:
We propose a novel benchmark for systematically assessing how well vision encoders recognize geometric premises in plane geometry diagrams~(\cref{sec:visual_feature}); We introduce \geoclip{}, a vision encoder capable of accurately detecting visual geometric premises~(\cref{sec:geoclip}), and a few-shot domain adaptation technique that efficiently transfers this capability across different diagram styles (\cref{sec:domain_adaptation});
We show that our VLM, incorporating domain-adapted GeoCLIP, surpasses existing specialized PGPS VLMs and generalist VLMs on the MathVerse benchmark~(\cref{sec:experiments}) and effectively interprets diverse diagram styles~(\cref{sec:abl}).

\iffalse
\begin{itemize}
    \item We propose a novel benchmark for systematically assessing how well vision encoders recognize geometric premises, e.g., perpendicularity and angle measures, in plane geometry diagrams.
	\item We introduce \geoclip{}, a vision encoder capable of accurately detecting visual geometric premises, and a few-shot domain adaptation technique that efficiently transfers this capability across different diagram styles.
	\item We show that our final VLM, incorporating GeoCLIP-DA, effectively interprets diverse diagram styles and achieves state-of-the-art performance on the MathVerse benchmark, surpassing existing specialized PGPS models and generalist VLM models.
\end{itemize}
\fi

\iffalse

Large language models (LLMs) have made significant strides in automated math word problem solving. In particular, their code-generation capabilities combined with proof assistants~\citep{lean,isabelle} help minimize computational errors~\citep{POT}, improve solution precision~\citep{autoformalization}, and offer rigorous feedback and evaluation~\citep{MATH}. Although LLMs excel in generating solution steps and correct answers for algebra and calculus~\citep{math_solving}, their uni-modal nature limits performance in domains like plane geometry, where both diagrams and text are vital.

Plane geometry problem solving (PGPS) tasks typically include diagrams and textual descriptions, requiring solvers to interpret premises from both sources. To facilitate automated solutions for these problems, several studies have introduced formal languages tailored for plane geometry to represent solution steps as a program with training datasets composed of diagrams, textual descriptions, and solution programs~\citep{geoqa,unigeo,intergps,pgps}. Building on these datasets, a number of PGPS specialized vision-language models (VLMs) have been developed so far~\citep{GOLD, LANS, geox}.

Most existing VLMs, however, fail to use diagrams when solving geometry problems. Well-known PGPS datasets such as GeoQA~\citep{geoqa}, UniGeo~\citep{unigeo}, and PGPS9K~\citep{pgps}, can be solved without accessing diagrams, as their problem descriptions often contain all geometric information. \cref{fig:pgps_examples} shows an example from GeoQA and PGPS9K datasets, where one can deduce the solution steps without knowing the diagrams. 
As a result, models trained on these datasets rely almost exclusively on textual information, leaving the vision encoder under-utilized~\citep{GOLD}. 
Consequently, the VLMs trained on these datasets cannot solve the plane geometry problem when necessary geometric properties or relations are excluded from the problem statement.

Some studies seek to enhance the recognition of geometric premises from a diagram by directly predicting the premises from the diagram~\citep{GOLD, intergps} or as an auxiliary task for vision encoders~\citep{geoqa,geoqa-plus}. However, these approaches remain highly domain-specific because the labels for training are difficult to obtain, thus limiting generalization across different domains. While self-supervised learning (SSL) methods that depend exclusively on geometric diagrams, e.g., vector quantized variational auto-encoder (VQ-VAE)~\citep{unimath} and masked auto-encoder (MAE)~\citep{scagps,geox}, have also been explored, the effectiveness of the SSL approaches on recognizing geometric features has not been thoroughly investigated.

We introduce a benchmark constructed with a synthetic data engine to evaluate the effectiveness of SSL approaches in recognizing geometric premises from diagrams. Our empirical results with the proposed benchmark show that the vision encoders trained with SSL methods fail to capture visual \geofeat{}s such as perpendicularity between two lines and angle measure.
Furthermore, we find that the pre-trained vision encoders often used in general-purpose VLMs, e.g., OpenCLIP~\citep{clip} and DinoV2~\citep{dinov2}, fail to recognize geometric premises from diagrams.

To improve the vision encoder for PGPS, we propose \geoclip{}, a model trained with a massive amount of diagram-caption pairs.
Since the amount of diagram-caption pairs in existing benchmarks is often limited, we develop a plane diagram generator that can randomly sample plane geometry problems with the help of existing proof assistant~\citep{alphageometry}.
To make \geoclip{} robust against different styles, we vary the visual properties of diagrams, such as color, font size, resolution, and line width.
We show that \geoclip{} performs better than the other SSL approaches and commonly used vision encoders on the newly proposed benchmark.

Another major challenge in PGPS is developing a domain-agnostic VLM capable of handling multiple PGPS benchmarks. As shown in \cref{fig:pgps_examples}, the main difficulties arise from variations in diagram styles. 
To address the issue, we propose a few-shot domain adaptation technique for \geoclip{} which transfers its visual \geofeat{} perception from the synthetic diagrams to the real-world diagrams efficiently. 

We study the efficacy of the domain adapted \geoclip{} on PGPS when equipped with the language model. To be specific, we compare the VLM with the previous PGPS models on MathVerse~\citep{mathverse}, which is designed to evaluate both the PGPS and visual \geofeat{} perception performance on various domains.
While previous PGPS models are inapplicable to certain types of MathVerse problems, we modify the prediction target and unify the solution program languages of the existing PGPS training data to make our VLM applicable to all types of MathVerse problems.
Results on MathVerse demonstrate that our VLM more effectively integrates diagrammatic information and remains robust under conditions of various diagram styles.

\begin{itemize}
    \item We propose a benchmark to measure the visual \geofeat{} recognition performance of different vision encoders.
    % \item \sh{We introduce geometric CLIP (\geoclip{} and train the VLM equipped with \geoclip{} to predict both solution steps and the numerical measurements of the problem.}
    \item We introduce \geoclip{}, a vision encoder which can accurately recognize visual \geofeat{}s and a few-shot domain adaptation technique which can transfer such ability to different domains efficiently. 
    % \item \sh{We develop our final PGPS model, \geovlm{}, by adapting \geoclip{} to different domains and training with unified languages of solution program data.}
    % We develop a domain-agnostic VLM, namely \geovlm{}, by applying a simple yet effective domain adaptation method to \geoclip{} and training on the refined training data.
    \item We demonstrate our VLM equipped with GeoCLIP-DA effectively interprets diverse diagram styles, achieving superior performance on MathVerse compared to the existing PGPS models.
\end{itemize}

\fi 

\section{COCO Problem}




On round $t,$ the online policy first chooses an admissible action $x_t \in \mathcal{X}\subset \bbR^d,$ 
 and then the adversary chooses a convex cost function $f_t: \mathcal{X} \to \mathbb{R}$ and a constraint of the form $g_{t}(x) \leq 0,$ where $g_{t}: \mathcal{X} \to \mathbb{R}$ is a convex function. Once the action $x_t$ has been chosen, we let $\nabla f_t(x_t)$ and full function $g_t$ or the set $\{x: g_t(x)\le 0\}$ to be revealed, as is standard in the literature.
 We now state the  standard assumptions made  in the  literature while studying the COCO problem \cite{guo2022online, yi2021regret, neely2017online, Sinha2024}.
\begin{assumption}[Convexity] \label{cvx}
$\mathcal{X} \subset \bbR^d$ is the admissible set that is closed, convex and has a finite Euclidean diameter $D$.  The cost function $f_t: \mathcal{X} \mapsto \mathbb{R}$ and the constraint function $g_{t}: \mathcal{X} \mapsto \mathbb{R}$ are convex for all $t\geq 1$.  
 %Moreover, $D$ is known ahead of time.
\end{assumption}
%\vspace{-0.18in}
\begin{assumption}[Lipschitzness] \label{bddness}
 %We have $\textrm{diam}(\mathcal{X}) \leq D, ||\nabla f_t(x)||_2 \leq G/2, \textrm{and}~ ||\nabla g_t(x))||_2 \leq G/2,~\forall t, \forall x\in \mathcal{X}$ for some finite constants $D$ and $G.$ If the functions are not necessarily differentiable, we require that the maximum magnitude of the subgradients be bounded accordingly.  Each
All cost functions $\{f_t\}_{t\geq 1}$ and the constraint functions $\{g_{t}\}_{ t\geq 1}$'s are $G$-Lipschitz, i.e., for any $x, y \in \mathcal{X},$ we have 
 \begin{eqnarray*}
 	|f_t(x)-f_t(y)| \leq G||x-y||,~
 	|g_{t}(x)-g_{t}(y)| \leq G||x-y||, ~\forall t\geq 1.
 \end{eqnarray*}
	\end{assumption}
 \begin{assumption}[Feasibility] \label{feas-constr}
With ${\cal G}_t = \{x\in \cX : g_t(x)\le 0\}$, we assume that  $\mathcal{X}^\star = \cap_{t=1}^T G_t  \neq \varnothing $.	
Any action $x^\star \in \cX^\star$ is defined to be feasible. %There exists some feasible action $x^\star \in \mathcal{X} $ s.t. $g_{t}(x^\star) \leq 0, \ \forall t \ \in T.$ The set of all feasible actions, denoted by $\mathcal{X}^\star,$ is called the feasible set. The feasibility assumption implies that $\mathcal{X}^\star \neq \emptyset.$
\end{assumption}
The feasibility assumption distinguishes the cost functions from the constraint functions and is common across all previous literature on COCO \cite{guo2022online, neely2017online, yu2016low,yuan2018online,yi2023distributed, georgios-cautious,Sinha2024}. 

%In light of Assumption \ref{feas-constr}, including multiple constraints at each time is straightforward because of the existence of a common feasible point for all $g_t, t=1, \dots, T$.

For any real number $z$, we define $(z)^+ \equiv \max(0,z).$ Since $g_{t}$'s are revealed after the action $x_t$ is chosen, any online policy need not necessarily take feasible actions on each round. 
 Thus in addition to the static\footnote{ The static-ness refers to the fixed benchmark using only one action $x^\star$ throughout the horizon of length $T$}  regret defined below
%\vspace{-0.1in}
\begin{eqnarray} \label{regret-def}
	\textrm{Regret}_{[1:T]} \equiv \sup_{\{f_t\}_{t=1}^T} \sup_{x^\star \in \mathcal{X}^\star} \textrm{Regret}_{[1:T]}(x^\star), \end{eqnarray}
	where $\textrm{Regret}_{[1:T]}(x^\star) \equiv \sum_{t=1}^T f_t(x_t) - \sum_{t=1}^T f_t(x^\star)$, 
an additional obvious metric of interest is  the total cumulative constraint violation (CCV) defined as 
 %\vspace{-0.1in}
  \begin{eqnarray} \label{gen-oco-goal}
 	\textrm{CCV}_{[1:T]}  = \sum_{t=1}^T (g_{t}(x_t))^+. 
	\end{eqnarray}
	 Under the standard assumption (Assumption \ref{feas-constr}) that $\mathcal{X}^\star$ is not empty, the goal is to design an online policy to simultaneously achieve a small regret \eqref{intro-regret-def} with $x^\star \in \mathcal{X}^\star$ and a small CCV \eqref{intro-gen-oco-goal}. We refer to this problem as the constrained OCO (COCO). 
	 
For simplicity, we define set 
\begin{equation}\label{defn:S}
 \ S_t = \cap_{\tau=1}^t {\cal G}_\tau,
\end{equation}
where $G_t$ is as defined in Assumption \ref{feas-constr}.
All ${\cal G}_t$'s are convex and consequently, all $S_t$'s are convex and are nested, i.e. $S_t\subseteq S_{t-1}$. Moreover, because of Assumption \ref{feas-constr},  each $S_t$ is non-empty and in particular $\cX^\star\in S_t$ for all $t$. After action $x_t$ has been chosen, set $S_t$ controls the constraint violation, which can be used to write an upper bound on the $\textrm{CCV}_{[1:T]}$ as follows.

\begin{definition}
For a convex set $\chi$ and a point $x\notin \chi$, 
$$\text{dist}(x,\chi) = \min_{y\in \chi} || x-y||.$$
\end{definition}

Thus, the constraint violation at time $t$, 
\begin{equation}\label{eq:distviolationrelation}
(g_t(x_t))^+ \le G\text{dist}(x_t,S_t), \ \text{and} \  \textrm{CCV}_{[1:T]}  \le G\sum_{t=1}^T \text{dist}(x_t,S_t),
\end{equation}
where $G$ is the common Lipschitz constants for all $g_t$'s.


	 
	 
\section{Algorithm from \cite{Sinha2024}}
The best known algorithm (Algorithm \ref{coco_sinha}) to solve COCO \cite{Sinha2024} was shown to have the following guarantee. 
\begin{theorem}\label{thm:sinha2024}[\cite{Sinha2024}]
Algorithm \ref{coco_sinha}'s $\textrm{Regret}_{[1:T]} = O(\sqrt{T})$ and  $\textrm{CCV}_{[1:T]} = O(\sqrt{T}\log T)$ when $f_t,g_t$ are convex.
\end{theorem}
We next show that in fact the analysis of \cite{Sinha2024} is tight for the CCV even when $d=1$ and $f_t(x)=f(x)$ and $g_t(x) =g(x)$ for all $t$. 
With finite diameter $D$ and the fact that any $x^\star \in \cX^\star$ belongs to all nested convex bodies $S_t$'s, when $d=1$, one expects that the CCV for any algorithm in this case will be $O(D)$. However, we as we show next,  Algorithm \ref{coco_sinha} does not effectively make use of geometric constraints imposed by nested convex bodies $S_t$'s.

\begin{algorithm}[tb]
   \caption{Online Algorithm from  \cite{Sinha2024}}
   \label{coco_sinha}
\begin{algorithmic}[1]
   \State {\bfseries Input:} Sequence of convex cost functions $\{f_t\}_{t=1}^T$ and constraint functions $\{g_t\}_{t=1}^T,$ $G=$ a common Lipschitz constant, $T=$ Horizon length,
   %an upper bound $G$ to the Euclidean norm of their (sub)-gradients, 
    $D=$ Euclidean diameter of the admissible set $\mathcal{X},$ $\mathcal{P}_\mathcal{X}(\cdot)=$ Euclidean projection oracle on the set $\mathcal{X}$ 
     \State {\bfseries Parameter settings:} 
     \begin{enumerate}
     	\item \textbf{Convex cost functions:} $\beta = (2GD)^{-1}, V=1, \lambda = \frac{1}{2\sqrt{T}}, \Phi(x)= \exp(\lambda x)-1.$
     
    \item \textbf{$\alpha$-Strongly convex cost functions:} $\beta =1, V=\frac{8G^2 \ln(Te)}{\alpha}, \Phi(x)= x^2.$
    \end{enumerate}
     %$ \alpha=\frac{1}{2GD}, n=\max(2, \lceil \ln T \rceil), V=(n-1)^{n-1}T^{\frac{n-1}{2}}, \Phi(x)=x^n.$ 
%   \REPEAT
  \State {\bfseries Initialization:} Set $ x_1={\bf 0}, \text{CCV}(0)=0$.
   \State {\bf For} $t=1:T$
   %\For{$t=1:T$}
   \State \quad Play $x_t,$ observe $f_t, g_t,$ incur a cost of $f_t(x_t)$ and constraint violation of $(g_t(x_t))^+$
   \State \quad $\tilde{f}_t \gets \beta f_t, \tilde{g}_t \gets \beta \max(0,g_t).$
   \State \quad $\text{CCV}(t)=\text{CCV}(t-1)+\tilde{g}_t(x_t).$
   \State \quad Compute $\nabla_t = \nabla \hat{f}_t(x_t),$ where $\hat{f}_t(x):= V\tilde{f}_t(x)+ \Phi'(\text{CCV}(t)) \tilde{g}_t(x), ~~ t \geq 1$.
   \State \quad $x_{t+1} = \mathcal{P}_\mathcal{X}(x_t - \eta_t \nabla_t)$, where 
   \quad \begin{eqnarray*}
   \eta_t =\begin{cases}
   	\frac{\sqrt{2}D}{2\sqrt{\sum_{\tau=1}^{t} ||\nabla_\tau||_2^2}}, ~&~\textrm{for convex costs} \\
   	\frac{1}{\sum_{s=1}^t H_s}, ~ &~ \textrm{for strongly convex costs } (H_t \textrm{ is the strong convexity parameter of } f_t). 
   	\end{cases}
   	\end{eqnarray*}
   	
%   \IF{$x_i > x_{i+1}$}
%   \STATE Swap $x_i$ and $x_{i+1}$
%   \STATE $noChange = false$
%   \ENDIF
   %\EndFor
   \State {\bf EndFor}
%   \UNTIL{$noChange$ is $true$}
\end{algorithmic}
\end{algorithm}




\begin{lemma}\label{lem:algwc}
Even when $d=1$ and $f_t(x)=f(x)$ and $g_t(x) =g(x)$ for all $t$, for Algorithm \ref{coco_sinha}, its $\textrm{CCV}_{[1:T]}  = \Omega(\sqrt{T} \log T)$.
\end{lemma}
\begin{proof}
{\bf Input:} Consider $d=1$, and let $\cX=[1, a], a>2$. Moreover, let $f_t(x)=f(x)$ and $g_t(x) =g(x)$ for all $t$. Let $f(x) = c x^2$ for some (large) $c>0$ and $g(x)$ be such that $G=\{x: g(x)\le 0\} \subseteq [a/2, a]$ and let $|\nabla g(x)|\le1$ for all $x$.

Let $1< x_1 < a/2$. Note that $\text{CCV}(t)$ (defined in Algorithm \ref{coco_sinha}) is a non-decreasing function, and let $t^\star$ be the earliest time $t$ such that $\Phi'(\text{CCV}(t)) \nabla g(x) <- c $. 
For $f(x) = c x^2$, $\nabla f(x) \ge c$ for all $x>1$. 
Thus, using Algorithm \ref{coco_sinha}'s definition, it follows that for all $t\le t^\star$, $x_t < a/2$, since the derivative of $f$ dominates the derivative of $\Phi'(\text{CCV}(t))  g(x)$ until then.


Since  $\Phi(x)= \exp(\lambda x)-1$ with $\lambda = \frac{1}{2\sqrt{T}}$, and by definition $|\nabla g(x)| \le 1$ for all $x$, thus, we have that by time $t^\star$, 
$\textrm{CCV}_{[1:t^\star]} = \Omega(\sqrt{T} \log T)$. Therefore, $\textrm{CCV}_{[1:T]} =\Omega(\sqrt{T} \log T)$.

 %Moreover, Algorithm \ref{coco_sinha}'s actions $x_t$'s keeps oscillating around $a/2$ after  time $t^\star$ leading to $\Omega(\sqrt{T})$ regret.
 \end{proof}
 
Essentially, Algorithm \ref{coco_sinha} is treating minimizing the $\text{CCV}$ problem as regret minimization for function $g$ similar to function $f$ and this leads to its CCV of $\Omega(\sqrt{T}\log T)$.
For any given input instance with $d=1$, an alternate algorithm that  chooses its actions following online gradient descent (OGD) projected on to the most recently revealed feasible set $S_t$ achieves $O(\sqrt{T})$ regret (irrespective of the starting action $x_1$) and $O(D)$ $\text{CCV}$ (since any $x^\star \in S_t$ for all $t$).  We extend this intuition in the next section, and present an algorithm that tries to exploit the geometry of the nested convex sets $S_t$ for general $d$.






%\input{NCBC}
\subsection{$d$-Set Semi-Bandits}
\label{sec:ds}
\begin{algorithm}[t]
    \LinesNumbered
    \SetAlgoLined
    \caption{DS-BARBAT: d-Sets-BARBAT}
    \label{algs:DS-BARBAT}
    
    \textbf{Initialization:} 
    Set the initial round \( T_0 = 0 \), \( \Delta_k^0 = 1 \), and \( r_k^0 = 0 \) for all \( k \in [K] \).

    \For{epochs $m = 1,2,\cdots$}{
        Set $\zeta_m \leftarrow (m + 4)2^{2(m+4)}\ln (K)$, and $\delta_m \leftarrow 1/(K\zeta_m)$
        
        Set $\lambda_m \leftarrow 2^8 \ln{\left(4K / \delta_m\right)}$ and $\beta_m \leftarrow \delta_m / K$.
        
        Set $n_k^m = \lambda_m (\Delta_k^{m-1})^{-2}$ for all arms $k \in [K]$.
        
        Set $N_m \leftarrow \lceil K \lambda_m 2^{2(m-1)}/d \rceil$ and $T_m \leftarrow T_{m-1} + N_m$.

        Select the arm subsets \( \cK_m = \mathop{\arg\max}_{k \in [K]} r_k^{m-1} \) with $|\cK_m| = d$. 

        Set
        $   
            \widetilde{n}_k^m = \begin{cases}
                n_k^m & k \not\in \cK_m \\
                N_m - \sum_{k \not\in k_m}n_k^m / d & k \in \cK_m
            \end{cases}
        $.

        \For{$t = T_{m-1} + 1$ to $T_m$}{
            Choose arm $I_t\sim p_m$ where $p_m(k)= \widetilde{n}_k^m / N_m$.

            Observe the corrupted reward $\widetilde{r}_{t,I_t}$ and update the total reward $S_{I_t}^m = S_{I_t}^m + \widetilde{r}_{t,I_t}$.
        }

        Set $r_k^m \leftarrow \min \{S_k^m / \widetilde{n}_k^m, 1\}$.

        Set $r_*^m \leftarrow \top(d)_{k \in [K]}\left\{r_k^m - \sqrt{\frac{4\ln(4/\beta_m)}{\widetilde{n}_k^m}}\right\}$, where $\top(d)$
        represents that the $d$ largest number.
        
        Set $\Delta_k^m \leftarrow \max\{2^{-m}, r_*^m - r_k^m\}$.
    }
\end{algorithm}

% In this paper, we only consider the d-sets setting, which is a special case of semi-bandits. In d-sets setting, the agent needs to select arbitrary $d$ arms in each round, which means that we need to consider $d$ arms instead of just one arm.

% In previous works~\citep{wei2018more,zimmert2019beating,ito2021hybrid,tsuchiya2023further}, the algorithms require explicitly computing the arm-selection probability distribution over all arms. Although the computational complexity is typically $O(K)$ in many cases, this can become problematic in d-sets setting, where the number of possible actions grows exponentially in $K$.

% Motivated by this issue, we extend BARBAT to \emph{DS-BARBAT}, presented in Algorithm~\ref{algs:DS-BARBAT}. In this setting, the agent observes the rewards of $d$ arms for each pull. Following a similar approach to MA-BARBAT, we set $N_m \leftarrow \lceil K \lambda_m 2^{2(m-1)} / d \rceil$.
% However, unlike MAB in which the optimal action is a single arm, here we consider a $d$-set setting in which the optimal action is a subset of $d$ arms. Consequently, we choose $\cK_m$ to be the $d$ arms with the highest empirical rewards in the previous epoch. When estimating the suboptimality gap, we similarly focus on the $d$ largest empirical values, reflecting the fact that all $d$ selected arms can be viewed as optimal.

% We show that the DS-BARBAT algorithm have the following regret bound. The proof is given in Appendix \ref{ape:ds}.
% \begin{theorem}
% \label{the:ds-erb}    % The expected regret of Algorithm DS-BARBAT 
%     Following \cite{zimmert2019beating}, let $\mu_1 \geq \mu_2 \geq \cdots \geq \mu_K$ be the ordering of mean rewards for the $K$ arms and $\Delta_k = \mu_k - \mu_d$ for all arms $k > d$. The expected regret of DS-BARBAT satisfies
%     \[R(T) = O\left(dC + \sum_{k=d+1}^{K}\frac{\log(T)\log(KT)}{\Delta_{k}} + \frac{K\log(1/\Delta)\log(K/\Delta)}{\Delta}\right).\]
% \end{theorem}
% \begin{remark}
%     Unlike previous works~\citep{wei2018more,zimmert2019beating,ito2021hybrid,tsuchiya2023further}, DS-BARBAT does not require computing the arm-selection probability in each round, thereby reducing the computational complexity to $O(K\log(T))$, which is very low.
% \end{remark}

In this paper, we focus on the $d$-sets setting, which is a special case of semi-bandits where the agent must select $d$ arms in each round. Specifically, let $d \in \{1, 2, \ldots, K - 1\}$ be a fixed parameter, and define the $d$-sets as
\[
\mathcal{X} = \left\{ \mathbf{x} \in \{0, 1\}^K \mid \sum_{k=1}^K x_k = d \right\},
\]
where $x_k = 1$ indicates that arm $k$ is selected, and $x_k = 0$ indicates that arm $k$ is not selected.
Following~\citep{zimmert2019beating,dann2023blackbox}, let $\mu_1 \geq \mu_2 \geq \cdots \geq \mu_K$ and $\Delta_k = \mu_k - \mu_d$ for all arms $k > d$.

%In previous works~\citep{wei2018more,zimmert2019beating,ito2021hybrid,tsuchiya2023further}, algorithms require explicitly computing the arm-selection probability distribution over all arms. While the computational complexity is typically $O(K)$ in many cases, this becomes problematic in the d-sets setting, where the number of possible actions grows exponentially with $K$.

We extend BARBAT to DS-BARBAT, presented in Algorithm~\ref{algs:DS-BARBAT}. In this setting, the agent observes the rewards of $d$ arms per pull. Similar to MA-BARBAT, we set $N_m \leftarrow \lceil K \lambda_m 2^{2(m-1)} / d \rceil$. However, unlike the traditional MAB where the optimal action is a single arm, here the optimal action is a subset of $d$ arms. Thus, we choose $\cK_m$ to be the $d$ arms with the highest empirical rewards in the previous epoch. When estimating the suboptimality gap, we focus on the $d$ largest empirical values, reflecting the fact that all $d$ selected arms are considered optimal.

The regret bound for DS-BARBAT is as follows, with the proof provided in Appendix~\ref{ape:ds}:
\begin{theorem}
\label{the:ds-erb}
The expected regret of DS-BARBAT satisfies
\[
\BE\left[R(T)\right] = O\left(dC + \sum_{k=d+1}^{K}\frac{\log(T)\log(KT)}{\Delta_{k}} + \frac{dK\log\left(1/\Delta\right)\log\left(K/\Delta\right)}{\Delta}\right).
\]
\end{theorem}


    Notice that our DS-BARBAT algorithm can efficient compute the sampling probability $p_m$, while FTRL-based methods~\citep{wei2018more,zimmert2019beating,ito2021hybrid,tsuchiya2023further} need to solve a complicated convex optimization problem in each round, which is rather expensive.



\section{Bounding the Total Movement Cost $M_T$ \eqref{defn:totalmovementcost1}}



%The movement cost $M$ is a general distance measure that counts the maximum total distance 
%that can be traversed when projections are taken from arbitrary points between nested convex sets, and is typically expected to be much larger than the actual violation $\sum_{\tau=1}^t \text{dist}(x_{\tau}, S_{\tau})$ that Algorithm \ref{coco_alg_1} will incur. 
%In Theorem \ref{thm:tvmonotone}, we saw that if the projections satisfy the monotonicity property, then the total CCV can be bounded independent of $T$ only in terms of $d$ and $D$. In general, however, Algorithm \ref{coco_alg_1} can have arbitrary projections. Thus, we upper bound $M$ itself, which when multiplied by $G$ is an upper bound on the CCV of Algorithm \ref{coco_alg_1}.

We start by considering two simple cases where bounding $M_T$ is easy.
 \begin{lemma}\label{lem:spheres}
If all nested convex bodies $S_1\supseteq S_2  \supseteq \dots \supseteq S_T$ are spheres then $M_T\le d^{3/2}D$.
\end{lemma}

\begin{proof}
Recall the definition that $x_t\in \partial S_{t-1}, b_t=\cP_{S_{t}}(x_t)\in S_t$ from \eqref{defn:genconvxmovement}.
Let $||x_t-b_t||=r$, then since all $S_t$'s are spheres, at least along one of the $d$-orthogonal canonical basis vectors, $\text{diameter}(S_{t})\le \text{diameter}(S_{t-1}) - \frac{r}{\sqrt{d}}$. Since the diameter along any of the $d$-axis is $D$, we get the answer.
\end{proof}

%Similar results can be shown for nice convex bodies, such as parallel cuboids. 
 \begin{lemma}\label{lem:square}
If all nested convex bodies $S_1\supseteq S_2  \supseteq \dots \supseteq S_T$ are cuboids that are axis parallel to each other, then $M\le d^{3/2}D$.
\end{lemma}
Proof is identical to Lemma \ref{lem:spheres}.
Note that similar results can be obtained when $S_t$'s are regular polygons that are axis parallel with each other.


 
 After exhausting the universal results for an upper bound on $M_T$ for `nice' nested convex bodies, we next give a general bound on $M_T$ for any sequence of nested convex bodies which depends on the geometry of the nested convex bodies (instance dependent).
 To state the result we need the following preliminaries.
 
 Following \eqref{defn:genconvxmovement}, $b_t=\cP_{S_t}(x_t)$ where $x_t\in \partial S_{t-1}$. Without loss of generality, 
 $x_t\notin S_{t}$ since otherwise the distance $||x_t-b_t||=0$.
 Let $m_t$ be the mid-point of $x_t$ and $b_t$, i.e. $m_t = \frac{x_t+b_t}{2}$.
 \begin{definition}\label{defn:anglewidth}
Let the convex hull of $m_t \cup S_{t}$ be $\cC_t$.
 Let $w_t$ be a unit vector such that there exists $c_t>0$ such that the cone 
 $$C_{w_t}(c_t) = \left\{z\in \bbR^d: -w_t^T\frac{(z-m_t)}{||(z-m_t)||} \ge c_t\right\}$$ contains $\cC_t$. Since $S_{t}$ is convex, such $w_t, c_t>0$ exist. For example, $w_t=b_t-x_t$ is one such choice for which $c_t>0$ since $m_t \notin S_t$.
See Fig. \ref{fig:anglewidthmain} for a pictorial representation.
 
 Let $c^\star_{w_t,t} = \arg \min_{c_t} C_{w_t}(c_t)$,
 $c^\star_t = \min_{w_t} c^\star_{w_t,t}$, and $w_t^\star= \arg \min_{w_t} c^\star_{w_t,t}$.
 Moreover, let $c^\star = \min_t  c^\star_t$, where by definition, $c^\star <1$. 
 \end{definition}
 
% \begin{figure}[]
%  \begin{center}
%%\begin{tikzpicture}
%\begin{tikzpicture}[scale=1.5, dot/.style={circle,inner sep=1pt,fill,label={#1},name=#1},
%  extended line/.style={shorten >=-#1,shorten <=-#1},
%  extended line/.default=1cm]
%% Define coordinates
%\coordinate (w_t) at (0,0);
%\coordinate (z_t) at (0,-.5);
%\coordinate (x_t) at (-1,-1);
%\coordinate (y_t) at (1,0);
%\coordinate (C_top) at (1,2.15);
%\coordinate (C_bottom) at (3.55,-0.55);
%
%
%% Draw object (shaded region)
%%\draw[thick, gray, fill=pink!50] (C_top) to[out=-.5,in=-15] (C_bottom) -- cycle;
%
%
%
%
%
%% Draw projected image (S_t+1)
%\draw[thick, cyan, fill=cyan!50,opacity=0.75] (1,0) -- (1.5,1.5) -- (3,1) -- (3.2,0.15)  -- cycle;
%\draw[thick, blue!10, fill=blue!10,opacity=0.5] (z_t) --(1.5,1.495) --  (y_t)   -- cycle;
%\draw[thick, blue!10, fill=blue!10,opacity=0.5] (z_t) --  (y_t)  -- (3.2,0.15) -- cycle;
%
%\filldraw[blue] (y_t) circle (2pt) node[below right] {$b_t$};
%\filldraw[black] (z_t) circle (2pt) node[below right] {$m_t$};
%% Draw camera and viewing ray
%%\begin{scope}[on background layer]
%\draw[thick,->] (x_t) -- (y_t);
%%\end{scope}
%\draw[dashed,->] (.65,0) -- (z_t) -- (-0.65,-1)  node[below right] {$w_t$};
%\draw[dashed,->] (.65,0) -- (z_t) -- (-0.65,-1)  node[below right] {$w_t$};;
%%\draw[dashed,->] (.65,0) -- (x_t) -- (-0.65,-1)  node[below right] {$w_t$};;
%
%\coordinate (u_1) at (0,.5);
%\coordinate (u_2) at (0,-2);
%
%%\draw [ dashed, <-] (u_1) -- (z_t) -- (u_2) node[above left] {$H_u$};
%%\draw [ dashed, ->] (1.3,-.7) -- (z_t) -- (-2.8,.2) node[above left] {$u$};
%
%%\draw [extended line, ->] ($(u_1)!(z_t)!(u_2)$) -- (-.5,-0.05) node[above left ] {$u$};
%
%%\draw [extended line,<-] ($(-0.25,.5)!(z_t)!(.25,-2)$) node[above left] {$u$};
%
%%\tkzDefLine[orthogonal=through (z_t)](x_t,y_t);
%
%
%%\filldraw[black] (w_t) circle (2pt) node[above left] {$w_t$};
%\filldraw[black] (x_t) circle (2pt) node[below left] {$a_t$};
%
%%\begin{scope}[on background layer]
%\draw[thick, gray, fill=gray!50,opacity=0.76] (C_top) to[out=.5,in=25] (C_bottom) -- cycle;
%%\end{scope}
%\draw[thick, gray, fill=gray!50,opacity=0.76] (C_bottom) to[out=-155,in=-165] (C_top) -- cycle;
%\node at (2.25,0.75) {$S_{t}$};
%
%% Draw lines connecting camera to object and projected image
%\draw[dotted] (z_t) -- (1.5,1.5);
%\draw[dotted] (z_t) -- (3.2,0.15);
%%\draw[dotted] (w_t) -- (3,1);
%%\draw[dotted] (w_t) -- (2.5,0.5);
%%\draw[dotted] (w_t) -- (1.5,0.5);
%%\begin{scope}[on background layer]
%\draw[gray!30,  thick,fill=gray!50,opacity=0.6] (z_t) -- (C_top) -- (C_bottom) -- (z_t);
%%\end{scope}
%%\draw[dotted] (z_t) -- (C_bottom);
%
%% Add labels for object and camera
%\node[above right] at (C_top) {$C_{w_t}(c_t)$};
%%\begin{axis}[
%%  axis lines=center,
%%  axis on top,
%%  %xlabel={$x$}, ylabel={$y$}, zlabel={$t$},
%%  domain=1:1,
%%  y domain=0:2*pi,
%%  xmin=-1.5, xmax=1.5,
%%  ymin=-1.5, ymax=1.5, zmin=0.0,
%%        %every axis x label/.style={at={(rel axis cs:0,0.5,0)},anchor=south},
%%        %every axis y label/.style={at={(rel axis cs:0.5,0,0)},anchor=north},
%%        %every axis z label/.style={at={(rel axis cs:0.5,0.5,0.9)},anchor=west},
%%  samples=30]
%%  \addplot3 [surf, colormap/blackwhite, shader=flat] ({x*cos(deg(y))},{x*sin(deg(y))},{x});
%%    
%%    %\addplot3[surf, samples=20, domain=-1:1, y domain=-1:1] {x^2 + y^2}; % Example surface plot
%%
%%\end{axis}
%\end{tikzpicture}
%\caption{Figure representing the cone $C_{w_t}(c_t)$ that contains the convex hull of $m_t$ and $S_{t}$ with unit vector $w_t$.} 
%%$u$ is a unit vector perpendicular to $H_u$ an hyperplane that is a supporting hyperplane $C_t$ at $m_t$ such that $\cC_t \cap H_u = \{z_t\}$ and 
%%$u^T (a_t-m_t)\ge 0$ }
%\label{fig:anglewidthmain}
%\end{center}
%\end{figure}

\begin{figure*}
\begin{center}
\includegraphics[width=10cm,keepaspectratio,angle=0]{Fig-ConeDef.png}
\caption{Figure representing the cone $C_{w_t}(c_t)$ that contains the convex hull of $m_t$ and $S_{t}$ with unit vector $w_t$.} 
\label{fig:anglewidthmain}
\end{center}
\end{figure*}


 
Essentially, $2\cos^{-1}(c^\star_t)$ is the angle width of $\cC_t$ with respect to $w_t^\star$, i.e. each element of $\cC_t$ makes an  angle of at most $ \cos^{-1}(c^\star_t)$ with $w_t^\star$.  
 
 

%\begin{rem} Instead of 
%\end{rem}
 
% \begin{rem} Theorem \ref{thm:manselli} obtains a universal result on $M$ by exploiting the self-expanded property that implies that $c^\star_t \ge \frac{1}{\sqrt{d}}$ (VI \cite{Manselli}) independent of the 
% shape of the convex bodies. Even when $x_t$ and $x_{t+1}$ are arbitrary points from $S_{t-1}$ and $S_{t}$ as is the considered case, it is still true that each sub-curve $(x_t, y_t)$ has the self-expanded property with respect to all the sub-curves $(x_\tau, y_\tau), \ \tau\ge t$ but not the whole curve $x_1,y_1, x_2,y_2, \dots, x_t,y_t,x_{t+1},y_{t+1}, \dots$.
% \end{rem}

\begin{rem}\label{rem:cbound}
Note that $c_t^\star$ is only a function of the distance $ ||x_t-b_t||$   and the shape of $S_t$'s, in particular, the maximum width of $S_t$ 
 along the directions perpendicular to vector $x_t-b_t$ $\forall \ t$ which can be at most the diameter $D$. 
 $c_t^\star$ decreases (increasing the ``width" of cone $C_{w_t^\star}(c_t^\star)$) as $||x_t-b_t||$ decreases, but small $x_t-b_t$ also implies small  violation at time $t$ from \eqref{defn:genconvxmovement}.
Across time slots, $d_{\min} = \min_t ||x_t-b_t||$ and shape of $S_t$'s control $c^\star$, where $d_{\min} > 0$ is inherent from the definition of $c^\star$ since a bound on $||x_t-b_t||$ is only needed for the case when $x_t\ne b_t$. 
\end{rem} 
 %Interestingly, if $d_{\min}$ is small then $c^\star$ but then $\text{CCV}_{[1:t]}$ in \eqref{defn:genconvx

\begin{rem}  Projecting $x_t\in \partial S_{t-1}$ onto $S_t$ to get $b_t=\cP_{S_t}(x_t)$, the diameter of $S_t$ is at most diameter of $S_{t-1} - ||x_t-b_t||$, however, only along the direction $b_t-x_t$. Since the shape of $S_{t}$ is arbitrary, as a result, the diameter of $S_t$ need not be smaller than the diameter of $S_{t-1}$ along any pre-specified direction, which was the main idea used to derive Lemma \ref{lem:spheres}.  Thus, to relate the distance $||x_t-b_t||$ with the decrease in the diameter of the convex bodies $S_t$'s, we use the concept of {\bf mean width} of a convex body that is defined as the expected width of the convex body along all the directions that are chosen uniformly randomly (formal definition is provided in Definition \ref{defn:avgwidth}).
\end{rem}

Next, we upper bound $M_T$ by connecting the distance $||x_t-b_t||$ to the decrease in mean width (to be defined ) of convex bodies $S_{t-1}$ and $S_t$'s.
 
 \begin{lemma}\label{lem:movementcost}
 The total movement cost 
$M_T$ in \eqref{defn:totalmovementcost1} is at most $$\frac{2V_d(d-1)}{V_{d-1}} \left(\frac{1}{c^\star}\right)^{d}D,$$ where 
$V_d$ is the $(d-1)$-dimensional Lebesgue measure of  the unit sphere in $d$ dimensions. 
\end{lemma}
%By definition, $c^\star <1$. Thus as $c^\star$ decreases, $M_T$ increases exponentially with exponent being $d$. 

Note that $V_d/V_{d-1} = O(1/\sqrt{d})$.
 Thus, we get the following {\bf main result} of the paper for Algorithm \ref{coco_alg_1} combining Lemma \ref{lem:regretbound} and Lemma \ref{lem:movementcost}. 
 \begin{theorem}\label{thm:main1}
For solving COCO, Algorithm \ref{coco_alg_1} has  $$\textrm{Regret}_{[1:T]} = O(\sqrt{T}), \ \text{and} \ \text{CCV}_{[1:T]}= O\left(\sqrt{d} \left(\frac{1}{c^\star}\right)^{d}D\right).$$
\end{theorem}

%\begin{rem} Specializing $a_t=x_t$ in \eqref{defn:genconvxmovement}, where $x_t$ is the action chosen by Algorithm \ref{coco_alg_1}, we will get a better bound on the $\text{CCV}_{[1:T]}$ via improved bound for $c^\star$, however, that will be both algorithm and instance dependent.
%\end{rem}
Compared to all prior results on COCO, that were universal (instance independent), where the best known one \cite{Sinha2024} has $\textrm{Regret}_{[1:T]} = O(\sqrt{T})$, and $\text{CCV}_{[1:T]}=O(\sqrt{T}\log T)$, Theorem \ref{thm:main1} is an instance 
dependent result for the CCV. In particular, it exploits the geometric structure of the nested convex sets $S_t$'s and 
derives an upper bound on the CCV that only depends on the `shape' of $S_t$'s. It can be the case that the instance is `badly' behaved and $c^\star$ is very small or dependent on $T$. If that is the case, in Section \ref{sec:algswitch} we show how to limit the CCV to $O(\sqrt{T}\log T)$. However, when $S_t$'s are `nice', e.g., $c^\star$ is independent of $T$ (Remark \ref{rem:cbound}) or $S_t$'s are spheres or axis parallel cuboids (Lemma \ref{lem:spheres} and \ref{lem:square}), the $\text{CCV}$ of Algorithm \ref{coco_alg_1} is independent of $T$, which is a fundamentally improved result compared to large body of prior work. In fact, in prior work this was largely assumed to be not possible. In particular, before the result of \cite{Sinha2024}, achieving simultaneous $\textrm{Regret}_{[1:T]} = O(\sqrt{T})$, and $\text{CCV}_{[1:T]}=O(\sqrt{T})$ itself was the final goal.




\section{Algorithm $\mathrm{Switch}$}\label{sec:algswitch}

Since Theorem \ref{thm:main1} provides an instance dependent bound on the CCV, that is a function of $c^\star$ which can be small, it can be the case that its CCV is larger than $O(\sqrt{T}\log T)$, thus providing a result that is inferior to that of Algorithm \ref{coco_sinha} \cite{Sinha2024}. Thus, next, we marry the two algorithms, Algorithm \ref{coco_sinha} and Algorithm \ref{coco_alg_1}, in Algorithm \ref{alg:switch} to provide a best of both results as follows. 


\begin{algorithm}[tb]
   \caption{$\mathrm{Switch}$}
   \label{alg:switch}
\begin{algorithmic}[1]
   \State {\bfseries Input:} Sequence of convex cost functions $\{f_t\}_{t=1}^T$ and constraint functions $\{g_t\}_{t=1}^T,$ $G=$ a common Lipschitz constant,  $d$ dimension  of the admissible set $\mathcal{X},$
   %an upper bound $G$ to the Euclidean norm of their (sub)gradients, 
    $D=$ Euclidean diameter of the admissible set $\mathcal{X},$ $\mathcal{P}_\mathcal{X}(\cdot)=$ Euclidean projection operator on the set $\mathcal{X}$,      \State {\bfseries Initialization:} Set $ x_1 \in \mathcal{X}$ arbitrarily, $\text{CCV}(0)=0$.
   \State {\bf For} \ {$t=1:T$}
   \State \quad {\bf If} {$\text{CCV}(t-1) \le \sqrt{T}\log T$}
   \State \quad \quad Follow Algorithm \ref{coco_alg_1}
   \State  \quad  \quad $\text{CCV}(t)=\text{CCV}(t-1)+\max\{g_t(x_t),0\}.$
   \State \quad {\bf Else} 
   \State \quad \quad Follow Algorithm \ref{coco_sinha} with resetting $\text{CCV}(t-1)=0$
   \State \quad {\bf EndIf} 
   \State {\bf EndFor}
\end{algorithmic}
\end{algorithm}

 \begin{theorem}\label{thm:main2}
$\mathrm{Switch}$ (Algorithm \ref{alg:switch}) has regret $\textrm{Regret}_{[1:T]} =O(\sqrt{T})$, while    $$\text{CCV}_{[1:T]}=
\min\left\{O\left(\sqrt{d} \left(\frac{1}{c^\star}\right)^{d}D\right), O(\sqrt{T}\log T)\right\}.$$
\end{theorem}
 
Algorithm $\mathrm{Switch}$ should be understood as the best of two worlds algorithm, where the two worlds corresponds to one having nice convex sets $S_t$'s that have CCV independent of $T$ or $o(\sqrt{T})$ for 
Algorithm \ref{coco_alg_1}, while in the other, CCV of Algorithm \ref{coco_alg_1} is large on its own, and the overall CCV is controlled by discontinuing the use of Algorithm \ref{coco_alg_1} once its CCV reaches $\sqrt{T}\log T$ and switching to Algorithm \ref{coco_sinha} thereafter that has universal guarantee of $O(\sqrt{T}\log T)$ on its CCV.
 
 After exhausting the general results on the CCV of Algorithm \ref{coco_alg_1}, we next consider the special case of $d=2$ and when the sets $S_t$ have a special structure defined by their projection hyperplanes. 
 Note that it is highly non-trivial to bound the CCV of Algorithm \ref{coco_alg_1} even when $d=2$.
 
  
 %\begin{remark}
 %\end{remark}

\section{Special case of $d=2$}
In this section, we show that if $d=2$ (all convex sets $S_t$'s lie in a plane) and the projections satisfy a monotonicity property depending on the problem instance, then we can bound the total CCV for Algorithm \ref{coco_alg_1} independent of the time horizon $T$ and consequently getting a $O(1)$ CCV. 




Recall from the definition of Algorithm \ref{coco_alg_1}, $y_t = \cP_{S_{t-1}}(x_t - \eta_t \nabla f_t(x_t))$ and 
$x_{t+1} = \cP_{S_t}(y_t)$.


\begin{definition}\label{defn:projhyperplane}
Let the hyperplane perpendicular to line segment $(y_t, x_{t+1})$ passing through $x_{t+1}$ be 
$F_t$. Without loss of generality, we let $y_t \notin S_t$, since then the projection is trivial. Essentially $F_t$ is the projection hyperplane at time $t$. %Let $N_t$ be the normal to $F_t$, then we
Let $\cH_t^+$ denote the positive half plane corresponding to $F_t$, i.e., 
$\cH_t^+ = \{z: z^T (y_t-x_{t+1})\ge 0\}$. 
Refer to Fig. \ref{fig:defF}.
Let the angle between $F_1$ and $F_t$ be $\theta_t$. 
\end{definition}
\begin{figure}


\includegraphics[width=10cm,keepaspectratio,angle=0]{FigDefF.pdf}


\caption{Definition of $F_t$'s.}
\label{fig:defF}
\end{figure}


\begin{definition}\label{defn:anglemonotone}
The instance $S_1 \supseteq S_2 \supseteq \dots \supseteq S_T$ is defined to be monotonic 
if $\theta_2 \le \theta_3 \le \dots \le \theta_T$.
\end{definition}




\begin{theorem}\label{thm:tvmonotone}
For $d=2$ when the instance is monotonic, $\text{CCV}_{[1:T]}$ for Algorithm \ref{coco_alg_1} is at most $O(GD)$.
\end{theorem}

Theorem \ref{thm:tvmonotone} provides a universal guarantee on the CCV of  Algorithm \ref{coco_alg_1} that is independent of the problem instance (as long as it is monotonic) unlike Lemma \ref{lem:movementcost}, even though it applies only for $d=2$. The proof is derived by using basic convex geometry results from \cite{Manselli} in combination with exploiting the definition of Algorithm \ref{coco_alg_1} and the monotonicity condition. It is worth noting that even under the monotonicity assumption it is non-trivial to upper bound the CCV since the successive angles made by $F_t$ with $F_1$ can increase arbitrarily slowly, making it difficult 
to control the total CCV. 
%It shows that the COCO problem when has structure can be exploited to get $O(1)$ CCV 
\section{OCS Problem}
In \cite{Sinha2024}, a special case of COCO, called the OCS problem, was introduced where $f_t\equiv0$ for all $t$. Essentially, with OCS, only constraint satisfaction is the objective.  In \cite{Sinha2024}, Algorithm \ref{coco_sinha} was shown to have CCV of $O(\sqrt{T}\log T)$. Next, we show that Algorithm \ref{coco_alg_1} has CCV of $O(1)$ for the OCS, a remarkable improvement.

 \begin{theorem}\label{thm:ocs}
For solving OCS, Algorithm \ref{coco_alg_1} has $\text{CCV}_{[1:T]}= O\left(d^{d/2} D\right)$.
\end{theorem}

As discussed in \cite{Sinha2024}, there are important applications of OCS, and it is important to find tight bounds on its CCV. Theorem \ref{thm:ocs} achieves this by showing that CCV of $O(1)$ can be achieved, where the constant depends only on the dimension of the action space and the diameter. This is a fundamental improvement compared to the CCV bound of $O(\sqrt{T}\log T)$ from \cite{Sinha2024}. Theorem  \ref{thm:ocs} is derived by using the connection between the curve obtained by successive projections on nested convex sets and self-expanded curves (Definition \ref{defn:se-curve}) and then using a classical result on self-expanded curves from \cite{Manselli}.

 
\chapter{\textcolor{black}{Conclusion}}
\label{ch: Conclusion}
\thispagestyle{plain}

In this thesis, the potential of \gls{sc} and \gls{goc} paradigms within modern digital networks has been explored and exploited. The rapid proliferation of data driven technologies such as the \gls{iot}, autonomous vehicles and smart cities has underscored the limitations of traditional bit-centric communication systems. These systems, grounded in Shannon's information theory, focus primarily on the accurate transmission of raw data without considering the contextual significance of the information being conveyed. This fundamental mismatch between data production and communication infrastructure capabilities has necessitated the exploration of more efficient and intelligent communication frameworks.

\cref{ch: SEMCOM} discussed how the core of this thesis focused on integrating of \gls{sc} principles with generative models and their potential applications in the context of edge computing. By focusing on the conveyance of relevant meaning rather than exact data reproduction, \gls{sc} reduces unnecessary bandwidth consumption and inefficiencies. In all those cases where it is possible and reasonable to discuss the semantics, then the faithful representation of the original data is unnecessary as long as the meaning has been conveyed. This paradigm also aligns with \gls{goc}, where the transmitted data is tailored to meet specific objectives, further reducing the communication overhead. The goal of the communication can either be the classical syntactic data transmission or the semantic preservation of the data. By focusing on the goal of the communication, it is possible to transmit only the most pertinent information, thereby reducing the load in communication networks and optimizing resource utilization.

In \cref{ch: SPIC}, the \gls{spic} framework was introduced as a novel method for semantic-aware image compression. The framework demonstrated the potential for high-fidelity image reconstruction from compressed semantic representations. The proposed modular transmitter-receiver architecture is based on a doubly conditioned \gls{ddpm} model, the \gls{semcore}, specifically designed to perform \gls{sr} under the conditioning of the \gls{ssm}. By doing so the reconstructed images preserve their semantic features at a fraction of the \gls{bpp} compared to classical methods such as \gls{bpg} and \gls{jpeg2000}.

Furthermore, the enhancement introduced by \gls{cspic} addressed a critical aspect in image reconstruction: the accurate representation of small and detailed objects. Without requiring extensive retraining of the underlying \gls{semcore} model, \gls{cspic} improved the preservation of important semantic classes, such as traffic signs.  The modular design at the core of the \gls{spic} and \gls{cspic} showcased the flexibility and adaptability of the system in different contexts.

The integration of \gls{sc} principles continued in \cref{ch: SQGAN}, where the \gls{sqgan} model was proposed. This architecture employed vector quantization in tandem with a semantic-aware masking mechanism, enabling selective transmission of semantically important regions of the image and the \gls{ssm}. By prioritizing critical semantic classes and utilizing techniques such as Semantic Relevant Classes Enhancement or the Semantic-Aware discriminator, the model excelled at maintaining high reconstruction quality even at very low bit rates, further emphasizing the efficiency gains of the proposed approach.

Finally, in \cref{ch: Goal_oriented}, the thesis was extended to include the \gls{goc} for resource allocation in \glspl{en}. By adopting the \gls{ib} principle to perform \gls{goc} was developed a framework to dynamically adjust compression and transmission parameters based on network conditions and resource constraints. This dynamic adaptation was crucial in balancing compression efficiency with semantic preservation, optimizing the use of computational and communication resources in edge networks.

By leveraging the \gls{sqgan} within the \gls{en}, the research demonstrated the synergy between \gls{sc} and \gls{goc}. Real-time network conditions informed adjustments to the masking process, enabling the edge network to operate autonomously and efficiently. This approach validated the potential of \gls{sgoc} to enhance resource utilization in modern network infrastructures.



% In this thesis, the potential of \gls{sc} and \gls{goc} paradigms within modern digital networks has been explored and exploited. The rapid proliferation of data driven technologies such as the \gls{iot}, autonomous vehicles, and smart cities has underscored the limitations of traditional bit-centric communication systems. These systems, grounded in Shannon's information theory, focus primarily on the accurate transmission of raw data without considering the contextual significance of the information being conveyed. This fundamental mismatch between data production and communication infrastructure capabilities has necessitated the exploration of more efficient and intelligent communication frameworks.

% As explained in \cref{ch: SEMCOM} at the core of this thesis lies the integration of \gls{sc} principles with generative models, particularly within the context of edge computing. \gls{sc}, which emphasizes the conveyance of meaning rather than mere symbol reconstruction, offers a pathway to significantly reduce bandwidth usage and enhance the efficiency of data transmission. This approach aligns seamlessly with the objectives of \gls{goc}, which prioritizes the transmission of information that is directly relevant to achieving specific goals. By focusing on the semantic content of the data, it becomes possible to transmit only the most pertinent information, thereby reducing the load in  communication networks and optimizing resource utilization.

% In \cref{ch: SPIC} the development and implementation of the \gls{spic} framework marked a significant stride in bridging \gls{sc} with practical image compression techniques. By leveraging diffusion models, \glspl{ddpm} were employed to reconstruct high-resolution images from compressed semantic representations. This modular approach, consisting of a transmitter and receiver architecture, facilitated the efficient encoding and decoding of both the low-resolution original image and the associated \gls{ssm}. The \gls{spic} framework demonstrated the capability to maintain high levels of semantic preservation while achieving substantial compression rates, thereby showcasing its potential as a viable alternative to classical image compression algorithms such as \gls{bpg} and \gls{jpeg2000}.

% Building upon the foundational work of \gls{spic}, the introduction of the \gls{cspic} further refined the approach by addressing the reconstruction of small and detailed objects within images. This enhancement was achieved without necessitating additional fine-tuning or retraining of the underlying \gls{semcore} model, thereby exploiting the framework's modularity and flexibility. The \gls{cspic} model underscored the importance of preserving critical semantic classes, ensuring that essential details (i.e. "traffic signs") are preserved. These level of semantic preservation was evaluated by the Traffic signs classification accuracy presented in \sref{sec: GM evaluation metrics}.

% In \cref{ch: SQGAN} the \gls{sqgan} model represented a novel integration of vector quantization and \gls{sc} principles. The \gls{sqgan} architecture incorporated a \gls{samm} to selectively transmit semantically relevant regions of the data. This selective encoding process significantly reduced redundancy and enhanced communication efficiency, particularly at extremely low \gls{bpp} values. The introduction of the \gls{samm} and the \gls{spe} facilitated the prioritization of latent vectors associated with critical semantic classes, thereby improving the overall reconstruction quality of important objects within images. Additionally, the designed Semantic Relevant Classes Enhancement data augmentation technique and the Semantic Aware Discriminator further refined the model's ability to preserve critical semantic information.

% In \cref{ch: Goal_oriented} asignificant contribution of this research was the exploration of goal-oriented resource allocation within \glspl{en}. By leveraging the \gls{ib} principle, the thesis addressed the challenge of dynamically adjusting compression parameters to balance the trade-off between compression efficiency and semantic preservation. The application of stochastic optimization techniques facilitated the optimal allocation of computational and communication resources, ensuring that the \gls{en} operates efficiently under varying network conditions and resource constraints. This integration of \gls{goc} principles with resource optimization strategies underscored the importance of adaptive and intelligent network management in modern communication infrastructures.

% Additionally, by employing the \gls{sqgan} model within the \gls{en} framework, the research demonstrated the potential of \gls{sgoc}. The integration of the \gls{sqgan} model within the \gls{en} architecture enabled the dynamic adjustment of the masking fractions based on real-time network conditions and resource availability. This approach ensured that the \gls{en} could autonomously optimize its operations, thereby enhancing communication efficiency and resource utilization in a goal-oriented fashion with focus on \gls{sc}.

% Throughout the research, the importance of modular and flexible framework design was emphasized. The proposed models, \gls{spic}, \gls{cspic}, and \gls{sqgan}, were designed to be easily integrated into existing communication systems without the need for extensive modifications. This design philosophy ensures that the advancements in \gls{sc} can be readily adopted in practical applications, facilitating the transition from traditional to intelligent communication paradigms.

% The comparative analysis of the proposed models against classical compression algorithms highlighted the superiority of semantic-aware approaches in preserving critical information at low bit rates. While traditional algorithms excel in minimizing pixel-level distortions, they fall short in maintaining the semantic integrity of the data. In contrast, the proposed semantic and goal-oriented models demonstrated enhanced performance in preserving meaningful content, thereby offering a more effective solution for applications where semantic accuracy is crucial.


%\bibliographystyle{IEEEtran}
%\bibliography{../IEEEabrv,../Research}
\bibliography{OCO.bib} 
\newpage
\section{Proof of Lemma \ref{lem:regretbound}}
\begin{proof}
From the convexity of $f_t$'s, for $x^\star$ satisfying Assumption \eqref{feas-constr}, we have 
$$f_t(x_t) - f_t(x^\star) \le \nabla f_t^T (x_t-x^\star).$$
From the choice of Algorithm \ref{coco_alg_1} for $x_{t+1}$, we have 
\begin{align*} ||x_{t+1}-x^\star||^2 & = || \cP_{S_{t}}(y_t) - x^\star||^2 \\
& \stackrel{(a)}\le || y_t - x^\star||^2, \\
& = ||\cP_{S_{t-1}}\left(x_t - \eta_t \nabla f_t(x_t)\right)-x^\star||^2, \\
& \stackrel{(n)}\le || (x_t-\eta_t\nabla f_t^T(x_t)) - x^\star||^2,
\end{align*}
where inequalities $(a)$ and $(b)$ follow since $x^\star \in S_t$ for all $t$.
Hence
\begin{align*}
 ||x_{t+1}-x^\star||^2 & \le  ||x_t-x^\star||^2 + \eta_t^2||\nabla f_t(x_t)||^2 - 2\eta_t \nabla f_t^T(x_t)(x_t-x^\star), \\
 \nabla f_t^T(x_t)(x_t-x^\star) & \le \frac{||x_t-x^\star||^2-||x_{t+1}-x^\star||^2 }{\eta_t} + \eta_t G^2.
\end{align*}
Summing this over $t=1$ to $T$, we get 
\begin{align*}
2\sum_{t=1}^T (f_t(x_t) - f_t(x^\star)) & \le \sum_{t=1}^T\nabla f_t^T (x_t-x^\star), \\
& \le \sum_{t=1}^T  \frac{||x_t-x^\star||^2-||x_{t+1}-x^\star||^2 }{\eta_t} + \sum_{t=1}^T\eta_t G^2, \\
& \le D^2 \frac{1}{\eta_T} + G^2 \sum_{t=1}^T\eta_t,\\
& \le O( DG \sqrt{T}),
\end{align*}
where the final inequality follows by choosing $\eta_t = \frac{D}{G\sqrt{t}}$.
\end{proof}
\section{Proof of Theorem \ref{lem:movementcost}}
\begin{proof}
We need the following preliminaries.

 \begin{definition}\label{defn:avgwidth}
Let $K$ be a non-empty convex bounded set in $\bbR^d$. Let $u$ be a unit vector, and $\ell_u$ a line through the origin parallel to $u$. 
Let $K_u$ be the orthogonal projection of $K$ onto $\ell_u$, with length $|K_u|$. The mean width of $K$ is defined as 
\begin{equation}\label{eq:projlength}
W(K) = \frac{1}{V_d} \int_{\bbS_1^d} |K_u| du,
\end{equation}
where $\bbS_1^d$ is the unit sphere in $d$ dimensions and $V_d$ its $(d-1)$-dimensional Lebesgue measure.
\end{definition}


The following is immediate. 
\begin{equation}\label{eq:WBound1}
0\le W(K) \le \text{diameter}(K).
\end{equation}

\begin{lemma}\label{lem:width2D}\cite{eggleston1966convexity}
For $d=2$, $$W(K)=\frac{\text{Perimeter}(K)}{\pi}.$$
\end{lemma}
Lemma \ref{lem:width2D} implies that $W(K) \ne W(K_1) + W(K_2)$ even if $K_1\cup K_2=K$ and $K_1\cap K_2=\phi$.




Recall from \eqref{defn:genconvxmovement} that $x_t\in  \partial S_{t-1}$ and $b_t$ is the projection of $x_t$ onto $S_{t}$, and $m_t$ is the mid-point of $x_t$ and $b_t$, i.e. $m_t = \frac{x_t+b_t}{2}$. Moreover, the convex sets $S_t$'s are nested, i.e., $S_1\supseteq S_2 \supseteq \dots \supseteq S_T$.
To prove Theorem \ref{lem:movementcost} we will bound the rate at which $W(S_t)$ (Definition \ref{defn:avgwidth}) decreases as a function of the length $||x_t-b_t||$. 

From Definition \ref{defn:anglewidth}, recall that $\cC_t$ is the convex hull of $m_t\cup S_{t}$. We also need to define $\cC_t^-$ as the convex hull of $x_t\cup S_{t}$. Since 
$S_t \subseteq \cC_t$ and $\cC_t^- \subseteq S_{t-1}$ (since $S_{t-1}$ is convex and $x_t\in S_{t-1}$), we have \begin{equation}\label{eq:widthlb}
W(S_{t}) - W(S_{t-1}) \le W(\cC_t) - W(\cC_t^-).
\end{equation} 
\begin{definition}\label{}
$\Delta_t = W(\cC_t) - W(\cC_t^-)$.
\end{definition}
%and derive a bound on $\Delta_t$.



The main ingredient of the proof is the following Lemma that bounds  $\Delta_t$ whose proof is provided after completing the proof of Theorem \ref{lem:movementcost}.
\begin{lemma}\label{lem:derivative} $$\Delta_t  \le -V_{d-1}\frac{||x_t-b_t||}{2V_d(d-1)}  (c_t^\star)^{d},$$
where $c_t^\star$ has been defined in Definition \ref{defn:anglewidth}.
\end{lemma}


Recalling that $c^\star = \min_t  c^\star_t$ from Definition \ref{defn:anglewidth}, and  combining  Lemma \ref{lem:derivative} with \eqref{eq:WBound1} and \eqref{eq:widthlb}, we get that 
$$\sum_{t=1}^T ||x_t-b_t|| \le \frac{2V_d(d-1)}{V_{d-1}} \left(\frac{1}{c^\star}\right)^{d}\text{diameter}(S_1),$$
since $S_1\supseteq S_2 \supseteq \dots \supseteq S_T$. Recalling that $\text{diameter}(S_1)\le D$, Theorem \ref{lem:movementcost} follows.
\end{proof}
%Note that $b_t$ being a projection of $a_t$ onto 
%$S_{t}$, 

\begin{proof}[Proof of Lemma \ref{lem:derivative}]
% \begin{figure}[]
%  \begin{center}
%%\begin{tikzpicture}
%\begin{tikzpicture}[scale=1.5, dot/.style={circle,inner sep=1pt,fill,label={#1},name=#1},
%  extended line/.style={shorten >=-#1,shorten <=-#1},
%  extended line/.default=1cm]
%% Define coordinates
%\coordinate (w_t) at (0,0);
%\coordinate (z_t) at (0,-.5);
%\coordinate (x_t) at (-1,-1);
%\coordinate (y_t) at (1,0);
%\coordinate (C_top) at (1,2.15);
%\coordinate (C_bottom) at (3.55,-0.55);
%
%
%% Draw object (shaded region)
%%\draw[thick, gray, fill=pink!50] (C_top) to[out=-.5,in=-15] (C_bottom) -- cycle;
%
%
%
%
%
%% Draw projected image (S_t+1)
%\draw[thick, cyan, fill=cyan!50,opacity=0.75] (1,0) -- (1.5,1.5) -- (3,1) -- (3.2,0.15)  -- cycle;
%\draw[thick, blue!10, fill=blue!10,opacity=0.5] (z_t) --(1.5,1.495) --  (y_t)   -- cycle;
%\draw[thick, blue!10, fill=blue!10,opacity=0.5] (z_t) --  (y_t)  -- (3.2,0.15) -- cycle;
%
%\filldraw[blue] (y_t) circle (2pt) node[below right] {$b_t$};
%\filldraw[black] (z_t) circle (2pt) node[below right] {$m_t$};
%% Draw camera and viewing ray
%\draw[thick,->] (x_t) -- (y_t);
%\draw[dashed,->] (.65,0) -- (z_t) -- (-0.65,-1)  node[below right] {$w_t$};
%\draw[dashed,->] (.65,0) -- (z_t) -- (-0.65,-1)  node[below right] {$w_t$};;
%%\draw[dashed,->] (.65,0) -- (x_t) -- (-0.65,-1)  node[below right] {$w_t$};;
%
%\coordinate (u_1) at (0,.5);
%\coordinate (u_2) at (0,-2);
%
%\draw [ dashed, <-] (-1,.5) -- (x_t) -- (-1,-2) node[above left] {$H_u'$};
%\draw [ dashed, <-] (u_1) -- (z_t) -- (u_2) node[above left] {$H_u$};
%\draw [ dashed, ->] (1.3,-.7) -- (z_t) -- (-2.8,.2) node[above left] {$u$};
%
%%\draw [extended line, ->] ($(u_1)!(z_t)!(u_2)$) -- (-.5,-0.05) node[above left ] {$u$};
%
%%\draw [extended line,<-] ($(-0.25,.5)!(z_t)!(.25,-2)$) node[above left] {$u$};
%
%%\tkzDefLine[orthogonal=through (z_t)](x_t,y_t);
%
%
%%\filldraw[black] (w_t) circle (2pt) node[above left] {$w_t$};
%\filldraw[black] (x_t) circle (2pt) node[below left] {$a_t$};
%
%\draw[thick, gray, fill=pink!50,opacity=0.6] (C_top) to[out=.5,in=25] (C_bottom) -- cycle;
%
%\draw[thick, gray, fill=gray!50,opacity=0.6] (C_bottom) to[out=-155,in=-165] (C_top) -- cycle;
%\node at (2.25,0.75) {$S_{t}$};
%
%% Draw lines connecting camera to object and projected image
%\draw[dotted] (z_t) -- (1.5,1.5);
%\draw[dotted] (z_t) -- (3.2,0.15);
%%\draw[dotted] (w_t) -- (3,1);
%%\draw[dotted] (w_t) -- (2.5,0.5);
%%\draw[dotted] (w_t) -- (1.5,0.5);
%\draw[dotted] (z_t) -- (C_top);
%\draw[dotted] (z_t) -- (C_bottom);
%
%% Add labels for object and camera
%\node[above right] at (C_top) {$C_{w_t}(c_t)$};
%\end{tikzpicture}
%\caption{Figure representing the cone $C_{w_t}(c_t)$ that contains the convex hull of $m_t$ and $S_{t}$ with respect to the unit vector $w_t$. $u$ is a unit vector perpendicular to $H_u$ an hyperplane that is a supporting hyperplane $C_t$ at $m_t$ such that $\cC_t \cap H_u = \{m_t\}$ and 
%$u^T (a_t-m_t)\ge 0$ }
%\label{fig:anglewidth}
%\end{center}
%\end{figure}

\begin{figure*}
\begin{center}
\includegraphics[width=10cm,keepaspectratio,angle=0]{FigConeDefHyperplane.png}
\caption{Figure representing the cone $C_{w_t}(c_t)$ that contains the convex hull of $m_t$ and $S_{t}$ with respect to the unit vector $w_t$. $u$ is a unit vector perpendicular to $H_u$ an hyperplane that is a supporting hyperplane $C_t$ at $m_t$ such that $\cC_t \cap H_u = \{m_t\}$ and 
$u^T (x_t-m_t)\ge 0$ }
\label{fig:anglewidth}
\end{center}
\end{figure*}



Let $H_u$ be the hyperplane perpendicular to vector $u$.
 Let $\cU_0$ be the set of unit vectors $u$ such that hyperplanes $H_u$ are supporting hyperplanes to $\cC_t$ at point $m_t$ such that $\cC_t \cap H_u = \{m_t\}$ and 
$u^T (x_t-m_t)\ge 0$.  See Fig. \ref{fig:anglewidth} for reference.

 Since $b_t$ is a projection of $x_t$ onto $S_{t}$, and $m_t$ is the mid-point of $x_t,b_t$, for $u\in \cU_0$, the hyperplane $H_u'$ containing $x_t$ and parallel to $H_u$ is a supporting hyperplane for $\cC_t^-$.  


Thus, using the definition of $K_u$ from \eqref{eq:projlength},
\begin{equation}\label{eq:dummy1}
\Delta_t  \le \frac{1}{V_d} \int_{\cU_0} (|\cC_{t,u}| - |\cC_{t,u}^-|) du= -\frac{||x_t-b_t||}{2V_d} \int_{\cU_0} u^T 
\frac{(x_t-m_t)}{||x_t-m_t||}  \ du,
\end{equation}
since $||x_t-m_t|| = ||x_t-b_t||/2$.

Recall the definition of $C_{w_t^\star}(c_t^\star)$ from Definition \ref{defn:anglewidth} which implies that the convex hull of $m_t$ and $S_{t}$, $\cC_t$ is contained in $C_{w_t^\star}(c_t^\star)$.
Next, we consider $\cU_1$ the set of unit vectors $u$ such that hyperplanes $H_u$ are supporting hyperplanes to $C_{w_t^\star}(c_t^\star)$ at point $m_t$ 
such that $u^T (x_t-m_t)\ge 0$. 
By definition $\cC_t\subseteq C_{w_t^\star}(c_t^\star)$, it follows that 
$\cU_1\subset \cU_0$.

Thus, from \eqref{eq:dummy1}
 \begin{equation}\label{eq:dummy2}
\Delta_t  \le -\frac{||x_t-b_t||}{2V_d} \int_{\cU_1} u^T. \frac{(x_t-m_t)}{||x_t-m_t||} du
\end{equation}

Recalling the definition of $w_t^\star$ (Definition \ref{defn:anglewidth}), 
vector $u\in \cU_1$ can be written as 
$$ u = \lambda u_{\perp} + \sqrt{1-\lambda^2} w_t^\star,$$
where $u_{\perp}^T w_t^\star=0$, $|u_{\perp}|=1$ and since $u\in \cU_1$
$$0 \le \lambda=\sqrt{1-(u^Tw_t^\star)} = u^Tu_{\perp}\le c_t^\star.$$

Let $\cS_{\perp} = \{u_\perp: |u_\perp|=1 , u_\perp^T w_t^\star=0\}$. Let $du_{\perp}$ be the 
$(n-2)$-dimensional Lebesgue measure of $\cS_{\perp}$. 

It is easy to verify that 
$du = \lambda^{d-2}(1-\lambda^2)^{-1/2} d\lambda du_{\perp}$ and hence from \eqref{eq:dummy2}

\begin{equation}\label{eq:dummy3}
\Delta_t  \le -\frac{||x_t-b_t||}{V_d} \int_{0}^{c_t^\star}  \lambda^{d-2}(1-\lambda^2)^{-1/2} d\lambda \int_{\cS_{\perp}} (\lambda u_{\perp} + \sqrt{1-\lambda^2} w_t^\star)^T \frac{(x_t-m_t)}{||x_t-m_t||}  du_{\perp}.
\end{equation}
 
 Note that $\int_{du_{\perp}} u_{\perp} du_{\perp}=0$. Thus,
 \begin{align}\nn \label{}
\Delta_t  & = -\frac{||x_t-b_t||}{2V_d} \frac{(w_t^\star)^T(x_t-m_t)}{||x_t-m_t||}  \int_{0}^{c_t^\star}  \lambda^{d-2}(1-\lambda^2)^{-1/2}   \sqrt{1-\lambda^2}\ d\lambda   \int_{\cS_{\perp}} du_{\perp},\\ 
\nn
& \stackrel{(a)}\le -V_{d-1} \frac{||x_t-b_t||}{2V_d}  \frac{ (w_t^\star)^T(x_t-m_t)}{||x_t-m_t||}  \int_{0}^{c_t^\star} \lambda^{d-2}\ d\lambda, \\ \nn
& \stackrel{(b)}\le  -V_{d-1}\frac{||x_t-b_t||}{2V_d(d-1)} c_t^\star (c_t^\star)^{d-1},\\ 
& = -V_{d-1}\frac{||x_t-b_t||}{2V_d(d-1)}  (c_t^\star)^{d},
\end{align}
where $(a)$ follows since  $\int_{\cS_{\perp}} du_{\perp} = V_{d-1}$ by definition, $(b)$ follows since $ \frac{(w_t^\star)^T(x_t-m_t)}{||x_t-m_t||} \ge c_t^\star$ from Definition \ref{defn:anglewidth}.

%Clearly, $W(S_{t} - W(S_t) \ge \Delta_t$. 
%We will bound $\Delta_t $ to get a bound on the movement cost $M$ \eqref{eq:totalDistance}


 \end{proof}
 
 %Note that the proof is conceptually similar to the proof of Theorem \ref{thm:manselli} in \cite{Manselli}, 
 %and has been suitably modified to fit the considered problem.
\section{Proof of Theorem \ref{thm:main2}}
 \begin{proof}
 Since $\text{CCV}(t)$ is a monotone non-decreasing function, let $t_{\min}$ be the largest time until which Algorithm \ref{coco_alg_1} is followed by $\mathrm{Switch}$.
 The regret guarantee is easy to prove.  From Theorem \ref{thm:main1},
 regret until time $t_{\min}$ is at most $O(\sqrt{t_{\min}})$. Moreover, starting from time $t_{\min}$ till $T$, from Theorem \ref{thm:sinha2024}, the regret of Algorithm \ref{coco_sinha} is at most $O(\sqrt{T-t_{\min}})$. Thus, the overall regret for $\mathrm{Switch}$ is at most $O(\sqrt{T})$.
 
 For the CCV, with $\mathrm{Switch}$, until time $t_{\min}$, $\text{CCV}(t_{\min})\le \sqrt{T}\log T$. At 
 time $t_{\min}$, $\mathrm{Switch}$ starts to use Algorithm \ref{coco_sinha} which has the following appealing property from (8) \cite{Sinha2024} that for any $t\ge t_{\min}$ where at time $t_{\min}$  Algorithm \ref{coco_sinha} was started to be used with resetting $\text{CCV}(t_{\min})=0$. 
 For any $t\ge t_{\min}$
 \begin{eqnarray} \label{gen-fn-ineq}
		\Phi(\text{CCV}(t)) +\textrm{Regret}_t(x^\star) \leq \sqrt{\sum_{\tau=t_{\min}}^t \big(\Phi'(\text{CCV}(\tau))\big)^2} + \sqrt{t-t_{\min}}.
\end{eqnarray}
where $\beta = (2GD)^{-1}, V=1, \lambda = \frac{1}{2\sqrt{T}}, \Phi(x)= \exp(\lambda x)-1, $ and $\lambda=\frac{1}{2\sqrt{T}}$.
We trivially have $\textrm{Regret}_t(x^\star)\geq -\frac{Dt}{2D} \geq -\frac{t}{2}.$ Hence, from \eqref{gen-fn-ineq}, we have that for any $\lambda = \frac{1}{2\sqrt{T}}$ and any $t \ge t_{\min}$
$$\text{CCV}_{[t_{\min},T]} \leq 4GD\ln(2\big(1+2T)\big)\sqrt{T}.$$
Since as argued before, with $\mathrm{Switch}$,  $\text{CCV}(t_{\min})\le \sqrt{T}\log T$, we get that  $\text{CCV}_{[1:T]}\le O(\sqrt{T}\log T)$.
 \end{proof}
%\section{Proof of Theorem \ref{thm:tvmonotone}}
\section{Preliminaries for Bounding the CCV in Theorem \ref{thm:ocs} and Theorem \ref{thm:tvmonotone}}
% 
% \begin{definition}\label{defn:avgwidth}
%Let $K$ be a non-empty convex bounded set in $\bbR^d$. Let $u$ be a unit vector, and $\ell_u$ a line through the origin parallel to $u$. 
%Let $K_u$ be the orthogonal projection of $K$ onto $\ell_u$, with length $|K_u|$. The mean width of $K$ is defined as 
%\begin{equation}\label{eq:projlength}
%W(K) = \frac{1}{V_d} \int_{S_1^d} |K_u| du,
%\end{equation}
%where $S_1^d$ is the unit sphere in $d$ dimensions and $V_d$ its $(d-1)$-dimensional Lebesgue measure.
%\end{definition}
%
%
%The following is immediate. 
%\begin{equation}\label{eq:WBound1}
%0\le W(K) \le \text{diameter}(K).
%\end{equation}
%
%
%\begin{lemma}\label{lem:width2D1}
%For $d=2$, $$W(K)=\frac{\text{Perimeter}(K)}{\pi}.$$
%\end{lemma}

Let $K_1, \dots, K_T$ be nested (i.e., $K_1 \supseteq K_2 \supset K_3 \supseteq \dots \supseteq K_T$) bounded convex subsets of $\bbR^d$. 

%Let the minimum distance between $K_i$ and $K_{i+1}$ be $d_{i,i+1}$ and 
%\begin{equation}
%d_{\min} = \min_i d_{i,i+1}.
%\end{equation}
\begin{definition}\label{defn:projectioncurve}
If $\sigma_1\in K_1$, and $\sigma_{t+1} = \cP_{K_{t+1}}(\sigma_t)$, for $t=1, \dots, T$. Then the curve 
$${\underline \sigma}= \{(\sigma_1,\sigma_2), (\sigma_2,\sigma_3), \dots, (\sigma_{T-1},\sigma_T)\}$$ is called the projection curve on $K_1, \dots, K_T$.
\end{definition}
%Let $x_t \in \partial K_t$ and $y_t$ be the projection of $x_t$ on set $K_{t+1}$.  


We are interested in upper bounding the quantity 
\begin{equation}\label{eq:totalDistance}
\Sigma = \max_{{\underline \sigma}} \sum_{t=1}^{T-1} ||\sigma_t - \sigma_{t+1}||.
\end{equation}
%If $x_t\in K_{t+1}$ then $||x_t - y_t||=0$.

 \begin{lemma}\label{lem:projection}
For a projection curve ${\underline \sigma}$, $\Sigma \le d^{d/2} \text{diameter}(K_1)$.
\end{lemma}


To prove the result we need the following definition.

\begin{definition}\label{defn:se-curve} A curve $\gamma: I \rightarrow \bbR^d$  is called self-expanded, if for every $t$ where 
$\gamma'(t)$ exists, we have 
$$< \gamma'(t), \gamma(t)-\gamma(u)> \ \ge 0$$ for all $u\in I$ with $u \le t$, where $<.,.>$ represents the inner product. 
In words, what this means is that $\gamma$ starting in a point $x_0$ is self expanded, if for every $x\in \gamma$ for which there exists the tangent line $\sfT$, the arc (sub-curve) $(x_0, x)$ is
contained in one of the two half-spaces, bounded by the hyperplane through
$x$ and orthogonal to $\sfT$. 
\end{definition}
For self-expanded curves the following classical result is known.
\begin{theorem}\label{thm:manselli}\cite{Manselli}
For any self-expanded curve $\gamma$ belonging to a closed bounded convex set of $\bbR^d$ with diameter $D$, its total length is at most $O(d^{d/2} D)$.
\end{theorem}
\begin{proof}[Proof of Lemma \ref{lem:projection}]
From Definition \ref{defn:projectioncurve}, the projection curve is 
$${\underline \sigma}=\{(\sigma_1,\sigma_2), (\sigma_2,\sigma_3), \dots, (\sigma_{T-1},\sigma_T)\}.$$ Let the reverse curve be ${\underline r} = \{r_t\}_{t=0, \dots, T-2}$, where $r_t = (\sigma_{T-t}, \sigma_{T-t-1})$. Thus we are reading ${\underline \sigma}$ backwards and calling it ${\underline r}$. Note that since $\sigma_{t}$ is the projection of $\sigma_{t-1}$ on $K_t$, each piece-wise linear segment $(\sigma_t, \sigma_{t+1})$ is a straight line and hence differentiable except at the end points. Moreover, since each $\sigma_t$ is obtained by projecting $\sigma_{t-1}$ onto $K_t$ and $K_{t+1}\subseteq K_t$, we have that the projection hyperplane 
$F_t$ that passes through $\sigma_t=\cP_{K_t}(\sigma_{t-1})$ and is perpendicular to $\sigma_t - \sigma_{t-1}$ separates the two sub curves $\{(\sigma_1,\sigma_2), (\sigma_2,\sigma_3), \dots, (\sigma_{t-1},\sigma_t)\}$ and $\{(\sigma_t,\sigma_{t+1}), (\sigma_{t+1},\sigma_{t+2}), \dots, (\sigma_{T-1},\sigma_T)\}$.


Thus, we have that 
for each segment $r_\tau$, at each point where it is differentiable, the curve $r_1, \dots r_{\tau-1}$ lies on one side of the hyperplane that passes through the point and is perpendicular to $ r_{\tau+1}$. Thus, we conclude that curve ${\underline r}$ is self-expanded.

% apply the result from \cite{Manselli} that bounds the total distance of a self-expanded curve belonging to a closed bounded convex set, to get the result.  

As a result, Theorem \ref{defn:se-curve} implies that the length of ${\underline r}$ is at most $O(d^{d/2} \text{diameter}(K_1))$, and the result follows since the length of ${\underline r}$ is same as that of ${\underline \sigma}$ which is $\Sigma$. 
\end{proof}

\section{Proof of Theorem \ref{thm:ocs}}
Clearly, with $f_t\equiv0$ for all $t$, with Algorithm \ref{coco_alg_1}, $y_t=x_t$ and the successive $x_t$'s are such that $x_{t+1} = \cP_{S_t}(x_{t})$. Thus, essentially, the curve ${\underline x} = (x_1, x_2), (x_2,x_3), \dots, (x_{T-1}, x_{T})$ formed by Algorithm \ref{coco_alg_1} for OCS is a projection curve (Definition \ref{defn:projectioncurve}) on $S_1\supseteq, \dots, \supseteq S_T$ and the result follows from Lemma \ref{lem:projection} and the fact that $\text{diameter}(S_1)\le D$.

%\begin{lemma}\label{lem:meanwidthfactorize}
%Let $K_1, K_2 \subseteq K$ be such that Lebesgue measure of $K_1\cap K_2=0$ and $K_1,K_2$ are convex.
%Then 
%\begin{equation}\label{}
%W(K_1)+W(K_2) \le W(K).
%\end{equation}
%\end{lemma}
\section{Proof of Theorem \ref{thm:tvmonotone}} 

%\begin{figure}
%
%\includegraphics[width=10cm,keepaspectratio,angle=0]{FigPhaseAnalysis.pdf}
%
%\caption{Depiction of the definition of phases and related quantities for a monotonic instance, where phase $1$ has $t^\star(1)=4$, thus curve till $F_4$ has been explored in phase $1$.  The next phase, phase $2$ is empty, since the angle between $F_4$ and $F_5$ is more than $\pi/4$. Phase $3$ begins from $F_5$ as its first hyperplane by setting $s(3)=5$.}
%\label{fig:monotone}
%\end{figure}





\begin{proof}
Recall that $d=2$, and the definition of $F_t$ from Definition \ref{defn:projhyperplane}. Let the center be $\sfc=\cP_{S_1}(x_1)$.  Let $t_{\text{orth}}$ be the earliest $t$ for which $\angle (F_t, F_1) = \pi$.

Initialize $\kappa=1$, $s(1)=1$, $\tau(1) =1$. 



{\bf BeginProcedure}
Step 1:Definition of Phase $\kappa$.
Consider $$\tau(\kappa) = \arg \max_{s(\kappa)< t \le t_{\text{orth}}, \angle(F_{s(\kappa)}, F_t) \le \pi/4} t.$$

{\bf If there is no such $\tau(\kappa)$}, 

\quad Phase $\kappa$ ends, define Phase $\kappa$ as {\bf Empty},  $s(\kappa+1) =  \tau(\kappa)+1$.


{\bf Else If} 

\quad $\angle(F_{\tau(\kappa)}, F_1)=\pi$ Exit

{\bf Else If} 

\quad $s(\kappa+1)=\tau(\kappa)$

{\bf End If}

Increment $\kappa=\kappa+1$,  and Go to Step 1.

{\bf EndProcedure}

\begin{example}\label{exm:phasedef} To better understand the definition of phases, consider Fig. \ref{fig:phases}, where the largest $t$ for which the angle between $F_t$ and $F_1$ is at most $\pi/4$ is $3$. Thus, $\tau(1)=3$, i.e., phase $1$ explores till time $t=3$ and phase $1$ ends. The starting hyperplane to consider in phase $2$ is $s(2)=3$ 
and given that angle between $F_3$ and and the next hyperplane $F_4$ is more than $\pi/4$, phase $2$ is empty and phase $2$ ends by exploring till $t=4$. The starting hyperplane to consider in phase $3$ is $s(3)=4$ and the process goes on. The first time $t$ such that the angle between $F_1$ and $F_t$ is $\pi$ is $t=6$, and thus $t_{\text{orth}}=6$, and the process stops at time $t=6$. 
This also implies that $S_6 \subset F_1$. 
Since $S_t$'s are nested, for all $t\ge 6$, $S_t\subset F_1$. Hence the total CCV after $t\ge t_{\text{orth}}$ is at most $GD$.
\end{example}

%Essentially, $\tau(1)$ is the largest time $t$ by which the angle between $F_1$ and $F_t$ is at most $\pi/4$. If there is no such $t$, then the angular region making angle of $\pi/4$ with $F_1$ is defined to the Empty. Thus, in phase $1$ hyperplanes till $F_{\tau(1)}$ are explored. The next phase begins by resetting $F_1$ as $F_{\tau(1)}$ by incrementing $s(\kappa) = \tau(\kappa)$. 
The main idea with defining phases, is to partition the whole space into empty and non-empty regions, where in each non-empty region, the starting and ending hyperplanes have an angle to at most $\pi/4$, while in an empty phase the starting and ending hyperplanes have an angle of at least $\pi/4$. Thus, we get the following simple result.

\begin{figure}


\includegraphics[width=15cm,keepaspectratio,angle=0]{Fig-phase.png}
\caption{Figure corresponding to Example \ref{exm:phasedef}.}
\label{fig:phases}
\end{figure}



\begin{lemma}\label{lem:nrphases} For $d=2$, there can be at most $4$ non-empty and $4$ empty phases.  
%that the number of phases (counting both non-empty and empty) till time $t_{\text{orth}}$ is at most $4$. 
\end{lemma}
Proof is immediate from the definition of the phases, since any consecutively occurring non-empty and empty phase exhausts an angle of at least $\pi/4$.

\begin{rem}\label{rem:aftertorth}
Since we are in $d=2$ dimensions, for all $t\ge t_{\text{orth}}$, the movement is along the hyperplane $F_1$ and thus the resulting constraint violation after time $t\ge t_{\text{orth}}$ is at most $GD$. Thus, in the phase definition above, we have only considered time till $t_{\text{orth}}$ and we only need to upper bound the CCV till time $t_{\text{orth}}$. \end{rem}





We next define the following required quantities.


\begin{definition}\label{defn:tstar}
With respect to the quantities defined for Algorithm \ref{coco_alg_1}, let for a non-empty phase $\kappa$ 
$$r_{\max}(\kappa)= \max_{s(\kappa) < t\le \tau(\kappa)} || y_t - \sfc||\ \text{and} \ t^\star(\kappa) = \arg \max_{s(\kappa) < t\le \tau(\kappa)}^T || y_t- \sfc||.$$
%Thus, a non-empty phase $\kappa$ consists of time slots $\cT(\kappa) = [t^\star(\kappa) - t^\star(\kappa-1), \tau(\kappa)]$ and $t^\star(\kappa) \in \cT(\kappa)$, and the angle $\angle(F_{t_1}, F_{t_2}) \le \pi/4$ when $t_1,t_2\in \cT(\kappa)$.
%
%If phase $\kappa$ is empty then $t^\star(\kappa)=t^\star(\kappa-1)+1$ %Consider the ball $\cB_1$ 
%with center $c$ as the projection of $x_1$ onto $S_1$ and radius $r_{\max}$. 
\end{definition}
$t^\star(\kappa)$ is the time index belonging to phase $\kappa$ for which $y_t$ is the farthest.

\begin{definition} 
A non-empty phase $\kappa$ consists of time slots $\cT(\kappa) = [\tau(\kappa-1), \tau(\kappa)]$ and the angle $\angle(F_{t_1}, F_{t_2}) \le \pi/4$ for all $t_1,t_2\in \cT(\kappa)$. Using Definition \ref{defn:tstar}, we partition $\cT(\kappa)$ as $\cT(\kappa) = \cT^-(\kappa) \cup \cT^+(\kappa)$, where $\cT^-(\kappa) = [\tau(\kappa-1)+1, t^\star(\kappa)+1]$ and $\cT^+(\kappa) = [ t^\star(\kappa)+2, \tau(\kappa)]$.
\end{definition}

Thus, $\cT(\kappa)$ and $\cT(\kappa+1)$ have one common time slot.

\begin{figure}
\includegraphics[width=15cm,keepaspectratio,angle=0]{Fig-newviolation.png}
\caption{Illustration of definition of $z_t(\kappa)$ for $t\in \cT(\kappa)$. In this example, for phase $1$, $t^\star(1)=3$ since the distance of $y_3$ from $\sfc$ is the farthest for phase $1$ that consists of time slots 
$\cT(1) = \{2,3\}$. Hence $z_{t^\star(1)+1}(1)=x_4$. For $t \in \cT(1) \backslash  t^\star(1)+1$,  $z_{t}(1)$ are such 
$z_{t+1}(1)$ is a projection of $z_{t}(1)$ onto $F_t$.}
\label{fig:MaxR}
\end{figure}
\begin{definition}\label{}
[Definition of $z_t(\kappa)$ \  for $t\in \cT^-(\kappa)$]. Let 
$z_{t^\star(\kappa)+1} = x_{t^\star(\kappa)+1}$.
For $t \in \cT^-(\kappa) \backslash t^\star(\kappa)+1$, define $z_t(\kappa)$ inductively as follows. 
$z_t(\kappa)$ is the pre-image of $z_{t+1}(\kappa)$ on $F_{t-1}$ such that the projection of $z_t(\kappa)$ on $F_t$ 
is $z_{t+1}(\kappa)$. 
%Consider the hyperplane $F_t$ as defined before. Extend this line till it intersects with ball $\cB_1 \cap \chi$, and call it $L_i'$. $S_i'$ is convex hull of $L_i' \cup S_{i+1}$. 
\end{definition}

\begin{definition}\label{}
[Definition of $z_t(\kappa)$ \  for \ $t\in \cT^+(\kappa)$]. 
For $t\in \cT^+(\kappa)$, define $z_t(\kappa)$ inductively as follows. 
$z_t(\kappa)$ is the projection of $z_{t-1}(\kappa)$ on $F_{t-1}$. 
%Consider the hyperplane $F_t$ as defined before. Extend this line till it intersects with ball $\cB_1 \cap \chi$, and call it $L_i'$. $S_i'$ is convex hull of $L_i' \cup S_{i+1}$. 
\end{definition}

See Fig. \ref{fig:MaxR} for a visual illustration of $t^\star(\kappa)$ and $z_t(\kappa)$.

The main idea behind defining $z_t(\kappa)$'s  is as follows. For each non-empty phase, we will construct a projection curve (Definition \ref{defn:projectioncurve}) using points $z_k$ such that the length of the projection curve upper bounds the CCV of Algorithm \ref{coco_alg_1} (shown in Lemma \ref{lem:violationub}), and then use Lemma \ref{lem:projection} to upper bound the length of the projection curve. %For an empty phase, the CCV of Algorithm \ref{coco_alg_1} is at most $D$, and there are at most $4$ empty phases. Thus, we will get the required bound.

\begin{definition}\label{}
[Definition of $S_t'$ for a non-empty phase $\kappa$:]  $S_{t^\star(\kappa)+1}' = S_{t^\star(\kappa)+1}$.
For $t \in \cT^-(\kappa) \backslash t^\star(\kappa)+1$, 
$S_t'$ is the convex hull of $z_{t+1}(\kappa) \cup S_t \cup S'_{t+1}(\kappa)$. For $t\in \cT^+(\kappa)$, 
$S_t' =S_t$.
See Fig. \ref{fig:defSprime}.
\end{definition}

\begin{lemma}\label{lem:nestedprime} For a non-empty phase $\kappa$, for any $t \in \cT(\kappa)$, $S_{t+1}' \subseteq S_t' $, i.e. they are nested.
\end{lemma}
\begin{figure}
\includegraphics[width=10cm,keepaspectratio,angle=0]{FigSprime.pdf}
\caption{Definition of $S_t$'s where $U_t$ are the extra regions that are added to $S_t$ to get $S_t'$.}
\label{fig:defSprime}
\end{figure}

\begin{definition} For a non-empty phase, 
 $\chi(\kappa) = S_{\tau(\kappa-1)}'  \cap \cH_{\tau(\kappa)}^+$, where $\cH_{\tau(\kappa)}^+$ has been defined in Definition \ref{defn:projhyperplane}.

\end{definition}

\begin{definition}\label{}
[New Violations for  $t\in \cT(\kappa)$:] 
For a non-empty phase $\kappa$, for $t\in \cT(\kappa) \backslash \tau(\kappa-1)$, let 
$$v_t(\kappa) = ||z_t(\kappa)-z_{t-1}(\kappa)||.$$
\end{definition}
%Phase $\kappa$ ends and increment $\kappa=\kappa+1$, reset $s(\kappa) =  t^\star(\kappa)$ $c=x_{t^\star(\kappa)}$ and Go to Step 1. 

%
%\begin{definition}
%For an empty phase $\kappa$, $\chi(\kappa) = S_{t^\star(\kappa-1)}' \cap \cH_{t^\star(\kappa-1)}^+$.
%\end{definition}
%Refer to Fig. \ref{fig:monotone} for a pictorial description of a two-dimensional monotonic instance.


\begin{lemma}\label{lem:membership} For each non-empty phase $\kappa$, all $z_{t}(\kappa)$'s for $t\in \cT(\kappa)$ belongs to $\cB(\sfc, \sqrt{2}D)$, where $\cB(c,r)$ is a ball with radius $r$ centered at $c$. In other words, $\chi(\kappa) \subseteq \cB(\sfc, \sqrt{2}D)$.
\end{lemma}
\begin{proof}
Recall that for a non-empty phase $\kappa$,  $\cT(\kappa) = \cT^-(\kappa) \cup  \cT^+(\kappa).$ We first argue about $t\in \cT^-(\kappa)$.
By definition, $z_{t^\star(\kappa)+1} = x_{t^\star(\kappa)+1}$ and $x_{t^\star(\kappa)+1}\in S_{t^\star(\kappa)}$. Thus, $z_{t^\star(\kappa)+1} \in \cB(\sfc, \sqrt{2}D)$.
Next we argue for $t \in \cT^-(\kappa) \backslash t^\star(\kappa)+1$.
Recall that the diameter of $\cX$ is $D$, and the fact that $y_t \in S_{t-1}$ from Algorithm \ref{coco_alg_1}. Thus, for any non-empty phase $\kappa$, the distance from $\sfc$ to the farthest $y_t$ belonging to the phase $\kappa$ is at most $D$, i.e., $r_{\max}(\kappa)\le D$. 
Let the pre-image of $z_{t^\star(\kappa)+1}(\kappa)$ onto $F_{s(\kappa)}$ (the base hyperplane with respect to which all hyperplanes have an angle of at most $\pi/4$ in phase $\kappa$) be $p(\kappa)$ such that projection of $p(\kappa)$ onto $F_{s(\kappa)}$ is $z_{t^\star(\kappa)+1}(\kappa)$. 
From the definition of any non-empty phase, the angle between $F_{s(\kappa)}$ and $F_{t}$ for $t\in \cT(\kappa)$ is at most $\pi/4$. 
Thus, the distance of $p(\kappa)$ from $\sfc$ is at most $\sqrt{2}D$. 

%Thus, if $t^\star(\kappa)+1-s(\kappa)=1$, we are done. 



%When $t^\star(\kappa)+1-s(\kappa)>1$, we proceed as follows. 
Consider the `triangle' $\Pi(\kappa)$ that is the convex hull of $\sfc, z_{t^\star(\kappa)+1}(\kappa)$ and $p(\kappa)$.
Given that the angle between $F_{t^\star(\kappa)}$ and $F_{t^\star(\kappa)-1}$ is at most $\pi/4$, the argument above implies that 
$z_t(\kappa) \in \Pi(\kappa)$ for $t=t^\star(\kappa)$. For $t= t^\star(\kappa)-1$, $z_t(\kappa) \in F_{t-1}$ is the projection of  $z_{t-1}(\kappa)$ onto $S_{t-1}'$. This implies that the distance of $z_t(\kappa)$ (for $t=t^\star(\kappa)-1$) from $\sfc$ is at most 
$$\frac{D}{\cos(\alpha_{t, t^\star(\kappa)}) \cos(\alpha_{t^\star(\kappa), t^\star(\kappa)+1})},$$ where 
$\alpha_{t_1,t_2}$ is the angle between $F_{t_1}$ and $F_{t_2}$.
From the monotonicity of angles $\theta_t$ (Definition \ref{defn:anglemonotone}), and the definition of a non-empty phase, we have that $\alpha_{t, t^\star(\kappa)}+\alpha_{t^\star(\kappa), t^\star(\kappa)+1} \le \pi/4$ and $\alpha_{t, t^\star(\kappa)}\ge 0, \alpha_{t^\star(\kappa), t^\star(\kappa)+1}\ge 0$.
Next, we appeal to the identity
\begin{equation}\label{eq:cosidentity}
\cos(A+B) \le \cos(A)\cos(B)
\end{equation} where $A+B\le \pi/4$, to claim that $z_t(\kappa) \in \Pi(\kappa)$ for $t=t^\star(\kappa)-1$. 

Iteratively using this argument while invoking the identity \eqref{eq:cosidentity} gives the result that for any $t \in \cT^-(\kappa)$, we have that $z_{t}(\kappa)$  belongs to $\Pi(\kappa)$. Since $\Pi(\kappa) \subseteq \cB(\sfc, \sqrt{2}D)$, we have the claim for all $t\in \cT^-(\kappa)$. 
%$t^\star(\kappa) - t^\star(\kappa-1)\le t \le t^\star(\kappa)$.


%
%
%Recall that for $t^\star(\kappa) - t^\star(\kappa-1)\le t\le t^\star(\kappa)$ 
%we have that $z_t(\kappa) \in F_{t-1}$ and is the projection of  $z_{t-1}(\kappa)$ onto $S_t'$.  Moreover, angle between any $F_t$ and $F_{t^\star(\kappa)}$ is at most $\pi/4$. Thus, as above $z_{t}(\kappa)$ for $t=t^\star(\kappa)-1$ 
%Thus, following the identity that 
%
%
%the monotonicity property of angles $\theta_t$ (Definition \ref{defn:anglemonotone}) and the non-expansive property of convex projections\footnote{Distance from $z_{t}(\kappa)$ to $z_{\tau(\kappa)}(\kappa)$  decreases as $t$ increases from $t^\star(\kappa) - t^\star(\kappa-1)\le t$ to $\tau(\kappa)$}, we have that $z_{t}(\kappa)$  belongs to $\Pi(\kappa)$. Since $\Pi(\kappa) \subseteq \cB(c, \sqrt{2}D)$, we have the claim for $t^\star(\kappa) - t^\star(\kappa-1)\le t \le t^\star(\kappa)$.

By definition $z_{t}(\kappa)$ for $t\in \cT^+(\kappa)$ belong to $S_{t-1}\subseteq S_1$. Thus, their distance from $\sfc$ is at most $D$. 
\end{proof}



\begin{lemma}\label{lem:violationub} For each non-empty phase $\kappa$, and for $t\in \cT(\kappa)$ the violation $v_t(\kappa)\ge  \text{dist}(x_t, S_t)$, where $\text{dist}(x_t, S_t)$ is the original violation.
\end{lemma}
\begin{proof}
By construction of any non-empty phase $\kappa$, for $t\in \cT(\kappa)$ both $x_t(\kappa)$ and $z_t(\kappa)$ belong to $F_{t-1}$. Moreover, by construction, the distance of $z_t(\kappa)$ from $\sfc$ is at least as much as the distance of $x_t$ from $\sfc$. Thus, using  the monotonicity property of angles $\theta_t$ (Definition \ref{defn:anglemonotone}) we get the result. See Fig. \ref{fig:MaxR} for a visual illustration.
\end{proof}



For each non-empty phase $\kappa$, by definition, the curve defined by sequence $z_{t}(\kappa)$ for $t\in\cT(\kappa)$ is a projection curve (Definition \ref{defn:projectioncurve}) on sets $S'_t(\kappa)$ (note that $S'_t(\kappa)$'s  are nested from Lemma \ref{lem:nestedprime}). Moreover, for all $t\in\cT(\kappa)$, set $S'_t(\kappa) \subset \chi(\kappa)$ which is a bounded convex set. 
Thus, for $d=2$ from Lemma \ref{lem:projection} the length of curve ${\underline z}(\kappa) = \{(z_{t}(\kappa), z_{t+1}(\kappa))\}_{t\in \cT(\kappa)}$
\begin{equation}\label{eq:totallengthprojection}
\sum_{t\in \cT(\kappa)} v_t(\kappa) \le 2
\text{diameter}(\chi(\kappa)).
\end{equation}


 

%Next, we make use of Lemma \ref{lem:width2D} that exactly characterizes the average width in $d=2$-dimensions. 
By definition, the number of non-empty phases till time $t_{\text{orth}}$ is at most $4$. Moreover, in each non-empty phase $\chi(\kappa) \subseteq \cB(\sfc, \sqrt{2}D)$ from Lemma \ref{lem:membership} . 

Thus, from  \eqref{eq:totallengthprojection}, we have that \begin{align}\nn
\sum_{\text{Phase} \ \kappa \ \text{is non-empty}} \ \ \ \sum_{t\in \cT(\kappa)} v_t(\kappa) &\le \sum_{\text{Phase} \ \kappa \ \text{is non-empty}} 2\ \text{diameter}(\chi(\kappa)) \\ \label{eq:summeanwidth}
&\le 8 \ \text{diameter}(\cB(\sfc, \sqrt{2}D))\le O(D).
\end{align}

Using Lemma \ref{lem:violationub}, we get 
\begin{align}\label{eq:finalviolation}
\sum_{\text{Phase} \ \kappa \ \text{is non-empty}} \ \ \ \sum_{t\in \cT(\kappa)}\text{dist}(x_t, S_t) \le O(D).
\end{align}

For any empty phase, the constraint violation is the length of line segment $(x_t,\cP_{S_t}(x_{t}))$ (Algorithm \ref{coco_alg_1}) crossing it is a straight line whose length is at most $O(D)$. 
%The length $||(y_t - x_{t+1})||$ of the curve is also an upper bound on the CCV incurred by Algorithm \ref{coco_alg_1} when crossing the two hyperplanes that define the empty phase. 
 Moreover, the total number of empty phases (Lemma \ref{lem:nrphases}) is a constant.
 %, so the total length of the violation curve (Algorithm \ref{coco_alg_1}) crossing all empty phases is $O(D)$. 
 Thus, the length of the curve $(x_t,\cP_{S_t}(x_{t}))$ for Algorithm \ref{coco_alg_1} corresponding to all empty phases is at $O(D)$.
%
%From Lemma \ref{lem:membership} it follows that  that $\cup_{\kappa} \chi_k \subseteq \cB(c, \sqrt{2}D)$. Moreover, $\chi_k \cap \chi_{k+1} = F_{t^\star(\kappa)}$ and $\chi_j \cap \chi_k = \phi$ for $|j- k|>1$. Since $F_{t^\star(\kappa)}$ is 
%a hyperplane that contributes zero mass to the integral in \eqref{eq:projlength}, we have that 
%
%\begin{equation}\label{eq:summeanwidth}
%\cup_{\kappa} W(\chi(\kappa)) \le W(\cB(c, \sqrt{2}D)) \stackrel{(a)}\le O(\sqrt{2}D).
%\end{equation}
%where $(a)$ follows from \eqref{eq:WBound1}.

Recall from \eqref{eq:distviolationrelation} that the CCV is at most $G$ times $\text{dist}(x_t, S_t)$.
Thus, from \eqref{eq:finalviolation} we get that the total violation incurred by Algorithm \ref{coco_alg_1} corresponding to non-empty phases is at most $O(GD)$, while corresponding to empty phases is at $O(GD)$.
Finally, accounting for the very first violation $\text{dist}(x_1, S_1)\le D$ and the fact that the CCV after time $t\ge t_{\text{orth}}$ (Remark \ref{rem:aftertorth}) is at most $GD$, we get that the total constraint violation $\text{CCV}_{[1:T]}$ for Algorithm \ref{coco_alg_1} is at most $O(G D)$. 

\end{proof}


%\bibliographystyle{IEEE}

%  \def\rvara{2}
%\begin{tikzpicture}
%    % Define the center point and radii
%    \coordinate (center) at (0,0);
%  \def\avar{1}
%    \def\bvar{2}
%    \def\cvar{3}
%    \def\dvar{4}
%    \def\evar{5}
%
%    % Draw the concentric circles
%    \draw[thick] (center) circle (\avar);
%    \draw[thick] (center) circle (\bvar);
%    \draw[thick] (center) circle (\cvar);
%    \draw[thick] (center) circle (\dvar);
%    \draw[thick] (center) circle (\evar);
%
%    % Place the nodes
%    
%    
%    
%    
%    \node[circle,fill=black,inner sep=2pt] at (center) {};
%    \node[below] at (center) {$x_n$};
%
%    \node[circle,fill=black,inner sep=2pt] at (\avar,0) {};
%    \node[below] at (\avar,0) {$x_{n-1}$};
%
%    \node[circle,fill=black,inner sep=2pt] at (1,3.872) {};
%    \node[above] at (1,3.872) {$x_{\ell+2}$};
%
%    \node[circle,fill=black,inner sep=2pt] at (0,\evar) {};
%    \node[below] at (0,\evar) {$x_{\ell+1}$};
%    
%    \node[circle,fill=black,inner sep=2pt] at (-6,6) {};
%    \node[below] at (-6,6) {$x_{0}$};
%    
%    \node[circle,fill=black,inner sep=2pt] at (-5.5,5.5) {};
%    \node[below] at (-5.5,5.5) {$x_{1}$};
%    
%    \node[circle,fill=black,inner sep=2pt] at (-4, 4) {};
%    \node[below] at (-4, 4) {$x_{\ell}$};
%    
%    \draw (-6.5,6.5) -- (0,0)  -- (5,-5);
%    
%     \draw[dashed] (-4, 4) -- (0,\evar);
%     
%     \draw (0,\evar+1) -- (0,0)  -- (0,-6);
%    
%    \draw[dashed] (0,0)  -- (1,3.872);
%    
%    \draw (0.35,1.414) arc (60:75:1.5);
%    
%    \node[above] at (.35, 1.414) {$\alpha_{\ell+1}$};
%    
%      \draw (0,0) -- node[below] {$r_{\ell+1}$}  ++(-5,0);
%
%      \draw (0,0) -- node[below] {$r_{\ell+2}$}  ++(-3.75,1.39);
%
%    
%\end{tikzpicture}
%
%
%
%Consider the 2-D case alone for analyzing the greedy algorithm for Nested Convex Body Chasing. Let the starting point be $x_0$. Let at time $t=1, \dots, n$, a convex body $S_t \subseteq S_{t-1}$ is shown and the point chosen by Greedy by $x_t \in S_n$.
%
%Then $x_0, x_1, x_2, \dots, x_n$ is the sequence of points generated by the Greedy algorithm and let curve $C$ denote the line segments joining $x_0, x_1, x_2, \dots, x_n$.
%
%
%Let $x_1, x_2, \dots, x_\ell$ be such that they lie on the line joining $x_0$ and $x_n$. If there is no such $\ell$, then let $\ell =0$. This implies that the total length of $C$ from $x_0$ till $x_\ell$ is at most $D$, i.e., 
%\begin{equation}\label{eq:prefix}
%\sum_{i=0}^{\ell-1}d(x_i,x_{i+1}) \le D.
%\end{equation}
%
%
% 
%Let the $y$-axis be oriented along the line segment $x_{\ell+1}, x_n$ and $x$-axis be perpendicular to the $y$-axis. 
%
%{\bf Property 1:} First thing to note is that all convex bodies $S_{\ell+1}, S_{\ell+2}, \dots, S_n$ lie on one side of the $y$-axis, since otherwise it will contradict the convexity of  $S_t, t\ge \ell+1$.
%
%
% Let $r_i = d(x_i, x_n)$ be the distance between point $x_i$ and the final point $x_n$. As we discussed, because of successive projections, $r_i\le r_{i-1}$. 
%
%Consider $x_n$ as the center and draw $n-{\ell+1}$ concentric circles with radius $r_i, i=\ell+1, 2,\dots, n-1$. 
%Let $\alpha_i$ be the angle between line segments $(x_i,x_n)$ and $(x_{i+1},x_n)$. 
%\begin{lemma}\label{lem:angle}
%$\alpha_i\ge0$ for all $i=\ell+1, 2,\dots, n-1$.
%\end{lemma}
%Proof follows similar to Property 1.
%
%Then from triangle inequality we get 
%$d(x_i,x_{i+1}) \le r_{i} - r_{i+1} + \alpha_i d(x_i, x_n)$. 
%Therefore, the total distance of the curve $C$ from $x_{\ell+1}$ till $x_n$
%$$\sum_{i=\ell+1}^{n-1}d(x_i,x_{i+1}) \le \sum_{i=1}^{n-1} r_{i} - r_{i+1}+ \alpha_i d(x_i, x_n) \le D+  D \sum_{i=1}^{n-1} \alpha_i.$$ 
%From Lemma \ref{lem:angle} and Property 1, $\sum_{i=1}^{n-1} \alpha_i\le \pi$, 
%hence 
%$\sum_{i=\ell+1}^{n-1}d(x_i,x_{i+1}) \le D+D\pi$.
%Since the distance between $(x_{\ell},x_{\ell+1})$ is at most $D$, combining this with \eqref{eq:prefix}, we get that the total length of the curve $C$ is $3D + \pi D$.

%\input{NCBCtoConstraintViolation}
 
%\section{Introduction}
%Consider the 2-D case for the moment, so the first convex body $K_1 \subset \bbR^2$.
%Let the initial position of the Greedy algorithm be $x_1$, and $x_i$ be its location once $K_i \subset K_{i-1}$ has been revealed and the projection has been taken from $x_{i-1}$ onto $K_i$.
%Let the convex hull of the $x_1, \dots, x_i$ be $S_i$. 
%
%
%\section{Preliminaries}
%    Let a curve be called Lipschitz if the map $\gamma:I \rightarrow \bbR^n$ is Lipschitz.
%
%\begin{definition}\label{defn:se-curve} A Lipschitz curve $\gamma: I \rightarrow \bbR^n$  is called self-expanded, if for every $t$ where 
%$\gamma'(t)$ exists, we have 
%$$< \gamma'(t), \gamma(t)-\gamma(u)> \ \ge 0$$ for all $u\in I$ with $u \le t$. 
%In words, what this means is that if $\gamma$ starting in a point $x_0$ is self expanded, if for every $x\in \gamma$ for which there exists the tangent line $T$, the arc (sub-curve) $(x_0, x)$ is
%contained in one of the two half-spaces, bounded by the hyperplane through
%$x$ and orthogonal to $T$. 
%\end{definition}
%
%
%
%\begin{definition}
%Let $x=x(s)$ be the representation
%of $\gamma$ as a function of the arc length; let us define, $\gamma(s)= \{x\in \bbR^2, x=x(\sigma), 0 \le \sigma\le s\}$.
%$\gamma\in \Gamma$ is defined to 
%satisfy to be self-expanding if $\gamma'(s)$ exists for any $s$, then the 
%\end{definition}
%
%
%Consider $\bbR^2$, where in addition to the two orthogonal basis $x$ and $y$ axes, consider the rotation of $x,y$ axes by angle $\pi/4$, denoted by   $x'$ and $y'$, respectively, as another orthogonal basis pair.
%The set of total axes is $A = \{x,y, x', y'\}$.
%
%\begin{definition} Consider any convex body $B \subset \bbR^2$.
%For $a\in A$, let $W_a(i)$ be the minimum distance between any two supporting hyperplanes of $B$ that are perpendicular to $a$-axis.
%\end{definition}
%
%Clearly, for any $a\in A$, $W_a(B)$ is a non-increasing quantity as a function of $B$.  
%We will prove the following result for  self-expanding curves in $\bbR^2$.
% 
%\begin{theorem}\label{thm:mainresult}
%For any self-expanded curve in $\bbR^2$, let $\gamma^+$ be its convex hull. Then 
%$$\max_a\{W'_a(\gamma^+)\} \ge \frac{1}{\sqrt{2}}.$$
%\end{theorem}
% 
%The curved produced by the Greedy algorithm satisfies the self-contracting property that is defined as follows.
%
%  \begin{definition} A curve $\gamma: I \rightarrow \bbR^n$ is self-contracted if for every $t_1 \le t_2 \le t_3$ in $I$, 
%  $$d(\gamma(t_2), \gamma(t_3)) \le d(\gamma(t_1), \gamma(t_3)).$$
%  \end{definition}
%  
%  \begin{lemma} Let $\gamma$ be a self-contracted curve and let $\gamma$ be differentiable at point $t$, then 
%  $$< \gamma'(t), \gamma(u)-\gamma(t)> \ \ge 0$$ for all $u\in I$ with $u>t$. 
%\end{lemma}
%
%\begin{prop}\label{prop:scGreedy} The affine extension $C$ of the sequence of points  $\{x_1, x_2, \dots, x_m\}$ produced by the Greedy algorithm for the CBC problem with sets $\{K_1, K_2, \dots, K_m\}$ is a self-contracting curve because of the orthogonality principle of projections on convex bodies.
%\end{prop}
%Next, we connect the self-contracted and self-expanded curves as follows. 
%
%  \begin{definition} Given a curve $\gamma:I \rightarrow \bbR^n$, denote by $I^- = \{-t |  t\in I\}$ and define the reverse curve 
%  $\gamma^-:I^- \rightarrow \bbR^n$ as 
%  $$\gamma^-(t) = \gamma(-t), \ \text{for} \ t\in I^-.$$
%    \end{definition}
%    
%    Essentially, $\gamma^-$ is $\gamma$ traced backwards or in reverse order. See Fig. \ref{fig:cont-expand} for an example.
%    
%    \begin{lemma}\label{lem:se-scconnection}
%Let $\gamma:I \rightarrow \bbR^n$ be a Lipschitz curve. Then $\gamma$ is self-contracted if and only if $\gamma^-$ is self-expanded curve.
%\end{lemma}
%
%Lemma \ref{lem:se-scconnection} implies that the length of self-expanded and self-contracted curves is the same. 
%Thus, from Proposition \ref{prop:scGreedy}, to bound the length of curve $C$ produced by the Greedy algorithm, it is sufficient to bound the length of the self-expanded curve corresponding to $C$.  
%  
%Let  the diameter of $K_1$, the first convex body, be $D$. Then, Theorem \eqref{thm:mainresult} implies that the length of curve $C$ produced by the Greedy algorithm is at most $4\sqrt{2} D$.
%
 

%\begin{figure}[h]
%\centering
%\includegraphics[width=90mm]{greedyCBC.jpg}
%\caption{}
%\label{fig:greedy}
%\end{figure}
%Refer to Fig. \ref{fig:greedy}, where once the algorithm reaches $x_2$, in next step, either both $x$ and $y$ coordinates increase or only one of them does, i.e. $x_3$ belongs to either sector ABE or ACD. 
%Lets consider ABE. Because of the projection step to get to $x_2$ from $x_1$, we know that the angle $x_1x_2B$ is at least $\pi/2$. Therefore, angle $\theta_1\le \pi/4$, and thus the projection $\Delta_y$ of $\ell_1$ ( the distance moved by Greedy in the next step to get to $x_3$ from $x_2$) is at least $\ell_1/\sqrt{2}$. Even though the x-coordinate is decreasing between $x_3$ and $x_2$, we have $$\max\{W'_x(i), W'_y(i)\} \ge \frac{1}{\sqrt{2}}.$$
%
%The same idea works when $x_3$ belongs to sector ACD with roles reversed between $W_x$ and $W_y$.  Moreover, when $x_3$ belongs to the Good Area, both $x$ and $y$ coordinates increase with at least one of $W_x$ and $W_y$ having gradient at least as much as $1/\sqrt{2}$. Thus, we have 
%$$\max\{W'_x(i), W'_y(i)\} \ge \frac{1}{\sqrt{2}}$$ in each step.
%
%Now I want to claim that the total length $L$ of the curve followed by the greedy algorithm is at most $2\times \sqrt{2} \times \text{diameter}$. 
%
%Essentially, because of $$\max\{W'_x(i), W'_y(i)\} \ge \frac{1}{\sqrt{2}},$$ if $L > 2\times \sqrt{2} \times \text{diameter}$ then the curves has to leaves the body in at least one dimension. 
%
%\section{$d$-dimensions}
%{\bf Algorithm:} On arrival of convex $S_j \subseteq S_{j-1}$, choose $x_j$ as the projection of $x_{j-1}$ on to $S_t$.
%
%Let $x_j$ be the point chosen by the algorithm at time step $j$ on arrival of convex set $S_j$.
%Let the convex hull of the $x_1, \dots, x_j$ be $C_j$.
%At time step $j$, for $i=1,\dots, d$, let $W_i(j)$ be the minimum distance between any two hyperplanes parallel to the $i$-axis that contain $C_j$. $W_i(j)$ is essentially the $i$-axis diameter of $C_j$. %Similarly define  $W_y(i)$ to be the minimum distance between any two hyperplanes parallel to the x-axis that contain $C_i$. 
%
%\begin{lemma}
%Either $x_{j+1}=x_{j}$ or at each time step $j+1$, $\exists \ i$ such that $W_i(j+1)-W_i(j) > 0$ for $i=1,\dots, d$.
%\end{lemma}
%\begin{proof}
%Note that the angle between $x_{j+1} -x_t$ and $x_{t} -x_{t-1}$ is more than $\pi/4$ for any $t\le j$, since each point $x_t$ is found by projecting $x_{t-1}$ on to the convex set $S_t$. Moreover, $S_{t+1} \subseteq S_t$ $\forall \ t$. This implies that either $x_{j+1}=x_{j}$ or at each time step $j+1$, $||x_t-x_{j+1}||  >  ||x_j-x_{j+1}||$ for all $t < j$.
%Thus,  either $x_{j+1}=x_{j}$,
%or $\exists \ i$ such that $W_i(j+1)-W_i(j) > 0$ for $i=1,\dots, d$.
% 
%\end{proof}
%
%\begin{lemma}
%Among the indices $i$ for which $W_i(j+1)-W_i(j) > 0$ for $i=1,\dots, d$, $\exists \  i'$ such that $W_{i'}(j+1)-W_{i'}(j) \ge \frac{1}{\sqrt{d}} ||x_{j+1}-x_j||$.
%\end{lemma}
%
%\begin{proof}
%Consider a hyperplane $H$ passing through the origin. Let any vector $v$ lie in one of the half space $G$ corresponding to $H$. Let the projection of $v$ onto the $i^{th}$-axis be $P_i(v) = \frac{v_i}{||v||}$, and its length be $|P_i|$. Then clearly the  $\max_i\{|P_i(v)|\}\ge \frac{1}{\sqrt{d}}$.
%
%Moreover, if $v \in G'$ where $G'\subset G$ with dimension $d' < d$, then 
%\begin{equation}\label{eq:growth}
%\max_{i=1, \dots, d'}\{|P_i(v)|\}\ge \frac{1}{\sqrt{d'}}.
%\end{equation}
%
%Let the set of indices for which $W_i(j+1)-W_i(j) > 0$ be $\cM$ with cardinality $|\cM|$. Consider the supporting hyperplane $H_j$ of $S_j$ at point $x_j$ that is normal to $x_{j}-x_{j-1}$. Let $G_j$ be the half space of $H_j$ that contains $S_{j+1}$. Consider the restriction of $G_j$ onto the indices of $\cM$, and consider point $x_j$ as the origin. 
%Then from \eqref{eq:growth} we get that  
%$W_{i'}(j+1)-W_{i'}(j) \ge \frac{1}{\sqrt{|\cM|}} ||x_{j+1}-x_j||$.
%Since $|\cM|\le d$, we get the result.
%
%
%\end{proof} 
%
%%Extending the above argument to $d$ dimensions. Let $x_j$ be the most recent point found by the algorithm when convex body $C_j$ was revealed. Consider $H_j$ the supporting hyperplane of $C_j$ at $x_j$ that is normal to $x_j-x_i$. Let the halfspace of $H_j$ containing $x_j-x_i$ be $G_j$. 
%%
%%Given that $x_j$ was found by projection onto $C_j$, $C_{j+1} \cap G_j = \phi$. Thus, the vector $x_{j+1}-x_j$ lies in  $G_j^c$, and hence $W_i(j+1)$ increases by amount $P_i(x_{j+1}-x_j)$ $\forall \ i=1, \dots, d$. Given that the projection  satisfies
%%$\max_i\{|P_i(x_{j+1}-x_j)|\}\ge \frac{1}{\sqrt{d}}$, we get that 
%%$$\max_{i, i=1,\dots, d} \{W'_i(t)\} \ge \frac{1}{\sqrt{d}}$$ in each step $t$.
%%
%%Given that there are a total of $d$ dimensions, we will get $L \le  d \times \sqrt{d} \times \text{diameter}$.
%
 \end{document}