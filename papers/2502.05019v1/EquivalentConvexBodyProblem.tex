\section{Bounding the Total Movement Cost $M_T$ \eqref{defn:totalmovementcost1}}



%The movement cost $M$ is a general distance measure that counts the maximum total distance 
%that can be traversed when projections are taken from arbitrary points between nested convex sets, and is typically expected to be much larger than the actual violation $\sum_{\tau=1}^t \text{dist}(x_{\tau}, S_{\tau})$ that Algorithm \ref{coco_alg_1} will incur. 
%In Theorem \ref{thm:tvmonotone}, we saw that if the projections satisfy the monotonicity property, then the total CCV can be bounded independent of $T$ only in terms of $d$ and $D$. In general, however, Algorithm \ref{coco_alg_1} can have arbitrary projections. Thus, we upper bound $M$ itself, which when multiplied by $G$ is an upper bound on the CCV of Algorithm \ref{coco_alg_1}.

We start by considering two simple cases where bounding $M_T$ is easy.
 \begin{lemma}\label{lem:spheres}
If all nested convex bodies $S_1\supseteq S_2  \supseteq \dots \supseteq S_T$ are spheres then $M_T\le d^{3/2}D$.
\end{lemma}

\begin{proof}
Recall the definition that $x_t\in \partial S_{t-1}, b_t=\cP_{S_{t}}(x_t)\in S_t$ from \eqref{defn:genconvxmovement}.
Let $||x_t-b_t||=r$, then since all $S_t$'s are spheres, at least along one of the $d$-orthogonal canonical basis vectors, $\text{diameter}(S_{t})\le \text{diameter}(S_{t-1}) - \frac{r}{\sqrt{d}}$. Since the diameter along any of the $d$-axis is $D$, we get the answer.
\end{proof}

%Similar results can be shown for nice convex bodies, such as parallel cuboids. 
 \begin{lemma}\label{lem:square}
If all nested convex bodies $S_1\supseteq S_2  \supseteq \dots \supseteq S_T$ are cuboids that are axis parallel to each other, then $M\le d^{3/2}D$.
\end{lemma}
Proof is identical to Lemma \ref{lem:spheres}.
Note that similar results can be obtained when $S_t$'s are regular polygons that are axis parallel with each other.


 
 After exhausting the universal results for an upper bound on $M_T$ for `nice' nested convex bodies, we next give a general bound on $M_T$ for any sequence of nested convex bodies which depends on the geometry of the nested convex bodies (instance dependent).
 To state the result we need the following preliminaries.
 
 Following \eqref{defn:genconvxmovement}, $b_t=\cP_{S_t}(x_t)$ where $x_t\in \partial S_{t-1}$. Without loss of generality, 
 $x_t\notin S_{t}$ since otherwise the distance $||x_t-b_t||=0$.
 Let $m_t$ be the mid-point of $x_t$ and $b_t$, i.e. $m_t = \frac{x_t+b_t}{2}$.
 \begin{definition}\label{defn:anglewidth}
Let the convex hull of $m_t \cup S_{t}$ be $\cC_t$.
 Let $w_t$ be a unit vector such that there exists $c_t>0$ such that the cone 
 $$C_{w_t}(c_t) = \left\{z\in \bbR^d: -w_t^T\frac{(z-m_t)}{||(z-m_t)||} \ge c_t\right\}$$ contains $\cC_t$. Since $S_{t}$ is convex, such $w_t, c_t>0$ exist. For example, $w_t=b_t-x_t$ is one such choice for which $c_t>0$ since $m_t \notin S_t$.
See Fig. \ref{fig:anglewidthmain} for a pictorial representation.
 
 Let $c^\star_{w_t,t} = \arg \min_{c_t} C_{w_t}(c_t)$,
 $c^\star_t = \min_{w_t} c^\star_{w_t,t}$, and $w_t^\star= \arg \min_{w_t} c^\star_{w_t,t}$.
 Moreover, let $c^\star = \min_t  c^\star_t$, where by definition, $c^\star <1$. 
 \end{definition}
 
% \begin{figure}[]
%  \begin{center}
%%\begin{tikzpicture}
%\begin{tikzpicture}[scale=1.5, dot/.style={circle,inner sep=1pt,fill,label={#1},name=#1},
%  extended line/.style={shorten >=-#1,shorten <=-#1},
%  extended line/.default=1cm]
%% Define coordinates
%\coordinate (w_t) at (0,0);
%\coordinate (z_t) at (0,-.5);
%\coordinate (x_t) at (-1,-1);
%\coordinate (y_t) at (1,0);
%\coordinate (C_top) at (1,2.15);
%\coordinate (C_bottom) at (3.55,-0.55);
%
%
%% Draw object (shaded region)
%%\draw[thick, gray, fill=pink!50] (C_top) to[out=-.5,in=-15] (C_bottom) -- cycle;
%
%
%
%
%
%% Draw projected image (S_t+1)
%\draw[thick, cyan, fill=cyan!50,opacity=0.75] (1,0) -- (1.5,1.5) -- (3,1) -- (3.2,0.15)  -- cycle;
%\draw[thick, blue!10, fill=blue!10,opacity=0.5] (z_t) --(1.5,1.495) --  (y_t)   -- cycle;
%\draw[thick, blue!10, fill=blue!10,opacity=0.5] (z_t) --  (y_t)  -- (3.2,0.15) -- cycle;
%
%\filldraw[blue] (y_t) circle (2pt) node[below right] {$b_t$};
%\filldraw[black] (z_t) circle (2pt) node[below right] {$m_t$};
%% Draw camera and viewing ray
%%\begin{scope}[on background layer]
%\draw[thick,->] (x_t) -- (y_t);
%%\end{scope}
%\draw[dashed,->] (.65,0) -- (z_t) -- (-0.65,-1)  node[below right] {$w_t$};
%\draw[dashed,->] (.65,0) -- (z_t) -- (-0.65,-1)  node[below right] {$w_t$};;
%%\draw[dashed,->] (.65,0) -- (x_t) -- (-0.65,-1)  node[below right] {$w_t$};;
%
%\coordinate (u_1) at (0,.5);
%\coordinate (u_2) at (0,-2);
%
%%\draw [ dashed, <-] (u_1) -- (z_t) -- (u_2) node[above left] {$H_u$};
%%\draw [ dashed, ->] (1.3,-.7) -- (z_t) -- (-2.8,.2) node[above left] {$u$};
%
%%\draw [extended line, ->] ($(u_1)!(z_t)!(u_2)$) -- (-.5,-0.05) node[above left ] {$u$};
%
%%\draw [extended line,<-] ($(-0.25,.5)!(z_t)!(.25,-2)$) node[above left] {$u$};
%
%%\tkzDefLine[orthogonal=through (z_t)](x_t,y_t);
%
%
%%\filldraw[black] (w_t) circle (2pt) node[above left] {$w_t$};
%\filldraw[black] (x_t) circle (2pt) node[below left] {$a_t$};
%
%%\begin{scope}[on background layer]
%\draw[thick, gray, fill=gray!50,opacity=0.76] (C_top) to[out=.5,in=25] (C_bottom) -- cycle;
%%\end{scope}
%\draw[thick, gray, fill=gray!50,opacity=0.76] (C_bottom) to[out=-155,in=-165] (C_top) -- cycle;
%\node at (2.25,0.75) {$S_{t}$};
%
%% Draw lines connecting camera to object and projected image
%\draw[dotted] (z_t) -- (1.5,1.5);
%\draw[dotted] (z_t) -- (3.2,0.15);
%%\draw[dotted] (w_t) -- (3,1);
%%\draw[dotted] (w_t) -- (2.5,0.5);
%%\draw[dotted] (w_t) -- (1.5,0.5);
%%\begin{scope}[on background layer]
%\draw[gray!30,  thick,fill=gray!50,opacity=0.6] (z_t) -- (C_top) -- (C_bottom) -- (z_t);
%%\end{scope}
%%\draw[dotted] (z_t) -- (C_bottom);
%
%% Add labels for object and camera
%\node[above right] at (C_top) {$C_{w_t}(c_t)$};
%%\begin{axis}[
%%  axis lines=center,
%%  axis on top,
%%  %xlabel={$x$}, ylabel={$y$}, zlabel={$t$},
%%  domain=1:1,
%%  y domain=0:2*pi,
%%  xmin=-1.5, xmax=1.5,
%%  ymin=-1.5, ymax=1.5, zmin=0.0,
%%        %every axis x label/.style={at={(rel axis cs:0,0.5,0)},anchor=south},
%%        %every axis y label/.style={at={(rel axis cs:0.5,0,0)},anchor=north},
%%        %every axis z label/.style={at={(rel axis cs:0.5,0.5,0.9)},anchor=west},
%%  samples=30]
%%  \addplot3 [surf, colormap/blackwhite, shader=flat] ({x*cos(deg(y))},{x*sin(deg(y))},{x});
%%    
%%    %\addplot3[surf, samples=20, domain=-1:1, y domain=-1:1] {x^2 + y^2}; % Example surface plot
%%
%%\end{axis}
%\end{tikzpicture}
%\caption{Figure representing the cone $C_{w_t}(c_t)$ that contains the convex hull of $m_t$ and $S_{t}$ with unit vector $w_t$.} 
%%$u$ is a unit vector perpendicular to $H_u$ an hyperplane that is a supporting hyperplane $C_t$ at $m_t$ such that $\cC_t \cap H_u = \{z_t\}$ and 
%%$u^T (a_t-m_t)\ge 0$ }
%\label{fig:anglewidthmain}
%\end{center}
%\end{figure}

\begin{figure*}
\begin{center}
\includegraphics[width=10cm,keepaspectratio,angle=0]{Fig-ConeDef.png}
\caption{Figure representing the cone $C_{w_t}(c_t)$ that contains the convex hull of $m_t$ and $S_{t}$ with unit vector $w_t$.} 
\label{fig:anglewidthmain}
\end{center}
\end{figure*}


 
Essentially, $2\cos^{-1}(c^\star_t)$ is the angle width of $\cC_t$ with respect to $w_t^\star$, i.e. each element of $\cC_t$ makes an  angle of at most $ \cos^{-1}(c^\star_t)$ with $w_t^\star$.  
 
 

%\begin{rem} Instead of 
%\end{rem}
 
% \begin{rem} Theorem \ref{thm:manselli} obtains a universal result on $M$ by exploiting the self-expanded property that implies that $c^\star_t \ge \frac{1}{\sqrt{d}}$ (VI \cite{Manselli}) independent of the 
% shape of the convex bodies. Even when $x_t$ and $x_{t+1}$ are arbitrary points from $S_{t-1}$ and $S_{t}$ as is the considered case, it is still true that each sub-curve $(x_t, y_t)$ has the self-expanded property with respect to all the sub-curves $(x_\tau, y_\tau), \ \tau\ge t$ but not the whole curve $x_1,y_1, x_2,y_2, \dots, x_t,y_t,x_{t+1},y_{t+1}, \dots$.
% \end{rem}

\begin{rem}\label{rem:cbound}
Note that $c_t^\star$ is only a function of the distance $ ||x_t-b_t||$   and the shape of $S_t$'s, in particular, the maximum width of $S_t$ 
 along the directions perpendicular to vector $x_t-b_t$ $\forall \ t$ which can be at most the diameter $D$. 
 $c_t^\star$ decreases (increasing the ``width" of cone $C_{w_t^\star}(c_t^\star)$) as $||x_t-b_t||$ decreases, but small $x_t-b_t$ also implies small  violation at time $t$ from \eqref{defn:genconvxmovement}.
Across time slots, $d_{\min} = \min_t ||x_t-b_t||$ and shape of $S_t$'s control $c^\star$, where $d_{\min} > 0$ is inherent from the definition of $c^\star$ since a bound on $||x_t-b_t||$ is only needed for the case when $x_t\ne b_t$. 
\end{rem} 
 %Interestingly, if $d_{\min}$ is small then $c^\star$ but then $\text{CCV}_{[1:t]}$ in \eqref{defn:genconvx

\begin{rem}  Projecting $x_t\in \partial S_{t-1}$ onto $S_t$ to get $b_t=\cP_{S_t}(x_t)$, the diameter of $S_t$ is at most diameter of $S_{t-1} - ||x_t-b_t||$, however, only along the direction $b_t-x_t$. Since the shape of $S_{t}$ is arbitrary, as a result, the diameter of $S_t$ need not be smaller than the diameter of $S_{t-1}$ along any pre-specified direction, which was the main idea used to derive Lemma \ref{lem:spheres}.  Thus, to relate the distance $||x_t-b_t||$ with the decrease in the diameter of the convex bodies $S_t$'s, we use the concept of {\bf mean width} of a convex body that is defined as the expected width of the convex body along all the directions that are chosen uniformly randomly (formal definition is provided in Definition \ref{defn:avgwidth}).
\end{rem}

Next, we upper bound $M_T$ by connecting the distance $||x_t-b_t||$ to the decrease in mean width (to be defined ) of convex bodies $S_{t-1}$ and $S_t$'s.
 
 \begin{lemma}\label{lem:movementcost}
 The total movement cost 
$M_T$ in \eqref{defn:totalmovementcost1} is at most $$\frac{2V_d(d-1)}{V_{d-1}} \left(\frac{1}{c^\star}\right)^{d}D,$$ where 
$V_d$ is the $(d-1)$-dimensional Lebesgue measure of  the unit sphere in $d$ dimensions. 
\end{lemma}
%By definition, $c^\star <1$. Thus as $c^\star$ decreases, $M_T$ increases exponentially with exponent being $d$. 

Note that $V_d/V_{d-1} = O(1/\sqrt{d})$.
 Thus, we get the following {\bf main result} of the paper for Algorithm \ref{coco_alg_1} combining Lemma \ref{lem:regretbound} and Lemma \ref{lem:movementcost}. 
 \begin{theorem}\label{thm:main1}
For solving COCO, Algorithm \ref{coco_alg_1} has  $$\textrm{Regret}_{[1:T]} = O(\sqrt{T}), \ \text{and} \ \text{CCV}_{[1:T]}= O\left(\sqrt{d} \left(\frac{1}{c^\star}\right)^{d}D\right).$$
\end{theorem}

%\begin{rem} Specializing $a_t=x_t$ in \eqref{defn:genconvxmovement}, where $x_t$ is the action chosen by Algorithm \ref{coco_alg_1}, we will get a better bound on the $\text{CCV}_{[1:T]}$ via improved bound for $c^\star$, however, that will be both algorithm and instance dependent.
%\end{rem}
Compared to all prior results on COCO, that were universal (instance independent), where the best known one \cite{Sinha2024} has $\textrm{Regret}_{[1:T]} = O(\sqrt{T})$, and $\text{CCV}_{[1:T]}=O(\sqrt{T}\log T)$, Theorem \ref{thm:main1} is an instance 
dependent result for the CCV. In particular, it exploits the geometric structure of the nested convex sets $S_t$'s and 
derives an upper bound on the CCV that only depends on the `shape' of $S_t$'s. It can be the case that the instance is `badly' behaved and $c^\star$ is very small or dependent on $T$. If that is the case, in Section \ref{sec:algswitch} we show how to limit the CCV to $O(\sqrt{T}\log T)$. However, when $S_t$'s are `nice', e.g., $c^\star$ is independent of $T$ (Remark \ref{rem:cbound}) or $S_t$'s are spheres or axis parallel cuboids (Lemma \ref{lem:spheres} and \ref{lem:square}), the $\text{CCV}$ of Algorithm \ref{coco_alg_1} is independent of $T$, which is a fundamentally improved result compared to large body of prior work. In fact, in prior work this was largely assumed to be not possible. In particular, before the result of \cite{Sinha2024}, achieving simultaneous $\textrm{Regret}_{[1:T]} = O(\sqrt{T})$, and $\text{CCV}_{[1:T]}=O(\sqrt{T})$ itself was the final goal.




\section{Algorithm $\mathrm{Switch}$}\label{sec:algswitch}

Since Theorem \ref{thm:main1} provides an instance dependent bound on the CCV, that is a function of $c^\star$ which can be small, it can be the case that its CCV is larger than $O(\sqrt{T}\log T)$, thus providing a result that is inferior to that of Algorithm \ref{coco_sinha} \cite{Sinha2024}. Thus, next, we marry the two algorithms, Algorithm \ref{coco_sinha} and Algorithm \ref{coco_alg_1}, in Algorithm \ref{alg:switch} to provide a best of both results as follows. 


\begin{algorithm}[tb]
   \caption{$\mathrm{Switch}$}
   \label{alg:switch}
\begin{algorithmic}[1]
   \State {\bfseries Input:} Sequence of convex cost functions $\{f_t\}_{t=1}^T$ and constraint functions $\{g_t\}_{t=1}^T,$ $G=$ a common Lipschitz constant,  $d$ dimension  of the admissible set $\mathcal{X},$
   %an upper bound $G$ to the Euclidean norm of their (sub)gradients, 
    $D=$ Euclidean diameter of the admissible set $\mathcal{X},$ $\mathcal{P}_\mathcal{X}(\cdot)=$ Euclidean projection operator on the set $\mathcal{X}$,      \State {\bfseries Initialization:} Set $ x_1 \in \mathcal{X}$ arbitrarily, $\text{CCV}(0)=0$.
   \State {\bf For} \ {$t=1:T$}
   \State \quad {\bf If} {$\text{CCV}(t-1) \le \sqrt{T}\log T$}
   \State \quad \quad Follow Algorithm \ref{coco_alg_1}
   \State  \quad  \quad $\text{CCV}(t)=\text{CCV}(t-1)+\max\{g_t(x_t),0\}.$
   \State \quad {\bf Else} 
   \State \quad \quad Follow Algorithm \ref{coco_sinha} with resetting $\text{CCV}(t-1)=0$
   \State \quad {\bf EndIf} 
   \State {\bf EndFor}
\end{algorithmic}
\end{algorithm}

 \begin{theorem}\label{thm:main2}
$\mathrm{Switch}$ (Algorithm \ref{alg:switch}) has regret $\textrm{Regret}_{[1:T]} =O(\sqrt{T})$, while    $$\text{CCV}_{[1:T]}=
\min\left\{O\left(\sqrt{d} \left(\frac{1}{c^\star}\right)^{d}D\right), O(\sqrt{T}\log T)\right\}.$$
\end{theorem}
 
Algorithm $\mathrm{Switch}$ should be understood as the best of two worlds algorithm, where the two worlds corresponds to one having nice convex sets $S_t$'s that have CCV independent of $T$ or $o(\sqrt{T})$ for 
Algorithm \ref{coco_alg_1}, while in the other, CCV of Algorithm \ref{coco_alg_1} is large on its own, and the overall CCV is controlled by discontinuing the use of Algorithm \ref{coco_alg_1} once its CCV reaches $\sqrt{T}\log T$ and switching to Algorithm \ref{coco_sinha} thereafter that has universal guarantee of $O(\sqrt{T}\log T)$ on its CCV.
 
 After exhausting the general results on the CCV of Algorithm \ref{coco_alg_1}, we next consider the special case of $d=2$ and when the sets $S_t$ have a special structure defined by their projection hyperplanes. 
 Note that it is highly non-trivial to bound the CCV of Algorithm \ref{coco_alg_1} even when $d=2$.
 
  
 %\begin{remark}
 %\end{remark}
