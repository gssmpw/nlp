\section{Special case of $d=2$}
In this section, we show that if $d=2$ (all convex sets $S_t$'s lie in a plane) and the projections satisfy a monotonicity property depending on the problem instance, then we can bound the total CCV for Algorithm \ref{coco_alg_1} independent of the time horizon $T$ and consequently getting a $O(1)$ CCV. 




Recall from the definition of Algorithm \ref{coco_alg_1}, $y_t = \cP_{S_{t-1}}(x_t - \eta_t \nabla f_t(x_t))$ and 
$x_{t+1} = \cP_{S_t}(y_t)$.


\begin{definition}\label{defn:projhyperplane}
Let the hyperplane perpendicular to line segment $(y_t, x_{t+1})$ passing through $x_{t+1}$ be 
$F_t$. Without loss of generality, we let $y_t \notin S_t$, since then the projection is trivial. Essentially $F_t$ is the projection hyperplane at time $t$. %Let $N_t$ be the normal to $F_t$, then we
Let $\cH_t^+$ denote the positive half plane corresponding to $F_t$, i.e., 
$\cH_t^+ = \{z: z^T (y_t-x_{t+1})\ge 0\}$. 
Refer to Fig. \ref{fig:defF}.
Let the angle between $F_1$ and $F_t$ be $\theta_t$. 
\end{definition}
\begin{figure}


\includegraphics[width=10cm,keepaspectratio,angle=0]{FigDefF.pdf}


\caption{Definition of $F_t$'s.}
\label{fig:defF}
\end{figure}


\begin{definition}\label{defn:anglemonotone}
The instance $S_1 \supseteq S_2 \supseteq \dots \supseteq S_T$ is defined to be monotonic 
if $\theta_2 \le \theta_3 \le \dots \le \theta_T$.
\end{definition}




\begin{theorem}\label{thm:tvmonotone}
For $d=2$ when the instance is monotonic, $\text{CCV}_{[1:T]}$ for Algorithm \ref{coco_alg_1} is at most $O(GD)$.
\end{theorem}

Theorem \ref{thm:tvmonotone} provides a universal guarantee on the CCV of  Algorithm \ref{coco_alg_1} that is independent of the problem instance (as long as it is monotonic) unlike Lemma \ref{lem:movementcost}, even though it applies only for $d=2$. The proof is derived by using basic convex geometry results from \cite{Manselli} in combination with exploiting the definition of Algorithm \ref{coco_alg_1} and the monotonicity condition. It is worth noting that even under the monotonicity assumption it is non-trivial to upper bound the CCV since the successive angles made by $F_t$ with $F_1$ can increase arbitrarily slowly, making it difficult 
to control the total CCV. 
%It shows that the COCO problem when has structure can be exploited to get $O(1)$ CCV 
\section{OCS Problem}
In \cite{Sinha2024}, a special case of COCO, called the OCS problem, was introduced where $f_t\equiv0$ for all $t$. Essentially, with OCS, only constraint satisfaction is the objective.  In \cite{Sinha2024}, Algorithm \ref{coco_sinha} was shown to have CCV of $O(\sqrt{T}\log T)$. Next, we show that Algorithm \ref{coco_alg_1} has CCV of $O(1)$ for the OCS, a remarkable improvement.

 \begin{theorem}\label{thm:ocs}
For solving OCS, Algorithm \ref{coco_alg_1} has $\text{CCV}_{[1:T]}= O\left(d^{d/2} D\right)$.
\end{theorem}

As discussed in \cite{Sinha2024}, there are important applications of OCS, and it is important to find tight bounds on its CCV. Theorem \ref{thm:ocs} achieves this by showing that CCV of $O(1)$ can be achieved, where the constant depends only on the dimension of the action space and the diameter. This is a fundamental improvement compared to the CCV bound of $O(\sqrt{T}\log T)$ from \cite{Sinha2024}. Theorem  \ref{thm:ocs} is derived by using the connection between the curve obtained by successive projections on nested convex sets and self-expanded curves (Definition \ref{defn:se-curve}) and then using a classical result on self-expanded curves from \cite{Manselli}.

 