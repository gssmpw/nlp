%\section{Proof of Theorem \ref{thm:tvmonotone}}
\section{Preliminaries for Bounding the CCV in Theorem \ref{thm:ocs} and Theorem \ref{thm:tvmonotone}}
% 
% \begin{definition}\label{defn:avgwidth}
%Let $K$ be a non-empty convex bounded set in $\bbR^d$. Let $u$ be a unit vector, and $\ell_u$ a line through the origin parallel to $u$. 
%Let $K_u$ be the orthogonal projection of $K$ onto $\ell_u$, with length $|K_u|$. The mean width of $K$ is defined as 
%\begin{equation}\label{eq:projlength}
%W(K) = \frac{1}{V_d} \int_{S_1^d} |K_u| du,
%\end{equation}
%where $S_1^d$ is the unit sphere in $d$ dimensions and $V_d$ its $(d-1)$-dimensional Lebesgue measure.
%\end{definition}
%
%
%The following is immediate. 
%\begin{equation}\label{eq:WBound1}
%0\le W(K) \le \text{diameter}(K).
%\end{equation}
%
%
%\begin{lemma}\label{lem:width2D1}
%For $d=2$, $$W(K)=\frac{\text{Perimeter}(K)}{\pi}.$$
%\end{lemma}

Let $K_1, \dots, K_T$ be nested (i.e., $K_1 \supseteq K_2 \supset K_3 \supseteq \dots \supseteq K_T$) bounded convex subsets of $\bbR^d$. 

%Let the minimum distance between $K_i$ and $K_{i+1}$ be $d_{i,i+1}$ and 
%\begin{equation}
%d_{\min} = \min_i d_{i,i+1}.
%\end{equation}
\begin{definition}\label{defn:projectioncurve}
If $\sigma_1\in K_1$, and $\sigma_{t+1} = \cP_{K_{t+1}}(\sigma_t)$, for $t=1, \dots, T$. Then the curve 
$${\underline \sigma}= \{(\sigma_1,\sigma_2), (\sigma_2,\sigma_3), \dots, (\sigma_{T-1},\sigma_T)\}$$ is called the projection curve on $K_1, \dots, K_T$.
\end{definition}
%Let $x_t \in \partial K_t$ and $y_t$ be the projection of $x_t$ on set $K_{t+1}$.  


We are interested in upper bounding the quantity 
\begin{equation}\label{eq:totalDistance}
\Sigma = \max_{{\underline \sigma}} \sum_{t=1}^{T-1} ||\sigma_t - \sigma_{t+1}||.
\end{equation}
%If $x_t\in K_{t+1}$ then $||x_t - y_t||=0$.

 \begin{lemma}\label{lem:projection}
For a projection curve ${\underline \sigma}$, $\Sigma \le d^{d/2} \text{diameter}(K_1)$.
\end{lemma}


To prove the result we need the following definition.

\begin{definition}\label{defn:se-curve} A curve $\gamma: I \rightarrow \bbR^d$  is called self-expanded, if for every $t$ where 
$\gamma'(t)$ exists, we have 
$$< \gamma'(t), \gamma(t)-\gamma(u)> \ \ge 0$$ for all $u\in I$ with $u \le t$, where $<.,.>$ represents the inner product. 
In words, what this means is that $\gamma$ starting in a point $x_0$ is self expanded, if for every $x\in \gamma$ for which there exists the tangent line $\sfT$, the arc (sub-curve) $(x_0, x)$ is
contained in one of the two half-spaces, bounded by the hyperplane through
$x$ and orthogonal to $\sfT$. 
\end{definition}
For self-expanded curves the following classical result is known.
\begin{theorem}\label{thm:manselli}\cite{Manselli}
For any self-expanded curve $\gamma$ belonging to a closed bounded convex set of $\bbR^d$ with diameter $D$, its total length is at most $O(d^{d/2} D)$.
\end{theorem}
\begin{proof}[Proof of Lemma \ref{lem:projection}]
From Definition \ref{defn:projectioncurve}, the projection curve is 
$${\underline \sigma}=\{(\sigma_1,\sigma_2), (\sigma_2,\sigma_3), \dots, (\sigma_{T-1},\sigma_T)\}.$$ Let the reverse curve be ${\underline r} = \{r_t\}_{t=0, \dots, T-2}$, where $r_t = (\sigma_{T-t}, \sigma_{T-t-1})$. Thus we are reading ${\underline \sigma}$ backwards and calling it ${\underline r}$. Note that since $\sigma_{t}$ is the projection of $\sigma_{t-1}$ on $K_t$, each piece-wise linear segment $(\sigma_t, \sigma_{t+1})$ is a straight line and hence differentiable except at the end points. Moreover, since each $\sigma_t$ is obtained by projecting $\sigma_{t-1}$ onto $K_t$ and $K_{t+1}\subseteq K_t$, we have that the projection hyperplane 
$F_t$ that passes through $\sigma_t=\cP_{K_t}(\sigma_{t-1})$ and is perpendicular to $\sigma_t - \sigma_{t-1}$ separates the two sub curves $\{(\sigma_1,\sigma_2), (\sigma_2,\sigma_3), \dots, (\sigma_{t-1},\sigma_t)\}$ and $\{(\sigma_t,\sigma_{t+1}), (\sigma_{t+1},\sigma_{t+2}), \dots, (\sigma_{T-1},\sigma_T)\}$.


Thus, we have that 
for each segment $r_\tau$, at each point where it is differentiable, the curve $r_1, \dots r_{\tau-1}$ lies on one side of the hyperplane that passes through the point and is perpendicular to $ r_{\tau+1}$. Thus, we conclude that curve ${\underline r}$ is self-expanded.

% apply the result from \cite{Manselli} that bounds the total distance of a self-expanded curve belonging to a closed bounded convex set, to get the result.  

As a result, Theorem \ref{defn:se-curve} implies that the length of ${\underline r}$ is at most $O(d^{d/2} \text{diameter}(K_1))$, and the result follows since the length of ${\underline r}$ is same as that of ${\underline \sigma}$ which is $\Sigma$. 
\end{proof}

\section{Proof of Theorem \ref{thm:ocs}}
Clearly, with $f_t\equiv0$ for all $t$, with Algorithm \ref{coco_alg_1}, $y_t=x_t$ and the successive $x_t$'s are such that $x_{t+1} = \cP_{S_t}(x_{t})$. Thus, essentially, the curve ${\underline x} = (x_1, x_2), (x_2,x_3), \dots, (x_{T-1}, x_{T})$ formed by Algorithm \ref{coco_alg_1} for OCS is a projection curve (Definition \ref{defn:projectioncurve}) on $S_1\supseteq, \dots, \supseteq S_T$ and the result follows from Lemma \ref{lem:projection} and the fact that $\text{diameter}(S_1)\le D$.

%\begin{lemma}\label{lem:meanwidthfactorize}
%Let $K_1, K_2 \subseteq K$ be such that Lebesgue measure of $K_1\cap K_2=0$ and $K_1,K_2$ are convex.
%Then 
%\begin{equation}\label{}
%W(K_1)+W(K_2) \le W(K).
%\end{equation}
%\end{lemma}
\section{Proof of Theorem \ref{thm:tvmonotone}} 

%\begin{figure}
%
%\includegraphics[width=10cm,keepaspectratio,angle=0]{FigPhaseAnalysis.pdf}
%
%\caption{Depiction of the definition of phases and related quantities for a monotonic instance, where phase $1$ has $t^\star(1)=4$, thus curve till $F_4$ has been explored in phase $1$.  The next phase, phase $2$ is empty, since the angle between $F_4$ and $F_5$ is more than $\pi/4$. Phase $3$ begins from $F_5$ as its first hyperplane by setting $s(3)=5$.}
%\label{fig:monotone}
%\end{figure}





\begin{proof}
Recall that $d=2$, and the definition of $F_t$ from Definition \ref{defn:projhyperplane}. Let the center be $\sfc=\cP_{S_1}(x_1)$.  Let $t_{\text{orth}}$ be the earliest $t$ for which $\angle (F_t, F_1) = \pi$.

Initialize $\kappa=1$, $s(1)=1$, $\tau(1) =1$. 



{\bf BeginProcedure}
Step 1:Definition of Phase $\kappa$.
Consider $$\tau(\kappa) = \arg \max_{s(\kappa)< t \le t_{\text{orth}}, \angle(F_{s(\kappa)}, F_t) \le \pi/4} t.$$

{\bf If there is no such $\tau(\kappa)$}, 

\quad Phase $\kappa$ ends, define Phase $\kappa$ as {\bf Empty},  $s(\kappa+1) =  \tau(\kappa)+1$.


{\bf Else If} 

\quad $\angle(F_{\tau(\kappa)}, F_1)=\pi$ Exit

{\bf Else If} 

\quad $s(\kappa+1)=\tau(\kappa)$

{\bf End If}

Increment $\kappa=\kappa+1$,  and Go to Step 1.

{\bf EndProcedure}

\begin{example}\label{exm:phasedef} To better understand the definition of phases, consider Fig. \ref{fig:phases}, where the largest $t$ for which the angle between $F_t$ and $F_1$ is at most $\pi/4$ is $3$. Thus, $\tau(1)=3$, i.e., phase $1$ explores till time $t=3$ and phase $1$ ends. The starting hyperplane to consider in phase $2$ is $s(2)=3$ 
and given that angle between $F_3$ and and the next hyperplane $F_4$ is more than $\pi/4$, phase $2$ is empty and phase $2$ ends by exploring till $t=4$. The starting hyperplane to consider in phase $3$ is $s(3)=4$ and the process goes on. The first time $t$ such that the angle between $F_1$ and $F_t$ is $\pi$ is $t=6$, and thus $t_{\text{orth}}=6$, and the process stops at time $t=6$. 
This also implies that $S_6 \subset F_1$. 
Since $S_t$'s are nested, for all $t\ge 6$, $S_t\subset F_1$. Hence the total CCV after $t\ge t_{\text{orth}}$ is at most $GD$.
\end{example}

%Essentially, $\tau(1)$ is the largest time $t$ by which the angle between $F_1$ and $F_t$ is at most $\pi/4$. If there is no such $t$, then the angular region making angle of $\pi/4$ with $F_1$ is defined to the Empty. Thus, in phase $1$ hyperplanes till $F_{\tau(1)}$ are explored. The next phase begins by resetting $F_1$ as $F_{\tau(1)}$ by incrementing $s(\kappa) = \tau(\kappa)$. 
The main idea with defining phases, is to partition the whole space into empty and non-empty regions, where in each non-empty region, the starting and ending hyperplanes have an angle to at most $\pi/4$, while in an empty phase the starting and ending hyperplanes have an angle of at least $\pi/4$. Thus, we get the following simple result.

\begin{figure}


\includegraphics[width=15cm,keepaspectratio,angle=0]{Fig-phase.png}
\caption{Figure corresponding to Example \ref{exm:phasedef}.}
\label{fig:phases}
\end{figure}



\begin{lemma}\label{lem:nrphases} For $d=2$, there can be at most $4$ non-empty and $4$ empty phases.  
%that the number of phases (counting both non-empty and empty) till time $t_{\text{orth}}$ is at most $4$. 
\end{lemma}
Proof is immediate from the definition of the phases, since any consecutively occurring non-empty and empty phase exhausts an angle of at least $\pi/4$.

\begin{rem}\label{rem:aftertorth}
Since we are in $d=2$ dimensions, for all $t\ge t_{\text{orth}}$, the movement is along the hyperplane $F_1$ and thus the resulting constraint violation after time $t\ge t_{\text{orth}}$ is at most $GD$. Thus, in the phase definition above, we have only considered time till $t_{\text{orth}}$ and we only need to upper bound the CCV till time $t_{\text{orth}}$. \end{rem}





We next define the following required quantities.


\begin{definition}\label{defn:tstar}
With respect to the quantities defined for Algorithm \ref{coco_alg_1}, let for a non-empty phase $\kappa$ 
$$r_{\max}(\kappa)= \max_{s(\kappa) < t\le \tau(\kappa)} || y_t - \sfc||\ \text{and} \ t^\star(\kappa) = \arg \max_{s(\kappa) < t\le \tau(\kappa)}^T || y_t- \sfc||.$$
%Thus, a non-empty phase $\kappa$ consists of time slots $\cT(\kappa) = [t^\star(\kappa) - t^\star(\kappa-1), \tau(\kappa)]$ and $t^\star(\kappa) \in \cT(\kappa)$, and the angle $\angle(F_{t_1}, F_{t_2}) \le \pi/4$ when $t_1,t_2\in \cT(\kappa)$.
%
%If phase $\kappa$ is empty then $t^\star(\kappa)=t^\star(\kappa-1)+1$ %Consider the ball $\cB_1$ 
%with center $c$ as the projection of $x_1$ onto $S_1$ and radius $r_{\max}$. 
\end{definition}
$t^\star(\kappa)$ is the time index belonging to phase $\kappa$ for which $y_t$ is the farthest.

\begin{definition} 
A non-empty phase $\kappa$ consists of time slots $\cT(\kappa) = [\tau(\kappa-1), \tau(\kappa)]$ and the angle $\angle(F_{t_1}, F_{t_2}) \le \pi/4$ for all $t_1,t_2\in \cT(\kappa)$. Using Definition \ref{defn:tstar}, we partition $\cT(\kappa)$ as $\cT(\kappa) = \cT^-(\kappa) \cup \cT^+(\kappa)$, where $\cT^-(\kappa) = [\tau(\kappa-1)+1, t^\star(\kappa)+1]$ and $\cT^+(\kappa) = [ t^\star(\kappa)+2, \tau(\kappa)]$.
\end{definition}

Thus, $\cT(\kappa)$ and $\cT(\kappa+1)$ have one common time slot.

\begin{figure}
\includegraphics[width=15cm,keepaspectratio,angle=0]{Fig-newviolation.png}
\caption{Illustration of definition of $z_t(\kappa)$ for $t\in \cT(\kappa)$. In this example, for phase $1$, $t^\star(1)=3$ since the distance of $y_3$ from $\sfc$ is the farthest for phase $1$ that consists of time slots 
$\cT(1) = \{2,3\}$. Hence $z_{t^\star(1)+1}(1)=x_4$. For $t \in \cT(1) \backslash  t^\star(1)+1$,  $z_{t}(1)$ are such 
$z_{t+1}(1)$ is a projection of $z_{t}(1)$ onto $F_t$.}
\label{fig:MaxR}
\end{figure}
\begin{definition}\label{}
[Definition of $z_t(\kappa)$ \  for $t\in \cT^-(\kappa)$]. Let 
$z_{t^\star(\kappa)+1} = x_{t^\star(\kappa)+1}$.
For $t \in \cT^-(\kappa) \backslash t^\star(\kappa)+1$, define $z_t(\kappa)$ inductively as follows. 
$z_t(\kappa)$ is the pre-image of $z_{t+1}(\kappa)$ on $F_{t-1}$ such that the projection of $z_t(\kappa)$ on $F_t$ 
is $z_{t+1}(\kappa)$. 
%Consider the hyperplane $F_t$ as defined before. Extend this line till it intersects with ball $\cB_1 \cap \chi$, and call it $L_i'$. $S_i'$ is convex hull of $L_i' \cup S_{i+1}$. 
\end{definition}

\begin{definition}\label{}
[Definition of $z_t(\kappa)$ \  for \ $t\in \cT^+(\kappa)$]. 
For $t\in \cT^+(\kappa)$, define $z_t(\kappa)$ inductively as follows. 
$z_t(\kappa)$ is the projection of $z_{t-1}(\kappa)$ on $F_{t-1}$. 
%Consider the hyperplane $F_t$ as defined before. Extend this line till it intersects with ball $\cB_1 \cap \chi$, and call it $L_i'$. $S_i'$ is convex hull of $L_i' \cup S_{i+1}$. 
\end{definition}

See Fig. \ref{fig:MaxR} for a visual illustration of $t^\star(\kappa)$ and $z_t(\kappa)$.

The main idea behind defining $z_t(\kappa)$'s  is as follows. For each non-empty phase, we will construct a projection curve (Definition \ref{defn:projectioncurve}) using points $z_k$ such that the length of the projection curve upper bounds the CCV of Algorithm \ref{coco_alg_1} (shown in Lemma \ref{lem:violationub}), and then use Lemma \ref{lem:projection} to upper bound the length of the projection curve. %For an empty phase, the CCV of Algorithm \ref{coco_alg_1} is at most $D$, and there are at most $4$ empty phases. Thus, we will get the required bound.

\begin{definition}\label{}
[Definition of $S_t'$ for a non-empty phase $\kappa$:]  $S_{t^\star(\kappa)+1}' = S_{t^\star(\kappa)+1}$.
For $t \in \cT^-(\kappa) \backslash t^\star(\kappa)+1$, 
$S_t'$ is the convex hull of $z_{t+1}(\kappa) \cup S_t \cup S'_{t+1}(\kappa)$. For $t\in \cT^+(\kappa)$, 
$S_t' =S_t$.
See Fig. \ref{fig:defSprime}.
\end{definition}

\begin{lemma}\label{lem:nestedprime} For a non-empty phase $\kappa$, for any $t \in \cT(\kappa)$, $S_{t+1}' \subseteq S_t' $, i.e. they are nested.
\end{lemma}
\begin{figure}
\includegraphics[width=10cm,keepaspectratio,angle=0]{FigSprime.pdf}
\caption{Definition of $S_t$'s where $U_t$ are the extra regions that are added to $S_t$ to get $S_t'$.}
\label{fig:defSprime}
\end{figure}

\begin{definition} For a non-empty phase, 
 $\chi(\kappa) = S_{\tau(\kappa-1)}'  \cap \cH_{\tau(\kappa)}^+$, where $\cH_{\tau(\kappa)}^+$ has been defined in Definition \ref{defn:projhyperplane}.

\end{definition}

\begin{definition}\label{}
[New Violations for  $t\in \cT(\kappa)$:] 
For a non-empty phase $\kappa$, for $t\in \cT(\kappa) \backslash \tau(\kappa-1)$, let 
$$v_t(\kappa) = ||z_t(\kappa)-z_{t-1}(\kappa)||.$$
\end{definition}
%Phase $\kappa$ ends and increment $\kappa=\kappa+1$, reset $s(\kappa) =  t^\star(\kappa)$ $c=x_{t^\star(\kappa)}$ and Go to Step 1. 

%
%\begin{definition}
%For an empty phase $\kappa$, $\chi(\kappa) = S_{t^\star(\kappa-1)}' \cap \cH_{t^\star(\kappa-1)}^+$.
%\end{definition}
%Refer to Fig. \ref{fig:monotone} for a pictorial description of a two-dimensional monotonic instance.


\begin{lemma}\label{lem:membership} For each non-empty phase $\kappa$, all $z_{t}(\kappa)$'s for $t\in \cT(\kappa)$ belongs to $\cB(\sfc, \sqrt{2}D)$, where $\cB(c,r)$ is a ball with radius $r$ centered at $c$. In other words, $\chi(\kappa) \subseteq \cB(\sfc, \sqrt{2}D)$.
\end{lemma}
\begin{proof}
Recall that for a non-empty phase $\kappa$,  $\cT(\kappa) = \cT^-(\kappa) \cup  \cT^+(\kappa).$ We first argue about $t\in \cT^-(\kappa)$.
By definition, $z_{t^\star(\kappa)+1} = x_{t^\star(\kappa)+1}$ and $x_{t^\star(\kappa)+1}\in S_{t^\star(\kappa)}$. Thus, $z_{t^\star(\kappa)+1} \in \cB(\sfc, \sqrt{2}D)$.
Next we argue for $t \in \cT^-(\kappa) \backslash t^\star(\kappa)+1$.
Recall that the diameter of $\cX$ is $D$, and the fact that $y_t \in S_{t-1}$ from Algorithm \ref{coco_alg_1}. Thus, for any non-empty phase $\kappa$, the distance from $\sfc$ to the farthest $y_t$ belonging to the phase $\kappa$ is at most $D$, i.e., $r_{\max}(\kappa)\le D$. 
Let the pre-image of $z_{t^\star(\kappa)+1}(\kappa)$ onto $F_{s(\kappa)}$ (the base hyperplane with respect to which all hyperplanes have an angle of at most $\pi/4$ in phase $\kappa$) be $p(\kappa)$ such that projection of $p(\kappa)$ onto $F_{s(\kappa)}$ is $z_{t^\star(\kappa)+1}(\kappa)$. 
From the definition of any non-empty phase, the angle between $F_{s(\kappa)}$ and $F_{t}$ for $t\in \cT(\kappa)$ is at most $\pi/4$. 
Thus, the distance of $p(\kappa)$ from $\sfc$ is at most $\sqrt{2}D$. 

%Thus, if $t^\star(\kappa)+1-s(\kappa)=1$, we are done. 



%When $t^\star(\kappa)+1-s(\kappa)>1$, we proceed as follows. 
Consider the `triangle' $\Pi(\kappa)$ that is the convex hull of $\sfc, z_{t^\star(\kappa)+1}(\kappa)$ and $p(\kappa)$.
Given that the angle between $F_{t^\star(\kappa)}$ and $F_{t^\star(\kappa)-1}$ is at most $\pi/4$, the argument above implies that 
$z_t(\kappa) \in \Pi(\kappa)$ for $t=t^\star(\kappa)$. For $t= t^\star(\kappa)-1$, $z_t(\kappa) \in F_{t-1}$ is the projection of  $z_{t-1}(\kappa)$ onto $S_{t-1}'$. This implies that the distance of $z_t(\kappa)$ (for $t=t^\star(\kappa)-1$) from $\sfc$ is at most 
$$\frac{D}{\cos(\alpha_{t, t^\star(\kappa)}) \cos(\alpha_{t^\star(\kappa), t^\star(\kappa)+1})},$$ where 
$\alpha_{t_1,t_2}$ is the angle between $F_{t_1}$ and $F_{t_2}$.
From the monotonicity of angles $\theta_t$ (Definition \ref{defn:anglemonotone}), and the definition of a non-empty phase, we have that $\alpha_{t, t^\star(\kappa)}+\alpha_{t^\star(\kappa), t^\star(\kappa)+1} \le \pi/4$ and $\alpha_{t, t^\star(\kappa)}\ge 0, \alpha_{t^\star(\kappa), t^\star(\kappa)+1}\ge 0$.
Next, we appeal to the identity
\begin{equation}\label{eq:cosidentity}
\cos(A+B) \le \cos(A)\cos(B)
\end{equation} where $A+B\le \pi/4$, to claim that $z_t(\kappa) \in \Pi(\kappa)$ for $t=t^\star(\kappa)-1$. 

Iteratively using this argument while invoking the identity \eqref{eq:cosidentity} gives the result that for any $t \in \cT^-(\kappa)$, we have that $z_{t}(\kappa)$  belongs to $\Pi(\kappa)$. Since $\Pi(\kappa) \subseteq \cB(\sfc, \sqrt{2}D)$, we have the claim for all $t\in \cT^-(\kappa)$. 
%$t^\star(\kappa) - t^\star(\kappa-1)\le t \le t^\star(\kappa)$.


%
%
%Recall that for $t^\star(\kappa) - t^\star(\kappa-1)\le t\le t^\star(\kappa)$ 
%we have that $z_t(\kappa) \in F_{t-1}$ and is the projection of  $z_{t-1}(\kappa)$ onto $S_t'$.  Moreover, angle between any $F_t$ and $F_{t^\star(\kappa)}$ is at most $\pi/4$. Thus, as above $z_{t}(\kappa)$ for $t=t^\star(\kappa)-1$ 
%Thus, following the identity that 
%
%
%the monotonicity property of angles $\theta_t$ (Definition \ref{defn:anglemonotone}) and the non-expansive property of convex projections\footnote{Distance from $z_{t}(\kappa)$ to $z_{\tau(\kappa)}(\kappa)$  decreases as $t$ increases from $t^\star(\kappa) - t^\star(\kappa-1)\le t$ to $\tau(\kappa)$}, we have that $z_{t}(\kappa)$  belongs to $\Pi(\kappa)$. Since $\Pi(\kappa) \subseteq \cB(c, \sqrt{2}D)$, we have the claim for $t^\star(\kappa) - t^\star(\kappa-1)\le t \le t^\star(\kappa)$.

By definition $z_{t}(\kappa)$ for $t\in \cT^+(\kappa)$ belong to $S_{t-1}\subseteq S_1$. Thus, their distance from $\sfc$ is at most $D$. 
\end{proof}



\begin{lemma}\label{lem:violationub} For each non-empty phase $\kappa$, and for $t\in \cT(\kappa)$ the violation $v_t(\kappa)\ge  \text{dist}(x_t, S_t)$, where $\text{dist}(x_t, S_t)$ is the original violation.
\end{lemma}
\begin{proof}
By construction of any non-empty phase $\kappa$, for $t\in \cT(\kappa)$ both $x_t(\kappa)$ and $z_t(\kappa)$ belong to $F_{t-1}$. Moreover, by construction, the distance of $z_t(\kappa)$ from $\sfc$ is at least as much as the distance of $x_t$ from $\sfc$. Thus, using  the monotonicity property of angles $\theta_t$ (Definition \ref{defn:anglemonotone}) we get the result. See Fig. \ref{fig:MaxR} for a visual illustration.
\end{proof}



For each non-empty phase $\kappa$, by definition, the curve defined by sequence $z_{t}(\kappa)$ for $t\in\cT(\kappa)$ is a projection curve (Definition \ref{defn:projectioncurve}) on sets $S'_t(\kappa)$ (note that $S'_t(\kappa)$'s  are nested from Lemma \ref{lem:nestedprime}). Moreover, for all $t\in\cT(\kappa)$, set $S'_t(\kappa) \subset \chi(\kappa)$ which is a bounded convex set. 
Thus, for $d=2$ from Lemma \ref{lem:projection} the length of curve ${\underline z}(\kappa) = \{(z_{t}(\kappa), z_{t+1}(\kappa))\}_{t\in \cT(\kappa)}$
\begin{equation}\label{eq:totallengthprojection}
\sum_{t\in \cT(\kappa)} v_t(\kappa) \le 2
\text{diameter}(\chi(\kappa)).
\end{equation}


 

%Next, we make use of Lemma \ref{lem:width2D} that exactly characterizes the average width in $d=2$-dimensions. 
By definition, the number of non-empty phases till time $t_{\text{orth}}$ is at most $4$. Moreover, in each non-empty phase $\chi(\kappa) \subseteq \cB(\sfc, \sqrt{2}D)$ from Lemma \ref{lem:membership} . 

Thus, from  \eqref{eq:totallengthprojection}, we have that \begin{align}\nn
\sum_{\text{Phase} \ \kappa \ \text{is non-empty}} \ \ \ \sum_{t\in \cT(\kappa)} v_t(\kappa) &\le \sum_{\text{Phase} \ \kappa \ \text{is non-empty}} 2\ \text{diameter}(\chi(\kappa)) \\ \label{eq:summeanwidth}
&\le 8 \ \text{diameter}(\cB(\sfc, \sqrt{2}D))\le O(D).
\end{align}

Using Lemma \ref{lem:violationub}, we get 
\begin{align}\label{eq:finalviolation}
\sum_{\text{Phase} \ \kappa \ \text{is non-empty}} \ \ \ \sum_{t\in \cT(\kappa)}\text{dist}(x_t, S_t) \le O(D).
\end{align}

For any empty phase, the constraint violation is the length of line segment $(x_t,\cP_{S_t}(x_{t}))$ (Algorithm \ref{coco_alg_1}) crossing it is a straight line whose length is at most $O(D)$. 
%The length $||(y_t - x_{t+1})||$ of the curve is also an upper bound on the CCV incurred by Algorithm \ref{coco_alg_1} when crossing the two hyperplanes that define the empty phase. 
 Moreover, the total number of empty phases (Lemma \ref{lem:nrphases}) is a constant.
 %, so the total length of the violation curve (Algorithm \ref{coco_alg_1}) crossing all empty phases is $O(D)$. 
 Thus, the length of the curve $(x_t,\cP_{S_t}(x_{t}))$ for Algorithm \ref{coco_alg_1} corresponding to all empty phases is at $O(D)$.
%
%From Lemma \ref{lem:membership} it follows that  that $\cup_{\kappa} \chi_k \subseteq \cB(c, \sqrt{2}D)$. Moreover, $\chi_k \cap \chi_{k+1} = F_{t^\star(\kappa)}$ and $\chi_j \cap \chi_k = \phi$ for $|j- k|>1$. Since $F_{t^\star(\kappa)}$ is 
%a hyperplane that contributes zero mass to the integral in \eqref{eq:projlength}, we have that 
%
%\begin{equation}\label{eq:summeanwidth}
%\cup_{\kappa} W(\chi(\kappa)) \le W(\cB(c, \sqrt{2}D)) \stackrel{(a)}\le O(\sqrt{2}D).
%\end{equation}
%where $(a)$ follows from \eqref{eq:WBound1}.

Recall from \eqref{eq:distviolationrelation} that the CCV is at most $G$ times $\text{dist}(x_t, S_t)$.
Thus, from \eqref{eq:finalviolation} we get that the total violation incurred by Algorithm \ref{coco_alg_1} corresponding to non-empty phases is at most $O(GD)$, while corresponding to empty phases is at $O(GD)$.
Finally, accounting for the very first violation $\text{dist}(x_1, S_1)\le D$ and the fact that the CCV after time $t\ge t_{\text{orth}}$ (Remark \ref{rem:aftertorth}) is at most $GD$, we get that the total constraint violation $\text{CCV}_{[1:T]}$ for Algorithm \ref{coco_alg_1} is at most $O(G D)$. 

\end{proof}