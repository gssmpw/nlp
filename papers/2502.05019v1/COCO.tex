\section{COCO Problem}




On round $t,$ the online policy first chooses an admissible action $x_t \in \mathcal{X}\subset \bbR^d,$ 
 and then the adversary chooses a convex cost function $f_t: \mathcal{X} \to \mathbb{R}$ and a constraint of the form $g_{t}(x) \leq 0,$ where $g_{t}: \mathcal{X} \to \mathbb{R}$ is a convex function. Once the action $x_t$ has been chosen, we let $\nabla f_t(x_t)$ and full function $g_t$ or the set $\{x: g_t(x)\le 0\}$ to be revealed, as is standard in the literature.
 We now state the  standard assumptions made  in the  literature while studying the COCO problem \cite{guo2022online, yi2021regret, neely2017online, Sinha2024}.
\begin{assumption}[Convexity] \label{cvx}
$\mathcal{X} \subset \bbR^d$ is the admissible set that is closed, convex and has a finite Euclidean diameter $D$.  The cost function $f_t: \mathcal{X} \mapsto \mathbb{R}$ and the constraint function $g_{t}: \mathcal{X} \mapsto \mathbb{R}$ are convex for all $t\geq 1$.  
 %Moreover, $D$ is known ahead of time.
\end{assumption}
%\vspace{-0.18in}
\begin{assumption}[Lipschitzness] \label{bddness}
 %We have $\textrm{diam}(\mathcal{X}) \leq D, ||\nabla f_t(x)||_2 \leq G/2, \textrm{and}~ ||\nabla g_t(x))||_2 \leq G/2,~\forall t, \forall x\in \mathcal{X}$ for some finite constants $D$ and $G.$ If the functions are not necessarily differentiable, we require that the maximum magnitude of the subgradients be bounded accordingly.  Each
All cost functions $\{f_t\}_{t\geq 1}$ and the constraint functions $\{g_{t}\}_{ t\geq 1}$'s are $G$-Lipschitz, i.e., for any $x, y \in \mathcal{X},$ we have 
 \begin{eqnarray*}
 	|f_t(x)-f_t(y)| \leq G||x-y||,~
 	|g_{t}(x)-g_{t}(y)| \leq G||x-y||, ~\forall t\geq 1.
 \end{eqnarray*}
	\end{assumption}
 \begin{assumption}[Feasibility] \label{feas-constr}
With ${\cal G}_t = \{x\in \cX : g_t(x)\le 0\}$, we assume that  $\mathcal{X}^\star = \cap_{t=1}^T G_t  \neq \varnothing $.	
Any action $x^\star \in \cX^\star$ is defined to be feasible. %There exists some feasible action $x^\star \in \mathcal{X} $ s.t. $g_{t}(x^\star) \leq 0, \ \forall t \ \in T.$ The set of all feasible actions, denoted by $\mathcal{X}^\star,$ is called the feasible set. The feasibility assumption implies that $\mathcal{X}^\star \neq \emptyset.$
\end{assumption}
The feasibility assumption distinguishes the cost functions from the constraint functions and is common across all previous literature on COCO \cite{guo2022online, neely2017online, yu2016low,yuan2018online,yi2023distributed, georgios-cautious,Sinha2024}. 

%In light of Assumption \ref{feas-constr}, including multiple constraints at each time is straightforward because of the existence of a common feasible point for all $g_t, t=1, \dots, T$.

For any real number $z$, we define $(z)^+ \equiv \max(0,z).$ Since $g_{t}$'s are revealed after the action $x_t$ is chosen, any online policy need not necessarily take feasible actions on each round. 
 Thus in addition to the static\footnote{ The static-ness refers to the fixed benchmark using only one action $x^\star$ throughout the horizon of length $T$}  regret defined below
%\vspace{-0.1in}
\begin{eqnarray} \label{regret-def}
	\textrm{Regret}_{[1:T]} \equiv \sup_{\{f_t\}_{t=1}^T} \sup_{x^\star \in \mathcal{X}^\star} \textrm{Regret}_{[1:T]}(x^\star), \end{eqnarray}
	where $\textrm{Regret}_{[1:T]}(x^\star) \equiv \sum_{t=1}^T f_t(x_t) - \sum_{t=1}^T f_t(x^\star)$, 
an additional obvious metric of interest is  the total cumulative constraint violation (CCV) defined as 
 %\vspace{-0.1in}
  \begin{eqnarray} \label{gen-oco-goal}
 	\textrm{CCV}_{[1:T]}  = \sum_{t=1}^T (g_{t}(x_t))^+. 
	\end{eqnarray}
	 Under the standard assumption (Assumption \ref{feas-constr}) that $\mathcal{X}^\star$ is not empty, the goal is to design an online policy to simultaneously achieve a small regret \eqref{intro-regret-def} with $x^\star \in \mathcal{X}^\star$ and a small CCV \eqref{intro-gen-oco-goal}. We refer to this problem as the constrained OCO (COCO). 
	 
For simplicity, we define set 
\begin{equation}\label{defn:S}
 \ S_t = \cap_{\tau=1}^t {\cal G}_\tau,
\end{equation}
where $G_t$ is as defined in Assumption \ref{feas-constr}.
All ${\cal G}_t$'s are convex and consequently, all $S_t$'s are convex and are nested, i.e. $S_t\subseteq S_{t-1}$. Moreover, because of Assumption \ref{feas-constr},  each $S_t$ is non-empty and in particular $\cX^\star\in S_t$ for all $t$. After action $x_t$ has been chosen, set $S_t$ controls the constraint violation, which can be used to write an upper bound on the $\textrm{CCV}_{[1:T]}$ as follows.

\begin{definition}
For a convex set $\chi$ and a point $x\notin \chi$, 
$$\text{dist}(x,\chi) = \min_{y\in \chi} || x-y||.$$
\end{definition}

Thus, the constraint violation at time $t$, 
\begin{equation}\label{eq:distviolationrelation}
(g_t(x_t))^+ \le G\text{dist}(x_t,S_t), \ \text{and} \  \textrm{CCV}_{[1:T]}  \le G\sum_{t=1}^T \text{dist}(x_t,S_t),
\end{equation}
where $G$ is the common Lipschitz constants for all $g_t$'s.


	 
	 