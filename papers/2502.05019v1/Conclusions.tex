\section{Conclusions}
One fundamental open question for COCO is: whether it is possible to simultaneously achieve $\cR_{[1:T]} =O(\sqrt{T})$ and $\text{CCV}_{[1:T]} = o(\sqrt{T})$ or $\text{CCV}_{[1:T]} = O(1)$.
In this paper, we have made substantial progress towards answering this question by proposing an algorithm that exploits the geometric properties of the nested convex sets $S_t$'s that effectively 
control the CCV. The state of the art algorithm \cite{Sinha2024} achieves a CCV of $\Omega(\sqrt{T}\log T)$ even for very simple instances as shown in Lemma \ref{lem:algwc}, 
and conceptually different algorithms were needed to achieve CCV of $o(\sqrt{T})$. 
We propose one such algorithm and show that when the nested convex constraint sets are `nice' (instances  is simple), achieving a CCV of $O(1)$ is possible without losing out on $O(\sqrt{T})$ regret guarantee. We also derived a bound on the CCV for general problem instances, that is as a function of the shape of nested convex constraint sets and the distance between them, and the diameter. 

In the absence of good lower bounds, the open question remains open in general, however,
 this paper significantly improves the conceptual understanding of COCO problem by demonstrating that good algorithms need to exploit the geometry of the nested convex constraint sets.

One remark we want to make at the end is that COCO is inherently a difficult problem, which is best exemplified by the fact, that even for the special case of COCO where $f_t=f$ for all $t$, essentially where $f$ is known ahead of time, our algorithm/prior work does not yield a better regret or CCV bound compared to when $f_t$'s are arbitrarily varying. 