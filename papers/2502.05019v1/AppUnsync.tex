\section{Proof of Theorem \ref{lem:movementcost}}
\begin{proof}
We need the following preliminaries.

 \begin{definition}\label{defn:avgwidth}
Let $K$ be a non-empty convex bounded set in $\bbR^d$. Let $u$ be a unit vector, and $\ell_u$ a line through the origin parallel to $u$. 
Let $K_u$ be the orthogonal projection of $K$ onto $\ell_u$, with length $|K_u|$. The mean width of $K$ is defined as 
\begin{equation}\label{eq:projlength}
W(K) = \frac{1}{V_d} \int_{\bbS_1^d} |K_u| du,
\end{equation}
where $\bbS_1^d$ is the unit sphere in $d$ dimensions and $V_d$ its $(d-1)$-dimensional Lebesgue measure.
\end{definition}


The following is immediate. 
\begin{equation}\label{eq:WBound1}
0\le W(K) \le \text{diameter}(K).
\end{equation}

\begin{lemma}\label{lem:width2D}\cite{eggleston1966convexity}
For $d=2$, $$W(K)=\frac{\text{Perimeter}(K)}{\pi}.$$
\end{lemma}
Lemma \ref{lem:width2D} implies that $W(K) \ne W(K_1) + W(K_2)$ even if $K_1\cup K_2=K$ and $K_1\cap K_2=\phi$.




Recall from \eqref{defn:genconvxmovement} that $x_t\in  \partial S_{t-1}$ and $p_t$ is the projection of $x_t$ onto $S_{t}$, and $m_t$ is the mid-point of $x_t$ and $p_t$, i.e. $m_t = \frac{x_t+p_t}{2}$. Moreover, the convex sets $S_t$'s are nested, i.e., $S_1\supseteq S_2 \supseteq \dots \supseteq S_T$.
To prove Theorem \ref{lem:movementcost} we will bound the rate at which $W(S_t)$ (Definition \ref{defn:avgwidth}) decreases as a function of the length $||x_t-p_t||$. 

From Definition \ref{defn:anglewidth}, recall that $\cC_t$ is the convex hull of $m_t\cup S_{t}$. We also need to define $\cC_t^-$ as the convex hull of $x_t\cup S_{t}$. Since 
$S_t \subseteq \cC_t$ and $\cC_t^- \subseteq S_{t-1}$, we have \begin{equation}\label{eq:widthlb}
W(S_{t}) - W(S_{t-1}) \le W(\cC_t) - W(\cC_t^-).
\end{equation} 
\begin{definition}\label{}
$\Delta_t = W(\cC_t) - W(\cC_t^-)$.
\end{definition}
%and derive a bound on $\Delta_t$.



The main ingredient of the proof is the following Lemma that bounds  $\Delta_t$ whose proof is provided after completing the proof of Theorem \ref{lem:movementcost}.
\begin{lemma}\label{lem:derivative} $$\Delta_t  \le -V_{d-1}\frac{||x_t-p_t||}{2V_d(d-1)}  (c_t^\star)^{d},$$
where $c_t^\star$ has been defined in Definition \ref{defn:anglewidth}.
\end{lemma}


Recalling that $c^\star = \min_t  c^\star_t$ from Definition \ref{defn:anglewidth}, and  combining  Lemma \ref{lem:derivative} with \eqref{eq:WBound1} and \eqref{eq:widthlb}, we get that 
$$\sum_{t=1}^T ||x_t-p_t|| \le \frac{2V_d(d-1)}{V_{d-1}} \left(\frac{1}{c^\star}\right)^{d}\text{diameter}(S_1),$$
since $S_1\supseteq S_2 \supseteq \dots \supseteq S_T$. Recalling that $\text{diameter}(S_1)\le D$, Theorem \ref{lem:movementcost} follows.
\end{proof}
%Note that $b_t$ being a projection of $a_t$ onto 
%$S_{t}$, 

\begin{proof}[Proof of Lemma \ref{lem:derivative}]
% \begin{figure}[]
%  \begin{center}
%%\begin{tikzpicture}
%\begin{tikzpicture}[scale=1.5, dot/.style={circle,inner sep=1pt,fill,label={#1},name=#1},
%  extended line/.style={shorten >=-#1,shorten <=-#1},
%  extended line/.default=1cm]
%% Define coordinates
%\coordinate (w_t) at (0,0);
%\coordinate (z_t) at (0,-.5);
%\coordinate (x_t) at (-1,-1);
%\coordinate (y_t) at (1,0);
%\coordinate (C_top) at (1,2.15);
%\coordinate (C_bottom) at (3.55,-0.55);
%
%
%% Draw object (shaded region)
%%\draw[thick, gray, fill=pink!50] (C_top) to[out=-.5,in=-15] (C_bottom) -- cycle;
%
%
%
%
%
%% Draw projected image (S_t+1)
%\draw[thick, cyan, fill=cyan!50,opacity=0.75] (1,0) -- (1.5,1.5) -- (3,1) -- (3.2,0.15)  -- cycle;
%\draw[thick, blue!10, fill=blue!10,opacity=0.5] (z_t) --(1.5,1.495) --  (y_t)   -- cycle;
%\draw[thick, blue!10, fill=blue!10,opacity=0.5] (z_t) --  (y_t)  -- (3.2,0.15) -- cycle;
%
%\filldraw[blue] (y_t) circle (2pt) node[below right] {$b_t$};
%\filldraw[black] (z_t) circle (2pt) node[below right] {$m_t$};
%% Draw camera and viewing ray
%\draw[thick,->] (x_t) -- (y_t);
%\draw[dashed,->] (.65,0) -- (z_t) -- (-0.65,-1)  node[below right] {$w_t$};
%\draw[dashed,->] (.65,0) -- (z_t) -- (-0.65,-1)  node[below right] {$w_t$};;
%%\draw[dashed,->] (.65,0) -- (x_t) -- (-0.65,-1)  node[below right] {$w_t$};;
%
%\coordinate (u_1) at (0,.5);
%\coordinate (u_2) at (0,-2);
%
%\draw [ dashed, <-] (-1,.5) -- (x_t) -- (-1,-2) node[above left] {$H_u'$};
%\draw [ dashed, <-] (u_1) -- (z_t) -- (u_2) node[above left] {$H_u$};
%\draw [ dashed, ->] (1.3,-.7) -- (z_t) -- (-2.8,.2) node[above left] {$u$};
%
%%\draw [extended line, ->] ($(u_1)!(z_t)!(u_2)$) -- (-.5,-0.05) node[above left ] {$u$};
%
%%\draw [extended line,<-] ($(-0.25,.5)!(z_t)!(.25,-2)$) node[above left] {$u$};
%
%%\tkzDefLine[orthogonal=through (z_t)](x_t,y_t);
%
%
%%\filldraw[black] (w_t) circle (2pt) node[above left] {$w_t$};
%\filldraw[black] (x_t) circle (2pt) node[below left] {$a_t$};
%
%\draw[thick, gray, fill=pink!50,opacity=0.6] (C_top) to[out=.5,in=25] (C_bottom) -- cycle;
%
%\draw[thick, gray, fill=gray!50,opacity=0.6] (C_bottom) to[out=-155,in=-165] (C_top) -- cycle;
%\node at (2.25,0.75) {$S_{t}$};
%
%% Draw lines connecting camera to object and projected image
%\draw[dotted] (z_t) -- (1.5,1.5);
%\draw[dotted] (z_t) -- (3.2,0.15);
%%\draw[dotted] (w_t) -- (3,1);
%%\draw[dotted] (w_t) -- (2.5,0.5);
%%\draw[dotted] (w_t) -- (1.5,0.5);
%\draw[dotted] (z_t) -- (C_top);
%\draw[dotted] (z_t) -- (C_bottom);
%
%% Add labels for object and camera
%\node[above right] at (C_top) {$C_{w_t}(c_t)$};
%\end{tikzpicture}
%\caption{Figure representing the cone $C_{w_t}(c_t)$ that contains the convex hull of $m_t$ and $S_{t}$ with respect to the unit vector $w_t$. $u$ is a unit vector perpendicular to $H_u$ an hyperplane that is a supporting hyperplane $C_t$ at $m_t$ such that $\cC_t \cap H_u = \{m_t\}$ and 
%$u^T (a_t-m_t)\ge 0$ }
%\label{fig:anglewidth}
%\end{center}
%\end{figure}

\begin{figure*}
\begin{center}
\includegraphics[width=10cm,keepaspectratio,angle=0]{FigConeDefHyperplane.png}
\caption{Figure representing the cone $C_{w_t}(c_t)$ that contains the convex hull of $m_t$ and $S_{t}$ with respect to the unit vector $w_t$. $u$ is a unit vector perpendicular to $H_u$ an hyperplane that is a supporting hyperplane $C_t$ at $m_t$ such that $\cC_t \cap H_u = \{m_t\}$ and 
$u^T (x_t-m_t)\ge 0$ }
\label{fig:anglewidth}
\end{center}
\end{figure*}



Let $H_u$ be the hyperplane perpendicular to vector $u$.
 Let $\cU_0$ be the set of unit vectors $u$ such that hyperplanes $H_u$ are supporting hyperplanes to $\cC_t$ at point $m_t$ such that $\cC_t \cap H_u = \{m_t\}$ and 
$u^T (x_t-m_t)\ge 0$.  See Fig. \ref{fig:anglewidth} for reference.

 Since $p_t$ is a projection of $x_t$ onto $S_{t}$, and $m_t$ is the mid-point of $x_t,p_t$, for $u\in \cU_0$, the hyperplane $H_u'$ containing $x_t$ and parallel to $H_u$ is a supporting hyperplane for $\cC_t^-$.  


Thus, using the definition of $K_u$ from \eqref{eq:projlength},
\begin{equation}\label{eq:dummy1}
\Delta_t  \le \frac{1}{V_d} \int_{\cU_0} (|\cC_{t,u}| - |\cC_{t,u}^-|) du= -\frac{||x_t-p_t||}{2V_d} \int_{\cU_0} u^T 
\frac{(x_t-m_t)}{||x_t-m_t||}  \ du,
\end{equation}
since $||x_t-m_t|| = ||x_t-p_t||/2$.

Recall the definition of $C_{w_t^\star}(c_t^\star)$ from Definition \ref{defn:anglewidth} which implies that the convex hull of $m_t$ and $S_{t}$, $\cC_t$ is contained in $C_{w_t^\star}(c_t^\star)$.
Next, we consider $\cU_1$ the set of unit vectors $u$ such that hyperplanes $H_u$ are supporting hyperplanes to $C_{w_t^\star}(c_t^\star)$ at point $m_t$ 
such that $u^T (x_t-m_t)\ge 0$. 
By definition $\cC_t\subseteq C_{w_t^\star}(c_t^\star)$, it follows that 
$\cU_1\subset \cU_0$.

Thus, from \eqref{eq:dummy1}
 \begin{equation}\label{eq:dummy2}
\Delta_t  \le -\frac{||x_t-p_t||}{2V_d} \int_{\cU_1} u^T. \frac{(x_t-m_t)}{||x_t-m_t||} du
\end{equation}

Recalling the definition of $w_t^\star$ (Definition \ref{defn:anglewidth}), 
vector $u\in \cU_1$ can be written as 
$$ u = \lambda u_{\perp} + \sqrt{1-\lambda^2} w_t^\star,$$
where $u_{\perp}^T w_t^\star=0$, $|u_{\perp}|=1$ and since $u\in \cU_1$
$$0 \le \lambda=\sqrt{1-(u^Tw_t^\star)} = u^Tu_{\perp}\le c_t^\star.$$

Let $\cS_{\perp} = \{u_\perp: |u_\perp|=1 , u_\perp^T w_t^\star=0\}$. Let $du_{\perp}$ be the 
$(n-2)$-dimensional Lebesgue measure of $\cS_{\perp}$. 

It is easy to verify that 
$du = \lambda^{d-2}(1-\lambda^2)^{-1/2} d\lambda du_{\perp}$ and hence from \eqref{eq:dummy2}

\begin{equation}\label{eq:dummy3}
\Delta_t  \le -\frac{||x_t-p_t||}{V_d} \int_{0}^{c_t^\star}  \lambda^{d-2}(1-\lambda^2)^{-1/2} d\lambda \int_{\cS_{\perp}} (\lambda u_{\perp} + \sqrt{1-\lambda^2} w_t^\star)^T \frac{(x_t-m_t)}{||x_t-m_t||}  du_{\perp}.
\end{equation}
 
 Note that $\int_{du_{\perp}} u_{\perp} du_{\perp}=0$. Thus,
 \begin{align}\nn \label{}
\Delta_t  & = -\frac{||x_t-p_t||}{2V_d} \frac{(w_t^\star)^T(x_t-m_t)}{||x_t-m_t||}  \int_{0}^{c_t^\star}  \lambda^{d-2}(1-\lambda^2)^{-1/2}   \sqrt{1-\lambda^2}\ d\lambda   \int_{\cS_{\perp}} du_{\perp},\\ 
\nn
& \stackrel{(a)}\le -V_{d-1} \frac{||x_t-p_t||}{2V_d}  \frac{ (w_t^\star)^T(x_t-m_t)}{||x_t-m_t||}  \int_{0}^{c_t^\star} \lambda^{d-2}\ d\lambda, \\ \nn
& \stackrel{(b)}\le  -V_{d-1}\frac{||x_t-p_t||}{2V_d(d-1)} c_t^\star (c_t^\star)^{d-1},\\ 
& = -V_{d-1}\frac{||x_t-p_t||}{2V_d(d-1)}  (c_t^\star)^{d},
\end{align}
where $(a)$ follows since  $\int_{\cS_{\perp}} du_{\perp} = V_{d-1}$ by definition, $(b)$ follows since $ \frac{(w_t^\star)^T(x_t-m_t)}{||x_t-m_t||} \ge c_t^\star$ from Definition \ref{defn:anglewidth}.

%Clearly, $W(S_{t} - W(S_t) \ge \Delta_t$. 
%We will bound $\Delta_t $ to get a bound on the movement cost $M$ \eqref{eq:totalDistance}
 \end{proof}
 Note that the proof is conceptually similar to the proof of Theorem \ref{thm:manselli} in \cite{Manselli}, 
 and has been adapted to fit the considered problem.