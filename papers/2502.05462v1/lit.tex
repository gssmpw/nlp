Early work on collaborative object transportation started in 1996. A virtual linkage model \cite{1996_khatib} represents the collaborative manipulation systems. It generates closed-chain constraints \cite{2023_Xu} between the object and the manipulators for motion synchronization and coordination. Multi-MMR cooperative manipulation and transportation of an object comprises of the centralized \cite{2013_Erhart}, decentralized \cite{2016_Petitti,2018_Culbertson,2018_Verginis} and distributed \cite{2017_Dai,2018_Marino,2020_Ren} cooperative control scheme.

Calculus of variation-based navigation method \cite{1997_Desai} demonstrated collision-free motion planning in the presence of static obstacles for a two MMRs system. The variational-based method does not scale well with more numbers of MMRs and is inadequate to solve generalized motion planning problems. Dipolar Inverse Lyapunov Functions (DILFs), combined with the potential field-based navigation function \cite{2003_Tanner}, find collision-free motion in static environments for deformable material transportation by multiple MMRs with meager control over the formation. The non-holonomic passive decomposition method \cite{2013_Yang} splits robots' motion into three components: formation shape, object transportation, and internal motions, and utilizes internal motions for static obstacle avoidance. The method does not incorporate the kinematic and dynamic constraints of the MMRs.

Constrained optimization-based methods \cite{2015a_AlonsoMora,2015b_AlonsoMora,2017_AlonsoMora} for trajectory planning in dynamic environments use obstacle-free regions around the formation in the position-time space and optimize for the object's pose to contain the robots in that region. These methods could only plan a collision-free trajectory when the planning horizon is adequately long and the poses of the robots in the current or the target formation are outside the position-time embedded obstacle-free convex region. A geometric path planning approach was proposed \cite{2018_Cao, 2017_Jiao} for multiple MMRs transporting an object for static obstacle avoidance. A rectangular passageway-based approach \cite{2018_Cao} is used to find the optimal system width and moving direction in the static obstacle-free area. However, these mentioned methods don't incorporate motion constraints and optimality guarantees.

A kinematic planning algorithm \cite{2019_Tallamraju} cooperatively manipulates spatial payload. The algorithm utilizes a hierarchical approach that conservatively approximates the obstacles as uniform cylinders and generates a real-time collision-free motion. This approach is highly restrictive for navigation in tight spaces with large heights and polygonal obstacles with high aspect ratios. MPC-based motion planning techniques for static obstacle avoidance have been presented in \cite{2017_Nikou,2024_Kennel}. An alternating direction method of multipliers (ADMM) based distributed trajectory planning algorithm \cite{2020_Shorinwa} plans trajectory in a static environment. A distributed formation control technique \cite{2021_Wu} utilizes constrained optimization for object transportation in a static environment. The formation moves along a predefined reference trajectory avoids collision with the static obstacles only. Motion Planning for deformable object transportation by multiple robots in a static environment using optimization was proposed in \cite{2022_Hu,2024_Pei}.

Motion planning for generic robotic systems also uses sampling-based motion planners like Probabilistic Road Maps (PRM) \cite{1996_Kavraki} and Rapidly exploring Random Tree (RRT) \cite{1998_Lavalle,2001_Lavalle,2019_Kleinbort} randomized planner which can plan for higher DoF configurations. A graph-based method with uniform sampling like cell discretization is proposed in \cite{2024_Liu}. However, these techniques work well for the static environment with global guarantees, but need complete re-planning for dynamic environments and poor scalability with the number of robots remains a challenge. The random sampling-based approach like \cite{2012_Krontiris} computes PRM from a set of formations. 

Constrained optimization-based static collision avoidance for a single robot was presented in \cite{2021_Zhang} and \cite{2016_Faulwasser}. A reciprocal collision avoidance algorithm \cite{2020_Mao} combined with MPC for multiple robots, does not maintain any formation. These generic planning algorithms cannot be used directly as they don't maintain the formation rigidly which is required for collaborative MMRs.
