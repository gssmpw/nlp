A system of $n$ nonholonomic MMRs grasps an object at its periphery as shown in Fig. \ref{fig:1} to collaboratively transport an object from a start location to a goal location in the obstacle-free region. $\{\boldsymbol{w}\}$ defines the world fixed reference frame. An object coordinate frame $\{\boldsymbol{o}\}$ is attached to the object center of mass (CoM), and each MMR has its own body coordinates $\{\boldsymbol{b}_i\}$ attached to the center of its mobile base. Without specific mention, all the quantities are defined in the world fixed frame $\boldsymbol{\{w\}}$. The collaborative manipulation system is defined in the following subsections.

\begin{figure}[htbp]
	\centerline{\includegraphics[width = 240px]{figure/Schematic/SystemDefination.eps}}
	\caption{Formation of five non-holonomic MMRs holding an object. The MMRs grasped the object to transport collaboratively from one place to another.}
	\label{fig:1}
\end{figure}

\subsection{Mobile Manipulator}
The mobile base of $i^{th}$ MMR is defined with pose $q_{m,i}=[p^T_i,\phi_i]^T$ where $p_i\in\mathbb{R}^2$ and $\phi_i\in\mathbb{R}$ are the position and orientation of the mobile base in $\{w\}$. The manipulator of MMR $i$ has $n_i$ number of joints. The joint states of the manipulator $i$ is defined as $q_{a,i} = [q_{a,i1},q_{a,i2},\cdots, q_{a,in_i}]^T$. The $i^{th}$ EE's position and orientation is defined in $\{w\}$ as $p_{ee,i}\in\mathbb{R}^3$ and $\phi_{ee,i}\in\mathbb{R}^3$. The joint state of the manipulator is defined by $q_i=[q_{m,i}^T,q_{a,i}^T]^T$. 
The $i^{th}$ non-holonomic MMR's first-order dynamics $\dot{q}_i =[v_{i}\cos(\phi_i), v_{i}\sin(\phi_i),\omega_{i},\dot{q}_{a,i}]^T$ is considered where the control inputs are mobile base's linear and angular velocities $v_{i}$, $\omega_{i}$ respectively and the manipulator's joint velocities $\dot{q}_{a,i}$, therefore, $u_i = [v_{i},\omega_{i},\dot{q}_{a,i} ]$. 

We represent the coupled first order system dynamics for $i^{th}$ MMR  by a discrete-time non-linear system as
\begin{equation}\label{stf}
	q^{k+1}_i =f(q^k_i ,u^k_i)
\end{equation}
where $k$ is the discrete time step. The admissible states and control inputs are defined by Eqn. \ref{eqn:limits}
\begin{subequations}
	\begin{align}
		\underline{q}_{a,i}\leq q_{a,i}\leq\overline{q}_{a,i} \label{jsl}\\
		\underline{u}_{i}\leq u_{i}\leq\overline{u}_{i}
		\label{ajrl}
	\end{align}
	\label{eqn:limits}
\end{subequations}
for all $i \in [1,n]$, where $\underline{q}_{a,i}$, $\overline{q}_{a,i}$ represents the manipulator's joint position limit vector and $\underline{u}_{i}$, $\overline{u}_{i}$ are the admissible control limits. The set of admissible states $\mathcal{Q}_i$ and control inputs $\mathcal{U}_i$ are indicated by joint position and velocity vectors' limit (Eqn. \eqref{eqn:limits}), $\mathcal{Q}_i = [\underline{q}_{a,i}, \overline{q}_{a,i}]$, $\mathcal{U}_i = [\underline{u}_{i}, \overline{u}_{i}]$

\subsection{Collaborative Formation}
The multi-MMR formation $\mathcal{F}$ (Fig \ref{fig:1}) of $n$ MMRs grasps a rigid object by the EEs. The $i^{th}$ EE grasps the object at $^or_i$, where the superscript $\boldsymbol{o}$ indicates it's reference frame $\{\boldsymbol{o}\}$. The formation configuration is defined by $\mathcal{X}=[p^T,o^T,Q^T]^T$, where $p\in\mathbb{R}^3$ is the position and $o\in\mathbb{R}^3$ is the orientation of the object CoM, $Q=[q_1^T,q_2^T,\cdots, q_n^T]^T$ is the configuration of $n$ MMRs. The space occupied by the formation is defined as $\mathcal{B}(\mathcal{X})$.

\subsection{Environments}
A structured and bounded environment having both static and dynamic obstacles is defined as $W$. $\mathcal{O}$ represents the set of static obstacles in $W$. We consider the static obstacles to be vertically upright. The static obstacle-free workspace is defined by 
\begin{equation}\label{eqn11}
    W_{free} = W\setminus\mathcal{O}\in\mathbb{R}^2
\end{equation}
The static obstacle map is known apriori. The set of dynamic obstacles within the sensing zone of the MMRs is defined as $\mathcal{O}_{dyn}$. The start position $p_s$ and the goal position $p_g$ are in the obstacle-free space $W_{free}$.

The planning objective is to design a motion planning framework for cooperative object transportation such that
\begin{enumerate}
	\item the MMRs can cooperatively transport the object from the start to goal position without any collision.
	\item the generated motion is kinodynamically feasible and within the admissible limits of the MMRs. 
	\item the planner can handle static concave obstacles directly.
	\item the control input of the MMRs are minimized.
\end{enumerate}