Robotic systems became integral to automation in manufacturing, remote exploration, warehouse management, and other areas. Cooperative multiple MMRs garner attention due to their low cost, small size, redundancy in heavy or oversized object transportation, and fixture-less multipart assembly requiring more Degrees of Freedom (DoF). A cooperative MMR system extends workspace coverage, flexibility, and redundancy with added complexity in robot coordination, communication, and motion planning. Multiple MMRs leverage the mobile bases' locomotion ability and the arms' manipulation ability for object transportation and manipulation in a large workspace.

Nonholonomic mobile bases are widespread in robotic applications because of their advantages in a reduced number of actuators, simplified wheels, and better load-carrying capacity. It also works better on uneven ground surfaces. Introducing a nonholonomic mobile base for the MMRs includes non-integrable constraints on kinematics. The nonholonomic mobile base's non-integrable kinematics constraints impact motion planning with its restricted side-wise motion capability. Hence, there are more intricacies in cooperative motion planning and trajectory generation than in the holonomic counterpart.

%%----------Literature----------
The study on collaborative manipulators started with a virtual linkage model \cite{1996_khatib} representing the collaborative manipulation systems to generate closed-chain constraints \cite{2023_Xu} between the object and the MMRs for motion synchronization and coordination. The dual arm cooperative control problem has been addressed by NMPC \cite{2024_Zhao}. The coordination scheme for multi-MMR cooperative manipulation and transportation comprises of centralized \cite{2013_Erhart}, decentralized \cite{2018_Culbertson,2018_Verginis} and distributed \cite{2017_Dai,2018_Marino,2020_Ren} control framework. Online task allocation \cite{2024a_Keshab} algorithm ensures efficient utilization of the capabilities of the cooperative manipulators. The collision-free navigation started with a variational-based method \cite{1997_Desai} that demonstrated static obstacle avoidance for a two MMR system with poor scalability. Dipolar inverse Lyapunov functions fused with the potential field-based navigation function \cite{2003_Tanner} plan collision-free motion in static environments to transport deformable material by multiple MMRs with a little scope of formation control.

Constrained optimization-based motion planning technique \cite{2017_AlonsoMora} for holonomic MMRs in dynamic environments uses obstacle-free convex polygons around the formation in the position-time space. It optimizes the holonomic MMRs' pose to retain the cooperative MMR system inside the obstacle-free polygon. A geometric path planning approach was proposed \cite{2018_Cao, 2017_Jiao} for multiple MMRs transporting an object for static obstacle avoidance. A rectangular passageway-based approach \cite{2018_Cao} is used to find the optimal system width and moving direction in the static obstacle-free area. These methods do not include motion constraints and guarantee feasible motion for nonholonomic MMRs.

A kinematic motion planning technique \cite{2019_Tallamraju} plans for spatial collaborative payload manipulation using a hierarchical approach. The technique's conservative approximation of the obstacles as uniform cylinders highly restricts navigation in tight spaces with high aspect ratios polygonal obstacles. MPC-based motion planning techniques for static obstacle avoidance have been presented in \cite{2017_Nikou,2024_Kennel}. An alternating direction method of multipliers-based distributed trajectory planning algorithm \cite{2020_Shorinwa} plans trajectory in a static environment. A distributed formation control technique \cite{2021_Wu} utilizes constrained optimization for object transportation in a static environment. The formation moves along a predefined reference trajectory to avoid collision with the static obstacles. Motion Planning for deformable object transportation \cite{2022_Hu,2024_Pei} in a static environment uses optimization techniques. A reciprocal collision avoidance algorithm \cite{2020_Mao} combined with MPC for multiple robots does not maintain any formation. These generic planning algorithms cannot be used as they do not maintain the rigid formation required for collaborative MMRs. An NMPC-based kinodynamic motion planning technique \cite{2024b_Keshab} plans motion for object transportation by multiple MMRs in static and dynamic environments. The proposed planning technique is limited to holonomic MMRs and environments with convex static obstacles.

We proposed a motion planning technique for collaborative object transportation using nonholonomic MMRs that eliminates the shortcomings. The proposed planning technique removes the restriction of convex obstacles and works with any polygonal static obstacles. The planning technique works in two steps: offline path planning and online motion planning. In offline path planning, the planner computes the shortest feasible piece-wise linear path between the start and the goal using visibility vertices \cite{2005_Choset} based technique considering the static obstacle. After computing the path, the convex polygon around path segments is computed for online motion planning. The online motion planner computes a kinodynamic feasible motion plan for the MMRs while transporting the object in a dynamic environment, taking the global path as an initial guess. The major contributions of this article are in the following:

\begin{enumerate}
	\item A novel visibility vertices-based offline path planning algorithm finds the shortest path in static environments with any convex and concave polygonal shape static obstacles. The algorithm is capable of finding paths through narrow corridors.
	\item A convex polygon computation algorithm to compute convex polygons around the path segments utilizing visible vertices of the path segments.
	\item An NMPC-based online motion planning technique jointly plans for nonholonomic MMR's base and the manipulator. The planner computes kinodynamic feasible collision-free motion for the multiple MMRs in a dynamic environment.
\end{enumerate}
  
%Motion planning for generic robotic systems also uses sampling-based motion planners like Probabilistic Road Maps (PRM) \cite{1996_Kavraki} and Rapidly exploring Random Tree (RRT) \cite{1998_Lavalle,2019_Kleinbort} randomized planner which can plan for higher DoF configurations. A graph-based method with uniform sampling like cell discretization is proposed in \cite{2024_Liu}. However, these techniques work well for the static environment with global guarantees, but need complete re-planning for dynamic environments and poor scalability with the number of robots remains a challenge. The random sampling-based approach like \cite{2012_Krontiris} computes PRM from a set of formations.