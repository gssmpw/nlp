We have validated the proposed kinodynamic motion planning algorithm in simulation and hardware experiments with the nonholonomic MMRs that accept velocity as a control input. The first order system dynamics for the nonholonomic MMR are approximated using the fourth-order Runge-Kutta method as a state transition function mentioned in Eqn. \eqref{stf}. The NMPC problem of the local motion planning and the nonlinear optimization of global planning is solved using the CasADi package \cite{2019_Andersson} with an Interior point optimization (Ipopt) method.

\subsection{Simulation}
The MMRs with a differential drive mobile base have the same forwarding and reversing capabilities. The manipulator's Denavit-Hartenberg (DH) parameters are mentioned in Table \ref{tab:0}. The MMRs rigidly grasp the object at the periphery, and their grasping location remains the same throughout the entire task.

\begin{table}[htbp]
	\caption{DH Parameters Value for the manipulators}
	\label{tab:0}
	\begin{center}
		\begin{tabular}{|c|c|c|c|c|}
			\hline
			\textbf{Joint} & $d\ (m)$ & $a\ (m)$ & $\alpha\ (rad)$& $\theta\ (rad)$\\
			\hline
			Joint 1 & 0.070 & 0 & 0 & $q_{a,1}$\\
			\hline
			Joint 2 & 0 & 0 & $0.5\ \pi$ & $q_{a,2}$\\
			\hline
			Joint 3 & 0.100 & 0 & $- \pi$ & $q_{a,3}$\\
			\hline
			Joint 4 & 0.125 & 0 & $ \pi$ & $q_{a,4}$\\
			\hline
			Joint 5 & 0 & 0.120 & $-0.5\ \pi$ & $q_{a,5}$\\
			\hline
			Gripper & 0 & 0 & 0 & 0\\
			\hline
		\end{tabular}
	\end{center}
\end{table}

We select the operational velocity of the formation $v_{op} = 0.15\ m/s$ and use prediction horizon time $T_h = 9\ s$, trajectory execution time $T_e = 3\ s$ and the discretization time step $T_c = 0.25\ s$. The safety margins  $d_{safe} = 0.05\ m$ and $d_{safe,dyn} = 0.1\ m$ for static and dynamic obstacle avoidance to keep the formation safe during object transportation. A higher margin restricts the formation from nearing the obstacles and hence reduces the obstacle-free space. The optimization weights are $\mathbf{W_u} = diag([0.05,0.05,5,0.5,5,0.05,0.05,5,0.5,5])$, $\mathbf{W_e} = diag([0.01,0.01])$ and $\mathbf{W_{N_h}} = 10^5$.


\begin{figure}[htbp]
	\subfigure[$t = 0\ s$]
	{
		\includegraphics[width=115px]{figure/Sim/vo15vd10Th90Te30Tc25_RVMU5NHF1AHIN4T13s1_0.png}
		\label{fig:sima}
	}
	\subfigure[$t = 32.5\ s$]
	{
		\includegraphics[width=115px]{figure/Sim/vo15vd10Th90Te30Tc25_RVMU5NHF1AHIN4T13s1_3250.png}
		\label{fig:simb}
	}\\
	\subfigure[$t = 40.5\ s$]
	{
		\includegraphics[width=115px]{figure/Sim/vo15vd10Th90Te30Tc25_RVMU5NHF1AHIN4T13s1_4050.png}
		\label{fig:simc}
	}
	\subfigure[$t = 79\ s$]
	{
		\includegraphics[width=115px]{figure/Sim/vo15vd10Th90Te30Tc25_RVMU5NHF1AHIN4T13s1_7900.png}
		\label{fig:simd}
	}\\
	\caption{The snapshots of cooperative object transportation by five MMRs (the MMR base in deep green and manipulator's arm in red line) in $10m\times10m$ environment. The red circular like dynamic obstacle is in its current state. The green convex polygons represents the static obstacle free region around the path. The object is being transported from the formation in Fig. \ref{fig:sima} to the formation in Fig. \ref{fig:simd}.}
	\label{fig:Sim}
\end{figure}

\begin{figure}[htbp]
	\centerline{\includegraphics[width = 235px]{figure/Sim/ObsMargin_vo15vd10Th90Te30Tc25_RVMU5NHF1AHIN4T13s1.png}}
	\caption{Safety margin during object transportation through the narrow passages. The horizontal lines plots safety margins $d_{safe}\ =\ 0.05\ m$ and $d_{safe,dyn}\ =\ 0.1\ m$ is for static and dynamic obstacles.}
	\label{fig:SafetyMargin}
\end{figure}

The five MMRs grasp an object at its periphery and start transporting (Fig. \ref{fig:sima}) through a narrow corridor of $1.9m$. While the MMRs come out of the corridor, it encounters dynamic obstacles in Fig. \ref{fig:simb}, while taking a sharp left turn. The generated motion plan successfully navigates the formation, avoiding dynamic obstacles, and turns toward (Fig. \ref{fig:simc}) the goal. The MMRs successfully transport the object through the narrow doors and complete the task without any collision (Fig. \ref{fig:simd}). Fig. \ref{fig:SafetyMargin} plots the shortest distance $d_{margin}$ from the formation to any static and dynamic obstacles. The $d_{margin}$ in Fig \ref{fig:SafetyMargin} for the static and the dynamic obstacles being always positive indicates successful collision avoidance behavior of the proposed motion planning techniques.

\subsection{Hardware Experiments}
\begin{figure}[htbp]
	\centerline{\includegraphics[width = 235px]{figure/Exp/NHSetup.png}}
	\caption{Experimental Setup of two in-house developed nonholonomic MMRs.}
	\label{fig:NonHoloSetup}
\end{figure}

We perform experiments with our in-house developed ROS-enabled MMRs to evaluate the motion planning algorithm in Section \ref{OMP} in an environment ($4\ m\times4\ m$) with static and dynamic obstacles. The nonholonomic MMR bases are made of two disc wheels each separately driven by geared motor with an encoder. Fig. \ref{fig:NonHoloSetup} shows two nonholonomic MMRs both grasped an object to transport it in an indoor environment shown in Fig. \ref{fig:expa}.

The manipulator of the MMRs shown in Fig. \ref{fig:NonHoloSetup} is same as the manipulator used for the simulation, described in Table \ref{tab:0}, except for the joint 5. We fixed the joint 5 because of the hardware limitations. The adjusted DH parameters of the gripper are $d = 0.120\ m, a = 0, \alpha = 0$, and $\theta = 0$ after fixing joint 5.
The planned trajectory and the control input for the MMRs by the online motion planner (Section \ref{OMP}) are sent to the respective MMRs. The trajectory tracking controllers for the mobile base and manipulator ensure desired trajectory tracking. The trajectory of each mobile base, object, and the EE of MMRs are measured using a Vicon motion capture system.

\begin{figure}[htbp]
	\subfigure[$t = 0\ s$]
	{
		\includegraphics[width=115px]{figure/Exp/E5RE2T1e_0000V1.png}
		\label{fig:expa}
	}
	\subfigure[$t = 15.10\ s$]
	{
		\includegraphics[width=115px]{figure/Exp/E5RE2T1e_1510.png}
		\label{fig:expb}
	}\\
	\subfigure[$t = 20\ s$]
	{
		\includegraphics[width=115px]{figure/Exp/E5RE2T1e_2000.png}
		\label{fig:expc}
	}
	\subfigure[$t = 47\ s$]
	{
		\includegraphics[width=115px]{figure/Exp/E5RE2T1e_4700.png}
		\label{fig:expd}
	}\\
	\caption{Two MMRs transport the rectangular object. The MMRs encounter a dynamic obstacle and started avoidance maneuver (Fig. \ref{fig:expb}). It successfully avoids the dynamic obstacle \ref{fig:expc}) and reaches the goal point \ref{fig:expd})}
	\label{fig:ExpSnap}
\end{figure}

Fig. \ref{fig:ExpSnap} shows the snap of the object transport from the start (Fig. \ref{fig:expa}) to the goal (Fig. \ref{fig:expd}). It encounters a dynamic obstacle and start avoidance maneuver. Fig. \ref{fig:expb} shows when the formation approaches the dynamic obstacle and finally avoids (Fig. \ref{fig:expc}) the obstacle to reach the goal (Fig. \ref{fig:expd}).

\begin{figure}[htbp]
	\centerline{\includegraphics[width = 240px]{figure/Exp/E5RE2T16_Obj_Pose_vi1.png}}
	\caption{Trajectory of the CoM of the object. The subscript d and m of the legend represents the planned and actual values.}
	\label{fig:obj_pose}
\end{figure}

%\begin{figure}[htbp]
%		\centerline{\includegraphics[width = 240px]{figure/Exp/E5RE2T16_Obj_Height_vi.png}}
%		\caption{Height of the CoM of the object.}
%		\label{fig:obj_height}
%\end{figure}

\begin{figure}[htbp]
	\centerline{\includegraphics[width = 240px]{figure/Exp/E5RE2T16_Arm_EE_Constr1.png}}
	\caption{The distance between the two EE during the object transportation.}
	\label{fig:ee_constr}
\end{figure}

Fig. \ref{fig:obj_pose} shows the planned and the actual trajectory of the CoM of the transported object. The position error remains within $0.05\ m$, and the orientation error remains within $0.15\ rad$. The $z$ height of the object's CoM is plotted separately ($z$ Vs $t$ plot) in Fig. \ref{fig:obj_pose} to better understand the object movement in 3D. The  $z$  height increases near $t = 10\ s$ and $t = 28\ s$ before taking sharp turn to reduce the inter robot distance and turning radius. The error in fixed distance between the EEs' grasping point in Fig. \ref{fig:ee_constr} shows that the coordination between the MMRs has been maintained. 
\subsection{Comparison} \label{ComparativeAnalysis}
We compare the computational time of our proposed online motion planning technique with the holonomic MMRs' planning algorithm proposed by \textit{Keshab et al.} \cite{2024b_Keshab} and  \textit{Alonso-Mora et al.} \cite{2017_AlonsoMora} in Table \ref{tab:ComputationTime}. All three algorithms have been implemented in Python for motion planning of two MMRs transporting an object cooperatively in an environment with dynamic obstacles. The MMRs are the same except for the base motion constraints: nonholonomic and holonomic. We have run the planning algorithm's computation study on a Laptop equipped with AMD Rayzen 5800H CPU and 16 GB RAM. The computation time measurements are taken for online local planning of each horizon from twenty-three cases in three different environments with varied goals and obstacle velocities. The details of the cases are available in the \href{https://drive.google.com/file/d/11kg1cFx5Pdsc1OzTVFD5Ny8kpwd0mBUL/view?usp=drive_link}{attached file}. We have also considered two dynamic obstacle cases. The results have been included in the attached \href{https://youtu.be/Hw2sWUVxgYU}{video}. The computation time in Table \ref{tab:ComputationTime} is slightly lower than the algorithm in \cite{2017_AlonsoMora} and higher than the algorithm in \cite{2024b_Keshab} due to its planning complexity arises because of the nonholonomic constraints of the MMRs base. The nonholonomic constraint for the mobile base reduces the solution space compared to the holonomic counterpart, which results in increased computation time for a feasible solution. The proposed approach demonstrates real-time performance using Python. We believe that significantly faster computation would be achieved with C++ implementation. 

\begin{table}[h]
	\centering
	\caption{Computation Time (in seconds) Comparison with the state-of-the-art algorithm.}
	\begin{tabular}{@{}lcccc@{}}
		\hline
		  & Min& Mean & Max & Standard deviation\\
		\hline
		Proposed&$0.199$&$0.580$ & $1.855$& $0.231$ \\
		\hline
        \textit{Keshab et al.}\cite{2024b_Keshab} &$0.227$&$0.272$ & $0.346$& $0.035$ \\
        \hline
		\textit{Alonso-Mora et al.}\cite{2017_AlonsoMora}&$0.46$&$0.857$ & $1.26$& $0.256$ \\
        \hline
	\end{tabular}
	\label{tab:ComputationTime}
\end{table}
