\section{Related Work}
\noindent\textbf{Information Gain. }
The challenge of efficiently selecting new information sources for LLMs can be viewed through the lens of optimal experiment design, a field pioneered by \citet{fedorov1972theory}. This framework emphasizes maximizing information gain while being strategic about resource allocation -- a particularly relevant consideration given the costs associated with integrating new data into LLM systems. Information gain itself has been conceptualized across various fields: in information theory, it relates to reductions in algorithmic information content \cite{cover1989kolmogorov}; in machine learning, it quantifies a feature's contribution to model performance \cite{ODHIAMBOOMUYA2021114765}; and in cognitive science, it represents uncertainty reduction in our experience of the world \cite{damiano2021visual}. While these theoretical frameworks provide valuable insights, they have nott been previously applied to the specific challenge of evaluating the potential value of text collections for enhancing LLM knowledge. Our work bridges this gap by proposing a practical, MCQ-based approach that quantifies information gain by measuring an LLM's ability to answer questions about a text collection with and without access to the source material.\\

\noindent\textbf{Knowledge Detection in LLMs. }
Prior research has developed several methods to analyze how LLMs process and retain textual information. Work on memorization \cite{hartmann2023sokmemorizationgeneralpurposelarge,shi2024detectingpretrainingdatalarge} and data contamination \cite{yax2024assessingcontaminationlargelanguage,golchin2024datacontaminationquiztool} focuses on identifying verbatim recall of training data, while hallucination detection \cite{farquhar_detecting_hallucinations_llms_semantic_entropy} aims to identify when models generate false information. Research on novelty detection has primarily focused on linguistic and semantic novelty \cite{language_novelty_1,semantic_linguistic_novelty_focus_paper}, with less attention paid to factual novelty. While these approaches provide valuable insights into model behavior, they are retrospective -- analyzing what models have already learned or memorized. In contrast, our work takes a prospective approach, developing metrics to evaluate the potential value of new information sources before investing in their integration into LLM systems.