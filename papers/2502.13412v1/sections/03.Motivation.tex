\section{Motivation}
In this section, we will provide a detailed explanation of the limitations of existing methods.
As shown in Fig.~\ref{fig: killing_example}, six texts are randomly selected from Stack Overflow posts, each describing the semantic relations between Java APIs.
Both methods are applied to construct KGs from these texts.

Fig.~\ref{fig: killing_example}-A illustrates the workflow of the representative schema-based method, short for MKC~\cite{huang2022se}.
First, this method relies on annotators to summarize entity types, relation types, and type triples to design the KG schema.
Based on this schema, preset rules (e.g., API\_1 be similar to API\_2) are then used to extract instance triples from the texts.
Finally, the method constructs an API KG containing instance triples with their type information.
For example, the entity type of \textit{Collection.sort()} is method, and the relation type of ``relies on'' is dependency.
Note that no instance triples are extracted from the fourth and sixth texts, as they do not match any type triples in the schema.
However, this method heavily depends on annotators, leading to high labor costs.

One representative schema-free method is GraphRAG~\cite{GraphRAG}.
As shown in Fig.~\ref{fig: killing_example}-B, GraphRAG requires users to specify the entity type according to the task.
Based on these entity types, an LLM is used to extract instance triples from the text, constructing a KG containing five instance triples.
However, due to the lack of guidance on relation types, LLMs fail to accurately analyze the relation between entities.
Therefore, the extracted instance triples are inconsistent with the text semantics, introducing noise and thus reducing the reliability of the KG.
For example, this method extracts the incorrect instance triple (\textit{SortedSet, is a type of, Set}) from the fifth text.
Furthermore, these instance triples lack relation types, which limits the utility of the KG.
For example, the instance triple (\textit{collection.sort(), relies on, Array.aslist()}) belongs to the type triple (\textit{method, null, method}).


Zhang et al.~\cite{EDC} propose another schema-free method, EDC, with its workflow illustrated in Fig.~\ref{fig: killing_example}-C.
First, the method uses an LLM to extract instance triples, forming an initial KG.
It then clusters the extracted relations and defines relation types (e.g., dependency and similarity).
Based on these relation types, EDC refines the initial KG.
By using a text similarity-based search, it recommends candidate relation types to help correct errors made during the extraction phase.
For example, (\textit{SortedSet, is a type of, Set}) can be corrected to (\textit{SortedSet, add elements like, Set}) due to the candidate relation type ``similarity''.
However, although refining phase can correct extracted relations, it cannot correct extracted entities due to the lack of entity types.
Therefore, there is still noise in the KG, which reduces the reliability of the KG.
For example, (\textit{component, depends on, InputStream}) is incorrect because component is not considered an API entity.
Furthermore, the lack of entity types also limits the utility of the KG.


To overcome these limitations, we propose a novel schema-based method, as shown in Fig.~\ref{fig: killing_example}-D.
To reduce labor costs, we use LLMs to extract instance triples from a small amount of user-provided seed text and explore the entity and relation types to generate a potential KG schema.
Based on this schema, LLMs extract instance triples from the target text (e.g., the given six texts) to construct a KG with complete type information.
This KG contains five instance triples (the sixth text does not match any type triples), each labeled with its type information.
However, since our method lacks manual verification, the generated schema may contain some invalid type triples, such as the type triple (\textit{class}, \textit{dependency}, \textit{method}).
This is because, in object-oriented design, class dependencies are typically directed towards other classes or interfaces, rather than specific methods.
Consequently, the KG constructed based on this schema contains suspicious instance triples, such as (\textit{FileWriter}, \textit{relies on}, \textit{flush()}).  
Therefore, it is necessary to verify the potential schema.

Therefore, we apply association rules to the KG construction by calculating the association strength of each type triple and filtering out those with association strength below a certain threshold, along with their instance triples.
Ultimately, we obtain a validated KG schema (including 3 type triples) and a reliable API KG (including 4 instance triples).
Our method can automate the design of the KG schema, thereby alleviating the issue of high labor costs.
Meanwhile, the schema-based approach minimizes noise as much as possible and ensures that the KG contains complete type information.


