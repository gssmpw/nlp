\section{DISCUSSION}
This section includes two parts: one is the threat to validity of our method, and the other is the advantages of this method.

\vspace{-2mm}
\subsection{Threats to Validity}
The validity of this paper faces three main threats.
The first is the manual annotation of experimental results, which may be influenced by the annotator's subjective judgment. 
To minimize this bias, we assign two annotators to each dataset and calculate the kappa coefficient to assess their agreement. 
All kappa values exceed 0.75, indicating a high level of consistency in the annotation results, thereby ensuring the reliability of the experimental results.

Another threat comes from the threshold setting in the KG filtering module.
If the threshold is set too high, it may filter out some correct and valuable type triples; if set too low, it may retain too many low-confidence type triples, reducing the reliability of the KG.
Since exhaustively testing all threshold combinations is impractical, we test five sets of thresholds and select the optimal one.
However, the chosen threshold may still not be the best, and a few valuable type triples still be filtered out, such as (\textit{method}, \textit{replacement}, \textit{method}).

The last threat comes from the choice of seed text.
To ensure a fair comparison, we treat the text used for designing the KG schema in the baseline~\cite{Manual} as the seed text and construct a KG schema with a more diverse entity types and relation types.
However, we do not test the method's performance on other seed texts.
Nevertheless, we find that the KG schema constructed by our method already covers the majority of type information, including 4 entity types and 13 relation types.
In the future, we will apply this method to more API texts to explore whether additional relation types can be discovered.

\subsection{Advantages of Our Method}
Although this method is currently applied to API data, theoretically, with minor adjustments to the prompts, it could be extended to other data domains, thus becoming a universal method for automatic KG construction.
However, achieving true universality presents some challenges. 
The data structures and semantic characteristics differ significantly across different fields, which may limit the adaptability of this method to other domains. 
In the future, we will focus on adjusting this method to adapt to the specific data structures and semantic requirements of different fields.

Integrating our method with existing KG retrieval technologies, such as GraphRAG~\cite{GraphRAG}, can form a comprehensive tool for knowledge extraction, analysis, and utilization. 
This integration leverages GraphRAG's expertise in KG retrieval and optimization, complementing the intelligent construction capabilities of the proposed method, thereby enabling a one-stop solution for KG construction, updating, and application.