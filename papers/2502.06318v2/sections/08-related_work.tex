\section{Related Work}
\label{sec:related_work}

\subsection{Distributed Tracing Systems}

Distributed tracing offers a holistic, end-to-end perspective on data requests as they navigate the myriad services within a distributed application.
% This capability empowers developers to efficiently diagnose service failures, identify performance bottlenecks, and address architectural complexities.
Pioneering works, such as Magpie~\cite{DBLP:conf/hotos/BarhamIMN03}, Whodunit~\cite{DBLP:conf/eurosys/ChandaCZ07}, X-Trace~\cite{DBLP:conf/nsdi/FonsecaPKSS07}, and Dapper~\cite{sigelman2010dapper}, have laid the foundation of tracing in distributed systems.
Recent studies take steps further to address specific challenges to pursue accurate and efficient tracing capabilities.
For example, Pivot Tracing~\cite{DBLP:conf/sosp/MaceRF15} and Canopy~\cite{DBLP:conf/sosp/KaldorMBGKOOSSV17} emphasize cross-application and cross-platform tracing to ensure seamless monitoring and diagnostics across diverse environments.
Panorama~\cite{DBLP:conf/osdi/HuangGLZD18} achieves more sophisticated observability by collecting inter-process and inter-thread error signals.
DeepFlow~\cite{DBLP:conf/sigcomm/ShenZXSLSZWY0XL23} is a non-intrusive distributed tracing framework for troubleshooting microservices.
It establishes a network-centric tracing plane with eBPF technique in the kernel. 
OpenTelemetry~\cite{opentelemetry} standardizes the APIs for telemetry data collection, instrumentation libraries, and semantic conventions.
It can be used with a broad variety of observability backends, such as Jaeger~\cite{jaeger}, Zipkin~\cite{zipkin}, and Prometheus~\cite{prometheus}.

\subsection{Trace-based System Management}

Besides the infrastructure for distributed tracing, trace data are extensively used in many system reliability management tasks.
The empirical study presented in~\cite{DBLP:journals/tse/ZhouPXSJLD21} demonstrates that the current industrial practices of microservice debugging can be enhanced by integrating appropriate tracing and visualization techniques.
MEPFL~\cite{DBLP:conf/sigsoft/Zhou0X0JLXH19} leverages system trace logs to perform latent error prediction and fault localization for microservice applications.
Some works employ traces to construct the dependency graph of microservices for root cause analysis.
This include machine learning-based approaches, such as random walk~\cite{DBLP:conf/sigmetrics/KimSS13}, PageRank~\cite{DBLP:conf/www/YuCCGHJWSL21}, hierarchical clustering and K-means~\cite{DBLP:journals/smr/YuHC23}, and spectral analysis~\cite{DBLP:conf/issre/ZhouZPYLLZZD23}, etc.
In recent years, trace-based fault diagnosis also resorts to deep learning-based approaches, e.g., 
\cite{DBLP:conf/asplos/GanZHCHPD19} combines CNN with LSTM to address the complexity of performance debugging.
TraceAnomaly~\cite{DBLP:conf/issre/ZhouZPYLLZZD23} uses deep Bayesian network to localize anomalous services in an unsupervised way based on trace representation learning.
Some work~\cite{DBLP:conf/sigsoft/YuCLCLZ23,DBLP:conf/kbse/ChenLSZWLYL21,DBLP:conf/icse/LeeYCSL23} utilizes a multi-modal approach, which integrates traces with logs~\cite{DBLP:journals/csur/HeHCYSL21,DBLP:journals/corr/abs-2107-05908,DBLP:conf/ccs/Du0ZS17} and metrics~\cite{DBLP:journals/corr/abs-2308-00393,DBLP:conf/icse/ChenL00LL22} to provide more comprehensive information about system status for microservice troubleshooting.
Traces also serve a crucial role in in analyzing system dependencies~\cite{DBLP:conf/cloud/LuoXLYXZDH021,DBLP:conf/nsdi/WuCP19}, critical paths~\cite{DBLP:conf/usenix/0002RRPSC22}, resource characterization~\cite{DBLP:conf/icpp/WangLWJCWDXHYZ22,DBLP:conf/kbse/LiuJGHCFYYL23}, and microservice architecture~\cite{DBLP:conf/usenix/HuyeSS23,DBLP:conf/sigsoft/0001ZZIGC22}.

\subsection{Trace Sampling and Compression}

In production environments, traces can carry comprehensive details, which, if not managed properly, can lead to significant overheads and potentially impact system performance.
To reduce overheads, the \textit{de facto} practice is to capture fewer traces.
Different from head-based sampling with random trace collection, tail-based solutions enable biased sampling towards more informative and uncommon traces.
Some learning-based approaches~\cite{DBLP:conf/IEEEcloud/ChenJSLZ24,DBLP:conf/icws/HuangCYCZ21,DBLP:conf/cloud/Las-CasasPAM19} have been proposed in this field.
Sieve~\cite{DBLP:conf/icws/HuangCYCZ21} employs Robust Random Cut Forest (RRCF) algorithm, a variant of Isolation Forest, to calculate an attention score for each trace, which is then used to determine its sampling probability.
Sifter~\cite{DBLP:conf/cloud/Las-CasasPAM19} captures edge-case traces by learning an unbiased, low-dimensional model to reconstruct the fixed-length sub-paths of traces.
A larger reconstruction loss indicates a higher sampling probability.
SampleHST~\cite{DBLP:conf/noms/GiasGSPOC23} and Perch utilize clustering techniques to divide traces into different groups, and sampling is then performed in each group.
STEAM~\cite{DBLP:conf/sigsoft/HeFLZ0LR023} preserves system observability by sampling mutually dissimilar traces.
It employs Graph Neural Networks (GNN) for trace representation, and requires human labeling to incorporate domain knowledge.
Hindsight~\cite{DBLP:conf/nsdi/ZhangXAVM23} introduces the idea of retroactive sampling, which combines the advantages of head-based and tail-based sampling.
Specifically, it allows the tracing for all requests at the service side, but reports trace data only after outlier symptoms are detected.

Different from prior research, \alias introduces an innovative methodology for mitigating the overhead associated with distributed tracing.
Rather than forecasting the significance of individual traces, \alias focuses on the compression of spans by leveraging their inherent redundancy.
Our approach can work with existing techniques complementarily to enhance the tracing performance.
It efficiently manages data volume while ensuring that the essential insights provided by trace data are preserved.

% \cite{DBLP:conf/vldb/BergHCJSNN23}

% \subsection{Distributed Tracing Systems}

% Distributed tracing has become an essential instrument for monitoring and optimizing modern service-based architectures. 
% It offers a holistic, end-to-end perspective on data requests as they navigate the myriad services within a distributed application.
% This capability empowers developers to efficiently diagnose service failures, identify performance bottlenecks, and address architectural complexities.
% Pioneering works, such as Magpie~\cite{DBLP:conf/hotos/BarhamIMN03}, Whodunit~\cite{DBLP:conf/eurosys/ChandaCZ07}, X-Trace~\cite{DBLP:conf/nsdi/FonsecaPKSS07}, and Dapper~\cite{sigelman2010dapper}, have laid the foundation of tracing in distributed systems.
% Recent studies take steps further to address specific challenges to pursue accurate and efficient tracing capabilities.
% Pivot Tracing~\cite{DBLP:conf/sosp/MaceRF15} and Canopy~\cite{DBLP:conf/sosp/KaldorMBGKOOSSV17} emphasize cross-application and cross-platform tracing to ensure seamless monitoring and diagnostics across diverse environments.
% Panorama~\cite{DBLP:conf/osdi/HuangGLZD18} achieves more sophisticated observability by collecting inter-process and inter-thread error signals.
% DeepFlow~\cite{DBLP:conf/sigcomm/ShenZXSLSZWY0XL23} is a non-intrusive distributed tracing framework for troubleshooting microservices.
% It establishes a network-centric tracing plane with eBPF technique in the kernel. 
% OpenTelemetry~\cite{opentelemetry} standardizes the APIs for telemetry data collection, instrumentation libraries, and semantic conventions.
% It can be used with a broad variety of observability backends, such as Jaeger~\cite{jaeger}, Zipkin~\cite{zipkin}, and Prometheus~\cite{prometheus}.

% \subsection{Trace-based System Management}

% Besides the infrastructure for distributed tracing, trace data are extensively used in many system reliability management tasks.
% The empirical study presented in~\cite{DBLP:journals/tse/ZhouPXSJLD21} demonstrates that the current industrial practices of microservice debugging can be enhanced by integrating appropriate tracing and visualization techniques.
% MEPFL~\cite{DBLP:conf/sigsoft/Zhou0X0JLXH19} leverages system trace logs to perform latent error prediction, fault localization, and fault type identification for microservice applications.
% Some works employ traces to construct the dependency graph of microservices for root cause analysis.
% This include machine learning-based approaches, such as random walk~\cite{DBLP:conf/sigmetrics/KimSS13}, PageRank~\cite{DBLP:conf/www/YuCCGHJWSL21}, hierarchical clustering and K-means~\cite{DBLP:journals/smr/YuHC23}, and spectral analysis~\cite{DBLP:conf/issre/ZhouZPYLLZZD23}, etc.
% In recent years, trace-based fault diagnosis also resorts to deep learning-based approaches, e.g., 
% \cite{DBLP:conf/asplos/GanZHCHPD19} combines CNN with LSTM to address the complexity of performance debugging.
% TraceAnomaly~\cite{DBLP:conf/issre/ZhouZPYLLZZD23} uses deep Bayesian network to localize anomalous services in an unsupervised way based on trace representation learning.
% Some work~\cite{DBLP:conf/sigsoft/YuCLCLZ23,DBLP:conf/kbse/ChenLSZWLYL21,DBLP:conf/icse/LeeYCSL23} utilizes a multi-modal approach, which integrates traces with logs~\cite{DBLP:journals/csur/HeHCYSL21,DBLP:journals/corr/abs-2107-05908,DBLP:conf/ccs/Du0ZS17} and metrics~\cite{DBLP:journals/corr/abs-2308-00393,DBLP:conf/icse/ChenL00LL22} to provide more comprehensive information about system status for microservice troubleshooting.
% Traces also serve a crucial role in in analyzing system dependencies~\cite{DBLP:conf/cloud/LuoXLYXZDH021,DBLP:conf/nsdi/WuCP19}, critical paths~\cite{DBLP:conf/usenix/0002RRPSC22}, resource characterization~\cite{DBLP:conf/icpp/WangLWJCWDXHYZ22,DBLP:conf/kbse/LiuJGHCFYYL23}, and microservice architecture~\cite{DBLP:conf/usenix/HuyeSS23,DBLP:conf/sigsoft/0001ZZIGC22}.

% \subsection{Trace Sampling and Compression}

% In production environments, traces can carry comprehensive details, which, if not managed properly, can lead to significant overheads and potentially impact system performance.
% To reduce overheads, the \textit{de facto} practice is to capture fewer traces.
% Different from head-based sampling with random trace collection, tail-based solutions enable biased sampling towards more informative and uncommon traces.
% Some learning-based approaches~\cite{DBLP:journals/corr/abs-2406-06975,DBLP:conf/icws/HuangCYCZ21,DBLP:conf/cloud/Las-CasasPAM19} have been proposed in this field.
% Sieve~\cite{DBLP:conf/icws/HuangCYCZ21} employs Robust Random Cut Forest (RRCF) algorithm, a variant of Isolation Forest, to calculate an attention score for each trace, which is then used to determine its sampling probability.
% Sifter~\cite{DBLP:conf/cloud/Las-CasasPAM19} captures edge-case traces by learning an unbiased, low-dimensional model to reconstruct the fixed-length sub-paths of traces.
% A larger reconstruction loss indicates a higher sampling probability.
% SampleHST~\cite{DBLP:conf/noms/GiasGSPOC23} and Perch utilize clustering techniques to divide traces into different groups, and sampling is then performed in each group.
% STEAM~\cite{DBLP:conf/sigsoft/HeFLZ0LR023} preserves system observability by sampling mutually dissimilar traces.
% It employs Graph Neural Networks (GNN) for trace representation, and requires human labeling to incorporate domain knowledge.
% Hindsight~\cite{DBLP:conf/nsdi/ZhangXAVM23} introduces the idea of retroactive sampling, which combines the advantages of head-based and tail-based sampling.
% Specifically, it allows the tracing for all requests at the service side, but reports trace data only after outlier symptoms are detected.

% Different from prior research, \alias introduces an innovative methodology for mitigating the overhead associated with distributed tracing.
% Rather than forecasting the significance of individual traces, \alias focuses on the compression of spans by leveraging their inherent redundancy.
% Our approach can work with existing techniques complementarily to enhance the tracing performance.
% It efficiently manages data volume while ensuring that the essential insights provided by trace data are preserved.

% % \cite{DBLP:conf/vldb/BergHCJSNN23}