%%
%% This is file `sample-acmsmall-conf.tex',
%% generated with the docstrip utility.
%%
%% The original source files were:
%%
%% samples.dtx  (with options: `all,proceedings,bibtex,acmsmall-conf')
%% 
%% IMPORTANT NOTICE:
%% 
%% For the copyright see the source file.
%% 
%% Any modified versions of this file must be renamed
%% with new filenames distinct from sample-acmsmall-conf.tex.
%% 
%% For distribution of the original source see the terms
%% for copying and modification in the file samples.dtx.
%% 
%% This generated file may be distributed as long as the
%% original source files, as listed above, are part of the
%% same distribution. (The sources need not necessarily be
%% in the same archive or directory.)
%%
%%
%% Commands for TeXCount
%TC:macro \cite [option:text,text]
%TC:macro \citep [option:text,text]
%TC:macro \citet [option:text,text]
%TC:envir table 0 1
%TC:envir table* 0 1
%TC:envir tabular [ignore] word
%TC:envir displaymath 0 word
%TC:envir math 0 word
%TC:envir comment 0 0
%%
%%
%% The first command in your LaTeX source must be the \documentclass
%% command.
%%
%% For submission and review of your manuscript please change the
%% command to \documentclass[manuscript, screen, review]{acmart}.
%%
%% When submitting camera ready or to TAPS, please change the command
%% to \documentclass[sigconf]{acmart} or whichever template is required
%% for your publication.
%%
%%
% \documentclass[acmsmall,screen,review,anonymous]{acmart}
\documentclass[acmsmall]{acmart}

% \usepackage{cite}
% \usepackage{amsmath,amssymb,amsfonts}
\usepackage{graphicx}
\usepackage{textcomp}
\usepackage{xcolor}
\usepackage{balance}
\usepackage{booktabs}
\usepackage{CJKutf8}
\usepackage{tikz}
\usepackage{forest}
\usepackage{booktabs}
\usepackage{makecell}
\usepackage{float}
\usepackage{algorithm}
\usepackage{algpseudocode}
\usepackage{xspace}
\usepackage{listings}
% \usepackage{minted}
\usepackage[symbol]{footmisc}
\usepackage{longtable}
\usepackage{nicematrix}
\usepackage{colortbl}
\usepackage{caption}
\usepackage{fontawesome5}
\usepackage{pifont}
\usepackage{subcaption}
\usepackage{multirow}
\usepackage{booktabs}
\usepackage{arydshln}
\usepackage{tabularx}
\usepackage{soul}
\usepackage{hyperref}
\usepackage[]{collab}

\renewcommand{\arraystretch}{1.72}
\setlength{\arrayrulewidth}{0.15mm}
\newcolumntype{C}[1]{>{\centering\arraybackslash}p{#1}}

\definecolor{lightgrey}{rgb}{0.96, 0.96, 0.96}
\definecolor{grey}{rgb}{0.92, 0.92, 0.92}
\definecolor{keycolor}{HTML}{2B4983}
\definecolor{valuecolor}{HTML}{61982D}

\newcommand\alias{\textsc{Tracezip}\xspace}
\newcommand\name{Span Retrieval Tree\xspace}
\newcommand\sname{SRT\xspace}
\newcommand{\red}[1]{\textcolor{red}{#1}}
\newcommand{\blue}[1]{\textcolor{blue}{#1}}
\collabAuthor{zb}{teal}{Zhuangbin}

%%
%% \BibTeX command to typeset BibTeX logo in the docs
\AtBeginDocument{%
  \providecommand\BibTeX{{%
    Bib\TeX}}}

%% Rights management information.  This information is sent to you
%% when you complete the rights form.  These commands have SAMPLE
%% values in them; it is your responsibility as an author to replace
%% the commands and values with those provided to you when you
%% complete the rights form.
% \setcopyright{acmlicensed}
% \copyrightyear{2025}
% \acmYear{2025}
% \acmDOI{10.1145/3728888}

%% These commands are for a PROCEEDINGS abstract or paper.
% \acmConference[ISSTA '25]{The ACM SIGSOFT International Symposium on Software Testing and Analysis}{June 25--28,
%   2025}{Trondheim, Norway}
%%
%%  Uncomment \acmBooktitle if the title of the proceedings is different
%%  from ``Proceedings of ...''!
%%
%%\acmBooktitle{Woodstock '18: ACM Symposium on Neural Gaze Detection,
%%  June 03--05, 2018, Woodstock, NY}
% \acmISBN{978-1-4503-XXXX-X/18/06}


%%
%% Submission ID.
%% Use this when submitting an article to a sponsored event. You'll
%% receive a unique submission ID from the organizers
%% of the event, and this ID should be used as the parameter to this command.
%%\acmSubmissionID{123-A56-BU3}

%%
%% For managing citations, it is recommended to use bibliography
%% files in BibTeX format.
%%
%% You can then either use BibTeX with the ACM-Reference-Format style,
%% or BibLaTeX with the acmnumeric or acmauthoryear sytles, that include
%% support for advanced citation of software artefact from the
%% biblatex-software package, also separately available on CTAN.
%%
%% Look at the sample-*-biblatex.tex files for templates showcasing
%% the biblatex styles.
%%

%%
%% The majority of ACM publications use numbered citations and
%% references.  The command \citestyle{authoryear} switches to the
%% "author year" style.
%%
%% If you are preparing content for an event
%% sponsored by ACM SIGGRAPH, you must use the "author year" style of
%% citations and references.
%% Uncommenting
%% the next command will enable that style.
%%\citestyle{acmauthoryear}


%%
%% end of the preamble, start of the body of the document source.
\begin{document}

%%
%% The "title" command has an optional parameter,
%% allowing the author to define a "short title" to be used in page headers.
\title{\alias: Efficient Distributed Tracing via Trace Compression}

\author{Zhuangbin Chen}
\orcid{0000-0001-5158-6716}
\affiliation{%
  \institution{School of Software Engineering, Sun Yat-sen University}
  \city{Zhuhai}
  \country{China}
}
\email{chenzhb36@mail.sysu.edu.cn}

\author{Junsong Pu}
\orcid{0009-0002-8309-1384}
\affiliation{%
  \institution{Beijing University of Posts and Telecommunication}
  \city{Beijing}
  \country{China}
}
\email{angrychow@bupt.edu.cn}

\author{Zibin Zheng}
\orcid{0000-0002-7878-4330}
\affiliation{%
  \institution{School of Software Engineering, Sun Yat-sen University}
  \city{Zhuhai}
  \country{China}
}
\email{zhzibin@mail.sysu.edu.cn}
\authornote{Zibin Zheng is the corresponding author.}

%%
%% The "author" command and its associated commands are used to define
%% the authors and their affiliations.
%% Of note is the shared affiliation of the first two authors, and the
%% "authornote" and "authornotemark" commands
%% used to denote shared contribution to the research.
% \author{Ben Trovato}
% \authornote{Both authors contributed equally to this research.}
% \email{trovato@corporation.com}
% \orcid{1234-5678-9012}
% \author{G.K.M. Tobin}
% \authornotemark[1]
% \email{webmaster@marysville-ohio.com}
% \affiliation{%
%   \institution{Institute for Clarity in Documentation}
%   \city{Dublin}
%   \state{Ohio}
%   \country{USA}
% }

% \author{Lars Th{\o}rv{\"a}ld}
% \affiliation{%
%   \institution{The Th{\o}rv{\"a}ld Group}
%   \city{Hekla}
%   \country{Iceland}}
% \email{larst@affiliation.org}


%%
%% By default, the full list of authors will be used in the page
%% headers. Often, this list is too long, and will overlap
%% other information printed in the page headers. This command allows
%% the author to define a more concise list
%% of authors' names for this purpose.
\renewcommand{\shortauthors}{Zhuangbin Chen, Junsong Pu, and Zibin Zheng}

%%
%% The abstract is a short summary of the work to be presented in the
%% article.
\begin{abstract}
\textit{Distributed tracing} serves as a fundamental building block in the monitoring and testing of cloud service systems.
To reduce computational and storage overheads, the \textit{de facto} practice is to capture fewer traces via sampling.
However, existing work faces a trade-off between the completeness of tracing and system overhead.
On one hand, \textit{head-based sampling} indiscriminately selects requests to trace when they enter the system, which may miss critical events.
On the other hand, \textit{tail-based sampling} first captures all requests and then selectively persists the edge-case traces, which entails the overheads related to trace collection and ingestion.
Taking a different path, we propose \alias in this paper to enhance the efficiency of distributed tracing via \textit{trace compression}.
Our key insight is that there exists significant redundancy among traces, which results in repetitive transmission of identical data between services and the backend.
We design a new data structure named \name (\sname) that continuously encapsulates such redundancy at the service side and transforms trace spans into a lightweight form.
At the backend, the complete traces can be seamlessly reconstructed by retrieving the common data that are already delivered by previous spans.
\alias includes a series of strategies to optimize the structure of \sname and a differential update mechanism to efficiently synchronize \sname between services and the backend.
Our evaluation on microservices benchmarks, popular cloud service systems, and production trace data demonstrates that \alias can achieve substantial performance gains in trace collection with negligible overhead.
We have implemented \alias inside the OpenTelemetry Collector, making it compatible with existing tracing APIs.
\end{abstract}


\setcopyright{acmlicensed}
\acmJournal{PACMSE}
\acmYear{2025} \acmVolume{2} \acmNumber{ISSTA} \acmArticle{ISSTA019} \acmMonth{7}\acmDOI{10.1145/3728888}


%%
%% The code below is generated by the tool at http://dl.acm.org/ccs.cfm.
%% Please copy and paste the code instead of the example below.
%%
\begin{CCSXML}
<ccs2012>
   <concept>
       <concept_id>10010520.10010521.10010537.10003100</concept_id>
       <concept_desc>Computer systems organization~Cloud computing</concept_desc>
       <concept_significance>500</concept_significance>
       </concept>
   <concept>
       <concept_id>10010520.10010575.10010577</concept_id>
       <concept_desc>Computer systems organization~Reliability</concept_desc>
       <concept_significance>500</concept_significance>
       </concept>
   <concept>
       <concept_id>10010520.10010575.10010579</concept_id>
       <concept_desc>Computer systems organization~Maintainability and maintenance</concept_desc>
       <concept_significance>500</concept_significance>
       </concept>
   <concept>
       <concept_id>10010520.10010575.10010755</concept_id>
       <concept_desc>Computer systems organization~Redundancy</concept_desc>
       <concept_significance>500</concept_significance>
       </concept>
   <concept>
       <concept_id>10011007.10011006.10011073</concept_id>
       <concept_desc>Software and its engineering~Software maintenance tools</concept_desc>
       <concept_significance>500</concept_significance>
       </concept>
 </ccs2012>
\end{CCSXML}

\ccsdesc[500]{Computer systems organization~Cloud computing}
\ccsdesc[500]{Computer systems organization~Reliability}
\ccsdesc[500]{Computer systems organization~Maintainability and maintenance}
\ccsdesc[500]{Computer systems organization~Redundancy}
\ccsdesc[500]{Software and its engineering~Software maintenance tools}

%%
%% Keywords. The author(s) should pick words that accurately describe
%% the work being presented. Separate the keywords with commas.
\keywords{Distributed tracing, Trace compression, Cloud computing, System monitoring}
%% A "teaser" image appears between the author and affiliation
%% information and the body of the document, and typically spans the
%% page.

% \received{20 February 2007}
% \received[revised]{12 March 2009}
% \received[accepted]{5 June 2009}

%%
%% This command processes the author and affiliation and title
%% information and builds the first part of the formatted document.
\maketitle


% humans are sensitive to the way information is presented.

% introduce framing as the way we address framing. say something about political views and how information is represented.

% in this paper we explore if models show similar sensitivity.

% why is it important/interesting.



% thought - it would be interesting to test it on real world data, but it would be hard to test humans because they come already biased about real world stuff, so we tested artificial.


% LLMs have recently been shown to mimic cognitive biases, typically associated with human behavior~\citep{ malberg2024comprehensive, itzhak-etal-2024-instructed}. This resemblance has significant implications for how we perceive these models and what we can expect from them in real-world interactions and decisionmaking~\citep{eigner2024determinants, echterhoff-etal-2024-cognitive}.

The \textit{framing effect} is a well-known cognitive phenomenon, where different presentations of the same underlying facts affect human perception towards them~\citep{tversky1981framing}.
For example, presenting an economic policy as only creating 50,000 new jobs, versus also reporting that it would cost 2B USD, can dramatically shift public opinion~\cite{sniderman2004structure}. 
%%%%%%%% 图1:  %%%%%%%%%%%%%%%%
\begin{figure}[t]
    \centering
    \includegraphics[width=\columnwidth]{Figs/01.pdf}
    \caption{Performance comparison (Top-1 Acc (\%)) under various open-vocabulary evaluation settings where the video learners except for CLIP are tuned on Kinetics-400~\cite{k400} with frozen text encoders. The satisfying in-context generalizability on UCF101~\cite{UCF101} (a) can be severely affected by static bias when evaluating on out-of-context SCUBA-UCF101~\cite{li2023mitigating} (b) by replacing the video background with other images.}
    \label{fig:teaser}
\end{figure}


Previous research has shown that LLMs exhibit various cognitive biases, including the framing effect~\cite{lore2024strategic,shaikh2024cbeval,malberg2024comprehensive,echterhoff-etal-2024-cognitive}. However, these either rely on synthetic datasets or evaluate LLMs on different data from what humans were tested on. In addition, comparisons between models and humans typically treat human performance as a baseline rather than comparing patterns in human behavior. 
% \gabis{looks good! what do we mean by ``most studies'' or ``rarely'' can we remove those? or we want to say that we don't know of previous work doing both at the same time?}\gili{yeah the main point is that some work has done each separated, but not all of it together. how about now?}

In this work, we evaluate LLMs on real-world data. Rather than measuring model performance in terms of accuracy, we analyze how closely their responses align with human annotations. Furthermore, while previous studies have examined the effect of framing on decision making, we extend this analysis to sentiment analysis, as sentiment perception plays a key explanatory role in decision-making \cite{lerner2015emotion}. 
%Based on this, we argue that examining sentiment shifts in response to reframing can provide deeper insights into the framing effect. \gabis{I don't understand this last claim. Maybe remove and just say we extend to sentiment analysis?}

% Understanding how LLMs respond to framing is crucial, as they are increasingly integrated into real-world applications~\citep{gan2024application, hurlin2024fairness}.
% In some applications, e.g., in virtual companions, framing can be harnessed to produce human-like behavior leading to better engagement.
% In contrast, in other applications, such as financial or legal advice, mitigating the effect of framing can lead to less biased decisions.
% In both cases, a better understanding of the framing effect on LLMs can help develop strategies to mitigate its negative impacts,
% while utilizing its positive aspects. \gabis{$\leftarrow$ reading this again, maybe this isn't the right place for this paragraph. Consider putting in the conclusion? I think that after we said that people have worked on it, we don't necessarily need this here and will shorten the long intro}


% If framing can influence their outputs, this could have significant societal effects,
% from spreading biases in automated decision-making~\citep{ghasemaghaei2024understanding} to reducing public trust in AI-generated content~\citep{afroogh2024trust}. 
% However, framing is not inherently negative -- understanding how it affects LLM outputs can offer valuable insights into both human and machine cognition.
% By systematically investigating the framing effect,


%It is therefore crucial to systematically investigate the framing effect, to better understand and mitigate its impact. \gabis{This paragraph is important - I think that right now it's saying that we don't want models to be influenced by framing (since we want to mitigate its impact, right?) When we talked I think we had a more nuanced position?}




To better understand the framing effect in LLMs in comparison to human behavior,
we introduce the \name{} dataset (Section~\ref{sec:data}), comprising 1,000 statements, constructed through a three-step process, as shown in Figure~\ref{fig:fig1}.
First, we collect a set of real-world statements that express a clear negative or positive sentiment (e.g., ``I won the highest prize'').
%as exemplified in Figure~\ref{fig:fig1} -- ``I won the highest prize'' positive base statement. (2) next,
Second, we \emph{reframe} the text by adding a prefix or suffix with an opposite sentiment (e.g., ``I won the highest prize, \emph{although I lost all my friends on the way}'').
Finally, we collect human annotations by asking different participants
if they consider the reframed statement to be overall positive or negative.
% \gabist{This allows us to quantify the extent of \textit{sentiment shifts}, which is defined as labeling the sentiment aligning with the opposite framing, rather then the base sentiment -- e.g., voting ``negative'' for the statement ``I won the highest prize, although I lost all my friends on the way'', as it aligns with the opposite framing sentiment.}
We choose to annotate Amazon reviews, where sentiment is more robust, compared to e.g., the news domain which introduces confounding variables such as prior political leaning~\cite{druckman2004political}.


%While the implications of framing on sensitive and controversial topics like politics or economics are highly relevant to real-world applications, testing these subjects in a controlled setting is challenging. Such topics can introduce confounding variables, as annotators might rely on their personal beliefs or emotions rather than focusing solely on the framing, particularly when the content is emotionally charged~\cite{druckman2004political}. To balance real-world relevance with experimental reliability, we chose to focus on statements derived from Amazon reviews. These are naturally occurring, sentiment-rich texts that are less likely to trigger strong preexisting biases or emotional reactions. For instance, a review like ``The book was engaging'' can be framed negatively without invoking specific cultural or political associations. 

 In Section~\ref{sec:results}, we evaluate eight state-of-the-art LLMs
 % including \gpt{}~\cite{openai2024gpt4osystemcard}, \llama{}~\cite{dubey2024llama}, \mistral{}~\cite{jiang2023mistral}, \mixtral{}~\cite{mistral2023mixtral}, and \gemma{}~\cite{team2024gemma}, 
on the \name{} dataset and compare them against human annotations. We find  that LLMs are influenced by framing, somewhat similar to human behavior. All models show a \emph{strong} correlation ($r>0.57$) with human behavior.
%All models show a correlation with human responses of more than $0.55$ in Pearson's $r$ \gabis{@Gili check how people report this?}.
Moreover, we find that both humans and LLMs are more influenced by positive reframing rather than negative reframing. We also find that larger models tend to be more correlated with human behavior. Interestingly, \gpt{} shows the lowest correlation with human behavior. This raises questions about how architectural or training differences might influence susceptibility to framing. 
%\gabis{this last finding about \gpt{} stands in opposition to the start of the statement, right? Even though it's probably one of the largest models, it doesn't correlate with humans? If so, better to state this explicitly}

This work contributes to understanding the parallels between LLM and human cognition, offering insights into how cognitive mechanisms such as the framing effect emerge in LLMs.\footnote{\name{} data available at \url{https://huggingface.co/datasets/gililior/WildFrame}\\Code: ~\url{https://github.com/SLAB-NLP/WildFrame-Eval}}

%\gabist{It also raises fundamental philosophical and practical questions -- should LLMs aim to emulate human-like behavior, even when such behavior is susceptible to harmful cognitive biases? or should they strive to deviate from human tendencies to avoid reproducing these pitfalls?}\gabis{$\leftarrow$ also following Itay's comment, maybe this is better in the dicsussion, since we don't address these questions in the paper.} %\gabis{This last statement brings the nuance back, so I think it contradicts the previous parapgraph where we talked about ``mitigating'' the effect of framing. Also, I think it would be nice to discuss this a bit more in depth, maybe in the discussion section.}






\section{Background on Causal Inference}
\label{sec:background-causal} 



 \newtextold{In this section, we 
 %formalize the notion of {\em Average Treatment Effect and understand the 
 review the basic concepts and key assumptions for inferring the effects of an intervention on the outcome on collected datasets without performing randomized controlled experiments. 
We use {\em Pearl's graphical causal model} for {\em observational causal analysis} \cite{pearl2009causal} to define these concepts.}


\par
\paratitle{Causal Inference and Causal DAGs} The primary goal of causal inference is to model causal dependencies between attributes and evaluate how changing one variable (referred to as intervention) would affect the other.
Pearl's Probabilistic Graphical Causal Model \cite{pearl2009causal} can be written as a tuple $(\exo, \edvar, Pr_{\exo}, \psi)$, where $\exo$ is a set of {\em exogenous} variables, $\Pr_{\exo}$ is the joint distribution of \exo, and $\edvar$ is a set of observed {\em endogenous variables}.
Here $\psi$ is a set of structural equations that encode dependencies among variables. The equation for $A \in \edvar$ takes the following form:
%that encode the dependencies among the variables.  These equations are of the form 
$$\psi_{A}: 
\dom(Pa_{\exo}(A)) {\times} \dom(Pa_{\edvar}(A)) \to \dom(A)$$
Here $Pa_{\exo}(A) {\subseteq} {\exo}$ and $Pa_{\edvar}(A) {\subseteq} \edvar \setminus \{A\}$ respectively denote the exogenous and endogenous parents of $A$. A causal relational model is associated with a directed acyclic graph ({\em causal DAG}) $G$, whose nodes are the endogenous variables $\edvar$ and there is a directed edge from $X$ to $O$ if  $X {\in} Pa_{\edvar}(O)$. The causal DAG obfuscates exogenous variables as they are unobserved. %Any given set of values for the exogenous variables completely determines the values of the endogenous variables by the structural equations (we do not need any known closed-form expressions of the structural equations in this work). 
The probability distribution $\Pr_{\exo}$ on exogenous variables $\exo$ induces a probability distribution  
on the endogenous variables $\edvar$ by the structural equations $\psi$.  A causal DAG can be constructed by a domain expert as in the above example, or using existing {\em causal discovery} algorithms~\cite{glymour2019review}. 



\begin{figure}
    \centering
    \small
    \begin{tikzpicture}[node distance=0.6cm and 1cm, every node/.style={minimum size=0.5cm}]
        \tikzset{vertex/.style = {draw, circle, align=center}}

        \node[vertex] (Ethnicity) {\bf\scriptsize{{Ethnicity}}};
        \node[vertex, right=0.3cm of Ethnicity] (Gender) {\bf{\scriptsize{Gender}}};
        \node[vertex, right=0.3cm of Gender] (Age) {\bf{\scriptsize{Age}}};
        \node[vertex, below=0.3cm of Gender] (Role) {\bf{\scriptsize{Role}}};
        \node[vertex, right=0.3cm of Role] (Education) {\bf{\small{\scriptsize{Education}}}};
        \node[vertex, below=0.3cm of Role] (Salary) {\bf{\scriptsize{Salary}}};

        \draw[->] (Ethnicity) -- (Salary);
        \draw[->] (Gender) -- (Role);
        \draw[->] (Age) -- (Role);
         \draw[->] (Education) -- (Role);
           \draw[->] (Education) -- (Salary);
             \draw[->] (Ethnicity) -- (Education);
                \draw[->] (Ethnicity) -- (Role);
             \draw[->] (Gender) -- (Education);
               \draw[->] (Age) -- (Education);
                 \draw[->] (Role) -- (Salary);
        \draw[->] (Gender) to[bend right] (Salary);
        \draw[->] (Age) -- (Salary);
    \end{tikzpicture}
    \caption{Partial causal DAG for the Stack Overflow dataset.}
    \label{fig:causal_DAG}
\end{figure}



 \begin{example}
Figure \ref{fig:causal_DAG} depicts a partial causal DAG for the SO dataset over the attributes in Table \ref{tab:data} as endogenous variables (we use a larger causal DAG with all 20 attributes in our experiments). 
  Given this causal DAG, we can observe that the role that a coder has in their company depends on their education, age gender and ethnicity.
\end{example}
\par


\par
\paratitle{Intervention} In Pearl's model, a treatment $T = t$ (on one or more variables) is considered as an {\em intervention} to a causal DAG by mechanically changing the DAG such that the values of node(s) of $T$ in $G$ are set to the value(s) in $t$, which is denoted by $\doop(T = t)$. Following this operation, the probability distribution of the nodes in the graph changes as the treatment nodes no longer depend on the values of their parents. Pearl's model gives an approach to estimate the new probability distribution by identifying the confounding factors $Z$ described earlier using conditions such as {\em d-separation} and {\em backdoor criteria} \cite{pearl2009causal}, which we do not discuss in this paper.


\par
\paratitle{Average Treatment Effect} The effects of an intervention are often measured by evaluating
% \par
% \paratitle{Causal inference, Treatment, ATE, and CATE}
% \newtextold{One of the primary goals  of {\em causal inference} is to estimate the effect of making a change in terms of a {\em treatment} $T$ (often referred to as an intervention)
% on the outcome $O$. 
% %A variable that is modified is often referred to as the treatment variable $T$ and the metric used to captures 
% The effect of treatment $T$ on outcome $O$ is measured by 
% %is known as 
{\em Conditional Average treatment effect (CATE)}, 
%a {\em treatment variable} $T$ on an outcome variable $O$ (e.g., what is the effect of higher \verb|Education| on \verb|Salary|). 
measuring the effect of an intervention on a subset of records~\cite{rubin1971use,holland1986statistics} by calculating the difference in average outcomes between the group that receives the treatment and the group that does not (called the {\em control} group), providing an estimate of how the intervention by $T$ influences an outcome $O$ for a given subpopulation. 
% Mathematically,
% \begin{equation}
%     %{\small ATE(T,O) = \mathbb{E}[O \mid \doop(T=1)] -      \mathbb{E}[O \mid \doop(T=0)]}
%     {\small ATE(T, O) = \mathbb{E}[O \mid \doop(T=1)] -  
%     \mathbb{E}[O \mid \doop(T=0)]}
% \label{eq:ate}
% \end{equation}
% In our work, where the treatment with maximum effect may vary among different subpopulations, we are interested in computing the \emph{Conditional Average Treatment Effect} (CATE), which measures the effect of a treatment on an outcome on \emph{a subset of input units}~\cite{rubin1971use,holland1986statistics}. 
Given a subset of the records defined by (a vector of) attributes $B$ and their values $b$, 
%g {\in} \Qagg(\db)$ defined by a predicate $G {=} g$ 
we can compute $CATE(T,O \mid B = b)$ as:
{
\begin{eqnarray}    
    %CATE(T,O \mid G=g) = \mathbb{E}[O \mid \doop(T=1)&, G=g] -  \mathbb{E}[O \mid \doop(T=0), G=g] 
   % CATE(T,O \mid B = b) = 
    \mathbb{E}[O \mid \doop(T=1), B = b] -  
    \mathbb{E}[O \mid \doop(T=0), B = b]\label{eq:cate}
\end{eqnarray}
}
Setting $B=\phi$ is equivalent to the ATE estimate.
The above definitions assumes that the treatment assigned to one unit does not affect the outcome of another unit (called the {Stable Unit Treatment Value Assumption (SUTVA)) \cite{rubin2005causal}}\footnote{This assumption does not hold for causal inference on multiple tables and even on a single table where tuples depend on each other.}. 


The ideal way of estimating the ATE and CATE is through {\em randomized controlled experiments}, 
where the population is randomly divided into two groups (treated and control, for binary treatments): 
%treated group that receives the treatment and control group that does not (denoted by 
%{the \em treated} group 
denoted by 
$\doop(T = 1)$ 
%for a binary treatment)  (the {\em control} group, 
and $\doop(T = 0)$ resp.)~\cite{pearl2009causal}.
%\sr{edited up to here, going to read the rest first, this section should not look like causumx}
%\par
%\par
However, randomized experiments cannot always be performed due to ethical or feasibility issues. In these scenarios, observational data is used to estimate the treatment effect, which requires the following additional assumptions. 
% {\em Observational Causal Analysis} still allows sound causal inference under additional assumptions. Randomization in controlled trials mitigates the effect of {\em confounding factors}, i.e., attributes that can affect the treatment assignment and outcome. Suppose we want to understand the causal effect of \verb|Education| on \verb|Salary| from the SO dataset.  %in Example~\ref{ex:running_example}. 
% We no longer apply Eq. (\ref{eq:ate}) since the values of \verb|Education| were not assigned at random in this data, and obtaining higher education largely depends on other attributes like \verb|Gender|, \verb|Age|, and \verb|Country|. 
% Pearl's model provides ways to account for these confounding attributes $Z$ to get an unbiased causal estimate from observational data under the following assumptions ($\independent$ denotes independence):
% \vspace{-2mm}
\newtextold{
The first assumption is called {\em unconfoundedness} or {\em strong ignorability}  \cite{rosenbaum1983central} says that the independence of outcome $O$ and treatment $T$ conditioning on a set of confounder variables  (covariates) $Z$, i.e.,
%\begin{eqnarray}
 $    O \independent T | Z {=} z$.
 %\label{eq:unconfoundedness}
%\end{eqnarray}
The second assumption called {\em overlap or positivity} says that there is a chance of observing individuals in both the treatment and control groups for every combination of covariate values, i.e., 
%\begin{eqnarray}
   $ 0 < Pr(T {=} 1 ~~|~~Z {=} z)< 1 $.
   %\label{eq:overlap}
%\end{eqnarray}
}
%\sg{Is this overlap or positivity? maybe both are the same?} \sr{yeah - same - from Google AI - The overlap assumption, also known as the positivity assumption, is a key assumption in causal inference that states that there is a chance of observing individuals in both the treatment and control groups for every combination of covariate values.}
% The above conditions are known as {\em Strong Ignorability} in Rubin's model \cite{rubin2005causal}.
The unconfoundedness assumption requires that the treatment $T$ and the outcome $O$ be independent when conditioned on a set of variables $Z$. In SO, assuming that only $Z$ =\{\verb|Gender|, \verb|Age|, \verb|Country|\} affects $T = $ \verb|Education|, if we condition on a fixed set of values of $Z$, i.e., consider people of a given gender, from a given country, and at a given age, then $T = $ \verb|Education| and $O = $ \verb|Salary| are independent. For such confounding factors $Z$,  Eq. (\ref{eq:cate}) reduces to the following form 
(positivity 
gives the feasibility of the expectation difference): 
 \vspace{-1mm}
{\small
\begin{flalign}    
% \begin{eqnarray}
   % % & ATE(T,O) = \mathbb{E}_Z \left[\mathbb{E}[O \mid T=1, Z = z] -  
   %  \mathbb{E}[O \mid T=0, Z = z] \right] \label{eq:conf-ate}\\
 & CATE(T,O {\mid} B {=} b) {=} \nonumber
    \mathbb{E}_Z \left[\mathbb{E}[O {\mid} T{=}1, B {=} b, Z {=} z] {-}  
    \mathbb{E}[O {\mid} T{=}0, B {=} b, Z {=} z]\right]\label{eq:conf-cate}
\end{flalign}
% \end{eqnarray}
}
% \vspace{-4mm}
This equation contains conditional probabilities and not $\doop(T = b)$, which can be estimated from an observed data. 
Pearl's model gives a systematic way to find such a $Z$ when a causal DAG is available. 




% \input{sections/03-empirical_study}
\section{Methodology}
\label{sec:methodology}

\subsection{Overview}

In this section, we present the design of \alias.
The system architecture is illustrated in Figure~\ref{fig:tracezip_system}, in which we add a \textit{compression module} (\ding{198}) to the service side and a corresponding \textit{decompression module} (\ding{199}) to the entry of the backend trace collectors.
In the compression module, we maintain two data structures, namely, a \name (\sname) and a dictionary, to constantly capture the redundancy across the spans.
Upon the generation of a span at the tracepoint, it undergoes compression utilizing the above data structures.
If the span carries a new redundancy pattern, it will be seamlessly integrated into the \sname and dictionary.
This integration is crucial as it enriches the structures, thereby facilitating the compression for subsequent spans.
We accelerate the above process by employing a combination of mapping and hashing techniques.
At the decompression module, the spans are restored to their original form by referring to the SRT and the dictionary.
To ensure a consistent and reliable data transmission,  it is imperative that these data structures are accurately synchronized between the service and backend sides.
To achieve this, we develop a differential update mechanism (\ding{200}).
This mechanism is designed to precisely pinpoint and propagate only the incremental changes in the data structures, ensuring an efficient synchronization process that minimizes overhead while maximizing data consistency.

\begin{figure}[t]
    \centering
    \includegraphics[width=0.62\linewidth]{figures/tracezip_system.pdf}
    \caption{System Architecture of \alias}
    \label{fig:tracezip_system}
\end{figure}


% \usetikzlibrary{trees}

% In microservice tracing technology, a microservice sends current trace data to the tracing backend. The trace data, known as Spans, are defined as individual units of work within a trace. As we mentioned above, in high-concurrency scenarios, it is likely that many key-value pairs in the spans array sent by a microservice to the tracing backend are repetitive. To better capture the redundancy in the Spans data sent by microservices, we have designed an algorithm named the Merging Algorithm. The Merging Algorithm transforms the Spans array data sent by microservices into a data structure similar to a Prefix Tree, which we will refer to as the Merging Tree in subsequent text. In the Merging Tree, all non-leaf nodes, except the root, are key-value pairs, while the leaf nodes are arrays of key-value pairs. By merging all the key-value pairs along the path from the root to any leaf node, one can reconstruct a span data entry. Figure \ref{fig:Spans_Array_Example} and Figure \ref{fig:Merging_Tree_Example} show an example of spans array and its corresponding Merging Tree. Additionally, we can capture uncommon Spans attributes and report anomalies to the tracing backend using the Spans attribute frequency information generated by the Merging Algorithm. Moreover, by using Dictionary Compression Algorithm, we can assign shorter codes to frequently occurring attribute values based on the frequency information produced by the Merging Algorithm.
% At the same time, when we use the Merging Algorithm and Dictionary Compression Algorithm to compress data, attribute value frequency data is generated. We can utilize this attribute value frequency data to identify potential anomalous link data. 
% The compression algorithm, while compressing the Spans data, will also mark potential anomalous link data and report it back to the Spans collection service backend.


% \begin{table}[]
%     \centering
%     \normalsize
%     \caption{Caption}
%     \begin{tabular}{c|c|c|c}
%     \toprule
%         name&db server&url&start\_time\\
%         \midrule
%         \midrule
%         interface\_1&GET&/interface1&... \\
%         interface\_1&GET&/interface1&... \\
%         interface\_2&PATCH&/interface2&... \\
%         interface\_2&PATCH&/interface2&... \\
%     \bottomrule
%     \end{tabular}
%     \label{tab:my_label}
% \end{table}

% \begin{table}[]
% \captionsetup{justification=centering}
% \centering
% \caption{Ablation study of components in}
% \label{tab: RQ2-ablation}
%  % \resizebox{0.48\textwidth}{!}{%\
% \begin{NiceTabular}{c|c|c|c|c}
% \toprule
% \rowcolor{grey}\textbf{name} & \textbf{operation} & \textbf{db server} & \textbf{results.count} & \textbf{other attrib.}\\
% \midrule
% \midrule
% Access DB & SELECT & MySQL & 10 & ...\\
% Access DB & INSERT & MySQL & 2 & ...\\
% Access DB & UPDATE & MongoDB & 5 & ...\\
% Access DB & DELETE & MongoDB & 1 & ...\\
% \bottomrule
% \end{NiceTabular}
% % }
% \end{table}

% \begin{figure}[H]
%     \centering
%     \begin{tabular}{c c c c}
%         \toprule
%         \multicolumn4c{Spans Array}\\
%         \cmidrule(lr){1-4}
%         Span Name&HTTP Method&URL&Details\\
%         \midrule
%         interface\_1&GET&/interface1&... \\
%         interface\_1&GET&/interface1&... \\
%         interface\_2&PATCH&/interface2&... \\
%         interface\_2&PATCH&/interface2&... \\
%         \bottomrule
%     \end{tabular}
%     \caption{An Example of Spans Array}
%     \label{fig:Spans_Array_Example}
% \end{figure}


% \tikzstyle{every node}=[draw=black,thick,anchor=west]
% \tikzstyle{selected}=[draw=red,fill=red!30]
% \tikzstyle{optional}=[dashed,fill=gray!50]
% \begin{figure*}
% \centering
% \begin{tikzpicture}[%
%   level distance=1.5cm,
%   sibling distance=2.5cm,
%   grow via three points={one child at (0.5,-0.7) and
%   two children at (0.5,-0.7) and (0.5,-1.4)},
%   edge from parent path={(\tikzparentnode.south) |- (\tikzchildnode.west)}]
%   \node {Merging Trees: global\_time\_base}
%     child { node {span\_name: interface\_1}
%         child {
%             node {http.method: GET}
%             child {
%                 node {url: /interface1}
%                 child [missing] {}
%                 child {
%                     node {\makecell{
%                             start\_time: start\_time\_offset \\
%                             end\_time: end\_time\_offset \\
%                             parent\_id: ... \\
%                             context: \{ ... \}
%                         }
%                     }
%                 }
%                 child [missing] {}
%                 child [missing] {}
%                 child {
%                     node {\makecell{
%                             start\_time: start\_time\_offset \\
%                             end\_time: end\_time\_offset \\
%                             parent\_id: ... \\
%                             context: \{ ... \}
%                         }
%                     }
%                 }
%                 child [missing] {}
%                 child [missing] {}
%             }
%         }
%         child [missing] {}
%     }		
%     child [missing] {}				
%     child [missing] {}				
%     child [missing] {}
%     child [missing] {}
%     child [missing] {}
%     child [missing] {}				
%     child [missing] {}	
%     child [missing] {}
%     child { node {span\_name: interface\_2}
%         child { node {http.method: PATCH}
%             child {
%                 node {url: /interface2}
%                 child [missing] {}
%                 child {
%                     node {\makecell{
%                             start\_time: start\_time\_offset \\
%                             end\_time: end\_time\_offset \\
%                             parent\_id: ... \\
%                             context: \{ ... \}
%                         }
%                     }
%                 }
%                 child [missing] {}
%                 child [missing] {}
%                 child {
%                     node {\makecell{
%                             start\_time: start\_time\_offset \\
%                             end\_time: end\_time\_offset \\
%                             parent\_id: ... \\
%                             context: \{ ... \}
%                         }
%                     }
%                 }
%                 child [missing] {}
%                 child [missing] {}
%             }
%         }
%     };
% \end{tikzpicture}
% \caption{Merging Tree Structure of Spans Array} \label{fig:Merging_Tree_Example}
% \end{figure*}


% The Merging Algorithm performs two operations on the microservice side: preprocessing and compressing the Spans Array, resulting in the creation of the Merging Tree. On the trace backend, a decompression algorithm is executed. During the preprocessing phase, the Merging Algorithm identifies which attribute in the Spans Array should be allocated to the non-leaf nodes of the Merging Tree by analyzing the repetition of these attribute values. This is done in linear time complexity. It also determines an order of attributes to minimize the number of non-leaf nodes in the Merging Tree. In the compression phase, the Merging Algorithm inserts each span into the Merging Tree according to the attribute order established during preprocessing. During the decompression phase, the Merging Algorithm reconstructs the Spans Array from the Merging Tree. The Merging Tree can be combined with other general compression algorithms, such as LZ77 and Huffman coding, to achieve higher compression ratios. Notably, the Merging Algorithm has superior time complexity compared to most existing general-purpose data compression algorithms, a topic that will be discussed in a subsequent subsection.

\subsection{Span Format Conventions}

To compress spans by leveraging their recurring patterns, we first stipulate the format of a span.
For simplicity and readibility, we assume that a span adheres to the standard JSON data format that defines it as a structured set of key-value pairs.
The key is a string, while the value can be either primitive types (strings, numbers, booleans, and null) or two structured types (nested key-value pairs and arrays).
This aligns with the format specifications used in many tracing frameworks and tools, e.g., OpenTelemetry~\cite{opentelemetry_traces}, Jaeger~\cite{jaeger}, Zipkin~\cite{zipkin}.
Typical fields (keys) of a span include: \textit{Name} (a human-readable string representing the operation done), \textit{Parent Span ID} (the span that caused the creation of this span, empty for root spans), \textit{Start and End Timestamps} (the start and end time of the span), \textit{Span Context} (the context of the span including the trace ID, the span ID, etc.), \textit{Attributes} (key-value pairs representing additional information about the span), \textit{Span Events} (structured log messages/annotations on a span), etc.
It is important to note that our proposed algorithm is not restricted to JSON or any particular serialization format.
For example, \alias can work effectively when Protobuf (Protocol Buffers)~\cite{protobuf} is used for trace data serialization.
With Protobuf's powerful deserialization capabilities, we can leverage its reflection-like APIs or direct-access methods to dynamically access the fields and values of spans.
Additionally, Protobuf is designed to be backward and forward compatible, allowing us to modify the message definition by adding or removing fields while maintaining compatibility with older data.
Such operations are essential for reducing trace redundancy, e.g., removing span elements that are deemed repetitive.

We also assume that spans possess \textit{structural locality}, meaning that during the continuous execution of a service or component, all spans sharing a common span Name will exhibit an identical structure.
In other words, spans with the same Name will consistently retain the same set of keys (e.g., attributes, tags, and metadata), differing only in the specific values associated with them.
This assumption arises naturally from the way distributed tracing systems operate, where spans typically represent predefined operations or events within the service workflow.
These operations are implemented as part of the service's codebase, which enforces a fixed schema or structure for spans generated by specific instrumentation points.
% For example, a span representing a database query will always include fields such as \texttt{db.statement}, \texttt{db.type}, or \texttt{db.instance}, while their values—such as the specific SQL query or database name—may vary.
This structural consistency allows for reliable trace analysis, optimization, and redundancy reduction, as the predictability of span structures minimizes the need for per-span schema discovery during processing.

% We also assume that spans possess \textit{structural locality}.
% That is, throughout the continuous execution of a service/component, all generated spans sharing a common span Name will have an identical structure.
% This implies they will retain the exact same keys, differing only in the values.
% Nevertheless, if it is violated\zb{what cases}, we xxx
% Such a property allow us to chain the values \zb{not finished}

% \newtheorem{definition}{Definition}

% A Spans Array is defined as follows:
% \begin{definition}\label{def_spans_array}
% A Spans Array is a set of key-value pair sets. Any key-value pair set in the Spans Array is referred to as a Span. Each Span has a key named "Span Name," and for all Spans with the same "Span Name," the set of keys for the remaining key-value pairs is identical. In other words, all Spans with the same Span Name have the same structure, meaning they have exactly the same keys, differing only in the values of those keys.
% \end{definition}


\subsection{Span Retrieval Compression and Uncompression}
\label{sec:compression_uncompression}

% A straightforward approach to compress spans by exploiting their repetitiveness involves the use of a dictionary.
A straightforward approach to compressing spans involves the use of a dictionary.
This method creates a dictionary where every unique key and value is assigned a unique identifier.
During the compression process, the keys and values of each span are substituted by the corresponding identifiers.
The size of the span can then be reduced as the identifiers are much smaller than the original data.
However, as revealed by our empirical study, there can still be redundant information among the identifiers.
The pure dictionary approach compresses data on a one-to-one basis, i.e., one identifier corresponds to one KV pair.
If multiple spans share a collection of common key-value pairs, it is possible to utilize a single identifier to represent this entire set of shared pairs, thereby amplifying the compression efficiency.
Thus, we propose to leverage the correlations among the values of spans to further eliminate repetitive information.

% \begin{algorithm}[t]
% \caption{Performance Anomaly Detection}
% \label{algo:anomaly_detection}
% \normalsize
% \SetAlgoLined
% \KwIn{$t$, $\mathcal{P}_a$, and $\mu_C$}
% \KwOut{Anomaly detection result for $t$}

% $\mathcal{D}_t\gets {\rm PairWiseDistance}(t, \mu_C)$

% $idx\gets {\rm MinIndex}(\mathcal{D}_t)$

% \eIf{$idx\in \mathcal{P}_a$}{
%     return True
% }{
%     return False
% }
% \end{algorithm}

% If there are more KV pairs shared across multiple spans, it is possible to represent this set of KV pairs with one identifier.
% \zb{Examples of events may include uncaught exceptions, button clicks, user logouts, network disconnections, etc. Its structure is also similar to that of Spans.}

\begin{algorithm}
    \caption{Span Retrieval Tree (SRT) Reconstruction and Span Compression}
    \label{algo:srt}
    \begin{algorithmic}[1]
    \State \textbf{Input:} a stream of continuously generated \textit{spans}, a threshold $\psi$
    \State \textbf{Output:} a constructed SRT $\mathcal{T}$, \textit{compressed spans}
    \State Initialize an empty SRT $\mathcal{T}$
    \For{each \textit{span} in \textit{spans}}
        \If{\textit{span Name} not in $\mathcal{T}$}
            \State Chain all key-value pairs of \textit{span} and add the path to the root of $\mathcal{T}$
            % \State Each key of $\mathcal{T}$ has a unique value number of 1
            \State Assign an identifier to this new path
        \Else
        \For{each \textit{key} at every depth of $\mathcal{T}$} \Comment{traverse $\mathcal{T}$ from the root to the leaf}
            \State Get the corresponding \{\textit{key}: \textit{value}\} from \textit{span}
            \If{\{\textit{key}: \textit{value}\} \textbf{does not exist} at the current depth of $\mathcal{T}$}
                \State Chain the remaining key-value pairs of \textit{span} and extend a new branch from the direct parent node of \textit{key}
                % \State Increase the unique value number of the corresponding keys by 1
                \State Calculate the number of unique nodes at each depth of $\mathcal{T}$
                \State Move the keys to the leaf whose unique value number exceeds $\psi$, i.e., \textit{local fields}
                \State Reorder the keys of $\mathcal{T}$ based on the ascending number of their unique values
                \State Reassign path identifiers
                \State \textbf{break}
            \EndIf
        \EndFor
        \EndIf

    \State Compress \textit{span} based on the corresponding path identifier and the values of local fields
    \EndFor
    \end{algorithmic}
\end{algorithm}

\begin{figure}[t]
    \centering
    \includegraphics[width=0.74\linewidth]{figures/span_retrieval_tree.pdf}
    \caption{An Example of \name}
    \label{fig:span_retrieval_tree}
\end{figure}

Our idea is that for spans generated in each service instance, we organize their key-value pairs as a prefix-tree-like data structure, i.e., \sname.
The \sname functions as a multi-way tree, with all non-leaf nodes (except for the root) associated with a key-value pair.
For each type of span (which is identified by a unique span Name), there is only one leaf node connected to all the last non-leaf nodes stemming from it.
This leaf node holds a collection of keys without values.
Figure~\ref{fig:span_retrieval_tree} illustrates an example of SRT, where the gray node and the yellow nodes represent the root and the leaves, respectively, while the remaining are the non-leaf nodes.
Each span can be ``spelled out'' by tracing a path from the root down to the leaf.
The path of \sname represents the set of KV pairs shared across multiple spans.
% The tracing process is done by searching the node that corresponds to the next key-value pair in the span.
The non-leaf nodes contain the fields that are more repetitive, which we refer to as \textit{universal fields}.
Although spans may exhibit commonality, they will still have some unique KV pairs, such as those related to ID and timestamps.
We refer to such pairs as \textit{local fields} and only store their keys at the leaf.
The rationale is that such unique fields are incompressible, i.e., not shared with other spans, so we discard their values.
\textit{Based on \sname, a span can be represented as a unique path identifier plus its exclusive values that are extracted according to the keys in the leaf node.}
Each path identifier collectively represents the KV pairs shared among spans, instead of one identifier for each key and value.
Since these common KV pairs constitute a significant portion, the trace size can be substantially reduced, enhancing the overall efficiency.

% Specifically, the \name is a multi-way tree where the root node is associated with a key-value pair having the key "Span Name."
% Except for the root node, all non-leaf nodes are associated with a key-value pair, and all leaf nodes are associated with a key-value pair set. We define the root node's depth as 1, and the keys of all nodes at depth 2 as "Span Name."
% The Merging Tree satisfies the following property: For a Merging Tree, except for the leaf nodes, all nodes at the same depth have key-value pairs with the same keys. We will refer to this property as the Keep-Order property.

\begin{table}
    \centering
    \caption{Span Examples of a Data-processing Service}
    \label{tab:span_examples}
    \begin{subtable}{1\textwidth}
    \small
        \centering
        \begin{NiceTabular}{c|c|c|c|c|c}
            % \toprule
            \specialrule{0.35mm}{0em}{0em}
            \rowcolor{grey}\textbf{name} & \textbf{operation} & \textbf{address} & \textbf{data\_size} & \textbf{span\_id} & \textbf{others}\\
            % \midrule
            % \midrule
            \specialrule{0.15mm}{0em}{0em}
            \specialrule{0.15mm}{.1em}{0em}
            Access Mem & WRITE & address1 & 64 bytes & id1 & ...\\
            Access Mem & READ & address2 & 128 bytes & id2 & ...\\
            Access Mem & READ & address2 & 64 bytes & id3 & ...\\
            Access Mem & WRITE & address1 & 64 bytes & id4 & ...\\
            Access Mem & READ & address2 & 256 bytes & id5 & ...\\
            % \bottomrule
            \specialrule{0.35mm}{0em}{0em}
        \end{NiceTabular}
        % \vspace{6pt}
        \caption{Span examples of ``Access Mem''}
    \end{subtable}
    \vfill
    \begin{subtable}{1\textwidth}
    \small
        \centering
        \begin{NiceTabular}{c|c|c|c|c|c}
            % \toprule
            \specialrule{0.35mm}{0em}{0em}
            \rowcolor{grey}\textbf{name} & \textbf{type} & \textbf{DB system} & \textbf{status} & \textbf{row.num} & \textbf{others}\\
            % \midrule
            % \midrule
            \specialrule{0.15mm}{0em}{0em}
            \specialrule{0.15mm}{.1em}{0em}
            Access DB & INSERT & MySQL & SUCCESS & 1 & ...\\
            Access DB & SELECT & MySQL & SUCCESS & 1 & ...\\
            Access DB & DELETE & MySQL & SUCCESS & 1 & ...\\
            % \bottomrule
            \specialrule{0.35mm}{0em}{0em}
        \end{NiceTabular}
        % \vspace{6pt}
        \caption{Span examples of ``Access DB''}
    \end{subtable}
\end{table}


% \begin{table}[h]
%     \centering
%     \caption{Span Examples of a Data-processing Service}
%     \label{tab:span_examples}
%     \centering
%     \begin{NiceTabular}{c|c|c|c|c|c}
%         % \toprule
%         \specialrule{0.35mm}{0em}{0em}
%         \rowcolor{grey}\textbf{name} & \textbf{operation} & \textbf{address} & \textbf{data.size} & \textbf{span.id} & \textbf{...}\\
%         % \midrule
%         % \midrule
%         \specialrule{0.15mm}{0em}{0em}
%         \specialrule{0.15mm}{.1em}{0em}
%         Access Mem & WRITE & address1 & 64 bytes & id1 & ...\\
%         Access Mem & READ & address2 & 128 bytes & id2 & ...\\
%         Access Mem & READ & address2 & 64 bytes & id3 & ...\\
%         Access Mem & WRITE & address1 & 64 bytes & id4 & ...\\
%         Access Mem & READ & address2 & 256 bytes & id5 & ...\\
%         % \bottomrule
%         \specialrule{0.35mm}{0em}{0em}
%     \end{NiceTabular}
%     \vspace{6pt}
%     \caption{Span examples of ``Access Mem''}
% \end{table}

% \begin{table}[h]
%     \centering
%     \caption{Span examples of ``Access DB''}
%     \begin{NiceTabular}{c|c|c|c|c}
%         % \toprule
%         \specialrule{0.35mm}{0em}{0em}
%         \rowcolor{grey}\textbf{name} & \textbf{type} & \textbf{DB system} & \textbf{status} & \textbf{row.num}\\
%         % \midrule
%         % \midrule
%         \specialrule{0.15mm}{0em}{0em}
%         \specialrule{0.15mm}{.1em}{0em}
%         Access DB & INSERT & MySQL & SUCCESS & 1\\
%         Access DB & SELECT & MySQL & SUCCESS & 1\\
%         Access DB & DELETE & MySQL & SUCCESS & 1\\
%         % \bottomrule
%         \specialrule{0.35mm}{0em}{0em}
%     \end{NiceTabular}
%     \vspace{6pt}
% \end{table}

We present our algorithm for \sname construction and span compression (i.e., Algorithm~\ref{algo:srt}) and explain it using span examples in Table~\ref{tab:span_examples}.
Suppose these spans are continuously generated by different tracepoints of a data-accessing service, including memory and database.
Each tracepoint produces a specific type of span with varying attributes.
The algorithm takes the stream of spans as input, and the resulting \sname is shown in Figure~\ref{fig:span_retrieval_tree}.
For each new type of span with a previously unseen span Name, we simply chain all fields of the span (line 6) and add the resulting path to the \sname root.
For example, the first row of Table~\ref{tab:span_examples}-(a) will be structured as \textit{Access Mem}$\hookrightarrow$ \textit{WRITE}$\hookrightarrow$\textit{address1}$\hookrightarrow$\textit{64~bytes}$\hookrightarrow$\textit{id1} (we omit the keys of the nodes and the other attributes), shown as the pink dashed rectangles.
For spans with a known Name, we traverse the \sname from the root to the leaf, and use the key at each depth to retrieve the corresponding key-value pair from the span (line 10).
If a retrieved pair does not exist in the SRT, the remaining key-value pairs are chained to construct a sub-path, which is then added as a new branch to the direct parent node (line 12).
% For the next spans with new values, they will be added to the corresponding branches of the \sname.
For example, the second row adds a new path, \textit{READ}$\hookrightarrow$\textit{address2}$\hookrightarrow$\textit{128~bytes}$\hookrightarrow$\textit{id2}, to node \textit{Access~Mem}.
Each path of \sname will be assigned a unique identifier, as described in Section~\ref{sec:hashing_acceleration}.
% In particular, as spans follow the JSON format, they can have nested JSON object.
% We represent this relationship by prefixing the keys of the child JSON with the parent's key, which facilitates the restoration of the nested structure at the backend.
% To reduce the complexity, the default nesting depth is set as two.
% Deeper nested JSON object (not recommended by OTel) will be converted to string.
% \red{We flatten the span as a list of key-value pairs by extracting all elements in it. use a separation symbol to record the span structure (but the other still cannot be kept?). Also there will be an event \sname. Otel recommends primitives kv pairs?}
% As spans with the same name share an identical structure, their original structure can be easily retrieved\zb{how?}.

% \zb{talk about the root node and time base}\zb{existing log compression can be used to further process values}

In \sname, once a new path emerges, we calculate the number of distinct nodes at the same depth (line 13), which represents the number of different values of a key, e.g., the key \textit{span.id} has five distinct values \textit{id1}$\backsim$\textit{id5}.
We set a threshold $\psi$ for the size of values a key can have.
A key with too many values will be regarded as a local field and moved to the leaf (line 14).
For example, \textit{span.id} will be in the leaf if $\psi=3$.
With the constructed SRT, the fourth row can be compactly represented as the identifier of the first path, i.e., the pink path, coupled with its unique value, i.e., \textit{id4} (line 21).
% Note the leaf will not store the values of local fields.
% This is to avoid the \sname growing to large and consumes excessive memory.
Time-related fields such as span start/end time is also a typical local field.
Since spans generated in a short time period will have close temporal fields, we set a \textit{time\_base} at the root node, which allows the leaves to store only the offset relative to the time base.
This is a common way to compress temporal data.
% \zb{more detail (64-bit unsigned integer), how the time base is updated}
Another special local field is the nested JSON object, such as \textit{\{``attributes'': \{``ip'': ``172.17.0.1'', ``port'': 26040\}\}}.
We represent the nested structure by prefixing the keys of the child JSON object with the parent's key (e.g., \textit{``attributes-ip'': ``172.17.0.1''} and \textit{``attributes-port'': 26040}), which allows the backend to easily restore the original hierarchy.
Technically, spans can extend to any depth as required by the tracing needs.
For the consideration of \sname's size, we set a depth limit, which defaults to two, and convert the values of the overly deep keys into pure string type.
Similar to other string fields, they will be moved to the leaves if exhibiting too much diversity.
% \red{Based on our study, most fields have a depth two.}
% In Figure~\ref{fig:span_retrieval_tree}, the pink and dashed rectangles presents a compete path for the fourth row in Table~\ref{tab:span_examples}-(a).

After compression, the span data that needs to be transmitted to the backend trace collector become significantly smaller in size, i.e., only the path identifier and the values of local fields specified by the leaf.
At the backend side, the uncompression process to restore the original span is straightforward and efficient.
This involves reconstructing the local fields based on the corresponding values received and combining them with the universal fields based on the path identifier.
In this process, the backend side should keep the latest copy of the \sname and the value of \textit{time\_base}.
We introduce an efficient synchronization mechanism later in Section~\ref{sec:differential_sync}.
For \textit{time\_base}, we periodically reset it, e.g., every second, ensuring that the time offset remains consistently small.

% For each span, we traverse the \sname and compare the KV pairs stored in the non-leaf nodes with the corresponding fields of the span.
% If all universal fields of the span \red{match}, the span can be compressed as a path identifier with the values of its local fields.

% In this section, we formally define the Spans Array, Merging Tree, and the execution steps of the Merging Algorithm.

% The Merging Tree is defined as follows:
% \begin{definition}\label{def_Merging_tree}

% \end{definition}

% We define the following operation as restoring a Span:
% \begin{definition}\label{def_restoring_span}
% Select a leaf node of the Merging Tree, and combine all the key-value pairs corresponding to the nodes on the unique path from this leaf node to the root node to form a new key-value pair set.
% \end{definition}

% For a Merging Tree, restoring all Spans means performing the span restoration operation for all leaf nodes of the Merging Tree, resulting in a new set of key-value pair sets. If all the Spans restored from a Merging Tree have the same "Span Name" and are identical to those in a Spans Array, then we say the Merging Tree and the Spans Array are equivalent.

% The goal of the Merging Algorithm is to convert the Spans Array into its equivalent Merging Tree to achieve data compression.
% As implied by the name Merging Algorithm, this algorithm merges identical key-value pairs together using a data structure similar to a Prefix Tree, thereby reducing the frequency of repeated key-value pairs.
% Specifically, Merging Tree reduces the occurrence frequency of

% \begin{equation*}\label{nt_reduce}
% f=\sum_{\mathrm{node} \in U} \mathrm{Son}(\mathrm{node}) -1
% \end{equation*}

% key-value pairs, where $U$ means a set of all non-leaf nodes of an Merging Tree, $\mathrm{Son}(x)$ equals the number of leaf nodes in the subtree formed by node $x$.

% In the above definition, we stipulate that the Merging Tree must satisfy the Keep-Order property. This means that for a Spans Array with a specific Span Name S, we can fix the shape of the Merging Tree by fixing an order of the spans' attribute keys. Specifically, for spans with Span Name equals S, if the attribute key A appears in the $i$th position in the permutation $P$ of their attribute keys, we place nodes corresponding to attribute key A at the $(i + 1)$th level in the subtree rooted at Span Name equals S in the Merging Tree. Therefore, given the permutation $P$, we can generate the corresponding Merging Tree according to this order. It is not difficult to see that the possible shapes of the Merging Tree correspond one-to-one with all possible permutations $P$.

% Here, it is easy to see that for different permutations $P$ corresponding to different Merging Trees, the value of $f$ may vary. This means that the order of attribute keys will ultimately affect the compression efficiency of the Merging Tree. Next, we will present the Merging Algorithm, which is the process of generating the Merging Tree, without further explanation. The impact of permutation $P$ on the value of $f$, as well as the optimization of the value of $f$, will be discussed in detail in the next subsection.

% \begin{algorithm}
% \caption{Preprocessing Stage}
% \begin{algorithmic}[1]
% \State \textbf{Input\:} Spans Array \texttt{spansArray}
% \State \textbf{Initialize:}
% \State \quad \texttt{O} $\leftarrow$ empty 2D dictionary
% \State \quad \texttt{E} $\leftarrow$ empty 3D dictionary
% \For{\texttt{s} \textbf{in} \texttt{spansArray}}
%     \For{\texttt{(k, v)} \textbf{in} \texttt{s}}
%         \If{\texttt{E[s.Name][k][v]} = \textbf{false}}
%             \State \texttt{O[s.Name][k]} $\leftarrow$ \texttt{O[s.Name][k]} $+ 1$
%             \State \texttt{E[s.Name][k][v]} $\leftarrow$ \textbf{true}
%         \EndIf
%     \EndFor
% \EndFor
% \For{\texttt{Name} \textbf{in} \texttt{O}}
%     \State \texttt{keys} $\leftarrow$ \texttt{O[Name].keys()}
%     \State \texttt{sortedKeys} $\leftarrow$ \texttt{sort(keys, key=lambda k: O[Name][k])}
%     \State \texttt{P[Name]} $\leftarrow$ \texttt{sortedKeys}
% \EndFor
% \State \textbf{Output\:} Permutations of keys \texttt{P}
% \end{algorithmic}
% \end{algorithm}

% Algorithm 1 describes the tasks accomplished during the preprocessing stage of the algorithm. In brief, for all spans with the same span name, we count the number of unique values for each attribute key. Then, we sort the attribute keys in ascending order based on the number of unique values, forming a permutation.

% \begin{algorithm}
% \caption{Compression Stage}
% \begin{algorithmic}[1]
% \State \textbf{Initialize:} Create node \texttt{Root}
% \For{\texttt{span} \textbf{in} \texttt{spansArray}}
%     \State \texttt{P\_span} $\leftarrow$ \texttt{P[span.Name]}
%     \State \texttt{now} $\leftarrow$ \texttt{Root\_span\_name}
%     \For{\texttt{i} \textbf{in} \texttt{range(len(P\_span))}}
%         \State \texttt{k} $\leftarrow$ \texttt{P\_span[i]}
%         \If{\texttt{now} has a child associated with \texttt{<k,span[k]>}}
%             \State \texttt{now} $\leftarrow$ \texttt{child}
%         \Else
%             \State Create node \texttt{newNode}
%             \State Associate \texttt{<k,span[k]>} with \texttt{newNode}
%             \State Connect \texttt{newNode} to \texttt{now}
%             \State \texttt{now} $\leftarrow$ \texttt{newNode}
%         \EndIf
%     \EndFor
%     \State Create leaf node \texttt{leafNode}
%     \State Associate rest span info. with \texttt{leafNode}
%     \State Connect \texttt{leafNode} to \texttt{now}
% \EndFor
% \end{algorithmic}
% \end{algorithm}

% Algorithm 2 describes the tasks accomplished during the compression stage of the algorithm. In brief, during the compression stage, we insert the span's attributes into the Merging Tree in the order determined during the preprocessing stage.

% \begin{algorithm}
% \caption{Decompression Stage}
% \begin{algorithmic}[1]
% \Function{D}{now}
%     \If{\texttt{now} is a leaf node}
%         \State \Return \texttt{rest of span key-value pairs}
%     \Else
%         \State \texttt{results} $\leftarrow$ empty list
%         \For{\texttt{child} \textbf{in} \texttt{now.children}}
%             \State \texttt{childResults} $\leftarrow$ \Call{D}{child}
%             \For{\texttt{span} \textbf{in} \texttt{childResults}}
%                 \State \texttt{append span to results}
%             \EndFor
%         \EndFor
%         \For{\texttt{span} \textbf{in} \texttt{results}}
%             \State \texttt{add key-value pairs associated with now to span}
%         \EndFor
%         \State \Return \texttt{results}
%     \EndIf
% \EndFunction
% \State \textbf{Output\:} \Call{D}{Root}
% \end{algorithmic}
% \end{algorithm}

% Algorithm 3 describes the tasks accomplished during the decompression stage of the algorithm. Simply put, we designed a recursive function $D(x)$ that restores the Spans Array from the bottom up.

% \begin{figure*}[htp]
%     \centering
%     \includegraphics[width=18cm]{figures/compress.pdf}
%     \caption{Process of compression}
% \end{figure*}


\subsection{Optimizations for \name}

So far, we have introduced the algorithms for span compression and uncompression.
It can be seen that \alias has a small computational complexity.
This is because for each span, these processes involve only a single path traversal of the \sname from the root to a leaf.
However, the issue of space complexity presents a more significant challenge.
The \sname can potentially grow too large and consume an excessive amount of memory.
Besides setting a hard constraint on the memory, we have also identified some opportunities to optimize its size.

\subsubsection{\name Restructuring}

During the construction of \sname in Figure~\ref{fig:span_retrieval_tree}, we simply follow the left-to-right order of keys in Table~\ref{tab:span_examples} to form the parent-child relations among nodes.
For example, key \textit{address} is the child of \textit{name} and also the parent of \textit{data.size}.
We observe that this may result in a sub-optimal \sname structure.
Specifically, for the \sname in Figure~\ref{fig:sft_restructuring} which is built based on the spans in Table~\ref{tab:span_examples}-(b), we can see that the three paths differ only in the \textit{type} field.
A better structure can be obtained by moving \textit{type} down to the bottom, which avoids the recurrence of the other three fields.
Based on this finding, we propose the following way to restructure the \sname.
In Section~\ref{sec:compression_uncompression}, we have calculated the number of possible values associated with each key once a new path emerges.
If a parent field has more values than its child, we swap their positions in the \sname.
That is, we reorder the keys of \sname based on the ascending number of their unique values (line 15).
% This process continues until all fields find their appropriate places.
After reordering, the identical nodes at the same depth will be merged, e.g., \textit{MySQL}, \textit{SUCCESS}, and \textit{1} in Figure~\ref{fig:sft_restructuring}.
Finally, the path identifiers of the restructured \sname will be adjusted (line 16).

% In the previous section, we mentioned that the efficiency of the Merging Tree compression is determined by the formula:
% \begin{equation*}
% f=\sum_{\mathrm{node} \in U} \mathrm{Son}(\mathrm{node}) -1
% \end{equation*}
% In other words, since the number of leaf nodes in the Merging Tree is fixed (which equals the number of Spans in the Spans Array), the fewer non-leaf nodes the Merging Tree has, the better the compression effect of the Merging Tree. Through our research on the dataset, we discovered the following pattern: attributes with more possible values often imply attributes with fewer possible values. The diagram below illustrates this principle. \texttt{func1} and \texttt{func2} are HTTP handlers for the GET method, and \texttt{func3} and \texttt{func4} are HTTP handlers for the POST method. When the \texttt{method} attribute is placed above the \texttt{handler} attribute, the number of non-leaf nodes is less than when the \texttt{handler} attribute is placed above the \texttt{method} attribute. Based on this principle, we construct the Merging Tree by sorting the attributes in ascending order of the number of possible values.

% \begin{figure}[htp]
%     \centering
%     \includegraphics[width=9cm]{figures/order.pdf}
%     \caption{How order affects compression}
% \end{figure}

\begin{figure}
    \centering
    \includegraphics[width=0.66\linewidth]{figures/sft_restructuring.pdf}
    \caption{Span Retrieval Tree Restructuring}
    \label{fig:sft_restructuring}
\end{figure}

% \begin{figure}[t]
%     \centering
%     \includegraphics[width=0.74\linewidth]{figures/span_retrieval_tree.pdf}
%     \caption{An Example of \name}
%     \label{fig:span_retrieval_tree}
% \end{figure}

\subsubsection{Mapping-based Tree Compression}
\label{sec:mapping-based tree compression}

Although we have restructured the \sname to eliminate redundant nodes, there could still be repeated keys and values in it.
For example, in Figure~\ref{fig:span_retrieval_tree}, key \textit{data.size} appears in all \textit{data.size} nodes, e.g., \textit{\{``data.size'': ``64 bytes''\}} and two of them also share value \textit{64 bytes.}
Thus, to further compress the size of \sname, we employ a dictionary to map keys/values that occur multiple times to shorter identifiers.
We construct the identifiers using the standard alphanumeric set, i.e., [0-9a-zA-Z].
Initially, the hashed output consists of a single character, from '0' to '9,' followed by 'a' through 'z,' and finally 'A' through 'Z.'
Upon exhausting the single character possibilities, the function increases the length of the hash output to two characters, starting from '00,' '01,' and so forth.
Since each universal field has limited distinct values, i.e., smaller than $\psi$, the dictionary will also be small in size.
Note we do not encode the values of local fields (not in the \sname), which may inevitably make the dictionary too big given their diversity.
% \zb{how to do the encoding: should include separation by space, comma, etc.}
% \zb{will the dictionary map the local values?}
Similar to the \sname synchronization process between services and the tracing backend, the dictionary will be sent to the backend every time it undergoes an update.
% \zb{enough details?}
% As discussed above, we use the Merging Algorithm to compress a large amount of repetitive attribute value data in the Spans data. However, even with the Merging Algorithm, due to different choices of the order of attributes, it is still possible for a certain attribute value to repeatedly appear on the compressed Merging Tree. Therefore, we count the number of occurrences of the same attribute value on the Merging Tree and then sort them in descending order of occurrences, assigning shorter codes to more frequently occurring attribute values.
% After Dictionary Compression, the Spans Collectors send the compressed data and the dictionary to the tracing backend.
% When compressing using a character stream format like JSON, the assigned codes are allocated from a character set (e.g., [0-9a-zA-Z]), and the corresponding codes are determined based on the character set and the ranking of frequency. For gRPC, since gRPC uses variable-length encoding for numbers, we can directly encode them as numbers.

Based on our empirical study (Section~\ref{sec:redundancy_study}), there exists structural redundancy among the attributes of a span.
% We propose the idea of \textit{recursive dictionary} to further compress both \red{\sname} and the dictionary.
We address this issue by examining the ingredients of the attributes.
Specifically, when constructing the dictionary, we encode the common sub-fields shared among spans (instead of the entire fields) as identifiers.
These sub-field identifiers are then used to compose the complete attributes.
% The idea is that instead of encoding the entire field as a single identifier, we represent the common sub-fields shared among spans as identifiers and use them to compose the complete attributes.
% if an attribute is (partially) composed by a combination of other fields, then the sub-fields will be replaced by the code of the fields.
% When restoring the attribute, such .
Take a span example from OpenTelemetry~\cite{opentelemetry_traces}, which contains the following fields:

\begin{figure}[h]
    \centering
    \includegraphics[width=0.6\linewidth]{figures/structural_correlation.pdf}
    \label{fig:structural_correlation}
\end{figure}

% \begin{lstlisting}
% {
%     "net.transport": "IP.TCP",
%     "net.peer.ip": "172.17.0.1",
%     "net.peer.port": "51820",
%     "net.host.ip": "10.177.2.152",
%     "net.host.port": "26040"
%     ...
% }
% \end{lstlisting}

\noindent We can see that in the keys, there are some words that appear multiple times, e.g., \texttt{net}, \texttt{host}, \texttt{port}.
Without considering such correlations, we could potentially introduce too much lengthy keys to the dictionary.
To remove such redundancy, we first separate each key into a list of tokens based on delimiters dot (``.'') and underline (``\_''), which are configurable.
When encoding the key, each of its tokens will be mapped to the corresponding identifier.
As the value part exhibits more diversity, we only apply this technique to the keys to avoid too much computational overhead.

% \subsection{Edge-case Span Detection}
% \zb{need to talk more about why we need to do this}
% Based on the idea that failure symptoms are locally observable~\cite{DBLP:conf/nsdi/ZhangXAVM23}, we incorporate a detection mechanism for edge-case spans in the compression process.\zb{why need it at the collection side (efficiency? we have already gather related information), cant we just do it at the backend?}
% Specifically, when performing \sname construction and dictionary-based tree compression, we collect information about the occurrence numbers of different values in the spans.
% This allows us to pinpoint the values that rarely occur, which often signify anomalous events or outliers.
% In particular, certain fields such as \textit{span ID} and \textit{start/end time} are intended to hold unique values.
% Thus, we allow users to configure the set of fields whose rare occurrences should be closely monitored\zb{examples?}.
% Upon the detection of such rare values, we send the corresponding \textit{trace ID}\zb{then we dont know which span is the anomaly} to the tracing backend.
% The backend can immediately trigger an alert and notify system operators to investigate potential issues or anomalies.
% \zb{One advantage is interpretability, how can we achieve it? add a flag along with the trace ID?}

% % When generating the Merging Tree and performing Dictionary Compression, we collect information about the frequency of attribute values in the Spans. Based on this, we can calculate the frequency of occurrence for a particular category of attribute values. By considering the frequency and the number of unique attribute values for a specific attribute, we can determine if an attribute value is rare or uncommon. Generally, a rare attribute value signifies an anomalous behavior pattern. We believe that while performing the compression algorithm at the Spans Collector, it is possible to identify potential exceptional Spans and send the Trace ID of those potential exceptional Spans to the tracing backend.

% We use the following two criteria to pinpoint rare spans\zb{which are commonly used?} based on the type of attributes under monitoring:

% \begin{itemize}
%     \item \textit{Categorical attributes}: We employ a frequency-based approach to manage categorical attributes such as HTTP status codes and IP addresses.
%     Each node preserving one possible value of the target attribute will maintain a frequency ratio, meaning the proportion of spans carrying that specific value relative to the total span count.
%     Upon the arrival of each span, if the frequency of its carried value is below a pre-defined threshold (the default setting is 0.01, which is configurable), that span will be regarded as a rare span.
%     Subsequently, the frequency ratio across all nodes will be updated.
%     This approach helps in pinpointing spans that may indicate \red{unusual system events}.

%     \item \textit{Numerical attributes}: For numerical attributes such as duration, we apply a statistical outlier detection method, i.e., checking whether the value falls within the three-sigma range.
%     To this end, each node will maintain the current average ($\mu$) and standard deviation $\sigma$ of the attribute values, along with a cumulative count.
%     These parameters are updated dynamically as new spans are processed.
%     If a particular value significantly deviates from the norm, it will be flagged as a rare span.
%     This allows us to spot any abnormal variations in the data.
% \end{itemize}

% If the ratio of the total number of spans with the same \texttt{span\_name} to the total number of sampled spans is lower than \texttt{abnormal\_span\_name\_rate}, the span is considered rare.
% When iterating through the attributes of a specific span:
% \begin{itemize}
%   \item If an attribute is present in the Merging Tree (indicating a low ratio of spans to possible attribute values, i.e., lower than \texttt{threshold\_rate}), and the ratio of the total number of spans with the same \texttt{span\_name} to the frequency of occurrence of that attribute is lower than \texttt{abnormal\_attributes\_frequency\_rate}, the span is considered rare.
%   \item If an attribute is not present in the Merging Tree (indicating a high ratio of spans to possible attribute values, i.e., higher than \texttt{threshold\_rate}), and the attribute's data type is integer or floating-point, the attribute is treated as continuous. The average $\mu$ and standard deviation $\sigma$ of this attribute are calculated. If the attribute value is not within the interval $[\mu - 3\sigma, \mu + 3\sigma]$, the span is considered rare.
% \end{itemize}
\section{Implementation}
\label{sec:implementation}

We have implemented \alias inside the OpenTelemetry Collector with around 3K lines of Golang code.
The OpenTelemetry Collector offers a vendor-agnostic implementation of how to manage telemetry data, which mainly includes four types of components: \textit{exporters}, \textit{processors}, \textit{receivers}, and \textit{extensions}.
We implement the span retrieval compression and decompression on the exporter and receiver, which run at the service side and backend side, respectively.
The exporter is responsible for building and updating the \sname and dictionary, compressing spans on the fly, and sending them to the remote backend.
After accepting the compressed data, the receiver performs span uncompression.
We outline some important details concerning the implementation.

\subsection{Search Acceleration by Hashing}
\label{sec:hashing_acceleration}

A straightforward data structure to implement \sname would be linked representation, which enjoys the benefits of dynamic size and efficient alterations (e.g., insertion and deletion).
However, in linked representation, the tree nodes are not stored contiguously or nearby in memory, potentially leading to more cache misses.
This factor can significantly impede the speed of path search within \sname.
To accelerate the search process, we apply hashing to convert each unique path of SRT to a path identifier, which is similar to that in Section~\ref{sec:mapping-based tree compression}.
Specifically, for each path, starting from the root we join the values of non-leaf nodes sequentially with a comma separator (similar to the CSV format). 
Based on the composed path string, we maintain a \{\textit{path}: \textit{identifier}\} mapping at the exporter.
% At the receiver, we maintain a consistent mapping at the opposite direction, i.e., \{\textit{identifier}: \textit{path}\}.
% These bi-directional mappings are suitable for compression and decompression at different sides.
When a new span is generated at the exporter, we extract the values of its universal fields based on the order in \sname.
The path search can then be quickly done for the span by checking if its path string exists in the map.
We use the \texttt{map} data type in Golang, which provides a highly efficient way to achieve this.
For any updates to the \sname, we only need to renew the affected paths as discussed in the next subsection.
% We then maintain a bi-directional mapping at the exporter, i.e., \{\textit{identifier}: \textit{path}\} and its reversed form, which will be synced with the receiver.

\subsection{Differential Data Synchronization}
\label{sec:differential_sync}

To ensure reliable span compression and uncompression, the exporter and receiver must maintain consistent copies of both the \sname and dictionary structures.
One simple strategy is for the exporter to send the latest versions of these structures upon any update.
However, given that updates often affect only a small segment of the overall structures, sending redundant (i.e., unchanged) data with each update would incur network overhead and potentially defer the uncompression process.
Thus, we implement a differential update mechanism for more resource-efficient synchronization.
The core idea is that at the receiver, instead of maintaining another \sname, we keep a path hashing in the opposite direction, i.e., \{\textit{identifier}: \textit{path}\}.
For any updates to the non-leaf nodes, we can easily pinpoint the affected paths and perform the renewal.
For example, in Figure~\ref{fig:tracezip_system}, the emergence of a new value (denoted by the pink dashed rectangle) gives rise to a novel path, i.e., \textit{path 3}.
In this case, we can add a new entry to the \{\textit{path}: \textit{identifier}\} mapping at the exporter and sync it with the receiver.
For path deletion, the exporter can simply send the corresponding identifier to the receiver for record elimination.
Other updates are essentially a combination of path addition and deletion.
% we hash each path of \sname (excluding the leaf node) into a path identifier (Section~\ref{sec:hashing_acceleration}), which is then synced with the receiver.
% For any updates to the non-leaf nodes, we can easily pinpoint the affected paths and (re)calculate the path identifiers for renewal.
% For example, in Figure~\ref{fig:tracezip_system}, the emergence of a new value (denoted by the pink dashed rectangle) gives rise to a novel path, i.e., \textit{path 3}.
% \red{In this case, only the hashed representation of this new path needs to be synced.}
% Such a design also benefits the search of spans in \sname, which will be elaborated in the next subsection.

For local fields and the mapping dictionary, it suffices to communicate only the changes to the receiver.
To ensure that the structures at the receiver is not outdated during the transmission of spans, we leverage the batch processor of OpenTelemetry Collector.
It caches the spans sent by SDK until the batch memory is full or its timer expires, instead of immediately forwarding them.
After compressing the spans in the buffer, we will make sure that the SRT and dictionary with updates (if any) have been synced with the receiver side before releasing the data.


% To verify the effectiveness of our proposed Merging Algorithm, we designed two prototype systems, which we refer to as the Static Spans Compressor($\approx 0.5k\mathrm{LOC}$) and the OpenTelemetry Collector instrumented with the Merging Tree Algorithm (hereafter referred to as OTelCol with Compression, $\approx 2k\mathrm{LOC}$, OTel Collector scaffolding codes are excluded). The Static Spans Compressor is a command-line tool used for compressing span records stored in CSV file format. OTelCol with Compression is a middleware based on the secondary development of the OpenTelemetry Collector. The OpenTelemetry Collector supports a plugin system, allowing developers to write their own plugins and build their own OTelCol, and control the data flow of OTelCol by writing configuration files. We have written Receiver and Exporter plugins for OTelCol. The compression algorithm constructs and sends the Merging Tree on the Exporter side, and decompresses the Merging Tree on the Receiver side, while also performing anomaly detection.

% \subsection{OTelCol with Compression}

% In OpenTelemetry, components are divided into three categories: Receiver, Processor and Exporter. The Receiver is the component in OpenTelemetry that ingests telemetry data (such as traces, metrics, and logs) from various sources. It acts as an entry point for data into the OpenTelemetry Collector. The Processor is the component responsible for processing and transforming telemetry data within the OpenTelemetry Collector. It operates between the Receiver and Exporter. The Exporter is the component that sends the processed telemetry data to various backends and storage systems for analysis and visualization. Therefore, we chose to develop the Receiver and Exporter components of the Merging Algorithm based on the OpenTelemetry Collector, use Batch component as processor, build a custom OTelCol that can use the components based on the Merging Algorithm, which we refer to as OTelCol with Compressor. Due to the vendor-agnostic nature of the OpenTelemetry Collector, our developed OTelCol with Compressor can easily integrate with various tracing platforms, such as Zipkin and Jaeger.

% \textbf{Gateway Collector Deploy Pattern} The OpenTelemetry documentation provides a deployment pattern known as the gateway collector, which starts multiple OTelCol instances and utilizes NGINX for load balancing. Considering a production environment with high access pressure on the tracing backend, we use our custom OTelCol with Compressor for each microservice, employing our Merging Algorithm Compressor as the Exporter, sending data to a remote OTelCol with Compressor for decoding operations (referred to as OTelCol with Compressor Gateway in the text). Using the gateway collector deployment pattern allows us to implement load balancing strategies, achieve decentralized trace collection, and further decouple the data collection platform.

% \begin{figure}[htp]
%     \centering
%     \includegraphics[width=8cm]{figures/arch.pdf}
%     \caption{Architecture of OtelCol with Compressor System}
%   \end{figure}

% \textbf{Batch Processor as Buffer} In OTelCol, there is an important component called the Batch Processor. The OpenTelemetry documentation also recommends developers use the Batch Processor. Essentially, the Batch Processor acts as a Spans Buffer. When the SDK collects spans from microservices, the Batch Processor caches the spans sent by the SDK until the batch in the Batch Processor is full or the Batch Processor's timer expires, instead of immediately forwarding them to the next node. We can configure the size of the Batch Processor and the duration of the timer in the configuration file. Since the Merging Algorithm relies on the high redundancy of spans data and the correlation between spans data, it is important to set a reasonable batch size and timeout value.

% \begin{table}[h!]
% \centering
% \begin{tabular}{|p{2cm}|p{6cm}|}
% \hline
% \textbf{Configuration Item} & \textbf{Description} \\
% \hline
% \texttt{sample buffer} & Sets the size of the sample buffer. The sample buffer is used to store and count the occurrences of various attribute values for anomaly detection. It differs from the batch size of the Batch Processor. \\
% \hline
% \texttt{abnormal span name rate} & Abnormal Span Name Rate. This parameter determines which span names with low frequency of occurrence will be considered abnormal. \\
% \hline
% \texttt{abnormal attributes frequency rate} & Abnormal Attributes Frequency Rate. This parameter determines which attribute values with low frequency of occurrence will be considered abnormal. \\
% \hline
% \texttt{enable number abnormal} & Enables numeric anomaly detection when set to true. This parameter controls whether numeric anomaly detection is enabled, where numeric attribute values are checked for anomalies based on their average value and variance. \\
% \hline
% \texttt{threshold rate} & Threshold Rate. This parameter is used to calculate the baseline threshold for inclusion in the Trie. Attribute values with an occurrence count exceeding \texttt{spanNameCount[spanName]/ThresholdRate} will not be included in the Trie compression sequence. \\
% \hline
% \texttt{enable gzip} & Whether to enable gzip compression. If enabled, spans data will be gzip-compressed when sent. \\
% \hline
% \end{tabular}
% \caption{Configuration parameters for exporter}
% \label{exporter}
% \end{table}

% \textbf{Mechanism of Exporter with Compressor} The Exporter with Compressor utilizes the Batch component to process tracing data sent by clients in batches. During the compression process, the attribute values are mapped to a dictionary, and the Attribute Name is synchronized between the Agent Side and the Gateway Side to restore the original attribute names. Additionally, during the compression process, it detects abnormal spans (as described in the Methodology section) and synchronizes the trace\_id of the abnormal spans to the Gateway. Here in Table \ref{exporter}, we present the configuration options of the Exporter and their meanings

% \textbf{Mechanism of Receiver with De-Compressor} The Receiver with De-Compressor will be deployed on the Gateway side. It receives synchronized dictionary data from the Agent Side, receives abnormal detection data in the form of an array of trace IDs from the Agent Side, and receives compressed spans arrays from the Agent Side. It then restores the compressed spans to their original uncompressed form.

% \zb{details of tree/dict synchronization: send path as dict (with an ID) to backend and only update the changed dicts; the base time is periodically updated}

\section{Experimental Evaluation}
\label{sec:exp_eval}

In this section, we present the evaluation of \alias.
We first introduce the experimental settings, including the deployed cloud services, the metric for evaluation, and the baseline methods.
Next, we demonstrate the experimental results, which include the effectiveness of trace compression and the analysis of both efficiency and overhead.

% \zb{Note in online scenarios, there exists a trade-off between the compression ratio and processing overhead at service instances}

\subsection{Experimental Setup}

\subsubsection{Deployed Cloud Systems}
\label{sec:deployed_cloud_systems}

To evaluate the compression performance in a realistic environment, we deploy popular cloud systems and collect their traces using the OpenTelemetry Collector instrumented with \alias.
We serialize the trace data into JSON format and transmit them utilizing HTTP protocols.
The selected services include one microservices benchmark named Train Ticket~\cite{DBLP:journals/tse/ZhouPXSJLD21} and six open-source application components, including gRPC, Apache Kafka, Servlet, MySQL, Redis, MongoDB.

Train Ticket is a railway ticketing application comprising 41 microservices, each responsible for a specific function, such as user authentication, ticket booking, payment processing, and notification.
This benchmark is implemented in different programming languages such as Java, Go, Node.js, Python, etc.
Train Ticket allows a comprehensive evaluation in a multifunctional scenario, which has been widely used in many trace-related topics, including trace sampling~\cite{DBLP:conf/IEEEcloud/ChenJSLZ24}, root cause localization~\cite{DBLP:journals/tse/ZhouPXSJLD21,DBLP:conf/issre/ZhouZPYLLZZD23}, service architecture measurement~\cite{DBLP:conf/sigsoft/0001ZZIGC22}, etc.
In order to replicate a live production environment, we employ Locust~\cite{locust}, an open-loop asynchronous workload generator, to drive the services.
The workloads are directly borrowed from the original work~\cite{DBLP:journals/tse/ZhouPXSJLD21} that introduces the Train Ticket microservices. 

The selected six application components have widespread adoption and play critical roles in modern cloud service architectures.
They represent a diverse cross-section of the technology stack, which play a foundational role in constructing robust, scalable, and high-performance distributed systems.
% from communication protocols like gRPC and web service frameworks such as Apache HTTP and Servlet, to messaging systems like Kafka, and data storage solutions including MySQL, Redis, and MongoDB.
% They are foundational in constructing robust, scalable, and high-performance distributed systems.
We generate workloads that reflect real-world usage patterns common in cloud-native and microservices environments.
For communication protocols like gRPC and web service frameworks such as Apache HTTP, we simulate typical traffic and user interactions.
In messaging systems like Kafka, workloads involve data streaming and message processing, while for data storage solutions like MySQL, Redis, and MongoDB, we focus on common database operations such as read/write transactions.
This approach ensures our findings are applicable and relevant to a wide range of real-world scenarios.
% By focusing on these components, our study captures a broad spectrum of interactions and operations that are common in today's cloud-native and microservices-oriented environments.
% This ensures that our findings are applicable and relevant to a vast array of real-world scenarios.
% \zb{How we generate the workload?}

% To comprehensively evaluate the application components, we generated workloads that closely mimic real-world usage patterns and stress typical operational scenarios in cloud-native and microservices-oriented environments. These workloads were designed to simulate a variety of interactions and operations across different layers of the technology stack.

% For communication protocols like gRPC, we created scenarios involving high-frequency request/response cycles to test latency and throughput. In the case of web service frameworks such as Apache HTTP and Servlet, we generated workloads that emulate typical web traffic, including concurrent user requests, varied content delivery, and session management.

% For messaging systems like Kafka, we crafted workloads that simulate real-time data streaming and message queuing, focusing on both high-volume data ingestion and complex processing pipelines. In terms of data storage solutions, including MySQL, Redis, and MongoDB, the workloads were designed to reflect common database operations, such as read/write transactions, complex queries, and data caching strategies.

% By employing such targeted and realistic workload generation, our study captures the dynamic and multifaceted nature of operations within these critical application components. This approach ensures that our evaluation reflects practical performance and interaction patterns, providing insights that are both applicable and valuable to developers and practitioners working in diverse cloud service architectures.


\subsubsection{Evaluation Metric}

To measure the effectiveness of \alias, we employ \textit{Compression Ratio} (CR) as the metric, which is widely used in the evaluation of existing compression methods for telemetry data~\cite{DBLP:conf/kbse/LiuZHHZL19,DBLP:conf/icse/LiZL024}.
The definition is given below:

\begin{equation*}
    CR=\frac{\mathrm{Original~File~Size}}{\mathrm{Compressed~File~Size}}
\end{equation*}

In each experiment, we run the same set of workloads, so the size of the original file remains constant.
With different compression approaches and configurations, the resulting compressed file may vary in size.
As the file size decreases, a higher level of compression is attained, indicating more effective compression performance.

\subsubsection{Baseline Methods}

Since we are the first to study the problem of trace compression in a live production scenario, there has not been any baseline methods/systems for comparison.
Note that \alias is orthogonal to existing trace sampling techniques, which compress traces via reducing the volume of data collected.
Thus, they cannot be directly compared to \alias.
In this case, we opt for general-purpose compression algorithms which can be used as out-of-the-box tools to compress traces.
Three prevalent and effective algorithms are selected, that is, gzip, bzip2, and lzma.
However, as they are not tailored for trace data, suggesting potential for further performance improvement.
Our goal is to illustrate the additional compression benefits that \alias can provide when applied in conjunction with these standard algorithms.

\begin{table*}[t]
    \centering
    \caption{Performance of Trace Compression on Open-source Cloud Systems}
    \label{tab:dynamic_compression_result}
    \centering
    \footnotesize
    \begin{NiceTabular}{C{1.89cm}|C{0.45cm} C{0.45cm}|C{0.45cm} C{0.45cm}|C{0.45cm} C{0.45cm}|C{0.45cm} C{0.45cm}|C{0.45cm} C{0.45cm}|C{0.45cm} C{0.45cm}|C{0.45cm} C{0.45cm}}
        % \toprule
        \specialrule{0.35mm}{0em}{0em}
        \multirow{2}{*}{} & \multicolumn{2}{c}{\textbf{Train Ticket}} & \multicolumn{2}{c}{\textbf{gRPC}} & \multicolumn{2}{c}{\textbf{Kafka}} & \multicolumn{2}{c}{\textbf{Servlet}} & \multicolumn{2}{c}{\textbf{MySQL}} & \multicolumn{2}{c}{\textbf{Redis}} & \multicolumn{2}{c}{\textbf{MongoDB}}\\
        % \cmidrule{2-15}
        % \midrule
        % \hdashline
        % \cdashline{2-8}
        \cline{2-15}
        & \textbf{Size} & \textbf{CR} & \textbf{Size} & \textbf{CR} & \textbf{Size} & \textbf{CR} & \textbf{Size} & \textbf{CR} & \textbf{Size} & \textbf{CR} & \textbf{Size} & \textbf{CR} & \textbf{Size} & \textbf{CR}\\
        % \midrule
        % \midrule
        \specialrule{0.15mm}{0em}{0em}
        \specialrule{0.15mm}{.1em}{0em}
        % \hdashline[2pt/2pt]
        % \hline
        % \hline
        Raw & 21.0 & 1 & 3.08 & 1 & 2.47 & 1 & 9.36 & 1 & 1.88 & 1 & 2.01  & 1 & 1.10 & 1 \\
        \rowcolor{grey} \alias & 5.19 & 4.05 & 0.58 & 5.31 & 0.627 & 3.94 & 1.45 & 6.46 & 0.30 & 6.27 & 0.33 & 6.09 & 0.21 & 5.24 \\
        % \midrule
        % \hline
        \hdashline[2pt/1pt]
        gzip & 1.93 & 10.90 & 0.163 & 18.91 & 0.143 & 17.27 & 0.506 & 18.50 & 0.112 & 16.85 & 0.084 & 23.93 & 0.065 & 16.92 \\
        \rowcolor{lightgrey} \alias (gzip) & \textbf{1.29} & 16.26 & 0.140 & 22.81 & 0.133 & 18.57 & 0.396 & 23.64 & 0.091 & 20.61 & 0.066 & 30.45 & 0.051 & 21.57 \\
        \rowcolor{grey} improvement & 33.0\% & 1.49x & 17.0\% & 1.17x & 7.0\% & 1.08x & 21.7\% & 1.28x & 18.2\% & 1.22x & 21.4\% & 1.27x & 21.5\% & 1.27x \\
        \hdashline[2pt/1pt]
        bzip2 & 2.41 & 8.71 & 0.135 & 21.96 & 0.124 & 19.92 & 0.421 & 22.23 & 0.097 & 19.38 & 0.056 & 35.89 & 0.054 & 20.37 \\
        \rowcolor{lightgrey} \alias (bzip2) & 1.34 & 15.62 & \textbf{0.128} & 24.00 & 0.116 & 21.29 & 0.365 & 25.64 & 0.087 & 21.56 & 0.044 & 45.68 & 0.048 & 22.92 \\
        \rowcolor{grey} improvement & 30.3\% & 1.75x & 8.5\% & 1.10x & 6.5\% & 1.07x & 13.3\% & 1.15x  & 10.3\% & 1.12x & 21.4\% & 1.27x & 11.1\% & 1.13x \\
        \hdashline[2pt/1pt]
        lzma & 1.93 & 10.89 & 0.174 & 17.67 & 0.128 & 19.29 & 0.487 & 19.22 & 0.121 & 15.61 & 0.064 & 31.41 & 0.065 & 16.92 \\
        \rowcolor{lightgrey} \alias (lzma) & 1.55& 13.54  & 0.146 & 21.14 & 0.012 & 20.58 & 0.412 & 22.72 & 0.097 & 19.48 & 0.054 & 37.2 & 0.055 & 20 \\
        \rowcolor{grey} improvement & 35.7\% & 1.24x & 16.4\% & 1.20x & 9.6\% & 1.07x & 20.1\% & 1.18x & 19.9\% & 1.2x & 17.4\% & 1.19x & 22.7\% & 1.18x \\
        % \midrule
        % \bottomrule
        \specialrule{0.35mm}{0em}{0em}
    \end{NiceTabular}
\end{table*}

\subsection{Effectiveness of Trace Compression}
\label{sec:compression_effectiveness}

\subsubsection{Open-source Cloud Systems}

Table~\ref{tab:dynamic_compression_result} presents the compression performance when collecting traces of the microservices benchmark and cloud applications components.
For each system, we calculate the total size of traces collected, the size after compression, and the resultant compression ratios (CRs) when applying different compression algorithms.
We can see that \alias, as a standalone solution, can achieve CRs ranging from 3.94 to 6.46.
This demonstrates that \alias can remove more amount of redundant information than that shown by our preliminary study in Section~\ref{sec:redundancy_study}.
Traditional compression tools, i.e., gzip, bzip2, and lzma, reduce the file size with a combination of different techniques such as dictionary-based compression and Huffman coding.
Among them, bzip2 generally outperforms the others across most systems, with gzip having the least effectiveness.
In the Train Ticket benchmark, the tools demonstrate the least effective compression with a CR of roughly 10, while on the cloud application components, they deliver a better performance, attaining a comparable CR of around 20.

When working in conjunction with the general-purpose compression algorithms, \alias can provide additional performance gain.
In general, the improvement achieved by \alias when combined with bzip2 is less pronounced than when paired with other algorithms.
This can be attributed to its already superior compression capability, which may reduce the incremental benefits that \alias can offer.
In the case of the microservices benchmark, namely Train Ticket, \alias achieves a more significant performance improvement of 30\%$\sim$35\%.
However, the improvement is less substantial in cloud application components, with Apache Kafka demonstrating an enhancement of less than 10\%.
As mentioned in Section~\ref{sec:redundancy_study}, the traces generated by Kafka include the data from its message queues, rendering the attributes more random.

So far we can make an important observation: compared to application backend components, general-purpose compression algorithms are less effective for processing the traces from Train Ticket, where \alias can offer more substantial improvement.
Our careful investigation reveals the following important insight.
% The spans generally encapsulate two types of data: connectivity details along the call chain and business-specific data.
Based on zero-code instrumentation, the spans collected encapsulate many attributes related to network connectivity (as specified by OpenTelemetry semantic conventions), such as the hostname, IP address, and port of the peer server.
For instance, MongoDB captures details of the requests; Kafka producers log information about their consumers.
Such information provides a comprehensive view of the request's journey across the distributed system.
In production systems, the invocations among different services and components constitute a complex graph, with each node potentially connected to dozens or more instances.
Our experimental environment may not be able to accurately replicate the conditions of the production scenarios.
Consequently, the connectivity information tends to be relatively static, especially for application backend components that operate at infrastructure and platform layer.
In this case, both \alias and traditional algorithms can properly compress such information, reducing the performance gain that \alias can offer.

On the other hand, Train Ticket comprises tens of microservices, which can form a invocation graph with moderate complexity.
Additional, as a service-oriented application, the traces from Train Ticket contain more information related to business logic.
% These two factors contribute to the increased diversity observed in the traces generated by Train Ticket.
These two factors render the traces produced in Train Ticket more diverse, and the compressible information is more scattered.
Traditional compression algorithms are limited to exploiting redundant information within a short sliding window (e.g., 32KB in gzip's Deflate algorithm).
On the other hand, \alias utilizes \sname to continuously capture the redundancy patterns across spans in a \textit{global} manner, which can further reduce the redundancy. 

% We've discovered that the structural correlations present in this architecture are not optimally leveraged by algorithms like Gzip, whereas our compression algorithm, \alias, can identify and compress these relational attributes within the invocation graph. In monolithic applications, the downstream connectivity information tends to be more static. Our experimental environment may not accurately replicate the conditions of the production environment, or it may include random data that are incompressible.

% When the downstream link data are completely identical, Gzip can effectively use its dictionary to identify these repetitions and compress them at a minimal cost. However, when the downstream data are entirely random, neither Gzip nor \alias can achieve effective compression. In such cases, our compression algorithm loses its effectiveness regarding the connectivity part of the information, compressing only the business-related span data, which leads to a reduction in the overall compression ratio.

% This observation underscores the tailored nature of \alias for handling the specific characteristics of trace data from complex, interconnected microservices, providing a more significant advantage over traditional compression tools that may not adapt as effectively to the intricate patterns found in modern distributed systems.


%spans 信息包含两部分,一部分信息是有关于链路下一条的连接信息,另一部分信息是与具体业务相关的信息。semantic convention 规定了很多关于链路下一跳的信息,例如下一条的服务器的 hostname、ip 地址、下一条服务器接受请求的 port 端口等。在实际的生产系统和测试的微服务系统中,所有的实例之间构成一个调用关系图,每一个微服务会与数十个或者更多的实例相连。我们发现这种结构性的关联 Gzip 算法难以很好的利用,而我们的压缩算法可以提取出调用关系图中属性值的关联并加以压缩。在单体应用服务中,链路的下一条信息 **往往是固定的** ,或者是 **无法被压缩的随机信息 **。当链路的下一条数据完全相同时,gzip 能够通过字典发现这种重复,并且用很小的代价进行压缩,当下一跳的链路数据信息完全随机时,gzip 和 \alias都不能够对其进行好的压缩。所以我们的压缩算法失去了关于链路下一条这一部分信息的压缩效果,只能够压缩和业务相关的链路数据,导致了压缩率的减损%

\subsubsection{Production Trace Data}

We also evaluate \alias using production trace data collected from Alibaba. Compared to existing microservices benchmarks, this dataset represents the call graphs of a large-scale deployment of over 20,000 microservices in production clusters. The participating microservices can be categorized into two types: stateless services and stateful services. Stateless services operate independently of any stored state data, whereas stateful services, including databases and systems like Memcached, are required to maintain state information.
There are three types of communication paradigms between pairs of microservices: inter-process communication, remote invocation, and indirect communication. In addition to this diversity, the trace data also exhibit statistical characteristics typical of industry scale. For example, the size of microservice call graphs follows a heavy-tail distribution; there is a non-negligible fraction of hot-spot microservices; and the microservices can form highly dynamic call dependencies at runtime.
This real-world application allows us to examine \alias's efficacy in handling large-scale, complex data, which is crucial for understanding its potential in practical, production-level scenarios.

Figure~\ref{fig:alibaba_compression} illustrates the evaluation results.
The raw size of the trace data used in our experiments is 26.15GB.
The CRs attained by gzip, bzip2, and lzma are 6.55, 7.30, and 8.52, respectively, reducing the data size to 3.99GB, 3.58GB, and 3.07GB.
These CRs are marginally lower than those recorded in the Train Ticket benchmark.
This observation aligns with our finding in Section~\ref{sec:compression_effectiveness}.
That is, relying predominantly on local information, traditional compression algorithms might find it challenging to efficiently compress data characterized by significant diversity and complexity.
This is where the strength of \alias becomes evident.
\alias, with its ability to identify global compression opportunities, enhances the compression performance by 35.1\%, 43.6\%, and 37.8\%, respectively.
For example, the combination of \alias and lzma achieves the optimal CR of 13.69, and compresses the data to a minimal size of 1.91GB.
This result underscores the benefits introduced by \alias, especially when dealing with large-scale and intricate trace data.

% The Static Spans Compressor, which is implemented by Go, is used to compress files in CSV format. The main purpose of implementing this system is to apply the Merging Tree compression algorithm to the Alibaba Cluster Trace 2022 dataset.

% \begin{table}[h!]
% \centering
% \begin{tabular}{|m{2cm}|m{1cm}|m{2cm}|m{2cm}|}
% \hline
% \textbf{Parameter} & \textbf{Type} & \textbf{Default Value} & \textbf{Description} \\ \hline
% \texttt{-path} & \texttt{string} & \texttt{""} & The path to the file to be compressed. \\ \hline
% \texttt{-chunk} & \texttt{int} & \texttt{0} & The size of chunks for compression. \\ \hline
% \texttt{-dirname} & \texttt{string} & \texttt{"output"} & The name of the output directory. \\ \hline
% \texttt{-j} & \texttt{int} & \texttt{1} & The maximum number of CPU cores to use. \\ \hline
% %\texttt{-huffman} & \texttt{bool} & \texttt{false} & Enable Huffman encoding for compression. \\ \hline
% \texttt{-merging} & \texttt{bool} & \texttt{false} & Enable Merging Algorithm. \\ \hline
% \end{tabular}
% \end{table}

% Here, we present the command-line parameters table for the Static Span Compressor. When using the Static Span Compressor, we do not use the Merging Algorithm by default, but only the dictionary compression algorithm, in order to facilitate the comparison of compression effects. Another important parameter is the chunk size. The Static Span Compressor can split the spans records and compress them in units of chunks. As the chunk size increases, the memory space occupied by the Merging Algorithm also increases, but the compression efficiency of the Merging Algorithm correspondingly improves. We will discuss this in detail in the Experiment Evaluation section.Additionally, we can specify the number of CPU cores to be used for the compression operation. Since the Static Spans Compressor compresses spans files, it does not implement anomaly detection algorithms.

\subsection{Performance Overhead}

We examine the overhead of \alias from the perspectives of space complexity and computational efficiency.

\subsubsection{Space Complexity}

At the service side, \alias maintains two types of data structures to capture the redundancy among trace spans and perform compression, namely, a \sname along with its hashed paths and a map for dictionary-based compression.
To prevent impeding the normal execution of the service, it is imperative that they are constrained in size without excessive memory consumption.
To study the space complexity of \alias, we select a microservice in Alibaba trace data with diverse spans and calculate the cumulative size of the three data structures after the compression procedure.
A critical parameter influencing this size is the threshold $\psi$, which dictates the maximum number of distinct values for universal attributes.
An attribute having an exceeding number of values will be moved to the leaf node, becoming a local attribute.
A larger $\psi$ enables \alias to compress a broader spectrum of span fields, enhancing performance but at the cost of a more substantial SRT and mapping structure.
Conversely, a small $\psi$ compromises the effectiveness but with lower space overhead in return.

\begin{figure}[t]
    \centering
    \includegraphics[width=0.6\linewidth]{figures/experiments/alibaba_compression.pdf}
    \caption{Compression on Alibaba Production Traces}
    \label{fig:alibaba_compression}
\end{figure}

Figure~\ref{fig:space_overhead} illustrates the results, where we can see both the data structure size and compression improvement grow with a larger $\psi$.
In the case of $\psi$=1,000, \sname and map together take up only 2.56MB of memory, but the performance gain that \alias achieves is significant, i.e., 33.8\%.
This result underscores \alias's capacity to achieve substantial compression efficiency while maintaining a balanced memory footprint.
However, we notice that the quantities of distinct values that span attributes can have tend to polarize.
This can also be observed in Figure~\ref{fig:space_overhead}.
The performance plateaus even when $\psi$=10,000, meaning there is no attributes whose value size falls in the range of [1,000, 10,000].
Certain attributes (e.g., authentication tokens, DB queries, span ID) might possess a substantially larger set of values compared to others.
Consequently, their inclusion (when $\psi$ is too large) in the \sname could potentially bloat its size.
To address the variability in attribute value distribution and maintain manageable memory usage, we also set a cap on the size of the data structures, e.g., limiting it to 5MB.

\subsubsection{Computational Efficiency}

To ensure \alias can be seamlessly integrated with services, all operations are designed for optimal efficiency.
The time complexity of \alias's core operations is analyzed as follows.
The construction and restructuring of \sname operate with a time complexity of $\mathcal{O}(m)$, where $m$ is the number of span attributes.
In many scenarios, $m$ typically remains below 20.
Other operations, such as hashing, dictionary mapping, and path searching, all have a complexity of $\mathcal{O}(1)$.
Thus, the overall time complexity of \alias is \textit{linear}, making it highly efficient.

\begin{figure}
    \centering
    \includegraphics[width=0.66\linewidth]{figures/experiments/space_overhead.pdf}
    \caption{Performance with Different $\psi$}
    \label{fig:space_overhead}
\end{figure}

To evaluate the efficiency of \alias, we measure the trace collection throughput for the \textit{basic} microservice~\cite{trainticket} of the Train Ticket benchmark.
Specifically, we deployed the instrumented OpenTelemetry system within a container (configured with one core and 1GB of memory) to compress and relay the spans.
The throughput is calculated as the uncompressed size of spans divided by the time token to transmit the traces from the service to the backend.
% This includes the time used for data compression (if \alias or gzip is applied) and JSON serialization.
% Specifically, we evaluate the time taken by different configurations to transmit the traces from the service to the backend, and then divide this duration by the overall size of the dataset.
In the most basic setting, referred to as \textit{Original}, the time is purely the period needed for data transmission and JSON serialization.
When compression techniques such as \alias and gzip are employed, we take into account the additional time required for data compression and decompression.
The results are present in Table~\ref{tab:throughput}.

\begin{table*}[h]
    \centering
    \caption{Performance of Throughput (MB/s) on Train Ticket}
    \label{tab:throughput}
    \centering
    \footnotesize
    \begin{NiceTabular}{C{2.5cm}|C{1cm}|C{1cm}|C{1cm}|C{1cm}}
        \specialrule{0.35mm}{0em}{0em}
        $\psi$ & \textbf{1} & \textbf{10} & \textbf{100} & \textbf{1,000} \\
        \specialrule{0.15mm}{0em}{0em}
        \specialrule{0.15mm}{.1em}{0em}
        Original & 13.98 & 13.57 & 13.78 & 14.05 \\
        \rowcolor{lightgrey} +\alias & 89.34 & 94.35 & \textbf{109.68} & 108.15 \\
        +gzip & 14.65 & 14.27 & 14.35 & 14.02 \\
        \rowcolor{lightgrey}+\alias (gzip) & 60.65 & 63.78 & 68.78 & 68.56 \\
        \specialrule{0.35mm}{0em}{0em}
    \end{NiceTabular}
\end{table*}

It can be seen that upon the integration of \alias, the throughput of trace collection is accelerated by nearly eight times (e.g., from 13.78MB/s to 109.68MB/s).
% This can be attributed to the fact that \alias requires a small amount of time for compression and decompression.
Another interesting observation is that gzip brings little performance gain to the throughput.
One important reason is that gzip compression is performed by the HTTP client library before the data is sent over the network.
Thus, gzip is applied after the data has been serialized into JSON.
As a time-consuming step, JSON serialization constitutes the performance bottleneck.
Moreover, gzip indiscriminately attempts to compress all information, including elements such as trace IDs, span IDs, and authentication tokens, which are inherently resistant to compression.
The (wasted) computational overhead of gzip compression and decompression thus offsets its benefits.
In contrast, \alias can accurately identify the incompressible attributes, i.e., the local fields, and bypass them.
Since \alias is applied before JSON serialization, it substantially reduces the volume of data that needs to be encoded.
% This greatly reduces the amount of data needed to be encoded as \alias is applied before the JSON serialization.
Such a design not only improves the throughput, but also benefits the CPU usage.
Our experiments indicate that the CPU utilization of Tracezip is merely 20\%$\sim$40\% of that in the Original and +gzip settings.


% In contrast, \alias is applied before the JSON serialization, which greatly decreases the data needed to be encoded.
% This also reduces the CPU usage of \alias by more than 60\% when compared to the Original and +gzip settings.

% Such a design choice significantly contributes to \alias's superior efficiency.
% In our lab environment, the processing speed of \alias can reach 100MB/s, while the fastest traditional algorithm, i.e., gzip, is only around 10MB/s.

% \zb{show a figure of CPU percentage and throughput}


\subsection{Threats to Validity}

% We discuss potential threats to the validity of our study.
When evaluating the performance and applicability of \alias, several potential threats to validity must be considered to ensure the robustness and generalizability of our findings.

\textbf{Internal validity}. One of the primary concerns regarding internal validity is the accuracy of our evaluation metrics and the potential biases in our experimental setup.
Real-world cloud services exhibit a vast array of complexities and variations, making it challenging to capture all possible scenarios within a single study.
To address this challenge, we carefully select a diverse set of microservices benchmarks and production trace data from Alibaba that we believe are representative of typical cloud service operations.
These benchmarks and dataset are chosen to reflect common patterns and behaviors observed in real-world applications, thus providing a meaningful context for evaluating \alias's performance.
Additionally, any configuration or tuning of \alias that is specific to these datasets might inadvertently favor our approach, potentially skewing the results.
To mitigate this, we ensure that the benchmarks and dataset are selected and configured independently of \alias's development process.

\textbf{External validity}. External validity pertains to the generalizability of our results to other settings or systems.
Our evaluation of \alias is specifically designed to address the diversity inherent in real-world cloud systems.
We implement and test \alias within the OpenTelemetry Collector framework and evaluate it across a range of cloud environments and backend components, including gRPC, Apache Kafka, MySQL, and others. 
These settings are carefully selected to reflect the variety of systems and technologies commonly used in cloud services, ensuring a comprehensive basis for assessing \alias's effectiveness.
By choosing such a diverse array of environments and applications, we aim to capture the broad spectrum of redundancy patterns and data characteristics found in typical cloud systems.
This approach helps to ensure that our findings are applicable to a wide range of real-world scenarios, demonstrating \alias's capability to perform effectively in diverse and dynamic cloud environments.
% \section{Discussion}
\label{sec:discussion}
\textsc{WWD} is a socio-technical infrastructure that supports the collection of cultural data, in the form of food, in a bottom-up community-led manner. Community members' needs and experiences actively shaped the architecture of WWD. Our data collection platform was constructed to be compatible with the established digital infrastructure and cultural norms of the communities we worked with. The types of data we collected (e.g., the attributes for each dish) were informed by community members who identified what was important to capture about a dish to accurately represent how the dish is prepared and consumed in their culture. 

Building the \textsc{WWD System} in a bottom-up, community-led manner required an immense amount of labour. Data did not simply flood in once the system architecture was built. Core Organisers and Community Ambassadors engaged in \textit{data work}---a socio-technical process through which data about local cuisines was produced. As many social computing scholars have noted, data work is often overlooked despite its essential role in shaping the epistemology of a dataset and consequently the downstream performance of ML systems~\cite{sambasivan2021everyone,ismailEngagingSolidarityData2018,mollerWhoDoesWork2020,scheuermanProductsPositionalityHow2024}. 

In our discussion, we surface the tensions that occurred during data work. These breakdowns in the data work process help us to reveal deeper structural issues in the AI/ML production pipeline that confound bottom-up, community-led approaches to dataset construction. Communities facing representational harms~\cite{weidinger2021ethical} and disparities in quality of service~\cite{shankar2017allocational, de2019doescvworkallocational} face a catch-22 when participating in efforts to improve dataset coverage: They can shoulder the burden of participation or be excluded from model ontology. New technologies, particularly GenAI tools, have been proposed as a way for communities to preserve their culture representation by participating in efforts to contribute data to model training~\cite{heritage7030070}. However, participation does not necessarily entail improved outcomes for communities~\cite{birhanePowerPeopleOpportunities2022}. We point to a difference in ethical frameworks between communities on the African continent with whom we worked and those of large tech companies that build and control GenAI technologies to illuminate why the promises of participation often fall short. 


\subsection{Tensions in data collection}
 
Our results show that there were tensions around data collection. Specifically, issues surrounding image provenance, the accuracy of information about a dish, and the benefits of participation arose throughout the data collection process. These issues reveal deeper structural problems with the AI/ML pipeline. 
\subsubsection{Establishing a clean bill of data provenance}
Recent efforts in participatory ML research attempt to safeguard the labour and intellectual property of community data contributors by creating dataset licenses restricting the use of the community's dataset, which assign ownership and terms of access and use to these datasets~\cite{birhanePowerPeopleOpportunities2022,longpre2024largelicence}. However, for the license to be effective, the data must have a clean bill of provenance. \textsc{World Wide Dishes} was built to be an open-source dataset with a Creative Commons license that could be used for model evaluation. As a result, the WWD dataset had to fulfil strict requirements for data quality, including ensuring that the dataset creators had a right to the images contained within the dataset.  In other words, the creators of \textsc{WWD} must then be able to claim rightful use and ownership over all the images collected as part of the project. However, many of the images that Contributors submitted during the data collection phase were taken from the Internet and lacked proper licensing. As a result, Community Ambassadors had to engage in extensive consultation and discussion with Contributors to ensure they understood the importance of data provenance in their submissions. Contributions, where the origins of the submitted were unclear, had to be deleted, erasing bits of cultural knowledge from our dataset. Ensuring a clean bill of data provenance was time-intensive and not easily scalable. It was difficult to enforce image upload guidelines in a volunteer effort, resulting in a smaller, less representative, dataset than we would have liked. However, the rigorous process of ensuring a clean bill of data provenance for each submission enabled us to, in good faith, release our dataset as an open-source project. 

The standard of open-sourcing datasets and applying Creative Commons license, while understandable, places massive burdens on small, community-led projects such as \textsc{WWD} to ensure clean bills of data provenance. To be clear, we are not arguing against open-source datasets or Creative Commons licenses, but rather are demonstrating the need to build infrastructures that support and fund the labour needed to verify that data for their projects can be used. As mentioned in~\cref{background}, understanding cultural nuance on a fine-grained, regional scale requires extensive (and non-extractive) consultation with community members who have the capacity to share local expertise. As such, non-exploitative and non-extractive community consultation is an important step in verifying the validity and veracity of cultural information. 

\subsubsection{Verifying cultural information}

Collecting accurate and representative cultural data is exceptionally difficult. Cultures are not bounded by government borders and/or other manufactured systems, but rather extend across larger regions and are often the product of intercultural exchanges~\cite{gupta2008beyondculture}. This makes determining the veracity of a data point in \textsc{WWD} almost impossible without extensive consultation with a community member with local expertise. In \textsc{WWD}, we sought to include as many Contributors as possible to collect a granular representation of cultural data. We accessed Contributors through our community ambassadors who had established relationships and trust with the folks they asked to contribute. Inclusion, however, can be a slippery slope~\cite{epstein2008rise,benjamin2016informed}. 

ML researchers continue to pursue the construction of ever more representative datasets in the name of improving model performance for \textit{everyone}~\cite{luccioni2021everyone, radford2018improvingeveryone}. Often, ML researchers have trouble accessing ``hard-to-reach'' populations, such as the communities we worked with to build \textsc{WWD}. Many recent projects have attempted to solicit engagement from ``hard-to-reach'' populations~\cite{kirk2024prism,ramaswamy2023geodegeographicallydiverseevaluation,singh2024aya_dataset}, yet none of these projects interrogate why these populations might be hard for researchers to access. Drawing on Benjamin's work~\cite{benjamin2016informed}, \textbf{we urge ML researchers to consider how research institutions and industry laboratories may engender distrust within communities that have endured centuries of extractive practices by actors from the Global North.} It is essential that researchers not only endeavour to make participation accessible to members of ``hard-to-reach'' communities but also work towards establishing themselves as trustworthy partners in the research process, in the same way that~\citet{singh2024aya_dataset} do this. 

\subsubsection{Explaining the benefits of participation}

Community Ambassadors wrestled with explaining the benefits of participation in \textsc{WWD} to potential Contributors. Participation was not financially compensated. The research team chose not to make use of professional data centre workers\footnote{Professional data centre workers are those people employed in a centralised manner to perform data collection tasks. Their livelihood is, therefore, connected to the requirement to engage in data contribution, which does not align with \textsc{WWD} goals. Additionally, even had we wished to use data centre workers, we lacked the resources to which a large technology company might have access, such as the ability to engage a business outsourcing company (e.g., Enlabler) to recruit and pay data workers.} because the nature of the data collection process argued for prioritising organic engagement through social networks to collect perspectives from people who do not, and have not, typically contributed to Internet datasets from around the world. We purposely chose a data collection method that would enable the use of social networks and allow us to reach participants other than those employed in a data worker centre, such as older generations and those across a wide socioeconomic range. We also wanted to empower participants to involve their families in the process. 

Although the research team would have preferred to individually compensate each Contributor, because \textsc{WWD} relies on a decentralised, global-scale data collection method, and, crucially, as of the time of data collection, normative standards and infrastructure do not exist to support such a decentralised payment process to effectively \textbf{pay data contributors}, we were \textit{unable} to pay them. The research team explored many possible avenues for paying participants but each time came up against prohibitively expensive and logistically insurmountable barriers. For example, money transfer services such as PayPal~\cite{paypal_countries_2023} and Wise~\cite{wise_usage_2023} were unavailable in many of the regions where \textsc{WWD} operates. These types of services also require that payment recipients have access to digital banking services, which many within our target communities do not. In addition, some of our Core Organisers, who are from the African continent and utilise digital banking services, provided anecdotal evidence of times when their transactions were flagged for seemingly no other reason than their nationality. Infrastructures to support financial remuneration for research participants in the Majority World are simply not commensurate with the many calls from Western researchers to engage participants in these parts of the world. \textbf{Researchers must therefore build the infrastructures to enable equitable participation with communities}; in particular, researchers should investigate how to address breakdowns in participant compensation infrastructures. Other similarly decentralised efforts have remunerated contributors with material items (e.g., sweaters and small gifts)~\cite{singh2024aya_dataset}. Still other researchers point to the limitations of financial compensation for participants and urge researchers to consider what kinds of remuneration would be useful given the context of their research site~\cite{hodge2020relational}.

Despite the lack of extrinsic, financial incentives, the Contributors did exhibit some intrinsic motivation. Contributors shared many different reasons for having participated, such as wanting to make a difference in GenAI outputs, supporting a friend, or contributing to a mission and team they believed in. The majority of data contributions came from the African continent. The authors have speculated why this might be, and have wondered if there is a common focus uniting these Contributors: a central philosophy of ``familyhood'' and unity. This is known by different terms across the continent, including djema’a (Arabic), ubuntu (Zulu), ujamaa (KiSwahili), umuntu (Chichewa), and unhu (Shona). Community Ambassadors also suspected that their positionality as members of the communities from which they were soliciting data contributions further strengthened sentiments of unity among participants who saw \textsc{WWD} as an extension of the growing ``By Africans, for Africans'' movement in ML~\cite{birhane_2024_for_africans}. 

Whilst we can only speculate about why participants engaged in the data contribution process, the authors recognise the responsibility they were given to respect and honour these Contributors and to avoid extractive and exploitative practices. 

\subsection{Participate or be excluded: A catch-22}


Cultural erasure and lack of representation are rooted in deep systemic issues that date back centuries. GenAI, especially T2I models, play an increasingly prominent role in shaping the media ecosystem. However, relying on these models to ``fix'' centuries of intentional cultural erasure overlooks the deeper systemic issues that will likely constrain the efficacy of these technocentric solutions. During the data collecting for \textsc{WWD}, Community Ambassadors often found themselves rationalising the uncompensated nature of data contributions by demonstrating that existing T2I models perform poorly when creating images of local dishes so Contributors should provide accurate data to teach the model what the dish should look like. Regardless, many Contributors, Community Ambassadors recalled, were eager to participate in an African-researcher-led ML effort. 

Through the reflection process, Community Ambassadors shared conflicting feelings about tapping into the shared philosophy of familyhood and unity that they suspected motivated Contributors' participation. On the one hand, local communities were engaging in the dataset creation process through the lens of \textit{Ubuntu} (broadly translated as ``I am because we are'')---an ethical framework that emphasises dignity, reciprocity, and the common good~\cite{ewuoso2019core}. In contrast, the models that would subsequently be trained by these datasets are developed in Western contexts and imbued with utilitarian ethics---a framework that emphasises the best for the greatest number of people~\cite{selbstFairnessAbstractionSociotechnical2019,west2004introduction}. These two distinct, yet interrelated elements of the ML pipeline---dataset production and model development---are therefore produced not only in distinct geographic regions~\cite{sambasivan2021everyone,scheuermanDatasetsHavePolitics2021} but also, in our case, in two distinct ethical frameworks. Contributors who engaged with us out of a sense of \textit{Ubuntu} are unlikely to see their values recognised and preserved in the actual functioning of the downstream T2I model that is optimising for fundamentally different well-being criteria.  

Participation in dataset construction is not a guaranteed way to achieve representational justice in T2I models. Thus, proposing to communities that to avoid being excluded from the future of media representation, they should participate in dataset development, is misleading. This false choice obfuscates (1) the deeper systematic issues that dictate whose culture gets preserved and represented and (2) the disjunction between the value system under which participants may contribute data and that of the models that are then trained on this data. 

Future research efforts should examine how to bring the ethical frameworks of dataset creation and model development into alignment by prioritising local, community ownership over AI. 


\section{Related Work \label{related}}
%
\noindent~{\bf Individual Device Attestation (\ra)}
is an extensively studied topic and numerous schemes have been proposed in the literature. These 
techniques generally fall into three categories: software-based, hardware-based, and hybrid.
%
Given the lack of rich hardware features on embedded platforms, lightweight Software-based \ra 
\cite{li2011viper,seshadri2006scuba,seshadri2004swatt,surminski2021realswatt} is only viable 
for legacy devices with no secure hardware features. 
It uses request-to-response time (between \vrf and \prv) to establish confidence 
in the integrity of the attestation report.
Nonetheless, network limitations (e.g. intermittent connection, network congestion) on \prv introduce noise to the request-to-response 
time, making software-based \ra impractical.

In contrast, hardware-based \ra techniques 
\cite{mccune2010trustvisor,noorman2013sancus,strackx2010efficient,
ling2021secure,chen2019opera,chen2022mage} either (1) embed \prv attestation functionality entirely 
within dedicated hardware, 
or (2) require substantial changes to the underlying hardware to support isolated execution of trusted 
software, e.g., SGX \cite{sgx} or TrustZone \cite{trustzone}.
However, such hardware features are often too complex and costly for low-end devices constrained by 
size, energy, and cost.

Given the limitations of both hardware- and software-based approaches in low-end embedded platforms, 
software/hardware co-design (hybrid) \cite{vrased,arkannezhadida,smart,tytan,nunes2020apex,trustlite} 
has recently emerged as a promising solution. It aims to provide equivalent security guarantees to hardware-based 
\ra while minimizing modifications to the underlying hardware.
The security features employed can be simplified to utilize a ROM or a memory protection unit (MPU).
Current hybrid \ra techniques implement the integrity-ensuring function (e.g., MAC) in software.
They use trusted hardware to control the execution of this software, preventing any violations that may compromise security, such as key leakage, or preemption of unprivileged software.

RealSWATT \cite{surminski2021realswatt} introduces a software-based approach designed for continuous attestation of real-time and multi-core systems, effectively solving the TOCTOU problem.
PISTIS \cite{grisafi2022pistis} is also a software trusted computing architecture enabling memory isolation, remote attestation, and secure update.
SANCUS \cite{noorman2013sancus} and TrustVisor \cite{mccune2010trustvisor} are hardware-based solutions offering attestation service with software module isolation. 
VRASED \cite{vrased} presents a formally verified hybrid RA architecture.
It implements the attestation function in software while employing small trusted hardware to enforce the attestation correctness and access control over the \ra secret key.
IDA \cite{arkannezhadida} proposes a novel hybrid attestation method that enables interrupts even during attestation, enhancing overall system security and flexibility.
Moreover, IDA monitors program memory between attestation requests to prevent TOCTOU attacks.
As previously mentioned in Section \ref{sec:bg}, \rata, \casu, and \garota are hybrid \ra architectures.
The first two provide constant-time computation for attestation requests (heartbeat requests in \casu) regardless of the size of the attested regions.
Meanwhile, the last provides a trusted timer and network that can be preemptively executed by authorized software.
Table \ref{table:comp_ind_att} compares various software, hardware, and hybrid \ra methods.

\noindent~{\bf Network Attestation (\sa)}
%
enables scalable attestation for large groups of interconnected devices. Few prior work \cite{asokan2015seda,ambrosin2016sana,carpent2017lightweight,ibrahim2017seed,ibrahim2016darpa,kohnhauser2017scapi,nunes2019towards,kohnhauser2018salad,petzi2022scraps,kuang2019esdra,abera2019diat} refers to this process as Swarm Attestation; we employ the term Network Attestation to denote the same concept. Table \ref{table:comp_swa_att} shows a comparison with other \sa schemes.

The first scheme, SEDA \cite{asokan2015seda}, employs secure hop-by-hop aggregation of \ra reports. 
Initially, \vrf broadcasts an attestation request to \prv-s. Each \prv attests its children nodes and forwards aggregated \ra reports to its parent. Finally, \vrf verifies only the last \ra reports to assess the status of all \prv-s.
SANA \cite{ambrosin2016sana} extends SEDA with a novel aggregate signature scheme, ensuring low verification overhead with minimal trust anchor.
It partitions \prv-s into subnetworks and aggregates \ra results across the entire network, facilitating public verification by multiple \vrf-s.
LISA \cite{carpent2017lightweight} introduces neighbor-based communication to propagate \ra reports. \prv-s verify \ra reports of other \prv-s before forwarding them to prevent replay attacks, and a quality metric for \sa techniques captures the information from each \prv.
SeED \cite{ibrahim2017seed} enhances the efficiency of SEDA and resilience against DoS attacks by enabling \prv-s to self-initiate \ra.
DARPA \cite{ibrahim2016darpa} detects physically compromised devices by exchanging heartbeat messages among \prv-s to identify compromised or absent devices.
SCAPI \cite{kohnhauser2017scapi} improves DARPA;  it introduces a leader that periodically generates and distributes secret session keys among \prv-s. To receive a new session key, \prv must be authenticated with the previous key.
SAP \cite{nunes2019towards} constructs a formal model encompassing desirable efficiency, soundness, and security notions for \sa. It systematically designs a synchronous attestation protocol compliant with security goals defined by the formal model.
SALAD \cite{kohnhauser2018salad} provides lightweight message aggregation for dynamic networks with intermittent connectivity, distributing \ra proofs among all devices.

SCRAPS \cite{petzi2022scraps} proposes a Pub/Sub network \sa protocol. It involves a proxy verifying \prv’s \ra reports on behalf of actual \vrf.
This proxy is implemented using smart contracts, i.e., untrusted entities hosted on a blockchain.
Once the proxy attests a \prv, \vrf-s can retrieve the \ra evidence from the proxy without trusting the proxy, enabling many-to-many attestation.
This enables many-to-many attestation by allowing \vrf-s to fetch \ra reports from the proxy.
ESDRA \cite{kuang2019esdra} designs a first many-to-one \sa scheme to eliminate fixed \vrf and reduce a single point of failure \vrf risks.
Moreover, the distributed attestation facilitates offering feedback on certain compromised \prv-s, thus suitable for half-dynamic networks.
DIAT \cite{abera2019diat} presents a control-flow attestation scheme for autonomous collaborative systems.
It combines data integrity attestation, modular attestation, and representation of execution paths, enabling efficient run-time attestation in a setting where embedded systems must act as both, \prv and \vrf.



%\vspace{0.2cm}
\section{Conclusion}

Subgroup analysis is an important, yet under-utilized tool in data science.
Our results suggest that combining algorithm-generated, rule-based insights with human intuition and experimentation in an interactive workflow can help practitioners develop a thorough understanding of complex datasets.
By implementing these interactions in a lightweight notebook-based tool, we hope to lower the barrier for data scientists to try subgroup discovery and to curate unexpected, interesting subpopulations in their data.
Divisi is available as an open-source package so that data scientists and HCI researchers can build on this work, helping to make exploratory subgroup analysis more feasible for a wider range of contexts.

% \section*{Data Availability}
% The source code and datasets of this work are publicly available on \url{https://anonymous.4open.science/r/tracezip-C74A}.

\section*{Acknowledgments}
The work described in this paper was supported by the National Natural Science Foundation of China (No. 62402536).
We extend our sincere gratitude to the anonymous reviewers for their insightful feedback.


\bibliographystyle{ACM-Reference-Format}
\bibliography{bibliography}

\end{document}
\endinput