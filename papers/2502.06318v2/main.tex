%%
%% This is file `sample-acmsmall-conf.tex',
%% generated with the docstrip utility.
%%
%% The original source files were:
%%
%% samples.dtx  (with options: `all,proceedings,bibtex,acmsmall-conf')
%% 
%% IMPORTANT NOTICE:
%% 
%% For the copyright see the source file.
%% 
%% Any modified versions of this file must be renamed
%% with new filenames distinct from sample-acmsmall-conf.tex.
%% 
%% For distribution of the original source see the terms
%% for copying and modification in the file samples.dtx.
%% 
%% This generated file may be distributed as long as the
%% original source files, as listed above, are part of the
%% same distribution. (The sources need not necessarily be
%% in the same archive or directory.)
%%
%%
%% Commands for TeXCount
%TC:macro \cite [option:text,text]
%TC:macro \citep [option:text,text]
%TC:macro \citet [option:text,text]
%TC:envir table 0 1
%TC:envir table* 0 1
%TC:envir tabular [ignore] word
%TC:envir displaymath 0 word
%TC:envir math 0 word
%TC:envir comment 0 0
%%
%%
%% The first command in your LaTeX source must be the \documentclass
%% command.
%%
%% For submission and review of your manuscript please change the
%% command to \documentclass[manuscript, screen, review]{acmart}.
%%
%% When submitting camera ready or to TAPS, please change the command
%% to \documentclass[sigconf]{acmart} or whichever template is required
%% for your publication.
%%
%%
% \documentclass[acmsmall,screen,review,anonymous]{acmart}
\documentclass[acmsmall]{acmart}

% \usepackage{cite}
% \usepackage{amsmath,amssymb,amsfonts}
\usepackage{graphicx}
\usepackage{textcomp}
\usepackage{xcolor}
\usepackage{balance}
\usepackage{booktabs}
\usepackage{CJKutf8}
\usepackage{tikz}
\usepackage{forest}
\usepackage{booktabs}
\usepackage{makecell}
\usepackage{float}
\usepackage{algorithm}
\usepackage{algpseudocode}
\usepackage{xspace}
\usepackage{listings}
% \usepackage{minted}
\usepackage[symbol]{footmisc}
\usepackage{longtable}
\usepackage{nicematrix}
\usepackage{colortbl}
\usepackage{caption}
\usepackage{fontawesome5}
\usepackage{pifont}
\usepackage{subcaption}
\usepackage{multirow}
\usepackage{booktabs}
\usepackage{arydshln}
\usepackage{tabularx}
\usepackage{soul}
\usepackage{hyperref}
\usepackage[]{collab}

\renewcommand{\arraystretch}{1.72}
\setlength{\arrayrulewidth}{0.15mm}
\newcolumntype{C}[1]{>{\centering\arraybackslash}p{#1}}

\definecolor{lightgrey}{rgb}{0.96, 0.96, 0.96}
\definecolor{grey}{rgb}{0.92, 0.92, 0.92}
\definecolor{keycolor}{HTML}{2B4983}
\definecolor{valuecolor}{HTML}{61982D}

\newcommand\alias{\textsc{Tracezip}\xspace}
\newcommand\name{Span Retrieval Tree\xspace}
\newcommand\sname{SRT\xspace}
\newcommand{\red}[1]{\textcolor{red}{#1}}
\newcommand{\blue}[1]{\textcolor{blue}{#1}}
\collabAuthor{zb}{teal}{Zhuangbin}

%%
%% \BibTeX command to typeset BibTeX logo in the docs
\AtBeginDocument{%
  \providecommand\BibTeX{{%
    Bib\TeX}}}

%% Rights management information.  This information is sent to you
%% when you complete the rights form.  These commands have SAMPLE
%% values in them; it is your responsibility as an author to replace
%% the commands and values with those provided to you when you
%% complete the rights form.
% \setcopyright{acmlicensed}
% \copyrightyear{2025}
% \acmYear{2025}
% \acmDOI{10.1145/3728888}

%% These commands are for a PROCEEDINGS abstract or paper.
% \acmConference[ISSTA '25]{The ACM SIGSOFT International Symposium on Software Testing and Analysis}{June 25--28,
%   2025}{Trondheim, Norway}
%%
%%  Uncomment \acmBooktitle if the title of the proceedings is different
%%  from ``Proceedings of ...''!
%%
%%\acmBooktitle{Woodstock '18: ACM Symposium on Neural Gaze Detection,
%%  June 03--05, 2018, Woodstock, NY}
% \acmISBN{978-1-4503-XXXX-X/18/06}


%%
%% Submission ID.
%% Use this when submitting an article to a sponsored event. You'll
%% receive a unique submission ID from the organizers
%% of the event, and this ID should be used as the parameter to this command.
%%\acmSubmissionID{123-A56-BU3}

%%
%% For managing citations, it is recommended to use bibliography
%% files in BibTeX format.
%%
%% You can then either use BibTeX with the ACM-Reference-Format style,
%% or BibLaTeX with the acmnumeric or acmauthoryear sytles, that include
%% support for advanced citation of software artefact from the
%% biblatex-software package, also separately available on CTAN.
%%
%% Look at the sample-*-biblatex.tex files for templates showcasing
%% the biblatex styles.
%%

%%
%% The majority of ACM publications use numbered citations and
%% references.  The command \citestyle{authoryear} switches to the
%% "author year" style.
%%
%% If you are preparing content for an event
%% sponsored by ACM SIGGRAPH, you must use the "author year" style of
%% citations and references.
%% Uncommenting
%% the next command will enable that style.
%%\citestyle{acmauthoryear}


%%
%% end of the preamble, start of the body of the document source.
\begin{document}

%%
%% The "title" command has an optional parameter,
%% allowing the author to define a "short title" to be used in page headers.
\title{\alias: Efficient Distributed Tracing via Trace Compression}

\author{Zhuangbin Chen}
\orcid{0000-0001-5158-6716}
\affiliation{%
  \institution{School of Software Engineering, Sun Yat-sen University}
  \city{Zhuhai}
  \country{China}
}
\email{chenzhb36@mail.sysu.edu.cn}

\author{Junsong Pu}
\orcid{0009-0002-8309-1384}
\affiliation{%
  \institution{Beijing University of Posts and Telecommunication}
  \city{Beijing}
  \country{China}
}
\email{angrychow@bupt.edu.cn}

\author{Zibin Zheng}
\orcid{0000-0002-7878-4330}
\affiliation{%
  \institution{School of Software Engineering, Sun Yat-sen University}
  \city{Zhuhai}
  \country{China}
}
\email{zhzibin@mail.sysu.edu.cn}
\authornote{Zibin Zheng is the corresponding author.}

%%
%% The "author" command and its associated commands are used to define
%% the authors and their affiliations.
%% Of note is the shared affiliation of the first two authors, and the
%% "authornote" and "authornotemark" commands
%% used to denote shared contribution to the research.
% \author{Ben Trovato}
% \authornote{Both authors contributed equally to this research.}
% \email{trovato@corporation.com}
% \orcid{1234-5678-9012}
% \author{G.K.M. Tobin}
% \authornotemark[1]
% \email{webmaster@marysville-ohio.com}
% \affiliation{%
%   \institution{Institute for Clarity in Documentation}
%   \city{Dublin}
%   \state{Ohio}
%   \country{USA}
% }

% \author{Lars Th{\o}rv{\"a}ld}
% \affiliation{%
%   \institution{The Th{\o}rv{\"a}ld Group}
%   \city{Hekla}
%   \country{Iceland}}
% \email{larst@affiliation.org}


%%
%% By default, the full list of authors will be used in the page
%% headers. Often, this list is too long, and will overlap
%% other information printed in the page headers. This command allows
%% the author to define a more concise list
%% of authors' names for this purpose.
\renewcommand{\shortauthors}{Zhuangbin Chen, Junsong Pu, and Zibin Zheng}

%%
%% The abstract is a short summary of the work to be presented in the
%% article.
\begin{abstract}
\textit{Distributed tracing} serves as a fundamental building block in the monitoring and testing of cloud service systems.
To reduce computational and storage overheads, the \textit{de facto} practice is to capture fewer traces via sampling.
However, existing work faces a trade-off between the completeness of tracing and system overhead.
On one hand, \textit{head-based sampling} indiscriminately selects requests to trace when they enter the system, which may miss critical events.
On the other hand, \textit{tail-based sampling} first captures all requests and then selectively persists the edge-case traces, which entails the overheads related to trace collection and ingestion.
Taking a different path, we propose \alias in this paper to enhance the efficiency of distributed tracing via \textit{trace compression}.
Our key insight is that there exists significant redundancy among traces, which results in repetitive transmission of identical data between services and the backend.
We design a new data structure named \name (\sname) that continuously encapsulates such redundancy at the service side and transforms trace spans into a lightweight form.
At the backend, the complete traces can be seamlessly reconstructed by retrieving the common data that are already delivered by previous spans.
\alias includes a series of strategies to optimize the structure of \sname and a differential update mechanism to efficiently synchronize \sname between services and the backend.
Our evaluation on microservices benchmarks, popular cloud service systems, and production trace data demonstrates that \alias can achieve substantial performance gains in trace collection with negligible overhead.
We have implemented \alias inside the OpenTelemetry Collector, making it compatible with existing tracing APIs.
\end{abstract}


\setcopyright{acmlicensed}
\acmJournal{PACMSE}
\acmYear{2025} \acmVolume{2} \acmNumber{ISSTA} \acmArticle{ISSTA019} \acmMonth{7}\acmDOI{10.1145/3728888}


%%
%% The code below is generated by the tool at http://dl.acm.org/ccs.cfm.
%% Please copy and paste the code instead of the example below.
%%
\begin{CCSXML}
<ccs2012>
   <concept>
       <concept_id>10010520.10010521.10010537.10003100</concept_id>
       <concept_desc>Computer systems organization~Cloud computing</concept_desc>
       <concept_significance>500</concept_significance>
       </concept>
   <concept>
       <concept_id>10010520.10010575.10010577</concept_id>
       <concept_desc>Computer systems organization~Reliability</concept_desc>
       <concept_significance>500</concept_significance>
       </concept>
   <concept>
       <concept_id>10010520.10010575.10010579</concept_id>
       <concept_desc>Computer systems organization~Maintainability and maintenance</concept_desc>
       <concept_significance>500</concept_significance>
       </concept>
   <concept>
       <concept_id>10010520.10010575.10010755</concept_id>
       <concept_desc>Computer systems organization~Redundancy</concept_desc>
       <concept_significance>500</concept_significance>
       </concept>
   <concept>
       <concept_id>10011007.10011006.10011073</concept_id>
       <concept_desc>Software and its engineering~Software maintenance tools</concept_desc>
       <concept_significance>500</concept_significance>
       </concept>
 </ccs2012>
\end{CCSXML}

\ccsdesc[500]{Computer systems organization~Cloud computing}
\ccsdesc[500]{Computer systems organization~Reliability}
\ccsdesc[500]{Computer systems organization~Maintainability and maintenance}
\ccsdesc[500]{Computer systems organization~Redundancy}
\ccsdesc[500]{Software and its engineering~Software maintenance tools}

%%
%% Keywords. The author(s) should pick words that accurately describe
%% the work being presented. Separate the keywords with commas.
\keywords{Distributed tracing, Trace compression, Cloud computing, System monitoring}
%% A "teaser" image appears between the author and affiliation
%% information and the body of the document, and typically spans the
%% page.

% \received{20 February 2007}
% \received[revised]{12 March 2009}
% \received[accepted]{5 June 2009}

%%
%% This command processes the author and affiliation and title
%% information and builds the first part of the formatted document.
\maketitle


The increasing reliance on LLMs for multimodal tasks across far-reaching sectors such as healthcare, finance, and manufacturing underscores the need to assess the accuracy and reliability of the information they generate. Vision-Language Models (VLM) have achieved state-of-the-art (SoTA) performance on Visual Question-Answering (VQA) benchmarks, and these models often utilize Retrieval-Augmented Generation (RAG) to maintain factual accuracy and relevance in a dynamic information environment. However, this has led to uncertainty in the information the LLM bases its answer on, as it may choose between parametric memory and retrieved sources. When models rely on memorized information instead of dynamically retrieving information, they may inadvertently propagate outdated or incorrect information, causing serious legal and ethical risks and undermining trust and reliability in AI systems \citep{huang2023survey}.
% The ability to strike a balance between generalization and specialization in AI systems is therefore crucial for ensuring the safe, reliable use of these technologies in real-world applications.

Despite these concerns, the way that Vision-Language models (VLMs) memorize and retrieve information, particularly in complex multimodal tasks, remains under-explored. Current research often focuses on either the general capabilities of large language models (LLMs) or the specialized retrieval mechanisms in retrieval augmented generation systems (RAG) \citep{incontext_rag,chen_murag_2022,liu_universal_2023}. Particularly in the context of multimodal retrieval and multihop reasoning, few studies analyze the tradeoff between finetuning for specialized tasks and zero-shot prompting for general-purpose vision-language capabilities. A lack of consensus on how to approach this tradeoff motivates the development of measures to quantify reliance on parametric memory, as well as metrics for quantifying the potential performance impact of extending LLMs with RAG systems.

To address this gap, we investigate how multimodal QA models balance accuracy with memorization on the WebQA benchmark. We compare finetuned multimodal systems against zero-shot VLMs, analyzing how retrieval performance influences QA accuracy. In particular, we focus on cases where retrieval fails, allowing us to measure reliance on parametric memory through two proposed metrics---the \ppr (\PPR) which quantifies how much model accuracy is influenced by retrieval quality, contrasting performance in best-case versus worst-case retrieval scenarios, and the \ucr (\UCR) which measures how often correct QA responses are generated when the retriever fails, providing a proxy for memorization.

To enable this analysis, we make several methodological contributions. For the finetuned QA models, we investigate Vision-Transformer (ViT) architectures, which allow for multihop reasoning over multiple sources. To investigate the impact of retrieval performance on trained LMs, we propose a variable-input Fusion-in-Decoder (FiD) model \cite{tanaka_slidevqa_2023, nlvr2}, building upon the VoLTA architecture \citep{pramanick_volta_2023}. For the zero-shot case, we build upon previous research on In-Context Retrieval \citep{incontext_rag} by demonstrating that LLMs such as GPT-4o are capable of performing the final ranking step of the retrieval process. In doing so, we find that GPT-4o, a general-purpose LLM, achieves SoTA performance on the WebQA task, outperforming existing finetuned RAG models by a significant margin (7\% higher accuracy). 

Crucially, our results reveal that while retrieval-augmented models reduce memorization, the training paradigm plays an important role. Finetuned models exhibit higher reliance on parametric memory, whereas zero-shot RAG approaches have lower memorization scores at the cost of accuracy. This suggests that while retrieval modules may mitigate the risks associated with outdated or incorrect information, SoTA performance requires that they be coupled with specialized QA models. Our memorization measures contribute to the development of transparent and reliable AI systems, particularly in applications where the sourcing of up-to-date, factual information is critical.



% We investigate the impact of question complexity on the ability of these models to integrate multiple data sources—such as images, text, and external retrievers—and produce coherent and accurate answers. We also explore whether in-context retrieval can be a viable alternative to traditional retrieval-augmented systems, offering a more streamlined approach to multimodal QA.

% To achieve this, we first compare zero-shot prompting multimodal LLMs with finetuned multimodal systems. We evaluate both types of models on the WebQA benchmark, a dataset designed for complex question answering that requires reasoning across both image and text sources. For the finetuned models, we use a Fusion-in-Decoder (FiD) architecture, which allows for multihop reasoning over multiple sources. Additionally, we introduce the concept of In-Context Retrieval Language Modeling (RLM), where the LLM itself performs retrieval tasks without the need for external retrievers. This method builds upon existing research in in-context learning  and aims to explore the viability of LLMs retrieving relevant sources and generating accurate answers directly from their context window.

% In order to investigate source utilization in finetuned multimodal models and LLMs, three lines of inquiry are established; 
% \begin{itemize}
%     \item Study 1: retrieval vs QA performance on webQA (motivating example, does QA answer correctly even with incorrect sources?)
%     \item Study 2: performance on adversarial examples where parametric knowledge would be incorrect by design
%     \item Study 3: improving performance on adversarial examples by fine-tuning (i.e model robustness)
% \end{itemize}

% Note, there is one weakness in this plan which is tying in the work we've already done. 
% If we added something from adversarial generation to the retrieval experiment (like a combination of study 1 + 3) it would be complete. So for instance we could try fine-tuning the retriever with adversarial examples (and not just the QA model)

% \begin{figure}
%     \centering
%     \includegraphics[width=0.95\linewidth]{figures/segmentation/webqa_segment_infill.png}
%     \caption{Example of the segmentation substitution pipeline from the WebQA task.}
%     % d5c76d760dba11ecb1e81171463288e9
%     \label{fig:seg_sub_pipeline}
% \end{figure}



% Retrieval augmented generation (RAG) with zero-shot prompting and fine-tuning Large Language Models (LLMs) have become the go-to methods for tasks relying on information retrieval and text generation. In many cases the LLMs parametric memory can sufficiently generalize to answer questions without being provided with retrieval mechanisms for out-of-domain knowledge. However, LLMs often hallucinate and provide wrong information in certain scenarios. This problem is amplified even further on open-domain Question Answering (QA) tasks involving multiple modalities. Grounded text generation using retrieved sources \citep{lewis2021retrievalaugmented} has been extensively studied for text-to-text QA tasks, but its application in multimodal settings has not been studied as much.


% Multimodal reasoning and question answering have gained prominence in recent research endeavors, with an increasing emphasis on handling various forms of data, particularly text and images. In this study, we address a specific gap in the existing literature by focusing on the development of a versatile multihop model capable of accommodating varying numbers of input images.

% Our motivation for this research lies in the growing complexity of answering questions using information on the web, where the challenge of navigating the open-domain setting is further complicated by the presence of multiple modalities and sometimes requires reasoning over multiple sources. WebQA is an ideal dataset on which to compare performance of finetuned RAG systems against general purpose LLMs; it is multimodal, with correct answers requiring reasoning over image and text sources. It is multihop, requiring a complex reasoning process over multiple sources. Finally, WebQA questions from different categories can be broken down into subdomains to analyze performance over domains of varying cardinality.

% Motivated by the real-world challenges of building retrieval and question answering (QA) systems, we design and finetune a closed domain, multimodal, multihop QA model, that is capable of reasoning over a varying number of sources taken as input from an external retriever module. This research contributes to the relatively underexplored domain of multihop reasoning across various input sources and modalities. Our goal is to explore the challenges posed by these scenarios and develop strategies that enable QA models to retrieve relevant information, conduct logical or numerical reasoning across diverse modalities, and generate coherent responses in natural language. To our knowledge, this is the first application of the Fusion-in-Decoder (FiD) architecture \cite{tanaka_slidevqa_2023, nlvr2} that is shown to work with a variable number of inputs, enabling multi-hop reasoning over sources.

% In-Context Learning refers to the ability of LLMs to perform any task by simply providing examples in the input prompt \citep{dong2022survey,min2022rethinking}. Inspired by this research, we propose a method to use the LLM itself as a multimodal retriever, potentially eschewing the requirement of a distinct retrieval module, thereby allowing the design of simpler retrieval-augmented QA systems. We dub this method In-Context Retrieval Language Modeling (RLM). To the best of the authors knowledge, In-Content RLM is disparate from other retrieval augmented approaches which utilize external retrieval modules \citep{incontext_rag,chen_murag_2022,liu_universal_2023}. Despite being a natural extension of In-Context learning, In-Context RLM has not yet been studied empirically.

% To expand on our contribution of In-Context Retrieval, this stems from the well-researched in-context learning of LLMs. In-context learning is the ability of a model to perform any task given a sufficient context window \citep{dong2022survey,min2022rethinking}. Such tasks could include retrieval and ranking, but typically, the go-to solution for tasks requiring retrieval has been RAG. To the best of the authors knowledge, In-Context Retrieval is distinct from In-Context Retrieval Augmented Language Modelling (RALM), and despite being a natural extension of In-Context learning, In-Context Retrieval has not yet been shown empirically.

% Finally, we explore the tradeoff between using zero-shot prompting LLMs and the fine-tuning approach. While we find that, overall, GPT-4o obtains SoTA performance on the WebQA task, outperforming the accuracy of existing finetuned RAG approaches by 7\%, finetuned approaches still perform better on more restricted subdomains\footnote{``In-Context RLM" @ \url{https://eval.ai/web/challenges/challenge-page/1255/leaderboard/3168}}. Finally, we validate that GPT-4o is relying on retrieval abilities to solve the task; we find that GPT-4o is capable of retrieving relevant sources in the presence of distractors and furthermore, when GPT-4o fails to retrieve correct sources, it answers incorrectly 75\% of the time, meaning that it is not relying on parametric memory for this task.

% \paragraph{Contributions}
% Based on our experimentation and analysis on the WebQA benchmark, we make the following contributions:
% \begin{itemize}
%     \item Propose a new architecture for multimodal multihop QA that takes variable number of input sources inspired by the Fusion-in-Decoder method.
%     \item Comparison of general purpose LLMs vs specialized models on the WebQA benchmark.
%     \item Observation of In-Context Multimodal Retrieval abilities of GPT-4o and that it does not rely on parametric memory for multimodal QA.
%     \item Analysis of relationship between retrieval and QA task performance.
%     \item Analysis of task and query complexity on the performance of retrieval and QA tasks.
% \end{itemize}
















% Throughout this paper, we will present our methodology, experiments, and findings, emphasizing our approach to multihop reasoning over varying numbers of input images. We believe that our work contributes to a deeper understanding of multimodal reasoning and has the potential to enhance the capabilities of question-answering systems in the intricate, multimodal landscape of web-based information.
\section{Background and Motivation}
\label{sec:background}

We introduce the background on serverless workload serving and motivate the use of runtime resource adaptation to address resource inefficiency in existing serverless platforms.

\subsection{Resource Inefficiency with Early Binding}
% In current serverless platforms, developers are required to specify immutable sizes for their deployed functions.
% Then, providers consider functions' runtime workloads  (e.g., concurrency)  and resource usage to scale out/in their instances.
% Moreover, due to high runtime variability, functions must size their functions for worst-case scenarios.
% This, however, incurs considerable resource inefficiency.
Current serverless workflow platforms (e.g., AWS Step Functions~\cite{aws-step-function} and Azure Durable Functions~\cite{azure-durable-function}) offer the opportunity for developers to build various applications with advanced logic like chaining, branching, and parallel execution.
These applications can be defined by JSON-based structured languages (e.g., Amazon States Language) or other programming languages.
Meanwhile, developers require to specify resource configurations, including memory size, CPU cores, and scaling options, for individual functions---an early-binding approach.
The serverless platform is responsible for monitoring the workload intensity and resource usage at runtime and scaling out/in function instances accordingly.
To account for potential runtime variability, developers must size the functions in their application workflow accounting for the worst case in order to provide SLO guarantees over the end-to-end delay of request processing, e.g., the 99th percentile (P99) of the end-to-end delay must be within a given target. 
After deployment, the function sizes become immutable. The worst case is not representative and over-shoots most of the time, leading to resource inefficiency. 


To verify this claim, we conduct a data-driven analysis with a dataset from Microsoft Azure Functions~\cite{azure-dataset} to explicitly demonstrate the resource inefficiency issue. % , deriving from the worst-case based early bind.
To quantify the inefficiency, we define a metric called \emph{slack}---the margin between the actual execution time and the SLO, which is calculated as $1-l/T$ with $l$ and $T$ representing end-to-end latency and SLO, respectively.
Under certain SLO defined with P99 latency as done by existing works (e.g., \cite{osdi22-orion,mac22-wisefuse}),  we can see from Figure \ref{fig:bg:slack} that more than 60\% function invocations have slacks over 60\%.
Particularly, we analyze slacks of the top 100 most popular functions, whose invocations account for 81.6\% of the total function invocations. % (depicted in Figure~\ref{fig:bg:popular_func}) of overall invocations.
The result shows that only 20\% of the invocations of the popular functions (blue line in Figure~\ref{fig:bg:slack}) have slacks less than 40\%.
This means the majority of requests are processed faster than necessary.
Notably, in DAG-based workloads (i.e., Azure Durable Functions), the resource inefficiency further deteriorates wherein the ratio between the 95th percentile and 50th percentile is by up to three times \cite{mac22-wisefuse}.

% \begin{figure}[t!]
% \centering
% \includegraphics[width=0.25\textwidth]{./figure/motivation/Average_P99_cdf_top=100.pdf}
% \vspace{-0.3cm}
% \caption{Sufficient function slacks in production traces.}
% \label{fig:bg:slack}
% \end{figure}

\subsection{Runtime Dynamics}
\label{sec:bg:worst-case}

The resource inefficiency caused by the large slack can be mainly attributed to the over-provisioning of resources by the developer. This is to ensure that the SLO is guaranteed even in the worst case (i.e., P99). However, normal cases deviate from the worst case significantly due to runtime dynamics. 
In particular, we observe that functions face two major dynamic factors at runtime: varying working sets and inevitable performance interference. These two factors contribute significantly to the variance of the function execution time. 
% Functions face two remarkably dynamic factors at runtime: working sets and performance interference, which lead to considerable variance of execution latency.

\begin{figure*}[!t]
	\centering
	\subfloat[]{
		\includegraphics[width=0.24\textwidth]{./figure/motivation/Average_P99_cdf_top=100.pdf}
		\label{fig:bg:slack}
	}
	\hspace{8mm}
	\subfloat[]{
		\includegraphics[width=0.25\textwidth]{./figure/motivation/function-latency-ml-analyze-varying-worksets.pdf}
		\label{fig:bg:ml-func-latency}
	}
	\hspace{8mm}
	\subfloat[]{
	\includegraphics[width=0.28\textwidth]{./figure/motivation/coresident-perf.pdf}   
	\label{fig:bg:perf-inteference}
	}
	%\vspace{-0.1cm}
	\caption{(a) slacks of function invocations in production traces, (b) function latency variance caused by varying input worksets for functions object detection (OD), question answering (QA), and and text-to-speech (TS), respectively,
 (c) performance interference attributed to co-location of homogeneous function with different dominant resource demands.}
 %\vspace{-0.4cm}
\end{figure*}

%'ml-analyze':{'text-to-speech': 'text-to-speech', 'question-answer': 'question answer',
%                      'object-detection': 'object detection'
\textbf{\textit{Varying working sets.}} 
The working set, i.e., input data like videos, audios, and texts, can have varying sizes.
Taking Microsoft Azure Function Blobs (storage service) as an example, their data size difference can be as high as nine orders of magnitude~\cite{azure-function-blob}.
Such a large difference results in substantial variance of the execution time even for the same function~\cite{socc21-faast,eurosys21-ofc}.
Specifically, we measure the execution time of three functions under different working sets (detailed in \S\ref{exp:setup}).
Figure~\ref{fig:bg:ml-func-latency} illustrates the results, where we can observe a variance of up to 3.8 times in function execution caused by varying working set sizes.

% \begin{figure}[t!]
% \centering
% \includegraphics[width=0.25\textwidth]{././figure/motivation/function-latency-ml-analyze-varying-worksets.pdf}
% \vspace{-0.3cm}
% \caption{Function latency variance caused by varying input worksets}
% \label{fig:bg:ml-func-latency}
% \end{figure}	

\textbf{\textit{Performance interference.}}
% On the other hand, function deployment, which decides when and where to deploy functions, is completely undertaken by providers.
For simplicity and security, commercial serverless platforms, such as Alibaba Function Compute, Microsoft Azure, and AWS Lambda, exclusively deploy function instances belonging to the same tenant, or even belonging to the same function, in the same virtual machine~\cite{socc22-owl,atc18-peek-bench}.
For example, the empirical study in~\cite{socc22-owl} shows that in Alibaba Function Compute 65\% of the virtual machines exclusively deploy instances of the same function.
This co-location of homogeneous function instances, however, can incur severe resource contention on the same resource dimensions, particularly for network bandwidth and memory bandwidth of virtual machines~\cite{sc21-gsight,micro19-faaSprofiler,socc22-owl,atc18-peek-bench}.
To verify this observation, we use a virtual machine to run a function increasing the number of co-located instances from one to six while measuring the execution time of four different functions with resource dominance on different dimensions namely computing, I/O, network, and memory, respectively (detailed in \S\ref{exp:setup}). 
As shown in Figure~\ref{fig:bg:perf-inteference}, the co-location of homogeneous functions leads to substantial resource contention and performance interference, prolonging the function execution time up to 8.1 times. The performance interference is often hard to model and predict.

% this co-residency results in substantial increase of execution latency by up to 8.1 times,leading to considerable variance in function execution time.
% when compared with that with concurrency as one.

%for CPU-, IO-, network- and memory-intensive functions as the concurrency rises from one to six.
%Figure shows that significant performance interference can be observed, . 
%compared with the inclusive deployment (concurrency as one), 
% this exclusive deployment (gray bar) results in substantial increase of execution latency by up to 8.1$\times$ for CPU-, IO-, network- and memory-intensive functions as the concurrency rises from one to six.

% this exclusive deployment (gray bar) results in substantial increase of execution latency by up to 8.1$\times$ for CPU-, IO-, network- and memory-intensive functions as the concurrency rises from one to six.
% As depicted in Figure~\ref{fig:bg:concurrent_latency}, with the concurrency rising  from one to six,  the exclusive deployment results in substantial increase of execution latency by up to 8.1$\times$.
% This significantly magnifies execution latency variance.

% \begin{figure}[t!]
% \centering
% \includegraphics[width=0.25\textwidth]{./figure/motivation/coresident-perf.pdf}
% \vspace{-0.3cm}
% \caption{Performance interference attributed to co-residency of homogeneous function.}
% \label{fig:bg:perf-inteference}
% \end{figure}




\subsection{Runtime Resource Adaptation}
\label{sec:bg:adaptive-allocation}
To tackle the aforementioned resource inefficiency issue, we can adopt a late-binding approach through \emph{runtime resource adaptation}, which resizes functions on the fly based on runtime information (e.g., function slacks), achieving higher resource efficiency without violating SLO. For example, given a workflow as a chain of functions, the resource allocation of the downstream functions can be adjusted when the first function finishes execution. This way, the slack from the first function can be exploited to optimize resource efficiency. 

The idea sounds straightforward and has been considered in some existing works \cite{infocom22-stepconf,middleware20-fifer,socc21-llama,socc21-kraken,middleware20-xanadu}.
However, most of these works make an unrealistic assumption that either the developer performs the adaptation decision with access to runtime information or the serverless platform provider performs the adaptation with domain knowledge of the application workflow. These assumptions render these solutions impractical to deploy in real-world serverless systems. The information barrier between the developer and the provider calls for a new solution. 

We identify the following challenges and opportunities for a full-fledged design for runtime resource adaptation. 

\textbf{\textit{Skewed function execution time distribution.}} 
Resource allocation for a serverless workflow is typically done by leveraging performance profiles of all the functions in the workflow. 
During the offline profiling, the execution time distribution for each function is first obtained by running the function with a variety of sample inputs under different resource conditions. Then, given a time budget, existing approaches typically use P99 of the function execution time as a target and calculate the corresponding resource demands. However, due to the high runtime variability, the distribution of the function execution time is highly skewed where the difference between P50 and P99 can be as high as 100 times~\cite{socc23-huawei-cloud}. This means that if only the function execution time at a single percentile (P50 or P99) is used for resource allocation, there will be significant resource under-provisioning and over-provisioning for most requests at runtime. To address this issue, our idea is to allow for the exploration of the function execution time at diverse percentiles during resource allocation. 


% It is a prerequisite to profile execution latency for adaptive resource allocation.  
% As aforementioned, owing to a variety of unexpected runtime dynamics,  execution latency demonstrates skewed distributions, by up to 100$\times$ between 99\% percentile and 50\% percentile on Huawei cloud serverless~\cite{socc23-huawei-cloud} .
% This makes the current a single statistic (e.g., mean) or 99\% percentile distribution based profiling suffer significant under- and over-estimation.
% To fix this issue, our insight is to \textit{introduce more diverse percentiles to profile execution latency}. 

\textbf{\textit{Dependencies of adaptation decisions.}}
As the function execution progresses, a sub-workflow will be generated by removing the finished function(s) from the workflow. Within each sub-workflow, the resource adaptation decisions for remaining functions are dependent on each other due to the constraint imposed by the end-to-end latency SLO. For example, under-provisioning a function will result in a reduction of the time budget for executing its downstream functions, thus calling for more resources for these downstream functions to avoid SLO violations. Meanwhile, the selection of the percentile for the execution time of each function dictates resource-latency tradeoff for that function. For example, a higher percentile means that more resources will be allocated to ensure that more requests processed by the function will finish within the given time budget. On the contrary, a lower percentile means that more requests will risk SLO violation, but at the benefits of reduced resource consumption. To address such complex dependencies, we propose the following ideas: (1) We introduce two metrics (i.e., the timeout metric and the resilience metric detailed in \S\ref{sec:profilier}) to balance the resource adaptation decisions of the head function of the current sub-workflow and those of the remaining downstream functions. These metrics help us connect the decision making across sub-workflows and avoids sub-optimal adaptation decisions in each sub-workflow. 
(2) We explore lower percentiles for the head function and a high percentile (i.e., P99) for other functions in each sub-workflow. Using lower percentiles maximizes the opportunity for resource optimization since any over-time execution of the head function can later be compensated by resource adaptation in the next round. The high percentile ensures that the resource adaptation is not too radical to cause SLO violations. 

% Each workflow generates multiple sub-workflows as the execution moves forwards. 
% Within sub-workflows, the provisioning is inter-corrected.
% For instance, under-provisioning upstream functions may directly shrink the time budget for downstream functions, which dictates more resources required by the latter against (sub-) SLO violation. 
% This makes sub-workflows generally adopted as the basic unit to make adaptation decisions~\cite{socc21-llama,rtas22-fa2}. 
%  Moreover,  due to the high variance of execution performance, runtime adaptation requires to carry out function by function, i.e.,  discrete adaptation.
%  This, however, can easily lead to a sub-optimal (analyzed in~\S~\ref{sec:synthesizer:generate}).
% Our insight is to \emph{introduce a metric (i.e., resilience detailed in \S~\ref{sec:profilier}) to quantify the inter-correlation as well as a heuristic design (i.e., heavier head explained in \S~\ref{sec:synthesizer:generate})  to calibrate the sub-optimal,  such that resource adaptation can explore higher resource efficiency without SLO guarantee}.

% In particular, latency percentiles (introduced by the profiling)  involves resource adaptation as a new knob.
% Specifically, higher percentile earns  stronger guarantees in SLOs but may be highly prone to resource over-allocation because of its latency over-estimation, impairing resource efficiency.
% In contrast, decreasing percentiles offers the opportunity to explore higher resource efficiency, but suffers the risk of timeout, i.e., execution latency beyond specified time budget, and  may thus incur  SLO violations.
% Here, our insight is to \emph{moderately explore percentiles (detailed in~\S~\ref{sec:synthesizer:generate}), where head functions of  (sub-)workflows can explore lower percentiles because this creates the opportunity to reap higher resource efficiency while possible timeout can be recovered by subsequent functions' re-adaptive allocation.
% On the other head, non head functions maintain percentiles as 99\%}.
% This can well keep the trade-off between opportunities of exploring higher resource efficiency and risks of SLO violations. 
% Additionally, it effectively shrinks the searching space, benefiting the adaptation with higher time-efficiency.


\textbf{\textit{Tight resource adaptation window.}}
Runtime resource adaptation requires to calculate a new resource allocation decision for the remaining sub-workflow immediately when a function finishes execution. Since serverless functions are typically short-lived (less than 1s on average)~\cite{atc18-peek-bench,socc22-owl,atc20-serverless-in-the-wild,socc23-huawei-cloud}, the window for resource adaptation is quite tight. Assuming the serverless platform will perform the runtime adaptation on behalf of the developer since the platform has access to full runtime information, the resource adaptation decision making should be fast without involving complex calculations and logic or exploring a large space. As discussed before, the serverless platform provider does not have domain knowledge of the serverless workflow. Hence, the developer must pass the necessary information to the serverless platform for runtime adaptation decision making. Our idea is to let the developer synthesize critical hints containing resource allocation rules and options, which the serverless platform provider utilizes to perform runtime resource adaptation. The hints should be highly condensed so the serverless platform can make adaptation decisions quickly enough. 


% Apart from highly varying execution performance, serverless functions are also short-living (less than 1s on average)~\cite{atc18-peek-bench,socc22-owl,atc20-serverless-in-the-wild,socc23-huawei-cloud}, so is the window for adaptive allocation. 
% This variance and volatility calls for a well-preparation of hints for all possible runtime situations while promising them compact and straightforward enough for providers to easily take action.

% Here, our insight is to \emph{holistically synthesize hints in an offline manner, and then utilize the discreteness of adaptive allocation in both decision-making and decision-executing (detailed in~\S~\ref{sec:synthesizer:condense}) to fully condense the hints.
% Finally, hints are warped into a simple and compact table.
% Base on that, providers can accomplish the runtime adaption promptly and properly}.

To demonstrate the potential of runtime resource adaptation incorporating all the above ideas, we take a real-world serverless workflow (explained in \S\ref{exp:setup}) as an example, and evaluate its end-to-end latency (denoted by E2E) and resource consumption (CPU cores).
As illustrated in Figure~\ref{fig:bg:size}, the late-binding (blue triangle) reduces the resource consumption by up to 42.2\% compared with existing early-binding solutions (orange circle), while ensuring SLO guarantees. This highlights the significant gains from runtime resource adaptation. 


\begin{figure}[t!]
\centering
\includegraphics[width=0.45\textwidth]{./figure/motivation/size_early_bind_vs_ours.pdf}
%\vspace{-0.1cm}
\caption{Performance comparison between early-binding (left)~\cite{eurosys19-grandslam} and late-binding (runtime resource adaptation), where the CPU consumption (right) is normalized by the optimal obtained with exhaustive search.} 
%\vspace{-0.3cm}
\label{fig:bg:size}
\end{figure}

   
	







% \input{sections/03-empirical_study}
\section {Study Design}
\label{sec:methodology}

%Bianca: Aqui estava faltando definir qual o research metodo que a gente utilizou para testar o que foi desenvolvido. Eu utilizei a ideia do small-scale evaluation, visto que segundo o trabalho do Wohlin o que fizemos nao é um estudo de caso, nem um experimento. 

% 5 steps Runeson (referencia/exemplo)
% We follow the five main steps for conducting case studies
% proposed by [21]: Design, preparation, collecting data, analysis
% and reporting.

% TODO: Explicar o Precison/Recall nessa seção e não nos Resultados -- %Bianca: Eu coloquei uma versao de definicao, veja o que acha. 

In this section, we present the study design. To evaluate our research questions, we performed a small-scale evaluation \cite{Wohlin2022cs}, following two main steps: (i) Data Collection and (ii) Design \& Execution. We describe the data collection to train and test our ML models in Section \ref{subsec:data}. In Section \ref{subsec:studydesing}, we detail how we designed and executed our strategy to train and configure the ML models. 

\subsection{Data Collection}
\label{subsec:data}

\begin{figure*} [ht]
    \centering
    \includegraphics[width=380pt]{figures/fig03-data-acquisition-v2.pdf}
    \caption{Data collection process}
    \label{fig:fig-data-selection}
\end{figure*}

We used an ongoing SLR update conducted by the same first three authors of this replication \cite{Wohlin2022} as the instrument of our study. We chose this ongoing SLR update since the inclusion and exclusion of new studies were rigorously conducted based on individual assessments and the consensus of three experienced SLR researchers. First, each researcher screened all the papers, analyzing the title, abstract, and keywords and registering his individual assessment (\textit{i.e.}, we had three assessments for each paper). Then, the full texts of the studies were analyzed, and discussions were held to reach a final consensus on the list of included and excluded papers. Hence, we had the results of the initial screening by each researcher and also the final list of papers to be included and excluded.

We had access to all the studies the team analyzed during the SLR update (.bib files): a total of 591 papers were analyzed for the SLR update, of which 39 were included, and 552 were excluded. We filtered the studies to consider only the studies in English and containing an abstract. In the end, we used 551 studies in our testing set, of which 38 were included by the team assessment and 513 were excluded. We used these studies to form the testing set for our ML models. 

To train our ML models, we used a training set with 128 studies, of which 45 were included and 83 were excluded. The 45 studies used to train our models with what should be included were the same studies included in the original SLR \cite{Wohlin2022}. Since we did not have access to the list of excluded studies during the study selection phase of the original SLR, we performed a backward snowballing \cite{Wohlin14} on the original references to obtain the 83 studies used to train our models with what should be excluded. Figure \ref{fig:fig-data-selection} summarizes this process. The bib files for the included and excluded studies of the training and testing sets are available in our open science repository~\cite{zenodoOpenScience}.

\subsection{Design \& Execution}
\label{subsec:studydesing}

We developed a pipeline with the following steps to automate the study selection process of an SLR update by using ML and answering our research questions. Our pipeline is illustrated in Figure \ref{fig:fig-study-design}. 

\begin{figure*} [ht]
    \centering
    \includegraphics[width=380pt]{figures/fig-pipeline-details-v4.pdf}
    \caption{Study design pipeline}
    \label{fig:fig-study-design}
\end{figure*}

In summary, our pipeline processes a set of .bib files containing the list of studies to train the ML models and the list of studies to be analyzed. After its execution, it returns a report file in .xlsx format with the ML model predictions, informing which studies should be included and excluded, as well as metrics about the ML model predictions and the configuration used.

The pipeline receives four different .bib files as input, one containing the list of studies that should be excluded and one containing the list of studies that should be included for each set (training and testing). In case there are any errors in the input files, the pipeline will stop its execution and will inform which entry was associated with each error as well as the type of error. 

As shown in Figure~\ref{fig:fig-study-design}, we first validated the .bib files of our testing and training sets to ensure the completeness of the set, avoiding duplicated entries or missing keys. Each study entry must have a title, the year of publication, an abstract text, and an author list. 
Secondly, we applied text filtering techniques with Natural Language Processing (NLP) \cite{NLTK}, such as Lemmatization and Tokenization, to remove irrelevant characters. Thirdly, we applied Text Vectorization on the filtered texts using  Term-frequency/Inverse-Document-Frequency (TF/IDF), a technique that transforms text data into a numerical matrix of features. Fourthly, we used statistical methods to compute and select the most relevant features. In the fifth step, we trained and tuned our ML models using our training set. Finally, in the last step, we used our ML models to predict which studies of our testing set should be included and excluded and compared the results with the final list of included and excluded studies.

Additionally, an optional .env file can be passed as input to our pipeline; this file allows some steps in our pipeline to use a specific configuration, such as choosing the configuration of the Feature Selection (FS) method to compute the features, as well as the number of features to be selected in step four, and choosing the configuration for the ML models regarding which algorithm to be used, or which metric should be targeted when tuning the model as well as the type of cross-validation to be performed, in step five. All parameters that can be configured are also shown in Figure~\ref{fig:fig-study-design}.

Building on the work by Napoleão \textit{et al.} \cite{Napoleao2021}, which highlighted Support Vector Machines (SVM) as one of the most commonly used machine learning classifiers for assisting study selection in systematic literature reviews (SLRs), we chose to evaluate SVM in our study. Additionally, inspired by the findings of Pintas \textit{et al.} \cite{pintas2021feature}, who analyzed the most widely adopted ML classifiers and feature selection techniques for text classification, identifying SVM, Naive Bayes, k-Nearest Neighbors, Decision Trees, and Random Forest (RF) as the top five classifiers, we conducted initial tests with these classifiers. Our preliminary results showed that SVM and RF outperformed the other classifiers. Consequently, we focused our evaluation on SVM and RF.

% We experimented multiple configurations of our pipeline and evaluated different configurations for Feature Selection and for training and tuning of our ML classifiers. To select the best features, we tested different statistical methods such as Chi-squared (Chi2) \cite{Chi2}, Pearson Correlation \cite{pearson_r} and Analysis of Variance (Anova-F) \cite{ANOVA}. We tested different techniques to tune our ML classifiers such as K-fold cross-validxation, Times-Series cross-validations and hyperparameter tuning with GirdSearch \cite{GridSearch}.

We experimented with multiple pipeline configurations and evaluated different configurations for FS and training and tuning of our ML classifiers. During step four, to compute the best features, we tested different statistical methods such as Chi-squared (Chi2) \cite{Chi2}, Pearson Correlation \cite{pearson_r} and Analysis of Variance (Anova-F) \cite{ANOVA} as well as different ranges of features. After applying Text Filtering and Text Vectorization techniques, presented in steps three and four of our pipeline, our training set comprised 23,630 features. We identified the range with the most relevant features in our training set as 900 to 1,500 features. Notably, the best results, both in terms of F-score and Recall, were consistently achieved with experiments that selected the 1200 best features.

For each evaluation, we executed the pipeline from start to finish in a clean environment using one statistical method at a time. To avoid data leakage and bias, feature selection was conducted solely based on the training set. Furthermore, we used GridSearch for parameter tuning when creating the model, which inherently includes k-fold cross-validation for measuring the most efficient parameter configuration when developing the model based on the training set. Finally, the trained model was applied to predict the inclusion or exclusion of the unseen (holdout) papers of the testing set. Hence, the hereafter reported results refer not to cross-validations conducted during model creation but to evaluating the tuned model based on the holdout testing set for which three experienced SE researchers had manually and rigorously crafted the information on inclusion and exclusion.

The complete Python code that automates our pipeline is available in our open science repository~\cite{zenodoOpenScience}.
\section{Implementation}
\label{sec:implementation}

We have implemented \alias inside the OpenTelemetry Collector with around 3K lines of Golang code.
The OpenTelemetry Collector offers a vendor-agnostic implementation of how to manage telemetry data, which mainly includes four types of components: \textit{exporters}, \textit{processors}, \textit{receivers}, and \textit{extensions}.
We implement the span retrieval compression and decompression on the exporter and receiver, which run at the service side and backend side, respectively.
The exporter is responsible for building and updating the \sname and dictionary, compressing spans on the fly, and sending them to the remote backend.
After accepting the compressed data, the receiver performs span uncompression.
We outline some important details concerning the implementation.

\subsection{Search Acceleration by Hashing}
\label{sec:hashing_acceleration}

A straightforward data structure to implement \sname would be linked representation, which enjoys the benefits of dynamic size and efficient alterations (e.g., insertion and deletion).
However, in linked representation, the tree nodes are not stored contiguously or nearby in memory, potentially leading to more cache misses.
This factor can significantly impede the speed of path search within \sname.
To accelerate the search process, we apply hashing to convert each unique path of SRT to a path identifier, which is similar to that in Section~\ref{sec:mapping-based tree compression}.
Specifically, for each path, starting from the root we join the values of non-leaf nodes sequentially with a comma separator (similar to the CSV format). 
Based on the composed path string, we maintain a \{\textit{path}: \textit{identifier}\} mapping at the exporter.
% At the receiver, we maintain a consistent mapping at the opposite direction, i.e., \{\textit{identifier}: \textit{path}\}.
% These bi-directional mappings are suitable for compression and decompression at different sides.
When a new span is generated at the exporter, we extract the values of its universal fields based on the order in \sname.
The path search can then be quickly done for the span by checking if its path string exists in the map.
We use the \texttt{map} data type in Golang, which provides a highly efficient way to achieve this.
For any updates to the \sname, we only need to renew the affected paths as discussed in the next subsection.
% We then maintain a bi-directional mapping at the exporter, i.e., \{\textit{identifier}: \textit{path}\} and its reversed form, which will be synced with the receiver.

\subsection{Differential Data Synchronization}
\label{sec:differential_sync}

To ensure reliable span compression and uncompression, the exporter and receiver must maintain consistent copies of both the \sname and dictionary structures.
One simple strategy is for the exporter to send the latest versions of these structures upon any update.
However, given that updates often affect only a small segment of the overall structures, sending redundant (i.e., unchanged) data with each update would incur network overhead and potentially defer the uncompression process.
Thus, we implement a differential update mechanism for more resource-efficient synchronization.
The core idea is that at the receiver, instead of maintaining another \sname, we keep a path hashing in the opposite direction, i.e., \{\textit{identifier}: \textit{path}\}.
For any updates to the non-leaf nodes, we can easily pinpoint the affected paths and perform the renewal.
For example, in Figure~\ref{fig:tracezip_system}, the emergence of a new value (denoted by the pink dashed rectangle) gives rise to a novel path, i.e., \textit{path 3}.
In this case, we can add a new entry to the \{\textit{path}: \textit{identifier}\} mapping at the exporter and sync it with the receiver.
For path deletion, the exporter can simply send the corresponding identifier to the receiver for record elimination.
Other updates are essentially a combination of path addition and deletion.
% we hash each path of \sname (excluding the leaf node) into a path identifier (Section~\ref{sec:hashing_acceleration}), which is then synced with the receiver.
% For any updates to the non-leaf nodes, we can easily pinpoint the affected paths and (re)calculate the path identifiers for renewal.
% For example, in Figure~\ref{fig:tracezip_system}, the emergence of a new value (denoted by the pink dashed rectangle) gives rise to a novel path, i.e., \textit{path 3}.
% \red{In this case, only the hashed representation of this new path needs to be synced.}
% Such a design also benefits the search of spans in \sname, which will be elaborated in the next subsection.

For local fields and the mapping dictionary, it suffices to communicate only the changes to the receiver.
To ensure that the structures at the receiver is not outdated during the transmission of spans, we leverage the batch processor of OpenTelemetry Collector.
It caches the spans sent by SDK until the batch memory is full or its timer expires, instead of immediately forwarding them.
After compressing the spans in the buffer, we will make sure that the SRT and dictionary with updates (if any) have been synced with the receiver side before releasing the data.


% To verify the effectiveness of our proposed Merging Algorithm, we designed two prototype systems, which we refer to as the Static Spans Compressor($\approx 0.5k\mathrm{LOC}$) and the OpenTelemetry Collector instrumented with the Merging Tree Algorithm (hereafter referred to as OTelCol with Compression, $\approx 2k\mathrm{LOC}$, OTel Collector scaffolding codes are excluded). The Static Spans Compressor is a command-line tool used for compressing span records stored in CSV file format. OTelCol with Compression is a middleware based on the secondary development of the OpenTelemetry Collector. The OpenTelemetry Collector supports a plugin system, allowing developers to write their own plugins and build their own OTelCol, and control the data flow of OTelCol by writing configuration files. We have written Receiver and Exporter plugins for OTelCol. The compression algorithm constructs and sends the Merging Tree on the Exporter side, and decompresses the Merging Tree on the Receiver side, while also performing anomaly detection.

% \subsection{OTelCol with Compression}

% In OpenTelemetry, components are divided into three categories: Receiver, Processor and Exporter. The Receiver is the component in OpenTelemetry that ingests telemetry data (such as traces, metrics, and logs) from various sources. It acts as an entry point for data into the OpenTelemetry Collector. The Processor is the component responsible for processing and transforming telemetry data within the OpenTelemetry Collector. It operates between the Receiver and Exporter. The Exporter is the component that sends the processed telemetry data to various backends and storage systems for analysis and visualization. Therefore, we chose to develop the Receiver and Exporter components of the Merging Algorithm based on the OpenTelemetry Collector, use Batch component as processor, build a custom OTelCol that can use the components based on the Merging Algorithm, which we refer to as OTelCol with Compressor. Due to the vendor-agnostic nature of the OpenTelemetry Collector, our developed OTelCol with Compressor can easily integrate with various tracing platforms, such as Zipkin and Jaeger.

% \textbf{Gateway Collector Deploy Pattern} The OpenTelemetry documentation provides a deployment pattern known as the gateway collector, which starts multiple OTelCol instances and utilizes NGINX for load balancing. Considering a production environment with high access pressure on the tracing backend, we use our custom OTelCol with Compressor for each microservice, employing our Merging Algorithm Compressor as the Exporter, sending data to a remote OTelCol with Compressor for decoding operations (referred to as OTelCol with Compressor Gateway in the text). Using the gateway collector deployment pattern allows us to implement load balancing strategies, achieve decentralized trace collection, and further decouple the data collection platform.

% \begin{figure}[htp]
%     \centering
%     \includegraphics[width=8cm]{figures/arch.pdf}
%     \caption{Architecture of OtelCol with Compressor System}
%   \end{figure}

% \textbf{Batch Processor as Buffer} In OTelCol, there is an important component called the Batch Processor. The OpenTelemetry documentation also recommends developers use the Batch Processor. Essentially, the Batch Processor acts as a Spans Buffer. When the SDK collects spans from microservices, the Batch Processor caches the spans sent by the SDK until the batch in the Batch Processor is full or the Batch Processor's timer expires, instead of immediately forwarding them to the next node. We can configure the size of the Batch Processor and the duration of the timer in the configuration file. Since the Merging Algorithm relies on the high redundancy of spans data and the correlation between spans data, it is important to set a reasonable batch size and timeout value.

% \begin{table}[h!]
% \centering
% \begin{tabular}{|p{2cm}|p{6cm}|}
% \hline
% \textbf{Configuration Item} & \textbf{Description} \\
% \hline
% \texttt{sample buffer} & Sets the size of the sample buffer. The sample buffer is used to store and count the occurrences of various attribute values for anomaly detection. It differs from the batch size of the Batch Processor. \\
% \hline
% \texttt{abnormal span name rate} & Abnormal Span Name Rate. This parameter determines which span names with low frequency of occurrence will be considered abnormal. \\
% \hline
% \texttt{abnormal attributes frequency rate} & Abnormal Attributes Frequency Rate. This parameter determines which attribute values with low frequency of occurrence will be considered abnormal. \\
% \hline
% \texttt{enable number abnormal} & Enables numeric anomaly detection when set to true. This parameter controls whether numeric anomaly detection is enabled, where numeric attribute values are checked for anomalies based on their average value and variance. \\
% \hline
% \texttt{threshold rate} & Threshold Rate. This parameter is used to calculate the baseline threshold for inclusion in the Trie. Attribute values with an occurrence count exceeding \texttt{spanNameCount[spanName]/ThresholdRate} will not be included in the Trie compression sequence. \\
% \hline
% \texttt{enable gzip} & Whether to enable gzip compression. If enabled, spans data will be gzip-compressed when sent. \\
% \hline
% \end{tabular}
% \caption{Configuration parameters for exporter}
% \label{exporter}
% \end{table}

% \textbf{Mechanism of Exporter with Compressor} The Exporter with Compressor utilizes the Batch component to process tracing data sent by clients in batches. During the compression process, the attribute values are mapped to a dictionary, and the Attribute Name is synchronized between the Agent Side and the Gateway Side to restore the original attribute names. Additionally, during the compression process, it detects abnormal spans (as described in the Methodology section) and synchronizes the trace\_id of the abnormal spans to the Gateway. Here in Table \ref{exporter}, we present the configuration options of the Exporter and their meanings

% \textbf{Mechanism of Receiver with De-Compressor} The Receiver with De-Compressor will be deployed on the Gateway side. It receives synchronized dictionary data from the Agent Side, receives abnormal detection data in the form of an array of trace IDs from the Agent Side, and receives compressed spans arrays from the Agent Side. It then restores the compressed spans to their original uncompressed form.

% \zb{details of tree/dict synchronization: send path as dict (with an ID) to backend and only update the changed dicts; the base time is periodically updated}

\section{Experimental Evaluation}
\label{sec:exp_eval}

In this section, we present the evaluation of \alias.
We first introduce the experimental settings, including the deployed cloud services, the metric for evaluation, and the baseline methods.
Next, we demonstrate the experimental results, which include the effectiveness of trace compression and the analysis of both efficiency and overhead.

% \zb{Note in online scenarios, there exists a trade-off between the compression ratio and processing overhead at service instances}

\subsection{Experimental Setup}

\subsubsection{Deployed Cloud Systems}
\label{sec:deployed_cloud_systems}

To evaluate the compression performance in a realistic environment, we deploy popular cloud systems and collect their traces using the OpenTelemetry Collector instrumented with \alias.
We serialize the trace data into JSON format and transmit them utilizing HTTP protocols.
The selected services include one microservices benchmark named Train Ticket~\cite{DBLP:journals/tse/ZhouPXSJLD21} and six open-source application components, including gRPC, Apache Kafka, Servlet, MySQL, Redis, MongoDB.

Train Ticket is a railway ticketing application comprising 41 microservices, each responsible for a specific function, such as user authentication, ticket booking, payment processing, and notification.
This benchmark is implemented in different programming languages such as Java, Go, Node.js, Python, etc.
Train Ticket allows a comprehensive evaluation in a multifunctional scenario, which has been widely used in many trace-related topics, including trace sampling~\cite{DBLP:conf/IEEEcloud/ChenJSLZ24}, root cause localization~\cite{DBLP:journals/tse/ZhouPXSJLD21,DBLP:conf/issre/ZhouZPYLLZZD23}, service architecture measurement~\cite{DBLP:conf/sigsoft/0001ZZIGC22}, etc.
In order to replicate a live production environment, we employ Locust~\cite{locust}, an open-loop asynchronous workload generator, to drive the services.
The workloads are directly borrowed from the original work~\cite{DBLP:journals/tse/ZhouPXSJLD21} that introduces the Train Ticket microservices. 

The selected six application components have widespread adoption and play critical roles in modern cloud service architectures.
They represent a diverse cross-section of the technology stack, which play a foundational role in constructing robust, scalable, and high-performance distributed systems.
% from communication protocols like gRPC and web service frameworks such as Apache HTTP and Servlet, to messaging systems like Kafka, and data storage solutions including MySQL, Redis, and MongoDB.
% They are foundational in constructing robust, scalable, and high-performance distributed systems.
We generate workloads that reflect real-world usage patterns common in cloud-native and microservices environments.
For communication protocols like gRPC and web service frameworks such as Apache HTTP, we simulate typical traffic and user interactions.
In messaging systems like Kafka, workloads involve data streaming and message processing, while for data storage solutions like MySQL, Redis, and MongoDB, we focus on common database operations such as read/write transactions.
This approach ensures our findings are applicable and relevant to a wide range of real-world scenarios.
% By focusing on these components, our study captures a broad spectrum of interactions and operations that are common in today's cloud-native and microservices-oriented environments.
% This ensures that our findings are applicable and relevant to a vast array of real-world scenarios.
% \zb{How we generate the workload?}

% To comprehensively evaluate the application components, we generated workloads that closely mimic real-world usage patterns and stress typical operational scenarios in cloud-native and microservices-oriented environments. These workloads were designed to simulate a variety of interactions and operations across different layers of the technology stack.

% For communication protocols like gRPC, we created scenarios involving high-frequency request/response cycles to test latency and throughput. In the case of web service frameworks such as Apache HTTP and Servlet, we generated workloads that emulate typical web traffic, including concurrent user requests, varied content delivery, and session management.

% For messaging systems like Kafka, we crafted workloads that simulate real-time data streaming and message queuing, focusing on both high-volume data ingestion and complex processing pipelines. In terms of data storage solutions, including MySQL, Redis, and MongoDB, the workloads were designed to reflect common database operations, such as read/write transactions, complex queries, and data caching strategies.

% By employing such targeted and realistic workload generation, our study captures the dynamic and multifaceted nature of operations within these critical application components. This approach ensures that our evaluation reflects practical performance and interaction patterns, providing insights that are both applicable and valuable to developers and practitioners working in diverse cloud service architectures.


\subsubsection{Evaluation Metric}

To measure the effectiveness of \alias, we employ \textit{Compression Ratio} (CR) as the metric, which is widely used in the evaluation of existing compression methods for telemetry data~\cite{DBLP:conf/kbse/LiuZHHZL19,DBLP:conf/icse/LiZL024}.
The definition is given below:

\begin{equation*}
    CR=\frac{\mathrm{Original~File~Size}}{\mathrm{Compressed~File~Size}}
\end{equation*}

In each experiment, we run the same set of workloads, so the size of the original file remains constant.
With different compression approaches and configurations, the resulting compressed file may vary in size.
As the file size decreases, a higher level of compression is attained, indicating more effective compression performance.

\subsubsection{Baseline Methods}

Since we are the first to study the problem of trace compression in a live production scenario, there has not been any baseline methods/systems for comparison.
Note that \alias is orthogonal to existing trace sampling techniques, which compress traces via reducing the volume of data collected.
Thus, they cannot be directly compared to \alias.
In this case, we opt for general-purpose compression algorithms which can be used as out-of-the-box tools to compress traces.
Three prevalent and effective algorithms are selected, that is, gzip, bzip2, and lzma.
However, as they are not tailored for trace data, suggesting potential for further performance improvement.
Our goal is to illustrate the additional compression benefits that \alias can provide when applied in conjunction with these standard algorithms.

\begin{table*}[t]
    \centering
    \caption{Performance of Trace Compression on Open-source Cloud Systems}
    \label{tab:dynamic_compression_result}
    \centering
    \footnotesize
    \begin{NiceTabular}{C{1.89cm}|C{0.45cm} C{0.45cm}|C{0.45cm} C{0.45cm}|C{0.45cm} C{0.45cm}|C{0.45cm} C{0.45cm}|C{0.45cm} C{0.45cm}|C{0.45cm} C{0.45cm}|C{0.45cm} C{0.45cm}}
        % \toprule
        \specialrule{0.35mm}{0em}{0em}
        \multirow{2}{*}{} & \multicolumn{2}{c}{\textbf{Train Ticket}} & \multicolumn{2}{c}{\textbf{gRPC}} & \multicolumn{2}{c}{\textbf{Kafka}} & \multicolumn{2}{c}{\textbf{Servlet}} & \multicolumn{2}{c}{\textbf{MySQL}} & \multicolumn{2}{c}{\textbf{Redis}} & \multicolumn{2}{c}{\textbf{MongoDB}}\\
        % \cmidrule{2-15}
        % \midrule
        % \hdashline
        % \cdashline{2-8}
        \cline{2-15}
        & \textbf{Size} & \textbf{CR} & \textbf{Size} & \textbf{CR} & \textbf{Size} & \textbf{CR} & \textbf{Size} & \textbf{CR} & \textbf{Size} & \textbf{CR} & \textbf{Size} & \textbf{CR} & \textbf{Size} & \textbf{CR}\\
        % \midrule
        % \midrule
        \specialrule{0.15mm}{0em}{0em}
        \specialrule{0.15mm}{.1em}{0em}
        % \hdashline[2pt/2pt]
        % \hline
        % \hline
        Raw & 21.0 & 1 & 3.08 & 1 & 2.47 & 1 & 9.36 & 1 & 1.88 & 1 & 2.01  & 1 & 1.10 & 1 \\
        \rowcolor{grey} \alias & 5.19 & 4.05 & 0.58 & 5.31 & 0.627 & 3.94 & 1.45 & 6.46 & 0.30 & 6.27 & 0.33 & 6.09 & 0.21 & 5.24 \\
        % \midrule
        % \hline
        \hdashline[2pt/1pt]
        gzip & 1.93 & 10.90 & 0.163 & 18.91 & 0.143 & 17.27 & 0.506 & 18.50 & 0.112 & 16.85 & 0.084 & 23.93 & 0.065 & 16.92 \\
        \rowcolor{lightgrey} \alias (gzip) & \textbf{1.29} & 16.26 & 0.140 & 22.81 & 0.133 & 18.57 & 0.396 & 23.64 & 0.091 & 20.61 & 0.066 & 30.45 & 0.051 & 21.57 \\
        \rowcolor{grey} improvement & 33.0\% & 1.49x & 17.0\% & 1.17x & 7.0\% & 1.08x & 21.7\% & 1.28x & 18.2\% & 1.22x & 21.4\% & 1.27x & 21.5\% & 1.27x \\
        \hdashline[2pt/1pt]
        bzip2 & 2.41 & 8.71 & 0.135 & 21.96 & 0.124 & 19.92 & 0.421 & 22.23 & 0.097 & 19.38 & 0.056 & 35.89 & 0.054 & 20.37 \\
        \rowcolor{lightgrey} \alias (bzip2) & 1.34 & 15.62 & \textbf{0.128} & 24.00 & 0.116 & 21.29 & 0.365 & 25.64 & 0.087 & 21.56 & 0.044 & 45.68 & 0.048 & 22.92 \\
        \rowcolor{grey} improvement & 30.3\% & 1.75x & 8.5\% & 1.10x & 6.5\% & 1.07x & 13.3\% & 1.15x  & 10.3\% & 1.12x & 21.4\% & 1.27x & 11.1\% & 1.13x \\
        \hdashline[2pt/1pt]
        lzma & 1.93 & 10.89 & 0.174 & 17.67 & 0.128 & 19.29 & 0.487 & 19.22 & 0.121 & 15.61 & 0.064 & 31.41 & 0.065 & 16.92 \\
        \rowcolor{lightgrey} \alias (lzma) & 1.55& 13.54  & 0.146 & 21.14 & 0.012 & 20.58 & 0.412 & 22.72 & 0.097 & 19.48 & 0.054 & 37.2 & 0.055 & 20 \\
        \rowcolor{grey} improvement & 35.7\% & 1.24x & 16.4\% & 1.20x & 9.6\% & 1.07x & 20.1\% & 1.18x & 19.9\% & 1.2x & 17.4\% & 1.19x & 22.7\% & 1.18x \\
        % \midrule
        % \bottomrule
        \specialrule{0.35mm}{0em}{0em}
    \end{NiceTabular}
\end{table*}

\subsection{Effectiveness of Trace Compression}
\label{sec:compression_effectiveness}

\subsubsection{Open-source Cloud Systems}

Table~\ref{tab:dynamic_compression_result} presents the compression performance when collecting traces of the microservices benchmark and cloud applications components.
For each system, we calculate the total size of traces collected, the size after compression, and the resultant compression ratios (CRs) when applying different compression algorithms.
We can see that \alias, as a standalone solution, can achieve CRs ranging from 3.94 to 6.46.
This demonstrates that \alias can remove more amount of redundant information than that shown by our preliminary study in Section~\ref{sec:redundancy_study}.
Traditional compression tools, i.e., gzip, bzip2, and lzma, reduce the file size with a combination of different techniques such as dictionary-based compression and Huffman coding.
Among them, bzip2 generally outperforms the others across most systems, with gzip having the least effectiveness.
In the Train Ticket benchmark, the tools demonstrate the least effective compression with a CR of roughly 10, while on the cloud application components, they deliver a better performance, attaining a comparable CR of around 20.

When working in conjunction with the general-purpose compression algorithms, \alias can provide additional performance gain.
In general, the improvement achieved by \alias when combined with bzip2 is less pronounced than when paired with other algorithms.
This can be attributed to its already superior compression capability, which may reduce the incremental benefits that \alias can offer.
In the case of the microservices benchmark, namely Train Ticket, \alias achieves a more significant performance improvement of 30\%$\sim$35\%.
However, the improvement is less substantial in cloud application components, with Apache Kafka demonstrating an enhancement of less than 10\%.
As mentioned in Section~\ref{sec:redundancy_study}, the traces generated by Kafka include the data from its message queues, rendering the attributes more random.

So far we can make an important observation: compared to application backend components, general-purpose compression algorithms are less effective for processing the traces from Train Ticket, where \alias can offer more substantial improvement.
Our careful investigation reveals the following important insight.
% The spans generally encapsulate two types of data: connectivity details along the call chain and business-specific data.
Based on zero-code instrumentation, the spans collected encapsulate many attributes related to network connectivity (as specified by OpenTelemetry semantic conventions), such as the hostname, IP address, and port of the peer server.
For instance, MongoDB captures details of the requests; Kafka producers log information about their consumers.
Such information provides a comprehensive view of the request's journey across the distributed system.
In production systems, the invocations among different services and components constitute a complex graph, with each node potentially connected to dozens or more instances.
Our experimental environment may not be able to accurately replicate the conditions of the production scenarios.
Consequently, the connectivity information tends to be relatively static, especially for application backend components that operate at infrastructure and platform layer.
In this case, both \alias and traditional algorithms can properly compress such information, reducing the performance gain that \alias can offer.

On the other hand, Train Ticket comprises tens of microservices, which can form a invocation graph with moderate complexity.
Additional, as a service-oriented application, the traces from Train Ticket contain more information related to business logic.
% These two factors contribute to the increased diversity observed in the traces generated by Train Ticket.
These two factors render the traces produced in Train Ticket more diverse, and the compressible information is more scattered.
Traditional compression algorithms are limited to exploiting redundant information within a short sliding window (e.g., 32KB in gzip's Deflate algorithm).
On the other hand, \alias utilizes \sname to continuously capture the redundancy patterns across spans in a \textit{global} manner, which can further reduce the redundancy. 

% We've discovered that the structural correlations present in this architecture are not optimally leveraged by algorithms like Gzip, whereas our compression algorithm, \alias, can identify and compress these relational attributes within the invocation graph. In monolithic applications, the downstream connectivity information tends to be more static. Our experimental environment may not accurately replicate the conditions of the production environment, or it may include random data that are incompressible.

% When the downstream link data are completely identical, Gzip can effectively use its dictionary to identify these repetitions and compress them at a minimal cost. However, when the downstream data are entirely random, neither Gzip nor \alias can achieve effective compression. In such cases, our compression algorithm loses its effectiveness regarding the connectivity part of the information, compressing only the business-related span data, which leads to a reduction in the overall compression ratio.

% This observation underscores the tailored nature of \alias for handling the specific characteristics of trace data from complex, interconnected microservices, providing a more significant advantage over traditional compression tools that may not adapt as effectively to the intricate patterns found in modern distributed systems.


%spans 信息包含两部分,一部分信息是有关于链路下一条的连接信息,另一部分信息是与具体业务相关的信息。semantic convention 规定了很多关于链路下一跳的信息,例如下一条的服务器的 hostname、ip 地址、下一条服务器接受请求的 port 端口等。在实际的生产系统和测试的微服务系统中,所有的实例之间构成一个调用关系图,每一个微服务会与数十个或者更多的实例相连。我们发现这种结构性的关联 Gzip 算法难以很好的利用,而我们的压缩算法可以提取出调用关系图中属性值的关联并加以压缩。在单体应用服务中,链路的下一条信息 **往往是固定的** ,或者是 **无法被压缩的随机信息 **。当链路的下一条数据完全相同时,gzip 能够通过字典发现这种重复,并且用很小的代价进行压缩,当下一跳的链路数据信息完全随机时,gzip 和 \alias都不能够对其进行好的压缩。所以我们的压缩算法失去了关于链路下一条这一部分信息的压缩效果,只能够压缩和业务相关的链路数据,导致了压缩率的减损%

\subsubsection{Production Trace Data}

We also evaluate \alias using production trace data collected from Alibaba. Compared to existing microservices benchmarks, this dataset represents the call graphs of a large-scale deployment of over 20,000 microservices in production clusters. The participating microservices can be categorized into two types: stateless services and stateful services. Stateless services operate independently of any stored state data, whereas stateful services, including databases and systems like Memcached, are required to maintain state information.
There are three types of communication paradigms between pairs of microservices: inter-process communication, remote invocation, and indirect communication. In addition to this diversity, the trace data also exhibit statistical characteristics typical of industry scale. For example, the size of microservice call graphs follows a heavy-tail distribution; there is a non-negligible fraction of hot-spot microservices; and the microservices can form highly dynamic call dependencies at runtime.
This real-world application allows us to examine \alias's efficacy in handling large-scale, complex data, which is crucial for understanding its potential in practical, production-level scenarios.

Figure~\ref{fig:alibaba_compression} illustrates the evaluation results.
The raw size of the trace data used in our experiments is 26.15GB.
The CRs attained by gzip, bzip2, and lzma are 6.55, 7.30, and 8.52, respectively, reducing the data size to 3.99GB, 3.58GB, and 3.07GB.
These CRs are marginally lower than those recorded in the Train Ticket benchmark.
This observation aligns with our finding in Section~\ref{sec:compression_effectiveness}.
That is, relying predominantly on local information, traditional compression algorithms might find it challenging to efficiently compress data characterized by significant diversity and complexity.
This is where the strength of \alias becomes evident.
\alias, with its ability to identify global compression opportunities, enhances the compression performance by 35.1\%, 43.6\%, and 37.8\%, respectively.
For example, the combination of \alias and lzma achieves the optimal CR of 13.69, and compresses the data to a minimal size of 1.91GB.
This result underscores the benefits introduced by \alias, especially when dealing with large-scale and intricate trace data.

% The Static Spans Compressor, which is implemented by Go, is used to compress files in CSV format. The main purpose of implementing this system is to apply the Merging Tree compression algorithm to the Alibaba Cluster Trace 2022 dataset.

% \begin{table}[h!]
% \centering
% \begin{tabular}{|m{2cm}|m{1cm}|m{2cm}|m{2cm}|}
% \hline
% \textbf{Parameter} & \textbf{Type} & \textbf{Default Value} & \textbf{Description} \\ \hline
% \texttt{-path} & \texttt{string} & \texttt{""} & The path to the file to be compressed. \\ \hline
% \texttt{-chunk} & \texttt{int} & \texttt{0} & The size of chunks for compression. \\ \hline
% \texttt{-dirname} & \texttt{string} & \texttt{"output"} & The name of the output directory. \\ \hline
% \texttt{-j} & \texttt{int} & \texttt{1} & The maximum number of CPU cores to use. \\ \hline
% %\texttt{-huffman} & \texttt{bool} & \texttt{false} & Enable Huffman encoding for compression. \\ \hline
% \texttt{-merging} & \texttt{bool} & \texttt{false} & Enable Merging Algorithm. \\ \hline
% \end{tabular}
% \end{table}

% Here, we present the command-line parameters table for the Static Span Compressor. When using the Static Span Compressor, we do not use the Merging Algorithm by default, but only the dictionary compression algorithm, in order to facilitate the comparison of compression effects. Another important parameter is the chunk size. The Static Span Compressor can split the spans records and compress them in units of chunks. As the chunk size increases, the memory space occupied by the Merging Algorithm also increases, but the compression efficiency of the Merging Algorithm correspondingly improves. We will discuss this in detail in the Experiment Evaluation section.Additionally, we can specify the number of CPU cores to be used for the compression operation. Since the Static Spans Compressor compresses spans files, it does not implement anomaly detection algorithms.

\subsection{Performance Overhead}

We examine the overhead of \alias from the perspectives of space complexity and computational efficiency.

\subsubsection{Space Complexity}

At the service side, \alias maintains two types of data structures to capture the redundancy among trace spans and perform compression, namely, a \sname along with its hashed paths and a map for dictionary-based compression.
To prevent impeding the normal execution of the service, it is imperative that they are constrained in size without excessive memory consumption.
To study the space complexity of \alias, we select a microservice in Alibaba trace data with diverse spans and calculate the cumulative size of the three data structures after the compression procedure.
A critical parameter influencing this size is the threshold $\psi$, which dictates the maximum number of distinct values for universal attributes.
An attribute having an exceeding number of values will be moved to the leaf node, becoming a local attribute.
A larger $\psi$ enables \alias to compress a broader spectrum of span fields, enhancing performance but at the cost of a more substantial SRT and mapping structure.
Conversely, a small $\psi$ compromises the effectiveness but with lower space overhead in return.

\begin{figure}[t]
    \centering
    \includegraphics[width=0.6\linewidth]{figures/experiments/alibaba_compression.pdf}
    \caption{Compression on Alibaba Production Traces}
    \label{fig:alibaba_compression}
\end{figure}

Figure~\ref{fig:space_overhead} illustrates the results, where we can see both the data structure size and compression improvement grow with a larger $\psi$.
In the case of $\psi$=1,000, \sname and map together take up only 2.56MB of memory, but the performance gain that \alias achieves is significant, i.e., 33.8\%.
This result underscores \alias's capacity to achieve substantial compression efficiency while maintaining a balanced memory footprint.
However, we notice that the quantities of distinct values that span attributes can have tend to polarize.
This can also be observed in Figure~\ref{fig:space_overhead}.
The performance plateaus even when $\psi$=10,000, meaning there is no attributes whose value size falls in the range of [1,000, 10,000].
Certain attributes (e.g., authentication tokens, DB queries, span ID) might possess a substantially larger set of values compared to others.
Consequently, their inclusion (when $\psi$ is too large) in the \sname could potentially bloat its size.
To address the variability in attribute value distribution and maintain manageable memory usage, we also set a cap on the size of the data structures, e.g., limiting it to 5MB.

\subsubsection{Computational Efficiency}

To ensure \alias can be seamlessly integrated with services, all operations are designed for optimal efficiency.
The time complexity of \alias's core operations is analyzed as follows.
The construction and restructuring of \sname operate with a time complexity of $\mathcal{O}(m)$, where $m$ is the number of span attributes.
In many scenarios, $m$ typically remains below 20.
Other operations, such as hashing, dictionary mapping, and path searching, all have a complexity of $\mathcal{O}(1)$.
Thus, the overall time complexity of \alias is \textit{linear}, making it highly efficient.

\begin{figure}
    \centering
    \includegraphics[width=0.66\linewidth]{figures/experiments/space_overhead.pdf}
    \caption{Performance with Different $\psi$}
    \label{fig:space_overhead}
\end{figure}

To evaluate the efficiency of \alias, we measure the trace collection throughput for the \textit{basic} microservice~\cite{trainticket} of the Train Ticket benchmark.
Specifically, we deployed the instrumented OpenTelemetry system within a container (configured with one core and 1GB of memory) to compress and relay the spans.
The throughput is calculated as the uncompressed size of spans divided by the time token to transmit the traces from the service to the backend.
% This includes the time used for data compression (if \alias or gzip is applied) and JSON serialization.
% Specifically, we evaluate the time taken by different configurations to transmit the traces from the service to the backend, and then divide this duration by the overall size of the dataset.
In the most basic setting, referred to as \textit{Original}, the time is purely the period needed for data transmission and JSON serialization.
When compression techniques such as \alias and gzip are employed, we take into account the additional time required for data compression and decompression.
The results are present in Table~\ref{tab:throughput}.

\begin{table*}[h]
    \centering
    \caption{Performance of Throughput (MB/s) on Train Ticket}
    \label{tab:throughput}
    \centering
    \footnotesize
    \begin{NiceTabular}{C{2.5cm}|C{1cm}|C{1cm}|C{1cm}|C{1cm}}
        \specialrule{0.35mm}{0em}{0em}
        $\psi$ & \textbf{1} & \textbf{10} & \textbf{100} & \textbf{1,000} \\
        \specialrule{0.15mm}{0em}{0em}
        \specialrule{0.15mm}{.1em}{0em}
        Original & 13.98 & 13.57 & 13.78 & 14.05 \\
        \rowcolor{lightgrey} +\alias & 89.34 & 94.35 & \textbf{109.68} & 108.15 \\
        +gzip & 14.65 & 14.27 & 14.35 & 14.02 \\
        \rowcolor{lightgrey}+\alias (gzip) & 60.65 & 63.78 & 68.78 & 68.56 \\
        \specialrule{0.35mm}{0em}{0em}
    \end{NiceTabular}
\end{table*}

It can be seen that upon the integration of \alias, the throughput of trace collection is accelerated by nearly eight times (e.g., from 13.78MB/s to 109.68MB/s).
% This can be attributed to the fact that \alias requires a small amount of time for compression and decompression.
Another interesting observation is that gzip brings little performance gain to the throughput.
One important reason is that gzip compression is performed by the HTTP client library before the data is sent over the network.
Thus, gzip is applied after the data has been serialized into JSON.
As a time-consuming step, JSON serialization constitutes the performance bottleneck.
Moreover, gzip indiscriminately attempts to compress all information, including elements such as trace IDs, span IDs, and authentication tokens, which are inherently resistant to compression.
The (wasted) computational overhead of gzip compression and decompression thus offsets its benefits.
In contrast, \alias can accurately identify the incompressible attributes, i.e., the local fields, and bypass them.
Since \alias is applied before JSON serialization, it substantially reduces the volume of data that needs to be encoded.
% This greatly reduces the amount of data needed to be encoded as \alias is applied before the JSON serialization.
Such a design not only improves the throughput, but also benefits the CPU usage.
Our experiments indicate that the CPU utilization of Tracezip is merely 20\%$\sim$40\% of that in the Original and +gzip settings.


% In contrast, \alias is applied before the JSON serialization, which greatly decreases the data needed to be encoded.
% This also reduces the CPU usage of \alias by more than 60\% when compared to the Original and +gzip settings.

% Such a design choice significantly contributes to \alias's superior efficiency.
% In our lab environment, the processing speed of \alias can reach 100MB/s, while the fastest traditional algorithm, i.e., gzip, is only around 10MB/s.

% \zb{show a figure of CPU percentage and throughput}


\subsection{Threats to Validity}

% We discuss potential threats to the validity of our study.
When evaluating the performance and applicability of \alias, several potential threats to validity must be considered to ensure the robustness and generalizability of our findings.

\textbf{Internal validity}. One of the primary concerns regarding internal validity is the accuracy of our evaluation metrics and the potential biases in our experimental setup.
Real-world cloud services exhibit a vast array of complexities and variations, making it challenging to capture all possible scenarios within a single study.
To address this challenge, we carefully select a diverse set of microservices benchmarks and production trace data from Alibaba that we believe are representative of typical cloud service operations.
These benchmarks and dataset are chosen to reflect common patterns and behaviors observed in real-world applications, thus providing a meaningful context for evaluating \alias's performance.
Additionally, any configuration or tuning of \alias that is specific to these datasets might inadvertently favor our approach, potentially skewing the results.
To mitigate this, we ensure that the benchmarks and dataset are selected and configured independently of \alias's development process.

\textbf{External validity}. External validity pertains to the generalizability of our results to other settings or systems.
Our evaluation of \alias is specifically designed to address the diversity inherent in real-world cloud systems.
We implement and test \alias within the OpenTelemetry Collector framework and evaluate it across a range of cloud environments and backend components, including gRPC, Apache Kafka, MySQL, and others. 
These settings are carefully selected to reflect the variety of systems and technologies commonly used in cloud services, ensuring a comprehensive basis for assessing \alias's effectiveness.
By choosing such a diverse array of environments and applications, we aim to capture the broad spectrum of redundancy patterns and data characteristics found in typical cloud systems.
This approach helps to ensure that our findings are applicable to a wide range of real-world scenarios, demonstrating \alias's capability to perform effectively in diverse and dynamic cloud environments.
% 

\section{Discussion}
\subsection{Giving Human-like Skills to AI} 
This study showed that for one form of humor - Gen-Z style Instagram image captioning humor - our AI-written humor was funnier than GPT's native humorsense, and as funny at the top 5 highest rated Instagram captions. We attribute this to a variety of features we added to the system. First, the visual detail extraction was able to find aspects of the image to poke fun at that were often sharper than GPT's joke target and more similar to the Instagram captions' joke target. Second, the narrative extrapolation step allowed the system to broaden its base of relatable joke targets - moving the focus away from making fun of the literal objects in the image, but using them as metaphors for relatable joke targets like relationship disasters, teamwork breakdowns, and the burden of student loans. This opened more creativity possibilities for joke targets. Lastly, we used an LLM-as-judge to rank the outputs accord to Gen-Z humor taste, thus giving the system some notion of the audience. These skills - detail observation, finding analogous and relatable social situations, and modeling the audience through fine tuning - are all considered somewhat ``human.'' Skills like reasoning and chaining are considered more typical of machines. But this shows that machines might be able to approach these more human skills with the right architecture and training.  


% In this paper, we showed that a model of GPT that is enhanced to have 3 human-like skills used in humor 
% \color{red}
% (observation, sense of story, and in-group knowledge) 
% \color{black}
% outperforms standard GPT-4o. Many other researchers have devised prompting techniques and architectures for improving LLM's reasoning capabilities such as reflection, chain-of-thought, and prompt chaining. However, fewer papers have explored how social skills can enhance LLMs communication abilities. Systems like Generative Agents ~\cite{joon_agents}
% and Character.AI \cite{characterAI} do this to great success. In this paper, we gave VLMs a few simple ``skills'' that were relevant to humor generation and showed that overall, it improved AI's ability to write humor. The focus of this paper was the human evaluation to see whether AI could get closer to parity with most upvoted human captions. However, if even a simple system like this can improve humor, perhaps more sophisticated systems could do better. 

There are many ways to improve the skills in HumorSkills. Building and testing a better Gen Z humor ranking would probably improve the filtering of bad captions. More fine-tuning could improve the breadth of Gen Z slang and references. More narratives and conflicts would expand its vocabulary of relatable situations. Finding ways to automatically collect narratives and conflicts to be applied would accelerate this process. Adding new skills would also be future research. Theories of humor abound. With recent advances in LLM's ability to do long chains of logical reasoning in DeepSeek and GPT-4o, it would be interesting to have AI try to analyze the humor and extract it's own theories or techniques for humor.

One of biggest shortcomings of the captions is that some of them are not logical enough to make sense, but are also not illogical enough to be absurd. These sound like mistakes. As future work, one could test whether an AI-based reflection step could think through the logic of a joke and decide whether it actually made sense or not. 
% The current rating system seems to let some of these by. 

% Other papers have tried fine-tuning b

Further testing or ablation studies could help shed light on which skills are most helpful. However, humor ratings have high variance among raters, and the data required to get statistical significance is often quite high. There may not be an effect of each skill individually - they might only work together. 



\subsection{Implications of Machine with Human-like Social Skills}
Human-like social skills - like humor - are often used for human bonding. If AI can write humor as well as the best people, the AI has the potential to both disingenuously create human bonding ~\cite{diresta2024spammersscammersleverageaigenerated,naaman_opinion} and to augment human's ability to bond~\cite{socialglue}. Either way, this has the potential to change the nature of human trust and communication.
In many ways, this is already happening in other domains. 
ChatGPT and Gmail SmartCompose~\cite{smartcompose} can already rewrite emails to sound more polite and we really are.
AI sales and scams can trick people into giving money to what they think are friend or loved ones in need~\cite{ai_scams}. 
AI has successfully been integrated into Gen Z dating apps that suggest messages to send to potential dates based on both a dating profile (for the opening line) or message history (for continued conversation)~\cite{majic2024rizz}. Many apps attempt this, but the quality of the suggested text sets them apart - the apps that generate more human-like texts have millions of active paying users~\cite{majic2024rizz}.
To some, this potential for disingenuousness is horrifying. Although disingenuous portrayals of oneself for dating purposes far precede the invention of generative AI, there is a possibility that AI will amplify this ability. 


As AI for social, cultural, and personally relevant communication improves, we may need a way to discern genuine from disingenuous communication. There are high-tech ways of doing this, such as making a video of oneself (until AI can do that). There are also low-tech ways of doing this, like talking in person. It would be highly ironic if the advancement of AI drives people to abandon technology, because it could not be trusted to be genuine. 

% For people who adopt these products, a common reason is that the bar for texting banter is so high for Gen Z, that help is appreciated, even when the sentiment is genuine. 

% Even at work, polite communication is socially demanded, and with ever-increasing amounts of communication, the emotional labor of even typing simple pleasantries is tiring. Early generative AI applications like Gmail Smart Compose~\cite{smartcompose} were noted for lowering the burden of writing a polite introductory line in emails. Although these lines are perfunctory and don't necessarily need to be genuine, it makes a difference to readers whether they are there or not. Social effort matters.



% \textit{Although hopefully AI will not force civilization back into a barter economy that necessatiate personal interaction to establish trust.} (LYDIA: TOO MUCH?)


% emotional labor. 

% Also with AI friends like Persona aI?



\section{Limitations}
This study targeted only one form of humor for only one audience:  Gen Z humor Instagram captions. This type of humor tends toward absurdities, which can be easier to generate than something that needs to be logically sound. Being illogically surprising is probably easier than being logical and surprising. Future work would have to test whether similar techniques work on other humor tastes. Some of our techniques, like fine-tuning, would likely work generically for all humor types, but other skills might need to be tried.

The caption humor is difficult, but it is more well-defined than other forms of humor. Caption humor only requires a punchline for a given image (the setup). Other forms of humor like standup comedy and popular humor magazines require generating both the setup and the punchline. A future direction is to explore what additional "skills" are needed to generate jokes with both setup and punchline. 

The humor generated here is for a public audience, but most humor made spontaneously is made for friends, and often users insider knowledge about the friends, their background, and their shared experiences. LLMs would likely struggle to make in-joke humor without a source of inside information to train on.

In our baseline captions, we crafted a simple prompt for GPT-4o to write humor. It is possible that with a better prompt or multiple generations, one could generate similar results. However, that is effectively what the system performs. It might be possible that there are prompts that don't employ any humor skills that can also generate jokes funnier than baseline GPT. Future work should test more prompts - both with and without skills to see if there are approaches other than skills that can enhance LLM humor generation.

% ensuring better generations and more consistent 







\section{Related Work \label{related}}
%
\noindent~{\bf Individual Device Attestation (\ra)}
is an extensively studied topic and numerous schemes have been proposed in the literature. These 
techniques generally fall into three categories: software-based, hardware-based, and hybrid.
%
Given the lack of rich hardware features on embedded platforms, lightweight Software-based \ra 
\cite{li2011viper,seshadri2006scuba,seshadri2004swatt,surminski2021realswatt} is only viable 
for legacy devices with no secure hardware features. 
It uses request-to-response time (between \vrf and \prv) to establish confidence 
in the integrity of the attestation report.
Nonetheless, network limitations (e.g. intermittent connection, network congestion) on \prv introduce noise to the request-to-response 
time, making software-based \ra impractical.

In contrast, hardware-based \ra techniques 
\cite{mccune2010trustvisor,noorman2013sancus,strackx2010efficient,
ling2021secure,chen2019opera,chen2022mage} either (1) embed \prv attestation functionality entirely 
within dedicated hardware, 
or (2) require substantial changes to the underlying hardware to support isolated execution of trusted 
software, e.g., SGX \cite{sgx} or TrustZone \cite{trustzone}.
However, such hardware features are often too complex and costly for low-end devices constrained by 
size, energy, and cost.

Given the limitations of both hardware- and software-based approaches in low-end embedded platforms, 
software/hardware co-design (hybrid) \cite{vrased,arkannezhadida,smart,tytan,nunes2020apex,trustlite} 
has recently emerged as a promising solution. It aims to provide equivalent security guarantees to hardware-based 
\ra while minimizing modifications to the underlying hardware.
The security features employed can be simplified to utilize a ROM or a memory protection unit (MPU).
Current hybrid \ra techniques implement the integrity-ensuring function (e.g., MAC) in software.
They use trusted hardware to control the execution of this software, preventing any violations that may compromise security, such as key leakage, or preemption of unprivileged software.

RealSWATT \cite{surminski2021realswatt} introduces a software-based approach designed for continuous attestation of real-time and multi-core systems, effectively solving the TOCTOU problem.
PISTIS \cite{grisafi2022pistis} is also a software trusted computing architecture enabling memory isolation, remote attestation, and secure update.
SANCUS \cite{noorman2013sancus} and TrustVisor \cite{mccune2010trustvisor} are hardware-based solutions offering attestation service with software module isolation. 
VRASED \cite{vrased} presents a formally verified hybrid RA architecture.
It implements the attestation function in software while employing small trusted hardware to enforce the attestation correctness and access control over the \ra secret key.
IDA \cite{arkannezhadida} proposes a novel hybrid attestation method that enables interrupts even during attestation, enhancing overall system security and flexibility.
Moreover, IDA monitors program memory between attestation requests to prevent TOCTOU attacks.
As previously mentioned in Section \ref{sec:bg}, \rata, \casu, and \garota are hybrid \ra architectures.
The first two provide constant-time computation for attestation requests (heartbeat requests in \casu) regardless of the size of the attested regions.
Meanwhile, the last provides a trusted timer and network that can be preemptively executed by authorized software.
Table \ref{table:comp_ind_att} compares various software, hardware, and hybrid \ra methods.

\noindent~{\bf Network Attestation (\sa)}
%
enables scalable attestation for large groups of interconnected devices. Few prior work \cite{asokan2015seda,ambrosin2016sana,carpent2017lightweight,ibrahim2017seed,ibrahim2016darpa,kohnhauser2017scapi,nunes2019towards,kohnhauser2018salad,petzi2022scraps,kuang2019esdra,abera2019diat} refers to this process as Swarm Attestation; we employ the term Network Attestation to denote the same concept. Table \ref{table:comp_swa_att} shows a comparison with other \sa schemes.

The first scheme, SEDA \cite{asokan2015seda}, employs secure hop-by-hop aggregation of \ra reports. 
Initially, \vrf broadcasts an attestation request to \prv-s. Each \prv attests its children nodes and forwards aggregated \ra reports to its parent. Finally, \vrf verifies only the last \ra reports to assess the status of all \prv-s.
SANA \cite{ambrosin2016sana} extends SEDA with a novel aggregate signature scheme, ensuring low verification overhead with minimal trust anchor.
It partitions \prv-s into subnetworks and aggregates \ra results across the entire network, facilitating public verification by multiple \vrf-s.
LISA \cite{carpent2017lightweight} introduces neighbor-based communication to propagate \ra reports. \prv-s verify \ra reports of other \prv-s before forwarding them to prevent replay attacks, and a quality metric for \sa techniques captures the information from each \prv.
SeED \cite{ibrahim2017seed} enhances the efficiency of SEDA and resilience against DoS attacks by enabling \prv-s to self-initiate \ra.
DARPA \cite{ibrahim2016darpa} detects physically compromised devices by exchanging heartbeat messages among \prv-s to identify compromised or absent devices.
SCAPI \cite{kohnhauser2017scapi} improves DARPA;  it introduces a leader that periodically generates and distributes secret session keys among \prv-s. To receive a new session key, \prv must be authenticated with the previous key.
SAP \cite{nunes2019towards} constructs a formal model encompassing desirable efficiency, soundness, and security notions for \sa. It systematically designs a synchronous attestation protocol compliant with security goals defined by the formal model.
SALAD \cite{kohnhauser2018salad} provides lightweight message aggregation for dynamic networks with intermittent connectivity, distributing \ra proofs among all devices.

SCRAPS \cite{petzi2022scraps} proposes a Pub/Sub network \sa protocol. It involves a proxy verifying \prv’s \ra reports on behalf of actual \vrf.
This proxy is implemented using smart contracts, i.e., untrusted entities hosted on a blockchain.
Once the proxy attests a \prv, \vrf-s can retrieve the \ra evidence from the proxy without trusting the proxy, enabling many-to-many attestation.
This enables many-to-many attestation by allowing \vrf-s to fetch \ra reports from the proxy.
ESDRA \cite{kuang2019esdra} designs a first many-to-one \sa scheme to eliminate fixed \vrf and reduce a single point of failure \vrf risks.
Moreover, the distributed attestation facilitates offering feedback on certain compromised \prv-s, thus suitable for half-dynamic networks.
DIAT \cite{abera2019diat} presents a control-flow attestation scheme for autonomous collaborative systems.
It combines data integrity attestation, modular attestation, and representation of execution paths, enabling efficient run-time attestation in a setting where embedded systems must act as both, \prv and \vrf.



%\vspace{0.2cm}
\section{Conclusion}



This paper introduces \sysname, an AI-assisted system designed to enhance the process of visual blend ideation by leveraging metaphors. 
%Our system utilizes large language models and commonsense knowledge bases to explore objects and their associated attributes, forming metaphorical connections with abstract concepts.
Our system utilizes LLMs and commonsense knowledge bases to explore objects and their associated attributes, forming metaphorical connections with abstract concepts. 
It offers the capability to automatically generate blending proposals based on user selections, facilitating rapid creative realization for verification through the T2I model.
To evaluate the system, we conducted a user study involving 24 participants who had AI experience. The findings demonstrate that \sysname\ has the potential to enhance the creativity of the generated ideation results and enable the expression of abstract concepts more metaphorically.
Additionally, this research offers insights into user preferences regarding visual blend design and potential future approaches for supporting design with generative AI.



% \section*{Data Availability}
% The source code and datasets of this work are publicly available on \url{https://anonymous.4open.science/r/tracezip-C74A}.

\section*{Acknowledgments}
The work described in this paper was supported by the National Natural Science Foundation of China (No. 62402536).
We extend our sincere gratitude to the anonymous reviewers for their insightful feedback.


\bibliographystyle{ACM-Reference-Format}
\bibliography{bibliography}

\end{document}
\endinput