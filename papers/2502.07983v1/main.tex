

%% 
%% Copyright 2007-2024 Elsevier Ltd
%% 
%% This file is part of the 'Elsarticle Bundle'.
%% ---------------------------------------------
%% 
%% It may be distributed under the conditions of the LaTeX Project Public
%% License, either version 1.3 of this license or (at your option) any
%% later version.  The latest version of this license is in
%%    http://www.latex-project.org/lppl.txt
%% and version 1.3 or later is part of all distributions of LaTeX
%% version 1999/12/01 or later.
%% 
%% The list of all files belonging to the 'Elsarticle Bundle' is
%% given in the file `manifest.txt'.
%% 
%% Template article for Elsevier's document class `elsarticle'
%% with numbered style bibliographic references
%% SP 2008/03/01
%% $Id: elsarticle-template-num.tex 249 2024-04-06 10:51:24Z rishi $
%%


\documentclass[preprint, 12pt]{elsarticle}

%% Use the option review to obtain double line spacing
%% \documentclass[authoryear,preprint,review,12pt]{elsarticle}

%% Use the options 1p,twocolumn; 3p; 3p,twocolumn; 5p; or 5p,twocolumn
%% for a journal layout:
%% \documentclass[final,1p,times]{elsarticle}
%% \documentclass[final,1p,times,twocolumn]{elsarticle}
%% \documentclass[final,3p,times]{elsarticle}
%% \documentclass[final,3p,times,twocolumn]{elsarticle}
%% \documentclass[final,5p,times]{elsarticle}
%% \documentclass[final,5p,times,twocolumn]{elsarticle}

%% For including figures, graphicx.sty has been loaded in
%% elsarticle.cls. If you prefer to use the old commands
%% please give \usepackage{epsfig}

%% The amssymb package provides various useful mathematical symbols
\usepackage{amssymb}
%% The amsmath package provides various useful equation environments.
\usepackage{amsmath}
\usepackage{hyperref}
\usepackage{xcolor}
\usepackage{pdfpages}
%\usepackage{selectp}
%\outputonly{0,1,2-22}
%\usepackage[3-22]{pagesel}


% for watermarking
%\usepackage[]{draftwatermark}
% for watermarking
%\SetWatermarkText{Confidential}
%\SetWatermarkScale{2}
%\SetWatermarkAngle{30}
%\SetWatermarkColor{red!25!white}

%\SetWatermarkHorCenter{0.25\paperwidth}
%\SetWatermarkVerCenter{0.3\paperheight}


%% The amsthm package provides extended theorem environments
%% \usepackage{amsthm}

%% The lineno packages adds line numbers. Start line numbering with
%% \begin{linenumbers}, end it with \end{linenumbers}. Or switch it on
%% for the whole article with \linenumbers.
%% \usepackage{lineno}

\journal{Maturitas}

\begin{document}

%TC:ignore
\begin{frontmatter}

%% Title, authors and addresses

%% use the tnoteref command within \title for footnotes;
%% use the tnotetext command for theassociated footnote;
%% use the fnref command within \author or \affiliation for footnotes;
%% use the fntext command for theassociated footnote;
%% use the corref command within \author for corresponding author footnotes;
%% use the cortext command for theassociated footnote;
%% use the ead command for the email address,
%% and the form \ead[url] for the home page:

%\title{\tnoteref{label1}}
%\tnotetext[label1]{}
\author[label1]{Bram van Dijk}
\author[label1]{Armel Lefebvre}
\author[label1]{Marco Spruit\corref{cor1}}

\affiliation[label1]{organization={Leiden University Medical Center},
             %addressline={},
             city={The Hague},
             %postcode={},
             %state={},
             country={The Netherlands}}
%\fntext[label3]{}
\ead{m_spruit@lumc.nl}
 %\ead[url]{home page}
 %\fntext[label2]{}
\cortext[cor1]{Corresponding author at: Health Campus The Hague, Turfmarkt 99, 2511 DP The Hague.}


\title{Welzijn.AI: A Conversational AI System for Monitoring Mental Well-being and a Use Case for Responsible AI Development}


%% use optional labels to link authors explicitly to addresses:
%% \author[label1,label2]{}
%% \affiliation[label1]{organization={},
%%             addressline={},
%%             city={},
%%             postcode={},
%%             state={},
%%             country={}}
%%
%% \affiliation[label2]{organization={},
%%             addressline={},
%%             city={},
%%             postcode={},
%%             state={},
%%             country={}}

%\author{} %% Author name

%% Author affiliation
%\affiliation{organization={},%Department and Organization
%            addressline={}, 
%            city={},
%            postcode={}, 
%            state={},
%            country={}}

%% Abstract
\begin{abstract}

\paragraph{Objectives} We present \texttt{Welzijn.AI} as new digital solution for monitoring mental well-being in the elderly, as a use case illustrating how recent guidelines on responsible Artificial Intelligence can inform \texttt{Welzijn.AI}'s \textit{Technology} and \textit{Value} dimensions.

\paragraph{Study design} Here \textit{Technology} concerns the description of an open, well-documented and interpretable envisioned architecture in light of the system's goals; \textit{Value} concerns stakeholder evaluations of \texttt{Welzijn.AI}. Stakeholders included, among others, informal/professional caregivers, a developer, patient and physician federations, and the elderly. Brief empirical evaluations comprised a SWOT-analysis, co-creation session, and user evaluation of a proof-of-concept implementation of \texttt{Welzijn.AI}.

\paragraph{Main outcome measures}
The SWOT analysis summarises stakeholder evaluations of \texttt{Welzijn.AI} in terms of its Strengths, Weaknesses, Opportunities and Threats. The co-creation session ranks technical, environmental and user-related requirements of \texttt{Welzijn.AI} with the Hundred Dollar Method. User evaluation comprises (dis)agreement on statements targeting \textit{Welzijn.AI}'s main characteristics, and a ranking of desired social characteristics.

\paragraph{Results}
Stakeholders stress different aspects of \texttt{Welzijn.AI}. For example, medical professionals highlight in the SWOT analysis \textit{Welzijn.AI} as the key unlocking an individual's social network, whereas in the co-creation session, more user-related aspects such as demo and practice sessions were emphasised. Stakeholders aligned on the importance of safe data storage and access. The elderly evaluated \texttt{Welzijn.AI}'s accessibility and perceived trust positively, but user comprehensibility and satisfaction negatively.

\paragraph{Conclusions} 
\textit{Welzijn.AI}'s architecture draws mostly on open models, as precondition for explainable language analysis. Also, we identified various stakeholder perspectives useful for researchers developing AI in health and beyond.  

\end{abstract}

%%Graphical abstract
%\begin{graphicalabstract}
%\includegraphics{grabs}
%\end{graphicalabstract}

%%Research highlights
%\begin{highlights}
%\item Research highlight 1
%\item Research highlight 2
%\end{highlights}


%% Keywords
\begin{keyword} mental well-being \sep elderly care \sep artificial intelligence \sep remote monitoring \sep language biomarkers 
%% keywords here, in the form: keyword \sep keyword

%% PACS codes here, in the form: \PACS code \sep code

%% MSC codes here, in the form: \MSC code \sep code
%% or \MSC[2008] code \sep code (2000 is the default)

\end{keyword}



\end{frontmatter}
%TC:endignore

%% Add \usepackage{lineno} before \begin{document} and uncomment 
%% following line to enable line numbers
%% \linenumbers

\section{Introduction}\label{intro}
\noindent The number of individuals above age 60 in the population is increasing and estimated to be one in six by 2030 \citep{WHO2023}. The elderly are vulnerable to mental health disorders, as they often experience social isolation \citep{WHO2021} and loss of cognitive functioning \citep{abdoli2022global}, which are linked to mental health issues \citep{WHO2023}.  Since the onset of mental issues in the elderly often foreshadows hospitalisation \citep{kane2001older}, monitoring mental well-being can support timely intervention to alleviate strain on healthcare systems. Artificial Intelligence (AI) as modern approach for data analysis, can play a key role in monitoring systems by recognising trends and disease onset in data from the elderly  \citep{chen2023digital}.

Earlier work on AI for elderly mental well-being concerned social robots to combat isolation \citep{vsabanovic2015robot}, applications that collect quiz data to monitor mental well-being \citep{valtolina2021design}, and recently also Large Language Models (LLMs) as conversational agents that support mental well-being \citep{alessa2023towards}. Extending the latter work, we introduce \texttt{Welzijn.AI}, a new LLM-driven AI system in its early phase that monitors mental well-being through conversation. We intend \texttt{Welzijn.AI} as an exemplary use case of how recent guidelines for responsible AI can inform the development of AI systems in healthcare for addressing two issues: the need to employ where possible open, documented models and explainable analysis \cite{polevikov2023advancing}, and the need for early stakeholder involvement in development \cite{siala2022shifting}. 

We approach these issues through recommendations from two guidelines for responsible AI: the Innovation Funnel for Valuable AI in Healthcare (IFVAIH) \citep{minVWS2022}, and the Ethics Guidelines for Trustworthy AI (EGTAI) \cite{EU2019trustworthyAI}. These guidelines offer recommendations regarding the \textit{Technology} and \textit{Value} dimensions of AI systems in early development.\footnote{We are aware that both guidelines cover many other dimensions covering ethics, privacy, bias, and sustainability, but here limit ourselves to two.} Technology covers, among other things, an outline of the system's intended architecture, that is preferably open, well-documented, accessible, and interpretable. Value concerns, among other things, early involvement and evaluation of stakeholders' perspectives on the (impact of the) system. 

Our contribution is providing \texttt{Welzijn.AI} as exemplary use case of how AI for healthcare can be shaped by AI guidelines already in an early phase. This is important, given that AI guidelines are hardly put to practice in system development \cite{hagendorff2020ethics}, although awareness of and legislation for responsible AI is increasing. In the remainder of this paper, Section \ref{methods} describes \texttt{Welzijn.AI}'s design and methods for stakeholder evaluations. Section \ref{results} presents results of these evaluations and Section \ref{discussion} reflects on our findings.

\section{Methods}\label{methods}
\noindent In the early Idea and Exploration phases as outlined in the IFVAIH, \textit{Technology} concerns designing a `start architecture' encompassing the envisioned required computational models and data. According to the EGTAI, open, documented, and interpretable resources are to be preferred. We proceed by discussing \texttt{Welzijn.AI}'s goals in Section \ref{sys_goals} and their architectural requirements in Section \ref{sys_arch}. Thereafter we describe \texttt{Welzijn.AI}'s Value dimension in terms of methods for stakeholder evaluations in Section \ref{stakeholder_methods}.

\subsection{Technology: goals}\label{sys_goals}
\noindent \texttt{Welzijn.AI} is a LLM-driven conversational AI system that continuously monitors the user's mental well-being and has the following core goals:

\begin{figure}[t]
\includegraphics[width=\textwidth]{img/maturitas_overview.png}
\caption{\texttt{Welzijn.AI} (green dashed box) comprises modules for \textit{interaction} (blue) and \textit{analysis} (orange). Regarding interaction, speech from the user is converted to text to serve as input for the LLM (speech-to-text module), and LLM text output is mapped back to speech (text-to-speech module). Regarding analysis, speech from the user is analysed in the speech/text analyser module, as well as the text output from the speech-to-text module. This text is also input to the text-to-score module, that maps it to a binned score indicating well-being. All modules operate locally and user data is also stored locally within the protected server environment of the Leiden University Medical Center, to ensure reliable and safe processing of and access to data. Connection and interaction with the app is made via secure API-calls over the internet.}
\label{fig:architecture}
\end{figure}

\begin{enumerate}
    \item Stimulating mental well-being by alleviating loneliness through a conversation partner;
    \item Providing conversations that are engaging and fun to optimise adherence;
    \item Providing explainable analysis of elderly language use through language biomarkers.
\end{enumerate}

\noindent Regarding 1), conversational AI systems have been identified as useful for providing mental health support \citep{koulouri2022chatbots}. Concerning 2), \texttt{Welzijn.AI}'s conversations are structured around validated well-being questionnaires like the EQ-5D-5L, that allow the user to elaborate on e.g. daily activities like hobbies. Regarding 3), the speech/text analyser and text-to-score modules incorporate established work on language biomarkers of mental disorders \citep{figueroa2022automatic, spruit2022exploring}. We discuss the implementation details of these goals in the next section.

\subsection{Technology: architecture}\label{sys_arch}
\noindent Here we describe \texttt{Welzijn.AI}'s start architecture in detail (see for an overview Figure \ref{fig:architecture}).

\textbf{Speech-to-text module --} This module maps user speech to text for further downstream use. We employ Whisper \cite{radford2023robust} as state-of-the-art semi-open\footnote{Here we mean by semi-open models that have their parameters and source code but not their training data publicly available.} speech-to-text model that was already successfully employed with multiple languages and individuals with mental disorders \citep{amorese2023automatic}. 

\textbf{Large Language Model module --} This module employs a state-of-the-art, fully open Dutch LLM optimised for instruction-following such as GEITje \citep{rijgersberg2024geitje_v2}. Prompt engineering is used to structure the interaction around EQ-5D-5L topics such as mobility and mood; similar work has shown that this is a successful approach to retrieve user input on health-related topics, while staying on topic and maintaining context over turns \cite{wei2024leveraging}.

\textbf{Text-to-speech module --} This module maps text output of the LLM module back to speech to ensure natural interaction. Here we employ recent fully open systems like ToucanTTS that in synthesising speech generalise well to many different languages \citep{lux2024massive}.  

\textbf{Speech/text analyser module --} Language biomarkers known to predict mental disorders in elderly speech are analysed in this module. This includes acoustic features like pitch, loudness, pauses, and linguistic features like frequencies of specific lexical items, syntactic complexity, and so on \cite{figueroa2022automatic,spruit2022exploring}. Features are extracted with open libraries like spaCy and librosa \citep{honnibal-johnson:2015:EMNLP,mcfee2015librosa}; the module returns values over time so that (in)formal caregivers can evaluate trends.

\textbf{Text-to-score module --} This module contains a smaller, fully open LLM (custom XLM-RobBERTa model) that maps user input (converted to text) on EQ-5D-5L topics to a binned score that indicates mental well-being, e.g. with 1 indicating mental distress and 5 indicating feeling well. This LLM is grounded in existing work in Natural Language Processing where a LLM infers which answer option (here well-being score) best fits the information provided in a given text (here user input) \citep{wang2024attend}. 

Regarding overall interpretability, this is key in \texttt{Welzijn.AI}'s \textit{analytical} modules (Figure \ref{fig:architecture}), as they produce feature patterns and well-being scores. The speech-text analyser module is feature-based hence interpretable by design. The text-to-score module is a fully open model that can be explained with software like SHAP \cite{lundberg2017unified}, that allows disclosing which words in the user's input are associated with specific well-being scores. Regarding reliable use of \textit{interaction} modules, early testing with the elderly is required to see whether fine-tuning on elderly data is necessary, but this is beyond the scope of this work.        

\subsection{Value: stakeholder evaluations}\label{stakeholder_methods}
\noindent In the Idea and Exploration phases of the IFVAIH, the \textit{Value} dimension concerns stakeholder perspectives on the value and impact of the application, which corresponds to the 'stakeholder participation' recommendation in the EGTAI to involve stakeholders early in development. Below we describe the methodologies of our stakeholder evaluations.

\subsubsection{Expert SWOT-analysis} \noindent This qualitative analysis evaluated \texttt{Welzijn.AI} from the perspective of six expert stakeholders: a General Practitioner (GP), two representatives of an association of Dutch physicians (Hadoks), a clinical psychologist, a representative of the Dutch Patient Federation, and an assistant professor intergenerational care. The system as discussed in \ref{sys_goals} was introduced, after which stakeholders were interviewed individually about the question how technology like \texttt{Welzijn.AI} can support vulnerable elderly. Results summarised in SWOT format (Strengths, Weaknesses, Opportunities and Threats) are presented in Section \ref{results:SWOT}.

\subsubsection{Stakeholder co-creation session} \noindent This session identified stakeholder evaluations of \texttt{Welzijn.AI}'s monitoring and signalling function in its broadest sense: besides tracking mental well-being, the system could also signal longer-than-usual periods without interaction. A representative of the Dutch association for informal caregivers (MantelzorgNL) and a robotics student represented two informal caregivers, a geriatrics specialist represented a professional caregiver, and a computer scientist represented a developer.

Three stakeholders (developer, professional caregiver, and informal caregivers) first individually created a use case about an elderly individual and \texttt{Welzijn.AI}'s monitoring function. Each participant then distributed 100 dollars over the options, i.e. ranked them (Hundred Dollar Method, HDM). For the use case with most dollars, core values were extracted by all stakeholders. Subsequently, value requirements that establish or resolve conflicts in core values were identified. These requirements targeted three aspects of \texttt{Welzijn.AI} and were ranked individually with the HDM: \texttt{Welzijn.AI}'s underlying \textit{technology}, its immediate \textit{environment}, and the \textit{user} itself. Results are discussed in Section \ref{results:co-creation}.

\begin{figure}[t]
\centering
\includegraphics[width=\textwidth]{img/welzijnAI_translated_framed_cropped.png}
\caption{Proof-of-concept static application interface of \texttt{Welzijn.AI}, translated to English. Left and middle panels illustrate how \texttt{Welzijn.AI} was presented to the elderly (Section \ref{elderly_eval}), with real conversations about various EQ-5D-5L topics (mobility, self-care and mood). Right panel shows a work-in-progress dashboard illustrating how text-to-score module score outputs can be tracked over multiple conversations and topics.}
\label{fig:proof-of-concept_ex}
\end{figure}

\subsubsection{User evaluation} \noindent This qualitative analysis presented a proof-of-concept static application interface (Figure \ref{fig:proof-of-concept_ex}) to an expert user panel consisting of 20 elderly individuals $(\bar{x} = 83.2, \sigma = 8.1)$, to identify 1) their user perspective and 2) \texttt{Welzijn.AI}'s desired social characteristics. 

\textbf{User perception} was operationalised with a survey\footnote{The full survey can be found in the supplementary materials.} targeting six characteristics connected to the use of a system in the literature: \textit{accessibility} \citep{diaz2014accessibility}, \textit{comprehensibility} \citep{davis1989perceived}, \textit{intention to use} \citep{heerink2008influence}, \textit{perceived trust} \citep{meng2022trust}, \textit{satisfaction} \citep{gelderman1998relation}, and \textit{anthropomorphism} \citep{li2021machinelike}. 

Accessibility, comprehensibility, intention to use, and perceived trust were measured with various statements concerning design features, how easily understandable the system was, intention to use the system, and reliability of the system, respectively. The elderly indicated (dis)agreement on a 5-point Likert scale. Satisfaction and human-likeness of the system were measured on a semantic differential scale, with various negative-positive pairs like frustrating-satisfying, fake-real and so on on the polar ends of a 5-point scale. 

All responses were converted for ease of interpretation and comparison. For the statements, only (strong) agreement indicated a positive perception regarding a characteristic, and for the semantic differential scale, only the two levels closest to the positive end of a negative-positive pair indicated a positive perception regarding a characteristic.        

\textbf{Desired social characteristics} of interactive AI systems were adapted from work by \cite{heerink2008influence, fong2003survey} and the elderly were asked to rank them. Social characteristics were: 1. responding empathetic (i.e. recognise and express emotions); 2. establishing social relationships (e.g. properly greeting the user); 3. using natural cues in interaction (e.g. use of fillers, emoticons); 4. exhibiting a distinctive personality/character (e.g. chatbot tells something about itself); 5. developed social competencies (e.g. capable of small talk); 6. building a relation of trust with the user (i.e. supporting the user); 7. behaving transparently (e.g. admitting errors). Rankings were converted to scores (7 points for rank 1, 6 for rank 2, etc.) and scores for each characteristic were summed into a final ranking.

\section{Results}\label{results}
\subsection{Expert SWOT-analysis}\label{results:SWOT}
\noindent \textbf{Strengths --} The experts held that an AI system, particularly when embedded in a plush toy/pet that can engage in conversation, can improve the quality of life of an elderly individual. It can reduce feelings of loneliness through interaction: the client may feel it has `something' to talk to. Also, the system may trigger a care response, or instil a sense of purpose or goal in the life of the elderly individual, which may reduce stress levels. Lastly, experts agreed that AI systems designed as social companions can help calming clients that feel restless, particularly when they can physically hold these systems as plush toys/pets when they receive care, which benefits both the elderly and caregivers. 

\textbf{Weaknesses --} Experts mentioned the possible dependence on AI systems like \texttt{Welzijn.AI}: they should not be considered replacements for genuine social contact, but as tools that can help reduce loneliness. Also, given that for monitoring other individuals must access the collected data, another weakness is inappropriate use of data by others. In addition, in monitoring it can be hard to decide on the boundary between small deviations from routines, and larger deviations that may require professional care.

\textbf{Opportunities --} Experts see value in AI systems signalling informal caregivers, as they know the elderly individual well, and can best decide whether follow-up action is needed. AI systems thus allow employing the social environment of an individual to alleviate strain on professional caregivers. This may also provide medical care avoiders an accessible way for receiving care from informal caregivers. Experts note however that employing a social environment would first require mapping it.

\textbf{Threats --} Experts mention privacy and safety issues. Concerning privacy, an issue is where the conversations that the system processes are stored, and who has access to them. Concerning safety, the system should not infer wrong conclusions from data. This implies that proper deployment of a system like \texttt{Welzijn.AI} for a specific individual also requires knowing the individual's routines.      

\subsection{Stakeholder co-creation session}\label{results:co-creation}
\noindent The use case allocated most dollars staged an elderly woman that interacts daily with \texttt{Welzijn.AI} at a fixed time after a stroll.\footnote{All use cases can be found in the supplementary materials.} One day she does this much later because she was afraid to use the stairs, causing the system to alert a third person. Stakeholders extracted consensual core values for the design of \texttt{Welzijn.AI} from this use case, and they were \textit{autonomy}, \textit{privacy}, \textit{living independently}, and \textit{consent/voluntary use}. Hereafter, value requirements were identified and allocated dollars by stakeholders individually (for an overview see Figure \ref{fig:cocreate}).  

\begin{figure}[t]
\includegraphics[width=\textwidth]{img/patientenfederatie.png}
\caption{Value requirements of \texttt{Welzijn.AI} regarding its technical, environmental, and user-related aspects, ranked with the HDM. Requirements are listed as follows. \textbf{Technical:} (T1) periodically asking consent, (T2) available on-off button, (T3) robust, portable design, (T4) automatic shutdown in case of multiple persons present, (T5) long battery life, (T6) limited human-likeness, (T7) gradual flow of the conversation, (T8) safe data storage, (T9) speech recognition. \textbf{Environment:} (E1) available WiFi, (E2) help desk, (E3) power outlet, (E4) social media functionality, (E5) promotional campaign, (E6) agreements on access to data. \textbf{User:} (U1) education for caregivers and user, (U2) demo, test and practice session, (U3) creating awareness to reduce stigma on using assistive technology.}
\label{fig:cocreate}
\end{figure}

\subsubsection{Consensus}
\noindent We first discuss requirements that all stakeholders valued more than zero dollar and thus all agreed on. These were a gradual flow of the conversation (T7), safe data storage (T8), help desk (E2), and demo, test and practice session (U2).\footnote{As the number of \textit{user} requirements is small, consensus is more likely, so the requirement with the highest aggregate ranking (U2) was chosen as consensual one here.} Some of these requirements clearly link to core values: safe data storage links to \textit{privacy}; a demo, test and practice session supports rendering the elderly \textit{autonomous} users, that as a result may also live \textit{autonomously} and \textit{independently} for longer; a gradual flow of the conversation as a design principle, and a help desk for troubleshooting in starting to use \texttt{Welzijn.AI} may also support \textit{autonomy} and \textit{living independently}.

\subsubsection{Disparity}
\noindent We also discuss some disparity in dollar allocation for different stakeholders. For example, \textbf{informal caregiver 2} allocated periodically asking consent (T1) zero dollar (contributes to \textit{privacy} and \textit{consent/voluntary use}), which the developer and professional caregiver gave much more. Interestingly, informal caregivers diverge themselves on many requirements, such as a promotional campaign for \texttt{Welzijn.AI} (E5) (less clear relation to core values), but sometimes also align, for example on robust, portable design (T3) (contributes to \textit{living independently}). Informal caregivers find requirements that contribute to \textit{autonomy} and \textit{living independently} much more important than other stakeholders, for example robust, portable design (T3) and speech recognition (T9).   

The \textbf{developer} allocated some requirements zero dollar that the professional and/or informal caregiver(s) give much more: for example the on-off button (T2) (contributes to \textit{autonomy, consent/voluntary use, privacy}), a robust, portable design (T3) (contributes to \textit{living independently}), and agreements on access to data (E6) (contributes to \textit{privacy}). The developer allocated much more dollars, relative to other stakeholders, to technical requirements like safe data storage (T8) (contributes to \textit{privacy}) and periodically asking consent (T1) (contributes to \textit{privacy} and \textit{consent/voluntary use}). It is striking that the developer overall allocated more dollars to a smaller number of technical and environmental requirements than other stakeholders, i.e. prioritises more.

The \textbf{professional caregiver}, unlike other stakeholders, allocated dollars to the automatic shutdown in case of multiple persons present (T4) (contributes to \textit{privacy}), but allocates zero dollar to a robust, portable design (T3) (contributes to \textit{autonomy} and \textit{living independently}), which both informal caregivers give much more. The professional caregiver often aligns with one or more informal caregivers, for example on agreements on access to data (E6) (contributes to \textit{privacy}), and demo, test and practice session (U2) (contributes to \textit{autonomy} and \textit{living independently}).  

\subsection{User evaluation}\label{elderly_eval}
\noindent Here we discuss 1) the elderly perception of \texttt{Welzijn.AI} and 2) their ranking of \texttt{Welzijn.AI}’s desired social characteristics.

\begin{table}[t] \small
\begin{center}
\begin{tabular}{c c c}
    \hline
    \textbf{Characteristic} & \textbf{Positive perception} & \textbf{Num. statements} \\   
    \hline
    Accessibility & 64\% & 160 \\
    Perceived trust & 59\% & 140 \\
    Human-likeness & 54\% & 100 \\
    Intention to use & 52\% & 80 \\
    Comprehensibility & 44\% & 160 \\
    Satisfaction & 42\% & 100 \\
    \hline
\end{tabular}
\end{center}
\caption{Positive perception on statements evaluating six characteristics of \texttt{Welzijn.AI}'s proof-of-concept interface (Figure \ref{fig:proof-of-concept_ex}). Num. statements refers to the total number of statements evaluated by 20 elderly individuals on a specific characteristic.}
\label{tab:user_eval}
\end{table}

\begin{table}[t] \small
\begin{center}
\begin{tabular}{c c c }
    \hline
    \textbf{Rank} & \textbf{Characteristic} & \textbf{Total points} \\   
    \hline
    1. & Responding empathetic & 101 \\
    2. & Exhibiting distinctive personality/character & 88 \\
    3. & Building a relation of trust & 86 \\
    4. & Developed social competencies & 82 \\ 
    5. & Establishing social relationships & 77  \\
    6. & Behaving transparantly & 65 \\
    7. & Using natural cues & 60 \\
    \hline
\end{tabular}
\end{center}
\caption{Ranking of desired social characteristics of Welzijn.AI by 20 elderly individuals.}
\label{tab:social_chars}
\end{table}

\subsubsection{User perception}
\noindent Our survey results are given in Table \ref{tab:user_eval}. Regarding \textbf{accessibility}, 64\% of the elderly responses evaluated the proof-of-concept app as accessible. Elderly were positive about font size, contrast and language use, but negative about understanding the meaning and use of the icons. Concerning \textbf{perceived trust}, 59\% of the responses indicated trust in the system. Elderly rated the system as reliable, and trusted the system's output, although they also thought that a human in the loop was required for proper functioning. For \textbf{human-likeness}, 54\% of the responses evaluated the system as human-like. Elderly experienced the system as natural, and more human than machine-like, but did not think it was conscious. Concerning \textbf{intention to use}, 52\% of the responses showed intention to use the app. The elderly planned to use the app also without recommendation by a doctor, but disagreed recommending it to others. Regarding \textbf{comprehensibility}, 44\% of the responses evaluated the app as comprehensible. Elderly perceived the app as complex, and taking time to understand, although they also thought the app was useful. For \textbf{satisfaction}, 42\% of the responses displayed satisfaction with the system. Most elderly displayed mixed feelings and indifference on whether the app was fun, hard to use, or relevant for them.

\subsubsection{Desired social characteristics}
\noindent The ranking of desired social characteristics for \texttt{Welzijn.AI} is given in Table \ref{tab:social_chars}. Being empathetic is clearly the most important characteristic, whereas the elderly consider the use of natural cues least important. Looking at the three most important characteristics, we can say that the elderly are looking for AI systems that can recognise and express emotions, that do not provide too generic or bland interaction but maintain some personality, and a system that is able to instil a sense of trust. Interestingly, they value this more than transparent model behaviour (e.g. being able to state what it does with the data), and showing natural behaviour (e.g. using fillers in interaction). 

\section{Discussion}\label{discussion}
\noindent This paper illustrated how the early development of AI systems in healthcare can be informed by recent guidelines on responsible AI. We discussed \texttt{Welzijn.AI}'s \textit{Technology} and \textit{Value} dimensions as illustrations of how this works in practice. 

Regarding \textit{Technology}, most of \texttt{Welzijn.AI}'s interaction modules are (semi-)open, though their generalisation capacity for elderly language remains to be evaluated. Analytical modules are fully open, but work remains to make the text-to-score module more interpretable.

Regarding \textit{Value}, stakeholders highlighted different aspects of \texttt{Welzijn.AI}. Experts in the SWOT analysis emphasised \texttt{Welzijn.AI}'s potential for leveraging the individual's social network, and reducing stress during receiving care. In the co-creation session we saw some divergence between stakeholders, where informal caregivers stressed requirements establishing elderly \textit{autonomy} and \textit{living independently}, but the developer technical aspects ensuring \textit{privacy}. Stakeholders did align in the SWOT analysis and co-creation session on the importance of data access and storage. These findings align with work suggesting that incorporating all stakeholder perspectives remains a key challenge for digital systems in health, but critical to enhance application functionality and efficacy \cite{sedhom2021mobile, zainal2023usability}.

The evaluation of \texttt{Welzijn.AI}'s proof-of-concept implementation with the elderly indicated that the majority of statements evaluated system accessibility and perceived trust positively, but evaluated system comprehensibility and satisfaction negatively. In addition, characteristics evaluated positively had only small majorities of positive evaluations, implying that for AI applications for the elderly there is a mismatch between what developers and the elderly find a comprehensible and satisfying application. This finding resonates with the emphasis on user involvement in both the EGTAI and IFVAIH, and with work arguing that many digital health applications are not being validated with a user-centred perspective in mind \cite{mathews2019digital}. 

We believe that our results are helpful for developers and researchers working on AI systems in and beyond healthcare. For \texttt{Welzijn.AI} a working prototype primarily for demo purposes has been developed, that will be available online, but for which currently a demo video can be consulted, that may inspire future work on AI systems for elderly well-being.\footnote{The video (and later also application) can be found at \url{https://osf.io/ezyna/}.}

%TC:ignore
\section{Contributions}
\noindent Bram van Dijk contributed to editing and reviewing the article; Armel Lefebvre contributed to editing and reviewing the article; Marco Spruit contributed to conceptualising the article, supervising stakeholder evaluations, supervising editing of the article, and final review. All authors saw and approved the final version and no other person made a substantial contribution to the paper.   
%TC:endignore

%TC:ignore
\section{Acknowledgments}\label{acknowledgments}
\noindent Many professionals and students have contributed already to the development of \texttt{Welzijn.AI}. For valuable input and/or work we thank Runda Wang and Casper van Wordragen (Leiden Institute of Advanced Computer Science), Danielle Matser and Vera Vroegop (Zorginstituut Nederland), Sytske van Bruggen and Anne-Wil Eewold (Hadoks), Els van de Boogart and Ildikó Vajda (Patiëntenfederatie Nederland), Mirjam de Haas and Melvin Alomerović (Hogeschool Utrecht), Wieke Hengeveld (MantelzorgNL), Nynke Slagboom and Hedwig Vos (Leiden University Medical Center), Edwin de Beurs (Leiden University), twenty elderly individuals from various Dutch nursing homes, 15 Dutch AI Parade Dialogue participants, and 11 software engineering students (Leiden Institute of Advanced Computer Science). 

The co-creation session took place within the project `Ethical appraisal of AI-driven MedTech for Trustworthy AI – development of a participatory, patient-centric methodology and toolkit', financed by ZonMw, grant number 10580012210019 of the program HTA Methodology 2021-2024.
%TC:endignore

%% If you have bib database file and want bibtex to generate the
%% bibitems, please use
%%
\bibliographystyle{elsarticle-num} 
\bibliography{natbib}

%% else use the following coding to input the bibitems directly in the
%% TeX file.

%% Refer following link for more details about bibliography and citations.
%% https://en.wikibooks.org/wiki/LaTeX/Bibliography_Management

\includepdf[pages=-]{supplementary_materials_welzijnAI_maturitas_arxiv.pdf}
\end{document}



\endinput
%%
%% End of file `elsarticle-template-num.tex'.
