%%%%%%%% ICML 2025 EXAMPLE LATEX SUBMISSION FILE %%%%%%%%%%%%%%%%%

\documentclass{article}

% Recommended, but optional, packages for figures and better typesetting:
\usepackage{microtype}
\usepackage{graphicx}
\usepackage{subfigure}
\usepackage{booktabs} % for professional tables

% hyperref makes hyperlinks in the resulting PDF.
% If your build breaks (sometimes temporarily if a hyperlink spans a page)
% please comment out the following usepackage line and replace
% \usepackage{icml2025} with \usepackage[nohyperref]{icml2025} above.
\usepackage{hyperref}
\usepackage{xspace}
\usepackage{enumitem}
%\usepackage{algorithm}
%\usepackage{algpseudocode}


% Attempt to make hyperref and algorithmic work together better:
%\newcommand{\theHalgorithm}{\arabic{algorithm}}

% Use the following line for the initial blind version submitted for review:
%\usepackage{icml2025}

% If accepted, instead use the following line for the camera-ready submission:
\usepackage[accepted]{icml2025}

% For theorems and such
\usepackage{amsmath, bm}
\usepackage{amssymb}
\usepackage{mathtools}
\usepackage{amsthm}



\usepackage{graphicx}      % For including graphics
\usepackage{subcaption}    % For sub-figures and sub-captions

% if you use cleveref..
\usepackage[capitalize,noabbrev]{cleveref}

%%%%%%%%%%%%%%%%%%%%%%%%%%%%%%%%
% THEOREMS
%%%%%%%%%%%%%%%%%%%%%%%%%%%%%%%%
\theoremstyle{plain}
\newtheorem{theorem}{Theorem}[section]
\newtheorem{proposition}[theorem]{Proposition}
\newtheorem{lemma}[theorem]{Lemma}
\newtheorem{corollary}[theorem]{Corollary}
\theoremstyle{definition}
\newtheorem{definition}[theorem]{Definition}
\newtheorem{assumption}[theorem]{Assumption}
\theoremstyle{remark}
\newtheorem{remark}[theorem]{Remark}

% Todonotes is useful during development; simply uncomment the next line
%    and comment out the line below the next line to turn off comments
%\usepackage[disable,textsize=tiny]{todonotes}
\usepackage[textsize=tiny]{todonotes}

\newcommand{\stap}{\bS_{\rm TAP}}
\newcommand{\slamp}{\bS_{\rm LAMP}}
\newcommand{\gout}{\bg_{\rm out}}

\newcommand{\Py}{\mathsf{Z}}
\newcommand{\I}{\mathbb{I}}
\newcommand{\Zout}{\Py}
\newcommand{\dgout}{\bG}

\newcommand{\bSigma}{\boldsymbol{\Sigma}}

% Probability
\renewcommand{\P}{\mathbb{P}}
\newcommand{\E}{\mathbb{E}}
\newcommand{\Var}{\text{Var}}
\newcommand{\Cov}{\mathrm{Cov}}
\newcommand{\cN}{\mathcal{N}}

% Sets
\newcommand{\Z}{\mathbb{Z}}
\newcommand{\R}{\mathbb{R}}
\newcommand{\C}{\mathbb{C}}
\newcommand{\N}{\mathbb{N}}
\renewcommand{\S}{\mathbb{S}}
\def\ball{{\mathsf B}}

% Variables
\newcommand{\eps}{\varepsilon} 
\newcommand{\vphi}{\varphi}
\def\id{{\mathbf I}}


% Math
\renewcommand{\d}{\textup{d}}
\renewcommand{\l}{\vert}
\newcommand{\dl}{\Vert}
\newcommand{\<}{\langle}
\renewcommand{\>}{\rangle}
\newcommand{\sign}{\text{sign}}
\newcommand{\diag}{\text{diag}}
%\newcommand{\tr}{\text{tr}}
%\newcommand{\op}{{\rm op}}
\newcommand{\ones}{\bm{1}}
\newcommand{\what}{\widehat}
%\newcommand{\grad}{\boldsymbol{\nabla}}
\def\sT{{\mathsf T}}
\def\bzero{{\boldsymbol 0}}
\newcommand{\bomega}{\boldsymbol{\omega}}
\newcommand{\bOmega}{\boldsymbol{\Omega}}
\newcommand{\flatten}{\operatorname{flat}}
\newcommand{\bcT}{\boldsymbol{\mathcal{T}}}


\DeclareMathOperator*{\argmin}{arg\,min}
\DeclareMathOperator*{\argmax}{arg\,max}
\DeclareMathOperator*{\argsup}{arg\,sup}
\DeclareMathOperator*{\arginf}{arg\,inf}
\newcommand{\eqnd}{\, {\buildrel d \over =} \,} 
\newcommand{\eqndef}{\mathrel{\mathop:}=}
\def\doteq{{\stackrel{\cdot}{=}}}
\newcommand{\goto}{\longrightarrow}
\newcommand{\gotod}{\buildrel d \over \longrightarrow} 
\newcommand{\gotoas}{\buildrel a.s. \over \longrightarrow} 
\def\simiid{{\stackrel{i.i.d.}{\sim}}}


% Notations 
\newcommand{\notate}[1]{\textcolor{red}{\textbf{[#1]}}}
\newcommand{\cc}[1]{\textcolor{blue}{\textbf{[CC:#1]}}}
\newcommand{\yw}[1]{\textcolor{pink}{\textbf{[YW:#1]}}}
\newcommand{\mc}[1]{\mathcal{#1}}
\newcommand{\mb}[1]{\mathbf{#1}}


% Theorem
\newtheorem{question}{Question}
\newtheorem{property}{Property}
\newtheorem{objective}{Objective}
\newtheorem{claim}{Claim}
\newtheorem{example}{Example}



%\usepackage[inline]{showlabels}

\DeclareSymbolFont{rsfs}{U}{rsfs}{m}{n}
\DeclareSymbolFontAlphabet{\mathscrsfs}{rsfs}



% Bold symbols
\def\bA{{\boldsymbol A}}
\def\bB{{\boldsymbol B}}
\def\bC{{\boldsymbol C}}
\def\bD{{\boldsymbol D}}
\def\bE{{\boldsymbol E}}
\def\bF{{\boldsymbol F}}
\def\bG{{\boldsymbol G}}

\def\bH{{\boldsymbol H}}
\def\bI{{\boldsymbol I}}
\def\bJ{{\boldsymbol J}}
\def\bK{{\boldsymbol K}}
\def\bL{{\boldsymbol L}}
\def\bM{{\boldsymbol M}}
\def\bN{{\boldsymbol N}}
\def\bO{{\boldsymbol O}}
\def\bP{{\boldsymbol P}}
\def\bQ{{\boldsymbol Q}}
\def\bR{{\boldsymbol R}}
\def\bS{{\boldsymbol S}}
\def\bT{{\boldsymbol T}}
\def\bU{{\boldsymbol U}}
\def\bV{{\boldsymbol V}}
\def\bW{{\boldsymbol W}}
\def\bX{{\boldsymbol X}}
\def\bY{{\boldsymbol Y}}
\def\bZ{{\boldsymbol Z}}

\def\ba{{\boldsymbol a}}
\def\bb{{\boldsymbol b}}
\def\be{{\boldsymbol e}}
\def\boldf{{\boldsymbol f}}
\def\bg{{\boldsymbol g}}
\def\bh{{\boldsymbol h}}
\def\bi{{\boldsymbol i}}
\def\bj{{\boldsymbol j}}
\def\bk{{\boldsymbol k}}
\def\bt{{\boldsymbol t}}
\def\bu{{\boldsymbol u}}
\def\bv{{\boldsymbol v}}
\def\bw{{\boldsymbol w}}
\def\bx{{\boldsymbol x}}
\def\by{{\boldsymbol y}}
\def\bz{{\boldsymbol z}}

\def\bmu{{\boldsymbol \mu}}
\def\bbeta{{\boldsymbol \beta}}
\def\bdelta{{\boldsymbol\delta}}
\def\beps{{\boldsymbol \eps}}
\def\blambda{{\boldsymbol \lambda}}
\def\bpsi{{\boldsymbol \psi}}
\def\bphi{{\boldsymbol \phi}}
\def\btheta{{\boldsymbol \theta}}
\def\bvphi{{\boldsymbol \vphi}}
\def\bxi{{\boldsymbol \xi}}

\def\bDelta{{\boldsymbol \Delta}}
\def\bLambda{{\boldsymbol \Lambda}}
\def\bPsi{{\boldsymbol \Psi}}
\def\bPhi{{\boldsymbol \Phi}}
\def\bSigma{{\boldsymbol \Sigma}}
\def\bTheta{{\boldsymbol \Theta}}

\def\bfzero{{\boldsymbol 0}}
\def\bfone{{\boldsymbol 1}}
\def\bPi{{\boldsymbol \Pi}}


% Symbols with hat
\def\hba{{\hat {\boldsymbol a}}}
\def\hf{{\hat f}}
\def\ha{{\hat a}}
\def\tcT{\widetilde{\mathcal T}}
\def\tK{\widetilde{K}}


\def\cR{\mathcal{R}}
\def\test{{\rm test}}
\def\train{{\rm train}}
\def\CV{\text{CV}}
\def\GCV{\text{GCV}}
\def\sfs{{\sf s}}

% rm symbols
\def\spn{{\rm span}}
\def\supp{{\rm supp}}
\def\Easy{{\rm E}}
\def\Hard{{\rm H}}
\def\post{{\rm post}}
\def\pre{{\rm pre}}
\def\Rot{{\rm Rot}}
\def\Sft{{\rm Sft}}
\def\endd{{\rm end}}
\def\KR{{\rm KR}}
\def\bbHe{{\rm He}}
\def\sk{{\rm sk}}
\def\de{{\rm d}}
\def\Tr{{\rm Tr}}
\def\lin{{\rm lin}}
\def\res{{\rm res}}
\def\degzero{{\rm deg0}}
\def\degone{{\rm deg1}}
\def\Poly{{\rm Poly}}
\def\Poly{{\rm Poly}}
\def\Coeff{{\rm Coeff}}
\def\de{{\rm d}}
\def\Unif{{\rm Unif}}
\def\lin{{\rm lin}}
\def\res{{\rm res}}
\def\RF{{\rm RF}}
\def\NT{{\rm NT}}
\def\Cyc{{\rm Cyc}}
\def\RC{{\rm RC}}
\def\kernel{\rm Ker}
\def\image{{\rm Im}}
\def\Easy{{\rm E}}
\def\Hard{{\rm H}}
\def\post{{\rm post}}
\def\pre{{\rm pre}}
\def\Rot{{\rm Rot}}
\def\Sft{{\rm Sft}}
\def\ddiag{{\rm ddiag}}
\def\KR{{\rm KR}}
\def\RR{{\rm RR}}
\def\bbHe{{\rm He}}
\def\eff{{\rm eff}}

\def\spn{{\rm span}}


%mathcal symbols
\def\cV{{\mathcal V}}
\def\cG{{\mathcal G}}
\def\cO{{\mathcal O}}
\def\cP{{\mathcal P}}
\def\cW{{\mathcal W}}
\def\cT{{\mathcal T}}
\def\cC{{\mathcal C}}
\def\cQ{{\mathcal Q}}
\def\cL{{\mathcal L}}
\def\cF{{\mathcal F}}
\def\cE{{\mathcal E}}
\def\cS{{\mathcal S}}
\def\cI{{\mathcal I}}
\def\cV{{\mathcal V}}
\def\cG{{\mathcal G}}
\def\cO{{\mathcal O}}
\def\cP{{\mathcal P}}
\def\cW{{\mathcal W}}
\def\cT{{\mathcal T}}
\def\cH{{\mathcal H}}
\def\cA{{\mathcal A}}


\def\tbA{\Tilde \bA}

%mathbb mathsf sf symbols
\def\K{{\mathbb K}}
\def\H{{\mathbb H}}
\def\T{{\mathbb T}}
\def\bbV{{\mathbb V}}
\def\W{{\mathbb W}}
\def\sM{{\mathsf M}}
\def\sW{{\mathsf W}}
\def\Unif{{\sf Unif}}
\def\normal{{\sf N}}
\def\proj{{\mathsf P}}
\def\ik{{\mathsf k}}
\def\il{{\mathsf l}}
\def\sM{{\sf M}}
\def\RKHS{{\sf RKHS}}
\def\RF{{\sf RF}}
\def\NT{{\sf NT}}
\def\NN{{\sf NN}}
\def\reals{{\mathbb R}}
\def\integers{{\mathbb Z}}
\def\naturals{{\mathbb N}}
\def\Top{{\mathbb T}}
\def\Kop{{\mathbb K}}
\def\Aop{{\mathbb A}}
\def\normal{{\sf N}}
\def\proj{{\mathsf P}}
\def\bbV{{\mathbb V}}
\def\sW{{\mathsf W}}
\def\sM{{\mathsf M}}
\def\T{{\mathbb T}}
\def\K{{\mathbb K}}
\def\H{{\mathbb H}}
\def\Unif{{\sf Unif}}
\def\normal{{\sf N}}
\def\Uop{{\mathbb U}}
\def\Hop{{\mathbb H}}
\def\Sop{{\mathbb S}}
\def\proj{{\mathsf P}}
\def\ik{{\mathsf k}}
\def\il{{\mathsf l}}
\def\sM{{\sf M}}
\def\RKHS{{\sf RKHS}}
\def\RF{{\sf RF}}
\def\NT{{\sf NT}}
\def\NN{{\sf NN}}
\def\reals{{\mathbb R}}
\def\integers{{\mathbb Z}}
\def\naturals{{\mathbb N}}
\def\proj{{\mathsf P}}
\def\Hop{{\mathbb H}}
\def\Uop{{\mathbb U}}
\def\App{{\rm App}}
\def\sU{{\sf U}}
\def\sV{{\sf V}}
\def\sfp{{\sf p}}
\def\tcE{\widetilde{\cE}}
\def\tmu{\widetilde  \mu}
\def\tbD{\widetilde{\bD}}




\def\stest{\mbox{\tiny\rm test}}

\def\seff{\mbox{\tiny\rm eff}}

\def\Ker{K}
\def\tKer{\tilde{K}}
\def\oKop{\overline{{\mathbb K}}}
\def\oKer{\overline{K}}
\def\ocV{\overline{{\mathcal V}}}

\def\th{\tilde{h}}
\def\tQ{\tilde{Q}}
\def\tsigma{\Tilde{\sigma}}


\def\hba{{\hat {\boldsymbol a}}}
\def\hf{{\hat f}}
\def\hy{{\hat y}}
\def\hU{\widehat{U}}
\def\hUop{\widehat{\mathbb U}}
\def\tbDelta{\widetilde{\bDelta}}


\def\tcT{\widetilde{\mathcal T}}

\def\Cyc{{\rm Cyc}}
\def\inv{{\rm inv}}


\def\cE{{\mathcal E}}
\def\cD{{\mathcal D}}
\def\cX{{\mathcal X}}
\def\cF{{\mathcal F}}
\def\cS{{\mathcal S}}
\def\cI{{\mathcal I}}



\def\He{{\rm He}}
\def\lin{{\rm lin}}
\def\res{{\rm res}}
\def\degzero{{\rm deg0}}
\def\degone{{\rm deg1}}
\def\Poly{{\rm Poly}}
\def\Coeff{{\rm Coeff}}
\def\de{{\rm d}}
\def\Unif{{\rm Unif}}
\def\RF{{\rm RF}}
\def\NT{{\rm NT}}
\def\Cyc{{\rm Cyc}}
\def\RC{{\rm RC}}

\def\tK{\widetilde{K}}
\def\stest{\mbox{\tiny\rm test}}


\def\ttau{\tilde{\tau}}


\def\cE{{\mathcal E}}
\def\bt{{\boldsymbol t}}
\def\normal{{\sf N}}

\def\bDelta{{\boldsymbol \Delta}}










\def\cX{{\mathcal X}}
\def\CKR{{\rm CKR}}
\def\bproj{{\overline \proj}}
\def\quadratic{{\rm quad}}
\def\cube{{\rm cube}}
\def\Cube{{\mathscrsfs Q}}

\def\Poly{{\rm Poly}}
\def\Coeff{{\rm Coeff}}
\def\RF{{\rm RF}}
\def\NT{{\rm NT}}
\def\bA{{\boldsymbol A}}
\def\btheta{{\boldsymbol \theta}}
\def\bTheta{{\boldsymbol \Theta}}
\def\bLambda{{\boldsymbol \Lambda}}
\def\blambda{{\boldsymbol \lambda}}

\def\cM{{\mathcal M}}

\def\cT{{\mathcal T}}
\def\cV{{\mathcal V}}
\def\bP{{\boldsymbol P}}
\def\diag{{\rm diag}}
\def\bS{{\boldsymbol S}}
\def\bO{{\boldsymbol O}}
\def\bD{{\boldsymbol D}}
\def\bPsi{{\boldsymbol \Psi}}
\def\bsh{{\boldsymbol h}}
\def\bL{{\boldsymbol L}}



\def\osigma{\overline{\sigma}}
\def\tbu{\Tilde \bu}
\def\tbZ{\Tilde \bZ}
\def\tbphi{\Tilde \bphi}
\def\tbpsi{\Tilde \bpsi}

\def\tbf{\Tilde \boldf}
\def\hbU{\hat{{\boldsymbol U}}_\lambda }
\def\hbUi{\hat{{\boldsymbol U}}_\lambda^{-1} }
\def\bb{{\boldsymbol b}}
\def\bsigma{{\boldsymbol \sigma}}

\def\hf{\hat f}
\def\hbf{\hat \boldf}
\def\bR{{\boldsymbol R}}
\def\bpsi{{\boldsymbol \psi}}
\def\cuH{\mathscrsfs{H}}

\def\noisestd{\sigma_{\varepsilon}}

\def\evn{{\mathsf m}}
\def\evN{{\mathsf M}}

\def\lvn{{\mathsf s}}
\def\lvN{{\mathsf S}}

\def\bc{{\boldsymbol c}}
\def\bC{{\boldsymbol C}}
\def\oba{\overline{{\boldsymbol a}}}
\def\uba{\underline{{\boldsymbol a}}}

\def\barsigma{\bar{\sigma}}

\def\tbN{\Tilde \bN}
\def\dv{{D}}

\def\tbV{\Tilde \bV}
\def\hiota{{\hat \iota}}
\def\biota{{\boldsymbol \iota}}
\def\hbiota{{\hat {\boldsymbol \iota}}}

\def\bzeta{{\boldsymbol \zeta}}
\def\hbzeta{{\hat {\boldsymbol \zeta}}}
\def\oproj{{\overline \proj}}
\def\barHop{\bar{\Hop}}
\def\barUop{\bar{\Uop}}
\def\barU{\bar{U}}
\def\barH{\bar{H}}
\def\ind{\mathbbm{1}}

\def\tC{\Tilde C}
\def\tQ{\Tilde Q}
\def\balpha{\boldsymbol{\alpha}}
\def\bgamma{\boldsymbol{\gamma}}
\def\cU{\mathcal{U}}
\def\tbC{\Tilde \bC}
\def\tba{\Tilde \ba}
\def\tbeta{\Tilde \beta}
\def\tbbeta{\Tilde \bbeta}
\def\boldf{\boldsymbol{f}}
\def\bXi{\boldsymbol{\Xi}}
\def\cB{\mathcal{B}}
\def\MP{{\rm MP}}
\def\complex{\mathbbm{C}}
\def\Im{{\rm Im}}
\def\tbM{\Tilde \bM}

\def\sR{\mathsf R}
\def\sV{\mathsf V}
\def\sB{\mathsf B}

\def\obR{\overline{\bR}}
\def\obM{\overline{\bM}}
\def\wbM{\widetilde{\bM}}
\def\tbR{\widetilde{\bR}}
\def\tbM{\widetilde{\bM}}

\def\ulambda{\overline{\lambda}}
\def\hbtheta{\hat \btheta}
\def\rr{{\rm r}}

\def\rC{\textcolor{red}{C}}

\def\rSQ{{\rm SQ}}

\def\rdc{{\rm dc}}
\def\rmc{{\rm mc}}
\def\cY{\mathcal{Y}}
\def\cZ{\mathcal{Z}}
\def\rdeg{{\rm deg}}


\def\dom{{\rm dom}}
\def\prox{{\rm prox}}
\def\hE{\widehat{\E}}
\def\okappa{\overline{\kappa}}
\def\otau{\overline{\tau}}
\def\br{{\boldsymbol r}}
\def\bGamma{{\boldsymbol \Gamma}}
\def\cJ{\mathcal{J}}
\def\oxi{\overline{\xi}}
\def\hbalpha{\hat{\balpha}}
\def\sfG{\textsf{G}}
\def\sfMG{\textsf{MG}}
\def\obz{\overline{\bz}}
\def\obZ{\overline{\bZ}}
\def\obg{\overline{\bg}}
\def\obG{\overline{\bG}}
\def\tbU{\Tilde{\bU}}
\def\obx{\overline{\bx}}
\def\ox{\overline{x}}



\def\tC{\Tilde C}
\def\tQ{\Tilde Q}
\def\balpha{\boldsymbol{\alpha}}
\def\bgamma{\boldsymbol{\gamma}}
\def\cU{\mathcal{U}}
\def\tbC{\Tilde \bC}
\def\tba{\Tilde \ba}
\def\tbeta{\Tilde \beta}
\def\tbbeta{\Tilde \bbeta}
\def\boldf{\boldsymbol{f}}
\def\bXi{\boldsymbol{\Xi}}
\def\cB{\mathcal{B}}
\def\MP{{\rm MP}}
\def\complex{\mathbbm{C}}
\def\Im{{\rm Im}}
\def\tbM{\Tilde \bM}

\def\sR{\mathsf R}
\def\sV{\mathsf V}
\def\sB{\mathsf B}

\def\ulambda{\overline{\lambda}}
\def\hbtheta{\hat \btheta}
\def\oPhi{\overline{\Phi}}
\def\sfPhi{\mathsf \Phi}

\def\hbSigma{\hat{\bSigma}}
\def\sfC{{\sf C}}
\def\sfc{{\sf c}}
\def\sfD{{\sf D}}
\def\sfM{{\sf M}}
\def\rmI{{\rm I}}
\def\rmII{{\rm II}}
\def\obQ{\overline{\bQ}}
\def\tS{\widetilde{S}} 
\def\tbS{\widetilde{\bS}}  
\def\obtheta{\overline{\btheta}}
\def\onu{\overline{\nu}}
\def\oT{\overline{T}}
\def\sL{\mathsf{L}}
\def\bq{\boldsymbol{q}}
\def\og{\overline{g}}
\def\oq{\overline{q}}
\def\ske{{\sf ske}}
\def\bs{{\boldsymbol s}}
\def\obD{\overline{\bD}}
\def\osfD{{\overline{{\sf D}}}}
\def\sflf{{\sf leaf}}
\def\sfT{{\sf T}}
\def\sfG{{\sf G}}
\def\bsfT{{\boldsymbol \sfT}}
\def\bsfG{{\boldsymbol \sfG}}
\def\obi{\overline{\bi}}
\def\obsfT{\overline{\bsfT}}
\def\obsfG{\overline{\bsfG}}
\def\oi{\overline{i}}
\def\osfT{\overline{\sfT}}
\def\osfG{\overline{\sfG}}
\def\sfH{{\sf H}}
\def\tbD{\widetilde{\bD}}
\def\polylog{\text{polylog}}
\def\tcL{{\widetilde{\cL}}}
\def\tsL{{\widetilde{\sL}}}

\def\seff{{\sf eff}}
\def\sG{\mathsf{G}}
\def\sKL{\mathsf{KL}}
\def\oevn{\overline{\evn}}
\def\obeta{\overline{\beta}}
\def\oC{\overline{C}}

\def\tnu{\Tilde{\nu}}
\def\hbSigma{\widehat{\bSigma}}
\def\tmu{\Tilde{\mu}}
\def\sK{{\sf K}}
\def\sA{{\sf A}}
\def\tPhi{\widetilde{\Phi}}
\def\obF{\overline{\bF}}
\def\oboldf{\overline{\boldf}}
\def\tr{\widehat{r}}
\def\hxi{\hat{\xi}}
\def\hr{\widehat{r}}
\def\hrho{\widehat{\rho}}
\def\trho{\widetilde{\rho}}
\def\tcA{\widetilde{\cA}}
\def\obv{\overline{\bv}}
\def\tsB{\widetilde{\sB}}
\def\tbG{\widetilde{\bG}}


\newcommand{\G}{\mathbf{G}}
\newcommand{\GT}{\mathbf{G}^\top}
\newcommand{\bet}{\boldsymbol{\beta}}
\newcommand{\U}{\mathbf{U}}
\newcommand{\V}{\mathbf{V}}
\newcommand{\D}{\mathbf{D}}
%\newcommand{\R}{\mathbb{R}}
%\newcommand{\E}{\mathbb{E}}
\newcommand{\Sph}{\mathbb{S}}
%\newcommand{\I}{\mathbb{I}}
%\newcommand{\Pr}{\mathbb{P}}
%\newcommand{\bx}{\boldsymbol{x}}
%\newcommand{\bw}{\boldsymbol{w}}
%\newcommand{\bz}{\boldsymbol{z}}
\newcommand{\bblV}{{\color{blue}\bV}}
\newcommand{\zq}[1]{{\leavevmode\color[rgb]{1,0,0}[Ziqian: #1]}}
% The \icmltitle you define below is probably too long as a header.
% Therefore, a short form for the running title is supplied here:
\icmltitlerunning{\mytitle}

\begin{document}

\twocolumn[
\icmltitle{\mytitle}

% It is OKAY to include author information, even for blind
% submissions: the style file will automatically remove it for you
% unless you've provided the [accepted] option to the icml2025
% package.

% List of affiliations: The first argument should be a (short)
% identifier you will use later to specify author affiliations
% Academic affiliations should list Department, University, City, Region, Country
% Industry affiliations should list Company, City, Region, Country

% You can specify symbols, otherwise they are numbered in order.
% Ideally, you should not use this facility. Affiliations will be numbered
% in order of appearance and this is the preferred way.
\icmlsetsymbol{equal}{*}

\begin{icmlauthorlist}
\icmlauthor{Ziqian Lin}{xxx}
\icmlauthor{Shubham Kumar Bharti}{xxx}
\icmlauthor{Kangwook Lee}{zzz}
%\icmlauthor{Firstname5 Lastname5}{yyy}
%\icmlauthor{Firstname6 Lastname6}{sch,yyy,comp}
%\icmlauthor{Firstname7 Lastname7}{comp}
%\icmlauthor{}{sch}
%\icmlauthor{Firstname8 Lastname8}{sch}
%\icmlauthor{Firstname8 Lastname8}{yyy,comp}
%\icmlauthor{}{sch}
%\icmlauthor{}{sch}
\end{icmlauthorlist}

\icmlaffiliation{xxx}{Department of Computer Science, University of Wisconsin-Madison, Madison, Wisconsin, USA}
\icmlaffiliation{zzz}{Department of Electrical \& Computer Engineering, University of Wisconsin-Madison, Madison, Wisconsin, USA}
%\icmlaffiliation{comp}{Company Name, Location, Country}
%\icmlaffiliation{sch}{School of ZZZ, Institute of WWW, Location, Country}

%\icmlcorrespondingauthor{Ziqian Lin}{zlin284@wisc.edu}
\icmlcorrespondingauthor{Kangwook Lee}{kangwook.lee@wisc.edu}

% You may provide any keywords that you
% find helpful for describing your paper; these are used to populate
% the "keywords" metadata in the PDF but will not be shown in the document
\icmlkeywords{Machine Learning, ICML}

\vskip 0.3in
]

% this must go after the closing bracket ] following \twocolumn[ ...

% This command actually creates the footnote in the first column
% listing the affiliations and the copyright notice.
% The command takes one argument, which is text to display at the start of the footnote.
% The \icmlEqualContribution command is standard text for equal contribution.
% Remove it (just {}) if you do not need this facility.

\printAffiliationsAndNotice{}  % leave blank if no need to mention equal contribution
%\printAffiliationsAndNotice{\icmlEqualContribution} % otherwise use the standard text.

\begin{abstract}
% In-context learning (ICL) enables large language models (LLMs) to adapt to new tasks by conditioning on examples.
Recent research has investigated the underlying mechanisms of in-context learning (ICL) both theoretically and empirically, often using data generated from simple function classes.
However, the existing work often focuses on the sequence consisting solely of labeled examples, while in practice, labeled examples are typically accompanied by an \emph{instruction}, providing some side information about the task. 
In this work, we propose \emph{ICL with hypothesis-class guidance (ICL-HCG)}, a novel synthetic data model for ICL where the input context consists of the literal description of a (finite) hypothesis class $\mathcal{H}$ and $(x,y)$ pairs from a hypothesis chosen from $\mathcal{H}$.
Under our framework ICL-HCG, we conduct extensive experiments to explore: 
(i) a variety of generalization abilities to new hypothesis classes; 
(ii) different model architectures;
(iii) sample complexity;
(iv) in-context data imbalance;
(v) the role of instruction; and
(vi) the effect of pretraining hypothesis diversity.
As a result, we show that 
(a) Transformers can successfully learn ICL-HCG and generalize to unseen hypotheses and unseen hypothesis classes, and (b) compared with ICL without instruction, ICL-HCG achieves significantly higher accuracy, demonstrating the role of instructions. 
The code is available at:  
\url{https://github.com/UW-Madison-Lee-Lab/ICL-HCG}.
\end{abstract}

\section{Introduction}
\label{sec:intro}
% Image editing methods in diffusion models depend on user-defined control directions - users can unlock their creativity using these methods by specifying the desired manipulation through prompts~\cite{gandikota2023concept}, reference images~\cite{ruiz2022dreambooth, kumari2022customdiffusion, gal2022image, chen2024trainingfreeregionalpromptingdiffusion}, or attribute vectors~\cite{parmar2023zero,hertz2022prompt}. In this work, we ask a fundamentally different question: \emph{Can we automatically discover the underlying visual structure of a concept within diffusion model's knowledge?} %Rather than requiring user-specified controls, we aim to decompose the model's internal knowledge into meaningful directions.

% This question touches on a fundamental limitation in how we interact with diffusion models. Current control methods ~\cite{zhang2023addingconditionalcontroltexttoimage, gandikota2023concept, ye2023ipadaptertextcompatibleimage,ye2023ipadaptertextcompatibleimage, hertz2024stylealignedimagegeneration, li2023photomaker, shi2024instantbooth, chen2024trainingfreeregionalpromptingdiffusion} require users to specify their desired manipulations in advance, limiting interactive creativity. This contrasts with natural human artistic workflows, where creators dynamically explore creative ideas while jointly refining them toward meaningful artistic outcomes~\cite{hoffmann2016modeling}. This synergy between specification and exploration is not new to generative models. Early GAN architectures naturally developed disentangled latent spaces that enabled continuous\cite{harkonen2020ganspace,radford2015unsupervised, wu2021stylespace, shen2020interfacegan}, compositional control over generated images. Users could explore these spaces to discover interesting variations that would be difficult to describe in words~\cite{wu2021stylespace}, then combine them to achieve their creative goals~\cite{grabe2022towards}. 


% While diffusion models have largely superseded GANs in conditional image synthesis~\cite{dhariwal2021diffusion},  their underlying structure remains less understood. Diffusion models achieve remarkable diversity through high-dimensional latents, unlike GANs' compact latent spaces.  With a single prompt, diffusion models can generate radically different variations through different random initializations of input noise. We ask - Is it possible to discover interpretable structure within this vast space of variations?

Text-to-image diffusion models are capable of generating remarkable visual variations from a single prompt through different random initializations. However, this vast creative potential remains largely opaque to users---while we can generate diverse images, we lack understanding of the underlying structure of these variations. This presents a fundamental challenge: how can we discover and expose the latent visual capabilities encoded within these models?

\let\thefootnote\relax \footnote{$^{*}$Correspondence to \texttt{gandikota.ro@northeastern.edu}}

The challenge touches on a key limitation in how we interact with diffusion models today. Current control methods require users to explicitly specify their desired edits in advance through prompts~\cite{gandikota2023concept}, reference images~\cite{zhang2023addingconditionalcontroltexttoimage, chen2024trainingfreeregionalpromptingdiffusion, ruiz2022dreambooth,kumari2022customdiffusion, Ryu_lora, hu2021lora}, or attribute vectors~\cite{ye2023ipadaptertextcompatibleimage, hertz2024stylealignedimagegeneration, li2023photomaker, shi2024instantbooth,parmar2023zero,hertz2022prompt}. That contrasts sharply with natural human creative workflows, where artists dynamically explore creative ideas and jointly refine them toward meaningful artistic outcomes~\cite{hoffmann2016modeling}. The need for pre-specified controls creates a barrier between users and the full creative potential of these models.

Interestingly, earlier generative models like GANs~\cite{gans,karras2019style,brock2018large} naturally developed more interpretable internal structures. Their compact latent spaces often exhibited emergent disentanglement~\cite{harkonen2020ganspace,radford2015unsupervised, wu2021stylespace, shen2020interfacegan}, enabling continuous and compositional control over generated images. Users could explore these spaces to discover interesting variations that would be difficult to describe in words~\cite{wu2021stylespace}, then combine them to achieve their creative goals~\cite{grabe2022towards}.

Diffusion models have largely superseded GANs in conditional image synthesis~\cite{dhariwal2021diffusion}, achieving greater diversity through much higher-dimensional latents. And yet an understanding of the underlying structure of these larger latent spaces has remained elusive. In this work, we ask a fundamental question: \emph{Can we automatically discover the visual structure within a diffusion model's knowledge of a concept?} Rather than requiring user-specified controls, we aim to decompose the model's internal representations into expressive directions that users can explore and combine.

To address these needs, we present \textbf{SliderSpace}, a framework that brings systematic explorability to diffusion models. Given just a text prompt, SliderSpace discovers a canonical set of meaningful, diverse, and controllable directions within the model's knowledge of that concept. Each direction is implemented as a low-rank adapter~\cite{hu2021lora} that can be scaled and composed with others, allowing users to explore and smoothly combine different aspects of variation, as shown in Figure~\ref{fig:intro}.

We ground SliderSpace discovery in three key requirements for meaningful decomposition of a diffusion model's visual manifold: 
\begin{enumerate}
    \item \textbf{Unsupervised Discovery:} The decomposition process should emerge from the intrinsic structure of the model's learned representation, rather than being guided by predefined attributes. This ensures we capture the true topology of the model's knowledge space rather than projecting our assumptions onto it.
    
    \item \textbf{Semantic Orthogonality:} Each discovered control must represent a distinct semantic direction. This is enforced in a semantic feature space, like CLIP, where every slider has an orthogonal effect in embeddings. This prevents discovering multiple controls that create similar semantic effects, making the system more efficient and easier.
    
    \item \textbf{Distribution Consistency:} Directions must induce consistent transformations across both random seeds and prompt variations. 
\end{enumerate}

These requirements naturally lead to our proposed framework, which we formalize in Section~\ref{sec:method}. As we show in our experiments, SliderSpace is architecture-agnostic, working with both conventional U-Net based models like Stable Diffusion~\cite{rombach2022high, rombach2022sd20, podell2023sdxl, turbo, dmd} and recent transformer-based architectures like Flux~\cite{flux}.

We demonstrate the expressiveness of SliderSpace through three applications: First, we show how SliderSpace can decompose high-level concepts into diverse and expressive components, revealing the natural axes of variation in the model's understanding. Second, we explore artistic style variation, where SliderSpace discovers directions that match or exceed the diversity of manually curated artist lists while being judged more useful by human evaluators. Finally, we show how SliderSpace can help reverse the mode collapse commonly observed in distilled diffusion models, restoring diversity while maintaining generation speed.

Beyond providing practical creative control, SliderSpace opens new avenues for understanding and utilizing the latent capabilities of diffusion models. By mapping these models' visual potential into intuitive, composable directions, we take a step toward making their creative possibilities more accessible and interpretable to users.

% Image editing methods in diffusion models unlock the creativity of users. In this work we ask an alternate question: \emph{Can we organize and expose what of the diffusion model is already capable of?}.
% Existing methods for controlling image generation typically require users to manually specify edit directions for desired changes. This process is time-consuming, requires technical expertise, and limits the spontaneity of the creative process. For instance, if a user wants to adjust the smile of a generated person, they must explicitly request this edit, often through imprecise prompt engineering or model fine-tuning. This approach of predefined controls or manual specifications restricts users from fully exploring the latent capabilities of the model. There may be interesting stylistic variations or attributes that the model can generate, but users have no easy way to discover or utilize these.

% Natural visual disentanglement was an emergent property in the latent space of Generative Adversarial Models (GANs) \cite{harkonen2020ganspace,radford2015unsupervised, wu2021stylespace, shen2020interfacegan}. In particular, it has been observed that StyleGAN~\cite{karras2019style} stylespace neurons offer detailed control over many meaningful aspects of images that would be difficult to describe in words~\cite{wu2021stylespace}. However, diffusion models do not share such a compact latent space~\cite{park2023unsupervised}; and efforts to uncover such a space in the semantic embeddings of the text conditioning have met with limited success \nik{Nick - is there a specific citation you were thinking about?}.

% In this work we introduce \textbf{SliderSpace}, which takes a step towards uncovering an analogous low dimensional representation of diffusion models' visual breadth; in essence treating the diffusion model as many generators sharing parameters, where a particular generator is defined by a specific prompt. For a given prompt we sample many random seeds (and optionally prompt expansions using an LLM), generate the corresponding images, and apply an off the shelf feature extractor (in this work CLIP, but our method can be applied to any differentiable feature extractor). We use PCA to analyze these features, and for each of the leading $k$ principal components we train a LoRA \cite{} which causes the diffusion model to produces images which increase the feature magnitude along that component when passed back through the same feature extractor. This leads to a 'Slider' for each principal component, because each LoRA can be scaled and applied to the original diffusion model, continuously varying those visual features in the generated results (as measured, in our case, by CLIP).

% There are many other works that enhance the controllability of diffusion models. One common approach is enabling users to add spatial constraints to a generation either manually, or via a reference image \cite{zhang2023addingconditionalcontroltexttoimage, chen2024trainingfreeregionalpromptingdiffusion}, a second is leveraging more abstract embeddings (e.g. identity, style) extracted from a reference image \cite{ye2023ipadaptertextcompatibleimage, hertz2024stylealignedimagegeneration, li2023photomaker, shi2024instantbooth}, a third is finetuning a foundation model to better generate a concept important to the user \cite{ruiz2022dreambooth, kumari2022customdiffusion, Ryu_lora, hu2021lora}, and a fourth (most relevant to this work) is finding low-rank adaptors of the model based on a prompt or small training set which can be scaled to provide continous control over one aspect of generated image (e.g. night vs day, basic vs luxury, etc.) \cite{gandikota2023concept}. SliderSpace is complementary to all of these methods and offers something distinct. All of the other methods we are aware require the user (and / or model designer) to know in advance what type of control they want. In contrast SliderSpace assists users in discovering and controlling hidden capabilities present in the diffusion model's distribution of possible generations.

%We propose that truly intuitive creative control in a text-to-image model should meet three key criteria: \emph{discoverability}, \emph{intuitiveness}, and \emph{specificity}. The model should reveal controllable attributes that may not be immediately obvious, offer controls that are easy to understand and manipulate, and ensure each control affects a distinct attribute of the generated image.

% We demonstrate the utility and power of SliderSpace using three applications built on top of SDXL-DMD \cite{dmd}, because its fast generation speed lends itself well to the continuous control offered by SliderSpace.

% First, we study concept decomposition (Section \ref{sec:concept_exp}), where we learn sliders for a specific concept (e.g. 'monster', 'waterfall', 'car'). Through quantitative metrics of diversity and text alignment we demonstrate that the learned sliders dramatically boost the diversity of generations when randomly applied without harming text alignment; we also ask humans to qualitatively judge these results in a user study where they find the SliderSpace results to be more 'Diverse', 'Useful', and 'Creative' than our baselines.

% Second, we attempt to compare the automatic discoveries of SliderSpace to a large scale manual study of artistic styles (Section \ref{sec:art_exp}), open-sourced by ParrotZone \cite{parrotzone}. In this study SDXL was prompted with over 4300 artist names,  and based on visual inspection the cases of successful stylistic mimicry recorded. Quantitatively SliderSpace more closely matches the distribution of artistic variation discovered by ParrotZone than other baselines, and in our user studies was judged to be significantly more 'Diverse' and 'Useful' than the baselines. To our surprise humans even judged SliderSpace results to be slightly more 'Diverse' than the results generated by the manually discovered artist names of \cite{parrotzone}.

% Third, we attempt to use SliderSpace to reverse the mode collapse commonly observed in distilled few-step diffusion models relative to the original teacher model (Section \ref{sec:diverse_exp}). We quantitatively demonstrate that applying SliderSpace to SDXL-DMD leads to more closely matching the distribution of images by the original teacher, SDXL.

%Through extensive experiments on various state-of-the-art text-to-image models, we demonstrate that SliderSpace significantly enhances user control and creative expression in AI-assisted image generation tasks. Our method enables a range of applications, including concept decomposition and control, diversity improvement in generated images, customization dissection and edits, and the exploration of artistic styles inherent in the model.

% SliderSpace goes beyond providing a practical tool for enhanced creative control. By mapping the visual potential of diffusion models it can open new avenues for generative creativity and deepens our understanding of each model's hidden potential.
\section{Related Work}

\paragraph{LLMs for Agent tasks.}

Our research is related to deploying large language models (LLMs) as agents for decision-making tasks in interactive environments~\citep{liu2023agentbench,zhou2023webarena,shridhar2020alfred,toyama2021androidenv}. Earlier works, such as~\citep{yao2023webshopscalablerealworldweb}, fine-tuned models like BERT~\citep{devlin2019bertpretrainingdeepbidirectional} for decision-making in simplified environments, such as online shopping or mobile phone manipulation. With the advent of large language models~\citep{brown2020languagemodelsfewshotlearners,openai2024gpt4technicalreport}, it became feasible to perform decision-making tasks through zero-shot or few-shot in-context learning. To better assess the capabilities of LLMs as agents, several models have been developed~\citep{deng2024mind2web,xiong2024watch,hong2023cogagent,yan2023gpt}. Most approaches~\citep{zheng2024seeact,deng2024mind2web} provide the agent with observation and action history, and the language model predicts the next action via in-context learning. Additionally, some methods~\citep{zhang2023building,li2023camel,song2024trial} attempt to distill trajectories from state-of-the-art language models to train more effective policy models. In contrast, our paper introduces a novel framework that automatically learns a reward model from LLM agent navigation, using it to guide the agents in making more effective plans.

\textbf{LLM Planning.} Our paper is also related to planning with large language models. Early researchers~\citep{brown2020languagemodelsfewshotlearners} often prompted large language models to directly perform agent tasks. Later, \citet{yao2022react} proposed ReAct, which combined LLMs for action prediction with chain-of-thought prompting~\citep{wei2022chain}. Several other works~\citep{yao2023treethoughtsdeliberateproblem,hao2023reasoning,zhao2023large,qiao2024agentplanningworldknowledge} have focused on enhancing multi-step reasoning capabilities by integrating LLMs with tree search methods. Our model differs from these previous studies in several significant ways. First, rather than solely focusing on text generation tasks, our pipeline addresses multi-step action planning tasks in interactive environments, where we must consider not only historical input but also multimodal feedback from the environment. Additionally, our pipeline involves automatic learning of the reward model from the environment without relying on human-annotated data, whereas previous works rely on prompting-based frameworks that require large commercial LLMs like GPT-4~\citep{openai2024gpt4technicalreport} to learn action prediction. Furthermore, \Model supports a variety of planning algorithms beyond tree search.

\textbf{Learning from AI Feedback.} In contrast to prior work on LLM planning, our approach also draws on recent advances in learning from AI feedback~\citep{bai2022constitutional,lee2023rlaif,yuan2024self,sharma2024critical,pan2024autonomous,koh2024tree}. These studies initially prompt state-of-the-art large language models to generate text responses that adhere to predefined principles and then potentially fine-tune the LLMs with reinforcement learning. Like previous studies, we also prompt large language models to generate synthetic data. However, unlike them, we focus not on fine-tuning a better generative model but on developing a classification model that evaluates how well action trajectories fulfill the intended instructions. This approach is simpler, requires no reliance on state-of-the-art LLMs, and is more efficient. We also demonstrate that our learned reward model can integrate with various LLMs and planning algorithms, consistently improving their performance.

\textbf{Inference-Time Scaling.} ~\citet{snell2024scaling} validates the efficacy of inference-time scaling for language models. Based on inference-time scaling, various methods have been proposed, such as random sampling~\citep{wang2022self} and tree-search methods~\citep{hao2023reasoning, zhang2024accessing, guan2025rstar}. Concurrently, several works have also leveraged inference-time scaling to improve the performance of agentic tasks. ~\citet{koh2024tree} adopts a training-free approach, employing MCTS to enhance policy model performance during inference and prompting the LLM to return the reward. ~\citet{gu2024your} introduces a novel speculative reasoning approach to bypass irreversible actions by leveraging LLMs or VLMs. It also employs tree search to improve performance and prompts an LLM to output rewards. ~\citet{yu2024exact} proposes Reflective-MCTS to perform tree search and fine-tune the GPT model, leading to improvements in ~\citet{koh2024visualwebarena}. ~\citet{putta2024agent} also utilizes MCTS to enhance performance on web-based tasks such as ~\citet{yao2023webshopscalablerealworldweb} and real-world booking environments. ~\cite{lin2025qlass} utilizes the stepwise reward to give effective intermediate guidance across different agentic tasks. Our work differs from previous efforts in two key aspects: (1) Broader Application Domain. Unlike prior studies that primarily focus on tasks from a single domain, our method demonstrates strong generalizability across web agents, mathematical reasoning, and scientific discovery domains, further proving its effectiveness. (2) Flexible and Effective Reward Modeling. Instead of simply prompting an LLM as a reward model, we finetune a small scale VLM~\citep{lin2023vila} to evaluate input trajectories. %Our reward scores range continuously between 0 and 1, in contrast to existing methods that rely on discrete scoring (e.g., 0 and 1, or 0, 0.5, and 1) through direct LLM prompting.

% Concurrently, several works have also leveraged inference-time scaling to improve the performance of agentic tasks. ~\citet{pan2024autonomous} demonstrates that LLMs and VLMs, such as the GPT series, can function as evaluators or reward models to provide guidance for fine-tuning or reflection, thereby enhancing digital agents. This lays the groundwork for subsequent studies that directly prompt LLMs as reward models. ~\citet{koh2024tree} adopts a training-free approach, employing MCTS to enhance policy model performance during inference. However, it is limited to web environments~\citep{koh2024visualwebarena}. Moreover, its value function relies on prompting an LLM, which is less effective than our proposed method. We validate our approach through ablation studies, demonstrating that our fine-tuned reward model is more effective. ~\citet{gu2024your} introduces a novel speculative reasoning approach to bypass irreversible actions, such as purchasing a product, by leveraging LLMs or VLMs. It also employs tree search to improve performance, but it remains restricted to the web domain~\citep{koh2024visualwebarena, deng2024mind2web}. Additionally, it lacks reward modeling and instead prompts an LLM to output rewards. ~\citet{yu2024exact} proposes Reflective-MCTS to perform tree search and fine-tune the GPT model, leading to improvements in ~\citep{koh2024visualwebarena}. However, this work focuses solely on a single web agent task, and its reward modeling is derived from multi-agent debate, differing from our more effective and efficient reward modeling approach. ~\citet{putta2024agent} also utilizes MCTS to enhance performance, but it is limited to web-based tasks such as ~\citep{yao2023webshopscalablerealworldweb} and real-world booking environments.
\section{Meta-Learning for ICL-HCG}
Training a learner to perform ICL aligns with the concept of meta-learning, as it enables adaptation to new tasks using in-context examples. While prior studies~\citep{garg2022can,fan2024transformers,raventos2024pretraining} train Transformers for ICL on sequences of the form \((x_1, y_1, x_2, y_2, \dots, x_k, y_k)\) without explicit instructions, our work investigates whether a Transformer trained for ICL with instructions, namely ICL-HCG, can generalize to new ICL-HCG tasks.

% While meta-learning trains across tasks and fine-tunes on new samples for rapid adaptation, MetaICL trains a learner on ICL tasks, learning to predict by using those new samples as in-context examples---feeding them along with the query sample (the one to be predicted) as input to the learner---eliminating the finetuning stage.

% Enabling a model to do in-context learning usually requires pretraining it on in-context learning sample of form $\text{training examples, test queries}$ sampled from a meta task distribution $P_{task}$ during pretraining of the large models.
% We use this section to introduce our meta-learning framework for ICL-HCG.
% We start by introducing the the meta tasks in ICL-HCG
%\zq{Is this meta learning? or just generalization? Since the task of meta in our work would be $(H,h,xyxy)$, which is a sample rather than a ``task''.}
%\subsection{Empirical Risk Minimization for Binary Classification (come later, what is the data ERM for a fix, fix hypothesis)}
\label{sec:ERM}
In this work, we consider the binary classification problem where the input space is finite. Formally, let the input space be defined as
$
\mathcal{X} = \{x_1, x_2, \ldots, x_{|\mathcal{X}|}\}
$,
and the label space as
$
\mathcal{Y} = \{0, 1\}
$.
The learning task is characterized by a hypothesis space
$
\mathcal{H} \subseteq \mathcal{Y}^{\mathcal{X}}
$,
where each hypothesis $ h \in \mathcal{H} $ is a deterministic function
$
h: \mathcal{X} \rightarrow \mathcal{Y}
$.
A realizable instance of the binary classification problem is specified by the tuple
$
(\mathcal{X}, \mathcal{Y}, P, h^*)
$,
where $P$ is a probability distribution over the input space $\mathcal{X}$, and $h^*:\mathcal{X} \rightarrow \mathcal{Y}$ is the target hypothesis.

During learning, a learner $A$ receives a training dataset $\mathcal{D}=\{(x^{(i)}, y^{(i)})\}_{i=1}^K$ and it is tasked to find a hypothesis that produces least generalization risk $R(A(\mathcal{D}))$ on future test sample drawn from $P$ and $h^*$.
Empirical risk minimization (ERM) is one of the most popular learning algorithm. It learns by minimizing the empirical risk on the training dataset $\mathcal D$ with respect to a loss function $l:\mathcal{Y} \times \mathcal{Y} \rightarrow \mathbb R^+$ and is given as,
$$
\hat h_\mathcal{D} = \text{ERM}(\mathcal{D},\mathcal{H}) \leftarrow \min_{h \in \mathcal{H}} \sum_{i=1}^K l(h(x^{(i)}), y^{(i)}).
$$
% Once learning is complete, the learner uses learnt hypothesis $\hat h_D$ to make prediction on samples from $P, h^*$ thereby suffering a risk of, $$R(\hat h_D) \leftarrow \mathbb E_{(x, y) \sim P, h^*}[l(\hat h_D(x), y)].$$
In this paper, we focus on $0/1$ risk, $$l(\hat{y},y)=
\begin{cases}
    1, & \text{if } \hat{y}=y\\
    0, & \text{if } \hat{y}\neq y
\end{cases}.
$$
\begin{remark}
ERM focuses mainly on sample complexity, that is, determining the minimum number of samples required to identify a hypothesis $\hat{h}$ with a risk below a specified threshold.
Differently, in this paper, we explore whether Transformer models can accurately pinpoint the exact hypothesis when provided with sufficient number of samples.
\end{remark}

% In theory, Probably Approximately Correct(PAC) learning completely characterizes the sample complexity of learning. It states that for any learning instance and $(\epsilon, \delta) \in (0,1)^2$ if $n\geq O(\frac{1}{\epsilon}(d(\mathcal H)+\log(\frac{1}{\delta})))$, then,  $$\text{w.p. $\geq 1-\delta$}, \qquad R(\hat h_D) \leq \epsilon.$$
% where $d(\mathcal H)$ is the VC dimension of hypothesis class $\mathcal H$ which is a measure of its complexity. Therefore, to learn the target hypothesis up to an error $\epsilon$, computing the ERM on an i.i.d. sample of size $O(\frac{d}{\epsilon})$ is sufficient.


%%% motivating incontext learning with finite hypothesis class

% In recent literature, several works {cite} have shown impressive capability of transformers to in-context learn ERM on a dataset $D$ at inference time which is known as In-context Learning(ICL). However, most of the works have focused on ICL learning settings where hypothesis class is defined implicitly by a linear $\mathbb R^d$ space. This begs the question of whether transformers can do ERM in more explicit learning settings where hypothesis class are specified as input to the model. Towards this end we seek to answer the following question:

% \textbf{Can a transformers learn to do in-context learning conditioned on finite/tabular hypothesis classes?}

% A couple of recent works have tried to address this problem by conditioning the hypothesis class on a noisy version of the target hypothesis {cite}. Contrary to this, a finite hypothesis class conditions the universal hypothesis space $\mathcal{A}^\mathcal{Y}$ by specifying a subset of hypothesis $\mathcal H$ that contains a the target hypothesis to solve the task instance. In real world learning scenario, this can be viewed   The explicit conditioning on finite hypothesis class also provide an avenue to test the generalization capability of ICL to generalize to different size of hypothesis tables thereby helping us to understand ERM capability of transformers in a more controlled setup. 

%%% Ziqian's previous writeup

% \subsection{Empirical Risk Minimization (ERM)}
% Consider a training dataset $\{(x_i, y_i)\}_{i=1}^n$, where each $x_i \in \mathcal{A}$ is a feature vector and $y_i \in \mathcal{Y}$ is the corresponding label or target value. Let $\mathcal{H}$ be a hypothesis class of functions $h: \mathcal{A} \to \mathcal{Y}$, and let $L:\mathcal{Y}\times\mathcal{Y}\to\mathbb{R}_{\geq 0}$ be a loss function that measures the discrepancy between the predicted value $h(x_i)$ and the true value $y_i$.

% The \emph{empirical risk} of a hypothesis $h \in \mathcal{H}$ is defined as:
% \[
% \hat{R}_n(h) = \frac{1}{n} \sum_{i=1}^n L(y_i, h(x_i)).
% \]

% The \emph{Empirical Risk Minimization} principle seeks a hypothesis $\hat{h} \in \mathcal{H}$ that minimizes the empirical risk:
% \[
% \hat{h} = \arg\min_{h \in \mathcal{H}} \hat{R}_n(h).
% \]


% we need to explain what is meta learning in plain english before going into notations 
% Meta-learning refers to the process of learning how to learn by leveraging experiences from multiple tasks.
% In our setup, each meta-task is defined by a hypothesis class, and a sample within a meta-task consists of a combination of the hypothesis class, multiple $(x,y)$ pairs, and the underlying hypothesis.
\subsection{Two Types of Tasks in ICL-HCG}
\label{sec:problem-definition}
We consider two types of tasks in ICL-HCG, both constructed from a finite hypothesis class
\(
    \mathcal{H} = \{h^{(1)}, h^{(2)}, \ldots, h^{|\mathcal{H}|}\}
\)
over a finite input space
\(
    \mathcal{X} = \{x_1, x_2, \ldots, x_{|\mathcal{X}|}\}
\)
and a binary output space
\(
    \mathcal{Y} = \{0, 1\}.
\)

\paragraph{Label prediction}
Consider a hypothesis class \(\mathcal{H}\) and a sequence consisting of training data and a test point
\[
    \Skmx = (x^{(1)}, y^{(1)}, \ldots, x^{(k-1)}, y^{(k-1)}, x^{(k)})
\]
where for all $i$, \(y^{(i)} = h(x^{(i)})\) for a specific \(h \in \mathcal{H}\), and \(x^{(k)}\) is a test query input. The objective is to predict the label 
\[
    y^{(k)} = h\bigl(x^{(k)}\bigr).
\]
We refer to this as label prediction, with input-output pairs:
\[
    i_{\text{I},k} = \bigl(\mathcal{H}, \Skmx\bigr),
    \quad
    o_{\text{I},k} = y^{(k)}.
\]

\paragraph{Hypothesis identification}
Given a hypothesis class \(\mathcal{H}\) and a sequence (namely \emph{ICL sequence}) 
\[
    \SK = (x^{(1)}, y^{(1)}, \ldots, x^{(K)}, y^{(K)}),
\]
where for all $i$, \(y^{(i)} = h(x^{(i)})\) for a specific  \(h \in \mathcal{H}\),
the goal is to identify the underlying hypothesis \(h\).
Denote this as hypothesis identification, with:
\[
    i_{\text{II},K} = \bigl(\mathcal{H}, \SK \bigr),
    \quad
    o_{\text{II},K} = h.
\]



\paragraph{Meta-learning}
%A task in meta learning is defined by a finite hypothesis class $\mathcal{H}$.
Label prediction uses \(k-1\) samples to predict the label of a new query \(x^{(k)}\), while hypothesis identification directly outputs \(h\).
Both label prediction and hypothesis identification can be viewed as attempts to identify \(h\) from $\mathcal{H}$ via empirical risk minimization (ERM) using the dataset 
\(\{(x^{(i)},y^{(i)})\}\).
Our meta-learning aims at learning to do ERM for different hypothesis classes when these hypothesis classes are given as input along with $(x,y)$ pairs.



\subsection{Sample Generation}
We consider the following two approaches for generating samples of ICL-HCG tasks.
\begin{asu}[i.i.d.\ Generation]
\label{asu:iid}
Given hypothesis classes
\(\{\mathcal{H}_i\}_{i=1}^{N}\), input space \(\mathcal{X}\), and an integer \(K\):\\
\subasu\label{asu:setting1}Sample a hypothesis class \(\mathcal{H}\) from \(\{\mathcal{H}_i\}_{i=1}^{N^\text{train}}\);\\
\subasu\label{asu:setting2}Sample a hypothesis \(h\) uniformly at random from \(\mathcal{H}\);\\
\subasu\label{asu:setting3}Sample \(K\) inputs \(\{x^{(i)}\}_{i=1}^{K}\) i.i.d.\ from \(\mathrm{Uniform}(\mathcal{X})\);\\
\subasu\label{asu:setting4}Generate \(y^{(i)} = h(x^{(i)})\) for each \(i \in [K]\);\\
\subasu\label{asu:setting5}\(\Skmx = [x^{(1)},y^{(1)}, \ldots, x^{(k)}]\) for label prediction;\\
\subasu\label{asu:setting6}{\tiny \(\SK = [x^{(1)},y^{(1)}, \ldots, x^{(K)},y^{(K)}]\)} for hypothesis identification.\\
\end{asu}

\begin{asu}[Opt-T Generation]
\label{asu:optt}
Given hypothesis classes
\(\{\mathcal{H}_i\}_{i=1}^{N}\), input space \(\mathcal{X}\), and an integer \(K\):\\
\subasu Sample a hypothesis class \(\mathcal{H}\) from \(\{\mathcal{H}_i\}_{i=1}^{N^\text{test}}\);\\
\subasu Sample a hypothesis \(h\) uniformly randomly from \(\mathcal{H}\);\\
\subasu Construct \emph{optimal teaching set}\footnote{The optimal teaching set~\citep{zhu2015machine} is the smallest set of \((x,y)\) pairs that uniquely identifies \(h\) among all candidates in \(\mathcal{H}\).} of \(h\) with respect to \(\mathcal{H}\);\\
\subasu Randomly duplicate elements from this optimal teaching set until its size reaches \(K\). Assign indices \(1\) through \(K\) arbitrarily to these \((x,y)\) pairs;\\
\subasu {\tiny \(\SK = [x^{(1)},y^{(1)}, \ldots, x^{(K)},y^{(K)}]\)} for hypothesis identification.
\end{asu}

% Under i.i.d.\ Generation (Assumption~\ref{asu:iid}), the samples \(\{(x^{(i)}, y^{(i)})\}\) are drawn uniformly at random from \(\mathcal{X}\). Under Opt-T Generation (Assumption~\ref{asu:optt}), a minimal \emph{teaching set} is first identified, and then possibly duplicated.

% Meta Training Objective**:
% The large models are finally trained to minimize the cross entropy loss on the meta dataset $D_M$,$$\theta^* \leftarrow \min_\theta \mathbb{E}_m\left[\sum_{i=1}^m \ell(f_\theta(x^{(i)}_M), y^{(i)}_M) \right].$$

% Pretraining dataset generation process : $$\mathcal H \sim P_\mathcal H, h \sim U(\mathcal H), D=\{(x_i, y_i\}|_{i=1}^n \sim P_{\mathcal X, \mathcal Y}$$. 


% Testing dataset generation process:
% 1. $$\mathcal H \sim P'_\mathcal H, h \sim U(\mathcal H), D=\{(x_i, y_i\}|_{i=1}^n \sim P_{\mathcal X, \mathcal Y}$$. 
% 2.  $$\mathcal H \sim P'_\mathcal H, h \sim U(\mathcal H), D= OPT(h^* \mathcal H)$$. 

% *****************************

% We consider two types of input $i$ and output $o$ for the meta learning tasks in this paper.
% The first type has input consists of a finite hypothesis class $\mathcal{H}=\{h^{(1)},h^{(2)},\ldots\}$ with finite input space
% $
% \mathcal{X} = \{x_1, x_2, \ldots, x_{|\mathcal{X}|}\},
% $
% and binary output space
% $
% \mathcal{Y} = \{0,1\},
% $, a sequence $\Skx=(x^{(1)},y^{(1)},\ldots,x^{(k)},y^{(k)},x^{(k+1)})$ containing $k$ pairs of $(x,y)$ satisfying $y=h(x),h\in\mathcal{H}$ and an additional query token $x^{(k+1)}$, and the output is $y^{(k+1)}=f(x^{(k+1)})$.
% The second type has input consists of the finite hypothesis class $\mathcal{H}$ and a sequence $\SK= (x^{(1)},y^{(1)},\ldots,x^{(K)},y^{(K)})$ containing $K$ pairs of $(x,y)$ satisfying $y=h(x),h\in\mathcal{H}$ and the output is the underline $h$.
% Both tasks need to distinguish the underline hypothesis of given $(x,y)$ pairs and then predict the underline hypothesis or use it to predict the $y$ of given $x$, which are loosely related to the empirical risk minimization (ERM), aiming at finding the hypothesis minimizing the risk on a given dataset $\mathcal{D}$.
% As a summary, the first type has $i_{\text{I},k}=(\mathcal{H},\Skx), o_{\text{I},k}=y^{(k+1)}$, namely Type I; and the second type has $i_{\text{I},k}=(\mathcal{H},\SK), o_{\text{I},k}=h$, namely Type II.

% We consider multi-task training on Type I ($k\in\{0,1,\ldots,K-1\}$) and Type II, with hypothesis class $\mathcal{H}$ sampled from a predefined training hypothesis classes $\{\mathcal{H}^\text{train}_i\}_{i=1}^{N^\text{train}}$.
% During training, for Type I and II, we generate $\Skx$, $y^{(k+1)}$ and $\SK$ via $h\sim\text{Uniform}(\mathcal{H})$, and then $\{x^{(i)}\}\overset{\text{i.i.d.}}{\sim}\text{Uniform}(\mathcal{X})$ and $y^{i}=h(x^{(i+1)})$.
% During testing, we perform Type I evaluation on new hypothesis classes $\{\mathcal{H}^\text{test}_i\}_{i=1}^{N^\text{test}}$ with $\Skx$ generated the same way as training.
% We perform Type II evaluation on new hypothesis classes $\{\mathcal{H}^\text{test}_i\}_{i=1}^{N^\text{test}}$ as well, but with two different generation for $\SK$: (i) the same as training, denoted (i.i.d), (ii) generating $\SK$ via firstly identifying the optimal teaching set of $h\in\mathcal{H}$, \ie, the set of minimum number of $(x,y)$ pairs distinguish $h$ from $\mathcal{H}$ and duplicating samples in it until number $K$.

\subsection{Meta Training and Testing}
\paragraph{Training}
Given a set of training hypothesis classes $\{\mathcal{H}_i^\text{train}\}_{i=1}^{N^\text{train}}$, the meta-learner is trained in a multi-task setting to minimize the following loss:
\begin{align}
\label{eq:loss}
    % \theta^* \leftarrow \min_\theta \mathbb{E}\left[\mathcal{L}_1(f_\theta(i_{\text{II}}), o_{\text{II}}) + \sum_{k=1}^K \mathcal{L}_2(f_\theta(i_{\text{I},k}), o_{\text{I},k}) \right],
    \mathcal{L} = \mathcal{L}_1(f_\theta(i_{\text{II},K}), o_{\text{II},K}) + \sum_{k=1}^K \mathcal{L}_2(f_\theta(i_{\text{I},k}), o_{\text{I},k}),
\end{align}
where we generate $\mathcal{H}$, $h$, and $\SK$ following \hyperref[asu:iid]{i.i.d. Generation}, inherently defining $(i_{\text{II},K}, o_{\text{II},K})$ and $(i_{\text{I},k}, o_{\text{I},k})$.
The loss is indeed implemented with additional terms, and we will further clarify the loss in Sec.~\ref{subsec:framework}, Eq.~\ref{eq:lossTF}.

\paragraph{Testing}
Given a set of testing hypothesis classes $\{\mathcal{H}_i^\text{test}\}_{i=1}^{N^\text{test}}$, we consider two types of testing.
\begin{itemize}[topsep=0.1em, partopsep=0em, leftmargin=*]
    \item \textbf{Label prediction}: We generate $(i_{\text{I},k}, o_{\text{I},k})$ following \hyperref[asu:iid]{i.i.d. Generation}, and then measure whether the learner $f$ predict $f(i_{\text{I},k})$ correctly for each $k\in[K]$;
    \item \textbf{Hypothesis identification}: We generate $(i_{\text{II},K}, o_{\text{II},K})$ using \hyperref[asu:optt]{Opt-T Generation} and evaluate whether the learner $f$ predicts $f(i_{\text{II}})$ correctly.
    This setting tests whether the learner acquires the ability to identify the underlying hypothesis with minimal information.
\end{itemize}


% Pretraining dataset generation process : $$\mathcal H \sim P_\mathcal H, h \sim U(\mathcal H), D=\{(x_i, y_i\}|_{i=1}^n \sim P_{\mathcal X, \mathcal Y}$$. 


% Testing dataset generation process:
% 1. $$\mathcal H \sim P'_\mathcal H, h \sim U(\mathcal H), D=\{(x_i, y_i\}|_{i=1}^n \sim P_{\mathcal X, \mathcal Y}$$. 
% 2.  $$\mathcal H \sim P'_\mathcal H, h \sim U(\mathcal H), D= OPT(h^* \mathcal H)$$. 

\subsection{Four Types of Generalization}
\paragraph{Hypothesis universe $\mathcal{H}^{\text{uni}}$}
Given an input space 
$
    \mathcal{X} = \{x_1, x_2, \ldots, x_{|\mathcal{X}|}\}
$
and a binary output space
$
    \mathcal{Y} = \{0, 1\},
$
We define the hypothesis universe $\mathcal{H}^{\text{uni}}=\mathcal{Y}^{\mathcal{X}}$ as the collection of all possible binary classification hypotheses.
This universe contains $M=2^{|\mathcal{X}|}$ distinct hypotheses, serving as a hypothesis pool to constructing training and testing hypothesis classes.

\begin{figure}[h!]
    \centering
    %\vspace{-1.1cm}
    \includegraphics[width = 0.5\textwidth]{fig/generalization.pdf}
    %\vspace{-0.2cm}
    \caption{\textbf{Four types of generalization.}
    An illustration of the four types of generalization.}
    \label{fig:framework}
    %\vspace{-0.3cm}
\end{figure}

In meta-learning, the goal is to train a model that is able to rapidly adapt to new tasks.  
Testing on new tasks can be considered as measuring the OOD generalization.
Under our ICL-HCG framework, we consider four types of OOD generalizations.
First, we examine whether the learner generalizes to a new testing hypothesis class (the hypothesis class is unseen during training) that may or may not contain hypotheses seen during training, referred to as in-distribution (ID) and out-of-distribution (OOD) hypothesis class generalization, respectively.

\begin{definition}[ID Hypothesis Class Generalization]
\label{def:SpaceGeneralization}
Given $\mathcal{H}^{\mathrm{uni}}$ of size $M$, we enumerate all $C(M, m) = \frac{M!}{m!(M - m)!}$ distinct hypothesis classes, each containing $m$ 
hypotheses.
We then randomly \emph{subsample} these classes into disjoint training and testing subsets, ensuring that no testing hypothesis class appears in the training set (although individual hypotheses may overlap).
By training on randomly selected training hypothesis classes and evaluating on unseen testing hypothesis classes, we assess generalization to new hypothesis classes consisting of ID hypotheses.
\end{definition}

\begin{definition}[OOD Hypothesis Class Generalization]
\label{def:HypothesisGeneralization}
Given $\mathcal{H}^{\text{uni}}$ of size $M$, 
we partition it into disjoint training and testing subsets of sizes 
$M^\text{ID}$ and $M^\text{OOD}$, respectively. 
We then generate training hypothesis classes from $M^\text{ID}$ and testing hypothesis classes from $M^\text{OOD}$, each containing $m$ hypotheses. 
We train the learner on the training hypothesis classes and evaluate on the testing hypothesis classes. 
Because no testing hypothesis appears during training, 
this setup probes how well the learner generalizes
to entirely new hypotheses, \ie, OOD hypotheses.
\end{definition}

We then consider whether the learner can generalize to hypothesis classes of various sizes. Building on the concepts of ID and OOD hypothesis class generalization, we introduce size generalizations as follows.

\begin{definition}[ID Hypothesis Class Size Generalization]
\label{def:Space&SizeGeneralization}
Building on the setting of ID hypothesis class generalization, while maintaining non-identical training and testing hypothesis classes, we allow training hypothesis class to include various number of hypotheses $m\in\mathcal{M}\subsetneqq[L]$.
We investigate whether the learner can perform well on hypothesis classes with other sizes $m\in [L]\setminus\mathcal{M}$, where $[L]=\{1,2,\ldots,L\}$.
\end{definition}

\begin{definition}[OOD Hypothesis Class Size Generalization]
\label{def:Hypothesis&SizeGeneralization}
Based on the setting of OOD hypothesis class generalization, while maintaining non-identical training and testing hypotheses, we allow training hypothesis class to include various number of hypotheses $m\in\mathcal{M}\subsetneqq[L]$.
We investigate whether the learner can perform well on hypothesis classes with various sizes $m\in [L]\setminus\mathcal{M}$, where $[L]=\{1,2,\ldots,L\}$.
\end{definition}



\begin{figure}[h!]
    \centering
    %\vspace{-1.1cm}
    \includegraphics[width = 0.475\textwidth]{fig/framework_simplified.pdf}
    %\vspace{-0.2cm}
    \caption{\textbf{Learning ICL-HCG via Transformer.}
    We begin by sampling a subset from the hypothesis universe as the hypothesis class $\mathcal{H}$.
    Next, we encode the hypothesis class $\mathcal{H}$ and concatenate it with context query into a unified sequences of token.
    This sequences is fed into a Transformer model for training with next-token prediction, and testing for evaluating the accuracy on $y$ and hypothesis identification.
    (This figure is an simplified illustration. Please refer to Appendix~\ref{app:prefix} and Fig.~\ref{fig:frameworkfull} for the full details.)}
    \label{fig:framework}
    %\vspace{-0.3cm}
\end{figure}
\subsection{Learning ICL-HCG via Transformer}
\label{subsec:framework}
This section details how Transformer learns ICL-HCG.
As shown in Fig.~\ref{fig:framework},
the hypothesis class $\mathcal{H}$ is first converted to a hypothesis prefix with randomly assigned hypothesis indexes, then concatenated with context query representing sequence $\SK$ as a unified sequence $s$.
%We list the pseudo algorithm of training and testing in the Appendix Algorithm~\ref{alg:framework}, namely ``Meta Training and Testing of ICL-HCG.'' 

% During training, $\mathcal{D}$ is constructed by $(x,h(x))$ pairs via uniformly randomly sampling $x$ from $\mathcal{X}$; while
% during testing, we consider two sample strategies of $(x,h(x))$ pairs in the context query: (i) ``i.i.d.'': the same as training; (ii) ``Opt-T'': given a hypothesis class $\mathcal{H}$ and a chosen hypothesis $h\in\mathcal{H}$, an optimal teaching set\footnote{An optimal teaching set is a set with the minimum number of samples to identify a hypothesis from the hypothesis class.} is derived and duplicated to number $K$ to fit the context query size $K$.
% \begin{definition}[Hypothesis Table]
%     Given $\mathcal{A}$ and hypothesis class $\mathcal{H}$ with $m$ $h$s on the domain, the corresponding hypothesis table is constructed with size $m\times |\mathcal{A}|$ such that each row in the table represents a hypothesis, each column indicates an $a\in\mathcal{A}$, and each value in the table represents the value of $h_m(x_n)$ in row $m$ and column $n$.
% \end{definition}

\paragraph{Hypothesis prefix\footnote{Please refer to Appendix~\ref{app:prefix} for the full version.}}
Given a hypothesis class $\mathcal{H}=\{h_4,h_6,h_7\}$, its hypothesis prefix with size $L=4$ is constructed as shown in Fig.~\ref{fig:framework}.
Blank hypothesis is used to fill the hypothesis prefix when $|\mathcal{H}|<L$.
A randomly assigned hypothesis index token \hz is used to label each hypothesis.
Leveraging Fig.~\ref{fig:framework} for $L=4$, {\hz}'s are assigned from a pool \{``\textcolor[RGB]{0,176,80}{A}'',``\textcolor[RGB]{0,176,80}{B}'',``\textcolor[RGB]{0,176,80}{C}'',``\textcolor[RGB]{0,176,80}{D}''\} of size $L$ without replacement\footnote{We use variable $z$ to represent the hypothesis index, and create a set of $L$ hypothesis index tokens as a pool from which each hypothesis is randomly assigned a unique index without replacement.}.

\paragraph{Context query}
Given an ICL sequence $\SK$, we append a query token ``\textcolor[RGB]{192,79,21}{\textgreater}'' after it to trigger trigger the prediction of the hypothesis index ss shown in Fig.~\ref{fig:framework}.
We name the combination of $\SK$ and ``\textcolor[RGB]{192,79,21}{\textgreater}'' as context query.

The Transformer predicts the $y$ tokens in the context query based on previous tokens and the index \hz of the underlying hypothesis based on all tokens in the sequence.
The training loss in Eq.~\ref{eq:loss} is further extended to all the tokens in the sequence and implemented as below:
\begin{align}
\label{eq:lossTF}
    \mathcal{L} = - \sum_{t=1}^{T} \log P_\theta(s_i \mid s_{<i}).
\end{align}
We summarize the pipeline in the Appendix~\ref{app:alg} Algorithm~\ref{alg:framework}.
%\section{Pseudo Algorithm for ICL-HCG}
\label{app:alg}
We summarize our meta framework for ICL-HCG in Algorithm~\ref{alg:framework}.
\begin{algorithm}
  \caption{Meta-Learning Framework for ICL-HCG}
  \label{alg:framework}
  \begin{algorithmic}[1]
    \STATE \textbf{Inputs:} a set of inputs $\mathcal{X}$, a training set of hypothesis classes $\mathcal{S}^{\text{train}}=\{\mathcal{H}_i^{\text{train}}\}_{i=1}^{N^{\text{train}}}$, a testing set of hypothesis classes $\mathcal{S}^{\text{test}}=\{\mathcal{H}_i^{\text{test}}\}_{i=1}^{N^{\text{test}}}$, batch size $B$, hypothesis prefix size $L$, and context query size $K$
    \FOR{\textbf{training epoch}}
      \STATE sample $\{\mathcal{H}_i\}_{i=1}^B \overset{\text{i.i.d.}}{\sim} \text{Uniform}(\mathcal{S}^{\text{train}})$
      \FOR{each hypothesis class $\mathcal{H} \in \{\mathcal{H}_i\}_{i=1}^B$}
        \STATE generate $h,\SK$ following \hyperref[asu:iid]{i.i.d. Generation}
        
        \STATE \textbf{// Construct sequence based on $\mathcal{H}$, $h$, and $\SK$}
        \STATE construct hypothesis prefix, context query, and hypothesis index $z$ based on $\mathcal{H}$, $h$, $\SK$
        \STATE $s \gets \text{concatenate}(\text{hypothesis prefix}, \text{context query}, z)$
    
        \STATE \textbf{// Cross-entropy loss for next token prediction}
        \STATE $\mathcal{L} \gets -\sum_{t=2}^{|s|} \log P(s_t \mid s_{<t})$
      \ENDFOR
      \STATE update model parameters using $\mathcal{L}$ of the batch
    \ENDFOR
    
    \FOR{\textbf{testing epoch}}
      \STATE sample $\{\mathcal{H}_i\}_{i=1}^B \overset{\text{i.i.d.}}{\sim} \text{Uniform}(\mathcal{S}^{\text{test}})$
      
      \FOR{each hypothesis class $\mathcal{H} \in \{\mathcal{H}_i\}_{i=1}^B$}
        \STATE generate $h,\SK$ via:
        \STATE \quad \textbf{either} following \hyperref[asu:iid]{i.i.d. Generation}
        \STATE \quad \textbf{or} following \hyperref[asu:optt]{Opt-T Generation}
        \STATE \textbf{construct sequence $s$ based on $\mathcal{H}$, $h$, and $\SK$}
        \STATE \textbf{evaluate the prediction accuracy on $y$, $z$, etc}
      \ENDFOR
    \ENDFOR
  \end{algorithmic}
\end{algorithm}



%\subsection{Empirical Risk Minimization for Binary Classification (come later, what is the data ERM for a fix, fix hypothesis)}
\label{sec:ERM}
In this work, we consider the binary classification problem where the input space is finite. Formally, let the input space be defined as
$
\mathcal{X} = \{x_1, x_2, \ldots, x_{|\mathcal{X}|}\}
$,
and the label space as
$
\mathcal{Y} = \{0, 1\}
$.
The learning task is characterized by a hypothesis space
$
\mathcal{H} \subseteq \mathcal{Y}^{\mathcal{X}}
$,
where each hypothesis $ h \in \mathcal{H} $ is a deterministic function
$
h: \mathcal{X} \rightarrow \mathcal{Y}
$.
A realizable instance of the binary classification problem is specified by the tuple
$
(\mathcal{X}, \mathcal{Y}, P, h^*)
$,
where $P$ is a probability distribution over the input space $\mathcal{X}$, and $h^*:\mathcal{X} \rightarrow \mathcal{Y}$ is the target hypothesis.

During learning, a learner $A$ receives a training dataset $\mathcal{D}=\{(x^{(i)}, y^{(i)})\}_{i=1}^K$ and it is tasked to find a hypothesis that produces least generalization risk $R(A(\mathcal{D}))$ on future test sample drawn from $P$ and $h^*$.
Empirical risk minimization (ERM) is one of the most popular learning algorithm. It learns by minimizing the empirical risk on the training dataset $\mathcal D$ with respect to a loss function $l:\mathcal{Y} \times \mathcal{Y} \rightarrow \mathbb R^+$ and is given as,
$$
\hat h_\mathcal{D} = \text{ERM}(\mathcal{D},\mathcal{H}) \leftarrow \min_{h \in \mathcal{H}} \sum_{i=1}^K l(h(x^{(i)}), y^{(i)}).
$$
% Once learning is complete, the learner uses learnt hypothesis $\hat h_D$ to make prediction on samples from $P, h^*$ thereby suffering a risk of, $$R(\hat h_D) \leftarrow \mathbb E_{(x, y) \sim P, h^*}[l(\hat h_D(x), y)].$$
In this paper, we focus on $0/1$ risk, $$l(\hat{y},y)=
\begin{cases}
    1, & \text{if } \hat{y}=y\\
    0, & \text{if } \hat{y}\neq y
\end{cases}.
$$
\begin{remark}
ERM focuses mainly on sample complexity, that is, determining the minimum number of samples required to identify a hypothesis $\hat{h}$ with a risk below a specified threshold.
Differently, in this paper, we explore whether Transformer models can accurately pinpoint the exact hypothesis when provided with sufficient number of samples.
\end{remark}

% In theory, Probably Approximately Correct(PAC) learning completely characterizes the sample complexity of learning. It states that for any learning instance and $(\epsilon, \delta) \in (0,1)^2$ if $n\geq O(\frac{1}{\epsilon}(d(\mathcal H)+\log(\frac{1}{\delta})))$, then,  $$\text{w.p. $\geq 1-\delta$}, \qquad R(\hat h_D) \leq \epsilon.$$
% where $d(\mathcal H)$ is the VC dimension of hypothesis class $\mathcal H$ which is a measure of its complexity. Therefore, to learn the target hypothesis up to an error $\epsilon$, computing the ERM on an i.i.d. sample of size $O(\frac{d}{\epsilon})$ is sufficient.


%%% motivating incontext learning with finite hypothesis class

% In recent literature, several works {cite} have shown impressive capability of transformers to in-context learn ERM on a dataset $D$ at inference time which is known as In-context Learning(ICL). However, most of the works have focused on ICL learning settings where hypothesis class is defined implicitly by a linear $\mathbb R^d$ space. This begs the question of whether transformers can do ERM in more explicit learning settings where hypothesis class are specified as input to the model. Towards this end we seek to answer the following question:

% \textbf{Can a transformers learn to do in-context learning conditioned on finite/tabular hypothesis classes?}

% A couple of recent works have tried to address this problem by conditioning the hypothesis class on a noisy version of the target hypothesis {cite}. Contrary to this, a finite hypothesis class conditions the universal hypothesis space $\mathcal{A}^\mathcal{Y}$ by specifying a subset of hypothesis $\mathcal H$ that contains a the target hypothesis to solve the task instance. In real world learning scenario, this can be viewed   The explicit conditioning on finite hypothesis class also provide an avenue to test the generalization capability of ICL to generalize to different size of hypothesis tables thereby helping us to understand ERM capability of transformers in a more controlled setup. 

%%% Ziqian's previous writeup

% \subsection{Empirical Risk Minimization (ERM)}
% Consider a training dataset $\{(x_i, y_i)\}_{i=1}^n$, where each $x_i \in \mathcal{A}$ is a feature vector and $y_i \in \mathcal{Y}$ is the corresponding label or target value. Let $\mathcal{H}$ be a hypothesis class of functions $h: \mathcal{A} \to \mathcal{Y}$, and let $L:\mathcal{Y}\times\mathcal{Y}\to\mathbb{R}_{\geq 0}$ be a loss function that measures the discrepancy between the predicted value $h(x_i)$ and the true value $y_i$.

% The \emph{empirical risk} of a hypothesis $h \in \mathcal{H}$ is defined as:
% \[
% \hat{R}_n(h) = \frac{1}{n} \sum_{i=1}^n L(y_i, h(x_i)).
% \]

% The \emph{Empirical Risk Minimization} principle seeks a hypothesis $\hat{h} \in \mathcal{H}$ that minimizes the empirical risk:
% \[
% \hat{h} = \arg\min_{h \in \mathcal{H}} \hat{R}_n(h).
% \]



% In order to have a set of hypotheses to construct $\mathcal{H}$, given $\mathcal{X}$, assuming $|\mathcal{Y}|=2$ (binary classification), we can generate $2^{|\mathcal{X}|}$ hypotheses (all possible hypotheses with binary labels), then uniformly randomly sample $m$ hypotheses from the set to construct a hypothesis class.
% Each space generalization with $m$ hypotheses can be further converted to $P(L,m) = \frac{L!}{(L - m)!}$ hypothesis prefixes.
% Then, $(x,y)$ pairs generated from a hypothesis $h$ sampled from the hypothesis class $\mathcal{H}$ are provided as dataset of context query for the Transformer to identify a correct hypothesis indexed \hz and predict the index.

% With this framework, we are interested in whether a trained Transformer is able to generalized to perform ERM on new hypothesis classes.
% Starting with introducing the hypothesis universe, we further deliver the four generalization cases considered in our experiments.


% \section{Experiments}

\section{Analysis}

\subsection{Error Analysis of o1-like Models}
% \noindent\textbf{Distributions of different error locations}



\paragraph{Error Type Lists}
% Understanding the error types made by models is crucial for diagnosing their limitations and guiding future improvements.
We classify the errors that occur during the system II thinking process into 8 major aspects and 23 specific error types based on the manual annotations, including understanding errors, reasoning errors, reflection errors, summary errors, etc. For detailed information about the error categories, see Appendix \ref{app: error_classification}.

\paragraph{What Are the Most Common Errors Across Domains?}

\begin{figure}[t]
    \centering
    \resizebox{1.0\textwidth}{!}
    {\includegraphics{figures/error_type_distribution.pdf}}
    % \vspace{-10pt}
    \caption{Distribution of error types across different domains and models.}
    % \vspace{-3mm}
    \label{fig: error_type}
\end{figure}

To analyze the characteristics of error distribution in different domains, we performed a uniform sampling of the data based on the model, the domain, and the query difficulty. Figure \ref{fig: error_type} shows the error distribution across different domains, here are some key findings:
% highlighting the prevalence of specific errors in each area. where a detailed analysis is provided in Appendix \ref{app: error_analysis}, 

\begin{itemize}[left=1em]
\item \textbf{Math:} The most frequent error type is \textit{Reasoning Error}(25.3\%), followed by \textit{Understanding Error}(15.7\%) and \textit{Calculation Error}(15.4\%). This indicates that while the models often struggle with logical reasoning and problem understanding, low-level computational mistakes also remain a significant issue.

\item \textbf{Programming}: 
\textit{Reasoning Error} (21.5\%) is the most common, followed by \textit{Formal Error} (16.7\%) and \textit{Understanding Error} (12.6\%). The high frequency of \textit{Formal Error} and \textit{Programming Error} (11.8\%) underscores the models' struggles with code-specific details and implementation. 

\item \textbf{PCB}: 
The dominant error types are \textit{Understanding Error} (20.4\%) and \textit{Knowledge Error} (17.3\%), closely followed by \textit{Reasoning Error} (17.3\%). This suggests that the main challenge for current models in the fields of physics, chemistry and biology is to understand field-specific concepts and accurately apply relevant knowledge.

\item \textbf{General Reasoning}: \textit{Reasoning Error} is the most prevalent, accounting for 43\%, followed by comprehension errors, accounting for 19\%, showing that logical reasoning is the primary bottleneck.

\end{itemize}

\paragraph{What Are the Model-Specific Error Patterns?}

% \begin{figure}[t]
%     \centering
%     \includegraphics[width=0.8\textwidth]{figures/error_type_model.pdf}
%     % \vspace{-3mm}
%     \caption{Distribution of Error Types Across Models.}
%     % \vspace{-3mm}
%     \label{fig: error_type_model}
% \end{figure}

We also analyzed errors specific to individual models, providing further insights into model weaknesses, as illustrated in Figure \ref{fig: error_type_model}. The error distributions reveal distinct patterns for each model, highlighting their unique strengths and areas for improvement. Here are some key findings:
%Due to space constraints, we focus here on the key findings from the most commonly used models, with a comprehensive analysis of all models provided in Appendix \ref{app: error_analysis}.

\begin{itemize}[leftmargin=4mm]

\item \textbf{DeepSeek-R1} exhibits its most pronounced weakness in \textit{Reasoning Errors} (22.7\%), indicating challenges in constructing coherent and accurate logical reasoning paths. However, it demonstrates relative strength in handling fundamental tasks, with minimal \textit{Calculation Errors} (3.1\%) and \textit{Programming Errors} (4.4\%).

%achieves strong performance in detail-oriented tasks such as formula computation and code syntax. Its primary limitation lies in reasoning and comprehension capabilities.

\item \textbf{QwQ-32B-Preview} excels at identifying correct problem-solving approaches. However, its effectiveness is significantly hindered by deficiencies in handling finer details, particularly in \textit{Calculation Errors} (17.9\%)

%but its effectiveness is often undermined by deficiencies in handling finer details.

% {QwQ-32B-Preview} demonstrates a relatively balanced performance but is notably weak in \textit{Calculation Errors} (17.9\%), indicating a significant limitation in numerical precision. It also shows a moderate frequency of \textit{Understanding Errors} (17.1\%), suggesting occasional difficulties in problem interpretation. 

\end{itemize}

\begin{tcolorbox}[colback=white!95!gray, colframe=gray!70!black,  title=Key Finding for Error Type]
The primary bottleneck of current models remains reasoning ability. However, detailed errors like calculation and formal mistakes also contribute significantly.
\end{tcolorbox}


\subsection{Reflection Analysis of o1-like Models}


\begin{figure}[t]
    \centering
    \includegraphics[width=0.95\textwidth]{figures/reflection.pdf}
    \caption{Distribution of effective reflection times by models and domains on a sample level. The segments within each pie chart represent how many times effective reflection occurs in one sample, with segment `0' indicating there is no effective reflection.}
    \label{fig: error_type_model}
\end{figure}

\paragraph{Statistics.}
We also conduct a analysis of the total number of reflections and the proportion of effective reflections in the long CoT output of all questions (including questions answered correctly and incorrectly by the model). 
% On average, 
%We observe that the long CoT contains \textit{five} times reflections, indicating that current o1-like models tend to reflect frequently. 

\paragraph{How Effective Are Model Reflections Across Different Models and Domains?}
We classify samples with reflections based on the number of valid reflections to evaluate the ability to produce valid reflections. Specifically, we label samples as \texttt{0} if no valid reflections occur, and \texttt{1}, \texttt{2}, or \texttt{>=3} for samples with one, two, or three and more valid reflections, respectively(all statistical analyses were performed under strictly controlled conditions, ensuring uniform sampling and balanced tasks for a fair comparison). In Figure \ref{fig: error_type_model}, {DeepSeek-R1} exhibits the highest proportion of effective reflections, and the models show a notably higher rate of effective reflections in the {math} domain. However, the overall proportion of valid reflections across all models remains relatively low, ranging between 30\% and 40\%. This suggests that the reflection capabilities of current models require further improvement.
%Detailed statistical data can be found in Appendix D.

\begin{tcolorbox}[colback=white!95!gray, colframe=gray!70!black,  title=Key Finding for Reflection]
Despite frequent reflection attempts, the proportion of effective reflections remains low across models, and  DeepSeek-R1 achieves the highest rate of valid reflections.
\end{tcolorbox}

\subsection{Effective Reasoning of o1-like Models}

\begin{figure}[t]
    \centering
    \includegraphics[width=0.98\textwidth]{figures/effetive_reasoning.pdf}
    \caption{Distribution of effective reasoning ratios.}
    
    \label{fig: effetive_reasoning}
\end{figure}

\paragraph{Statistics.} 
% As previously mentioned, 
Human annotators evaluate the usefulness of the reasoning in each section, enabling us to calculate the proportion of valid reasoning in each response. As illustrated in Figure \ref{fig: effetive_reasoning}, each graph shows the distribution of effective reasoning ratios for a particular model. The red dashed line in each graph indicates the average effective reasoning ratio.

\paragraph{What Proportion of Reasoning in Long CoT Responses is Effective?}
On average, only 73\% of the reasoning in the collected long CoT responses is useful, highlighting significant redundancy issues. Among the models analyzed, \textit{QwQ-32B-Preview} exhibited the lowest proportion of effective reasoning at 70\%, while \textit{DeepSeek-R1} achieved a notably higher proportion compared to the others, demonstrating superior reasoning efficiency.


\begin{tcolorbox}[colback=white!95!gray, colframe=gray!70!black,  title=Key Finding for Reasoning Efficiency]
On average, 27\% of reasoning in long CoT responses we collected is redundant, and DeepSeek-R1 outperforms others in reasoning efficiency.
\end{tcolorbox}
\vspace{-3mm}

\subsection{Reasoning Process Analysis}

Figure ~\ref{fig: action_roles} shows the distribution of each section's action roles in the system II thinking process of the o1-like models. Initially, problem analysis dominates, indicating that the model initially focuses on understanding the requirements and constraints of the problem. As the solution progresses, cognitive activities diversify significantly, with reflection and validation becoming more prominent. In the later part of the reasoning, the distribution of conclusion and summarization gradually increases. 
%As the model progresses from problem analysis, solution implementation and conclusion, it demonstrates the common reasoning template of o1-like models.


\begin{figure}[t]
    \centering
    \includegraphics[width=0.8\textwidth]{figures/action_role.pdf}
    \caption{Distribution of different task types throughout the progress of a long CoT response.}
    \vspace{-3mm}
    
    \label{fig: action_roles}
\end{figure}
\subsection{Results on DeltaBench}

% Please add the following required packages to your document preamble:
% \usepackage{multirow}
\begin{table*}[!t]
\centering
\resizebox{1.0\textwidth}{!}{%
    \begin{tabular}{cccccccccccccccc}
    \toprule
    \multirow{2}{*}{\textbf{Model}} & \multicolumn{3}{c}{\textbf{Overall}} & \textbf{Math} & \textbf{Code} & \textbf{PCB} & \textbf{General} \\
    \cmidrule(lr){2-4} \cmidrule(lr){5-5} \cmidrule(lr){6-6} \cmidrule(lr){7-7} \cmidrule(lr){8-8}
     & \textbf{\textit{Recall}} & \textbf{\textit{Precision}} & \textbf{\textit{F1}} & \textbf{\textit{F1}} & \textbf{\textit{F1}} & \textbf{\textit{F1}} & \textbf{\textit{F1}} \\
    \midrule
    \multicolumn{8}{c}{\textbf{\textit{Process Reward Models (PRMs)}}} \\
    \midrule
    \rowcolor[rgb]{ .988,  .949,  .8} Qwen2.5-Math-PRM-7B & \textbf{30.30} & \textbf{34.96} & \textbf{29.22}  &  \textbf{29.64} & \textbf{23.76} & \underline{31.09} & \underline{34.19}   \\
    \rowcolor[rgb]{ .988,  .949,  .8} Qwen2.5-Math-PRM-72B & \underline{28.16} & \underline{29.37} & \underline{26.38}  & \underline{24.16} & \underline{22.02} & \textbf{31.14} & \textbf{35.83}  \\
    \rowcolor[rgb]{ .988,  .949,  .8} Llama3.1-8B-PRM-Deepseek-Data & 11.7 & 15.59 & 12.02 &  12.28 & 10.95 & 16.76 & 12.59  \\
    \rowcolor[rgb]{ .988,  .949,  .8} Llama3.1-8B-PRM-Mistral-Data & 9.64 & 11.21 & 9.45 & 9.40 & 10.72 & 13.43 & 12.40  \\
    \rowcolor[rgb]{ .988,  .949,  .8} Skywork-o1-Qwen-2.5-1.5B & 3.32 & 3.84 & 3.07 & 1.30 & 6.66 & 5.43 & 7.87  \\
    \rowcolor[rgb]{ .988,  .949,  .8} Skywork-o1-Qwen-2.5-7B & 2.49 & 2.22 & 2.17 & 0.78 & 6.28 & 6.02 & 3.11  \\
    \midrule
     \multicolumn{8}{c}{\textbf{\textit{LLM as Critic Models}}} \\
    \midrule
    \rowcolor[rgb]{ .922,  .89,  .988} GPT-4-turbo-128k & \textbf{57.19} & \textbf{37.35} & \textbf{40.76} & \textbf{37.56} & \textbf{43.06} & \underline{45.54} & \underline{42.17} \\
    \rowcolor[rgb]{ .922,  .89,  .988} GPT-4o-mini & \underline{49.88} & 35.37 & \underline{37.82} & \underline{33.26} & 37.95 & \textbf{45.98} & \textbf{46.39} \\
    \rowcolor[rgb]{ .922,  .89,  .988} Doubao-1.5-Pro & 39.68 & \underline{37.02} & 35.25 & 32.46 & \underline{39.47} & 33.53 & 37.00 \\
    \rowcolor[rgb]{ .922,  .89,  .988} GPT-4o & 36.52 & 32.48 & 30.85 & 28.61 & 28.53 & 39.25 & 36.50 \\
    \rowcolor[rgb]{ .922,  .89,  .988} Qwen2.5-Max & 36.11 & 30.82 & 30.49 & 26.73 & 32.81 & 39.49 & 29.54 \\
    \rowcolor[rgb]{ .922,  .89,  .988} Gemini-1.5-pro & 35.51 & 30.32 & 29.59 & 26.56 & 28.20 & 40.13 & 33.66 \\
    \rowcolor[rgb]{ .922,  .89,  .988} DeepSeek-V3 & 32.33 & 28.13 & 27.33 & 27.04 & 27.73 & 27.35 & 27.45 \\
    \rowcolor[rgb]{ .922,  .89,  .988} Llama-3.1-70B-Instruct & 32.22 & 28.85 & 27.67 & 21.49 & 32.13 & 28.45 & 39.18 \\
    \rowcolor[rgb]{ .922,  .89,  .988} Qwen2.5-32B-Instruct & 30.12 & 28.63 & 26.73 & 22.34 & 31.37 & 33.78 & 24.37 \\
    \rowcolor[rgb]{ .882,  .949,  .89} DeepSeek-R1 & 29.20 & 32.66 & 28.43 & 24.17 & 29.28 & 34.78 & 35.87 \\
    \rowcolor[rgb]{ .882,  .949,  .89} o1-preview & 27.92 & 30.59 & 26.97 & 22.19 & 28.09 & 33.11 & 35.94 \\
    % Gemini-2.0-flash-thinking & 14.02 & 17.36 & 14.56 & 14.79 & 11.97 & 19.34 & 15.26 \\
    \rowcolor[rgb]{ .922,  .89,  .988} Qwen2.5-14B-Instruct & 26.64 & 27.27 & 24.73 & 21.51 & 29.05 & 29.98 & 20.59 \\
    \rowcolor[rgb]{ .922,  .89,  .988} Llama-3.1-8B-Instruct & 25.71 & 28.01 & 24.91 & 18.12 & 32.17 & 27.30 & 29.93 \\
    \rowcolor[rgb]{ .882,  .949,  .89} o1-mini & 22.90 & 22.90 & 19.89 & 16.71 & 21.70 & 20.37 & 26.94 \\
    \rowcolor[rgb]{ .922,  .89,  .988} Qwen2.5-7B-Instruct & 21.99 & 19.61 & 18.63 & 11.61 & 25.92 & 29.85 & 15.18 \\
    \rowcolor[rgb]{ .882,  .949,  .89} DeepSeek-R1-Distill-Qwen-32B & 17.19 & 18.65 & 16.28 & 13.02 & 23.55 & 15.05 & 11.56 \\
    % Gemini-2.0-flash-thinking & 14.02 & 17.36 & 14.56 & 14.79 & 11.97 & 19.34 & 15.26 \\
    \rowcolor[rgb]{ .882,  .949,  .89} DeepSeek-R1-Distill-Qwen-14B & 12.81 & 14.54 & 12.55 & 9.40 & 18.36 & 10.44 & 12.01 \\
    % \rowcolor[rgb]{ .882,  .949,  .89} QwQ-32B-Preview & 10.20 & 10.17 & 9.07 & 7.38 & 8.60 & 14.97 & 10.54 \\
    \bottomrule
    \end{tabular}
}
\caption{Experimental results of PRMs and critic models on DeltaBench. \textbf{Bold} indicates the best results within the same group of models, while \underline{ underline} indicates the second best.}
% \vspace{-4mm}
\label{tab: main}
\end{table*}

% \noindent\textbf{Evaluation Metrics.}
% % To accurately assess the performance of the PRM and critic models on DeltaBench, 
% We employ \textbf{recall}, \textbf{precision}, and \textbf{macro-F1 score} for error sections as evaluation metrics. For the PRMs, we utilize an outlier detection technique based on the Z-Score to make predictions. This method was chosen because threshold-based prediction methods determined from other step-level datasets, such as those used in ProcessBench~\citep{Zheng2024ProcessBenchIP}, may not be reliable due to significant differences in dataset distributions, particularly as DeltaBench focuses on long CoT. Outlier detection helps to avoid this bias. The threshold $t$ for determining the correctness of a section is defined as:
% % \begin{align}
% $t = \mu - \sigma$,
% % \nonumber
% % \label{eq: prm_threshold}
% % \end{align}
% where $\mu$ is the mean of the rewards distribution across the dataset, and $\sigma$ is the standard deviation. Sections falling below $t$ are predicted as error sections. For critic models, all erroneous sections within a long CoT are prompted to be identified. Given that error sections constitute a smaller proportion than correct sections across the dataset, we use macro-F1 to mitigate the potential impact of the imbalance between positive and negative sections. Macro-F1 independently calculates the F1 score for each sample
% % (for our metric, each case) 
% and then takes the average, providing a more balanced evaluation metric when dealing with class imbalance.

\noindent\textbf{Baseline Models.}
% 开源(中英模型,llama3)和闭源模型
% To comprehensively evaluate the performance of current PRMs and critic models, we extensively selected and evaluated a wide range of both open-source and closed-source models on DeltaBench.
% \paragraph{Process Reward Models}
For the \textbf{PRMs}, we select the following models: Qwen2.5-Math-PRM-7B\footnote{\href{https://huggingface.co/Qwen/Qwen2.5-Math-PRM-7B}{Qwen/Qwen2.5-Math-PRM-7B}}, Qwen2.5-Math-PRM-72B\footnote{\href{https://huggingface.co/Qwen/Qwen2.5-Math-PRM-72B}{Qwen/Qwen2.5-Math-PRM-72B}}, Llama3.1-8B-PRM-Deepseek-Data\footnote{\href{https://huggingface.co/RLHFlow/Llama3.1-8B-PRM-Deepseek-Data}{RLHFlow/Llama3.1-8B-PRM-Deepseek-Data}}, Llama3.1 -8B-PRM-Mistral-Data\footnote{\href{https://huggingface.co/RLHFlow/Llama3.1-8B-PRM-Mistral-Data}{RLHFlow/Llama3.1-8B-PRM-Mistral-Data}}, Skywork-o1-Open-PRM- Qwen-2.5-1.5B\footnote{\href{https://huggingface.co/Skywork/Skywork-o1-Open-PRM-Qwen-2.5-1.5B}{Skywork/Skywork-o1-Open-PRM-Qwen-2.5-1.5B}}, and Skywork-o1-Open-PRM-Qwen-2.5-7B\footnote{\href{https://huggingface.co/Skywork/Skywork-o1-Open-PRM-Qwen-2.5-7B}{Skywork/Skywork-o1-Open-PRM-Qwen-2.5-7B}}. 
% These represent some of the best open-source PRMs currently available.
% \paragraph{Critic Models}
We select a group of the most advanced open-source and closed-source LLMs to serve as \textbf{critic models} for evaluation, which includes various GPT-4~\citep{gpt4} variants (such as GPT-4-turbo-128K, GPT-4o-mini, GPT-4o), the Gemini model~\citep{Reid2024Gemini1U}(Gemini-1.5-pro), several Qwen models~\citep{qwen2.5} (such as Qwen2.5-32B-Instruct and Qwen2.5-14B-Instruct), Doubao-1.5-Pro~\citep{doubao2025}
and o1 models~\citep{openai-o1} (o1-preview-0912, o1-mini-0912).
% , and a GPT-3.5 variant (gpt-3.5-16K).



\subsubsection{Main Results}
In Table \ref{tab: main},
we provide the results of different LLMs on DeltaBench. 
For PRMs, we have the following observations: (1). Existing PRMs usually achieve low performance, which indicates that existing PRMs cannot identify the errors in long CoTs effectively and it is necessary to improve the performance of PRMs. (2). Larger PRMs
do not lead to better performance. For example, the Qwen2.5-Math-PRM-72B is inferior to wen2.5-Math-PRM-7B.
For critic models, we have the following findings: (1)
GPT-4-turbo-128k archives the best critique results, which is better than other models (e.g., GPT-4o) a lot in DeltaBench. (2) For o1-like models (e.g., DeepSeek-R1, o1-mini, o1-preview), we observe that the results of these models are not superior to non-o1-like models, with the performance of o1-preview is even lower than Qwen2.5-32B-Instruct.
%Additionally, we observe that the QWQ and DeepSeek-R1-Distill series models exhibit weaknesses in following instructions. 
A detailed analysis of underperforming models is provided in Appendix \ref{app: underperforming}.

% model size
% domains
% o1模型跟普通模型critic能力对比分析


\subsubsection{Further Analysis}

\paragraph{Effect of Long CoT Length.}
\begin{figure}[t]
    \centering
    \includegraphics[width=1.0\textwidth]{figures/4.5.1/length2.pdf}
    \caption{The effect of long CoT length.}
    \label{fig: crtic1}
\end{figure}
In Figure \ref{fig: crtic1}, we compare the average F1-Score performance of critic models and PRMs across varying LongCoT token lengths. 
For critic models, the performance notably declines as token length increases. Initially, models like Deepseek-R1 and GPT-4o exhibit strong performance with shorter sequences (1-3k tokens). However, as token length increases to mid-ranges (4-7k tokens), there is a marked decrease in performance across all models. This trend highlights the growing difficulty for critic models to maintain precision and recall as long CoT response become longer and more complex, likely due to the challenge of evaluating lengthy model outputs. In contrast, PRMs demonstrate greater stability across token lengths, as they evaluate sections sequentially rather than processing the entire output at once. Despite this advantage, PRMs achieve lower overall scores compared to critic models on our evaluation set.

\begin{tcolorbox}[colback=white!95!gray, colframe=gray!70!black, title=Key Finding]
  Critic models exhibit significant performance degradation with longer contexts, while PRMs demonstrate consistent evaluation capability across varying lengths.
\end{tcolorbox}


\paragraph{Performance Analysis Across Different Error Types.}
\begin{figure}[t]
    \centering
    \includegraphics[width=0.9\textwidth]{figures/4.5.2/top_models_per_task.pdf}
    \caption{Results of different LLMs on top-5 errors.}
    \label{fig: top_models_per_task}
\end{figure}
Figure \ref{fig: top_models_per_task} shows the performance of different models on the five most common error types. In terms of error types, most models demonstrate the highest accuracy in recognizing calculation errors. Conversely, the recognition of strategy errors is generally the weakest. In terms of models, there is significant variation in the ability of individual models to recognize different error types. For instance, DeepSeek-V3 achieves an F1 of 36\% on calculation errors but only 23\% on strategy errors. Meanwhile, Llama3.1-8B-PRM-Deepseek performs poorly, with an F1 score of 22\% on calculation errors, and shows a significant decline in performance across the other four error types. This highlights the limited generalization capabilities of most models when recognizing various error types.

\begin{tcolorbox}[colback=white!95!gray, colframe=gray!70!black, title=Key Finding]
  Models exhibit strong performance on calculation errors but struggle with strategy errors, revealing limited generalization across error types.
\end{tcolorbox}

\begin{table}[!ht]
    \centering
    % \scriptsize
    % \footnotesize
    \begin{tabular}{cccc}
    \toprule
        \multirow{2}{*}{Model} & \multicolumn{3}{c}{HitRate@$k$ - Avg(\%)} \\ \cline{2-4}
                           & $k=1$ & $k=3$ & $k=5$ \\ 
                           % \hline
                           \midrule
        Qwen2.5-Math-PRM-7B & \textbf{49.15} & \textbf{69.14} & \textbf{83.14} \\
        Qwen2.5-Math-PRM-72B & \underline{41.13} & \underline{62.70} & \underline{75.73} \\ 
        Llama3.1-8B-PRM-Deepseek-Data & 12.63 & 48.62 & 69.78 \\
        Llama3.1-8B-PRM-Mistral-Data & 8.99 & 42.97 & 65.33 \\
        Skywork-o1-Open-PRM-Qwen-2.5-1.5B & 31.90 & 53.82 & 69.23 \\
        Skywork-o1-Open-PRM-Qwen-2.5-7B & 31.58 & 52.59 & 69.16 \\
        % \hline
        \bottomrule
    \end{tabular}
    \vspace{+3mm}
    \caption{Results of HitRate@$k$. Bold and underlined results indicate the best and the second best.}
    % \vspace{-4mm}
\label{tab: hitrate}
\end{table}

\paragraph{Analysis on HitRate evaluation for PRMs.}

\begin{figure}[t]
    \centering
    \includegraphics[width=\textwidth]{figures/prm_rank.pdf}
    % \vspace{-10pt}
    \caption{Ranking of rewards for the first incorrect section for different PRMs.}
    % \vspace{-3mm}
    \label{fig: prm_rank}
\end{figure}

To better measure the ability of PRMs to identify erroneous sections in long CoTs, we use HitRate@$k$ to evaluate PRMs. Specifically, within a sample, we rank the sections in ascending order based on the rewards given by the PRM, select the smallest $k$ sections, and calculate the recall rate for the erroneous sections among them. Specifically, we define the sorted sections as $S = \{s_1, s_2, \ldots, s_n\}$, with $E$ being the set of erroneous sections. We select the top $k$ sections, denoted as $S_k = \{s_1, s_2, \ldots, s_k\}$. The HitRate@$k$ is  calculated as:
\begin{align}
\text{HitRate@}k = \frac{|S_k \cap E|}{\min(k, |E|)}
% \nonumber
\label{eq: hitrate}
\end{align}
In this formula, $|S_k \cap E|$ indicates the number of erroneous sections identified among the top $k$ sections. This metric reflects the ability of PRMs to effectively identify erroneous sections within the top $k$ candidate sections. In Table \ref{tab: hitrate}, the relative performance rankings among different PRMs are quite similar to the results in Table \ref{tab: main}. Additionally, we observe that for $k=3$ and $k=5$, the performance differences between various PRMs are not particularly significant. However, when $k=1$, the Qwen2.5-Math-PRM-7B shows a clear performance advantage. Figure \ref{fig: prm_rank} illustrates the ranking ability of different PRMs for the first incorrect section within the sample, which is generally consistent with the performance evaluation results of HitRate@k.
% This is because a smaller $k$ value imposes stricter requirements on the PRM's ability to identify errors.

% HitRate@$k$ evaluates the performance of PRMs from the perspective of reward ranking, providing additional evidence for the experimental results and conclusions in Table \ref{tab: main} from a different angle.

\begin{tcolorbox}[colback=white!95!gray, colframe=gray!70!black, title=Key Finding]
  HitRate@k evaluation aligns with the main results, with Qwen2.5-Math-PRM-7B demonstrating superior performance in identifying the first incorrect section.
\end{tcolorbox}


\begin{figure}[t]
    \centering
    \includegraphics[width=0.8\textwidth]{figures/4.5.4/self-critic.pdf}
    % \vspace{-10pt}
    \caption{F1-score comparison of self-critique and cross-model critique abilities for different models.}
    % \vspace{-5mm}
    \label{fig: self-critic}
\end{figure}

\paragraph{Comparative Analysis of Self-Critique Capabilities of LLMs.} We randomly sample queries based on domains and models that generate the long CoT output, followed by a statistical analysis of the model's performance in evaluating its own outputs as well as those of other models. In Figure \ref{fig: self-critic},  Gemini 2.0 Flash Thinking, DeepSeek-R1, and QwQ-32B-Preview show lower self-critique scores compared to their cross-model critique scores, indicating a prevalent deficiency in self-critic abilities. Notably, DeepSeek-R1 exhibits the largest discrepancy, with a 36\% decrease in self-evaluation compared to evaluations of other models. This suggests models' self-critic abilities remain underdeveloped.
% signaling an area that requires improvement.

\begin{tcolorbox}[colback=white!95!gray, colframe=gray!70!black, title=Key Finding]
  LLMs demonstrate weaker self-critique performance compared to cross-model critique, highlighting a fundamental limitation in self-critic capabilities.
\end{tcolorbox}



%%%

% \noindent\textbf{Performance Analysis Across Different Categories}

% \begin{figure}[htbp]
% \centering
% \includegraphics[width=\linewidth]{figures/prm_task_comparison.pdf}
% \caption{Performance of PRMs across different categories (outlier detection).}
% \label{fig: prm_task}
% % \vspace{-0.6cm}
% % \vspace{-4mm}
% \end{figure}


% \noindent\textbf{Performance Variation in Different Lengths of Long CoT}

% \noindent\textbf{Performance Analysis Across Different Error Types}

% \noindent\textbf{Analysis of In-Sample Reward Ranking}


% % \subsection{Evaluation Metrics}

% % \subsection{Main Results}

% % \subsection{Further Analysis}
% \subsection{Analysis on LLM Critics}
%  \textbf{error location}



% \subsubsection{The Performance across different domains}

% \begin{figure}[t]
%     \centering
%     \includegraphics[width=0.5\textwidth]{figures/critic6.pdf}
%     \caption{The score distributions across different domains.}
%     \label{fig: crtic2}
% \end{figure}

% In Figure \ref{fig: crtic2}, we illustrate the F1-score distribution of various large language models (LLMs) across different domains. Analyzing model performance across domains reveals that most models demonstrate stronger critiquing abilities in Physics, Chemistry, Biology, and General Reasoning compared to Mathematics and Programming, indicating higher proficiency in scientific and general reasoning tasks. Meanwhile, the performance of each model varies significantly depending on the domain, reflecting inherent strengths and weaknesses in handling different tasks. For instance, the Gemini-1.5-Pro model achieves an F1-score of 40.1\% in PCB, yet only 26.6\% in Mathematics. These discrepancies underscore challenges in the models' generalization capabilities.






\section{Conclusion}
In this paper, we introduced \textsc{mo-cbo} as a new problem class in order to optimize multiple target variables within a known causal graph by sequentially performing interventions on the system. We proved that a \textsc{mo-cbo} problem can be decomposed into a collection of $|\mathbb{P}(\mathbf{X})| = 2^{|\mathbf{X}|}$ local problems, and solving it essentially corresponds to solving these local problems. To reduce the search space, we derived theoretical results that identify possibly Pareto-optimal minimal intervention sets in a given causal graph. We proved that these sets comprise a minimal collection of local problems that are guaranteed to contain the optimal solutions of any \textsc{mo-cbo} problem. Moreover, we introduced \textsc{Causal ParetoSelect} as an algorithm that iteratively selects and solves local \textsc{mo-cbo} problems in the reduced search space based on relative hypervolume improvement.

Our theoretical and empirical findings highlight two distinct cases: When no unobserved confounders exist between target variables and their ancestors, both \textsc{mo-cbo} and \textsc{mobo} can recover the ground-truth causal Pareto front. However, our approach demonstrates greater cost efficiency while constructing a more diverse set of solutions. In contrast, when unobserved confounders are present between targets and their ancestors, traditional \textsc{mobo} approaches can fail to approximate the ground truth, whereas \textsc{mo-cbo} demonstrates efficient discovery of Pareto-optimal solutions. This occurs because unobserved confounders can propagate effects through the causal graph, and naively disrupting these paths can lead to suboptimal solutions.

In our algorithm, the surrogate model assumes independent outcomes which may limit efficiency since it overlooks shared endogenous confounders. Future work could enhance cost effectiveness by integrating multi-task Gaussian processes to better capture shared information across treatment variables. Other directions for future research include the adaptation of existing \textsc{cbo} variants to the multi-objective case. For instance, combining dynamic \textsc{cbo} \citep{NEURIPS2021_577bcc91} with \textsc{mo-cbo} would lead to a \textsc{mo-cbo} variant that can handle time-dynamic causal models. As the field of causal decision-making continues to grow, we anticipate more progress in the development of multi-objective frameworks to address complex, real-world challenges.
\section{Limitations and Future Work}
The proposed OpenFly platform incorporates various rendering engines/techniques to provide high-quality scenes. Specifically, this is the first attempt to use 3D GS reconstructed scenes to support real-to-sim training and testing, while in the reconstruction of large-scale areas, a few visual artifacts are inevitably present. Future work will focus on exploring more effective reconstruction methods to enhance realism in large-scale scenes. Besides, the proposed OpenFly-Agent is built upon the large VLN model architecture, which is not practical for real-time deployment on UAVs. To address this, future research should focus on developing more efficient architectures and effective quantization techniques. 


\section{Conclusion}
In this work, we present OpenFly, a platform designed for large-scale data collection in aerial Vision-and-Language Navigation (VLN). OpenFly integrates multiple rendering engines and advanced real-to-sim techniques for data generation, enabling efficient collection of diverse, high-quality aerial VLN data. The resulting large-scale dataset comprises 100k trajectories across 18 distinct scenes, spanning a wide range of altitudes and difficulty levels, which is significantly superior than existing ones. Furthermore, we propose OpenFly-Agent, a keyframe-aware aerial navigation model capable of directly predicting flight actions based on observations and language instructions. Extensive experiments validate the effectiveness of the proposed method, and establishing a comprehensive benchmark for future advancements in aerial navigation. 
%The toolchain, dataset, and code will be publicly released, providing a valuable resource for future research in this field.
%\section{Impact Statements}
By explicitly including instructions alongside training examples, this work contributes to a deeper understanding of in-context learning (ICL) and serves as a testbed for future theoretical and empirical research.
The primary goal is to advance the field of Machine Learning by demonstrating how models can leverage task descriptions to enhance generalization and reduce reliance on large labeled datasets.
In doing so, the findings may promote better insights into Large Language Models (LLMs) and inspire the development of more efficient, adaptable AI systems.

\section*{Acknowledgements}
This work of Kangwook Lee is supported 
%in part 
by NSF Award DMS-2023239, NSF CAREER Award CCF-2339978, Amazon Research Award, and a grant from FuriosaAI.

% \section*{Impact Statement}
% This paper presents work whose goal is to advance the field of Machine Learning. There are many potential societal consequences of our work, none which we feel must be specifically highlighted here.

%\input{tex_main/6_impactstatemtent}
% In the unusual situation where you want a paper to appear in the
% references without citing it in the main text, use \nocite
%\nocite{langley00}

\bibliography{example_paper}
\bibliographystyle{icml2025}


%%%%%%%%%%%%%%%%%%%%%%%%%%%%%%%%%%%%%%%%%%%%%%%%%%%%%%%%%%%%%%%%%%%%%%%%%%%%%%%
%%%%%%%%%%%%%%%%%%%%%%%%%%%%%%%%%%%%%%%%%%%%%%%%%%%%%%%%%%%%%%%%%%%%%%%%%%%%%%%
% APPENDIX
%%%%%%%%%%%%%%%%%%%%%%%%%%%%%%%%%%%%%%%%%%%%%%%%%%%%%%%%%%%%%%%%%%%%%%%%%%%%%%%
%%%%%%%%%%%%%%%%%%%%%%%%%%%%%%%%%%%%%%%%%%%%%%%%%%%%%%%%%%%%%%%%%%%%%%%%%%%%%%%
% \newpage
% The work of Kangwook Lee is supported in part by NSF CAREER Award CcF-2339978, Amazon Research Award, and a grant from
% FuriosaAl.
\appendix
\onecolumn
\section{Pseudo Algorithm for ICL-HCG}
\label{app:alg}
We summarize our meta framework for ICL-HCG in Algorithm~\ref{alg:framework}.
\begin{algorithm}
  \caption{Meta-Learning Framework for ICL-HCG}
  \label{alg:framework}
  \begin{algorithmic}[1]
    \STATE \textbf{Inputs:} a set of inputs $\mathcal{X}$, a training set of hypothesis classes $\mathcal{S}^{\text{train}}=\{\mathcal{H}_i^{\text{train}}\}_{i=1}^{N^{\text{train}}}$, a testing set of hypothesis classes $\mathcal{S}^{\text{test}}=\{\mathcal{H}_i^{\text{test}}\}_{i=1}^{N^{\text{test}}}$, batch size $B$, hypothesis prefix size $L$, and context query size $K$
    \FOR{\textbf{training epoch}}
      \STATE sample $\{\mathcal{H}_i\}_{i=1}^B \overset{\text{i.i.d.}}{\sim} \text{Uniform}(\mathcal{S}^{\text{train}})$
      \FOR{each hypothesis class $\mathcal{H} \in \{\mathcal{H}_i\}_{i=1}^B$}
        \STATE generate $h,\SK$ following \hyperref[asu:iid]{i.i.d. Generation}
        
        \STATE \textbf{// Construct sequence based on $\mathcal{H}$, $h$, and $\SK$}
        \STATE construct hypothesis prefix, context query, and hypothesis index $z$ based on $\mathcal{H}$, $h$, $\SK$
        \STATE $s \gets \text{concatenate}(\text{hypothesis prefix}, \text{context query}, z)$
    
        \STATE \textbf{// Cross-entropy loss for next token prediction}
        \STATE $\mathcal{L} \gets -\sum_{t=2}^{|s|} \log P(s_t \mid s_{<t})$
      \ENDFOR
      \STATE update model parameters using $\mathcal{L}$ of the batch
    \ENDFOR
    
    \FOR{\textbf{testing epoch}}
      \STATE sample $\{\mathcal{H}_i\}_{i=1}^B \overset{\text{i.i.d.}}{\sim} \text{Uniform}(\mathcal{S}^{\text{test}})$
      
      \FOR{each hypothesis class $\mathcal{H} \in \{\mathcal{H}_i\}_{i=1}^B$}
        \STATE generate $h,\SK$ via:
        \STATE \quad \textbf{either} following \hyperref[asu:iid]{i.i.d. Generation}
        \STATE \quad \textbf{or} following \hyperref[asu:optt]{Opt-T Generation}
        \STATE \textbf{construct sequence $s$ based on $\mathcal{H}$, $h$, and $\SK$}
        \STATE \textbf{evaluate the prediction accuracy on $y$, $z$, etc}
      \ENDFOR
    \ENDFOR
  \end{algorithmic}
\end{algorithm}

\section{Implementation Detail of Hypothesis Prefix and Context Query}
\label{app:prefix}
\begin{figure}[h!]
    \centering
    %\vspace{-1.1cm}
    \includegraphics[width = 0.75\textwidth]{fig/framework.pdf}
    %\vspace{-0.2cm}
    \caption{\textbf{The framework.} We convert hypothesis class $\mathcal{H}$ and ICL sequence $\SK$ into sequences of tokens, concatenate them and input to Transformer.
    Then we examine whether Transformer can predict correct $y$ and $z$ values.}
    \label{fig:frameworkfull}
    %\vspace{-0.3cm}
\end{figure}
\paragraph{Hypothesis prefix}
\label{def:HypothesisPrefixFull}
Given a hypothesis class $\mathcal{H}$ and its hypothesis table, the correspongding hypothesis prefix with hypothesis prefix's content length $L$ is constructed as shown in Fig.~\ref{fig:frameworkfull}.
The token ``\textcolor[RGB]{78,149,217}{P}'' serves as the padding token to separate hypotheses,
the token ``\textcolor[RGB]{216,110,204}{;}'' serves as the separation token to separate $(x,y)$ pairs,
the token ``\textcolor[RGB]{127,127,127}{N}'' serves as the empty token to fill a blank hypothesis,
and the token ``\textcolor[RGB]{192,79,21}{\textgreater}'' is used to connect $(x,y)$ pairs of the hypothesis to a randomly assigned hypothesis index \hz\footnote{We use variable $z$ to represent the hypothesis index.}.
In the illustrated example in Fig.~\ref{fig:frameworkfull}, the randomly assigned indexes {\hz}'s are sampled from $M=4$ hypothesis index tokens \{``\textcolor[RGB]{0,176,80}{A}'',``\textcolor[RGB]{0,176,80}{B}'',``\textcolor[RGB]{0,176,80}{C}'',``\textcolor[RGB]{0,176,80}{D}''\} without replacement\footnote{A set of $L$ hypothesis index tokens are created serve as the pool from which the hypothesis indexes are randomly sampled without replacement.}.

\paragraph{Context query}
Given an ICL sequence $\SK$ with $K$ pairs of $(x,y)$, the context query of size $K$ is constructed to represent the ICL sequence and trigger the prediction of the hypothesis index with padding token ``\textcolor[RGB]{78,149,217}{P}'', separation token token ``\textcolor[RGB]{216,110,204}{;}'', and query token ``\textcolor[RGB]{192,79,21}{\textgreater}'' as shown in Fig.~\ref{fig:frameworkfull}.
\section{Additional Details of Experiments}
\label{app:exp}

\subsection{Four Types of Generalization}
We share more training and testing curves in Fig.~\ref{fig:multiple_curves_IO_2x3} to provide additional results to Fig.~\ref{fig:multiple_curves_IO}, and in Fig.~\ref{fig:multiple_curves_IOS_9x5} to provide additional results to Fig.~\ref{fig:multiple_curves_IOS}.
\label{subapp:4generalization}
\begin{figure*}[th!]
    \centering
    %\vspace{-1.1cm}
    %\includegraphics[width = 0.8\textwidth]{fig/BASIC/table_generalization_multiple_curves_1x4_combined.pdf}
    \includegraphics[width = 0.8\textwidth]{new_fig/BASIC/multiple_curves_for_IO_2x3.pdf}
    %\vspace{-0.2cm}
    \caption{\textbf{Multiple runs for ID and OOD hypothesis class generalizations.}}
    \label{fig:multiple_curves_IO_2x3}
    %\vspace{-0.3cm}
\end{figure*}

\begin{figure*}[th!]
    \centering
    %\vspace{-1.1cm}
    %\includegraphics[width = 0.8\textwidth]{fig/BASIC/table_generalization_multiple_curves_1x4_combined.pdf}
    \includegraphics[width = 0.95\textwidth]{new_fig/BASIC/multiple_curves_for_IOS_9x5.pdf}
    %\vspace{-0.2cm}
    \caption{\textbf{Multiple runs for ID and OOD hypothesis class size generalizations.}}
    \label{fig:multiple_curves_IOS_9x5}
    %\vspace{-0.3cm}
\end{figure*}

\subsection{Compare with Other Model Architectures}
\label{subapp:4model}
We share more training and testing curves in Figs.~\ref{fig:multiple_models_IO_2x3} and~\ref{fig:multiple_models_IOS_9x5} to provide additional results to Figs.~\ref{fig:multiple_models_IO} and~\ref{fig:multiple_models_IOS}, respectively.
\begin{figure*}[th!]
    \centering
    %\vspace{-1.1cm}
    \includegraphics[width = 0.7\textwidth]{new_fig/MODEL/multiple_models_for_IO_2x3.pdf}
    %\vspace{-0.2cm}
    \caption{\textbf{Various models on ID and OOD hypothesis class generalizations.}}
    \label{fig:multiple_models_IO_2x3}
    %\vspace{-0.3cm}
\end{figure*}


\begin{figure*}[th!]
    \centering
    %\vspace{-1.1cm}
    \includegraphics[width = 0.95\textwidth]{new_fig/MODEL/multiple_models_for_IOS_9x5.pdf}
    %\vspace{-0.2cm}
    \caption{\textbf{Various models on ID and OOD hypothesis class generalizations.}}
    \label{fig:multiple_models_IOS_9x5}
    %\vspace{-0.3cm}
\end{figure*}



\subsection{Effect of Training Class Count}
We share more training and testing curves in Fig.~\ref{fig:num_train_IO_1x5} to provide additional results to Fig.~\ref{fig:num_train_IO}.
\label{subapp:numtrain}
\begin{figure*}[th!]
    \centering
    %\vspace{-1.1cm}
    %\includegraphics[width = 0.9\textwidth]{fig/NUMTRAIN/table_generalization_multiple_curves_1x5_combined.pdf}
    \includegraphics[width = 0.9\textwidth]{new_fig/NUMTRAIN/num_train_IO_1x5.pdf}
    %\vspace{-0.2cm}
    \caption{\textbf{Effect of training hypothesis class count on ID and OOD hypothesis class generalization.}}
    \label{fig:num_train_IO_1x5}
    %\vspace{-0.3cm}
\end{figure*}

\section{Experiment Setup}
\label{app:expsetup}
Each experiment is \textbf{repeated four times}, with the mean calculated across runs.
The shadow region's boundary is defined by \textbf{the minimum and maximum values} observed across the four runs.
\subsection{Learning Rate Scheduler}
\label{app:lrscheduler}
We set the train procedure with 768 total epochs, each epoch containing 1024 batches.
The learning rate (lr) is first warmed up linearly from an LR$/64$ at epoch $e=1$ to a peak value LR at epoch $e=64$, following:
$$
\text{lr}(e) = \text{LR}\cdot \frac{e}{64}, \quad 1 \leq e \leq 64.
$$
After epoch 64, the learning rate undergoes a quadratic decay over the remaining 704 epochs, given by
\[
\text{lr}(e) = \text{LR}\cdot \sqrt{\frac{64}{e}}, \quad 64 \le e \le 768.
\]

\subsection{Hyperparameter Search}
\label{app:hyperparameters}
We list the hyperparameter searching spaces used for Transformer, LSTM, GRU, and Mamba.
The best hyperparameter is searched using ID hypothesis class generalization with $\|\mathcal{X}\|=5$, $\|\mathcal{H}\|=8$, and then used for all other settings.
\begin{table}[h!]
    \caption{Hyperparameters}
    \label{tab:TrainingParams}
    \centering
    \begin{tabular}{l|l}
        \textbf{Parameters} & \textbf{Tuning} \\% & \textbf{Estimation}  \\
        \hline
        Sampling time                   & $0.05$  \\   %& $0.05$     \\
        Reward discount factor $\gamma$ & $0.99$  \\   %& $0.99$     \\
        Learning rate for actor         & $10^{-3}$ \\% & $10^{-3}$  \\
        Learning rate for critic        & $10^{-3}$ \\% & $10^{-3}$  \\
        $L_2$ Regularization factor     & $10^{-5}$ \\% & $10^{-4}$  \\
        Optimizer parameter $\epsilon$  & $10^{-8}$ \\% & $10^{-8}$  \\
        Minimum batch size              & $1024$  \\%   & $64$     \\
        Experience buffer length        & $10^{6}$  \\% & $10^{6}$   \\
    \end{tabular}
\end{table}

\subsection{Setup for Generating Training and Testing Hypothesis Classes}
\label{setup:generalization}
We list the experimental setup for each experiments in the following Table~\ref{table:setup}.
When conducting experiments to evaluate accuracy on $y$, we modified the experimental setup following Table~\ref{table:setupicl}.
\begin{table}[th!]
\centering
\caption{\textbf{Experimental setups of different generalizations.}
The expression \(\min\{512, \#\text{possible}\}\) indicates that when the number of possible hypothesis classes is fewer than 512, we evaluate all possible hypothesis classes for testing; otherwise, we limit the selection to at most 512 hypothesis classes.
For example, if \( |\mathcal{H}^{\text{OOD}}| = 16 \) and \( |\mathcal{H}| = 2 \), the total number of possible hypothesis classes is given by:
$\binom{|\mathcal{H}^{\text{OOD}}|}{|\mathcal{H}|} = \binom{16}{2} = \frac{16 \times 15}{2} = 120.$
Since \( 120 < 512 \), we evaluate all 120 hypothesis classes for testing in this scenario.}
\label{table:setup}
\resizebox{1.00\linewidth}{!}{
\begin{tabular}{lrrrr}
\toprule
\multicolumn{1}{c}{Generalization Setup} & \multicolumn{1}{c}{\begin{tabular}[c]{@{}c@{}}ID Hypothesis\\ Class Generalization\end{tabular}} & \multicolumn{1}{c}{\begin{tabular}[c]{@{}c@{}}OOD Hypothesis\\ Class Generalization\end{tabular}} & \multicolumn{1}{c}{\begin{tabular}[c]{@{}c@{}}ID Hypothesis\\ Class Size Generalization\end{tabular}} & \multicolumn{1}{c}{\begin{tabular}[c]{@{}c@{}}OOD Hypothesis\\ Class Size Generalization\end{tabular}} \\ \midrule
size of input space ($|\mathcal{X}|$)    & 5                                                                                                & 5                                                                                                 & 5                                                                                                     & 5                                                                                                      \\
size of label space ($|\mathcal{Y}|$)    & 2                                                                                                & 2                                                                                                 & 2                                                                                                     & 2                                                                                                      \\
size of context query ($K$)              & 5                                                                                                & 5                                                                                                 & 5                                                                                                     & 5                                                                                                      \\
size of training hypothesis class ($|\mathcal{H}^{\text{train}}|$) & 8                                                                                                & 8                                                                                                 & 7,8,9                                                                                                   & 7,8,9                                                                                                    \\
size of testing hypothesis class ($|\mathcal{H}^{\text{test}}|$)   & 8                                                                                                & 8                                                                                                 & 2$,\ldots,$14                                                                                                  & 2$,\ldots,$14                                                                                                   \\
size of hypothesis prefix ($L$)          & 8                                                                                                & 8                                                                                                 & 16                                                                                                    & 16                                                                                                     \\
\#all hypotheses ($|\mathcal{H}^{\text{uni}}|$)                    & 32                                                                                               & 32                                                                                                & 32                                                                                                    & 32                                                                                                     \\
\#hypotheses in ID pool ($|\mathcal{H}^{\text{ID}}|$)                & 16                                                                                               & 16                                                                                                & 16                                                                                                    & 16                                                                                                     \\
\#hypotheses in OOD pool ($|\mathcal{H}^{\text{OOD}}|$)              & 16                                                                                               & 16                                                                                                & 16                                                                                                    & 16                                                                                                     \\
\#training hypothesis classes            & 12358                                                                                            & 12358                                                                                             & 4096                                                                             & 4096                                                                               \\
\#testing hypothesis classes             & 512                                                                                              & 512                                                                                               & $\min\{512,\#\text{possible}\}$                                                                                & $\min\{512,\#\text{possible}\}$                                                                                 \\ \bottomrule
\end{tabular}
}
\end{table}

\begin{table}[th!]
\centering
\caption{\textbf{Additional setups.} Numbers that differ from those in Table~\ref{table:setup} are highlighted in bold for clarity.}
\label{table:setupicl}
\begin{tabular}{lrr}
\toprule
Section & Sec.~\ref{subsec:icl} & Sec.~\ref{subsec:diversity} \\ \midrule
size of input space ($|\mathcal{X}|$)    & \textbf{4}                                                                                       & \textbf{6}                                                                                        \\
size of label space ($|\mathcal{Y}|$)    & 2                                                                                                & 2                                                                                                 \\
size of context query ($K$)              & \textbf{12}                                                                                      & \textbf{12}                                                                                       \\
size of training hypothesis class ($|\mathcal{H}^{\text{train}}|$) & \textbf{4}                                                                                       & 8                                                                                                   \\
size of testing hypothesis class ($|\mathcal{H}^{\text{test}}|$)   & \textbf{4}                                                                                       & 8                                                                                                   \\
size of hypothesis prefix ($L$)          & \textbf{4}                                                                                       & 8                                                                                                 \\
\#all hypotheses ($|\mathcal{H}^{\text{uni}}|$)                    & \textbf{16}                                                                                & \textbf{64}                                                                                         \\
\#hypotheses in ID pool ($|\mathcal{H}^{\text{ID}}|$)              & \textbf{16}                                                                                & \textbf{8,16,24,32,48}                                                                                                 \\
\#hypotheses in OOD pool ($|\mathcal{H}^{\text{OOD}}|$)            & \textbf{0}                                                                                & 16                                                                                                  \\
\#training hypothesis classes            & \textbf{1308}                                                                                    & $\min\{12358,\#\text{possible}\}$                                                                                            \\
\#testing hypothesis classes             & 512                                                                                              & 512                                                                                               \\ \bottomrule
\end{tabular}
\end{table}
%%%%%%%%%%%%%%%%%%%%%%%%%%%%%%%%%%%%%%%%%%%%%%%%%%%%%%%%%%%%%%%%%%%%%%%%%%%%%%%
%%%%%%%%%%%%%%%%%%%%%%%%%%%%%%%%%%%%%%%%%%%%%%%%%%%%%%%%%%%%%%%%%%%%%%%%%%%%%%%


\end{document}


% This document was modified from the file originally made available by
% Pat Langley and Andrea Danyluk for ICML-2K. This version was created
% by Iain Murray in 2018, and modified by Alexandre Bouchard in
% 2019 and 2021 and by Csaba Szepesvari, Gang Niu and Sivan Sabato in 2022.
% Modified again in 2023 and 2024 by Sivan Sabato and Jonathan Scarlett.
% Previous contributors include Dan Roy, Lise Getoor and Tobias
% Scheffer, which was slightly modified from the 2010 version by
% Thorsten Joachims & Johannes Fuernkranz, slightly modified from the
% 2009 version by Kiri Wagstaff and Sam Roweis's 2008 version, which is
% slightly modified from Prasad Tadepalli's 2007 version which is a
% lightly changed version of the previous year's version by Andrew
% Moore, which was in turn edited from those of Kristian Kersting and
% Codrina Lauth. Alex Smola contributed to the algorithmic style files.
