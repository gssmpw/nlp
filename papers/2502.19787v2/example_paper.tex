%%%%%%%% ICML 2025 EXAMPLE LATEX SUBMISSION FILE %%%%%%%%%%%%%%%%%

\documentclass{article}

% Recommended, but optional, packages for figures and better typesetting:
\usepackage{microtype}
\usepackage{graphicx}
\usepackage{subfigure}
\usepackage{booktabs} % for professional tables

% hyperref makes hyperlinks in the resulting PDF.
% If your build breaks (sometimes temporarily if a hyperlink spans a page)
% please comment out the following usepackage line and replace
% \usepackage{icml2025} with \usepackage[nohyperref]{icml2025} above.
\usepackage{hyperref}
\usepackage{xspace}
\usepackage{enumitem}
%\usepackage{algorithm}
%\usepackage{algpseudocode}


% Attempt to make hyperref and algorithmic work together better:
%\newcommand{\theHalgorithm}{\arabic{algorithm}}

% Use the following line for the initial blind version submitted for review:
%\usepackage{icml2025}

% If accepted, instead use the following line for the camera-ready submission:
\usepackage[accepted]{icml2025}

% For theorems and such
\usepackage{amsmath, bm}
\usepackage{amssymb}
\usepackage{mathtools}
\usepackage{amsthm}



\usepackage{graphicx}      % For including graphics
\usepackage{subcaption}    % For sub-figures and sub-captions

% if you use cleveref..
\usepackage[capitalize,noabbrev]{cleveref}

%%%%%%%%%%%%%%%%%%%%%%%%%%%%%%%%
% THEOREMS
%%%%%%%%%%%%%%%%%%%%%%%%%%%%%%%%
\theoremstyle{plain}
\newtheorem{theorem}{Theorem}[section]
\newtheorem{proposition}[theorem]{Proposition}
\newtheorem{lemma}[theorem]{Lemma}
\newtheorem{corollary}[theorem]{Corollary}
\theoremstyle{definition}
\newtheorem{definition}[theorem]{Definition}
\newtheorem{assumption}[theorem]{Assumption}
\theoremstyle{remark}
\newtheorem{remark}[theorem]{Remark}

% Todonotes is useful during development; simply uncomment the next line
%    and comment out the line below the next line to turn off comments
%\usepackage[disable,textsize=tiny]{todonotes}
\usepackage[textsize=tiny]{todonotes}

% Lists
\usepackage[inline]{enumitem}
\newlist{enuminline}{enumerate*}{1}
\setlist[enuminline]{label=(\roman*)}

% Math
\newcommand{\vol}{\mathrm{vol}}
\newcommand{\E}{\mathbb E}
\newcommand{\var}{\mathrm{var}}
\newcommand{\cov}{\mathrm{cov}}
\newcommand{\Normal}{\mathcal N}
\newcommand{\slowvar}{\mathcal L}
\newcommand{\bbR}{\mathbb R}
\newcommand{\bbE}{\mathbb E}
\newcommand{\bbP}{\mathbb P}
\newcommand{\calC}{\mathcal C}
\newcommand{\calR}{\mathcal R}
\newcommand{\calT}{\mathcal T}
\newcommand{\loA}{\underline{A}}
\newcommand{\upA}{\overline{A}} 
\newcommand{\cv}{\mathrm{cv}} 
\newcommand{\pp}{\mathrm{pp}}
\newcommand{\HS}{\mathrm{HS}}
\newcommand{\erfi}{\mathrm{erfi}}
\newcommand{\tX}{\widetilde X}
\newcommand{\OneFOne}{{}_1F_1}
\newcommand{\PP}{\mathrm{\texttt{PP}}}
\newcommand{\PPpp}{\mathrm{\texttt{PP+}}}
\newcommand{\BPP}{\mathrm{\texttt{BPP}}}
\newcommand{\BPPpp}{\mathrm{\texttt{BPP+}}}
\newcommand{\FABPPI}{\mathrm{\texttt{FABPP}}}
\newcommand{\asto}{\overset{\text{a.s.}}{\to}}
\newcommand{\zq}[1]{{\leavevmode\color[rgb]{1,0,0}[Ziqian: #1]}}
% The \icmltitle you define below is probably too long as a header.
% Therefore, a short form for the running title is supplied here:
\icmltitlerunning{\mytitle}

\begin{document}

\twocolumn[
\icmltitle{\mytitle}

% It is OKAY to include author information, even for blind
% submissions: the style file will automatically remove it for you
% unless you've provided the [accepted] option to the icml2025
% package.

% List of affiliations: The first argument should be a (short)
% identifier you will use later to specify author affiliations
% Academic affiliations should list Department, University, City, Region, Country
% Industry affiliations should list Company, City, Region, Country

% You can specify symbols, otherwise they are numbered in order.
% Ideally, you should not use this facility. Affiliations will be numbered
% in order of appearance and this is the preferred way.
\icmlsetsymbol{equal}{*}

\begin{icmlauthorlist}
\icmlauthor{Ziqian Lin}{xxx}
\icmlauthor{Shubham Kumar Bharti}{xxx}
\icmlauthor{Kangwook Lee}{zzz}
%\icmlauthor{Firstname5 Lastname5}{yyy}
%\icmlauthor{Firstname6 Lastname6}{sch,yyy,comp}
%\icmlauthor{Firstname7 Lastname7}{comp}
%\icmlauthor{}{sch}
%\icmlauthor{Firstname8 Lastname8}{sch}
%\icmlauthor{Firstname8 Lastname8}{yyy,comp}
%\icmlauthor{}{sch}
%\icmlauthor{}{sch}
\end{icmlauthorlist}

\icmlaffiliation{xxx}{Department of Computer Science, University of Wisconsin-Madison, Madison, Wisconsin, USA}
\icmlaffiliation{zzz}{Department of Electrical \& Computer Engineering, University of Wisconsin-Madison, Madison, Wisconsin, USA}
%\icmlaffiliation{comp}{Company Name, Location, Country}
%\icmlaffiliation{sch}{School of ZZZ, Institute of WWW, Location, Country}

%\icmlcorrespondingauthor{Ziqian Lin}{zlin284@wisc.edu}
\icmlcorrespondingauthor{Kangwook Lee}{kangwook.lee@wisc.edu}

% You may provide any keywords that you
% find helpful for describing your paper; these are used to populate
% the "keywords" metadata in the PDF but will not be shown in the document
\icmlkeywords{Machine Learning, ICML}

\vskip 0.3in
]

% this must go after the closing bracket ] following \twocolumn[ ...

% This command actually creates the footnote in the first column
% listing the affiliations and the copyright notice.
% The command takes one argument, which is text to display at the start of the footnote.
% The \icmlEqualContribution command is standard text for equal contribution.
% Remove it (just {}) if you do not need this facility.

\printAffiliationsAndNotice{}  % leave blank if no need to mention equal contribution
%\printAffiliationsAndNotice{\icmlEqualContribution} % otherwise use the standard text.

\begin{abstract}
% In-context learning (ICL) enables large language models (LLMs) to adapt to new tasks by conditioning on examples.
Recent research has investigated the underlying mechanisms of in-context learning (ICL) both theoretically and empirically, often using data generated from simple function classes.
However, the existing work often focuses on the sequence consisting solely of labeled examples, while in practice, labeled examples are typically accompanied by an \emph{instruction}, providing some side information about the task. 
In this work, we propose \emph{ICL with hypothesis-class guidance (ICL-HCG)}, a novel synthetic data model for ICL where the input context consists of the literal description of a (finite) hypothesis class $\mathcal{H}$ and $(x,y)$ pairs from a hypothesis chosen from $\mathcal{H}$.
Under our framework ICL-HCG, we conduct extensive experiments to explore: 
(i) a variety of generalization abilities to new hypothesis classes; 
(ii) different model architectures;
(iii) sample complexity;
(iv) in-context data imbalance;
(v) the role of instruction; and
(vi) the effect of pretraining hypothesis diversity.
As a result, we show that 
(a) Transformers can successfully learn ICL-HCG and generalize to unseen hypotheses and unseen hypothesis classes, and (b) compared with ICL without instruction, ICL-HCG achieves significantly higher accuracy, demonstrating the role of instructions. 
The code is available at:  
\url{https://github.com/UW-Madison-Lee-Lab/ICL-HCG}.
\end{abstract}

\section{Introduction}
Backdoor attacks pose a concealed yet profound security risk to machine learning (ML) models, for which the adversaries can inject a stealth backdoor into the model during training, enabling them to illicitly control the model's output upon encountering predefined inputs. These attacks can even occur without the knowledge of developers or end-users, thereby undermining the trust in ML systems. As ML becomes more deeply embedded in critical sectors like finance, healthcare, and autonomous driving \citep{he2016deep, liu2020computing, tournier2019mrtrix3, adjabi2020past}, the potential damage from backdoor attacks grows, underscoring the emergency for developing robust defense mechanisms against backdoor attacks.

To address the threat of backdoor attacks, researchers have developed a variety of strategies \cite{liu2018fine,wu2021adversarial,wang2019neural,zeng2022adversarial,zhu2023neural,Zhu_2023_ICCV, wei2024shared,wei2024d3}, aimed at purifying backdoors within victim models. These methods are designed to integrate with current deployment workflows seamlessly and have demonstrated significant success in mitigating the effects of backdoor triggers \cite{wubackdoorbench, wu2023defenses, wu2024backdoorbench,dunnett2024countering}.  However, most state-of-the-art (SOTA) backdoor purification methods operate under the assumption that a small clean dataset, often referred to as \textbf{auxiliary dataset}, is available for purification. Such an assumption poses practical challenges, especially in scenarios where data is scarce. To tackle this challenge, efforts have been made to reduce the size of the required auxiliary dataset~\cite{chai2022oneshot,li2023reconstructive, Zhu_2023_ICCV} and even explore dataset-free purification techniques~\cite{zheng2022data,hong2023revisiting,lin2024fusing}. Although these approaches offer some improvements, recent evaluations \cite{dunnett2024countering, wu2024backdoorbench} continue to highlight the importance of sufficient auxiliary data for achieving robust defenses against backdoor attacks.

While significant progress has been made in reducing the size of auxiliary datasets, an equally critical yet underexplored question remains: \emph{how does the nature of the auxiliary dataset affect purification effectiveness?} In  real-world  applications, auxiliary datasets can vary widely, encompassing in-distribution data, synthetic data, or external data from different sources. Understanding how each type of auxiliary dataset influences the purification effectiveness is vital for selecting or constructing the most suitable auxiliary dataset and the corresponding technique. For instance, when multiple datasets are available, understanding how different datasets contribute to purification can guide defenders in selecting or crafting the most appropriate dataset. Conversely, when only limited auxiliary data is accessible, knowing which purification technique works best under those constraints is critical. Therefore, there is an urgent need for a thorough investigation into the impact of auxiliary datasets on purification effectiveness to guide defenders in  enhancing the security of ML systems. 

In this paper, we systematically investigate the critical role of auxiliary datasets in backdoor purification, aiming to bridge the gap between idealized and practical purification scenarios.  Specifically, we first construct a diverse set of auxiliary datasets to emulate real-world conditions, as summarized in Table~\ref{overall}. These datasets include in-distribution data, synthetic data, and external data from other sources. Through an evaluation of SOTA backdoor purification methods across these datasets, we uncover several critical insights: \textbf{1)} In-distribution datasets, particularly those carefully filtered from the original training data of the victim model, effectively preserve the model’s utility for its intended tasks but may fall short in eliminating backdoors. \textbf{2)} Incorporating OOD datasets can help the model forget backdoors but also bring the risk of forgetting critical learned knowledge, significantly degrading its overall performance. Building on these findings, we propose Guided Input Calibration (GIC), a novel technique that enhances backdoor purification by adaptively transforming auxiliary data to better align with the victim model’s learned representations. By leveraging the victim model itself to guide this transformation, GIC optimizes the purification process, striking a balance between preserving model utility and mitigating backdoor threats. Extensive experiments demonstrate that GIC significantly improves the effectiveness of backdoor purification across diverse auxiliary datasets, providing a practical and robust defense solution.

Our main contributions are threefold:
\textbf{1) Impact analysis of auxiliary datasets:} We take the \textbf{first step}  in systematically investigating how different types of auxiliary datasets influence backdoor purification effectiveness. Our findings provide novel insights and serve as a foundation for future research on optimizing dataset selection and construction for enhanced backdoor defense.
%
\textbf{2) Compilation and evaluation of diverse auxiliary datasets:}  We have compiled and rigorously evaluated a diverse set of auxiliary datasets using SOTA purification methods, making our datasets and code publicly available to facilitate and support future research on practical backdoor defense strategies.
%
\textbf{3) Introduction of GIC:} We introduce GIC, the \textbf{first} dedicated solution designed to align auxiliary datasets with the model’s learned representations, significantly enhancing backdoor mitigation across various dataset types. Our approach sets a new benchmark for practical and effective backdoor defense.



\section{Related Work}
\label{sec:related-works}
\subsection{Novel View Synthesis}
Novel view synthesis is a foundational task in the computer vision and graphics, which aims to generate unseen views of a scene from a given set of images.
% Many methods have been designed to solve this problem by posing it as 3D geometry based rendering, where point clouds~\cite{point_differentiable,point_nfs}, mesh~\cite{worldsheet,FVS,SVS}, planes~\cite{automatci_photo_pop_up,tour_into_the_picture} and multi-plane images~\cite{MINE,single_view_mpi,stereo_magnification}, \etal
Numerous methods have been developed to address this problem by approaching it as 3D geometry-based rendering, such as using meshes~\cite{worldsheet,FVS,SVS}, MPI~\cite{MINE,single_view_mpi,stereo_magnification}, point clouds~\cite{point_differentiable,point_nfs}, etc.
% planes~\cite{automatci_photo_pop_up,tour_into_the_picture}, 


\begin{figure*}[!t]
    \centering
    \includegraphics[width=1.0\linewidth]{figures/overview-v7.png}
    %\caption{\textbf{Overview.} Given a set of images, our method obtains both camera intrinsics and extrinsics, as well as a 3DGS model. First, we obtain the initial camera parameters, global track points from image correspondences and monodepth with reprojection loss. Then we incorporate the global track information and select Gaussian kernels associated with track points. We jointly optimize the parameters $K$, $T_{cw}$, 3DGS through multi-view geometric consistency $L_{t2d}$, $L_{t3d}$, $L_{scale}$ and photometric consistency $L_1$, $L_{D-SSIM}$.}
    \caption{\textbf{Overview.} Given a set of images, our method obtains both camera intrinsics and extrinsics, as well as a 3DGS model. During the initialization, we extract the global tracks, and initialize camera parameters and Gaussians from image correspondences and monodepth with reprojection loss. We determine Gaussian kernels with recovered 3D track points, and then jointly optimize the parameters $K$, $T_{cw}$, 3DGS through the proposed global track constraints (i.e., $L_{t2d}$, $L_{t3d}$, and $L_{scale}$) and original photometric losses (i.e., $L_1$ and $L_{D-SSIM}$).}
    \label{fig:overview}
\end{figure*}

Recently, Neural Radiance Fields (NeRF)~\cite{2020NeRF} provide a novel solution to this problem by representing scenes as implicit radiance fields using neural networks, achieving photo-realistic rendering quality. Although having some works in improving efficiency~\cite{instant_nerf2022, lin2022enerf}, the time-consuming training and rendering still limit its practicality.
Alternatively, 3D Gaussian Splatting (3DGS)~\cite{3DGS2023} models the scene as explicit Gaussian kernels, with differentiable splatting for rendering. Its improved real-time rendering performance, lower storage and efficiency, quickly attract more attentions.
% Different from NeRF-based methods which need MLPs to model the scene and huge computational cost for rendering, 3DGS has stronger real-time performance, higher storage and computational efficiency, benefits from its explicit representation and gradient backpropagation.

\subsection{Optimizing Camera Poses in NeRFs and 3DGS}
Although NeRF and 3DGS can provide impressive scene representation, these methods all need accurate camera parameters (both intrinsic and extrinsic) as additional inputs, which are mostly obtained by COLMAP~\cite{colmap2016}.
% This strong reliance on COLMAP significantly limits their use in real-world applications, so optimizing the camera parameters during the scene training becomes crucial.
When the prior is inaccurate or unknown, accurately estimating camera parameters and scene representations becomes crucial.

% In early works, only photometric constraints are used for scene training and camera pose estimation. 
% iNeRF~\cite{iNerf2021} optimizes the camera poses based on a pre-trained NeRF model.
% NeRFmm~\cite{wang2021nerfmm} introduce a joint optimization process, which estimates the camera poses and trains NeRF model jointly.
% BARF~\cite{barf2021} and GARF~\cite{2022GARF} provide new positional encoding strategy to handle with the gradient inconsistency issue of positional embedding and yield promising results.
% However, they achieve satisfactory optimization results when only the pose initialization is quite closed to the ground-truth, as the photometric constrains can only improve the quality of camera estimation within a small range.
% Later, more prior information of geometry and correspondence, \ie monocular depth and feature matching, are introduced into joint optimisation to enhance the capability of camera poses estimation.
% SC-NeRF~\cite{SCNeRF2021} minimizes a projected ray distance loss based on correspondence of adjacent frames.
% NoPe-NeRF~\cite{bian2022nopenerf} chooses monocular depth maps as geometric priors, and defines undistorted depth loss and relative pose constraints for joint optimization.
In earlier studies, scene training and camera pose estimation relied solely on photometric constraints. iNeRF~\cite{iNerf2021} refines the camera poses using a pre-trained NeRF model. NeRFmm~\cite{wang2021nerfmm} introduces a joint optimization approach that simultaneously estimates camera poses and trains the NeRF model. BARF~\cite{barf2021} and GARF~\cite{2022GARF} propose a new positional encoding strategy to address the gradient inconsistency issues in positional embedding, achieving promising results. However, these methods only yield satisfactory optimization when the initial pose is very close to the ground truth, as photometric constraints alone can only enhance camera estimation quality within a limited range. Subsequently, 
% additional prior information on geometry and correspondence, such as monocular depth and feature matching, has been incorporated into joint optimization to improve the accuracy of camera pose estimation. 
SC-NeRF~\cite{SCNeRF2021} minimizes a projected ray distance loss based on correspondence between adjacent frames. NoPe-NeRF~\cite{bian2022nopenerf} utilizes monocular depth maps as geometric priors and defines undistorted depth loss and relative pose constraints.

% With regard to 3D Gaussian Splatting, CF-3DGS~\cite{CF-3DGS-2024} also leverages mono-depth information to constrain the optimization of local 3DGS for relative pose estimation and later learn a global 3DGS progressively in a sequential manner.
% InstantSplat~\cite{fan2024instantsplat} focus on sparse view scenes, first use DUSt3R~\cite{dust3r2024cvpr} to generate a set of densely covered and pixel-aligned points for 3D Gaussian initialization, then introduce a parallel grid partitioning strategy in joint optimization to speed up.
% % Jiang et al.~\cite{Jiang_2024sig} proposed to build the scene continuously and progressively, to next unregistered frame, they use registration and adjustment to adjust the previous registered camera poses and align unregistered monocular depths, later refine the joint model by matching detected correspondences in screen-space coordinates.
% \gjh{Jiang et al.~\cite{Jiang_2024sig} also implemented an incremental approach for reconstructing camera poses and scenes. Initially, they perform feature matching between the current image and the image rendered by a differentiable surface renderer. They then construct matching point errors, depth errors, and photometric errors to achieve the registration and adjustment of the current image. Finally, based on the depth map, the pixels of the current image are projected as new 3D Gaussians. However, this method still exhibits limitations when dealing with complex scenes and unordered images.}
% % CG-3DGS~\cite{sun2024correspondenceguidedsfmfree3dgaussian} follows CF-3DGS, first construct a coarse point cloud from mono-depth maps to train a 3DGS model, then progressively estimate camera poses based on this pre-trained model by constraining the correspondences between rendering view and ground-truth.
% \gjh{Similarly, CG-3DGS~\cite{sun2024correspondenceguidedsfmfree3dgaussian} first utilizes monocular depth estimation and the camera parameters from the first frame to initialize a set of 3D Gaussians. It then progressively estimates camera poses based on this pre-trained model by constraining the correspondences between the rendered views and the ground truth.}
% % Free-SurGS~\cite{freesurgs2024} matches the projection flow derived from 3D Gaussians with optical flow to estimate the poses, to compensate for the limitations of photometric loss.
% \gjh{Free-SurGS~\cite{freesurgs2024} introduces the first SfM-free 3DGS approach for surgical scene reconstruction. Due to the challenges posed by weak textures and photometric inconsistencies in surgical scenes, Free-SurGS achieves pose estimation by minimizing the flow loss between the projection flow and the optical flow. Subsequently, it keeps the camera pose fixed and optimizes the scene representation by minimizing the photometric loss, depth loss and flow loss.}
% \gjh{However, most current works assume camera intrinsics are known and primarily focus on optimizing camera poses. Additionally, these methods typically rely on sequentially ordered image inputs and incrementally optimize camera parameters and scene representation. This inevitably leads to drift errors, preventing the achievement of globally consistent results. Our work aims to address these issues.}

Regarding 3D Gaussian Splatting, CF-3DGS~\cite{CF-3DGS-2024} utilizes mono-depth information to refine the optimization of local 3DGS for relative pose estimation and subsequently learns a global 3DGS in a sequential manner. InstantSplat~\cite{fan2024instantsplat} targets sparse view scenes, initially employing DUSt3R~\cite{dust3r2024cvpr} to create a densely covered, pixel-aligned point set for initializing 3D Gaussian models, and then implements a parallel grid partitioning strategy to accelerate joint optimization. Jiang \etal~\cite{Jiang_2024sig} develops an incremental method for reconstructing camera poses and scenes, but it struggles with complex scenes and unordered images. 
% Similarly, CG-3DGS~\cite{sun2024correspondenceguidedsfmfree3dgaussian} progressively estimates camera poses using a pre-trained model by aligning the correspondences between rendered views and actual scenes. Free-SurGS~\cite{freesurgs2024} pioneers an SfM-free 3DGS method for reconstructing surgical scenes, overcoming challenges such as weak textures and photometric inconsistencies by minimizing the discrepancy between projection flow and optical flow.
%\pb{SF-3DGS-HT~\cite{ji2024sfmfree3dgaussiansplatting} introduced VFI into training as additional photometric constraints. They separated the whole scene into several local 3DGS models and then merged them hierarchically, which leads to a significant improvement on simple and dense view scenes.}
HT-3DGS~\cite{ji2024sfmfree3dgaussiansplatting} interpolates frames for training and splits the scene into local clips, using a hierarchical strategy to build 3DGS model. It works well for simple scenes, but fails with dramatic motions due to unstable interpolation and low efficiency.
% {While effective for simple scenes, it struggles with dramatic motion due to unstable view interpolation and suffers from low computational efficiency.}

However, most existing methods generally depend on sequentially ordered image inputs and incrementally optimize camera parameters and 3DGS, which often leads to drift errors and hinders achieving globally consistent results. Our work seeks to overcome these limitations.

\section{Meta-Learning for ICL-HCG}
Training a learner to perform ICL aligns with the concept of meta-learning, as it enables adaptation to new tasks using in-context examples. While prior studies~\citep{garg2022can,fan2024transformers,raventos2024pretraining} train Transformers for ICL on sequences of the form \((x_1, y_1, x_2, y_2, \dots, x_k, y_k)\) without explicit instructions, our work investigates whether a Transformer trained for ICL with instructions, namely ICL-HCG, can generalize to new ICL-HCG tasks.

% While meta-learning trains across tasks and fine-tunes on new samples for rapid adaptation, MetaICL trains a learner on ICL tasks, learning to predict by using those new samples as in-context examples---feeding them along with the query sample (the one to be predicted) as input to the learner---eliminating the finetuning stage.

% Enabling a model to do in-context learning usually requires pretraining it on in-context learning sample of form $\text{training examples, test queries}$ sampled from a meta task distribution $P_{task}$ during pretraining of the large models.
% We use this section to introduce our meta-learning framework for ICL-HCG.
% We start by introducing the the meta tasks in ICL-HCG
%\zq{Is this meta learning? or just generalization? Since the task of meta in our work would be $(H,h,xyxy)$, which is a sample rather than a ``task''.}
%\subsection{Empirical Risk Minimization for Binary Classification (come later, what is the data ERM for a fix, fix hypothesis)}
\label{sec:ERM}
In this work, we consider the binary classification problem where the input space is finite. Formally, let the input space be defined as
$
\mathcal{X} = \{x_1, x_2, \ldots, x_{|\mathcal{X}|}\}
$,
and the label space as
$
\mathcal{Y} = \{0, 1\}
$.
The learning task is characterized by a hypothesis space
$
\mathcal{H} \subseteq \mathcal{Y}^{\mathcal{X}}
$,
where each hypothesis $ h \in \mathcal{H} $ is a deterministic function
$
h: \mathcal{X} \rightarrow \mathcal{Y}
$.
A realizable instance of the binary classification problem is specified by the tuple
$
(\mathcal{X}, \mathcal{Y}, P, h^*)
$,
where $P$ is a probability distribution over the input space $\mathcal{X}$, and $h^*:\mathcal{X} \rightarrow \mathcal{Y}$ is the target hypothesis.

During learning, a learner $A$ receives a training dataset $\mathcal{D}=\{(x^{(i)}, y^{(i)})\}_{i=1}^K$ and it is tasked to find a hypothesis that produces least generalization risk $R(A(\mathcal{D}))$ on future test sample drawn from $P$ and $h^*$.
Empirical risk minimization (ERM) is one of the most popular learning algorithm. It learns by minimizing the empirical risk on the training dataset $\mathcal D$ with respect to a loss function $l:\mathcal{Y} \times \mathcal{Y} \rightarrow \mathbb R^+$ and is given as,
$$
\hat h_\mathcal{D} = \text{ERM}(\mathcal{D},\mathcal{H}) \leftarrow \min_{h \in \mathcal{H}} \sum_{i=1}^K l(h(x^{(i)}), y^{(i)}).
$$
% Once learning is complete, the learner uses learnt hypothesis $\hat h_D$ to make prediction on samples from $P, h^*$ thereby suffering a risk of, $$R(\hat h_D) \leftarrow \mathbb E_{(x, y) \sim P, h^*}[l(\hat h_D(x), y)].$$
In this paper, we focus on $0/1$ risk, $$l(\hat{y},y)=
\begin{cases}
    1, & \text{if } \hat{y}=y\\
    0, & \text{if } \hat{y}\neq y
\end{cases}.
$$
\begin{remark}
ERM focuses mainly on sample complexity, that is, determining the minimum number of samples required to identify a hypothesis $\hat{h}$ with a risk below a specified threshold.
Differently, in this paper, we explore whether Transformer models can accurately pinpoint the exact hypothesis when provided with sufficient number of samples.
\end{remark}

% In theory, Probably Approximately Correct(PAC) learning completely characterizes the sample complexity of learning. It states that for any learning instance and $(\epsilon, \delta) \in (0,1)^2$ if $n\geq O(\frac{1}{\epsilon}(d(\mathcal H)+\log(\frac{1}{\delta})))$, then,  $$\text{w.p. $\geq 1-\delta$}, \qquad R(\hat h_D) \leq \epsilon.$$
% where $d(\mathcal H)$ is the VC dimension of hypothesis class $\mathcal H$ which is a measure of its complexity. Therefore, to learn the target hypothesis up to an error $\epsilon$, computing the ERM on an i.i.d. sample of size $O(\frac{d}{\epsilon})$ is sufficient.


%%% motivating incontext learning with finite hypothesis class

% In recent literature, several works {cite} have shown impressive capability of transformers to in-context learn ERM on a dataset $D$ at inference time which is known as In-context Learning(ICL). However, most of the works have focused on ICL learning settings where hypothesis class is defined implicitly by a linear $\mathbb R^d$ space. This begs the question of whether transformers can do ERM in more explicit learning settings where hypothesis class are specified as input to the model. Towards this end we seek to answer the following question:

% \textbf{Can a transformers learn to do in-context learning conditioned on finite/tabular hypothesis classes?}

% A couple of recent works have tried to address this problem by conditioning the hypothesis class on a noisy version of the target hypothesis {cite}. Contrary to this, a finite hypothesis class conditions the universal hypothesis space $\mathcal{A}^\mathcal{Y}$ by specifying a subset of hypothesis $\mathcal H$ that contains a the target hypothesis to solve the task instance. In real world learning scenario, this can be viewed   The explicit conditioning on finite hypothesis class also provide an avenue to test the generalization capability of ICL to generalize to different size of hypothesis tables thereby helping us to understand ERM capability of transformers in a more controlled setup. 

%%% Ziqian's previous writeup

% \subsection{Empirical Risk Minimization (ERM)}
% Consider a training dataset $\{(x_i, y_i)\}_{i=1}^n$, where each $x_i \in \mathcal{A}$ is a feature vector and $y_i \in \mathcal{Y}$ is the corresponding label or target value. Let $\mathcal{H}$ be a hypothesis class of functions $h: \mathcal{A} \to \mathcal{Y}$, and let $L:\mathcal{Y}\times\mathcal{Y}\to\mathbb{R}_{\geq 0}$ be a loss function that measures the discrepancy between the predicted value $h(x_i)$ and the true value $y_i$.

% The \emph{empirical risk} of a hypothesis $h \in \mathcal{H}$ is defined as:
% \[
% \hat{R}_n(h) = \frac{1}{n} \sum_{i=1}^n L(y_i, h(x_i)).
% \]

% The \emph{Empirical Risk Minimization} principle seeks a hypothesis $\hat{h} \in \mathcal{H}$ that minimizes the empirical risk:
% \[
% \hat{h} = \arg\min_{h \in \mathcal{H}} \hat{R}_n(h).
% \]


% we need to explain what is meta learning in plain english before going into notations 
% Meta-learning refers to the process of learning how to learn by leveraging experiences from multiple tasks.
% In our setup, each meta-task is defined by a hypothesis class, and a sample within a meta-task consists of a combination of the hypothesis class, multiple $(x,y)$ pairs, and the underlying hypothesis.
\subsection{Two Types of Tasks in ICL-HCG}
\label{sec:problem-definition}
We consider two types of tasks in ICL-HCG, both constructed from a finite hypothesis class
\(
    \mathcal{H} = \{h^{(1)}, h^{(2)}, \ldots, h^{|\mathcal{H}|}\}
\)
over a finite input space
\(
    \mathcal{X} = \{x_1, x_2, \ldots, x_{|\mathcal{X}|}\}
\)
and a binary output space
\(
    \mathcal{Y} = \{0, 1\}.
\)

\paragraph{Label prediction}
Consider a hypothesis class \(\mathcal{H}\) and a sequence consisting of training data and a test point
\[
    \Skmx = (x^{(1)}, y^{(1)}, \ldots, x^{(k-1)}, y^{(k-1)}, x^{(k)})
\]
where for all $i$, \(y^{(i)} = h(x^{(i)})\) for a specific \(h \in \mathcal{H}\), and \(x^{(k)}\) is a test query input. The objective is to predict the label 
\[
    y^{(k)} = h\bigl(x^{(k)}\bigr).
\]
We refer to this as label prediction, with input-output pairs:
\[
    i_{\text{I},k} = \bigl(\mathcal{H}, \Skmx\bigr),
    \quad
    o_{\text{I},k} = y^{(k)}.
\]

\paragraph{Hypothesis identification}
Given a hypothesis class \(\mathcal{H}\) and a sequence (namely \emph{ICL sequence}) 
\[
    \SK = (x^{(1)}, y^{(1)}, \ldots, x^{(K)}, y^{(K)}),
\]
where for all $i$, \(y^{(i)} = h(x^{(i)})\) for a specific  \(h \in \mathcal{H}\),
the goal is to identify the underlying hypothesis \(h\).
Denote this as hypothesis identification, with:
\[
    i_{\text{II},K} = \bigl(\mathcal{H}, \SK \bigr),
    \quad
    o_{\text{II},K} = h.
\]



\paragraph{Meta-learning}
%A task in meta learning is defined by a finite hypothesis class $\mathcal{H}$.
Label prediction uses \(k-1\) samples to predict the label of a new query \(x^{(k)}\), while hypothesis identification directly outputs \(h\).
Both label prediction and hypothesis identification can be viewed as attempts to identify \(h\) from $\mathcal{H}$ via empirical risk minimization (ERM) using the dataset 
\(\{(x^{(i)},y^{(i)})\}\).
Our meta-learning aims at learning to do ERM for different hypothesis classes when these hypothesis classes are given as input along with $(x,y)$ pairs.



\subsection{Sample Generation}
We consider the following two approaches for generating samples of ICL-HCG tasks.
\begin{asu}[i.i.d.\ Generation]
\label{asu:iid}
Given hypothesis classes
\(\{\mathcal{H}_i\}_{i=1}^{N}\), input space \(\mathcal{X}\), and an integer \(K\):\\
\subasu\label{asu:setting1}Sample a hypothesis class \(\mathcal{H}\) from \(\{\mathcal{H}_i\}_{i=1}^{N^\text{train}}\);\\
\subasu\label{asu:setting2}Sample a hypothesis \(h\) uniformly at random from \(\mathcal{H}\);\\
\subasu\label{asu:setting3}Sample \(K\) inputs \(\{x^{(i)}\}_{i=1}^{K}\) i.i.d.\ from \(\mathrm{Uniform}(\mathcal{X})\);\\
\subasu\label{asu:setting4}Generate \(y^{(i)} = h(x^{(i)})\) for each \(i \in [K]\);\\
\subasu\label{asu:setting5}\(\Skmx = [x^{(1)},y^{(1)}, \ldots, x^{(k)}]\) for label prediction;\\
\subasu\label{asu:setting6}{\tiny \(\SK = [x^{(1)},y^{(1)}, \ldots, x^{(K)},y^{(K)}]\)} for hypothesis identification.\\
\end{asu}

\begin{asu}[Opt-T Generation]
\label{asu:optt}
Given hypothesis classes
\(\{\mathcal{H}_i\}_{i=1}^{N}\), input space \(\mathcal{X}\), and an integer \(K\):\\
\subasu Sample a hypothesis class \(\mathcal{H}\) from \(\{\mathcal{H}_i\}_{i=1}^{N^\text{test}}\);\\
\subasu Sample a hypothesis \(h\) uniformly randomly from \(\mathcal{H}\);\\
\subasu Construct \emph{optimal teaching set}\footnote{The optimal teaching set~\citep{zhu2015machine} is the smallest set of \((x,y)\) pairs that uniquely identifies \(h\) among all candidates in \(\mathcal{H}\).} of \(h\) with respect to \(\mathcal{H}\);\\
\subasu Randomly duplicate elements from this optimal teaching set until its size reaches \(K\). Assign indices \(1\) through \(K\) arbitrarily to these \((x,y)\) pairs;\\
\subasu {\tiny \(\SK = [x^{(1)},y^{(1)}, \ldots, x^{(K)},y^{(K)}]\)} for hypothesis identification.
\end{asu}

% Under i.i.d.\ Generation (Assumption~\ref{asu:iid}), the samples \(\{(x^{(i)}, y^{(i)})\}\) are drawn uniformly at random from \(\mathcal{X}\). Under Opt-T Generation (Assumption~\ref{asu:optt}), a minimal \emph{teaching set} is first identified, and then possibly duplicated.

% Meta Training Objective**:
% The large models are finally trained to minimize the cross entropy loss on the meta dataset $D_M$,$$\theta^* \leftarrow \min_\theta \mathbb{E}_m\left[\sum_{i=1}^m \ell(f_\theta(x^{(i)}_M), y^{(i)}_M) \right].$$

% Pretraining dataset generation process : $$\mathcal H \sim P_\mathcal H, h \sim U(\mathcal H), D=\{(x_i, y_i\}|_{i=1}^n \sim P_{\mathcal X, \mathcal Y}$$. 


% Testing dataset generation process:
% 1. $$\mathcal H \sim P'_\mathcal H, h \sim U(\mathcal H), D=\{(x_i, y_i\}|_{i=1}^n \sim P_{\mathcal X, \mathcal Y}$$. 
% 2.  $$\mathcal H \sim P'_\mathcal H, h \sim U(\mathcal H), D= OPT(h^* \mathcal H)$$. 

% *****************************

% We consider two types of input $i$ and output $o$ for the meta learning tasks in this paper.
% The first type has input consists of a finite hypothesis class $\mathcal{H}=\{h^{(1)},h^{(2)},\ldots\}$ with finite input space
% $
% \mathcal{X} = \{x_1, x_2, \ldots, x_{|\mathcal{X}|}\},
% $
% and binary output space
% $
% \mathcal{Y} = \{0,1\},
% $, a sequence $\Skx=(x^{(1)},y^{(1)},\ldots,x^{(k)},y^{(k)},x^{(k+1)})$ containing $k$ pairs of $(x,y)$ satisfying $y=h(x),h\in\mathcal{H}$ and an additional query token $x^{(k+1)}$, and the output is $y^{(k+1)}=f(x^{(k+1)})$.
% The second type has input consists of the finite hypothesis class $\mathcal{H}$ and a sequence $\SK= (x^{(1)},y^{(1)},\ldots,x^{(K)},y^{(K)})$ containing $K$ pairs of $(x,y)$ satisfying $y=h(x),h\in\mathcal{H}$ and the output is the underline $h$.
% Both tasks need to distinguish the underline hypothesis of given $(x,y)$ pairs and then predict the underline hypothesis or use it to predict the $y$ of given $x$, which are loosely related to the empirical risk minimization (ERM), aiming at finding the hypothesis minimizing the risk on a given dataset $\mathcal{D}$.
% As a summary, the first type has $i_{\text{I},k}=(\mathcal{H},\Skx), o_{\text{I},k}=y^{(k+1)}$, namely Type I; and the second type has $i_{\text{I},k}=(\mathcal{H},\SK), o_{\text{I},k}=h$, namely Type II.

% We consider multi-task training on Type I ($k\in\{0,1,\ldots,K-1\}$) and Type II, with hypothesis class $\mathcal{H}$ sampled from a predefined training hypothesis classes $\{\mathcal{H}^\text{train}_i\}_{i=1}^{N^\text{train}}$.
% During training, for Type I and II, we generate $\Skx$, $y^{(k+1)}$ and $\SK$ via $h\sim\text{Uniform}(\mathcal{H})$, and then $\{x^{(i)}\}\overset{\text{i.i.d.}}{\sim}\text{Uniform}(\mathcal{X})$ and $y^{i}=h(x^{(i+1)})$.
% During testing, we perform Type I evaluation on new hypothesis classes $\{\mathcal{H}^\text{test}_i\}_{i=1}^{N^\text{test}}$ with $\Skx$ generated the same way as training.
% We perform Type II evaluation on new hypothesis classes $\{\mathcal{H}^\text{test}_i\}_{i=1}^{N^\text{test}}$ as well, but with two different generation for $\SK$: (i) the same as training, denoted (i.i.d), (ii) generating $\SK$ via firstly identifying the optimal teaching set of $h\in\mathcal{H}$, \ie, the set of minimum number of $(x,y)$ pairs distinguish $h$ from $\mathcal{H}$ and duplicating samples in it until number $K$.

\subsection{Meta Training and Testing}
\paragraph{Training}
Given a set of training hypothesis classes $\{\mathcal{H}_i^\text{train}\}_{i=1}^{N^\text{train}}$, the meta-learner is trained in a multi-task setting to minimize the following loss:
\begin{align}
\label{eq:loss}
    % \theta^* \leftarrow \min_\theta \mathbb{E}\left[\mathcal{L}_1(f_\theta(i_{\text{II}}), o_{\text{II}}) + \sum_{k=1}^K \mathcal{L}_2(f_\theta(i_{\text{I},k}), o_{\text{I},k}) \right],
    \mathcal{L} = \mathcal{L}_1(f_\theta(i_{\text{II},K}), o_{\text{II},K}) + \sum_{k=1}^K \mathcal{L}_2(f_\theta(i_{\text{I},k}), o_{\text{I},k}),
\end{align}
where we generate $\mathcal{H}$, $h$, and $\SK$ following \hyperref[asu:iid]{i.i.d. Generation}, inherently defining $(i_{\text{II},K}, o_{\text{II},K})$ and $(i_{\text{I},k}, o_{\text{I},k})$.
The loss is indeed implemented with additional terms, and we will further clarify the loss in Sec.~\ref{subsec:framework}, Eq.~\ref{eq:lossTF}.

\paragraph{Testing}
Given a set of testing hypothesis classes $\{\mathcal{H}_i^\text{test}\}_{i=1}^{N^\text{test}}$, we consider two types of testing.
\begin{itemize}[topsep=0.1em, partopsep=0em, leftmargin=*]
    \item \textbf{Label prediction}: We generate $(i_{\text{I},k}, o_{\text{I},k})$ following \hyperref[asu:iid]{i.i.d. Generation}, and then measure whether the learner $f$ predict $f(i_{\text{I},k})$ correctly for each $k\in[K]$;
    \item \textbf{Hypothesis identification}: We generate $(i_{\text{II},K}, o_{\text{II},K})$ using \hyperref[asu:optt]{Opt-T Generation} and evaluate whether the learner $f$ predicts $f(i_{\text{II}})$ correctly.
    This setting tests whether the learner acquires the ability to identify the underlying hypothesis with minimal information.
\end{itemize}


% Pretraining dataset generation process : $$\mathcal H \sim P_\mathcal H, h \sim U(\mathcal H), D=\{(x_i, y_i\}|_{i=1}^n \sim P_{\mathcal X, \mathcal Y}$$. 


% Testing dataset generation process:
% 1. $$\mathcal H \sim P'_\mathcal H, h \sim U(\mathcal H), D=\{(x_i, y_i\}|_{i=1}^n \sim P_{\mathcal X, \mathcal Y}$$. 
% 2.  $$\mathcal H \sim P'_\mathcal H, h \sim U(\mathcal H), D= OPT(h^* \mathcal H)$$. 

\subsection{Four Types of Generalization}
\paragraph{Hypothesis universe $\mathcal{H}^{\text{uni}}$}
Given an input space 
$
    \mathcal{X} = \{x_1, x_2, \ldots, x_{|\mathcal{X}|}\}
$
and a binary output space
$
    \mathcal{Y} = \{0, 1\},
$
We define the hypothesis universe $\mathcal{H}^{\text{uni}}=\mathcal{Y}^{\mathcal{X}}$ as the collection of all possible binary classification hypotheses.
This universe contains $M=2^{|\mathcal{X}|}$ distinct hypotheses, serving as a hypothesis pool to constructing training and testing hypothesis classes.

\begin{figure}[h!]
    \centering
    %\vspace{-1.1cm}
    \includegraphics[width = 0.5\textwidth]{fig/generalization.pdf}
    %\vspace{-0.2cm}
    \caption{\textbf{Four types of generalization.}
    An illustration of the four types of generalization.}
    \label{fig:framework}
    %\vspace{-0.3cm}
\end{figure}

In meta-learning, the goal is to train a model that is able to rapidly adapt to new tasks.  
Testing on new tasks can be considered as measuring the OOD generalization.
Under our ICL-HCG framework, we consider four types of OOD generalizations.
First, we examine whether the learner generalizes to a new testing hypothesis class (the hypothesis class is unseen during training) that may or may not contain hypotheses seen during training, referred to as in-distribution (ID) and out-of-distribution (OOD) hypothesis class generalization, respectively.

\begin{definition}[ID Hypothesis Class Generalization]
\label{def:SpaceGeneralization}
Given $\mathcal{H}^{\mathrm{uni}}$ of size $M$, we enumerate all $C(M, m) = \frac{M!}{m!(M - m)!}$ distinct hypothesis classes, each containing $m$ 
hypotheses.
We then randomly \emph{subsample} these classes into disjoint training and testing subsets, ensuring that no testing hypothesis class appears in the training set (although individual hypotheses may overlap).
By training on randomly selected training hypothesis classes and evaluating on unseen testing hypothesis classes, we assess generalization to new hypothesis classes consisting of ID hypotheses.
\end{definition}

\begin{definition}[OOD Hypothesis Class Generalization]
\label{def:HypothesisGeneralization}
Given $\mathcal{H}^{\text{uni}}$ of size $M$, 
we partition it into disjoint training and testing subsets of sizes 
$M^\text{ID}$ and $M^\text{OOD}$, respectively. 
We then generate training hypothesis classes from $M^\text{ID}$ and testing hypothesis classes from $M^\text{OOD}$, each containing $m$ hypotheses. 
We train the learner on the training hypothesis classes and evaluate on the testing hypothesis classes. 
Because no testing hypothesis appears during training, 
this setup probes how well the learner generalizes
to entirely new hypotheses, \ie, OOD hypotheses.
\end{definition}

We then consider whether the learner can generalize to hypothesis classes of various sizes. Building on the concepts of ID and OOD hypothesis class generalization, we introduce size generalizations as follows.

\begin{definition}[ID Hypothesis Class Size Generalization]
\label{def:Space&SizeGeneralization}
Building on the setting of ID hypothesis class generalization, while maintaining non-identical training and testing hypothesis classes, we allow training hypothesis class to include various number of hypotheses $m\in\mathcal{M}\subsetneqq[L]$.
We investigate whether the learner can perform well on hypothesis classes with other sizes $m\in [L]\setminus\mathcal{M}$, where $[L]=\{1,2,\ldots,L\}$.
\end{definition}

\begin{definition}[OOD Hypothesis Class Size Generalization]
\label{def:Hypothesis&SizeGeneralization}
Based on the setting of OOD hypothesis class generalization, while maintaining non-identical training and testing hypotheses, we allow training hypothesis class to include various number of hypotheses $m\in\mathcal{M}\subsetneqq[L]$.
We investigate whether the learner can perform well on hypothesis classes with various sizes $m\in [L]\setminus\mathcal{M}$, where $[L]=\{1,2,\ldots,L\}$.
\end{definition}



\begin{figure}[h!]
    \centering
    %\vspace{-1.1cm}
    \includegraphics[width = 0.475\textwidth]{fig/framework_simplified.pdf}
    %\vspace{-0.2cm}
    \caption{\textbf{Learning ICL-HCG via Transformer.}
    We begin by sampling a subset from the hypothesis universe as the hypothesis class $\mathcal{H}$.
    Next, we encode the hypothesis class $\mathcal{H}$ and concatenate it with context query into a unified sequences of token.
    This sequences is fed into a Transformer model for training with next-token prediction, and testing for evaluating the accuracy on $y$ and hypothesis identification.
    (This figure is an simplified illustration. Please refer to Appendix~\ref{app:prefix} and Fig.~\ref{fig:frameworkfull} for the full details.)}
    \label{fig:framework}
    %\vspace{-0.3cm}
\end{figure}
\subsection{Learning ICL-HCG via Transformer}
\label{subsec:framework}
This section details how Transformer learns ICL-HCG.
As shown in Fig.~\ref{fig:framework},
the hypothesis class $\mathcal{H}$ is first converted to a hypothesis prefix with randomly assigned hypothesis indexes, then concatenated with context query representing sequence $\SK$ as a unified sequence $s$.
%We list the pseudo algorithm of training and testing in the Appendix Algorithm~\ref{alg:framework}, namely ``Meta Training and Testing of ICL-HCG.'' 

% During training, $\mathcal{D}$ is constructed by $(x,h(x))$ pairs via uniformly randomly sampling $x$ from $\mathcal{X}$; while
% during testing, we consider two sample strategies of $(x,h(x))$ pairs in the context query: (i) ``i.i.d.'': the same as training; (ii) ``Opt-T'': given a hypothesis class $\mathcal{H}$ and a chosen hypothesis $h\in\mathcal{H}$, an optimal teaching set\footnote{An optimal teaching set is a set with the minimum number of samples to identify a hypothesis from the hypothesis class.} is derived and duplicated to number $K$ to fit the context query size $K$.
% \begin{definition}[Hypothesis Table]
%     Given $\mathcal{A}$ and hypothesis class $\mathcal{H}$ with $m$ $h$s on the domain, the corresponding hypothesis table is constructed with size $m\times |\mathcal{A}|$ such that each row in the table represents a hypothesis, each column indicates an $a\in\mathcal{A}$, and each value in the table represents the value of $h_m(x_n)$ in row $m$ and column $n$.
% \end{definition}

\paragraph{Hypothesis prefix\footnote{Please refer to Appendix~\ref{app:prefix} for the full version.}}
Given a hypothesis class $\mathcal{H}=\{h_4,h_6,h_7\}$, its hypothesis prefix with size $L=4$ is constructed as shown in Fig.~\ref{fig:framework}.
Blank hypothesis is used to fill the hypothesis prefix when $|\mathcal{H}|<L$.
A randomly assigned hypothesis index token \hz is used to label each hypothesis.
Leveraging Fig.~\ref{fig:framework} for $L=4$, {\hz}'s are assigned from a pool \{``\textcolor[RGB]{0,176,80}{A}'',``\textcolor[RGB]{0,176,80}{B}'',``\textcolor[RGB]{0,176,80}{C}'',``\textcolor[RGB]{0,176,80}{D}''\} of size $L$ without replacement\footnote{We use variable $z$ to represent the hypothesis index, and create a set of $L$ hypothesis index tokens as a pool from which each hypothesis is randomly assigned a unique index without replacement.}.

\paragraph{Context query}
Given an ICL sequence $\SK$, we append a query token ``\textcolor[RGB]{192,79,21}{\textgreater}'' after it to trigger trigger the prediction of the hypothesis index ss shown in Fig.~\ref{fig:framework}.
We name the combination of $\SK$ and ``\textcolor[RGB]{192,79,21}{\textgreater}'' as context query.

The Transformer predicts the $y$ tokens in the context query based on previous tokens and the index \hz of the underlying hypothesis based on all tokens in the sequence.
The training loss in Eq.~\ref{eq:loss} is further extended to all the tokens in the sequence and implemented as below:
\begin{align}
\label{eq:lossTF}
    \mathcal{L} = - \sum_{t=1}^{T} \log P_\theta(s_i \mid s_{<i}).
\end{align}
We summarize the pipeline in the Appendix~\ref{app:alg} Algorithm~\ref{alg:framework}.
%\section{Pseudo Algorithm for ICL-HCG}
\label{app:alg}
We summarize our meta framework for ICL-HCG in Algorithm~\ref{alg:framework}.
\begin{algorithm}
  \caption{Meta-Learning Framework for ICL-HCG}
  \label{alg:framework}
  \begin{algorithmic}[1]
    \STATE \textbf{Inputs:} a set of inputs $\mathcal{X}$, a training set of hypothesis classes $\mathcal{S}^{\text{train}}=\{\mathcal{H}_i^{\text{train}}\}_{i=1}^{N^{\text{train}}}$, a testing set of hypothesis classes $\mathcal{S}^{\text{test}}=\{\mathcal{H}_i^{\text{test}}\}_{i=1}^{N^{\text{test}}}$, batch size $B$, hypothesis prefix size $L$, and context query size $K$
    \FOR{\textbf{training epoch}}
      \STATE sample $\{\mathcal{H}_i\}_{i=1}^B \overset{\text{i.i.d.}}{\sim} \text{Uniform}(\mathcal{S}^{\text{train}})$
      \FOR{each hypothesis class $\mathcal{H} \in \{\mathcal{H}_i\}_{i=1}^B$}
        \STATE generate $h,\SK$ following \hyperref[asu:iid]{i.i.d. Generation}
        
        \STATE \textbf{// Construct sequence based on $\mathcal{H}$, $h$, and $\SK$}
        \STATE construct hypothesis prefix, context query, and hypothesis index $z$ based on $\mathcal{H}$, $h$, $\SK$
        \STATE $s \gets \text{concatenate}(\text{hypothesis prefix}, \text{context query}, z)$
    
        \STATE \textbf{// Cross-entropy loss for next token prediction}
        \STATE $\mathcal{L} \gets -\sum_{t=2}^{|s|} \log P(s_t \mid s_{<t})$
      \ENDFOR
      \STATE update model parameters using $\mathcal{L}$ of the batch
    \ENDFOR
    
    \FOR{\textbf{testing epoch}}
      \STATE sample $\{\mathcal{H}_i\}_{i=1}^B \overset{\text{i.i.d.}}{\sim} \text{Uniform}(\mathcal{S}^{\text{test}})$
      
      \FOR{each hypothesis class $\mathcal{H} \in \{\mathcal{H}_i\}_{i=1}^B$}
        \STATE generate $h,\SK$ via:
        \STATE \quad \textbf{either} following \hyperref[asu:iid]{i.i.d. Generation}
        \STATE \quad \textbf{or} following \hyperref[asu:optt]{Opt-T Generation}
        \STATE \textbf{construct sequence $s$ based on $\mathcal{H}$, $h$, and $\SK$}
        \STATE \textbf{evaluate the prediction accuracy on $y$, $z$, etc}
      \ENDFOR
    \ENDFOR
  \end{algorithmic}
\end{algorithm}



%\subsection{Empirical Risk Minimization for Binary Classification (come later, what is the data ERM for a fix, fix hypothesis)}
\label{sec:ERM}
In this work, we consider the binary classification problem where the input space is finite. Formally, let the input space be defined as
$
\mathcal{X} = \{x_1, x_2, \ldots, x_{|\mathcal{X}|}\}
$,
and the label space as
$
\mathcal{Y} = \{0, 1\}
$.
The learning task is characterized by a hypothesis space
$
\mathcal{H} \subseteq \mathcal{Y}^{\mathcal{X}}
$,
where each hypothesis $ h \in \mathcal{H} $ is a deterministic function
$
h: \mathcal{X} \rightarrow \mathcal{Y}
$.
A realizable instance of the binary classification problem is specified by the tuple
$
(\mathcal{X}, \mathcal{Y}, P, h^*)
$,
where $P$ is a probability distribution over the input space $\mathcal{X}$, and $h^*:\mathcal{X} \rightarrow \mathcal{Y}$ is the target hypothesis.

During learning, a learner $A$ receives a training dataset $\mathcal{D}=\{(x^{(i)}, y^{(i)})\}_{i=1}^K$ and it is tasked to find a hypothesis that produces least generalization risk $R(A(\mathcal{D}))$ on future test sample drawn from $P$ and $h^*$.
Empirical risk minimization (ERM) is one of the most popular learning algorithm. It learns by minimizing the empirical risk on the training dataset $\mathcal D$ with respect to a loss function $l:\mathcal{Y} \times \mathcal{Y} \rightarrow \mathbb R^+$ and is given as,
$$
\hat h_\mathcal{D} = \text{ERM}(\mathcal{D},\mathcal{H}) \leftarrow \min_{h \in \mathcal{H}} \sum_{i=1}^K l(h(x^{(i)}), y^{(i)}).
$$
% Once learning is complete, the learner uses learnt hypothesis $\hat h_D$ to make prediction on samples from $P, h^*$ thereby suffering a risk of, $$R(\hat h_D) \leftarrow \mathbb E_{(x, y) \sim P, h^*}[l(\hat h_D(x), y)].$$
In this paper, we focus on $0/1$ risk, $$l(\hat{y},y)=
\begin{cases}
    1, & \text{if } \hat{y}=y\\
    0, & \text{if } \hat{y}\neq y
\end{cases}.
$$
\begin{remark}
ERM focuses mainly on sample complexity, that is, determining the minimum number of samples required to identify a hypothesis $\hat{h}$ with a risk below a specified threshold.
Differently, in this paper, we explore whether Transformer models can accurately pinpoint the exact hypothesis when provided with sufficient number of samples.
\end{remark}

% In theory, Probably Approximately Correct(PAC) learning completely characterizes the sample complexity of learning. It states that for any learning instance and $(\epsilon, \delta) \in (0,1)^2$ if $n\geq O(\frac{1}{\epsilon}(d(\mathcal H)+\log(\frac{1}{\delta})))$, then,  $$\text{w.p. $\geq 1-\delta$}, \qquad R(\hat h_D) \leq \epsilon.$$
% where $d(\mathcal H)$ is the VC dimension of hypothesis class $\mathcal H$ which is a measure of its complexity. Therefore, to learn the target hypothesis up to an error $\epsilon$, computing the ERM on an i.i.d. sample of size $O(\frac{d}{\epsilon})$ is sufficient.


%%% motivating incontext learning with finite hypothesis class

% In recent literature, several works {cite} have shown impressive capability of transformers to in-context learn ERM on a dataset $D$ at inference time which is known as In-context Learning(ICL). However, most of the works have focused on ICL learning settings where hypothesis class is defined implicitly by a linear $\mathbb R^d$ space. This begs the question of whether transformers can do ERM in more explicit learning settings where hypothesis class are specified as input to the model. Towards this end we seek to answer the following question:

% \textbf{Can a transformers learn to do in-context learning conditioned on finite/tabular hypothesis classes?}

% A couple of recent works have tried to address this problem by conditioning the hypothesis class on a noisy version of the target hypothesis {cite}. Contrary to this, a finite hypothesis class conditions the universal hypothesis space $\mathcal{A}^\mathcal{Y}$ by specifying a subset of hypothesis $\mathcal H$ that contains a the target hypothesis to solve the task instance. In real world learning scenario, this can be viewed   The explicit conditioning on finite hypothesis class also provide an avenue to test the generalization capability of ICL to generalize to different size of hypothesis tables thereby helping us to understand ERM capability of transformers in a more controlled setup. 

%%% Ziqian's previous writeup

% \subsection{Empirical Risk Minimization (ERM)}
% Consider a training dataset $\{(x_i, y_i)\}_{i=1}^n$, where each $x_i \in \mathcal{A}$ is a feature vector and $y_i \in \mathcal{Y}$ is the corresponding label or target value. Let $\mathcal{H}$ be a hypothesis class of functions $h: \mathcal{A} \to \mathcal{Y}$, and let $L:\mathcal{Y}\times\mathcal{Y}\to\mathbb{R}_{\geq 0}$ be a loss function that measures the discrepancy between the predicted value $h(x_i)$ and the true value $y_i$.

% The \emph{empirical risk} of a hypothesis $h \in \mathcal{H}$ is defined as:
% \[
% \hat{R}_n(h) = \frac{1}{n} \sum_{i=1}^n L(y_i, h(x_i)).
% \]

% The \emph{Empirical Risk Minimization} principle seeks a hypothesis $\hat{h} \in \mathcal{H}$ that minimizes the empirical risk:
% \[
% \hat{h} = \arg\min_{h \in \mathcal{H}} \hat{R}_n(h).
% \]



% In order to have a set of hypotheses to construct $\mathcal{H}$, given $\mathcal{X}$, assuming $|\mathcal{Y}|=2$ (binary classification), we can generate $2^{|\mathcal{X}|}$ hypotheses (all possible hypotheses with binary labels), then uniformly randomly sample $m$ hypotheses from the set to construct a hypothesis class.
% Each space generalization with $m$ hypotheses can be further converted to $P(L,m) = \frac{L!}{(L - m)!}$ hypothesis prefixes.
% Then, $(x,y)$ pairs generated from a hypothesis $h$ sampled from the hypothesis class $\mathcal{H}$ are provided as dataset of context query for the Transformer to identify a correct hypothesis indexed \hz and predict the index.

% With this framework, we are interested in whether a trained Transformer is able to generalized to perform ERM on new hypothesis classes.
% Starting with introducing the hypothesis universe, we further deliver the four generalization cases considered in our experiments.




\section{Fine-Tuning Experiments}
This section validates that our dataset can enhance the GUI grounding capabilities of VLMs and that the proposed functionality grounding and referring are effective fine-tuning tasks.
\subsection{Experimental Settings}
\noindent\textbf{Evaluation Benchmarks} We base our evaluation on the UI grounding benchmarks for various scenarios: \textbf{FuncPred} is the test split from our collected functionality dataset. This benchmark requires a model to locate the element specified by its functionality description. \textbf{ScreenSpot}~\citep{cheng2024seeclick} is a benchmark comprising test samples on mobile, desktop, and web platforms. It requires the model to locate elements based on short instructions. \textbf{RefExp}~\citep{Bai2021UIBertLG} is to locate elements given crowd-sourced referring expressions. \textbf{VisualWebBench (VWB)}~\citep{liu2024visualwebbench} is a comprehensive multi-modal benchmark assessing the understanding capabilities of VLMs in web scenarios. We select the element and action grounding tasks from this benchmark. To better align with high-level semantic instructions for potential agent requirements and avoid redundancy evaluation with ScreenSpot, we use ChatGPT to expand the OCR text descriptions in the original task instructions, such as \textit{Abu Garcia College Fishing} into functionality descriptions like \textit{This element is used to register for the Abu Garcia College Fishing event}.
\textbf{MOTIF}~\citep{Burns2022ADF} requires an agent to complete a natural language command in mobile Apps.
For all of these benchmarks, we report the grounding accuracy (\%): $\text { Acc }= \sum_{i=1}^N \mathbf{1}\left(\text {pred}_i \text { inside GT } \text {bbox}_i\right) / N \times 100 $ where $\mathbf{1}$ is an indicator function and $N$ is the number of test samples. This formula denotes the percentage of samples with the predicted points lying within the bounding boxes of the target elements.

\noindent\textbf{Training Details}
We select Qwen-VL-10B~\citep{bai2023qwen} and SliME-8B~\citep{slime} as the base models and fine-tune them on 25k, 125k, and 702k samples of the AutoGUI training data to investigate how the AutoGUI data enhances the UI grounding capabilities of the VLMs. The models are fine-tuned on 8 A100 GPUs for one epoch. We follow SeeClick~\citep{cheng2024seeclick} to fine-tune Qwen-VL with LoRA~\citep{hu2022lora} and follow the recipe of SliME~\citep{slime} to fine-tune it with only the visual encoder frozen (More details in Sec.~\ref{sec:supp:impl details}).

\noindent\textbf{Compared VLMs}
We compare with both general-purpose VLMs (i.e., LLaVA series~\citep{liu2023llava,liu2024llavanext}, SliME~\citep{slime}, and Qwen-VL~\citep{bai2023qwen}) and UI-oriented ones (i.e., Qwen2-VL~\citep{qwen2vl}, SeeClick~\citep{cheng2024seeclick}, CogAgent~\citep{hong2023cogagent}). SeeClick finetunes Qwen-VL with around 1 million data combining various data sources, including a large proportion of human-annotated UI grounding/referring samples. CogAgent is trained with a huge amount of text recognition, visual grounding, UI understanding, and publicly available text-image datasets, such as LAION-2B~\citep{LAION5B}. During the evaluation, we manually craft grounding prompts suitable for these VLMs.
\subsection{Experimental Results and Analysis}
\begin{table}[]
\scriptsize
\centering
\caption{\textbf{Element grounding accuracy on the used benchmarks.} We compare the base models fine-tuned with our AutoGUI data and representative open-source VLMs. The results show that the two base models (i.e. Qwen-VL and SliME-8B) obtain significant performance gains over the benchmarks after being fine-tuned with AutoGUI data. Moreover, increasing the AutoGUI data size consistently improves grounding accuracy, demonstrating notable scaling effects. $\dag$ means the metric value is borrowed from the benchmark paper. $*$ means using additional SeeClick training data.}
\label{tab:eval results}
\begin{tabular}{@{}cccccccccc@{}}
\toprule
Type & Model    & Size    & FuncPred & VWB EG & VWB AG & MoTIF & RefExp & ScreenSpot  \\ \midrule
\multirow{5}{*}{General} & LLaVA-1.5~\citep{liu2023llava} & 7B & 3.2      &        12.1$^{\dag}$        &     13.6$^{\dag}$           &  7.2   &  4.2 & 5.0 & \\
 & LLaVA-1.5~\citep{liu2023llava} & 13B & 5.8      &           16.7     &        9.7        &   12.3 &  20.3   & 11.2 &  \\
 & LLaVA-1.6~\citep{liu2024llavanext} & 34B &  4.4      &      19.9          &    17.0            &   7.0 &  29.1  & 10.3 &  \\
 & SliME~\citep{slime} & 8B &  3.2  &   6.1       &     4.9     & 7.0  &  8.3  &  13.0  \\ 

 & Qwen-VL~\citep{bai2023qwen} & 10B &  3.0     &      1.7          &      3.9          &    7.8 &  8.0  & 5.2$^{\dag}$   \\ 
 \midrule
\multirow{3}{*}{UI-VLM} &  Qwen2-VL~\citep{bai2023qwen}  & 7B     &     7.8       &    3.9        &  3.9  &  16.7 & 32.4 & 26.1    \\
 & CogAgent~\citep{hong2023cogagent} & 18B    &  29.3   &    \underline{55.7}      &    \textbf{59.2}      & \textbf{24.7}   & 35.0 &  47.4$^{\dag}$  \\
 & SeeClick~\citep{cheng2024seeclick} & 10B    &    19.8     &    39.2           &     27.2           & 11.1  &  \textbf{58.1}  & \underline{53.4}$^{\dag}$ \\ 
\midrule
\multirow{4}{*}{Finetuned} &  Qwen-VL-AutoGUI25k & 10B      &    14.2     &      12.8         &    12.6           &   10.8    &  12.0 & 19.0    \\
 & Qwen-VL-AutoGUI125k  & 10B       &     25.5     &      23.2         &        29.1       &    11.5   &  14.9 & 32.0     \\ 
 & Qwen-VL-AutoGUI702k  & 10B       &   43.1   &    38.0       &     32.0    &  15.5  & 23.9 &    38.4   \\
& Qwen-VL-AutoGUI702k$^*$   & 10B     &  \underline{50.0}  &    \textbf{56.2}    &  \underline{45.6}  & \underline{21.0} & \underline{51.5} & \textbf{54.2}      \\
\midrule
\multirow{3}{*}{Finetuned} & SliME-AutoGUI25k  & 8B     &   28.0   &     14.0      &      10.6      &  14.3   & 18.4 & 27.2   \\
 & SliME-AutoGUI125k   & 8B      &   39.9    &  22.0   &     12.0       &  17.8  & 22.1 &  35.0     \\
 & SliME-AutoGUI702k   & 8B      &     \textbf{62.6}   &       25.4        &     13.6          &   20.6    & 26.7 & 44.0 &          \\
\bottomrule
\end{tabular}
\end{table}
\vspace{-2mm}


\noindent\textbf{A) AutoGUI functionality annotations effectively enhance VLMs' UI grounding capabilities and achieve scaling effects.} We endeavor to show that the element functionality data autonomously collected by AutoGUI contributes to high grounding accuracy. The results in Tab.~\ref{tab:eval results} demonstrate that on all benchmarks the two base models achieve progressively rising grounding accuracy as the functionality data size scales from 25k to 702k, with SliME-8B's accuracy increasing from merely \textbf{3.2} and \textbf{13.0} to \textbf{62.6} and \textbf{44.0} on FuncPred and ScreenSpot, respectively. This increase is visualized in Fig.~\ref{fig:funcpred scaling success} showing that increasing AutoGUI data amount leads to more precise localization performance.

After fine-tuning with AutoGUI 702k data, the two base models surpass SeeClick, the strong UI-oriented VLM on FuncPred and MOTIF. We notice that the base models lag behind SeeClick and CogAgent on ScreenSpot and RefExp, as the two benchmarks contain test samples whose UIs cannot be easily recorded (e.g., Apple devices and Desktop software) as training data, causing a domain gap. Nevertheless, SliME-8B still exhibits noticeable performance improvements on ScreenSpot and RefExp when scaling up the AutoGUI data, suggesting that the AutoGUI data helps to enhance grounding accuracy on the out-of-domain tasks.

To further unleash the potential of the AutoGUI data, the base model, Qwen-VL, is finetuned with the combination of the AutoGUI and SeeClick UI-grounding data. This model becomes the new state-of-the-art on FuncPred, ScreenSpot, and VWB EG, surpassing SeeClick and CogAgent. This result suggests that our AutoGUI data can be mixed with existing UI grounding training data to foster better UI grounding capabilities.

In summary, our functionality data can endow a general VLM with stronger UI grounding ability and exhibit clear scaling effects as the data size increases.


\begin{table}[]
\centering
\footnotesize
\caption{\textbf{Comparing the AutoGUI functionality annotation type with existing types}. Qwen-VL is fine-tuned with the three annotation types. The results show that our functionality data leads to superior grounding accuracy compared with the naive element-HTML data and the condensed functionality annotations.}
\label{tab:ablation}
\begin{tabular}{@{}ccccc@{}}
\toprule
Data Size             & Variant          & FuncPred & RefExp & ScreenSpot \\ \midrule
\multirow{3}{*}{25k}  & w/ Elem-HTML data     &  5.3      &  4.5   &    5.7     \\
                      & w/ Condensed Func. Anno.     &  3.8   &  3.0  &   4.8      \\
                      & w/ Func. Anno. (Ours full)         &    \textbf{21.1}    &   \textbf{10.0}   &   \textbf{16.4}    \\ \midrule
\multirow{3}{*}{125k} & w/ Elem-HTML data     &  15.5   &  7.8  &   17.0      \\
                      & w/ Condensed Func. Anno.     &  14.1   &  11.7  &   23.8      \\
                      & w/ Func. Anno. (Ours full)         &  \textbf{24.6}   &  \textbf{12.7}  &   \textbf{27.0}    \\ \bottomrule
\end{tabular}
\end{table}



\noindent\textbf{B) Our functionality annotations are effective for enhancing UI grounding capabilities.} To assess the effectiveness of functionality annotations, we compare this annotation type with two existing types: 1) \textbf{Naive element-HTML pairs}, which are directly obtained from the UI source code~\citep{hong2023cogagent} and associate HTML code with elements in specified areas of a screenshot. Examples are shown in Fig.~\ref{fig: functionality vs others}. To create these pairs, we replace the functionality annotations with the corresponding HTML code snippets recorded during trajectory collection. 2) \textbf{Brief functionality descriptions} that are generated by prompting GPT-4o-mini\footnote{https://openai.com/index/gpt-4o-mini-advancing-cost-efficient-intelligence/} to condense the AutoGUI functionality annotations. For example, a full description such as \textit{`This element provides access to a documentation category, allowing users to explore relevant information and guides'} is shortened to \textit{`Documentation category access'}.

After experimenting with Qwen-VL~\citep{bai2023qwen} at the 25k and 125k scales, the results in Tab.~\ref{tab:ablation} show that fine-tuning with the complete functionality annotations is superior to the other two types. Notably, our functionality annotation type yields the largest gain on the challenging FuncPred benchmark that emphasizes contextual functionality grounding. In contrast, the Elem-HTML type performs poorly due to the noise inherent in HTML code (e.g., numerous redundant tags), which reduces fine-tuning efficiency. The condensed functionality annotations are inferior, as the consensing loses details necessary for fine-grained UI understanding. In summary, the AutoGUI functionality annotations provide a clear advantage in enhancing UI grounding capabilities.


\subsection{Failure Case Analysis}
After analyzing the grounding failure cases, we identified several failure patterns in the fine-tuned models: a) difficulty in accurately locating small elements; b) challenges in distinguishing between similar but incorrect elements; and c) issues with recognizing icons that have uncommon shapes. Please refer to Sec.~\ref{sec:supp:case analysis} for details.



\section{Discussion}

This study presents PathFinder, a multi-modal, multi-agent AI framework designed to emulate the multi-scale, iterative diagnostic approach of expert pathologists for histopathology whole slide images (WSIs). By integrating Triage, Navigation, Description, and Diagnosis Agents, PathFinder collaboratively gathers evidence to deliver accurate, interpretable diagnoses with natural language explanations. Notably, it surpasses state-of-the-art methods and the average performance of human experts in melanoma diagnosis, setting a new benchmark in AI-driven pathology.

PathFinder has the potential to accelerate diagnostic workflows, reducing the reliance on manual examination and enabling timely patient care in clinical settings. Its natural language descriptions provide interpretability, facilitating the validation of AI-generated diagnoses by pathologists. Moreover, its integration of vision-language models (VLMs) and large language models (LLMs) highlights the promise of multi-modal AI in delivering scalable, specialized diagnostic tools that could improve access to pathology expertise.

\noindent\textbf{Limitations.} Despite its strengths, PathFinder has limitations. The framework relies on pre-existing datasets and significant computational resources, posing challenges in resource-constrained environments. Additionally, the complexity of the Navigation Agent’s decision-making process and occasional hallucinations by the Description Agent could affect transparency and accuracy of the decision-making process. Future work should address these issues by enhancing dataset diversity, computational efficiency, and patch selection strategies, further advancing PathFinder's potential as a transformative tool in AI-assisted pathology.

In this paper, we systematically investigate the position bias problem in the multi-constraint instruction following. To quantitatively measure the disparity of constraint order, we propose a novel Difficulty Distribution Index (CDDI). Based on the CDDI, we design a probing task. First, we construct a large number of instructions consisting of different constraint orders. Then, we conduct experiments in two distinct scenarios. Extensive results reveal a clear preference of LLMs for ``hard-to-easy'' constraint orders. To further explore this, we conduct an explanation study. We visualize the importance of different constraints located in different positions and demonstrate the strong correlation between the model's attention distribution and its performance.
%\section{Impact Statements}
By explicitly including instructions alongside training examples, this work contributes to a deeper understanding of in-context learning (ICL) and serves as a testbed for future theoretical and empirical research.
The primary goal is to advance the field of Machine Learning by demonstrating how models can leverage task descriptions to enhance generalization and reduce reliance on large labeled datasets.
In doing so, the findings may promote better insights into Large Language Models (LLMs) and inspire the development of more efficient, adaptable AI systems.

\section*{Acknowledgements}
This work of Kangwook Lee is supported 
%in part 
by NSF Award DMS-2023239, NSF CAREER Award CCF-2339978, Amazon Research Award, and a grant from FuriosaAI.

% \section*{Impact Statement}
% This paper presents work whose goal is to advance the field of Machine Learning. There are many potential societal consequences of our work, none which we feel must be specifically highlighted here.

%\input{tex_main/6_impactstatemtent}
% In the unusual situation where you want a paper to appear in the
% references without citing it in the main text, use \nocite
%\nocite{langley00}

\bibliography{example_paper}
\bibliographystyle{icml2025}


%%%%%%%%%%%%%%%%%%%%%%%%%%%%%%%%%%%%%%%%%%%%%%%%%%%%%%%%%%%%%%%%%%%%%%%%%%%%%%%
%%%%%%%%%%%%%%%%%%%%%%%%%%%%%%%%%%%%%%%%%%%%%%%%%%%%%%%%%%%%%%%%%%%%%%%%%%%%%%%
% APPENDIX
%%%%%%%%%%%%%%%%%%%%%%%%%%%%%%%%%%%%%%%%%%%%%%%%%%%%%%%%%%%%%%%%%%%%%%%%%%%%%%%
%%%%%%%%%%%%%%%%%%%%%%%%%%%%%%%%%%%%%%%%%%%%%%%%%%%%%%%%%%%%%%%%%%%%%%%%%%%%%%%
% \newpage
% The work of Kangwook Lee is supported in part by NSF CAREER Award CcF-2339978, Amazon Research Award, and a grant from
% FuriosaAl.
\appendix
\onecolumn
\section{Pseudo Algorithm for ICL-HCG}
\label{app:alg}
We summarize our meta framework for ICL-HCG in Algorithm~\ref{alg:framework}.
\begin{algorithm}
  \caption{Meta-Learning Framework for ICL-HCG}
  \label{alg:framework}
  \begin{algorithmic}[1]
    \STATE \textbf{Inputs:} a set of inputs $\mathcal{X}$, a training set of hypothesis classes $\mathcal{S}^{\text{train}}=\{\mathcal{H}_i^{\text{train}}\}_{i=1}^{N^{\text{train}}}$, a testing set of hypothesis classes $\mathcal{S}^{\text{test}}=\{\mathcal{H}_i^{\text{test}}\}_{i=1}^{N^{\text{test}}}$, batch size $B$, hypothesis prefix size $L$, and context query size $K$
    \FOR{\textbf{training epoch}}
      \STATE sample $\{\mathcal{H}_i\}_{i=1}^B \overset{\text{i.i.d.}}{\sim} \text{Uniform}(\mathcal{S}^{\text{train}})$
      \FOR{each hypothesis class $\mathcal{H} \in \{\mathcal{H}_i\}_{i=1}^B$}
        \STATE generate $h,\SK$ following \hyperref[asu:iid]{i.i.d. Generation}
        
        \STATE \textbf{// Construct sequence based on $\mathcal{H}$, $h$, and $\SK$}
        \STATE construct hypothesis prefix, context query, and hypothesis index $z$ based on $\mathcal{H}$, $h$, $\SK$
        \STATE $s \gets \text{concatenate}(\text{hypothesis prefix}, \text{context query}, z)$
    
        \STATE \textbf{// Cross-entropy loss for next token prediction}
        \STATE $\mathcal{L} \gets -\sum_{t=2}^{|s|} \log P(s_t \mid s_{<t})$
      \ENDFOR
      \STATE update model parameters using $\mathcal{L}$ of the batch
    \ENDFOR
    
    \FOR{\textbf{testing epoch}}
      \STATE sample $\{\mathcal{H}_i\}_{i=1}^B \overset{\text{i.i.d.}}{\sim} \text{Uniform}(\mathcal{S}^{\text{test}})$
      
      \FOR{each hypothesis class $\mathcal{H} \in \{\mathcal{H}_i\}_{i=1}^B$}
        \STATE generate $h,\SK$ via:
        \STATE \quad \textbf{either} following \hyperref[asu:iid]{i.i.d. Generation}
        \STATE \quad \textbf{or} following \hyperref[asu:optt]{Opt-T Generation}
        \STATE \textbf{construct sequence $s$ based on $\mathcal{H}$, $h$, and $\SK$}
        \STATE \textbf{evaluate the prediction accuracy on $y$, $z$, etc}
      \ENDFOR
    \ENDFOR
  \end{algorithmic}
\end{algorithm}

\section{Implementation Detail of Hypothesis Prefix and Context Query}
\label{app:prefix}
\begin{figure}[h!]
    \centering
    %\vspace{-1.1cm}
    \includegraphics[width = 0.75\textwidth]{fig/framework.pdf}
    %\vspace{-0.2cm}
    \caption{\textbf{The framework.} We convert hypothesis class $\mathcal{H}$ and ICL sequence $\SK$ into sequences of tokens, concatenate them and input to Transformer.
    Then we examine whether Transformer can predict correct $y$ and $z$ values.}
    \label{fig:frameworkfull}
    %\vspace{-0.3cm}
\end{figure}
\paragraph{Hypothesis prefix}
\label{def:HypothesisPrefixFull}
Given a hypothesis class $\mathcal{H}$ and its hypothesis table, the correspongding hypothesis prefix with hypothesis prefix's content length $L$ is constructed as shown in Fig.~\ref{fig:frameworkfull}.
The token ``\textcolor[RGB]{78,149,217}{P}'' serves as the padding token to separate hypotheses,
the token ``\textcolor[RGB]{216,110,204}{;}'' serves as the separation token to separate $(x,y)$ pairs,
the token ``\textcolor[RGB]{127,127,127}{N}'' serves as the empty token to fill a blank hypothesis,
and the token ``\textcolor[RGB]{192,79,21}{\textgreater}'' is used to connect $(x,y)$ pairs of the hypothesis to a randomly assigned hypothesis index \hz\footnote{We use variable $z$ to represent the hypothesis index.}.
In the illustrated example in Fig.~\ref{fig:frameworkfull}, the randomly assigned indexes {\hz}'s are sampled from $M=4$ hypothesis index tokens \{``\textcolor[RGB]{0,176,80}{A}'',``\textcolor[RGB]{0,176,80}{B}'',``\textcolor[RGB]{0,176,80}{C}'',``\textcolor[RGB]{0,176,80}{D}''\} without replacement\footnote{A set of $L$ hypothesis index tokens are created serve as the pool from which the hypothesis indexes are randomly sampled without replacement.}.

\paragraph{Context query}
Given an ICL sequence $\SK$ with $K$ pairs of $(x,y)$, the context query of size $K$ is constructed to represent the ICL sequence and trigger the prediction of the hypothesis index with padding token ``\textcolor[RGB]{78,149,217}{P}'', separation token token ``\textcolor[RGB]{216,110,204}{;}'', and query token ``\textcolor[RGB]{192,79,21}{\textgreater}'' as shown in Fig.~\ref{fig:frameworkfull}.
\section{Additional Details of Experiments}
\label{app:exp}

\subsection{Four Types of Generalization}
We share more training and testing curves in Fig.~\ref{fig:multiple_curves_IO_2x3} to provide additional results to Fig.~\ref{fig:multiple_curves_IO}, and in Fig.~\ref{fig:multiple_curves_IOS_9x5} to provide additional results to Fig.~\ref{fig:multiple_curves_IOS}.
\label{subapp:4generalization}
\begin{figure*}[th!]
    \centering
    %\vspace{-1.1cm}
    %\includegraphics[width = 0.8\textwidth]{fig/BASIC/table_generalization_multiple_curves_1x4_combined.pdf}
    \includegraphics[width = 0.8\textwidth]{new_fig/BASIC/multiple_curves_for_IO_2x3.pdf}
    %\vspace{-0.2cm}
    \caption{\textbf{Multiple runs for ID and OOD hypothesis class generalizations.}}
    \label{fig:multiple_curves_IO_2x3}
    %\vspace{-0.3cm}
\end{figure*}

\begin{figure*}[th!]
    \centering
    %\vspace{-1.1cm}
    %\includegraphics[width = 0.8\textwidth]{fig/BASIC/table_generalization_multiple_curves_1x4_combined.pdf}
    \includegraphics[width = 0.95\textwidth]{new_fig/BASIC/multiple_curves_for_IOS_9x5.pdf}
    %\vspace{-0.2cm}
    \caption{\textbf{Multiple runs for ID and OOD hypothesis class size generalizations.}}
    \label{fig:multiple_curves_IOS_9x5}
    %\vspace{-0.3cm}
\end{figure*}

\subsection{Compare with Other Model Architectures}
\label{subapp:4model}
We share more training and testing curves in Figs.~\ref{fig:multiple_models_IO_2x3} and~\ref{fig:multiple_models_IOS_9x5} to provide additional results to Figs.~\ref{fig:multiple_models_IO} and~\ref{fig:multiple_models_IOS}, respectively.
\begin{figure*}[th!]
    \centering
    %\vspace{-1.1cm}
    \includegraphics[width = 0.7\textwidth]{new_fig/MODEL/multiple_models_for_IO_2x3.pdf}
    %\vspace{-0.2cm}
    \caption{\textbf{Various models on ID and OOD hypothesis class generalizations.}}
    \label{fig:multiple_models_IO_2x3}
    %\vspace{-0.3cm}
\end{figure*}


\begin{figure*}[th!]
    \centering
    %\vspace{-1.1cm}
    \includegraphics[width = 0.95\textwidth]{new_fig/MODEL/multiple_models_for_IOS_9x5.pdf}
    %\vspace{-0.2cm}
    \caption{\textbf{Various models on ID and OOD hypothesis class generalizations.}}
    \label{fig:multiple_models_IOS_9x5}
    %\vspace{-0.3cm}
\end{figure*}



\subsection{Effect of Training Class Count}
We share more training and testing curves in Fig.~\ref{fig:num_train_IO_1x5} to provide additional results to Fig.~\ref{fig:num_train_IO}.
\label{subapp:numtrain}
\begin{figure*}[th!]
    \centering
    %\vspace{-1.1cm}
    %\includegraphics[width = 0.9\textwidth]{fig/NUMTRAIN/table_generalization_multiple_curves_1x5_combined.pdf}
    \includegraphics[width = 0.9\textwidth]{new_fig/NUMTRAIN/num_train_IO_1x5.pdf}
    %\vspace{-0.2cm}
    \caption{\textbf{Effect of training hypothesis class count on ID and OOD hypothesis class generalization.}}
    \label{fig:num_train_IO_1x5}
    %\vspace{-0.3cm}
\end{figure*}

\section{Experiment Setup}
\label{app:expsetup}
Each experiment is \textbf{repeated four times}, with the mean calculated across runs.
The shadow region's boundary is defined by \textbf{the minimum and maximum values} observed across the four runs.
\subsection{Learning Rate Scheduler}
\label{app:lrscheduler}
We set the train procedure with 768 total epochs, each epoch containing 1024 batches.
The learning rate (lr) is first warmed up linearly from an LR$/64$ at epoch $e=1$ to a peak value LR at epoch $e=64$, following:
$$
\text{lr}(e) = \text{LR}\cdot \frac{e}{64}, \quad 1 \leq e \leq 64.
$$
After epoch 64, the learning rate undergoes a quadratic decay over the remaining 704 epochs, given by
\[
\text{lr}(e) = \text{LR}\cdot \sqrt{\frac{64}{e}}, \quad 64 \le e \le 768.
\]

\subsection{Hyperparameter Search}
\label{app:hyperparameters}
We list the hyperparameter searching spaces used for Transformer, LSTM, GRU, and Mamba.
The best hyperparameter is searched using ID hypothesis class generalization with $\|\mathcal{X}\|=5$, $\|\mathcal{H}\|=8$, and then used for all other settings.
\section{Hyperparameter Search}\label{app:hype}
\normalsize
We exclusively conduct hyperparameter search on fold 0. 
For \textbf{GraFITi}~\citep{Yalavarthi2024.GraFITi} the hyperparameters for the search are as follows:
\begin{itemize}
    \item The number of layers, with possible values [1, 2, 3, 4].
    \item The number of attention heads, with possible values [1, 2, 4].
    \item The latent dimension, with possible values [16, 32, 64, 128, 256].
\end{itemize}

For the \textbf{LinODEnet} model~\citep{Scholz2022.Latenta} we search the hyperparameters from:
\begin{itemize}
    \item The hidden dimension, with possible values [16, 32, 64, 128].
    \item The latent dimension, with possible values [64, 128, 192, 256].
\end{itemize}

For \textbf{GRU-ODE-Bayes}~\citep{DeBrouwer2019.GRUODEBayesd} we tune the hidden size from [16, 32, 64, 128, 256]

For \textbf{Neural Flows}~\citep{Bilos2021.Neurald} we define the hyperparameter spaces for the search are as follows:
\begin{itemize}
    \item The number of flow layers, with possible values [1, 2, 4].
    \item The hidden dimension, with possible values [16, 32, 64, 128, 256].
    \item The flow model type, with possible values [GRU, ResNet].
\end{itemize}

For the \textbf{CRU}~\citep{Schirmer2022.Modelingb} the hyperparameter space is as follows:
\begin{itemize}
    \item The latent state dimension, with possible values [10, 20, 30].
    \item The number of basis functions, with possible values [10, 20].
    \item The bandwidth with possible values [3, 10].
\end{itemize}


\subsection{Setup for Generating Training and Testing Hypothesis Classes}
\label{setup:generalization}
We list the experimental setup for each experiments in the following Table~\ref{table:setup}.
When conducting experiments to evaluate accuracy on $y$, we modified the experimental setup following Table~\ref{table:setupicl}.
\begin{table}[th!]
\centering
\caption{\textbf{Experimental setups of different generalizations.}
The expression \(\min\{512, \#\text{possible}\}\) indicates that when the number of possible hypothesis classes is fewer than 512, we evaluate all possible hypothesis classes for testing; otherwise, we limit the selection to at most 512 hypothesis classes.
For example, if \( |\mathcal{H}^{\text{OOD}}| = 16 \) and \( |\mathcal{H}| = 2 \), the total number of possible hypothesis classes is given by:
$\binom{|\mathcal{H}^{\text{OOD}}|}{|\mathcal{H}|} = \binom{16}{2} = \frac{16 \times 15}{2} = 120.$
Since \( 120 < 512 \), we evaluate all 120 hypothesis classes for testing in this scenario.}
\label{table:setup}
\resizebox{1.00\linewidth}{!}{
\begin{tabular}{lrrrr}
\toprule
\multicolumn{1}{c}{Generalization Setup} & \multicolumn{1}{c}{\begin{tabular}[c]{@{}c@{}}ID Hypothesis\\ Class Generalization\end{tabular}} & \multicolumn{1}{c}{\begin{tabular}[c]{@{}c@{}}OOD Hypothesis\\ Class Generalization\end{tabular}} & \multicolumn{1}{c}{\begin{tabular}[c]{@{}c@{}}ID Hypothesis\\ Class Size Generalization\end{tabular}} & \multicolumn{1}{c}{\begin{tabular}[c]{@{}c@{}}OOD Hypothesis\\ Class Size Generalization\end{tabular}} \\ \midrule
size of input space ($|\mathcal{X}|$)    & 5                                                                                                & 5                                                                                                 & 5                                                                                                     & 5                                                                                                      \\
size of label space ($|\mathcal{Y}|$)    & 2                                                                                                & 2                                                                                                 & 2                                                                                                     & 2                                                                                                      \\
size of context query ($K$)              & 5                                                                                                & 5                                                                                                 & 5                                                                                                     & 5                                                                                                      \\
size of training hypothesis class ($|\mathcal{H}^{\text{train}}|$) & 8                                                                                                & 8                                                                                                 & 7,8,9                                                                                                   & 7,8,9                                                                                                    \\
size of testing hypothesis class ($|\mathcal{H}^{\text{test}}|$)   & 8                                                                                                & 8                                                                                                 & 2$,\ldots,$14                                                                                                  & 2$,\ldots,$14                                                                                                   \\
size of hypothesis prefix ($L$)          & 8                                                                                                & 8                                                                                                 & 16                                                                                                    & 16                                                                                                     \\
\#all hypotheses ($|\mathcal{H}^{\text{uni}}|$)                    & 32                                                                                               & 32                                                                                                & 32                                                                                                    & 32                                                                                                     \\
\#hypotheses in ID pool ($|\mathcal{H}^{\text{ID}}|$)                & 16                                                                                               & 16                                                                                                & 16                                                                                                    & 16                                                                                                     \\
\#hypotheses in OOD pool ($|\mathcal{H}^{\text{OOD}}|$)              & 16                                                                                               & 16                                                                                                & 16                                                                                                    & 16                                                                                                     \\
\#training hypothesis classes            & 12358                                                                                            & 12358                                                                                             & 4096                                                                             & 4096                                                                               \\
\#testing hypothesis classes             & 512                                                                                              & 512                                                                                               & $\min\{512,\#\text{possible}\}$                                                                                & $\min\{512,\#\text{possible}\}$                                                                                 \\ \bottomrule
\end{tabular}
}
\end{table}

\begin{table}[th!]
\centering
\caption{\textbf{Additional setups.} Numbers that differ from those in Table~\ref{table:setup} are highlighted in bold for clarity.}
\label{table:setupicl}
\begin{tabular}{lrr}
\toprule
Section & Sec.~\ref{subsec:icl} & Sec.~\ref{subsec:diversity} \\ \midrule
size of input space ($|\mathcal{X}|$)    & \textbf{4}                                                                                       & \textbf{6}                                                                                        \\
size of label space ($|\mathcal{Y}|$)    & 2                                                                                                & 2                                                                                                 \\
size of context query ($K$)              & \textbf{12}                                                                                      & \textbf{12}                                                                                       \\
size of training hypothesis class ($|\mathcal{H}^{\text{train}}|$) & \textbf{4}                                                                                       & 8                                                                                                   \\
size of testing hypothesis class ($|\mathcal{H}^{\text{test}}|$)   & \textbf{4}                                                                                       & 8                                                                                                   \\
size of hypothesis prefix ($L$)          & \textbf{4}                                                                                       & 8                                                                                                 \\
\#all hypotheses ($|\mathcal{H}^{\text{uni}}|$)                    & \textbf{16}                                                                                & \textbf{64}                                                                                         \\
\#hypotheses in ID pool ($|\mathcal{H}^{\text{ID}}|$)              & \textbf{16}                                                                                & \textbf{8,16,24,32,48}                                                                                                 \\
\#hypotheses in OOD pool ($|\mathcal{H}^{\text{OOD}}|$)            & \textbf{0}                                                                                & 16                                                                                                  \\
\#training hypothesis classes            & \textbf{1308}                                                                                    & $\min\{12358,\#\text{possible}\}$                                                                                            \\
\#testing hypothesis classes             & 512                                                                                              & 512                                                                                               \\ \bottomrule
\end{tabular}
\end{table}
%%%%%%%%%%%%%%%%%%%%%%%%%%%%%%%%%%%%%%%%%%%%%%%%%%%%%%%%%%%%%%%%%%%%%%%%%%%%%%%
%%%%%%%%%%%%%%%%%%%%%%%%%%%%%%%%%%%%%%%%%%%%%%%%%%%%%%%%%%%%%%%%%%%%%%%%%%%%%%%


\end{document}


% This document was modified from the file originally made available by
% Pat Langley and Andrea Danyluk for ICML-2K. This version was created
% by Iain Murray in 2018, and modified by Alexandre Bouchard in
% 2019 and 2021 and by Csaba Szepesvari, Gang Niu and Sivan Sabato in 2022.
% Modified again in 2023 and 2024 by Sivan Sabato and Jonathan Scarlett.
% Previous contributors include Dan Roy, Lise Getoor and Tobias
% Scheffer, which was slightly modified from the 2010 version by
% Thorsten Joachims & Johannes Fuernkranz, slightly modified from the
% 2009 version by Kiri Wagstaff and Sam Roweis's 2008 version, which is
% slightly modified from Prasad Tadepalli's 2007 version which is a
% lightly changed version of the previous year's version by Andrew
% Moore, which was in turn edited from those of Kristian Kersting and
% Codrina Lauth. Alex Smola contributed to the algorithmic style files.
