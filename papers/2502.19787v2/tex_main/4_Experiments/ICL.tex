\subsection{The Benefit of Hypothesis Prefix}
\label{subsec:icl}
In this section, we demonstrate how the hypothesis prefix influences the accuracy of ICL.
We compare ICL accuracy on $y$ with hypothesis prefix and without hypothesis prefix, under the setting of ID hypothesis class generalization.
\begin{highlight}
    \paragraph{Finding 5:} 
    \emph{Incorporating hypothesis prefix as instruction significantly boost the accuracy of ICL.}
\end{highlight}
As shown in Fig.~\ref{fig:ICL}, the hypothesis prefix significantly enhances the training and testing accuracy on $y$ of ICL.
Using position 3 as an example, predictions with three $(x,y)$ pairs as demonstrations achieve approximately 0.95 accuracy with instruction but only around 0.8 without, highlighting the effectiveness of instruction.
%Notably, in the hypothesis generalization setting, the underlying function of testing ICL samples corresponds to an unseen hypothesis. This suggests that instructions can facilitate ICL even for novel hypotheses.
\begin{figure}[h!]
    \centering
    %\vspace{-1.1cm}
    % \includegraphics[width = 0.475\textwidth]{fig/ICL/table_generalization_multiple_curves_1x2_combined.pdf}
    \includegraphics[width = 0.475\textwidth]{fig/ICL/ICL_revision.pdf}
    %\vspace{-0.2cm}
    \caption{\textbf{The effect of instruction.}
    Under ID hypothesis class generalization, providing an instruction (hypothesis prefix) significantly boosts ICL performance, especially when the $y$ token appears early (indicating only a few demonstration examples precede it).
    }
    \label{fig:ICL}
    %\vspace{-0.3cm}
\end{figure}