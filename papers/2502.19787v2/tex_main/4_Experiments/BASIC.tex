\begin{figure}[h!]
\centering
    \subfigure[testing curves of ID hypothesis class generalization.\label{fig:multiple_curves_I_1x1}]
    {
        %\includegraphics[width=0.22\textwidth]{fig/BASIC/multiple_curves_for_table_generalization_test_-optimal_acc_z.pdf}
        \includegraphics[width=0.22\textwidth]{new_fig/BASIC/multiple_curves_for_I_1x1.pdf}
    }
    \subfigure[testing curves of OOD hypothesis class generalization.\label{fig:multiple_curves_O_1x1}]{
        %\includegraphics[width=0.22\textwidth]{fig/BASIC/multiple_curves_for_hypothesis_generalization_test_-optimal_acc_z.pdf}
        \includegraphics[width=0.22\textwidth]{new_fig/BASIC/multiple_curves_for_O_1x1.pdf}
        }
    \caption{\textbf{Multiple runs on ID and OOD hypothesis class generalizations.} (Different runs imply training and testing with different random seeds.) Transformer successfully learns ICL-HCG, and generalizes to new hypothesis classes and hypotheses.
    Generalization on ID hypotheses is easier than on OOD hypotheses.
    %The generalization on new hypotheses is harder than spaces.
    Refer to Appendix~\ref{subapp:4generalization}, Fig.~\ref{fig:multiple_curves_IO_2x3} for more curves of loss, training and testing accuracy.%\zq{add ID curves}
    }
    \label{fig:multiple_curves_IO}
\end{figure}
\begin{figure}[h!]
\centering
    \subfigure[testing curves of ID hypothesis class size generalization.\label{fig:multiple_curves_IS_1x1}]
    {
        %\includegraphics[width=0.475\textwidth]{fig/BASIC/table_length_generalization_multiple_curves_1x4_combined.pdf}
        \includegraphics[width=0.475\textwidth]{new_fig/BASIC/multiple_curves_for_IS_4x3.pdf}
    }
    %\hspace{0.1em}
    \subfigure[testing curves of OOD hypothesis class size generalization.\label{fig:multiple_curves_OS_1x1}]{
        %\includegraphics[width=0.475\textwidth]{fig/BASIC/hypothesis_length_generalization_multiple_curves_1x4_combined.pdf}
        \includegraphics[width=0.475\textwidth]{new_fig/BASIC/multiple_curves_for_OS_4x3.pdf}
        }
    \caption{\textbf{Multiple runs on ID and OOD hypothesis class size generalizations.}
    (Different runs imply training and testing with different random seeds.)
    Transformers trained on hypothesis classes with sizes \(\lvert \mathcal{H} \rvert \in \{7,8,9\}\) successfully generalizes to hypothesis classes with sizes \(\lvert \mathcal{H} \rvert \in \{2,3,\ldots,13,14\}\) under ID hypothesis class size generalization.
    In contrast, the same trained Transformer exhibits poorer performance on OOD hypothesis class size generalization.
    In the figure, \emph{IS} stands for ``in-size,'' indicating the hypothesis class sizes included in the training, while \emph{OOS} stands for ``out-of-size,'' indicating the sizes that are \textbf{not} included in the training.
    Refer to Appendix~\ref{subapp:4generalization}, Fig.~\ref{fig:multiple_curves_IOS_9x5} for training accuracy curves.
    %\zq{extend the length}
    }
    \label{fig:multiple_curves_IOS}
\end{figure}
\subsection{Four Types of Generalization}
\label{subsec:4generalization}
%\zq{add a question to each experiments}
This section investigates whether a Transformer trained on ICL-HCG tasks can generalize to new tasks, \ie, new hypothesis classes.
We explore four types of generalization scenarios, defined in Definitions~\ref{def:SpaceGeneralization},~\ref{def:HypothesisGeneralization},~\ref{def:Space&SizeGeneralization}, and~\ref{def:Hypothesis&SizeGeneralization}. Detailed hyperparameters of settings are provided in Appendix~\ref{setup:generalization}.
\begin{highlight}
    \paragraph{Finding 1:} 
    \emph{Transformer can successfully learn ICL-HCG tasks and such a learned ability can generalize to new hypothesis, hypothesis class, and hypothesis size, whereas the generalization on OOD hypotheses is harder than ID hypotheses.}
\end{highlight}
We first demonstrate that the Transformer successfully learns ICL-HCG and that this capability generalizes effectively on ID and OOD hypothesis class generalizations.
As illustrated in Figs.~\ref{fig:multiple_curves_I_1x1} and~\ref{fig:multiple_curves_O_1x1}, the Trained Transformers on 4 runs with different random seeds all achieve near-perfect accuracy (close to 1.00) on ID hypothesis class generalization, and around 0.8 to 0.9 accuracy on OOD hypothesis class generalization.
Furthermore, we show that the learned ICL-HCG ability generalizes to hypothesis classes of various sizes.
As depicted in Figs.~\ref{fig:multiple_curves_IS_1x1} and~\ref{fig:multiple_curves_OS_1x1}, the trained Transformers achieve near 1.00 accuracy for $|\mathcal{H}|\in\{2,\ldots,12\}$ on ID hypothesis class size generalization, while exhibiting moderately lower accuracy on OOD hypothesis class size generalization.
%The accuracy curves exhibit multiple plateaus, with the duration of each plateau varying across different runs.
Both Figs.~\ref{fig:multiple_curves_IO} and~\ref{fig:multiple_curves_IOS} indicate that generalization on OOD hypotheses is more challenging compared to ID hypotheses.
%, exhibiting worse accuracy and larger variance.
% \begin{figure*}[th!]
%     \centering
%     %\vspace{-1.1cm}
%     \includegraphics[width = 0.8\textwidth]{fig/BASIC/table_length_generalization_multiple_curves_1x4_combined.pdf}
%     %\vspace{-0.2cm}
%     \caption{\textbf{Multiple runs for space\&size generalization.}}
%     \label{fig:Space&SizeGeneralization-MultipleRuns}
%     %\vspace{-0.3cm}
% \end{figure*}
% \begin{figure*}[th!]
%     \centering
%     %\vspace{-1.1cm}
%     \includegraphics[width = 0.8\textwidth]{fig/BASIC/hypothesis_length_generalization_multiple_curves_1x4_combined.pdf}
%     %\vspace{-0.2cm}
%     \caption{\textbf{Multiple runs for hypothesis\&size generalization.}}
%     \label{fig:Hypothesis&SizeGeneralization-MultipleRuns}
%     %\vspace{-0.3cm}
% \end{figure*}