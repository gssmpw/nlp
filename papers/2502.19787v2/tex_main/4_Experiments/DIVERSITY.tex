\subsection{The Effect of Pretraining Hypothesis Diversity}
\label{subsec:diversity}
% In this section, we examine the effect of hypothesis diversity together with instruction on the ICL accuracy.
% As shown in Fig.~\ref{fig:diversity}, under the setting of hypothesis generalization, We TBC
% Noticing under the setting of hypothesis generalization, the underlying function of testing ICL samples is an unseen hypothesis, this means instruction can help ICL even on new hypothesis.
% In this section, we examine the impact of hypothesis diversity in conjunction with instructional information on the accuracy of ICL.
% We conduct experiments with $|\mathcal{X}|=6$ and hypothesis universe $\mathcal{H^{\text{uni}}}$ with size $2^{|\mathcal{X}|}=64$.
% We sample $M^{\text{train}}\in\{4,8,16,32\}$ hypotheses from the hypothesis universe $H^{\text{uni}}$ for training, and then sample $M^{\text{test}}=16$ hypotheses from the remaining $M-M^{\text{train}}$ hypotheses for testing.
% As shown in Fig.~\ref{fig:diversity}, under thehypothesis generalization setting, the Transformer trained with instruction achieves higher ICL accuracy compared to models trained without instruction.
% Notably, in the hypothesis generalization setting, the underlying function of testing ICL samples corresponds to unseen hypotheses.
% This suggests that incorporating instructions can enhance ICL performance even when dealing with novel hypotheses.
In this section, we investigate the impact of hypothesis diversity combined with instruction (hypothesis prefix) on ICL accuracy.  
We conduct experiments under OOD hypothesis class generalization with an input space size of $|\mathcal{X}| = 6$, leading to a hypothesis universe $\mathcal{H}^{\text{uni}}$ of $2^{|\mathcal{X}|} = 64$ hypotheses.  
$\mathcal{H}^{\text{uni}}$ is split into $\mathcal{H}^{\text{ID}}$ with 48 hypotheses and $\mathcal{H}^{\text{OOD}}$ with 16.  
For training, we sample $M^{\text{train}} \in \{8, 16, 24, 32\}$ hypotheses from $\mathcal{H}^{\text{ID}}$ to examine the effect of training hypothesis diversity, while testing uses all hypotheses in $\mathcal{H}^{\text{ID}}$.
\begin{highlight}
    \paragraph{Finding 6:} 
    \emph{Increasing the diversity of pretraining hypotheses significantly boosts the performance of ICL when instructions are provided.}
\end{highlight}
As illustrated in Fig.~\ref{fig:diversity}, under OOD hypothesis class generalization, the Transformer trained with instructions achieves similar ICL accuracy to a standard ICL approach when pretraining hypothesis diversity is low, but significantly outperforms it when pretraining hypothesis diversity is high.
Using position 10 as an example, with instruction, increasing $M^{\text{train}}$ from 8 to 32 raises accuracy from 0.80 to approximately 0.99.  
Without instruction, the same increase in diversity improves accuracy only from 0.80 to 0.90.
Notably, the testing ICL samples are derived from unseen hypotheses, indicating that incorporating instructions can enhance ICL performance for new hypotheses.

\begin{figure}[h!]
    \centering
    %\vspace{-1.1cm}
    %\includegraphics[width = 0.475\textwidth]{fig/DIVERISTY/hypothesis_generalization_multiple_curves_2x2_combined.pdf}
    \includegraphics[width = 0.475\textwidth]{new_fig/DIVERSITY/diversity_2x2_combined.pdf}
    %\vspace{-0.2cm}
    \caption{
    \textbf{The effect of pretraining hypothesis diversity.} Under hypothesis generalization, increasing the diversity of pretraining hypotheses significantly boosts the performance of ICL when instructions are provided. However, without instructions, this effect is limited.
    }
    \label{fig:diversity}
    %\vspace{-0.3cm}
\end{figure}