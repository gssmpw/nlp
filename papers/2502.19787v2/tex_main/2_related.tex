\section{Related Works}
\label{sec:RelatedWork}
% We classify synthetic data models used in ICL research into two broad categories. In the first category, called the \emph{Interleaved Input-Output Format} in this paper, the pretraining data is an interleaved sequence:
% \[
%     \bigl(\vx^{(1)}, \vy^{(1)}, \vx^{(2)}, \vy^{(2)}, \ldots \bigr),
% \]
% where each label \(\vy^{(i)}\) is generated by applying a function \(f\) to the corresponding input \(\vx^{(i)}\), i.e., \(\vy^{(i)} = f(\vx^{(i)})\).

% In the second category, called the \emph{Recurrent Input-Output Format} in this paper, the pretraining sequence follows:
% \[
%     \bigl(\vx^{(1)}, \vx^{(2)}, \vx^{(3)}, \ldots\bigr),
% \]
% and each input \(\vx^{(i+1)}\) is generated from the previous one following the same function
% \(
%     \vx^{(i+1)} = f\bigl(\vx^{(i)}\bigr).
% \)

%\paragraph{Interleaved Input-Output Format}
\citet{garg2022can} first construct synthetic data with pretraining and testing sequences generated from noiseless linear regression tasks.
Specifically, \citet{garg2022can} sample all input vectors \(\vx\) from an isotropic Gaussian distribution \(\mathcal{N}(\vzero, \mathbf{I})\). Within each sequence, the outputs are given by
\(
y^{(i)} = \langle \vw, \vx^{(i)} \rangle,
\)
where \(\vw\) is drawn from \(\mathcal{N}(\vzero, \mathbf{I}_d)\). 
For ICL inference, the prompt takes the form
\(
(\vx^{(1)}, y^{(1)}, \vx^{(2)}, y^{(2)}, \ldots, \vx^{(k-1)}, y^{(k-1)}, \vx_{\text{query}})
\)
where the \(k-1\) pairs \(\{(\vx^{(i)}, y^{(i)})\}_{i=1}^{k-1}\) serve as demonstrations illustrating the linear relationship governed by \(\vw\).
The model then predicts the output for \(\vx_{\text{query}}\).
%We refer to this sequence format as the "Interleaved Input-Output Format."
%\citet{garg2022can} also extends such an experiment to non-linear cases including decision trees and two-layer neural networks.

\paragraph{Noiseless Linear Regression} Based on the well-defined problem setup by~\citet{garg2022can} using noiseless linear regression, researchers systematically study the mechanisms of ICL and properties of Transformer.
For instance, there is a particular interesting line of research on connecting ICL to gradient descent, firstly hinted by~\citet{garg2022can}.
\citet{akyurek2022learning} and \citet{von2022transformers} then show that one attention layer can be exactly constructed to perform gradient descent, and empirically find similarities between ICL and gradient descent. 
Further, \citet{ahn2023transformers} theoretically show that under certain conditions, Transformers trained on noiseless linear regression tasks minimizing the pretraining loss will implement gradient descent algorithm.
Nevertheless, \citet{fu2024transformers} show that Transformers learn to approximate second-order optimization methods for ICL, sharing a similar convergence rate as Iterative Newton’s Method.
Besides gradient descent, there are lots of other interesting topics on ICL and Transformers based on this linear regression setting, such as looped Transformer~\citep{Yang0NP24,GatmirySRJK24}, training dynamic~\citep{zhang2024trained, HuangCL24, kim2024transformers}, generalization~\citep{panwar2024incontext}, etc.

\paragraph{Noisy Linear Regression} Such a simple noiseless linear regression task is further extended to variants.
By extending the linear regression to noisy linear regression---$y=\langle \vx, \vw \rangle + \epsilon$,
\citet{li2023transformers} analyze the generalization and stability of ICL. 
\citet{WuZCBGB24} and \citet{raventos2024pretraining} analyze the effect of task diversity on the attention model's ICL risk.
Via extending the regression tasks sampling from Gaussian to Gaussian mixture, \citet{lin2024dual} show ICL exhibits two different modes including task retrieval and learning.
With the tasks of $\vy=\mW\vx+\vepsilon$ where $\mW$ is a matrix rather than a vector, \citet{ChenSWY24} examine the training dynamic of multi-head attention for ICL.

\paragraph{More than Linear Regression} Beyond linear regression, researchers are also interested in non-linear regression and classification.
The research directions are scattered, and we list them as follows.
\citet{BaiCWXM23} show that Transformers can perform in-context algorithm selection, \ie, adaptively selecting different ICL algorithms such as gradient descent, least square, or ridge regression.
\citet{BhattamishraPBK24} show Transformer can learn a variety of Boolean function classes.
\citet{cheng2024transformers} provide evidence that Transformers can learn to implement gradient descent to enable them to learn non-linear functions.
\citet{0004HMWXS024} show that trained Transformer achieves near-optimal ICL performance under $y=\langle \vw, f(\vx)\rangle$, where $f$ is a shallow neural network (MLP).
Examining linear and non-linear regression tasks, \citet{fan2024transformers} and~\citet{tripuraneni2024can} show Transformer can perform ICL on composited or mixed tasks of pretrained linear or non-linear tasks, and \citet{yadlowsky2024can} examine whether trained Transformers can generalize to new tasks beyond pretraining.
\citet{park2024can} examine whether Mamba can in-context learn a variety of synthetic tasks.
Via examining regression and classification tasks, \citet{kim2024task} show task diversity helps shorten the ICL plateau pretraining.
\citet{rameshcompositional} assume there are multiple functions composited to connect $\vx$ and $\vy$ pair, \eg, $\vy=f_1\circ f_2\circ f_3(\vx)$ to study the compositional capabilities of Trasnformer.
\citet{li2024nonlinear} study how non-linear Transformer learns binary classification.


% \paragraph{Recurrent Input-Output Format}
% % This type of pretraining sequence has the format $(\vx^{(1)},\vx^{(2)},\vx^{(3)},\ldots)$, a sequence of tokens generated from a next-token generative model, such as the Hidden Markov Model (HMM)~\citep{xie2021explanation}, Probabilistic Finite Automata (PFA)~\citep{akyurek2024incontext}, or simply a function $\vx_{i+1} = f(\vx^{(i)})+\epsilon$~\citep{li2023transformers,sander2024how}.
% %with noise or $\vx_{i+1} = f(\vx^{(i)})$~\citep{sander2024how} without noise.
% \citet{xie2021explanation} first proposes to use multiple Hidden Markov Models (HMMs) to represent latent concepts in real-world language and generate pretraining sequences, then explain the ICL phenomenon via a Bayesian perspective.
% %The training sequences are then generated from those HMMs for pretraining, while testing sequences are constructed by concatenating multiple sequences of the same length.
% %Such misalignment between the training and testing dataset mimics the misalignment under real-world LLMs' training and ICL testing scenarios, which is the key feature of the first category.
% \citet{han2023explaining} leverages the same setting and explains ICL via kernel regression.
% Sequences under this format are also generated by \citet{akyurek2024incontext} using Probabilistic Finite Automata (PFAs), \citet{edelman2024evolution} using Markov Chain, and \citet{nichani2024transformers} using Markov Chain with causal structure, to study the induction head~\cite{olsson2022context} of LLMs.
% While the abovementioned datasets only consider single-step dependence between tokens, i.e., the next token only depends on one of the previous tokens, \citet{ashokMCICLR} further explore
% %study how Transformer learns both 1-order and 
% higher-order Markov chains.


% Beyond the setting of $(\vx^{(1)}, \vy^{(1)}, \vx^{(2)}, \vy^{(2)}, \ldots)$, multiple steps are further introduced by~\citet{li2024how} to the mapping from $\vx$ to $\vy$, \ie, the training sequence becomes $\vx^{(1)}, \vy^{(1)}^{'}, \vy^{(1)}^{''}, \vx^{(2)}, \vy^{(2)}^{'}, \vy^{(2)}^{''}, \ldots$ to mimic the multiple steps in Chain-of-Thought (CoT)~\citep{wei2022chain}.


% \paragraph{Synthetic Dataset with Image.}
% Beyond simply vector serving as $\vx$, researchers also leverages image as those $\vx$, \ie, the training sequence $(\vx^{(1)},\vy^{(1)},\vx^{(2)},\vy^{(2)},\ldots)$ has image serves as in-context samples.
% \citet{chan2022data} first examine how the pretraining data properly affects the ICL phenomenon.
% \citet{singh2024transient, reddy2023mechanistic} further examine the effect of data on the dynamics of ICL and in-weight learning (IWL).
% \citet{fubreaking} study the learning plateaus of in-context learning with similar image label paired pretraining sequences.

\paragraph{Synthetic Dataset with Instruction}
To the best of our knowledge, there are two articles on synthetic datasets with instructions.
\citet{huang2024task} append an additional vector $\vmu$ to the sequences with interleaved input-output format, which leads to the sequence $(\vmu, \vx^{(1)}, \vw^\top\vx^{(1)}, \vx^{(2)}, \vw^\top\vx^{(2)}, \ldots)$ in which $\vx^{(i)}\sim\mathcal{N}(\vmu,\mI)$, and show that the trained Transformer can achieve significantly lower loss on ICL when the task descriptor $\vmu$ is provided.
The work of \citet{xuanyuan2024on} is most closely related to us.
It develops a new synthetic dataset based on task $((a\cdot x)\circ(b\cdot y))\mod p=r$, where $(x,y)$ is the input, $r$ is the output, $\circ$ is an operation ($+,-,/$), and each task is defined by the parameters $(a,b,\circ)$ ($p$ is a constant).
The instruction is constructed as $(a_l,a_u,b_l,b_u,\circ)$, where $a_l$ and $a_u$ are the lower and upper bounds of $a$ (similar for $b$), and $\circ$ is the operation.
Therefore, the instruction constrains the possible tasks, \ie, provide information on the underlying true task of in-context samples.
With such a setting, \citet{xuanyuan2024on} study how the information provided by instruction affects the ICL accuracy.
