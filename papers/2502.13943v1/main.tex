
\documentclass{article}

\usepackage[accepted]{icml2025_arxiv}

\usepackage{microtype}
\usepackage{booktabs} % for professional tables

\usepackage{pifont}    % For \ding symbols
\usepackage{xcolor}

\usepackage{times}
\usepackage{latexsym}

\usepackage{epsfig}
\usepackage{multirow}
\usepackage{multicol}
\usepackage{array}
\usepackage{arydshln}
\usepackage{amsmath}
\usepackage{ulem}
\usepackage{amsfonts}

\usepackage[utf8]{inputenc}


\usepackage{microtype}

\usepackage{inconsolata}
\usepackage{graphicx}
\usepackage{lipsum}
\usepackage{hyperref}
\usepackage[capitalize,noabbrev]{cleveref}



\usepackage{tabularx}

\newcommand{\theHalgorithm}{\arabic{algorithm}}

\newcommand{\MethodName}{ASPRM}
\newcommand{\chuheng}[1]{{\color{cyan} #1}}
\newcommand{\yuliang}[1]{{\color{red} #1}}
\newcommand{\junjie}[1]{{\color{yellow} #1}}
\newcommand{\weishen}[1]{{\color{blue} #1}}



\usepackage{amsmath}
\usepackage{amssymb}
\usepackage{mathtools}
\usepackage{amsthm}
\usepackage{subcaption}
\theoremstyle{plain}
\newtheorem{theorem}{Theorem}[section]
\newtheorem{proposition}[theorem]{Proposition}
\newtheorem{lemma}[theorem]{Lemma}
\newtheorem{corollary}[theorem]{Corollary}
\theoremstyle{definition}
\newtheorem{definition}[theorem]{Definition}
\newtheorem{assumption}[theorem]{Assumption}
\theoremstyle{remark}
\newtheorem{remark}[theorem]{Remark}

\usepackage[textsize=tiny]{todonotes}


% The \icmltitle you define below is probably too long as a header.
% Therefore, a short form for the running title is supplied here:
\icmltitlerunning{AdaptiveStep: Automatically Dividing Reasoning Step through Model Confidence}

\begin{document}

\twocolumn[

\icmltitle{AdaptiveStep: Automatically Dividing Reasoning Step through Model Confidence}


\icmlsetsymbol{equal}{*}

\begin{icmlauthorlist}
\icmlauthor{Yuliang Liu}{equal,nju}
\icmlauthor{Junjie Lu}{equal,uts}
\icmlauthor{Zhaoling Chen}{nju}
\icmlauthor{Chaofeng Qu}{}
\icmlauthor{Jason Klein Liu}{}
\icmlauthor{Chonghan Liu}{}
\icmlauthor{Zefan Cai}{uwm}
%\icmlauthor{}{sch}
\icmlauthor{Yunhui Xia}{}
\icmlauthor{Li Zhao}{msra}
\icmlauthor{Jiang Bian}{msra}
\icmlauthor{Chuheng Zhang}{msra}
\icmlauthor{Wei Shen}{}
\icmlauthor{Zhouhan Lin}{sjtu}
\end{icmlauthorlist}

\icmlaffiliation{nju}{Nanjing University}
\icmlaffiliation{uts}{University of Technology Sydney}
% \icmlaffiliation{sii}{Shanghai Innovation Institute}
\icmlaffiliation{uwm}{UW-Madison}
\icmlaffiliation{sjtu}{Shanghai Jiaotong University}
\icmlaffiliation{msra}{MSRA}


% \icmlaffiliation{sch}{School of ZZZ, Institute of WWW, Location, Country}

\icmlcorrespondingauthor{Yuliang Liu}{liuyl03181@gmail.com}
\icmlcorrespondingauthor{Junjie Lu}{lux17999@gmail.com}
\icmlcorrespondingauthor{Chuheng Zhang}{zhangchuheng123@live.com}
\icmlcorrespondingauthor{Wei Shen}{shenwei0917@126.com}
\icmlcorrespondingauthor{Zhouhan Lin}{hantek@sjtu.edu.cn}

% You may provide any keywords that you
% find helpful for describing your paper; these are used to populate
% the "keywords" metadata in the PDF but will not be shown in the document
\icmlkeywords{Process reward model, LLM reasoning}

\vskip 0.3in
]

\printAffiliationsAndNotice{\icmlEqualContribution} % otherwise use the standard text.


\begin{abstract}
Current approaches for training Process Reward Models (PRMs) often involve breaking down responses into multiple reasoning steps using rule-based techniques, such as using predefined placeholder tokens or setting the reasoning step's length into a fixed size.
These approaches overlook the fact that specific words do not typically mark true decision points in a text. To address this, we propose AdaptiveStep, a method that divides reasoning steps based on the model's confidence in predicting the next word. This division method provides more decision-making information at each step, enhancing downstream tasks, such as reward model learning. Moreover, our method does not require manual annotation. We demonstrate its effectiveness through experiments with AdaptiveStep-trained PRMs in mathematical reasoning and code generation tasks. Experimental results indicate that the outcome PRM achieves state-of-the-art Best-of-N performance, surpassing greedy search strategy with token-level value-guided decoding, while also reducing construction costs by over 30\% compared to existing open-source PRMs. In addition, we provide a thorough analysis and case study on the PRM's performance, transferability, and generalization capabilities. We provide our code on \href{https://github.com/Lux0926/ASPRM/tree/main}{GitHub}.


\end{abstract}

\iffalse


\fi

%!TEX root = gcn.tex
\section{Introduction}
Graphs, representing structural data and topology, are widely used across various domains, such as social networks and merchandising transactions.
Graph convolutional networks (GCN)~\cite{iclr/KipfW17} have significantly enhanced model training on these interconnected nodes.
However, these graphs often contain sensitive information that should not be leaked to untrusted parties.
For example, companies may analyze sensitive demographic and behavioral data about users for applications ranging from targeted advertising to personalized medicine.
Given the data-centric nature and analytical power of GCN training, addressing these privacy concerns is imperative.

Secure multi-party computation (MPC)~\cite{crypto/ChaumDG87,crypto/ChenC06,eurocrypt/CiampiRSW22} is a critical tool for privacy-preserving machine learning, enabling mutually distrustful parties to collaboratively train models with privacy protection over inputs and (intermediate) computations.
While research advances (\eg,~\cite{ccs/RatheeRKCGRS20,uss/NgC21,sp21/TanKTW,uss/WatsonWP22,icml/Keller022,ccs/ABY318,folkerts2023redsec}) support secure training on convolutional neural networks (CNNs) efficiently, private GCN training with MPC over graphs remains challenging.

Graph convolutional layers in GCNs involve multiplications with a (normalized) adjacency matrix containing $\numedge$ non-zero values in a $\numnode \times \numnode$ matrix for a graph with $\numnode$ nodes and $\numedge$ edges.
The graphs are typically sparse but large.
One could use the standard Beaver-triple-based protocol to securely perform these sparse matrix multiplications by treating graph convolution as ordinary dense matrix multiplication.
However, this approach incurs $O(\numnode^2)$ communication and memory costs due to computations on irrelevant nodes.
%
Integrating existing cryptographic advances, the initial effort of SecGNN~\cite{tsc/WangZJ23,nips/RanXLWQW23} requires heavy communication or computational overhead.
Recently, CoGNN~\cite{ccs/ZouLSLXX24} optimizes the overhead in terms of  horizontal data partitioning, proposing a semi-honest secure framework.
Research for secure GCN over vertical data  remains nascent.

Current MPC studies, for GCN or not, have primarily targeted settings where participants own different data samples, \ie, horizontally partitioned data~\cite{ccs/ZouLSLXX24}.
MPC specialized for scenarios where parties hold different types of features~\cite{tkde/LiuKZPHYOZY24,icml/CastigliaZ0KBP23,nips/Wang0ZLWL23} is rare.
This paper studies $2$-party secure GCN training for these vertical partition cases, where one party holds private graph topology (\eg, edges) while the other owns private node features.
For instance, LinkedIn holds private social relationships between users, while banks own users' private bank statements.
Such real-world graph structures underpin the relevance of our focus.
To our knowledge, no prior work tackles secure GCN training in this context, which is crucial for cross-silo collaboration.


To realize secure GCN over vertically split data, we tailor MPC protocols for sparse graph convolution, which fundamentally involves sparse (adjacency) matrix multiplication.
Recent studies have begun exploring MPC protocols for sparse matrix multiplication (SMM).
ROOM~\cite{ccs/SchoppmannG0P19}, a seminal work on SMM, requires foreknowledge of sparsity types: whether the input matrices are row-sparse or column-sparse.
Unfortunately, GCN typically trains on graphs with arbitrary sparsity, where nodes have varying degrees and no specific sparsity constraints.
Moreover, the adjacency matrix in GCN often contains a self-loop operation represented by adding the identity matrix, which is neither row- nor column-sparse.
Araki~\etal~\cite{ccs/Araki0OPRT21} avoid this limitation in their scalable, secure graph analysis work, yet it does not cover vertical partition.

% and related primitives
To bridge this gap, we propose a secure sparse matrix multiplication protocol, \osmm, achieving \emph{accurate, efficient, and secure GCN training over vertical data} for the first time.

\subsection{New Techniques for Sparse Matrices}
The cost of evaluating a GCN layer is dominated by SMM in the form of $\adjmat\feamat$, where $\adjmat$ is a sparse adjacency matrix of a (directed) graph $\graph$ and $\feamat$ is a dense matrix of node features.
For unrelated nodes, which often constitute a substantial portion, the element-wise products $0\cdot x$ are always zero.
Our efficient MPC design 
avoids unnecessary secure computation over unrelated nodes by focusing on computing non-zero results while concealing the sparse topology.
We achieve this~by:
1) decomposing the sparse matrix $\adjmat$ into a product of matrices (\S\ref{sec::sgc}), including permutation and binary diagonal matrices, that can \emph{faithfully} represent the original graph topology;
2) devising specialized protocols (\S\ref{sec::smm_protocol}) for efficiently multiplying the structured matrices while hiding sparsity topology.


 
\subsubsection{Sparse Matrix Decomposition}
We decompose adjacency matrix $\adjmat$ of $\graph$ into two bipartite graphs: one represented by sparse matrix $\adjout$, linking the out-degree nodes to edges, the other 
by sparse matrix $\adjin$,
linking edges to in-degree nodes.

%\ie, we decompose $\adjmat$ into $\adjout \adjin$, where $\adjout$ and $\adjin$ are sparse matrices representing these connections.
%linking out-degree nodes to edges and edges to in-degree nodes of $\graph$, respectively.

We then permute the columns of $\adjout$ and the rows of $\adjin$ so that the permuted matrices $\adjout'$ and $\adjin'$ have non-zero positions with \emph{monotonically non-decreasing} row and column indices.
A permutation $\sigma$ is used to preserve the edge topology, leading to an initial decomposition of $\adjmat = \adjout'\sigma \adjin'$.
This is further refined into a sequence of \emph{linear transformations}, 
which can be efficiently computed by our MPC protocols for 
\emph{oblivious permutation}
%($\Pi_{\ssp}$) 
and \emph{oblivious selection-multiplication}.
% ($\Pi_\SM$)
\iffalse
Our approach leverages bipartite graph representation and the monotonicity of non-zero positions to decompose a general sparse matrix into linear transformations, enhancing the efficiency of our MPC protocols.
\fi
Our decomposition approach is not limited to GCNs but also general~SMM 
by 
%simply 
treating them 
as adjacency matrices.
%of a graph.
%Since any sparse matrix can be viewed 

%allowing the same technique to be applied.

 
\subsubsection{New Protocols for Linear Transformations}
\emph{Oblivious permutation} (OP) is a two-party protocol taking a private permutation $\sigma$ and a private vector $\xvec$ from the two parties, respectively, and generating a secret share $\l\sigma \xvec\r$ between them.
Our OP protocol employs correlated randomnesses generated in an input-independent offline phase to mask $\sigma$ and $\xvec$ for secure computations on intermediate results, requiring only $1$ round in the online phase (\cf, $\ge 2$ in previous works~\cite{ccs/AsharovHIKNPTT22, ccs/Araki0OPRT21}).

Another crucial two-party protocol in our work is \emph{oblivious selection-multiplication} (OSM).
It takes a private bit~$s$ from a party and secret share $\l x\r$ of an arithmetic number~$x$ owned by the two parties as input and generates secret share $\l sx\r$.
%between them.
%Like our OP protocol, o
Our $1$-round OSM protocol also uses pre-computed randomnesses to mask $s$ and $x$.
%for secure computations.
Compared to the Beaver-triple-based~\cite{crypto/Beaver91a} and oblivious-transfer (OT)-based approaches~\cite{pkc/Tzeng02}, our protocol saves ${\sim}50\%$ of online communication while having the same offline communication and round complexities.

By decomposing the sparse matrix into linear transformations and applying our specialized protocols, our \osmm protocol
%($\prosmm$) 
reduces the complexity of evaluating $\numnode \times \numnode$ sparse matrices with $\numedge$ non-zero values from $O(\numnode^2)$ to $O(\numedge)$.

%(\S\ref{sec::secgcn})
\subsection{\cgnn: Secure GCN made Efficient}
Supported by our new sparsity techniques, we build \cgnn, 
a two-party computation (2PC) framework for GCN inference and training over vertical
%ly split
data.
Our contributions include:

1) We are the first to explore sparsity over vertically split, secret-shared data in MPC, enabling decompositions of sparse matrices with arbitrary sparsity and isolating computations that can be performed in plaintext without sacrificing privacy.

2) We propose two efficient $2$PC primitives for OP and OSM, both optimally single-round.
Combined with our sparse matrix decomposition approach, our \osmm protocol ($\prosmm$) achieves constant-round communication costs of $O(\numedge)$, reducing memory requirements and avoiding out-of-memory errors for large matrices.
In practice, it saves $99\%+$ communication
%(Table~\ref{table:comm_smm}) 
and reduces ${\sim}72\%$ memory usage over large $(5000\times5000)$ matrices compared with using Beaver triples.
%(Table~\ref{table:mem_smm_sparse}) ${\sim}16\%$-

3) We build an end-to-end secure GCN framework for inference and training over vertically split data, maintaining accuracy on par with plaintext computations.
We will open-source our evaluation code for research and deployment.

To evaluate the performance of $\cgnn$, we conducted extensive experiments over three standard graph datasets (Cora~\cite{aim/SenNBGGE08}, Citeseer~\cite{dl/GilesBL98}, and Pubmed~\cite{ijcnlp/DernoncourtL17}),
reporting communication, memory usage, accuracy, and running time under varying network conditions, along with an ablation study with or without \osmm.
Below, we highlight our key achievements.

\textit{Communication (\S\ref{sec::comm_compare_gcn}).}
$\cgnn$ saves communication by $50$-$80\%$.
(\cf,~CoGNN~\cite{ccs/KotiKPG24}, OblivGNN~\cite{uss/XuL0AYY24}).

\textit{Memory usage (\S\ref{sec::smmmemory}).}
\cgnn alleviates out-of-memory problems of using %the standard 
Beaver-triples~\cite{crypto/Beaver91a} for large datasets.

\textit{Accuracy (\S\ref{sec::acc_compare_gcn}).}
$\cgnn$ achieves inference and training accuracy comparable to plaintext counterparts.
%training accuracy $\{76\%$, $65.1\%$, $75.2\%\}$ comparable to $\{75.7\%$, $65.4\%$, $74.5\%\}$ in plaintext.

{\textit{Computational efficiency (\S\ref{sec::time_net}).}} 
%If the network is worse in bandwidth and better in latency, $\cgnn$ shows more benefits.
$\cgnn$ is faster by $6$-$45\%$ in inference and $28$-$95\%$ in training across various networks and excels in narrow-bandwidth and low-latency~ones.

{\textit{Impact of \osmm (\S\ref{sec:ablation}).}}
Our \osmm protocol shows a $10$-$42\times$ speed-up for $5000\times 5000$ matrices and saves $10$-2$1\%$ memory for ``small'' datasets and up to $90\%$+ for larger ones.




\section{Related Work}

\subsection{Personalization and Role-Playing}
Recent works have introduced benchmark datasets for personalizing LLM outputs in tasks like email, abstract, and news writing, focusing on shorter outputs (e.g., 300 tokens for product reviews \citep{kumar2024longlamp} and 850 for news writing \citep{shashidhar-etal-2024-unsupervised}). These methods infer user traits from history for task-specific personalization \citep{sun-etal-2024-revealing, sun-etal-2025-persona, pal2024beyond, li2023teach, salemi2025reasoning}. In contrast, we tackle the more subjective problem of long-form story writing, with author stories averaging 1500 tokens. Unlike prior role-playing approaches that use predefined personas (e.g., Tony Stark, Confucius) \citep{wang-etal-2024-rolellm, sadeq-etal-2024-mitigating, tu2023characterchat, xu2023expertprompting}, we propose a novel method to infer story-writing personas from an author’s history to guide role-playing.


\subsection{Story Understanding and Generation}  
Prior work on persona-aware story generation \citep{yunusov-etal-2024-mirrorstories, bae-kim-2024-collective, zhang-etal-2022-persona, chandu-etal-2019-way} defines personas using discrete attributes like personality traits, demographics, or hobbies. Similarly, \citep{zhu-etal-2023-storytrans} explore story style transfer across pre-defined domains (e.g., fairy tales, martial arts, Shakespearean plays). In contrast, we mimic an individual author's writing style based on their history. Our approach differs by (1) inferring long-form author personas—descriptions of an author’s style from their past works, rather than relying on demographics, and (2) handling long-form story generation, averaging 1500 tokens per output, exceeding typical story lengths in prior research.



\vspace{-5pt}
\section{Method}
\label{sec:method}
\section{Overview}

\revision{In this section, we first explain the foundational concept of Hausdorff distance-based penetration depth algorithms, which are essential for understanding our method (Sec.~\ref{sec:preliminary}).
We then provide a brief overview of our proposed RT-based penetration depth algorithm (Sec.~\ref{subsec:algo_overview}).}



\section{Preliminaries }
\label{sec:Preliminaries}

% Before we introduce our method, we first overview the important basics of 3D dynamic human modeling with Gaussian splatting. Then, we discuss the diffusion-based 3d generation techniques, and how they can be applied to human modeling.
% \ZY{I stopp here. TBC.}
% \subsection{Dynamic human modeling with Gaussian splatting}
\subsection{3D Gaussian Splatting}
3D Gaussian splatting~\cite{kerbl3Dgaussians} is an explicit scene representation that allows high-quality real-time rendering. The given scene is represented by a set of static 3D Gaussians, which are parameterized as follows: Gaussian center $x\in {\mathbb{R}^3}$, color $c\in {\mathbb{R}^3}$, opacity $\alpha\in {\mathbb{R}}$, spatial rotation in the form of quaternion $q\in {\mathbb{R}^4}$, and scaling factor $s\in {\mathbb{R}^3}$. Given these properties, the rendering process is represented as:
\begin{equation}
  I = Splatting(x, c, s, \alpha, q, r),
  \label{eq:splattingGA}
\end{equation}
where $I$ is the rendered image, $r$ is a set of query rays crossing the scene, and $Splatting(\cdot)$ is a differentiable rendering process. We refer readers to Kerbl et al.'s paper~\cite{kerbl3Dgaussians} for the details of Gaussian splatting. 



% \ZY{I would suggest move this part to the method part.}
% GaissianAvatar is a dynamic human generation model based on Gaussian splitting. Given a sequence of RGB images, this method utilizes fitted SMPLs and sampled points on its surface to obtain a pose-dependent feature map by a pose encoder. The pose-dependent features and a geometry feature are fed in a Gaussian decoder, which is employed to establish a functional mapping from the underlying geometry of the human form to diverse attributes of 3D Gaussians on the canonical surfaces. The parameter prediction process is articulated as follows:
% \begin{equation}
%   (\Delta x,c,s)=G_{\theta}(S+P),
%   \label{eq:gaussiandecoder}
% \end{equation}
%  where $G_{\theta}$ represents the Gaussian decoder, and $(S+P)$ is the multiplication of geometry feature S and pose feature P. Instead of optimizing all attributes of Gaussian, this decoder predicts 3D positional offset $\Delta{x} \in {\mathbb{R}^3}$, color $c\in\mathbb{R}^3$, and 3D scaling factor $ s\in\mathbb{R}^3$. To enhance geometry reconstruction accuracy, the opacity $\alpha$ and 3D rotation $q$ are set to fixed values of $1$ and $(1,0,0,0)$ respectively.
 
%  To render the canonical avatar in observation space, we seamlessly combine the Linear Blend Skinning function with the Gaussian Splatting~\cite{kerbl3Dgaussians} rendering process: 
% \begin{equation}
%   I_{\theta}=Splatting(x_o,Q,d),
%   \label{eq:splatting}
% \end{equation}
% \begin{equation}
%   x_o = T_{lbs}(x_c,p,w),
%   \label{eq:LBS}
% \end{equation}
% where $I_{\theta}$ represents the final rendered image, and the canonical Gaussian position $x_c$ is the sum of the initial position $x$ and the predicted offset $\Delta x$. The LBS function $T_{lbs}$ applies the SMPL skeleton pose $p$ and blending weights $w$ to deform $x_c$ into observation space as $x_o$. $Q$ denotes the remaining attributes of the Gaussians. With the rendering process, they can now reposition these canonical 3D Gaussians into the observation space.



\subsection{Score Distillation Sampling}
Score Distillation Sampling (SDS)~\cite{poole2022dreamfusion} builds a bridge between diffusion models and 3D representations. In SDS, the noised input is denoised in one time-step, and the difference between added noise and predicted noise is considered SDS loss, expressed as:

% \begin{equation}
%   \mathcal{L}_{SDS}(I_{\Phi}) \triangleq E_{t,\epsilon}[w(t)(\epsilon_{\phi}(z_t,y,t)-\epsilon)\frac{\partial I_{\Phi}}{\partial\Phi}],
%   \label{eq:SDSObserv}
% \end{equation}
\begin{equation}
    \mathcal{L}_{\text{SDS}}(I_{\Phi}) \triangleq \mathbb{E}_{t,\epsilon} \left[ w(t) \left( \epsilon_{\phi}(z_t, y, t) - \epsilon \right) \frac{\partial I_{\Phi}}{\partial \Phi} \right],
  \label{eq:SDSObservGA}
\end{equation}
where the input $I_{\Phi}$ represents a rendered image from a 3D representation, such as 3D Gaussians, with optimizable parameters $\Phi$. $\epsilon_{\phi}$ corresponds to the predicted noise of diffusion networks, which is produced by incorporating the noise image $z_t$ as input and conditioning it with a text or image $y$ at timestep $t$. The noise image $z_t$ is derived by introducing noise $\epsilon$ into $I_{\Phi}$ at timestep $t$. The loss is weighted by the diffusion scheduler $w(t)$. 
% \vspace{-3mm}

\subsection{Overview of the RTPD Algorithm}\label{subsec:algo_overview}
Fig.~\ref{fig:Overview} presents an overview of our RTPD algorithm.
It is grounded in the Hausdorff distance-based penetration depth calculation method (Sec.~\ref{sec:preliminary}).
%, similar to that of Tang et al.~\shortcite{SIG09HIST}.
The process consists of two primary phases: penetration surface extraction and Hausdorff distance calculation.
We leverage the RTX platform's capabilities to accelerate both of these steps.

\begin{figure*}[t]
    \centering
    \includegraphics[width=0.8\textwidth]{Image/overview.pdf}
    \caption{The overview of RT-based penetration depth calculation algorithm overview}
    \label{fig:Overview}
\end{figure*}

The penetration surface extraction phase focuses on identifying the overlapped region between two objects.
\revision{The penetration surface is defined as a set of polygons from one object, where at least one of its vertices lies within the other object. 
Note that in our work, we focus on triangles rather than general polygons, as they are processed most efficiently on the RTX platform.}
To facilitate this extraction, we introduce a ray-tracing-based \revision{Point-in-Polyhedron} test (RT-PIP), significantly accelerated through the use of RT cores (Sec.~\ref{sec:RT-PIP}).
This test capitalizes on the ray-surface intersection capabilities of the RTX platform.
%
Initially, a Geometry Acceleration Structure (GAS) is generated for each object, as required by the RTX platform.
The RT-PIP module takes the GAS of one object (e.g., $GAS_{A}$) and the point set of the other object (e.g., $P_{B}$).
It outputs a set of points (e.g., $P_{\partial B}$) representing the penetration region, indicating their location inside the opposing object.
Subsequently, a penetration surface (e.g., $\partial B$) is constructed using this point set (e.g., $P_{\partial B}$) (Sec.~\ref{subsec:surfaceGen}).
%
The generated penetration surfaces (e.g., $\partial A$ and $\partial B$) are then forwarded to the next step. 

The Hausdorff distance calculation phase utilizes the ray-surface intersection test of the RTX platform (Sec.~\ref{sec:RT-Hausdorff}) to compute the Hausdorff distance between two objects.
We introduce a novel Ray-Tracing-based Hausdorff DISTance algorithm, RT-HDIST.
It begins by generating GAS for the two penetration surfaces, $P_{\partial A}$ and $P_{\partial B}$, derived from the preceding step.
RT-HDIST processes the GAS of a penetration surface (e.g., $GAS_{\partial A}$) alongside the point set of the other penetration surface (e.g., $P_{\partial B}$) to compute the penetration depth between them.
The algorithm operates bidirectionally, considering both directions ($\partial A \to \partial B$ and $\partial B \to \partial A$).
The final penetration depth between the two objects, A and B, is determined by selecting the larger value from these two directional computations.

%In the Hausdorff distance calculation step, we compute the Hausdorff distance between given two objects using a ray-surface-intersection test. (Sec.~\ref{sec:RT-Hausdorff}) Initially, we construct the GAS for both $\partial A$ and $\partial B$ to utilize the RT-core effectively. The RT-based Hausdorff distance algorithms then determine the Hausdorff distance by processing the GAS of one object (e.g. $GAS_{\partial A}$) and set of the vertices of the other (e.g. $P_{\partial B}$). Following the Hausdorff distance definition (Eq.~\ref{equation:hausdorff_definition}), we compute the Hausdorff distance to both directions ($\partial A \to \partial B$) and ($\partial B \to \partial A$). As a result, the bigger one is the final Hausdorff distance, and also it is the penetration depth between input object $A$ and $B$.


%the proposed RT-based penetration depth calculation pipeline.
%Our proposed methods adopt Tang's Hausdorff-based penetration depth methods~\cite{SIG09HIST}. The pipeline is divided into the penetration surface extraction step and the Hausdorff distance calculation between the penetration surface steps. However, since Tang's approach is not suitable for the RT platform in detail, we modified and applied it with appropriate methods.

%The penetration surface extraction step is extracting overlapped surfaces on other objects. To utilize the RT core, we use the ray-intersection-based PIP(Point-In-Polygon) algorithms instead of collision detection between two objects which Tang et al.~\cite{SIG09HIST} used. (Sec.~\ref{sec:RT-PIP})
%RT core-based PIP test uses a ray-surface intersection test. For purpose this, we generate the GAS(Geometry Acceleration Structure) for each object. RT core-based PIP test takes the GAS of one object (e.g. $GAS_{A}$) and a set of vertex of another one (e.g. $P_{B}$). Then this computes the penetrated vertex set of another one (e.g. $P_{\partial B}$). To calculate the Hausdorff distance, these vertex sets change to objects constructed by penetrated surface (e.g. $\partial B$). Finally, the two generated overlapped surface objects $\partial A$ and $\partial B$ are used in the Hausdorff distance calculation step.

Our goal is to increase the robustness of T2I models, particularly with rare or unseen concepts, which they struggle to generate. To do so, we investigate a retrieval-augmented generation approach, through which we dynamically select images that can provide the model with missing visual cues. Importantly, we focus on models that were not trained for RAG, and show that existing image conditioning tools can be leveraged to support RAG post-hoc.
As depicted in \cref{fig:overview}, given a text prompt and a T2I generative model, we start by generating an image with the given prompt. Then, we query a VLM with the image, and ask it to decide if the image matches the prompt. If it does not, we aim to retrieve images representing the concepts that are missing from the image, and provide them as additional context to the model to guide it toward better alignment with the prompt.
In the following sections, we describe our method by answering key questions:
(1) How do we know which images to retrieve? 
(2) How can we retrieve the required images? 
and (3) How can we use the retrieved images for unknown concept generation?
By answering these questions, we achieve our goal of generating new concepts that the model struggles to generate on its own.

\vspace{-3pt}
\subsection{Which images to retrieve?}
The amount of images we can pass to a model is limited, hence we need to decide which images to pass as references to guide the generation of a base model. As T2I models are already capable of generating many concepts successfully, an efficient strategy would be passing only concepts they struggle to generate as references, and not all the concepts in a prompt.
To find the challenging concepts,
we utilize a VLM and apply a step-by-step method, as depicted in the bottom part of \cref{fig:overview}. First, we generate an initial image with a T2I model. Then, we provide the VLM with the initial prompt and image, and ask it if they match. If not, we ask the VLM to identify missing concepts and
focus on content and style, since these are easy to convey through visual cues.
As demonstrated in \cref{tab:ablations}, empirical experiments show that image retrieval from detailed image captions yields better results than retrieval from brief, generic concept descriptions.
Therefore, after identifying the missing concepts, we ask the VLM to suggest detailed image captions for images that describe each of the concepts. 

\vspace{-4pt}
\subsubsection{Error Handling}
\label{subsec:err_hand}

The VLM may sometimes fail to identify the missing concepts in an image, and will respond that it is ``unable to respond''. In these rare cases, we allow up to 3 query repetitions, while increasing the query temperature in each repetition. Increasing the temperature allows for more diverse responses by encouraging the model to sample less probable words.
In most cases, using our suggested step-by-step method yields better results than retrieving images directly from the given prompt (see 
\cref{subsec:ablations}).
However, if the VLM still fails to identify the missing concepts after multiple attempts, we fall back to retrieving images directly from the prompt, as it usually means the VLM does not know what is the meaning of the prompt.

The used prompts can be found in \cref{app:prompts}.
Next, we turn to retrieve images based on the acquired image captions.

\vspace{-3pt}
\subsection{How to retrieve the required images?}

Given $n$ image captions, our goal is to retrieve the images that are most similar to these captions from a dataset. 
To retrieve images matching a given image caption, we compare the caption to all the images in the dataset using a text-image similarity metric and retrieve the top $k$ most similar images.
Text-to-image retrieval is an active research field~\cite{radford2021learning, zhai2023sigmoid, ray2024cola, vendrowinquire}, where no single method is perfect.
Retrieval is especially hard when the dataset does not contain an exact match to the query \cite{biswas2024efficient} or when the task is fine-grained retrieval, that depends on subtle details~\cite{wei2022fine}.
Hence, a common retrieval workflow is to first retrieve image candidates using pre-computed embeddings, and then re-rank the retrieved candidates using a different, often more expensive but accurate, method \cite{vendrowinquire}.
Following this workflow, we experimented with cosine similarity over different embeddings, and with multiple re-ranking methods of reference candidates.
Although re-ranking sometimes yields better results compared to simply using cosine similarity between CLIP~\cite{radford2021learning} embeddings, the difference was not significant in most of our experiments. Therefore, for simplicity, we use cosine similarity between CLIP embeddings as our similarity metric (see \cref{tab:sim_metrics}, \cref{subsec:ablations} for more details about our experiments with different similarity metrics).

\vspace{-3pt}
\subsection{How to use the retrieved images?}
Putting it all together, after retrieving relevant images, all that is left to do is to use them as context so they are beneficial for the model.
We experimented with two types of models; models that are trained to receive images as input in addition to text and have ICL capabilities (e.g., OmniGen~\cite{xiao2024omnigen}), and T2I models augmented with an image encoder in post-training (e.g., SDXL~\cite{podellsdxl} with IP-adapter~\cite{ye2023ip}).
As the first model type has ICL capabilities, we can supply the retrieved images as examples that it can learn from, by adjusting the original prompt.
Although the second model type lacks true ICL capabilities, it offers image-based control functionalities, which we can leverage for applying RAG over it with our method.
Hence, for both model types, we augment the input prompt to contain a reference of the retrieved images as examples.
Formally, given a prompt $p$, $n$ concepts, and $k$ compatible images for each concept, we use the following template to create a new prompt:
``According to these examples of 
$\mathord{<}c_1\mathord{>:<}img_{1,1}\mathord{>}, ... , \mathord{<}img_{1,k}\mathord{>}, ... , \mathord{<}c_n\mathord{>:<}img_{n,1}\mathord{>}, ... , $
$\mathord{<}img_{n,k}\mathord{>}$,
generate $\mathord{<}p\mathord{>}$'', 
where $c_i$ for $i\in{[1,n]}$ is a compatible image caption of the image $\mathord{<}img_{i,j}\mathord{>},  j\in{[1,k]}$. 

This prompt allows models to learn missing concepts from the images, guiding them to generate the required result. 

\textbf{Personalized Generation}: 
For models that support multiple input images, we can apply our method for personalized generation as well, to generate rare concept combinations with personal concepts. In this case, we use one image for personal content, and 1+ other reference images for missing concepts. For example, given an image of a specific cat, we can generate diverse images of it, ranging from a mug featuring the cat to a lego of it or atypical situations like the cat writing code or teaching a classroom of dogs (\cref{fig:personalization}).
\vspace{-2pt}
\begin{figure}[htp]
  \centering
   \includegraphics[width=\linewidth]{Assets/personalization.pdf}
   \caption{\textbf{Personalized generation example.}
   \emph{ImageRAG} can work in parallel with personalization methods and enhance their capabilities. For example, although OmniGen can generate images of a subject based on an image, it struggles to generate some concepts. Using references retrieved by our method, it can generate the required result.
}
   \label{fig:personalization}\vspace{-10pt}
\end{figure}



\section{Experiments}

\subsection{Experimental Setup}

\textbf{Datasets.} We use three categories from the Amazon Reviews dataset~\cite{mcauley2015image} for our experiments: ``Sports and Outdoors'' (\textbf{Sports}), ``Beauty'' (\textbf{Beauty}), and ``CDs and Vinyl'' (\textbf{CDs}). Each user’s historical reviews are considered ``actions'' and are sorted chronologically as action sequences, with earlier reviews appearing first. To evaluate the models, we adopt the widely used leave-last-out protocol~\cite{kang2018sasrec,zhao2022revisiting,rajput2023tiger}, where the last item and second-to-last item in each action sequence are used for testing and validation, respectively. More details about the datasets can be found in~\Cref{app:datasets}.

\textbf{Compared methods.} We compare the performance of ActionPiece with the following methods: (1)~ID-based sequential recommendation methods, including BERT4Rec~\cite{sun2019bert4rec}, and SASRec~\cite{kang2018sasrec}; (2)~feature-enhanced sequential recommendation methods, such as FDSA~\cite{zhang2019fdsa}, S$^3$-Rec~\cite{zhou2020s3}, and VQ-Rec~\cite{hou2023vqrec}; and (3)~generative recommendation methods, including P5-CID~\cite{hua2023p5cid}, TIGER~\cite{rajput2023tiger}, LMIndexer~\cite{jin2024lmindexer}, HSTU~\cite{zhai2024hstu}, and SPM-SID~\cite{singh2024spmsid}, each representing a different action tokenization method (\Cref{tab:act_tokenization}). A detailed description of these baselines is provided in~\Cref{appendix:baselines}.

\textbf{Evaluation settings.} Following~\citet{rajput2023tiger}, we use Recall@$K$ and NDCG@$K$ as metrics to evaluate the methods, where $K \in \{5, 10\}$. Model checkpoints with the best performance on the validation set are used for evaluation on the test set. We run the experiments with five random seeds and report the average metrics.

\textbf{Implementation details.} Please refer to~\Cref{appendix:implementation} for detailed implementation and hyperparameter settings.


\subsection{Overall Performance}

We compare ActionPiece with sequential recommendation and generative recommendation baselines, which use various action tokenization methods, across three public datasets. The results are shown in~\Cref{tab:performance}. 

For the compared methods, we observe that those using item features generally outperform item ID-only methods. This indicates that incorporating features enhances recommendation performance. Among the methods leveraging item features (``Feature + ID'' and ``Generative''), generative recommendation models achieve better performance. These results further confirm that injecting semantics into item indexing and optimizing at a sub-item level enables generative models to better use semantic information and improve recommendation performance. Among all the baselines, SPM-SID achieves the best results. By incorporating the SentencePiece model~\cite{kudo2018sentencepiece}, SPM-SID replaces popular semantic ID patterns within each item with new tokens, benefiting from a larger vocabulary.

\begin{table}[t!]
    \small
    \centering
	\caption{Ablation analysis of ActionPiece. The recommendation performance is measured using NDCG@$10$. The best performance is denoted in \textbf{bold} fonts.}
	\label{tab:ablation}
	\vskip 0.1in
% 	\setlength{\tabcolsep}{1mm}{
% \resizebox{2.1\columnwidth}{!}{
    \begin{tabular}{lccc}
	\toprule
	\multicolumn{1}{c}{\textbf{Variants}} & \textbf{Sports} & \textbf{Beauty} & \textbf{CDs} \\
	\midrule
	\midrule
    \multicolumn{4}{@{}c}{\textit{TIGER with larger vocabularies}} \\
    \midrule
    (1.1) TIGER\ -\ 1k ($4 \times 2^8$) & 0.0225 & 0.0384 & 0.0411 \\
    (1.2) TIGER-49k ($6 \times 2^{13}$) & 0.0162 & 0.0317 & 0.0338 \\
    (1.3) TIGER-66k ($4 \times 2^{14}$) & 0.0194 & N/A$^\dag$ & 0.0319 \\
    \midrule
    \multicolumn{4}{@{}c}{\textit{Vocabulary construction}} \\
    \midrule
    (2.1) \emph{w/o} tokenization & 0.0215 & 0.0389 & 0.0346 \\
    (2.2) \emph{w/o} context-aware & 0.0258 & 0.0416 & 0.0429 \\
    (2.3) \emph{w/o} weighted counting & 0.0257 & 0.0412 & 0.0435 \\
    \midrule
    \multicolumn{4}{@{}c}{\textit{Set permutation regularization}} \\
    \midrule
    (3.1) only for inference & 0.0192 & 0.0316 & 0.0329 \\
    (3.2) only for training & 0.0244 & 0.0387 & 0.0422 \\
    \midrule
    ActionPiece (40k) & \textbf{0.0264} & \textbf{0.0424} & \textbf{0.0451} \\
    \bottomrule
	\end{tabular}
	\vspace{0.05cm}
	\begin{flushleft}
        $^\dag$ not applicable as $2^{14}$ is larger than \#items in Beauty.
    \end{flushleft}
% 	}}
    \vskip -0.2in
\end{table}

\begin{figure*}[t!]
    \begin{center}
    \includegraphics[width=\linewidth]{fig/vocab_size.pdf}
    \vskip -0.1in
    \caption{Analysis of recommendation performance (NDCG@10, $\uparrow$) and average tokenized sequence length (NSL, $\downarrow$) \wrt vocabulary size across three datasets.
    % NSL refers to the normalized sequence length, calculated relative to the number of initial tokens.
    ``N/A’’ indicates that ActionPiece is not applied, \ie action sequences are represented solely by initial tokens.}
    \label{fig:vocab_size}
    \end{center}
    \vskip -0.2in
\end{figure*}

Our proposed ActionPiece consistently outperforms all baselines across three datasets, achieving a significant improvement in NDCG@$10$. It surpasses the best-performing baseline method by $6.00\%$ to $12.82\%$. Unlike existing methods, ActionPiece is the first context-aware action sequence tokenizer, \ie the same action can be tokenized into different tokens depending on its surrounding context. This allows ActionPiece to capture important sequence-level feature patterns that enhance recommendation performance.

% \begin{figure}[t]
% % \vskip 0.2in
% \begin{center}
% \centerline{\includegraphics[width=0.85\columnwidth]{fig/ndcg_vs_vocab_size.pdf}}
% \end{center}
% % \vskip -0.3in
% \vspace{-0.3in}
% \caption{Comparison of performance and vocabulary size (\#token for TIGER, SPM-SID, and ActionPiece; \#item for SASRec; and \#item+\#attribute for S$^3$-Rec) on ``Sports'' dataset.
% % By adjusting the vocabulary size, ActionPiece is shown to balance memory efficiency and recommendation performance.
% }
% \label{fig:intro}
% % \vskip -0.2in
% \vspace{-0.1in}
% \end{figure}


\subsection{Ablation Study}\label{sec:ablation}

We conduct ablation analyses in~\Cref{tab:ablation} to study how each proposed technique contributes to ActionPiece.\\
\hspace*{3mm} (1)~We increase the vocabulary size of TIGER, to determine whether the performance gain of ActionPiece is solely due to scaling up the number of tokens in the vocabulary. By increasing the number of semantic ID digits per item~($4 \rightarrow 6$) and the number of candidate semantic IDs per digit~($2^8 \rightarrow 2^{13}\ \text{or}\ 2^{14}$), we create two variants with vocabularies larger than ActionPiece. However, these TIGER variants perform worse than ActionPiece, and even the original TIGER with only $1024$ tokens. The experimental results suggest that scaling up the vocabulary size for generative recommendation models is challenging, consistent with the observations from~\citet{zhang2024moc}.\\
\hspace*{3mm} (2)~To evaluate the effectiveness of the proposed vocabulary construction techniques, we introduce the following variants: \emph{(2.1)~w/o tokenization}, which skips vocabulary construction, using item features directly as tokens; \emph{(2.2)~w/o context-aware}, which only considers co-occurrences and merges tokens within each action during vocabulary construction and segmentation; and \emph{(2.3)~w/o weighted counting}, which treats all token pairs equally rather than using the weights defined in~\Cref{eq:p_one_set,eq:p_two_sets}. The results indicate that removing any of these techniques reduces performance, demonstrating the importance of these methods for building a context-aware tokenizer.\\
\hspace*{3mm} (3)~To evaluate the effectiveness of SPR, we revert to naive segmentation, as described in~\Cref{subsubsec:segmentation}, during model training and inference, respectively. The results show that replacing SPR with naive segmentation in either training or inference degrades performance.

\begin{figure}[t!]
    \begin{center}
    \includegraphics[width=0.95\columnwidth]{fig/token_util.pdf}
    \vskip -0.1in
    \caption{Analysis of token utilization rate (\%) during model training \wrt segmentation strategy.
    % ``SPR'' denotes set permutation regularization.
    }
    \label{fig:token_util}
    \end{center}
    % \vskip -0.3in
    \vskip -0.3in
\end{figure}

\subsection{Further Analysis}

% In this section, we analyze the impact of key hyperparameters in vocabulary construction and segmentation.

\subsubsection{Performance and Efficiency \wrt Vocabulary Size}

Vocabulary size is a key hyperparameter for language tokenizers~\cite{meta2024llama3,dagan2024getting}. In this study, we investigate how adjusting vocabulary size affects the generative recommendation models. We use the normalized sequence length (NSL)~\cite{dagan2024getting} to measure the length of tokenized sequences, where a smaller NSL indicates fewer tokens per tokenized sequence. We experiment with vocabulary sizes in \{N/A, 5k, 10k, 20k, 30k, 40k\}, where ``N/A'' represents the direct use of item features as tokens. As shown in~\Cref{fig:vocab_size}, increasing the vocabulary size improves recommendation performance and reduces the tokenized sequence length. Conversely, reducing the vocabulary size lowers the number of model parameters, improving memory efficiency. This analysis demonstrates that adjusting vocabulary size enables a trade-off between model performance, sequence length, and memory efficiency.

\subsubsection{Token Utilization Rate \wrt Segmentation Strategy}\label{sec:token_utilization}

As described in~\Cref{subsubsec:training}, applying SPR augments the training corpus by producing multiple token sequences that share the same semantics. In~\Cref{tab:ablation}, we observe that incorporating SPR significantly improves recommendation performance. One possible reason is that SPR increases token utilization rates. To validate this assumption, we segment the action sequences in each training epoch using two strategies: naive segmentation and SPR. As shown in~\Cref{fig:token_util}, naive segmentation uses only $56.89\%$ of tokens for model training, limiting the model's ability to generalize to unseen action sequences. In contrast, SPR achieves a token utilization rate of $87.01\%$ after the first training epoch, with further increases as training progresses. These results demonstrate that the proposed SPR segmentation strategy improves the utilization of ActionPiece tokens, enabling better generalization and enhanced performance.


\subsubsection{Performance \wrt Inference-Time Ensembles}\label{sec:inference_time_ensemble}

As described in~\Cref{subsubsec:inference}, ActionPiece supports inference-time ensembling by using SPR segmentation. We vary the number of ensembled segments, $q$, in \{N/A, 1, 3, 5, 7\}, where ``N/A'' indicates using naive segmentation during model inference. As shown in~\Cref{fig:ensemble}, ensembling more tokenized sequences improves ActionPiece's recommendation performance. However, the performance gains slow down as $q$ increases to $5$ and $7$. Since a higher $q$ also increases the computational cost of inference, this creates a trade-off between performance and computational budget in practice.

\begin{figure}[t!]
    \begin{center}
    \includegraphics[width=\columnwidth]{fig/ensemble.pdf}
    \vskip -0.15in
    \caption{Analysis of performance (NDCG@10, $\uparrow$) \wrt the number of ensembled segments $q$ during model inference.}
    \label{fig:ensemble}
    \end{center}
    \vskip -0.25in
\end{figure}

\subsection{Case Study}\label{subsec:case}

To understand how GR models benefit from the unordered feature setting and context-aware action sequence tokenization, we present an illustrative example in~\Cref{fig:case}.

Each item in the action sequence is represented as a feature set, with each item consisting of five features. The features within an item do not require a specific order. The first step of tokenization leverages the unordered nature of the feature set and applies set permutation regularization~(\Cref{subsubsec:segmentation}). This process arranges each feature set into a specific permutation and iteratively groups features based on the constructed vocabulary~(\Cref{subsubsec:vocab_construct}). This results in different segments that convey the same semantics. Each segment is represented as a sequence of sets, where each set corresponds to a token in the vocabulary.

By examining the segments and their corresponding token sequences, we identify four types of tokens, as annotated in~\Cref{fig:case}: (1) a subset of features from a single item (token {\setlength{\fboxsep}{0pt}\colorbox{myblue}{14844}} corresponds to features {\setlength{\fboxsep}{0pt}\colorbox{myblue}{747}} and {\setlength{\fboxsep}{0pt}\colorbox{myblue}{923}} of the T-shirt); (2) a set containing a single feature (feature {\setlength{\fboxsep}{0pt}\colorbox{mygreen}{76}} of the socks); (3) all features of a single item (token {\setlength{\fboxsep}{0pt}\colorbox{myyellow}{7995}} corresponds to all features of the shorts); and (4) features from multiple items (\eg token {\setlength{\fboxsep}{0pt}\colorbox{myblue}{83}\colorbox{mygreen}{16}} includes feature {\setlength{\fboxsep}{0pt}\colorbox{myblue}{923}} from the T-shirt and feature {\setlength{\fboxsep}{0pt}\colorbox{mygreen}{679}} from the socks, while token {\setlength{\fboxsep}{0pt}\colorbox{mygreen}{19}\colorbox{myyellow}{895}} includes feature {\setlength{\fboxsep}{0pt}\colorbox{mygreen}{1100}} from the socks as well as features {\setlength{\fboxsep}{0pt}\colorbox{myyellow}{560}} and {\setlength{\fboxsep}{0pt}\colorbox{myyellow}{943}} from the shorts). Notably, the fourth type of token demonstrates that the features of one action can be segmented and grouped with features from adjacent actions. This results in different tokens for the same action depending on the surrounding context, showcasing the context-aware tokenization process of ActionPiece.








\bibliography{reference}
\bibliographystyle{icml2025}



\newpage
\appendix
\onecolumn


\clearpage
\section{Appendix}

\subsection{feature statistic}
\label{app:feature_statistic}

In this part, we present the statistics of the decision token types in our dataset. Table~\ref{tab:Math_Mistral_confused_proportion} and Table~\ref{tab:Math_Llama_confused_proportion} shows the statistical information of the math data, and Table~\ref{tab:Code_confused_proportion} shows that of the code. 
We adopt en\textunderscore core\textunderscore web\textunderscore sm model from Spacy library as tokenizer and POS tagger to make statistics. We show the cases for types of tokens in Appendix~\ref{app:case_study}.

\subsection{Dataset Information Statistic}
\label{app:dataset_statistic}

In Figure~\ref{fig:statistic-information}, we report the statistical information of math training data for our dataset and ER-PRM, Math-Shepherd, PRM800K\protect\footnotemark[2]. In Figure~\ref{fig:statistic-information-bon}, we show the statistical information for our math BoN candidates.

\subsection{Case Study}
\label{app:case_study}


\footnotetext[2]{Same to~\cite{wang2024mathshepherdverifyreinforcellms}, We counted the number of samples for PRM800K and is a quarter of that of Math-Shepherd. }


\begin{table*}[htbp]
  \caption{MetaMath-Mistral generated data statistic results: percentage of tokens types and percentage of decision tokens types for math domain. \textbf{Natural Sentence} stands for a piece of text separated by a \textit{New line break} or \textit{Punctuation} like Period and Question Mark. \textbf{Reasoning} represents \textit{symbolic reasoning} or \textit{Math Formula}; \textbf{Entity} represents \textit{Noun} like apple or personal name; \textbf{Semantics} represents \textit{Conjunction}, \textit{Verb} and \textit{Determiner}. We also find that there are few word level splits represented by \textbf{Split Word}; we retained these segmentation points to enhance the model's generalization at these points during PRM training.}
    \noindent 
    \begin{minipage}{\textwidth}
    
   
   \setlength{\abovecaptionskip}{7pt}
   
   \vspace{-0.05in}
   
   \setlength{\cmidrulewidth}{0.01em}
   \renewcommand{\tabcolsep}{10pt}
   \renewcommand{\arraystretch}{1.2}
   \centering
   
   \begin{tabular}{cl|cc}
   
   \toprule
        \multirow{2}{*}{Categories} & \hspace{2.0em}\multirow{2}{*}{Subtypes} & \multicolumn{2}{c}{Position}\\
        \cmidrule[\cmidrulewidth](lr){3-4}
          && Token type proportion (78m) & decision token proportion (1517k)\\
         \midrule
         \multirow{2}{*}{Natural Sentence} &
          \vline \hspace{0.7em} New line break  & 3.85\% & 2.70\% \\
         & \raisebox{0pt}[0.3cm][0pt]{\rule{0.5pt}{1cm}}\hspace{1.4em} Punctuation  & 26.92\% & 4.61\% \\

         
         \multirow{2}{*}{Reasoning} & \vline \hspace{0.1em} Symbolic Reasoning  & 15.39\% & 6.79\% \\

         & \vline \hspace{0.8em} Math Formula &  3.85\% & 21.03\% \\
        
        
         \multirow{1}{*}{Entity}&  \hspace{2.7em} Noun & 15.38\% & 11.01\% \\

        
         \multirow{3}{*}{Semantics}& \vline \hspace{1.4em} Conjunction & 20.51\% & 29.00\% \\
         & \vline \hspace{2.8em} Verb & 6.41\% & 5.34\% \\
         & \raisebox{0pt}[0.3cm][0pt]{\rule{0.5pt}{1cm}} \hspace{1.5em} Determiner & 7.69\% & 2.64\% \\
         
         \bottomrule
   \end{tabular}
   

  \label{tab:Math_Mistral_confused_proportion}
    \end{minipage}
\end{table*}


\clearpage


\begin{table*}[htbp]
   
  \caption{Proportion of decision tokens in the original data of the same type for math domain generated by MetaMath-Llama.}
    \noindent 
    \begin{minipage}{\textwidth}
    
   
   \setlength{\abovecaptionskip}{7pt}
   
   % \label{tab:sg_blocks}
   \vspace{-0.05in}
   
   \setlength{\cmidrulewidth}{0.01em}
   \renewcommand{\tabcolsep}{10pt}
   \renewcommand{\arraystretch}{1.2}
   \centering
   
    \begin{tabular}{cl|cc}
   
   \toprule
        \multirow{2}{*}{Categories} & \hspace{2.0em}\multirow{2}{*}{Subtypes} & \multicolumn{2}{c}{Position}\\
        \cmidrule[\cmidrulewidth](lr){3-4}
          && Token type proportion (81m) & decision token proportion (1413k)\\
         \midrule
         \multirow{2}{*}{Natural Sentence} &
          \vline \hspace{0.7em} New line break  & 2.47\% & 6.69\% \\
         & \raisebox{0pt}[0.3cm][0pt]{\rule{0.5pt}{1cm}}\hspace{1.4em} Punctuation  &  28.40\% & 14.91\% \\

         
         \multirow{2}{*}{Reasoning} & \vline \hspace{0.1em} Symbolic Reasoning & 16.05\% & 5.66\% \\

         & \vline \hspace{0.8em} Math Formula &   3.7\% & 20.24\% \\
        
        
         \multirow{1}{*}{Entity}&  \hspace{2.7em} Noun &  14.82\% & 7.35\% \\

        
         \multirow{3}{*}{Semantics}& \vline \hspace{1.4em} Conjunction &  20.99\% & 23.48\% \\
         & \vline \hspace{2.8em} Verb &  6.17\% & 5.24\% \\
         & \raisebox{0pt}[0.3cm][0pt]{\rule{0.5pt}{1cm}} \hspace{1.5em} Determiner &  7.4\% & 2.99\% \\
         
         
         \bottomrule
   \end{tabular}

  \label{tab:Math_Llama_confused_proportion}
    \end{minipage}
\end{table*}



\begin{table*}[htbp]
  \caption{Proportion of decision tokens in the original data of the same type for code domain}

    \begin{minipage}{\textwidth}
    
   \setlength{\abovecaptionskip}{7pt}
   \setlength{\cmidrulewidth}{0.01em}
   \renewcommand{\tabcolsep}{10pt}
   \renewcommand{\arraystretch}{1.2}
   \begin{tabular}{cl|ccc}
   
   \toprule
        \multirow{2}{*}{Categories} & \hspace{2em}\multirow{2}{*}{Subtypes} & \multicolumn{2}{c}{Position}\\
        \cmidrule[\cmidrulewidth](lr){3-4}
          & & \hspace{0.2em}Token type proportion (17m) &\hspace{1.0em}decision token proportion (47k)\\
         \midrule
         \multirow{2}{*}{Syntax Symbol} &
          \vline \hspace{0.4em} New line break  & 6.99\% & 11.79\% \\
         & \vline \hspace{0.5em} Space Character  & 77.58\% & 1.60\% \\
             

        \multirow{1}{*}{Numbers}& \hspace{2.0em} Number & 4.21\% & 0.84\% \\
        
         \multirow{2}{*}{Logical Operators} &
        \vline \hspace{0.3em} Boolean Operators  & 0.26\% & 3.21\% \\
        & \vline \hspace{0em}Arithmetic Operators & 2.04\% & 3.54\% \\
        
        
        \multirow{1}{*}{Definition} & \hspace{1.5em} Def / Class & 0.53\% & 1.82\% \\
      
         \multirow{1}{*}{Import Statement}& \hspace{0.8em} From / Import & 0.58\% & 0.76\% \\

         \multirow{3}{*}{Function}& \vline \hspace{0.6em} Type Defination & 0.16\% & 0.48\% \\
         & \vline \hspace{0.5em} Build-in Function & 0.49\% & 0.77\% \\
         & \vline \hspace{0.8em} Instance Method & 0.09\% & 0.26\% \\
         
         \multirow{1}{*}{Control Statements} & \hspace{1.3em} If / Else / Elif &  0.64\% & 3.51\% \\
        

        \multirow{1}{*}{Loop Statements} & \hspace{1.8em} For / While &  0.62\% & 1.73\% \\
         
         \multirow{2}{*}{Others} & \vline\hspace{2.2em} Return  & 0.68\% & 0.58\% \\
        & \vline \hspace{0.2em} Punctuation Mark  & 4.99\% & 6.52\% \\
         
         \bottomrule
   \end{tabular}
   

  \label{tab:Code_confused_proportion}
    \end{minipage}
\end{table*}




\begin{table*}
 \caption{Proportion of decision tokens in Code and Code Comment}
  \centering
  \begin{tabular}{c|ccc}
    \hline
    Categories & Trigger type(234k) & Token type (17m) & Line number(1599k) \\
    \hline
    Code & 47k (19.95\%) & 4m (26.15\%) & 1280k(80.02\%)     \\
    Code comment & 187k (80.05\%) & 13m (73.85\%) & 319k(19.98\%)   \\
    \hline
  \end{tabular}
 
  \label{tab:code_and_comments}
\end{table*}

\begin{table*}[htbp]
  \caption{Samples of decision tokens for math domain.}
    \noindent
    \begin{minipage}{\textwidth}
   \setlength{\abovecaptionskip}{7pt}
   \vspace{-0.05in}
   \setlength{\cmidrulewidth}{0.01em}
   \renewcommand{\tabcolsep}{10pt}
   \renewcommand{\arraystretch}{1.2}
   \centering
   \begin{tabular}{cl|c@{}}
   
   \toprule
        \multirow{1}{*}{Categories} & \hspace{2.0em}\multirow{1}{*}{Subtypes} & Sample\\
         \midrule
         \multirow{2}{*}{Sentence} &
         \vline\hspace{0.9em} New line break &  works on 4 of them each day.\textcolor{red}{\textbackslash \textbf{n}}After 5 days,\\
         & \vline\hspace{1.4em} Punctuation & If Billie has 18 crayons\textcolor{red}{\textbf{,}} and Bobbie has three times\\
         
         \multirow{2}{*}{Reasoning} & \hspace{0.8em} Text reasoning & gives them \textcolor{red}{\textbf{3}} points. So in total, Joe's team has 3 + 3 = 6\\
         
         & \hspace{1.0em} Math formula & so x \textcolor{red}{\textbf{+}} 4x - 10 = 25\\
         

         \multirow{1}{*}{Entity}& \hspace{2.7em} Noun & Ron gets to pick a new \textcolor{red}{\textbf{book}} 1 out of 13\\
         
         \multirow{3}{*}{Semantics}& \vline \hspace{1.4em} Conjunction & their ages is 34, \textcolor{red}{\textbf{so}} we can write the equation L + (L + 4) = 34.\\
         & \vline \hspace{2.8em} Verb & In 14 days, each dog will \textcolor{red}{\textbf{eat}} 250 grams/day\\
         & \vline \hspace{1.5em} Determiner & we can round \textcolor{red}{\textbf{this}} to the nearest whole number. \\
         
         \bottomrule
   \end{tabular}
   

  \label{tab:Math_confused_samples}
    \end{minipage}
\end{table*}

\begin{table*}[htbp]
  \caption{Samples of decision tokens for code domain}
    \noindent 

    \begin{minipage}{\textwidth}
    
   \setlength{\abovecaptionskip}{7pt}
   \setlength{\cmidrulewidth}{0.01em}
   \renewcommand{\tabcolsep}{10pt}
   \renewcommand{\arraystretch}{1.2}
   \centering
   \begin{tabular}{cl|c}
   
   \toprule
        Categories & \hspace{2.3em}Subtypes & Sample\\
        
         \midrule
         \multirow{2}{*}{Syntax Symbol} &
          \vline \hspace{0.9em} New line break  & \textcolor{red}{\textbackslash n} i += num\_bytes \\
         & \vline \hspace{0.8em} Space Character & dp[i][j] += dp[i - 1][j] * (j - k) \textcolor{red}{\textbackslash \textbf{s}} \\
             

        \multirow{1}{*}{Numbers}& \hspace{2.0em} Number & j = (target - x * \textcolor{red}{\textbf{2}}) // 2\\
        
         \multirow{2}{*}{Logical Operators} &
        \vline \hspace{0.3em} Boolean Operators & if c in count \textcolor{red}{\textbf{and}} c != a:\\
        & \vline \hspace{0em}Arithmetic Operators & dp = [[0] * (n\textcolor{red}{\textbf{+}}1) for \_ in range(n+1)]\\
         \multirow{2}{*}{Definition} & \hspace{2.8em} Def & \textcolor{red}{\textbf{def}} is\_valid(r, c):\\
         & \hspace{2.8em}Class & \textcolor{red}{\textbf{class}} Solution:\\
         \multirow{2}{*}{Import Statement}& \hspace{2.7em} From &\textcolor{red}{\textbf{from}} collections import defaultdict \\
         & \hspace{2.3em} Import & \textcolor{red}{\textbf{import}} collections\\
         \multirow{3}{*}{Function}& \vline \hspace{1.0em} Type Defination & for size in \textcolor{red}{\textbf{list}}(dp[curr\_sum]):\\
         & \vline \hspace{0.8em} Build-in Function & if \textcolor{red}{\textbf{abs}}(next\_count + 1) > 0:\\
         & \vline \hspace{0.8em} Instance Method & \textcolor{red}{\textbf{self}}.count = 0\\
         
         \multirow{3}{*}{Control Statements} & \hspace{3.5em} If & \textcolor{red}{\textbf{if}} len(tokens) < 4:\\
         &\hspace{3.3em}Else & \textcolor{red}{\textbf{else}}:\\
         & \hspace{3.3em}Elif & \textcolor{red}{\textbf{elif}} level == 0 and expression[i] == ' ':\\
        \multirow{2}{*}{Loop Statements} & \hspace{3.1em} For & \textcolor{red}{for} i in range(len(fronts)):\\
         & \hspace{2.7em} While & \textcolor{red}{\textbf{while}} x != self.parent[x]:\\
         \multirow{2}{*}{Others} & \vline\hspace{2.2em} Return & \textcolor{red}{\textbf{return}} (merged[n // 2 - 1] + merged[n // 2]) / 2.0\\
        & \vline \hspace{0.2em} Punctuation Mark  & digit\_sum = \textcolor{red}{\textbf{(}}l1.val if l1 else 0) + (l2.val if l2 else 0)\\
         
         \bottomrule
   \end{tabular}
   
  \label{tab:Code_confused_samples}
    \end{minipage}
\end{table*}




\begin{figure*}[htbp]
    \centering
    \begin{minipage}[b]{0.48\textwidth}
        \centering
        \includegraphics[width=\textwidth]{files/figures/figure-datasets-steps.pdf} 
        \subcaption{}\label{fig:datasets-average-step}
    \end{minipage}%
    \hfill
    \begin{minipage}[b]{0.48\textwidth}
        \centering
        \includegraphics[width=\textwidth]{files/figures/figure-datasets-samples.pdf} 
        \subcaption{}\label{fig:datasets-sample-number}
    \end{minipage}%
    \hfill
    \begin{minipage}[b]{0.48\textwidth}
        \centering
        \includegraphics[width=\textwidth]{files/figures/figure-datasets-tokens.pdf} 
        \subcaption{}\label{fig:datasets-tokens-per-step}
    \end{minipage}%
    \hfill
    \begin{minipage}[b]{0.48\textwidth}
        \centering
        \includegraphics[width=\textwidth]{files/figures/figure-datasets-length.pdf} 
        \subcaption{}\label{fig:datasets-tokens-per-sample}
    \end{minipage}%
    \hfill
    \caption{Statistic Information of our math dataset, Ours-M represents data constructed by Mistral, and Ours-L represents data constructed by Llama. ER-PRM, Math-Shepherd (M-S), PRM800K. (a): Average step; (b): Sample number; (c): Average tokens per reasoning step; (d): Sample length. We use a Mistral tokenizer for statistics.}
    \label{fig:statistic-information}
\end{figure*}


\begin{figure*}[htbp]
    \centering
    \begin{minipage}[b]{0.48\textwidth}
        \centering
        \includegraphics[width=\textwidth]{files/figures/figure-bon-mistral.pdf} 
        \subcaption{}\label{fig:eval-datasets-mistral-length}
    \end{minipage}%
    \hfill
    \begin{minipage}[b]{0.48\textwidth}
        \centering
        \includegraphics[width=\textwidth]{files/figures/figure-bon-Llama.pdf} 
        \subcaption{}\label{fig:eval-datasets-llama-length}
    \end{minipage}%
    \caption{Statistic Information of our BoN dataset (a): Statistic with Mistral tokenizer; (b): Statistic with Llama tokenizer.}
    \label{fig:statistic-information-bon}
\end{figure*}



\end{document}

