
%\newtheorem{exercise}{Exercise}[chapter]
%\newtheorem{example}{Example}[section]  % the number of the example will be different
\newtheorem{theoremraw}{Theorem}[chapter]

\makeatletter
\newcommand{\paragrapharrow}{%
	\@startsection{paragraph}{4}{\z@}%
	{3.25ex \@plus1ex \@minus.2ex}%
	{-1em}%
	{\normalfont\normalsize\bfseries$\blacktriangleright$\ }}
\makeatother

%%%%%%%%%%%%%%%%%%%%%%%%%%%%%%
% colors
\usepackage[noframe]{showframe}
\usepackage{framed}
\usepackage{lipsum}

% margins 
\newcommand{\roundcornertheorem}{0pt}
\newcommand{\linewidththeorem}{0.1pt}
\newcommand{\frametitlerulewidththeorem}{0.1pt}
\newcommand{\innerbottommargintheorem}{2pt}
\newcommand{\innerleftmargintheorem}{2pt}
\newcommand{\innerrightmargintheorem}{2pt}
\newcommand{\innertopmargintheorem}{2pt}
\newcommand{\outerlinewidththeorem}{1pt}
\newcommand{\frametitleaboveskiptheorem}{1pt}

\newcommand{\textremark}{3pt}
%%%%%%%%%%%%%%%%%%%%%%%%%%%%%%
% new theorems
%\renewcommand{\thesection}{\@arabic\c@section}
%\newcounter{theo}[chapter] \setcounter{theo}{0}
%%\renewcommand{\thetheo}{\arabic{section}.\arabic{theo}}
%%\renewcommand{\thetheo}{\thechapter.\thesection.\arabic{theo}}
%\renewcommand{\thetheo}{\thechapter.\arabic{theo}}



%%%%%%%%%%%%%%%%%%%%%%%%%%%%%%
%Definition
%\newenvironment{definition}[1][]{%
	%	\refstepcounter{theo}%
	%	\ifstrempty{#1}%
	%	{\newcommand{\defName}{}}
	%	{\newcommand{\defName}{:~#1}}
	%	\mdfsetup{%
		%		frametitle={%
			%			\tikz[baseline=(current bounding box.east),outer sep=0pt]
			%			\node[anchor=east,rectangle,fill=\mdframecolorDefinition]
			%			{\strut Definition~\thetheo\defName};}}%
	%	\mdfsetup{innertopmargin=10pt,linecolor=\mdframecolorDefinition,%
		%		linewidth=2pt,topline=true,%
		%		frametitleaboveskip=\dimexpr-\ht\strutbox\relax
		%	}
	%	\begin{mdframed}[]\relax%
		%	}{\end{mdframed}}
%%%%%%%%%%%%%%%%%%%%%%%%%%%%%%
%Remark
\newenvironment{remarkOLD}[1][]{%
	\refstepcounter{theo}%
	\ifstrempty{#1}%
	{\newcommand{\defName}{}}
	{\newcommand{\defName}{:~#1}}
	\mdfsetup{%
		frametitle={%
			\tikz[baseline=(current bounding box.east),outer sep=0pt]
			\node[anchor=east,rectangle,fill=structurecolorHighTheorem]
			{\strut Remark~\thetheo\defName};}}%
	\mdfsetup{innertopmargin=10pt,linecolor=structurecolorHighTheorem,%
		linewidth=2pt,topline=true,%
		frametitleaboveskip=\dimexpr-\ht\strutbox\relax
		% new added
		frametitlerule=false, 
		frametitlerulewidth=\frametitlerulewidththeorem,%    options when using {tcolorbox}?
		%		roundcorner=\textremark,
		%		linewidth=\textremark,
		innertopmargin=\textremark,
		innerbottommargin=\textremark,
		innerleftmargin=\textremark,
		innerrightmargin=\textremark,
		frametitlebelowskip=3pt,
		outerlinewidth=1.2pt,
	}
	\begin{mdframed}[]\relax%
	}{\end{mdframed}}
%----------------------------------------------------------------------------------------
% Theorem Blues
%----------------------------------------------------------------------------------------
%%%%%%%%%%%%%%%%%%%%%%%%%%%%%%
%Theorem Blue
\newenvironment{theoremHigh}[1][]{%
\refstepcounter{theo}%
\ifstrempty{#1}%
{\newcommand{\theoName}{}}
{\newcommand{\theoName}{:~(#1)}}
\mdfsetup{
	backgroundcolor=structurecolorHighTheorem,
	linecolor=structurecolorHighTheorem,
	% ----------------------------------
	frametitlerulewidth=\frametitlerulewidththeorem,%    options when using {tcolorbox}?
	roundcorner=\roundcornertheorem,
	linewidth=\linewidththeorem,
	innerbottommargin=\innerbottommargintheorem,
	innerleftmargin=\innerleftmargintheorem,
	innerrightmargin=\innerrightmargintheorem,
	innertopmargin=\innertopmargintheorem,
	outerlinewidth=\outerlinewidththeorem,
	topline=false,
	innertopmargin=-5pt,
	innerbottommargin=1pt,
	linewidth=0,
	startinnercode=\paragraph{{\strut Theorem~\thetheo\theoName}}
}
\begin{mdframed}[]\relax%
}{\end{mdframed}}
%%%%%%%%%%%%%%%%%%%%%%%%%%%%%%
%Corollary Blue
\newenvironment{corollaryHigh}[1][]{%
\refstepcounter{theo}%
\ifstrempty{#1}%
{\newcommand{\theoName}{}}
{\newcommand{\theoName}{:~(#1)}}
\mdfsetup{
	backgroundcolor=structurecolorHighTheorem,
	linecolor=structurecolorHighTheorem,
	% ----------------------------------
	frametitlerulewidth=\frametitlerulewidththeorem,%    options when using {tcolorbox}?
	roundcorner=\roundcornertheorem,
	linewidth=\linewidththeorem,
	innerbottommargin=\innerbottommargintheorem,
	innerleftmargin=\innerleftmargintheorem,
	innerrightmargin=\innerrightmargintheorem,
	innertopmargin=\innertopmargintheorem,
	outerlinewidth=\outerlinewidththeorem,
	topline=false,
	innertopmargin=-5pt,
	innerbottommargin=1pt,
	linewidth=0,
	startinnercode=\paragraph{{\strut Corollary~\thetheo\theoName}}
}
\begin{mdframed}[]\relax%
}{\end{mdframed}}
%%%%%%%%%%%%%%%%%%%%%%%%%%%%%%
%Lemma Blue
\newenvironment{lemmaHigh}[1][]{%
\refstepcounter{theo}%
\ifstrempty{#1}%
{\newcommand{\theoName}{}}
{\newcommand{\theoName}{:~(#1)}}
\mdfsetup{
	backgroundcolor=structurecolorHighTheorem,
	linecolor=structurecolorHighTheorem,
	% ----------------------------------
	frametitlerulewidth=\frametitlerulewidththeorem,%    options when using {tcolorbox}?
	roundcorner=\roundcornertheorem,
	linewidth=\linewidththeorem,
	innerbottommargin=\innerbottommargintheorem,
	innerleftmargin=\innerleftmargintheorem,
	innerrightmargin=\innerrightmargintheorem,
	innertopmargin=\innertopmargintheorem,
	outerlinewidth=\outerlinewidththeorem,
	topline=false,
	innertopmargin=-5pt,
	innerbottommargin=1pt,
	linewidth=0,
	startinnercode=\paragraph{{\strut Lemma~\thetheo\theoName}}
}
\begin{mdframed}[]\relax%
}{\end{mdframed}}
%%%%%%%%%%%%%%%%%%%%%%%%%%%%%%
%Proposition Blue
\newenvironment{propositionHigh}[1][]{%
\refstepcounter{theo}%
\ifstrempty{#1}%
{\newcommand{\theoName}{}}
{\newcommand{\theoName}{:~(#1)}}
\mdfsetup{
	backgroundcolor=structurecolorHighTheorem,
	linecolor=structurecolorHighTheorem,
	% ----------------------------------
	frametitlerulewidth=\frametitlerulewidththeorem,%    options when using {tcolorbox}?
	roundcorner=\roundcornertheorem,
	linewidth=\linewidththeorem,
	innerbottommargin=\innerbottommargintheorem,
	innerleftmargin=\innerleftmargintheorem,
	innerrightmargin=\innerrightmargintheorem,
	innertopmargin=\innertopmargintheorem,
	outerlinewidth=\outerlinewidththeorem,
	topline=false,
	innertopmargin=-5pt,
	innerbottommargin=1pt,
	linewidth=0,
	startinnercode=\paragraph{{\strut Proposition~\thetheo\theoName}}
}
\begin{mdframed}[]\relax%
}{\end{mdframed}}

%----------------------------------------------------------------------------------------
%   Theorem with grays
%----------------------------------------------------------------------------------------
%%%%%%%%%%%%%%%%%%%%%%%%%%%%%%
%Theorem
\newenvironment{theorem}[1][]{%
	\refstepcounter{theo}%
	\ifstrempty{#1}%
	{\newcommand{\theoName}{}}
	{\newcommand{\theoName}{:~(#1)}}
	\mdfsetup{
		backgroundcolor=\mdframecolorTheorem,
		linecolor=\mdframecolorTheorem,
		% ----------------------------------
		frametitlerulewidth=\frametitlerulewidththeorem,%    options when using {tcolorbox}?
		roundcorner=\roundcornertheorem,
		linewidth=\linewidththeorem,
		innerbottommargin=\innerbottommargintheorem,
		innerleftmargin=\innerleftmargintheorem,
		innerrightmargin=\innerrightmargintheorem,
		innertopmargin=\innertopmargintheorem,
		outerlinewidth=\outerlinewidththeorem,
		topline=false,
		innertopmargin=-5pt,
		innerbottommargin=1pt,
		linewidth=0,
		startinnercode=\paragraph{{\strut Theorem~\thetheo\theoName}}
	}
	\begin{mdframed}[]\relax%
	}{\end{mdframed}}
%%%%%%%%%%%%%%%%%%%%%%%%%%%%%%
%Corollary
\newenvironment{corollary}[1][]{%
	\refstepcounter{theo}%
\ifstrempty{#1}%
{\newcommand{\theoName}{}}
{\newcommand{\theoName}{:~(#1)}}
\mdfsetup{
	backgroundcolor=\mdframecolorTheorem,
	linecolor=\mdframecolorTheorem,
	% ----------------------------------
	frametitlerulewidth=\frametitlerulewidththeorem,%    options when using {tcolorbox}?
	roundcorner=\roundcornertheorem,
	linewidth=\linewidththeorem,
	innerbottommargin=\innerbottommargintheorem,
	innerleftmargin=\innerleftmargintheorem,
	innerrightmargin=\innerrightmargintheorem,
	innertopmargin=\innertopmargintheorem,
	outerlinewidth=\outerlinewidththeorem,
	topline=false,
	innertopmargin=-5pt,
	innerbottommargin=1pt,
	linewidth=0,
	startinnercode=\paragraph{{\strut Corollary~\thetheo\theoName}}
}
	\begin{mdframed}[]\relax%
	}{\end{mdframed}}
%%%%%%%%%%%%%%%%%%%%%%%%%%%%%%
%Lemma
\newenvironment{lemma}[1][]{%
	\refstepcounter{theo}%
\ifstrempty{#1}%
{\newcommand{\theoName}{}}
{\newcommand{\theoName}{:~(#1)}}
\mdfsetup{
	backgroundcolor=\mdframecolorTheorem,
	linecolor=\mdframecolorTheorem,
	% ----------------------------------
	frametitlerulewidth=\frametitlerulewidththeorem,%    options when using {tcolorbox}?
	roundcorner=\roundcornertheorem,
	linewidth=\linewidththeorem,
	innerbottommargin=\innerbottommargintheorem,
	innerleftmargin=\innerleftmargintheorem,
	innerrightmargin=\innerrightmargintheorem,
	innertopmargin=\innertopmargintheorem,
	outerlinewidth=\outerlinewidththeorem,
	topline=false,
	innertopmargin=-5pt,
	innerbottommargin=1pt,
	linewidth=0,
	startinnercode=\paragraph{{\strut Lemma~\thetheo\theoName}}
}
\begin{mdframed}[]\relax%
}{\end{mdframed}}

%%%%%%%%%%%%%%%%%%%%%%%%%%%%%%
%Proposition
\newenvironment{proposition}[1][]{%
\refstepcounter{theo}%
\ifstrempty{#1}%
{\newcommand{\theoName}{}}
{\newcommand{\theoName}{:~(#1)}}
\mdfsetup{
	backgroundcolor=\mdframecolorTheorem,
	linecolor=\mdframecolorTheorem,
	% ----------------------------------
	frametitlerulewidth=\frametitlerulewidththeorem,%    options when using {tcolorbox}?
	roundcorner=\roundcornertheorem,
	linewidth=\linewidththeorem,
	innerbottommargin=\innerbottommargintheorem,
	innerleftmargin=\innerleftmargintheorem,
	innerrightmargin=\innerrightmargintheorem,
	innertopmargin=\innertopmargintheorem,
	outerlinewidth=\outerlinewidththeorem,
	topline=false,
	innertopmargin=-5pt,
	innerbottommargin=1pt,
	linewidth=0,
	startinnercode=\paragraph{{\strut Proposition~\thetheo\theoName}}
}
\begin{mdframed}[]\relax%
}{\end{mdframed}}


%%%%%%%%%%%%%%%%%%%%%%%%%%%%%%%%%%%%%%%%%%%%%%%%%%%%%%%%%%%%%%%%%%%%%%%%%%%%%


%----------------------------------------------------------------------------------------
% Definition box		
%----------------------------------------------------------------------------------------
\usepackage{amsthm}
\newtheoremstyle{normalfontstyle} % Name
{3pt}                           % Space above
{3pt}                           % Space below
{\normalfont}                   % Body font
{}                              % Indent amount
{\bfseries}                     % Theorem head font
{}                             % Punctuation after theorem head
{ }                             % Space after theorem head
{}                              % Theorem head spec (can be left empty, meaning 'normal')

\usepackage{thmtools}
% Define a new theorem style that uses normal font for the body and bold font for the head
\declaretheoremstyle[
spaceabove=3pt,
spacebelow=3pt,
headfont=\bfseries,
notefont=\bfseries, % This ensures the optional note is in bold
notebraces={(}{)}, % This ensures the optional note is in brackets
bodyfont=\normalfont,
postheadspace=1em, % adds a space after the theorem head.
]{normalfontboldhead}


\theoremstyle{normalfontstyle}
\newcommand{\BlackBox}{\rule{1.5ex}{1.5ex}}  % end of proof
\renewenvironment{proof}{\par\noindent{\bf Proof\ }}{\hfill\BlackBox\\[2mm]}

%\newtheorem{definitionT}[theo]{Definition} % Use the same counter as theo
%\newtheorem{definitionT}{Definition}[chapter]
\declaretheorem[style=normalfontboldhead, name=Definition, numberlike=theo]{definitionT}
\newmdenv[skipabove=7pt,
skipbelow=7pt,
rightline=false,
leftline=true,
topline=false,
bottomline=false,
linecolor=mydarkblue,
innerleftmargin=5pt,
innerrightmargin=5pt,
innertopmargin=0pt,
leftmargin=2cm,
rightmargin=0cm,
linewidth=4pt,
innerbottommargin=0pt]{dBox}
\newenvironment{definition}{\begin{dBox}\begin{definitionT}}{\end{definitionT}\end{dBox}}


%\newtheorem{exerciseC}[theo]{Exercise} % Use the same counter as theo
%\newtheorem{exerciseC}{Exercise}[chapter]
\declaretheorem[style=normalfontboldhead, name=Exercise, numberlike=theo]{exerciseC}
\newmdenv[skipabove=7pt,
skipbelow=7pt,
rightline=false,
leftline=true,
topline=false,
bottomline=false,
linecolor=mydarkgreen,
innerleftmargin=5pt,
innerrightmargin=5pt,
innertopmargin=0pt,
leftmargin=2cm,
rightmargin=0cm,
linewidth=4pt,
innerbottommargin=0pt]{eBox}
\newenvironment{exercise}{\begin{eBox}\begin{exerciseC}}{\end{exerciseC}\end{eBox}}


%\newtheorem{remarekC}[theo]{Remark} % Use the same counter as theo
%\newtheorem{remarekC}{Remark}[chapter]
\declaretheorem[style=normalfontboldhead, name=Remark, numberlike=theo]{remarekC}
\newmdenv[skipabove=7pt,
skipbelow=7pt,
rightline=false,
leftline=true,
topline=false,
bottomline=false,
linecolor=mydarkpurple,
innerleftmargin=5pt,
innerrightmargin=5pt,
innertopmargin=0pt,
leftmargin=2cm,
rightmargin=0cm,
linewidth=4pt,
innerbottommargin=0pt]{rBox}
\newenvironment{remark}{\begin{rBox}\begin{remarekC}}{\end{remarekC}\end{rBox}}


\declaretheorem[style=normalfontboldhead, name=Assumption, numberlike=theo]{assumptionC}
\newmdenv[skipabove=7pt,
skipbelow=7pt,
rightline=false,
leftline=true,
topline=false,
bottomline=false,
linecolor=mydarkpurple,
innerleftmargin=5pt,
innerrightmargin=5pt,
innertopmargin=0pt,
leftmargin=2cm,
rightmargin=0cm,
linewidth=4pt,
innerbottommargin=0pt]{asBox}
\newenvironment{assumption}{\begin{asBox}\begin{assumptionC}}{\end{assumptionC}\end{asBox}}


%\newtheorem{exampleC}[theo]{Example} % Use the same counter as theo
\declaretheorem[style=normalfontboldhead, name=Example, numberlike=theo]{exampleC}
\newmdenv[skipabove=7pt,
skipbelow=7pt,
rightline=false,
leftline=false,
topline=false,
bottomline=false,
linecolor=mydarkgreen,
innerleftmargin=1pt,
innerrightmargin=5pt,
innertopmargin=0pt,
leftmargin=2cm,
rightmargin=0cm,
linewidth=4pt,
innerbottommargin=0pt]{xBox}
\newenvironment{example}{\begin{xBox}\begin{exampleC}}{\exampbar\end{exampleC}\end{xBox}}


\usepackage{adforn}  % symbols of \adftripleflourishleft
\newcommand{\xchaptertitle}{Chapter~\thechapter~}
\newcommand{\problemname}{Problems}
\newenvironment{problemset}[1][\xchaptertitle~\problemname]{
\vspace*{10pt}
\begin{center}
\phantomsection\addcontentsline{toc}{section}{\texorpdfstring{\xchaptertitle~\problemname}{\problemname}}
% \markboth{#1}{\rightmark}
\markright{#1}
\textcolor{structurecolor}{\Large\bfseries\adftripleflourishleft~#1~\adftripleflourishright}
\end{center}
\begin{enumerate}[ref=\thechapter.\theenumi]}{
\end{enumerate}} 


\newenvironment{note}{
\par\noindent\makebox[-3pt][r]{
\scriptsize\color{red!90}\textdbend\quad}
\textbf{\color{black}\notename} \citshape}{\par}

\newenvironment{itemcircle}{
\begin{enumerate}[label=\small\protect\circled{\arabic*}]}{
\end{enumerate}}


%----------------------------------------------------------------------------------------
% item distance & style 		
%----------------------------------------------------------------------------------------
\usepackage[shortlabels]{enumitem}  % enum style [](i)]
\usepackage{enumitem}
% item distance
%\setenumerate[1]{itemsep=0pt,partopsep=0pt,parsep=\parskip,topsep=5pt}
%\setitemize[1]{itemsep=0pt,partopsep=0pt,parsep=\parskip,topsep=5pt}
%\setdescription{itemsep=0pt,partopsep=0pt,parsep=\parskip,topsep=5pt}
% item style
\newcommand*{\eitemi}{\tikz \draw [baseline, ball color=structurecolor,draw=none] circle (2pt);}
\newcommand*{\eitemii}{\tikz \draw [baseline, fill=structurecolor,draw=none,circular drop shadow] circle (2pt);}
\newcommand*{\eitemiii}{\tikz \draw [baseline, fill=structurecolor,draw=none] circle (2pt);}
\setlist[enumerate,1]{label=\color{black}\arabic*.,itemsep=0pt,partopsep=0pt,parsep=\parskip,topsep=3pt}
\setlist[enumerate,2]{label=\color{black}(\alph*).,itemsep=0pt,partopsep=0pt,parsep=\parskip,topsep=3pt}
\setlist[enumerate,3]{label=\color{black}\Roman*.,itemsep=0pt,partopsep=0pt,parsep=\parskip,topsep=3pt}
\setlist[enumerate,4]{label=\color{black}\Alph*.,itemsep=0pt,partopsep=0pt,parsep=\parskip,topsep=3pt}
\setlist[itemize,1]{label={\eitemi},itemsep=0pt,partopsep=0pt,parsep=\parskip,topsep=3pt}
\setlist[itemize,2]{label={\eitemii},itemsep=0pt,partopsep=0pt,parsep=\parskip,topsep=3pt}
\setlist[itemize,3]{label={\eitemiii},itemsep=0pt,partopsep=0pt,parsep=\parskip,topsep=3pt}








