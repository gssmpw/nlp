\documentclass[10pt,letterpaper]{article}

\usepackage{cogsci}

\cogscifinalcopy

\usepackage{pslatex}
\usepackage{apacite}
\usepackage{graphicx}
\usepackage{linguex}
\usepackage{float}


\setlength\titlebox{8cm}

\title{FutureVision: A methodology for the investigation of future cognition}

\author{
\begin{tabular}{c} {\bf Tiago Timponi Torrent} {(tiago.torrent@ufjf.br)} \\ FrameNet Brasil \\ Federal University of Juiz de Fora - CNPq \\ \\
{\bf Nicolás Hinrichs} {(hinrichsn@cbs.mpg.de)} \\ Research Group Cognition and Plasticity \\ Max Planck Inst. for Human Cognitive and Brain Sciences \\ \\
{\bf Igor Lourenço} {(ials@ufu.br)} \\ Institute for Language and Linguistics \\ Federal University of Uberlândia - CNPq \\ \\
{\bf Marcelo Viridiano} {(marcelo.viridiano@case.edu)} \\ FrameNet Brasil \\ Federal University of Juiz de Fora - CNPq \end{tabular}
\begin{tabular}{c} {\bf Mark Turner} {(turner@case.edu)} \\ Red Hen Lab \\ Case Western Reserve University \\ \\
{\bf Frederico Belcavello} {(fred.belcavello@ufjf.br)} \\ FrameNet Brasil \\ Federal University of Juiz de Fora - CNPq \\ \\
{\bf Arthur Lorenzi Almeida} {(arthur.lorenzi@estudante.ufjf.br)} \\ FrameNet Brasil \\ Federal University of Juiz de Fora \\ \\
{\bf Ely Edison Matos} {(ely.matos@ufjf.br)} \\ FrameNet Brasil \\ Federal University of Juiz de Fora \end{tabular} 
}
\date{}

\begin{document}

\maketitle



\begin{abstract}
This paper presents a methodology combining multimodal semantic analysis with an eye-tracking experimental protocol to investigate the cognitive effort involved in understanding the communication of future scenarios. To demonstrate the methodology, we conduct a pilot study examining how visual fixation patterns vary during the evaluation of valence and counterfactuality in fictional ad pieces describing futuristic scenarios, using a portable eye tracker. Participants' eye movements are recorded while evaluating the stimuli and describing them to a conversation partner. Gaze patterns are analyzed alongside semantic representations of the stimuli and participants' descriptions, constructed from a frame semantic annotation of both linguistic and visual modalities. Preliminary results show that far-future and pessimistic scenarios are associated with longer fixations and more erratic saccades, supporting the hypothesis that fractures in the base spaces underlying the interpretation of future scenarios increase cognitive load for comprehenders.

\textbf{Keywords:} 
future cognition; linguistic cognition; multimodality; eye tracking; FrameNet annotation.
\end{abstract}


\section{Introduction}

The fields of Speculative Futures and Quantitative Futurism \cite{webbsignals,hoffman2022speculative,mcgonigal2022imaginable} aim at conceiving of and communicating future scenarios that allow people and organizations to plan their actions towards their long-term goals. Key authors devise methodologies for effective future scenario forecasting, as well as propose diffent kinds of interventions––workshops, games, immersive experiences and fictional ad pieces such as the one in Figure \ref{fig:mystique}––to teach people how to engage in future-oriented thinking.

Human cognition relies on a range of processes to construct, interpret, and communicate counterfactual and imagined scenarios. In this paper, we present a methodology aimed at investigating the factors impacting the cognitive effort required from comprehenders to understand multimodal communication of futuristic scenarios. Specifically, we focus on evaluating the extent to which violations of the semantic frames used as foundational structures for analogical projections influence cognitive effort in explaining future scenarios.

\citeA{fauconnier1997mappings}, throughout his work on mental spaces, emphasized the power of base spaces and the dependence of mental space networks on base frames. Adopting a distinct but related perspective, Fillmore, throughout his work on Frame Semantics, has supported the claim that frames structure our experience of the world, being defined as systems of interrelated concepts \cite{Fillmore1982}, which can take alternate perspectives \cite{Fillmore1985} and be organized into a network-like database \cite{Baker:1998:BFP:980845.980860,baker2003}.

\begin{figure}[t]
\begin{center}
\includegraphics[width=\columnwidth]{figures/mystique.png}
\end{center}
\caption{Fictional ad piece from the 2022 Future Today Institue Tech Trends Report.} 
\label{fig:mystique}
\end{figure}

\citeA{torrent&turnerfuturemind} observe that, even with the proliferation of spaces in a mental space network and the reformulation and reconstruction of its contents and connections, the base spaces and the frames structuring them generally persist when humans engage in counterfactual thinking. The mental operations and communicative constructions needed to build these networks depend on and favor the persistence of the base. The base is especially important for imagining and understanding unreal worlds, such as counterfactual, future or fictional scenarios. 

In this research, we use the theoretical-methodological foundations of Cognitive Linguistics \cite{Fillmore1982,Fillmore1985,fauconnier1997mappings,fauconnier2002way}, multimodal frame-semantic annotation \cite{belcavello-etal-2020-frame,belcavello-etal-2022-charon}, and eye-tracking techniques \cite{10.5555/3134162,Conklinetal2018} to gauge the extent to which violations of the base spaces used for futuristic projections correlate with an increase in the cognitive effort required to explain scenarios presented in the form of fictitious advertising pieces. In the remainder of this paper, we present the Persistence of the Base hypothesis and, next, the FutureVision methodology. Last, we analyze data from a pilot experiment and discuss its contributions and limitations.

\section{Persistence of The Base in Future Cognition}

Analogical and projective processes \cite{fauconnier2002way} are mobilized to anchor future scenarios in experiences that are culturally shared between the interlocutors involved in the communicative process in which they engage. \citeA{torrent&turnerfuturemind} demonstrate specific patterns of basic structure persistence by analyzing scenario forecasting.

As an example, consider the fictional ad piece from the 2022 Future Today Institute Tech Trends Report \cite{fti2022} in Figure \ref{fig:mystique}. The story told by the ad relies on a background of knowledge that the storyteller needs the audience to activate so that the future scenario the former had in mind––Future A––is maximally similar to the one constructed by the latter––Future B. Based on the ground, the storyteller will engage in setting up a joint attention scene to which the audience is expected to attend. When talking about the future, the storyteller and the audience are not expected to focus their attention on the here and now of the conversation, but rather on a there and then in the future. Hence, this is the case where a blended joint attention scene is needed \cite{Turner2015}. 

Once the scene is set, the storyteller prompts the audience to build mental spaces required for comprehending the ad. Those mental spaces are triggered by both linguistic elements and other communicative modes, such as visuals. Mental spaces are structured by frames. In this case, the frame is \texttt{Body\_decoration}. Each element of the frame can be mapped to elements in the ad. The \textsc{Decoration} is the bottle of Privée Mystique, the \textsc{Decorated\_individual} is the model, and her pixelated face is the \textsc{Body\_location}.\footnote{The complete description of this and the other frames indicated in \texttt{Courier} font in this paper can be found at \texttt{http://webtool.frame.net.br/report/frame}}

Nonetheless, this is not a regular ad for a common cosmetic foundation. This is an ad for a cosmetic foundation of the future. Hence, the audience is invited, by the communicative context––note that this ad was published in the 2022 FTI Report as an example of a near-future optimistic scenario––to project another mental space 50 years from now, where the product is to be sold to people represented by the model. To be able to do so, the audience must recognize the position of this new space in a time scale, so that they can calibrate their expectations accordingly.

From this new space, the audience is invited to build two other spaces: one in the past, which can serve as a base for thinking about how the notion of make-up has evolved, given its uses through time, and another one also around 50 years from now, but in which a new element is to be included. The space in the past allows the audience to reason about how make-up has changed its function and use patterns since it first started being used––from its early uses for ceremonial events, to its deployment in theater in lieu of personal masks, to our current days. The counterfactual space allows for thinking about the idea that make-up can serve a different purpose than it currently does: namely, it can protect one's identity against facial recognition technology. Since the notion of concealing still persists, a generic mental space with this concept is brought into play.

Vital relations \cite{fauconnier2002way} connect the elements in all the spaces. The different people wearing make-up throughout the years are linked via Analogy, which allows for the compression of all the different instances of make-up users into one single identity. Disanalogy, in turn, allows for the connection between a product primarily made to conceal flaws in the skin surface to one aimed at concealing identities. To be able to make this connection, the audience must bring into play yet another frame: that of \texttt{Protecting}, where an \textsc{Asset} is to be protected from some \textsc{Danger} by the deployment of some \textsc{Protection}. Those new connections build an access point for the construction of a new future concept.

From the access point indicated above, the audience will be able to reconstrue the frame blending intended by the storyteller. The clues for building this new blended space in the future include linguistic elements––\textit{"Hide in plain sight while still getting noticed"}––working together with visual elements––the camera icon in the foundation bottle and the pixelated aspect of the model's face. Other less direct clues help the audience fit the new frame into their system of concepts: the model in the ad is black and black women are amongst the most prominent victims of harmful algorithmic biases.

\begin{figure}[t]
\begin{center}
\includegraphics[width=\columnwidth]{figures/persistence.png}
\end{center}
\caption{Cognitive mechanisms involved in the comprehension of the fictional ad piece.} 
\label{fig:persistence}
\end{figure}

Throughout the process just described, both storytellers and audience rely on embodiment \cite{kirchhoff2018} to be able to recruit frames and reframe concepts. They also resort to image schemas and force dynamics \cite{talmy1988force} to make sense of generalities involved in specific concepts. For the case under analysis, the make-up is conceived of as a barrier precluding an agonist––the face recognition technology––to invade a bounded region––the model's identity. Figure \ref{fig:persistence} summarizes the cognitive mechanisms involved in the comprehension of the ad. In the next section, we present FutureVision: a methodology devised for investigating the extent to which differences in shared bakground knowledge, counterfactuality and valence may influence the cognitive effort needed for understanding communication of future scenarios.  

\section{FutureVision: A Methodology for Investigating Future Cognition}

The FutureVision methodology builds on recent advances on multimodal frame-semantic annotation \cite{belcavello-etal-2020-frame,belcavello-etal-2022-charon} and eye-tracking techiniques \cite{10.5555/3134162,Conklinetal2018} to shed light on the cognitive effort involved in the comprehension of future scenarios.

To test the feasibility of the methodology, a pilot study employed an experimental setup involving eye-tracking glasses to evaluate gaze fixation patterns during scenario evaluation.\footnote{The experimental protocol was approved by the Committee for Ethics in Research at the Federal University of Juiz de Fora under the number 84376424.0.0000.5147.} A conversation pair is formed at random, based on availability. This pair is invited to enter a room, sit in chairs about 50 centimeters apart at an approximate 90-degree angle, in front of a 40-inch video monitor equipped with an image capture camera. In this configuration, participants are in a comfortable position both to look at each other and to look at the monitor, and they are also in the visual field captured by the camera. 

Participant A, chosen at random from one of the two participants, is asked to put on the Tobii Pro Glasses 3 eye tracker glasses and to take on the role of "speaker". Participant B, therefore, does not wear the glasses and takes on the role of "listener". Participants are asked to read the instructions, which inform them that the experiment will last approximately 25 minutes, with an optional break of 5 minutes; that they will be shown a set of 8 advertisements (text and image) on the monitor screen that do not exist in real life, but which, in theory, could exist in the future.\footnote{For the pilot experiment, text in the ads was translated into Brazilian Portuguese.} The ads representing future scenarios have been selected from the Future Today Institute's 2022 Technology Trends Report \cite{fti2022} and have been grouped into four sets according to Valence (pessimistic/optimistic) and Counterfactuality (near/far future). Participant A's task is to explain to participant B the future scenario that brings relevance to whatever product or service the ad presents. The data collected are the fixation points of participant A's gaze on the screen and the utterances participant A makes––which are transcribed into text.

\subsection{Participant selection}

Participants were recruited and selected based on their knowledge of Brazilian Portuguese, which was the language used in the experiment. Inclusion criteria required participants to be over 18 years of age, native speakers of Brazilian Portuguese, literate, and without hearing complaints or visual problems (unless the visual problems can be corrected by the use of lenses). Subjects are not filtered or separated by sex or gender. Participation in the experiment is free and consensual.

Subjects are excluded from participating if they have evidence or history of central nervous system disease, of problematic alcohol consumption, of abuse of other substances, of learning or memory problems, of psychiatric disorder, of serious medical illness, or of photosensitive epilepsy, or are taking any medication known to have centrally acting effects. Compliance with these requirements is  measured through the application of a triage questionnaire.

\subsection{Risks and Benefits}

The risks involved in participating are minimal. We would highlight the risk of participant identification. These risks are lessened to the extent that all participants are registered with a random control number which are only accessible by the researcher in charge and kept confidential. We have also considered the possibility of the participant's feeling embarrassment, discomfort, stress, or tiredness when answering the questions asked by the researcher or interacting during the experimental task with an unknown person. In these cases, the participant can request a break, stop answering any question, or even stop taking part in the research at any time, without any sanction.

The benefits of this research for the participants are indirect, in that their participation contributes to a better understanding of how the semantic processing of visions of the future takes place. This understanding can bring benefits to society as a whole as well as foster research.

\subsection{Data analysis}

\subsubsection{Multimodal frame-semantic annotation of experimental stimuli}

Eight fictional ads from the FTI 2022 Tech Trends Report––two for each combination: near-future optimistic, near-future pessimistic, far-future optimistic and far-future pessimistic––were annotated  by two experts with 10+ years of experience in annotation for frames and frame elements for both the linguistic and the visual communicative modes. Semantic representations built from the annotations were stored for each stimulus. 

\subsubsection{Frame-semantic annotation of Participant A's descriptions}

Participant A's verbal descriptions of the stimuli were automatically transcribed and annotated using the categories of frame semantics.\footnote{Participants descriptions were transcribed using Notta.ai (available at \texttt{https://app.notta.ai/}) and manually revised by a trained linguist. Frame-semantic annotation was performed on an instance of LOME \cite{xia-etal-2021-lome} trained with FrameNet Brasil data.} This involves identifying the main conceptual structures (frames) that correspond to the participants' linguistic output, providing a qualitative context for the gaze pattern data.

\subsubsection{Processing eye-tracking data}

Raw gaze tracking data was extracted from Tobii Pro Glasses 3, focusing on key metrics such as fixation duration, fixation count and saccadic movement patterns. Gaze data was segmented according to the different stimuli presented during the session, allowing a detailed comparison of fixation patterns in various valence and counterfactual conditions. Areas of interest (AOIs) were defined so as to keep track of the multimodal composition of the ads, including both text blocks and profiled images, as well as the whole screen.

\subsubsection{Statistical analysis}

Mixed-effects regression models were employed to analyze the relationship between gaze behavior, cognitive effort, and scenario interpretation. Fixation duration, pupil dilation, and AOI transition probabilities were treated as dependent variables, while scenario valence and future distance served as fixed effects. Random intercepts accounted for individual differences in cognitive flexibility and baseline attention patterns. Mediation analysis tested whether pupil dilation mediated the relationship between scenario distance and semantic similarity scores. Cluster analysis was conducted to identify gaze behavior patterns, classifying participants into low and high cognitive load groups based on fixation variability and saccadic patterns. Markov chain models were applied to AOIs transition data to quantify gaze regressions and assess conceptual integration difficulty across conditions. 

\subsubsection{Integration of behavioral data and gaze pattern data}

Following the methodology proposed by \citeA{viridiano-etal-2022-case}, cosine similarity between the frame-semantic representations of the stimuli and that of the participants' descriptions was calculated to assess how close their interpretations are to the intended valence and counterfactuality. This is correlated with fixation variability to explore how cognitive costs influence perception and interpretation processes. The results of the comparative and regression analyses are interpreted to understand how the fracture of the base space manifests itself in attachment patterns and cognitive processing.

\section{Results}

In this section we present the results obtained from the pilot application of the FutureVision methodology to three pairs of participants.

\subsection{Similarities between semantic representations}

Cosine similarities for the frame-semantic representations obtained from the annotation of the stimuli and of the participants' descriptions of the fictional ads are given in Table \ref{tbl:cosine}. The first trend to be noted is that near-future scenarios present, on average, higher cosine similarity scores than far-future scenarios. Also, optimistic scenarios tend to have higher similarity scores than pessimistic ones.

Also noteworthy is the discrepancy between the similarity scores for the two near-future optimistic scenarios. One of the fictional ads in this condition is precisely the Privée Mystique ad (Figure \ref{fig:mystique}). Although the FTI Tech Trends Report \cite{fti2022} presents this as an optimistic scenario, since it would allow population profiles more prone to suffering from algorithmic bias to protect themselves agains harmful technologies, all three participants mentioned in their descriptions various frames other than \texttt{Protecting}, namely: \texttt{Personal\_relationship}, evoked by lexical units such as \textit{traição.n 'betrayal.n'}, and \texttt{Theft}, evoked by \textit{roubo.n 'theft.n'}, as shown in \ref{ex:betrayal}.
\ex.\label{ex:betrayal} Acho que é uma visão negativa, justamente porque me remete muito a questão da \textbf{traição ou roubo}, porque não vejo sentido de você usar uma peça de beleza para poder se esconder de câmeras. (\textit{I think it is a negative perspective [on the future], precisely because it leads me into thinking about the issues of \textbf{betrayal and theft}, because I see no reason for you to wear some beauty product to hide from cameras.})

Therefore, for the participants in the pilot, Privée Mystique presents a pessimistic perspective on the future: one in which cosmetic technology is used for betraying loved ones and committing crimes. Gaze plot data extracted using the eye-tracking glasses (Figure \ref{fig:mystique_gazeplot}), as well as the heatmap generated from the same data (Figure \ref{fig:mystique_heatmap}), indicate that participants did explore the image composition and the text in the expected fashion. They focus on the pixelated face, on the name of the foundation and the icon below it and they read the text. Hence, the most likely explanation for the discrepancy in the cosine similarity relates to lack of a shared background story for the interpretation of the ad, which, in turn, led to a base space violation: instead of being conceived of as a protecting device that benefits the person who applies it to their face, the proposed cosmetic product was framed as a mechanism for deceiving law enforcement authorities and loved ones. 

\begin{table}[t]
\begin{center} 
\caption{Cosine similarities between semantic representations of stimuli and participants' descriptions.} 
\label{tbl:cosine} 
\vskip 0.12in
\begin{tabular}{lllll} 
\hline
Stimulus               &  PA1  & PA2   & PA3   & AVG \\
\hline
Near-Fut Optimistic 1  & 0.519 & 0.532 & 0.494 & 0.515 \\
Near-Fut Optimistic 2  & 0.386 & 0.377 & 0.342 & 0.369 \\
Near-Fut Pessimistic 1 & 0.399 & 0.401 & 0.556 & 0.452 \\
Near-Fut Pessimistic 2 & 0.214 & 0.341 & 0.492 & 0.349 \\
Far-Fut Optimistic 1   & 0.262 & 0.354 & 0.408 & 0.341 \\
Far-Fut Optimistic 2   & 0.317 & 0.288 & 0.319 & 0.308 \\
Far-Fut Pessimistic 1  & 0.381 & 0.288 & 0.404 & 0.358 \\
Far-Fut Pessimistic 2  & 0.342 & 0.265 & 0.390 & 0.332 \\
\hline
\end{tabular} 
\end{center} 
\end{table}

\begin{figure}[h]
\begin{center}
\includegraphics[width=\columnwidth]{figures/mystique_gazeplot.png}
\end{center}
\caption{PA1 gaze plot for the Privée Mystique ad for the first 15 seconds.} 
\label{fig:mystique_gazeplot}
\end{figure}

\begin{figure}[h]
\begin{center}
\includegraphics[width=\columnwidth]{figures/mystique_heatmap.png}
\end{center}
\caption{Heatmap for the Privée Mystique ad based on all three participants for the first 60 seconds.} 
\label{fig:mystique_heatmap}
\end{figure}

\subsection{Eye-Tracking insights into cognitive load}

To examine the cognitive effort involved in processing future scenarios, we conducted a series of statistical analyses integrating eye-tracking metrics, multimodal similarity of semantic representations, and behavioral responses. These analyses were designed to evaluate whether base space violations influence cognitive load and conceptual reframing. Our approach consisted of four key components:

\begin{enumerate}
\item \textbf{Mediation analysis:}
cognitive load may serve as an explanatory mechanism for differences in how future scenarios are interpreted. We tested whether pupil dilation—an indicator of autonomic cognitive effort—mediates the relationship between future scenario distance and semantic similarity scores. This analysis helps determine whether increased cognitive effort accounts for interpretative divergence between near and far-future scenarios.

A mediation analysis was conducted to examine whether cognitive load, operationalized as pupil dilation, mediated the relationship between future distance (near vs. distant) and semantic similarity scores. 

\begin{table}[h]
\centering
\caption{Pupil dilation mediates cognitive load}
\label{tbl:mediation}
\begin{tabular}{lccc}
\hline
Predictor & Estimate ($\beta$) & 95\% CI & $p$-value \\
\hline
Fut. Dist. & -0.21 & [-0.38, -0.04] & 0.024 \\
Pup. Dilat. & -0.14 & [-0.29, 0.01] & 0.08 \\
Intercept & 0.45 & [0.20, 0.70] & 0.002 \\
\hline
\end{tabular}
\end{table}

The results indicated:

\begin{itemize}
\item Future distance was a significant predictor of pupil dilation ($\beta = -0.21$, $p < 0.05$, 95\% CI = [-0.38, -0.04]), confirming that distant future scenarios led to lower autonomic engagement.
\item When pupil dilation was included as a mediator, the effect of future distance on semantic similarity decreased ($\beta = -0.14$, $p = 0.08$, 95\% CI = [-0.29, 0.01]), suggesting partial mediation.
\item The mediation model explained 23\% of the variance in semantic similarity scores ($R^2 = 0.23$), reinforcing the role of cognitive effort in interpretative divergence.
\end{itemize}

This result suggests that when cognitive effort is high, as indexed by increased pupil dilation, participants struggle to construct mental spaces that align with the intended valence of future scenarios. The fact that mediation is partial rather than full indicates that additional cognitive mechanisms, such as prior knowledge and linguistic framing, may also play a role in shaping interpretation.

\item \textbf{Cluster analysis:} not all participants process counterfactual information in the same way. To capture variability in gaze behavior, we applied K-means and hierarchical clustering to fixation duration and saccadic variability. This allowed us to identify distinct cognitive processing strategies and assess whether certain gaze patterns are systematically linked to increased cognitive effort when engaging with distant and pessimistic future scenarios. Hierarchical clustering confirmed the robustness of the initial K-means results, revealing two stable clusters. 

Cluster 1, characterized by low cognitive load, showed shorter fixations ($\beta = -0.19$, $p = 0.041$, $R^2 = 0.27$) and fewer saccades, with gaze concentrated on textual elements. These participants efficiently extracted relevant information and engaged in faster conceptual alignment with the stimuli. In contrast, Cluster 2, defined by high cognitive load, exhibited longer fixations ($\beta = -0.19$, $p = 0.041$, $R^2 = 0.27$) and erratic saccades, with frequent gaze shifts between AOIs, particularly in far-future pessimistic scenarios. These findings suggest that individuals experiencing greater cognitive strain are more likely to engage in exploratory gaze patterns, indicating difficulty in reconciling the scenario with prior knowledge.

\item \textbf{Markov Chain analysis of AOI transitions:}
fixation behavior alone does not reveal how participants navigate multimodal stimuli. To assess how participants transition between text and images, we applied a Markov chain model to gaze shifts between AOIs. This method enables us to quantify regressions (back-and-forth shifts indicative of processing difficulty) and compare transition patterns between different clusters of gaze behavior.

To further analyze cognitive effort, a Markov chain model was applied to participants' gaze transitions between AOIs (as summarized in Table 3 below). 

\begin{table}[h]
\centering
\caption{Gaze shifts between AOIs}
\label{tbl:aoi_transitions}
\begin{tabular}{lcc}
\hline
AOI Transition & Cluster 1 & Cluster 2 \\
\hline
Text → Image & 0.25 & 0.48 \\
Image → Text & 0.30 & 0.52 \\
Image → Screen & 0.20 & 0.38 \\
Screen → Image & 0.25 & 0.62 \\
\hline
\end{tabular}
\end{table}

The analysis revealed that:

\begin{itemize}
\item Participants in Cluster 2 made frequent back-and-forth transitions between text and images, indicating difficulty integrating conceptual information.
\item In contrast, Cluster 1 participants showed smoother transitions, suggesting more efficient conceptual integration.
\end{itemize}

This pattern suggests that high cognitive load is associated with increased reliance on visual re-exploration, potentially due to difficulties in forming a coherent conceptual representation of the future scenario.

\item \textbf{Interaction effects and individual differences:}
if scenario reframing is influenced by cognitive load, fixation duration may interact with scenario type to predict conceptual reinterpretation. We conducted regression analyses to test whether gaze behavior interacts with scenario valence and future distance to influence reframing patterns. Additionally, we explored whether individual differences in cognitive flexibility correlate with pupil dilation and fixation patterns, providing insight into how prior cognitive adaptability influences scenario comprehension.

A regression model was conducted to examine the interaction between fixation duration and scenario reframing, as summarised in Table 4 (see below).

\begin{table}[h]
\centering
\caption{Interaction between fixations and reframing}
\label{tbl:interaction}
\begin{tabular}{lccc}
\hline
Predictor & Estimate ($\beta$) & 95\% CI & $p$-value \\
\hline
Fut. Dist. & -0.27 & [-0.52, -0.02] & 0.048 \\
Fix. Durat. & -0.19 & [-0.35, -0.03] & 0.041 \\
Intercept & 0.45 & [0.06, 0.84] & 0.03 \\
\hline
\end{tabular}
\end{table}

The results indicated that future distance significantly interacted with fixation duration ($\beta = -0.19$, $p = 0.041$, $R^2 = 0.27$), suggesting that participants with longer fixations in far-future scenarios were more likely to reframe the scenario. The 95\% confidence interval for the regression estimate was [-0.35, -0.03], reinforcing the significance of the interaction. This reinforces the hypothesis that base space violations disrupt automatic conceptual alignment, leading to increased fixation durations and a greater likelihood of conceptual reframing. Longer fixations reflect delayed comprehension, while scenario reframing suggests an effort to reconstruct coherence when the original interpretation is incompatible with prior knowledge.

\end{enumerate}


\section{Discussion}

The findings align with theories of cognitive effort in mental space construction \cite{fauconnier1997mappings, Fillmore1982}, supporting the idea that base space violations increase cognitive load. Specifically, the longer fixation durations observed for pessimistic and far-future scenarios may reflect the greater effort required to reconcile these scenarios with existing mental frameworks. This observation is consistent with prior evidence showing that mental effort and counterfactual scenarios modulate language understanding, as demonstrated through ERP findings in older adults \cite{SalasHerrera2024}. 

Building on this, the results highlight the importance of base space persistence and fracture in counterfactual reasoning and future-oriented communication, with cognitive load theories explaining increased fixation durations in far-future and pessimistic scenarios. The observed differences in AOI transition probabilities suggest that higher cognitive load leads to more frequent regressions, indicating cognitive instability when processing counterfactual elements. This supports the concept of iterative mental space revisions. Additionally, pupil dilation serves as an index of cognitive effort, emphasizing the role of cognitive load in interpretation variability, with the overall findings suggesting that processing distant or counterfactual future scenarios demands more cognitive resources.

\section{Limitations}

This pilot study aimed to assess the feasibility of combining eye-tracking and multimodal semantic annotation methods to explore how cognitive effort varies with counterfactual and valence-based stimuli. As an exploratory effort, the goal was to test the FutureVision methodology and refine hypotheses for larger investigations. The results demonstrated the potential of combining gaze-tracking data and frame semantics for analyzing cognitive processes. While the experimental setup, including the use of Tobii Pro Glasses 3, successfully captured eye movement metrics relevant to cognitive load, challenges in participant recruitment and task engagement highlighted areas for refinement. A key limitation was the small sample size, restricting the generalizability of the findings. Although the sample was adequate for testing feasibility, larger and more diverse populations are needed for confirmation. The study focused on a limited set of variables and did not account for potential confounding factors like individual differences in working memory or cultural influences. Despite these limitations, the study demonstrated the feasibility of the experimental paradigm and identified promising avenues for research on future cognition.

\section{Conclusion}
Our central finding that the FutureVision methodology allows for assessing the impact of base space violations on cognitive load is an important contribution to several fields of reasearch where future cognition plays a central role. Beyond its contributions to Scenario Forecasting, Speculative Futures and Quantitative Futurism, the methodology presented in this paper can inform research in other fields such as human-computer interaction, cognitive neuroscience, and clinical decision-making. Understanding how individuals process counterfactual and valence-based scenarios can support the development of adaptive systems that respond to users' cognitive states, such as virtual reality environments for cognitive training or rehabilitation. 

\section{Data and code availability}
Data and scripts are available via \texttt{https://osf.io/9kvu2}

\section{Acknowledgments}

TTT is a Research Productivity Grantee of the Brazilian National Council for Scientific and Technological Development (CNPq, 315749/2021-0). NH was supported by the Brazilian National Council for Scientific and Technological Development (CNPq, 420360/2022-0). FB was supported by the Brazilian National Council for Scientific and Technological Development (CNPq, 200270/2023-0). IL is a Research Productivity Grantee of the Brazilian National Council for Scientific and Technological Development (CNPq, 316193/2023-2). ALA was supported by the Minas Gerais State Research Support Foundation (FAPEMIG, RED-00106-21/09604293605). MV was supported by the Brazilian National Council for Scientific and Technological Development (CNPq, 201299/2024-0).

\bibliographystyle{apacite}

\setlength{\bibleftmargin}{.125in}
\setlength{\bibindent}{-\bibleftmargin}

\bibliography{CogSci_Template}


\end{document}
