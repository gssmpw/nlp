% if you use cleveref..

%%%%%%%%%%%%%%%%%%%%%%%%%%%%%%%%
% THEOREMS
%%%%%%%%%%%%%%%%%%%%%%%%%%%%%%%%



\usepackage[accepted]{icml2025} %% Babak: put [accepted] back once accepted 

\usepackage{hyperref}
% Recommended, but optional, packages for figures and better typesetting:
\usepackage{pgfplots}
\pgfplotsset{compat=1.18}  % Use the latest version
\usepackage{tikz}
\usepackage[many]{tcolorbox}
\usepackage{xcolor}
\usepackage{amsmath}
\usepackage[capitalize,noabbrev]{cleveref}

\usepackage{microtype}
\usepackage{graphicx}
\usepackage{subfigure}
\usepackage{booktabs} % for professional tables

\DeclareMathOperator*{\argmax}{arg\,max}
\usepackage{amssymb}
\usepackage{mathtools}
\usepackage{amsthm}
\usepackage{xspace} 
\usepackage{mdframed}
\usepackage{xcolor}

\usepackage[textsize=tiny]{todonotes}
% Attempt to make hyperref and algorithmic work together better:
\newcommand{\theHalgorithm}{\arabic{algorithm}}

% Use the following line for the initial blind version submitted for review:


% If accepted, instead use the following line for the camera-ready submission:
% \usepackage[accepted]{icml2025}

% For theorems and such


\theoremstyle{plain}
\newtheorem{theorem}{Theorem}
\newtheorem{proposition}[theorem]{Proposition}
\newtheorem{lemma}[theorem]{Lemma}
\newtheorem{corollary}[theorem]{Corollary}

\theoremstyle{definition}
\newtheorem{definition}{Definition}
\newtheorem{assumption}[definition]{Assumption}

\theoremstyle{remark}
\newtheorem{remark}[theorem]{Remark}

% \theoremstyle{plain}
% \newtheorem{theorem}{Theorem}[section]
% \newtheorem{proposition}[theorem]{Proposition}
% \newtheorem{lemma}[theorem]{Lemma}
% \newtheorem{corollary}[theorem]{Corollary}
% \theoremstyle{definition}
% \newtheorem{definition}[theorem]{Definition}
% \newtheorem{assumption}[theorem]{Assumption}
% \theoremstyle{remark}
% \newtheorem{remark}[theorem]{Remark}

%%%%%%%%%%%%%%%%%%%%%%%%%%%%%%
% Macros for Notation
%%%%%%%%%%%%%%%%%%%%%%%%%%%%%%
\newcommand{\sys}{\textsc{TokenSwap}\xspace}
\newcommand{\pmain}{\mathbf{p}^{\text{main}}}       % Main model distribution
\newcommand{\paux}{\mathbf{p}^{\text{aux}}}         % Auxiliary model distribution
\newcommand{\pfinal}{\mathbf{p}^{\text{final}}}   
\newcommand{\ptrue}{\mathbf{p}^{\text{true}}} % Final distribution after combination
\newcommand{\gramset}{\mathcal{G}}
                 % Grammar subset

\mdfdefinestyle{exampleframe}{%
    linecolor=gray!30,
    linewidth=0.5pt,
    backgroundcolor=gray!5,
    roundcorner=2pt,
    innertopmargin=8pt,
    innerbottommargin=8pt,
    innerleftmargin=8pt,
    innerrightmargin=8pt,
    skipabove=\baselineskip,
    skipbelow=\baselineskip
}

% Optional: Custom colors for different sections
\definecolor{prefixcolor}{RGB}{0,0,0}
\definecolor{standardcolor}{RGB}{100,100,100}
\definecolor{tokenswapcolor}{RGB}{0,90,180}

% Optional: Custom commands for consistent formatting
\newcommand{\generationexample}[4]{%
    \begin{mdframed}[style=exampleframe]
        #1 #2 #3 #4
    \end{mdframed}
}
  
    
 