\documentclass[journal]{IEEEtran}

\ifCLASSINFOpdf
\else
   \usepackage[dvips]{graphicx}
\fi
\usepackage{url}

\hyphenation{op-tical net-works semi-conduc-tor}

\usepackage{graphicx}
\usepackage{booktabs}
\usepackage[table,xcdraw]{xcolor}
\usepackage{colortbl}
\definecolor{Gray}{gray}{0.9}
\usepackage{cite}           % compressed citation lists
\usepackage{amsmath}
\usepackage{amssymb}% http://ctan.org/pkg/amssymb
\usepackage{pifont}% http://ctan.org/pkg/pifont
\newcommand{\cmark}{\ding{51}}%
\newcommand{\xmark}{\ding{55}}%
\usepackage{multirow}
\usepackage[caption=false]{subfig}
\usepackage{hyperref}
\definecolor{blue}{RGB}{0,0,255}

\begin{document}

\title{mWhisper-Flamingo for Multilingual Audio-Visual Noise-Robust Speech Recognition}

\author{Andrew Rouditchenko,
        Samuel Thomas,~\IEEEmembership{Senior Member, IEEE},
        Hilde Kuehne,~\IEEEmembership{Member, IEEE},
        Rogerio Feris,~\IEEEmembership{Senior Member, IEEE},
        and James Glass,~\IEEEmembership{Fellow, IEEE}
\thanks{Code at \href{https://github.com/roudimit/whisper-flamingo}%{\texttt{\color{blue}{https://github.com/roudimit/whisper-flamingo}}}. 
{\texttt{https://github.com/roudimit/whisper-flamingo}}.
Andrew Rouditchenko and James Glass are with MIT, USA (e-mail: \{roudi, glass\}@mit.edu).
Samuel Thomas and Rogerio
Feris are with MIT-IBM Watson AI Lab, USA.
Hilde Kuehne is with University of Tuebingen, DE.
% A.R. and J.G. are with MIT, USA (e-mail: \{roudi, glass\}@mit.edu).
% S.T. and R.F. are with MIT-IBM Watson AI Lab, USA.
% H.K. is with University of Tuebingen, DE.
We thank Videet Mehta for help with the demo and Tatiana Likhomanenko and Saurabhchand Bhati for helpful discussions. This research was supported by MIT-IBM Watson AI Lab and an NDSEG Fellowship to A.R.}}

% \markboth{Journal of \LaTeX\ Class Files, Vol. 14, No. 8, August 2015}
% \markboth{IEEE Signal Processing Letters}
\markboth{Preprint}
{Shell \MakeLowercase{\textit{et al.}}: Bare Demo of IEEEtran.cls for IEEE Journals}
\maketitle

\begin{abstract}
Audio-Visual Speech Recognition (AVSR) combines lip-based video with audio and can improve performance in noise, but most methods are trained only on English data.
One limitation is the lack of large-scale multilingual video data, which makes it hard hard to train models from scratch.
In this work, we propose mWhisper-Flamingo for multilingual AVSR which combines the strengths of a pre-trained audio model (Whisper) and video model (AV-HuBERT).
To enable better multi-modal integration and improve the noisy multilingual performance, we introduce decoder modality dropout where the model is trained both on paired audio-visual inputs and separate audio/visual inputs.
mWhisper-Flamingo achieves state-of-the-art WER on MuAViC, an AVSR dataset of 9 languages.
Audio-visual mWhisper-Flamingo consistently outperforms audio-only Whisper on all languages in noisy conditions.
\end{abstract}

\begin{IEEEkeywords}
audio-visual speech recognition, multilingual
\end{IEEEkeywords}

\IEEEpeerreviewmaketitle

\section{Introduction}
Automatic Speech Recognition (ASR) has seen great progress thanks to large-scale training~\cite{radford2023robust,puvvada24_interspeech,rouditchenko23_interspeech,shi23g_interspeech}, but models still struggle with background noise~\cite{gong23d_interspeech,hu2024large}.
To improve performance, Audio-Visual Speech Recognition (AVSR) models combine lip-based video with audio inputs~\cite{afouras2018adeep,petridis2018audio,petridis2018end,xu2020discriminative,ma2021end,serdyuk22_interspeech,shi22_interspeech,ma2023auto,burchi2023audio,cappellazzo2024large}.
In particular, Whisper-Flamingo~\cite{rouditchenko24_interspeech} proposed an audio-visual adaptation of Whisper~\cite{radford2023robust}, a pre-trained ASR model, and showed significant improvements in noise robustness compared to the original audio-only model.
Whisper-Flamingo was motivated by the limitation of previous AVSR systems which often lack large-scale transcribed videos and face difficulty training models from scratch on only a few hundred hours of data.
To overcome this, Whisper-Flamingo combines the strength of Whisper's audio encoder and text decoder trained on 680k hours with AV-HuBERT~\cite{shi2022learning}, a pre-trained lip-reading visual encoder.
The model integrates lip-based visual features from AV-HuBERT into Whisper's decoder and achieves State-of-the-Art (SOTA) performance on English AVSR.

In this work, we propose mWhisper-Flamingo: a novel multilingual extension of Whisper-Flamingo which achieves SOTA performance on multilingual AVSR.
mWhisper-Flamingo combines Whisper's strong multilingual audio encoder and text decoder with a new AV-HuBERT visual encoder pre-trained on multilingual videos~\cite{kim_2024}.
Unlike the previous Whisper-Flamingo model which could only take English videos as input, the new mWhisper-Flamingo model can handle videos in 9 different languages (including English).
However, we show that Whisper-Flamingo's default training process applied to multilingual videos yields poor noisy multilingual AVSR performance, despite achieving good English performance.
To address this, we propose a novel decoder modality dropout technique by training the model both on paired audio-visual inputs and separate audio/video inputs.
We show this to be key for good noisy multilingual AVSR performance with a thorough analysis and ablation study.

We test our method across 9 languages in clean and noisy conditions on the MuAViC dataset~\cite{anwar23_interspeech}.
In clean audio conditions, mWhisper-Flamingo outperforms previous audio-visual methods and achieves SOTA across multilingual languages.
In noisy conditions, mWhisper-Flamingo consistently outperforms the audio-only Whisper model on 6 different noise types and 5 levels of noise.
We release our code and models.

\begin{figure}[t]
    \centering
    \includegraphics[width=\linewidth]{figures/mWhisper-Flamingo_fig.pdf}
    \caption{In mWhisper-Flamingo, the AV-HuBERT and Whisper encoders extract visual and audio features from multilingual videos. 
    Separate cross attention layers in Whisper's decoder attend to the visual and audio features.
    Decoder modality dropout randomly replaces the audio or video features by 0, forcing the decoder to train on video-only and audio-only inputs.}
    \label{fig:overview}
    \vspace{-0.4cm}
\end{figure}

\section{Method}
Our method builds upon Whisper-Flamingo~\cite{rouditchenko24_interspeech}, an audio-visual extension of Whisper~\cite{radford2023robust}.
Whisper-Flamingo adds new gated cross-attention layers (originally proposed for the Flamingo vision-language model~\cite{alayrac2022flamingo}) into each of Whisper’s decoder blocks which attend to the visual features from the AV-HuBERT visual encoder~\cite{shi2022learning}.
The layers are initialized as identity functions and initially ignore the visual input, but the weights are adjusted to integrate the visual features for AVSR during training on audio-visual inputs.

Whisper-Flamingo uses two-stages of training.
First, all of Whisper's parameters are fine-tuned on noisy audio inputs, enhancing its domain-specific performance and noise robustness. 
Second, the gated cross-attention layers are initialized and trained on audio-visual data. 
During this stage, all of Whisper's and AV-HuBERT's parameters are frozen to preserve the pre-trained knowledge and to facilitate multi-modal integration.

mWhisper-Flamingo inherits Whisper-Flamingo's architecture, except the English AV-HuBERT~\cite{shi2022learning} is replaced with a version pre-trained on multilingual videos~\cite{kim_2024}.
We propose to train mWhisper-Flamingo on multilingual videos jointly with English, similar to Whisper's training process.
However, using Whisper-Flamingo's default training process on multilingual videos yielded only minor improvements in noisy multilingual AVSR performance, despite achieving significant improvements for English (see Section~\ref{sec:analysis}). 
We suspect this is due to English having on average 13.6x more data than other languages in our dataset.
To address this, we introduce decoder modality dropout by training the model both on paired audio-visual inputs and separate audio/video inputs.


Dropout, originally proposed to prevent overfitting in neural networks~\cite{hinton2012improving}, was extended to multi-modal learning as modality dropout~\cite{neverova2015moddrop}. 
Modality dropout randomly drops input modalities during training to better capture cross-modality correlations while preserving unique contributions of specific modalities.
Modality dropout has been used in AVSR~\cite{makino2019recurrent,shi2022learning,hsu2022u,lian2023av,rouditchenko2023av} to prevent models from over-relying on the audio modality, which is typically easier to transcribe than the lip-based visual modality.
However, current methods mainly use early-fusion where audio and visual features are combined and used as input into a single Transformer~\cite{vaswani2017attention} encoder.
During modality dropout, the Transformer encoder must handle the incomplete input stream and output an embedding sequence which is used as input to the decoder's cross-attention layers.
In contrast, Whisper-Flamingo uses late-fusion where separate Transformer encoders process audio and visual features independently.
Each encoder outputs an embedding sequence used as input to separate audio and video cross attention layers in the decoder, where the modality fusion occurs.
During modality dropout, one of the audio and video embedding sequences is replaced by a 0-vector, forcing the decoder to handle the incomplete input stream in each of the cross-attention layers corresponding to the missing modality.
This enables the encoders to remain specialized on their specific modalities while the decoder learns better multi-modal integration.
This process is shown in Figure~\ref{fig:overview}.
Note that cross-attention with a 0-vector results in a 0-vector, but the output from the layer incorporates the bias in the linear transformations.

Our method is conceptually similar to LayerDrop~\cite{Fan2020Reducing}, which randomly drops Transformer layers for model pruning on text-based tasks. 
However, our focus lies in enhancing the decoder's ability to learn better multi-modal integration and to handle inputs where one modality may be unreliable.

We define the probabilities of using audio-visual (\( p_{AV} \)), audio-only (\( p_{A} \)), and visual-only (\( p_{V} \)) inputs during training as follows. 
If audio-visual inputs are selected, both modalities are used.  
If audio-only inputs are selected, visual features are zeroed out at the decoder.  
If video-only inputs are selected, audio features are zeroed out at the decoder.  
In Section~\ref{sec:analysis}, we show results using different probabilities.
The setting \( p_{AV} = 0.5,\ p_{A} = 0,\ p_{V} = 0.5 \) performs the best on noisy multilingual AVSR, so we use it for training our main models.

\section{Experiments}

\subsection{Experimental Setup}

We use the MuAViC~\cite{anwar23_interspeech} dataset of 1,141h of videos in 9 languages.
The dataset is based on the LRS3 English video dataset~\cite{afouras2018lrs3} and mTedX dataset~\cite{salesky21_interspeech}.
The hours of video per language are: 
English (En): 433, Arabic (Ar): 16, German (De): 10, Greek (El): 25, Spanish (Es): 178, French (Fr): 176, Italian (It): 101, Portuguese (Pt): 153 and Russian (Ru): 49.

We use Whisper~\cite{radford2023robust} small, medium, and large-v2 with 244M, 769M, and 1.55B parameters.
We fine-tuned Whisper small and medium on 4 A6000 GPUs with 48GB memory, but could not fine-tune Whisper large due to the GPU memory limits.
We use AV-HuBERT pre-trained on unlabeled multilingual videos~\cite{kim_2024} with 325M parameters.
We selected this model instead of other English-only lip-reading models~\cite{haliassos2023jointly,haliassos2024braven,haliassos2024unified} due to its superiority on multilingual videos~\cite{ma2022visual,zinonos2023learning,kim2023lip,yeo2024visual}.
Different to Whisper-Flamingo, we also fine-tune its parameters.
The gated cross attention layers add 82M and 296M for the small and medium models, bringing the total to 651M and 1.39B parameters for mWhisper-Flamingo small / medium. 
The other dataloading details and hyperparameters closely follow Whisper-Flamingo~\cite{rouditchenko2023av}.
Spectrogram frames are used as input to Whisper at 100 Hz while grayscale videos are used as input to AV-HuBERT at 25 fps.
The videos are cropped on the lips using Dlib~\cite{king2009dlib} and are aligned to a reference mean face~\cite{martinez2020lipreading}.
Models were trained using PyTorch~\cite{paszke2019pytorch} and PyTorch Lightning~\cite{Falcon_PyTorch_Lightning_2019} with the AdamW optimizer~\cite{loshchilov2018decoupled}.

During training, we randomly add noise to the audio with a Signal-to-Noise Ratio (SNR) of 0 dB. 
Based on prior work~\cite{shi22_interspeech,anwar23_interspeech}, ``natural'', ``music'' and ``babble'' noise are sampled from the MUSAN dataset~\cite{snyder2015musan}, and overlapping ``speech'' is sampled from LRS3~\cite{afouras2018lrs3}.
We monitor the token prediction accuracy on the noisy validation set every 1k steps to select the best checkpoints.
We normalize the text by lower-casing and removing punctuation except single apostrophes.
Our goal is to improve the multilingual Word Error Rate (WER), so we compute average WER on all languages except English. 
Given the data imbalance, we also separately compute the average on the ``higher resource'' languages with over 100h of data (Es, Fr, It, Pt), and the other ``lower resource'' languages (Ar, De, El, Ru) with as low as 10h of data.

% This must be in the first 5 lines to tell arXiv to use pdfLaTeX, which is strongly recommended.
\pdfoutput=1
% In particular, the hyperref package requires pdfLaTeX in order to break URLs across lines.

\documentclass[11pt]{article}

% Change "review" to "final" to generate the final (sometimes called camera-ready) version.
% Change to "preprint" to generate a non-anonymous version with page numbers.
\usepackage{acl}

% Standard package includes
\usepackage{times}
\usepackage{latexsym}

% Draw tables
\usepackage{booktabs}
\usepackage{multirow}
\usepackage{xcolor}
\usepackage{colortbl}
\usepackage{array} 
\usepackage{amsmath}

\newcolumntype{C}{>{\centering\arraybackslash}p{0.07\textwidth}}
% For proper rendering and hyphenation of words containing Latin characters (including in bib files)
\usepackage[T1]{fontenc}
% For Vietnamese characters
% \usepackage[T5]{fontenc}
% See https://www.latex-project.org/help/documentation/encguide.pdf for other character sets
% This assumes your files are encoded as UTF8
\usepackage[utf8]{inputenc}

% This is not strictly necessary, and may be commented out,
% but it will improve the layout of the manuscript,
% and will typically save some space.
\usepackage{microtype}
\DeclareMathOperator*{\argmax}{arg\,max}
% This is also not strictly necessary, and may be commented out.
% However, it will improve the aesthetics of text in
% the typewriter font.
\usepackage{inconsolata}

%Including images in your LaTeX document requires adding
%additional package(s)
\usepackage{graphicx}
% If the title and author information does not fit in the area allocated, uncomment the following
%
%\setlength\titlebox{<dim>}
%
% and set <dim> to something 5cm or larger.

\title{Wi-Chat: Large Language Model Powered Wi-Fi Sensing}

% Author information can be set in various styles:
% For several authors from the same institution:
% \author{Author 1 \and ... \and Author n \\
%         Address line \\ ... \\ Address line}
% if the names do not fit well on one line use
%         Author 1 \\ {\bf Author 2} \\ ... \\ {\bf Author n} \\
% For authors from different institutions:
% \author{Author 1 \\ Address line \\  ... \\ Address line
%         \And  ... \And
%         Author n \\ Address line \\ ... \\ Address line}
% To start a separate ``row'' of authors use \AND, as in
% \author{Author 1 \\ Address line \\  ... \\ Address line
%         \AND
%         Author 2 \\ Address line \\ ... \\ Address line \And
%         Author 3 \\ Address line \\ ... \\ Address line}

% \author{First Author \\
%   Affiliation / Address line 1 \\
%   Affiliation / Address line 2 \\
%   Affiliation / Address line 3 \\
%   \texttt{email@domain} \\\And
%   Second Author \\
%   Affiliation / Address line 1 \\
%   Affiliation / Address line 2 \\
%   Affiliation / Address line 3 \\
%   \texttt{email@domain} \\}
% \author{Haohan Yuan \qquad Haopeng Zhang\thanks{corresponding author} \\ 
%   ALOHA Lab, University of Hawaii at Manoa \\
%   % Affiliation / Address line 2 \\
%   % Affiliation / Address line 3 \\
%   \texttt{\{haohany,haopengz\}@hawaii.edu}}
  
\author{
{Haopeng Zhang$\dag$\thanks{These authors contributed equally to this work.}, Yili Ren$\ddagger$\footnotemark[1], Haohan Yuan$\dag$, Jingzhe Zhang$\ddagger$, Yitong Shen$\ddagger$} \\
ALOHA Lab, University of Hawaii at Manoa$\dag$, University of South Florida$\ddagger$ \\
\{haopengz, haohany\}@hawaii.edu\\
\{yiliren, jingzhe, shen202\}@usf.edu\\}



  
%\author{
%  \textbf{First Author\textsuperscript{1}},
%  \textbf{Second Author\textsuperscript{1,2}},
%  \textbf{Third T. Author\textsuperscript{1}},
%  \textbf{Fourth Author\textsuperscript{1}},
%\\
%  \textbf{Fifth Author\textsuperscript{1,2}},
%  \textbf{Sixth Author\textsuperscript{1}},
%  \textbf{Seventh Author\textsuperscript{1}},
%  \textbf{Eighth Author \textsuperscript{1,2,3,4}},
%\\
%  \textbf{Ninth Author\textsuperscript{1}},
%  \textbf{Tenth Author\textsuperscript{1}},
%  \textbf{Eleventh E. Author\textsuperscript{1,2,3,4,5}},
%  \textbf{Twelfth Author\textsuperscript{1}},
%\\
%  \textbf{Thirteenth Author\textsuperscript{3}},
%  \textbf{Fourteenth F. Author\textsuperscript{2,4}},
%  \textbf{Fifteenth Author\textsuperscript{1}},
%  \textbf{Sixteenth Author\textsuperscript{1}},
%\\
%  \textbf{Seventeenth S. Author\textsuperscript{4,5}},
%  \textbf{Eighteenth Author\textsuperscript{3,4}},
%  \textbf{Nineteenth N. Author\textsuperscript{2,5}},
%  \textbf{Twentieth Author\textsuperscript{1}}
%\\
%\\
%  \textsuperscript{1}Affiliation 1,
%  \textsuperscript{2}Affiliation 2,
%  \textsuperscript{3}Affiliation 3,
%  \textsuperscript{4}Affiliation 4,
%  \textsuperscript{5}Affiliation 5
%\\
%  \small{
%    \textbf{Correspondence:} \href{mailto:email@domain}{email@domain}
%  }
%}

\begin{document}
\maketitle
\begin{abstract}
Recent advancements in Large Language Models (LLMs) have demonstrated remarkable capabilities across diverse tasks. However, their potential to integrate physical model knowledge for real-world signal interpretation remains largely unexplored. In this work, we introduce Wi-Chat, the first LLM-powered Wi-Fi-based human activity recognition system. We demonstrate that LLMs can process raw Wi-Fi signals and infer human activities by incorporating Wi-Fi sensing principles into prompts. Our approach leverages physical model insights to guide LLMs in interpreting Channel State Information (CSI) data without traditional signal processing techniques. Through experiments on real-world Wi-Fi datasets, we show that LLMs exhibit strong reasoning capabilities, achieving zero-shot activity recognition. These findings highlight a new paradigm for Wi-Fi sensing, expanding LLM applications beyond conventional language tasks and enhancing the accessibility of wireless sensing for real-world deployments.
\end{abstract}

\section{Introduction}

In today’s rapidly evolving digital landscape, the transformative power of web technologies has redefined not only how services are delivered but also how complex tasks are approached. Web-based systems have become increasingly prevalent in risk control across various domains. This widespread adoption is due their accessibility, scalability, and ability to remotely connect various types of users. For example, these systems are used for process safety management in industry~\cite{kannan2016web}, safety risk early warning in urban construction~\cite{ding2013development}, and safe monitoring of infrastructural systems~\cite{repetto2018web}. Within these web-based risk management systems, the source search problem presents a huge challenge. Source search refers to the task of identifying the origin of a risky event, such as a gas leak and the emission point of toxic substances. This source search capability is crucial for effective risk management and decision-making.

Traditional approaches to implementing source search capabilities into the web systems often rely on solely algorithmic solutions~\cite{ristic2016study}. These methods, while relatively straightforward to implement, often struggle to achieve acceptable performances due to algorithmic local optima and complex unknown environments~\cite{zhao2020searching}. More recently, web crowdsourcing has emerged as a promising alternative for tackling the source search problem by incorporating human efforts in these web systems on-the-fly~\cite{zhao2024user}. This approach outsources the task of addressing issues encountered during the source search process to human workers, leveraging their capabilities to enhance system performance.

These solutions often employ a human-AI collaborative way~\cite{zhao2023leveraging} where algorithms handle exploration-exploitation and report the encountered problems while human workers resolve complex decision-making bottlenecks to help the algorithms getting rid of local deadlocks~\cite{zhao2022crowd}. Although effective, this paradigm suffers from two inherent limitations: increased operational costs from continuous human intervention, and slow response times of human workers due to sequential decision-making. These challenges motivate our investigation into developing autonomous systems that preserve human-like reasoning capabilities while reducing dependency on massive crowdsourced labor.

Furthermore, recent advancements in large language models (LLMs)~\cite{chang2024survey} and multi-modal LLMs (MLLMs)~\cite{huang2023chatgpt} have unveiled promising avenues for addressing these challenges. One clear opportunity involves the seamless integration of visual understanding and linguistic reasoning for robust decision-making in search tasks. However, whether large models-assisted source search is really effective and efficient for improving the current source search algorithms~\cite{ji2022source} remains unknown. \textit{To address the research gap, we are particularly interested in answering the following two research questions in this work:}

\textbf{\textit{RQ1: }}How can source search capabilities be integrated into web-based systems to support decision-making in time-sensitive risk management scenarios? 
% \sq{I mention ``time-sensitive'' here because I feel like we shall say something about the response time -- LLM has to be faster than humans}

\textbf{\textit{RQ2: }}How can MLLMs and LLMs enhance the effectiveness and efficiency of existing source search algorithms? 

% \textit{\textbf{RQ2:}} To what extent does the performance of large models-assisted search align with or approach the effectiveness of human-AI collaborative search? 

To answer the research questions, we propose a novel framework called Auto-\
S$^2$earch (\textbf{Auto}nomous \textbf{S}ource \textbf{Search}) and implement a prototype system that leverages advanced web technologies to simulate real-world conditions for zero-shot source search. Unlike traditional methods that rely on pre-defined heuristics or extensive human intervention, AutoS$^2$earch employs a carefully designed prompt that encapsulates human rationales, thereby guiding the MLLM to generate coherent and accurate scene descriptions from visual inputs about four directional choices. Based on these language-based descriptions, the LLM is enabled to determine the optimal directional choice through chain-of-thought (CoT) reasoning. Comprehensive empirical validation demonstrates that AutoS$^2$-\ 
earch achieves a success rate of 95–98\%, closely approaching the performance of human-AI collaborative search across 20 benchmark scenarios~\cite{zhao2023leveraging}. 

Our work indicates that the role of humans in future web crowdsourcing tasks may evolve from executors to validators or supervisors. Furthermore, incorporating explanations of LLM decisions into web-based system interfaces has the potential to help humans enhance task performance in risk control.






\section{Related Work}
\label{sec:relatedworks}

% \begin{table*}[t]
% \centering 
% \renewcommand\arraystretch{0.98}
% \fontsize{8}{10}\selectfont \setlength{\tabcolsep}{0.4em}
% \begin{tabular}{@{}lc|cc|cc|cc@{}}
% \toprule
% \textbf{Methods}           & \begin{tabular}[c]{@{}c@{}}\textbf{Training}\\ \textbf{Paradigm}\end{tabular} & \begin{tabular}[c]{@{}c@{}}\textbf{$\#$ PT Data}\\ \textbf{(Tokens)}\end{tabular} & \begin{tabular}[c]{@{}c@{}}\textbf{$\#$ IFT Data}\\ \textbf{(Samples)}\end{tabular} & \textbf{Code}  & \begin{tabular}[c]{@{}c@{}}\textbf{Natural}\\ \textbf{Language}\end{tabular} & \begin{tabular}[c]{@{}c@{}}\textbf{Action}\\ \textbf{Trajectories}\end{tabular} & \begin{tabular}[c]{@{}c@{}}\textbf{API}\\ \textbf{Documentation}\end{tabular}\\ \midrule 
% NexusRaven~\citep{srinivasan2023nexusraven} & IFT & - & - & \textcolor{green}{\CheckmarkBold} & \textcolor{green}{\CheckmarkBold} &\textcolor{red}{\XSolidBrush}&\textcolor{red}{\XSolidBrush}\\
% AgentInstruct~\citep{zeng2023agenttuning} & IFT & - & 2k & \textcolor{green}{\CheckmarkBold} & \textcolor{green}{\CheckmarkBold} &\textcolor{red}{\XSolidBrush}&\textcolor{red}{\XSolidBrush} \\
% AgentEvol~\citep{xi2024agentgym} & IFT & - & 14.5k & \textcolor{green}{\CheckmarkBold} & \textcolor{green}{\CheckmarkBold} &\textcolor{green}{\CheckmarkBold}&\textcolor{red}{\XSolidBrush} \\
% Gorilla~\citep{patil2023gorilla}& IFT & - & 16k & \textcolor{green}{\CheckmarkBold} & \textcolor{green}{\CheckmarkBold} &\textcolor{red}{\XSolidBrush}&\textcolor{green}{\CheckmarkBold}\\
% OpenFunctions-v2~\citep{patil2023gorilla} & IFT & - & 65k & \textcolor{green}{\CheckmarkBold} & \textcolor{green}{\CheckmarkBold} &\textcolor{red}{\XSolidBrush}&\textcolor{green}{\CheckmarkBold}\\
% LAM~\citep{zhang2024agentohana} & IFT & - & 42.6k & \textcolor{green}{\CheckmarkBold} & \textcolor{green}{\CheckmarkBold} &\textcolor{green}{\CheckmarkBold}&\textcolor{red}{\XSolidBrush} \\
% xLAM~\citep{liu2024apigen} & IFT & - & 60k & \textcolor{green}{\CheckmarkBold} & \textcolor{green}{\CheckmarkBold} &\textcolor{green}{\CheckmarkBold}&\textcolor{red}{\XSolidBrush} \\\midrule
% LEMUR~\citep{xu2024lemur} & PT & 90B & 300k & \textcolor{green}{\CheckmarkBold} & \textcolor{green}{\CheckmarkBold} &\textcolor{green}{\CheckmarkBold}&\textcolor{red}{\XSolidBrush}\\
% \rowcolor{teal!12} \method & PT & 103B & 95k & \textcolor{green}{\CheckmarkBold} & \textcolor{green}{\CheckmarkBold} & \textcolor{green}{\CheckmarkBold} & \textcolor{green}{\CheckmarkBold} \\
% \bottomrule
% \end{tabular}
% \caption{Summary of existing tuning- and pretraining-based LLM agents with their training sample sizes. "PT" and "IFT" denote "Pre-Training" and "Instruction Fine-Tuning", respectively. }
% \label{tab:related}
% \end{table*}

\begin{table*}[ht]
\begin{threeparttable}
\centering 
\renewcommand\arraystretch{0.98}
\fontsize{7}{9}\selectfont \setlength{\tabcolsep}{0.2em}
\begin{tabular}{@{}l|c|c|ccc|cc|cc|cccc@{}}
\toprule
\textbf{Methods} & \textbf{Datasets}           & \begin{tabular}[c]{@{}c@{}}\textbf{Training}\\ \textbf{Paradigm}\end{tabular} & \begin{tabular}[c]{@{}c@{}}\textbf{\# PT Data}\\ \textbf{(Tokens)}\end{tabular} & \begin{tabular}[c]{@{}c@{}}\textbf{\# IFT Data}\\ \textbf{(Samples)}\end{tabular} & \textbf{\# APIs} & \textbf{Code}  & \begin{tabular}[c]{@{}c@{}}\textbf{Nat.}\\ \textbf{Lang.}\end{tabular} & \begin{tabular}[c]{@{}c@{}}\textbf{Action}\\ \textbf{Traj.}\end{tabular} & \begin{tabular}[c]{@{}c@{}}\textbf{API}\\ \textbf{Doc.}\end{tabular} & \begin{tabular}[c]{@{}c@{}}\textbf{Func.}\\ \textbf{Call}\end{tabular} & \begin{tabular}[c]{@{}c@{}}\textbf{Multi.}\\ \textbf{Step}\end{tabular}  & \begin{tabular}[c]{@{}c@{}}\textbf{Plan}\\ \textbf{Refine}\end{tabular}  & \begin{tabular}[c]{@{}c@{}}\textbf{Multi.}\\ \textbf{Turn}\end{tabular}\\ \midrule 
\multicolumn{13}{l}{\emph{Instruction Finetuning-based LLM Agents for Intrinsic Reasoning}}  \\ \midrule
FireAct~\cite{chen2023fireact} & FireAct & IFT & - & 2.1K & 10 & \textcolor{red}{\XSolidBrush} &\textcolor{green}{\CheckmarkBold} &\textcolor{green}{\CheckmarkBold}  & \textcolor{red}{\XSolidBrush} &\textcolor{green}{\CheckmarkBold} & \textcolor{red}{\XSolidBrush} &\textcolor{green}{\CheckmarkBold} & \textcolor{red}{\XSolidBrush} \\
ToolAlpaca~\cite{tang2023toolalpaca} & ToolAlpaca & IFT & - & 4.0K & 400 & \textcolor{red}{\XSolidBrush} &\textcolor{green}{\CheckmarkBold} &\textcolor{green}{\CheckmarkBold} & \textcolor{red}{\XSolidBrush} &\textcolor{green}{\CheckmarkBold} & \textcolor{red}{\XSolidBrush}  &\textcolor{green}{\CheckmarkBold} & \textcolor{red}{\XSolidBrush}  \\
ToolLLaMA~\cite{qin2023toolllm} & ToolBench & IFT & - & 12.7K & 16,464 & \textcolor{red}{\XSolidBrush} &\textcolor{green}{\CheckmarkBold} &\textcolor{green}{\CheckmarkBold} &\textcolor{red}{\XSolidBrush} &\textcolor{green}{\CheckmarkBold}&\textcolor{green}{\CheckmarkBold}&\textcolor{green}{\CheckmarkBold} &\textcolor{green}{\CheckmarkBold}\\
AgentEvol~\citep{xi2024agentgym} & AgentTraj-L & IFT & - & 14.5K & 24 &\textcolor{red}{\XSolidBrush} & \textcolor{green}{\CheckmarkBold} &\textcolor{green}{\CheckmarkBold}&\textcolor{red}{\XSolidBrush} &\textcolor{green}{\CheckmarkBold}&\textcolor{red}{\XSolidBrush} &\textcolor{red}{\XSolidBrush} &\textcolor{green}{\CheckmarkBold}\\
Lumos~\cite{yin2024agent} & Lumos & IFT  & - & 20.0K & 16 &\textcolor{red}{\XSolidBrush} & \textcolor{green}{\CheckmarkBold} & \textcolor{green}{\CheckmarkBold} &\textcolor{red}{\XSolidBrush} & \textcolor{green}{\CheckmarkBold} & \textcolor{green}{\CheckmarkBold} &\textcolor{red}{\XSolidBrush} & \textcolor{green}{\CheckmarkBold}\\
Agent-FLAN~\cite{chen2024agent} & Agent-FLAN & IFT & - & 24.7K & 20 &\textcolor{red}{\XSolidBrush} & \textcolor{green}{\CheckmarkBold} & \textcolor{green}{\CheckmarkBold} &\textcolor{red}{\XSolidBrush} & \textcolor{green}{\CheckmarkBold}& \textcolor{green}{\CheckmarkBold}&\textcolor{red}{\XSolidBrush} & \textcolor{green}{\CheckmarkBold}\\
AgentTuning~\citep{zeng2023agenttuning} & AgentInstruct & IFT & - & 35.0K & - &\textcolor{red}{\XSolidBrush} & \textcolor{green}{\CheckmarkBold} & \textcolor{green}{\CheckmarkBold} &\textcolor{red}{\XSolidBrush} & \textcolor{green}{\CheckmarkBold} &\textcolor{red}{\XSolidBrush} &\textcolor{red}{\XSolidBrush} & \textcolor{green}{\CheckmarkBold}\\\midrule
\multicolumn{13}{l}{\emph{Instruction Finetuning-based LLM Agents for Function Calling}} \\\midrule
NexusRaven~\citep{srinivasan2023nexusraven} & NexusRaven & IFT & - & - & 116 & \textcolor{green}{\CheckmarkBold} & \textcolor{green}{\CheckmarkBold}  & \textcolor{green}{\CheckmarkBold} &\textcolor{red}{\XSolidBrush} & \textcolor{green}{\CheckmarkBold} &\textcolor{red}{\XSolidBrush} &\textcolor{red}{\XSolidBrush}&\textcolor{red}{\XSolidBrush}\\
Gorilla~\citep{patil2023gorilla} & Gorilla & IFT & - & 16.0K & 1,645 & \textcolor{green}{\CheckmarkBold} &\textcolor{red}{\XSolidBrush} &\textcolor{red}{\XSolidBrush}&\textcolor{green}{\CheckmarkBold} &\textcolor{green}{\CheckmarkBold} &\textcolor{red}{\XSolidBrush} &\textcolor{red}{\XSolidBrush} &\textcolor{red}{\XSolidBrush}\\
OpenFunctions-v2~\citep{patil2023gorilla} & OpenFunctions-v2 & IFT & - & 65.0K & - & \textcolor{green}{\CheckmarkBold} & \textcolor{green}{\CheckmarkBold} &\textcolor{red}{\XSolidBrush} &\textcolor{green}{\CheckmarkBold} &\textcolor{green}{\CheckmarkBold} &\textcolor{red}{\XSolidBrush} &\textcolor{red}{\XSolidBrush} &\textcolor{red}{\XSolidBrush}\\
API Pack~\cite{guo2024api} & API Pack & IFT & - & 1.1M & 11,213 &\textcolor{green}{\CheckmarkBold} &\textcolor{red}{\XSolidBrush} &\textcolor{green}{\CheckmarkBold} &\textcolor{red}{\XSolidBrush} &\textcolor{green}{\CheckmarkBold} &\textcolor{red}{\XSolidBrush}&\textcolor{red}{\XSolidBrush}&\textcolor{red}{\XSolidBrush}\\ 
LAM~\citep{zhang2024agentohana} & AgentOhana & IFT & - & 42.6K & - & \textcolor{green}{\CheckmarkBold} & \textcolor{green}{\CheckmarkBold} &\textcolor{green}{\CheckmarkBold}&\textcolor{red}{\XSolidBrush} &\textcolor{green}{\CheckmarkBold}&\textcolor{red}{\XSolidBrush}&\textcolor{green}{\CheckmarkBold}&\textcolor{green}{\CheckmarkBold}\\
xLAM~\citep{liu2024apigen} & APIGen & IFT & - & 60.0K & 3,673 & \textcolor{green}{\CheckmarkBold} & \textcolor{green}{\CheckmarkBold} &\textcolor{green}{\CheckmarkBold}&\textcolor{red}{\XSolidBrush} &\textcolor{green}{\CheckmarkBold}&\textcolor{red}{\XSolidBrush}&\textcolor{green}{\CheckmarkBold}&\textcolor{green}{\CheckmarkBold}\\\midrule
\multicolumn{13}{l}{\emph{Pretraining-based LLM Agents}}  \\\midrule
% LEMUR~\citep{xu2024lemur} & PT & 90B & 300.0K & - & \textcolor{green}{\CheckmarkBold} & \textcolor{green}{\CheckmarkBold} &\textcolor{green}{\CheckmarkBold}&\textcolor{red}{\XSolidBrush} & \textcolor{red}{\XSolidBrush} &\textcolor{green}{\CheckmarkBold} &\textcolor{red}{\XSolidBrush}&\textcolor{red}{\XSolidBrush}\\
\rowcolor{teal!12} \method & \dataset & PT & 103B & 95.0K  & 76,537  & \textcolor{green}{\CheckmarkBold} & \textcolor{green}{\CheckmarkBold} & \textcolor{green}{\CheckmarkBold} & \textcolor{green}{\CheckmarkBold} & \textcolor{green}{\CheckmarkBold} & \textcolor{green}{\CheckmarkBold} & \textcolor{green}{\CheckmarkBold} & \textcolor{green}{\CheckmarkBold}\\
\bottomrule
\end{tabular}
% \begin{tablenotes}
%     \item $^*$ In addition, the StarCoder-API can offer 4.77M more APIs.
% \end{tablenotes}
\caption{Summary of existing instruction finetuning-based LLM agents for intrinsic reasoning and function calling, along with their training resources and sample sizes. "PT" and "IFT" denote "Pre-Training" and "Instruction Fine-Tuning", respectively.}
\vspace{-2ex}
\label{tab:related}
\end{threeparttable}
\end{table*}

\noindent \textbf{Prompting-based LLM Agents.} Due to the lack of agent-specific pre-training corpus, existing LLM agents rely on either prompt engineering~\cite{hsieh2023tool,lu2024chameleon,yao2022react,wang2023voyager} or instruction fine-tuning~\cite{chen2023fireact,zeng2023agenttuning} to understand human instructions, decompose high-level tasks, generate grounded plans, and execute multi-step actions. 
However, prompting-based methods mainly depend on the capabilities of backbone LLMs (usually commercial LLMs), failing to introduce new knowledge and struggling to generalize to unseen tasks~\cite{sun2024adaplanner,zhuang2023toolchain}. 

\noindent \textbf{Instruction Finetuning-based LLM Agents.} Considering the extensive diversity of APIs and the complexity of multi-tool instructions, tool learning inherently presents greater challenges than natural language tasks, such as text generation~\cite{qin2023toolllm}.
Post-training techniques focus more on instruction following and aligning output with specific formats~\cite{patil2023gorilla,hao2024toolkengpt,qin2023toolllm,schick2024toolformer}, rather than fundamentally improving model knowledge or capabilities. 
Moreover, heavy fine-tuning can hinder generalization or even degrade performance in non-agent use cases, potentially suppressing the original base model capabilities~\cite{ghosh2024a}.

\noindent \textbf{Pretraining-based LLM Agents.} While pre-training serves as an essential alternative, prior works~\cite{nijkamp2023codegen,roziere2023code,xu2024lemur,patil2023gorilla} have primarily focused on improving task-specific capabilities (\eg, code generation) instead of general-domain LLM agents, due to single-source, uni-type, small-scale, and poor-quality pre-training data. 
Existing tool documentation data for agent training either lacks diverse real-world APIs~\cite{patil2023gorilla, tang2023toolalpaca} or is constrained to single-tool or single-round tool execution. 
Furthermore, trajectory data mostly imitate expert behavior or follow function-calling rules with inferior planning and reasoning, failing to fully elicit LLMs' capabilities and handle complex instructions~\cite{qin2023toolllm}. 
Given a wide range of candidate API functions, each comprising various function names and parameters available at every planning step, identifying globally optimal solutions and generalizing across tasks remains highly challenging.



\section{Preliminaries}
\label{Preliminaries}
\begin{figure*}[t]
    \centering
    \includegraphics[width=0.95\linewidth]{fig/HealthGPT_Framework.png}
    \caption{The \ourmethod{} architecture integrates hierarchical visual perception and H-LoRA, employing a task-specific hard router to select visual features and H-LoRA plugins, ultimately generating outputs with an autoregressive manner.}
    \label{fig:architecture}
\end{figure*}
\noindent\textbf{Large Vision-Language Models.} 
The input to a LVLM typically consists of an image $x^{\text{img}}$ and a discrete text sequence $x^{\text{txt}}$. The visual encoder $\mathcal{E}^{\text{img}}$ converts the input image $x^{\text{img}}$ into a sequence of visual tokens $\mathcal{V} = [v_i]_{i=1}^{N_v}$, while the text sequence $x^{\text{txt}}$ is mapped into a sequence of text tokens $\mathcal{T} = [t_i]_{i=1}^{N_t}$ using an embedding function $\mathcal{E}^{\text{txt}}$. The LLM $\mathcal{M_\text{LLM}}(\cdot|\theta)$ models the joint probability of the token sequence $\mathcal{U} = \{\mathcal{V},\mathcal{T}\}$, which is expressed as:
\begin{equation}
    P_\theta(R | \mathcal{U}) = \prod_{i=1}^{N_r} P_\theta(r_i | \{\mathcal{U}, r_{<i}\}),
\end{equation}
where $R = [r_i]_{i=1}^{N_r}$ is the text response sequence. The LVLM iteratively generates the next token $r_i$ based on $r_{<i}$. The optimization objective is to minimize the cross-entropy loss of the response $\mathcal{R}$.
% \begin{equation}
%     \mathcal{L}_{\text{VLM}} = \mathbb{E}_{R|\mathcal{U}}\left[-\log P_\theta(R | \mathcal{U})\right]
% \end{equation}
It is worth noting that most LVLMs adopt a design paradigm based on ViT, alignment adapters, and pre-trained LLMs\cite{liu2023llava,liu2024improved}, enabling quick adaptation to downstream tasks.


\noindent\textbf{VQGAN.}
VQGAN~\cite{esser2021taming} employs latent space compression and indexing mechanisms to effectively learn a complete discrete representation of images. VQGAN first maps the input image $x^{\text{img}}$ to a latent representation $z = \mathcal{E}(x)$ through a encoder $\mathcal{E}$. Then, the latent representation is quantized using a codebook $\mathcal{Z} = \{z_k\}_{k=1}^K$, generating a discrete index sequence $\mathcal{I} = [i_m]_{m=1}^N$, where $i_m \in \mathcal{Z}$ represents the quantized code index:
\begin{equation}
    \mathcal{I} = \text{Quantize}(z|\mathcal{Z}) = \arg\min_{z_k \in \mathcal{Z}} \| z - z_k \|_2.
\end{equation}
In our approach, the discrete index sequence $\mathcal{I}$ serves as a supervisory signal for the generation task, enabling the model to predict the index sequence $\hat{\mathcal{I}}$ from input conditions such as text or other modality signals.  
Finally, the predicted index sequence $\hat{\mathcal{I}}$ is upsampled by the VQGAN decoder $G$, generating the high-quality image $\hat{x}^\text{img} = G(\hat{\mathcal{I}})$.



\noindent\textbf{Low Rank Adaptation.} 
LoRA\cite{hu2021lora} effectively captures the characteristics of downstream tasks by introducing low-rank adapters. The core idea is to decompose the bypass weight matrix $\Delta W\in\mathbb{R}^{d^{\text{in}} \times d^{\text{out}}}$ into two low-rank matrices $ \{A \in \mathbb{R}^{d^{\text{in}} \times r}, B \in \mathbb{R}^{r \times d^{\text{out}}} \}$, where $ r \ll \min\{d^{\text{in}}, d^{\text{out}}\} $, significantly reducing learnable parameters. The output with the LoRA adapter for the input $x$ is then given by:
\begin{equation}
    h = x W_0 + \alpha x \Delta W/r = x W_0 + \alpha xAB/r,
\end{equation}
where matrix $ A $ is initialized with a Gaussian distribution, while the matrix $ B $ is initialized as a zero matrix. The scaling factor $ \alpha/r $ controls the impact of $ \Delta W $ on the model.

\section{HealthGPT}
\label{Method}


\subsection{Unified Autoregressive Generation.}  
% As shown in Figure~\ref{fig:architecture}, 
\ourmethod{} (Figure~\ref{fig:architecture}) utilizes a discrete token representation that covers both text and visual outputs, unifying visual comprehension and generation as an autoregressive task. 
For comprehension, $\mathcal{M}_\text{llm}$ receives the input joint sequence $\mathcal{U}$ and outputs a series of text token $\mathcal{R} = [r_1, r_2, \dots, r_{N_r}]$, where $r_i \in \mathcal{V}_{\text{txt}}$, and $\mathcal{V}_{\text{txt}}$ represents the LLM's vocabulary:
\begin{equation}
    P_\theta(\mathcal{R} \mid \mathcal{U}) = \prod_{i=1}^{N_r} P_\theta(r_i \mid \mathcal{U}, r_{<i}).
\end{equation}
For generation, $\mathcal{M}_\text{llm}$ first receives a special start token $\langle \text{START\_IMG} \rangle$, then generates a series of tokens corresponding to the VQGAN indices $\mathcal{I} = [i_1, i_2, \dots, i_{N_i}]$, where $i_j \in \mathcal{V}_{\text{vq}}$, and $\mathcal{V}_{\text{vq}}$ represents the index range of VQGAN. Upon completion of generation, the LLM outputs an end token $\langle \text{END\_IMG} \rangle$:
\begin{equation}
    P_\theta(\mathcal{I} \mid \mathcal{U}) = \prod_{j=1}^{N_i} P_\theta(i_j \mid \mathcal{U}, i_{<j}).
\end{equation}
Finally, the generated index sequence $\mathcal{I}$ is fed into the decoder $G$, which reconstructs the target image $\hat{x}^{\text{img}} = G(\mathcal{I})$.

\subsection{Hierarchical Visual Perception}  
Given the differences in visual perception between comprehension and generation tasks—where the former focuses on abstract semantics and the latter emphasizes complete semantics—we employ ViT to compress the image into discrete visual tokens at multiple hierarchical levels.
Specifically, the image is converted into a series of features $\{f_1, f_2, \dots, f_L\}$ as it passes through $L$ ViT blocks.

To address the needs of various tasks, the hidden states are divided into two types: (i) \textit{Concrete-grained features} $\mathcal{F}^{\text{Con}} = \{f_1, f_2, \dots, f_k\}, k < L$, derived from the shallower layers of ViT, containing sufficient global features, suitable for generation tasks; 
(ii) \textit{Abstract-grained features} $\mathcal{F}^{\text{Abs}} = \{f_{k+1}, f_{k+2}, \dots, f_L\}$, derived from the deeper layers of ViT, which contain abstract semantic information closer to the text space, suitable for comprehension tasks.

The task type $T$ (comprehension or generation) determines which set of features is selected as the input for the downstream large language model:
\begin{equation}
    \mathcal{F}^{\text{img}}_T =
    \begin{cases}
        \mathcal{F}^{\text{Con}}, & \text{if } T = \text{generation task} \\
        \mathcal{F}^{\text{Abs}}, & \text{if } T = \text{comprehension task}
    \end{cases}
\end{equation}
We integrate the image features $\mathcal{F}^{\text{img}}_T$ and text features $\mathcal{T}$ into a joint sequence through simple concatenation, which is then fed into the LLM $\mathcal{M}_{\text{llm}}$ for autoregressive generation.
% :
% \begin{equation}
%     \mathcal{R} = \mathcal{M}_{\text{llm}}(\mathcal{U}|\theta), \quad \mathcal{U} = [\mathcal{F}^{\text{img}}_T; \mathcal{T}]
% \end{equation}
\subsection{Heterogeneous Knowledge Adaptation}
We devise H-LoRA, which stores heterogeneous knowledge from comprehension and generation tasks in separate modules and dynamically routes to extract task-relevant knowledge from these modules. 
At the task level, for each task type $ T $, we dynamically assign a dedicated H-LoRA submodule $ \theta^T $, which is expressed as:
\begin{equation}
    \mathcal{R} = \mathcal{M}_\text{LLM}(\mathcal{U}|\theta, \theta^T), \quad \theta^T = \{A^T, B^T, \mathcal{R}^T_\text{outer}\}.
\end{equation}
At the feature level for a single task, H-LoRA integrates the idea of Mixture of Experts (MoE)~\cite{masoudnia2014mixture} and designs an efficient matrix merging and routing weight allocation mechanism, thus avoiding the significant computational delay introduced by matrix splitting in existing MoELoRA~\cite{luo2024moelora}. Specifically, we first merge the low-rank matrices (rank = r) of $ k $ LoRA experts into a unified matrix:
\begin{equation}
    \mathbf{A}^{\text{merged}}, \mathbf{B}^{\text{merged}} = \text{Concat}(\{A_i\}_1^k), \text{Concat}(\{B_i\}_1^k),
\end{equation}
where $ \mathbf{A}^{\text{merged}} \in \mathbb{R}^{d^\text{in} \times rk} $ and $ \mathbf{B}^{\text{merged}} \in \mathbb{R}^{rk \times d^\text{out}} $. The $k$-dimension routing layer generates expert weights $ \mathcal{W} \in \mathbb{R}^{\text{token\_num} \times k} $ based on the input hidden state $ x $, and these are expanded to $ \mathbb{R}^{\text{token\_num} \times rk} $ as follows:
\begin{equation}
    \mathcal{W}^\text{expanded} = \alpha k \mathcal{W} / r \otimes \mathbf{1}_r,
\end{equation}
where $ \otimes $ denotes the replication operation.
The overall output of H-LoRA is computed as:
\begin{equation}
    \mathcal{O}^\text{H-LoRA} = (x \mathbf{A}^{\text{merged}} \odot \mathcal{W}^\text{expanded}) \mathbf{B}^{\text{merged}},
\end{equation}
where $ \odot $ represents element-wise multiplication. Finally, the output of H-LoRA is added to the frozen pre-trained weights to produce the final output:
\begin{equation}
    \mathcal{O} = x W_0 + \mathcal{O}^\text{H-LoRA}.
\end{equation}
% In summary, H-LoRA is a task-based dynamic PEFT method that achieves high efficiency in single-task fine-tuning.

\subsection{Training Pipeline}

\begin{figure}[t]
    \centering
    \hspace{-4mm}
    \includegraphics[width=0.94\linewidth]{fig/data.pdf}
    \caption{Data statistics of \texttt{VL-Health}. }
    \label{fig:data}
\end{figure}
\noindent \textbf{1st Stage: Multi-modal Alignment.} 
In the first stage, we design separate visual adapters and H-LoRA submodules for medical unified tasks. For the medical comprehension task, we train abstract-grained visual adapters using high-quality image-text pairs to align visual embeddings with textual embeddings, thereby enabling the model to accurately describe medical visual content. During this process, the pre-trained LLM and its corresponding H-LoRA submodules remain frozen. In contrast, the medical generation task requires training concrete-grained adapters and H-LoRA submodules while keeping the LLM frozen. Meanwhile, we extend the textual vocabulary to include multimodal tokens, enabling the support of additional VQGAN vector quantization indices. The model trains on image-VQ pairs, endowing the pre-trained LLM with the capability for image reconstruction. This design ensures pixel-level consistency of pre- and post-LVLM. The processes establish the initial alignment between the LLM’s outputs and the visual inputs.

\noindent \textbf{2nd Stage: Heterogeneous H-LoRA Plugin Adaptation.}  
The submodules of H-LoRA share the word embedding layer and output head but may encounter issues such as bias and scale inconsistencies during training across different tasks. To ensure that the multiple H-LoRA plugins seamlessly interface with the LLMs and form a unified base, we fine-tune the word embedding layer and output head using a small amount of mixed data to maintain consistency in the model weights. Specifically, during this stage, all H-LoRA submodules for different tasks are kept frozen, with only the word embedding layer and output head being optimized. Through this stage, the model accumulates foundational knowledge for unified tasks by adapting H-LoRA plugins.

\begin{table*}[!t]
\centering
\caption{Comparison of \ourmethod{} with other LVLMs and unified multi-modal models on medical visual comprehension tasks. \textbf{Bold} and \underline{underlined} text indicates the best performance and second-best performance, respectively.}
\resizebox{\textwidth}{!}{
\begin{tabular}{c|lcc|cccccccc|c}
\toprule
\rowcolor[HTML]{E9F3FE} &  &  &  & \multicolumn{2}{c}{\textbf{VQA-RAD \textuparrow}} & \multicolumn{2}{c}{\textbf{SLAKE \textuparrow}} & \multicolumn{2}{c}{\textbf{PathVQA \textuparrow}} &  &  &  \\ 
\cline{5-10}
\rowcolor[HTML]{E9F3FE}\multirow{-2}{*}{\textbf{Type}} & \multirow{-2}{*}{\textbf{Model}} & \multirow{-2}{*}{\textbf{\# Params}} & \multirow{-2}{*}{\makecell{\textbf{Medical} \\ \textbf{LVLM}}} & \textbf{close} & \textbf{all} & \textbf{close} & \textbf{all} & \textbf{close} & \textbf{all} & \multirow{-2}{*}{\makecell{\textbf{MMMU} \\ \textbf{-Med}}\textuparrow} & \multirow{-2}{*}{\textbf{OMVQA}\textuparrow} & \multirow{-2}{*}{\textbf{Avg. \textuparrow}} \\ 
\midrule \midrule
\multirow{9}{*}{\textbf{Comp. Only}} 
& Med-Flamingo & 8.3B & \Large \ding{51} & 58.6 & 43.0 & 47.0 & 25.5 & 61.9 & 31.3 & 28.7 & 34.9 & 41.4 \\
& LLaVA-Med & 7B & \Large \ding{51} & 60.2 & 48.1 & 58.4 & 44.8 & 62.3 & 35.7 & 30.0 & 41.3 & 47.6 \\
& HuatuoGPT-Vision & 7B & \Large \ding{51} & 66.9 & 53.0 & 59.8 & 49.1 & 52.9 & 32.0 & 42.0 & 50.0 & 50.7 \\
& BLIP-2 & 6.7B & \Large \ding{55} & 43.4 & 36.8 & 41.6 & 35.3 & 48.5 & 28.8 & 27.3 & 26.9 & 36.1 \\
& LLaVA-v1.5 & 7B & \Large \ding{55} & 51.8 & 42.8 & 37.1 & 37.7 & 53.5 & 31.4 & 32.7 & 44.7 & 41.5 \\
& InstructBLIP & 7B & \Large \ding{55} & 61.0 & 44.8 & 66.8 & 43.3 & 56.0 & 32.3 & 25.3 & 29.0 & 44.8 \\
& Yi-VL & 6B & \Large \ding{55} & 52.6 & 42.1 & 52.4 & 38.4 & 54.9 & 30.9 & 38.0 & 50.2 & 44.9 \\
& InternVL2 & 8B & \Large \ding{55} & 64.9 & 49.0 & 66.6 & 50.1 & 60.0 & 31.9 & \underline{43.3} & 54.5 & 52.5\\
& Llama-3.2 & 11B & \Large \ding{55} & 68.9 & 45.5 & 72.4 & 52.1 & 62.8 & 33.6 & 39.3 & 63.2 & 54.7 \\
\midrule
\multirow{5}{*}{\textbf{Comp. \& Gen.}} 
& Show-o & 1.3B & \Large \ding{55} & 50.6 & 33.9 & 31.5 & 17.9 & 52.9 & 28.2 & 22.7 & 45.7 & 42.6 \\
& Unified-IO 2 & 7B & \Large \ding{55} & 46.2 & 32.6 & 35.9 & 21.9 & 52.5 & 27.0 & 25.3 & 33.0 & 33.8 \\
& Janus & 1.3B & \Large \ding{55} & 70.9 & 52.8 & 34.7 & 26.9 & 51.9 & 27.9 & 30.0 & 26.8 & 33.5 \\
& \cellcolor[HTML]{DAE0FB}HealthGPT-M3 & \cellcolor[HTML]{DAE0FB}3.8B & \cellcolor[HTML]{DAE0FB}\Large \ding{51} & \cellcolor[HTML]{DAE0FB}\underline{73.7} & \cellcolor[HTML]{DAE0FB}\underline{55.9} & \cellcolor[HTML]{DAE0FB}\underline{74.6} & \cellcolor[HTML]{DAE0FB}\underline{56.4} & \cellcolor[HTML]{DAE0FB}\underline{78.7} & \cellcolor[HTML]{DAE0FB}\underline{39.7} & \cellcolor[HTML]{DAE0FB}\underline{43.3} & \cellcolor[HTML]{DAE0FB}\underline{68.5} & \cellcolor[HTML]{DAE0FB}\underline{61.3} \\
& \cellcolor[HTML]{DAE0FB}HealthGPT-L14 & \cellcolor[HTML]{DAE0FB}14B & \cellcolor[HTML]{DAE0FB}\Large \ding{51} & \cellcolor[HTML]{DAE0FB}\textbf{77.7} & \cellcolor[HTML]{DAE0FB}\textbf{58.3} & \cellcolor[HTML]{DAE0FB}\textbf{76.4} & \cellcolor[HTML]{DAE0FB}\textbf{64.5} & \cellcolor[HTML]{DAE0FB}\textbf{85.9} & \cellcolor[HTML]{DAE0FB}\textbf{44.4} & \cellcolor[HTML]{DAE0FB}\textbf{49.2} & \cellcolor[HTML]{DAE0FB}\textbf{74.4} & \cellcolor[HTML]{DAE0FB}\textbf{66.4} \\
\bottomrule
\end{tabular}
}
\label{tab:results}
\end{table*}
\begin{table*}[ht]
    \centering
    \caption{The experimental results for the four modality conversion tasks.}
    \resizebox{\textwidth}{!}{
    \begin{tabular}{l|ccc|ccc|ccc|ccc}
        \toprule
        \rowcolor[HTML]{E9F3FE} & \multicolumn{3}{c}{\textbf{CT to MRI (Brain)}} & \multicolumn{3}{c}{\textbf{CT to MRI (Pelvis)}} & \multicolumn{3}{c}{\textbf{MRI to CT (Brain)}} & \multicolumn{3}{c}{\textbf{MRI to CT (Pelvis)}} \\
        \cline{2-13}
        \rowcolor[HTML]{E9F3FE}\multirow{-2}{*}{\textbf{Model}}& \textbf{SSIM $\uparrow$} & \textbf{PSNR $\uparrow$} & \textbf{MSE $\downarrow$} & \textbf{SSIM $\uparrow$} & \textbf{PSNR $\uparrow$} & \textbf{MSE $\downarrow$} & \textbf{SSIM $\uparrow$} & \textbf{PSNR $\uparrow$} & \textbf{MSE $\downarrow$} & \textbf{SSIM $\uparrow$} & \textbf{PSNR $\uparrow$} & \textbf{MSE $\downarrow$} \\
        \midrule \midrule
        pix2pix & 71.09 & 32.65 & 36.85 & 59.17 & 31.02 & 51.91 & 78.79 & 33.85 & 28.33 & 72.31 & 32.98 & 36.19 \\
        CycleGAN & 54.76 & 32.23 & 40.56 & 54.54 & 30.77 & 55.00 & 63.75 & 31.02 & 52.78 & 50.54 & 29.89 & 67.78 \\
        BBDM & {71.69} & {32.91} & {34.44} & 57.37 & 31.37 & 48.06 & \textbf{86.40} & 34.12 & 26.61 & {79.26} & 33.15 & 33.60 \\
        Vmanba & 69.54 & 32.67 & 36.42 & {63.01} & {31.47} & {46.99} & 79.63 & 34.12 & 26.49 & 77.45 & 33.53 & 31.85 \\
        DiffMa & 71.47 & 32.74 & 35.77 & 62.56 & 31.43 & 47.38 & 79.00 & {34.13} & {26.45} & 78.53 & {33.68} & {30.51} \\
        \rowcolor[HTML]{DAE0FB}HealthGPT-M3 & \underline{79.38} & \underline{33.03} & \underline{33.48} & \underline{71.81} & \underline{31.83} & \underline{43.45} & {85.06} & \textbf{34.40} & \textbf{25.49} & \underline{84.23} & \textbf{34.29} & \textbf{27.99} \\
        \rowcolor[HTML]{DAE0FB}HealthGPT-L14 & \textbf{79.73} & \textbf{33.10} & \textbf{32.96} & \textbf{71.92} & \textbf{31.87} & \textbf{43.09} & \underline{85.31} & \underline{34.29} & \underline{26.20} & \textbf{84.96} & \underline{34.14} & \underline{28.13} \\
        \bottomrule
    \end{tabular}
    }
    \label{tab:conversion}
\end{table*}

\noindent \textbf{3rd Stage: Visual Instruction Fine-Tuning.}  
In the third stage, we introduce additional task-specific data to further optimize the model and enhance its adaptability to downstream tasks such as medical visual comprehension (e.g., medical QA, medical dialogues, and report generation) or generation tasks (e.g., super-resolution, denoising, and modality conversion). Notably, by this stage, the word embedding layer and output head have been fine-tuned, only the H-LoRA modules and adapter modules need to be trained. This strategy significantly improves the model's adaptability and flexibility across different tasks.


\section{Experiment}
\label{s:experiment}

\subsection{Data Description}
We evaluate our method on FI~\cite{you2016building}, Twitter\_LDL~\cite{yang2017learning} and Artphoto~\cite{machajdik2010affective}.
FI is a public dataset built from Flickr and Instagram, with 23,308 images and eight emotion categories, namely \textit{amusement}, \textit{anger}, \textit{awe},  \textit{contentment}, \textit{disgust}, \textit{excitement},  \textit{fear}, and \textit{sadness}. 
% Since images in FI are all copyrighted by law, some images are corrupted now, so we remove these samples and retain 21,828 images.
% T4SA contains images from Twitter, which are classified into three categories: \textit{positive}, \textit{neutral}, and \textit{negative}. In this paper, we adopt the base version of B-T4SA, which contains 470,586 images and provides text descriptions of the corresponding tweets.
Twitter\_LDL contains 10,045 images from Twitter, with the same eight categories as the FI dataset.
% 。
For these two datasets, they are randomly split into 80\%
training and 20\% testing set.
Artphoto contains 806 artistic photos from the DeviantArt website, which we use to further evaluate the zero-shot capability of our model.
% on the small-scale dataset.
% We construct and publicly release the first image sentiment analysis dataset containing metadata.
% 。

% Based on these datasets, we are the first to construct and publicly release metadata-enhanced image sentiment analysis datasets. These datasets include scenes, tags, descriptions, and corresponding confidence scores, and are available at this link for future research purposes.


% 
\begin{table}[t]
\centering
% \begin{center}
\caption{Overall performance of different models on FI and Twitter\_LDL datasets.}
\label{tab:cap1}
% \resizebox{\linewidth}{!}
{
\begin{tabular}{l|c|c|c|c}
\hline
\multirow{2}{*}{\textbf{Model}} & \multicolumn{2}{c|}{\textbf{FI}}  & \multicolumn{2}{c}{\textbf{Twitter\_LDL}} \\ \cline{2-5} 
  & \textbf{Accuracy} & \textbf{F1} & \textbf{Accuracy} & \textbf{F1}  \\ \hline
% (\rownumber)~AlexNet~\cite{krizhevsky2017imagenet}  & 58.13\% & 56.35\%  & 56.24\%& 55.02\%  \\ 
% (\rownumber)~VGG16~\cite{simonyan2014very}  & 63.75\%& 63.08\%  & 59.34\%& 59.02\%  \\ 
(\rownumber)~ResNet101~\cite{he2016deep} & 66.16\%& 65.56\%  & 62.02\% & 61.34\%  \\ 
(\rownumber)~CDA~\cite{han2023boosting} & 66.71\%& 65.37\%  & 64.14\% & 62.85\%  \\ 
(\rownumber)~CECCN~\cite{ruan2024color} & 67.96\%& 66.74\%  & 64.59\%& 64.72\% \\ 
(\rownumber)~EmoVIT~\cite{xie2024emovit} & 68.09\%& 67.45\%  & 63.12\% & 61.97\%  \\ 
(\rownumber)~ComLDL~\cite{zhang2022compound} & 68.83\%& 67.28\%  & 65.29\% & 63.12\%  \\ 
(\rownumber)~WSDEN~\cite{li2023weakly} & 69.78\%& 69.61\%  & 67.04\% & 65.49\% \\ 
(\rownumber)~ECWA~\cite{deng2021emotion} & 70.87\%& 69.08\%  & 67.81\% & 66.87\%  \\ 
(\rownumber)~EECon~\cite{yang2023exploiting} & 71.13\%& 68.34\%  & 64.27\%& 63.16\%  \\ 
(\rownumber)~MAM~\cite{zhang2024affective} & 71.44\%  & 70.83\% & 67.18\%  & 65.01\%\\ 
(\rownumber)~TGCA-PVT~\cite{chen2024tgca}   & 73.05\%  & 71.46\% & 69.87\%  & 68.32\% \\ 
(\rownumber)~OEAN~\cite{zhang2024object}   & 73.40\%  & 72.63\% & 70.52\%  & 69.47\% \\ \hline
(\rownumber)~\shortname  & \textbf{79.48\%} & \textbf{79.22\%} & \textbf{74.12\%} & \textbf{73.09\%} \\ \hline
\end{tabular}
}
\vspace{-6mm}
% \end{center}
\end{table}
% 

\subsection{Experiment Setting}
% \subsubsection{Model Setting.}
% 
\textbf{Model Setting:}
For feature representation, we set $k=10$ to select object tags, and adopt clip-vit-base-patch32 as the pre-trained model for unified feature representation.
Moreover, we empirically set $(d_e, d_h, d_k, d_s) = (512, 128, 16, 64)$, and set the classification class $L$ to 8.

% 

\textbf{Training Setting:}
To initialize the model, we set all weights such as $\boldsymbol{W}$ following the truncated normal distribution, and use AdamW optimizer with the learning rate of $1 \times 10^{-4}$.
% warmup scheduler of cosine, warmup steps of 2000.
Furthermore, we set the batch size to 32 and the epoch of the training process to 200.
During the implementation, we utilize \textit{PyTorch} to build our entire model.
% , and our project codes are publicly available at https://github.com/zzmyrep/MESN.
% Our project codes as well as data are all publicly available on GitHub\footnote{https://github.com/zzmyrep/KBCEN}.
% Code is available at \href{https://github.com/zzmyrep/KBCEN}{https://github.com/zzmyrep/KBCEN}.

\textbf{Evaluation Metrics:}
Following~\cite{zhang2024affective, chen2024tgca, zhang2024object}, we adopt \textit{accuracy} and \textit{F1} as our evaluation metrics to measure the performance of different methods for image sentiment analysis. 



\subsection{Experiment Result}
% We compare our model against the following baselines: AlexNet~\cite{krizhevsky2017imagenet}, VGG16~\cite{simonyan2014very}, ResNet101~\cite{he2016deep}, CECCN~\cite{ruan2024color}, EmoVIT~\cite{xie2024emovit}, WSCNet~\cite{yang2018weakly}, ECWA~\cite{deng2021emotion}, EECon~\cite{yang2023exploiting}, MAM~\cite{zhang2024affective} and TGCA-PVT~\cite{chen2024tgca}, and the overall results are summarized in Table~\ref{tab:cap1}.
We compare our model against several baselines, and the overall results are summarized in Table~\ref{tab:cap1}.
We observe that our model achieves the best performance in both accuracy and F1 metrics, significantly outperforming the previous models. 
This superior performance is mainly attributed to our effective utilization of metadata to enhance image sentiment analysis, as well as the exceptional capability of the unified sentiment transformer framework we developed. These results strongly demonstrate that our proposed method can bring encouraging performance for image sentiment analysis.

\setcounter{magicrownumbers}{0} 
\begin{table}[t]
\begin{center}
\caption{Ablation study of~\shortname~on FI dataset.} 
% \vspace{1mm}
\label{tab:cap2}
\resizebox{.9\linewidth}{!}
{
\begin{tabular}{lcc}
  \hline
  \textbf{Model} & \textbf{Accuracy} & \textbf{F1} \\
  \hline
  (\rownumber)~Ours (w/o vision) & 65.72\% & 64.54\% \\
  (\rownumber)~Ours (w/o text description) & 74.05\% & 72.58\% \\
  (\rownumber)~Ours (w/o object tag) & 77.45\% & 76.84\% \\
  (\rownumber)~Ours (w/o scene tag) & 78.47\% & 78.21\% \\
  \hline
  (\rownumber)~Ours (w/o unified embedding) & 76.41\% & 76.23\% \\
  (\rownumber)~Ours (w/o adaptive learning) & 76.83\% & 76.56\% \\
  (\rownumber)~Ours (w/o cross-modal fusion) & 76.85\% & 76.49\% \\
  \hline
  (\rownumber)~Ours  & \textbf{79.48\%} & \textbf{79.22\%} \\
  \hline
\end{tabular}
}
\end{center}
\vspace{-5mm}
\end{table}


\begin{figure}[t]
\centering
% \vspace{-2mm}
\includegraphics[width=0.42\textwidth]{fig/2dvisual-linux4-paper2.pdf}
\caption{Visualization of feature distribution on eight categories before (left) and after (right) model processing.}
% 
\label{fig:visualization}
\vspace{-5mm}
\end{figure}

\subsection{Ablation Performance}
In this subsection, we conduct an ablation study to examine which component is really important for performance improvement. The results are reported in Table~\ref{tab:cap2}.

For information utilization, we observe a significant decline in model performance when visual features are removed. Additionally, the performance of \shortname~decreases when different metadata are removed separately, which means that text description, object tag, and scene tag are all critical for image sentiment analysis.
Recalling the model architecture, we separately remove transformer layers of the unified representation module, the adaptive learning module, and the cross-modal fusion module, replacing them with MLPs of the same parameter scale.
In this way, we can observe varying degrees of decline in model performance, indicating that these modules are indispensable for our model to achieve better performance.

\subsection{Visualization}
% 


% % 开始使用minipage进行左右排列
% \begin{minipage}[t]{0.45\textwidth}  % 子图1宽度为45%
%     \centering
%     \includegraphics[width=\textwidth]{2dvisual.pdf}  % 插入图片
%     \captionof{figure}{Visualization of feature distribution.}  % 使用captionof添加图片标题
%     \label{fig:visualization}
% \end{minipage}


% \begin{figure}[t]
% \centering
% \vspace{-2mm}
% \includegraphics[width=0.45\textwidth]{fig/2dvisual.pdf}
% \caption{Visualization of feature distribution.}
% \label{fig:visualization}
% % \vspace{-4mm}
% \end{figure}

% \begin{figure}[t]
% \centering
% \vspace{-2mm}
% \includegraphics[width=0.45\textwidth]{fig/2dvisual-linux3-paper.pdf}
% \caption{Visualization of feature distribution.}
% \label{fig:visualization}
% % \vspace{-4mm}
% \end{figure}



\begin{figure}[tbp]   
\vspace{-4mm}
  \centering            
  \subfloat[Depth of adaptive learning layers]   
  {
    \label{fig:subfig1}\includegraphics[width=0.22\textwidth]{fig/fig_sensitivity-a5}
  }
  \subfloat[Depth of fusion layers]
  {
    % \label{fig:subfig2}\includegraphics[width=0.22\textwidth]{fig/fig_sensitivity-b2}
    \label{fig:subfig2}\includegraphics[width=0.22\textwidth]{fig/fig_sensitivity-b2-num.pdf}
  }
  \caption{Sensitivity study of \shortname~on different depth. }   
  \label{fig:fig_sensitivity}  
\vspace{-2mm}
\end{figure}

% \begin{figure}[htbp]
% \centerline{\includegraphics{2dvisual.pdf}}
% \caption{Visualization of feature distribution.}
% \label{fig:visualization}
% \end{figure}

% In Fig.~\ref{fig:visualization}, we use t-SNE~\cite{van2008visualizing} to reduce the dimension of data features for visualization, Figure in left represents the metadata features before model processing, the features are obtained by embedding through the CLIP model, and figure in right shows the features of the data after model processing, it can be observed that after the model processing, the data with different label categories fall in different regions in the space, therefore, we can conclude that the Therefore, we can conclude that the model can effectively utilize the information contained in the metadata and use it to guide the model for classification.

In Fig.~\ref{fig:visualization}, we use t-SNE~\cite{van2008visualizing} to reduce the dimension of data features for visualization.
The left figure shows metadata features before being processed by our model (\textit{i.e.}, embedded by CLIP), while the right shows the distribution of features after being processed by our model.
We can observe that after the model processing, data with the same label are closer to each other, while others are farther away.
Therefore, it shows that the model can effectively utilize the information contained in the metadata and use it to guide the classification process.

\subsection{Sensitivity Analysis}
% 
In this subsection, we conduct a sensitivity analysis to figure out the effect of different depth settings of adaptive learning layers and fusion layers. 
% In this subsection, we conduct a sensitivity analysis to figure out the effect of different depth settings on the model. 
% Fig.~\ref{fig:fig_sensitivity} presents the effect of different depth settings of adaptive learning layers and fusion layers. 
Taking Fig.~\ref{fig:fig_sensitivity} (a) as an example, the model performance improves with increasing depth, reaching the best performance at a depth of 4.
% Taking Fig.~\ref{fig:fig_sensitivity} (a) as an example, the performance of \shortname~improves with the increase of depth at first, reaching the best performance at a depth of 4.
When the depth continues to increase, the accuracy decreases to varying degrees.
Similar results can be observed in Fig.~\ref{fig:fig_sensitivity} (b).
Therefore, we set their depths to 4 and 6 respectively to achieve the best results.

% Through our experiments, we can observe that the effect of modifying these hyperparameters on the results of the experiments is very weak, and the surface model is not sensitive to the hyperparameters.


\subsection{Zero-shot Capability}
% 

% (1)~GCH~\cite{2010Analyzing} & 21.78\% & (5)~RA-DLNet~\cite{2020A} & 34.01\% \\ \hline
% (2)~WSCNet~\cite{2019WSCNet}  & 30.25\% & (6)~CECCN~\cite{ruan2024color} & 43.83\% \\ \hline
% (3)~PCNN~\cite{2015Robust} & 31.68\%  & (7)~EmoVIT~\cite{xie2024emovit} & 44.90\% \\ \hline
% (4)~AR~\cite{2018Visual} & 32.67\% & (8)~Ours (Zero-shot) & 47.83\% \\ \hline


\begin{table}[t]
\centering
\caption{Zero-shot capability of \shortname.}
\label{tab:cap3}
\resizebox{1\linewidth}{!}
{
\begin{tabular}{lc|lc}
\hline
\textbf{Model} & \textbf{Accuracy} & \textbf{Model} & \textbf{Accuracy} \\ \hline
(1)~WSCNet~\cite{2019WSCNet}  & 30.25\% & (5)~MAM~\cite{zhang2024affective} & 39.56\%  \\ \hline
(2)~AR~\cite{2018Visual} & 32.67\% & (6)~CECCN~\cite{ruan2024color} & 43.83\% \\ \hline
(3)~RA-DLNet~\cite{2020A} & 34.01\%  & (7)~EmoVIT~\cite{xie2024emovit} & 44.90\% \\ \hline
(4)~CDA~\cite{han2023boosting} & 38.64\% & (8)~Ours (Zero-shot) & 47.83\% \\ \hline
\end{tabular}
}
\vspace{-5mm}
\end{table}

% We use the model trained on the FI dataset to test on the artphoto dataset to verify the model's generalization ability as well as robustness to other distributed datasets.
% We can observe that the MESN model shows strong competitiveness in terms of accuracy when compared to other trained models, which suggests that the model has a good generalization ability in the OOD task.

To validate the model's generalization ability and robustness to other distributed datasets, we directly test the model trained on the FI dataset, without training on Artphoto. 
% As observed in Table 3, compared to other models trained on Artphoto, we achieve highly competitive zero-shot performance, indicating that the model has good generalization ability in out-of-distribution tasks.
From Table~\ref{tab:cap3}, we can observe that compared with other models trained on Artphoto, we achieve competitive zero-shot performance, which shows that the model has good generalization ability in out-of-distribution tasks.


%%%%%%%%%%%%
%  E2E     %
%%%%%%%%%%%%


\section{Conclusion}
In this paper, we introduced Wi-Chat, the first LLM-powered Wi-Fi-based human activity recognition system that integrates the reasoning capabilities of large language models with the sensing potential of wireless signals. Our experimental results on a self-collected Wi-Fi CSI dataset demonstrate the promising potential of LLMs in enabling zero-shot Wi-Fi sensing. These findings suggest a new paradigm for human activity recognition that does not rely on extensive labeled data. We hope future research will build upon this direction, further exploring the applications of LLMs in signal processing domains such as IoT, mobile sensing, and radar-based systems.

\section*{Limitations}
While our work represents the first attempt to leverage LLMs for processing Wi-Fi signals, it is a preliminary study focused on a relatively simple task: Wi-Fi-based human activity recognition. This choice allows us to explore the feasibility of LLMs in wireless sensing but also comes with certain limitations.

Our approach primarily evaluates zero-shot performance, which, while promising, may still lag behind traditional supervised learning methods in highly complex or fine-grained recognition tasks. Besides, our study is limited to a controlled environment with a self-collected dataset, and the generalizability of LLMs to diverse real-world scenarios with varying Wi-Fi conditions, environmental interference, and device heterogeneity remains an open question.

Additionally, we have yet to explore the full potential of LLMs in more advanced Wi-Fi sensing applications, such as fine-grained gesture recognition, occupancy detection, and passive health monitoring. Future work should investigate the scalability of LLM-based approaches, their robustness to domain shifts, and their integration with multimodal sensing techniques in broader IoT applications.


% Bibliography entries for the entire Anthology, followed by custom entries
%\bibliography{anthology,custom}
% Custom bibliography entries only
\bibliography{main}
\newpage
\appendix

\section{Experiment prompts}
\label{sec:prompt}
The prompts used in the LLM experiments are shown in the following Table~\ref{tab:prompts}.

\definecolor{titlecolor}{rgb}{0.9, 0.5, 0.1}
\definecolor{anscolor}{rgb}{0.2, 0.5, 0.8}
\definecolor{labelcolor}{HTML}{48a07e}
\begin{table*}[h]
	\centering
	
 % \vspace{-0.2cm}
	
	\begin{center}
		\begin{tikzpicture}[
				chatbox_inner/.style={rectangle, rounded corners, opacity=0, text opacity=1, font=\sffamily\scriptsize, text width=5in, text height=9pt, inner xsep=6pt, inner ysep=6pt},
				chatbox_prompt_inner/.style={chatbox_inner, align=flush left, xshift=0pt, text height=11pt},
				chatbox_user_inner/.style={chatbox_inner, align=flush left, xshift=0pt},
				chatbox_gpt_inner/.style={chatbox_inner, align=flush left, xshift=0pt},
				chatbox/.style={chatbox_inner, draw=black!25, fill=gray!7, opacity=1, text opacity=0},
				chatbox_prompt/.style={chatbox, align=flush left, fill=gray!1.5, draw=black!30, text height=10pt},
				chatbox_user/.style={chatbox, align=flush left},
				chatbox_gpt/.style={chatbox, align=flush left},
				chatbox2/.style={chatbox_gpt, fill=green!25},
				chatbox3/.style={chatbox_gpt, fill=red!20, draw=black!20},
				chatbox4/.style={chatbox_gpt, fill=yellow!30},
				labelbox/.style={rectangle, rounded corners, draw=black!50, font=\sffamily\scriptsize\bfseries, fill=gray!5, inner sep=3pt},
			]
											
			\node[chatbox_user] (q1) {
				\textbf{System prompt}
				\newline
				\newline
				You are a helpful and precise assistant for segmenting and labeling sentences. We would like to request your help on curating a dataset for entity-level hallucination detection.
				\newline \newline
                We will give you a machine generated biography and a list of checked facts about the biography. Each fact consists of a sentence and a label (True/False). Please do the following process. First, breaking down the biography into words. Second, by referring to the provided list of facts, merging some broken down words in the previous step to form meaningful entities. For example, ``strategic thinking'' should be one entity instead of two. Third, according to the labels in the list of facts, labeling each entity as True or False. Specifically, for facts that share a similar sentence structure (\eg, \textit{``He was born on Mach 9, 1941.''} (\texttt{True}) and \textit{``He was born in Ramos Mejia.''} (\texttt{False})), please first assign labels to entities that differ across atomic facts. For example, first labeling ``Mach 9, 1941'' (\texttt{True}) and ``Ramos Mejia'' (\texttt{False}) in the above case. For those entities that are the same across atomic facts (\eg, ``was born'') or are neutral (\eg, ``he,'' ``in,'' and ``on''), please label them as \texttt{True}. For the cases that there is no atomic fact that shares the same sentence structure, please identify the most informative entities in the sentence and label them with the same label as the atomic fact while treating the rest of the entities as \texttt{True}. In the end, output the entities and labels in the following format:
                \begin{itemize}[nosep]
                    \item Entity 1 (Label 1)
                    \item Entity 2 (Label 2)
                    \item ...
                    \item Entity N (Label N)
                \end{itemize}
                % \newline \newline
                Here are two examples:
                \newline\newline
                \textbf{[Example 1]}
                \newline
                [The start of the biography]
                \newline
                \textcolor{titlecolor}{Marianne McAndrew is an American actress and singer, born on November 21, 1942, in Cleveland, Ohio. She began her acting career in the late 1960s, appearing in various television shows and films.}
                \newline
                [The end of the biography]
                \newline \newline
                [The start of the list of checked facts]
                \newline
                \textcolor{anscolor}{[Marianne McAndrew is an American. (False); Marianne McAndrew is an actress. (True); Marianne McAndrew is a singer. (False); Marianne McAndrew was born on November 21, 1942. (False); Marianne McAndrew was born in Cleveland, Ohio. (False); She began her acting career in the late 1960s. (True); She has appeared in various television shows. (True); She has appeared in various films. (True)]}
                \newline
                [The end of the list of checked facts]
                \newline \newline
                [The start of the ideal output]
                \newline
                \textcolor{labelcolor}{[Marianne McAndrew (True); is (True); an (True); American (False); actress (True); and (True); singer (False); , (True); born (True); on (True); November 21, 1942 (False); , (True); in (True); Cleveland, Ohio (False); . (True); She (True); began (True); her (True); acting career (True); in (True); the late 1960s (True); , (True); appearing (True); in (True); various (True); television shows (True); and (True); films (True); . (True)]}
                \newline
                [The end of the ideal output]
				\newline \newline
                \textbf{[Example 2]}
                \newline
                [The start of the biography]
                \newline
                \textcolor{titlecolor}{Doug Sheehan is an American actor who was born on April 27, 1949, in Santa Monica, California. He is best known for his roles in soap operas, including his portrayal of Joe Kelly on ``General Hospital'' and Ben Gibson on ``Knots Landing.''}
                \newline
                [The end of the biography]
                \newline \newline
                [The start of the list of checked facts]
                \newline
                \textcolor{anscolor}{[Doug Sheehan is an American. (True); Doug Sheehan is an actor. (True); Doug Sheehan was born on April 27, 1949. (True); Doug Sheehan was born in Santa Monica, California. (False); He is best known for his roles in soap operas. (True); He portrayed Joe Kelly. (True); Joe Kelly was in General Hospital. (True); General Hospital is a soap opera. (True); He portrayed Ben Gibson. (True); Ben Gibson was in Knots Landing. (True); Knots Landing is a soap opera. (True)]}
                \newline
                [The end of the list of checked facts]
                \newline \newline
                [The start of the ideal output]
                \newline
                \textcolor{labelcolor}{[Doug Sheehan (True); is (True); an (True); American (True); actor (True); who (True); was born (True); on (True); April 27, 1949 (True); in (True); Santa Monica, California (False); . (True); He (True); is (True); best known (True); for (True); his roles in soap operas (True); , (True); including (True); in (True); his portrayal (True); of (True); Joe Kelly (True); on (True); ``General Hospital'' (True); and (True); Ben Gibson (True); on (True); ``Knots Landing.'' (True)]}
                \newline
                [The end of the ideal output]
				\newline \newline
				\textbf{User prompt}
				\newline
				\newline
				[The start of the biography]
				\newline
				\textcolor{magenta}{\texttt{\{BIOGRAPHY\}}}
				\newline
				[The ebd of the biography]
				\newline \newline
				[The start of the list of checked facts]
				\newline
				\textcolor{magenta}{\texttt{\{LIST OF CHECKED FACTS\}}}
				\newline
				[The end of the list of checked facts]
			};
			\node[chatbox_user_inner] (q1_text) at (q1) {
				\textbf{System prompt}
				\newline
				\newline
				You are a helpful and precise assistant for segmenting and labeling sentences. We would like to request your help on curating a dataset for entity-level hallucination detection.
				\newline \newline
                We will give you a machine generated biography and a list of checked facts about the biography. Each fact consists of a sentence and a label (True/False). Please do the following process. First, breaking down the biography into words. Second, by referring to the provided list of facts, merging some broken down words in the previous step to form meaningful entities. For example, ``strategic thinking'' should be one entity instead of two. Third, according to the labels in the list of facts, labeling each entity as True or False. Specifically, for facts that share a similar sentence structure (\eg, \textit{``He was born on Mach 9, 1941.''} (\texttt{True}) and \textit{``He was born in Ramos Mejia.''} (\texttt{False})), please first assign labels to entities that differ across atomic facts. For example, first labeling ``Mach 9, 1941'' (\texttt{True}) and ``Ramos Mejia'' (\texttt{False}) in the above case. For those entities that are the same across atomic facts (\eg, ``was born'') or are neutral (\eg, ``he,'' ``in,'' and ``on''), please label them as \texttt{True}. For the cases that there is no atomic fact that shares the same sentence structure, please identify the most informative entities in the sentence and label them with the same label as the atomic fact while treating the rest of the entities as \texttt{True}. In the end, output the entities and labels in the following format:
                \begin{itemize}[nosep]
                    \item Entity 1 (Label 1)
                    \item Entity 2 (Label 2)
                    \item ...
                    \item Entity N (Label N)
                \end{itemize}
                % \newline \newline
                Here are two examples:
                \newline\newline
                \textbf{[Example 1]}
                \newline
                [The start of the biography]
                \newline
                \textcolor{titlecolor}{Marianne McAndrew is an American actress and singer, born on November 21, 1942, in Cleveland, Ohio. She began her acting career in the late 1960s, appearing in various television shows and films.}
                \newline
                [The end of the biography]
                \newline \newline
                [The start of the list of checked facts]
                \newline
                \textcolor{anscolor}{[Marianne McAndrew is an American. (False); Marianne McAndrew is an actress. (True); Marianne McAndrew is a singer. (False); Marianne McAndrew was born on November 21, 1942. (False); Marianne McAndrew was born in Cleveland, Ohio. (False); She began her acting career in the late 1960s. (True); She has appeared in various television shows. (True); She has appeared in various films. (True)]}
                \newline
                [The end of the list of checked facts]
                \newline \newline
                [The start of the ideal output]
                \newline
                \textcolor{labelcolor}{[Marianne McAndrew (True); is (True); an (True); American (False); actress (True); and (True); singer (False); , (True); born (True); on (True); November 21, 1942 (False); , (True); in (True); Cleveland, Ohio (False); . (True); She (True); began (True); her (True); acting career (True); in (True); the late 1960s (True); , (True); appearing (True); in (True); various (True); television shows (True); and (True); films (True); . (True)]}
                \newline
                [The end of the ideal output]
				\newline \newline
                \textbf{[Example 2]}
                \newline
                [The start of the biography]
                \newline
                \textcolor{titlecolor}{Doug Sheehan is an American actor who was born on April 27, 1949, in Santa Monica, California. He is best known for his roles in soap operas, including his portrayal of Joe Kelly on ``General Hospital'' and Ben Gibson on ``Knots Landing.''}
                \newline
                [The end of the biography]
                \newline \newline
                [The start of the list of checked facts]
                \newline
                \textcolor{anscolor}{[Doug Sheehan is an American. (True); Doug Sheehan is an actor. (True); Doug Sheehan was born on April 27, 1949. (True); Doug Sheehan was born in Santa Monica, California. (False); He is best known for his roles in soap operas. (True); He portrayed Joe Kelly. (True); Joe Kelly was in General Hospital. (True); General Hospital is a soap opera. (True); He portrayed Ben Gibson. (True); Ben Gibson was in Knots Landing. (True); Knots Landing is a soap opera. (True)]}
                \newline
                [The end of the list of checked facts]
                \newline \newline
                [The start of the ideal output]
                \newline
                \textcolor{labelcolor}{[Doug Sheehan (True); is (True); an (True); American (True); actor (True); who (True); was born (True); on (True); April 27, 1949 (True); in (True); Santa Monica, California (False); . (True); He (True); is (True); best known (True); for (True); his roles in soap operas (True); , (True); including (True); in (True); his portrayal (True); of (True); Joe Kelly (True); on (True); ``General Hospital'' (True); and (True); Ben Gibson (True); on (True); ``Knots Landing.'' (True)]}
                \newline
                [The end of the ideal output]
				\newline \newline
				\textbf{User prompt}
				\newline
				\newline
				[The start of the biography]
				\newline
				\textcolor{magenta}{\texttt{\{BIOGRAPHY\}}}
				\newline
				[The ebd of the biography]
				\newline \newline
				[The start of the list of checked facts]
				\newline
				\textcolor{magenta}{\texttt{\{LIST OF CHECKED FACTS\}}}
				\newline
				[The end of the list of checked facts]
			};
		\end{tikzpicture}
        \caption{GPT-4o prompt for labeling hallucinated entities.}\label{tb:gpt-4-prompt}
	\end{center}
\vspace{-0cm}
\end{table*}
% \section{Full Experiment Results}
% \begin{table*}[th]
    \centering
    \small
    \caption{Classification Results}
    \begin{tabular}{lcccc}
        \toprule
        \textbf{Method} & \textbf{Accuracy} & \textbf{Precision} & \textbf{Recall} & \textbf{F1-score} \\
        \midrule
        \multicolumn{5}{c}{\textbf{Zero Shot}} \\
                Zero-shot E-eyes & 0.26 & 0.26 & 0.27 & 0.26 \\
        Zero-shot CARM & 0.24 & 0.24 & 0.24 & 0.24 \\
                Zero-shot SVM & 0.27 & 0.28 & 0.28 & 0.27 \\
        Zero-shot CNN & 0.23 & 0.24 & 0.23 & 0.23 \\
        Zero-shot RNN & 0.26 & 0.26 & 0.26 & 0.26 \\
DeepSeek-0shot & 0.54 & 0.61 & 0.54 & 0.52 \\
DeepSeek-0shot-COT & 0.33 & 0.24 & 0.33 & 0.23 \\
DeepSeek-0shot-Knowledge & 0.45 & 0.46 & 0.45 & 0.44 \\
Gemma2-0shot & 0.35 & 0.22 & 0.38 & 0.27 \\
Gemma2-0shot-COT & 0.36 & 0.22 & 0.36 & 0.27 \\
Gemma2-0shot-Knowledge & 0.32 & 0.18 & 0.34 & 0.20 \\
GPT-4o-mini-0shot & 0.48 & 0.53 & 0.48 & 0.41 \\
GPT-4o-mini-0shot-COT & 0.33 & 0.50 & 0.33 & 0.38 \\
GPT-4o-mini-0shot-Knowledge & 0.49 & 0.31 & 0.49 & 0.36 \\
GPT-4o-0shot & 0.62 & 0.62 & 0.47 & 0.42 \\
GPT-4o-0shot-COT & 0.29 & 0.45 & 0.29 & 0.21 \\
GPT-4o-0shot-Knowledge & 0.44 & 0.52 & 0.44 & 0.39 \\
LLaMA-0shot & 0.32 & 0.25 & 0.32 & 0.24 \\
LLaMA-0shot-COT & 0.12 & 0.25 & 0.12 & 0.09 \\
LLaMA-0shot-Knowledge & 0.32 & 0.25 & 0.32 & 0.28 \\
Mistral-0shot & 0.19 & 0.23 & 0.19 & 0.10 \\
Mistral-0shot-Knowledge & 0.21 & 0.40 & 0.21 & 0.11 \\
        \midrule
        \multicolumn{5}{c}{\textbf{4 Shot}} \\
GPT-4o-mini-4shot & 0.58 & 0.59 & 0.58 & 0.53 \\
GPT-4o-mini-4shot-COT & 0.57 & 0.53 & 0.57 & 0.50 \\
GPT-4o-mini-4shot-Knowledge & 0.56 & 0.51 & 0.56 & 0.47 \\
GPT-4o-4shot & 0.77 & 0.84 & 0.77 & 0.73 \\
GPT-4o-4shot-COT & 0.63 & 0.76 & 0.63 & 0.53 \\
GPT-4o-4shot-Knowledge & 0.72 & 0.82 & 0.71 & 0.66 \\
LLaMA-4shot & 0.29 & 0.24 & 0.29 & 0.21 \\
LLaMA-4shot-COT & 0.20 & 0.30 & 0.20 & 0.13 \\
LLaMA-4shot-Knowledge & 0.15 & 0.23 & 0.13 & 0.13 \\
Mistral-4shot & 0.02 & 0.02 & 0.02 & 0.02 \\
Mistral-4shot-Knowledge & 0.21 & 0.27 & 0.21 & 0.20 \\
        \midrule
        
        \multicolumn{5}{c}{\textbf{Suprevised}} \\
        SVM & 0.94 & 0.92 & 0.91 & 0.91 \\
        CNN & 0.98 & 0.98 & 0.97 & 0.97 \\
        RNN & 0.99 & 0.99 & 0.99 & 0.99 \\
        % \midrule
        % \multicolumn{5}{c}{\textbf{Conventional Wi-Fi-based Human Activity Recognition Systems}} \\
        E-eyes & 1.00 & 1.00 & 1.00 & 1.00 \\
        CARM & 0.98 & 0.98 & 0.98 & 0.98 \\
\midrule
 \multicolumn{5}{c}{\textbf{Vision Models}} \\
           Zero-shot SVM & 0.26 & 0.25 & 0.25 & 0.25 \\
        Zero-shot CNN & 0.26 & 0.25 & 0.26 & 0.26 \\
        Zero-shot RNN & 0.28 & 0.28 & 0.29 & 0.28 \\
        SVM & 0.99 & 0.99 & 0.99 & 0.99 \\
        CNN & 0.98 & 0.99 & 0.98 & 0.98 \\
        RNN & 0.98 & 0.99 & 0.98 & 0.98 \\
GPT-4o-mini-Vision & 0.84 & 0.85 & 0.84 & 0.84 \\
GPT-4o-mini-Vision-COT & 0.90 & 0.91 & 0.90 & 0.90 \\
GPT-4o-Vision & 0.74 & 0.82 & 0.74 & 0.73 \\
GPT-4o-Vision-COT & 0.70 & 0.83 & 0.70 & 0.68 \\
LLaMA-Vision & 0.20 & 0.23 & 0.20 & 0.09 \\
LLaMA-Vision-Knowledge & 0.22 & 0.05 & 0.22 & 0.08 \\

        \bottomrule
    \end{tabular}
    \label{full}
\end{table*}




\end{document}

\begin{table*}[t]
    \centering
    \setlength{\tabcolsep}{8pt}
    \caption{
    Multilingual WER on MuAViC (Babble noise added at 0-SNR).
    Hours of video used to fine-tune each model are shown. 
    Mod. = Modality.
    Avg. non-En: average WER without English.
    H.R. = High Resource (Es, Fr, It, Pt). L.R. = Low Resource (Ar, De, El, Ru).
    }
    \label{tab:noisy}
    \vspace{-3mm}
\resizebox{\linewidth}{!}{%
\begin{tabular}{lcccrrrrrrrrrrrr}
\toprule
Model & \begin{tabular}[c]{@{}l@{}}Total \\ Params\end{tabular} & \begin{tabular}[c]{@{}l@{}}FT Vid. \\ Hours\end{tabular} & Mod. & \multicolumn{1}{l}{En} & \multicolumn{1}{l}{Ar} & \multicolumn{1}{l}{De} & \multicolumn{1}{l}{El} & \multicolumn{1}{l}{Es} & \multicolumn{1}{l}{Fr} & \multicolumn{1}{l}{It} & \multicolumn{1}{l}{Pt} & \multicolumn{1}{l}{Ru} & \multicolumn{1}{l}{\begin{tabular}[c]{@{}l@{}}Avg \\ non-En\end{tabular}} & \multicolumn{1}{l}{\begin{tabular}[c]{@{}l@{}}Avg \\ H.R.\end{tabular}} & \multicolumn{1}{l}{\begin{tabular}[c]{@{}l@{}}Avg\\ L.R. \end{tabular}} \\
% \midrule
\hline
\rowcolor{Gray}\multicolumn{16}{c}{\it{Small Models}} \\
Whisper Small Zero-Shot & 244M & - & A & 27.9 & 102 & 61.2 & 76.4 & 63.3 & 57.8 & 76.2 & 73.4 & 57.6 & 71.0 & 67.7 & 74.3 \\
Whisper Small Fine-Tuned & 244M & - & A & 16.1 & \bf{101} & 59.9 & 56.8 & 41.7 & 35.9 & 50.6 & 50.2 & 46.7 & 55.3 & 44.6 & 66.0 \\
mWhisper-Flamingo Small & 651M & \multicolumn{1}{r}{1,141} & AV & \bf{8.3} & \bf{101} & \bf{56.7} & \bf{51.1} & \bf{33.6} & \bf{31.9} & \bf{41.2} & \bf{42.8} & \bf{44.6} & \bf{50.4} & \bf{37.4} & \bf{63.5} \\
Relative Improvement &  & - &  & \it{48.4} & \it{-0.8} & \it{5.3} & \it{10.0} & \it{19.4} & \it{11.1} & \it{18.6} & \it{14.7} & \it{4.5} & \it{10.4} & \it{16.0} & \it{4.8} \\
% \midrule
\hline
\rowcolor{Gray}\multicolumn{16}{c}{\it{Medium Models}} \\
Whisper Medium Zero-Shot & 769M & - & A & 22.5 & 105 & 53.1 & 61.3 & 49.1 & 47.7 & 60.3 & 60.9 & 47.5 & 60.7 & 54.5 & 66.8 \\
Whisper Medium Fine-Tuned & 769M & - & A & 12.3 & 96.4 & 51.8 & 45.7 & 36.1 & 30.4 & 43.5 & 42.2 & 37.9 & 48.0 & 38.1 & 58.0 \\
mWhisper-Flamingo Medium & 1.39B & \multicolumn{1}{r}{1,141} & AV & \bf{7.4} & \bf{95.3} & \bf{49.4} & \bf{41.8} & \bf{28} & \bf{27.5} & \bf{35.2} & \bf{36} & \bf{36.1} & \bf{43.7} & \bf{31.7} & \bf{55.7} \\
Relative Improvement &  & - &  & \it{39.8} & \it{1.1} & \it{4.6} & \it{8.5} & \it{22.4} & \it{9.5} & \it{19.1} & \it{14.7} & \it{4.7} & \it{10.6} & \it{16.4} & \it{4.8} \\
% \midrule
\hline
\rowcolor{Gray}\multicolumn{16}{c}{\it{Large Models}} \\
Whisper Large Zero-Shot & 1.5B & - & A & \bf{21.3} & \bf{96.5} & \bf{51.5} & \bf{54.7} & \bf{46.9} & \bf{41.7} & \bf{56.6} & \bf{58.7} & \bf{41.3} & \bf{56.0} & \bf{51.0} & \bf{61.0}\\
\bottomrule
\end{tabular}%
}
\vspace{-0.4cm}
\end{table*}

\subsection{Clean Results}
In Table~\ref{tab:main}, we show the results on MuAViC using the original, clean audio.
For previous SOTA audio-visual methods, initial work fine-tuned pre-trained English AV-HuBERT~\cite{shi2022learning} on multilingual videos in MuAViC~\cite{anwar23_interspeech,hong-etal-2023-intuitive}.
Some methods trained individual models for specific non-En languages~\cite{li2023parameter,gimeno2024tailored,li24i_interspeech}, however, they are outperformed by multilingual models.
The current SOTA models are XLAVS-R~\cite{han2024xlavs} and Fast Conformer~\cite{burchi2024multilingual}.
The former adapted XLS-R~\cite{babu2021xls}, a multilingual self-supervised audio model, and the latter trains a model from scratch. 
Moreover, Fast Conformer achieved even better results on the higher-resource languages by using extra multilingual training videos from other datasets.
While these methods worked well on the higher-resource languages, their performance on lower-resource languages is less satisfactory.

Next, we establish baselines using Whisper models.
Compared to XLAVS-R 2B, Whisper small zero-shot achieves better performance on the lower resource languages (37.5\% vs 41.9\%), while Whisper medium zero-shot achieves better overall average non-En WER (22.9\% vs 26.4\%).
This shows Whisper's strong multilingual capabilities and motivates its selection as the foundation of our proposed models.

Fine-tuning Whisper on MuAViC leads to further gains.
Fine-tuned Whisper medium achieves a new \textbf{SOTA non-En average WER of 20.1\%,} surpassing all previous audio-visual methods fine-tuning solely on MuAViC. 
Notably, the English performance is also improved from 2.3\% (zero-shot) to 0.74\%, approaching the current SOTA of 0.68\% reported in Whisper-Flamingo~\cite{rouditchenko24_interspeech}. 
This shows that fine-tuning improves multilingual performance without compromising English accuracy.  

mWhisper-Flamingo performs similarly to audio-only fine-tuned Whisper (20.4\% vs. 20.1\% WER for the medium models). 
This small difference suggests that video provides limited benefit for clean audio. 
However, mWhisper-Flamingo still significantly outperforms all previous AVSR models fine-tuned solely on the 1,141h MuAViC videos, achieving a new SOTA on all languages with this setup. 
It even surpasses Fast Conformer trained with 4,957h of videos on all languages except Fr and Pt. 
This shows the strength of using Whisper as initialization for AVSR. 
Finally, mWhisper-Flamingo medium consistently outperforms mWhisper-Flamingo small, showing the benefit of increased model size.

\subsection{Noisy Results}
Table~\ref{tab:noisy} shows the performance on MuAViC with babble noise at 0-SNR. 
The babble noise is from Whisper-Flamingo~\cite{rouditchenko24_interspeech} and was constructed from 30 LRS3 speakers.
We compare Whisper zero-shot with Whisper fine-tuned and mWhisper-Flamingo.
The noisy results show a significant performance degredation compared to the clean results (Table~\ref{tab:main}).
For example, the average non-En WER for Whisper small zero-shot increases from 27.0\% in the clean setting to 71.0\% under 0-SNR babble noise. 
However, fine-tuning significantly improves WER. 
For the small model, fine-tuning improves the average non-En WER from 71.0\% (zero-shot) to 55.3\%. 
A similar trend is observed for the medium model, with WER decreasing from 60.7\% to 48.0\%. 

The integration of visual features in mWhisper-Flamingo provides further gains. 
mWhisper-Flamingo small achieves an average non-En WER of 50.4\%, a \textbf{10.4\% relative improvement} over fine-tuned audio-only Whisper small (55.3\%).
Similarly, mWhisper-Flamingo medium achieves an average non-En WER of 43.7\%, a \textbf{10.6\% relative improvement} compared to fine-tuned audio-only Whisper medium (48.0\%). 
The relative improvements are better for the higher-resource languages (\textbf{16.0\% and 16.4\%} for small and medium models), indicating that more video training data is helpful.
The relative improvements for English are much better at 48.4\% and 39.8\% for the small and medium models, which we attribute to having more English training data.

mWhisper-Flamingo medium achieves the best multilingual WER (43.7\%), even outperforming Whisper large zero-shot (65.0\%) despite having fewer parameters (1.39B vs 1.5B).
This shows that the performance gains are not only due to more parameters but also due to the visual modality. 
As additional evidence, mWhisper-Flamingo small (651M parameters) outperforms audio-only fine-tuned Whisper medium (769M parameters) on the higher-resource languages (37.4\% vs. 38.1\%). 
This highlights that smaller audio-visual models can achieve better performance than larger audio-only models.

Finally, we tested the models on 6 different noise types, 5 SNR levels $\{-10,-5,0,5,10 \}$, and 4 languages (Es, Fr, It, Pt). 
The noise setups follow Whisper-Flamingo~\cite{rouditchenko24_interspeech} and AV-HuBERT~\cite{shi22_interspeech}.
Figure~\ref{fig:noise} shows the WER for Es, Fr, It, Pt averaged over all SNR values for each noise type. 
There is a clear trend of mWhisper-Flamingo outperforming the audio-only models, especially for the more challenging noises (3 types of babble and side speech).
Overall, these results confirm the advantage of mWhisper-Flamingo on diverse noise types.

\begin{figure}[t]
    \centering
    \includegraphics[width=\linewidth]{figures/fig_avg.pdf}
    \vspace{-2em}
    \caption{Multilingual WER ($\downarrow$ is better) for different noise types averaged over 4 languages (Es, Fr, It, Pt) and 5 SNR levels $\{-10,-5,0,5,10 \}$.}
    \label{fig:noise}
    \vspace{-0.5cm}
\end{figure}


\subsection{Ablation Study and Analysis}
\label{sec:analysis}
In Table~\ref{tab:ablation}, we present an ablation study and model analysis.
Due to computational constraints, we trained and evaluated the small model on 5 of the 9 languages.
We compare the average non-En WER between models.
We show the noisy results since the performance differences between models in the clean results were minor.
The baseline audio-only fine-tuned Whisper small achieves 44.4\%.

% \begin{table}[!t]
% \centering
% \scalebox{0.68}{
%     \begin{tabular}{ll cccc}
%       \toprule
%       & \multicolumn{4}{c}{\textbf{Intellipro Dataset}}\\
%       & \multicolumn{2}{c}{Rank Resume} & \multicolumn{2}{c}{Rank Job} \\
%       \cmidrule(lr){2-3} \cmidrule(lr){4-5} 
%       \textbf{Method}
%       &  Recall@100 & nDCG@100 & Recall@10 & nDCG@10 \\
%       \midrule
%       \confitold{}
%       & 71.28 &34.79 &76.50 &52.57 
%       \\
%       \cmidrule{2-5}
%       \confitsimple{}
%     & 82.53 &48.17
%        & 85.58 &64.91
     
%        \\
%        +\RunnerUpMiningShort{}
%     &85.43 &50.99 &91.38 &71.34 
%       \\
%       +\HyReShort
%         &- & -
%        &-&-\\
       
%       \bottomrule

%     \end{tabular}
%   }
% \caption{Ablation studies using Jina-v2-base as the encoder. ``\confitsimple{}'' refers using a simplified encoder architecture. \framework{} trains \confitsimple{} with \RunnerUpMiningShort{} and \HyReShort{}.}
% \label{tbl:ablation}
% \end{table}
\begin{table*}[!t]
\centering
\scalebox{0.75}{
    \begin{tabular}{l cccc cccc}
      \toprule
      & \multicolumn{4}{c}{\textbf{Recruiting Dataset}}
      & \multicolumn{4}{c}{\textbf{AliYun Dataset}}\\
      & \multicolumn{2}{c}{Rank Resume} & \multicolumn{2}{c}{Rank Job} 
      & \multicolumn{2}{c}{Rank Resume} & \multicolumn{2}{c}{Rank Job}\\
      \cmidrule(lr){2-3} \cmidrule(lr){4-5} 
      \cmidrule(lr){6-7} \cmidrule(lr){8-9} 
      \textbf{Method}
      & Recall@100 & nDCG@100 & Recall@10 & nDCG@10
      & Recall@100 & nDCG@100 & Recall@10 & nDCG@10\\
      \midrule
      \confitold{}
      & 71.28 & 34.79 & 76.50 & 52.57 
      & 87.81 & 65.06 & 72.39 & 56.12
      \\
      \cmidrule{2-9}
      \confitsimple{}
      & 82.53 & 48.17 & 85.58 & 64.91
      & 94.90&78.40 & 78.70& 65.45
       \\
      +\HyReShort{}
       &85.28 & 49.50
       &90.25 & 70.22
       & 96.62&81.99 & \textbf{81.16}& 67.63
       \\
      +\RunnerUpMiningShort{}
       % & 85.14& 49.82
       % &90.75&72.51
       & \textbf{86.13}&\textbf{51.90} & \textbf{94.25}&\textbf{73.32}
       & \textbf{97.07}&\textbf{83.11} & 80.49& \textbf{68.02}
       \\
   %     +\RunnerUpMiningShort{}
   %    & 85.43 & 50.99 & 91.38 & 71.34 
   %    & 96.24 & 82.95 & 80.12 & 66.96
   %    \\
   %    +\HyReShort{} old
   %     &85.28 & 49.50
   %     &90.25 & 70.22
   %     & 96.62&81.99 & 81.16& 67.63
   %     \\
   % +\HyReShort{} 
   %     % & 85.14& 49.82
   %     % &90.75&72.51
   %     & 86.83&51.77 &92.00 &72.04
   %     & 97.07&83.11 & 80.49& 68.02
   %     \\
      \bottomrule

    \end{tabular}
  }
\caption{\framework{} ablation studies. ``\confitsimple{}'' refers using a simplified encoder architecture. \framework{} trains \confitsimple{} with \RunnerUpMiningShort{} and \HyReShort{}. We use Jina-v2-base as the encoder due to its better performance.
}
\label{tbl:ablation}
\end{table*}
% Note that the 5 language model presented in these experiments achieves slightly better than the 9 language model (36.6 vs 37.4).
First, we show that the combination of decoder modality dropout and a fine-tunable visual encoder leads to the best performance (36.6\%).
The performance is significantly worse if decoder modality dropout is disabled using either a fine-tunable visual encoder (44.6\%) or a frozen visual encoder encoder (42.6\%).
It shows that our proposed decoder modality dropout is \textit{crucial} for obtaining the best multilingual performance.
Note that the latter setup represents the default training configuration from Whisper-Flamingo~\cite{rouditchenko24_interspeech}.
Considering that the latter setup improved the English noisy WER from 16.2\% to 11.5\%, it is reasonable that modality dropout was not necessary in the original, English-only Whisper-Flamingo.
Finally, using decoder modality dropout with a fine-tunable visual encoder is better than using it with a frozen visual encoder (36.6\% vs 40.6\%).

mWhisper-Flamingo uses decoder modality dropout probabilities of \( p_{AV} = 0.5,\ p_{A} = 0,\ p_{V} = 0.5 \), meaning that the model trains with audio-visual inputs 50\% of the time and visual-only inputs 50\% of the time.
Without dropout, the modal trains only on audio-visual inputs (\( p_{AV} = 1,\ p_{A} = 0,\ p_{V} = 0 \)), and the performance is much worse (44.6\% vs 36.6\%).
Using dropout with audio-only inputs 50\% of the time instead of video-only inputs (\( p_{AV} = 0.5,\ p_{A} = 0.5,\ p_{V} = 0 \)) does not improve performance (44.6\%). 
It shows that it is necessary for the model to train with video-only inputs.
Trying different probabilities between AV, A, and V such as \( p_{AV} = 0.5,\ p_{A} = 0.25,\ p_{V} = 0.25 \) and \( p_{AV} = 0.25,\ p_{A} = 0.25,\ p_{V} = 0.5 \), the performance is improved to 37.7\%, however the best performance is achieved without audio-only inputs (\( p_{AV} = 0.5,\ p_{A} = 0,\ p_{V} = 0.5 \)).
Our explanation is that Whisper was fine-tuned in the first stage of training, so the model can already handle audio-only inputs well.
Training on video-only inputs allows the model to better integrate the visual modality.
We also tried other proportions of audio-visual and video-only training such as \( p_{AV} = 0.75,\ p_{A} = 0,\ p_{V} = 0.25 \) and \( p_{AV} = 0.25,\ p_{A} = 0,\ p_{V} = 0.75 \) which performed slightly worse.

Finally, we compare multilingual AV-HuBERT~\cite{kim_2024} with AV-HuBERT pre-trained only on English videos~\cite{shi2022learning}.
The model using the multilingual AV-HuBERT encoder achieves better multilingual performance (36.6\% vs 37.4\%) but worse English performance (8.0\% vs 7.5\%), which is reasonable.

\section{Conclusion}
We introduce mWhisper-Flamingo, a novel multilingual model for AVSR.
mWhisper-Flamingo outperforms all audio-visual methods trained on MuAViC, and even surpasses a model trained with substantially more videos on most languages. 
mWhisper-Flamingo outperforms audio-only Whisper in diverse noise settings. 
Our proposed decoder modality dropout significantly improved the noisy multilingual WER.
% Future work could try scaling training data and trying new audio and visual encoders.


\clearpage

\bibliographystyle{IEEEtran}
\bibliography{ref}

% Appendix for ArXiv version
\setcounter{table}{0}
\renewcommand{\thetable}{A\arabic{table}}
\section{Appendix}
\subsection{Full Noisy Results}
\begin{table*}[t]
    \centering
    \setlength{\tabcolsep}{5pt}
    \caption{Multilingual WER ($\downarrow$ is better) on MuAViC with different noise types and SNR levels (small models). The results for Music and Natural noise from MUSAN are averaged.
    }
    \label{tab:noise-full-small}
    \vspace{-3mm}
\resizebox{\linewidth}{!}{%
\begin{tabular}{lrrrrrrrrrrrrrrrrrrrrrrrrr}
\toprule
Method &  \multicolumn{5}{c}{Babble (LRS3), SNR=} & \multicolumn{5}{c}{Babble (MuAViC), SNR=} & \multicolumn{5}{c}{Babble (MUSAN), SNR=} & \multicolumn{5}{c}{Speech (LRS3), SNR=} & \multicolumn{5}{c}{Music+Natural, SNR=} \\
 & -10 & -5 & 0 & 5 & 10 & -10 & -5 & 0 & 5 & 10 & -10 & -5 & 0 & 5 & 10 & -10 & -5 & 0 & 5 & 10 & -10 & -5 & 0 & 5 & 10 \\
 
\hline
\rowcolor{Gray} \multicolumn{26}{l}{\textit{English (En)}} \\
Whisper, Zero-shot & 98.8 & 88.5 & 27.9 & 7.2 & 3.7 & 98.9 & 90.9 & 33.9 & 8.6 & 4.2 & 98.9 & 80.9 & 24.9 & 7.0 & 3.8 & 104 & 86.7 & 36.4 & 8.6 & \textbf{4.0} & 42.7 & 19.3 & 7.5 & 3.9 & \textbf{2.9} \\
Whisper, Fine-tuned & 109 & 79.4 & 16.0 & 5.6 & 4.1 & 120 & 100 & 20.4 & 6.2 & \bf{4.1} & 108 & 63.4 & 13.2 & 4.9 & 4.2 & 74.8 & 51.4 & 26.7 & 12.3 & 6.4 & 36.3 & 13.6 & 5.7 & \bf{4.0} & 3.7 \\
mWhisper-Flamingo & \bf{39} & \bf{27.2} & \bf{8.7} & \bf{4.8} & \bf{4.0} & \bf{39.7} & \bf{32.0} & \bf{10.2} & \bf{5.0} & 4.2 & \bf{38.7} & \bf{24.4} & \bf{7.8} & \bf{4.6} & \bf{4.1} & \bf{41.1} & \bf{26.3} & \bf{13.1} & \bf{6.6} & 4.7 & \bf{15.1} & \bf{7.7} & \bf{4.7} & 4.3 & 4.2 \\


\hline
\rowcolor{Gray} \multicolumn{26}{l}{\textit{Spanish (Es)}} \\
Whisper, Zero-shot & 98.2 & 98.8 & 63.3 & 28.1 & 17.0 & 98.2 & 97.2 & 64.9 & 30.1 & 17.6 & 98.3 & 95.3 & 59.8 & 27.6 & 16.6 & 116 & 111.3 & 76.1 & 30.9 & 16.7 & 62.6 & 40.4 & 24.3 & 16.7 & 13.8 \\
Whisper, Fine-tuned & 108 & 103 & 41.7 & 18.5 & 12.2 & 138 & 117 & 47.5 & 19.8 & 12.9 & 118 & 94.8 & 38.3 & 18.0 & 11.9 & 54.5 & 32.3 & 18.2 & \bf{12.7} & 11.0 & 56.2 & 29.9 & 17.3 & 12.5 & \bf{10.6} \\
mWhisper-Flamingo & \bf{90.1} & \bf{75.7} & \bf{33.6} & \bf{16.9} & \bf{11.7} & \bf{97.4} & \bf{83} & \bf{38.8} & \bf{17.5} & \bf{12.1} & \bf{92.1} & \bf{69.6} & \bf{31.8} & \bf{16.3} & \bf{11.6} & \bf{43.3} & \bf{25.4} & \bf{16.5} & 13.0 & \bf{10.8} & \bf{43.3} & \bf{25.0} & \bf{16.3} & \bf{11.9} & \bf{10.6} \\

\hline
\rowcolor{Gray} \multicolumn{26}{l}{\textit{French (Fr)}} \\
Whisper, Zero-shot & 100 & 96.8 & 57.8 & 31.9 & 22.4 & \textbf{101} & 97.2 & 64.5 & 33.3 & 22.5 & 101 & 95.0 & 54.5 & 30.3 & 22.2 & 111 & 100 & 60.3 & 29.9 & 22.0 & 63.7 & 45.4 & 30.3 & 23.0 & 20.0 \\
Whisper, Fine-tuned & 112 & 90.9 & 35.8 & 20.5 & 16.3 & 144 & 104.0 & 43.3 & 21.7 & 16.0 & 116 & 78.5 & 34.5 & 19.9 & 15.7 & 53.4 & 33.0 & 21.0 & 16.6 & \bf{15} & 53.1 & 31.8 & 21.1 & 15.9 & \bf{14.5} \\
mWhisper-Flamingo & \bf{98.7} & \bf{70.6} & \bf{31.9} & \bf{19.7} & \bf{15.8} & 110 & \bf{83.0} & \bf{39.3} & \bf{20.6} & \bf{15.8} & \bf{98.2} & \bf{63.9} & \bf{31.1} & \bf{19.1} & \bf{15.3} & \bf{44.7} & \bf{28.1} & \bf{19.8} & \bf{16.2} & 15.1 & \bf{44.9} & \bf{27.5} & \bf{19.6} & \bf{15.8} & 14.6 \\

\hline
\rowcolor{Gray} \multicolumn{26}{l}{\textit{Italian (It)}} \\
Whisper, Zero-shot & 99.3 & 101 & 76.2 & 42.7 & 27.7 & 99.8 & 100 & 78.7 & 44.9 & 27.8 & 99.4 & 101 & 71.5 & 39.7 & 26.3 & 112 & 107 & 78.1 & 35.7 & 28.1 & 69.5 & 52.1 & 36.0 & 27.1 & 22.1 \\
Whisper, Fine-tuned & 105 & 102 & 50.6 & 25.6 & 16.6 & 128 & 109 & 53.7 & 25.9 & 16.8 & 110 & 95.0 & 46.9 & 23.9 & 16.4 & 60.7 & 37.2 & 22.9 & 16.5 & \bf{14.4} & 57.5 & 35.1 & 21.9 & 16.2 & 14.1 \\
mWhisper-Flamingo & \bf{86.6} & \bf{75.9} & \bf{41.2} & \bf{22.8} & \bf{16.1} & \bf{94.7} & \bf{81.4} & \bf{43.4} & \bf{23.1} & \bf{16.0} & \bf{87.5} & \bf{71.1} & \bf{37.6} & \bf{22.0} & \bf{15.6} & \bf{48.5} & \bf{31.1} & \bf{21.1} & \bf{16.2} & \bf{14.4} & \bf{46.3} & \bf{30.2} & \bf{19.9} & \bf{15.7} & \bf{13.9} \\

\hline
\rowcolor{Gray} \multicolumn{26}{l}{\textit{Portuguese (Pt)}} \\
Whisper, Zero-shot & 97.8 & 98.9 & 73.4 & 41.3 & 26.9 & \bf{97.5} & 98.6 & 76.2 & 44.0 & 26.5 & 98.0 & 95.3 & 68.8 & 38.8 & 25.5 & 110 & 108 & 86.2 & 43.9 & 25 & 67.4 & 51.8 & 34.8 & 25.9 & 21.2 \\
Whisper, Fine-tuned & 109 & 107 & 50.2 & 26.9 & 18.0 & 140 & 116 & 56.3 & 28.2 & 18.0 & 114 & 96.3 & 45.5 & 24.2 & 17.5 & 67.3 & 41.5 & 25.1 & 18.7 & 15.7 & 60.7 & 36.5 & 23.0 & 17.4 & 15.0 \\
mWhisper-Flamingo & \bf{91.3} & \bf{80.9} & \bf{42.8} & \bf{23.9} & \bf{17.2} & 104 & \bf{89.8} & \bf{45.7} & \bf{25.0} & \bf{17.5} & \bf{90.1} & \bf{71.9} & \bf{37.7} & \bf{22.8} & \bf{16.7} & \bf{53.0} & \bf{35.2} & \bf{23.4} & \bf{18.3} & \bf{15.5} & \bf{48.1} & \bf{31.1} & \bf{21.0} & \bf{16.7} & \bf{14.6} \\

\bottomrule
\end{tabular}%
}
\end{table*}


\begin{table*}[t]
    \centering
    \setlength{\tabcolsep}{5pt}
    \caption{Multilingual WER ($\downarrow$ is better) on MuAViC with different noise types and SNR levels (medium models). The results for Music and Natural noise from MUSAN are averaged.
    }
    \label{tab:noise-full-medium}
    \vspace{-3mm}
\resizebox{\linewidth}{!}{%
\begin{tabular}{lrrrrrrrrrrrrrrrrrrrrrrrrr}
\toprule
Method &  \multicolumn{5}{c}{Babble (LRS3), SNR=} & \multicolumn{5}{c}{Babble (MuAViC), SNR=} & \multicolumn{5}{c}{Babble (MUSAN), SNR=} & \multicolumn{5}{c}{Speech (LRS3), SNR=} & \multicolumn{5}{c}{Music+Natural, SNR=} \\
 & -10 & -5 & 0 & 5 & 10 & -10 & -5 & 0 & 5 & 10 & -10 & -5 & 0 & 5 & 10 & -10 & -5 & 0 & 5 & 10 & -10 & -5 & 0 & 5 & 10 \\
 
\hline
\rowcolor{Gray} \multicolumn{26}{l}{\textit{English (En)}} \\
Whisper, Zero-shot & 98.9 & 84.0 & 22.5 & 5.6 & \textbf{2.6} & 99.7 & 86.1 & 26.6 & 6.6 & 3.0 & 99.2 & 74.0 & 20.1 & 5.5 & \textbf{2.8} & 99.2 & 74.7 & 29.2 & \textbf{5.9} & \textbf{2.8} & 37.8 & 16.0 & 5.8 & \bf{3.1} & \bf{2.4} \\
Whisper, Fine-tuned & 110 & 72.3 & 12.3 & 4.5 & 3.6 & 116 & 91.6 & 16.6 & 5.0 & 3.8 & 108 & 55.9 & 11.4 & 4.2 & 3.5 & 59.7 & 40.3 & 23.4 & 12.7 & 7.5 & 32.6 & 11.4 & 5.0 & 4.1 & 3.8 \\
mWhisper-Flamingo & \bf{40.0} & \bf{25.7} & \bf{7.5} & \bf{3.8} & 3.4 & \bf{40.9} & \bf{32.2} & \bf{8.7} & \bf{4.0} & \bf{3.6} & \bf{39.7} & \bf{22.1} & \bf{6.1} & \bf{3.7} & 3.6 & \bf{34.4} & \bf{23.1} & \bf{13.5} & 7.2 & 4.7 & \bf{13.5} & \bf{6.4} & \bf{4.1} & 3.5 & 3.5 \\


\hline
\rowcolor{Gray} \multicolumn{26}{l}{\textit{Spanish (Es)}} \\
Whisper, Zero-shot & 99.7 & 92.6 & 49.1 & 20.7 & 12.6 & 99.8 & 95.0 & 52.2 & 21.2 & 13.1 & 100.8 & 88.2 & 46.3 & 20.1 & 12.5 & 112.3 & 92.7 & 49.8 & 19.2 & 13.5 & 55.6 & 33.4 & 18.8 & 12.8 & 10.8 \\
Whisper, Fine-tuned & 111 & 97.0 & 36.1 & 14.9 & 11.3 & 129 & 108 & 38.5 & 15.4 & 11.4 & 114.9 & 83.2 & 31.1 & 14.1 & \bf{9.9} & 37.3 & 20.2 & 13.1 & 10.4 & \bf{8.9} & 47.6 & 25.6 & 13.6 & 10.5 & \bf{8.8} \\
mWhisper-Flamingo & \bf{91.2} & \bf{70.3} & \bf{28.0} & \bf{13.9} & \bf{9.9} & \bf{97.6} & \bf{77.8} & \bf{32.2} & \bf{14.4} & \bf{10.1} & \bf{91.0} & \bf{63.1} & \bf{26.2} & \bf{13.1} & 10.4 & \bf{31.0} & \bf{18.0} & \bf{12.1} & \bf{10.0} & 9.0 & \bf{38.6} & \bf{20.3} & \bf{12.5} & \bf{9.8} & \bf{8.8} \\

\hline
\rowcolor{Gray} \multicolumn{26}{l}{\textit{French (Fr)}} \\
Whisper, Zero-shot & 100 & 90.2 & 47.7 & 25.1 & 19.3 & \bf{99.9} & 93.9 & 52.2 & 27.1 & 18.7 & 100 & 85.5 & 44.0 & 24.5 & 19.3 & 103.3 & 73.6 & 37.2 & 23.0 & 19.4 & 56.4 & 36.2 & 24.6 & 18.9 & 16.6 \\
Whisper, Fine-tuned & 116 & 78.2 & 30.4 & 16.2 & \bf{12.4} & 139 & 97.6 & 35.6 & 17.5 & \bf{12.6} & 118 & 72.7 & 28.4 & 15.6 & \bf{12.4} & 36.9 & 22.9 & \bf{15.8} & \bf{12.9} & 12.0 & 47.4 & 25.4 & 16.0 & \bf{13.0} & \bf{11.8} \\
mWhisper-Flamingo & \bf{98.1} & \bf{63.3} & \bf{27.5} & \bf{16.0} & 12.7 & 107 & \bf{75.1} & \bf{30.7} & \bf{17.1} & 12.7 & \bf{97.7} & \bf{58.3} & \bf{26.2} & \bf{15.5} & 12.7 & \bf{32.4} & \bf{21.0} & 15.9 & 13.2 & \bf{11.9} & \bf{39.3} & \bf{23.1} & \bf{15.8} & \bf{13.0} & 12.1 \\

\hline
\rowcolor{Gray} \multicolumn{26}{l}{\textit{Italian (It)}} \\
Whisper, Zero-shot & 99.9 & 96.8 & 60.3 & 28.9 & 16.8 & 99.9 & 99.6 & 62.6 & 29.8 & 17.2 & 100 & 93.3 & 55.7 & 27.2 & 16.1 & 110.1 & 96.2 & 53.6 & 24.1 & 15.0 & 59.6 & 39.2 & 23.2 & 16.1 & 13.3 \\
Whisper, Fine-tuned & 112 & 102 & 43.5 & 19.8 & 13.0 & 121 & 102 & 47.2 & 20.0 & 12.9 & 117 & 87.9 & 39.3 & 18.7 & 12.6 & 44.5 & 25.3 & 16.6 & 13.1 & 11.5 & 52.1 & 29.6 & 17.5 & 12.8 & \bf{11.2} \\
mWhisper-Flamingo & \bf{92.7} & \bf{73.3} & \bf{35.2} & \bf{18.0} & \bf{12.6} & \bf{94.6} & \bf{77.4} & \bf{37.0} & \bf{18.3} & \bf{12.6} & \bf{89.8} & \bf{66.5} & \bf{31.6} & \bf{17.2} & \bf{12.1} & \bf{35.4} & \bf{22.2} & \bf{15.5} & \bf{12.5} & \bf{11.2} & \bf{42.6} & \bf{24.8} & \bf{15.9} & \bf{12.6} & 11.4 \\

\hline
\rowcolor{Gray} \multicolumn{26}{l}{\textit{Portuguese (Pt)}} \\
Whisper, Zero-shot & 97.0 & 93.2 & 60.9 & 31.2 & 18.9 & \textbf{97.1} & 94.8 & 63.1 & 32.1 & 19.5 & 97.0 & 90.7 & 55.7 & 28.4 & 18.5 & 109 & 98.9 & 69.2 & 31.9 & 18.9 & 60.1 & 41.1 & 26.1 & 18.3 & 15.2 \\
Whisper, Fine-tuned & 109 & 95.4 & 42.3 & 21.7 & 13.9 & 130 & 104 & 46.3 & 21.5 & 14.2 & 114.4 & 85.1 & 37.7 & 19.4 & 13.7 & 45.4 & 27.9 & 18.2 & 14.3 & 12.4 & 52.3 & 30.3 & 18.7 & 14.0 & 12.2 \\
mWhisper-Flamingo & \bf{94.5} & \bf{74.3} & \bf{36} & \bf{20.0} & \bf{13.7} & 100 & \bf{79.7} & \bf{38.6} & \bf{20.1} & \bf{14.1} & \bf{93.6} & \bf{65.4} & \bf{32.8} & \bf{18.2} & \bf{13.5} & \bf{38.1} & \bf{25.0} & \bf{17.2} & \bf{13.8} & \bf{12.3} & \bf{43.4} & \bf{26.3} & \bf{17.4} & \bf{13.5} & \bf{11.9} \\

\bottomrule
\end{tabular}%
}
\end{table*}



Table~\ref{tab:noise-full-small} and Table~\ref{tab:noise-full-medium} show the full decoding results for different noise types and SNRs.
Table~\ref{tab:noise-full-small} compares the small models and Table~\ref{tab:noise-full-medium} compares the medium models.
Note that the results in Table~\ref{tab:noise-full-small} were used to create Figure~\ref{fig:noise} (excluding the English results).

\end{document}