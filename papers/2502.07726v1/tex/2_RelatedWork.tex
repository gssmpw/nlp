Prediction of odometry on the basis of \acp{imu} has a long history in the military and civilian domains alike~\cite{titterton2004strapdown}. Considering the explicit goal of inertial-only estimation of odometry for pedestrians with \acp{imu} integrated on the shoes, \cite{angermann2012footslam} utilized a Bayesian approach to achieve long-term error stability. Exploiting the power of \acp{dnn}, the work in~\cite{ionet} considers segments of inertial data and achieves to estimate non-periodic motion trajectories. Focusing on pedestrians carrying a smartphone and performing natural motions, the work in~\cite{ronin} explored different \ac{dnn} architectures and demonstrated robust performance. TLIO~\cite{liuTLIOTightLearned2020a} built on top of such ideas and proposed a tightly-coupled \ac{ekf}-based method for odometry estimation relying exclusively on inertial sensor data. 

Currently, the domain of exclusively \ac{imu}-based odometry estimation has developed into a diverse field including a multitude of methods deployed across diverse robot configurations~\cite{cohen2023inertial}. The work in~\cite{cioffiLearnedInertialOdometry2023a} successfully deployed learned inertial odometry in the framework of autonomous drone racing. Considering neural prediction as part of an on-manifold \ac{ekf} estimator, the method in~\cite{bajwa2024dive} presented low prediction errors for quadrotor trajectories. For ground vehicles, the authors in~\cite{karlsson2021speed} presented speed estimation based on a \ac{cnn} with accelerometer and gyroscope measurements as inputs. This line of works has extended to legged robots~\cite{buchananLearningInertialOdometry2022}, as well as underwater systems~\cite{liDIEMEDeepInertial2024}. Methodologically, the literature presents diversity both regarding the neural architectures employed~\cite{chen2024deep} and in terms of using different estimation techniques, e.g., factor graphs~\cite{buchananDeepIMUBias2023a}. 

However, reliance exclusively on inertial sensing presents significant limitations especially when it comes to long-term consistency. At the same time, complementary proprioceptive signals may be seamlessly available onboard a robotic system raising the potential for multi-modal approaches to improve resilience. To that end, the work in~\cite{zhangDIDODeepInertial2022} proposed Deep Inertial quadrotor Dynamical Odometry (DIDO) which employs \acp{dnn} to learn both the IMU and dynamics properties and thus better support state estimation. Key to its approach is the use of tachometer sensing onboard the quadrotor's motors. From a different perspective, the authors in~\cite{joshiSMVIORobust2023a} consider underwater \ac{vio} combined with a model-based motion estimator such that when \ac{vio}-alone fails the collective system presents robustness. 

Compared to current literature, our contributions include: \textit{i)} a dynamics-aware proprioceptive method that accurately predicts an underwater robot's velocities and associated covariances through the fusion of \ac{imu} and motor information, \textit{ii)} an implementation based on single input-single output with the temporal context being propagated over the hidden states of the \ac{gru}, significantly reducing the computational overhead compared to a buffer of inputs, \textit{iii)} use of readily available motor commands instead of specialised RPM sensing, \textit{iv)} a network learning the velocity prediction with about $28$k parameters which allows for fast inference on both CPU and GPU, \textit{v)} the use of an ensemble of networks to enhance velocity and uncertainty prediction, and \textit{vi)} flexible fusion of the method into odometry estimation either without or with visual updates. 