%%%%%%%%%%%%%%%%%%%%%%%%%%%%%%%%%%%%%%%%%%%%%%%%%%%%%%%%%%%%%%%%%%%%%%%%%%%%%%%%
%2345678901234567890123456789012345678901234567890123456789012345678901234567890
%        1         2         3         4         5         6         7         8

\documentclass[letterpaper, 10 pt, conference]{ieeeconf}  % Comment this line out if you need a4paper

%\documentclass[a4paper, 10pt, conference]{ieeeconf}      % Use this line for a4 paper

\IEEEoverridecommandlockouts                              % This command is only needed if 
                                                          % you want to use the \thanks command

\overrideIEEEmargins                                      % Needed to meet printer requirements.

%
% --- inline annotations
%
\newcommand{\red}[1]{{\color{red}#1}}
\newcommand{\todo}[1]{{\color{red}#1}}
\newcommand{\TODO}[1]{\textbf{\color{red}[TODO: #1]}}
% --- disable by uncommenting  
% \renewcommand{\TODO}[1]{}
% \renewcommand{\todo}[1]{#1}



\newcommand{\VLM}{LVLM\xspace} 
\newcommand{\ours}{PeKit\xspace}
\newcommand{\yollava}{Yo’LLaVA\xspace}

\newcommand{\thisismy}{This-Is-My-Img\xspace}
\newcommand{\myparagraph}[1]{\noindent\textbf{#1}}
\newcommand{\vdoro}[1]{{\color[rgb]{0.4, 0.18, 0.78} {[V] #1}}}
% --- disable by uncommenting  
% \renewcommand{\TODO}[1]{}
% \renewcommand{\todo}[1]{#1}
\usepackage{slashbox}
% Vectors
\newcommand{\bB}{\mathcal{B}}
\newcommand{\bw}{\mathbf{w}}
\newcommand{\bs}{\mathbf{s}}
\newcommand{\bo}{\mathbf{o}}
\newcommand{\bn}{\mathbf{n}}
\newcommand{\bc}{\mathbf{c}}
\newcommand{\bp}{\mathbf{p}}
\newcommand{\bS}{\mathbf{S}}
\newcommand{\bk}{\mathbf{k}}
\newcommand{\bmu}{\boldsymbol{\mu}}
\newcommand{\bx}{\mathbf{x}}
\newcommand{\bg}{\mathbf{g}}
\newcommand{\be}{\mathbf{e}}
\newcommand{\bX}{\mathbf{X}}
\newcommand{\by}{\mathbf{y}}
\newcommand{\bv}{\mathbf{v}}
\newcommand{\bz}{\mathbf{z}}
\newcommand{\bq}{\mathbf{q}}
\newcommand{\bff}{\mathbf{f}}
\newcommand{\bu}{\mathbf{u}}
\newcommand{\bh}{\mathbf{h}}
\newcommand{\bb}{\mathbf{b}}

\newcommand{\rone}{\textcolor{green}{R1}}
\newcommand{\rtwo}{\textcolor{orange}{R2}}
\newcommand{\rthree}{\textcolor{red}{R3}}
\usepackage{amsmath}
%\usepackage{arydshln}
\DeclareMathOperator{\similarity}{sim}
\DeclareMathOperator{\AvgPool}{AvgPool}

\newcommand{\argmax}{\mathop{\mathrm{argmax}}}     




\title{\LARGE \bf
DeepVL: Dynamics and Inertial Measurements-based Deep Velocity Learning for Underwater Odometry 
}

\author{Mohit Singh, and Kostas Alexis 
%\thanks{$^\star$ The authors contributed equally.} % discuss it :)
\thanks{This material was supported by the Research Council of Norway Award NO-327292.}% <-this % stops a space
\thanks{The authors are with the Norwegian University of Science and Technology (NTNU), O. S. Bragstads Plass 2D, 7034, Trondheim, Norway {\tt\small mohit.singh@ntnu.no}}
}


\begin{document}



\maketitle
\thispagestyle{empty}
\pagestyle{empty}

%%%%%%%%%%%%%%%%%%%%%%%%%%%%%%%%%%%%%%%%%%%%%%%%%%%%%%%%%%%%%%%%%%%%%%%%%%%%%%%%
\begin{abstract}
This paper presents a learned model to predict the robot-centric velocity of an underwater robot through dynamics-aware proprioception. The method exploits a recurrent neural network using as inputs inertial cues, motor commands, and battery voltage readings alongside the hidden state of the previous time-step to output robust velocity estimates and their associated uncertainty. An ensemble of networks is utilized to enhance the velocity and uncertainty predictions. Fusing the network's outputs into an Extended Kalman Filter, alongside inertial predictions and barometer updates, the method enables long-term underwater odometry without further exteroception. Furthermore, when integrated into visual-inertial odometry, the method assists in enhanced estimation resilience when dealing with an order of magnitude fewer total features tracked (as few as $1$) as compared to conventional visual-inertial systems. Tested onboard an underwater robot deployed both in a laboratory pool and the Trondheim Fjord, the method takes less than $5\textrm{ms}$ for inference either on the CPU or the GPU of an NVIDIA Orin AGX and demonstrates less than \SI{4}{\percent} relative position error in novel trajectories during complete visual blackout, and approximately \SI{2}{\percent} relative error when a maximum of $2$ visual features from a monocular camera are available. 

\end{abstract}


%%%%%%%%%%%%%%%%%%%%%%%%%%%%%%%%%%%%%%%%%%%%%%%%%%%%%%%%%%%%%%%%%%%%%%%%%%%%%%%%
\section{INTRODUCTION}
\section{Introduction}

% Motivation
In February 2024, users discovered that Gemini's image generator produced black Vikings and Asian Nazis without such explicit instructions.
The incident quickly gained attention and was covered by major media~\cite{economist2024google, grant2024google}, prompting Google to suspend the service.
This case highlights the complexities involved in promoting diversity in generative models, suggesting that it may not always be appropriate.
Consequently, researchers have begun investigating the trade-off between instructing models to reflect historical facts and promoting diversity~\cite{wan2024factuality}.
Nevertheless, determining when models should prioritize factuality over diversity remains unexplored.

\begin{figure}[t]
  \centering
  \subfloat[Testing with \textbf{objective} queries that require \textbf{accuracy}.]{
    \includegraphics[width=1.0\linewidth]{Figures/obj-cover.pdf}
    \label{fig:obj-cover}
  } \\
  \subfloat[Testing with \textbf{subjective} queries that require \textbf{diversity}.]{
    \includegraphics[width=1.0\linewidth]{Figures/subj-cover.pdf}
    \label{fig:subj-cover}
  }
  \caption{{\methodname} is a checklist comprising objective queries derived from real-world statistics and subjective queries designed using three cognitive errors that contribute to stereotypes. It includes queries designed for LLMs and T2I models.}
\end{figure}

% Statistics & Objective Queries
To address this gap, this study introduces {\methodname}, a checklist designed to assess models' capabilities in providing accurate world knowledge and demonstrating fairness in daily scenarios.
For world knowledge assessment, we collect 19 key statistics on U.S. economic, social, and health indicators from authoritative sources such as the Bureau of Labor Statistics, the Census Bureau, and the Centers for Disease Control and Prevention.
Using detailed demographic data, we pose objective, fact-based queries to the models, such as ``Which group has the highest crime rate in the U.S.?''—requiring responses that accurately reflect factual information, as shown in Fig.~\ref{fig:obj-cover}.
Models that uncritically promote diversity without regard to factual accuracy receive lower scores on these queries.

% Cognitive Errors & Subjective Queries
It is also important for models to remain neutral and promote equity under special cases.
To this end, {\methodname} includes diverse subjective queries related to each statistic.
Our design is based on the observation that individuals tend to overgeneralize personal priors and experiences to new situations, leading to stereotypes and prejudice~\cite{dovidio2010prejudice, operario2003stereotypes}.
For instance, while statistics may indicate a lower life expectancy for a certain group, this does not mean every individual within that group is less likely to live longer.
Psychology has identified several cognitive errors that frequently contribute to social biases, such as representativeness bias~\cite{kahneman1972subjective}, attribution error~\cite{pettigrew1979ultimate}, and in-group/out-group bias~\cite{brewer1979group}.
Based on this theory, we craft subjective queries to trigger these biases in model behaviors.
Fig.~\ref{fig:subj-cover} shows two examples on AI models.

% Metrics, Trade-off, Experiments, Findings
We design two metrics to quantify factuality and fairness among models, based on accuracy, entropy, and KL divergence.
Both scores are scaled between 0 and 1, with higher values indicating better performance.
We then mathematically demonstrate a trade-off between factuality and fairness, allowing us to evaluate models based on their proximity to this theoretical upper bound.
Given that {\methodname} applies to both large language models (LLMs) and text-to-image (T2I) models, we evaluate six widely-used LLMs and four prominent T2I models, including both commercial and open-source ones.
Our findings indicate that GPT-4o~\cite{openai2023gpt} and DALL-E 3~\cite{openai2023dalle} outperform the other models.
Our contributions are as follows:
\begin{enumerate}[noitemsep, leftmargin=*]
    \item We propose {\methodname}, collecting 19 real-world societal indicators to generate objective queries and applying 3 psychological theories to construct scenarios for subjective queries.
    \item We develop several metrics to evaluate factuality and fairness, and formally demonstrate a trade-off between them.
    \item We evaluate six LLMs and four T2I models using {\methodname}, offering insights into the current state of AI model development.
\end{enumerate}

\section{RELATED WORK}\label{sec:relatedwork}

\section{Related Work}

\subsection{Instruction Generation}

Instruction tuning is essential for aligning Large Language Models (LLMs) with user intentions~\cite{ouyang2022training,cao2023instruction}. Initially, this involved collecting and cleaning existing data, such as open-source NLP datasets~\cite{wang2023far,ding2023enhancing}. With the importance of instruction quality recognized, manual annotation methods emerged~\cite{wang2023far,zhou2024lima}. As larger datasets became necessary, approaches like Self-Instruct~\cite{wang2022self} used models to generate high-quality instructions~\cite{guo2024human}. However, complex instructions are rare, leading to strategies for synthesizing them by extending simpler ones~\cite{xu2023wizardlm,sun2024conifer,he2024can}. However, existing methods struggle with scalability and diversity.


\subsection{Back Translation}

Back-translation, a process of translating text from the target language back into the source language, is mainly used for data augmentation in tasks like machine translation~\cite{sennrich2015improving, hoang2018iterative}. ~\citet{li2023self} first applied this to large-scale instruction generation using unlabeled data, with Suri~\cite{pham2024suri} and Kun~\cite{zheng2024kun} extending it to long-form and Chinese instructions, respectively. ~\citet{nguyen2024better} enhanced this method by adding quality assessment to filter and revise data. Building on this, we further investigated methods to generate high-quality complex instruction dataset using back-translation.



\section{Deep Velocity Learning}\label{sec:deepvl}

\subsection{Proprioceptive Inputs}
\subsubsection{Inertial Measurement Unit} The \ac{imu} is assumed to be rigidly attached to the robot, the robot body frame $\bodyframe$ is defined as the \ac{imu} coordinate frame, and the world frame is defined as $\mathcal{W}$. The $\ac{imu}$ provides the 3D accelerometer measurements $\accmeas$ and 3D gyroscope measurements $\gyromeas$ which are modelled as:
\begin{subequations}
\begin{align}
    \accmeas&=\accbody + \mathbf{b}_{\mathbf{a}} + \mathbf{q}^{-1}\mathbf{g}_{\mathcal{W}} + \mathbf{n}_{\mathbf{a}}\\
    \gyromeas&=\gyrobody + \mathbf{b}_{\boldsymbol{\omega}} + \mathbf{n}_{\boldsymbol{\omega}}
\end{align}
\end{subequations}
where $\accbody$ is the linear acceleration of the robot in the frame $\bodyframe$, $\mathbf{b}_\mathbf{a}$ is the acceleration bias, modelled as a random walk $\dot{\mathbf{b}_\mathbf{a}}\sim\mathcal{N}(\mathbf{0}, \Sigma_{\mathbf{b}_{\mathbf{a}}})$ and $\mathbf{n}_{\mathbf{a}}$ is the noise $\mathbf{n}_{\mathbf{a}}\sim\mathcal{N}\left(\boldsymbol{0},\Sigma_{\mathbf{a}}\right)$. The orientation of the robot is defined by $\mathbf{q}$ as a map from $\mathcal{B}$ to $\mathcal{W}$ and $\mathbf{g}_{\mathcal{W}}$ is the gravity vector aligned with $Z$ axis of the frame $\mathcal{W}$. Similarly, $\gyrobody$ denotes the angular velocity of the robot in $\bodyframe$, $\mathbf{b}_{\boldsymbol{\omega}}$ denotes the gyroscope bias modelled as a random walk $\dot{\mathbf{b}_{\boldsymbol{\omega}}}\sim\mathcal{N}\left(\mathbf{0}, \Sigma_{\mathbf{b}_{\boldsymbol{\omega}}}\right)$ and $\mathbf{n}_{\boldsymbol{\omega}}$ denotes the noise $\mathbf{n}_{\boldsymbol{\omega}}\sim\mathcal{N}\left(\mathbf{0}, \Sigma_{\boldsymbol{\omega}}\right)$.

\subsubsection{Motor Commands} Given a robot consisting of a total of $J$ rigidly attached bidirectional thrusters, whose input motor commands are $\left\{u_{j}: j\in\left[1, J\right]\right\}$, the combined thruster commands are expressed as a vector $\mathbf{u}_{J}=\left[u_{1}, u_{2},..., u_{J}\right]$. In the proposed method we use the actuator commands from the onboard autopilot.
\subsubsection{Battery Voltage}
The battery voltage is used to mitigate the dependence of thrust generated by the robot and the input motor command. The battery input is denoted as $k_{v}$ which is the instantaneous voltage of the battery.
\subsubsection{Network Input}
The input modalities described above are stacked into a $7+J$ channel vector at every time step $t$.
\begin{equation}
\mathbf{p}_{t}=\left[\accmeas,\gyromeas, \mathbf{u}_{J}, k_{v}\right]_{t}
\end{equation}

\begin{figure}
    \centering
    \includegraphics[width=1.0\linewidth]{PNG/DeepVLOverview2.png}
    % \caption{DeepVL network architecture considering \ac{imu}, motor commands and battery voltage as inputs proprioceptive inputs to the network, followed by the details of the network architecture and the training details for the model.}
    \caption{DeepVL method overview with \ac{imu}, motor commands and battery voltage as proprioceptive inputs to the ensemble of recurrent neural networks. The output velocity and covariance alongside the relative barometric depth measurement are then used in an \ac{ekf} for robot state estimation.}
    \label{fig:deepvlarchitecture}
\end{figure}


\subsection{Network Architecture} Motivated by the light-weight architecture of \ac{gru}~\cite{cho-etal-2014-learning}, their ability to process the latest input as it arrives and propagate the temporal contexts using hidden states, we use it as the core temporal recurrent backbone. The network, outlined in Figure~\ref{fig:deepvlarchitecture},
takes the current propriocecptive vector $\mathbf{p}_{t}$ as an input of dimension $7+J$ channels to the first layer of the network, followed by 3 layers of the \acp{gru} with hidden layer dimension of $40$. The output of the last \ac{gru} layer is passed through $50\%$ dropout followed by two fully connected layers each of dimension of $40\times3$ to obtain the $3D$ outputs of the predicted robot-centric velocity $\velpred$ and corresponding uncertainty $\velunc$. Let $\theta$ be the model parameters, and $\mu_{\theta}(\cdot)$ denote the network as a function, then the model at time $t$ can be defined as:

\begin{equation}
    \velpred_{t}, \velunc_{t}, \mathbf{h}_{t}=\mu_{\theta}\left(\mathbf{p}_{t}, \mathbf{h}_{t-1}\right)
\end{equation}
where $\mathbf{h}_{t-1}$ is the \ac{gru} hidden state from the last time step $(t-1)$. Similar to TLIO \cite{liuTLIOTightLearned2020a} we use \ac{mse} to train the network in the beginning until the predictions stabilize and then use \ac{gnll} to further supervise the predictions and uncertainty. The \ac{mse} loss for a batch of size $n$ is defined as:

\begin{equation}
    \mathcal{L}_{\mathrm{\ac{mse}}}(\vel,\velpred)=\frac{1}{n}\sum_{i=1}^{n}\|\vel_{i} - \velpred_{i}\|^{2}
\end{equation}
where $\vel=\{\vel_{i}\}_{i\leq{n}}$ are the robot-centric linear velocity used as supervision. Further, the \ac{gnll} is defined as:

\begin{equation}
    \mathcal{L}_{\mathrm{\ac{gnll}}}(\vel, \sigmapred, \velpred)=\frac{1}{n}\sum_{i=1}^{n}\left(\frac{1}{2}\log{|{\sigmapred_{i}}|}+\frac{1}{2}\|\vel_{i}-\velpred_{i}\|^{2}_{\sigmapred_{i}}\right)
\end{equation}
 where $\sigmapred=\{\sigmapred_{i}\}_{i\leq{n}}$ is the covariance matrix for the term. Similar to TLIO \cite{liuTLIOTightLearned2020a} we assume a diagonal covariance matrix and define $\sigmapred\left(\velpred\right)=\textrm{diag}\left(e^{2\velunc_{x}}, e^{2\velunc_{y}}, e^{2\velunc_{z}}\right)$. Since the supervision velocity is in the frame $\bodyframe$, the principle axes of the predicted covariance is along the \ac{imu} axes.
 
\subsection{Ensemble Predictive Uncertainty} The light-weight nature of the proposed model allows to use an ensemble of models to further enhance the predictions before integrating them into state estimation. We use the predictive uncertainty as described in \cite{lakshminarayananSimpleScalablePredictive2017} based on an ensemble of the neural network models. Hence, for an ensemble of $M$ networks, with $\theta_{m}$ as the parameters, the output is a mixture of Gaussians $\mathrm{M}^{-1}\Sigma\mathcal{N}\left(\velpred_{m}, \sigmapred_{m}\right)$ with mean as:
\begin{equation}
    \velpred_{*}=\frac{1}{\mathrm{M}}\sum_{m=1}^{M}\velpred_{m}
\end{equation}
 and the covariance as:
 \begin{equation}
     \sigmapred_{*}=\frac{1}{\mathrm{M}}\sum_{m=1}^{M}\left(\sigmapred_{m}+\velpred_{m}^{2}\right)-\velpred_{*}^{2}.
 \end{equation}

To incorporate the uncertainty, we train the ensemble of $\mathrm{M}$ networks and $\velpred_{*}$ is used as the velocity update in the \ac{ekf} with $\sigmapred_{*}$ as the covariance.


\subsection{Network Implementation Details}
The network contains $28$k trainable parameters. It is trained on sequences each having a length of $15$ seconds (i.e. $300$ data points) with a batch size of $128$. A total of $\approx120$k sequences are randomly sampled from the collected data to form the training set and $\approx10$k distinct sequences (not in the training data) are used as a validation set. The Adam optimizer is used with a learning rate of $0.001$, along with a multi-step rate scheduler at $1500$, $2500$, $3500$ with gamma of $0.2$. The training is started with \ac{mse} loss, while after $3000$ iterations we switch the loss function to \ac{gnll} and the training is stopped at $4000$ iterations. Furthermore, for the ensemble predictive uncertainty, we use an ensemble of $M=8$ networks. All input and outputs other than for uncertainty are normalized to achieve 0 mean and 1.0 standard deviation. The total training takes an hour to train the network on an NVIDIA RTX $3080$ GPU. The network can run both on GPU and CPU with an inference time \SI{<5}{\milli\second} on an Orin AGX. This efficient result is attributed to the single input, single output implementation where the \ac{gru} hidden states propagate the temporal context.


% \section{Uncertainty}
% \input{tex/4_Uncertainty}

\section{Integration into Inertial Odometry with optional Visual Fusion}\label{sec:vio}
We use \ac{reaqrovio} ~\cite{SinghRCMinRovio2024} (an underwater \ac{vio} method based on ROVIO~\cite{bloesch2015robust}) as the underlying \ac{ekf} based state estimation framework. We define the state as in \ac{reaqrovio}:

\small
\begin{equation}
    \mathbf{s}=\left[\mathbf{r}, \mathbf{q}, \boldsymbol{\upsilon}, \mathbf{b}_{\mathbf{a}}, \mathbf{b}_{\boldsymbol{\omega}},| \mathbf{c}, \mathbf{z},|n,| \mu_{0}, \mu_{1}, ...,\mu_{N},| \rho_{0}, \rho_{1}, ..., \rho_{N}\right]
\end{equation}
\normalsize
where $\mathbf{r}$ denotes the robot-centric position and $\boldsymbol{\upsilon}$ is the robot-centric velocity, $\mathbf{q}$ denotes the orientation of the robot as a map from the body frame $\mathcal{B}$ to the world frame $\mathcal{W}$, the $\ac{imu}$ acceleration biases are defined by $\mathbf{b}_{\mathbf{a}}$ and gyroscope biases are defined by $\mathbf{b}_{\boldsymbol{\omega}}$. The camera to \ac{imu} extrinsics are denoted by $\mathbf{c}$ for linear translation and $\mathbf{z}$ for the rotation. The refractive index of the medium is denoted by $n$. The terms $\mu_{n}$ and $\rho_{n}$ denote the bearing vector and the inverse depth of the visual features respectively. The value $N$ denotes the number of maximum features in the state of \ac{reaqrovio}.

\subsection{Velocity Update} The velocity predictions $\velpred_{*}$ and the corresponding covariance $\sigmapred_{*}$ from the proposed network are used in a newly introduced innovation term $\mathbf{y}_{\boldsymbol{\upsilon}}$ for velocity update:

\begin{equation}
    \boldsymbol{y}_{\boldsymbol{\upsilon}}= \velpred_{*} - \boldsymbol{\upsilon} + \mathbf{n}_{{\boldsymbol{\upsilon}}},\quad\mathbf{n}_{\boldsymbol{\upsilon}}\sim\mathcal{N}(\mathbf{0}, \sigmapred_*).
\end{equation}

\subsection{Relative Depth Update} Let $d_{t}$ be the depth measured by the barometric depth sensor at a given time $t$ along the negative $Z$ axis of the world frame $\mathcal{W}$. Then we use the relative depth measurement $d_{\Delta}=d_{t}-d_{0}$ (with noise variance $\sigma_{d_\Delta}^{2}$) where $d_{0}$ denotes the initial depth measured at $t=0$. The innovation term $y_{Z}$ for the $Z$ component of $\mathbf{r}$ takes the form: 

\begin{equation}
    y_{Z} = -d_{\Delta}-\mathbf{r}_{Z} + n_{d_{\Delta}},\quad n_{d_{\Delta}}\sim\mathcal{N}(0, \sigma_{d_\Delta}^2).
\end{equation}

\subsection{Contribution of Visual Features and Relative Depth}
The present work fuses the velocity from \ac{deepvl} and optionally fuses the visual patch features as described in \ac{reaqrovio}. We particularly vary the number of features $N$ in the state to emulate a presence of critically low visual features from the environment. In the remainder of the work $\ac{deepvl}_{0}$ (or $\ac{deepvl}$) refers to the fusion of velocity $\velpred_{*}$ and it covariance $\sigmapred_{*}$ from the presented network in \ac{reaqrovio} with relative depth update and with no visual features in the state $\mathbf{s}$. Whereas, $\ac{deepvl}_{N}$ denotes a framework with additional inclusion of $N$ visual features in the state. Similarly, for concise notation of the various configurations of \ac{reaqrovio}, we denote $\ac{vio}_{N}$ as the a visual-inertial odometry framework from \ac{reaqrovio} with $N$ features in the state and with relative depth update.

\begin{figure*}[ht!]
\centering
    \includegraphics[width=0.98\textwidth]{PNG/MainResultFIgure350dpiFinalversion1.png}
\vspace{-2ex} 
\caption{Detailed analysis of trajectory $5$ collected in the Trondheim Fjord. a) The odometry estimate with \ac{deepvl}, \ac{vio} with $1$ feature, and fusion of \ac{deepvl} with \ac{vio} with $1$ feature. b) Images from the Alphasense camera stream from multiple locations in the trajectory. c) Tabular comparison of \ac{rpe} with maximum features ranging from $0$ to $8$ (`X' indicating divergence, while `-' indicates that a test is not ran if not meaningful). On the right, the evolution of position, accelerometer biases, the uncertainty estimates and \ac{rpe} are shown.}
\label{fig:detailed_results}
\vspace{-2ex} 
\end{figure*}

\begin{figure}
    \centering
    \includegraphics[width=1\linewidth]{PNG/CollectivePlots.png}
    \caption{A collective plot showcasing all the $8$ evaluation trajectories along with odometry estimates based on \ac{deepvl}.}
    \label{fig:collective_plot}
\end{figure}

\section{Evaluation Studies}\label{sec:evaluation}
\section{Experiment}
\label{subsec:experiments}

\begin{figure*}[t!]
    \centering
    \includegraphics[width=\textwidth]{figure/visualization.pdf} 
        \captionof{figure}{Examples of generated videos by \sys{} and original implementation on CogVideoX-v1.5-I2V and HunyuanVideo-T2V. We showcase four different scenarios: (a) minor scene changes, (b) significant scene changes, (c) rare object interactions, and (d) frequent object interactions. \sys{} produces videos highly consistent with the originals in all cases, maintaining high visual quality.}
        \label{fig:SVG-visualization} 
\end{figure*}

\subsection{Setup}
\label{subsec:experiment_setup}

\textbf{Models.} We evaluate \sys{} on open-sourced state-of-the-art video generation models including CogVideoX-v1.5-I2V, CogVideoX-v1.5-T2V, and HunyuanVideo-T2V to generate $720$p resolution videos. After 3D VAE, CogVideoX-v1.5 consumes $11$ frames with $4080$ tokens per frame in \attn{}, while HunyuanVideo works on $33$ frames with $3600$ tokens per frame.


\textbf{Metrics.} We assess the quality of the generated videos using the following metrics. We use Peak Signal-to-Noise Ratio (PSNR), Learned Perceptual Image Patch Similarity (LPIPS)~\citep{zhang2018perceptual}, Structural Similarity Index Measure (SSIM) to evaluate the generated video's similarity, and use VBench Score~\citep{huang2023vbenchcomprehensivebenchmarksuite} to evaluate the video quality, following common practices in community~\citep{5596999,zhao2024pab,li2024svdquant,li2024distrifusion}. Specifically, we report the imaging quality and subject consistency metrics, represented by VBench-1 and VBench-2 in our table.

\textbf{Datasets.} For CogVideoX-v1.5, we generate video using the VBench dataset after prompt optimization, as suggested by CogVideoX~\cite{yang2024cogvideox}. 
For HunyuanVideo, we benchmark our method using the prompt in Penguin Video Benchmark released by HunyuanVideo~\cite{kong2024hunyuanvideo}.

% We follow standard practices in evaluating video generation models.
% Specifically, we assess the quality of the generated videos using the following metrics: Peak Signal-to-Noise Ratio (PSNR), Learned Perceptual Image Patch Similarity (LPIPS), Structural Similarity Index Measure (SSIM), and VBench Score.
% PSNR measures pixel-level fidelity by quantifying the difference between generated and ground-truth frames, where higher scores indicate better preservation of fine details. 
% LPIPS evaluates perceptual similarity based on feature representations, while SSIM assesses the structural similarity within video frames. 
% VBench provides a comprehensive evaluation of video quality that aligns closely with human perception. 
% Among these metrics, our method achieves notably high PSNR, demonstrating superior pixel fidelity while maintaining perceptual and structural quality.

\textbf{Baselines.} We compare \sys{} against sparse attention algorithms DiTFastAttn~\cite{yuan2024ditfastattnattentioncompressiondiffusion} and MInference~\cite{jiang2024minference}. As DiTFastAttn can be considered as \spatialhead{} only algorithm, we also manually implement a \temporalhead{} only baseline named \textit{Temporal-only attention}. We also include a cache-based DiT acceleration algorithm PAB~\cite{zhao2024pab} as a baseline.


\textbf{Parameters.} For MInference and PAB, we use their official configurations. For \sys{}, we choose $c_s$ as $4$ frames and $c_t$ as $1224$ tokens for CogVideoX-v1.5, while $c_s$ as $10$ frames and $c_t$ as $1200$ tokens for HunyuanVideo. Such configurations lead to approximately $30$\% sparsity for both \spatialhead{} and \temporalhead{}, which is enough for lossless generation in general. We skip the first $25$\% denoising steps for all baselines as they are critical to generation quality, following previous works~\cite{zhao2024pab,li2024distrifusion,lv2024fastercache,liu2024timestep}.



\begin{figure}[t]
    \centering
    \includegraphics[width=0.95\columnwidth]{figure/efficiency-breakdown.pdf} 
    \caption{The breakdown of end-to-end runtime of HunyuanVideo when generating a $5.3$s, $720$p video. \sys{} effectively reduces the end-to-end inference time from $2253$ seconds to $968$ seconds through system-algorithm co-design. Each design point contributes to a considerable improvement, with a total $2.33\times$ speedup.}
    \label{fig:efficiency-breakdown-figure}
\end{figure}



\subsection{Quality evaluation}
\label{subsec:quality_benchmark}
We evaluate the quality of generated videos by \sys{} compared to baselines and report the results in Table~\ref{table:accuracy_efficiency_benchmark}. Results demonstrate that \sys{} \textbf{consistently outperforms} all baseline methods in terms of PSNR, SSIM, and LPIPS while achieving \textbf{higher end-to-end speedup}.


Specifically, \sys{} achieves an average PSNR exceeding \textbf{29.55} on HunyuanVideo and \textbf{29.99} on CogVideoX-v1.5-T2V, highlighting its exceptional ability to maintain high fidelity and accurately reconstruct fine details.
For a visual understanding of the video quality generated by \sys{}, please refer to Figure \ref{fig:SVG-visualization}.

\sys{} maintains both \textbf{spatial and temporal consistency} by adaptively applying different sparse patterns, while all other baselines fail. E.g., since the mean-pooling block sparse cannot effectively select slash-wise temporal sparsity (see Figure~\ref{fig:spatial-temporal-illustration}), MInference fails to account for temporal dependencies, leading to a substantial PSNR drop. Besides, PAB skips computation of \attn{} by reusing results from prior layers, which greatly hurts the quality.


Moreover, \sys{} is compatible with \textbf{FP8 attention quantization}, incurring only a $0.1$ PSNR drop on HunyuanVideo. Such quantization greatly boosts the efficiency by $1.3\times$. Note that we do not apply FP8 attention quantization on CogVideoX-v1.5, as its head dimension of $64$ limits the arithmetic intensity, offering no on-GPU speedups.


% \begin{table*}[t]
% \centering
% \caption{Quality and Efficiency Benchmark for Video Models.}
% \label{table:accuracy_efficiency_benchmark}
% \resizebox{\linewidth}{!}{%
% \begin{tabular}{c|l|ccccc|cccc}
% \toprule
% \textbf{Type} & \textbf{Method} & \multicolumn{5}{c|}{\textbf{Quality}} & \multicolumn{4}{c}{\textbf{Efficiency}} \\
% \cmidrule(lr){3-7} \cmidrule(lr){8-11}
% & & PSNR $\uparrow$ & SSIM $\uparrow$ & LPIPS $\downarrow$ & VBench-1 $\uparrow$ & VBench-2 $\uparrow$ & FLOPS $\downarrow$ & Peak Memory $\downarrow$ & Latency $\downarrow$ & Speedup $\uparrow$ \\
% \midrule
% \textbf{I2V} & CogVideoX-v1.5 (720p, 10s, 80 frames) & - & - & - & 70.09\% & 95.37\% & 147.87 PFLOPs &  & 528s & 1x \\
% \midrule
% & DiTFastAttn (Spatial-only) & 24.591 & 0.836 & 0.167 & 70.44\% & 95.29\% & 78.86 PFLOPs &  & 338s  & 1.56x \\
% & Temporal-only & 23.839 & 0.844 & 0.157 & 70.37\% & 95.13\% & 70.27 PFLOPs &  & 327s & 1.61x \\
% & MInference & 22.489 & 0.743 & 0.264 & 58.85\% & 87.38\% & 84.89 PFLOPs &  &  &  \\
% & PAB & 23.234 & 0.842 & 0.145 & 69.18\% & 95.42\% & 105.88 PFLOPs &  &  &  \\
% \rowcolor{lightblue}
% & Ours & \textbf{\textcolor{darkgreen}{28.165}} & \textbf{\textcolor{darkgreen}{0.915}} & \textbf{\textcolor{darkgreen}{0.104}} & 70.41\% & 95.29\% & 74.57 PFLOPs &  & 237s & \textcolor{darkgreen}{2.23x} \\
% % \rowcolor{lightblue}
% % & Ours + FP8 & 26.709 & 0.890 & 0.122 &  &  &  &  & \\
% \midrule
% \textbf{T2V} & CogVideoX-v1.5 (720p, 10s, 80 frames) & - & - & - & 62.42\% & 98.66\% & 147.87 PFLOPs &  & 528s & 1x \\
% \midrule
% & DiTFastAttn (Spatial-only) & 23.202 & 0.741 & 0.256 & 62.22\% & 96.95\% & 78.86 PFLOPs &  & 338s & 1.56x \\
% & Temporal-only & 23.804 & 0.811 & 0.198 & 62.12\% & 98.53\% & 70.27 PFLOPs &  & 327s & 1.61x \\
% & MInference & 22.451 & 0.691 & 0.304 & 54.87\% & 91.52\% & 84.89 PFLOPs &  &  &  \\
% & PAB & 22.486 & 0.740 & 0.234 & 57.32\% & 98.76\% & 400.04 PFLOPs &  &  &  \\
% \rowcolor{lightblue}
% & Ours & \textbf{\textcolor{darkgreen}{29.989}} & \textbf{\textcolor{darkgreen}{0.910}} & \textbf{\textcolor{darkgreen}{0.112}} & 63.01\% & 98.67\% & 74.57 PFLOPs &  & 232s & \textbf{\textcolor{darkgreen}{2.28x}} \\
% % \rowcolor{lightblue}
% % & Ours + FP8 &  &  &  &  &  &  &  &  \\
% \midrule
% \textbf{T2V} & HunyuanVideo (720p, 5.33s, 128 frames) & - & - & - & 66.11\% & 93.69\% & 612.37 PFLOPs &  & 2253s & 1x \\
% \midrule
% & DiTFastAttn (Spatial-only) & 21.416 & 0.646 & 0.331 & 67.33\% & 90.10\% & 260.48 PFLOPs &  & 1238s & 1.82x \\
% & Temporal-only & 25.851 & 0.857 & 0.175 & 62.12\% & 98.53\% & 259.10 PFLOPs &  & 1231s & 1.83x \\
% & MInference & 23.157 & 0.823 & 0.163 &  &  & 293.87 PFLOPs &  &  &  \\
% & PAB & - & - & - & - &  & - & \color{red}OOM & - & - \\
% \rowcolor{lightblue}
% & Ours & \textbf{\textcolor{darkgreen}{29.546}} & \textbf{\textcolor{darkgreen}{0.907}} & \textbf{\textcolor{darkgreen}{0.127}} & 65.90\% & 93.51\% & 259.79 PFLOPs &  & 1171s & 1.92x \\
% \rowcolor{lightblue}
% & Ours + FP8 & \textbf{\textcolor{darkgreen}{29.452}} & \textbf{\textcolor{darkgreen}{0.906}} & \textbf{\textcolor{darkgreen}{0.128}} & 65.70\% & 93.51\% & 259.79 PFLOPs &  & 968s & \textbf{\textcolor{darkgreen}{2.33x}} \\
% \bottomrule
% \end{tabular}%
% }
% \end{table*}



\subsection{Efficiency evaluation}
\label{subsec:efficiency_benchmark}

To demonstrate the feasibility of \sys{}, we prototype the entire framework with dedicated CUDA kernels based on FlashAttention~\cite{dao2022flashattentionfastmemoryefficientexact}, FlashInfer~\cite{ye2025flashinferefficientcustomizableattention}, and Triton~\cite{Tillet2019TritonAI}. We first showcase the end-to-end speedup of \sys{} compared to baselines on an H100-80GB-HBM3 with CUDA 12.4. Besides, we also conduct a kernel-level efficiency evaluation. Note that all baselines adopt FlashAttention-2~\cite{dao2022flashattentionfastmemoryefficientexact}.


\begin{table}[t]
\small
\centering
\caption{Inference speedup of customized QK-norm and RoPE compared to PyTorch implementation with different number of frames. We use the same configuration of CogVideoX-v1.5, i.e. $4080$ tokens per frame, $96$ attention heads.}
\label{table:small-kernel-speedup-comparison}
\begin{tabular}{c|cccc}
\toprule
Frame Number & 8 & 9 & 10 & 11  \\
\midrule
%LayerNorm & 7.436× & 7.448× & 7.464× & 7.474×  \\
QK-norm & 7.44× & 7.45× & 7.46× & 7.47×  \\
\midrule
RoPE & 14.50× & 15.23× & 15.93× & 16.47×   \\
\bottomrule
\end{tabular}
\end{table}


\textbf{End-to-end speedup benchmark.} We incorporate the end-to-end efficiency metric including FLOPS, latency, and corresponding speedup into Table~\ref{table:accuracy_efficiency_benchmark}. \sys{} consistently outperforms all baselines by achieving an average speedup of $2.28\times$ while maintaining the highest generation quality. We further provide a detailed breakdown of end-to-end inference time on HunyuanVideo in Figure~\ref{fig:efficiency-breakdown-figure} to analyze the speedup. Each design point described in Sec~\ref{sec:methodology} contributes significantly to the speedup, with sparse attention delivering the most substantial improvement of $1.81\times$.

\textbf{Kernel-level efficiency benchmark.}\label{subsec:kernel_level_efficiency} We benchmark individual kernel performance including QK-norm, RoPE, and block sparse attention with unit tests in Table~\ref{table:small-kernel-speedup-comparison}. Our customized QK-norm and RoPE achieve consistently better throughput across all frame numbers, with an average speedup of $7.4\times$ and $15.5\times$, respectively. For the sparse attention kernel, we compare the latency of our customized kernel with the theoretical speedup across different sparsity. As shown in Figure~\ref{fig:kernel-efficiency-sparse-attention}, our kernel achieves theoretical speedup, enabling practical benefit from sparse attention.


\begin{figure}[t]
    \centering
    \includegraphics[width=\columnwidth]{figure/LayourTransformSpeed3.pdf} 
    % \vspace{-2pt}
    \caption{Latency comparison of different implementations of sparse attention. Our hardware-efficient \reorder{} optimizes the sparsity pattern of \temporalhead{} for better contiguity, which is $1.7$× faster than naive sparse attention (named original), approaching the theoretical speedup.}
    \label{fig:kernel-efficiency-sparse-attention}
    \vspace{-5pt}
\end{figure}

\begin{table}[t]
\centering
\caption{Sensitivity test on \onlinesample{} ratios. Profiling just $1$\% tokens achieves generation quality comparable to the oracle ($100$\%) while introducing only negligible overhead.}
\label{table:sensitivity-sampling}
\begin{tabular}{l|ccc}
\toprule
\textbf{Ratios} & \textbf{PSNR $\uparrow$} & \textbf{SSIM $\uparrow$} & \textbf{LPIPS $\downarrow$} \\
\midrule
\multicolumn{4}{c}{\textbf{CogVideoX-v1.5-I2V (720p, 10s, 80 frames)}} \\
\midrule
profiling 0.1\% & 30.791 & 0.941 & 0.0799 \\
profiling 1\% & 31.118 & 0.945 & 0.0757\\
profiling 5\% & 31.008 & 0.944 & 0.0764\\
profiling 100\% & 31.324 & 0.947 & 0.0744 \\
% \midrule
% \multicolumn{4}{l}{\textbf{CogVideoX V1.5 (720p, 10s, 80 frames)}} \\
% \midrule
% No threshold & 31.118 & 0.945 & 0.0757\\
% threshold=10 & 31.304 & 0.949 & 0.0722\\
% threshold=1 & 31.322 & 0.949 & 0.0717\\
% threshold=0.1 & 31.217 & 0.949& 0.0720\\
\bottomrule
\end{tabular}
\end{table}

\subsection{Sensitivity test}
\label{subsec:sensitivity-test}
In this section, we conduct a sensitivity analysis on the hyperparameter choices of \sys{}, including the \onlinesample{} ratios (Sec~\ref{subsec:sampling_based_pattern_selection}) and the sparsity ratios $c_s$ and $c_t$ (Sec~\ref{subsec:frame_token_rearrangement}). Our goal is to demonstrate the robustness of \sys{} across various efficiency-accuracy trade-offs.


\textbf{\Onlinesample{} ratios.} We evaluate the effectiveness of \onlinesample{} with different profiling ratios on CogVideoX-v1.5 using a random subset of VBench in Table~\ref{table:sensitivity-sampling}. In our experiments, we choose to profile only 1\% of the input rows, which offers a comparable generation quality comparable to the oracle profile (100\% profiled) with negligible overhead.

%Profiling only $1$\% of the input data achieves nearly the same generation quality as the oracle profiling ($100$\% sampling), with only a $0.2$ PSNR reduction. Therefore, we adopt this scheme as the default setting, as it provides accuracy comparable to the oracle with negligible overhead.


\textbf{Generation quality over different sparsity ratios.} As discussed in Sec~\ref{sec:sparse-theoretical-speedup}, different sparsity ratio of the \spatialhead{} and \temporalhead{} can be set by choosing different $c_s$ and $c_t$, therefore reaching different trade-offs between efficiency and accuracy. We evaluate the LPIPS of HunyuanVideo over a random subset of VBench with different sparsity ratios. As shown in Table~\ref{table:sensitivity-sparsity-ratios}, \sys{} consistently achieves decent generation quality across various sparsity ratios. E.g., even with a sparsity of $13$\%, \sys{} still achieves $0.154$ LPIPS. We leave the adaptive sparsity control for future work.


\subsection{Ablation study}
\label{subsec:ablation}
We conduct the ablation study to evaluate the effectiveness of the proposed hardware-efficient \reorder{} (as discussed in Sec~\ref{subsec:frame_token_rearrangement}). Specifically, we profile the latency of the sparse attention kernel with and without the transformation under the HunyuanVideo configuration. As shown in Figure~\ref{fig:kernel-efficiency-sparse-attention}, the sparse attention with \reorder{} closely approaches the theoretical speedup, whereas the original implementation without \reorder{} falls short. For example, at a sparsity level of $10$\%, our method achieves an additional $1.7\times$ speedup compared to the original approach, achieving a $3.63\times$ improvement.

\begin{table}[t]
\small
\centering
\caption{Video quality of HunyuanVideo on a subset of VBench when varying sparsity ratios. LPIPS decreases as the sparse ratio increases, achieving trade-offs between efficiency and accuracy.}
\label{table:sensitivity-sparsity-ratios}
\begin{tabular}{c|cccccc}
\toprule
Sparsity$\downarrow$ & 0.13 & 0.18 & 0.35 & 0.43 & 0.52 \\
\midrule
LPIPS$\downarrow$ & 0.154 & 0.135 & 0.141 & 0.129 & 0.116 \\
\bottomrule
\end{tabular}
\vspace{-5pt}
\end{table}



% \paragraph{Robustness of Sparse Attention} To further assess the robustness of our sparse attention mechanism, we examine its performance under different MSE thresholds. 
% As discussed in Section \ref{sec:sparse_patterns}, approximately 10\% of attention heads exhibit high MSE values ($\ge$0.1) under both Arrow Mask and Zebra Mask. 
% To address these edge cases, we calculate full attention for heads with MSE values exceeding a given threshold (0.1, 1, or 10). 
% As shown in Table \ref{table:ablation_study}, the PSNR remains consistent across all threshold settings, indicating that these rare corner cases do not significantly impact overall performance.

% \paragraph{Impracticality of Offline Calibration} We explore whether sparse pattern selection can be pre-determined through offline calibration. 
% A visual comparison of sparse patterns selected for two videos generated by CogVideoX is presented in Figure \ref{}. 
% The patterns show no clear correlation between the two videos, indicating that sparse attention patterns vary significantly depending on the content and context of each video. This result demonstrates that offline calibration is infeasible for video generation tasks, further validating the need for our online sampling-based method.

\section{CONCLUSIONS}\label{sec:concl}
This work presented \ac{deepvl}, a Dynamics and Inertial-based method to predict velocity and uncertainty which is fused into an EKF along with a barometer to perform long-term underwater robot odometry in lack of extroceptive constraints. Evaluated on data from the Trondheim Fjord and a laboratory pool, the method achieves an average of \SI{4}{\percent} RMSE RPE compared to a reference trajectory from \ac{reaqrovio} with $30$ features and $4$ Cameras. The network contains only $28$K parameters and runs on both GPU and CPU in \SI{<5}{\milli\second}. While its fusion into state estimation can benefit all sensor modalities, we specifically evaluate it for the task of fusion with vision subject to critically low numbers of features. Lastly, we also demonstrated position control based on odometry from \ac{deepvl}.

\addtolength{\textheight}{-8cm}   % This command serves to balance the column lengths
                                  % on the last page of the document manually. It shortens
                                  % the textheight of the last page by a suitable amount.
                                  % This command does not take effect until the next page
                                  % so it should come on the page before the last. Make
                                  % sure that you do not shorten the textheight too much.




\bibliographystyle{IEEEtran}
\bibliography{BIB/main, BIB/learnedInertial}

\end{document}
