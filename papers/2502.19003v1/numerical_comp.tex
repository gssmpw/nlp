%
%
\section{Numerical tests for the coupling conditions and the discrete conservation property}
\label{sec:tests}
%
%

In order to test numerical schemes, it is useful to have a problem with exact solutions. 
For the diffusion equation $w_t-Dw_{xx} =0$, one can easily generate solutions
by separation of variables, see. e.g.\ John \cite[Section 7.1]{bJOH}.
For us, an interesting family of solutions on the interval $[0,1]$ are
%
$$
w_n(x,t)=\e^{-D(n\pi)^2 t}\cos \left(n\pi x\right)+1
$$
%
for $n=1,2,3,\ldots$. They satisfy the homogeneous Neumann boundary
condition. We take $n=1$ then the initial data are $w(x)=\cos \left(\pi x\right)+1$ 
with initial total mass concentration $\int_{0}^{1} [\cos(\pi x)+1] \,dx=1$.
Due to the homogeneous Neumann boundary conditions, the total mass concentration 
remains unchanged for the exact solution. 
The solution will approach the constant $1$ asymptotically in time. So, 
total mass concentration is a good indicator for finding flaws in the code.
Using this solution, we validated our discretizations and the numerical Dirichlet-Neumann coupling conditions. 
The latter should reproduce the same exact solution up to rounding errors in a bi-domain computation. 
In our computations, this was indeed the case.

All our computations were done with {\sc Matlab} R2023b on the interval $[0,1]$ with double precision. 
We chose the spatial mesh size and set the time step to be in the relation
$\Delta t =\frac{0.4(\Delta x)^2}{\max (D_-,D_+)}$, i.e.\ $\nu_\pm\le 0.4$, 
to have a safety margin towards the stability bound at $0.5$.

%
%
\subsection{The coupling conditions}
%
%

The Dirichlet-Neumann coupling and the heat flux coupling are well established. 
The biophysical channel and pumping conditions are slightly
different. So, for comparison, we show how they act using a few examples.
We took the initial data $\cos(\pi x)+1$ as well as the diffusion coefficients $D_-=0.1$ and $D_+=1$.
Shown in Figure \ref{comp1} are the initial data and
six computations with the nodal based scheme. 

We used the coupling conditions \eqref{f16} and \eqref{giless1} together with 
the Dirichlet condition $u_m^{n+1}=v_m^{n+1}$ and then \eqref{2001}
with the heat flux coefficients $H=1$ and $H=0.1$. Further, we took the channel pumping with $\psi = 9.3954\text{E}-7$,
$\alpha = 1.497$, $\beta = 1.1949\text{E}-04$, $\gamma = 1.1556\text{E}-07$ and $\delta = 1.1444\text{E}-07$. 
For the membrane pumping we chose $P_l=0.02$, $P_p=1$ and $K_d=0.2$.
The values were almost all taken from a table in Thul and Falcke \cite{l11}. 
These authors require $\psi = 9.3954$ and $P_p =40$. The values correspond to a larger heat flux.
But this leads to a bad scaling within the figure. So, we modified them for the sake of our comparison.
The higher values were used for local pumping in a 3 dimensional model where the concentration spreads out and becomes more diluted.
A one dimensional model needs only a smaller value for the coefficients in order to produce the same effect.

%
\begin{figure}
%
\begin{subfigure}{.49\textwidth}
  \centering
  \includegraphics[width=1 \linewidth]{comp1_new.pdf}
  \caption{}
  %\label{fig:sfig1}
\end{subfigure}
%
\begin{subfigure}{.49\textwidth}
  \centering
  \includegraphics[width=1 \linewidth]{comp1_zoom_new.pdf}
  \caption{}
  %\label{fig:sfig2}
\end{subfigure}
%
\caption{\textbf{Bi-domain computations with six couplings:} 
\label{comp1}
The interval [0,1] is divided into 100 sub-intervals of length $\Delta x=0.01$. 
There were $3000$ time steps of length $\Delta t=4\cdot 10^{-5}$, giving a final time $T=0.12$. 
The figure on the right is a zoom into the coupling region.}
%
\end{figure}

We made the spatial mesh rather course with $\Delta x=0.01$ in order to highlight differences in the couplings.
In the zoom, we cut off the extreme values of the membrane coupling to show the other
solutions more clearly. To the left of $x=0.5$ from top to bottom, the solutions are with
membrane, then channel coupling, the initial data, heat flux coupling with $H=0.1$, then $H=1$, and finally Dirichlet-Neumann and Giles coupling.
The latter two are continuous. To the right, the order of the discontinuous solutions is reversed
and they all are below the continuous couplings.

We see in the computational results that the solution approaches a quasi constant state 
faster on the right sub-interval due to the larger diffusion coefficient. 
Zooming in even further into the Graph of the Dirichlet-Neumann coupling would show
that the Giles formula does deviate from the correct coupling.
The nodal values at $x=0.5$ differ by $2.730344211099967\text{E}-03$.
The smaller heat flux coefficient of $H=0.05$ models a less permeable membrane than the case $H=1$. 
So, the overall equilibrium value $u(x)=v(x)=1$ is approached much slower. 

The pumping goes from right to left, increasing the concentration on the left hand side 
at the expense of the right hand side. 
The small value of $H=0.1$ represents a slower flux through the membrane
at $x=0.5$ than with $H=1$.
So, the left-hand side is loosing mass at a slower rate.

%
\begin{figure}
%
\begin{subfigure}{.49\textwidth}
  \centering
  \includegraphics[width=1 \linewidth]{comp7_zoom_new.pdf}
  \caption{}
  %\label{fig:sfig1}
\end{subfigure}
%
\begin{subfigure}{.49\textwidth}
  \centering
  \includegraphics[width=1 \linewidth]{comp6_zoom_new.pdf}
  \caption{}
  %\label{fig:sfig2}
\end{subfigure}
%
\caption{\textbf{Bi-domain computations with negative heat fluxes:} 
\label{comp5}
The interval [0,1] is divided into 100,000 sub-intervals of length $\Delta x=0.00001$. 
There were $10^6$ time steps of length $\Delta t=4\cdot 10^{-11}$, giving a final time $T=0.00004$. The figure on the left 
is a zoom into the coupling region for the cosine initial data.
The figure on the right is a zoom into the upper part of the coupling region
for the piecewise constant initial data.}
%
\end{figure}

For comparison of the influence of the heat flux coefficients in the coupling, 
we also made two computations with negative heat flux coefficients using $H=-1$, $H=-0.01$, $\psi = -9.3954\text{E}-7$
and $P_l=-0.02$. Figure \ref{comp5} shows two zooms into the solutions.
Negative heat flux coefficients mean that there is no natural heat or concentration flux 
against the gradient but an active flux mechanism working
in the direction of the gradient. We used the finer mesh, which we will be using below. The first computation 
is with the above cosine initial data. The second has piecewise constant initial data
with $u(x)=1$ and $v(x)=0.06$, cp.\ Figure \ref{comp3}. In Figure \ref{comp5}(a), to the left of the interface from top
to bottom are the membrane coupling, $H=-1$, $H=-0.1$, and channel coupling solutions. The latter three are quite
close to each other. In Figure \ref{comp5}(a), we have the order of solutions $H=-1$, membrane coupling, $H=-0.1$, and channel
coupling. The latter is close to the initial data. We clearly see that in each case more heat or mass 
is moved from right to left as compared to the other computations with positive coefficients.

%
\begin{figure}
%
\begin{subfigure}{.49\textwidth}
  \centering
  \includegraphics[width=1 \linewidth]{comp2_new.pdf}
  \caption{}
  %\label{fig:sfig1}
\end{subfigure}
%
\begin{subfigure}{.49\textwidth}
  \centering
  \includegraphics[width=1 \linewidth]{comp2_zoom_new.pdf}
  \caption{}
  %\label{fig:sfig2}
\end{subfigure}
%
\caption{\textbf{Bi-domain computations with six couplings:} 
\label{comp2}
The interval [0,1] is divided into 10,000 sub-intervals of length $\Delta x=0.0001$. 
There were $10^5$ time steps of length $\Delta t=4\cdot 10^{-9}$, giving a final time $T=0.0004$. 
The figure on the right is a zoom into the coupling region.}
%
\end{figure}

%
%
\subsection{Discrete mass conservation of coupling conditions}
%
%

We made the same computation as in the previous section on a finer mesh with $\Delta x = 0.0001$ and $10^5$ time steps 
of size $\Delta t=4\cdot 10^{-9}$ in order to
demonstrate the mass conservation property. 
The solutions are shown in Figure \ref{comp2}. 
The extreme values at $x=0.5$ are taken by the membrane pumping solution. 
The channel pumping solution is on the left hand side only slightly above and on the right hand side
below the $H=0.1$ heat flux solution. Then come the $H=1$ heat flux solution 
and the continuous values of the initial data.
Due to the resolution of the figure, the graph of the
Dirichlet-Neumann coupling solutions are hidden behind the one using 
the Giles coupling. They are optically quite close. 
But, their difference in solution value at $x=0.5$ is $8.133493461626173\text{E}-05$.

We had a discrete initial total mass concentration of
$C(0)=1.000000000000001$. Then, we obtained the final values 
$C(T)=0.9999975790344722$ for the correct Dirichlet-Neumann coupling and 
$C(T)=0.9980606224422512$ for the Giles coupling. All of these were computed using \eqref{dnc2}.
The computational error of the correct coupling was $|C(T)-C((0)|=4.432343381211012\text{E}-12$. 
We clearly see that the Giles coupling produces 
a much larger error. It is $|C(T)-C(0)|=2.420965528937558\text{E}-06$ by losing mass.
The heat flux couplings gave $C(T) = 0.9999999999955704$
and $C(T)= 9.999999999955695$. For the pumping the values were $C(T) = 0.9999999999955702$ 
and $C(T) = 0.9999999999955707$. Except for the Giles coupling, the values 
are very reasonable deviations of order $10^{-12}$ from the exact value $1$.

%
\begin{figure}
%
\begin{subfigure}{.49\textwidth}
  \centering
  \includegraphics[width=1 \linewidth]{comp3_new.pdf}
  \caption{}
  %\label{fig:sfig1}
\end{subfigure}
%
\begin{subfigure}{.49\textwidth}
  \centering
  \includegraphics[width=1 \linewidth]{comp3_zoom_new.pdf}
  \caption{}
  %\label{fig:sfig2}
\end{subfigure}
%
\caption{\textbf{Bi-domain computations with six couplings:} 
\label{comp3}
The interval [0,1] is divided into 100,000 sub-intervals of length $\Delta x=0.00001$. 
There were $10^6$ time steps of length $\Delta t=4\cdot 10^{-11}$, giving a final time $T=0.00004$. 
The figure on the right is a zoom into the upper part of the coupling region.}
%
\end{figure}

Next, we took piecewise constant initial data with $u(x)=1$ and $v(x)=0.06$, see Figure \ref{comp3}.
The exact initial total concentration is $0.53$. The initial discretized total 
concentration was $C(0)=0.5300000000000005$. 
For the Dirichlet-Neumann coupling we obtained $C(T)=0.5300009399993384$
and for the Giles coupling $C(T)=0.5299973682972777$. The results for 
the heat fluxes were $C(T)=0.5299999999993753$, $C(T)=0.5299999999994137$
and for the pumping couplings $C(T)=0.5299999999994603$, $C(T)=0.5299999999993950$.
The Dirichlet-Neumann coupling has a larger computational error of $|C(T)-C(0)|=9.399993379233251\text{E}-07$
then the other correct couplings, which are of the order $10^{-13}$. 
This seems to be caused initially by the Dirichlet condition
which does not do well with the initial discontinuity. With an error
of $|C(T)-C(0)|=2.631702722744045\text{E}-06$, the error in the Giles coupling is not
so pronounced in the example.

We did the same computation taking the non-conservative one-sided differences
giving \eqref{300} and \eqref{3001} for the flux couplings. We obtained for the heat fluxes
$C(T)=0.5300000706650079$, $C(T)=0.5300000706650079$ and for the pumping fluxes
$C(T)=0.5300000005493521$, $C(T)=0.5299999951641959$. The computational error
$|C(T)-C(0)|$ is considerably higher with the non-conservative differences. The order
is $10^{-8}$ for the heat fluxes, $10^{-10}$ for the channel pumping flux and
$10^{-9}$ for the membrane pumping flux. The plots are not much different from Figure \ref{comp3}.



%
%
\subsection*{The finite volume type discretization}
%
%


%
\begin{figure}
%
\begin{subfigure}{.49\textwidth}
  \centering
  \includegraphics[width=1 \linewidth]{comp4_new.pdf}
  \caption{}
  %\label{fig:sfig1}
\end{subfigure}
%
\begin{subfigure}{.49\textwidth}
  \centering
  \includegraphics[width=1 \linewidth]{comp4_zoom_new.pdf}
  \caption{}
  %\label{fig:sfig2}
\end{subfigure}
%
\caption{\textbf{Finite volume type bi-domain computations with six couplings:} 
\label{comp4}
The interval [0,1] is divided into 100,000 sub-intervals of length $\Delta x=0.00001$. 
There were $10^6$ time steps of length $\Delta t=4\cdot 10^{-11}$, giving a final time $T=0.00004$. 
The figure on the right is a zoom into the upper part of the coupling region.}
%
\end{figure}


Finally, we redid the preceding computations with the piecewise constant initial data 
using the same parameters with the finite volume type discretization
that was used for the nodal based scheme, see Figure \ref{comp4}.
The nodal points were shifted, the mass conserving flux coupling formulas
\eqref{300} and \eqref{3001} as well as boundary condition
formulas \eqref{onesided_bdry} were used. Further, the discrete total mass concentration was
computed using \eqref{fv_cons}. The initial discrete total mass concentration was again $C(0)=0.5300000000000005$.
The Dirichlet-Neumann coupling gave a much better result of $C(T)=0.5299999999993383$ with an error of
$|C(T)-C(0)|=6.621370118864434\text{E}-13$. Here, we see that the node at the initial discontinuity caused the large
error in the nodal based scheme. The Giles coupling coupling gives 
$C(T)= 0.5299964295714319$ with an error of  $|C(T)-C(0)|=3.570428568577810e-06$. 
Again, it is much larger than the correct coupling. The heat flux couplings give $C(T)=0.5299999999993757$,
$C(T)=0.5299999999994144$ and the pumping couplings $C(T)=0.5299999999994608$,
$C(T)=0.5299999999993955$. All computational errors are of size $10^{-13}$
as before.

Some further computational results for the conservation property were given in Munir \cite[Section 5.5]{MUN}. 

%
%
\subsection{The homogeneous Neumann boundary condition}
%
%

For the discrete mass conservation, let us first look at the homogeneous Neumann boundary conditions
\eqref{onesided_bdry} and \eqref{central_bdry}. We mentioned that the
former is good with the finite volume type scheme and the latter with the nodal-based scheme. 
The error when taking the non-conservative discretization at
the boundary is most pronounced when the solution has a steep gradient
at the boundary. We want to show this effect.

We did some computations for a simple single-domain problem with the nodal-based scheme with $D=0.001$. 
As initial data we took the function $u(x) = 100\sqrt{x(1-x)}$ for $x\in [0,1]$
with the initial total mass concentration $C=100\int_{0}^{1} \sqrt{x(1-x)}\,dx=39.269908169872415$.

First, we took the boundary condition \eqref{central_bdry}.
We ran three computations, the first with $\Delta x = 0.01$ and 500 time steps
of size $\Delta t = 0.04$ to the final time $T=20$.
The total approximated initial concentration was $C(0)=39.22835638873124$
with a large initial dicretization error of $|C(T)-C(0)|= 4.155178114117319\text{E}-02$. As the final
total concentration, we obtained $C(T) =39.22835638873113$. Only the last two decimals
are changed, giving a computational error of $|C(T)-C(0)|=1.136868377216160\text{E}-13$. Here, the initial error heavily outweighs the computational error.

Next we used $\Delta x = 0.001$ with $50.000$ time steps of size $\Delta t = 0.0004$ to
the final time $T=20$.
We obtained $C(0) = 39.26859346252676$ with an initial discretization error of $|C(T)-C(0)|=1.314707345656529\text{E}-03$. The final total concentration was
$C(T)=39.26859346251628$. 
\text{E}ven though the initial total concentration is
more precise due to the finer spatial mesh, we have a higher computational error of
$|C(T)-C(0)|=1.048050535246148\text{E}-11$. We see the effect of error accumulation due to the
many time steps. 

We also used a much finer mesh of $\Delta x = 10^{-7}$ to get 
a better discretization of the initial values.
We had a time step of $\Delta t = 4\cdot 10^{-12}$ due to the stiffness of the problem and took $10^5$ time steps to the
final time $T=4\cdot 10^{-7}$. We have the initial total
concentration $C(0)=39.26990816855763$ with an initial discretization error of $|C-C(0)|=1.314788278250489e-09$ and the final
total concentration $C(T)=3.926990816855260$ with a computational error of $|C(T)-C(0)|=5.023537141823908\text{E}-12$.

We repeated these computations with the one-sided differences \eqref{onesided_bdry} at both boundaries. The initial errors were the same.
In the first case, we have $C(T)=38.91134647144038$, giving a computational error of $|C(T)-C(0)|=0.3170099172908607$, 
which is very large compared to the result with the conservative boundary condition.
On the finer mesh, we obtained $C(T)=39.23609630251410$ with a slightly
smaller computational error of $|C(T)-C(0)|= 0.03249716001266023$ due to the finer mesh. On the very fine mesh 
at the smaller final time, we obtained $C(T)=39.26990812490996$ with
a computational error $|C(T)-C(0)|=4.364766681419496\text{E}-08$. It is clearly larger than in the case of the conservative boundary condition.

