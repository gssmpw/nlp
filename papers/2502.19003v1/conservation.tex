%
%
\section{The discrete conservation property for bi-domain diffusion equations}
\label{sec:cons}
%
%

Consider our coupling problem \eqref{eqn1} with homogeneous Neumann boundary data.
The total amount of concentration at any time $t$ is $C(t) =\int_0^{\frac 12}u(s,t)\d s+\int_{\frac 12}^1v(s,t)\d s$.
Differentiating, then using the equations \eqref{eqn1} and the boundary conditions, we obtain
\begin{eqnarray*}
\frac \d{\d t} C(t) &=&\int_0^{\frac 12}\frac\partial{\partial t}u(s,t)\d s+\int_{\frac 12}^1\frac\partial{\partial t}v(s,t)\d s
=D_-\int_0^{\frac 12}\frac{\partial^2}{\partial s^2}u(s,t)\d s+D_+\int_{\frac 12}^1\frac{\partial^2}{\partial s^2}v(s,t)\d s\\
&=& D_- \frac{\partial}{\partial x}u(\frac 12,t) -D_+\frac{\partial}{\partial x}v(\frac 12,t) =0
\end{eqnarray*}
%
if the coupling condition $D_- \frac{\partial}{\partial x}u(\frac 12,t) =D_+\frac{\partial}{\partial x}v(\frac 12,t)$ holds.
This is true for all the conditions that we use. Note that the Dirichlet condition or the nature of fluxes is not relevant.
The result means that $C(t)$ remains constant in time. Obviously, this property should be shared by the discrete schemes that we use.
It should hold up to machine accuracy in any computation. Behind this conservation property are important physical principles such as
mass conservation for concentrations or energy conservation in the case of the heat equation.

%
%
\subsection{Discrete mass conservation}
%
%

In order to check whether the conservation property holds throughout the computational time interval, we will
compute the discrete total mass $C_{total}$ in the domain at each time step $t_n=n\Delta t$ for $n\in\N_0$.

%
%
\subsubsection*{The nodal-based scheme}
%
%

For this purpose, we have to introduce cells for our nodal-based scheme. The nodal value then represents a constant
value over the whole cell. The total amount of our discrete variables on a cell is the value times the length of the cell. 
We define cell boundary points $x_{j\pm\frac 12} =j\Delta x\pm\frac{\Delta x}2$ for any $j=1,\cdots , N-1$
and introduce the cells $\sigma_j =[x_{j-\frac 12},x_{j+\frac 12}]$
for $j=1,\cdots ,m-1,m+1,\cdots N-1$. This is analogous to our finite volume type scheme. But the nodes and cells 
are shifted by $\frac{\Delta x}2$. For the boundary nodes
we take the cells to be $\sigma_0 =[x_0,x_{0+\frac 12}]$ and $\sigma_N=[x_{N-\frac 12}, x_N]$. At the interface, we introduce
two cells $\sigma_{m-}=[x_{m-\frac 12},x_{m-}]$ and $\sigma_{m+}=[x_{m+},x_{m+\frac 12}]$. 
These smaller cells are the reason why
the flux boundary and coupling conditions are different for the schemes.

For $n\in \mathbb{N}_0$, we define concentration sum
%
\begin{equation}
\label{dnc2}
C_n=\frac{1}{2}u^n_0+u^n_1+\cdots +u^n_{m-1}+\frac{1}{2}u^n_m+\frac{1}{2}v^n_m+v^n_{m+1}
+\cdots+v^n_{N-1}+\frac{1}{2}v^n_{N}
\end{equation}
%
and the discrete total concentrations $\overline{C}_n = \Delta x C_n$.
The integral of the initial concentration over the whole domain gives the initial total concentration $\overline{C}_0$.
If we take the initial nodal values to be the integral averages over the cells of the exact initial data, 
then they give the same initial total concentration $\overline{C}_0 = \Delta x C_0$. 
Any approximation of the integral averages, e.g.,  by quadrature, will introduce a small initial error in total mass.
The common factor $\Delta x$ is not needed for the considerations that follow. Therefore, we will work with $C_n$.
The { discrete conservation property} for a nodal-based scheme then is that $C_{n+1} = C_n$ for $n\in\N_0$.
An attempt to introduce the discrete conservation property was made in Morton and Mayers \cite[Section 2.14]{bMOMA}, 
but not quite in an adequate way and not put to much use.

\noindent
{\bf Boundary conditions:}
It is easily seen that the numerical homogeneous Neumann boundary conditions \eqref{central_bdry} obtained via central 
differencing give the discrete conservation property. The same is true for Neumann boundary conditions 
with non-zero fluxes. The boundary conditions \eqref{onesided_bdry} using the 
one-sided differences produce
errors of the order $\frac{\nu_-} 2(u_1^n-u_0^n)$ and $\frac{\nu_+} 2(v_N^n-v_{N-1}^n)$.

\noindent
{\bf Dirichlet-Neumann coupling:}
We will need the above form of discretization in which the cell $\sigma_m = [x_{m-\frac{1}{2}}, x_{m+\frac{1}{2}}]$ 
is split into two sub-cells for the
coupling conditions that do not involve the Dirichlet condition.
When we consider the discrete Dirichlet condition $u_m^n =v_m^n$, 
the term $\frac{1}{2}u^n_m+\frac{1}{2}v^n_m$ is replaced by $u_m^n$.
This is the reason why the Dirichlet-Neumann coupling does not need the extra factors 
of $2$ that are present in the boundary conditions.

We will make use of the numerical flux functions \eqref{num_flux}.
With these, we can rewrite the nodal-based schemes. 
For instance \eqref{scheme_u} and the boundary condition \eqref{central_bdry} for $j=0$ are
%
$$
u_j^{n+1} = u_j^n + F_-^{j+\frac 12,n}(u_{j+1}^n,u_j^n)-F_-^{j-\frac 12,n}(u_j^n,u_{j-1}^n)
\qquad\text{and}\qquad u_0^{n+1} =u_0^n + 2F_-^{\frac 12,n}(u_{1}^n,u_0^n)
$$
%
Analogously, we can treat the other updates and use $F_-^{j+\frac 12,n}(v_{j+1}^n,v_j^n)$ as well as
$F_-^{j-\frac 12,n}(v_j^n,v_{j-1}^n)$ for $j\ge m$. This is basically a finite volume type formulation
of the nodal-based scheme using shifted cells.
Since $F_\pm^{j+\frac 12,n}(u_{j+1}^n,u_j^n) = F_\pm^{(j+1)-\frac 12,n}(u_{(j+1)}^n,u_{(j+1)-1}^n)$ 
and $F_\pm^{j-\frac 12,n}(u_j^n,u_{j-1}^n)=F_\pm^{(j-1)+\frac 12,}(u_{(j-1)+1}^n,u_{(j-1)}^n)$ we see that the
fluxes cancel when the updates are inserted into $C_{n+1}$ and summed up. The updates for $j=0,m,N$ need the extra factor of $2$
in the flux to compensate for the half-sized cells there. Then we obtain $C_{n+1} =C_n$. This is the discrete conservation
property.

For the Dirichlet-Neumann coupling, we use the cell $\sigma_m$ and $\frac{1}{2}u^n_m+\frac{1}{2}v^n_m =u_m^n$ in \eqref{dnc2}.
We see that \eqref{scheme_c1} can be written just using the fluxes
\eqref{num_flux} and all fluxes cancel to give $C_{n+1} =C_n$.
Giles \cite{GIL} used \eqref{giless1} which for $r=1$ contains an extra factor of $2$ and therefore gives
$C_{n+1} = C_n -\nu_-(u_m^n-u_{m-1}^n) +\nu_+(v_{m+1}^n-u_m^n)$. So, it does not have the conservation property.
The correct scheme \eqref{giless2} is identical to \eqref{scheme_c1} for $r=1$. So, it has the conservation property
in this case. For the case $r\ne 1$ in \eqref{giless2}, one would have to modify \eqref{dnc2} to take the different mesh sizes 
into account. This will be done in a further paper \cite{CMW3}.

\noindent
{\bf Flux couplings:}
For the heat flux coupling, we have to take the central difference coupling \eqref{2001} to compensate for $\frac 12u_m^n$ and
$\frac 12v_m^n$ in $C_{n+1}$. We have
%
\begin{eqnarray*}
u_m^{n+1} &=& u_m^n -2 F_-^{m-\frac12}(u_m^n,u_{m-1}^n) -\frac{2\Delta t}{\Delta x} J_{heat}(u^n_m,v_m^n),\\
v_m^{n+1} &=& v_m^n +2 F_-^{m+\frac12}(v_{m+1}^n,v_m^n) +\frac{2\Delta t}{\Delta x} J_{heat}(u^n_m,v_m^n).
\end{eqnarray*}
%
The extra heat flux terms cancel in the summation in $C_{n+1}$ giving $C_{n+1}=C_n$. 
Obviously, this will hold for all other discrete coupling conditions.

%
%
\subsubsection*{Piecewise linear finite elements}
%
%

Note that for piecewise linear finite elements on our nodal mesh, one can compute the total integral
of the solution over the interval $[0,1]$ using exact quadrature by the mid-point or trapezoidal rules as
%
\begin{eqnarray*}
\overline{C}_n&=&\Delta x \left(\sum_{j=1}^m \frac{u_{j-1}^n+u_j^n}2 +\sum_{j=m+1}^N \frac{v_{j-1}^n+v_j^n}2\right)\\
&=&\Delta x \left(\frac 12 u_0^n + \sum_{j=1}^{m-1} u_j^n +\frac 12 (u_m^n +v_m^n) +\sum_{j=m+1}^{N-1} v_j^n
+\frac 12v_N^n\right).
\end{eqnarray*}
%
This is the same formula as when $\overline{C}_n$ is obtained using \eqref{dnc2}. The
natural boundary conditions and mass lumping for the time derivative maintain the conservation property.

%
%
\subsubsection*{Discrete mass conservation for finite volume type coupling schemes}
%
%

In the case of the finite volume type scheme, we can simplify \eqref{dnc2} to
%
\begin{equation}
    \label{fv_cons}
C_n=u^n_1+\cdots +u^n_m+v^n_{m+1}+\cdots+v^n_{N}. 
\end{equation}
%
We see that the factors of $2$ are not needed,
so we obtain the conservation property using the discrete boundary conditions \eqref{onesided_bdry}
and heat flux coupling conditions \eqref{300} and \eqref{3001}. Again, we can substitute
any of the other fluxes for $J_{heat}(u_m^n,v_m^n)$.

The conservation property also extends to multidimensional equations. For finite volume schemes, this is straightforward.
The total computed quantity is just summed up of all cells. 
For finite elements on regular rectangular or quadrilateral meshes,
it is also clear how to define cells around the nodes. 
For Delauney triangulations, one can use the Voronoi cells around the nodes.
These cells are used in some finite volume methods, see e.g.\ Mishev \cite{MIS} and literature cited there.


