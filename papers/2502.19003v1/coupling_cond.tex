%
%
\section{Bi-domain modeling and coupling conditions}% for diffusion\\ equations}
\label{sec:bi}
%
%

The main motivation for this paper is the interest in coupling conditions arising 
in the mathematical modeling of calcium dynamics in living cells.
In a living cell, calcium $Ca^{2+}$ is transported through channels by pumps, and it diffuses into the cytosol
as well as into the endoplasmic reticulum (ER), and it reacts with buffers. The ER and the cytosol are
separated by a membrane that contains channels through which calcium is exchanged.
The calcium concentration in the ER is denoted by $E$, and in the cytosol by $c$. 
A three-dimensional calcium dynamics model for these processes was proposed by Falcke \cite{l8}, see also 
Chamakuri \cite{l9}, Thul and Falcke \cite{l11} as well as Thul \cite{l10} for more detailed descriptions 
of the mathematical model. 


\begin{wrapfigure}{l}{0pt}
% \begin{center}
\includegraphics[width=3 in]{newcube.eps}
\centering
\caption{Bi-domain cubic volume distribution of ER and cytosolic domains, modification of a figure from \cite{l9}.}
\label{fig901}
\end{wrapfigure}

In the model, the membrane is a surface that divides
the model domain into two subdomains $\Omega_c$ and $\Omega_E$. When modeling a small section of a cell,
the simplest domain for such a model
is to take a rectangular box or cuboid with a planar surface for the membrane, see Figure~\ref{fig901}.
The calcium dynamics is described
by a system of coupled reaction-diffusion equations for the concentrations of calcium $Ca^{2+}$ in the cytosol, the ER,
 and a number of buffers, see Falcke \cite{l8}. The equations are coupled through reaction terms, which we will disregard. Instead, we will focus on the bi-domain coupling via the membrane. To simplify, we will consider a one-dimensional approximation of the three-dimensional model, as shown in Figure~\ref{fig901}, by taking a one-dimensional slice along the $z$-axis.

%
%
\subsection{Biophysical coupling conditions}
\label{subsec:biocouple}
%
%
The bi-domain coupling occurs through fluxes that represent channel flow or pumping. We describe these in some detail as we aim to incorporate similar terms into our one-dimensional model. The transport across the ER membrane involves three distinct contributions. Calcium is moved
from the ER into the cytosol through a leak current $P_l (E-c)$ in a pumping term and some channels.
Here, $P_l$ is the coefficient of the leak flux density. Calcium is re-sequestered into the ER by 
pumps modeled by a term proportional to the maximal pump strength $P_p$. The term contains a dissociation constant $K_d>0$. 
In the current model, the channel cluster fluxes $J_{ch}$ and the membrane pumping fluxes $J_{pump}$ are 
normal to the surface, as shown in Figure \ref{fig901}. They are given as follows  \cite[(2.1)]{l10}
%
\begin{equation}
\label{ch66}
J_{ch}(c,E)=\Psi \frac{E-\alpha c}{\beta+\gamma E+\delta c}
\end{equation}
%
with some constants $\alpha$, $\beta$, $\gamma$, $\delta$ and $\Psi$. Outside of the channels on the membrane, one has
%
\begin{equation}
\label{987}
J_{pump}(c,E)=P_{l}(E-c) -P_{p}\frac{c^{2}}{K_{d} ^2 +c^2}
\end{equation}
%
as membrane pumping condition with some constants $P_l$, $P_p$ and $K_d$.
Specific values of these parameters can be found in Thul \cite{l10} or Thul and Falcke \cite{l11}.

The model consists of reaction-diffusion type equations for $c$ and $E$ as well as a number of other quantities, 
see e.g.\ Thul \cite[Chapter 2]{l10} or Thul and Falcke \cite{l11}.
The respective diffusion coefficients are $D$ and $D_E$.
The currents are incorporated into the volume dynamics by setting the 
flux type coupling conditions on the interface $\Gamma_c$ at the $ER$
membrane to be
%
\begin{equation}
\label{couple_bio}
D\nabla c \cdot \mathbf{n}_c= D_E \nabla E \cdot \mathbf{n}_c=-J_{ch,pump}(c,E) 
\end{equation}
%
or more specifically $D\frac{\partial c}{\partial z}=D_E \frac{\partial E}{\partial z}=-J_{ch,pump}(c,E)$.

In Fourier's or Fick's law, the flux $J$ is always in the direction of the negative gradient of the diffused quantity. 
Therefore, a positive slope of the solution must correspond to a negative flux. In the coupling conditions, this means that the 
gradients or normal derivatives of the solution are equal to minus the fluxes.
Note that, in practice, the dynamics of this model are further complicated by the spontaneous and stochastic opening and closing of channels, see Falcke \cite{l8}. However, this is not pertinent to our objectives here.


%
% 
\subsection[Bi-domain diffusion equation]{The one dimensional bi-domain diffusion equation}
%\label{subsec:bidom}
%
%
 
We now consider a bi-domain one-dimensional diffusion model, which is a one-dimensional reduction of the three-dimensional model discussed above. We assume that the concentration only varies in the vertical $z$-direction
and use $x$ as our one-dimensional variable. If it suffices to restrict ourselves to one diffusion equation in order to study 
the coupling conditions. We consider as domain the interval $\Omega=[0,1]\subset\R$ and
divide the domain into two sub-domains by the midpoint $x=\frac 12$, which is the common
interface boundary. The two sub-domains are $\Omega_-=[0,\frac 12]$  and $\Omega_+=[\frac 12, 1]$. 
Let $D_\pm>0$ be the diffusion coefficients on the sub-domains, which may differ. 
For the consideration of the coupling conditions, we want to distinguish the solutions clearly
on the sub-domains. 
We therefore take
$u:\Omega_-\times\mathbb{R}_{\ge 0}\to \mathbb{R}$ and $v:\Omega_+\times\mathbb{R}_{\ge 0}\to \mathbb{R}$ 
to be the solutions that describe a concentration or temperature at
position $x\in\Omega$ and time $t\in\R_{\ge 0}$ in the sub-domains.
We provide some initial data $u_0:\Omega_-\to \mathbb{R}$
and $v_0:\Omega_+\to \mathbb{R}$. The initial boundary value problem for the bi-domain diffusion equations is then defined as
%
\begin{flalign}
\label{eqn1}
&\quad\frac{\partial u}{\partial t}= D_{-}\frac {\partial ^2 u}{\partial x^2} 
\quad \mbox{for all} \quad (x,t) \in \Omega_+\times\mathbb{R}_{\ge 0},
\qquad\frac{\partial v}{\partial t}= D_{+}\frac {\partial ^2 v}{\partial x^2} 
\quad \mbox{for all} \quad (x,t) \in \Omega_-\times\mathbb{R}_{\ge 0},\nonumber\\
&\quad u(x,0) = u_0(x)\quad\text{for}\;x\in\Omega_-,\qquad\qquad\qquad\qquad v(x,0) = v_0(x)\quad\text{for}\; x\in\Omega_+,\nonumber\\
&\quad\frac{\partial}{\partial x}u(0,t)=\frac{\partial}{\partial x}v(1,t) = 0 \quad\text{for}\; t\in\R_{\ge 0}.
\end{flalign}
%
We take the outer boundary conditions to be the homogeneous no flux
Neumann conditions. We aim to solve a well-posed problem by coupling $u$ and $v$ across the interface at $x=\frac 12$. The problem
defined in \eqref{eqn1} also needs two appropriate internal coupling conditions at this interface. 

%
%
\subsection{Coupling conditions}
\label{subsec:couple}
%
%

We now introduce a number of useful coupling conditions that can be found in the literature.

\noindent
{\bf Dirichlet-Neumann coupling:}
The simplest case is to assume the continuity of the solution and the flux at the interface.
%
\begin{equation}
\label{nnm}
u({\s\frac 12},t)=v({\s\frac 12},t),\qquad D_- 
\frac{\partial u(\frac 12,t)}{\partial x} =D_+ \frac{\partial v(\frac 12,t)}{\partial x}.
\end{equation}
%
It is generally called the Dirichlet-Neumann coupling.
This type of coupling is used in domain decomposition methods, see e.g.\ Quarteroni and Valli in \cite{bQUVA}. 
In heat conduction, it applies to different materials in perfect contact, as described by Carslaw and Jaeger \cite[p.\ 23]{b8}. Carr and March \cite{l116} refer to it as the perfect contact condition.

Note that in case $D_-=D_+=D$, this coupling will give a solution $w$ to
the single domain diffusion equation. The
common factor $D$ then drops out of \eqref{nnm}. This can be used as a numerical test case for
coupling algorithms.

\noindent
{\bf Heat flux coupling conditions:}
We assume that the heat flow is proportional to the temperature difference and flowing from higher to lower temperature.
%
\begin{equation}
D_- \frac{\partial u(\frac 12,t)}{\partial x}=D_+ \frac{\partial v(\frac 12,t)}{\partial x} 
=-J_{heat}(u,v) =H(v({\s\frac 12},t)-u({\s\frac 12},t)),
\label{eq111}
\end{equation}
%
see e.g.\ Carslaw and Jaeger \cite[p.~23]{b8}. Here $H>0$ is the contact transfer coefficient at $x=\frac 12$. 

\noindent
{\bf General interface conditions:}
For completeness, we mention a more general case than the previous coupling conditions.  
All of the above coupling conditions can be considered within a general form as 
%
\begin{equation}
\label{flux_gen}
D_- \frac{\partial u(\frac 12,t)}{\partial x} =D_+ \frac{\partial v(\frac 12,t)}{\partial x} =-J_{gen}(u,v)
\end{equation}
%
with $J(u,v)=J_{gen}(u,v) =-H(\theta v({\s\frac 12},t)-u({\s\frac 12},t))$. Here $\theta>0$ is the
partition coefficient at $x=\frac 12$. For $\theta =1$, we have the heat flux coupling conditions \eqref{eq111}.

Using one of the flux equations, dividing it by $H$ and taking $H \to \infty$ produces 
the {\bf partition conditions} given by Carr and March \cite{l116} 
% 
\begin{equation}
u({\s\frac 12},t)=\theta v({\s\frac 12},t),\qquad D_- \frac{\partial u(\frac 12,t)}{\partial x} 
=D_+ \frac{\partial  v(\frac 12,t)}{\partial x}.
\label{part1}
\end{equation}
%
for $t>0$. The condition defined in case $\theta\neq 1$ maintains a constant
ratio between the discontinuous solutions at the interface. For references to applications, see Carr and March \cite{l116}.
If $\theta =1$, the coupling conditions are just the Dirichlet-Neumann coupling \eqref{nnm}. They are the limiting case for $H\to\infty$ of the heat flux coupling conditions.

\noindent
{\bf Channel and pumping interface conditions:}
Now, we want to consider the coupling conditions \eqref{couple_bio} using
\eqref{ch66} or \eqref{987}.
The general combined coupled interface conditions for the biophysical model discussed above are defined 
using $J(u,v)=J_{ch,pump}(u,v)$ in \eqref{flux_gen} for channel and membrane pumping. The latter flux is given by \eqref{ch66} or \eqref{987} for $u=E$ and $v=c$. 
Note that both cases can be viewed as generalizations of the heat flux conditions \eqref{eq111}.

%\noindent
%{\bf Simplified membrane pumping conditions:}
%We consider the special case of the membrane pumping conditions obtained %by 
%setting $P_l=H$ and $P_p=P$ and $K_d=1$ in $J_{pump}(u,v)$ to give
%$J_{pump,simp}(u,v)=-H(v-u)+P\frac{v^2}{1+v^2}$ in \eqref{flux_gen}. 
%Note that the minus sign of the heat flux coefficient means that we are %pumping towards a higher concentration.

%\noindent
%{\bf Linearized membrane pumping coupling conditions:}
%Now, we linearize the non-linear simplified membrane pumping conditions. %For this, we take 
%the function $f(v)=-H(v-u)+P_p\frac{v^2}{1+v^2}$. To linearize it,
%we use a Taylor expansion at $v=1$ for $g(v)=\frac{v^2}{1+v^2}$ giving %$g(v)\approx\frac{1}{2}v$. 
%The new form of the flux is $J_{pump,lin}(u,v) =-H(v-u)+\frac{P_p}{2}v$ in \eqref{flux_gen}.

Note that most coupling conditions have the same form \eqref{couple_bio} or \eqref{flux_gen}. The three terms have to be equal. 
For practical use, one chooses two out of the three possibilities of equating a pair of these terms. These are then discretized and used
in the numerical scheme.