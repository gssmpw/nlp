%
%
\section{Introduction}
\label{sec:int}
%
%

The subject of this paper is a study of various numerical interface coupling conditions for 
diffusion or heat equations.     
Many important real-world problems in physics, engineering, and biology are modeled via bi-domain or multi-domain
partial differential equations (PDEs) with coupling conditions at the sub-domain interfaces. 
This modeling may be due to the nature of the problem when it has a
physical interface at which physical coupling conditions occur. Or it may be an artificial mathematical interface, 
e.g.\ in domain decomposition methods that are used for computational purposes. 

% Diffusion equations are partial differential equations that are frequently used to model heat conduction via Fourier's
% law or diffusion of mass concentrations via Fick's law.
% A particular application we have in mind in biophysics is the diffusion of calcium concentrations
% in living cells.
% A complex three-dimensional biophysical model consisting of a system of reaction-diffusion equations 
% and coupling conditions was developed by Falcke \cite{l8}. 
% This model was used by Thul \cite{l10} and Chamakuri \cite{l9}. 
% The equations
% model the intracellular calcium dynamics in a realistic fashion between the cytosolic region and the
% endoplasmic reticulum (ER) region of a living cell via channels and pumps on the membrane which separates both
% regions. The channels and pumps on the membrane are mathematically modeled as 
% coupling conditions on the interface between the two regions.

Diffusion equations, commonly used to model heat conduction via Fourier's law or the diffusion of mass concentrations via Fick's law, are central to 
many of these models. One such application is the diffusion of calcium concentrations in living cells, which is modeled using a system of reaction-
diffusion equations. A complex three-dimensional biophysical model, developed by Falcke \cite{l8}, incorporates coupling conditions to describe 
intracellular calcium dynamics between the cytosolic and endoplasmic reticulum (ER) regions of a cell. The channels and pumps on the membrane 
separating these regions are mathematically modeled as coupling conditions on the interface between them, 
as explored by Thul \cite{l10}, Thul and Falcke \cite{l11}, as well as 
Chamakuri \cite{l9}. These coupling conditions were the main motivation for this study.
Another motivation came from the analysis of Giles \cite{GIL} in the context of fluid-structure interactions.


% Our aim is to investigate important mathematical properties of such coupling 
% conditions that are used in numerical computations.
% For this purpose we choose the simpler model of a diffusion or heat equation in one space dimension
% in order to analyze mathematical problems concerning the numerical 
% discretization of coupling conditions for such types of equations. This allows us to better
% understand the fundamental numerical issues of the conservation property and stability of the coupling.

While the diffusion equations need only one boundary condition at outer boundaries,
there are always two conditions needed for coupling them at an internal interface. 
The coupling conditions of the above model are channel pumping and membrane pumping fluxes. 
For the purpose of comparison, we also consider some of the other types of coupling conditions used in conjunction with
bi-domain diffusion or heat equations.
These include the well-known Dirichlet-Neumann coupling and the heat flux coupling. 

The simplest coupling conditions are the Dirichlet-Neumann (DN) coupling conditions that
have been extensively used for parallel computing in non-overlapping domain decomposition methods, see e.g.\ 
Toselli and Widlund \cite{bTOWI}, Quarteroni and Valli \cite{bQUVA} or Quarteroni \cite[Ch.\ 19]{bQUA}.
The same type of problems also arise in heat flow at interfaces between different materials 
where heat flux coupling conditions are well
established, see e.g.\ Carslaw and Jaeger \cite{b8}. 
Carr and March \cite{l116} considered various interface coupling conditions based on the heat flux coupling conditions. 
The coupling conditions in which we are mainly interested are the more complex channel pumping and membrane
pumping conditions modeling calcium transport within cells mentioned above, 
see these conditions in \eqref{ch66} and \eqref{987} below. 
They are extensions of the heat flux coupling conditions, e.g.\ by the addition of a non-linear
pumping term. 
  
Our analysis in this first paper is focused on the essential property of conservation of concentration or heat.
This property reflects the fundamental principles of mass conservation for mass concentration or energy conservation
for temperature.
Indeed, this property must be maintained at the discrete level by numerical schemes. We will show
how this is relevant to discretized flux boundary conditions and various
flux coupling conditions for bi-domain diffusion equations. As a first step, we only look at schemes that
are explicit in time. We are aware of the fact that implicit methods are
actually the methods of choice for these stiff problems. The implicit schemes are addressed in a further paper \cite{CMW3}.
In a companion paper \cite{CMW2}, we will consider the property of 
Godunov-Ryabenkii stability is based on normal mode solutions for the coupling schemes as
well as for the boundary conditions. Results of the joint research of the authors 
were included in the thesis of Munir \cite{MUN}.
Until now, the bio-physical coupling conditions that we are considering
have not been analyzed in terms of the discrete conservation property and numerical stability.

The use of GR stability for coupling conditions was first introduced by Giles \cite{GIL}, which served as a major starting point for our studies. 
However, Giles's approach employed an inconsistent scheme, leading to artificial instabilities and loss of conservation. This inconsistency was 
pointed out by Zhang et al.\ \cite{zhang}. We will address it in our discussion of the schemes. Giles focused on one-dimensional bi-domain 
thermal diffusion equations with Dirichlet-Neumann coupling conditions, which he used in engineering heat conduction problems involving fluid-
structure interactions. For simplicity, we consider only the basic diffusion equation.

Roe et al.\ \cite{roe2007stability} extended Giles's work by introducing a moving interface, using both finite difference/finite difference (FD/FD) and finite volume/finite element (FV/FEM) discretizations. In a later study \cite{roe2008combined}, they used higher-order combined methods for explicit and implicit coupling. Errera and Chemin \cite{errera2013optimal} explored Dirichlet-Robin and Robin-Robin coupling conditions for the same equations, and Errera with Moretti et al.\ \cite{moretti2018stability} focused on stability and convergence for various coupling schemes.

Henshaw and Chand \cite{chand} studied a heat transfer problem as a multi-domain issue with Dirichlet-Neumann coupling and mixed Robin conditions, proposing the use of central differences for discretizing interface equations to improve accuracy and stability. Lemari\'{e} et al. \cite{lemarie2015analysis} explored ocean-atmosphere coupling conditions using implicit and explicit methods, while Zhang et al.\ \cite{zhang} investigated multi-domain PDEs in climate models, employing both explicit and implicit coupling discretizations.


% For the discretization in space, we use the standard central difference for the second derivative. We restrict ourselves 
% to a first-order method in time.
% For the explicit methods in this paper, we take a simple forward Euler time step, giving the forward in time central in space (FTCS)
% update. The time steps in this discretization of 
% diffusion equations have a severe stability restriction. It is a stiff problem. 
% Therefore, implicit methods are more important in practice. 
% But this paper is already quite lengthy. Implicit schemes will be
% taken up in the separate paper \cite{CMW3}. Concerning basic numerical methods and concepts for single-domain diffusion equations, 
% we refer to Hundsdorfer and Verwer \cite{bHUVE}, Morton and Mayers \cite{bMOMA} or Thomas \cite{bTHO}. 
For spatial discretization, we use the standard second order central difference for the second derivative and limit ourselves to a first-order 
method in time. For explicit methods, we employ a simple forward Euler time step, resulting in the forward in time, central in space (FTCS) update. This 
discretization of diffusion equations comes with a significant stability restriction, making it a stiff problem. Consequently, implicit methods are 
generally more practical. However, due to the length of this paper, implicit schemes will be discussed in a separate publication \cite{CMW3}. For basic 
numerical methods and concepts related to single-domain diffusion equations, we refer to Hundsdorfer and Verwer \cite{bHUVE}, Morton and Mayers \cite{bMOMA}, or Thomas \cite{bTHO}.
 
As outer boundary conditions, we are interested in the more commonly used flux conditions rather than the Dirichlet conditions
used by Giles \cite{GIL}. For simplicity, we took
homogeneous Neumann conditions. They have the advantage that total concentration should be conserved on the domain 
and we can test that numerically.
Non-zero fluxes can be easily added. 
Ghost values are used to determine the numerical updates of the outer boundaries and the interface coupling 
conditions. To find these ghost values, we use either the central difference method or a one-sided
difference method with respect to the mesh points at or next to the boundary of the computational domain. 
The ghost values are thereby eliminated from the updates.

Below in Section \ref{sec:cons}, we will argue that it is important to maintain the conservation property in discretizations.
We emphasize that it is necessary to maintain e.g.\ the mass conservation principle for a concentration
in the type of applied problems that we are considering. 
This is the analogue of energy conservation in case the same type of equations model heat flow. 
The numerical conservation property that we introduce restricts the discretization of boundary and coupling conditions.
This is the first important result of the paper. 

The next key result comes from the comparison of nodal based finite element type and cell based finite volume type schemes. 
For the Dirichlet-Neumann coupling there is no difference in the schemes due to
the Dirichlet coupling condition. But, for Neumann type boundary conditions and
flux coupling conditions, the schemes differ. Nodal based schemes need a central
difference with respect to the boundary or interface node
in order to have the conservation property.
In finite volume type schemes, which are cell based, there are no nodal values 
at boundaries or interfaces. The domain boundary or coupling interface is a cell boundary. Here, a one-sided difference is actually a central difference 
for the domain boundary and it is conservative. This distinction is the second main result of the paper.
We also show that the Giles coupling 
\cite{GIL} fails to satisfy the conservation property. 

One of our test cases has discontinuous data at the coupling interface.
In this case, we show that for the Dirichlet-Neumann coupling the nodal based scheme produces a much larger 
computational error in the total mass concentration than the finite volume type scheme. This is due to the fact that the former has a node at the
interface.

% The paper is organized as follows. In Section \ref{sec:bi}, we first give a short review of the application that motivated this study.
% This is necessary in order to understand the background of the bi-domain diffusion problem and
% some of the coupling conditions that we consider. Then, we define the bi-domain diffusion problem that we are studying.
% Next, we introduce various coupling conditions. Some are from the literature, and others are derived by simplification. 
% In Section \ref{sec:exp}, we introduce our explicit scheme in time nodal discretization of equations as well as the 
% numerical boundary and coupling conditions. These include the conditions used by Giles \cite{GIL}. Finally, we introduce a
% finite volume type scheme. Section \ref{sec:cons} is the main section of the paper. We introduce the concept of the discrete
% conservation property for the nodal-based scheme and the finite volume type scheme. Then, we give all the results that we outlined above.
% The final Section \ref{sec:tests} gives some numerical computations that give examples for the coupling conditions and
% verify the conservation property in practice.

The paper is organized as follows. Section \ref{sec:bi} begins with a 
brief review of the biophysical application that motivated this study, providing the necessary background for understanding the bi-domain diffusion problem and 
the associated coupling conditions. We then define the specific bi-domain diffusion problem under investigation and introduce various coupling 
conditions, some drawn from the literature. In Section \ref{sec:exp}, we present our explicit time-
stepping scheme and the nodal discretization of the equations, along with the numerical boundary and coupling conditions. 
We include those suggested by 
Giles \cite{GIL}. We also introduce a finite volume type scheme. Section \ref{sec:cons} is the core of the paper. We introduce the concept of discrete conservation property for both the nodal-based and finite volume type schemes. This is followed by the results discussed earlier. Finally, Section \ref{sec:tests} presents numerical computations that demonstrate the coupling as well as boundary conditions and verify the conservation property in practice.
