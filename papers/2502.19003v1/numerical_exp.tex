%
%
\section{Explicit discretization of the bi-domain diffusion problems}
\label{sec:exp}
%
%

In this section, we discuss numerical methods for the discretization of the bi-domain diffusion 
equation with homogeneous Neumann boundary conditions in one space dimension. We discretize the diffusion equation with 
an explicit nodal-based finite difference scheme and a finite volume discretization method.

%
%
\subsection{The explicit nodal based scheme}
\label{sub:node}
%
%

We take $\Omega=[0,1]$ partitioned into $\Omega_- =[0,\frac 12]$ and $\Omega_+ =[\frac 12, 1]$
with the coupling boundary at $x=\frac{1}{2}$. 
We introduce grid points $x_j\in [0,1]$ for the spatio-temporal discretization of the bi-domain diffusion system. 
We set $N=2m$ for some $m\in\mathbb{N}$ and $\Delta x=1/N$. Grid points for the two sub-domains $\Omega_-=[0,1/2]$
and $\Omega_+=[1/2,1]$ are defined as
% 
\begin{equation*}
x_{j}=j \Delta x,\quad\mbox{for}\quad j=0,1,...,m-1, j=m+1,..., N=2m.
\end{equation*}
%
The nodes $x_0$ and $x_N$ are the boundary nodes, the others the interior nodes.
Further, at the interface $c=\frac{1}{2}$ we introduce at $x_m =m\Delta x$ a double node $x_{m_{-}}=x_{m_{+}}=m\Delta x$, see
Figure \ref{bid1} below. The node $x_{m_{-}}$ is used 
in conjunction with $\Omega_-$ and $x_{m_{+}}$ with $\Omega_+$. We call a scheme based on this mesh a nodal-based scheme, in contrast to the cell-based finite volume mesh introduced below.
%
\begin{figure}[htp]
\setlength{\unitlength}{1cm}
\begin{picture}(15,2) 
\put(2.0,1){\line(1,0){12}}       
\put(2.0,1){\circle*{0.1}}
\put(1.3,0.5){$x_0=0$}
\put(1.8,1.5){$u_0^n$}
\put(3.2,1){\circle*{0.1}}
\put(3,0.5){$x_1$}
\put(3,1.5){$u_1^n$}
\put(4.7,0.5){$\cdots$}
\put(6.8,1){\circle*{0.1}}
\put(6.6,0.5){$x_{m-1}$}
\put(6.3,1.5){$u_{m-1}^n$}
\put(8,1){\circle*{0.1}}
\put(6.6,0){$x_m =x_{m\pm}=\frac 12$}
\put(8,0.3){\vector(0,1){0.6}}
\put(7.5,1.5){$u_m^n$}
\put(8.1,1.5){$v_m^n$}
\put(9.2,1){\circle*{0.1}}
\put(8.7,0.5){$x_{m+1}$}
\put(8.9,1.5){$v_{m+1}^n$}
\put(10.8,0.5){$\cdots$}
\put(12.8,1){\circle*{0.1}}
\put(12.3,0.5){$x_{N-1}$}
\put(12.3,1.5){$v_{N-1}^n$}
\put(14,1){\circle*{0.1}}
\put(13.5,0.5){$x_N=1$}
\put(13.8,1.5){$v_N^n$}
\end{picture}
%
\caption{Grid points and discrete values for bi-domain equations}
\label{bid1}
\end{figure}
%

We introduce a fixed time step $\Delta t>0$ and define discrete times $t_n=n\Delta t$ for $n\in\N_0$.
Thus we obtain nodal grid point $(x_j,t_n)\in [0,1]\times\R_{\ge 0}$ for our computational domain.
For the functions $u$ and $v$ introduced in the previous section, we set nodal values to be
% 
$$%\begin{eqnarray*}
u_{j}^n \approx u(x_{j},t_n)\quad \text{for}\; j=0,1,...,m-1,
\quad v_{j}^n\approx v(x_{j},t_n) \quad\text{for}\;j=m+1,m+2,...,N
$$%\end{eqnarray*}
%
as well as at the interface node $u_{m}^n\approx u(x_{m_{-}},t_n)$ and 
$v_{m}^n\approx v(x_{m_{+}},t_n)$. 

The discretization of the diffusion equation is achieved via an explicit forward time step of length $\Delta t>0$ and 
the standard central difference in space for the second derivative in space. This gives us the forward in time central in space
(FTCS) scheme. 
%We note that the use of linear finite elements leads to such a discretization after the application of quadrature. 

We introduce the parameters $\nu_-=D_{-}\frac{\Delta t}{(\Delta x)^2}$ and $\nu_+=D_{+}\frac{\Delta t}{(\Delta x)^2}$.
The updates determining the interior node values for the system \eqref{eqn1} written as explicit iterations
are then given as
%
\begin{eqnarray}
\label{scheme_u}
u^{n+1}_{j}&=&u_j^n +\nu_- (u^n_{j+1}-u_{j}^{n})-\nu_-(u_j^n- u_{j-1}^{n}) \qquad\mbox{for}\;j=1,...,m-1\\
\label{scheme_v}
v^{n+1}_{j}&=&v_j^n +\nu_+ (v^n_{j+1}-v_{j}^{n}) -\nu_+ (v_j^n-v_{j-1}^{n}) \qquad\mbox{for}\;j=m+1,...,N-1.
\end{eqnarray}
%
It is well-known by 
stability analysis, see e.g.\ Morton and Mayers \cite{bMOMA} or Thomas \cite{bTHO}, 
that the time steps of this scheme are restricted by 
$\Delta t\le \frac{(\Delta x)^2}{2 \max\{D_{-},D_{+}\}}$ giving $0<\nu_\pm< \frac 12$.
Due to the square term, the time steps become very small on fine meshes. This leads to the preference for implicit schemes in practice.

%
%
\subsection{Numerical homogeneous Neumann boundary conditions}
%
%

In order to determine $u_0^{n+1}$ and $v_N^{n+1}$ we need numerical boundary conditions. We cannot use the
formulas of type \eqref{scheme_u} for $j=0$ or \eqref{scheme_v} for $j=N$ directly. 
To compute these nodal values we introduce the ghost points $x_{-1} =-\Delta x$ and $x_{N+1} =(N+1)\Delta x$ as well as 
the corresponding ghost values $u_{-1}^n$ and $v_{N+1}^n$.
We implement discrete homogeneous outer Neumann boundary conditions. 

Let us consider the boundary at $x=0$. The simplest
two finite difference approximations we can use, namely the one-sided and the central difference with respect
to the node $x_0$, are
%
\begin{equation}
\label{bdry_fd}
\frac{u_0^n-u_{-1}^n}{\Delta x} = 0 \qquad \mbox{and}\qquad \frac{u_1^n-u_{-1}^n}{2\Delta x}=0.
\end{equation}
%
These give the values $u_{-1}^n=u_0^n$ and $u_{-1}^n = u_1^n$ respectively. They are inserted into the
updates \eqref{scheme_u} for $j=0$. Analogously, we proceed for $j=N$ using \eqref{scheme_v}. The one-sided differences give
%
\begin{equation}
\label{onesided_bdry}
u^{n+1}_{0}=u_{0}^{n}+\nu_- (u^n_{1}-u_{0}^{n}) \qquad\mbox{and}\qquad v^{n+1}_{N}=v_{N}^{n}-\nu_+ (v^n_{N}-v_{N-1}^{n}).
\end{equation}
%
The central differences lead to extra factors of $2$ in the differences.
%
\begin{equation}
\label{central_bdry}
u^{n+1}_{0}=u_{0}^{n}+2\nu_- (u^n_{1}-u_{0}^{n})\qquad\mbox{or}\qquad v^{n+1}_{N}=v_{N}^{n}-2\nu_+( v^n_{N}-v_{N-1}^{n}).
\end{equation}
%
We will see in Section \ref{sec:cons} that the conservation property will determine when 
each of the formulas \eqref{onesided_bdry} or \eqref{central_bdry}
is useful. We will see for the nodal-based scheme that only \eqref{central_bdry} correctly represents
the conservation property. 
Later, we will also consider a finite volume type scheme. It needs \eqref{onesided_bdry}
as boundary conditions. In finite volume schemes, the one-sided difference formula turns out to be a central 
difference with respect to the cell
boundary at $x=0$ or $x=1$. This gives the correct boundary fluxes there. 

In the literature, these boundary conditions seem to have only been compared
in terms of the fact that \eqref{central_bdry} is a second-order approximation in space and \eqref{onesided_bdry}
only of first order, while the interior updates \eqref{scheme_u} and \eqref{scheme_v} are second order in space, 
see e.g.\ Hundsdorfer and Verwer \cite[Subsection I.5.3]{bHUVE} 
or Thomas \cite[Section 1.4]{bTHO}. In \cite{bHUVE}, there is
a nice symmetry argument for using \eqref{central_bdry}. For $L^p$ error estimates, the truncation error of the boundary
discretization is allowed to be one order lower than in the interior. The total estimates are determined by summing up
the cells or elements. The number of those that contain boundary conditions is one order of $\frac 1h$ less than those
in the interior. Therefore, the overall order in the space of a scheme using \eqref{onesided_bdry} 
is maintained despite the lower
order of the truncation error. So, the order of the scheme does not make much of a distinction 
between the two numerical boundary conditions.

For a complete scheme, it remains to determine updates $u_{m}^{n+1}$ and $v_m^{n+1}$ 
via various numerical coupling conditions below.
We will obtain the numerical coupling conditions results analogous to the numerical boundary conditions for the 
discrete numerical coupling conditions via two fluxes.
The Dirichlet-Neumann coupling differs due to the Dirichlet condition. 

%
%
\subsubsection*{The piecewise linear finite element method}
%
%

The above updates \eqref{scheme_u}, \eqref{scheme_v} and \eqref{central_bdry} can be achieved on our regular mesh by 
taking piecewise linear finite elements as the spatial semi-discretization. The finite element functions are
linear on the intervals $[x_{j-1},x_j]$ for $j=1,\ldots ,N$, continuous at the nodes and with nodal values as above.
In the case of flux couplings below, we would allow a discontinuity at $x_m$. But let us ignore that for the moment.
After quadrature, this finite element method gives a system of ordinary
differential equations in time. Let $\ul w(t)\in\R^{N+1}$ be the vector of nodal values of the piecewise linear
approximations at time $t\ge 0$. Then the resulting system has the form $\ul M\dot{\ul w}(t) +\ul A\ul w(t)=\ul 0$ with
the mass matrix $\ul M\in\R^{(N+1)\times (N+1)}$ and the stiffness matrix $\ul A\in\R^{(N+1)\times (N+1)}$ 
that are calculated using the hat basis functions.
In the literature, these tri-diagonal matrices can be found as $(N-1)\times(N-1)$ matrices 
for the single domain heat equation 
with zero Dirichlet boundary data. The mass matrix has the entries $\Delta x\frac 46$ on the diagonal and
$\Delta x\frac 16$ on the two secondary diagonals, analogously the entries are $\frac 2{\Delta x}$ 
and $\frac {-1}{\Delta x}$ for the stiffness matrix, see e.g.\ Wait and Mitchell \cite[Section 5.2]{bWAMI}.

The homogenous Neumann condition is automatically satisfied for finite elements. 
This comes from the variational principle behind them.
Dirichlet conditions need to be enforced by putting boundary values to zero. A flux boundary condition is
put into the underlying functional for the Galerkin method. In case the boundary fluxes vanish, nothing has to be done. 
For this reason, Neumann boundary conditions
are also called natural or do nothing boundary conditions, see e.g.\ Johnson \cite[Section 1.7]{bJOHN}.
The homogeneous Neumann condition requires the use of the nodal values at $x_0$ and $x_N$. The mass and stiffness matrices
are then tri-diagonal $(N+1)\times (N+1)$ matrices. The entries for the interior nodes $x_j$ for $j=1,\ldots, N-1$ are the same
as those for $j=2,\ldots ,N-2$ in the case of the Dirichlet boundary condition. But for $j=0,N$ the homogeneous Neumann boundary
condition gives the entries $\Delta x\frac 26$ on the diagonal and
$\Delta x\frac 16$ on the two secondary diagonals, analogously the entries are $\frac 1{\Delta x}$ 
and $\frac {-1}{\Delta x}$ for the stiffness matrix. This comes from the fact that only half of a hat basis function is
used at the boundary nodes for the Neumann boundary data.

In computations, the mass matrix $M$ is dealt with using a commonly applied procedure called mass lumping, 
see Hundsdorfer and Verwer \cite[Section III.5]{bHUVE}, Quarteroni and Valli \cite[Section 11.4]{bQUVA1}
or Thom\'ee \cite[Chapter 15]{bTHOME}. 
One adds in each row all entries of $\ul M$ and puts them as entries into the regular diagonal
matrix $\ul D$. Thus, the need to solve a system of linear equations in each time step is eliminated. 
This gives the system $\dot{\ul w}(t) +\ul D^{-1}\ul A\ul w(t)=\ul 0$. An exact quadrature to obtain $\ul M$
requires the Simpson rule on each interval since the integrands are quadratic. Taking the trapezoidal rule as an
approximate quadrature gives the lumped matrix. Quarteroni and Valli, as well as Thom\'ee
discussed this for the case of two space dimensions. Thom\'ee gave a second
interpretation of the procedure.
The matrix $\ul D$ for the Neumann boundary conditions has the entries $1$ for the interior nodes and $\frac 12$ for the boundary
nodes. Then a forward time step leads to the boundary updates \eqref{central_bdry} with the factor of $2$ 
as well as the interior updates \eqref{scheme_u}, \eqref{scheme_v}.
We will see in Section \ref{sec:cons} that the finite element method 
is consistent with the conservation property.

%
%
\subsection{Discrete Dirichlet-Neumann coupling conditions}
%
%

As a first step we consider the Dirichlet-Neumann conditions \eqref{nnm} for the bi-domain diffusion model, i.e.\ the Dirichlet condition
$u(\frac 12,t)=v(\frac 12,t)$ and the Neumann condition 
$D_{-}\frac{\partial  u(\frac 12,t)}{\partial x} =D_{+}\frac{\partial  v(\frac 12,t)}{\partial x}$.
We must choose one of them in order to determine $u_m^{n+1}$ and the other for $v_m^{n+1}$.
We take the Neumann condition for $u_m^{n+1}$ as our first step and discretize it using forward differences and a ghost point value $u_{m+1}^n$ as
$D_{-}\frac{u_{m+1}^n-u_{m}^n}{\Delta x}=D_{+}\frac{v^n_{m+1}-v^n_{m}}{\Delta x}$. 
This is solved for $u_{m+1}^n-u_{m}^n$ and the result inserted into \eqref{scheme_u} for $j=m$. Using $\nu_+=\frac{D_{+}}{D_{-}}\nu_-$
and $u_m^n=v_m^n$ we obtain the following update
% 
\begin{equation}
u^{n+1}_{m}= u^n_{m}+ \nu_{+} (v_{m+1}^n-u_{m}^n)-\nu_{-}(u_{m}^n-u_{m-1}^n).
 \label{f16}
\end{equation}
%
Note that Giles \cite{GIL} suggested a derivation using the fluxes to arrive at the same update.

Then we use the Dirichlet condition to set $v_m^{n+1}=u_m^{n+1}$ as our second step to achieve the coupling.
We assume that the discrete initial data satisfy $v_m^0=u_m^0$. So, we will always have $v_m^n=u_m^n$, and
we could eliminate the variable $v_m^n$ from explicit schemes using the Dirichlet coupling condition.
We keep it for comparison with conditions where this is not the case.

Using the backward differences above means that one introduces the ghost value $v_{m-1}^n$. This can be inserted
into \eqref{scheme_v} and with the Dirichlet condition $u_m^n=v_m^n$ gives the same scheme, the Dirichlet update
becoming $u_m^{n+1}=v_m^{n+1}$. Taking a central difference here does not make much sense, since we would be introducing
two ghost values into one Neumann condition.

Now, the fully discretized explicit FTCS nodal scheme for the bi-domain diffusion model with Dirichlet-Neumann
coupling conditions is given as 
%
\begin{eqnarray}
\label{scheme_c1}
u_{0}^{n+1}=&u_{0}^n+2\nu_- (u_{1}^{n}-u_{0}^n) \quad\qquad\qquad\qquad\qquad&\quad\mbox{for $j=0$},\nonumber \\
u_{j}^{n+1}=&u_j^n +\nu_-( u_{j+1}^{n}-u^n_j) -\nu_-(u_j^n-u^n_{j-1}) \;\quad&\quad\mbox{for $0<j<m$},\nonumber \\
v_{j}^{n+1}=&v_j^n+\nu_+ (v_{j+1}^{n}-v^{n}_j)-\nu_+(v_j^n-v^n_{j-1})\;\quad &\quad\mbox{for $m<j<N$}, \nonumber \\
v_{N}^{n+1}=&v_{N}^n-2\nu_+ (v_{N}^{n}-v_{N-1}^n) \quad\qquad\qquad\qquad&\quad\mbox{for $j=N$},
\end{eqnarray}
%
with the coupling conditions
%
\begin{eqnarray}
\label{dn_disc}
u_{m}^{n+1}=& u^n_{m}+\nu_{+}(v^n_{m+1}-u^n_{m})-\nu_- (u^n_{m}-u^n_{m-1})& \quad \mbox{for $j=m$},\nonumber \\
v^{n+1}_m=&u^{n+1}_m  \quad\qquad\qquad\qquad\qquad\qquad\qquad\qquad&\quad\mbox{for $j=m$}.\nonumber \\
\end{eqnarray}

%
%
\subsubsection*{The coupling scheme of Giles}
%
%

Giles \cite{GIL} considered heat diffusion with a heat capacity $c$ and conductivity $k$, i.e.\ in our notation an
equation of the form $cu_t-ku_{xx}=0$. He further considered 
different mesh sizes $\Delta x_{\pm}$ in the two sub-domains. 
Note that Giles \cite[Eq.~(26)]{GIL} did not distinguish $u$ and $v$ in the bi-domain case, and he did not
introduce double values at the interface node. Therefore, 
$v^{n+1}_m=u^{n+1}_m$ is implicitly automatically implied in his 
scheme. As we mentioned, this is something we could have also done above without changing the outcome of the explicit computations. 
But, we will need our approach with double values at the interface for the other 
coupling conditions that do not include the Dirichlet condition.

Before making a supposed simplification, Giles had derived a correct coupling scheme \cite[Eqs.~(15),(16)]{GIL}. 
In our notation, the update is given as
%
\begin{equation}
\label{giless2}
u_{m}^{n+1}= u^n_{m}+\frac{2r\nu_{+}}{1+r}(v^n_{m+1}-u^n_{m})-\frac{2\nu_{-}}{1+r}(u^n_{m}-u^n_{m-1})
\end{equation}
%
where $r=\frac{c_-\Delta x_-}{c_+\Delta x_+}$ and $\nu_{\pm}=\frac{k_{\pm}\Delta t}{c_{\pm}(\Delta x_{\pm})^2}$. 
Note that setting $c_{\pm}=1$, $k_{\pm}=D_{\pm}$, $\Delta x_-=\Delta x_+$ and $r=1$ in \eqref{giless2} gives our formula \eqref{f16}.
 
However, in Gile's stability analysis, \cite[(26)]{GIL} used the update for
% 
\begin{equation}
\label{giless1}
u_{m}^{n+1}= u^n_{m}+2 r\nu_{+}(v^n_{m+1}-u^n_{m})-2\nu_- (u^n_{m}-u^n_{m-1}).
\end{equation}
%
This discretization was obtained by introducing an inconsistency in the time discretization. It leads to the loss of 
the conservation property, see Section \ref{sec:cons}, and some instabilities, see Giles \cite{GIL}.
 

%
%
\subsection{Discrete other coupling conditions}
%
%

We discretize the heat flux coupling conditions defined in \eqref{eq111} via an explicit discretization method with one-sided 
differences. The heat flux coupling conditions are 
$D_{-}\frac{\partial u}{\partial x}= D_{+}\frac{\partial v}{\partial x}=H(v-u).$
For the nodes $j\ne m$, we use the formulas in \eqref{scheme_c1}. Only the coupling conditions for $j=m$ will be replaced.
Now for the interface node $j=m$, the updates \eqref{scheme_u} and \eqref{scheme_v} are
%
\begin{eqnarray}
 \label{n32}
 u^{n+1}_{m}&=&u^{n}_{m}+\nu_-(u^{n}_{m+1}-u^{n}_{m})-\nu_- (u^{n}_{m}-u^n_{m-1})\nonumber\\
 v^{n+1}_{m}&=&v^{n}_{m}+\nu_+(v^{n}_{m+1}-v^{n}_{m})-\nu_+(v^{n}_{m}-v^n_{m-1}).
\end{eqnarray}
%
In these two formulas, we have two ghost point values $u^n_{m+1}$ and $v^n_{m-1}$. We calculate these values by discretizing
the two coupling conditions \eqref{eq111}.
For the first one, we take the forward difference approximation, and for
the second one is the backward difference approximation to obtain these ghost point values
% 
\begin{equation}
D_{-}\frac{u_{m+1}^n-u_{m}^n}{\Delta x}=D_{+}\frac{v_{m}^n-v_{m-1}^n}{\Delta x}= -J_{heat}(u_m^n,v_m^n) = H(v_{m}^n-u_{m}^n).
\label{y7}
\end{equation}
%
Solving the first equation gives $u_{m+1}^{n}=u_{m}^n+\frac{H\Delta x}{D_{-}}(v_{m}^n-u_{m}^n)$.
We substitute into \eqref{n32} and obtain using $\nu_-\frac{H\Delta x}{D_{-}} 
=\frac{ D_{-}\Delta t}{H(\Delta x)^2}\frac{\Delta x}{D_{-}} =\frac{H \Delta t}{\Delta x}$ the update
%
\begin{eqnarray} 
u_{m}^{n+1}&=&u^n_m+\nu_- \Big(u_{m}^n+\frac{H\Delta x}{D_{-}}(v_{m}^n-u_{m}^n)-u^n_m\Big)-\nu_-( u_{m}^{n}-u^n_{m-1})\nonumber\\
&=&u^n_m-\nu_-(u^n_m-u^n_{m-1})+\frac{H \Delta t}{\Delta x}(v^n_m-u^n_m),\nonumber\\
&=&u^n_m-\nu_-(u^n_m-u^n_{m-1})-\frac{\Delta t}{\Delta x}J_{heat}(u^n_m,v_m^n).
\label{300}
\end{eqnarray}
%
Solving the second equation of \eqref{y7} we get $v_{m-1}^{n}=v_{m}^n-\frac{H \Delta x}{D_{+}} (v_{m}^n-u_{m}^n)$ and
analogously
%
\begin{equation}
v_{m}^{n+1}=v^n_m+\nu_+(v^n_{m+1}-v^n_m)+\frac{\Delta t}{\Delta x}J_{heat}(u^n_m,v_m^n).
\label{3001}
\end{equation}
%
For the coupled scheme, we proceed analogously as in \eqref{scheme_c1}. Only we are replacing the 
updates for the Dirichlet-Neumann coupling at $j=m$
by the new formulas  \eqref{300} and \eqref{3001} for the heat flux coupling conditions.

Now, we want to discretize coupling conditions \eqref{eq111} via central difference approximations as
%
\begin{equation}
 D_{-}\frac{u_{m+1}^{n}-u_{m-1}^n}{2\Delta x}=D_{+}\frac{v_{m+1}^{n}-v_{m-1}^n}{2\Delta x}=H(v^n_m-u^n_m).
\end{equation}
%
This gives $u_{m+1}^{n}=u_{m-1}^n+\frac{2H\Delta x}{D_{-}}(v_{m}^n-u_{m}^n)$.
Substituting into \eqref{n32} gives additional factors of $2$ in the updates
%
\begin{eqnarray}
\label{2001}
u_{m}^{n+1}&=&u^n_m-2\nu_-(u^n_{m}-u_{m-1}^n)-\frac{2\Delta t}{\Delta x} J_{heat}(u^n_m,v_m^n)\nonumber\\
v_{m}^{n+1}&=&v^n_m+2\nu_+ (v_{m+1}^n-v^n_m)+\frac{2\Delta t}{\Delta x}J_{heat}(u^n_m,v_m^n).
\end{eqnarray}
%
Note the analogy to the numerical homogeneous Neumann conditions \eqref{central_bdry}. These are the coupling
conditions that will prove to maintain the conservation property for the scheme \eqref{scheme_c1}. The coupling conditions
\eqref{2001} replace \eqref{dn_disc}.

We now refer to the other coupling conditions defined in Subsection \ref{subsec:couple}.  
We obtain, for example, the explicit discretization for the channel pumping conditions
by replacing the heat flux $J_{heat}(u_m^n,v_m^n)=-H(v^n_m-u^n_m)$ 
by the channel flux $J_{ch}(u_m^n,v_m^n)=\Psi \frac{u^n_m-\alpha v^n_m}{\beta+\gamma u^n_m+\delta v^n_m}$ 
in the updates \eqref{300} and \eqref{3001} or in \eqref{2001}. Analogously, we proceed to the membrane coupling conditions.

%
%
\subsection{An explicit finite volume type scheme}
\label{sub_fv}
%
%

We now want to consider a finite volume type scheme. These types of schemes are quite popular for compressible fluid flow computations
and useful when the conservation property of quantities comes into play. 
For this scheme we define for $m\in\N$ the $N=2m$ cells of length $\Delta x=1/N$ with midpoints $x_j = \Delta x(j-1/2)$ and 
boundary points $x_{j\pm\frac{1}{2}}=x_j\pm \frac{\Delta x}{2}$
for $j=1,\ldots ,N$. The cells are the sub-intervals $\sigma_j=[x_{j-\frac{1}{2}}, x_{j+\frac{1}{2}}]$, see
Figure \ref{fbb1}. The number of cells and nodes is always even in our coupling problems. Each sub-domain $[0,\frac 12]$
and $[\frac 12,1]$ has $m$ cells. The nodal points of
the nodal-based schemes considered above
have become the cell boundary points.  

On the cells $\sigma_j$, we consider the solutions to be constant and assign the values $u_j^n$ for $j=1,\ldots ,m$
as wells as $v_j^n$ for $j=m+1,\ldots ,N$ to the nodes $x_j$ at the cell center. Since the interface sits at cell boundaries,
we do not have to use a node with a double value for the coupling.

%
\begin{figure}[htp]
\setlength{\unitlength}{1cm}
\begin{picture}(15,2) 
\put(2.0,1){\line(1,0){12}} 
\put(1.9,0.3){$0$}
\put(2.0,0.7){\line(0,1){0.6}}    
\put(2.7,1){\circle*{0.1}}
\put(2.5,0.5){$x_1$}
\put(2.5,-0.1){$\sigma_1$}
\put(2.4,1.5){$u_1^n$}
\put(3.4,0.7){\line(0,1){0.6}}
\put(4.1,1){\circle*{0.1}}
\put(3.9,0.5){$x_2$}
\put(3.9,-0.1){$\sigma_2$}
\put(4,1.5){$u_2^n$}
\put(4.8,0.7){\line(0,1){0.6}}
\put(5.7,0.5){$\cdots$}
\put(6.6,0.7){\line(0,1){0.6}}
\put(7.3,1){\circle*{0.1}}
\put(7.1,0.5){$x_m$}
\put(7.1,-0.1){$\sigma_m$}
\put(7.1,1.5){$u_m^n$}
\put(8,0.7){\line(0,1){0.6}}
\put(7.9,0.2){$\frac 12$}
\put(8.7,1){\circle*{0.1}}
\put(8.5,0.5){$x_{m+1}$}
\put(8.5,-0.1){$\sigma_{m+1}$}
\put(8.5,1.5){$v_{m+1}^n$}
\put(9.4,0.7){\line(0,1){0.6}}
\put(10.8,0.5){$\cdots$}
\put(12.6,0.7){\line(0,1){0.6}}
\put(13.3,1){\circle*{0.1}}
\put(13.1,0.5){$x_N$}
\put(13.1,-0.1){$\sigma_N$}
\put(13.1,1.5){$v_N^n$}
\put(13.9,0.3){$1$}
\put(14.0,0.7){\line(0,1){0.6}} 
\end{picture}
\caption{Cells, nodes, and cell or nodal values for the finite volume scheme.}
\label{fbb1}
\end{figure}
%

We seek approximations of $u$ and $v$ by integral averages on the cells to represent our solutions, i.e.\ by
$u_{j}^n\approx \frac{1}{\Delta x}\int_{x_{j-\frac{1}{2}}}^{x_{j+\frac{1}{2}}} u(x,t_n)\,dx$ for $j=1,...,m$
and $v_{j}^n\approx\frac{1}{\Delta x}\int_{x_{j-\frac{1}{2}}}^{x_{j+\frac{1}{2}}} v(x,t_n)\,dx$ 
for $j=m+1,...,N=2m$.
For the discretization of the initial data, we can use these integral averages. The integrals may be
replaced by a quadrature rule.

Finite volume schemes use a numerical flux formulation that is very useful in the presence of the conservation property.
Using it, they maintain this property automatically. We introduce the right and left-hand numerical flux functions for 
cell $\sigma_j$ for $j=1,\ldots,m$ as
%
\begin{equation}
\label{num_flux}
F_\pm^{j+\frac 12,n}(u_{j+1}^n,u_j^n) = \nu_\pm(u_{j+1}^n-u_j^n)\qquad\text{and} 
\qquad F_\pm^{j-\frac 12,n}(u_j^n,u_{j-1}^n) = \nu_\pm(u_j^n-u_{j-1}^n).
\end{equation}
%
The numerical fluxes for $j=m+1,\ldots, N$ are analogous with $u_j^n$ replaced by $v_j^n$.
Note that usually $\frac{\Delta t}{(\Delta x)^2}$ is not included in the fluxes for the purpose of numerical analysis. 
But we find it convenient to do so here. An important property is that $F_\pm^{j+\frac 12,n}(u_{j+1}^n,u_j^n) =
F_\pm^{(j+1)-\frac 12,n}(u_{(j+1)}^n,u_{(j+1)-1}^n)$, i.e.\ the right hand flux for cell $\sigma_j$ is equal to the left hand
flux of cell $\sigma_{j+1}$. In the updates, they will appear with opposite signs. 
With these numerical flux functions, we obtain the flux form of the updates. 
%
\begin{eqnarray*}
u_j^{n+1} &=& u_j^n + F_-^{j+\frac 12,n}(u_{j+1}^n,u_j^n) -F_-^{j-\frac 12,n}(u_j^n,u_{j-1}^n) \quad\text{for}\; j=2,\ldots ,m-1,\\
v_j^{n+1} &=& u_j^n + F_+^{j+\frac 12,n}(v_{j+1}^n,v_j^n) -F_+^{j-\frac 12,n}(v_j^n,v_{j-1}^n) \quad\text{for}\; j=m+2,\ldots ,N-1=2m-1.
\end{eqnarray*}
%
For the interior cells $\sigma_j$ for $1<j<m$ and $m+1<j<N$, this gives the FTCS updates used in \eqref{scheme_u} and \eqref{scheme_v}. 

As numerical boundary conditions for the boundary cells $\sigma_1$ and $\sigma_N$,
we use the one-sided difference formulas from \eqref{onesided_bdry}. For the left boundary, we have to replace $j=0$ by $j=1$, e.g.\ 
$u_1^{n+1} =u_1^n +F_-^{\frac 32,n}(u_2^n,u_1^n)$. This will be justified in Section \ref{sec:cons}. 

In finite difference form, the explicit FTCS finite volume type scheme is
%
\begin{eqnarray}
\label{fvol2}
u^{n+1}_{1}&=&u^n_1+\nu_-( u^n_{2}-u_{1}^{n})\nonumber\\
u^{n+1}_{j}&=&u_j^n+\nu_-( u^n_{j+1}-u_{j}^{n}) -\nu_-(u_j^n- u_{j-1}^{n}) \qquad\mbox{for}\;j=2,...,m-1\nonumber\\
v^{n+1}_{j}&=&v_j^n+\nu_+( v^n_{j+1}-v_{j}^{n}) - \nu_+ (v_j^n-v_{j-1}^{n} )\qquad\mbox{for}\;j=m+2,...,N-1\\
v^{n+1}_{N}&=&v_N^n-\nu_+( v^n_{N}-v_{N-1}^{n}).\nonumber
\end{eqnarray}
%

Now we derive the discretization scheme of the Dirichlet-Neumann coupling \eqref{nnm} for our finite volume type scheme using
ghost cell values $v^n_m$ and $u^n_{m+1}$.
We discretize the Neumann coupling condition via the central 
differences with respect to the boundary. These are one-sided differences for $u_m^n$ and $v_{m+1}^n$. We set
$D_{+}\frac{v^n_{m+1}-v^n_{m}}{\Delta x}=D_{-}\frac{u^n_{m+1}-u^n_m}{\Delta x}$.
Using the Dirichlet condition $u^n_m=v^n_m$  
this implies that $\nu_{-}(u^n_{m+1}-u^n_m)=\nu_{+}(v^n_{m+1}-v^n_m)=\nu_{+}(v^n_{m+1}-u^n_m)$. 
This gives for $j=m, m+1$ the Dirichlet-Neumann updates
%
\begin{eqnarray*}
u^{n+1}_{m}&=&u^n_m+\nu_{-}(u^n_{m+1}-u^n_m)-\nu_{-}(u^n_m-u^n_{m-1})\\ 
&=&u^n_m+\nu_+(v^n_{m+1}-u^n_m)-\nu_-(u^n_m-u^n_{m-1}),\\
v^{n+1}_{m+1}&=&v_{m+1}^n+\nu_+ (v^n_{m+2}-v_{m+1}^{n}) - \nu_+ (v_{m+1}^n-u_m^{n} )
\end{eqnarray*}
%
Note that the first formula for $u_m^n$ is the same as in the nodal-based case \eqref{scheme_c1}. 
A difference is only in the outer boundary conditions and the interpretation of the discrete values.

For the various flux coupling conditions, we take the updates \eqref{300} and \eqref{3001} and insert the appropriate fluxes.
The justification will be given in the next section. The numerical flux functions for $j=m+\frac 12$ 
then contain the added extra term $\frac {\Delta t}{\Delta x}J(u_m^n,v_m^n)$.