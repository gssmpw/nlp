\pdfoutput=1
\documentclass[a4paper,english]{jnsao}
\usepackage[utf8]{inputenc}
\usepackage[english]{babel}
\usepackage{graphicx}
\usepackage{float}
\usepackage{wrapfig}
\usepackage{subcaption}
\usepackage[textsize=footnotesize,color=DarkOrange!40]{todonotes}
\usepackage[indLines=false,noEnd=false]{algpseudocodex}
\usepackage[nameinlink,capitalise]{cleveref}
\usepackage{algorithm}
\usepackage{enumitem}

\theoremstyle{definition}
\newtheorem{assumption}[theorem]{Assumption}
\crefname{assumption}{Assumption}{Assumptions}

\numberwithin{algorithm}{section}

\usepackage{tikz,pgfplots,pgfplotstable}
\usepgfplotslibrary{colorbrewer,fillbetween}
\pgfplotsset{compat=newest}
\pgfplotsset{
    fb/.style                 = { color = Set2-A,          line width = 1pt },
    projgrad/.style           = { color = Set2-B,          line width = 1pt, dashed },
    legend style = {
        inner sep = 0pt,
        outer xsep = 5pt,
        outer ysep = 0pt,
        legend cell align = left,
        align = left,
        draw = none,
        fill = none,
    },
}

\newcommand{\va}[1]{\left| #1 \right|}
\newcommand{\cj}[1]{\{ #1\}}
\newcommand{\proxold}[2]{\prox_{#1}(#2)}
\DeclareMathOperator{\prox}{prox}
\DeclareMathOperator{\proj}{proj}
\newcommand{\fl}[3]{#1 :#2 \to #3}
\newcommand{\df}[2]{#1 ( #2 )}
\newcommand{\nr}[2]{ \| #1 \|_{#2}}
\newcommand{\pd}[2]{ \langle #1,#2 \rangle}
\newcommand{\rea}[1]{\mathbb{R}^{#1}}
\newcommand{\f}[5]{\begin{array}{crclr}
#1 : & #2  & \to & #3 &\\
  & #4 & \longmapsto  &  #5 &
  \end{array}}
\newcommand{\fpp}[5]{ #1 = \left \{
\begin{array}{clr}
#2 & \textit{si} & #3\\
#4 & \textit{si} & #5
\end{array}
\right. }
\newcommand{\vg}[1]{\mathbf{#1}}
\def\grad{\nabla}
\def\norm#1{\|#1\|}
\def\linear{\mathbb{L}}
\def\term#1{\emph{#1}}
\def\iprod#1#2{\langle #1, #2 \rangle}
\def\defeq{:=}
\def\N{\mathbb{N}}
\def\R{\mathbb{R}}
\def\C{\mathbb{C}}
\def\extR{\widebar{\R}}
\DeclareMathOperator{\Dom}{dom}
\DeclareMathOperator{\sublev}{sub}
\DeclareMathOperator{\diam}{diam}
\DeclareMathOperator{\interior}{int}
\DeclareMathOperator{\boundary}{bd}
\DeclareMathOperator{\ccone}{ccone}
\DeclareMathOperator{\Sym}{Sym}
\let\Re\relax\DeclareMathOperator{\Re}{Re}
\let\Im\relax\DeclareMathOperator{\Im}{Im}
\newcommand{\freevar}{\,\boldsymbol\cdot\,}
\def\abs#1{|#1|}
\def\polar#1{#1^\circ}
\def\bipolar#1{#1^{\circ\circ}}
\def\ortho{o}
\def\relerr{\rho}

\manuscriptcopyright{}
\manuscriptlicense{}

\title{Multigrid methods for total variation\texorpdfstring{$^\ddagger$}{}}
\shorttitle{Multigrid methods for total variation}

\date{2024-11-18}

\author{%
    Felipe Guerra\thanks{Research Center in Mathematical Modeling and Optimization (MODEMAT), Quito, Ecuador \emph{and} Department of Mathematics, Escuela Politécnica Nacional (EPN), Quito, Ecuador. \email{edison.guerra@epn.edu.ec}}
    \and
    Tuomo Valkonen\thanks{MODEMAT \emph{and} EPN \emph{and} Department of Mathematics and Statistics, University of Helsinki, Finland. \email{tuomo.valkonen@iki.fi}, \orcid{0000-0001-6683-3572}}
}

\acknowledgements{This research has been supported by the EPN internal project PIS-23-04.

\medskip

${\ \ }^\ddagger$ Deanonymised version submitted to SSVM 2025. This version does not incorporate post peer review improvements.}

\begin{document}

\maketitle

\begin{abstract}
    Based on a \emph{nonsmooth coherence condition}, we construct and prove the convergence of a forward-backward splitting method that alternates between steps on a fine and a coarse grid.
    Our focus is an total variation regularised inverse imaging problems, specifically, their dual problems, for which we develop in detail the relevant coarse-grid problems.
    We demonstrate the performance of our method on total variation denoising and magnetic resonance imaging.
\end{abstract}

\section{Introduction}

In this work, we consider composite optimisation problems of the form
\begin{equation}
    \label{eq:intro:problem:fine-grid}
    \min_{x \in X}~ F(x) + G(x),
\end{equation}
where $F$ is convex and smooth, and $G$ is convex but possibly nonsmooth on a Hilbert space $X$.
We want to apply forward-backward splitting \cite{lions1979splitting}
to this problem, while reducing computational effort by occasionally passing to a lower-dimensional problem.
Such multigrid methods make possible the computationally efficient high-precision solution of partial differential equations \cite{briggs2000multigrid}.
A few works \cite{nash2000multigrid,parpas2017multilevel,kornhuber1994monotone,ang2024mgprox} have looked into applying the same principle to large-scale optimisation problems and variational inequalities.
However, none, so far, treat nonsmoothness completely satisfactorily: while \cite{ang2024mgprox} allows $G$ to be nonsmooth, it requires $F$ to be strongly convex in addition to smooth.
Such an assumption is rarely satisfied in problems of practical interest.
In \cite{kornhuber1994monotone} only constrained quadratic problems are considered.
In \cite{parpas2017multilevel}, less assumptions are imposed on the fine-grid problem, however, the coarse-grid problems are required to be smooth.
We want to exploit the inherent nonsmooth properties of the problem on the coarse grid as well.

We focus on imaging with isotropic total variation regularisation, i.e.,
\[
    \min_{y \in Y}~ E(y) + \alpha\norm{\grad_h y}_{2,1},
\]
where $E$ is a data fitting term; $\grad_h \in \linear(Y; X)$ a (discrete) gradient operator; and $\norm{\freevar}_{2,1}$ the sum over a pixelwise $2$-norms.
Since, in the present work, we are limited to forward-backward splitting, and the proximal map of total variation is not prox-simple, i.e., not easily calculated, we have to work with the dual problem
\begin{equation}
    \label{eq:intro:tv-dual}
    \min_{x \in X}~ E^*(-\grad_h^* x) + \delta_{\alpha B_{2,1}}(x).
\end{equation}
To calculate $\grad E^*$ efficiently, indeed, for $E^*$ to be smooth, we will, unfortunately, need $E$ to be strongly convex (but not necessarily smooth).
Typically $E(y)=\frac{1}{2}\norm{Ty-b}^2$ for a forward operator $A$ mapping an image to its measurements, so we will need $T$ to be invertible.
This holds with fully or over-sampled data.

The problem \eqref{eq:intro:tv-dual} is of the form \eqref{eq:intro:problem:fine-grid}.
A significant step in forming a multigrid optimisation method is deciding on a coarse-grid version of the problem. While MGProx \cite{ang2024mgprox} allows any smooth coarse-grid problem, we will follow the approach of \cite{parpas2017multilevel}---where gradient descent for smooth problems was considered---in proposing in \cref{sec:nscc} a \term{nonsmooth coherence condition} that locally determines the coarse-grid problem. For \cref{eq:intro:tv-dual}, this will form coarse-grid constraints more difficult than $\alpha B_{2,1}$.
We analyse the projection to those constraints in \cref{sec:tv} after proposing and proving the convergence of the general method in \cref{sec:fb}.
We finish with denoising and magnetic resonance imaging (MRI) experiments in \cref{sec:numerical}.

This work builds upon the Master's thesis of the first author \cite{felipe-msc} with simplified proofs and coarse problem, and expanded numerics with a faster implementation \cite{multigrid-codes-zenodo}.

\paragraph{Notation}

Let $X$ and $Y$ be Hilberts spaces, $A \subset X$.
We write $\linear(X; Y)$ for the bounded linear operators between $X$ and $Y$, $\delta_A$ for the $\{0,\infty\}$-valued indicator function of $A$, and
$B(x,\alpha)$ for the closed ball of radius $\alpha$ and centre $x$ in $X$.
We write $\polar A \defeq \{ z \mid \iprod{z}{x} \le 0\ \forall x \in A\}$ for the polar, and $\bipolar A \defeq \polar{(\polar A)}$ for the bipolar.
These satisfy $\bipolar A \supset A$ and $\polar{(\bipolar A)} = \polar A$.
The smallest closed convex cone containing $A$ is $\ccone A \defeq \bipolar A$.
The normal cone to $A$ at $x$ is $N_A(x) \defeq \{ z \mid \iprod{z}{\tilde x - x} \le 0 \ \forall \tilde x \in A\}$.
For a convex $F: X \to \extR$, the subdifferential at $x$ is $\partial F(x)$, the proximal operator $\prox_F$, and the Fenchel conjugate $F^*$.
We have $\partial \delta_A(x) = N_A(x)$.
We refer to \cite{clason2020introduction} for more details on these concepts.
Finally $x_{\freevar j} \in \R^D$ is the $j$:th row of $x \in \R^{D \times n}$.

\section{The coarse problem}
\label{sec:nscc}

The first question we have to answer is how to build the coarse problem?
In this section, we construct a general guideline, the \emph{nonsmooth coherence condition}, and prove that a forward-backward method applied to a coarse-grid problem satisfying this condition, will construct a descent direction for the fine grid.

\subsection{The nonsmooth coherence condition}

Write $I_h^H \in \linear(X; X_H)$ for the \term{restriction} operator from the fine grid modelled by the Hilbert space $X$ to the coarse grid modelled by the Hilbert space $X_H$. Typically $X$ and $X_H$ are finite-dimensional with $\dim X_H \ll \dim X$.
We call $I_H^h \in \linear(X_H; X)$ satisfying $\mu I_H^h=(I_h^H)^*$ for some $\mu>0$ the \term{prolongation} operator.

In \cite{nash2000multigrid,parpas2017multilevel}, treating the smooth problem $\min_x F(x)$, the coarse-grid problems $\min_x F_H^k$ were built by introducing an arbitrary coarse objective $F_H$ that satisfies the smooth \term{coherence condition}
$
    I_h^H \nabla F(x^{k}) = \nabla F_H(\zeta ^{k,0}),
$
for an initial coarse point $\zeta^{k,0}$, typically $\zeta^{k,0}=I_h^H x^k \in X_H$, and setting
\[
    F_H^k(\zeta) \defeq F_H(\zeta) + \iprod{w_H^k}{\zeta-\zeta^{k,0}}
    \quad\text{for}\quad
    w_H^k \defeq I_h^H \grad F(x^k) - \grad F_H(\zeta^{k,0}) \in X_H.
\]
Then $\grad F_H^k(\zeta^{k,0})=\grad F(x^k)$, so if $x^k$ solves the fine-grid problem, the coarse grid problem triggers no change: it is solved by $\zeta^{k,0}$.
A descent direction for the coarse-grid problem also allows constructing a find-grid descent direction \cite{nash2000multigrid}.

We extend this approach to the nonsmooth problems \eqref{eq:intro:problem:fine-grid}.
Specifically, for $G_H^k$ satisfying the following two assumptions, we take as our \term{coarse-grid problem}
\begin{align}
    \label{eq:notation:problem:coarse-grid}
    \min_{\zeta \in X_{H}} F_H^k(\zeta) + G_H^k(\zeta).
\end{align}

\begin{assumption}[Basic coarse structure]
    \label{ass:coarse:basic}
    $F_H: X_H \to \R$ is convex with $L_H$-Lipschitz gradient. For all $k \in \N$, $G_H^k: X \to \extR$ is convex, proper, lower semicontinuous.
    The step length parameter $\tau_H>0$ satisfies $\epsilon \defeq 2- \tau_HL_H > 0$.
\end{assumption}

\begin{assumption}[Nonsmooth coherence condition]
    \label{ass:coarse:coherence}
    The fine-grid iterate $x^k$ and the initial coarse iterate $\zeta^{k,0}$
    (typically $I_h^H x^k$) satisfy
    $
        \df{I_{h}^{H}\partial G}{x^{k}}\subseteq \df{\partial G_{H}^{k}}{\zeta^{k,0}}.
    $
\end{assumption}

\begin{example}
    \label{ex:coarse:canonical:gH}
    Take $G_H^k=\delta_{\Omega^k}$ for $\Omega^k \defeq  \zeta^{k,0} + \polar{(I_h^H\partial G(x^k))}$.
    Obviously, $\Omega^k \subset X_H$ is nonempty and convex, and $\partial G_H^k(\zeta^{k,0})=N_{\Omega_k}(\zeta^{k,0})=\bipolar{(I_h^H\partial G(x^k))} \supset I_h^H\partial G(x^k)$.
\end{example}

\subsection{Coarse-grid algorithm and descent directions}

We apply forward-backward splitting to the coarse-grid problems \eqref{eq:notation:problem:coarse-grid}.
For an initial coarse point $\zeta^{k,0}$ and an iteration count $m \in \N$, we  thus iterate
\begin{equation}
    \label{eq:coarse:alg}
    \zeta^{k,j+1}
    \defeq
    \prox_{\tau_H G_H^k}(\zeta^{k,j} - \tau_H \grad F_H^k(\zeta^{k,j}))
    \quad
    (j=0,\ldots,m-1).
\end{equation}
In the following we show that $d=\zeta^{k,m}-\zeta^{k,0}$ is a fine-grid descent direction.

\begin{lemma}
    \label{lemma:coarse:descent}
    If \cref{ass:coarse:basic} holds, and we apply \eqref{eq:coarse:alg} for any $\zeta ^{k,0} \in X_H$, then
    \[
        \pd{I_h^H \nabla F(x^k)}{\zeta^{k,m}-\zeta^{k,0}}+\frac{\epsilon}{2\tau_H}\sum _{j=0}^{m-1}\nr{\zeta^{k,j+1}-\zeta^{k,j}}{X_H}^{2}\leq \df{G_{H}^{k}}{\zeta^{k,0}} - \df{G_{H}^{k}}{\zeta^{k,m}}.
    \]
\end{lemma}

\begin{proof}
    We abbreviate $J_H^k \defeq G_H^k + F_H$ and $\zeta^j \defeq \zeta^{k,j}$.
    In implicit form, \eqref{eq:coarse:alg} reads
    \begin{align}
        \label{eq:coarse:ei}
        0\in \df{\partial G_{H}^k}{\zeta^{j+1}} + \df{\nabla F_{H}}{\zeta ^{k,j}}+w_{H}^{k}+ \tau_H^{-1}\df{}{\zeta^{j+1}-\zeta^{j}}.
    \end{align}
    By the subdifferentiability of $G_H^k$ and the descent inequality
    $F_H(\zeta+h) \leq F_H(\zeta) + \pd{\grad F_H(\zeta)}{h} + \frac{L_H}{2}\nr{h}{X}^2$, valid for any $\zeta,h$ (see, e.g., \cite[Theorem 7.1]{clason2020introduction}),
    \[
        \pd{\df{\partial G_{H}^k}{\zeta^{j+1}}+\df{\nabla F_{H}}{\zeta^{j}}}{\zeta^{j+1}-\zeta^{j}}
        \ge
        \df{J_{H}^k}{\zeta^{j+1}}-\df{J_{H}^k}{\zeta^{j}}
        - \frac{L_H}{2}\nr{\zeta^{j+1}-\zeta^{j}}{X_H}^{2}.
    \]
    Applying \(\pd{\,\cdot\,}{\zeta^{j+1}-\zeta^{j}}\) on both sides of \cref{eq:coarse:ei} and using $\epsilon \defeq 2- \tau_HL_H$ thus yields
    \[
        \pd{w_H^k}{\zeta^{j+1}-\zeta^{j}}
        + \frac{\epsilon}{2\tau_H}\nr{\zeta^{j+1}-\zeta^{j}}{X_H}^{2}
        \le
        \df{J_{H}^k}{\zeta^{j}}-\df{J_{H}^k}{\zeta^{j+1}}.
    \]
    Summing  over $j=0,\ldots,m-1$, it follows
    \[
        \pd{w_{H}^{k}}{\zeta^{m}-\zeta^{0}}+\frac{\epsilon}{2\tau_H}\sum _{j=0}^{m-1}\nr{\zeta^{j+1}-\zeta^{j}}{X_H}^{2}
        \le
        \df{J_{H}^{k}}{\zeta^{0}} - \df{J_{H}^{k}}{\zeta^{m}}.
    \]
    Since $\epsilon>0$, by the construction of $w_{H}^{k}$, we get
    using the convexity of $F_H$,
    \[
        \pd{w_{H}^{k}}{\zeta^{m}-\zeta^{0}} \geq \pd{I_{h}^{H} \nabla F(x^k)}{\zeta^{m}-\zeta^{0}}+\df{F_{H}}{\zeta^{0}}-\df{F_{H}}{\zeta^{m}}.
    \]
    Combining these two estimates and simplifying, we obtain the claim.
\end{proof}

\begin{corollary}
    \label{cor:fb:cord}
    Suppose \cref{ass:coarse:basic,ass:coarse:coherence} hold.
    Let $d \defeq I_H^h(\zeta ^{k,m}-\zeta^{k,0})$.
    Then
    \[
        [G+F]'(x^k;d)
        =
        \sup _{g\in \df{\partial G}{x^{k}}}\pd{g+\df{\nabla F}{x^{k}}}{d}
        \leq
        -\frac{\epsilon}{2\mu\tau_H }\sum _{j=0}^{m-1}\nr{\zeta^{k,j+1}-\zeta^{k,j}}{}^{2} \le 0.
    \]
\end{corollary}

\begin{proof}
    By \cref{ass:coarse:coherence},
    $
        \pd{I_h^H g}{\zeta -\zeta ^{k,0}} \leq G_H^k(\zeta)-G_H^k(\zeta^{k,0})
    $
    for all $\zeta$ and $g \in \partial G (x^k)$.
    Taking $\zeta = \zeta^{k,m}$, combining with \cref{lemma:coarse:descent}, we obtain
    \[
        \pd{I_h^H g + I_{h}^{H}\nabla F(x^{k})}{\zeta -\zeta ^{k,0}} +\frac{\epsilon}{2\tau_H}\sum _{j=0}^{m-1}\nr{\zeta^{k,j+1}-\zeta^{k,j}}{X_H}^{2}\leq 0
        \quad \forall\,g \in \partial G (x^k).
    \]
    Since $(I_h^H)^*=\mu I_H^h$, taking the supremum over $g$, we obtain the middle inequality of the claim.
    The equality is standard, e.g., \cite[Lemma 4.4]{clason2020introduction}.
\end{proof}

Now, as $d$ is aligned with a negative subdifferential of the fine-grid objective, we readily show that it is a fine-grid descent direction.

\begin{theorem}
    \label{thm:fb:teosd}
    Suppose \cref{ass:coarse:basic,ass:coarse:coherence} hold.
    Then for $d = I_H^h(\zeta ^{k,m}-\zeta^{k,0})$ and any $\kappa \in (0,1)$, there exists $\theta >0$ such that
    \[
        \label{eq:fb:descent}
        \df{[G+F]}{x^{k}+ \theta d}< \df{[G+F]}{x^{k}}+\kappa\theta \df{{[G+F]}'}{x^{k};d}.
    \]
\end{theorem}

\begin{proof}
    In \cref{cor:fb:cord}, we use the definition of the directional derivative.
\end{proof}

\section{Forward-backward multigrid}
\label{sec:fb}

We now develop the overall multigrid algorithm for \cref{eq:intro:problem:fine-grid}.
Our starting point, again, is the classical forward-backward splitting
\begin{align*}
    x^{k+1} = \proxold{\tau G}{x^k - \tau \nabla F(x^k)}.
\end{align*}
As this is a monotone descent method, \cref{thm:fb:teosd} suggests that performing  coarse-grid iterations between its iterations would not ruin the convergence.
However, details remain to attend to.
To prove convergence, we adapt the technique of \cite{ang2024mgprox}.

\subsection{Algorithm}
\label{sec:full-algorithm}

Our proposed \cref{alg:fb:mg} is, for simplicity, limited to two grid levels, but easily extended to multiple levels.
It depends on \emph{line search} to guarantee \eqref{eq:fb:descent}, as well as a \emph{trigger condition} to perform coarse corrections. Options for the latter include:
\begin{enumerate}[nosep]
    \item heuristic, after enough progress since the previous correction \cite{parpas2017multilevel};
    \item a fixed number of fine iterations between coarse corrections; or
    \item a bounded number of coarse corrections during the algorithm runtime.
\end{enumerate}
Although a standard line search procedure can be used, in practise, for efficiency, we either take at each coarse correction a fixed $\theta_{k} =\bar\theta_k$, or, if this does not satisfy the sufficient decrease condition  \eqref{eq:fb:descent}, reject the coarse grid result with $\theta_k=0$.

\subsection{Convergence}
\label{sec:convergence}

\begin{algorithm}[t]
    \caption{Forward-backward multigrid (FBMG)}
    \label{alg:fb:mg}
    \begin{algorithmic}[1]
        \Require $F,G,F_H$ and $\tau,\tau_H>0$ satisfying \cref{ass:fb:general,ass:coarse:basic}. Sufficient descent parameter $\kappa \in (0, 1)$ as well as a line search procedure and trigger condition.
        \State Choose an initial iterate $x^0 \in X$. Set $J \defeq G+F$.
        \ForAll{$k=0,1,2,\ldots$ until a chosen stopping criterion is fulfilled}
            \If{a trigger condition is satisfied}
                \State Choose an initial coarse point $\zeta^{k,0}$ (e.g., $I_h^H x^k$).
                \State Design $G_H^k$ satisfying the nonsmooth coherence condition (\cref{ass:coarse:coherence})
                \State Set
                $
                    w_{H}^{k} \defeq I_h^H\nabla F(x^k) - \df{\nabla F_{H}}{\zeta ^{k,0}},
                $
                \ForAll{$j=1,\ldots,m-1$}
                    \State $\zeta^{k,j+1} \defeq \proxold{\tau _H G_H^k}{\zeta^{k,j}-\tau _H [\nabla F_H(\zeta^{k,j}) + w_H^k]}$
                \Comment{Coarse-grid FB}
                \EndFor
                \State Set $d \defeq I_H^h(\zeta ^{k,m}-\zeta ^{k,0})$.
                \State Find $\theta_k \ge 0$ such that {\(\df{J}{x^{k}+\theta_k d} \le \df{J}{x^{k}}+\kappa \theta_k\df{{J}'}{x^{k};d}\)}.
                \label{line:fb:mg:linesearch}
                \Comment{Line search}
                \State
                $x^{k+1} \defeq \proxold{\tau G}{z^k + \nabla F(z^k)}$
                \quad\text{for}\quad $z^k \defeq x^k + \theta_k d$
                \label{line:fb:mg:fine-fb1}
                \Comment{Fine-grid FB}
            \Else
                \State\label{line:fb:mg:fine-fb2}
                \smash{$x^{k+1} \defeq \proxold{\tau G}{x^k + \nabla F(x^k)}$}
                \Comment{Fine-grid FB}
            \EndIf
        \EndFor
    \end{algorithmic}
\end{algorithm}

\begin{assumption}
    \label{ass:fb:general}
    $F,G: X \to \extR$ are proper, convex and lower semicontinuous, $F$ Fréchet differentiable with $L$-Lipschitz gradient.
    The step length $\tau \in (0, 1/L)$.
\end{assumption}

\begin{theorem}[Sublinear convergence]
    \label{thm:fb:teoconvsub}
    Suppose \cref{ass:fb:general} holds, and $x^{*}\in \df{\left[\partial J\right]^{-1}}{0}$ where $J:= G+F$.
    Let $\{x^k\}_{k \ge 1}$ be generated by \cref{alg:fb:mg} for an initial $x^0\in X$ such that the corresponding sublevel set is bounded, i.e.,
    \begin{gather}
        \nonumber
        \varepsilon \defeq \diam(\sublev_{J(x^0)}J) := \sup \{ \nr{x-y}{2} \mid J(x), J(y) \le J(x^0)\} < \infty.
    \shortintertext{Then}
        \label{eq:fb:prop}
        J(x^{k+1})-J(x^*) \leq \frac{1}{k}\max \cj{4C,J(x^0)-J(x^*)}
        \quad\text{for}\quad
        C = \frac{2\varepsilon ^2}{\tau(2-\tau L)}>0.
    \end{gather}
\end{theorem}

\begin{proof}
    \Cref{thm:fb:teosd} guarantees the line search on \cref{line:fb:mg:linesearch} of \cref{alg:fb:mg} to be satisfiable (strictly for a $\theta_k>0$, although we also allow $\theta_k=0$ and mere non-increase).
    By standard arguments based on convexity, the descent lemma, and the Pythagoras identity (see the proof of \cite[Theorem 11.4]{clason2020introduction}), \cref{line:fb:mg:fine-fb1} satisfies
    \begin{align}
        \label{eq:fb:descent3}
        J(x^{k+1}) -J(x^*) +\frac{1}{2\tau} \nr{x^{k+1}-x^*}{X}^2 + \frac{1-\tau L}{2\tau}\nr{x^{k+1}-z^k}{X}^2\leq \frac{1}{2\tau} \nr{z^k-x^*}{X}^2.
    \end{align}
    Since $\tau L<1$ we have
    \[
        \begin{split}
            J(x^{k+1}) -J(x^*)
            &
            \leq \frac{1}{2\tau} \df{}{\nr{z^k-x^*}{X}^2-\nr{x^{k+1}-x^*}{X}^2}
            \\
            &
            = \frac{1}{2\tau} \df{}{\nr{z^k-x^*}{X}-\nr{x^{k+1}-x^*}{X}}\df{}{\nr{z^k-x^*}{X}+\nr{x^{k+1}-x^*}{X}}
            \\
            &
            \leq \frac{\varepsilon}{\tau} \df{}{\nr{z^k-x^*}{X}-\nr{x^{k+1}-x^*}{X}}
            \\
            &
            \leq \frac{\varepsilon}{\tau} \nr{x^{k+1}-z^k}{X}.
        \end{split}
    \]
    Rearranging gives
    \begin{equation}
        \label{eq:fb:ecu1}
        \df{}{J(x^{k+1})-J(x^*)}^2 \leq (\varepsilon \tau ^{-1})^2 \nr{x^{k+1}-z^k}{X}^2.
    \end{equation}
    On the other hand, taking $x^*=z^k$ in \eqref{eq:fb:descent3}, and continuing with $J(z^k) \le J(x^k)$ established by the line search procedure on \cref{line:fb:mg:linesearch} of \cref{alg:fb:mg}, we obtain
    \[
        \frac{2-\tau L}{2\tau}\nr{x^{k+1}-z^k}{X}^2
        \le
        J(z^k) - J(x^{k+1})
        \le
        J(x^k) - J(x^{k+1}).
    \]
    Combining with \cref{eq:fb:ecu1} yields
    $
        (J(x^{k+1})-J(x^*))^2\leq C[J(x^k) - J(x^{k+1})].
    $
    Repeating the analysis with $z^k=x^k$ establishes the same result for \cref{line:fb:mg:fine-fb2}.
    Now, according to \cite[Lemma 4]{karimi2017imro}, the monotonically decreasing sequence $\cj{J(x^k)}_{k\in \mathbb{N}}$ satisfies \cref{eq:fb:prop}.
\end{proof}

\section{Total variation regularised imaging problems}
\label{sec:tv}

We will in \cref{sec:numerical} apply \cref{alg:fb:mg} to image processing problems of the form
\begin{equation}
    \label{eq:tv:primal-problem}
    \min _{y\in \R^n}~ \phi(y) + \alpha \nr{\nabla_h y}{2,1}
    \quad\text{for}\quad
    \df{\phi}{y} := \frac{1}{2}\sum _{s=1}^t \nr{T_sy-b_s}{2}^2,
\end{equation}
where $\grad_h \in \linear(\R^n; \R^{D \times n})$ for some dimension $D$.
Since the nonsmooth total variation regulariser is not prox-simple, to derive an efficient method, we will need to work with the dual problem.
Since MRI involves complex numbers, we allow $T_s \in \linear(\C^n; \C^n)$ and $b_s \in \C^n$.
Thus the dual formulation is
\begin{equation}
    \label{eq:tv:dual-problem}
    \min _{x\in \R^{D\times n}}\, \phi ^*(-\nabla_h ^* x) + G(x)
    \quad\text{for}\quad
    \df{G}{x} \defeq (\alpha\nr{\cdot}{2,1})^*(x) = \sum_{i=1}^n \delta _{B(0,\alpha)}(x_{\freevar i}).
\end{equation}
We construct $\phi^*$ in \cref{sec:tv:data-fenchel}, after we have first constructed the coarse nonsmooth function $G_H^k$ and its proximal operator in \cref{sec:tv:coarse,sec:tv:coarse-prox}.

\subsection{The coarse problem}
\label{sec:tv:coarse}

We use the standard restriction operator\footnote{For MRI we could replace $I_H^h$ by $\mathcal{F}^*I_H^h\mathcal{F}$, where $\mathcal{F}$ is the Fourier transform in the relevant grid, but do not currently use this form.} $I_h^H = R  \otimes R$, where, in stencil notation,
$R = \begin{bsmallmatrix}
        \frac{1}{2}& 1 & \frac{1}{2}
       \end{bsmallmatrix};$
see \cite{briggs2000multigrid}.
The prolongation operator is then $I_h^H = \frac{1}{4} (I_H^h)^*$.
We take
\begin{equation}
    \label{eq:tv:gHk}
    G_H^k(\zeta) \defeq \sum_{l=1}^N \delta_{\Omega_l}(\zeta_{\freevar l})
    \quad\text{for}\quad
    \Omega_l \defeq \zeta_{\freevar l}^{k,0} + \polar \Gamma_l,
\end{equation}
where for all coarse pixel indices $l=1,\ldots,N$, we define
\begin{equation}
    \label{eq:tv:gammal}
    \Gamma_l
    \defeq
    [I_h^H \partial G(x^k)]_l
    =
    \left\{
        \sum\nolimits_{p\in A_l}q^k_{\freevar p}
        \,\middle|\,
        q_{\freevar p}^k \in [\partial G(x^k)]_p\ \forall p\in A_l
    \right\},
\end{equation}
with $A_l \subset \{1,\ldots,n\}$ the subset of fine pixel indices $i$ that contribute to the coarse pixel $l$ via $I_h^H$, that is, $[I_h^H]_{li} \ne 0$.
Note that $\Omega_l$ is nonempty, closed and convex for all $l=1,\ldots,N$.
Apart from the Lipschitz gradient, there are no theoretical restrictions on $F_H$, but we
take it as a coarse version of $\phi ^*\circ -\nabla _h^*$, as we describe later in \cref{ex:coarse:FH}, after forming $\phi^*$.

\begin{lemma}
    \Cref{ass:coarse:coherence} holds for $G$ and $G_H^k$ as in \cref{eq:tv:gHk,eq:tv:dual-problem}.
\end{lemma}

\begin{proof}
    We recall for all fine pixel indices $i=1,..,n$ that
    \begin{equation}
        \label{eq:cap4:subdGpsr}
        \left[\partial \df{G}{x^k}\right]_i
        =\df{\partial \delta _{\df{B}{0,\alpha}}}{x^k_{\freevar i}}=\left\{\begin{array}{ll}
        \cj{0}, & x^k_{\freevar i} \in \interior B(0, \alpha),  \\
        \cj{\beta x^k_{\freevar i} \mid \beta\geq 0}, & x^k_{\freevar i} \in \boundary B(0, \alpha), \\
        \emptyset & \text{otherwise}.
        \end{array}\right.
    \end{equation}
    It follows that $\Gamma_l$ is a (possibly non-convex) cone in $\R^D$.
    We then deduce
    \[
        \partial \delta_{\Omega_l}(\zeta_{\freevar l}^{k,0})
        =
        N_{\Omega_l}(\zeta_{\freevar l}^{k,0})
        =
        N_{\Gamma_l^\circ}(0)
        = (\Gamma_l^\circ)^\circ \supset \Gamma_l,
    \]
    and further
    \[
        I_h^H \partial G(x^k)
        = \Gamma_1 \times \cdots \times \Gamma_N
        \subset
        \partial \delta _{\Omega _l}(\zeta _{\freevar 1}^{k,0})
        \times \cdots \times
        \partial \delta _{\Omega _l}(\zeta _{\freevar N}^{k,0})
        = \partial G_H^k(\zeta ^{k,0}).
        \qedhere
    \]
\end{proof}

\subsection{The coarse proximal operator}
\label{sec:tv:coarse-prox}

By \eqref{eq:tv:gHk}, $\Omega_l - \zeta_l^{k,0} = \polar\Gamma_l = \polar{(\bipolar\Gamma_l)}$ is a closed convex cone for each $l$.
When $D=2$, it is therefore defined by at most two “director” vectors.
If we can identify them, it will be possible to write the proximal operator of $G_H^k$ in a simple form.

\begin{lemma}
    \label{lemma:tv:gamma}
    Suppose $D=2$.
    For any coarse pixel index $l \in \{1,\ldots,N\}$, $\bipolar \Gamma_l$ can only take for some fine pixel indices $j,s \in A_l$ one of the forms
    \[
        \cj{0},\quad \ccone\cj{x_{\freevar j}},\quad \ccone\cj{x_{\freevar j},x_{\freevar s}},\quad \ccone\cj{x_{\freevar j},x_{\freevar s},-x_{\freevar j}},\quad \R^2.
    \]
\end{lemma}

\begin{proof}
    We construct the director vectors algorithmically, starting with empty sets $V_l$ and $O_l$. The former will eventually contain the directors, while $O_l$ tracks subspaces generated by opposing vectors.
    For all $p \in A_l$, we repeat steps 1 and 2:
    \begin{enumerate}
        \item
        We omit $x^k_{\freevar p} \in  \interior \df{B}{0,\alpha}$ since, by \cref{eq:tv:gammal}, the subgradient $q_{\freevar p}^k=0$, does not contribute to $\Gamma_l$.
        By contrast, if $x^k_{\freevar p} \in \boundary \df{B}{0,\alpha}$, we add this vector to $V_l$.

        \item
        If $V_l=\{z_{j}, z_{s}, z_{p}\}$ for three distinct vectors, one of them must be superfluous for forming $\polar\Gamma_l$.
        To discard it, we form the linear system
        $
            z_{j} \beta_1 + z_{s} \beta_2 = z_{p}
        $
        for the unknowns $\beta_1$ and $\beta_2$, and consider several cases:
        \begin{enumerate}[label=(\alph*),nosep]
            \item  For a non-unique solution, $z_{s} = c z_{j}$ for some $c \ne 0$.
            When $c > 0$, $z_s$ is superfluous.
            When $c < 0$, $\Gamma_l$ must be contained in the subspace orthogonal to $z_s$.
            In both cases, we take $V_l = \{z_{j},z_{p}\}$, and in the latter add $z_s$ to $O_l$.
        \end{enumerate}
        Otherwise the system has a unique solution, and we continue with the cases:
        \begin{enumerate}[resume*]
            \item If $\beta _1 \geq 0$ and $\beta _2 \geq 0$, then $z_{p} \in \ccone\{z_{j}, z_{s}\}$, so we remove $z_{p}$ from $V_l$.
            \item If $\beta _1 >0$ and $\beta_2 <0$, then $z_{j} \in \ccone \{z_{s}, z_{p}\}$, so we remove $z_{j}$ from $V_l$.
            \item If $\beta _1 <0$ and $\beta_2 >0$, then $z_{s} \in \ccone\{z_{j}, z_{p}\}$, so we remove $z_{s}$ from $V_l$.
            \item If $\beta _1 = 0$ and $\beta_2 <0$ or $\beta _1 < 0$ and $\beta_2 = 0$, then $z_{s} = -z_{p}$ or $z_{j} = -z_{p}$, so we eliminate $z_{p}$ from $V_l$ but add it to $O_l$.
            \item If $\beta _1 < 0$ and $\beta_2 <0$, then $\Gamma_l^\circ=\{0\}$, so we terminate with $\bipolar \Gamma_l = \rea{2}$.
        \end{enumerate}
    \end{enumerate}

    If the construction did not terminate in the above loop, we consider:
    \begin{enumerate}[label=(\roman*)]
        \item If $\abs{V_l}+\abs{O_l} \le 2$, then by construction $O_l=\emptyset$.
        If $V_l=\emptyset$, also $\bipolar \Gamma_l = \cj{0}$.
        Otherwise $\bipolar \Gamma_l = \ccone V_l$.
        \item If $|V_l| +|O_l| = 3$, then $O_l=\{z_p\}$ and $V_l=\{z_{j}, z_{s}\}$ with $z_{p} = -z_{j}$ or $z_{p} = -z_{s}$. In other words, the three vectors define the half-space $\bipolar \Gamma_l = \ccone\cj{z_{j},z_{s},-z_{j}}$.
        \item $|V_l| +|O_l| \geq 4$ then $O_l$ defines two distinct subspaces that contain $\polar \Gamma_l$, which must then by $\{0\}$. Hence $\bipolar \Gamma _l = \rea{2}$.
        \qedhere
    \end{enumerate}
\end{proof}

In the next proposition, $z_{j}^\ortho$ denotes a vector orthogonal to $z_{j}$ with $\sup_{z \in \Gamma} \iprod{z}{z_j^\ortho} \le 0$.
We also write
$
    p(\zeta, z) \defeq \max \cj{0,\pd{\zeta}{z}}z/\nr{z}{2}^2
$
for the projection of $\zeta$ to $z[0,\infty)$.

\begin{proposition}
    \label{prop:tv:prox}
    The proximal operator of $G_H^k$ defined in \eqref{eq:tv:gHk} is given by
    \begin{subequations}%
    \begin{equation}
        \label{eq:tv:prox1}
        [\proxold{\gamma G_H^k}{\zeta}]_l = \left \{\begin{array}{ll}
            \zeta _l, & \bipolar \Gamma _l = \cj{0}, \\
            \zeta _l -  p(\zeta_l - \zeta _l^{k,0},z_{j}), & \bipolar \Gamma _l = \ccone\cj{z_{j}}, \\
            \zeta _l^{k,0} + p(\zeta_l - \zeta _l^{k,0},z_j^\ortho), & \bipolar \Gamma _l = \ccone\cj{z_{j},z_{s},-z_{j}}, \\
            \zeta _l^{k,0}, & \bipolar \Gamma_l = \rea{2}, \\
            \text{see below} &\bipolar  \Gamma_l = \ccone\cj{z_{j},z_{s}},
        \end{array}\right.
    \end{equation}
    for each component $l=1,\ldots,m$, where in the final case,
    \begin{equation}
        \label{eq:tv:prox2}
        [\proxold{\gamma G_H^k}{\zeta}]_l =  \left \{\begin{array}{ll}
            \zeta _l^{k,0}, & \zeta _l\in \ccone\cj{z_{j},z_{s}}, \\
            \zeta _l, & \zeta _l\in \ccone\cj{z_{j}^\ortho,z_{s}^\ortho}, \\
            \zeta _l^{k,0} + p(\zeta_l - \zeta _l^{k,0},z_{j}^\ortho), & \zeta _l\in \ccone\cj{z_{j},z_{j}^\ortho}, \\
            \zeta _l^{k,0} + p(\zeta_l - \zeta _l^{k,0},z_{s}^\ortho), & \zeta _l\in \ccone\cj{z_{s},z_{s}^\ortho}.
        \end{array}\right.
    \end{equation}%
    \end{subequations}%
\end{proposition}

\begin{figure}[t]
    \centering
    \includegraphics[width=0.4\textwidth]{a_deg.png}
    \hfil
    \includegraphics[width=0.37\textwidth]{b.png}
    \caption{Illustration of the middle case of \eqref{eq:tv:prox1} (left) and \eqref{eq:tv:prox2} (right).}
    \label{fig:prox}
\end{figure}

\begin{proof}
    The proximal map of $G_H^k$ separates into individual Euclidean projections onto each $\Omega_l$.
    These are determined by $\bipolar \Gamma_l$, so we consider the cases of \cref{lemma:tv:gamma}:
    \begin{enumerate}[label=(\alph*)]
        \item
        If $\bipolar \Gamma_l = \cj{0}$, we have $\Omega_l = \rea{2}$.
        Hence the proximal map is the identity.

        \item
        If $\bipolar \Gamma _l = \rea{2}$, we have $\Omega _l = \cj{\zeta _l^{k,0}}$, so the proximal map is the constant $\zeta _l^{k,0}$.

        \item
        If $\bipolar \Gamma_l = \ccone\cj{z_{j}}$, then
        $
            \Omega _l
             = \zeta_l^{k,0} + \polar{(\bipolar \Gamma_l)}
             = \zeta_l^{k,0} + \polar{\{z_j\}}.
        $
        Thus $[\proxold{\gamma G_H^k}{\zeta}]_l = \zeta _ l -p(\zeta _l - \zeta _l^{k,0},z_{j})$.

        \item
        If $\bipolar \Gamma _l = \ccone\cj{z_{j},z_{s},-z_{j}}$, then, likewise, $\Omega _l =  \zeta _l^{k,0} + [0, \infty) z_j^\ortho$, resulting in
        $[\proxold{\gamma G_H^k}{\zeta}]_l = \zeta _ l^{k,0} -p(\zeta _l - \zeta _l^{k,0},z_{j}^\ortho)$; see \cref{fig:prox}.

        \item
        When $\bipolar \Gamma_l = \ccone \cj{z_{j},z_{s}}$, we can divide $\R^2$ into four cones, each with distinct projection operator agreeing with \cref{eq:tv:prox2}; see \cref{fig:prox}.
        \qedhere
    \end{enumerate}
\end{proof}

\subsection{The Fenchel conjugate of the (complex) data term}
\label{sec:tv:data-fenchel}

To form the Fenchel conjugate of $\phi$, we start with its derivative.
We write $\norm{\freevar}_{\R^n}$ for the Euclidean norm in $\R^n$ and
$\norm{x}_{\C^n} \defeq \sqrt{\sum_{k=1}^n \abs{x_i}^2}$ for $x=(x_1,\ldots,x_n) \in \C^n$.

\begin{lemma}
    \label{lemma:tv:phi-derivative}
    $\nabla \phi (y) = Ty -e$ for $T \defeq \sum _{s=1}^t \Re T_s^*T_s$ and $e \defeq \sum _{s=1}^t \Re T_s^* b_s$.
\end{lemma}

\begin{proof}
    Due to properties of the complex inner product, for any $y, h \in \R ^n$,
    \begin{align*}
        \nr{T_s(y+h) -b_s}{\C^n}^2 - \nr{T_s y -b_s}{\C^n}^2
        &
        = 2 \Re\pd{T_sy-b_s}{T_sh}_{\mathbb{C}^n} + \nr{T_s h}{\C^n}^{2}
        \\
        &
        =
        2 \iprod{\Re T_s^*(T_sy-b_s)}{h}_{\R^n} + \nr{T_s h}{\C^n}^{2}.
    \end{align*}
    Dividing by $2$ and summing over $s=1,\ldots,t$, therefore
    \[
        \df{\phi}{y+h}-\df{\phi}{y}
        =
        \iprod{Ty-e}{h}_{\R^n} + \frac{1}{2} \sum _{s=1}^t \nr{T_s h}{\C^n}^{2}.
    \]
    By the definition of the Fréchet derivative, the claim follows.
\end{proof}

From the definition of the Fenchel conjugate and the Fermat principle, now
\begin{equation}
    \label{eq:tv:phi-conjugate-0}
    \phi ^*(z) = \pd{z}{T^{-1}(z+e)} - \phi(T^{-1}(z+e)).
\end{equation}
We next develop a more tractable form, which shows \eqref{eq:tv:dual-problem} to have equivalent form
\begin{equation}
    \label{eq:tv:equiv-dual}
    \min _{x\in \R^{D\times n}} \frac{1}{2} \nr{T^{-1/2}(\nabla _h^*x-e)}{2}^2 + \sum _i \delta_{B(0,\alpha)}(x_{\freevar i}).
\end{equation}

\begin{lemma}
    \label{lemma:tv:complex-operator-norm}
    If $A \in \linear(\C^n; \C^n)$.
    Then $\nr{Ay}{\C^n}^2 = \pd{\Re(A^*A)y}{y}_{\R^n}$ for all $y\in \R^n$.
\end{lemma}
\begin{proof}
    Writing $A = A_1 +iA_2$ for $A_1,A_2 \in \rea{n\times n}$, we have $A^* = A_1^T -iA_2^T$.
    Thus
    \begin{align*}
        \nr{Ay}{\C^n}^2 &
        =
        \pd{A^*Ay}{y}_{\mathbb{C}^n}
        =\pd{(A_1^T -iA_2^T)(A_1 +iA_2)y}{y}_{\mathbb{C}^n}
        \\
        & = \pd{(A_1^TA_1 + A_2^TA_2)y}{y}_{\R^n} + i\pd{(A_1^TA_2 - A_2^TA_1)y}{y}_{\R^n}.
    \end{align*}
    The last term is zero by the properties of the real inner product and transpose.
\end{proof}

\begin{lemma}
    Let $r_s = b_s - T_sT^{-1}e$ for $T$ and $e$ from \cref{lemma:tv:phi-derivative}.
    Then
    \[
        \phi^*(z)
        = \frac{1}{2} \nr{T^{-1/2}(z+e)}{\R^n}^2 - \frac{1}{2}\sum _{s=1}^t\nr{r_s}{\C^n}^2 - \nr{T^{-1/2}e}{\R^n}^2
        \quad\forall z \in \R^n.
    \]
\end{lemma}

\begin{proof}
    \Cref{lemma:tv:complex-operator-norm} and the properties of the complex inner product yield
    \begin{equation}
        \label{eq:tv:conjugate:0}
        \begin{aligned}[t]
        \nr{T_s(T^{-1}(z+e))-b_s}{\C^n}^2
        &
        = \nr{T_sT^{-1}z - r_s}{\C^n}^2
        \\
        &
        = \nr{T_sT^{-1}z}{\C^n}^2 - \pd{T_s T^{-1}z}{r_s}_{\mathbb{C}^n} - \pd{r_s}{T_s T^{-1}z}_{\mathbb{C}^n} +\nr{r_s}{\C^n}^2
        \\
        &
        = \iprod{\Re(T_s^*T_s) T^{-1}z}{T^{-1}z}_{\R^n}
        - 2 \Re \pd{T^{-1}z}{T_s^* r_s}_{\mathbb{C}^n}
        + \nr{r_s}{\C^n}^2.
        \end{aligned}
    \end{equation}
    We have $\sum_{s=1}^t T_s^*r_s=\sum_{s=1}^t T_s^* b_s - \sum_{s=1}^t T_s^*T_s T^{-1} e$, hence $\Re \sum_{s=1}^t T_s^*r_s= 0$.
    Since $T^{-1}z \in \R^n$, it follows that $\sum _{s=1}^t\Re \pd{T^{-1}z}{T_s^* r_s}_{\mathbb{C}^n} = 0$.
    Dividing \eqref{eq:tv:conjugate:0} by $2$ and summing over $s=1,\ldots,t$, therefore
    \[
        \phi(T^{-1}(z+e))
        = \frac{1}{2}\pd{z}{T^{-1}z}_{\R^n} + \frac{1}{2}\sum _{s=1}^t\nr{r_s}{\C^n}^2.
    \]
    Using this expression in \eqref{eq:tv:phi-conjugate-0}, the claim readily follows.
\end{proof}

We can now finally suggest one way to form the coarse function $F_H$:

\begin{example}
    \label{ex:coarse:FH}
    Form a coarse discrete gradient operator $\grad_H \in \linear(\R^N; X_H)$ and set
    \[
        F_H(\zeta ) \defeq \frac{1}{2} \nr{T_H^{-1/2}(\grad _H ^*\zeta - b_H)}{\R^N}^2
        \quad\text{for}\quad
        T_H \defeq I_h^H T
        \quad\text{and}\quad
        b_H \defeq I_h^H b.
    \]
\end{example}

\section{Numerical experience}
\label{sec:numerical}

We now report our numerical experience with denoising and MRI.
Both problems have the primal form \eqref{eq:tv:primal-problem}.
We work with the equivalent dual problem \eqref{eq:tv:dual-problem}.
Our Julia implementation is available on Zenodo \cite{multigrid-codes-zenodo}.

\subsection{Denoising}
\label{sec:numerical:denoising}

\makeatletter
\def\hlinewd#1{%
\noalign{\ifnum0=`}\fi\hrule \@height #1
\futurelet\reserved@a\@xhline}
\makeatother

\begin{table}[t]
    \caption{Time (seconds) to reach relative error $\relerr_1$ and $\relerr_2$ for both experiments.}
    \label{tab:error-time}
    \centering
    \begin{tabular}{l@{\quad}|@{\quad}lll@{\quad}|@{\quad}lll}
        \hlinewd{1pt}
        Experiment & $\relerr _1$ & FB & FBMG & $\relerr _2$ & FB & FBMG \\
        \hline
        Denoising & 0.01 & 0.030498 & 0.007404 & 0.001 & 0.231761 & 0.102651 \\
        MRI & 0.01  & 0.069281 & 0.005343 & 0.001  & 0.142469  & 0.046784 \\
        \hlinewd{1pt}
    \end{tabular}
\end{table}

For denoising we use one full sample, i.e., solve \eqref{eq:tv:primal-problem} with $t=1$ and $T_s=I$.
The dual problem \eqref{eq:tv:equiv-dual} is then $\min_{x\in \R^{D\times n}} \nr{\nabla _h^* x - e}{\R^n}^2 +\sum _{i} \delta _{B(0,\alpha)}(x_{\freevar ,i})$.
We use the the \emph{Blue Marble} public domain test image with resolution $3002\times 3000$. We add pixelwise Gaussian noise of standard deviation $\sigma = 0.4$.
The Lipschitz constant $L=L_H=8$ \cite{chambolle2004algorithm}.
We take $\alpha = 0.85$ and $\tau = 0.95/L$ and $\tau_H = 1.95/L$.
In FBMG, we perform $m=6$ coarse steps, based on trial and error, before the first 110 fine iterations only.
For line search, we try $\theta_k = \bar\theta_k \defeq \omega_k\pd{T^{-1}(e-\grad ^* x^{k})}{\grad^* d}/ \nr{T^{-1/2}\grad ^*d}{2}^2 \ge 0$ for the scaling factor $\omega_k = 2/5$, and otherwise fail with $\theta_k=0$.\footnote{When $\omega_k <2$, $\bar\theta_k$ is a scaled-down exact solution to \eqref{eq:fb:descent} for $F=\phi^*$ and $G=0$. Small $\omega_k$ attempts to ensure $x^k+\bar\theta_k d \in B(0, \alpha)^n$. By convexity, this check guarantees descent.}
We illustrate the data and reconstructions in \cref{fig:denoising:compare:FB:FBMG}, and the performance in \cref{fig:denoising:graphs,tab:error-time}, where,
for $x^*$ computed by 100000 iterations of FBMGs, the \emph{relative error}
\begin{equation}
    \label{eq:relerror}
    \relerr=\relerr^k \defeq (v(x^k)-v(x^*))/(v(x^0)-v(x^*))
    \quad\text{with}\quad
    v(x) \defeq \phi^*(-\nabla _h^* x) + G(x).
\end{equation}
The \emph{iteration comparison number} in \cref{fig:denoising:graphs} scales coarse iterations by the ratio of the number of coarse to fine pixels.

\begin{figure}[t]
    \centering
    \begin{subfigure}{0.48\columnwidth}
    \begin{tikzpicture}
        \begin{axis}[%
            width = \linewidth,
            height = 0.6\linewidth,
            axis x line*=bottom,
            axis y line*=left,
            xlabel={Iteration comparison number},
            ylabel={Relative error},
            ymode=log,
            xmode=normal,
            legend pos = north east,
            ]

            \addplot [fb] table[x=iter,y=relative]{fb_simple_final_rel.txt};
            \addlegendentry{FB}

            \addplot [projgrad] table[x=iter,y=relative]{fbmg_simple_final_rel.txt};
            \addlegendentry{FBMG}
        \end{axis}
    \end{tikzpicture}
    \end{subfigure}
    \hfill
    \begin{subfigure}{0.48\columnwidth}
    \begin{tikzpicture}
        \begin{axis}[%
            width = \linewidth,
            height = 0.6\linewidth,
            axis x line*=bottom,
            axis y line*=left,
            xlabel={CPU time},
            ylabel={Relative error},
            ymode=log,
            xmode=normal,
            legend pos = north east,
            ]

            \addplot [fb] table[x=cputime,y=relative]{fb_simple_final_rel.txt};
            \addlegendentry{FB}

            \addplot [projgrad] table[x=cputime,y=relative]{fbmg_simple_final_rel.txt};
            \addlegendentry{FBMG}
        \end{axis}
    \end{tikzpicture}
    \end{subfigure}
    \caption{Relative error \eqref{eq:relerror} versus iteration count and CPU time for denoising.}
    \label{fig:denoising:graphs}
\end{figure}

\begin{figure}[t]
    \centering
    \begin{subfigure}{.2\textwidth}
        \centering
        \includegraphics[width=0.95\linewidth]{original_final_res.png}
        \caption{Original}
        \label{fig:original:image}
    \end{subfigure}%
    \begin{subfigure}{.2\textwidth}
        \centering
        \includegraphics[width=0.95\linewidth]{noisy_final_res.png}
        \caption{Noisy}
        \label{fig:noisy:image}
    \end{subfigure}
    \begin{subfigure}{.2\textwidth}
        \centering
        \includegraphics[width=0.95\linewidth]{fbsol_final_res.png}
        \caption{FB}
        \label{fig:fb:solution:image}
    \end{subfigure}%
    \begin{subfigure}{.2\textwidth}
        \centering
        \includegraphics[width=0.95\linewidth]{fbmgsol_final_res.png}
        \caption{FBMG}
        \label{fig:fbmg:solution:image}
    \end{subfigure}%
    \caption{Denoising data and results at relative error $\relerr= 0.001$.
    }
    \label{fig:denoising:compare:FB:FBMG}
\end{figure}

\subsection{Magnetic resonance imaging}
\label{sec:numerical:mri}

We take $T_s \defeq S_s\mathcal{F}$, where $\mathcal{F}$ is the discrete Fourier transform and $S_s$ is a frequency subsampling operator. Then $T$ of \cref{lemma:tv:phi-derivative} becomes
$
    T = \Re \mathcal{F}^* S \mathcal{F} = \mathcal{F}^* \Sym S \mathcal{F}
$
for $S \defeq \sum _{s=1}^t S_s^* S_s$ and $\Sym S$ its symmetrisation over positive and negative frequencies in both axes.
Thus $T^{-1}$, required for the dual problem \eqref{eq:tv:equiv-dual}, exists and is easily calculated when each frequency is sampled by some $S_s$.
With $t=21$, we form each subsampling mask $S_1,\ldots,S_t$ by random sampling 150 lines in the Fourier space from a uniform distribution of such subsets of lines.
We use the MRI phantom of \cite{belzunce2018high} with resolution $583\times 493$, and add complex Gaussian noise with standard deviation $\sigma = 50$.
We take $\alpha = 1.15$, and the step length parameters and line search as for denoising with $L = 8\norm{T^{-1}}$, $L_H = 8 \norm{T_H^{-1}}$.
We perform $m=6$ coarse steps before the first 500 fine iterations only.
We illustrate the data, reconstructions, and performance in \cref{fig:mri:compare:FB:FBMG:mri,fig:mri:graphs,tab:error-time}.

\begin{figure}[t]
    \centering
    \begin{subfigure}{0.48\columnwidth}
    \begin{tikzpicture}
        \begin{axis}[%
            width = \linewidth,
            height = 0.6\linewidth,
            axis x line*=bottom,
            axis y line*=left,
            xlabel={Iteration comparison number},
            ylabel={Relative error},
            ymode=log,
            xmode=normal,
            legend pos = north east,
            ]
            \addplot [fb] table[x=iter,y=relative]{fb_mri_final_noise50_21m.txt};
            \addlegendentry{FB}

            \addplot [projgrad] table[x=iter,y=relative]{fbmg_mri_final_noise50_21m.txt};
            \addlegendentry{FBMG}
        \end{axis}
    \end{tikzpicture}
    \end{subfigure}
    \begin{subfigure}{0.48\columnwidth}
    \begin{tikzpicture}
        \begin{axis}[%
            width = \linewidth,
            height = 0.6\linewidth,
            axis x line*=bottom,
            axis y line*=left,
            xlabel={CPU time},
            ylabel={Relative error},
            ymode=log,
            xmode=normal,
            legend pos = north east,
            xtick = {0.0, 0.04, 0.08, 0.12},
            xticklabel style={
                /pgf/number format/precision=3,
                /pgf/number format/fixed,
            }
            ]

            \addplot [fb] table[x=cputime,y=relative]{fb_mri_final_noise50_21m.txt};
            \addlegendentry{FB}

            \addplot [projgrad] table[x=cputime,y=relative]{fbmg_mri_final_noise50_21m.txt};
            \addlegendentry{FBMG}
        \end{axis}
    \end{tikzpicture}
    \end{subfigure}
    \caption{Relative error \eqref{eq:relerror} versus both iteration count and CPU time for MRI.}
    \label{fig:mri:graphs}
\end{figure}

\begin{figure}[t]
    \centering
    \begin{subfigure}{.195\textwidth}
        \centering
        \includegraphics[width=0.95\linewidth]{original_mri_final.png}
        \caption{Original}
        \label{fig:original:image:mri}
    \end{subfigure}%
    \begin{subfigure}{.195\textwidth}
        \centering
        \includegraphics[width=0.95\linewidth]{21muestras50_st20.png}
        \caption{Sample}
        \label{fig:noisy:line:sample:mri}
    \end{subfigure}%
    \begin{subfigure}{.195\textwidth}
        \centering
        \includegraphics[width=0.95\linewidth]{21muestras50_ct20.png}
        \caption{Backproj.}
        \label{fig:noisy:transform:sample:mri}
    \end{subfigure}%
    \begin{subfigure}{.195\textwidth}
        \centering
        \includegraphics[width=0.95\linewidth]{fbsol_mri_final_noise50.png}
        \caption{FB}
        \label{fig:fb:solution:image:mri}
    \end{subfigure}%
    \begin{subfigure}{.195\textwidth}
        \centering
        \includegraphics[width=0.95\linewidth]{fbmgsol_mri_final_noise50.png}
        \caption{FBMG}
        \label{fig:fbmg:solution:image:mri}
    \end{subfigure}%
    \caption{MRI data and results at relative error $\relerr=0.01$.
    (\subref{fig:noisy:transform:sample:mri}) is the backprojection of the Fourier line sample (\subref{fig:noisy:line:sample:mri}).
    There are altogether $t=100$ such samples.
    }
    \label{fig:mri:compare:FB:FBMG:mri}
\end{figure}

\subsection{Conclusions}

\Cref{fig:denoising:graphs,fig:mri:graphs,tab:error-time} indicate that while the performance improvements in denoising are noticeable, they are \emph{very significant} for the much more expensive MRI problem.
This can be expected, as the fine grid Fourier transform is an expensive operation.
The situation is comparable to \cite{parpas2017multilevel}, who do deblurring directly with the primal problem.
This requires proximal map of total variation to be solved numerically (as a denoising problem) on each fine-grid step, while in the coarse grid they avoid this by using a smooth problem and gradient steps.
For multigrid optimisation methods to be meaningful, it therefore appears that the coarse-grid problems have to be significantly cheaper than the fine-grid problems.

\bibliographystyle{jnsao}
% This must be in the first 5 lines to tell arXiv to use pdfLaTeX, which is strongly recommended.
\pdfoutput=1
% In particular, the hyperref package requires pdfLaTeX in order to break URLs across lines.

\documentclass[11pt]{article}

% Change "review" to "final" to generate the final (sometimes called camera-ready) version.
% Change to "preprint" to generate a non-anonymous version with page numbers.
\usepackage{acl}

% Standard package includes
\usepackage{times}
\usepackage{latexsym}

% Draw tables
\usepackage{booktabs}
\usepackage{multirow}
\usepackage{xcolor}
\usepackage{colortbl}
\usepackage{array} 
\usepackage{amsmath}

\newcolumntype{C}{>{\centering\arraybackslash}p{0.07\textwidth}}
% For proper rendering and hyphenation of words containing Latin characters (including in bib files)
\usepackage[T1]{fontenc}
% For Vietnamese characters
% \usepackage[T5]{fontenc}
% See https://www.latex-project.org/help/documentation/encguide.pdf for other character sets
% This assumes your files are encoded as UTF8
\usepackage[utf8]{inputenc}

% This is not strictly necessary, and may be commented out,
% but it will improve the layout of the manuscript,
% and will typically save some space.
\usepackage{microtype}
\DeclareMathOperator*{\argmax}{arg\,max}
% This is also not strictly necessary, and may be commented out.
% However, it will improve the aesthetics of text in
% the typewriter font.
\usepackage{inconsolata}

%Including images in your LaTeX document requires adding
%additional package(s)
\usepackage{graphicx}
% If the title and author information does not fit in the area allocated, uncomment the following
%
%\setlength\titlebox{<dim>}
%
% and set <dim> to something 5cm or larger.

\title{Wi-Chat: Large Language Model Powered Wi-Fi Sensing}

% Author information can be set in various styles:
% For several authors from the same institution:
% \author{Author 1 \and ... \and Author n \\
%         Address line \\ ... \\ Address line}
% if the names do not fit well on one line use
%         Author 1 \\ {\bf Author 2} \\ ... \\ {\bf Author n} \\
% For authors from different institutions:
% \author{Author 1 \\ Address line \\  ... \\ Address line
%         \And  ... \And
%         Author n \\ Address line \\ ... \\ Address line}
% To start a separate ``row'' of authors use \AND, as in
% \author{Author 1 \\ Address line \\  ... \\ Address line
%         \AND
%         Author 2 \\ Address line \\ ... \\ Address line \And
%         Author 3 \\ Address line \\ ... \\ Address line}

% \author{First Author \\
%   Affiliation / Address line 1 \\
%   Affiliation / Address line 2 \\
%   Affiliation / Address line 3 \\
%   \texttt{email@domain} \\\And
%   Second Author \\
%   Affiliation / Address line 1 \\
%   Affiliation / Address line 2 \\
%   Affiliation / Address line 3 \\
%   \texttt{email@domain} \\}
% \author{Haohan Yuan \qquad Haopeng Zhang\thanks{corresponding author} \\ 
%   ALOHA Lab, University of Hawaii at Manoa \\
%   % Affiliation / Address line 2 \\
%   % Affiliation / Address line 3 \\
%   \texttt{\{haohany,haopengz\}@hawaii.edu}}
  
\author{
{Haopeng Zhang$\dag$\thanks{These authors contributed equally to this work.}, Yili Ren$\ddagger$\footnotemark[1], Haohan Yuan$\dag$, Jingzhe Zhang$\ddagger$, Yitong Shen$\ddagger$} \\
ALOHA Lab, University of Hawaii at Manoa$\dag$, University of South Florida$\ddagger$ \\
\{haopengz, haohany\}@hawaii.edu\\
\{yiliren, jingzhe, shen202\}@usf.edu\\}



  
%\author{
%  \textbf{First Author\textsuperscript{1}},
%  \textbf{Second Author\textsuperscript{1,2}},
%  \textbf{Third T. Author\textsuperscript{1}},
%  \textbf{Fourth Author\textsuperscript{1}},
%\\
%  \textbf{Fifth Author\textsuperscript{1,2}},
%  \textbf{Sixth Author\textsuperscript{1}},
%  \textbf{Seventh Author\textsuperscript{1}},
%  \textbf{Eighth Author \textsuperscript{1,2,3,4}},
%\\
%  \textbf{Ninth Author\textsuperscript{1}},
%  \textbf{Tenth Author\textsuperscript{1}},
%  \textbf{Eleventh E. Author\textsuperscript{1,2,3,4,5}},
%  \textbf{Twelfth Author\textsuperscript{1}},
%\\
%  \textbf{Thirteenth Author\textsuperscript{3}},
%  \textbf{Fourteenth F. Author\textsuperscript{2,4}},
%  \textbf{Fifteenth Author\textsuperscript{1}},
%  \textbf{Sixteenth Author\textsuperscript{1}},
%\\
%  \textbf{Seventeenth S. Author\textsuperscript{4,5}},
%  \textbf{Eighteenth Author\textsuperscript{3,4}},
%  \textbf{Nineteenth N. Author\textsuperscript{2,5}},
%  \textbf{Twentieth Author\textsuperscript{1}}
%\\
%\\
%  \textsuperscript{1}Affiliation 1,
%  \textsuperscript{2}Affiliation 2,
%  \textsuperscript{3}Affiliation 3,
%  \textsuperscript{4}Affiliation 4,
%  \textsuperscript{5}Affiliation 5
%\\
%  \small{
%    \textbf{Correspondence:} \href{mailto:email@domain}{email@domain}
%  }
%}

\begin{document}
\maketitle
\begin{abstract}
Recent advancements in Large Language Models (LLMs) have demonstrated remarkable capabilities across diverse tasks. However, their potential to integrate physical model knowledge for real-world signal interpretation remains largely unexplored. In this work, we introduce Wi-Chat, the first LLM-powered Wi-Fi-based human activity recognition system. We demonstrate that LLMs can process raw Wi-Fi signals and infer human activities by incorporating Wi-Fi sensing principles into prompts. Our approach leverages physical model insights to guide LLMs in interpreting Channel State Information (CSI) data without traditional signal processing techniques. Through experiments on real-world Wi-Fi datasets, we show that LLMs exhibit strong reasoning capabilities, achieving zero-shot activity recognition. These findings highlight a new paradigm for Wi-Fi sensing, expanding LLM applications beyond conventional language tasks and enhancing the accessibility of wireless sensing for real-world deployments.
\end{abstract}

\section{Introduction}

In today’s rapidly evolving digital landscape, the transformative power of web technologies has redefined not only how services are delivered but also how complex tasks are approached. Web-based systems have become increasingly prevalent in risk control across various domains. This widespread adoption is due their accessibility, scalability, and ability to remotely connect various types of users. For example, these systems are used for process safety management in industry~\cite{kannan2016web}, safety risk early warning in urban construction~\cite{ding2013development}, and safe monitoring of infrastructural systems~\cite{repetto2018web}. Within these web-based risk management systems, the source search problem presents a huge challenge. Source search refers to the task of identifying the origin of a risky event, such as a gas leak and the emission point of toxic substances. This source search capability is crucial for effective risk management and decision-making.

Traditional approaches to implementing source search capabilities into the web systems often rely on solely algorithmic solutions~\cite{ristic2016study}. These methods, while relatively straightforward to implement, often struggle to achieve acceptable performances due to algorithmic local optima and complex unknown environments~\cite{zhao2020searching}. More recently, web crowdsourcing has emerged as a promising alternative for tackling the source search problem by incorporating human efforts in these web systems on-the-fly~\cite{zhao2024user}. This approach outsources the task of addressing issues encountered during the source search process to human workers, leveraging their capabilities to enhance system performance.

These solutions often employ a human-AI collaborative way~\cite{zhao2023leveraging} where algorithms handle exploration-exploitation and report the encountered problems while human workers resolve complex decision-making bottlenecks to help the algorithms getting rid of local deadlocks~\cite{zhao2022crowd}. Although effective, this paradigm suffers from two inherent limitations: increased operational costs from continuous human intervention, and slow response times of human workers due to sequential decision-making. These challenges motivate our investigation into developing autonomous systems that preserve human-like reasoning capabilities while reducing dependency on massive crowdsourced labor.

Furthermore, recent advancements in large language models (LLMs)~\cite{chang2024survey} and multi-modal LLMs (MLLMs)~\cite{huang2023chatgpt} have unveiled promising avenues for addressing these challenges. One clear opportunity involves the seamless integration of visual understanding and linguistic reasoning for robust decision-making in search tasks. However, whether large models-assisted source search is really effective and efficient for improving the current source search algorithms~\cite{ji2022source} remains unknown. \textit{To address the research gap, we are particularly interested in answering the following two research questions in this work:}

\textbf{\textit{RQ1: }}How can source search capabilities be integrated into web-based systems to support decision-making in time-sensitive risk management scenarios? 
% \sq{I mention ``time-sensitive'' here because I feel like we shall say something about the response time -- LLM has to be faster than humans}

\textbf{\textit{RQ2: }}How can MLLMs and LLMs enhance the effectiveness and efficiency of existing source search algorithms? 

% \textit{\textbf{RQ2:}} To what extent does the performance of large models-assisted search align with or approach the effectiveness of human-AI collaborative search? 

To answer the research questions, we propose a novel framework called Auto-\
S$^2$earch (\textbf{Auto}nomous \textbf{S}ource \textbf{Search}) and implement a prototype system that leverages advanced web technologies to simulate real-world conditions for zero-shot source search. Unlike traditional methods that rely on pre-defined heuristics or extensive human intervention, AutoS$^2$earch employs a carefully designed prompt that encapsulates human rationales, thereby guiding the MLLM to generate coherent and accurate scene descriptions from visual inputs about four directional choices. Based on these language-based descriptions, the LLM is enabled to determine the optimal directional choice through chain-of-thought (CoT) reasoning. Comprehensive empirical validation demonstrates that AutoS$^2$-\ 
earch achieves a success rate of 95–98\%, closely approaching the performance of human-AI collaborative search across 20 benchmark scenarios~\cite{zhao2023leveraging}. 

Our work indicates that the role of humans in future web crowdsourcing tasks may evolve from executors to validators or supervisors. Furthermore, incorporating explanations of LLM decisions into web-based system interfaces has the potential to help humans enhance task performance in risk control.






\section{Related Work}
\label{sec:relatedworks}

% \begin{table*}[t]
% \centering 
% \renewcommand\arraystretch{0.98}
% \fontsize{8}{10}\selectfont \setlength{\tabcolsep}{0.4em}
% \begin{tabular}{@{}lc|cc|cc|cc@{}}
% \toprule
% \textbf{Methods}           & \begin{tabular}[c]{@{}c@{}}\textbf{Training}\\ \textbf{Paradigm}\end{tabular} & \begin{tabular}[c]{@{}c@{}}\textbf{$\#$ PT Data}\\ \textbf{(Tokens)}\end{tabular} & \begin{tabular}[c]{@{}c@{}}\textbf{$\#$ IFT Data}\\ \textbf{(Samples)}\end{tabular} & \textbf{Code}  & \begin{tabular}[c]{@{}c@{}}\textbf{Natural}\\ \textbf{Language}\end{tabular} & \begin{tabular}[c]{@{}c@{}}\textbf{Action}\\ \textbf{Trajectories}\end{tabular} & \begin{tabular}[c]{@{}c@{}}\textbf{API}\\ \textbf{Documentation}\end{tabular}\\ \midrule 
% NexusRaven~\citep{srinivasan2023nexusraven} & IFT & - & - & \textcolor{green}{\CheckmarkBold} & \textcolor{green}{\CheckmarkBold} &\textcolor{red}{\XSolidBrush}&\textcolor{red}{\XSolidBrush}\\
% AgentInstruct~\citep{zeng2023agenttuning} & IFT & - & 2k & \textcolor{green}{\CheckmarkBold} & \textcolor{green}{\CheckmarkBold} &\textcolor{red}{\XSolidBrush}&\textcolor{red}{\XSolidBrush} \\
% AgentEvol~\citep{xi2024agentgym} & IFT & - & 14.5k & \textcolor{green}{\CheckmarkBold} & \textcolor{green}{\CheckmarkBold} &\textcolor{green}{\CheckmarkBold}&\textcolor{red}{\XSolidBrush} \\
% Gorilla~\citep{patil2023gorilla}& IFT & - & 16k & \textcolor{green}{\CheckmarkBold} & \textcolor{green}{\CheckmarkBold} &\textcolor{red}{\XSolidBrush}&\textcolor{green}{\CheckmarkBold}\\
% OpenFunctions-v2~\citep{patil2023gorilla} & IFT & - & 65k & \textcolor{green}{\CheckmarkBold} & \textcolor{green}{\CheckmarkBold} &\textcolor{red}{\XSolidBrush}&\textcolor{green}{\CheckmarkBold}\\
% LAM~\citep{zhang2024agentohana} & IFT & - & 42.6k & \textcolor{green}{\CheckmarkBold} & \textcolor{green}{\CheckmarkBold} &\textcolor{green}{\CheckmarkBold}&\textcolor{red}{\XSolidBrush} \\
% xLAM~\citep{liu2024apigen} & IFT & - & 60k & \textcolor{green}{\CheckmarkBold} & \textcolor{green}{\CheckmarkBold} &\textcolor{green}{\CheckmarkBold}&\textcolor{red}{\XSolidBrush} \\\midrule
% LEMUR~\citep{xu2024lemur} & PT & 90B & 300k & \textcolor{green}{\CheckmarkBold} & \textcolor{green}{\CheckmarkBold} &\textcolor{green}{\CheckmarkBold}&\textcolor{red}{\XSolidBrush}\\
% \rowcolor{teal!12} \method & PT & 103B & 95k & \textcolor{green}{\CheckmarkBold} & \textcolor{green}{\CheckmarkBold} & \textcolor{green}{\CheckmarkBold} & \textcolor{green}{\CheckmarkBold} \\
% \bottomrule
% \end{tabular}
% \caption{Summary of existing tuning- and pretraining-based LLM agents with their training sample sizes. "PT" and "IFT" denote "Pre-Training" and "Instruction Fine-Tuning", respectively. }
% \label{tab:related}
% \end{table*}

\begin{table*}[ht]
\begin{threeparttable}
\centering 
\renewcommand\arraystretch{0.98}
\fontsize{7}{9}\selectfont \setlength{\tabcolsep}{0.2em}
\begin{tabular}{@{}l|c|c|ccc|cc|cc|cccc@{}}
\toprule
\textbf{Methods} & \textbf{Datasets}           & \begin{tabular}[c]{@{}c@{}}\textbf{Training}\\ \textbf{Paradigm}\end{tabular} & \begin{tabular}[c]{@{}c@{}}\textbf{\# PT Data}\\ \textbf{(Tokens)}\end{tabular} & \begin{tabular}[c]{@{}c@{}}\textbf{\# IFT Data}\\ \textbf{(Samples)}\end{tabular} & \textbf{\# APIs} & \textbf{Code}  & \begin{tabular}[c]{@{}c@{}}\textbf{Nat.}\\ \textbf{Lang.}\end{tabular} & \begin{tabular}[c]{@{}c@{}}\textbf{Action}\\ \textbf{Traj.}\end{tabular} & \begin{tabular}[c]{@{}c@{}}\textbf{API}\\ \textbf{Doc.}\end{tabular} & \begin{tabular}[c]{@{}c@{}}\textbf{Func.}\\ \textbf{Call}\end{tabular} & \begin{tabular}[c]{@{}c@{}}\textbf{Multi.}\\ \textbf{Step}\end{tabular}  & \begin{tabular}[c]{@{}c@{}}\textbf{Plan}\\ \textbf{Refine}\end{tabular}  & \begin{tabular}[c]{@{}c@{}}\textbf{Multi.}\\ \textbf{Turn}\end{tabular}\\ \midrule 
\multicolumn{13}{l}{\emph{Instruction Finetuning-based LLM Agents for Intrinsic Reasoning}}  \\ \midrule
FireAct~\cite{chen2023fireact} & FireAct & IFT & - & 2.1K & 10 & \textcolor{red}{\XSolidBrush} &\textcolor{green}{\CheckmarkBold} &\textcolor{green}{\CheckmarkBold}  & \textcolor{red}{\XSolidBrush} &\textcolor{green}{\CheckmarkBold} & \textcolor{red}{\XSolidBrush} &\textcolor{green}{\CheckmarkBold} & \textcolor{red}{\XSolidBrush} \\
ToolAlpaca~\cite{tang2023toolalpaca} & ToolAlpaca & IFT & - & 4.0K & 400 & \textcolor{red}{\XSolidBrush} &\textcolor{green}{\CheckmarkBold} &\textcolor{green}{\CheckmarkBold} & \textcolor{red}{\XSolidBrush} &\textcolor{green}{\CheckmarkBold} & \textcolor{red}{\XSolidBrush}  &\textcolor{green}{\CheckmarkBold} & \textcolor{red}{\XSolidBrush}  \\
ToolLLaMA~\cite{qin2023toolllm} & ToolBench & IFT & - & 12.7K & 16,464 & \textcolor{red}{\XSolidBrush} &\textcolor{green}{\CheckmarkBold} &\textcolor{green}{\CheckmarkBold} &\textcolor{red}{\XSolidBrush} &\textcolor{green}{\CheckmarkBold}&\textcolor{green}{\CheckmarkBold}&\textcolor{green}{\CheckmarkBold} &\textcolor{green}{\CheckmarkBold}\\
AgentEvol~\citep{xi2024agentgym} & AgentTraj-L & IFT & - & 14.5K & 24 &\textcolor{red}{\XSolidBrush} & \textcolor{green}{\CheckmarkBold} &\textcolor{green}{\CheckmarkBold}&\textcolor{red}{\XSolidBrush} &\textcolor{green}{\CheckmarkBold}&\textcolor{red}{\XSolidBrush} &\textcolor{red}{\XSolidBrush} &\textcolor{green}{\CheckmarkBold}\\
Lumos~\cite{yin2024agent} & Lumos & IFT  & - & 20.0K & 16 &\textcolor{red}{\XSolidBrush} & \textcolor{green}{\CheckmarkBold} & \textcolor{green}{\CheckmarkBold} &\textcolor{red}{\XSolidBrush} & \textcolor{green}{\CheckmarkBold} & \textcolor{green}{\CheckmarkBold} &\textcolor{red}{\XSolidBrush} & \textcolor{green}{\CheckmarkBold}\\
Agent-FLAN~\cite{chen2024agent} & Agent-FLAN & IFT & - & 24.7K & 20 &\textcolor{red}{\XSolidBrush} & \textcolor{green}{\CheckmarkBold} & \textcolor{green}{\CheckmarkBold} &\textcolor{red}{\XSolidBrush} & \textcolor{green}{\CheckmarkBold}& \textcolor{green}{\CheckmarkBold}&\textcolor{red}{\XSolidBrush} & \textcolor{green}{\CheckmarkBold}\\
AgentTuning~\citep{zeng2023agenttuning} & AgentInstruct & IFT & - & 35.0K & - &\textcolor{red}{\XSolidBrush} & \textcolor{green}{\CheckmarkBold} & \textcolor{green}{\CheckmarkBold} &\textcolor{red}{\XSolidBrush} & \textcolor{green}{\CheckmarkBold} &\textcolor{red}{\XSolidBrush} &\textcolor{red}{\XSolidBrush} & \textcolor{green}{\CheckmarkBold}\\\midrule
\multicolumn{13}{l}{\emph{Instruction Finetuning-based LLM Agents for Function Calling}} \\\midrule
NexusRaven~\citep{srinivasan2023nexusraven} & NexusRaven & IFT & - & - & 116 & \textcolor{green}{\CheckmarkBold} & \textcolor{green}{\CheckmarkBold}  & \textcolor{green}{\CheckmarkBold} &\textcolor{red}{\XSolidBrush} & \textcolor{green}{\CheckmarkBold} &\textcolor{red}{\XSolidBrush} &\textcolor{red}{\XSolidBrush}&\textcolor{red}{\XSolidBrush}\\
Gorilla~\citep{patil2023gorilla} & Gorilla & IFT & - & 16.0K & 1,645 & \textcolor{green}{\CheckmarkBold} &\textcolor{red}{\XSolidBrush} &\textcolor{red}{\XSolidBrush}&\textcolor{green}{\CheckmarkBold} &\textcolor{green}{\CheckmarkBold} &\textcolor{red}{\XSolidBrush} &\textcolor{red}{\XSolidBrush} &\textcolor{red}{\XSolidBrush}\\
OpenFunctions-v2~\citep{patil2023gorilla} & OpenFunctions-v2 & IFT & - & 65.0K & - & \textcolor{green}{\CheckmarkBold} & \textcolor{green}{\CheckmarkBold} &\textcolor{red}{\XSolidBrush} &\textcolor{green}{\CheckmarkBold} &\textcolor{green}{\CheckmarkBold} &\textcolor{red}{\XSolidBrush} &\textcolor{red}{\XSolidBrush} &\textcolor{red}{\XSolidBrush}\\
API Pack~\cite{guo2024api} & API Pack & IFT & - & 1.1M & 11,213 &\textcolor{green}{\CheckmarkBold} &\textcolor{red}{\XSolidBrush} &\textcolor{green}{\CheckmarkBold} &\textcolor{red}{\XSolidBrush} &\textcolor{green}{\CheckmarkBold} &\textcolor{red}{\XSolidBrush}&\textcolor{red}{\XSolidBrush}&\textcolor{red}{\XSolidBrush}\\ 
LAM~\citep{zhang2024agentohana} & AgentOhana & IFT & - & 42.6K & - & \textcolor{green}{\CheckmarkBold} & \textcolor{green}{\CheckmarkBold} &\textcolor{green}{\CheckmarkBold}&\textcolor{red}{\XSolidBrush} &\textcolor{green}{\CheckmarkBold}&\textcolor{red}{\XSolidBrush}&\textcolor{green}{\CheckmarkBold}&\textcolor{green}{\CheckmarkBold}\\
xLAM~\citep{liu2024apigen} & APIGen & IFT & - & 60.0K & 3,673 & \textcolor{green}{\CheckmarkBold} & \textcolor{green}{\CheckmarkBold} &\textcolor{green}{\CheckmarkBold}&\textcolor{red}{\XSolidBrush} &\textcolor{green}{\CheckmarkBold}&\textcolor{red}{\XSolidBrush}&\textcolor{green}{\CheckmarkBold}&\textcolor{green}{\CheckmarkBold}\\\midrule
\multicolumn{13}{l}{\emph{Pretraining-based LLM Agents}}  \\\midrule
% LEMUR~\citep{xu2024lemur} & PT & 90B & 300.0K & - & \textcolor{green}{\CheckmarkBold} & \textcolor{green}{\CheckmarkBold} &\textcolor{green}{\CheckmarkBold}&\textcolor{red}{\XSolidBrush} & \textcolor{red}{\XSolidBrush} &\textcolor{green}{\CheckmarkBold} &\textcolor{red}{\XSolidBrush}&\textcolor{red}{\XSolidBrush}\\
\rowcolor{teal!12} \method & \dataset & PT & 103B & 95.0K  & 76,537  & \textcolor{green}{\CheckmarkBold} & \textcolor{green}{\CheckmarkBold} & \textcolor{green}{\CheckmarkBold} & \textcolor{green}{\CheckmarkBold} & \textcolor{green}{\CheckmarkBold} & \textcolor{green}{\CheckmarkBold} & \textcolor{green}{\CheckmarkBold} & \textcolor{green}{\CheckmarkBold}\\
\bottomrule
\end{tabular}
% \begin{tablenotes}
%     \item $^*$ In addition, the StarCoder-API can offer 4.77M more APIs.
% \end{tablenotes}
\caption{Summary of existing instruction finetuning-based LLM agents for intrinsic reasoning and function calling, along with their training resources and sample sizes. "PT" and "IFT" denote "Pre-Training" and "Instruction Fine-Tuning", respectively.}
\vspace{-2ex}
\label{tab:related}
\end{threeparttable}
\end{table*}

\noindent \textbf{Prompting-based LLM Agents.} Due to the lack of agent-specific pre-training corpus, existing LLM agents rely on either prompt engineering~\cite{hsieh2023tool,lu2024chameleon,yao2022react,wang2023voyager} or instruction fine-tuning~\cite{chen2023fireact,zeng2023agenttuning} to understand human instructions, decompose high-level tasks, generate grounded plans, and execute multi-step actions. 
However, prompting-based methods mainly depend on the capabilities of backbone LLMs (usually commercial LLMs), failing to introduce new knowledge and struggling to generalize to unseen tasks~\cite{sun2024adaplanner,zhuang2023toolchain}. 

\noindent \textbf{Instruction Finetuning-based LLM Agents.} Considering the extensive diversity of APIs and the complexity of multi-tool instructions, tool learning inherently presents greater challenges than natural language tasks, such as text generation~\cite{qin2023toolllm}.
Post-training techniques focus more on instruction following and aligning output with specific formats~\cite{patil2023gorilla,hao2024toolkengpt,qin2023toolllm,schick2024toolformer}, rather than fundamentally improving model knowledge or capabilities. 
Moreover, heavy fine-tuning can hinder generalization or even degrade performance in non-agent use cases, potentially suppressing the original base model capabilities~\cite{ghosh2024a}.

\noindent \textbf{Pretraining-based LLM Agents.} While pre-training serves as an essential alternative, prior works~\cite{nijkamp2023codegen,roziere2023code,xu2024lemur,patil2023gorilla} have primarily focused on improving task-specific capabilities (\eg, code generation) instead of general-domain LLM agents, due to single-source, uni-type, small-scale, and poor-quality pre-training data. 
Existing tool documentation data for agent training either lacks diverse real-world APIs~\cite{patil2023gorilla, tang2023toolalpaca} or is constrained to single-tool or single-round tool execution. 
Furthermore, trajectory data mostly imitate expert behavior or follow function-calling rules with inferior planning and reasoning, failing to fully elicit LLMs' capabilities and handle complex instructions~\cite{qin2023toolllm}. 
Given a wide range of candidate API functions, each comprising various function names and parameters available at every planning step, identifying globally optimal solutions and generalizing across tasks remains highly challenging.



\section{Preliminaries}
\label{Preliminaries}
\begin{figure*}[t]
    \centering
    \includegraphics[width=0.95\linewidth]{fig/HealthGPT_Framework.png}
    \caption{The \ourmethod{} architecture integrates hierarchical visual perception and H-LoRA, employing a task-specific hard router to select visual features and H-LoRA plugins, ultimately generating outputs with an autoregressive manner.}
    \label{fig:architecture}
\end{figure*}
\noindent\textbf{Large Vision-Language Models.} 
The input to a LVLM typically consists of an image $x^{\text{img}}$ and a discrete text sequence $x^{\text{txt}}$. The visual encoder $\mathcal{E}^{\text{img}}$ converts the input image $x^{\text{img}}$ into a sequence of visual tokens $\mathcal{V} = [v_i]_{i=1}^{N_v}$, while the text sequence $x^{\text{txt}}$ is mapped into a sequence of text tokens $\mathcal{T} = [t_i]_{i=1}^{N_t}$ using an embedding function $\mathcal{E}^{\text{txt}}$. The LLM $\mathcal{M_\text{LLM}}(\cdot|\theta)$ models the joint probability of the token sequence $\mathcal{U} = \{\mathcal{V},\mathcal{T}\}$, which is expressed as:
\begin{equation}
    P_\theta(R | \mathcal{U}) = \prod_{i=1}^{N_r} P_\theta(r_i | \{\mathcal{U}, r_{<i}\}),
\end{equation}
where $R = [r_i]_{i=1}^{N_r}$ is the text response sequence. The LVLM iteratively generates the next token $r_i$ based on $r_{<i}$. The optimization objective is to minimize the cross-entropy loss of the response $\mathcal{R}$.
% \begin{equation}
%     \mathcal{L}_{\text{VLM}} = \mathbb{E}_{R|\mathcal{U}}\left[-\log P_\theta(R | \mathcal{U})\right]
% \end{equation}
It is worth noting that most LVLMs adopt a design paradigm based on ViT, alignment adapters, and pre-trained LLMs\cite{liu2023llava,liu2024improved}, enabling quick adaptation to downstream tasks.


\noindent\textbf{VQGAN.}
VQGAN~\cite{esser2021taming} employs latent space compression and indexing mechanisms to effectively learn a complete discrete representation of images. VQGAN first maps the input image $x^{\text{img}}$ to a latent representation $z = \mathcal{E}(x)$ through a encoder $\mathcal{E}$. Then, the latent representation is quantized using a codebook $\mathcal{Z} = \{z_k\}_{k=1}^K$, generating a discrete index sequence $\mathcal{I} = [i_m]_{m=1}^N$, where $i_m \in \mathcal{Z}$ represents the quantized code index:
\begin{equation}
    \mathcal{I} = \text{Quantize}(z|\mathcal{Z}) = \arg\min_{z_k \in \mathcal{Z}} \| z - z_k \|_2.
\end{equation}
In our approach, the discrete index sequence $\mathcal{I}$ serves as a supervisory signal for the generation task, enabling the model to predict the index sequence $\hat{\mathcal{I}}$ from input conditions such as text or other modality signals.  
Finally, the predicted index sequence $\hat{\mathcal{I}}$ is upsampled by the VQGAN decoder $G$, generating the high-quality image $\hat{x}^\text{img} = G(\hat{\mathcal{I}})$.



\noindent\textbf{Low Rank Adaptation.} 
LoRA\cite{hu2021lora} effectively captures the characteristics of downstream tasks by introducing low-rank adapters. The core idea is to decompose the bypass weight matrix $\Delta W\in\mathbb{R}^{d^{\text{in}} \times d^{\text{out}}}$ into two low-rank matrices $ \{A \in \mathbb{R}^{d^{\text{in}} \times r}, B \in \mathbb{R}^{r \times d^{\text{out}}} \}$, where $ r \ll \min\{d^{\text{in}}, d^{\text{out}}\} $, significantly reducing learnable parameters. The output with the LoRA adapter for the input $x$ is then given by:
\begin{equation}
    h = x W_0 + \alpha x \Delta W/r = x W_0 + \alpha xAB/r,
\end{equation}
where matrix $ A $ is initialized with a Gaussian distribution, while the matrix $ B $ is initialized as a zero matrix. The scaling factor $ \alpha/r $ controls the impact of $ \Delta W $ on the model.

\section{HealthGPT}
\label{Method}


\subsection{Unified Autoregressive Generation.}  
% As shown in Figure~\ref{fig:architecture}, 
\ourmethod{} (Figure~\ref{fig:architecture}) utilizes a discrete token representation that covers both text and visual outputs, unifying visual comprehension and generation as an autoregressive task. 
For comprehension, $\mathcal{M}_\text{llm}$ receives the input joint sequence $\mathcal{U}$ and outputs a series of text token $\mathcal{R} = [r_1, r_2, \dots, r_{N_r}]$, where $r_i \in \mathcal{V}_{\text{txt}}$, and $\mathcal{V}_{\text{txt}}$ represents the LLM's vocabulary:
\begin{equation}
    P_\theta(\mathcal{R} \mid \mathcal{U}) = \prod_{i=1}^{N_r} P_\theta(r_i \mid \mathcal{U}, r_{<i}).
\end{equation}
For generation, $\mathcal{M}_\text{llm}$ first receives a special start token $\langle \text{START\_IMG} \rangle$, then generates a series of tokens corresponding to the VQGAN indices $\mathcal{I} = [i_1, i_2, \dots, i_{N_i}]$, where $i_j \in \mathcal{V}_{\text{vq}}$, and $\mathcal{V}_{\text{vq}}$ represents the index range of VQGAN. Upon completion of generation, the LLM outputs an end token $\langle \text{END\_IMG} \rangle$:
\begin{equation}
    P_\theta(\mathcal{I} \mid \mathcal{U}) = \prod_{j=1}^{N_i} P_\theta(i_j \mid \mathcal{U}, i_{<j}).
\end{equation}
Finally, the generated index sequence $\mathcal{I}$ is fed into the decoder $G$, which reconstructs the target image $\hat{x}^{\text{img}} = G(\mathcal{I})$.

\subsection{Hierarchical Visual Perception}  
Given the differences in visual perception between comprehension and generation tasks—where the former focuses on abstract semantics and the latter emphasizes complete semantics—we employ ViT to compress the image into discrete visual tokens at multiple hierarchical levels.
Specifically, the image is converted into a series of features $\{f_1, f_2, \dots, f_L\}$ as it passes through $L$ ViT blocks.

To address the needs of various tasks, the hidden states are divided into two types: (i) \textit{Concrete-grained features} $\mathcal{F}^{\text{Con}} = \{f_1, f_2, \dots, f_k\}, k < L$, derived from the shallower layers of ViT, containing sufficient global features, suitable for generation tasks; 
(ii) \textit{Abstract-grained features} $\mathcal{F}^{\text{Abs}} = \{f_{k+1}, f_{k+2}, \dots, f_L\}$, derived from the deeper layers of ViT, which contain abstract semantic information closer to the text space, suitable for comprehension tasks.

The task type $T$ (comprehension or generation) determines which set of features is selected as the input for the downstream large language model:
\begin{equation}
    \mathcal{F}^{\text{img}}_T =
    \begin{cases}
        \mathcal{F}^{\text{Con}}, & \text{if } T = \text{generation task} \\
        \mathcal{F}^{\text{Abs}}, & \text{if } T = \text{comprehension task}
    \end{cases}
\end{equation}
We integrate the image features $\mathcal{F}^{\text{img}}_T$ and text features $\mathcal{T}$ into a joint sequence through simple concatenation, which is then fed into the LLM $\mathcal{M}_{\text{llm}}$ for autoregressive generation.
% :
% \begin{equation}
%     \mathcal{R} = \mathcal{M}_{\text{llm}}(\mathcal{U}|\theta), \quad \mathcal{U} = [\mathcal{F}^{\text{img}}_T; \mathcal{T}]
% \end{equation}
\subsection{Heterogeneous Knowledge Adaptation}
We devise H-LoRA, which stores heterogeneous knowledge from comprehension and generation tasks in separate modules and dynamically routes to extract task-relevant knowledge from these modules. 
At the task level, for each task type $ T $, we dynamically assign a dedicated H-LoRA submodule $ \theta^T $, which is expressed as:
\begin{equation}
    \mathcal{R} = \mathcal{M}_\text{LLM}(\mathcal{U}|\theta, \theta^T), \quad \theta^T = \{A^T, B^T, \mathcal{R}^T_\text{outer}\}.
\end{equation}
At the feature level for a single task, H-LoRA integrates the idea of Mixture of Experts (MoE)~\cite{masoudnia2014mixture} and designs an efficient matrix merging and routing weight allocation mechanism, thus avoiding the significant computational delay introduced by matrix splitting in existing MoELoRA~\cite{luo2024moelora}. Specifically, we first merge the low-rank matrices (rank = r) of $ k $ LoRA experts into a unified matrix:
\begin{equation}
    \mathbf{A}^{\text{merged}}, \mathbf{B}^{\text{merged}} = \text{Concat}(\{A_i\}_1^k), \text{Concat}(\{B_i\}_1^k),
\end{equation}
where $ \mathbf{A}^{\text{merged}} \in \mathbb{R}^{d^\text{in} \times rk} $ and $ \mathbf{B}^{\text{merged}} \in \mathbb{R}^{rk \times d^\text{out}} $. The $k$-dimension routing layer generates expert weights $ \mathcal{W} \in \mathbb{R}^{\text{token\_num} \times k} $ based on the input hidden state $ x $, and these are expanded to $ \mathbb{R}^{\text{token\_num} \times rk} $ as follows:
\begin{equation}
    \mathcal{W}^\text{expanded} = \alpha k \mathcal{W} / r \otimes \mathbf{1}_r,
\end{equation}
where $ \otimes $ denotes the replication operation.
The overall output of H-LoRA is computed as:
\begin{equation}
    \mathcal{O}^\text{H-LoRA} = (x \mathbf{A}^{\text{merged}} \odot \mathcal{W}^\text{expanded}) \mathbf{B}^{\text{merged}},
\end{equation}
where $ \odot $ represents element-wise multiplication. Finally, the output of H-LoRA is added to the frozen pre-trained weights to produce the final output:
\begin{equation}
    \mathcal{O} = x W_0 + \mathcal{O}^\text{H-LoRA}.
\end{equation}
% In summary, H-LoRA is a task-based dynamic PEFT method that achieves high efficiency in single-task fine-tuning.

\subsection{Training Pipeline}

\begin{figure}[t]
    \centering
    \hspace{-4mm}
    \includegraphics[width=0.94\linewidth]{fig/data.pdf}
    \caption{Data statistics of \texttt{VL-Health}. }
    \label{fig:data}
\end{figure}
\noindent \textbf{1st Stage: Multi-modal Alignment.} 
In the first stage, we design separate visual adapters and H-LoRA submodules for medical unified tasks. For the medical comprehension task, we train abstract-grained visual adapters using high-quality image-text pairs to align visual embeddings with textual embeddings, thereby enabling the model to accurately describe medical visual content. During this process, the pre-trained LLM and its corresponding H-LoRA submodules remain frozen. In contrast, the medical generation task requires training concrete-grained adapters and H-LoRA submodules while keeping the LLM frozen. Meanwhile, we extend the textual vocabulary to include multimodal tokens, enabling the support of additional VQGAN vector quantization indices. The model trains on image-VQ pairs, endowing the pre-trained LLM with the capability for image reconstruction. This design ensures pixel-level consistency of pre- and post-LVLM. The processes establish the initial alignment between the LLM’s outputs and the visual inputs.

\noindent \textbf{2nd Stage: Heterogeneous H-LoRA Plugin Adaptation.}  
The submodules of H-LoRA share the word embedding layer and output head but may encounter issues such as bias and scale inconsistencies during training across different tasks. To ensure that the multiple H-LoRA plugins seamlessly interface with the LLMs and form a unified base, we fine-tune the word embedding layer and output head using a small amount of mixed data to maintain consistency in the model weights. Specifically, during this stage, all H-LoRA submodules for different tasks are kept frozen, with only the word embedding layer and output head being optimized. Through this stage, the model accumulates foundational knowledge for unified tasks by adapting H-LoRA plugins.

\begin{table*}[!t]
\centering
\caption{Comparison of \ourmethod{} with other LVLMs and unified multi-modal models on medical visual comprehension tasks. \textbf{Bold} and \underline{underlined} text indicates the best performance and second-best performance, respectively.}
\resizebox{\textwidth}{!}{
\begin{tabular}{c|lcc|cccccccc|c}
\toprule
\rowcolor[HTML]{E9F3FE} &  &  &  & \multicolumn{2}{c}{\textbf{VQA-RAD \textuparrow}} & \multicolumn{2}{c}{\textbf{SLAKE \textuparrow}} & \multicolumn{2}{c}{\textbf{PathVQA \textuparrow}} &  &  &  \\ 
\cline{5-10}
\rowcolor[HTML]{E9F3FE}\multirow{-2}{*}{\textbf{Type}} & \multirow{-2}{*}{\textbf{Model}} & \multirow{-2}{*}{\textbf{\# Params}} & \multirow{-2}{*}{\makecell{\textbf{Medical} \\ \textbf{LVLM}}} & \textbf{close} & \textbf{all} & \textbf{close} & \textbf{all} & \textbf{close} & \textbf{all} & \multirow{-2}{*}{\makecell{\textbf{MMMU} \\ \textbf{-Med}}\textuparrow} & \multirow{-2}{*}{\textbf{OMVQA}\textuparrow} & \multirow{-2}{*}{\textbf{Avg. \textuparrow}} \\ 
\midrule \midrule
\multirow{9}{*}{\textbf{Comp. Only}} 
& Med-Flamingo & 8.3B & \Large \ding{51} & 58.6 & 43.0 & 47.0 & 25.5 & 61.9 & 31.3 & 28.7 & 34.9 & 41.4 \\
& LLaVA-Med & 7B & \Large \ding{51} & 60.2 & 48.1 & 58.4 & 44.8 & 62.3 & 35.7 & 30.0 & 41.3 & 47.6 \\
& HuatuoGPT-Vision & 7B & \Large \ding{51} & 66.9 & 53.0 & 59.8 & 49.1 & 52.9 & 32.0 & 42.0 & 50.0 & 50.7 \\
& BLIP-2 & 6.7B & \Large \ding{55} & 43.4 & 36.8 & 41.6 & 35.3 & 48.5 & 28.8 & 27.3 & 26.9 & 36.1 \\
& LLaVA-v1.5 & 7B & \Large \ding{55} & 51.8 & 42.8 & 37.1 & 37.7 & 53.5 & 31.4 & 32.7 & 44.7 & 41.5 \\
& InstructBLIP & 7B & \Large \ding{55} & 61.0 & 44.8 & 66.8 & 43.3 & 56.0 & 32.3 & 25.3 & 29.0 & 44.8 \\
& Yi-VL & 6B & \Large \ding{55} & 52.6 & 42.1 & 52.4 & 38.4 & 54.9 & 30.9 & 38.0 & 50.2 & 44.9 \\
& InternVL2 & 8B & \Large \ding{55} & 64.9 & 49.0 & 66.6 & 50.1 & 60.0 & 31.9 & \underline{43.3} & 54.5 & 52.5\\
& Llama-3.2 & 11B & \Large \ding{55} & 68.9 & 45.5 & 72.4 & 52.1 & 62.8 & 33.6 & 39.3 & 63.2 & 54.7 \\
\midrule
\multirow{5}{*}{\textbf{Comp. \& Gen.}} 
& Show-o & 1.3B & \Large \ding{55} & 50.6 & 33.9 & 31.5 & 17.9 & 52.9 & 28.2 & 22.7 & 45.7 & 42.6 \\
& Unified-IO 2 & 7B & \Large \ding{55} & 46.2 & 32.6 & 35.9 & 21.9 & 52.5 & 27.0 & 25.3 & 33.0 & 33.8 \\
& Janus & 1.3B & \Large \ding{55} & 70.9 & 52.8 & 34.7 & 26.9 & 51.9 & 27.9 & 30.0 & 26.8 & 33.5 \\
& \cellcolor[HTML]{DAE0FB}HealthGPT-M3 & \cellcolor[HTML]{DAE0FB}3.8B & \cellcolor[HTML]{DAE0FB}\Large \ding{51} & \cellcolor[HTML]{DAE0FB}\underline{73.7} & \cellcolor[HTML]{DAE0FB}\underline{55.9} & \cellcolor[HTML]{DAE0FB}\underline{74.6} & \cellcolor[HTML]{DAE0FB}\underline{56.4} & \cellcolor[HTML]{DAE0FB}\underline{78.7} & \cellcolor[HTML]{DAE0FB}\underline{39.7} & \cellcolor[HTML]{DAE0FB}\underline{43.3} & \cellcolor[HTML]{DAE0FB}\underline{68.5} & \cellcolor[HTML]{DAE0FB}\underline{61.3} \\
& \cellcolor[HTML]{DAE0FB}HealthGPT-L14 & \cellcolor[HTML]{DAE0FB}14B & \cellcolor[HTML]{DAE0FB}\Large \ding{51} & \cellcolor[HTML]{DAE0FB}\textbf{77.7} & \cellcolor[HTML]{DAE0FB}\textbf{58.3} & \cellcolor[HTML]{DAE0FB}\textbf{76.4} & \cellcolor[HTML]{DAE0FB}\textbf{64.5} & \cellcolor[HTML]{DAE0FB}\textbf{85.9} & \cellcolor[HTML]{DAE0FB}\textbf{44.4} & \cellcolor[HTML]{DAE0FB}\textbf{49.2} & \cellcolor[HTML]{DAE0FB}\textbf{74.4} & \cellcolor[HTML]{DAE0FB}\textbf{66.4} \\
\bottomrule
\end{tabular}
}
\label{tab:results}
\end{table*}
\begin{table*}[ht]
    \centering
    \caption{The experimental results for the four modality conversion tasks.}
    \resizebox{\textwidth}{!}{
    \begin{tabular}{l|ccc|ccc|ccc|ccc}
        \toprule
        \rowcolor[HTML]{E9F3FE} & \multicolumn{3}{c}{\textbf{CT to MRI (Brain)}} & \multicolumn{3}{c}{\textbf{CT to MRI (Pelvis)}} & \multicolumn{3}{c}{\textbf{MRI to CT (Brain)}} & \multicolumn{3}{c}{\textbf{MRI to CT (Pelvis)}} \\
        \cline{2-13}
        \rowcolor[HTML]{E9F3FE}\multirow{-2}{*}{\textbf{Model}}& \textbf{SSIM $\uparrow$} & \textbf{PSNR $\uparrow$} & \textbf{MSE $\downarrow$} & \textbf{SSIM $\uparrow$} & \textbf{PSNR $\uparrow$} & \textbf{MSE $\downarrow$} & \textbf{SSIM $\uparrow$} & \textbf{PSNR $\uparrow$} & \textbf{MSE $\downarrow$} & \textbf{SSIM $\uparrow$} & \textbf{PSNR $\uparrow$} & \textbf{MSE $\downarrow$} \\
        \midrule \midrule
        pix2pix & 71.09 & 32.65 & 36.85 & 59.17 & 31.02 & 51.91 & 78.79 & 33.85 & 28.33 & 72.31 & 32.98 & 36.19 \\
        CycleGAN & 54.76 & 32.23 & 40.56 & 54.54 & 30.77 & 55.00 & 63.75 & 31.02 & 52.78 & 50.54 & 29.89 & 67.78 \\
        BBDM & {71.69} & {32.91} & {34.44} & 57.37 & 31.37 & 48.06 & \textbf{86.40} & 34.12 & 26.61 & {79.26} & 33.15 & 33.60 \\
        Vmanba & 69.54 & 32.67 & 36.42 & {63.01} & {31.47} & {46.99} & 79.63 & 34.12 & 26.49 & 77.45 & 33.53 & 31.85 \\
        DiffMa & 71.47 & 32.74 & 35.77 & 62.56 & 31.43 & 47.38 & 79.00 & {34.13} & {26.45} & 78.53 & {33.68} & {30.51} \\
        \rowcolor[HTML]{DAE0FB}HealthGPT-M3 & \underline{79.38} & \underline{33.03} & \underline{33.48} & \underline{71.81} & \underline{31.83} & \underline{43.45} & {85.06} & \textbf{34.40} & \textbf{25.49} & \underline{84.23} & \textbf{34.29} & \textbf{27.99} \\
        \rowcolor[HTML]{DAE0FB}HealthGPT-L14 & \textbf{79.73} & \textbf{33.10} & \textbf{32.96} & \textbf{71.92} & \textbf{31.87} & \textbf{43.09} & \underline{85.31} & \underline{34.29} & \underline{26.20} & \textbf{84.96} & \underline{34.14} & \underline{28.13} \\
        \bottomrule
    \end{tabular}
    }
    \label{tab:conversion}
\end{table*}

\noindent \textbf{3rd Stage: Visual Instruction Fine-Tuning.}  
In the third stage, we introduce additional task-specific data to further optimize the model and enhance its adaptability to downstream tasks such as medical visual comprehension (e.g., medical QA, medical dialogues, and report generation) or generation tasks (e.g., super-resolution, denoising, and modality conversion). Notably, by this stage, the word embedding layer and output head have been fine-tuned, only the H-LoRA modules and adapter modules need to be trained. This strategy significantly improves the model's adaptability and flexibility across different tasks.


\section{Experiment}
\label{s:experiment}

\subsection{Data Description}
We evaluate our method on FI~\cite{you2016building}, Twitter\_LDL~\cite{yang2017learning} and Artphoto~\cite{machajdik2010affective}.
FI is a public dataset built from Flickr and Instagram, with 23,308 images and eight emotion categories, namely \textit{amusement}, \textit{anger}, \textit{awe},  \textit{contentment}, \textit{disgust}, \textit{excitement},  \textit{fear}, and \textit{sadness}. 
% Since images in FI are all copyrighted by law, some images are corrupted now, so we remove these samples and retain 21,828 images.
% T4SA contains images from Twitter, which are classified into three categories: \textit{positive}, \textit{neutral}, and \textit{negative}. In this paper, we adopt the base version of B-T4SA, which contains 470,586 images and provides text descriptions of the corresponding tweets.
Twitter\_LDL contains 10,045 images from Twitter, with the same eight categories as the FI dataset.
% 。
For these two datasets, they are randomly split into 80\%
training and 20\% testing set.
Artphoto contains 806 artistic photos from the DeviantArt website, which we use to further evaluate the zero-shot capability of our model.
% on the small-scale dataset.
% We construct and publicly release the first image sentiment analysis dataset containing metadata.
% 。

% Based on these datasets, we are the first to construct and publicly release metadata-enhanced image sentiment analysis datasets. These datasets include scenes, tags, descriptions, and corresponding confidence scores, and are available at this link for future research purposes.


% 
\begin{table}[t]
\centering
% \begin{center}
\caption{Overall performance of different models on FI and Twitter\_LDL datasets.}
\label{tab:cap1}
% \resizebox{\linewidth}{!}
{
\begin{tabular}{l|c|c|c|c}
\hline
\multirow{2}{*}{\textbf{Model}} & \multicolumn{2}{c|}{\textbf{FI}}  & \multicolumn{2}{c}{\textbf{Twitter\_LDL}} \\ \cline{2-5} 
  & \textbf{Accuracy} & \textbf{F1} & \textbf{Accuracy} & \textbf{F1}  \\ \hline
% (\rownumber)~AlexNet~\cite{krizhevsky2017imagenet}  & 58.13\% & 56.35\%  & 56.24\%& 55.02\%  \\ 
% (\rownumber)~VGG16~\cite{simonyan2014very}  & 63.75\%& 63.08\%  & 59.34\%& 59.02\%  \\ 
(\rownumber)~ResNet101~\cite{he2016deep} & 66.16\%& 65.56\%  & 62.02\% & 61.34\%  \\ 
(\rownumber)~CDA~\cite{han2023boosting} & 66.71\%& 65.37\%  & 64.14\% & 62.85\%  \\ 
(\rownumber)~CECCN~\cite{ruan2024color} & 67.96\%& 66.74\%  & 64.59\%& 64.72\% \\ 
(\rownumber)~EmoVIT~\cite{xie2024emovit} & 68.09\%& 67.45\%  & 63.12\% & 61.97\%  \\ 
(\rownumber)~ComLDL~\cite{zhang2022compound} & 68.83\%& 67.28\%  & 65.29\% & 63.12\%  \\ 
(\rownumber)~WSDEN~\cite{li2023weakly} & 69.78\%& 69.61\%  & 67.04\% & 65.49\% \\ 
(\rownumber)~ECWA~\cite{deng2021emotion} & 70.87\%& 69.08\%  & 67.81\% & 66.87\%  \\ 
(\rownumber)~EECon~\cite{yang2023exploiting} & 71.13\%& 68.34\%  & 64.27\%& 63.16\%  \\ 
(\rownumber)~MAM~\cite{zhang2024affective} & 71.44\%  & 70.83\% & 67.18\%  & 65.01\%\\ 
(\rownumber)~TGCA-PVT~\cite{chen2024tgca}   & 73.05\%  & 71.46\% & 69.87\%  & 68.32\% \\ 
(\rownumber)~OEAN~\cite{zhang2024object}   & 73.40\%  & 72.63\% & 70.52\%  & 69.47\% \\ \hline
(\rownumber)~\shortname  & \textbf{79.48\%} & \textbf{79.22\%} & \textbf{74.12\%} & \textbf{73.09\%} \\ \hline
\end{tabular}
}
\vspace{-6mm}
% \end{center}
\end{table}
% 

\subsection{Experiment Setting}
% \subsubsection{Model Setting.}
% 
\textbf{Model Setting:}
For feature representation, we set $k=10$ to select object tags, and adopt clip-vit-base-patch32 as the pre-trained model for unified feature representation.
Moreover, we empirically set $(d_e, d_h, d_k, d_s) = (512, 128, 16, 64)$, and set the classification class $L$ to 8.

% 

\textbf{Training Setting:}
To initialize the model, we set all weights such as $\boldsymbol{W}$ following the truncated normal distribution, and use AdamW optimizer with the learning rate of $1 \times 10^{-4}$.
% warmup scheduler of cosine, warmup steps of 2000.
Furthermore, we set the batch size to 32 and the epoch of the training process to 200.
During the implementation, we utilize \textit{PyTorch} to build our entire model.
% , and our project codes are publicly available at https://github.com/zzmyrep/MESN.
% Our project codes as well as data are all publicly available on GitHub\footnote{https://github.com/zzmyrep/KBCEN}.
% Code is available at \href{https://github.com/zzmyrep/KBCEN}{https://github.com/zzmyrep/KBCEN}.

\textbf{Evaluation Metrics:}
Following~\cite{zhang2024affective, chen2024tgca, zhang2024object}, we adopt \textit{accuracy} and \textit{F1} as our evaluation metrics to measure the performance of different methods for image sentiment analysis. 



\subsection{Experiment Result}
% We compare our model against the following baselines: AlexNet~\cite{krizhevsky2017imagenet}, VGG16~\cite{simonyan2014very}, ResNet101~\cite{he2016deep}, CECCN~\cite{ruan2024color}, EmoVIT~\cite{xie2024emovit}, WSCNet~\cite{yang2018weakly}, ECWA~\cite{deng2021emotion}, EECon~\cite{yang2023exploiting}, MAM~\cite{zhang2024affective} and TGCA-PVT~\cite{chen2024tgca}, and the overall results are summarized in Table~\ref{tab:cap1}.
We compare our model against several baselines, and the overall results are summarized in Table~\ref{tab:cap1}.
We observe that our model achieves the best performance in both accuracy and F1 metrics, significantly outperforming the previous models. 
This superior performance is mainly attributed to our effective utilization of metadata to enhance image sentiment analysis, as well as the exceptional capability of the unified sentiment transformer framework we developed. These results strongly demonstrate that our proposed method can bring encouraging performance for image sentiment analysis.

\setcounter{magicrownumbers}{0} 
\begin{table}[t]
\begin{center}
\caption{Ablation study of~\shortname~on FI dataset.} 
% \vspace{1mm}
\label{tab:cap2}
\resizebox{.9\linewidth}{!}
{
\begin{tabular}{lcc}
  \hline
  \textbf{Model} & \textbf{Accuracy} & \textbf{F1} \\
  \hline
  (\rownumber)~Ours (w/o vision) & 65.72\% & 64.54\% \\
  (\rownumber)~Ours (w/o text description) & 74.05\% & 72.58\% \\
  (\rownumber)~Ours (w/o object tag) & 77.45\% & 76.84\% \\
  (\rownumber)~Ours (w/o scene tag) & 78.47\% & 78.21\% \\
  \hline
  (\rownumber)~Ours (w/o unified embedding) & 76.41\% & 76.23\% \\
  (\rownumber)~Ours (w/o adaptive learning) & 76.83\% & 76.56\% \\
  (\rownumber)~Ours (w/o cross-modal fusion) & 76.85\% & 76.49\% \\
  \hline
  (\rownumber)~Ours  & \textbf{79.48\%} & \textbf{79.22\%} \\
  \hline
\end{tabular}
}
\end{center}
\vspace{-5mm}
\end{table}


\begin{figure}[t]
\centering
% \vspace{-2mm}
\includegraphics[width=0.42\textwidth]{fig/2dvisual-linux4-paper2.pdf}
\caption{Visualization of feature distribution on eight categories before (left) and after (right) model processing.}
% 
\label{fig:visualization}
\vspace{-5mm}
\end{figure}

\subsection{Ablation Performance}
In this subsection, we conduct an ablation study to examine which component is really important for performance improvement. The results are reported in Table~\ref{tab:cap2}.

For information utilization, we observe a significant decline in model performance when visual features are removed. Additionally, the performance of \shortname~decreases when different metadata are removed separately, which means that text description, object tag, and scene tag are all critical for image sentiment analysis.
Recalling the model architecture, we separately remove transformer layers of the unified representation module, the adaptive learning module, and the cross-modal fusion module, replacing them with MLPs of the same parameter scale.
In this way, we can observe varying degrees of decline in model performance, indicating that these modules are indispensable for our model to achieve better performance.

\subsection{Visualization}
% 


% % 开始使用minipage进行左右排列
% \begin{minipage}[t]{0.45\textwidth}  % 子图1宽度为45%
%     \centering
%     \includegraphics[width=\textwidth]{2dvisual.pdf}  % 插入图片
%     \captionof{figure}{Visualization of feature distribution.}  % 使用captionof添加图片标题
%     \label{fig:visualization}
% \end{minipage}


% \begin{figure}[t]
% \centering
% \vspace{-2mm}
% \includegraphics[width=0.45\textwidth]{fig/2dvisual.pdf}
% \caption{Visualization of feature distribution.}
% \label{fig:visualization}
% % \vspace{-4mm}
% \end{figure}

% \begin{figure}[t]
% \centering
% \vspace{-2mm}
% \includegraphics[width=0.45\textwidth]{fig/2dvisual-linux3-paper.pdf}
% \caption{Visualization of feature distribution.}
% \label{fig:visualization}
% % \vspace{-4mm}
% \end{figure}



\begin{figure}[tbp]   
\vspace{-4mm}
  \centering            
  \subfloat[Depth of adaptive learning layers]   
  {
    \label{fig:subfig1}\includegraphics[width=0.22\textwidth]{fig/fig_sensitivity-a5}
  }
  \subfloat[Depth of fusion layers]
  {
    % \label{fig:subfig2}\includegraphics[width=0.22\textwidth]{fig/fig_sensitivity-b2}
    \label{fig:subfig2}\includegraphics[width=0.22\textwidth]{fig/fig_sensitivity-b2-num.pdf}
  }
  \caption{Sensitivity study of \shortname~on different depth. }   
  \label{fig:fig_sensitivity}  
\vspace{-2mm}
\end{figure}

% \begin{figure}[htbp]
% \centerline{\includegraphics{2dvisual.pdf}}
% \caption{Visualization of feature distribution.}
% \label{fig:visualization}
% \end{figure}

% In Fig.~\ref{fig:visualization}, we use t-SNE~\cite{van2008visualizing} to reduce the dimension of data features for visualization, Figure in left represents the metadata features before model processing, the features are obtained by embedding through the CLIP model, and figure in right shows the features of the data after model processing, it can be observed that after the model processing, the data with different label categories fall in different regions in the space, therefore, we can conclude that the Therefore, we can conclude that the model can effectively utilize the information contained in the metadata and use it to guide the model for classification.

In Fig.~\ref{fig:visualization}, we use t-SNE~\cite{van2008visualizing} to reduce the dimension of data features for visualization.
The left figure shows metadata features before being processed by our model (\textit{i.e.}, embedded by CLIP), while the right shows the distribution of features after being processed by our model.
We can observe that after the model processing, data with the same label are closer to each other, while others are farther away.
Therefore, it shows that the model can effectively utilize the information contained in the metadata and use it to guide the classification process.

\subsection{Sensitivity Analysis}
% 
In this subsection, we conduct a sensitivity analysis to figure out the effect of different depth settings of adaptive learning layers and fusion layers. 
% In this subsection, we conduct a sensitivity analysis to figure out the effect of different depth settings on the model. 
% Fig.~\ref{fig:fig_sensitivity} presents the effect of different depth settings of adaptive learning layers and fusion layers. 
Taking Fig.~\ref{fig:fig_sensitivity} (a) as an example, the model performance improves with increasing depth, reaching the best performance at a depth of 4.
% Taking Fig.~\ref{fig:fig_sensitivity} (a) as an example, the performance of \shortname~improves with the increase of depth at first, reaching the best performance at a depth of 4.
When the depth continues to increase, the accuracy decreases to varying degrees.
Similar results can be observed in Fig.~\ref{fig:fig_sensitivity} (b).
Therefore, we set their depths to 4 and 6 respectively to achieve the best results.

% Through our experiments, we can observe that the effect of modifying these hyperparameters on the results of the experiments is very weak, and the surface model is not sensitive to the hyperparameters.


\subsection{Zero-shot Capability}
% 

% (1)~GCH~\cite{2010Analyzing} & 21.78\% & (5)~RA-DLNet~\cite{2020A} & 34.01\% \\ \hline
% (2)~WSCNet~\cite{2019WSCNet}  & 30.25\% & (6)~CECCN~\cite{ruan2024color} & 43.83\% \\ \hline
% (3)~PCNN~\cite{2015Robust} & 31.68\%  & (7)~EmoVIT~\cite{xie2024emovit} & 44.90\% \\ \hline
% (4)~AR~\cite{2018Visual} & 32.67\% & (8)~Ours (Zero-shot) & 47.83\% \\ \hline


\begin{table}[t]
\centering
\caption{Zero-shot capability of \shortname.}
\label{tab:cap3}
\resizebox{1\linewidth}{!}
{
\begin{tabular}{lc|lc}
\hline
\textbf{Model} & \textbf{Accuracy} & \textbf{Model} & \textbf{Accuracy} \\ \hline
(1)~WSCNet~\cite{2019WSCNet}  & 30.25\% & (5)~MAM~\cite{zhang2024affective} & 39.56\%  \\ \hline
(2)~AR~\cite{2018Visual} & 32.67\% & (6)~CECCN~\cite{ruan2024color} & 43.83\% \\ \hline
(3)~RA-DLNet~\cite{2020A} & 34.01\%  & (7)~EmoVIT~\cite{xie2024emovit} & 44.90\% \\ \hline
(4)~CDA~\cite{han2023boosting} & 38.64\% & (8)~Ours (Zero-shot) & 47.83\% \\ \hline
\end{tabular}
}
\vspace{-5mm}
\end{table}

% We use the model trained on the FI dataset to test on the artphoto dataset to verify the model's generalization ability as well as robustness to other distributed datasets.
% We can observe that the MESN model shows strong competitiveness in terms of accuracy when compared to other trained models, which suggests that the model has a good generalization ability in the OOD task.

To validate the model's generalization ability and robustness to other distributed datasets, we directly test the model trained on the FI dataset, without training on Artphoto. 
% As observed in Table 3, compared to other models trained on Artphoto, we achieve highly competitive zero-shot performance, indicating that the model has good generalization ability in out-of-distribution tasks.
From Table~\ref{tab:cap3}, we can observe that compared with other models trained on Artphoto, we achieve competitive zero-shot performance, which shows that the model has good generalization ability in out-of-distribution tasks.


%%%%%%%%%%%%
%  E2E     %
%%%%%%%%%%%%


\section{Conclusion}
In this paper, we introduced Wi-Chat, the first LLM-powered Wi-Fi-based human activity recognition system that integrates the reasoning capabilities of large language models with the sensing potential of wireless signals. Our experimental results on a self-collected Wi-Fi CSI dataset demonstrate the promising potential of LLMs in enabling zero-shot Wi-Fi sensing. These findings suggest a new paradigm for human activity recognition that does not rely on extensive labeled data. We hope future research will build upon this direction, further exploring the applications of LLMs in signal processing domains such as IoT, mobile sensing, and radar-based systems.

\section*{Limitations}
While our work represents the first attempt to leverage LLMs for processing Wi-Fi signals, it is a preliminary study focused on a relatively simple task: Wi-Fi-based human activity recognition. This choice allows us to explore the feasibility of LLMs in wireless sensing but also comes with certain limitations.

Our approach primarily evaluates zero-shot performance, which, while promising, may still lag behind traditional supervised learning methods in highly complex or fine-grained recognition tasks. Besides, our study is limited to a controlled environment with a self-collected dataset, and the generalizability of LLMs to diverse real-world scenarios with varying Wi-Fi conditions, environmental interference, and device heterogeneity remains an open question.

Additionally, we have yet to explore the full potential of LLMs in more advanced Wi-Fi sensing applications, such as fine-grained gesture recognition, occupancy detection, and passive health monitoring. Future work should investigate the scalability of LLM-based approaches, their robustness to domain shifts, and their integration with multimodal sensing techniques in broader IoT applications.


% Bibliography entries for the entire Anthology, followed by custom entries
%\bibliography{anthology,custom}
% Custom bibliography entries only
\bibliography{main}
\newpage
\appendix

\section{Experiment prompts}
\label{sec:prompt}
The prompts used in the LLM experiments are shown in the following Table~\ref{tab:prompts}.

\definecolor{titlecolor}{rgb}{0.9, 0.5, 0.1}
\definecolor{anscolor}{rgb}{0.2, 0.5, 0.8}
\definecolor{labelcolor}{HTML}{48a07e}
\begin{table*}[h]
	\centering
	
 % \vspace{-0.2cm}
	
	\begin{center}
		\begin{tikzpicture}[
				chatbox_inner/.style={rectangle, rounded corners, opacity=0, text opacity=1, font=\sffamily\scriptsize, text width=5in, text height=9pt, inner xsep=6pt, inner ysep=6pt},
				chatbox_prompt_inner/.style={chatbox_inner, align=flush left, xshift=0pt, text height=11pt},
				chatbox_user_inner/.style={chatbox_inner, align=flush left, xshift=0pt},
				chatbox_gpt_inner/.style={chatbox_inner, align=flush left, xshift=0pt},
				chatbox/.style={chatbox_inner, draw=black!25, fill=gray!7, opacity=1, text opacity=0},
				chatbox_prompt/.style={chatbox, align=flush left, fill=gray!1.5, draw=black!30, text height=10pt},
				chatbox_user/.style={chatbox, align=flush left},
				chatbox_gpt/.style={chatbox, align=flush left},
				chatbox2/.style={chatbox_gpt, fill=green!25},
				chatbox3/.style={chatbox_gpt, fill=red!20, draw=black!20},
				chatbox4/.style={chatbox_gpt, fill=yellow!30},
				labelbox/.style={rectangle, rounded corners, draw=black!50, font=\sffamily\scriptsize\bfseries, fill=gray!5, inner sep=3pt},
			]
											
			\node[chatbox_user] (q1) {
				\textbf{System prompt}
				\newline
				\newline
				You are a helpful and precise assistant for segmenting and labeling sentences. We would like to request your help on curating a dataset for entity-level hallucination detection.
				\newline \newline
                We will give you a machine generated biography and a list of checked facts about the biography. Each fact consists of a sentence and a label (True/False). Please do the following process. First, breaking down the biography into words. Second, by referring to the provided list of facts, merging some broken down words in the previous step to form meaningful entities. For example, ``strategic thinking'' should be one entity instead of two. Third, according to the labels in the list of facts, labeling each entity as True or False. Specifically, for facts that share a similar sentence structure (\eg, \textit{``He was born on Mach 9, 1941.''} (\texttt{True}) and \textit{``He was born in Ramos Mejia.''} (\texttt{False})), please first assign labels to entities that differ across atomic facts. For example, first labeling ``Mach 9, 1941'' (\texttt{True}) and ``Ramos Mejia'' (\texttt{False}) in the above case. For those entities that are the same across atomic facts (\eg, ``was born'') or are neutral (\eg, ``he,'' ``in,'' and ``on''), please label them as \texttt{True}. For the cases that there is no atomic fact that shares the same sentence structure, please identify the most informative entities in the sentence and label them with the same label as the atomic fact while treating the rest of the entities as \texttt{True}. In the end, output the entities and labels in the following format:
                \begin{itemize}[nosep]
                    \item Entity 1 (Label 1)
                    \item Entity 2 (Label 2)
                    \item ...
                    \item Entity N (Label N)
                \end{itemize}
                % \newline \newline
                Here are two examples:
                \newline\newline
                \textbf{[Example 1]}
                \newline
                [The start of the biography]
                \newline
                \textcolor{titlecolor}{Marianne McAndrew is an American actress and singer, born on November 21, 1942, in Cleveland, Ohio. She began her acting career in the late 1960s, appearing in various television shows and films.}
                \newline
                [The end of the biography]
                \newline \newline
                [The start of the list of checked facts]
                \newline
                \textcolor{anscolor}{[Marianne McAndrew is an American. (False); Marianne McAndrew is an actress. (True); Marianne McAndrew is a singer. (False); Marianne McAndrew was born on November 21, 1942. (False); Marianne McAndrew was born in Cleveland, Ohio. (False); She began her acting career in the late 1960s. (True); She has appeared in various television shows. (True); She has appeared in various films. (True)]}
                \newline
                [The end of the list of checked facts]
                \newline \newline
                [The start of the ideal output]
                \newline
                \textcolor{labelcolor}{[Marianne McAndrew (True); is (True); an (True); American (False); actress (True); and (True); singer (False); , (True); born (True); on (True); November 21, 1942 (False); , (True); in (True); Cleveland, Ohio (False); . (True); She (True); began (True); her (True); acting career (True); in (True); the late 1960s (True); , (True); appearing (True); in (True); various (True); television shows (True); and (True); films (True); . (True)]}
                \newline
                [The end of the ideal output]
				\newline \newline
                \textbf{[Example 2]}
                \newline
                [The start of the biography]
                \newline
                \textcolor{titlecolor}{Doug Sheehan is an American actor who was born on April 27, 1949, in Santa Monica, California. He is best known for his roles in soap operas, including his portrayal of Joe Kelly on ``General Hospital'' and Ben Gibson on ``Knots Landing.''}
                \newline
                [The end of the biography]
                \newline \newline
                [The start of the list of checked facts]
                \newline
                \textcolor{anscolor}{[Doug Sheehan is an American. (True); Doug Sheehan is an actor. (True); Doug Sheehan was born on April 27, 1949. (True); Doug Sheehan was born in Santa Monica, California. (False); He is best known for his roles in soap operas. (True); He portrayed Joe Kelly. (True); Joe Kelly was in General Hospital. (True); General Hospital is a soap opera. (True); He portrayed Ben Gibson. (True); Ben Gibson was in Knots Landing. (True); Knots Landing is a soap opera. (True)]}
                \newline
                [The end of the list of checked facts]
                \newline \newline
                [The start of the ideal output]
                \newline
                \textcolor{labelcolor}{[Doug Sheehan (True); is (True); an (True); American (True); actor (True); who (True); was born (True); on (True); April 27, 1949 (True); in (True); Santa Monica, California (False); . (True); He (True); is (True); best known (True); for (True); his roles in soap operas (True); , (True); including (True); in (True); his portrayal (True); of (True); Joe Kelly (True); on (True); ``General Hospital'' (True); and (True); Ben Gibson (True); on (True); ``Knots Landing.'' (True)]}
                \newline
                [The end of the ideal output]
				\newline \newline
				\textbf{User prompt}
				\newline
				\newline
				[The start of the biography]
				\newline
				\textcolor{magenta}{\texttt{\{BIOGRAPHY\}}}
				\newline
				[The ebd of the biography]
				\newline \newline
				[The start of the list of checked facts]
				\newline
				\textcolor{magenta}{\texttt{\{LIST OF CHECKED FACTS\}}}
				\newline
				[The end of the list of checked facts]
			};
			\node[chatbox_user_inner] (q1_text) at (q1) {
				\textbf{System prompt}
				\newline
				\newline
				You are a helpful and precise assistant for segmenting and labeling sentences. We would like to request your help on curating a dataset for entity-level hallucination detection.
				\newline \newline
                We will give you a machine generated biography and a list of checked facts about the biography. Each fact consists of a sentence and a label (True/False). Please do the following process. First, breaking down the biography into words. Second, by referring to the provided list of facts, merging some broken down words in the previous step to form meaningful entities. For example, ``strategic thinking'' should be one entity instead of two. Third, according to the labels in the list of facts, labeling each entity as True or False. Specifically, for facts that share a similar sentence structure (\eg, \textit{``He was born on Mach 9, 1941.''} (\texttt{True}) and \textit{``He was born in Ramos Mejia.''} (\texttt{False})), please first assign labels to entities that differ across atomic facts. For example, first labeling ``Mach 9, 1941'' (\texttt{True}) and ``Ramos Mejia'' (\texttt{False}) in the above case. For those entities that are the same across atomic facts (\eg, ``was born'') or are neutral (\eg, ``he,'' ``in,'' and ``on''), please label them as \texttt{True}. For the cases that there is no atomic fact that shares the same sentence structure, please identify the most informative entities in the sentence and label them with the same label as the atomic fact while treating the rest of the entities as \texttt{True}. In the end, output the entities and labels in the following format:
                \begin{itemize}[nosep]
                    \item Entity 1 (Label 1)
                    \item Entity 2 (Label 2)
                    \item ...
                    \item Entity N (Label N)
                \end{itemize}
                % \newline \newline
                Here are two examples:
                \newline\newline
                \textbf{[Example 1]}
                \newline
                [The start of the biography]
                \newline
                \textcolor{titlecolor}{Marianne McAndrew is an American actress and singer, born on November 21, 1942, in Cleveland, Ohio. She began her acting career in the late 1960s, appearing in various television shows and films.}
                \newline
                [The end of the biography]
                \newline \newline
                [The start of the list of checked facts]
                \newline
                \textcolor{anscolor}{[Marianne McAndrew is an American. (False); Marianne McAndrew is an actress. (True); Marianne McAndrew is a singer. (False); Marianne McAndrew was born on November 21, 1942. (False); Marianne McAndrew was born in Cleveland, Ohio. (False); She began her acting career in the late 1960s. (True); She has appeared in various television shows. (True); She has appeared in various films. (True)]}
                \newline
                [The end of the list of checked facts]
                \newline \newline
                [The start of the ideal output]
                \newline
                \textcolor{labelcolor}{[Marianne McAndrew (True); is (True); an (True); American (False); actress (True); and (True); singer (False); , (True); born (True); on (True); November 21, 1942 (False); , (True); in (True); Cleveland, Ohio (False); . (True); She (True); began (True); her (True); acting career (True); in (True); the late 1960s (True); , (True); appearing (True); in (True); various (True); television shows (True); and (True); films (True); . (True)]}
                \newline
                [The end of the ideal output]
				\newline \newline
                \textbf{[Example 2]}
                \newline
                [The start of the biography]
                \newline
                \textcolor{titlecolor}{Doug Sheehan is an American actor who was born on April 27, 1949, in Santa Monica, California. He is best known for his roles in soap operas, including his portrayal of Joe Kelly on ``General Hospital'' and Ben Gibson on ``Knots Landing.''}
                \newline
                [The end of the biography]
                \newline \newline
                [The start of the list of checked facts]
                \newline
                \textcolor{anscolor}{[Doug Sheehan is an American. (True); Doug Sheehan is an actor. (True); Doug Sheehan was born on April 27, 1949. (True); Doug Sheehan was born in Santa Monica, California. (False); He is best known for his roles in soap operas. (True); He portrayed Joe Kelly. (True); Joe Kelly was in General Hospital. (True); General Hospital is a soap opera. (True); He portrayed Ben Gibson. (True); Ben Gibson was in Knots Landing. (True); Knots Landing is a soap opera. (True)]}
                \newline
                [The end of the list of checked facts]
                \newline \newline
                [The start of the ideal output]
                \newline
                \textcolor{labelcolor}{[Doug Sheehan (True); is (True); an (True); American (True); actor (True); who (True); was born (True); on (True); April 27, 1949 (True); in (True); Santa Monica, California (False); . (True); He (True); is (True); best known (True); for (True); his roles in soap operas (True); , (True); including (True); in (True); his portrayal (True); of (True); Joe Kelly (True); on (True); ``General Hospital'' (True); and (True); Ben Gibson (True); on (True); ``Knots Landing.'' (True)]}
                \newline
                [The end of the ideal output]
				\newline \newline
				\textbf{User prompt}
				\newline
				\newline
				[The start of the biography]
				\newline
				\textcolor{magenta}{\texttt{\{BIOGRAPHY\}}}
				\newline
				[The ebd of the biography]
				\newline \newline
				[The start of the list of checked facts]
				\newline
				\textcolor{magenta}{\texttt{\{LIST OF CHECKED FACTS\}}}
				\newline
				[The end of the list of checked facts]
			};
		\end{tikzpicture}
        \caption{GPT-4o prompt for labeling hallucinated entities.}\label{tb:gpt-4-prompt}
	\end{center}
\vspace{-0cm}
\end{table*}
% \section{Full Experiment Results}
% \begin{table*}[th]
    \centering
    \small
    \caption{Classification Results}
    \begin{tabular}{lcccc}
        \toprule
        \textbf{Method} & \textbf{Accuracy} & \textbf{Precision} & \textbf{Recall} & \textbf{F1-score} \\
        \midrule
        \multicolumn{5}{c}{\textbf{Zero Shot}} \\
                Zero-shot E-eyes & 0.26 & 0.26 & 0.27 & 0.26 \\
        Zero-shot CARM & 0.24 & 0.24 & 0.24 & 0.24 \\
                Zero-shot SVM & 0.27 & 0.28 & 0.28 & 0.27 \\
        Zero-shot CNN & 0.23 & 0.24 & 0.23 & 0.23 \\
        Zero-shot RNN & 0.26 & 0.26 & 0.26 & 0.26 \\
DeepSeek-0shot & 0.54 & 0.61 & 0.54 & 0.52 \\
DeepSeek-0shot-COT & 0.33 & 0.24 & 0.33 & 0.23 \\
DeepSeek-0shot-Knowledge & 0.45 & 0.46 & 0.45 & 0.44 \\
Gemma2-0shot & 0.35 & 0.22 & 0.38 & 0.27 \\
Gemma2-0shot-COT & 0.36 & 0.22 & 0.36 & 0.27 \\
Gemma2-0shot-Knowledge & 0.32 & 0.18 & 0.34 & 0.20 \\
GPT-4o-mini-0shot & 0.48 & 0.53 & 0.48 & 0.41 \\
GPT-4o-mini-0shot-COT & 0.33 & 0.50 & 0.33 & 0.38 \\
GPT-4o-mini-0shot-Knowledge & 0.49 & 0.31 & 0.49 & 0.36 \\
GPT-4o-0shot & 0.62 & 0.62 & 0.47 & 0.42 \\
GPT-4o-0shot-COT & 0.29 & 0.45 & 0.29 & 0.21 \\
GPT-4o-0shot-Knowledge & 0.44 & 0.52 & 0.44 & 0.39 \\
LLaMA-0shot & 0.32 & 0.25 & 0.32 & 0.24 \\
LLaMA-0shot-COT & 0.12 & 0.25 & 0.12 & 0.09 \\
LLaMA-0shot-Knowledge & 0.32 & 0.25 & 0.32 & 0.28 \\
Mistral-0shot & 0.19 & 0.23 & 0.19 & 0.10 \\
Mistral-0shot-Knowledge & 0.21 & 0.40 & 0.21 & 0.11 \\
        \midrule
        \multicolumn{5}{c}{\textbf{4 Shot}} \\
GPT-4o-mini-4shot & 0.58 & 0.59 & 0.58 & 0.53 \\
GPT-4o-mini-4shot-COT & 0.57 & 0.53 & 0.57 & 0.50 \\
GPT-4o-mini-4shot-Knowledge & 0.56 & 0.51 & 0.56 & 0.47 \\
GPT-4o-4shot & 0.77 & 0.84 & 0.77 & 0.73 \\
GPT-4o-4shot-COT & 0.63 & 0.76 & 0.63 & 0.53 \\
GPT-4o-4shot-Knowledge & 0.72 & 0.82 & 0.71 & 0.66 \\
LLaMA-4shot & 0.29 & 0.24 & 0.29 & 0.21 \\
LLaMA-4shot-COT & 0.20 & 0.30 & 0.20 & 0.13 \\
LLaMA-4shot-Knowledge & 0.15 & 0.23 & 0.13 & 0.13 \\
Mistral-4shot & 0.02 & 0.02 & 0.02 & 0.02 \\
Mistral-4shot-Knowledge & 0.21 & 0.27 & 0.21 & 0.20 \\
        \midrule
        
        \multicolumn{5}{c}{\textbf{Suprevised}} \\
        SVM & 0.94 & 0.92 & 0.91 & 0.91 \\
        CNN & 0.98 & 0.98 & 0.97 & 0.97 \\
        RNN & 0.99 & 0.99 & 0.99 & 0.99 \\
        % \midrule
        % \multicolumn{5}{c}{\textbf{Conventional Wi-Fi-based Human Activity Recognition Systems}} \\
        E-eyes & 1.00 & 1.00 & 1.00 & 1.00 \\
        CARM & 0.98 & 0.98 & 0.98 & 0.98 \\
\midrule
 \multicolumn{5}{c}{\textbf{Vision Models}} \\
           Zero-shot SVM & 0.26 & 0.25 & 0.25 & 0.25 \\
        Zero-shot CNN & 0.26 & 0.25 & 0.26 & 0.26 \\
        Zero-shot RNN & 0.28 & 0.28 & 0.29 & 0.28 \\
        SVM & 0.99 & 0.99 & 0.99 & 0.99 \\
        CNN & 0.98 & 0.99 & 0.98 & 0.98 \\
        RNN & 0.98 & 0.99 & 0.98 & 0.98 \\
GPT-4o-mini-Vision & 0.84 & 0.85 & 0.84 & 0.84 \\
GPT-4o-mini-Vision-COT & 0.90 & 0.91 & 0.90 & 0.90 \\
GPT-4o-Vision & 0.74 & 0.82 & 0.74 & 0.73 \\
GPT-4o-Vision-COT & 0.70 & 0.83 & 0.70 & 0.68 \\
LLaMA-Vision & 0.20 & 0.23 & 0.20 & 0.09 \\
LLaMA-Vision-Knowledge & 0.22 & 0.05 & 0.22 & 0.08 \\

        \bottomrule
    \end{tabular}
    \label{full}
\end{table*}




\end{document}


\end{document}
