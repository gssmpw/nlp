\section{Related work}
\subsection{Video Quality Assessment}
In the field of video quality assessment, there are currently two main approaches: traditional methods based on handcrafted feature extraction and modern methods based on deep learning. Traditional methods____ rely on manually designed features to assess video quality; however, due to the limitations of handcrafted features, these methods typically only capture shallow quality representations and fail to account for the complex factors that influence video quality.

With the rapid advancement of deep learning, models____ based on this technology have demonstrated superior feature extraction capabilities in video quality assessment tasks. Notably, networks such as 3D-CNN____ and Video Swin Transformers____ are able to capture deep spatiotemporal features from video data. However, while these models have improved video quality assessment capabilities, they still struggle to fully capture the complex characteristics that affect video quality, making it difficult to meet the demands of comprehensive video quality assessment tasks.

\subsection{Video Foundation Models}

Video foundation models are commonly used for video understanding tasks such as action recognition, typically leveraging convolutional____ and attention mechanisms____ to extract video features. UniFormer____ combines convolution and attention to reduce spatiotemporal redundancy and mitigate global dependency issues. Building on this, UniFormer2____ incorporates pre-trained ViTs to further capture rich image priors. InternVideo____ enhances the understanding of complex video content by integrating UniFormer2 with video-language contrastive learning and video-masked modeling. InternVideo2____ adopts a progressive training strategy, leveraging its massive parameter size and extensive data through masked distillation training, achieving superior performance.