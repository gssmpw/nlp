
In this section, we first present the overall results of the data leakage detection analysis. Following that, we provide detailed results and answers to each research question (RQ).



\vspace{0.2cm}
\noindent
\textbf{Overall Results.}
Before diving into the specific RQs, we first present an overview of the experimental results. The selected LLM's pre-training datasets consist of 12M samples for Python, 20M samples for Java, and 14M samples for C/C++. In comparison, the diverse SE benchmarks we studied collectively comprise 46k samples for Python, 42k samples for Java, and 21k samples for C/C++.
To investigate potential data leakage, each SE benchmark sample was compared against all pre-training data samples for its corresponding programming language. This process resulted in an astounding total of over 1.7 trillion comparisons. The sheer scale of this computational effort highlights the complexity and resource-intensive nature of studying data leakage regarding LLMs. 


From an overall perspective, as depicted in Figure~\ref{fig:data_review}, only 2\% of the benchmark samples from all the SE benchmarks studied were flagged by the automated tool MinHash+LSH as potentially forming at least one duplicate pair with the pre-training data of StarCoder. Moreover, of the pairs flagged by MinHash+LSH, 28\% were confirmed as duplicates after manual labeling, while the remaining 72\% were determined not to be duplicates.

Next, we will discuss the detailed results and answers to each RQ.


\begin{table}
  \centering
\caption{\textcolor{black}{Effect of duplicate code removal on structural metrics. (+ve) indicates positive impact; (-ve) indicates negative impact; (-) indicates metric remains unaffected, \textbf{bold} indicates statistical significance; \textit{italic} indicates improvement.}}
\label{Table:Metrics Suites and Metrics Tools Summary}
%\begin{sideways}
\begin{adjustbox}{width=1.0\textwidth,center}
%\begin{adjustbox}{width=\textheight,totalheight=\textwidth,keepaspectratio}
\begin{tabular}{lllll}\hline
\toprule
\bfseries Quality Attribute & \bfseries Metric & \bfseries Impact & \bfseries \textit{p}-value & \bfseries Cliff's delta ($\delta$) \\
\midrule
%\multicolumn{2}{l}{\textbf{\textit{Internal Quality Attribute }}}\\
%\midrule
Cohesion &  LCOM5  & +ve & \textit{\textbf{7.72e-41}} & 0.54 (Large)
%(Small)   
\\ 
Coupling &  \cellcolor{gray!30}CBO  & \cellcolor{gray!30}+ve & \cellcolor{gray!30}\textit{\textbf{9.49e-76}} & \cellcolor{gray!30}0.6 (Large)
%(Small) 
\\
         &  RFC & +ve & \textit{\textbf{1.25e-68}}  & 0.55 (Large)

\\
         &  \cellcolor{gray!30}NII & \cellcolor{gray!30}-ve &  \cellcolor{gray!30}\textbf{0} & \cellcolor{gray!30}0.47 (Large)

\\
         &  NOI & +ve & \textit{\textbf{0}}  & 0.26 (Small)

\\
Complexity &  \cellcolor{gray!30}CC & \cellcolor{gray!30}- & \cellcolor{gray!30}\textbf{0} & \cellcolor{gray!30}0.14 (Small)

\\
           &  WMC & +ve & \textit{\textbf{6.51e-70}} & 0.5 (Large)

\\
         &  \cellcolor{gray!30}NL &  \cellcolor{gray!30}- &  \cellcolor{gray!30}\textbf{3.92e-05}&  \cellcolor{gray!30}0.03 (Negligible)

\\
         &  NLE &  - & \textbf{0.004}  & 0.02 (Negligible)

\\
         &  \cellcolor{gray!30}HCPL & \cellcolor{gray!30}+ve &  \cellcolor{gray!30}\textit{\textbf{0}} &  \cellcolor{gray!30}0.14 (Negligible)

\\
         &  HDIF & +ve & \textit{\textbf{0}}  &  0.08 (Negligible)

\\
         &  \cellcolor{gray!30}HEFF & \cellcolor{gray!30}+ve &  \cellcolor{gray!30}\textit{\textbf{2.45e-271}} &  \cellcolor{gray!30}0.13 (Negligible)

\\
         &  HNDB & +ve &  \textit{\textbf{1.07e-266}} & 0.13  (Negligible)

\\
         &  \cellcolor{gray!30}HPL & \cellcolor{gray!30}+ve & \cellcolor{gray!30}\textit{\textbf{0}} & \cellcolor{gray!30}0.13  (Negligible)

\\
         &  HPV & +ve & \textit{\textbf{0}}  & 0.14  (Negligible)

\\
         &  \cellcolor{gray!30}HTRP & \cellcolor{gray!30}+ve &  \cellcolor{gray!30}\textit{\textbf{2.48e-271}} & \cellcolor{gray!30}0.13  (Negligible)

\\
         &  HVOL & +ve & \textit{\textbf{0}}  &  0.13  (Negligible)

\\
         &  \cellcolor{gray!30}MIMS & \cellcolor{gray!30}+ve & \cellcolor{gray!30}\textit{\textbf{7.23e-227}}  &  \cellcolor{gray!30}0.13  (Negligible)

\\
         &   MI & +ve &  \textit{\textbf{7.22e-227}} &  0.13  (Negligible)

\\
         &   \cellcolor{gray!30}MISEI & \cellcolor{gray!30}+ve & \cellcolor{gray!30}\textit{\textbf{0}} & \cellcolor{gray!30}0.16  (Small)

\\
         &   MISM &  +ve&  \textit{\textbf{0}}  & 0.16  (Small)

\\
Inheritance &   \cellcolor{gray!30}DIT & \cellcolor{gray!30}-ve & \cellcolor{gray!30}\textbf{3.81e-199} & \cellcolor{gray!30}0.6 (Large) 
 
\\
            &  NOC & +ve & \textbf{\textit{3.61e-130}} & 0.83 (Large)  

\\
            &  \cellcolor{gray!30}NOA & \cellcolor{gray!30}-ve & \cellcolor{gray!30}\textbf{2.37e-196}  & \cellcolor{gray!30}0.63 (Large)
 
\\ 
Design Size &  LOC & +ve & \textbf{\textit{0}} &   0.14 (Small)

\\
         &  \cellcolor{gray!30}TLOC & \cellcolor{gray!30}+ve &   \cellcolor{gray!30}\textit{\textbf{0}} &  \cellcolor{gray!30}0.16 (Small)

\\
            &  LLOC &  +ve &  \textbf{\textit{0}} &   0.13 (Negligible)

\\
         &  \cellcolor{gray!30}TLLOC & \cellcolor{gray!30}+ve & \cellcolor{gray!30}\textit{\textbf{0}}  &   \cellcolor{gray!30}0.15 (Small)

\\
            &   CLOC & - &  \textbf{1.43e-05} &   0.02 (Negligible)

\\
            &  \cellcolor{gray!30}NPM & \cellcolor{gray!30}- & \cellcolor{gray!30}\textbf{4.42e-193} & \cellcolor{gray!30}0.5 (Large)

\\
         &  NOS &  +ve&  \textit{\textbf{0}} &  0.07 (Negligible)

\\
         &  \cellcolor{gray!30}TNOS & \cellcolor{gray!30}+ve & \cellcolor{gray!30}\textit{\textbf{0}}  &  \cellcolor{gray!30}0.08 (Negligible)

\\
\bottomrule
\end{tabular}
\end{adjustbox}
%\end{sideways}
\end{table}
\begin{comment}
    


%\begin{sidewaystable}
\begin{table*}

\caption{The impact of duplicate code removal on quality metrics.}
\label{composites}


\fontsize{10}{14}\selectfont
	\tabcolsep=0.1cm
\resizebox{\textwidth}{!}{
\begin{tabular}{llcccccccccccccccccccccc} \toprule
\multicolumn{1}{c}{}                             & \multicolumn{1}{c}{}                          & \multicolumn{3}{c}{Cohesion}                                                                                                            & \multicolumn{5}{c}{Coupling}                                                                                                          & \multicolumn{4}{c}{Complexity}                                                                                                                                                         & \multicolumn{3}{c}{Inheritance}                                                         & \multicolumn{7}{c}{Design Size}                                                                                                                                                                                                                                                                         \\
\multicolumn{1}{c}{\multirow{-2}{*}{}} & \multicolumn{1}{c}{\multirow{-2}{*}{Measure}} & LCOM5                                        &                                          &                                          & CBO                                         & RFC                                         &     &    &                 & WMC                                         & CC                                         &                                  &                                          & DIT                                         & NOC                 &   NOA            & SLOC                                         & LLOC                                        & CLOC                &      NPM                                  &                                        &                                          &                                         \\ \hline
                                                 & Refactoring Impact                            & 0                                           & \cellcolor[HTML]{0350F8}1                   & \cellcolor[HTML]{0350F8}1                   & 0                                           & \cellcolor[HTML]{0350F8}-12                 & 0        & 0        & 0                   & \cellcolor[HTML]{0C56F7}-5                  & \cellcolor[HTML]{1B61F8}-3                  & \cellcolor[HTML]{1B61F8}-3                  & \cellcolor[HTML]{A3BBED}-1                  & 0                                           & 0                   & 0                   & \cellcolor[HTML]{7DA2F1}-2                  & 0                                           & 0                   & 7                                           & \cellcolor[HTML]{F8ADAD}2                   & \cellcolor[HTML]{0350F8}-20                 & 0                                           \\
                                                 & Behavior                                      & -                                           & \cellcolor[HTML]{0350F8}haut                & \cellcolor[HTML]{0350F8}haut                & -                                           & \cellcolor[HTML]{0350F8}bas                 & -        & -        & -                   & \cellcolor[HTML]{0C56F7}bas                 & \cellcolor[HTML]{1B61F8}bas                 & \cellcolor[HTML]{1B61F8}bas                 & \cellcolor[HTML]{A3BBED}bas                 & -                                           & -                   & -                   & \cellcolor[HTML]{7DA2F1}bas                 & -                                           & -                   & haut                                        & \cellcolor[HTML]{F8ADAD}haut                & \cellcolor[HTML]{0350F8}bas                 & -                                           \\
\multirow{-3}{*}{}                         & P-value ($\delta$)                            & 1 (N)                                       & \cellcolor[HTML]{0350F8}\textless{}0.05 (S) & \cellcolor[HTML]{0350F8}\textless{}0.05 (S) & 1 (N)                                       & \cellcolor[HTML]{0350F8}\textless{}0.05 (M) & 0.07 (N) & 0.07 (N) & \textless{}0.05 (N) & \cellcolor[HTML]{0C56F7}\textless{}0.05(S)  & \cellcolor[HTML]{1B61F8}\textless{}0.05 (S) & \cellcolor[HTML]{1B61F8}\textless{}0.05(S)  & \cellcolor[HTML]{A3BBED}\textless{}0.05(N)  & 0.08 (N)                                    & 0.23 (S)            & 0.19 (N)            & \cellcolor[HTML]{7DA2F1}0.06 (N)            & 0.16 (N)                                    & \textless{}0.05 (N) & 0.09 (S)                                    & \cellcolor[HTML]{F8ADAD}\textless{}0.05 (N) & \cellcolor[HTML]{0350F8}\textless{}0.05 (M) & 0.12 (N)                                    \\
 \bottomrule
\end{tabular}
}
\end{table*}
%\end{sidewaystable}

\end{comment}

\subsection{RQ-2 How do different configurations affect the effectiveness of LLMs?}
\label{sec:rq2}


\noindent 
\textbf{Impact of different example numbers.}
As previous studies~\cite{brown2020language,A3CodGen} have shown, the number of examples provided has a significant impact on LLMs' performance. 
To explore this, we adjust the number of examples while keeping other parameters and hyperparameters constant to ensure a fair comparison.
We do not conduct experiments in a zero-shot setting, as LLMs may generate unnormalized outputs without a prompt template, which would hinder automated extraction. 
From Fig.~\ref{fig:ablation}, we observe that as the number of examples increases, both the average token length and time cost rise sharply, while the improvement in Pass@k remains modest.
Based on these findings, we perform our ablation studies (Table~\ref{tab:rq1} and \ref{tab:ablation}) using a one-shot setting in \mytitle.


\noindent 
\textbf{Impact of different selection strategies.}
% Our case study reveals that the RAG method improves the performance of LLMs.
RAG retrieves relevant codes from a retrieval database and supplements this information for code generation~\cite{parvez2021retrieval}. 
To ensure a fair comparison, we set the number of examples to one and evaluated the results of RAG versus random selection on the same LLM (i.e., DeepSeek-V3). From Table~\ref{tab:ablation}, Pass@1 and Compile@1 are higher when RAG is enabled, indicating that it improves the effectiveness of code generation.


\begin{figure}[htbp]
    \centering
    \includegraphics[width=\linewidth]{figs/ablation.pdf}
    \caption{Performance of Qwen2.5-Coder-7B. The x-axis represents the number of shots.}
    \label{fig:ablation}
\end{figure}
\vspace{-0.2cm}

\noindent 
\textbf{Impact of Context Information.}
Since that relevant context typically enhances performance in other programming languages, we conduct an ablation study to examine the influence of context on the quality of LLM-generated contracts. Table~\ref{tab:ablation} shows that providing context information improves both Pass@1 and Compile@1. 
However, there is no clear correlation between gas fees, vulnerability rate, and the presence of context information.


% \vspace{-0.1cm}
\begin{table}[htbp]
    \centering
    \caption{Ablation study on the effect of RAG and Context on DeepSeek-V3's (one-shot) performance.}
    \resizebox{\linewidth}{!}
    {
        \begin{tabular}{cc|cccc}
        \toprule
        RAG & Context & Pass@1 & Compile@1 & Fee & Vul \\
        \midrule
        \ding{51} & \ding{51} & \textbf{21.72\%}& \textbf{53.35\%}&  \textbf{-7525}& 26.61\% \\ 
        \ding{55} & \ding{51} & 20.24\% & 51.08\% & 3828& \textbf{23.68\%}\\ 
        \ding{51} & \ding{55} & 21.28\% & 52.54\% & -708& 26.13\%\\
        \ding{55} & \ding{55} & 20.17\% & 50.32\% & 768& 26.83\%\\   
        \bottomrule
        \end{tabular}
    }
    \label{tab:ablation}
\end{table}
% \vspace{-0.4cm}

\subsection{RQ-3 Gas Efficiency and Scalability Analysis}
\label{sec:rq3}


\noindent
\textbf{Objective.}
In \Chain, we take \textit{Gas Fee} and \textit{Scalability} into consideration.
In blockchain query databases, the gas fee and scaleability are essential since the former will largely impact the practicality of the designed methods (i.e., a higher gas fee means more money spent when running queries) and the latter will impact compatibility (i.e., poor compatibility will lead to massive code modification when adjusting to another blockchain system).

\noindent
\textbf{Experimental Design.}
First, we investigate the impact of different data structures on gas fee and design four variants of \Chain. 
vChain+$_{F}$ represents vChain+ replicated and enhanced to support multimodal queries.
vChain+$_{O}$ is vChain+$_{F}$ without the off-chain query module.
MulChain$_{BT}$ is MulChain with its underlying data structure replaced with B\(+\)Tree for time range queries.
MulChain$_{BH}$ is MulChain with its underlying data structure replaced with our gas-efficient BHashTree for time range queries.
MulChain$_{T}$ is MulChain with its underlying data structure replaced with our verifiable trie for fuzzy queries.
This approach allows us to examine the individual effects of each component.


\noindent
\textbf{Results.} We discuss the results from the aspects of gas consumption and scalability, respectively.




\noindent
\textbf{\underline{Gas Consumption Analysis.}}
The average gas fees for BHashTree are much lower than those of vChain+$_{F}$, slightly lower than B\(+\)Tree-based methods, as illustrated in Fig.~\ref{fig:Gas Consumption}(a).
Fig.~\ref{fig:Gas Consumption}(b) presents the average gas fees of the trie in comparison to the accumulator from vChain+. 
Notably, the gas consumption of MulChain$_{T}$ exceeds that of vChain+$_{F}$ due to our strategic trade-off of space for time. 
We deem this trade-off acceptable, as the reduction in query latency is particularly valuable in the context of fuzzy queries on blockchains.




\begin{figure}[htbp]
    \centering
    \includegraphics[width=.7\linewidth]{figures/Gas_BHT_BT.pdf}
    % \captionsetup{skip=0pt}
    \caption{Gas Consumption}
    \label{fig:Gas Consumption}
\end{figure}


\noindent
\textbf{\underline{Scalability Analysis.}}
\Chain supports all six SQL primitives (i.e., insert, delete, update, simple, time range, fuzzy queries) on Ethereum and FISCO BCOS. 
In contrast, the CRUD Service of FISCO BCOS does not support time range and fuzzy queries. 
We test \Chain using time range queries on BTC and ETH datasets. 
From Fig.~\ref{fig:FISCO BCOS}(a), we can see that \Chain undergoes a decline of up to 3.78\% when the number of blocks grows.
In Fig.~\ref{fig:FISCO BCOS}(b), we observe that MulChain$_{BT}$ is faster on the BTC dataset than on ETH for block counts below 128 and above 1024. 
This performance difference is due to the varying timestamp densities of the two datasets and the initialization cost of the B\(+\)Tree.


\begin{figure}[htbp]
    \centering
    \includegraphics[width=.7\linewidth]{figures/FISCO_all.pdf}
    % \captionsetup{skip=0pt}
    \caption{Query Performance on FISCO BCOS}
    \label{fig:FISCO BCOS}
\end{figure}


\intuition{
{\bf Answer to RQ-3}: 
(1) The five data structures (i.e., accumulator of vChain+$_{F}$, vChain+$_{O}$, MulChain$_{BT}$, MulChain$_{BH}$ and MulChain$_T$) contribute substantially to \Chain, and combining them achieves the best performance of blockchain query on different scenarios.
(2) The gas fee of MulChain$_{T}$ exceeds that of vChain+$_{F}$ due to our strategic trade-off of space for time.
(3) \Chain supports blockchains based on Ethereum virtual machine and Hyperledger Fabric.
}


\begin{table*}[t!]
    \centering
    \tabcolsep=1.5pt
    \renewcommand{\arraystretch}{0.92} 
    \caption{Experimental results of RQ6}
    \begin{tabular}{ccccc}
    \hline
    \multicolumn{1}{c|}{\textbf{Attempt Times}}   & \multicolumn{1}{c|}{\textbf{First Attempt}} & \multicolumn{1}{c|}{\textbf{Second Attempt}} & \multicolumn{1}{c|}{\textbf{Third Attempt}} & \textbf{More-time Attempt} \\ \hline
    \multicolumn{1}{c|}{\textbf{IllusionCAPTCHA}} & 86.95\%                                     & 8.69\%                                       & 0.00\%                                      & 4.34\%                     \\ \hline
    \multicolumn{1}{l}{}                          & \multicolumn{1}{l}{}                        & \multicolumn{1}{l}{}                         & \multicolumn{1}{l}{}                        & \multicolumn{1}{l}{}      
    \end{tabular}
\label{tex:RQ4}
\end{table*}