\documentclass[acmsmall,screen]{acmart}
\usepackage[shortlabels]{enumitem}
\usepackage{fancybox, graphicx}
\usepackage{color}
\usepackage{listings}
\usepackage{tcolorbox}
\usepackage{balance}
\usepackage{tabularx}
\usepackage{multirow}
\usepackage{xspace}
\usepackage{tikz}
\usepackage{textcomp}
\usepackage{booktabs}
\usetikzlibrary{patterns}
\usepackage{url}


\definecolor{darkblue}{rgb}{0.0, 0.0, 0.55}  

\usepackage[normalem]{ulem}
\usepackage[utf8]{inputenc}
\usepackage{amsmath}  
\usepackage{graphicx}
\usepackage{comment}
\usepackage{multirow}
\usepackage{enumitem}
\usepackage{array}
\usepackage{pifont}
\usepackage{xcolor}
\usepackage{balance}
\usepackage{mathtools}
\usepackage{xspace}
\usepackage{url}
\usepackage{eucal}
\usepackage{amsfonts}
\usepackage{color}
\usepackage{booktabs}
\usepackage{bbding}
\usepackage{float}

\usepackage{graphicx}
\usepackage{hyperref}
\usepackage{url}
\usepackage{multirow}
\usepackage{comment}
\usepackage{colortbl}
\usepackage{arydshln}
\usepackage{graphicx}
\usepackage{subcaption}
\usepackage[linewidth=0.1pt]{mdframed}
\usepackage[linesnumbered,ruled,lined]{algorithm2e}


\usepackage{wrapfig,lipsum,booktabs}
\usepackage{subfloat}
\usepackage[export]{adjustbox}% 

\usepackage{graphicx}
\usepackage{adjustbox}
\usepackage{forest}
\usepackage{tikz}
\usepackage{pifont}


\usepackage{hyperref}
\usepackage{hyperxmp}

    
\AtBeginDocument{%
  \providecommand\BibTeX{{%
    \normalfont B\kern-0.5em{\scshape i\kern-0.25em b}\kern-0.8em\TeX}}}

\newcommand{\basicalerter}[2]{\fbox{\bfseries\sffamily\scriptsize\color{blue} #1}{\sf\small$\blacktriangleright$\textit{\color{blue} #2}$\blacktriangleleft$}}

\newcommand{\mrevised}[1]{\textcolor{red}{#1}}
\newcommand{\revised}[1]{\textcolor{black}{#1}}
\newcommand{\checked}[1]{\textcolor{blue}{#1}}

\newcommand{\todoorange}[1]{\textcolor{orange}{\textbf{[[#1]]}}}
\newcommand{\added}[1]{\textcolor{black}{#1}}


\pdfpagewidth=\paperwidth
\pdfpageheight=\paperheight

\makeatletter
\newcommand{\showfontsize}{\f@size{} pt}
\newcommand\usemm[1]{%
  \strip@pt\dimexpr0.3514598\dimexpr #1\relax\relax mm%
}
\newcommand\usein[1]{%
  \strip@pt\dimexpr0.013837\dimexpr #1\relax\relax in%
}
\makeatother

\usepackage{graphicx}
\usepackage{hyperref}
\usepackage{url}
\usepackage{multirow}
\usepackage{comment}
\usepackage{colortbl}
\usepackage{graphicx}
\usepackage{tcolorbox}

\usepackage{algpseudocode}  
\usepackage{amsmath}  
\usepackage{caption} 
\usepackage{algpseudocode}
\usepackage{float}
\usepackage{soul}



\usepackage{xcolor}
\usepackage{enumitem}

\usepackage{amsmath}
\newcommand{\basicalert}[2]{\fbox{\bfseries\sffamily\scriptsize\color{black} #1}{\sf\small$\blacktriangleright$\textit{\color{red} #2}$\blacktriangleleft$}}


\newcommand{\xin}[1]{\basicalert{From Xin}{#1}}




\definecolor{dark-red}{RGB}{255,0,0}
\definecolor{dark-green}{RGB}{0,200,0}
\definecolor{lightgreen}{rgb}{0.56, 0.93, 0.56} % 定义浅绿色
\definecolor{lightred}{rgb}{1.0, 0.7, 0.7}

\lstdefinelanguage{java-pretty}
{
  language=java,
  numbers=left,
  frame=shadowbox,
  rulesepcolor= \color{red!20!green!20!blue!20},
  basicstyle=\footnotesize\ttfamily,
  numberstyle=\scriptsize,
  breaklines=true,
  columns=fullflexible,
  xleftmargin=16pt,
  showstringspaces=false,
  keywordstyle=\color{blue}\bfseries,
  stringstyle=\color{javared},
  commentstyle=\color{javagreen},
  morecomment=[s][\color{javadocblue}]{/**}{*/},
}
\colorlet{punct}{red!60!black}
\definecolor{background}{HTML}{EEEEEE}
\definecolor{delim}{RGB}{20,105,176}
\colorlet{numb}{magenta!60!black}
\lstdefinelanguage{json}{
    basicstyle=\normalfont\ttfamily,
    numbers=left,
    numberstyle=\scriptsize,
    stepnumber=1,
    numbersep=8pt,
    showstringspaces=false,
    breaklines=true,
    frame=lines,
    literate=
     *{:}{{{\color{punct}{:}}}}{1}
      {,}{{{\color{punct}{,}}}}{1}
      {\{}{{{\color{delim}{\{}}}}{1}
      {\}}{{{\color{delim}{\}}}}}{1}
      {[}{{{\color{delim}{[}}}}{1}
      {]}{{{\color{delim}{]}}}}{1},
}

\lstdefinelanguage{json-pretty}
{
  language=json,
  numbers=left,
  frame=shadowbox,
  rulesepcolor= \color{red!20!green!20!blue!20},
  basicstyle=\footnotesize\ttfamily,
  numberstyle=\scriptsize,
  breaklines=true,
  columns=fullflexible,
  xleftmargin=16pt,
  showstringspaces=false,
  keywordstyle=\color{blue}\bfseries,
  stringstyle=\color{javared},
  commentstyle=\color{javagreen},
  morecomment=[s][\color{javadocblue}]{/**}{*/},
}

\lstdefinelanguage{diff}{
    basicstyle=\ttfamily\small,
    morecomment=[f][\color{diffstart}]{@@},
    morecomment=[f][\color{javagreen}]{+\ },
    morecomment=[f][\color{javared}]{-\ },
  }

\def\diffinclcolor{\color{javagreen}}
\def\diffremcolor{\color{javared}}
  
\newcommand{\InputWithSpace}[1]{\bgroup\def\arraystretch{1.15}\input{#1}\egroup}


\settopmatter{printacmref=false}

\setcopyright{acmcopyright}
\copyrightyear{2024}
\acmYear{2024}
\acmDOI{XXXXXXX.XXXXXXX}


\begin{document}

\title{LessLeak-Bench: A First Investigation of Data Leakage in LLMs Across 83 Software Engineering Benchmarks}

\author{Xin Zhou}
\affiliation{%
  \institution{Singapore Management University}
  \country{Singapore}
}
\email{xinzhou.2020@phdcs.smu.edu.sg}


\author{Martin Weyssow}
\affiliation{%
  \institution{Singapore Management University}
  \country{Singapore}
}
\email{mweyssow@smu.edu.sg}

\author{Ratnadira WIDYASARI}
\affiliation{%
  \institution{Singapore Management University}
  \country{Singapore}
}
\email{ratnadiraw.2020@phdcs.smu.edu.sg}

\author{Ting Zhang}
\affiliation{%
  \institution{Singapore Management University}
  \country{Singapore}
}
\email{tingzhang.2019@phdcs.smu.edu.sg}

\author{Junda He}
\affiliation{%
  \institution{Singapore Management University}
  \country{Singapore}
}
\email{jundahe@smu.edu.sg}


\author{Yunbo Lyu}
\affiliation{%
  \institution{Singapore Management University}
  \country{Singapore}
}
\email{yunbolyu@smu.edu.sg}



\author{Jianming Chang}
\email{jianmingchang@seu.edu.cn}
\affiliation{%
  \institution{Southeast University}
  \city{Nanjing}
  \state{JiangSu}
  \country{China}
}


\author{Beiqi Zhang}
\email{zhangbeiqi@whu.edu.cn}
\affiliation{%
  \institution{Wuhan University}
  \city{Wuhan}
  \country{China}
}

\author{Dan Huang}
\affiliation{%
  \institution{Singapore Management University}
  \country{Singapore}
}
\email{dan.huang.2024@phdcs.smu.edu.sg}



\author{David Lo}
\affiliation{%
  \institution{Singapore Management University}
  \country{Singapore}
}
\email{davidlo@smu.edu.sg}






\renewcommand{\shortauthors}{Zhou et al.}




\begin{abstract}
Large Language Models (LLMs) are widely utilized in software engineering (SE) tasks, such as code generation and automated program repair. However, their reliance on extensive and often undisclosed pre-training datasets raises significant concerns about data leakage, where the evaluation benchmark data is unintentionally ``seen'' by LLMs during the model's construction phase.
The data leakage issue could largely undermine the validity of LLM-based research and evaluations. 
Despite the increasing use of LLMs in the SE community, there is no comprehensive study that assesses the extent of data leakage in SE benchmarks for LLMs yet.
To address this gap, this paper presents the first large-scale analysis of data leakage in 83 SE benchmarks concerning LLMs. 
We systematically investigated whether, and to what extent, popular SE benchmark datasets were included in a LLM's pre-training data. 
Our approach involved using an efficient near-duplicate data detection algorithm, MinHash+LSH, to identify potential duplicate pairs between the SE benchmarks and LLM's pre-training dataset. Subsequently, we conducted extensive manual labeling on these potential duplicates to identify true duplicates.
Those true duplicates reveal and confirm the data leakage of SE benchmarks. 
Our results show that in general, data leakage in SE benchmarks is minimal, with average leakage ratios of only 4.8\%, 2.8\%, and 0.7\% for Python, Java, and C/C++ benchmarks, respectively. However, some benchmarks exhibit relatively higher leakage ratios, which raises concerns about their bias in evaluation. For instance, QuixBugs and BigCloneBench have leakage ratios of 100.0\% and 55.7\%, respectively.
Furthermore, we observe that data leakage has a substantial impact on LLM evaluation. On the APPS benchmark, StarCoder-7B achieves a Pass@1 score that is 4.9 times higher on leaked samples than on non-leaked samples, highlighting how leaked benchmark data can lead to inflated metrics.
We also identify key causes of high data leakage, such as the direct inclusion of benchmark data in pre-training datasets and the use of coding platforms like LeetCode for benchmark construction.
To address the data leakage, we introduce \textbf{LessLeak-Bench}, a new benchmark that removes leaked samples from the 83 SE benchmarks, enabling more reliable LLM evaluations in future research.
Our study enhances the understanding of data leakage in SE benchmarks and provides valuable insights for future research involving LLMs in SE.



\end{abstract}

\maketitle

\newcommand{\XSpace}[1]{}

\makeatletter
\newenvironment{btHighlight}[1][]
{\begingroup\tikzset{bt@Highlight@par/.style={#1}}\begin{lrbox}{\@tempboxa}}
{\end{lrbox}\bt@HL@box[bt@Highlight@par]{\@tempboxa}\endgroup}

\newcommand\btHL[1][]{%
  \begin{btHighlight}[#1]\bgroup\aftergroup\bt@HL@endenv%
}
\def\bt@HL@endenv{%
  \end{btHighlight}%
  \egroup
}
\newcommand{\bt@HL@box}[2][]{%
  \tikz[#1]{%
    \pgfpathrectangle{\pgfpoint{1pt}{0pt}}{\pgfpoint{\wd #2}{\ht #2}}%
    \pgfusepath{use as bounding box}%
    \node[anchor=base west, fill=yellow!30,outer sep=0pt,inner xsep=1pt, inner ysep=0pt, rounded corners=0pt, minimum height=\ht\strutbox+1pt,#1]{\raisebox{1pt}{\strut}\strut\usebox{#2}};
  }%
}
\makeatother
\lstdefinestyle{Highlight}{
    moredelim=**[is][\btHL]{`}{`},
    moredelim=**[is][{\btHL[fill=orange!50]}]{´}{´},
    moredelim=**[is][{\btHL[fill=red!50]}]{@}{@},
}


\newcommand{\wholedata}{{\textit{complete data}}}
\newcommand{\singledata}{{\textit{single-line data}}}



\section{Introduction}
\section{Introduction}
\label{section:introduction}

% redirection is unique and important in VR
Virtual Reality (VR) systems enable users to embody virtual avatars by mirroring their physical movements and aligning their perspective with virtual avatars' in real time. 
As the head-mounted displays (HMDs) block direct visual access to the physical world, users primarily rely on visual feedback from the virtual environment and integrate it with proprioceptive cues to control the avatar’s movements and interact within the VR space.
Since human perception is heavily influenced by visual input~\cite{gibson1933adaptation}, 
VR systems have the unique capability to control users' perception of the virtual environment and avatars by manipulating the visual information presented to them.
Leveraging this, various redirection techniques have been proposed to enable novel VR interactions, 
such as redirecting users' walking paths~\cite{razzaque2005redirected, suma2012impossible, steinicke2009estimation},
modifying reaching movements~\cite{gonzalez2022model, azmandian2016haptic, cheng2017sparse, feick2021visuo},
and conveying haptic information through visual feedback to create pseudo-haptic effects~\cite{samad2019pseudo, dominjon2005influence, lecuyer2009simulating}.
Such redirection techniques enable these interactions by manipulating the alignment between users' physical movements and their virtual avatar's actions.

% % what is hand/arm redirection, motivation of study arm-offset
% \change{\yj{i don't understand the purpose of this paragraph}
% These illusion-based techniques provide users with unique experiences in virtual environments that differ from the physical world yet maintain an immersive experience. 
% A key example is hand redirection, which shifts the virtual hand’s position away from the real hand as the user moves to enhance ergonomics during interaction~\cite{feuchtner2018ownershift, wentzel2020improving} and improve interaction performance~\cite{montano2017erg, poupyrev1996go}. 
% To increase the realism of virtual movements and strengthen the user’s sense of embodiment, hand redirection techniques often incorporate a complete virtual arm or full body alongside the redirected virtual hand, using inverse kinematics~\cite{hartfill2021analysis, ponton2024stretch} or adjustments to the virtual arm's movement as well~\cite{li2022modeling, feick2024impact}.
% }

% noticeability, motivation of predicting a probability, not a classification
However, these redirection techniques are most effective when the manipulation remains undetected~\cite{gonzalez2017model, li2022modeling}. 
If the redirection becomes too large, the user may not mitigate the conflict between the visual sensory input (redirected virtual movement) and their proprioception (actual physical movement), potentially leading to a loss of embodiment with the virtual avatar and making it difficult for the user to accurately control virtual movements to complete interaction tasks~\cite{li2022modeling, wentzel2020improving, feuchtner2018ownershift}. 
While proprioception is not absolute, users only have a general sense of their physical movements and the likelihood that they notice the redirection is probabilistic. 
This probability of detecting the redirection is referred to as \textbf{noticeability}~\cite{li2022modeling, zenner2024beyond, zenner2023detectability} and is typically estimated based on the frequency with which users detect the manipulation across multiple trials.

% version B
% Prior research has explored factors influencing the noticeability of redirected motion, including the redirection's magnitude~\cite{wentzel2020improving, poupyrev1996go}, direction~\cite{li2022modeling, feuchtner2018ownershift}, and the visual characteristics of the virtual avatar~\cite{ogawa2020effect, feick2024impact}.
% While these factors focus on the avatars, the surrounding virtual environment can also influence the users' behavior and in turn affect the noticeability of redirection.
% One such prominent external influence is through the visual channel - the users' visual attention is constantly distracted by complex visual effects and events in practical VR scenarios.
% Although some prior studies have explored how to leverage user blindness caused by visual distractions to redirect users' virtual hand~\cite{zenner2023detectability}, there remains a gap in understanding how to quantify the noticeability of redirection under visual distractions.

% visual stimuli and gaze behavior
Prior research has explored factors influencing the noticeability of redirected motion, including the redirection's magnitude~\cite{wentzel2020improving, poupyrev1996go}, direction~\cite{li2022modeling, feuchtner2018ownershift}, and the visual characteristics of the virtual avatar~\cite{ogawa2020effect, feick2024impact}.
While these factors focus on the avatars, the surrounding virtual environment can also influence the users' behavior and in turn affect the noticeability of redirection.
This, however, remains underexplored.
One such prominent external influence is through the visual channel - the users' visual attention is constantly distracted by complex visual effects and events in practical VR scenarios.
We thus want to investigate how \textbf{visual stimuli in the virtual environment} affect the noticeability of redirection.
With this, we hope to complement existing works that focus on avatars by incorporating environmental visual influences to enable more accurate control over the noticeability of redirected motions in practical VR scenarios.
% However, in realistic VR applications, the virtual environment often contains complex visual effects beyond the virtual avatar itself. 
% We argue that these visual effects can \textbf{distract users’ visual attention and thus affect the noticeability of redirection offsets}, while current research has yet taken into account.
% For instance, in a VR boxing scenario, a user’s visual attention is likely focused on their opponent rather than on their virtual body, leading to a lower noticeability of redirection offsets on their virtual movements. 
% Conversely, when reaching for an object in the center of their field of view, the user’s attention is more concentrated on the virtual hand’s movement and position to ensure successful interaction, resulting in a higher noticeability of offsets.

Since each visual event is a complex choreography of many underlying factors (type of visual effect, location, duration, etc.), it is extremely difficult to quantify or parameterize visual stimuli.
Furthermore, individuals respond differently to even the same visual events.
Prior neuroscience studies revealed that factors like age, gender, and personality can influence how quickly someone reacts to visual events~\cite{gillon2024responses, gale1997human}. 
Therefore, aiming to model visual stimuli in a way that is generalizable and applicable to different stimuli and users, we propose to use users' \textbf{gaze behavior} as an indicator of how they respond to visual stimuli.
In this paper, we used various gaze behaviors, including gaze location, saccades~\cite{krejtz2018eye}, fixations~\cite{perkhofer2019using}, and the Index of Pupil Activity (IPA)~\cite{duchowski2018index}.
These behaviors indicate both where users are looking and their cognitive activity, as looking at something does not necessarily mean they are attending to it.
Our goal is to investigate how these gaze behaviors stimulated by various visual stimuli relate to the noticeability of redirection.
With this, we contribute a model that allows designers and content creators to adjust the redirection in real-time responding to dynamic visual events in VR.

To achieve this, we conducted user studies to collect users' noticeability of redirection under various visual stimuli.
To simulate realistic VR scenarios, we adopted a dual-task design in which the participants performed redirected movements while monitoring the visual stimuli.
Specifically, participants' primary task was to report if they noticed an offset between the avatar's movement and their own, while their secondary task was to monitor and report the visual stimuli.
As realistic virtual environments often contain complex visual effects, we started with simple and controlled visual stimulus to manage the influencing factors.

% first user study, confirmation study
% collect data under no visual stimuli, different basic visual stimuli
We first conducted a confirmation study (N=16) to test whether applying visual stimuli (opacity-based) actually affects their noticeability of redirection. 
The results showed that participants were significantly less likely to detect the redirection when visual stimuli was presented $(F_{(1,15)}=5.90,~p=0.03)$.
Furthermore, by analyzing the collected gaze data, results revealed a correlation between the proposed gaze behaviors and the noticeability results $(r=-0.43)$, confirming that the gaze behaviors could be leveraged to compute the noticeability.

% data collection study
We then conducted a data collection study to obtain more accurate noticeability results through repeated measurements to better model the relationship between visual stimuli-triggered gaze behaviors and noticeability of redirection.
With the collected data, we analyzed various numerical features from the gaze behaviors to identify the most effective ones. 
We tested combinations of these features to determine the most effective one for predicting noticeability under visual stimuli.
Using the selected features, our regression model achieved a mean squared error (MSE) of 0.011 through leave-one-user-out cross-validation. 
Furthermore, we developed both a binary and a three-class classification model to categorize noticeability, which achieved an accuracy of 91.74\% and 85.62\%, respectively.

% evaluation study
To evaluate the generalizability of the regression model, we conducted an evaluation study (N=24) to test whether the model could accurately predict noticeability with new visual stimuli (color- and scale-based animations).
Specifically, we evaluated whether the model's predictions aligned with participants' responses under these unseen stimuli.
The results showed that our model accurately estimated the noticeability, achieving mean squared errors (MSE) of 0.014 and 0.012 for the color- and scale-based visual stimili, respectively, compared to participants' responses.
Since the tested visual stimuli data were not included in the training, the results suggested that the extracted gaze behavior features capture a generalizable pattern and can effectively indicate the corresponding impact on the noticeability of redirection.

% application
Based on our model, we implemented an adaptive redirection technique and demonstrated it through two applications: adaptive VR action game and opportunistic rendering.
We conducted a proof-of-concept user study (N=8) to compare our adaptive redirection technique with a static redirection, evaluating the usability and benefits of our adaptive redirection technique.
The results indicated that participants experienced less physical demand and stronger sense of embodiment and agency when using the adaptive redirection technique. 
These results demonstrated the effectiveness and usability of our model.

In summary, we make the following contributions.
% 
\begin{itemize}
    \item 
    We propose to use users' gaze behavior as a medium to quantify how visual stimuli influences the noticebility of redirection. 
    Through two user studies, we confirm that visual stimuli significantly influences noticeability and identify key gaze behavior features that are closely related to this impact.
    \item 
    We build a regression model that takes the user's gaze behavioral data as input, then computes the noticeability of redirection.
    Through an evaluation study, we verify that our model can estimate the noticeability with new participants under unseen visual stimuli.
    These findings suggest that the extracted gaze behavior features effectively capture the influence of visual stimuli on noticeability and can generalize across different users and visual stimuli.
    \item 
    We develop an adaptive redirection technique based on our regression model and implement two applications with it.
    With a proof-of-concept study, we demonstrate the effectiveness and potential usability of our regression model on real-world use cases.

\end{itemize}

% \delete{
% Virtual Reality (VR) allows the user to embody a virtual avatar by mirroring their physical movements through the avatar.
% As the user's visual access to the physical world is blocked in tasks involving motion control, they heavily rely on the visual representation of the avatar's motions to guide their proprioception.
% Similar to real-world experiences, the user is able to resolve conflicts between different sensory inputs (e.g., vision and motor control) through multisensory integration, which is essential for mitigating the sensory noise that commonly arises.
% However, it also enables unique manipulations in VR, as the system can intentionally modify the avatar's movements in relation to the user's motions to achieve specific functional outcomes,
% for example, 
% % the manipulations on the avatar's movements can 
% enabling novel interaction techniques of redirected walking~\cite{razzaque2005redirected}, redirected reaching~\cite{gonzalez2022model}, and pseudo haptics~\cite{samad2019pseudo}.
% With small adjustments to the avatar's movements, the user can maintain their sense of embodiment, due to their ability to resolve the perceptual differences.
% % However, a large mismatch between the user and avatar's movements can result in the user losing their sense of embodiment, due to an inability to resolve the perceptual differences.
% }

% \delete{
% However, multisensory integration can break when the manipulation is so intense that the user is aware of the existence of the motion offset and no longer maintains the sense of embodiment.
% Prior research studied the intensity threshold of the offset applied on the avatar's hand, beyond which the embodiment will break~\cite{li2022modeling}. 
% Studies also investigated the user's sensitivity to the offsets over time~\cite{kohm2022sensitivity}.
% Based on the findings, we argue that one crucial factor that affects to what extent the user notices the offset (i.e., \textit{noticeability}) that remains under-explored is whether the user directs their visual attention towards or away from the virtual avatar.
% Related work (e.g., Mise-unseen~\cite{marwecki2019mise}) has showcased applications where adjustments in the environment can be made in an unnoticeable manner when they happen in the area out of the user's visual field.
% We hypothesize that directing the user's visual attention away from the avatar's body, while still partially keeping the avatar within the user's field-of-view, can reduce the noticeability of the offset.
% Therefore, we conduct two user studies and implement a regression model to systematically investigate this effect.
% }

% \delete{
% In the first user study (N = 16), we test whether drawing the user's visual attention away from their body impacts the possibility of them noticing an offset that we apply to their arm motion in VR.
% We adopt a dual-task design to enable the alteration of the user's visual attention and a yes/no paradigm to measure the noticeability of motion offset. 
% The primary task for the user is to perform an arm motion and report when they perceive an offset between the avatar's virtual arm and their real arm.
% In the secondary task, we randomly render a visual animation of a ball turning from transparent to red and becoming transparent again and ask them to monitor and report when it appears.
% We control the strength of the visual stimuli by changing the duration and location of the animation.
% % By changing the time duration and location of the visual animation, we control the strengths of attraction to the users.
% As a result, we found significant differences in the noticeability of the offsets $(F_{(1,15)}=5.90,~p=0.03)$ between conditions with and without visual stimuli.
% Based on further analysis, we also identified the behavioral patterns of the user's gaze (including pupil dilation, fixations, and saccades) to be correlated with the noticeability results $(r=-0.43)$ and they may potentially serve as indicators of noticeability.
% }

% \delete{
% To further investigate how visual attention influences the noticeability, we conduct a data collection study (N = 12) and build a regression model based on the data.
% The regression model is able to calculate the noticeability of the offset applied on the user's arm under various visual stimuli based on their gaze behaviors.
% Our leave-one-out cross-validation results show that the proposed method was able to achieve a mean-squared error (MSE) of 0.012 in the probability regression task.
% }

% \delete{
% To verify the feasibility and extendability of the regression model, we conduct an evaluation study where we test new visual animations based on adjustments on scale and color and invite 24 new participants to attend the study.
% Results show that the proposed method can accurately estimate the noticeability with an MSE of 0.014 and 0.012 in the conditions of the color- and scale-based visual effects.
% Since these animations were not included in the dataset that the regression model was built on, the study demonstrates that the gaze behavioral features we extracted from the data capture a generalizable pattern of the user's visual attention and can indicate the corresponding impact on the noticeability of the offset.
% }

% \delete{
% Finally, we demonstrate applications that can benefit from the noticeability prediction model, including adaptive motion offsets and opportunistic rendering, considering the user's visual attention. 
% We conclude with discussions of our work's limitations and future research directions.
% }

% \delete{
% In summary, we make the following contributions.
% }
% % 
% \begin{itemize}
%     \item 
%     \delete{
%     We quantify the effects of the user's visual attention directed away by stimuli on their noticeability of an offset applied to the avatar's arm motion with respect to the user's physical arm. 
%     Through two user studies, we identified gaze behavioral features that are indicative of the changes in noticeability.
%     }
%     \item 
%     \delete{We build a regression model that takes the user's gaze behavioral data and the offset applied to the arm motion as input, then computes the probability of the user noticing the offset.
%     Through an evaluation study, we verified that the model needs no information about the source attracting the user's visual attention and can be generalizable in different scenarios.
%     }
%     \item 
%     \delete{We demonstrate two applications that potentially benefit from the regression model, including adaptive motion offsets and opportunistic rendering.
%     }

% \end{itemize}

\begin{comment}
However, users will lose the sense of embodiment to the virtual avatars if they notice the offset between the virtual and physical movements.
To address this, researchers have been exploring the noticing threshold of offsets with various magnitudes and proposing various redirection techniques that maintain the sense of embodiment~\cite{}.

However, when users embody virtual avatars to explore virtual environments, they encounter various visual effects and content that can attract their attention~\cite{}.
During this, the user may notice an offset when he observes the virtual movement carefully while ignoring it when the virtual contents attract his attention from the movements.
Therefore, static offset thresholds are not appropriate in dynamic scenarios.

Past research has proposed dynamic mapping techniques that adapted to users' state, such as hand moving speed~\cite{frees2007prism} or ergonomically comfortable poses~\cite{montano2017erg}, but not considering the influence of virtual content.
More specifically, PRISM~\cite{frees2007prism} proposed adjusting the C/D ratio with a non-linear mapping according to users' hand moving speed, but it might not be optimal for various virtual scenarios.
While Erg-O~\cite{montano2017erg} redirected users' virtual hands according to the virtual target's relative position to reduce physical fatigue, neglecting the change of virtual environments. 

Therefore, how to design redirection techniques in various scenarios with different visual attractions remains unknown.
To address this, we investigate how visual attention affects the noticing probability of movement offsets.
Based on our experiments, we implement a computational model that automatically computes the noticing probability of offsets under certain visual attractions.
VR application designers and developers can easily leverage our model to design redirection techniques maintaining the sense of embodiment adapt to the user's visual attention.
We implement a dynamic redirection technique with our model and demonstrate that it effectively reduces the target reaching time without reducing the sense of embodiment compared to static redirection techniques.

% Need to be refined
This paper offers the following contributions.
\begin{itemize}
    \item We investigate how visual attractions affect the noticing probability of redirection offsets.
    \item We construct a computational model to predict the noticing probability of an offset with a given visual background.
    \item We implement a dynamic redirection technique adapting to the visual background. We evaluate the technique and develop three applications to demonstrate the benefits. 
\end{itemize}



First, we conducted a controlled experiment to understand how users perceived the movement offset while subjected to various distractions.
Since hand redirection is one of the most frequently used redirections in VR interactions, we focused on the dynamic arm movements and manually added angular offsets to the' elbow joint~\cite{li2022modeling, gonzalez2022model, zenner2019estimating}. 
We employed flashing spheres in the user's field of view as distractions to attract users' visual attention.
Participants were instructed to report the appearing location of the spheres while simultaneously performing the arm movements and reporting if they perceived an offset during the movement. 
(\zhipeng{Add the results of data collection. Analyze the influence of the distance between the gaze map and the offset.}
We measured the visual attraction's magnitude with the gaze distribution on it.
Results showed that stronger distractions made it harder for users to notice the offset.)
\zhipeng{Need to rewrite. Not sure to use gaze distribution or a metric obtained from the visual content.}
Secondly, we constructed a computational model to predict the noticing probability of offsets with given visual content.
We analyzed the data from the user studies to measure the influence of visual attractions on the noticing probability of offsets.
We built a statistical model to predict the offset's noticing probability with a given visual content.
Based on the model, we implement a dynamic redirection technique to adjust the redirection offset adapted to the user's current field of view.
We evaluated the technique in a target selection task compared to no hand redirection and static hand redirection.
\zhipeng{Add the results of the evaluation.}
Results showed that the dynamic hand redirection technique significantly reduced the target selection time with similar accuracy and a comparable sense of embodiment.
Finally, we implemented three applications to demonstrate the potential benefits of the visual attention adapted dynamic redirection technique.
\end{comment}

% This one modifies arm length, not redirection
% \citeauthor{mcintosh2020iteratively} proposed an adaptation method to iteratively change the virtual avatar arm's length based on the primary tasks' performance~\cite{mcintosh2020iteratively}.



% \zhipeng{TO ADD: what is redirection}
% Redirection enables novel interactions in Virtual Reality, including redirected walking, haptic redirection, and pseudo haptics by introducing an offset to users' movement.
% \zhipeng{TO ADD: extend this sentence}
% The price of this is that users' immersiveness and embodiment in VR can be compromised when they notice the offset and perceive the virtual movement not as theirs~\cite{}.
% \zhipeng{TO ADD: extend this sentence, elaborate how the virtual environment attracts users' attention}
% Meanwhile, the visual content in the virtual environment is abundant and consistently captures users' attention, making it harder to notice the offset~\cite{}.
% While previous studies explored the noticing threshold of the offsets and optimized the redirection techniques to maintain the sense of embodiment~\cite{}, the influence of visual content on the probability of perceiving offsets remains unknown.  
% Therefore, we propose to investigate how users perceive the redirection offset when they are facing various visual attractions.


% We conducted a user study to understand how users notice the shift with visual attractions.
% We used a color-changing ball to attract the user's attention while instructing users to perform different poses with their arms and observe it meanwhile.
% \zhipeng{(Which one should be the primary task? Observe the ball should be the primary one, but if the primary task is too simple, users might allocate more attention on the secondary task and this makes the secondary task primary.)}
% \zhipeng{(We need a good and reasonable dual-task design in which users care about both their pose and the visual content, at least in the evaluation study. And we need to be able to control the visual content's magnitude and saliency maybe?)}
% We controlled the shift magnitude and direction, the user's pose, the ball's size, and the color range.
% We set the ball's color-changing interval as the independent factor.
% We collect the user's response to each shift and the color-changing times.
% Based on the collected data, we constructed a statistical model to describe the influence of visual attraction on the noticing probability.
% \zhipeng{(Are we actually controlling the attention allocation? How do we measure the attracting effect? We need uniform metrics, otherwise it is also hard for others to use our knowledge.)}
% \zhipeng{(Try to use eye gaze? The eye gaze distribution in the last five seconds to decide the attention allocation? Basically constructing a model with eye gaze distribution and noticing probability. But the user's head is moving, so the eye gaze distribution is not aligned well with the current view.)}

% \zhipeng{Saliency and EMD}
% \zhipeng{Gaze is more than just a point: Rethinking visual attention
% analysis using peripheral vision-based gaze mapping}

% Evaluation study(ideal case): based on the visual content, adjusting the redirection magnitude dynamically.

% \zhipeng{(The risk is our model's effect is trivial.)}

% Applications:
% Playing Lego while watching demo videos, we can accelerate the reaching process of bricks, and forbid the redirection during the manipulation.

% Beat saber again: but not make a lot of sense? Difficult game has complicated visual effects, while allows larger shift, but do not need large shift with high difficulty





\section{Background}\label{section2}
%!TEX root = 2024_auv_mola_drl6dof_main.tex
%%%%%%%%%%%%%%%%%%%%%%%%%%%%%%%%%%%%%%%%%%%%%%%%%%%%%%%%%%%%%%%%%%%%%%%
\section{Background}
\label{sec:background}

%%%%%%%%%%%%%%%%%%%%%%%%%%%%%%%%%%%%%%%%%%%%%%%%%%
%%% Reinforcement learning
\subsection{Deep Reinforcement learning}

\ac{drl} \cite{RichardSutton20} is a method that aims to train an agent's policy $\pi$ to map states into actions by interacting with the environment. This is achieved by maximizing a numerical reward signal and using a \ac{mdp} framework to regulate the interaction between the \ac{rl} agent’s policy and the environment. At each time step, the agent observes a state $\bm{s}$, takes an action $\bm{a}$, and upon transitioning to the next state, receives a reward $r$. Once the episode (i.e., process) is complete, the accumulated reward is calculated as the sum of all time steps rewards in that episode.

\ac{drl} methods can be model-based or model-free. Model-based methods use a model to predict the next state and reward, while model-free methods learn solely from experiencing the unmodeled and unknown consequences of an action. While learning from trial and error may result in less efficient learning, model-free methods have the advantage when a model is unavailable or inaccurate.

\begin{figure}[t!]
\centering
\includegraphics[width=0.45\textwidth]{figures/reward_vs_step.pdf}%
\caption{Average reward per episode over a moving window of 100 episodes obtained by the TQC, SAC, and TD3 algorithms during a $2.5\times10^6$ step training, equivalent to 3125 episodes.}
\label{fig:rewards}
\end{figure}

%%%%%%%%%%%%%%%%%%%%%%%%%%%%%%%%%%%%%%%%%%%%%%%%%%
%% 6DOF Error Computation
\subsection{\ac{6dof} Error Computation}

The position errors are determined by the difference between the current position $(x, y, z)$ and the goal position $(x_d, y_d, z_d)$ following the North-East-Down (NED) convention, computed as
%%%
% Keep to remove space between equations and paragraph
%%%
\begin{equation}
    e_x(t) = x^t - x_d^t,\; e_y(t) = y^t - y_d^t,\; e_z(t) = z^t - z_d^t.
\label{eq:errors}
\end{equation}

To compute the error in attitude, we will evaluate the difference between the current orientation and the goal attitude, both with respect to the fixed world frame. This involves representing both poses as rotation matrices ($\bm{R}\in SO(3)$) and converting their difference to exponential coordinates $[\bm{{e_\theta}}]\in so(3)$ through the matrix logarithm:
%%%
% Keep to remove space between equations and paragraph
%%%
\begin{equation}
     [\bm{{e_\theta}}(t)] = \log(\bm{R}(t)^T \cdot \bm{R}_d)
\end{equation}

Then, the skew-symmetric matrix $[\bm{{e_\theta}}(t)]$ is converted into its vector representation $\bm{{e_\theta}}(t) \in \mathbb{R}^3$, where its entries correspond to the element-wise error for the attitude, defined as
%%%
% Keep to remove space between equations and paragraph
%%%
\begin{equation}
    \begin{bmatrix} \theta_{x}^t & \theta_{y}^t & \theta_{z}^t \end{bmatrix} = \bm{{e_\theta}}(t).
    \label{eq:attitude_error}
\end{equation}

Furthermore, to provide a single metric for attitude error evaluation, we compute $\theta^t$ based on the axis-angle representation for $\bm{{e_\theta}}(t)$, as described in \eqref{eq:theta_error}. By using this metric, we obtain a global evaluation of orientation, which aligns the controller's performance with practical manual navigation comparisons.
%%%
% Keep to remove space between equations and paragraph
%%%
\begin{equation}
    \theta^t = ||\bm{{e_\theta}}(t)||
    \label{eq:theta_error}
\end{equation}




\section{Methodology}\label{section3}
\section{Methodology}

We utilized LLMs to tackle the ASQP task across 0-, 10-, 20-, 30-, 40-, and 50-shot settings on different datasets. The performance is compared to that achieved using a dedicated training set to fine-tune smaller pre-trained language models. Furthermore, we report performance results for the TASD task.

\subsection{Evaluation}

\subsubsection{Datasets}

\begin{table*}[h]
\centering
\resizebox{1.8\columnwidth}{!}{%
\begin{tabular}{lccccc}
\hline
\textbf{}                    & \textbf{Rest15} & \textbf{Rest16} & \textbf{FlightABSA} & \textbf{OATS Coursera} & \textbf{OATS Hotels} \\ \hline
\textbf{\# Train}             & 834             & 1,264           & 1,351               & 1,400               & 1,400                \\
\textbf{\# Test}              & 537             & 544             & 387               & 400                 & 400                  \\
\textbf{\# Dev}              & 209             & 316             & 192               & 200                 & 200                  \\ \hline
\textbf{\# Aspect Categories} & 13              & 13              & 13              & 28                  & 33                   \\
\textbf{Language} & en              & en              & en              & en                  & en                   \\
\textbf{Domain} & restaurant              & restaurant              & airline              & e-learning                  & hotel                   \\
\hline
\end{tabular}
}
\caption{Overview of all ASQP datasets considered for evaluation. The datasets cover a range of different numbers of considered aspect categories and domains. }
\label{tab:overview-datasets}
\end{table*}

Table \ref{tab:overview-datasets} presents an overview of the datasets used in this study, including Rest15 and Rest16, along with three additional datasets covering diverse domains.


\textbf{Rest15 \& Rest16:} ASQP annotations originate from \citet{zhang2021aspect} and the TASD annotations from \citet{wan2020target}. This ensured comparability with the performance scores reported in previous research.

\textbf{FlightABSA:} A novel dataset containing 1,930 sentences annotated for ASQP. Properties of the annotated dataset are provided in Appendix \ref{appendix:flightabsa}. 

\textbf{OATS Hotels \& OATS Coursera:} We utilized a subset of two corpora recently introduced by \citet{chebolu2024oats} comprising ASQP-annotated sentences from reviews on hotels and e-learning courses. A detailed description of the data preprocessing for the OATS datasets can be found in Appendix \ref{appendix:oats-dataset}.

For the TASD task, we removed the opinion terms from the quadruples in annotations from FlightABSA, OATS Coursera and OATS Hotels. Subsequently, any duplicate triplets (\textit{a}, \textit{c}, \textit{p}) that appeared twice in a sentence were discarded.

\subsubsection{Setting}

For evaluation, the test dataset was considered for all datasets. An LLM was prompted five times with different seeds (0 to 4) for each combination of ABSA task (ASQP and TASD), dataset and amount of random few-shot examples (0, 10, 20, 30, 40 or 50) taken from the training set in order to get five label predictions. For all seeds, the same few-shot examples were used; however, they were shuffled differently for each prompt execution. The average performance across all five runs is reported.

\subsubsection{Metrics}

As in previous works in the field of ABSA, we report the micro-averaged F1 score as well as precision and recall to assess the model's performance. The F1 score is the harmonic mean of precision and recall. Precision measures the proportion of correctly predicted positive instances out of all instances predicted as positive \cite[p.~67]{jurafsky2000speech}. Recall quantifies the proportion of correctly predicted positive instances out of all actual positive instances in the dataset \cite[p.~67]{jurafsky2000speech}.

%, and is computed as follows:

%\[
%F1 = 2 \times \frac{\text{Precision} \times \text{Recall}}{\text{Precision} + \text{Recall}}
%\]
Similar to \citet{zhang2021aspect}, a quad prediction was considered correct if all the predicted sentiment elements are exactly the same as the gold labels. Recognizing the potential interest in class-level performance metrics for subsequent research, we have shared the predicted labels for every evaluated setting in our GitHub repository, allowing detailed class-level analysis.

\begin{figure*}[!htbp]
    \centering
    \includegraphics[width=2.1\columnwidth]{material/prompt.pdf}
    \caption{The prompt includes both a task description and specification of the output format. The LLM is run with five different seeds and in the case of self-consistency prompting, the tuple that appears most often across the five predictions is incorporated into the final label.}

\end{figure*}
\label{figure:study-prompt}

\subsection{Large Language Models}

We employed Gemma-2-27B\footnote{google/gemma-2-27b: \url{https://ollama.com/library/gemma2:27b}} by Google, which comprises 27.2 billion parameters \citep{team2024gemma}. Ollama\footnote{ollama: \url{https://ollama.com}} was employed for inference, and the LLMs were loaded with 4-bit quantization. The model was chosen for its efficiency in terms of generated tokens per second, which is a critical factor given the extensive prompt execution requirements. Notably, our study required over 342,720 prompts to be executed, with many few-shot learning prompts encompassing over a thousand tokens. For larger models, such as Llama-3.3-70B \citet{dubey2024llama}, the required computational costs would have been hardly feasible with our resources. For comparison purposes, we also report performance for the smaller-sized LLM, Gemma-2-9B\footnote{google/gemma-2-9b: \url{https://ollama.com/library/gemma2:9b}}.

The experiments were conducted on two NVIDIA RTX A5000 GPUs equipped with 24 GB of VRAM each. The LLM's temperature parameter was set to 0.8 and generation was terminated upon encountering the closing square bracket character (\texttt{"]"}) signifying the ending of a predicted label.

\subsection{Prompt}

\subsubsection{Components}

We adopted the prompting framework introduced by \citet{gou2023mvp} with some modifications. The employed prompt is illustrated in Figure \ref{figure:study-prompt} and an example is provided in Appendix \ref{appendix:prompt-example}. The main components of the prompt include a list of explanation on all considered sentiment elements and the specification of the output format. 

Unlike \citet{gou2023mvp}, our prompt instructs the LLM to pay attention to case sensitivity when returning aspect and opinion terms. Hence, the identified phrases should appear in the predicted tuple as they do in the sentence, similar to all supervised approaches mentioned in the related work section. Therefore, in the prompt, we clearly stated that the exact phrases should appear in the predicted label. 

Since we execute each prompt with five different seeds, we also report the performance when employing the self-consistency prompting technique introduced by \citet{wang2022self}. The key idea is to select the most consistent answer from multiple prompt executions. We adapted the approach for ABSA by incorporating a tuple into the merged label if it appears in the majority of the predicted labels. As illustrated in Figure \ref{figure:study-prompt}, this corresponds to a tuple appearing in at least 3 out of 5 predicted labels.

\subsection{Output Validation}

Since LLMs such as Gemma-2-27B cannot be strictly constrained to a fixed output format, we programmatically validated the output of the LLM. For the predicted label, several criteria needed to be met for the generation to be considered valid:

\begin{itemize}
    \item \textbf{Format}: The output must be a list of one or more tuples consisting of strings (quadruples for ASQP, triplets for TASD).
    \item \textbf{Sentiment}: The sentiment must be either 'positive', 'negative' or 'neutral'.
    \item \textbf{Aspect category}: Only the categories considered for the respective dataset and thus being mentioned in the prompt should be predicted as a part of a tuple.
    \item \textbf{Aspect and opinion terms}: Both must appear in the given sentence as predicted.
\end{itemize}

If any of the specified criteria for reasoning or label validation is not met, a regeneration attempt was triggered. If the predicted label was still invalid after 10 attempts, an empty label (\texttt{[]}) was considered as the predicted label.

\subsection{Baseline Model}

We compared the previously mentioned zero- and few-shot conditions against three SOTA baseline approaches, which are, the three best-performing methods for ASQP and TASD on the Rest15 and Rest16 datasets: Paraphrase \citep{zhang2021aspect}, DLO \citep{hu2022improving} and MVP \citep{gou2023mvp}.

\begin{description}
    \item[Paraphrase \citep{zhang2021aspect}:] \textit{Paraphrase} is used to linearize sentiment quads into a natural language sequence to construct the input target pair.
    \item[DLO \citep{hu2022improving}:] \textit{Dataset-level order} is a method designed for ASQP that leverages the order-free property of quadruplets. It identifies and utilizes optimal template orders through entropy minimization and combines multiple effective templates for data augmentation.
    \item[MVP \citep{gou2023mvp}:] \textit{Multi-view-Prompting} introduces element order prompts. The language model is guided to generate multiple sentiment tuples, with a different element order each, and then selects the most reasonable tuples by a voting mechanism. This method is highly resource-intensive, as multiple input-output pairs are created for each example in the train set, each comprising different sentiment element positions.
\end{description}

For all three approaches, we conducted training using the entire dataset and performed training with only 10, 20, 30, 40, or 50 training examples equally to the ones employed for the few-shot learning conditions. Training was conducted using five different random seeds (0 to 4). Moreover, to facilitate comparisons across datasets, we trained models using 800 training examples, as this represents the largest multiple of 100 examples available for all train sets (900 training examples are not available for Rest15). The results obtained using the full training sets of Rest15 and Rest16 were extracted from the works of \citet{zhang2021aspect}, \citet{hu2022improving}, and \citet{gou2023mvp}.

For all methods, we used the hyperparameter configurations used by \citet{zhang2021aspect}, \citet{hu2022improving} and \citet{gou2023mvp}. The only exception was the 10-shot condition, where batch size was set to 8 instead of 16, as the limited number of examples (10) could not form a batch of 16 examples.




\section{Experimental Setup}\label{section4}
In this section, we discuss the selection process for the studied LLM and SE benchmarks. Additionally, we outline the implementation details and define the research questions in this study.

\vspace{-0.2cm}
\subsection{LLM Selection} 
\label{llm_selection}

 



 \textbf{Selection Criteria.} Selecting an appropriate target LLM was a crucial step in this study due to the extensive manual annotation involved, which makes repeated experiments across multiple models infeasible. To guide our selection, we established the following selection criteria:
 

\noindent
\begin{enumerate}[left=0pt]
    \item \textit{Full Open-Source Availability.} The selected LLM must be fully open-source, encompassing both model parameters and pre-training data. However, since re-training the model is outside the scope of our study, open-source pre-training algorithms and scripts are not required.
    \item \textit{High Effectiveness.} The model should demonstrate strong effectiveness on widely used SE benchmarks, such as HumanEval~\cite{chen2021evaluating} and MBPP~\cite{MBPP_1}.
    \item \textit{Influence and Adoption.} 
We prioritized models with significant influence, particularly those that have inspired or laid the groundwork for the development of newer/better LLMs.
\end{enumerate}

\begin{table*}[t]
\caption{The Overview of Large Language Models (LLMs) for Code.}
 \vspace{-0.2cm}
\label{tab:model_selection}
\centering 
\scalebox{0.75}{ 
\rotatebox{0}{
\begin{tabular}{lllcccrr}
\hline
\textbf{Model}                                             & \textbf{Institution} & \textbf{Size}                                                                                                                              & \textbf{Date} & \textbf{\begin{tabular}[c]{@{}c@{}}Open Source\\ (Model)\end{tabular}} & \textbf{\begin{tabular}[c]{@{}c@{}}Open Source\\ (Data)\end{tabular}} & \multicolumn{1}{c}{\textbf{\begin{tabular}[c]{@{}c@{}}HumanEval \\ (pass@1)\end{tabular}}} & \multicolumn{1}{c}{\textbf{\begin{tabular}[c]{@{}c@{}}MBPP\\ (pass@1)\end{tabular}}} \\ \hline

GPT-C~\cite{svyatkovskiy2020intellicode}  & Microsoft            & 366M                                                                                                                                       & 2020-05       &                                                                        & \multicolumn{1}{l}{}                                                  & -                                                                                          & -                                                                                    \\ 
\rowcolor{lightgreen!50}  CodeBERT~\cite{feng2020codebert}            & Microsoft            & 124M                                                                                                                                       & 2020-09       & \CheckmarkBold                                          & \CheckmarkBold                                         & -                                                                                          & -                                                                                    \\
\rowcolor{lightgreen!50}  GraphCodeBERT~\cite{guo2020graphcodebert}            & Microsoft            & 124M                                                                                                                                       & 2021-02       & \CheckmarkBold                                          & \CheckmarkBold                                         & -                                                                                          & -                                                                                    \\
\rowcolor{lightgreen!50} 
CodeGPT~\cite{lu2021codexglue}            & Microsoft            & 124M                                                                                                                                       & 2021-02       & \CheckmarkBold                                          & \CheckmarkBold                                         & -                                                                                          & -                                                                                    \\ \rowcolor{lightgreen!50} 
GPT-Neo~\cite{gpt-neo}                     & EleutherAI           & 125M, 1.3B, 2.7B
                                                                                       & 2021-03       & \CheckmarkBold                                          & \CheckmarkBold                                         & 6.41                                                                                       & 5.89   \\ 
\rowcolor{lightgreen!50}  PLBART\cite{ahmad2021unified} & UCLA & 140M &	2021-03 & \CheckmarkBold   & \CheckmarkBold & - & -\\\rowcolor{lightgreen!50} 
GPT-J~\cite{gpt-j}                        & EleutherAI           & 6B                                                                                                                                         & 2021-05       & \CheckmarkBold                                          & \CheckmarkBold                                         & 11.62                                                                                      & 11.30                                                                                \\ 
Codex~\cite{chen2021evaluating}           & OpenAI               & 12M-12B
 & 2021-07       &                                                                        & \multicolumn{1}{l}{}                                                  & 28.81                                                                                      & -                                                                                    \\
\rowcolor{lightgreen!50}  CodeT5 \cite{wang2021codet5} & Salesforce & 60M, 220M, 770M & 2021-09 & \CheckmarkBold & \CheckmarkBold & - & - \\
\rowcolor{lightgreen!50}  CodeParrot~\cite{tunstall2022natural}     & Hugging Face         & 110M, 1.5B                                                                                                                                 & 2021-11       & \CheckmarkBold                                          & \CheckmarkBold                                         & 3.99                                                                                       & 1.29                                                                                 \\
\rowcolor{lightgreen!50}  PolyCoder~\cite{xu2022systematic}         & CMU                  & 160M, 400M, 2.7B                                                                                                                           & 2022-02       & \CheckmarkBold                                          & \CheckmarkBold                                         & 5.59                                                                                       & 4.39                                                                                 \\
\rowcolor{lightgreen!50}  CodeGen~\cite{nijkamp2022codegen}         & Salesforce           & 350M-16.1B                                                 & 2022-03       & \CheckmarkBold                                          & \CheckmarkBold                                         & 29.28                                                                                      & 35.28                                                                                \\
\rowcolor{lightgreen!50}  GPT-NeoX~\cite{black2022gpt}              & EleutherAI           & 20B                                                                                                                                        & 2022-04       & \CheckmarkBold                                          & \CheckmarkBold                                         & 15.4                                                                                       & -                                                                                    \\
PaLM-Coder~\cite{chowdhery2023palm}       & Google               & 8B, 62B, 540B                                                                                                                              & 2022-04       &                                                                        & \multicolumn{1}{l}{}                                                  & 36                                                                                         & 47                                                                                   \\
\rowcolor{lightgreen!50}  InCoder~\cite{fried2022incoder}           & Meta                 & 1.3B, 6.7B                                                                                                                                 & 2022-04       & \CheckmarkBold                                          & \CheckmarkBold                                         & 15.2                                                                                       & 21.3                                                                                 \\
PanGu-Coder~\cite{christopoulou2022pangu} & Huawei               & 317M, 2.6B                                                                                                                                 & 2022-07       &                                                                        & \multicolumn{1}{l}{}                                                  & 23.78                                                                                      & 23.0                                                                                 \\
PyCodeGPT~\cite{zan2022cert}              & Microsoft            & 110M                                                                                                                                       & 2022-06       & \CheckmarkBold                                          & \multicolumn{1}{l}{}                                                  & 8.33                                                                                       & 9.39                                                                                 \\
\rowcolor{lightgreen!50}  CodeGeeX~\cite{zheng2023codegeex}         & Tsinghua             & 13B                                                                                                                                        & 2022-09       & \CheckmarkBold                                          & \CheckmarkBold                                         & 22.89                                                                                      & 24.4                                                                                 \\
\rowcolor{lightgreen!50}  BLOOM~\cite{le2023bloom}                  & BigScience           & 176B                                                                                                                                       & 2022-11       & \CheckmarkBold                                          & \CheckmarkBold                                         & 15.52                                                                                      & -                                                                                    \\
GPT-3.5-Turbo~\cite{gpt-3.5-turbo}        & OpenAI               & -                                                                                                                                          & 2022-11       &                                                                        & \multicolumn{1}{l}{}                                                  & 76.2                                                                                       & 52.2                                                                                 \\
\rowcolor{lightgreen!50}  SantaCoder~\cite{allal2023santacoder}     & Hugging Face         & 1.1B                                                                                                                                       & 2022-12       & \CheckmarkBold                                          & \CheckmarkBold                                         & 18                                                                                         & 3.65                                                                                 \\
GPT-4~\cite{gpt4}                & OpenAI               & -                                                                                                                                          & 2023-03       &                                                                        & \multicolumn{1}{l}{}                                                  & 84.1                                                                                       & -                                                                                    \\
\rowcolor{lightgreen!50}  CodeGen2~\cite{nijkamp2023codegen2}       & Salesforce           & 1B-16B                                                                                                                          & 2023-05       & \CheckmarkBold                                          & \CheckmarkBold                                         & 20.46                                                                                      & -                                                                                    \\
\rowcolor{lightred!60}  StarCoder~\cite{starcoder_one}          & Hugging Face         & 15.5B                                                                                                                                      & 2023-05       & \CheckmarkBold                                          & \CheckmarkBold                                         & 33.60                                                                                      & \textcolor{black}{\textbf{52.7}}                                                                                 \\
PanGu-Coder2~\cite{shen2023pangu}         & Huawei               & 15B                                                                                                                                        & 2023-07       &                                                                        & \multicolumn{1}{l}{}                                                  & 61.64                                                                                      & -                                                                                    \\
Llama 2~\cite{llama2}          & Meta                 & 7B, 13B, 70B                                                                                                                               & 2023-07       & \CheckmarkBold                                          & \multicolumn{1}{l}{}                                                  & -                                                                                          & 45.4                                                                                 \\
Code Llama~\cite{roziere2023code}         & Meta                 & 7B, 13B, 34B                                                                                                                               & 2023-08       & \CheckmarkBold                                          & \multicolumn{1}{l}{}                                                  & 48.8                                                                                       & 55                                                                                   \\
phi-1.5~\cite{li2023textbooks}            & Microsoft            & 1.3B                                                                                                                                       & 2023-09       & \CheckmarkBold                                          & \multicolumn{1}{l}{}                                                  & 41.4                                                                                       & 43.5                                                                                 \\
phi-2~\cite{phi-2}                        & Microsoft            & 2.7B                                                                                                                                       & 2023-12       & \CheckmarkBold                                          & \multicolumn{1}{l}{}                                                  & 49.4                                                                                       & 64                                                                                   \\
DeepSeek-Coder~\cite{deepseekcoder}     & DeepSeek             & 1.3B, 6.7B, 33B                                                                                                                            & 2023-11       & \CheckmarkBold                                          & \multicolumn{1}{l}{}                                                  & 56.1                                                                                       & 66                                                                                   \\
\rowcolor{lightgreen!60}  StarCoder2~\cite{starcoder2}   & Hugging Face         & 15B                                                                                                                                        & 2024-02       & \CheckmarkBold                                          & \CheckmarkBold                                         & \textcolor{black}{\textbf{46.3}}                                                                                       & 50.6                                                                                \\
Claude-3-Opus~\cite{claude3}              & Anthropic            & -                                                                                                                                          & 2024-03       &                                                                        & \multicolumn{1}{l}{}                                                  & 82.9                                                                                       & 89.4                                                                                 \\
CodeGemma~\cite{codegemma_2024}          & Google               & 2B, 7B                                                                                                                                     & 2024-04       & \CheckmarkBold                                          & \multicolumn{1}{l}{}                                                  & 44.5                                                                                       & 65.1                                                                                 \\
Code-Qwen~\cite{codeqwen}                 & Qwen Group           & 7B                                                                                                                                         & 2024-04       & \CheckmarkBold                                          & \multicolumn{1}{l}{}                                                  & 45.1                                                                                       & 51.4                                                                                 \\
Llama3~\cite{llama3}                      & Meta                 & 8B, 70B                                                                                                                                    & 2024-04       & \CheckmarkBold                                          & \multicolumn{1}{l}{}                                                  & 81.7                                                                                       & -               \\                                                                 \bottomrule
\end{tabular} 
}} \vspace{-0.4cm}
\end{table*}


\vspace{0.1cm}
\noindent
 \textbf{Target LLM Selection Process.}
 \revised{
Table~\ref{tab:model_selection} presents our analysis of various code-related LLMs in chronological order, till April 2024, when this study was initiated. Models highlighted in light green meet our full open-source criteria, while those marked in light red represent the most suitable candidates based on our selection criteria. After thorough analysis, we selected \textbf{StarCoder}~\cite{starcoder_one} as the research LLM for this study.}


In accordance with our first criterion, we excluded proprietary models such as GPT-3.5~\cite{gpt-3.5-turbo} and GPT-4~\cite{gpt4}, along with recent open-source models like LLaMA3~\cite{llama3}, CodeQwen~\cite{codeqwen}, and DeepSeek-Coder~\cite{deepseekcoder}, which do not fully comply with open-source data requirements. Among the fully open-source options, StarCoder~\cite{starcoder_one} and StarCoder2~\cite{starcoder2} ranked highest on two widely used code generation benchmarks, HumanEval~\cite{chen2021evaluating} and MBPP~\cite{MBPP_1}, thereby fulfilling our second criterion.



When comparing StarCoder~\cite{starcoder_one} with StarCoder2~\cite{starcoder2}, we found that StarCoder’s earlier release positioned it as a foundational model in the field, serving as the basis for several other strong/newer LLMs after fine-tuning, such as PanGu-Coder2~\cite{shen2023pangu}, WizardCoder~\cite{luo2023wizardcoder}, and OctoPack~\cite{muennighoff2023octopack}. This suggests that findings from StarCoder extend to these derivative models. Specifically, if SE benchmark data has leaked to StarCoder, it also affects its descendant LLMs as well, thereby enhancing the broader applicability of our findings and conclusions. Consequently, we opted to select StarCoder over StarCoder2.

Lastly, while some other LLMs do not directly build upon StarCoder, they utilize the pre-training dataset of StarCoder or adopt data curation techniques inspired by StarCoder’s pre-training practices. For instance, CodeShell~\cite{xie2024codeshell} incorporates StarCoder’s pre-training data, while DeepSeek-Coder~\cite{deepseekcoder} employs data-filtering methods akin to those used by StarCoder to gather more recent GitHub data, extending to February 2023. This suggests that findings from StarCoder can be generalized to CodeShell and are likely applicable to DeepSeek-Coder as well, as DeepSeek-Coder follows a similar methodology to extend its datasets. 
In summary, we believe StarCoder is a suitable LLM for us to better understand the SE benchmark leakage status.


\vspace{0.2cm}
\noindent
\textbf{StarCoder's Pre-training Data.} 
StarCoder’s pre-training data is sourced from \textit{The Stack~\cite{stack}} dataset. \textit{The Stack~\cite{stack}} dataset comprises over 6TB of permissively licensed code spanning 358 programming languages. The Stack is collected from public GitHub repositories between 2015 and 2022. For pre-training, StarCoder focused on the 86 programming languages that either contained more than 500MB of data or ranked in the top 50 on popularity indices such as \textit{Githut 2.0}\footnote{\url{https://githut.info/}} or the \textit{December 2022 TIOBE Index}\footnote{\url{https://web.archive.org/web/20221229040526/https://www.tiobe.com/tiobe-index/}}. GitHub repositories for these popular programming languages are recognized as valuable data sources within the SE community, ensuring the representativeness of StarCoder’s pre-training data.



Specifically, StarCoder was pre-trained on a dataset of 305M files, totaling 800+ GB.
In this study, we focus on SE benchmarks in three popular programming languages: Python, Java, and C/C++. Therefore, we restrict our analysis to the corresponding subsets within StarCoder's pre-training data. 
These subsets include 12M files for Python, 20M files for Java, and 14M files for C/C++. 
Despite narrowing our scope to these languages, the data volume remains substantial, supporting our decision to use an efficient automated tool to identify potential duplicate pairs before proceeding with manual labeling. 


\vspace{-0.2cm}
 \subsection{Benchmark Data Selection}
 \label{benchmark_selection}


The whole SE community has developed numerous high-quality benchmarks. However, the extensive manual annotation required for each makes it impractical to cover all valuable SE benchmarks. To investigate the leakage status of SE benchmarks in relation to LLMs, we focus on selecting benchmarks that have been previously used for LLM evaluation. Our selection is further constrained to benchmarks within three widely used PLs: Python, Java, and C/C++.



\vspace{0.1cm}
\noindent
\textbf{Studied SE Benchmark Selection.} 
To identify relevant benchmarks, we leverage a recent and comprehensive survey on LLMs for SE tasks~\cite{hou2024large}, which analyzed 395 research papers from January 2017 to January 2024. 
From the papers referenced in this survey, we selected benchmarks and datasets used for evaluating LLMs. We further expanded our selection by examining the citations and related works of these papers, excluding those lacking replication packages or inaccessible datasets.
In building our benchmark collection, we prioritized including datasets across a variety of SE tasks, rather than concentrating on any one task with extensive datasets. 



To ensure clarity in our analysis, benchmarks with multiple variants—such as those involving different programming languages or scenarios—were assigned distinguishing tags appended to their original names.
For example, the CodeEditorBench~\cite{CodeEditorBench} benchmark includes variants in three programming languages: Python, Java, and C++. Additionally, it features four scenarios: Code Debug, Code Translate, Code Polish, and Code Requirement Switch. Since each variant contains distinct benchmark data samples, we assign a unique name to each variant.
The naming convention combines the following components:
(1) benchmark original name, (2) scenario name, and (3) programming language name. Name tags for the scenario and programming language are included only if there are multiple variations across different scenarios or programming languages.
For example, the CodeEditorBench~\cite{CodeEditorBench} variant containing Python data samples for the Code Debug scenario is renamed as ``CodeEditorBench-\textit{debug-py}'', following the naming convention. Here, ``\textit{debug}'' indicates the Code Debug scenario, and ``\textit{py}'' represents the Python data.




Through this process, we ultimately compiled a set of 83 SE datasets covering a broad range of SE tasks, including code generation, program repair, code editing, code translation, code review, debugging, code execution, test output prediction, secure code generation, GitHub issue fixing, clone detection, log generation, vulnerability repair, and vulnerability detection.
A brief overview of the benchmarks we studied is provided in Tables~\ref{tab:main_results_py}, ~\ref{tab:main_results_java}, and ~\ref{tab:main_results_c}.








\vspace{-0.2cm}
\subsection{Implementation Details}
We retrieve StarCoder's pre-trained data from its official HuggingFace homepage.\footnote{\url{https://huggingface.co/bigcode/starcoder}} We execute computations on an NVIDIA GeForce A5000 GPU with 24 GB of memory. We acquire the data for each SE benchmark dataset directly from its replication packages or official websites. For the Minhash+LSH method, we adopt the BigCode Team’s implementation~\cite{bigcode}, which leverages the \textit{\textsf{DataSketches}}~\cite{datasketch} library.



\vspace{-0.2cm}
\subsection{Research Questions}
Our work aims to mainly answer four Research Questions (RQs). 
\begin{itemize}[leftmargin=*]
\item \textbf{RQ1: To what extent does data leakage exist in the studied SE benchmarks?} 
In RQ1, we evaluate an extensive set of 83 SE benchmarks to investigate potential data leakage into an advanced LLM StarCoder~\cite{starcoder_one}.
\item \textbf{RQ2: What factors contribute to high leakage rates in SE benchmarks?}
In RQ2, we analyze the top benchmarks with high leakage rates, discussing the potential reasons behind their high leakage ratios. 
\item \textbf{RQ3: How does the leakage of benchmark data affect the effectiveness of LLMs?} 
In RQ3, we measure the effectiveness differences between leaked and non-leaked portions of the data to explore the impact caused by data leakage.
\item \textbf{RQ4: How effective is the automated metric in detecting data leakage when lacking access to LLM pre-training data?} 
Since pre-training data for many LLMs is inaccessible, this RQ investigates whether SE benchmark leakage can be inferred solely from LLM behaviors (i.e., without access to pre-training data), such as the Perplexity scores of LLMs. 
\end{itemize}








\section{Experimental Results}\label{section5}

In this section, we first present the overall results of the data leakage detection analysis. Following that, we provide detailed results and answers to each research question (RQ).



\vspace{0.2cm}
\noindent
\textbf{Overall Results.}
Before diving into the specific RQs, we first present an overview of the experimental results. The selected LLM's pre-training datasets consist of 12M samples for Python, 20M samples for Java, and 14M samples for C/C++. In comparison, the diverse SE benchmarks we studied collectively comprise 46k samples for Python, 42k samples for Java, and 21k samples for C/C++.
To investigate potential data leakage, each SE benchmark sample was compared against all pre-training data samples for its corresponding programming language. This process resulted in an astounding total of over 1.7 trillion comparisons. The sheer scale of this computational effort highlights the complexity and resource-intensive nature of studying data leakage regarding LLMs. 


From an overall perspective, as depicted in Figure~\ref{fig:data_review}, only 2\% of the benchmark samples from all the SE benchmarks studied were flagged by the automated tool MinHash+LSH as potentially forming at least one duplicate pair with the pre-training data of StarCoder. Moreover, of the pairs flagged by MinHash+LSH, 28\% were confirmed as duplicates after manual labeling, while the remaining 72\% were determined not to be duplicates.

Next, we will discuss the detailed results and answers to each RQ.


\begin{table}
  \centering
\caption{\textcolor{black}{Effect of duplicate code removal on structural metrics. (+ve) indicates positive impact; (-ve) indicates negative impact; (-) indicates metric remains unaffected, \textbf{bold} indicates statistical significance; \textit{italic} indicates improvement.}}
\label{Table:Metrics Suites and Metrics Tools Summary}
%\begin{sideways}
\begin{adjustbox}{width=1.0\textwidth,center}
%\begin{adjustbox}{width=\textheight,totalheight=\textwidth,keepaspectratio}
\begin{tabular}{lllll}\hline
\toprule
\bfseries Quality Attribute & \bfseries Metric & \bfseries Impact & \bfseries \textit{p}-value & \bfseries Cliff's delta ($\delta$) \\
\midrule
%\multicolumn{2}{l}{\textbf{\textit{Internal Quality Attribute }}}\\
%\midrule
Cohesion &  LCOM5  & +ve & \textit{\textbf{7.72e-41}} & 0.54 (Large)
%(Small)   
\\ 
Coupling &  \cellcolor{gray!30}CBO  & \cellcolor{gray!30}+ve & \cellcolor{gray!30}\textit{\textbf{9.49e-76}} & \cellcolor{gray!30}0.6 (Large)
%(Small) 
\\
         &  RFC & +ve & \textit{\textbf{1.25e-68}}  & 0.55 (Large)

\\
         &  \cellcolor{gray!30}NII & \cellcolor{gray!30}-ve &  \cellcolor{gray!30}\textbf{0} & \cellcolor{gray!30}0.47 (Large)

\\
         &  NOI & +ve & \textit{\textbf{0}}  & 0.26 (Small)

\\
Complexity &  \cellcolor{gray!30}CC & \cellcolor{gray!30}- & \cellcolor{gray!30}\textbf{0} & \cellcolor{gray!30}0.14 (Small)

\\
           &  WMC & +ve & \textit{\textbf{6.51e-70}} & 0.5 (Large)

\\
         &  \cellcolor{gray!30}NL &  \cellcolor{gray!30}- &  \cellcolor{gray!30}\textbf{3.92e-05}&  \cellcolor{gray!30}0.03 (Negligible)

\\
         &  NLE &  - & \textbf{0.004}  & 0.02 (Negligible)

\\
         &  \cellcolor{gray!30}HCPL & \cellcolor{gray!30}+ve &  \cellcolor{gray!30}\textit{\textbf{0}} &  \cellcolor{gray!30}0.14 (Negligible)

\\
         &  HDIF & +ve & \textit{\textbf{0}}  &  0.08 (Negligible)

\\
         &  \cellcolor{gray!30}HEFF & \cellcolor{gray!30}+ve &  \cellcolor{gray!30}\textit{\textbf{2.45e-271}} &  \cellcolor{gray!30}0.13 (Negligible)

\\
         &  HNDB & +ve &  \textit{\textbf{1.07e-266}} & 0.13  (Negligible)

\\
         &  \cellcolor{gray!30}HPL & \cellcolor{gray!30}+ve & \cellcolor{gray!30}\textit{\textbf{0}} & \cellcolor{gray!30}0.13  (Negligible)

\\
         &  HPV & +ve & \textit{\textbf{0}}  & 0.14  (Negligible)

\\
         &  \cellcolor{gray!30}HTRP & \cellcolor{gray!30}+ve &  \cellcolor{gray!30}\textit{\textbf{2.48e-271}} & \cellcolor{gray!30}0.13  (Negligible)

\\
         &  HVOL & +ve & \textit{\textbf{0}}  &  0.13  (Negligible)

\\
         &  \cellcolor{gray!30}MIMS & \cellcolor{gray!30}+ve & \cellcolor{gray!30}\textit{\textbf{7.23e-227}}  &  \cellcolor{gray!30}0.13  (Negligible)

\\
         &   MI & +ve &  \textit{\textbf{7.22e-227}} &  0.13  (Negligible)

\\
         &   \cellcolor{gray!30}MISEI & \cellcolor{gray!30}+ve & \cellcolor{gray!30}\textit{\textbf{0}} & \cellcolor{gray!30}0.16  (Small)

\\
         &   MISM &  +ve&  \textit{\textbf{0}}  & 0.16  (Small)

\\
Inheritance &   \cellcolor{gray!30}DIT & \cellcolor{gray!30}-ve & \cellcolor{gray!30}\textbf{3.81e-199} & \cellcolor{gray!30}0.6 (Large) 
 
\\
            &  NOC & +ve & \textbf{\textit{3.61e-130}} & 0.83 (Large)  

\\
            &  \cellcolor{gray!30}NOA & \cellcolor{gray!30}-ve & \cellcolor{gray!30}\textbf{2.37e-196}  & \cellcolor{gray!30}0.63 (Large)
 
\\ 
Design Size &  LOC & +ve & \textbf{\textit{0}} &   0.14 (Small)

\\
         &  \cellcolor{gray!30}TLOC & \cellcolor{gray!30}+ve &   \cellcolor{gray!30}\textit{\textbf{0}} &  \cellcolor{gray!30}0.16 (Small)

\\
            &  LLOC &  +ve &  \textbf{\textit{0}} &   0.13 (Negligible)

\\
         &  \cellcolor{gray!30}TLLOC & \cellcolor{gray!30}+ve & \cellcolor{gray!30}\textit{\textbf{0}}  &   \cellcolor{gray!30}0.15 (Small)

\\
            &   CLOC & - &  \textbf{1.43e-05} &   0.02 (Negligible)

\\
            &  \cellcolor{gray!30}NPM & \cellcolor{gray!30}- & \cellcolor{gray!30}\textbf{4.42e-193} & \cellcolor{gray!30}0.5 (Large)

\\
         &  NOS &  +ve&  \textit{\textbf{0}} &  0.07 (Negligible)

\\
         &  \cellcolor{gray!30}TNOS & \cellcolor{gray!30}+ve & \cellcolor{gray!30}\textit{\textbf{0}}  &  \cellcolor{gray!30}0.08 (Negligible)

\\
\bottomrule
\end{tabular}
\end{adjustbox}
%\end{sideways}
\end{table}
\begin{comment}
    


%\begin{sidewaystable}
\begin{table*}

\caption{The impact of duplicate code removal on quality metrics.}
\label{composites}


\fontsize{10}{14}\selectfont
	\tabcolsep=0.1cm
\resizebox{\textwidth}{!}{
\begin{tabular}{llcccccccccccccccccccccc} \toprule
\multicolumn{1}{c}{}                             & \multicolumn{1}{c}{}                          & \multicolumn{3}{c}{Cohesion}                                                                                                            & \multicolumn{5}{c}{Coupling}                                                                                                          & \multicolumn{4}{c}{Complexity}                                                                                                                                                         & \multicolumn{3}{c}{Inheritance}                                                         & \multicolumn{7}{c}{Design Size}                                                                                                                                                                                                                                                                         \\
\multicolumn{1}{c}{\multirow{-2}{*}{}} & \multicolumn{1}{c}{\multirow{-2}{*}{Measure}} & LCOM5                                        &                                          &                                          & CBO                                         & RFC                                         &     &    &                 & WMC                                         & CC                                         &                                  &                                          & DIT                                         & NOC                 &   NOA            & SLOC                                         & LLOC                                        & CLOC                &      NPM                                  &                                        &                                          &                                         \\ \hline
                                                 & Refactoring Impact                            & 0                                           & \cellcolor[HTML]{0350F8}1                   & \cellcolor[HTML]{0350F8}1                   & 0                                           & \cellcolor[HTML]{0350F8}-12                 & 0        & 0        & 0                   & \cellcolor[HTML]{0C56F7}-5                  & \cellcolor[HTML]{1B61F8}-3                  & \cellcolor[HTML]{1B61F8}-3                  & \cellcolor[HTML]{A3BBED}-1                  & 0                                           & 0                   & 0                   & \cellcolor[HTML]{7DA2F1}-2                  & 0                                           & 0                   & 7                                           & \cellcolor[HTML]{F8ADAD}2                   & \cellcolor[HTML]{0350F8}-20                 & 0                                           \\
                                                 & Behavior                                      & -                                           & \cellcolor[HTML]{0350F8}haut                & \cellcolor[HTML]{0350F8}haut                & -                                           & \cellcolor[HTML]{0350F8}bas                 & -        & -        & -                   & \cellcolor[HTML]{0C56F7}bas                 & \cellcolor[HTML]{1B61F8}bas                 & \cellcolor[HTML]{1B61F8}bas                 & \cellcolor[HTML]{A3BBED}bas                 & -                                           & -                   & -                   & \cellcolor[HTML]{7DA2F1}bas                 & -                                           & -                   & haut                                        & \cellcolor[HTML]{F8ADAD}haut                & \cellcolor[HTML]{0350F8}bas                 & -                                           \\
\multirow{-3}{*}{}                         & P-value ($\delta$)                            & 1 (N)                                       & \cellcolor[HTML]{0350F8}\textless{}0.05 (S) & \cellcolor[HTML]{0350F8}\textless{}0.05 (S) & 1 (N)                                       & \cellcolor[HTML]{0350F8}\textless{}0.05 (M) & 0.07 (N) & 0.07 (N) & \textless{}0.05 (N) & \cellcolor[HTML]{0C56F7}\textless{}0.05(S)  & \cellcolor[HTML]{1B61F8}\textless{}0.05 (S) & \cellcolor[HTML]{1B61F8}\textless{}0.05(S)  & \cellcolor[HTML]{A3BBED}\textless{}0.05(N)  & 0.08 (N)                                    & 0.23 (S)            & 0.19 (N)            & \cellcolor[HTML]{7DA2F1}0.06 (N)            & 0.16 (N)                                    & \textless{}0.05 (N) & 0.09 (S)                                    & \cellcolor[HTML]{F8ADAD}\textless{}0.05 (N) & \cellcolor[HTML]{0350F8}\textless{}0.05 (M) & 0.12 (N)                                    \\
 \bottomrule
\end{tabular}
}
\end{table*}
%\end{sidewaystable}

\end{comment}

\subsection{RQ-2 How do different configurations affect the effectiveness of LLMs?}
\label{sec:rq2}


\noindent 
\textbf{Impact of different example numbers.}
As previous studies~\cite{brown2020language,A3CodGen} have shown, the number of examples provided has a significant impact on LLMs' performance. 
To explore this, we adjust the number of examples while keeping other parameters and hyperparameters constant to ensure a fair comparison.
We do not conduct experiments in a zero-shot setting, as LLMs may generate unnormalized outputs without a prompt template, which would hinder automated extraction. 
From Fig.~\ref{fig:ablation}, we observe that as the number of examples increases, both the average token length and time cost rise sharply, while the improvement in Pass@k remains modest.
Based on these findings, we perform our ablation studies (Table~\ref{tab:rq1} and \ref{tab:ablation}) using a one-shot setting in \mytitle.


\noindent 
\textbf{Impact of different selection strategies.}
% Our case study reveals that the RAG method improves the performance of LLMs.
RAG retrieves relevant codes from a retrieval database and supplements this information for code generation~\cite{parvez2021retrieval}. 
To ensure a fair comparison, we set the number of examples to one and evaluated the results of RAG versus random selection on the same LLM (i.e., DeepSeek-V3). From Table~\ref{tab:ablation}, Pass@1 and Compile@1 are higher when RAG is enabled, indicating that it improves the effectiveness of code generation.


\begin{figure}[htbp]
    \centering
    \includegraphics[width=\linewidth]{figs/ablation.pdf}
    \caption{Performance of Qwen2.5-Coder-7B. The x-axis represents the number of shots.}
    \label{fig:ablation}
\end{figure}
\vspace{-0.2cm}

\noindent 
\textbf{Impact of Context Information.}
Since that relevant context typically enhances performance in other programming languages, we conduct an ablation study to examine the influence of context on the quality of LLM-generated contracts. Table~\ref{tab:ablation} shows that providing context information improves both Pass@1 and Compile@1. 
However, there is no clear correlation between gas fees, vulnerability rate, and the presence of context information.


% \vspace{-0.1cm}
\begin{table}[htbp]
    \centering
    \caption{Ablation study on the effect of RAG and Context on DeepSeek-V3's (one-shot) performance.}
    \resizebox{\linewidth}{!}
    {
        \begin{tabular}{cc|cccc}
        \toprule
        RAG & Context & Pass@1 & Compile@1 & Fee & Vul \\
        \midrule
        \ding{51} & \ding{51} & \textbf{21.72\%}& \textbf{53.35\%}&  \textbf{-7525}& 26.61\% \\ 
        \ding{55} & \ding{51} & 20.24\% & 51.08\% & 3828& \textbf{23.68\%}\\ 
        \ding{51} & \ding{55} & 21.28\% & 52.54\% & -708& 26.13\%\\
        \ding{55} & \ding{55} & 20.17\% & 50.32\% & 768& 26.83\%\\   
        \bottomrule
        \end{tabular}
    }
    \label{tab:ablation}
\end{table}
% \vspace{-0.4cm}

\subsection{RQ-3 Gas Efficiency and Scalability Analysis}
\label{sec:rq3}


\noindent
\textbf{Objective.}
In \Chain, we take \textit{Gas Fee} and \textit{Scalability} into consideration.
In blockchain query databases, the gas fee and scaleability are essential since the former will largely impact the practicality of the designed methods (i.e., a higher gas fee means more money spent when running queries) and the latter will impact compatibility (i.e., poor compatibility will lead to massive code modification when adjusting to another blockchain system).

\noindent
\textbf{Experimental Design.}
First, we investigate the impact of different data structures on gas fee and design four variants of \Chain. 
vChain+$_{F}$ represents vChain+ replicated and enhanced to support multimodal queries.
vChain+$_{O}$ is vChain+$_{F}$ without the off-chain query module.
MulChain$_{BT}$ is MulChain with its underlying data structure replaced with B\(+\)Tree for time range queries.
MulChain$_{BH}$ is MulChain with its underlying data structure replaced with our gas-efficient BHashTree for time range queries.
MulChain$_{T}$ is MulChain with its underlying data structure replaced with our verifiable trie for fuzzy queries.
This approach allows us to examine the individual effects of each component.


\noindent
\textbf{Results.} We discuss the results from the aspects of gas consumption and scalability, respectively.




\noindent
\textbf{\underline{Gas Consumption Analysis.}}
The average gas fees for BHashTree are much lower than those of vChain+$_{F}$, slightly lower than B\(+\)Tree-based methods, as illustrated in Fig.~\ref{fig:Gas Consumption}(a).
Fig.~\ref{fig:Gas Consumption}(b) presents the average gas fees of the trie in comparison to the accumulator from vChain+. 
Notably, the gas consumption of MulChain$_{T}$ exceeds that of vChain+$_{F}$ due to our strategic trade-off of space for time. 
We deem this trade-off acceptable, as the reduction in query latency is particularly valuable in the context of fuzzy queries on blockchains.




\begin{figure}[htbp]
    \centering
    \includegraphics[width=.7\linewidth]{figures/Gas_BHT_BT.pdf}
    % \captionsetup{skip=0pt}
    \caption{Gas Consumption}
    \label{fig:Gas Consumption}
\end{figure}


\noindent
\textbf{\underline{Scalability Analysis.}}
\Chain supports all six SQL primitives (i.e., insert, delete, update, simple, time range, fuzzy queries) on Ethereum and FISCO BCOS. 
In contrast, the CRUD Service of FISCO BCOS does not support time range and fuzzy queries. 
We test \Chain using time range queries on BTC and ETH datasets. 
From Fig.~\ref{fig:FISCO BCOS}(a), we can see that \Chain undergoes a decline of up to 3.78\% when the number of blocks grows.
In Fig.~\ref{fig:FISCO BCOS}(b), we observe that MulChain$_{BT}$ is faster on the BTC dataset than on ETH for block counts below 128 and above 1024. 
This performance difference is due to the varying timestamp densities of the two datasets and the initialization cost of the B\(+\)Tree.


\begin{figure}[htbp]
    \centering
    \includegraphics[width=.7\linewidth]{figures/FISCO_all.pdf}
    % \captionsetup{skip=0pt}
    \caption{Query Performance on FISCO BCOS}
    \label{fig:FISCO BCOS}
\end{figure}


\intuition{
{\bf Answer to RQ-3}: 
(1) The five data structures (i.e., accumulator of vChain+$_{F}$, vChain+$_{O}$, MulChain$_{BT}$, MulChain$_{BH}$ and MulChain$_T$) contribute substantially to \Chain, and combining them achieves the best performance of blockchain query on different scenarios.
(2) The gas fee of MulChain$_{T}$ exceeds that of vChain+$_{F}$ due to our strategic trade-off of space for time.
(3) \Chain supports blockchains based on Ethereum virtual machine and Hyperledger Fabric.
}


\begin{table*}[t!]
    \centering
    \tabcolsep=1.5pt
    \renewcommand{\arraystretch}{0.92} 
    \caption{Experimental results of RQ6}
    \begin{tabular}{ccccc}
    \hline
    \multicolumn{1}{c|}{\textbf{Attempt Times}}   & \multicolumn{1}{c|}{\textbf{First Attempt}} & \multicolumn{1}{c|}{\textbf{Second Attempt}} & \multicolumn{1}{c|}{\textbf{Third Attempt}} & \textbf{More-time Attempt} \\ \hline
    \multicolumn{1}{c|}{\textbf{IllusionCAPTCHA}} & 86.95\%                                     & 8.69\%                                       & 0.00\%                                      & 4.34\%                     \\ \hline
    \multicolumn{1}{l}{}                          & \multicolumn{1}{l}{}                        & \multicolumn{1}{l}{}                         & \multicolumn{1}{l}{}                        & \multicolumn{1}{l}{}      
    \end{tabular}
\label{tex:RQ4}
\end{table*}



\section{Discussion}\label{section6}
\section{Discussion and Conclusion}
\label{sec:discussion}


\textbf{Conclusion.} In this paper, we propose LRM to utilize diffusion models for step-level reward modeling, based on the insights that diffusion models possess text-image alignment abilities and can perceive noisy latent images across different timesteps. To facilitate the training of LRM, the MPCF strategy is introduced to address the inconsistent preference issue in LRM's training data. We further propose LPO, a method that employs LRM for step-level preference optimization, operating entirely within the latent space. LPO not only significantly reduces training time but also delivers remarkable performance improvements across various evaluation dimensions, highlighting the effectiveness of employing the diffusion model itself to guide its preference optimization. We hope our findings can open new avenues for research in preference optimization for diffusion models and contribute to advancing the field of visual generation.

\textbf{Limitations and Future Work.} (1) The experiments in this work are conducted on UNet-based models and the DDPM scheduling method. Further research is needed to adapt these findings to larger DiT-based models \cite{sd3} and flow matching methods \cite{flow_match}. (2) The Pick-a-Pic dataset mainly contains images generated by SD1.5 and SDXL, which generally exhibit low image quality. Introducing higher-quality images is expected to enhance the generalization of the LRM. (3) As a step-level reward model, the LRM can be easily applied to reward fine-tuning methods \cite{alignprop, draft}, avoiding lengthy inference chain backpropagation and significantly accelerating the training speed. (4) The LRM can also extend the best-of-N approach to a step-level version, enabling exploration and selection at each step of image generation, thereby achieving inference-time optimization similar to GPT-o1 \cite{gpt_o1}.



\section{Related Work}\label{section7}

\section{Related Work}
\label{sec:related_work}
\vspace{-1.5mm}
\noindent\paragraph{LLM Serving.} 
Orca~\cite{orca} introduces iteration-level scheduling that batches the LLM requests at the token level. vLLM~\cite{vllm} proposes PageAttention~\cite{vllm}
\vspace{-0.5mm}
\noindent\paragraph{LLM Serving Scheduler.} 



% \noindent\paragraph{Job Scheduling.} Cluster managers implement schedulers to orchestrate jobs and tasks over a fixed set of resources with a scheduling policy. First, a cluster scheduler decides when to run a job via a queueing policy, such as First Come First Serve (FCFS) \cite{yarn, spark, mesos}, Shortest Job First (SJF) \cite{alibaba}, Least Attained Service (LAS) \cite{tiresias}, or general priority-based policies \cite{kubernetes, slurm}. Second, a scheduler determines the placement strategy (i.e. which node) via a bin-packing policy, such as First-Fit, Best-fit \cite{slurm}, Worst-fit \cite{kubernetes, borg}. Bin-packing can also occur over single or multiple resource types \cite{drf, tetris, synergy, allox, optimus}.  Finally, schedulers may optimize for various objectives such as job completion time (JCT) \cite{tiresias, synergy}, throughput \cite{gavel, gandiva}, fairness \cite{drf, delay}, and makespan \cite{gavel, synergy}.
% \vspace{-0.5mm}
% \noindent\paragraph{Cluster autoscaling.} To orchestrate resources, cluster managers also implement cluster autoscalers to dynamically adjust cluster size via provisioned cloud instances. The default Kubernetes cluster autoscaler immediately scales cloud instances when tasks (i.e. pods) are pending due to insufficient cluster resources\cite{kubernetes}. Various cloud services, including managed instance groups ~\cite{awsautoscale,gcpautoscale} and custom Kubernetes solutions ~\cite{gke, karpenter}, scale by relying on rule-based mechanisms (e.g. CPU utilization $\geq$ $60\%$) and/or prediction-based methods to predict future bursts. Autoscalers may also optimize instance costs by right-sizing cloud instances ~\cite{kubecost} or managing multiple node pools~\cite{canap}. Other studies~\cite{seagull, sc11autoscaling} focus on cloud bursting for on-premise jobs, emphasizing on cost-efficient data migration and meeting deadlines.
% \vspace{-0.5mm}

% Orthogonally, there exists a large body of work on horizontal and vertical autoscaling, primarily focusing on resizing the number of tasks and task resource requirements per job. Works like Autopilot~\cite{rzadca2020autopilot}, Sinan~\cite{zhang2021sinan}, DS2~\cite{kalavri2018three}, and FIRM~\cite{qiu2020firm} either use direct feedback from workloads to maintain SLOs (e.g. latency or OOM rates) or use a model to predict resource needs. These approaches, which resubmit modified tasks to the scheduler and do not provision compute, are complementary to our work.

% In particular, the constant waiting policy generalizes cluster autoscalers ($C=0$). Existing auto-scaling policies schedule jobs on the cluster when available, and otherwise immediately provisions resources in the cloud when the cluster is full \cite{docker, kubernetes, slurm}.


% Other studies explore cloud bursting for on-premise jobs. For instance, Seagull \cite{seagull} performs cloud bursting for enterprise applications by calculating the cost-benefit tradeoff of each workload and opportunistically migrating VM state to the cloud. Likewise, \cite{sc11autoscaling} schedules workflows on the most cost-efficient instances to minimize cost while meeting deadlines. These instance selection methods from these studies could further enhance Starburst's cost savings.


% \section{Related Work}
% \label{sec:related_work}

% \noindent\paragraph{Waiting Policies.} Our work's closest neighbor is the Waiting Game \cite{waiting_game}, which introduces the concept of a waiting policy and investigates the optimal reserved cluster size that balances costs between reserved and on-demand instances. This research examines various selective and non-selective waiting policies, juggling between cost, reserved cluster provisioning, and Job Completion Time (JCT). One of their proposals, the Constant Wait policy (\S\ref{sec:naive_constant_waiting}), serves as one of our baselines. Our selective waiting policies shift CPU-jobs and jobs in extended queues to the cloud, which is different from the Waiting Game's strategy of moving short-run jobs to the cloud. As we've found that short jobs can account for 10-25\% of total costs, this approach doesn't suit our usecase. Additionally, they propose a balking waiting policy, which necessitates predicting job waiting times. To address this complex task, they utilized a DNN model to predict job waiting times \cite{good_wait}.

% \paragraph{Cluster Autoscaling.}

% There is a large body of work on autoscaling, primarily focusing on resizing workloads to utilize more resources. Works like Autopilot~\cite{rzadca2020autopilot}, Sinan~\cite{zhang2021sinan}, DS2~\cite{kalavri2018three}, and FIRM~\cite{qiu2020firm} either use direct feedback from workloads to maintain SLOs (e.g., latency or OOM rates) or use a model to predict resource needs. These approaches, which aim to prevent over or under-provisioning of the cluster, is orthagonal to our work.

% Other studies explore cloud bursting for on-premise jobs. For instance, Seagull \cite{seagull} performs cloud bursting for enterprise applications by calculating the cost-benefit tradeoff of each workload and opportunistically migrating VM state to the cloud. In contrast, Starburst uses a waiting mechanism to balance cost and job completion time. Likewise, \cite{sc11autoscaling} schedules workflows on the most cost-efficient instances to minimize cost while meeting deadlines. These instance selection methods from these studies could further enhance Starburst's cost savings.

% % \paragraph{DNN Scheduling}
% % Deployment of Deep Learning workloads has given rise to many different schedulers. Gandiva~\cite{gandiva} introduces scheduling mechanisms such as time-slicing, migration and bin-packing to maximize cluster utilization. Some works, such as Optimus~\cite{optimus}, assume job runtime is predictable and allocate resources to minimize JCT, while others, such as Tiresias~\cite{tiresias}, operate under partial job runtime information to minimize the average JCT by discretizing priorities in the job queue. Gavel~\cite{gavel} maximizes the goodput of jobs in a heterogeneous cluster by considering the relative performance differences between different resource types. Starburst \textcolor{red}{TODO - How is DNN scheduling related here?}



\section{Conclusion and Future Work}\label{section8}
\section{Conclusion}
We introduced \Bench, the first ever IMTS forecasting benchmark.
\Bench's datasets are created with ODE models, that were defined in decades of research and published on
the Physiome Model Repository. Our experiments showed that LinODEnet and CRU are actually
better than previous evaluation on established datasets indicated. Nevertheless,
we also provided a few datasets, on which models are unable to outperform a
constant baseline model. We believe that our datasets, especially the very difficult ones,
can help to identify deficits of current architectures and support future research on
IMTS forecasting.




\balance
\bibliographystyle{acm}
\bibliography{main}

\end{document}
\endinput
%%
%% End of file `sample-manuscript.tex'.