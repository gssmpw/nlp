\documentclass[acmsmall,screen]{acmart}
\usepackage[shortlabels]{enumitem}
\usepackage{fancybox, graphicx}
\usepackage{color}
\usepackage{listings}
\usepackage{tcolorbox}
\usepackage{balance}
\usepackage{tabularx}
\usepackage{multirow}
\usepackage{xspace}
\usepackage{tikz}
\usepackage{textcomp}
\usepackage{booktabs}
\usetikzlibrary{patterns}
\usepackage{url}


\definecolor{darkblue}{rgb}{0.0, 0.0, 0.55}  

\usepackage[normalem]{ulem}
\usepackage[utf8]{inputenc}
\usepackage{amsmath}  
\usepackage{graphicx}
\usepackage{comment}
\usepackage{multirow}
\usepackage{enumitem}
\usepackage{array}
\usepackage{pifont}
\usepackage{xcolor}
\usepackage{balance}
\usepackage{mathtools}
\usepackage{xspace}
\usepackage{url}
\usepackage{eucal}
\usepackage{amsfonts}
\usepackage{color}
\usepackage{booktabs}
\usepackage{bbding}
\usepackage{float}

\usepackage{graphicx}
\usepackage{hyperref}
\usepackage{url}
\usepackage{multirow}
\usepackage{comment}
\usepackage{colortbl}
\usepackage{arydshln}
\usepackage{graphicx}
\usepackage{subcaption}
\usepackage[linewidth=0.1pt]{mdframed}
\usepackage[linesnumbered,ruled,lined]{algorithm2e}


\usepackage{wrapfig,lipsum,booktabs}
\usepackage{subfloat}
\usepackage[export]{adjustbox}% 

\usepackage{graphicx}
\usepackage{adjustbox}
\usepackage{forest}
\usepackage{tikz}
\usepackage{pifont}


\usepackage{hyperref}
\usepackage{hyperxmp}

    
\AtBeginDocument{%
  \providecommand\BibTeX{{%
    \normalfont B\kern-0.5em{\scshape i\kern-0.25em b}\kern-0.8em\TeX}}}

\newcommand{\basicalerter}[2]{\fbox{\bfseries\sffamily\scriptsize\color{blue} #1}{\sf\small$\blacktriangleright$\textit{\color{blue} #2}$\blacktriangleleft$}}

\newcommand{\mrevised}[1]{\textcolor{red}{#1}}
\newcommand{\revised}[1]{\textcolor{black}{#1}}
\newcommand{\checked}[1]{\textcolor{blue}{#1}}

\newcommand{\todoorange}[1]{\textcolor{orange}{\textbf{[[#1]]}}}
\newcommand{\added}[1]{\textcolor{black}{#1}}


\pdfpagewidth=\paperwidth
\pdfpageheight=\paperheight

\makeatletter
\newcommand{\showfontsize}{\f@size{} pt}
\newcommand\usemm[1]{%
  \strip@pt\dimexpr0.3514598\dimexpr #1\relax\relax mm%
}
\newcommand\usein[1]{%
  \strip@pt\dimexpr0.013837\dimexpr #1\relax\relax in%
}
\makeatother

\usepackage{graphicx}
\usepackage{hyperref}
\usepackage{url}
\usepackage{multirow}
\usepackage{comment}
\usepackage{colortbl}
\usepackage{graphicx}
\usepackage{tcolorbox}

\usepackage{algpseudocode}  
\usepackage{amsmath}  
\usepackage{caption} 
\usepackage{algpseudocode}
\usepackage{float}
\usepackage{soul}



\usepackage{xcolor}
\usepackage{enumitem}

\usepackage{amsmath}
\newcommand{\basicalert}[2]{\fbox{\bfseries\sffamily\scriptsize\color{black} #1}{\sf\small$\blacktriangleright$\textit{\color{red} #2}$\blacktriangleleft$}}


\newcommand{\xin}[1]{\basicalert{From Xin}{#1}}




\definecolor{dark-red}{RGB}{255,0,0}
\definecolor{dark-green}{RGB}{0,200,0}
\definecolor{lightgreen}{rgb}{0.56, 0.93, 0.56} % 定义浅绿色
\definecolor{lightred}{rgb}{1.0, 0.7, 0.7}

\lstdefinelanguage{java-pretty}
{
  language=java,
  numbers=left,
  frame=shadowbox,
  rulesepcolor= \color{red!20!green!20!blue!20},
  basicstyle=\footnotesize\ttfamily,
  numberstyle=\scriptsize,
  breaklines=true,
  columns=fullflexible,
  xleftmargin=16pt,
  showstringspaces=false,
  keywordstyle=\color{blue}\bfseries,
  stringstyle=\color{javared},
  commentstyle=\color{javagreen},
  morecomment=[s][\color{javadocblue}]{/**}{*/},
}
\colorlet{punct}{red!60!black}
\definecolor{background}{HTML}{EEEEEE}
\definecolor{delim}{RGB}{20,105,176}
\colorlet{numb}{magenta!60!black}
\lstdefinelanguage{json}{
    basicstyle=\normalfont\ttfamily,
    numbers=left,
    numberstyle=\scriptsize,
    stepnumber=1,
    numbersep=8pt,
    showstringspaces=false,
    breaklines=true,
    frame=lines,
    literate=
     *{:}{{{\color{punct}{:}}}}{1}
      {,}{{{\color{punct}{,}}}}{1}
      {\{}{{{\color{delim}{\{}}}}{1}
      {\}}{{{\color{delim}{\}}}}}{1}
      {[}{{{\color{delim}{[}}}}{1}
      {]}{{{\color{delim}{]}}}}{1},
}

\lstdefinelanguage{json-pretty}
{
  language=json,
  numbers=left,
  frame=shadowbox,
  rulesepcolor= \color{red!20!green!20!blue!20},
  basicstyle=\footnotesize\ttfamily,
  numberstyle=\scriptsize,
  breaklines=true,
  columns=fullflexible,
  xleftmargin=16pt,
  showstringspaces=false,
  keywordstyle=\color{blue}\bfseries,
  stringstyle=\color{javared},
  commentstyle=\color{javagreen},
  morecomment=[s][\color{javadocblue}]{/**}{*/},
}

\lstdefinelanguage{diff}{
    basicstyle=\ttfamily\small,
    morecomment=[f][\color{diffstart}]{@@},
    morecomment=[f][\color{javagreen}]{+\ },
    morecomment=[f][\color{javared}]{-\ },
  }

\def\diffinclcolor{\color{javagreen}}
\def\diffremcolor{\color{javared}}
  
\newcommand{\InputWithSpace}[1]{\bgroup\def\arraystretch{1.15}\input{#1}\egroup}


\settopmatter{printacmref=false}

\setcopyright{acmcopyright}
\copyrightyear{2024}
\acmYear{2024}
\acmDOI{XXXXXXX.XXXXXXX}


\begin{document}

\title{LessLeak-Bench: A First Investigation of Data Leakage in LLMs Across 83 Software Engineering Benchmarks}

\author{Xin Zhou}
\affiliation{%
  \institution{Singapore Management University}
  \country{Singapore}
}
\email{xinzhou.2020@phdcs.smu.edu.sg}


\author{Martin Weyssow}
\affiliation{%
  \institution{Singapore Management University}
  \country{Singapore}
}
\email{mweyssow@smu.edu.sg}

\author{Ratnadira WIDYASARI}
\affiliation{%
  \institution{Singapore Management University}
  \country{Singapore}
}
\email{ratnadiraw.2020@phdcs.smu.edu.sg}

\author{Ting Zhang}
\affiliation{%
  \institution{Singapore Management University}
  \country{Singapore}
}
\email{tingzhang.2019@phdcs.smu.edu.sg}

\author{Junda He}
\affiliation{%
  \institution{Singapore Management University}
  \country{Singapore}
}
\email{jundahe@smu.edu.sg}


\author{Yunbo Lyu}
\affiliation{%
  \institution{Singapore Management University}
  \country{Singapore}
}
\email{yunbolyu@smu.edu.sg}



\author{Jianming Chang}
\email{jianmingchang@seu.edu.cn}
\affiliation{%
  \institution{Southeast University}
  \city{Nanjing}
  \state{JiangSu}
  \country{China}
}


\author{Beiqi Zhang}
\email{zhangbeiqi@whu.edu.cn}
\affiliation{%
  \institution{Wuhan University}
  \city{Wuhan}
  \country{China}
}

\author{Dan Huang}
\affiliation{%
  \institution{Singapore Management University}
  \country{Singapore}
}
\email{dan.huang.2024@phdcs.smu.edu.sg}



\author{David Lo}
\affiliation{%
  \institution{Singapore Management University}
  \country{Singapore}
}
\email{davidlo@smu.edu.sg}






\renewcommand{\shortauthors}{Zhou et al.}




\begin{abstract}
Large Language Models (LLMs) are widely utilized in software engineering (SE) tasks, such as code generation and automated program repair. However, their reliance on extensive and often undisclosed pre-training datasets raises significant concerns about data leakage, where the evaluation benchmark data is unintentionally ``seen'' by LLMs during the model's construction phase.
The data leakage issue could largely undermine the validity of LLM-based research and evaluations. 
Despite the increasing use of LLMs in the SE community, there is no comprehensive study that assesses the extent of data leakage in SE benchmarks for LLMs yet.
To address this gap, this paper presents the first large-scale analysis of data leakage in 83 SE benchmarks concerning LLMs. 
We systematically investigated whether, and to what extent, popular SE benchmark datasets were included in a LLM's pre-training data. 
Our approach involved using an efficient near-duplicate data detection algorithm, MinHash+LSH, to identify potential duplicate pairs between the SE benchmarks and LLM's pre-training dataset. Subsequently, we conducted extensive manual labeling on these potential duplicates to identify true duplicates.
Those true duplicates reveal and confirm the data leakage of SE benchmarks. 
Our results show that in general, data leakage in SE benchmarks is minimal, with average leakage ratios of only 4.8\%, 2.8\%, and 0.7\% for Python, Java, and C/C++ benchmarks, respectively. However, some benchmarks exhibit relatively higher leakage ratios, which raises concerns about their bias in evaluation. For instance, QuixBugs and BigCloneBench have leakage ratios of 100.0\% and 55.7\%, respectively.
Furthermore, we observe that data leakage has a substantial impact on LLM evaluation. On the APPS benchmark, StarCoder-7B achieves a Pass@1 score that is 4.9 times higher on leaked samples than on non-leaked samples, highlighting how leaked benchmark data can lead to inflated metrics.
We also identify key causes of high data leakage, such as the direct inclusion of benchmark data in pre-training datasets and the use of coding platforms like LeetCode for benchmark construction.
To address the data leakage, we introduce \textbf{LessLeak-Bench}, a new benchmark that removes leaked samples from the 83 SE benchmarks, enabling more reliable LLM evaluations in future research.
Our study enhances the understanding of data leakage in SE benchmarks and provides valuable insights for future research involving LLMs in SE.



\end{abstract}

\maketitle

\newcommand{\XSpace}[1]{}

\makeatletter
\newenvironment{btHighlight}[1][]
{\begingroup\tikzset{bt@Highlight@par/.style={#1}}\begin{lrbox}{\@tempboxa}}
{\end{lrbox}\bt@HL@box[bt@Highlight@par]{\@tempboxa}\endgroup}

\newcommand\btHL[1][]{%
  \begin{btHighlight}[#1]\bgroup\aftergroup\bt@HL@endenv%
}
\def\bt@HL@endenv{%
  \end{btHighlight}%
  \egroup
}
\newcommand{\bt@HL@box}[2][]{%
  \tikz[#1]{%
    \pgfpathrectangle{\pgfpoint{1pt}{0pt}}{\pgfpoint{\wd #2}{\ht #2}}%
    \pgfusepath{use as bounding box}%
    \node[anchor=base west, fill=yellow!30,outer sep=0pt,inner xsep=1pt, inner ysep=0pt, rounded corners=0pt, minimum height=\ht\strutbox+1pt,#1]{\raisebox{1pt}{\strut}\strut\usebox{#2}};
  }%
}
\makeatother
\lstdefinestyle{Highlight}{
    moredelim=**[is][\btHL]{`}{`},
    moredelim=**[is][{\btHL[fill=orange!50]}]{´}{´},
    moredelim=**[is][{\btHL[fill=red!50]}]{@}{@},
}


\newcommand{\wholedata}{{\textit{complete data}}}
\newcommand{\singledata}{{\textit{single-line data}}}



\section{Introduction}
\begin{figure}[ht]
    \centering
    \includegraphics[width=0.8\linewidth]{graphs/greater_than_naive.pdf}
    \vspace{0.5cm}
    \includegraphics[width=0.8\linewidth]{graphs/p1_bottom.png}
    \vspace{-5pt}
    \caption{\textcolor{positional}{Positional} vs.\ \textcolor{nonpositional}{non-positional} circuits. In a \textcolor{nonpositional}{non-positional} circuit, the same edges must be included at all positions. A \textcolor{positional}{positional} circuit can distinguish between the same edge at different positions. This specificity yields better trade-offs between circuit size and faithfulness. It can also increase both precision and recall.}
    \label{fig:p1}
    \vspace{-5pt}
\end{figure}

\section{Introduction}

\looseness=-1
A primary goal of interpretability research is to characterize the internal mechanisms in language models (LMs) and other NLP models. 
A core approach in this area is \textbf{circuit discovery}---identifying the minimal subgraph within the model's computation graph that performs a specific task \citep{olah2021framework,olah-mech}.
Typically, the nodes of a circuit represent model components (e.g., attention heads, neurons, or layers).
While manual circuit discovery methods can yield position-specific insights \citep{wanginterpretability,goldowskydill2023localizingmodelbehaviorpath}, \emph{automatic methods often overlook positional information}, treating components as uniformly relevant across all input token positions \citep{conmytowards,syed2023attribution}. 
For instance, if an attention head is included in a circuit, it is assumed to contribute equally to the computation for every position in the input sequence.
The assumption that circuits are position-invariant ignores the fact that different positions often require distinct computations.
By ignoring positions, current methods limit their ability to capture mechanisms that operate across positions, such as interactions between attention heads across positions.

In this study, we start by demonstrating that positional agnosticism is a significant limitation (\S\ref{sec:motivating}). Then, to address these limitations, we introduce a new approach: position-aware edge attribution patching (PEAP; \S\ref{sec:full_circ_discovery}; Figure~\ref{fig:p1}). Current approaches  assume that if an edge is in a circuit, then the same edge will be in the circuit at all positions, thus leading to low precision. It is also assumed that an edge's importance should be aggregated across positions before deciding whether it should be included in the circuit; this can lead to cancellation effects, and thus low recall. PEAP instead allows us to compute the importance of cross-positional edges, and separately evaluates edge importance at each position. We show that this leads to smaller and more accurate circuits; see Figure~\ref{fig:p1}.

Incorporating positional information into circuit discovery is straightforward when inputs have the same length and structure across examples.

However, realistic datasets are not nearly this templatic.
How, then, can we incorporate positional information into automatic circuit discovery?
To address this challenge, we propose \textbf{schemas} (\S\ref{sec:schema}). 
Schemas assign semantic labels to spans of tokens, enabling information aggregation across examples even when the spans differ in length.

For example, in the input ``The \textcolor{positional}{war} lasted from 1453 to 14\underline{\hspace{1em}},'' the span ``\textcolor{positional}{war}'' could be labeled as ``\emph{Subject}''.
This enables handling spans with varying lengths: the phrase ``\textcolor{positional}{Black Plague}'' in another example can be treated as a single positional span with the same role as ``\textcolor{positional}{war}''.
In experiments with two LMs and three tasks, we find that circuits discovered using schemas achieve a better trade-off between circuit size and faithfulness to the model's behavior than position-agnostic circuits.
Importantly, position-aware circuits offer a more precise representation of the underlying mechanisms, providing a more concise foundation for mechanistic explanations.

We also present a fully automated pipeline for schema generation and application (\S\ref{sec:schema-generation}) using large language models (LLMs). 
We evaluate the quality of the generated schemas and their utility in discovering position-aware circuits (\S\ref{sec:schema-eval}).
Notably, circuits derived using automatically generated and applied schemas achieve comparable faithfulness scores to circuits discovered with human-designed and manually applied schemas.

We summarize our contributions as follows:
\begin{itemize}[noitemsep,leftmargin=*,topsep=1pt,parsep=1pt]
    \item Introduce a position-aware circuit discovery method, which obtains better faithfulness than position-agnostic discovery.  
    \item Introduce dataset schemas,  facilitating positional circuit discovery in more naturalistic settings. 
    \item Develop an automated schema generation and application pipeline with LLMs, yielding schemas that are comparable to manually-annotated ones.
\end{itemize}



\section{Background}\label{section2}
\section{Background of Cost Estimation} \label{sec:background}
This section first gives a brief overview of classical and learned cost estimation. 
Afterwards, we describe the learning procedure of \lcms and provide a taxonomy that guides our selection of recent \lcms for this study in \Cref{sec:methodology}.

\subsection{Traditional \& Learned Cost Estimation}
\textbf{Traditional Cost Estimation.} 
Precise cost estimates for different plan candidates in a database are crucial for the query optimizer to select optimal plans from a large search space.
Thus, a lot of engineering effort has been spent since the beginning of database development to estimate the execution costs of a query plan.
Most database systems such as MySQL \cite{widenius2002}, Oracle, PostgreSQL, or System R \cite{astrahan1976} use hand-crafted cost models to reason about the execution costs of a query plan.
These models typically provide a cost function for each physical operator in a query plan that estimates its runtime costs according to CPU usage, I/O operations, memory consumption, expected tuples, and random or sequential page accesses.
However, due to the wide variety of data, queries, and data layouts, traditional cost models need to make simplifying assumptions (e.g., independence of attributes).
These often lead to incorrect predictions of the execution cost. 
Consequently, the query optimizer makes sub-optimal decisions that degrade the query performance by increasing its runtime \cite{leis_how_2015}.
%-------------------------------------------------------

\noindent\textbf{Learned Cost Estimation.}
The need to improve prediction accuracy and the rise of machine learning motivated the idea of \lcms. 
The main idea is to approximate the complex cost functions with a learned model.
Generally, a typical model learns from previous query executions to predict execution costs like runtime.
In contrast to traditional cost models, the promise of \lcms is that they can better learn arbitrarily complex functions.
Thus, improved prediction accuracy can be expected in contrast to traditional approaches based on simplifying assumptions.
Overall, the higher accuracy is expected to lead to a selection of query plans with improved query performance.

%-----------------------------------------
\subsection{Learning Procedure of \lcms}
For our study, we look at effects that also result from the learning procedure of \lcms.
As such, we briefly review the traditional procedure as depicted in \Cref{fig:learning_procedure} to provide the necessary background:
\circles{A}~At first, a workload generator is used to create a large set of randomized, synthetic SQL-Strings that involve a variety of representative query properties such as filter predicates, joins, or aggregation types.
\circles{B}~These queries are executedses (e.g., an airline or movie database) to collect the actual costs of queries.
An important aspect here is that training procedures of many \lcms leads to biases in the dataset due to timeouts and pre-optimized queries, as discussed later.
\circles{C}~Next, various information is extracted from the workload execution.
Most importantly, the physical query plans are extracted, which serve as input to cost models.
In addition to physical plans, \lcms require different information, such as data characteristics like histograms or sample bitmaps.
\circles{D}+\circles{E}~Finally, the workload (i.e., plans and runtime) is then split for training and testing the \lcms. 

\begin{figure*}
    \centering
    \includegraphics[width=\linewidth]{./figures/training_procedure.pdf}
    \caption{
    Learning procedure of \lcms. 
    \circles{\textsc{A}} Generation of synthetic training queries. \circles{\textsc{B}} Query execution on training databases. 
    \circles{\textsc{C}} Feature (query plans, data characteristics, and sample bitmaps) and label (query runtimes) extraction to generate the training and test dataset. 
    \circles{\textsc{D}} Training of the \lcm with supervised learning. 
    \circles{\textsc{E}} Evaluation of the \lcm against unseen test data.}
    \label{fig:learning_procedure}
\end{figure*}

%------------------------------------------------------
\subsection{Taxonomy of \lcms} \label{subsec:taxonomy}
\lcms developed in the last years differ in various dimensions.
This section provides a brief taxonomy of recent \lcms to structure the different methodological approaches. 
This taxonomy will guide the selection of \lcms that we use in this study and ensure that we cover the different methodologies to analyze how they affect the ability of \lcms to support query optimization.

\noindent\textbf{Input Features.}
The first crucial dimension is the input features that a \lcm learns from.
The input features are extracted from the executed workloads (cf. \Cref{fig:learning_procedure}\circles{C}).
The query plan and the underlying data distribution need to be modeled so that a \lcm is informed to make reasonable predictions about the execution costs, which in turn affects query optimization, as we will show.
However, \lcms make use of different information for cost estimation.

\begin{enumerate}[leftmargin=*, nosep]
\item \textbf{SQL-String vs. Query Plans}: 
Some of the first models rely on the SQL string to describe a query, as it gives insights about the tables, predicates, and joins. 
However, details of the execution plan, such as physical operators or the order of joins, are not described there. 
Thus, most \lcms utilize the physical query plan, which includes the operators (e.g., scans, joins) and physical operator types  (e.g., nested loop vs. hash join).
As we will see later, this is fundamental for query optimization.

\item \textbf{Cardinalities}:
Intermediate cardinalities are an important input signal for the overall cost of a plan as they denote the number of tuples an operator needs to process \cite{leis_how_2015}.
Thus, many \lcms leverage intermediate cardinalities as input features, which are either annotated by the databases' cardinality estimator or obtained through an additional learned estimator from related work \cite{hilprecht2020deepdb, kipf2019, yang2020}.
While some \lcms also ignore cardinalities as input for cost prediction, we show in our study that they, in fact, improve the usefulness of cost estimates from \lcms for various query optimization tasks.

\item\textbf{Data Distribution}:
Another helpful factor in estimating cost is understanding the data distribution in the base tables, especially if no cardinalities are used.
For instance, the fact of how many distinct values exist in a column might influence the efficiency of physical operators (e.g., hash join). 
As such, some \lcms use data distribution represented as database statistics and histograms or sample bitmaps (which we explain later) from the base tables as inputs.
However, as we will show in our study, their effect on query optimization tasks remains unclear.

\item \textbf{Cost Estimates}: 
Finally, some of the most recent \lcms even leverage the cost estimates provided by a classical cost estimator as an input feature, which serves as a strong input signal.
This idea renders these \lcms to \textit{hybrid} as they combine a traditional cost model with a learned approach.
The study shows that this provides significant benefits.
\end{enumerate}

%-----------------------------------------------------
\noindent \textbf{Query Representation.}
Many \lcms use model architectures use a graph-based representation to encode query plans as input to the models\footnote{The graph-based representation of the queries refers to the fact whether a model leverages the query graph structure and not to the model learning architecture itself.}.
These approaches thus explicitly leverage information about the order (parent-child relationships) of operators in plans.
However, other \lcms \cite{kipf2019, akdere2012} represent a query plan (or the SQL string) as a flat vector of fixed size without modeling the operator dependencies, which we refer to as \textit{flat} representation in this paper.
While intuitively, capturing the structure and not using a flat representation should be beneficial for \lcms, the results of using graph structure in this study are not that clear. 

%-------------------------------------------------------
\noindent \textbf{Database Dependency.}
Furthermore, an important aspect is whether \lcms can generalize to unseen databases (i.e., a new set of tables) or not.
\textit{Database-agnostic} \lcms were designed \cite{hilprecht2022, zibo_liang_dace_2024} to enable cost predictions for unseen databases that were not part of the training data.
This approach has the advantage of directly providing results without requiring database-specific training data. 
In contrast, \textit{database-specific} \cite{sun2019, zhao2022, marcus2019} models cannot generalize for unknown databases.
For this study, an interesting question is if one of these classes is better suited to support query optimization tasks as database-specific can better adapt to one single database while database-agnostic models can generalize better.

%----------------------------------------
\noindent \textbf{Model Architecture.}
Finally, the presented \lcms differ largely in their learning approach.
Various learning architectures were proposed, including decision trees, tree-structured neural networks, neural units, graph neural networks, and transformer architectures.
While different architectures show different results on the cost estimation tasks, it is still open to see which architecture provides the best results for query optimization.
%----------------------------------------



\section{Methodology}\label{section3}
\section{Methodology}
 

\begin{figure*}[t]
    \centering
    \scalebox{0.9}{
    % Define a style for the special labels (for 1, 2, 3)
\begin{tikzpicture}[
    >=Stealth,     % arrow tips
    thick,         % line thickness
    node distance=2.5cm
]

%--- User (person) on the left ---
\node[
  person,
  minimum size=1.25cm,
  label=below:User
] (user) {};

%--- Rectangle in the middle: "1" inside, "operations" below ---
\node[
  draw,
  rectangle,
  minimum width=2.0cm,
  minimum height=1.2cm,
  label=below:operations,
  right=of user
] (box) {\textcircled{1}};

%--- Cylinder (DB) on the right ---
\node[
  draw,
  cylinder,
  shape border rotate=90,
  aspect=1.5,
  minimum height=1.5cm,
  minimum width=1cm,
  label=below:DB,
  right=of box
] (db) {};

%--- Bidirectional arrows with special labels ---
% Arrows between user and box labeled "3"
\draw[->] (user) to[bend left=10] node[above]{\textcircled{3}} (box);
\draw[->] (box)  to[bend left=10]  (user);

% Arrows between box and DB labeled "2"
\draw[->] (box)  to[bend left=10] node[above]{\textcircled{2}} (db);
\draw[->] (db)   to[bend left=10]  (box);

\end{tikzpicture}}
    \caption{An overview of {\sc Demotic} is shown. {\sc Demotic} takes a Verilog instance describing a combinational circuit and parse it into its corresponding probabilistic model described in PyTorch. The embedding layer converts the learnable real-value inputs into probabilities. The $\ell_2$-loss function is calculated in each training iteration and the input variables are updated using GD.}
    \label{fig1}
\end{figure*}


In this section, we describe our differentiable solver/sampler for multi-level digital circuits. While the common approach in solving CircuitSAT typically involves converting the underlying circuit into CNF and employing a SAT solver to find the satisfying solution, we take a completely different approach. Instead, we re-frame the CircuitSAT problem as a multi-output regression task, transforming it into a learning problem. Digital circuits are inherently discrete and non-differentiable. Therefore, we first need to relax the CircuitSAT problem into a continuous form while accurately capturing the structure and behavior of the circuit. To accomplish this, we leverage the probability model of digital gates, as shown in Table \ref{tab1}. This probability model is commonly used in different domains such as stochastic computing \cite{Ardakani2017SC} and dynamic power estimation of digital circuits \cite{harris2010cmos}. We use these probabilities to model each gate in the circuit. The result of such modeling is a differentiable formulation of the underlying circuit that accurately describes its functionality while preserving its spatial structure. Of course, the outcome of this model remains identical to the original circuit in its discrete form for any binary input valuations.




\begin{table}[t]
    \centering
    \caption{Probability model of logic gates.}
    \vspace{-0.25cm}
    \begin{table*}[!th]
\centering
\resizebox{\textwidth}{!}{%
\begin{tabular}{@{}llcccccccccc@{}}
\toprule
& & \multicolumn{2}{c}{\textbf{Intent Detection}} & \multicolumn{2}{c}{\textbf{Topic Mining}} & \multicolumn{2}{c}{\textbf{Domain Discovery}} & \multicolumn{1}{c}{\textbf{Type}} & \multicolumn{1}{c}{\textbf{Emotion}} & \\
\cmidrule(lr){3-4} \cmidrule(lr){5-6} \cmidrule(lr){7-8} \cmidrule(lr){9-9} \cmidrule(lr){10-10}  %\cmidrule(lr){11-11}
\textbf{Model} & \textbf{Method} & \textbf{BANKING} & \textbf{CLINC} & \textbf{Reddit} & \textbf{StackEx} & \textbf{MTOP} & \textbf{CLINC(D)} & \textbf{FewEvent} & \textbf{GoEmotion} & \textbf{AVG} \\ \midrule \midrule
GPT-4o-mini & Standard Prompting & 0.652 & 0.792 & 0.534 & 0.482 & 0.896 & 0.536 & 0.630 & 0.378 & 0.613 \\
& Self-Consistency & 0.666 & 0.802 & 0.586 & 0.494 & 0.902 & 0.530 & 0.640 & 0.382 & 0.625 \\
& TestNUC & 0.712 & 0.858 & 0.614 & 0.528 & 0.936 & 0.544 & 0.674 & 0.410 & 0.660 \\
& \cellcolor{gray!18}TestNUC\textdagger & \cellcolor{gray!18}\textbf{0.764} & \cellcolor{gray!18}\textbf{0.864} & \cellcolor{gray!18}\textbf{0.646} & \cellcolor{gray!18}\textbf{0.540} & \cellcolor{gray!18}\textbf{0.948} & \cellcolor{gray!18}\textbf{0.554} & \cellcolor{gray!18}\textbf{0.680} & \cellcolor{gray!18}\textbf{0.414} & \cellcolor{gray!18}\textbf{0.676} \\ \midrule \midrule
Llama-3.1-8B & Standard Prompting & 0.572 & 0.726 & 0.502 & 0.492 & 0.892 & 0.528 & 0.530 & 0.332 & 0.572 \\
& Self-Consistency & 0.620 & 0.774 & 0.564 & 0.526 & 0.902 & 0.518 & 0.564 & 0.340 & 0.601 \\
& TestNUC & 0.694 & 0.806 & 0.618 & 0.558 & 0.934 & 0.528 & 0.596 & 0.356 & 0.636 \\
& \cellcolor{gray!18}TestNUC\textdagger & \cellcolor{gray!18}\textbf{0.724} & \cellcolor{gray!18}\textbf{0.812} & \cellcolor{gray!18}\textbf{0.646} & \cellcolor{gray!18}\textbf{0.576} & \cellcolor{gray!18}\textbf{0.940} & \cellcolor{gray!18}\textbf{0.542} & \cellcolor{gray!18}\textbf{0.614} & \cellcolor{gray!18}\textbf{0.360} & \cellcolor{gray!18}\textbf{0.652} \\ \midrule \midrule
Claude-3-Haiku & Standard Prompting & 0.680 & 0.848 & 0.486 & 0.564 & 0.892 & 0.552 & 0.594 & 0.336 & 0.619 \\
& Self-Consistency & 0.702 & 0.870 & 0.510 & 0.578 & 0.904 & 0.564 & 0.568 & 0.350 & 0.631 \\
& TestNUC & 0.762 & 0.894 & 0.596 & 0.588 & 0.940 & 0.590 & 0.620 & 0.348 & 0.667 \\
& \cellcolor{gray!18}TestNUC\textdagger & \cellcolor{gray!18}\textbf{0.804} & \cellcolor{gray!18}\textbf{0.902} & \cellcolor{gray!18}\textbf{0.612} & \cellcolor{gray!18}\textbf{0.600} & \cellcolor{gray!18}\textbf{0.946} & \cellcolor{gray!18}\textbf{0.622} & \cellcolor{gray!18}\textbf{0.660} & \cellcolor{gray!18}\textbf{0.368} & \cellcolor{gray!18}\textbf{0.689} \\ \midrule \midrule
GPT-4o & Standard Prompting & 0.746 & 0.924 & 0.712 & 0.674 & 0.962 & 0.614 & 0.682 & 0.406 & 0.715 \\
& Self-Consistency & 0.758 & 0.922 & 0.720 & 0.688 & 0.958 & 0.624 & 0.696 & 0.426 & 0.724 \\
&TestNUC & 0.804 & 0.934 & 0.744 & \textbf{0.710} & 0.974 & 0.644 & 0.692 & 0.446 & 0.744 \\
& \cellcolor{gray!18}TestNUC\textdagger & \cellcolor{gray!18}\textbf{0.824} & \cellcolor{gray!18}\textbf{0.940} & \cellcolor{gray!18}\textbf{0.750} & \cellcolor{gray!18}\textbf{0.710} & \cellcolor{gray!18}\textbf{0.978} & \cellcolor{gray!18}\textbf{0.654} & \cellcolor{gray!18}\textbf{0.708} & \cellcolor{gray!18}\textbf{0.464} & \cellcolor{gray!18}\textbf{0.754} \\
\bottomrule
\end{tabular}%
}
\caption{Accuracy comparison with Standard Prompting and Self-Consistency across four diverse LLMs. TestNUC consistently improves the inference performance on all benchmark datasets. $\dagger$ denotes that 50 neighbors are utilized.}
\label{tab:main_compare_sc}
\end{table*}
    \label{tab1}
    \vspace{-0.25cm}
\end{table}






Given the differentiable model of the circuit obtained by replacing its discrete logic gates with their corresponding probability model, our objective now is to generate a set of inputs that satisfy a desired constraint. This constraint could pertain to any desired valuation of intermediate signals or outputs. To generate satisfying solutions to the CircuitSAT problem, we represent the input variables to the circuit as $\textbf{V} \in \mathbb{R}^{b\times n}$, where $n$ represents the number of variables and $b$ denotes the batch size. We define the matrix $\textbf{V}$ as the parameters of an embedding layer in our circuit model, which will be updated during the learning process. It is worth mentioning that the number of variables in our sampling method is significantly fewer than that of SAT samplers, remaining the same as the number of inputs in the circuit. This discrepancy arises because SAT samplers deal with the CNF of the circuit, where each gate or component introduces additional variables. The embedding layer converts the real-value input variables of the circuit into probabilities in the range from $0$ to $1$ using the sigmoid function $\sigma(\cdot)$, expressed as:
\begin{equation}
    \textbf{P} = \sigma(\textbf{V}) = \dfrac{1}{1 + e^{-\textbf{V}}},
\end{equation}
where $\textbf{P} \in [0, 1]^{b\times n}$ represents the input probabilities to the underlying circuit. The circuit functionality is then computed as:
\begin{equation}
    \textbf{Y} = \mathcal{F}(\textbf{P}),
\end{equation}
where $\mathcal{F}:[0, 1]^{b \times n} \rightarrow [0, 1]^{b \times m}$ denotes the probabilistic model of the circuit. The matrix $\textbf{Y} \in [0, 1]^{b \times m}$ denotes the $m$ outputs across $b$ data batches. The $\ell_2$-loss function $\mathcal{L}$ can be constructed by measuring the distance between $\textbf{Y}$ and the target output valuation matrix $\textbf{T} \in \{0, 1\}^{b \times m}$ as follows:
\begin{equation}
    \mathcal{L} = \sum_{b,m} \left|\left| \textbf{Y} - \textbf{T} \right|\right|^2_2.
\end{equation}
The above loss function can be minimized, and the input variables (i.e., $\textbf{V}$) can be updated using GD in an iterative manner. Upon convergence, the $b$ solutions to the CircuitSAT problem are obtained by converting the soft input values (i.e., $\textbf{V}$) into hard values (i.e., $\widetilde{\textbf{V}} \in \{0, 1\}^{b\times n}$).

Fig. \ref{fig1} illustrates the overview of {\sc Demotic}. {\sc Demotic} is equipped with a parser to covert the circuit described in either bit-blasted Verilog or Berkeley Logic Interchange Format (BLIF) into its corresponding probabilistic model. Consequently, {\sc Demotic} can describe combinational circuits and generate satisfying solutions for any arbitrary constraint on the circuit. Such a sampling paradigm can also benefit from GPU acceleration due to the parallel independent computations across the data batches, enabling a high-throughput sampling procedure. 

To better understand our methodology, let us consider a quantitative example using the module ``c$15$'' shown in Fig. \ref{fig1}. We set the output node $G19$ to $1$ as an output constraint, while the output node $G22$ can take any value of either $0$ or $1$. Therefore, the goal in this example is to find a pair of inputs such that the output node $G19$ is equal to $1$. In this example, the input nodes contributing to our output constraint are $G3$, $G6$, and $G7$. These inputs are learned iteratively using gradient descent. The remaining input nodes, $G1$, $G2$, and $G3$, will not be updated and can take any arbitrary binary values. During each training iteration, each input node is updated by computing the derivative of the loss function with respect to each input node.

To illustrate the process, we generate two samples. In the first step, we randomly assign two values to each input node as follows:
\begin{equation}
    \textbf{v}_{G3} = \begin{bmatrix}
           0.1 \\
           -0.2 
         \end{bmatrix}, \textbf{v}_{G6} = \begin{bmatrix}
           0.5 \\
           -0.4 
         \end{bmatrix}, \textbf{v}_{G7} = \begin{bmatrix}
           -0.7 \\
           -0.8 
         \end{bmatrix},
\end{equation}
where the concatenation of the above vectors forms the matrix $\textbf{V}$. Next, the input probabilities to the circuit are calculated using the sigmoid function:
\begin{equation}
    \textbf{p}_{G3} = \begin{bmatrix}
           0.5250 \\
           0.4502
         \end{bmatrix}, \textbf{p}_{G6} = \begin{bmatrix}
           0.6225 \\
           0.4013
         \end{bmatrix}, \textbf{p}_{G7} = \begin{bmatrix}
           0.3318 \\
           0.3100
         \end{bmatrix}.
\end{equation}
Using the probability model of each gate shown in Table \ref{tab1}, the probabilities of the intermediate node $G11$ and the output node $G19$ are calculated as follows:
\begin{equation}
    \textbf{p}_{G11} = \begin{bmatrix}
           0.4939 \\
           0.4902
         \end{bmatrix}, \textbf{p}_{G19} = \begin{bmatrix}
           0.1639 \\
           0.1520
         \end{bmatrix}.
\end{equation}
Given the target value of 1 for the output node $G19$, the loss is calculated as:
\begin{equation}
    \mathcal{L} = (\textbf{p}_{G19} - 1)^2 = \begin{bmatrix}
           (0.1639 - 1)^2  \\
           (0.1520 - 1)^2 
         \end{bmatrix} = \begin{bmatrix}
           0.6991  \\
           0.7192 
         \end{bmatrix}.
\end{equation}


The above computations are commonly referred to as forward computations. To update the value of the input variables, we need to calculate the derivative of the loss with respect to each input variable, which is referred to as backward computations. This involves using the derivatives of each gate (as shown in Table \ref{tab1}) and applying the chain rule. The process is derived as follows:
\begin{align}
    \dfrac{\partial \mathcal{L}}{\partial \textbf{v}_{G3}} &= \dfrac{\partial \mathcal{L}}{\partial \textbf{p}_{G19}} \dfrac{\partial \textbf{p}_{G19}}{\partial \textbf{p}_{G11}} \dfrac{\partial \textbf{p}_{G11}} {\partial \textbf{p}_{G3}}
    \dfrac{\partial \textbf{p}_{G3}} {\partial \textbf{v}_{G3}} = 2\textbf{p}_{G19} \cdot \textbf{p}_{G7} \cdot (1 - 2\textbf{p}_{G6}) \nonumber \\ 
    &\cdot \sigma(\textbf{v}_{G3})\cdot (1 - \sigma(\textbf{v}_{G3})) = \begin{bmatrix}
           0.0339  \\
           -0.0257 
         \end{bmatrix}, \nonumber 
\end{align}
\begin{align}
    \dfrac{\partial \mathcal{L}}{\partial \textbf{v}_{G6}} &= \dfrac{\partial \mathcal{L}}{\partial \textbf{p}_{G19}} \dfrac{\partial \textbf{p}_{G19}}{\partial \textbf{p}_{G11}} \dfrac{\partial \textbf{p}_{G11}} {\partial \textbf{p}_{G6}}
    \dfrac{\partial \textbf{p}_{G6}} {\partial \textbf{v}_{G6}} = 2\textbf{p}_{G19} \cdot \textbf{p}_{G7} \cdot (1 - 2\textbf{p}_{G3})  \nonumber 
 \\ 
    & \cdot \sigma(\textbf{v}_{G6}) \cdot (1 - \sigma(\textbf{v}_{G6}))  = \begin{bmatrix}
           0.0065   \\
           -0.0126 
         \end{bmatrix}, \nonumber 
\end{align}
\begin{align}
    \dfrac{\partial \mathcal{L}}{\partial \textbf{v}_{G7}} &= \dfrac{\partial \mathcal{L}}{\partial \textbf{p}_{G19}} \dfrac{\partial \textbf{p}_{G19}}{\partial \textbf{p}_{G7}} \dfrac{\partial \textbf{p}_{G7}} {\partial \textbf{v}_{G7}} = 2\textbf{p}_{G19} \cdot \textbf{p}_{G11} \nonumber 
 \\ 
    & \cdot \sigma(\textbf{v}_{G7})\cdot (1 - \sigma(\textbf{v}_{G7})) = \begin{bmatrix}
           -0.1831   \\
           -0.1778 
         \end{bmatrix},
\end{align}
where ``$\cdot$'' denotes element-wise multiplication.


At this point, each variable is updated using the gradient descent update rule. This involves subtracting the derivative of the loss, scaled by the learning rate, from the corresponding input variables. Given a learning rate of $\gamma = 10$, the new values of the input variables at the end of this iteration are obtained as follows:
\begin{align}
    \textbf{v}_{G3} &= \textbf{v}_{G3} - \gamma \dfrac{\partial \mathcal{L}}{\partial \textbf{v}_{G3}} =  \begin{bmatrix}
           -0.2389 \\
           0.0569
         \end{bmatrix}, \textbf{v}_{G6} = \begin{bmatrix}
           0.4349 \\
           -0.2741
         \end{bmatrix}, \nonumber \\ \textbf{v}_{G7} & = \begin{bmatrix}
           1.1311 \\
           0.9783
         \end{bmatrix}.
\end{align}
This process can be repeated multiple times until convergence. However, even after one iteration in this specific example, we obtain two valid and distinct solutions by rounding the input variables to their nearest discrete values after applying the sigmoid function. In this example, the two input pairs of $(v_{G3} = -0.2389, v_{G6} = 0.4349, v_{G7} = 1.1311)$ and $(v_{G3} = 0.0569, v_{G6} = -0.2741, v_{G7} = 0.9783)$ are rounded to $(\widetilde{v}_{G3} = 0, \widetilde{v}_{G6} = 1, \widetilde{v}_{G7} = 1)$ and $(\widetilde{v}_{G3} = 1, \widetilde{v}_{G6} = 0, \widetilde{v}_{G7} = 1)$, respectively. As demonstrated through this example, the forward and backward computations of the two samples are independent of each other. This allows for the parallel execution of the learning process across multiple samples (i.e., batches), enabling GPU acceleration.


% \begin{figure*}[t]
%     \centering
%      \begin{subfigure}[b]{0.6\textwidth}
%          \centering
%          \scalebox{0.9}{\begin{tikzpicture}[auto, node distance=2cm,>=latex']
    % Combinational block 1
    \node [draw, fill = dateblue!30, shape=rectangle, minimum width=2.5cm, minimum height=1.5cm, text width=2cm, align = center, line width=1.5pt] (comb1) {Combinational Circuit ($\mathcal{F}_h$)};
    % Flip-Flop block
    \node [draw, fill = datemagenta!40, shape=rectangle, right = 1.25cm of comb1, align = center, minimum height=2cm, line width=1.5pt] (ff) {Flip-Flops};
    % Combinational block 2
    \node [draw, fill = dateblue!30,shape=rectangle, minimum width=2.5cm, minimum height=1.5cm, right =1.25cm of ff, text width=2cm, align = center, yshift=0.5cm, line width=1.5pt] (comb2) {Combinational Circuit ($\mathcal{F}_o$)};
    % Input nodes
    \node [left =1cm of comb1, coordinate] (input1) {};
    % Output node
    \node [right of=comb2, coordinate] (output) {};
    % Connection arrows
    \draw [->, >=stealth, line width=1.5pt] (input1) -- node[left, xshift = -0.5cm] {$\mathbf P_{t}$} (comb1);
    \draw [->, >=stealth, line width=1.5pt] (comb1) -- node {$\mathbf H_{t}$} (ff);
    \draw [->, >=stealth, line width=1.5pt] (comb1.east) -- node {} (ff);
    \draw [->, >=stealth, line width=1.5pt] (ff) -- node {$\mathbf H_{t-1}$} ([yshift=-0.5cm]comb2.west);
    \draw [->, >=stealth, line width=1.5pt] ([xshift=0.5cm]ff.east) |-  ([yshift=-1.5cm,xshift = -0.5cm]comb1.west) |-  ([yshift=-0.25cm, xshift = -0.15cm]comb1.west) |- ([yshift=-0.25cm]comb1.west);
    \draw [->, >=stealth, line width=1.5pt] (comb2) -- node[right, xshift = 0.5cm] {$\mathbf Y_t$} (output);
    \draw [->, >=stealth, line width=1.5pt, dashed] ([yshift=0cm, xshift = -0.5cm]comb1.west) |- ([yshift=1cm, xshift = -0.75cm]comb2.west) |- ([yshift=0.5cm, xshift =0cm]comb2.west);
\end{tikzpicture}}
%          \vspace{-0.5cm}
%          \caption{}
%          \label{fig2a}
%      \end{subfigure}
%      \hfill
%      \begin{subfigure}[b]{0.39\textwidth}
%          \centering
%          \scalebox{0.9}{\begin{tikzpicture}[scale=1, transform shape];
                \node [nnlayer]                     at ( 0,     0)    (sig1)          {$\mathcal{F}_o$};
                \node [nnlayer]                     at ( 1.5,   0)    (sig2)          {$\mathcal{F}_h$};

                \node [anchor=east]                 at ( -1,  -0.5) (hiddenlast)    {$\mathbf H_{t-1}$};
                \node [anchor=west]                 at ( 3.5,     1.) (hiddennext)    {$\mathbf H_{t}$};

                \node [anchor=west]                 at ( 0,     2.25) (hiddennext1)    {$\mathbf Y_{t}$};

                \node [anchor=east]                 at ( -1, -1.5)  (input)         {$\mathbf P_{t}$};

                \draw [ultra thick, rounded corners=0.2cm] (hiddenlast) -| (sig1);
                \draw [ultra thick, rounded corners=0.2cm] (hiddenlast) -| (sig2);


                \draw [->, >=stealth, ultra thick, rounded corners=0.2cm] (sig1.north) -- ([yshift = 1.5cm]sig1.north);;

                % \draw [->, >=stealth, ultra thick, rounded corners=0.2cm] (sig2.north) |- ([xshift=2cm, yshift=0.5cm]sig2.north) |- ( 6,  -0.5);

                \draw [->, >=stealth, ultra thick, rounded corners=0.2cm] (sig2.north) |- ([xshift=1.5cm, yshift=0.5cm]sig2.north);

                \draw [ultra thick, rounded corners=0.2cm] (input) -- ++(1.2,0) |- (0.5,-0.5);

                \begin{pgfonlayer}{background}
                \draw [fill=dateblue!10, rounded corners=.5cm] (-.5, -1) rectangle (3,1.5);
                \end{pgfonlayer}


            \end{tikzpicture}}
%          \vspace{-0.5cm}
%          \caption{}
%          \label{fig2b}
%      \end{subfigure}
%     \caption{The general form of a sequential circuit is shown in (a), and the recurrent cell for sequential circuits is depicted in (b).}
%     \label{fig2}
%     \vspace{-0.5cm}
% \end{figure*}



% \section{Sequential Circuits}
% So far, we have described how combinational circuits can be modeled and analyzed using {\sc Demotic}. In contrast to combinational circuits, where outputs are determined solely by their present inputs, the output in sequential circuits depends on both the past behavior of the circuit and the present values of inputs. The temporal operations of sequential circuits are controlled by a clock signal. The contents of memory elements (i.e., flip-flops) represent the past behavior of such a circuit, which is commonly referred to as the \textit{state} of the circuit. 


% Solving CircuitSAT problems for sequential circuits presents a unique challenge as it requires finding a sequence of inputs that satisfies the target constraint over a series of clock cycles. To tackle such problems, we can leverage a novel technique inspired by recurrent neural networks (RNNs). In RNNs, backpropagation through time is utilized during the learning process, allowing for updates to the network's hidden state at each time step. Similarly, in the context of solving CircuitSAT problems, we perform forward computations to iteratively update the state values at each clock cycle. During backward computations, gradients are backpropagated through time, extending back to the initial time step (i.e., the first clock cycle), to adjust the input sequence accordingly. While this approach draws parallels to RNN training, it is tailored to the unique challenges posed by solving CircuitSAT problems for sequential circuits. 



% Fig. \ref{fig2a} shows the general structure of a sequential circuit. We use this structure to formulate the CircuitSAT problem for sequential circuits to find satisfying solutions using {\sc Demotic}. In this structure, there are two combination circuits: one to update the state of the circuit (i.e., the content values of flip-flops) and the other one to generate the output. It is worth mentioning that both of these combinational circuits take the present values of flip-flops and primary inputs at the current time step as their inputs. Let us represent the primary input variables at time step $t$ as $\textbf{V}_t \in \mathbb{R}^{b\times n}$. We encode the primary input variables at each time step as learnable parameters to an embedding layer followed by the sigmoid function to provide input probabilities at time $t$ as $\textbf{P}_t \in [0, 1]^{p\times n}$ to the combinational circuits, i.e.,
% \begin{equation}
%     \textbf{P}_t = \sigma(\textbf{V}_t).
% \end{equation}
% The present output of the circuit (i.e., $\textbf{Y}_t \in [0, 1]^{b\times m}$) is computed as: 
% \begin{equation}
%     \textbf{Y}_{t} = \mathcal{F}_o(\textbf{P}_t, \textbf{H}_{t-1}),
% \end{equation}
% where $\mathcal{F}_o$ and $\textbf{H}_t \in [0, 1]^{b \times r}$ denote the functionality of the combinational circuit generating outputs and the present values of flip-flops at each time step, respectively. The number of flip-flops in the circuit is represented by $r$. The state of the circuit for the next time step is obtained as:
% \begin{equation}
%     \textbf{H}_{t} = \mathcal{F}_h(\textbf{P}_t, \textbf{H}_{t-1}),
% \end{equation}
% where the functionality of the combinational circuit updating the values of flip-flops is denoted by $\mathcal{F}_h$. The $\ell_2$-loss function $\mathcal{L}$ can then be constructed by measuring the distance between $\textbf{Y}_t$ at the desired time step $N$ and the target output valuation matrix $\textbf{T} \in \{0, 1\}^{b \times m}$ as follows:
% \begin{equation}
%     \mathcal{L} = \sum_{b,m} \left|\left| \textbf{T} - \textbf{Y}_N \right|\right|^2_2.
% \end{equation}
% With such a formulation for sequential circuits, {\sc Demotic} can solve the CircuitSAT problem and provide $b$ solutions. The general form of the recurrent cell for sequential circuits is shown in Fig. \ref{fig2b}, which is analogous to the RNN cell.





\section{Experimental Setup}\label{section4}
\section{Experimental Setup}
\label{sec_setup}

\begin{table}[h] \footnotesize  \centering\resizebox{0.48\textwidth}{!}{\begin{tabular}{c|l|c|c|c}
\toprule
\textbf{Task} & \textbf{Dataset} & \textbf{N-shot} & \multirowcell{\textbf{Train texts} \\ \textbf{for STMD}} & \multirowcell{\textbf{Evaluation} \\ \textbf{texts}} \\
\midrule
\multirow{3}{*}{\multirowcell{Text \\ Summarization}} & CNN/DailyMail & 0 & 2,000 & 2,000 \\
& XSum & 0 & 2,000 & 2,000 \\
& SamSum & 0 & 2,000 & 819 \\
\midrule
\multirow{4}{*}{\multirowcell{QA \\ Long answer}} & PubMedQA & 0 & 2,000 & 2,000 \\
& MedQUAD & 5 & 2,000 & 2,000 \\
& TruthfulQA & 5 & 408 & 409 \\
& GSM8k & 5 & 2,000 & 1,319 \\
\midrule
\multirow{4}{*}{\multirowcell{QA \\ Short answer}} & SciQ & 0 & 5,000 & 1,000 \\
& CoQA & \multirowcell{all preceding \\ questions} & 5,000 & 2,000 \\
& TriviaQA & 5 & 5,000 & 2,000 \\
\midrule
\multirow{1}{*}{\multirowcell{MCQA}} & MMLU & 5 & 5,000 & 2,000 \\
\bottomrule
\end{tabular}
}\caption{\label{tab:dataset_stat} The statistics of the datasets used for evaluation.}
\end{table}


To demonstrate the utility of the framework for measuring \ENDow{}, we describe the various SLU pipeline configurations on which we apply the framework and conduct analyses (discussed in \S\ref{sec_results}).
%Our experiments are conducted on three downstream tasks, four task-models, and several cleaning techniques, as outlined below.

\subsection{Preparing Transcript Sets}
\label{sec_setup_transcripts}

\paragraph{Text-to-speech model.}
Some of the SLU datasets in our experiments lack accompanying audio files, and in any case, we would like our experiments to be based on a controlled speech environment. We used the \texttt{toirtoise-tts} \citep{betker2023tortoisetts} Python library\footnote{\url{https://github.com/neonbjb/tortoise-tts}} as the text-to-speech model, and implemented a procedure for handling lengthy speech (see Appendix \ref{sec_appendix_implementation_tts}). The TTS stage produces the initial set of audio files for each of the SLU datasets in our experiments. 
%See Appendix \todo{TTS appendix} for technical details on the TTS procedure applied for producing high-quality audio files.

\paragraph{Noising method.}
Each audio file was reverberated with the \texttt{rir-generator} \citep{werner2023rirgenerator} Python library,\footnote{\url{https://github.com/audiolabs/rir-generator}} and then recreated with background office sounds \citep[a clipped audio file;][]{myNoise2020office} with one of five signal-to-noise ratios (see Appendix \ref{sec_appendix_implementation_noising}). 
%We produced noised audio files at various levels in two steps: (1) an audio file is reverberated with the \texttt{rir-generator} \citep{werner2023rirgenerator} Python library,\footnote{\url{https://github.com/audiolabs/rir-generator}} and then (2) background office sounds are added, using a clipped audio file \citep{myNoise2020office}, at five different signal-to-noise ratios (see Appendix \ref{sec_appendix_implementation_noising}).
After this process there are six sets of increasingly tampered audio files.

\paragraph{ASR system.}
We used Whisper \citep{Radford2023whisper}\footnote{\texttt{openai/whisper-small.en}} for conducting speech-to-text (see Appendix \ref{sec_appendix_implementation_stt}). 
In all there are seven sets of increasingly noised transcripts (the first is the clean reference set). In our setting, seven noise levels provided a satisfactory analysis for examining the behavior of the SLU pipeline. The WER scores distribute within 0 and 0.9, and the curves empirically exhibit sufficiently clear behavioral patterns.
%In all there are six sets of ASR-generated transcripts and the reference set, overall providing seven levels of noise (the first set with no noise). We aimed for seven levels as this provides a satisfactory analysis for examining the behavior of the SLU pipeline.\footnote{The range of reverberation we used to impair the audio files yield transcripts with WER scores of up to $\sim$0.9. Also, the seven levels of noise empirically provide curves with sufficiently clear behavioral patterns.}

\paragraph{Cleaning techniques.}
In our experiments, we use the cleaning component to study the effect of different types of words, e.g., nouns, on downstream tasks. This analysis also simulates an SLU pipeline in which the ASR system prioritizes accuracy for specific word types, guiding where to focus efforts in the transcription process.

To clean transcripts, we first aligned a noised transcript to its respective reference transcript with \texttt{jiwer}.\footnote{\url{https://github.com/jitsi/jiwer}} Then, any non-equivalent alignment that involves the targeted word-type was repaired. We separately target nouns, verbs, adjectives, adverbs, any of the above (``content words''), none of the above (``non-content words''), and named entities -- seven techniques in all. Details in Appendix \ref{sec_appendix_implementation_cleaning}.


\subsection{Downstream Tasks}
\label{sec_setup_tasks}

We experiment with three downstream tasks, characterized by different output objectives. Summarization is a generation task where text is synthesized based on the collective understanding of a dialog. Question-answering is framed here as an extraction task that retrieves spans from the transcript. Dialog-act categorization is a classification task that assigns a communicative goal label (e.g., `statement', `question', etc.) to conversational utterances. The first two tasks are on the full transcript level, while the latter task is on the utterance level. These differences offer insights into potential distinctions in SLU pipelines.

In our experiments we focus on \textit{long spoken dialogues}, as opposed to short or written dialogues, as they impose a more challenging setting for task models.
%In our experiments we focus on \textit{long spoken dialogues}, as opposed to dialogues with few turns or those based on written language such as chats or forums.
See \autoref{tab_datasets} for a summary of the tasks, and Appendix \ref{sec_appendix_datasets} for examples of task instances.

\paragraph{Summarization.}
For summarization, we use the QMSum dataset\footnote{\url{https://github.com/Yale-LILY/QMSum}} \citep{zhong-etal-2021-qmsum}, a generic and query-focused dialog summarization benchmark. It consists of transcripts and summaries of product meetings \citep[AMI;][]{carletta2006ami}, academic meetings \citep[ICSI;][]{janin2003icsi} and parliament committee meetings.
%, all as transcripts with their respective queries and reference summaries.

To evaluate system summaries we use standard ROUGE metrics\footnote{\url{huggingface.co/spaces/evaluate-metric/rouge}, with the default arguments.} \citep{lin-2004-rouge} and pairwise comparison ranking \citep{qin-etal-2024-large} with \texttt{GPT-4o-mini} as a judge for overall quality \citep{liu-etal-2024-benchmarking} (see Appendix \ref{sec_appendix_implementation_pairwise} for details).

\paragraph{Question-answering.}
The QAConv dataset\footnote{\url{https://github.com/salesforce/QAConv}} \citep{wu-etal-2022-qaconv} consists of dialogues with questions whose answers are short spans in the dialog. We only use the instances based on court cases or interviews (since these are long spoken dialogues).

For evaluation, predicted answers are compared against reference answers with exact match accuracy, token-level $F_1$ and fuzzy matching, following the QAConv benchmark.

\paragraph{Dialog-act classification.}
%The objective of this task is to label an utterance with the proper communicative function, such as a question or statement. We use 
The MRDA dataset\footnote{\url{https://github.com/NathanDuran/MRDA-Corpus}} \citep{shriberg-etal-2004-icsi} consists of meetings from the ICSI corpus and research-oriented group meetings. Each utterance in the transcripts is labeled with one of 12 dialog act labels \citep{dhillon2004mrdaLabeling}. We utilize the first and last 50 utterances from each transcript (100 of $\sim$1392), for efficiency purposes. See Appendix \ref{sec_appendix_datasets} for more details.
%In the dialog-act classification task, the objective is to label an utterance with the proper communicative function, such as a question or statement. The label-sets vary from one dataset to another, and have different levels of granularity (from 5 to 50+ labels). We use the MRDA dataset\footnote{\url{https://github.com/NathanDuran/MRDA-Corpus}} \citep{shriberg-etal-2004-icsi}, which consists of meetings from the ICSI corpus and research-oriented group meetings. Each utterance is labeled with a dialog act on three granularity levels. For our experiments, we used the middle granularity level \citep[tagset with 12 dialog acts;][]{dhillon2004mrdaLabeling}, and the first and last 50 utterances from each transcript (100 of $\sim$1392), for efficiency purposes.

The MRDA results were traditionally evaluated with the accuracy metric, but we also evaluate with macro-$F_1$ due to the high class imbalance in the dataset, as suggested by \citet{miah-etal-2023-hierarchical}.

\paragraph{Models.}
For all three tasks, we experiment with four instruct-tuned LLMs in zero-shot mode:
%, on the test sets of the tasks.
%The models used are 
Mistral-7B,\footnote{\texttt{mistralai/Mistral-7B-Instruct-v0.3}} Llama3-8B,\footnote{\texttt{meta-llama/Meta-Llama-3-8B-Instruct}} Llama3.1-8B,\footnote{\texttt{meta-llama/Llama-3.1-8B-Instruct}} and GPT-4o-mini.\footnote{\texttt{gpt-4o-mini-2024-07-18}} 
They were selected for their modest hardware requirements and affordability.
%Additionally, our goal is to investigate whether these models behave differently when challenged with transcripts with varying noise levels and types.
%We use the same general prompts for the four models, which were scripted with some light prompt engineering on a few instances.
Since the context size of Mistral and Llama-3 cannot fit most of the transcripts in full, summarization and QA were conducted on these models in segments.
%, however GPT and Llama-3.1 received the full transcripts as input.
See details and prompts in Appendix \ref{sec_appendix_prompts}.




\section{Experimental Results}\label{section5}

In this section, we first present the overall results of the data leakage detection analysis. Following that, we provide detailed results and answers to each research question (RQ).



\vspace{0.2cm}
\noindent
\textbf{Overall Results.}
Before diving into the specific RQs, we first present an overview of the experimental results. The selected LLM's pre-training datasets consist of 12M samples for Python, 20M samples for Java, and 14M samples for C/C++. In comparison, the diverse SE benchmarks we studied collectively comprise 46k samples for Python, 42k samples for Java, and 21k samples for C/C++.
To investigate potential data leakage, each SE benchmark sample was compared against all pre-training data samples for its corresponding programming language. This process resulted in an astounding total of over 1.7 trillion comparisons. The sheer scale of this computational effort highlights the complexity and resource-intensive nature of studying data leakage regarding LLMs. 


From an overall perspective, as depicted in Figure~\ref{fig:data_review}, only 2\% of the benchmark samples from all the SE benchmarks studied were flagged by the automated tool MinHash+LSH as potentially forming at least one duplicate pair with the pre-training data of StarCoder. Moreover, of the pairs flagged by MinHash+LSH, 28\% were confirmed as duplicates after manual labeling, while the remaining 72\% were determined not to be duplicates.

Next, we will discuss the detailed results and answers to each RQ.

\subsection{Pre-trained Model Selection (RQ1)} \label{subsec:rq1}

\sectopic{Methodology. }  We shortlist the ST models for investigation in our work based on the NLP  leaderboard, which reports the 38 most accurate pre-trained models\footnote{\url{https://www.sbert.net/docs/pretrained_models.html}}. These models have been extensively evaluated for their ability to generate sentence embeddings (i.e., capturing the semantics of the whole text) and their performance in semantic search (i.e., finding relevant answers to a given query). Both tasks closely align with our objectives. 
To identify trace links, we apply the pre-trained models in a zero-shot setting as follows. 
We let each model compute the similarity matrix equivalent to the output of step~5 in our approach (see Fig.~\ref{fig:approach}). 
We then predict a trace link if the similarity value exceeds 
a predefined threshold. Since zero-shot does not require training, we run EXPI on the entire \texttt{HIPAA} dataset. 


\sectopic{Evaluation Metrics. } To better assess the performance irrespective of the selected threshold, we compute the \textit{Area Under the Curve (AUC)} for the receiver operating characteristic (ROC) across different threshold values,  ranging from $0.1$ to $0.9$. 
The ROC curve captures the trade-off between the true positive rate (TPR) and the false positive rate (FPR). TPR is the proportion of positives correctly identified as such (i.e., the percentage of trace links correctly identified for a given threshold). FPR is the proportion of negatives incorrectly identified as positives (i.e., the percentage of trace links wrongly identified as not trace links). The AUC of the ROC curve (computed as micro-average over all the provisions to avoid the dominance of some provisions)  provides a single aggregate performance measure across all possible thresholds and, hence, is a suitable evaluation metric to compare the ST models.  We posit that the model with the highest AUC value demonstrates the best overall performance in identifying trace links in a zero-shot setting, as a higher AUC value indicates a better balance between correctly identifying true trace links (high TPR) and minimizing the identification of false links (low FPR). 
%
%
\sectopic{Results. }
Table~\ref{tab:rq1} presents the \texttt{AUC} values of the ST pre-trained models on the \texttt{HIPAA} dataset and also  reports $K$, indicating the ranking of the models in the NLP community based on their accuracy~\cite{Reimers:19}, as well as $K^\dag$, indicating the ranking based on \texttt{AUC} achieved on \texttt{HIPAA}. 



\begin{table}
%\footnotesize
\centering
\caption{AUC of ST models for LRT on \texttt{HIPAA} (\textbf{RQ1}). 
% \TBD{@Romina: come on! you don't leave footnotes on TRACES in the table. Please revise your changes. Also, you don't need "HIPAA" in the header if the results are only for HIPAA, @Romina: please remove and adjust the header accordingly}
}
\label{tab:rq1}
% \begin{threeparttable}[t]
\begin{tabularx}{\textwidth}{@{} p{0.05\textwidth} @{\hskip 0.5em} p{0.05\textwidth} @{\hskip 3em} p{0.05\textwidth} @{\hskip 20em} *{5}{>{\centering\arraybackslash}X}@{}}
    \toprule
    \multirow{1}{*}{$K$\tnote{1}} & \multirow{1}{*}{Model\tnote{2}} & \multirow{1}{*}{Name\tnote{1}} & \multirow{1}{*}{\texttt{AUC}\tnote{1}} & \multirow{1}{*}{$K^\dag$\tnote{1}} \\%\multicolumn{2}{c}{\texttt{HIPAA}} \\ %& \multicolumn{2}{c}{\texttt{TRACES}} & \multicolumn{2}{c}{Average}\\
    % \cmidrule(lr){4-5}
    % &&& \texttt{AUC} & $K^\dag$ \\ %&\texttt{AUC} & $K^\ddag$ &\texttt{AUC} & $K^*$  \\
    \midrule
1 &   \texttt{ST1}  & \texttt{all-mpnet-base-v2} & 0.744 & 16 \\ % & 0.331 & 29 & 0.538 & 27\\
2 &   \texttt{ST2}  & \texttt{gtr-t5-xxl} & 0.725 & 21 \\ % & \textbf{0.685} & 1 & 0.705 & 7\\
3 &   \texttt{ST3}  &\texttt{gtr-t5-xl} & 0.789 & 6 \\ % & 0.678 & 2 & 0.733 & 2\\
4 &   \texttt{ST4}  &\texttt{sentence-t5-xxl} & 0.720 & 22 \\ % & 0.666 & 3 & 0.693 & 8\\
5 &   \texttt{ST5}  &\texttt{gtr-t5-large} & 0.743 & 17 \\ % & 0.640 & 7 & 0.692 & 9\\
6 &   \texttt{ST6}  &\texttt{all-mpnet-base-v1} & 0.712 & 25 \\ % & 0.338 & 27 & 0.525 & 29\\
7 &   \texttt{ST7}  &\texttt{multi-qa-mpnet-base-dot-v1} & 0.688 & 27 \\ % & 0.631 & 8 & 0.659 & 12\\
8 &   \texttt{ST8}  &\texttt{multi-qa-mpnet-base-cos-v1} & 0.603 & 34 \\ % & 0.222 & 36 & 0.413 & 36\\
9 &   \texttt{ST9}  &\texttt{all-roberta-large-v1} & 0.601 & 35 \\ % & 0.333 & 28 & 0.467 & 34\\
10 &   \texttt{ST10}  &\texttt{sentence-t5-xl} & 0.769 & 10 \\ % & 0.644 & 6 & 0.706 & 5\\
11 &   \texttt{ST11}  &\texttt{all-distilroberta-v1} & 0.719 & 23 \\ % & 0.284 & 34 & 0.501 & 32\\
12 &   \texttt{ST12}  &\texttt{all-MiniLM-L12-v1} & 0.729 & 19 \\ % & 0.318 & 30 & 0.523 & 30\\
13 &   \texttt{ST13}  &\texttt{all-MiniLM-L12-v2} & 0.747 & 15 \\ % & 0.339 & 26 & 0.543 & 26\\
14 &   \texttt{ST14}  &\texttt{multi-qa-distilbert-dot-v1} & 0.563 & 36 \\ % & 0.546 & 17 & 0.555 & 25\\
15 &   \texttt{ST15}  &\texttt{multi-qa-distilbert-cos-v1} & 0.640 & 33 \\ % & 0.228 & 35 & 0.434 & 35\\
16 &   \texttt{ST16}  &\texttt{gtr-t5-base} & 0.770 & 9 \\ % & 0.655 & 5 & 0.712 & 4\\
17 &   \texttt{ST17}  &\texttt{sentence-t5-large} & 0.748 & 14 \\ % & 0.663 & 4 & 0.706 & 6\\
18 &   \texttt{ST18}  &\texttt{all-MiniLM-L6-v2} & 0.761 & 11 \\ % & 0.285 & 33 & 0.523 & 31\\
19 &   \texttt{ST19}  &\texttt{multi-qa-MiniLM-L6-cos-v1} & 0.670 & 29 \\ % & 0.313 & 31 & 0.492 & 33\\
20 &   \texttt{ST20}  &\texttt{all-MiniLM-L6-v1} & 0.749 & 13 \\ % & 0.307 & 32 & 0.528 & 28\\
21 &   \texttt{ST21}  &\texttt{paraphrase-mpnet-base-v2} & 0.850 & 2 \\ % & 0.587 & 14 & 0.719 & 3\\
22 &   \texttt{ST22}  &\texttt{msmarco-bert-base-dot-v5} & 0.644 & 32 \\ % & 0.503 & 20 & 0.574 & 24\\
23 &   \texttt{ST23}  & \texttt{multi-qa-MiniLM-L6-dot-v1} & 0.715 & 24 \\ % & 0.605 & 12 & 0.660 & 11\\
24 &   \texttt{ST24}  & \texttt{sentence-t5-base} & 0.726 & 20 \\ % & 0.584 & 15 & 0.655 & 13\\
25 &   \texttt{ST25}  & \texttt{msmarco-distilbert-base-tas-b} & 0.701 & 26 \\ % & 0.557 & 16 & 0.629 & 18\\
26 &   \texttt{ST26}  & \texttt{msmarco-distilbert-dot-v5} & 0.685 & 28 \\ % & 0.600 & 13 & 0.643 & 15\\
27 &   \texttt{ST27}  & \texttt{paraphrase-distilroberta-base-v2} & 0.801 & 4 \\ % & 0.455 & 24 & 0.628 & 19\\
28 &   \texttt{ST28}  & \texttt{paraphrase-MiniLM-L12-v2} & 0.794 & 5 \\ % & 0.496 & 22 & 0.645 & 14\\
29 &   \texttt{ST29}  & \texttt{paraphrase-multilingual-mpnet-base-v2} & \textbf{0.859} & 1 \\ % & 0.614 & 10 & \textbf{0.736} & 1\\
30 &   \texttt{ST30}  & \texttt{paraphrase-TinyBERT-L6-v2} & 0.787 & 7 \\ % & 0.464 & 23 & 0.625 & 21\\
31 &   \texttt{ST31}  & \texttt{paraphrase-MiniLM-L6-v2} & 0.770 & 8 \\ % & 0.511 & 18 & 0.641 & 16\\
32 &   \texttt{ST32}  & \texttt{paraphrase-albert-small-v2} & 0.737 & 18 \\ % & 0.499 & 21 & 0.618 & 22\\
33 &   \texttt{ST33}  & \texttt{paraphrase-multilingual-MiniLM-L12-v2} & 0.811 & 3 \\ % & 0.511 & 19 & 0.661 & 10\\
34 &   \texttt{ST34}  & \texttt{paraphrase-MiniLM-L3-v2} & 0.757 & 12 \\ % & 0.441 & 25 & 0.599 & 23\\
35 &   \texttt{ST35}  & \texttt{distiluse-base-multilingual-cased-v1} & 0.349 & 37 \\ % & 0.092 & 37 & 0.220 & 37\\
36 &   \texttt{ST36}  & \texttt{distiluse-base-multilingual-cased-v2} & 0.341 & 38 \\ % & 0.090 & 38 & 0.216 & 38\\
37 &   \texttt{ST37}  & \texttt{average\_word\_embeddings\_komninos} & 0.647 & 31 \\ % & 0.606 & 11 & 0.627 & 20\\
38 &   \texttt{ST38}  & \texttt{average\_word\_embeddings\_glove.6B.300d} & 0.636 & 30 \\ % & 0.625 & 9 & 0.630 & 17\\ 
\bottomrule
\end{tabularx}
\begin{tablenotes}
     \item[1] $K$: The average performance ranking of the models, as reported in the NLP community. $K^\dag$: The ranking of the models based on AUC values computed on \texttt{HIPAA} ($K=1$ indicates the highest AUC). 
      \item [2] \texttt{ST1}--\texttt{ST38} correspond to the models reported at this link (sorted by average accuracy in descending order):     \url{https://www.sbert.net/docs/pretrained_models.html}. %, where \texttt{ST29} is \texttt{paraphrase-multilingual-mpnet-base-v2}.
     \end{tablenotes}
 % \end{threeparttable}
 %\vspace*{-2em}
 \end{table}

 

The best-performing model on \texttt{HIPAA} is \texttt{ST29} ($K^\dag=1$), with an AUC value of 0.859. The next best performing model is \texttt{ST21} with an AUC value of 0.850. The difference between these two AUC values is only marginal. A possible explanation is that  \texttt{ST29} uses  \texttt{ST21} as its base model.  \texttt{ST29}  has been, however, trained on more (multi-lingual) data.   

Additionally, we observe a discrepancy in the performance of the models on the \texttt{HIPAA} dataset compared to that reported by the NLP community.  
The best NLP model, \texttt{ST1}, does not perform well  on \texttt{HIPAA}, ranked 16. 
This observation indicates that well-performing models in NLP are not necessarily as effective for RE-specific problems. 
%The datasets in RE are typically domain-specific increasing the level of complexity to deal with.    

\begin{tcolorbox}[arc=1mm,width=\columnwidth,
                  top=0mm,left=0mm,  right=0mm, bottom=0mm,
                  boxrule=1pt, colback=violet!15!white,colframe=white]
\textbf{The answer RQ1} is that \texttt{ST29} is the best-performing pre-trained model for LRT (corresponding to \texttt{paraphrase-multilingual-mpnet-base-v2}). 
\end{tcolorbox}%The goal of RQ1 is to select a robust ST model that performs consistently well across datasets. 
% Table~\ref{tab:rq1} shows the \texttt{AUC} values of the ST pre-trained models on the \texttt{HIPAA} and \texttt{TRACES} datasets. The table also reports $K$ indicating the ranking of the models in the NLP community based on their accuracy~\cite{Reimers:19}, as well as $K^\dag$,  $K^\ddag$ and $K^*$,  indicating the rankings based on \texttt{AUC} achieved on \texttt{HIPAA},  \texttt{TRACES} and on average across the two datasets, respectively. The AUC for the ROC curve metric enables fair comparison, irrespective of the selected threshold values. 

%\input{Files/tab1-RQ1}

% The table shows that the models perform considerably poorly on the \texttt{TRACES} dataset. A plausible reason is that \texttt{TRACES} has a total of 26 regulatory codes, some of which are seemingly closely related (e.g., the regulatory code \textit{TIM}---the period for which personal data is stored is semantically close to \textit{DUR}---the duration of data processing). 
% To reduce the degree of confusion that ST models exhibit, we compute the AUC values for \texttt{TRACES} at the category level \TBD{is this what we report in the table? (yes)}. Recall from Section~\ref{tab:datasets} that the 26 regulatory codes in  \texttt{TRACES} are grouped into 10 different categories (listed in Table~\ref{tab:datasets}). Once the ST model computes the similarity values of single regulatory codes, we then assign to each category the maximum similarity values among the single regulatory codes in that category. For example, \textit{TIM} and \textit{DUR} belong to the category \textit{data retention} (\textit{RTN} in Table~\ref{tab:datasets}). If the similarity value between a given requirement $r_i$ and \textit{TIM} and \textit{DUR} is 0.3 and 0.47, respectively, then we assign the similarity value 0.47 between $r_i$ and the category \textit{data retention}.  


% The table further shows a discrepancy in the performance of the models across our datasets compared to that reported by the NLP community.  
% The best NLP model, \texttt{ST1}, does not perform well on our datasets as it is ranked 16 on \texttt{HIPAA} and 29 on \texttt{TRACES}. This indicates that well-performing models in NLP are not necessarily robust for RE-specific problems where the models are confronted with datasets spanning specific-domains and potentially different requirement types.  

% The best-performing model on \texttt{HIPAA} is \texttt{ST29} ($K^\dag=1$), with an AUC value of 0.859. The same model, \texttt{ST29}, is however ranked 10 on \texttt{TRACES} with an AUC of 0.614, 0.07 lower than the best model \texttt{ST2} ($K^\ddag=1$). However, \texttt{ST2} yields  0.13 lower AUC value on \texttt{HIPAA} when compared with \texttt{ST29}. 
% Overall, \texttt{ST29} achieves the best average AUC value of 0.736 on both datasets \texttt{HIPAA} and \texttt{TRACES} ($K^*=1$), leaving \texttt{ST2} six ranks behind. 
% Additionally, we observe that, on average, \texttt{ST3} fares fairly close to \texttt{ST29}. Still, according to the NLP leaderboard, \texttt{ST29} has the advantage of being much faster and smaller in size than \texttt{ST3}: \texttt{ST29}'s size is 970 MB, whereas \texttt{ST3}'s size is 2370 MB. 

% \begin{tcolorbox}[arc=1mm,width=\columnwidth,
%                   top=0mm,left=0mm,  right=0mm, bottom=0mm,
%                   boxrule=1pt, colback=violet!15!white,colframe=white]
% In view of the above analysis, \textbf{the answer RQ1}, we select \texttt{ST29} (corresponding to \texttt{paraphrase-multilingual-mpnet-base-v2}) as the best-performing ST model in identifying trace links using a zero-shot setting. 
% \end{tcolorbox}




\begin{table}[t!]
    \centering
    \tabcolsep=1.5pt
    \renewcommand{\arraystretch}{0.92} 
    \caption{Experimental results of RQ4}
    \begin{tabular}{c|cc|cc}
    \hline
    \textbf{Method}             & \multicolumn{2}{c|}{\textbf{Zero-Shot}}                                  & \multicolumn{2}{c}{\textbf{COT}}                               \\ \hline
    \textbf{Metric}             & \multicolumn{2}{c|}{\textbf{Success Rate}}                               & \multicolumn{2}{c}{\textbf{Success Rate}}                                \\ \hline
    \textbf{Model}              & \multicolumn{1}{c|}{\textbf{GPT4o}} & \textbf{Gemini} & \multicolumn{1}{c|}{\textbf{GPT4o}} & \textbf{Gemini} \\ \hline
    \textbf{Illu Text} & 0.00\%                                 & 0.00\%    &0.00\%                                 & 0.00\%                  \\ \hline
    \textbf{Illu Image} & 0.00\%                                 & 0.00\%    &0.00\%                                & 0.00\%                 \\ \hline
    \end{tabular}
    \label{tex:RQ2}
\end{table}

\begin{table}[!t]
\small
\caption{The sum of feature contribution scores (FCS) for the comparison schemes during training and testing in both closed and open-world settings. AG and GR represent the baselines APIGraph~\cite{apigraph} and GuideRetraining~\cite{guide_retraining}.}
\label{tab:rq3}
\centering
\begin{tblr}{
  cells = {c},
  rowsep = -1.0pt,
  cell{1}{1} = {c=9}{},
  cell{2}{1} = {r=2}{},
  cell{2}{2} = {c=4}{},
  cell{2}{6} = {c=4}{},
  vline{2-3,6} = {2}{},
  vline{2,6} = {3}{},
  vline{2,6} = {4-14}{},
  hline{1-2,4,15} = {-}{},
  hline{3} = {2-9}{},
  colsep = 2.5pt,
}
DeepDrebin\cite{Grossedeepdrebin} &                 &    &    &      &            &    &    &      \\
           & Closed world    &    &    &      & Open world &    &    &      \\
           & w/o             & AG & GR & Ours & w/o        & AG & GR & Ours \\
Train      & 27.22 & 28.33 & 25.15 & 32.15 &  27.22 & 28.33 & 25.15 & 32.15  \\
1          &  23.98 & 24.49 & 19.04 &27.98 & 22.91  & 23.62 & 18.58 &  27.33 \\
2          &  21.70 & 23.53 & 18.44 & 26.14 &  19.46 & 22.57 & 17.04 & 25.74 \\
3          &  19.78 & 20.27& 17.12 & 22.92 &  17.56 &19.29  & 16.10 & 21.97 \\
4          & 17.94 &18.04 & 15.11& 20.16& 16.68 & 16.74 & 13.72 & 19.26 \\
5          & 15.73 & 15.82 &13.52 & 18.72 & 14.69  & 15.09& 12.06 & 18.57 \\
6          & 14.26& 14.23 &12.46 & 16.63 & 13.56 & 13.75 & 11.10& 16.59 \\
7          & 13.14 & 13.21 & 11.67 & 14.89 & 12.12   & 13.19 & 11.63 & 14.75 \\
8          & 12.58 & 12.47 &10.88 &13.83 & 11.75  &11.83 & 10.05 & 13.66  \\
9          & 12.26 & 12.31 & 10.52 & 12.91& 11.24  &11.12 & 9.30 &12.38 \\
10          &  11.98  &12.01 &6.83& 12.52 & 10.51 & 10.62& 6.01& 11.62
\end{tblr}
\end{table}


\begin{table}[htb!]
\caption{IDP Comparison of CodeImprove with CodeImprove-Random}
\label{Tab:rq4}
\renewcommand{\arraystretch}{1.12}
\resizebox{\columnwidth}{!}{
\begin{tabular}{|c|c|c|c|c|c|c|c|}
\hline
 \multirow{2}{*}{\textbf{Experiment}} &\multicolumn{3}{c|}{\textbf{Vulnerability Detection}} & \multicolumn{3}{c|}{\textbf{Defect Prediction}}\\ \cline{2-7}
      & CodeBERT & RoBERTa & BERT  & CodeBERT & RoBERTa & BERT \\ \hline

    CodeImprove-Random  &4.5  & 4.18  & 3.95  & 5.04 & 8.35 & 5.45   \\\hline

     CodeImprove & 16.77  & 6.32  &8.78  & 12.04 &  10.06&11.12 \\\hline

     



    
  
  
\end{tabular}
}
\end{table}



\section{Discussion}\label{section6}
\section{Discussion}

In this paper, we explored the relationship between human evaluations and NLP benchmarks of chat-finetuned language models (chat LMs). Our work is motivated by the recent shift towards human evaluations as the primary means of assessing chat LM performance, and the desire to determine the role that NLP benchmarks should play.

Through a large-scale study of the Chat Llama 2 model family on a diverse set of human and NLP evaluations, we demonstrated that NLP benchmarks are generally well-correlated with human judgments of chat LM quality. However, our analysis also reveals some notable exceptions to this overall trend. In particular, we find that adversarial and safety-focused evaluations, as well as language assistance and open question answering tasks, exhibit weaker or negative correlations respectively with NLP benchmarks. We also explored predicting human evaluation scores from NLP evaluation scores using overparameterized linear regression models. Our results suggest that NLP benchmarks can indeed be used to predict aggregate human preferences, although we caution that the limited sample size and the assumptions of our models may limit the generalizability of these findings. Our results suggest that NLP benchmarks can serve as fast and cheap proxies of slower and expensive human evaluations in assessing chat LMs.

Additionally, our work highlights the need for further research into NLP evaluations that can effectively capture important aspects of LM behavior, such as safety, robustness to adversarial inputs, and performance on complex, open-ended tasks. It is possible that new NLP benchmarks can provide signals on these topics, e.g., \citep{wang2023decodingtrust}. Of particular interest is developing human-interpretable and scaling-predictable evaluation processes, e.g., \citep{schaeffer2024emergent, ruan2024observational,schaeffer2024predictingdownstreamcapabilitiesfrontier}. Developing and refining such evaluation methods \citep{madaan2024quantifyingvarianceevaluationbenchmarks}, as well as detecting whether evaluations scores faithfully capture models' true performance \citep{oren2023proving,schaeffer2023pretrainingtestsetneed,roberts2023cutoff,jiang2024investigatingdatacontaminationpretraining,zhang2024careful,duan2024uncoveringlatentmemoriesassessing} will be crucial for ensuring that LMs are safe, reliable, and beneficial as they become increasingly integrated into society.

% In conclusion, our study provides insights into the relationship between human evaluations and NLP benchmarks of chat language models. By leveraging the complementary strengths of both human and NLP benchmarks, we can build a more complete understanding of LM capabilities and behaviors, ultimately enabling the development of models more capable, trustworthy, and beneficial to society.




\section{Related Work}\label{section7}

Yang et al.~\cite{DBLP:conf/icse/0003ZWS0H024} investigates memorization in Code LLMs, focusing on CodeParrot and CodeParrot-small. The authors use code clone detection tools to identify duplicates in model-generated outputs and analyze patterns of memorization. Our work differs in three key ways. First, we focus on the StarCoder series, a larger and more advanced family of Code LLMs. Second, while Yang et al.~\cite{DBLP:conf/icse/0003ZWS0H024} examines general memorization capabilities using pre-training data not specifically tied to SE benchmarks, our study directly addresses data leakage in SE benchmarks, offering insights into the risks and biases such leakage introduces. Third, unlike their reliance on clone detection tools, which can yield false positives, our approach incorporates large-scale manual labeling to ensure the accuracy and reliability of identified duplicates and leaked samples.

Lopez et al.~\cite{lópez2024interdatasetcodeduplicationdata} investigated dataset leakage in lightweight LLMs, such as CodeBERT, by leveraging an automatic clone detection tool to identify duplicates between pre-training data and three SE benchmarks.
Compared to Lopez et al., our study investigates larger and more advanced LLMs, such as StarCoder, which are approximately 125 times larger than the models analyzed by Lopez et al. and significantly more capable in generation tasks like code generation and program repair. Additionally, rather than focusing on only three SE benchmarks, our work evaluates a much broader set of 83 SE benchmarks across three popular programming languages, providing a more comprehensive analysis of benchmark leakage. To ensure the accuracy of identified leakage, our study also integrates large-scale manual labeling.


In addition to academic literature, several studies in the gray (non-peer-reviewed) literature have explored related topics.
Matton et al.~\cite{matton2024leakagecodegenerationevaluation} analyzed potential data leakage in the HumanEval and MBPP benchmarks by checking for occurrences of their prompts in public GitHub repositories. Similarly, Riddell et al.~\cite{riddell2024quantifyingcontaminationevaluatingcode} studied leakage in these benchmarks using a multi-step approach. They calculated Levenshtein similarity scores to identify overlaps between the benchmarks and LLM pre-training data, selected the top 500 pre-training samples with the highest similarity for each test case, and used an automatic code plagiarism detection tool to identify duplicates. 
In contrast to these gray literature studies, which focus on only two or three SE benchmarks, our work evaluates a significantly broader set of 83 SE benchmarks across three popular programming languages, enabling a more comprehensive analysis of benchmark leakage. Furthermore, our approach incorporates large-scale manual labeling to enhance the accuracy and reliability of leakage detection. We also provide actionable recommendations for LLM developers and users to effectively mitigate data leakage.




\section{Conclusion and Future Work}\label{section8}



In this study, we performed the first large-scale analysis of data leakage across 83 software engineering (SE) benchmarks, covering three popular programming languages—Python, Java, and C/C++. By combining an efficient near-duplicate detection algorithm with extensive manual labeling, we ensured the accurate identification of leaked data.



Our findings show that while data leakage is generally low, with average leakage ratios of 4.8\%, 2.8\%, and 0.7\% for Python, Java, and C/C++ benchmarks respectively, some benchmarks exhibit higher leakage that requires attention. We identified four main causes of leakage: direct inclusion of benchmark data in pre-training datasets, overlap between source repositories, reliance on platforms like LeetCode, and shared data sources such as GitHub issues.
We also found that automatic detection methods, like Perplexity-based metrics, struggle to distinguish between leaked and non-leaked samples. Additionally, our experiments reveal that data leakage inflates evaluation metrics, with models performing significantly better on leaked samples. For instance, StarCoder-7b achieved a Pass@1 score 4.9 times higher on leaked samples, underlining the need to address leakage to ensure fair evaluations.
This study offers insights into data leakage status in SE benchmarks and its impact on LLM evaluation.


In the future, we aim to expand the analysis to additional benchmarks and explore new methods to prevent or further reduce data leakage.





\vspace{0.2cm}
\noindent \textbf{Acknowledgement.}  This research / project is supported by the National Research Foundation, under its Investigatorship Grant (NRF-NRFI08-2022-0002). Any opinions, findings and conclusions or recommendations expressed in this material are those of the author(s) and do not reflect the views of National Research Foundation, Singapore.





\balance
\bibliographystyle{acm}
\bibliography{main}

\end{document}
\endinput
%%
%% End of file `sample-manuscript.tex'.