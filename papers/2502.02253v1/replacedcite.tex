\section{Related Works}
\label{sec:related}

By leveraging the capabilities of NDT, digital replicas of networked environments can be created to offer unparalleled insights, predictive capabilities, and opportunities for optimization. 5G's low latency, high bandwidth, and massive device connectivity provide the foundation for real-time communication and control, enabling robots to readily respond to unexpected dynamics in the deployment environment. 

In this context, communication-aware motion planning in robotic networks is becoming of key importance in the robotic domain, due to the impelling need of offloading heavy computation tasks to edge platforms. 
The authors of____ investigate radio-aware semantic map creation along the exploration of unknown environments, and provide simulation-based results and a ROS package to deal with such scenarios. Similarly, ____ investigate joint optimization of 5G and robotic domain as a way to improve resource allocation and energy consumption KPIs. %____ and
In the domain of wireless channel estimation within robotic networks, the authors of____ propose a framework for estimating spatial variations in a wireless channel within a robotic network. They introduce a multiscale probabilistic model to characterize the wireless channel and develop an estimator based on this model.
In the area of NDT, most of the works in the literature consider holistic approaches by assuming full knowledge of the network infrastructure and environment, for example, the authors of____ propose a knowledge graph-based construction method for NDTs, exploiting a graph representation to model the system dynamics. %In ____ a scalable NDT for network slicing is developed, aiming to capture the intertwined relationships among slices and monitor the end-to-end (E2E) metrics of slices under diverse network environments. 
% ____
% ____
Altogether, this highlights the ongoing innovation and evolution in both the fields of robotics and communication to build an NDT, especially from a robot perspective, which needs to discover its environment and available network infrastructure. %, driven by the goal of meeting the demands of contemporary and future applications.
\new{Key challenges in building an NDT for robots stand on the model accuracy and real-time data acquisition. In this paper we introduce an online NDT model for robots exploring unknown areas, utilizing ROS-based real-time data to create and update NDT efficiently.}
% in terms of timeliness as well as the way to acquire and store data that the robots collect in real time. In this paper, we propose an online NDT model based on a robot when exploring an unknown area and develop a pipeline to exploit ROS-based real time information for creating and updating NDT over time.

%%%%%%%%%%%%%%%%%%%%%%%%%%%%%%%%%%%%%%%%%%%%%%%%%%%%%%%%%%%%%%%%%%%%%%%%%%%%%%%%
\subsection{Architecture Framework}
To build such an NDT to integrate with robotic data and applications and use it for joint network and robotic control, we design the architecture framework depicted in Fig.~\ref{fig:architecture}. It includes the Physical System, Digital Twin Layer, and Digital Twin Application layer. The Digital Twin Layer is the digital representation of the Physical System, which are interconnected with each other in real-time to realize the closed loop for the control of the real work system.  The digital representation is composed of Network Modeling and Physical Environment Modeling. The physical environment modeling creates the digital modelling of the physical environment, including the space, topology, objects, devices, humans, etc. 
The Network Modeling allows for building the virtual representation of the network behavior, as well as estimating the expected network performance for various applications including resource prediction, fault resilience, anomaly detection, and network control, together with robot control or navigation planning. 
The construction of the NDT models can be done by many different means such as using analytical models like queuing theory and network calculus, by means of simulations, or by adopting Machine Learning (ML)-based approaches, e.g., based on Graph Neural Networks (GNN), Deep Reinforcement Learning (DRL) or Deep Neural Network (DNN) models. Building the NDT models relies on the inputs from the collection of real data from various data sources from the Physical world system including monitoring data, system or operation data, or even raw data provided by the robots or UEs and different network and system entities. These data can be stored in the Digital Twin Layer for continuous updates and processing to facilitate the efficient usage of the time-series and large-scale data.  
 \vspace{2mm}
\begin{figure}[t]
\centering
\includegraphics[width=0.80\columnwidth, clip, trim = 4cm 0cm 3cm 1.2cm]{Figures/NDT_concept.pdf}
\caption{Architecture of the Network Digital Twin for the joint network and robot control}
\label{fig:architecture}
 \vspace{-3mm}
\end{figure}