\section{MODALITY INCOMPLETENESS SETTING}
\label{setting}
\begin{figure}[t] 
    \centering
    \includegraphics[width=\linewidth]{Figures/intro.pdf} % 
    \caption{Illustration of Modality Incompleteness Setting. Dash boxes denote the missing modality, while each pair of boxes represents an instance. }
    \label{fig:setting}
\end{figure}

In this study, we focus on two representative brain imaging modalities: PET and MRI, and simulate a more realistic setting by considering both client-level and instance-level modality incompleteness. Each instance is represented as a triplet \( (\mathbf{x}_P, \mathbf{x}_M, y) \), where \( \mathbf{x}_P \) corresponds to data from the PET modality, \( \mathbf{x}_M \) corresponds to data from the MRI modality, and \( y \) represents the associated label. Based on the availability of modalities, data instances are categorized into the following three types, as shown in \Cref{fig:setting}:

\begin{enumerate}
    \item \textbf{PET-only instances:} These instances contain data solely from the PET modality, with no corresponding MRI data. Such instances are formally denoted as \( d_P = (\mathbf{x}_P, \varnothing, y) \).

    \item \textbf{MRI-only instances:} These instances exclusively contain data from the MRI modality, with no accompanying PET data. These are represented as \( d_M = (\varnothing, \mathbf{x}_M, y) \).

    \item \textbf{Multimodal instances:} These instances include data from both modalities, and are represented as \( d_B = (\mathbf{x}_P, \mathbf{x}_M, y) \).
\end{enumerate}

At the client level, we categorize clients into three groups based on the composition of their instances:

\begin{enumerate}
    \item \textbf{PET-only clients:} These clients exclusively host PET-only instances. Their datasets are denoted as \( \mathcal{D}_P = \bigl\{d_P^i\bigr\}_{i=1}^{n} \), where \( n \) is the number of instances.

    \item \textbf{MRI-only clients:} These clients solely host MRI-only instances, denoted as \( \mathcal{D}_M = \bigl\{d_M^i\bigr\}_{i=1}^{n} \).

    \item \textbf{Multimodal clients:} These clients contain a mix of all three types of instances. Their datasets are represented as:
    \[
    \mathcal{D}_B = \bigl\{d_P^i\bigr\}_{i=1}^{\beta_1 n} \cup \bigl\{d_M^i\bigr\}_{i=1}^{\beta_2 n} \cup \bigl\{d_B^i\bigr\}_{i=1}^{(1-\beta_1-\beta_2)n},
    \]
    where \( \beta_1 \) and \( \beta_2 \) indicate the proportions of PET-only and MRI-only instances on multimodal clients, respectively, while \( 1-\beta_1-\beta_2 \) denotes the proportion of multimodal instances on multimodal clients.
\end{enumerate}

For client $i$, we denote the number of instances on this client as $n_i$. The overall dataset is defined as follows:

\begin{equation}
    \mathcal{D} = \bigcup_{i=1}^{\alpha_1 N} \mathcal{D}_P^i \cup \bigcup_{i=1}^{\alpha_2 N} \mathcal{D}_M^i \cup \bigcup_{i=1}^{(1-\alpha_1-\alpha_2) N} \mathcal{D}_B^i,
\end{equation}
where \( \alpha_1 \), \( \alpha_2 \), and \( 1-\alpha_1-\alpha_2 \) represent the proportions of PET-only, MRI-only, and multimodal clients, respectively, and \( N \) denotes the total number of clients participating in the federated learning framework.



% For each client, we denote the total number of instances as \( m \), with \( \mathbf{X}_P \) and \( \mathbf{X}_M \) representing all PET and MRI data on this client, respectively, along with their corresponding labels \( \mathbf{y}_P \) and \( \mathbf{y}_M \). 






