%%%%%%%%%%%%%%%%%%%%%%%%%%%%%%%%%%%%%%%%%%%%%%%%%%%%%%%%%%%%%%%%%%%%%%%%%%%%%%%%
%2345678901234567890123456789012345678901234567890123456789012345678901234567890
%        1         2         3         4         5         6         7         8

\documentclass[letterpaper, 10 pt, conference]{ieeeconf}  % Comment this line out if you need a4paper

%\documentclass[a4paper, 10pt, conference]{ieeeconf}      % Use this line for a4 paper

\IEEEoverridecommandlockouts                              % This command is only needed if 
                                                          % you want to use the \thanks command

\overrideIEEEmargins                                      % Needed to meet printer requirements.
\usepackage{amssymb}
\usepackage{graphicx}
\usepackage{times} % assumes new font selection scheme installed
\usepackage{amsmath} % assumes amsmath package installed
\usepackage{cleveref}
\usepackage{multicol}
\usepackage{multirow}
\usepackage{cite}
\usepackage{balance}
\usepackage[affil-it]{authblk}
\usepackage{xcolor}
\crefformat{figure}{Fig.~#1#2}
\pdfminorversion=4
% \Crefname{figure}{Fig.}{Figs.}
% \crefdefaultlabelformat{#2}

%In case you encounter the following error:
%Error 1010 The PDF file may be corrupt (unable to open PDF file) OR
%Error 1000 An error occurred while parsing a contents stream. Unable to analyze the PDF file.
%This is a known problem with pdfLaTeX conversion filter. The file cannot be opened with acrobat reader
%Please use one of the alternatives below to circumvent this error by uncommenting one or the other
%\pdfobjcompresslevel=0
%\pdfminorversion=4

% See the \addtolength command later in the file to balance the column lengths
% on the last page of the document

% The following packages can be found on http:\\www.ctan.org
% \usepackage{graphics} % for pdf, bitmapped graphics files
% \usepackage{epsfig} % for postscript graphics files
% \usepackage{mathptmx} % assumes new font selection scheme installed

% \usepackage{amssymb}  % assumes amsmath package installed

\title{\LARGE \bf
ClusMFL: A Cluster-Enhanced Framework for Modality-Incomplete Multimodal Federated Learning in Brain Imaging Analysis
}

\author{Xinpeng Wang$^\dagger$, Rong Zhou$^\ddagger$, Han Xie$^\S$, Xiaoying Tang$^{\ast\dagger}$, Lifang He$^\ddagger$ and Carl Yang$^\S$% <-this % stops a space
\thanks{*Xiaoying Tang is the corresponding author.}% <-this % stops a space
\thanks{$^\dagger$Xinpeng Wang and Xiaoying Tang are with the School of Science and Engineering, the Chinese University of Hong Kong, Shenzhen. {\tt\small xinpengwang@link.cuhk.edu.cn}, {\tt\small tangxiaoying@cuhk.edu.cn}}%
\thanks{$^\ddagger$Rong Zhou and Lifang He are with the Department of Computer Science and Engineering, Lehigh University. {\tt\small \{roz322, lih319\}@lehigh.edu}}%
\thanks{$^\S$Han Xie and Carl Yang are with the Department of Computer Science, Emory University. {\tt\small\{han.xie, j.carlyang\}@emory.edu} 
}
}
\begin{document}



\maketitle
\thispagestyle{empty}
\pagestyle{empty}


%%%%%%%%%%%%%%%%%%%%%%%%%%%%%%%%%%%%%%%%%%%%%%%%%%%%%%%%%%%%%%%%%%%%
\begin{abstract}

Multimodal Federated Learning (MFL) has emerged as a promising approach for collaboratively training multimodal models across distributed clients, particularly in healthcare domains. In the context of brain imaging analysis, modality incompleteness presents a significant challenge, where some institutions may lack specific imaging modalities (e.g., PET, MRI, or CT) due to privacy concerns, device limitations, or data availability issues. While existing work typically assumes modality completeness or oversimplifies missing-modality scenarios, we simulate a more realistic setting by considering both client-level and instance-level modality incompleteness in this study. 
Building on this realistic simulation, we propose ClusMFL, a novel MFL framework that leverages feature clustering for cross-institutional brain imaging analysis under modality incompleteness. Specifically, ClusMFL utilizes the FINCH algorithm to construct a pool of cluster centers for the feature embeddings of each modality-label pair, effectively capturing fine-grained data distributions. These cluster centers are then used for feature alignment within each modality through supervised contrastive learning, while also acting as proxies for missing modalities, allowing cross-modal knowledge transfer. Furthermore, ClusMFL employs a modality-aware aggregation strategy, further enhancing the model’s performance in scenarios with severe modality incompleteness. 
We evaluate the proposed framework on the ADNI dataset, utilizing structural MRI and PET scans. 
% We conduct our experiments using the ADNI dataset, incorporating structural MRI and PET scans to evaluate our framework. 
Extensive experimental results demonstrate that ClusMFL achieves state-of-the-art performance compared to various baseline methods across varying levels of modality incompleteness, providing a scalable solution for cross-institutional brain imaging analysis.
\end{abstract}



%%%%%%%%%%%%%%%%%%%%%%%%%%%%%%%%%%%%%%%%%%%%%%%%%%%%%%%%%%%%%%%%%%%%%%%%%%%%%%%%
\section{Introduction}

Probabilistic modeling is at the heart of modern machine learning, where data is often conceptualized as samples drawn from a probability distribution in a high-dimensional space. 
%In this context, a common objective in machine learning is regression, where we predict a subset of variables based on the values of other variables. 
%Whereas, 
In this context, the field of generative modeling, which focuses on drawing new samples from probability distributions given a few samples from the distribution, has witnessed spectacular advancements in recent years, empowering researchers to create synthetic images, videos, and other data with astonishing quality. A concept that has catalyzed significant progress in generative modeling is score-based diffusion modeling \cite{song_generative_2019,song_score-based_2021,ho_denoising_2020}. This technique incrementally introduces noise to the data until it reaches a state of pure noise, all while learning how to reverse this process effectively. 

These new ideas have found applications in the classical sampling problem, wherein the objective is to generate samples from a probability distribution given only its unnormalized density \cite{zhang_path_2022,berner_optimal_2023,vargas_denoising_2022,richter_improved_2023,grenioux_stochastic_2024,huang_reverse_2023}. 
%The classical sampling problem amounts to drawing samples from a probability distribution given its unnormalized probability density. 
In the realm of computational science and statistics, sampling techniques play a pivotal role in various applications ranging from Bayesian inference to machine learning algorithms. The ability to efficiently sample from complex probability distributions underpins the success of numerous computational methodologies, and traditional sampling methods, such as Markov Chain Monte Carlo (MCMC), have been widely employed for this purpose. However, these methods often encounter challenges in high-dimensional spaces or distributions with intricate geometries. Moreover, MCMC methods, which construct a Markov chain with a stationary distribution aligned with the target distribution, often suffer from slow convergence due to long mixing times.

In light of the effectiveness of diffusion processes in generative modeling, there is significant interest in leveraging these processes for sampling.
The aim is to find diffusion processes such that starting with the samples from a tractable distribution such as Gaussian, the diffusion process should produce a sample from the desired distribution at the final time. Unlike conventional MCMC methods, diffusion-based approaches do not require tuning of proposal distributions or acceptance probabilities. Furthermore, diffusion-based methods seem to mitigate the slow convergence of MCMC methods.

Traditional score-based generative modeling generates samples from a target distribution by learning how to reverse a forward diffusion process that maps the target distribution to a prior distribution. Ideally, these diffusion processes require an infinite time horizon for convergence.
%Typically, the forward processes in diffusion models converge exponentially fast to the prior distribution. Consequently, running the diffusion process for only a finite amount of time still provides a reasonably good approximation for sampling from the target distribution. 
In this work, we design computationally efficient diffusion-based samplers using ideas from the \textit{stochastic interpolants} framework, and \textit{forward-backward SDEs}, to generate exact samples from the target distribution within a finite time.
\\[4pt]
%In this work, we study diffusion-based methods for sampling from high-dimensional probability distributions.
%Due to the significant amount of time required for conventional methods to generate samples from the target distribution, there is a pressing need for techniques capable of sampling from probability distributions within \textit{finite time}. 
%In light of the effectiveness of diffusion processes in generative modeling, there is significant interest in using diffusion processes for sampling.
%Traditional score-based generative modeling generates samples from a target distribution by learning how to reverse a forward diffusion process that maps the target distribution to a prior distribution. 
%Typically, these forward processes require an infinite duration to traverse from the target distribution to the prior distribution. Consequently, this method remains approximate, as ideally, simulating the diffusion process for an infinite duration would be necessary. Whereas, diffusion processes capable of generating samples from the target distribution in a finite time remains largely unexplored. Our study precisely focuses on this aspect.
%{\color{blue}Typically, the forward processes in diffusion models converge exponentially fast to the prior distribution. Consequently, running the diffusion process for only a finite amount of time still provides a reasonably good approximation for sampling from the target distribution. 
%Building on this, our study focuses on diffusion processes designed to generate exact samples from the target distribution within finite time.
%Taking a step further, we design computationally efficient diffusion processes using the stochastic interpolant framework that can generate exact samples from the target distribution within a finite time.}
%%%%%%%%%%%%%%%%%%%%%%%%
\textbf{Problem Statement:}
We are interested in sampling from a probability distribution $\pi$ with support on $\R^d$ by constructing diffusion processes which run for a finite time. We are given an unnormalized probability density function $\hat{\pi}$, such that $\pi(x) = \frac{\hat{\pi}(x)}{\int\hat{\pi}(x)dx}$.

\textbf{Notations:}
We denote a Gaussian distribution (and its density) with mean $\mu$ and covariance $\Sigma$ by $\cN{\mu,\Sigma}$. A uniform random variable in the interval $[0,T]$ is denoted by $U([0,T])$. A derivative  of a scalar function $f$ with respect to time $t\in \mathbb{R}_+$ is denoted by $\dot{f}$. A gradient with respect to the space variables $x\in \mathbb{R}^d$ is denoted with $\nabla$, and $\Delta$ denotes the Laplacian. We use $\Vert\cdot\Vert$ for the  Euclidean norm in $\mathbb{R}^d$.
%%%%%%%%%%%%%%%%%%%%%%%%%

\subsection{Our Contributions}

There are infinitely many diffusion processes that can drive samples from a prior distribution to a target distribution in a finite time. However, computationally tractable ways to find and simulate diffusions that have distributions other than Dirac distribution at the initial time and given target distribution at finite time are still unknown to the best of our knowledge. We take a step towards this by providing a principled approach for generating such diffusions when the prior distribution is Gaussian. \emph{Specifically, we propose a class of diffusion-based sampling methods that, starting with samples from the Gaussian distribution, can produce samples from a target distribution in finite time given its unnormalized density.} We achieve this by taking an approach based on the stochastic interpolants framework \cite{albergo_stochastic_2023}. To the best of our knowledge, this is the first time that stochastic interpolants have been used for classical sampling. Our approach reduces the sampling problem to solving Hamilton-Jacobi-Bellman (HJB) equations; a class of well-studied partial differential equations (PDEs) that arise frequently in the field of optimal control. Traditionally, HJB PDEs are solved by minimizing the corresponding control costs \cite{zhang_path_2022}. Instead, for important reasons that will become clear later (see the discussion in Section~\ref{sec:solving_fbsde}), \emph{our approach uses the theory of forward-backward stochastic differential equations (FBSDE) that connects solutions of HJB PDEs (more generally, nonlinear parabolic PDEs) to solutions of a certain set of stochastic differential equations called FBSDEs}. 
%We will explain some of the advantages of this approach later on. 
Moreover, we solve these FBSDEs using machine learning-based methods \cite{han_solving_2018,e_algorithms_2022,raissi_forward-backward_2018}. One of the advantages of our methods is that they allow solving HJB PDEs without the need to compute computationally expensive Neural SDE gradients \cite{zhang_path_2022,berner_optimal_2023}. The techniques that we develop to solve HJB PDEs using FBSDEs can be of independent interest.  
%We start by defining a time dependent density function satisfying the constraints that at $t=0$ it has a density of a Gaussian and at $t=T$, it has the density of the target distribution. Thereafter, we develop methods to implement a stochastic process with above defined marginal density. We then identify that certain quantities related to the density satisfies certain HJB PDEs. We use a method based on FBSDEs and machine learning to solve the PDEs.
%We present here glimpses of the numerical results for our linear interpolant-based sampler. 

\subsection{Our Techniques}
We draw inspiration from the stochastic interpolants framework, which begins with a family of time-indexed random variables defining the density of the diffusion process at each time instant $t$, and then explores methods to realize such a diffusion process. Such a collection of random variables that have desired probability distribution at the time boundaries are known as stochastic interpolants \cite{albergo_stochastic_2023}. 
%We study a specific case known as the linear one-sided stochastic interpolants (linear interpolants in short). 
To illustrate our techniques, in this section we restrict our focus to one of our methods based on \textit{half interpolants}. 
%The half interpolant-based method and another based on linear interpolant will be presented later in detail. 
A half interpolant is a collection of random variables $\{x_t\}_{t\in[0,T]}$ given by $x_t = g(t)x^*+r(t)z,$ for $0\le t\le T$, where $g,r:[0,T]\rightarrow \R_+$ are functions such that $\frac{g}{r}$ is a non-decreasing. Furthermore, $g$ satisfies the boundary condition $g(0)=0$. Here, $x^*\sim\nu$ and $z\sim\cN{0,I_d}$ are independent random variables.

 Our aim is to implement a diffusion process $\{S_t\}_{t\in[0,T]}$ such that the distribution of $S_t$ is same as $x_t$ for all $t\in[0,T]$ and also enforce the constraint that $x_T,S_T\sim\pi$ (thereby implicitly choosing $\nu$). If realized, such a diffusion process can drive samples $S_0\sim\cN{0,r^2(0)I_d}$ to $S_T\sim\pi$. 
 
 Let $\rho(t,\cdot)$ denote the probability density of $x_t$ and let $s(t,x) := \nabla\log{\rho(t,x)}$ denote the so-called score function of density $\rho$. Lemma~\ref{lemma:pde_density} shows that $\rho$ satisfies a ``Fokker-Planck'' PDE given by
\begin{equation}\label{eqn:FPE_tech}
    \partial_t\rho-\frac{\eps^2(t)}{2}\Delta\rho+\nabla\cdot\left(\left(b(t,x)+\frac{\eps^2(t)}{2}s(t,x)\right)\rho\right) = 0,%\quad \rho(0,\cdot) \equiv \cN{0,r^2(0)I_d}.
\end{equation}
with initial condition $\rho(0,\cdot) \equiv \cN{0,r^2(0)I_d}$, where $b(t,x) = \dot g(t)\expectCond{x^*}{x_t=x}-r(t)\dot r(t)s(t,x)$ and $\eps:[0,T]\rightarrow \R_+$ is an arbitrary function.

Equation~\ref{eqn:FPE_tech} suggests that we can realize a process $S_t$ with density $\rho$ by simulating a stochastic differential equation (SDE) given by:
\begin{equation}\label{eqn:SDEforSampling_tech}
   dS_t = \left(b(t,S_t)+\frac{\eps^2(t)}{2}s(t,S_t)\right)dt+\eps(t)dW_t,%;\quad S_0\sim\cN{0,r^2(0)I_d}, 
\end{equation}
with $S_0\sim\cN{0,r^2(0)I_d}$, where $\{W_t\}_{t\in{0,T}}$ is a standard Brownian motion.
Therefore, it suffices to have access to functions $b$ and $s$ (responsible for the drift term) to implement a process $S_t$ that has the same marginal distribution as $x_t$. We will show (Lemma~\ref{lemma:b_and_s}) that both $b$ and $s$ can be expressed in terms of $\expectCond{x^*}{x_t=x}$. Thus, it is sufficient to learn the function $\expectCond{x^*}{x_t=x}$. Towards this, for some $\beta:[0,T]\rightarrow\R_+$, we consider a function $u:[0,T]\times\R^d\rightarrow\R$ given by
\begin{align}\label{eqn:velocity_denf_tech}
    u(t,x) &= \log\frac{\rho(t,\beta(t)x)}{\psi(t,\beta(t)x)}\\ &= \log\int_{\R^d}\nu(x^*)e^{\frac{\beta(t)g(t)}{r^2(t)}<x,x^*>-\frac{g^2(t)}{2r^2(t)}\norm{x^*}^2}dx^*,\nonumber
\end{align}
where $\psi(t,\cdot)$ is the density of the isotropic Gaussian with variance $r^2(t)$. Taking the gradient of (\ref{eqn:velocity_denf_tech}), we note that $\expectCond{x^*}{x_t=x} = \frac{r^2(t)}{\beta(t)g(t)}\nabla u(t,\frac{x}{\beta(t)})$. Hence, a feasible way to obtain $\expectCond{x^*}{x_t=x}$ is to compute $\nabla u$. A direct calculation (Lemma~\ref{lemma:velociyt_HJB_PDE}) shows that $u$ satisfies the following HJB equation:
\begin{equation}\label{eqn:velocity_pde_tech}
    \partial_t u + \frac{\sigma^2}{2}\Delta u + \frac{\sigma^2}{2}\norm{\nabla u}^2-\partial_t\log\left(\frac{\beta(t)g(t)}{r^2(t)}\right)x^T\nabla u = 0,
\end{equation}
where $\sigma^2(t) = 2\frac{r^2(t)}{\beta^2(t)}\partial_t\log\frac{g(t)}{r(t)}$. The condition that $\frac{g}{r}$ is non-decreasing assures that $\sigma^2$ is a positive function. 
Observe that (\ref{eqn:velocity_pde_tech}) is a backward Kolmogorov PDE and can be solved given a terminal condition. To satisfy the constraint $x_T\sim\pi$, we want $\rho(T,\cdot)=\pi(\cdot)$, which gives the terminal condition $u(T, x)= \varphi(x) \equiv \log\frac{\pi(\beta(T)x)}{\psi(T,\beta(T)x)}$. The function $\beta$ is a design parameter, which we can choose such that the coefficients of the PDE are well-defined for $t\in[0,T]$. 

There are several ways to obtain the solution $u$ (more importantly $\nabla u$) of HJB PDE (\ref{eqn:velocity_pde_tech}) under the terminal condition $\varphi$. In the optimal control literature, a prominent approach for solving HJB PDEs involves minimizing the sum of control costs and the terminal cost (see Lemma~\ref{lemma_app:oc_optimization}). %However, for reasons that will become clear later, we opt not to pursue this route. 
Instead, we exploit the connections between the solutions of non-linear PDEs and solutions to the corresponding FBSDE to solve (\ref{eqn:velocity_pde_tech}). The remainder of our approach then relies on machine learning-based techniques to solve an FBSDE associated with the PDE (\ref{eqn:velocity_pde_tech}). Solving the FBSDE gives us access to the function $\nabla u$ on an appropriate domain. Once we have the $\nabla u$, subsequently we obtain functions $b$ and $s$. We then can realize the process $S_t$ using (\ref{eqn:SDEforSampling_tech}). Figure~\ref{fig:trajectory_tech} shows sample trajectories of the diffusion process thus obtained when the target distribution is a mixture of Gaussians.

\begin{figure}
  \centering
  \includegraphics[scale = 0.3]{images/trajectory.png}
  \caption{Sample trajectories of diffusion process for sampling from Gaussian mixture.}
  \label{fig:trajectory_tech}
\end{figure}

\subsection{Related Works}
\paragraph{MCMC methods:} For decades, MCMC has stood as the primary method for sampling from unnormalized densities. Techniques that integrate MCMC with annealing and importance sampling methods have consistently yielded superior results in this domain. Among these, Annealed Importance Sampling (AIS) \cite{neal_annealed_2001} and its Sequential Monte Carlo (SMC) \cite{del_moral_sequential_2006} extensions are widely regarded as state-of-the-art in numerous sampling tasks. Nonetheless, in many practical scenarios, the convergence of these methods can be notably slow. Additionally, analyzing their performance can pose significant challenges, further complicating their application in real-world settings.
\paragraph{Diffusion-based methods:} 
The utilization of diffusions for sampling has been prevalent for a considerable period, with the Langevin diffusion standing out as a prominent example. However, the usage of non-equilibrium dynamics of diffusions for sampling has gained popularity only recently. Noteworthy examples of diffusion-based samplers include Path Integral Sampler (PIS) \cite{zhang_path_2022}, Denoised Diffusion Sampler (DDS) \cite{vargas_denoising_2022}, and time reversed Diffusion Sampler (DIS) \cite{berner_optimal_2023}, Generalized Bridge Sampler (GBS) \cite{richter_improved_2023}, among others. These samplers leverage advancements in machine learning to address an optimization problem, with the solution being a control function that guides samples from a prior density to samples from the target density. For a detailed comparison of the performance of these methods, we refer the reader to \cite{blessing_beyond_2024}. More recently, the concept of time-reversing diffusion processes has been combined with MCMC techniques to develop samplers that do not require training \cite{grenioux_stochastic_2024,huang_reverse_2023}.
\paragraph{Stochastic interpolants:} 
The framework of stochastic interpolants was recently introduced in \cite{albergo_stochastic_2023}. Despite its conceptual simplicity, this framework provides a unified approach to utilizing diffusions for sampling, particularly in generative modeling tasks. Stochastic interpolants play a significant role in the derivation of our methods. In particular, our methods strive to learn certain quantities related to the densities defined by the stochastic interpolants. Subsequently, these quantities are used for sampling.
\paragraph{Schr\"odinger bridge:}
For arbitrary prior and target distributions, the task of finding a diffusion process that maps one to the other can be formulated as an optimization problem known as a the Schr\"odinger bridge problem. Concretely, the dynamical formulation of Schr\"odinger Bridge is the optimization problem $\min_{\mathbb{Q}\in\mathcal{P}(\mathbb{P}_0,\mathbb{P}_T)} D_{\text{KL}}(\mathbb{Q}||\mathbb{P})$, where $\mathcal{P}(\mathbb{P}_0,\mathbb{P}_T)$ is the set of all path measures having density $\mathbb{P}_0$ at $t=0$ and $\mathbb{P}_T$ at $t=T$ and $\mathbb{P}$ is a reference path measure. Solving a Schr\"odinger bridge problem is generally challenging, as it requires solving a set of coupled partial differential equations (PDEs) \cite{chen_likelihood_2021}. However, an instance of the Schrödinger bridge problem that is relatively easier to solve arises when the prior distribution $\mathbb{P}_0$ is a Dirac distribution. In this scenario, the Schrödinger bridge problem reduces to solving a single Hamilton-Jacobi-Bellman (HJB) PDE. The resultant diffusion process is known as a F\"ollmer process \cite{follmer_time_1986}. The PIS \cite{zhang_path_2022} algorithm--a special case of the sampling method we propose--is an implementation of F\"ollmer process.


\section{MODALITY INCOMPLETENESS SETTING}
\label{setting}
\begin{figure}[t] 
    \centering
    \includegraphics[width=\linewidth]{Figures/intro.pdf} % 
    \caption{Illustration of Modality Incompleteness Setting. Dash boxes denote the missing modality, while each pair of boxes represents an instance. }
    \label{fig:setting}
\end{figure}

In this study, we focus on two representative brain imaging modalities: PET and MRI, and simulate a more realistic setting by considering both client-level and instance-level modality incompleteness. Each instance is represented as a triplet \( (\mathbf{x}_P, \mathbf{x}_M, y) \), where \( \mathbf{x}_P \) corresponds to data from the PET modality, \( \mathbf{x}_M \) corresponds to data from the MRI modality, and \( y \) represents the associated label. Based on the availability of modalities, data instances are categorized into the following three types, as shown in \Cref{fig:setting}:

\begin{enumerate}
    \item \textbf{PET-only instances:} These instances contain data solely from the PET modality, with no corresponding MRI data. Such instances are formally denoted as \( d_P = (\mathbf{x}_P, \varnothing, y) \).

    \item \textbf{MRI-only instances:} These instances exclusively contain data from the MRI modality, with no accompanying PET data. These are represented as \( d_M = (\varnothing, \mathbf{x}_M, y) \).

    \item \textbf{Multimodal instances:} These instances include data from both modalities, and are represented as \( d_B = (\mathbf{x}_P, \mathbf{x}_M, y) \).
\end{enumerate}

At the client level, we categorize clients into three groups based on the composition of their instances:

\begin{enumerate}
    \item \textbf{PET-only clients:} These clients exclusively host PET-only instances. Their datasets are denoted as \( \mathcal{D}_P = \bigl\{d_P^i\bigr\}_{i=1}^{n} \), where \( n \) is the number of instances.

    \item \textbf{MRI-only clients:} These clients solely host MRI-only instances, denoted as \( \mathcal{D}_M = \bigl\{d_M^i\bigr\}_{i=1}^{n} \).

    \item \textbf{Multimodal clients:} These clients contain a mix of all three types of instances. Their datasets are represented as:
    \[
    \mathcal{D}_B = \bigl\{d_P^i\bigr\}_{i=1}^{\beta_1 n} \cup \bigl\{d_M^i\bigr\}_{i=1}^{\beta_2 n} \cup \bigl\{d_B^i\bigr\}_{i=1}^{(1-\beta_1-\beta_2)n},
    \]
    where \( \beta_1 \) and \( \beta_2 \) indicate the proportions of PET-only and MRI-only instances on multimodal clients, respectively, while \( 1-\beta_1-\beta_2 \) denotes the proportion of multimodal instances on multimodal clients.
\end{enumerate}

For client $i$, we denote the number of instances on this client as $n_i$. The overall dataset is defined as follows:

\begin{equation}
    \mathcal{D} = \bigcup_{i=1}^{\alpha_1 N} \mathcal{D}_P^i \cup \bigcup_{i=1}^{\alpha_2 N} \mathcal{D}_M^i \cup \bigcup_{i=1}^{(1-\alpha_1-\alpha_2) N} \mathcal{D}_B^i,
\end{equation}
where \( \alpha_1 \), \( \alpha_2 \), and \( 1-\alpha_1-\alpha_2 \) represent the proportions of PET-only, MRI-only, and multimodal clients, respectively, and \( N \) denotes the total number of clients participating in the federated learning framework.



% For each client, we denote the total number of instances as \( m \), with \( \mathbf{X}_P \) and \( \mathbf{X}_M \) representing all PET and MRI data on this client, respectively, along with their corresponding labels \( \mathbf{y}_P \) and \( \mathbf{y}_M \). 







\section{METHOD}
\begin{figure*}[htbp]
    \centering
    \includegraphics[width=\textwidth]{Figures/Method_Overview.png} % 使用 \textwidth 来指定图片宽度为双栏宽度
    \caption{Overview of ClusMFL. In this figure, PET-only instances are used as examples of single-modality instances in local training. Different patterns represent different modalities, and different colors indicate different labels.}  
    \label{fig:overview}
\end{figure*}
\subsection{Preliminary}
\label{sec: preliminary}
In this study, we adopt a typical architecture for multimodal models, which includes two encoders—one for each modality—and a classifier. Let \( f_P \) and \( f_M \) denote the encoders for PET modality and MRI modality, respectively. The encoders \( f_P \) and \( f_M \) are responsible for extracting the relevant features from each modality. The model also includes a classifier, denoted as \( g \), which concatenates the embeddings from the encoders and performs the final prediction.

During the inference stage, for instances containing both modalities, the input data \( \mathbf{x}_P \) and \( \mathbf{x}_M \) are processed through their respective encoders \( f_P \) and \( f_M \). Specifically, the feature embeddings are obtained as \( \mathbf{z}_P = f_P(\mathbf{x}_P) \) and \( \mathbf{z}_M = f_M(\mathbf{x}_M)\), which are subsequently concatenated and passed to the classifier \( g \) for the final prediction, \( \hat{y} = g(\mathbf{z}_P, \mathbf{z}_M) \). For instances with only a single modality, the feature embedding of the missing modality is replaced with a tensor of zeros, denoted as \( \mathbf{0} \). 
% For example, in the case of PET-only instances, the feature embedding \( \mathbf{z}_M \) is replaced by a tensor of zeros, \textit{i.e.}, \( \mathbf{z}_M = \mathbf{0} \).

The federated training process repeats the construction of the pool of cluster centers and cluster sizes (\textit{i.e.}, \( \mathbf{C}^{\text{global}} \) and \( \mathbf{S}^{\text{global}} \)), followed by local training and aggregation in each round, which are explained in detail in the following sections. We provide an overview of the construction of  \( \mathbf{C}^{\text{global}} \) and \( \mathbf{S}^{\text{global}} \) and local training in \Cref{fig:overview}.

\subsection{Constructing The Pool of Cluster Centers}
Unlike traditional prototype learning methods, which use the mean of features as the prototype and result in a shift between individual samples and the prototype, this study adopts the FINCH \cite{FINCH} algorithm to construct a pool of cluster centers, thereby more effectively representing the clients' data distribution.
% Unlike traditional prototype learning methods, which fail to adequately represent the data distribution due to the significant shift between individual samples and the prototype, this study employs the FINCH algorithm to cluster feature embeddings and more accurately capture the underlying data distribution.

On each client, we begin by applying the FINCH clustering algorithm to calculate the local cluster centers for feature embeddings of each pair of modality and label. The resulting cluster centers, along with cluster sizes, are then uploaded to the server. For instance, consider the modality PET. The feature embeddings associated with modality PET are extracted as \( \mathbf{Z}_P = f_P(\mathbf{X}_P) \), where $\mathbf{X}_P$ represents all PET data on this client. The FINCH algorithm is then applied to \( \mathbf{Z}_P \) for clustering, yielding the corresponding cluster centers and cluster sizes. Specifically, for label \( j \) in modality PET, FINCH identifies the cluster centers as:
\begin{equation}
    (\mathbf{C}_{P,j},\mathbf{S}_{P,j})= \text{FINCH}(\mathbf{Z}_{P,j}),
\end{equation}
where \( \mathbf{Z}_{P,j} \) denotes the feature embeddings with modality PET corresponding to label \( j \), and \( \mathbf{C}_{P,j} = ( \mathbf{c}_{P,j}^{1}, \mathbf{c}_{P,j}^{2}, \dots, \mathbf{c}_{P,j}^{K_{P,j}} )\) represents the \( K_{P,j} \) cluster centers obtained from the FINCH algorithm for label \( j \). The set \( \mathbf{S}_{P,j} =( s_{P,j}^{1}, s_{P,j}^{2}, \dots, s_{P,j}^{K_{P,j}} ) \) represents the sizes of the corresponding clusters, where each \( s_{P,j}^{k} \) denotes the number of feature embeddings assigned to the \( k \)-th cluster for label \( j \), with \( \mathbf{c}_{P,j}^{k} \) being the cluster center. $K_{P,j}$ represents the number of clusters, which is determined automatically by FINCH.

For client \(i\), the cluster centers and sizes associated with modality PET and label \(j\) are denoted as \(\mathbf{C}_{P,j}^i\) and \(\mathbf{S}_{P,j}^i\), respectively. If client \(i\) lacks data for modality PET, we set \(\mathbf{C}_{P,j}^i = \varnothing\) and \(\mathbf{S}_{P,j}^i = \varnothing\).

Once the cluster centers and cluster sizes for each client are computed, they are sent to the server, where they are collected to form a global pool of cluster centers. Specifically, the global pool of cluster centers for label \( j \) in modality PET, denoted as \( \mathbf{C}_{P,j}^{\text{global}} \), is constructed by concatenating the cluster centers from all clients:

\begin{equation}
    \mathbf{C}_{P,j}^{\text{global}} = \bigoplus_{i=1}^N \mathbf{C}_{P,j}^i.
\end{equation}
Similarly, the corresponding cluster sizes for label \( j \) are cancatenated into a global pool, denoted as \( \mathbf{S}_{P,j}^{\text{global}} \):

\begin{equation}
    \mathbf{S}_{P,j}^{\text{global}} = \bigoplus_{i=1}^N \mathbf{S}_{P,j}^i.
\end{equation}

The global pools \( \mathbf{C}_{P,j}^{\text{global}} \) and \( \mathbf{S}_{P,j}^{\text{global}} \) encapsulate the aggregated cluster centers and their respective sizes for each label \( j \) across all clients, enabling the model to leverage a comprehensive representation of the data distribution. These global pools are then distributed back to each client, allowing them to utilize the global information during local training.

The same process is performed for MRI modality. Let \( \mathbf{C}_{M,j}^i \) and \( \mathbf{S}_{M,j}^i \) represent the cluster centers and cluster sizes, respectively, for MRI and label \( j \) on client \( i \). The global pools for MRI are constructed as:

\begin{equation}
    \mathbf{C}_{M,j}^{\text{global}} = \bigoplus_{i=1}^N \mathbf{C}_{M,j}^i, \quad \mathbf{S}_{M,j}^{\text{global}} = \bigoplus_{i=1}^N \mathbf{S}_{M,j}^i.
\end{equation}

These global pools for MRI are also distributed back to the clients to ensure that the global knowledge from both modalities is available for further training and optimization.

\subsection{Feature Alignment}

To ensure that the encoders \( f_M \) and \( f_P \) extract the correct modality-specific features and mitigate overfitting under severe modality incompleteness, we combine global cluster centers and local feature embeddings and apply supervised contrastive loss for feature alignment.

For any multimodal client with dataset $\mathcal{D}_B$, let \( \mathbf{X}_P \) and \( \mathbf{X}_M \) represent all PET and MRI data on this client, respectively, along with their corresponding labels \( \mathbf{y}_P \) and \( \mathbf{y}_M \). The feature embeddings for PET modality and MRI modality are computed as \( \mathbf{Z}_P = f_P(\mathbf{X}_P) \) and  \( \mathbf{Z}_M = f_M(\mathbf{X}_M) \) respectively. To incorporate global information, the feature embeddings \( \mathbf{Z}_P \) are concatenated with the global cluster centers \( \mathbf{C}_{P}^{\text{global}} \):
\begin{equation}
    \mathbf{Z}_{P,G} = \mathbf{Z}_P \oplus \bigoplus_{j=1}^J \mathbf{C}_{P,j}^{\text{global}},
\end{equation}
where \( J \) is the total number of unique labels. Similarly, the label \( \mathbf{y}_P \) is extended as:
\begin{equation}
    \mathbf{y}_{P,G} = \mathbf{y}_P \oplus \bigoplus_{j=1}^J (j )^{|\mathbf{C}_{P,j}^{\text{global}}|},
\end{equation}
where \( ( j )^{|\mathbf{C}_{P,j}^{\text{global}}|} \) denotes a list containing label \( j \) repeated \( |\mathbf{C}_{P,j}^{\text{global}}| \) times.

We compute the supervised contrastive loss over \( \mathbf{Z}_{P,G} \). Let \( \mathcal{I} = \{ 1, 2, \dots, |\mathbf{y}_{P,G}| \} \) denote the index set, and define \( \mathcal{A}(i) = \mathcal{I} \backslash \{ i \} \) as the set of all indices excluding \( i \). The supervised contrastive loss for PET modality is defined as:
\begin{equation}
    \mathcal{L}_{\text{ctr}}^P(\mathbf{X}_P) = \mathbb{E}_{i \in \mathcal{I}} \left[ -\frac{1}{|\mathcal{P}(i)|} \sum_{p \in \mathcal{P}(i)} \log \frac{v_{i,p}}{\sum_{a \in \mathcal{A}(i)} v_{i,a}} \right],
\end{equation}
where:
\begin{itemize}
    \item \( \mathcal{P}(i) = \{ p \in \mathcal{A}(i) \mid \mathbf{y}_{P,G}^{(p)} = \mathbf{y}_{P,G}^{(i)} \} \) represents the set of indices corresponding to positive examples for \( \mathbf{Z}_{P,G}^{(i)} \).
    \item \( v_{i,p} = \exp(\text{sim}(\mathbf{Z}_{P,G}^{(i)}, \mathbf{Z}_{P,G}^{(p)}) / \tau) \), where \( \text{sim}(\cdot, \cdot) \) denotes the cosine similarity between two embeddings, and \( \tau \) is a temperature parameter.
\end{itemize}

The similar process is applied to calculate \( \mathcal{L}_{\text{ctr}}^M(\mathbf{X}_M) \) and the overall supervised contrastive loss for $\mathcal{D}_B$ is then computed as:
\begin{equation}
    \mathcal{L}_{\text{CTR}} (\mathcal{D}_B)= 
    \frac{
        |\mathbf{y}_P| \cdot \mathcal{L}_{\text{ctr}}^P(\mathbf{X}_P) 
        + 
        |\mathbf{y}_M| \cdot \mathcal{L}_{\text{ctr}}^M(\mathbf{X}_M)
    }{
        |\mathbf{y}_P| + |\mathbf{y}_M|
    }.
\end{equation}

For clients with data from only one modality, the loss is computed solely for the corresponding modality, without involving the other modality. 
% For example, for a PET-only client, the overall supervised contrastive loss \( \mathcal{L}_{\text{CTR}}(\mathcal{D}_P) \) is simply given by \( \mathcal{L}_{\text{ctr}}^P(\mathbf{X}_P) \).

\subsection{Modality Completion Loss}
To further enhance the model's classification performance in the presence of missing modalities, we use $\mathbf{C}^{\mathrm{global}}$ as approximations for feature embeddings of the missing modality. Specifically, for a PET-only instance with data \( d_P = (\mathbf{x}_P, \varnothing, y) \) or an MRI-only instance with data \( d_M = (\varnothing, \mathbf{x}_M, y) \), we first pass the available modality through the corresponding encoder to obtain the feature embedding \( \mathbf{z}_P \) and \( \mathbf{z}_M \). We then approximate the missing modality using the \( i \)-th cluster center from the corresponding global cluster centers. The prediction for each instance is given by:
\begin{equation}
\hat{y}_c^i = \begin{cases}
g(\mathbf{z}_P, \mathbf{C}_{M,y}^{\text{global},(i)}) & \text{for PET-only instance}, \\
g(\mathbf{C}_{P,y}^{\text{global},(i)}, \mathbf{z}_M) & \text{for MRI-only instance},
\end{cases}
\end{equation}
where \( \mathbf{C}_{M,y}^{\text{global},(i)} \) and \( \mathbf{C}_{P,y}^{\text{global},(i)} \) represents the \( i \)-th cluster center from \( \mathbf{C}_{M,y}^{\text{global}} \) and \( \mathbf{C}_{P,y}^{\text{global}} \) respectively.

The loss for a PET-only instance is then computed as:
\begin{equation}
\mathcal{L}_{\text{mc}}(d_P) =  \frac{1}{T_{M,y}} \sum_{i=1}^{|\mathbf{S}_{M,y}^{\text{global}}|} \mathbf{S}_{M,y}^{\text{global},(i)} \cdot \mathcal{L}_{\text{CE}}(\hat{y}_c^i, y),
\end{equation}
where \( \mathcal{L}_{\text{CE}} \) denotes the cross-entropy loss function, and \( T_{M,y} = \sum_{i=1}^{|\mathbf{S}_{M,y}^{\text{global}}|} \mathbf{S}_{M,y}^{\text{global},(i)} \).

Similarly, for an MRI-only instance, the loss is computed as:
\begin{equation}
\mathcal{L}_{\text{mc}}(d_M) =  \frac{1}{T_{P,y}} \sum_{i=1}^{|\mathbf{S}_{P,y}^{\text{global}}|} \mathbf{S}_{P,y}^{\text{global},(i)} \cdot \mathcal{L}_{\text{CE}}(\hat{y}_c^i, y),
\end{equation}
where \( T_{P,y} = \sum_{i=1}^{|\mathbf{S}_{P,y}^{\text{global}}|} \mathbf{S}_{P,y}^{\text{global},(i)} \).

For any client $i$, let \( \mathcal{D}_i^{\text{single}} = \{ d \mid d \in \mathcal{D}_i, \, d \text{ is a single modality instance} \} \) be the subset of instances with a single modality (either PET-only or MRI-only). The overall modality completion loss is calculated as:
\begin{equation}
    \mathcal{L}_{\text{MC}}(\mathcal{D}_i)=\mathbb{E}_{d\in\mathcal{D}_{\text{single}}}\left[\mathcal{L}_{\text{mc}}(d)\right].
\end{equation}

In this manner, we use the global cluster centers of the available modality to approximate the missing modality, allowing the model to still make meaningful predictions in the presence of incomplete data.



\subsection{Overall Loss Function}

For client \( i \) with dataset \( \mathcal{D}_i \), the prediction \( \hat{y}_i \) is generated as described in \Cref{sec: preliminary} for each instance \( d_i \). Let \( \hat{\mathbf{y}} = (\hat{y}_1, \hat{y}_2, \dots, \hat{y}_{n_i}) \) denote the vector of predicted labels for the \( m \) instances in the client's dataset, and \( \mathbf{y} = (y_1, y_2, \dots, y_{n_i}) \) denote the corresponding true labels. The overall loss is computed as:

\begin{equation}
\mathcal{L}(\mathcal{D}_i) = \mathcal{L}_{\text{CE}}(\hat{\mathbf{y}}, \mathbf{y}) + \lambda_1 \mathcal{L}_{\text{CTR}}(\mathcal{D}_i) + \lambda_2 \mathcal{L}_{\text{MC}}(\mathcal{D}_i),
\end{equation}
where \( \lambda_1 \) and \( \lambda_2 \) are the regularization coefficients that balance the contributions of the contrastive loss and the modality completion loss, respectively.



\subsection{Modality-Aware Aggregation}

In this section, we describe the modality-aware aggregation (MAA) method adopted in our framework. Specifically, we use different aggregation weights for different modules within the model, based on the number of instances for each modality at each client.

Let \( n_P^i \) represent the number of instances with the PET modality at client \( i \), and \( n_M^i \) represent the number of instances with the MRI modality at client \( i \). The total numbers of instances with PET and MRI modalities across all clients are denoted as \( n^{\text{total}}_P \) and \( n^{\text{total}}_M \), and are computed as:
\begin{equation}
n^{\text{total}}_P = \sum_{i=1}^{N} n_P^i, \quad n^{\text{total}}_M = \sum_{i=1}^{N} n_M^i.    
\end{equation}

For client \( i \), the aggregation weights for the encoders are computed as:
\begin{equation}
w_P^i = \frac{n_P^i}{n_{\text{total}}^P}, \quad w_M^i = \frac{n_M^i}{n_{\text{total}}^M},
\end{equation}
where \( w_P^i \) and \( w_M^i \) are the aggregation weights for encoder $f_P$ and encoder $f_M$, respectively. 

The aggregation processes for encoders $f_P$ and $f_M$ are as follows:
\begin{equation}
    \mathbf{f}_P^{\text{global}} = \sum_{i=1}^{N} w_P^i \mathbf{f}_P^i, \quad
\mathbf{f}_M^{\text{global}} = \sum_{i=1}^{N} w_M^i \mathbf{f}_M^i,
\end{equation}
where \( \mathbf{f}_P^i \), \( \mathbf{f}_M^i \) are the model parameters of the PET encoder and MRI encoder, respectively.


For the classifier \( g \), the aggregation process is given by:
\begin{equation}
    \mathbf{g}^{\text{global}} = \frac{1}{\sum_{i=1}^{N}n_i}\sum_{i=1}^{N} n_i\cdot\mathbf{g}^i,
\end{equation}
where \( \mathbf{g}^i \) denotes the model parameters for classifier $g$.

% This approach ensures that the model updates are more accurately representative of the data distribution at each client, while assigning different importance to the PET and MRI modalities based on the number of instances available at each client.

% \subsection{Communication Complexity Analysis}
               
\section{EXPERIMENT}
\subsection{Experimental Setup}
\begin{table*}[t]
\resizebox{\textwidth}{!}{
\begin{tabular}{cccccccccccc}
\hline
\multirow{2}{*}{$\alpha$} & \multirow{2}{*}{Method} & \multicolumn{5}{c}{$\beta = 0.2$}                                                                                                         & \multicolumn{5}{c}{$\beta = 0.4$}                                                                                     \\ \cline{3-12} 
                          &                         & Precision                 & Recall                    & F1                        & Accuracy                  & AUC                       & Precision             & Recall                & F1                    & Accuracy              & AUC                   \\ \hline
\multirow{7}{*}{0.2}      & FedAvg                  & 53.94 $\pm$ 3.56          & 54.05 $\pm$ 3.53          & 53.21 $\pm$ 3.90          & 53.41 $\pm$ 3.71          & 69.39 $\pm$ 2.05          & 53.58 $\pm$ 2.34      & 53.80 $\pm$ 2.77      & 52.30 $\pm$ 2.32      & 52.64 $\pm$ 2.14      & 68.20 $\pm$ 2.72      \\
                          & FedProx                 & 55.81 $\pm$ 1.58          & 54.37 $\pm$ 5.19          & 54.32 $\pm$ 2.21          & 54.62 $\pm$ 2.18          & 68.29 $\pm$ 2.93          & 54.39 $\pm$ 2.69      & 54.36 $\pm$ 1.82      & 53.27 $\pm$ 2.99      & 53.52 $\pm$ 2.97      & 68.11 $\pm$ 2.74      \\
                          & FedMed-GAN              & 54.62 $\pm$ 1.02          & 55.18 $\pm$ 2.73          & 53.46 $\pm$ 0.97          & 53.52 $\pm$ 1.07          & 68.80 $\pm$ 2.16          & 53.87 $\pm$ 3.66      & 54.66 $\pm$ 3.57      & 53.35 $\pm$ 3.61      & 53.41 $\pm$ 3.63      & 68.09 $\pm$ 2.53      \\
                          & FedMI                   & 54.77 $\pm$ 2.94          & 55.22 $\pm$ 4.36          & 54.26 $\pm$ 2.82          & 54.40 $\pm$ 2.83          & 69.15 $\pm$ 2.73          & 54.03 $\pm$ 3.88      & 54.46 $\pm$ 3.56      & 53.54 $\pm$ 4.22      & 53.63 $\pm$ 4.14      & 67.61 $\pm$ 3.93      \\
                          & MFCPL                   & 54.54 $\pm$ 1.88          & \textbf{56.06 $\pm$ 4.14} & 53.88 $\pm$ 1.64          & 54.07 $\pm$ 1.93          & 69.39 $\pm$ 2.71          & 53.86 $\pm$ 5.42      & 54.01 $\pm$ 5.44      & 53.22 $\pm$ 5.16      & 53.30 $\pm$ 5.34      & 67.58 $\pm$ 3.67      \\
                          & PmcmFL                  & 53.17 $\pm$ 1.60          & 52.73 $\pm$ 3.50          & 52.58 $\pm$ 1.81          & 52.64 $\pm$ 2.03          & 67.51 $\pm$ 4.07          & 49.03 $\pm$ 1.55      & 48.01 $\pm$ 3.03      & 48.38 $\pm$ 2.00      & 48.46 $\pm$ 2.31      & 63.71 $\pm$ 2.50      \\
                          & ClusMFL                    & \textbf{57.16 $\pm$ 2.32}     & 54.73 $\pm$ 3.93              & \textbf{56.56 $\pm$ 2.36}     & \textbf{56.92 $\pm$ 2.41}     & \textbf{72.81 $\pm$ 3.64}     & \textbf{56.06 $\pm$ 1.31} & \textbf{55.44 $\pm$ 2.19} & \textbf{55.38 $\pm$ 1.07} & \textbf{55.49 $\pm$ 1.10} & \textbf{72.50 $\pm$ 2.02} \\ \hline
\multirow{7}{*}{0.4}      & FedAvg                  & 53.75 $\pm$ 3.86          & 54.26 $\pm$ 3.76          & 52.76 $\pm$ 4.64          & 53.08 $\pm$ 4.34          & 67.91 $\pm$ 3.48          & 52.60 $\pm$ 3.00      & 53.95 $\pm$ 3.66      & 51.84 $\pm$ 3.06      & 51.98 $\pm$ 3.07      & 67.03 $\pm$ 3.22      \\
                          & FedProx                 & 54.36 $\pm$ 3.50          & 54.28 $\pm$ 4.17          & 53.77 $\pm$ 3.48          & 53.96 $\pm$ 3.51          & 68.26 $\pm$ 3.53          & 52.16 $\pm$ 2.20      & 52.93 $\pm$ 3.76      & 51.35 $\pm$ 2.13      & 51.54 $\pm$ 2.07      & 67.43 $\pm$ 3.64      \\
                          & FedMed-GAN              & 52.97 $\pm$ 1.63          & 52.89 $\pm$ 2.41          & 52.43 $\pm$ 1.74          & 52.53 $\pm$ 1.63          & 67.25 $\pm$ 3.45          & 53.56 $\pm$ 3.57      & 54.02 $\pm$ 3.95      & 52.54 $\pm$ 3.97      & 52.86 $\pm$ 3.63      & 67.91 $\pm$ 2.93      \\
                          & FedMI                   & 55.40 $\pm$ 2.85          & 53.82 $\pm$ 2.45          & 54.84 $\pm$ 2.75          & 54.95 $\pm$ 2.69          & 68.59 $\pm$ 2.48          & 54.01 $\pm$ 2.07      & 52.36 $\pm$ 1.98      & 53.62 $\pm$ 2.14      & 53.85 $\pm$ 2.06      & 66.70 $\pm$ 2.63      \\
                          & MFCPL                   & 55.07 $\pm$ 4.34          & 54.66 $\pm$ 3.93          & 54.34 $\pm$ 4.37          & 54.51 $\pm$ 4.57          & 66.60 $\pm$ 2.58          & 52.84 $\pm$ 4.14      & 54.69 $\pm$ 3.92      & 51.70 $\pm$ 3.89      & 51.87 $\pm$ 3.74      & 67.81 $\pm$ 3.48      \\
                          & PmcmFL                  & 51.71 $\pm$ 5.54          & 49.80 $\pm$ 6.21          & 50.77 $\pm$ 5.21          & 50.88 $\pm$ 5.42          & 65.54 $\pm$ 6.22          & 52.45 $\pm$ 3.96      & 50.03 $\pm$ 5.74      & 50.40 $\pm$ 4.72      & 50.77 $\pm$ 4.66      & 63.31 $\pm$ 5.17      \\
                          & ClusMFL                    & \textbf{56.59 $\pm$ 4.53} & \textbf{54.68 $\pm$ 4.49} & \textbf{56.17 $\pm$ 4.08} & \textbf{56.37 $\pm$ 4.10} & \textbf{73.25 $\pm$ 4.11} & \textbf{54.83 $\pm$ 6.13} & \textbf{54.88 $\pm$ 6.09} & \textbf{54.22 $\pm$ 5.89} & \textbf{54.40 $\pm$ 6.03} & \textbf{72.22 $\pm$ 4.47} \\ \hline
\end{tabular}
}
\caption{Performance Comparison Across Different Federated Learning Methods (Mean $\pm$ Standard Deviation \%) under Different Settings.}
\label{main result}
\end{table*}

\begin{figure*}[h]
    \centering
    \includegraphics[width=\linewidth]{Figures/convergence.png}
    \caption{Training curves of different methods.}
    \label{fig:training curve}
\end{figure*}
% dataset description
% In this study, we use AV45-PET and VBM-MRI modality.
The brain imaging dataset used in this study is sourced from the Alzheimer's Disease Neuroimaging Initiative (ADNI) public repository~\cite{mueller2005alzheimer} and comprises 915 participants stratified into three diagnostic categories: 297 healthy controls (HC), 451 mild cognitive impairment (MCI), and 167 Alzheimer's disease (AD) patients. Multimodal neuroimaging acquisitions encompass structural Magnetic Resonance Imaging (VBM-MRI) and 18 F-florbetapir PET (AV45-PET), enabling examination of brain structure and amyloid plaque deposition, respectively.

Consistent with established neuroimaging processing pipelines \cite{barshan2015stage,zhu2010graphene}, we preprocess neuroimaging data to region-of-interest (ROI) features from each participant’s images. 
First, the multi-modality imaging scans are aligned to each participant's same visit. All imaging scans are aligned to a T1-weighted template image. 
Subsequently, the images are segmented into gray matter (GM), white matter (WM) and cerebrospinal fluid (CSF) maps. 
They are normalized to the standard Montreal Neurological Institute (MNI) space as $2 \times 2 \times 2$ mm$^3$ voxels, being smoothed with an $8$ mm full-width at half-maximum (FWHM) Gaussian kernel.
We preprocess the structural MRI scans with voxel-based morphometry (VBM) by using the SPM software \cite{ashburner2000voxel}, and register the AV45-PET scans to the MNI space by SPM. 
For both MRI and PET scans, we parcellate the brain into 90 ROIs (excluding the cerebellum and vermis) based on the AAL-90 atlas \cite{tzourio2002automated}, and computed ROI-level measures by averaging voxel-wise values within each region.

In the original dataset, each instance contains both modalities. We first split the dataset into a training set and a test set with a ratio of 1:4. For the test set, we ensure that the proportions of the three types of instances are equal, \textit{i.e.}, each type accounts for $\frac{1}{3}$ of the total. 
In the federated learning setup, we distribute the training instances equally across clients while preserving label distribution. Each instance is then assigned a type (multimodal or single-modality) based on the MFL settings controlled by $\alpha_1$, $\alpha_2$, $\beta_1$, and $\beta_2$ as described in \Cref{setting}. Single-modality instances drop the corresponding modality. For simplicity, we set $\alpha_1 = \alpha_2 = \alpha$ and $\beta_1 = \beta_2 = \beta$.

% baselinese
To evaluate the effectiveness of our proposed method, we compare it against several baseline approaches, including FedAvg \cite{FedAvg}, FedProx \cite{FedProx}, FedMed-GAN \cite{FedMed-GAN}, FedMI \cite{FedMI}, MFCPL \cite{MFCPL}, and PmcmFL \cite{PmcmFL}. FedAvg and FedProx are traditional federated learning algorithms, with FedAvg employing simple parameter averaging, while FedProx introduces a proximal term to the objective to mitigate client heterogeneity. FedMed-GAN employs CycleGAN \cite{cyclegan} to complete missing modalities, enhancing diagnosis accuracy. FedMI, MFCPL, and PmcmFL leverage prototype learning to model class distributions and align features, effectively addressing incomplete modalities and heterogeneous data across clients. All methods are evaluated under the same experimental setup to ensure a fair comparison.

% inplementation details
In our experiment, we set the number of clients to $N=10$, with the number of communication rounds fixed at 30 and each client performing 10 local training epochs. For optimization, we employ the Adam \cite{adam} optimizer with an initial learning rate of 0.01. To dynamically adjust the learning rate during training, we use cosine annealing strategy. We conduct 5-fold cross-validation and report the results as the mean $\pm$ standard deviation across all folds. For evaluation metrics, we adopt a weighted average for precision, F1-score, and ROC-AUC, while using a macro average for recall.

\subsection{Main Results}
\Cref{main result} presents a comprehensive comparison of the performance of our proposed method against several baseline approaches. The experiments were conducted under varying configurations of $\beta$ and $\alpha$, which denote modality incompleteness and client diversity, respectively.

As indicated in \Cref{main result}, our method consistently outperforms the baseline approaches across different settings of modality incompleteness, achieving superior results in terms of precision, recall, F1 score, accuracy, and AUC. Notably, certain algorithms specifically designed for modality-incomplete MFL fail to outperform traditional federated learning methods, such as FedProx, in some scenarios.

Moreover, as the values of $\alpha$ and $\beta$ increase—corresponding to a higher proportion of instances with only a single modality—most of the baseline algorithms exhibit a noticeable decline in performance. In contrast, our proposed method demonstrates stable and robust performance, highlighting its effectiveness and resilience in handling varying degrees of modality incompleteness.



\subsection{Ablation Study}

 In order to assess the contributions of each component, we conduct an ablation study under the setting of $\alpha = 0.4$ and $\beta = 0.2$, as shown in Table \ref{table:ablation study}. The results demonstrate that applying modality-aware aggregation (MAA) alone yields lower performance across all metrics. Incorporating the contrastive loss ($\mathcal{L}_{\text{CTR}}$) improves the results significantly, while the modality completion loss ($\mathcal{L}_{\text{MC}}$) also leads to moderate gains. Combining both losses without MAA further enhances performance. The full method, which includes both MAA and the two loss functions, achieves the highest performance, with precision, recall, F1-score, accuracy, and AUC all showing notable improvements. These findings highlight the effectiveness of combining modality-aware aggregation with the contrastive and modality completion losses in addressing modality incompleteness in MFL.
\begin{table}[t!]
\centering
\scriptsize
\caption{The ablation study results of \name using LLaMA3-8B-Instruct as the base model on the TFV task. \textcolor{Maroon}{Red} signifies degradation in percentage.}
\resizebox{\columnwidth}{!}{
\begin{tabular}{lllll}
\toprule
 \multirow{2}{*}[-0.5ex]{\textbf{Methods}} & \multicolumn{4}{c}{\textbf{\;\;TFV}} \\
\cmidrule(l){2-5}
 &\bf\%Acc. &\bf\%F1 &\bf\%Prec. &\bf\%Recall \\
\midrule
 \rowcolor{myblue} \textbf{\name (Ours)} &77.20  \basex{0.00} &77.46 \basex{0.00} &79.98 \basex{0.00} &77.20  \basex{0.00}  \\ 
  % w/o LoRA &- \downbad{0.0} &- \downbad{0.0} &- \downbad{0.0} &- \downbad{0.0} \\
  w/o PHL &70.96 \downbad{6.24} &70.86 \downbad{6.60} &71.35 \downbad{8.63} &70.96 \downbad{6.24}\\
  w/o PHL, w/ HGNN &72.70 \downbad{4.50} &72.54 \downbad{4.92} &73.03 \downbad{6.95} &72.70 \downbad{4.50}\\
  w/o Inquiry Emb. & 72.63 \downbad{4.57} & 73.39 \downbad{4.07} & 74.22 \downbad{5.76} & 74.22 \downbad{2.98}\\

\bottomrule
\end{tabular}}
\label{tab:ablation_study}
\vspace{-0.1in}
\end{table}

\subsection{Convergence Efficiency}

To better analyze the convergence behavior of different algorithms in modality-incompleteness scenarios, we conduct experiments with four different random seeds in the setting of the first fold, with $\alpha = 0.4$ and $\beta = 0.2$. The training curves, displayed in \Cref{fig:training curve}, represent the mean values across these random seeds, while the shaded areas indicate the standard deviation. From \Cref{fig:training curve}, we make the following observations:

\begin{itemize}
    \item \textbf{F1 Score and AUC:} Our method exhibits a rapid and consistent increase in both F1 score and AUC during the initial communication rounds, and subsequently stabilizes at higher values compared to the baseline methods. This indicates its superior ability to quickly capture relevant patterns. Furthermore, the consistent improvement in both metrics suggests that our approach enhances class discrimination more effectively as training progresses.
    \item \textbf{Test Loss:} Our method achieves convergence with fewer communication rounds compared to the baseline methods. Specifically, it converges around the 11th communication round, whereas most baseline methods require approximately 20 rounds to reach convergence, highlighting the superior efficiency of our approach.
    \item \textbf{Time:} Our method achieves significant performance improvements without a substantial increase in training time. Furthermore, the training time of our method remains significantly lower than that of FedMed-GAN, which requires adversarial training for GANs.
\end{itemize}

\section{CONCLUSION}
In this study, we propose ClusMFL, a novel framework designed to address modality incompleteness in multimodal federated learning. ClusMFL enhances local data representation with the FINCH clustering algorithm, mitigates missing modality effects through supervised contrastive loss and modality completion loss, and employs a modality-aware aggregation strategy to adaptively integrate client contributions. Extensive experiments on the ADNI dataset demonstrate that ClusMFL outperforms state-of-the-art methods, particularly in scenarios with severe modality incompleteness, though at a slightly higher computational cost. Future research will explore the incorporation of additional modalities to further enhance the proposed framework's applicability.
% \begin{thebibliography}{99}

\bibitem{chaplot2020neural} Chaplot, Devendra Singh, et al. "Neural topological slam for visual navigation." Proceedings of the IEEE/CVF conference on computer vision and pattern recognition. 2020.

\bibitem{maksymets2021thda} Maksymets, Oleksandr, et al. "Thda: Treasure hunt data augmentation for semantic navigation." Proceedings of the IEEE/CVF International Conference on Computer Vision. 2021.

\bibitem{mezghan2022memory} Mezghan, Lina, et al. "Memory-augmented reinforcement learning for image-goal navigation." 2022 IEEE/RSJ International Conference on Intelligent Robots and Systems (IROS). IEEE, 2022.

\bibitem{al2022zero} Al-Halah, Ziad, Santhosh Kumar Ramakrishnan, and Kristen Grauman. "Zero experience required: Plug \& play modular transfer learning for semantic visual navigation." Proceedings of the IEEE/CVF Conference on Computer Vision and Pattern Recognition. 2022.

\bibitem{ye2021auxiliary} Ye, Joel, et al. "Auxiliary tasks and exploration enable objectgoal navigation." Proceedings of the IEEE/CVF international conference on computer vision. 2021.

\bibitem{chaplot2020object} Chaplot, Devendra Singh, et al. "Object goal navigation using goal-oriented semantic exploration." Advances in Neural Information Processing Systems 33 (2020)

\bibitem{ramakrishnan2022poni} Ramakrishnan, Santhosh Kumar, et al. "Poni: Potential functions for objectgoal navigation with interaction-free learning." Proceedings of the IEEE/CVF Conference on Computer Vision and Pattern Recognition. 2022.

\bibitem{ramrakhya2022habitat} Ramrakhya, Ram, et al. "Habitat-web: Learning embodied object-search strategies from human demonstrations at scale." Proceedings of the IEEE/CVF Conference on Computer Vision and Pattern Recognition. 2022.

\bibitem{mousavian2019visual} Mousavian, Arsalan, et al. "Visual representations for semantic target driven navigation." 2019 International Conference on Robotics and Automation (ICRA). IEEE, 2019.

\bibitem{dhariwal2021diffusion} Dhariwal, Prafulla, and Alexander Nichol. "Diffusion models beat gans on image synthesis." Advances in neural information processing systems 34 (2021)

\bibitem{ho2022classifier} Ho, Jonathan, and Tim Salimans. "Classifier-free diffusion guidance." arXiv preprint arXiv:2207.12598 (2022).

\bibitem{nichol2021glide} Nichol, Alex, et al. "Glide: Towards photorealistic image generation and editing with text-guided diffusion models." arXiv preprint arXiv:2112.10741 (2021)

\bibitem{brooks2023instructpix2pix} Brooks, Tim, Aleksander Holynski, and Alexei A. Efros. "Instructpix2pix: Learning to follow image editing instructions." Proceedings of the IEEE/CVF Conference on Computer Vision and Pattern Recognition. 2023.

\bibitem{fu2023guiding} Fu, Tsu-Jui, et al. "Guiding instruction-based image editing via multimodal large language models." arXiv preprint arXiv:2309.17102 (2023).

\bibitem{geng2024instructdiffusion} Geng, Zigang, et al. "Instructdiffusion: A generalist modeling interface for vision tasks." Proceedings of the IEEE/CVF Conference on Computer Vision and Pattern Recognition. 2024.

\bibitem{zhou2024minedreamer} Zhou, Enshen, et al. "Minedreamer: Learning to follow instructions via chain-of-imagination for simulated-world control." arXiv preprint arXiv:2403.12037 (2024).

\bibitem{zhou2023esc} Zhou, Kaiwen, et al. "Esc: Exploration with soft commonsense constraints for zero-shot object navigation." International Conference on Machine Learning. PMLR, 2023.

\bibitem{yu2023l3mvn} Yu, Bangguo, Hamidreza Kasaei, and Ming Cao. "L3mvn: Leveraging large language models for visual target navigation." 2023 IEEE/RSJ International Conference on Intelligent Robots and Systems (IROS). IEEE, 2023.

\bibitem{gadre2023cows} Gadre, Samir Yitzhak, et al. "Cows on pasture: Baselines and benchmarks for language-driven zero-shot object navigation." Proceedings of the IEEE/CVF Conference on Computer Vision and Pattern Recognition. 2023.

\bibitem{shah2023navigation} Shah, Dhruv, et al. "Navigation with large language models: Semantic guesswork as a heuristic for planning." Conference on Robot Learning. PMLR, 2023.

\bibitem{cai2024bridging} Cai, Wenzhe, et al. "Bridging zero-shot object navigation and foundation models through pixel-guided navigation skill." 2024 IEEE International Conference on Robotics and Automation (ICRA). IEEE, 2024.

\bibitem{yu2023co} Yu, Bangguo, Hamidreza Kasaei, and Ming Cao. "Co-NavGPT: Multi-robot cooperative visual semantic navigation using large language models." arXiv preprint arXiv:2310.07937 (2023).

\bibitem{wu2024voronav} Wu, Pengying, et al. "Voronav: Voronoi-based zero-shot object navigation with large language model." arXiv preprint arXiv:2401.02695 (2024).

\bibitem{qin2023mp5} Qin, Yiran, et al. "Mp5: A multi-modal open-ended embodied system in minecraft via active perception." arXiv preprint arXiv:2312.07472 (2023).

\bibitem{du2024learning} Du, Yilun, et al. "Learning universal policies via text-guided video generation." Advances in Neural Information Processing Systems 36 (2024).

\bibitem{ajay2024compositional} Ajay, Anurag, et al. "Compositional foundation models for hierarchical planning." Advances in Neural Information Processing Systems 36 (2024).

\bibitem{liang2024skilldiffuser} Liang, Zhixuan, et al. "Skilldiffuser: Interpretable hierarchical planning via skill abstractions in diffusion-based task execution." Proceedings of the IEEE/CVF Conference on Computer Vision and Pattern Recognition. 2024.

\bibitem{heusel2017gans} Heusel, Martin, et al. "Gans trained by a two time-scale update rule converge to a local nash equilibrium." Advances in neural information processing systems 30 (2017).

\bibitem{zhang2018unreasonable} Zhang, Richard, et al. "The unreasonable effectiveness of deep features as a perceptual metric." Proceedings of the IEEE conference on computer vision and pattern recognition. 2018.

\bibitem{brown2020language} Brown, Tom B. "Language models are few-shot learners." arXiv preprint arXiv:2005.14165 (2020).

\bibitem{podell2023sdxl} Podell, Dustin, et al. "Sdxl: Improving latent diffusion models for high-resolution image synthesis." arXiv preprint arXiv:2307.01952 (2023).

\bibitem{brohan2022rt} Brohan, Anthony, et al. "Rt-1: Robotics transformer for real-world control at scale." arXiv preprint arXiv:2212.06817 (2022).

\bibitem{brohan2023rt} Brohan, Anthony, et al. "Rt-2: Vision-language-action models transfer web knowledge to robotic control." arXiv preprint arXiv:2307.15818 (2023).

\bibitem{li2024manipllm} Li, Xiaoqi, et al. "Manipllm: Embodied multimodal large language model for object-centric robotic manipulation." Proceedings of the IEEE/CVF Conference on Computer Vision and Pattern Recognition. 2024.

\bibitem{shah2023vint} Shah, Dhruv, et al. "ViNT: A foundation model for visual navigation." arXiv preprint arXiv:2306.14846 (2023).

\bibitem{liu2024visual} Liu, Haotian, et al. "Visual instruction tuning." Advances in neural information processing systems 36 (2024).

\bibitem{hu2021lora} Hu, Edward J., et al. "Lora: Low-rank adaptation of large language models." arXiv preprint arXiv:2106.09685 (2021).

\bibitem{qin2023supfusion} Qin, Yiran, et al. "SupFusion: Supervised LiDAR-camera fusion for 3D object detection." Proceedings of the IEEE/CVF International Conference on Computer Vision. 2023.

\bibitem{qin2024worldsimbench} Qin, Yiran, et al. "Worldsimbench: Towards video generation models as world simulators." arXiv preprint arXiv:2410.18072 (2024).

\bibitem{yu2025gamefactory} Yu, Jiwen, et al. "GameFactory: Creating New Games with Generative Interactive Videos." arXiv preprint arXiv:2501.08325 (2025).

\bibitem{zhou2024code} Zhou, Enshen, et al. "Code-as-Monitor: Constraint-aware Visual Programming for Reactive and Proactive Robotic Failure Detection." arXiv preprint arXiv:2412.04455 (2024).

\bibitem{zhang2024ad} Zhang, Zaibin, et al. "AD-H: Autonomous Driving with Hierarchical Agents." arXiv preprint arXiv:2406.03474 (2024).

\bibitem{wang2024toward} Wang, Chaoqun, et al. "Toward Accurate Camera-based 3D Object Detection via Cascade Depth Estimation and Calibration." arXiv preprint arXiv:2402.04883 (2024).

\bibitem{huang2024story3d} Huang, Yuzhou, et al. "Story3d-agent: Exploring 3d storytelling visualization with large language models." arXiv preprint arXiv:2408.11801 (2024).

\bibitem{savinov2018semi} Savinov, Nikolay, Alexey Dosovitskiy, and Vladlen Koltun. "Semi-parametric topological memory for navigation." arXiv preprint arXiv:1803.00653 (2018).

\bibitem{majumdar2022zson} Majumdar, Arjun, et al. "Zson: Zero-shot object-goal navigation using multimodal goal embeddings." Advances in Neural Information Processing Systems 35 (2022): 32340-32352.

\bibitem{yadav2023offline} Yadav, Karmesh, et al. "Offline visual representation learning for embodied navigation." Workshop on Reincarnating Reinforcement Learning at ICLR 2023. 2023.

\bibitem{yadav2023ovrl} Yadav, Karmesh, et al. "Ovrl-v2: A simple state-of-art baseline for imagenav and objectnav." arXiv preprint arXiv:2303.07798 (2023).

\bibitem{sun2024fgprompt} Sun, Xinyu, et al. "FGPrompt: fine-grained goal prompting for image-goal navigation." Advances in Neural Information Processing Systems 36 (2024).

\bibitem{zhu2017target} Zhu, Yuke, et al. "Target-driven visual navigation in indoor scenes using deep reinforcement learning." 2017 IEEE international conference on robotics and automation (ICRA). IEEE, 2017.

\bibitem{koh2024generating} Koh, Jing Yu, Daniel Fried, and Russ R. Salakhutdinov. "Generating images with multimodal language models." Advances in Neural Information Processing Systems 36 (2024).

\bibitem{krantz2022instance} Krantz, Jacob, et al. "Instance-specific image goal navigation: Training embodied agents to find object instances." arXiv preprint arXiv:2211.15876 (2022).

\bibitem{schulman2017proximal} Schulman, John, et al. "Proximal policy optimization algorithms." arXiv preprint arXiv:1707.06347 (2017).

\bibitem{anderson2018evaluation} Anderson, Peter, et al. "On evaluation of embodied navigation agents." arXiv preprint arXiv:1807.06757 (2018).

\bibitem{lin2024navcot} Lin, Bingqian, et al. "NavCoT: Boosting LLM-Based Vision-and-Language Navigation via Learning Disentangled Reasoning." arXiv preprint arXiv:2403.07376 (2024).

\bibitem{NavGPT} Zhou, Gengze, Yicong Hong, and Qi Wu. "Navgpt: Explicit reasoning in vision-and-language navigation with large language models." Proceedings of the AAAI Conference on Artificial Intelligence.

\bibitem{hahn2021no} Hahn, Meera, et al. "No rl, no simulation: Learning to navigate without navigating." Advances in Neural Information Processing Systems 34 (2021): 26661-26673.

\bibitem{li2025t2isafety} Li, Lijun, et al. "T2ISafety: Benchmark for Assessing Fairness, Toxicity, and Privacy in Image Generation." arXiv preprint arXiv:2501.12612 (2025).

\bibitem{an2024agfsync} An, Jingkun, et al. "AGFSync: Leveraging AI-Generated Feedback for Preference Optimization in Text-to-Image Generation." arXiv preprint arXiv:2403.13352 (2024).


\end{thebibliography}
\addtolength{\textheight}{-12cm}   % This command serves to balance the column lengths
                                  % on the last page of the document manually. It shortens
                                  % the textheight of the last page by a suitable amount.
                                  % This command does not take effect until the next page
                                  % so it should come on the page before the last. Make
                                  % sure that you do not shorten the textheight too much.

%%%%%%%%%%%%%%%%%%%%%%%%%%%%%%%%%%%%%%%%%%%%%%%%%%%%%%%%%%%%%%%%%%%%%%%%%%%%%%%%



%%%%%%%%%%%%%%%%%%%%%%%%%%%%%%%%%%%%%%%%%%%%%%%%%%%%%%%%%%%%%%%%%%%%%%%%%%%%%%%%



%%%%%%%%%%%%%%%%%%%%%%%%%%%%%%%%%%%%%%%%%%%%%%%%%%%%%%%%%%%%%%%%%%%%%%%%%%%%%%%%
% \section*{APPENDIX}

% Appendixes should appear before the acknowledgment.

% \section*{ACKNOWLEDGMENT}

% The preferred spelling of the word ÒacknowledgmentÓ in America is without an ÒeÓ after the ÒgÓ. Avoid the stilted expression, ÒOne of us (R. B. G.) thanks . . .Ó  Instead, try ÒR. B. G. thanksÓ. Put sponsor acknowledgments in the unnumbered footnote on the first page.



% %%%%%%%%%%%%%%%%%%%%%%%%%%%%%%%%%%%%%%%%%%%%%%%%%%%%%%%%%%%%%%%%%%%%%%%%%%%%%%%%

% References are important to the reader; therefore, each citation must be complete and correct. If at all possible, references should be commonly available publications.


% \clearpage
\bibliographystyle{IEEEtran}
% \bibliography{main}
% This must be in the first 5 lines to tell arXiv to use pdfLaTeX, which is strongly recommended.
\pdfoutput=1
% In particular, the hyperref package requires pdfLaTeX in order to break URLs across lines.

\documentclass[11pt]{article}

% Change "review" to "final" to generate the final (sometimes called camera-ready) version.
% Change to "preprint" to generate a non-anonymous version with page numbers.
\usepackage{acl}

% Standard package includes
\usepackage{times}
\usepackage{latexsym}

% Draw tables
\usepackage{booktabs}
\usepackage{multirow}
\usepackage{xcolor}
\usepackage{colortbl}
\usepackage{array} 
\usepackage{amsmath}

\newcolumntype{C}{>{\centering\arraybackslash}p{0.07\textwidth}}
% For proper rendering and hyphenation of words containing Latin characters (including in bib files)
\usepackage[T1]{fontenc}
% For Vietnamese characters
% \usepackage[T5]{fontenc}
% See https://www.latex-project.org/help/documentation/encguide.pdf for other character sets
% This assumes your files are encoded as UTF8
\usepackage[utf8]{inputenc}

% This is not strictly necessary, and may be commented out,
% but it will improve the layout of the manuscript,
% and will typically save some space.
\usepackage{microtype}
\DeclareMathOperator*{\argmax}{arg\,max}
% This is also not strictly necessary, and may be commented out.
% However, it will improve the aesthetics of text in
% the typewriter font.
\usepackage{inconsolata}

%Including images in your LaTeX document requires adding
%additional package(s)
\usepackage{graphicx}
% If the title and author information does not fit in the area allocated, uncomment the following
%
%\setlength\titlebox{<dim>}
%
% and set <dim> to something 5cm or larger.

\title{Wi-Chat: Large Language Model Powered Wi-Fi Sensing}

% Author information can be set in various styles:
% For several authors from the same institution:
% \author{Author 1 \and ... \and Author n \\
%         Address line \\ ... \\ Address line}
% if the names do not fit well on one line use
%         Author 1 \\ {\bf Author 2} \\ ... \\ {\bf Author n} \\
% For authors from different institutions:
% \author{Author 1 \\ Address line \\  ... \\ Address line
%         \And  ... \And
%         Author n \\ Address line \\ ... \\ Address line}
% To start a separate ``row'' of authors use \AND, as in
% \author{Author 1 \\ Address line \\  ... \\ Address line
%         \AND
%         Author 2 \\ Address line \\ ... \\ Address line \And
%         Author 3 \\ Address line \\ ... \\ Address line}

% \author{First Author \\
%   Affiliation / Address line 1 \\
%   Affiliation / Address line 2 \\
%   Affiliation / Address line 3 \\
%   \texttt{email@domain} \\\And
%   Second Author \\
%   Affiliation / Address line 1 \\
%   Affiliation / Address line 2 \\
%   Affiliation / Address line 3 \\
%   \texttt{email@domain} \\}
% \author{Haohan Yuan \qquad Haopeng Zhang\thanks{corresponding author} \\ 
%   ALOHA Lab, University of Hawaii at Manoa \\
%   % Affiliation / Address line 2 \\
%   % Affiliation / Address line 3 \\
%   \texttt{\{haohany,haopengz\}@hawaii.edu}}
  
\author{
{Haopeng Zhang$\dag$\thanks{These authors contributed equally to this work.}, Yili Ren$\ddagger$\footnotemark[1], Haohan Yuan$\dag$, Jingzhe Zhang$\ddagger$, Yitong Shen$\ddagger$} \\
ALOHA Lab, University of Hawaii at Manoa$\dag$, University of South Florida$\ddagger$ \\
\{haopengz, haohany\}@hawaii.edu\\
\{yiliren, jingzhe, shen202\}@usf.edu\\}



  
%\author{
%  \textbf{First Author\textsuperscript{1}},
%  \textbf{Second Author\textsuperscript{1,2}},
%  \textbf{Third T. Author\textsuperscript{1}},
%  \textbf{Fourth Author\textsuperscript{1}},
%\\
%  \textbf{Fifth Author\textsuperscript{1,2}},
%  \textbf{Sixth Author\textsuperscript{1}},
%  \textbf{Seventh Author\textsuperscript{1}},
%  \textbf{Eighth Author \textsuperscript{1,2,3,4}},
%\\
%  \textbf{Ninth Author\textsuperscript{1}},
%  \textbf{Tenth Author\textsuperscript{1}},
%  \textbf{Eleventh E. Author\textsuperscript{1,2,3,4,5}},
%  \textbf{Twelfth Author\textsuperscript{1}},
%\\
%  \textbf{Thirteenth Author\textsuperscript{3}},
%  \textbf{Fourteenth F. Author\textsuperscript{2,4}},
%  \textbf{Fifteenth Author\textsuperscript{1}},
%  \textbf{Sixteenth Author\textsuperscript{1}},
%\\
%  \textbf{Seventeenth S. Author\textsuperscript{4,5}},
%  \textbf{Eighteenth Author\textsuperscript{3,4}},
%  \textbf{Nineteenth N. Author\textsuperscript{2,5}},
%  \textbf{Twentieth Author\textsuperscript{1}}
%\\
%\\
%  \textsuperscript{1}Affiliation 1,
%  \textsuperscript{2}Affiliation 2,
%  \textsuperscript{3}Affiliation 3,
%  \textsuperscript{4}Affiliation 4,
%  \textsuperscript{5}Affiliation 5
%\\
%  \small{
%    \textbf{Correspondence:} \href{mailto:email@domain}{email@domain}
%  }
%}

\begin{document}
\maketitle
\begin{abstract}
Recent advancements in Large Language Models (LLMs) have demonstrated remarkable capabilities across diverse tasks. However, their potential to integrate physical model knowledge for real-world signal interpretation remains largely unexplored. In this work, we introduce Wi-Chat, the first LLM-powered Wi-Fi-based human activity recognition system. We demonstrate that LLMs can process raw Wi-Fi signals and infer human activities by incorporating Wi-Fi sensing principles into prompts. Our approach leverages physical model insights to guide LLMs in interpreting Channel State Information (CSI) data without traditional signal processing techniques. Through experiments on real-world Wi-Fi datasets, we show that LLMs exhibit strong reasoning capabilities, achieving zero-shot activity recognition. These findings highlight a new paradigm for Wi-Fi sensing, expanding LLM applications beyond conventional language tasks and enhancing the accessibility of wireless sensing for real-world deployments.
\end{abstract}

\section{Introduction}

In today’s rapidly evolving digital landscape, the transformative power of web technologies has redefined not only how services are delivered but also how complex tasks are approached. Web-based systems have become increasingly prevalent in risk control across various domains. This widespread adoption is due their accessibility, scalability, and ability to remotely connect various types of users. For example, these systems are used for process safety management in industry~\cite{kannan2016web}, safety risk early warning in urban construction~\cite{ding2013development}, and safe monitoring of infrastructural systems~\cite{repetto2018web}. Within these web-based risk management systems, the source search problem presents a huge challenge. Source search refers to the task of identifying the origin of a risky event, such as a gas leak and the emission point of toxic substances. This source search capability is crucial for effective risk management and decision-making.

Traditional approaches to implementing source search capabilities into the web systems often rely on solely algorithmic solutions~\cite{ristic2016study}. These methods, while relatively straightforward to implement, often struggle to achieve acceptable performances due to algorithmic local optima and complex unknown environments~\cite{zhao2020searching}. More recently, web crowdsourcing has emerged as a promising alternative for tackling the source search problem by incorporating human efforts in these web systems on-the-fly~\cite{zhao2024user}. This approach outsources the task of addressing issues encountered during the source search process to human workers, leveraging their capabilities to enhance system performance.

These solutions often employ a human-AI collaborative way~\cite{zhao2023leveraging} where algorithms handle exploration-exploitation and report the encountered problems while human workers resolve complex decision-making bottlenecks to help the algorithms getting rid of local deadlocks~\cite{zhao2022crowd}. Although effective, this paradigm suffers from two inherent limitations: increased operational costs from continuous human intervention, and slow response times of human workers due to sequential decision-making. These challenges motivate our investigation into developing autonomous systems that preserve human-like reasoning capabilities while reducing dependency on massive crowdsourced labor.

Furthermore, recent advancements in large language models (LLMs)~\cite{chang2024survey} and multi-modal LLMs (MLLMs)~\cite{huang2023chatgpt} have unveiled promising avenues for addressing these challenges. One clear opportunity involves the seamless integration of visual understanding and linguistic reasoning for robust decision-making in search tasks. However, whether large models-assisted source search is really effective and efficient for improving the current source search algorithms~\cite{ji2022source} remains unknown. \textit{To address the research gap, we are particularly interested in answering the following two research questions in this work:}

\textbf{\textit{RQ1: }}How can source search capabilities be integrated into web-based systems to support decision-making in time-sensitive risk management scenarios? 
% \sq{I mention ``time-sensitive'' here because I feel like we shall say something about the response time -- LLM has to be faster than humans}

\textbf{\textit{RQ2: }}How can MLLMs and LLMs enhance the effectiveness and efficiency of existing source search algorithms? 

% \textit{\textbf{RQ2:}} To what extent does the performance of large models-assisted search align with or approach the effectiveness of human-AI collaborative search? 

To answer the research questions, we propose a novel framework called Auto-\
S$^2$earch (\textbf{Auto}nomous \textbf{S}ource \textbf{Search}) and implement a prototype system that leverages advanced web technologies to simulate real-world conditions for zero-shot source search. Unlike traditional methods that rely on pre-defined heuristics or extensive human intervention, AutoS$^2$earch employs a carefully designed prompt that encapsulates human rationales, thereby guiding the MLLM to generate coherent and accurate scene descriptions from visual inputs about four directional choices. Based on these language-based descriptions, the LLM is enabled to determine the optimal directional choice through chain-of-thought (CoT) reasoning. Comprehensive empirical validation demonstrates that AutoS$^2$-\ 
earch achieves a success rate of 95–98\%, closely approaching the performance of human-AI collaborative search across 20 benchmark scenarios~\cite{zhao2023leveraging}. 

Our work indicates that the role of humans in future web crowdsourcing tasks may evolve from executors to validators or supervisors. Furthermore, incorporating explanations of LLM decisions into web-based system interfaces has the potential to help humans enhance task performance in risk control.






\section{Related Work}
\label{sec:relatedworks}

% \begin{table*}[t]
% \centering 
% \renewcommand\arraystretch{0.98}
% \fontsize{8}{10}\selectfont \setlength{\tabcolsep}{0.4em}
% \begin{tabular}{@{}lc|cc|cc|cc@{}}
% \toprule
% \textbf{Methods}           & \begin{tabular}[c]{@{}c@{}}\textbf{Training}\\ \textbf{Paradigm}\end{tabular} & \begin{tabular}[c]{@{}c@{}}\textbf{$\#$ PT Data}\\ \textbf{(Tokens)}\end{tabular} & \begin{tabular}[c]{@{}c@{}}\textbf{$\#$ IFT Data}\\ \textbf{(Samples)}\end{tabular} & \textbf{Code}  & \begin{tabular}[c]{@{}c@{}}\textbf{Natural}\\ \textbf{Language}\end{tabular} & \begin{tabular}[c]{@{}c@{}}\textbf{Action}\\ \textbf{Trajectories}\end{tabular} & \begin{tabular}[c]{@{}c@{}}\textbf{API}\\ \textbf{Documentation}\end{tabular}\\ \midrule 
% NexusRaven~\citep{srinivasan2023nexusraven} & IFT & - & - & \textcolor{green}{\CheckmarkBold} & \textcolor{green}{\CheckmarkBold} &\textcolor{red}{\XSolidBrush}&\textcolor{red}{\XSolidBrush}\\
% AgentInstruct~\citep{zeng2023agenttuning} & IFT & - & 2k & \textcolor{green}{\CheckmarkBold} & \textcolor{green}{\CheckmarkBold} &\textcolor{red}{\XSolidBrush}&\textcolor{red}{\XSolidBrush} \\
% AgentEvol~\citep{xi2024agentgym} & IFT & - & 14.5k & \textcolor{green}{\CheckmarkBold} & \textcolor{green}{\CheckmarkBold} &\textcolor{green}{\CheckmarkBold}&\textcolor{red}{\XSolidBrush} \\
% Gorilla~\citep{patil2023gorilla}& IFT & - & 16k & \textcolor{green}{\CheckmarkBold} & \textcolor{green}{\CheckmarkBold} &\textcolor{red}{\XSolidBrush}&\textcolor{green}{\CheckmarkBold}\\
% OpenFunctions-v2~\citep{patil2023gorilla} & IFT & - & 65k & \textcolor{green}{\CheckmarkBold} & \textcolor{green}{\CheckmarkBold} &\textcolor{red}{\XSolidBrush}&\textcolor{green}{\CheckmarkBold}\\
% LAM~\citep{zhang2024agentohana} & IFT & - & 42.6k & \textcolor{green}{\CheckmarkBold} & \textcolor{green}{\CheckmarkBold} &\textcolor{green}{\CheckmarkBold}&\textcolor{red}{\XSolidBrush} \\
% xLAM~\citep{liu2024apigen} & IFT & - & 60k & \textcolor{green}{\CheckmarkBold} & \textcolor{green}{\CheckmarkBold} &\textcolor{green}{\CheckmarkBold}&\textcolor{red}{\XSolidBrush} \\\midrule
% LEMUR~\citep{xu2024lemur} & PT & 90B & 300k & \textcolor{green}{\CheckmarkBold} & \textcolor{green}{\CheckmarkBold} &\textcolor{green}{\CheckmarkBold}&\textcolor{red}{\XSolidBrush}\\
% \rowcolor{teal!12} \method & PT & 103B & 95k & \textcolor{green}{\CheckmarkBold} & \textcolor{green}{\CheckmarkBold} & \textcolor{green}{\CheckmarkBold} & \textcolor{green}{\CheckmarkBold} \\
% \bottomrule
% \end{tabular}
% \caption{Summary of existing tuning- and pretraining-based LLM agents with their training sample sizes. "PT" and "IFT" denote "Pre-Training" and "Instruction Fine-Tuning", respectively. }
% \label{tab:related}
% \end{table*}

\begin{table*}[ht]
\begin{threeparttable}
\centering 
\renewcommand\arraystretch{0.98}
\fontsize{7}{9}\selectfont \setlength{\tabcolsep}{0.2em}
\begin{tabular}{@{}l|c|c|ccc|cc|cc|cccc@{}}
\toprule
\textbf{Methods} & \textbf{Datasets}           & \begin{tabular}[c]{@{}c@{}}\textbf{Training}\\ \textbf{Paradigm}\end{tabular} & \begin{tabular}[c]{@{}c@{}}\textbf{\# PT Data}\\ \textbf{(Tokens)}\end{tabular} & \begin{tabular}[c]{@{}c@{}}\textbf{\# IFT Data}\\ \textbf{(Samples)}\end{tabular} & \textbf{\# APIs} & \textbf{Code}  & \begin{tabular}[c]{@{}c@{}}\textbf{Nat.}\\ \textbf{Lang.}\end{tabular} & \begin{tabular}[c]{@{}c@{}}\textbf{Action}\\ \textbf{Traj.}\end{tabular} & \begin{tabular}[c]{@{}c@{}}\textbf{API}\\ \textbf{Doc.}\end{tabular} & \begin{tabular}[c]{@{}c@{}}\textbf{Func.}\\ \textbf{Call}\end{tabular} & \begin{tabular}[c]{@{}c@{}}\textbf{Multi.}\\ \textbf{Step}\end{tabular}  & \begin{tabular}[c]{@{}c@{}}\textbf{Plan}\\ \textbf{Refine}\end{tabular}  & \begin{tabular}[c]{@{}c@{}}\textbf{Multi.}\\ \textbf{Turn}\end{tabular}\\ \midrule 
\multicolumn{13}{l}{\emph{Instruction Finetuning-based LLM Agents for Intrinsic Reasoning}}  \\ \midrule
FireAct~\cite{chen2023fireact} & FireAct & IFT & - & 2.1K & 10 & \textcolor{red}{\XSolidBrush} &\textcolor{green}{\CheckmarkBold} &\textcolor{green}{\CheckmarkBold}  & \textcolor{red}{\XSolidBrush} &\textcolor{green}{\CheckmarkBold} & \textcolor{red}{\XSolidBrush} &\textcolor{green}{\CheckmarkBold} & \textcolor{red}{\XSolidBrush} \\
ToolAlpaca~\cite{tang2023toolalpaca} & ToolAlpaca & IFT & - & 4.0K & 400 & \textcolor{red}{\XSolidBrush} &\textcolor{green}{\CheckmarkBold} &\textcolor{green}{\CheckmarkBold} & \textcolor{red}{\XSolidBrush} &\textcolor{green}{\CheckmarkBold} & \textcolor{red}{\XSolidBrush}  &\textcolor{green}{\CheckmarkBold} & \textcolor{red}{\XSolidBrush}  \\
ToolLLaMA~\cite{qin2023toolllm} & ToolBench & IFT & - & 12.7K & 16,464 & \textcolor{red}{\XSolidBrush} &\textcolor{green}{\CheckmarkBold} &\textcolor{green}{\CheckmarkBold} &\textcolor{red}{\XSolidBrush} &\textcolor{green}{\CheckmarkBold}&\textcolor{green}{\CheckmarkBold}&\textcolor{green}{\CheckmarkBold} &\textcolor{green}{\CheckmarkBold}\\
AgentEvol~\citep{xi2024agentgym} & AgentTraj-L & IFT & - & 14.5K & 24 &\textcolor{red}{\XSolidBrush} & \textcolor{green}{\CheckmarkBold} &\textcolor{green}{\CheckmarkBold}&\textcolor{red}{\XSolidBrush} &\textcolor{green}{\CheckmarkBold}&\textcolor{red}{\XSolidBrush} &\textcolor{red}{\XSolidBrush} &\textcolor{green}{\CheckmarkBold}\\
Lumos~\cite{yin2024agent} & Lumos & IFT  & - & 20.0K & 16 &\textcolor{red}{\XSolidBrush} & \textcolor{green}{\CheckmarkBold} & \textcolor{green}{\CheckmarkBold} &\textcolor{red}{\XSolidBrush} & \textcolor{green}{\CheckmarkBold} & \textcolor{green}{\CheckmarkBold} &\textcolor{red}{\XSolidBrush} & \textcolor{green}{\CheckmarkBold}\\
Agent-FLAN~\cite{chen2024agent} & Agent-FLAN & IFT & - & 24.7K & 20 &\textcolor{red}{\XSolidBrush} & \textcolor{green}{\CheckmarkBold} & \textcolor{green}{\CheckmarkBold} &\textcolor{red}{\XSolidBrush} & \textcolor{green}{\CheckmarkBold}& \textcolor{green}{\CheckmarkBold}&\textcolor{red}{\XSolidBrush} & \textcolor{green}{\CheckmarkBold}\\
AgentTuning~\citep{zeng2023agenttuning} & AgentInstruct & IFT & - & 35.0K & - &\textcolor{red}{\XSolidBrush} & \textcolor{green}{\CheckmarkBold} & \textcolor{green}{\CheckmarkBold} &\textcolor{red}{\XSolidBrush} & \textcolor{green}{\CheckmarkBold} &\textcolor{red}{\XSolidBrush} &\textcolor{red}{\XSolidBrush} & \textcolor{green}{\CheckmarkBold}\\\midrule
\multicolumn{13}{l}{\emph{Instruction Finetuning-based LLM Agents for Function Calling}} \\\midrule
NexusRaven~\citep{srinivasan2023nexusraven} & NexusRaven & IFT & - & - & 116 & \textcolor{green}{\CheckmarkBold} & \textcolor{green}{\CheckmarkBold}  & \textcolor{green}{\CheckmarkBold} &\textcolor{red}{\XSolidBrush} & \textcolor{green}{\CheckmarkBold} &\textcolor{red}{\XSolidBrush} &\textcolor{red}{\XSolidBrush}&\textcolor{red}{\XSolidBrush}\\
Gorilla~\citep{patil2023gorilla} & Gorilla & IFT & - & 16.0K & 1,645 & \textcolor{green}{\CheckmarkBold} &\textcolor{red}{\XSolidBrush} &\textcolor{red}{\XSolidBrush}&\textcolor{green}{\CheckmarkBold} &\textcolor{green}{\CheckmarkBold} &\textcolor{red}{\XSolidBrush} &\textcolor{red}{\XSolidBrush} &\textcolor{red}{\XSolidBrush}\\
OpenFunctions-v2~\citep{patil2023gorilla} & OpenFunctions-v2 & IFT & - & 65.0K & - & \textcolor{green}{\CheckmarkBold} & \textcolor{green}{\CheckmarkBold} &\textcolor{red}{\XSolidBrush} &\textcolor{green}{\CheckmarkBold} &\textcolor{green}{\CheckmarkBold} &\textcolor{red}{\XSolidBrush} &\textcolor{red}{\XSolidBrush} &\textcolor{red}{\XSolidBrush}\\
API Pack~\cite{guo2024api} & API Pack & IFT & - & 1.1M & 11,213 &\textcolor{green}{\CheckmarkBold} &\textcolor{red}{\XSolidBrush} &\textcolor{green}{\CheckmarkBold} &\textcolor{red}{\XSolidBrush} &\textcolor{green}{\CheckmarkBold} &\textcolor{red}{\XSolidBrush}&\textcolor{red}{\XSolidBrush}&\textcolor{red}{\XSolidBrush}\\ 
LAM~\citep{zhang2024agentohana} & AgentOhana & IFT & - & 42.6K & - & \textcolor{green}{\CheckmarkBold} & \textcolor{green}{\CheckmarkBold} &\textcolor{green}{\CheckmarkBold}&\textcolor{red}{\XSolidBrush} &\textcolor{green}{\CheckmarkBold}&\textcolor{red}{\XSolidBrush}&\textcolor{green}{\CheckmarkBold}&\textcolor{green}{\CheckmarkBold}\\
xLAM~\citep{liu2024apigen} & APIGen & IFT & - & 60.0K & 3,673 & \textcolor{green}{\CheckmarkBold} & \textcolor{green}{\CheckmarkBold} &\textcolor{green}{\CheckmarkBold}&\textcolor{red}{\XSolidBrush} &\textcolor{green}{\CheckmarkBold}&\textcolor{red}{\XSolidBrush}&\textcolor{green}{\CheckmarkBold}&\textcolor{green}{\CheckmarkBold}\\\midrule
\multicolumn{13}{l}{\emph{Pretraining-based LLM Agents}}  \\\midrule
% LEMUR~\citep{xu2024lemur} & PT & 90B & 300.0K & - & \textcolor{green}{\CheckmarkBold} & \textcolor{green}{\CheckmarkBold} &\textcolor{green}{\CheckmarkBold}&\textcolor{red}{\XSolidBrush} & \textcolor{red}{\XSolidBrush} &\textcolor{green}{\CheckmarkBold} &\textcolor{red}{\XSolidBrush}&\textcolor{red}{\XSolidBrush}\\
\rowcolor{teal!12} \method & \dataset & PT & 103B & 95.0K  & 76,537  & \textcolor{green}{\CheckmarkBold} & \textcolor{green}{\CheckmarkBold} & \textcolor{green}{\CheckmarkBold} & \textcolor{green}{\CheckmarkBold} & \textcolor{green}{\CheckmarkBold} & \textcolor{green}{\CheckmarkBold} & \textcolor{green}{\CheckmarkBold} & \textcolor{green}{\CheckmarkBold}\\
\bottomrule
\end{tabular}
% \begin{tablenotes}
%     \item $^*$ In addition, the StarCoder-API can offer 4.77M more APIs.
% \end{tablenotes}
\caption{Summary of existing instruction finetuning-based LLM agents for intrinsic reasoning and function calling, along with their training resources and sample sizes. "PT" and "IFT" denote "Pre-Training" and "Instruction Fine-Tuning", respectively.}
\vspace{-2ex}
\label{tab:related}
\end{threeparttable}
\end{table*}

\noindent \textbf{Prompting-based LLM Agents.} Due to the lack of agent-specific pre-training corpus, existing LLM agents rely on either prompt engineering~\cite{hsieh2023tool,lu2024chameleon,yao2022react,wang2023voyager} or instruction fine-tuning~\cite{chen2023fireact,zeng2023agenttuning} to understand human instructions, decompose high-level tasks, generate grounded plans, and execute multi-step actions. 
However, prompting-based methods mainly depend on the capabilities of backbone LLMs (usually commercial LLMs), failing to introduce new knowledge and struggling to generalize to unseen tasks~\cite{sun2024adaplanner,zhuang2023toolchain}. 

\noindent \textbf{Instruction Finetuning-based LLM Agents.} Considering the extensive diversity of APIs and the complexity of multi-tool instructions, tool learning inherently presents greater challenges than natural language tasks, such as text generation~\cite{qin2023toolllm}.
Post-training techniques focus more on instruction following and aligning output with specific formats~\cite{patil2023gorilla,hao2024toolkengpt,qin2023toolllm,schick2024toolformer}, rather than fundamentally improving model knowledge or capabilities. 
Moreover, heavy fine-tuning can hinder generalization or even degrade performance in non-agent use cases, potentially suppressing the original base model capabilities~\cite{ghosh2024a}.

\noindent \textbf{Pretraining-based LLM Agents.} While pre-training serves as an essential alternative, prior works~\cite{nijkamp2023codegen,roziere2023code,xu2024lemur,patil2023gorilla} have primarily focused on improving task-specific capabilities (\eg, code generation) instead of general-domain LLM agents, due to single-source, uni-type, small-scale, and poor-quality pre-training data. 
Existing tool documentation data for agent training either lacks diverse real-world APIs~\cite{patil2023gorilla, tang2023toolalpaca} or is constrained to single-tool or single-round tool execution. 
Furthermore, trajectory data mostly imitate expert behavior or follow function-calling rules with inferior planning and reasoning, failing to fully elicit LLMs' capabilities and handle complex instructions~\cite{qin2023toolllm}. 
Given a wide range of candidate API functions, each comprising various function names and parameters available at every planning step, identifying globally optimal solutions and generalizing across tasks remains highly challenging.



\section{Preliminaries}
\label{Preliminaries}
\begin{figure*}[t]
    \centering
    \includegraphics[width=0.95\linewidth]{fig/HealthGPT_Framework.png}
    \caption{The \ourmethod{} architecture integrates hierarchical visual perception and H-LoRA, employing a task-specific hard router to select visual features and H-LoRA plugins, ultimately generating outputs with an autoregressive manner.}
    \label{fig:architecture}
\end{figure*}
\noindent\textbf{Large Vision-Language Models.} 
The input to a LVLM typically consists of an image $x^{\text{img}}$ and a discrete text sequence $x^{\text{txt}}$. The visual encoder $\mathcal{E}^{\text{img}}$ converts the input image $x^{\text{img}}$ into a sequence of visual tokens $\mathcal{V} = [v_i]_{i=1}^{N_v}$, while the text sequence $x^{\text{txt}}$ is mapped into a sequence of text tokens $\mathcal{T} = [t_i]_{i=1}^{N_t}$ using an embedding function $\mathcal{E}^{\text{txt}}$. The LLM $\mathcal{M_\text{LLM}}(\cdot|\theta)$ models the joint probability of the token sequence $\mathcal{U} = \{\mathcal{V},\mathcal{T}\}$, which is expressed as:
\begin{equation}
    P_\theta(R | \mathcal{U}) = \prod_{i=1}^{N_r} P_\theta(r_i | \{\mathcal{U}, r_{<i}\}),
\end{equation}
where $R = [r_i]_{i=1}^{N_r}$ is the text response sequence. The LVLM iteratively generates the next token $r_i$ based on $r_{<i}$. The optimization objective is to minimize the cross-entropy loss of the response $\mathcal{R}$.
% \begin{equation}
%     \mathcal{L}_{\text{VLM}} = \mathbb{E}_{R|\mathcal{U}}\left[-\log P_\theta(R | \mathcal{U})\right]
% \end{equation}
It is worth noting that most LVLMs adopt a design paradigm based on ViT, alignment adapters, and pre-trained LLMs\cite{liu2023llava,liu2024improved}, enabling quick adaptation to downstream tasks.


\noindent\textbf{VQGAN.}
VQGAN~\cite{esser2021taming} employs latent space compression and indexing mechanisms to effectively learn a complete discrete representation of images. VQGAN first maps the input image $x^{\text{img}}$ to a latent representation $z = \mathcal{E}(x)$ through a encoder $\mathcal{E}$. Then, the latent representation is quantized using a codebook $\mathcal{Z} = \{z_k\}_{k=1}^K$, generating a discrete index sequence $\mathcal{I} = [i_m]_{m=1}^N$, where $i_m \in \mathcal{Z}$ represents the quantized code index:
\begin{equation}
    \mathcal{I} = \text{Quantize}(z|\mathcal{Z}) = \arg\min_{z_k \in \mathcal{Z}} \| z - z_k \|_2.
\end{equation}
In our approach, the discrete index sequence $\mathcal{I}$ serves as a supervisory signal for the generation task, enabling the model to predict the index sequence $\hat{\mathcal{I}}$ from input conditions such as text or other modality signals.  
Finally, the predicted index sequence $\hat{\mathcal{I}}$ is upsampled by the VQGAN decoder $G$, generating the high-quality image $\hat{x}^\text{img} = G(\hat{\mathcal{I}})$.



\noindent\textbf{Low Rank Adaptation.} 
LoRA\cite{hu2021lora} effectively captures the characteristics of downstream tasks by introducing low-rank adapters. The core idea is to decompose the bypass weight matrix $\Delta W\in\mathbb{R}^{d^{\text{in}} \times d^{\text{out}}}$ into two low-rank matrices $ \{A \in \mathbb{R}^{d^{\text{in}} \times r}, B \in \mathbb{R}^{r \times d^{\text{out}}} \}$, where $ r \ll \min\{d^{\text{in}}, d^{\text{out}}\} $, significantly reducing learnable parameters. The output with the LoRA adapter for the input $x$ is then given by:
\begin{equation}
    h = x W_0 + \alpha x \Delta W/r = x W_0 + \alpha xAB/r,
\end{equation}
where matrix $ A $ is initialized with a Gaussian distribution, while the matrix $ B $ is initialized as a zero matrix. The scaling factor $ \alpha/r $ controls the impact of $ \Delta W $ on the model.

\section{HealthGPT}
\label{Method}


\subsection{Unified Autoregressive Generation.}  
% As shown in Figure~\ref{fig:architecture}, 
\ourmethod{} (Figure~\ref{fig:architecture}) utilizes a discrete token representation that covers both text and visual outputs, unifying visual comprehension and generation as an autoregressive task. 
For comprehension, $\mathcal{M}_\text{llm}$ receives the input joint sequence $\mathcal{U}$ and outputs a series of text token $\mathcal{R} = [r_1, r_2, \dots, r_{N_r}]$, where $r_i \in \mathcal{V}_{\text{txt}}$, and $\mathcal{V}_{\text{txt}}$ represents the LLM's vocabulary:
\begin{equation}
    P_\theta(\mathcal{R} \mid \mathcal{U}) = \prod_{i=1}^{N_r} P_\theta(r_i \mid \mathcal{U}, r_{<i}).
\end{equation}
For generation, $\mathcal{M}_\text{llm}$ first receives a special start token $\langle \text{START\_IMG} \rangle$, then generates a series of tokens corresponding to the VQGAN indices $\mathcal{I} = [i_1, i_2, \dots, i_{N_i}]$, where $i_j \in \mathcal{V}_{\text{vq}}$, and $\mathcal{V}_{\text{vq}}$ represents the index range of VQGAN. Upon completion of generation, the LLM outputs an end token $\langle \text{END\_IMG} \rangle$:
\begin{equation}
    P_\theta(\mathcal{I} \mid \mathcal{U}) = \prod_{j=1}^{N_i} P_\theta(i_j \mid \mathcal{U}, i_{<j}).
\end{equation}
Finally, the generated index sequence $\mathcal{I}$ is fed into the decoder $G$, which reconstructs the target image $\hat{x}^{\text{img}} = G(\mathcal{I})$.

\subsection{Hierarchical Visual Perception}  
Given the differences in visual perception between comprehension and generation tasks—where the former focuses on abstract semantics and the latter emphasizes complete semantics—we employ ViT to compress the image into discrete visual tokens at multiple hierarchical levels.
Specifically, the image is converted into a series of features $\{f_1, f_2, \dots, f_L\}$ as it passes through $L$ ViT blocks.

To address the needs of various tasks, the hidden states are divided into two types: (i) \textit{Concrete-grained features} $\mathcal{F}^{\text{Con}} = \{f_1, f_2, \dots, f_k\}, k < L$, derived from the shallower layers of ViT, containing sufficient global features, suitable for generation tasks; 
(ii) \textit{Abstract-grained features} $\mathcal{F}^{\text{Abs}} = \{f_{k+1}, f_{k+2}, \dots, f_L\}$, derived from the deeper layers of ViT, which contain abstract semantic information closer to the text space, suitable for comprehension tasks.

The task type $T$ (comprehension or generation) determines which set of features is selected as the input for the downstream large language model:
\begin{equation}
    \mathcal{F}^{\text{img}}_T =
    \begin{cases}
        \mathcal{F}^{\text{Con}}, & \text{if } T = \text{generation task} \\
        \mathcal{F}^{\text{Abs}}, & \text{if } T = \text{comprehension task}
    \end{cases}
\end{equation}
We integrate the image features $\mathcal{F}^{\text{img}}_T$ and text features $\mathcal{T}$ into a joint sequence through simple concatenation, which is then fed into the LLM $\mathcal{M}_{\text{llm}}$ for autoregressive generation.
% :
% \begin{equation}
%     \mathcal{R} = \mathcal{M}_{\text{llm}}(\mathcal{U}|\theta), \quad \mathcal{U} = [\mathcal{F}^{\text{img}}_T; \mathcal{T}]
% \end{equation}
\subsection{Heterogeneous Knowledge Adaptation}
We devise H-LoRA, which stores heterogeneous knowledge from comprehension and generation tasks in separate modules and dynamically routes to extract task-relevant knowledge from these modules. 
At the task level, for each task type $ T $, we dynamically assign a dedicated H-LoRA submodule $ \theta^T $, which is expressed as:
\begin{equation}
    \mathcal{R} = \mathcal{M}_\text{LLM}(\mathcal{U}|\theta, \theta^T), \quad \theta^T = \{A^T, B^T, \mathcal{R}^T_\text{outer}\}.
\end{equation}
At the feature level for a single task, H-LoRA integrates the idea of Mixture of Experts (MoE)~\cite{masoudnia2014mixture} and designs an efficient matrix merging and routing weight allocation mechanism, thus avoiding the significant computational delay introduced by matrix splitting in existing MoELoRA~\cite{luo2024moelora}. Specifically, we first merge the low-rank matrices (rank = r) of $ k $ LoRA experts into a unified matrix:
\begin{equation}
    \mathbf{A}^{\text{merged}}, \mathbf{B}^{\text{merged}} = \text{Concat}(\{A_i\}_1^k), \text{Concat}(\{B_i\}_1^k),
\end{equation}
where $ \mathbf{A}^{\text{merged}} \in \mathbb{R}^{d^\text{in} \times rk} $ and $ \mathbf{B}^{\text{merged}} \in \mathbb{R}^{rk \times d^\text{out}} $. The $k$-dimension routing layer generates expert weights $ \mathcal{W} \in \mathbb{R}^{\text{token\_num} \times k} $ based on the input hidden state $ x $, and these are expanded to $ \mathbb{R}^{\text{token\_num} \times rk} $ as follows:
\begin{equation}
    \mathcal{W}^\text{expanded} = \alpha k \mathcal{W} / r \otimes \mathbf{1}_r,
\end{equation}
where $ \otimes $ denotes the replication operation.
The overall output of H-LoRA is computed as:
\begin{equation}
    \mathcal{O}^\text{H-LoRA} = (x \mathbf{A}^{\text{merged}} \odot \mathcal{W}^\text{expanded}) \mathbf{B}^{\text{merged}},
\end{equation}
where $ \odot $ represents element-wise multiplication. Finally, the output of H-LoRA is added to the frozen pre-trained weights to produce the final output:
\begin{equation}
    \mathcal{O} = x W_0 + \mathcal{O}^\text{H-LoRA}.
\end{equation}
% In summary, H-LoRA is a task-based dynamic PEFT method that achieves high efficiency in single-task fine-tuning.

\subsection{Training Pipeline}

\begin{figure}[t]
    \centering
    \hspace{-4mm}
    \includegraphics[width=0.94\linewidth]{fig/data.pdf}
    \caption{Data statistics of \texttt{VL-Health}. }
    \label{fig:data}
\end{figure}
\noindent \textbf{1st Stage: Multi-modal Alignment.} 
In the first stage, we design separate visual adapters and H-LoRA submodules for medical unified tasks. For the medical comprehension task, we train abstract-grained visual adapters using high-quality image-text pairs to align visual embeddings with textual embeddings, thereby enabling the model to accurately describe medical visual content. During this process, the pre-trained LLM and its corresponding H-LoRA submodules remain frozen. In contrast, the medical generation task requires training concrete-grained adapters and H-LoRA submodules while keeping the LLM frozen. Meanwhile, we extend the textual vocabulary to include multimodal tokens, enabling the support of additional VQGAN vector quantization indices. The model trains on image-VQ pairs, endowing the pre-trained LLM with the capability for image reconstruction. This design ensures pixel-level consistency of pre- and post-LVLM. The processes establish the initial alignment between the LLM’s outputs and the visual inputs.

\noindent \textbf{2nd Stage: Heterogeneous H-LoRA Plugin Adaptation.}  
The submodules of H-LoRA share the word embedding layer and output head but may encounter issues such as bias and scale inconsistencies during training across different tasks. To ensure that the multiple H-LoRA plugins seamlessly interface with the LLMs and form a unified base, we fine-tune the word embedding layer and output head using a small amount of mixed data to maintain consistency in the model weights. Specifically, during this stage, all H-LoRA submodules for different tasks are kept frozen, with only the word embedding layer and output head being optimized. Through this stage, the model accumulates foundational knowledge for unified tasks by adapting H-LoRA plugins.

\begin{table*}[!t]
\centering
\caption{Comparison of \ourmethod{} with other LVLMs and unified multi-modal models on medical visual comprehension tasks. \textbf{Bold} and \underline{underlined} text indicates the best performance and second-best performance, respectively.}
\resizebox{\textwidth}{!}{
\begin{tabular}{c|lcc|cccccccc|c}
\toprule
\rowcolor[HTML]{E9F3FE} &  &  &  & \multicolumn{2}{c}{\textbf{VQA-RAD \textuparrow}} & \multicolumn{2}{c}{\textbf{SLAKE \textuparrow}} & \multicolumn{2}{c}{\textbf{PathVQA \textuparrow}} &  &  &  \\ 
\cline{5-10}
\rowcolor[HTML]{E9F3FE}\multirow{-2}{*}{\textbf{Type}} & \multirow{-2}{*}{\textbf{Model}} & \multirow{-2}{*}{\textbf{\# Params}} & \multirow{-2}{*}{\makecell{\textbf{Medical} \\ \textbf{LVLM}}} & \textbf{close} & \textbf{all} & \textbf{close} & \textbf{all} & \textbf{close} & \textbf{all} & \multirow{-2}{*}{\makecell{\textbf{MMMU} \\ \textbf{-Med}}\textuparrow} & \multirow{-2}{*}{\textbf{OMVQA}\textuparrow} & \multirow{-2}{*}{\textbf{Avg. \textuparrow}} \\ 
\midrule \midrule
\multirow{9}{*}{\textbf{Comp. Only}} 
& Med-Flamingo & 8.3B & \Large \ding{51} & 58.6 & 43.0 & 47.0 & 25.5 & 61.9 & 31.3 & 28.7 & 34.9 & 41.4 \\
& LLaVA-Med & 7B & \Large \ding{51} & 60.2 & 48.1 & 58.4 & 44.8 & 62.3 & 35.7 & 30.0 & 41.3 & 47.6 \\
& HuatuoGPT-Vision & 7B & \Large \ding{51} & 66.9 & 53.0 & 59.8 & 49.1 & 52.9 & 32.0 & 42.0 & 50.0 & 50.7 \\
& BLIP-2 & 6.7B & \Large \ding{55} & 43.4 & 36.8 & 41.6 & 35.3 & 48.5 & 28.8 & 27.3 & 26.9 & 36.1 \\
& LLaVA-v1.5 & 7B & \Large \ding{55} & 51.8 & 42.8 & 37.1 & 37.7 & 53.5 & 31.4 & 32.7 & 44.7 & 41.5 \\
& InstructBLIP & 7B & \Large \ding{55} & 61.0 & 44.8 & 66.8 & 43.3 & 56.0 & 32.3 & 25.3 & 29.0 & 44.8 \\
& Yi-VL & 6B & \Large \ding{55} & 52.6 & 42.1 & 52.4 & 38.4 & 54.9 & 30.9 & 38.0 & 50.2 & 44.9 \\
& InternVL2 & 8B & \Large \ding{55} & 64.9 & 49.0 & 66.6 & 50.1 & 60.0 & 31.9 & \underline{43.3} & 54.5 & 52.5\\
& Llama-3.2 & 11B & \Large \ding{55} & 68.9 & 45.5 & 72.4 & 52.1 & 62.8 & 33.6 & 39.3 & 63.2 & 54.7 \\
\midrule
\multirow{5}{*}{\textbf{Comp. \& Gen.}} 
& Show-o & 1.3B & \Large \ding{55} & 50.6 & 33.9 & 31.5 & 17.9 & 52.9 & 28.2 & 22.7 & 45.7 & 42.6 \\
& Unified-IO 2 & 7B & \Large \ding{55} & 46.2 & 32.6 & 35.9 & 21.9 & 52.5 & 27.0 & 25.3 & 33.0 & 33.8 \\
& Janus & 1.3B & \Large \ding{55} & 70.9 & 52.8 & 34.7 & 26.9 & 51.9 & 27.9 & 30.0 & 26.8 & 33.5 \\
& \cellcolor[HTML]{DAE0FB}HealthGPT-M3 & \cellcolor[HTML]{DAE0FB}3.8B & \cellcolor[HTML]{DAE0FB}\Large \ding{51} & \cellcolor[HTML]{DAE0FB}\underline{73.7} & \cellcolor[HTML]{DAE0FB}\underline{55.9} & \cellcolor[HTML]{DAE0FB}\underline{74.6} & \cellcolor[HTML]{DAE0FB}\underline{56.4} & \cellcolor[HTML]{DAE0FB}\underline{78.7} & \cellcolor[HTML]{DAE0FB}\underline{39.7} & \cellcolor[HTML]{DAE0FB}\underline{43.3} & \cellcolor[HTML]{DAE0FB}\underline{68.5} & \cellcolor[HTML]{DAE0FB}\underline{61.3} \\
& \cellcolor[HTML]{DAE0FB}HealthGPT-L14 & \cellcolor[HTML]{DAE0FB}14B & \cellcolor[HTML]{DAE0FB}\Large \ding{51} & \cellcolor[HTML]{DAE0FB}\textbf{77.7} & \cellcolor[HTML]{DAE0FB}\textbf{58.3} & \cellcolor[HTML]{DAE0FB}\textbf{76.4} & \cellcolor[HTML]{DAE0FB}\textbf{64.5} & \cellcolor[HTML]{DAE0FB}\textbf{85.9} & \cellcolor[HTML]{DAE0FB}\textbf{44.4} & \cellcolor[HTML]{DAE0FB}\textbf{49.2} & \cellcolor[HTML]{DAE0FB}\textbf{74.4} & \cellcolor[HTML]{DAE0FB}\textbf{66.4} \\
\bottomrule
\end{tabular}
}
\label{tab:results}
\end{table*}
\begin{table*}[ht]
    \centering
    \caption{The experimental results for the four modality conversion tasks.}
    \resizebox{\textwidth}{!}{
    \begin{tabular}{l|ccc|ccc|ccc|ccc}
        \toprule
        \rowcolor[HTML]{E9F3FE} & \multicolumn{3}{c}{\textbf{CT to MRI (Brain)}} & \multicolumn{3}{c}{\textbf{CT to MRI (Pelvis)}} & \multicolumn{3}{c}{\textbf{MRI to CT (Brain)}} & \multicolumn{3}{c}{\textbf{MRI to CT (Pelvis)}} \\
        \cline{2-13}
        \rowcolor[HTML]{E9F3FE}\multirow{-2}{*}{\textbf{Model}}& \textbf{SSIM $\uparrow$} & \textbf{PSNR $\uparrow$} & \textbf{MSE $\downarrow$} & \textbf{SSIM $\uparrow$} & \textbf{PSNR $\uparrow$} & \textbf{MSE $\downarrow$} & \textbf{SSIM $\uparrow$} & \textbf{PSNR $\uparrow$} & \textbf{MSE $\downarrow$} & \textbf{SSIM $\uparrow$} & \textbf{PSNR $\uparrow$} & \textbf{MSE $\downarrow$} \\
        \midrule \midrule
        pix2pix & 71.09 & 32.65 & 36.85 & 59.17 & 31.02 & 51.91 & 78.79 & 33.85 & 28.33 & 72.31 & 32.98 & 36.19 \\
        CycleGAN & 54.76 & 32.23 & 40.56 & 54.54 & 30.77 & 55.00 & 63.75 & 31.02 & 52.78 & 50.54 & 29.89 & 67.78 \\
        BBDM & {71.69} & {32.91} & {34.44} & 57.37 & 31.37 & 48.06 & \textbf{86.40} & 34.12 & 26.61 & {79.26} & 33.15 & 33.60 \\
        Vmanba & 69.54 & 32.67 & 36.42 & {63.01} & {31.47} & {46.99} & 79.63 & 34.12 & 26.49 & 77.45 & 33.53 & 31.85 \\
        DiffMa & 71.47 & 32.74 & 35.77 & 62.56 & 31.43 & 47.38 & 79.00 & {34.13} & {26.45} & 78.53 & {33.68} & {30.51} \\
        \rowcolor[HTML]{DAE0FB}HealthGPT-M3 & \underline{79.38} & \underline{33.03} & \underline{33.48} & \underline{71.81} & \underline{31.83} & \underline{43.45} & {85.06} & \textbf{34.40} & \textbf{25.49} & \underline{84.23} & \textbf{34.29} & \textbf{27.99} \\
        \rowcolor[HTML]{DAE0FB}HealthGPT-L14 & \textbf{79.73} & \textbf{33.10} & \textbf{32.96} & \textbf{71.92} & \textbf{31.87} & \textbf{43.09} & \underline{85.31} & \underline{34.29} & \underline{26.20} & \textbf{84.96} & \underline{34.14} & \underline{28.13} \\
        \bottomrule
    \end{tabular}
    }
    \label{tab:conversion}
\end{table*}

\noindent \textbf{3rd Stage: Visual Instruction Fine-Tuning.}  
In the third stage, we introduce additional task-specific data to further optimize the model and enhance its adaptability to downstream tasks such as medical visual comprehension (e.g., medical QA, medical dialogues, and report generation) or generation tasks (e.g., super-resolution, denoising, and modality conversion). Notably, by this stage, the word embedding layer and output head have been fine-tuned, only the H-LoRA modules and adapter modules need to be trained. This strategy significantly improves the model's adaptability and flexibility across different tasks.


\section{Experiment}
\label{s:experiment}

\subsection{Data Description}
We evaluate our method on FI~\cite{you2016building}, Twitter\_LDL~\cite{yang2017learning} and Artphoto~\cite{machajdik2010affective}.
FI is a public dataset built from Flickr and Instagram, with 23,308 images and eight emotion categories, namely \textit{amusement}, \textit{anger}, \textit{awe},  \textit{contentment}, \textit{disgust}, \textit{excitement},  \textit{fear}, and \textit{sadness}. 
% Since images in FI are all copyrighted by law, some images are corrupted now, so we remove these samples and retain 21,828 images.
% T4SA contains images from Twitter, which are classified into three categories: \textit{positive}, \textit{neutral}, and \textit{negative}. In this paper, we adopt the base version of B-T4SA, which contains 470,586 images and provides text descriptions of the corresponding tweets.
Twitter\_LDL contains 10,045 images from Twitter, with the same eight categories as the FI dataset.
% 。
For these two datasets, they are randomly split into 80\%
training and 20\% testing set.
Artphoto contains 806 artistic photos from the DeviantArt website, which we use to further evaluate the zero-shot capability of our model.
% on the small-scale dataset.
% We construct and publicly release the first image sentiment analysis dataset containing metadata.
% 。

% Based on these datasets, we are the first to construct and publicly release metadata-enhanced image sentiment analysis datasets. These datasets include scenes, tags, descriptions, and corresponding confidence scores, and are available at this link for future research purposes.


% 
\begin{table}[t]
\centering
% \begin{center}
\caption{Overall performance of different models on FI and Twitter\_LDL datasets.}
\label{tab:cap1}
% \resizebox{\linewidth}{!}
{
\begin{tabular}{l|c|c|c|c}
\hline
\multirow{2}{*}{\textbf{Model}} & \multicolumn{2}{c|}{\textbf{FI}}  & \multicolumn{2}{c}{\textbf{Twitter\_LDL}} \\ \cline{2-5} 
  & \textbf{Accuracy} & \textbf{F1} & \textbf{Accuracy} & \textbf{F1}  \\ \hline
% (\rownumber)~AlexNet~\cite{krizhevsky2017imagenet}  & 58.13\% & 56.35\%  & 56.24\%& 55.02\%  \\ 
% (\rownumber)~VGG16~\cite{simonyan2014very}  & 63.75\%& 63.08\%  & 59.34\%& 59.02\%  \\ 
(\rownumber)~ResNet101~\cite{he2016deep} & 66.16\%& 65.56\%  & 62.02\% & 61.34\%  \\ 
(\rownumber)~CDA~\cite{han2023boosting} & 66.71\%& 65.37\%  & 64.14\% & 62.85\%  \\ 
(\rownumber)~CECCN~\cite{ruan2024color} & 67.96\%& 66.74\%  & 64.59\%& 64.72\% \\ 
(\rownumber)~EmoVIT~\cite{xie2024emovit} & 68.09\%& 67.45\%  & 63.12\% & 61.97\%  \\ 
(\rownumber)~ComLDL~\cite{zhang2022compound} & 68.83\%& 67.28\%  & 65.29\% & 63.12\%  \\ 
(\rownumber)~WSDEN~\cite{li2023weakly} & 69.78\%& 69.61\%  & 67.04\% & 65.49\% \\ 
(\rownumber)~ECWA~\cite{deng2021emotion} & 70.87\%& 69.08\%  & 67.81\% & 66.87\%  \\ 
(\rownumber)~EECon~\cite{yang2023exploiting} & 71.13\%& 68.34\%  & 64.27\%& 63.16\%  \\ 
(\rownumber)~MAM~\cite{zhang2024affective} & 71.44\%  & 70.83\% & 67.18\%  & 65.01\%\\ 
(\rownumber)~TGCA-PVT~\cite{chen2024tgca}   & 73.05\%  & 71.46\% & 69.87\%  & 68.32\% \\ 
(\rownumber)~OEAN~\cite{zhang2024object}   & 73.40\%  & 72.63\% & 70.52\%  & 69.47\% \\ \hline
(\rownumber)~\shortname  & \textbf{79.48\%} & \textbf{79.22\%} & \textbf{74.12\%} & \textbf{73.09\%} \\ \hline
\end{tabular}
}
\vspace{-6mm}
% \end{center}
\end{table}
% 

\subsection{Experiment Setting}
% \subsubsection{Model Setting.}
% 
\textbf{Model Setting:}
For feature representation, we set $k=10$ to select object tags, and adopt clip-vit-base-patch32 as the pre-trained model for unified feature representation.
Moreover, we empirically set $(d_e, d_h, d_k, d_s) = (512, 128, 16, 64)$, and set the classification class $L$ to 8.

% 

\textbf{Training Setting:}
To initialize the model, we set all weights such as $\boldsymbol{W}$ following the truncated normal distribution, and use AdamW optimizer with the learning rate of $1 \times 10^{-4}$.
% warmup scheduler of cosine, warmup steps of 2000.
Furthermore, we set the batch size to 32 and the epoch of the training process to 200.
During the implementation, we utilize \textit{PyTorch} to build our entire model.
% , and our project codes are publicly available at https://github.com/zzmyrep/MESN.
% Our project codes as well as data are all publicly available on GitHub\footnote{https://github.com/zzmyrep/KBCEN}.
% Code is available at \href{https://github.com/zzmyrep/KBCEN}{https://github.com/zzmyrep/KBCEN}.

\textbf{Evaluation Metrics:}
Following~\cite{zhang2024affective, chen2024tgca, zhang2024object}, we adopt \textit{accuracy} and \textit{F1} as our evaluation metrics to measure the performance of different methods for image sentiment analysis. 



\subsection{Experiment Result}
% We compare our model against the following baselines: AlexNet~\cite{krizhevsky2017imagenet}, VGG16~\cite{simonyan2014very}, ResNet101~\cite{he2016deep}, CECCN~\cite{ruan2024color}, EmoVIT~\cite{xie2024emovit}, WSCNet~\cite{yang2018weakly}, ECWA~\cite{deng2021emotion}, EECon~\cite{yang2023exploiting}, MAM~\cite{zhang2024affective} and TGCA-PVT~\cite{chen2024tgca}, and the overall results are summarized in Table~\ref{tab:cap1}.
We compare our model against several baselines, and the overall results are summarized in Table~\ref{tab:cap1}.
We observe that our model achieves the best performance in both accuracy and F1 metrics, significantly outperforming the previous models. 
This superior performance is mainly attributed to our effective utilization of metadata to enhance image sentiment analysis, as well as the exceptional capability of the unified sentiment transformer framework we developed. These results strongly demonstrate that our proposed method can bring encouraging performance for image sentiment analysis.

\setcounter{magicrownumbers}{0} 
\begin{table}[t]
\begin{center}
\caption{Ablation study of~\shortname~on FI dataset.} 
% \vspace{1mm}
\label{tab:cap2}
\resizebox{.9\linewidth}{!}
{
\begin{tabular}{lcc}
  \hline
  \textbf{Model} & \textbf{Accuracy} & \textbf{F1} \\
  \hline
  (\rownumber)~Ours (w/o vision) & 65.72\% & 64.54\% \\
  (\rownumber)~Ours (w/o text description) & 74.05\% & 72.58\% \\
  (\rownumber)~Ours (w/o object tag) & 77.45\% & 76.84\% \\
  (\rownumber)~Ours (w/o scene tag) & 78.47\% & 78.21\% \\
  \hline
  (\rownumber)~Ours (w/o unified embedding) & 76.41\% & 76.23\% \\
  (\rownumber)~Ours (w/o adaptive learning) & 76.83\% & 76.56\% \\
  (\rownumber)~Ours (w/o cross-modal fusion) & 76.85\% & 76.49\% \\
  \hline
  (\rownumber)~Ours  & \textbf{79.48\%} & \textbf{79.22\%} \\
  \hline
\end{tabular}
}
\end{center}
\vspace{-5mm}
\end{table}


\begin{figure}[t]
\centering
% \vspace{-2mm}
\includegraphics[width=0.42\textwidth]{fig/2dvisual-linux4-paper2.pdf}
\caption{Visualization of feature distribution on eight categories before (left) and after (right) model processing.}
% 
\label{fig:visualization}
\vspace{-5mm}
\end{figure}

\subsection{Ablation Performance}
In this subsection, we conduct an ablation study to examine which component is really important for performance improvement. The results are reported in Table~\ref{tab:cap2}.

For information utilization, we observe a significant decline in model performance when visual features are removed. Additionally, the performance of \shortname~decreases when different metadata are removed separately, which means that text description, object tag, and scene tag are all critical for image sentiment analysis.
Recalling the model architecture, we separately remove transformer layers of the unified representation module, the adaptive learning module, and the cross-modal fusion module, replacing them with MLPs of the same parameter scale.
In this way, we can observe varying degrees of decline in model performance, indicating that these modules are indispensable for our model to achieve better performance.

\subsection{Visualization}
% 


% % 开始使用minipage进行左右排列
% \begin{minipage}[t]{0.45\textwidth}  % 子图1宽度为45%
%     \centering
%     \includegraphics[width=\textwidth]{2dvisual.pdf}  % 插入图片
%     \captionof{figure}{Visualization of feature distribution.}  % 使用captionof添加图片标题
%     \label{fig:visualization}
% \end{minipage}


% \begin{figure}[t]
% \centering
% \vspace{-2mm}
% \includegraphics[width=0.45\textwidth]{fig/2dvisual.pdf}
% \caption{Visualization of feature distribution.}
% \label{fig:visualization}
% % \vspace{-4mm}
% \end{figure}

% \begin{figure}[t]
% \centering
% \vspace{-2mm}
% \includegraphics[width=0.45\textwidth]{fig/2dvisual-linux3-paper.pdf}
% \caption{Visualization of feature distribution.}
% \label{fig:visualization}
% % \vspace{-4mm}
% \end{figure}



\begin{figure}[tbp]   
\vspace{-4mm}
  \centering            
  \subfloat[Depth of adaptive learning layers]   
  {
    \label{fig:subfig1}\includegraphics[width=0.22\textwidth]{fig/fig_sensitivity-a5}
  }
  \subfloat[Depth of fusion layers]
  {
    % \label{fig:subfig2}\includegraphics[width=0.22\textwidth]{fig/fig_sensitivity-b2}
    \label{fig:subfig2}\includegraphics[width=0.22\textwidth]{fig/fig_sensitivity-b2-num.pdf}
  }
  \caption{Sensitivity study of \shortname~on different depth. }   
  \label{fig:fig_sensitivity}  
\vspace{-2mm}
\end{figure}

% \begin{figure}[htbp]
% \centerline{\includegraphics{2dvisual.pdf}}
% \caption{Visualization of feature distribution.}
% \label{fig:visualization}
% \end{figure}

% In Fig.~\ref{fig:visualization}, we use t-SNE~\cite{van2008visualizing} to reduce the dimension of data features for visualization, Figure in left represents the metadata features before model processing, the features are obtained by embedding through the CLIP model, and figure in right shows the features of the data after model processing, it can be observed that after the model processing, the data with different label categories fall in different regions in the space, therefore, we can conclude that the Therefore, we can conclude that the model can effectively utilize the information contained in the metadata and use it to guide the model for classification.

In Fig.~\ref{fig:visualization}, we use t-SNE~\cite{van2008visualizing} to reduce the dimension of data features for visualization.
The left figure shows metadata features before being processed by our model (\textit{i.e.}, embedded by CLIP), while the right shows the distribution of features after being processed by our model.
We can observe that after the model processing, data with the same label are closer to each other, while others are farther away.
Therefore, it shows that the model can effectively utilize the information contained in the metadata and use it to guide the classification process.

\subsection{Sensitivity Analysis}
% 
In this subsection, we conduct a sensitivity analysis to figure out the effect of different depth settings of adaptive learning layers and fusion layers. 
% In this subsection, we conduct a sensitivity analysis to figure out the effect of different depth settings on the model. 
% Fig.~\ref{fig:fig_sensitivity} presents the effect of different depth settings of adaptive learning layers and fusion layers. 
Taking Fig.~\ref{fig:fig_sensitivity} (a) as an example, the model performance improves with increasing depth, reaching the best performance at a depth of 4.
% Taking Fig.~\ref{fig:fig_sensitivity} (a) as an example, the performance of \shortname~improves with the increase of depth at first, reaching the best performance at a depth of 4.
When the depth continues to increase, the accuracy decreases to varying degrees.
Similar results can be observed in Fig.~\ref{fig:fig_sensitivity} (b).
Therefore, we set their depths to 4 and 6 respectively to achieve the best results.

% Through our experiments, we can observe that the effect of modifying these hyperparameters on the results of the experiments is very weak, and the surface model is not sensitive to the hyperparameters.


\subsection{Zero-shot Capability}
% 

% (1)~GCH~\cite{2010Analyzing} & 21.78\% & (5)~RA-DLNet~\cite{2020A} & 34.01\% \\ \hline
% (2)~WSCNet~\cite{2019WSCNet}  & 30.25\% & (6)~CECCN~\cite{ruan2024color} & 43.83\% \\ \hline
% (3)~PCNN~\cite{2015Robust} & 31.68\%  & (7)~EmoVIT~\cite{xie2024emovit} & 44.90\% \\ \hline
% (4)~AR~\cite{2018Visual} & 32.67\% & (8)~Ours (Zero-shot) & 47.83\% \\ \hline


\begin{table}[t]
\centering
\caption{Zero-shot capability of \shortname.}
\label{tab:cap3}
\resizebox{1\linewidth}{!}
{
\begin{tabular}{lc|lc}
\hline
\textbf{Model} & \textbf{Accuracy} & \textbf{Model} & \textbf{Accuracy} \\ \hline
(1)~WSCNet~\cite{2019WSCNet}  & 30.25\% & (5)~MAM~\cite{zhang2024affective} & 39.56\%  \\ \hline
(2)~AR~\cite{2018Visual} & 32.67\% & (6)~CECCN~\cite{ruan2024color} & 43.83\% \\ \hline
(3)~RA-DLNet~\cite{2020A} & 34.01\%  & (7)~EmoVIT~\cite{xie2024emovit} & 44.90\% \\ \hline
(4)~CDA~\cite{han2023boosting} & 38.64\% & (8)~Ours (Zero-shot) & 47.83\% \\ \hline
\end{tabular}
}
\vspace{-5mm}
\end{table}

% We use the model trained on the FI dataset to test on the artphoto dataset to verify the model's generalization ability as well as robustness to other distributed datasets.
% We can observe that the MESN model shows strong competitiveness in terms of accuracy when compared to other trained models, which suggests that the model has a good generalization ability in the OOD task.

To validate the model's generalization ability and robustness to other distributed datasets, we directly test the model trained on the FI dataset, without training on Artphoto. 
% As observed in Table 3, compared to other models trained on Artphoto, we achieve highly competitive zero-shot performance, indicating that the model has good generalization ability in out-of-distribution tasks.
From Table~\ref{tab:cap3}, we can observe that compared with other models trained on Artphoto, we achieve competitive zero-shot performance, which shows that the model has good generalization ability in out-of-distribution tasks.


%%%%%%%%%%%%
%  E2E     %
%%%%%%%%%%%%


\section{Conclusion}
In this paper, we introduced Wi-Chat, the first LLM-powered Wi-Fi-based human activity recognition system that integrates the reasoning capabilities of large language models with the sensing potential of wireless signals. Our experimental results on a self-collected Wi-Fi CSI dataset demonstrate the promising potential of LLMs in enabling zero-shot Wi-Fi sensing. These findings suggest a new paradigm for human activity recognition that does not rely on extensive labeled data. We hope future research will build upon this direction, further exploring the applications of LLMs in signal processing domains such as IoT, mobile sensing, and radar-based systems.

\section*{Limitations}
While our work represents the first attempt to leverage LLMs for processing Wi-Fi signals, it is a preliminary study focused on a relatively simple task: Wi-Fi-based human activity recognition. This choice allows us to explore the feasibility of LLMs in wireless sensing but also comes with certain limitations.

Our approach primarily evaluates zero-shot performance, which, while promising, may still lag behind traditional supervised learning methods in highly complex or fine-grained recognition tasks. Besides, our study is limited to a controlled environment with a self-collected dataset, and the generalizability of LLMs to diverse real-world scenarios with varying Wi-Fi conditions, environmental interference, and device heterogeneity remains an open question.

Additionally, we have yet to explore the full potential of LLMs in more advanced Wi-Fi sensing applications, such as fine-grained gesture recognition, occupancy detection, and passive health monitoring. Future work should investigate the scalability of LLM-based approaches, their robustness to domain shifts, and their integration with multimodal sensing techniques in broader IoT applications.


% Bibliography entries for the entire Anthology, followed by custom entries
%\bibliography{anthology,custom}
% Custom bibliography entries only
\bibliography{main}
\newpage
\appendix

\section{Experiment prompts}
\label{sec:prompt}
The prompts used in the LLM experiments are shown in the following Table~\ref{tab:prompts}.

\definecolor{titlecolor}{rgb}{0.9, 0.5, 0.1}
\definecolor{anscolor}{rgb}{0.2, 0.5, 0.8}
\definecolor{labelcolor}{HTML}{48a07e}
\begin{table*}[h]
	\centering
	
 % \vspace{-0.2cm}
	
	\begin{center}
		\begin{tikzpicture}[
				chatbox_inner/.style={rectangle, rounded corners, opacity=0, text opacity=1, font=\sffamily\scriptsize, text width=5in, text height=9pt, inner xsep=6pt, inner ysep=6pt},
				chatbox_prompt_inner/.style={chatbox_inner, align=flush left, xshift=0pt, text height=11pt},
				chatbox_user_inner/.style={chatbox_inner, align=flush left, xshift=0pt},
				chatbox_gpt_inner/.style={chatbox_inner, align=flush left, xshift=0pt},
				chatbox/.style={chatbox_inner, draw=black!25, fill=gray!7, opacity=1, text opacity=0},
				chatbox_prompt/.style={chatbox, align=flush left, fill=gray!1.5, draw=black!30, text height=10pt},
				chatbox_user/.style={chatbox, align=flush left},
				chatbox_gpt/.style={chatbox, align=flush left},
				chatbox2/.style={chatbox_gpt, fill=green!25},
				chatbox3/.style={chatbox_gpt, fill=red!20, draw=black!20},
				chatbox4/.style={chatbox_gpt, fill=yellow!30},
				labelbox/.style={rectangle, rounded corners, draw=black!50, font=\sffamily\scriptsize\bfseries, fill=gray!5, inner sep=3pt},
			]
											
			\node[chatbox_user] (q1) {
				\textbf{System prompt}
				\newline
				\newline
				You are a helpful and precise assistant for segmenting and labeling sentences. We would like to request your help on curating a dataset for entity-level hallucination detection.
				\newline \newline
                We will give you a machine generated biography and a list of checked facts about the biography. Each fact consists of a sentence and a label (True/False). Please do the following process. First, breaking down the biography into words. Second, by referring to the provided list of facts, merging some broken down words in the previous step to form meaningful entities. For example, ``strategic thinking'' should be one entity instead of two. Third, according to the labels in the list of facts, labeling each entity as True or False. Specifically, for facts that share a similar sentence structure (\eg, \textit{``He was born on Mach 9, 1941.''} (\texttt{True}) and \textit{``He was born in Ramos Mejia.''} (\texttt{False})), please first assign labels to entities that differ across atomic facts. For example, first labeling ``Mach 9, 1941'' (\texttt{True}) and ``Ramos Mejia'' (\texttt{False}) in the above case. For those entities that are the same across atomic facts (\eg, ``was born'') or are neutral (\eg, ``he,'' ``in,'' and ``on''), please label them as \texttt{True}. For the cases that there is no atomic fact that shares the same sentence structure, please identify the most informative entities in the sentence and label them with the same label as the atomic fact while treating the rest of the entities as \texttt{True}. In the end, output the entities and labels in the following format:
                \begin{itemize}[nosep]
                    \item Entity 1 (Label 1)
                    \item Entity 2 (Label 2)
                    \item ...
                    \item Entity N (Label N)
                \end{itemize}
                % \newline \newline
                Here are two examples:
                \newline\newline
                \textbf{[Example 1]}
                \newline
                [The start of the biography]
                \newline
                \textcolor{titlecolor}{Marianne McAndrew is an American actress and singer, born on November 21, 1942, in Cleveland, Ohio. She began her acting career in the late 1960s, appearing in various television shows and films.}
                \newline
                [The end of the biography]
                \newline \newline
                [The start of the list of checked facts]
                \newline
                \textcolor{anscolor}{[Marianne McAndrew is an American. (False); Marianne McAndrew is an actress. (True); Marianne McAndrew is a singer. (False); Marianne McAndrew was born on November 21, 1942. (False); Marianne McAndrew was born in Cleveland, Ohio. (False); She began her acting career in the late 1960s. (True); She has appeared in various television shows. (True); She has appeared in various films. (True)]}
                \newline
                [The end of the list of checked facts]
                \newline \newline
                [The start of the ideal output]
                \newline
                \textcolor{labelcolor}{[Marianne McAndrew (True); is (True); an (True); American (False); actress (True); and (True); singer (False); , (True); born (True); on (True); November 21, 1942 (False); , (True); in (True); Cleveland, Ohio (False); . (True); She (True); began (True); her (True); acting career (True); in (True); the late 1960s (True); , (True); appearing (True); in (True); various (True); television shows (True); and (True); films (True); . (True)]}
                \newline
                [The end of the ideal output]
				\newline \newline
                \textbf{[Example 2]}
                \newline
                [The start of the biography]
                \newline
                \textcolor{titlecolor}{Doug Sheehan is an American actor who was born on April 27, 1949, in Santa Monica, California. He is best known for his roles in soap operas, including his portrayal of Joe Kelly on ``General Hospital'' and Ben Gibson on ``Knots Landing.''}
                \newline
                [The end of the biography]
                \newline \newline
                [The start of the list of checked facts]
                \newline
                \textcolor{anscolor}{[Doug Sheehan is an American. (True); Doug Sheehan is an actor. (True); Doug Sheehan was born on April 27, 1949. (True); Doug Sheehan was born in Santa Monica, California. (False); He is best known for his roles in soap operas. (True); He portrayed Joe Kelly. (True); Joe Kelly was in General Hospital. (True); General Hospital is a soap opera. (True); He portrayed Ben Gibson. (True); Ben Gibson was in Knots Landing. (True); Knots Landing is a soap opera. (True)]}
                \newline
                [The end of the list of checked facts]
                \newline \newline
                [The start of the ideal output]
                \newline
                \textcolor{labelcolor}{[Doug Sheehan (True); is (True); an (True); American (True); actor (True); who (True); was born (True); on (True); April 27, 1949 (True); in (True); Santa Monica, California (False); . (True); He (True); is (True); best known (True); for (True); his roles in soap operas (True); , (True); including (True); in (True); his portrayal (True); of (True); Joe Kelly (True); on (True); ``General Hospital'' (True); and (True); Ben Gibson (True); on (True); ``Knots Landing.'' (True)]}
                \newline
                [The end of the ideal output]
				\newline \newline
				\textbf{User prompt}
				\newline
				\newline
				[The start of the biography]
				\newline
				\textcolor{magenta}{\texttt{\{BIOGRAPHY\}}}
				\newline
				[The ebd of the biography]
				\newline \newline
				[The start of the list of checked facts]
				\newline
				\textcolor{magenta}{\texttt{\{LIST OF CHECKED FACTS\}}}
				\newline
				[The end of the list of checked facts]
			};
			\node[chatbox_user_inner] (q1_text) at (q1) {
				\textbf{System prompt}
				\newline
				\newline
				You are a helpful and precise assistant for segmenting and labeling sentences. We would like to request your help on curating a dataset for entity-level hallucination detection.
				\newline \newline
                We will give you a machine generated biography and a list of checked facts about the biography. Each fact consists of a sentence and a label (True/False). Please do the following process. First, breaking down the biography into words. Second, by referring to the provided list of facts, merging some broken down words in the previous step to form meaningful entities. For example, ``strategic thinking'' should be one entity instead of two. Third, according to the labels in the list of facts, labeling each entity as True or False. Specifically, for facts that share a similar sentence structure (\eg, \textit{``He was born on Mach 9, 1941.''} (\texttt{True}) and \textit{``He was born in Ramos Mejia.''} (\texttt{False})), please first assign labels to entities that differ across atomic facts. For example, first labeling ``Mach 9, 1941'' (\texttt{True}) and ``Ramos Mejia'' (\texttt{False}) in the above case. For those entities that are the same across atomic facts (\eg, ``was born'') or are neutral (\eg, ``he,'' ``in,'' and ``on''), please label them as \texttt{True}. For the cases that there is no atomic fact that shares the same sentence structure, please identify the most informative entities in the sentence and label them with the same label as the atomic fact while treating the rest of the entities as \texttt{True}. In the end, output the entities and labels in the following format:
                \begin{itemize}[nosep]
                    \item Entity 1 (Label 1)
                    \item Entity 2 (Label 2)
                    \item ...
                    \item Entity N (Label N)
                \end{itemize}
                % \newline \newline
                Here are two examples:
                \newline\newline
                \textbf{[Example 1]}
                \newline
                [The start of the biography]
                \newline
                \textcolor{titlecolor}{Marianne McAndrew is an American actress and singer, born on November 21, 1942, in Cleveland, Ohio. She began her acting career in the late 1960s, appearing in various television shows and films.}
                \newline
                [The end of the biography]
                \newline \newline
                [The start of the list of checked facts]
                \newline
                \textcolor{anscolor}{[Marianne McAndrew is an American. (False); Marianne McAndrew is an actress. (True); Marianne McAndrew is a singer. (False); Marianne McAndrew was born on November 21, 1942. (False); Marianne McAndrew was born in Cleveland, Ohio. (False); She began her acting career in the late 1960s. (True); She has appeared in various television shows. (True); She has appeared in various films. (True)]}
                \newline
                [The end of the list of checked facts]
                \newline \newline
                [The start of the ideal output]
                \newline
                \textcolor{labelcolor}{[Marianne McAndrew (True); is (True); an (True); American (False); actress (True); and (True); singer (False); , (True); born (True); on (True); November 21, 1942 (False); , (True); in (True); Cleveland, Ohio (False); . (True); She (True); began (True); her (True); acting career (True); in (True); the late 1960s (True); , (True); appearing (True); in (True); various (True); television shows (True); and (True); films (True); . (True)]}
                \newline
                [The end of the ideal output]
				\newline \newline
                \textbf{[Example 2]}
                \newline
                [The start of the biography]
                \newline
                \textcolor{titlecolor}{Doug Sheehan is an American actor who was born on April 27, 1949, in Santa Monica, California. He is best known for his roles in soap operas, including his portrayal of Joe Kelly on ``General Hospital'' and Ben Gibson on ``Knots Landing.''}
                \newline
                [The end of the biography]
                \newline \newline
                [The start of the list of checked facts]
                \newline
                \textcolor{anscolor}{[Doug Sheehan is an American. (True); Doug Sheehan is an actor. (True); Doug Sheehan was born on April 27, 1949. (True); Doug Sheehan was born in Santa Monica, California. (False); He is best known for his roles in soap operas. (True); He portrayed Joe Kelly. (True); Joe Kelly was in General Hospital. (True); General Hospital is a soap opera. (True); He portrayed Ben Gibson. (True); Ben Gibson was in Knots Landing. (True); Knots Landing is a soap opera. (True)]}
                \newline
                [The end of the list of checked facts]
                \newline \newline
                [The start of the ideal output]
                \newline
                \textcolor{labelcolor}{[Doug Sheehan (True); is (True); an (True); American (True); actor (True); who (True); was born (True); on (True); April 27, 1949 (True); in (True); Santa Monica, California (False); . (True); He (True); is (True); best known (True); for (True); his roles in soap operas (True); , (True); including (True); in (True); his portrayal (True); of (True); Joe Kelly (True); on (True); ``General Hospital'' (True); and (True); Ben Gibson (True); on (True); ``Knots Landing.'' (True)]}
                \newline
                [The end of the ideal output]
				\newline \newline
				\textbf{User prompt}
				\newline
				\newline
				[The start of the biography]
				\newline
				\textcolor{magenta}{\texttt{\{BIOGRAPHY\}}}
				\newline
				[The ebd of the biography]
				\newline \newline
				[The start of the list of checked facts]
				\newline
				\textcolor{magenta}{\texttt{\{LIST OF CHECKED FACTS\}}}
				\newline
				[The end of the list of checked facts]
			};
		\end{tikzpicture}
        \caption{GPT-4o prompt for labeling hallucinated entities.}\label{tb:gpt-4-prompt}
	\end{center}
\vspace{-0cm}
\end{table*}
% \section{Full Experiment Results}
% \begin{table*}[th]
    \centering
    \small
    \caption{Classification Results}
    \begin{tabular}{lcccc}
        \toprule
        \textbf{Method} & \textbf{Accuracy} & \textbf{Precision} & \textbf{Recall} & \textbf{F1-score} \\
        \midrule
        \multicolumn{5}{c}{\textbf{Zero Shot}} \\
                Zero-shot E-eyes & 0.26 & 0.26 & 0.27 & 0.26 \\
        Zero-shot CARM & 0.24 & 0.24 & 0.24 & 0.24 \\
                Zero-shot SVM & 0.27 & 0.28 & 0.28 & 0.27 \\
        Zero-shot CNN & 0.23 & 0.24 & 0.23 & 0.23 \\
        Zero-shot RNN & 0.26 & 0.26 & 0.26 & 0.26 \\
DeepSeek-0shot & 0.54 & 0.61 & 0.54 & 0.52 \\
DeepSeek-0shot-COT & 0.33 & 0.24 & 0.33 & 0.23 \\
DeepSeek-0shot-Knowledge & 0.45 & 0.46 & 0.45 & 0.44 \\
Gemma2-0shot & 0.35 & 0.22 & 0.38 & 0.27 \\
Gemma2-0shot-COT & 0.36 & 0.22 & 0.36 & 0.27 \\
Gemma2-0shot-Knowledge & 0.32 & 0.18 & 0.34 & 0.20 \\
GPT-4o-mini-0shot & 0.48 & 0.53 & 0.48 & 0.41 \\
GPT-4o-mini-0shot-COT & 0.33 & 0.50 & 0.33 & 0.38 \\
GPT-4o-mini-0shot-Knowledge & 0.49 & 0.31 & 0.49 & 0.36 \\
GPT-4o-0shot & 0.62 & 0.62 & 0.47 & 0.42 \\
GPT-4o-0shot-COT & 0.29 & 0.45 & 0.29 & 0.21 \\
GPT-4o-0shot-Knowledge & 0.44 & 0.52 & 0.44 & 0.39 \\
LLaMA-0shot & 0.32 & 0.25 & 0.32 & 0.24 \\
LLaMA-0shot-COT & 0.12 & 0.25 & 0.12 & 0.09 \\
LLaMA-0shot-Knowledge & 0.32 & 0.25 & 0.32 & 0.28 \\
Mistral-0shot & 0.19 & 0.23 & 0.19 & 0.10 \\
Mistral-0shot-Knowledge & 0.21 & 0.40 & 0.21 & 0.11 \\
        \midrule
        \multicolumn{5}{c}{\textbf{4 Shot}} \\
GPT-4o-mini-4shot & 0.58 & 0.59 & 0.58 & 0.53 \\
GPT-4o-mini-4shot-COT & 0.57 & 0.53 & 0.57 & 0.50 \\
GPT-4o-mini-4shot-Knowledge & 0.56 & 0.51 & 0.56 & 0.47 \\
GPT-4o-4shot & 0.77 & 0.84 & 0.77 & 0.73 \\
GPT-4o-4shot-COT & 0.63 & 0.76 & 0.63 & 0.53 \\
GPT-4o-4shot-Knowledge & 0.72 & 0.82 & 0.71 & 0.66 \\
LLaMA-4shot & 0.29 & 0.24 & 0.29 & 0.21 \\
LLaMA-4shot-COT & 0.20 & 0.30 & 0.20 & 0.13 \\
LLaMA-4shot-Knowledge & 0.15 & 0.23 & 0.13 & 0.13 \\
Mistral-4shot & 0.02 & 0.02 & 0.02 & 0.02 \\
Mistral-4shot-Knowledge & 0.21 & 0.27 & 0.21 & 0.20 \\
        \midrule
        
        \multicolumn{5}{c}{\textbf{Suprevised}} \\
        SVM & 0.94 & 0.92 & 0.91 & 0.91 \\
        CNN & 0.98 & 0.98 & 0.97 & 0.97 \\
        RNN & 0.99 & 0.99 & 0.99 & 0.99 \\
        % \midrule
        % \multicolumn{5}{c}{\textbf{Conventional Wi-Fi-based Human Activity Recognition Systems}} \\
        E-eyes & 1.00 & 1.00 & 1.00 & 1.00 \\
        CARM & 0.98 & 0.98 & 0.98 & 0.98 \\
\midrule
 \multicolumn{5}{c}{\textbf{Vision Models}} \\
           Zero-shot SVM & 0.26 & 0.25 & 0.25 & 0.25 \\
        Zero-shot CNN & 0.26 & 0.25 & 0.26 & 0.26 \\
        Zero-shot RNN & 0.28 & 0.28 & 0.29 & 0.28 \\
        SVM & 0.99 & 0.99 & 0.99 & 0.99 \\
        CNN & 0.98 & 0.99 & 0.98 & 0.98 \\
        RNN & 0.98 & 0.99 & 0.98 & 0.98 \\
GPT-4o-mini-Vision & 0.84 & 0.85 & 0.84 & 0.84 \\
GPT-4o-mini-Vision-COT & 0.90 & 0.91 & 0.90 & 0.90 \\
GPT-4o-Vision & 0.74 & 0.82 & 0.74 & 0.73 \\
GPT-4o-Vision-COT & 0.70 & 0.83 & 0.70 & 0.68 \\
LLaMA-Vision & 0.20 & 0.23 & 0.20 & 0.09 \\
LLaMA-Vision-Knowledge & 0.22 & 0.05 & 0.22 & 0.08 \\

        \bottomrule
    \end{tabular}
    \label{full}
\end{table*}




\end{document}

\end{document}
