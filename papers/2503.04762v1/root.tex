% \documentclass[conference]{IEEEtran}
\documentclass[letterpaper, 10 pt, conference]{ieeeconf}
\IEEEoverridecommandlockouts
% The preceding line is only needed to identify funding in the first footnote. If that is unneeded, please comment it out.
\usepackage{cite}
\usepackage[letterpaper, top=57pt, left=52pt, right=52pt, bottom=57pt]{geometry}
% \usepackage[noproof]{amsthm}
\usepackage{amsmath,amssymb,amsfonts}
\usepackage{algorithmic}
\usepackage{graphicx}
\usepackage{textcomp}
\usepackage{xcolor}
\usepackage{dsfont}
\def\BibTeX{{\rm B\kern-.05em{\sc i\kern-.025em b}\kern-.08em
    T\kern-.1667em\lower.7ex\hbox{E}\kern-.125emX}}

%%%%%%%%%%%%%%%%%%%%%%%%%%%%%%%%%%%%%%%%%%%%%%%%%%%%%%%
%%%%%%%%%%%%%%%    theorems %%%%%%%%%%%%%%%%%%%%%%%%%%%
%%%%%%%%%%%%%%%%%%%%%%%%%%%%%%%%%%%%%%%%%%%%%%%%%%%%%%%
% \usepackage{mdframed}
\usepackage{kantlipsum}

%%%%%%%%%%%%%%%%%%%%%%%%%%%%%%%%%%%%%%%%%%%%%%%%%%%%%%%
%%%%%%%%%%%%%%%    theorems %%%%%%%%%%%%%%%%%%%%%%%%%%%
%%%%%%%%%%%%%%%%%%%%%%%%%%%%%%%%%%%%%%%%%%%%%%%%%%%%%%%
\theoremstyle{plain}
\newtheorem{theorem}{Theorem}[section]
\newtheorem{proposition}[theorem]{Proposition}
\newtheorem{lemma}[theorem]{Lemma}
\newtheorem{example}[theorem]{Example}
\newtheorem{corollary}[theorem]{Corollary}
\theoremstyle{definition}
\newtheorem{definition}[theorem]{Definition}
\newtheorem{assumption}[theorem]{Assumption}
\theoremstyle{remark}
\newtheorem{remark}[theorem]{Remark}


% \titleformat{\subsection}[runin]% runin puts it in the same paragraph
%        {\normalfont\bfseries}% formatting commands to apply to the whole heading
%        {\thesubsection}% the label and number
%        {0.5em}% space between label/number and subsection title
%        {}% formatting commands applied just to subsection title
%        [.]% punctuation or other commands following subsection title


%%%%%%%%%%%%%%%%%%%%%%%%%%%%%%%%%%%%%%%%%%%%%%%%%%%%%%%
%%%%%%%%%%%%%%%  mathematical notations%%%%%%%%%%%%%%%%
% \usepackage[english]{babel}
% \usepackage[utf8]{inputenc}
% \usepackage[T1]{fontenc}

%% Figures, tables and lists
\usepackage[dvipsnames]{xcolor}
\usepackage{paralist}
\usepackage{graphicx}
\usepackage{subcaption}
\usepackage{longtable} 
\usepackage{multirow}
\usepackage{listings}
\usepackage{makecell}
\usepackage{array}
\usepackage{float}
\usepackage{dsfont}
\usepackage{rotating}
\usepackage{booktabs}
\usepackage{enumerate}
\usepackage{tikz}
\usepackage{pgf}
\usepackage{enumitem}
\usepackage{lipsum} % for generating filler text
\usepackage{titlesec}

%% Math
% \usepackage{amssymb, amsthm,bbm}
\usepackage{mathtools}
\usepackage{mathrsfs}
%% References and author info 
\mathtoolsset{showonlyrefs}
\usepackage{natbib}
\usepackage{authblk}
\usepackage{todonotes}
\usepackage{xr-hyper}


%%%%%%%%%%%%%%%%%%%%%%%%%%%%%%%%%%%%%%%%%%%%%%%%%%%%%%%
\newcommand{\R}{\mathbb R}
\newcommand{\EE}{\mathbb{E}}

\DeclareMathOperator{\Tr}{Tr}
\DeclareMathOperator*{\argmin}{argmin}
\DeclareMathOperator*{\argmax}{argmax}

\newcommand{\bs}[1]{\ensuremath{\boldsymbol{#1}}}
\newcommand{\mc}{\mathcal}
\newcommand{\opt}{^\star}


\newcommand{\diff}{\textnormal{d}}


\def \iid {\stackrel{\textnormal{i.i.d.}}{\sim}}
\def \iidtext {\textnormal{i.i.d.}}





%%%%%%%%%%%%%%%%%%%%%%%%%%%%%%%%%%%%%%%%%%%%%%%%%%%%%%%
%%%%%%%%%%%%%%%%%%%%% colors     %%%%%%%%%%%%%%%%%%%%%%
%%%%%%%%%%%%%%%%%%%%%%%%%%%%%%%%%%%%%%%%%%%%%%%%%%%%%%%
\definecolor{myblue}{rgb}{.8, .8, 1}
\definecolor{mathblue}{rgb}{0.2472, 0.24, 0.6} % mathematica's Color[1, 1--3]
\definecolor{mathred}{rgb}{0.6, 0.24, 0.442893}
\definecolor{mathyellow}{rgb}{0.6, 0.547014, 0.24}


% May add more in future.






    
\begin{document}

\title{Safety Verification of Stochastic Systems under Signal Temporal Logic Specifications
}

\author{Liqian Ma, Zishun Liu, Hongzhe Yu, and Yongxin Chen% <-this % stops a space
\thanks{The authors are with Georgia Institute of Technology, Atlanta, GA 30332 
        {\tt\small \{mlq\}\{zliu910\}\{hyu419\}\{yongchen\}@gatech.edu}}%
}


% \author{\IEEEauthorblockN{1\textsuperscript{st} Given Name Surname}
% \IEEEauthorblockA{\textit{dept. name of organization (of Aff.)} \\
% \textit{name of organization (of Aff.)}\\
% City, Country \\
% email address or ORCID}
% \and
% \IEEEauthorblockN{2\textsuperscript{nd} Given Name Surname}
% \IEEEauthorblockA{\textit{dept. name of organization (of Aff.)} \\
% \textit{name of organization (of Aff.)}\\
% City, Country \\
% email address or ORCID}
% \and
% \IEEEauthorblockN{3\textsuperscript{rd} Given Name Surname}
% \IEEEauthorblockA{\textit{dept. name of organization (of Aff.)} \\
% \textit{name of organization (of Aff.)}\\
% City, Country \\
% email address or ORCID}
% \and
% \IEEEauthorblockN{4\textsuperscript{th} Given Name Surname}
% \IEEEauthorblockA{\textit{dept. name of organization (of Aff.)} \\
% \textit{name of organization (of Aff.)}\\
% City, Country \\
% email address or ORCID}
% \and
% \IEEEauthorblockN{5\textsuperscript{th} Given Name Surname}
% \IEEEauthorblockA{\textit{dept. name of organization (of Aff.)} \\
% \textit{name of organization (of Aff.)}\\
% City, Country \\
% email address or ORCID}
% \and
% \IEEEauthorblockN{6\textsuperscript{th} Given Name Surname}
% \IEEEauthorblockA{\textit{dept. name of organization (of Aff.)} \\
% \textit{name of organization (of Aff.)}\\
% City, Country \\
% email address or ORCID}
% }

\maketitle

\begin{abstract}
We study the verification problem of stochastic systems under signal temporal logic (STL) specifications. We propose a novel approach that enables the verification of the probabilistic satisfaction of STL specifications for nonlinear systems subject to both bounded deterministic disturbances and stochastic disturbances. Our method, referred to as the STL erosion strategy, reduces the probabilistic verification problem into a deterministic verification problem with a tighter STL specification. The degree of tightening is determined by leveraging recent results on bounding the deviation between the stochastic trajectory and the deterministic trajectory. Our approach can be seamlessly integrated with any existing deterministic STL verification algorithm. Numerical experiments are conducted to showcase the efficacy of our method.
\end{abstract}

% \begin{IEEEkeywords}

% \end{IEEEkeywords}

\section{Introduction}
Safety is a critical consideration in various applications, including robots, autonomous vehicles, smart grids, and transportation control systems~\cite{wolf2017safety}. These safety-critical scenarios demand formal guarantees to ensure that systems operate as expected, as failures may result in severe consequences, such as harm to humans or significant financial costs. Safety verification refers to the task of determining whether a system satisfies a given safety specification over a specified period~\cite{guiochet2017safety, vicentini2019safety}. 
Conventional safe set specifications primarily focus on spatial requirements, ensuring that the system state never enters an unsafe region~\cite{prajna2004safety}. However, as the complexity of autonomous systems increases, many real-world tasks require specifications that are not only spatial but also temporal in nature. For instance, a mobile robot needs to pass Area A before entering Area B. In this paper, we focus on safety verification under Signal Temporal Logic (STL) specification, which uses both boolean and temporal logic operators to formulate constraints for continuous-valued systems~\cite{maler2004monitoring}. 

Real-world systems are subject to various types of uncertainty. It is essential for safety verification algorithms to account for disturbances. Many existing approaches model these uncertainties as bounded disturbances and employ worst-case analysis to guarantee the satisfaction of safety specifications. Examples of such approaches for safe set specification include Hamilton–Jacobi Reachability (HJ Reachability)~\cite{bansal2017hamilton}, reachability analysis, and barrier certificates~\cite{prajna2004safety}. For STL specifications, methods such as HJ Reachability~\cite{chen2018signal} and reachability analysis~\cite{roehm2016stl, lercher2024using, kochdumper2024fully} have been employed to formally verify STL satisfaction under bounded disturbance inputs.


In many practical situations, disturbances are better modeled as stochastic noise, which provides a more realistic representation, as in the case of sensor noise. When considering stochastic disturbances, the aforementioned deterministic methods are not applicable or tend to be overly conservative, as they focus on worst-case scenarios that rarely occur in practice. To better account for stochastic disturbances, we adopt a probabilistic setting, where the goal is to ensure the safety specification is satisfied with high probability, e.g., greater than 99.9\%. 
For safe set specifications, several methods have been proposed to verify stochastic systems, including martingale-based approaches~\cite{steinhardt2012finite, santoyo2021barrier}, risk estimation~\cite{frey2020collision}, and sampling-based methods~\cite{janson2017monte}. Our recent paper significantly reduces the conservativeness of the verification algorithms for safe set specifications~\cite{liu2024safety}.
For STL specifications, most existing approaches are limited to handling the probability constraint for a single trajectory satisfying the STL specification~\cite{sadigh2016safe, farahani2018shrinking, yang2023distributed, vlahakis2024probabilistic, kordabad2024control}. Very few studies have focused on STL verification under both bounded and stochastic disturbances. In \cite{salamati2021data}, a method is proposed to address this problem for linear systems under Gaussian noise. In this work, we focus on the problem of STL verification for nonlinear systems under both bounded and stochastic disturbances. 

In this work, we present a novel framework for verifying the probabilistic STL satisfaction of discrete-time nonlinear stochastic systems. To the best of our knowledge, this is the first approach capable of addressing this problem for nonlinear systems under both bounded and stochastic disturbances. Given a desired probability requirement, our method first erodes the superlevel set of the predicates in an STL formula to get a tighter STL formula. If the deterministic system is verified to satisfy the tighter STL formula, then the stochastic system is guaranteed to satisfy the original STL formula with the specified probability constraint. As a result, the stochastic verification problem is transformed into a deterministic one. The depth of erosion is determined by the sharp probabilistic bound proposed in our previous work~\cite{liu2024probabilistic}, which helps reduce the conservativeness of the verification result, especially when the probability tolerance is low and the time horizon is long. Our method does not rely on restrictive assumptions, such as linear system dynamics or affine predicates, which is common in previous work~\cite{vlahakis2024probabilistic}. This broader applicability makes our approach suitable for real-world applications.



\textit{Notations.}
% \textit{Vectors, matrices, and probability.} 
Denote by $\real$ and $\n$ the sets of real numbers and nonnegative integers, and define $\n_{[a,b]}=\setb{a, a+1, \dots, b}$ where $a,b\in \n$ and $a<b$. Given a vector sequence $\{x_t\}$, define $\boldsymbol{x}_{[t_1,t_2]} = (x_{t_1}, x_{t_1+1}, \dots, x_{t_2}) = [x\tran_{t_1}, x\tran_{t_1+1}, \dots, x\tran_{t_2}]\tran$ , $t\in\n_{[t_1,t_2]}$. When $X_t$ are random vectors, $\boldsymbol{X}_{[t_1,t_2]}$ is a random process. We use $\bP$ to denote probability. A random vector $X \sim \mathcal{N}(\mu, \Sigma)$ follows a multivariate Gaussian distribution with mean $\mu$ and covariance $\Sigma$.
Given a vector $x\in \real^n$, $\|x\|$ denotes the euclidean norm and $\|x\|_P = \sqrt{x\tran P x}$, where $P\in\real^{n\times n}$ is a positive definite matrix.
% \textit{Sets.} 
The $n$ dimensional ball with radius $r$ and center $y$ is denoted by $\BB^n(r, y)=\setb{x\in \real^n : \|x-y\| \leq r}$. Denote the complement of set $A$ as $\setcomp{A}$ and $-B = \setb{-y: \forall y\in B}$. Given sets $A$ and $B$, define the Minkowski sum of $A$ and $B$ by $A\oplus B = \setb{x+y: x\in A,~ y\in B}$, and the Minkowski difference or Pontryargin difference of $A$ and $B$ by $A\ominus B=\setb{x:x+y\in A, \forall y\in B}$ \cite{kolmanovsky1998theory}. The Minkowski sum and difference satisfy the relation $(A\ominus B)\oplus B \subseteq A$.





    
\section{Problem Formulation}
In this section, we provide an overview of signal temporal logic and then formulate the STL verification problem for nonlinear stochastic systems.

\subsection{Signal Temporal Logic}
In this work, we use STL to specify the spatio-temporal properties of systems for safety verification. 
We consider the STL formula with a standard syntax
\begin{equation}
\label{eq: standard STL syntax}
    \varphi := \mathsf{T} {}\mid{} \pi {}\mid{} \neg\varphi {}\mid{} \varphi_1\wedge\varphi_2 {}\mid{} \varphi_1 \UU_{[t_1,t_2]} \varphi_2,
\end{equation}
where $\mathsf{T}$ denotes \textit{True}, $\pi:=(\mu(x)\geq 0)$ is a predicate, $\neg$ and $\wedge$ denote negation and conjunction, and $\UU_{[t_1,t_2]}$ is the temporal \textit{until} operator. An STL formula $\varphi$ is recursively constructed using the operators introduced above. We use $\pi \in \varphi$ to denote that the predicate $\pi$ is used to construct $\varphi$. Other operators can be constructed using these operators. For example, temporal \textit{eventually} operator $\lozenge_{[t_1,t_2]}\varphi = \mathsf{T} \UU_{[t_1,t_2]}\varphi$, temporal \textit{globally} operator $\square_{[t_1,t_2]}\varphi = \neg \lozenge_{[t_1,t_2]}\neg \varphi$, and disjunction $\varphi_1\vee \varphi_2 = \neg (\neg \varphi_1 \wedge \neg \varphi_2)$.

The boolean semantics of the STL formula are defined over the system trajectory $\boldsymbol{x}_{[t,\infty]}$ ~\cite{maler2004monitoring}.
Denote by $\boldsymbol{x}_{[t,\infty]} \models \varphi$ a trajectory $\boldsymbol{x}_{[t,\infty]}$ satisfies $\varphi$. The boolean semantics can be recursively defined by: 
$\boldsymbol{x}_{[t,\infty]} \models \pi \Leftrightarrow \mu(x_t) \geq 0$, 
$\boldsymbol{x}_{[t,\infty]} \models \neg \varphi \Leftrightarrow \neg(\boldsymbol{x}_{[t,\infty]} \models \varphi)$, 
$\boldsymbol{x}_{[t,\infty]} \models \varphi_1 \wedge \varphi_2 \Leftrightarrow \boldsymbol{x}_{[t,\infty]} \models \varphi_1 \wedge \boldsymbol{x}_{[t,\infty]} \models \varphi_2$,
$\boldsymbol{x}_{[t,\infty]} \models \varphi_1 \UU_{[t_1,t_2]}\varphi_2 \Leftrightarrow \exists \tau\in \n_{[t+t_1,t+t_2]}$, 
$\st \boldsymbol{x}_{[\tau,\infty]} \models \varphi_2 \wedge \forall \tau'\in \n_{[t,\tau]}, \boldsymbol{x}_{[\tau',\infty]} \models \varphi_1$. 
% We can check whether $\varphi$ is satisfied by recursively applying boolean semantics.

The horizon of an STL formula $\varphi$ is denoted as $h^\varphi$, which is the length of the trajectory that is required to determine the satisfaction of $\varphi$~\cite{belta2019formal}. $h^\varphi$ can be calculated recursively by: $h^\pi=0$, $h^{\neg \varphi} = h^\varphi$, $h^{\varphi_1 \wedge \varphi_2} = \max(h^\varphi_1, h^\varphi_2)$, $h^{\varphi_1 \UU_{[t_1,t_2]}\varphi_2} = t_2 + \max(h^\varphi_1, h^\varphi_2)$.


To simplify the analysis, we assume that all STL formulas are converted into a negation-free form, meaning the formulas do not contain any negations. This conversion is always possible by first transforming the formula into Negation Normal Form~\cite{fainekos2009robustness} and then introducing new predicates with reversed inequalities as needed~\cite[Proposition 2]{belta2019formal, sadraddini2015robust}.

\subsection{Stochastic Systems}
{\em Dynamics:}
Consider the discrete-time stochastic system 
\begin{equation}
\label{eq: stochastic dynamics}
     X_{t+1}=f(X_t,d_t,t)+w_t
\end{equation}
where $X_t\in \real^n$, $d_t\in \DD \subset \real^m$, $w_t\in \real^n$ are the state, input, and stochastic disturbance at time $t \in \n$. The input $d_t$ represents bounded disturbances, whose statistical property is unknown. We assume $f: \real^n\times\real^p\times \n \ra\real^n$ is globally Lipschitz (Assumption \ref{ass: Lipschitz f}). 
\begin{assumption}[Lipschitz]
\label{ass: Lipschitz f}
For $\forall t \in \n$, there exists a $L_t$, such that for all $x, y\in \real^n$, and $d\in\DD$, 
\begin{equation*}
    \|f(x,d,t) - f(y,d,t)\| \leq L_t\|x-y\|.
\end{equation*}
\end{assumption}

{\em Stochastic Disturbance:}
We assume the stochastic disturbance $w_t$ is sub-Gaussian, which covers a wide range of distributions in the real world including Gaussian distribution and uniform distribution with bounded support.

\begin{definition}
A random vector $ X \in \real^n$ is said to be sub-Gaussian if $\EE(X)=0$ and there exists a positive constant $ \sigma > 0 $ such that for any $\ell$ on the unit sphere, $\EE_X \left[ e^{\lambda \langle\ell, X\rangle} \right] \leq e^{\frac{\sigma^2 \lambda^2}{2}}$ holds for all $ \lambda \in \real $. Here, $ \sigma^2 $ is called variance proxy. We use $X\sim subG(\sigma^2)$ to denote $X$ is sub-Gaussian.
\end{definition}
% \hongzhe{This assumption seems too short to be an Assumption, maybe can move to the main paper with one sentence.}
\begin{assumption}\label{ass:subG}
    For system~\eqref{eq: stochastic dynamics}, 
    $w_t \sim subG(\sigma_t^2)$, where $\sigma_t>0$, $\forall t>0$.
\end{assumption}

\subsection{Problem Statement}
We consider the safety verification problem of the stochastic system \eqref{eq: stochastic dynamics} under the STL specification \eqref{eq: standard STL syntax}.
We first recall the STL satisfaction of a deterministic system under an STL specification.
\begin{definition}
Consider the deterministic version of the system dynamics
    \begin{equation}
    \label{eq: deterministic dynamics}
    x_{t+1}=f(x_t,d_t,t)
    \end{equation}
with a set of initial states $\XX_0$, and a bounded set of input $\DD$. Given an STL specification $\varphi$ with a bounded horizon $T$, system \eqref{eq: deterministic dynamics} is said to satisfy $\varphi$ if,
    \begin{equation}\label{eq: probability constraint}
        \forall x_0 \in \XX_0, \forall t\in\n_{[0,T]}, d_t \in \DD: \boldsymbol{x}_{[0,T]} \models \varphi.
    \end{equation}
\end{definition}

This definition can be conservative for the stochastic system \eqref{eq: stochastic dynamics} whose trajectories are often unbounded, inevitably leading to a violation of the STL specification. Therefore, we turn to a probabilistic version of STL satisfaction~\cite{sadigh2016safe, farahani2018shrinking}.
\begin{definition}
Consider the stochastic system \eqref{eq: stochastic dynamics} with a set of initial states $\XX_0$, and a bounded set of input $\DD$. Given an STL specification $\varphi$ with a bounded horizon $T$, and a probability tolerance $\delta \in [0,1]$, the system is said to satisfy $\varphi$ with $1-\delta$ guarantee, if
    \begin{equation}\label{eq: probability constraint}
        \forall x_0 \in \XX_0, \forall t\in\n_{[0,T]}, d_t \in \DD: \bP \big(\boldsymbol{X}_{[0,T]} \models \varphi \big) \geq 1-\delta.
    \end{equation}

\end{definition}


\begin{remark}
$\XX_0$ and $\DD$ may each contain only a single point. For instance, if $\XX_0 = \{x_0\}$ and $\DD = \{0\}$, this corresponds to considering a single initial state $x_0$ and no deterministic disturbance, which is a common setting in optimal control problems~\cite{vlahakis2024probabilistic}. 
\end{remark}

We are interested in the problem of verifying a stochastic system under an STL specification.
\begin{problem}[Verification]
\label{prob: verification}
Consider a stochastic system \eqref{eq: stochastic dynamics} under Assumptions \ref{ass: Lipschitz f}-\ref{ass:subG}, and an STL specification $\varphi$. Verify whether the system satisfies $\varphi$ with $1-\delta$ guarantee. 
\end{problem}


\section{Method}

\subsection{Problem Statement}
Our problem setting involves a natural language task instruction $\task$ and a scene graph $\gG = (\gV, \gE)$, where $\gV$ and $\gE$ denote vertices and edges, respectively. Each node $\gV_i$ represents an object along with its attributes, such as coordinates or colors, while each edge indicates a type of spatial relationship, such as inside or on top of. Additionally, we assume access to the \textit{scene graph schema} $\gs$, which is a textual description of types, formats, and the semantics of the graph vertices and edges. Our objective is to generate the solution of $\task$ using LLMs, based on the available information above, expressed as $\sol = f(\task, \gG, \gs; LLMs)$.

%%%%% figure %%%%%%%%%%%%%%%%%%%%%%%%%%%%%%%%%%%%%%%%%%%%%%%%%
\begin{figure*}[t!]
 
  % \vspace*{-0.1in}
  \centering
  \scalebox{0.92}{
    \begin{tikzpicture}
     \node[anchor=north west] at (0in,0in)
      {{\includegraphics[width=1.0\textwidth,clip=true,trim=0
      0pt 0 0]{figs/method.pdf}}};
%     \node[yshift=-0pt,anchor=north west] at (0.1in,0.0in) {\bf \small (a)};
%     \node[anchor=north west] at (0.92in,-0.05in) {\textbf{(a)}};
%   \node[anchor=north west] at (2.00in,-0.05in) {\textbf{(b)}};
%     \node[anchor=north west] at (3.09in,-0.05in) {\textbf{(c)}};
    \end{tikzpicture}
  }
  \vspace*{-0.12in}
  \caption{\textbf{\RwR Workflow}.
  It solves tasks on scene graphs through the cooperation of two LLM agents: Reasoner and Retriever. Reasoner iteratively queries Retriever for graph information and reasons based on the received data from the Retriever. 
  The scene graph schema is prompted to synergize the reasoning and retrieval.
  Additionally, both agents employ the code-writing skill: Retriever programs to retrieve graph information based on the schema, while the Reasoner writes code to utilize external tools for solving complex atomic problems. In the graph, 
  \protect{\raisebox{-.05cm}{\includegraphics[height=.30cm]{figs/code.png}}} and 
  \protect{\raisebox{-.05cm}{\includegraphics[height=.30cm]{figs/code_exe.png}}}
  represent code writing and execution, respectively.
  They retrieve graph information $\bm{\gG}^\prime$ or enhance the analysis $\bm{\anly}$.
  }
  \vspace*{-0.15in}
 \label{fig:Method}
\end{figure*}
%%%%%%%%%%%%%%%%%%%%%%%%%%%%%%%%%%%%%%%%%%%%%%%%%%%%%%%%%%%%%

\subsection{Overview of \RwR}
%TODO: remove repeated statements and add two-agent framework formula.
We explore grounding the reasoning process to scene graphs based on the scene graph schema $\gs$ and the code-writing ability of LLMs.
We develop \RwR, a two-agent framework that iteratively reasons through the next steps and retrieves necessary information from the graph.
As shown in Figure \ref{fig:Method}, our method contains two LLM agents: a \textit{Reasoner} and a \textit{Retriever}. 
The system initializes with the Scene Graph Schema, the Environment Description, general Guidance to direct the cooperation process, and task-dependent information such as the description of Agent Actions and Reasoning Tools. 
Given a task, the Reasoner determines the next substep to approach the task and identifies necessary scene graph information. It then raises a natural language query to the Retriever for this information. Upon receiving the query, the Retriever processes the scene graph through code-writing and sends the data back to Reasoner. By iteratively performing these steps, both agents collaborate to solve the task. 
Formally, at each time step $t$:
\begin{align}
    \anly_t, \query_t &= Reasoner(\{\anly_0, \query_0, \gG^\prime_0\}, \{\anly_{1}, \query_{1}, \gG^\prime_{1}\}, \cdots; \gs) \\
    \code_t &= Retriever(\query_t; \gs) \\
    \gG^\prime_{t} &= \code_t(\gG)
\end{align}
where $\anly$ denotes the current analysis by the Reasoner; $\query$ denotes natural language query for the graph information; $\code$ denotes the retrieval code following the query; and $\gG^\prime$ denotes the retrieved information by executing the code on the scene graph $\gG$.

Importantly, unlike previous iterative methods \cite{react, iterRG} that uses a single LLM to process the entire history, the two agents in \RwR only exchange the query $\query$ and the corresponding graph data $\gG^\prime$, excluding the underlying thought process, such as $\anly$ and $\code$. 
As we will show, this agent-level context filtering, enabled by our two-agent design, is critical for eliminating the interference from irrelevant conversation history, thereby ensuring a seamless and automated cooperative task-solving process.

% including the {\em Explanation} of intermediate conclusions in language and the {\em Reasoning Code} for tool using or sub-task solving;

% Our system initializes with the Scene Graph Schema, the Environment Description, general Guidance to direct the cooperation process, and task-dependent information such as the description of Agent Actions and Reasoning Tools. Then, given the Task, the Reasoner outputs analysis in natural language labeled as {\em Explanation}, and {\em Query} the Retriever. In turn, given the Scene Graph and a Query, the Retriever provides structured responses grounded in the Scene Graph. This process iterates until the Reasoner outputs a plan.

% The next subsections explain workflows of each agent, as well as techniques that ensure a fluent and automated task-solving process.

\subsection{Reasoner}

Reasoner is the central agent steering the task-solving iterations. We prompt it with the schema $\gs$, environment and task information (such as action description for the planning task), annotations of reasoning tools, general guidance to ensure automated task-solving conversation, and optionally, few-shot task-level examples. Reasoner then initiates the conversation with Retriever to solve a given task.

Concretely, without any knowledge about the graph data initially, the Reasoner analyzes only the task $\task$ and graph schema $\gs$, generates the first analysis $\anly_0$, and sends out the first associated query $\query_0$ to the Retriever. At the $t^{th}$ round of conversation, the Reasoner consumes past analyses, queries, and retrieved information: $\{(\anly_0, \query_0, \gG^\prime_0), \cdots, (\anly_{t-1}, \query_{t-1}, \gG^\prime_{t-1})\}$.
It then generates the next corresponding analysis $\anly_t$ and query $\query_{t}$, where $\anly_t$ involves intermediate conclusions and the next subtask to be solved, which informs and justifies $\query_{t}$.
For example, in the $2^{nd}$ round of conversation shown in Figure  \ref{fig:Method}, Reasoner processes previously retrieved agent and red box room and location ($\{(\anly_0, \query_0, \gG^\prime_0), (\anly_{1}, \query_{1}, \gG^\prime_{1})$),
identifies that the next subtask is to find \texttt{\small "the path between two rooms"} ($\anly_2$), and then query for the \texttt{\small "door IDs and attributes"} that connect two rooms ($\query_2$) for solving the subtask. In this way, each reasoning step is grounded to the environment by factoring in the retrieved information. %, and the graph data processed by LLMs is filtered by the reasoning.

The analysis $\anly_t$ might involve solving complex spatial sub-problems, such as navigation and object search. Past literature shows that LLMs give unreliable solutions to quantitative problems \citep{llmMathReason}. To circumvent the deficiency, we follow prior work \citep{toolformer, ART} to enable code-writing and tool-use for the Reasoner. We provide programmatic functions to address atomic problems critical to the given task family. As shown in Figure \ref{fig:Method}, at the $t^{th}$ round of conversation, the Reasoner uses the provided pathfinding tool \texttt{\small traverse\_room} to identify obstacles that need to be removed to traverse to the key, a problem beyond the capacity of LLMs. We include tool annotations in the prompt to guide the Reasoner in querying for the information necessary. 
The introduction of tools prevents hallucination on complex problems and reduces the burden of LLMs by leveraging known algorithms.

%% Below is redundant
% Since the Reasoner controls the iterative process to address a task, it is critical to control its behavior to ensure a smooth flow of the conversation. We control the message exchange between the Reasoner and the Retriever through both prompt guidance and manual interference. Specifically, we prompt the Reasoner with the graph schema and the guidance to \texttt{\small "Communicate using the terms in the graph schema"} to avoid confusion. 
% \red{full prompt linked to appendix?}
% \red{The part below can be updated based on the rebuttal.}
% We also filter out only the next query $\query_{t+1}$ to send to the Retriever, removing the analysis $\anly_t$ and the past conversation. We find that without doing so, the Retriever might attempt to realize all plan steps in the language analysis in the conversation, while omitting the actual desired information, which leads to a failure eventually.

\subsection{Retriever}

The Retriever assists the Reasoner by processing its free-form queries and returning the requested information from the graph. 
Specifically, given a query $\query$, the Retriever generates code $\code$ that executes on the scene graph to retrieve the required information $\gG^\prime$. Here, $\gV^\prime$ and $\gE^\prime$ denote subsets of graph nodes and edges, respectively. While the Reasoner may query for either the entire node or edge or just a subset of their attributes, we use $\gV^\prime$ and $\gE^\prime$ as the general representation for either case.
The code-writing strategy offers significant advantages over traditional API-calling methods.
As shown in Fig. \ref{fig:Method}, it enables efficient graph traversal by iterating through nodes, edges, or attributes using loops. 
It also supports query-oriented information filtering through logical structures such as conditional statements. 
These capabilities ensure that the retrieved information is well-aligned with the reasoning demands.

Similar to the prompt for the Reasoner, the prompt for the  Retriever includes the environment description, the scene graph schema $\gs$, and general guidance. The key difference is that $\gs$ guides the Retriever in writing the information retrieval code. Confusion is avoided by ensuring that both agents communicate using the same terms from the schema.

\subsection{Self-debugging and Error prevention in code-writing} 

Even with adequate context, LLMs are not guaranteed to write executable code in a single attempt. Therefore, we introduce a self-debugging mechanism to both the Retriever and the Reasoner to ensure the successful code execution \citep{selfdebug}. Specifically, we establish an inner iteration between the code-writing LLM and the code executor. At each round, we prompt the history of attempts, including the initial query $\query$, previous programs ${\code_{0}, \cdots, \code_{i-1}}$, and execution outcomes ${\code_{0}(\gG), \cdots, \code_{i-1}(\gG)}$, back to the LLM. If execution errors exist, the code-writing LLM corrects the code and repeats the process. Conversely, if the code execution is successful, then the debugging iteration terminates.

What's more, we observe hallucination in the code written by LLMs as prior work \citep{liu2024exploring}. In our case, the Reasoner might hallucinate about scene information without querying for it from the Retriever. To prevent this, we design a reprompting technique based on keyword detection. Specifically, we detect the keywords \textit{"assuming"} and \textit{"assume"} in the code written by LLMs, and prompt the code back to the Reasoner with the query to remove any assumptions in the code. We observe that the simple technique prevents scene information hallucination in most cases.
\section{Experiments
\label{sec:experiments}
}

\begin{figure*}[t]
\centering
\begin{tabular}{ccc}
\includegraphics[scale=0.29]
{toy_prediction_exact.pdf} &
\includegraphics[scale=0.29]
{toy_prediction_trace.pdf} &
\includegraphics[scale=0.29]
{toy_prediction_log.pdf} \\
(a) & (b) & (c) \\                
\includegraphics[scale=0.29]
{toy_all_predictions.pdf} &
\includegraphics[scale=0.25]
{toy_all_losses.pdf} &
\includegraphics[scale=0.25]
{toy_all_variances.pdf} \\
(d) & (e) & (f)              
\end{tabular}
\caption{First row shows posterior predictions (means with 2-standard deviations) after
  fitting the exact GP (a), and the sparse GPs with either the standard collapsed SGPR bound (b) or the proposed SGPR-new collapsed bound (c). In panels (b),(c) the seven inducing points are intiliazed to the same random locations (shown on top with crosses) while the optimized values are shown at the bottom.
  Panel (d) superimposes all predictions in order to provide a more comparative visualization.
  Finally, panel (e) shows the ELBO (or exact log marginal likelihood for the exact GP) values across optimization steps while (f) shows the corresponding values for the noise variance $\sigma^2$.}
\label{fig:toy}
\end{figure*}


\subsection{Illustration in 1-D Regression}

In the first regression experiment we consider the 1-D  Snelson dataset \cite{Snelson2006}. We took a subset of 40 examples of this dataset and we fitted the exact GP with the squared exponential kernel $k(x, x') = \sigma_f^2 \exp( - \frac{ (x - x')^2}{2 \ell^2})$. We also fitted sparse variational GPs %, denoted as SGPR, 
with either the standard collapsed bound \cite{titsias2009variational} from \Cref{eq:collapsedbound_old} (SGPR) or the new collapsed bound from \Cref{eq:newcollapsedbound} (SGPR-new).
Both sparse GP methods use seven inducing points initialized at the
same values as shown in Figure \ref{fig:toy}. All methods are initialized to the same hyperparameter values; see \Cref{app:furtherresults}.

Figure \ref{fig:toy} shows the results. %Specifically,
Note that both SGPR and SGPR-new find similar inducing point locations. But SGPR-new,  as a tighter bound (see panel (e)), is able to reduce some bias when estimating
the hyperparameters since it finds a noise variance $\sigma^2$ closer to the one by exact GP (see panel (f)).  
This results in better predictions that match better the exact GP, as shown by the comparative visualization in panel (d). From panel (d), observe that both the mean and variances of SGPR-new are closer to the exact GP than SGPR.  


\subsection{Medium Size Regression Datasets
\label{sec:mediumregress}
}

To further investigate the findings from the previous section, we consider three medium size real-world UCI regression datasets (Pol, Bike, and Elevators)
with roughly 10k training data points each, and for which we can still run the exact GP. We choose the ARD squared exponential kernel $k(\bx, \bx') = \sigma_f^2 \exp( - \sum_{i=1}^d \frac{(x_i - x_i')^2}{2 \ell_i^2})$.
We run all three previous methods (Exact GP, SGPR, SGPR-new) five times with different random train-test splits;
see \Cref{app:furtherresults} for experimental details. We also include
in the comparison a fourth method (discussed in Related Work)
which is the \citet{artemevburt2021cglb}'s bound  (SGPR-artemev) that does training using the collapsed bound from \Cref{eq:artemvecollapsedbound} in \Cref{app:artemevbound}. 
All sparse GP methods use $M=1024$ or $M=2048$ inducing points initialized by k-means.  Figure \ref{fig:mediumsize1024} shows the objective function and the noise variance $\sigma^2$ across $10k$ optimization steps using Adam with base learning rate $0.01$ and for $M=1024$.  \Cref{fig:mediumsize2048} in \Cref{app:mediumsizeRegress} shows the corresponding plots for $M=2048$.  We observe that for Pol and Bike, SGPR-new matches closer the exact GP training than SGPR and SGPR-artemev. Specifically, SGPR-new gives higher ELBO and estimates the noise variance with reduced underfitting bias.
For the Elevators dataset, $M=1024$ inducing points were enough for sparse GP methods to closely match exact GP training. This happens because in this case $\bQ_{\f \f}$ accurately approximates $\bK_{\f \f}$, i.e., the elements $k_{ii} - q_{ii}$ get close to zero. Table \ref{table:smalldatasetsTestLL} reports test log-likelihood predictions which show that 
SGPR-new outperforms SGPR and SGPR-artemev.  

\begin{table}[t]
  \caption{Average test log-likelihoods for the medium size regression datasets.
  The numbers in parentheses are standard errors.
    %The SGPR methods used $M=1024$ inducing points.
  }
\label{table:smalldatasetsTestLL}
\vskip 0.15in
%\begin{small}
\begin{center}
%  \begin{sc}
\resizebox{\linewidth}{!}{%
\begin{tabular}{lcccr}
\toprule
& Pol  & Bike & Elevators \\
\midrule
Exact GP & $1.089(0.011)$ & $3.105(0.022)$ & $-0.386(0.001)$ \\
% Exact GP & $1.089(0.011)$ & $3.105(0.022)$ & $-0.386(0.001)$  \\
\midrule
 $M=1024$ & & & \\
SGPR & $0.821(0.008)$ & $2.176(0.020)$ & $-0.387(0.001)$\\
% SGPR-trace & $0.958(0.008)$  & $2.337(0.030)$ & $-0.387(0.001)$ \\
SGPR-artemev & $0.859(0.007)$ & $2.199(0.024)$ & $-0.387(0.001)$  \\
SGPR-new & $0.920(0.006)$ & $2.326(0.026)$  & $-0.387(0.001)$  \\
%SGPR-log & $0.998(0.008)$  & $2.511(0.021)$ & $-0.387(0.001)$ \\
\midrule
$M=2048$ & & & \\
% SGPR-trace & $0.821(0.008)$ & $2.176(0.020)$ & $-0.387(0.001)$\\
SGPR & $0.958(0.008)$  & $2.337(0.030)$ & $-0.387(0.001)$ \\
% SGPR-log & $0.920(0.006)$ & $2.326(0.026)$  & $-0.387(0.001)$  \\
SGPR-artemev & $0.976(0.008)$ & $2.356(0.029)$ & $-0.387(0.001)$  \\
SGPR-new & $0.998(0.008)$  & $2.511(0.021)$ & $-0.387(0.001)$ \\
\bottomrule
\end{tabular}}
%\end{sc}
%\end{small}
\end{center}
\vskip -0.1in
\end{table}


\begin{figure*}[t]
\centering
\begin{tabular}{ccc}
\includegraphics[scale=0.25]
{smallscale_elbo_pol_1024.pdf} &
\includegraphics[scale=0.25]
{smallscale_elbo_bike_1024.pdf} &
\includegraphics[scale=0.25]
{smallscale_elbo_elevators_1024.pdf} \\
\includegraphics[scale=0.25]
{smallscale_sigma2_pol_1024.pdf} &
\includegraphics[scale=0.25]
{smallscale_sigma2_bike_1024.pdf} &
\includegraphics[scale=0.25]
{smallscale_sigma2_elevators_1024.pdf} 
\end{tabular}
\caption{The two plots in each column correspond to the same dataset: first row shows the ELBO (or log-likelihood)
 for all four methods (Exact GP, SGPR, SGPR-new and SGPR-artemev) with the number of iterations, and the plot in the second row shows the
  corresponding values for $\sigma^2$. SGPR methods use $M=1024$ inducing points initialized by k-means. For each line we plot the mean and standard error
  after repeating the experiment five times with different train-test dataset splits; see \Cref{app:furtherresults} for further experimental details.       
}
\label{fig:mediumsize1024}
\end{figure*}


\subsection{Large Scale Regression Datasets
\label{sec:largeregress}
}

\begin{table*}[t]
\caption{Test log-likelihoods for the large scale regression datasets with standard errors in parentheses. Best mean values are highlighted.} 
% Uses random 80\% / 20\% training and test splits, repeated 5 times. }
\label{table:largescaleTestLL}
\makebox[\textwidth][c]{
\resizebox{1.02\textwidth}{!}{
\setlength\tabcolsep{2pt}
\begin{tabular}{ l l cc cc cc cc}
\toprule
& & Kin40k &  Protein & \footnotesize KeggDirected & KEGGU &  3dRoad & Song &  Buzz & \footnotesize HouseElectric \\
\cmidrule(lr){3-10}
& $N$ & 25,600 & 29,267 & 31,248 & 40,708 & 278,319 & 329,820 & 373,280 & 1,311,539  \\
& $d$ & 8 & 9 & 20 & 27 & 3 & 90 & 77 & 9  \\
\midrule
%\multirow{2}{*}{SVGP}
%& $1024$  
%& 0.094(0.003) & -0.963(0.006) & 0.967(0.005) & 0.678(0.004) & -0.698(0.002) & -1.193(0.001) & -0.079(0.002) & 1.304(0.002)  \\
%& $1536$  
%& 0.129(0.003) & -0.949(0.005) & 0.944(0.006) & 0.673(0.004) & -0.674(0.003) & -1.193(0.001) & -0.079(0.002) & 1.304(0.003) \\
%\midrule
From \citet{shietal2020} \\ 
ODVGP & $1024+1024$ 
& 0.137(0.003) & -0.956(0.005) & -0.199(0.067) & 0.105(0.033) & -0.664(0.003) & -1.193(0.001) & -0.078(0.001) & 1.317(0.002) \\
& $1024+8096$  
& 0.144(0.002) & -0.946(0.005) & -0.136(0.063) & 0.109(0.033) & -0.657(0.003) & -1.193(0.001) & -0.079(0.001) & 1.319(0.004) \\
SOLVE-GP & $1024 + 1024$ & 0.187(0.002) & -0.943(0.005) &  0.973(0.003) &  0.680(0.003) & -0.659(0.002) & -1.192(0.001) &  -0.071(0.001) & 1.333(0.003) \\
%\midrule
%SVGP
% \\
%& $2048$
%& 0.137(0.003) & {\bf -0.940}(0.005) & 0.907(0.003) & 0.665(0.004) & -0.669(0.002) & {\bf -1.192}(0.001) & -0.079(0.002) & 1.304(0.003) \\
\midrule
SVGP [ours] & 1024 & $0.108(0.002)$ & $-0.969(0.006)$ & $1.042(0.009)$ & $0.699(0.005)$ & $-0.704(0.003)$ & $-1.192(0.001)$ & $-0.069(0.002)$ & $1.383(0.002)$ \\
& 2048 & $0.237(0.002)$ & $-0.944(0.006)$ & ${\bf 1.050}(0.009)$ & ${\bf 0.703}(0.005)$ & ${\bf -0.650}(0.003)$ & ${\bf -1.190}(0.001)$ & $-0.063(0.001)$ & $1.419(0.002)$ \\
SVGP-new [ours]  & 1024 & $0.152(0.003)$ & $-0.965(0.006)$ & $1.044(0.009)$ & $0.699(0.005)$ & $-0.701(0.003)$ & $-1.192(0.001)$ & $-0.065(0.002)$ & $1.387(0.003)$ \\
 & 2048 & ${\bf 0.286}(0.002)$ & ${\bf -0.938}(0.006)$ & $1.051(0.009)$ & ${\bf 0.703}(0.005)$ & $-0.651(0.004)$ & ${\bf -1.190}(0.001)$ & ${\bf -0.060}(0.001)$ & ${\bf 1.421}(0.002)$ \\
\bottomrule 
\end{tabular}
}
}
\end{table*}


We consider 8 UCI regression datasets, with training data sizes ranging from tens of thousands to millions. 
%Results of exact GP regression have been reported on these datasets with distributed training~\citep{wang2019exact}. 
We implemented the stochastic optimization versions of the two scalable sparse GP methods: (i) the one that trains using the previous uncollapsed bound from
 \citet{hensman2013gaussian} (SVGP) and (ii) our new bound from    
\Cref{eq:newuncollapsedbound} (SVGP-new). We denote these stochastic optimization versions by SVGP to distinguish them from the corresponding
SGPR methods that use the more expensive collapsed bounds. We run the SVGP methods with $M=1024$ and $2048$ inducing points, Matern3/2 kernel with common lengthscale, minibatch size $1024$, Adam with base learning rate $0.01$ and $100$ epochs. These experimental settings match the ones in \citet{wang2019exact} and \citet{shietal2020} as further described  in \Cref{app:largescaleRegress}. Table \ref{table:largescaleTestLL} reports the test log likelihood scores
for all datasets. In the comparison we also included two strong baselines from Table 2 in \citet{shietal2020}, i.e., SOLVE-GP and ODVGP \cite{salimbeni2018orthogonally}.


\begin{figure*}
\centering
\begin{tabular}{ccc}
\includegraphics[scale=0.24]
{poisson_toy_all_predictions.pdf} &
\includegraphics[scale=0.24]
{poisson_toy_all_losses.pdf} &
\includegraphics[scale=0.24]
{poisson_elbo_nybicycle_16.pdf} \\
% (a) & (b) & (c)
\end{tabular}
\caption{({\bf left}) shows the % posterior 
predictions (means with 2-standard deviations) over counts (black dots) in the artificial data example  after
  fitting the Full GP, and the two SVGPs. This plot superimposes all predictions in order to provide a comparative visualization.
  %; see \Cref{app:poisson} for individual plots. 
  ({\bf middle})  shows the ELBO  across optimization steps for the artificial data example. ({\bf right}) shows the ELBO for the NYBikes dataset and $M=16$.}
\label{fig:poisson}
\end{figure*}


From the predictive log likelihood scores in Table \ref{table:largescaleTestLL} and also the corresponding Root Mean Squared Error (RMSE)
scores reported in  \Cref{table:largescaleRMSE} in \Cref{app:largescaleRegress}, we can conclude that training with the new SVGP-new variational bound
provides a clear improvement compared to training with the previous SVGP bound. Note that this improvement requires no change in the computational
cost, and in fact there is only a minor modification needed to be done in an existing SVGP implementation in order to run SVGP-new.  

\vspace{-1mm}

\subsection{Poisson Regression
  \label{sec:poisson}
}

\vspace{-1mm}

We consider a non-Gaussian likelihood example where the output data are counts modeled  by a Poisson likelihood 
$p(\y | \f) = \prod_{i=1}^N \frac{e^{f_i}}{y_i !} e^{-e^{f_i}}$  where the log intensities values follow a GP prior. For such 
case the new variational approximation includes a single additional variational parameter denoted by $v$, which is optimized together 
with the remaining parameters; see \cref{sec:nongaussian}. We will compare training with the new ELBO 
 from \Cref{eq:nonGaussian_bound_tractable}  (we denote this method by SVGP-new) with the standard ELBO that is obtained by restricting  $v=1$  (SVGP). 
 
Firstly, we consider an artificial example of $50$ observations with 1-D inputs placed in the grid $[-10, 10]$ where counts are
generated using Poisson intensities given by $\lambda(x) = 3.5 + 3  \sin(x)$. We train the GP model with the SVGP bound and the proposed SVGP-new bound using $6$ inducing points initialized to the same values for both methods; see \Cref{app:poisson}. 
\Cref{fig:poisson}(left) shows the 
% observed counts together with the 
predictions obtained by SVGP, SVGP-new 
and non-sparse %or full 
variational % inference 
GP (Full GP). From
this figure and from the
ELBO values, 
we observe that SVGP-new
remains closer to Full GP.  

Secondly, we consider a real dataset (NYBikes) about bicycles crossings going over bridges in New York City\footnote{This dataset is freely available from
\url{https://www.kaggle.com/datasets/new-york-city/nyc-east-river-bicycle-crossings}.}.
This dataset is a daily record of the number of bicycles crossing into or out of Manhattan via one of the East River bridges over a period 9 months. The data contains $210$  points and we randomly choose $90\%$ for training and $10\%$ for test.   
We apply GP Poisson regression for the Brooklyn bridge counts where the input vector $\bx$ is taken to be two-dimensional consisted of 
 maximum and minimum daily temperatures.  We train the sparse GPs with either SVGP or SVGP-new and with $M=8,16,32$ 
 inducing points initialized by k-means.  Since the dataset is small  we also run the non-sparse  Full GP. The ELBO across iterations in \Cref{fig:poisson} (right) and the test log likelihood scores (\Cref{table:poisson_nybikes} 
 in \Cref{app:poisson})
 indicate that  SVGP-new provides a better approximation than SVGP.  
 
  
   
   










\section{Concluding remarks}
We propose an STL erosion strategy for probabilistic STL verification of discrete-time nonlinear stochastic systems under sub-Gaussian disturbances. Leveraging a tight bound on stochastic deviation, our strategy reduces the probabilistic verification problem into a deterministic verification problem with a tightened STL specification which can then be solved using existing deterministic methods. Our strategy is validated on both linear and nonlinear systems. Control synthesis for nonlinear stochastic systems with probabilistic STL specifications based on our results is an interesting and promising future direction.

\bibliographystyle{IEEEtran}        
\bibliography{references}

\end{document}
