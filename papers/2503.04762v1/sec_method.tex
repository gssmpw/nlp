\section{Stochastic STL verification}\label{sec:main}
Our strategy to solve Problem \ref{prob: verification} is to convert the probabilistic STL satisfaction problem into a deterministic one with a tighter STL formula, as described in Section~\ref{sec: stl erosion} and Figure~\ref{fig: method}. Combining this strategy and a sharp bound on the deviation of a stochastic trajectory from its deterministic counterpart, we address Problem \ref{prob: verification} in Section~\ref{sec: probabilistic bound}.

\subsection{STL Erosion}
\label{sec: stl erosion}

\begin{figure}
\centering
\includegraphics[width =0.9\linewidth]{figures/method.pdf}
\caption{An illustration of the STL erosion method. Here we consider a simple STL specification: $\varphi:=\square_{[t_1,t_2]}\pi$, where the super level set of the predicate $\pi$ is $\CC$. This STL specification requires the state to remain inside a safe set $\CC$ from $t_1$ to $t_2$. The predicate $\pi$ is eroded by $\tilde{E}$. If the deterministic trajectory can satisfy the tightened STL formula $\tilde \varphi = \square_{[t_1,t_2]}\tilde\pi$, then the stochastic trajectory would satisfy $\varphi$ with a high probability.}
\label{fig: method}
\end{figure}

The deterministic system~\eqref{eq: deterministic dynamics} can be viewed as a noise-free version of the stochastic system~\eqref{eq: stochastic dynamics}. We refer to a deterministic trajectory of \eqref{eq: deterministic dynamics} and a stochastic trajectory of \eqref{eq: stochastic dynamics} from the same initial state and subjected to the same sequence of $d_t$ as \textit{associated} trajectories. Our strategy relies on the intuition that the trajectory of \eqref{eq: stochastic dynamics} should fluctuate around and remain close to its associated trajectory of \eqref{eq: deterministic dynamics} with high probability. 


Building on the set erosion strategy~\cite{liu2024safety}, we introduce the \textit{predicate erosion strategy}. 
For a predicate $\mu(\cdot)$, let $\CC$ be its superlevel set, \ie, $\CC=\setb{x\in \real^n:\mu(x)\geq 0}$. We shrink $\CC$ based on the amount of fluctuation to get a subset $\tilde{\CC}\subset\CC$. If for the deterministic state $x_t$, $x_t \in \tilde{\CC}$, then for the stochastic state $X_t$, $X_t\in \CC$ holds with a high probability since the fluctuation of the stochastic trajectory probably not exceed the margin $\CC \backslash \tilde{\CC}$. Therefore, we can verify whether the stochastic trajectory satisfies $\varphi$ with a high probability by verifying whether the deterministic trajectory satisfies $\varphi$ with the eroded predicates. 

To quantify the fluctuations of a stochastic trajectory around its deterministic counterpart, we formalize the concepts of stochastic fluctuation and stochastic deviation.

\begin{definition}
    Define $e_t = X_t - x_t$ as the stochastic fluctuation of the stochastic trajectory around the associated deterministic trajectory at time step $t$. Define $\|e_t\| = \| X_t-x_t\|$ as the stochastic deviation.
\end{definition}

Analogous to the reachability analysis of the system state, we introduce the probabilistic reachable set of the stochastic fluctuation. 

\begin{definition}[PRS] 
    Let $e_t$ be the stochastic fluctuation at time step $t$.
    % t \in \n_{[0, T]}$. 
    A set $E_{\theta, t}$ is called a Probabilistic Reachable Set (PRS) of $e_t$ at probability level $1-\theta$, $\theta \in (0,1)$, if for any $x_0=X_0\in \XX_0$ and any $d_s \in \DD, 0\le s\le t$, it holds that
    \begin{equation}
        \bP(e_t\in E_{\theta, t})\geq 1-\theta.
    \end{equation}
\end{definition}

Note that the PRS of $e_t$ is not unique. If $E_{\theta, t}$ is a PRS, then $E'_{\theta, t}$ is also a PRS if $E_{\theta, t}\subseteq E'_{\theta, t}$ We say $E_{\theta, t}$ is tighter than $E'_{\theta, t}$ if $E'_{\theta, t} \subseteq E_{\theta, t}$. We present a tight PRS in \ref{sec: probabilistic bound}.

For ease of analysis, we introduce the following sets:
\begin{equation} \label{eq: E_theta}
    \tilde{E}_\theta = \bigcup_{t=0}^{T} E_{\theta, t},
\end{equation}
\begin{equation} \label{eq: E_theta T}
    \boldsymbol{\tilde E}_{\theta} = \underbrace{\tilde{E}_\theta \times \tilde{E}_\theta \times \dots \times \tilde{E}_\theta}_{T \text{ times}}.
\end{equation}
At any time step \( t \), the stochastic fluctuation \( e_t \) belongs to the set \( \tilde{E}_\theta \) with probability at least \( 1 - \theta \). By union bound, the probability that \( e_t \) belongs to \( \tilde{E}_\theta \) for all time steps is
\begin{equation}
   \mathbb{P}(e_t \in \tilde{E}_\theta, \forall t \in \mathbb{N}_{[0,T]}) = \mathbb{P}(\boldsymbol{e}_{[0,T]} \in \boldsymbol{\tilde E}_{\theta}) \geq 1 - T\theta
\end{equation}

Using the PRS, we can tighten the predicates to eliminate the probabilistic term in \eqref{eq: probability constraint}.

\begin{proposition}[Predicate erosion]
    \label{prop: predicate erosion}
    Assume $E_{\theta, t}$ is a PRS of $e_t$. Let $\tilde{E}_\theta$ be as defined in \eqref{eq: E_theta}. Consider a predicate $\pi = (\mu(\cdot)\geq 0)$ with superlevel set $\CC$. Let $\tilde \pi$ be a tighter version of $\pi$, \st $x_t \models \tilde \pi \Leftrightarrow x_t \in \CC \ominus \tilde{E}_\theta$, then for any time step $t$, $t \in \n_{[0, T]}$, $x_t \models \tilde \pi \implies x_t + e_t \models \pi, \forall e_t \in \tilde{E}_\theta$.
\end{proposition}
\begin{proof}
    $x_t\models \tilde \pi \implies x_t \in \CC \ominus \tilde E_{\theta} \implies \forall e_t \in \tilde E_{\theta}, x_t + e_t \in (\CC \ominus \tilde E_{\theta, t})\oplus \tilde E_{\theta} \subseteq \CC$. Therefore $ x_t + e_t \models \pi, \forall e_t\in \tilde E_{\theta}$.
\end{proof}
Proposition~\ref{prop: predicate erosion} extends \cite{vlahakis2024probabilistic} which considers predicate tightening for affine predicates. 
In Proposition~\ref{prop: predicate erosion}, we erode $\CC$ with $\tilde{E}_\theta$, as verifying an STL formula may require evaluating a predicate at any timestep along a trajectory. Compared to the Set Erosion method for the safe set specification~\cite{liu2024safety}, the STL specification involves both spatial and temporal constraints, requiring us to shrink $\CC$ for all time steps using the union of the PRSs, as opposed to the time-varying safe set in~\cite{liu2024safety}. The following theorem shows that the stochastic STL satisfaction problem can be converted into a deterministic one by eroding every predicate in an STL formula.

\begin{theorem}[STL formula erosion]
\label{thm: STL erosion}
    Consider associated trajectory $\boldsymbol{X}_{[0,T]}$ and $\boldsymbol{x}_{[0,T]}$. Let $\tilde{E}_\theta$ be as defined in \eqref{eq: E_theta}, and $\boldsymbol{\tilde E}_{\theta}$ be as defined in \eqref{eq: E_theta T}. Given an STL formula $\varphi$ constructed with predicates $\pi_1, \pi_2, \dots, \pi_m$, and corresponding superlevel sets $\CC_1, \CC_2, \dots, \CC_m$. For every predicate $\pi_i,i \in \n_{[1,m]}$, we substitute $\pi_i$ with $\tilde \pi_i$, where the superlevel set of $\tilde \pi_i$ is $\CC_i \ominus \tilde{E}_\theta$ and keep all other operators unchanged to get $\tilde{\varphi}$, a eroded version of $\varphi$. Then $\boldsymbol{x}_{[0,T]} \models \tilde{\varphi} \implies \bP(\boldsymbol{X}_{[0,T]}\models \varphi) \geq 1-T\theta$
\end{theorem}
\begin{proof}
    By definition, it suffices to show $\boldsymbol{x}_{[0,T]} \models \tilde{\varphi} \implies \boldsymbol{x}_{[0,T]} +  \boldsymbol{e}_{[0,T]}\models \varphi, \forall \boldsymbol{e}_{[0,T]} \in \boldsymbol{\tilde E}_{\theta}$. Since the STL formulas are recursively defined, we prove by induction.

    \textit{Base Case}: By Proposition~\ref{prop: predicate erosion}, $\forall t \in \n_{[0,T]}$, $\boldsymbol{x}_{[t,T]} \models \tilde \pi_i\implies \boldsymbol{x}_{[t,T]} +  \boldsymbol{e}_{[t,T]} \models \pi_i, \forall \boldsymbol{e}_{[t,T]} \in \boldsymbol{\tilde E}_{\theta [t, T]}$. 

   \textit{Induction Hypothesis}: For any STL sub-formulas $\varphi_j$ and their eroded version $\tilde \varphi_j$, assume $\forall t \in \n_{[0,T]}$, $\boldsymbol{x}_{[t,T]} \models \tilde \varphi_j\implies \boldsymbol{x}_{[t,T]} +  \boldsymbol{e}_{[t,T]} \models \varphi_j, \forall \boldsymbol{e}_{[t,T]} \in \boldsymbol{\tilde E}_{\theta [t, T]}.$ 

   \textit{Induction Step}: We discuss logical operators and temporal operators respectively:

    \begin{enumerate}
        \item \textit{Logical operators:}  Let $\varphi' = \varphi_1 \wedge \varphi_2$ and assume $\boldsymbol{x}_{[t,T]} \models \tilde{\varphi}'$, which is equivalent to $\boldsymbol{x}_{[t,T]}\models\tilde \varphi_1 \wedge \boldsymbol{x}_{[t,T]}\models \tilde\varphi_2$. This implies $\boldsymbol{x}_{[t,T]} +  \boldsymbol{e}_{[t,T]} \models \varphi_1 \wedge \boldsymbol{x}_{[t,T]} +  \boldsymbol{e}_{[t,T]} \models \varphi_2$, which is equivalent to $\boldsymbol{x}_{[t,T]} +  \boldsymbol{e}_{[t,T]} \models \varphi'$, $\forall \boldsymbol{e}_{[t,T]} \in \boldsymbol{\tilde E}_{\theta [t, T]}$. We can similarly prove the induction step for the disjunctive operator. 
        \item \textit{Temporal Operators:} Let $\varphi' = \varphi_1\UU_{[t_1,t_2]}\varphi_2$ and assume $\boldsymbol{x}_{[t,T]} \models \tilde{\varphi}'$, which is equivalent to $\exists \tau \in \n_{[t+t_1,t+t_2]}, \boldsymbol{x}_{[\tau,T]} \models\tilde{\varphi}'_2 \wedge \forall \tau' \in \n_{[t, \tau]}, \boldsymbol{x}_{[\tau',T]} \models \tilde{\varphi}'_1$. This implies that, for the same $\tau$, $\boldsymbol{x}_{[\tau,T]} + \boldsymbol{e}_{[\tau,T]} \models {\varphi}'_2 \wedge \forall \tau' \in \n_{[t, \tau]}, \boldsymbol{x}_{[\tau',T]} + \boldsymbol{e}_{[\tau',T]} \models {\varphi}'_1$, which is equivalent to $\boldsymbol{x}_{[t,T]} + \boldsymbol{e}_{[t,T]} \models \varphi', \forall \boldsymbol{e}_{[t,T]} \in \boldsymbol{\tilde E}_{\theta [t, T]}$. We can similarly prove the induction step for other temporal operators.
        
    \end{enumerate}
    
    By induction, if $\boldsymbol{x}_{[0,T]} \models \tilde{\varphi}$, then $\boldsymbol{x}_{[0,T]} +  \boldsymbol{e}_{[0,T]}\models \varphi, \forall \boldsymbol{e}_{[0,T]} \in \boldsymbol{E}_{\theta,T}$.
\end{proof}

An illustration of the proposed method is shown in Figure~\ref{fig: method}. According to Theorem~\ref{thm: STL erosion}, instead of verifying whether a stochastic trajectory satisfies an STL formula with a certain probability, one can verify whether a corresponding deterministic trajectory satisfies a tighter STL formula. Consequently, the stochastic STL verification problem (Problem~\ref{prob: verification}) reduces to a deterministic verification problem, which can be solved by any existing methods.

\begin{theorem}[STL verification of stochastic systems]\label{thm: verification}
    Consider the stochastic system \eqref{eq: stochastic dynamics} and the associated deterministic system \eqref{eq: deterministic dynamics} with initial set $\XX_0$ and bounded input set $\DD$. Given an STL specification $\varphi$ with bounded horizon $T$, $\delta \in (0,1)$, define $\tilde{E}_\theta$ as in \eqref{eq: E_theta} with $\theta = \delta/T$ and construct $\tilde \varphi$ as in Theorem~\ref{thm: STL erosion}. If \eqref{eq: deterministic dynamics} satisfies $\tilde{\varphi}$, then the stochastic system \eqref{eq: stochastic dynamics} satisfies $\varphi$ with $1-\delta$ guarantee.
\end{theorem}

\begin{proof}
    The deterministic system satisfies $\tilde{\varphi}$ implies that, for every initial state $x_0 \in \XX_0$ and every bounded disturbance $d_t \in \DD$, it holds that $\boldsymbol{x}_{[0,T]} \models \tilde \varphi$. By Theorem~\ref{thm: STL erosion}, for every initial state $x_0 \in \XX_0$ and every bounded disturbance $d_t \in \DD$, $\bP(\boldsymbol{X}_{[0,T]}\models \varphi) \geq 1-T\theta = 1-\delta$. Therefore the stochastic system satisfies $\varphi$ with $1-\delta$ guarantee.
\end{proof}

\subsection{STL verification}
\label{sec: probabilistic bound}
To effectively apply the STL erosion strategy, a tight PRS of the stochastic fluctuation is the key. Next, we utilize recent results on bounding stochastic derivation to obtain a tight PRS~\cite{liu2024probabilistic,jafarpour2024probabilistic}. 
% To achieve less conservative verification results by using Theorem \ref{thm: STL erosion}, a tight PRS of the stochastic fluctuation is desired. In this section, we leverage our recent result on bounding stochastic derivation to achieve this tight PRS~\cite{liu2024probabilistic,jafarpour2024probabilistic}. Another widely used approach for acquiring PRS is introduced as the baseline.
% \subsubsection{Probabilistic Bound on Stochastic Deviation}
% A tight bound on stochastic deviation $\|X_t-x_t\|$ is derived in~\cite{liu2024probabilistic}, which we restate in the following proposition.
\begin{proposition}\label{prop: stochastic deviation}\cite{liu2024probabilistic}
Let $\boldsymbol{X}_{[0,\infty]}$ be the trajectory of the stochastic system \eqref{eq: stochastic dynamics} and $\boldsymbol{x}_{[0,\infty]}$ be the trajectory of the deterministic system \eqref{eq: deterministic dynamics} with the same initial state $x_0 \in \XX_0$ and the same input sequence $\boldsymbol{d}_{[0, \infty]}$. Then for any $\varepsilon \in (0,1)$, $\theta \in (0,1)$, and $t\geq 0$,
\begin{equation}\label{eq: bound r}
 \bP\Big( \|X_t-x_t\|\leq r_{\theta, t} \Big) \geq 1-\theta,
\end{equation}
 where 
$r_{\theta, t} = \sqrt{\Psi_t(\varepsilon_1n+\varepsilon_2\log(1/\theta))}$, $\psi_t=\prod_{k=0}^{t}L_k^{2}$,\quad
$\Psi_t=\psi_{t-1}\sum_{k=0}^{t-1}\sigma_{k}^2\psi_k^{-1}$, 
$\varepsilon_1=\frac{2\log(1+2/\varepsilon)}{(1-\varepsilon)^2}, \varepsilon_2=\frac{2}{(1-\varepsilon)^2}$.
\end{proposition}

The bound in \eqref{eq: bound r} holds for both the Euclidean norm and the weighted norm \( \|\cdot \|_P \)~\cite[Section V-D]{jafarpour2024probabilistic}.
% Under the weighted norm \( \|\cdot \|_P \), a tighter bound can be obtained due to the smaller Lipschitz constant
The above bound scales logarithmically with $T$ and $1/\delta$, which is tight for stochastic systems \cite{liu2024probabilistic}. By Proposition~\ref{prop: stochastic deviation}, the ball $\BB(r_{\theta, t}, 0)$ (or an ellipsoid when using the weighted norm) serves as a tight PRS of $e_t$.

With the tight PRS of $e_t$, we can combine it with any existing methods for STL verification for deterministic systems, and leverage the STL erosion strategy stated in Theorem~\ref{thm: STL erosion} to verify STL satisfaction of the stochastic system.

\begin{theorem}\label{thm: overall}
        Consider the stochastic system \eqref{eq: stochastic dynamics} and the associated deterministic system \eqref{eq: deterministic dynamics} with initial set $\XX_0$, and the bounded input set $\DD$. Given an STL specification $\varphi$ with bounded horizon $T$, $\delta \in (0,1)$, let $\BB(r_{\delta, t}, 0)$ serve as the PRS of $e_t$ in Theorem~\ref{thm: verification}, where $r_{\delta, t} = \sqrt{\Psi_t(\varepsilon_1n+\varepsilon_2\log(T/\delta))}$. If the deterministic system satisfies $\tilde{\varphi}$, then the stochastic system satisfies $\varphi$ with $1-\delta$ guarantee.
\end{theorem}

\begin{proof}
    Plug the bound stated in Proposition~\ref{prop: stochastic deviation} into Theorem~\ref{thm: verification}, then Theorem~\ref{thm: overall} follows.
\end{proof}

% \subsubsection{Probabilistic Bound by Worst-Case Analysis} 
% \label{sec: worst-case analysis}
Worst-case analysis is used to verify the safety of systems under bounded disturbances~\cite{prajna2007framework}. It can also be applied to systems with unbounded noise for probabilistic verification by treating the noise as bounded noise with a high probability. This idea is used in \cite{vlahakis2024probabilistic} to handle probabilistic STL constraints for linear stochastic systems with affine predicates. The same idea can be readily extended to nonlinear stochastic systems.
% 
By~\cite[Section IV-B]{liu2024safety},
\begin{equation}\label{eq: sd by worst}
    \|X_t-x_t\|\leq \sqrt{\psi_{t-1}}\sum_{k=0}^{t-1}\sigma_{k}\sqrt{\psi_k^{-1}(\varepsilon_1n+\varepsilon_2\log\frac{T}{\delta})}
\end{equation}
holds for all $t\in\n_{[0,T]}$ with probability at least $1-\delta$. Denote the right-hand side as $r_{\delta,t}^{\text w}$. It is shown that $r_{\delta,t}^{\text w}$ is always greater than $r_{\delta,t}$ \cite{liu2024safety}. Thus, using the worst-case bound $r_{\delta,t}^{\text w}$ in STL erosion is inherently more conservative than Theorem~\ref{thm: overall}. 


     