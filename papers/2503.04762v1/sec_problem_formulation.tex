\section{Problem Formulation}
In this section, we provide an overview of signal temporal logic and then formulate the STL verification problem for nonlinear stochastic systems.

\subsection{Signal Temporal Logic}
In this work, we use STL to specify the spatio-temporal properties of systems for safety verification. 
We consider the STL formula with a standard syntax
\begin{equation}
\label{eq: standard STL syntax}
    \varphi := \mathsf{T} {}\mid{} \pi {}\mid{} \neg\varphi {}\mid{} \varphi_1\wedge\varphi_2 {}\mid{} \varphi_1 \UU_{[t_1,t_2]} \varphi_2,
\end{equation}
where $\mathsf{T}$ denotes \textit{True}, $\pi:=(\mu(x)\geq 0)$ is a predicate, $\neg$ and $\wedge$ denote negation and conjunction, and $\UU_{[t_1,t_2]}$ is the temporal \textit{until} operator. An STL formula $\varphi$ is recursively constructed using the operators introduced above. We use $\pi \in \varphi$ to denote that the predicate $\pi$ is used to construct $\varphi$. Other operators can be constructed using these operators. For example, temporal \textit{eventually} operator $\lozenge_{[t_1,t_2]}\varphi = \mathsf{T} \UU_{[t_1,t_2]}\varphi$, temporal \textit{globally} operator $\square_{[t_1,t_2]}\varphi = \neg \lozenge_{[t_1,t_2]}\neg \varphi$, and disjunction $\varphi_1\vee \varphi_2 = \neg (\neg \varphi_1 \wedge \neg \varphi_2)$.

The boolean semantics of the STL formula are defined over the system trajectory $\boldsymbol{x}_{[t,\infty]}$ ~\cite{maler2004monitoring}.
Denote by $\boldsymbol{x}_{[t,\infty]} \models \varphi$ a trajectory $\boldsymbol{x}_{[t,\infty]}$ satisfies $\varphi$. The boolean semantics can be recursively defined by: 
$\boldsymbol{x}_{[t,\infty]} \models \pi \Leftrightarrow \mu(x_t) \geq 0$, 
$\boldsymbol{x}_{[t,\infty]} \models \neg \varphi \Leftrightarrow \neg(\boldsymbol{x}_{[t,\infty]} \models \varphi)$, 
$\boldsymbol{x}_{[t,\infty]} \models \varphi_1 \wedge \varphi_2 \Leftrightarrow \boldsymbol{x}_{[t,\infty]} \models \varphi_1 \wedge \boldsymbol{x}_{[t,\infty]} \models \varphi_2$,
$\boldsymbol{x}_{[t,\infty]} \models \varphi_1 \UU_{[t_1,t_2]}\varphi_2 \Leftrightarrow \exists \tau\in \n_{[t+t_1,t+t_2]}$, 
$\st \boldsymbol{x}_{[\tau,\infty]} \models \varphi_2 \wedge \forall \tau'\in \n_{[t,\tau]}, \boldsymbol{x}_{[\tau',\infty]} \models \varphi_1$. 
% We can check whether $\varphi$ is satisfied by recursively applying boolean semantics.

The horizon of an STL formula $\varphi$ is denoted as $h^\varphi$, which is the length of the trajectory that is required to determine the satisfaction of $\varphi$~\cite{belta2019formal}. $h^\varphi$ can be calculated recursively by: $h^\pi=0$, $h^{\neg \varphi} = h^\varphi$, $h^{\varphi_1 \wedge \varphi_2} = \max(h^\varphi_1, h^\varphi_2)$, $h^{\varphi_1 \UU_{[t_1,t_2]}\varphi_2} = t_2 + \max(h^\varphi_1, h^\varphi_2)$.


To simplify the analysis, we assume that all STL formulas are converted into a negation-free form, meaning the formulas do not contain any negations. This conversion is always possible by first transforming the formula into Negation Normal Form~\cite{fainekos2009robustness} and then introducing new predicates with reversed inequalities as needed~\cite[Proposition 2]{belta2019formal, sadraddini2015robust}.

\subsection{Stochastic Systems}
{\em Dynamics:}
Consider the discrete-time stochastic system 
\begin{equation}
\label{eq: stochastic dynamics}
     X_{t+1}=f(X_t,d_t,t)+w_t
\end{equation}
where $X_t\in \real^n$, $d_t\in \DD \subset \real^m$, $w_t\in \real^n$ are the state, input, and stochastic disturbance at time $t \in \n$. The input $d_t$ represents bounded disturbances, whose statistical property is unknown. We assume $f: \real^n\times\real^p\times \n \ra\real^n$ is globally Lipschitz (Assumption \ref{ass: Lipschitz f}). 
\begin{assumption}[Lipschitz]
\label{ass: Lipschitz f}
For $\forall t \in \n$, there exists a $L_t$, such that for all $x, y\in \real^n$, and $d\in\DD$, 
\begin{equation*}
    \|f(x,d,t) - f(y,d,t)\| \leq L_t\|x-y\|.
\end{equation*}
\end{assumption}

{\em Stochastic Disturbance:}
We assume the stochastic disturbance $w_t$ is sub-Gaussian, which covers a wide range of distributions in the real world including Gaussian distribution and uniform distribution with bounded support.

\begin{definition}
A random vector $ X \in \real^n$ is said to be sub-Gaussian if $\EE(X)=0$ and there exists a positive constant $ \sigma > 0 $ such that for any $\ell$ on the unit sphere, $\EE_X \left[ e^{\lambda \langle\ell, X\rangle} \right] \leq e^{\frac{\sigma^2 \lambda^2}{2}}$ holds for all $ \lambda \in \real $. Here, $ \sigma^2 $ is called variance proxy. We use $X\sim subG(\sigma^2)$ to denote $X$ is sub-Gaussian.
\end{definition}
% \hongzhe{This assumption seems too short to be an Assumption, maybe can move to the main paper with one sentence.}
\begin{assumption}\label{ass:subG}
    For system~\eqref{eq: stochastic dynamics}, 
    $w_t \sim subG(\sigma_t^2)$, where $\sigma_t>0$, $\forall t>0$.
\end{assumption}

\subsection{Problem Statement}
We consider the safety verification problem of the stochastic system \eqref{eq: stochastic dynamics} under the STL specification \eqref{eq: standard STL syntax}.
We first recall the STL satisfaction of a deterministic system under an STL specification.
\begin{definition}
Consider the deterministic version of the system dynamics
    \begin{equation}
    \label{eq: deterministic dynamics}
    x_{t+1}=f(x_t,d_t,t)
    \end{equation}
with a set of initial states $\XX_0$, and a bounded set of input $\DD$. Given an STL specification $\varphi$ with a bounded horizon $T$, system \eqref{eq: deterministic dynamics} is said to satisfy $\varphi$ if,
    \begin{equation}\label{eq: probability constraint}
        \forall x_0 \in \XX_0, \forall t\in\n_{[0,T]}, d_t \in \DD: \boldsymbol{x}_{[0,T]} \models \varphi.
    \end{equation}
\end{definition}

This definition can be conservative for the stochastic system \eqref{eq: stochastic dynamics} whose trajectories are often unbounded, inevitably leading to a violation of the STL specification. Therefore, we turn to a probabilistic version of STL satisfaction~\cite{sadigh2016safe, farahani2018shrinking}.
\begin{definition}
Consider the stochastic system \eqref{eq: stochastic dynamics} with a set of initial states $\XX_0$, and a bounded set of input $\DD$. Given an STL specification $\varphi$ with a bounded horizon $T$, and a probability tolerance $\delta \in [0,1]$, the system is said to satisfy $\varphi$ with $1-\delta$ guarantee, if
    \begin{equation}\label{eq: probability constraint}
        \forall x_0 \in \XX_0, \forall t\in\n_{[0,T]}, d_t \in \DD: \bP \big(\boldsymbol{X}_{[0,T]} \models \varphi \big) \geq 1-\delta.
    \end{equation}

\end{definition}


\begin{remark}
$\XX_0$ and $\DD$ may each contain only a single point. For instance, if $\XX_0 = \{x_0\}$ and $\DD = \{0\}$, this corresponds to considering a single initial state $x_0$ and no deterministic disturbance, which is a common setting in optimal control problems~\cite{vlahakis2024probabilistic}. 
\end{remark}

We are interested in the problem of verifying a stochastic system under an STL specification.
\begin{problem}[Verification]
\label{prob: verification}
Consider a stochastic system \eqref{eq: stochastic dynamics} under Assumptions \ref{ass: Lipschitz f}-\ref{ass:subG}, and an STL specification $\varphi$. Verify whether the system satisfies $\varphi$ with $1-\delta$ guarantee. 
\end{problem}

