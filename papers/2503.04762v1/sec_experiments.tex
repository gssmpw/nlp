\section{Numerical examples}
\label{sec: experiments}
We illustrate the proposed method through two numerical examples, one on linear dynamics and one on nonlinear dynamics. 

\subsection{Double Integrator}
\label{sec: double integrator}

Consider the discrete-time double integrator system: 
\begin{equation}\label{eq: double integrator}
    X_{t+1} = 
    \underbrace{
    \begin{bmatrix}
    1 & 0 & \Delta t & 0 \\
    0 & 1 & 0 & \Delta t \\
    0 & 0 & 1 & 0 \\
    0 & 0 & 0 & 1
    \end{bmatrix}}_{\textbf{A}}
    X_t +
    \underbrace{
    \begin{bmatrix}
    0 & 0 \\
    0 & 0 \\
    \frac{\Delta t}{m} & 0 \\
    0 & \frac{\Delta t}{m}
    \end{bmatrix}}_{\textbf{B}}
    (u_t + d_t) + w_t,
\end{equation}
where $X_t \in \mathbb{R}^4$ represents the state vector $[p_x, p_y, v_x, v_y]\tran$, $\Delta t$ is the discretization step size, $u_t$ is the control input, $d_t\in\DD$ is the bounded disturbance, $w_t \sim \mathcal{N}(0, \Sigma)$ is the noise term. We set $\Delta t = 0.01$, $\Sigma = \Delta t\cdot 0.04 \cdot I_4$, and mass $m=0.1$. The set of initial states is $\XX_0 = [0.3, 0.5]\times[0.3,0.5]\times\{0\}\times\{0\}$.

\begin{figure}
    \centering
    \includegraphics[width =0.49\linewidth]{figures/double_integrator_determinisitc_trajectorie_75.png}
    \includegraphics[width =0.49\linewidth]{figures/double_integrator_stochastic_trajectorie_75.png}\\
    \vspace{0.2cm}
    \includegraphics[width =0.49\linewidth]{figures/double_integrator_determinisitc_trajectorie_100.png}
    \includegraphics[width =0.49\linewidth]{figures/double_integrator_stochastic_trajectorie_100.png}
    \caption{Stochastic STL verification of the double integrator system~\eqref{eq: double integrator} with $1-10^{-4}$ guarantee. \textbf{Left:} Stochastic STL verification using our STL erosion strategy. The gray circle and the green circle represent the obstacle and the goal area respectively. The corresponding eroded predicates are represented as yellow areas. Each curve is an independent trajectory of the deterministic system. \textbf{Right:} Each curve is an independent trajectory of the stochastic system.}
    \label{fig: double integrator}
\end{figure}

The task is to reach a goal area and stay in it for some time while avoiding obstacles, as shown in Figure~\ref{fig: double integrator}. The task horizon is $T=100$. To specify this task using an STL formula, we define two predicates, $\pi_{\text{obs}}:= \big(\mu_{\text{obs}}(x) = (p_x-2.2)^2+(p_y-2.2)^2 - 1.2^2 \geq 0\big)$, 
and $\pi_{\text{goal}}:= \big(\mu_{\text{goal}}(x) = (p_x-4.9)^2+(p_y-3.2)^2 - 0.55^2 \leq 0\big)$. 
The STL specification is $\varphi = (\square_{[0,T]}\pi_{\text{obs}})\wedge(\lozenge_{[0,T-10]}\square_{[0,10]}\pi_{\text{goal}})$, which requires the position to be inside the goal area continuously for 10 steps, and to be always collision-free with the obstacles.


We train a neural network to generate a reference trajectory for the deterministic system to complete the task following \cite{meng2023signal} and then use a static feedback gain $K = \begin{bmatrix}
    -2&0&-1&0\\
    0&-2&0&-1
\end{bmatrix}$ to track the reference trajectory. To get a small Lipschitz constant, we use weighted norm in~\eqref{ass: Lipschitz f}. We construct a semi-definite program to search for the best weight $P$ \cite{fan2017simulation}. The bound on stochastic fluctuation is given by the weighted norm $\|\cdot\|_P$, representing an ellipsoid in 4-dimensional space. Since all the predicates are specified in the $p_x - p_y$ plane, we project the ellipsoid to this plane to do predicate erosion.

Our goal is to verify whether the stochastic system with initial set $\XX_0$ and bounded disturbance $\DD = [-0.1, 0.1]\times [-0.1, 0.1]$ would satisfy $\varphi$ with probability $1-10^{-4}$. We calculate $\tilde{E}$ using our bound $r = r_{\delta, t} = 0.644$ (Theorem~\ref{thm: overall}). The result is visualized in Figure~\ref{fig: double integrator}. The deterministic system is verified to satisfy $\tilde \varphi$ by the algorithms implemented in CORA~\cite{roehm2016stl, lercher2024using, Althoff2015ARCH}. By Theorem~\ref{thm: overall}, the stochastic system satisfies $\varphi$ with probability at least $1-10^{-4}$. We simulate $10^5$ trajectories for both the deterministic system and the stochastic system to validate our method. All the sampled deterministic trajectories do not intersect with the obstacle enlarged by $\tilde{E}$, and stay inside the shrunk goal area for over 10 steps. Meanwhile, all the sampled stochastic trajectories satisfy the original STL formula $\varphi$, validating our strategy.

\subsection{Nonlinear Unicycle}
Consider the following nonlinear kinematic unicycle
\begin{equation}
    \begin{split}
        X_{t+1} &= X_t + \Delta t
        \begin{bmatrix}
            v_t\cos(\theta_t) \\
            v_t\sin(\theta_t) \\
            \omega_t + d_t
        \end{bmatrix}
        + w_t, % \\
        % &= f(X_t, d_t) + w_t.
    \end{split}
\end{equation}
where $X_t \in \mathbb{R}^3$ represents the state vector $[p_x, p_y,\theta]\tran$, $d_t\in \DD$ is the bounded disturbance, $w_t$ is the stochastic noise, and $v_t$ and $\omega_t$ are control inputs. $\Delta t$ is the discretization step size set to $0.05$. 

\begin{figure}
    \centering
    \includegraphics[width = 0.49\linewidth]{figures/unicycle_deterministic.png}
    \includegraphics[width = 0.49\linewidth]{figures/unicycle_stochastic.png}
    \caption{Stochastic STL verification of the unicycle system with $1-10^{-4}$ guarantee. \textbf{Left:} Stochastic STL verification using our STL erosion strategy. The blue circle represents the first goal area. The green circle represents the second goal area. The red hexagon is the obstacle. The corresponding eroded predicates are represented as yellow areas. Each curve is an independent trajectory of the deterministic system. \textbf{Right:} Each curve is an independent trajectory of the stochastic system.}
    \label{fig: unicycle}
\end{figure}

The task is to pass the first goal area (Goal 1) and finally enter the second goal area (Goal 2) while avoiding an obstacle in the middle, as shown in Figure~\ref{fig: unicycle}. Define three predicates, $\pi_{\text{goal}_1} = \big(\mu_{\text{goal}_1}(x) = (p_x+2.3)^2+(p_y-2.5)^2 - 0.7^2 \leq 0\big)$, $\pi_{\text{goal}_2} = \big(\mu_{\text{goal}_2}(x) = (p_x-2.45)^2+(p_y-2.65)^2 - 0.75^2 \leq 0\big)$, and $\pi_{\text{obs}}$ is the complement of the inscribed hexagon of the circle $(p_x-0)^2+(p_y-2.4)^2 - 1.2^2 \geq 0$. We can specify the task using the following STL formula, $\varphi = (\square_{[0,T]}\pi_{\text{obs}})\wedge(\lozenge_{[0,T]}\pi_{\text{goal}_2})\wedge(\neg \pi_{\text{goal}_2}\UU_{[0,T]} \pi_{\text{goal}_1})$.

Similar to Section~\ref{sec: double integrator}, we generate a reference trajectory $t\mapsto(p^*_x, p^*_y, \theta^*, v^*,\omega^*)$ for this task and use the a feedback tracking controller to track the reference trajectory:
$
    v_t = v^* + K_x\big(\cos\theta (p^*_x - p_x) + \sin\theta (p^*_y - p_y)\big),
    \omega_t = \omega^* + K_y\big(-\sin\theta (p^*_x - p_x) + \cos\theta (p^*_y - p_y) \big)
                + K_\theta(\theta^*-\theta).
$
The Lipschitz constant is estimated by sampling.

Our goal is to verify whether the stochastic system with initial set $\XX_0 = [-0.1, 0.1]\times[0.1, 0.3]\times[\pi/2 - 0.1, \pi/2+0.1]$, bounded disturbance $\DD = [-0.02, 0.02]$ and stochastic disturbance $w_t \sim \mathcal{N}(0, \Sigma)$, $\Sigma =\Delta t \cdot 0.001 \cdot I_3$ would satisfy $\varphi$ with probability $1-10^{-4}$. 

We first use our bound (Theorem~\ref{thm: overall}) to erode the STL formula. The result is visualized in Figure~\ref{fig: unicycle}. The deterministic system is verified to satisfy the tighter STL formula using CORA~\cite{Althoff2015ARCH}. We sample $10^5$ trajectories for both the deterministic system and the stochastic system, all of which satisfy the tighter STL formula and the original STL formula, respectively. 

Finally, we compare our method with the worst-case analysis described at the end of Section~\ref{sec:main}. The latter is significantly more conservative. In this example, $r_{\delta, t}^{\text w} \approx 6.91$, while our method yields $r_{\delta, t}=0.63$. As a result, the obstacle covers all goal areas and the goal areas become empty sets. The verification algorithm returns false due to conservativeness.


