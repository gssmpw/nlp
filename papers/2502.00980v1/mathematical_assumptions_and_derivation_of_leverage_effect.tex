
\section{Mathematical assumptions and derivation of \eqref{eqn: vt = vt hat + et}} \label{wide width and deep depth}

\subsection{Model setup}
% 이 얘기는 GARCH처럼 R^e_{t-1} 이 V_t 에 영향을 줄 수 있는 모델에서 할 수 있는 얘기임. stoch vol model 에서는 불가능
We consider a stochastic volatility model framework under the risk-neutral measure $\mathbb{Q}$ for the underlying asset price $\{S_t\}_{t\geq 0}$ and the VIX $\{V_t\}_{t\geq 0}$. Specifically, we assume 
\begin{equation} \label{eqn: stoch vol model}
\begin{gathered}
\frac{dS_t}{S_t} = r dt + V_t dW^s_t, \\
dV_t = k \left( \theta - V_t \right) dt + \eta_t dW^v_t,
\end{gathered}
\end{equation}
where $r$ is the risk-free rate, and $W^s$ and $W^v$ are correlated Brownian motions under $\mathbb{Q}$. 
The stochastic process $\eta_t$ is a square-integrable process adapted to the filtration $\mathcal{F}^{s, v}_t := \sigma (\{W^s_l, W^v_l : 0 \leq l \leq t \})$, and is chosen to ensure that $V_t$ remains nonnegative (e.g., $\eta_t = \sqrt{V_t}.$). 
Since the VIX is derived from option prices, we focus on these risk-neutral dynamics throughout.
% Since $\eta_t$ is adapted and satisfies the usual square-integrability condition, we have $\mathbb{E} \left[ \int_t^T \eta_l dW^v_l \right] = 0$.


To discretely approximate the continuous dynamics \eqref{eqn: stoch vol model}, we consider the following Euler–Maruyama scheme with a uniform time step $\Delta t$:
\begin{equation} \label{eqn: discrete stoch vol model}
\begin{gathered}
\frac{S_{t} - S_{t-1}}{S_{t-1}} = r \Delta t + V_{t-1} \Delta_t W^s, \\
V_{t} - V_{t-1} = \kappa \left( \theta - V_{t-1} \right) + \eta_{t-1}  \Delta_t W^v,
\end{gathered}
\end{equation}
where $\kappa := k \Delta t$, $\Delta_t W^s=W_t^s - W_{t-1}^s$ and $\Delta_t W^v = W_t^v - W_{t-1}^v$.
... % 아예 이 아래 문장을 빼버릴까.. 그냥 이렇게 놔두고 
% 걸리는거 1. R_{t-1}^e가 아니고 R_t^e 인것 
% 걸리는거 2. R^e 식에 보면 V_{t-1}이 들어가서 effect가 섞여버리지않나햇는데 아닌듯. 이건 ㄱㅊ은듯
Under this discrete approximation, the (simple) excess return of the underlying asset, denoted by $R_{t}^e$, can be expressed with the multiplication of the previous volatility and the uncertainty in the underlying asset price dynamics as
\begin{equation*}
    R_{t}^e = \frac{S_t - S_{t-1}}{S_{t-1}} - r\Delta t = V_{t-1} \Delta_t W^s.
\end{equation*}
In many stochastic volatility models, the leverage effect is often incorporated into the model by a negative correlation between $W^s$ and $W^v$. 
\begin{equation*}
    \mathbb{E} \left[ R_t^e e_t \right]
\end{equation*}
$\Delta_t W^v := \rho \Delta_t W^s + \sqrt{1-\rho^2} \Delta_t (W^v)^{\perp}$



... %가장 크게 문제되는 것: 지금 얘기면 근데 eta에서 그 uncertainty of S 가 온다는 얘기고... vol of vol 에서 leverage effect가 온다는건데, 이걸 leverage effect라고 잘 안하지 않나? W^s 랑 W^v가 negative corr 인거를 그렇게 부를텐데 왜이게 이렇ㄱ ㅔ됐지?


... % 우리는 t시점의 VIX를 예측할 때 t시점의 excess return을 알 수 없으니
% rama cont에 따르면, leverage effect는 R_{t-k}^e가 Vt와 negative correlation을 갖고, 이것이 k가 커질수록 점차 줄어든다는 것이다. 우리는 Vt를 예측함에 있어서 R_t^e의 값을 참조하는 것은 미래참조에 해당하니, 가용가능한 데이터 중 가장 영향을 크게 줄 수 있는 R_{t-1}^e로부터 영향을 확인하고자 한다.

....

In this section, we suppose that the dynamics of the underlying asset price $S_t$ and the VIX is given as a stochastic volatility model given as
\begin{equation} \label{eqn: stoch vol model}
\begin{gathered}
\frac{dS_t}{S_t} = r dt + V_t dW^s_t, \\
dV_t = k \left( \theta - V_t \right) dt + \eta_t dW^v_t,
\end{gathered}
\end{equation}
where $r$ is the risk-free rate, $W^s$ and $W^v$ are correlated Brownian motions under $\mathbb{Q}$\footnote{We are concerned with the VIX, which is extracted from option price data. Thus, we focus on the dynamics under the risk-neutral measure.}, and $\eta_t$ is a square-integrable process adapted to the filtration $\mathcal{F}^{s, v}_t := \sigma (\{W^s_l, W^v_l \})_{\{l\leq t \}}$. Then, the expectation $\mathbb{E} [\int_t^T \eta_l dW_l^v=0]$.

% In \eqref{eqn: stoch vol model}, the stochastic process $\eta_t$ can even be associated with $V_t$ and $S_t$. 

The discretely approximated dynamics can be written as 
\begin{equation}
\begin{gathered}
\frac{S_{t} - S_{t-1}}{S_{t-1}} = r \Delta t + V_{t-1} \Delta_t W^s, \\
V_{t} - V_{t-1} = \kappa \left( \theta - V_{t-1} \right) + \eta_{t-1}  \Delta_t W^v,
\end{gathered}
\end{equation}
where $\kappa := k \Delta t$, and $\Delta_t W := \left( W_t - W_{t-1} \right)$. In this modeling, the (simple) excess return of the underlying asset can be expresses with the Brownian motion of the underlying asset as $\frac{S_{t} - S_{t-1}}{S_{t-1}} - r \Delta t = V_{t-1} \Delta_t W^s$.

The leverage effect under such stochastic volatility model is often represented by the negative correlation between the Brownian motions of the underlying asset and the volatility, $W^s$ and $W^v$.

Turning back to KAN, suppose that KAN perfectly matches the mean-reverting drift terms and the predicted value of KAN $\hat{V}_t$ follows \eqref{eqn: delta Vt}. Then the discrepancy between the VIX $V_t$ and the predicted value $\hat{V}_t$ is given by
\begin{equation}
\begin{aligned}
V_t - \hat{V}_t &= \left(V_t - V_{t-1} \right) - \left(\hat{V}_t - V_{t-1} \right) \\
&= \left( \kappa \left(\theta - V_{t-1} \right) + \eta_{t-1} \Delta_t W^v \right) - \left( \kappa \left(\theta - V_{t-1} \right) + \epsilon_t \right) \\
&= \eta_{t-1} \Delta_t W^v - \epsilon_t := e_t
\end{aligned}
\end{equation}
... % 여기서 epsilon_t 는 V_t의 과거 정보로부터 얻을 수 없는 미래의 uncertainty임 (만약 KAN이 perfectly forecasting based on past information 이라면) 앞에있는 \Delta_t W^v 도 알수 없는 미래임. 그래서 e_t는 그 관점에서 봤을 때 알 수 없는 값이긴 함.
% 아니지 epsilon_t가  V_t의 과거 정보로부터 얻을 수 없는 미래의 uncertainty가 아니라, e_t가 그 역할을 하는거 아닌가?

% 관점을 다시 생각해야함. V_t - \hat{V}_t = e_t 라고 하면, \hat{V}_t가 상식적으로 과거의 V에 관한 정보로 얻은 최적의 estimator (projection to such space) 라고 보는게 맞고, 그러면 이 e_t가 V의 과거 정보들과는 orthogonal 하고 뭔지몰루는 값임/

Note that our KAN predicts $\hat{V}_t$ based on the information of $V_l$ up to time $t-1$.
In other words, $\hat{V}_t$ and $\epsilon_t$ are adapted to $\mathcal{F}^v_t := \sigma (W^v_l )_{\{l\leq t \}}$, while $\eta_t$ is generally not adapted to $\mathcal{F}^v_t$.
That is, $e_t$ has some room to be able to be captured from the uncertainty from the underlying excess return.
Thus, we assume that $V_t = \hat{V}_t + e_t$ and capture some structure in $e_t$ with the underlying excess return. (that structure is a projection of $e_t$ to the space spanned by the excess return of the underlying asset.)
% 그러면 W^v는 W^s와 correlated되어있으니 \sigma (Vt) 만으로는 다 표현할 수 없음. 



This result coincides with the result from the assumption that $\alpha=0$ and $\beta=1$ in \eqref{eqn: MZ equation}, which is not rejected hypothesis. Then $V_t - \hat{V}_t = u_t$. Under the additional assumption that $u_t$ is a stochastic process adapted to $\mathcal{F}^{s, v}_t$, then $u_t$ is equivalent to $e_t$.

To sum up, this experiment is designed to express the discrepancy $e_t$ with the excess return of the underlying asset as much as possible. If it exhibits negative effects, then we can interpret this that KAN reflects the leverage effect.

