% This must be in the first 5 lines to tell arXiv to use pdfLaTeX, which is strongly recommended.
\pdfoutput=1
% In particular, the hyperref package requires pdfLaTeX in order to break URLs across lines.

\documentclass[11pt]{article}

% Remove the "review" option to generate the final version.
\usepackage{ACL2023}

% Standard package includes
\usepackage{times}
\usepackage{latexsym}


\usepackage{amsmath,amssymb}
\usepackage{subcaption}
\usepackage{graphicx}
\usepackage{float}
\usepackage{stfloats}
\usepackage{fdsymbol}
\usepackage{pifont}
\usepackage{xcolor}
\usepackage{cleveref}
\usepackage{adjustbox}
\usepackage{diagbox}
\usepackage{booktabs}
\usepackage{array}
\usepackage{booktabs}
\usepackage{multirow}
\usepackage{mdframed}
\usepackage{enumitem}
\usepackage{tabularx}
\usepackage{algorithm}
\usepackage{algorithmicx}  
\usepackage{algpseudocode}  
\usepackage{diagbox}
\crefname{section}{§}{§§}
\Crefname{section}{§}{§§}
\usepackage{pifont}
\usepackage{tikz}
\usepackage{xspace}
\usepackage{expl3}
\def\TODO#1{\todo[color=TodoColor,size=small,inline]{TODO: #1}}

\usepackage{todonotes}

%
\setlength\unitlength{1mm}
\newcommand{\twodots}{\mathinner {\ldotp \ldotp}}
% bb font symbols
\newcommand{\Rho}{\mathrm{P}}
\newcommand{\Tau}{\mathrm{T}}

\newfont{\bbb}{msbm10 scaled 700}
\newcommand{\CCC}{\mbox{\bbb C}}

\newfont{\bb}{msbm10 scaled 1100}
\newcommand{\CC}{\mbox{\bb C}}
\newcommand{\PP}{\mbox{\bb P}}
\newcommand{\RR}{\mbox{\bb R}}
\newcommand{\QQ}{\mbox{\bb Q}}
\newcommand{\ZZ}{\mbox{\bb Z}}
\newcommand{\FF}{\mbox{\bb F}}
\newcommand{\GG}{\mbox{\bb G}}
\newcommand{\EE}{\mbox{\bb E}}
\newcommand{\NN}{\mbox{\bb N}}
\newcommand{\KK}{\mbox{\bb K}}
\newcommand{\HH}{\mbox{\bb H}}
\newcommand{\SSS}{\mbox{\bb S}}
\newcommand{\UU}{\mbox{\bb U}}
\newcommand{\VV}{\mbox{\bb V}}


\newcommand{\yy}{\mathbbm{y}}
\newcommand{\xx}{\mathbbm{x}}
\newcommand{\zz}{\mathbbm{z}}
\newcommand{\sss}{\mathbbm{s}}
\newcommand{\rr}{\mathbbm{r}}
\newcommand{\pp}{\mathbbm{p}}
\newcommand{\qq}{\mathbbm{q}}
\newcommand{\ww}{\mathbbm{w}}
\newcommand{\hh}{\mathbbm{h}}
\newcommand{\vvv}{\mathbbm{v}}

% Vectors

\newcommand{\av}{{\bf a}}
\newcommand{\bv}{{\bf b}}
\newcommand{\cv}{{\bf c}}
\newcommand{\dv}{{\bf d}}
\newcommand{\ev}{{\bf e}}
\newcommand{\fv}{{\bf f}}
\newcommand{\gv}{{\bf g}}
\newcommand{\hv}{{\bf h}}
\newcommand{\iv}{{\bf i}}
\newcommand{\jv}{{\bf j}}
\newcommand{\kv}{{\bf k}}
\newcommand{\lv}{{\bf l}}
\newcommand{\mv}{{\bf m}}
\newcommand{\nv}{{\bf n}}
\newcommand{\ov}{{\bf o}}
\newcommand{\pv}{{\bf p}}
\newcommand{\qv}{{\bf q}}
\newcommand{\rv}{{\bf r}}
\newcommand{\sv}{{\bf s}}
\newcommand{\tv}{{\bf t}}
\newcommand{\uv}{{\bf u}}
\newcommand{\wv}{{\bf w}}
\newcommand{\vv}{{\bf v}}
\newcommand{\xv}{{\bf x}}
\newcommand{\yv}{{\bf y}}
\newcommand{\zv}{{\bf z}}
\newcommand{\zerov}{{\bf 0}}
\newcommand{\onev}{{\bf 1}}

% Matrices

\newcommand{\Am}{{\bf A}}
\newcommand{\Bm}{{\bf B}}
\newcommand{\Cm}{{\bf C}}
\newcommand{\Dm}{{\bf D}}
\newcommand{\Em}{{\bf E}}
\newcommand{\Fm}{{\bf F}}
\newcommand{\Gm}{{\bf G}}
\newcommand{\Hm}{{\bf H}}
\newcommand{\Id}{{\bf I}}
\newcommand{\Jm}{{\bf J}}
\newcommand{\Km}{{\bf K}}
\newcommand{\Lm}{{\bf L}}
\newcommand{\Mm}{{\bf M}}
\newcommand{\Nm}{{\bf N}}
\newcommand{\Om}{{\bf O}}
\newcommand{\Pm}{{\bf P}}
\newcommand{\Qm}{{\bf Q}}
\newcommand{\Rm}{{\bf R}}
\newcommand{\Sm}{{\bf S}}
\newcommand{\Tm}{{\bf T}}
\newcommand{\Um}{{\bf U}}
\newcommand{\Wm}{{\bf W}}
\newcommand{\Vm}{{\bf V}}
\newcommand{\Xm}{{\bf X}}
\newcommand{\Ym}{{\bf Y}}
\newcommand{\Zm}{{\bf Z}}

% Calligraphic

\newcommand{\Ac}{{\cal A}}
\newcommand{\Bc}{{\cal B}}
\newcommand{\Cc}{{\cal C}}
\newcommand{\Dc}{{\cal D}}
\newcommand{\Ec}{{\cal E}}
\newcommand{\Fc}{{\cal F}}
\newcommand{\Gc}{{\cal G}}
\newcommand{\Hc}{{\cal H}}
\newcommand{\Ic}{{\cal I}}
\newcommand{\Jc}{{\cal J}}
\newcommand{\Kc}{{\cal K}}
\newcommand{\Lc}{{\cal L}}
\newcommand{\Mc}{{\cal M}}
\newcommand{\Nc}{{\cal N}}
\newcommand{\nc}{{\cal n}}
\newcommand{\Oc}{{\cal O}}
\newcommand{\Pc}{{\cal P}}
\newcommand{\Qc}{{\cal Q}}
\newcommand{\Rc}{{\cal R}}
\newcommand{\Sc}{{\cal S}}
\newcommand{\Tc}{{\cal T}}
\newcommand{\Uc}{{\cal U}}
\newcommand{\Wc}{{\cal W}}
\newcommand{\Vc}{{\cal V}}
\newcommand{\Xc}{{\cal X}}
\newcommand{\Yc}{{\cal Y}}
\newcommand{\Zc}{{\cal Z}}

% Bold greek letters

\newcommand{\alphav}{\hbox{\boldmath$\alpha$}}
\newcommand{\betav}{\hbox{\boldmath$\beta$}}
\newcommand{\gammav}{\hbox{\boldmath$\gamma$}}
\newcommand{\deltav}{\hbox{\boldmath$\delta$}}
\newcommand{\etav}{\hbox{\boldmath$\eta$}}
\newcommand{\lambdav}{\hbox{\boldmath$\lambda$}}
\newcommand{\epsilonv}{\hbox{\boldmath$\epsilon$}}
\newcommand{\nuv}{\hbox{\boldmath$\nu$}}
\newcommand{\muv}{\hbox{\boldmath$\mu$}}
\newcommand{\zetav}{\hbox{\boldmath$\zeta$}}
\newcommand{\phiv}{\hbox{\boldmath$\phi$}}
\newcommand{\psiv}{\hbox{\boldmath$\psi$}}
\newcommand{\thetav}{\hbox{\boldmath$\theta$}}
\newcommand{\tauv}{\hbox{\boldmath$\tau$}}
\newcommand{\omegav}{\hbox{\boldmath$\omega$}}
\newcommand{\xiv}{\hbox{\boldmath$\xi$}}
\newcommand{\sigmav}{\hbox{\boldmath$\sigma$}}
\newcommand{\piv}{\hbox{\boldmath$\pi$}}
\newcommand{\rhov}{\hbox{\boldmath$\rho$}}
\newcommand{\upsilonv}{\hbox{\boldmath$\upsilon$}}

\newcommand{\Gammam}{\hbox{\boldmath$\Gamma$}}
\newcommand{\Lambdam}{\hbox{\boldmath$\Lambda$}}
\newcommand{\Deltam}{\hbox{\boldmath$\Delta$}}
\newcommand{\Sigmam}{\hbox{\boldmath$\Sigma$}}
\newcommand{\Phim}{\hbox{\boldmath$\Phi$}}
\newcommand{\Pim}{\hbox{\boldmath$\Pi$}}
\newcommand{\Psim}{\hbox{\boldmath$\Psi$}}
\newcommand{\Thetam}{\hbox{\boldmath$\Theta$}}
\newcommand{\Omegam}{\hbox{\boldmath$\Omega$}}
\newcommand{\Xim}{\hbox{\boldmath$\Xi$}}


% Sans Serif small case

\newcommand{\Gsf}{{\sf G}}

\newcommand{\asf}{{\sf a}}
\newcommand{\bsf}{{\sf b}}
\newcommand{\csf}{{\sf c}}
\newcommand{\dsf}{{\sf d}}
\newcommand{\esf}{{\sf e}}
\newcommand{\fsf}{{\sf f}}
\newcommand{\gsf}{{\sf g}}
\newcommand{\hsf}{{\sf h}}
\newcommand{\isf}{{\sf i}}
\newcommand{\jsf}{{\sf j}}
\newcommand{\ksf}{{\sf k}}
\newcommand{\lsf}{{\sf l}}
\newcommand{\msf}{{\sf m}}
\newcommand{\nsf}{{\sf n}}
\newcommand{\osf}{{\sf o}}
\newcommand{\psf}{{\sf p}}
\newcommand{\qsf}{{\sf q}}
\newcommand{\rsf}{{\sf r}}
\newcommand{\ssf}{{\sf s}}
\newcommand{\tsf}{{\sf t}}
\newcommand{\usf}{{\sf u}}
\newcommand{\wsf}{{\sf w}}
\newcommand{\vsf}{{\sf v}}
\newcommand{\xsf}{{\sf x}}
\newcommand{\ysf}{{\sf y}}
\newcommand{\zsf}{{\sf z}}


% mixed symbols

\newcommand{\sinc}{{\hbox{sinc}}}
\newcommand{\diag}{{\hbox{diag}}}
\renewcommand{\det}{{\hbox{det}}}
\newcommand{\trace}{{\hbox{tr}}}
\newcommand{\sign}{{\hbox{sign}}}
\renewcommand{\arg}{{\hbox{arg}}}
\newcommand{\var}{{\hbox{var}}}
\newcommand{\cov}{{\hbox{cov}}}
\newcommand{\Ei}{{\rm E}_{\rm i}}
\renewcommand{\Re}{{\rm Re}}
\renewcommand{\Im}{{\rm Im}}
\newcommand{\eqdef}{\stackrel{\Delta}{=}}
\newcommand{\defines}{{\,\,\stackrel{\scriptscriptstyle \bigtriangleup}{=}\,\,}}
\newcommand{\<}{\left\langle}
\renewcommand{\>}{\right\rangle}
\newcommand{\herm}{{\sf H}}
\newcommand{\trasp}{{\sf T}}
\newcommand{\transp}{{\sf T}}
\renewcommand{\vec}{{\rm vec}}
\newcommand{\Psf}{{\sf P}}
\newcommand{\SINR}{{\sf SINR}}
\newcommand{\SNR}{{\sf SNR}}
\newcommand{\MMSE}{{\sf MMSE}}
\newcommand{\REF}{{\RED [REF]}}

% Markov chain
\usepackage{stmaryrd} % for \mkv 
\newcommand{\mkv}{-\!\!\!\!\minuso\!\!\!\!-}

% Colors

\newcommand{\RED}{\color[rgb]{1.00,0.10,0.10}}
\newcommand{\BLUE}{\color[rgb]{0,0,0.90}}
\newcommand{\GREEN}{\color[rgb]{0,0.80,0.20}}

%%%%%%%%%%%%%%%%%%%%%%%%%%%%%%%%%%%%%%%%%%
\usepackage{hyperref}
\hypersetup{
    bookmarks=true,         % show bookmarks bar?
    unicode=false,          % non-Latin characters in AcrobatÕs bookmarks
    pdftoolbar=true,        % show AcrobatÕs toolbar?
    pdfmenubar=true,        % show AcrobatÕs menu?
    pdffitwindow=false,     % window fit to page when opened
    pdfstartview={FitH},    % fits the width of the page to the window
%    pdftitle={My title},    % title
%    pdfauthor={Author},     % author
%    pdfsubject={Subject},   % subject of the document
%    pdfcreator={Creator},   % creator of the document
%    pdfproducer={Producer}, % producer of the document
%    pdfkeywords={keyword1} {key2} {key3}, % list of keywords
    pdfnewwindow=true,      % links in new window
    colorlinks=true,       % false: boxed links; true: colored links
    linkcolor=red,          % color of internal links (change box color with linkbordercolor)
    citecolor=green,        % color of links to bibliography
    filecolor=blue,      % color of file links
    urlcolor=blue           % color of external links
}
%%%%%%%%%%%%%%%%%%%%%%%%%%%%%%%%%%%%%%%%%%%



\newcommand{\mrinmaya}[1]{\textcolor{blue}{[mrinmaya: #1]}}


\usepackage{setspace}
\renewcommand{\algorithmicrequire}{ \textbf{Input:}}  
\renewcommand{\algorithmicensure}{ \textbf{Output:}}   

% For proper rendering and hyphenation of words containing Latin characters (including in bib files)
\usepackage[T1]{fontenc}
% For Vietnamese characters
% \usepackage[T5]{fontenc}
% See https://www.latex-project.org/help/documentation/encguide.pdf for other character sets

% This assumes your files are encoded as UTF8
\usepackage[utf8]{inputenc}

% This is not strictly necessary, and may be commented out.
% However, it will improve the layout of the manuscript,
% and will typically save some space.
\usepackage{microtype}

% This is also not strictly necessary, and may be commented out.
% However, it will improve the aesthetics of text in
% the typewriter font.
\usepackage{inconsolata}

\newcommand{\zoneicon}[1]{%
  \begin{tikzpicture}
    \fill[#1] (0,0) circle (0.13cm);
  \end{tikzpicture}%
}
\newcommand{\zone}[2]{%
  \begin{tikzpicture}
    \fill[#1] (0,0) -- ++(90:0.13cm) arc (90:270:0.13cm) -- cycle;
    \fill[#2] (0,0) -- ++(-90:0.13cm) arc (-90:90:0.13cm) -- cycle;
    % \draw (0,0) circle (0.15cm);
  \end{tikzpicture}%
}
\definecolor{z_red}{rgb}{0.8, 0.25, 0.25}
\definecolor{z_green}{rgb}{0.18, 0.55, 0.34}
\definecolor{z_orange}{rgb}{0.93, 0.57, 0.13}
\definecolor{z_blue}{rgb}{0.16, 0.32, 0.75}

\newcommand{\zonex}{\xspace \zoneicon{z_green}{$\mathcal{Z}_{\text{\ding{51}}}$}\xspace}
\newcommand{\zoney}{\xspace \zoneicon{z_blue}{$\mathcal{Z}_{\text{\ding{55}} \rightarrow \text{\ding{51}}}$}\xspace}
\newcommand{\zonez}{\xspace \zoneicon{z_red}{$\mathcal{Z}_{\text{\ding{55}} \rightarrow \text{\ding{55}}}$}\xspace}
\newcommand{\zonexx}{\xspace \zoneicon{z_orange}{$\mathcal{Z}_{\text{\ding{51}} \rightarrow \text{\ding{55}}}$}\xspace}


\ExplSyntaxOn
\NewDocumentCommand{\cb}{m}
{
  \fp_set:Nn \l_tmpa_fp { 60 } 
  \adjustbox{margin=1.5pt, bgcolor=gray!\fp_eval:n{\l_tmpa_fp*#1}}{#1}
}
\ExplSyntaxOff

% \newcommand{\zonex}{\zone{z_green}{z_green}{$\mathcal{Z}_{\text{\ding{51}}}$}\xspace}
% \newcommand{\zoney}{\zone{z_red}{z_green}{$\mathcal{Z}_{\text{\ding{55}} \rightarrow \text{\ding{51}}}$}\xspace}
% \newcommand{\zonez}{\zone{z_red}{z_red}{$\mathcal{Z}_{\text{\ding{55}} \rightarrow \text{\ding{55}}}$}\xspace}
% \newcommand{\zonexx}{\zone{z_orange}{z_orange}{$\mathcal{Z}_{\text{\ding{51}} \rightarrow \text{\ding{55}}}$}\xspace}
% If the title and author information does not fit in the area allocated, uncomment the following
%
%\setlength\titlebox{<dim>}
%
% and set <dim> to something 5cm or larger.


\title{Investigating the Zone of Proximal Development of Language Models\\ for In-Context Learning}


\author{
Peng Cui~\qquad~Mrinmaya Sachan \\
Department of Computer Science, ETH Zürich \\
\texttt{
\href{mailto:peng.cui@inf.ethz.ch}{peng.cui@inf.ethz.ch}
}\\
}


\begin{document}
\maketitle
\begin{abstract}


In this paper, we introduce a learning analytics framework to analyze the in-context learning (ICL) behavior of large language models (LLMs) through the lens of the Zone of Proximal Development (ZPD), an %well-
established theory in educational psychology. 
ZPD delineates the space between what a learner is capable of doing unsupported and what the learner cannot do even with support.
We adapt this concept to ICL, measuring the ZPD of LLMs based on model performance on individual examples before and after ICL.
Furthermore, we propose an item response theory (IRT) model to predict the distribution of zones for LLMs.
Our findings reveal a series of intricate and multifaceted behaviors of ICL, providing new insights into understanding and leveraging this technique. 
Finally, we demonstrate how our framework can enhance LLM in both inference and fine-tuning scenarios:
(1) By predicting a model’s zone of proximal development, we selectively apply ICL to queries that are most likely to benefit from demonstrations, achieving a better balance between inference cost and performance; 
(2) We propose a human-like curriculum for fine-tuning, which prioritizes examples within the model’s ZPD. 
The curriculum results in improved performance, and we explain its effectiveness through an analysis of the training dynamics of LLMs.\footnote{Code is available at \href{https://github.com/nlpcui/llm_zpd}{https://github.com/nlpcui/llm-zpd}} 

\end{abstract}

\section{Introduction}
Human learning is a dynamic and progressive process where learners integrate new information into their knowledge base through interactions with the environment \cite{piaget1977development}.
Research in education and learning sciences has extensively explored what makes learning most effective and efficient.
Among them, the Zone of Proximal Development (ZPD) emphasizes the alignment between the learner's capability and the problem's difficulty \cite{vygotsky1978mind}.
Specifically, ZPD refers to the range of problems that a learner can solve with appropriate scaffolding but cannot tackle independently.
This concept is essential in education as
it identifies knowledge that is valuable for learning, feasible to acquire, and not yet mastered. 
Therefore, learning within ZPD is believed to foster more effective cognitive development \cite{chaiklin2003zone, tharp1991rousing}.


In this paper, we propose a learning analytics framework to study the \emph{learning behavior} of language models through the lens of ZPD. 
In particular, we focus on in-context learning (ICL), an emerging ability of LLMs that allows them to learn from a few demonstrations \cite{brown2020language,wei2022emergent}.
Previous studies have primarily focused on strategies for demonstration optimization \citep{liu2021makes,qin2023context,rubin2021learning,ye2023compositional}. 
However, even with high-quality demonstrations, the performance of ICL still varies significantly across tasks and data \cite{srivastava-etal-2024-nice}.
This variability calls for a more comprehensive examination of the \emph{inherent} in-context learnability of LLMs on individual queries. 

\begin{figure}
    \centering
    \includegraphics[width=\columnwidth]{figures/motivation.pdf}
    \caption{We conceptualize an LLM's Zone of Proximal Development (ZPD) for ICL as the set of queries on which the model's performance can be improved with demonstrations. We introduce a framework to measure and predict this zone and explore its applications.}
    \label{fig:illustration}
\end{figure}

We first formalize the concept of ZPD in ICL. 
Drawing on the parallel between ICL and human learning from worked examples, we view LLMs as learners and in-context demonstrations as a form of scaffolding. 
Then, based on the model's prior knowledge and its response to ICL, a query set can be divided into three \textbf{zones} ($\mathcal{Z}$): 
(1) The first zone, denoted as \zonex, consists of queries that can be solved by the model via direct prompting, representing the model's prior knowledge;
(2) The second zone, denoted as \zoney, includes queries that can be solved by the model only with ICL, representing the model's ZPD; and
(3) The third zone, denoted as \zonez, contains queries that the model cannot solve even with ICL, representing the knowledge beyond the model's reach. 
Figure \ref{fig:illustration} illustrates this conceptualization. 
This categorization provides a granular look at the model's capability, limitations, and interaction with specific interventions. 

We begin by measuring the task-specific zones of various models (\cref{sec: measure}). 
Since the ICL performance is sensitive to the choice of demonstrations and the ground-truth demonstrations are not available, it is non-trivial to determine whether a problem can potentially benefit from ICL.
To address this, we employ a greedy algorithm to construct \emph{Oracle} demonstrations for each query and use them to approximate the zone distribution empirically. 
Then, we propose to predict the zones of unseen queries using the item Response theory (IRT; \citet{santor1998progress}), which jointly captures the latent traits of the model and query (e.g., ability, difficulty). 
In particular, we introduce a variant of IRT that further takes into account the model's in-context learnability to capture the performance changes with or without ICL (\cref{sec: pred}).
We find that the ICL behavior of LLMs is generally predictable even without demonstration information, although the degree of predictability varies across different datasets and tasks.

Finally, we showcase how our framework enhances LLMs in both inference and fine-tuning scenarios (\cref{Sec: zone_application}). 
For inference, we propose a selective ICL strategy, which first predicts the zone of input queries and then applies ICL only to queries that are most likely to benefit from ICL (i.e., within the model’s ZPD \zoney). 
Experimental results show this approach achieves competitive or even better performance with reduced inference cost.
For fine-tuning, we propose a ZPD-based curriculum that prioritizes challenging yet learnable training examples.
We find such a curriculum improves fine-tuning outcomes.
Upon further analysis of training dynamics, we find LLMs exhibit  \emph{consistent learnability} under both ICL and fine-tuning settings.
This consistency explains the effectiveness of our ZPD-based curriculum and suggests potential connections between these two learning paradigms.

In summary, our contributions are threefold:
\begin{itemize}
 \item We conceptualize the ZPD framework for LLMs, which provides a new perspective on analyzing their ICL behavior.
 \item We introduce a novel IRT variant that captures LLMs’ in-context learnability and predicts their performance with or without ICL.
 \item We showcase two applications of our framework: a selective ICL strategy and a ZPD-based curriculum, demonstrating its potential to enhance both LLM training and inference.
\end{itemize}

\section{Related Work}
\noindent \textbf{In-Context Learning} \cite{brown2020language} has become a popular paradigm for enhancing the capabilities of LLMs across a wide range of tasks. 
Previous work has extensively focused on optimizing demonstrations, particularly through the selection \cite{liu-etal-2022-makes,rubin-etal-2022-learning,li2023unified} and ranking \cite{pmlr-v139-zhao21c,lu-etal-2022-fantastically} of in-context examples. 
In this paper, we shift the focus from demonstration optimization to the LLM and the target query themselves, highlighting the inherent in-context learnability of LLMs on individual queries. 
Our study complements these works, contributing to a holistic understanding of what makes ICL (un)successful.
Another line of research explores how the ICL capability emerges and functions, with various hypotheses proposed, such as task recognition \cite{xie2022incontext, 10.5555/3666122.3666809}, composition \cite{li-etal-2024-language}, meta-gradient learning \cite{garg2022can,akyurek2023what}. 
This paper also aims to understand ICL but from an empirical perspective by collecting, analyzing, and predicting ICL behaviors.

\vspace{1.5mm}
\noindent \textbf{Adoption of IRT in NLP.}
IRT is a set of statistical models used in educational assessments to measure the latent abilities of individuals through standardized testing \cite{lord2008statistical, santor1998progress}. 
In recent years, it has become increasingly popular in NLP. 
\citet{byrd2022predicting} uses IRT to estimate question difficulty and model
skills. 
\citet{gor2024great} proposes a content-aware and identifiable IRT to analyze human-AI complementarity. 
\citet{polo2024tinybenchmarks} argues for the adoption of IRT to build benchmarks for efficient evaluation.
In this work, we use IRT to predict LLM in-context learnability on individual queries (conceptualized as ZPD) by capturing the behavior of LLMs before and after (in-context) learning.

\vspace{1.5mm}
\noindent \textbf{Curriculum Learning} \cite{bengio2009curriculum} is the approach that organizes the training examples such that the model converges faster and better, which has been successfully applied in various NLP tasks \cite{tay2019simple,platanios2019competence,sachan2016easy}. 
Typically, curriculum learning algorithms organize training examples in increasing order of difficulty. 
Conversely, there is another line of research that works in the opposite way to start with hard examples, namely Hard Example Mining \cite{shrivastava2016training,jin2018unsupervised}. 
In this paper, we propose a ZPD-based curriculum that strikes a middle point between the two techniques: prioritizing training examples that are challenging and yet learnable (i.e., within the model's ZPD).
Similar strategies have been proven effective in various scenarios \cite{mindermann2022prioritized}. 
However, this paper proposes a new framework for discovering such desired examples, which can be incorporated into existing approaches. 



\section{Measuring ZPD of LLMs}\label{sec: measure}
\subsection{Preliminaries} \label{sec: preliminary}
Let $\mathcal{D}=\{(x_1, y_1), ..., (x_n, y_n)\}$ be a dataset where $x_i$ is a query and $y_i$ is the ground-truth answer. 
We define the ZPD (\zoney) of a model $\mathcal{M}$ on $D$ as a subset of examples on which the model's performance can be improved through a learning trial. 
In this study, we focus on the ICL setting and measure learning outcomes by comparing the model's performance with and without ICL. 
Specifically, let $c = \{ (x_1, y_1)...(x_{k}, y_{k}) | x_{j} \in \mathcal{D}\}$ be a set of demonstrations for $x$ $(x \notin c)$, we define \zoney as:
\begin{gather}
    \text{\zoney} \triangleq \{x | \mathcal{F}(y^\varnothing)<\tau, \mathcal{F}(y^{c}) > \tau \},
\end{gather}
where $\mathcal{F}$ is a scoring function and $\tau$ is a threshold deciding whether the predicted answer is acceptable. 
$y^{\varnothing}$ and $y^{c}$ represent the model's output with direct prompting and with in-context demonstrations, respectively:
\begin{gather}
    y^{\varnothing} = \mathcal{M}(\mathcal{T}(x)), y^{c} = \mathcal{M}(\mathcal{T}(c_1) \oplus ... \oplus \mathcal{T}(x)).
\end{gather}
where $\mathcal{T}$ is a template function and $\oplus$ denotes string concatenation. 
Due to the potential interference between instruction and demonstration \cite{srivastava-etal-2024-nice}, we adopted a simple prompt template with minimal instruction to focus on the effect of demonstration (See Appendix Table \ref{tab:app_prompt}).
 
Similarly, we can define the other two subsets as follows:
\begin{gather}
    \text{\zonex} \triangleq \{x | \mathcal{F}(y^\varnothing)>\tau \}, \\
    \text{\zonez} \triangleq \{x | \mathcal{F}(y^\varnothing)<\tau, \mathcal{F}(y^{c})<\tau \},
\end{gather}
representing queries that can be solved by $\mathcal{M}$ with direct prompting,
and queries that cannot be solved even with ICL. 

This formalization is flexible and can be applied to other settings.
For example, future work could replace ICL with other prompting strategies or analyze fine-tuning behaviors by examining the performance across different epochs. 

\subsection{Approximating \zoney and \zonez}
While \zonex is deterministic from the model's base performance $\{y^{\varnothing}_1, y^{\varnothing}_2, ...\}$, \zoney and \zonez depend on the choice of demonstrations $c$.
In this paper, we aim to investigate the ideal ICL behavior of LLMs with \emph{optimal} demonstrations.
This is because our goal is to understand the model’s \emph{inherent} in-context learnability on individual queries rather than the behavior of a specific ICL strategy. 
Since optimal demonstrations for each query are unavailable, precise measurements of \zoney and \zonez are infeasible. 
To address this, we first create \emph{Oracle} demonstrations---the best demonstrations achievable in a practical setting (with a limited demonstration pool and restricted computation resources).
Then, we use them to approximate \zoney and \zonez.

In concrete, we adopt a retrieve and rank method to construct Oracle demonstrations.
Firstly, we retrieve a candidate set $\mathcal{C}$ for each query. 
The common belief is that demonstrations that are similar to the query are most likely to enhance performance \cite{liu-etal-2022-makes}. 
Following previous work \cite{rubin-etal-2022-learning}, we employ BM25 \cite{robertson2009probabilistic}, a sparse retriever based on surface features, and SBERT \cite{reimers2019sentence}, which is based on dense sentence encoding. 
For the two retrievers, we calculate similarities based on both the $(x, y)$ pair and the ground-truth answer $y$ only, resulting in 2 $\times$ 2 $\times$ $K$ candidates.  
However, similarity may not be the only criterion for demonstration selection. 
To further enrich the candidate set and recall effective but dissimilar demonstrations, we randomly sample $K$ candidates from the bottom 50 percentile of the retrieving results, doubling the candidate size. 

Next, we select Oracle demonstrations $c$ using a greedy scoring approach:
\begin{gather}
    c_i = \mathop{{\rm argmax}}\limits_{\mathcal{C} \setminus \{c_1,..,c_{i-1}\} } {\rm Prob}_{\mathcal{M}} (y|c_1 \oplus ... c_{i} \oplus x),
\end{gather}
where $c_i$ is the $i^{th}$ selected demonstration and ${\rm Prob_{\mathcal{M}}(\cdot)}$ is the probability from the model $\mathcal{M}$. 
In other words, we greedily choose demonstrations that can maximize the likelihood of the ground-truth answer.
With these demonstrations, the resulting \zoney is a subset of the actual ZPD while \zonez is a superset of the actual one.
In the rest of the paper, we use \zoney and \zonez to denote for the approximated zones unless otherwise specified. 

\section{Zone Prediction} \label{sec: pred}
In this section, we attempt to build a model to predict an LLM’s zone distribution on unseen queries. 
Essentially, the goal is to predict the model's performance, i.e., whether it can solve a query directly (\zonex) or with ICL (\zoney), or not at all (\zonez). 
We propose a novel variant of item response theory (\textsc{Irt}) to capture the latent traits of the LLM and the queries.
A graphic view of our model is shown in Figure \ref{fig:model}.

\subsection{Background of \textsc{Irt}}
\textsc{Irt} is a statistical model that predicts the probability of individual respondents correctly answering a set of queries (or items).
In this work, we take a collection of LLMs $\{\mathcal{M}_1, \mathcal{M}_2, ..., \mathcal{M}_m\}$ as respondents. 
The basic 1 Parameter Logistic (1PL) \textsc{Irt} is defined as:
\begin{gather}
    P(r_{i,j}=1|\mathcal{M}_{i}, x_{j}) = \sigma(\theta_{i}-d_{j}), \label{eq:irt_1PL}
    \end{gather} 
where $r_{i,j}$ is the binary correctness label of $\mathcal{M}$'s prediction on $x_i$. 
$\sigma$ is the ${\rm sigmoid}$ function. 
$\theta_i$ and $d_j$ are latent variables (scalars) to be estimated, representing the ability of the $i$th model $\mathcal{M}_{i}$ and the difficulty of the $j$th query $x_j$. 
Simply put, \textsc{Irt} predicts the correctness label based on the gap between model ability and query difficulty.

The 1PL \textsc{Irt} assumes the monotonic relationship between item difficulty and respondent ability.
To relax this, we employ the multi-dimensional IRT (\textsc{MIrt}, \citet{reckase200618}), which is defined as:
\begin{gather}
    P(r_{i,j}=1|\mathcal{M}_{i}, x_{j}) = \sigma(\boldsymbol{\theta}_{i}^{\mathrm{T}}\boldsymbol{\alpha}_{j}-d_{j}), \label{eq:MIRT-DP}
\end{gather}
where the model's ability is represented as a \emph{skill vector} $\boldsymbol{\theta}_j \in \mathbb{R}^{\rm H}$.
Correspondingly, an item-wise \emph{discrimination vector} $\boldsymbol{\alpha}_i \in \mathbb{R}^{\rm H}$ is introduced to represent its latent traits.
A closer alignment between $\boldsymbol{\theta}_{i}$ and $\boldsymbol{\alpha}_j$ indicates a higher likelihood of a correct response.
%\mrinmaya{I think the above part about MIRT can be written better. We}

The training objective of \textsc{Irt} is defined as:
\begin{gather}
    \mathcal{L}_{\rm IRT}=\sum_{i=1}^{\rm M}\sum_{j=1}^{\rm N} {\rm CE}(P(r_{i,j}), y_{j}),
\end{gather}
where ${\rm CE(\cdot)}$ stands for the cross-entropy loss between predicted probability and the groud-truth label.

\begin{figure}[t]
    \centering
    \includegraphics[width=\columnwidth]{figures/method.pdf}
    \caption{
    We assume that a model's performance on a given query, $y^{c}$ (with ICL) or $y^{\varnothing}$ (without ICL), is determined by latent traits (shadowed nodes, bottom) of both the model and the query, including the model's skill $\boldsymbol{\theta}$, ICL skill $\boldsymbol{\theta^{c}}$, the query's discrimination $\boldsymbol{\alpha}$, ICL discrimination $\boldsymbol{\alpha^c}$, and overall difficulty $d$. 
    }
    \label{fig:model}
\end{figure}


\subsection{Content-Aware \textsc{Mirt}}
A limitation of \textsc{MIrt} is that it relies on the response data to infer item traits $\boldsymbol{\alpha}_i$. 
Therefore, it cannot generalize to unseen queries during inference.
To overcome this limitation, we use a lightweight neural network to parameterize item traits based on their text features. 
Specifically, for a given query $x_{j}$, we first use an embedding model to obtain its representation $\boldsymbol{e}_j$.
Then, we compute its traits by:
\begin{gather}
    d_{j} = f({\rm {\bf W}_{d}} \boldsymbol{e}_{j} + {\rm {\bf b}_{d}}); 
    \boldsymbol{\alpha}_j = f({\rm {\bf W}_{\alpha}} \boldsymbol{e}_{j} + {\rm {\bf b}_{\alpha}})
\end{gather}
where ${\rm {\bf W}_{d}}, {\rm {\bf W}_{\alpha}}, {\rm {\bf b}_{d}, {\rm {\bf b}_{\alpha}}}$ are learnable weights, trained together with the IRT model, and $f$ is the ${\rm Relu}$ function.

\subsection{Adapting \textsc{Mirt} to Learning Dynamics}
While the above model can predict the model's performance on an unseen query, it cannot predict one query's correctness label under two settings and thus cannot predict three zones simultaneously.
We propose a variant that incorporates the dynamics of ICL. 
Concretely, we introduce an additional \emph{ICL skill vector} $\boldsymbol{\theta}^{c}$ for the model and similarly an \emph{ICL discrimination vector} $\boldsymbol{\alpha}^{c}$ for the item:
\begin{gather}
         P(r_{i,j}=1|\mathcal{M}_{i}, x_{j}) = \sigma(\boldsymbol{\theta}_{i}^{\mathrm{T}}\boldsymbol{\alpha}_j-d_{j}+\boldsymbol{\theta}^{c\mathrm{T}}_{i }\boldsymbol{\alpha}_{j}^{c}), \label{eq:MIRT-ICL}
\end{gather}
where the alignment between $\boldsymbol{\theta}^{c}_{i}$ and $\boldsymbol{\alpha}_{j}^{c}$ represents the in-context \emph{learnability} of $\mathcal{M}_{i}$ with respect to $x_j$. 
Similar to $d$ and $\boldsymbol{\alpha}$, $\boldsymbol{\alpha}^{c}_{j}$ is computed based on the embedding of x:
\begin{gather}
    \alpha^{c}_{j} = f({\rm {\bf W}^{c}_{\alpha}} e_{j} + {\rm {\bf b}^{c}_{\alpha}}).
\end{gather}
Combining Eq. \ref{eq:MIRT-DP}, and \ref{eq:MIRT-ICL}, we have:
\begin{gather}
         P(r_{i,j}^{\{ \varnothing, c \}}=1) = \sigma(\boldsymbol{\theta_{i}} \boldsymbol{\alpha_j} - d_{j} + g^{\{ \varnothing, c \} } \boldsymbol{\theta}^{c}_{i }\boldsymbol{\alpha}_{j}^{c}), \label{eq:combined}
\end{gather}
where $r^{\varnothing}$ and $r^{c}$ are the correctness labels under direct prompting and ICL.
$\{g^\varnothing=0, g^c=1\}$ is a gating parameter in align with $r$ to ensure that $\theta^{c}_{i }\alpha_{j}^{c}$ are only enabled in the ICL setting. 
In doing this, the latent factors are learned such that:
\begin{gather}
    \left\{
        \begin{array}{cc}
            \boldsymbol{\theta}^{\mathrm{T}} \boldsymbol{\alpha} > d, \boldsymbol{\theta}^{\mathrm{T}} \boldsymbol{\alpha} + \boldsymbol{\theta}^{c\mathrm{T}} \boldsymbol{\alpha}^{c} >d, \ \text{if} \ \ r^{\varnothing}=1, &\\
             \boldsymbol{\theta}^{\mathrm{T}} \boldsymbol{\alpha} <d, \boldsymbol{\theta}^{\mathrm{T}} \boldsymbol{\alpha} + \boldsymbol{\theta}^{c\mathrm{T}} \boldsymbol{\alpha}^{c} >d, \ \text{if} \ \ r^{\varnothing}=0, r^{c}=1,  &\\
             \boldsymbol{\theta}^\mathrm{T} \boldsymbol{\alpha} <d, \boldsymbol{\theta}^\mathrm{T} \boldsymbol{\alpha} + \boldsymbol{\theta}^{c} \mathrm{T} \boldsymbol{\alpha}^{c} <d, \ \text{if} \ \ r^{\varnothing}=0, r^{c}=0. &
        \end{array}
    \right.
\end{gather}
The above three situations correspond to \zonex, \zoney, and \zonez, respectively. 
We refer to the proposed model as \textsc{Mirt}$_{\textsc{Icl}}$. 

From a multi-task learning perspective, our model can be seen as jointly training two IRT models, each with its own ability ($\theta, \theta^{c}$) and discrimination ($\alpha, \alpha^{c}$) parameters, while sharing the overall item difficulty ($d$). 
This allows the model to better capture the relationships between LM behaviors across the two settings.

\section{Experiments}
We experiment with 8 LLaMA models \cite{touvron2023llama,dubey2024llama} of various sizes, including \texttt{LLaMA-2-7B}, \texttt{LLaMA-2-7B-chat},
\texttt{LLaMA-2-13B}, \texttt{LLaMA-2-13B-chat}, \texttt{LLaMA-3-8B}, \texttt{LLaMA-3-8B-Instruct},  \texttt{LLaMA-3-70B}, and \texttt{LLaMA-3-70B-Instruct}. 
In particular, we consider both instruction-tuned (IT) (\texttt{-chat/Instruct} models) or non-IT versions to examine the influence of instruction tuning on the model’s ZPD.
In this study, we focus on the \emph{mathematical reasoning} and \emph{text understanding} abilities of LLMs, using the MathQA dataset \textbf{GSM8K} \cite{cobbe2021gsm8k} and the Stance detection (Favor, Neutral, Against) dataset \textbf{EZStance}  \cite{zhao2023ez} for stance detection. 
Detailed experiment setup can be found in Appendix \ref{sec:app_setup}.


We first present and analyze the zone distribution of various LLaMA models (\cref{Sec:zone_dist}).
Then, we evaluate the performance of IRT models on zone prediction (\cref{Sec: zone_pred}). 
Finally, we demonstrate two applications of our framework (\cref{Sec: zone_application}). 

\subsection{Zone Distribution Analysis} \label{Sec:zone_dist}
\begin{figure}
    \centering
    \includegraphics[width=\columnwidth]{figures/zone_dist.png}
    \caption{Zone distribution of various LLMs on the two datasets. Yellow lines represent the accuracy of \textsc{Kate}.}
    \label{fig:zone_dist}
\end{figure}
We measure the three zones of LLaMA models on the test set of GSM8K and the validation set of EZStance. 
Our observations are as follows.

\vspace{1.5mm}
\noindent \textbullet\ \emph{\textbf{The potential of ICL remains largely untapped}}. 
In Figure \ref{fig:zone_dist}, we present the zone distributions of various models. 
Ideally, the accuracy of ICL should be the combined proportion of \zonex and \zoney, which highlights the great potential of ICL. 
For instance, on the GSM8K dataset, the 8B-Instruct model, with the help of Oracle demonstrations, can achieve competitive performance compared to the two 70B models.
Note that, however, this is only a lower bound of ideal ICL performance, as the Oracle demonstrations are still sub-optimal.
Nevertheless, the current method still falls short of fully utilizing even this lower bound. 
For reference, we highlight the accuracy (yellow line) of \textsc{Kate} \cite{liu2021makes}, a similarity-based demonstration selection strategy (with \texttt{paraphrase-mpnet-base-v2}).
On average, it lags by around 20\% on the two datasets.

\begin{figure}[t]
    \centering
    \includegraphics[width=\columnwidth]{figures/neg_icl.png}
    \caption{Increased and decreased accuracy by \textsc{Kate} on GSM8K (left) and EZStance (right).}
    \label{fig:neg_icl}
\end{figure}

\vspace{1.5mm}
\noindent \textbullet\ \emph{\textbf{In-context demonstrations can be harmful.}}
In Section \cref{sec: measure}, we divide a query set into three zones according to the model's performance difference with and without ICL. 
However, sometimes, ICL can also degrade the performance.
We denote the collection of such queries as \zonexx. 
We do not frame \zonexx into our formalization (\cref{sec: preliminary}) but merge it into \zonex because we focus on the ideal ICL behavior given Oracle demonstrations.
However, this negative effect of ICL is non-negligible in a practical setting.

With \textsc{Kate} as a case study, we compare its increased accuracy (i.e., the proportion of recalled ZPD (\zoney) examples) and decreased accuracy (i.e., the proportion of \zonexx examples) in Figure \ref{fig:neg_icl}. 
The sum of the two is the overall performance of $\textsc{Kate}$. 
We can see \zonexx can reduce up to 14\% and 18\% accuracy on GSM8K and EZStance.
Besides, this negative effect is also model-dependent. 
For example, \texttt{LLaMA-2-7B-chat} and \texttt{LLaMA-2-13B-chat} are particularly vulnerable to harmful demonstrations, and this negative effect even overwhelms the benefit for \texttt{LLaMA-3-70B}.
% More investigations are needed to understand this behavior, which we leave for future work. 
This granular look at the ICL performance provides a new perspective to improve ICL strategy: recalling examples in \zonex while minimizing \zonexx. 
Previous work mainly focused on the first direction and we will showcase how our IRT model can enhance ICL through the second way in \cref{App1}.


\begin{table}[h]
\small
\centering
\begin{tabular}{@{}lllllll@{}}
\toprule
\multirow{2}{*}{\textbf{Zones}} & \multicolumn{3}{c}{\textbf{GSM8K}}         & \multicolumn{3}{c}{\textbf{EZStance}}      \\ \cmidrule(l){2-7} 
                                & \textbf{Max} & \textbf{Min} & \textbf{Avg} & \textbf{Max} & \textbf{Min} & \textbf{Avg} \\ \midrule
\zonex                              & \cb{0.89}         & \cb{0.74}         & \cb{0.84}         & \cb{0.91}             & \cb{0.46}              & \cb{0.70}             \\
\zoney                              & \cb{0.74}         & \cb{0.21}         & \cb{0.58}         &  \cb{0.78}            & \cb{0.34}             & \cb{0.58}             \\
\zonez                              & \cb{0.58}         & \cb{0.20}         & \cb{0.42}         &   \cb{0.87}           &  \cb{0.32}            &  \cb{0.53}            \\ \bottomrule
\end{tabular}
\caption{Pairwise overlap coefficients among zones of different LLMs.} 
\label{tab: overlap}
\end{table}

\vspace{1.5mm}
\noindent \textbullet\ \emph{\textbf{ZPD (\zoney) of LLMs differ significantly}}. 
We measure the overlap between zones of different models by calculating their averaged pairwise \emph{Overlap Coefficient}, defined as:
\begin{gather}
    \textsc{Overlap} (A, B) = \frac{|A \cap B|}{{\rm min}(|A|, |B|)},
\end{gather}
where $A$ and $B$ are the zones to compare. 
The results are shown in Table \ref{tab: overlap}, where we can see examples in \zonex are largely shared across various models, while examples in \zoney and \zonez do not highly overlap, indicating each LLM has its own ZPD.
This suggests that ICL strategies should take into account both the data aspect (e.g., similarity) and the model, corroborating the conclusion of \citet{peng-etal-2024-revisiting}. 



\subsection{Zone Prediction Evaluation}\label{Sec: zone_pred}

We compare our proposed IRT model \textsc{Mirt}$_{\rm ICL}$ (Eq. \ref{eq:combined}) with the following baselines:
i) \underline{1PL model} (\textsc{Irt}$_{\rm 1PL}$, Eq. \ref{eq:irt_1PL}), 
ii) \underline{2PL model}, which is similar to Eq. \ref{eq:MIRT-DP} but with $\theta$ and $\alpha$ as scalars, 
and iii) \underline{Multi-Dimensional IRT} \textsc{Mirt} (Eq. \ref{eq:MIRT-DP}).
We evaluate their ability to predict LLM performance under both direct prompting (DP) and ICL, using AUC as the primary metric. See Appendix \ref{sec:app_irt_implementation} for the implementation details. 
Note that aside from our \textsc{Mirt}$_{\textsc{Icl}}$, other baseline models are trained solely on DP data. 
Nevertheless, we can assess their generalization ability to the ICL setting: since AUC assesses the relative ranking of predicted probabilities, these models should also achieve good AUC if \emph{the LLM's probabilities of correctly answering individual queries are consistent across both settings}. 


\begin{table}[t]
\small
\centering
\setlength\tabcolsep{4pt}
\begin{tabular}{@{}lcccccc@{}}
\toprule
\multirow{2}{*}{\textbf{Model}} & \multicolumn{3}{c}{\textbf{GSM8K}} & \multicolumn{3}{c}{\textbf{EZStance}} \\ \cmidrule(l){2-7} 
                                & DP        & ICL       & Overall    & DP         & ICL        & Overall     \\ \midrule
\textsc{Irt}$_{\rm 1PL}$                             & 0.808     & 0.769     & 0.748      & 0.736      & 0.617      & 0.644       \\
\textsc{Irt}$_{\rm 2PL}$                             & 0.788     & 0.740     & 0.728      & 0.739      & 0.631      & 0.651       \\
\textsc{Mirt}                            & \textbf{0.837}     & 0.770     & 0.743      & 0.760      & 0.608      & 0.799       \\
\textsc{Mirt}$_{\rm ICL}$                            & 0.833     & \textbf{0.821}     & \textbf{0.862}      & \textbf{0.770}      & \textbf{0.662}      & \textbf{0.799}       \\ \bottomrule
\end{tabular}
\caption{Performance (AUC) of various IRT models on the two datasets. The best results are in \textbf{bold}. Results of Accuracy can be found in Appendix Table \ref{tab:app_irt_acc}.}
\label{tab:irt_auc}
\end{table}


\vspace{1.5mm}
\noindent \textbullet\ \emph{\textbf{ICL behavior is, to varying degrees, predictable without demonstrations}}. 
We present the AUC results in Table \ref{tab:irt_auc}. 
As a demonstration-agnostic model, \textsc{Mirt}$_\textsc{\ Icl}$ achieves reasonably decent performance GSM8K but comparatively weaker results on EZStance. 
We interpret the difference through the \emph{predictability} and \emph{sensitivity} of ICL: for certain tasks and datasets, ICL performance may hinge more on the model’s inherent ICL capacity and the query's difficulty. 
While for others, it may depend more on the demonstrations or prompts, making the ICL behavior less predictable without the information of demonstrations.
Existing work has been focusing on measuring and mitigating sensitivity \cite{zhao2021calibrate}. 
We highlight a complementary perspective: measuring and leveraging (See \cref{Sec: zone_application} for applications) the predictability of ICL behavior.
\vspace{1.5mm}


\begin{table}[t]
\small
\centering
\setlength\tabcolsep{2pt}
\begin{tabular}{@{}lcccccccc@{}}
\toprule
\multirow{2}{*}{} & \multicolumn{2}{c}{\textbf{L2-7B}} & \multicolumn{2}{c}{\textbf{L2-13B}} & \multicolumn{2}{c}{\textbf{L3-8B}} & \multicolumn{2}{c}{\textbf{L3-70B}} \\
                  & base    & chat   & base    & chat    & base    & instr.   & base    & instr.    \\ \midrule
\textbf{GSM8K}    & $+$.10            & $+$.40            & $+$.13$^*$             & $+$.24             & $+$.29             & $+$.07            & $+$.31             & $+$.53             \\
\textbf{EZStance} & $-$.45            & $-$.60           & $-$.47            & $-$.25            & $-$.35            & $-$.28           & $-$.48            & $-$.36            \\ \bottomrule
\end{tabular}
\caption{Pearson Correlation between $\boldsymbol{\theta}^{\mathrm{T}}\boldsymbol{\alpha}-d$ (model's ability to solve the query with direct prompting) and $\boldsymbol{\theta}^{c\mathrm{T}} \boldsymbol{\alpha}^{c}$ (the additional gain obtained by ICL). Results with $^{*}$ indicate $p$-value$>0.05$.} \label{tab:pearson}
\vspace*{-2mm}
\end{table}


\noindent \textbullet\ \emph{\textbf{(In)consistency between difficulty and in-context learnability.}}
In Eq. \ref{eq:combined}, $\boldsymbol{\theta}\boldsymbol{\alpha} - d$ represents the model's ability to solve the query with DP (or the query's \textit{difficulty}), while $\boldsymbol{\theta}^{c} \boldsymbol{\alpha}^c$ captures the additional gain achieved through ICL, reflecting the model's \textit{in-context learnability} of the query.
To examine the relationship between the two terms, we compute their Pearson correlation. 
The results, presented in Table \ref{tab:pearson}, reveal that for the GSM8K dataset, these two terms exhibit weak or moderate positive correlations (from $+$0.07 to $+$0.53). 
Interestingly, the correlation on EZStance is stronger but negative, meaning difficult examples under direct prompting (lower $\boldsymbol{\theta}^{\mathrm{T}}\boldsymbol{\alpha} - d$) seem to benefit more from ICL (higher $\boldsymbol{\theta}^{c \mathrm{T}} \boldsymbol{\alpha}^c$) and vice versa. 
% However, due to the weaker performance on EZstance, this phenomenon should be interpreted with caution. 
This suggests that a query’s difficulty and its in-context learnability are not always aligned. We attribute this phenomenon to the differing abilities required for direct prompting versus ICL. 
The former primarily relies on the model’s prior knowledge of the query, while the latter depends on its ability to leverage contextual information. 
As a result, this inconsistency could arise in certain tasks and queries where the knowledge is missing but easy to learn in context. 
A notable example is classification with flipped or semantically unrelated labels \cite{wei2023larger}, where an LM struggles to solve the disrupted task in the regular setting but can successfully learn the new mapping through demonstrations.



\subsection{Applications} \label{Sec: zone_application}
In this section, we demonstrate how our framework can improve in-context learning through a selective ICL strategy (\cref{App1}) and a ZPD-derived curriculum for fine-tuning LLMs (\cref{App2}).

\subsubsection{Selective ICL}\label{App1}
\textbf{Approach.}
While ICL has demonstrated effectiveness across a wide range of tasks, it costs $k$ times additional input tokens ($k=$ the number of demonstrations).
Moreover, as discussed in \cref{Sec:zone_dist}, ICL sometimes results in worse performance, even with carefully retrieved demonstrations.
To address these issues, we propose Selective ICL (\textsc{SelIcl}). 
In specific, given a query $x_i$, we first predict its correct probability with direct prompting $p_{i}^{\varnothing}$ and the correct probability with ICL $p_{i}^{c}$ using Eq. \ref{eq:combined} with $g=0$ and $g=1$ respectively. 
Then, we decide the inference prompt for $x_i$ by:
\begin{gather}
    \left\{
        \begin{array}{ll}
             \mathcal{T}(\tilde{c}_1) ...  \oplus \mathcal{T}(x) \:  & \text{if} \; p^{\varnothing} < \tau_1 \; \text{and} \; p^{c} > \tau_2  \\
            \mathcal{T}(x) &\text{Otherwise.} 
        \end{array}
    \right. \label{eq:sel_icl}
\end{gather}
where $\{ \tilde{c}_{1}, ..., \tilde{c}_{n} \}$ are demonstrations retrieved by a certain strategy.
$\tau_1$ and $\tau_2$ are predefined thresholds.
A lower $p^{\varnothing}$ ($< \tau_1 $) and a higher $p^{c}$ ( $>\tau_2$) indicate the model is unable to solve this query with direct prompting but is likely to solve it with ICL. 
In other words, we apply ICL only to queries within the model's ZPD. 
By doing so, we aim to reduce unnecessary costs by avoiding ICL for either too easy ($p^{\varnothing} < \tau_1 $) or too hard ($p^{c} >\tau_2$) queries.
Furthermore, this can also potentially improve performance by mitigating the negative effect of ICL observed in Figure \ref{fig:neg_icl}.    

\vspace{1.5mm}


\begin{figure*}[th]
    \centering
    \includegraphics[width=\textwidth]{figures/gsm8k_sel_icl.png}
    \caption{Accuracy and inference cost (number of input tokens) of different ICL strategies on the GSM8K dataset. \textcolor{z_red}{$\blacktriangledown$} is the performance of the baseline \textsc{FulIcl}, which applies ICL to all the queries. \textcolor{z_blue}{$\bigcirc$} and \textcolor{z_green}{$\bigstar$} are the performance of \textsc{SelIcl} under various thresholds $\tau_1$ and $\tau_2$ (not shown), where \textcolor{z_green}{$\bigstar$} highlights cases in which \textsc{SelIcl} achieves better or equal accuracy with less input tokens compared to the baseline (\textcolor{z_red}{$\blacktriangledown$}).}
    \label{fig:gsm8k_sel_icl}
\end{figure*}

\noindent \textbf{Result and Analysis.}
We compare our \textsc{SelIcl} with the vanilla ICL that applies demonstrations to all queries  (denoted as \textsc{FulIcl}). 
Specifically, we use \textsc{Kate} to retrieve demonstrations for \textsc{FulIcl}.
However, it is worth noting that \textsc{SelIcl} is orthogonal to other ICL strategies for two reasons: 
(1) It focuses on determining when to apply ICL, independent of how demonstrations are selected or organized; 
(2) The IRT model is trained to predict the model’s ICL performance given Oracle demonstrations. 
Consequently, $p^{c}$ is expected to serve as the predicted upper bound for any ICL strategy.


To select $\tau_{1}$ and $\tau_{2}$ for \textsc{SelIcl}, we perform a grid search on the IRT validation set by varying their values within the range $[0.01, 0.02, \dots, 0.99]$.
For each combination, we decide whether or not to apply ICL to each query according to Eq. \ref{eq:sel_icl} and compute the overall accuracy and number of input tokens.  
Since the prompts and model outputs are already collected when constructing the IRT dataset (Appendix \ref{sec:app_irt_implementation}), these results can be obtained without additional model inference.


Then, we plot the Pareto curve \cite{deb2011multi} of \textsc{SelIcl}, approximated with scatter points.
In multi-objective optimization, each point on the Pareto curve represents a Pareto-optimal solution that cannot be further improved in one objective without compromising the other (in our case, accuracy and number of input tokens). 


Results for GSM8K are shown in Figure \ref{fig:gsm8k_sel_icl}, and results for EZstance are available in Appendix Figure \ref{fig:ezstance_sel_icl}.
Solutions that are dominated\footnote{
In the context of a Pareto curve, a solution dominates another if it is at least as good in all objectives and strictly better in at least one objective.} by others are discarded (apart from the baseline results (\textcolor{z_red}{$\blacktriangledown$}) for comparison).
As can be seen, for 6 out of 8 models, \textsc{SelIcl} with proper thresholds (\textcolor{z_green}{$\bigstar$}) can dominate \textsc{fulIcl}. 
Overall, \textsc{SelIcl} can serve as a tool to trade off accuracy and cost in resource-limited scenarios. 
\textsc{SelIcl} is paticularly successful for \texttt{LLaMA-2-7b-chat} and \texttt{LLaMA-70B-Instruct}. 
Combining with previous findings, both models have relatively narrow ZPD (Figure \ref{fig:zone_dist}) and are more susceptible to the negative effects of ICL (Figure \ref{fig:neg_icl}), suggesting that greater caution is needed when applying ICL to them.

\subsubsection{ZPD-based Curriculum}\label{App2}
It is generally believed that the success of ICL relies on the model's prior knowledge about the query \cite{xie2022incontext,li-etal-2024-language}.
Therefore, we assume that queries that can be enhanced by ICL (\zoney) are more learnable than those unsolvable by ICL (\zonez) but also more valuable for learning than those already solvable by DP (\zonex). 
Motivated by this, we proposed a ZPD-based curriculum learning algorithm for fine-tuning.

\begin{algorithm}[t]
\caption{ZPD-based Curriculum}
\label{alg:DCQGFramework}
\small
\setstretch{1.2}
    \begin{algorithmic}[1]
    \Require Training set $\mathcal{D}$, model $\mathcal{M}$, correct probability with DP $p^{\varnothing}$ and with ICL $p^{c}$, bucket $k$, epoch $e$
    \Ensure  Trained model $\mathcal{M}^{*}$ 
        \State  $\mathcal{D}^{*} \leftarrow $ Sort($\mathcal{D}$, $p^{c}_{i}-p^{\varnothing}_{i} $)
        \State $\{ \mathcal{D}_{1}, ..., \mathcal{D}_{n} \} \leftarrow \mathbf{SplitData}(\mathcal{D}^{*})$ ; $\mathcal{D}_{train} \leftarrow \varnothing$
        \For {$i=1, i\leq k, i$++}
            \State $\mathcal{D}_{train} \leftarrow \mathcal{D}_{train} \cup \mathcal{D}_{i}$ \Comment{Update training set}
            \For {$j=1, j\leq e, j$++}
            \State $\mathbf{Train}(\mathcal{M}, \mathcal{D}_{train})$;
            \EndFor
        \EndFor
    % \State \textbf{return} $\mathcal{M}^{*}$
    \end{algorithmic}
    \label{algo:cl_scheduler}
\end{algorithm}

\noindent \textbf{Approach.} 
Typically, curriculum learning consists of a ranking algorithm, which sorts examples according to a certain criterion, and a scheduling algorithm, which sequences examples for training. 
In our approach, we rank training examples according to $p^{c} - p^{\varnothing}$ (Eq. \ref{eq:sel_icl}), which represents the learning gain brought by ICL. 
For scheduling, we employ the baby-step algorithm \cite{spitkovsky2010baby}, which splits examples into buckets and accumulatively introduces new buckets. The overall process is outlined in Algorithm \ref{algo:cl_scheduler}.

\begin{figure}[t]
    \centering
    \includegraphics[width=\columnwidth]{figures/curriculum.png}
    \caption{Comparison between random and our \textsc{ZPD}-based curriculum on two datasets.}
    \label{fig:curriculum}
\end{figure}

\noindent \textbf{Results and Analysis.}
We compare our algorithm against a random baseline.
Although simple, random is the most widely used baseline in practice and is not necessarily a weak one, as many curriculum strategies fail to outperform it in language modeling \cite{campos2021curriculum}.
% \TODO{Random is not a weak baseline.}
We fine-tune the \texttt{LLaMA-8B-Instruct} model separately using the two methods with the same scheduler for $6$ epochs. 
See experimental details in Appendix \ref{sec:app_finetune}. 
As shown in Figure \ref{fig:curriculum}, our curriculum results in faster convergence and improved performance in most cases.
To understand why it works, we analyze the training loss of examples in different zones.
Specifically, we compute the \textbf{mean} and \textbf{variance} of each example's loss across epochs. 
The two metrics reflect the convergence behavior of individual examples: a higher mean indicates the example is harder to learn, while a higher variance indicates the model is ambiguous about the example \cite{swayamdipta-etal-2020-dataset}. 



For fair analysis, we fine-tune a new \texttt{LLaMA-8B-Instruct} model on the GSM8K dataset for $5$ epochs without any curriculum. 
Figure \ref{fig:training_loss} shows the loss information.
We found \emph{consistent learnability} between in-context learning and fine-tuning scenarios: 
examples in \zonez are the hardest to learn, 
followed by \zoney
\footnote{(Since we use sub-optimal Oracle demonstrations, some \zoney examples are not recalled and misclassified into \zonez. As a result, the actual loss value of \zoney data tends to be slightly closer to \zonez.)}, 
and lastly \zonex. 
This confirms that our curriculum works as expected: prioritizing examples that are learnable and informative (not yet learned).  
Such a strategy has been shown effective for various tasks and model architectures \cite{mindermann2022prioritized,fan2023irreducible}, and our framework provides a new way to discover these examples.


\begin{figure}[t]
    \centering
    \includegraphics[width=\columnwidth]{figures/loss.png}
    \caption{Mean and variance of training loss for queries in different zones. Results are computed over 5 epochs.}
    \label{fig:training_loss}
\end{figure}

\section{Conclusion}
This work presents a novel framework based on the Zone of Proximal Development (ZPD) theory to analyze the ICL behaviors of LLMs. 
We thoroughly discuss the formalization, measurement, prediction, and application of ZPD in LLMs. 
Our framework serves as an effective tool for understanding the potential, limitations, and complex dynamics of ICL. 
Furthermore, we demonstrate its applicability in both inference and training scenarios.

\section*{Limitations}
We discuss the limitations of this work from the following aspects.
First, due to the unavailability of optimal in-context demonstrations, we can only approximate the ZPD of LLMs, which is a lower bound of the model’s actual in-context learnability. 
This challenge is as nuanced and complex as understanding human learning: one can never precisely measure the potential of human learners.
Second, we investigate the ZPD of LLMs in a simplified scenario where we only consider demonstrations as guidance and use basic templates without instructions to minimize confounding factors. 
In practice, ICL is often combined with other prompting strategies, whose influence may warrant further exploration.
Finally, the ZPD is a dynamic range that evolves with the learner’s knowledge development. 
Our framework is designed to measure and leverage an LLM’s current ZPD, but it is less suited to modeling its developing process (e.g., across different checkpoints during pre-training or fine-tuning). 
In the future, more advanced learning analytics approaches, such as knowledge tracing, could be adopted to enhance our framework.


% This must be in the first 5 lines to tell arXiv to use pdfLaTeX, which is strongly recommended.
\pdfoutput=1
% In particular, the hyperref package requires pdfLaTeX in order to break URLs across lines.

\documentclass[11pt]{article}

% Remove the "review" option to generate the final version.
\usepackage[]{ACL2023}

% Standard package includes
\usepackage{times}
\usepackage{latexsym}
\usepackage{graphicx}
\usepackage{amsfonts}
\usepackage{booktabs}
\usepackage{multirow}
\usepackage{makecell}
\usepackage{amsmath}
\usepackage{subfigure}

% For proper rendering and hyphenation of words containing Latin characters (including in bib files)
\usepackage[T1]{fontenc}
% For Vietnamese characters
% \usepackage[T5]{fontenc}
% See https://www.latex-project.org/help/documentation/encguide.pdf for other character sets

% This assumes your files are encoded as UTF8
\usepackage[utf8]{inputenc}

% This is not strictly necessary, and may be commented out.
% However, it will improve the layout of the manuscript,
% and will typically save some space.
\usepackage{microtype}

% This is also not strictly necessary, and may be commented out.
% However, it will improve the aesthetics of text in
% the typewriter font.
\usepackage{inconsolata}

% If the title and author information does not fit in the area allocated, uncomment the following
%
%\setlength\titlebox{<dim>}
%
% and set <dim> to something 5cm or larger.

\title{Back Attention: Understanding and Enhancing \\ Multi-Hop Reasoning in Large Language Models}

% Author information can be set in various styles:
% For several authors from the same institution:
% \author{Author 1 \and ... \and Author n \\
%         Address line \\ ... \\ Address line}
% if the names do not fit well on one line use
%         Author 1 \\ {\bf Author 2} \\ ... \\ {\bf Author n} \\
% For authors from different institutions:
% \author{Author 1 \\ Address line \\  ... \\ Address line
%         \And  ... \And
%         Author n \\ Address line \\ ... \\ Address line}
% To start a seperate ``row'' of authors use \AND, as in
 \author{Zeping Yu \\ University of Manchester          \And
         Yonatan Belinkov \\ Technion - IIT, Israel  \And
         Sophia Ananiadou \\ University of Manchester}

\begin{document}
\maketitle
\begin{abstract}
We investigate how large language models perform latent multi-hop reasoning in prompts like ``Wolfgang Amadeus Mozart's mother's spouse is''. To analyze this process, we introduce logit flow, an interpretability method that traces how logits propagate across layers and positions toward the final prediction. Using logit flow, we identify four distinct stages in single-hop knowledge prediction: (A) entity subject enrichment, (B) entity attribute extraction, (C) relation subject enrichment, and (D) relation attribute extraction. Extending this analysis to multi-hop reasoning, we find that failures often stem from the relation attribute extraction stage, where conflicting logits reduce prediction accuracy. To address this, we propose back attention, a novel mechanism that enables lower layers to leverage higher-layer hidden states from different positions during attention computation. With back attention, a 1-layer transformer achieves the performance of a 2-layer transformer. Applied to four LLMs, back attention improves accuracy on five reasoning datasets, demonstrating its effectiveness in enhancing latent multi-hop reasoning ability.
\end{abstract}

\section{Introduction}
Enhancing the multi-hop reasoning capabilities of large language models (LLMs) has become a central research focus in recent studies \cite{openaio1,qi2024mutual,snell2024scaling,luo2024improve}. A widely used approach, chain-of-thought (COT) reasoning \cite{wei2022chain}, improves accuracy by explicitly articulating intermediate reasoning steps. Many studies have expanded on this idea by generating explicit reasoning chains to further enhance performance \cite{zhou2022least,creswell2022selection,shum2023automatic,yao2024tree}. However, these methods often require substantial computational resources due to multiple inference steps or extensive sampling, leading to high costs and deployment challenges, particularly in large-scale or resource-constrained scenarios.

Therefore, enhancing the ability of latent multi-hop reasoning is crucial for reducing the cost. For example, predicting ``Wolfgang Amadeus Mozart's mother's spouse is'' -> ``Leopold'' demonstrates a model’s ability to internally retrieve and integrate relevant knowledge. Recent studies have investigated the mechanisms underlying latent multi-hop reasoning. Given two hops <e1, r1, e2> and <e2, r2, e3>, where ``e'' represents an ``entity'' and ``r'' a ``relation'', \citet{yang2024large} observe that LLMs can sometimes successfully predict queries like ``The r2 of the r1 of e1 is'' \texttt{->} ``e3'' by latently identifying the bridge entity ``e2''. However, \citet{biran2024hopping} find that the accuracy of latent multi-hop reasoning remains low, even when both individual hops are correct. They hypothesize that the low accuracy arises because factual knowledge is primarily stored in the early layers. If the first hop is resolved too late, the later layers may fail to encode the knowledge for subsequent reasoning steps.

Although latent multi-hop reasoning has been explored, its underlying mechanism remains unclear. First, previous studies primarily focus on the format ``The r2 of the r1 of e1 is''. In this format, the e1 position and the last position inherently obtain the information of r1 and r2, making it unsurprising that information flows between them. A more complex format, ``e1's r1's r2 is'', introduces additional challenges. Due to the autoregressive nature of decoder-only LLMs, earlier positions cannot access later tokens, hindering relational knowledge propagation and leading to lower accuracy than ``The r2 of the r1 of e1 is'' prompts. Second, several studies have shown that the higher attention and feed-forward network (FFN) layers also store knowledge \cite{geva2023dissecting,yu2024neuron}, challenging the prevailing hypothesis about multi-hop reasoning mechanisms. Last, how to leverage interpretability insights to enhance reasoning remains uncertain. Previous studies \cite{sakarvadia2023memory, li2024understanding} rely on model editing methods, which may cause potential risks \cite{gu2024model, gupta2024model}.

\begin{figure}[thb]
  \centering
  \includegraphics[width=0.75\columnwidth]{figure1_new.pdf}
  \caption{Four stages in single-hop knowledge prediction. At entity position: (A) entity subject enrichment by FFN neurons; (B) entity attribute extraction by attention neurons. At relation and last positions: (C) relation subject enrichments by FFN neurons; (D) relation attribute extraction by attention neurons and FFN neurons.}
\vspace{-10pt}
\end{figure}

In this study, we focus on addressing these challenges. First, we propose an innovative interpretability analysis method named ``logit flow'', which analyzes how logits propagate across different layers and positions toward the final prediction on neuron-level. We use logit flow and activation patching \cite{wang2022interpretability} to analyze the mechanism of single-hop knowledge prediction. We examine prompts such as ``e1's r1 is'' \texttt{->} ``e2'', where e1 represents an entity (e.g. Mozart), r1 represents a relation (e.g. mother), and e2 is the correct answer (e.g. Maria), which is also an entity. We find four main stages, as shown in Figure 1: (A) entity subject enrichment by FFN neurons at e1 position, (B) entity attribute extraction by attention neurons at e1 position, (C) relation subject enrichment by FFN neurons at r1 and last positions, and (D) relation attribute extraction by attention neurons and FFN neurons at r1 and last positions. The first two stages align with \citet{geva2023dissecting}, where entity-related features are enriched and extracted (``e1'' \texttt{->} ``e1 features''). Our analysis further reveals that the last two stages integrate these enriched entity features with the relation, facilitating the prediction of the final token (``e1 features \& r1'' \texttt{->} ``e2'').

Next, we use logit flow and activation patching to analyze correct cases and false cases in two-hop reasoning queries like ``e1's r1's r2 is'', where the correct answer is ``e3'' and the false answer is ``e2''. In false cases, the relation attribute extraction stage strongly captures r1 position's high layer information. Since this attribution occurs at a later stage than when the model encodes ``e2'' -> ``e2 features'' and ``e2 features \& r2'' -> ``e3'', it reinforces e2 more than e3, ultimately reducing two-hop reasoning accuracy. Based on the interpretability findings, we propose an innovative method named ``back attention'' to enhance the multi-hop ability, which allows lower layers to capture higher hidden states. When trained from scratch on arithmetic tasks, a 1-layer transformer with back attention achieves the accuracy of a 2-layer transformer. When applied to four LLMs, back attention boosts accuracy across five reasoning datasets, highlighting its effectiveness in improving multi-hop reasoning ability.

Overall, our contributions are as follow:

a) We introduce logit flow, an innovative interpretability method that traces how logits propagate across layers and positions. We demonstrate its effectiveness in both single-hop and multi-hop reasoning. Specifically, for single-hop knowledge prediction, we identify four key stages: entity subject enrichment, entity attribute extraction, relation subject enrichment, and relation attribute extraction.

b) We apply logit flow to analyze both correct and incorrect multi-hop reasoning cases. Our findings reveal that failures often stem from the relation attribute extraction stage, where conflicting logits disrupt accurate predictions.

c) We propose back attention, a novel technique that enhances feature capture in lower layers by integrating higher-level information. This method is effective both for training from scratch and for adapting pretrained LLMs.

\section{Experimental Settings}
In Section 3 and 4, we use the TwoHop reasoning dataset \cite{biran2024hopping}. Each data instance contains two hops like <e1, r1, e2> and <e2, r2, e3>, where e1, e2, e3 are entities and r1, r2 are relations. For instance, <Wolfgang Amadeus Mozart, mother, Maria Anna Mozart> and <Maria Anna Mozart, spouse, Leopold Mozart> represent two such hops.

We formulate prompts for first-hop, second-hop, and two-hop queries as ``e1's r1 is'', ``e2's r2 is'', and ``e1's r1's r2 is'', respectively. Following \citet{biran2024hopping}, we remove shortcut cases \cite{ju2024investigating} and retain the instances where both the first-hop and two-hop predictions are correct. Then we exclude ⟨e1, e2, e3⟩ triplets appearing fewer than 30 times, ensuring that the model has sufficient exposure to the retained knowledge types. To prevent excessive data duplication, we limit the number of cases where the correct answer e3 appears more than five times. In Section 3, we analyze 889 cases where the first-hop, second-hop, and two-hop queries are all answered correctly. In Section 4, we focus on 568 cases where e1, e2, and e3 are all human entities. This set includes both correct and incorrect two-hop reasoning cases, enabling a broader evaluation of multi-hop reasoning by comparing successful and failed cases.

\section{Mechanism of Single-Hop Prediction}
In Section 3.1, we introduce the background. In Section 3.2, we introduce the proposed interpretability method ``logit flow''. In Section 3.3, we utilize logit flow method and identify the four stages in single-hop knowledge prediction.

\subsection{Background}

\paragraph{Residual Stream.} To better understand how logit flow captures information propagation in decoder-only LLMs, we first introduce the residual stream \cite{elhage2021mathematical}. Given an input sentence $X=[t_1, t_2, ..., t_T]$ with $T$ tokens, the model processes it through residual connections, ultimately producing the probability distribution $y$ over $B$ tokens in vocabulary $V$ for the next token prediction. Each token $t_i$ at position $i$ is transformed into a word embedding $h_i^0 \in \mathbb{R}^{d}$ by the embedding matrix $E \in \mathbb{R}^{B \times d}$. Next, the word embeddings are taken as the $0th$ layer input and transformed by $L+1$ transformer layers ($0th-Lth$). The output of layer $l$ is the sum of the layer input, the attention layer output $A_i^l$ and the FFN layer output $F_i^l$:
\begin{equation}
h_i^l = h_i^{l-1} + A_i^{l} + F_i^{l}
\end{equation}
The probability distribution $y$ is computed by multiplying $h_T^L$ (the final layer $L$ output at the last position $T$) and the unembedding matrix $E_u \in \mathbb{R}^{B \times d}$.
\begin{equation}
y = softmax(E_u \, h_T^{L})
\end{equation}
The attention layer output $A_i^l$ can be regarded as the sum of vectors on different heads and positions:

\begin{equation}
A_i^l = \sum_{j=1}^H \sum_{p=1}^T \alpha_{i,j,p}^l W^o_{j,l} (W^v_{j,l} h_p^{l-1})
\end{equation}
\begin{equation}
\alpha_{i,j,p}^l = softmax(W^q_{j,l} h_i^{l-1} \cdot W^k_{j,l} h_p^{l-1})
\end{equation}
where $H$ is the head number and $\alpha$ is the attention score. $W^q$, $W^k$, $W^v$, $W^o$ are the query, key, value and output matrices in each attention head.

\paragraph{FFN and attention neurons.} 
Based on the computation of FFN output (Eq.5), \citet{geva2020transformer} find that the FFN output is a weighted sum of neurons, where each neuron's contribution is determined by its learned weights and input interactions:
\begin{equation}
F_i^l = W_{fc2}^l\sigma (W_{fc1}^l (h_i^{l-1}+A_i^l))
\end{equation}
\begin{equation}
F_i^l = \sum_{k=1}^N {m_{i,k}^l fc2_{k}^l}
\end{equation}
\begin{equation}
m_{i,k}^l = \sigma (fc1_k^l \cdot (h_i^{l-1}+A_i^l))
\end{equation}
Here, $fc2_k^l$ is the $kth$ column of the second MLP $W_{fc2}^l \in \mathbb{R}^{d \times N}$. Its coefficient score $m$ is computed by the inner product between the residual output and $fc1_k^l$ (the $kth$ row of the first MLP $W_{fc1}^l \in \mathbb{R}^{N \times d}$). Similarly, in attention mechanisms, neuron activations are influenced by key-value transformations \cite{yu2024neuron}. These activations shape how information is stored and propagated through layers, ultimately influencing the model’s predictions:
\begin{equation}
A_i^l = \sum_{j=1}^H \sum_{p=1}^T \sum_{e=1}^{d/H} \alpha_{i,j,p}^l \beta_{j,p,e}^l wo_{j,e}^l
\end{equation}
\begin{equation}
\beta_{j,p,e}^l = wv_{j,e}^l \cdot h_p^{l-1}
\end{equation}
Here, $wo_{j,e}^l$ is the $eth$ column of $W^o_{j,l}$, whose coefficient score $\alpha\beta$ is computed by the inner product between the layer input $h_p^{l-1}$ and $wv_{j,e}^l$ (the $eth$ row of $W^v_{j,l}$), combined with the attention score $\alpha$. 

In this study, we define: 1) A \textbf{subvalue} as the column of the second MLP ($fc2$ in FFN and $wo$ in the attention head). 2) A \textbf{subkey} as the row of the first MLP ($fc1$ in FFN and $wv$ in the attention head). 3) A \textbf{neuron} as the product of the coefficient score and the subvalue (Eq. 6 and Eq. 8).

\subsection{Logit Flow: Tracing the Logits on Different Layers and Positions}

\paragraph{Identifying important neurons in deep layers.} Many studies \cite{dar2022analyzing,geva2022transformer,wang2022interpretability,katz2023visit,yu2024neuron,nikankin2024arithmetic} find that the layer-level and neuron-level vectors in deep layers store logits related to final predictions. When we say a vector stores logits about $s$, we mean that multiplying this vector with the unembedding matrix results in a high log probability for $s$, where the probability of a vector is obtained by multiplying this vector with the unembedding matrix (replacing $h^L_T$ with this vector in Eq.2) \cite{nostalgebraist2020}.

The final vector $h_T^L$ stores large logits about the prediction $s$. The logit increase, $log(p(s|h_T^L))-log(p(s|h_T^0))$, can be decomposed into contributions from $L \times N$ FFN neurons and $L \times H \times T \times d/H$ attention neurons. To identify the neurons in deep layers, we use the log probability increase \cite{yu2024neuron} as importance score:
\begin{equation}
Imp(v^l) = log(p(s|v^l+h^{l-1})) - log(p(s|h^{l-1}))
\end{equation}
If the importance score $Imp(v^l)$ of a neuron $v^l$ is large, it indicates that adding this neuron on its layer input $h^{l-1}$ significantly enhances the log probability of the final prediction $s$. 

\paragraph{Identifying important neurons in shallow layers.} Although shallow neurons typically do not store logits directly related to the final prediction, they can contribute by amplifying the coefficient scores of deeper neurons. For instance, in Eq.9, $\beta$ is computed by the inner product between the attention subkey $wv$ and the layer input $h^{l-1}$, where the layer input is the sum of the neurons from previous layers in the residual stream at this position. 

To analyze this effect, we compute the inner product between the subkey of the 300 most important attention neurons and each preceding FFN neuron, weighting the result by the importance score of the attention neuron. This approach allows us to identify the most influential shallow FFN neurons. If a shallow FFN neuron has a high summed inner product score, it indicates that this neuron activates multiple important attention neurons, thereby indirectly increasing the logits of the final prediction. Unlike previous studies \cite{yu2024neuron}, we retain the inner product of each FFN neuron at every position, rather than summing the scores across all positions. This method enables us to analyze which specific positions and layers contribute the most to activating attention neurons.

\paragraph{Logit flow: an interpretability method for analyzing the logits in different positions and layers.} After identifying the deep FFN and attention neurons that store the final logits, we compute and visualize the sum of their importance scores across different layers and positions. A large score in a specific layer or position indicates that it stores crucial information related to the final prediction. Additionally, we compute and illustrate the weighted sum of inner products of FFN neurons at each layer and position, revealing which layers and positions play a significant role in activating important attention neurons. This approach allows us to distinguish the layers and positions that contribute to predictions both directly and indirectly.

\subsection{Four Stages in Single-Hop Prediction} We utilize logit flow to analyze 889 first-hop queries (``e1's r1 is'' -> ``e2''). We compute the average scores across all cases using LLama2-7B \cite{touvron2023llama2}. If an entity or relation consists of multiple BPE tokens, we sum the scores of these tokens across their respective positions in each layer. The average scores on each layer and position are illustrated in Figure 2. In this and all subsequent logit flow visualizations, the horizontal axis represents the layers, while the vertical axis represents the positions. Darker colors indicate higher logits at a specific position and layer.

\begin{figure}[thb]
  \centering
  \includegraphics[width=0.8\columnwidth]{firsthop.pdf}
  \caption{Results of logit flow: ``e1's r1 is'' \texttt{->} ``e2''}
\vspace{-10pt}
\end{figure}

The attention neurons storing logits are distributed across the e1, r1, and last positions, with the layers at e1 being lower than those at r1 and the last position. Similarly, FFN neurons with large inner products are also concentrated at e1, r1, and the last positions, but they generally appear just before the average layers of the attention neurons. The stages at entity position align with the layer-level conclusions in \citet{geva2023dissecting}, where FFN features are activated by the entity's word embeddings and subsequently processed by attention layers.

Additionally, we find that subject enrichment and attribute extraction occur not only at entity position but also at relation and last positions. Due to the autoregressive nature of decoder-only LLMs, the mechanisms at the entity position and r1/last positions differ. At entity position, lower-layer FFN and attention neurons encode knowledge about ``e1 -> e1 features''. In contrast, at the relation and last positions, deeper FFN and attention neurons store knowledge of ``e1 features \& r1 -> e2''. For example, consider ``Mozart's mother is -> Maria'' and ``Mozart's father is -> Leopold''. The hidden states at the position of ``Mozart's'' are identical in both cases, meaning these positions cannot directly determine whether the final prediction is ``Maria'' or ``Leopold''. Instead, at the entity position, lower layers extract Mozart's features containing both ``Maria'' and ``Leopold''. At the relation and last positions, deeper layers refine this information, encoding ``Mozart's features \& mother -> Maria'' and ``Mozart's features \& father -> Leopold'', which enables the model to generate the correct prediction. To verify this, we compute the average logit difference of each layer's hidden state between the correct answer (e.g. Maria) and the conflicting answer (e.g. Leopold) at entity, relation and last positions across all correct human->human cases. The results align with our analysis, detailed in Appendix A. The entity position cannot distinguish the correct answer and the conflicting answer, while the relation and last positions' logit difference start to increase after the entity attribute extraction stage.

We also analyze the logit flow of 889 second-hop cases ``e2's r2 is'' \texttt{->} ``e3'', detailed in Appendix B. Similar to the first-hop results, we observe the same four stages in the second-hop predictions, further validating the single-hop prediction mechanism. In addition, we utilize the activation patching \cite{wang2022interpretability} method to analyze the layer-level information flow, as presented in Appendix C, also observing the importance in entity, relation and last positions. Compared to the layer-level approach, our method provides a neuron-level perspective on information flow, offering a more granular and detailed understanding.

\section{Mechanism of Two-Hop Prediction}
\citet{biran2024hopping} find that the two-hop accuracy remains low, even when both the first-hop and second-hop queries are correct. In this section, we investigate the cause of this phenomenon. We focus on the prompt like ``e1's r1's r2 is'', where the correct answer is ``e3''. We use the logit flow method to analyze the 889 correct two-hop queries, as shown in appendix D. We find that the importance of attention neurons at relation positions is significantly lower than that in single-hop queries. Based on this observation, we hypothesize that the model may incorrectly predict the entity corresponding to ``e1's r1'' or ``e1's r2'' instead of ``e3''. This interference could lead the model to favor intermediate entities over the correct final answer, ultimately reducing the accuracy of two-hop reasoning. 

To verify this, we analyze 568 human->human->human cases with the prompt ``e1's r1's r2 is'' and the correct answer ``e3'' in Llama2-7B, where e1, e2, e3 are all human entities. We compare the ranking of the correct answer ``e3'' against two conflicting answers: ``e1's r1'' and ``e1's r2''. For example, for ``Mozart's mother's spouse is'', the correct answer is ``Leopold'', and the conflicting answers are ``Maria'' (Mozart's mother) and ``Constanze'' (Mozart's spouse). Among 568 cases, 52.3\% correctly predict ``e3'', 42.4\% predict ``e2'' (the answer of ``e1's r1''), and 5.3\% predict the answer of ``e1's r2''. This indicates that the conflicting entities can cause the accuracy decrease. 

\begin{figure}[htb]
  \centering
  \includegraphics[width=0.88\columnwidth]{twohop-human.pdf}
  \caption{Results of logit flow on correct and false human->human->human cases in Llama2-7B.}
\vspace{-10pt}
\end{figure}

To further investigate this phenomenon, we use the logit flow method to compare correct cases (where the predicted answer is ``e3'') with false cases (where the predicted answer is ``e2''), as shown in Figure 3. We observe that in the false cases, the influence at the r1 position is significantly stronger. The results of activation patching (Appendix E) and Llama3.1-8B \& Llama3.2-3B (Appendix F) reveal a similar trend. This finding appears counterintuitive—why does the model predict the wrong answer when it relies more heavily on the features at the r1 position?

A closer look at the single-hop analysis provides an explanation. In the case of ``e1's r1 is'', the high layers at the r1 position store logits related to ``e2''. Due to the autoregressive nature of decoder-only LLMs, the hidden states at r1 position remain the same in both ``e1's r1 is'' and ``e1's r1's r2 is''. Consequently, when the high-layer information at the r1 position is extracted in ``e1's r1's r2 is'', it inadvertently reinforces the probability of ``e2'', leading to lower accuracy in two-hop reasoning.

This phenomenon can also be understood through the four stages of knowledge storage. In the single-hop analysis (Figure 2), the knowledge of ``e1 -> e1 features'' and ``e2 -> e2 features'' is stored in lower layers (layers 7–20), whereas the knowledge of ``e1 features \& r1 -> e2'' and ``e2 features \& r2 -> e3'' is stored in deeper layers (layers 20–31). In two-hop false cases (Figure 3), when the features at r1 positions, which are related to e2, are extracted at layer 28, they only activate the ``e2 features \& r2 -> e3'' parameters in layers 28–31. Although this process does enhance the probability of e3, it amplifies the probability of e2 even more. This imbalance leads to the model predicting e2 instead of e3, resulting in lower accuracy for two-hop reasoning. From this perspective, our results partially align with the "hopping too late" hypothesis \cite{biran2024hopping}. However, our findings reveal a key difference: while some parameters encoding "e2 \& r2 -> e3" are still activated, their contribution is weaker compared to the direct influence of ``e2''.

\section{Back Attention: Letting Lower Layers Capture Higher-Layer features}
Based on the single-hop mechanism, if we can restore the r1 position's deep layer features back to later positions' shallow layers, the parameters storing ``e2 -> e2 features'' and ``e2 features \& r2 -> e3'' can be activated, thereby strengthening the competitiveness of the correct answer. Motivated by this, we propose an innovative technique, ``back attention'', to allow the lower layers capture higher features. The computations of the original attention output $A$ and the back attention output $B$ are shown in Eq. 11-12. In the original attention computation, the query, key, and value vectors are computed by the hidden states $\mathbf{h}$ on the same layer:
\begin{equation}
\begin{small}
\text{A} = \text{Softmax} \left( \frac{\mathbf{h} \mathbf{W}^q (\mathbf{h} \mathbf{W}^k)^\top}{\sqrt{d'}} \right) (\mathbf{h} \mathbf{W}^v) \mathbf{W}^{o}.
\end{small}
\end{equation}
In contrast, back attention modifies this mechanism by computing queries from a lower source layer $\mathbf{hs}$ while obtaining keys and values from a target layer $\mathbf{ht}$, which are the hidden states on a higher layer or the stack of all higher layers' hidden states. This adjustment allows a lower layer to capture richer representations stored in higher layers:
\begin{equation}
\begin{small}
\text{B} = \text{Softmax} \left( \frac{\mathbf{hs} \mathbf{W}^q_{B} (\mathbf{ht} \mathbf{W}^k_{B})^\top}{\sqrt{d'}} \right) (\mathbf{ht} \mathbf{W}^v_{B}) \mathbf{W}^{o}_{B}.
\end{small}
\end{equation}

\begin{figure}[thb]
  \centering
  \includegraphics[width=0.75\columnwidth]{backattention.pdf}
  \caption{Back attention on a 1-layer transformer.}
\vspace{-10pt}
\end{figure}

Figure 4 illustrates how back attention is integrated into a single-layer transformer. Back attention occurs after the original inference pass, during which the hidden states of all layers and positions are calculated. The query vector is computed from the 0th layer input ($\mathbf{hs}$), while the key and value vectors are computed from the 0th layer output ($\mathbf{ht}$). Then the back attention output $B$ is added back onto the 0th layer input, and recompute the forward pass again. Notably, Figure 4 only shows the back attention computation at the last position, while similar computation happens at all positions on the 0th layer input. Back attention restores high-layer features at different positions using the back attention scores. If the back attention score is 1.0 at r1 position and 0.0 at other positions, it means that the r1 position's 0th layer output is added at the last position's 0th layer input. The results when adding back attention in training and fine-tuning stages are as follows.

\paragraph{Training from scratch: back attention enhances the ability of 1-layer transformer.} We conduct experiments on a 2-digit addition arithmetic dataset. In each training and testing set, there are 12,150 single-sum cases (``a+b=''), and 6,188 double-sum cases (``c+d+e=''), where ``a'', ``b'', ``c'', ``d'', and ``e'' are integers ranging from 0 to 99. The model needs to ``memorize'' the single-sum cases and ``learn'' the double-sum patterns. We utilize the Llama tokenizer, representing each digit as a separate token (e.g., 12 is tokenized as [``1'', ``2'']), ensuring that each token appears sufficiently during training. We compare the results of a 1-layer transformer, a 2-layer transformer, and a 1-layer transformer with back attention. In all models, the dimension is 440 for attention/FFN layers, and 160 for back attention. We use the AdamW optimizer \cite{loshchilov2017decoupled} with a learning rate of 0.0001, a batch size of 64, and a maximum of 500 epochs. 

The accuracy of 1-layer transformer, 1-layer transformer with attention, and 2-layer transformer are 83.8\%, 93.8\%, and 92.5\%, respectively. The details of loss and accuracy are shown in Appendix G. The 2-layer transformer and the 1-layer transformer with back attention converge faster than the 1-layer transformer. Notably, the 1-layer transformer with back attention requires only 56.7\% of the parameters of the 2-layer transformer. Therefore, incorporating back attention during the training stage can significantly enhance the model's performance while reducing parameter requirements.

\begin{figure}[thb]
  \centering
  \includegraphics[width=0.99\columnwidth]{layeracc.pdf}
  \caption{Test accuracy of back attention on each layer.}
\vspace{-10pt}
\end{figure}

\paragraph{Adding back attention in pre-trained LLMs: back attention increases the reasoning accuracy.} Back attention can also be integrated into a pre-trained LLM, using all higher-layer states to compute the keys and values. For instance, if back attention is added to the 6th layer, the keys and values are calculated using the layer outputs of all positions from the 6th layer to the final layer. We add back attention on each layer in Llama-7B \cite{touvron2023llama}, fine-tuning on the double-sum arithmetic cases (the same in the previous section). Figure 5 shows the accuracy when fine-tuning back attention on each layer (freezing LLM parameters), where the original accuracy is 67.1\%. The accuracy across the 0-5 layers exhibits significant fluctuation. Adding back attention to the 6th layer achieves a peak accuracy of 93.2\%, followed by a steady decline compared with higher layers.

\begin{table}[htb]
\centering
\begin{small}
%\begin{sc}
\begin{tabular}{cccccc}
\toprule
  & 1DC & SVAMP & MA & TwoHop & SQA \\
\midrule
Llama3 & 72.7 & 55.7 & 21.1 & 11.5 & 65.1 \\
+backattn & 97.0 & 69.3 & 88.9 & 47.8 & 86.2 \\
\midrule
Llama3.1 & 74.6 & 56.0 & 30.0 & 8.8 & 65.4 \\
+backattn & 98.5 & 70.7 & 86.2 & 42.7 & 87.0 \\
\midrule
Llama3.2 & 49.3 & 44.3 & 15.0 & 6.5 & 62.0 \\
+backattn & 92.9 & 62.0 & 52.8 & 37.0 & 86.3 \\
\midrule
Mistral & 51.9 & 63.0 & 26.1 & 8.8 & 71.5 \\
+backattn & 87.4 & 71.7 & 47.2 & 40.1 & 87.8 \\
\bottomrule
\end{tabular}
%\end{sc}
\end{small}
\caption{Accuracy (\%) on 5 datasets before/after adding back attention on 6th layer in four LLMs.}
\vspace{-10pt}
\end{table}

Then we do experiments on 5 arithmetic and reasoning datasets 1-Digit-Composite (1DC) \cite{brown2020language}, SVAMP \cite{patel2021nlp}, MultiArith (MA) \cite{roy2016solving}, TwoHop \cite{biran2024hopping}, and StrategyQA (SQA) \cite{geva2021did}. We fine-tune back attention on the 6th layer in Llama3-8B \cite{meta2024introducing}, Llama3.1-8B \cite{dubey2024llama}, Llama3.2-3B \cite{meta2024llama}, and Mistral-7B \cite{jiang2023mistral}. The accuracy is shown in Table 1. On all LLMs, adding back attention achieves a significant accuracy increase. 

\begin{figure}[thb]
  \centering
  \includegraphics[width=0.9\columnwidth]{casestudy.pdf}
  \caption{Back attention scores at all positions and higher layers when adding on the 6th layer.}
\vspace{-10pt}
\end{figure}

To evaluate whether back attention functions as intended, we analyze the case ``Mozart's mother's spouse is'' \texttt{->} ``Leopold'' in TwoHop dataset and visualize the back attention scores (darker larger) in Figure 6. Back attention effectively learns to recover ``mother'' position's 27-30 layers' hidden states into the last position's 6th layer. This visualization proves that back attention successfully propagates high-layer information from important positions to lower layers, enabling the model to better utilize knowledge for accurate predictions.

\paragraph{Advantages of back attention.} First, back attention can be incorporated during fine-tuning, enabling flexible enhancement of a powerful pre-trained LLM without retraining the model from scratch. Second, the back attention parameters are remarkably lightweight compared to those of pre-trained LLMs. For instance, the back attention parameters account for just 0.002\% of LLama3's 8 billion parameters. Third, back attention exhibits substantial potential, increasing the average accuracy from 46.9\% to 77.0\% in Llama3.1-8B, even when applied to a single layer. Finally, the visualization of back attention scores (Figure 6) serves as an interpretability tool, offering insights into which positions are most critical for a given task, thereby improving our understanding of the mechanisms.

\section{Related Work}
\subsection{Multi-Hop Reasoning in LLMs}
Improving the reasoning ability of LLMs has become a key focus of recent research \cite{lightman2023let,huang2023large,li2024chain,wang2024chain}. \citet{wei2022chain} use chain-of-thought to enhance the reasoning ability by articulating intermediate steps. \citet{fu2022complexity} propose complexity-based prompting, showing that selecting and generating reasoning chains with higher complexity significantly improves reasoning accuracy. \citet{wang2022self} combine chain-of-thought with the self-consistency decoding strategy, achieving significant improvements by sampling diverse reasoning paths and selecting the most consistent answer. \citet{chen2024self} propose self-play fine-tuning, which enhances LLMs' reasoning abilities by refining their outputs through self-generated data, thereby reducing reliance on human-annotated datasets. \citet{brown2024large} propose scaling inference compute by increasing the number of generated samples, demonstrating significant improvements across tasks like coding and math. \citet{hao2023reasoning,yao2024tree} use tree-based methods to improve the performance.

\subsection{Mechanistic Interpretability}
Mechanistic interpretability \cite{Chris2022} aims to reverse engineer the internal mechanisms of LLMs. Logit lens \cite{nostalgebraist2020} is a widely used method \cite{dar2022analyzing,katz2023visit,yu2024interpreting} to analyze the information of hidden states, by multiplying the vectors with the unembedding matrix. A commonly used localization method is causal mediation analysis \cite{vig2020investigating,meng2022locating,stolfo2023mechanistic,geva2023dissecting}, whose core idea is to compute the change of the output when modifying a hidden state. Another types of studies focus on constructing the circuit in the model \cite{olsson2022context,zhang2023towards,gould2023successor,hanna2024does,wang2022interpretability}. Due to the superposition phenomenon \cite{elhage2022toy,scherlis2022polysemanticity,bricken2023towards}, sparse auto-encoder (SAE) is useful for interpreting the features \cite{gao2024scaling,templeton2024scaling,cunningham2023sparse}. A useful characteristic is the residual stream \cite{elhage2021mathematical}, revealing that the final embedding can be represented as the sum of layer outputs. Furthermore, \citet{geva2020transformer,geva2022transformer} find that the FFN output is the weighted sum of FFN neurons. \citet{yu2024neuron} find that the attention head outputs can also be regarded as the weighted sum of attention neurons. 

While previous neuron-level studies primarily focus on ``localization''—identifying which neurons are important—they often lack a deeper ``analysis'' of how these neurons influence predictions. By applying our logit flow method, we gain a clearer understanding of how neurons are activated and contribute to the final prediction.

\section{Conclusion}
We investigate the mechanisms of latent multi-hop reasoning in LLMs and identify key factors affecting the accuracy. Through our interpretability method logit flow, we uncover four distinct stages in single-hop knowledge prediction: entity subject enrichment, entity attribute extraction, relation subject enrichment, and relation attribute extraction. Analyzing two-hop queries, we find that failures often arise in the relation attribute extraction stage, where conflicting logits lower prediction accuracy. To address this, we propose back attention, a novel method that enables lower layers to access higher-layer hidden states, effectively restoring important features. Back attention significantly enhances reasoning performance, allowing a 1-layer transformer to match the accuracy of a 2-layer transformer. When applied to pre-trained LLMs, it improves accuracy across five datasets and four models, demonstrating its effectiveness in multi-hop reasoning. Overall, our analysis provides new insights and introduces a powerful approach for improving reasoning accuracy in LLMs.

\clearpage
\section{Limitations}
In this study, the interpretability analysis primarily focuses on single-hop and two-hop knowledge queries, which represent specific reasoning scenarios. While these cases provide valuable insights, it is important to acknowledge that other types of reasoning tasks might involve different mechanisms not captured in our analysis. Despite these constraints, the observed performance improvements across a variety of reasoning tasks and LLMs suggest that the proposed back attention method and the derived insights possess a degree of general applicability. Further investigations will be needed to validate these findings on more diverse reasoning tasks and refine the interpretability framework for broader applicability.

In this work, back attention is applied to only a single layer, where it has demonstrated promising results. Nevertheless, back attention can also be extended to two or more layers, potentially yielding even greater improvements. We view the success of the single-layer application as a foundational step, paving the way for future research aimed at exploring and optimizing back attention in more complex and multi-layer configurations.

\bibliography{custom}
\bibliographystyle{acl_natbib}

\clearpage
\appendix

\section{Logit Difference at Different Positions}
\begin{figure}[thb]
  \centering
  \includegraphics[width=0.88\columnwidth]{logitdiff.pdf}
  \caption{Logit difference at entity, relation and last positions on human->human cases in Llama2-7B. The logit difference is small at entity position, but large on relation and last positions' deep layers.}
\vspace{-10pt}
\end{figure}

We compute the average logit difference at entity, relation and last positions across all correct human -> human cases, shown in Figure 7. Take ``Mozart's mother is -> Maria'' as an example. We compute the logit difference between ``Maria'' and ``Leopold'' (Mozart's father). At the entity position, the logit difference is small on all layers. At the relation and last positions, the logit difference increases sharply after the entity subject enrichment and entity attribute extraction stages (layers 19–20). This indicates that the entity position primarily extracts general features of ``Mozart'', including information relevant to both ``Maria'' and ``Leopold''. In contrast, the deeper layers at the relation and last positions encode specific knowledge, such as ``Mozart's features \& mother -> Maria'' and ``Mozart's features \& father -> Leopold'', which ultimately differentiate the correct prediction.

\section{Results of Logit Flow on Second-Hop Queries in Llama2-7B}
\begin{figure}[thb]
  \centering
  \includegraphics[width=0.8\columnwidth]{secondhop.pdf}
  \caption{Results of logit flow on second-hop queries ``e2's r2 is'' \texttt{->} ``e3'' in Llama2-7B. There are four similar stages with the first-hop queries: (A) entity subject enrichment, (B) entity attribute extraction, (C) relation subject enrichment, and (D) relation subject extraction.}
\vspace{-10pt}
\end{figure}

The results of logit flow on second-hop queries ``e2's r2 is'' \texttt{->} ``e3'' are shown in Figure 8. There are also four stages existing in the second-hop queries, similar to those in the first-hop queries (Figure 2).

\section{Results of Activation Patching on Single-Hop Queries in Llama2-7B}
The results of activation patching on single-hop queries are shown in Figure 9, using the pyvene \cite{wu2024pyvene} and NNsight \cite{fiotto2024nnsight} libraries. Compared to the logit flow results (Figure 2), the entity and last positions exhibit higher importance, while the relation position appears less significant. This difference arises because activation patching aggregates the importance of both FFN and attention modules into a single visualization. In contrast, the logit flow method distinguishes and separately visualizes the importance of FFN and attention neurons, offering a more granular, neuron-level understanding of the information flow.

\begin{figure}[thb]
  \centering
  \includegraphics[width=0.88\columnwidth]{activationpatching.pdf}
  \caption{Results of activation patching on single-hop queries in Llama2-7B. Similar to logit flow (but not as obvious as logit flow), there is also importance on r1 position's high layers.}
\vspace{-10pt}
\end{figure}

\section{Results of Logit Flow on Two-Hop Queries in Llama2-7B}
\begin{figure}[thb]
  \centering
  \includegraphics[width=0.8\columnwidth]{twohop.pdf}
  \caption{Results of logit flow on two-hop queries ``e1's r1's r2 is'' \texttt{->} ``e3''. The importance of relation positions (r1 and r2) is lower than single-hop queries.}
\vspace{-10pt}
\end{figure}

The results of logit flow on the two-hop queries ``e1's r1's r2 is'' \texttt{->} ``e3'' are shown in Figure 10. Compared to the logit flow results on single-hop queries (Figure 2), the importance of relation positions is significantly lower. This suggests that e1's features at the e1 position are primarily extracted into the last position, potentially activating the parameters associated with ``e1's r1'', ``e1's r2'', and ``e1's r1's r2''. This motivates our exploration between the correct and false human->human->human cases in Section 4.

\section{Results of Activation Patching on Correct and False Two-Hop Queries in Llama2-7B}

The results of activation patching on correct and false human->human->human cases in Llama2-7B are shown in Figure 11. Compared with the correct cases, the false cases show a much clearer influence at r1 position's high layers. This trend is similar to the findings of logit flow method (Figure 3), indicating that the r1 position's high features increase the probability of ``e2'', thereby reducing the accuracy of two-hop reasoning.

\begin{figure}[thb]
  \centering
  \includegraphics[width=0.88\columnwidth]{twohop-human-activationpatching.pdf}
  \caption{Results of activation patching on correct and false human->human->human cases in Llama2-7B. The importance of r1 position is 1.66\% in correct cases and 5.43\% in false cases.}
\vspace{-10pt}
\end{figure}

\section{Results of Logit Flow and Activation Patching on Correct and False Two-Hop Queries in Llama3.1-8B and Llama3.2-3B}

\begin{figure}[thb]
  \centering
  \includegraphics[width=0.88\columnwidth]{twohop-human-llama3.pdf}
  \caption{Results of logit flow on correct and false human->human->human cases in Llama3.1-8B. The importance of r1 position is 6.38\% in correct cases and 32.18\% in false cases.}
\vspace{-10pt}
\end{figure}

\begin{figure}[thb]
  \centering
  \includegraphics[width=0.88\columnwidth]{twohop-human-llama3-activationpatching.pdf}
  \caption{Results of activation patching on correct and false human->human->human cases in Llama3.1-8B. The importance of r1 position is 4.98\% in correct cases and 18.00\% in false cases.}
\vspace{-10pt}
\end{figure}

\begin{figure}[thb]
  \centering
  \includegraphics[width=0.88\columnwidth]{twohop-human-llama3.2.pdf}
  \caption{Results of logit flow on correct and false human->human->human cases in Llama3.2-3B. The importance of r1 position is 17.50\% in correct cases and 40.36\% in false cases.}
\vspace{-10pt}
\end{figure}

\begin{figure}[thb]
  \centering
  \includegraphics[width=0.88\columnwidth]{twohop-human-llama3.2-activationpatching.pdf}
  \caption{Results of activation patching on correct and false human->human->human cases in Llama3.2-3B. The importance of r1 position is 11.23\% in correct cases and 21.52\% in false cases.}
\vspace{-10pt}
\end{figure}

The comparison of correct and false human->human->human cases in Llama3.1-8B are shown in Figure 12 (results of logit flow) and Figure 13 (results of activation patching). Similar results of Llama3.2-3B are shown in Figure 14 (results of logit flow) and Figure 15 (results of activation patching). In both methods and models, the impact of r1 position's high layers in the false cases are larger than that in the correct cases. These results show similar trends with the results of Llama2-7B.

\section{Loss and Accuracy of Back Attention on 1-Layer Transformer}

The loss and accuracy of 1-layer transformer, 1-layer transformer with back attention, and 2-layer transformer are shown in Figure 16. The performance of 1-layer transformer with 2-layer transformer is similar, much better than that of 1-layer transformer.

\begin{figure}[thb]
  \centering
  \includegraphics[width=0.97\columnwidth]{arithmetic-experiment.pdf}
  \caption{Loss (left) and accuracy (right) on arithmetic dataset of 1-layer transformer, 1-layer transformer with back attention, and 2-layer transformer.}
\vspace{-10pt}
\end{figure}


\end{document}

\bibliographystyle{acl_natbib}

\clearpage
\appendix
\newpage
\section{Experimental Setups}\label{sec:app_setup}
\subsection{Inference Setting} \label{sec:app_inference_setting}
The datasets are used under the MIT License and with their intended use.
For models, we use \texttt{LLaMA} checkpoints from Hugging Face Transformers \cite{wolf2020transformers}.
We run experiments with up to $8 \times$ RTX 4090 24G GPUs. e.
Due to memory constraints, we use Float16 precision for inference, with each run taking around 1\textasciitilde4 hours, depending on the model and data size. 
The prompt template for GSM8K and EZstance are in Appendix Table \ref{tab:app_prompt}.
For ICL, we set the number of demonstrations to $8$ following \cite{li2023unified,rubin2021learning}.


\begin{table}[h]
\small
\centering
\begin{tabular}{@{}p{1.5cm}p{6cm}@{}}
\toprule
\textbf{Dataset}                   & \multicolumn{1}{c}{\textbf{Prompt Template}}                                                                    \\ \midrule
\multirow{2}{*}{\textbf{GSM8K}}    & \texttt{Question:} \colorbox{yellow}{\texttt{\{math\_problem\}}}                                                                            \\
                                   & \texttt{Answer:} \colorbox{yellow}{\texttt{\{step\_by\_step\_answer\}.}}                                                                    \\ \midrule
\multirow{3}{*}{\textbf{EZStance}} & \texttt{Text:} \colorbox{yellow}{\texttt{\{sentence\}}}                                                                                           \\
                                   & \texttt{Question: Which stance-"favor," "against," or "neutral"-does the above text express toward} \colorbox{yellow}{\texttt{\{target\}}}? \\
                                   & \texttt{Answer:} \colorbox{yellow}{\texttt{\{stance\}}}.                                                                                    \\ \bottomrule
\end{tabular}
\caption{Prompt templates for the two datasets. \colorbox{yellow}{highlighted} parts are inputs.} \label{tab:app_prompt}
\end{table}

\subsection{Implementation Details of IRT} \label{sec:app_irt_implementation}
\noindent \textbf{Dataset Construction.}
The dataset for the IRT model is built upon LLM outputs. 
First, we construct Oracle demonstrations using the approach described in \cref{sec: measure}. 
Then, we run LLMs using prompts in Appendix Table \ref{tab:app_prompt} 
 in different settings (DP or ICL). 
The outputs are represented as tuples consisting of $<$\texttt{model\_id}, \texttt{example\_id}, \texttt{input}, \texttt{output}, \texttt{setting}, \texttt{label}$>$.  
This results in total $2$ (Direct prompting or ICL setting) $\times$ $M$ (Number of LLMs) $\times$ $N$ (Number of queries) instances, where $M=8$, $N_{\rm {\tiny GSM8K}}=1319, N_{\rm {\tiny EZStance}}=6703$.  
We further split them into 80\% training set, 10\% validation set, and 10\% test set.


\noindent \textbf{Training Setup.}
We set the dimension of latent traits $\boldsymbol{\theta}, \boldsymbol{\alpha}, \boldsymbol{\theta}^{c},$ $\boldsymbol{\alpha}^{c}$ to 32.
Queries are encoded with SBERT (\texttt{paraphrase-mpnet-base-v2}) with an embedding size of 768.
We train all the models for 10 epochs with a learning rate of $2e-4$ and batch size of 16. 
Traditionally, IRT is optimized by marginalized maximum likelihood estimation \cite{chalmers2012mirt}.
However, this does not scale well to large datasets \cite{lalor2023py}.
We follow \citet{gor2024great} to use Adam \cite{kingma2014adam} to optimize our model.
The best model is selected based on the performance on the validation set.


\subsection{Details of Fine-tuning} \label{sec:app_finetune}
We fine-tune \texttt{LLaMA-3-8B-Instruct} to evaluate our curriculum learning algorithm (\cref{App2}).
Since \texttt{LLaMA} models might already be fine-tuned on the training set of GSM8K \cite{zhang2024careful}, we randomly sample 1,000 instances from the test set for fine-tuning and use the remaining 319 instances for evaluation.
The EZStance dataset is curated after the release of \texttt{LLaMA-3} and, therefore, has no such concern.
We sample 5,000 examples from the training set for fine-tuning and directly evaluate the model on the test set. 
With the scheduler in Algorithm \ref{algo:cl_scheduler}, we split the dataset into $3$ buckets and fine-tune the model on each bucket for $2$ epochs with a learning rate of $1e-5$ and batch size of $4$. 


\section{Additional Results}\label{sec:app_add_results}
\subsection{Additional Results of IRT}
The accuracy of IRT models is in Table \ref{tab:app_irt_acc}. 
Note that baseline models are not trained on ICL data and therefore their accuracy is not indicative. 
We report it only for the completeness of the results. 
\begin{table}[h]
\setlength\tabcolsep{4pt}
\small
\centering
\begin{tabular}{@{}lcccccc@{}}
\toprule
\multirow{2}{*}{\textbf{Model}} & \multicolumn{3}{c}{\textbf{GSM8K}} & \multicolumn{3}{c}{\textbf{EZStance}} \\ \cmidrule(l){2-7} 
                                & DP      & ICL      & Overall     & DP       & ICL       & Overall      \\ \midrule
\textsc{Irt}$_{\rm 1Pl}$                       & 69.1      & 39.6     & 56.7        &   76.1         &   45.6        &   63.1           \\
\textsc{Irt}$_{\rm 2PL}$                       & 70.4      & 40.2     & 58.7        &   75.3         &  46.4         &   63.0           \\
MIRT                            & 68.9      & 47.2          & 59.8             & 76.6           & 45.9          & 63.5            \\ 
MIRT$_{\textsc{ICL}}$                            & \textbf{77.4}      & \textbf{78.4}     & \textbf{77.9}        &    \textbf{77.0}       & \textbf{68.5}          &   \textbf{72.8}           \\ \bottomrule
\end{tabular}
\caption{Performance (Accuracy $\%$ ) of various IRT models. The best results are in \textbf{bold}.} \label{tab:app_irt_acc}
\end{table}

\subsection{Additional Results of \textsc{SelIcl}}
The result of \textsc{SelIcl} on the EZStance dataset is in Appendix Figure \ref{fig:ezstance_sel_icl}.

\begin{figure*}[h]
    \centering
    \includegraphics[width=\textwidth]{figures/ez_stance_sel_icl.png}
    \caption{Results of \textsc{SelIcl} on EZStance. See detailed explanations in Figure \ref{fig:gsm8k_sel_icl}.}
    \label{fig:ezstance_sel_icl}
\end{figure*}

\end{document}
