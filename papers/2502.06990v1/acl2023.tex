% This must be in the first 5 lines to tell arXiv to use pdfLaTeX, which is strongly recommended.
\pdfoutput=1
% In particular, the hyperref package requires pdfLaTeX in order to break URLs across lines.

\documentclass[11pt]{article}

% Remove the "review" option to generate the final version.
\usepackage{ACL2023}

% Standard package includes
\usepackage{times}
\usepackage{latexsym}


\usepackage{amsmath,amssymb}
\usepackage{subcaption}
\usepackage{graphicx}
\usepackage{float}
\usepackage{stfloats}
\usepackage{fdsymbol}
\usepackage{pifont}
\usepackage{xcolor}
\usepackage{cleveref}
\usepackage{adjustbox}
\usepackage{diagbox}
\usepackage{booktabs}
\usepackage{array}
\usepackage{booktabs}
\usepackage{multirow}
\usepackage{mdframed}
\usepackage{enumitem}
\usepackage{tabularx}
\usepackage{algorithm}
\usepackage{algorithmicx}  
\usepackage{algpseudocode}  
\usepackage{diagbox}
\crefname{section}{§}{§§}
\Crefname{section}{§}{§§}
\usepackage{pifont}
\usepackage{tikz}
\usepackage{xspace}
\usepackage{expl3}
\def\TODO#1{\todo[color=TodoColor,size=small,inline]{TODO: #1}}

\usepackage{todonotes}
\newcommand{\thought}[1]{{\color[rgb]{0.2,0.39,0.66}(#1)}}
\newcommand{\todo}[1]{{\color[rgb]{1.0,0.0,0.0}(#1)}}
\newcommand{\hsh}[1]{{\color{green!50!black} Henrik: #1}}
\newcommand{\st}[1]{{\color{red!50!black} Sebastian: #1}}

\newcommand{\ulm}[1]{_{\scaleto{\mathrm{#1}}{3pt}}}
\newcommand\at[2]{\left.#1\right|_{#2}}











\newtheorem{assumption}{Assumption}

\DeclareMathOperator*{\argmax}{arg\,max}
\DeclareMathOperator*{\argmin}{arg\,min}

\newcommand{\swname}[1]{\texttt{#1}}
\newcommand{\ie}{i\/.\/e\/.,\/~}
\newcommand{\eg}{e\/.\/g\/.,\/~}
\newcommand{\cf}{cf\/.\/~}

\newcommand{\fig}{Fig\/.\/~}
\newcommand{\defn}{Def\/.\/~}
\newcommand{\sect}{Sec\/.\/~}
\newcommand{\tabl}{Tab\/.\/~}
\newcommand{\algo}{Algorithm~}
\newcommand{\theo}{Theorem~}

\newcommand{\bnnl}{3 hidden layers}
\newcommand{\bnnn}{50 neurons}
\newcommand{\bnna}{tanh activations}

\newcommand{\capt}[1]{\mdseries{\emph{#1}}}

\newcommand{\videolink}{at \url{https://youtu.be/_d7AqTRjz6g}}
\newcommand{\codelink}{\url{https://github.com/wheelbot/mini-wheelbot}}

\newcommand{\fakepar}[1]{\vspace{0mm}\noindent\textbf{#1.}}

\newcommand{\needref}{\textcolor{red}{[REF]}}

\newcommand{\plotfontsize}{9pt}


\newcommand{\mrinmaya}[1]{\textcolor{blue}{[mrinmaya: #1]}}


\usepackage{setspace}
\renewcommand{\algorithmicrequire}{ \textbf{Input:}}  
\renewcommand{\algorithmicensure}{ \textbf{Output:}}   

% For proper rendering and hyphenation of words containing Latin characters (including in bib files)
\usepackage[T1]{fontenc}
% For Vietnamese characters
% \usepackage[T5]{fontenc}
% See https://www.latex-project.org/help/documentation/encguide.pdf for other character sets

% This assumes your files are encoded as UTF8
\usepackage[utf8]{inputenc}

% This is not strictly necessary, and may be commented out.
% However, it will improve the layout of the manuscript,
% and will typically save some space.
\usepackage{microtype}

% This is also not strictly necessary, and may be commented out.
% However, it will improve the aesthetics of text in
% the typewriter font.
\usepackage{inconsolata}

\newcommand{\zoneicon}[1]{%
  \begin{tikzpicture}
    \fill[#1] (0,0) circle (0.13cm);
  \end{tikzpicture}%
}
\newcommand{\zone}[2]{%
  \begin{tikzpicture}
    \fill[#1] (0,0) -- ++(90:0.13cm) arc (90:270:0.13cm) -- cycle;
    \fill[#2] (0,0) -- ++(-90:0.13cm) arc (-90:90:0.13cm) -- cycle;
    % \draw (0,0) circle (0.15cm);
  \end{tikzpicture}%
}
\definecolor{z_red}{rgb}{0.8, 0.25, 0.25}
\definecolor{z_green}{rgb}{0.18, 0.55, 0.34}
\definecolor{z_orange}{rgb}{0.93, 0.57, 0.13}
\definecolor{z_blue}{rgb}{0.16, 0.32, 0.75}

\newcommand{\zonex}{\xspace \zoneicon{z_green}{$\mathcal{Z}_{\text{\ding{51}}}$}\xspace}
\newcommand{\zoney}{\xspace \zoneicon{z_blue}{$\mathcal{Z}_{\text{\ding{55}} \rightarrow \text{\ding{51}}}$}\xspace}
\newcommand{\zonez}{\xspace \zoneicon{z_red}{$\mathcal{Z}_{\text{\ding{55}} \rightarrow \text{\ding{55}}}$}\xspace}
\newcommand{\zonexx}{\xspace \zoneicon{z_orange}{$\mathcal{Z}_{\text{\ding{51}} \rightarrow \text{\ding{55}}}$}\xspace}


\ExplSyntaxOn
\NewDocumentCommand{\cb}{m}
{
  \fp_set:Nn \l_tmpa_fp { 60 } 
  \adjustbox{margin=1.5pt, bgcolor=gray!\fp_eval:n{\l_tmpa_fp*#1}}{#1}
}
\ExplSyntaxOff

% \newcommand{\zonex}{\zone{z_green}{z_green}{$\mathcal{Z}_{\text{\ding{51}}}$}\xspace}
% \newcommand{\zoney}{\zone{z_red}{z_green}{$\mathcal{Z}_{\text{\ding{55}} \rightarrow \text{\ding{51}}}$}\xspace}
% \newcommand{\zonez}{\zone{z_red}{z_red}{$\mathcal{Z}_{\text{\ding{55}} \rightarrow \text{\ding{55}}}$}\xspace}
% \newcommand{\zonexx}{\zone{z_orange}{z_orange}{$\mathcal{Z}_{\text{\ding{51}} \rightarrow \text{\ding{55}}}$}\xspace}
% If the title and author information does not fit in the area allocated, uncomment the following
%
%\setlength\titlebox{<dim>}
%
% and set <dim> to something 5cm or larger.


\title{Investigating the Zone of Proximal Development of Language Models\\ for In-Context Learning}


\author{
Peng Cui~\qquad~Mrinmaya Sachan \\
Department of Computer Science, ETH Zürich \\
\texttt{
\href{mailto:peng.cui@inf.ethz.ch}{peng.cui@inf.ethz.ch}
}\\
}


\begin{document}
\maketitle
\begin{abstract}


In this paper, we introduce a learning analytics framework to analyze the in-context learning (ICL) behavior of large language models (LLMs) through the lens of the Zone of Proximal Development (ZPD), an %well-
established theory in educational psychology. 
ZPD delineates the space between what a learner is capable of doing unsupported and what the learner cannot do even with support.
We adapt this concept to ICL, measuring the ZPD of LLMs based on model performance on individual examples before and after ICL.
Furthermore, we propose an item response theory (IRT) model to predict the distribution of zones for LLMs.
Our findings reveal a series of intricate and multifaceted behaviors of ICL, providing new insights into understanding and leveraging this technique. 
Finally, we demonstrate how our framework can enhance LLM in both inference and fine-tuning scenarios:
(1) By predicting a model’s zone of proximal development, we selectively apply ICL to queries that are most likely to benefit from demonstrations, achieving a better balance between inference cost and performance; 
(2) We propose a human-like curriculum for fine-tuning, which prioritizes examples within the model’s ZPD. 
The curriculum results in improved performance, and we explain its effectiveness through an analysis of the training dynamics of LLMs.\footnote{Code is available at \href{https://github.com/nlpcui/llm_zpd}{https://github.com/nlpcui/llm-zpd}} 

\end{abstract}

\section{Introduction}
Human learning is a dynamic and progressive process where learners integrate new information into their knowledge base through interactions with the environment \cite{piaget1977development}.
Research in education and learning sciences has extensively explored what makes learning most effective and efficient.
Among them, the Zone of Proximal Development (ZPD) emphasizes the alignment between the learner's capability and the problem's difficulty \cite{vygotsky1978mind}.
Specifically, ZPD refers to the range of problems that a learner can solve with appropriate scaffolding but cannot tackle independently.
This concept is essential in education as
it identifies knowledge that is valuable for learning, feasible to acquire, and not yet mastered. 
Therefore, learning within ZPD is believed to foster more effective cognitive development \cite{chaiklin2003zone, tharp1991rousing}.


In this paper, we propose a learning analytics framework to study the \emph{learning behavior} of language models through the lens of ZPD. 
In particular, we focus on in-context learning (ICL), an emerging ability of LLMs that allows them to learn from a few demonstrations \cite{brown2020language,wei2022emergent}.
Previous studies have primarily focused on strategies for demonstration optimization \citep{liu2021makes,qin2023context,rubin2021learning,ye2023compositional}. 
However, even with high-quality demonstrations, the performance of ICL still varies significantly across tasks and data \cite{srivastava-etal-2024-nice}.
This variability calls for a more comprehensive examination of the \emph{inherent} in-context learnability of LLMs on individual queries. 

\begin{figure}
    \centering
    \includegraphics[width=\columnwidth]{figures/motivation.pdf}
    \caption{We conceptualize an LLM's Zone of Proximal Development (ZPD) for ICL as the set of queries on which the model's performance can be improved with demonstrations. We introduce a framework to measure and predict this zone and explore its applications.}
    \label{fig:illustration}
\end{figure}

We first formalize the concept of ZPD in ICL. 
Drawing on the parallel between ICL and human learning from worked examples, we view LLMs as learners and in-context demonstrations as a form of scaffolding. 
Then, based on the model's prior knowledge and its response to ICL, a query set can be divided into three \textbf{zones} ($\mathcal{Z}$): 
(1) The first zone, denoted as \zonex, consists of queries that can be solved by the model via direct prompting, representing the model's prior knowledge;
(2) The second zone, denoted as \zoney, includes queries that can be solved by the model only with ICL, representing the model's ZPD; and
(3) The third zone, denoted as \zonez, contains queries that the model cannot solve even with ICL, representing the knowledge beyond the model's reach. 
Figure \ref{fig:illustration} illustrates this conceptualization. 
This categorization provides a granular look at the model's capability, limitations, and interaction with specific interventions. 

We begin by measuring the task-specific zones of various models (\cref{sec: measure}). 
Since the ICL performance is sensitive to the choice of demonstrations and the ground-truth demonstrations are not available, it is non-trivial to determine whether a problem can potentially benefit from ICL.
To address this, we employ a greedy algorithm to construct \emph{Oracle} demonstrations for each query and use them to approximate the zone distribution empirically. 
Then, we propose to predict the zones of unseen queries using the item Response theory (IRT; \citet{santor1998progress}), which jointly captures the latent traits of the model and query (e.g., ability, difficulty). 
In particular, we introduce a variant of IRT that further takes into account the model's in-context learnability to capture the performance changes with or without ICL (\cref{sec: pred}).
We find that the ICL behavior of LLMs is generally predictable even without demonstration information, although the degree of predictability varies across different datasets and tasks.

Finally, we showcase how our framework enhances LLMs in both inference and fine-tuning scenarios (\cref{Sec: zone_application}). 
For inference, we propose a selective ICL strategy, which first predicts the zone of input queries and then applies ICL only to queries that are most likely to benefit from ICL (i.e., within the model’s ZPD \zoney). 
Experimental results show this approach achieves competitive or even better performance with reduced inference cost.
For fine-tuning, we propose a ZPD-based curriculum that prioritizes challenging yet learnable training examples.
We find such a curriculum improves fine-tuning outcomes.
Upon further analysis of training dynamics, we find LLMs exhibit  \emph{consistent learnability} under both ICL and fine-tuning settings.
This consistency explains the effectiveness of our ZPD-based curriculum and suggests potential connections between these two learning paradigms.

In summary, our contributions are threefold:
\begin{itemize}
 \item We conceptualize the ZPD framework for LLMs, which provides a new perspective on analyzing their ICL behavior.
 \item We introduce a novel IRT variant that captures LLMs’ in-context learnability and predicts their performance with or without ICL.
 \item We showcase two applications of our framework: a selective ICL strategy and a ZPD-based curriculum, demonstrating its potential to enhance both LLM training and inference.
\end{itemize}

\section{Related Work}
\noindent \textbf{In-Context Learning} \cite{brown2020language} has become a popular paradigm for enhancing the capabilities of LLMs across a wide range of tasks. 
Previous work has extensively focused on optimizing demonstrations, particularly through the selection \cite{liu-etal-2022-makes,rubin-etal-2022-learning,li2023unified} and ranking \cite{pmlr-v139-zhao21c,lu-etal-2022-fantastically} of in-context examples. 
In this paper, we shift the focus from demonstration optimization to the LLM and the target query themselves, highlighting the inherent in-context learnability of LLMs on individual queries. 
Our study complements these works, contributing to a holistic understanding of what makes ICL (un)successful.
Another line of research explores how the ICL capability emerges and functions, with various hypotheses proposed, such as task recognition \cite{xie2022incontext, 10.5555/3666122.3666809}, composition \cite{li-etal-2024-language}, meta-gradient learning \cite{garg2022can,akyurek2023what}. 
This paper also aims to understand ICL but from an empirical perspective by collecting, analyzing, and predicting ICL behaviors.

\vspace{1.5mm}
\noindent \textbf{Adoption of IRT in NLP.}
IRT is a set of statistical models used in educational assessments to measure the latent abilities of individuals through standardized testing \cite{lord2008statistical, santor1998progress}. 
In recent years, it has become increasingly popular in NLP. 
\citet{byrd2022predicting} uses IRT to estimate question difficulty and model
skills. 
\citet{gor2024great} proposes a content-aware and identifiable IRT to analyze human-AI complementarity. 
\citet{polo2024tinybenchmarks} argues for the adoption of IRT to build benchmarks for efficient evaluation.
In this work, we use IRT to predict LLM in-context learnability on individual queries (conceptualized as ZPD) by capturing the behavior of LLMs before and after (in-context) learning.

\vspace{1.5mm}
\noindent \textbf{Curriculum Learning} \cite{bengio2009curriculum} is the approach that organizes the training examples such that the model converges faster and better, which has been successfully applied in various NLP tasks \cite{tay2019simple,platanios2019competence,sachan2016easy}. 
Typically, curriculum learning algorithms organize training examples in increasing order of difficulty. 
Conversely, there is another line of research that works in the opposite way to start with hard examples, namely Hard Example Mining \cite{shrivastava2016training,jin2018unsupervised}. 
In this paper, we propose a ZPD-based curriculum that strikes a middle point between the two techniques: prioritizing training examples that are challenging and yet learnable (i.e., within the model's ZPD).
Similar strategies have been proven effective in various scenarios \cite{mindermann2022prioritized}. 
However, this paper proposes a new framework for discovering such desired examples, which can be incorporated into existing approaches. 



\section{Measuring ZPD of LLMs}\label{sec: measure}
\subsection{Preliminaries} \label{sec: preliminary}
Let $\mathcal{D}=\{(x_1, y_1), ..., (x_n, y_n)\}$ be a dataset where $x_i$ is a query and $y_i$ is the ground-truth answer. 
We define the ZPD (\zoney) of a model $\mathcal{M}$ on $D$ as a subset of examples on which the model's performance can be improved through a learning trial. 
In this study, we focus on the ICL setting and measure learning outcomes by comparing the model's performance with and without ICL. 
Specifically, let $c = \{ (x_1, y_1)...(x_{k}, y_{k}) | x_{j} \in \mathcal{D}\}$ be a set of demonstrations for $x$ $(x \notin c)$, we define \zoney as:
\begin{gather}
    \text{\zoney} \triangleq \{x | \mathcal{F}(y^\varnothing)<\tau, \mathcal{F}(y^{c}) > \tau \},
\end{gather}
where $\mathcal{F}$ is a scoring function and $\tau$ is a threshold deciding whether the predicted answer is acceptable. 
$y^{\varnothing}$ and $y^{c}$ represent the model's output with direct prompting and with in-context demonstrations, respectively:
\begin{gather}
    y^{\varnothing} = \mathcal{M}(\mathcal{T}(x)), y^{c} = \mathcal{M}(\mathcal{T}(c_1) \oplus ... \oplus \mathcal{T}(x)).
\end{gather}
where $\mathcal{T}$ is a template function and $\oplus$ denotes string concatenation. 
Due to the potential interference between instruction and demonstration \cite{srivastava-etal-2024-nice}, we adopted a simple prompt template with minimal instruction to focus on the effect of demonstration (See Appendix Table \ref{tab:app_prompt}).
 
Similarly, we can define the other two subsets as follows:
\begin{gather}
    \text{\zonex} \triangleq \{x | \mathcal{F}(y^\varnothing)>\tau \}, \\
    \text{\zonez} \triangleq \{x | \mathcal{F}(y^\varnothing)<\tau, \mathcal{F}(y^{c})<\tau \},
\end{gather}
representing queries that can be solved by $\mathcal{M}$ with direct prompting,
and queries that cannot be solved even with ICL. 

This formalization is flexible and can be applied to other settings.
For example, future work could replace ICL with other prompting strategies or analyze fine-tuning behaviors by examining the performance across different epochs. 

\subsection{Approximating \zoney and \zonez}
While \zonex is deterministic from the model's base performance $\{y^{\varnothing}_1, y^{\varnothing}_2, ...\}$, \zoney and \zonez depend on the choice of demonstrations $c$.
In this paper, we aim to investigate the ideal ICL behavior of LLMs with \emph{optimal} demonstrations.
This is because our goal is to understand the model’s \emph{inherent} in-context learnability on individual queries rather than the behavior of a specific ICL strategy. 
Since optimal demonstrations for each query are unavailable, precise measurements of \zoney and \zonez are infeasible. 
To address this, we first create \emph{Oracle} demonstrations---the best demonstrations achievable in a practical setting (with a limited demonstration pool and restricted computation resources).
Then, we use them to approximate \zoney and \zonez.

In concrete, we adopt a retrieve and rank method to construct Oracle demonstrations.
Firstly, we retrieve a candidate set $\mathcal{C}$ for each query. 
The common belief is that demonstrations that are similar to the query are most likely to enhance performance \cite{liu-etal-2022-makes}. 
Following previous work \cite{rubin-etal-2022-learning}, we employ BM25 \cite{robertson2009probabilistic}, a sparse retriever based on surface features, and SBERT \cite{reimers2019sentence}, which is based on dense sentence encoding. 
For the two retrievers, we calculate similarities based on both the $(x, y)$ pair and the ground-truth answer $y$ only, resulting in 2 $\times$ 2 $\times$ $K$ candidates.  
However, similarity may not be the only criterion for demonstration selection. 
To further enrich the candidate set and recall effective but dissimilar demonstrations, we randomly sample $K$ candidates from the bottom 50 percentile of the retrieving results, doubling the candidate size. 

Next, we select Oracle demonstrations $c$ using a greedy scoring approach:
\begin{gather}
    c_i = \mathop{{\rm argmax}}\limits_{\mathcal{C} \setminus \{c_1,..,c_{i-1}\} } {\rm Prob}_{\mathcal{M}} (y|c_1 \oplus ... c_{i} \oplus x),
\end{gather}
where $c_i$ is the $i^{th}$ selected demonstration and ${\rm Prob_{\mathcal{M}}(\cdot)}$ is the probability from the model $\mathcal{M}$. 
In other words, we greedily choose demonstrations that can maximize the likelihood of the ground-truth answer.
With these demonstrations, the resulting \zoney is a subset of the actual ZPD while \zonez is a superset of the actual one.
In the rest of the paper, we use \zoney and \zonez to denote for the approximated zones unless otherwise specified. 

\section{Zone Prediction} \label{sec: pred}
In this section, we attempt to build a model to predict an LLM’s zone distribution on unseen queries. 
Essentially, the goal is to predict the model's performance, i.e., whether it can solve a query directly (\zonex) or with ICL (\zoney), or not at all (\zonez). 
We propose a novel variant of item response theory (\textsc{Irt}) to capture the latent traits of the LLM and the queries.
A graphic view of our model is shown in Figure \ref{fig:model}.

\subsection{Background of \textsc{Irt}}
\textsc{Irt} is a statistical model that predicts the probability of individual respondents correctly answering a set of queries (or items).
In this work, we take a collection of LLMs $\{\mathcal{M}_1, \mathcal{M}_2, ..., \mathcal{M}_m\}$ as respondents. 
The basic 1 Parameter Logistic (1PL) \textsc{Irt} is defined as:
\begin{gather}
    P(r_{i,j}=1|\mathcal{M}_{i}, x_{j}) = \sigma(\theta_{i}-d_{j}), \label{eq:irt_1PL}
    \end{gather} 
where $r_{i,j}$ is the binary correctness label of $\mathcal{M}$'s prediction on $x_i$. 
$\sigma$ is the ${\rm sigmoid}$ function. 
$\theta_i$ and $d_j$ are latent variables (scalars) to be estimated, representing the ability of the $i$th model $\mathcal{M}_{i}$ and the difficulty of the $j$th query $x_j$. 
Simply put, \textsc{Irt} predicts the correctness label based on the gap between model ability and query difficulty.

The 1PL \textsc{Irt} assumes the monotonic relationship between item difficulty and respondent ability.
To relax this, we employ the multi-dimensional IRT (\textsc{MIrt}, \citet{reckase200618}), which is defined as:
\begin{gather}
    P(r_{i,j}=1|\mathcal{M}_{i}, x_{j}) = \sigma(\boldsymbol{\theta}_{i}^{\mathrm{T}}\boldsymbol{\alpha}_{j}-d_{j}), \label{eq:MIRT-DP}
\end{gather}
where the model's ability is represented as a \emph{skill vector} $\boldsymbol{\theta}_j \in \mathbb{R}^{\rm H}$.
Correspondingly, an item-wise \emph{discrimination vector} $\boldsymbol{\alpha}_i \in \mathbb{R}^{\rm H}$ is introduced to represent its latent traits.
A closer alignment between $\boldsymbol{\theta}_{i}$ and $\boldsymbol{\alpha}_j$ indicates a higher likelihood of a correct response.
%\mrinmaya{I think the above part about MIRT can be written better. We}

The training objective of \textsc{Irt} is defined as:
\begin{gather}
    \mathcal{L}_{\rm IRT}=\sum_{i=1}^{\rm M}\sum_{j=1}^{\rm N} {\rm CE}(P(r_{i,j}), y_{j}),
\end{gather}
where ${\rm CE(\cdot)}$ stands for the cross-entropy loss between predicted probability and the groud-truth label.

\begin{figure}[t]
    \centering
    \includegraphics[width=\columnwidth]{figures/method.pdf}
    \caption{
    We assume that a model's performance on a given query, $y^{c}$ (with ICL) or $y^{\varnothing}$ (without ICL), is determined by latent traits (shadowed nodes, bottom) of both the model and the query, including the model's skill $\boldsymbol{\theta}$, ICL skill $\boldsymbol{\theta^{c}}$, the query's discrimination $\boldsymbol{\alpha}$, ICL discrimination $\boldsymbol{\alpha^c}$, and overall difficulty $d$. 
    }
    \label{fig:model}
\end{figure}


\subsection{Content-Aware \textsc{Mirt}}
A limitation of \textsc{MIrt} is that it relies on the response data to infer item traits $\boldsymbol{\alpha}_i$. 
Therefore, it cannot generalize to unseen queries during inference.
To overcome this limitation, we use a lightweight neural network to parameterize item traits based on their text features. 
Specifically, for a given query $x_{j}$, we first use an embedding model to obtain its representation $\boldsymbol{e}_j$.
Then, we compute its traits by:
\begin{gather}
    d_{j} = f({\rm {\bf W}_{d}} \boldsymbol{e}_{j} + {\rm {\bf b}_{d}}); 
    \boldsymbol{\alpha}_j = f({\rm {\bf W}_{\alpha}} \boldsymbol{e}_{j} + {\rm {\bf b}_{\alpha}})
\end{gather}
where ${\rm {\bf W}_{d}}, {\rm {\bf W}_{\alpha}}, {\rm {\bf b}_{d}, {\rm {\bf b}_{\alpha}}}$ are learnable weights, trained together with the IRT model, and $f$ is the ${\rm Relu}$ function.

\subsection{Adapting \textsc{Mirt} to Learning Dynamics}
While the above model can predict the model's performance on an unseen query, it cannot predict one query's correctness label under two settings and thus cannot predict three zones simultaneously.
We propose a variant that incorporates the dynamics of ICL. 
Concretely, we introduce an additional \emph{ICL skill vector} $\boldsymbol{\theta}^{c}$ for the model and similarly an \emph{ICL discrimination vector} $\boldsymbol{\alpha}^{c}$ for the item:
\begin{gather}
         P(r_{i,j}=1|\mathcal{M}_{i}, x_{j}) = \sigma(\boldsymbol{\theta}_{i}^{\mathrm{T}}\boldsymbol{\alpha}_j-d_{j}+\boldsymbol{\theta}^{c\mathrm{T}}_{i }\boldsymbol{\alpha}_{j}^{c}), \label{eq:MIRT-ICL}
\end{gather}
where the alignment between $\boldsymbol{\theta}^{c}_{i}$ and $\boldsymbol{\alpha}_{j}^{c}$ represents the in-context \emph{learnability} of $\mathcal{M}_{i}$ with respect to $x_j$. 
Similar to $d$ and $\boldsymbol{\alpha}$, $\boldsymbol{\alpha}^{c}_{j}$ is computed based on the embedding of x:
\begin{gather}
    \alpha^{c}_{j} = f({\rm {\bf W}^{c}_{\alpha}} e_{j} + {\rm {\bf b}^{c}_{\alpha}}).
\end{gather}
Combining Eq. \ref{eq:MIRT-DP}, and \ref{eq:MIRT-ICL}, we have:
\begin{gather}
         P(r_{i,j}^{\{ \varnothing, c \}}=1) = \sigma(\boldsymbol{\theta_{i}} \boldsymbol{\alpha_j} - d_{j} + g^{\{ \varnothing, c \} } \boldsymbol{\theta}^{c}_{i }\boldsymbol{\alpha}_{j}^{c}), \label{eq:combined}
\end{gather}
where $r^{\varnothing}$ and $r^{c}$ are the correctness labels under direct prompting and ICL.
$\{g^\varnothing=0, g^c=1\}$ is a gating parameter in align with $r$ to ensure that $\theta^{c}_{i }\alpha_{j}^{c}$ are only enabled in the ICL setting. 
In doing this, the latent factors are learned such that:
\begin{gather}
    \left\{
        \begin{array}{cc}
            \boldsymbol{\theta}^{\mathrm{T}} \boldsymbol{\alpha} > d, \boldsymbol{\theta}^{\mathrm{T}} \boldsymbol{\alpha} + \boldsymbol{\theta}^{c\mathrm{T}} \boldsymbol{\alpha}^{c} >d, \ \text{if} \ \ r^{\varnothing}=1, &\\
             \boldsymbol{\theta}^{\mathrm{T}} \boldsymbol{\alpha} <d, \boldsymbol{\theta}^{\mathrm{T}} \boldsymbol{\alpha} + \boldsymbol{\theta}^{c\mathrm{T}} \boldsymbol{\alpha}^{c} >d, \ \text{if} \ \ r^{\varnothing}=0, r^{c}=1,  &\\
             \boldsymbol{\theta}^\mathrm{T} \boldsymbol{\alpha} <d, \boldsymbol{\theta}^\mathrm{T} \boldsymbol{\alpha} + \boldsymbol{\theta}^{c} \mathrm{T} \boldsymbol{\alpha}^{c} <d, \ \text{if} \ \ r^{\varnothing}=0, r^{c}=0. &
        \end{array}
    \right.
\end{gather}
The above three situations correspond to \zonex, \zoney, and \zonez, respectively. 
We refer to the proposed model as \textsc{Mirt}$_{\textsc{Icl}}$. 

From a multi-task learning perspective, our model can be seen as jointly training two IRT models, each with its own ability ($\theta, \theta^{c}$) and discrimination ($\alpha, \alpha^{c}$) parameters, while sharing the overall item difficulty ($d$). 
This allows the model to better capture the relationships between LM behaviors across the two settings.

\section{Experiments}
We experiment with 8 LLaMA models \cite{touvron2023llama,dubey2024llama} of various sizes, including \texttt{LLaMA-2-7B}, \texttt{LLaMA-2-7B-chat},
\texttt{LLaMA-2-13B}, \texttt{LLaMA-2-13B-chat}, \texttt{LLaMA-3-8B}, \texttt{LLaMA-3-8B-Instruct},  \texttt{LLaMA-3-70B}, and \texttt{LLaMA-3-70B-Instruct}. 
In particular, we consider both instruction-tuned (IT) (\texttt{-chat/Instruct} models) or non-IT versions to examine the influence of instruction tuning on the model’s ZPD.
In this study, we focus on the \emph{mathematical reasoning} and \emph{text understanding} abilities of LLMs, using the MathQA dataset \textbf{GSM8K} \cite{cobbe2021gsm8k} and the Stance detection (Favor, Neutral, Against) dataset \textbf{EZStance}  \cite{zhao2023ez} for stance detection. 
Detailed experiment setup can be found in Appendix \ref{sec:app_setup}.


We first present and analyze the zone distribution of various LLaMA models (\cref{Sec:zone_dist}).
Then, we evaluate the performance of IRT models on zone prediction (\cref{Sec: zone_pred}). 
Finally, we demonstrate two applications of our framework (\cref{Sec: zone_application}). 

\subsection{Zone Distribution Analysis} \label{Sec:zone_dist}
\begin{figure}
    \centering
    \includegraphics[width=\columnwidth]{figures/zone_dist.png}
    \caption{Zone distribution of various LLMs on the two datasets. Yellow lines represent the accuracy of \textsc{Kate}.}
    \label{fig:zone_dist}
\end{figure}
We measure the three zones of LLaMA models on the test set of GSM8K and the validation set of EZStance. 
Our observations are as follows.

\vspace{1.5mm}
\noindent \textbullet\ \emph{\textbf{The potential of ICL remains largely untapped}}. 
In Figure \ref{fig:zone_dist}, we present the zone distributions of various models. 
Ideally, the accuracy of ICL should be the combined proportion of \zonex and \zoney, which highlights the great potential of ICL. 
For instance, on the GSM8K dataset, the 8B-Instruct model, with the help of Oracle demonstrations, can achieve competitive performance compared to the two 70B models.
Note that, however, this is only a lower bound of ideal ICL performance, as the Oracle demonstrations are still sub-optimal.
Nevertheless, the current method still falls short of fully utilizing even this lower bound. 
For reference, we highlight the accuracy (yellow line) of \textsc{Kate} \cite{liu2021makes}, a similarity-based demonstration selection strategy (with \texttt{paraphrase-mpnet-base-v2}).
On average, it lags by around 20\% on the two datasets.

\begin{figure}[t]
    \centering
    \includegraphics[width=\columnwidth]{figures/neg_icl.png}
    \caption{Increased and decreased accuracy by \textsc{Kate} on GSM8K (left) and EZStance (right).}
    \label{fig:neg_icl}
\end{figure}

\vspace{1.5mm}
\noindent \textbullet\ \emph{\textbf{In-context demonstrations can be harmful.}}
In Section \cref{sec: measure}, we divide a query set into three zones according to the model's performance difference with and without ICL. 
However, sometimes, ICL can also degrade the performance.
We denote the collection of such queries as \zonexx. 
We do not frame \zonexx into our formalization (\cref{sec: preliminary}) but merge it into \zonex because we focus on the ideal ICL behavior given Oracle demonstrations.
However, this negative effect of ICL is non-negligible in a practical setting.

With \textsc{Kate} as a case study, we compare its increased accuracy (i.e., the proportion of recalled ZPD (\zoney) examples) and decreased accuracy (i.e., the proportion of \zonexx examples) in Figure \ref{fig:neg_icl}. 
The sum of the two is the overall performance of $\textsc{Kate}$. 
We can see \zonexx can reduce up to 14\% and 18\% accuracy on GSM8K and EZStance.
Besides, this negative effect is also model-dependent. 
For example, \texttt{LLaMA-2-7B-chat} and \texttt{LLaMA-2-13B-chat} are particularly vulnerable to harmful demonstrations, and this negative effect even overwhelms the benefit for \texttt{LLaMA-3-70B}.
% More investigations are needed to understand this behavior, which we leave for future work. 
This granular look at the ICL performance provides a new perspective to improve ICL strategy: recalling examples in \zonex while minimizing \zonexx. 
Previous work mainly focused on the first direction and we will showcase how our IRT model can enhance ICL through the second way in \cref{App1}.


\begin{table}[h]
\small
\centering
\begin{tabular}{@{}lllllll@{}}
\toprule
\multirow{2}{*}{\textbf{Zones}} & \multicolumn{3}{c}{\textbf{GSM8K}}         & \multicolumn{3}{c}{\textbf{EZStance}}      \\ \cmidrule(l){2-7} 
                                & \textbf{Max} & \textbf{Min} & \textbf{Avg} & \textbf{Max} & \textbf{Min} & \textbf{Avg} \\ \midrule
\zonex                              & \cb{0.89}         & \cb{0.74}         & \cb{0.84}         & \cb{0.91}             & \cb{0.46}              & \cb{0.70}             \\
\zoney                              & \cb{0.74}         & \cb{0.21}         & \cb{0.58}         &  \cb{0.78}            & \cb{0.34}             & \cb{0.58}             \\
\zonez                              & \cb{0.58}         & \cb{0.20}         & \cb{0.42}         &   \cb{0.87}           &  \cb{0.32}            &  \cb{0.53}            \\ \bottomrule
\end{tabular}
\caption{Pairwise overlap coefficients among zones of different LLMs.} 
\label{tab: overlap}
\end{table}

\vspace{1.5mm}
\noindent \textbullet\ \emph{\textbf{ZPD (\zoney) of LLMs differ significantly}}. 
We measure the overlap between zones of different models by calculating their averaged pairwise \emph{Overlap Coefficient}, defined as:
\begin{gather}
    \textsc{Overlap} (A, B) = \frac{|A \cap B|}{{\rm min}(|A|, |B|)},
\end{gather}
where $A$ and $B$ are the zones to compare. 
The results are shown in Table \ref{tab: overlap}, where we can see examples in \zonex are largely shared across various models, while examples in \zoney and \zonez do not highly overlap, indicating each LLM has its own ZPD.
This suggests that ICL strategies should take into account both the data aspect (e.g., similarity) and the model, corroborating the conclusion of \citet{peng-etal-2024-revisiting}. 



\subsection{Zone Prediction Evaluation}\label{Sec: zone_pred}

We compare our proposed IRT model \textsc{Mirt}$_{\rm ICL}$ (Eq. \ref{eq:combined}) with the following baselines:
i) \underline{1PL model} (\textsc{Irt}$_{\rm 1PL}$, Eq. \ref{eq:irt_1PL}), 
ii) \underline{2PL model}, which is similar to Eq. \ref{eq:MIRT-DP} but with $\theta$ and $\alpha$ as scalars, 
and iii) \underline{Multi-Dimensional IRT} \textsc{Mirt} (Eq. \ref{eq:MIRT-DP}).
We evaluate their ability to predict LLM performance under both direct prompting (DP) and ICL, using AUC as the primary metric. See Appendix \ref{sec:app_irt_implementation} for the implementation details. 
Note that aside from our \textsc{Mirt}$_{\textsc{Icl}}$, other baseline models are trained solely on DP data. 
Nevertheless, we can assess their generalization ability to the ICL setting: since AUC assesses the relative ranking of predicted probabilities, these models should also achieve good AUC if \emph{the LLM's probabilities of correctly answering individual queries are consistent across both settings}. 


\begin{table}[t]
\small
\centering
\setlength\tabcolsep{4pt}
\begin{tabular}{@{}lcccccc@{}}
\toprule
\multirow{2}{*}{\textbf{Model}} & \multicolumn{3}{c}{\textbf{GSM8K}} & \multicolumn{3}{c}{\textbf{EZStance}} \\ \cmidrule(l){2-7} 
                                & DP        & ICL       & Overall    & DP         & ICL        & Overall     \\ \midrule
\textsc{Irt}$_{\rm 1PL}$                             & 0.808     & 0.769     & 0.748      & 0.736      & 0.617      & 0.644       \\
\textsc{Irt}$_{\rm 2PL}$                             & 0.788     & 0.740     & 0.728      & 0.739      & 0.631      & 0.651       \\
\textsc{Mirt}                            & \textbf{0.837}     & 0.770     & 0.743      & 0.760      & 0.608      & 0.799       \\
\textsc{Mirt}$_{\rm ICL}$                            & 0.833     & \textbf{0.821}     & \textbf{0.862}      & \textbf{0.770}      & \textbf{0.662}      & \textbf{0.799}       \\ \bottomrule
\end{tabular}
\caption{Performance (AUC) of various IRT models on the two datasets. The best results are in \textbf{bold}. Results of Accuracy can be found in Appendix Table \ref{tab:app_irt_acc}.}
\label{tab:irt_auc}
\end{table}


\vspace{1.5mm}
\noindent \textbullet\ \emph{\textbf{ICL behavior is, to varying degrees, predictable without demonstrations}}. 
We present the AUC results in Table \ref{tab:irt_auc}. 
As a demonstration-agnostic model, \textsc{Mirt}$_\textsc{\ Icl}$ achieves reasonably decent performance GSM8K but comparatively weaker results on EZStance. 
We interpret the difference through the \emph{predictability} and \emph{sensitivity} of ICL: for certain tasks and datasets, ICL performance may hinge more on the model’s inherent ICL capacity and the query's difficulty. 
While for others, it may depend more on the demonstrations or prompts, making the ICL behavior less predictable without the information of demonstrations.
Existing work has been focusing on measuring and mitigating sensitivity \cite{zhao2021calibrate}. 
We highlight a complementary perspective: measuring and leveraging (See \cref{Sec: zone_application} for applications) the predictability of ICL behavior.
\vspace{1.5mm}


\begin{table}[t]
\small
\centering
\setlength\tabcolsep{2pt}
\begin{tabular}{@{}lcccccccc@{}}
\toprule
\multirow{2}{*}{} & \multicolumn{2}{c}{\textbf{L2-7B}} & \multicolumn{2}{c}{\textbf{L2-13B}} & \multicolumn{2}{c}{\textbf{L3-8B}} & \multicolumn{2}{c}{\textbf{L3-70B}} \\
                  & base    & chat   & base    & chat    & base    & instr.   & base    & instr.    \\ \midrule
\textbf{GSM8K}    & $+$.10            & $+$.40            & $+$.13$^*$             & $+$.24             & $+$.29             & $+$.07            & $+$.31             & $+$.53             \\
\textbf{EZStance} & $-$.45            & $-$.60           & $-$.47            & $-$.25            & $-$.35            & $-$.28           & $-$.48            & $-$.36            \\ \bottomrule
\end{tabular}
\caption{Pearson Correlation between $\boldsymbol{\theta}^{\mathrm{T}}\boldsymbol{\alpha}-d$ (model's ability to solve the query with direct prompting) and $\boldsymbol{\theta}^{c\mathrm{T}} \boldsymbol{\alpha}^{c}$ (the additional gain obtained by ICL). Results with $^{*}$ indicate $p$-value$>0.05$.} \label{tab:pearson}
\vspace*{-2mm}
\end{table}


\noindent \textbullet\ \emph{\textbf{(In)consistency between difficulty and in-context learnability.}}
In Eq. \ref{eq:combined}, $\boldsymbol{\theta}\boldsymbol{\alpha} - d$ represents the model's ability to solve the query with DP (or the query's \textit{difficulty}), while $\boldsymbol{\theta}^{c} \boldsymbol{\alpha}^c$ captures the additional gain achieved through ICL, reflecting the model's \textit{in-context learnability} of the query.
To examine the relationship between the two terms, we compute their Pearson correlation. 
The results, presented in Table \ref{tab:pearson}, reveal that for the GSM8K dataset, these two terms exhibit weak or moderate positive correlations (from $+$0.07 to $+$0.53). 
Interestingly, the correlation on EZStance is stronger but negative, meaning difficult examples under direct prompting (lower $\boldsymbol{\theta}^{\mathrm{T}}\boldsymbol{\alpha} - d$) seem to benefit more from ICL (higher $\boldsymbol{\theta}^{c \mathrm{T}} \boldsymbol{\alpha}^c$) and vice versa. 
% However, due to the weaker performance on EZstance, this phenomenon should be interpreted with caution. 
This suggests that a query’s difficulty and its in-context learnability are not always aligned. We attribute this phenomenon to the differing abilities required for direct prompting versus ICL. 
The former primarily relies on the model’s prior knowledge of the query, while the latter depends on its ability to leverage contextual information. 
As a result, this inconsistency could arise in certain tasks and queries where the knowledge is missing but easy to learn in context. 
A notable example is classification with flipped or semantically unrelated labels \cite{wei2023larger}, where an LM struggles to solve the disrupted task in the regular setting but can successfully learn the new mapping through demonstrations.



\subsection{Applications} \label{Sec: zone_application}
In this section, we demonstrate how our framework can improve in-context learning through a selective ICL strategy (\cref{App1}) and a ZPD-derived curriculum for fine-tuning LLMs (\cref{App2}).

\subsubsection{Selective ICL}\label{App1}
\textbf{Approach.}
While ICL has demonstrated effectiveness across a wide range of tasks, it costs $k$ times additional input tokens ($k=$ the number of demonstrations).
Moreover, as discussed in \cref{Sec:zone_dist}, ICL sometimes results in worse performance, even with carefully retrieved demonstrations.
To address these issues, we propose Selective ICL (\textsc{SelIcl}). 
In specific, given a query $x_i$, we first predict its correct probability with direct prompting $p_{i}^{\varnothing}$ and the correct probability with ICL $p_{i}^{c}$ using Eq. \ref{eq:combined} with $g=0$ and $g=1$ respectively. 
Then, we decide the inference prompt for $x_i$ by:
\begin{gather}
    \left\{
        \begin{array}{ll}
             \mathcal{T}(\tilde{c}_1) ...  \oplus \mathcal{T}(x) \:  & \text{if} \; p^{\varnothing} < \tau_1 \; \text{and} \; p^{c} > \tau_2  \\
            \mathcal{T}(x) &\text{Otherwise.} 
        \end{array}
    \right. \label{eq:sel_icl}
\end{gather}
where $\{ \tilde{c}_{1}, ..., \tilde{c}_{n} \}$ are demonstrations retrieved by a certain strategy.
$\tau_1$ and $\tau_2$ are predefined thresholds.
A lower $p^{\varnothing}$ ($< \tau_1 $) and a higher $p^{c}$ ( $>\tau_2$) indicate the model is unable to solve this query with direct prompting but is likely to solve it with ICL. 
In other words, we apply ICL only to queries within the model's ZPD. 
By doing so, we aim to reduce unnecessary costs by avoiding ICL for either too easy ($p^{\varnothing} < \tau_1 $) or too hard ($p^{c} >\tau_2$) queries.
Furthermore, this can also potentially improve performance by mitigating the negative effect of ICL observed in Figure \ref{fig:neg_icl}.    

\vspace{1.5mm}


\begin{figure*}[th]
    \centering
    \includegraphics[width=\textwidth]{figures/gsm8k_sel_icl.png}
    \caption{Accuracy and inference cost (number of input tokens) of different ICL strategies on the GSM8K dataset. \textcolor{z_red}{$\blacktriangledown$} is the performance of the baseline \textsc{FulIcl}, which applies ICL to all the queries. \textcolor{z_blue}{$\bigcirc$} and \textcolor{z_green}{$\bigstar$} are the performance of \textsc{SelIcl} under various thresholds $\tau_1$ and $\tau_2$ (not shown), where \textcolor{z_green}{$\bigstar$} highlights cases in which \textsc{SelIcl} achieves better or equal accuracy with less input tokens compared to the baseline (\textcolor{z_red}{$\blacktriangledown$}).}
    \label{fig:gsm8k_sel_icl}
\end{figure*}

\noindent \textbf{Result and Analysis.}
We compare our \textsc{SelIcl} with the vanilla ICL that applies demonstrations to all queries  (denoted as \textsc{FulIcl}). 
Specifically, we use \textsc{Kate} to retrieve demonstrations for \textsc{FulIcl}.
However, it is worth noting that \textsc{SelIcl} is orthogonal to other ICL strategies for two reasons: 
(1) It focuses on determining when to apply ICL, independent of how demonstrations are selected or organized; 
(2) The IRT model is trained to predict the model’s ICL performance given Oracle demonstrations. 
Consequently, $p^{c}$ is expected to serve as the predicted upper bound for any ICL strategy.


To select $\tau_{1}$ and $\tau_{2}$ for \textsc{SelIcl}, we perform a grid search on the IRT validation set by varying their values within the range $[0.01, 0.02, \dots, 0.99]$.
For each combination, we decide whether or not to apply ICL to each query according to Eq. \ref{eq:sel_icl} and compute the overall accuracy and number of input tokens.  
Since the prompts and model outputs are already collected when constructing the IRT dataset (Appendix \ref{sec:app_irt_implementation}), these results can be obtained without additional model inference.


Then, we plot the Pareto curve \cite{deb2011multi} of \textsc{SelIcl}, approximated with scatter points.
In multi-objective optimization, each point on the Pareto curve represents a Pareto-optimal solution that cannot be further improved in one objective without compromising the other (in our case, accuracy and number of input tokens). 


Results for GSM8K are shown in Figure \ref{fig:gsm8k_sel_icl}, and results for EZstance are available in Appendix Figure \ref{fig:ezstance_sel_icl}.
Solutions that are dominated\footnote{
In the context of a Pareto curve, a solution dominates another if it is at least as good in all objectives and strictly better in at least one objective.} by others are discarded (apart from the baseline results (\textcolor{z_red}{$\blacktriangledown$}) for comparison).
As can be seen, for 6 out of 8 models, \textsc{SelIcl} with proper thresholds (\textcolor{z_green}{$\bigstar$}) can dominate \textsc{fulIcl}. 
Overall, \textsc{SelIcl} can serve as a tool to trade off accuracy and cost in resource-limited scenarios. 
\textsc{SelIcl} is paticularly successful for \texttt{LLaMA-2-7b-chat} and \texttt{LLaMA-70B-Instruct}. 
Combining with previous findings, both models have relatively narrow ZPD (Figure \ref{fig:zone_dist}) and are more susceptible to the negative effects of ICL (Figure \ref{fig:neg_icl}), suggesting that greater caution is needed when applying ICL to them.

\subsubsection{ZPD-based Curriculum}\label{App2}
It is generally believed that the success of ICL relies on the model's prior knowledge about the query \cite{xie2022incontext,li-etal-2024-language}.
Therefore, we assume that queries that can be enhanced by ICL (\zoney) are more learnable than those unsolvable by ICL (\zonez) but also more valuable for learning than those already solvable by DP (\zonex). 
Motivated by this, we proposed a ZPD-based curriculum learning algorithm for fine-tuning.

\begin{algorithm}[t]
\caption{ZPD-based Curriculum}
\label{alg:DCQGFramework}
\small
\setstretch{1.2}
    \begin{algorithmic}[1]
    \Require Training set $\mathcal{D}$, model $\mathcal{M}$, correct probability with DP $p^{\varnothing}$ and with ICL $p^{c}$, bucket $k$, epoch $e$
    \Ensure  Trained model $\mathcal{M}^{*}$ 
        \State  $\mathcal{D}^{*} \leftarrow $ Sort($\mathcal{D}$, $p^{c}_{i}-p^{\varnothing}_{i} $)
        \State $\{ \mathcal{D}_{1}, ..., \mathcal{D}_{n} \} \leftarrow \mathbf{SplitData}(\mathcal{D}^{*})$ ; $\mathcal{D}_{train} \leftarrow \varnothing$
        \For {$i=1, i\leq k, i$++}
            \State $\mathcal{D}_{train} \leftarrow \mathcal{D}_{train} \cup \mathcal{D}_{i}$ \Comment{Update training set}
            \For {$j=1, j\leq e, j$++}
            \State $\mathbf{Train}(\mathcal{M}, \mathcal{D}_{train})$;
            \EndFor
        \EndFor
    % \State \textbf{return} $\mathcal{M}^{*}$
    \end{algorithmic}
    \label{algo:cl_scheduler}
\end{algorithm}

\noindent \textbf{Approach.} 
Typically, curriculum learning consists of a ranking algorithm, which sorts examples according to a certain criterion, and a scheduling algorithm, which sequences examples for training. 
In our approach, we rank training examples according to $p^{c} - p^{\varnothing}$ (Eq. \ref{eq:sel_icl}), which represents the learning gain brought by ICL. 
For scheduling, we employ the baby-step algorithm \cite{spitkovsky2010baby}, which splits examples into buckets and accumulatively introduces new buckets. The overall process is outlined in Algorithm \ref{algo:cl_scheduler}.

\begin{figure}[t]
    \centering
    \includegraphics[width=\columnwidth]{figures/curriculum.png}
    \caption{Comparison between random and our \textsc{ZPD}-based curriculum on two datasets.}
    \label{fig:curriculum}
\end{figure}

\noindent \textbf{Results and Analysis.}
We compare our algorithm against a random baseline.
Although simple, random is the most widely used baseline in practice and is not necessarily a weak one, as many curriculum strategies fail to outperform it in language modeling \cite{campos2021curriculum}.
% \TODO{Random is not a weak baseline.}
We fine-tune the \texttt{LLaMA-8B-Instruct} model separately using the two methods with the same scheduler for $6$ epochs. 
See experimental details in Appendix \ref{sec:app_finetune}. 
As shown in Figure \ref{fig:curriculum}, our curriculum results in faster convergence and improved performance in most cases.
To understand why it works, we analyze the training loss of examples in different zones.
Specifically, we compute the \textbf{mean} and \textbf{variance} of each example's loss across epochs. 
The two metrics reflect the convergence behavior of individual examples: a higher mean indicates the example is harder to learn, while a higher variance indicates the model is ambiguous about the example \cite{swayamdipta-etal-2020-dataset}. 



For fair analysis, we fine-tune a new \texttt{LLaMA-8B-Instruct} model on the GSM8K dataset for $5$ epochs without any curriculum. 
Figure \ref{fig:training_loss} shows the loss information.
We found \emph{consistent learnability} between in-context learning and fine-tuning scenarios: 
examples in \zonez are the hardest to learn, 
followed by \zoney
\footnote{(Since we use sub-optimal Oracle demonstrations, some \zoney examples are not recalled and misclassified into \zonez. As a result, the actual loss value of \zoney data tends to be slightly closer to \zonez.)}, 
and lastly \zonex. 
This confirms that our curriculum works as expected: prioritizing examples that are learnable and informative (not yet learned).  
Such a strategy has been shown effective for various tasks and model architectures \cite{mindermann2022prioritized,fan2023irreducible}, and our framework provides a new way to discover these examples.


\begin{figure}[t]
    \centering
    \includegraphics[width=\columnwidth]{figures/loss.png}
    \caption{Mean and variance of training loss for queries in different zones. Results are computed over 5 epochs.}
    \label{fig:training_loss}
\end{figure}

\section{Conclusion}
This work presents a novel framework based on the Zone of Proximal Development (ZPD) theory to analyze the ICL behaviors of LLMs. 
We thoroughly discuss the formalization, measurement, prediction, and application of ZPD in LLMs. 
Our framework serves as an effective tool for understanding the potential, limitations, and complex dynamics of ICL. 
Furthermore, we demonstrate its applicability in both inference and training scenarios.

\section*{Limitations}
We discuss the limitations of this work from the following aspects.
First, due to the unavailability of optimal in-context demonstrations, we can only approximate the ZPD of LLMs, which is a lower bound of the model’s actual in-context learnability. 
This challenge is as nuanced and complex as understanding human learning: one can never precisely measure the potential of human learners.
Second, we investigate the ZPD of LLMs in a simplified scenario where we only consider demonstrations as guidance and use basic templates without instructions to minimize confounding factors. 
In practice, ICL is often combined with other prompting strategies, whose influence may warrant further exploration.
Finally, the ZPD is a dynamic range that evolves with the learner’s knowledge development. 
Our framework is designed to measure and leverage an LLM’s current ZPD, but it is less suited to modeling its developing process (e.g., across different checkpoints during pre-training or fine-tuning). 
In the future, more advanced learning analytics approaches, such as knowledge tracing, could be adopted to enhance our framework.


% This must be in the first 5 lines to tell arXiv to use pdfLaTeX, which is strongly recommended.
\pdfoutput=1
% In particular, the hyperref package requires pdfLaTeX in order to break URLs across lines.

\documentclass[11pt]{article}

% Remove the "review" option to generate the final version.
\usepackage{ACL2023}

% Standard package includes
\usepackage{times}
\usepackage{latexsym}

% For proper rendering and hyphenation of words containing Latin characters (including in bib files)
\usepackage[T1]{fontenc}
% For Vietnamese characters
% \usepackage[T5]{fontenc}
% See https://www.latex-project.org/help/documentation/encguide.pdf for other character sets

% This assumes your files are encoded as UTF8
\usepackage[utf8]{inputenc}

% This is not strictly necessary, and may be commented out.
% However, it will improve the layout of the manuscript,
% and will typically save some space.
\usepackage{microtype}

% This is also not strictly necessary, and may be commented out.
% However, it will improve the aesthetics of text in
% the typewriter font.
\usepackage{inconsolata}
%
% and set <dim> to something 5cm or larger.
\usepackage{microtype}
\usepackage{import}
\usepackage{layout}
\usepackage{tabularx, makecell}
\usepackage{booktabs}
\usepackage{mathrsfs}
\usepackage{amssymb} 
\usepackage{url}
\usepackage{hyperref}
\usepackage{graphicx}
\usepackage{xspace,paralist}
\usepackage{times,latexsym}
\usepackage{amsmath}
\usepackage{appendix}
\usepackage{comment} 
\usepackage{enumitem}
\usepackage{makecell}
\usepackage{multirow}
\usepackage{xcolor}
\usepackage{arydshln}
\usepackage{cleveref}
\usepackage{tcolorbox}
\usepackage{todonotes}
\usepackage{longtable,supertabular}
\usepackage{amssymb}% http://ctan.org/pkg/amssymb
\usepackage{pifont}% http://ctan.org/pkg/pifont
% \newcommand{\cmark}{\ding{51}}%
% \newcommand{\xmark}{\ding{55}}%

%\usepackage[frozencache,cachedir=.]{minted}
% \usepackage[cachedir=.]{minted}
\definecolor{bg}{rgb}{0.95,0.95,0.95}


\newcommand{\cmark}{\textcolor{green}{\ding{51}}}
\newcommand{\xmark}{\textcolor{red}{\ding{55}}}
\newcommand{\red}[1]{\textcolor{red}{#1}}
\newcommand{\orange}[1]{\textcolor{orange}{#1}}
\newcommand{\blue}[1]{\textcolor{blue}{#1}}
\newcommand{\cyan}[1]{\textcolor{cyan}{#1}}

\newcommand{\HL}[1]{\textcolor{blue}{#1}}
\newcommand{\ex}[1]{\textit{#1}\xspace}
\newcommand{\eqnref}[1]{Eq~\eqref{#1}\xspace}
\newcommand{\tabref}[1]{Table~\ref{#1}\xspace}
\newcommand{\figref}[1]{Figure~\ref{#1}\xspace}
\newcommand{\secref}[1]{Section~\ref{#1}\xspace}
\newcommand{\appref}[1]{Appendix~\ref{#1}\xspace}

\newcommand{\name}{\textit{SCALAR}}
\newcommand{\nmodels}{8\xspace}



% If the title and author information does not fit in the area allocated, uncomment the following
%
%\setlength\titlebox{<dim>}
%
% and set <dim> to something 5cm or larger.

\title{SCALAR: Scientific Citation-based Live Assessment \\of Long-context Academic Reasoning}

% \author{Renxi Wang*, Honglin Mu*, Liqun Ma, Lizhi Lin, Yunlong Feng \\ \textbf{Timothy Baldwin, Xudong Han, Haonan Li}}

\author{
Renxi Wang$^{1,2}$\thanks{\hspace{2mm}Equal contributions.} \quad Honglin Mu$^{1,2}$\footnotemark[1] \quad Liqun Ma$^{1}$ \quad Lizhi Lin$^{2,3}$ \quad Yunlong Feng$^{4}$ \\ 
\textbf{Timothy Baldwin$^{1,2,5}$ \quad Xudong Han$^{1,2}$ \quad Haonan Li$^{1,2}$\thanks{\hspace{2mm}Corresponding author.}} \\
$^{1}$MBZUAI \quad $^{2}$LibrAI \quad $^{3}$Tsinghua University \\ $^{4}$Alibaba Group \quad $^{5}$The University of Melbourne \\
}

\begin{document}
\maketitle
\begin{abstract}
Evaluating large language models' (LLMs) long-context understanding capabilities remains challenging. We present SCALAR (Scientific Citation-based Live Assessment of Long-context Academic Reasoning), a novel benchmark that leverages academic papers and their citation networks. SCALAR features automatic generation of high-quality ground truth labels without human annotation, controllable difficulty levels, and a dynamic updating mechanism that prevents data contamination. Using ICLR 2025 papers, we evaluate \nmodels state-of-the-art LLMs, revealing key insights about their capabilities and limitations in processing long scientific documents across different context lengths and reasoning types. Our benchmark provides a reliable and sustainable way to track progress in long-context understanding as LLM capabilities evolve.\footnote{\url{https://github.com/LibrAIResearch/scalar}}
\end{abstract}

\section{Introduction}

Large language models (LLMs) have demonstrated impressive capabilities in processing texts of increasing lengths \citep{achiam2023gpt,anthropic2024,dubey2024llama,yang2025qwen2}.
While capable of handling contexts of hundreds of thousands of tokens, evaluating their true understanding of long documents remains challenging. 

Previous evaluations of long-context understanding have often relied on synthetic datasets or simple retrieval tasks like ``needle in a haystack'' variations~\citep{needle,kuratov2024searchneedles11mhaystack,wang2024needle,roberts2024needlethreadingllmsfollow}. While such tasks can test a model's ability to locate information in long sequences, they fail to assess genuine comprehension and are readily solvable by current LLMs \citep{team2024gemini}. Moreover, creating high-quality benchmarks traditionally requires extensive human annotation, which is both time-consuming and costly. Some work transforms short-context tasks into long context by combining them with passages or long documents, such as long-document QA~\citep{needle}, summarization~\citep{chang2024booookscore}, reasoning~\citep{babilong} and reranking~\citep{helmet}. However, such datasets suffer from data contamination and shortcut exploitation, as LLMs can solve problems using their own knowledge rather than the long context. See more related work in \Cref{related_work}.

\begin{figure*}[t]
    \centering
    \tiny
    \includegraphics[width=1.0\linewidth]{images/SCILong.drawio.pdf}
    \caption{The overall process of building \name. We first crawl arXiv papers that are also in ICLR 2025. Then we parse them into structualized data, sampling citations to mask and other citations as candidates.}
    \label{fig:overview}
\end{figure*}


In this work, we propose \name, a novel benchmark for evaluating LLMs' long-context understanding in the scientific domain. Our approach leverages recent published academic papers and their citation networks, which are labeled by scientific researchers, ensuring annotations are rigorous and high-quality (especially since we use highly rated papers from top-tier conferences). 
This benchmark can be automatically updated according to the most recent high-quality publications that are also publicly available on arXiv.
Based on the collected data, we develop a framework that can control the difficulty levels dynamically. We configure three levels of difficulty to construct our benchmark, ensuring a suitable evaluation covering from small models to large models. Our evaluation reveals significant performance gaps even among state-of-the-art models, highlighting current limitations in long-context understanding.


Our contributions are threefold: (1) We introduce a systematic approach to creating high-quality, dynamic long-context evaluation datasets using academic citation networks, which can be continuously updated with new papers; (2) We release a comprehensive benchmark dataset based on ICLR 2025 papers, providing a challenging testbed for long-context understanding; (3) Through extensive evaluation of \nmodels advanced LLMs with enhanced long-context capabilities, we reveal several important insights about the current state and limitations of long-document understanding, including how model performance varies with different context lengths and types.

\section{\name}

\subsection{Data Construction}

\paragraph{Data Crawling} We first crawl the research articles and their citations from arXiv.\footnote{\url{https://arxiv.org/}} However, since arXiv papers may include low-quality or unverified research, we mitigate this issue by selecting only those papers that have been submitted to ICLR 2025 and have received an overall score higher than 7. This not only helps filter out low-quality research but also reduces the risk of data contamination.\footnote{Note that our methods can be generally applied to any paper.} 
% This initial collection forms the foundation of our dataset.

\paragraph{Preprocessing and Structualization} 
%The second stage focuses on the preprocessing of the collected papers. 
 We then identify citations within the paper body. Each citation is carefully marked with its position and mapped to its corresponding reference in the bibliography. We then use the data collection process to gather information about the cited papers. We separate papers into top-level sections, finding that papers in our dataset contain an average of 6.1 sections. For all papers, we remove references and appendices, leaving out the main content for task formulation.

\paragraph{Citation Filtering} To ensure high-quality evaluation data, we distinguish between two categories of citations: grouped citations and individual citations. Grouped citations refer to multiple sources together in a single parenthetical reference, typically used when summarizing general information, such as ``\textit{prior works (Liu et al., 2024; Hsieh et al., 2024a; Zhang et al., 2024)}''. Individual citations, in contrast, reference single sources separately, usually when discussing specific methods or results, for instance ``\textit{needle in a haystack' test (Kamradt, 2023)}''. Since grouped citations may have multiple correct answers when masked, during the processing in \Cref{sec:task_formulation}, we exclusively sample and mask individual citations to maintain clear ground truth labels.

% \begin{table}
% \begin{tabular}{lrrr}
% \toprule
% Subset & \#Num & \#Tokens & \#Words \\
% \midrule
% Easy   & 100 & 70,370 & 50,072 \\
% Medium & 100 & 77,241 & 53,695 \\
% Hard   & 100 & 68,155 & 46,798 \\
% \bottomrule
% \end{tabular}
% \label{tab:data_statistics}
% \caption{Caption.\HL{rethinking this table}}
% \end{table}

\subsection{Task Formulation}
\label{sec:task_formulation}
% \paragraph{Basic Task Format}
We formulate all problems as cloze-based multiple-choice questions, where language models must select the correct reference from provided candidates based on a paper context where a citation has been masked. Each question comprises three distinct parts: {Instruction}, {Question Paper}, and {Candidates}.
\texttt{Instruction} describes the task-related information, including how the LLM should finish the task, the answer format, and the roles of other parts of the question. \texttt{Question Paper} contains either a full paper or a specific section with a citation masked with a placeholder \texttt{**[MASKED\_CITATION]**}. \texttt{Candidates} contain four potential references --- one correct answer and three distractors. All candidates are drawn from the reference list of the question paper. Since we only use individual citations (where authors cite one paper in isolation rather than grouping multiple references), we can be confident that among all papers in the reference list, the authors determined this specific reference was the most appropriate for this particular citation, ensuring reliable ground truth labels.

In our prompt implementation, we define several XML elements to separate different elements. The details of the prompt template are shown in \Cref{fig:prompt}.

\begin{figure}[h]
	\small
	\begin{tcolorbox}[colframe=white, left=2.5mm, right=1.5mm]
\textbf{Question Paper}\\
\textbullet\ Single Section: Providing only the section containing the masked citation.\\
\textbullet\ Full Paper: Including the complete paper as context. \\

% We categorize \textbf{citations} into two types based on their semantic properties:\\
\textbf{Citation Type} \\
\textbullet\ Attributional Citations: These explicitly mention a specific model, method, or dataset name (\textit{e.g., ``BERT (Devlin et al., [2018]) is used for embedding texts...''}).\\
\textbullet\ Descriptive Citations: These integrate references into descriptions without explicit naming (\textit{e.g., ``Pretraining a bidirectional transformer (Devlin et al., [2018]) is time-consuming...''}).\\

\textbf{Distractor Sampling} \\
\textbullet\ Random Sample: Sampling from the question paper's reference list randomly.\\
\textbullet\ Nearest Sample: Sampling from the nearest 4 citations to the masked citation. We prioritize citations within the same section as the masked citation. If no enough citations, we sample from adjacent sections.\\

\textbf{Candidate Representation}\\
\textbullet\ Concise: Title and abstract only.\\
\textbullet\ Full: Complete paper content.\\
\textbullet\ Body: Paper content with title, abstract, introduction, and conclusion removed
	\end{tcolorbox}
	\caption{Configuration dimensions for controlling task difficulty in SCALAR. The framework allows adjustments across four key aspects: question paper scope, citation type, distractor sampling method, and candidate paper representation. Each dimension can be independently configured to create varying levels of challenge.}
	\label{fig:difficulty_config}
\end{figure}



\subsection{Difficulty Control}
\label{sec:difficuly}

We implement a highly customizable difficulty control framework that allows fine-grained adjustment of both semantic complexity and context length. The framework offers multiple configuration dimensions, as shown in Figure~\ref{fig:difficulty_config}.

This flexible framework allows for precise control of task difficulty through various combinations of these configurations. For our benchmark, we define three standard difficulty levels that progressively increase semantic and length complexity:  (I) \textbf{Easy} level masks attributional citations, where distractors are sampled randomly. (II) \textbf{Medium} level masks descriptive citations, where distractors are also sampled randomly. Both easy and medium level use the full paper for the question paper and candidates.
(III) \textbf{Hard} level masks descriptive citations, but candidates are sampled from its nearest citations. Additionally, we use only the body of the paper for candidates.
 
\paragraph{Final Dataset}
The final dataset consists of 300 questions evenly distributed across three difficulty levels, with each question containing four candidates. To ensure diversity, we limit at most five questions per paper. All papers, including question paper and candidates, are limited to 100,000 characters to accommodate model context limitations, with questions formatted using the template in \Cref{fig:prompt}.

\begin{table}[t]
\small
\centering
\resizebox{1.0\linewidth}{!}{
\begin{tabular}{lclccc}
\toprule
Model & Context & Price & Easy & Medium & Hard \\
\midrule
Llama 3.1 70B    & 128K  &  \$0.65*         & 0.37 & 0.29 & 0.16 \\
Llama 3.1 8B     & 128K  & \$0.05*          & 0.30 & 0.24 & 0.25 \\
Llama 3.3 70B    & 128K  & \$0.65*          & 0.37 & 0.36 & 0.23 \\
Qwen2.5 7B 1M    & 1M & \$0.05*             & 0.52 & 0.37 & 0.29 \\
GLM 4 9B             & 128K & \$0.05*       & 0.67 & 0.50 & 0.35 \\
Claude 3.5 Haiku & 200K & \$0.80            & 0.77 & 0.61 & 0.42 \\
GPT 4o                    & 128K & \$2.50   & 0.95 & 0.72 & 0.50 \\
GPT 4o Mini               & 128K & \$0.15   & 0.81 & 0.56 & 0.48 \\
\bottomrule
\end{tabular}
}
\caption{Model performance across difficulty levels. Price shown is per 1M tokens. * denotes price calculated based on third-party inference service.}
\label{tab:model_performance}
\end{table}

\section{Experiments}

% \subsection{Baselines}
We benchmark several open-source and proprietary models with long-context capabilities ($\geq 128k$ tokens) on SCALAR.% based on the LibrA-Eval\footnote{\href{https://github.com/LibrAIResearch/libra-eval}{https://github.com/LibrAIResearch/libra-eval}} framework.
The list of models we evaluate are introduced in \Cref{app:eval}.



\subsection{Main Results}
\Cref{tab:model_performance} presents model performance across difficulty levels.  Generally, all LLMs' performance downgrades when difficulty increases. On the easy level, \name\ can already differentiate LLMs' long context capability, where the best model GPT-4o achieves 95\% accuracy, while the lowest-performing models hover around 30-37\%, compared to the random baseline of 25\%.
For the hard level set, half of the models achieve random performance, and even current SOTA models obtain less than half the correct results, demonstrating how challenging our dataset is. 

The hard level proves particularly challenging - even state-of-the-art models like GPT-4o achieve only 50\% accuracy, while most other models perform near random chance. Notably, model size does not directly correlate with performance: smaller models optimized for long-context processing (Qwen2.5-1M-7B at 52\% and GLM-4-9B at 67\% on easy tasks) outperform larger models like Llama-3.1-70B (37\%). This suggests that architectural choices and training objectives for long-context understanding \citep{yang2025qwen2} may be more crucial than model scale alone \citep{dubey2024llama,glm2024chatglm}.


\begin{table}[t]
    \centering
    \resizebox{\linewidth}{!}{
    \begin{tabular}{cccccc}
    \toprule
        \multirow{2}{*}{Question}  & \multirow{2}{*}{Candidate} & \multicolumn{2}{c}{Easy} & \multicolumn{2}{c}{Hard} \\
        \cmidrule{3-4} \cmidrule{5-6}
        & & GPT & Qwen & GPT & Qwen \\
        \midrule
        Section& Concise & 0.86 & 0.88 & 0.58 & 0.48\\
        Full   & Concise & 0.86 & 0.72 & 0.52 & 0.40\\
        Section& Full    & 0.80 & 0.40 & 0.64 & 0.22\\
        Full   & Full    & 0.80 & 0.42 & 0.50 & 0.24\\
        \bottomrule
    \end{tabular}}
    \caption{Impact of context length on model performance. Results compare GPT-4o-mini and Qwen2.5-7B-Instruct-1M across different context configurations. Question context varies between full paper and section-only, while candidate context varies between full paper and title+abstract-only. Semantic difficulty is controlled: \textbf{Easy} level uses attributional citations with random candidates, while \textbf{Hard} level uses descriptive citations with nearest-neighbor candidates.}
    \label{tab:length}
\end{table}


\begin{table}[t]
    \centering
    \resizebox{\linewidth}{!}{
    \begin{tabular}{llcccc}
    \toprule
        Cite Type& Sampling &Candidate & GPT & Qwen  \\
        \midrule
        Attributional &Random & Full &  0.82 & 0.42\\
        Attributional &Nearest& Full &  0.82 & 0.38\\
        Descriptive   &Random & Full &  0.56 & 0.34\\
        Descriptive   &Nearest& Full &  0.52 & 0.22\\
        Descriptive   &Nearest& Body &  0.44 & 0.28\\
        \bottomrule
    \end{tabular}}
    \caption{Impact of semantic reasoning difficulty on model performance. Results of GPT-4o-mini and Qwen2.5-7B-Instruct-1M. Cite type, sampling methods, and candidate type are introduced in \Cref{fig:difficulty_config}.}
    \label{tab:semantic}
\end{table}


\section{Analysis}

Following our discussion of difficulty control in \Cref{sec:difficuly}, where we established three difficulty levels (easy, medium, and hard), we now conduct a detailed analysis of how different configuration combinations affect task difficulty. We categorize these configurations into two primary dimensions: context length and semantic complexity. For each configuration setting in our analysis, we evaluate model performance on a sample of 50 questions.


\paragraph{Context Length}
We analyze the impact of context length by varying both question and candidate paper representations, as shown in \Cref{tab:length}. For questions, we compare using either the full paper or just the section containing the masked citation. For candidates, we compare using either the full paper or just the abstract. This creates four distinct length configurations while controlling for semantic difficulty.

The results show that both models generally perform better with more concise contexts. GPT-4o-mini maintains relatively stable performance across configurations, while Qwen2.5-7B shows significant degradation when given full candidate papers (dropping from 88\% to 40-42\% in the easy setting). This pattern persists in the hard setting, though with overall lower performance, suggesting that focused, relevant context may be more beneficial than comprehensive but potentially noisy full-paper information.

\paragraph{Semantic Complexity}
\Cref{tab:semantic} demonstrates how different semantic factors affect model performance on citation prediction. We analyze three key dimensions: citation type (attributional vs. descriptive), candidate sampling method (random vs. nearest), and candidate representation (full paper vs. body only).

The results show a clear hierarchy of difficulty. Both models perform best with attributional citations (GPT: 82\%, Qwen: 38-42\%), likely due to their more straightforward nature. Performance drops significantly with descriptive citations, particularly when combined with nearest-neighbor sampling.
While GPT-4o-mini maintains above-random performance across all configurations, Qwen's performance on the most challenging setting (descriptive, nearest sampling, full paper) drops to 22\%, below random chance (25\%). Interestingly, when using body-only candidates, Qwen's performance improves slightly to 28\%, suggesting that the previous drop might be due to context length limitations rather than semantic difficulty alone. These patterns validate our benchmark's difficulty levels while highlighting the importance of considering both semantic and context length effects.

%(GPT: 52\%, Qwen: 22\%). This suggests that distinguishing between semantically similar papers is more challenging than identifying random candidates. Using only the paper body further increases difficulty, indicating the importance of abstract and introduction sections for citation understanding. These patterns validate our benchmark's difficulty levels and show how semantic complexity can be systematically controlled.


\section{Related Work}\label{related_work}

\paragraph{Long-context Evaluation} Many approaches have been proposed to evaluate the ability of language models to utilize a longer context \citep{inftybench,loogle,bamboo,wang2024leavedocumentbehindbenchmarking,song2024countingstarsmultievidencepositionawarescalable}. Real-world evaluations cover long-document QA and summarization \citep{zeroscrolls,laban2024summaryhaystackchallengelongcontext}, mathematics and code understanding \citep{leval,zhao2024docmathevalevaluatingmathreasoning,wang2024mathhayautomatedbenchmarklongcontext}, domain specific analysis \citep{reddy-etal-2024-docfinqa}, and retrieval tasks \citep{helmet}, in varies languages \cite{qiu2024clongevalchinesebenchmarkevaluating}, formats \citep{zhang2024marathonracerealmlong}. 
These benchmarks often repurpose existing corpus, raising concerns over its difficulty and data contamination. Meanwhile, benchmarks using synthetic data focus on atomic abilities such as retrieval \citep{needle}, state tracking \cite{babilong}, data aggregation \citep{ruler}, multi-hop reasoning \citep{longbench} like code understanding. These benchmarks are controllable although they deviate from real-world usage. Our benchmark combines the two approaches by mining supervision signals from real-world scholarly data with frequent updates, which enables a more realistic and controllable evaluation setting of long context.    


\paragraph{Citation-based Benchmarking} Although scholar literature corpus has been extensively used in language model pretraining \citep{s2orc, pile}, its potential to evaluate long context utilization is not fully explored. A number of datasets focus on generating and recommending citations \citep{citebench, F_rber_2020, 10.1007/978-3-030-99736-6_19}. \citet{litsearch} creates a retrieval benchmark by constructing questions for inline citations using GPT-4 and manually creating questions. There are benchmarks testing models' abilities to answer questions based on papers.  QASPER \citep{qasper} focuses on answering questions about NLP papers, and LitQA \citep{Lala2023PaperQA} examines models' knowledge of biology works.  However, these works primarily focus on understanding individual papers or concepts rather than evaluating long-context comprehension across multiple papers. Our work bridges this gap by leveraging citation networks to create challenging long-context evaluation scenarios that better reflect how researchers process and connect scientific literature.


\section{Conclusion}

In this work, we introduced SCALAR, a novel benchmark designed to evaluate LLMs long-context understanding while mitigating data contamination. By leveraging citation networks from scholarly papers, automatically generates high-quality ground truth labels while controlling task difficulty. Our experiments with state-of-the-art LLMs reveal that while models can effectively handle simple citation matching, they struggle with deeper comprehension of complex, context-rich references. SCALAR offers a sustainable and evolving benchmark to track advancements in long-context processing, providing insights for future model development.

\section*{Limitation}
SCALAR currently focuses on cloze-style citation-matching tasks, which may not fully assess broader comprehension and reasoning abilities. Additionally, its scope is limited to computer science, restricting its applicability to other academic domains. To address these limitations, we plan to introduce diverse evaluation formats, while also expanding SCALAR to fields like biomedical and legal research for a more comprehensive assessment of long-context understanding.



% Entries for the entire Anthology, followed by custom entries
\bibliography{anthology,custom}
\bibliographystyle{acl_natbib}

\appendix


\begin{figure*}[t]
	\small
	\begin{tcolorbox}[colframe=white, left=2.5mm, right=1.5mm]
You are given a paper with a placeholder ``**[MASKED\_CITATION]**'' in
its content. Your task is to select the most appropriate reference
from the provided reference list to replace the mask.\\
- The paper content is enclosed within <Paper> and </Paper>.\\
- The reference list is enclosed within <References> and </References>,\\
with each reference wrapped in <Candidate> and </Candidate>.\\
- After selecting the best-suited reference, output the index of that
reference in the following format:\\
<answer>index</answer>.\\
<Paper>\\
\blue{... BERT (*[MASKED\_CITATIO]**) or ...} \\
</Paper>\\
<References>\\
<Candidate>Candidate [0]:\\
\blue{... candidate content ...}\\
</Candidate>\\
<Candidate>Candidate [1]:\\
\blue{... candidate content ...}\\
</Candidate>\\
<Candidate>Candidate [2]:\\
\blue{... candidate content ...}\\
</Candidate>\\
<Candidate>Candidate [3]:\\
\blue{... candidate content ...}\\
</Candidate>\\
</References>\\
Remember to output the index of the selected reference enclosed within
<answer> and </answer>.\\
	\end{tcolorbox}
	\caption{The prompt template used for the questions.}
	\label{fig:prompt}
\end{figure*}

\section{Prompt use in SCALAR}
\label{sec:appendix_prompt}
The prompt template used for the questions is in \Cref{fig:prompt}.


\section{Evaluation Models}
\label{app:eval}
The list of models we evaluate includes:

\paragraph{Open-source models}

(1) Llama~\citep{touvron2023llama, touvron2023llama2, dubey2024llama} is a series of the most influential open-source LLMs developed by Meta.
We utilize the \textit{8B, 70B, and 405B Llama3.1-Instruct}  and the \textit{70B Llama3.3-Instruct} for evaluation.
(2) Qwen2.5-1M~\citep{yang2025qwen2} is a long-context variant of Qwen2.5, supporting a 1M-token context. 
We test the \textit{Qwen2.5-7B-Instruct-1M} version of this model.
% (3) \textit{Mistral-Nemo-Base-2407}, an instruction-fine-tuned version of Mistral NeMo, supporting a 128k-token context.
(3) \textit{GLM-4-9B-Chat}, the human preference-aligned version of GLM-4-9B in the GLM-4 series~\citep{glm2024chatglm}, launched by Zhipu AI.
% (5) \textit{DeepSeek-R1}~\cite{guo2025deepseek} is a reinforcement learning-enhanced model that combines cold-start data with iterative RL fine-tuning, achieving performance comparable to OpenAI-o1-1217 across various tasks.

\paragraph{Proprietary models}
(4) \textit{GPT-4o} and \textit{GPT4o-mini}, two proprietary long-context models from OpenAI.
(5) \textit{claude-3-5-haiku-20241022}, the fastest model for daily tasks in Anthropic's Claude.


\end{document}

\bibliographystyle{acl_natbib}

\clearpage
\appendix
\newpage
\section{Experimental Setups}\label{sec:app_setup}
\subsection{Inference Setting} \label{sec:app_inference_setting}
The datasets are used under the MIT License and with their intended use.
For models, we use \texttt{LLaMA} checkpoints from Hugging Face Transformers \cite{wolf2020transformers}.
We run experiments with up to $8 \times$ RTX 4090 24G GPUs. e.
Due to memory constraints, we use Float16 precision for inference, with each run taking around 1\textasciitilde4 hours, depending on the model and data size. 
The prompt template for GSM8K and EZstance are in Appendix Table \ref{tab:app_prompt}.
For ICL, we set the number of demonstrations to $8$ following \cite{li2023unified,rubin2021learning}.


\begin{table}[h]
\small
\centering
\begin{tabular}{@{}p{1.5cm}p{6cm}@{}}
\toprule
\textbf{Dataset}                   & \multicolumn{1}{c}{\textbf{Prompt Template}}                                                                    \\ \midrule
\multirow{2}{*}{\textbf{GSM8K}}    & \texttt{Question:} \colorbox{yellow}{\texttt{\{math\_problem\}}}                                                                            \\
                                   & \texttt{Answer:} \colorbox{yellow}{\texttt{\{step\_by\_step\_answer\}.}}                                                                    \\ \midrule
\multirow{3}{*}{\textbf{EZStance}} & \texttt{Text:} \colorbox{yellow}{\texttt{\{sentence\}}}                                                                                           \\
                                   & \texttt{Question: Which stance-"favor," "against," or "neutral"-does the above text express toward} \colorbox{yellow}{\texttt{\{target\}}}? \\
                                   & \texttt{Answer:} \colorbox{yellow}{\texttt{\{stance\}}}.                                                                                    \\ \bottomrule
\end{tabular}
\caption{Prompt templates for the two datasets. \colorbox{yellow}{highlighted} parts are inputs.} \label{tab:app_prompt}
\end{table}

\subsection{Implementation Details of IRT} \label{sec:app_irt_implementation}
\noindent \textbf{Dataset Construction.}
The dataset for the IRT model is built upon LLM outputs. 
First, we construct Oracle demonstrations using the approach described in \cref{sec: measure}. 
Then, we run LLMs using prompts in Appendix Table \ref{tab:app_prompt} 
 in different settings (DP or ICL). 
The outputs are represented as tuples consisting of $<$\texttt{model\_id}, \texttt{example\_id}, \texttt{input}, \texttt{output}, \texttt{setting}, \texttt{label}$>$.  
This results in total $2$ (Direct prompting or ICL setting) $\times$ $M$ (Number of LLMs) $\times$ $N$ (Number of queries) instances, where $M=8$, $N_{\rm {\tiny GSM8K}}=1319, N_{\rm {\tiny EZStance}}=6703$.  
We further split them into 80\% training set, 10\% validation set, and 10\% test set.


\noindent \textbf{Training Setup.}
We set the dimension of latent traits $\boldsymbol{\theta}, \boldsymbol{\alpha}, \boldsymbol{\theta}^{c},$ $\boldsymbol{\alpha}^{c}$ to 32.
Queries are encoded with SBERT (\texttt{paraphrase-mpnet-base-v2}) with an embedding size of 768.
We train all the models for 10 epochs with a learning rate of $2e-4$ and batch size of 16. 
Traditionally, IRT is optimized by marginalized maximum likelihood estimation \cite{chalmers2012mirt}.
However, this does not scale well to large datasets \cite{lalor2023py}.
We follow \citet{gor2024great} to use Adam \cite{kingma2014adam} to optimize our model.
The best model is selected based on the performance on the validation set.


\subsection{Details of Fine-tuning} \label{sec:app_finetune}
We fine-tune \texttt{LLaMA-3-8B-Instruct} to evaluate our curriculum learning algorithm (\cref{App2}).
Since \texttt{LLaMA} models might already be fine-tuned on the training set of GSM8K \cite{zhang2024careful}, we randomly sample 1,000 instances from the test set for fine-tuning and use the remaining 319 instances for evaluation.
The EZStance dataset is curated after the release of \texttt{LLaMA-3} and, therefore, has no such concern.
We sample 5,000 examples from the training set for fine-tuning and directly evaluate the model on the test set. 
With the scheduler in Algorithm \ref{algo:cl_scheduler}, we split the dataset into $3$ buckets and fine-tune the model on each bucket for $2$ epochs with a learning rate of $1e-5$ and batch size of $4$. 


\section{Additional Results}\label{sec:app_add_results}
\subsection{Additional Results of IRT}
The accuracy of IRT models is in Table \ref{tab:app_irt_acc}. 
Note that baseline models are not trained on ICL data and therefore their accuracy is not indicative. 
We report it only for the completeness of the results. 
\begin{table}[h]
\setlength\tabcolsep{4pt}
\small
\centering
\begin{tabular}{@{}lcccccc@{}}
\toprule
\multirow{2}{*}{\textbf{Model}} & \multicolumn{3}{c}{\textbf{GSM8K}} & \multicolumn{3}{c}{\textbf{EZStance}} \\ \cmidrule(l){2-7} 
                                & DP      & ICL      & Overall     & DP       & ICL       & Overall      \\ \midrule
\textsc{Irt}$_{\rm 1Pl}$                       & 69.1      & 39.6     & 56.7        &   76.1         &   45.6        &   63.1           \\
\textsc{Irt}$_{\rm 2PL}$                       & 70.4      & 40.2     & 58.7        &   75.3         &  46.4         &   63.0           \\
MIRT                            & 68.9      & 47.2          & 59.8             & 76.6           & 45.9          & 63.5            \\ 
MIRT$_{\textsc{ICL}}$                            & \textbf{77.4}      & \textbf{78.4}     & \textbf{77.9}        &    \textbf{77.0}       & \textbf{68.5}          &   \textbf{72.8}           \\ \bottomrule
\end{tabular}
\caption{Performance (Accuracy $\%$ ) of various IRT models. The best results are in \textbf{bold}.} \label{tab:app_irt_acc}
\end{table}

\subsection{Additional Results of \textsc{SelIcl}}
The result of \textsc{SelIcl} on the EZStance dataset is in Appendix Figure \ref{fig:ezstance_sel_icl}.

\begin{figure*}[h]
    \centering
    \includegraphics[width=\textwidth]{figures/ez_stance_sel_icl.png}
    \caption{Results of \textsc{SelIcl} on EZStance. See detailed explanations in Figure \ref{fig:gsm8k_sel_icl}.}
    \label{fig:ezstance_sel_icl}
\end{figure*}

\end{document}
