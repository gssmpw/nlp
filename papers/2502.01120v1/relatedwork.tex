\section{Related Work}
\label{sec:related}

%
We focus on Lipschitz decomposition and on capped decomposition,
that was introduced in \cite{FN22}, 
but the literature studies several different decompositions of metric spaces
into low-diameter clusters, see e.g.~\cite{MN07, Filtser24}.
In particular, the notion of padded decomposition~\cite{Rao99, KL07}
is closely related and was used extensively,
see for example~\cite{Rao99, Bartal04, LN03draft, MN07, KLMN05}.
%
While a Lipschitz decomposition guarantees that nearby points are likely to be clustered together,
a padded decomposition guarantees that each point is, with good probability,
together with all its nearby points in the same cluster.
%
Remarkably, if a metric space admits a padded decomposition then it admits
also a Lipschitz decomposition with almost the same parameters~\cite{LN03draft},
however the other direction is not true, as demonstrated by $\ell_2^d$.
%


%
The problem of computing efficiently the optimal decomposition parameters
for an input metric space $(X,\rho)$ was studied in~\cite{KR11}.
Specifically for Lipschitz decomposition, %
they show that $\beta^*(X)$ can be $O(1)$-approximated in polynomial time (in $n$).