\section{Literature Review}
Network simulation is a widely used technique for evaluating communication protocols, security mechanisms, and performance characteristics in a controlled environment. It provides scalability and cost-effectiveness while eliminating the need for physical infrastructure. However, traditional network simulations often lack the realism observed in testbeds that employ real devices and applications \cite{gomez2023survey}. Various network simulation tools exist, each offering distinct capabilities, making selection and implementation a challenge for users \cite{smera2022networks}. While many frameworks focus on protocol-level interactions, they often overlook application-specific behavior, which plays a crucial role in real-world network performance.

Several simulation frameworks have been developed to bridge the gap between abstract network models and real-world deployments. For instance, the \textit{Network Simulation Bridge (NSB)} facilitates integration between real applications and network simulators, ensuring scalability while minimizing performance overhead \cite{kuttivelil2023network}. Similarly, the \textit{Network Research Simulator (NRS)} provides a flexible programming interface to conduct experiments on network robustness, including simulations of large-scale attacks and predictive analytics using AI \cite{marzo2022network}. Despite their contributions, these approaches do not fundamentally model application-level behavior but rather provide simulation support for network-layer functionalities.

Efforts have been made to incorporate real application traffic into network simulations. The use of captured application-specific traffic flows allows for the modeling of application performance under different network conditions \cite{zacks2021systems}. Studies on web traffic classification have demonstrated that application-level traffic exhibits distinct statistical properties that can be leveraged for simulation and anomaly detection \cite{karayaka2022application}. However, such approaches rely on empirical data collection rather than a fully statistical modeling approach, making them less adaptable to varying conditions. Furthermore, solutions like Android client emulation \cite{hetu2014similitude} provide a means to simulate user interactions but require resource-intensive setups.

A key limitation of existing methodologies is the lack of a true application-layer modeling approach. While application-level traffic data has been correlated with network-level traffic for improved insight \cite{sharma2021correlating}, most studies still treat applications as static sources of predefined traffic patterns rather than dynamic entities that influence network behavior. Probabilistic models generated from historical network traffic have been proposed to emulate application statistics on virtualized environments \cite{ganapathi2020data}, but these models do not fully replicate the underlying application behavior. Similarly, simulation tools for enterprise multicast \cite{wu2022ip} focus on protocol efficiency rather than the behavioral intricacies of applications.

The challenge of integrating real application behavior into network simulations has been long recognized. Existing simulators like NS3 offer extensive functionality for protocol testing, but their credibility is often questioned due to the lack of real-world application interactions \cite{rampfl2013network}. Other approaches attempt to introduce real applications into simulated networks \cite{ meszaros2019inet}, but they require hardware-intensive setups and are not always easily replicable. While existing network simulation techniques provide insights into network operations, they do not fundamentally simulate how applications influence network behavior \cite{kuttivelil2023network}.

Given these limitations, this paper proposes \emph{a fully statistical approach that replicates application behavior and builds network behavior from it}. Unlike traditional methods, which either simulate network traffic without understanding application behavior or require empirical traffic captures, this approach \emph{models application behavior probabilistically using statistical distributions}, allowing it to dynamically generate network traffic patterns. This methodology is \emph{easily replicable}, as it does not depend on complex data collection processes or specialized hardware. Additionally, the statistical nature of the approach makes it \emph{lightweight and adaptable}, enabling its use across different network conditions without significant overhead. By focusing on application-layer behavior as the foundation of network simulation, this work advances the realism and usability of network simulation tools in monitoring, anomaly detection, and performance analysis.

\begin{comment}