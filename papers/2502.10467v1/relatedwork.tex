\section{Related Work}
In this section, we introduce three common music notation systems—MIDI, MusicXML, and ABC Notation—and use \textit{Boat on Tai Lake}, a Chinese folk tune, as an example to demonstrate each format.

\begin{figure}[htbp]
    \centering
    \subfloat[Sheet Music]{
        \begin{minipage}[t]{\textwidth}
            \centering
            \includegraphics[width=\textwidth]{images/1_a_sheet_music.png}  % Replace with your image file
            % \label{fig:sheet-music}
        \end{minipage}
    } \quad
    \centering
    \subfloat[MIDI (ASCII Code Representation)]{
        \begin{minipage}[b]{0.46\textwidth}
            \lstinputlisting[basicstyle=\fontsize{7}{9}\selectfont\ttfamily]{notations/1_b_midi.txt}
            % \label{code:midi}
        \end{minipage}
    } \quad
    \subfloat[MusicXML]{
        \begin{minipage}[b]{0.47\textwidth}
            \lstinputlisting[basicstyle=\fontsize{6}{8}\selectfont\ttfamily, language=XML]{notations/1_c_musicxml.xml}
            % \label{code:musicxml}
        \end{minipage}
    }
    \qquad
    \subfloat[ABC Notation]{
        \begin{minipage}[b]{0.47\textwidth}
            \lstinputlisting[basicstyle=\fontsize{7}{9}\selectfont\ttfamily]{notations/1_d_abc.txt}
            % \label{code:abc}
        \end{minipage}
    } \quad
    \subfloat[YNote]{
        \begin{minipage}[b]{0.46\textwidth}
            \lstinputlisting[basicstyle=\fontsize{7}{9}\selectfont\ttfamily]{notations/1_e_ynote.txt}
            % \label{code:ynote}
        \end{minipage}
    }
    \caption{\textit{Boat on Tai Lake} in Various Music Notations}
    \label{fig:code-table}
\end{figure}

\subsection{MIDI}
MIDI (Musical Instrument Digital Interface) \cite{de2017understanding} is a standardized protocol for communication between musical devices, first developed in 1983 by several electronic music equipment manufacturers. Unlike audio signals, MIDI transmits digital control information, enabling electronic instruments, computers, synthesizers, and audio equipment to communicate with one another. Its primary purpose is to send instructions, such as which notes to play, their velocity, and timbre, guiding devices on how to generate sound.

The MIDI protocol specifies a transmission rate of 31.25 kbps, which is fast enough to ensure low-latency signal transmission necessary for real-time musical control. MIDI data is generally divided into three main categories: Channel Messages, System Messages, and Control Change Messages.

\begin{itemize}
    \item Channel Messages convey essential performance information, such as note-on and note-off events, note velocity, and other parameters needed for playing back notes.
    \item System Messages handle global controls, including timing synchronization and system resets.
    \item Control Change Messages are used to make real-time adjustments to sound characteristics, such as pitch bends, modulation, and pedal controls.
\end{itemize}

MIDI technology is widely used across various fields, including music production, live performance control, sheet music generation, music education, film scoring, and sound design. It is also integral to automated and interactive music systems. Thanks to its efficient data transmission and standardized protocol, MIDI allows musicians, producers, and sound designers to easily compose music, control effects, and synchronize digital audio elements, providing powerful support for both music creation and performance. Figure 1\_b shows an example of a MIDI file for \textit{Boat on Tai Lake} represented in ASCII code.

% \begin{figure}[!h]
% \centering
% \includegraphics[width=\textwidth]{images/midi.png}
% % where an .eps filename suffix will be assumed under latex, 
% % and a .pdf suffix will be assumed for pdflatex; or what has been declared
% % via \DeclareGraphicsExtensions.
% \caption{MIDI Example for \textit{Boat on Tai Lake}}
% \label{fig:midi}
% \end{figure}

\subsection{MusicXML}
MusicXML (Music Extensible Markup Language) \cite{MAKEMUSIC,MUSICNOTATIONCOMMUNITYGROUP} is a standard open format based on XML, creating a universal format for common Western musical notation used for the interchange and distribution of digital scores. Musical information can be utilized by score programs, sequencers, other performance applications, music education software, and music databases. The hierarchical structure of MusicXML reflects the structure of music itself — from higher-level elements like pieces and sections, down to lower-level details like notes and beats.

MusicXML has two top-level elements, representing the partwise and timewise score formats, which are determined by the root element. The top-level element also defines the structural framework of the lower-level elements. If the root element is <score-partwise>, the musical part is the primary structure, containing multiple <part> elements, each of which includes several <measure> elements. On the other hand, if the root element is <score-timewise>, the measure is the primary structure, containing multiple <measure> elements, each of which includes several <part> elements.

The actual music in the score is represented by lower-level elements, each of which contains a group of music data. These elements are composed of zero or more <note>, <backup>, <direction>, <attributes>, <sound>, and other elements. Sub-elements like <key> and <time> within <attributes>, as well as <pitch> within <note>, define the basic structure of each measure and the specific details of the notes.

The score header refers to the elements located at the top of the score, providing basic information and settings about the piece. This includes the title, composer, arranger, performer, copyright details, and more. It is typically composed of elements like <work>, <movement-number>, <movement-title>, <part-list>, and others.

Overall, MusicXML provides a standardized format that is widely supported by many applications and offers high readability for developers. Its ability to thoroughly describe elements like notes, harmony, and rhythm allows for a more complete representation of the score. However, the detailed information also leads to larger file sizes, and in some cases, the structure may be overly complex or verbose. Additionally, MusicXML can be more difficult for musicians to understand. Figure 1\_c shows an example of MusicXML for \textit{Boat on Tai Lake}.

% \begin{figure}[!h]
% \centering
% \includegraphics[width=\textwidth]{images/musicxml.png}
% % where an .eps filename suffix will be assumed under latex, 
% % and a .pdf suffix will be assumed for pdflatex; or what has been declared
% % via \DeclareGraphicsExtensions.
% \caption{MusicXML Example for \textit{Boat on Tai Lake}}
% \label{fig:musicxml}
% \end{figure}

\subsection{ABC Notation}
ABC Notation \cite{ABCNotation} is a simple, text-based music notation system that is relatively easy for humans to read. Over the past three decades, it has been used to represent tens of thousands of musical pieces, which are commonly referred to as ABC tunes.

An ABC tune consists of two main parts: the tune header and the tune body. The tune header contains basic information about the music, including reference number, title, composer, origin, region, meter, unit note length, tempo, parts, transcription, key, rhythm, and other metadata such as background information and instructions.

The tune body records the actual musical content. Notes are represented by their English names ("A", "B", "C", etc.), and octaves are indicated with commas for lower octaves and apostrophes (or lowercase letters) for higher octaves. For example, a "C" in a lower octave is written as "C,", while a "C" in a higher octave is written as "C'" or "c".

The note duration is defined by the L value in the tune header. For instance, if L is set to 1/8, then "C" represents an eighth note, "C/2" represents a sixteenth note, and "C2" represents a quarter note.

Bars are separated by the "|" symbol, and repeated sections can be indicated using ":|". Additionally, "K" and "V" symbols are used to represent key signatures and clefs, and custom symbols can be defined by the user.

Although ABC Notation is more readable than traditional sheet music, its flexible and varied representations can lead to the same piece being notated in multiple ways. This flexibility, along with the presence of optional information, makes it difficult to parse automatically with high accuracy. An example of ABC Notation is shown in Figure 1\_d.

% ---- Introduction of YNote ----