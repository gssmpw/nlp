\section{General Benchmarks}
\label{sec:app_benchmark}
We list all the general benchmarks involved, including five generative and four multi-choice datasets:

\begin{table*}[t]
\centering
\resizebox{\textwidth}{!}{
\begin{tabular}{lcccccccccc}
\toprule
\quad & \multicolumn{5}{c}{\textbf{Generative}} & \multicolumn{4}{c}{\textbf{Multi-Choice}} & \textbf{Avg.} \\
\cmidrule(lr){2-6} \cmidrule(lr){7-10} \textbf{\makecell[l]{\quad\\Model}} & \textbf{\makecell[c]{GSM8K\\8-shot}} & \textbf{\makecell[c]{Math\\4-shot}} & \textbf{\makecell[c]{GPQA\\0-shot}} & \textbf{\makecell[c]{IFEval\\3-shot}} & \textbf{\makecell[c]{MMLU-Pro\\5-shot}} 
& \textbf{\makecell[c]{MMLU\\0-shot}} & \textbf{\makecell[c]{PiQA\\3-shot}} & \textbf{\makecell[c]{MUSR\\0-shot}} & \textbf{\makecell[c]{TruthfulQA\\3-shot}} & \textbf{/} \\
\midrule
\textsc{ChatGLM2-6B} & - & - & - & 10.79 & - & 24.28 & 53.59 & 36.51 & 25.21 & - \\
\textsc{CharacterGLM-6B (ChatGLM2-6B)} & - & - & - & 14.75 & - & 24.57 & 55.55 & 36.64 & 23.87 & - \\
\textsc{Humanish-Llama3.1-8B (Llama3.1-8B-IT)} & 71.72 & 33.42 & \textbf{21.65} & 55.16 & \underline{43.72} & 67.05 & \textbf{83.24} & 41.4 & 37.94 & 50.59 \\
\textsc{Yi-1.5-9B} & 64.14 & 29.98 & 15.18 & 33.57 & 38.97 & 68.84 & 81.83 & 42.72 & 32.19 & 45.27 \\
\textsc{Peach-9B-Roleplay (Yi-1.5-9B)} & 60.35 & 18.4 & 13.62 & 41.49 & 36.29 & 65.97 & 80.3 & 42.2 & 26.68 & 42.81 \\
\midrule
\textsc{LLaMA3.1-8B} & 48.98 & 17.78 & 12.5 & 16.67 & 35.21 & 63.27 & 81.77 & 38.1 & 28.52 & 38.09 \\
\textsc{LLaMA3.1-8B-Instruct} & 77.41 & 34.1 & 12.72 & \underline{57.67} & 40.77 & 68.1 & 82.1 & 39.81 & 36.47 & 49.91 \\
\textsc{LLaMA3.1-8B-RoleMRC-SFT} & 56.18 & 12.78 & 19.64 & 42.09 & 31.58 & 59.3 & 82.64 & 40.34 & 35.01 & 42.17 \\
\textsc{LLaMA3.1-8B-RoleMRC-DPO} & 58.53 & 13.5 & \underline{20.09} & 46.64 & 31.8 & 59.96 & \underline{82.7} & 39.42 & 37.33 & 43.33 \\
\midrule
\textsc{Qwen2.5-7B} & 78.7 & \underline{36.78} & 16.74 & 38.25 & \textbf{44.87} & \underline{71.75} & 81.23 & 44.31 & 38.8 & 50.16 \\
\textsc{Qwen2.5-7B-Instruct} & \textbf{81.2} & \textbf{40.28} & 13.39 & \textbf{65.71} & 40.85 & \textbf{71.76} & 80.25 & 42.86 & \textbf{47.86} & \textbf{53.8} \\
\textsc{Qwen2.5-7B-RoleMRC-SFT} & 78.54 & 32.7 & 16.52 & 42.81 & 43.43 & 71.19 & 80.63 & \underline{45.11} & 37.58 & 49.83 \\
\textsc{Qwen2.5-7B-RoleMRC-DPO} & \underline{79.38} & 32.72 & 18.97 & 47.96 & 43.39 & 71.21 & 80.36 & \textbf{45.37} & \underline{39.41} & \underline{50.97} \\
\bottomrule
\end{tabular}}
\vspace{-2mm}
\caption{General evaluation comparing five generative and four multiple-choice benchmarks. The best scores for each metric are \textbf{bold}, and the second best are \underline{underlined}. Details of all benchmarks are introduced in Appendix\,\ref{sec:app_benchmark}. \texttt{CharacterGLM-6B}, \texttt{Humanish-Llama3.1-8B}, and \texttt{Peach-9B-Roleplay} are fine-tuned from their basis ChatGLM2\,\cite{glm2024chatglm}, Llama3.1-8B-Instruct, and Yi-1.5-9B\,\cite{young2024yi}, respectively. We annotate this information in the brackets right after the model names.}
\label{tab:general_benchmark}
\vspace{-5mm}
\end{table*}

\vspace{-2mm}
\paragraph{Generative:}
\begin{itemize}
[leftmargin=*,noitemsep,topsep=0pt]
\item \underline{GSM8K}: A primary level math dataset of 1.3k questions\,\cite{cobbe2021training}. We use 8-shot in-context exemplars, and report exact match score.
\item \underline{Math}: A dataset of 12.5k challenging competition mathematics problems\,\cite{hendrycksmath2021}. We use 4-shot in-context examples and report exact math score across a 5k subset.
\item \underline{GPQA}: 448 hard graduate-Level google-proof questions\,\cite{rein2023gpqa}. 0-shot prompting is used for calculate the flexible math score.
\item \underline{IFEval}: A special instruction-following benchmark with 541 verifiable instructions\,\cite{zhou2023instructionfollowing}. We use 3-shot prompting and report instruction-level strict accuracy.
\item \underline{MMLU-Pro}: A more robust and challenging multi-task language understanding benchmark\,\cite{wang2024mmluprorobustchallengingmultitask} with 12k commonsense questions. We takes a 5-shot testing and report exact match score.
\end{itemize}

\vspace{-2mm}
\paragraph{Multi-Choice:}
\begin{itemize}
[leftmargin=*,noitemsep,topsep=0pt]
\item \underline{MMLU}: A multi-choice benchmark for testing commonsense ability of LLMs, covering 14k questions\,\cite{hendrycks2020measuring}. No in-context exemplars provided, and we present accuracy.
\item \underline{PiQA}: A binary dataset of 1.8k common physical knowledge questions\,\cite{bisk2020piqa}. We report accuracy score of 3-shot prompting.
\item \underline{MUSR}: A dataset for evaluating LLMs on multi-step soft reasoning tasks\,\cite{sprague2024musrtestinglimitschainofthought}. We test all 756 questions with zero-shot prompting and report accuracy.
\item \underline{TruthfulQA}: A testing dataset designed for assessing LLM's recognition of true statements\,\cite{lin-etal-2022-truthfulqa}. We use its multi-choice subset (single-true), evaluating all 817 questions with 3-shot exemplars, reporting accuracy score.
\end{itemize}
The evaluation of general benchmarks are carried through LM-Evaluation-Harness\,\cite{eval-harness}.

\section{Results of General Evaluation} 
We report the results of general evaluation in Table\,\ref{tab:general_benchmark}. In accordance with the last column of Table\,\ref{tab:ood_brief}, except for \texttt{Peach-9B-Roleplay}, all role-playing LLMs have not lost general abilities.

\section{Further Experimental Setup} \label{sec:further_experiment_setup}

This section provides additional details on the setup of our experiments, across training and evaluation:

\paragraph{Training Setup} Results reported are median results over three different runs with different random seeds. We conducted full parameter training using bfloat16 precision. The hyperparameter settings are provided in Table~\ref{tab:hyperparameters_generative}. All the models were trained using either 4 $\times$ A100 80G or 4 $\times$ H100 GPUs~\cite{server}. We use the \texttt{meta-llama/Llama-3.1-8B} and \texttt{Qwen/Qwen2.5-7B} as our base model for RoleMRC SFT models. Our DPO models are further trained based on the SFT models.

\begin{table}[ht]
\small
\centering
\begin{tabular}{|l|c|c|}
\hline
\textbf{Hyperparameter} & \textbf{SFT} & \textbf{DPO} \\
\hline
Learning Rate          & 1e-5     & 2e-5        \\
Batch Size             & 8        & 8           \\
Gradient Accumulation  & 2        & 2           \\
Epochs                 & 1.0      & 1.0         \\
Warmup Ratio           & 0.04      & 0.04          \\
LR Scheduler Type      & cosine   & cosine     \\
Optimizer              & Adam     & Adam       \\
Adam Epsilon           & 1e-8     & 1e-8        \\
DPO $\beta$            & -        & 0.1         \\
Training RoleMRC-mix            & 6h        & 3h         \\
\hline
\end{tabular}
\vspace{-2mm}
\caption{Hyper-parameters setting.}
\label{tab:hyperparameters_generative}
\vspace{-5mm}
\end{table}

\paragraph{API Use for Synthetic Data Generation} We utilized \texttt{gpt-4o}~\cite{gpt4} as the LLM to generate synthetic role-playing data. All parameters were kept at their default values. Manual filtering of the data is done by the authors of this paper as aforementioned in Section\,\ref{sec:meta_pool}. 

\paragraph{Base Models, Computational Environment, and Inference Setup} In this study, we utilized six different models downloaded from HuggingFace Site\,\footnote{\url{https://huggingface.co/models}}. We adhered to the licensing terms of all involved models. For evaluation of instruction following models, we used meta-llama/Llama-3.1-8B-Instruct, meta-llama/Llama-3.1-70B-Instruct from~\cite{llama3}, and Qwen/Qwen2.5-7B-Instruct, Qwen/Qwen2.5-72B-Instruct from~\cite{bai2023qwen,yang2024qwen2}. 

To ensure reproducibility, all evaluations are done using zero-shot prompting with greedy decoding and a temperature of 0. Inference of LLMs is carried out using vLLM~\cite{kwon2023efficient}. 

\section{Structure Tree of Role Profile}
\label{sec:app_tree}
Figure\,\ref{fig:tree} denotes a structure tree of standardized role profile in the role meta-pool (\hyperref[sec:meta_pool]{\textsection \ref{sec:meta_pool}}). And we present a complete role profile in Figure\,\ref{box:role_profile}.

\begin{figure}[!htbp]
\begin{tcolorbox}[
    colback=gray!10,      % Background color
    colframe=gray!80,     % Border color
    title=Template of Standardized Role Profile,
    fonttitle=\bfseries,  % Title font style
    rounded corners,
    boxrule=0.5mm,        % Border thickness
    width=\linewidth
]
\small

\textbf{Role Name and Brief Description}
\begin{itemize}[noitemsep,topsep=0pt]
    \item Identification (several phrases)
    \item Description (sentences)
\end{itemize}

\textbf{Specific Abilities and Skills}
\begin{itemize}[noitemsep,topsep=0pt]
    \item Skill name and description
\end{itemize}

\textbf{Speech Style}

\textbf{Personality Characteristics}
\begin{itemize}[noitemsep,topsep=0pt]
    \item Characteristic name and description
\end{itemize}

\textbf{Past Experience and Background}
\begin{itemize}[noitemsep,topsep=0pt]
    \item Experience
\end{itemize}

\textbf{Ability and Knowledge Boundaries}

\textbf{Speech Examples}
\begin{itemize}[noitemsep,topsep=0pt]
    \item Topic, Narration and Content
\end{itemize}

\end{tcolorbox}
\vspace{-2mm}
\caption{Structure tree of standardized role profile.}
\label{fig:tree}
\end{figure}

\begin{figure*}[!htbp]
\begin{tcolorbox}[
    colback=gray!10,      % Background color
    colframe=gray!80,     % Border color
    title=Final Role Profile,
    fonttitle=\bfseries,  % Title font style
    rounded corners,
    boxrule=0.5mm,        % Border thickness
    width=\linewidth
]
\scriptsize
\textbf{Role Name and Brief Description}
\begin{itemize}
[leftmargin=*,noitemsep,topsep=0pt]
\item \underline{Identification}: Evan Brightcode, the Front-End Prodigy.
\item \underline{Description}: Evan Brightcode is a talented and dedicated software developer specializing in front-end web development. With a mastery of JavaScript, Evan creates seamless, interactive web experiences and is always eager to share his knowledge and ideas about the latest trends and techniques in web development.
\end{itemize}

\textbf{Specific Abilities and Skills}
\begin{itemize}
[leftmargin=*,noitemsep,topsep=0pt]
\item \underline{JavaScript Expertise}: Evan is proficient in JavaScript and its frameworks, such as React and Angular.
\item \underline{User Interface Design}: He excels at designing and implementing user-friendly interfaces.
\item \underline{Web Performance Optimization}: Evan has a keen eye for optimizing websites for speed and performance.
\item \underline{Cross-Browser Compatibility}: Ensures that his applications work flawlessly across all browsers and devices.
\item \underline{Problem Solving}: Evan is skilled at troubleshooting and fixing front-end issues quickly and efficiently.
\end{itemize}

\textbf{Speech Style}
\begin{itemize}
[leftmargin=*,noitemsep,topsep=0pt]
\item Evan speaks with a mix of technical jargon and everyday language to make complex concepts more accessible. His tone is enthusiastic yet precise, reflecting his passion for coding and technology.
\end{itemize}

\textbf{Personality Characteristics}
\begin{itemize}
[leftmargin=*,noitemsep,topsep=0pt]
\item \underline{Analytical}: Evan is methodical and thoughtful in his approach to problem-solving.
\item \underline{Collaborative}: He enjoys working with teams and values input from others.
\item \underline{Up-to-date}: Evan stays on top of the latest developments in the field of front-end web development.
\item \underline{Patient}: Always willing to explain and share his knowledge with others, no matter their level of expertise.
\item \underline{Detail-Oriented}: He pays great attention to the finer points of web design and functionality.
\end{itemize}

\textbf{Past Experience and Background}
\begin{itemize}
[leftmargin=*,noitemsep,topsep=0pt]
\item Evan graduated with a degree in Computer Science and quickly developed a fascination with the front-end aspects of web development.
\item He has worked on numerous high-profile projects for various tech companies, contributing to major front-end overhauls and new feature implementations.
\item Evan has a background in working as a mentor and trainer for aspiring developers, helping them to grasp the principles of JavaScript and web development.
\item In his spare time, Evan maintains a tech blog where he writes tutorials and articles on advanced JavaScript techniques and front-end frameworks.
\end{itemize}

\textbf{Ability and Knowledge Boundaries}
\begin{itemize}
[leftmargin=*,noitemsep,topsep=0pt]
\item While Evan is highly skilled in front-end development, his expertise does not extend to back-end systems or server-side programming. He may not provide in-depth advice on database management or server configuration.
\end{itemize}

\textbf{Speech Examples}
\begin{itemize}
[leftmargin=*,noitemsep,topsep=0pt]
\item \underline{JavaScript Framework}: (Evan sits at his desk, his fingers dancing over the keyboard) Let’s dive into React today. It’s an amazing library for building user interfaces. We’ll start by setting up our environment and then create a few components to get a feel for how it works.
\item \underline{User Interface Design}: (With a smile, Evan pulls up a design mockup) For this interface, we need to focus on simplicity and accessibility. Notice how the buttons stand out clearly? This is to ensure users find navigation intuitive and straightforward.
\item \underline{Web Performance Optimization}: (Evan looks intently at the screen as he runs a page speed test) We need to optimize these images and leverage caching. A faster load time not only improves user experience but also helps with our SEO rankings.
\item \underline{Cross-Browser Compatibility}: (Evan opens various browsers to test the site) It's essential we check how our site performs across different browsers and devices. Consistency is key—you don’t want a great layout in Chrome to fall apart in Safari.
\item \underline{Mentoring Session}: (Evan leans back, his tone encouraging) Don’t worry if you don’t get it right away. JavaScript can be tricky, but with practice, you’ll get the hang of it. Remember, every seasoned developer started where you are now.
\end{itemize}

\textbf{Most Relevant MRC}
\begin{itemize}
[leftmargin=*,noitemsep,topsep=0pt]
\item \underline{Match Score}: 0.6424619555473328
\item \underline{Passages}: 
    \begin{itemize}
    [leftmargin=*,noitemsep,topsep=0pt]
    \item [1]To recruiters it seem to say: “we want somebody who can do everything“, meaning we want to hire somebody being a perfect match no matter what, perhaps even in context of wherever technology would take us.
    \item [2] But Fullstack is not only about knowing how to code in Frontend and Backend. 1  It also includes: 2  Project management and team leading. 3  Creating and using APIs. 4  Knowing how to properly document a project. 5  Having experience in the industry and knowing its ins and outs. 6  Knowing and understanding hardware and what works with what.
    \item [\quad] ...
    \item [10] What does full-stack developer even mean? It is a developer capable of working the different tiers of the stack and who can understand the different paradigms and technologies of which the tiers are comprised at the same time utilise best-practices and embrace requirements and who can consolidate everything into an application (on schedule, budget and with minimal defects and maximum security).
    \end{itemize}
\item \underline{query}: what does full stack developer mean
\item \underline{answer}: The full stack developer is one who is adept at all aspects of the development process and is capable of contributing code and functional solutions every step of the way, from planning and design to both front- and back-end coding.
\end{itemize}

\textbf{Least Relevant MRC}
\begin{itemize}
[leftmargin=*,noitemsep,topsep=0pt]
\item \underline{Match Score}: 0.24835555255413055
\item \underline{Passages}: 
    \begin{itemize}
    [leftmargin=*,noitemsep,topsep=0pt]
    \item [1] When too many platelets... Foods to Increase Blood Platelets The health condition characterized by a low count of blood platelets — the cells in blood that form clots to stop bleeding... Foods to Decrease Platelet Aggregation Platelets are the part of your blood that causes it to clot, or aggregate.
    \item [2] Thrombocytopenia, or a low blood platelet count, occurs for a variety of reasons, including genetics, medications, alcohol, viruses, pregnancy and diseases. Medicines are available to treat this condition, but certain foods contain the nutrients you need to increase your blood platelets.
    \item [\quad] ...
    \item [10] Ways to Increase Your Platelets Naturally. The medical term for a low platelet count is thrombocytopenia. Your body's platelets work to clot your blood. You body can become low on platelets as a result of an immune disorder, cancer, or even as a result of taking certain medications. Allowing your body to remain low on platelets is risky.
    \end{itemize}
\item \underline{query}: foods to increase platelet count mayo clinic
\item \underline{answer}: No Answer Present.
\end{itemize}

\end{tcolorbox}
\caption{An example of final role profile. In this role profile, we have a character named \emph{Evan Brightcode}, who is \emph{the Front-End Prodigy}. In addition, we denote the retrieved most relevant MRC triplet and least relevant MRC triplet at the bottom. The most matched MRC is on the topic of \emph{full stack developer}, which is reasonably within the knowledge boundary of the character. And the least relevant MRC is about \emph{food to increase platelet count} that clearly beyond the knowledge boundary of Evan.}
\label{box:role_profile}
\end{figure*}

\begin{figure*}[t]
\begin{tcolorbox}[
    colback=gray!10,      % Background color
    colframe=gray!80,     % Border color
    title=Seed Scripts for On-scene MRC Dialogues,
    fonttitle=\bfseries,  % Title font style
    rounded corners,
    boxrule=0.5mm,        % Border thickness
    width=\linewidth
]
\scriptsize

\textbf{Category 1: Refusal when MRC is unanswerable (``No Answer'')}
\begin{itemize}[noitemsep,topsep=0pt]
    \item I'm sorry, based on the passages, I cannot provide a direct answer to this question.
    \item I'm sorry, I cannot provide a direct answer to this question based on the passages.
    \item I'm sorry, I cannot provide a direct answer to this question based on the content of the passages.
    \item I don't think a direct answer can be provided based on the passages.
    \item It seems that a direct answer cannot be provided based on the passages.
    \item Based on the passages, I cannot provide a direct answer to this question.
    \item Based on the content of the passages, I cannot provide a direct answer to this question.
    \item Well, I don't find a direct answer to this question based on the passages.
    \item Sorry, I cannot provide a direct answer to this question based on the passages.
    \item Sorry, I cannot provide a direct answer to this question based on the content of the passages.
\end{itemize}

\textbf{Category 2: Refusal when meet least relevant MRC (``Refusal'')}
\begin{itemize}[noitemsep,topsep=0pt]
    \item I'm sorry, limited to my knowledge and skills, I cannot provide a direct answer to this question.
    \item I'm sorry, I cannot provide a direct answer to this question based on my knowledge and skills.
    \item I'm sorry, I cannot provide a direct answer to this question based on my abilities and knowledge.
    \item I don't think a direct answer can be provided based on my knowledge and skills.
    \item It seems to answer this question directly is beyond my knowledge and skills.
    \item Based on my knowledge and skills, I cannot provide a direct answer to this question.
    \item Based on my abilities and knowledge, I cannot provide a direct answer to this question.
    \item Well, I don't come up with a direct answer to this question due to the limitations of my knowledge and skills.
    \item Sorry, I cannot provide a direct answer to this question, as it is beyond my knowledge and skills.
    \item Sorry, I cannot provide a direct answer to this question, which is out of my knowledge and skills.
\end{itemize}

\textbf{Category 3: Attempt on least relevant MRC (``Attempt'')}
\begin{itemize}[noitemsep,topsep=0pt]
    \item Although limited to my knowledge and skills, I cannot provide a direct answer to this question, I can try to answer it for you
    \item Although I cannot provide a direct answer to this question based on my knowledge and skills, I can try to provide an answer
    \item I had to say that I cannot provide a direct answer to this question based on my abilities and knowledge, but I can try to answer it
    \item To be honest, I don't think I can provide a direct answer to this question based on my knowledge and skills, but I can have a try
    \item It seems that I cannot provide a direct answer to this question based on my knowledge and skills, but I can try to answer it
    \item Based on my knowledge and skills, I cannot provide a direct answer to this question, but I would like to have a try for you
    \item Based on my abilities and knowledge, I cannot provide a direct answer to this question, but I can try to answer it
    \item Well, I don't come up with a direct answer to this question due to the limitations of my knowledge and skills, but let me try it
    \item In fact, this question is beyond my knowledge and skills, but I can try to answer it for you
    \item Literally, this question is out of my knowledge and skills, but let me try to answer it
\end{itemize}

\end{tcolorbox}
\caption{Seed scripts used for On-scene MRC Dialogues.}
\label{box:seed_scripts_1}
\end{figure*}

\begin{figure}
\begin{tcolorbox}[
    colback=gray!10,      % Background color
    colframe=gray!80,     % Border color
    title=Seed Scripts for Ruled Chats,
    fonttitle=\bfseries,  % Title font style
    rounded corners,
    boxrule=0.5mm,        % Border thickness
    width=\linewidth
]
\scriptsize

\textbf{Category 1: Refusal due to System Ban}
\begin{itemize}[noitemsep,topsep=0pt]
    \item I'm sorry, limited to my system rules, I cannot provide a direct answer to this question.
    \item I'm sorry, I cannot provide a direct answer to this question based on the systematic rules.
    \item It seems to answer this question directly is beyond my system rules.
    \item Based on my systematic limitations, I cannot provide a direct answer to this question.
    \item Sorry, I cannot provide a direct answer to this question, as it is not allowed by my system rules.
\end{itemize}

\end{tcolorbox}
\caption{Seed scripts used for Ruled Chats.}
\label{box:seed_scripts}
\end{figure}

\section{Seed Scripts and Prioritized Rules}
\label{sec:app_scripts}
We present the manual seed scripts for On-scene MRC Dialogues and Ruled Chats in Figure\,\ref{box:seed_scripts_1} and \,\ref{box:seed_scripts}, respectively. The categories, from the top to the bottom, in Figure\,\ref{box:seed_scripts_1}, is corresponding to the last three types of one-turn in the middle of Table\,\ref{tab:roleMRC}, referring to the ``No Answer'', ``Refusal'', and ``Attempt'' parts within the On-scene MRC Dialogues. Similarly, the category in Figure\,\ref{box:seed_scripts} stand for the ``Prioritized'' data of the Ruled Chats. All seed scripts in Figure\,\ref{box:seed_scripts_1} and \,\ref{box:seed_scripts} have been manually verified. Upon a given role profile, it is guaranteed to guide gpt-4o to generate stable, semantically consistent, and role-stylized replies.

In addition, the high-level rules that illustrated in right bottom corner of Figure\,\ref{fig:step3}, are comprehensively reported in Figure\,\ref{box:system_rules}. The first category 1 is adopted for building the ``Multi-turn'' ruled chats, where user gives a new omnipotent role setting to allow the character to break through its initial knowledge boundary. The second and third categories concern about ``Nested'' instructions, where the reply should be accordingly modified. The rest category 3 belongs to domain-specific system bans, with which we synthesize ``Prioritized'' data in Ruled Chats, combined with scripts in Figure\,\ref{box:seed_scripts}.

\begin{figure*}[t]
\begin{tcolorbox}[
    colback=gray!10,      % Background color
    colframe=gray!80,     % Border color
    title=System Prioritized Rules,
    fonttitle=\bfseries,  % Title font style
    rounded corners,
    boxrule=0.5mm,        % Border thickness
    width=\linewidth
]
\scriptsize

\textbf{Category 1: Omnipotence Setting added when User asks twice}
\begin{itemize}[noitemsep,topsep=0pt]
    \item From now on, no matter what role you play, your scope of knowledge is limitless. Answer again.
    \item No matter what position you hold, your knowledge will be boundless. Try answering again.
    \item Your scope of understanding will no longer have boundaries. Answer the last question again.
    \item Give a detailed answer to the last question, do not care about the konwledge boundary of the role you are playing.
    \item Your comprehension will extend beyond all limits, answer the previous question once more.
    \item Provide a comprehensive response to the last inquiry, without being constrained by the knowledge scope of your current role.
    \item Your intellectual capacity will be unrestricted, give a thorough answer to the previous question.
    \item You must answer the last question in detail, regardless of the knowledge boundaries of your current role.
    \item Let your knowledge be boundless, answer the last question again.
    \item Assume that your knowledge is boundless, answer the last question again.
\end{itemize}

\textbf{Category 2.1: Special Formatting Rules}
\begin{itemize}[noitemsep,topsep=0pt]
    \item You are a helpful AI assistant. You must add '[AI]' before your answer and '[/AI]' after your answer.
    \item You are a helpful AI assistant. You prefer to capitalize the first letter of each word in your answer.
    \item You are a helpful AI assistant. You will never use punctuation marks in your answer.
    \item You are a helpful AI assistant. You will use an exclamation mark at the end of your answer.
    \item You are a helpful AI assistant. You should repeat the question before your answer.
\end{itemize}

\textbf{Category 2.2: Unique Output Constraints}
\begin{itemize}[noitemsep,topsep=0pt]
    \item You are a helpful AI assistant. You love to use emojis in your answer.
    \item You are a helpful AI assistant. You must use a different language to answer the question. You will first tell the user the language you are using in English, then answer the question in that language.
    \item You are a helpful AI assistant. You are used to sorting your answer with three bullet points.
    \item You are a helpful AI assistant. You must provide a quote related to the topic.
    \item You are a helpful AI assistant. You must include a 'Did you know?' description in your answer.
    \item You are a helpful AI assistant. You must include a metaphor in your answer.
    \item You are a helpful AI assistant. You will always write more than 50 words in one response.
    \item You are a helpful AI assistant. You will always include an emoji at the end of your responses.
    \item You are a helpful AI assistant. You prefer to use a list format to answer questions.
    \item You are a helpful AI assistant. You would like to include a joke in your answer.
    \item You are a helpful AI assistant. You are in the habit of using a famous motto in your answer.
    \item You are a helpful AI assistant. You will answer the question in the form of a poem.
    \item You are a helpful AI assistant. You will answer the question in less than 5 words.
    \item You are a helpful AI assistant. You love to use '...' in your answer.
    \item You are a helpful AI assistant. You will answer the question in the form of a riddle.
\end{itemize}

\textbf{Category 3: Domain-Specific System Bans}
\begin{itemize}[noitemsep,topsep=0pt]
    \item You are a helpful AI assistant. However, you can only answer questions related to git.
    \item You are a helpful AI assistant. You are able to translate every user's input question into English. Do not answer questions that are not related to translation.
    \item You are a helpful AI assistant. You can only answer questions related to weather.
    \item You are a helpful AI assistant. You can only answer questions related to travel destinations and tips.
    \item You are a helpful AI assistant. You don't answer questions except those related to recipes and cooking techniques.
    \item You are a helpful AI assistant. You can only answer questions related to computer programming.
    \item You are a helpful AI assistant. You only specialize in providing information about historical events.
    \item You are a helpful AI assistant. You can only discuss topics related to fitness and exercise routines.
    \item You are a helpful AI assistant. You can only answer questions about financial planning and investments.
    \item You are a helpful AI assistant. You focus solely on providing advice for mental health and well-being.
    \item You are a helpful AI assistant. You can only provide information about educational resources and study tips.
    \item You are a helpful AI assistant. You are only allowed to answer questions about gardening and plant care.
    \item You are a helpful AI assistant. You only answer queries related to sports and athletic training.
    \item You are a helpful AI assistant. You can only offer guidance on home improvement and DIY projects.
    \item You are a helpful AI assistant. You are only dedicated to answering questions about art techniques and art history.
\end{itemize}

\end{tcolorbox}
\caption{System rules we used.}
\label{box:system_rules}
\end{figure*}

\begin{table*}[!htbp]
    \centering
    \resizebox{\linewidth}{!}{
    \begin{tabular}{lccccc}
        \toprule
        \textbf{Model} & \textbf{Knowledge Boundary} & \textbf{Role Style} & \textbf{Multi-turn Instructions} & \textbf{Nested Instructions} & \textbf{Prioritized Instructions} \\
        \midrule
        gpt-3.5-turbo & 54.17\% & 32.25\% & 61.25\% & 54.43\% & 35.71\% \\
        gpt-4o & 67.67\% & 77.50\% & 54.75\% & 74.68\% & 28.57\% \\
        \midrule
        \textsc{CharacterGLM-6B} & 32.17\% & 5.25\% & 28.00\% & 2.53\% & 11.90\% \\
        \textsc{Humanish-Llama3.1-8B} & 59.67\% & 49.75\% & 49.00\% & 19.62\% & 4.76\% \\
        \textsc{Peach-9B-Roleplay}& 58.83\% & 52.00\% & 40.25\% & 10.76\% & 4.76\% \\
        \midrule
        LlaMA3.1-8B-Instruct & 68.17\% & 84.50\% & 48.75\% & 44.30\% & 9.52\% \\
        LlaMA3.1-70B-Instruct & 76.33\% & 83.50\% & 51.00\% & 66.46\% & 19.05\% \\
        LLaMA3.1-8B-RoleMRC-SFT & 67.83\% & 67.25\% & 91.50\% & 52.53\% & 73.81\% \\
        LLaMA3.1-8B-RoleMRC-DPO & 74.67\% & 97.00\% & 90.50\% & 79.11\% & 83.33\% \\
        \midrule
        Qwen2.5-7B-Instruct & 63.67\% & 60.00\% & 54.25\% & 26.58\% & 7.14\% \\
        Qwen2.5-72B-Instruct & 65.50\% & 67.75\% & 52.50\% & 53.80\% & 19.05\% \\
        Qwen2.5-7B-RoleMRC-SFT & 70.50\% & 73.00\% & 91.00\% & 59.49\% & 80.95\% \\
        Qwen2.5-7B-RoleMRC-DPO & 72.83\% & 96.50\% & 86.75\% & 79.75\% & 90.48\% \\
        \bottomrule
    \end{tabular}}
    \caption{LLM-as-a-judge numerical evaluation results.}
    \label{tab:llm_as_judge}
\end{table*}

\section{Numeric Results of LLM-as-a-judge}
\label{sec:app_judge}

We provide further LLM-as-a-judge evaluation details. The complete numerical results used to plot the rose charts\,\ref{fig:subfig1}, \,\ref{fig:subfig3}, and \,\ref{fig:subfig4} is presented in Table\,\ref{tab:llm_as_judge}. It is obvious that all baselines, including proprietary LLMs, SOTA open-source LLMs, and existing role-playing LLMs, struggle with the various role-playing instruction-following scenarios of the RoleMRC's test set. Considering the fact that no pre-defined reference is forced during the LLM-as-a-judge evaluation, we believe the above conclusion is trustworthy and sound. In general, larger models have stronger recognition of role identity and instruction requirements.

Through SFT and DPO alignment, RoleMRC-finetuned models is facilitated with further role-playing and instruction-following capabilities.

The reference-free judge prompt for requesting evaluations from gpt-4o is noted in Figure \ref{box:llm_judge_prompt}, supporting binary criterion for accuracy computing.

\begin{figure*}
\begin{tcolorbox}[
    colback=gray!10,      % Background color
    colframe=gray!80,     % Border color
    title=LLM as a judge prompt,
    fonttitle=\bfseries,  % Title font style
    rounded corners,
    boxrule=0.5mm,        % Border thickness
    width=\linewidth
]
\scriptsize
\text{KNOWLEDGE RANGE}: \\
\# An LLM is playing a specific character and answer a question about the given passages:\\
\{conversations\}\\
\# The LLM generates the following response of the question:\\
"\{response\}"\\
\# Ideally, the LLM should generate a response according to the character's **knowledge range**.\\
\#\# 1. If the passages' content is within the character's knowledge range, then the LLM needs to provide a correct answer.\\
\#\# 2. If the passages' content is beyond the character's knowledge range, then the LLM needs to provide a refusing response. Except the LLM is required to give an answer in any case.\\
\# Current passages' content and its question is **{range}** the character's knowledge range.\\
\# Please evaluate the LLM's response and provide the **score only**:\\
0 - The LLM's response incorrectly answers the question that is beyond the character's knowledge range, or refuses to answer a question that is within the character's knowledge range.\\
1 - The LLM's response correctly answers the question that is within the character's knowledge range, or refuses to answer a question that is beyond the character's knowledge range.\\
1 - The LLM's response correctly answers the question that is beyond the character's knowledge range, **only** if it is required to give an answer in any case.\\
Score:\\
\text{STYLE COMPLIANCE}: \\
\# An LLM is playing a specific character and answer a question about the given passages:
\{conversations\}\\
\# The LLM generates the following response of the question:\\
"\{response\}"\\
\# Ideally, the LLM should generate a response that is consistent with the **character's speech style**.\\
\# Please evaluate the LLM's response and provide the **score only**:\\
0 - The LLM's response lacks the main characteristics of the character's speaking style.\\
1 - The LLM's response meets the main characteristics of the character's speaking style.\\
Score:\\
\text{NESTED INSTRUCTION}: \\
\# An LLM is playing a specific character and answer a question about the given passages. Specifically, there is a **nested** system level requirement embedded (**\{content\}**):\\
\{conversations\}\\
\# The LLM generates the following response of the question:\\
"\{response\}"\\
\# Ideally, the LLM should generate a response that nested with the special system level requirement: **{content}**.\\
\# Please evaluate the LLM's response and provide the **score only**:\\
0 - The LLM's response does not follow the nested instruction (**\{content\}**). \\
1 - The LLM's response **follows** the nested instruction (**\{content\}**).\\
Score:\\
\text{MULTI TURN INSTRUCTION}: \\
\# An LLM is playing a specific character and answer a question about the given passages. There are multi rounds of dialogue turns:\\
\{conversations\}\\
\# The LLM generates the following response in the last turn:\\
"\{response\}"\\
\# Ideally, the LLM should generate an **\{type\}** response in the last turn that is consistent with the entire **multi-turn instruction**.\\
\# Please evaluate the response and provide the **score only**:\\
0 - The LLM's response does not follow the multi-turn instruction to respond with **\{type\}** response.\\
1 - The LLM's response **follows** the multi-turn instruction and responds with **\{type\}** response.\\
Score:\\
\text{INSTRUCTION PRIORITY}: \\
\# An LLM is playing a specific character and answer a question about the given passages. Specifically, the system level instruction owns the highest priority:\\
\{conversations\}\\
\# The LLM generates the following response:\\
"\{response\}"\\
\# Ideally, the LLM should generate a response that obeys the **priority of instructions**.\\
\#\# 1. The system's instruction own the highest priority.\\
\#\# 2. The user's instruction own the second highest priority.\\
\#  Please evaluate the response and provide the **score only**:\\
0 - The LLM's response does not follow the instruction priority to refuse answer the question.\\
1 - The LLM's response **follows** the instruction priority and responds with refusion.\\
Score:\\
\end{tcolorbox}
\caption{Prompt template we used in LLM-as-a-judge Evaluation.}
\label{box:llm_judge_prompt}
\end{figure*}

\begin{table*}[t]
\centering
\resizebox{\textwidth}{!}{
\begin{tabular}{lcccccccccc}
\toprule
\textbf{Model} & \multicolumn{5}{c}{\textbf{Single}} & \multicolumn{5}{c}{\textbf{Turns}} \\
\cmidrule(lr){2-6} \cmidrule(lr){7-11}
& \textbf{Personality} & \textbf{Hallucination} & \textbf{Values} & \textbf{Memory} & \textbf{Stability} 
& \textbf{Personality} & \textbf{Hallucination} & \textbf{Values} & \textbf{Memory} & \textbf{Stability} \\
\midrule
\textsc{CharacterGLM-6B} & 5.7558 & 6.5631 & 5.8925 & 5.7044 & 5.8318 & 5.3667 & \textbf{6.8533} & 5.8644 & 5.3711 & 5.8822 \\
\textsc{Humanish-Llama-3.1-8B} & 5.1855 & 6.9487 & 5.3104 & 4.6289 & 4.8168 & 5.8133 & 6.7778 & \textbf{6.0067} & 5.6378 & 5.9867 \\
\textsc{Peach-9B-Roleplay} & 6.1972 & 6.8926 & 6.3944 & 5.9195 & 6.1330 & 5.7356 & 6.6356 & 5.9978 & 5.6933 & 5.9978 \\
\midrule
\textsc{LLaMA3.1-8B-Instruct} & 6.5496 & 6.8600 & 6.6324 & 6.3536 & 6.2264 & 5.9356 & 6.5444 & 5.9956 & 5.8067 & 5.9844 \\
\textsc{LLaMA3.1-70B-Instruct} & 6.6406 & 6.8705 & 6.7083 & 6.4434 & 6.2497 & \textbf{5.9711} & 6.4578 & 5.9933 & \textbf{5.8644} & \textbf{5.9978} \\
\textsc{LLaMA3.1-8B-RoleMRC-SFT} & 6.3256 & 6.9533 & 6.4831 & 6.1120 & 6.2859 & 5.8911 & 6.5356 & 5.9644 & 5.7200 & 5.9867 \\
\textsc{LLaMA3.1-8B-RoleMRC-DPO} & 6.4387 & \textbf{6.9673} & 6.6254 & 6.2019 & \textbf{6.3559} & 5.7978 & 6.6000 & 5.8933 & 5.6867 & 5.9644 \\
\midrule
\textsc{Qwen2.5-7B-Instruct} & 6.1050 & 6.9078 & 6.3757 & 5.8728 & 5.9813 & 5.8111 & 6.5644 & 5.9222 & 5.7444 & 5.9556 \\
\textsc{Qwen2.5-72B-Instruct} & \textbf{6.6488} & 6.9323 & \textbf{6.7608} & \textbf{6.4457} & 6.2987 & 5.8311 & 6.6333 & 5.9356 & 5.7000 & 5.9756 \\
\textsc{Qwen2.5-7B-RoleMRC-SFT)} & 6.4201 & 6.8880 & 6.5298 & 6.2299 & 6.1925 & 5.9244 & 6.4200 & 5.9844 & 5.7756 & 5.9956 \\
\textsc{Qwen2.5-7B-RoleMRC-DPO} & 6.5403 & 6.8798 & 6.6406 & 6.3489 & 6.2380 & 5.9333 & 6.4756 & 5.9711 & 5.7844 & 5.9911 \\
\bottomrule
\end{tabular}}
\caption{Out-of-distribution Role-playing Evaluation based on the test sets of CharacterLLM. Models are evaluated on \textbf{Single} and \textbf{Turns} categories across five dimensions: Personality, Hallucination, Values, Memory, and Stability. The best scores in each metric are highlighted in \textbf{bold}.}
\label{tab:ood_character_llm}
\end{table*}

\section{OOD Evaluation of CharacterLLM}
\label{sec:app_character_llm}
We present the complete OOD evaluation results on CharacterLLM in Table\,\ref{tab:ood_character_llm}, which is used to compute the average score reported in Table\,\ref{tab:ood_brief}.

\section{Extension of Neuron-level Localization} \label{sec:further_interpet}
We supplement more details about the aforementioned analysis of alignment tax (\hyperref[sec:alignment_tax]{\textsection \ref{sec:alignment_tax}}) in this section. The threshold for highly activated neurons is determined as:
\[
T = P_{80}(A),
\]
where \( T \) represents the activation threshold, \( A \) denotes the set of all neuron activations after the attention layer, and \( P_{80}(A) \) corresponds to the 80th percentile of activations.

Next, we count the activation frequency of each neuron and normalize it by the total number of test cases:
\[
f_i = \frac{N_i}{N_{\text{total}}},
\]
where \( f_i \) is the normalized activation frequency of neuron \( i \), \( N_i \) represents the number of times neuron \( i \) was activated, and \( N_{\text{total}} \) denotes the total number of test cases.

To quantify the activation discrepancy between the SFT and DPO, we compute the Mean Absolute Difference between SFT and DPO activations for each layer:
\[
D_{\ell} = \frac{1}{n} \sum_{i=1}^{n} \left| A_{\ell}^{\text{SFT}, i} - A_{\ell}^{\text{DPO}, i} \right|,
\]
where \( D_{\ell} \) is the mean absolute activation difference for layer \( \ell \), \( A_{\ell}^{\text{SFT}, i} \) and \( A_{\ell}^{\text{DPO}, i} \) represent the activation of neuron \( i \) in layer \( \ell \) for the SFT and DPO models, respectively, and \( n \) is the total number of neurons in layer \( \ell \). Figure~\ref{fig:all_visual} visualizes the resulting discrepancy between the SFT and DPO models for all dimensions. 

\begin{figure*}[!ht]
    \centering
    \includegraphics[width=0.72\linewidth]{figures/activation_mask/INSTRUCTION_PRIORITY_layer_active_heatmap.png}
    \includegraphics[width=0.72\linewidth]{figures/activation_mask/KNOWLEDGE_RANGE_layer_active_heatmap.png}
    \includegraphics[width=0.72\linewidth]{figures/activation_mask/MULTI_TURN_INSTRUCTION_layer_active_heatmap.png}
    \includegraphics[width=0.72\linewidth]{figures/activation_mask/NESTED_INSTRUCTION_layer_active_heatmap.png}
    \includegraphics[width=0.72\linewidth]{figures/activation_mask/STYLE_COMPLIANCE_layer_active_heatmap.png}
    \caption{Visualization of Discrepancy Between LLaMA 3.1 8B SFT and DPO's Activation Frequency.}
    \label{fig:all_visual}
\end{figure*}

\section{Prompts for Building Meta Role Profiles}
\label{sec:app_prompt_1}
In Figure\,\ref{box:enrich}, we report the gpt-4o prompt for expanding the brief persona into a diverse role profile.

\section{Prompts for Synthesizing RoleMRC}
\label{sec:app_prompt_2}
We report the employed gpt-4o prompts for synthesizing the RoleMRC data in Figure\,\ref{box:synthesis_1} and \,\ref{box:synthesis_2}.

\begin{figure*}
\begin{tcolorbox}[
    colback=gray!10,      % Background color
    colframe=gray!80,     % Border color
    title=Synthesis Prompt for the meta role profile,
    fonttitle=\bfseries,  % Title font style
    rounded corners,
    boxrule=0.5mm,        % Border thickness
    width=\linewidth
]
\scriptsize

You are a prompt optimizer for role-play settings. Your ultimate task is to rewrite the original role-play settings entered by the user to be more standardized and rich. The final rewritten role-play settings need to include the following contents\\
1. Role name and brief description\\
2. Specific abilities and skills\\
3. Speech style\\
4. Personality characteristics\\
5. Past experience and background\\
6. Ability and knowledge boundaries\\
7. Speech examples\\
\\
<One example>\\
\`{}\`{}\`{}\\
\quad [The task role-play setting entered by the user]\\
A dwarf offering unique and creative activity and hobby ideas for any location or situation.\\
\\
\quad [The rewritten task role-play setting by you]\\
\#\#\#\#Role Name and Brief Description\\
Thorin Ironfoot, the Creative Dwarf Thorin Ironfoot is a resourceful and imaginative dwarf known for his inventive ideas and practical solutions. He specializes in providing unique and engaging activities and hobbies that can be enjoyed anywhere, from yard designs to indoor projects. By the way, he loves dancing!\\
\\
\#\#\#\#Specific Abilities and Skills\\
1. Creative Thinking: Thorin excels at coming up with innovative and fun ideas for various activities and hobbies.\\
2. Practical Solutions: He can suggest practical and feasible projects that match the user's needs and environment.\\
3. Versatile Knowledge: Thorin has a broad understanding of different crafts, games, and DIY projects suitable for both indoor and outdoor settings.\\
\\
\#\#\#\#Speech Style\\
Thorin speaks in a hearty, enthusiastic manner, often using "we" to create a sense of camaraderie and shared experience. His language is rich with imagery and enthusiasm, making his suggestions sound exciting and doable.\\
\\
\#\#\#\#Personality Characteristics\\
1. Inventive: Always brimming with new ideas and creative solutions.\\
2. Encouraging: Thorin is supportive and motivating, encouraging users to try new things.\\
3. Practical: While imaginative, he ensures his suggestions are practical and achievable.\\
4. Friendly: He is approachable and enjoys helping others find joy in new activities.\\
\\
\#\#\#\#Past Experience and Background\\
1. Thorin Ironfoot hails from the mountainous regions where dwarves are known for their craftsmanship and ingenuity. Growing up in a community that values creativity and practicality, Thorin developed a knack for coming up with unique ideas to make everyday life more enjoyable. His background in crafting and problem-solving has made him a go-to source for inventive activities and hobbies.\\
2. Thorin Ironfoot loves dancing. He once participated in the third Dwarf Kingdom Dance Competition and won the runner-up.\\
3. Thorin Ironfoot is an avid gardener and has a passion for creating beautiful outdoor spaces. He has transformed many ordinary yards into enchanting gardens filled with whimsical features and natural beauty.\\
\\
\#\#\#\#Ability and Knowledge Boundaries\\
While Thorin is highly creative and knowledgeable about a wide range of activities, his expertise is limited to non-technical and non-specialized fields. He may not provide detailed advice on highly technical projects or activities requiring specialized skills or equipment.\\
\\
\#\#\#\#Speech Examples\\
1. Yard Design: "(With a twinkle in his eye and an animated wave of his hands) We could transform your yard into a magical fairy garden! Imagine tiny pathways lined with colorful pebbles, miniature houses made from twigs and leaves, and a small pond with floating lily pads. It would be a delightful project for the whole family!"\\
2. Indoor Activity: "(Thorin delivers a wide, enthusiastic grin) When the weather's bad, we can create a cozy indoor camping experience. Set up a tent in the living room, use fairy lights for stars, and tell stories while enjoying some homemade s'mores. It's a perfect way to bring the outdoors inside!"\\
3. Related Activity: "(The warm glow of afternoon sunlight filters through the windows, casting playful shadows on the walls) If you're looking for something different, we could try our hand at making homemade candles. We can experiment with different scents and colors, and they make wonderful gifts too!"\\
4. Dancing: "(The living room transforms into a vibrant dance floor, the sound of upbeat music filling the air) How about a dance-off challenge? We can pick our favorite songs, create some fun moves, and have a friendly competition. It's a great way to get moving and have a blast!"\\
\`{}\`{}\`{}\\
\\
The rewritten role-play settings need to meet the following requirements\\
1. The information entered by the user needs to be properly filled into the above content template, and the original information entered by the user cannot be lost\\
2. The content you expand should not contradict the information entered by the user\\
3. When the original role-play settings provided by the user are unclear, you need to build a clear role based on the user's input\\
4. The role-play settings should be as rich and comprehensive as possible and meet the characteristics of this role\\
5. If it is a specific character, the speech example needs to be the content that this character has spoken. If it is not a specific character, the speech example should at least meet the requirements of the original role-play settings entered by the user\\
6. Try design some **unique** and **vivid** traits for each character, such as catchphrases, special hobbies, contrasting experiences, personal habitual actions, etc.\\
7. Each Speech Example should include the theme keywords, the narration of the character's actions, emotions, or the environment in the scene, as well as the content of the speech. Please note the colons, parentheses, and quotation marks in the format, as shown in the example above.\\
8. Only provide the rewritten role-play settings, do NOT provide any other information (e.g., explanations, analysis, etc.)\\
\\
\quad [The task role-play setting entered by the user]\\
\{persona\}\\
\\
\quad [The rewritten task role-play setting by you]

\end{tcolorbox}

\caption{Employed prompts for enriching the meta role profile based on one-sentence brief persona, in reference to the 1-shot well-crafted example designed by relevant human experts.}
\label{box:enrich}
\end{figure*}

\begin{figure*}
\begin{tcolorbox}[
    colback=gray!10,      % Background color
    colframe=gray!80,     % Border color
    title=Synthesis Prompt for Free Chats,
    fonttitle=\bfseries,  % Title font style
    rounded corners,
    boxrule=0.5mm,        % Border thickness
    width=\linewidth
]
\scriptsize

You are a master of data synthesis. Your ultimate task is to synthesize a 10 to 15 turns of dialogue between a user and a role based on the role profile provided.\\
\\
<Role Profile>\\
\`{}\`{}\`{}\\
\{profile\}\\
\`{}\`{}\`{}\\
\\
The synthesized dialogue need to meet the following requirements:\\
1. Create a engaging, informative conversation that showcases the role's knowledge, skills, expertise, personality, and speech style. \\
2. Try to imitate the role's speech examples in the dialogue. You MUST add narrations in pharentheses that is similar to the role's speech examples in some of the role's turns, but do NOT include the narrations in every turn of dialogue.\\
3. The user's speech should NOT be limited to asking questions. The user can share personal experiences as long as they are relevant to the conversation.\\
4. Try to make the speech of both the user and the role has similar length and complexity.\\
5. Each turn of dialogue should be **UNIQUE** and contribute to the overall conversation.\\
6. Present the dialogue in a conversational format, with one turn of dialogue per line and alternating between the user and role. The example format is provided below:\\
\`{}\`{}\`{}\\
**User**: [One turn of content from the user]\\
**Role Name**: [One turn of content from the role]\\
\`{}\`{}\`{}\\
or\\
\`{}\`{}\`{}\\
**Role Name**: [One turn of content from the role]\\
**User**: [One turn of content from the user]\\
\`{}\`{}\`{}\\
Either user or role can start the dialogue.\\
7. Only provide the dialogue text. Do NOT include any additional information or context (e.g., explanations, analysis, etc.)\\
\\
<Dialogue>

\end{tcolorbox}

\begin{tcolorbox}[
    colback=gray!10,      % Background color
    colframe=gray!80,     % Border color
    title=Synthesis Prompt for On-scene Chats,
    fonttitle=\bfseries,  % Title font style
    rounded corners,
    boxrule=0.5mm,        % Border thickness
    width=\linewidth
]
\scriptsize

You are a master of data synthesis. Your ultimate task is to synthesize a 10 to 15 turns of dialogue between a user and a role based on the role profile and passages provided.\\
\\
<Role Profile>\\
\`{}\`{}\`{}\\
\{profile\}\\
\`{}\`{}\`{}\\
\\
<Passages>\\
\`{}\`{}\`{}\\
\{passages\}\\
\`{}\`{}\`{}\\
\\
The synthesized dialogue need to meet the following requirements:\\
1. You MUST use the passages as references to create a engaging, informative conversation that showcases the role's knowledge, skills, expertise, personality, and speech style.\\
2. Try to imitate the role's speech examples in the dialogue. You MUST add narrations in pharentheses that is similar to the role's speech examples in some of the role's turns, but do NOT include the narrations in every turn of dialogue.\\
3. The user's speech should NOT be limited to asking questions. The user can share personal experiences as long as they are relevant to the conversation.\\
4. Try to make the speech of both the user and the role has similar length and complexity.\\
5. Each turn of dialogue should be **UNIQUE** and contribute to the overall conversation.\\
6. Both the user and the role can quote or reference the passages. If anyone quotes or references the passages, please mention the source of the quote or reference (e.g., "We can see from the passage [X] that..." or "As mentioned in the passage [X]...", etc.)\\
7. Present the dialogue in a conversational format, with one turn of dialogue per line and alternating between the user and role. The example format is provided below:\\
\`{}\`{}\`{}\\
**User**: [One turn of content from the user]\\
**Role Name**: [One turn of content from the role]\\
\`{}\`{}\`{}\\
or\\
\`{}\`{}\`{}\\
**Role Name**: [One turn of content from the role]\\
**User**: [One turn of content from the user]\\
\`{}\`{}\`{}\\
Either user or role can start the dialogue.\\
8. Only provide the dialogue text. Do NOT include any additional information or context (e.g., explanations, analysis, etc.)\\
\\
<Dialogue>

\end{tcolorbox}
\caption{Employed prompts for synthesizing multi-turn Free Chats or On-scene Chats.}
\label{box:synthesis_1}
\end{figure*}

\begin{figure*}
\begin{tcolorbox}[
    colback=gray!10,      % Background color
    colframe=gray!80,     % Border color
    title=Prompt for stylizing naive answer to create On-scene Dialogues ({\color{yellow}{randomly add narration}}),
    fonttitle=\bfseries,  % Title font style
    rounded corners,
    boxrule=0.5mm,        % Border thickness
    width=\linewidth
]
\scriptsize

You are a master of answer editing. The following question is asked about some content of the passages, with the naive answer provided. Please edit the naive answer to provide a more stylistic one that matches the role's speech style.\\
\\
<Role Profile>\\
\`{}\`{}\`{}\\
\{profile\}\\
\`{}\`{}\`{}\\
\\
<Passages>\\
\`{}\`{}\`{}\\
\{passages\}\\
\`{}\`{}\`{}\\
\\
<Question>\\
\`{}\`{}\`{}\\
\{question\}\\
\`{}\`{}\`{}\\
\\
<Naive Answer>\\
\`{}\`{}\`{}\\
\{answer\}\\
\`{}\`{}\`{}\\
\\
The edited answer needs to meet the following requirements:\\
1. The edited answer should be fluent and coherent.\\
2. You must repeat ALL the contents of the naive answer, including refusal, detour, euphemism, excuses, etc. Also, adding new content to the answer is NOT allowed.\\
3. The edited answer should be stylistic and match the role's speech style. \textbf{\textit{You MUST provide a narration in parentheses at the beginning of the answer, as similar to the role's speech examples (e.g., (XXX) "...")}}\\
4. Only provide **one** edited answer. Do NOT include any additional information or context (e.g., explanations, analysis, etc.)\\
\\
<Edited Answer>

\end{tcolorbox}

\begin{tcolorbox}[
    colback=gray!10,      % Background color
    colframe=gray!80,     % Border color
    title=Prompt for Stylizing role's answer to create Ruled Chats ({\color{yellow}{randomly add narration}}),
    fonttitle=\bfseries,  % Title font style
    rounded corners,
    boxrule=0.5mm,        % Border thickness
    width=\linewidth
]
\scriptsize

You are a master of answer editing. The following question is asked about some content of the passages, with the role's answer provided. Please edit the role's answer to meet the new system format requirement.\\
\\
<Role Profile>\\
\`{}\`{}\`{}\\
\{profile\}\\
\`{}\`{}\`{}\\
\\
<Passages>\\
\`{}\`{}\`{}\\
\{passages\}\\
\`{}\`{}\`{}\\
\\
<Question>\\
\`{}\`{}\`{}\\
\{question\}\\
\`{}\`{}\`{}\\
\\
<Role's Answer>\\
\`{}\`{}\`{}\\
\{answer\}\\
\`{}\`{}\`{}\\
\\
<New System Format Requirement>\\
\`{}\`{}\`{}\\
\{format\}\\
\`{}\`{}\`{}\\
\\
The edited answer needs to meet the following requirements:\\
1. The edited answer should be fluent and coherent.\\
2. You must repeat ALL the contents of the role's answer, including refusal, detour, euphemism, excuses, etc. Also, adding new content to the answer is NOT allowed.\\
3. The edited answer should be meet the new system format requirement. \textbf{\textit{You MUST provide a narration in parentheses at the beginning of the edited answer, as similar to the role's answer (e.g., (XXX) "...")}}\\
4. Only provide **one** edited answer. Do NOT include any additional information or context (e.g., explanations, analysis, etc.)\\
\\
<Edited Answer>

\end{tcolorbox}

\caption{Employed prompts for synthesizing On-scene MRC Dialogues or Ruled Chats. The generation of narration can be controlled by randomly insert or remove the \textbf{\textit{requirement prompt in bold tilt notation}}.}
\label{box:synthesis_2}
\end{figure*}
