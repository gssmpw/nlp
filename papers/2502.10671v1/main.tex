
\newcommand{\CLASSINPUTtoptextmargin}{0.751in}
\newcommand{\CLASSINPUTbottomtextmargin}{1.001in}
\documentclass[conference%, draftcls
]{IEEEtran}
%\usepackage{showframe}
%\usepackage{layout}
%\usepackage[top=0.75in,left=0.67in,right=0.67in]{geometry}
\IEEEoverridecommandlockouts    %delete ?
\usepackage{verbatim} % for comments
\usepackage{cite}
\usepackage{nicematrix}
\usepackage{longtable}
\usepackage{booktabs}
\usepackage{bm}
\usepackage{amsmath,amssymb,amsfonts,flushend}
%%%
\makeatletter
\let\MYcaption\@makecaption
\makeatother
\usepackage[font=footnotesize]{subcaption}
\makeatletter
\let\@makecaption\MYcaption
\makeatother
%%%%
\captionsetup[subfigure]{skip=2pt}
%%%%
\usepackage{algorithmic}

\usepackage{graphicx}
\usepackage{tikz}
\usepackage{circuitikz}
\usepackage{siunitx}
\usetikzlibrary{shapes.geometric, arrows,chains, positioning}
\usetikzlibrary{arrows.meta} 
\usetikzlibrary { decorations.pathmorphing, decorations.pathreplacing, decorations.shapes, } 
\usepackage{pbox}
\newcommand{\ctikzlabel}[2]{\pbox{\textwidth}{#1\\#2}} % multiple-lines labels
\usepackage{relsize}
\tikzset{
    pin/.style = {font = \relsize{-2}} % pin font size
}
\ctikzset{
    bipoles/length = 4em, % bipole size
    font = \relsize{-1}, % default font size
}
\usepackage{textcomp}
\usepackage{xcolor}
\def\BibTeX{{\rm B\kern-.05em{\sc i\kern-.025em b}\kern-.08em
    T\kern-.1667em\lower.7ex\hbox{E}\kern-.125emX}}
\usepackage{mathtools}
\usepackage{schemabloc}
\usetikzlibrary{circuits}
%\usepackage{physics}
\usetikzlibrary{arrows}

\usepackage{pgfplotstable}
%\pgfplotsset{compat=1.14,grid style={dashed,gray}}
\usepackage{multirow}
%gia norm
\usepackage{mathtools}
\usepackage{commath}
\let\oldnorm\norm   % <-- Store original \norm as \oldnorm
\let\norm\undefined % <-- "Undefine" \norm
\DeclarePairedDelimiter\norm{\lVert}{\rVert}

\newcommand{\RNum}[1]{\uppercase\expandafter{\romannumeral #1\relax}}
\graphicspath{ {./confimages/} }
%matlab2tikz
 \usepackage{pgfplots}
  \pgfplotsset{compat=newest}
  %% the following commands are needed for some matlab2tikz features
  \usetikzlibrary{plotmarks}
  \usetikzlibrary{arrows.meta}
  \usepgfplotslibrary{patchplots}
  \usepackage{grffile}
  \usepackage{amsmath}
  
  % package for Proofs
\usepackage{amsthm}

% gia algorithmo
\usepackage{fancyhdr,graphicx,amsmath,amssymb}
\usepackage[ruled,vlined]{algorithm2e}


%\include{pythonlisting}
\usepackage{times}
%gia pinaka
\usepackage{hhline}

%space saver
%\usepackage{titlesec}
%\titlespacing\section{0pt}{3pt plus 2pt minus 2pt}{3pt plus 1pt minus 2pt}
%\titlespacing\subsection{0pt}{2pt plus 1pt minus 1pt}{2pt plus 1pt minus 1pt} %minus 2pt
\usepackage{hhline}
\usepackage{booktabs} % for textbook-quality tables
\usepackage{circuitikz}
\usepackage{siunitx}

\newtheorem{Proposition}{Proposition}

\usepackage[hyphens]{url}
\usepackage{balance}
\usepackage[commandnameprefix=always]{changes}
\setlength{\marginparwidth}{1.7cm}
\setlength{\marginparsep}{0cm}
\definechangesauthor[color=blue,name=Vassilis]{VP}
%\setlength{\textfloatsep}{1pt plus 0pt minus 2.0pt}		% space between last top float or first bottom float and the text
\setlength{\floatsep}{0pt plus 0pt minus 2.0pt}			% space left between floats
\setlength{\dblfloatsep}{0pt plus 0pt minus 2.0pt}			% space left between floats
\setlength{\dbltextfloatsep}{0pt plus 0pt minus 2.0pt}			% space left between floats
\setlength{\abovecaptionskip}{0pt plus 0pt minus 10pt}	% space above caption
\setlength{\belowcaptionskip}{0pt plus 0pt minus 3pt}	% space below caption
\setlength{\abovedisplayskip}{2pt plus 1pt minus 2pt}	% space before maths
\setlength{\belowdisplayskip}{2pt plus 1pt minus 2pt}	% space after maths

\renewcommand{\topfraction}{0.90}
\renewcommand{\bottomfraction}{0.90}
\renewcommand{\floatpagefraction}{0.9}
\renewcommand{\textfraction}{0.02}
%extra compression with title spacings

\usepackage{titlesec}
\titlespacing*{\section}{0pt}{0pt plus 0pt minus 0pt}{0pt plus 1pt minus 5pt}
\titlespacing{\subsection}{0pt}{1pt plus 0pt minus 1pt}{0pt plus 0pt minus 1pt}
\titlespacing{\subsubsection}{0pt}{1pt plus 0pt minus 12pt}{1pt}

%%%title for proof
\makeatletter
\renewenvironment{proof}[1][\relax]{\par
  \pushQED{\qed}%
  \normalfont \topsep6\p@\@plus6\p@\relax
  \trivlist
  \item[\hskip\labelsep\itshape
    \ifx#1\relax \proofname\else\proofname{} of #1\fi\@addpunct{.}]\ignorespaces
}{%
  \popQED\endtrivlist\@endpefalse
}
\makeatother


\begin{document}

%magic 
%\setlength{\abovedisplayskip}{0.2pt}
%\setlength{\belowdisplayskip}{0.2pt}
%\setlength{\abovecaptionskip}{1pt}
%\setlength{\belowcaptionskip}{1pt}
%\setlength{\dbltextfloatsep}{0.1pt}
%\setlength{\textfloatsep}{0.1pt}
%\setlength{\dblfloatsep}{0.1pt}
%\setlength{\floatsep}{0.1pt}
%
\allowdisplaybreaks % page breaks for equations

% Angle of Arrival Estimation
% Evaluating Beam Sweeping for AoA Estimation
%Evaluation of Beam Sweeping for AoA Estimation
\title{Evaluating Beam Sweeping for AoA Estimation with an RIS Prototype: Indoor/Outdoor Field Trials\vspace{-0.3cm}}


\author{\IEEEauthorblockN{ 
Dimitris Vordonis\IEEEauthorrefmark{2}, Dimitris Kompostiotis\IEEEauthorrefmark{2}, Vassilis Paliouras\IEEEauthorrefmark{2}, George C. Alexandropoulos\IEEEauthorrefmark{5}, and Florin Grec\IEEEauthorrefmark{4}}
\IEEEauthorblockA{\IEEEauthorrefmark{2}Electrical and Computer Engineering Department, University of Patras, Greece,\\
 \IEEEauthorrefmark{5}Department of Informatics and Telecommunications, National and Kapodistrian University of Athens, Greece,\\
\IEEEauthorrefmark{4}European Space Agency (ESA), Noordwijk, Netherlands\\
e-mails:\{d.vordonis, d.kompostiotis\}@ac.upatras.gr, paliuras@upatras.gr, alexandg@di.uoa.gr, Florin-Catalin.Grec@esa.int}\vspace{-1.07cm}}




% The paper headers
\markboth{NOTES}%
{Shell \MakeLowercase{\textit{et al.}}: Bare Demo of IEEEtran.cls for IEEE Journals}


\maketitle

%\layout

\begin{abstract}
Reconfigurable Intelligent Surfaces (RISs) have emerged as a promising technology to enhance wireless communication systems by enabling dynamic control over the propagation environment. However, practical experiments are crucial towards the validation of the theoretical potential of RISs while establishing their real-world applicability, especially since most studies rely on simplified models and lack comprehensive field trials. In this paper, we present an efficient method for configuring a 1-bit RIS prototype at sub-6 GHz, resulting in a codebook oriented for beam sweeping; an essential protocol for initial access and Angle of Arrival (AoA) estimation. The measured radiation patterns of the RIS validate the theoretical model, demonstrating consistency between the experimental results and the predicted beamforming behavior. Furthermore, we experimentally prove that RIS can alter channel properties and by harnessing the diversity it provides, we evaluate beam sweeping as an AoA estimation technique. Finally, we investigate the frequency selectivity of the RIS and propose an approach to address indoor challenges by leveraging the geometry of environment.
\end{abstract}

\begin{IEEEkeywords}
Reconfigurable intelligent surfaces, MIMO, beam sweeping, radiation pattern, prototype, field trials.%, sub-6 GHz band.
\end{IEEEkeywords}



\section{Introduction}
Reconfigurable Intelligent Surfaces (RISs) are emerging as a transformative technology for 6G~\cite{huang2019reconfigurable,bjornson2022reconfigurable,pan2022overview}, offering the ability to dynamically control wireless propagation environments with minimal energy consumption. In the context of wireless communication systems, RISs hold great promise for positioning applications~\cite{bjornson2022reconfigurable,pan2022overview,wymeersch2020radio,wymeersch2022radio,keykhosravi2023leveraging,alexandropoulos2023ris}, enabling precise user localization even in challenging scenarios such as urban areas with blockages or dense multipath. By introducing additional degrees of freedom through controllable reflections, RISs can improve the Signal-to-Noise Ratio (SNR), which is critical for positioning algorithms, while leveraging their large aperture size and acting as extra reference points to enhance localization accuracy. Furthermore, RISs can exploit and control multipath propagation, turning it into a valuable resource by providing diverse and rich signal measurements to aid positioning systems.

%A challenge for 5G-NR positioning techniques is their limited accuracy and reliability in multipath and interference-heavy environments~\cite{bjornson2022reconfigurable,pan2022overview,wymeersch2020radio,wymeersch2022radio,RIS_loc,kompostiotis2024evaluation}. RISs can enhance the performance of existing methods such as super-resolution algorithms~\cite{kompostiotis2024evaluation}, and stand as a complementary technology to  Global Navigation Satellite Systems (GNSS), which is restricted to outdoor use, and  to Ultra-WideBand (UWB), which is limited to short-range applications; thus bridging critical gaps in localization services.
Building on the ability of RIS to improve wireless positioning, efficient beamforming strategies are essential to fully exploit its capabilities. One such practical and widely adopted protocol in wireless communications is beam sweeping~\cite{che2023efficient,10048686,ouyang2023computer}, which plays a critical role in both initial access and Angle of Arrival (AoA) estimation.
%In this context, a practical and efficient protocol that is commonly used in wireless communications, for both initial access and Angle of Arrival (AoA) estimation, is beam sweeping~\cite{che2023efficient,10048686,ouyang2023computer}. 
In RIS-aided systems, an RIS node enables beam sweeping by steering the incoming signal and reflecting it toward the desired angles, systematically scanning the served area with narrow beams. To reduce resource usage and pilot signaling, variations such as hierarchical codebooks~\cite{albanese2021papir,RIS_hierarchical}, multi-beam patterns~\cite{wang2023hierarchical}, or broad beams~\cite{ramezani2023dual,ramezani2024broad,10256051,lin2024design} can be used, offering efficient coverage and supporting user mobility.

% from theory to practice: challenges from real ris (freq sel, discrete, hw imperfections) + indoor challenges in sub6ghz (multipath, nlos components)- not easy to have a pattern with one peak only with 1bit. models that capture ris behavior no validated
%\textcolor{blue}{Maybe 2nd paragraph:}

% notes:
 %As the scanning angle increases, the measured gain decreases and the main
 %beam broadens(2bit ris)~\cite{dai2020reconfigurable} also, The half-power beamwidths are 9.1 and 8.8 , respectively, and the measured sidelobe levels are 16.7 dB and 16.4 dB
 %bj -> 8 db side lobe (1bit) sto greedy
 
Despite its great potential, the RIS technology necessitates rigorous experimental validation to substantiate its theoretical benefits. The imperfections of RIS prototypes significantly complicate the design of effective codebook-based beamforming strategies. Issues such as strong side lobes in radiation patterns of 1-bit RIS designs~\cite{10498085,sayanskiy20222d}, phase-dependent amplitude response~\cite{10095408,kompostiotis2023secrecy}, frequency selectivity~\cite{rains2023fully}, dependence on the angle of incoming signals~\cite{9551980,weinberger2024validating}, and the practical difficulty of achieving precise binary phase shifts~\cite{9551980,rains2023fully}, meeting the $180^\circ$ difference, are critical limitations. Furthermore, mutual coupling between adjacent RIS elements~\cite{zheng2024mutual,9837634}, limited scanning range, and the impracticality of achieving narrow beams at certain angles with 1-bit resolution~\cite{dai2020reconfigurable}, further degrade performance. These hardware-related imperfections, combined with environmental challenges such as multipath propagation~\cite{kompostiotis2024evaluation,nikonowicz2024indoor}, ground-bounce effects~\cite{trichopoulos2022design}, and interference, make the task of experimental validation even more difficult. Overcoming these challenges is crucial to ensure the practical viability of RIS-aided systems in real-world deployments.



Recent measurement-based studies have focused mainly on evaluating SNR improvements and understanding the radiation patterns of RIS~\cite{9551980,amri2021reconfigurable,dai2020reconfigurable,trichopoulos2022design}, particularly highlighting the presence of strong side lobes in low-resolution phase configurations. Some works aim to derive theoretical path loss models~\cite{tang2022path,9713744}, which have been validated through controlled experiments in anechoic chambers. However, extending these models to real outdoor scenarios~\cite{lan2023measurement} has revealed discrepancies caused by hardware imperfections and challenges inherent to real-world deployments. In this paper, we validate a subcase~\cite{ramezani2023dual,ramezani2024broad,10256051} of the model~\cite{9713744} that eliminates the need for approximations of transmitter (Tx) or receiver (Rx) antenna radiation patterns, simplifying real-world deployment considerations. Beyond system-level performance, RISs have gained significant attention for positioning applications. Ouyang et al.~\cite{ouyang2023computer} combine beam sweeping with visual information from a camera attached to the RIS to improve positioning adaptively in dynamic environments. Rahal et al.~\cite{rahal2023ris} integrate beam sweeping in combination with the Space-Alternating Generalized Expectation-maximization (SAGE) algorithm to mitigate side lobe challenges and multipath interference, refining position estimates in complex indoor settings. In contrast, our approach leverages beam sweeping, not only with offline codebooks, but also introduces a real-time configuration methodology that exploits the indoor environment's geometry and leverages non line-of-sight (NLoS) components. In addition, we propose an alternative beam sweeping strategy that minimizes the received signal power; an extra solution that complements the traditional objective of maximizing signal strength. %Furthermore, 
Spatial Time Coding Modulation (STCM) has emerged as another method for AoA estimation~\cite{gholami2024wireless,dos2024assessing}, where RIS elements are time-modulated to generate harmonics that encode spatial information. However, STCM faces significant practical challenges~\cite{gholami2024wireless,dos2024assessing}, including strict timing constraints and phase synchronization issues, which are exacerbated by RIS configuration mechanisms, such as script-based control methods that introduce timing discrepancies, making it difficult to achieve the precise modulation intervals required for clean harmonics and reducing the accuracy of AoA estimation. In contrast, beam sweeping offers a simpler and more practical solution.



%main papers: ~\cite{tang2022path,9713744} path loss model anechoic
%9713744 for fr1
%outdoor:sub6ghz~\cite{lan2023measurement} - trying to validate theoretic path loss model 

%SNR and basic test: %ris prototypes~\cite{9551980} - greedy
%SNR ~\cite{amri2021reconfigurable} - 1bit 
%2bit~\cite{dai2020reconfigurable} - insights for rad pattern
%SNR outdoor,  prototype~\cite{trichopoulos2022design}
%max +20, min -60~\cite{tewes2023comprehensive}

%positioning:
%beam sw + camera for dynamic loc~\cite{ouyang2023computer}
%~\cite{rahal2023ris}: beam sweeping and SAGE, side lobes lead to large errors, mmwave, 1bit RIS, indoor. does not exploit indoor real time beamforming. beam sw 5deg step. best or 3 max or sage.
%STCM~\cite{gholami2024realization,gholami2024wireless,dos2024assessing}
%In this paper,


The remainder of the paper is organized as follows: Section~\ref{s:theoretical_model} introduces the theoretical model for the radiation pattern of RIS, which is validated through real outdoor measurements. Section~\ref{s:phase_optimization} presents one phase optimization method to configure the RIS prototype during the experiments. Section~\ref{s:measurement_setup} describes the measurement setup, including both outdoor and indoor environments, as well as the key system components. Then in Section~\ref{s:experimental_results}, we make several experiments
 to verify the system performance. This section further introduces the proposed beam sweeping methodology and evaluates its performance in both outdoor and indoor environments. Finally, Section~\ref{s:conclusion} concludes the paper.%, while Section~\ref{s:future_work} outlines future research directions.


\section{Radiation Pattern of RIS: Theoretical model}\label{s:theoretical_model}
\vspace{0.05cm}
%This section presents the baseband model of the RIS-aided system and the theoretical model for the RIS radiation pattern. 
We consider a single-antenna Tx, an RIS with $N {=} N_x {\times} N_y$ elements, where $N_x$ and $N_y$ denote the number of rows and columns, respectively, and a single-antenna Rx. The received signal, over a single subcarrier, can be modeled as in~\cite{rabault2024tacit}:
\begin{equation}
    y =  (\bm{h}_{r}^\mathsf{H}\bm{\Theta}\bm{h}_{g} + h_{d})s + z, 
    \label{e:rx_signal_A}
\end{equation}
where \( \bm{h}_{g},\bm{h}_{r} {\in} \mathbb{C}^{N \times 1} \) %and \( \bm{h}_{r} {\in} \mathbb{C}^{N \times 1} \) 
are the channels from the Tx to the RIS and from the RIS to the Rx, respectively; \( h_{d} {\in} \mathbb{C} \) is the direct link channel between the Tx and Rx; and \( s \) is the transmitted symbol with zero-mean and unit variance \( \mathbb{E}(|s|^2) {=} 1 \). The RIS is configured with a diagonal phase-shift matrix
\begin{comment}
\begin{equation}
    \bm{\Theta} = \operatorname{diag}(\bm{\omega_\theta}^\mathsf{H}), \quad \bm{\omega_\theta} = [e^{\text{j}\theta_{1}}, \dots, e^{\text{j}\theta_{N}}]^\mathsf{T},
\end{equation}
\end{comment}
$ \bm{\Theta} = \operatorname{diag}(\bm{\omega_\theta}^\mathsf{H})$, where $\bm{\omega_\theta} = [e^{\text{j}\theta_{1}}, \dots, e^{\text{j}\theta_{N}}]^\mathsf{T}$ and \( \theta_n \) represents the phase shift applied by the \(n\)-th RIS element. As \eqref{e:rx_signal_A} shows, \( \bm{\Theta} \) directly influences the received signal, allowing the RIS to manipulate the end-to-end channel. 

Regarding the theoretical model of the RIS radiation pattern~\cite{ramezani2023dual,ramezani2024broad,10256051}, it represents a specific subcase of the general model introduced in~\cite{9713744}, eliminating the need for approximations of the Tx or Rx antenna radiation patterns. This simplification is particularly valuable for facilitating practical deployment in real-world scenarios. The overall radiation pattern \( G(\phi,\theta) \) is expressed as~\cite{ramezani2024broad,10256051},
\begin{equation}
    G(\phi,\theta) = A(\phi,\theta) G_{R0}(\tilde{\phi}, \tilde{\theta}) G_{R0}(\phi,\theta),
    \label{eq:radiation_pattern}
\end{equation}
where \( (\tilde{\phi}, \tilde{\theta}) \) denote the AoA at the RIS, \( (\phi, \theta) \) are the angles of departure (AoD) from the RIS, and \( G_{R0} \) represents the radiation pattern of a single RIS element, modeled using the 3GPP antenna gain model~\cite{ramezani2024broad,10256051}. The power-domain array factor is given by,
\begin{equation}
    A(\phi, \theta) = \big| \bm{\omega}_{\bm{\theta}}^T \big(\mathbf{a}_{\text{RIS}}(\tilde{\phi},\tilde{\theta}) \odot \mathbf{a}^{*}_{\text{RIS}}(\phi, \theta)\big) \big|^2,
    \label{eq:power_factor}
\end{equation}
where \( \odot \) denotes the Hadamard product, and \( \mathbf{a}_{\text{RIS}}(\phi, \theta) \) is the array response vector for the RIS. For an RIS element located at \( \mathbf{u}_i \), \( \mathbf{a}_{\text{RIS}}(\phi, \theta) \) is defined as~\cite{ramezani2024broad,10256051}:
\begin{equation}
    \mathbf{a}_{\text{RIS}}(\phi, \theta) = \Big[e^{-\jmath\mathbf{k}(\phi, \theta)^\mathsf{T}\mathbf{u}_1}, \dots, e^{-\jmath\mathbf{k}(\phi, \theta)^\mathsf{T}\mathbf{u}_N}\Big]^\mathsf{T},
    \label{e:array_response_vec}
\end{equation}
where \( \mathbf{u}_i \) is the position of the \( i\)-th RIS element, and \( \mathbf{k}(\phi, \theta) \) represents the wave vector of the transmitted signal:
\begin{equation}
    \mathbf{k}(\phi, \theta) = \frac{2\pi}{\lambda} \Big[\cos(\theta)\cos(\phi), \cos(\theta)\sin(\phi), \sin(\theta)\Big]^\mathsf{T},
    \label{e:wave_vector}
\end{equation}
with \( \lambda \) being the wavelength.


\section{Practical Phase Optimization Method}\label{s:phase_optimization} 
The $\text{A}(\varphi,\theta)$ is maximized for
\begin{equation}
    \bm{\omega}_{\bm{\theta}}^{T} = (\mathbf{a}_{\text{RIS}}(\varphi_{\text{AoA}},\theta_{\text{AoA}}) \odot \mathbf{a}^{*}_{\text{RIS}}(\varphi,\theta))^{H}.
    \label{opt_config}
\end{equation}
%The result in~\eqref{opt_config} corresponds 
in %to 
the continuous case, where each RIS element achieves infinite angular resolution. However, practical RIS implementations often restrict each element to discrete phase-shift values, such as \( \theta_i {\in} \{-\frac{\pi}{2}, \frac{\pi}{2}\}, \forall i \). Quantizing the optimal continuous solution (1-bit resolution) does not guarantee that the RIS will steer the beam toward the desired AoD and results in strong sidelobes~\cite{10498085,sayanskiy20222d}, which can significantly degrade performance. Therefore, this section presents a practical RIS phase optimization method based on column-row scanning. Grouping elements, such as by columns or rows, is a well-established approach to reduce complexity, leveraging the fact that adjacent elements typically exhibit similar channel coefficients~\cite{bjornson2021optimizing}. A similar idea to column/row changes is discussed in~\cite{9551980}. Iterative algorithms that optimize each element individually often require more iterations and, in long-range scenarios without low-noise amplifiers (LNAs), small differences in received signal power may be challenging to detect~\cite{weinberger2024validating,tewes2023comprehensive}. The applied method starts with a homogeneous surface, initializing all elements of the RIS to the same configuration. The optimization proceeds horizontally by modifying the phase-shift states of individual columns, measuring the received power at the Rx. If an improvement is observed, the change is retained; otherwise, the configuration is reverted. Next, a similar procedure is applied vertically across rows. This process runs for one complete iteration, providing a low-complexity yet effective configuration strategy. The complete procedure is detailed in Algorithm~\ref{algo:ris_config}.
\begin{algorithm}[b]
\footnotesize
\parbox[t]{0.7\textwidth}{\caption{\strut\small\textbf{RIS Configuration via Column-Row Scanning\vskip-5ex}} \label{algo:ris_config}}
\tcp{\( \bm{\omega_\theta} \) is mapped to the dual-polarized RIS's configuration matrix \( \bm{\Phi} \) of size \( N_x \times 2N_y \).}
\KwIn{Initial RIS configuration matrix \( \bm{\Phi} \) with all elements set to 0.}
\KwOut{Optimized RIS configuration \( \bm{\Phi} \).}
\nl Measure the initial received power \( P_{\text{max}} \) with \( \bm{\Phi} \) set to all 0\;
\nl 
    \tcc{\textbf{Step 1: Column-Wise Scanning}}
    \nl \For{\texttt{\( col = 1:2:2N_y \)}}{
        \nl Invert the states of columns \( col, col+1 \) in \( \bm{\Phi} \)\;
        \nl Measure the received power \( P_r \) for the current configuration\;
        \nl \If{\( P_r > P_{\text{max}} \)}{
            \nl \( P_{\text{max}} \gets P_r \) \tcp*{Keep new configuration.}
        }\Else{
            \nl Revert columns \( col, col+1 \) to its previous state\;
        }
    }
    \tcc{\textbf{Step 2: Row-Wise Scanning}}
    \nl \For{\texttt{\( row = 1, \dots, N_x \)}}{
        \nl Invert the states of row \( row \) in \( \bm{\Phi} \)\;
        \nl Measure the received power \( P_r \) for the current configuration\;
        \nl \If{\( P_r > P_{\text{max}} \)}{
            \nl \( P_{\text{max}} \gets P_r \) \tcp*{Keep new configuration.}
        }\Else{
            \nl Revert row \( row \) to its previous state\;
        }
    }
\nl \Return{\( \bm{\Phi} \)} \tcp*[l]{Return optimized RIS configuration.}
\end{algorithm}




\section{Indoor and Outdoor Measurement Setups}\label{s:measurement_setup} 
The measurement setups involve both outdoor and indoor experiments (Figs.~\ref{fig:2a} and~\ref{fig:2c}). The RIS, used in both setups, consists of four tiles, each containing a $16{\times}16$ array of unit cells~\cite{rains2023fully}. Each unit cell is equipped with two varactor diodes, each independently driven by two programmable bias voltage levels: $V_1 {=} 11$ V and $V_2 {=} 7.5$ V. The RIS operates around 3.5 GHz and supports dual-polarization (horizontal and vertical). For both transmission and reception, we used flat-panel directional antennas, specifically the PAT3519XP model from ITELITE. These antennas are dual-polarized and have a beamwidth of around $20^\circ$, which reduces the impact of unwanted reflections and interference. The antennas have a working frequency range of 3.5--3.8 GHz. The Vector Network Analyzer (VNA) used in this study is the Copper Mountain S5243, a 2-port device capable of measuring frequencies up to 44 GHz. The VNA was used to measure the S parameters, providing insight into the transmission characteristics between the Tx and the Rx. For accurate measurements, a SOLT (Short-Open-Load-Through) calibration kit T4311 was employed to calibrate the VNA, ensuring precise results.

The outdoor measurement setup (Fig.~\ref{fig:2a}) was conducted in an environment with minimal multipath interference. The Tx and Rx were positioned at a distance of 8.5 meters from the RIS, while all components were placed at a height of 1.3 meters. The measurements were taken with a step size of $5^\circ$ in the azimuth domain. In contrast, the indoor measurement setup (Fig.~\ref{fig:2c}) was conducted in a more challenging environment with strong multipath effects. The nodes were placed at the same height of 1.3 meters, but the distance between the Tx and the RIS was decreased to 5.5 meters. The multipath environment in the indoor setup introduces additional complexity, as signals reflected from walls, floors, and other objects can distort the measurements. The measurement step was $15^\circ$. Furthermore, for communication between the RIS controller (Raspberry Pi) and the VNA, a script-based system was developed. The RIS configuration and VNA were controlled via MATLAB scripts from a laptop, enabling automated measurement and data collection. This setup facilitated efficient testing and ensured that the RIS configuration could be adjusted in real-time during the measurement process, allowing for dynamic investigation.% under different conditions.


  

\begin{figure}[t]
\centering
\begin{minipage}{0.47\textwidth}
  \vspace*{2mm}
  \centering
  \includegraphics[width=0.9\textwidth]{outdoor_setuppp.jpg}
  \subcaption{Outdoor setup: The Tx-RIS distance is 8.5~m, and the RIS-Rx distance is 8.5~m for all Rx positions, with all nodes positioned at a height of 1.3~m. Rx positions range from $0^\circ$ to $60^\circ$, with a step size of $5^\circ$, %Rx positions measured in $5^\circ$ increments from $0^\circ$ to $60^\circ$, 
  and the Tx is fixed at $-15^\circ$. RIS configurations are optimized to steer the beam toward each Rx position.}\label{fig:2a}
\end{minipage}%
%\hfill
\vspace{0.2cm}
\begin{minipage}{0.47\textwidth}
  \centering
  \includegraphics[width=0.9\textwidth]{confimages/indoor_setup_new.jpg}
  \subcaption{Indoor setup: The Tx-RIS distance is 5.5~m, and the RIS-Rx distance is 8.5~m for all Rx positions, with all nodes positioned at a height of 1.3~m. Rx positions range from $0^\circ$ to $45^\circ$, with a step size of $15^\circ$
  %Rx positions measured in $15^\circ$ increments from $0^\circ$ to $45^\circ$, 
  and the Tx if fixed at $-15^\circ$. RIS configurations are optimized to steer the beam toward each Rx position.}\label{fig:2c}
\end{minipage}%
%\vspace{0.1cm}
\caption{Experimental setups for outdoor and indoor environments.}\label{fig:exp_4images}
\end{figure}


\begin{comment}
    

\begin{figure*}[t]
\centering
\begin{minipage}{0.48\textwidth}
  \centering
  \includegraphics[width=0.9\textwidth]{confimages/outdoor_setupp.jpg}
  \subcaption{Outdoor setup: The Tx-RIS distance is 8.5~m, and the RIS-Rx distance is 8.5~m for all Rx positions, with all nodes positioned at a height of 1.3~m.}\label{fig:2a}
\end{minipage}%
\hfill
\begin{minipage}{0.48\textwidth}
  \centering
  \includegraphics[width=0.9\textwidth]{confimages/outdoor_setup.png}
  \subcaption{Outdoor setup: Rx positions measured in $5^\circ$ increments from $0^\circ$ to $60^\circ$, with the Tx fixed at $-15^\circ$. RIS configurations are optimized to steer the beam toward each Rx position.}\label{fig:2b}
\end{minipage}%
\vspace{0.2cm}
\begin{minipage}{0.48\textwidth}
  \centering
  \includegraphics[width=0.9\textwidth]{confimages/indoor_setup.jpg}
  \subcaption{Indoor setup: The Tx-RIS distance is 5.5~m, and the RIS-Rx distance is 8.5~m for all Rx positions, with all nodes positioned at a height of 1.3~m.}\label{fig:2c}
\end{minipage}%
\hfill
\begin{minipage}{0.48\textwidth}
  \centering
  \includegraphics[width=0.9\textwidth]{confimages/test_matlab_ind.PNG}
  \subcaption{Indoor setup: Rx positions measured in $15^\circ$ increments from $0^\circ$ to $45^\circ$, with the Tx fixed at $-15^\circ$. RIS configurations are optimized to steer the beam toward each Rx position.}\label{fig:2d}
\end{minipage}%
\vspace{0.2cm}
\caption{Experimental setups for outdoor and indoor environments.}\label{fig:exp_4images}
\end{figure*}

\end{comment}







\section{Experimental Results and Discussion}\label{s:experimental_results}
\begin{comment}
 
\subsection{RIS Characterization}
The RIS characterization experiment was conducted in an outdoor environment (Fig.~\ref{fig:2a}) using a specular setup. The transmitter  and receiver were equipped with directional antennas, positioned at $-15^\circ$ and $+15^\circ$, respectively, relative to the RIS. The selected outdoor area ensures a specular reflection path with limited multipath, providing controlled conditions to evaluate the RIS performance. A VNA was used to measure the S21 signal under two specific RIS configurations: \textit{all-0s} (all RIS elements set to apply a $-\pi/2$ phase shift) and \textit{all-1s} (all RIS elements set to apply a $+\pi/2$ phase shift). The key condition to validate the proper functioning of the RIS is that the phase difference between the S21 signals for the \textit{all-0} and \textit{all-1} configurations must be approximately $180^\circ$. The experiment analyzes the bandwidth limitations and frequency-dependent behavior of the RIS by examining the consistency of this phase difference across the $3.1$–$3.9$ GHz frequency band. Any deviations in the phase difference of S21 over the frequency range highlight the frequency-selective nature of the RIS and its practical operating bandwidth.


The results, presented in Fig.~\ref{fig:ris_char_combined}, show that the phase difference constraint of $180^\circ$ is not easily feasible in practice. The RIS operating frequency approximately spanning from $3.45$ GHz to $3.65$ GHz, being reasonable as it is designed for operation around $3.5$ GHz. Relaxing the phase difference constraint to $155^\circ$ extends the bandwidth to $12$ MHz, while stricter constraints, such as $175^\circ$, limit the bandwidth to just $1$ MHz. These findings underline the challenges of achieving consistent phase differences across a wide frequency range, which is critical for wideband RIS-aided communication systems.





\begin{figure}[t]
\centering
\begin{minipage}{0.45\textwidth}
\centering
\scalebox{0.5}{% This file was created by matlab2tikz.
%
%The latest updates can be retrieved from
%  http://www.mathworks.com/matlabcentral/fileexchange/22022-matlab2tikz-matlab2tikz
%where you can also make suggestions and rate matlab2tikz.
%
\definecolor{mycolor1}{rgb}{0.00000,0.44700,0.74100}%
%
\begin{tikzpicture}

\begin{axis}[%
width=4.521in,
height=3.566in,
at={(0.758in,0.481in)},
scale only axis,
xmin=3.1,
xmax=3.9,
xlabel style={font=\color{white!15!black}},
xlabel={Frequency (GHz)},
ymin=0,
ymax=180,
ylabel style={font=\color{white!15!black}},
ylabel={Phase Difference ($^\circ$)},
axis background/.style={fill=white},
xmajorgrids,
ymajorgrids
]
\addplot [color=mycolor1, line width=1.5pt, forget plot]
  table[row sep=crcr]{%
3.1	11.7880061887\\
3.101	11.8559964855\\
3.102	11.1810185932\\
3.103	12.0218751214\\
3.104	11.760122202\\
3.105	11.383098201\\
3.106	11.751409369\\
3.107	10.9066951968\\
3.108	10.86369243853\\
3.109	12.1984568878\\
3.11	10.966570302\\
3.111	10.336025096\\
3.112	10.708763815\\
3.113	11.5563543627\\
3.114	11.2771986956\\
3.115	12.5539890647\\
3.116	11.4563966027\\
3.117	11.509522179\\
3.118	10.75963182\\
3.119	10.882369948\\
3.12	10.7010044143\\
3.121	10.97906275299\\
3.122	10.4718761593\\
3.123	10.5845300173\\
3.124	10.837224157\\
3.125	11.369139049\\
3.126	11.6516660984\\
3.127	11.3360210127\\
3.128	11.58372072028\\
3.129	11.1544041567\\
3.13	10.994907313\\
3.131	11.271081292\\
3.132	11.395267547\\
3.133	10.8259919392\\
3.134	11.4889769841\\
3.135	11.1248267479\\
3.136	10.5794715726\\
3.137	10.400751352\\
3.138	11.273497133\\
3.139	10.967366075\\
3.14	11.0723512896\\
3.141	11.10745768945\\
3.142	10.46024469\\
3.143	9.895826013\\
3.144	9.54192384700002\\
3.145	10.461608283\\
3.146	9.97086881659999\\
3.147	10.1715721972\\
3.148	9.36962974479999\\
3.149	10.0705821912\\
3.15	9.88047848499997\\
3.151	10.189086581\\
3.152	10.900456354\\
3.153	10.3566605621\\
3.154	10.5816231479\\
3.155	10.9693501422\\
3.156	10.4103542476\\
3.157	9.585279802\\
3.158	9.84316208799999\\
3.159	9.70429594500001\\
3.16	9.8582440619\\
3.161	9.28821389509\\
3.162	10.0332098285\\
3.163	10.092330942\\
3.164	9.748914071\\
3.165	9.588316066\\
3.166	9.31486540970002\\
3.167	8.78279153580002\\
3.168	8.83253061299999\\
3.169	8.4518152169\\
3.17	8.24696997199999\\
3.171	8.40786600199999\\
3.172	8.34683231399998\\
3.173	8.0285724213\\
3.174	7.545617177\\
3.175	8.24557076010001\\
3.176	7.94496444560002\\
3.177	8.620504133\\
3.178	8.88873561700001\\
3.179	8.9050446153\\
3.18	8.44913116609999\\
3.181	8.55394826196601\\
3.182	7.96121762709998\\
3.183	8.25341936999999\\
3.184	8.05745511699999\\
3.185	7.49804454899999\\
3.186	8.31021644079999\\
3.187	8.15578236249999\\
3.188	7.4804991183\\
3.189	7.58169235190002\\
3.19	7.124678362\\
3.191	6.54326872399997\\
3.192	5.71277420299998\\
3.193	6.0149135824\\
3.194	6.1302881691\\
3.195	6.47597859370001\\
3.196	5.5140366091\\
3.197	5.756786285\\
3.198	6.29575072\\
3.199	6.68861409800002\\
3.2	6.11146409700001\\
3.201	6.11143018358001\\
3.202	6.91339805300001\\
3.203	5.80858151199999\\
3.204	5.621496909\\
3.205	5.61128087399999\\
3.206	6.13796594180002\\
3.207	6.03122125250002\\
3.208	6.34169969958\\
3.209	5.6222271259\\
3.21	4.93532542899999\\
3.211	5.093976266\\
3.212	4.18905107500001\\
3.213	4.73657611689998\\
3.214	4.3718982771\\
3.215	4.8408883107\\
3.216	3.93655588979999\\
3.217	3.306037162\\
3.218	3.64223383000001\\
3.219	2.741449085\\
3.22	3.45162503429998\\
3.221	3.2066893656\\
3.222	2.9194362552\\
3.223	3.4813150064\\
3.224	3.89759475\\
3.225	4.023142552\\
3.226	3.2454034669\\
3.227	4.63520011129998\\
3.228	3.94114496202999\\
3.229	3.7138260927\\
3.23	2.042418809\\
3.231	2.74628848100002\\
3.232	2.54730544999998\\
3.233	2.90331481600001\\
3.234	2.17270849260001\\
3.235	1.91560081860001\\
3.236	1.42006408029999\\
3.237	1.36060264100001\\
3.238	1.63372580399999\\
3.239	1.690506253\\
3.24	2.29114209649998\\
3.241	1.74549991550001\\
3.242	2.00535708500001\\
3.243	1.17490013150001\\
3.244	0.707427812999981\\
3.245	0.555735435000003\\
3.246	1.08019991200001\\
3.247	1.3480463157\\
3.248	1.51189864259999\\
3.249	1.57893445560001\\
3.25	0.607947233999994\\
3.251	0.0156513150000137\\
3.252	1.68153865099998\\
3.253	1.01110805000002\\
3.254	2.43553225300002\\
3.255	1.49003257430999\\
3.256	2.23304681870002\\
3.257	2.20941524400001\\
3.258	1.70134384300002\\
3.259	1.16716197400001\\
3.26	0.191881297800023\\
3.261	0.449921569200001\\
3.262	0.220819060129998\\
3.263	0.665742564400006\\
3.264	0.907668188000002\\
3.265	1.241845789\\
3.266	0.783336693999985\\
3.267	1.15452959020001\\
3.268	1.7251693636\\
3.269	1.6544981504\\
3.27	1.80027746069999\\
3.271	2.01377650000001\\
3.272	1.85619612100001\\
3.273	1.00304873100001\\
3.274	0.927334985199991\\
3.275	2.10829549510001\\
3.276	2.66255981259999\\
3.277	4.10614429980001\\
3.278	3.76630766700001\\
3.279	4.99387596800003\\
3.28	4.949821727\\
3.281	5.19657759519998\\
3.282	6.12957993910001\\
3.283	5.73701025139999\\
3.284	6.8055413731\\
3.285	5.93347537700001\\
3.286	5.49701574700001\\
3.287	6.262810449\\
3.288	5.3900407244\\
3.289	5.40180204263001\\
3.29	4.41471571389999\\
3.291	4.74898862200001\\
3.292	5.55962486200002\\
3.293	5.373909492\\
3.294	3.82199298770001\\
3.295	4.94190331339999\\
3.296	5.30374793764\\
3.297	4.8258124896\\
3.298	5.00405140399999\\
3.299	5.712459645\\
3.3	6.98251591499999\\
3.301	6.365151588\\
3.302	7.45258382989999\\
3.303	8.16227761213\\
3.304	9.54812519749999\\
3.305	10.227170114\\
3.306	9.19113835800002\\
3.307	11.592729715\\
3.308	9.8370287839\\
3.309	9.91738846460001\\
3.31	10.1265259406\\
3.311	10.4568331278\\
3.312	10.618862442\\
3.313	12.334225756\\
3.314	10.975551204\\
3.315	12.1727011026\\
3.316	13.20798355209\\
3.317	12.8409402887\\
3.318	11.4809037965\\
3.319	10.897642976\\
3.32	11.690380956\\
3.321	12.5875677409\\
3.322	11.0116671746\\
3.323	9.80663456455\\
3.324	9.91112419680002\\
3.325	9.49427251269998\\
3.326	10.370544179\\
3.327	9.367014837\\
3.328	9.6901024638\\
3.329	9.68308364770002\\
3.33	10.60300932934\\
3.331	10.7056990603\\
3.332	10.763028213\\
3.333	12.607987281\\
3.334	13.338158055\\
3.335	14.8356526301\\
3.336	16.0382913851\\
3.337	17.48703353788\\
3.338	19.8920288895\\
3.339	20.449509966\\
3.34	20.612859495\\
3.341	21.502851542\\
3.342	24.6016783048\\
3.343	23.33917604148\\
3.344	23.2765516124\\
3.345	22.2124258338\\
3.346	21.132198788\\
3.347	21.091259636\\
3.348	20.8893593669\\
3.349	19.789529506\\
3.35	20.96979651714\\
3.351	18.8019439056\\
3.352	19.0984666159\\
3.353	18.31231143\\
3.354	18.470941041\\
3.355	17.4243070598\\
3.356	16.99123567\\
3.357	16.7094411636\\
3.358	16.9585550139\\
3.359	18.8901042973\\
3.36	19.561519556\\
3.361	21.382271116\\
3.362	23.9915579925\\
3.363	23.1149608315\\
3.364	24.4920494248\\
3.365	26.4167402902\\
3.366	28.3181730354\\
3.367	29.638792239\\
3.368	35.128679873\\
3.369	35.3819665292\\
3.37	38.4758231231\\
3.371	40.92371578756\\
3.372	48.8673332054\\
3.373	51.396528617\\
3.374	47.240946317\\
3.375	45.9572352835\\
3.376	43.8034102988\\
3.377	41.28919980385\\
3.378	41.8217703284\\
3.379	38.5639702915\\
3.38	34.909395566\\
3.381	31.773973524\\
3.382	32.8389664803\\
3.383	33.2190461191\\
3.384	32.67237448122\\
3.385	35.8821729492\\
3.386	38.92209158\\
3.387	32.218562424\\
3.388	32.453955678\\
3.389	34.009153246\\
3.39	35.3311272388\\
3.391	38.98989512087\\
3.392	39.2223677472\\
3.393	44.054941962\\
3.394	47.309764021\\
3.395	52.888796305\\
3.396	59.7156087215\\
3.397	66.7435078804\\
3.398	72.9613020348\\
3.399	74.5086720499\\
3.4	73.1202743866\\
3.401	71.855907914\\
3.402	70.535616722\\
3.403	68.6618491731\\
3.404	64.49542546701\\
3.405	62.33070630239\\
3.406	62.6640551492\\
3.407	61.275022343\\
3.408	60.925089284\\
3.409	64.1166793284\\
3.41	60.0388269257\\
3.411	55.31338578938\\
3.412	54.81713226916\\
3.413	54.4017577165\\
3.414	54.91900337\\
3.415	53.172095027\\
3.416	53.3393595349\\
3.417	53.8440144568\\
3.418	53.26161442263\\
3.419	56.45534313902\\
3.42	65.1185952423\\
3.421	66.074652677\\
3.422	72.823358435\\
3.423	73.6130836939\\
3.424	80.2342238097\\
3.425	83.6815479843\\
3.426	91.703000859789\\
3.427	90.6826097953\\
3.428	94.559283679\\
3.429	102.7682570533\\
3.43	111.2757253168\\
3.431	117.8888766532\\
3.432	112.0104775433\\
3.433	108.4955096767\\
3.434	106.5741736518\\
3.435	101.895974573\\
3.436	95.4847551974\\
3.437	85.0695322906\\
3.438	81.2392790517\\
3.439	73.54594573683\\
3.44	68.818292701\\
3.441	69.237830342\\
3.442	67.998241593\\
3.443	69.4336899893\\
3.444	71.1994212043\\
3.445	73.638417642\\
3.446	78.5768728761\\
3.447	78.0549340963\\
3.448	80.1033280268\\
3.449	78.723457577\\
3.45	82.8251337881\\
3.451	88.1471335035\\
3.452	98.2648325247\\
3.453	109.97323279549\\
3.454	124.2502901736\\
3.455	143.3337154515\\
3.456	151.4602567113\\
3.457	156.53584821458\\
3.458	162.8638290984\\
3.459	162.1991757932\\
3.46	152.89910532028\\
3.461	141.3352136904\\
3.462	131.215685688\\
3.463	117.6802778295\\
3.464	105.0845828594\\
3.465	93.3373415805\\
3.466	89.23507376397\\
3.467	90.6192306381\\
3.468	89.2382638856\\
3.469	87.85065113\\
3.47	84.5163557122\\
3.471	85.0170160077\\
3.472	87.9253617029\\
3.473	81.993155436\\
3.474	71.0526584645\\
3.475	71.3584654482\\
3.476	74.955955994\\
3.477	86.202522435\\
3.478	87.5079254808\\
3.479	94.9339150591\\
3.48	116.1288063022\\
3.481	146.9711616848\\
3.482	174.6739547812\\
3.483	170.5486363396\\
3.484	165.5711925756\\
3.485	172.6883772595\\
3.486	177.3935393311\\
3.487	174.1339266812\\
3.488	174.5776147232\\
3.489	173.1473351327\\
3.49	164.4591637392\\
3.491	156.6631152955\\
3.492	155.1983649273\\
3.493	147.1467299453\\
3.494	132.74876746126\\
3.495	132.3936039901\\
3.496	117.188075653\\
3.497	111.5885160156\\
3.498	108.7760880973\\
3.499	97.0387734208\\
3.5	86.9613525818\\
3.501	79.4754128675\\
3.502	78.7350927011\\
3.503	82.630918789\\
3.504	81.1591505896\\
3.505	77.1327058844\\
3.506	78.2996944253\\
3.507	86.6139129795\\
3.508	90.3153921479\\
3.509	96.827366116\\
3.51	98.434098158\\
3.511	110.6412695509\\
3.512	117.0511130197\\
3.513	121.6751221331\\
3.514	141.7565624692\\
3.515	162.36390990089\\
3.516	177.1654303912\\
3.517	164.9399035766\\
3.518	159.4301954272\\
3.519	163.6794448117\\
3.52	162.2116752484\\
3.521	163.8123819425\\
3.522	167.2500704777\\
3.523	172.1597276043\\
3.524	153.2352513945\\
3.525	137.9671508396\\
3.526	108.0527834717\\
3.527	90.8093349709\\
3.528	82.1363278887\\
3.529	90.6106900447\\
3.53	89.708280649\\
3.531	89.1631405386\\
3.532	89.3261933386\\
3.533	94.7315169059\\
3.534	97.3655664306\\
3.535	96.9777853684\\
3.536	92.2591790987\\
3.537	95.365626434\\
3.538	98.865094208\\
3.539	108.6190198876\\
3.54	119.758399579\\
3.541	128.4951164984\\
3.542	140.65543578742\\
3.543	157.8010459738\\
3.544	169.9407224583\\
3.545	173.442269647\\
3.546	172.72393749992\\
3.547	170.4650773192\\
3.548	170.190765385\\
3.549	179.2091898737\\
3.55	177.0190479763\\
3.551	168.6065082226\\
3.552	158.1213350533\\
3.553	151.7883066428\\
3.554	150.8835465088\\
3.555	177.4713389236\\
3.556	175.308243714\\
3.557	152.5387711181\\
3.558	118.1760082581\\
3.559	97.1217740864\\
3.56	87.164623742136\\
3.561	78.7760209145\\
3.562	79.1086027914\\
3.563	75.0982692289\\
3.564	82.671461461\\
3.565	88.21699603\\
3.566	86.7189771998\\
3.567	94.22647347494\\
3.568	96.8537817101\\
3.569	106.76873102282\\
3.57	111.8767813095\\
3.571	121.0444577528\\
3.572	130.9796469406\\
3.573	142.1572073597\\
3.574	151.8945856683\\
3.575	168.3506195316\\
3.576	176.2751664062\\
3.577	173.50798513059\\
3.578	161.0415927726\\
3.579	152.5897373655\\
3.58	150.91298113163\\
3.581	141.6381396672\\
3.582	130.2642359345\\
3.583	125.5216253476\\
3.584	110.650936767\\
3.585	94.9445296281\\
3.586	81.009687804\\
3.587	72.56800991077\\
3.588	57.7005481634\\
3.589	49.1987111691\\
3.59	14.985838519\\
3.591	28.169509067\\
3.592	44.277869342\\
3.593	78.4060060772\\
3.594	99.5390441836\\
3.595	98.7912944841\\
3.596	105.0555003313\\
3.597	117.4708712775\\
3.598	129.1271478948\\
3.599	128.463138683\\
3.6	138.4949076993\\
3.601	148.3329925137\\
3.602	161.6415576234\\
3.603	163.9126551147\\
3.604	162.2840489696\\
3.605	170.0043839074\\
3.606	177.3776086202\\
3.607	172.3906445095\\
3.608	170.9789487113\\
3.609	169.2382896579\\
3.61	159.5921328747\\
3.611	156.2380585053\\
3.612	157.4083736284\\
3.613	158.2508918381\\
3.614	155.8591548579\\
3.615	151.5672276871\\
3.616	153.8525182151\\
3.617	160.5885855872\\
3.618	168.2099474727\\
3.619	175.7610309198\\
3.62	176.4848173414\\
3.621	172.8483192263\\
3.622	161.6619760156\\
3.623	153.1888528846\\
3.624	149.2516803857\\
3.625	138.9506182182\\
3.626	132.9172999573\\
3.627	129.4107133529\\
3.628	130.1186820148\\
3.629	130.6861283685\\
3.63	132.7969435308\\
3.631	135.78284834031\\
3.632	137.9857808772\\
3.633	136.914998979\\
3.634	136.7540410534\\
3.635	139.0229562249\\
3.636	141.8690575728\\
3.637	142.7522977343\\
3.638	144.08356767802\\
3.639	144.3374460501\\
3.64	144.48376453\\
3.641	144.211989083\\
3.642	143.81288501257\\
3.643	143.2509393694\\
3.644	141.6568495648\\
3.645	140.1306625779\\
3.646	137.2172596618\\
3.647	130.0116177137\\
3.648	123.99457181\\
3.649	119.9080522163\\
3.65	119.3701712964\\
3.651	117.7857181766\\
3.652	115.5134184217\\
3.653	111.325267819\\
3.654	110.0148992942\\
3.655	109.6433812283\\
3.656	108.2712151268\\
3.657	106.7071658525\\
3.658	103.1121753546\\
3.659	101.1508321856\\
3.66	101.224605127\\
3.661	100.4342347928\\
3.662	99.3265003904\\
3.663	99.2024591239\\
3.664	99.3158799257\\
3.665	99.73515294\\
3.666	101.1029851713\\
3.667	100.34973735\\
3.668	99.2636007347\\
3.669	98.57913942237\\
3.67	97.0997201127\\
3.671	96.26356757954\\
3.672	93.1016793938\\
3.673	90.316277234\\
3.674	89.703607579\\
3.675	89.580129181\\
3.676	88.81159007148\\
3.677	85.9528021339\\
3.678	85.4086930315\\
3.679	83.8908840645\\
3.68	81.81107333\\
3.681	80.302395837\\
3.682	78.2218312779\\
3.683	76.3376348887\\
3.684	75.41428753048\\
3.685	74.5375591967\\
3.686	72.609998395\\
3.687	72.325240261\\
3.688	71.6944468967\\
3.689	70.8993579914\\
3.69	71.0166204638\\
3.691	70.12581135225\\
3.692	69.3753608944\\
3.693	69.585991686\\
3.694	68.651567088\\
3.695	67.3536062935\\
3.696	65.7064001471\\
3.697	65.0304264053\\
3.698	64.7966066365\\
3.699	64.6778006883\\
3.7	63.619390704\\
3.701	62.669057159\\
3.702	61.6581763005\\
3.703	60.84363801986\\
3.704	59.77524418636\\
3.705	58.120915848\\
3.706	57.709870376\\
3.707	57.321842686\\
3.708	56.6055071365\\
3.709	55.6246490348\\
3.71	54.5597935404\\
3.711	54.077578164\\
3.712	53.5698673109\\
3.713	52.419936378\\
3.714	51.798824494\\
3.715	51.2116479901\\
3.716	51.1824231358\\
3.717	50.9421842561\\
3.718	50.0820634096\\
3.719	49.4448175092\\
3.72	49.170136132\\
3.721	48.566049543\\
3.722	47.9020934617\\
3.723	47.2207323023\\
3.724	46.5418382791\\
3.725	45.8904894615\\
3.726	44.5444030986\\
3.727	44.154434128\\
3.728	43.557948735\\
3.729	43.0530834937\\
3.73	42.89480221045\\
3.731	41.96664893948\\
3.732	41.0513923274\\
3.733	40.496430693\\
3.734	40.506799501\\
3.735	40.0707850945\\
3.736	39.2813863917\\
3.737	38.62221252042\\
3.738	37.731500143\\
3.739	37.4448564391\\
3.74	37.015062606\\
3.741	36.334303791\\
3.742	36.3161319043\\
3.743	35.8279182143\\
3.744	35.5494992082\\
3.745	34.7795488989\\
3.746	34.3449937485\\
3.747	33.852000524\\
3.748	33.115649323\\
3.749	33.3464200791\\
3.75	32.9750411088\\
3.751	32.65718372297\\
3.752	32.42133091\\
3.753	31.862340699\\
3.754	31.429378963\\
3.755	30.905969394\\
3.756	30.230391149\\
3.757	30.22161654897\\
3.758	29.5023753675\\
3.759	29.1628852489\\
3.76	29.026794004\\
3.761	28.207640439\\
3.762	27.8395077927\\
3.763	28.0336530989\\
3.764	27.409873752\\
3.765	26.8200085517\\
3.766	26.5029313036\\
3.767	25.936129125\\
3.768	25.230497699\\
3.769	24.9806015265\\
3.77	24.1554965005\\
3.771	23.79243916012\\
3.772	23.288760952\\
3.773	23.140876797\\
3.774	23.210562725\\
3.775	23.166396539\\
3.776	22.789755745\\
3.777	22.4940425739\\
3.778	21.8285006679\\
3.779	21.2374819489\\
3.78	21.172014497\\
3.781	20.215018277\\
3.782	20.048089711\\
3.783	19.517620518\\
3.784	19.38154639165\\
3.785	18.8468349317\\
3.786	19.259974071\\
3.787	19.308257499\\
3.788	19.816218266\\
3.789	19.417810599\\
3.79	19.3131943697\\
3.791	18.32961243443\\
3.792	17.6760089398\\
3.793	17.2581607938\\
3.794	16.616805179\\
3.795	16.233867987\\
3.796	15.9841916826\\
3.797	15.6410918072\\
3.798	15.29830828782\\
3.799	15.0003190832\\
3.8	14.679006512\\
3.801	14.27587669\\
3.802	14.182465428\\
3.803	14.0499739178\\
3.804	12.8255146894\\
3.805	12.1483872234\\
3.806	12.0399551783\\
3.807	11.719973967\\
3.808	11.521030257\\
3.809	11.465426868\\
3.81	11.3694231943\\
3.811	11.3587203428\\
3.812	11.1869610977\\
3.813	10.6486210716\\
3.814	10.628963973\\
3.815	10.8225396400001\\
3.816	10.17842473\\
3.817	9.68780045559998\\
3.818	9.40595620409999\\
3.819	9.63311420510001\\
3.82	9.6553728129\\
3.821	9.110895011\\
3.822	9.42313630000001\\
3.823	9.45367280599999\\
3.824	9.28752773389999\\
3.825	9.24006162478\\
3.826	8.71900130930001\\
3.827	8.40586223240001\\
3.828	8.30899121900001\\
3.829	8.18424796799999\\
3.83	7.90444780999999\\
3.831	7.39732185420002\\
3.832	7.24275420160001\\
3.833	6.62157037129998\\
3.834	6.31718547399998\\
3.835	6.33243808699999\\
3.836	5.66814982400001\\
3.837	5.48117536019998\\
3.838	5.0048866329\\
3.839	4.81385693596999\\
3.84	5.17972996040001\\
3.841	4.38110561400001\\
3.842	4.77755114700003\\
3.843	4.946321966\\
3.844	5.01513047840001\\
3.845	5.75243450560001\\
3.846	5.12997057717999\\
3.847	4.80607156460002\\
3.848	4.23292431599998\\
3.849	4.36557243299998\\
3.85	4.05810386000002\\
3.851	4.42118007660002\\
3.852	4.65737706139998\\
3.853	4.9339311139\\
3.854	5.38922666529999\\
3.855	4.821039203\\
3.856	4.524028222\\
3.857	5.28069965899999\\
3.858	6.17225216419999\\
3.859	5.9727379878\\
3.86	6.00490370590001\\
3.861	4.92261119360001\\
3.862	5.01423653000001\\
3.863	4.11643514400001\\
3.864	4.03787666099998\\
3.865	3.74066275709998\\
3.866	3.5651489258\\
3.867	4.32433523859999\\
3.868	3.9944203881\\
3.869	3.76852024300001\\
3.87	3.768949007\\
3.871	3.464028133\\
3.872	3.84875609310001\\
3.873	3.90078418752\\
3.874	4.31120908259999\\
3.875	4.04150900159999\\
3.876	3.403641252\\
3.877	3.19615107899997\\
3.878	2.94627821799997\\
3.879	3.23277762000001\\
3.88	3.59439958329801\\
3.881	4.25345172459998\\
3.882	5.11984524050001\\
3.883	5.55646666299998\\
3.884	5.01277333799999\\
3.885	4.97461288860001\\
3.886	5.0815367898\\
3.887	5.213851003\\
3.888	5.0124787149\\
3.889	4.743263278\\
3.89	5.20817840800001\\
3.891	5.088728291\\
3.892	5.18361465880002\\
3.893	4.9178975568\\
3.894	4.9916947886\\
3.895	4.34546739019999\\
3.896	4.509439853\\
3.897	4.578094002\\
3.898	4.12978930399998\\
3.899	3.64897356040001\\
3.9	3.78986402620001\\
};
\end{axis}

\begin{axis}[%
width=5.833in,
height=4.375in,
at={(0in,0in)},
scale only axis,
xmin=0,
xmax=1,
ymin=0,
ymax=1,
axis line style={draw=none},
ticks=none,
axis x line*=bottom,
axis y line*=left
]
\end{axis}
\end{tikzpicture}%}
%\vspace{-0.2cm}
\subcaption{RIS characterization}
\label{fig:ris_char_plot}
\end{minipage}%
\hfill
\begin{minipage}{0.45\textwidth}
\centering
\begin{NiceTabular}{>{\raggedright}p{3.5cm}p{2cm}}
\CodeBefore
\rowcolor{teal!10}{1}
\Body
\toprule
\textbf{Phase Difference Constraint} &  \textbf{Bandwidth} \\
\midrule
$\geq$$175^\circ$ & 1 MHz \\
\midrule
$\geq$$170^\circ$ & 5 MHz \\
\midrule
$\geq$$160^\circ$ & 8 MHz \\
\midrule
$\geq$$155^\circ$ & 12 MHz \\
\midrule
$\geq$$150^\circ$ & 21 MHz \\
\bottomrule
\end{NiceTabular}
\subcaption{Phase Difference Constraints and Bandwidth limitations}
\label{tab:ris_phase_diff}
\end{minipage}
\vspace{0.2cm}
\caption{RIS characterization results: (a) phase difference measurements between "all 0s" and "all 1s" configurations under specular reflection scenarios. (b) bandwidth limitations under different phase difference constraints. The results highlight the challenge of achieving a consistent phase difference (ideal $180^\circ$)  across a wide frequency range.}
\label{fig:ris_char_combined}
\end{figure}

\end{comment}





\subsection{Radiation Pattern of RIS: Codebook-based Beamforming}
In this subsection, we analyze the radiation patterns of the RIS obtained from outdoor experiments (Fig.~\ref{fig:2a}), conducted under a fixed Tx position at $-15^\circ$ and a Rx azimuth range of $0^\circ$ to $60^\circ$ in steps of $5^\circ$ (Fig.~\ref{fig:2a}). At each Rx position, Algorithm~1 was executed to optimize the RIS configuration for maximum received power. For the resulting configurations, we measured the amplitude of the $S_{21}$ signal at all Rx positions, allowing us to extract the RIS radiation patterns. 

Bj{\"o}rnson et al.~\cite{bjornson2021optimizing} note that the channel coefficients of RIS elements are approximately equal within each column of the array, when scattering objects are predominantly located in the horizontal plane (limited vertical scattering). In our outdoor setup (Fig.~\ref{fig:2a}), this assumption is reasonable, as the environment lacks tall buildings or other significant vertical scatterers. The area is characterized by sparse trees and open ground, and the Tx, Rx, and RIS were all positioned at the same height of $1.3$ meters, further minimizing vertical reflections. When Algorithm~1 was executed in real time to maximize the received power, the resulting RIS configurations revealed unexpected behavior in the last rows of the array (Fig.~\ref{fig:ris_configs}). Specifically, the lower rows deviated from the expected columnar structure. Given the limited vertical scattering in the outdoor environment, this deviation is likely caused by ground bounce effects, where signals reflected off the ground interact with the direct signal, enhancing or distorting the received power at the Rx. To further investigate this phenomenon, we compared the measured radiation patterns with the theoretical model of Sec.~\ref{s:theoretical_model} under two scenarios: using the measured RIS configuration as input to the model (Case 1), and modifying the last rows by extending the states of the corresponding upper rows within the same columns (Case 2). 

\begin{figure*}[t]
    \centering
    \begin{minipage}{0.185\textwidth}
        \centering
        \scalebox{0.24}{% This file was created by matlab2tikz.
%
%The latest updates can be retrieved from
%  http://www.mathworks.com/matlabcentral/fileexchange/22022-matlab2tikz-matlab2tikz
%where you can also make suggestions and rate matlab2tikz.
%
\begin{tikzpicture}
\useasboundingbox (0,-0.5) rectangle (11.5,6.1);
\begin{axis}[%
width=4.521in,
height=2.26in,
%at={(0.758in,1.134in)},
scale only axis,
point meta min=0,
point meta max=1,
axis on top,
xmin=0.5,
xmax=64.5,
xtick={\empty},
xlabel style={font=\color{white!15!black}},
xlabel={\Huge Columns},
y dir=reverse,
ymin=0.5,
ymax=32.5,
ytick={\empty},
ylabel style={font=\color{white!15!black}},
ylabel={\Huge Rows},
axis background/.style={fill=white}
]
\addplot [forget plot] graphics [xmin=0.5, xmax=64.5, ymin=0.5, ymax=32.5] {RIS_conf0-1.png};
\end{axis}

% \begin{axis}[%
% width=5.833in,
% height=4.375in,
% at={(0in,0in)},
% scale only axis,
% xmin=0,
% xmax=1,
% ymin=0,
% ymax=1,
% axis line style={draw=none},
% ticks=none,
% axis x line*=bottom,
% axis y line*=left
% ]
% \end{axis}
\end{tikzpicture}%}
        \subcaption{RIS Configuration, $0^\circ$}
        \label{fig:ris_conf0}
    \end{minipage}
    \begin{minipage}{0.185\textwidth}
        \centering
        \scalebox{0.24}{% This file was created by matlab2tikz.
%
%The latest updates can be retrieved from
%  http://www.mathworks.com/matlabcentral/fileexchange/22022-matlab2tikz-matlab2tikz
%where you can also make suggestions and rate matlab2tikz.
%
\begin{tikzpicture}
\useasboundingbox (0,-0.5) rectangle (11.5,6.1);

\begin{axis}[%
width=4.521in,
height=2.26in,
%at={(0.758in,1.134in)},
scale only axis,
point meta min=0,
point meta max=1,
axis on top,
xmin=0.5,
xmax=64.5,
xtick={\empty},
xlabel style={font=\color{white!15!black}},
xlabel={\Huge Columns},
y dir=reverse,
ymin=0.5,
ymax=32.5,
ytick={\empty},
ylabel style={font=\color{white!15!black}},
ylabel={\Huge Rows},
axis background/.style={fill=white}
]
\addplot [forget plot] graphics [xmin=0.5, xmax=64.5, ymin=0.5, ymax=32.5] {RIS_conf15-1.png};
\end{axis}

% \begin{axis}[%
% width=5.833in,
% height=4.375in,
% at={(0in,0in)},
% scale only axis,
% xmin=0,
% xmax=1,
% ymin=0,
% ymax=1,
% axis line style={draw=none},
% ticks=none,
% axis x line*=bottom,
% axis y line*=left
% ]
% \end{axis}
\end{tikzpicture}%}
        \subcaption{RIS Configuration, $15^\circ$}
        \label{fig:ris_conf15}
    \end{minipage}
    \begin{minipage}{0.185\textwidth}
        \centering
        \scalebox{0.24}{% This file was created by matlab2tikz.
%
%The latest updates can be retrieved from
%  http://www.mathworks.com/matlabcentral/fileexchange/22022-matlab2tikz-matlab2tikz
%where you can also make suggestions and rate matlab2tikz.
%
\begin{tikzpicture}
\useasboundingbox (0,-0.5) rectangle (11.5,6.1);

\begin{axis}[%
width=4.521in,
height=2.26in,
%at={(0.758in,1.134in)},
scale only axis,
point meta min=0,
point meta max=1,
axis on top,
xmin=0.5,
xmax=64.5,
xtick={\empty},
xlabel style={font=\color{white!15!black}},
xlabel={\Huge Columns},
y dir=reverse,
ymin=0.5,
ymax=32.5,
ytick={\empty},
ylabel style={font=\color{white!15!black}},
ylabel={\Huge Rows},
axis background/.style={fill=white}
]
\addplot [forget plot] graphics [xmin=0.5, xmax=64.5, ymin=0.5, ymax=32.5] {RIS_conf30-1.png};
\end{axis}

\begin{axis}[%
width=5.833in,
height=4.375in,
at={(0in,0in)},
scale only axis,
xmin=0,
xmax=1,
ymin=0,
ymax=1,
axis line style={draw=none},
ticks=none,
axis x line*=bottom,
axis y line*=left
]
\end{axis}
\end{tikzpicture}%}
        \subcaption{RIS Configuration, $30^\circ$}
        \label{fig:ris_conf30}
    \end{minipage}
    \begin{minipage}{0.185\textwidth}
        \centering
        \scalebox{0.24}{% This file was created by matlab2tikz.
%
%The latest updates can be retrieved from
%  http://www.mathworks.com/matlabcentral/fileexchange/22022-matlab2tikz-matlab2tikz
%where you can also make suggestions and rate matlab2tikz.
%
\begin{tikzpicture}
\useasboundingbox (0,-0.5) rectangle (11.5,6.1);

\begin{axis}[%
width=4.521in,
height=2.26in,
%at={(0.758in,1.134in)},
scale only axis,
point meta min=0,
point meta max=1,
axis on top,
xmin=0.5,
xmax=64.5,
xtick={\empty},
xlabel style={font=\color{white!15!black}},
xlabel={\Huge Columns},
y dir=reverse,
ymin=0.5,
ymax=32.5,
ytick={\empty},
ylabel style={font=\color{white!15!black}},
ylabel={\Huge Rows},
axis background/.style={fill=white}
]
\addplot [forget plot] graphics [xmin=0.5, xmax=64.5, ymin=0.5, ymax=32.5] {RIS_conf45-1.png};
\end{axis}

\begin{axis}[%
width=5.833in,
height=4.375in,
at={(0in,0in)},
scale only axis,
xmin=0,
xmax=1,
ymin=0,
ymax=1,
axis line style={draw=none},
ticks=none,
axis x line*=bottom,
axis y line*=left
]
\end{axis}
\end{tikzpicture}%}
        \subcaption{RIS Configuration, $45^\circ$}
        \label{fig:ris_conf45}
    \end{minipage}
    \begin{minipage}{0.185\textwidth}
        \centering
        \scalebox{0.24}{% This file was created by matlab2tikz.
%
%The latest updates can be retrieved from
%  http://www.mathworks.com/matlabcentral/fileexchange/22022-matlab2tikz-matlab2tikz
%where you can also make suggestions and rate matlab2tikz.
%
\begin{tikzpicture}
\useasboundingbox (0,-0.5) rectangle (11.5,6.1);

\begin{axis}[%
width=4.521in,
height=2.26in,
%at={(0.758in,1.134in)},
scale only axis,
point meta min=0,
point meta max=1,
axis on top,
xmin=0.5,
xmax=64.5,
xtick={\empty},
xlabel style={font=\color{white!15!black}},
xlabel={\Huge Columns},
y dir=reverse,
ymin=0.5,
ymax=32.5,
ytick={\empty},
ylabel style={font=\color{white!15!black}},
ylabel={\Huge Rows},
axis background/.style={fill=white}
]
\addplot [forget plot] graphics [xmin=0.5, xmax=64.5, ymin=0.5, ymax=32.5] {RIS_conf60-1.png};
\end{axis}

\begin{axis}[%
width=5.833in,
height=4.375in,
at={(0in,0in)},
scale only axis,
xmin=0,
xmax=1,
ymin=0,
ymax=1,
axis line style={draw=none},
ticks=none,
axis x line*=bottom,
axis y line*=left
]
\end{axis}
\end{tikzpicture}%}
        \subcaption{RIS Configuration, $60^\circ$}
        \label{fig:ris_conf60}
    \end{minipage}
    % 
    \caption{RIS configurations obtained after executing Algorithm~1 at different Rx azimuth angles. Each configuration corresponds to a $32 \times 64$ RIS, where each element supports both horizontal and vertical polarizations, resulting in double columns. The last rows deviate from the expected column-wise behavior~\cite{bjornson2021optimizing}, likely due to ground bounce effects~\cite{trichopoulos2022design}. This behavior is consistently observed across all the Rx positions of Fig.~\ref{fig:2a}.}
    \label{fig:ris_configs}
\end{figure*}


Fig.~\ref{fig:validation} compares the measured radiation patterns with the theoretical patterns derived from Case 1 and Case 2. 
%While the overall agreement between the measured and theoretical patterns validates the RIS radiation pattern framework, the alignment improves significantly when Case 2 is used. 
The measurements strongly validate the theoretical radiation pattern model, with Case 2 providing enhanced alignment by accounting for ground-bounce effects in the last rows. Furthermore, each radiation pattern consistently exhibits a pronounced main lobe directed toward the desired Rx angle, even for angles near the edge of the RIS scanning range (e.g., $60^\circ$), demonstrating the beamforming capability of the RIS. This highlights that the proposed codebook is well-suited for beam sweeping and is further utilized in the subsequent subsections.


\begin{figure*}[t]
    \centering
    \begin{minipage}{0.185\textwidth}
        \centering
        \scalebox{0.24}{% This file was created by matlab2tikz.
%
%The latest updates can be retrieved from
%  http://www.mathworks.com/matlabcentral/fileexchange/22022-matlab2tikz-matlab2tikz
%where you can also make suggestions and rate matlab2tikz.
%
\definecolor{mycolor1}{rgb}{0.00000,0.44700,0.74100}%
\definecolor{mycolor2}{rgb}{0.85000,0.32500,0.09800}%
\definecolor{mycolor3}{rgb}{0.92900,0.69400,0.12500}%
%
\begin{tikzpicture}

\begin{axis}[%
width=4.521in,
height=3.541in,
at={(0.758in,0.506in)},
scale only axis,
xmin=0,
xmax=60,
xlabel style={font=\color{white!15!black}},
xlabel={Angle (degrees)},
ymin=-30,
ymax=0,
ylabel style={font=\color{white!15!black}},
ylabel={Magnitude (dB)},
axis background/.style={fill=white},
title style={font=\bfseries},
%title={$\text{RIS Config 1: Beam Steering at 0}^\circ$},
xmajorgrids,
ymajorgrids,
legend style={legend cell align=left, align=left, draw=white!15!black}
]
\addplot [color=mycolor1, line width=1.2pt, mark=triangle, mark options={solid, mycolor1}]
  table[row sep=crcr]{%
0	-11.5129348688376\\
1	-13.7050842292267\\
2	-13.5756354850128\\
3	-12.0408566670578\\
4	-11.6339438730106\\
5	-12.9775644874467\\
6	-15.8663377960082\\
7	-17.5955834158791\\
8	-16.5117596140838\\
9	-16.700816214909\\
10	-19.7169500428193\\
11	-19.9624994758085\\
12	-15.0022435367885\\
13	-12.8091258016531\\
14	-13.9418852079273\\
15	-18.9262497449786\\
16	-12.8332786998903\\
17	-6.22732797303474\\
18	-2.55914261947748\\
19	-0.631228758202518\\
20	0\\
21	-0.525657606295546\\
22	-2.25564895609199\\
23	-5.46797142321581\\
24	-10.9360780541338\\
25	-19.1250610027069\\
26	-16.3453438898502\\
27	-13.6272204222029\\
28	-13.9371144140041\\
29	-16.4973481799196\\
30	-20.4720019840877\\
31	-21.5468063933925\\
32	-19.9929298640421\\
33	-19.5337341959903\\
34	-19.5038764608812\\
35	-18.2283863144249\\
36	-16.3542556973449\\
37	-15.2892342440581\\
38	-15.5880196112846\\
39	-17.6844486876801\\
40	-22.4377703441346\\
41	-23.915606287981\\
42	-17.5474642742416\\
43	-13.6685476321782\\
44	-11.5351828584377\\
45	-10.5240429230645\\
46	-10.356031774147\\
47	-10.9004805112211\\
48	-12.1044597909139\\
49	-13.9661506202799\\
50	-16.5146171995214\\
51	-19.7235599975998\\
52	-23.0531427235759\\
53	-24.6091379146533\\
54	-23.8241481214414\\
55	-22.6550774647166\\
56	-21.9317621160654\\
57	-21.6919247828968\\
58	-21.8505122500997\\
59	-22.3426837826813\\
60	-23.1411392308224\\
};
\addlegendentry{Case 1}

\addplot [color=mycolor2, line width=1.2pt, mark=square, mark options={solid, mycolor2}]
  table[row sep=crcr]{%
0	0\\
1	-3.61718194209512\\
2	-6.30775855399673\\
3	-4.08431714739389\\
4	-2.26004814377226\\
5	-2.48411233998906\\
6	-5.07308659816087\\
7	-11.7625485225134\\
8	-24.1229055218373\\
9	-11.4014759081966\\
10	-9.51309529735951\\
11	-12.1704003573212\\
12	-24.8123298476353\\
13	-15.2169935316516\\
14	-9.48234819159022\\
15	-8.29491530945658\\
16	-10.2309542605905\\
17	-17.1083156827063\\
18	-22.7812368936563\\
19	-12.9809103045822\\
20	-10.8830196066196\\
21	-12.6102529969854\\
22	-20.6037618849795\\
23	-19.7114508347825\\
24	-11.571741756848\\
25	-9.07312559765606\\
26	-9.42684574108975\\
27	-12.8417004099238\\
28	-24.1721062885048\\
29	-18.1976350861399\\
30	-12.0537489407605\\
31	-10.7833599241706\\
32	-12.9315977842362\\
33	-22.3882558768902\\
34	-17.0106993983827\\
35	-9.06911597886639\\
36	-5.66096040007726\\
37	-4.28070781418486\\
38	-4.35414709706529\\
39	-5.71057255953222\\
40	-7.92163657409978\\
41	-8.54906052083734\\
42	-6.06111590014163\\
43	-3.41092665690334\\
44	-1.65617049529254\\
45	-0.747105026630287\\
46	-0.544171426263375\\
47	-0.94532573983777\\
48	-1.87798123808841\\
49	-3.27255832035343\\
50	-5.01723185988101\\
51	-6.88115115312706\\
52	-8.46413940915802\\
53	-9.40304457553604\\
54	-9.76429985396491\\
55	-9.93606894308226\\
56	-10.2299228740659\\
57	-10.7829557584668\\
58	-11.6290868061431\\
59	-12.7578161812497\\
60	-14.1392620024048\\
};
\addlegendentry{Case 2}

\addplot [color=mycolor3, dashed, line width=1.5pt, mark size=4.0pt, mark=o, mark options={solid, mycolor3}]
  table[row sep=crcr]{%
0	0\\
5	-9.83385\\
10	-28.10913\\
15	-18.83296\\
20	-20.11346\\
25	-13.50132\\
30	-6.51126\\
35	-6.59193999999999\\
40	-6.69179\\
45	-3.09581\\
50	-9.64471999999999\\
55	-8.58743\\
60	-20.44139\\
};
\addlegendentry{Real Measurements}

\end{axis}

\begin{axis}[%
width=5.833in,
height=4.375in,
at={(0in,0in)},
scale only axis,
xmin=0,
xmax=1,
ymin=0,
ymax=1,
axis line style={draw=none},
ticks=none,
axis x line*=bottom,
axis y line*=left
]
\end{axis}
\end{tikzpicture}%}
        \subcaption{RIS Configuration, $0^\circ$}
        \label{fig:ris_val0}
    \end{minipage}
    \begin{minipage}{0.185\textwidth}
        \centering
        \scalebox{0.24}{% This file was created by matlab2tikz.
%
%The latest updates can be retrieved from
%  http://www.mathworks.com/matlabcentral/fileexchange/22022-matlab2tikz-matlab2tikz
%where you can also make suggestions and rate matlab2tikz.
%
\definecolor{mycolor1}{rgb}{0.00000,0.44700,0.74100}%
\definecolor{mycolor2}{rgb}{0.85000,0.32500,0.09800}%
\definecolor{mycolor3}{rgb}{0.92900,0.69400,0.12500}%
%
\begin{tikzpicture}

\begin{axis}[%
width=4.521in,
height=3.541in,
at={(0.758in,0.506in)},
scale only axis,
xmin=0,
xmax=60,
xlabel style={font=\color{white!15!black}},
xlabel={Angle (degrees)},
ymin=-45,
ymax=0,
ylabel style={font=\color{white!15!black}},
ylabel={Magnitude (dB)},
axis background/.style={fill=white},
title style={font=\bfseries},
%title={$\text{RIS Config 4: Beam Steering at 15}^\circ$},
xmajorgrids,
ymajorgrids,
legend style={legend cell align=left, align=left, draw=white!15!black}
]
\addplot [color=mycolor1, line width=1.2pt, mark=triangle, mark options={solid, mycolor1}]
  table[row sep=crcr]{%
0	-14.6786445593638\\
1	-17.0800408733408\\
2	-14.8221117841331\\
3	-13.5056047891839\\
4	-15.7856436102923\\
5	-29.5169701153514\\
6	-15.9040707197746\\
7	-9.40503594088598\\
8	-6.93284718905295\\
9	-6.77561145313041\\
10	-8.35403798275549\\
11	-9.40938717767622\\
12	-7.01557171159772\\
13	-4.50856116296803\\
14	-3.49113732709947\\
15	-3.83236348857269\\
16	-4.59908566402909\\
17	-3.87428667854773\\
18	-1.87680686276372\\
19	-0.372671046466081\\
20	0\\
21	-0.892699298254037\\
22	-3.09602062734172\\
23	-6.01945913050772\\
24	-6.64963169336834\\
25	-4.82937441625639\\
26	-3.67830114366071\\
27	-3.90136616491346\\
28	-5.49019095298853\\
29	-8.15968422143313\\
30	-10.1649551842161\\
31	-9.47124381848423\\
32	-8.35807711694064\\
33	-8.44069093363919\\
34	-10.0983986446108\\
35	-13.9183630270922\\
36	-22.690098202867\\
37	-26.8884590976916\\
38	-17.8890155948305\\
39	-15.5750266534219\\
40	-15.7167231911708\\
41	-17.4382163622038\\
42	-19.3545874196389\\
43	-18.6468032239888\\
44	-16.4785826708205\\
45	-15.0035579496104\\
46	-14.5455228123771\\
47	-15.0332126212438\\
48	-16.4219941456527\\
49	-18.7575172475617\\
50	-22.2283723048995\\
51	-27.337995423458\\
52	-35.5774703149042\\
53	-42.2588259608004\\
54	-36.3544858784211\\
55	-34.0019158308331\\
56	-32.3725032130566\\
57	-30.4442894575694\\
58	-28.5674418649826\\
59	-27.1597522715976\\
60	-26.3365627168834\\
};
\addlegendentry{Case 1}

\addplot [color=mycolor2, line width=1.2pt, mark=square, mark options={solid, mycolor2}]
  table[row sep=crcr]{%
0	-11.6192839004329\\
1	-13.296753802071\\
2	-12.358943880634\\
3	-11.9407319361271\\
4	-14.8366253171932\\
5	-31.4335197884916\\
6	-13.4867683306669\\
7	-7.35332925400504\\
8	-4.81062451847774\\
9	-4.25584248268624\\
10	-4.87051615838661\\
11	-4.92688111611302\\
12	-3.11083845103845\\
13	-1.07327373125371\\
14	0\\
15	-0.0795775819830951\\
16	-1.20274085762289\\
17	-2.87425948114361\\
18	-3.87692904398538\\
19	-3.6699421474094\\
20	-3.39697984751712\\
21	-3.72908721758092\\
22	-4.1337299181391\\
23	-3.47326511809111\\
24	-2.00041443853403\\
25	-0.864923095827358\\
26	-0.597350407232916\\
27	-1.31098076855157\\
28	-2.94289087447152\\
29	-5.03319920149285\\
30	-6.27090317931094\\
31	-6.02489368202804\\
32	-5.64862211995583\\
33	-6.17053697587328\\
34	-8.02761031739265\\
35	-11.8196325140833\\
36	-19.8323641158631\\
37	-27.3346272917987\\
38	-17.1076569192978\\
39	-14.4727395003843\\
40	-14.2614450000246\\
41	-15.2375839462344\\
42	-15.8975274502406\\
43	-14.9043805370014\\
44	-13.262392466174\\
45	-12.1202409338504\\
46	-11.7511724347751\\
47	-12.1476408691622\\
48	-13.2757960159723\\
49	-15.1279967574309\\
50	-17.7217750986647\\
51	-21.0384713716032\\
52	-24.7078840076072\\
53	-27.2621392903744\\
54	-27.5734737197778\\
55	-27.0632556535158\\
56	-26.6904425836637\\
57	-26.4637470033469\\
58	-26.2336846752061\\
59	-25.974297356617\\
60	-25.7641067915393\\
};
\addlegendentry{Case 2}

\addplot [color=mycolor3, dashed, line width=1.5pt, mark size=4.0pt, mark=o, mark options={solid, mycolor3}]
  table[row sep=crcr]{%
0	-14.0981\\
5	-16.30763\\
10	-13.7896\\
15	0\\
20	-6.97848\\
25	-5.51293\\
30	-5.40864999999999\\
35	-18.30925\\
40	-27.21271\\
45	-9.65184\\
50	-21.56723\\
55	-22.70132\\
60	-16.94833\\
};
\addlegendentry{Real Measurements}

\end{axis}

\begin{axis}[%
width=5.833in,
height=4.375in,
at={(0in,0in)},
scale only axis,
xmin=0,
xmax=1,
ymin=0,
ymax=1,
axis line style={draw=none},
ticks=none,
axis x line*=bottom,
axis y line*=left
]
\end{axis}
\end{tikzpicture}%}
        \subcaption{RIS Configuration, $15^\circ$}
        \label{fig:ris_val15}
    \end{minipage}
    \begin{minipage}{0.185\textwidth}
        \centering
        \scalebox{0.24}{% This file was created by matlab2tikz.
%
%The latest updates can be retrieved from
%  http://www.mathworks.com/matlabcentral/fileexchange/22022-matlab2tikz-matlab2tikz
%where you can also make suggestions and rate matlab2tikz.
%
\definecolor{mycolor1}{rgb}{0.00000,0.44700,0.74100}%
\definecolor{mycolor2}{rgb}{0.85000,0.32500,0.09800}%
\definecolor{mycolor3}{rgb}{0.92900,0.69400,0.12500}%
%
\begin{tikzpicture}

\begin{axis}[%
width=4.521in,
height=3.541in,
at={(0.758in,0.506in)},
scale only axis,
xmin=0,
xmax=60,
xlabel style={font=\color{white!15!black}},
xlabel={Angle (degrees)},
ymin=-50,
ymax=0,
ylabel style={font=\color{white!15!black}},
ylabel={Magnitude (dB)},
axis background/.style={fill=white},
title style={font=\bfseries},
%title={$\text{RIS Config 7: Beam Steering at 30}^\circ$},
xmajorgrids,
ymajorgrids,
legend style={legend cell align=left, align=left, draw=white!15!black}
]
\addplot [color=mycolor1, line width=1.2pt, mark=triangle, mark options={solid, mycolor1}]
  table[row sep=crcr]{%
0	-17.7940355684854\\
1	-24.8654847636706\\
2	-27.0415588051027\\
3	-18.3690621348448\\
4	-16.2425604075084\\
5	-17.8110183609841\\
6	-24.6902647585152\\
7	-22.1241577003038\\
8	-15.240427890282\\
9	-12.4217459165332\\
10	-11.4217475083697\\
11	-11.022218120302\\
12	-10.5092189880369\\
13	-10.3589333142213\\
14	-11.8526018919031\\
15	-17.5642403397297\\
16	-17.4388124538197\\
17	-8.11068712431223\\
18	-3.52708031953844\\
19	-1.06257227213744\\
20	0\\
21	-0.110337394548267\\
22	-1.38931498464537\\
23	-4.0588189160162\\
24	-8.81571178117845\\
25	-17.3411064361609\\
26	-15.7203634950053\\
27	-11.1617988296857\\
28	-9.82885069937697\\
29	-10.2460622347398\\
30	-11.6529094704845\\
31	-13.1205657730966\\
32	-13.7757949981142\\
33	-14.006807910519\\
34	-14.789347235054\\
35	-16.7499169788315\\
36	-20.4362027334653\\
37	-26.2059852441214\\
38	-26.7460532691807\\
39	-22.9739971871815\\
40	-21.364130833956\\
41	-21.5097884001658\\
42	-23.122489367314\\
43	-26.3299137274895\\
44	-32.0369880577097\\
45	-46.6070941634486\\
46	-38.2416980512857\\
47	-32.4368496167291\\
48	-29.7052958865212\\
49	-27.8931035308398\\
50	-26.5275663736342\\
51	-25.5678802310666\\
52	-25.0580858406368\\
53	-25.0240634879658\\
54	-25.467094962961\\
55	-26.3757589133858\\
56	-27.7328903361272\\
57	-29.5143073800928\\
58	-31.6776310771356\\
59	-34.1362829794769\\
60	-36.7140919648661\\
};
\addlegendentry{Case 1}

\addplot [color=mycolor2, line width=1.2pt, mark=square, mark options={solid, mycolor2}]
  table[row sep=crcr]{%
0	-9.73516826797754\\
1	-11.4236082251063\\
2	-15.3757750542248\\
3	-14.9036208680596\\
4	-10.8968557130599\\
5	-9.11128485301003\\
6	-8.87957394654747\\
7	-8.01170149262224\\
8	-5.20894224840857\\
9	-2.40160240808873\\
10	-0.631637342016823\\
11	0\\
12	-0.47105172111776\\
13	-2.0272805732443\\
14	-4.55512480151047\\
15	-7.26708127045966\\
16	-8.43083212972437\\
17	-8.48202901634885\\
18	-9.30326265028334\\
19	-11.5034724720747\\
20	-13.4896003945048\\
21	-11.9331521139202\\
22	-9.753356503448\\
23	-8.87917566843652\\
24	-8.92323971109674\\
25	-8.35697763814883\\
26	-6.20499387496982\\
27	-3.77391484854353\\
28	-2.04196832071218\\
29	-1.19794002050613\\
30	-1.23073863868959\\
31	-2.12255311225512\\
32	-3.86447176692133\\
33	-6.35251219651573\\
34	-9.00020930197668\\
35	-10.4946711456623\\
36	-10.7606987382784\\
37	-11.270884068418\\
38	-12.9354624205865\\
39	-16.0474719354224\\
40	-18.7313303268885\\
41	-16.6208634041064\\
42	-13.9170845511137\\
43	-12.6329132496385\\
44	-12.6799774284245\\
45	-13.9700980886243\\
46	-16.5111492086205\\
47	-19.6395549912566\\
48	-19.7752853511462\\
49	-17.158292141968\\
50	-15.13586365843\\
51	-14.1512814805332\\
52	-14.0375709783467\\
53	-14.637415325167\\
54	-15.7961695458648\\
55	-17.2455089743144\\
56	-18.4573115858446\\
57	-18.8503275942779\\
58	-18.4844862187641\\
59	-17.9326286839047\\
60	-17.6054051301608\\
};
\addlegendentry{Case 2}

\addplot [color=mycolor3, dashed, line width=1.5pt, mark size=4.0pt, mark=o, mark options={solid, mycolor3}]
  table[row sep=crcr]{%
0	-10.48016\\
5	-8.56865999999999\\
10	-7.24165\\
15	-8.69526\\
20	-14.31\\
25	-23.88117\\
30	0\\
35	-22.79826\\
40	-24.16076\\
45	-15.24653\\
50	-23.59478\\
55	-19.09318\\
60	-9.30128999999999\\
};
\addlegendentry{Real Measurements}

\end{axis}

\begin{axis}[%
width=5.833in,
height=4.375in,
at={(0in,0in)},
scale only axis,
xmin=0,
xmax=1,
ymin=0,
ymax=1,
axis line style={draw=none},
ticks=none,
axis x line*=bottom,
axis y line*=left
]
\end{axis}
\end{tikzpicture}%}
        \subcaption{RIS Configuration, $30^\circ$}
        \label{fig:ris_val30}
    \end{minipage}
    \begin{minipage}{0.185\textwidth}
        \centering
        \scalebox{0.24}{% This file was created by matlab2tikz.
%
%The latest updates can be retrieved from
%  http://www.mathworks.com/matlabcentral/fileexchange/22022-matlab2tikz-matlab2tikz
%where you can also make suggestions and rate matlab2tikz.
%
\definecolor{mycolor1}{rgb}{0.00000,0.44700,0.74100}%
\definecolor{mycolor2}{rgb}{0.85000,0.32500,0.09800}%
\definecolor{mycolor3}{rgb}{0.92900,0.69400,0.12500}%
%
\begin{tikzpicture}

\begin{axis}[%
width=4.521in,
height=3.541in,
at={(0.758in,0.506in)},
scale only axis,
xmin=0,
xmax=60,
xlabel style={font=\color{white!15!black}},
xlabel={Angle (degrees)},
ymin=-50,
ymax=0,
ylabel style={font=\color{white!15!black}},
ylabel={Magnitude (dB)},
axis background/.style={fill=white},
title style={font=\bfseries},
%title={$\text{RIS Config 10: Beam Steering at 45}^\circ$},
xmajorgrids,
ymajorgrids,
legend style={legend cell align=left, align=left, draw=white!15!black}
]
\addplot [color=mycolor1, line width=1.2pt, mark=triangle, mark options={solid, mycolor1}]
  table[row sep=crcr]{%
0	-7.38627530305651\\
1	-10.7227815011013\\
2	-17.5019522148272\\
3	-29.02114441177\\
4	-21.0000911566035\\
5	-20.5380614191636\\
6	-24.7330295001669\\
7	-23.5145081882792\\
8	-19.5763086532215\\
9	-19.7654673357314\\
10	-24.898521549952\\
11	-22.8626534073272\\
12	-16.769823726512\\
13	-15.4037443007262\\
14	-19.4816900723444\\
15	-24.7943589805062\\
16	-10.578216071032\\
17	-4.95706714953497\\
18	-1.87802127536114\\
19	-0.333420097852631\\
20	0\\
21	-0.80718096875006\\
22	-2.8826374529654\\
23	-6.70752725580306\\
24	-14.1120472686386\\
25	-29.5526834010377\\
26	-14.6644971281699\\
27	-11.9359125098044\\
28	-12.3085743043788\\
29	-15.1319317797964\\
30	-21.5833990345836\\
31	-27.1922508609845\\
32	-19.7663689800197\\
33	-17.109614122701\\
34	-16.9923768026088\\
35	-18.8354433861367\\
36	-22.658322943291\\
37	-24.8619215188787\\
38	-20.5374310500864\\
39	-16.8699336588488\\
40	-14.5499015078944\\
41	-13.0942177657324\\
42	-12.2240437466192\\
43	-11.7981901089804\\
44	-11.7501612652247\\
45	-12.0543013201944\\
46	-12.7076044978569\\
47	-13.7202511487152\\
48	-15.109797069213\\
49	-16.8920989362379\\
50	-19.0532484625294\\
51	-21.4633615610408\\
52	-23.6955858677253\\
53	-25.0489026536013\\
54	-25.3706705464504\\
55	-25.3202857442556\\
56	-25.4637721261425\\
57	-25.997036979652\\
58	-26.8920729667458\\
59	-27.9458579763688\\
60	-28.7716854387964\\
};
\addlegendentry{Case 1}

\addplot [color=mycolor2, line width=1.2pt, mark=square, mark options={solid, mycolor2}]
  table[row sep=crcr]{%
0	0\\
1	-2.02789022329744\\
2	-5.60404570651727\\
3	-11.1056885409837\\
4	-15.6396008541408\\
5	-14.174954597591\\
6	-13.8966124958242\\
7	-16.3463646120122\\
8	-21.7262979882597\\
9	-22.7573878840743\\
10	-18.7405447062638\\
11	-17.0562946419164\\
12	-16.7011369200612\\
13	-16.3545900674608\\
14	-15.4439095819127\\
15	-14.564199890685\\
16	-14.3786969432511\\
17	-15.2564732646953\\
18	-17.6331390887217\\
19	-22.82936025942\\
20	-48.1875267876959\\
21	-23.9273326423491\\
22	-18.3492841788161\\
23	-15.8776414290346\\
24	-14.9278074706159\\
25	-15.0105445147241\\
26	-15.8070894060117\\
27	-16.8471523559926\\
28	-17.5259534211015\\
29	-17.8183483837649\\
30	-18.508251514734\\
31	-20.51457462968\\
32	-24.2753859897679\\
33	-23.9474642168593\\
34	-19.1965419999276\\
35	-16.3938779825637\\
36	-15.5656308201291\\
37	-16.4732831411678\\
38	-17.768201931446\\
39	-14.8689333954203\\
40	-10.3055748485942\\
41	-6.86358094649352\\
42	-4.5303333025673\\
43	-3.07294043952391\\
44	-2.3424689592481\\
45	-2.25944886836809\\
46	-2.79441282708032\\
47	-3.96144930867318\\
48	-5.82313810190711\\
49	-8.50229560060824\\
50	-12.145506539294\\
51	-16.3048777532599\\
52	-17.7273549694122\\
53	-16.0741461238658\\
54	-14.8425833898965\\
55	-14.7024316090321\\
56	-15.6136299626227\\
57	-17.6190241277302\\
58	-20.9426136273968\\
59	-25.4732790385581\\
60	-26.2049244949485\\
};
\addlegendentry{Case 2}

\addplot [color=mycolor3, dashed, line width=1.5pt, mark size=4.0pt, mark=o, mark options={solid, mycolor3}]
  table[row sep=crcr]{%
0	-5.3293\\
5	-6.95337\\
10	-11.39936\\
15	-14.53505\\
20	-24.18032\\
25	-12.80109\\
30	-21.61014\\
35	-16.92446\\
40	-4.4235\\
45	0\\
50	-9.73663\\
55	-12.59637\\
60	-46.17454\\
};
\addlegendentry{Real Measurements}

\end{axis}

\begin{axis}[%
width=5.833in,
height=4.375in,
at={(0in,0in)},
scale only axis,
xmin=0,
xmax=1,
ymin=0,
ymax=1,
axis line style={draw=none},
ticks=none,
axis x line*=bottom,
axis y line*=left
]
\end{axis}
\end{tikzpicture}%}
        \subcaption{RIS Configuration, $45^\circ$}
        \label{fig:ris_val45}
    \end{minipage}
    \begin{minipage}{0.185\textwidth}
        \centering
        \scalebox{0.24}{% This file was created by matlab2tikz.
%
%The latest updates can be retrieved from
%  http://www.mathworks.com/matlabcentral/fileexchange/22022-matlab2tikz-matlab2tikz
%where you can also make suggestions and rate matlab2tikz.
%
\definecolor{mycolor1}{rgb}{0.00000,0.44700,0.74100}%
\definecolor{mycolor2}{rgb}{0.85000,0.32500,0.09800}%
\definecolor{mycolor3}{rgb}{0.92900,0.69400,0.12500}%
%
\begin{tikzpicture}

\begin{axis}[%
width=4.521in,
height=3.541in,
at={(0.758in,0.506in)},
scale only axis,
xmin=0,
xmax=60,
xlabel style={font=\color{white!15!black}},
xlabel={Angle (degrees)},
ymin=-70,
ymax=0,
ylabel style={font=\color{white!15!black}},
ylabel={Magnitude (dB)},
axis background/.style={fill=white},
title style={font=\bfseries},
%title={$\text{RIS Config 13: Beam Steering at 60}^\circ$},
xmajorgrids,
ymajorgrids,
legend style={at={(0.664,0.325)}, anchor=south west, legend cell align=left, align=left, draw=white!15!black}
]
\addplot [color=mycolor1, line width=1.2pt, mark=triangle, mark options={solid, mycolor1}]
  table[row sep=crcr]{%
0	-25.5203457592778\\
1	-23.183618367366\\
2	-20.7668994643841\\
3	-19.7806055700259\\
4	-20.9011543325049\\
5	-24.6387292563742\\
6	-25.4425040343947\\
7	-21.0621120395934\\
8	-19.4758675266653\\
9	-20.5166277919642\\
10	-19.6795497065858\\
11	-15.0207208735997\\
12	-11.949336531828\\
13	-11.2984708391168\\
14	-13.9159198401671\\
15	-29.8442511031028\\
16	-13.2085336148329\\
17	-5.99534716625482\\
18	-2.38528097257403\\
19	-0.551505869010235\\
20	0\\
21	-0.587397892058291\\
22	-2.36701234591677\\
23	-5.64325392325986\\
24	-11.4082667841874\\
25	-25.9269882293028\\
26	-19.9987379990393\\
27	-15.4570995152843\\
28	-15.6085289993516\\
29	-18.3295500039001\\
30	-20.864947074112\\
31	-18.7435430936635\\
32	-16.6914930762859\\
33	-16.4889992313291\\
34	-18.1639496724935\\
35	-22.0190405621759\\
36	-27.5621675469527\\
37	-26.0182125347212\\
38	-23.0319649886917\\
39	-22.4598688953885\\
40	-23.8755853501399\\
41	-26.6499893459564\\
42	-27.5422335052737\\
43	-25.1792222045868\\
44	-23.3574111597398\\
45	-22.96350055915\\
46	-23.9995626718365\\
47	-26.5018055938734\\
48	-29.2585218133303\\
49	-27.1234849225123\\
50	-23.0077203437503\\
51	-19.850420693136\\
52	-17.5349345227492\\
53	-15.7963188938229\\
54	-14.4835147436439\\
55	-13.5215939548102\\
56	-12.8760624776877\\
57	-12.532025095722\\
58	-12.4838201634369\\
59	-12.7309359741697\\
60	-13.2773527213823\\
};
\addlegendentry{Case 1}

\addplot [color=mycolor2, line width=1.2pt, mark=square, mark options={solid, mycolor2}]
  table[row sep=crcr]{%
0	-12.3668794253823\\
1	-12.1180947834066\\
2	-16.0060402779874\\
3	-35.0024402362127\\
4	-17.8825089110535\\
5	-14.0135959999599\\
6	-15.0871876462829\\
7	-19.6083797919764\\
8	-14.1985513232049\\
9	-8.98972815588516\\
10	-6.67887911461571\\
11	-6.35517575534485\\
12	-7.76477020954745\\
13	-11.0761231471171\\
14	-16.9061666223806\\
15	-23.2377333549026\\
16	-21.5019133291553\\
17	-21.5719754230241\\
18	-25.2993632398403\\
19	-33.8116849609858\\
20	-62.6081082135945\\
21	-35.4123726799376\\
22	-26.350864618765\\
23	-22.2794107155319\\
24	-21.6805704462616\\
25	-23.7922781375901\\
26	-20.1171601626603\\
27	-13.9181583894865\\
28	-10.0787696669255\\
29	-8.03693554549653\\
30	-7.43147386536054\\
31	-8.18702544604027\\
32	-10.4862374988419\\
33	-14.8923818710259\\
34	-20.9556607917509\\
35	-18.7659202479479\\
36	-16.0544313707786\\
37	-16.2745564038064\\
38	-19.7991362088531\\
39	-33.1861787809855\\
40	-22.9388044653328\\
41	-16.7899892751314\\
42	-14.6697127377439\\
43	-14.7357177987329\\
44	-16.8404191929102\\
45	-21.7355561590666\\
46	-25.9153486451269\\
47	-19.840220488153\\
48	-16.2038444534699\\
49	-14.2831949508353\\
50	-12.7221769675957\\
51	-10.6177057606074\\
52	-8.05879750660637\\
53	-5.60975080754059\\
54	-3.57852737964693\\
55	-2.03099454406419\\
56	-0.944996240948548\\
57	-0.280224467351651\\
58	0\\
59	-0.076867884641274\\
60	-0.49379692649061\\
};
\addlegendentry{Case 2}

\addplot [color=mycolor3, dashed, line width=1.5pt, mark size=4.0pt, mark=o, mark options={solid, mycolor3}]
  table[row sep=crcr]{%
0	-11.531\\
5	-16.1673\\
10	-16.41876\\
15	-20.17275\\
20	-23.38745\\
25	-13.47651\\
30	-16.29332\\
35	-19.83092\\
40	-11.36462\\
45	-18.27192\\
50	-18.66712\\
55	-3.5953\\
60	0\\
};
\addlegendentry{Real Measurements}

\end{axis}

\begin{axis}[%
width=5.833in,
height=4.375in,
at={(0in,0in)},
scale only axis,
xmin=0,
xmax=1,
ymin=0,
ymax=1,
axis line style={draw=none},
ticks=none,
axis x line*=bottom,
axis y line*=left
]
\end{axis}
\end{tikzpicture}%}
        \subcaption{RIS Configuration, $60^\circ$}
        \label{fig:ris_val60}
    \end{minipage}
    \caption{Validation of RIS radiation patterns measured in the outdoor setup (Fig.~\ref{fig:2a}) at Rx positions ranging from $0^\circ$ to $60^\circ$, with a step size of $5^\circ$ (3.55 GHz). Case 1 corresponds to the theoretical radiation pattern model~\cite{ramezani2024broad,10256051} derived using the RIS configurations generated by Algorithm 1, while Case 2 represents the theoretical model by modifying the last rows of the RIS configuration matrix, specifically extending the states of the corresponding upper rows within the same columns. The gains are normalized, with the maximum value set to $0\,\mathrm{dB}$. It is noted that the theoretical patterns for Case 1 and Case 2 are plotted with a resolution of $1^\circ$ while the measured patterns are taken at $5^\circ$ intervals. The maximum gain across all Rx positions is observed at the location where beam optimization via Algorithm 1 was performed.}
    \label{fig:validation}
\end{figure*}



\subsection{Impact of Frequency Selectivity on RIS Beamforming} 
\begin{figure}[t]
\centering
\begin{minipage}{0.24\textwidth}
    \centering
    \scalebox{0.29}{% This file was created by matlab2tikz.
%
%The latest updates can be retrieved from
%  http://www.mathworks.com/matlabcentral/fileexchange/22022-matlab2tikz-matlab2tikz
%where you can also make suggestions and rate matlab2tikz.
%
\definecolor{mycolor1}{rgb}{0.00000,0.44700,0.74100}%
\definecolor{mycolor2}{rgb}{0.85000,0.32500,0.09800}%
\definecolor{mycolor3}{rgb}{0.92900,0.69400,0.12500}%
%
\begin{tikzpicture}

\begin{axis}[%
width=4.844in,
height=3.396in,
at={(0.812in,0.458in)},
scale only axis,
xmin=0,
xmax=60,
xlabel style={font=\color{white!15!black}},
xlabel={Angle (degrees)},
ymin=-90,
ymax=-50,
ylabel style={font=\color{white!15!black}},
ylabel={Magnitude (dB)},
axis background/.style={fill=white},
title style={font=\bfseries},
%title={RIS Configuration 1},
axis x line*=bottom,
axis y line*=left,
xmajorgrids,
ymajorgrids,
legend style={legend cell align=left, align=left, draw=white!15!black}
]
\addplot [color=mycolor1, line width=1.5pt, mark=o, mark options={solid, mycolor1}]
  table[row sep=crcr]{%
0	-60.12\\
5	-72.06488\\
10	-72.0265\\
15	-76.54573\\
20	-73.92595\\
25	-72.11463\\
30	-70.84872\\
35	-71.36137\\
40	-66.66101\\
45	-67.20325\\
50	-66.59303\\
55	-71.88374\\
60	-74.61409\\
};
\addlegendentry{3.50 GHz}

\addplot [color=mycolor2, dashed, line width=1.5pt, mark size=4.0pt, mark=asterisk, mark options={solid, mycolor2}]
  table[row sep=crcr]{%
0	-56.87895\\
5	-70.73302\\
10	-68.75427\\
15	-69.49359\\
20	-64.73093\\
25	-72.60362\\
30	-66.88623\\
35	-64.37363\\
40	-64.31977\\
45	-63.09956\\
50	-63.89492\\
55	-68.32399\\
60	-70.94555\\
};
\addlegendentry{3.55 GHz}

\addplot [color=mycolor3, dotted, line width=1.5pt, mark size=6.0pt, mark=diamond, mark options={solid, mycolor3}]
  table[row sep=crcr]
    \subcaption{RIS Configuration at $0^\circ$}\label{fig:freqsel0}
\end{minipage}%
\hfill
\begin{minipage}{0.24\textwidth}
    \centering
    \scalebox{0.29}{% This file was created by matlab2tikz.
%
%The latest updates can be retrieved from
%  http://www.mathworks.com/matlabcentral/fileexchange/22022-matlab2tikz-matlab2tikz
%where you can also make suggestions and rate matlab2tikz.
%
\definecolor{mycolor1}{rgb}{0.00000,0.44700,0.74100}%
\definecolor{mycolor2}{rgb}{0.85000,0.32500,0.09800}%
\definecolor{mycolor3}{rgb}{0.92900,0.69400,0.12500}%
%
\begin{tikzpicture}

\begin{axis}[%
width=4.844in,
height=3.396in,
at={(0.812in,0.458in)},
scale only axis,
xmin=0,
xmax=60,
xlabel style={font=\color{white!15!black}},
xlabel={Angle (degrees)},
ymin=-90,
ymax=-50,
ylabel style={font=\color{white!15!black}},
ylabel={Magnitude (dB)},
axis background/.style={fill=white},
title style={font=\bfseries},
%title={RIS Configuration 7},
axis x line*=bottom,
axis y line*=left,
xmajorgrids,
ymajorgrids,
legend style={legend cell align=left, align=left, draw=white!15!black}
]
\addplot [color=mycolor1, line width=1.5pt, mark=o, mark options={solid, mycolor1}]
  table[row sep=crcr]{%
0	-65.02938\\
5	-67.92269\\
10	-65.64189\\
15	-66.28907\\
20	-68.47385\\
25	-77.32084\\
30	-60.38505\\
35	-72.44364\\
40	-75.05635\\
45	-85.20625\\
50	-81.35133\\
55	-71.66863\\
60	-76.00131\\
};
\addlegendentry{3.50 GHz}

\addplot [color=mycolor2, dashed, line width=1.5pt, mark size=4.0pt, mark=asterisk, mark options={solid, mycolor2}]
  table[row sep=crcr]{%
0	-63.39021\\
5	-67.2532\\
10	-62.14983\\
15	-60.80414\\
20	-75.7338\\
25	-73.21961\\
30	-56.40607\\
35	-74.7845\\
40	-78.35597\\
45	-79.3804\\
50	-72.04745\\
55	-69.43168\\
60	-75.70787\\
};
\addlegendentry{3.55 GHz}

\addplot [color=mycolor3, dotted, line width=1.5pt, mark size=6.0pt, mark=diamond, mark options={solid, mycolor3}]
  table[row sep=crcr]
    \subcaption{RIS Configuration at $30^\circ$}\label{fig:freq_sel30}
\end{minipage}%
\caption{Comparison of measured radiation patterns across frequencies. The results demonstrate that the angle direction of the main lobe is preserved across the frequency range, showcasing the effectiveness of the proposed codebook for beam sweeping in wideband applications. Variations are observed in the side lobes. Similar results in all Rx positions.}
\label{fig:freq_selectivity}
\end{figure}

In this subsection, the proposed codebook is further examined in terms of frequency selectivity. Fig.~\ref{fig:freq_selectivity} analyzes the radiation patterns, extending the analysis to multiple frequencies. Specifically, we plot the patterns at 3.5 GHz, 3.55 GHz, and 3.6 GHz. While the angle direction of the main lobe is consistently maintained, highlighting the suitability of the codebook for beam sweeping and for wideband applications that require bandwidths of several MHz; we observe variations in the side lobes.  

%\subsection{RIS Impact on Received Power}

\begin{comment}
    

\subsection{Channel Reciprocity}
In this experiment, we investigate the channel reciprocity in a RIS-aided communication system. Channel reciprocity is a key feature in time division duplex (TDD) systems, where uplink (UL) and downlink (DL) transmissions share the same frequency band but occur in alternating time slots. This mechanism leverages that the channel properties remain consistent in both directions, thus significantly reducing the signaling overhead in systems with multiple antennas, as channel estimation can be performed in one direction and applied in the other. For RIS-aided communications, channel reciprocity ensures that the RIS configuration optimized for beamforming in the UL remains effective in the DL without requiring reconfiguration. To verify UL/DL reciprocity, we conducted experiments using a VNA to measure the S-parameters in a realistic indoor environment (Fig.~\ref{fig:2b}). The RIS configuration remains fixed during the measurements. The results, shown in Fig.~\ref{fig:channel_recipr}, compare the amplitude (in dB) and phase (in degrees) of S21 and S12 across the frequency range. The close agreement between S21 and S12 confirms the reciprocity of the UL and DL channels, despite the presence of multipath effects in the indoor environment. Similar results are observed across all the tests conducted in this work, even when the directional antennas at the receiver are replaced with omnidirectional ones to emulate more realistic conditions. This finding demonstrates that RIS can reliably support TDD systems without requiring separate configurations for UL and DL.
\begin{figure}[t]
\centering
\begin{minipage}{0.24\textwidth}
    \centering
    \scalebox{0.29}{% This file was created by matlab2tikz.
%
%The latest updates can be retrieved from
%  http://www.mathworks.com/matlabcentral/fileexchange/22022-matlab2tikz-matlab2tikz
%where you can also make suggestions and rate matlab2tikz.
%
\definecolor{mycolor1}{rgb}{0.00000,0.44700,0.74100}%
\definecolor{mycolor2}{rgb}{0.85000,0.32500,0.09800}%
%
\begin{tikzpicture}

\begin{axis}[%
width=4.521in,
height=3.566in,
at={(0.758in,0.481in)},
scale only axis,
xmin=3.4,
xmax=3.6,
xlabel style={font=\color{white!15!black}},
xlabel={Frequency (GHz)},
ymin=-52,
ymax=-38,
ylabel style={font=\color{white!15!black}},
ylabel={Amplitude (dB)},
axis background/.style={fill=white},
xmajorgrids,
ymajorgrids,
legend style={legend cell align=left, align=left, draw=white!15!black}
]
\addplot [color=mycolor1, line width=2.0pt]
  table[row sep=crcr]{%
3.4	-46.890374399546\\
3.40025	-47.2162547033839\\
3.4005	-47.225879820215\\
3.40075	-47.243133613579\\
3.401	-47.3694422250919\\
3.40125	-47.5336925981263\\
3.4015	-47.6925325289531\\
3.40175	-48.2650431939417\\
3.402	-48.7002029282503\\
3.40225	-48.4765637313154\\
3.4025	-47.5074563189457\\
3.40275	-46.4367161806794\\
3.403	-45.8027740685414\\
3.40325	-45.3021461288976\\
3.4035	-44.9888039757175\\
3.40375	-44.9614354501425\\
3.404	-44.9711693392334\\
3.40425	-45.0558234884675\\
3.4045	-45.1615922879765\\
3.40475	-45.025283182814\\
3.405	-44.7648183917659\\
3.40525	-44.3360306540865\\
3.4055	-43.7845328697304\\
3.40575	-43.2782633879116\\
3.406	-42.9990619055769\\
3.40625	-42.6839634301666\\
3.4065	-42.5190534781645\\
3.40675	-42.4134815814612\\
3.407	-42.3717635696205\\
3.40725	-42.3436623497069\\
3.4075	-42.3591991298598\\
3.40775	-42.352610040334\\
3.408	-42.4190942049564\\
3.40825	-42.3873829552613\\
3.4085	-42.3256369331177\\
3.40875	-42.2718080266645\\
3.409	-42.1735704782584\\
3.40925	-42.1428831647445\\
3.4095	-42.1606519501693\\
3.40975	-42.1692455026019\\
3.41	-42.2938390590492\\
3.41025	-42.2142675609793\\
3.4105	-42.1509330284936\\
3.41075	-42.1208028182584\\
3.411	-42.0206458897566\\
3.41125	-41.9643230506821\\
3.4115	-41.8670845731376\\
3.41175	-41.8042133947988\\
3.412	-41.7528951014165\\
3.41225	-41.8186846862107\\
3.4125	-41.8947251511575\\
3.41275	-41.9317100633314\\
3.413	-41.9249855075474\\
3.41325	-41.8629657326065\\
3.4135	-41.7487561204317\\
3.41375	-41.5560312538824\\
3.414	-41.4917518527007\\
3.41425	-41.4195592219134\\
3.4145	-41.3897433153934\\
3.41475	-41.3753392874571\\
3.415	-41.384366205442\\
3.41525	-41.4086592351956\\
3.4155	-41.4684078951668\\
3.41575	-41.5700670192832\\
3.416	-41.5765173483448\\
3.41625	-41.5869584134593\\
3.4165	-41.6333833747078\\
3.41675	-41.6217409355538\\
3.417	-41.6656260020995\\
3.41725	-41.6426547843471\\
3.4175	-41.6227837376001\\
3.41775	-41.6157167914023\\
3.418	-41.6310037405066\\
3.41825	-41.6913685926624\\
3.4185	-41.7318630685892\\
3.41875	-41.8940529921674\\
3.419	-41.9624483443684\\
3.41925	-42.068399106379\\
3.4195	-42.134651735748\\
3.41975	-42.1502061394636\\
3.42	-42.0762391284286\\
3.42025	-41.9382475791436\\
3.4205	-41.7925264852757\\
3.42075	-41.6436628902717\\
3.421	-41.548635648144\\
3.42125	-41.4682636809434\\
3.4215	-41.4515010208891\\
3.42175	-41.5084726480078\\
3.422	-41.4847243319785\\
3.42225	-41.5040018201076\\
3.4225	-41.4776613681336\\
3.42275	-41.4362819313256\\
3.423	-41.4429094157967\\
3.42325	-41.4963980080744\\
3.4235	-41.6254545957844\\
3.42375	-41.7445271317719\\
3.424	-41.8728036756065\\
3.42425	-41.8511195886164\\
3.4245	-41.7731031646758\\
3.42475	-41.6909268930699\\
3.425	-41.6097007714893\\
3.42525	-41.5302858667813\\
3.4255	-41.4904625499044\\
3.42575	-41.5431735650591\\
3.426	-41.6727163845541\\
3.42625	-41.9080473198453\\
3.4265	-42.1660713380632\\
3.42675	-42.3627208346774\\
3.427	-42.4793794725305\\
3.42725	-42.4116791956386\\
3.4275	-42.2586768047628\\
3.42775	-42.1089041269074\\
3.428	-42.0352486965864\\
3.42825	-42.0240229428099\\
3.4285	-42.0269533409021\\
3.42875	-42.1029926289922\\
3.429	-42.214770353178\\
3.42925	-42.429455603113\\
3.4295	-42.6076162628967\\
3.42975	-42.791458916275\\
3.43	-43.0145298093329\\
3.43025	-43.2469783997517\\
3.4305	-43.4288176825753\\
3.43075	-43.6682520314795\\
3.431	-43.8693375120328\\
3.43125	-44.0563523081586\\
3.4315	-43.9488212324007\\
3.43175	-43.9126293337268\\
3.432	-43.7345760928725\\
3.43225	-43.7571364895304\\
3.4325	-43.710087613524\\
3.43275	-43.8389539728138\\
3.433	-44.2311041730937\\
3.43325	-44.668503645933\\
3.4335	-45.0371993323096\\
3.43375	-45.4140915335899\\
3.434	-45.4983762602026\\
3.43425	-45.2789957056646\\
3.4345	-44.8319068626018\\
3.43475	-44.4127633970388\\
3.435	-44.064344402818\\
3.43525	-43.9474411992593\\
3.4355	-43.8414426571127\\
3.43575	-43.9576045472624\\
3.436	-44.0952136824733\\
3.43625	-44.4409121702477\\
3.4365	-44.9741070083708\\
3.43675	-45.2751068008134\\
3.437	-45.5812004508539\\
3.43725	-45.5132792268167\\
3.4375	-45.4646377364799\\
3.43775	-45.3249739393265\\
3.438	-45.2073117137954\\
3.43825	-45.1000719292617\\
3.4385	-44.992805967763\\
3.43875	-45.0225899166734\\
3.439	-44.975576799351\\
3.43925	-44.9700195796796\\
3.4395	-45.1479029508502\\
3.43975	-45.3422607607186\\
3.44	-45.3774918789175\\
3.44025	-45.3931871705311\\
3.4405	-45.6109190521244\\
3.44075	-45.8644026432368\\
3.441	-46.1133848833489\\
3.44125	-46.3615258874319\\
3.4415	-46.5654833231396\\
3.44175	-46.6099329261073\\
3.442	-46.6057866915654\\
3.44225	-46.6573953734459\\
3.4425	-46.7332046532466\\
3.44275	-46.6522972430655\\
3.443	-46.7520514190521\\
3.44325	-46.6902254591473\\
3.4435	-46.8036666522464\\
3.44375	-47.0019352392377\\
3.444	-47.6410380955419\\
3.44425	-48.1416706299013\\
3.4445	-48.9815429057541\\
3.44475	-49.6868134911802\\
3.445	-50.2062310865024\\
3.44525	-50.8163774674933\\
3.4455	-50.4564600963283\\
3.44575	-49.6613141303559\\
3.446	-48.7987369381006\\
3.44625	-47.7537554604865\\
3.4465	-47.639867661138\\
3.44675	-47.4177269876856\\
3.447	-47.4026039084664\\
3.44725	-47.461682657682\\
3.4475	-47.5708404392274\\
3.44775	-47.8731898776105\\
3.448	-48.0161647972269\\
3.44825	-47.9854791673465\\
3.4485	-47.8514631502911\\
3.44875	-47.45639060142\\
3.449	-46.9989492276153\\
3.44925	-46.7157987978712\\
3.4495	-46.6297113034008\\
3.44975	-46.5997202221215\\
3.45	-46.5036051128488\\
3.45025	-46.592803404444\\
3.4505	-46.7446348517743\\
3.45075	-47.1043227295976\\
3.451	-47.4274120119688\\
3.45125	-47.6376015809257\\
3.4515	-47.7019603647385\\
3.45175	-47.2478982466238\\
3.452	-46.848438378518\\
3.45225	-46.3898473837448\\
3.4525	-46.0262462617676\\
3.45275	-46.0083157573113\\
3.453	-45.9099791712407\\
3.45325	-46.1052700335507\\
3.4535	-46.3822962109694\\
3.45375	-46.6980957715334\\
3.454	-47.013730494713\\
3.45425	-47.0401429528029\\
3.4545	-46.8879087604931\\
3.45475	-46.6342349405584\\
3.455	-46.3538805510473\\
3.45525	-46.1027087434682\\
3.4555	-45.756140189606\\
3.45575	-45.5317044837358\\
3.456	-45.3117996268237\\
3.45625	-45.1246828808441\\
3.4565	-44.9385140450253\\
3.45675	-44.7333383213833\\
3.457	-44.5660383096575\\
3.45725	-44.5427992155471\\
3.4575	-44.6174961521371\\
3.45775	-44.5579044402\\
3.458	-44.5695841107256\\
3.45825	-44.5365373986631\\
3.4585	-44.39091782198\\
3.45875	-44.1681548800809\\
3.459	-43.9275151431596\\
3.45925	-43.6182020745629\\
3.4595	-43.3420841587448\\
3.45975	-43.1071527832274\\
3.46	-43.0707938839642\\
3.46025	-43.0686822965256\\
3.4605	-43.185804937995\\
3.46075	-43.4064166487083\\
3.461	-43.7073279461805\\
3.46125	-43.9955355118992\\
3.4615	-44.194952180981\\
3.46175	-44.2146994247127\\
3.462	-44.2249167934648\\
3.46225	-44.265891641144\\
3.4625	-44.2177068261852\\
3.46275	-44.2759498288703\\
3.463	-44.2332278158819\\
3.46325	-44.276090920151\\
3.4635	-44.4079809047892\\
3.46375	-44.5420928441302\\
3.464	-44.6895247443918\\
3.46425	-44.6411190693815\\
3.4645	-44.5063748707642\\
3.46475	-44.5777329816435\\
3.465	-44.6008543402598\\
3.46525	-44.6880936123083\\
3.4655	-44.8667382377366\\
3.46575	-44.8909362346833\\
3.466	-44.8500028438132\\
3.46625	-44.7856877071252\\
3.4665	-44.5707630314317\\
3.46675	-44.3643588391982\\
3.467	-44.1783350434873\\
3.46725	-43.9914671989855\\
3.4675	-43.8879204922286\\
3.46775	-43.8648772750745\\
3.468	-43.7862197809254\\
3.46825	-43.7364570612938\\
3.4685	-43.6339674556528\\
3.46875	-43.5552382398729\\
3.469	-43.401305183416\\
3.46925	-43.3404274036419\\
3.4695	-43.1526177628255\\
3.46975	-43.20284824074\\
3.47	-43.3151656068442\\
3.47025	-43.4003993103314\\
3.4705	-43.5464638533641\\
3.47075	-43.5194102205131\\
3.471	-43.5070940055393\\
3.47125	-43.4836563694157\\
3.4715	-43.4376747689271\\
3.47175	-43.4686465196102\\
3.472	-43.6083551216163\\
3.47225	-43.8103756096876\\
3.4725	-43.9832293440175\\
3.47275	-44.270125640992\\
3.473	-44.5231495692493\\
3.47325	-44.4320701760429\\
3.4735	-44.0661939600876\\
3.47375	-43.6795466042499\\
3.474	-43.1555672082126\\
3.47425	-42.8438107928607\\
3.4745	-42.5677994914051\\
3.47475	-42.4683138658758\\
3.475	-42.4126372216948\\
3.47525	-42.5411190390962\\
3.4755	-42.699900546517\\
3.47575	-42.7528861216894\\
3.476	-42.6728254843486\\
3.47625	-42.3966127049811\\
3.4765	-42.0896837778333\\
3.47675	-41.8156979127407\\
3.477	-41.6203721526683\\
3.47725	-41.4077256267675\\
3.4775	-41.2809948137605\\
3.47775	-41.2855358277572\\
3.478	-41.2923684740437\\
3.47825	-41.3571245105015\\
3.4785	-41.3654500903302\\
3.47875	-41.3367151715899\\
3.479	-41.2878956603008\\
3.47925	-41.2137792291253\\
3.4795	-41.1874562733657\\
3.47975	-41.0927787177187\\
3.48	-40.9665409075643\\
3.48025	-40.8296446678891\\
3.4805	-40.700331868743\\
3.48075	-40.6022150352354\\
3.481	-40.5168277428743\\
3.48125	-40.4327020939608\\
3.4815	-40.3924958862454\\
3.48175	-40.3446271138471\\
3.482	-40.3231105614896\\
3.48225	-40.2982333603998\\
3.4825	-40.287319384914\\
3.48275	-40.2547866636528\\
3.483	-40.2458485310276\\
3.48325	-40.2089782635807\\
3.4835	-40.1685236234243\\
3.48375	-40.1230371629148\\
3.484	-40.1402714826228\\
3.48425	-40.2180615567527\\
3.4845	-40.3025144041486\\
3.48475	-40.4271113389071\\
3.485	-40.5317653355273\\
3.48525	-40.6817131362062\\
3.4855	-40.7964032864389\\
3.48575	-40.9321268514527\\
3.486	-40.9952654483121\\
3.48625	-41.0659204708511\\
3.4865	-41.1415848705739\\
3.48675	-41.1139882686329\\
3.487	-41.1832641992236\\
3.48725	-41.1758054964935\\
3.4875	-41.2077868598987\\
3.48775	-41.1744829326366\\
3.488	-41.0997280648632\\
3.48825	-41.1728512647062\\
3.4885	-41.2221758034469\\
3.48875	-41.2985063148246\\
3.489	-41.3363123155654\\
3.48925	-41.412151599312\\
3.4895	-41.4603082177602\\
3.48975	-41.4693081701988\\
3.49	-41.5218392386004\\
3.49025	-41.5060556695742\\
3.4905	-41.4771466696009\\
3.49075	-41.4496256861889\\
3.491	-41.4335943808233\\
3.49125	-41.4350476696561\\
3.4915	-41.3737929830313\\
3.49175	-41.3358451880866\\
3.492	-41.3235464803928\\
3.49225	-41.3020645026094\\
3.4925	-41.322989023666\\
3.49275	-41.3382307202838\\
3.493	-41.3819631226415\\
3.49325	-41.4376108764008\\
3.4935	-41.5852026226994\\
3.49375	-41.6582062873248\\
3.494	-41.7864118712463\\
3.49425	-41.8794611409777\\
3.4945	-41.9026966241289\\
3.49475	-41.9256251239831\\
3.495	-41.9077074581976\\
3.49525	-41.793893167031\\
3.4955	-41.7005225069618\\
3.49575	-41.6165419911385\\
3.496	-41.5814403608979\\
3.49625	-41.5956659481791\\
3.4965	-41.6726778790993\\
3.49675	-41.804313544727\\
3.497	-41.976775040708\\
3.49725	-42.1306750502611\\
3.4975	-42.2159116014904\\
3.49775	-42.2111212697493\\
3.498	-42.1160913087468\\
3.49825	-41.9620193199042\\
3.4985	-41.7872637307075\\
3.49875	-41.585991030893\\
3.499	-41.5040954498696\\
3.49925	-41.4440839051557\\
3.4995	-41.480727344568\\
3.49975	-41.6041503614985\\
3.5	-41.7398188249797\\
3.50025	-41.9168897230158\\
3.5005	-42.029853528127\\
3.50075	-42.1105976971607\\
3.501	-42.1472203935726\\
3.50125	-42.0481859265468\\
3.5015	-41.9723458439023\\
3.50175	-41.8482255350959\\
3.502	-41.7244820122799\\
3.50225	-41.5797453486964\\
3.5025	-41.5046764273319\\
3.50275	-41.4254747033787\\
3.503	-41.389396188334\\
3.50325	-41.3901795720897\\
3.5035	-41.3914780330537\\
3.50375	-41.4303541283557\\
3.504	-41.5100908903111\\
3.50425	-41.5829468461943\\
3.5045	-41.6503358083567\\
3.50475	-41.6296031666687\\
3.505	-41.548766467192\\
3.50525	-41.4691471810042\\
3.5055	-41.4055638585248\\
3.50575	-41.349799715002\\
3.506	-41.3739363574567\\
3.50625	-41.4325920278164\\
3.5065	-41.5636841674145\\
3.50675	-41.7637543653108\\
3.507	-42.1193434869844\\
3.50725	-42.3787935572955\\
3.5075	-42.6668713101716\\
3.50775	-42.79465621662\\
3.508	-42.7716871148806\\
3.50825	-42.7642971131581\\
3.5085	-42.75831136761\\
3.50875	-42.780631444073\\
3.509	-42.8226321261097\\
3.50925	-42.8466609584842\\
3.5095	-42.9438249333715\\
3.50975	-43.0748650403565\\
3.51	-43.073093256925\\
3.51025	-43.0081239495097\\
3.5105	-43.0069168681927\\
3.51075	-43.0122490560491\\
3.511	-42.9980893812792\\
3.51125	-43.0576103410044\\
3.5115	-42.9860723755815\\
3.51175	-42.8792969621278\\
3.512	-42.6787280456621\\
3.51225	-42.4223976211644\\
3.5125	-42.1438602967827\\
3.51275	-41.8777939119416\\
3.513	-41.7762870388926\\
3.51325	-41.7042340258915\\
3.5135	-41.7248554096974\\
3.51375	-41.7617240539862\\
3.514	-41.8743997246564\\
3.51425	-41.8872641287304\\
3.5145	-41.8510066277322\\
3.51475	-41.749655718587\\
3.515	-41.6356351349785\\
3.51525	-41.5256945813205\\
3.5155	-41.5015124248481\\
3.51575	-41.5632087973184\\
3.516	-41.730020735366\\
3.51625	-41.9702553437541\\
3.5165	-42.2155585357173\\
3.51675	-42.5435392678026\\
3.517	-42.8087745109951\\
3.51725	-42.9520851135595\\
3.5175	-42.9007475538126\\
3.51775	-42.8575058952939\\
3.518	-42.7535261586179\\
3.51825	-42.6735315440844\\
3.5185	-42.6761990442139\\
3.51875	-42.7740273238683\\
3.519	-42.9613632727915\\
3.51925	-43.1181958161546\\
3.5195	-43.3737588535234\\
3.51975	-43.5323911654589\\
3.52	-43.5814009552922\\
3.52025	-43.4516399766687\\
3.5205	-43.3912669005921\\
3.52075	-43.4006771474741\\
3.521	-43.556250924487\\
3.52125	-43.794259268857\\
3.5215	-44.1427878708001\\
3.52175	-44.6343326941459\\
3.522	-45.0790468331118\\
3.52225	-45.3290236748418\\
3.5225	-45.140133022938\\
3.52275	-44.8481196688215\\
3.523	-44.3734651624591\\
3.52325	-44.1262222263099\\
3.5235	-44.0191557245688\\
3.52375	-44.1991024450823\\
3.524	-44.4591984092883\\
3.52425	-44.9889948784858\\
3.5245	-45.634267307883\\
3.52475	-46.2530220015084\\
3.525	-46.6488160927412\\
3.52525	-46.5073012661487\\
3.5255	-46.031835508315\\
3.52575	-45.6617529437928\\
3.526	-45.4631080719181\\
3.52625	-45.2908669181057\\
3.5265	-45.295479821914\\
3.52675	-45.162985972547\\
3.527	-45.2091286532848\\
3.52725	-45.5524722804142\\
3.5275	-45.8104313786318\\
3.52775	-46.2247081035741\\
3.528	-46.6835411349767\\
3.52825	-47.1966493881205\\
3.5285	-47.3705201906193\\
3.52875	-47.6599141331456\\
3.529	-47.9352424780696\\
3.52925	-48.1648233970668\\
3.5295	-48.3358519434901\\
3.52975	-47.5316416341847\\
3.53	-47.0161838342247\\
3.53025	-46.3829686060982\\
3.5305	-45.9009249979248\\
3.53075	-45.6758974426448\\
3.531	-45.5813529371144\\
3.53125	-45.642973627949\\
3.5315	-45.7962309830791\\
3.53175	-46.0657246506486\\
3.532	-46.3590933599179\\
3.53225	-46.5549061004637\\
3.5325	-46.127586001411\\
3.53275	-45.5810452674693\\
3.533	-45.0296368383728\\
3.53325	-44.4532799576188\\
3.5335	-44.0833131230425\\
3.53375	-43.751060024798\\
3.534	-43.4977176371998\\
3.53425	-43.3371973788972\\
3.5345	-43.0746638549575\\
3.53475	-42.8880346823226\\
3.535	-42.6919501249042\\
3.53525	-42.5520844624512\\
3.5355	-42.5203139811161\\
3.53575	-42.4885887681836\\
3.536	-42.5487166842137\\
3.53625	-42.5268642441107\\
3.5365	-42.5315980398933\\
3.53675	-42.5075583680974\\
3.537	-42.489161314302\\
3.53725	-42.3651777546284\\
3.5375	-42.2160720037954\\
3.53775	-42.0907522147598\\
3.538	-42.0756818381072\\
3.53825	-42.070042793255\\
3.5385	-42.2100320877705\\
3.53875	-42.3633390586244\\
3.539	-42.5160475322636\\
3.53925	-42.6716706208176\\
3.5395	-42.7309095025411\\
3.53975	-42.8899470890633\\
3.54	-43.0033933040663\\
3.54025	-43.0482640058444\\
3.5405	-43.1260471273874\\
3.54075	-43.2158850141786\\
3.541	-43.2446467802552\\
3.54125	-43.2115010146419\\
3.5415	-43.0860435376419\\
3.54175	-42.9865717865965\\
3.542	-42.9264149151803\\
3.54225	-42.9137226896408\\
3.5425	-42.8761930816671\\
3.54275	-42.8774412533563\\
3.543	-42.841813268394\\
3.54325	-42.7341493533429\\
3.5435	-42.7239674977245\\
3.54375	-42.6254458551552\\
3.544	-42.5746928908011\\
3.54425	-42.5556668644227\\
3.5445	-42.5328198945193\\
3.54475	-42.559446758109\\
3.545	-42.6024242080662\\
3.54525	-42.602148751856\\
3.5455	-42.6146831867512\\
3.54575	-42.4878540559018\\
3.546	-42.2108959749001\\
3.54625	-41.9517486778475\\
3.5465	-41.6889949846001\\
3.54675	-41.5067902157909\\
3.547	-41.39112292039\\
3.54725	-41.3617749164508\\
3.5475	-41.3094170079765\\
3.54775	-41.3145812596365\\
3.548	-41.2488893265099\\
3.54825	-41.1629409040593\\
3.5485	-40.9526019692018\\
3.54875	-40.7852533838192\\
3.549	-40.5732345641892\\
3.54925	-40.4529291818648\\
3.5495	-40.3977909015534\\
3.54975	-40.4743020344871\\
3.55	-40.6397770292495\\
3.55025	-40.8268141301401\\
3.5505	-40.9858248332678\\
3.55075	-41.0996641336362\\
3.551	-41.1115070320301\\
3.55125	-40.9979278392644\\
3.5515	-40.8311816427203\\
3.55175	-40.6883212351223\\
3.552	-40.5974189993929\\
3.55225	-40.5920368870391\\
3.5525	-40.6811741762803\\
3.55275	-40.918010199168\\
3.553	-41.1729658508423\\
3.55325	-41.4105132318468\\
3.5535	-41.6393563108215\\
3.55375	-41.6730643673795\\
3.554	-41.6119220540279\\
3.55425	-41.4997172240479\\
3.5545	-41.3217163372126\\
3.55475	-41.1475467325092\\
3.555	-41.0474101019797\\
3.55525	-41.0556769517152\\
3.5555	-41.140857668672\\
3.55575	-41.3599361691102\\
3.556	-41.5587357567512\\
3.55625	-41.7897351184784\\
3.5565	-41.958844390921\\
3.55675	-42.0257261544832\\
3.557	-41.9772524675935\\
3.55725	-41.9949950727521\\
3.5575	-41.9390101775789\\
3.55775	-41.9514533134897\\
3.558	-41.9833057764242\\
3.55825	-41.9541145187067\\
3.5585	-42.0304952534862\\
3.55875	-42.0956829323985\\
3.559	-42.1758828634933\\
3.55925	-42.3111817036348\\
3.5595	-42.3513239419325\\
3.55975	-42.5069834779226\\
3.56	-42.7279902991587\\
3.56025	-42.9938965323699\\
3.5605	-43.0756963641184\\
3.56075	-43.1100928805095\\
3.561	-43.1114165458442\\
3.56125	-43.0625445682163\\
3.5615	-42.9468607510269\\
3.56175	-42.8832435998816\\
3.562	-42.91013989055\\
3.56225	-43.0567246145338\\
3.5625	-43.1920785323963\\
3.56275	-43.4138970853978\\
3.563	-43.5994464380823\\
3.56325	-43.7728806828978\\
3.5635	-43.8671713824445\\
3.56375	-43.8805798015088\\
3.564	-43.8743747264688\\
3.56425	-43.6470935025236\\
3.5645	-43.6088047811528\\
3.56475	-43.5977786517968\\
3.565	-43.5909270354723\\
3.56525	-43.6579875239357\\
3.5655	-43.71939264331\\
3.56575	-43.8358705733959\\
3.566	-43.981166979832\\
3.56625	-44.1582779623714\\
3.5665	-44.4000969602433\\
3.56675	-44.7856761602289\\
3.567	-45.0854411153554\\
3.56725	-45.3231665246012\\
3.5675	-45.7369685816242\\
3.56775	-46.0276929283452\\
3.568	-46.2538557454797\\
3.56825	-46.0645004229964\\
3.5685	-45.6501584187001\\
3.56875	-45.1787451914312\\
3.569	-44.9327783182948\\
3.56925	-44.8012324505235\\
3.5695	-44.891967363839\\
3.56975	-44.8355373668023\\
3.57	-44.9591451259259\\
3.57025	-44.978963152953\\
3.5705	-45.0427782303743\\
3.57075	-44.9822754875566\\
3.571	-44.6600603936843\\
3.57125	-44.2396573312225\\
3.5715	-44.0149613066939\\
3.57175	-43.8075284651697\\
3.572	-43.8154744427426\\
3.57225	-43.8625904856346\\
3.5725	-43.8602687492417\\
3.57275	-43.7705795286459\\
3.573	-43.6746973244995\\
3.57325	-43.5831036800425\\
3.5735	-43.4635674571377\\
3.57375	-43.2381674060802\\
3.574	-43.0553086187538\\
3.57425	-42.9696406606697\\
3.5745	-43.1285239628757\\
3.57475	-43.280602460044\\
3.575	-43.504200064852\\
3.57525	-43.7177614379153\\
3.5755	-43.8260157610042\\
3.57575	-43.7718267815732\\
3.576	-43.5462455534209\\
3.57625	-43.4214813481382\\
3.5765	-43.2039026370801\\
3.57675	-43.0688733638269\\
3.577	-43.0987250759733\\
3.57725	-43.1192015728099\\
3.5775	-43.2148415865053\\
3.57775	-43.3912572722328\\
3.578	-43.616197539949\\
3.57825	-43.7565434927981\\
3.5785	-43.7696417095317\\
3.57875	-43.7657652271391\\
3.579	-43.8247663581801\\
3.57925	-44.0324130710859\\
3.5795	-44.201846148078\\
3.57975	-44.2797638533652\\
3.58	-44.1895206875142\\
3.58025	-43.8522191174678\\
3.5805	-43.6587284609281\\
3.58075	-43.6141849660188\\
3.581	-43.5515683820929\\
3.58125	-43.6852365354894\\
3.5815	-43.9510134448692\\
3.58175	-44.1563722283853\\
3.582	-44.2096431112818\\
3.58225	-43.9589926436797\\
3.5825	-43.4810476667715\\
3.58275	-42.9875267771622\\
3.583	-42.5566062915072\\
3.58325	-42.3155265570266\\
3.5835	-42.2359183511244\\
3.58375	-42.360255292434\\
3.584	-42.5545702777436\\
3.58425	-42.7935137162605\\
3.5845	-42.9104372359877\\
3.58475	-42.7742392942528\\
3.585	-42.4743109071147\\
3.58525	-42.1168067895268\\
3.5855	-41.7576254180924\\
3.58575	-41.4497920584065\\
3.586	-41.2983860713147\\
3.58625	-41.3019480892238\\
3.5865	-41.418755272803\\
3.58675	-41.5862991088443\\
3.587	-41.6684356666225\\
3.58725	-41.6605561226193\\
3.5875	-41.5124208654963\\
3.58775	-41.40898927524\\
3.588	-41.2239351953905\\
3.58825	-41.0413526200908\\
3.5885	-40.96545110611\\
3.58875	-40.9386920569293\\
3.589	-41.0109266514956\\
3.58925	-41.1554271624351\\
3.5895	-41.3202146994938\\
3.58975	-41.297098935766\\
3.59	-41.2175428395496\\
3.59025	-41.0668555934056\\
3.5905	-40.9595380980271\\
3.59075	-40.8317023558659\\
3.591	-40.7928518824267\\
3.59125	-40.8089857681194\\
3.5915	-40.9325554602737\\
3.59175	-41.1510102575207\\
3.592	-41.4125555423212\\
3.59225	-41.7249327975602\\
3.5925	-41.9040067547133\\
3.59275	-41.8731888885735\\
3.593	-41.7172994010799\\
3.59325	-41.4965214799303\\
3.5935	-41.3162769886523\\
3.59375	-41.213163432821\\
3.594	-41.1335591890901\\
3.59425	-41.0653908857391\\
3.5945	-41.0352575051389\\
3.59475	-41.0301489285053\\
3.595	-41.0851558496342\\
3.59525	-41.1259647071151\\
3.5955	-41.1049151563167\\
3.59575	-41.0373574271985\\
3.596	-40.9718921129304\\
3.59625	-40.8895622516084\\
3.5965	-40.7797634420677\\
3.59675	-40.6884482642739\\
3.597	-40.5506537594157\\
3.59725	-40.4571863044338\\
3.5975	-40.4035789788337\\
3.59775	-40.3542051891538\\
3.598	-40.3041383887501\\
3.59825	-40.2337779813201\\
3.5985	-40.2122725755543\\
3.59875	-40.1695357833949\\
3.599	-40.1547437633335\\
3.59925	-40.1519360583217\\
3.5995	-40.0931055778447\\
3.59975	-40.0150121090802\\
3.6	-39.9060068705567\\
};
\addlegendentry{S21}

\addplot [color=mycolor2, dashed, line width=1.5pt]
  table[row sep=crcr]{%
3.4	-46.7512639629951\\
3.40025	-47.107959066408\\
3.4005	-47.2133237702258\\
3.40075	-47.2275402122231\\
3.401	-47.3332588176036\\
3.40125	-47.2881998371223\\
3.4015	-47.7503870559877\\
3.40175	-48.2832086842319\\
3.402	-48.4686899134193\\
3.40225	-48.2872403311633\\
3.4025	-47.4160644670838\\
3.40275	-46.4268555253839\\
3.403	-45.6342207661202\\
3.40325	-45.1954244185385\\
3.4035	-44.9581491622913\\
3.40375	-44.8881962905103\\
3.404	-44.9198926486703\\
3.40425	-45.0271367528674\\
3.4045	-44.9997369862566\\
3.40475	-44.911475417259\\
3.405	-44.7613971600507\\
3.40525	-44.2687851251859\\
3.4055	-43.762437090143\\
3.40575	-43.3246226042214\\
3.406	-42.9270532230437\\
3.40625	-42.7100808921266\\
3.4065	-42.5743599250116\\
3.40675	-42.3799131012635\\
3.407	-42.334310692296\\
3.40725	-42.322692736522\\
3.4075	-42.3275703441095\\
3.40775	-42.3520767097445\\
3.408	-42.4160282854886\\
3.40825	-42.3675403354935\\
3.4085	-42.3150429261892\\
3.40875	-42.1709158029933\\
3.409	-42.1210120364556\\
3.40925	-42.0925593663579\\
3.4095	-42.0805641507037\\
3.40975	-42.1197954304548\\
3.41	-42.189048488157\\
3.41025	-42.1404949647944\\
3.4105	-42.0498581445008\\
3.41075	-42.0250679806015\\
3.411	-41.932356615775\\
3.41125	-41.8531018261812\\
3.4115	-41.7425387388998\\
3.41175	-41.6831489071694\\
3.412	-41.6762376021865\\
3.41225	-41.7127637178119\\
3.4125	-41.8000298990292\\
3.41275	-41.8236074968084\\
3.413	-41.8108162673678\\
3.41325	-41.7638485688679\\
3.4135	-41.6341121720219\\
3.41375	-41.5319640011779\\
3.414	-41.4658130372826\\
3.41425	-41.3670124317905\\
3.4145	-41.3435746163774\\
3.41475	-41.3354266333136\\
3.415	-41.3511368343041\\
3.41525	-41.3988436860457\\
3.4155	-41.4750073659158\\
3.41575	-41.5153380093211\\
3.416	-41.5959183035795\\
3.41625	-41.6202048671682\\
3.4165	-41.651672202056\\
3.41675	-41.6339818539267\\
3.417	-41.6659014704926\\
3.41725	-41.6829702848168\\
3.4175	-41.6536977444131\\
3.41775	-41.6232972079033\\
3.418	-41.6217279696851\\
3.41825	-41.6542016497204\\
3.4185	-41.759844031524\\
3.41875	-41.8457280288679\\
3.419	-41.9278075702173\\
3.41925	-42.0394172054033\\
3.4195	-42.1000606157616\\
3.41975	-42.1486499290559\\
3.42	-42.0621397817757\\
3.42025	-41.9482510641255\\
3.4205	-41.7527350248242\\
3.42075	-41.6430784132477\\
3.421	-41.5355068157638\\
3.42125	-41.4598074462459\\
3.4215	-41.4326370604962\\
3.42175	-41.4731964338396\\
3.422	-41.454117655391\\
3.42225	-41.4556002021234\\
3.4225	-41.4367370994356\\
3.42275	-41.3956908545033\\
3.423	-41.3917721152691\\
3.42325	-41.4846308422571\\
3.4235	-41.5581387277927\\
3.42375	-41.7062853579193\\
3.424	-41.7999946620956\\
3.42425	-41.7742527490391\\
3.4245	-41.7057790485962\\
3.42475	-41.6476310508791\\
3.425	-41.5393439455799\\
3.42525	-41.4767158308831\\
3.4255	-41.4533575893831\\
3.42575	-41.4863626547888\\
3.426	-41.6540679519521\\
3.42625	-41.8982068083475\\
3.4265	-42.1541052350123\\
3.42675	-42.3619741651483\\
3.427	-42.4264992881125\\
3.42725	-42.354051242938\\
3.4275	-42.2960342092\\
3.42775	-42.1845972445795\\
3.428	-42.0798925940893\\
3.42825	-42.0266159991461\\
3.4285	-42.0419580960459\\
3.42875	-42.1118803550142\\
3.429	-42.2951487622461\\
3.42925	-42.4360734167106\\
3.4295	-42.6362406146908\\
3.42975	-42.8652234148879\\
3.43	-43.0227390673651\\
3.43025	-43.2905366589534\\
3.4305	-43.5106192501558\\
3.43075	-43.7495073895169\\
3.431	-43.9133160893652\\
3.43125	-43.9908610002299\\
3.4315	-44.0156567331432\\
3.43175	-43.8557024571906\\
3.432	-43.7271121535203\\
3.43225	-43.6933929250209\\
3.4325	-43.7249170541686\\
3.43275	-43.8786344321786\\
3.433	-44.189829071863\\
3.43325	-44.5528727780486\\
3.4335	-44.9849342355465\\
3.43375	-45.4089160658256\\
3.434	-45.493061711563\\
3.43425	-45.2113256325067\\
3.4345	-44.7445734355544\\
3.43475	-44.3602165146849\\
3.435	-43.9760458528501\\
3.43525	-43.8336950237638\\
3.4355	-43.7648035711971\\
3.43575	-43.9276463012898\\
3.436	-44.1049972625378\\
3.43625	-44.3888514971777\\
3.4365	-44.9046819880502\\
3.43675	-45.1765341367926\\
3.437	-45.5141670007767\\
3.43725	-45.5460022204125\\
3.4375	-45.3311405794448\\
3.43775	-45.2551089548983\\
3.438	-45.1400255366494\\
3.43825	-45.1115048681394\\
3.4385	-45.0036614854031\\
3.43875	-44.9568879415083\\
3.439	-44.9256328149816\\
3.43925	-44.9808051898268\\
3.4395	-45.1247814433216\\
3.43975	-45.2870993607299\\
3.44	-45.3324076156269\\
3.44025	-45.3968975268408\\
3.4405	-45.6281952208448\\
3.44075	-45.8490584475793\\
3.441	-46.1786723593501\\
3.44125	-46.3568747358891\\
3.4415	-46.5158758474101\\
3.44175	-46.7723573061476\\
3.442	-46.6028774102792\\
3.44225	-46.638181857238\\
3.4425	-46.7443368177759\\
3.44275	-46.7621798909437\\
3.443	-46.6109704583721\\
3.44325	-46.6806120636463\\
3.4435	-46.8597338290798\\
3.44375	-47.073926388103\\
3.444	-47.4880389216515\\
3.44425	-48.3000624448234\\
3.4445	-48.9171328115952\\
3.44475	-49.523458882141\\
3.445	-50.3457429692808\\
3.44525	-50.7247558977166\\
3.4455	-50.464357109705\\
3.44575	-49.537884486423\\
3.446	-48.522272867492\\
3.44625	-47.8727831881243\\
3.4465	-47.618763166713\\
3.44675	-47.2864741970819\\
3.447	-47.2485012697332\\
3.44725	-47.3241760890227\\
3.4475	-47.4545096642833\\
3.44775	-47.591209792117\\
3.448	-47.7813162940394\\
3.44825	-47.8357984874292\\
3.4485	-47.5226990853656\\
3.44875	-47.3787979771286\\
3.449	-47.0815053371294\\
3.44925	-46.6421833163741\\
3.4495	-46.4155040565323\\
3.44975	-46.3492647519949\\
3.45	-46.3892266428047\\
3.45025	-46.4800587889983\\
3.4505	-46.7013629832502\\
3.45075	-47.0278154377184\\
3.451	-47.3033077800151\\
3.45125	-47.6789634574923\\
3.4515	-47.5747013597152\\
3.45175	-47.2738924818799\\
3.452	-46.8687894151895\\
3.45225	-46.3200934474975\\
3.4525	-45.9738341858831\\
3.45275	-45.8716958976441\\
3.453	-45.829821085803\\
3.45325	-46.0402272234067\\
3.4535	-46.2899878595985\\
3.45375	-46.5511914014733\\
3.454	-46.8050615857163\\
3.45425	-47.091805752782\\
3.4545	-46.9193847517883\\
3.45475	-46.6172734212859\\
3.455	-46.2853472941801\\
3.45525	-46.0606331691476\\
3.4555	-45.7445960631608\\
3.45575	-45.4947516957966\\
3.456	-45.2589655080267\\
3.45625	-44.9576158836572\\
3.4565	-44.8965096894826\\
3.45675	-44.6131274667305\\
3.457	-44.5518047008214\\
3.45725	-44.4991175168685\\
3.4575	-44.4992805572984\\
3.45775	-44.4806863821245\\
3.458	-44.5108137313486\\
3.45825	-44.4638596944304\\
3.4585	-44.3739284732198\\
3.45875	-44.1305122839141\\
3.459	-43.8324207593807\\
3.45925	-43.4861954164626\\
3.4595	-43.2577224951398\\
3.45975	-43.0607707046026\\
3.46	-42.973438379429\\
3.46025	-43.0006920510551\\
3.4605	-43.1549453305466\\
3.46075	-43.3254402085115\\
3.461	-43.6105411700009\\
3.46125	-43.9459195899593\\
3.4615	-44.1286276140071\\
3.46175	-44.2957195677285\\
3.462	-44.2643519791061\\
3.46225	-44.2474674466344\\
3.4625	-44.2800209639142\\
3.46275	-44.2701561505322\\
3.463	-44.1834776739706\\
3.46325	-44.3146242593454\\
3.4635	-44.4604958463894\\
3.46375	-44.5388327701478\\
3.464	-44.687344474494\\
3.46425	-44.7076763927922\\
3.4645	-44.5936722334331\\
3.46475	-44.5590753596103\\
3.465	-44.6086003904156\\
3.46525	-44.7976319347779\\
3.4655	-44.8850670543162\\
3.46575	-44.9612472055181\\
3.466	-44.8635677364265\\
3.46625	-44.7821784039217\\
3.4665	-44.6110764814066\\
3.46675	-44.3559391615581\\
3.467	-44.1705912190979\\
3.46725	-44.0195510462044\\
3.4675	-43.8952862098957\\
3.46775	-43.83637388238\\
3.468	-43.7724288465626\\
3.46825	-43.7224313921389\\
3.4685	-43.6655873330213\\
3.46875	-43.4850358367636\\
3.469	-43.3778397694151\\
3.46925	-43.2166498723148\\
3.4695	-43.1989525926474\\
3.46975	-43.1589557557612\\
3.47	-43.261692886501\\
3.47025	-43.3533383983852\\
3.4705	-43.4414731055853\\
3.47075	-43.4762875803482\\
3.471	-43.4025477072202\\
3.47125	-43.4125719625516\\
3.4715	-43.432571190903\\
3.47175	-43.4393338855633\\
3.472	-43.5246210571351\\
3.47225	-43.6427068337378\\
3.4725	-43.9518113755295\\
3.47275	-44.192089683326\\
3.473	-44.4719824926141\\
3.47325	-44.4195683384264\\
3.4735	-44.0759458666412\\
3.47375	-43.5970141067518\\
3.474	-43.1720016841412\\
3.47425	-42.7915841903468\\
3.4745	-42.6056273484945\\
3.47475	-42.4657580226003\\
3.475	-42.4410098586734\\
3.47525	-42.5345862348663\\
3.4755	-42.7454714870074\\
3.47575	-42.7848325840051\\
3.476	-42.6746376470715\\
3.47625	-42.404578106499\\
3.4765	-42.1444972250131\\
3.47675	-41.8975988673383\\
3.477	-41.6262339928687\\
3.47725	-41.4424057569418\\
3.4775	-41.3246453134855\\
3.47775	-41.3098627468193\\
3.478	-41.3445824231603\\
3.47825	-41.344762961727\\
3.4785	-41.373786363332\\
3.47875	-41.3693921857931\\
3.479	-41.3044002604191\\
3.47925	-41.2503212350499\\
3.4795	-41.1840095402054\\
3.47975	-41.0805744322333\\
3.48	-40.9691909228303\\
3.48025	-40.7989050451515\\
3.4805	-40.6947029651093\\
3.48075	-40.5600370835834\\
3.481	-40.4893406377664\\
3.48125	-40.408952371389\\
3.4815	-40.3557003930378\\
3.48175	-40.3165987737423\\
3.482	-40.2799865105809\\
3.48225	-40.2477229362589\\
3.4825	-40.2324785485091\\
3.48275	-40.2041367879241\\
3.483	-40.1894985710559\\
3.48325	-40.1402225911269\\
3.4835	-40.0906559674415\\
3.48375	-40.0454557495217\\
3.484	-40.0674343881671\\
3.48425	-40.1410762694496\\
3.4845	-40.227373286119\\
3.48475	-40.3390448513128\\
3.485	-40.5029600066864\\
3.48525	-40.625803591183\\
3.4855	-40.7477738639765\\
3.48575	-40.8533298430519\\
3.486	-40.9529403963158\\
3.48625	-41.0559999435246\\
3.4865	-41.0850701909729\\
3.48675	-41.1237337000837\\
3.487	-41.1363604997733\\
3.48725	-41.2022468175124\\
3.4875	-41.1562433296367\\
3.48775	-41.1335844588538\\
3.488	-41.1414896749646\\
3.48825	-41.1883615377623\\
3.4885	-41.2407314460918\\
3.48875	-41.3064547869501\\
3.489	-41.3766903503316\\
3.48925	-41.4530078307254\\
3.4895	-41.4656164923513\\
3.48975	-41.559744491607\\
3.49	-41.5807033465639\\
3.49025	-41.5487929928972\\
3.4905	-41.5552452058466\\
3.49075	-41.4960412202182\\
3.491	-41.4695578021145\\
3.49125	-41.4319376671771\\
3.4915	-41.38259769156\\
3.49175	-41.3708982337775\\
3.492	-41.3488225951285\\
3.49225	-41.3280784552563\\
3.4925	-41.2982453108894\\
3.49275	-41.3137970631219\\
3.493	-41.3565778149806\\
3.49325	-41.4331126559097\\
3.4935	-41.5024175314938\\
3.49375	-41.6380052063834\\
3.494	-41.7554788829777\\
3.49425	-41.8185374795911\\
3.4945	-41.895019313155\\
3.49475	-41.8600549298071\\
3.495	-41.8438261182443\\
3.49525	-41.77314345586\\
3.4955	-41.6223050007643\\
3.49575	-41.5723720625492\\
3.496	-41.5168671435843\\
3.49625	-41.528882095954\\
3.4965	-41.6365711637872\\
3.49675	-41.8258368763142\\
3.497	-41.9299901862632\\
3.49725	-42.0886817420277\\
3.4975	-42.1723859960191\\
3.49775	-42.1555843648248\\
3.498	-42.143396626003\\
3.49825	-41.9716788093288\\
3.4985	-41.7739999237853\\
3.49875	-41.5851681688476\\
3.499	-41.4826791860476\\
3.49925	-41.421041084394\\
3.4995	-41.4702318181361\\
3.49975	-41.6202414956643\\
3.5	-41.8003130931643\\
3.50025	-41.9286816506392\\
3.5005	-42.0466431095543\\
3.50075	-42.1304869699679\\
3.501	-42.122074984018\\
3.50125	-42.0833906905288\\
3.5015	-41.9506933577687\\
3.50175	-41.8312804892272\\
3.502	-41.6688526776321\\
3.50225	-41.581182713553\\
3.5025	-41.4981101218359\\
3.50275	-41.4284985471111\\
3.503	-41.3918166158196\\
3.50325	-41.3968675173184\\
3.5035	-41.3819939586345\\
3.50375	-41.4072974784416\\
3.504	-41.4739250888766\\
3.50425	-41.5487920039541\\
3.5045	-41.6049220966359\\
3.50475	-41.5775657466311\\
3.505	-41.5055237958522\\
3.50525	-41.4182347752462\\
3.5055	-41.3675914697245\\
3.50575	-41.2577429397322\\
3.506	-41.269226090391\\
3.50625	-41.3187507513966\\
3.5065	-41.4963673600533\\
3.50675	-41.7000057406138\\
3.507	-42.0369831502289\\
3.50725	-42.3468646596353\\
3.5075	-42.5831692071779\\
3.50775	-42.6779519860295\\
3.508	-42.7350154989542\\
3.50825	-42.6897334024174\\
3.5085	-42.6818811878842\\
3.50875	-42.6646875836424\\
3.509	-42.7303494162252\\
3.50925	-42.808927052188\\
3.5095	-42.888848670733\\
3.50975	-42.9794689581796\\
3.51	-42.9937880235064\\
3.51025	-42.983909612605\\
3.5105	-42.9438957486404\\
3.51075	-42.9623599398276\\
3.511	-42.9865348623955\\
3.51125	-42.9798658725901\\
3.5115	-42.9918508306861\\
3.51175	-42.9406836490315\\
3.512	-42.6518116359534\\
3.51225	-42.4318399038873\\
3.5125	-42.1144672877284\\
3.51275	-41.9476234885931\\
3.513	-41.754322459347\\
3.51325	-41.7053151799106\\
3.5135	-41.7166389099847\\
3.51375	-41.7547643860771\\
3.514	-41.8292735717471\\
3.51425	-41.8856964469461\\
3.5145	-41.8560879242443\\
3.51475	-41.7745648815075\\
3.515	-41.5787322852497\\
3.51525	-41.5274339459095\\
3.5155	-41.4946566566528\\
3.51575	-41.5750163696711\\
3.516	-41.704996317267\\
3.51625	-41.9016699258296\\
3.5165	-42.1898750009088\\
3.51675	-42.4713125662774\\
3.517	-42.7086385341122\\
3.51725	-42.8711554330677\\
3.5175	-42.8703401760951\\
3.51775	-42.7409442984901\\
3.518	-42.6783740266237\\
3.51825	-42.6115279213851\\
3.5185	-42.5922354950864\\
3.51875	-42.6906121739851\\
3.519	-42.9007036046518\\
3.51925	-43.1396649745747\\
3.5195	-43.3623357699205\\
3.51975	-43.4416118545908\\
3.52	-43.491831056496\\
3.52025	-43.367255456696\\
3.5205	-43.3880698728395\\
3.52075	-43.4041980650767\\
3.521	-43.4753262500487\\
3.52125	-43.7037872517758\\
3.5215	-44.0862270027948\\
3.52175	-44.5834562009185\\
3.522	-45.0569525200502\\
3.52225	-45.2323818656779\\
3.5225	-45.1149592334992\\
3.52275	-44.6804215008095\\
3.523	-44.3301658905679\\
3.52325	-44.1104327295084\\
3.5235	-44.0674809363344\\
3.52375	-44.1823870660026\\
3.524	-44.5392271109255\\
3.52425	-45.0018036215374\\
3.5245	-45.6366869398221\\
3.52475	-46.2869204413493\\
3.525	-46.6337408332608\\
3.52525	-46.5003490620333\\
3.5255	-46.1047837542848\\
3.52575	-45.767393823903\\
3.526	-45.4509345873905\\
3.52625	-45.3279811237877\\
3.5265	-45.1805894211165\\
3.52675	-45.2028680019466\\
3.527	-45.2635112147816\\
3.52725	-45.5454157255222\\
3.5275	-45.7575426978202\\
3.52775	-46.2501968167266\\
3.528	-46.6470139843323\\
3.52825	-47.0877604167234\\
3.5285	-47.3105471823209\\
3.52875	-47.552206094756\\
3.529	-48.0108647579318\\
3.52925	-48.0550947781386\\
3.5295	-48.0510431183256\\
3.52975	-47.549322882214\\
3.53	-46.9093724332818\\
3.53025	-46.2928964456058\\
3.5305	-45.8302321296717\\
3.53075	-45.5864005542648\\
3.531	-45.4367225417968\\
3.53125	-45.5015040679191\\
3.5315	-45.6604735303261\\
3.53175	-45.9104848261617\\
3.532	-46.334631986148\\
3.53225	-46.3433038229942\\
3.5325	-46.0664540353311\\
3.53275	-45.5762237859854\\
3.533	-44.9249137842941\\
3.53325	-44.4195775671844\\
3.5335	-44.0235220375689\\
3.53375	-43.7376618374792\\
3.534	-43.4586899100892\\
3.53425	-43.279680161169\\
3.5345	-43.0563859669072\\
3.53475	-42.8342070839298\\
3.535	-42.6581467228341\\
3.53525	-42.4945795656598\\
3.5355	-42.4568621934877\\
3.53575	-42.4787241322279\\
3.536	-42.529347612088\\
3.53625	-42.5187892640036\\
3.5365	-42.5173074491929\\
3.53675	-42.477279093886\\
3.537	-42.4517132596272\\
3.53725	-42.3064665544925\\
3.5375	-42.1905999025547\\
3.53775	-42.0959553819501\\
3.538	-42.0312818504925\\
3.53825	-42.1212238726934\\
3.5385	-42.1582427064088\\
3.53875	-42.3465882102217\\
3.539	-42.4519808307387\\
3.53925	-42.6269896999717\\
3.5395	-42.7195752896668\\
3.53975	-42.8663290677298\\
3.54	-42.9103236630996\\
3.54025	-43.0026850763156\\
3.5405	-43.0601443742278\\
3.54075	-43.0895599286632\\
3.541	-43.1817946969653\\
3.54125	-43.0919878421997\\
3.5415	-43.0879497694634\\
3.54175	-42.9552694723731\\
3.542	-42.856715773737\\
3.54225	-42.8297129514576\\
3.5425	-42.8176988648042\\
3.54275	-42.809421285673\\
3.543	-42.7497896829185\\
3.54325	-42.7126592723501\\
3.5435	-42.6089880391047\\
3.54375	-42.5370387523968\\
3.544	-42.4952355678628\\
3.54425	-42.4565304515411\\
3.5445	-42.4413451219089\\
3.54475	-42.4877163958162\\
3.545	-42.5079785126274\\
3.54525	-42.539831065947\\
3.5455	-42.4758119371035\\
3.54575	-42.3737018472538\\
3.546	-42.1723629344581\\
3.54625	-41.8399151684129\\
3.5465	-41.6264658796939\\
3.54675	-41.4295498782784\\
3.547	-41.3397058481138\\
3.54725	-41.2807007761675\\
3.5475	-41.2930237819422\\
3.54775	-41.2665803181928\\
3.548	-41.218976255287\\
3.54825	-41.1093824154269\\
3.5485	-40.9312726953832\\
3.54875	-40.7107856503416\\
3.549	-40.5419246015197\\
3.54925	-40.4280853336999\\
3.5495	-40.4200912206255\\
3.54975	-40.4357470560001\\
3.55	-40.6070312964486\\
3.55025	-40.8186362879276\\
3.5505	-41.003101551862\\
3.55075	-41.0929341686215\\
3.551	-41.1023903674134\\
3.55125	-40.9557644468167\\
3.5515	-40.8223341319667\\
3.55175	-40.6380811401784\\
3.552	-40.5465019880668\\
3.55225	-40.5602933153314\\
3.5525	-40.6517683708769\\
3.55275	-40.8322069722002\\
3.553	-41.0877712387301\\
3.55325	-41.3664889824628\\
3.5535	-41.5243947980406\\
3.55375	-41.597676953403\\
3.554	-41.5395991777965\\
3.55425	-41.3998534331327\\
3.5545	-41.2259059407387\\
3.55475	-41.0378813365676\\
3.555	-40.9452334595842\\
3.55525	-40.9491108492595\\
3.5555	-41.0626880349262\\
3.55575	-41.2754545132824\\
3.556	-41.5158522355708\\
3.55625	-41.7471802886427\\
3.5565	-41.9019713760278\\
3.55675	-41.9571026534382\\
3.557	-41.9601058996716\\
3.55725	-41.9384246767079\\
3.5575	-41.87502074352\\
3.55775	-41.8839970060925\\
3.558	-41.8724445466969\\
3.55825	-41.9182415977791\\
3.5585	-42.0317171996788\\
3.55875	-42.1068690268072\\
3.559	-42.1834247034204\\
3.55925	-42.2187952382416\\
3.5595	-42.2640867323573\\
3.55975	-42.4722465639707\\
3.56	-42.7043264677745\\
3.56025	-42.9005761107447\\
3.5605	-43.0999952143769\\
3.56075	-43.111305593646\\
3.561	-43.1021851118782\\
3.56125	-43.0150930685314\\
3.5615	-42.9185547096215\\
3.56175	-42.8881854271253\\
3.562	-42.9226358077431\\
3.56225	-42.9826460670016\\
3.5625	-43.2127844933853\\
3.56275	-43.4289168210384\\
3.563	-43.6118522815342\\
3.56325	-43.7567984305013\\
3.5635	-43.8327553768577\\
3.56375	-43.8681719091935\\
3.564	-43.778736467969\\
3.56425	-43.6609212206332\\
3.5645	-43.5799216327621\\
3.56475	-43.5979603452464\\
3.565	-43.5843024118817\\
3.56525	-43.6081472268482\\
3.5655	-43.6898018064855\\
3.56575	-43.8008028060111\\
3.566	-43.9468415850081\\
3.56625	-44.1372180247538\\
3.5665	-44.3633662749528\\
3.56675	-44.6951894467974\\
3.567	-44.9787280333782\\
3.56725	-45.3154132596816\\
3.5675	-45.6747834959266\\
3.56775	-45.9662940818434\\
3.568	-46.1173806249524\\
3.56825	-45.8777441496218\\
3.5685	-45.490600058407\\
3.56875	-45.1411482240335\\
3.569	-44.9429745658766\\
3.56925	-44.7568512774578\\
3.5695	-44.7265935154474\\
3.56975	-44.8313023912576\\
3.57	-44.804700971699\\
3.57025	-45.0015370001782\\
3.5705	-45.0329082469352\\
3.57075	-44.9697840746049\\
3.571	-44.6929530050665\\
3.57125	-44.2716047741074\\
3.5715	-44.0100079472688\\
3.57175	-43.865526771781\\
3.572	-43.8593033351205\\
3.57225	-43.9117869670138\\
3.5725	-43.8265207657628\\
3.57275	-43.7930567611047\\
3.573	-43.6487474995882\\
3.57325	-43.5779485904687\\
3.5735	-43.4715021605889\\
3.57375	-43.2842860464784\\
3.574	-43.0656567444369\\
3.57425	-43.0166063731468\\
3.5745	-43.09093392376\\
3.57475	-43.2763972005868\\
3.575	-43.4965912568305\\
3.57525	-43.7045072033359\\
3.5755	-43.7930736433049\\
3.57575	-43.7054472965229\\
3.576	-43.5596171986783\\
3.57625	-43.3445650246771\\
3.5765	-43.1515097262374\\
3.57675	-43.0420133448391\\
3.577	-43.0399195350455\\
3.57725	-43.0857692028813\\
3.5775	-43.1528819762725\\
3.57775	-43.3274769253039\\
3.578	-43.5152472053468\\
3.57825	-43.6717989059276\\
3.5785	-43.8398110130047\\
3.57875	-43.7716434575695\\
3.579	-43.801067641637\\
3.57925	-44.0069393434259\\
3.5795	-44.1629976284539\\
3.57975	-44.2317765170408\\
3.58	-44.0652234862187\\
3.58025	-43.8904942417718\\
3.5805	-43.6915050643853\\
3.58075	-43.5701397783942\\
3.581	-43.5429576678464\\
3.58125	-43.6572349970946\\
3.5815	-43.8581143732527\\
3.58175	-44.0796019878269\\
3.582	-44.114708825778\\
3.58225	-43.9673288156376\\
3.5825	-43.5256432034382\\
3.58275	-43.0101210673474\\
3.583	-42.5562898847813\\
3.58325	-42.3032363872803\\
3.5835	-42.2197914532405\\
3.58375	-42.3306656678081\\
3.584	-42.5764406076065\\
3.58425	-42.8307716663526\\
3.5845	-42.9078693305924\\
3.58475	-42.7396921722885\\
3.585	-42.5064270541853\\
3.58525	-42.0965160326687\\
3.5855	-41.7535323986097\\
3.58575	-41.4628580387358\\
3.586	-41.3056426012992\\
3.58625	-41.3239744989997\\
3.5865	-41.4355354218473\\
3.58675	-41.5845010219994\\
3.587	-41.6655576831777\\
3.58725	-41.6630483718276\\
3.5875	-41.5484770246624\\
3.58775	-41.3746007279855\\
3.588	-41.2122153246445\\
3.58825	-41.0540971979462\\
3.5885	-40.943537361729\\
3.58875	-40.8873526897383\\
3.589	-41.0120965555171\\
3.58925	-41.1263912081225\\
3.5895	-41.2438385872028\\
3.58975	-41.266391107538\\
3.59	-41.1866790254265\\
3.59025	-41.0093550069683\\
3.5905	-40.897830831601\\
3.59075	-40.8072921495431\\
3.591	-40.7531950929834\\
3.59125	-40.7843750725412\\
3.5915	-40.8846604516848\\
3.59175	-41.087116352069\\
3.592	-41.3828128130325\\
3.59225	-41.6683568539602\\
3.5925	-41.8381394583191\\
3.59275	-41.8649607207782\\
3.593	-41.6517789653121\\
3.59325	-41.4637795286154\\
3.5935	-41.3319487194249\\
3.59375	-41.2010277430407\\
3.594	-41.1131021754132\\
3.59425	-41.0471643576821\\
3.5945	-41.0393889956516\\
3.59475	-41.055300736693\\
3.595	-41.098062770546\\
3.59525	-41.1444963570542\\
3.5955	-41.1224887450217\\
3.59575	-41.0384262234249\\
3.596	-40.9625488340111\\
3.59625	-40.8708626572639\\
3.5965	-40.8117848603184\\
3.59675	-40.6879090890713\\
3.597	-40.604037720111\\
3.59725	-40.4918902416552\\
3.5975	-40.4188589685185\\
3.59775	-40.3673157114414\\
3.598	-40.3317860296852\\
3.59825	-40.281939883407\\
3.5985	-40.227349182205\\
3.59875	-40.2003477070137\\
3.599	-40.1521984320376\\
3.59925	-40.1571202218639\\
3.5995	-40.1056459325787\\
3.59975	-40.0285600295092\\
3.6	-39.8947954828109\\
};
\addlegendentry{S12}

\end{axis}

\begin{axis}[%
width=5.833in,
height=4.375in,
at={(0in,0in)},
scale only axis,
xmin=0,
xmax=1,
ymin=0,
ymax=1,
axis line style={draw=none},
ticks=none,
axis x line*=bottom,
axis y line*=left
]
\end{axis}
\end{tikzpicture}%}
    \subcaption{Amplitude \emph{vs.} Frequency}\label{fig:amp}
\end{minipage}%
\hfill
\begin{minipage}{0.24\textwidth}
    \centering
    \scalebox{0.29}{% This file was created by matlab2tikz.
%
%The latest updates can be retrieved from
%  http://www.mathworks.com/matlabcentral/fileexchange/22022-matlab2tikz-matlab2tikz
%where you can also make suggestions and rate matlab2tikz.
%
\definecolor{mycolor1}{rgb}{0.00000,0.44700,0.74100}%
\definecolor{mycolor2}{rgb}{0.85000,0.32500,0.09800}%
%
\begin{tikzpicture}

\begin{axis}[%
width=4.521in,
height=3.566in,
at={(0.758in,0.481in)},
scale only axis,
xmin=3.4,
xmax=3.6,
xlabel style={font=\color{white!15!black}},
xlabel={Frequency (GHz)},
ymin=-200,
ymax=200,
ylabel style={font=\color{white!15!black}},
ylabel={Phase ($^\circ$)},
axis background/.style={fill=white},
xmajorgrids,
ymajorgrids,
legend style={legend cell align=left, align=left, draw=white!15!black}
]
\addplot [color=mycolor1, line width=2.0pt]
  table[row sep=crcr]{%
3.4	173.000048554429\\
3.40025	165.90163872167\\
3.4005	159.859930833925\\
3.40075	150.793348390318\\
3.401	139.362482099527\\
3.40125	129.165600445795\\
3.4015	116.064372247279\\
3.40175	108.253220971599\\
3.402	109.686677329128\\
3.40225	108.924207761729\\
3.4025	108.76377666351\\
3.40275	97.878861803686\\
3.403	81.7337761309665\\
3.40325	63.3243303961302\\
3.4035	45.7191182160472\\
3.40375	29.5145969654008\\
3.404	13.2351762100539\\
3.40425	-1.09854974101614\\
3.4045	-12.5133351990404\\
3.40475	-23.4613948671745\\
3.405	-34.0870976023105\\
3.40525	-45.2957862555199\\
3.4055	-59.9090295292999\\
3.40575	-76.5815909518218\\
3.406	-94.7711414124599\\
3.40625	-113.943246192066\\
3.4065	-132.226331569217\\
3.40675	-151.211876796963\\
3.407	-169.496514318995\\
3.40725	171.711633333555\\
3.4075	153.456460192594\\
3.40775	136.68562951938\\
3.408	119.771323954142\\
3.40825	103.686951438009\\
3.4085	87.5142102837987\\
3.40875	70.0488803752439\\
3.409	52.9425162641923\\
3.40925	35.1452817390024\\
3.4095	16.602266751211\\
3.40975	-0.102613349372407\\
3.41	-15.8801092988727\\
3.41025	-32.4675965710451\\
3.4105	-48.38634599492\\
3.41075	-64.5943116425808\\
3.411	-81.4474872192197\\
3.41125	-98.0152810494796\\
3.4115	-114.821442829661\\
3.41175	-132.482637990681\\
3.412	-150.613651470468\\
3.41225	-167.424690868864\\
3.4125	176.399400183131\\
3.41275	160.617029643001\\
3.413	145.558279178449\\
3.41325	131.642187850322\\
3.4135	116.258231855357\\
3.41375	99.9189291926722\\
3.414	82.9138396414578\\
3.41425	66.7441449274966\\
3.4145	49.6142388593034\\
3.41475	33.4055549871104\\
3.415	16.9171595856085\\
3.41525	0.457076884320938\\
3.4155	-15.0275980958726\\
3.41575	-30.2553500997\\
3.416	-45.0114496683183\\
3.41625	-59.8408928550383\\
3.4165	-74.5397123670086\\
3.41675	-89.7276368645655\\
3.417	-103.855992721279\\
3.41725	-118.459825300471\\
3.4175	-132.844949923444\\
3.41775	-147.321693487227\\
3.418	-162.974917810968\\
3.41825	-177.789449670506\\
3.4185	166.908457850756\\
3.41875	152.663266370657\\
3.419	139.382576945968\\
3.41925	126.062282720137\\
3.4195	113.890000554637\\
3.41975	102.090208310944\\
3.42	89.8389652693689\\
3.42025	76.7916275989042\\
3.4205	62.9047629190615\\
3.42075	48.3349313840176\\
3.421	33.5553603576026\\
3.42125	18.9933035573644\\
3.4215	3.87104778851741\\
3.42175	-11.1477180067401\\
3.422	-25.4792876667279\\
3.42225	-39.4262716102277\\
3.4225	-54.1901608644596\\
3.42275	-68.9366369022648\\
3.423	-84.3365206450137\\
3.42325	-99.6417807432999\\
3.4235	-113.757370329693\\
3.42375	-128.300548767404\\
3.424	-141.280203326142\\
3.42425	-154.159218486252\\
3.4245	-166.885336827316\\
3.42475	179.208640983435\\
3.425	164.73359529087\\
3.42525	149.899998298136\\
3.4255	133.133046700604\\
3.42575	116.512618648879\\
3.426	100.394931129911\\
3.42625	85.0089113688714\\
3.4265	71.611196332625\\
3.42675	59.0438434773755\\
3.427	47.4220454091001\\
3.42725	35.2062732403623\\
3.4275	22.7666817499081\\
3.42775	8.56253224086613\\
3.428	-6.69626413289118\\
3.42825	-23.2832391398059\\
3.4285	-39.4783868262393\\
3.42875	-55.8603093932653\\
3.429	-72.0071686235143\\
3.42925	-87.5042115474893\\
3.4295	-103.433973476105\\
3.42975	-118.637425144202\\
3.43	-133.253624205125\\
3.43025	-147.69373313188\\
3.4305	-161.728724375322\\
3.43075	-174.492143549936\\
3.431	173.436768648921\\
3.43125	161.651549020803\\
3.4315	150.297527293313\\
3.43175	137.079115050669\\
3.432	121.504019430519\\
3.43225	105.064341081403\\
3.4325	88.1912543020461\\
3.43275	70.2176411901588\\
3.433	52.8903170013428\\
3.43325	37.4196635748103\\
3.4335	25.5118961938276\\
3.43375	14.2961951999823\\
3.434	6.82473124497159\\
3.43425	-2.95378773814755\\
3.4345	-15.4045444898118\\
3.43475	-31.3393727082517\\
3.435	-49.1325116347242\\
3.43525	-69.6822491242133\\
3.4355	-90.4636478922377\\
3.43575	-111.437688891054\\
3.436	-132.091440611779\\
3.43625	-150.992480463033\\
3.4365	-169.029224922358\\
3.43675	176.290649963288\\
3.437	162.857674015553\\
3.43725	149.415409778569\\
3.4375	135.329196584167\\
3.43775	118.90799925569\\
3.438	102.123958175934\\
3.43825	83.8636500062582\\
3.4385	66.4321909355555\\
3.43875	46.7903626179899\\
3.439	28.227184925361\\
3.43925	8.26860166861185\\
3.4395	-10.7420932507726\\
3.43975	-28.3440532387923\\
3.44	-46.233362705938\\
3.44025	-65.1092470260223\\
3.4405	-83.0954077500133\\
3.44075	-100.849383899488\\
3.441	-117.93653910621\\
3.44125	-133.50383501995\\
3.4415	-146.963319591339\\
3.44175	-158.895795772392\\
3.442	-175.347249901388\\
3.44225	171.148799261557\\
3.4425	155.295436290138\\
3.44275	139.016759070071\\
3.443	121.892276345946\\
3.44325	105.713053766229\\
3.4435	87.1273074820435\\
3.44375	67.2840147938294\\
3.444	50.2144031738267\\
3.44425	35.8082728895193\\
3.4445	23.3486863386294\\
3.44475	14.9082826442242\\
3.445	6.61184058316685\\
3.44525	8.38768614744744\\
3.4455	11.0988788746922\\
3.44575	10.3058661355741\\
3.446	-2.26875624509403\\
3.44625	-15.9743731054474\\
3.4465	-34.0681211898793\\
3.44675	-52.7312867591775\\
3.447	-68.2079529068774\\
3.44725	-85.4886311388942\\
3.4475	-98.1236867384916\\
3.44775	-111.412364942976\\
3.448	-124.25727969984\\
3.44825	-133.798426250329\\
3.4485	-142.104483734994\\
3.44875	-149.900155347301\\
3.449	-164.718227472616\\
3.44925	-179.739994092675\\
3.4495	162.417825556328\\
3.44975	146.257352495267\\
3.45	130.783817317834\\
3.45025	116.671171066436\\
3.4505	100.350861218653\\
3.45075	86.8043833778144\\
3.451	74.6821221582645\\
3.45125	67.3687274805061\\
3.4515	60.4673558281836\\
3.45175	51.2316023421393\\
3.452	40.9061960887615\\
3.45225	27.4253921753226\\
3.4525	11.7149165427448\\
3.45275	-7.17176332699657\\
3.453	-25.1750125377915\\
3.45325	-41.4500511618434\\
3.4535	-57.8423028038739\\
3.45375	-69.6648584895823\\
3.454	-79.6129074404841\\
3.45425	-87.5914200715235\\
3.4545	-96.9056734860037\\
3.45475	-107.07127761547\\
3.455	-118.077398333461\\
3.45525	-131.598044035608\\
3.4555	-145.250366100278\\
3.45575	-158.884775079794\\
3.456	-173.916152289815\\
3.45625	170.440815801384\\
3.4565	154.432717618132\\
3.45675	138.675620408354\\
3.457	121.829722535415\\
3.45725	106.198537547224\\
3.4575	89.0371553884142\\
3.45775	74.2978891215614\\
3.458	60.4146945487711\\
3.45825	46.6027930176584\\
3.4585	32.6199133449636\\
3.45875	19.0642388489395\\
3.459	4.59797794722243\\
3.45925	-11.8193044255418\\
3.4595	-29.6212072420846\\
3.45975	-48.8115470575945\\
3.46	-68.216622075599\\
3.46025	-87.9929842409548\\
3.4605	-106.986639588466\\
3.46075	-126.168821259212\\
3.461	-142.925479309827\\
3.46125	-158.514618669074\\
3.4615	-172.226596715938\\
3.46175	174.589783947911\\
3.462	159.863909270467\\
3.46225	145.853407240515\\
3.4625	129.209418136046\\
3.46275	114.880445334976\\
3.463	99.2052387284656\\
3.46325	82.0783900702589\\
3.4635	66.5438128245193\\
3.46375	50.1067082246819\\
3.464	37.0419796847156\\
3.46425	23.4447378092038\\
3.4645	8.09659018336952\\
3.46475	-7.23369853881957\\
3.465	-22.9831009100876\\
3.46525	-38.802147332072\\
3.4655	-52.6164666949131\\
3.46575	-64.8991039544884\\
3.466	-77.9701967314038\\
3.46625	-90.2275258162572\\
3.4665	-104.501564395663\\
3.46675	-120.109602096497\\
3.467	-134.762736070822\\
3.46725	-149.639278608963\\
3.4675	-165.964558004492\\
3.46775	177.623544369196\\
3.468	162.256515198781\\
3.46825	146.740006624724\\
3.4685	132.404796975932\\
3.46875	117.068235465673\\
3.469	101.627900193293\\
3.46925	85.4016849202681\\
3.4695	69.3569520129395\\
3.46975	52.6078097775065\\
3.47	36.1662825753959\\
3.47025	20.3376082367549\\
3.4705	6.83203360597204\\
3.47075	-7.33569591990232\\
3.471	-21.087789857668\\
3.47125	-35.3194045080662\\
3.4715	-49.836961544812\\
3.47175	-64.7763136666289\\
3.472	-79.7013665695138\\
3.47225	-95.0085360075053\\
3.4725	-108.304884996159\\
3.47275	-120.385206982124\\
3.473	-127.70452065485\\
3.47325	-135.030917321164\\
3.4735	-142.602335981657\\
3.47375	-153.404954286017\\
3.474	-166.924609599278\\
3.47425	178.503835128822\\
3.4745	161.500019125158\\
3.47475	145.409463885958\\
3.475	129.550662659293\\
3.47525	114.340096707138\\
3.4755	101.149603279535\\
3.47575	89.2499578191463\\
3.476	79.5490251855776\\
3.47625	68.1491961183289\\
3.4765	54.9475549981641\\
3.47675	41.2245655474246\\
3.477	26.2909147256694\\
3.47725	10.229977938666\\
3.4775	-5.63974830271824\\
3.47775	-21.7992043860323\\
3.478	-36.227695964579\\
3.47825	-50.2532846400767\\
3.4785	-63.6533398112788\\
3.47875	-77.1217764877766\\
3.479	-90.422619876873\\
3.47925	-103.559150532996\\
3.4795	-116.771068585487\\
3.47975	-130.392944522256\\
3.48	-144.01270541354\\
3.48025	-157.782419722304\\
3.4805	-172.090728801575\\
3.48075	173.691126509875\\
3.481	158.409638733845\\
3.48125	143.522816825248\\
3.4815	128.93464016843\\
3.48175	113.517384799192\\
3.482	98.9715991071285\\
3.48225	84.2067782174926\\
3.4825	69.5549767126016\\
3.48275	54.676528789569\\
3.483	40.1857894230323\\
3.48325	25.22267255495\\
3.4835	10.3398355281753\\
3.48375	-5.37483090765601\\
3.484	-21.1590811048694\\
3.48425	-37.3939661358854\\
3.4845	-52.7026602627364\\
3.48475	-68.1527101995428\\
3.485	-82.6816137698789\\
3.48525	-97.3392085625284\\
3.4855	-111.944917805508\\
3.48575	-125.58145908184\\
3.486	-138.876062724177\\
3.48625	-152.567885984002\\
3.4865	-166.255285619555\\
3.48675	179.797869595963\\
3.487	166.241650916708\\
3.48725	152.287987453649\\
3.4875	138.379985407037\\
3.48775	123.900033833639\\
3.488	108.652759278644\\
3.48825	93.1758835287522\\
3.4885	78.6430787409618\\
3.48875	64.0571309640279\\
3.489	49.5064319344352\\
3.48925	35.1797175877785\\
3.4895	20.7202280976144\\
3.48975	5.8408824793158\\
3.49	-8.41697766355503\\
3.49025	-22.5031991716443\\
3.4905	-37.1876162664579\\
3.49075	-52.2057676390611\\
3.491	-67.4992379399587\\
3.49125	-83.6831715748593\\
3.4915	-98.7249196551187\\
3.49175	-114.588808784633\\
3.492	-130.935847904217\\
3.49225	-147.217124477084\\
3.4925	-164.348626714775\\
3.49275	179.215416719681\\
3.493	162.193939296318\\
3.49325	144.551440067824\\
3.4935	127.851318106897\\
3.49375	111.500572878217\\
3.494	95.5243519961908\\
3.49425	80.9858546574775\\
3.4945	64.7777963606625\\
3.49475	49.9706180748181\\
3.495	34.1462688066512\\
3.49525	18.08647681487\\
3.4955	2.01314158785181\\
3.49575	-15.3806938136039\\
3.496	-33.8288923043028\\
3.49625	-52.2002361064304\\
3.4965	-69.8709208557411\\
3.49675	-86.9264081553598\\
3.497	-103.098997143995\\
3.49725	-118.090085634772\\
3.4975	-132.289183144335\\
3.49775	-146.288209511002\\
3.498	-160.298998986226\\
3.49825	-174.628711424963\\
3.4985	170.146595990163\\
3.49875	153.155677193319\\
3.499	135.695539775215\\
3.49925	117.731742981204\\
3.4995	100.113350937136\\
3.49975	83.3633951544854\\
3.5	67.2069748675425\\
3.50025	52.5027460190588\\
3.5005	38.8912202913535\\
3.50075	26.322170043728\\
3.501	13.3652173131262\\
3.50125	0.433531264179654\\
3.5015	-13.4266224173492\\
3.50175	-26.5717360328165\\
3.502	-41.2599934008706\\
3.50225	-55.8227456847695\\
3.5025	-71.2116913660494\\
3.50275	-86.4212933320358\\
3.503	-101.877391658841\\
3.50325	-116.900532244204\\
3.5035	-132.40879511475\\
3.50375	-147.384682273708\\
3.504	-162.154599367389\\
3.50425	-176.179875591129\\
3.5045	170.553040051884\\
3.50475	157.155377927972\\
3.505	143.624389936772\\
3.50525	129.540120281424\\
3.5055	114.893639394003\\
3.50575	99.0895310921445\\
3.506	83.5856914254806\\
3.50625	67.2027727823648\\
3.5065	50.4369390060373\\
3.50675	35.553566524394\\
3.507	21.5094281838759\\
3.50725	8.10579172644154\\
3.5075	-2.29883825547281\\
3.50775	-13.0443605506117\\
3.508	-24.6658438505955\\
3.50825	-35.7763693901101\\
3.5085	-48.0728208273872\\
3.50875	-61.1731190549958\\
3.509	-74.1957148339461\\
3.50925	-86.5986521922799\\
3.5095	-99.4723435189173\\
3.50975	-110.848715710597\\
3.51	-121.7216797362\\
3.51025	-133.628380844626\\
3.5105	-145.553822865463\\
3.51075	-157.780864128778\\
3.511	-170.203550521437\\
3.51125	178.090268904567\\
3.5115	167.80287455435\\
3.51175	157.647822096307\\
3.512	146.926795511888\\
3.51225	134.922664652353\\
3.5125	120.641460964151\\
3.51275	106.000080073489\\
3.513	90.7788591111227\\
3.51325	75.2544444813448\\
3.5135	59.671190639137\\
3.51375	45.0306972630002\\
3.514	30.7838189693583\\
3.51425	17.8115350743118\\
3.5145	5.51537742483956\\
3.51475	-7.60020247163197\\
3.515	-22.6376988015466\\
3.51525	-37.9965449944629\\
3.5155	-53.9303221399408\\
3.51575	-70.487829242143\\
3.516	-86.3284985380169\\
3.51625	-101.67984963309\\
3.5165	-115.437475153205\\
3.51675	-128.798116000636\\
3.517	-140.014530665233\\
3.51725	-150.948291979077\\
3.5175	-161.269438954383\\
3.51775	-173.455285456219\\
3.518	173.338789549535\\
3.51825	159.04361505372\\
3.5185	144.419304933761\\
3.51875	129.287795633622\\
3.519	113.63339707102\\
3.51925	100.066823056483\\
3.5195	88.7289636950002\\
3.51975	76.9726844461774\\
3.52	65.9264214957448\\
3.52025	52.4351847874765\\
3.5205	38.8614579512971\\
3.52075	23.3524728152037\\
3.521	7.80431275580518\\
3.52125	-9.00885173393957\\
3.5215	-23.7565791136133\\
3.52175	-36.5028939096589\\
3.522	-44.8570776047887\\
3.52225	-51.49617759336\\
3.5225	-58.4106920636761\\
3.52275	-68.3830292803776\\
3.523	-83.524821013311\\
3.52325	-100.846592208699\\
3.5235	-119.138297371647\\
3.52375	-138.427018229926\\
3.524	-157.539385462703\\
3.52425	-174.730568840626\\
3.5245	170.303741893769\\
3.52475	159.544847320207\\
3.525	154.379492727346\\
3.52525	147.024844082774\\
3.5255	135.288658570989\\
3.52575	119.757659806346\\
3.526	101.499007355214\\
3.52625	81.7051938776326\\
3.5265	60.7474525897539\\
3.52675	37.4163432143683\\
3.527	11.620867466009\\
3.52725	-11.8717345771823\\
3.5275	-38.4755812805486\\
3.52775	-64.0963840375942\\
3.528	-88.5658843457733\\
3.52825	-112.241832876291\\
3.5285	-135.421392850677\\
3.52875	-157.288037286892\\
3.529	-176.006638296625\\
3.52925	167.783166903668\\
3.5295	154.672465428705\\
3.52975	140.972693685426\\
3.53	124.516266588005\\
3.53025	103.745784595094\\
3.5305	81.509113625118\\
3.53075	60.6518567867832\\
3.531	39.5796004126697\\
3.53125	21.069537852824\\
3.5315	3.9134145577557\\
3.53175	-10.9930185369916\\
3.532	-21.8431013252577\\
3.53225	-29.9776853693777\\
3.5325	-37.0069805550094\\
3.53275	-46.6678305563691\\
3.533	-58.0792329943707\\
3.53325	-72.6840547850284\\
3.5335	-88.7741485958243\\
3.53375	-105.234912755364\\
3.534	-121.505775287705\\
3.53425	-136.89623072432\\
3.5345	-152.29074688471\\
3.53475	-167.425712888379\\
3.535	176.238514403182\\
3.53525	160.337217986553\\
3.5355	143.939076788641\\
3.53575	128.42451874861\\
3.536	113.347398620357\\
3.53625	99.0835042698188\\
3.5365	85.4483152047767\\
3.53675	72.0102825130444\\
3.537	57.4047388866822\\
3.53725	44.876314428877\\
3.5375	30.8133709711661\\
3.53775	14.830374673735\\
3.538	-1.27709614810263\\
3.53825	-16.5687911321705\\
3.5385	-32.2278338099817\\
3.53875	-46.2161287339521\\
3.539	-60.1391263100869\\
3.53925	-72.9820405045233\\
3.5395	-86.1043493993653\\
3.53975	-98.5789712855459\\
3.54	-110.596332446254\\
3.54025	-122.638262381266\\
3.5405	-134.047814871987\\
3.54075	-145.556229667135\\
3.541	-156.579910889019\\
3.54125	-168.081844353754\\
3.5415	-179.127572134553\\
3.54175	168.555262917577\\
3.542	155.985525148147\\
3.54225	143.465953864027\\
3.5425	130.611967483138\\
3.54275	118.84423101507\\
3.543	106.122547505224\\
3.54325	94.8087340502195\\
3.5435	81.9488213650939\\
3.54375	69.3577309685993\\
3.544	56.8159096876153\\
3.54425	44.450613171566\\
3.5445	30.3160407084884\\
3.54475	17.8280515438086\\
3.545	5.79803983802171\\
3.54525	-4.93406336244206\\
3.5455	-16.1478725393626\\
3.54575	-25.6457650619097\\
3.546	-36.9732056439152\\
3.54625	-49.1524037476311\\
3.5465	-63.6248613650787\\
3.54675	-78.0404350621086\\
3.547	-92.7132088175903\\
3.54725	-107.527773758178\\
3.5475	-121.145115931844\\
3.54775	-134.032204243748\\
3.548	-146.815776476214\\
3.54825	-159.4525672233\\
3.5485	-172.823795225272\\
3.54875	172.878248053713\\
3.549	157.595947681121\\
3.54925	141.836893002407\\
3.5495	125.312167989026\\
3.54975	108.458270663785\\
3.55	92.9821952051093\\
3.55025	78.2158328372186\\
3.5505	65.3002799121097\\
3.55075	53.3642904946494\\
3.551	41.8089522427607\\
3.55125	29.7341768456426\\
3.5515	16.3762106898408\\
3.55175	1.37665630442696\\
3.552	-14.6372318744271\\
3.55225	-31.184401568528\\
3.5525	-47.2569633504733\\
3.55275	-63.2037879153187\\
3.553	-77.9191446429346\\
3.55325	-90.6325377200271\\
3.5535	-101.578416144992\\
3.55375	-112.775783346337\\
3.554	-123.842531946226\\
3.55425	-136.28856229123\\
3.5545	-149.273207772428\\
3.55475	-163.954961827435\\
3.555	179.461670933842\\
3.55525	163.180257948807\\
3.5555	146.196320788004\\
3.55575	130.325353672416\\
3.556	115.645253229545\\
3.55625	102.469432146271\\
3.5565	90.0992814383981\\
3.55675	77.952231205009\\
3.557	65.0648332688868\\
3.55725	51.8317773120012\\
3.5575	38.305499639065\\
3.55775	23.5329461035372\\
3.558	7.97155842149409\\
3.55825	-7.65912701803618\\
3.5585	-22.5009993179288\\
3.55875	-37.155095542666\\
3.559	-51.6725973484358\\
3.55925	-66.5747236809727\\
3.5595	-82.7904220794355\\
3.55975	-97.7019659945352\\
3.56	-113.253408280171\\
3.56025	-126.766534328715\\
3.5605	-138.303205211169\\
3.56075	-151.987787105791\\
3.561	-164.694246268285\\
3.56125	-179.300022059786\\
3.5615	165.545933131902\\
3.56175	148.759794206171\\
3.562	131.937008409349\\
3.56225	114.480338095179\\
3.5625	98.2575911967507\\
3.56275	81.8367912268534\\
3.563	66.2371051243857\\
3.56325	52.523502820133\\
3.5635	37.7505917784335\\
3.56375	23.0938538361394\\
3.564	8.38579749066954\\
3.56425	-8.43300931161481\\
3.5645	-25.8094353007148\\
3.56475	-44.9585829837691\\
3.565	-63.5622162666116\\
3.56525	-82.0492560342264\\
3.5655	-101.848565614495\\
3.56575	-120.693753675815\\
3.566	-141.096851946086\\
3.56625	-160.862631693766\\
3.5665	-179.62325693391\\
3.56675	161.454209918903\\
3.567	143.65129736749\\
3.56725	126.714309243618\\
3.5675	111.82553927614\\
3.56775	98.0962005657821\\
3.568	87.187894904394\\
3.56825	77.6230246320531\\
3.5685	64.3630875169691\\
3.56875	48.4577377751366\\
3.569	29.4670005690659\\
3.56925	11.2246780939549\\
3.5695	-6.9907587146482\\
3.56975	-23.2727441260814\\
3.57	-38.5807449942477\\
3.57025	-54.1683119334548\\
3.5705	-68.1570689626369\\
3.57075	-79.2981874142661\\
3.571	-91.5929682102406\\
3.57125	-106.859016854298\\
3.5715	-123.512368999641\\
3.57175	-142.590831804251\\
3.572	-159.539795550376\\
3.57225	-173.914302928736\\
3.5725	171.014435728783\\
3.57275	156.201333695955\\
3.573	142.094845738447\\
3.57325	127.795031420637\\
3.5735	113.034640106042\\
3.57375	98.4773723087593\\
3.574	80.9843560268097\\
3.57425	64.0033316302011\\
3.5745	45.9268263753624\\
3.57475	29.2866090389438\\
3.575	15.0584185650608\\
3.57525	2.37668289008138\\
3.5755	-8.6546854016902\\
3.57575	-21.0441693291286\\
3.576	-32.6371264255603\\
3.57625	-46.5128001632956\\
3.5765	-60.7381190451778\\
3.57675	-77.1854127850648\\
3.577	-93.8921569450698\\
3.57725	-110.132110298058\\
3.5775	-125.105787243987\\
3.57775	-140.114437184792\\
3.578	-153.914196443565\\
3.57825	-166.390003040327\\
3.5785	-179.547365304\\
3.57875	167.8530090953\\
3.579	154.372518978039\\
3.57925	140.482235613882\\
3.5795	130.834359232695\\
3.57975	119.360901843462\\
3.58	109.366647109407\\
3.58025	98.0525830504025\\
3.5805	83.4103068280454\\
3.58075	67.8105664278899\\
3.581	53.4322623434563\\
3.58125	38.9013668302673\\
3.5815	25.0160977301\\
3.58175	14.6108732310207\\
3.582	5.90952017621428\\
3.58225	-1.48126823370042\\
3.5825	-11.855525041838\\
3.58275	-25.2975518883566\\
3.583	-41.7796051953919\\
3.58325	-59.8361580276222\\
3.5835	-78.5485191856065\\
3.58375	-96.7590661320624\\
3.584	-111.739133988632\\
3.58425	-124.51652369663\\
3.5845	-134.903953425378\\
3.58475	-145.431746385316\\
3.585	-156.37104565714\\
3.58525	-169.579967913961\\
3.5855	174.890997598458\\
3.58575	157.396827383299\\
3.586	138.913445601787\\
3.58625	120.626473149465\\
3.5865	103.489847579909\\
3.58675	88.6787257867389\\
3.587	74.5449797310761\\
3.58725	61.1966234084359\\
3.5875	47.9441826766831\\
3.58775	33.5084801931362\\
3.588	18.466029037687\\
3.58825	1.97461998550141\\
3.5885	-14.8300449370679\\
3.58875	-32.8148706883941\\
3.589	-49.9347281861156\\
3.58925	-66.0600535033775\\
3.5895	-80.1500521145937\\
3.58975	-93.4432367888602\\
3.59	-107.708343714007\\
3.59025	-122.637095096554\\
3.5905	-138.603685263227\\
3.59075	-155.660931017825\\
3.591	-173.214872414871\\
3.59125	168.80505511891\\
3.5915	151.038247714131\\
3.59175	134.390046709324\\
3.592	118.369519146803\\
3.59225	104.680495858814\\
3.5925	93.8038303177449\\
3.59275	82.1394142410554\\
3.593	69.9041916395151\\
3.59325	55.3314514818674\\
3.5935	40.4332126349487\\
3.59375	23.8359003857225\\
3.594	7.75385476796541\\
3.59425	-8.18620962891693\\
3.5945	-24.5954583347253\\
3.59475	-41.0264011582589\\
3.595	-56.8453600240633\\
3.59525	-71.1613414064403\\
3.5955	-85.4708203223409\\
3.59575	-100.273082086677\\
3.596	-115.024025946716\\
3.59625	-130.840518728668\\
3.5965	-145.877153020323\\
3.59675	-161.434070357594\\
3.597	-177.05009078638\\
3.59725	166.429438978689\\
3.5975	150.437873561865\\
3.59775	134.453252407042\\
3.598	118.530461790572\\
3.59825	102.732634455712\\
3.5985	87.0034763997185\\
3.59875	70.9228956298116\\
3.599	54.819774304731\\
3.59925	39.4875256441435\\
3.5995	24.756546838429\\
3.59975	9.47878411691982\\
3.6	-6.19630402008241\\
};
\addlegendentry{S21}

\addplot [color=mycolor2, dashed, line width=1.5pt]
  table[row sep=crcr]{%
3.4	172.792264895113\\
3.40025	166.114176259761\\
3.4005	160.022308483348\\
3.40075	150.771684211975\\
3.401	139.709269140375\\
3.40125	128.110451126765\\
3.4015	115.75945125577\\
3.40175	111.16460202902\\
3.402	107.489323061446\\
3.40225	110.738590235911\\
3.4025	108.174790670399\\
3.40275	97.7328190416587\\
3.403	82.5902346406083\\
3.40325	64.7922446501575\\
3.4035	46.6129053277534\\
3.40375	29.8099809697271\\
3.404	13.5445086235575\\
3.40425	-1.14717172752084\\
3.4045	-12.4914086623659\\
3.40475	-23.2636971435841\\
3.405	-33.6971178971777\\
3.40525	-45.000323042917\\
3.4055	-59.3147241481974\\
3.40575	-77.0808023806127\\
3.406	-94.9091023920398\\
3.40625	-113.511840024089\\
3.4065	-131.967320897974\\
3.40675	-150.883247222456\\
3.407	-169.02181396714\\
3.40725	172.899633358603\\
3.4075	154.436404428828\\
3.40775	137.378003921888\\
3.408	120.769070403994\\
3.40825	105.009800315857\\
3.4085	88.1488031938205\\
3.40875	71.0451496004227\\
3.409	54.0156567668129\\
3.40925	35.6186497811308\\
3.4095	17.4890370583076\\
3.40975	0.602185816409668\\
3.41	-15.5278651038406\\
3.41025	-31.7292256031871\\
3.4105	-47.6625026045406\\
3.41075	-64.2599794830143\\
3.411	-80.5283891404866\\
3.41125	-97.1956705521983\\
3.4115	-114.686950217382\\
3.41175	-132.166892553744\\
3.412	-150.182905394161\\
3.41225	-167.376236716137\\
3.4125	176.057893005758\\
3.41275	160.641762372594\\
3.413	145.629918411118\\
3.41325	130.907592546258\\
3.4135	115.821143962028\\
3.41375	99.1599435509107\\
3.414	82.6650827799032\\
3.41425	65.7909084103747\\
3.4145	49.3607432194167\\
3.41475	32.8150362743814\\
3.415	16.0577513591437\\
3.41525	0.437170010663082\\
3.4155	-15.4970368606436\\
3.41575	-30.9725330468359\\
3.416	-45.4341767136284\\
3.41625	-59.9333100026309\\
3.4165	-75.3304164452217\\
3.41675	-89.3410651679666\\
3.417	-103.803344891867\\
3.41725	-117.96935216041\\
3.4175	-132.686579803022\\
3.41775	-147.296104448943\\
3.418	-162.799376892137\\
3.41825	-178.319997802626\\
3.4185	167.69572559029\\
3.41875	153.638629176611\\
3.419	140.002063128975\\
3.41925	126.263911011999\\
3.4195	114.598124818976\\
3.41975	101.998457204635\\
3.42	90.8406740622166\\
3.42025	77.8268725815436\\
3.4205	64.3954794410639\\
3.42075	49.3632681631539\\
3.421	34.8134465825047\\
3.42125	20.1572283171027\\
3.4215	4.32302054120995\\
3.42175	-10.1571677297864\\
3.422	-24.5372874704402\\
3.42225	-38.9250862802783\\
3.4225	-53.254502713376\\
3.42275	-68.0985403215527\\
3.423	-83.8099051737517\\
3.42325	-98.7615574233289\\
3.4235	-113.289691061142\\
3.42375	-127.952949722829\\
3.424	-141.200981241033\\
3.42425	-153.273336402535\\
3.4245	-166.754439942383\\
3.42475	-179.898225953742\\
3.425	165.189142197953\\
3.42525	149.635719795808\\
3.4255	133.15430724618\\
3.42575	116.719299116465\\
3.426	100.141852117382\\
3.42625	85.0495982188814\\
3.4265	71.4104294453043\\
3.42675	58.8154016034093\\
3.427	47.4339795077866\\
3.42725	35.5848961712424\\
3.4275	21.7445252993133\\
3.42775	8.4416368318239\\
3.428	-6.7617835008093\\
3.42825	-23.2400725567857\\
3.4285	-39.1445632256369\\
3.42875	-55.3811826520397\\
3.429	-71.7378022432944\\
3.42925	-87.1271475030805\\
3.4295	-102.799075304638\\
3.42975	-117.867947205057\\
3.43	-132.300324557366\\
3.43025	-146.618269823054\\
3.4305	-160.654588539555\\
3.43075	-173.637077003652\\
3.431	174.514623461792\\
3.43125	163.794386152171\\
3.4315	150.731267653965\\
3.43175	137.803348740869\\
3.432	123.271025744054\\
3.43225	107.051326510809\\
3.4325	89.3462363397997\\
3.43275	71.8713039449884\\
3.433	54.2779223656745\\
3.43325	39.5692679208608\\
3.4335	25.9318348821185\\
3.43375	15.7842010812786\\
3.434	7.76493152552197\\
3.43425	-1.37096390054038\\
3.4345	-13.3515086317388\\
3.43475	-29.7094324437849\\
3.435	-49.0159602667695\\
3.43525	-68.682592353894\\
3.4355	-89.0609909270232\\
3.43575	-111.2722608736\\
3.436	-131.033683111805\\
3.43625	-151.379784922558\\
3.4365	-168.033272052552\\
3.43675	176.15213406926\\
3.437	164.132981211201\\
3.43725	150.373221867868\\
3.4375	135.690707855898\\
3.43775	120.275650321952\\
3.438	102.064370429237\\
3.43825	85.1644799906148\\
3.4385	66.1815705537222\\
3.43875	47.6191933524327\\
3.439	27.996396241058\\
3.43925	8.98033429653141\\
3.4395	-10.6371728912525\\
3.43975	-27.3195989179045\\
3.44	-45.3280411273969\\
3.44025	-64.0062130695611\\
3.4405	-82.7051911694913\\
3.44075	-101.511607132822\\
3.441	-116.284233301676\\
3.44125	-132.00463785727\\
3.4415	-146.555796830748\\
3.44175	-159.412655860394\\
3.442	-172.754765477645\\
3.44225	171.178157963153\\
3.4425	156.112492730766\\
3.44275	140.995160280325\\
3.443	123.853532154614\\
3.44325	107.592102449706\\
3.4435	89.2681874017641\\
3.44375	70.8260360568853\\
3.444	52.2773159341968\\
3.44425	35.2082428604844\\
3.4445	25.3218284828398\\
3.44475	17.057594286017\\
3.445	8.5091934471269\\
3.44525	9.94798707994525\\
3.4455	14.6518554185303\\
3.44575	13.4265979787889\\
3.446	2.3046633937159\\
3.44625	-14.7246281301032\\
3.4465	-33.0237085968212\\
3.44675	-50.6838489380861\\
3.447	-67.925745121407\\
3.44725	-83.1357973539695\\
3.4475	-97.5351245699656\\
3.44775	-109.843857241087\\
3.448	-123.310035119586\\
3.44825	-130.157905602638\\
3.4485	-139.033352457615\\
3.44875	-150.107999418407\\
3.449	-164.787415919437\\
3.44925	179.446019642013\\
3.4495	163.766465736558\\
3.44975	148.680413847535\\
3.45	130.815820633464\\
3.45025	115.049309411738\\
3.4505	100.448453608985\\
3.45075	87.6332388952337\\
3.451	74.0816445381582\\
3.45125	66.7523898869122\\
3.4515	59.509266682132\\
3.45175	53.4096974923005\\
3.452	41.2420604022362\\
3.45225	26.3500850191374\\
3.4525	10.8604471209958\\
3.45275	-5.88927705589923\\
3.453	-24.6181825321794\\
3.45325	-40.99237760095\\
3.4535	-55.9996006008129\\
3.45375	-68.8866710835722\\
3.454	-77.9313121965979\\
3.45425	-87.3087158047856\\
3.4545	-96.8250321603299\\
3.45475	-106.73713920202\\
3.455	-117.912213520494\\
3.45525	-131.380183344619\\
3.4555	-145.281290077587\\
3.45575	-157.791299962777\\
3.456	-173.895100954895\\
3.45625	172.13927999604\\
3.4565	155.692013643234\\
3.45675	139.81913476732\\
3.457	122.63360904091\\
3.45725	106.051909986295\\
3.4575	90.1221091771685\\
3.45775	74.7890854781862\\
3.458	60.1229438410156\\
3.45825	46.9431351996491\\
3.4585	34.0478507512678\\
3.45875	20.1905781255056\\
3.459	4.61915706149659\\
3.45925	-11.899028245311\\
3.4595	-29.4779333414303\\
3.45975	-48.8742150920261\\
3.46	-68.5566321550566\\
3.46025	-88.1506954640069\\
3.4605	-107.820926768428\\
3.46075	-125.622453659621\\
3.461	-142.979137145525\\
3.46125	-158.751830923856\\
3.4615	-172.182075593011\\
3.46175	174.85101623773\\
3.462	160.528882172716\\
3.46225	145.212849019234\\
3.4625	130.464735669026\\
3.46275	114.724879711146\\
3.463	98.8157767518138\\
3.46325	81.6896206029659\\
3.4635	66.285755578478\\
3.46375	50.9329244527271\\
3.464	35.9810060867478\\
3.46425	23.5681478246117\\
3.4645	8.66993760370116\\
3.46475	-7.01405216445424\\
3.465	-23.5227683726345\\
3.46525	-37.6978878787379\\
3.4655	-52.3915220135415\\
3.46575	-65.4216084257388\\
3.466	-77.7861066712225\\
3.46625	-89.8011394529857\\
3.4665	-104.085501980636\\
3.46675	-118.408569523536\\
3.467	-133.093703234252\\
3.46725	-148.73254388115\\
3.4675	-165.298431906705\\
3.46775	179.070518247503\\
3.468	162.675074101867\\
3.46825	148.309795760155\\
3.4685	133.382837714983\\
3.46875	117.853060361875\\
3.469	102.827850513513\\
3.46925	86.2542908834678\\
3.4695	69.8624044197285\\
3.46975	53.2963433294927\\
3.47	36.8278165584684\\
3.47025	21.9394421679135\\
3.4705	8.352758537272\\
3.47075	-6.08829845571928\\
3.471	-19.647666047129\\
3.47125	-34.7050126537972\\
3.4715	-49.5094717377835\\
3.47175	-64.4188345298874\\
3.472	-79.488973778099\\
3.47225	-94.6454372702839\\
3.4725	-108.646560113433\\
3.47275	-120.193638133544\\
3.473	-128.440410754237\\
3.47325	-135.015725361355\\
3.4735	-142.129889338948\\
3.47375	-153.967478969668\\
3.474	-167.552320668863\\
3.47425	177.785807196734\\
3.4745	161.457118089634\\
3.47475	144.899326157105\\
3.475	128.876247207701\\
3.47525	114.115146778093\\
3.4755	101.234800143771\\
3.47575	89.7310902114478\\
3.476	79.5979606513278\\
3.47625	68.8318045850166\\
3.4765	55.7275330607732\\
3.47675	41.1678572006804\\
3.477	26.6290619379944\\
3.47725	10.8293771751727\\
3.4775	-5.12090705614158\\
3.47775	-20.6576708233671\\
3.478	-35.8485027004083\\
3.47825	-49.3653063293714\\
3.4785	-62.7908801240645\\
3.47875	-76.5032513768034\\
3.479	-89.32689418425\\
3.47925	-102.729597302678\\
3.4795	-116.350663631658\\
3.47975	-129.136068436328\\
3.48	-142.785257844958\\
3.48025	-156.942408697203\\
3.4805	-171.103681787897\\
3.48075	174.489731696763\\
3.481	159.417669809374\\
3.48125	144.220223527517\\
3.4815	129.810592295351\\
3.48175	114.879627647331\\
3.482	99.5895902136069\\
3.48225	84.9866502095591\\
3.4825	70.2637027999617\\
3.48275	55.5263099528371\\
3.483	40.4278783103443\\
3.48325	25.8559105609323\\
3.4835	10.7179096564137\\
3.48375	-4.90541573625031\\
3.484	-21.0278406693135\\
3.48425	-36.8300996382413\\
3.4845	-52.7779699185451\\
3.48475	-67.2803064541342\\
3.485	-82.8729871513352\\
3.48525	-97.77144407309\\
3.4855	-112.090402376179\\
3.48575	-126.06581167947\\
3.486	-139.358779111455\\
3.48625	-153.154506196824\\
3.4865	-166.705225459308\\
3.48675	179.430207673537\\
3.487	165.644782730008\\
3.48725	151.703164303061\\
3.4875	138.153810154621\\
3.48775	123.461379826877\\
3.488	108.304186497856\\
3.48825	93.0998961848459\\
3.4885	78.6855624250216\\
3.48875	63.883064233201\\
3.489	49.4923127310612\\
3.48925	34.993666264088\\
3.4895	20.529793464388\\
3.48975	5.97568374440118\\
3.49	-8.32176579615517\\
3.49025	-22.0971192161503\\
3.4905	-37.0080670540494\\
3.49075	-51.4935051098415\\
3.491	-67.1749054442144\\
3.49125	-82.4390865281517\\
3.4915	-97.9646637610015\\
3.49175	-113.894510437353\\
3.492	-129.638081581743\\
3.49225	-146.089154151084\\
3.4925	-162.676444041041\\
3.49275	179.846669078932\\
3.493	162.968314618022\\
3.49325	145.673291429575\\
3.4935	129.027255099687\\
3.49375	112.716929196475\\
3.494	97.1744767506004\\
3.49425	81.3009533209668\\
3.4945	66.1745875336756\\
3.49475	50.6548725574349\\
3.495	34.5223356916486\\
3.49525	18.4221964311492\\
3.4955	1.88200178859362\\
3.49575	-15.0186798561397\\
3.496	-33.2996955266728\\
3.49625	-51.3154728888298\\
3.4965	-69.6063496752795\\
3.49675	-86.878712175893\\
3.497	-102.777020346024\\
3.49725	-118.521388832478\\
3.4975	-133.03686108011\\
3.49775	-146.425599480202\\
3.498	-159.927682602886\\
3.49825	-174.297803566193\\
3.4985	169.820035832211\\
3.49875	153.879415490663\\
3.499	136.137677742867\\
3.49925	118.41445669503\\
3.4995	100.415712950491\\
3.49975	83.7408717387082\\
3.5	67.8953881130193\\
3.50025	52.9543522361885\\
3.5005	39.5099299732529\\
3.50075	26.8708757392221\\
3.501	14.0429956591229\\
3.50125	0.94166339895036\\
3.5015	-12.7869215890208\\
3.50175	-26.0344291763034\\
3.502	-40.2270378317645\\
3.50225	-55.0089454678738\\
3.5025	-70.4155675642513\\
3.50275	-86.0164385778386\\
3.503	-101.409554815794\\
3.50325	-116.249581824547\\
3.5035	-131.330228351239\\
3.50375	-146.826915608093\\
3.504	-161.296029305794\\
3.50425	-175.748272307996\\
3.5045	171.619444346145\\
3.50475	158.124316013248\\
3.505	145.124181106037\\
3.50525	130.470662999258\\
3.5055	115.857403093433\\
3.50575	100.188930347849\\
3.506	83.9750493380062\\
3.50625	67.7749748488507\\
3.5065	52.0594516613028\\
3.50675	36.1795757692435\\
3.507	21.4956657836874\\
3.50725	8.51255217402512\\
3.5075	-2.53512365934073\\
3.50775	-13.4493049086719\\
3.508	-24.248185572144\\
3.50825	-35.5255115500635\\
3.5085	-48.2436375271621\\
3.50875	-61.2818187172419\\
3.509	-74.6559003139513\\
3.50925	-87.083527713526\\
3.5095	-99.5436782500301\\
3.50975	-111.725590786132\\
3.51	-121.817184254136\\
3.51025	-133.864867928362\\
3.5105	-146.056091936804\\
3.51075	-158.925811962199\\
3.511	-170.514931972047\\
3.51125	178.289936491596\\
3.5115	167.230236560753\\
3.51175	157.019528507437\\
3.512	146.281122766441\\
3.51225	134.480274980266\\
3.5125	120.252140206235\\
3.51275	106.314875926183\\
3.513	90.3645514751522\\
3.51325	74.7180610065709\\
3.5135	59.0351891445838\\
3.51375	44.806467106081\\
3.514	31.0342919673814\\
3.51425	18.0737730173061\\
3.5145	5.4742436395131\\
3.51475	-7.94296590645257\\
3.515	-21.6856103966563\\
3.51525	-37.4573651505082\\
3.5155	-53.5261976247971\\
3.51575	-70.2567332848358\\
3.516	-86.8788145609929\\
3.51625	-101.856745938087\\
3.5165	-115.712219336184\\
3.51675	-128.982602106459\\
3.517	-140.995914852812\\
3.51725	-150.299946616365\\
3.5175	-161.195365564348\\
3.51775	-172.799311829409\\
3.518	173.459481182907\\
3.51825	159.329940826282\\
3.5185	144.203608162644\\
3.51875	128.598603385031\\
3.519	113.800359170279\\
3.51925	100.382756817694\\
3.5195	87.7535031165559\\
3.51975	76.6219153917103\\
3.52	65.0081217258856\\
3.52025	52.7637843415624\\
3.5205	37.5278994073874\\
3.52075	22.509256676057\\
3.521	6.15691925328364\\
3.52125	-8.82388253292008\\
3.5215	-24.6084725850977\\
3.52175	-36.7805483917595\\
3.522	-46.5123425225787\\
3.52225	-52.7187203931302\\
3.5225	-59.9563263357071\\
3.52275	-69.6662559337219\\
3.523	-84.9048424557557\\
3.52325	-101.954477786034\\
3.5235	-119.945387826845\\
3.52375	-137.980191663439\\
3.524	-157.839028221728\\
3.52425	-175.54658951653\\
3.5245	169.713680942083\\
3.52475	159.690668811204\\
3.525	154.003472210196\\
3.52525	147.26306111905\\
3.5255	135.403828930835\\
3.52575	118.737195663997\\
3.526	101.061132933775\\
3.52625	81.597746913814\\
3.5265	59.7732432715988\\
3.52675	38.2514046623539\\
3.527	12.7954501923888\\
3.52725	-11.5253447192765\\
3.5275	-37.384835310307\\
3.52775	-63.4149988293652\\
3.528	-88.3140085724628\\
3.52825	-111.045780619709\\
3.5285	-134.964440225741\\
3.52875	-157.504073838909\\
3.529	-175.810600299643\\
3.52925	167.694618705835\\
3.5295	153.012114804192\\
3.52975	140.088163390728\\
3.53	123.883378955906\\
3.53025	104.390195873538\\
3.5305	81.8254514430972\\
3.53075	60.1117075259615\\
3.531	40.4393515636826\\
3.53125	21.4922951259721\\
3.5315	3.45901765553841\\
3.53175	-11.0036051852106\\
3.532	-22.6516151854596\\
3.53225	-30.3955995777715\\
3.5325	-37.4997371887941\\
3.53275	-46.2854188094762\\
3.533	-58.4199412314353\\
3.53325	-73.8826038786791\\
3.5335	-88.4933166310213\\
3.53375	-105.543091657632\\
3.534	-121.373558817464\\
3.53425	-136.65444249176\\
3.5345	-151.91361564753\\
3.53475	-167.787187278595\\
3.535	176.669645141594\\
3.53525	159.403133896271\\
3.5355	143.146417974474\\
3.53575	128.473771198304\\
3.536	113.510858972184\\
3.53625	99.0319481179873\\
3.5365	85.3380489349863\\
3.53675	72.1653263332556\\
3.537	58.2916470483764\\
3.53725	45.0220289413397\\
3.5375	31.1595087806878\\
3.53775	15.261350433645\\
3.538	0.0421572714081927\\
3.53825	-17.058235533364\\
3.5385	-31.4204575838642\\
3.53875	-45.4615988961322\\
3.539	-59.3141574952713\\
3.53925	-71.8622283937195\\
3.5395	-84.9883459948274\\
3.53975	-97.7148662650242\\
3.54	-110.685643718429\\
3.54025	-121.832084786892\\
3.5405	-133.631726767889\\
3.54075	-145.363022495079\\
3.541	-156.088717997159\\
3.54125	-167.342003829144\\
3.5415	-178.284042365478\\
3.54175	169.610939289508\\
3.542	156.951788629775\\
3.54225	143.935485701596\\
3.5425	131.351594048906\\
3.54275	118.912341664996\\
3.543	107.294685739606\\
3.54325	95.3690908396569\\
3.5435	82.4007553224551\\
3.54375	69.4990998012673\\
3.544	57.7489358379847\\
3.54425	44.3148260503593\\
3.5445	30.7787754974923\\
3.54475	18.3980089973405\\
3.545	6.02233813670356\\
3.54525	-5.68686870006679\\
3.5455	-15.8086487996479\\
3.54575	-26.2834560563173\\
3.546	-37.0306020839699\\
3.54625	-49.5174224179668\\
3.5465	-63.2250384997978\\
3.54675	-77.8570147822892\\
3.547	-92.6160106693091\\
3.54725	-107.394739612155\\
3.5475	-121.106925467477\\
3.54775	-134.065776088409\\
3.548	-146.71038667364\\
3.54825	-159.640067004379\\
3.5485	-172.819729719877\\
3.54875	172.86084810413\\
3.549	157.829172451647\\
3.54925	141.583490842931\\
3.5495	125.187102405056\\
3.54975	108.85910223347\\
3.55	93.1128532704069\\
3.55025	78.7745023827271\\
3.5505	65.9248347992254\\
3.55075	53.9074404118686\\
3.551	42.8297914883792\\
3.55125	30.5981045984033\\
3.5515	16.9475631947227\\
3.55175	2.05865709918068\\
3.552	-13.969379246605\\
3.55225	-30.1666322849107\\
3.5525	-46.7486839709549\\
3.55275	-62.3965287620216\\
3.553	-77.3381256969647\\
3.55325	-90.3183218280452\\
3.5535	-101.3741040042\\
3.55375	-112.017313060804\\
3.554	-123.134889853273\\
3.55425	-135.703215095659\\
3.5545	-149.359324282835\\
3.55475	-164.11681964343\\
3.555	179.878248331445\\
3.55525	162.855016128648\\
3.5555	146.221101097358\\
3.55575	130.299234691448\\
3.556	115.206112171171\\
3.55625	102.249811720961\\
3.5565	89.6972699556701\\
3.55675	77.0795486918993\\
3.557	64.8155413408656\\
3.55725	52.1616378736079\\
3.5575	37.4864274114925\\
3.55775	22.8661152521865\\
3.558	7.82301692870602\\
3.55825	-7.69053212809381\\
3.5585	-22.5765169840426\\
3.55875	-37.2322239825801\\
3.559	-51.8230832830951\\
3.55925	-66.7936424213577\\
3.5595	-82.5420973775277\\
3.55975	-97.9129462568253\\
3.56	-112.991833861302\\
3.56025	-126.14151019759\\
3.5605	-139.632327468859\\
3.56075	-151.704776214897\\
3.561	-164.883956369669\\
3.56125	-178.837848866027\\
3.5615	165.702092748515\\
3.56175	149.16947671613\\
3.562	131.652503315669\\
3.56225	115.080495659651\\
3.5625	98.7146483891457\\
3.56275	81.9892023265429\\
3.563	67.4149541145772\\
3.56325	53.0425209842211\\
3.5635	38.2618407299494\\
3.56375	23.4911681179492\\
3.564	8.70515766748914\\
3.56425	-7.87426439493089\\
3.5645	-25.3350222823502\\
3.56475	-44.1991808575968\\
3.565	-62.2762834429257\\
3.56525	-81.33057557585\\
3.5655	-102.014907689005\\
3.56575	-121.166140314759\\
3.566	-140.536564562398\\
3.56625	-160.870381117116\\
3.5665	179.595296606473\\
3.56675	161.985045336509\\
3.567	143.716794206629\\
3.56725	128.087134428329\\
3.5675	111.296087356181\\
3.56775	98.9846947892341\\
3.568	88.8054611823977\\
3.56825	78.4249515840617\\
3.5685	65.1606605521853\\
3.56875	48.5429254543214\\
3.569	29.8683185497822\\
3.56925	11.594584291704\\
3.5695	-5.58769523165586\\
3.56975	-22.787122968412\\
3.57	-40.104427708607\\
3.57025	-54.3211864661001\\
3.5705	-67.7392443832838\\
3.57075	-79.0930169281648\\
3.571	-91.7346285815674\\
3.57125	-107.287461431035\\
3.5715	-123.740983084127\\
3.57175	-141.871456569887\\
3.572	-158.765103023829\\
3.57225	-173.827457356394\\
3.5725	171.358701179004\\
3.57275	156.528067394907\\
3.573	142.094004279119\\
3.57325	127.824055724989\\
3.5735	114.03110764942\\
3.57375	98.3234231224705\\
3.574	81.4764284416547\\
3.57425	65.3308139390499\\
3.5745	46.8628803514268\\
3.57475	30.3312701287608\\
3.575	15.1628445553993\\
3.57525	2.44838312812393\\
3.5755	-7.87871237584795\\
3.57575	-19.2627541530551\\
3.576	-31.76516109112\\
3.57625	-45.4164087421735\\
3.5765	-61.3857855827755\\
3.57675	-76.4964887803653\\
3.577	-92.807979791548\\
3.57725	-109.647144783112\\
3.5775	-125.42509168472\\
3.57775	-140.210270185499\\
3.578	-153.671215278714\\
3.57825	-166.1103494282\\
3.5785	-178.702836273242\\
3.57875	168.20616109432\\
3.579	154.333946549572\\
3.57925	140.271191402033\\
3.5795	129.027769543033\\
3.57975	119.447205894207\\
3.58	109.0841796675\\
3.58025	96.8776221991196\\
3.5805	83.1015576405682\\
3.58075	68.1482602423134\\
3.581	53.9192244426878\\
3.58125	38.2376002894964\\
3.5815	24.8540652818633\\
3.58175	14.0926133151623\\
3.582	5.61454018126472\\
3.58225	-2.4115164741306\\
3.5825	-12.1517384605208\\
3.58275	-25.2384730625121\\
3.583	-41.5478059344138\\
3.58325	-59.2440837924843\\
3.5835	-78.2469603688748\\
3.58375	-95.8417919262682\\
3.584	-111.37255325468\\
3.58425	-123.546485212516\\
3.5845	-134.151976888891\\
3.58475	-144.139820320319\\
3.585	-155.853585068575\\
3.58525	-169.691299235871\\
3.5855	175.077692697672\\
3.58575	157.885862310527\\
3.586	139.256312117514\\
3.58625	121.460107423459\\
3.5865	104.25162272218\\
3.58675	89.0087667730879\\
3.587	75.1349725612615\\
3.58725	62.1307446794377\\
3.5875	48.8176303777287\\
3.58775	34.2587483997685\\
3.588	19.1264203135427\\
3.58825	2.81549228696533\\
3.5885	-14.2735067439386\\
3.58875	-31.8564468956852\\
3.589	-49.1529733021094\\
3.58925	-64.9668019562508\\
3.5895	-79.2092710834369\\
3.58975	-92.5692345035792\\
3.59	-106.820774359329\\
3.59025	-122.13529739152\\
3.5905	-138.006348576258\\
3.59075	-154.95846466754\\
3.591	-172.491016206301\\
3.59125	169.506498251316\\
3.5915	151.464316524242\\
3.59175	134.608916902277\\
3.592	118.250477692854\\
3.59225	104.33208142862\\
3.5925	93.0570611215516\\
3.59275	81.9920311918144\\
3.593	69.6381849889401\\
3.59325	55.6431605226543\\
3.5935	40.2292808912898\\
3.59375	23.7313010444689\\
3.594	7.68916206704291\\
3.59425	-7.8325943240347\\
3.5945	-24.4682642713573\\
3.59475	-41.0565311170022\\
3.595	-56.4972498923719\\
3.59525	-70.8410399130283\\
3.5955	-85.2694310339699\\
3.59575	-99.8212895606105\\
3.596	-114.999860732043\\
3.59625	-130.134192399281\\
3.5965	-145.365337444462\\
3.59675	-160.71306552\\
3.597	-176.601950260269\\
3.59725	167.623254843175\\
3.5975	151.525153763696\\
3.59775	135.306628568857\\
3.598	118.983841086087\\
3.59825	103.594547826089\\
3.5985	87.3706540583943\\
3.59875	71.5522957118293\\
3.599	55.7846990395561\\
3.59925	40.9103805774\\
3.5995	25.4092839540223\\
3.59975	10.4156911769967\\
3.6	-5.28785038758727\\
};
\addlegendentry{S12}

\end{axis}

\begin{axis}[%
width=5.833in,
height=4.375in,
at={(0in,0in)},
scale only axis,
xmin=0,
xmax=1,
ymin=0,
ymax=1,
axis line style={draw=none},
ticks=none,
axis x line*=bottom,
axis y line*=left
]
\end{axis}
\end{tikzpicture}%}
    \subcaption{Phase \emph{vs.} Frequency}\label{fig:phase}
\end{minipage}%
\caption{Channel reciprocity experiment: Measured S12 and S21 parameters of the RIS wireless channel, frequency range from 3.4 GHz to 3.6 GHz.}
\label{fig:channel_recipr}
\end{figure}

\end{comment}





\subsection{Proposed Beam Sweeping Methodology for AoA estimation}
In this subsection, we evaluate beam sweeping as a method for AoA estimation in both outdoor and indoor experimental setups (Fig.~\ref{fig:exp_4images}). For these measurements, the Rx signal power is measured over a frequency range of 3.4 GHz to 3.6 GHz, sampled at 801 points using a VNA; further investigating the methodology's accuracy for wideband applications.
%For both outdoor and indoor scenarios, measurements were taken using a Vector Network Analyzer (VNA) across 801 frequency points, ranging from 3.4 GHz to 3.6 GHz. 
In Figs.~\ref{fig:out_beamsw}--\ref{fig:m3_indoor}
%, \ref{fig:indoor_with_out}, \ref{fig:m1_indoor}, and \ref{fig:m3_indoor}, 
the red dot represents the RIS configuration where the maximum or minimum signal is expected. Furthermore, the Rx power is normalized, with the maximum set to 0 dB (Figs.~\ref{fig:out_beamsw}, \ref{fig:indoor_with_out}, and \ref{fig:m1_indoor}) or the minimum set to 0 dB (Fig.~\ref{fig:m3_indoor}).










\subsubsection{Beam Sweeping in Outdoor Setup}
%\vspace{0.1cm}
As shown in Fig.~\ref{fig:out_beamsw}, the codebook generated by Algorithm~1 ensures a pronounced main lobe toward the desired Rx position, achieving at least a 5 dB gain over the second-best RIS configuration for all Rx positions. This highlights the robustness of the beamforming strategy; achieving perfect AoA estimation in the outdoor environment (limited multipath).


\subsubsection{Beam Sweeping in Indoor Setup}
Fig.~\ref{fig:indoor_with_out} illustrates the performance of the precomputed codebook, in the indoor setup, showing a maximum AoA estimation error of $10^\circ$. This aligns with the findings of~\cite{kompostiotis2024evaluation}, which reported similar errors in ray-tracing simulations,\linebreak and~\cite{rahal2023ris}, which corroborated these results in a real RIS FR2 setup. 
To refine the AoA estimation, we propose a methodology that employs RIS for both maximizing and minimizing the received signal power. This dual-mode operation highlights the diversity that the RIS can provide on measurements. For the maximization case, as shown in Fig.~\ref{fig:m1_indoor}, the real-time execution of Algorithm~1 achieves perfect AoA estimation accuracy. Regarding the minimization case, depicted in Fig.~\ref{fig:m3_indoor}, only one test exhibits a mismatch, while the others achieve perfect AoA estimation. This mismatch may be arised due to frequency selectivity, as variations in the side lobes are observed depending on the frequency (Fig.~\ref{fig:freq_selectivity}). As mentioned, the results are evaluated across the entire tested frequency range, and it is important to note that for specific frequencies, this mismatch does not occur; highlighting the impact of frequency-dependent effects on the beamforming performance. 
%As shown in Fig.~\ref{fig:m3_indoor}, the number of iterations in minimization mode has a greater impact compared to the maximization case (Fig.~\ref{fig:m1_indoor}). This is justified, as observed from the measurements and the emerging RIS configurations, rows play a more significant role in minimization mode. Algorithm~1 starts with column scanning, therefore the minimization mode benefits from further improvements in the second iteration. So, 
The real-time execution of Algorithm~1 allows for dynamic adjustments to the indoor environment's geometry, while both maximization and minimization beam sweeping modes provide complementary solutions, enhancing AoA estimation accuracy and offering a more robust approach for real-world indoor environments.



\vspace{-0.1cm}
\section{Conclusion}~\label{s:conclusion} 
This paper provides practical insights for the RIS technology, proposing a methodology for beam sweeping in both outdoor and indoor environments; addressing the challenges of multipath and frequency-selective behavior of RIS.
%, where its phase response varies across frequencies. 
The applied RIS configuration method leads to a codebook that is practical for beam sweeping, and the real measurements validate the theoretical model. Outdoor measurements show that the generated codebook performs effectively, while for the challenging indoor setup, an additional methodology is proposed for improved AoA estimation, reducing the initial mismatch of around $10^\circ$ by adapting the RIS configuration to the specific multipath conditions and geometry of the environment.
%Additionally, our measurement campaign confirms channel reciprocity in RIS-aided communication systems and demonstrates how RIS can significantly modify the channel properties of a static environment, enhancing the received signal power by +X dB when utilized for receive signal power maximization or reducing it by -Y dB when optimized for signal suppression.

% \section{Future Work}~\label{s:future_work}
% In future work, we will conduct real measurements to evaluate a hierarchical beam sweeping approach by enriching the RIS codebook with multi-beam patterns. These patterns will enable the RIS to transmit signals toward multiple desired angles simultaneously, segmenting the served area and reducing the iterations required for conventional beam sweeping. Furthermore, leveraging the dual-polarized RIS in our setup, we plan to test the feasibility of generating broad beam patterns using techniques such as Golay complementary pairs. The omni pattern can serve as a benchmark to establish the initial received power for the user. Subsequently, through the hierarchical beam sweeping protocol, the AoA estimation accuracy will be gradually improving.




\vspace{-0.1cm}
\section{Acknowledgement}
This work has been supported by the ESA Project PRISM: RIS-enabled Positioning and Mapping (NAVISP-EL1-063).







\begin{figure*}[t]
    \vspace{1mm}
    \centering
    \begin{minipage}{0.185\textwidth}
        \centering
        \scalebox{0.24}{% This file was created by matlab2tikz.
%
%The latest updates can be retrieved from
%  http://www.mathworks.com/matlabcentral/fileexchange/22022-matlab2tikz-matlab2tikz
%where you can also make suggestions and rate matlab2tikz.
%
\definecolor{mycolor1}{rgb}{0.20000,0.60000,0.80000}%
%
\begin{tikzpicture}

\begin{axis}[%
width=4.844in,
height=3.396in,
at={(0.812in,0.458in)},
scale only axis,
bar shift auto,
xmin=-30,
xmax=0,
xlabel style={font=\color{white!15!black}},
xlabel={Normalized Power (dB)},
ymin=-0.2,
ymax=14.2,
ytick={ 1,  2,  3,  4,  5,  6,  7,  8,  9, 10, 11, 12, 13},
    yticklabels={$0^\circ$, $5^\circ$, $10^\circ$, $15^\circ$, $20^\circ$, $25^\circ$, $30^\circ$, $35^\circ$, $40^\circ$, $45^\circ$, $50^\circ$, $55^\circ$, $60^\circ$},
ylabel style={font=\color{white!15!black}},
ylabel={RIS Configuration},
axis background/.style={fill=white},
title style={font=\bfseries},
%title={$\text{Beam Sweeping - RX Position 0}^\circ$},
axis x line*=bottom,
axis y line*=left,
xmajorgrids,
ymajorgrids
]
\addplot[xbar, bar width=0.8, fill=mycolor1, draw=black, area legend] table[row sep=crcr] {%
0	1\\
-9.44701692725761	2\\
-20.4938110984389	3\\
-18.2215729458743	4\\
-19.3423000161992	5\\
-13.2961174665309	6\\
-6.65518493147796	7\\
-5.87754101561218	8\\
-6.98375988854762	9\\
-3.71912167808735	10\\
-10.3819347348705	11\\
-9.37497079929054	12\\
-19.3254123211246	13\\
};
\addplot[forget plot, color=white!15!black] table[row sep=crcr] {%
0	-0.2\\
0	14.2\\
};
\addplot[only marks, mark=*, mark options={}, mark size=2.2361pt, color=red, fill=red, forget plot] table[row sep=crcr]
        \subcaption{Rx position at $0^\circ$}
        \label{fig:beamsw_out0}
    \end{minipage}
    \begin{minipage}{0.185\textwidth}
        \centering
        \scalebox{0.24}{% This file was created by matlab2tikz.
%
%The latest updates can be retrieved from
%  http://www.mathworks.com/matlabcentral/fileexchange/22022-matlab2tikz-matlab2tikz
%where you can also make suggestions and rate matlab2tikz.
%
\definecolor{mycolor1}{rgb}{0.20000,0.60000,0.80000}%
%
\begin{tikzpicture}

\begin{axis}[%
width=4.844in,
height=3.396in,
at={(0.812in,0.458in)},
scale only axis,
bar shift auto,
xmin=-30,
xmax=0,
xlabel style={font=\color{white!15!black}},
xlabel={Normalized Power (dB)},
ymin=-0.2,
ymax=14.2,
ytick={ 1,  2,  3,  4,  5,  6,  7,  8,  9, 10, 11, 12, 13},
    yticklabels={$0^\circ$, $5^\circ$, $10^\circ$, $15^\circ$, $20^\circ$, $25^\circ$, $30^\circ$, $35^\circ$, $40^\circ$, $45^\circ$, $50^\circ$, $55^\circ$, $60^\circ$},
ylabel style={font=\color{white!15!black}},
ylabel={RIS Configuration},
axis background/.style={fill=white},
title style={font=\bfseries},
%title={$\text{Beam Sweeping - RX Position 15}^\circ$},
axis x line*=bottom,
axis y line*=left,
xmajorgrids,
ymajorgrids
]
\addplot[xbar, bar width=0.8, fill=mycolor1, draw=black, area legend] table[row sep=crcr] {%
-11.4319704177238	1\\
-13.9614533664424	2\\
-10.234631161402	3\\
0	4\\
-6.12230349350461	5\\
-4.83810975091131	6\\
-6.46597916579952	7\\
-12.241166064019	8\\
-13.0297536395889	9\\
-10.1663664046928	10\\
-14.0201408092067	11\\
-13.5358192784159	12\\
-13.4227979155706	13\\
};
\addplot[forget plot, color=white!15!black] table[row sep=crcr] {%
0	-0.2\\
0	14.2\\
};
\addplot[only marks, mark=*, mark options={}, mark size=2.2361pt, color=red, fill=red, forget plot] table[row sep=crcr]
        \subcaption{Rx position at $15^\circ$}
        \label{fig:beamsw_out15}
    \end{minipage}
    \begin{minipage}{0.185\textwidth}
        \centering
        \scalebox{0.24}{% This file was created by matlab2tikz.
%
%The latest updates can be retrieved from
%  http://www.mathworks.com/matlabcentral/fileexchange/22022-matlab2tikz-matlab2tikz
%where you can also make suggestions and rate matlab2tikz.
%
\definecolor{mycolor1}{rgb}{0.20000,0.60000,0.80000}%
%
\begin{tikzpicture}

\begin{axis}[%
width=4.844in,
height=3.396in,
at={(0.812in,0.458in)},
scale only axis,
bar shift auto,
xmin=-30,
xmax=0,
xlabel style={font=\color{white!15!black}},
xlabel={Normalized Power (dB)},
ymin=-0.2,
ymax=14.2,
ytick={ 1,  2,  3,  4,  5,  6,  7,  8,  9, 10, 11, 12, 13},
    yticklabels={$0^\circ$, $5^\circ$, $10^\circ$, $15^\circ$, $20^\circ$, $25^\circ$, $30^\circ$, $35^\circ$, $40^\circ$, $45^\circ$, $50^\circ$, $55^\circ$, $60^\circ$},
ylabel style={font=\color{white!15!black}},
ylabel={RIS Configuration},
axis background/.style={fill=white},
title style={font=\bfseries},
%title={$\text{Beam Sweeping - RX Position 30}^\circ$},
axis x line*=bottom,
axis y line*=left,
xmajorgrids,
ymajorgrids
]
\addplot[xbar, bar width=0.8, fill=mycolor1, draw=black, area legend] table[row sep=crcr] {%
-9.99352923435171	1\\
-7.60561512425468	2\\
-6.86197214320735	3\\
-8.51801882247999	4\\
-13.1258417597495	5\\
-18.0113336483307	6\\
0	7\\
-19.3672845909811	8\\
-17.2760209284037	9\\
-12.6625475217441	10\\
-20.1223729371069	11\\
-15.5199866971492	12\\
-10.2060252791208	13\\
};
\addplot[forget plot, color=white!15!black] table[row sep=crcr] {%
0	-0.2\\
0	14.2\\
};
\addplot[only marks, mark=*, mark options={}, mark size=2.2361pt, color=red, fill=red, forget plot] table[row sep=crcr]
        \subcaption{Rx position at $30^\circ$}
        \label{fig:beamsw_out30}
    \end{minipage}
    \begin{minipage}{0.185\textwidth}
        \centering
        \scalebox{0.24}{% This file was created by matlab2tikz.
%
%The latest updates can be retrieved from
%  http://www.mathworks.com/matlabcentral/fileexchange/22022-matlab2tikz-matlab2tikz
%where you can also make suggestions and rate matlab2tikz.
%
\definecolor{mycolor1}{rgb}{0.20000,0.60000,0.80000}%
%
\begin{tikzpicture}

\begin{axis}[%
width=4.844in,
height=3.396in,
at={(0.812in,0.458in)},
scale only axis,
bar shift auto,
xmin=-30,
xmax=0,
xlabel style={font=\color{white!15!black}},
xlabel={Normalized Power (dB)},
ymin=-0.2,
ymax=14.2,
ytick={ 1,  2,  3,  4,  5,  6,  7,  8,  9, 10, 11, 12, 13},
    yticklabels={$0^\circ$, $5^\circ$, $10^\circ$, $15^\circ$, $20^\circ$, $25^\circ$, $30^\circ$, $35^\circ$, $40^\circ$, $45^\circ$, $50^\circ$, $55^\circ$, $60^\circ$},
ylabel style={font=\color{white!15!black}},
ylabel={RIS Configuration},
axis background/.style={fill=white},
title style={font=\bfseries},
%title={$\text{Beam Sweeping - RX Position 45}^\circ$},
axis x line*=bottom,
axis y line*=left,
xmajorgrids,
ymajorgrids
]
\addplot[xbar, bar width=0.8, fill=mycolor1, draw=black, area legend] table[row sep=crcr] {%
-5.29345980933891	1\\
-7.62364946882121	2\\
-11.7327585397704	3\\
-15.1315223789185	4\\
-24.1553764203118	5\\
-12.2215085934111	6\\
-21.624419473844	7\\
-16.2499711696139	8\\
-4.52090925233889	9\\
0	10\\
-9.19676901356801	11\\
-12.5337483272285	12\\
-22.216815734731	13\\
};
\addplot[forget plot, color=white!15!black] table[row sep=crcr] {%
0	-0.2\\
0	14.2\\
};
\addplot[only marks, mark=*, mark options={}, mark size=2.2361pt, color=red, fill=red, forget plot] table[row sep=crcr]
        \subcaption{Rx position at $45^\circ$}
        \label{fig:beamsw_out45}
    \end{minipage}
    \begin{minipage}{0.185\textwidth}
        \centering
        \scalebox{0.24}{% This file was created by matlab2tikz.
%
%The latest updates can be retrieved from
%  http://www.mathworks.com/matlabcentral/fileexchange/22022-matlab2tikz-matlab2tikz
%where you can also make suggestions and rate matlab2tikz.
%
\definecolor{mycolor1}{rgb}{0.20000,0.60000,0.80000}%
%
\begin{tikzpicture}

\begin{axis}[%
width=4.844in,
height=3.396in,
at={(0.812in,0.458in)},
scale only axis,
bar shift auto,
xmin=-30,
xmax=0,
xlabel style={font=\color{white!15!black}},
xlabel={Normalized Power (dB)},
ymin=-0.2,
ymax=14.2,
ytick={ 1,  2,  3,  4,  5,  6,  7,  8,  9, 10, 11, 12, 13},
    yticklabels={$0^\circ$, $5^\circ$, $10^\circ$, $15^\circ$, $20^\circ$, $25^\circ$, $30^\circ$, $35^\circ$, $40^\circ$, $45^\circ$, $50^\circ$, $55^\circ$, $60^\circ$},
ylabel style={font=\color{white!15!black}},
ylabel={RIS Configuration},
axis background/.style={fill=white},
title style={font=\bfseries},
%title={$\text{Beam Sweeping - RX Position 60}^\circ$},
axis x line*=bottom,
axis y line*=left,
xmajorgrids,
ymajorgrids
]
\addplot[xbar, bar width=0.8, fill=mycolor1, draw=black, area legend] table[row sep=crcr] {%
-9.41516888312268	1\\
-14.1738385093149	2\\
-13.4048739907347	3\\
-16.1853162549615	4\\
-17.6233751281677	5\\
-13.7022503349284	6\\
-11.8131893581137	7\\
-16.3225709356877	8\\
-12.1850497581985	9\\
-15.4550154342765	10\\
-14.6098927014232	11\\
-3.36654273309183	12\\
0	13\\
};
\addplot[forget plot, color=white!15!black] table[row sep=crcr] {%
0	-0.2\\
0	14.2\\
};
\addplot[only marks, mark=*, mark options={}, mark size=2.2361pt, color=red, fill=red, forget plot] table[row sep=crcr]
        \subcaption{Rx position at $60^\circ$}
        \label{fig:beamsw_out60}
    \end{minipage}
    \caption{Outdoor setup: Beam sweeping results}
    \label{fig:out_beamsw}
\end{figure*}



\begin{figure*}[t]
    \centering
    \begin{minipage}{0.24\textwidth}
        \centering
        \scalebox{0.28}{% This file was created by matlab2tikz.
%
%The latest updates can be retrieved from
%  http://www.mathworks.com/matlabcentral/fileexchange/22022-matlab2tikz-matlab2tikz
%where you can also make suggestions and rate matlab2tikz.
%
\definecolor{mycolor1}{rgb}{0.20000,0.60000,0.80000}%
%
\begin{tikzpicture}

\begin{axis}[%
width=4.844in,
height=3.396in,
at={(0.812in,0.458in)},
scale only axis,
bar shift auto,
xmin=-30,
xmax=0,
xlabel style={font=\color{white!15!black}},
xlabel={Normalized Power Metric (dB)},
ymin=0.5,
ymax=13.5,
ytick={ 1,  2,  3,  4,  5,  6,  7,  8,  9, 10, 11, 12, 13},
    yticklabels={$0^\circ$, $5^\circ$, $10^\circ$, $15^\circ$, $20^\circ$, $25^\circ$, $30^\circ$, $35^\circ$, $40^\circ$, $45^\circ$, $50^\circ$, $55^\circ$, $60^\circ$},
ylabel style={font=\color{white!15!black}},
ylabel={RIS Configuration},
axis background/.style={fill=white},
title style={font=\bfseries},
%title={$\text{Normalized Beam Sweeping - RX Position 0}^\circ$},
axis x line*=bottom,
axis y line*=left,
xmajorgrids,
ymajorgrids
]
\addplot[xbar, bar width=0.8, fill=mycolor1, draw=black, area legend] table[row sep=crcr] {%
-12.9107723137461	1\\
-2.73862231211901	2\\
0	3\\
-11.9397246051296	4\\
-7.38691555900036	5\\
-3.26783917777293	6\\
-2.04516825088477	7\\
-5.11344900205867	8\\
-4.08550018629995	9\\
-10.8553865617718	10\\
-11.7329574611128	11\\
-14.4791219904472	12\\
-9.77057203755779	13\\
};
\addplot[forget plot, color=white!15!black, line width=1.5pt] table[row sep=crcr] {%
0	0.5\\
0	13.5\\
};
\addplot[only marks, mark=*, mark options={}, mark size=2.2361pt, color=red, fill=red, forget plot] table[row sep=crcr]
        \subcaption{Rx position at $0^\circ$}
        \label{fig:ind_beamsw_out0}
    \end{minipage}
    \begin{minipage}{0.24\textwidth}
        \centering
        \scalebox{0.28}{% This file was created by matlab2tikz.
%
%The latest updates can be retrieved from
%  http://www.mathworks.com/matlabcentral/fileexchange/22022-matlab2tikz-matlab2tikz
%where you can also make suggestions and rate matlab2tikz.
%
\definecolor{mycolor1}{rgb}{0.20000,0.60000,0.80000}%
%
\begin{tikzpicture}

\begin{axis}[%
width=4.844in,
height=3.396in,
at={(0.812in,0.458in)},
scale only axis,
bar shift auto,
xmin=-30,
xmax=0,
xlabel style={font=\color{white!15!black}},
xlabel={Normalized Power Metric (dB)},
ymin=0.5,
ymax=13.5,
ytick={ 1,  2,  3,  4,  5,  6,  7,  8,  9, 10, 11, 12, 13},
    yticklabels={$0^\circ$, $5^\circ$, $10^\circ$, $15^\circ$, $20^\circ$, $25^\circ$, $30^\circ$, $35^\circ$, $40^\circ$, $45^\circ$, $50^\circ$, $55^\circ$, $60^\circ$},
ylabel style={font=\color{white!15!black}},
ylabel={RIS Configuration},
axis background/.style={fill=white},
title style={font=\bfseries},
%title={$\text{Normalized Beam Sweeping - RX Position 15}^\circ$},
axis x line*=bottom,
axis y line*=left,
xmajorgrids,
ymajorgrids
]
\addplot[xbar, bar width=0.8, fill=mycolor1, draw=black, area legend] table[row sep=crcr] {%
-9.05769334868475	1\\
-8.31192009670491	2\\
-9.04457383685525	3\\
-4.30040179904159	4\\
-2.15822465676875	5\\
0	6\\
-8.89203862886973	7\\
-8.31700676862815	8\\
-11.5286560284076	9\\
-11.8658155926617	10\\
-10.1694528255634	11\\
-12.7714223747898	12\\
-8.3503237107953	13\\
};
\addplot[forget plot, color=white!15!black, line width=1.5pt] table[row sep=crcr] {%
0	0.5\\
0	13.5\\
};
\addplot[only marks, mark=*, mark options={}, mark size=2.2361pt, color=red, fill=red, forget plot] table[row sep=crcr]
        \subcaption{Rx position at $15^\circ$}
        \label{fig:ind_beamsw_out15}
    \end{minipage}
    \begin{minipage}{0.24\textwidth}
        \centering
        \scalebox{0.28}{% This file was created by matlab2tikz.
%
%The latest updates can be retrieved from
%  http://www.mathworks.com/matlabcentral/fileexchange/22022-matlab2tikz-matlab2tikz
%where you can also make suggestions and rate matlab2tikz.
%
\definecolor{mycolor1}{rgb}{0.20000,0.60000,0.80000}%
%
\begin{tikzpicture}

\begin{axis}[%
width=4.844in,
height=3.396in,
at={(0.812in,0.458in)},
scale only axis,
bar shift auto,
xmin=-30,
xmax=0,
xlabel style={font=\color{white!15!black}},
xlabel={Normalized Power Metric (dB)},
ymin=0.5,
ymax=13.5,
ytick={ 1,  2,  3,  4,  5,  6,  7,  8,  9, 10, 11, 12, 13},
    yticklabels={$0^\circ$, $5^\circ$, $10^\circ$, $15^\circ$, $20^\circ$, $25^\circ$, $30^\circ$, $35^\circ$, $40^\circ$, $45^\circ$, $50^\circ$, $55^\circ$, $60^\circ$},
ylabel style={font=\color{white!15!black}},
ylabel={RIS Configuration},
axis background/.style={fill=white},
title style={font=\bfseries},
%title={$\text{Normalized Beam Sweeping - RX Position 30}^\circ$},
axis x line*=bottom,
axis y line*=left,
xmajorgrids,
ymajorgrids
]
\addplot[xbar, bar width=0.8, fill=mycolor1, draw=black, area legend] table[row sep=crcr] {%
-6.8944050079383	1\\
-5.66031293307523	2\\
-4.25002119340891	3\\
-13.7551552384071	4\\
-14.0193744017144	5\\
-9.6291075319989	6\\
-13.0940070434133	7\\
0	8\\
-3.74509164205062	9\\
-10.3979020769042	10\\
-12.3303138725052	11\\
-7.94572172162777	12\\
-13.0489854492803	13\\
};
\addplot[forget plot, color=white!15!black, line width=1.5pt] table[row sep=crcr] {%
0	0.5\\
0	13.5\\
};
\addplot[only marks, mark=*, mark options={}, mark size=2.2361pt, color=red, fill=red, forget plot] table[row sep=crcr]
        \subcaption{Rx position at $30^\circ$}
        \label{fig:ind_beamsw_out30}
    \end{minipage}
    \begin{minipage}{0.24\textwidth}
        \centering
        \scalebox{0.28}{% This file was created by matlab2tikz.
%
%The latest updates can be retrieved from
%  http://www.mathworks.com/matlabcentral/fileexchange/22022-matlab2tikz-matlab2tikz
%where you can also make suggestions and rate matlab2tikz.
%
\definecolor{mycolor1}{rgb}{0.20000,0.60000,0.80000}%
%
\begin{tikzpicture}

\begin{axis}[%
width=4.844in,
height=3.396in,
at={(0.812in,0.458in)},
scale only axis,
bar shift auto,
xmin=-30,
xmax=0,
xlabel style={font=\color{white!15!black}},
xlabel={Normalized Power Metric (dB)},
ymin=0.5,
ymax=13.5,
ytick={ 1,  2,  3,  4,  5,  6,  7,  8,  9, 10, 11, 12, 13},
    yticklabels={$0^\circ$, $5^\circ$, $10^\circ$, $15^\circ$, $20^\circ$, $25^\circ$, $30^\circ$, $35^\circ$, $40^\circ$, $45^\circ$, $50^\circ$, $55^\circ$, $60^\circ$},
ylabel style={font=\color{white!15!black}},
ylabel={RIS Configuration},
axis background/.style={fill=white},
title style={font=\bfseries},
%title={$\text{Normalized Beam Sweeping - RX Position 45}^\circ$},
axis x line*=bottom,
axis y line*=left,
xmajorgrids,
ymajorgrids
]
\addplot[xbar, bar width=0.8, fill=mycolor1, draw=black, area legend] table[row sep=crcr] {%
-3.56139278239337	1\\
-7.61564335874992	2\\
-10.8691034039982	3\\
-11.3367028591996	4\\
-14.7172716668008	5\\
-11.0924497633497	6\\
-15.3765862114877	7\\
-12.1016533771549	8\\
-8.61678803525348	9\\
0	10\\
-3.30239975627171	11\\
-13.6511576168329	12\\
-14.1264785918066	13\\
};
\addplot[forget plot, color=white!15!black, line width=1.5pt] table[row sep=crcr] {%
0	0.5\\
0	13.5\\
};
\addplot[only marks, mark=*, mark options={}, mark size=2.2361pt, color=red, fill=red, forget plot] table[row sep=crcr]
        \subcaption{Rx position at $45^\circ$}
        \label{fig:ind_beamsw_out45}
    \end{minipage}
    
    \caption{Indoor setup: Beam sweeping results utilizing the precomputed codebook, with maximum AoA estimation error equal to $10^\circ$ (cases~\ref{fig:ind_beamsw_out0} and \ref{fig:ind_beamsw_out15}).}
    \label{fig:indoor_with_out}
\end{figure*}





\begin{figure*}[t]
    \centering
    \begin{minipage}{0.24\textwidth}
        \centering
        \scalebox{0.28}{% This file was created by matlab2tikz.
%
%The latest updates can be retrieved from
%  http://www.mathworks.com/matlabcentral/fileexchange/22022-matlab2tikz-matlab2tikz
%where you can also make suggestions and rate matlab2tikz.
%
\definecolor{mycolor1}{rgb}{0.20000,0.60000,0.80000}%
%
\begin{tikzpicture}

\begin{axis}[%
width=4.844in,
height=3.396in,
at={(0.812in,0.458in)},
scale only axis,
bar shift auto,
xmin=-30,
xmax=0,
xlabel style={font=\color{white!15!black}},
xlabel={Normalized Power Metric (dB)},
ymin=0.5,
ymax=8.5,
ytick={1, 2, 3, 4, 5, 6, 7, 8},
    yticklabels={$0^\circ$-1st iter, $0^\circ$-2nd iter, $15^\circ$-1st iter, $15^\circ$-2nd iter, $30^\circ$-1st iter, $30^\circ$-2nd iter, $45^\circ$-1st iter, $45^\circ$-2nd iter},
ylabel style={font=\color{white!15!black}},
ylabel={RIS Configuration},
axis background/.style={fill=white},
title style={font=\bfseries},
%title={$\text{Normalized Beam Sweeping - RX Position 0}^\circ$},
axis x line*=bottom,
axis y line*=left,
xmajorgrids,
ymajorgrids
]
\addplot[xbar, bar width=0.8, fill=mycolor1, draw=black, area legend] table[row sep=crcr] {%
-0.338885874853034	1\\
0	2\\
-14.21294203993	3\\
-13.7635333493692	4\\
-10.140545719259	5\\
-9.09809874944966	6\\
-3.06616271812548	7\\
-3.36497427259179	8\\
};
\addplot[forget plot, color=white!15!black, line width=1.5pt] table[row sep=crcr] {%
0	0.5\\
0	8.5\\
};
\addplot[only marks, mark=*, mark options={}, mark size=2.2361pt, color=red, fill=red, forget plot] table[row sep=crcr]
        \subcaption{Rx position at $0^\circ$}
        \label{fig:ind_beamsw_maxin0}
    \end{minipage}
    \begin{minipage}{0.24\textwidth}
        \centering
        \scalebox{0.28}{% This file was created by matlab2tikz.
%
%The latest updates can be retrieved from
%  http://www.mathworks.com/matlabcentral/fileexchange/22022-matlab2tikz-matlab2tikz
%where you can also make suggestions and rate matlab2tikz.
%
\definecolor{mycolor1}{rgb}{0.20000,0.60000,0.80000}%
%
\begin{tikzpicture}

\begin{axis}[%
width=4.844in,
height=3.396in,
at={(0.812in,0.458in)},
scale only axis,
bar shift auto,
xmin=-30,
xmax=0,
xlabel style={font=\color{white!15!black}},
xlabel={Normalized Power Metric (dB)},
ymin=0.5,
ymax=8.5,
ytick={1, 2, 3, 4, 5, 6, 7, 8},
yticklabels={$0^\circ$-1st iter, $0^\circ$-2nd iter, $15^\circ$-1st iter, $15^\circ$-2nd iter, $30^\circ$-1st iter, $30^\circ$-2nd iter, $45^\circ$-1st iter, $45^\circ$-2nd iter},
ylabel style={font=\color{white!15!black}},
ylabel={RIS Configuration},
axis background/.style={fill=white},
title style={font=\bfseries},
%title={$\text{Normalized Beam Sweeping - RX Position 15}^\circ$},
axis x line*=bottom,
axis y line*=left,
xmajorgrids,
ymajorgrids
]
\addplot[xbar, bar width=0.8, fill=mycolor1, draw=black, area legend] table[row sep=crcr] {%
-11.7830591389383	1\\
-11.4120709740691	2\\
-0.143340601772081	3\\
0	4\\
-5.71565626489543	5\\
-5.12892298599902	6\\
-8.97111529264295	7\\
-6.56237911245133	8\\
};
\addplot[forget plot, color=white!15!black, line width=1.5pt] table[row sep=crcr] {%
0	0.5\\
0	8.5\\
};
\addplot[only marks, mark=*, mark options={}, mark size=2.2361pt, color=red, fill=red, forget plot] table[row sep=crcr]
        \subcaption{Rx position at $15^\circ$}
        \label{fig:ind_beamsw_maxin15}
    \end{minipage}
    \begin{minipage}{0.24\textwidth}
        \centering
        \scalebox{0.28}{% This file was created by matlab2tikz.
%
%The latest updates can be retrieved from
%  http://www.mathworks.com/matlabcentral/fileexchange/22022-matlab2tikz-matlab2tikz
%where you can also make suggestions and rate matlab2tikz.
%
\definecolor{mycolor1}{rgb}{0.20000,0.60000,0.80000}%
%
\begin{tikzpicture}

\begin{axis}[%
width=4.844in,
height=3.396in,
at={(0.812in,0.458in)},
scale only axis,
bar shift auto,
xmin=-30,
xmax=0,
xlabel style={font=\color{white!15!black}},
xlabel={Normalized Power Metric (dB)},
ymin=0.5,
ymax=8.5,
ytick={1, 2, 3, 4, 5, 6, 7, 8},
yticklabels={$0^\circ$-1st iter, $0^\circ$-2nd iter, $15^\circ$-1st iter, $15^\circ$-2nd iter, $30^\circ$-1st iter, $30^\circ$-2nd iter, $45^\circ$-1st iter, $45^\circ$-2nd iter},
ylabel style={font=\color{white!15!black}},
ylabel={RIS Configuration},
axis background/.style={fill=white},
title style={font=\bfseries},
%title={$\text{Normalized Beam Sweeping - RX Position 30}^\circ$},
axis x line*=bottom,
axis y line*=left,
xmajorgrids,
ymajorgrids
]
\addplot[xbar, bar width=0.8, fill=mycolor1, draw=black, area legend] table[row sep=crcr] {%
-3.4886760879522	1\\
-2.65544881564408	2\\
-3.30869249437968	3\\
-2.37971672279907	4\\
-0.665999262367848	5\\
0	6\\
-9.0090572620844	7\\
-6.70795064267047	8\\
};
\addplot[forget plot, color=white!15!black, line width=1.5pt] table[row sep=crcr] {%
0	0.5\\
0	8.5\\
};
\addplot[only marks, mark=*, mark options={}, mark size=2.2361pt, color=red, fill=red, forget plot] table[row sep=crcr]
        \subcaption{Rx position at $30^\circ$}
        \label{fig:ind_beamsw_maxin30}
    \end{minipage}
    \begin{minipage}{0.24\textwidth}
        \centering
        \scalebox{0.28}{% This file was created by matlab2tikz.
%
%The latest updates can be retrieved from
%  http://www.mathworks.com/matlabcentral/fileexchange/22022-matlab2tikz-matlab2tikz
%where you can also make suggestions and rate matlab2tikz.
%
\definecolor{mycolor1}{rgb}{0.20000,0.60000,0.80000}%
%
\begin{tikzpicture}

\begin{axis}[%
width=4.844in,
height=3.396in,
at={(0.812in,0.458in)},
scale only axis,
bar shift auto,
xmin=-30,
xmax=0,
xlabel style={font=\color{white!15!black}},
xlabel={Normalized Power Metric (dB)},
ymin=0.5,
ymax=8.5,
ytick={1, 2, 3, 4, 5, 6, 7, 8},
yticklabels={$0^\circ$-1st iter, $0^\circ$-2nd iter, $15^\circ$-1st iter, $15^\circ$-2nd iter, $30^\circ$-1st iter, $30^\circ$-2nd iter, $45^\circ$-1st iter, $45^\circ$-2nd iter},
ylabel style={font=\color{white!15!black}},
ylabel={RIS Configuration},
axis background/.style={fill=white},
title style={font=\bfseries},
%title={$\text{Normalized Beam Sweeping - RX Position 45}^\circ$},
axis x line*=bottom,
axis y line*=left,
xmajorgrids,
ymajorgrids
]
\addplot[xbar, bar width=0.8, fill=mycolor1, draw=black, area legend] table[row sep=crcr] {%
-4.19209448105993	1\\
-5.09574629996613	2\\
-12.113619763628	3\\
-15.6057824141141	4\\
-14.7521958130981	5\\
-11.4781886358824	6\\
-0.276869637951478	7\\
0	8\\
};
\addplot[forget plot, color=white!15!black, line width=1.5pt] table[row sep=crcr] {%
0	0.5\\
0	8.5\\
};
\addplot[only marks, mark=*, mark options={}, mark size=2.2361pt, color=red, fill=red, forget plot] table[row sep=crcr]
        \subcaption{Rx position at $45^\circ$}
        \label{fig:ind_beamsw_maxin45}
    \end{minipage}
    
    \caption{Indoor setup: Beam sweeping results, executing two iterations of Algorithm~1 in real time (Maximization).}
    \label{fig:m1_indoor}
\end{figure*}


\begin{figure*}[t]
    \centering
    \begin{minipage}{0.24\textwidth}
        \centering
        \scalebox{0.28}{% This file was created by matlab2tikz.
%
%The latest updates can be retrieved from
%  http://www.mathworks.com/matlabcentral/fileexchange/22022-matlab2tikz-matlab2tikz
%where you can also make suggestions and rate matlab2tikz.
%
\definecolor{mycolor1}{rgb}{0.20000,0.60000,0.80000}%
%
\begin{tikzpicture}

\begin{axis}[%
width=4.844in,
height=3.396in,
at={(0.812in,0.458in)},
scale only axis,
bar shift auto,
xmin=0,
xmax=60,
xlabel style={font=\color{white!15!black}},
xlabel={Normalized Power Metric (dB)},
ymin=0.5,
ymax=8.5,
ytick={1,2,3,4,5,6,7,8},
%yticklabels={{17},{18},{19},{20},{21},{22},{23},{24}},
yticklabels={$0^\circ$-1st iter, $0^\circ$-2nd iter, $15^\circ$-1st iter, $15^\circ$-2nd iter, $30^\circ$-1st iter, $30^\circ$-2nd iter, $45^\circ$-1st iter, $45^\circ$-2nd iter},
ylabel style={font=\color{white!15!black}},
ylabel={RIS Configuration},
axis background/.style={fill=white},
title style={font=\bfseries},
%title={$\text{Minimum Power Analysis - RX Position 0}^\circ$},
axis x line*=bottom,
axis y line*=left,
xmajorgrids,
ymajorgrids
]
\addplot[xbar, bar width=0.8, fill=mycolor1, draw=black, area legend] table[row sep=crcr] {%
24.0764	1\\
0	2\\
53.56713	3\\
51.98386	4\\
20.0303	5\\
24.97505	6\\
30.9861	7\\
32.47019	8\\
};
\addplot[forget plot, color=white!15!black, line width=1.5pt] table[row sep=crcr] {%
0	0.5\\
0	8.5\\
};
\addplot[only marks, mark=*, mark options={}, mark size=2.2361pt, color=red, fill=red, forget plot] table[row sep=crcr]
        \subcaption{Rx position at $0^\circ$}
        \label{fig:ind_beamsw_minin0}
    \end{minipage}
    \begin{minipage}{0.24\textwidth}
        \centering
        \scalebox{0.28}{% This file was created by matlab2tikz.
%
%The latest updates can be retrieved from
%  http://www.mathworks.com/matlabcentral/fileexchange/22022-matlab2tikz-matlab2tikz
%where you can also make suggestions and rate matlab2tikz.
%
\definecolor{mycolor1}{rgb}{0.20000,0.60000,0.80000}%
%
\begin{tikzpicture}

\begin{axis}[%
width=4.844in,
height=3.396in,
at={(0.812in,0.458in)},
scale only axis,
bar shift auto,
xmin=0,
xmax=60,
xlabel style={font=\color{white!15!black}},
xlabel={Normalized Power Metric (dB)},
ymin=0.5,
ymax=8.5,
ytick={1,2,3,4,5,6,7,8},
%yticklabels={{17},{18},{19},{20},{21},{22},{23},{24}},
yticklabels={$0^\circ$-1st iter, $0^\circ$-2nd iter, $15^\circ$-1st iter, $15^\circ$-2nd iter, $30^\circ$-1st iter, $30^\circ$-2nd iter, $45^\circ$-1st iter, $45^\circ$-2nd iter},
ylabel style={font=\color{white!15!black}},
ylabel={RIS Configuration},
axis background/.style={fill=white},
title style={font=\bfseries},
%title={$\text{Minimum Power Analysis - RX Position 15}^\circ$},
axis x line*=bottom,
axis y line*=left,
xmajorgrids,
ymajorgrids
]
\addplot[xbar, bar width=0.8, fill=mycolor1, draw=black, area legend] table[row sep=crcr] {%
5.60259000000001	1\\
0	2\\
10.54198	3\\
7.51519	4\\
25.36362	5\\
25.26755	6\\
26.36424	7\\
26.18726	8\\
};
\addplot[forget plot, color=white!15!black, line width=1.5pt] table[row sep=crcr] {%
0	0.5\\
0	8.5\\
};
\addplot[only marks, mark=*, mark options={}, mark size=2.2361pt, color=red, fill=red, forget plot] table[row sep=crcr]
        \subcaption{Rx position at $15^\circ$}
        \label{fig:ind_beamsw_minin15}
    \end{minipage}
    \begin{minipage}{0.24\textwidth}
        \centering
        \scalebox{0.28}{% This file was created by matlab2tikz.
%
%The latest updates can be retrieved from
%  http://www.mathworks.com/matlabcentral/fileexchange/22022-matlab2tikz-matlab2tikz
%where you can also make suggestions and rate matlab2tikz.
%
\definecolor{mycolor1}{rgb}{0.20000,0.60000,0.80000}%
%
\begin{tikzpicture}

\begin{axis}[%
width=4.844in,
height=3.396in,
at={(0.812in,0.458in)},
scale only axis,
bar shift auto,
xmin=0,
xmax=60,
xlabel style={font=\color{white!15!black}},
xlabel={Normalized Power Metric (dB)},
ymin=0.5,
ymax=8.5,
ytick={1,2,3,4,5,6,7,8},
%yticklabels={{17},{18},{19},{20},{21},{22},{23},{24}},
yticklabels={$0^\circ$-1st iter, $0^\circ$-2nd iter, $15^\circ$-1st iter, $15^\circ$-2nd iter, $30^\circ$-1st iter, $30^\circ$-2nd iter, $45^\circ$-1st iter, $45^\circ$-2nd iter},
ylabel style={font=\color{white!15!black}},
ylabel={RIS Configuration},
axis background/.style={fill=white},
title style={font=\bfseries},
%title={$\text{Minimum Power Analysis - RX Position 30}^\circ$},
axis x line*=bottom,
axis y line*=left,
xmajorgrids,
ymajorgrids
]
\addplot[xbar, bar width=0.8, fill=mycolor1, draw=black, area legend] table[row sep=crcr] {%
26.38895	1\\
26.16275	2\\
41.42751	3\\
42.59774	4\\
10.4021	5\\
0	6\\
33.87541	7\\
35.18893	8\\
};
\addplot[forget plot, color=white!15!black, line width=1.5pt] table[row sep=crcr] {%
0	0.5\\
0	8.5\\
};
\addplot[only marks, mark=*, mark options={}, mark size=2.2361pt, color=red, fill=red, forget plot] table[row sep=crcr]
        \subcaption{Rx position at $30^\circ$}
        \label{fig:ind_beamsw_minin30}
    \end{minipage}
    \begin{minipage}{0.24\textwidth}
        \centering
        \scalebox{0.28}{% This file was created by matlab2tikz.
%
%The latest updates can be retrieved from
%  http://www.mathworks.com/matlabcentral/fileexchange/22022-matlab2tikz-matlab2tikz
%where you can also make suggestions and rate matlab2tikz.
%
\definecolor{mycolor1}{rgb}{0.20000,0.60000,0.80000}%
%
\begin{tikzpicture}

\begin{axis}[%
width=4.844in,
height=3.396in,
at={(0.812in,0.458in)},
scale only axis,
bar shift auto,
xmin=0,
xmax=60,
xlabel style={font=\color{white!15!black}},
xlabel={Normalized Power Metric (dB)},
ymin=0.5,
ymax=8.5,
ytick={1,2,3,4,5,6,7,8},
%yticklabels={{17},{18},{19},{20},{21},{22},{23},{24}},
yticklabels={$0^\circ$-1st iter, $0^\circ$-2nd iter, $15^\circ$-1st iter, $15^\circ$-2nd iter, $30^\circ$-1st iter, $30^\circ$-2nd iter, $45^\circ$-1st iter, $45^\circ$-2nd iter},
ylabel style={font=\color{white!15!black}},
ylabel={RIS Configuration},
axis background/.style={fill=white},
title style={font=\bfseries},
%title={$\text{Minimum Power Analysis - RX Position 45}^\circ$},
axis x line*=bottom,
axis y line*=left,
xmajorgrids,
ymajorgrids
]
\addplot[xbar, bar width=0.8, fill=mycolor1, draw=black, area legend] table[row sep=crcr] {%
7.12939999999999	1\\
12.7032	2\\
37.64182	3\\
38.28399	4\\
23.72995	5\\
24.439	6\\
2.54640000000001	7\\
0	8\\
};
\addplot[forget plot, color=white!15!black, line width=1.5pt] table[row sep=crcr] {%
0	0.5\\
0	8.5\\
};
\addplot[only marks, mark=*, mark options={}, mark size=2.2361pt, color=red, fill=red, forget plot] table[row sep=crcr]
        \subcaption{Rx position at $45^\circ$}
        \label{fig:ind_beamsw_minin45}
    \end{minipage}
    
    \caption{Indoor setup: Beam sweeping results, executing two iterations of Algorithm~1 in real time (Minimization).}
    \label{fig:m3_indoor}
\end{figure*}


\bibliographystyle{IEEEtran}
%\bibliographystyle{IEEEabrv}
%\bibliography{IEEEfull,irs}
\bibliography{irs}

\end{document}
