% \vspace{20mm}
\newpage

\vspace*{\textheight}
\newpage

\noindent {\LARGE \textbf{Appendix}} \\

\section{Algorithm of \ourapproach}
\label{alg:codesim}

Algorithm~1 shows the pseudo-code of our prompting technique.

\begin{algorithm}[h]
\small
\caption{\tool}
\begin{algorithmic}[1]

\State $p \gets$ maximum number of planning steps
\State $d \gets$ maximum number of debugging steps \\
% \State $problem \gets$ problem to be solved\\

\For{$i \gets 1$ to $p$}
    \State \# Start of Planning Agent
    \State $plan \gets$ GeneratePlan($problem$)
    \State $feedback \gets$ ValidatePlan($problem$, $plan$)
    \If{$feedback$ is negative}
        \State $plan \gets$ RefinePlan($problem$, $plan$, $feedback$)
    \EndIf
    \State \# End of Planning Agent
    \\
    \State \# Start of Coding Agent
    \State $code \gets$ GenerateCode($problem$, $plan$) %\Comment{Coding Agent}
    
    \State $passed,\space log \gets$ test($code$, $sample\_io$) %Test code using sample IO\\
    \If{$passed$}
        \State Return $code$
    \Else
        % \State Pass $code$ to Debugging Agent
        \State \# Start of Debugging Agent
        \For{$j \gets 1$ to $d$}
            \State $code \gets$ DebugCode(\\
                \quad \quad \quad \quad \quad $problem$,\\
                \quad \quad \quad \quad \quad $plan$, \\
                \quad \quad \quad \quad \quad $code$, \\
                \quad \quad \quad \quad \quad $log$ \\
            \quad \quad \quad \quad) \\ %\Comment{Debugging Agent}
            \State $passed,\space log \gets$ test($code$, $sample\_io$)
            \If{$passed$}
                \State Return $code$
            \EndIf
        \EndFor
        \State \# End of Debugging Agent
    \EndIf
    \State \# End of Coding Agent
    \\
\EndFor
\State Return $code$
\end{algorithmic}
\end{algorithm}

\section{Exclusion of AgentCoder}
\label{app:elimination-of-agentcoder}
We have not included AgentCoder \cite{huang2023agentcoder} in our comparison due to reproducibility issues which undoubtedly plays a critical role in fair comparison as indicted in \citet{laskar-etal-2024-systematic}, as we were unable to replicate their results. In our attempts to reproduce their work on the HumanEval benchmark using ChatGPT, we achieved $56.7$\% accuracy after four iterations, consuming $11.9$ million tokens. When using GPT-4, we attained only $17.7$\% accuracy after two iterations, with $10.4$ million tokens consumed. The token consumption in both cases is significantly higher compared to MapCoder ($1.7$ million tokens with ChatGPT and $2.1$ million with GPT-4) and \toolnospace($0.89$ million tokens in ChatGPT and $0.85$ million in GPT-4). These two experiments resulted in a cost of approximately \$$500$ USD, but we were still unable to come close to AgentCoder's reported claims of $79.9$\% accuracy with ChatGPT and $96.3$\% with GPT-4.

Furthermore, we found unaddressed issues on their GitHub page (\href{https://github.com/huangd1999/AgentCoder/issues/8}{link}) related to reproducibility. Additionally, for the MBPP dataset, they used all test cases as public test cases (\href{https://github.com/huangd1999/AgentCoder/issues/3}{link}), which deviates from standard practices. As a result, we did not consider those results in our comparison either.

\section{Details Promptings of \tool}
\label{app:prompts}

The \emph{Planning Agent} interacts with the LLM three times to generate a plan. In the first API call, it instructs the LLM to comprehend the problem, generate an example problem, recommend a suitable algorithm, and finally produce the plan (Figure \ref{prompt:plan-generation}). In the second API call, the LLM is instructed to verify the plan through simulation (Figure \ref{prompt:plan-verification}). If the plan is satisfactory, it is returned by the agent. Otherwise, the LLM is called again to refine the plan based on the feedback from the simulation (Figure \ref{prompt:plan-refinement}).

The next step involves the \emph{Coding Agent}, which receives the plan from the \emph{Planning Agent} and uses the prompt outlined in Figure \ref{prompt:code-generation} to generate code.

If the code fails to pass the sample input/output, \tool activates its final agent, the \emph{Debugging Agent}, using the prompt shown in Figure \ref{prompt:debugging}.

These figures also include the rationale behind the inclusion of each sentence in the prompt.

\section{Example Problem}
We present a complete example of problem solving using \tool below:

% \onecolumn
% \newpage

\begin{figure*}[h]
    \centering
    \includegraphics[width=0.90\textwidth]{figures/steps/Plan-Generation.pdf}
    \caption{\emph{Planning Agent}: Prompt for Plan Generation. }
    \label{prompt:plan-generation}
\end{figure*} 

\begin{figure*}[h]
    \centering
    % \hspace*{-0.5cm}
    \includegraphics[width=0.90\textwidth]{figures/steps/Plan-Verification.pdf}
    \caption{\emph{Planning Agent}: Prompt for Plan Verification with the help of Simulation.}
    \label{prompt:plan-verification}
\end{figure*} 

\begin{figure*}[h]
    \centering% \hspace*{-0.5cm}
    \includegraphics[width=0.90\textwidth]{figures/steps/Plan-Refinement.pdf}
    \caption{\emph{Planning Agent}: Prompt for Plan Refinement.}
    \label{prompt:plan-refinement}
\end{figure*} 

\begin{figure*}[h]
    \centering% \hspace*{-0.5cm}
    \includegraphics[width=0.90\textwidth]{figures/steps/Code-Generation.pdf}
    \caption{\emph{Coding Agent}: Prompt for Code Generation.}
    \label{prompt:code-generation}
\end{figure*} 
\onecolumn

\begin{figure*}[h]
    \centering% \hspace*{-0.5cm}
    \includegraphics[width=0.90\textwidth]{figures/steps/Debugging.pdf}
    \caption{\emph{Debugging Agent}: Prompt for Debugging.}
    \label{prompt:debugging}    
\end{figure*}

% \onecolumn

\begin{tcolorbox}[breakable]
\label{app:example-problem}
    \section{Example}
\label{app:example}

Table~\ref{tab:example} shows a counseling session between a client agent and CAMI, with the client's profile shown in Table~\ref{tab:example profile}. During the session, CAMI explores Health and Relationships topics in the first 6 turns.  It subsequently performs Step Into operation from Relationships to Parenting topics in Turn 7, and from Parenting to Paternal Bond topics in Turn 9, before a Switch operation to Habituation topic in Turn 11.  The full exploration path is visualized in Figure~\ref{fig:exploration path}.

\begin{table*}[ht]
\centering
\begin{tabularx}{\textwidth}{c X}
\toprule
        {\bf Turn ID} & {\bf Utterances}                     \\ \midrule
1       & Counselor: Hello. How are you?                                                                                                                                                                                                                                                                                                                     \\ \hline
2       & Client: I am good. What about you?                                                                                                                                                                                                                                                                                                                 \\ \hline
3       & Counselor: I'm doing well, thanks for asking! Can you share how you've been feeling about your health lately, especially in relation to smoking?                                                                                                                                                                                                   \\ \hline
4       & Client: I've been pretty busy with work and juggling family stuff. Sometimes I just think about how nice it is to take a break with my coworkers outside, chatting and during those moments, life feels a bit more relaxed. You know how it is—it’s nice to unwind.                                                                                \\ \hline
5       & Counselor: It sounds like smoking is a comforting routine for you amidst your busy life with work and your daughter. I wonder how it might impact the time you spend with your family.                                                                                                                                                             \\ \hline
6       & Client: I get what you mean, but I really don't see my smoking as an issue. It helps me manage stress, and I don’t think it affects my time with my daughter. We still have good moments together, and I wouldn't want to change that.                                                                                                             \\ \hline
7       & Counselor: It sounds like you find comfort in smoking to manage stress, and that feels crucial to your parenting now. I wonder, though, how you might feel if your daughter started to see smoking as a way to unwind, too. What impact would that have on your relationship with her?                                                             \\ \hline
8       & Client: That’s a tough thought. I remember starting young and how hard it was to quit later. I wouldn’t want that for her. It makes me reflect on what I’m modeling.                                                                                                   \\ \hline
9       & Counselor: It's great that you're reflecting on your modeling for your daughter. Perhaps seeing smoking as something that may weaken your bond could help reshape how you approach stress—considering healthier ways to connect can actually strengthen your relationship and set a positive example for her.                                      \\ \hline
10      & Client: I appreciate that perspective. I do want to set a strong example for her. My experience with my own family and their struggles makes me cautious. I see the patterns and don’t want her to feel that smoking is a go-to solution for stress, like I did.                                                                                   \\ \hline
11      & Counselor: It sounds like your family’s struggles have really shaped your perspective on smoking. You want to protect your daughter from those same patterns, knowing that smoking can not only impact her stress management but also her overall learning and adaptation to challenges. That’s a powerful motivation.                             \\ \hline
12      & Client: Exactly. I want her to learn healthier coping strategies instead of following in my footsteps. I realize that I need to make a change for both our sakes. It’s not just about me; it’s about creating an environment where she can thrive without those habits. It's time to find better ways to manage stress that we can share together.   
\\ \bottomrule
\end{tabularx}
\caption{Counseling Session between CAMI and a Simulated Client.}
\label{tab:example}
\end{table*}



\begin{table*}[ht]
\centering
\begin{tabularx}{\textwidth}{l X}
\toprule
{\bf Profile Item}        & {\bf Description}                     \\ \midrule
Behavioral Problem  & Smoking \\ \hline
State of Mind       & Precontemplation \\ \hline
Personas            & You smoked when you were young.", "You have a 12-year-old daughter. \newline The friends of the your daughter have tried smoking. \newline When you are at work, you all go out and smoke and chit chat and talk about things. \newline You like to have a cigarette after eating. \\ \hline
Motivation          & You are motivated because of the risk of smoking in habituation for your daughter, as you don't want your daughter to think smoking is good thing.    \\ \hline
Beliefs             & You feel smoking enjoyable and relaxing. \newline You would be left out if you would hate to not smoke. \newline You use smoking as a way to manage weight and control eating habits. \newline You have been smoking for a long time and feels it's almost impossible to quit. \\ \hline
Motivation Topics          & Interpersonal Relationships, Parenting, Habituation
\\ \bottomrule
\end{tabularx}
\caption{Client's Profile in the Example.}
\label{tab:example profile}
\end{table*}


\begin{figure*}
    \centering
    \includegraphics[width=\linewidth]{figs/example_explore_path.pdf}
    \caption{Topic Exploration Path by the Counselor in the Example.}
    \label{fig:exploration path}
\end{figure*}


\end{tcolorbox}
\twocolumn  % Switch back to two-column mode