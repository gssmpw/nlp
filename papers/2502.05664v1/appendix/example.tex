\vspace{2mm}
\textbf{An Example from HumanEval dataset for demonstrating how \tool works}
\hrule 
\vspace{5mm}
Input for Planning: 1

You are a programmer tasked with generating appropriate plan to solve a given problem using the \*\*Python3\*\* programming language.\\
 \\
\#\# Problem

\begin{verbatim}
def generate_integers(a, b):
    """
    Given two positive integers a and b, return the even digits between a
    and b, in ascending order.

    For example:
    generate_integers(2, 8) => [2, 4, 6, 8]
    generate_integers(8, 2) => [2, 4, 6, 8]
    generate_integers(10, 14) => []
    """
\end{verbatim}

\*\*Expected Output:\*\*\\
 \\
Your response must be structured as follows:\\

\#\#\# Problem Understanding\\
 \\
- Think about the original problem. Develop an initial understanding about the problem.\\

\#\#\# Recall Example Problem\\
 \\
Recall a relevant and distinct problems (different from problem mentioned above) and\\
- Describe it\\
- Generate Python3 code step by step to solve that problem\\
- Discuss the algorithm to solve this problem\\
- Finally generate a planning to solve that problem\\

\#\#\# Algorithm to solve the original problem\\
 \\
- Write down the algorithm that is well suited for the original problem\\
- Give some tutorials to about the algorithm for example:\\
    - How to approach this type of algorithm\\
    - Important things to consider\\

\#\#\# Plan\\
 \\
- Write down a detailed, step-by-step plan to solve the \*\*original problem\*\*.\\

--------\\
\*\*Important Instruction:\*\*\\
- Strictly follow the instructions.\\
- Do not generate code.\\

\vspace{0.75cm} 
\hrule
\vspace{0.75cm}

Response from Planning: 1\\

\#\#\# Problem Understanding

The problem requires generating a list of even integers between two given positive integers, $a$ and $b$, inclusive. The list should be in ascending order, regardless of whether $a$ is less than or greater than $b$. If there are no even integers in the range, the function should return an empty list. \\

\#\#\# Recall Example Problem

\*\*Example Problem:\*\*\\

Find all odd numbers between two given integers $x$ and $y$, inclusive, and return them in descending order.\\

\*\*Step-by-step Solution:\*\*\\

1. Determine the range between $x$ and $y$.\\
2. Identify all odd numbers within this range.\\
3. Sort the odd numbers in descending order.\\
4. Return the sorted list.\\

\*\*Algorithm:\*\*\\
 \\
1. Check if $x$ is greater than $y$. If so, swap them to ensure the range is correct.\\
2. Iterate through the range from $x$ to $y$.\\
3. For each number, check if it is odd (i.e., $number \% 2 != 0$).\\
4. Collect all odd numbers in a list.\\
5. Sort the list in descending order.\\
6. Return the sorted list.\\

\*\*Planning:\*\*\\
 \\
- Define a function $find\_odd\_numbers(x, y)$.\\
- Use a conditional to swap $x$ and $y$ if $x > y$.\\
- Use a loop to iterate through the range from $x$ to $y$.\\
- Check each number for oddness and add it to a list if it is odd.\\
- Sort the list in descending order.\\
- Return the list.\\

\#\#\# Algorithm to Solve the Original Problem\\
 \\
1. \*\*Determine the Range:\*\*\\
   - Identify the start and end of the range by taking the minimum and maximum of $a$ and $b$.\\

2. \*\*Iterate Through the Range:\*\*\\
   - Loop through each number in the range from the minimum to the maximum of $a$ and $b$.\\

3. \*\*Check for Even Numbers:\*\*\\
   - For each number, check if it is even by using the condition $number \% 2 == 0$.\\

4. \*\*Collect Even Numbers:\*\*\\
   - Add each even number to a list.\\

5. \*\*Return the List:\*\*\\
   - Since the numbers are collected in ascending order, return the list directly.\\
 \\
\*\*Tutorials:\*\*\\
 \\
- \*\*Approach:\*\*\\
  - Always ensure the range is correctly defined, especially when the order of inputs can vary.\\
  - Use modulus operation to check for evenness or oddness.\\
 \\
- \*\*Important Considerations:\*\*\\
  - Handle edge cases where there might be no even numbers in the range.\\
  - Ensure the function is robust to handle any positive integer inputs.\\

\#\#\# Plan \\
 \\
1. \*\*Define the Function:\*\* \\
   - Create a function $generate\_integers(a, b)$. \\

2. \*\*Determine the Range:\*\* \\
   - Use $min(a, b)$ and $max(a, b)$ to define the start and end of the range. \\

3. \*\*Initialize an Empty List:\*\* \\
   - Prepare an empty list to store even numbers. \\

4. \*\*Loop Through the Range:\*\* \\
   - Use a $for$ loop to iterate from the start to the end of the range. \\

5. \*\*Check for Evenness:\*\* \\
   - Within the loop, use $if number \% 2 == 0$ to check if a number is even. \\

6. \*\*Collect Even Numbers:\*\* \\
   - Append even numbers to the list. \\

7. \*\*Return the Result:\*\* \\
   - Return the list of even numbers, which will be in ascending order. \\

By following this plan, the function will correctly generate a list of even integers between the given numbers $a$ and $b$. \\
 \\

\vspace{0.5cm} \hrule   \vspace{0.5cm}
Input for Simulation: 1


You are a programmer tasked with verifying a plan to solve a given problem using the \*\*Python3\*\* programming language. \\

\#\# Problem: \\
\begin{verbatim}
def generate_integers(a, b):
    """
    Given two positive integers a and b, return the even digits between a
    and b, in ascending order.

    For example:
    generate_integers(2, 8) => [2, 4, 6, 8]
    generate_integers(8, 2) => [2, 4, 6, 8]
    generate_integers(10, 14) => []
    """
\end{verbatim}

\#\#\# Plan \\

1. \*\*Define the Function:\*\* \\
   - Create a function $generate\_integers(a, b)$. \\

2. \*\*Determine the Range:\*\* \\
   - Use $min(a, b)$ and $max(a, b)$ to define the start and end of the range.v

3. \*\*Initialize an Empty List:\*\* \\
   - Prepare an empty list to store even numbers. \\

4. \*\*Loop Through the Range:\*\* \\
   - Use a $for$ loop to iterate from the start to the end of the range. \\

5. \*\*Check for Evenness:\*\* \\
   - Within the loop, use $if \space number \% 2 == 0$ to check if a number is even. \\

6. \*\*Collect Even Numbers:\*\* \\
   - Append even numbers to the list. \\

7. \*\*Return the Result:\*\* \\
   - Return the list of even numbers, which will be in ascending order. \\

By following this plan, the function will correctly generate a list of even integers between the given numbers $a$ and $b$. \\

\*\*Expected Output:\*\* \\
 \\
Your response must be structured as follows: \\

\#\#\# Simulation \\
 \\
- Take a sample input and apply plan step by step to get the output. \\
- Compare the generated output with the sample output to verify if your plan works as expected. \\

\#\#\# Plan Evaluation \\
 \\
- If the simulation is successful write \*\*No Need to Modify Plan\*\*. \\
- Otherwise write \*\*Plan Modification Needed\*\*. \\


\vspace{0.5cm} \hrule   \vspace{0.5cm}
Response from Simulation: 1


\#\#\# Simulation \\
 \\
Let's apply the plan step by step using the sample input $generate\_integers(2, 8)$. \\

1. \*\*Define the Function:\*\* \\
   - We define the function $generate\_integers(a, b)$. \\

2. \*\*Determine the Range:\*\* \\
   - Calculate $start = min(2, 8) = 2$ and $end = max(2, 8) = 8$. \\

3. \*\*Initialize an Empty List:\*\* \\
   - Create an empty list $even_numbers = []$. \\

4. \*\*Loop Through the Range:\*\* \\
   - Iterate over the range from $start$ to $end$ (inclusive). \\

5. \*\*Check for Evenness:\*\* \\
   - For each number in the range, check if it is even using $if \space number \% 2 == 0$. \\

6. \*\*Collect Even Numbers:\*\* \\
   - Append even numbers to the list $even_numbers$. \\

7. \*\*Return the Result:\*\* \\
   - Return the list $even_numbers$. \\

Applying these steps: \\

- Range: 2 to 8 \\
- Iteration: \\
  - 2 is even, append to list: $[2]$ \\
  - 3 is not even \\
  - 4 is even, append to list: $[2, 4]$ \\
  - 5 is not even \\
  - 6 is even, append to list: $[2, 4, 6]$ \\
  - 7 is not even \\
  - 8 is even, append to list: $[2, 4, 6, 8]$ \\

The output is $[2, 4, 6, 8]$, which matches the expected output. \\

\#\#\# Plan Evaluation \\
 \\
\*\*No Need to Modify Plan\*\* \\

The plan works as expected for the given problem. The function correctly generates a list of even integers between the given numbers $a$ and $b$, regardless of their order. \\


\vspace{0.5cm} \hrule   \vspace{0.5cm}
Input for final code generation: \\
 \\

You are a programmer tasked with solving a given problem using the \*\*Python3\*\* programming language. See the plan to solve the plan and implement code to solve it. \\

\#\# Problem: \\
\begin{verbatim}
def generate_integers(a, b):
    """
    Given two positive integers a and b, return the even digits between a
    and b, in ascending order.

    For example:
    generate_integers(2, 8) => [2, 4, 6, 8]
    generate_integers(8, 2) => [2, 4, 6, 8]
    generate_integers(10, 14) => []
    """
\end{verbatim}

\#\#\# Plan \\
 \\
1. \*\*Define the Function:\*\* \\
   - Create a function $generate\_integers(a, b)$. \\

2. \*\*Determine the Range:\*\* \\
   - Use $min(a, b)$ and $max(a, b)$ to define the start and end of the range. \\

3. \*\*Initialize an Empty List:\*\* \\
   - Prepare an empty list to store even numbers. \\

4. \*\*Loop Through the Range:\*\* \\
   - Use a $for$ loop to iterate from the start to the end of the range. \\

5. \*\*Check for Evenness:\*\* \\
   - Within the loop, use $if  \space number \% 2 == 0$ to check if a number is even. \\

6. \*\*Collect Even Numbers:\*\* \\
   - Append even numbers to the list. \\

7. \*\*Return the Result:\*\* \\
   - Return the list of even numbers, which will be in ascending order. \\
 \\
By following this plan, the function will correctly generate a list of even integers between the given numbers $a$ and $b$. \\

-------- \\
\*\*Important Instructions:\*\* \\
- Do not add any explanation. \\
- The generated \*\*Python3\*\* code must be inside a triple backtick (\$\$\$) code block. \\ \\



\vspace{0.5cm} \hrule   \vspace{0.5cm}
Response from final code generation: \\
 \\

\begin{verbatim}
def generate_integers(a, b):
    start = min(a, b)
    end = max(a, b)
    even_numbers = []
    for number in range(start, end + 1):
        if number % 2 == 0:
            even_numbers.append(number)
    return even_numbers
\end{verbatim}


\vspace{0.5cm} \hrule   \vspace{0.5cm}
Input for Improving code: 1, 1 \\
 \\
You are a programmer who has received a solution of a problem written in \*\*Python3\*\* that fails to pass certain test cases. Your task is to modify the code in such a way so that it can pass all the test cases. Do not generate same code. \\

\#\# Problem: \\
\begin{verbatim}
def generate_integers(a, b):
    """
    Given two positive integers a and b, return the even digits between a
    and b, in ascending order.

    For example:
    generate_integers(2, 8) => [2, 4, 6, 8]
    generate_integers(8, 2) => [2, 4, 6, 8]
    generate_integers(10, 14) => []
    """
\end{verbatim}
\#\#\# Plan \\
 \\
1. \*\*Define the Function:\*\* \\
   - Create a function $generate\_integers(a, b)$. \\

2. \*\*Determine the Range:\*\* \\
   - Use $min(a, b)$ and $max(a, b)$ to define the start and end of the range. \\

3. \*\*Initialize an Empty List:\*\* \\
   - Prepare an empty list to store even numbers. \\

4. \*\*Loop Through the Range:\*\* \\
   - Use a $for$ loop to iterate from the start to the end of the range. \\

5. \*\*Check for Evenness:\*\* \\
   - Within the loop, use $if \space number \% 2 == 0$ to check if a number is even. \\

6. \*\*Collect Even Numbers:\*\* \\
   - Append even numbers to the list. \\

7. \*\*Return the Result:\*\* \\
   - Return the list of even numbers, which will be in ascending order. \\

By following this plan, the function will correctly generate a list of even integers between the given numbers $a$ and $b$. \\

\#\#\# Buggy Code \\
 \\
\begin{verbatim}
def generate_integers(a, b):
    start = min(a, b)
    end = max(a, b)
    even_numbers = []
    for number in range(start, end + 1):
        if number % 2 == 0:
            even_numbers.append(number)
    return even_numbers
\end{verbatim}

\#\#\# Test Report \\
 \\
Test Cases where the generated code failed to generate the expected output: \\
$assert generate_integers(10, 14) == []$ \\

\*\*Expected Output:\*\* \\
 \\
Your response must be structured as follows: \\

\#\#\# Simulation with failed test case \\
To detect where is the bug: \\
    - Take a sample test case where it fails. \\
    - Take the input go through each step according to the plan \\
    - You will get a output that must be different from the expected output.  \\

\#\#\# Debugging Notes \\
Based on this simulation detect any of the following cases: \\
    - Plan is wrong \\
    - Plan is correct but plan to code generation is wrong. \\

- Finally, discuss how to correct this code. \\

\#\#\# Modified Code \\
 \\
\texttt{```}Python3 \\
\# Your corrected code, with comments explaining each correction. \\
\texttt{```} \\

-------- \\
\*\*Important Instructions:\*\* \\
- Strictly follow the instructions. \\
- Do not add testing code for example assert statement in your code. \\
- Do not be overconfident that the generated code is correct. It is wrong. \\
- The modified \*\*Python3\*\* code must be enclosed within triple backticks (\texttt{```}). \\
- Your response must contain \*\*Simulation with failed test case\*\*, \*\*Debugging Notes\*\*, and \*\*Modified Code\*\* section. \\



\vspace{0.5cm} \hrule   \vspace{0.5cm}
Response from Improving code: 1, 1 \\
 \\

\#\#\# Simulation with failed test case \\
 \\
Let's take the test case $generate\_integers(10, 14)$:v

1. \*\*Determine the Range:\*\* \\
   - $start = min(10, 14) = 10$ \\
   - $end = max(10, 14) = 14$ \\

2. \*\*Loop Through the Range:\*\* \\
   - The loop iterates over $10, 11, 12, 13, 14$. \\

3. \*\*Check for Evenness:\*\* \\
   - $10 \% 2 == 0$ (even, added to list) \\
   - $11 \% 2 != 0$ (odd, not added) \\
   - $12 \% 2 == 0$ (even, added to list) \\
   - $13 \% 2 != 0$ (odd, not added) \\
   - $14 \% 2 == 0$ (even, added to list) \\

4. \*\*Collect Even Numbers:\*\* \\
   - The list becomes $[10, 12, 14]$. \\

5. \*\*Return the Result:\*\* \\
   - The function returns $[10, 12, 14]$. \\

\*\*Expected Output:\*\* $[]$ \\

\#\#\# Debugging Notes \\
 \\
- \*\*Plan is correct\*\*: The plan correctly identifies the need to find even numbers between $a$ and $b$. \\
- \*\*Plan to code generation is wrong\*\*: The code does not account for the requirement that the even digits themselves must be between $a$ and $b$. It incorrectly includes numbers like $10, 12, 14$ when the digits themselves (1, 2, 4) are not between $10$ and $14$. \\

\#\#\# Modified Code \\
\begin{verbatim}
def generate_integers(a, b):
    start = min(a, b)
    end = max(a, b)
    even_numbers = []
    for number in range(start, end + 1):
        # Check each digit in the number
        for digit in str(number):
            if int(digit) % 2 == 0 and start <= int(digit) <= end:
                even_numbers.append(int(digit))
    # Remove duplicates and sort the result
    return sorted(set(even_numbers))
\end{verbatim}

\*\*Explanation of Corrections:\*\* \\
 \\
- \*\*Digit Check\*\*: The code now checks each digit of the numbers in the range to see if it is even and within the specified range. \\
- \*\*Set for Uniqueness\*\*: A set is used to ensure that each even digit is only included once. \\
- \*\*Sorting\*\*: The final list is sorted to ensure ascending order. \\
