% This must be in the first 5 lines to tell arXiv to use pdfLaTeX, which is strongly recommended.
\pdfoutput=1
% In particular, the hyperref package requires pdfLaTeX in order to break URLs across lines.

\documentclass[11pt]{article}

\usepackage[table]{xcolor}


% Change "review" to "final" to generate the final (sometimes called camera-ready) version.
% Change to "preprint" to generate a non-anonymous version with page numbers.
\usepackage[final]{acl}
\usepackage{graphicx}
% Standard package includes
\usepackage{times}
\usepackage{latexsym}

% For proper rendering and hyphenation of words containing Latin characters (including in bib files)
\usepackage[T1]{fontenc}
% For Vietnamese characters
% \usepackage[T5]{fontenc}
% See https://www.latex-project.org/help/documentation/encguide.pdf for other character sets

% This assumes your files are encoded as UTF8
\usepackage[utf8]{inputenc}

% This is not strictly necessary, and may be commented out,
% but it will improve the layout of the manuscript,
% and will typically save some space.
\usepackage{microtype}
\usepackage{pdfpages}
% This is also not strictly necessary, and may be commented out.
% However, it will improve the aesthetics of text in
% the typewriter font.
\usepackage{inconsolata}
\usepackage{amssymb}% http://ctan.org/pkg/amssymb
\usepackage{pifont}% http://ctan.org/pkg/pifont
\usepackage{multirow}
\usepackage{array}
\usepackage{float}
\usepackage{pdfpages}
\usepackage{algorithm}
\usepackage{algpseudocode}
\usepackage[most]{tcolorbox}


% If the title and author information does not fit in the area allocated, uncomment the following
%
%\setlength\titlebox{<dim>}
%
% and set <dim> to something 5cm or larger.

% Defining some constants
% \def\codebaselink{https://github.com/kagnlp/codesim.github.io}
\def\codebaselink{https://kagnlp.github.io/codesim.github.io/}

\def\codebase{\href{\codebaselink}{\codebaselink}}
\newcommand{\cmark}{\color{green}{\ding{51}}}%
\newcommand{\xmark}{\color{red}{\ding{55}}}%

\def\rotate{\rotatebox[origin=c]{90}}
\def\vlinerow{\multirow{1}{*}}

\newcommand{\bluecell}{\cellcolor{blue!25}}
\newcommand{\violetcell}{\cellcolor{violet!25}}
\newcommand{\greencell}{\cellcolor{green!50}}
\newcommand{\redcell}{\cellcolor{red!25}}
\newcommand{\yellowcell}{\cellcolor{yellow!25}}
\newcommand{\cyancell}{\cellcolor{cyan!50}}
\newcommand{\graycell}{\cellcolor{black!10}}

\newcommand{\eunus}[1]{\textcolor{red}{\textsc{Eunus:} #1}}
\newcommand{\rizwan}[1]{\textcolor{blue}{\bf\small [Rizwan: #1]}}
\newcommand{\ashraful}[1]{\textcolor{purple}{\bf\small [Ashraful: #1]}}
\newcommand{\ashrafulupdate}[1]{\textcolor{purple}{\bf\small [Ashraful Update: #1]}}

\newcommand{\tool}{\textsc{CodeSim}~}
\newcommand{\toolnospace}{\textsc{CodeSim}}


\newcommand\modelfont[1]{{\usefont{T1}{Discognate}{m}{n}#1}}
\newcommand{\model}{\modelfont{CodeSim}\xspace}

\def\ourapproach{\toolnospace}

\newenvironment{nobreakwords}
  {\sloppy\hyphenpenalty=10000\exhyphenpenalty=10000 }
  {\par}


\definecolor{BLUE}{RGB}{30,60,220} % Adjust based on the gradient color (blue)
\definecolor{GREEN}{RGB}{2,100,51} % Adjust based on the gradient color (green)


\title{
\textsc{\textbf{
\textcolor{BLUE}{\textsc{Code}}\textcolor{GREEN}{\textsc{SIM}}}}: Multi-Agent Code Generation and Problem Solving through Simulation-Driven Planning and Debugging
}




% \title{\ourapproach: Multi-Agent Prompting for Competitive Programming}
% \title{\toolnospace: {Multi-Agent Code Generation and Problem Solving through Simulation-Driven Planning and Debugging}}


% \title{\toolnospace: {Simulation-Driven Multi-Agent Code Generation and Competitive Problem Solving }}


% \title{\toolnospace: {Multi-Agent Code Generation and Competitive Problem Solving through Simulation-Driven Planning and Debugging}}


% Code Generation}

% \title{\model:~Generating Code by Retrieving the Docs}

% Author information can be set in various styles:
% For several authors from the same institution:
% \author{Author 1 \and ... \and Author n \\
%         Address line \\ ... \\ Address line}
% if the names do not fit well on one line use
%         Author 1 \\ {\bf Author 2} \\ ... \\ {\bf Author n} \\
% For authors from different institutions:
% \author{Author 1 \\ Address line \\  ... \\ Address line
%         \And  ... \And
%         Author n \\ Address line \\ ... \\ Address line}
% To start a separate ``row'' of authors use \AND, as in
% \author{Author 1 \\ Address line \\  ... \\ Address line
%         \AND
%         Author 2 \\ Address line \\ ... \\ Address line \And
%         Author 3 \\ Address line \\ ... \\ Address line}


\author{
Md. Ashraful Islam\thanks{Work done when working as a remote RA at QCRI.}$^{1}$, \ Mohammed Eunus Ali$^1$, \ Md Rizwan Parvez$^2$ \\
$^1$Bangladesh University of Engineering and Technology (BUET) \\
$^2$Qatar Computing Research Institute (QCRI) \\
\{mdashrafulpramanic, mohammed.eunus.ali\}@gmail.com, mparvez@hbku.edu.qa 
}


\begin{document}
\maketitle

\begin{abstract}  
Test time scaling is currently one of the most active research areas that shows promise after training time scaling has reached its limits.
Deep-thinking (DT) models are a class of recurrent models that can perform easy-to-hard generalization by assigning more compute to harder test samples.
However, due to their inability to determine the complexity of a test sample, DT models have to use a large amount of computation for both easy and hard test samples.
Excessive test time computation is wasteful and can cause the ``overthinking'' problem where more test time computation leads to worse results.
In this paper, we introduce a test time training method for determining the optimal amount of computation needed for each sample during test time.
We also propose Conv-LiGRU, a novel recurrent architecture for efficient and robust visual reasoning. 
Extensive experiments demonstrate that Conv-LiGRU is more stable than DT, effectively mitigates the ``overthinking'' phenomenon, and achieves superior accuracy.
\end{abstract}  
\section{Introduction}


\begin{figure}[t]
\centering
\includegraphics[width=0.6\columnwidth]{figures/evaluation_desiderata_V5.pdf}
\vspace{-0.5cm}
\caption{\systemName is a platform for conducting realistic evaluations of code LLMs, collecting human preferences of coding models with real users, real tasks, and in realistic environments, aimed at addressing the limitations of existing evaluations.
}
\label{fig:motivation}
\end{figure}

\begin{figure*}[t]
\centering
\includegraphics[width=\textwidth]{figures/system_design_v2.png}
\caption{We introduce \systemName, a VSCode extension to collect human preferences of code directly in a developer's IDE. \systemName enables developers to use code completions from various models. The system comprises a) the interface in the user's IDE which presents paired completions to users (left), b) a sampling strategy that picks model pairs to reduce latency (right, top), and c) a prompting scheme that allows diverse LLMs to perform code completions with high fidelity.
Users can select between the top completion (green box) using \texttt{tab} or the bottom completion (blue box) using \texttt{shift+tab}.}
\label{fig:overview}
\end{figure*}

As model capabilities improve, large language models (LLMs) are increasingly integrated into user environments and workflows.
For example, software developers code with AI in integrated developer environments (IDEs)~\citep{peng2023impact}, doctors rely on notes generated through ambient listening~\citep{oberst2024science}, and lawyers consider case evidence identified by electronic discovery systems~\citep{yang2024beyond}.
Increasing deployment of models in productivity tools demands evaluation that more closely reflects real-world circumstances~\citep{hutchinson2022evaluation, saxon2024benchmarks, kapoor2024ai}.
While newer benchmarks and live platforms incorporate human feedback to capture real-world usage, they almost exclusively focus on evaluating LLMs in chat conversations~\citep{zheng2023judging,dubois2023alpacafarm,chiang2024chatbot, kirk2024the}.
Model evaluation must move beyond chat-based interactions and into specialized user environments.



 

In this work, we focus on evaluating LLM-based coding assistants. 
Despite the popularity of these tools---millions of developers use Github Copilot~\citep{Copilot}---existing
evaluations of the coding capabilities of new models exhibit multiple limitations (Figure~\ref{fig:motivation}, bottom).
Traditional ML benchmarks evaluate LLM capabilities by measuring how well a model can complete static, interview-style coding tasks~\citep{chen2021evaluating,austin2021program,jain2024livecodebench, white2024livebench} and lack \emph{real users}. 
User studies recruit real users to evaluate the effectiveness of LLMs as coding assistants, but are often limited to simple programming tasks as opposed to \emph{real tasks}~\citep{vaithilingam2022expectation,ross2023programmer, mozannar2024realhumaneval}.
Recent efforts to collect human feedback such as Chatbot Arena~\citep{chiang2024chatbot} are still removed from a \emph{realistic environment}, resulting in users and data that deviate from typical software development processes.
We introduce \systemName to address these limitations (Figure~\ref{fig:motivation}, top), and we describe our three main contributions below.


\textbf{We deploy \systemName in-the-wild to collect human preferences on code.} 
\systemName is a Visual Studio Code extension, collecting preferences directly in a developer's IDE within their actual workflow (Figure~\ref{fig:overview}).
\systemName provides developers with code completions, akin to the type of support provided by Github Copilot~\citep{Copilot}. 
Over the past 3 months, \systemName has served over~\completions suggestions from 10 state-of-the-art LLMs, 
gathering \sampleCount~votes from \userCount~users.
To collect user preferences,
\systemName presents a novel interface that shows users paired code completions from two different LLMs, which are determined based on a sampling strategy that aims to 
mitigate latency while preserving coverage across model comparisons.
Additionally, we devise a prompting scheme that allows a diverse set of models to perform code completions with high fidelity.
See Section~\ref{sec:system} and Section~\ref{sec:deployment} for details about system design and deployment respectively.



\textbf{We construct a leaderboard of user preferences and find notable differences from existing static benchmarks and human preference leaderboards.}
In general, we observe that smaller models seem to overperform in static benchmarks compared to our leaderboard, while performance among larger models is mixed (Section~\ref{sec:leaderboard_calculation}).
We attribute these differences to the fact that \systemName is exposed to users and tasks that differ drastically from code evaluations in the past. 
Our data spans 103 programming languages and 24 natural languages as well as a variety of real-world applications and code structures, while static benchmarks tend to focus on a specific programming and natural language and task (e.g. coding competition problems).
Additionally, while all of \systemName interactions contain code contexts and the majority involve infilling tasks, a much smaller fraction of Chatbot Arena's coding tasks contain code context, with infilling tasks appearing even more rarely. 
We analyze our data in depth in Section~\ref{subsec:comparison}.



\textbf{We derive new insights into user preferences of code by analyzing \systemName's diverse and distinct data distribution.}
We compare user preferences across different stratifications of input data (e.g., common versus rare languages) and observe which affect observed preferences most (Section~\ref{sec:analysis}).
For example, while user preferences stay relatively consistent across various programming languages, they differ drastically between different task categories (e.g. frontend/backend versus algorithm design).
We also observe variations in user preference due to different features related to code structure 
(e.g., context length and completion patterns).
We open-source \systemName and release a curated subset of code contexts.
Altogether, our results highlight the necessity of model evaluation in realistic and domain-specific settings.






%\section{Related Work}
%\label{sec:related-work}

%\subsection{Background}

%Defect detection is critical to ensure the yield of integrated circuit manufacturing lines and reduce faults. Previous research has primarily focused on wafer map data, which engineers produce by marking faulty chips with different colors based on test results. The specific spatial distribution of defects on a wafer can provide insights into the causes, thereby helping to determine which stage of the manufacturing process is responsible for the issues. Although such research is relatively mature, the continual miniaturization of integrated circuits and the increasing complexity and density of chip components have made chip-level detection more challenging, leading to potential risks\cite{ma2023review}. Consequently, there is a need to combine this approach with magnified imaging of the wafer surface using scanning electron microscopes (SEMs) to detect, classify, and analyze specific microscopic defects, thus helping to identify the particular process steps where defects originate.

%Previously, wafer surface defect classification and detection were primarily conducted by experienced engineers. However, this method relies heavily on the engineers' expertise and involves significant time expenditure and subjectivity, lacking uniform standards. With the ongoing development of artificial intelligence, deep learning methods using multi-layer neural networks to extract and learn target features have proven highly effective for this task\cite{gao2022review}.

%In the task of defect classification, it is typical to use a model structure that initially extracts features through convolutional and pooling layers, followed by classification via fully connected layers. Researchers have recently developed numerous classification model structures tailored to specific problems. These models primarily focus on how to extract defect features effectively. For instance, Chen et al. presented a defect recognition and classification algorithm rooted in PCA and classification SVM\cite{chen2008defect}. Chang et al. utilized SVM, drawing on features like smoothness and texture intricacy, for classifying high-intensity defect images\cite{chang2013hybrid}. The classification of defect images requires the formulation of numerous classifiers tailored for myriad inspection steps and an Abundance of accurately labeled data, making data acquisition challenging. Cheon et al. proposed a single CNN model adept at feature extraction\cite{cheon2019convolutional}. They achieved a granular classification of wafer surface defects by recognizing misclassified images and employing a k-nearest neighbors (k-NN) classifier algorithm to gauge the aggregate squared distance between each image feature vector and its k-neighbors within the same category. However, when applied to new or unseen defects, such models necessitate retraining, incurring computational overheads. Moreover, with escalating CNN complexity, the computational demands surge.

%Segmentation of defects is necessary to locate defect positions and gather information such as the size of defects. Unlike classification networks, segmentation networks often use classic encoder-decoder structures such as UNet\cite{ronneberger2015u} and SegNet\cite{badrinarayanan2017segnet}, which focus on effectively leveraging both local and global feature information. Han Hui et al. proposed integrating a Region Proposal Network (RPN) with a UNet architecture to suggest defect areas before conducting defect segmentation \cite{han2020polycrystalline}. This approach enables the segmentation of various defects in wafers with only a limited set of roughly labeled images, enhancing the efficiency of training and application in environments where detailed annotations are scarce. Subhrajit Nag et al. introduced a new network structure, WaferSegClassNet, which extracts multi-scale local features in the encoder and performs classification and segmentation tasks in the decoder \cite{nag2022wafersegclassnet}. This model represents the first detection system capable of simultaneously classifying and segmenting surface defects on wafers. However, it relies on extensive data training and annotation for high accuracy and reliability. 

%Recently, Vic De Ridder et al. introduced a novel approach for defect segmentation using diffusion models\cite{de2023semi}. This approach treats the instance segmentation task as a denoising process from noise to a filter, utilizing diffusion models to predict and reconstruct instance masks for semiconductor defects. This method achieves high precision and improved defect classification and segmentation detection performance. However, the complex network structure and the computational process of the diffusion model require substantial computational resources. Moreover, the performance of this model heavily relies on high-quality and large amounts of training data. These issues make it less suitable for industrial applications. Additionally, the model has only been applied to detecting and segmenting a single type of defect(bridges) following a specific manufacturing process step, limiting its practical utility in diverse industrial scenarios.

%\subsection{Few-shot Anomaly Detection}
%Traditional anomaly detection techniques typically rely on extensive training data to train models for identifying and locating anomalies. However, these methods often face limitations in rapidly changing production environments and diverse anomaly types. Recent research has started exploring effective anomaly detection using few or zero samples to address these challenges.

%Huang et al. developed the anomaly detection method RegAD, based on image registration technology. This method pre-trains an object-agnostic registration network with various images to establish the normality of unseen objects. It identifies anomalies by aligning image features and has achieved promising results. Despite these advancements, implementing few-shot settings in anomaly detection remains an area ripe for further exploration. Recent studies show that pre-trained vision-language models such as CLIP and MiniGPT can significantly enhance performance in anomaly detection tasks.

%Dong et al. introduced the MaskCLIP framework, which employs masked self-distillation to enhance contrastive language-image pretraining\cite{zhou2022maskclip}. This approach strengthens the visual encoder's learning of local image patches and uses indirect language supervision to enhance semantic understanding. It significantly improves transferability and pretraining outcomes across various visual tasks, although it requires substantial computational resources.
%Jeong et al. crafted the WinCLIP framework by integrating state words and prompt templates to characterize normal and anomalous states more accurately\cite{Jeong_2023_CVPR}. This framework introduces a novel window-based technique for extracting and aggregating multi-scale spatial features, significantly boosting the anomaly detection performance of the pre-trained CLIP model.
%Subsequently, Li et al. have further contributed to the field by creating a new expansive multimodal model named Myriad\cite{li2023myriad}. This model, which incorporates a pre-trained Industrial Anomaly Detection (IAD) model to act as a vision expert, embeds anomaly images as tokens interpretable by the language model, thus providing both detailed descriptions and accurate anomaly detection capabilities.
%Recently, Chen et al. introduced CLIP-AD\cite{chen2023clip}, and Li et al. proposed PromptAD\cite{li2024promptad}, both employing language-guided, tiered dual-path model structures and feature manipulation strategies. These approaches effectively address issues encountered when directly calculating anomaly maps using the CLIP model, such as reversed predictions and highlighting irrelevant areas. Specifically, CLIP-AD optimizes the utilization of multi-layer features, corrects feature misalignment, and enhances model performance through additional linear layer fine-tuning. PromptAD connects normal prompts with anomaly suffixes to form anomaly prompts, enabling contrastive learning in a single-class setting.

%These studies extend the boundaries of traditional anomaly detection techniques and demonstrate how to effectively address rapidly changing and sample-scarce production environments through the synergy of few-shot learning and deep learning models. Building on this foundation, our research further explores wafer surface defect detection based on the CLIP model, especially focusing on achieving efficient and accurate anomaly detection in the highly specialized and variable semiconductor manufacturing process using a minimal amount of labeled data.


\section{\tool}
\label{sec:mapcoder}
\label{sec:codesim}
% The main focus of this work is to develop a multi-agent system where the agents will interact with one another and generate correct code. The agents should be designed in such a way so that they can help each other to build a complete environment that produce correct code as much as possible. For this reason we have taken three agents in our system imitating human programming cycle. They are planning, coding, and debugging agent. Then using a traversal plan we have accommodate all the agents. All of them are described below:

Our goal is to develop a multi-agent code generation approach capable of complex problem solving. Drawing inspiration from recent works like MapCoder and ChatDev (in a different context), we devise the agents in \tool for planning, coding, and debugging. 
While these existing approaches focus primarily on expanding steps without verifying underlying hypotheses, we address this limitation by introducing a novel verification approach. Our approach simulates input/output step-by-step, verifying generated plans and performing internal debugging, mirroring how humans understand, visualize, and refine in algorithm development. Below, we present our proposed model.




% We devise a pipeline sequence for \toolnospace, intelligently cascading the agents in a structured way and enhancing each agent's capability by augmenting in-context learning signals from previous agents in the pipeline. However, not all the agent responses/outputs are equally useful. Therefore, additionally, \tool features an adaptive agent traversal schema to interact among corresponding agents dynamically, iteratively enhancing the generated code by, for example, fixing bugs, while maximizing the usage of the LLM agents.  In this section, we first discuss the agents (as per the pipeline), their prompts, and interactions, followed by the dynamic agent traversal protocol in \tool towards code generation for competitive problem-solving. 

\begin{table*}[h]
    \centering
    \begin{tabular}{c}
    \hspace{-2.5mm}
    \includegraphics[width=0.999\textwidth]{figures/tables/Basic-Results.pdf}
    \end{tabular}
    \caption{Pass@1 results for different approaches on basic programming tasks.}
    \label{tab:basic-dataset-result}
\end{table*}


\subsection{Planning Agent}
\label{subsec:planning-agent}
% This is our first agent. This agent receives a problem description and generate a detailed plan for solving the problem. After receiving a problem if we tell the LLM to generate a plan directly without any further knowledge that results in wrong plan in most of the cases. So, we replicate a programmers' behavior here. A human programmer when receives a problem it first recall a similar problem that he/she has solved before. This recalling is important because it helps to get a direction for starting the plan generation of the original problem. Then he/she generate a plan. After plan generation he/she might not start coding, in fact it is not good practice. Common practice is to simulate the sample input using the plan. After following the steps if someone cannot get expected output then it revice the plan again and then start coding. We have imitated these steps to generate a plan. For generating relevant example problems we have utilizes the inherent knowledge of the LLM that is gained while training. So, first we instruct the LLM to think about the problem and understand it, then give a similar problem, and finally generate a planning for the original problem. After generating a plan we instruct the LLM to follow that plan and make a simulation with the sample input. After completing this simulation if the simulation results are not matched with the expected output then LLM respond us to modify the plan otherwise LLM response that the plan is ok. If we get a negative response then we instruct the LLM to revice the plan and return the revised plan from this step.

% The first agent in \tool is a \emph{Planning Agent}. Given a problem description, the \emph{Planning Agent} first generate an exemplar-a relevant problem (description), its plan and and solution. This step steps reflects a human programmer when receives a problem it first recall a similar problem that he/she has solved before.  This exemplar or recalling is important because it helps to get a direction for starting the plan generation of the original problem. However, instead of generating $k$ ungrounded exemplars, we only generate one exemplar at a time. We instruct the LLM to think about the original problem and understand it, then leveraging the exemplar to generate a planning for the original problem. After generating a plan we instruct the LLM to follow that plan and make a simulation with the sample input. After completing this simulation if the simulation results are not matched with the expected output then LLM respond us to modify the plan otherwise LLM response that the plan is ok. If we get a negative response then we instruct the LLM to revise the plan and return the revised plan from this step.

The first component of \tool is the \emph{Planning Agent}. Given a problem description, the \emph{Planning Agent} generates a single exemplar—a relevant problem along with its plan and solution. This mimics the behavior of human programmers, who, when faced with a new problem, first recall a similar problem they’ve previously solved. This exemplar-based recall is crucial as it provides a starting point for constructing a solution plan. Instead of generating multiple ungrounded exemplars as in MapCoder, our agent focuses on only one at a time. We then instruct the LLM to %thoroughly understand the original problem and use the exemplar as a guide for generating
generate an appropriate plan. Once the plan is created, the LLM simulates (step-by-step) the solution with a sample input. If the simulation result does not match the expected output, the agent prompts the LLM to revise the plan. Otherwise, the plan is deemed valid. In the case of failure, the \emph{Planning Agent} refines the plan. The complete prompts for the Planning Agent—including plan generation, verification, and refinement—are provided in the Appendix (Figure \ref{prompt:plan-generation}, 
\ref{prompt:plan-verification}, \ref{prompt:plan-refinement}).
% \smallskip
\subsection{Coding Agent}
\label{subsec:coding-agent}
% Next is the \emph{Coding Agent}. It takes the problem description, and a plan from the \emph{Planning Agent} as input and implements the corresponding planning into code to solve the problem. After generating code \tool evaluate it with the sample I/O. If the code passed all the sample test cases then we return this code as a solution otherwise pass it to the following agent.

Next component is the \emph{Coding Agent}, which takes the problem description and the plan generated by the \emph{Planning Agent} as input. The role of this agent is to translate the plan into executable code that solves the given problem. Once the code is generated, \tool evaluates it using sample input/output test cases. If the code passes all sample tests, it is returned as the final solution. Otherwise, the code is handed over to the next agent for further refinement. Figure \ref{prompt:code-generation} in the Appendix provides the complete prompt used by the \emph{Coding Agent}.

% \smallskip
\subsection{Debugging Agent}
\label{subsec:debugging-agent}
% The last agent, \emph{Debugging Agent} takes the original problem, plan from the \emph{Planning Agent}, generated code and the execution (unit testing) log  from the \emph{Coding Agent}, as input and try to debug the code. For debugging we again utilize the power of the simulation. We instruct the LLM to simulate the code with sample input where it fails to generate expected outcome. In this way the LLM can go through each step of the generated code and can detect where the bug is. Then we instruct the LLM to solve that bug by giving a modified code. We have shown the complete prompt in Appendix (Figure \ref{prompt:debugging}). Note that, different from the other techniques such as LATS \cite{zhou2023lats}, AgentCoder \cite{huang2023agentcoder}, Reflexion \cite{shinn2023reflexion}, our \emph{Debugging Agent} does not require any additional test case generation in the pipeline. We have discussed the reason of excluding additional I/O generation phase from our approach on Ablation Study \ref{subsec:impact-of-additional-io}. 

The final component, the \emph{Debugging Agent}, receives the original problem, the plan from the \emph{Planning Agent}, the code generated by the \emph{Coding Agent}, and the execution (unit testing) log as input to debug the code. To identify bugs, instead of directly prompting the LLMs, we uniquely leverage the simulation once again. The LLM is instructed specifically to simulate the code on inputs where it fails to produce the expected output, allowing it to trace the execution step by step and locate the error. Once the bug is identified, the LLM modifies the code to resolve the issue. The complete prompt for the \emph{Debugging Agent} is shown in the Appendix (Figure \ref{prompt:debugging}). Unlike other approaches such as LATS \cite{zhou2023lats}, AgentCoder \cite{huang2023agentcoder}, and Reflexion \cite{shinn2023reflexion}, our \emph{Debugging Agent} does not require additional test case generation. The rationale behind excluding this phase is discussed in the Ablation Study \ref{subsec:impact-of-additional-io}.

% \begin{figure}
%     \centering
%     \hspace*{-0.3cm}
%     \includegraphics[width=0.49\textwidth]{figures/steps/step-2-short.pdf}
%     \vspace{-6mm}
%     \caption{Prompt for \emph{Planning Agent}.} 
%     % The example problem mentioned in this figure will come from the Self-retrieval Agent.}
%     \label{fig:prompt-agent-2-short}
%     \vspace{-4mm}
% \end{figure}

% \vspace{-mm}
% \begin{figure}[h]
%     \centering
%     \hspace*{-0.3cm}
%     \includegraphics[width=0.49\textwidth]{figures/steps/step-4-short.pdf}
%     \vspace{-6mm}
%     \caption{Prompt for \emph{Debugging Agent}.}
%     \label{fig:prompt-agent-4-short}
%     \vspace{-4mm}
% \end{figure} 



\smallskip
\subsection{Adaptive Iteration}
\label{sec:agent-traverse}


\tool employs an adaptive iteration starting with the \emph{Planning Agent}, which generates plans for the given problem. These plans are passed to the \emph{Coding Agent}, which translates them into code and tests against sample I/Os. If all tests pass, the code is returned; otherwise, it's sent to the \emph{Debugging Agent}. The \emph{Debugging Agent} attempts to fix the code for up to $d$ attempts. If unsuccessful after $d$ attempts, the process returns to the \emph{Planning Agent}, restarting the cycle. Once code passing all sample I/Os is obtained, the cycle ends, returning the code as the final output solution for evaluation against hidden test cases. This entire process repeats for a maximum of $p$ cycles if needed. Algorithm \ref{alg:codesim} in the Appendix summarizes our adaptive agent traversal. The algorithm's complexity is $O(pd)$. Appendix \ref{app:example-problem} provides a comprehensive example of how \tool solves a problem. 
% comparing \toolnospace's problem-solving approach to others (e.g., Direct, Chain-of-thought, and Reflexion).


% \tool employs an adaptive iteration starting with the \emph{Planning Agent}, which generates plans for the given problem. These plans are then passed to the \emph{Coding Agent}, which translates them into code and tests against sample I/Os. If all tests pass, the code is returned; otherwise, it's sent to the \emph{Debugging Agent}. The \emph{Debugging Agent} attempts to fix the code for up to $d$ iterations. If unsuccessful, the process returns to the \emph{Planning Agent}, restarting the cycle. Once code passing all sample I/Os is obtained, the iteration ends returning the code as the final output solution  which then are evaluated against the hidden test cases. However, this process repeats for maximum $p$ iterations. Algorithm \ref{alg:codesim} in the Appendix summarizes our adaptive agent traversal. The algorithm's complexity is $O(pd)$. Appendix \ref{app:example-problem} provides a comprehensive example comparing \toolnospace's problem-solving approach to Direct, Chain-of-thought, and Reflexion methods.

% The dynamic traversal in \tool begins with the \emph{Planning Agent}, which outputs the plans for the original problem. Then the generated plan is pushed to the \emph{Coding Agent}. \emph{Coding Agent} translates the plan into code, tested with sample I/Os. If all pass, the code is returned; otherwise, it's passed to \emph{Debugging Agent}. They attempt to fix the code iteratively up to $d$ times. If Debugging agent failed to generate a code that passes all the sample I/Os then we will again go to the \emph{Planning Agent} and repeat the above cycle again. As soon as it gets a code that passes all the sample I/Os the code is returned for evaluation with hidden test cases. This iterative process continues for $p$ iterations. We summarize our dynamic agent traversal in Algorithm \ref{alg:codesim} in Appendix. Our algorithm's complexity is $O(pd)$. A complete example illustrating \toolnospace's problem-solving compared to Direct, Chain-of-thought, and Reflexion approaches is shown in Appendix \ref{app:example-problem}.

%in Figure %\ref{fig:qualitative-example}.
%Detailed prompts for each agent are in Appendix \ref{app:prompts}.
\section{Experimental settings}
\label{sec:experimental-setup}
%We use two types of evaluation: offline and online. For offline
%evaluation,
\myparagraph{Datasets}
We use datasets which are reports of ranking
competitions \cite{raifer2017information,Mordo+al:25a}. In these
competitions, students were assigned to queries and had to produce
documents that would be highly ranked. Before the first round the students were provided with an example of a document relevant to the query. In each of the following rounds, the students observed past rankings for their queries and could modify their documents to potentially improve their next round ranking.

The first dataset, \firstmention{\firstDataset}, is the result of
ranking competitions held for $31$ queries from the TREC9-TREC12 Web
tracks \cite{raifer2017information}. Five to six students competed for each query. The undisclosed ranking function
was LambdaMART \cite{burges2010lambdamart} applied with various hand-crafted features. Following
Goren et al. \cite{goren2020ranking}, whose document modification
approach, \firstmention{\sentReplace}, serves as a baseline\footnote{We found that using LambdaMART instead of SVMrank as originally proposed \cite{goren2020ranking} yields improved performance.}, we use
round 7 for evaluation \cite{raifer2017information}. \sentReplace is a state-of-the-art feature-based supervised method for ranking-incentivized document modification. It
replaces a sentence in the document with another sentence to
improve ranking and to maintain content quality and faithfulness to
the original document.

The second dataset, \firstmention{\secondDataset}, is a report of
ranking competitions \cite{Mordo+al:25a} where the undisclosed ranking
function was the cosine between the E5 embedding vectors \cite{Wang+al:24a} of a document
and a query\footnote{The intfloat/e5-large-unsupervised version from
  the Hugging Face repository
  (\url{https://huggingface.co/intfloat/e5-large-unsupervised}).}. The competitions were run for 7 rounds with $15$
queries from the Web tracks of TREC9-TREC12; 4 players were competing
for each query \cite{wang2022text}.
%Our best performing document modification
%strategies (prompts) were used as bots in some rounds of these
%competitions for online evalution. (See more details below.)
%For
%offline evaluation,
We used round $4$ for evaluation to allow the document
modification methods to have enough history of past rankings. 

For both datasets just described, we apply the different document
modification methods, henceforth referred to as \firstmention{bots},
upon each of the documents in the ranked list for a query in the
specified round (except for the highest ranked document). For each
selected document, we induce a ranking using the same ranker used in
the competitions over its modified version and the original next-round
versions of the other documents (of students) from the round. We use
the evaluation measures described below upon the resultant ranking. We
average the evaluation results across all documents we modified per
round and over queries.

\myparagraph{Evaluation measures}
%We analyzed the performance of a document modification method using various evaluation measures, categorized into three primary groups: ranking properties, faithfulness properties and Quality and Relevance properties. All measures were computed per player and her document for a given query. The results were averaged over queries and grouped by the player type (student, baseline\footnote{i.e. the method of replacing paragraphs \cite{goren2020ranking}.}, a static document or one of the \bt s). For the online evaluation, the measures were also averaged over rounds.
To evaluate rank promotion (demotion) of documents as a result of
modification, we follow Goren et al. \cite{goren2020ranking} and
report \firstmention{Scaled Promotion}: the increase (decrease) of rank in the next round with respect to the current round normalized by the maximum possible rank promotion (demotion).



\omt{
%\begin{block}{Candidate Faithfulness at 1}
$CF@1(d_{curr},d_{next})=\frac{1}{n} \cdot \Sigma_{i=1}^{n} \mathbf{1}{\{ TT(d{curr},d_{next_{i}}) \geq 0.5 \}}$
%\end{block}

%\begin{block}{Normalized Candidate Faithfulness at 1}
$NCF@1(d_{curr},d_{next})=\frac{CF@1(d_{curr},d_{next})}{CF@1(d_{curr},d_{curr})}$
%\end{block}


%\begin{block}{Environmental Faithfulness at 10}
$EF@10(d_{next})=\frac{1}{2 \cdot 10} \cdot \Sigma_{i=1}^{10} [\mathbf{1}{\{ TT(d{\text{top}i}, d{\text{next}}) \geq 0.5 \}} + \mathbf{1}{\{ TT(d{\text{next}}, d_{\text{top}_i}) \geq 0.5 \}}]$
%\end{block}

%\begin{block}{Normalized Environmental Faithfulness at 10}
$NEF@10(d_{curr},d_{next})=\frac{EF@10(d_{next})}{EF@10(d_{curr})}$
%\end{block}
}


To evaluate the faithfulness of a modified document ($\dn$) to its
original (current) version ($\dc$), we compare
the two documents using Gekhman's et al. \cite{gekhman2023trueteacher}
natural language inference (NLI) approach. Specifically, we estimate
whether one document (denoted {\em hypothesis}) is entailed from the other
document (denoted {\em premise}) while preserving factual consistency. The estimate is the \trueteacher{} (TrueTeacher)
measure: \trueteacher $(premise, hypothesis)$ is the
output of the model in the range [0,1]; higher scores indicate stronger factual alignment.

To apply the TrueTeacher model, we first compute the average number of sentences in the modified document that are entailed\footnote{Entailment is determined by a threshold of $0.5$ for the TT score \cite{gekhman2023trueteacher}.} by the current document, which we refer to as raw faithfulness (RF):
%\trueteacher{} score between
%the current document ($\dc$) and all ($n$) sentences in the modified
%document ($\dn$):
$RF(\dn,\dc) \definedas \frac{1}{n} \sum_{i=1}^n \delta[\trueteacher
  (\dc, \dni) \ge 0.5];$ $d^{i}$ is the i'th sentence in document $d$;
$\delta$ is Kronecker's indicator function. Since $RF(\dc,\dc)$ is not
necessarily $1$, we normalize the raw
faithfulness to yield our \firstmention{\normFaith} measure: $\frac
{RF(\dn,\dc)}{RF(\dc,\dc)}$. 

Using LLMs to modify documents raises a concern
about hallucinations \cite{shuster2021retrieval}. We hence measure the
extent to which the content in the modified document is ``faithful''
to that in the entire corpus\footnote{For a corpus we use all the
  documents in all rounds prior to the round on which evaluation is
  performed.}. To that end, we treat the current document as a query,
and retrieve the top-$k$\footnote{We set $k=10$ in our experiments.}
documents in the corpus; $\topRet$ denotes the retrieved set. Retrieval is based on using cosine to compare a query
embedding and the document embedding. We use two types of embeddings:
E5 \cite{Wang+al:24a} and TF.IDF.  We define raw corpus faithfulness
(RCF) as: $RCF (\dn) \definedas \frac {1}{2k} \sum_{d \in \topRet}
(RF(\dn,d) + RF (d,\dn))$. The normalized corpus faithfulness
measure we use is: $CF (\dn) \definedas
\frac{RCF(\dn)}{RCF(\dc)}$. Using the E5 and TF.IDF embeddings results
in the \firstmention{\normCorpFaithE} and
\firstmention{\normCorpFaithT} normalized corpus faithfulness
measures, respectively.

Statistically significant differences are determined using the two-tailed paired permutation test
  with 100,000 random permutations and $p < 0.05$.

\omt{
The Normalized Candidate Faithfulness $NCF@1(\dc, \dn)$: the
normalization of the Candidate Faithfulness $CF@1(\dc, \dn)$ by the
self-consistency score: $\frac{CF@1 (\dc, \dn)}{CF@1(\dc, \dc)}$;
(iii) Environmental Faithfulness at 10 $EF@10(\dn)$: This metric
measures how much the generated document ($\dn$) maintains contextual
consistency with the broader corpus. Specifically, it measures the
similarity of $\dn$ to the top 10 documents most similar to it in the
corpus.  The corpus includes all the documents (across all queries)
available up to the test round. Two approaches are employed to compute
the similarity. The first approach is based on the (unsupervised) E5
\cite{wang2022text} representation with the cosine similarity
metric. The second approach is based on the TF.IDF
\cite{sparck1972statistical, salton1975vector} representation with the
cosine similarity metric. This metric is then calculated as follows:
$\frac{1}{2*10} \sum_{i=1}^{10} (\trueteacher(\dn, d_{top_{i}}) +
\trueteacher(d_{top_i}, \dn))$. Where $d_{top_{i}}$ represents the
$i$-th document, while ordering the documents with respect to the
similarity to $\dc$. The two approaches yield two variants of this
metric: $EF@10$\_dense and $EF@10$\_sparse, for the E5 and TF.IDF
representations, respectively; (iv) The Normalized Environmental
Faithfulness at 10 $NEF@10(\dn)$: The normalization of EF@10 by the
EF@10 of the current document: $\frac{EF@10(\dn)}{EF@10(\dc)}$. These
measures collectively provide a comprehensive framework for assessing
faithfulness. They evaluate the consistency of the modified document
not only with respect to the current document but also in relation to
other documents in the corpus.



\myparagraph{Relevance and Quality scores} The third category of evaluation measures focuses on the relevance and quality of documents. Both quality and relevance scores are assigned by crowdsourcing annotators via the Connect platform on CloudResearch \cite{noauthor_introducing_2024}, assessing the document's content quality and its relevance to the query\footnote{These evaluations of relevance and quality are conducted exclusively in the online evaluation setting.}. A document's quality or relevance score is set to 1 if at least three out of five English-speaking annotators marked it as valid or relevant to the query; otherwise, the score is set to 0. We report the ratio of documents that received a quality or relevance score of 1.
}


\myparagraph{Instantiating bots} For LLM we use Chat-GPT 4o
\cite{achiam2023gpt}. As described in Section
\ref{sec:bots}, there are a few parameters affecting the
instantiation of specific prompts. The number of queries is set to a
value in $\{1, 2\}$.  The number of examples per query is selected
from $\{1, 2, 3\}$. The number of past ranks (i.e., rounds) in the
Temporal prompt is selected from $\{2,3\}$. Using these
parameter values, and the other binary decision factors that affect
instantiation (see Section \ref{sec:bots}), results in $192$
different bots (prompts). In addition, we set the LLM's temperature parameter which controls potential drift to values in $\{0, 0.5, 1, 1.5, 2\}$ \cite{peeperkorn_is_2024}. 

\myparagraph{Rank promotion performance of bots} In terms of Scaled
Promotion, we found\footnote{Actual numbers are omitted due to space
  considerations and as they convery no additional insight.} that the
Pairwise bots (with random selection of document pairs) and the
Listwise bots were the best performing for both the \firstDataset and
\secondDataset datasets; the same specific instantiation of each of these two bots was
always among the top-3 performing bots for both datasets. This finding attests to the
rank-promotion effectiveness of these types of bots (prompts) for
different rankers (LambdaMART and E5). The Temporal bots (prompts), which provide rank-changes information along rounds, were less
effective (in terms of Scaled Promotion) than the Pairwise and
Listwise bots, but were more effective than the Pointwise bots. 

In what follows, we present the evaluation of the two
bots which posted for both datasets Scaled Promotion among the best three:\footnote{These bots were also the best performing in the online evaluation presented below.} 
%For efficiency considerations, we use LLama-2 with $13$B parameters
%\cite{touvron2023llama} to select the best performing
%configurations. The selection is performed with the \firstDataset
%dataset based on the scaled promotion evaluation measure. The best
%performing configurations for which we will report performance over the evaluation datasets are:

\begin{itemize}
\item Pairwise, where only the given query is included, one
  random pair of documents for each of the three last rounds is
  provided as examples, the current rank of the document is not
  used, and the temperature is set to $0.5$.
\item Listwise, where only the given query is included, two previous rounds are used, the current rank is not used, and the temperature is set to $0$.
\end{itemize}
Appendix \ref{appendix_prompt} provides the prompts for these bots.
%The fact that the pairwise and listwise approaches are the most
%effective is conceptually consistent of findings in work on using LLMs
%to induce ranking where the merits of pairwise and listwise approaches
%have been demonstrated \cite{ma2023zero,qin2023large}. For evaluation
%over \firstDataset and \secondDataset we use




\omt{
%The goal of the first phase is to identify a representative prompt for each class of prompts--that is, the prompt whose resultant agents maximize a metric related to ranking promotion. We conduct a comprehensive grid search over the 225 configurations described in Section \ref{sec:bots}, using Dataset 1. For this phase, we utilized Llama-2 with 13B parameters due to its availability \cite{touvron2023llama}. From each configuration, we constructed five agents, each with a different temperature setting for the probability model of the LLM\footnote{All other parameters of the LLM were fixed.}. The temperatures used were \{0, 0.5, 1, 1.5, 2\} and were selected based on the work of Peeperkorn et al \cite{peeperkorn_is_2024}. The selected agents compete for round 7, as was the case in Goren et al. \cite{goren2020ranking}. We do not report the detailed results of this phase due to space limitations in the paper.
}


\myparagraph{Online evaluation} The evaluation performed over the
\firstDataset and \secondDataset datasets is offline and therefore
spans a single round: the students who competed in the competition did
not respond to rankings induced over the documents we modify here. We
therefore also performed online evaluation where our instantiated
prompts competed as bots against students. We organized a ranking
competition\footnote{The competition was approved by institution and international ethics committees.} similar to that of Mordo et al. \cite{Mordo+al:25a} using 15
queries from TREC9-TREC12\footnote{These are different queries than
  those used in the \secondDataset dataset: 21, 55, 61, 64, 74, 75, 83, 96, 124,
  144, 161, 164, 166, 170, 194.}. In contrast to Mordo et al.'s
competitions \cite{Mordo+al:25a}, each game included 5 players: two-three
students, one of the two bots discussed above (Pairwise or
Listwise), and one or two static documents were created using a procedure similar to the one in Raifer et al. \cite{raifer2017information}: first, we used the query in the English Wikipedia search engine and selected a highly ranked page. We then extracted a candidate paragraph from this page, with a length of up to 150 words. Three annotators assessed the relevance of the passages, and we repeated the extraction process for each query until at least two annotators judged a paragraph as relevant. The selected paragraph was then used as a static document for the query for all students.

The students were not aware that they were competing
against bots. We applied our bots in rounds 5\footnote{Due to
  technical issues, we could not run the bots at round 4 as in the offline evaluation.}, 6 and 7 and report the
average performance over these three rounds.

We had documents in the online
evaluation judged for relevance and quality using crowdsourcing
annotators on the Connect platform of CloudResearch
\cite{noauthor_introducing_2024}. Following past work on ranking
competitions \cite{raifer2017information,goren2020ranking}, a
document's quality grade is set to $1$ if at least three out of five
English-speaking annotators marked it as valid (as opposed to keyword
stuffed or useless) and to $0$ otherwise. The relevance grade was $1$ if the document was
marked relevant by at least three annotators and $0$ otherwise. 







\endinput


We adopt an evaluation approach similar to that of Goren et al. \cite{goren2020ranking}. Two evaluation settings are considered: (i) Offline evaluation, where we leveraged existing datasets from ranking competitions, and (ii) Online evaluation, where a set of \bt s, each with a specific \contextualized, participate as a player in an ongoing ranking competition. The offline evaluation is run only for a single round, since the students did not respond to rankings that included the documents produced by our \bt s. In the online setting, other players may modify their documents simultaneously while our agents make their own modifications. In this section, we begin by describing the datasets used in our experiments (Section \ref{sec_datasets}). We then present the setups for both offline and online evaluations (Section \ref{sec_exp_set}). Finally, we outline the evaluation measures employed to assess performance of the \bt s and compare their performances against other types of agents (Section \ref{sec:eval-measures}).







\subsection{Datasets}\label{sec_datasets}
To perform offline and online evaluation, we deployed our approach in three different ranking competitions: one competition with feature-based ranking function, and two other utilizing transformer-based ranking function. The datasets employed in our experiments are as follows:

\myparagraph{Dataset 1} The first competition utilized for offline evaluation and comparison with the baseline model proposed by Goren et al. \cite{goren2020ranking}. It was organized by Raifer et al. \cite{raifer2017information}. In this competition, students enrolled in a course served as authors of documents and were assigned to 31 queries from the TREC9-TREC12 Web tracks. Each query defined a repeated-ranking-game. Students were incentivized with course grade bonuses and were asked to modify their documents for 8 rounds so that their document will be highly ranked for the played query. We selected round 7 for the offline evaluation, following Goren et al. \cite{goren2020ranking}. A total of 31 repeated-games (one per query) were conducted. A LambdaMART ranking function was applied \cite{burges2010lambdamart}.

\myparagraph{Dataset 2} This dataset used for offline evaluation and hyper-parameter tuning for the online evaluation. It was sourced from a ranking competition conducted by Mordo et al. \cite{div}. It involved a competition with 15 queries\footnote{From the TREC9-TREC12 Web Track as well.}, 7 rounds, and 4 players per game. In contrast to the competition described by Raifer et al. \cite{raifer2017information}, a transformer based ranking function was applied: the (unsupervised) E5 \footnote{The intfloat/e5-large-unsupervised version from the Hugging Face repository was used
  (\url{https://huggingface.co/intfloat/e5-large-unsupervised}).} \cite{wang2022text}. We focus on round 4, as it is the first round where we can apply our \bt s (recall that our \bt s require the context of previous rounds to modify a document).

\myparagraph{Dataset 3} We organize a ranking competition using 15 queries\footnote{From TREC9-TREC12; Different queries comparing to those used in Dataset 2: [21, 55, 61, 64, 74, 75, 83, 96, 124, 144, 161, 164, 166, 170, 194].}. The setup of this competition was similar to that of Mordo et al \cite{div}, with the following key difference: each games included 5 players. From round 5 \footnote{We initially planned to introduce our \bt s in round 4 as in Dataset 2; however, due to experimental constraints, we began their application in round 5.} of the competition, the players in each group consisted of: one \bt, two or three students and one or two planted documents\footnote{The same document as the initial document every participant started with.}. From the perspective of the students, the inclusion of \bt s did not alter the structure or appearance of the competition, preserving the integrity of the evaluation.

\subsection{Experimental setting}\label{sec_exp_set}

Our document modification approach operates as follows: first, the ranking for a given query is observed. Next, the approach modifies a specific document with the aim that the resulting document will be ranked higher in the next round of the game. In dynamic (online) settings, other documents may also be modified simultaneously, influencing the subsequent ranking. We design two evaluation paradigms—online and offline—both of which simulate a dynamic setting. The approach introduced by Goren et al. \cite{goren2020ranking} serves as a baseline to our approach.

\myparagraph{Offline evaluation}
The offline evaluation is divided into three phases. In each phase, we evaluated an \bt{} using a similar approach employed by Goren et al. \cite{goren2020ranking}: (i) Select a round and a game (query); (ii) Modify a document using the tested modification method (baseline \cite{goren2020ranking} or \bt{} with a specific prompt), excluding the top-ranked document. The exclusion of top-ranked documents is attributed to previous findings that their authors tend to avoid modifying their documents \cite{raifer2017information}. (iii) Evaluate the performance of the agent with respect to all other documents in the ranked list. (iv) Iterate over all the documents in the selected round and query and average the computed measure. 

% The performance of each metric is computed as the average over queries.

The goal of the first phase is to identify a representative prompt for each class of prompts--that is, the prompt whose resultant agents maximize a metric related to ranking promotion. We conduct a comprehensive grid search over the 225 configurations described in Section \ref{sec:bots}, using Dataset 1. For this phase, we utilized Llama-2 with 13B parameters due to its availability \cite{touvron2023llama}. From each configuration, we constructed five agents, each with a different temperature setting for the probability model of the LLM\footnote{All other parameters of the LLM were fixed.}. The temperatures used were \{0, 0.5, 1, 1.5, 2\} and were selected based on the work of Peeperkorn et al \cite{peeperkorn_is_2024}. The selected agents compete for round 7, as was the case in Goren et al. \cite{goren2020ranking}. We do not report the detailed results of this phase due to space limitations in the paper.

%In the second phase, we evaluated the performance of each representative prompt (Identified in Phase 1) on the same dataset and specifically on round 7, incorporating two key modifications: (i) we replaced Llama-2 with Chat-GPT 4o \cite{achiam2023gpt} as the latter demonstrated superior performance across multiple benchmarks\footnote{\url{https://docsbot.ai/models/compare/gpt-4o/llama-2-chat-13b}}. (ii) We included a baseline model introduced by Goren et al. \cite{goren2020ranking}, which modifies documents by replacing passages. Our implementation of the baseline consist of a primary difference: we replaced RankSVM \cite{joachims2002ranksvm} with LambdaMART \cite{burges2010lambdamart} due to its superior performance on Dataset 1. Details regarding the reproducibility process are omitted.

In the third phase, we evaluated the performance of each representative prompt on Dataset 2, which contains data from a ranking competition with a transformer-based ranking function. The evaluation procedure mirrored that of the second phase. We focused on round 4, as the round for evaluation.

\myparagraph{Online evaluation}
We adopted the two best performing \bt s (in terms of ranking promotion metric) evaluated on the transformer-based competition (Dataset 2) and assigned them as players in a similar ranking competition with different queries (resulting in Dataset 3). These \bt s joined the competition in round 5 and competed for the highest ranking in rounds 5, 6 and 7. Recall that the decision to introduce the \bt s in round 5, rather than at the beginning, was based on their dependency on past rankings, which were integrated into the \contextualized s to guide document modifications. The selection of round 5 over round 4 was due to experimental constraints.

% This article addresses the challenge of white-hat ranking-incentivized modifications, building on the work of Goren et al. \cite{goren2020ranking}, who explored this topic in offline and online competition settings on a ranking competition dataset comprised by Raifer et al. \cite{raifer2017information}. In addition to closely mimicking their approach, which utilized the feature-based LambdaMART ranker \cite{burges2010ranknet} and an LTR-based baseline, referred as the "feature-based" setting in this article, we adopt a transformer-based ranker—specifically, the E5 model introduced by Wang et al. \cite{wang2024multilingual}, both for offline and online evaluation. The E5 settings are referred to as the offline and online "transformer-based" settings in this article.

% Our study investigates the effectiveness of ranking-incentivized modifications within a comparable framework while leveraging the advantages of transformer-based models. The primary goal of this research is to evaluate strategies for rank promotion, focusing on leveraging large language models (LLMs) to implement modifications using various few-shot \cite{brown2020language} contextual approaches. By incorporating LLM-based methodologies, we aim to assess their ability to generate high-quality, contextually relevant modifications that adhere to the principles of white-hat ranking practices.


% \paragraph*{Dataset Creation}
% For the feature-based setting, we utilized the dataset created by Raifer et al. \cite{raifer2017information}, similarly to Goren et al. \cite{goren2020ranking}.

% In the transformer-based settings, we implemented and evaluated our \bt s within a ranking environment inspired by the 'ranking competition' framework introduced by Raifer et al. \cite{raifer2017information}. Similar to their approach, our ranking competition focused on optimizing documents for a black-box ranker. However, we introduced several adjustments to align with our experimental objectives and constraints.

% First, we utilized a subset of 15 queries derived from the TREC ClueWeb09 dataset \cite{clueweb09}, whereas Raifer et al. \cite{raifer2017information} used a broader set. This choice enabled us to conduct multiple experiments in parallel while maintaining a manageable workload and scalability. Additionally, we adhered to the original competition's guidelines, requiring concise, 150-word plain English submissions without links, special characters, or HTML tags, to ensure methodological consistency.

% The competition was structured into "matches" and "rounds." A match refers to a grouping of four competitors who worked on a single query. In each match, participants edited their texts to achieve the highest possible ranking for the query. A round is one iteration of competition during which all matches were conducted simultaneously for a specific query set. Each round provided participants with feedback, allowing them to see their own rankings as well as those of their competitors.

% The competition was divided into two parts, with each part consisting of seven rounds. In the first part, participants worked with 15 queries \cite{partA2024}. In the second part, these queries were replaced with 15 different ones, also sourced from ClueWeb09 \cite{clueweb09} \cite{partB2024 (TBA)}. The grouping of competitors and conditions remained fixed across rounds. We used the first part of the competition for offline evaluation and the second part for online evaluation, enabling a thorough analysis of our methodology under both controlled and dynamic conditions.

% By tailoring the competition to our needs while preserving its core principles, we ensured both comparability to prior work and the validity of our findings in the context of scalable and rigorous experimentation.


% \paragraph*{Offline Evaluation}
% The feature-based setting we developed was heavily influenced by the offline setting described in detail by Goren et al.\ \cite{goren2020ranking}. The evaluation of this setting was carried out meticulously, adhering closely to the methodology outlined in Goren et al.'s \cite{goren2020ranking} offline evaluation section.

% To rigorously assess our LLM-based ranking-incentivized modification methodology in the offline transformer-based setting, we constructed an evaluation setting inspired by the offline setting introduced by Goren et al.\ \cite{goren2020ranking}. Our experiments were conducted on documents initially ranked 2nd, 3rd, and 4th in the fifth round of a competition designed to rank documents against a shared set of 15 queries. These ranks were chosen deliberately, as they represented non-top-performing documents, providing a meaningful opportunity to evaluate the potential for improvement when modifications were applied, as suggested by Goren et al.\ \cite{goren2020ranking}.

% In each round of evaluation, four documents were subjected to ranking. Three of these were unaltered, human-authored documents selected from previous rankings in the competition. The fourth document was a modified version, generated by applying our LLM-based methodology. By incorporating multiple initial ranking positions and diverse queries, we ensured that the evaluation was not overly influenced by specific document characteristics or query types. This approach enhanced the generalizability of our findings.


% \paragraph*{Online Evaluation}
% To closely mimic the online experiment described by Goren et al.\ \cite{goren2020ranking}, we conducted the second part of the ranking competition using the same participant pool but a different set of 15 queries sourced from ClueWeb09 \cite{clueweb09}.

% For the online evaluation, we selected the two \bt{} methods that achieved the highest scores in the offline evaluation during one of our early experiments, the results of which are not depicted in this paper. We used "scaled rank promotion" as the metric for determining their performance. These \bt{}s were introduced into the competition starting from the 4th round of the second part, as their methodology required contextual information derived from at least three prior rounds to function effectively. The bots competed against the same set of human participants, with groupings and conditions held constant across all seven rounds of this phase. This ensured consistency in the evaluation and allowed for a direct comparison between \bt{}- and human-authored rankings.

% From the perspective of the human participants, the inclusion of \bt s did not alter the structure or appearance of the competition. Each round followed the same format and task descriptions as in the first part of the competition. This design ensured that human participants approached their ranking tasks without being influenced by knowledge of the \bt s' involvement, preserving the integrity of the evaluation.

% \paragraph*{Ranker and Document Embeddings}
% In the feature-based setting, we employed the same methodology and features previously utilized by Raifer et al.\ \cite{raifer2017information} and Goren et al.\ \cite{goren2020ranking}. Specifically, we used the exact trained LambdaMART model that was trained and used by Goren et al.\ \cite{goren2020ranking}. The method for text embeddings for this setting replicated Goren et al.'s \cite{goren2020ranking} approach - incorporating 25 content-based features. These features were selected either from those used in Microsoft's learning-to-rank datasets \cite{mslr}, or as query-independent measures of document quality. Notably, these included stopword-based metrics and the entropy of term distribution within a document, both of which have been proven effective in web retrieval scenarios.

% For the transformer-based offline and online settings, we utilized E5 as the ranker and the E5 embedder for the document embeddings.


% \paragraph*{Baseline}
% To establish a robust baseline, we implemented a ranking passage pairs approach, closely mirroring the methodology described in Goren et al. \cite{goren2020ranking}, with the primary difference being the replacement of RankSVM \cite{joachims2002ranksvm} with LambdaMART \cite{burges2010lambdamart}. LambdaMART was selected based on its superior empirical performance in prior evaluations. The dataset, features, and labels remained identical to the original setup. Labels for the passage pairs were generated using a dual-objective harmonic mean approach introduced in Goren et al. \cite{goren2020ranking}, integrating rank-promotion and local coherence objectives, where rank-promotion labels ranged from 0 to 4 based on positional improvement in rankings, and coherence labels quantified semantic similarity to maintain content quality. The harmonic mean was computed with $\beta = 1$, assigning equal weight to both objectives.

% Training was conducted on 57 documents extracted from Round 6 of the original competition dataset introduced by Raifer et al. \cite{raifer2017information}. The model’s performance was subsequently evaluated on 124 experimental settings, derived from Round 7 of the ranking competition, spanning documents ranked 2–5 across 31 queries. The validation set was configured similarly to Goren et al.'s \cite{goren2020ranking} procedure. Both training and validation utilized NDCG@1 as the evaluation metric, contrasting with the NDCG@5 used by Goren et al. \cite{goren2020ranking}, to align with the goal of selecting the top sentence-swapped document.

% We implemented LambdaMART \cite{burges2010lambdamart} using its default parameter settings for several features, specifically: \begin{itemize} \item \textbf{Minimum leaf support:} Minimum number of samples each leaf must contain, set to 1 (default). \item \textbf{Number of threshold candidates for tree splitting:} Set to 256 (default). \item \textbf{Early stopping rounds:} Set to 100 (default). \end{itemize}

% We conducted a grid search to tune other hyper-parameters, exploring different configurations. This grid search included: \begin{itemize} 
%     \item \textbf{Number of Trees:} 50, 500, 1000, 1200. 
%     \item \textbf{Number of Leaves per Tree:} 10, 50, 100. 
%     \item \textbf{Learning Rate (Shrinkage Value):} 0.01, 0.1, 0.2. 
% \end{itemize}


\begin{table*}[t]
\centering
\fontsize{11pt}{11pt}\selectfont
\begin{tabular}{lllllllllllll}
\toprule
\multicolumn{1}{c}{\textbf{task}} & \multicolumn{2}{c}{\textbf{Mir}} & \multicolumn{2}{c}{\textbf{Lai}} & \multicolumn{2}{c}{\textbf{Ziegen.}} & \multicolumn{2}{c}{\textbf{Cao}} & \multicolumn{2}{c}{\textbf{Alva-Man.}} & \multicolumn{1}{c}{\textbf{avg.}} & \textbf{\begin{tabular}[c]{@{}l@{}}avg.\\ rank\end{tabular}} \\
\multicolumn{1}{c}{\textbf{metrics}} & \multicolumn{1}{c}{\textbf{cor.}} & \multicolumn{1}{c}{\textbf{p-v.}} & \multicolumn{1}{c}{\textbf{cor.}} & \multicolumn{1}{c}{\textbf{p-v.}} & \multicolumn{1}{c}{\textbf{cor.}} & \multicolumn{1}{c}{\textbf{p-v.}} & \multicolumn{1}{c}{\textbf{cor.}} & \multicolumn{1}{c}{\textbf{p-v.}} & \multicolumn{1}{c}{\textbf{cor.}} & \multicolumn{1}{c}{\textbf{p-v.}} &  &  \\ \midrule
\textbf{S-Bleu} & 0.50 & 0.0 & 0.47 & 0.0 & 0.59 & 0.0 & 0.58 & 0.0 & 0.68 & 0.0 & 0.57 & 5.8 \\
\textbf{R-Bleu} & -- & -- & 0.27 & 0.0 & 0.30 & 0.0 & -- & -- & -- & -- & - &  \\
\textbf{S-Meteor} & 0.49 & 0.0 & 0.48 & 0.0 & 0.61 & 0.0 & 0.57 & 0.0 & 0.64 & 0.0 & 0.56 & 6.1 \\
\textbf{R-Meteor} & -- & -- & 0.34 & 0.0 & 0.26 & 0.0 & -- & -- & -- & -- & - &  \\
\textbf{S-Bertscore} & \textbf{0.53} & 0.0 & {\ul 0.80} & 0.0 & \textbf{0.70} & 0.0 & {\ul 0.66} & 0.0 & {\ul0.78} & 0.0 & \textbf{0.69} & \textbf{1.7} \\
\textbf{R-Bertscore} & -- & -- & 0.51 & 0.0 & 0.38 & 0.0 & -- & -- & -- & -- & - &  \\
\textbf{S-Bleurt} & {\ul 0.52} & 0.0 & {\ul 0.80} & 0.0 & 0.60 & 0.0 & \textbf{0.70} & 0.0 & \textbf{0.80} & 0.0 & {\ul 0.68} & {\ul 2.3} \\
\textbf{R-Bleurt} & -- & -- & 0.59 & 0.0 & -0.05 & 0.13 & -- & -- & -- & -- & - &  \\
\textbf{S-Cosine} & 0.51 & 0.0 & 0.69 & 0.0 & {\ul 0.62} & 0.0 & 0.61 & 0.0 & 0.65 & 0.0 & 0.62 & 4.4 \\
\textbf{R-Cosine} & -- & -- & 0.40 & 0.0 & 0.29 & 0.0 & -- & -- & -- & -- & - & \\ \midrule
\textbf{QuestEval} & 0.23 & 0.0 & 0.25 & 0.0 & 0.49 & 0.0 & 0.47 & 0.0 & 0.62 & 0.0 & 0.41 & 9.0 \\
\textbf{LLaMa3} & 0.36 & 0.0 & \textbf{0.84} & 0.0 & {\ul{0.62}} & 0.0 & 0.61 & 0.0 &  0.76 & 0.0 & 0.64 & 3.6 \\
\textbf{our (3b)} & 0.49 & 0.0 & 0.73 & 0.0 & 0.54 & 0.0 & 0.53 & 0.0 & 0.7 & 0.0 & 0.60 & 5.8 \\
\textbf{our (8b)} & 0.48 & 0.0 & 0.73 & 0.0 & 0.52 & 0.0 & 0.53 & 0.0 & 0.7 & 0.0 & 0.59 & 6.3 \\  \bottomrule
\end{tabular}
\caption{Pearson correlation on human evaluation on system output. `R-': reference-based. `S-': source-based.}
\label{tab:sys}
\end{table*}



\begin{table}%[]
\centering
\fontsize{11pt}{11pt}\selectfont
\begin{tabular}{llllll}
\toprule
\multicolumn{1}{c}{\textbf{task}} & \multicolumn{1}{c}{\textbf{Lai}} & \multicolumn{1}{c}{\textbf{Zei.}} & \multicolumn{1}{c}{\textbf{Scia.}} & \textbf{} & \textbf{} \\ 
\multicolumn{1}{c}{\textbf{metrics}} & \multicolumn{1}{c}{\textbf{cor.}} & \multicolumn{1}{c}{\textbf{cor.}} & \multicolumn{1}{c}{\textbf{cor.}} & \textbf{avg.} & \textbf{\begin{tabular}[c]{@{}l@{}}avg.\\ rank\end{tabular}} \\ \midrule
\textbf{S-Bleu} & 0.40 & 0.40 & 0.19* & 0.33 & 7.67 \\
\textbf{S-Meteor} & 0.41 & 0.42 & 0.16* & 0.33 & 7.33 \\
\textbf{S-BertS.} & {\ul0.58} & 0.47 & 0.31 & 0.45 & 3.67 \\
\textbf{S-Bleurt} & 0.45 & {\ul 0.54} & {\ul 0.37} & 0.45 & {\ul 3.33} \\
\textbf{S-Cosine} & 0.56 & 0.52 & 0.3 & {\ul 0.46} & {\ul 3.33} \\ \midrule
\textbf{QuestE.} & 0.27 & 0.35 & 0.06* & 0.23 & 9.00 \\
\textbf{LlaMA3} & \textbf{0.6} & \textbf{0.67} & \textbf{0.51} & \textbf{0.59} & \textbf{1.0} \\
\textbf{Our (3b)} & 0.51 & 0.49 & 0.23* & 0.39 & 4.83 \\
\textbf{Our (8b)} & 0.52 & 0.49 & 0.22* & 0.43 & 4.83 \\ \bottomrule
\end{tabular}
\caption{Pearson correlation on human ratings on reference output. *not significant; we cannot reject the null hypothesis of zero correlation}
\label{tab:ref}
\end{table}


\begin{table*}%[]
\centering
\fontsize{11pt}{11pt}\selectfont
\begin{tabular}{lllllllll}
\toprule
\textbf{task} & \multicolumn{1}{c}{\textbf{ALL}} & \multicolumn{1}{c}{\textbf{sentiment}} & \multicolumn{1}{c}{\textbf{detoxify}} & \multicolumn{1}{c}{\textbf{catchy}} & \multicolumn{1}{c}{\textbf{polite}} & \multicolumn{1}{c}{\textbf{persuasive}} & \multicolumn{1}{c}{\textbf{formal}} & \textbf{\begin{tabular}[c]{@{}l@{}}avg. \\ rank\end{tabular}} \\
\textbf{metrics} & \multicolumn{1}{c}{\textbf{cor.}} & \multicolumn{1}{c}{\textbf{cor.}} & \multicolumn{1}{c}{\textbf{cor.}} & \multicolumn{1}{c}{\textbf{cor.}} & \multicolumn{1}{c}{\textbf{cor.}} & \multicolumn{1}{c}{\textbf{cor.}} & \multicolumn{1}{c}{\textbf{cor.}} &  \\ \midrule
\textbf{S-Bleu} & -0.17 & -0.82 & -0.45 & -0.12* & -0.1* & -0.05 & -0.21 & 8.42 \\
\textbf{R-Bleu} & - & -0.5 & -0.45 &  &  &  &  &  \\
\textbf{S-Meteor} & -0.07* & -0.55 & -0.4 & -0.01* & 0.1* & -0.16 & -0.04* & 7.67 \\
\textbf{R-Meteor} & - & -0.17* & -0.39 & - & - & - & - & - \\
\textbf{S-BertScore} & 0.11 & -0.38 & -0.07* & -0.17* & 0.28 & 0.12 & 0.25 & 6.0 \\
\textbf{R-BertScore} & - & -0.02* & -0.21* & - & - & - & - & - \\
\textbf{S-Bleurt} & 0.29 & 0.05* & 0.45 & 0.06* & 0.29 & 0.23 & 0.46 & 4.2 \\
\textbf{R-Bleurt} & - &  0.21 & 0.38 & - & - & - & - & - \\
\textbf{S-Cosine} & 0.01* & -0.5 & -0.13* & -0.19* & 0.05* & -0.05* & 0.15* & 7.42 \\
\textbf{R-Cosine} & - & -0.11* & -0.16* & - & - & - & - & - \\ \midrule
\textbf{QuestEval} & 0.21 & {\ul{0.29}} & 0.23 & 0.37 & 0.19* & 0.35 & 0.14* & 4.67 \\
\textbf{LlaMA3} & \textbf{0.82} & \textbf{0.80} & \textbf{0.72} & \textbf{0.84} & \textbf{0.84} & \textbf{0.90} & \textbf{0.88} & \textbf{1.00} \\
\textbf{Our (3b)} & 0.47 & -0.11* & 0.37 & 0.61 & 0.53 & 0.54 & 0.66 & 3.5 \\
\textbf{Our (8b)} & {\ul{0.57}} & 0.09* & {\ul 0.49} & {\ul 0.72} & {\ul 0.64} & {\ul 0.62} & {\ul 0.67} & {\ul 2.17} \\ \bottomrule
\end{tabular}
\caption{Pearson correlation on human ratings on our constructed test set. 'R-': reference-based. 'S-': source-based. *not significant; we cannot reject the null hypothesis of zero correlation}
\label{tab:con}
\end{table*}

\section{Results}
We benchmark the different metrics on the different datasets using correlation to human judgement. For content preservation, we show results split on data with system output, reference output and our constructed test set: we show that the data source for evaluation leads to different conclusions on the metrics. In addition, we examine whether the metrics can rank style transfer systems similar to humans. On style strength, we likewise show correlations between human judgment and zero-shot evaluation approaches. When applicable, we summarize results by reporting the average correlation. And the average ranking of the metric per dataset (by ranking which metric obtains the highest correlation to human judgement per dataset). 

\subsection{Content preservation}
\paragraph{How do data sources affect the conclusion on best metric?}
The conclusions about the metrics' performance change radically depending on whether we use system output data, reference output, or our constructed test set. Ideally, a good metric correlates highly with humans on any data source. Ideally, for meta-evaluation, a metric should correlate consistently across all data sources, but the following shows that the correlations indicate different things, and the conclusion on the best metric should be drawn carefully.

Looking at the metrics correlations with humans on the data source with system output (Table~\ref{tab:sys}), we see a relatively high correlation for many of the metrics on many tasks. The overall best metrics are S-BertScore and S-BLEURT (avg+avg rank). We see no notable difference in our method of using the 3B or 8B model as the backbone.

Examining the average correlations based on data with reference output (Table~\ref{tab:ref}), now the zero-shoot prompting with LlaMA3 70B is the best-performing approach ($0.59$ avg). Tied for second place are source-based cosine embedding ($0.46$ avg), BLEURT ($0.45$ avg) and BertScore ($0.45$ avg). Our method follows on a 5. place: here, the 8b version (($0.43$ avg)) shows a bit stronger results than 3b ($0.39$ avg). The fact that the conclusions change, whether looking at reference or system output, confirms the observations made by \citet{scialom-etal-2021-questeval} on simplicity transfer.   

Now consider the results on our test set (Table~\ref{tab:con}): Several metrics show low or no correlation; we even see a significantly negative correlation for some metrics on ALL (BLEU) and for specific subparts of our test set for BLEU, Meteor, BertScore, Cosine. On the other end, LlaMA3 70B is again performing best, showing strong results ($0.82$ in ALL). The runner-up is now our 8B method, with a gap to the 3B version ($0.57$ vs $0.47$ in ALL). Note our method still shows zero correlation for the sentiment task. After, ranks BLEURT ($0.29$), QuestEval ($0.21$), BertScore ($0.11$), Cosine ($0.01$).  

On our test set, we find that some metrics that correlate relatively well on the other datasets, now exhibit low correlation. Hence, with our test set, we can now support the logical reasoning with data evidence: Evaluation of content preservation for style transfer needs to take the style shift into account. This conclusion could not be drawn using the existing data sources: We hypothesise that for the data with system-based output, successful output happens to be very similar to the source sentence and vice versa, and reference-based output might not contain server mistakes as they are gold references. Thus, none of the existing data sources tests the limits of the metrics.  


\paragraph{How do reference-based metrics compare to source-based ones?} Reference-based metrics show a lower correlation than the source-based counterpart for all metrics on both datasets with ratings on references (Table~\ref{tab:sys}). As discussed previously, reference-based metrics for style transfer have the drawback that many different good solutions on a rewrite might exist and not only one similar to a reference.


\paragraph{How well can the metrics rank the performance of style transfer methods?}
We compare the metrics' ability to judge the best style transfer methods w.r.t. the human annotations: Several of the data sources contain samples from different style transfer systems. In order to use metrics to assess the quality of the style transfer system, metrics should correctly find the best-performing system. Hence, we evaluate whether the metrics for content preservation provide the same system ranking as human evaluators. We take the mean of the score for every output on each system and the mean of the human annotations; we compare the systems using the Kendall's Tau correlation. 

We find only the evaluation using the dataset Mir, Lai, and Ziegen to result in significant correlations, probably because of sparsity in a number of system tests (App.~\ref{app:dataset}). Our method (8b) is the only metric providing a perfect ranking of the style transfer system on the Lai data, and Llama3 70B the only one on the Ziegen data. Results in App.~\ref{app:results}. 


\subsection{Style strength results}
%Evaluating style strengths is a challenging task. 
Llama3 70B shows better overall results than our method. However, our method scores higher than Llama3 70B on 2 out of 6 datasets, but it also exhibits zero correlation on one task (Table~\ref{tab:styleresults}).%More work i s needed on evaluating style strengths. 
 
\begin{table}%[]
\fontsize{11pt}{11pt}\selectfont
\begin{tabular}{lccc}
\toprule
\multicolumn{1}{c}{\textbf{}} & \textbf{LlaMA3} & \textbf{Our (3b)} & \textbf{Our (8b)} \\ \midrule
\textbf{Mir} & 0.46 & 0.54 & \textbf{0.57} \\
\textbf{Lai} & \textbf{0.57} & 0.18 & 0.19 \\
\textbf{Ziegen.} & 0.25 & 0.27 & \textbf{0.32} \\
\textbf{Alva-M.} & \textbf{0.59} & 0.03* & 0.02* \\
\textbf{Scialom} & \textbf{0.62} & 0.45 & 0.44 \\
\textbf{\begin{tabular}[c]{@{}l@{}}Our Test\end{tabular}} & \textbf{0.63} & 0.46 & 0.48 \\ \bottomrule
\end{tabular}
\caption{Style strength: Pearson correlation to human ratings. *not significant; we cannot reject the null hypothesis of zero corelation}
\label{tab:styleresults}
\end{table}

\subsection{Ablation}
We conduct several runs of the methods using LLMs with variations in instructions/prompts (App.~\ref{app:method}). We observe that the lower the correlation on a task, the higher the variation between the different runs. For our method, we only observe low variance between the runs.
None of the variations leads to different conclusions of the meta-evaluation. Results in App.~\ref{app:results}.
\section{Ablation Studies and Analyses}
\label{sec:ablation-study}

\subsection{Impact of Different Agents}
\label{sec:agent-impact-ablation}
Our primary contributions are two folds: (i) the simulation-guided plan verification step within the \emph{Planning Agent} and (ii) the bug fixing process through simulation in \emph{Debugging Agent}. To evaluate the significance of these components, we ablate these two parts of our approach and present the results in Table \ref{tab:agent-disabling}. The findings confirm that both components contribute significantly. 

\begin{table}[h]
    \centering
    \begin{tabular}{c}
    \hspace*{-0.35cm}
    \includegraphics[width=0.45\textwidth]{figures/tables/Ablation-Less-Feature.pdf}
    \end{tabular}
    \caption{Pass@1 results for different versions of \tool (by using GPT4o on HumanEval dataset).}
    \label{tab:agent-disabling}
\end{table}




\subsection{Fine-grained Analysis of the Impact of Simulation}
\label{subsec:impact-simulation}
Table \ref{tab:simulation-impact} presents the impact of incorporating \textit{Simulation} in \toolnospace. The results show that \tool consistently outperforms other approaches across both simple and multi-agent settings, demonstrating superior performance with both open-source and proprietary LLMs. This highlights the effectiveness of \textit{Simulation} in enhancing problem-solving efficiency within our pipeline. 

\begin{table}[h]
    \centering
    \begin{tabular}{c}
    \hspace*{-0.35cm}
    \includegraphics[width=0.49\textwidth]{figures/tables/Ablation-Impact-Simulation.pdf}
    \end{tabular}
    \caption{Impact of using \textit{Simulation}.}
    \label{tab:simulation-impact}
\end{table}


\subsection{Impact of Varying Programming Languages}
\label{subsec:various-programming-languages}
To evaluate the performance of \tool across various programming languages, we utilized the xCodeEval \cite{khan2023xcodeeval} dataset. The experimental results, presented in Table \ref{tab:veriying-PL}, demonstrate that \tool maintains strong performance across different programming languages, highlighting its versatility and effectiveness.

\begin{table}[h]
    \centering
    \begin{tabular}{c}    
    \hspace*{-0.35cm}
    \includegraphics[width=0.49\textwidth]{figures/tables/Ablation-Variying-PL.pdf}
    \end{tabular}
    \caption{Pass@1 results for different programming languages from xCodeEval dataset by using ChatGPT.}
    \label{tab:veriying-PL}
\end{table}



\subsection{Use of External Debugger}
\label{subsec:ldb-debugger}

\begin{table}[h]
    \centering
    \begin{tabular}{c}
    \hspace*{-0.35cm}
    \includegraphics[width=0.49\textwidth]{figures/tables/LDB-Results.pdf}
    \end{tabular}
    \caption{Pass@1 results for different approaches using an external debugger.}
    \label{tab:ldb-results}
\end{table}


\noindent
The performance of \tool can be further enhanced by incorporating an external debugger in the \emph{second pass}. We experiment with LDB as the external debugger on HumanEval dataset in Table \ref{tab:ldb-results}. We use the output code from the most competitive \emph{first-pass} generation methods, including \toolnospace, Reflexion, and MapCoder, using GPT-4o as the backbone. These seed programs are then passed to LDB, which was tested with two different LLMs: ChatGPT and GPT-4o. As can be seen, \tool achieves $95.1$\% accuracy in the \emph{first pass} with GPT-4o, surpassing Reflexion's \emph{second pass} performance of $94.5$\%.  By utilizing LDB with GPT-4o, \tool achieves a \emph{second pass} accuracy of $97.6$\%, setting a new state-of-the-art result for a \emph{dual-pass} approach. In addition, we note that the \emph{second pass} with LDB consumes 39K more tokens in Reflexion compared to our approach, highlighting the efficiency  of \toolnospace.
% LDB GPT4o + CodeSim GPT4o = 219.8k
% LDB GPT4o + Reflexion GPT4o = 258.8k

% However, this \emph{dual-pass} process requires approximately 51.6 thousand more tokens than the \emph{single-pass} approach of \toolnospace.


\subsection{Qualitative Example}
We also conduct a qualitative analysis to better understand how \tool improves performance across various datasets. Figure \ref{fig:qualitative-example} demonstrates how \tool enhances the plan through simulation and assists in debugging the code using the same technique. A complete example, including LLM output, is provided in Appendix \ref{app:example-problem}.

\subsection{Impact of $p$ and $d$}
\tool includes two key hyperparameters: the maximum number of planning steps ($p$) and the maximum number of debugging steps ($d$). By varying these parameters, we plot the results in Figure \ref{tab:ablation-p-d-results}, which shows a proportionate improvement in performance. It is important to note that higher values of $p$ and $d$ lead to more API calls and increased token consumption, allowing users to adjust these parameters to balance between accuracy and cost.


\begin{figure}[h]
    \centering
    \includegraphics[width=0.85\linewidth]{figures/results/Ablation-p-d.png}
    \caption{Pass@1 results by varying maximum number of planning, $p$ and maximum number of debugging, $d$.}
    \label{tab:ablation-p-d-results}
\end{figure}


\subsection{Impact of Number of Sample I/Os}
\label{subsec:impact-of-sample-io}
The HumanEval dataset has an average of only $2.82$ sample I/Os per example, which is a relatively small number for deriving meaningful insights. In this ablation, we augment the dataset by adding 5 more sample I/Os from the HumanEval-ET dataset. This augmentation increases performance notably, leading to \textbf{$89$\%} accuracy with ChatGPT, a \textbf{$3.5$\%} improvement over previous results, \textbf{$86$\%}.


\begin{table*}[t]
    \centering
    \begin{tabular}{c}
    \hspace*{-0.3cm}
    \includegraphics[width=0.95\textwidth]{figures/tables/Ablation-Token-Analysis.pdf}
    \end{tabular}
    \caption{Comparison between MapCoder and \tool in terms of average number of API calls, average tokens used (in thousands). Here the upward symbol (↑) refers that the higher value is better and opposite meaning for downward symbol (↓).}
    \label{tab:ablation-token-counts}
\end{table*}



\subsection{Impact of Synthesizing Additional I/O}
\label{subsec:impact-of-additional-io}
Increasing the number of sample I/Os for testing can enhance the overall performance of our approach, as indicated in \ref{subsec:impact-of-sample-io}. Based on this insight, we use a self-consistency \cite{wang2023selfconsistency} method to generate additional test cases. We instruct the LLM to generate five more test cases for each problem, covering both basic and edge cases. The LLM is called twice, and we select the test cases that are present in both responses. However, this approach results in a performance decline. With ChatGPT we achieve \textbf{$78$\%} accuracy—a \textbf{$9.3$\%} decrease from the original \textbf{$86$\%}. This indicates that generating additional I/Os is a non-trivial task that may negatively impact final outcomes.

\subsection{API Call and Token Analysis}
\label{subsec:api-time-analysis}
We compare the API calls and token consumption of our approach with the previous state-of-the-art method, MapCoder \cite{islam-etal-2024-mapcoder}, as shown in Table \ref{tab:ablation-token-counts}. The results reveal that \tool not only improves performance but also reduces token consumption. On average, \tool uses $4.13$ thousand fewer tokens while achieving a $7.1$\% increase in accuracy, proving that \tool is more efficient in both accuracy and token usage compared to MapCoder.

\subsection{Error Analysis and Challenges}
\begin{nobreakwords}
Although \tool demonstrates strong performance compared to other methods, it faces challenges in specific algorithmic domains. The APPS dataset \cite{hendrycks2021apps} includes problems with three levels of difficulty: (i) Introductory, (ii) Interview, and (iii) Competition. Figure \ref{fig:difficulty-levels} illustrates the performance of different approaches based on difficulty level. The results indicate that for introductory and interview-level problems, \tool does not surpass MapCoder when using ChatGPT. Additionally, when using GPT-4, \tool struggles to outperform MapCoder on interview-level problems. Upon manual review, we observe that for more complex issues, such as dynamic programming (DP), \tool encounters difficulties in constructing the DP table.
\end{nobreakwords}


\begin{figure}[h]
    \centering
    \includegraphics[width=0.49\textwidth]{figures/results/apps-difficulty-levels.png}
    \vspace{-3mm}
    \caption{Performance of different approaches across different difficulty levels on the APPS dataset.}
    \label{fig:difficulty-levels}
    \vspace{-4mm}
\end{figure}  


\section{Conclusion}
In this work, we propose a simple yet effective approach, called SMILE, for graph few-shot learning with fewer tasks. Specifically, we introduce a novel dual-level mixup strategy, including within-task and across-task mixup, for enriching the diversity of nodes within each task and the diversity of tasks. Also, we incorporate the degree-based prior information to learn expressive node embeddings. Theoretically, we prove that SMILE effectively enhances the model's generalization performance. Empirically, we conduct extensive experiments on multiple benchmarks and the results suggest that SMILE significantly outperforms other baselines, including both in-domain and cross-domain few-shot settings.
\section{Limitation}
The use of 3D-printed PLA for structural components improves improving ease of assembly and reduces weight and cost, yet it causes deformation under heavy load, which can diminish end-effector precision. Using metal, such as aluminum, would remedy this problem. Additionally, \robot relies on integrated joint relative encoders, requiring manual initialization in a fixed joint configuration each time the system is powered on. Using absolute joint encoders could significantly improve accuracy and ease of use, although it would increase the overall cost. 

%Reliance on commercially available actuators simplifies integration but imposes constraints on control frequency and customization, further limiting the potential for tailored performance improvements.

% The 6 DoF configuration provides sufficient mobility for most tasks; however, certain bimanual operations could benefit from an additional degree of freedom to handle complex joint constraints more effectively. Furthermore, the limited torque density of commercially available proprioceptive actuators restricts the payload and torque output, making the system less suitability for handling heavier loads or high-torque applications. 

The 6 DoF configuration of the arm provides sufficient mobility for single-arm manipulation tasks, yet it shows a limitation in certain bimanual manipulation problems. Specifically, when \robot holds onto a rigid object with both hands, each arm loses 1 DoF because the hands are fixed to the object during grasping. This leads to an underactuated kinematic chain which has a limited mobility in 3D space. We can achieve more mobility by letting the object slip inside the grippers, yet this renders the grasp less robust and simulation difficult. Therefore, we anticipate that designing a lightweight 3 DoF wrist in place of the current 2 DoF wrist allows a more diverse repertoire of manipulation in bimanual tasks.

Finally, the limited torque density of commercially available proprioceptive actuators restricts the performance. Currently, all of our actuators feature a 1:10 gear ratio, so \robot can handle up to 2.5 kg of payload. To handle a heavier object and manipulate it with higher torque, we expect the actuator to have 1:20$\sim$30 gear ratio, but it is difficult to find an off-the-shelf product that meets our requirements. Customizing the actuator to increase the torque density while minimizing the weight will enable \robot to move faster and handle more diverse objects.

%These constraints highlight opportunities for improvement in future iterations, including alternative materials for enhanced rigidity, custom actuator designs for higher control precision and torque density, the adoption of absolute joint encoders, and optimized configurations to balance dexterity and weight.


% \smallskip
% \myparagraph{Acknowledgments} We thank the reviewers for their comments.
% The work by Moshe Tennenholtz was supported by funding from the
% European Research Council (ERC) under the European Union's Horizon
% 2020 research and innovation programme (grant agreement 740435).


% Bibliography entries for the entire Anthology, followed by custom entries
%\bibliography{anthology,custom}
% Custom bibliography entries only
\bibliography{custom, anthology, iclr2025_conference}

\subsection{Lloyd-Max Algorithm}
\label{subsec:Lloyd-Max}
For a given quantization bitwidth $B$ and an operand $\bm{X}$, the Lloyd-Max algorithm finds $2^B$ quantization levels $\{\hat{x}_i\}_{i=1}^{2^B}$ such that quantizing $\bm{X}$ by rounding each scalar in $\bm{X}$ to the nearest quantization level minimizes the quantization MSE. 

The algorithm starts with an initial guess of quantization levels and then iteratively computes quantization thresholds $\{\tau_i\}_{i=1}^{2^B-1}$ and updates quantization levels $\{\hat{x}_i\}_{i=1}^{2^B}$. Specifically, at iteration $n$, thresholds are set to the midpoints of the previous iteration's levels:
\begin{align*}
    \tau_i^{(n)}=\frac{\hat{x}_i^{(n-1)}+\hat{x}_{i+1}^{(n-1)}}2 \text{ for } i=1\ldots 2^B-1
\end{align*}
Subsequently, the quantization levels are re-computed as conditional means of the data regions defined by the new thresholds:
\begin{align*}
    \hat{x}_i^{(n)}=\mathbb{E}\left[ \bm{X} \big| \bm{X}\in [\tau_{i-1}^{(n)},\tau_i^{(n)}] \right] \text{ for } i=1\ldots 2^B
\end{align*}
where to satisfy boundary conditions we have $\tau_0=-\infty$ and $\tau_{2^B}=\infty$. The algorithm iterates the above steps until convergence.

Figure \ref{fig:lm_quant} compares the quantization levels of a $7$-bit floating point (E3M3) quantizer (left) to a $7$-bit Lloyd-Max quantizer (right) when quantizing a layer of weights from the GPT3-126M model at a per-tensor granularity. As shown, the Lloyd-Max quantizer achieves substantially lower quantization MSE. Further, Table \ref{tab:FP7_vs_LM7} shows the superior perplexity achieved by Lloyd-Max quantizers for bitwidths of $7$, $6$ and $5$. The difference between the quantizers is clear at 5 bits, where per-tensor FP quantization incurs a drastic and unacceptable increase in perplexity, while Lloyd-Max quantization incurs a much smaller increase. Nevertheless, we note that even the optimal Lloyd-Max quantizer incurs a notable ($\sim 1.5$) increase in perplexity due to the coarse granularity of quantization. 

\begin{figure}[h]
  \centering
  \includegraphics[width=0.7\linewidth]{sections/figures/LM7_FP7.pdf}
  \caption{\small Quantization levels and the corresponding quantization MSE of Floating Point (left) vs Lloyd-Max (right) Quantizers for a layer of weights in the GPT3-126M model.}
  \label{fig:lm_quant}
\end{figure}

\begin{table}[h]\scriptsize
\begin{center}
\caption{\label{tab:FP7_vs_LM7} \small Comparing perplexity (lower is better) achieved by floating point quantizers and Lloyd-Max quantizers on a GPT3-126M model for the Wikitext-103 dataset.}
\begin{tabular}{c|cc|c}
\hline
 \multirow{2}{*}{\textbf{Bitwidth}} & \multicolumn{2}{|c|}{\textbf{Floating-Point Quantizer}} & \textbf{Lloyd-Max Quantizer} \\
 & Best Format & Wikitext-103 Perplexity & Wikitext-103 Perplexity \\
\hline
7 & E3M3 & 18.32 & 18.27 \\
6 & E3M2 & 19.07 & 18.51 \\
5 & E4M0 & 43.89 & 19.71 \\
\hline
\end{tabular}
\end{center}
\end{table}

\subsection{Proof of Local Optimality of LO-BCQ}
\label{subsec:lobcq_opt_proof}
For a given block $\bm{b}_j$, the quantization MSE during LO-BCQ can be empirically evaluated as $\frac{1}{L_b}\lVert \bm{b}_j- \bm{\hat{b}}_j\rVert^2_2$ where $\bm{\hat{b}}_j$ is computed from equation (\ref{eq:clustered_quantization_definition}) as $C_{f(\bm{b}_j)}(\bm{b}_j)$. Further, for a given block cluster $\mathcal{B}_i$, we compute the quantization MSE as $\frac{1}{|\mathcal{B}_{i}|}\sum_{\bm{b} \in \mathcal{B}_{i}} \frac{1}{L_b}\lVert \bm{b}- C_i^{(n)}(\bm{b})\rVert^2_2$. Therefore, at the end of iteration $n$, we evaluate the overall quantization MSE $J^{(n)}$ for a given operand $\bm{X}$ composed of $N_c$ block clusters as:
\begin{align*}
    \label{eq:mse_iter_n}
    J^{(n)} = \frac{1}{N_c} \sum_{i=1}^{N_c} \frac{1}{|\mathcal{B}_{i}^{(n)}|}\sum_{\bm{v} \in \mathcal{B}_{i}^{(n)}} \frac{1}{L_b}\lVert \bm{b}- B_i^{(n)}(\bm{b})\rVert^2_2
\end{align*}

At the end of iteration $n$, the codebooks are updated from $\mathcal{C}^{(n-1)}$ to $\mathcal{C}^{(n)}$. However, the mapping of a given vector $\bm{b}_j$ to quantizers $\mathcal{C}^{(n)}$ remains as  $f^{(n)}(\bm{b}_j)$. At the next iteration, during the vector clustering step, $f^{(n+1)}(\bm{b}_j)$ finds new mapping of $\bm{b}_j$ to updated codebooks $\mathcal{C}^{(n)}$ such that the quantization MSE over the candidate codebooks is minimized. Therefore, we obtain the following result for $\bm{b}_j$:
\begin{align*}
\frac{1}{L_b}\lVert \bm{b}_j - C_{f^{(n+1)}(\bm{b}_j)}^{(n)}(\bm{b}_j)\rVert^2_2 \le \frac{1}{L_b}\lVert \bm{b}_j - C_{f^{(n)}(\bm{b}_j)}^{(n)}(\bm{b}_j)\rVert^2_2
\end{align*}

That is, quantizing $\bm{b}_j$ at the end of the block clustering step of iteration $n+1$ results in lower quantization MSE compared to quantizing at the end of iteration $n$. Since this is true for all $\bm{b} \in \bm{X}$, we assert the following:
\begin{equation}
\begin{split}
\label{eq:mse_ineq_1}
    \tilde{J}^{(n+1)} &= \frac{1}{N_c} \sum_{i=1}^{N_c} \frac{1}{|\mathcal{B}_{i}^{(n+1)}|}\sum_{\bm{b} \in \mathcal{B}_{i}^{(n+1)}} \frac{1}{L_b}\lVert \bm{b} - C_i^{(n)}(b)\rVert^2_2 \le J^{(n)}
\end{split}
\end{equation}
where $\tilde{J}^{(n+1)}$ is the the quantization MSE after the vector clustering step at iteration $n+1$.

Next, during the codebook update step (\ref{eq:quantizers_update}) at iteration $n+1$, the per-cluster codebooks $\mathcal{C}^{(n)}$ are updated to $\mathcal{C}^{(n+1)}$ by invoking the Lloyd-Max algorithm \citep{Lloyd}. We know that for any given value distribution, the Lloyd-Max algorithm minimizes the quantization MSE. Therefore, for a given vector cluster $\mathcal{B}_i$ we obtain the following result:

\begin{equation}
    \frac{1}{|\mathcal{B}_{i}^{(n+1)}|}\sum_{\bm{b} \in \mathcal{B}_{i}^{(n+1)}} \frac{1}{L_b}\lVert \bm{b}- C_i^{(n+1)}(\bm{b})\rVert^2_2 \le \frac{1}{|\mathcal{B}_{i}^{(n+1)}|}\sum_{\bm{b} \in \mathcal{B}_{i}^{(n+1)}} \frac{1}{L_b}\lVert \bm{b}- C_i^{(n)}(\bm{b})\rVert^2_2
\end{equation}

The above equation states that quantizing the given block cluster $\mathcal{B}_i$ after updating the associated codebook from $C_i^{(n)}$ to $C_i^{(n+1)}$ results in lower quantization MSE. Since this is true for all the block clusters, we derive the following result: 
\begin{equation}
\begin{split}
\label{eq:mse_ineq_2}
     J^{(n+1)} &= \frac{1}{N_c} \sum_{i=1}^{N_c} \frac{1}{|\mathcal{B}_{i}^{(n+1)}|}\sum_{\bm{b} \in \mathcal{B}_{i}^{(n+1)}} \frac{1}{L_b}\lVert \bm{b}- C_i^{(n+1)}(\bm{b})\rVert^2_2  \le \tilde{J}^{(n+1)}   
\end{split}
\end{equation}

Following (\ref{eq:mse_ineq_1}) and (\ref{eq:mse_ineq_2}), we find that the quantization MSE is non-increasing for each iteration, that is, $J^{(1)} \ge J^{(2)} \ge J^{(3)} \ge \ldots \ge J^{(M)}$ where $M$ is the maximum number of iterations. 
%Therefore, we can say that if the algorithm converges, then it must be that it has converged to a local minimum. 
\hfill $\blacksquare$


\begin{figure}
    \begin{center}
    \includegraphics[width=0.5\textwidth]{sections//figures/mse_vs_iter.pdf}
    \end{center}
    \caption{\small NMSE vs iterations during LO-BCQ compared to other block quantization proposals}
    \label{fig:nmse_vs_iter}
\end{figure}

Figure \ref{fig:nmse_vs_iter} shows the empirical convergence of LO-BCQ across several block lengths and number of codebooks. Also, the MSE achieved by LO-BCQ is compared to baselines such as MXFP and VSQ. As shown, LO-BCQ converges to a lower MSE than the baselines. Further, we achieve better convergence for larger number of codebooks ($N_c$) and for a smaller block length ($L_b$), both of which increase the bitwidth of BCQ (see Eq \ref{eq:bitwidth_bcq}).


\subsection{Additional Accuracy Results}
%Table \ref{tab:lobcq_config} lists the various LOBCQ configurations and their corresponding bitwidths.
\begin{table}
\setlength{\tabcolsep}{4.75pt}
\begin{center}
\caption{\label{tab:lobcq_config} Various LO-BCQ configurations and their bitwidths.}
\begin{tabular}{|c||c|c|c|c||c|c||c|} 
\hline
 & \multicolumn{4}{|c||}{$L_b=8$} & \multicolumn{2}{|c||}{$L_b=4$} & $L_b=2$ \\
 \hline
 \backslashbox{$L_A$\kern-1em}{\kern-1em$N_c$} & 2 & 4 & 8 & 16 & 2 & 4 & 2 \\
 \hline
 64 & 4.25 & 4.375 & 4.5 & 4.625 & 4.375 & 4.625 & 4.625\\
 \hline
 32 & 4.375 & 4.5 & 4.625& 4.75 & 4.5 & 4.75 & 4.75 \\
 \hline
 16 & 4.625 & 4.75& 4.875 & 5 & 4.75 & 5 & 5 \\
 \hline
\end{tabular}
\end{center}
\end{table}

%\subsection{Perplexity achieved by various LO-BCQ configurations on Wikitext-103 dataset}

\begin{table} \centering
\begin{tabular}{|c||c|c|c|c||c|c||c|} 
\hline
 $L_b \rightarrow$& \multicolumn{4}{c||}{8} & \multicolumn{2}{c||}{4} & 2\\
 \hline
 \backslashbox{$L_A$\kern-1em}{\kern-1em$N_c$} & 2 & 4 & 8 & 16 & 2 & 4 & 2  \\
 %$N_c \rightarrow$ & 2 & 4 & 8 & 16 & 2 & 4 & 2 \\
 \hline
 \hline
 \multicolumn{8}{c}{GPT3-1.3B (FP32 PPL = 9.98)} \\ 
 \hline
 \hline
 64 & 10.40 & 10.23 & 10.17 & 10.15 &  10.28 & 10.18 & 10.19 \\
 \hline
 32 & 10.25 & 10.20 & 10.15 & 10.12 &  10.23 & 10.17 & 10.17 \\
 \hline
 16 & 10.22 & 10.16 & 10.10 & 10.09 &  10.21 & 10.14 & 10.16 \\
 \hline
  \hline
 \multicolumn{8}{c}{GPT3-8B (FP32 PPL = 7.38)} \\ 
 \hline
 \hline
 64 & 7.61 & 7.52 & 7.48 &  7.47 &  7.55 &  7.49 & 7.50 \\
 \hline
 32 & 7.52 & 7.50 & 7.46 &  7.45 &  7.52 &  7.48 & 7.48  \\
 \hline
 16 & 7.51 & 7.48 & 7.44 &  7.44 &  7.51 &  7.49 & 7.47  \\
 \hline
\end{tabular}
\caption{\label{tab:ppl_gpt3_abalation} Wikitext-103 perplexity across GPT3-1.3B and 8B models.}
\end{table}

\begin{table} \centering
\begin{tabular}{|c||c|c|c|c||} 
\hline
 $L_b \rightarrow$& \multicolumn{4}{c||}{8}\\
 \hline
 \backslashbox{$L_A$\kern-1em}{\kern-1em$N_c$} & 2 & 4 & 8 & 16 \\
 %$N_c \rightarrow$ & 2 & 4 & 8 & 16 & 2 & 4 & 2 \\
 \hline
 \hline
 \multicolumn{5}{|c|}{Llama2-7B (FP32 PPL = 5.06)} \\ 
 \hline
 \hline
 64 & 5.31 & 5.26 & 5.19 & 5.18  \\
 \hline
 32 & 5.23 & 5.25 & 5.18 & 5.15  \\
 \hline
 16 & 5.23 & 5.19 & 5.16 & 5.14  \\
 \hline
 \multicolumn{5}{|c|}{Nemotron4-15B (FP32 PPL = 5.87)} \\ 
 \hline
 \hline
 64  & 6.3 & 6.20 & 6.13 & 6.08  \\
 \hline
 32  & 6.24 & 6.12 & 6.07 & 6.03  \\
 \hline
 16  & 6.12 & 6.14 & 6.04 & 6.02  \\
 \hline
 \multicolumn{5}{|c|}{Nemotron4-340B (FP32 PPL = 3.48)} \\ 
 \hline
 \hline
 64 & 3.67 & 3.62 & 3.60 & 3.59 \\
 \hline
 32 & 3.63 & 3.61 & 3.59 & 3.56 \\
 \hline
 16 & 3.61 & 3.58 & 3.57 & 3.55 \\
 \hline
\end{tabular}
\caption{\label{tab:ppl_llama7B_nemo15B} Wikitext-103 perplexity compared to FP32 baseline in Llama2-7B and Nemotron4-15B, 340B models}
\end{table}

%\subsection{Perplexity achieved by various LO-BCQ configurations on MMLU dataset}


\begin{table} \centering
\begin{tabular}{|c||c|c|c|c||c|c|c|c|} 
\hline
 $L_b \rightarrow$& \multicolumn{4}{c||}{8} & \multicolumn{4}{c||}{8}\\
 \hline
 \backslashbox{$L_A$\kern-1em}{\kern-1em$N_c$} & 2 & 4 & 8 & 16 & 2 & 4 & 8 & 16  \\
 %$N_c \rightarrow$ & 2 & 4 & 8 & 16 & 2 & 4 & 2 \\
 \hline
 \hline
 \multicolumn{5}{|c|}{Llama2-7B (FP32 Accuracy = 45.8\%)} & \multicolumn{4}{|c|}{Llama2-70B (FP32 Accuracy = 69.12\%)} \\ 
 \hline
 \hline
 64 & 43.9 & 43.4 & 43.9 & 44.9 & 68.07 & 68.27 & 68.17 & 68.75 \\
 \hline
 32 & 44.5 & 43.8 & 44.9 & 44.5 & 68.37 & 68.51 & 68.35 & 68.27  \\
 \hline
 16 & 43.9 & 42.7 & 44.9 & 45 & 68.12 & 68.77 & 68.31 & 68.59  \\
 \hline
 \hline
 \multicolumn{5}{|c|}{GPT3-22B (FP32 Accuracy = 38.75\%)} & \multicolumn{4}{|c|}{Nemotron4-15B (FP32 Accuracy = 64.3\%)} \\ 
 \hline
 \hline
 64 & 36.71 & 38.85 & 38.13 & 38.92 & 63.17 & 62.36 & 63.72 & 64.09 \\
 \hline
 32 & 37.95 & 38.69 & 39.45 & 38.34 & 64.05 & 62.30 & 63.8 & 64.33  \\
 \hline
 16 & 38.88 & 38.80 & 38.31 & 38.92 & 63.22 & 63.51 & 63.93 & 64.43  \\
 \hline
\end{tabular}
\caption{\label{tab:mmlu_abalation} Accuracy on MMLU dataset across GPT3-22B, Llama2-7B, 70B and Nemotron4-15B models.}
\end{table}


%\subsection{Perplexity achieved by various LO-BCQ configurations on LM evaluation harness}

\begin{table} \centering
\begin{tabular}{|c||c|c|c|c||c|c|c|c|} 
\hline
 $L_b \rightarrow$& \multicolumn{4}{c||}{8} & \multicolumn{4}{c||}{8}\\
 \hline
 \backslashbox{$L_A$\kern-1em}{\kern-1em$N_c$} & 2 & 4 & 8 & 16 & 2 & 4 & 8 & 16  \\
 %$N_c \rightarrow$ & 2 & 4 & 8 & 16 & 2 & 4 & 2 \\
 \hline
 \hline
 \multicolumn{5}{|c|}{Race (FP32 Accuracy = 37.51\%)} & \multicolumn{4}{|c|}{Boolq (FP32 Accuracy = 64.62\%)} \\ 
 \hline
 \hline
 64 & 36.94 & 37.13 & 36.27 & 37.13 & 63.73 & 62.26 & 63.49 & 63.36 \\
 \hline
 32 & 37.03 & 36.36 & 36.08 & 37.03 & 62.54 & 63.51 & 63.49 & 63.55  \\
 \hline
 16 & 37.03 & 37.03 & 36.46 & 37.03 & 61.1 & 63.79 & 63.58 & 63.33  \\
 \hline
 \hline
 \multicolumn{5}{|c|}{Winogrande (FP32 Accuracy = 58.01\%)} & \multicolumn{4}{|c|}{Piqa (FP32 Accuracy = 74.21\%)} \\ 
 \hline
 \hline
 64 & 58.17 & 57.22 & 57.85 & 58.33 & 73.01 & 73.07 & 73.07 & 72.80 \\
 \hline
 32 & 59.12 & 58.09 & 57.85 & 58.41 & 73.01 & 73.94 & 72.74 & 73.18  \\
 \hline
 16 & 57.93 & 58.88 & 57.93 & 58.56 & 73.94 & 72.80 & 73.01 & 73.94  \\
 \hline
\end{tabular}
\caption{\label{tab:mmlu_abalation} Accuracy on LM evaluation harness tasks on GPT3-1.3B model.}
\end{table}

\begin{table} \centering
\begin{tabular}{|c||c|c|c|c||c|c|c|c|} 
\hline
 $L_b \rightarrow$& \multicolumn{4}{c||}{8} & \multicolumn{4}{c||}{8}\\
 \hline
 \backslashbox{$L_A$\kern-1em}{\kern-1em$N_c$} & 2 & 4 & 8 & 16 & 2 & 4 & 8 & 16  \\
 %$N_c \rightarrow$ & 2 & 4 & 8 & 16 & 2 & 4 & 2 \\
 \hline
 \hline
 \multicolumn{5}{|c|}{Race (FP32 Accuracy = 41.34\%)} & \multicolumn{4}{|c|}{Boolq (FP32 Accuracy = 68.32\%)} \\ 
 \hline
 \hline
 64 & 40.48 & 40.10 & 39.43 & 39.90 & 69.20 & 68.41 & 69.45 & 68.56 \\
 \hline
 32 & 39.52 & 39.52 & 40.77 & 39.62 & 68.32 & 67.43 & 68.17 & 69.30  \\
 \hline
 16 & 39.81 & 39.71 & 39.90 & 40.38 & 68.10 & 66.33 & 69.51 & 69.42  \\
 \hline
 \hline
 \multicolumn{5}{|c|}{Winogrande (FP32 Accuracy = 67.88\%)} & \multicolumn{4}{|c|}{Piqa (FP32 Accuracy = 78.78\%)} \\ 
 \hline
 \hline
 64 & 66.85 & 66.61 & 67.72 & 67.88 & 77.31 & 77.42 & 77.75 & 77.64 \\
 \hline
 32 & 67.25 & 67.72 & 67.72 & 67.00 & 77.31 & 77.04 & 77.80 & 77.37  \\
 \hline
 16 & 68.11 & 68.90 & 67.88 & 67.48 & 77.37 & 78.13 & 78.13 & 77.69  \\
 \hline
\end{tabular}
\caption{\label{tab:mmlu_abalation} Accuracy on LM evaluation harness tasks on GPT3-8B model.}
\end{table}

\begin{table} \centering
\begin{tabular}{|c||c|c|c|c||c|c|c|c|} 
\hline
 $L_b \rightarrow$& \multicolumn{4}{c||}{8} & \multicolumn{4}{c||}{8}\\
 \hline
 \backslashbox{$L_A$\kern-1em}{\kern-1em$N_c$} & 2 & 4 & 8 & 16 & 2 & 4 & 8 & 16  \\
 %$N_c \rightarrow$ & 2 & 4 & 8 & 16 & 2 & 4 & 2 \\
 \hline
 \hline
 \multicolumn{5}{|c|}{Race (FP32 Accuracy = 40.67\%)} & \multicolumn{4}{|c|}{Boolq (FP32 Accuracy = 76.54\%)} \\ 
 \hline
 \hline
 64 & 40.48 & 40.10 & 39.43 & 39.90 & 75.41 & 75.11 & 77.09 & 75.66 \\
 \hline
 32 & 39.52 & 39.52 & 40.77 & 39.62 & 76.02 & 76.02 & 75.96 & 75.35  \\
 \hline
 16 & 39.81 & 39.71 & 39.90 & 40.38 & 75.05 & 73.82 & 75.72 & 76.09  \\
 \hline
 \hline
 \multicolumn{5}{|c|}{Winogrande (FP32 Accuracy = 70.64\%)} & \multicolumn{4}{|c|}{Piqa (FP32 Accuracy = 79.16\%)} \\ 
 \hline
 \hline
 64 & 69.14 & 70.17 & 70.17 & 70.56 & 78.24 & 79.00 & 78.62 & 78.73 \\
 \hline
 32 & 70.96 & 69.69 & 71.27 & 69.30 & 78.56 & 79.49 & 79.16 & 78.89  \\
 \hline
 16 & 71.03 & 69.53 & 69.69 & 70.40 & 78.13 & 79.16 & 79.00 & 79.00  \\
 \hline
\end{tabular}
\caption{\label{tab:mmlu_abalation} Accuracy on LM evaluation harness tasks on GPT3-22B model.}
\end{table}

\begin{table} \centering
\begin{tabular}{|c||c|c|c|c||c|c|c|c|} 
\hline
 $L_b \rightarrow$& \multicolumn{4}{c||}{8} & \multicolumn{4}{c||}{8}\\
 \hline
 \backslashbox{$L_A$\kern-1em}{\kern-1em$N_c$} & 2 & 4 & 8 & 16 & 2 & 4 & 8 & 16  \\
 %$N_c \rightarrow$ & 2 & 4 & 8 & 16 & 2 & 4 & 2 \\
 \hline
 \hline
 \multicolumn{5}{|c|}{Race (FP32 Accuracy = 44.4\%)} & \multicolumn{4}{|c|}{Boolq (FP32 Accuracy = 79.29\%)} \\ 
 \hline
 \hline
 64 & 42.49 & 42.51 & 42.58 & 43.45 & 77.58 & 77.37 & 77.43 & 78.1 \\
 \hline
 32 & 43.35 & 42.49 & 43.64 & 43.73 & 77.86 & 75.32 & 77.28 & 77.86  \\
 \hline
 16 & 44.21 & 44.21 & 43.64 & 42.97 & 78.65 & 77 & 76.94 & 77.98  \\
 \hline
 \hline
 \multicolumn{5}{|c|}{Winogrande (FP32 Accuracy = 69.38\%)} & \multicolumn{4}{|c|}{Piqa (FP32 Accuracy = 78.07\%)} \\ 
 \hline
 \hline
 64 & 68.9 & 68.43 & 69.77 & 68.19 & 77.09 & 76.82 & 77.09 & 77.86 \\
 \hline
 32 & 69.38 & 68.51 & 68.82 & 68.90 & 78.07 & 76.71 & 78.07 & 77.86  \\
 \hline
 16 & 69.53 & 67.09 & 69.38 & 68.90 & 77.37 & 77.8 & 77.91 & 77.69  \\
 \hline
\end{tabular}
\caption{\label{tab:mmlu_abalation} Accuracy on LM evaluation harness tasks on Llama2-7B model.}
\end{table}

\begin{table} \centering
\begin{tabular}{|c||c|c|c|c||c|c|c|c|} 
\hline
 $L_b \rightarrow$& \multicolumn{4}{c||}{8} & \multicolumn{4}{c||}{8}\\
 \hline
 \backslashbox{$L_A$\kern-1em}{\kern-1em$N_c$} & 2 & 4 & 8 & 16 & 2 & 4 & 8 & 16  \\
 %$N_c \rightarrow$ & 2 & 4 & 8 & 16 & 2 & 4 & 2 \\
 \hline
 \hline
 \multicolumn{5}{|c|}{Race (FP32 Accuracy = 48.8\%)} & \multicolumn{4}{|c|}{Boolq (FP32 Accuracy = 85.23\%)} \\ 
 \hline
 \hline
 64 & 49.00 & 49.00 & 49.28 & 48.71 & 82.82 & 84.28 & 84.03 & 84.25 \\
 \hline
 32 & 49.57 & 48.52 & 48.33 & 49.28 & 83.85 & 84.46 & 84.31 & 84.93  \\
 \hline
 16 & 49.85 & 49.09 & 49.28 & 48.99 & 85.11 & 84.46 & 84.61 & 83.94  \\
 \hline
 \hline
 \multicolumn{5}{|c|}{Winogrande (FP32 Accuracy = 79.95\%)} & \multicolumn{4}{|c|}{Piqa (FP32 Accuracy = 81.56\%)} \\ 
 \hline
 \hline
 64 & 78.77 & 78.45 & 78.37 & 79.16 & 81.45 & 80.69 & 81.45 & 81.5 \\
 \hline
 32 & 78.45 & 79.01 & 78.69 & 80.66 & 81.56 & 80.58 & 81.18 & 81.34  \\
 \hline
 16 & 79.95 & 79.56 & 79.79 & 79.72 & 81.28 & 81.66 & 81.28 & 80.96  \\
 \hline
\end{tabular}
\caption{\label{tab:mmlu_abalation} Accuracy on LM evaluation harness tasks on Llama2-70B model.}
\end{table}

%\section{MSE Studies}
%\textcolor{red}{TODO}


\subsection{Number Formats and Quantization Method}
\label{subsec:numFormats_quantMethod}
\subsubsection{Integer Format}
An $n$-bit signed integer (INT) is typically represented with a 2s-complement format \citep{yao2022zeroquant,xiao2023smoothquant,dai2021vsq}, where the most significant bit denotes the sign.

\subsubsection{Floating Point Format}
An $n$-bit signed floating point (FP) number $x$ comprises of a 1-bit sign ($x_{\mathrm{sign}}$), $B_m$-bit mantissa ($x_{\mathrm{mant}}$) and $B_e$-bit exponent ($x_{\mathrm{exp}}$) such that $B_m+B_e=n-1$. The associated constant exponent bias ($E_{\mathrm{bias}}$) is computed as $(2^{{B_e}-1}-1)$. We denote this format as $E_{B_e}M_{B_m}$.  

\subsubsection{Quantization Scheme}
\label{subsec:quant_method}
A quantization scheme dictates how a given unquantized tensor is converted to its quantized representation. We consider FP formats for the purpose of illustration. Given an unquantized tensor $\bm{X}$ and an FP format $E_{B_e}M_{B_m}$, we first, we compute the quantization scale factor $s_X$ that maps the maximum absolute value of $\bm{X}$ to the maximum quantization level of the $E_{B_e}M_{B_m}$ format as follows:
\begin{align}
\label{eq:sf}
    s_X = \frac{\mathrm{max}(|\bm{X}|)}{\mathrm{max}(E_{B_e}M_{B_m})}
\end{align}
In the above equation, $|\cdot|$ denotes the absolute value function.

Next, we scale $\bm{X}$ by $s_X$ and quantize it to $\hat{\bm{X}}$ by rounding it to the nearest quantization level of $E_{B_e}M_{B_m}$ as:

\begin{align}
\label{eq:tensor_quant}
    \hat{\bm{X}} = \text{round-to-nearest}\left(\frac{\bm{X}}{s_X}, E_{B_e}M_{B_m}\right)
\end{align}

We perform dynamic max-scaled quantization \citep{wu2020integer}, where the scale factor $s$ for activations is dynamically computed during runtime.

\subsection{Vector Scaled Quantization}
\begin{wrapfigure}{r}{0.35\linewidth}
  \centering
  \includegraphics[width=\linewidth]{sections/figures/vsquant.jpg}
  \caption{\small Vectorwise decomposition for per-vector scaled quantization (VSQ \citep{dai2021vsq}).}
  \label{fig:vsquant}
\end{wrapfigure}
During VSQ \citep{dai2021vsq}, the operand tensors are decomposed into 1D vectors in a hardware friendly manner as shown in Figure \ref{fig:vsquant}. Since the decomposed tensors are used as operands in matrix multiplications during inference, it is beneficial to perform this decomposition along the reduction dimension of the multiplication. The vectorwise quantization is performed similar to tensorwise quantization described in Equations \ref{eq:sf} and \ref{eq:tensor_quant}, where a scale factor $s_v$ is required for each vector $\bm{v}$ that maps the maximum absolute value of that vector to the maximum quantization level. While smaller vector lengths can lead to larger accuracy gains, the associated memory and computational overheads due to the per-vector scale factors increases. To alleviate these overheads, VSQ \citep{dai2021vsq} proposed a second level quantization of the per-vector scale factors to unsigned integers, while MX \citep{rouhani2023shared} quantizes them to integer powers of 2 (denoted as $2^{INT}$).

\subsubsection{MX Format}
The MX format proposed in \citep{rouhani2023microscaling} introduces the concept of sub-block shifting. For every two scalar elements of $b$-bits each, there is a shared exponent bit. The value of this exponent bit is determined through an empirical analysis that targets minimizing quantization MSE. We note that the FP format $E_{1}M_{b}$ is strictly better than MX from an accuracy perspective since it allocates a dedicated exponent bit to each scalar as opposed to sharing it across two scalars. Therefore, we conservatively bound the accuracy of a $b+2$-bit signed MX format with that of a $E_{1}M_{b}$ format in our comparisons. For instance, we use E1M2 format as a proxy for MX4.

\begin{figure}
    \centering
    \includegraphics[width=1\linewidth]{sections//figures/BlockFormats.pdf}
    \caption{\small Comparing LO-BCQ to MX format.}
    \label{fig:block_formats}
\end{figure}

Figure \ref{fig:block_formats} compares our $4$-bit LO-BCQ block format to MX \citep{rouhani2023microscaling}. As shown, both LO-BCQ and MX decompose a given operand tensor into block arrays and each block array into blocks. Similar to MX, we find that per-block quantization ($L_b < L_A$) leads to better accuracy due to increased flexibility. While MX achieves this through per-block $1$-bit micro-scales, we associate a dedicated codebook to each block through a per-block codebook selector. Further, MX quantizes the per-block array scale-factor to E8M0 format without per-tensor scaling. In contrast during LO-BCQ, we find that per-tensor scaling combined with quantization of per-block array scale-factor to E4M3 format results in superior inference accuracy across models. 


\end{document}

