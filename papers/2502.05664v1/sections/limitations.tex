% \newpage

\section{Limitations}
In Section \ref{subsec:ldb-debugger}, we observe that utilizing an external debugger can further enhance our results. Our next research goal is to achieve the best performance without relying on any external tools. Although we have reduced token consumption compared to the previous state-of-the-art method, MapCoder, it still remains high compared to the direct prompting approach. Direct prompting consumes an average of 560 tokens, while our method consumes around 13,640 tokens. This indicates room for enhancement in efficiency. While in this work, we generate the exemplars with the LLMs themselves, in general they are found from external resource \cite{parvez-chang-2021-evaluating}. Although this has its own challenges such as noisy retrievals \cite{filco}, inconsistent generations \cite{islam-etal-2024-open, parvez2024evidence, sadat-etal-2023-delucionqa} this direction could also be a possible improvement. Another limitation is the use of external tools for assistance during simulation. We have not explored this avenue in the current research, leaving it for future work. Additionally, more sample I/Os could potentially improve performance, and our future research will focus on investigating methods for generating accurate additional I/Os. Moreover, we would like to note that in this work, we focus solely on generated code's correctness and did not study its optimizations such as test-time, memory. Finally, it is advisable to run the machine generated code inside a sandbox to avoid any potential risks.
