The Mini Wheelbot is a balancing, reaction wheel unicycle robot designed as a testbed for learning-based control.
It is an unstable system with highly nonlinear yaw dynamics, non-holonomic driving, and discrete contact switches in a small, powerful, and rugged form factor.
The Mini Wheelbot can use its wheels to stand up from any initial orientation -- enabling automatic environment resets in repetitive experiments and even challenging half flips.
We illustrate the effectiveness of the Mini Wheelbot as a testbed by implementing two popular learning-based control algorithms.
First, we showcase Bayesian optimization for tuning the balancing controller.
Second, we use imitation learning from an expert nonlinear MPC that uses gyroscopic effects to reorient the robot and can track higher-level velocity and orientation commands.
The latter allows the robot to drive around based on user commands -- for the first time in this class of robots.
The Mini Wheelbot is not only compelling for testing learning-based control algorithms, but it is also just fun to work with, as demonstrated in the video of our experiments \videolink.
