Small-scale driving~\cite{mondada1994mobile,johnson2006mobile,mondada2009puck,rubenstein2015aerobot,pickem2015gritsbot,wilson2016pheeno, paull2017duckietown, hsieh2006economical, liniger2015optimization, carron2023chronos, bodmer2024optimization, goldfain2019autorally, o2020f1tenth, gonzales2018planning} and balancing~\cite{gajamohan2012cubli, mayr2015mechatronic, muehlebach2016nonlinear, hofer2023one, klemm2019ascento, nagarajan2014ballbot, geist2022wheelbot} robots are a popular choice for control experiments, though none is specifically designed for testing learning algorithms.
This section summarizes existing designs, highlighting that either articulated driving or balancing of an unstable system is achieved.
In contrast, the Mini Wheelbot combines both with highly nonlinear yaw dynamics and automatic environment resets ideal for learning experiments.









\subsection{Driving Robots}
Driving robots have been developed for decades, commonly with slow and stable differential drives~\cite{mondada1994mobile,johnson2006mobile,mondada2009puck,rubenstein2015aerobot,pickem2015gritsbot,wilson2016pheeno, paull2017duckietown} or fast and car-like dynamics for autonomous racing~\cite{hsieh2006economical, liniger2015optimization, carron2023chronos, bodmer2024optimization, goldfain2019autorally, o2020f1tenth, gonzales2018planning}.
Differential-drive platforms are cheap to produce in large quantities for testing networked and multi-robot control algorithms~\cite{johnson2006mobile, paull2017duckietown, wilson2016pheeno,pickem2015gritsbot, pickem2017robotarium, schwab2020experimental}, or as education systems~\cite{mondada1994mobile, mondada2009puck, rubenstein2015aerobot}.
Research questions arise mainly from high-level perception and coordination while simple kinematic planning and state-feedback controllers perform well in steering.
On car-like robots, pushing the limits of autonomous racing requires precise control of fast dynamics and difficult-to-model tire slip effects, however, the underlying contouring control problem can be linearized and is inherently stable when driving slowly, hence even PID schemes succeed in this task~\cite{carron2023chronos}.
In comparison, the Mini Wheelbot is unstable and has interesting nonlinear tasks (yaw control) that can not be achieved by classic linear methods.

\subsection{Balancing Robots}
Balancing is a long-standing challenge in robotic research, for example in pendulum sculptures~\cite{trimpe2012balancing,gajamohan2012cubli, mayr2015mechatronic,hofer2023one}, unicycle robots~\cite{geist2022wheelbot, schoonwinkel1988design, vos1990dynamics, xu2011pendulum, daud2017dynamic, lee2012decoupled, jae2011fuzzy, li2012attitude, rosyidi2016speed, neves2021discrete, rizal2015point, jin2010balancing} like the Mini Wheelbot, and even legged~\cite{klemm2019ascento} or ball~\cite{nagarajan2014ballbot} robots.
Cube-like robots stabilize standing on a corner with leavers~\cite{trimpe2012balancing} or reaction wheels~\cite{gajamohan2012cubli,mayr2015mechatronic,hofer2023one}, where nonlinearity~\cite{muehlebach2016nonlinear} and underactuation~\cite{hofer2023one} inspire research.
While some cube-like robots can automatically stand up~\cite{gajamohan2012cubli, mayr2015mechatronic, muehlebach2016nonlinear}, their mobility is restricted to walking-like sequences of controlled stand-up and falling.
Similarly, the Mini Wheelbot can use its reaction wheel to balance on the point contact of the driving wheel, stand up from any initial position, and has underactuated yaw dynamics.
That is, when in perfect balance, the Mini Wheelbot has no direct control over it's yaw angle.
Compared to cube-like robots, however, the driving wheel allows for mobility.

Balancing unicycle robots use levers~\cite{schoonwinkel1988design,jin2010balancing,xu2011pendulum,daud2017dynamic} or reaction wheels~\cite{geist2022wheelbot,neves2021discrete,jae2011fuzzy, rosyidi2016speed,lee2012decoupled,rizal2015point}.
Except for an early prototype~\cite{geist2022wheelbot}, existing designs' actuators lack power for a stand-up and are asymmetric, which prohibits interesting chaining of stand-up maneuvers.
Roll and pitch balancing with linear state-feedback or fuzzy controllers is well understood~\cite{xu2011pendulum, jae2011fuzzy, lee2012decoupled, neves2021discrete}, also with reference velocities~\cite{rosyidi2016speed}.
However, so far, a third turntable actuator was required to control the yaw orientation~\cite{schoonwinkel1988design,vos1990dynamics,rizal2015point,jin2010balancing} and thus permit meaningful driving.
In contrast, we use nonlinear methods to control yaw without an additional turntable actuator for the first time.
