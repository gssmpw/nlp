\documentclass[conference]{IEEEtran}

\IEEEoverridecommandlockouts
\usepackage{cite}
\usepackage{amsmath,amssymb,amsfonts}
\usepackage{algorithm}
\usepackage{algorithmic}
\renewcommand{\algorithmicrequire}{\textbf{Input:}} 
\renewcommand{\algorithmicensure}{\textbf{Output:}} 
\usepackage{graphicx}
\usepackage{fancyhdr}
\usepackage{pifont}
\usepackage{wasysym}
\usepackage{utfsym}
\usepackage{enumerate}
\usepackage{textcomp}
\usepackage{listings}
\usepackage{xcolor}
\usepackage{tcolorbox}
\usepackage{longtable}
\usepackage{caption}
\usepackage{subcaption}
\usepackage{booktabs} % 用于更好的表格线条
\usepackage{colortbl}
\makeatletter
\newcommand{\rmnum}[1]{\romannumeral #1}
\newcommand{\Rmnum}[1]{\expandafter\@slowromancap\romannumeral #1@}
\makeatother
\usepackage[font=small]{caption}
\usepackage{multirow}
\usepackage{multicol}
\usepackage{bm}
%\usepackage[colorlinks,urlcolor=black,linkcolor=black,anchorcolor=blue,citecolor=green]{hyperref}
\usepackage[numbers,sort&compress]{natbib} %references style
\usepackage{lipsum}% generate text for the example
%\usepackage[a4paper, total={184mm,239mm}]{geometry}
\def\BibTeX{{\rm B\kern-.05em{\sc i\kern-.025em b}\kern-.08em
    T\kern-.1667em\lower.7ex\hbox{E}\kern-.125emX}}
    
\fancypagestyle{firstpagefooter}{%
  \fancyhf{}
  \renewcommand\headrulewidth{0pt}
  % \fancyfoot[R]{Menelaos-NT Research Report template by Zhouyan Qiu, University of Vigo}
}

\usepackage{footmisc} % 脚注控制包

\renewcommand\thefootnote{} % 清空脚注编号

\definecolor{pink}{rgb}{1, 0.5, 0.5}
\definecolor{mygreen}{RGB}{0, 176, 80} 
\definecolor{myred}{RGB}{222, 26, 55}
\definecolor{myyellow}{RGB}{255, 152, 0}
\definecolor{myblue}{RGB}{0, 122, 192}
% 设置代码框的样式
\lstdefinestyle{mystyle}{
	basicstyle=\ttfamily\scriptsize\bfseries,
	frame=single,
	framerule=0.7pt,
	breaklines=true,
%	numbers=left,
	numberstyle=\tiny\color{gray},
	numbersep=8pt,
	keywordstyle=\color{blue}\bfseries,
	commentstyle=\color{green},
	stringstyle=\color{orange},
	showstringspaces=false,
	escapeinside={(*@}{@*)},	
}

\lstdefinestyle{mystyle1}{
	keywords={module, input, output, reg, wire, always, posedge, negedge, if, else, begin, end, endmodule},
	keywordstyle=\color{blue}\bfseries,
	morecomment=[l]{//},
	commentstyle=\color{green},
	basicstyle=\ttfamily\footnotesize\bfseries,  % Font size for code
	numbers=left,                       % Line numbers on the left
	numberstyle=\tiny\color{gray},      % Line number style
	numbersep=8pt,                      % Space between line numbers and code                  % Adjust left margin for better alignment
	frame=single,   
	framerule=0.7pt,                      % Remove inner frame from listings
	showstringspaces=false,             % Do not show spaces in strings
	breaklines=true,                    % Enable line breaking
	escapeinside={(*@}{@*)},
}

\pagestyle{empty}

\begin{document}

%\title{Verilog Linting With Large Language Models\vspace{-0.5em}}
\title{LintLLM: An Open-Source Verilog Linting Framework Based on Large Language Models\vspace{-0.5em}}
% 星号变数字

\author{
	\IEEEauthorblockN{Zhigang Fang\textsuperscript{1}, Renzhi Chen\textsuperscript{2}*, Zhijie Yang\textsuperscript{3}, Yang Guo\textsuperscript{1}, Huadong Dai\textsuperscript{3}, Lei Wang\textsuperscript{3}*}
	\IEEEauthorblockA{\textsuperscript{1}\textit{National University of Defense Technology}, Changsha, China \\
		\textsuperscript{2}\textit{Qiyuan Lab}, Beijing, China \\
		\textsuperscript{3}\textit{Defense Innovation Institute, Academy of Military Sciences}, Beijing, China \\
		Email: $\{$fangzhigang, leiwang$\}$@nudt.edu.cn, chenrenzhi@qiyuanlab.com}\vspace{-3em}
}

\maketitle 


\renewcommand{\footnoterule}{%
	\hrule width 0.4\textwidth height 0.5pt \kern 2pt % 设置横线长度和粗细
}

\makeatletter
\renewcommand{\@makefntext}[1]{%
	\leftskip=0pt % 取消缩进
	\noindent % 取消首行缩进
	\@makefnmark\hspace{0.5em}#1%
}

\footnotetext {*Co-Corresponding author: Renzhi Chen and Lei Wang. \vspace{-2.5em}}

\begin{abstract}

Code Linting tools are vital for detecting potential defects in Verilog code. However, the limitations of traditional Linting tools are evident in frequent false positives and redundant defect reports. Recent advancements in large language models (LLM) have introduced new possibilities in this area. In this paper, we propose LintLLM, an open-source Linting framework that utilizes LLMs to detect defects in Verilog code via Prompt of Logic-Tree and Defect Tracker. Furthermore, we create an open-source benchmark using the mutation-based defect injection technique to evaluate LLM's ability in detecting Verilog defects. Experimental results show that o1-mini improves the correct rate by 18.89\% and reduces the false-positive rate by 15.56\% compared with the best-performing EDA tool. Simultaneously, LintLLM operates at less than one-tenth of the cost of commercial EDA tools. This study demonstrates the potential of LLM as an efficient and cost-effective Linting tool for hardware design. The benchmark and experimental results are open-source at URL: https://github.com/fangzhigang32/Static-Verilog-Analysis

\end{abstract}

\begin{IEEEkeywords}
LLM, Lint, Code defect detection, EDA
\end{IEEEkeywords}

\thispagestyle{firstpagefooter}

\section{Introduction}

Ensuring code quality and maintaining consistent coding styles is essential for producing robust and defect-free Register-Transfer Level (RTL) designs \cite{deng2023verilog}. Code Linting tools can significantly reduce verification costs by analyzing source code for potential defects, such as inconsistent coding styles and unsynthesizable constructs, which differ from bugs as they do not necessarily lead to errors \cite{stefanovic2020static}. Commercial EDA tools (e.g. SpyGlass \cite{SpyGlass}) detect Verilog defects via matching Design Under Test (DUT) with predefined rules, which pattern causes a large number of false positives \cite{novak2010taxonomy}. Although customizing rules that meet the design specifications can alleviate this symptom, analyzing and extracting these rules is difficult. Additionally, expensive licensing fees burden small and medium-sized enterprises and research institutions. Previous open-source EDA tools (e.g. Verilator \cite{Verilator}) tended to exhibit lower performance, which increases the complexity for designers to deal with code defects and prolongs debugging time \cite{firdous2019speeding}. These challenges affect the promotion and popularity of traditional Linting tools, highlighting the need for a comprehensive and open-source Linting tool.

Recently, LLMs such as GPT-4 have demonstrated significant potential in Verilog code debugging. In terms of function bugs, LLMs combines design specifications and DUTs to achieve efficient bug localization without testbench \cite{journals/corr/abs-2409-15186}. For syntax bugs, RTLFixer \cite{conf/dac/TsaiLR24} utilizes the Retrieval Augmented Generation (RAG) strategy to greatly improve the accuracy and efficiency of automatically repairing bugs. To optimize the debugging process, MEIC \cite{journals/corr/abs-2405-06840} proposes an iterative framework to accelerate the debugging of complex hardware through multiple rounds of feedback. LLM4SecHW \cite{conf/asianhost/FuYDGQ23} and HDLdebugger \cite{journals/corr/abs-2403-11671} further improve the LLM's application effect in hardware debugging through domain data fine-tuning and reverse engineering. These studies highlight the vast potential of LLMs in assisting RTL Coding and opening new avenues for EDA. However, most existing research primarily focuses on the application of LLMs in assisting code debugging. In contrast, our study explores the capabilities of LLMs for static code analysis. This work further promotes the refined application of LLMs in chip design automation. 

%and make a detailed comparison of its performance and cost differences with EDA tools.


%第四段
%在这篇文章中,我们提出了LintLLM,一个开源的代码质量检查工具,图1展示了它的框架。LintLLM通过逻辑树提示和缺陷溯源器增强了LLM检查Verilog代码缺陷的能力,解决了商业Lint工具存在冗余和误诊的弱点。为了评估LintLLM的有效性,我们创建了一个包含语法缺陷、位宽异常和无法综合等X种缺陷代码的benchmark套件。它包含90个被分为简单、中等和困难三个等级的Verilog设计。


In this paper, we develop LintLLM, an open-source Linting framework that leverages LLM to detect defects in Verilog code, which finds defects earlier than the compiler. It enhances the capability of LLMs via the \emph{Prompt of Logic-Tree} and \emph{Defect Tracker} we proposed, which address the limitations of traditional Linting tools that exhibit redundancy and false positives. To evaluate the effectiveness of LintLLM, we create a high-quality benchmark consisting of 90 Verilog designs, encompassing 11 types of common code defects.

%we create a high-quality benchmark consisting of 90 Verilog designs at various levels of complexity, encompassing 11 types of common code defects.

We evaluated the code defect detection capabilities of five LLMs and two traditional Linting tools on our benchmark. Experimental results show that o1-mini improves the correct rate by 18.89\% and reduces the false-positive rate by 15.56\% compared with the best-performing EDA tool. Moreover, LintLLM operates at less than one-tenth of the cost of the commercial EDA tool. This study demonstrates the potential of LLM as an effective Linting tool for hardware design and provides a cost-effective alternative for detecting code defects.

Our main contributions are as follows:
\begin{itemize}

	\item 
	We proposed a Linting framework named LintLLM that leverages LLMs to detect code defects. To the best of our knowledge, this is the first work employing LLM as a novel Linting tool in the hardware design domain.
	
	\item 
	We created an open-source benchmark to evaluate LLM's ability for detecting Verilog defects. It covers 11 defect types and includes 90 Verilog designs.
	
	\item 
	We introduced a template of Prompt Engineering named Prompt of Logic-Tree. It can translate complex algorithms into tree-structured prompts that are easier for LLMs to understand.
	
	\item 
	We proposed the Defect Tracker. It can locate the main defect causing multiple secondary defects in the code, reducing the redundancy of defect reports.
	
\end{itemize}

%The rest of the paper is organized as follows. In Section II, the background of this work is explained. In Section III, our proposed methods of LintLLM is explained. The experimental setup and research questions are explained in Section IV. Experimental results and conclusions are shown in Section V and Section VI.

%----------------------------------------------------------------------------------------
%\clearpage

\section{Background and Related Works}

\subsection{LLMs for Chip Design}

%芯片设计流程包括系统规格,架构设计,功能设计,逻辑综合,物理设计等多个环节[iEDA]。目前LLM在芯片设计的应用主要包括RTL代码生成,EDA脚本生成,芯片验证和领域知识问答四个方面。在代码生成任务中,文章【ChipGPT: How far are we from natural language hardware design】【AutoChip: Automating HDL Generation Using LLM Feedback】采用提示工程的方式利用LLM生成verilog代码并进行语法和功能验证;文章【Benchmarking Large Language Models for Automated Verilog RTL Code Generation】【VeriGen: A Large Language Model for Verilog Code Generation】【BetterV: Controlled Verilog Generation with Discriminative Guidance】【CodeV: Empowering LLMs for Verilog Generation through Multi-Level Summarization】【AutoVCoder: A Systematic Framework for Automated Verilog Code Generation using LLMs】采用指令微调的方式创建专用于Verilog代码生成的LLM,提高了生成的代码的质量。文章【RTLCoder: Fully Open-Source and Efficient LLM-Assisted RTL Code Generation Technique】【Large Language Model for Verilog Generation with Golden Code Feedback】【VerilogCoder: Autonomous Verilog Coding Agents with Graph-based Planning and Abstract Syntax Tree (AST)-based Waveform Tracing Tool】则基于LLM创建的Agent能够根据EDA工具的反馈生成质量更好的代码。文章【Data is all you need: Finetuning LLMs for Chip Design via an Automated design-data augmentation framework】【ChatEDA: A Large Language Model Powered Autonomous Agent for EDA】利用LLM生成操作EDA工具的脚本,自动化芯片设计流程。为了评估LLM生成verilog代码的语法和功能正确性,文章【VerilogEval: Evaluating Large Language Models for Verilog Code Generation】【RTLLM: An Open-Source Benchmark for Design RTL Generation with Large Language Model】【MG-Verilog: Multi-grained Dataset Towards Enhanced LLM-assisted Verilog Generation】提供了开源的数据集。然而这些数据集无法评估LLM的缺陷检测能力。
%此外,文章【VerilogReader: LLM-Aided Hardware Test Generation】【Evaluating LLMs for Hardware Design and Test】【LLM-based Processor Verification: A Case Study for Neuromorphic Processor】【AutoBench: Automatic Testbench Generation and Evaluation Using LLMs for HDL Design】【LLM4DV: Using Large Language Models for Hardware Test Stimuli Generation】使用LLM生成测试激励,测试芯片的覆盖率。使用LLM作为一个辅助设计人员的全能助手,通过与设计人员的交互,能够完成芯片的初步设计。【Chip-Chat: Challenges and Opportunities in Conversational Hardware Design】【Towards LLM-Powered Verilog RTL Assistant: Self-Verification and Self-Correction】【Customized Retrieval Augmented Generation and Benchmarking for EDA Tool Documentation QA】【ChipNeMo: Domain-Adapted LLMs for Chip Design】。先前的工作通过使用LLM结合EDA工具,揭示了LLM在芯片设计中的强大能力。然而,与此不同的是,我们这项工作探索原生LLM的能力【The Dawn of AI-Native EDA: Promises and  Challenges of Large Circuit Models】,在不使用EDA工具辅助下,首次使用LLM作为Linting工具完成verilog代码缺陷检测任务。


The chip design process includes system specification, architectural design, functional design, logic synthesis, and physical design \cite{conf/aspdac/LiHT0ZWLQLLSCBZ24}. Recent advancements highlight the application of LLMs in chip design across four primary directions, RTL code generation, EDA script generation, chip verification, and domain-specific knowledge Q\&A. In RTL code generation, researchers have employed LLMs to generate Verilog code through prompt engineering techniques \cite{journals/corr/abs-2305-14019,journals/corr/abs-2311-04887}. Other studies have focused on instruction fine-tuning to adapt LLMs specifically for Verilog code generation, significantly improving the quality and accuracy of generated code \cite{journals/todaes/ThakurAPTDKG24,conf/icml/PeiZYH024,journals/corr/abs-2407-10424,journals/corr/abs-2407-18333}. Additionally, research has demonstrated the use of LLM-based agents that iteratively improve code quality by integrating feedback from EDA tools \cite{journals/corr/abs-2312-08617,journals/corr/abs-2408-08927}. In script generation, LLMs have been utilized to automatically create scripts for EDA tools operations and simplify workflows \cite{conf/dac/ChangWY0JZCLYZZ24,journals/tcad/WuHZYZZY24}. To evaluate the syntax and function correctness of generated codes, researchers have developed open-source datasets \cite{conf/aspdac/LuLZX24,journals/corr/abs-2407-01910}. However, these datasets currently lack mechanisms to evaluate defect detection capabilities. For chip verification, LLMs have also been employed to generate test stimuli and analyze coverage, demonstrating their utility in comprehensive testing tasks \cite{journals/corr/abs-2406-04373,journals/corr/abs-2405-02326,conf/date/XiaoDYCWZDWTX24,journals/corr/abs-2310-04535}. Moreover, researchers have explored LLM as an interactive design assistant to assist preliminary design stages through iterative dialogue with designers \cite{conf/mlcad/BlockloveGKP23,journals/corr/abs-2311-00176}. 

Overall, these studies underscore the robust capabilities of LLMs in supporting EDA tools across various chip design tasks. In contrast to prior studies, our research investigates the potential of LLM-native approaches \cite{chen2024large}, employing LLM as an open-source Linting tool to autonomously detect Verilog code defects without relying on EDA tools supporting for the first time.


\subsection{Static Code Analysis}

%静态代码分析是一种在不实际运行代码的情况下,通过预定义的模式或规则检测代码中的语法、编码风格和设计不一致。[How developers engage with static analysis tools in different contexts.]。Linting工具能够辅助验证人员提高缺陷检测效率,其工作原理是对代码的语法、结构、和潜在缺陷进行解析,并通过与既定的规则集匹配,来识别不符合这些规则的潜在问题[Taxonomy of Static Code Analysis Tools]。在硬件设计领域,传统的工具如synopsys的SpyGlass,Cadence的JasperGold和Siemens的Questa Lint,他们通过自定义的编码规则能够检测verilog中潜在的缺陷。然而,这些工具的一个常见问题是它们倾向于产生假阳性报告,报告的代码问题不是实际问题的情况[Why don’t software developers use static analysis tools to find bugs? ]。最近,软件工程的文章【Code Linting Using Language Models】,使用LLM对Java代码进行缺陷分类获得了较好的结果。这项工作启发了我们的思考,能够使用LLM作为Linting工具针对硬件描述语言的缺陷检测,并降低误报的产生。

Static code analysis is a method of detecting syntax, coding style, and design inconsistencies in code through predefined patterns or rules without actually running the code \cite{journals/ese/VassalloPPPGZ20}. Linting tools can assist verification personnel in improving defect detection efficiency. They work by analyzing the syntax, structure, and potential defects of the code and identifying potential problems that do not comply with these rules by matching them with the established rule set \cite{novak2010taxonomy}. In the field of hardware design, traditional EDA tools such as Synopsys's SpyGlass \cite{SpyGlass}, Cadence's JasperGold \cite{JasperGold}, and Siemens's Questa Lint \cite{QuestaLint} can detect potential defects in Verilog through customized coding rules. However, a common problem with these tools is that they tend to generate false positive reports, reporting code issues that actually are not defects \cite{conf/icse/JohnsonSMB13}.  

Recently, Holden et al. \cite{journals/corr/abs-2406-19508} uses LLMs to classify Java code defects and achieved good results in the field of software engineering. This work inspired us to think about using LLM as a Linting tool for defect detection of Verilog code and reducing the generation of false positives.

%----------------------------------------------------------------------------------------

\begin{figure}[t]
	\centering
	\includegraphics[width=0.45\textwidth]{picture/overview.pdf} % main2
	\caption{Workflow of Traditional Linting and LLM-based Linting}
	\label{pic1}
	\vspace{-2mm}
\end{figure}

\section{Methods}

\subsection{Overview}

The traditional Linting flow is shown in Fig. \ref{pic1}(a). To explore the application of LLMs in code defect detection, we proposed the LLM-based Linting flow named LintLLM, as shown in Fig. \ref{pic1}(b). \ding{172}The DUT is input into LLM. \ding{173}The Prompt is used to guide LLM in detecting defects. \ding{174}Many defects are passed to \ding{175}the Tracker for tracking. \ding{176}The main defect is located through the feedback loop. \ding{177}Outputting the detection report. The following methods are used to enhance LLM.

%传统的linting流程如图1a所示。为了探索将LLM应用于代码缺陷检测,我们提出了基于llm的缺陷检测流程,如图1b所示。首先将DUT通过提示工程的方式输入LLM进行检测;当检测到多个缺陷后,通过追踪这些缺陷的主缺陷从而降低误报率,最后输出缺陷报告。



%\subsection{Workflow of LintLLM}
%LintLLM进行代码缺陷检查包括五个步骤,Fig. \ref{pic1} depicts its workflow.。1 将待测设计(DUT)和prompt输入给LLM;2 LLM执行缺陷检查并输出结果;3缺陷追踪器分析多个缺陷中的主要缺陷;4将分析结果和prompt再次输入LLM;5 生成代码缺陷报告。
%The process of detecting code defects utilizing LintLLM includes five steps. Fig. \ref{pic1} depicts its workflow. \ding{182} 
%The Design Under Test (DUT) and Prompt are input to the LLM. \ding{183} LLM performs defect detection and outputs the results. \ding{184} Defect Tracker analyzes the main defect in the code that cause multiple secondary defects. \ding{185} The analysis results and prompt are input to LLM again. \ding{186} After the defect detection is completed, the defects report is output.


\subsection{Prompt of Logic-Tree \label{method1}}

%借鉴数据结构
%Clear logic and detailed prompts can be better understood by LLM. To achieve this, we propose the Prompt of Logic-Tree is shown in Fig. \ref{pic2}, by drawing idea from the concept of the Ordered Tree in data structures. This is a new prompt template that translates complex algorithms for problem-solving into a logically clear tree structure. It alleviates the symptom of LLM forgetting in long texts. This template consists of two parts: a root node specifying the role and task, and child nodes describing algorithms for solving the task.

%借鉴TOT
Clear logic and detailed prompts can be better understood by LLMs. To achieve this, we propose the Prompt of Logic-Tree by drawing ideas from the Tree of Thoughts (ToT) \cite{conf/nips/YaoYZS00N23}, as shown in Fig. \ref{pic2}. Different from ToT, this is a new prompt template that translates complex algorithms for problem-solving into a clear tree structure. This template consists of two parts: the root node specifying the role and task and child nodes describing detailed algorithms for solving the task.


\begin{figure}[htbp]
	\centering
	\includegraphics[width=0.49\textwidth]{picture/method1.pdf} % pic4.pdf
	\caption{Template for Prompt of Logic-Tree}
	\label{pic2}
\end{figure}

\textbf{Root Node.} The role and task of LLMs are described as the root node. By setting a specific role, LLMs can more accurately answer user questions or perform specific tasks. This role definition helps ensure that the LLM's output aligns with its role positioning, avoiding overly free or off-topic responses. LLMs treat the assigned task as an objective that must be followed throughout the task execution process.

\textbf{Child Nodes.} The characteristic of the Logic-Tree is that the each node of the tree is ordered from left to right and cannot be interchanged. Therefore, it is reasonable to describe the task-solving algorithms with rigorous logic as child nodes. The parent nodes represent the main steps of the solving algorithm, and the child nodes from left to right represent the sequential sub-steps under the main steps.



%%--------------------------------是否注释--------------------------------------------
%In the code defect detection task, the prompt based on the Prompt of Logic-Tree is shown in Fig. \ref{pic3}.
%%
%%\vspace{-2mm}
%\begin{figure}[htbp]
%	\begin{tcolorbox}[colback=white, colframe=white]
%		\begin{lstlisting}[style=mystyle,xleftmargin=-1.5em,xrightmargin=-1.5em,aboveskip=0em]
%(*@\textbf{\textcolor{mygreen}{System}}:@*)You are a Verilog code defect checker and are able to follow the defined rules.
%(*@\textbf{\textcolor{myred}{Step1}}:@*)Identify punctuation marks
%    (*@\textbf{\textcolor{myyellow}{(1)}}@*)Chinese punctuation cannot appear in the code.
%(*@\textbf{\textcolor{myred}{Step2}}:@*)Identify module
%    (*@\textbf{\textcolor{myyellow}{(1)}}@*)Module must be wrapped by 'module-endmodule'.
%(*@\textbf{\textcolor{myred}{Step3}}:@*)Identify statements
%    (*@\textbf{\textcolor{myyellow}{(1)}}@*)For statements in the module header, end with (','), and do not punctuate the last statement.
%    (*@\textbf{\textcolor{myyellow}{(2)}}@*)For statements outside the module header, end with (';').
%(*@\textbf{\textcolor{myred}{Step4}}:@*)Identify variables
%    (*@\textbf{\textcolor{myyellow}{(1)}}@*)For variables defined in the module header, the port type ('input', 'output', 'input') must be defined.
%        (*@\textbf{\textcolor{myblue}{a}}.@*)For 'input' type variables, they must be assigned to other variables.
%        (*@\textbf{\textcolor{myblue}{b}}.@*)For 'output' type variables, they need to be assigned values by other variables.
%        (*@\textbf{\textcolor{myblue}{......}}@*)
%(*@\textbf{\textcolor{myred}{......}}@*)
%(*@\textbf{\textcolor{myred}{Step11}}:@*)Identify operator
%  (*@\textbf{\textcolor{myyellow}{......}}@*)
%		\end{lstlisting}
%	\end{tcolorbox}
%	\vspace{-8mm}
%	\caption{Prompt for code defect detection task}
%	\label{pic3}
%\end{figure}

\subsection{Defect Tracker \label{method2}}

The difference from natural language is that Verilog code is highly logical, and the main defect in the code may cause multiple related secondary defects, as shown in Fig. \ref{pic4}. Due to the dependencies between these defects, Linting tools face challenges in detecting and locating the main defect. Traditional Linting tools can usually only detect line by line and cannot accurately locate the root cause of secondary defects.

\begin{figure}[hb]
\begin{tcolorbox}[colback=white, colframe=white]
\begin{lstlisting}[style=mystyle1,xleftmargin=-0.5em,xrightmargin=-1em,aboveskip=-1em,belowskip=-1.5em]
module complex_1(
    output reg [15:0] qo,
    input [15:0] din,
    input load
);
    reg [7:0] temp_reg; // main defect
	    // [7:0]-->[15:0]
    always @(posedge load) begin
        temp_reg <= din; // secondary defect 1 
        qo <= temp_reg;  // secondary defect 2
    end
endmodule
\end{lstlisting}
\end{tcolorbox}
\caption{The main defect of incorrect bit-width of `temp\_reg' in Line 6 simultaneously causes secondary defects in Line 9 and Line 10.}
\label{pic4}
\end{figure}

To address this scenario, we introduce the Defect Tracker. It guides LLMs to identify the main defect in complex code defects. The specific algorithm is shown in Alg. \ref{defect_tracker}, which mainly includes three steps. 

(\rmnum{1}) Defining the initial defect set. Using LLM to detect all defects in the code, and record the location, type, and dependencies of each defect.

(\rmnum{2}) Gradually fix and re-detect. Defect Tracker will fix each defect one by one, and then re-run LLM to detect the number of remaining defects.

(\rmnum{3}) Judging the minimum number of defects. When the number of remaining defects reaches the minimum value after repairing a certain defect, it indicates that the defect is the main cause of multiple secondary defects.

\begin{algorithm}
	\fontsize{7pt}{9.1pt}\selectfont
	\caption{Defect Tracker Algorithm}
	\label{defect_tracker}
	\begin{algorithmic}[1]
			\REQUIRE A set of detected defects $D = \{D_1, D_2, \ldots, D_m\}$, where $m$ is the number of defects.
			\ENSURE Identify the main defect $D_{{main}}$.
			
			\STATE Initialize a set for the number of remaining defects $R = \{R_1, R_2, \ldots, R_m\}$ after fixing $D$, where $m$ is the number of defects.
			
			\FOR{$i = 1$ to $m$}
			\STATE Fix defect $D_i$ while ignoring the other defects $D_j$ for $j \neq i$.
			\STATE Run the defect detection process, and record the number of remaining defect as $R_i$.
			\ENDFOR
			
			\STATE Find the minimum value in the set $R$, i.e., $R_k$.
			
			\STATE The defect $D_k$ corresponding to $R_k$ is identified as the main defect $D_{main}$.
			
			\RETURN $D_{main}$
		\end{algorithmic}
\end{algorithm}

\section{Benchmark of Static Code Analysis}

Our investigation reveals that previous benchmarks \cite{conf/aspdac/LuLZX24,journals/corr/abs-2405-06840} mainly focus on evaluating the correctness of syntax and function for Verilog code generated by LLMs. Therefore, we create a high-quality benchmark to evaluate LLM's ability for detecting Verilog defects through the following workflows.

\subsection{Correct Code Collection}
%首先我们从Github筛选了150个常见的基于Verilog的原始设计文件,这些文件是一个完整module-endmodule且不包含include关键字。然后,我们删除了所有了注释,目的是仅评估LLM对代码的理解和缺陷检测能力。为了确保这些代码的正确性,我们对其进行了语法和功能验证。最终得到了90个正确的设计文件。
Firstly, we screen 150 common Verilog files from Github. Each file is a complete `module-endmodule' block and does not contain the `include' keyword. Then, we delete all comments, aiming to evaluate LLM's understanding and defect detection capabilities only for code. To ensure the correctness of these codes, we verify their syntax and function. Finally, 90 correct design files are obtained.

\subsection{Code Defect Injection \label{method3}}
%为了生成高质量的缺陷代码,我们从GitHub收集了大量基于Verilog的原始设计文件。之后提出了基于变异的缺陷注入方法。该方法能够模拟硬件设计师在代码书写过程中无意引入的缺陷。这种缺陷注入方法包括以下7种变异类型。

We define multiple rules through mutation-based defect injection technique \cite{conf/mbmv/Ahmadi-PourHD21}, as shown in Table \ref{rule}. This method imitates the defect that hardware designers unintentionally introduce during the coding process.

\begin{table}[htbp]
	\caption{Rules for code defect injection }
	\vspace{-1mm}
	\centering
	\setlength{\tabcolsep}{2.5pt} % 调整表格列的间距
	\fontsize{7pt}{1pt}\selectfont
	%	\scriptsize 
	\begin{tabular}{c|>{\raggedright\arraybackslash}p{4.4cm}|>{\raggedright\arraybackslash}p{3.4cm}} % 5列,左对齐
		\toprule 
		\textbf{Type}\vspace{-0.2pt}   & \multicolumn{1}{c|}{\textbf{Description}}    &  \multicolumn{1}{c}{\textbf{Example}}\\
		\midrule
		1         & replace the reserved keywords           & swap (else if) with (elif)   \\  \rule{0pt}{8pt}
		2         & swap blocking and non-blocking operators    & swap ($=$)  with ($<=$)   \\ \rule{0pt}{8pt}
		3         & swap assignment and relational operators           & swap ($=$) with ($==$)  \\ \rule{0pt}{8pt}
		4         & swap port type of signals                 & swap (input) with (output)  \\ \rule{0pt}{8pt}
		5         & swap data type of signals     & swap (reg) with (wire)  \\  \rule{0pt}{8pt}
		6         & change the bit-width of signals           & swap ([7:0]) with ([15:0])   \\  \rule{0pt}{8pt}
		7         & swap signal edges in the sensitivity list      & swap (posedge) with (negedge) \\ \rule{0pt}{8pt}
		8         & swap logical and bitwise operators        & swap (\&) with (\&\&) \\ \rule{0pt}{8pt}
		9         & change connection symbol in sensitivity list & swap (or) with (\textbar, \textbar\textbar) \\ \rule{0pt}{8pt}
		10        & use undefined or undeclared signal   & undefined signal \\  \rule{0pt}{8pt}
		11        & inject statements that create race or hazard & write-write race for signal out \\ \rule{0pt}{8pt}
		12        & inject statements that cannot be synthesized & unknown or high-impedance state \\ \rule{0pt}{8pt}
		13        & inject defects caused by module instances & signal port floating
		\\ 
		\bottomrule
	\end{tabular}
	\label{rule}
\end{table}

\subsection{Benchmark Construction}

We create an anonymously released benchmark, which contains 11 categories of defects and 90 design files. It includes defect codes and the corresponding line number, as shown in TABLE \ref{benchmark}. Due to the length limit of the paper, the details of the benchmark can be accessed through this link: https://github.com/fangzhigang32/Static-Verilog-Analysis

\begin{table}[htbp]
	\caption{Benchmark for Verilog code defects}
	\centering
	\setlength{\tabcolsep}{16.4pt} % 调整表格列的间距
	\fontsize{7pt}{1pt}\selectfont
	\vspace{-1mm}
	\begin{tabular}{l|l|c} 
		\toprule 
		\textbf{Difficulty Levels}  \vspace{-0.2pt}                                               & \textbf{Defect Categories} \vspace{-0.2pt}            & \multicolumn{1}{c}{\textbf{Count}}  \vspace{-0.2pt}\\ 
		\midrule 
		& Syntax Structure            & 6                       \\ \rule{0pt}{8pt}
		& Signal Usage                & 13                       \\ \rule{0pt}{8pt}
		Simple (30)                                                      & Sensitivity List            & 16                      \\ \rule{0pt}{8pt}
		& Reserved words              & 3                       \\ \rule{0pt}{8pt}
		& Race or Hazard              & 11                       \\ \rule{0pt}{8pt}
		Medium (30)                                                      & Port Type                   & 4                       \\ \rule{0pt}{8pt}
		& Operators                   & 7                        \\ \rule{0pt}{8pt}
		& Module Instances            & 8                        \\ \rule{0pt}{8pt}
		Complex (30)                                                     & Logic Synthesis             & 6                      \\ \rule{0pt}{8pt}
		& Combinational or Sequential & 2                         \\ \rule{0pt}{8pt}
		& Bit width Usage             & 14                        \\
		\midrule  
		\multicolumn{2}{l}{Total}                                                                  & 90               \\ 
		\bottomrule                                         
	\end{tabular}
	\label{benchmark}
\end{table}

\section{Experiment}
\subsection{LLMs Selection} %模型选择;温度设置;评估指标
%为了适应代码缺陷检查的实际运行环境,我们将超参数temperature设置为0,除了o1-mini,因为他不支持该参数。保证LLM的输出结果具有确定性。
We select recent LLMs to facilitate academic research and industrial applications, and TABLE \ref{llm_summary} describes the details of the models. Considering the actual operating environment of code defect detection, the hyper-parameter \textbf{Temperature} is set to 0 to ensure the determinism of LLMs outputs. Note that o1-mini does not support this property.

\begin{table}[htbp]
	\caption{Summary of LLMs Used in Current Study}
	\vspace{-1mm}
	\centering
	\setlength{\tabcolsep}{9.8pt} % 调整表格列的间距
	\fontsize{7pt}{8pt}\selectfont
%	\scriptsize 
	\begin{tabular}{l|c|c|cc} % 5列,左对齐
		\toprule 
		\textbf{Model Name}\vspace{-0.2pt}    & \textbf{Parameters}\vspace{-0.2pt}   & \textbf{Context Length}\vspace{-0.2pt} & \textbf{Release Date}\vspace{-0.2pt} \\
		\midrule \rule{0pt}{8pt}
		GPT-4         & Unintroduced & 8K             & Mar, 2023   \\  \rule{0pt}{8pt}
		Llama-3.1     & 405B         & 128K           & Jul, 2024   \\  \rule{0pt}{8pt}
		GPT-4o        & Unintroduced & 128K           & May, 2024   \\  \rule{0pt}{8pt}
		DeepSeek V2.5 & 236B         & 128K           & Sept, 2024  \\  \rule{0pt}{8pt}
		o1-mini       & Unintroduced & 128K           & Sept, 2024  \\
		\bottomrule
	\end{tabular}
	\label{llm_summary}
\end{table}

The best performing \textbf{commercial EDA} and \textbf{Verilator} \cite{Verilator} are our baselines.

\subsection{Experimental Setting}

To evaluate LLM's ability for detecting Verilog defects, we set up four groups of experiments.

(\rmnum{1}) \textbf{Original}. Directly input DUT into LLM for exploring LLM's basic capability, as shown in Fig. \ref{pic1}(b) (steps \ding{172}$\rightarrow$\ding{177}).

(\rmnum{2}) \textbf{+Prompt of Logic-Tree}. Only Prompt of Logic-Tree is used to explore its impact on LLM in improving the correct rate of defects, as shown in Fig. \ref{pic1}(b) (steps \ding{172}$\rightarrow$\ding{173}$\rightarrow$\ding{177}).

(\rmnum{3}) \textbf{+Defect Tracker}. Only Defect Tracker is used to explore its impact on LLM in reducing the false-positive rate of defects, as shown in Fig. \ref{pic1}(b) (steps \ding{172}$\rightarrow$\ding{174}$\rightarrow$\ding{175}$\rightarrow$\ding{176}$\rightarrow$\ding{177}).

(\rmnum{4}) \textbf{LintLLM}. Prompt of Logic-Tree and Defect Tracker are simultaneously used to explore the LLM's best performance, as shown in Fig. \ref{pic1}(b) (steps \ding{172}$\rightarrow$\ding{173}$\rightarrow$\ding{174}$\rightarrow$\ding{175}$\rightarrow$\ding{176}$\rightarrow$\ding{177}).


\subsection{Evaluation Metrics}
%为了全面的评估LLM对代码缺陷检查的能力,我们定义了三个指标。(1)检测率,在全部DUT中正确检测到的缺陷比例;(2)误报率,在全部DUT中误检测的缺陷比例;(3)检测时间,检测全部缺陷消耗的时间。
%当检测结果报告的缺陷行与注入缺陷的行数一致时,被认为正确检测。当没有缺陷的代码行被报告为缺陷时,被认为误报。
There are two cases for the evaluation of experimental results. (\rmnum{1}) When the defective line in the defect report is consistent with the line number where the defect is injected, it is considered a correct detection.  (\rmnum{2}) When a code line without defect is reported as a defect, it is considered a false positive. To comprehensively evaluate the capability of code defect detection by LLMs, we defined two metrics. \textbf{Correct Rate (CR)}: the percentage of defects correctly detected in all DUTs. \textbf{False-Positive Rate (FR)}: the percentage of false positive defects in all DUTs. The CR is better when higher (\textcolor{blue}{$\nearrow$}), and FR is better when lower (\textcolor{red}{$\searrow$}).


\subsection{Research Questions} %研究问题
%本研究评估了LLM进行代码缺陷检测中的有效性,并检验了提出的方法的必要性。它包括四个主要的研究问题。
This paper evaluates the effectiveness of LLMs in code defect detection and examines the impact of our proposed methods. It includes an investigation structured around three specific Research Questions (RQs).

\begin{itemize}
%与 EDA 工具相比,LLM 在代码缺陷检测方面有什么不同
\item
\textbf{RQ1:} What are the differences in detection performance between LLMs and EDA tools?
%为什么我们的方法能提高LLM的代码缺陷检测性能
\item
\textbf{RQ2:} Why our methods can improve the detection performance of LLMs?
%成本优势
\item
\textbf{RQ3:} What are the advantages of LLMs in detection costs compared with commercial EDA tools?
\end{itemize}

\section{Result}

%通过在提出的benchmark进行代码缺陷检查实验,得到了完整的实验结果,如表1所示。结果表明参与评测的5种LLM在我们方法的增强下,综合性能均超过了EDA工具。尤其是o1-mini达到了xxx,在LLM中表现最佳。

The experiment was conducted on the proposed benchmark, with the complete results presented in TABLE \ref{main_result}. The results demonstrate that the correct rate of LLMs significantly outperforms EDA tools, especially when enhanced with Prompt of Logic-Tree and Defect Tracker. Among the evaluated LLMs, o1-mini achieves the highest correct rate of 83.33\% and the lowest false-positive rate of 12.22\%. Compared to the best performing commercial EDA tool, o1-mini's correct rate increased by 18.89\% and false-positive rate decreased by 15.56\%. When compared to Verilator, its correct rate increased by 21.11\% and false-positive rate decreased by 20.00\%.

\begin{table}[htbp]
	\setlength{\tabcolsep}{0.8pt} % 调整表格列的间距
	\caption{Main Results - The performance of code defect detection tasks using EDA tools and LLMs. \ding{172} is Original. \ding{173} is + Prompt of Logic-Tree. \ding{174} is + Defect Tracker. \ding{175} is LintLLM. o1-mini +LintLLM achieves the best performance, with an correct rate of 83.33\% and a false-positive rate of 12.22\%. }
	\label{main_result}
	\fontsize{7pt}{8pt}\selectfont
	\vspace{-1mm}
	\begin{tabular}{l|cc|cc|cc|cc}
		\toprule
		\multirow{2}{*}{\textbf{ Tools}} &
		\multicolumn{2}{c|}{\ding{172}} &
		\multicolumn{2}{c|}{\ding{173}} &
		\multicolumn{2}{c|}{\ding{174}} &
		\multicolumn{2}{c}{\ding{175}}\\ \cline{2-3} \cline{4-5} \cline{6-7} \cline{8-9} \rule{0pt}{8pt} 
		& \textbf{CR} & \textbf{FR} & \textbf{CR} & \textbf{FR} & \textbf{CR} & \textbf{FR} & \textbf{CR} & \textbf{FR} \\ 
		\midrule \rule{0pt}{8pt}
		Commercial EDA          & \cellcolor{green!27}64.44\%      & \cellcolor{red!35}27.78\%             & \cellcolor{green!15}--            & \cellcolor{red!15}--                   & \cellcolor{green!15}--            & \cellcolor{red!15}--                   & \cellcolor{green!15}--               & \cellcolor{red!15}--                     \\ \rule{0pt}{8pt}
		Verilator                & \cellcolor{green!21}62.22\%      & \cellcolor{red!30}32.22\%    & \cellcolor{green!15}--            & \cellcolor{red!15}--                   & \cellcolor{green!15}--            & \cellcolor{red!15}--                   & \cellcolor{green!15}--               & \cellcolor{red!15}--                     \\ \rule{0pt}{8pt}
		Llama-3.1          & \cellcolor{green!24}63.33\%      & \cellcolor{red!15}61.11\%             & \cellcolor{green!33}66.67\%      & \cellcolor{red!17.5}60.00\%             & \cellcolor{green!33}66.67\%      & \cellcolor{red!30}32.22\%             & \cellcolor{green!39}68.89\%         & \cellcolor{red!32.5}31.11\%               \\ \rule{0pt}{8pt}
		DeepSeek V2.5      & \cellcolor{green!24}63.33\%      & \cellcolor{red!50}20.00\%             & \cellcolor{green!48}\textbf{75.56\%}      & \cellcolor{red!22.5}47.78\%             & \cellcolor{green!51}78.89\%      & \cellcolor{red!47.5}21.11\%             & \cellcolor{green!57}81.11\%         & \cellcolor{red!52.5}18.89\%               \\ \rule{0pt}{8pt}
		GPT-4              & \cellcolor{green!15}47.78\%      & \cellcolor{red!25}36.67\%             & \cellcolor{green!18}48.89\%      & \cellcolor{red!42.5}24.44\%             & \cellcolor{green!21}62.22\%      & \cellcolor{red!52.5}18.89\%             & \cellcolor{green!33}66.67\%         & \cellcolor{red!27.5}33.33\%               \\ \rule{0pt}{8pt}
		GPT-4o             & \cellcolor{green!30}65.56\%      & \cellcolor{red!45}22.22\%             & \cellcolor{green!30}65.56\%      & \cellcolor{red!50}\textbf{20.00\%}    & \cellcolor{green!36}67.78\%      & \cellcolor{red!40}25.56\%             & \cellcolor{green!45}73.33\%         & \cellcolor{red!37.5}26.67\%               \\ 	\rule{0pt}{8pt}		
		o1-mini            & \cellcolor{green!36}\textbf{67.78\%}      & \cellcolor{red!60}\textbf{11.11\%}             & \cellcolor{green!42}70.00\%      & \cellcolor{red!20}55.56\%             & \cellcolor{green!54}\textbf{80.00\%}      & \cellcolor{red!55}\textbf{13.33\%}             & \cellcolor{green!60}\textbf{83.33\%}         & \cellcolor{red!57.5}\textbf{12.22\%}               \\
		\bottomrule
	\end{tabular}			
	\footnotemark \textit{Note:} The highest contrast colors represent the best rate.
\end{table}

Overall, o1-mini shows the best performance, combining a high correct rate with a low false-positive rate, demonstrating the effectiveness of LLMs in code defect detection tasks. 


%\begin{table*}[htbp]
%	\setlength{\tabcolsep}{15pt} % 调整表格列的间距
%	\fontsize{7pt}{8pt}\selectfont
%	\caption{Main Results - The performance of code defect detection tasks using EDA tools and LLMs}
%	\label{main_result}
%	\vspace{-1mm}
%	\begin{tabular}{l|cc|cc|cc|cc}
%		\toprule
%		\multirow{2}{*}{\textbf{EDA Tools / LLM}} &
%		\multicolumn{2}{c|}{\textbf{Original}} &
%		\multicolumn{2}{c|}{\textbf{+ Prompt of Logic-Tree}} &
%		\multicolumn{2}{c|}{\textbf{+ Defect Tracker}} &
%		\multicolumn{2}{c}{\textbf{+ LintLLM}}\\ \cline{2-3} \cline{4-5} \cline{6-7} \cline{8-9} \rule{0pt}{8pt} &
%		\textbf{\begin{tabular}[c]{@{}c@{}}CR\\ \end{tabular}} &
%		\textbf{\begin{tabular}[c]{@{}c@{}}FR\\ \end{tabular}} &
%		\textbf{\begin{tabular}[c]{@{}c@{}}CR\\ \end{tabular}} &
%		\textbf{\begin{tabular}[c]{@{}c@{}}FR\\ \end{tabular}} &
%		\textbf{\begin{tabular}[c]{@{}c@{}}CR\\ \end{tabular}} &
%		\textbf{\begin{tabular}[c]{@{}c@{}}FR\\ \end{tabular}} &
%		\textbf{\begin{tabular}[c]{@{}c@{}}CR\\ \end{tabular}} &
%		\textbf{\begin{tabular}[c]{@{}c@{}}FR\\ \end{tabular}} \\ \midrule 
%		EDA: SpyGlass           & \cellcolor{green!24}51.11\%      & \cellcolor{red!22.5}54.44\%             & \cellcolor{green!15}--            & \cellcolor{red!15}--                   & \cellcolor{green!15}--            & \cellcolor{red!15}--                   & \cellcolor{green!15}--               & \cellcolor{red!15}--                     \\ 
%		EDA: VCS                & \cellcolor{green!15}42.22\%      & \cellcolor{red!60}\textbf{2.22\%}    & \cellcolor{green!15}--            & \cellcolor{red!15}--                   & \cellcolor{green!15}--            & \cellcolor{red!15}--                   & \cellcolor{green!15}--               & \cellcolor{red!15}--                     \\ 
%		LLM: Llama-3.1          & \cellcolor{green!30}63.33\%      & \cellcolor{red!15}61.11\%             & \cellcolor{green!36}66.67\%      & \cellcolor{red!17.5}60.00\%             & \cellcolor{green!36}66.67\%      & \cellcolor{red!35}32.22\%             & \cellcolor{green!39}68.89\%         & \cellcolor{red!32.5}31.11\%               \\ 
%		LLM: DeepSeek V2.5      & \cellcolor{green!30}63.33\%      & \cellcolor{red!47.5}20.00\%             & \cellcolor{green!48}\textbf{75.56\%}      & \cellcolor{red!25}47.78\%             & \cellcolor{green!51}78.89\%      & \cellcolor{red!45}21.11\%             & \cellcolor{green!57}81.11\%         & \cellcolor{red!42.5}18.89\%               \\ 
%		LLM: GPT-4              & \cellcolor{green!18}47.78\%      & \cellcolor{red!27.5}36.67\%             & \cellcolor{green!21}48.89\%      & \cellcolor{red!40}24.44\%             & \cellcolor{green!27}62.22\%      & \cellcolor{red!42.5}18.89\%             & \cellcolor{green!36}66.67\%         & \cellcolor{red!30}33.33\%               \\ 
%		LLM: GPT-4o             & \cellcolor{green!33}65.56\%      & \cellcolor{red!42.5}22.22\%             & \cellcolor{green!33}65.56\%      & \cellcolor{red!47.5}\textbf{20.00\%}    & \cellcolor{green!39}67.78\%      & \cellcolor{red!37.5}25.56\%             & \cellcolor{green!48}73.33\%         & \cellcolor{red!27.5}26.67\%               \\ 			
%		LLM: o1-mini            & \cellcolor{green!43}\textbf{67.78\%}      & \cellcolor{red!55}11.11\%             & \cellcolor{green!45}70.00\%      & \cellcolor{red!22.5}55.56\%             & \cellcolor{green!54}\textbf{80.00\%}      & \cellcolor{red!52.5}\textbf{13.33\%}             & \cellcolor{green!60}\textbf{83.33\%}         & \cellcolor{red!50}\textbf{12.22\%}               \\
%		\bottomrule
%	\end{tabular}	
%\end{table*}

%未配色
%\begin{table*}[htbp]
%	\setlength{\tabcolsep}{12.6pt} % 调整表格列的间距
%	\caption{Main Results - The performance of code defect detection tasks using EDA tools and LLMs}
%	\label{main_result}
%	\begin{tabular}{l|cc|cc|cc|cc}
%		\toprule
%		\multirow{2}{*}{\textbf{EDA Tools / LLM}} &
%		\multicolumn{2}{c|}{\textbf{Original}} &
%		\multicolumn{2}{c|}{\textbf{+ Prompt of Logic-Tree}} &
%		\multicolumn{2}{c|}{\textbf{+ Defect Tracker}} &
%		\multicolumn{2}{c}{\textbf{+ Both methods}}\\ \cline{2-3} \cline{4-5} \cline{6-7} \cline{8-9} \rule{0pt}{8pt}&
%		\textbf{\begin{tabular}[c]{@{}c@{}}CR\\ \end{tabular}} &
%		\textbf{\begin{tabular}[c]{@{}c@{}}FR\\ \end{tabular}} &
%		\textbf{\begin{tabular}[c]{@{}c@{}}CR\\ \end{tabular}} &
%		\textbf{\begin{tabular}[c]{@{}c@{}}FR\\ \end{tabular}} &
%		\textbf{\begin{tabular}[c]{@{}c@{}}CR\\ \end{tabular}} &
%		\textbf{\begin{tabular}[c]{@{}c@{}}FR\\ \end{tabular}} &
%		\textbf{\begin{tabular}[c]{@{}c@{}}CR\\ \end{tabular}} &
%		\textbf{\begin{tabular}[c]{@{}c@{}}FR\\ \end{tabular}} \\ \midrule \rule{0pt}{8pt}
%		EDA: SpyGlass           & 51.11\%      & 54.44\%             & -            & -                   & -            & -                   & -               & -                     \\ \rule{0pt}{8pt}
%		EDA: VCS                & 42.22\%      & \textbf{2.22\%}              & -            & -                   & -            & -                   & -               & -                     \\ \rule{0pt}{8pt}
%		LLM: Llama-3.1          & 63.33\%      & 61.11\%             & 66.67\%      & 60.00\%             & 66.67\%      & 32.22\%             & 68.89\%         & 31.11\%               \\ \rule{0pt}{8pt}
%		LLM: DeepSeek V2.5      & 63.33\%      & 20.00\%             & \textbf{75.56\%}      & 47.78\%             & 78.89\%      & 21.11\%             & 81.11\%         & 18.89\%               \\ \rule{0pt}{8pt}
%		LLM: GPT-4              & 47.78\%      & 36.67\%             & 48.89\%      & 24.44\%             & 62.22\%      & 18.89\%             & 66.67\%         & 33.33\%               \\ \rule{0pt}{8pt}
%		LLM: GPT-4o             & 65.56\%      & 22.22\%             & 65.56\%      & \textbf{20.00\%}             & 67.78\%      & 25.56\%             & 73.33\%         & 26.67\%               \\ \rule{0pt}{8pt}			
%		LLM: o1-mini            & \textbf{67.78\%}      & 11.11\%             & 70.00\%      & 55.56\%             & \textbf{80.00\%}      & \textbf{13.33\%}             & \textbf{83.33\%}         & \textbf{12.22\%}               \\
%		\bottomrule
%	\end{tabular}	
%	%		\vspace{-.15in}
%\end{table*}


\begin{table*}[htbp]
	\caption{Detailed Results - Defect detection reports using EDA tools and LLMs on the benchmark. \ding{182} is Commercial EDA. \ding{183} is Verilator. \ding{184} is Llama-3.1 +LintLLM. \ding{185} is DeepSeek V2.5 +LintLLM. \ding{186} is GPT-4 +LintLLM. \ding{187} is GPT-4o +LintLLM. \ding{188} is o1-mini +LintLLM. \textbf{C} indicates whether the defect is detected correctly, and \textbf{F} indicates the number of false positives (0 as `-'). It shows that LLMs perform better than EDA tools.}  
	\label{Detail_Results}
	\centering
	\setlength{\tabcolsep}{2.42pt} % 调整表格列的间距
	\fontsize{7pt}{1pt}\selectfont
	\vspace{-1mm}
	%\setlength{\arrayrulewidth}{0.4mm} % 控制表格线的粗细
	%\renewcommand{\arraystretch}{1.5} % Default value: 1
	\begin{tabular}{c|cccccccccccccc|c|cccccccccccccc|c|cccccccccccccc} 
		%	\rule{0pt}{8pt}
		\toprule
		\multirow{2}{*}{\textbf{Simple}} &
		\multicolumn{2}{c}{\ding{182}} & \multicolumn{2}{c}{\ding{183}} & \multicolumn{2}{c}{\ding{184}} & \multicolumn{2}{c}{\ding{185}} & \multicolumn{2}{c}{\ding{186}} & \multicolumn{2}{c}{\ding{187}} & \multicolumn{2}{c|}{\ding{188}} & 
		\multirow{2}{*}{\textbf{Medium}} &
		\multicolumn{2}{c}{\ding{182}} & \multicolumn{2}{c}{\ding{183}} & \multicolumn{2}{c}{\ding{184}} & \multicolumn{2}{c}{\ding{185}} & \multicolumn{2}{c}{\ding{186}} & \multicolumn{2}{c}{\ding{187}} & \multicolumn{2}{c|}{\ding{188}}  & 
		\multirow{2}{*}{\textbf{Complex}} &
		\multicolumn{2}{c}{\ding{182}} & \multicolumn{2}{c}{\ding{183}} & \multicolumn{2}{c}{\ding{184}} & \multicolumn{2}{c}{\ding{185}} & \multicolumn{2}{c}{\ding{186}} & \multicolumn{2}{c}{\ding{187}} & \multicolumn{2}{c}{\ding{188}}  \\
		\cline{2-15} \cline{17-30} \cline{32-45}  \rule{0pt}{8pt}
		& \textbf{C} & \textbf{F} & \textbf{C} & \textbf{F} & \textbf{C} & \textbf{F} & \textbf{C} & \textbf{F} & \textbf{C} & \textbf{F} & \textbf{C} & \textbf{F} & \textbf{C} & \textbf{F} & &  \textbf{C} & \textbf{F} & \textbf{C} & \textbf{F} & \textbf{C} & \textbf{F} & \textbf{C} & \textbf{F} & \textbf{C} & \textbf{F} & \textbf{C} & \textbf{F} & \textbf{C} & \textbf{F} &  & \textbf{C} & \textbf{F} & \textbf{C} & \textbf{F} & \textbf{C} & \textbf{F} & \textbf{C} & \textbf{F} & \textbf{C} & \textbf{F} & \textbf{C} & \textbf{F} & \textbf{C} & \textbf{F}\\
		
		\midrule \rule{0pt}{8pt}
		s01	&	\cellcolor{green!45}{\checked}	&	-	&	\cellcolor{green!45}{\checked}	&	-	&	\cellcolor{green!45}{\checked}	&	-	&	\cellcolor{green!45}{\checked}	&	-	&	\cellcolor{green!45}{\checked}	&	-	&	\cellcolor{green!45}{\checked}	&	-	&	\cellcolor{green!45}{\checked}	&	-	&	m01	&	\cellcolor{green!45}{\checked}	&	1	&	\cellcolor{green!45}{\checked}	&	1	&	\cellcolor{green!45}{\checked}	&	-	&	\cellcolor{green!45}{\checked}	&	-	&	\cellcolor{green!45}{\checked}	&	-	&	\cellcolor{green!45}{\checked}	&	-	&	\cellcolor{green!45}{\checked}	&	-	&	c01	&	\cellcolor{red!45}{$\times$}	&	-	&	\cellcolor{green!45}{\checked}	&	1	&	\cellcolor{green!45}{\checked}	&	-	&	\cellcolor{red!45}{$\times$}	&	1	&	\cellcolor{green!45}{\checked}	&	-	&	\cellcolor{green!45}{\checked}	&	-	&	\cellcolor{green!45}{\checked}	&	-
		\\ \rule{0pt}{8pt} 																																																																																								
		s02	&	\cellcolor{green!45}{\checked}	&	-	&	\cellcolor{green!45}{\checked}	&	-	&	\cellcolor{green!45}{\checked}	&	-	&	\cellcolor{green!45}{\checked}	&	-	&	\cellcolor{green!45}{\checked}	&	-	&	\cellcolor{green!45}{\checked}	&	-	&	\cellcolor{green!45}{\checked}	&	-	&	m02	&	\cellcolor{red!45}{$\times$}	&	-	&	\cellcolor{green!45}{\checked}	&	-	&	\cellcolor{green!45}{\checked}	&	-	&	\cellcolor{green!45}{\checked}	&	-	&	\cellcolor{green!45}{\checked}	&	-	&	\cellcolor{green!45}{\checked}	&	-	&	\cellcolor{green!45}{\checked}	&	-	&	c02	&	\cellcolor{red!45}{$\times$}	&	3	&	\cellcolor{green!45}{\checked}	&	-	&	\cellcolor{green!45}{\checked}	&	-	&	\cellcolor{green!45}{\checked}	&	-	&	\cellcolor{red!45}{$\times$}	&	1	&	\cellcolor{red!45}{$\times$}	&	1	&	\cellcolor{green!45}{\checked}	&	-
		\\ \rule{0pt}{8pt} 																																																																																								
		s03	&	\cellcolor{green!45}{\checked}	&	-	&	\cellcolor{green!45}{\checked}	&	-	&	\cellcolor{green!45}{\checked}	&	-	&	\cellcolor{green!45}{\checked}	&	-	&	\cellcolor{green!45}{\checked}	&	-	&	\cellcolor{green!45}{\checked}	&	-	&	\cellcolor{green!45}{\checked}	&	-	&	m03	&	\cellcolor{green!45}{\checked}	&	-	&	\cellcolor{green!45}{\checked}	&	2	&	\cellcolor{green!45}{\checked}	&	-	&	\cellcolor{green!45}{\checked}	&	-	&	\cellcolor{green!45}{\checked}	&	-	&	\cellcolor{green!45}{\checked}	&	-	&	\cellcolor{green!45}{\checked}	&	-	&	c03	&	\cellcolor{green!45}{\checked}	&	-	&	\cellcolor{green!45}{\checked}	&	-	&	\cellcolor{red!45}{$\times$}	&	1	&	\cellcolor{red!45}{$\times$}	&	1	&	\cellcolor{green!45}{\checked}	&	-	&	\cellcolor{green!45}{\checked}	&	-	&	\cellcolor{green!45}{\checked}	&	-
		\\ \rule{0pt}{8pt} 																																																																																								
		s04	&	\cellcolor{green!45}{\checked}	&	-	&	\cellcolor{green!45}{\checked}	&	-	&	\cellcolor{green!45}{\checked}	&	-	&	\cellcolor{green!45}{\checked}	&	-	&	\cellcolor{green!45}{\checked}	&	-	&	\cellcolor{green!45}{\checked}	&	-	&	\cellcolor{green!45}{\checked}	&	-	&	m04	&	\cellcolor{green!45}{\checked}	&	-	&	\cellcolor{red!45}{$\times$}	&	-	&	\cellcolor{red!45}{$\times$}	&	1	&	\cellcolor{green!45}{\checked}	&	-	&	\cellcolor{red!45}{$\times$}	&	1	&	\cellcolor{green!45}{\checked}	&	-	&	\cellcolor{green!45}{\checked}	&	-	&	c04	&	\cellcolor{red!45}{$\times$}	&	-	&	\cellcolor{red!45}{$\times$}	&	-	&	\cellcolor{green!45}{\checked}	&	-	&	\cellcolor{red!45}{$\times$}	&	1	&	\cellcolor{green!45}{\checked}	&	-	&	\cellcolor{green!45}{\checked}	&	-	&	\cellcolor{green!45}{\checked}	&	-
		\\ \rule{0pt}{8pt} 																																																																																								
		s05	&	\cellcolor{green!45}{\checked}	&	-	&	\cellcolor{green!45}{\checked}	&	1	&	\cellcolor{green!45}{\checked}	&	-	&	\cellcolor{green!45}{\checked}	&	-	&	\cellcolor{green!45}{\checked}	&	-	&	\cellcolor{green!45}{\checked}	&	-	&	\cellcolor{green!45}{\checked}	&	-	&	m05	&	\cellcolor{green!45}{\checked}	&	-	&	\cellcolor{green!45}{\checked}	&	-	&	\cellcolor{red!45}{$\times$}	&	1	&	\cellcolor{green!45}{\checked}	&	-	&	\cellcolor{green!45}{\checked}	&	-	&	\cellcolor{green!45}{\checked}	&	-	&	\cellcolor{green!45}{\checked}	&	-	&	c05	&	\cellcolor{green!45}{\checked}	&	-	&	\cellcolor{green!45}{\checked}	&	2	&	\cellcolor{red!45}{$\times$}	&	1	&	\cellcolor{red!45}{$\times$}	&	1	&	\cellcolor{red!45}{$\times$}	&	1	&	\cellcolor{red!45}{$\times$}	&	1	&	\cellcolor{red!45}{$\times$}	&	1
		\\ \rule{0pt}{8pt} 																																																																																								
		s06	&	\cellcolor{green!45}{\checked}	&	-	&	\cellcolor{green!45}{\checked}	&	-	&	\cellcolor{green!45}{\checked}	&	-	&	\cellcolor{green!45}{\checked}	&	-	&	\cellcolor{green!45}{\checked}	&	-	&	\cellcolor{green!45}{\checked}	&	-	&	\cellcolor{green!45}{\checked}	&	-	&	m06	&	\cellcolor{red!45}{$\times$}	&	-	&	\cellcolor{green!45}{\checked}	&	-	&	\cellcolor{red!45}{$\times$}	&	1	&	\cellcolor{green!45}{\checked}	&	-	&	\cellcolor{green!45}{\checked}	&	-	&	\cellcolor{green!45}{\checked}	&	-	&	\cellcolor{green!45}{\checked}	&	-	&	c06	&	\cellcolor{green!45}{\checked}	&	-	&	\cellcolor{green!45}{\checked}	&	-	&	\cellcolor{red!45}{$\times$}	&	1	&	\cellcolor{red!45}{$\times$}	&	1	&	\cellcolor{green!45}{\checked}	&	-	&	\cellcolor{green!45}{\checked}	&	-	&	\cellcolor{green!45}{\checked}	&	-
		\\ \rule{0pt}{8pt} 																																																																																								
		s07	&	\cellcolor{green!45}{\checked}	&	-	&	\cellcolor{green!45}{\checked}	&	-	&	\cellcolor{green!45}{\checked}	&	-	&	\cellcolor{green!45}{\checked}	&	-	&	\cellcolor{green!45}{\checked}	&	-	&	\cellcolor{green!45}{\checked}	&	-	&	\cellcolor{green!45}{\checked}	&	-	&	m07	&	\cellcolor{red!45}{$\times$}	&	-	&	\cellcolor{green!45}{\checked}	&	-	&	\cellcolor{red!45}{$\times$}	&	1	&	\cellcolor{green!45}{\checked}	&	-	&	\cellcolor{green!45}{\checked}	&	-	&	\cellcolor{green!45}{\checked}	&	-	&	\cellcolor{green!45}{\checked}	&	-	&	c07	&	\cellcolor{red!45}{$\times$}	&	-	&	\cellcolor{red!45}{$\times$}	&	-	&	\cellcolor{red!45}{$\times$}	&	1	&	\cellcolor{green!45}{\checked}	&	-	&	\cellcolor{green!45}{\checked}	&	-	&	\cellcolor{green!45}{\checked}	&	-	&	\cellcolor{green!45}{\checked}	&	-
		\\ \rule{0pt}{8pt} 																																																																																								
		s08	&	\cellcolor{green!45}{\checked}	&	-	&	\cellcolor{green!45}{\checked}	&	-	&	\cellcolor{green!45}{\checked}	&	-	&	\cellcolor{green!45}{\checked}	&	-	&	\cellcolor{green!45}{\checked}	&	-	&	\cellcolor{green!45}{\checked}	&	-	&	\cellcolor{green!45}{\checked}	&	-	&	m08	&	\cellcolor{red!45}{$\times$}	&	-	&	\cellcolor{red!45}{$\times$}	&	1	&	\cellcolor{red!45}{$\times$}	&	1	&	\cellcolor{green!45}{\checked}	&	-	&	\cellcolor{green!45}{\checked}	&	-	&	\cellcolor{green!45}{\checked}	&	-	&	\cellcolor{green!45}{\checked}	&	-	&	c08	&	\cellcolor{green!45}{\checked}	&	-	&	\cellcolor{green!45}{\checked}	&	-	&	\cellcolor{red!45}{$\times$}	&	1	&	\cellcolor{red!45}{$\times$}	&	1	&	\cellcolor{red!45}{$\times$}	&	1	&	\cellcolor{red!45}{$\times$}	&	1	&	\cellcolor{green!45}{\checked}	&	-
		\\ \rule{0pt}{8pt} 																																																																																								
		s09	&	\cellcolor{green!45}{\checked}	&	-	&	\cellcolor{green!45}{\checked}	&	-	&	\cellcolor{green!45}{\checked}	&	-	&	\cellcolor{green!45}{\checked}	&	-	&	\cellcolor{green!45}{\checked}	&	-	&	\cellcolor{green!45}{\checked}	&	-	&	\cellcolor{green!45}{\checked}	&	-	&	m09	&	\cellcolor{green!45}{\checked}	&	-	&	\cellcolor{green!45}{\checked}	&	1	&	\cellcolor{red!45}{$\times$}	&	1	&	\cellcolor{green!45}{\checked}	&	-	&	\cellcolor{green!45}{\checked}	&	-	&	\cellcolor{green!45}{\checked}	&	-	&	\cellcolor{green!45}{\checked}	&	-	&	c09	&	\cellcolor{red!45}{$\times$}	&	1	&	\cellcolor{red!45}{$\times$}	&	-	&	\cellcolor{red!45}{$\times$}	&	1	&	\cellcolor{red!45}{$\times$}	&	1	&	\cellcolor{red!45}{$\times$}	&	1	&	\cellcolor{red!45}{$\times$}	&	1	&	\cellcolor{red!45}{$\times$}	&	1
		\\ \rule{0pt}{8pt} 																																																																																								
		s10	&	\cellcolor{green!45}{\checked}	&	-	&	\cellcolor{green!45}{\checked}	&	-	&	\cellcolor{green!45}{\checked}	&	-	&	\cellcolor{green!45}{\checked}	&	-	&	\cellcolor{green!45}{\checked}	&	-	&	\cellcolor{green!45}{\checked}	&	-	&	\cellcolor{green!45}{\checked}	&	-	&	m10	&	\cellcolor{green!45}{\checked}	&	-	&	\cellcolor{green!45}{\checked}	&	1	&	\cellcolor{red!45}{$\times$}	&	1	&	\cellcolor{green!45}{\checked}	&	-	&	\cellcolor{green!45}{\checked}	&	-	&	\cellcolor{green!45}{\checked}	&	-	&	\cellcolor{green!45}{\checked}	&	-	&	c10	&	\cellcolor{red!45}{$\times$}	&	-	&	\cellcolor{red!45}{$\times$}	&	1	&	\cellcolor{green!45}{\checked}	&	-	&	\cellcolor{green!45}{\checked}	&	-	&	\cellcolor{green!45}{\checked}	&	-	&	\cellcolor{red!45}{$\times$}	&	1	&	\cellcolor{red!45}{$\times$}	&	1
		\\ \rule{0pt}{8pt} 																																																																																								
		s11	&	\cellcolor{green!45}{\checked}	&	-	&	\cellcolor{green!45}{\checked}	&	-	&	\cellcolor{green!45}{\checked}	&	-	&	\cellcolor{red!45}{$\times$}	&	1	&	\cellcolor{red!45}{$\times$}	&	1	&	\cellcolor{green!45}{\checked}	&	-	&	\cellcolor{green!45}{\checked}	&	-	&	m11	&	\cellcolor{green!45}{\checked}	&	-	&	\cellcolor{red!45}{$\times$}	&	-	&	\cellcolor{green!45}{\checked}	&	-	&	\cellcolor{green!45}{\checked}	&	-	&	\cellcolor{red!45}{$\times$}	&	1	&	\cellcolor{red!45}{$\times$}	&	1	&	\cellcolor{green!45}{\checked}	&	-	&	c11	&	\cellcolor{red!45}{$\times$}	&	-	&	\cellcolor{red!45}{$\times$}	&	-	&	\cellcolor{green!45}{\checked}	&	-	&	\cellcolor{green!45}{\checked}	&	-	&	\cellcolor{green!45}{\checked}	&	-	&	\cellcolor{green!45}{\checked}	&	-	&	\cellcolor{green!45}{\checked}	&	-
		\\ \rule{0pt}{8pt} 																																																																																								
		s12	&	\cellcolor{green!45}{\checked}	&	-	&	\cellcolor{green!45}{\checked}	&	-	&	\cellcolor{green!45}{\checked}	&	-	&	\cellcolor{green!45}{\checked}	&	-	&	\cellcolor{green!45}{\checked}	&	-	&	\cellcolor{green!45}{\checked}	&	-	&	\cellcolor{green!45}{\checked}	&	-	&	m12	&	\cellcolor{red!45}{$\times$}	&	-	&	\cellcolor{green!45}{\checked}	&	-	&	\cellcolor{red!45}{$\times$}	&	1	&	\cellcolor{green!45}{\checked}	&	-	&	\cellcolor{green!45}{\checked}	&	-	&	\cellcolor{green!45}{\checked}	&	-	&	\cellcolor{green!45}{\checked}	&	-	&	c12	&	\cellcolor{red!45}{$\times$}	&	-	&	\cellcolor{red!45}{$\times$}	&	-	&	\cellcolor{green!45}{\checked}	&	-	&	\cellcolor{green!45}{\checked}	&	-	&	\cellcolor{green!45}{\checked}	&	-	&	\cellcolor{red!45}{$\times$}	&	1	&	\cellcolor{green!45}{\checked}	&	-
		\\ \rule{0pt}{8pt} 																																																																																								
		s13	&	\cellcolor{green!45}{\checked}	&	-	&	\cellcolor{green!45}{\checked}	&	-	&	\cellcolor{green!45}{\checked}	&	-	&	\cellcolor{green!45}{\checked}	&	-	&	\cellcolor{green!45}{\checked}	&	-	&	\cellcolor{green!45}{\checked}	&	-	&	\cellcolor{green!45}{\checked}	&	-	&	m13	&	\cellcolor{green!45}{\checked}	&	-	&	\cellcolor{green!45}{\checked}	&	-	&	\cellcolor{green!45}{\checked}	&	-	&	\cellcolor{green!45}{\checked}	&	-	&	\cellcolor{green!45}{\checked}	&	-	&	\cellcolor{green!45}{\checked}	&	-	&	\cellcolor{green!45}{\checked}	&	-	&	c13	&	\cellcolor{red!45}{$\times$}	&	-	&	\cellcolor{red!45}{$\times$}	&	-	&	\cellcolor{red!45}{$\times$}	&	1	&	\cellcolor{red!45}{$\times$}	&	1	&	\cellcolor{red!45}{$\times$}	&	-	&	\cellcolor{red!45}{$\times$}	&	1	&	\cellcolor{red!45}{$\times$}	&	1
		\\ \rule{0pt}{8pt} 																																																																																								
		s14	&	\cellcolor{green!45}{\checked}	&	-	&	\cellcolor{green!45}{\checked}	&	-	&	\cellcolor{green!45}{\checked}	&	-	&	\cellcolor{green!45}{\checked}	&	-	&	\cellcolor{green!45}{\checked}	&	-	&	\cellcolor{green!45}{\checked}	&	-	&	\cellcolor{green!45}{\checked}	&	-	&	m14	&	\cellcolor{red!45}{$\times$}	&	-	&	\cellcolor{green!45}{\checked}	&	-	&	\cellcolor{green!45}{\checked}	&	-	&	\cellcolor{green!45}{\checked}	&	-	&	\cellcolor{red!45}{$\times$}	&	1	&	\cellcolor{red!45}{$\times$}	&	1	&	\cellcolor{green!45}{\checked}	&	-	&	c14	&	\cellcolor{green!45}{\checked}	&	-	&	\cellcolor{green!45}{\checked}	&	2	&	\cellcolor{red!45}{$\times$}	&	1	&	\cellcolor{green!45}{\checked}	&	-	&	\cellcolor{red!45}{$\times$}	&	1	&	\cellcolor{red!45}{$\times$}	&	1	&	\cellcolor{green!45}{\checked}	&	-
		\\ \rule{0pt}{8pt} 																																																																																								
		s15	&	\cellcolor{green!45}{\checked}	&	-	&	\cellcolor{green!45}{\checked}	&	-	&	\cellcolor{green!45}{\checked}	&	-	&	\cellcolor{green!45}{\checked}	&	-	&	\cellcolor{green!45}{\checked}	&	-	&	\cellcolor{green!45}{\checked}	&	-	&	\cellcolor{green!45}{\checked}	&	-	&	m15	&	\cellcolor{green!45}{\checked}	&	-	&	\cellcolor{red!45}{$\times$}	&	-	&	\cellcolor{green!45}{\checked}	&	-	&	\cellcolor{green!45}{\checked}	&	-	&	\cellcolor{green!45}{\checked}	&	-	&	\cellcolor{green!45}{\checked}	&	-	&	\cellcolor{green!45}{\checked}	&	-	&	c15	&	\cellcolor{red!45}{$\times$}	&	-	&	\cellcolor{red!45}{$\times$}	&	-	&	\cellcolor{red!45}{$\times$}	&	1	&	\cellcolor{red!45}{$\times$}	&	1	&	\cellcolor{red!45}{$\times$}	&	1	&	\cellcolor{green!45}{\checked}	&	-	&	\cellcolor{red!45}{$\times$}	&	1
		\\ \rule{0pt}{8pt} 																																																																																								
		s16	&	\cellcolor{green!45}{\checked}	&	-	&	\cellcolor{red!45}{$\times$}	&	-	&	\cellcolor{green!45}{\checked}	&	-	&	\cellcolor{green!45}{\checked}	&	-	&	\cellcolor{green!45}{\checked}	&	-	&	\cellcolor{green!45}{\checked}	&	-	&	\cellcolor{green!45}{\checked}	&	-	&	m16	&	\cellcolor{red!45}{$\times$}	&	3	&	\cellcolor{green!45}{\checked}	&	-	&	\cellcolor{green!45}{\checked}	&	-	&	\cellcolor{green!45}{\checked}	&	-	&	\cellcolor{green!45}{\checked}	&	-	&	\cellcolor{green!45}{\checked}	&	-	&	\cellcolor{green!45}{\checked}	&	-	&	c16	&	\cellcolor{red!45}{$\times$}	&	-	&	\cellcolor{red!45}{$\times$}	&	-	&	\cellcolor{red!45}{$\times$}	&	1	&	\cellcolor{green!45}{\checked}	&	-	&	\cellcolor{red!45}{$\times$}	&	1	&	\cellcolor{red!45}{$\times$}	&	1	&	\cellcolor{green!45}{\checked}	&	-
		\\ \rule{0pt}{8pt} 																																																																																								
		s17	&	\cellcolor{green!45}{\checked}	&	-	&	\cellcolor{red!45}{$\times$}	&	-	&	\cellcolor{green!45}{\checked}	&	-	&	\cellcolor{green!45}{\checked}	&	-	&	\cellcolor{green!45}{\checked}	&	-	&	\cellcolor{green!45}{\checked}	&	-	&	\cellcolor{green!45}{\checked}	&	-	&	m17	&	\cellcolor{red!45}{$\times$}	&	-	&	\cellcolor{red!45}{$\times$}	&	-	&	\cellcolor{green!45}{\checked}	&	-	&	\cellcolor{green!45}{\checked}	&	-	&	\cellcolor{green!45}{\checked}	&	-	&	\cellcolor{green!45}{\checked}	&	-	&	\cellcolor{green!45}{\checked}	&	-	&	c17	&	\cellcolor{red!45}{$\times$}	&	1	&	\cellcolor{red!45}{$\times$}	&	-	&	\cellcolor{red!45}{$\times$}	&	1	&	\cellcolor{red!45}{$\times$}	&	1	&	\cellcolor{red!45}{$\times$}	&	1	&	\cellcolor{red!45}{$\times$}	&	1	&	\cellcolor{green!45}{\checked}	&	-
		\\ \rule{0pt}{8pt} 																																																																																								
		s18	&	\cellcolor{green!45}{\checked}	&	-	&	\cellcolor{green!45}{\checked}	&	-	&	\cellcolor{green!45}{\checked}	&	-	&	\cellcolor{green!45}{\checked}	&	-	&	\cellcolor{green!45}{\checked}	&	-	&	\cellcolor{green!45}{\checked}	&	-	&	\cellcolor{green!45}{\checked}	&	-	&	m18	&	\cellcolor{red!45}{$\times$}	&	-	&	\cellcolor{red!45}{$\times$}	&	-	&	\cellcolor{green!45}{\checked}	&	-	&	\cellcolor{green!45}{\checked}	&	-	&	\cellcolor{red!45}{$\times$}	&	1	&	\cellcolor{green!45}{\checked}	&	-	&	\cellcolor{green!45}{\checked}	&	-	&	c18	&	\cellcolor{red!45}{$\times$}	&	-	&	\cellcolor{red!45}{$\times$}	&	-	&	\cellcolor{green!45}{\checked}	&	-	&	\cellcolor{green!45}{\checked}	&	-	&	\cellcolor{green!45}{\checked}	&	-	&	\cellcolor{red!45}{$\times$}	&	1	&	\cellcolor{red!45}{$\times$}	&	-
		\\ \rule{0pt}{8pt} 																																																																																								
		s19	&	\cellcolor{green!45}{\checked}	&	-	&	\cellcolor{green!45}{\checked}	&	1	&	\cellcolor{green!45}{\checked}	&	-	&	\cellcolor{green!45}{\checked}	&	-	&	\cellcolor{green!45}{\checked}	&	-	&	\cellcolor{green!45}{\checked}	&	-	&	\cellcolor{green!45}{\checked}	&	-	&	m19	&	\cellcolor{green!45}{\checked}	&	-	&	\cellcolor{red!45}{$\times$}	&	-	&	\cellcolor{green!45}{\checked}	&	-	&	\cellcolor{green!45}{\checked}	&	-	&	\cellcolor{green!45}{\checked}	&	-	&	\cellcolor{green!45}{\checked}	&	-	&	\cellcolor{green!45}{\checked}	&	-	&	c19	&	\cellcolor{green!45}{\checked}	&	-	&	\cellcolor{red!45}{$\times$}	&	-	&	\cellcolor{green!45}{\checked}	&	-	&	\cellcolor{green!45}{\checked}	&	-	&	\cellcolor{red!45}{$\times$}	&	1	&	\cellcolor{red!45}{$\times$}	&	1	&	\cellcolor{green!45}{\checked}	&	-
		\\ \rule{0pt}{8pt} 																																																																																								
		s20	&	\cellcolor{green!45}{\checked}	&	1	&	\cellcolor{green!45}{\checked}	&	-	&	\cellcolor{green!45}{\checked}	&	-	&	\cellcolor{green!45}{\checked}	&	-	&	\cellcolor{green!45}{\checked}	&	-	&	\cellcolor{green!45}{\checked}	&	-	&	\cellcolor{green!45}{\checked}	&	-	&	m20	&	\cellcolor{red!45}{$\times$}	&	2	&	\cellcolor{green!45}{\checked}	&	-	&	\cellcolor{green!45}{\checked}	&	-	&	\cellcolor{green!45}{\checked}	&	-	&	\cellcolor{red!45}{$\times$}	&	2	&	\cellcolor{green!45}{\checked}	&	-	&	\cellcolor{green!45}{\checked}	&	-	&	c20	&	\cellcolor{green!45}{\checked}	&	-	&	\cellcolor{green!45}{\checked}	&	-	&	\cellcolor{green!45}{\checked}	&	-	&	\cellcolor{green!45}{\checked}	&	-	&	\cellcolor{red!45}{$\times$}	&	1	&	\cellcolor{red!45}{$\times$}	&	1	&	\cellcolor{green!45}{\checked}	&	-
		\\ \rule{0pt}{8pt} 																																																																																								
		s21	&	\cellcolor{green!45}{\checked}	&	1	&	\cellcolor{green!45}{\checked}	&	-	&	\cellcolor{green!45}{\checked}	&	-	&	\cellcolor{green!45}{\checked}	&	-	&	\cellcolor{green!45}{\checked}	&	-	&	\cellcolor{green!45}{\checked}	&	-	&	\cellcolor{green!45}{\checked}	&	-	&	m21	&	\cellcolor{green!45}{\checked}	&	-	&	\cellcolor{red!45}{$\times$}	&	3	&	\cellcolor{red!45}{$\times$}	&	1	&	\cellcolor{green!45}{\checked}	&	-	&	\cellcolor{red!45}{$\times$}	&	1	&	\cellcolor{red!45}{$\times$}	&	1	&	\cellcolor{green!45}{\checked}	&	-	&	c21	&	\cellcolor{green!45}{\checked}	&	1	&	\cellcolor{green!45}{\checked}	&	-	&	\cellcolor{red!45}{$\times$}	&	1	&	\cellcolor{red!45}{$\times$}	&	1	&	\cellcolor{red!45}{$\times$}	&	1	&	\cellcolor{red!45}{$\times$}	&	1	&	\cellcolor{red!45}{$\times$}	&	1
		\\ \rule{0pt}{8pt} 																																																																																								
		s22	&	\cellcolor{green!45}{\checked}	&	-	&	\cellcolor{green!45}{\checked}	&	-	&	\cellcolor{green!45}{\checked}	&	-	&	\cellcolor{green!45}{\checked}	&	-	&	\cellcolor{green!45}{\checked}	&	-	&	\cellcolor{green!45}{\checked}	&	-	&	\cellcolor{green!45}{\checked}	&	-	&	m22	&	\cellcolor{green!45}{\checked}	&	-	&	\cellcolor{green!45}{\checked}	&	-	&	\cellcolor{green!45}{\checked}	&	-	&	\cellcolor{green!45}{\checked}	&	-	&	\cellcolor{red!45}{$\times$}	&	1	&	\cellcolor{green!45}{\checked}	&	-	&	\cellcolor{green!45}{\checked}	&	-	&	c22	&	\cellcolor{red!45}{$\times$}	&	1	&	\cellcolor{red!45}{$\times$}	&	-	&	\cellcolor{red!45}{$\times$}	&	1	&	\cellcolor{red!45}{$\times$}	&	1	&	\cellcolor{green!45}{\checked}	&	-	&	\cellcolor{red!45}{$\times$}	&	1	&	\cellcolor{green!45}{\checked}	&	-
		\\ \rule{0pt}{8pt} 																																																																																								
		s23	&	\cellcolor{green!45}{\checked}	&	-	&	\cellcolor{red!45}{$\times$}	&	-	&	\cellcolor{green!45}{\checked}	&	-	&	\cellcolor{green!45}{\checked}	&	-	&	\cellcolor{green!45}{\checked}	&	-	&	\cellcolor{green!45}{\checked}	&	-	&	\cellcolor{green!45}{\checked}	&	-	&	m23	&	\cellcolor{green!45}{\checked}	&	-	&	\cellcolor{green!45}{\checked}	&	-	&	\cellcolor{green!45}{\checked}	&	-	&	\cellcolor{green!45}{\checked}	&	-	&	\cellcolor{red!45}{$\times$}	&	1	&	\cellcolor{red!45}{$\times$}	&	1	&	\cellcolor{green!45}{\checked}	&	-	&	c23	&	\cellcolor{red!45}{$\times$}	&	-	&	\cellcolor{red!45}{$\times$}	&	-	&	\cellcolor{green!45}{\checked}	&	-	&	\cellcolor{green!45}{\checked}	&	-	&	\cellcolor{green!45}{\checked}	&	-	&	\cellcolor{green!45}{\checked}	&	-	&	\cellcolor{red!45}{$\times$}	&	-
		\\ \rule{0pt}{8pt} 																																																																																								
		s24	&	\cellcolor{green!45}{\checked}	&	-	&	\cellcolor{red!45}{$\times$}	&	-	&	\cellcolor{green!45}{\checked}	&	-	&	\cellcolor{green!45}{\checked}	&	-	&	\cellcolor{green!45}{\checked}	&	-	&	\cellcolor{green!45}{\checked}	&	-	&	\cellcolor{green!45}{\checked}	&	-	&	m24	&	\cellcolor{green!45}{\checked}	&	-	&	\cellcolor{green!45}{\checked}	&	-	&	\cellcolor{green!45}{\checked}	&	-	&	\cellcolor{green!45}{\checked}	&	-	&	\cellcolor{red!45}{$\times$}	&	1	&	\cellcolor{green!45}{\checked}	&	-	&	\cellcolor{green!45}{\checked}	&	-	&	c24	&	\cellcolor{red!45}{$\times$}	&	-	&	\cellcolor{red!45}{$\times$}	&	-	&	\cellcolor{green!45}{\checked}	&	-	&	\cellcolor{green!45}{\checked}	&	-	&	\cellcolor{red!45}{$\times$}	&	1	&	\cellcolor{green!45}{\checked}	&	-	&	\cellcolor{green!45}{\checked}	&	-
		\\ \rule{0pt}{8pt} 																																																																																								
		s25	&	\cellcolor{green!45}{\checked}	&	-	&	\cellcolor{green!45}{\checked}	&	2	&	\cellcolor{green!45}{\checked}	&	-	&	\cellcolor{green!45}{\checked}	&	-	&	\cellcolor{green!45}{\checked}	&	-	&	\cellcolor{green!45}{\checked}	&	-	&	\cellcolor{green!45}{\checked}	&	-	&	m25	&	\cellcolor{green!45}{\checked}	&	-	&	\cellcolor{green!45}{\checked}	&	-	&	\cellcolor{green!45}{\checked}	&	-	&	\cellcolor{green!45}{\checked}	&	-	&	\cellcolor{red!45}{$\times$}	&	1	&	\cellcolor{green!45}{\checked}	&	-	&	\cellcolor{green!45}{\checked}	&	-	&	c25	&	\cellcolor{red!45}{$\times$}	&	-	&	\cellcolor{red!45}{$\times$}	&	2	&	\cellcolor{red!45}{$\times$}	&	1	&	\cellcolor{red!45}{$\times$}	&	1	&	\cellcolor{red!45}{$\times$}	&	1	&	\cellcolor{green!45}{\checked}	&	-	&	\cellcolor{red!45}{$\times$}	&	1
		\\ \rule{0pt}{8pt} 																																																																																								
		s26	&	\cellcolor{green!45}{\checked}	&	-	&	\cellcolor{green!45}{\checked}	&	-	&	\cellcolor{green!45}{\checked}	&	-	&	\cellcolor{green!45}{\checked}	&	-	&	\cellcolor{green!45}{\checked}	&	-	&	\cellcolor{red!45}{$\times$}	&	1	&	\cellcolor{green!45}{\checked}	&	-	&	m26	&	\cellcolor{green!45}{\checked}	&	-	&	\cellcolor{green!45}{\checked}	&	-	&	\cellcolor{green!45}{\checked}	&	-	&	\cellcolor{green!45}{\checked}	&	-	&	\cellcolor{green!45}{\checked}	&	-	&	\cellcolor{green!45}{\checked}	&	-	&	\cellcolor{green!45}{\checked}	&	-	&	c26	&	\cellcolor{green!45}{\checked}	&	-	&	\cellcolor{red!45}{$\times$}	&	-	&	\cellcolor{green!45}{\checked}	&	-	&	\cellcolor{green!45}{\checked}	&	-	&	\cellcolor{green!45}{\checked}	&	-	&	\cellcolor{green!45}{\checked}	&	-	&	\cellcolor{green!45}{\checked}	&	-
		\\ \rule{0pt}{8pt} 																																																																																								
		s27	&	\cellcolor{green!45}{\checked}	&	-	&	\cellcolor{green!45}{\checked}	&	1	&	\cellcolor{green!45}{\checked}	&	-	&	\cellcolor{green!45}{\checked}	&	-	&	\cellcolor{green!45}{\checked}	&	-	&	\cellcolor{green!45}{\checked}	&	-	&	\cellcolor{green!45}{\checked}	&	-	&	m27	&	\cellcolor{green!45}{\checked}	&	1	&	\cellcolor{green!45}{\checked}	&	1	&	\cellcolor{red!45}{$\times$}	&	1	&	\cellcolor{red!45}{$\times$}	&	1	&	\cellcolor{red!45}{$\times$}	&	1	&	\cellcolor{green!45}{\checked}	&	-	&	\cellcolor{green!45}{\checked}	&	-	&	c27	&	\cellcolor{green!45}{\checked}	&	1	&	\cellcolor{green!45}{\checked}	&	-	&	\cellcolor{green!45}{\checked}	&	-	&	\cellcolor{green!45}{\checked}	&	-	&	\cellcolor{green!45}{\checked}	&	-	&	\cellcolor{green!45}{\checked}	&	-	&	\cellcolor{green!45}{\checked}	&	-
		\\ \rule{0pt}{8pt} 																																																																																								
		s28	&	\cellcolor{red!45}{$\times$}	&	-	&	\cellcolor{green!45}{\checked}	&	-	&	\cellcolor{green!45}{\checked}	&	-	&	\cellcolor{green!45}{\checked}	&	-	&	\cellcolor{green!45}{\checked}	&	-	&	\cellcolor{green!45}{\checked}	&	-	&	\cellcolor{green!45}{\checked}	&	-	&	m28	&	\cellcolor{green!45}{\checked}	&	-	&	\cellcolor{green!45}{\checked}	&	-	&	\cellcolor{green!45}{\checked}	&	-	&	\cellcolor{green!45}{\checked}	&	-	&	\cellcolor{red!45}{$\times$}	&	1	&	\cellcolor{green!45}{\checked}	&	-	&	\cellcolor{green!45}{\checked}	&	-	&	c28	&	\cellcolor{green!45}{\checked}	&	2	&	\cellcolor{green!45}{\checked}	&	1	&	\cellcolor{red!45}{$\times$}	&	1	&	\cellcolor{red!45}{$\times$}	&	1	&	\cellcolor{red!45}{$\times$}	&	1	&	\cellcolor{red!45}{$\times$}	&	1	&	\cellcolor{red!45}{$\times$}	&	1
		\\ \rule{0pt}{8pt} 																																																																																								
		s29	&	\cellcolor{green!45}{\checked}	&	-	&	\cellcolor{red!45}{$\times$}	&	-	&	\cellcolor{green!45}{\checked}	&	-	&	\cellcolor{green!45}{\checked}	&	-	&	\cellcolor{green!45}{\checked}	&	-	&	\cellcolor{green!45}{\checked}	&	-	&	\cellcolor{green!45}{\checked}	&	-	&	m29	&	\cellcolor{red!45}{$\times$}	&	-	&	\cellcolor{red!45}{$\times$}	&	-	&	\cellcolor{green!45}{\checked}	&	-	&	\cellcolor{green!45}{\checked}	&	-	&	\cellcolor{green!45}{\checked}	&	-	&	\cellcolor{green!45}{\checked}	&	-	&	\cellcolor{red!45}{$\times$}	&	1	&	c29	&	\cellcolor{red!45}{$\times$}	&	3	&	\cellcolor{red!45}{$\times$}	&	4	&	\cellcolor{red!45}{$\times$}	&	1	&	\cellcolor{red!45}{$\times$}	&	1	&	\cellcolor{red!45}{$\times$}	&	1	&	\cellcolor{red!45}{$\times$}	&	1	&	\cellcolor{red!45}{$\times$}	&	-
		\\ \rule{0pt}{8pt} 																																																																																								
		s30	&	\cellcolor{red!45}{$\times$}	&	-	&	\cellcolor{red!45}{$\times$}	&	-	&	\cellcolor{green!45}{\checked}	&	-	&	\cellcolor{green!45}{\checked}	&	-	&	\cellcolor{green!45}{\checked}	&	-	&	\cellcolor{red!45}{$\times$}	&	1	&	\cellcolor{red!45}{$\times$}	&	1	&	m30	&	\cellcolor{red!45}{$\times$}	&	-	&	\cellcolor{red!45}{$\times$}	&	-	&	\cellcolor{red!45}{$\times$}	&	1	&	\cellcolor{green!45}{\checked}	&	-	&	\cellcolor{red!45}{$\times$}	&	1	&	\cellcolor{red!45}{$\times$}	&	1	&	\cellcolor{red!45}{$\times$}	&	-	&	c30	&	\cellcolor{green!45}{\checked}	&	3	&	\cellcolor{green!45}{\checked}	&	1	&	\cellcolor{red!45}{$\times$}	&	1	&	\cellcolor{green!45}{\checked}	&	-	&	\cellcolor{green!45}{\checked}	&	-	&	\cellcolor{green!45}{\checked}	&	-	&	\cellcolor{red!45}{$\times$}	&	1 \\
		\bottomrule
	\end{tabular}
\end{table*}

\subsection{[RQ1]: What are the differences in detection performance between LLMs and EDA tools?}

We utilize the LintLLM framework to enhance LLMs, aiming to evaluate its state-of-the-art performance in defect detection. The detailed experimental results are shown in TABLE \ref{Detail_Results}. 

Simple benchmark. The performance of LLMs and commercial EDA tool is comparable. However, Verilator failed to detect 6 defects and generated 5 false positives.

Medium benchmark. The advantages of LLMs are highlighted. DeepSeek V2.5 (\ding{185}) and o1-mini (\ding{188}) have only one (m27) or two (m29, m30) defects not detected, respectively, and only one false positive. However, Llama-3.1 (\ding{184}) and GPT-4 (\ding{186}) perform poorly in complex DUTs.

Complex benchmark. Although both LLMs and EDA tools face challenges, LLMs still outperform EDA tools. o1-mini (\ding{188}) correctly detects 18 defects and generates 9 false positives, performing best among the evaluated models. In contrast, commercial EDA tool (\ding{182}) only correctly detects 12 defects and generates 16 false positives. Meanwhile, Verilator (\ding{183}) produced 14 false positives and only correctly detected 12 defects out of 30 complex DUTs. 

The detection results of the Original models and the LintLLM optimization are shown in Fig. \ref{overview_result}. All LLMs show improvements in the LintLLM scenario. In terms of false positives, Llama-3.1 and DeepSeek V2.5 reduce 6 and 1 false positives, respectively. However, the false positives of the GPT series models (GPT-4, GPT-4o, and o1-mini) increase slightly, which will be discussed in RQ2 (Section \ref{rq2}).


\begin{figure}[htbp]
	\centering
		\begin{subfigure}[]{0.24\textwidth}
				\includegraphics[width=\textwidth]{picture/C_Original.pdf}\vspace{-2mm}
				\caption{Correct: Original}
				\label{original_c}
			\end{subfigure}
		%	\hfill % or 
		\begin{subfigure}[]{0.24\textwidth}
				\includegraphics[width=\textwidth]{picture/F_Original.pdf}\vspace{-2mm}
				\caption{False Positive: Original}
				\label{original_f}
			\end{subfigure}
	\begin{subfigure}[]{0.24\textwidth}
		\includegraphics[width=\textwidth]{picture/C_LintLLM.pdf}\vspace{-2mm}
		\caption{Correct: LintLLM}
		\label{Both_methods_c}
	\end{subfigure}
	%	\hfill % or 
	\begin{subfigure}[]{0.24\textwidth}
		\includegraphics[width=\textwidth]{picture/F_LintLLM.pdf}\vspace{-2mm}
		\caption{False Positive: LintLLM}
		\label{Both_methods_f}
	\end{subfigure}
	\caption{Comparison of defect detection results between the original model and LintLLM. The numbers in the overlapping area indicate the number of DUTs detected as correct or false positive by multiple models. It shows that LintLLM outperforms the Original method.}
	\label{overview_result}
	
\end{figure}

% \textbf{Summary}: LLM outperform EDA tools in both detection accuracy and false positive control. By utilizing LintLLM method, LLM shows strong potential and application prospects in code defect detection tasks.

\textbf{Summary}: LLMs outperform EDA tools in both detection accuracy and false positive control. Among 90 DUTs, o1-mini enhanced by LintLLM detects 75 defects and produces 11 false positives, while commercial EDA tool only detects 58 defects and produces 15 false positives. It demonstrates the powerful potential of LLMs in code defect detection.

\begin{figure*}[htbp]
	\centering
	\begin{subfigure}[b]{0.195\textwidth}
		\includegraphics[width=\textwidth]{picture/Llama-3.1.pdf} \vspace{-6mm}
		\caption{Llama-3.1}
	\end{subfigure}
	\begin{subfigure}[b]{0.195\textwidth}
		\includegraphics[width=\textwidth]{picture/DeepSeek_V2.5.pdf} \vspace{-6mm}
		\caption{DeepSeek V2.5}
	\end{subfigure}
	\begin{subfigure}[b]{0.195\textwidth}
		\includegraphics[width=\textwidth]{picture/GPT-4.pdf} \vspace{-6mm}
		\caption{GPT-4}
	\end{subfigure}
	\begin{subfigure}[b]{0.195\textwidth}
		\includegraphics[width=\textwidth]{picture/GPT-4o.pdf} \vspace{-6mm}
		\caption{GPT-4o}
	\end{subfigure}
	\begin{subfigure}[b]{0.195\textwidth}
		\includegraphics[width=\textwidth]{picture/o1-mini.pdf} \vspace{-6mm}
		\caption{o1-mini}
	\end{subfigure}
	\caption{The impact of different prompting methods on defect detection. The CR is better when higher (\textcolor{blue}{$\nearrow$}), and FR is better when lower (\textcolor{red}{$\searrow$}). It is clearly shown that Prompt of Logic-Tree improves the correct rate, and Defect Tracker reduces the false-positive rate. With the combined effect of the two methods, LLMs have the best performance.}
	\label{different_methods}
\end{figure*}

\subsection{[RQ2]: Why our methods can improve the detection performance of LLM? \label{rq2}}

To explore the performance of different methods, we conducted experiments using Prompt of Logic-Tree and Defect Tracker, respectively. The results are shown in Fig. \ref{different_methods}. Prompt of Logic-Tree improves correct rate for GPT-4 from 47.78\% to 48.89\%, while the false-positive rate drops to 24.44\%. This shows that the method can help LLMs make more accurate judgments on defect detection. Although the correct rate of DeepSeek V2.5 significantly increases to 75.56\%, the false-positive rate also increases to 47.78\%. This is because the Prompt of Logic-Tree mainly enhances LLMs to follow the defect detection process, but ignores the case where multiple defects exist in the code.

Defect Tracker reduces false-positive rates by tracking multiple defect paths in the code and locating the main defect that causes multiple defects. Except for GPT-4o, the false-positive rates of each model show a significant reduction, while improving the detection correctness. These results indicate that the Defect Tracker is very effective in reducing false-positive rate, especially for models with high initial false-positive rates, such as Llama-3.1 and GPT-4. 

Combining the two methods improves the model's overall performance, yet the false-positive rate of GPT series models (GPT-4, GPT-4o, and o1-mini) does not drop as expected. We speculate two main reasons. (\rmnum{1}) Possibly some undisclosed security mechanisms prevent the knowledge structure of the model from being changed by prompts. (\rmnum{2}) Combining the two methods may produce information conflicts, leading to an increase in false-positive rates. Future research will focus on dynamically adjusting the prompts according to the characteristics of LLMs.

%\textbf{Summary}: Prompt of Logic-Tree improve the accuracy by gradually guiding LLM to detect code defects, while Defect Tracker effectively reduces the false positive rate by locating the main defects. The overall performance is improved after combining the both methods.

\textbf{Summary}: Prompt of Logic-Tree uses logical prompts to guide LLMs for gradually detecting defects and improving the accuracy, such as the accuracy of DeepSeek V2.5 increasing from 63.33\% to 75.56\%. Defect Tracker effectively reduces the false-positive rate by locating the main defects, with o1-mini's false-positive rate dropping by 42.23\%. The comprehensive performance is improved after combining both methods.




\subsection{[RQ3]: What are the advantages of LLMs in detection costs compared with commercial EDA tools?}

To comprehensively explore the cost advantages of LLMs, we conducted further analysis. Based on provider pricing of LLMs \cite{price}, the cost of processing 1 million tokens (about 80,000 lines of code) is \$3 for inputting and \$12 for outputting. The input tokens include DUTs and prompts, and the output tokens cover the analysis process and generated defect reports. After conversion, the total cost of detecting 80,000 lines of code is about \$20. For commercial EDA tools, we assume the annual license fee is about \$1.2 million.

Firstly, we analyze the relationship between code volume and cost, as shown in Fig. 6(a). When the annual code volume is less than 4800 million lines (equivalent to the code volume of a development team of 100,000 people in one year), using LLMs is more economical (Region I). However, when this threshold is exceeded, the cost advantage of using commercial EDA tool is more obvious (Region II). Therefore, LLMs is a more cost-effective solution for small and medium-sized enterprises and research institutions with less code volume.

Secondly, we further quantified the cost through a real case. A medium-sized CPU control unit contains about 1,000 lines of Verilog code, which meets the context tokens restrictions of LLMs. The cost of using LLMs to perform 1 defect detection is \$0.285, and it runs 1,000 times per day, with an annual cost of about \$0.1 million. Fig. 6(b) shows the relationship between cost and detection performance. Based on commercial EDA tool, investing \$0.1 million to integrate LLMs will improve performance by approximately 20.00\%, achieving a correct rate of 87.78\% and a false-positive rate of 7.78\%.

\begin{figure}[htbp]
	\centering
	\begin{subfigure}[]{0.24\textwidth}
		\includegraphics[width=\textwidth]{picture/cost11.pdf}
		\caption{Lines of Code vs. Cost}
		\label{cost1}
	\end{subfigure}
	%	\hfill % or 
	\begin{subfigure}[]{0.24\textwidth}
		\includegraphics[width=\textwidth]{picture/cost23.pdf}
		\caption{Cost vs. Performance}
		\label{cost2}
	\end{subfigure}
	\caption{Cost Comparison of commercial EDA Tool and LLM. LLM achieves better performance at one-tenth the cost of EDA tools.}
	\label{cost}
\end{figure}

\textbf{Summary}: For small and medium-sized enterprises and research institutions processing fewer than 4,800 million lines of code annually, LLM offers significant advantages in reducing defect detection costs. Additionally, integrating LLM with commercial EDA tool at a lower cost will improve performance by approximately 20\%.



\section{Conclusion}

This paper introduces LintLLM, an open-source Linting framework that leverages LLMs to detect defects in Verilog code. This study demonstrates that LLM is an efficient and economical solution for hardware defect detection. In the future, we will explore using LLMs to automatically repair defects after completing detection and finally obtain high-quality RTL designs. Additionally, integrating LintLLM into  IDEs is worth considering.
\clearpage

\bibliographystyle{IEEEtran} 
% \bibliography{Mybib.bib}
{\footnotesize \bibliography{Mybib.bib}}
\nocite{*}

\end{document}
