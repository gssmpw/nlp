\section{Related Works}
The study of computational aesthetics and art analysis has grown significantly with advancements in \gls{ai}\gls{ml}. These technologies have provided new methods for exploring artistic styles and understanding cultural patterns. Datasets play an essential role in this progress, serving as the foundation for training and testing models. The quality and diversity of datasets greatly impact model performance and generalizability to various artistic and cultural contexts. Over recent decades, many initiatives have contributed to digitizing art collections, making them accessible for computational analysis and enabling the creation of robust datasets.

One of the most well-known datasets for aesthetic evaluation is the aesthetic visual analysis (AVA) dataset, created by____. It contains over 250,000 images, each annotated with scores that reflect their aesthetic quality. This dataset was primarily focused on photographs, as was the aesthetic and attribute database (AADB) introduced by____ containing aesthetic scores and meaningful attributes assigned to each image by multiple human raters.
 While these datasets helped researchers study the aesthetics of photographs, they were not suitable for analyzing artistic works such as paintings and drawings. To address this, art-specific datasets like WikiArt\footnote{\url{https://www.wikiart.org/}} and the Rijksmuseum Challenge____ introduced thousands of images of artworks, enabling tasks such as artist identification and style recognition. %However, these datasets lacked detailed aesthetic annotations that could allow for deeper analysis of artistic content.

\gls{dl} methods have advanced the analysis of art, making it possible to classify styles ____ and attribute artworks to their creators with high accuracy. One common approach is transfer learning, where convolutional neural networks trained on large datasets like ImageNet____ are fine-tuned for tasks specific to art analysis____. This method was effective for style classification and artist attribution. Neural style transfer (NST)____ is another approach allowing models to separate an artwork's style from its content. NST contributed for further research into how stylistic features can be analyzed and even replicated computationally. Researchers have also studied how styles evolve and connect with cultural patterns. For example,____ showed that neural networks could detect stylistic features that align with historical concepts in art, while datasets like OmniArt____ have been used to map stylistic diversity across different cultures. 

However, while \gls{dl} models have significantly transformed the analysis of art by identifying, for example, patterns and predict aesthetic scores, they often struggle to interpret the symbolic meanings or historical significance of artworks____. Moreover, \gls{dl} models rely heavily on labeled dataset while creating such datasets for art analysis is time-consuming and expensive knowing that they often focus on a limited range of artistic styles and concepts. Furthermore, models designed for one type of artistic task, such as style classification, struggle to adapt to other tasks without significant adjustments____.
\begin{figure*}[!ht] % picture
    \centering
    \includegraphics[width=\textwidth]{figs/Fig-1.pdf}
    \caption{Distribution of the number of artworks across different styles for individual artists. We retrieved more than 15,000 artworks across 23 artists for our study.}
    \label{fig:fig1}
\end{figure*}
Multimodal approaches, which combine text and images, have become an important area of research. Projects like Artpedia____ and SemArt____ introduced models that align both visual and textual description (contextual descriptions)  in a shared semantic space. 
This alignment enables advanced searches based on content. %and provides a more holistic understanding of artworks. 
For example, a user could search for paintings that match a specific theme or historical context, and the system would retrieve relevant results by analyzing both visual features and textual descriptions. The OmniArt dataset has been particularly useful in this field, as it includes detailed metadata alongside visual and textual annotations, supporting deeper studies of iconography and stylistic elements. These methods highlight the potential of combining different types of data to improve how we study and interpret art.

The rise of LLM such as GPT____,  Gemeni____, and CLIP____ has further contributed to improve computational aesthetics by bridging the gap between visual and textual data. \glspl{llm} have expanded the possibilities for multimodal analysis, allowing models to process text and images in a unified framework. For example, CLIP aligns textual and visual embeddings, enabling the evaluation of artworks based on textual descriptions, such as ``a serene landscape with balanced composition''. Such models offer new capabilities for integrating narrative and aesthetic elements, facilitating tasks like multimodal retrieval and aesthetic judgment.  Generative \glspl{llm}____ generates artworks based on textual prompts, providing insights into aesthetic preferences and stylistic variations.

\glspl{llm} with their advanced ability to process and understand complex textual data, can provide an interesting approach to interpreting and analyzing artworks. In fact, \glspl{llm} can analyze not just the visual elements of an artwork but also its associated textual descriptions, historical background, and cultural significance.  This enables them to bridge the gap between visual content and human interpretation, offering deeper insights into the meaning, symbolism, and emotional resonance of art. Glimpses of \glspl{llm} potential in art analysis have emerged, yet a framework that comprehensively examines both technical and conceptual elements of art remains underexplored, motivating the need for this study.