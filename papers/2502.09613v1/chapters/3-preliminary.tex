\section{Preliminaries}
\noindent \textbf{Variational autoencoder.}
A variational autoencoder (VAE) ~\citep{kingma2013auto} is a generative model that represents high-dimensional data distributions in a lower-dimensional latent space. 
The encoder maps the input data $\mathbf{x}$ to a latent variable $\mathbf{z}$ by estimating the parameters of a posterior distribution $q_{\phi}(\mathbf{z}|\mathbf{x})$. The posterior is typically assumed to follow the Gaussian distribution, parameterized by a mean $\mu_{\phi}(\mathbf{x})$ and a variance $\sigma_{\phi}(\mathbf{x})$. The latent variable $\mathbf{z}$ is sampled from this posterior distribution, i.e., $\mathbf{z} \sim q_{\phi}(\mathbf{z}|\mathbf{x}) = \mathcal{N}(\mathbf{z}; \mu_{\phi}(\mathbf{x}), \sigma_{\phi}(\mathbf{x})^2)$. The decoder reconstructs the input $\mathbf{x}$ by mapping $\mathbf{z}$ back to the data space through the likelihood $p_{\theta}(\mathbf{x}|\mathbf{z})$. The learning objective of is:
\begin{equation}
\mathcal{L}_{\text{VAE}}(\theta, \phi; \mathbf{X}) = \mathbb{E}_{q_{\phi}(\mathbf{Z}|\mathbf{X})}[\log p_{\theta}(\mathbf{X}|\mathbf{Z})] - \text{KL}(q_{\phi}(\mathbf{Z}|\mathbf{X}) \| p(\mathbf{Z})).
\label{eq:vae}
\end{equation}



\noindent \textbf{3D Gaussian Splatting.}
3DGS~\citep{kerbl3Dgaussians} is an efficient NVS framework that uses a set of 3D Gaussian primitives to represent a scene explicitly. Each Gaussian primitive has a position vector $\boldsymbol{\mu} \in \mathbb{R}^3$, a 3D covariance matrix $\boldsymbol{\Sigma} \in \mathbb{R}^{3\times 3}$, an opacity $\alpha \in \mathbb{R}$, and a spherical harmonics (SH) coefficient  $\boldsymbol{c} \in \mathbb{R}^k$ \citep{ramamoorthi2001efficient} representing the view dependent colors.
\begin{equation}
G(x)=e^{-\frac{1}{2}(x-\boldsymbol{\mu})^T \boldsymbol{\Sigma}^{-1}(x-\boldsymbol{\mu)}},
\end{equation}
where ${\Sigma}={R}{S}{S}^T{R}^T$, ${S}$ denotes the scaling matrix and ${R}$ is the rotation matrix. Then, rasterization~\citep{zwicker2001surface} can transform the 3D Gaussian spheres to the 2D camera plane to calculate the 2D covariance matrix in the camera space as 
\begin{equation}
{\Sigma}^{'} = {J}{W}{\Sigma} {W}^T{J}^T,
\end{equation}
where ${W}$ is the perspective transformation matrix and ${J}$ is Jacobin of the projection matrix.
For every pixel, the Gaussians are traversed in depth order from the image plane, and their pixel colors $c_i$  are combined through alpha compositing, forming pixel color ${C}$ as
\begin{equation}
{C}=\sum_{i \in N} c_i \alpha_i \prod_{j=1}^{i-1}\left(1-\alpha_j\right).
\end{equation}

