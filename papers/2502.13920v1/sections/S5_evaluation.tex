\section{Evaluation}\label{sec.evaluation}

We conducted an in-the-wild deployment study with 16 participants to evaluate our system compared to a baseline chatbot. 
We aimed to answer the following research questions:
\begin{itemize}
    \item How does \name{} impact users' sleep health behaviors and outcomes?
    \item How effective are \name{}'s activity recommendations and theory-guided conversations in motivating behavior change?
    \item What are users' perceived benefits and limitations towards using \name{}?
\end{itemize}

\subsection{Study Design}\label{subsec.study_design}
We employed a within-subjects design to evaluate \name{} against a baseline system. The study lasted for eight weeks and consisted of three phases:
\begin{itemize}
    \item \textbf{Observation (16 days):} Participants maintained their regular sleep and activity patterns, such that we can collect user historical data to initialize our models.
    \item \textbf{\name{} (20 days):} Our chatbot~\jianben{engaged} users in theory-guided and data-driven conversations to motivate users' sleep-enhancing behavior change. 
    \item \textbf{Baseline (20 days):} Participants used a ChatGPT-style chatbot capable of answering general health-related questions, providing general sleep advice, and answering questions about users' health data.
\end{itemize}
To mitigate ordering effects, we counterbalanced the Baseline and \name{} phases across participants. 

\subsection{System Implementation and Setup}
Both \name{} and the Baseline were implemented as web-based applications hosted on AWS EC2, accessible via URL. Key differences between the two systems were:
\begin{itemize}
    \item \name{}: Implemented as described in \autoref{sec.chatbot}, featuring theory-guided conversations and personalized activity recommendations.
    \item Baseline: Followed a similar framework but lacked the theory-guided approach and personalized activity recommendations of \name{}. Specifically, the baseline leverages agent coordinators to decide whether users ask questions about their data. If yes, the data insight agent uses a Python tool to retrieve and analyze the target data subsets and provide a response based on the conversation context.  Otherwise, the chatbot responds to the user message directly.
\end{itemize}
Both systems were initialized with a system prompt to act as a sleep expert, helping users improve sleep quality.

\subsection{Participants}\label{subsec.participants}
We recruited (N=16)
participants (age range: 18-44 years, 5 females, 10 males, 1 prefer not to say) from diverse sources, including email listservs and posters at university, personal connections, and social clubs (\eg, running clubs).
\autoref{tab:demographics} summarizes participant demographics.
They were primarily Asians (9) and Whites (5); most held graduate degrees (14).
Regarding work arrangements,
7 split time evenly between home and office, and 5 worked mainly in the office; almost all resided in urban areas.

Prior to the study, 7 participants did not own or wear a wearable device, and 9 reported using a wearable device for varying periods from less than 6 months to over 2 years.
Participants were highly motivated—with 12 (75\%) very or extremely motivated to improve physical activity and 13 (81.25\%) to improve sleep health.
Sleep patterns varied among participants, with 6 self-identifying as night owls, 5 as average sleepers, 3 as early birds, and 2 as irregular.
The most common bedtimes were between 11 to 12 pm, and the most common wake times were between 6 to 9 am.
This varied sample enabled the evaluation of our sleep health chatbot across different profiles, work arrangements, and sleep habits.



\begin{table}[htbp]
\centering
\caption{Participant demographics.}
\label{tab:demographics}
\small % Reduces font size for better fit
\resizebox{\columnwidth}{!}{
\begin{tabular}{p{0.35\columnwidth} p{0.42\columnwidth} r}
\hline
\textbf{Characteristic} & \textbf{Category} & \textbf{N (\%)} \\
\hline
\textbf{Age} & 18 to 24 & 3 (18.75) \\
 & 25 to 34 & 9 (56.25) \\
 & 35 to 44 & 4 (25.00) \\
\hline
\textbf{Gender} & Female & 5 (31.25) \\
 & Male & 10 (62.50) \\
 & Prefer not to say & 1 (6.25) \\
\hline
\textbf{Race/Ethnicity} & Asian & 9 (56.25) \\
 & Black or African American & 1 (6.25) \\
 & Other & 1 (6.25) \\
 & White & 5 (31.25) \\
\hline
\textbf{Occupation} & Full-time employed & 8 (50.00) \\
 & Full-time employed and Student & 2 (12.50) \\
 & Not employed & 1 (6.25) \\
 & Student & 5 (31.25) \\
\hline
\textbf{Wearable Device Usage} & No prior ownership & 6 (37.50) \\
 & Own but never use & 1 (6.25) \\
 & Less than 6 months & 3 (18.75) \\
 & 6 months to 2 years & 3 (18.75) \\
 & Over 2 years & 3 (18.75) \\
\hline
\textbf{Motivation to Improve Physical Activity} & Extremely motivated & 4 (25.00) \\
 & Very motivated & 8 (50.00) \\
 & Somewhat motivated & 4 (25.00) \\
\hline
\textbf{Motivation to Improve Sleep Health} & Extremely motivated & 5 (31.25) \\
 & Very motivated & 8 (50.00) \\
 & Somewhat motivated & 2 (12.50) \\
 & Not very motivated & 1 (6.25) \\
\hline
\textbf{Sleep Pattern} & Night owl & 6 (37.50) \\
 & Average sleeper & 5 (31.25) \\
 & Early bird & 3 (18.75) \\
 & Irregular & 2 (12.50) \\
\hline
\end{tabular}
}
\end{table}




\subsection{Study Procedures}

\subsubsection{Pre-study Preparations}
We took comprehensive measures to ensure ethical conduct and participant readiness.
First, participants were thoroughly informed about the study protocols, such as data collection methods and privacy measures. They signed consent forms detailing the study's purpose, procedures, risks, and benefits.
Participants were instructed on how to interact with the chatbots, asking questions about their sleep and activity data and requesting suggestions. Sample questions were provided to help initiate conversations with both systems.
For data protection, participants were assigned unique identifiers (user ID and password) to access the chatbots.
They were informed about the chatbots' access to their Oura ring metrics and instructed not to input sensitive content. 
Oura rings were provided along with instructions for granting and later revoking API access for data collection.


\subsubsection{Study Phases}
After the observation periods, participants used either the Baseline or \name{} for three weeks. 
Finally, semi-structured interviews (45 minutes) were conducted at the study's end to gather in-depth feedback on both systems.
Throughout the study, participants were asked to wear their Oura ring continuously and carry their iPhones, complete daily questionnaires, participate in post-system questionnaires, and engage in exit interviews at the study's conclusion.

\textbf{Daily questionnaires}
% Participants were reminded to complete a brief daily survey to track 
collected users' perceived sleep quality, motivation levels, and responses to chatbot recommendations.
The daily survey included the following key components:
(1) sleep quality rating (5-point scale: ``Terrible'' to ``Excellent''); 
(2) motivation level to improve sleep/activity (5-point scale);
(3) reasons for low motivation (if applicable);
(4) relevance of chatbot recommendations (5-point scale: ``Not Relevant at all'' to ``Very Relevant'');
(5) adherence to recommendations (Yes/No);


\textbf{Post-system surveys}
% After each phase (Baseline and SleepBot), participants needed to complete a comprehensive survey to evaluate their experience with the respective system. 
% The post-system survey 
assessed user experience on a 5-point Likert scale:
(1) technical aspects (e.g., response time, accuracy, user interface design, overall usability);
(2) personalization and relevance of recommendations;
(3) impact on understanding sleep-activity relationships;
(4) motivation and behavior change;
(5) perceived improvements in sleep quality.
Open-ended questions were also collected on behavior changes and system strengths and limitations.

\textbf{Exit interviews}
% At the conclusion of the study, participants engaged in semi-structured interviews to provide in-depth feedback on two systems based on their overall experience:
gathered feedback on
(1) overall impressions of both systems;
(2) perceived impacts on sleep patterns and physical activity;
(3) perceived engagement and usage over time;
(3) most and least helpful features;
(4) challenges in adopting recommendations;
(5) suggestions for improvement;
(6) likelihood of continued use.

\subsection{Data Collection}
We collected the following data daily throughout the study. 

\textbf{Wearable data.}
Sleep and activity data were collected using the Oura ring (see \autoref{tab:data_collection}). The metrics included duration, efficiency, quality, average HRV, and lowest heart rate, stress levels, and readiness scores.
Activity types and scores were also recorded.

\textbf{Questionnaires and interview data.}
We collected daily questionnaires, post-system questionnaires, and exit interview feedback (detailed in \autoref{subsec.study_design}).

\textbf{System usage data.}
We collected data on users' conversation history with the chatbots.
Based on that, we computed the frequency of interactions and length of conversations.

\subsection{Data Security and Ethical Considerations}
All data was securely stored on a university-hosted Amazon Web Services (AWS) server. 
To protect participants’ privacy, personal information was stored separately from research data, with identifying details securely kept on the EZ backup system.
% Participants' personal information was stored separately from research data, with identifying information kept on the EZ backup system.
To ensure ethical use of chatbots, 
We implemented guardrails using OpenAI's moderation API\footnote{\url{https://platform.openai.com/docs/guides/moderation}} to detect and block the generation of potentially harmful content.
Regular check-ins were also conducted to address participants' concerns about chatbot responses.

\subsection{Data analysis}
% For each research question, we conducted the following data analyses:
We conducted analyses corresponding to each research question:

\subsubsection{Impact on Sleep Health Behaviors and Outcomes}
\begin{itemize}
\item Paired t-tests to compare sleep and activity metrics (\eg, sleep duration, sleep efficiency, and readiness score) between Baseline and \name{} conditions
\item Descriptive statistics (mean and standard deviation) for stress level distribution comparison
\item Thematic analysis of user responses regarding perceived impacts on sleep patterns and physical activity
\end{itemize}
\subsubsection{Effectiveness of Recommendations and Theory-guided Conversations}
\begin{itemize}
\item Wilcoxon signed-rank test to compare personalization and relevance ratings, and adherence rates between Baseline and \name{} conditions
% \item Chi-square tests to compare recommendation adherence rates between conditions
\item Descriptive statistics for the frequency of behavior change techniques employed by \name{}
\item Thematic analysis of user feedback on the most and least helpful features of \name{}
\item Content analysis of chat logs to identify prevalent behavior change techniques and their implementation
\item Expert feedback on system responses to users' messages
\end{itemize}
\subsubsection{User Engagement and Perceived Benefits and Limitations}
\begin{itemize}
\item Paired t-tests to compare user engagement metrics (ratio of active days, conversation length) between Baseline and \name{} conditions
\item Linear regression analysis to examine engagement trends over time for both systems
\item Wilcoxon signed-rank test to compare user ratings on various aspects of system performance, usability, and perceived benefits between Baseline and \name{} conditions
\item Thematic analysis of interview responses and open-ended survey questions to identify:
\begin{itemize}
\item Perceived benefits and limitations of \name{}
\item Challenges in adopting recommendations
\item Suggestions for improvement
\end{itemize}
\item Descriptive statistics of user satisfaction ratings
\end{itemize}






