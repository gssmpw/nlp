\section{Introduction}\label{sec.intro}

Sleep is a fundamental pillar of human health, profoundly influencing physical well-being, cognitive function, and emotional resilience~\cite{medic2017short}. 
With modern wearable devices like smartwatches and rings, monitoring sleep patterns has become more accessible than ever. These technologies provide detailed metrics—sleep duration, stages, heart rate variability, and more—offering valuable insights into nightly rest. However, they often fall short in delivering actionable guidance.
Users are frequently presented with an abundance of metrics but lack clear guidance on how to interpret and apply this data to improve their sleep quality.

Many systems have aimed to bridge this gap through personalized sleep interventions, but most have shown significant limitations. 
Commercial devices like Fitbit, Whoop, and the Oura Ring, as well as research prototypes like SleepCoach~\cite{daskalova2016sleepcoacher}, provide detailed insights into sleep patterns by correlating sleep factors with sleep quality but often offer generic recommendations that rely on predefined static templates. These templates are not flexible enough to adapt to users' evolving preferences, schedules, or environmental factors. 
SleepGuru~\cite{lee2022sleepguru} incorporates users' calendars to predict sleep pressure and recommends sleep schedules accordingly. 
However, it falls short in addressing crucial real-life variables that extend beyond calendar events, such as physical activity or changes in personal habits.

Conversational chatbots have emerged as a promising interaction paradigm for eliciting user personal contexts and preferences and delivering personalized health interventions~\cite{bickmore2005establishing, fitzpatrick2017delivering}.
However, many health-focused chatbots reply on template-based or rule-based methods, limiting their conversational flexibility and their ability to provide dynamic personalization~\cite{laranjo2018conversational}. 
This rigidity can lead to repetitive or irrelevant suggestions, potentially reducing engagement and motivation for behavior change over time.

Recent advances in large language models (LLMs) have shown great potential for enabling more flexible, context-aware conversations.
However, most existing LLM applications in health are still in their early stages. 
They struggle to natively support processing and interpreting raw and complex time-series sensor data, as well as translating data insights into actionable plans tailored to users' personal contexts. This gap hinders their effectiveness in driving sustained behavior change in a structured and systematic manner.

To address these identified gaps, we present \name{}, a novel LLM-powered chatbot augmented by a multi-agent framework to provide personalized, data-driven, and theory-guided sleep health support. 
\name{} seamlessly integrates wearable device data, contextual information, and established behavior change theories to deliver adaptive recommendations and motivational support. 
To tackle the challenge of delivering context-aware, adaptive interventions, 
we first build a contextual multi-armed bandit (MAB) model that dynamically learns and suggests sleep-enhancing activities by considering real-time environment factors (\eg, time and weather) and individual physiological sleep data collected from wearable devices.
Informed by behavioral theories~\cite{bandura1977self,lally2010habits}, this model balances exploiting previously successful sleep-enhancing activities and exploring new, potentially beneficial options. This ensures that recommendations remain relevant and adapt to users' evolving contexts and preferences.
To enhance real-time adaptability and intervention effectiveness, we leverage LLMs that incorporate behavior change techniques to deliver the activity recommendations in a natural and motivational conversation.
Combining the strengths of LLMs and the contextual MAB model while overcoming their limitations, we design a multi-agent framework where specialized agents coordinate LLMs' conversational and reasoning capabilities with context-aware dynamic recommendations. 
These agents work collaboratively to interpret and integrate wearable data insights with context information and deliver adaptive recommendations tailored to users' unique circumstances in a theory-guided conversation.
To evaluate \name{}, we conduct an eight-week in-the-wild deployment study with 16 participants, comparing our system to a baseline chatbot without context-aware recommendations or theory-guided conversations.
Our study adopts a within-subjects design.
The results show significant improvements in sleep metrics (\eg, longer sleep duration) and better activity scores, when using \name{} compared to the baseline. Additionally, participants report higher levels of motivation to engage in sleep-enhancing behaviors, attributed to the personalized and context-aware recommendations provided by the system.


The major contributions of this work are:
\begin{itemize}
    \item We introduce a novel LLM-powered chatbot with a multi-agent framework that integrates wearable data analysis, contextual factors, and a multi-armed bandit model to deliver personalized, timely, and adaptive sleep recommendations guided by behavior change theories.
    \item We conduct a eight-week in-the-wild deployment study to evaluate the effectiveness and usability of \name{} compared to a baseline, showing improvements through sleep metrics and user feedback.
    \item We provide design implications for future personalized health system designs and LLM applications in health applications. 
\end{itemize}


