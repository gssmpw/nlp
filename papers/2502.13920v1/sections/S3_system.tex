\section{HealthGuru}\label{sec.bg}
\name{} is a novel personalized health support chatbot designed to promote behavior change for better sleep health. By integrating wearable data, contextual information, and established behavior change theories, it offers data-driven, theory-guided support through an LLM-based multi-agent conversational interface.


\subsection{Data-Driven and Theory-Guided Health Support}

In designing \name{}, we focused on four key requirements:

\begin{itemize}
     \item \textbf{Provide quantitative health data analytics.} 
     Wearable devices facilitate health monitoring by providing rich, longitudinal data streams~\cite{piwek2016rise}. \name{} integrates relevant wearable data to provide a holistic view of users' health status, offering tangible metrics to track sleep and activities while providing nuanced, data-driven health advice.
 
    \item \textbf{Adapt to personal contexts and preferences.} 
    Health behaviors are deeply influenced by environmental and personal contexts~\cite{sallis2015ecological}.
    Static recommendations often fail to capture the dynamic nature of users' lives~\cite{yardley2016understanding}.
    Guided by just-in-time adaptive intervention (JITAI) principles, our system delivers personalized and timely support by considering users' realtime internal states (\eg, physiological metrics) and external contexts (\eg, environment)~\cite{nahum2018just}.
    We employ a contextual multi-armed bandit approach to dynamically adapt recommendations to users' evolving health states (\eg, physiological metrics) and external contexts (\eg, time, weather).
    This ensures the relevance and effectiveness of health interventions in changing environments. 

    \item \textbf{Support theory-informed motivational practice.} 
    While providing health information and knowledge is essential, it is often insufficient on its own to drive sustained behavior change~\cite{kelly2016changing}.
    Established behavioral change theories~\cite{michie2013behavior, atkins2017guide} offer structured approaches to understanding and influencing health behaviors.
    Therefore, we ground our \jianben{system-generated} advice on the theoretical techniques and frameworks (\eg, positive reinforcement and goal-setting) to motivate users in their health improvement journey.

    \item \textbf{Enable natural conversational interaction.} 
    The mode of interaction significantly impacts user engagement and the effectiveness of digital health interventions~\cite{bickmore2005establishing}. 
    Conversational interactions have been shown to provide human-like interaction, and increase user satisfaction, trust, and adherence in health applications~\cite{laranjo2018conversational, kocaballi2019personalization}. Through an LLM-based chatbot, \name{} facilitates nature and engaging interactions that can deliver JITAI to address users' immediate sleep health concerns and improve user receptivity to interventions according to user feedback.
   
\end{itemize}

\subsection{System and Data}\label{sec.system_data}
\name{} is built on a multi-agent architecture (\autoref{fig:system_framework}) that processes user inputs, integrates contextual data, and generates personalized, theory-guided responses to promote behavior change and improve sleep health. 
This section outlines the system workflow and the data types used for personalized feedback.


\subsubsection{System Workflow}
The system processes user messages and chat history by engaging specialized agents that consider behavior change theory, wearable data, context data, and activity recommendations.
When a user sends a message (\eg, ``What do you recommend me to do?''), the agent coordinator selects the appropriate agents. For example, the behavior change agent identifies motivational techniques, while the recommendation agent uses current context (\eg, sunny weather at 81°F) to suggest suitable activities. 
The response agent then compiles these inputs into a concise, coherent reply (as shown in \autoref{fig:system_framework}, where a late afternoon run is suggested to avoid peak heat, with benefits explained).


\begin{figure*}
  \includegraphics[width=\textwidth]{./figures/chatbot_framework.pdf}
  \caption{Overview of \name{} system. The system framework (left) features a multi-agent architecture, incorporating behavior change techniques, wearable data analysis, and context-aware recommendations. 
  The user interface (right) engages users in multi-turn conversations to gain insights into their sleep and activity data and receive personalized advice.}
  \label{fig:system_framework}
\end{figure*}

\subsubsection{Data Collection \& Preparation}
As summarized in \autoref{tab:data_collection}, we collect three primary types of wearable data for sleep health analysis and activity recommendation: physiological, sleep, and activity data.
The selection of these data types is based on their direct relevance to both sleep quality and physical activity and their widespread availability in existing commercial wearable devices (\eg, Oura ring, Fitbit, and Whoop).
Moreover, the data selection is supported by extensive research in sleep science~\cite{shaffer2017overview, ohayon2017national, kredlow2015effects, peake2018critical} and human-computer interaction~\cite{liang2016sleep}.
% + more reasons why we choose these data types, not others...
Physiological data (\eg, HRV)~\cite{shaffer2017overview}
offer quantitative insights into sleep quality and activity intensity.
Sleep data details sleep patterns, efficiency, and quality~\cite{ohayon2017national}.
Activity data captures the type, intensity, and duration of physical activities~\cite{kredlow2015effects}.
We also include context data for generating activity recommendations (in \autoref{subsec.act_rec}) and behavior change theory components (in \autoref{sec.chatbot}) for formulating responses.


\begin{table}[t]
\caption{Collected wearable data types and definitions.}
\label{tab:data_collection}
\small
\begin{tabular}{@{}p{0.3\columnwidth}p{0.65\columnwidth}@{}}
\toprule
\textbf{Category --- Measurement} & \textbf{Definition} \\
\midrule
\multicolumn{2}{@{}l@{}}{\textit{Physiological Data}} \\
Average Heart Rate  & Mean heart rate (in bpm) measured during sleep or activity to assess cardiovascular health. \\
Lowest Heart Rate   & The lowest heart rate (in bpm) recorded during sleep—often occurring in deep sleep stages. \\
Average HRV         & Mean heart rate variability (in ms) during sleep; higher values (e.g., 50–100 ms) suggest better stress resilience. \\
Stress Level        & A computed metric based on physiological markers (such as HRV), typically scaled (e.g., 1–100). \\
\midrule
\multicolumn{2}{@{}l@{}}{\textit{Sleep Data}} \\
Bedtime Start \& End & Timestamps indicating when sleep begins and ends (e.g., 2024-07-26T00:26:28-04:00). \\
Day                 & The date corresponding to the sleep data (e.g., 2024-07-26). \\
Sleep Efficiency    & Ratio of sleep time to time in bed, expressed as a percentage (1–100). \\
Readiness Score     & A daily readiness score (1–100) derived from sleep quality and physiological measures. \\
Time in Bed         & Total time spent in bed (in seconds), including periods of wakefulness. \\
Total Sleep Duration& Actual sleep time excluding wake periods (in seconds). \\
Sleep Score         & Overall sleep quality score (1–100) based on duration, efficiency, and physiological data. \\
\midrule
\multicolumn{2}{@{}l@{}}{\textit{Activity Data}} \\
Activity Type       & Type of physical activity (e.g., Running, Walking, Cycling). \\
Intensity           & Activity intensity level (e.g., easy, moderate, hard). \\
Day                 & The date on which the activity was recorded. \\
Start \& End Time   & Timestamps marking the beginning and end of the activity period. \\
\bottomrule
\end{tabular}
\end{table}

\subsection{Physical Activity Recommendations}
\label{subsec.act_rec}
We present a novel contextual multi-arm bandit (MAB) model (\autoref{fig:context_mab}) to recommend personalized activities to improve health outcomes.
This approach emphasizes adaptability to user behavior and environmental context factors while maintaining a balance between familiar and novel experiences to motivate behavior change.

\begin{figure*}
  \includegraphics[width=\textwidth]{./figures/contextual_MAB_v2.pdf}
  \caption{Contextual MAB algorithm workflow. The algorithm generates activity recommendations by balancing the exploration and exploitation benefits of activities based on the current context and the corresponding activity history and sleep scores. After the recommendation, the sleep quality for the recommended activities will provide feedback to update the model.}
  \label{fig:context_mab}
\end{figure*}


\textbf{Context consideration.}
We consider the time of day, temperature, and weather \cite{hardeman2019systematic, klenk2012walking, herbolsheimer2016physical, giles2002relative}.
These factors are crucial in determining the activity feasibility and effectiveness. 
For example, the time of day affects user schedules and energy levels. 
Temperature impacts the comfort of outdoor activities.

\textbf{Problem formulation.}
Framed as an optimization task, the system aims to maximize sleep quality scores (measured by devices like the Oura ring) by recommending activities (\eg, Gym, Walking, Yoga, Reading, Meditation) based on current contexts.

\textbf{Motiviation of contextual multi-arm bandit model}
for activity recommendation is guided by both theoretical and practical considerations.
Self-efficacy theory \cite{bandura1977self} suggests users are more likely to engage in activities they believe they can successfully perform based on their past experience, while habit formation theory \cite{lally2010habits} emphasizes the importance of consistent positive outcomes.
Practically, the system must address the ``cold start'' problem of limited initial user data and adapt to changing user preferences and environmental context over time.
It also needs to balance between recommending known effective and introducing novel options to prevent user boredom and discover new beneficial behaviors.

To address these requirements, we adopt a contextual multi-arm bandit (MAB) model to guide our activity recommendations. In this framework, each ``arm'' represents a physical activity and the ``reward'' is the corresponding sleep quality, while additional context (\eg, time, location, weather) informs decision-making. 
The algorithm balances exploitation (selecting proven activities) with exploration (trying new or less certain ones), aligning with behavioral theories by recommending feasible, context-appropriate options and fostering habit formation through novel experiences. 
The model's adaptive learning capability allows it to respond to changing user preferences and environmental conditions, while also addressing the cold start problem by quickly learning from limited initial data.

\textbf{Implementation of contextual MAB}.
We implement the Linear Upper Confidence Bound (LinUCB) algorithm~\cite{li2010contextual}
for its ability to handle context information and enable online learning with strong theoretical performance.
The core idea behind LinUCB is to model the expected reward of each action as a linear function of the contextual features and incorporate an uncertainty term to balance exploration and exploitation. 
Considering both estimated rewards and confidence in the estimates, the model selects the actions that have the highest potential for reward.
The implementation of LinUCB is described as follows:
\begin{itemize}
    \item \textbf{Context vector construction:} For each recommendation at time $t$, we create a context vector $x_t \in \mathbb{R}^d$, where $d$ is the dimensionality of the context features. The context vector incorporates relevant factors, including time of day, temperature, and weather conditions. Time of day is discretized into seven intervals based on hours: `0-6', `6-9', `9-12', `12-15', `15-18', `18-21', and `21-24'.
    Temperature values are converted into `cold', `mild', `warm', and `hot'.
    Weather is categorized into `sunny',
    `rain', `clear', `windy', and `snow'.
    These features are one-hot encoded and concatenated to form the context vector.
    \item \textbf{Action set initialization:} For each user, we define a set $\mathcal{A}$ of possible sleep-enhancing activities (\eg, evening walk, meditation, reading).

    \item \textbf{Decision rule:} At each time step $t$, for a given context $x_t$, LinUCB selects the action $a_t$ that maximizes:
    \[
    a_t = \arg\max_{a \in \mathcal{A}} \left( \hat{\theta}_a^\top \mathbf{x}_t + \alpha \sqrt{\mathbf{x}_t^\top \mathbf{A}_a^{-1} \mathbf{x}_t} \right),
    \]
    where:
    \begin{itemize}
        \item The term $\hat{\theta}_a^\top \mathbf{x}_t$ represents the estimated expected reward for action $a$ given context $\mathbf{x}_t$. $\hat{\theta}_a \in \mathbb{R}^d$ is the estimated parameter vector for action $a$, representing the learned relationship between the context and the reward (sleep quality) for that action.
        
        \item The term $\sqrt{\mathbf{x}_t^\top \mathbf{A}_a^{-1} \mathbf{x}_t}$ represents the uncertainty (standard deviation) in the estimate, forming the upper confidence bound. $\mathbf{A}_a \in \mathbb{R}^{d \times d}$ is the covariance matrix (or precision matrix) for action $a$, initialized as a $d \times d$ identity matrix and updated over time. It records the certainty of our estimates.
        
        \item $\alpha$ (>0) is a positive scalar that controls the trade-off between exploration and exploitation. A higher $\alpha$ encourages more exploration.
        
    \end{itemize}
    
    \item \textbf{Action execution:} The system recommends the selected activity $a_t$ to the user.
    
    \item \textbf{Reward observation:} After recommendation, the system observes a reward $r_t$
    (\eg, a normalized sleep score between 0 and 1).
    
    \item \textbf{Update procedure:} We then update the parameters for the suggested action $a_t$ based on the observed reward $r_t$:
    \begin{enumerate}
        \item Update the covariance matrix by incorporating context information at time $t$ to refine the certainty in parameter estimates:
        $\mathbf{A}_{a_t} = \mathbf{A}_{a_t} + \mathbf{x}_t \mathbf{x}_t^\top$.
        
        \item Update the accumulated reward vector for action $a_t$:
        $\mathbf{b}_{a_t} = \mathbf{b}_{a_t} + r_t \mathbf{x}_t$.
        
        \item Recompute the parameter vector for action $a_t$:
        $\hat{\theta}_{a_t} = \mathbf{A}_{a_t}^{-1} \mathbf{b}_{a_t}$.
    \end{enumerate}    
\end{itemize}


\subsection{Sleep Health Chatbot}\label{sec.chatbot}
While quantitative data from users (\eg, sleep, activity, context) is valuable for promoting sleep health, it often overlooks nuanced personal factors such as lifestyle preferences and individual barriers. 
To bridge this limitation, we develop an LLM-powered chatbot that enables natural, flexible interactions. 
By integrating quantitative evidence with behavior change frameworks, the chatbot delivers data-driven and theory-informed advice tailored to users' unique circumstances.
This integrated approach aims to create a positive feedback loop where recommended behavior changes lead to improved sleep outcomes.

\subsubsection{Build Theoretical Framework for Conversation}\label{subsec.chatbot_framework}
We incorporate widely recognized behavior change theories
to inform the chatbot's conversational flow and recommendation strategies.
% into the conversation flow. 
The use of these theories provides a structured and solid foundation for identifying the determinants of behavior change and selecting appropriate strategies to address them.
They can guide LLMs to generate more relevant and reliable responses.

\textbf{Selection of behavior change theories.}
Following \citet{moullin2020ten}'s recommendations, our framework selection is guided by four key criteria:
1) purpose alignment with behavior change and implementation;
2) individual-level target capability;
3) capability to address implementation elements (\eg, barriers and strategies);
and 4) contextual fit with individual lifestyles and chatbot interventions.
After evaluating several frameworks, 
we choose Theoretical Domains Framework (TDF)~\cite{atkins2017guide} and the Behavior Change Technique (BCT) Taxonomy~\cite{michie2013behavior} because of their comprehensive coverage of behavior determinants and actionable techniques, and their suitability for individual-level interventions.
Specifically,
TDF offers 14 domains for identifying behavior change factors,
such as knowledge, skills, belief about capabilities, and environmental context.
BCT Taxonomy provides 93 specific techniques organized into 14 categories (\eg, goals and planning, feedback and monitoring) for implementing behavior change interventions.
It offers practical methods for translating theoretical understanding into actionable interventions.
This combination allows us to identify behavioral determinants and address them through actionable strategies.

We have also considered alternative frameworks. 
Behavior Change Wheel (BCW)~\cite{michie2011behaviour} combines the COM-B model (Capability, Opportunity, Motivation—Behavior) with intervention functions and policy categories to design comprehensive behavior change interventions.
It emphasizes policy enablers and broader intervention functions that are less directly applicable to our context.
Similarly, we rule out Consolidated Framework for Implementation Research (CFIR)~\cite{damschroder2009fostering} since it highlights contextual factors such as organizational culture and policies.
The Transtheoretical Model (TTM)~\cite{prochaska1997transtheoretical} describes behavior change as a process through six stages, from precomtemplation to termination.
It focuses more on tailoring interventions for transitions of change over extended periods, which is less aligned with our immediate, adaptive interventions.

\textbf{Consolidating theoretical frameworks.}
We first identify TDF domains that are most relevant for sleep behavior.
Some domains like Social Role and Identity and Social Influences are excluded due to the challenges of simulating users' social connections in a chatbot context.
Then, we select BCT categories that align with the identified TDF domains and are practical for a chatbot to deliver.
Finally, we consolidate the concepts derived from TDF and BCT into seven key techniques to inform our chatbot design. 
For example, the ``intention'' and ``goals''
domains from the TDF are merged as one category because they both represent a commitment to achieve a desired outcome. 
The TDF domain ``Behavior Regulation'' is merged with BCT ``Feedback and Monitoring'', both involving objective tracking and management of observed behaviors.
BCT ``Social Support'' category is scoped to TDF ``Emotion'' domain.
The final framework is detailed in \autoref{tab:sleep-techniques}.

\subsubsection{Implementing Data-Driven and Theory-Guided Chatbot}\label{subsec.chatbot_implementation}
We implement a multi-agent chatbot where each agent, powered by LLMs, performs specialized analytical tasks to generate comprehensive, personalized responses. 
They coordinate with each other by custom prompts and rules.

First, \textbf{Agent Coordinator} analyzes incoming messages and conversation history to determine which analytical tasks are needed. It then directs requests to corresponding specialized agents based on classification prompts.
The agents then process these requests:


\begin{itemize}
    \item \textbf{Technique Extraction Agent:} 
    This agent identifies relevant behavior change strategies (see \autoref{tab:sleep-techniques}) for generating responses.
    It uses tailored prompts that fuse users' current messages, conversation history, and behavior change theories to instruct LLM selection of response strategies.

    \item \textbf{Personal Data Agent:} This agent retrieves and analyzes the user's personal data (\autoref{tab:data_collection}), such as sleep patterns and activity levels, that are relevant to the user's messages.
    Given the user's chat messages and daily wearable data (organized as a dataframe) as input, the agent leverages the LLM's capabilities to generate Python code for data retrieval, transform, and analysis.
    Based on the code results and user message, another LLM reports the insights (\eg, average sleep duration over the past week).\footnote{If there are data and processing errors, the system will respond ``I am sorry, I am not able to provide the information at the moment.''}

    \item \textbf{Recommendation Agent} is responsible for delivering context-aware personalized activity suggestions.
    Given the current users' context data (\eg, weather, time of day), our contextual MAB model suggests sleep-enhancing activities (\eg, run, walk). Then, to improve relevance and actionability, the agent prompts LLMs to adjust and tailor the model-generated activities based on the user's current contexts, locations, and situations implied in the messages. For example, it may suggest indoor treadmill running when it is raining outside or walking in the nearby park.
   
\end{itemize}

Finally, the \textbf{Response Agent} synthesizes outputs from all agents into cohesive replies that integrate data insights, personalized recommendations, and behavior change techniques. 
This modular approach decomposes complex chatbot's conversational and analytical reasoning tasks into specialized agents, ensuring responses are relevant, theory-grounded, and adaptive to user's specific circumstances.


\begin{table*}[h]
\centering
\caption{Seven core techniques identified for sleep health improvement.}
\label{tab:sleep-techniques}
\resizebox{.95\textwidth}{!}{
\begin{tabular}{|p{0.2\textwidth}|p{0.4\textwidth}|p{0.4\textwidth}|}
\hline
\textbf{Technique Domain} & \textbf{Definition} & \textbf{Example} \\
\hline
1. Consequences and Reinforcement & Discussing anticipated outcomes of sleep behaviors and providing feedback on user's actions & ``Based on your Oura ring data, your consistent 10 PM bedtime has led to a 15\% increase in your deep sleep. This improvement can enhance your memory and cognitive function.'' \\
\hline
2. Feedback and Monitoring & Tracking sleep patterns and providing users with insights into their progress & ``I notice you've been using your Oura ring consistently. Let's review your sleep efficiency over the past week and identify areas for improvement.'' \\
\hline
3. Goals & Setting clear, achievable sleep-related objectives tailored to the user's current habits and desired outcomes & ``Given your current average of 6 hours of sleep, shall we set a goal to gradually increase this to 7 hours over the next month?'' \\
\hline
4. Knowledge & Providing users with tailored information about sleep health, addressing gaps in their understanding & ``Did you know that exposure to blue light from devices before bedtime can disrupt your melatonin production? Let's discuss some strategies to minimize this effect.'' \\
\hline
5. Environmental Context and Resources & Addressing the user's physical sleep environment and available resources to optimize sleep conditions & ``I see you live in a noisy urban area. Have you considered using a white noise machine to mask disruptive sounds during the night?'' \\
\hline
6. Skills and Capabilities & Teaching users specific techniques to improve their sleep and building their confidence in implementing these strategies & ``Let's practice a simple breathing exercise that can help you relax before bed. Inhale for 4 counts, hold for 7, and exhale for 8. How does that feel?'' \\
\hline
7. Emotional Support & Providing empathy, encouragement, and motivation to support users' sleep improvement efforts & ``I understand that changing sleep habits can be challenging. Remember, every small step you take is progress. How can I help you feel more confident about making these changes?'' \\
\hline
\end{tabular}
}
\end{table*}


\section{System Implementation}\label{sec.sys_implementation}
\name{} is implemented as an interactive web-based application using Streamlit, a Python framework to build data-driven web apps.
Upon user content, the system accesses the user's location data via the browser, which is then used to retrieve local time, weather conditions, and temperature information through the WeatherAPI\footnote{\url{https://www.weatherapi.com/}}. 
This contextual information is used to build the personalized activity recommendation model.
Meanwhile, users' wearable data, including activity, sleep, and physiological data (\eg, heart rate), is accessed through Oura API\footnote{\url{https://cloud.ouraring.com/v2/docs}}.
The chatbot's core functionality is powered by OpenAI's language models. 
GPT-4o-mini is employed for managing general conversation flow, theoretical framework selection, and recommendation model integration and results refinement, ensuring a balance between response quality and speed. 
For the more complex task of retrieving and analyzing wearable data, we leverage GPT-4o to use the Python tool to generate analysis codes and results.
Then, we build the LLM-based multi-agent framework and chatbot by using Python LangChain Library\footnote{\url{https://github.com/langchain-ai/langchain}}. LangChain provides a suite of tools for constructing prompts, managing conversations, and orchestrating the flow between agents and LLMs via prompt chaining. It allows us to modularize the chatbot's functionalities into specialized agents that interact seamlessly.
% to build prompts, manage conversation history, and orchestrate the flow between agents.}
To enhance user experience, the system implements streamed output, reducing perceived latency and allowing for real-time interaction.
The application is hosted on Amazon Elastic Compute Cloud (EC2), providing scalable computing capacity. 

