\section{Related Work}
\label{pepfliteraturereview}
EPF has emerged as a critical research area due to the increasing complexity and volatility of electricity markets. Accurate forecasting is essential for market participants to make informed decisions, mitigate financial risks, and optimise operations. Probabilistic forecasting methods, in particular, address the inherent uncertainties in these markets by providing PIs rather than single-point estimates, allowing stakeholders to assess the range of potential outcomes. Table \ref{TableLR} provides a comprehensive summary of the literature on probabilistic EPF methods, highlighting the evolution from traditional point forecasts to advanced probabilistic techniques, including QR and CP. This section critically examines the role of forecasting in electricity spot markets, foundational concepts of uncertainty in EPF, recent advancements in probabilistic methods, and the growing role of CP as an alternative framework, while evaluating their strengths, limitations, and practical implications.

\subsection{Electricity Spot Markets}
%Day-Ahead Market
The DAM is a cornerstone of electricity trading, where market participants submit bids for electricity trades up to one day in advance, with the bidding process closing in the afternoon to determine market-clearing prices and volumes based on supply, demand, and grid constraints \cite{newbery2016benefits, ilea2018european}. 
While the DAM enhances cross-border electricity trading and grid stability, it faces challenges from forecast uncertainties driven by the increasing penetration of intermittent renewable energy sources such as wind and solar \cite{miraki2025probabilistic}. These challenges necessitate advanced forecasting models capable of managing the variability introduced by weather-dependent renewables, such as privacy-preserving frameworks that enhance accuracy by integrating diverse data sources \cite{liu2025probabilistic}. Initially dominated by traditional statistical methods, forecasting has shifted toward advanced machine learning (ML) techniques, which excel at capturing the complex, non-linear relationships inherent in electricity markets \cite{ugurlu2018electricity, lago2018forecasting, chen2019brim, li2021day}. Hybrid approaches combining ML with other techniques have further improved precision, particularly in modelling supply-demand dynamics and adapting to varying market conditions \cite{lago2021forecasting, elrobrini2024federated}.

%Intra-Day Market
The IDM is crucial for real-time electricity trading, bridging the DAM and BM by allowing continuous trading up to a few hours before delivery. This enables participants to adjust their positions in response to unforeseen events, such as plant outages or fluctuations in renewable energy output \cite{shinde2019literature}. In contrast to the DAM's hourly settlement intervals, the IDM allows trades in 30-minute blocks, notably in markets like the Irish Single Electricity Market.The IDM's real-time nature necessitates advanced forecasting models capable of accurately predicting short-term price movements. These challenges necessitate advanced forecasting models capable of managing the variability introduced by weather-dependent renewables \cite{piantadosi2024photovoltaic}, with explainable frameworks helping ensure adaptability and reliability in dynamic conditions \cite{samarajeewa2024artificial}. Traditional statistical methods \cite{uniejewski2019understanding, narajewski2020econometric} have shown success in processing large datasets and adapting to the market's dynamic conditions. Additionally, in markets with a high penetration of renewable energy, the IDM optimises trading strategies by enabling participants to refine their actions in near real-time, thereby mitigating risks associated with price volatility \cite{koch2019short, birkeland2024research}. Although the BM shares several structural similarities with the IDM—particularly in their real-time nature and focus on short-term trading—our work centres primarily on the BM due to its finer granularity and its role in balancing supply and demand in real-time.

%Balancing Market
Despite the importance of the BM in maintaining equilibrium between supply and demand, the literature on BM forecasting is relatively underdeveloped compared to that of the DAM and IDM. The BM plays a key role in balancing supply and demand with finer granularity than the IDM, particularly when responding to the fluctuations associated with renewable energy \cite{ortner2019future}. Accurate BM price forecasts are essential, especially where large-scale energy storage is not viable \cite{MAZZI2017259}. Studies have compared various forecasting models and their performance in BM settings, typically with limited horizons. Probabilistic forecasting methods for BM prices, such as those proposed by \cite{dumas2019probabilistic}, show promise but suggest further enhancements and integration into bidding strategies. Notably, existing research often lacks critical analysis and broader context. Recent studies, including \cite{narajewski2022probabilistic} and \cite{OCONNOR2024101436}, explore diverse models for BM price forecasting, with simpler models often outperforming the more complex ML and DL models. These studies highlight the need for continued research to address current limitations and explore emerging methodologies, ensuring sharp and reliable BM price predictions. Given the challenges of renewable energy integration and market volatility, European electricity market reforms increasingly prioritise resilient mechanisms \cite{zachmann2023design}.





\begin{table}[]
\begin{tabular}{p{4.3cm} p{1.0cm} p{1.5cm} p{1.0cm} p{1.4cm} p{1.5cm}}
% \begin{tabular}{lllllll}
\textbf{Reference} & \textbf{Point} & \textbf{Proba-bilistic} & \textbf{CP} & \textbf{Day-Ahead} &  \textbf{Real-Time}   \\

\cite{nowotarski2015computing}     &        &    \checkmark   &       &    \checkmark   &       \\
\cite{hagfors2016modeling}       &        &  \checkmark      &       &       &   \checkmark     \\
\cite{maciejowska2016probabilistic}&        &    \checkmark    &       &     \checkmark   &      \\
\cite{hagfors2016modeling} &        &    \checkmark   &       &       &   \checkmark    \\

\cite{lago2018forecasting}          &   \checkmark     &       &       &    \checkmark   &      \\
\cite{nowotarski2018recent}     &        &   \checkmark    &       &    \checkmark   &        \\
\cite{koch2019short}             &        &       &       &       &  \checkmark      \\
\cite{shinde2019literature}      &        &       &       &       &  \checkmark    \\
% \cite{kostrzewski2019probabilistic} &        &   \checkmark    &       &   \checkmark    &   \\
\cite{cheng2019hybrid}             &  \checkmark &       &       &  \checkmark &        \\
\cite{dumas2019probabilistic}      &        &    \checkmark   &       &       &    \checkmark     \\
\cite{uniejewski2019understanding} &    \checkmark    &       &       &       &   \checkmark    \\

% \cite{toubeau2020capturing}    &        &   \checkmark    &       &    \checkmark   &     \\
\cite{narajewski2020econometric}     &        &   \checkmark    &       &       &   \checkmark     \\
\cite{lucas2020price}               &   \checkmark     &       &       &       &   \checkmark     \\
\cite{marcjasz2020probabilistic}     &        &   \checkmark    &       &   \checkmark    &      \\
\cite{lago2021forecasting}       &    \checkmark    &       &       &    \checkmark   &       \\
\cite{uniejewski2021regularized}     &        & \checkmark &       & \checkmark &            \\
\cite{kath2021conformal}             &        & \checkmark & \checkmark  &  \checkmark   & \checkmark   \\
% \cite{wen2021probabilistic}   &        &   \checkmark    &       &  \checkmark    &      \\
\cite{narajewski2022probabilistic}   &        &   \checkmark    &       &       &    \checkmark   \\
\cite{khajeh2022applications}        &        &    \checkmark   &       &       &       \\
\cite{jensen2022ensemble}          &        &  \checkmark     &   \checkmark    &       &      \\
\cite{hu2022conformalized}      &        &    \checkmark    &    \checkmark    &       &      \\
\cite{zaffran2022adaptive}      &        &    \checkmark   &   \checkmark    &       &     \\

\cite{uniejewski2023smoothing}       &        & \checkmark &       &       &      \\
\cite{marcjasz2023distributional}  &        &   \checkmark    &       &     \checkmark  &     \\
\cite{subhankar2023probabilistically} &        &  \checkmark     &  \checkmark     &       &      \\

% \cite{ryu2023quantile}    &        &   \checkmark    &       &   \checkmark    &      \\
% \cite{wang2023probabilistic} &        &   \checkmark    &       &  \checkmark    &      \\
\cite{dewolf2023valid}          &        &   \checkmark    &   \checkmark    &       &       \\
\cite{trebbien2023probabilistic}        &        &   \checkmark    &       &   \checkmark   &      \\
\cite{cordier2023flexible}      &        &   \checkmark    &   \checkmark    &       &      \\
% \cite{OCONNOR2024101436}             &    \checkmark     &       &       &     \checkmark  &   \checkmark      \\
% \cite{monjazeb2024wholesale}       &        &    \checkmark   &       &   \checkmark    &      \\

% \cite{de2024electricity}        &        &   \checkmark    &   \checkmark    &    \checkmark   &      \\
% \cite{lee2024transformer}       &        &    \checkmark    &     \checkmark   &       &       \\
% \cite{xu2024conformal}          &        &     \checkmark  &   \checkmark    &       &       \\
% \cite{angelopoulos2024conformal}&        &   \checkmark     \\
% \textbf{Our approach} &       &   \checkmark    &   \checkmark    &   \checkmark    &    \checkmark      \\
\end{tabular}
% \caption{Literature on PEPF. The \checkmark is the approach presented herein}
% \label{TableLR}
\end{table}



\begin{table}[]
\begin{tabular}{p{4.0cm} p{1.0cm} p{1.4cm} p{1.0cm} p{1.5cm} p{1.5cm}}
% \begin{tabular}{lllllll}
\textbf{Reference} & \textbf{Point} & \textbf{Probab-ilistic} & \textbf{CP} & \textbf{Day-Ahead} &  \textbf{Real-Time}   \\
% \cite{dewolf2023valid}          &        &   \checkmark    &   \checkmark    &       &       \\
% \cite{trebbien2023probabilistic}        &        &   \checkmark    &       &   \checkmark   &      \\
% \cite{cordier2023flexible}      &        &   \checkmark    &   \checkmark    &       &      \\
\cite{OCONNOR2024101436}             &    \checkmark     &       &       &     \checkmark  &   \checkmark      \\
\cite{monjazeb2024wholesale}       &        &    \checkmark   &       &   \checkmark    &      \\

\cite{de2024electricity}        &        &   \checkmark    &   \checkmark    &    \checkmark   &      \\
\cite{lee2024transformer}       &        &    \checkmark    &     \checkmark   &       &       \\
\cite{xu2024conformal}          &        &     \checkmark  &   \checkmark    &       &       \\
\cite{angelopoulos2024conformal}&        &   \checkmark     \\
\textbf{Our approach} &       &   \checkmark    &   \checkmark    &   \checkmark    &    \checkmark      \\
\end{tabular}
\caption{Literature on PEPF. The \checkmark is the approach presented herein}
\label{TableLR}
\end{table}


\subsection{Probabilistic Methods}
Probabilistic forecasting methods have evolved substantially to meet the need for robust PIs in volatile electricity markets. Techniques such as Bayesian approaches, bootstrap methods, distribution-based forecasts, Monte Carlo simulations, and historical simulations each bring unique strengths but also face practical limitations. Bayesian methods effectively quantify uncertainty by combining prior knowledge with observed data, as seen in advancements like Bayesian neural networks (BNNs), stochastic volatility models, and hybrid Bayesian optimisation \cite{vahidinasab2010bayesian, kostrzewski2019probabilistic, cheng2019hybrid}. However, their high computational demands and reliance on strong priors limit scalability, particularly in non-stationary conditions. Bootstrap methods, such as residual-based bootstrapping, improve empirical coverage in BM settings but often produce overly wide intervals during volatile periods \cite{uniejewski2019importance, khosravi2013quantifying}. Distribution-based forecasts, including copula processes and generalised additive models, capture extreme price behaviours but are constrained by their reliance on correct distributional assumptions \cite{klein2023deep, serinaldi2011distributional, monteiro2018new}. Monte Carlo simulations, enhanced through Bayesian networks and sequential Markov Chain Monte Carlo techniques, improve real-time adaptability but remain computationally prohibitive for high-frequency trading \cite{yuanchen2023electricity}. Historical simulations offer simplicity and robustness but struggle to address structural shifts or unprecedented events, particularly in markets integrating renewable energy \cite{maciejowska2024multiple, janczura2024expectile}. Recent studies have explored the use of meta-heuristic optimization techniques for forecasting models in short-term load forecasting \cite{bhatnagar2025using}.

These limitations have driven interest in Conformal Prediction (CP) techniques, which offer a flexible, distribution-free framework for generating reliable PIs. CP avoids the restrictive assumptions of traditional methods while ensuring predefined confidence levels, even in non-stationary markets \cite{gammerman1998learning, vovk2005algorithmic}. Recent advancements, such as Ensemble Batch Prediction Intervals (EnbPI) and Sequential Predictive Conformal Inference (SPCI), extend CP’s utility to time-series applications \cite{xu2023sequential}. Despite these strengths, CP’s reliance on large calibration datasets and sensitivity to structural breaks can undermine performance in dynamic electricity markets \cite{foygel2022conformal}. Further research should focus on enhancing CP’s adaptability and robustness, particularly for handling extreme price events and evolving market complexities.



\subsubsection*{Quantile Regression \& Conformal Prediction}
QR and QRA are widely used in PEPF for their simplicity, efficiency, and ability to capture uncertainty across quantiles. Recent advancements have refined these techniques, such as Lasso QRA and smoothed QRA with kernel estimation, which improve predictive performance \cite{uniejewski2021regularized, uniejewski2023smoothing}. DL integrations have also shown promise, with methods like QR combined with LSTNet and SHAP-based feature selection enhancing sharpness through kernel density estimation \cite{liu2023day}, or hybrid QR-LSTM frameworks optimised via multi-objective algorithms improving accuracy under diverse market conditions \cite{xu2024novel}. While effective, QR and QRA face challenges in uncertainty quantification, particularly in achieving reliable coverage, motivating the exploration of CP as a complementary framework.

CP generates distribution-free PIs with predefined confidence levels, making it a robust alternative for uncertain and volatile markets \cite{gammerman1998learning, vovk2005algorithmic}. Extensions like Ensemble Batch Prediction Intervals (EnbPI), which combines ensemble learning with block bootstrap methods, have demonstrated success in addressing seasonality and trends in energy forecasting \cite{xu2021conformal, zaffran2022adaptive}. Similarly, Sequential Predictive Conformal Inference (SPCI) dynamically recalibrates PIs to improve reliability in non-stationary settings, with Transformer-based extensions further enhancing its adaptability \cite{lee2024transformer}. Hybrid approaches such as Ensemble Conformal QR and SCP with QR, offer a balance between computational efficiency and robust coverage, making them suitable for real-time applications \cite{jensen2022ensemble, cordier2023flexible}.  However, CP still faces practical limitations, including sensitivity to structural breaks and reliance on large calibration datasets \cite{foygel2022conformal}. Enhancing CP’s robustness, addressing extreme price events, and integrating CP with DL architectures to improve its accuracy and adaptability in dynamic electricity markets have proven beneficial \cite{angelopoulos2024conformal, xu2024conformal}.