\section{Related Works}
\label{sec:relatedworks}
\subsection{Ontological Knowledge graphs and Retrieval Augmented Generation}

Ontological Knowledge Graphs (OKGs) have been employed to enhance the interpretability and accuracy of LLMs. By structuring knowledge in a graph format, these systems can provide deeper insights into the relationships between concepts and data points. Buehler et al.____ demonstrated the utility of OKGs in materials science, particularly for interpreting and predicting material properties and behaviors. The integration of OKGs with LLMs enables the models to not only retrieve but also contextualize and relate information in a more meaningful way, significantly improving generative performance and reducing hallucinations.

Recent advancements have also explored the use of multi-agent systems, where multiple LLMs with specialized capabilities collaborate to solve complex problems. These systems leverage the strengths of individual agents, such as specific domain expertise or particular functionalities like code generation and execution. Buehler's work highlights the efficacy of multi-agent strategies in materials design and other engineering tasks. By using a combination of agents, each equipped with distinct roles and access to specialized data sources, these systems can perform intricate tasks ranging from hypothesis generation to simulation-based data analysis.

Buehler demonstrated the efficacy of RAG in enhancing the query-answering capabilities of LLMs on specialized topics not covered during their fine-tuning. By integrating contextually relevant information from external knowledge sources, RAG significantly reduces the likelihood of hallucinations and inaccuracies compared to raw LLM responses. This approach allows models to provide more precise and context-rich answers, particularly in fields like nanotechnology, where Buehler showcased improved performance and interpretability using Ontological Knowledge Graphs alongside RAG techniques. 

In the context of nanotechnology, the combination of LLMs and RAG, augmented by OKGs and multi-agent systems, presents a powerful approach to addressing complex research queries. Nanotechnology research often involves interdisciplinary knowledge spanning chemistry, physics, and materials science, necessitating sophisticated tools for information retrieval and integration. 
NANOGPT exemplifies the potential of these advanced AI techniques. Unlike Buehler's work, NANOGPT differentiates itself through the use of a sophisticated real-time retrieval system to provide accurate, contextually enriched responses to intricate queries. While NANOGPT does not currently utilize OKGs, incorporating OKGs into its framework represents an exciting avenue for future research, promising even greater advances in supporting nanotechnology discovery.

\subsection{Large Language Models in Chemistry}

The exploration of LLMs and generative artificial intelligence (GAI) in educational contexts, particularly in the domain of chemistry, has garnered significant attention in recent years. Yik et al.____ contributes to this growing body of literature by investigating the capabilities and limitations of ChatGPT-3.5 in explaining organic chemistry reaction mechanisms.

In chemistry education, the use of GAI tools has been examined primarily through their performance on chemistry assessments. Studies such as those by Leon and Vidhani(2023)____ and Clark(2023)____ evaluated ChatGPT's ability to answer general chemistry multiple-choice and open-response questions, revealing varied performance with notable challenges in complex problem-solving scenarios. For example, Leon and Vidhani reported that ChatGPT-3.5 achieved a mean score of 27\% on multiple-choice and free response general chemistry questions, while Clark found an accuracy of 47\% on non-numeric questions. These findings underscore the mixed efficacy of LLMs in accurately responding to academic queries.

The application of GAI in organic chemistry, however, has been less explored. Watts et al.____ conducted a study comparing student-generated and ChatGPT-generated responses to organic chemistry writing-to-learn assignments. Their analysis indicated that, while ChatGPT could engage in mechanistic reasoning, it often lacked the depth and accuracy found in student responses, particularly in describing electron movement, a crucial aspect of mechanistic reasoning. The study builds on this foundation by specifically examining ChatGPT's performance in explaining organic chemistry reaction mechanisms. It evaluates the accuracy and sophistication of the generated explanations, situating the findings within the frameworks of mechanistic reasoning and prompt engineering. The research highlights that while ChatGPT can produce highly sophisticated explanations, accuracy remains a challenge, with a significant portion of responses containing minor inaccuracies that can mislead learners.
NANOGPT, which integrates retrieval-augmented generation to draw directly from verified sources, can significantly reduce hallucinations and enhance the accuracy of generated explanations compared to standalone LLMs like ChatGPT.

\subsection{Natural Language Processing for Materials Science}

Recent advancements in NLP have opened new avenues for analyzing and extracting information from scientific literature in materials science. Choudhary and Kelley____ introduced ChemNLP, a comprehensive library for applying NLP techniques to materials chemistry text data. ChemNLP offers several key functionalities relevant to our work. The library provides curated open-access datasets from arXiv and PubChem, specifically tailored for materials and chemistry literature. This approach to creating domain-specific datasets aligns with our goal of developing a specialized system for nanotechnology research.
ChemNLP implements various machine learning models, including traditional algorithms (e.g., SVM, Random Forest), transformers, and graph neural networks, for classifying and clustering scientific texts. The library also incorporates Named Entity Recognition models trained on materials science data, capable of extracting entities such as material names, properties, and characterization methods. This functionality is particularly relevant for identifying key concepts and entities in nanotechnology literature.
The ChemNLP project explores the use of large language models like T5 for abstractive summarization and OPT for text generation in the context of scientific articles. Furthermore, ChemNLP demonstrates the potential of integrating NLP-derived insights with existing materials databases, such as density functional theory (DFT) datasets. This integration showcases the possibility of combining text-based knowledge with structured scientific data, a concept that could enhance the capabilities of our retrieval-augmented generation system.
The development of a web application for chemical formula searches in the ChemNLP project highlights the importance of user-friendly interfaces for accessing processed scientific information, a feature we aim to incorporate in NANOGPT. Overall, the ChemNLP library's approach to processing materials science literature provides valuable insights for our work on NANOGPT, particularly in areas of dataset preparation, entity recognition, and the potential for integrating text-based and structured scientific data in the field of nanotechnology.