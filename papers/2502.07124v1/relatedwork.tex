\section{Related Studies}
The structural design of large language models (LLMs) has been extensively studied, with particular emphasis on the role of neuron activations in shaping language representations \cite{choquet2024exploiting}. Research has shown that specific neurons within transformer-based architectures correspond to distinct syntactic and semantic properties, indicating that individual neuron activations contribute to language modeling through specialized functions \cite{golatkar2024cpr}. Certain approaches have sought to manipulate these activations to enhance model interpretability, particularly through controlled interventions that modulate neuron responses in a targeted manner \cite{sang2024evaluating}. Studies have demonstrated that neuron suppression or amplification techniques can significantly influence model outputs, offering insights into how linguistic abstractions are encoded within the network \cite{higasigi2024novel}. The interpretability of neuron behavior has also been explored through attribution-based methods that aim to quantify the contribution of individual neurons to specific tokens or linguistic structures \cite{nabovina2024neural, soikao2024novel}. Although these methods provide a granular understanding of neuron functions, challenges remain in effectively modifying neuron activations to improve model behavior without introducing unintended biases or performance degradation \cite{ pagacheva2024dynamic}. 

Techniques for optimizing the efficiency and adaptability of LLM architectures have focused on various neuron manipulation strategies \cite{nademort2024innovative}. Approaches such as structured neuron pruning and sparsity-inducing regularization methods have been implemented to reduce redundant activations while preserving overall model expressivity \cite{bernar2024exploring}. Results have indicated that targeted pruning schemes can lead to more efficient information routing within the network, enabling models to retain relevant linguistic abstractions while reducing computational overhead \cite{geline2024linguistic}. Compression techniques, including quantization and weight pruning, have been explored to improve the inference efficiency of LLMs without substantially affecting their generative capabilities \cite{shan2024benchmarking}. Additionally, investigations into the plasticity of neuron connections have revealed that retraining selected neurons rather than entire network layers can facilitate more targeted adaptation to evolving language patterns \cite{ sefeni2024game}. However, existing strategies primarily focus on reducing model size rather than restructuring neuron encapsulation for improved information aggregation, leaving a gap in methodologies that emphasize adaptive neuron connectivity \cite{korbanov2024hierarchical}. 

Efforts to integrate external knowledge sources into LLMs have been widely examined as a means of enhancing factual consistency and reducing the risk of hallucination in generated text \cite{aguiluz2024dynamic}. The incorporation of structured knowledge graphs into transformer-based architectures has been explored as a method for improving factual consistency by linking token representations to verifiable knowledge sources \cite{fa2024modality}. Additionally, pretraining strategies incorporating domain-specific corpora have demonstrated improvements in contextual relevance for specialized applications \cite{lu2024large}. However, studies have identified limitations in knowledge retrieval-based methods, particularly when attempting to balance memorization with the ability to generalize across different domains \cite{mcintosh2024game}. The challenge of continuously updating internal knowledge representations has also been addressed through lifelong learning paradigms that seek to refine LLM parameters incrementally without catastrophic forgetting \cite{tarel2024transformative}. While these approaches have shown promise in expanding model knowledge, their reliance on external retrieval mechanisms often leads to latency issues and inconsistencies in information integration across different contexts \cite{atox2024evaluating}. The underlying difficulty lies in ensuring that internalized knowledge representations remain both dynamic and efficiently aggregated without introducing instability in model behavior \cite{foster2024token}. 

The interaction between neuron activation patterns and information retrieval mechanisms within LLMs remains an open area of research \cite{chard2024auditing}. Studies have suggested that the alignment of activation dynamics with external knowledge retrieval can play a crucial role in maintaining response coherence, particularly in scenarios requiring complex reasoning \cite{hisaharo2024optimizing}. Computational models have been proposed to analyze the interplay between activation distributions and contextual embeddings, revealing that certain neuron clusters are responsible for encoding long-range dependencies in textual data \cite{eamen2024neural}. Investigations into the hierarchical structure of transformer layers have indicated that deeper layers contribute more significantly to abstract reasoning, whereas earlier layers tend to encode surface-level syntactic relationships \cite{ vitiello2024context}. The adaptive tuning of intermediate representations has been explored as a means of enhancing multi-step reasoning capabilities, but existing methods have faced difficulties in ensuring stable convergence across diverse input distributions \cite{kanax2024contextualized, navjord2023beyond}. Without explicit mechanisms for regulating the way information is encapsulated and distributed across neurons, existing models often exhibit inconsistencies in knowledge retention and retrieval, leading to suboptimal generalization performance \cite{wench2024factored}. 

Existing neuron manipulation and information integration methods, while effective in their respective domains, fail to address the need for a comprehensive framework that unifies these approaches within LLM architectures \cite{lapov2024dynamic}. The lack of structured neuron encapsulation mechanisms results in inefficient information aggregation, particularly when models are tasked with processing heterogeneous data sources \cite{beaumont2024neural}. Current approaches largely focus on static modifications to neuron connectivity or external knowledge retrieval, rather than reconfiguring neuron encapsulation to facilitate a more adaptive and efficient aggregation of linguistic abstractions \cite{lemal2024dynamic}. The approach introduced in this study aims to bridge this gap by proposing a novel method for neuron encapsulation that optimizes the internal structure of LLMs for more effective information aggregation, addressing the limitations of both prior neuron manipulation techniques and external knowledge integration strategies \cite{merrick2024upscaling}.