\section{Related Work}
Learning-augmented algorithms have been in the epicenter of the recent research literature, with numerous related contributions in recent years. Many classical problems have been reconsidered, and new algorithms, enhanced with (possibly erroneous) machine-learned predictions about their input, are designed and analyzed with respect to the achievable consistency and robustness. Representative problem domains include data structures, online and approximation algorithms for combinatorial optimization, streaming and sublinear algorithms, and many more. For a more detailed exposition see the early survey by \cite{mitzenmacher2022algorithms} as well as the online repository \url{algorithms-with-predictions.github.io}.

In algorithmic game theory and computational social choice, the concept of prediction has been considered for problems related to mechanism design \citep{agrawal2022learning,BGT23,balkanski2023online,balkanski2024randomized,colini2024trust,christodoulou2024mechanism}, the price of anarchy of cost sharing \citep{gkatzelis2022improved}, and the distortion of voting \citep{berger2023optimal}. More closely to our work, the prediction framework has been considered in auction environments. Namely, \cite{XL22} were the first to study revenue-maximizing auctions with the challenging benchmark of the highest valuation, \cite{caragiannis2024randomized} leveraged a fundamental setting from the theory of \cite{myerson1981optimal} and developed tight randomized auctions with the same challenging benchmark of the highest valuation. On the other hand, \cite{lu2024competitive} adopted several benchmarks and studied competitive auctions and digital goods with predictions and lastly \cite{gkatzelis2025clock} studied clock auctions with unreliable advice.

Our assumption of using a revenue benchmark in order to evaluate the efficiency of our mechanisms is similar in spirit with the framework of competitive auctions, initiated with the work of \cite{goldberg2006competitive} and adopted in the algorithms with predictions by \cite{lu2024competitive}. Finally, close to our work is the study of revenue-maximizing auctions with a \textit{single sample} by \cite{dhangwatnotai2010revenue}, where they initiated the study of \textit{random reserve} prices. One of our mechanisms implicitly leverages random reserve prices and we use a characterization developed in their paper. Generally the field of Bayesian mechanism design, which uses extensively statistical information about the agent valuations, is surveyed by \cite{hartline2013mechanism}.