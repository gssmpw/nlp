\documentclass[11pt,a4paper]{article}
\usepackage{times}
\usepackage{soul}
\usepackage{url}
\usepackage[hidelinks]{hyperref}
\usepackage[utf8]{inputenc}
\usepackage[small]{caption}
\usepackage{graphicx}
\usepackage{amsmath}
\usepackage{amsthm}
\usepackage{amssymb}
\usepackage{booktabs}
\usepackage{algorithm}
\usepackage{algorithmic}
\usepackage[switch]{lineno}
%\usepackage[a4paper, total={6in, 8in}]{geometry}
\usepackage{natbib}
\usepackage{fullpage}
%\pagenumbering{gobble}



\newtheorem{example}{Example}
\newtheorem{theorem}{Theorem}
\newtheorem{corollary}{Corollary}[theorem]
\newtheorem{lemma}[theorem]{Lemma}
\newtheorem{definition}{Definition}[section]
\newtheorem{remark}{Remark}
\newtheorem{prop}{Proposition}


\title{Mechanisms for Selling an Item Among a Strategic Bidder\\and a Profiled Agent}
\author{
\textbf{Ioannis Caragiannis}\\
\small{Aarhus University, Denmark}\\
\small{iannis@cs.au.dk}
\newline
\and
\textbf{Georgios Kalantzis}\\
\small{University of Edinburgh, United Kingddom}\\
\small{G.Kalantzis@sms.ed.ac.uk}
}
\date{}

\allowdisplaybreaks
\sloppy

\begin{document}


\maketitle

\begin{abstract}
We consider a scenario where a single item can be sold to one of two agents. Both agents draw their valuation for the item from the same probability distribution. However, only one of them submits a bid to the mechanism. For the other, the mechanism receives a \textit{prediction} for her valuation, which can be true or false. Our goal is to design mechanisms for selling the item which make as high revenue as possible in cases of a correct or incorrect prediction. As benchmark for proving our revenue-approximation guarantees, we use the maximum expected revenue that can be obtained by a strategic and a honest bidder. We study two mechanisms. The first one yields optimal revenue when the prediction is guaranteed to be correct and a constant revenue approximation when the prediction is incorrect, assuming that the agent valuations are drawn from a monotone hazard rate (MHR) distribution. Our second mechanism ignores the prediction for the second agent and simulates the revenue-optimal mechanism when no bid information for the bidders is available. We prove, again assuming that valuations are drawn from MHR distributions, that this mechanism achieves a constant revenue approximation guarantee compared to the revenue-optimal mechanism for a honest and a strategic bidder. The MHR assumption is necessary; we show that there are regular probability distributions for which no constant revenue approximation is possible.
\end{abstract}

\section{Introduction}
In standard auction settings, the participating bidders have some private value for the items which are for sale; then some elicitation procedure enables the bidders to submit a bid to the auctioneer, who will decide which bidders get the items and at which price. However, if we consider a setting where some bidders, while desiring to buy some item, abstain from submitting a bid to the \textit{mechanism}, then the auctioneer will not have any other choice but to come up with a price for the item and make a take-or-leave it offer to that particular bidder. The bidder then could either accept this offer and win the item or reject it. Note that a bidder could abstain from submitting her own bid to a mechanism for a plethora of reasons, for instance if the bids eventually become publicly available or known to the other bidders then someone who wants to retain her privacy would avoid revealing her private monetary valuation of the items for sale. On the other hand, in some cases the bidder may not be sure of what bid to submit for an item and is more willing to respond to an offer made to her. 

The challenge that arises in the design of an auction in such a setting is to determine the value of the offer that the auctioneer has to make to the bidder that didn't submit a bid. However, by following the \textit{data-driven} approach and the very prominent research agenda of \textit{learning-augmented} algorithms we could assume that a prediction model is available for the private valuation of that bidder. Though following the paradigm of \cite{mitzenmacher2022algorithms} these predictions stemming from statistical models could possibly be erroneous, thus an efficient mechanism should be robust against an incorrect prediction. Also, since the prediction model essentially provides information for the private valuation of that bidder, for the rest of the paper we will call the bidder, who doesn't submit a bid, a \textit{profiled agent}.

We initiate the study of such auction environments by focusing on the model in which a strategic bidder and a profiled agent participate in a single-item auction. The strategic bidder submits a bid to the auction and acts strategically by not necessarily revealing her true valuation and the profiled agent doesn't submit a bid, but a prediction is available for her valuation which the auctioneer can use to make a take-or-leave it offer to her. With guideline the beautiful theory of \cite{myerson1981optimal}, our goal is to design a \textit{truthful} auction where we incentivize the strategic bidder to reveal her true valuation to the mechanism and at the same time extract as much revenue as possible.

Most of the prior work on algorithms and mechanisms with predictions, assumes that the prediction available is a numerical non-probabilistic value representing the unknown feature of the environment we want to predict, see survey \cite{mitzenmacher2022algorithms}. Recent work \citep{dinitz2024binary} in order to more accurately simulate the output format of most statistical models, assumes it is a distribution which represents the distribution that feature draws values from.

In this paper to further exploit the richness of the prediction and adapt to the Bayesian environment of the auction, we assume that the prediction is a stochastic variable following the same distribution with the feature we want to predict and essentially achieves this by exhibiting certain correlations with it. Namely, by following the literature we consider two cases; first the \textit{consistency} where the prediction is always correct and returns the true value that feature drew from its distribution, or the case of \textit{robustness} where the prediction could be incorrectly correlated with the feature and return arbitrarily wrong values.

\subsection{Our Contribution}
We develop two mechanisms in order to extract high revenue from a new auction environment with a strategic bidder and a profiled agent. We focus on auctions in a Bayesian environment, which is inherently tied with revenue-maximizing auctions and has not been studied through the lens of learning-augmented algorithms.

In our way of doing so, we define a prediction model for the private valuation of the profiled agent, that allows us to adapt to the Bayesian setting. In this setting, the bidders draw their valuations from a distribution and we study the average (or expected) outcome and revenue of the auction. Thus, using a simple non-probabilistic prediction, will not allow us to perform the average analysis of the auction along consistency and robustness. In our model the prediction is a stochastic variable that naturally fits in the Bayesian environment by exhibiting certain correlations with the valuation we are predicting.

Note that if the prediction is always correct then essentially we know the exact private valuation of the profiled agent and we can always make her an offer she will accept, therefore in that case our setting boils down to a strategic and a honest bidder, who never misreports her valuation. We use the optimal revenue we can retrieve in that case as the challenging revenue benchmark for evaluating the efficiency of our mechanisms.

The first mechanism we study has perfect consistency and we also prove the positive result that has constant $\frac{1}{4e}$ robustness. Though with the aim of improving the robustness of the first mechanism, we develop a second one with better robustness of $\frac{1}{e}$, which essentially is prediction neutral and does not leverage the prediction. Finally, we combine these two mechanisms to get another one that lies on a \textit{Pareto frontier} along the objectives of consistency and robustness.

\subsection{Related Work}
Learning-augmented algorithms have been in the epicenter of the recent research literature, with numerous related contributions in recent years. Many classical problems have been reconsidered, and new algorithms, enhanced with (possibly erroneous) machine-learned predictions about their input, are designed and analyzed with respect to the achievable consistency and robustness. Representative problem domains include data structures, online and approximation algorithms for combinatorial optimization, streaming and sublinear algorithms, and many more. For a more detailed exposition see the early survey by \cite{mitzenmacher2022algorithms} as well as the online repository \url{algorithms-with-predictions.github.io}.

In algorithmic game theory and computational social choice, the concept of prediction has been considered for problems related to mechanism design \citep{agrawal2022learning,BGT23,balkanski2023online,balkanski2024randomized,colini2024trust,christodoulou2024mechanism}, the price of anarchy of cost sharing \citep{gkatzelis2022improved}, and the distortion of voting \citep{berger2023optimal}. More closely to our work, the prediction framework has been considered in auction environments. Namely, \cite{XL22} were the first to study revenue-maximizing auctions with the challenging benchmark of the highest valuation, \cite{caragiannis2024randomized} leveraged a fundamental setting from the theory of \cite{myerson1981optimal} and developed tight randomized auctions with the same challenging benchmark of the highest valuation. On the other hand, \cite{lu2024competitive} adopted several benchmarks and studied competitive auctions and digital goods with predictions and lastly \cite{gkatzelis2025clock} studied clock auctions with unreliable advice.

Our assumption of using a revenue benchmark in order to evaluate the efficiency of our mechanisms is similar in spirit with the framework of competitive auctions, initiated with the work of \cite{goldberg2006competitive} and adopted in the algorithms with predictions by \cite{lu2024competitive}. Finally, close to our work is the study of revenue-maximizing auctions with a \textit{single sample} by \cite{dhangwatnotai2010revenue}, where they initiated the study of \textit{random reserve} prices. One of our mechanisms implicitly leverages random reserve prices and we use a characterization developed in their paper. Generally the field of Bayesian mechanism design, which uses extensively statistical information about the agent valuations, is surveyed by \cite{hartline2013mechanism}.

\section{Model and Preliminaries}\label{sec:prelims}
We begin by introducing some notation and concepts on \textit{revenue-optimal} auctions. Most of this information and further details can be found in classical textbooks and auction theory literature, e.g. see \cite{roughgarden2016twenty,hartline2013mechanism} and \cite{krishna2009auction}. In our setup, there is a set of bidders $\mathcal{N} = \{1,2, \ldots ,n\}$ and a single item for sale. Each bidder has a private valuation $v_i$ for the item which is drawn from a commonly known (cumulative) distribution  function $F$, with positive density function $f$, which belongs to a family of distributions $\mathcal{F}$ with support $\mathbb{R}^{+}$. Each bidder $i \in \mathcal{N}$ submits a bid $b_i$ to the auctioneer, who after collecting all the bids $\mathbf{b} = (b_1,b_2, ...,b_n)$, decides who will receive the item and at which price. 

Specifically, there are two components; the auctioneer has to derive an allocation rule $\mathbf{x}:\mathcal{B} \rightarrow U$, where $\mathcal{B}$ denotes the set of bid vectors and $U$ denotes a set of probability distributions over the bidders, and a payment function $\mathbf{p}:\mathcal{B} \rightarrow \mathbb{R}^{n}$. Note that since we focus on deterministic single-item auctions we assume $U$ to be a set of binary vectors with the additional constraint that $\sum_{i \in \mathcal{N}}x_i(\mathbf{b}) = 1$. Also, let $\mathcal{M}(\cdot) = (\mathbf{x}(\mathbf{b}),\mathbf{p}(\mathbf{b}))$ denote the outcome of an auction (or mechanism).

Bidders are utility maximizers. Bidder $i$’s utility, from the outcome of an auction that uses the allocation rule $\mathbf{x}$ and the payment rule $\mathbf{p}$ when the bidders submit the bid vector $\mathbf{b}$, is defined as $u_i(\mathbf{b}) = v_i \cdot x_i(\mathbf{b}) - p_i(\mathbf{b})$. Generally, the bidders can be \textit{honest} and only submit their true valuations, irrespectively of what the other bidders submit, or be \textit{strategic} and misreport their true valuation in order to increase their utility. For strategic bidders, we define the desired property of \textit{truthfulness}, which requires that each bidder $i$, maximizes her utility by reporting her true valuation as bid. Formally, this requirement can be written as $u_i(v_i, b_{-i}) \geq u_i(z, b_{-i})$, for every bid vector $\mathbf{b}_{-i}$ submitted by the agents different than $i$ and for every possible bid $z$ bidder $i$ may consider to submit. In addition, the property of \textit{individual rationality} requires that every truthful bidder $i \in \mathcal{N}$ has non-negative utility $u_i(v_i, \mathbf{b}_{-i})$, for any possible bids $\mathbf{b}_{-i}$ by the other bidders. An auction that is both truthful and individually rational is called \textit{dominant-strategy incentive-compatible} (DSIC).

\cite{myerson1981optimal} in his seminal work characterized the DSIC single-item auction with strategic bidders. First he proved that any DSIC auction must have a monotone \textit{non-decreasing} allocation rule and provided an explicit payment rule that is unique for any DSIC auction. Namely we have:
\begin{lemma}[\cite{myerson1981optimal}]\label{prelims:myerson-lemma-mon}
    An auction ($\mathcal{B},\mathbf{x},\mathbf{p}$) is DSIC if and only if the allocation rule $\mathbf{x}$ is monotone non-decreasing, in which case the payment formula for bidder $i$ is given as:
    \[
    p_i(\mathbf{b}) = b_i \cdot x_i(\mathbf{b})- \int_0^{b_i}x_i(z,\mathbf{b}_{-i}) \, dz
    \]
\end{lemma}
An important implication of Myerson's Lemma \ref{prelims:total-exp-rev} is that in deterministic single-item auctions the winning bidder will be charged what we refer to as her \textit{critical bid}. The critical bid $c(\mathbf{b}_{-i})$ of bidder $i$ essentially is the minimum bid she has to make in order to have the highest-bid, formally defined as $c(\mathbf{b}_{-i}) = \inf\{z_i: \forall j \neq i, z_i \geq b_j \}$.

Next, if we associate the private valuation $v_i$ of each bidder $i$ with the so-called \textit{virtual valuation} function given as $\phi(v_i)= v_i - \frac{1-F(u_i)}{f(u_i)}$, then by using Myerson's payment formula from Lemma \ref{prelims:myerson-lemma-mon} we get the following characterization for the expected total payment when bidders independently draw their valuations from a common distribution $F$.
\begin{lemma}[\cite{myerson1981optimal}]\label{prelims:myerson-lemma-exp}
    Every DSIC auction ($\mathcal{B},\mathbf{x},\mathbf{p}$) with $\mathcal{N}$ bidders who draw their valuations independently from a common distribution $F$, has total expected revenue that is given as:
    \[
    \mathbb{E}_{\mathbf{v}\sim F}\bigg[\sum_{i \in \mathcal{N}}p_i(\mathbf{v})\bigg] = \mathbb{E}_{\mathbf{v}\sim F}\bigg[\sum_{i \in \mathcal{N}}\phi(v_i)\cdot x_i(\mathbf{v})\bigg] 
    \]
\end{lemma}
So, the expected revenue of an auction equals the expected \textit{virtual welfare} and maximizing it over the space of DSIC auctions reduces to maximizing the expected virtual welfare. So in a single-item auction, if we aim to maximize the expected revenue we should maximize-pointwise and give the item to the bidder with the highest virtual valuation $\phi(v_i)$. However this rule is not necessarily monotone non-decreasing which is required from Myerson's Lemma. This depends on the form of the virtual valuation function $\phi$. For instance in simple terms, if the virtual valuation function is decreasing, then bidders with higher valuations will have smaller chances of winning or the virtual valuation function could even be negative.

This gives rise to the desired property of \textit{regularity} for the virtual valuation function, which assumes that the function $\phi$ is non-decreasing. A stronger assumption which we adapt throughout the paper is the \textit{monotone hazard rate} (MHR), where by defining the \textit{hazard rate} for a bidder $i$ as $h(v_i) = \frac{f(u_i)}{1-F(u_i)}$, we assume that $h(v_i)$ is monotone non-decreasing. Note that monotone hazard rate implies regularity, but the other way around does not hold always. Also, several important distributions are MHR, e.g. the uniform, the normal, the exponential etc. For further details the interested reader can refer to \cite{roughgarden2016twenty} and \cite{hartline2013mechanism}.

In a sense the virtual valuation function $\phi$ acts as an offset for guaranteeing truthfulness. Because if the bidders were \textit{honest} and only submitted their true valuations, by the utility function an individually rational and revenue-optimal auction would simply charge the winning bidder her true valuation, and we would not need Myerson's Lemma to derive a payment formula. In that case the total expected revenue would simply be the expected sum of the true valuations of the bidders over the allocation rule.

Moving on to our setting, we have one bidder who is strategic and submits a bid to the auction and a profiled agent who does not submit a bid, but responds to an offer. Myerson's toolkit cannot be applied directly, since it solely handles the case where all the bidders are strategic and furthermore we have to somehow leverage the prediction available, to come up with a good offer to the profiled agent.

Importantly, we consider the setup where the auctioneer will collect the bid from the strategic bidder and by using the prediction for the profiled agent as well, would have to make an irrevocable decision; allocate the item to the strategic bidder and charge her respectively or make an offer to the profiled agent and in case she accepts, give her the item and charge her.

Note that, if the auctioneer decides to not give the item to the strategic bidder and makes an offer to the profiled agent, then if the profiled agent rejects the item, the item does not get allocated to anyone. Because if in this case the auctioneer gave the item back to the strategic, then an allocation rule which uses the bid information from the strategic bidder to deduce whether or not to make an offer to the agent, would not be monotone. Since the strategic bidder can potentially manipulate her bid and win in the event that the profiled agent rejects the offer.

Furthermore, no multiple offers are allowed to the profiled agent. Finally, we make the rationality assumption that if the offer to the profiled agent is less or equal than her true valuation, then she is going to accept it, otherwise reject it. 

\paragraph{Prediction Model}\label{prediction-model}
Next we define the prediction model. Let $\widehat{v}$ be the prediction, a random variable with range $\mathbb{X} \subseteq \mathbb{R}^{+}$ and let $v_p$ be the valuation of the profiled agent, a random variable again with range $\mathbb{Y} \subseteq \mathbb{R}^{+}$. Both random variables follow the same distribution $F$. The motivation behind this model is that we want to perform expected (or average) analysis over the distribution $F$ that the valuation follows. If the prediction is correct, then essentially $\widehat{v}$ becomes the true valuation which is drawn from $F$, in case the prediction is wrong we still want the same expected analysis to hold.

We distinguish two cases, i.e. consistency and robustness, with different conditional distributions between $v_p$ and $\widehat{v}$ in each case. Namely, we define two \textit{joint} (cumulative) probability functions $C_{(\widehat{v},v_p)}$ and $R_{(\widehat{v}, v_p)}$ for consistency and robustness respectively, where in both cases the marginal distributions of each variable are the same. We show the conditional distributions, when $v_p$ and $\widehat{v}$ are discrete random variables which is more straightforward. Namely, for the consistency we denote $(x,y) \in \mathbb{X}\times \mathbb{Y}$ realizations of $\widehat{v}$ and $v_p$ respectively, and we have the following conditional distribution:
\[
C_{\widehat{v}|v_p}(x,y) = \Pr[\widehat{v} = x | v_p = y] = \begin{cases}
        1 &\text{if } x=y \\
        0 &\text{otherwise}
    \end{cases}
\]
Thus for any value $y$ the profiled agent draws then the prediction always realizes the same value. 

On the other hand, for the robustness we have the following conditional distribution. Let $0 < \delta(x,y) \leq 1$ denote some positive function of $(x,y) \in \mathbb{X}\times \mathbb{Y}$, then
\[
R_{\widehat{v}|v_p}(x,y) = \Pr[\widehat{v} = x | v_p = y] = \begin{cases}
        \delta(x,y) &\text{if } x\neq y \\
        1 - \delta(x,y) &\text{otherwise}
    \end{cases}
\]
So, for any value $y$ the profiled agent draws, the prediction can realize any different value $x$ with positive probability. The conditional distribution $R_{\widehat{v}|v_p}(x,y)$ depends on the $\delta(x,y)$ function which characterizes the arbitrary correlation that the prediction $\widehat{v}$ can have with the valuation $v_p$. The values $x$, that $\widehat{v}$ realizes can be arbitrarily different than the values $y$ that $v_p$ does, with some positive probability. 

If $v_p$ and $\widehat{v}$ were continuous, then in our definitions we would adapt the standard discretization approach from the literature. Specifically, for the consistency we would like the prediction to approach $v_p$ from below, i.e. $\lim_{\epsilon \rightarrow 0}\Pr[\widehat{v}\in\{x - \epsilon, x\} \hspace{0.5mm}|\hspace{0.5mm} v_p \in \{y, y + \epsilon\}] = 1$ if $ x = y$. In a similar manner, we define the conditional distribution for robustness.

\paragraph{Revenue Benchmark}\label{sec:benchmark}
In order to evaluate the efficiency of our mechanisms, both their consistency and robustness, we use the challenging benchmark of the optimal revenue we can obtain when the prediction is always correct. 

Note that when the prediction is correct and the auctioneer knows the true valuation of the profiled agent, then essentially the agent becomes an honest bidder who submits her true valuation to the auction. This is because the auctioneer could simply always retrieve her true valuation, just by offering her the prediction value, which given our assumptions she will accept.

Essentially, the total expected revenue given by Lemma \ref{prelims:myerson-lemma-exp} for a strategic and a honest bidder simplifies to the summation of two terms over the allocation rule. The first will include the virtual valuation term, since one bidder is strategic and we require truthfulness, and the second would simply be the true valuation of the agent, since she acts as an honest bidder and the auctioneer knows her value in advance. By linearity of expectation we get the total expected revenue:
\begin{equation}\label{prelims:total-exp-rev}
    \mathbb{E}_{\mathbf{v}\sim F}\left[\sum_{i \in \{p,s\}}p_i(\mathbf{v})\right] = \mathbb{E}_{\mathbf{v}\sim F}[\phi(v_s)\cdot x_s(\mathbf{v}) + v_p\cdot x_p(\mathbf{v})] 
\end{equation}
with $\mathbf{v} = (v_s,v_p)$, where $v_s$ denotes the valuation of the strategic bidder and $v_p$ the valuation of the profiled agent and $p_s,p_p$ denote the payment functions respectively.

Therefore, from Equation (\ref{prelims:total-exp-rev}) similarly as before, we can design the revenue optimal allocation. Simply compare the realizations of $\phi(v_s)$ and the prediction $\widehat{v}=v_p$ \footnote{Since we have a DSIC auction we assume that $v_s$ is the bid realization of the strategic bidder, same as her true valuation.} and give the item to the strategic bidder or offer $\widehat{v}$ to the profiled agent. The payment for the strategic bidder, by following Lemma \ref{prelims:myerson-lemma-mon} will be given by her critical bid, $\phi^{-1}(v_p)$ and for the profiled agent would simply be her true valuation $v_p$. To summarize we have the following proposition:
\begin{prop}\label{propos: revenue-optimal-auction}
    Let $M_{\text{opt-c}}$ be a DSIC mechanism ($\mathcal{B},\mathbf{x},\mathbf{p}$) with prediction input $\widehat{v}$ and the following components 
    \[
        x_s = \begin{cases}
            1 &\text{if } v_s \geq \phi^{-1}(\widehat{v}) \\
            0 &\text{otherwise}
        \end{cases}
    \]
    \[
        x_p = \begin{cases}
            1 &\text{if } \widehat{v} \geq \phi(v_s) \\
            0 &\text{otherwise}
        \end{cases}
    \]
    where $\mathbf{x} = (x_s,x_p)$ is the allocation rule, $x_s$ denotes the allocation to the strategic bidder, $x_p$ to the profiled agent and $v_s$ is the bid realization of the strategic bidder. And
    \[
        p_s = \begin{cases}
            \phi^{-1}(\widehat{v}) &\text{if } v_s \geq \phi^{-1}(\widehat{v}) \\
            0 &\text{otherwise}
        \end{cases}
    \]
    \[
        p_p = \begin{cases}
            \widehat{v} &\text{if } \widehat{v} \geq \phi(v_s)\\
            0 &\text{otherwise}
        \end{cases}
    \]
    where $\mathbf{p} = (p_s,p_p)$ is the payment rule respectively. 
    
    Then $M_{\text{opt-c}}$ is the revenue-optimal mechanism when the prediction input $\widehat{v}$ is always correct, i.e. $\widehat{v} = v_p$, where $v_p$ the valuation of the profiled agent.
\end{prop}
The proof is provided in the Supplementary material along with the omitted proofs of other following Lemmas and Corollaries, due to lack of space.

\paragraph{Consistency and Robustness}
We define the two objectives of the learning-augmented framework, i.e. the consistency and the robustness. We consider the expected revenue of $M_{\text{opt-c}}$ when the prediction is correct as our benchmark and evaluate the consistency and robustness of our mechanisms with respect to it. In the case of consistency we always have $v_p =\widehat{v}$ but for robustness $\widehat{v}$ can be arbitrarily correlated with $v_p$. So, for DSIC auctions we have the following objectives
\[
\text{consistency}(\mathcal{M}) = \sup_{F \in \mathcal{F}}\frac{\mathbb{E}_{\mathbf{v}\sim F}[\mathcal{R}(\mathcal{M}(\mathbf{v}))]}{\mathbb{E}_{\mathbf{v}\sim F}[\mathcal{R}(M_{\text{opt-c}})]}
\]
and 
\[
\text{robustness}(\mathcal{M}) = \sup_{F \in\mathcal{F}}\sup_{R_{\widehat{v}|v_p}}\frac{\mathbb{E}_{(\mathbf{v},\widehat{v})\sim F}[\mathcal{R}(\mathcal{M}(\mathbf{v},\widehat{v}))]}{\mathbb{E}_{\mathbf{v}\sim F}[\mathcal{R}(M_{\text{opt-c}})]}
\]
where $\sup_{R_{\widehat{v}|v_p}}$ denotes the worst-case conditional distribution between $\widehat{v}$ and $v_p$ and $\mathcal{R}(\cdot)$ denotes the revenue obtained from a mechanism. 

\section{The Robustness of the Benchmark Mechanism}
Next we show that the benchmark mechanism $M_{\text{opt-c}}$ besides being the revenue-optimal mechanism when the prediction is always correct, it is also robust when the prediction is wrong.

Before doing that we derive closed formulas for the benchmark revenue, that $M_{\text{opt-c}}$ achieves in the case of consistency, i.e. when the prediction is always correct. Let $\mathcal{R}_s$ be the expected revenue extracted from the strategic bidder. We temporarily denote as $\widehat{v} = v_p$ and $v_s$ as the realizations of the prediction and the strategic bidder respectively. Then by following the blueprint of Proposition \ref{propos: revenue-optimal-auction}, we know that the strategic bidder in case she wins, i.e. $v_s \geq \phi^{-1}(\widehat{v})$, will pay her critical bid $\phi^{-1}(\widehat{v})$, thus:
\begin{align}\label{strategic-revenue}
    \mathcal{R}_s &= \mathbb{E}_{(\mathbf{v},\widehat{v})\sim F}\left[\phi^{-1}(\widehat{v}) \hspace{0.5mm}| \hspace{0.5mm} v_s \geq \phi^{-1}(\widehat{v})\right] \cdot \Pr\left[v_s \geq \phi^{-1}(\widehat{v})\right] \nonumber \\
    &= \int_0^{\infty}\bigg(\int_{\phi^{-1}(\widehat{v})}^{\infty} \phi^{-1}(\widehat{v})\cdot f(v_s) \, dv_s \bigg) \cdot f(\widehat{v}) \, d\widehat{v} \nonumber \\
    &= \int_0^{\infty} \phi^{-1}(\widehat{v}) \cdot (1 - F(\phi^{-1}(\widehat{v})))\cdot f(\widehat{v}) \, d\widehat{v} \nonumber \\
    &= \int_0^{\infty} \phi^{-1}(z) \cdot (1 - F(\phi^{-1}(z)))\cdot f(z) \, dz 
\end{align}
In the last step we just changed the notation from $\widehat{v}$ to $z$, and we maintain this simpler notation from now on.

On the other hand let $\mathcal{R}_p$ be the expected revenue drawn from the profiled agent. Based on Proposition \ref{propos: revenue-optimal-auction}, the agent will accept the offer in the case of consistency and pay the prediction value $\widehat{v} = v_p$ if $\widehat{v} \geq \phi(v_s)$, or alternatively if $v_s \leq \phi^{-1}(\widehat{v})$, thus
\begin{align}\label{honest-revenue}
    \mathcal{R}_p &= \mathbb{E}_{(\mathbf{v},\widehat{v})\sim F}\left[\widehat{v} \hspace{0.5mm}| \hspace{0.5mm} \widehat{v} \geq \phi(v_s)\right] \cdot \Pr\left[\widehat{v} \geq \phi(v_s)\right] \nonumber \\
    &= \int_{0}^{\infty} \bigg(\int_{0}^{\phi^{-1}(\widehat{v})} \widehat{v}\cdot f(v_s) \, dv_s \bigg) 
    \cdot f(\widehat{v})\, d\widehat{v} \nonumber \\
    &= \int_0^{\infty} \widehat{v} \cdot F(\phi^{-1}(\widehat{v})) \cdot f(\widehat{v}) \, d\widehat{v} \nonumber \\
    &= \int_0^{\infty} z \cdot F(\phi^{-1}(z)) \cdot f(z) \, dz
\end{align}
So, by the law of total expectation when the prediction is always correct we get the benchmark expected revenue of $\mathcal{R}_s + \mathcal{R}_p$. 

Before proving the robustness of $M_{\text{opt-c}}$ and try to approximate $\mathcal{R}_s + \mathcal{R}_p$ when the prediction is incorrect, we make the following negative observation. If the prediction $\widehat{v}$ is arbitrarily correlated with $v_p$, then we can not always extract revenue from the profiled agent with $M_{\text{opt-c}}$. To see this it suffices to consider the simple case where $v_p$ and $\widehat{v}$ are uniformly random in $[0,1]$. Then, from the definition of the conditional distribution $R_{\widehat{v}|v_p}$ in Section \ref{prediction-model}, we consider the following instance. Let a negligibly small but positive $\epsilon >0$. Then denote as $y$ the realization of $v_p$, if $y \in [0,1-\epsilon]$ we have that $\delta(x,y) = 1$ for a realization $x = y + \epsilon$ of the prediction $\widehat{v}$. On the other hand, if $y \in (1-\epsilon, 1]$ then we have $\delta(x,y) = 1$ for $x = y - 1 + \epsilon$. 

Thus, the above realizations $(x,y)$ occur with probability one in each case. Therefore, the offer that the mechanism $M_{\text{opt-c}}$ makes to the profiled agent can (almost) always be higher than her actual value and she is going to reject it, since in total the probability that $\widehat{v}$ is not higher than $v_p$ is only $\epsilon$ which can be arbitrarily close to $0$.

Thus, for the robustness of $M_{\text{opt-c}}$ in the worst case we can assume that we can only extract revenue from the strategic bidder, which is still given by $\mathcal{R}_s$ even if $\widehat{v} \neq v_p$. In our attempt to approximate the optimal revenue $\mathcal{R}_s + \mathcal{R}_p$ with the revenue $\mathcal{R}_s$ drawn only from the strategic bidder, we will assume that the distribution $F$, which the valuations and the prediction are drawn from is MHR, since we have the following inapproximability result.
\begin{corollary}
    There exists a class of regular, but not MHR, distributions for which the ratio $\frac{\mathcal{R}_s + \mathcal{R}_p}{\mathcal{R}_s}$ is arbitrarily large.
\end{corollary}
Next we establish a series of tools we will need tailored for MHR distributions. Specifically for the MHR distribution it's a folklore fact by now that we can bound the quantile of the Myerson's reserve $r^{*} = \phi^{-1}(0)$ as $1 - F(r^{*}) \geq \frac{1}{e}$. Next we provide a more general version of this bound.
\begin{lemma}\label{quantiles-lemma}
    For any MHR distribution $F \in \mathcal{F}$, let $z \in \mathbb{R}^{+}$ and $\phi$ the virtual valuation function, then it holds that
    \[
    \frac{1 - F(\phi^{-1}(z))}{1 - F(z)}\geq \frac{1}{e}
    \]
\end{lemma}
\begin{proof}
    First, we define the cumulative hazard rate as $H(z) =\int_0^{z}h(y) \,dy $ it is well-known that $1 - F(z) = e^{-H(z)}$ holds \footnote{It actually holds for any differentiable function of $z$, we can verify by taking the natural logarithm and then differentiate.}. Now let's look at the inverse ratio and substitute with this transformation and with the defintion of $H(z)$
    \begin{align}\label{1}
        \frac{1 - F(z)}{1 - F(\phi^{-1}(z))} &= \frac{e^{-H(z)}}{e^{-H(\phi^{-1}(z))}} \nonumber \\
        &= \exp{\bigg(\int_{0}^{\phi^{-1}(z)}h(y) \,dy-\int_{0}^{z}h(y) \,dy\bigg)} \nonumber \\
        &= \exp{\bigg(\int_{z}^{\phi^{-1}(z)}h(y) \,dy\bigg)}
    \end{align}
     We claim that the value of the innermost integral is at most 1. Since, $h(y)$ is monotone non-decreasing by the MHR assumption, we can bound the area covered by the integral with a parallelogram of height $h(\phi^{-1}(z))$. So it holds that,
    \begin{align}\label{2}
        \int_{z}^{\phi^{-1}(z)}h(y) \,dy &\leq h(\phi^{-1}(z)) \cdot (\phi^{-1}(z)-z) \nonumber \\
        &= h(\phi^{-1}(z)) \cdot \frac{1}{h(\phi^{-1}(z))} \nonumber \\
        &= 1
    \end{align}
    In the first equality we used the definition of the inverse valuation function given by $\phi^{-1}(z) = z + \frac{1}{h(\phi^{-1}(z))}$. Therefore, by (\ref{1}) and (\ref{2}) we conclude that:
    \[
    \frac{1 - F(z)}{1 - F(\phi^{-1}(z))} \leq e
    \]
\end{proof}
Note that for $z=0$ from Lemma \ref{quantiles-lemma} we can recover the previous folklore bound of the quantile of Myerson's reserve $r^{*}$.

Our setup resembles the setup of revenue-maximization auctions with \textit{random reserves} by \cite{dhangwatnotai2010revenue}, since essentially for the strategic bidder we offer a random reserve price of $\phi^{-1}(u)$ where $u$ is drawn from a distribution $F$. And by denoting $R(r) = r\cdot(1-F(r))$ the well-known \textit{revenue function} with reserve price $r$, then essentially for the expected revenue of the strategic bidder it holds that $\mathcal{R}_s = \mathbb{E}_{u \sim F}[R(\phi^{-1}(u))]$. An implicit consequence of Lemma \ref{quantiles-lemma} is the fact that $\mathcal{R}_s$ achieves constant approximation to the expected revenue with a random reserve $u$ drawn from $F$.
\begin{corollary}\label{corollary-bound-strategic}
    For any MHR distribution $F \in \mathcal{F}$, let $u$ denote a valuation drawn from $F$, it holds that
    \[
    \mathcal{R}_s \geq \frac{1}{e}\cdot\mathbb{E}_{u \sim F}[R(u)]
    \]
\end{corollary}
Furthermore \cite{dhangwatnotai2010revenue} proved a tight for MHR distributions constant approximation of the expected revenue obtained with a random reserve to the expected social welfare. Namely, they proved the following
\begin{lemma}[\cite{dhangwatnotai2010revenue}]\label{Dhang-improved}
For any MHR distribution $F \in \mathcal{F}$, let $u$ denote a valuation drawn from $F$, it holds that
\[
\mathbb{E}_{u\sim F}[R(u)] \geq \frac{1}{4} \cdot \mathbb{E}_{u \sim F}[u]
\]
\end{lemma}
By leveraging the above we derive the constant robustness of $M_{\text{opt-c}}$ when $F$ is MHR and we get the following theorem
\begin{theorem}
     $M_{\text{opt-c}}$ is a DSIC mechansim ($\mathcal{B},\mathbf{x},\mathbf{p}$) that is $1$-consistent and $\frac{1}{4e}$-robust.
\end{theorem}
\begin{proof}
    The perfect consistency is clear, since $M_{\text{opt-c}}$ it's actually the revenue-optimal auction and the benchmark when the prediction is correct.

    For the robustness, from Lemma \ref{quantiles-lemma} we have that
    \[
    F(\phi^{-1}(z)) \leq 1 - \frac{1}{e}\cdot(1-F(z))
    \] 
    We multiply both sides $\forall z \in \mathbb{R}^{+}$ with $z\cdot f(z) \geq 0$ and then we integrate and use the definition of the expected value $\mathbb{E}_{u \sim F}[u]$ and the revenue function $R(\cdot)$ to get the following
    \begin{align}\label{robustness-derivation}
        \mathcal{R}_p &\leq \mathbb{E}_{u \sim F}[u] - \frac{1}{e}\cdot\mathbb{E}_{u\sim F}[R(u)] \nonumber \\ 
        &\leq 4\cdot\mathbb{E}_{u \sim F}[R(u)] - \frac{1}{e}\cdot\mathbb{E}_{u \sim F}[R(u)] \nonumber \\
        &= \left(4 - \frac{1}{e}\right)\cdot \mathbb{E}_{u \sim F}[R(u)] \nonumber \\
        &\leq \left(4 - \frac{1}{e}\right)\cdot e \cdot \mathcal{R}_s 
    \end{align}
    In the second inequality we used Lemma \ref{Dhang-improved} and in the last we used Corollary \ref{corollary-bound-strategic}.
    
    Finally from (\ref{robustness-derivation}) we add in both sides the strategic revenue $\mathcal{R}_s$ and we get the desired bound
    \begin{align*}
        \mathcal{R}_p + \mathcal{R}_s &\leq 4e \cdot \mathcal{R}_s
    \end{align*}    
\end{proof}
Thus we established the nice result that the revenue-optimal mechanism $M_{\text{opt-c}}$ when the prediction is always correct, it is also robust when it is not. However, we observed that if we use the prediction, in the case of the robustness we will not be able to make an offer to the profiled agent that she will accept, since adversarially the prediction can almost always be higher than her valuation and thus get no revenue from her. Therefore, we shift our focus to the other extreme in order to improve the robustness, where we disregard the "risky" prediction completely and design a mechanism without value information. Ultimately, our goal is to combine these two mechanisms in order to achieve a Pareto frontier along the objectives of consistency and robustness.

\section{A Mechanism with Improved Robustness}
If we disregard the prediction, then it is well-known \cite{hartline2013mechanism} by exploiting the prior information $F$ that the take-it-or-leave it offer we can make to the agent for optimal expected revenue is the Myerson's reserve $r^{*} = \phi^{-1}(0)$. We study a DSIC class of mechanisms where we use some positive deterministic reserve price $\tau \geq r^{*}$ for the strategic bidder and if she does not make it, then we offer the Myerson's reserve $r^{*}$ to the profiled agent. We can deduce the reserve price $\tau$ for the strategic bidder from the formula for the total expected revenue. We have:
\begin{equation}\label{total-revenue-without-pred}
    \mathbb{E}_{\mathbf{v}\sim F}\left[\sum_{i \in \{p,s\}}p_i(\mathbf{v})\right] = \mathbb{E}_{\mathbf{v}\sim F}[\phi(v_s)\cdot x_s(\mathbf{v}) + R(r^{*})\cdot x_p(\mathbf{v})]
\end{equation}
The first term is given again from Lemma \ref{prelims:myerson-lemma-exp} since the bidder is strategic and for the second term we know that if we offer Myerson's reserve $r^{*}$ to the profiled agent, then the expected revenue would be $R(r^{*})$. Thus from Equation (\ref{total-revenue-without-pred}) and similarly as in Proposition \ref{propos: revenue-optimal-auction}, the revenue-optimal mechanism would give the item to the strategic bidder if $v_s \geq \phi^{-1}(R(r^{*}))$ otherwise will offer $r^{*}$ to the profiled agent. Unfortunately though being the optimal mechanism, the analysis to deduce some constant approximation to the benchmark revenue from using the reserve price $\tau = \phi^{-1}(R(r^{*}))$ is untractable. Instead, we prove that there exist a simpler reserve price $\tau \geq r^{*}$ that still provides improved robustness against the benchmark.

Similarly as before, we prove that there is a distribution outside of MHR where we can not approximate the benchmark revenue. By denoting as $\overline{\mathcal{R}}$ the revenue we can extract from the above class of mechanisms we get the following.
\begin{corollary}
    There exists a class of regular, but not MHR, distributions for which the ratio $\frac{\mathcal{R}_s + \mathcal{R}_p}{\overline{\mathcal{R}}}$ is arbitrarily large.
\end{corollary}
Next similarly as before we establish a series of tools specifically for MHR distributions. First we upper bound the benchmark revenue $\mathcal{R}_p + \mathcal{R}_s$ with the sum of the social welfare and the revenue with Myerson's reserve:
\begin{lemma}\label{make-my-life-easy-lemma}
    For any MHR distribution $F \in \mathcal{F}$, let $u$ denote a valuation drawn from $F$, $r^{*}$ be the Myerson's reserve and $R$ the revenue function, then it holds that
    \[
    \mathcal{R}_p + \mathcal{R}_s \leq \mathbb{E}_{u \sim F}[u] + R(r^{*})
    \]
\end{lemma}
Next we use a known lemma from \cite{hartline2013mechanism} where we can approximate the social welfare with the revenue obtained by using a reserve price $\tau \geq r^{*}$.
\begin{lemma}[Lemma 4.38 in \cite{hartline2013mechanism}]\label{hartline-lemma}
    For any MHR distribution $F \in \mathcal{F}$, let $u$ denote a valuation drawn from $F$, $r^{*}$ be the Myerson's reserve and $R$ the revenue function, then for any $\tau \in \mathbb{R}^{+}$ such that $\tau \geq r^{*}$ it holds that
    \[
        R(\tau) \geq \frac{1}{e} \cdot \mathbb{E}_{u \sim F}[u]
    \]
\end{lemma}
Specifically in \cite{hartline2013mechanism} the bound is about revenue with just the reserve price $r^{*}$. In the Supplementary Material we provide a slight refinement of their proof and show that the bound also holds for any $\tau \geq r^{*}$. 

Next let $M_{\tau}$ be the mechanism that first uses some reserve price $\tau \geq r^{*}$ for the strategic bidder and if her bid falls below it, then offers the Myerson's reserve to the profiled agent. We show that there exists a $\tau$ such that the expected revenue from $M_{\tau}$ provides a constant approximation to our revenue benchmark.
\begin{theorem}
    Let $\tau \geq r^{*}$. Then $M_{\tau}$ is a DSIC mechansim ($\mathcal{B},\mathbf{x},\mathbf{p}$) that is $\frac{1}{e}$-consistent and $\frac{1}{e}$-robust.
\end{theorem}
\begin{proof}
     Let $\tau = \max(F^{-1}(\frac{1}{e}), r^{*})$ and $r^{*} = \phi^{-1}(0)$. First, from Lemma \ref{make-my-life-easy-lemma} we make use of Lemma
     \ref{hartline-lemma} with the reserve price $\tau$ since by definition it holds that $\tau \geq r^{*}$ and we extend the upper bound for the revenue benchmark as
    \begin{equation}\label{revenue-bench-upper}
        \mathcal{R}_s + \mathcal{R}_p \leq e \cdot \tau \cdot (1-F(\tau)) + R(r^{*})
    \end{equation}
    Let $\overline{\mathcal{R}}$ denote the revenue of $M_{\tau}$, we have
    \begin{equation}\label{revenue-formula}
        \overline{\mathcal{R}} = \tau \cdot (1-F(\tau)) + R(r^{*}) \cdot F(\tau)
    \end{equation}
    We consider the two cases on the possible values of $\tau$. If $F^{-1}(\frac{1}{e})\geq \phi^{-1}(0)$ we have that $\tau = F^{-1}(\frac{1}{e})$ and by substituting on (\ref{revenue-formula}) we get:
    \begin{equation}\label{9}
        \overline{\mathcal{R}} = F^{-1}\bigg(\frac{1}{e}\bigg) \cdot \bigg(1-\frac{1}{e}\bigg) + R(r^{*}) \cdot \frac{1}{e}
    \end{equation}
    Then by substituting on (\ref{revenue-bench-upper}) with $\tau = F^{-1}(\frac{1}{e})$ and using equality (\ref{9}) we get
    \begin{align*}
        \mathcal{R}_s + \mathcal{R}_p &\leq e \cdot F^{-1}\bigg(\frac{1}{e}\bigg) \cdot \bigg(1-\frac{1}{e}\bigg) + R(r^{*}) \\
        &= e \cdot \bigg(F^{-1}\bigg(\frac{1}{e}\bigg) \cdot \bigg(1-\frac{1}{e}\bigg) + R(r^{*})\cdot \frac{1}{e}\bigg) \\
        &= e \cdot\overline{\mathcal{R}}
    \end{align*}
    hence the we get the $\frac{1}{e}$-approximation to the benchmark revenue in this case. 
    
    In the second case, where it holds $F^{-1}(\frac{1}{e})< \phi^{-1}(0)$, then $\tau = r^{*}$ and from the previous upper bound (\ref{revenue-bench-upper}) we now get that
    \begin{equation}\label{10}
        \mathcal{R}_s + \mathcal{R}_p \leq (1 + e) \cdot R(r^{*})
    \end{equation}
    Note that since $F^{-1}(\frac{1}{e})< \phi^{-1}(0)$ we have that $F(\phi^{-1}(0)) > \frac{1}{e}$. Therefore for the revenue $\overline{\mathcal{R}}$ from (\ref{revenue-formula}) we get 
    \begin{align*}
        \overline{\mathcal{R}} &= (1 + F(\phi^{-1}(0))) \cdot R(r^{*}) \\
        &\geq \frac{1 + F(\phi^{-1}(0))}{1 + e} \cdot (\mathcal{R}_s + \mathcal{R}_p)  \\ 
        &\geq \frac{1 + \frac{1}{e}}{1 + e} \cdot (\mathcal{R}_s + \mathcal{R}_p)\\
        &\geq \frac{1}{e} \cdot (\mathcal{R}_s + \mathcal{R}_p)
    \end{align*}
    In the second inequality we used the upper bound from (\ref{10}). Thus, we conclude that in either case the auction $M_{\tau}$ yields an $\frac{1}{e}$-approximation benchmark revenue.
\end{proof}
Thus, we conclude that the mechanism $M_{\tau}$ actually has better robustness than $M_{\text{opt-c}}$, but at the expense of consistency. In order to balance between these two objectives, we use the following randomized algorithm, where with probability $\lambda$ we run mechanism $M_{\text{opt-c}}$, otherwise $M_{\tau}$ and get the Pareto frontier.
\begin{corollary}
    Let $0 \leq \lambda \leq 1$. There exists a randomized DSIC mechanism ($\mathcal{B},\mathbf{x},\mathbf{p}$) that is $\left(\frac{1}{e} + (\frac{e-1}{e}) \cdot \lambda\right)$-consistent and $\left(\frac{1}{e} - \frac{3}{4e}\cdot \lambda\right)$-robust.
\end{corollary}
Thus by choosing $\lambda$ appropriately we can obtain a mechanism that is geared towards consistency or robustness.
\section{Conclusion}
We initiated the study of auction environments with a strategic bidder and a profiled agent, yet another environment where the algorithms with predictions framework has application. We believe that such auctions are of independent interest and several future directions exist. For instance, we can consider the multi-agent and multi-bidder case, where it's not clear how the correlations of the predictions with the valuations of each of the profiled agents will behave in the worst-case of robustness. Furthermore, we defined a prediction model tailored for Bayesian settings that allows us to perform average analysis along consistency and robustness. Potentially this prediction model can be leveraged for other problems in a Bayesian setting, besides auctions. 

\bibliographystyle{apalike}
\bibliography{refs}

%\clearpage
\appendix
\section{Missing Proofs from Section 2}
\paragraph{Proof of Proposition 1.}
If the prediction input is always correct, i.e. $\widehat{v} = v_p$ then the auctioneer knows the true valuation of the profiled agent, and essentially she becomes an honest bidder who submits her true valuation to the auction. This is because the auctioneer could always retrieve her true valuation as if she was an honest bidder, just by offering her the prediction value. Thus from Lemma 2 the total expected revenue is given as 
\[
\mathbb{E}_{\mathbf{v}\sim F}\left[\sum_{i \in \{p,s\}}p_i(\mathbf{v})\right] = \mathbb{E}_{\mathbf{v}\sim F}[\phi(v_s)\cdot x_s(\mathbf{v}) + v_p\cdot x_p(\mathbf{v})]
\]
Thus, in order to deduce the revenue-optimal mechanism we maximize point-wise. We achieve this by first collecting the bid of the strategic bidder $\phi(v_s)$ and then comparing it with the realization of the prediction $\widehat{v}=v_p$. If $\phi(v_s) \geq \widehat{v}$, then in order to maximize point-wise we should give the item to the strategic bidder, i.e $x_s(\mathbf{v}) = 1$. Otherwise, if $\phi(v_s) < \widehat{v}$, then we will extract higher revenue by offering $\widehat{v}$ to the profiled agent, which she is going to accept, since the prediction is correct.

Note that since we assume MHR distributions, then due to the regularity of the virtual valuation function the rule is monotone non-decreasing and satisfies the DSIC condition. Next, the payment for the strategic bidder, by following Lemma 1 will be given by her critical bid, $\phi^{-1}(v_p)$. The payment for the profiled agent would simply be her true valuation $v_p$ since in this case she acts as a honest bidder and in order to extract as much revenue as possible we charge her the true valuation given by the prediction, which is the highest offer we can make and she will accept. Furthermore, we do not have to handle the case where the virtual valuation function can be negative, since due to regularity and the fact that $F$ has support $\mathbb{R}^{+}$, $v_p$ would simply be higher and the strategic bidder would not get the item anyway. \qed

\section{Missing Proofs from Section 3}
\paragraph{Proof of Corollary 2.1.}
    For any $c > 1$, consider the distribution $F(z) = 1 - \frac{1}{(z + 1)^c}$ with density $f(z) = \frac{c}{(z+1)^{c+1}}$
    ; we can calculate its virtual valuation function as
    \begin{equation*}
        \phi(z) = z - \frac{z + 1}{c} = z\cdot\bigg(1 - \frac{1}{c}\bigg) - \frac{1}{c}
    \end{equation*}
    The hazard rate is given as $h(z) = \frac{c}{z+1}$ which indicates that $F$ is not a MHR distribution, since $h$ it's a decreasing function of $z$ for any constant $c > 1$, but the virtual valuation function $\phi$ is clearly increasing thus it is a regular distribution.
    
    Moving on, we calculate the inverse valuation function as
    \begin{equation*}
        \phi^{-1}(z) = \frac{z + \frac{1}{c}}{1 - \frac{1}{c}} = \frac{z\cdot c + 1}{c - 1}
    \end{equation*}
    And also we have
    \begin{equation*}
        F(\phi^{-1}(z)) = 1 - \frac{1}{\left(\frac{z\cdot c + 1}{c - 1} + 1\right)^c} = 1 - \frac{1}{c^c \cdot \left(\frac{u+1}{c-1}\right)^c} = 1 -\left(\frac{c-1}{c}\right)^c \cdot \frac{1}{(z+1)^c}
    \end{equation*}
    Now we are ready to separately calculate the expected revenue of the strategic bidder $\mathcal{R}_s$ and the expected revenue of the honest bidder $\mathcal{R}_p$. We have,
    \begin{align*}
        \mathcal{R}_s &= \int_0^{\infty} \phi^{-1}(z) \cdot (1 - F(\phi^{-1}(z)))\cdot f(z) \, dz \\
        &= \int_0^{\infty}\frac{z\cdot c + 1}{c - 1} \cdot \left(\frac{c-1}{c}\right)^c \cdot \frac{c}{(z+1)^{2c + 1}} \,dz \\
        &= \left(\frac{c-1}{c}\right)^{c-1} \cdot \int_0^{\infty} \frac{z\cdot c + 1 + c - c }{(z+1)^{2c + 1}} \, dz \\
        &= \left(\frac{c-1}{c}\right)^{c-1} \cdot \left(\int_0^{\infty} \frac{c}{(z+1)^{2c}} \,dz + \int_0^{\infty} \frac{1-c}{(z+1)^{2c+1}} \,dz \right)
    \end{align*}
    Next we calculate both of these integrals, thus
    \begin{align*}
        \mathcal{R}_s &= \left(\frac{c-1}{c}\right)^{c-1} \cdot \left(c\cdot \frac{(z+1)^{-2c + 1}}{-2c + 1}\Big|_0^{\infty} + (1- c)\cdot \frac{(z+1)^{-2c}}{-2c}\Big|_0^{\infty}\right) \\
        &= \left(\frac{c-1}{c}\right)^{c-1} \cdot \left(\frac{c}{2c - 1} - \frac{c-1}{2c}\right) \\
        &= \left(\frac{c-1}{c}\right)^{c-1} \cdot \left(\frac{2c^2 - (c-1)\cdot(2c-1)}{(2c-1)\cdot 2c}\right) \\
        &= \left(\frac{c-1}{c}\right)^{c-1} \cdot \left(\frac{3c - 1}{2c \cdot(2c - 1)}\right)
    \end{align*}
    Now, we move on to the calculation of the honest bidder revenue $\mathcal{R}_p$. Similarly we have
    \begin{align*}
        \mathcal{R}_p &= \int_0^{\infty} z \cdot F(\phi^{-1}(z)) \cdot f(z) \, dz \\ 
        &= \int_0^{\infty}\frac{z \cdot c}{(z+1)^{c+1}} \,dz - \left(\frac{c-1}{c}\right)^c \cdot \int_0^{\infty}\frac{z\cdot c}{(z+1)^{2c+1}} \,dz \\
        &= \int_0^{\infty}\frac{c}{(z+1)^c} \, dz - \int_0^{\infty}\frac{c}{(z+1)^{c+1}} \,dz - \left(\frac{c-1}{c}\right)^c \cdot \left(\int_0^{\infty}\frac{c}{(z+1)^{2c}} \,dz - \int_0^{\infty}\frac{c}{(z+1)^{2c+1}} \,dz\right) \\
        &= c \cdot \frac{(z+1)^{-c+1}}{-c+1}\Big|_0^{\infty} - c \cdot \frac{(z+1)^{-c}}{-c}\Big|_0^{\infty} - \left(\frac{c-1}{c}\right)^c \cdot \left(c \cdot \frac{(z+1)^{-2c+1}}{-2c+1}\Big|_0^{\infty} - c \cdot \frac{(z+1)^{-2c}}{-2c}\Big|_0^{\infty}\right) \\
        &= \frac{c}{c-1} - 1 - \left(\frac{c-1}{c}\right)^c \cdot \left(\frac{c}{2c - 1} - \frac{1}{2}\right) \\
        &= \frac{1}{c-1} - \left(\frac{c-1}{c}\right)^c \cdot \frac{1}{4c-2}
    \end{align*}
    Now let's consider the approximation ratio
    \begin{align*}
        \frac{\mathcal{R}_s + \mathcal{R}_p}{\mathcal{R}_s} &= 1 + \frac{\mathcal{R}_p}{\mathcal{R}_s} \\ 
        &= 1 + \frac{\frac{1}{c-1}}{\left(\frac{c-1}{c}\right)^{c-1} \cdot \left(\frac{3c - 1}{2c \cdot(2c - 1)}\right)} - \left(\frac{c-1}{c}\right)\cdot \frac{\frac{1}{4c-2}}{\frac{3c-1}{c\cdot (4c - 2)}} \\
        &= \left(\frac{c}{c-1}\right)^c \cdot \frac{c \cdot (4c-2)}{3c-1} - \frac{c-1}{3c-1}
    \end{align*}
    Now if we let $c \rightarrow 1^{+}$, then we get that $\frac{\mathcal{R}_s + \mathcal{R}_p}{\mathcal{R}_s} \rightarrow \infty$, therefore if the valuations and the prediction follow this particular distribution, which is regular, but not MHR, then we are not able to approximate the revenue benchmark in the case of the robustness. \qed

    \paragraph{Proof of Corollary 3.1.}
    We know that 
    \[
    \mathcal{R}_s = \int_0^{\infty} \phi^{-1}(z) \cdot (1 - F(\phi^{-1}(z))) \cdot f(z) \,dz
    \]
     On the other hand it holds that $\mathbb{E}_{u \sim F}[R(u)] = \int_0^{\infty} z \cdot (1 - F(z))\cdot f(z) \,dz$. Also, by the definition of the inverse valuation function we have $\forall u\geq0$ it holds that $\phi^{-1}(u) \geq u$. Finally, since $f$ is a positive density function we get $\forall z\geq0$, that $z\cdot f(z) \geq 0$. Multiplying both sides by $z\cdot f(z)$ and integrating we get the desired.\qed


\section{Missing Proofs from Section 4}

\paragraph{Proof of Corollary 5.1.}
For any $c > 1$, we consider the same distribution as in Corollary 2.1, i.e. $F(z) = 1 - \frac{1}{(z+1)^c}$. Then we know the revenue function that first offers a reserve price $\tau$ and if the bidder doesn't make it, offers Myerson's reserve $r^{*} = \phi^{-1}(0)$ to the other one, has revenue given as:
\begin{align*}
    \overline{\mathcal{R}} &= \tau \cdot (1-F(\tau)) + r^{*}\cdot(1-F(r^{*}))\cdot F(\tau) \\
    &= \frac{\tau}{(\tau+1)^c} + \frac{(c-1)^{c-1}}{c^c}\cdot \left(1 - \frac{1}{(\tau + 1)^c}\right)
\end{align*}
From corollary 2.1 we know that:
\[
\mathcal{R}_s + \mathcal{R}_p = \left(\frac{c-1}{c}\right)^{c-1} \cdot \left(\frac{3c - 1}{2c \cdot(2c - 1)}\right) +\frac{1}{c-1} - \left(\frac{c-1}{c}\right)^c \cdot \frac{1}{4c-2} 
\]
For $c \rightarrow 1^{+}$ and any $\tau \in \mathbb{R}^{+}$, we can easily verify that:
\[
\lim_{c \rightarrow 1^{+}} \frac{\mathcal{R}_s + \mathcal{R}_p}{\overline{\mathcal{R}}} = \infty
\] \qed


\paragraph{Proof of Lemma 6.}
 First we can expand the strategic revenue $\mathcal{R}_s$ as follows
\begin{align*}
    \mathcal{R}_s &= \int_0^{\infty}\phi^{-1}(z)\cdot f(z) \, dz -\int_0^{\infty}\phi^{-1}(z)\cdot F(\phi^{-1}(z))\cdot f(z) \, du \nonumber \\
    &= \mathbb{E}_{u \sim F}[\phi^{-1}(u)] - \int_0^{\infty}z \cdot F(\phi^{-1}(z))\cdot f(z) \, dz - \int_0^{\infty} \frac{1}{h(\phi^{-1}(z))}\cdot F(\phi^{-1}(z))\cdot f(z) \, dz
\end{align*}
In the second line, we used the definition of the inverse valuation function given by $\phi^{-1}(z) = z + \frac{1}{h(\phi^{-1}(z))}$, where $h$ is the hazard rate. On the other hand we have:
\[
    \mathcal{R}_p = \int_0^{\infty} z \cdot F(\phi^{-1}(z)) \cdot f(z) \, dz
\]
Then using the transformation $1 - F(\phi^{-1}(z)) = e^{-H(\phi^{-1}(z))}$, we have the following simplification for the total expected revenue:
\begin{align}\label{12}
    \mathcal{R}_s + \mathcal{R}_p &= \mathbb{E}_{u \sim F}[\phi^{-1}(u)] - \int_0^{\infty}\frac{1}{h(\phi^{-1}(z))}\cdot F(\phi^{-1}(z)) \cdot f(z) \, dz\nonumber \\
    &= \mathbb{E}_{u \sim F}[\phi^{-1}(u)] - \int_0^{\infty}\frac{1}{h(\phi^{-1}(z))} \cdot (1 - e^{-H(\phi^{-1}(z))})\cdot f(z) \, dz \nonumber \\
    &= \mathbb{E}_{u \sim F}[\phi^{-1}(u)] - \int_0^{\infty}\frac{1}{h(\phi^{-1}(z))}\cdot f(z) \, dz + \int_0^{\infty} \frac{1}{h(\phi^{-1}(z))}\cdot e^{-H(\phi^{-1}(z))}\cdot f(z) \, dz \nonumber \\
    &= \mathbb{E}_{u \sim F}[u] + \int_0^{\infty} \frac{1}{h(\phi^{-1}(z))}\cdot e^{-H(\phi^{-1}(z))}\cdot f(z) \, dz \nonumber \\
    &= \int_0^{\infty} (1-F(z)) \, dz + \int_0^{\infty} \frac{1}{h(\phi^{-1}(z))}\cdot e^{-H(\phi^{-1}(z))}\cdot f(z) \, dz 
\end{align}
Now note that from the MHR assumption, since the hazard rate $h$ and the inverse valuation function $\phi^{-1}$ are non-decreasing, we have for $\forall z \geq 0$ that $\frac{1}{h(\phi^{-1}(z))} \leq \frac{1}{h(\phi^{-1}(0))}$ and $e^{-H(\phi^{-1}(z))} \leq e^{-H(\phi^{-1}(0))}$. Therefore from (\ref{12}) we derive the following bound.
\begin{align*}
    \mathcal{R}_s + \mathcal{R}_p &\leq \int_0^{\infty} (1-F(z)) \,dz + \frac{1}{h(\phi^{-1}(0))}\cdot e^{-H(\phi^{-1}(0))}\int_0^{\infty} f(z) \, dz \nonumber \\
    &= \int_0^{\infty} (1-F(z)) \,dz + \phi^{-1}(0)\cdot (1-F(\phi^{-1}(0)) \\
    &= \mathbb{E}_{u \sim F}[u] + R(r^{*})
\end{align*}
\qed

\paragraph{Proof of Lemma 7.}
    Let $H(z) = \int_0^{z}h(y) \, dy$ be the cumulative hazard rate of the distribution $F$ and we know that $1 - F(z) = e^{-H(z)}$. Furthermore, we have
    \begin{align}\label{3}
        \mathbb{E}_{u \sim F}[u] &= \int_0^{\infty}(1-F(z)) \, dz \nonumber \\
        &= \int_0^{\infty}e^{-H(z)} \, dz
    \end{align}  
    Note that, due to the MHR assumption, the cumulative hazard rate $H(z)$ is a convex function, so we can get a lower bound of $H(z)$ by the line tangent to it at $\tau$, as follows
    \begin{align}\label{4}
        H(z) &\geq H(\tau) + h(\tau)\cdot (z - \tau) \nonumber \\
        &\geq H(\tau) + \frac{z - \tau}{\tau}
    \end{align}
    In the first inequality we used the tangent line and in the second one we have $h(\tau) = \frac{1}{\tau - \phi(\tau)} \geq \frac{1}{\tau}$, since $\tau \geq \phi^{-1}(0) \Rightarrow \phi(\tau) \geq 0$. So by substituting (\ref{4}) into (\ref{3}), we get:
    \begin{align}
        \mathbb{E}_{u \sim F}[u] &\leq \int_0^{\infty}\exp(-H(\tau) - \frac{z}{\tau} + 1) \, dz \nonumber \\
        &= e \cdot e^{-H(\tau)} \cdot \int_0^{\infty}e^{\frac{z}{\tau}} \, dz \nonumber \\
        &= e \cdot e^{-H(\tau)} \cdot \tau \nonumber \\
        &= e \cdot (1 - F(\tau)) \cdot \tau \nonumber \\
        &= e \cdot R(\tau)
    \end{align}
    which gives the desired.\qed

\paragraph{Proof of Corollary 8.1.}
We run mechanism $M_{\text{opt-c}}$ which is 1-consistent and $\frac{1}{4e}$-robust with probability $\lambda$ and otherwise we run mechanism $M_{\tau}$ which is $\frac{1}{e}$-consistent and $\frac{1}{e}$-robust. Thus, in total we get the expected consistency of:
\[
\lambda \cdot 1 + (1-\lambda)\cdot \frac{1}{e} = \lambda + \frac{1}{e} - \frac{\lambda}{e} = \frac{1}{e} + \left(\frac{e-1}{e}\right)\cdot \lambda
\]
and the expected robustness of:
\[
\lambda \cdot \frac{1}{4e} + (1-\lambda)\cdot \frac{1}{e} = \frac{1}{e} - \left(\frac{\lambda}{e} - \frac{\lambda}{4e}\right) = \frac{1}{e} - \frac{3}{4e}\cdot \lambda 
\]\qed


\end{document}
\typeout{get arXiv to do 4 passes: Label(s) may have changed. Rerun}
