\section{Conclusion} \label{sec:conclusion}
This paper proposed the problem of restoring a pruned CNN in a way free of training data and fine-tuning. We mathematically formulated how filter pruning can make a damage to the output of the pruned network, and focused on how to model the information carried by each pruned filter to be delivered to the other remaining filters. Our proposed assumption is inspired by the fact that the more the remaining filters participate in the recovery process, the better the approximation we can obtain for the original output. With this assumption, we successfully decomposed the reconstruction error into the three different components, and thereby designed a data-free loss function along with its closed form solution. Our future work would be to extend the proposed loss function so as to cover different activation functions other than ReLU.