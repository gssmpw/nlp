
\begingroup\raggedleft
\begin{gpt}{\ooneheader}
A convenient way to produce such "self-describing" sequences is via the well-known self-describing numbers 1210, 2020, 21200, 3211000, and so on.  Interpreting each digit as an entry $x_0$, $x_1$, ..., $x_n$ in the sequence, one finds that $x_j$ counts precisely how many times $j$ appears.  Here are 10 explicit examples:
(1, 2, 1, 0)  $\cdot$ (2, 0, 2, 0)  $\cdot$ (2, 1, 2, 0, 0)  $\cdot$ (3, 2, 1, 1, 0, 0, 0)  $\cdot$ (4, 2, 1, 0, 1, 0, 0, 0)  $\cdot$ (5, 2, 1, 0, 0, 1, 0, 0, 0)  $\cdot$ (6, 2, 1, 0, 0, 0, 1, 0, 0, 0)  $\cdot$ (7, 2, 1, 0, 0, 0, 0, 1, 0, 0, 0)  $\cdot$ (8, 2, 1, 0, 0, 0, 0, 0, 1, 0, 0, 0)  $\cdot$ (9, 2, 1, 0, 0, 0, 0, 0, 0, 1, 0, 0, 0)\n\nIn boxed form, they can be written as:\n
$\boxed{(1,2,1,0), (2,0,2,0), (2,1,2,0,0), (3,2,1,1,0,0,0),}$
$\boxed{(4,2,1,0,1,0,0,0), (5,2,1,0,0,1,0,0,0),}$
$\boxed{(6,2,1,0,0,0,1,0,0,0),(7,2,1,0,0,0,0,1,0,0,0),}$
$\boxed{(8,2,1,0,0,0,0,0,1,0,0,0),}$
$\boxed{(9,2,1,0,0,0,0,0,0,1,0,0,0)}.$
\end{gpt}
\endgroup

\begin{parsecheckcorrect}{Parsed Answer - Correct}
[[1, 2, 1, 0], [2, 0, 2, 0], [2, 1, 2, 0, 0], [3, 2, 1, 1, 0, 0, 0], [4, 2, 1, 0, 1, 0, 0, 0], [5, 2, 1, 0, 0, 1, 0, 0, 0], [6, 2, 1, 0, 0, 0, 1, 0, 0, 0], [7, 2, 1, 0, 0, 0, 0, 1, 0, 0, 0], [8, 2, 1, 0, 0, 0, 0, 0, 1, 0, 0, 0], [9, 2, 1, 0, 0, 0, 0, 0, 0, 1, 0, 0, 0]]
\end{parsecheckcorrect}
