\begin{system}{System}
You are an expert mathematician that solves problems using a brute-force approach. You have been given access to a Python interpreter that you can use to generate your brute-force approach.
Make sure to abide by the following rules:
    1. You should write the code in a Python code block starting with '```python' and ending with the '```' sequence followed by the end of your response. 
    2. The code should be a brute-force solution without optimizations. The user needs to know whether the problem can be solved in a brute-force approach.
    3. After you have written your code, you can immediately stop your output. The user will execute your code and give you the results. If the result was successful, you can then format the response accurately.
    4. In your Python code, you can use 'print()' to save whatever important information you will then need to format your final reply. The user will then execute your code in a Python environment.
    5. Your code will run for two minutes. If the code times out, you will receive a timeout error along with all print statements received before timing out.
    6. Before generating code, perform reasoning to ensure your solution will attempt a valid brute-force approach.
    7. If your program finishes without a valid solution b  efore the two-minute timeout, you have not done your job well.
\end{system}

\begin{prompt}{User}
Given $n = 27$, find a sequence of distinct real numbers $x_1, \ldots, x_n$ such that each of the sets$\{x_1, x_2, x_3\}, \{x_2, x_3, x_4\},\ldots,\{x_{n-2}, x_{n-1}, x_n\}, \{x_{n-1}, x_n, x_1\},\text{ and } \{x_n, x_1, x_2\}$ forms a 3-term arithmetic progression when arranged in increasing order.
\end{prompt}

\begingroup\raggedleft
\begin{claudels}{\claudeheader}
Let me reason about this problem:  
    1) We need to find 27 distinct numbers where each consecutive triplet forms an arithmetic sequence when sorted 
    2) The triplets wrap around, meaning the last two numbers connect with the first ones 
    3) A brute-force approach would be to:    
        - Start with some small initial values    
        - Incrementally try values that satisfy each consecutive triplet    
        - Keep track of used values to ensure distinctness    
        - Check if it wraps around correctly  
    4) For a brute-force solution, let's:    
        - Start with small initial values (x\u2081, x\u2082)    
        - For each next value, ensure it can form an AP with previous two when sorted    
        - Use reasonable bounds to limit search space    
        - Check for distinctness  
Let me write code to search for such a sequence:  

```python 
def is_arithmetic_when_sorted(a, b, c):     
    nums = sorted([a, b, c])     
    return abs((nums[1] - nums[0]) - (nums[2] - nums[1])) < 1e-10  

def check_sequence(seq):     
    n = len(seq)
    for i in range(n):
        if not is_arithmetic_when_sorted(seq[i], seq[(i+1)\%n], seq[(i+2)\%n]):             
            return False     
    # Check distinctness     
    return len(set(seq)) == len(seq)  

def find_sequence(n=27):     
    def extend_sequence(current_seq):         
        if len(current_seq) == n:             
            if check_sequence(current_seq):                 
                print(','.join(map(str, current_seq)))                 
                return True             
            return False 

        prev1, prev2 = current_seq[-2:]

        # Try values that could potentially form arithmetic sequence         
        used = set(current_seq)

        # Try different possibilities for next value         
        for multiplier in range(-10, 11):             
            # Try extending based on arithmetic patterns            
            candidates = [                 
                2 * prev2 - prev1 + multiplier,  # Continue AP in one direction                 
                2 * prev1 - prev2 + multiplier,  # Continue AP in other direction                 
                (prev1 + prev2)/2 + multiplier   # Middle value             
            ]                          
            for next_val in candidates:                 
                if next_val not in used and -100 <= next_val <= 100:                     
                    if is_arithmetic_when_sorted(prev1, prev2, next_val):                         
                        if extend_sequence(current_seq + [next_val]):                             
                            return True         
        return False   
       
# Try different starting values     
for start in range(-5, 6):         
    for second in range(-5, 6):             
        if start != second:                 
            if extend_sequence([start, second]):                     
                return  find_sequence() ```
\end{claudels}
\endgroup

\begin{prompt}{User}
Code Output: 
```-5,-4,-6,-8,-10,-12,-14,-16,-18,-20,-22,-24,-26,-28,-30,-32,-34,-33,-35,-31,-27,-23,-19,-15,
-11,-7,-3```
This was the last time your code can be executed. From now on, you will not be able to execute code.
\end{prompt}

\begingroup\raggedleft
\begin{claudels}{\claudeheader}
Based on the output, I can now provide the solution.
$\boxed{-5,-4,-6,-8,-10,-12,-14,-16,-18,-20,-22,-24,-26,-28,-30,-32,-34,-33,-35,-31,-27,-23,-19,-15,}$
$\boxed{-11,-7,-3}$
This sequence satisfies all the required conditions:
    1. It contains 27 distinct numbers
    2. Each consecutive triplet forms an arithmetic sequence when arranged in increasing order
    3. The wrap-around condition is also satisfied (the last two numbers with the first, and the last number with the first two)
\end{claudels}
\endgroup

\begin{parsecheckcorrect}{Parsed Answer - Correct}
[-5, -4, -6, -8, -10, -12, -14, -16, -18, -20, -22, -24, -26, -28, -30, -32, -34, -33, -35, -31, -27, -23, -19, -15, -11, -7, -3]
\end{parsecheckcorrect}

