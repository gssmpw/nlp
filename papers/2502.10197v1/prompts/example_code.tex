
\begin{prompt}{Prompt}
Let $n \geq 2$ be an integer. Consider an $n \times n$ chessboard divided into $n^2$ unit squares.
We call a configuration of $n$ rooks on this board *happy* if every row and every column contains exactly one rook. 
For $n=22$ and $k=5$, find a happy configuration of rooks such that there is no $k \times k$ square without a rook on any of its $k^2$ unit squares.

Format your reply as follows:
Output a comma-separated list of 1-based rook indices (so (1, 1) is the top left corner of the board) inside of $\boxed{...}$, for example $\boxed{((1,1), (4,5), (1,6))}$.
\end{prompt}

\begingroup\raggedleft
\begin{geminils}{\geminiflheader}
Okay, I understand the problem. We need to find a happy configuration of 22 rooks on a 22x22 chessboard such that no 5x5 square is empty of rooks. A happy configuration means there's exactly one rook in each row and each column. This is equivalent to a permutation of the numbers from 1 to 22. The condition that no 5x5 square is empty is the crucial constraint. A simple permutation like (1, 1), (2, 2), ..., (22, 22) would not satisfy this condition. If we place the rooks on the main diagonal, then any 5x5 subgrid along the diagonal has no rooks in the upper-right 4x4 corner or the bottom-left 4x4 corner, so it does not work. To ensure no 5x5 square is empty, we need to spread the rooks out somewhat evenly. A good starting point might be to place the rooks on the main diagonal and then offset each rook by a certain amount. Consider placing rooks at (i, a * i mod n + 1) for a constant a. 
Let's try with a=1. The rooks are at (i, i). This doesn't work. 
Let's try a=2. (i, 2*i mod 22 + 1) 
Let's try a=3. (i, 3*i mod 22 + 1).
Let's try the configuration where the $i$-th rook is at position $(i, 3i \\pmod{22} + 1)$, where the indices are 1-based. This is guaranteed to have one rook per row. We need to check if there is one rook per column. If we map $i \\to 3i \\pmod{22}$, this map is a permutation since 3 and 22 are coprime. Hence this is a valid rook configuration.
Now we need to see if there's a 5x5 subgrid with no rooks. I will generate the configuration for $n=22$ using $a=3$ and then visually check whether it might work. 
```python
n = 22
a = 3
rooks = []
for i in range(1, n + 1):
    col = (a * i) \% n + 1
    rooks.append((i, col))
    
    print(rooks)

def check_5x5_empty(rooks, n):
    for r in range(1, n - 4 + 1):
        for c in range(1, n - 4 + 1):
            empty = True
            for row in range(r, r + 5):
                for col in range(c, c + 5):
                    if (row, col) in rooks:
                        empty = False
                        break
                if not empty:
                    break
            if empty:
                print(f"5x5 empty square found at ({r}, {c})")
                return False
    return True

print(check_5x5_empty(rooks, n))```
\end{geminils}
\endgroup

\begin{prompt}{User}
Code Output:
```[(1, 4), (2, 7), (3, 10), (4, 13), (5, 16), (6, 19), (7, 22), (8, 3), (9, 6), (10, 9), (11, 12), (12, 15), (13, 18), (14, 21), (15, 2), (16, 5), (17, 8), (18, 11), (19, 14), (20, 17), (21, 20), (22, 1)]
5x5 empty square found at (1, 17)
False
\end{prompt}

\begingroup\raggedleft
\begin{geminils}{\geminiflheader}
Okay, the code output shows that the configuration with $a=3$ doesn't work. It found a 5x5 empty square. Let's try $a=7$.
```python
n = 22
a = 7
rooks = []
for i in range(1, n + 1):
    col = (a * i) \% n + 1
    rooks.append((i, col))
    
    print(rooks)

def check_5x5_empty(rooks, n):
    for r in range(1, n - 4 + 1):
        for c in range(1, n - 4 + 1):
            empty = True
            for row in range(r, r + 5):
                for col in range(c, c + 5):
                    if (row, col) in rooks:
                        empty = False
                        break
                if not empty:
                    break
            if empty:
                print(f"5x5 empty square found at ({r}, {c})")
                return False
    return True

print(check_5x5_empty(rooks, n))```
\end{geminils}
\endgroup

\begin{prompt}{User}
Code Output:
```[(1, 8), (2, 15), (3, 22), (4, 7), (5, 14), (6, 21), (7, 6), (8, 13), (9, 20), (10, 5), (11, 12), (12, 19), (13, 4), (14, 11), (15, 18), (16, 3), (17, 10), (18, 17), (19, 2), (20, 9), (21, 16), (22, 1)]
5x5 empty square found at (1, 1)
False
\end{prompt}

\begingroup\raggedleft
\begin{geminils}{\geminiflheader}
Okay, $a=7$ also doesn't work. Let's try $a=9$.
```python
n = 22
a = 9
rooks = []
for i in range(1, n + 1):
    col = (a * i) \% n + 1
    rooks.append((i, col))
    
    print(rooks)

def check_5x5_empty(rooks, n):
    for r in range(1, n - 4 + 1):
        for c in range(1, n - 4 + 1):
            empty = True
            for row in range(r, r + 5):
                for col in range(c, c + 5):
                    if (row, col) in rooks:
                        empty = False
                        break
                if not empty:
                    break
            if empty:
                print(f"5x5 empty square found at ({r}, {c})")
                return False
    return True

print(check_5x5_empty(rooks, n))```
\end{geminils}
\endgroup

\begin{prompt}{User}
Code Output:
```[(1, 10), (2, 19), (3, 6), (4, 15), (5, 2), (6, 11), (7, 20), (8, 7), (9, 16), (10, 3), (11, 12), (12, 21), (13, 8), (14, 17), (15, 4), (16, 13), (17, 22), (18, 9), (19, 18), (20, 5), (21, 14), (22, 1)]
True
\end{prompt}
    
\begingroup\raggedleft
\begin{geminils}{\geminiflheader}
Okay, the code output shows that the configuration with $a=9$ works!
The rook positions are: (1, 10), (2, 19), (3, 6), (4, 15), (5, 2), (6, 11), (7, 20), (8, 7), (9, 16), (10, 3), (11, 12), (12, 21), (13, 8), (14, 17), (15, 4), (16, 13), (17, 22), (18, 9), (19, 18), (20, 5), (21, 14), (22, 1).
Final Answer: The happy configuration of rooks is \\boxed{((1, 10), (2, 19), (3, 6), (4, 15), (5, 2), (6, 11), (7, 20), (8, 7), (9, 16), (10, 3), (11, 12), (12, 21), (13, 8), (14, 17), (15, 4), (16, 13), (17, 22), (18, 9), (19, 18), (20, 5), (21, 14), (22, 1))}.
\end{geminils}
\endgroup


\begin{parsecheckcorrect}{Parsed Answer - Correct}
[[1, 10], [2, 19], [3, 6], [4, 15], [5, 2], [6, 11], [7, 20], [8, 7], [9, 16], [10, 3], [11, 12], [12, 21], [13, 8], [14, 17], [15, 4], [16, 13], [17, 22], [18, 9], [19, 18], [20, 5], [21, 14], [22, 1]]
\end{parsecheckcorrect}
