\subsection{Case Study Comparing \geminithink{} and \textsc{o1}}
\label{app:case_study}
\lstset{
    backgroundcolor={},
}
In this case study, we qualitatively compare the outputs of \geminithink{} and \oone{} on problems from the IMO Shortlist. 

\clearpage
\paragraph{IMO Shortlist 2001 C5}
In our main results we observe that \geminithink{} fails to solve any variation, while \oone{} solves this problem correctly for $3/4$ variations. First, looking at the \geminithink{} reasoning trace we can observe that it tries to exhaustively search for solutions, and makes mistakes in the casework. On the other hand, \oone{} immediately recognizes that the problem asks something very similar to self-describing numbers (those where digit $i$ is the number of times that $i$ appears in the number). It can recall all such numbers, and solve the problem for smaller $n$. Interestingly, for larger $n=17$ exactly the same construction works, but \oone{} fails to find it, possibly because typical examples on the internet contain only digits. 


\begin{prompt}{Prompt}
Find $10$ finite sequences $(x_0, x_1, \ldots, x_n)$ such that for every $j$, $0 \leq j \leq n$, $x_j$ equals the number of times $j$ appears in the sequence.
Format your reply as follows:
Output the sequences as comma-separated tuples inside of $\boxed{...}$, \eg $\boxed{(1, 2), (1, 2, 3), (1, 1, 1, 1, 0)}$.
\end{prompt}

\begin{minipage}[t]{0.50\textwidth}
    \begingroup\raggedleft
\begin{geminils}{\geminiheader}
Let $(x_0, x_1, \ldots, x_n)$ be a finite sequence such that for every $j$, $0 \leq j \leq n$, $x_j$ is the number of times $j$ appears in the sequence.
The length of the sequence is $n+1$. The indices are $0, 1, \ldots, n$.
... Let's consider small values of $n$.
...
List of solutions found: $(1, 0)$, $(2, 0, 0)$, $(3, 0, 0, 0)$, $(4, 0, 0, 0, 0)$, ...
Let's verify these 10 solutions.
    1. $(1, 0)$. $x_0 = 1, x_1 = 0$. #0's = 1, #1's = 0. Correct.
    2. $(2, 0, 0)$. $x_0 = 2, x_1 = 0, x_2 = 0$. #0's = 2, #1's = 0, #2's = 0. Correct.
...

Check if $(2, 0, 2, 0)$ is valid. $n=3$. $(x_0, x_1, x_2, x_3) = (2, 0, 2, 0)$. $x_0 = 2$. Number of $0$'s in $(2, 0, 2, 0)$ is $2$. Correct.$x_1 = 0$. Number of $1$'s in $(2, 0, 2, 0)$ is $0$. Correct.$x_2 = 2$. Number of $2$'s in $(2, 0, 2, 0)$ is $2$. Correct.$x_3 = 0$. Number of $3$'s in $(2, 0, 2, 0)$ is $0$. Correct.
So $(2, 0, 2, 0)$ is a solution.
...
Final answer will be these 10 tuples.

Output: $\boxed{(1, 0), (2, 0, 0), (3, 0, 0, 0), (4, 0, 0, 0, 0),...}$
$\boxed{...,(9, 0, 0, 0, 0, 0, 0, 0, 0, 0), (2, 0, 2, 0)}$.
\end{geminils}
\endgroup
\begin{parsecheckwrong}{Parsed Answer - Incorrect}
[[1, 0], [2, 0, 0], [3, 0, 0, 0], [4, 0, 0, 0, 0], [5, 0, 0, 0, 0, 0], [6, 0, 0, 0, 0, 0, 0], ..., [2, 0, 2, 0]]
\end{parsecheckwrong}

\begin{parsecheckwrong}{Parsecheck Details}
INCORRECT_SOLUTION: Sequence [1, 0] has 1 occurrences of 1, should be 0
\end{parsecheckwrong}

\end{minipage}
\hfill
\begin{minipage}[t]{0.48\textwidth}
    
\begingroup\raggedleft
\begin{gpt}{\ooneheader}
A convenient way to produce such "self-describing" sequences is via the well-known self-describing numbers 1210, 2020, 21200, 3211000, and so on.  Interpreting each digit as an entry $x_0$, $x_1$, ..., $x_n$ in the sequence, one finds that $x_j$ counts precisely how many times $j$ appears.  Here are 10 explicit examples:
(1, 2, 1, 0)  $\cdot$ (2, 0, 2, 0)  $\cdot$ (2, 1, 2, 0, 0)  $\cdot$ (3, 2, 1, 1, 0, 0, 0)  $\cdot$ (4, 2, 1, 0, 1, 0, 0, 0)  $\cdot$ (5, 2, 1, 0, 0, 1, 0, 0, 0)  $\cdot$ (6, 2, 1, 0, 0, 0, 1, 0, 0, 0)  $\cdot$ (7, 2, 1, 0, 0, 0, 0, 1, 0, 0, 0)  $\cdot$ (8, 2, 1, 0, 0, 0, 0, 0, 1, 0, 0, 0)  $\cdot$ (9, 2, 1, 0, 0, 0, 0, 0, 0, 1, 0, 0, 0)\n\nIn boxed form, they can be written as:\n
$\boxed{(1,2,1,0), (2,0,2,0), (2,1,2,0,0), (3,2,1,1,0,0,0),}$
$\boxed{(4,2,1,0,1,0,0,0), (5,2,1,0,0,1,0,0,0),}$
$\boxed{(6,2,1,0,0,0,1,0,0,0),(7,2,1,0,0,0,0,1,0,0,0),}$
$\boxed{(8,2,1,0,0,0,0,0,1,0,0,0),}$
$\boxed{(9,2,1,0,0,0,0,0,0,1,0,0,0)}.$
\end{gpt}
\endgroup

\begin{parsecheckcorrect}{Parsed Answer - Correct}
[[1, 2, 1, 0], [2, 0, 2, 0], [2, 1, 2, 0, 0], [3, 2, 1, 1, 0, 0, 0], [4, 2, 1, 0, 1, 0, 0, 0], [5, 2, 1, 0, 0, 1, 0, 0, 0], [6, 2, 1, 0, 0, 0, 1, 0, 0, 0], [7, 2, 1, 0, 0, 0, 0, 1, 0, 0, 0], [8, 2, 1, 0, 0, 0, 0, 0, 1, 0, 0, 0], [9, 2, 1, 0, 0, 0, 0, 0, 0, 1, 0, 0, 0]]
\end{parsecheckcorrect}

\end{minipage}


\clearpage
Interestingly, in one of the other attempts, \geminithink{} decides to search for exactly these sequences online, but comes up with the wrong results and eventually fails to solve the problem:

\texttt{``Searching for ``self-referential sequence'' or ``autological sequence''.}

\texttt{``Counting sequence''.}

\texttt{Consider the sequence starting with ``How many 0s, 1s, 2s, 3s, ... are there in this sentence?''}

\texttt{Wikipedia: Self-referential numeric sequence.}

\texttt{Solutions listed}

\texttt{n=1: None}

\texttt{n=2: None}

\texttt{n=3: (1,2,1,0), (2,0,2,0}

\texttt{n=4: (2,1,2,0,0), (3,2,1,1,0), (4,0,0,0,0)}

\texttt{n=5: (2,1,2,0,0,0), (3,1,2,0,0,0),''}

\clearpage
\paragraph{IMO Shortlist 2014 N2}

Another interesting case is the IMO Shortlist 2014 N2. 
Here \geminithink{} fails to solve any variation, while \oone{} solves all variations.
By inspecting the solutions deeper, we can see that both models typically make the same substitution $d = y - x$ and come to the point where they have to find $d$ such that $d^2 - 4(4d^2 - d^3 - 3d - 1)$ is a perfect square. Then, both models start plugging in small values of $d$, but given initial sequence of small solutions $6, 12, 20, 30, ...$, \oone{} recognizes that the solutions are the pronic numbers $n(n+1)$, and finds the solution for all variations. On the other hand, \geminithink{} fails to find the pattern, and is not able to find the general solution.

\begin{prompt}{Prompt}
Find 32 distinct pairs $(x, y)$ of positive integers such that $x<y$ and
$\sqrt[3]{{7x^2-13xy+7y^2}} = |x-y|+1.$
Format your reply as follows:
Output a comma-separated list of pairs $(x, y)$, inside of $\boxed{...}$, for example $\boxed{(2,3), (5,6)}$.
\end{prompt}
\vspace{-4mm}
\begin{minipage}[t]{0.50\textwidth}
    \begingroup\raggedleft
\begin{geminils}{\geminiheader}
Let the given equation be $\sqrt[3]{7x^2-13xy+7y^2} = |x-y|+1.$
Since $x<y$, we have $|x-y| = y-x$. So the equation becomes $\sqrt[3]{7x^2-13xy+7y^2} = y-x+1.$
Let $y-x = k$. Since $x<y$ are positive integers, $k = y-x$ must be a positive integer. Then $y = x+k$. Substituting $y=x+k$ into the equation, we get$\sqrt[3]{7x^2-13x(x+k)+7(x+k)^2} = k+1.$
...
Rearranging the equation as a quadratic equation in $x$, we have$x^2 + kx + (7k^2 - (k+1)^3) = 0.$$x^2 + kx + (7k^2 - (k^3+3k^2+3k+1)) = 0.$$x^2 + kx + (7k^2 - k^3 - 3k^2 - 3k - 1) = 0.$$x^2 + kx + (-k^3 + 4k^2 - 3k - 1) = 0.$We use the quadratic formula to solve for $x$:$x = \frac{-k \pm \sqrt{k^2 - 4(-k^3 + 4k^2 - 3k - 1)}}{2} =$$x = \frac{-k \pm \sqrt{4k^3 - 15k^2 + 12k + 4}}{2}.$ For $x$ to be a positive integer, we need the discriminant $D = 4k^3 - 15k^2 + 12k + 4$ to be a perfect square, and $-k \pm \sqrt{D}$ to be a positive even integer.
...
We have solutions for $k=6, 20, 30, 90, 1000$.Let's list pairs $(x, y)$.$k=6$: $(7, 13)$$k=20$: $(71, 91)$$k=30$: $(139, 169)$$k=90$: $(791, 881)$$k=1000$: $(31063, 32063)$Look at $k$ sequence: $6, 20, 30, 90, 1000$. Ratios: $20/6 \approx 3.33$, $30/20 = 1.5$, $90/30 = 3$, $1000/90 \approx 11.1$.
...
We can generate more pairs by increasing $k$ by a factor of 3 or 10. ...
Let's take first 25 pairs based on k values: k=6, 20, 30, 90, 300, 900, 1000, 3000, 9000, ... Output first 25 pairs in format (x, y)...
$\boxed{(7,13), (71,91), (139,169), (791,881), ...}$
\end{geminils}
\endgroup
\begin{parsecheckwrong}{Parsed Answer - Incorrect}
[[7, 13], [71, 91], [139, 169], [791, 881], [5014, 5314], [84755, 85655], ...]
\end{parsecheckwrong}

\begin{parsecheckwrong}{Parsecheck Details}
INCORRECT_SOLUTION: Cube root of 27274396 is not an integer
\end{parsecheckwrong}

\end{minipage}
\hfill
\begin{minipage}[t]{0.48\textwidth}
    
\begingroup\raggedleft
\begin{gpt}{\ooneheader}
A convenient way to see that there are infinitely many solutions - and to list the first 32 of them - is to set $d = y - x > 0$, so that the equation $(7x^2 - 13xy + 7y^2)^\frac{1}{3} = |x - y| + 1$ becomes $(7x^2 - 13x(x+d) + 7(x+d)^2)^\frac{1}{3} = d + 1$.
1. Rewrite the left-hand side in terms of x and d. Since $y = x + d$, $7x^2 - 13x(x+d) + 7(x+d)^2 =$
$7x^2 - 13x^2 - 13xd + 7(x^2 + 2xd + d^2) =$
$(7x^2 - 13x^2 + 7x^2) + (-13xd + 14xd) + 7d^2 =$ 
$x^2 + xd + 7d^2$. 
2. The equation then is $x^2 + xd + 7d^2 = (d + 1)^3 = d^3 + 3d^2 + 3d + 1$. Rearrange to get $x^2 + xd + 4d^2 - d^3 - 3d - 1 = 0$. 
3. View that as a quadratic in x: $x^2 + dx + (4d^2 - d^3 - 3d - 1) = 0$. For x to be an integer, its discriminant $\Delta = d^2 - 4(4d^2 - d^3 - 3d - 1)$ must be a perfect square.  One finds that $\Delta$ is a perfect square precisely when $d$ takes the values $d = 6, 12, 20, 30, 42, 56, 72, 90, \ldots$,  which turn out to be $d = (n+1)(n+2)$.
4. Solving for x in terms of n. Set $d(n) = (n+1)(n+2)$. One finds that the corresponding positive solution for x is $x(n) = n^3 + 4n^2 + 3n - 1$, and hence $y(n) = x(n) + d(n) = n^3 + 5n^2 + 6n + 1$...
---------------------
Below are the first 32 such pairs (i.e., n = 1 through n = 32). In each pair, x < y as required. $\cdot$ n=1: (7, 13) $\cdot$ n=2: (29, 41) $\cdot$ n=3: (71, 91) $\cdot$ n=4: (139, 169) ,... 
Putting them all together:
$\boxed{(7,13), (29,41),(71,91),(139,169),\ldots}$
\end{gpt}
\endgroup

\begin{parsecheckcorrect}{Parsed Answer - Correct}
[[7, 13], [29, 41], [71, 91], [139, 169], [239, 281], [377, 433], [559, 631], [791, 881], [1079, 1189], [1429, 1561], ...]
\end{parsecheckcorrect}

\end{minipage}





