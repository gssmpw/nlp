\section*{Supplementary Materials}
\label{sec:supplementary}

\begin{table}[h]%
\centering%
\caption{\textbf{ETHOS performance on the ARES tasks with a breakdown for demographic subgroups.} This table presents the predictive performance (AUROC with 95\% confidence intervals) of ETHOS (top) and MEDS-Tab (bottom) for four critical clinical outcomes used in ARES: Hospital Mortality, ICU Admission, Prolonged Hospital Stay (>10 days), and a Composite Risk Score (HM+IA+PS). The prevalence rates of each outcome are provided for reference. Performance metrics are further stratified by gender and race to assess potential disparities in model performance across demographic subgroups.}%
\label{tab:ares-results}%
\begin{tabular}{lcccc}%
\toprule%
&\textbf{Hospital Mortality}&\textbf{ICU Admission}&\textbf{Prolonged Stay}&\textbf{Composite (HM+IA+PS)}\\%
\textit{Prevalence} (\%)&1.85&15.44&9.01&20.39\\%
\toprule%
\multicolumn{5}{c}{\textbf{\Large{ETHOS}}}\\%
\toprule%
\textbf{Overall}&0.936 {[}0.930, 0.944{]}&0.927 {[}0.927, 0.929{]}&0.849 {[}0.845, 0.848{]}&0.899 {[}0.898, 0.902{]}\\%
\midrule%
\textbf{Gender}&&&&\\%
\hspace{1em} Female&0.939 {[}0.927, 0.945{]}&0.926 {[}0.918, 0.928{]}&0.852 {[}0.843, 0.862{]}&0.897 {[}0.894, 0.902{]}\\%
\hspace{1em} Male&0.932 {[}0.921, 0.938{]}&0.927 {[}0.924, 0.931{]}&0.844 {[}0.836, 0.851{]}&0.900 {[}0.897, 0.906{]}\\%
\midrule%
\textbf{Race}&&&&\\%
\hspace{1em} Asian&0.953 {[}0.947, 0.965{]}&0.939 {[}0.938, 0.955{]}&0.871 {[}0.854, 0.878{]}&0.913 {[}0.907, 0.926{]}\\%
\hspace{1em} Black&0.959 {[}0.954, 0.970{]}&0.927 {[}0.926, 0.939{]}&0.867 {[}0.860, 0.877{]}&0.899 {[}0.888, 0.910{]}\\%
\hspace{1em} Hispanic&0.942 {[}0.933, 0.981{]}&0.940 {[}0.934, 0.948{]}&0.872 {[}0.849, 0.883{]}&0.909 {[}0.893, 0.919{]}\\%
\hspace{1em} Other&0.977 {[}0.962, 0.990{]}&0.961 {[}0.957, 0.969{]}&0.863 {[}0.837, 0.883{]}&0.927 {[}0.918, 0.940{]}\\%
\hspace{1em} Unknown&0.890 {[}0.875, 0.910{]}&0.933 {[}0.921, 0.941{]}&0.796 {[}0.773, 0.799{]}&0.933 {[}0.927, 0.947{]}\\%
\hspace{1em} White&0.920 {[}0.909, 0.925{]}&0.918 {[}0.915, 0.920{]}&0.840 {[}0.838, 0.843{]}&0.890 {[}0.888, 0.893{]}\\%
\toprule%
\multicolumn{5}{c}{\textbf{\Large{MEDS-Tab}}}\\%
\toprule%
\textbf{Overall}&0.887 {[}0.882, 0.896{]}&0.918 {[}0.916, 0.919{]}&0.815 {[}0.810, 0.819{]}&0.879 {[}0.875, 0.879{]}\\%
\midrule%
\textbf{Gender}&&&&\\%
\hspace{1em} Female&0.898 {[}0.892, 0.902{]}&0.916 {[}0.912, 0.919{]}&0.822 {[}0.817, 0.825{]}&0.877 {[}0.873, 0.881{]}\\%
\hspace{1em} Male&0.876 {[}0.871, 0.884{]}&0.918 {[}0.913, 0.918{]}&0.807 {[}0.801, 0.808{]}&0.878 {[}0.872, 0.880{]}\\%
\midrule%
\textbf{Race}&&&&\\%
\hspace{1em} Asian&0.895 {[}0.825, 0.909{]}&0.916 {[}0.906, 0.920{]}&0.819 {[}0.795, 0.847{]}&0.877 {[}0.860, 0.885{]}\\%
\hspace{1em} Black&0.918 {[}0.892, 0.948{]}&0.920 {[}0.909, 0.928{]}&0.836 {[}0.824, 0.836{]}&0.874 {[}0.862, 0.883{]}\\%
\hspace{1em} Hispanic&0.890 {[}0.846, 0.954{]}&0.922 {[}0.901, 0.927{]}&0.851 {[}0.839, 0.853{]}&0.900 {[}0.880, 0.907{]}\\%
\hspace{1em} Other&0.933 {[}0.889, 0.945{]}&0.943 {[}0.938, 0.965{]}&0.830 {[}0.804, 0.854{]}&0.915 {[}0.902, 0.930{]}\\%
\hspace{1em} Unknown&0.789 {[}0.779, 0.824{]}&0.953 {[}0.950, 0.958{]}&0.750 {[}0.705, 0.783{]}&0.926 {[}0.922, 0.942{]}\\%
\hspace{1em} White&0.871 {[}0.868, 0.878{]}&0.907 {[}0.904, 0.909{]}&0.806 {[}0.798, 0.807{]}&0.867 {[}0.865, 0.867{]}\\%
\bottomrule%
\end{tabular}%
\end{table}



\begin{table}%
\centering%
\caption{\textbf{Demographic characteristics of the dataset analyzed in this study.} Performance Comparison of ETHOS and MEDS-Tab Across Clinical Outcomes and Demographic Subgroups. This table presents the predictive performance (AUROC with 95\% confidence intervals) of ETHOS (top) and MEDS-Tab (bottom) for four critical clinical outcomes used in ARES: Hospital Mortality, ICU Admission, Prolonged Hospital Stay (>10 days), and a Composite Risk Score (HM+IA+PS). The prevalence rates of each outcome are provided for reference. Performance metrics are further stratified by gender and race to assess potential disparities in model performance across demographic subgroups.}%
\label{tab:population-demographics}%
\begin{tabular}{lrrr}%
\toprule%
&\textbf{Train/Validation}&\textbf{Test}&\textbf{Total}\\%
\toprule%
\textbf{Patient Number}&269,741&29,971&299,712\\%
\textbf{Mean Age (Std.)}&48.5 (20.9)&48.6 (20.9)&48.5 (20.9)\\%
\midrule%
\textbf{Gender (\%)}&&&\\%
\hspace{1em} Female&142,696 (52.9)&15,857 (52.9)&158,553 (52.9)\\%
\hspace{1em} Male&127,045 (47.1)&14,114 (47.1)&141,159 (47.1)\\%
\midrule%
\textbf{Race (\%)}&&&\\%
\hspace{1em} Unknown&115,437 (42.8)&12,684 (42.3)&128,121 (42.7)\\%
\hspace{1em} White&110,408 (40.9)&12,369 (41.3)&122,777 (41.0)\\%
\hspace{1em} Black&21,410 (7.9)&2,321 (7.7)&23,731 (7.9)\\%
\hspace{1em} Hispanic&9,214 (3.4)&1,023 (3.4)&10,237 (3.4)\\%
\hspace{1em} Asian&6,802 (2.5)&787 (2.6)&7,589 (2.5)\\%
\hspace{1em} Other&6,470 (2.4)&787 (2.6)&7,257 (2.4)\\%
\midrule%
\textbf{Marital Status (\%)}&&&\\%
\hspace{1em} Unknown&114,234 (42.3)&12,603 (42.1)&126,837 (42.3)\\%
\hspace{1em} Married&70,269 (26.1)&7,811 (26.1)&78,080 (26.1)\\%
\hspace{1em} Single&60,915 (22.6)&6,793 (22.7)&67,708 (22.6)\\%
\hspace{1em} Widowed&14,243 (5.3)&1,670 (5.6)&15,913 (5.3)\\%
\hspace{1em} Divorced&10,080 (3.7)&1,094 (3.7)&11,174 (3.7)\\%
\bottomrule%
\end{tabular}%
\end{table}

\begin{table}
\caption{\textbf{Prediction of Hospitalization At Triage.} Performance comparison of various models for predicting hospitalization at triage, evaluated using AUROC, AUPRC, sensitivity, and specificity (95\% confidence intervals in brackets). The thresholds for sensitivity and specificity were determined by finding the operating point on the ROC curve closest to (0,1). ETHOS demonstrates superior performance across all metrics, achieving the highest AUROC (0.912), AUPRC (0.887), sensitivity (0.849), and specificity (0.820), outperforming all other methods, including traditional scoring systems and machine learning models.}
\label{tab:ed-hospitalization}
\begin{tabular}{lcccc}
\toprule
& AUROC & AUPRC & Sensitivity & Specificity \\
\midrule
LR & 0.809 [0.807, 0.813] & 0.773 [0.770, 0.780] & 0.751 [0.729, 0.757] & 0.723 [0.720, 0.747] \\
RF & 0.817 [0.814, 0.821] & 0.785 [0.778, 0.791] & 0.767 [0.735, 0.773] & 0.716 [0.714, 0.748] \\
GB & 0.819 [0.816, 0.822] & 0.792 [0.787, 0.797] & 0.753 [0.732, 0.771] & 0.728 [0.712, 0.750] \\
MLP & 0.823 [0.821, 0.827] & 0.797 [0.792, 0.803] & 0.748 [0.740, 0.778] & 0.740 [0.720, 0.752] \\
ESI & 0.712 [0.708, 0.716] & 0.632 [0.628, 0.639] & 0.584 [0.577, 0.591] & 0.784 [0.782, 0.790] \\
NEWS & 0.581 [0.577, 0.585] & 0.555 [0.548, 0.559] & 0.563 [0.554, 0.568] & 0.546 [0.542, 0.552] \\
NEWS2 & 0.565 [0.561, 0.569] & 0.538 [0.532, 0.543] & 0.519 [0.510, 0.526] & 0.570 [0.567, 0.576] \\
REMS & 0.666 [0.661, 0.669] & 0.605 [0.599, 0.610] & 0.605 [0.553, 0.717] & 0.641 [0.544, 0.712] \\
MEWS & 0.558 [0.555, 0.561] & 0.521 [0.516, 0.527] & 0.296 [0.292, 0.299] & 0.812 [0.807, 0.817] \\
CART & 0.673 [0.668, 0.676] & 0.617 [0.609, 0.622] & 0.703 [0.699, 0.707] & 0.578 [0.571, 0.583] \\
Med2Vec & 0.815 [0.812, 0.818] & 0.779 [0.774, 0.783] & 0.741 [0.732, 0.756] & 0.739 [0.726, 0.754] \\
AutoScore & 0.794 [0.791, 0.797] & 0.755 [0.750, 0.761] & 0.745 [0.714, 0.749] & 0.698 [0.692, 0.730] \\
MEDS-Tab & 0.846 [0.842, 0.850] & 0.862 [0.859, 0.866] & 0.723 [0.713, 0.729] & 0.807 [0.801, 0.819] \\
ETHOS (ours) & 0.912 [0.909, 0.915] & 0.887 [0.881, 0.890] & 0.849 [0.846, 0.852] & 0.820 [0.817, 0.823] \\
\bottomrule
\end{tabular}
\end{table}

\begin{table}
\caption{\textbf{Prediction of Critical Outcome Within 12h At Triage.} Performance comparison of various models for predicting critical outcomes within 12 hours of triage, evaluated using AUROC, AUPRC, sensitivity, and specificity (95\% confidence intervals in brackets). The thresholds for sensitivity and specificity were determined by finding the operating point on the ROC curve closest to (0,1). ETHOS achieves the highest performance across most of the metrics, with an AUROC of 0.937, AUPRC of 0.649, sensitivity of 0.858, and specificity of 0.863, substantially outperforming all other methods, including traditional scoring systems and machine learning models.}
\label{tab:ed-critical-outcome}
\begin{tabular}{lcccc}
\toprule
& AUROC & AUPRC & Sensitivity & Specificity \\
\midrule
LR & 0.869 [0.864, 0.872] & 0.327 [0.309, 0.338] & 0.804 [0.795, 0.822] & 0.777 [0.763, 0.784] \\
RF & 0.875 [0.868, 0.881] & 0.385 [0.368, 0.405] & 0.813 [0.789, 0.825] & 0.785 [0.783, 0.809] \\
GB & 0.884 [0.880, 0.887] & 0.404 [0.386, 0.415] & 0.804 [0.799, 0.829] & 0.799 [0.774, 0.800] \\
MLP & 0.888 [0.883, 0.891] & 0.397 [0.378, 0.407] & 0.817 [0.803, 0.838] & 0.795 [0.782, 0.812] \\
ESI & 0.807 [0.802, 0.812] & 0.198 [0.188, 0.207] & 0.877 [0.870, 0.888] & 0.640 [0.637, 0.644] \\
NEWS & 0.638 [0.625, 0.654] & 0.155 [0.144, 0.165] & 0.458 [0.439, 0.480] & 0.798 [0.796, 0.803] \\
NEWS2 & 0.621 [0.609, 0.635] & 0.141 [0.132, 0.149] & 0.596 [0.404, 0.612] & 0.536 [0.535, 0.826] \\
REMS & 0.679 [0.668, 0.689] & 0.109 [0.102, 0.117] & 0.674 [0.656, 0.693] & 0.605 [0.602, 0.608] \\
MEWS & 0.621 [0.610, 0.631] & 0.112 [0.106, 0.117] & 0.442 [0.421, 0.461] & 0.774 [0.770, 0.778] \\
CART & 0.706 [0.695, 0.713] & 0.154 [0.142, 0.161] & 0.591 [0.569, 0.603] & 0.723 [0.719, 0.727] \\
Med2Vec & 0.874 [0.870, 0.878] & 0.334 [0.313, 0.347] & 0.825 [0.794, 0.838] & 0.767 [0.761, 0.797] \\
AutoScore & 0.857 [0.853, 0.861] & 0.311 [0.294, 0.319] & 0.789 [0.753, 0.813] & 0.762 [0.745, 0.794] \\
MEDS-Tab & 0.814 [0.805, 0.822] & 0.420 [0.403, 0.440] & 0.776 [0.761, 0.790] & 0.663 [0.658, 0.667] \\
ETHOS (ours) & 0.937 [0.932, 0.943] & 0.649 [0.628, 0.666] & 0.858 [0.850, 0.867] & 0.863 [0.856, 0.869] \\
\bottomrule
\end{tabular}
\end{table}

\begin{table}
\caption{\textbf{Prediction of Emergency Department Re{-}presentation Within 72h.} Performance comparison of various models for predicting emergency department re-presentation within 72 hours, evaluated using AUROC, AUPRC, sensitivity, and specificity (95\% confidence intervals in brackets). The thresholds for sensitivity and specificity were determined by finding the operating point on the ROC curve closest to (0,1). ETHOS demonstrates superior performance, achieving the highest AUROC (0.740), AUPRC (0.199), sensitivity (0.659), and specificity (0.696), outperforming all other methods and showcasing its effectiveness for this challenging task.}
\label{tab:ed-representation}
\begin{tabular}{lcccc}
\toprule
& AUROC & AUPRC & Sensitivity & Specificity \\
\midrule
LR & 0.677 [0.661, 0.693] & 0.160 [0.138, 0.179] & 0.571 [0.535, 0.645] & 0.683 [0.626, 0.712] \\
RF & 0.672 [0.654, 0.680] & 0.147 [0.125, 0.158] & 0.570 [0.552, 0.652] & 0.688 [0.599, 0.690] \\
GB & 0.698 [0.681, 0.713] & 0.165 [0.147, 0.181] & 0.616 [0.576, 0.673] & 0.667 [0.609, 0.709] \\
MLP & 0.694 [0.681, 0.706] & 0.162 [0.144, 0.180] & 0.628 [0.591, 0.659] & 0.660 [0.615, 0.690] \\
Med2Vec & 0.656 [0.646, 0.670] & 0.133 [0.121, 0.150] & 0.564 [0.547, 0.600] & 0.663 [0.624, 0.679] \\
LSTM & 0.689 [0.676, 0.702] & 0.168 [0.147, 0.183] & 0.654 [0.574, 0.661] & 0.612 [0.609, 0.679] \\
AutoScore & 0.624 [0.612, 0.636] & 0.074 [0.068, 0.082] & 0.589 [0.548, 0.635] & 0.601 [0.563, 0.640] \\
MEDS-Tab & 0.696 [0.679, 0.713] & 0.163 [0.143, 0.185] & 0.635 [0.598, 0.680] & 0.661 [0.633, 0.704] \\
ETHOS (ours) & 0.740 [0.723, 0.760] & 0.199 [0.171, 0.230] & 0.659 [0.643, 0.678] & 0.696 [0.685, 0.708] \\
\bottomrule
\end{tabular}
\end{table}

\begin{table}%
\centering%
\caption{\textbf{Summary of Token and Timeline Statistics.} This table presents a comprehensive overview of the token and timeline data in the training, test, and combined datasets. Key metrics include the total number of tokens and timelines, along with statistics on timeline lengths such as the longest timeline, median, mean, and shortest timeline. The number of unique timeline tokens is also reported. The final section breaks down the encoding of timeline tokens into categories, such as time intervals, quantiles, medications, diagnoses, procedures, laboratory results, vitals, and other clinical features. This summary highlights the diversity and complexity of the tokenized data used in the study.}%
\label{tab:simple-pht-stats}%
\begin{tabular}{lrrr}%
\toprule%
&\textbf{Train/Validation}&\textbf{Test}&\textbf{Total}\\%
\toprule%
\textbf{Tokens}&321,238,835&35,942,101&357,180,936\\%
\midrule%
\textbf{Timelines}&257,082&28,540&285,622\\%
\midrule%
\textbf{Timeline Lengths}\\%
\hspace{1em} Longest&221,122&106,936&221,122\\%
\hspace{1em} Q3&1,041&1,052&1,042\\%
\hspace{1em} Median&322&331&323\\%
\hspace{1em} Mean&1,249&1,259&1,250\\%
\hspace{1em} Q1&114&115&114\\%
\hspace{1em} Shortest&2&2&2\\%
\hspace{1em} Unique&13,094&4,940&13,631\\%
\midrule%
\textbf{Unique Timeline Tokens}&4,495&3,947&4,495\\%
\midrule%
\textbf{Timeline Tokens Encoding}\\%
\hspace{1em} Time Intervals&19&19&19\\%
\hspace{1em} Quantiles&10&10&10\\%
\hspace{1em} Medications&312&275&312\\%
\hspace{1em} Diagnoses&2,989&2,542&2,989\\%
\hspace{1em} Procedures&34&34&34\\%
\hspace{1em} Labs&200&200&200\\%
\hspace{1em} Vitals&6&6&6\\%
\hspace{1em} HCPCS&66&37&66\\%
\hspace{1em} Inpatient Stays&29&29&29\\%
\hspace{1em} Emergency Department&7&7&7\\%
\hspace{1em} DRGs&772&737&772\\%
\hspace{1em} BMI&10&10&10\\%
\bottomrule%
\end{tabular}%
\end{table}

\begin{table}%
\centering%
\caption{\textbf{Overview of the data sources and their corresponding columns used in this work from the MIMIC-IV database and its extension MIMIC-IV-ED.} The table groups the data into three main categories: ED (Emergency Department), hosp (Hospital), and ICU (Intensive Care Unit). For each category, the associated tables and the specific columns extracted for the study are listed, highlighting key variables relevant to patient care and outcomes, such as identifiers (e.g., stay\_id, hadm\_id), timestamps (e.g., intime, charttime), and clinical observations (e.g., vitalsign, labresults). These selections were guided by the objectives of the study to comprehensively model patient trajectories and outcomes.}%
\label{tab:data-sources}%
\begin{tabular}{ll}%
\toprule%
Data Source&Used Columns\\%
\toprule%
\textbf{ed}&\\%
\vspace{0.2em}%
\hspace{1em} diagnosis&\makecell[l]{icd\_version, icd\_code, stay\_id}\\%
\vspace{0.2em}%
\hspace{1em} edstays&\makecell[l]{intime, arrival\_transport, hadm\_id\\stay\_id, outtime, disposition\\hadm\_id, stay\_id}\\%
\vspace{0.2em}%
\hspace{1em} pyxis&\makecell[l]{name, charttime, stay\_id}\\%
\vspace{0.2em}%
\hspace{1em} triage&\makecell[l]{acuity, stay\_id}\\%
\vspace{0.2em}%
\hspace{1em} vitalsign&\makecell[l]{temperature, charttime, stay\_id\\heartrate, charttime, stay\_id\\resprate, charttime, stay\_id\\o2sat, charttime, stay\_id\\sbp, dbp, charttime\\stay\_id, pain, charttime\\stay\_id}\\%
\midrule%
\textbf{hosp}&\\%
\vspace{0.2em}%
\hspace{1em} admissions&\makecell[l]{admission\_type, admission\_location, admittime\\insurance, marital\_status, race\\hadm\_id, discharge\_location, dischtime\\hadm\_id}\\%
\vspace{0.2em}%
\hspace{1em} diagnoses\_icd&\makecell[l]{icd\_version, icd\_code, hadm\_id}\\%
\vspace{0.2em}%
\hspace{1em} drgcodes&\makecell[l]{drg\_type, drg\_code, description\\hadm\_id}\\%
\vspace{0.2em}%
\hspace{1em} emar&\makecell[l]{medication, event\_txt, charttime\\hadm\_id, emar\_id, emar\_seq}\\%
\vspace{0.2em}%
\hspace{1em} hcpcsevents&\makecell[l]{short\_description, hadm\_id, chartdate}\\%
\vspace{0.2em}%
\hspace{1em} labevents&\makecell[l]{itemid, valueuom, hadm\_id\\charttime, valuenum, value}\\%
\vspace{0.2em}%
\hspace{1em} omr&\makecell[l]{result\_name, result\_value, chartdate}\\%
\vspace{0.2em}%
\hspace{1em} patients&\makecell[l]{gender, dod}\\%
\vspace{0.2em}%
\hspace{1em} procedures\_icd&\makecell[l]{icd\_version, icd\_code, hadm\_id\\chartdate}\\%
\vspace{0.2em}%
\hspace{1em} transfers&\makecell[l]{eventtype, careunit, intime\\hadm\_id}\\%
\midrule%
\textbf{icu}&\\%
\vspace{0.2em}%
\hspace{1em} icustays&\makecell[l]{first\_careunit, intime, hadm\_id\\stay\_id, last\_careunit, outtime\\hadm\_id, stay\_id}\\%
\bottomrule%
\end{tabular}%
\end{table}

\begin{figure}
    \centering
    \caption{\textbf{Model Architecture and Hyperparameter Overview.} (Left) The architecture of the transformer-based model, following the standard GPT design, includes multiple layers of masked multi-head attention and feed-forward modules, normalized at each step and combined with positional encodings. (Right) Summary of the hyperparameters used for model training and their explored ranges. The final model uses 6 layers, a context size of 2048, an embedding size of 768, 12 attention heads, a dropout rate of 0.3, and a batch size of 32. Additional information includes the percentage of discarded ambiguous inference repetitions (0.2–0.3\%) that appear when doing zero-shot inference.}
    \includegraphics[width=0.7\textwidth]{model_info.png}
    \label{fig:model-info}
\end{figure}

\begin{figure}
    \centering
    \caption{\textbf{ROC Curves for ETHOS Across All Prediction Tasks.} ROC curves and corresponding area under the curve (AUC) values with 95\% confidence intervals are shown for seven prediction tasks: Hospital Mortality, ICU Admission, Prolonged Stay (>10 days), Composite Outcome (Hospital Mortality + ICU Admission + Prolonged Stay), Hospitalization at Triage, Critical Outcome Within 12h at Triage, and Emergency Department (ED) Re-presentation Within 72h. Each plot includes the fitted ROC curve (orange), unique thresholds (crosses), and the 95\% confidence interval (gray shading). ETHOS demonstrates high predictive performance across all tasks, with AUC values ranging from 0.740 (ED Re-presentation) to 0.936 (Hospital Mortality).}
    \includegraphics[width=\textwidth]{ethos_auroc.pdf}
    \label{fig:ethos-auroc}
\end{figure}

\begin{figure}
    \centering
    \caption{\textbf{Calibration Curves for ETHOS Predictions Across Clinical Outcomes with 95\% Confidence Intervals Determined by Bootstrapping.} This figure presents calibration curves evaluating the reliability of ETHOS probability predictions across six key clinical outcomes: hospital mortality, ICU admission, prolonged hospital stay, composite risk score (HM+IA+PS), hospitalization at triage, critical outcome within 12 hours at triage, and ED re-presentation within 72 hours. The calibration curves compare predicted probabilities (x-axis) against observed event frequencies (y-axis), with perfect calibration represented by the dashed diagonal line, while the solid orange line shows ETHOS calibration performance, and the shaded gray region represents the 95\% confidence interval (CI) derived from bootstrapping. Each plot includes the Brier score, a metric assessing probabilistic prediction accuracy, where lower values indicate better calibration, with 0.00–0.05 classified as excellent, 0.05–0.10 as good, 0.10–0.20 as acceptable, and values above 0.20 as poor calibration. ETHOS demonstrates excellent calibration for hospital mortality (Brier score: 0.014), critical outcome within 12 hours (0.031), and ED re-presentation (0.041), while ICU admission (0.064), prolonged stay (0.067), and the composite risk score (0.090) exhibit good calibration, closely following the ideal calibration curve. Hospitalization at triage (0.143) is categorized as acceptable calibration, with some deviations at higher predicted probabilities, suggesting areas for potential improvement. Overall, ETHOS exhibits strong calibration across most clinical tasks, particularly in predicting mortality, early critical deterioration, and ED re-presentation, with acceptable performance for hospitalization risk at triage. These findings highlight ETHOS’s reliability in translating probability estimates into clinically meaningful risk stratifications, supporting its potential as a robust AI-driven decision support tool for real-time risk prediction and clinical decision-making.}
    \includegraphics[width=\textwidth]{ethos_calibration.pdf}
    \label{fig:ethos-calibration}
\end{figure}

\clearpage
\begin{longtable}{c}
\caption{\textbf{Risk Trajectories for Eight Patients from ED Presentation to Discharge, ICU Admission, or Death.}. This figure presents examples of risk trajectories for eight different patients, illustrating the dynamic evolution of risk predictions following presentation at the emergency department (ED). Each risk value is estimated from multiple (n=100) simulated fPHTs. The shaded area around each risk curve represents the 95\% confidence interval (CI) for the predicted risk. The primary graphs plot risk progression as a function of the number of tokens generated since ED presentation, effectively modeling the temporal evolution of patient risk. The visualisation of ARES score is schematically represented below using 10 color-coded symbols corresponding to key risk categories (see Figures 1 and 2 in main paper). In some graphs, symbols corresponding to ICU admission risk are absent (e.g., E, F, G, and H) because these patients were already admitted to the ICU earlier, leading ARES to automatically exclude this risk component from consideration. The time axis under ARES represents actual elapsed time (in hours and days) since ED presentation. However, time progression on these axes is not linear, as the number of generated tokens does not directly correspond to real-time intervals. Instead, token generation occurs in discrete units determined by patient events. Notably, in case H, a sudden drop in prolonged stay risk occurs because ARES automatically reclassifies a risk of prolonged stay >10 days into prolonged stay >15 days, leading to an observed risk reduction. This drop is an inherent property of ARES modeling rather than a true change in patient status. All trajectories ultimately conclude when the patient either dies or is discharged.} 
\label{tab:timeline-examples} \\

\includegraphics[width=1\textwidth]{add_timeline1.png} \\
\textbf{(A) Ends with death.} \\[10pt]

\includegraphics[width=1\textwidth]{add_timeline2.png} \\
\textbf{(B) Ends with ICU admission.} \\[10pt]

\includegraphics[width=1\textwidth]{add_timeline3.png} \\
\textbf{(C) Ends with ICU admission.} \\[10pt]

\includegraphics[width=1\textwidth]{add_timeline4.png} \\
\textbf{(D) Ends with hospital discharge.} \\[10pt]

\includegraphics[width=1\textwidth]{add_timeline5.png} \\
\textbf{(E) Ends with death.} \\[10pt]

\includegraphics[width=1\textwidth]{add_timeline6.png} \\
\textbf{(F) Ends with death.} \\[10pt]

\includegraphics[width=1\textwidth]{add_timeline7.png} \\
\textbf{(G) Ends with death.} \\[10pt]

\includegraphics[width=1\textwidth]{add_timeline8.png} \\
\textbf{(H) Ends with death.} \\[10pt]

\end{longtable}

\begin{longtable}{lcccccc}
\caption{\textbf{Detailed Token Statistics}. The table provides a detailed breakdown of the total number of tokens and unique tokens for each code group in the training, test, and combined datasets. Each code group represents a specific type of information, such as laboratory results (LAB), clinical classifications (e.g., ATC, ICD\_CM), time intervals (e.g., 15m-45m, 12h-18h), and other key features like BMI, vitals, or discharge locations. The statistics summarize the diversity (\#Unique) and frequency (Count) of tokens across datasets, offering insights into the distribution and variability of features used in the modeling process.} \label{tab:detailed-token-stats} \\
\toprule
 & \multicolumn{2}{c}{Train} & \multicolumn{2}{c}{Test} & \multicolumn{2}{c}{Total} \\
 & \#Unique & Count & \#Unique & Count & \#Unique & Count \\
Code Group &  &  &  &  &  &  \\
\midrule
\endfirsthead
\toprule
 & \multicolumn{2}{c}{Train} & \multicolumn{2}{c}{Test} & \multicolumn{2}{c}{Total} \\
 & \#Unique & Count & \#Unique & Count & \#Unique & Count \\
Code Group &  &  &  &  &  &  \\
\midrule
\endhead
\midrule
\multicolumn{7}{r}{Continued on next page} \\
\midrule
\endfoot
\bottomrule
\endlastfoot
LAB & 200 & 90,250,118 & 200 & 10,098,515 & 200 & 100,348,633 \\
ATC & 87 & 26,773,380 & 81 & 2,997,648 & 87 & 29,771,028 \\
ATC\_4 & 12 & 26,773,367 & 12 & 2,997,644 & 12 & 29,771,011 \\
ATC\_SFX & 213 & 26,658,727 & 182 & 2,984,558 & 213 & 29,643,285 \\
Q1 & 1 & 13,313,065 & 1 & 1,476,714 & 1 & 14,789,779 \\
Q2 & 1 & 12,153,214 & 1 & 1,353,936 & 1 & 13,507,150 \\
Q3 & 1 & 11,631,525 & 1 & 1,299,028 & 1 & 12,930,553 \\
Q4 & 1 & 10,483,733 & 1 & 1,172,049 & 1 & 11,655,782 \\
Q5 & 1 & 10,315,908 & 1 & 1,156,166 & 1 & 11,472,074 \\
Q6 & 1 & 10,154,034 & 1 & 1,141,348 & 1 & 11,295,382 \\
VITAL & 6 & 9,946,752 & 6 & 1,113,072 & 6 & 11,059,824 \\
Q7 & 1 & 9,574,210 & 1 & 1,076,334 & 1 & 10,650,544 \\
ICD\_CM & 2,989 & 9,330,094 & 2,542 & 1,036,475 & 2,989 & 10,366,569 \\
Q8 & 1 & 8,954,426 & 1 & 1,006,563 & 1 & 9,960,989 \\
Q9 & 1 & 8,593,863 & 1 & 966,320 & 1 & 9,560,183 \\
Q10 & 1 & 7,900,178 & 1 & 888,383 & 1 & 8,788,561 \\
ICD\_PCS & 34 & 3,998,316 & 34 & 442,617 & 34 & 4,440,933 \\
15m-45m & 1 & 2,234,231 & 1 & 251,165 & 1 & 2,485,396 \\
1h15m-2h & 1 & 2,082,216 & 1 & 232,659 & 1 & 2,314,875 \\
2h-3h & 1 & 1,925,854 & 1 & 214,816 & 1 & 2,140,670 \\
3h-5h & 1 & 1,877,497 & 1 & 209,154 & 1 & 2,086,651 \\
45m-1h15m & 1 & 1,678,348 & 1 & 188,677 & 1 & 1,867,025 \\
5m-15m & 1 & 1,549,919 & 1 & 173,374 & 1 & 1,723,293 \\
BMI & 10 & 1,485,790 & 10 & 169,939 & 10 & 1,655,729 \\
5h-8h & 1 & 1,122,479 & 1 & 124,573 & 1 & 1,247,052 \\
8h-12h & 1 & 980,545 & 1 & 109,975 & 1 & 1,090,520 \\
TRANSFER & 38 & 750,441 & 38 & 83,393 & 38 & 833,834 \\
12h-18h & 1 & 708,241 & 1 & 79,051 & 1 & 787,292 \\
2mt-6mt & 1 & 465,225 & 1 & 52,313 & 1 & 517,538 \\
=6mt & 1 & 456,699 & 1 & 50,085 & 1 & 506,784 \\
30d-2mt & 1 & 430,807 & 1 & 48,696 & 1 & 479,503 \\
12d-20d & 1 & 388,256 & 1 & 44,259 & 1 & 432,515 \\
DRG & 772 & 388,255 & 737 & 42,977 & 772 & 431,232 \\
HOSPITAL\_DISCHARGE & 1 & 388,254 & 1 & 42,977 & 1 & 431,231 \\
DISCHARGE\_LOCATION & 10 & 388,254 & 10 & 42,977 & 10 & 431,231 \\
INSURANCE & 3 & 388,254 & 3 & 42,977 & 3 & 431,231 \\
HOSPITAL\_ADMISSION & 1 & 388,254 & 1 & 42,977 & 1 & 431,231 \\
ADMISSION\_TYPE & 3 & 388,254 & 3 & 42,977 & 3 & 431,231 \\
ED\_REGISTRATION & 1 & 382,614 & 1 & 42,473 & 1 & 425,087 \\
ED\_OUT & 1 & 382,614 & 1 & 42,473 & 1 & 425,087 \\
ED\_ACUITY & 1 & 382,614 & 1 & 42,473 & 1 & 425,087 \\
ED\_TRANSPORT & 4 & 382,614 & 4 & 42,473 & 4 & 425,087 \\
20d-30d & 1 & 340,809 & 1 & 38,149 & 1 & 378,958 \\
4d-7d & 1 & 333,877 & 1 & 38,375 & 1 & 372,252 \\
7d-12d & 1 & 328,988 & 1 & 37,916 & 1 & 366,904 \\
1d-2d & 1 & 307,351 & 1 & 34,627 & 1 & 341,978 \\
TIMELINE\_END & 1 & 257,082 & 1 & 28,540 & 1 & 285,622 \\
2d-4d & 1 & 227,549 & 1 & 25,932 & 1 & 253,481 \\
18h-1d & 1 & 225,224 & 1 & 25,242 & 1 & 250,466 \\
HCPCS & 66 & 127,052 & 37 & 13,731 & 66 & 140,783 \\
ICU\_ADMISSION & 1 & 65,816 & 1 & 7,365 & 1 & 73,181 \\
ICU\_TYPE & 9 & 65,816 & 9 & 7,365 & 9 & 73,181 \\
ICU\_DISCHARGE & 1 & 65,816 & 1 & 7,365 & 1 & 73,181 \\
SOFA & 1 & 65,816 & 1 & 7,365 & 1 & 73,181 \\
MEDS\_DEATH & 1 & 26,200 & 1 & 2,876 & 1 & 29,076 \\
\end{longtable}


\section{Monte Carlo Justification for Probability Estimation}
\label{sec:monte-carlo-just}

Let $p(\mathbf{x})$ denote the probability distribution over fPHTs as modeled by ETHOS where by $\mathbf{x}$ we indicate an fPHT. Suppose we want to estimate the probability of some event $A$ regarding the future timeline. For instance, $A$ could be the event ``the patient death when admitted'' or ``the patient admitted to ICU.'' Formally,

\[
\Pr(A) \;=\; \sum_{\mathbf{x} \in A} p(\mathbf{x}),
\]

where the sum is over all sequences $\mathbf{x}$ for which the event $A$ holds (i.e., $\mathbf{x} \in A$).

{\bf A. Monte Carlo Estimator}\\

A straightforward Monte Carlo approach to approximate $\Pr(A)$ is as follows:

\begin{enumerate}
    \item \textbf{Draw} $N$ i.i.d.\ samples $\{\mathbf{x}^{(1)}, \mathbf{x}^{(2)}, \ldots, \mathbf{x}^{(N)}\}$ from the model $p(\mathbf{x})$.
    \item \textbf{Define} an indicator function $I(\mathbf{x}^{(i)} \in A)$, which is $1$ if the sample $\mathbf{x}^{(i)}$ lies in $A$, and $0$ otherwise.
    \item \textbf{Estimate} $\Pr(A)$ by the ratio
    \[
       \hat{\Pr}(A) 
       \;=\; \frac{1}{N} \sum_{i=1}^{N} I\bigl(\mathbf{x}^{(i)} \in A\bigr).
    \]
\end{enumerate}

In other words, $\hat{\Pr}(A)$ is simply the fraction of samples whose corresponding timelines satisfy event $A$ indicated as $M/N$ in the text.

{\bf B. Unbiasedness}\\

If the samples $\mathbf{x}^{(i)}$ are drawn exactly from $p(\mathbf{x})$, then for each sample,

\[
\mathbb{E}[I(\mathbf{x}^{(i)} \in A)]
\;=\; \Pr(\mathbf{x}^{(i)} \in A)
\;=\; \Pr(A).
\]

Hence,
\[
\mathbb{E}\bigl[\hat{\Pr}(A)\bigr]
\;=\; \mathbb{E}\Bigl[\frac{1}{N}\sum_{i=1}^N I(\mathbf{x}^{(i)} \in A)\Bigr]
\;=\; \Pr(A),
\]
showing that $\hat{\Pr}(A)$ is an \emph{unbiased} estimator of $\Pr(A)$.

{\bf{C. Convergence by the Law of Large Numbers}}\\

By the Law of Large Numbers (LLN), as $N \to \infty$,
\[
\hat{\Pr}(A) \;\xrightarrow{a.s.}\; \Pr(A),
\]
meaning the simple ratio of ``successes'' (i.e., samples satisfying $A$) to total draws converges almost surely to the true probability.
