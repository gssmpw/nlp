\section{Related Work}
A ``mocobot'' is a mobile variant of the ``cobot,'' robots designed for physical collaboration with humans, originally introduced in____. Mocobots enhance cobots with mobility for diverse tasks. The Omnid mocobot, designed for human-multirobot collaborative manipulation, is described in____. 

The work in this paper extends previous work by providing a method for discovering payload kinematic and inertial properties, building on previous work in cooperative robot manipulation and payload estimation.
\subsubsection{Cooperative Robot Manipulation}
Cooperative robot manipulation involves multiple robots working together to manipulate objects, with applications in manufacturing, construction, and hazardous environments____. Various control architectures ranging from centralized to distributed systems, including those with no explicit communication, have been utilized for manipulation tasks____.

\begin{figure} 
\centering
$\vcenter{\hbox{\subfloat{\includegraphics[width=0.44\textwidth]{lin_omnids.PNG}}}}$
\caption{Three Omnid mocobots collaborate safely and intuitively with a human operator on a simulated plane wing assembly task.}
\label{fig:demo}
\end{figure}

%Research can be categorized into autonomous manipulation, human-single-robot collaboration, and human-multi-robot collaboration. 
Approaches like haptic feedback for leader-follower formations____ and force amplification strategies____ have been explored. Recent advancements include machine learning techniques to enhance human-robot interaction and adaptability____. Some systems use distributed adaptive control of omnidirectional mobile bases equipped with robot arms to autonomously transport payloads____. Other configurations include customized robots for collaborative rigid-body manipulation____ and strategies for manipulating deformable payloads____. 
%However, these systems often require full knowledge of payload dynamics or specific configurations for attachment, limiting flexibility in unknown environments.

%%%%%%%%%%%%%%%%%%%%%%%%%%%%%%%%%%%%%%%%%%%%%%%%
\subsubsection{Payload Estimation}
Knowledge of payload properties, particularly the center of mass (CoM) and inertia matrix, is crucial for the dynamic performance and stability of robotic systems. Early research on payload inertia matrix estimation involved suspending payloads and measuring oscillation periods, which provided reliable results for vehicles and aircraft____. However, these techniques are impractical for robotic applications, especially in dynamic and unstructured environments.

In the field of robotics, force-torque sensors, combined with Newton-Euler equations, have become a standard tool for the estimation of inertial properties by a single robot____. For the case of manipulation by multiple mobile manipulators, decentralized approaches have been developed for estimating properties of the common payload____. %gao2022estimating}. 
%However, such methods often struggle with payloads that have complex shapes as they typically rely on simplified models and assumptions that do not account for the full range of potential payload dynamics.
In____, the robots cooperatively estimate the centroid of the payload, but the motion is limited to the plane and mass and inertial parameters are not considered. The estimation goals of____ are closely related to those of this paper, except the algorithms described in that work do not consider the orientations of the robots' grasp frames and have only been tested in simulation.
%However, these methods model the payload dynamics through simplified assumptions that may not fully capture the characteristics of payloads with irregular geometries or complex mass distributions. This limitation can restrict their direct applicability to our research, which involves payloads that exhibit these complex features under an unstructured environment.

%Decentralized approaches have been developed to handle cooperative manipulation tasks for estimation using multi-robot systems. These methods focus on estimating the CoM and other properties by leveraging the distributed sensing capabilities of multiple robots


%%%%%%%%%%%%%%%%%%%%%%%%%%%%%%%%%%%%%%%%%%%%%%%%