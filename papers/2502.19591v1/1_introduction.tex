In applications such as sanding, polishing, wiping, or sensor scanning, a robot manipulator needs to move its end-effector to cover an entire surface. In cases where the robot has redundant degrees of freedom or the application allows certain tolerances, the system obtains some \textit{flexibility}, \textit{i.e.}, there are infinitely many inverse kinematics (IK) solutions for positioning the end-effector at each surface point. A coverage path planner should exploit the flexibility to find a joint motion that fully covers the surface, while minimizing objectives like joint movements and maintaining short computation times.

%Consequently, a motion planner needs to determine 1) a Cartesian path for the end-effector to move on the target surface, and 2) a joint-space trajectory for the robot to follow the Cartesian path. 
% In order to minimize specific costs in joint space, \textit{e.g.}, joint movements, the motion planner must leverage the flexibility brought by the redundancy to find optimal joint motions.

% The coverage path planning problem for a robot manipulator with some redundancy has been formulated as a generalized traveling salesman problem (GTSP) whose objective is to find the shortest possible route that visits exactly one node from each of several predefined sets. Here, each IK solution is a node and all nodes that move the end-effector to the same pose are grouped in the same sets. A GTSP solver finds a series IK solutions that move the end-effector to each end-effector exactly once. However, this approach can produce a large graph and the GTSP is NP-hard. Therefore, this approach doesn't scale \cite{hess2012null, suarez2018robotsp}. 


This paper provides a solution to the problem of coverage path planning for redundant robot manipulator systems, aiming to minimize specific costs in joint space.
%, such as joint movements.
Following prior work \cite{hess2012null, kaljaca2020coverage, zhang2024jpmdp}, we formulate this problem as a Generalized Traveling Salesman Problem (GTSP), where the objective is to find the shortest route in a graph that visits exactly one node from each predefined set. Each node represents an IK solution, with sets grouping IK solutions that cover the same surface point, and edge weights representing the movement costs between two IK configurations. A GTSP route corresponds to a sequence of IK solutions that move the end-effector to cover all required surface points. Using this formulation, prior work \cite{hess2012null, kaljaca2020coverage, zhang2024jpmdp} randomly samples multiple IK solutions for each surface point, constructs a graph, and solves the GTSP. 
However, this approach often leads to a large graph, and due to the NP-hard nature of GTSP, it requires significant computation time \cite{hess2012null, suarez2018robotsp}. 

In this paper, we present an efficient and effective coverage path planning approach for redundant manipulator systems. 
In contrast to prior work which randomly samples IK solutions, we incorporate a strategy to identify IK samples that are likely to lead to good solutions. By identifying guide paths that roughly cover the surface and sample IK solutions near them, our strategy accelerates the computation by solving a sequence of smaller GTSP. Our approach enables frequent identification of effective solutions within a significantly reduced timeframe. We provide an open-source implementation of our approach\footnote{Open-sourced code \url{https://github.com/uwgraphics/arm_coverage}}.

The central contribution of this paper is an effective and efficient coverage path planning approach for robot manipulators (\cref{sec:technical_details}). To facilitate understanding of the problem, we describe and analyze two baseline approaches based on prior work (\ref{sec:cart-tsp} and \cref{sec:joint_gtsp}). All three approaches were evaluated using four simulated benchmarks (\cref{sec:evaluation}). Our results show that the proposed approach generates higher-quality motions with shorter computation time compared to the baselines. 
