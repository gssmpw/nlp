
% In this section, we describe related works in the areas of coverage path planning for robot manipulators, generalized traveling salesman problems, and arm reconfigurations. 

\vspace{-1mm}
\subsection{Coverage Path Planning for Robot Manipulators}
\vspace{-1mm}
Coverage path planning is required for a robot manipulator to perform tasks such as polishing \cite{schneyer2023segmentation}, bush trimming \cite{kaljaca2020coverage}, surface cleaning \cite{hess2012null, sakata2023coverage, moura2018automation}, surface inspection \cite{jing2017sampling, jing2018computational, leidner2016robotic}, surface disinfection \cite{zhang2024jpmdp, thakar2022area}, and post-processing after 3D printing \cite{do2023geometry}. Coverage path planning approaches for robotic arms generally fall into two categories. The first group focuses on planning Cartesian paths for specific tasks, which the robotic arm then follows. These approaches often involve specialized end-effectors tailored for specific tasks, such as sanding using different parts of the sanding disk \cite{schneyer2023segmentation}, post-processing after 3D printing using a blast nozzle  \cite{do2023geometry},  spraying with aerosol cans \cite{thakar2022area}, wiping with a combination of three actions \cite{leidner2016robotic}, and cleaning surfaces of unknown geometry \cite{moura2018automation}. 
These works assume that the robot can accurately move the end-effector along the planned path. Planning in the Cartesian space, they can not achieve objectives in joint space such as minimizing joint movements \cite{zhang2024jpmdp}. 

The second group of methods plans directly in joint space for both non-redundant \cite{yang2024improved, yang2020cellular, yang2020non} and redundant robot manipulators \cite{hess2012null, kaljaca2020coverage, zhang2024jpmdp}. For redundant systems, as outlined in the Introduction, one approach samples multiple IK solutions for each surface point, constructs a joint space graph, and solves a GTSP. However, due to GTSP's NP-hard nature, this method, which we call Joint-GTSP, is sensitive to graph size. To reduce computation time, prior works often restrict the number of IK samplings $m$ for each surface point, \textit{e.g.}, $m{=}16$ \cite{kaljaca2020coverage} and $m{=}9$ \cite{hess2012null}. However, this constraint restricts the exploration of the solution space. 
Due to Joint-GTSP's limitations, prior works \cite{ hess2012null, suarez2018robotsp} use an alternative method that builds a graph in Cartesian space, solves a Traveling Salesman Problem (TSP), and uses a path tracking method to follow the TSP route.  In the Cartesian graph, nodes represent surface points, and edge weights depend on distance and curvature between points. We call this approach Cart-TSP-IKLink, where IKLink is a path tracking method. Although more efficient due to a smaller graph size compared to Joint-GTSP, Cart-TSP-IKLink does not plan in joint space and can not optimize joint-space objectives such as minimizing joint movements. 
We will further describe Joint-GTSP and Cart-TSP-IKLink in \cref{sec:technical_overview} and use them as baselines in our experiments. In this work, we propose a coverage path planning approach that is more efficient and effective than both Joint-GTSP and Cart-TSP-IKLink. 


\vspace{-3mm}
\subsection{Generalized Traveling Salesman Problem} \label{sec:gtsp}
\vspace{-1mm}
In addition to coverage path planning for robot manipulators, other robotics problems can be formulated as Generalized Traveling Salesman Problems (GTSP), such as task sequencing for robot manipulators \cite{suarez2018robotsp, saha2006planning, alatartsev2015robotic} and viewpoint coverage path planning for unmanned aerial vehicles \cite{song2018surface, obermeyer2012sampling, dhami2024gatsbi, bahnemann2021revisiting, cao2020hierarchical}.
Both exact and approximate methods are used to solve GTSP \cite{pop2023comprehensive}. While exact methods \cite{pop2007new, noon1991lagrangian, kara2012new} guarantee optimal solutions, they require significant computation time. 
Approximation methods are polynomial time algorithms that
produce approximate solutions. 
Among many approximate GTSP solvers \cite{hu2008effective, helsgaun2015solving, snyder2006random}, 
% Example include 2-opt neighborhood method \cite{hu2008effective} and Lin-Kernighan heuristic \cite{helsgaun2015solving}, and generic algorithm \cite{snyder2006random}. 
% Another approach is to convert GTSP into TSP \cite{dimitrijevic1997efficient, behzad2002new}, allowing the use of many widely available approximate TSP solvers. % These approximate algorithms are often iterative, meaning that they can generate multiple solutions over time. 
we choose GLKH \cite{helsgaun2015solving} for its open-source availability \cite{GLKH} and state-of-the-art performance on large scale datasets \cite{suarez2018robotsp, smith2017glns}. 

The solution of GTSP is a \textit{cycle}, \textit{i.e.}, a closed loop with no defined start or end. However, coverage path planning requires a \textit{path}, where the manipulator does not return to the starting point. In order to extract a path from a cycle, Hess \textit{et al.} \cite{hess2012null} select the starting point as the node closest to the robot's current configuration. However, this method does not extract the shortest path even if the cycle is the shortest. In this work, we construct the graph in a way so that a GTSP solution directly yields the shortest path 
 (\cref{sec:joint_gtsp}-Step C).


% RobotTSP \cite{suarez2018robotsp}: arm; first do TSP in task space, then IKLink-like in joint space; compare 2-opt, nearest-neighbor, and integer programming. Found 2-opt to be the best. 

% Hierarchical coverage path planning \cite{cao2020hierarchical}, non-arm.

\vspace{-3mm}
\subsection{Arm Reconfigurations}
\vspace{-1mm}

Most coverage path planners assume that the robot manipulator can cover the given surface as a whole without interruptions. The uninterrupted coverage of some surfaces is infeasible due to the robot’s joint limits, self-collision, or obstacles elsewhere in the environment.
In such cases, it becomes necessary to divide the surface into smaller segments, requiring the robot to perform \textit{arm reconfigurations} -- a process where the robot deviates from the reference surface, repositions itself in configuration space, and resumes task execution. 
Arm reconfigurations, also known as retractions \cite{kaljaca2020coverage} or discontinuities \cite{yang2020cellular}, should be minimized as they increase time and energy consumption.

Previous work has proposed coverage path planners to minimize reconfigurations \cite{kaljaca2020coverage, yang2024improved, yang2020cellular, yang2020non}. 
However, these methods are designed for non-redundant robots and their extension to redundant robots remains unclear. This paper presents a coverage path planner for redundant manipulator systems, with the goal of minimizing reconfigurations.

\vspace{-3mm}
\subsection{Flexibility in Task Execution}
\vspace{-1mm}

In cases where the manipulator is redundant, \textit{i.e.}, the robot has more degrees of freedom than the task requires, or the task has certain tolerances, \textit{i.e.}, allowable position or rotation inaccuracy, there are infinitely many IK solutions that satisfy the task requirements. Such flexibility has been leveraged to achieve active compliance \cite{sadeghian2013task}, emotional expression \cite{claret2017exploiting}, efficient trajectory tracking \cite{wang2025anytime}, and smooth telemanipulation \cite{wang2023exploiting}. In this work, we exploit the flexibility to optimize arm reconfigurations and joint movements in coverage path planning. 