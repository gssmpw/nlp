

\begin{figure*} [tb]
  \centering
  \includegraphics[width=7in]{figures/tasks.pdf}
  \vspace{-5mm}
  \caption{Our experiment involves four benchmark applications. For each, we provide visualizations of \protect\circled{1} the robot performing the task, \protect\circled{2} task tolerances, and \protect\circled{3} \protect\circled{4} \protect\circled{5} how the surface is covered using different approaches. We optimize for the number of reconfigurations (breakpoints on the paths) and joint movements; motions with shorter joint movements tend to result in complex Cartesian space paths.
  The robot visualizations and green traces were generated using \textit{Motion Comparator}\protect\footnotemark \cite{wang2024motion}.
  }
  
  \label{fig: tasks}
  \vspace{-3mm}
\end{figure*}


\begin{table*}[tb]
\caption{Experiment Results and Metrics of the Target Surface} 
\vspace{-3mm}
\label{tab:results}
\begin{center}
\begin{tabular}{c|l|rrr|cc|c}
\hline
Benchmark & \makecell[c]{Method} & \makecell[c]{Mean Num\\ of Reconfig} & \makecell[c]{Mean Joint \\ Movements (rad)$^\dagger$}& \makecell[c]{Mean Comput-\\ation Time (s)} & \makecell{Max Position \\ Error (m)$^\ddagger$} & \makecell{Max Rotation \\ Error (rad)$^\ddagger$} & \makecell{Number of End-\\Effector Targets $n$} \\  
\hline

%%%%%%%%%%%%%%%%%%%% paste data from python %%%%%%%%%%%%%%%%%%%%  
\rule{0pt}{1.05\normalbaselineskip}%
\multirow{3}{*}{\makecell[c]{Vacuum\\ Stairs}} & Cart-TSP-IKLink & 
    5.00$\pm$0.00 & 32.20$\pm$0.41 \hspace{2mm} & 47.40$\pm$\hspace{1ex}0.35   &  
7.7e-4 & \hspace{1ex}9.5e-3 & 
\multirow{3}{*}{208} \\

& Joint-GTSP & 
5.10$\pm$0.94 & 33.00$\pm$1.11 \hspace{2mm} & 146.22$\pm$35.27   &
 7.9e-4 & 10.0e-3 & \\

& H-Joint-GTSP & 
\textbf{4.30}$\pm$0.46 &  
\textbf{26.66}$\pm$0.51 \hspace{2mm} & 
\textbf{43.59}$\pm$14.07 &
9.8e-4 & \hspace{1ex}9.9e-3 & \\


\hline
\rule{0pt}{1.05\normalbaselineskip}%
\multirow{3}{*}{\makecell[c]{Polish\\Wok}} & Cart-TSP-IKLink &  \textbf{0.00}$\pm$0.00 & 57.78$\pm$0.33 \hspace{2mm} & 99.82$\pm$\hspace{1ex}1.00  &
8.3e-4 & \hspace{1ex}9.2e-3 & 
\multirow{3}{*}{209} \\

& Joint-GTSP & 
14.70$\pm$3.41 & 58.27$\pm$4.93 \hspace{2mm} & 566.26$\pm$95.83  &
7.6e-4 & \hspace{1ex}9.7e-3 & \\

& H-Joint-GTSP & 
\textbf{0.00}$\pm$0.00 & \textbf{41.71}$\pm$0.83 \hspace{2mm} & \textbf{74.56}$\pm$19.71  &
9.6e-4 & 10.0e-3 & \\

\hline
\rule{0pt}{1.05\normalbaselineskip}%
\multirow{3}{*}{\makecell[c]{Clean\\Toilet\\Bowl}} & Cart-TSP-IKLink & 
\textbf{0.00}$\pm$0.00 & 80.78$\pm$1.63 \hspace{2mm} & 89.62$\pm$\hspace{2.2ex}0.37  &
9.0e-4 & n/a & 
\multirow{3}{*}{157} \\

& Joint-GTSP & 
17.29$\pm$6.09 & 61.58$\pm$3.93 \hspace{2mm} & 624.35$\pm$211.82 &
9.5e-4 & n/a & \\

& H-Joint-GTSP & 
\textbf{0.00}$\pm$0.00 & \textbf{17.91}$\pm$1.66 \hspace{2mm} & \textbf{61.35}$\pm$\hspace{2.2ex}9.13  &
9.8e-4 & n/a & \\

\hline
\rule{0pt}{1.05\normalbaselineskip}%
\multirow{3}{*}{\makecell[c]{Scan\\ Floor}} & Cart-TSP-IKLink & 
2.00$\pm$\hspace{1ex}0.00 & 166.23$\pm$0.52 \hspace{2mm} & 262.18$\pm$\hspace{2.2ex}5.12   &
\hspace{1ex}9.9e-4 & \hspace{1ex}9.7e-3 & 
\multirow{3}{*}{674} \\

& Joint-GTSP & 
21.60$\pm$13.71 & 161.29$\pm$9.88 \hspace{2mm} & 1106.65$\pm$198.65   &
\hspace{1ex}9.9e-4 & \hspace{1ex}7.9e-3 & \\

& H-Joint-GTSP & 
\textbf{1.30}$\pm$\hspace{1ex}0.46 & \textbf{123.37}$\pm$1.29 \hspace{2mm} & \textbf{207.84}$\pm$\hspace{1.1ex}69.13  &
10.0e-4 & 10.0e-3 & \\

%%%%%%%%%%%%%%%%%%%%%%%%%%%%%%%%%%%%%%%%%%%%%%%%%%%%%%%%%%%% 

\hline
\multicolumn{8}{l}{\rule{0pt}{1\normalbaselineskip}%
The range values are standard deviations.
$\dagger$: Joint movements do not involve movements to perform arm reconfigurations.} \\
\multicolumn{8}{l}{
\makecell[l]{$\ddagger$: In our prototype, we set the positional and rotational accuracy of the IK solver as 1e-3 m and 1e-2 rad. As noted in \cref{sec:implementation}, the accuracy can be\\ improved with more computation time. Position and rotation errors are measured only in degrees of freedom without tolerance. }}
\vspace{-3mm}
\end{tabular}
\end{center}
\end{table*}

\begin{figure*} [tb]
  \centering
  \includegraphics[width=6.5in]{figures/plot.pdf}
  \vspace{-1mm}
  \caption{
    To demonstrate the scalability of the approaches, we increased the number of end-effector targets sampled on the surface in the wok polishing task. The results show that the proposed method, H-Joint-GTSP, has superior scalability compared to both Cart-TSP-IKLink and H-Joint-GTSP because it consistently has faster performance and produces higher-quality motion, even when applied to increased sampling density of the input.}
  \label{fig: plot}
  \vspace{-6mm}
\end{figure*}

In this section, we compare our approach, H-Joint-GTSP, with two alternative approaches described in \cref{sec:technical_overview}, Cart-TSP-IKLink and Joint-GTSP, on four simulated benchmark tasks.  

\vspace{-1mm}
\subsection{Implementation Details} \label{sec:implementation}
\vspace{-1mm}

Our prototype system uses the open-source RangedIK\footnote{IK solver: \url{https://github.com/uwgraphics/relaxed\_ik\_core/tree/ranged-ik}} library. RangedIK \cite{wang2023rangedik} is an optimization-based inverse kinematics (IK) solver that can incorporate various requirements such as placing the end-effector within specified task tolerances. RangedIK generates more precise IK solutions with increased iterations. 
%We use GLKH \cite{helsgaun2015solving} and LKH \cite{helsgaun2000effective} as GTSP and TSP solvers, respectively. 
In addition, we modified the original GLKH solver \cite{GLKH} to accept a sparse graph as input.
All evaluations were performed on a laptop with Intel Core i9-13950HX 5.50 GHz CPU and 64 GB of RAM.



\vspace{-1mm}
\subsection{Benchmarks}
\vspace{-1mm}
We developed four benchmark applications involving both concave and convex surfaces, as well as 6- and 7-degree-of-freedom (DoF) manipulators, each with varying task tolerances. Fig. \ref{fig: tasks} shows visualizations of these applications. 

\subsubsection{Vacuum Stairs}

This task involves a Boston Dynamics Spot robot vacuuming stairs. The quadruped robot positions its body statically before using its 6 DoF arm to maneuver the vacuum hose. At each stop, the robot cleans both the vertical and horizontal surfaces of two steps. The robot can freely rotate the vacuum hose about its principal axis. 

\subsubsection{Polish Wok} 

This task replicates the wok polishing task in prior work \cite{yang2020cellular, yang2023template}, involving a Universal Robotics UR5 manipulator polishing the outer surface of a wok. In contrast to prior work which locks the robot's last joint to obtain a non-redundant setting, our benchmark has redundancy for coverage path planners to exploit. Specifically, the 6 DoF robot can freely rotate the tool along its principal axis.
In addition, following prior work \cite{schneyer2023segmentation}, we assume that the entire finishing disk, not just the center, can be used for polishing. This enables the robot to move the disk along the tangent plane at the end-effector target (the $xy$ plane) within a specified distance threshold. 

\subsubsection{Clean Toilet Bowl}

This task replicates the toilet bowl cleaning task in prior work \cite{sakata2023coverage} using a 6 DoF Universal Robotics UR5 manipulator.
Due to the semi-spherical shape of the toilet brush head, the robot can freely rotate the brush along its principal axis and adjust its tilt.

\subsubsection{Scan Floor}

This task replicates the floor scanning task in prior work \cite{hyde2023spot} where a robot uses a handheld detector to scan the floor to detect potential chemical or radioactive leaks. 
We use a 7 DoF Franka Emika Panda robot with a rectangular sensor. We assume that the robot needs to precisely control all 6 DoF of the rectangular sensor.

\vspace{-1mm}
\subsection{Experimental Procedure and Results}
\vspace{-1mm}

To account for the randomness in IK sampling and TSP/GTSP solvers, we repeat each benchmark 10 times and report average performance. 
All three approaches use an iteration-based TSP/GTSP solver, \textit{i.e.}, they tend to find improved solutions over time. We consider them converged if motion quality remains unchanged over 30 seconds, at which point we report the motion metrics and computation time.

As shown in Tab. \ref{tab:results}, H-Joint-GTSP consistently outperformed Cart-TSP-IKLink and Joint-GTSP across all four benchmark applications, generating higher-quality motions with shorter computation time.
H-Joint-GTSP produced motions with fewer or equal arm reconfigurations, and when reconfigurations were the same, its motions had shorter joint movements compared to the baseline approaches.
All approaches generated accurate motions.
To further demonstrate the scalability of the approaches, we increased the number of end-effector targets sampled on the surface in the wok polishing task. 
This denser sampling is required when using smaller polishing tools.
As shown in Fig. \ref{fig: plot}, H-Joint-GTSP has better scalability than the baseline approaches.

\vspace{-1mm}
\subsection{Real-Robot Demonstration}
\vspace{-1mm}
To further demonstrate the effectiveness of our proposed approach, we implemented a mock wok polishing task on a physical Universal Robotics UR5 robot. A joint motion $\boldsymbol{\xi}$ found by H-Joint-GTSP was sent to a PID controller to control the robot. As shown in Fig. \ref{fig: teaser} and the supplementary video, the robot successfully covered the bottom of the wok with a smooth and accurate motion.
