
Our proposed method, H-Joint-GTSP, reduces the graph size compared to Joint-GTSP by first computing a \textit{guide path} and then sampling inverse kinematics (IK) solutions near it.
As shown in Fig. \ref{fig: approaches}, this approach involves extracting end-effector target exemplars, sampling IK solutions for the exemplars, solving an upper-level GTSP to find guide paths, sampling IK solutions near the guide paths, and solving a lower-level GTSP to find joint motions. 

\textit{Step 1: Extract End-Effector Target Exemplars} ---
We extract key end-effector poses using the affinity propagation clustering algorithm \cite{frey2007clustering} with the distance defined in Equation \ref{eq:cartesian_dist}. The affinity propagation algorithm can automatically determine the number of clusters and return exemplars. The output of the clustering algorithm is $n'$ clusters $\{\mathcal{C}_1, \mathcal{C}_2, ..., \mathcal{C}_{n'} \}$, where the $i$-th cluster involves the indices of $m_i {\geq} 1$ end-effector targets $\mathcal{C}_i {=} \{c_i^1, c_i^2, ..., c_i^{m_i}\}$ and the first item $c_i^1$ corresponds to the center of the cluster.

\textit{Step 2: Sample IK Solutions for Exemplars} ---
Following Step A of Joint-GTSP (\cref{sec:joint_gtsp}), we sample $m$ IK solutions and cluster them for each exemplar (we set $m{=}100$ in our implementation). Additionally, for each IK solution, we verify that the robot can move from the configuration to other end-effector targets in the cluster within certain velocity constraints. Specifically, we use an optimization-based IK solver, RangedIK \cite{wang2023rangedik}, which incorporates velocity constraints and can be expressed as a function $\text{IK}(\mathbf{x}, \theta_{init}, \tau)$. The function returns a valid IK solution that moves the end-effector to the target $\mathbf{x}$ within tolerance $\tau$ and meets the velocity constraint from the initial configuration $\theta_{init}$. If no valid IK solution is found, it returns $\emptyset$. 
We add an IK solution $\theta$ to the set of IK solutions for the $i$-th exemplar $\Theta_i$ only if $\theta$ can reach all targets in the cluster, $\Theta_i {=} \Theta_i \cup \theta$ if $\text{IK}(\mathbf{x}_c, \theta, \tau) {\neq} \emptyset$, $\forall c \in \mathcal{C}_i / c_i^1$.
If $|\Theta_i|{=}0$, we subdivide the cluster $\mathcal{C}_i$ and repeat the sampling process. 

\textit{Step 3: Solve Upper-Level GTSP} ---  
The IK solutions sampled in Step 2 are used as nodes to construct a graph and guide paths are found by solving GTSP, following a process similar to Steps B \& C of Joint-GTSP (\cref{sec:joint_gtsp}). In contrast to Joint-GTSP, which uses IK solutions for all $n$ end-effector targets, here we only use IK samplings for the $n'$ exemplars, resulting in a smaller graph. A guide path $\boldsymbol{\xi}'{=}(\theta'_1, \theta'_2,..., \theta'_{n'}) $ consists of joint configurations that travel through all the exemplars. Given a $\boldsymbol{\xi}'$, we also obtain the order in which the exemplars are visited, denoted by $\boldsymbol{\pi}{=}(\pi_1, \pi_2,..., \pi_{n'})$, where $\pi_i$ is the index of the $i$-th visited exemplar.  Because we use an iteration-based GTSP solver, multiple guide paths are found over time. 

\textit{Step 4: Sample IK Solution near Guide Path} ---
This step generates IK solutions for non-exemplar end-effector targets. Using RangedIK, we greedily propagate to nearby non-exemplars from an IK solution $\theta'_i$  on a guide path. Specifically, we compute $\text{IK}(\mathbf{x}, \theta'_i, \tau)$, where $\mathbf{x}$ represents a non-exemplar end-effector target near $\mathbf{x}_{\pi_i}$. The reachability check in Step 2 ensures that all end-effector targets have at least one IK solution. 

\textit{Step 5: Solve Lower-Level GTSP} ---
In the final setup, we construct a graph and find joint motions by solving the lower-level GTSP. The nodes of this graph are IK solutions on guide paths as well as IK solutions propagated in Step 4. The process follows the same approach as described in Steps B \& C of Joint-GTSP in \cref{sec:joint_gtsp}.

\textit{Analysis} --- The upper-level GTSP graph has $n'm$ nodes, where $n'$ is the number of end-effector target exemplars and $m$ is the average number of sampled IK solutions. The lower-level GTSP graph has $nm'$ nodes, where $n$ is the number of end-effector targets and $m'$ is the number of IK solutions sampled near the guide path ($m'{\geq}1$ because multiple guide paths may be found over time in Step 3, and a guide path may produce multiple IK solutions for an end-effector target in Step 4). Both graphs are smaller than the graph in Joint-GTSP, leading to improved efficiency.



