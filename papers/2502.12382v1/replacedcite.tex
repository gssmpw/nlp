\section{Related Work}
\label{sec:Section2}
Given their growing and linked character, IoT networks—which are increasingly vulnerable to cyber-attacks—rely on IDS to maintain their security. Using both binary and multi-class intrusion detection models, several machine ML and DL methods have been employed to improve IDS. This review of the literature investigates several current works addressing intrusion detection problems using various techniques, both as stand-alone models and in hybrid setups.

The authors of ____ presented a new AI framework that detects malware in IoT devices to mitigate cyber-attacks. The authors focus on enhancing security in various use cases for smart environments through an all-inclusive AI-enabled approach in this paper. Emulation of a smart environment employing the Raspberry Pi and NVIDIA Jetson as gateways in capturing data from IoT devices connected via the MQTT protocol, therefore enabling monitoring of real-time malware attacks for their prediction. In this work, many models of AI have been evaluated, among which the DNN model demonstrated superior accuracy and classification capability with an F1-score of 92\% and detection accuracy of 93\% on Edge-IIoTset and IoT-23. Concerns about the impact on system resources by specifying metrics are drawn to traffic and CPU usage on both devices, while challenges include the lack of ground-truth data in most cyberattacks. Future research shall 
be on few-shot learning, lightweight model implementation, DL cutting-edge methodologies, penetration testing, and the use of additional sensor and actuator data to enhance the anomaly detection system.

Using the IoT 2023 dataset as a thorough benchmark, the study of ____ tackles the issue of feature extraction from IoT data. The goal is to gain a better understanding of the dataset's properties and possible uses by evaluating both classic statistical approaches and ML-based methods. Feature extraction takes on more significance in the context of the IoT because of the "curse of dimensionality," the well-known fact that data processing and analysis get more complicated as the number of dimensions grows. Various techniques have been surveyed to place them in the context of their strengths in capturing relevant information, reducing dimensionality, and improving performance in IoT analytics. Some key findings in this respect include the Hughes phenomenon: classifier performance may get better with more features up to some optimal point before deteriorating. This paper guides the choice of suitable feature extraction methods to be deployed for various IoT applications via ample experiments and performance analysis. This will, therefore, help in the practical development of IoT solutions in 2023 and beyond. Besides, according to the authors, little effect of reducing features on the model performance is up to an accuracy of 93.04\% using Decision Trees and 93.05\% using Random Forest (RF) models. 

%This paper reviews CNN for anomaly detection within the Internet of Things networks ____ and, thus, tries to evaluate the performance of dimensions CNN1D, CNN2D, and CNN3D in the presence of normal and anomalous network data. It shows the models’ trustworthiness in detecting different cyber-attack types and maintaining the integrity of the IoT network traffic. Various datasets are used, including IoT Network Intrusion, Bot-IoT, MQTT-IoT-IDS2020, IoT-DS-2, IoT-23, and IoT-DS 1. This study concludes that CNN2D and CNN1D are very good at identifying anomalies in IoT networks because they are accurate and fast. Thus, from the current and future perspectives, these models are very promising in building a solid structure for intrusion detection in the network of computers. Moreover, the authors recommend that future research should be directed along the lines of other deep learning approaches, such as FFN, RNN, and GAN, that are suitable for transforming this system into a high anomaly detection one to rise to the challenges of the shifting paradigm in cyber security.

Using ML approaches, Prazeres \textit{et al.} (2022) evaluates AI-based malware detection in IoT network traffic. The study makes use of the IoT-23 dataset using real IoT network traffic of both benign and malicious, including numerous forms of malware including botnets and DDoS attacks. Key strategies to categorize network traffic are feature selection, data normalization, and the application of several ML models (Logistic Regression, RF, ANN, and Naïve Bayes) ____. 

%____ evaluates different algorithms of anomaly detection and classification using the IoT-23 dataset. The author has found that, out of those, the RF algorithm was most effective with an accuracy of 99.5\% and precision. It can be seen that ANNs have biases toward classes with higher occurrences, possibly due to neuron weight configurations, and the Support Vector Machine turned in the poorest result at an accuracy of 60\% which can’t predict benign captures, but it turned in a relatively high recall rating. The study concludes that RF is the best algorithm for detecting and classifying anomalies in the IoT-23 dataset, which was also revealed in the past by related studies, and proposes further research into the causes of high accuracy by simpler models and the potential of advanced neural networks to enable improved performance.

Especially in multi-class issues, RF has repeatedly shown to be a useful classifier in identifying intrusions. For example, with high accuracy rates in multi-class classification, a RF application to the IoT-23 dataset revealed better performance than other ML models____.

Likewise, XGBoost, known for its boosting power, has demonstrated extraordinary intrusion detection capability. As network IDS built for IoT networks show, XGBoost lowers mistakes and raises classification accuracy by iteratively strengthening weak models ____. Research also shows how well it can handle prevalent network dataset class imbalance problems ____.

%Intrusion detection systems have used ANNs to learn intricate connections inside network traffic data. Their versatility and great performance while managing big datasets help them shine in multi-class detection challenges. In one study, for example, ANNs were tested against various supervised learning techniques; Random Forest turned up as the best performance for multi-class intrusion detection. Nevertheless, ANNs also displayed competitive performance in precisely categorizing fraudulent traffic ____.

%CNNs' capacity to automatically learn geographic information from data has driven their increasing application to IDS. CNNs have proved quite effective in the framework of multi-class classification in spotting network breaches. For multi-class detection on the Bot-IoT dataset, for example, a CNN model was able to reach accuracy values near 99.9986\% ____. To increase the long-range dependent detection capability in network traffic, CNNs have also been coupled with other DL models such as Recurrent Neural Networks (RNNs). There has been a considerable improvement in the detection accuracy of learning the spatial and temporal components of the network traffic data when CNN is paired with Gated Recurrent Units (GRU) ____. 

Using multiclassification models within the PySpark architecture, a 2024 study by Alrefaei et al. ____ offers an IoT network real-time intrusion detection system (IDS). One-Vs- Rest (OVR) method ML approaches include RF, Decision Trees, Logistic Regression, and XGBoost help to enhance detection accuracy and minimize prediction latency. Class imbalance is solved via data cleansing, scaling, and SMote using the IoT-23 dataset. RF displayed the fastest prediction time at 0.0311 seconds, but XGBoost achieved the highest accuracy at 98.89\%, underlining the system's value in real-time IoT threat detection and so reducing security concerns.

Combining the advantages of several classifiers, ensemble approaches have shown notable gains in IDS performance. In particular, hybrid models, which combine classifiers in voting systems to improve overall accuracy and detection rate, have encouraging results. Upadhyay et al. (2021) presented a majority voting ensemble combining RF, XGBoost, and KNN with additional classifiers for SCADA-based power grids. Their model demonstrated gains in binary and multi-class classification by selecting features using Recursive Feature Elimination and then utilizing majority voting to increase precision and recall ____.

%Similarly, Hussein et al. (2021) ____ achieved greater detection rates across several assault categories by integrating AdaBoost, Random Forest, and XGBoost in an ensemble strategy for multi-class and binary classification. Their approach was verified on datasets like NSL-KDD and UNSW-NB15, and the ensemble performed better than individual classifiers regarding accuracy and precision.

A hybrid model that included RF, XGBoost, and KNN and was improved using feature selection approaches was also investigated by Liu et al. (2023). When tested on many datasets, their approach showed an increase in overall detection accuracy and a decrease in false positives. In this study, the voting system of a hybrid classifier, which combines other individual classifiers, is achieving better performance than stand-alone classifiers ____.

%Voting classifiers have emerged as a popular IDS technique, especially in IoT contexts where various threat types necessitate adaptable detection systems. A voting classifier that included AdaBoost, Random Forest, KNN, and SVC was suggested in research by Mhawi et al. (2022) ____, to handle the high-dimensional nature of network traffic. Particularly in situations involving multi-class classification, their hybrid model outperformed solo classifiers. On the CICIDS2017 dataset, the model showed accuracy gains of up to  99.7\%.

Research shows that a voting ensemble of classifiers including RF, XGBoost, AdaBoost, KNN, and SVC performs better than any classifier taken on alone. Leevy et al. (2021) evaluated different ensemble models, including XGBoost and RF, to find assaults in the framework of the IoT. Their studies show that in terms of adaptation and accuracy, ensemble models usually outperform individual classifiers ____.

%Rahman et al. (2021), who combined AdaBoost and XGBoost with different feature selection methods, further highlight the significance of feature selection in these hybrid models. Their hybrid model decreased computing overhead, which is crucial for real-time intrusion detection systems while increasing detection accuracy ____.

When testing IDS, the gold standard is the IoT-23 dataset, which simulates real-world IoT network traffic. It provides several attack scenarios that researchers can use to test the multi-class classification capabilities of DL and ML models. Detection rates surpass 99\% when models based on RF and CNN are used ____.

ElKashlan \textit{et al.} proposed a ML-based IDS to safeguard electric vehicle charging stations (EVCSs) in an IoT environment. They evaluated many machine-learning techniques to identify fraudulent traffic in both binary and multiclass traffic models using a real-world IoT dataset.  The suggested solution seeks to improve EVCS security by successfully detecting and thwarting cyberattacks, guaranteeing the ecosystem's stability, and guarding against possible interruptions to day-to-day operations.  Along with defining the EV charging ecosystem, the article looks at the many attack types that could happen and the vulnerabilities that each component faces. The traffic and possible assaults on an EVCS were represented by them using a native IoT dataset, IoT-23, which is derived from actual IoT devices ____.

To improve intrusion detection in IoT networks, Ullah et al. (2021) suggested a hybrid model that combines Convolutional Neural Networks (CNN) with Gated Recurrent Units (GRU). To accomplish reliable binary and multiclass classifications, it applies sophisticated preprocessing and feature selection algorithms to a variety of datasets (BoT-IoT, MQTT-IoT-IDS2020, etc.). The model outperforms earlier techniques with high accuracy (up to 99.96\%) and efficient detection rates across a range of cyberattack types. Its architecture uses strategies like dropout layers and early halting to overcome common problems like overfitting and data imbalance. In order to improve detection skills, future research areas will use Generative Adversarial Networks (GANs) ____. 

Using generative adversarial networks (GANs), Park et al. suggest an improved AI-based network intrusion detection system (NIDS). To identify anomalies and learn the typical behavior of network traffic, the system makes use of a DL model.  In order to increase the NIDS's accuracy, the study also creates fake data using GANs.  When tested on a benchmark dataset, the suggested system outperforms the most advanced NIDSs, according to the results.  The study comes to the conclusion that the suggested approach can be utilized to increase computer network security and is efficient at identifying network intrusions ____.

To identify cyberattacks in IoT networks, Abdalgawad et al. investigated the use of generative DL models, such as Adversarial Autoencoders (AAE) and Bidirectional Generative Adversarial Networks (BiGAN). They analyzed the IoT-23 dataset, which includes network traffic information from actual IoT devices, such as Amazon Echoes and smart door locks. These models were trained to recognize a variety of attacks, such as botnet activity from Mirai, Okiru, and Torii, as well as DDoS attacks. According to the study, these generative models outperformed more conventional ML techniques like RF in terms of intrusion detection. This study shows how generative DL can be used to enhance cybersecurity in the expanding IoT environment ____.


%