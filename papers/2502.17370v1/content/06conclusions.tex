\section{Conclusions}
A summary of the improvements, expressed in terms of improvement ratios in the bonuses, regret upper bounds, and empirical regret, is reported in Table~\ref{tab:ratios}. First, we compare our versions of the \texttt{UCBVI} algorithms with the original ones from~\citep{azar2017minimax}. The algorithmic structure remains the same, though we re-derived the bonus terms to make them as tight as possible, resulting in an improvement of $7/2$ and $\sqrt{2}$ for the Chernoff-Hoeffding and Bernstein-Friedmann bonuses, in the dominant terms, respectively. This reduction in over-exploration has significant empirical effects, as shown in Section~\ref{sec:experiments}, where, as reported in Table~\ref{tab:ratios}, we achieve an improvement in the empirical regret of $1.87$ times. Additionally, this impacts the regret analysis, where we were able to reduce the regret bound by a factor of $10$ and $5/4$ for the Chernoff-Hoeffding and Bernstein-Friedmann bonuses, respectively, in terms of dominant terms. However, lower order terms also have an impact on the performance, and through a refined analysis we were able to reduce them by a factor of $\sim \! 30$ and $\sim \! 4$ for the Chernoff-Hoeffding and Bernstein-Friedmann bonuses, respectively.