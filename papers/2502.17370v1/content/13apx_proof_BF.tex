\section{Proof of Theorem~\ref{thr:ucbvibfUB}}
\label{apx:proof_BF}

\ucbvibfUB*

Similarly to the proof of Theorem~\ref{thr:ucbvichUB} in Appendix~\ref{apx:proof_CH}, in order to demonstrate the upper bound of \ucbvibf, we first need to demonstrate optimism. However, in order to remove the additional $\sqrt{H}$ term, we are required to both demonstrate optimism as well as to bound by how much the optimistic value function estimator exceeds the true optimal value function.

We start by observing that:

\begin{align*}
    \regbf{K} &= \sum_{k \in \dsb{K}} \Vstar{1}(x_{k,1}) - \Vpi{1}(x_{k,1}) \\
    &\le \sum_{k \in \dsb{K}} \Vest{k,1}(x_{k,1}) - \Vpi{1}(x_{k,1}) \\
    &= \sum_{k \in \dsb{K}} \Vdifftil{k,1}(x_{k,1}) \\
    &= \regbftilde{K}.
\end{align*}

According to Lemma~\ref{lem:regr_dec}, under the events $\mathcal{E}$ and $\Omega_{k,h}$, we can decompose the pseudo-regret as:

\begin{equation}
    \sum_{i=1}^k \Vdifftil{i,h}(x_{i,h}) \le e \sum_{i=1}^k \sum_{j=h}^{H-1} \left[ \varepsilon_{i,j} + 2 \sqrt{L} \overline{\varepsilon}_{i,j} + \bonus_{i,j}(x_{i,j},a_{i,j}^{\pi_i}) + \xi_{i,j} (x_{i,j}, a_{i,j}^{\pi_i}) + \frac{8H^2SL}{3 N_i (x_{i,j}, a_{i,j}^{\pi_i})} \right]. \label{thr:ucbvibfUB:1}
\end{equation}

We also define, by trivially modifying the derivation of Lemma~\ref{lem:regr_dec}, the pseudo-regret considering only the episodes in which, at stage $h \in \dsb{H}$ a specific state $x \in \Ss$ was occupied:

\begin{align}
    \sum_{i=1}^k \mathbb{I}(x_{i,h} = x) \Vdiff{i,h}(x_{i,h}) &\le \sum_{i=1}^k \mathbb{I}(x_{i,h} = x) \Vdifftil{i,h}(x_{i,h}) \nonumber \\
    \begin{split}
    \label{thr:ucbvibfUB:2}
    &\le e \sum_{i=1}^k \mathbb{I}(x_{i,h} = x) \sum_{j=h}^{H-1} \bigg[ \varepsilon_{i,j} + 2 \sqrt{L} \overline{\varepsilon}_{i,j} + \bonus_{i,j}(x_{i,j},a_{i,j}^{\pi_i}) \\
    &\quad + \xi_{i,j} (x_{i,j}, a_{i,j}^{\pi_i}) + \frac{8H^2SL}{3 N_i (x_{i,j}, a_{i,j}^{\pi_i})} \bigg]. 
    \end{split}
\end{align}

By applying Lemmas~\ref{lem:sum_eps} and ~\ref{lem:sum_eps_x}, we can upper bound Equations~\eqref{thr:ucbvibfUB:1} and \eqref{thr:ucbvibfUB:2} as:

\begin{align*}
    \sum_{i=1}^k \Vdifftil{i,h}(x_{i,h}) &\le e \sum_{i=1}^k \sum_{j=h}^{H-1} \left[ \bonus_{i,j}(x_{i,j},a_{i,j}^{\pi_i}) + \xi_{i,j} (x_{i,j}, a_{i,j}^{\pi_i}) + \frac{8H^2SL}{3 N_i (x_{i,j}, a_{i,j}^{\pi_i})} \right] \\
    &\quad + 2e\sqrt{H^2T_kL} + 4e\sqrt{T_kL} \\
    &= \proxyUB,
\end{align*}

and:

\begin{align*}
    \sum_{i=1}^k \mathbb{I}(x_{i,h} = x) \Vdifftil{i,h}(x_{i,h}) &\le e \sum_{i=1}^k \mathbb{I}(x_{i,h} = x) \sum_{j=h}^{H-1} \left[ \bonus_{i,j}(x_{i,j},a_{i,j}^{\pi_i}) + \xi_{i,j} (x_{i,j}, a_{i,j}^{\pi_i}) + \frac{8H^2SL}{3 N_i (x_{i,j}, a_{i,j}^{\pi_i})} \right]\\
    &\quad + 2e\sqrt{H^3 N'_{k,h}(x) L} + 4e\sqrt{H N'_{k,h}(x) L} \\
    &= \proxyUBx,
\end{align*}

where we denote the upper bounds of $\sum_{i=1}^k \Vdifftil{i,h}(x_{i,h})$ and $\sum_{i=1}^k \mathbb{I}(x_{i,h} = x) \Vdifftil{i,h}(x_{i,h})$ as $\proxyUB$ and $\proxyUBx$, respectively, for ease of notation.

We now demonstrate optimism, which requires us to show that, with high probability, the event $\Omega_{k,h}$ holds.

\begin{lemma}[Optimism under Bernstein-Freedman bonus]
\label{lem:BF_opt}
Let the optimistic bonus be defined as:

\begin{align*}
    \bonus_{k,h}(x,a) &= \sqrt{\frac{4 L \Var_{y \sim \Pest_k (\cdot | x,a)} [\Vest{k,h+1}(y)]}{N_k(x,a)}} + \frac{7HL}{3 (N_k(x,a) - 1)} \\
    &\quad + \sqrt{\frac{4 \min\{\mathbb{E}_{y \sim \Pest_k(\cdot|x,a)} [\frac{84^2H^3S^2AL^2}{N'_{k,h+1}(y)}] , H^2\}}{N_k (x,a)}}.
\end{align*}

Then, under event $\mathcal{E}$, the following set of events hold:

\begin{equation*}
    \Omega_{k,h} \coloneqq \left\{ \Vest{i,j}(x) \ge \Vstar{j}(x), \forall (i,j) \in [k,h]_{\mathrm{hist}}, x \in \Ss \right\},
\end{equation*}

for $k \in \dsb{K}$ and $h \in \dsb{H}$, where:

\begin{equation*}
    [k,h]_{\mathrm{hist}} \coloneqq \left\{ (i,j) \in \dsb{K} \times \dsb{H} : i < k \vee (i=h, j \ge h) \right\}.
\end{equation*}
\end{lemma}

\begin{proof}
We demonstrate the result by induction. We begin by observing that $\Vest{k, H+1}(x) = \Vstar{H+1}(x) = 0$ for every $k \in \dsb{K}$ and $x \in \Ss$.
To prove the induction, we need to prove that, if $\Omega_{k,h}$ holds, then also $\Omega_{k,h-1}$ holds. We prove this for a generic $k \in \dsb{K}$, and we can then apply this procedure for increasing values of $k$, starting from $k=1$.

If $\Omega_{k,h}$ holds, then $\Vest{k,h}(x) \ge \Vstar{h}(x)$ for every $x \in \Ss$. We now bound the estimation error due to the optimistic approach:

\begin{align}
    \Vest{k,h}(x) - \Vstar{h}(x) &= \frac{1}{N'_{k,h}(x)} \sum_{i=1}^k \mathbb{I}(x_{i,h}=x) (\Vest{k,h}(x) - \Vstar{h}(x)) \nonumber \\
    &\le \frac{1}{N'_{k,h}(x)} \sum_{i=1}^k \mathbb{I}(x_{i,h}=x) (\Vest{i,h}(x) - \Vpi[i]{h}(x)) \label{lem:BF_opt:1} \\
    &= \frac{1}{N'_{k,h}(x)} \sum_{i=1}^k \mathbb{I}(x_{i,h} = x) \Vdifftil{i,h}(x_{i,h}), \label{lem:BF_opt:2}
\end{align}

where Equation~\eqref{lem:BF_opt:1} follows from the fact that $\Vest{k,h}$ is monotonically decreasing in $k$ by definition, and by observing that $\Vstar{h} \ge \Vpi[i]{h}$.

Recalling the upper bound of $\sum_{i=1}^k \mathbb{I}(x_{i,h} = x) \Vdifftil{i,h}(x_{i,h})$:

\begin{align}
    \sum_{i=1}^k \mathbb{I}(x_{i,h} = x) \Vdifftil{i,h}(x_{i,h}) &\le \proxyUBx \nonumber \\
    &= e \sum_{i=1}^k \mathbb{I}(x_{i,h} = x) \sum_{j=h}^{H-1} \bigg[ \bonus_{i,j}(x_{i,j},a_{i,j}^{\pi_i}) + \xi_{i,j} (x_{i,j}, a_{i,j}^{\pi_i}) \nonumber \\ 
    &\quad + \frac{8H^2SL}{3 N_i (x_{i,j}, a_{i,j}^{\pi_i})} \bigg] + 2e\sqrt{H^3 N'_{k,h}(x) L} + 4e\sqrt{H N'_{k,h}(x) L}, \nonumber
\end{align}

we now bound the summations over the terms in the summation over episodes and stages. By applying Lemma~\ref{lem:sum_b}, we can bound the summation over typical episodes of the bonus terms as:

\begin{align*}
    \sum_{i=1}^k \mathbb{I}(i \in \ktypx, x_{i,h}=x) \sum_{j=h}^{H-1} \bonus_{i,j}(x_{i,j}, a_{i,j}^{\pi_i}) &\le \sqrt{28 H^2SA N'_{k,h}(x) L^2} + \frac{7}{3} HSAL^2 \\
    &\quad + 2 \sqrt{84^2 H^3 S^4 A^2 L^4} + \sqrt{8 H^2 SAL^2 \proxyUBx},
\end{align*}

by observing that $\sum_{i=1}^k \mathbb{I}(x_{i,h} = x) \Vdiff{i,h}(x_{i,h}) \le \proxyUBx$ and that the series of $\proxyUBx$ terms is decreasing in $h$, as each term $\proxyUBx$ is a summation of elements each of which includes the next term, and as such we can upper bound $\sum_{j=h}^{H-1} U_{k,j,x}$ with $H \proxyUBx$.

In a similar way, we can apply the result of Lemma~\ref{lem:sum_c} to bound the summation over typical episodes of the state-action wise model error terms as:

\begin{align*}
    \sum_{i=1}^k \mathbb{I}(i \in \ktypx, x_{i,h}=x) &\sum_{j=h}^{H-1} \big[ \Pest_i(\cdot | x_{i,j}, a_{i,j}^{\pi_i}) - P(\cdot | x_{i,j}, a_{i,j}^{\pi_i}) \big]^\transpose \Vstar{j+1}(\cdot) \\
    &\le \sqrt{6 H^2 S A N'_{k,h}(x) L^2} + \frac{2}{3} HSAL^2 + 2\sqrt{H^2 S A L^2 \proxyUBx}.
\end{align*}

With the same procedure as in the proof of Lemma~\ref{lem:CH_opt}, we obtain the following upper bound:

\begin{equation}
\label{lem:BF_opt:9}
    \sum_{i=1}^K \sum_{j=1}^H \frac{8H^2SL}{3N_i(x_{i,j},a_{i,j}^{\pi_i})} \le \frac{8}{3} H^2 S^2 A L^2.
\end{equation}

By combining these result, and accounting for the regret on non-typical episodes, we can write:

\begin{align}
    \sum_{i=1}^k \mathbb{I}(x_{i,h} = x) \Vdiff{i,h}(x_{i,h}) &\le \proxyUBx \nonumber \\
    &\le e \bigg[ \sqrt{28 H^2 S A N'_{k,h}(x) L^2} + \frac{7}{3} HSAL^2 + 168 \sqrt{H^3 S^4 A^2 L^4} \nonumber \\
    &\quad + \sqrt{8 H^2 S A L^2 \proxyUBx} + \sqrt{6 H^2 S A N'_{k,h}(x) L^2} + \frac{2}{3}HSAL^2 \nonumber \\
    &\quad + 2\sqrt{H^2 S A L^2 \proxyUBx} + \frac{8}{3} H^2 S^2 A L^2 + 2 \sqrt{H^3 N'_{k,h}(x) L} \nonumber \\
    &\quad + 4 \sqrt{H N'_{k,h}(x)L} + 100 H^2 S^2 A L^2 \bigg] \nonumber \\
    &\le e \bigg[ 12 \sqrt{H^2 S A N'_{k,h}(x) L^2} + 5 \sqrt{H^2 S A L^2 \proxyUBx} \nonumber \\
    &\quad + \frac{821}{3} H^2 S^2 A L^2 + 2 \sqrt{H^3 N'_{k,h}(x) L}\bigg]. \nonumber
\end{align}

Letting:

\begin{align*}
    \alpha &= e \left[ 12 \sqrt{H^2 S A N'_{k,h}(x) L^2} + \frac{821}{3} H^2 S^2 A L^2 + 2 \sqrt{H^3 N'_{k,h}(x) L}\right], \\
    \beta &= 5e\sqrt{H^2 S^2 A L^2},
\end{align*}

we can solve for $\proxyUBx$ and obtain the following upper bound:

\begin{equation*}
    \proxyUBx \le \beta^2 + 2\alpha,
\end{equation*}

which we can write as:

\begin{align}
    \proxyUBx &\le 25 e^2 H^2 S^2 A L^2 + 24 e \sqrt{H^2 S A N'_{k,h}(x) L^2} + \frac{1642}{3} e H^2 S^2 A L^2 + 4 e \sqrt{H^3 N'_{k,h}(x) L} \nonumber \\
    &\le 24 e \sqrt{H^2 S A N'_{k,h}(x) L^2} + \frac{1846}{3} e H^2 S^2 A L^2 + 4 e \sqrt{H^3 N'_{k,h}(x) L} \nonumber \\
    &\le 28 e \sqrt{H^2 S A N'_{k,h}(x) L^2} + \frac{1846}{3} e H^2 S^2 A L^2 \label{lem:BF_opt:3} \\
    &\le 28 \cdot \frac{12}{11} e \sqrt{H^3 S^2 A N'_{k,h}(x) L^2} \label{lem:BF_opt:4} \\
    &\le 84 \sqrt{H^3 S^2 A N'_{k,h}(x) L^2}, \nonumber
\end{align}

where Equation~\eqref{lem:BF_opt:3} holds if $SA \ge H$, and Equation~\eqref{lem:BF_opt:4} holds under the condition of $\ktypx$.

Plugging this result into Equation~\eqref{lem:BF_opt:2}, and observing that the error cannot be greater than $H$, we get the following upper bound to the estimation error due to the optimistic approach:

\begin{equation}
\label{lem:BF_opt:5}
    \Vest{k,h}(x) - \Vstar{h}(x) \le \min \left\{ 84 \sqrt{\frac{H^3 S^2 A L^2}{N'_{k,h}(x)}}, H \right\}.
\end{equation}

Using this result, we now prove that $\Vest{k,h-1}(x) \ge \Vstar{h-1}(x)$. Let us recall the definition of $\Vest{k,h-1}(x)$:

\begin{equation*}
    \Vest{k,h-1}(x) = \min \left\{ \Vest{k-1, h-1}(x), H, \mathcal{T}_{h-1}^{\pi_k} \Vest{k,h} \right\},
\end{equation*}

where $\mathcal{T}_{h-1}^{\pi_k} \Vest{k,h} \coloneqq R^{\pi_k}(x_{k,h-1}) + \bonus_{k, h-1}(x_{k,h-1}, a_{k,h-1}^{\pi_k}) + \mathbb{E}_{y \sim \Pest_k(\cdot|x_{k,h-1}, a_{k,h-1}^{\pi_k})} \Vest{k,h}(y)$.
Observe that, if $\Vest{k, h-1}(x) = H$, then the optimism holds trivially. Also, if $\Vest{k, h-1}(x) = \Vest{k-1, h-1}(x)$, the optimism holds trivially under $\Omega_{k,h}$. As such, we only need to demonstrate the case in which $\Vest{k,h-1}(x) = \mathcal{T}_{h-1}^{\pi_k} \Vest{k,h}$. As such, we derive the following:

\begin{align}
    \Vest{k,h-1}(x) &- \Vstar{h-1}(x) \nonumber \\
    &= \max_{a \in \As} \left\{ R(x, a) + \bonus_{k,h-1}(x,a) + \sum_{y \in \Ss} \Pest_k (y | x, a) \Vest{k, h}(y) \right\} \nonumber \\
    &\quad - R(x, a_{k,h-1}^{\pi^*}) - \sum_{y \in \Ss} P(y | x, a_{k,h-1}^{\pi^*}) \Vstar{h}(y) \nonumber \\
    &\ge  \bonus_{k, h-1}(x, a_{k,h-1}^{\pi^*}) + \sum_{y \in \Ss} \Pest_k (y | x,a_{k,h-1}^{\pi^*}) \Vest{k, h}(y) \nonumber \\
    &\quad - \sum_{y \in \Ss} P(y | x, a_{k,h-1}^{\pi^*}) \Vstar{h}(y) \nonumber \\
    &= \bonus_{k, h-1}(x, a_{k,h-1}^{\pi^*}) + \sum_{y \in \Ss} \Pest_k (y | x,a_{k,h-1}^{\pi^*}) \left[ \Vest{k,h}(y) - \Vstar{h}(y) \right] \nonumber \\
    &\quad + \sum_{y \in \Ss} \left[ \Pest_k (y | x,a_{k,h-1}^{\pi^*}) - P(y | x, a_{k,h-1}^{\pi^*}) \right] \Vstar{h}(y) \nonumber \\
    &\ge \bonus_{k, h-1}(x, a_{k,h-1}^{\pi^*}) + \sum_{y \in \Ss} \left[ \Pest_k (y | x,a_{k,h-1}^{\pi^*}) - P(y | x, a_{k,h-1}^{\pi^*}) \right] \Vstar{h}(y) \label{lem:BF_opt:6}
\end{align}

where Equation~\eqref{lem:BF_opt:6} follows from the induction assumption.

Under event $\mathcal{E}$, we can apply the empirical Bernstein inequality \citep{maurer2009}:

\begin{equation*}
    \left| \sum_{y \in \Ss} \left[ \Pest_k (y | x,a) - P(y | x, a) \right] \Vstar{h}(y) \right| \le \sqrt{\frac{2 \VVestopt{k,h}(x,a)L}{N_k (x,a)}} + \frac{7 HL}{3 (N_k(x,a)-1)},
\end{equation*}

where $\VVestopt{k,h}(x,a) \coloneqq \Var_{y \sim \Pest_k (\cdot | x,a)} [\Vstar{h}(y)]$. As such, we obtain:

\begin{align}
    \Vest{k,h-1}(x) - \Vstar{h-1}(x) &\ge \bonus_{k, h-1}(x, a_{k,h-1}^{\pi^*}) - \sqrt{\frac{2 \VVestopt{k,h}(x, a_{k,h-1}^{\pi^*})L}{N_k (x, a_{k,h-1}^{\pi^*})}} - \frac{7 HL}{3 (N_k(x, a_{k,h-1}^{\pi^*})-1)} \nonumber \\
    \begin{split}
    \label{lem:BF_opt:7}
    &= \sqrt{\frac{4 \VVest{k,h}(x, a_{k,h-1}^{\pi^*})L}{N_k (x, a_{k,h-1}^{\pi^*})}} + \sqrt{\frac{4L \mathbb{E}_{y \sim \Pest_k (\cdot | x, a_{k,h-1}^{\pi^*})}\bonus'_{k,h}(y)}{N_k (x, a_{k,h-1}^{\pi^*})}} \\
    &\quad - \sqrt{\frac{2 \VVestopt{k,h}(x, a_{k,h-1}^{\pi^*})L}{N_k (x,x, a_{k,h-1}^{\pi^*})}}.
    \end{split}
\end{align}

We now bound $\VVestopt{k, h}$ in terms of $\VVest{k, h}$. Observing that:

\begin{align*}
    \Var[X] &= \mathbb{E}[X - \mathbb{E}[X]]^2 \\
    &= \mathbb{E}[X \pm Y - \mathbb{E}[X] \pm \mathbb{E}[Y]]^2 \\
    &= \mathbb{E}[(X-Y) - \mathbb{E}[X-Y] + Y - \mathbb{E}[Y]]^2 \\
    &\le 2 \mathbb{E}[(X-Y) - \mathbb{E}[X-Y]]^2 + 2 \mathbb{E}[Y - \mathbb{E}[Y]]^2 \\
    &= \Var[X-Y] + 2 \Var[Y],
\end{align*}

we can then rewrite:

\begin{align*}
    \VVestopt{k, h}(x, a_{k,h-1}^{\pi^*}) &\le 2 \VVest{k, h}(x, a_{k,h-1}^{\pi^*}) + 2 \Var_{y \sim \Pest_k (\cdot | x, a_{k,h-1}^{\pi^*})} [\Vstar{h}(y) - \Vest{k,h}(y)] \\
    &\le 2 \VVest{k, h}(x, a_{k,h-1}^{\pi^*}) + 2 \sum_{y \in \Ss} \Pest_k (\cdot | x, a_{k,h-1}^{\pi^*}) (\Vest{k,h}(y) - \Vstar{y})^2.
\end{align*}

By plugging this result into Equation~\eqref{lem:BF_opt:7}, we get:

\begin{equation*}
    \Vest{k,h-1}(x) - \Vstar{h-1}(x) \ge \sqrt{\frac{4L \mathbb{E}_{y \sim \Pest_k (\cdot | x, a_{k,h-1}^{\pi^*})}\bonus'_{k,h}(y)}{N_k (x, a_{k,h-1}^{\pi^*})}} - \sqrt{\frac{4L \mathbb{E}_{y \sim \Pest_k (\cdot | x, a_{k,h-1}^{\pi^*})}(\Vest{k,h}(y) - \Vstar{h}(y))^2}{N_k (x, a_{k,h-1}^{\pi^*})}}.
\end{equation*}

By applying the result of Equation~\eqref{lem:BF_opt:5} and the definition of $\bonus'_{k,h}(y)$, we finally obtain that $\Vest{k,h-1}(x) - \Vstar{h-1}(x) \ge 0$, thus demonstrating optimism.
\end{proof}

Having demonstrated optimism, we now prove the upper bound of the regret $\regbf$:

\begin{align}
    \regbftilde{K} &\le U_{K,1} \nonumber \\
    \begin{split}
    \label{lem:BF_opt:8}
    &= e \bigg[ \sqrt{28 HSATL^2} + \frac{7}{3}HSAL^2 + 2\sqrt{84^2 H^3 S^4 A^2 L^4} \\
    &\quad + \sqrt{8 HSAL^2 U_{K1}} +\sqrt{6 HSATL^2} + \frac{2}{3} HSAL^2 \\
    &\quad + 2 \sqrt{HSAL^2 U_{K,1}} + \frac{8}{3} H^2S^2AL^2 + 2\sqrt{H^2TL} \\
    &\quad + 4 \sqrt{TL} + 100 H^2 S^2 A L \bigg]
    \end{split} \\
    &\le e \bigg[ 12 \sqrt{HSATL^2} + 5 \sqrt{H^2 SAL^2 U_{K,1}} \nonumber \\
    &\quad + \frac{821}{3} H^2 S^2 A L^2 + 2 \sqrt{H^2 T L} \bigg]
\end{align}

where Equation~\eqref{lem:BF_opt:8} is obtained by applying the results of Lemmas~\ref{lem:sum_b} and \ref{lem:sum_c}, by applying the result of Equation~\eqref{lem:BF_opt:9}, and by accounting for the regret of non-typical episodes.

As done in Lemma~\ref{lem:BF_opt}, by letting:

\begin{align*}
    \alpha &= e \left[ 12 \sqrt{H S A T L^2} + \frac{821}{3} H^2 S^2 A L^2 + 2 \sqrt{H^2 T L}\right], \\
    \beta &= 5e\sqrt{H^2 S^2 A L^2},
\end{align*}

we can solve for $U_{K,1}$ and obtain:

\begin{align*}
    \regbftilde{K} &\le 24e \sqrt{HSATL^2} + \frac{1846}{3}e H^2 S^2 A L^2 + 4e \sqrt{H^2 T L} \\
    &\le 24e \sqrt{HSATL^2} + 616 e H^2 S^2 A L^2 + 4e \sqrt{H^2 T L}
\end{align*}

thus completing the proof.