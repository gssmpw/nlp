\section{A Refined \texttt{UCBVI} Analysis}
\label{sec:analysis}

In this section, we analyze the \ucbvi algorithm. We start in Section~\ref{sec:analysis:ch} by providing the regret upper bound for \ucbvi with \emph{Chernoff-Hoeffding} bonus (Theorem~\ref{thr:ucbvichUB}). Then, we consider the more elaborated \emph{Bernstein-Freedman} bonus, and we characterize the regret of \ucbvi we get when we consider such a bonus (Theorem~\ref{thr:ucbvibfUB}). Our contribution consists of refining the analysis to obtain tighter bonuses and, as a direct consequence, tighter regret bounds.

\subsection{Chernoff-Hoeffding}
\label{sec:analysis:ch}

Let us start with the \emph{Chernoff-Hoeffding} bound, taking the opportunity to correct some typos of the original analysis of~\citep{azar2017minimax}.\footnotetext[2]{We assume that, by definition, $\bonus_{k,H}(s,a) = 0$, as at the last stage, there is no need for exploration, and the rewards are deterministic.}
\addtocounter{footnote}{1}

\begin{restatable}[Regret for \ucbvi with Chernoff-Hoeffding bonus]{thr}{ucbvichUB}\label{thr:ucbvichUB}
Let $\delta > 0$. Considering:\footnotemark[2]
\begin{equation*}
    b_{k,h}(x,a) = \frac{2HL}{\sqrt{\max \{ N_{k}(x,a), 1 \}}}, 
\end{equation*}
then, \wp at least $1-\delta$, the regret of \ucbvich is bounded by:
\begin{equation*}
    \regch{K} \le 10 e HL \sqrt{SAT} + \frac{8}{3} e H^2 S^2 A L^2,
\end{equation*}
where $L = \ln{(5HSAT / \delta)}$. For $T \ge \Omega ( H^2 S^3 A )$, this bound translates to $\widetilde{\BigO}(H \sqrt{SAT})$.
\end{restatable}%
Theorem~\ref{thr:ucbvichUB} should be compared to Theorem~1 of~\citep{azar2017minimax}. Since the analysis is a refinement of the original analysis in terms of constants, the order of the regret does not change between the two theorems. However, our analysis provides a smaller value for the constants.\footnote{To the best of the authors' knowledge, the original analysis of~\citep{azar2017minimax} is missing a multiplicative $e$ factor in the regret bound.} Moreover, observe how the minimum value of $T$ for which the regret bound holds according to our analysis is $H$ times higher than the one reported in the original theorem. This is due to the fact that the minimum $T$ in the statement of Theorem~1 of~\citep{azar2017minimax} is incorrect, although the derivation in the appendix provides the same minimum value of $T$ we obtain.

\subsection{Bernstein-Freedman}
\label{sec:analysis:bf}

We now move to the \emph{Bernstein-Freedman} bonus.
\begin{restatable}[Regret for \ucbvi with Bernstein-Freedman bonus]{thr}{ucbvibfUB}\label{thr:ucbvibfUB}
Let $\delta > 0$. Considering:\footnotemark[2]  
\begin{align*}
    b_{k,h}(x,a) = & \underbrace{\sqrt{\frac{4L \Var_{y \sim \widehat{P}_k(\cdot | x,a)}(V_{k,h+1}(y))}{\max \{ N_{k}(x,a), 1 \}}}}_{\textup{\texttt{(A)}}} + \underbrace{\frac{7HL}{ 3 \max \{ N_{k}(x,a) - 1, 1 \}}}_{\textup{\texttt{(B)}}} \\ & \quad + \underbrace{\sqrt{\frac{4 \sum_{y \in \mathcal{S}} \left( \widehat{P}(y | x, a) \cdot \min \left\{ \frac{84^2 H^3 S^2 A L^2}{\max \{1, N_{k,h+1}'(y) \} } , H^2 \right\} \right)}{\max \{ N_{k}(x,a), 1 \}}}}_{\textup{\texttt{(C)}}},
\end{align*}
then, \wp at least $1-\delta$, the regret of \ucbvibf is bounded by:
\begin{equation*}
    \regbf{K} \le 24 e L \sqrt{HSAT} + 616 e H^2 S^2 A L^2 + 4 e \sqrt{H^2 TL},
\end{equation*}
where $L = \ln (5HSAT / \delta )$. For $T \ge \Omega ( H^3 S^3 A )$ and $SA \ge H$, this bound translates to $\widetilde{\BigO}(\sqrt{HSAT})$.
\end{restatable}

Theorem~\ref{thr:ucbvibfUB} should be compared to Theorem~2 of~\citep{azar2017minimax}. Again, as the analysis is a refinement in terms of constants, the order of the regret does not change. Moreover, also the minimum value of $T$ under which the regret bound matches the lower bound in unchanged between the two analyses. It is important to notice, however, that the constant terms of our analysis are strictly smaller than the ones of~\citep{azar2017minimax}.\footnotemark[3] In particular, comparing the bonus term $b_{k,h}$ presented in Theorem~\ref{thr:ucbvibfUB}  \wrt the one of \citep[][Algorithm~4]{azar2017minimax} we observe the following differences:
\begin{itemize}[topsep=1pt,noitemsep,leftmargin=15pt]
    \item in term \texttt{(A)}, we have a $\sqrt{4}$ multiplicative factor instead of $\sqrt{8}$;
    \item in term \texttt{(B)}, we have a $7$ multiplicative factor instead of $14$;
    \item in term \texttt{(C)}, we have a $\sqrt{4}$ multiplicative factor instead of $\sqrt{8}$ and, inside the minimum, a term multiplicative factor $84^2$ instead of $100^2$.
\end{itemize}
The reader shall refer to Appendices~\ref{apx:proof_CH} and \ref{apx:proof_BF} for the proofs of Theorems~\ref{thr:ucbvichUB} and \ref{thr:ucbvibfUB}, respectively. The derivations provided in the appendices closely follow the proofs of~\citep{azar2017minimax}, focusing on lowering the constant terms.
A full description of the notation employed throughout the paper is reported in Appendix~\ref{apx:notation}. Both proofs are conducted under the condition that concentration inequalities hold for the next state estimator and its variance. Those conditions fall under event $\mathcal{E}$, which is presented in Appendix~{B.4} of~\citep{azar2017minimax} and restated in Appendix~\ref{apx:event}. Finally, additional lemmas necessary to show the regret decomposition and to bound the summation of the term it comprises are demonstrated in Appendix~\ref{apx:lemmas}.