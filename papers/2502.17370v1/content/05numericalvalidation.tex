\section{Numerical Validation}
\label{sec:experiments}

In this section, we numerically compare the performances of \ucbvi, both with the Chernoff-Hoeffding and Bernstein-Freedman bonuses of~\citep{azar2017minimax} and with the improved Bernstein-Freedman bonus of this paper, against the \mvp algorithm.

In order to fairly compare to the \mvp algorithm, all the $N_h(x,a)$ terms are considered as $N(x,a)$, removing the discriminant of the stage from the algorithm, and the $c_2$ constant (which refers to the uncertainty in the estimation of the rewards) is set to $0$, to remove the exploration factor needed due to the stochasticity of the reward in the original paper.

\subsection{Illustrative Environments}

As a first experimental evaluation, we consider a set of illustrative environments. We consider an MDP with parameters $S=3$, $A=3$, $H \in \{5, 10\}$, and we consider a number of episodes $K \in \{10^5, 10^6\}$. We evaluate each experiment by averaging over 10 runs. In each run, the rewards and transition probabilities of the MDP are randomly generated. Then, the clairvoyant optimum is calculated for the purpose of regret computation, and the algorithms are evaluated.

\begin{figure*}[t!]
    \centering

    \hspace{0.8cm}
    \subfloat[$H=5, \ K=10^5$.]{\resizebox{0.37\linewidth}{!}{\includegraphics{img/h5k1e5_standalone.pdf}}}
    \hfill
    \subfloat[$H=5, \ K=10^6$.]{\resizebox{0.37\linewidth}{!}{\includegraphics{img/h5k1e6_standalone.pdf}}}
    \hspace{1cm}

    \vspace{0.3cm}

    \hspace{0.8cm}
    \subfloat[$H=10, \ K=10^5$.]{\resizebox{0.37\linewidth}{!}{\includegraphics{img/h10k1e5_standalone.pdf}}}
    \hfill
    \subfloat[$H=10, \ K=10^6$.]{\resizebox{0.37\linewidth}{!}{\includegraphics{img/h10k1e6_standalone.pdf}}}
    \hspace{1cm}
    
    \caption{Performances in terms of cumulative regret in toy environments with $S=3$ states and $A=3$ actions ($10$ runs, mean $\pm$ $95\%$ C.I.).}
    \label{fig:TODO2}
\end{figure*}


% \begin{figure}[t!]
% \centering

% %\hspace{0.6cm}
% \begin{subfigure}[b]{0.37\textwidth}
% \resizebox{.4\linewidth}{!}{% This file was created with tikzplotlib v0.10.1.
\begin{tikzpicture}

\definecolor{crimson2143940}{RGB}{214,39,40}
\definecolor{darkgray176}{RGB}{176,176,176}
\definecolor{darkorange25512714}{RGB}{255,127,14}
\definecolor{forestgreen4416044}{RGB}{44,160,44}
\definecolor{lightgray204}{RGB}{204,204,204}
\definecolor{mediumpurple148103189}{RGB}{148,103,189}
\definecolor{steelblue31119180}{RGB}{31,119,180}

\begin{axis}[
legend cell align={left},
legend style={
  fill opacity=0.8,
  draw opacity=1,
  text opacity=1,
  at={(0.03,0.97)},
  anchor=north west,
  draw=lightgray204,
  font=\small
},
tick align=outside,
tick pos=left,
x grid style={darkgray176},
xmin=0, xmax=100000,
xtick style={color=black},
y grid style={darkgray176},
ymin=-10000, ymax=210000,
ytick style={color=black},
grid=both,
xlabel={Episodes},
ylabel={Cumulative Regret},
grid style={dashed,gray!50!white}
]
\path [fill=steelblue31119180, fill opacity=0.3]
(axis cs:0,2.12672796801025)
--(axis cs:0,0.989704837421551)
--(axis cs:1010,1136.0508166159)
--(axis cs:2020,2205.91884463453)
--(axis cs:3030,3175.61943475829)
--(axis cs:4040,3922.81626692682)
--(axis cs:5050,4450.99133499234)
--(axis cs:6060,4843.68464295278)
--(axis cs:7070,5120.90961538935)
--(axis cs:8080,5360.28893824791)
--(axis cs:9090,5578.49298366381)
--(axis cs:10100,5778.49607680844)
--(axis cs:11111,5957.62703038494)
--(axis cs:12121,6124.09549236818)
--(axis cs:13131,6276.10942981811)
--(axis cs:14141,6414.51087959212)
--(axis cs:15151,6531.85196569804)
--(axis cs:16161,6637.28934693636)
--(axis cs:17171,6785.98803154887)
--(axis cs:18181,6854.53939962624)
--(axis cs:19191,6922.16498151456)
--(axis cs:20201,6977.50039679582)
--(axis cs:21211,7041.90692588171)
--(axis cs:22222,7104.1732632182)
--(axis cs:23232,7161.52133365692)
--(axis cs:24242,7215.1031432492)
--(axis cs:25252,7271.61866064002)
--(axis cs:26262,7322.67199496642)
--(axis cs:27272,7372.76467124873)
--(axis cs:28282,7419.17782638811)
--(axis cs:29292,7460.3643649432)
--(axis cs:30302,7508.43617125909)
--(axis cs:31312,7606.47773517732)
--(axis cs:32322,7647.52731071697)
--(axis cs:33333,7678.89002654522)
--(axis cs:34343,7715.60935040458)
--(axis cs:35353,7754.78307579332)
--(axis cs:36363,7782.87779335615)
--(axis cs:37373,7819.95872687578)
--(axis cs:38383,7846.95387917532)
--(axis cs:39393,7876.67738111017)
--(axis cs:40403,7913.63355893242)
--(axis cs:41413,7942.72458052481)
--(axis cs:42423,7972.95690815507)
--(axis cs:43433,7999.35963469123)
--(axis cs:44444,8031.00615642694)
--(axis cs:45454,8063.33188990578)
--(axis cs:46464,8091.34359371308)
--(axis cs:47474,8114.07822458533)
--(axis cs:48484,8142.1686980654)
--(axis cs:49494,8167.26443312439)
--(axis cs:50504,8193.23969541275)
--(axis cs:51514,8216.09921743858)
--(axis cs:52524,8240.3840691964)
--(axis cs:53534,8266.80071174889)
--(axis cs:54544,8291.29280746059)
--(axis cs:55555,8312.52877408823)
--(axis cs:56565,8331.13483664669)
--(axis cs:57575,8358.47260826635)
--(axis cs:58585,8380.86590979398)
--(axis cs:59595,8406.60934654177)
--(axis cs:60605,8431.31771033467)
--(axis cs:61615,8452.32336411823)
--(axis cs:62625,8475.83366259029)
--(axis cs:63635,8494.87522040904)
--(axis cs:64645,8519.27142110574)
--(axis cs:65655,8531.068838442)
--(axis cs:66666,8547.57863297076)
--(axis cs:67676,8569.01378260896)
--(axis cs:68686,8590.67743850856)
--(axis cs:69696,8613.17655061685)
--(axis cs:70706,8629.99456978704)
--(axis cs:71716,8648.46775042744)
--(axis cs:72726,8669.289755548)
--(axis cs:73736,8686.07607262032)
--(axis cs:74746,8699.24385038215)
--(axis cs:75756,8721.31457863979)
--(axis cs:76766,8740.51694967127)
--(axis cs:77777,8761.87475753713)
--(axis cs:78787,8783.6906337013)
--(axis cs:79797,8800.81058176716)
--(axis cs:80807,8814.68800314613)
--(axis cs:81817,8834.35178612937)
--(axis cs:82827,8851.83330217317)
--(axis cs:83837,8869.60299115167)
--(axis cs:84847,8886.45102047156)
--(axis cs:85857,8902.89131964998)
--(axis cs:86867,8918.84575711282)
--(axis cs:87877,8930.91320709942)
--(axis cs:88888,8951.72730602213)
--(axis cs:89898,8967.78396088857)
--(axis cs:90908,8979.81807973907)
--(axis cs:91918,8999.16694401072)
--(axis cs:92928,9017.11241899148)
--(axis cs:93938,9027.65605850855)
--(axis cs:94948,9045.80400989767)
--(axis cs:95958,9061.99562622224)
--(axis cs:96968,9075.74236970258)
--(axis cs:97978,9091.99560969165)
--(axis cs:98988,9107.62151463014)
--(axis cs:99999,9120.76699479393)
--(axis cs:99999,15568.2586879302)
--(axis cs:99999,15568.2586879302)
--(axis cs:98988,15551.9848958295)
--(axis cs:97978,15537.9811908127)
--(axis cs:96968,15523.5813638177)
--(axis cs:95958,15509.929507421)
--(axis cs:94948,15492.7232885444)
--(axis cs:93938,15477.2679747048)
--(axis cs:92928,15464.3318128948)
--(axis cs:91918,15448.5985083849)
--(axis cs:90908,15433.8841907391)
--(axis cs:89898,15421.8223710532)
--(axis cs:88888,15409.6913176567)
--(axis cs:87877,15393.0625930562)
--(axis cs:86867,15381.9278131395)
--(axis cs:85857,15366.572470013)
--(axis cs:84847,15350.9206713614)
--(axis cs:83837,15334.285237299)
--(axis cs:82827,15317.5067451947)
--(axis cs:81817,15299.2951377862)
--(axis cs:80807,15283.627066185)
--(axis cs:79797,15270.4962576473)
--(axis cs:78787,15252.2797721739)
--(axis cs:77777,15234.8665116188)
--(axis cs:76766,15216.2112438468)
--(axis cs:75756,15198.2232905611)
--(axis cs:74746,15178.0862881637)
--(axis cs:73736,15164.0327976221)
--(axis cs:72726,15145.8233712246)
--(axis cs:71716,15126.8436990611)
--(axis cs:70706,15107.0521088158)
--(axis cs:69696,15088.09081038)
--(axis cs:68686,15070.3223922406)
--(axis cs:67676,15053.7878357117)
--(axis cs:66666,15034.1231283877)
--(axis cs:65655,15017.7666421836)
--(axis cs:64645,15003.4822969855)
--(axis cs:63635,14981.5935600091)
--(axis cs:62625,14963.467694792)
--(axis cs:61615,14945.1798295189)
--(axis cs:60605,14926.9018948212)
--(axis cs:59595,14903.6874133403)
--(axis cs:58585,14883.0794715714)
--(axis cs:57575,14859.4745801951)
--(axis cs:56565,14833.7119947982)
--(axis cs:55555,14817.4328301016)
--(axis cs:54544,14795.9757465606)
--(axis cs:53534,14769.9330644217)
--(axis cs:52524,14745.2131947184)
--(axis cs:51514,14721.9553632155)
--(axis cs:50504,14700.3720568868)
--(axis cs:49494,14678.2827726314)
--(axis cs:48484,14655.7869716149)
--(axis cs:47474,14629.4583698984)
--(axis cs:46464,14605.9429692829)
--(axis cs:45454,14580.1801032162)
--(axis cs:44444,14549.7257004302)
--(axis cs:43433,14523.5191017989)
--(axis cs:42423,14495.1602361893)
--(axis cs:41413,14468.6892238494)
--(axis cs:40403,14440.4474743196)
--(axis cs:39393,14406.4035829461)
--(axis cs:38383,14378.5176664427)
--(axis cs:37373,14350.7770757511)
--(axis cs:36363,14319.6467071614)
--(axis cs:35353,14295.4859321004)
--(axis cs:34343,14255.9052841242)
--(axis cs:33333,14222.1099659713)
--(axis cs:32322,14192.6964802492)
--(axis cs:31312,14154.7034099434)
--(axis cs:30302,14098.6707805062)
--(axis cs:29292,14049.9132282567)
--(axis cs:28282,14007.3172198887)
--(axis cs:27272,13964.7586528122)
--(axis cs:26262,13923.3320888599)
--(axis cs:25252,13875.9597285784)
--(axis cs:24242,13824.6647773989)
--(axis cs:23232,13774.2210936361)
--(axis cs:22222,13718.2831366745)
--(axis cs:21211,13658.2296156459)
--(axis cs:20201,13600.945416934)
--(axis cs:19191,13447.2575767183)
--(axis cs:18181,13281.8834672117)
--(axis cs:17171,13118.6951812175)
--(axis cs:16161,12895.0460474458)
--(axis cs:15151,12602.3028635388)
--(axis cs:14141,12304.214024216)
--(axis cs:13131,12000.5650072858)
--(axis cs:12121,11692.6207725768)
--(axis cs:11111,11341.949069278)
--(axis cs:10100,10920.449199245)
--(axis cs:9090,10438.7661440383)
--(axis cs:8080,9858.14791596023)
--(axis cs:7070,9263.53004916521)
--(axis cs:6060,8653.41362390574)
--(axis cs:5050,7847.06583392513)
--(axis cs:4040,6805.028910857)
--(axis cs:3030,5561.54678376982)
--(axis cs:2020,3992.38245297008)
--(axis cs:1010,2123.2965777991)
--(axis cs:0,2.12672796801025)
--cycle;

\path [fill=darkorange25512714, fill opacity=0.3]
(axis cs:0,2.21792612537202)
--(axis cs:0,1.08276371379671)
--(axis cs:1010,1107.01114082508)
--(axis cs:2020,1974.6007501779)
--(axis cs:3030,2441.47694356578)
--(axis cs:4040,2698.15945770495)
--(axis cs:5050,2911.66213291425)
--(axis cs:6060,3084.58627730553)
--(axis cs:7070,3230.94773902739)
--(axis cs:8080,3355.14372039489)
--(axis cs:9090,3461.4027205001)
--(axis cs:10100,3521.72902468663)
--(axis cs:11111,3582.19744994567)
--(axis cs:12121,3641.72088373608)
--(axis cs:13131,3693.91638276472)
--(axis cs:14141,3742.01319338605)
--(axis cs:15151,3787.99685563272)
--(axis cs:16161,3851.78632471594)
--(axis cs:17171,3887.97818372042)
--(axis cs:18181,3924.05397494187)
--(axis cs:19191,3956.61556714529)
--(axis cs:20201,3989.91404635267)
--(axis cs:21211,4018.25892045985)
--(axis cs:22222,4044.35381840265)
--(axis cs:23232,4070.81248003743)
--(axis cs:24242,4096.39415419437)
--(axis cs:25252,4119.33807003159)
--(axis cs:26262,4148.1619470031)
--(axis cs:27272,4177.23142078862)
--(axis cs:28282,4203.22441542001)
--(axis cs:29292,4232.33094729769)
--(axis cs:30302,4252.2958169679)
--(axis cs:31312,4271.13430315652)
--(axis cs:32322,4293.06996850285)
--(axis cs:33333,4313.40599227668)
--(axis cs:34343,4336.3209738843)
--(axis cs:35353,4357.57193261946)
--(axis cs:36363,4378.48645281197)
--(axis cs:37373,4398.98842692469)
--(axis cs:38383,4417.7791405796)
--(axis cs:39393,4436.84615970368)
--(axis cs:40403,4456.92306239621)
--(axis cs:41413,4473.2359381349)
--(axis cs:42423,4494.08385997216)
--(axis cs:43433,4514.45105863491)
--(axis cs:44444,4532.84272077841)
--(axis cs:45454,4548.39747467405)
--(axis cs:46464,4565.30061420836)
--(axis cs:47474,4581.21391969083)
--(axis cs:48484,4594.65529146882)
--(axis cs:49494,4606.7031457859)
--(axis cs:50504,4624.29613702471)
--(axis cs:51514,4635.40341887548)
--(axis cs:52524,4644.04111847811)
--(axis cs:53534,4652.27017644359)
--(axis cs:54544,4667.82039010597)
--(axis cs:55555,4685.44682815021)
--(axis cs:56565,4700.47585416054)
--(axis cs:57575,4715.13628261898)
--(axis cs:58585,4729.49981664382)
--(axis cs:59595,4749.32832513821)
--(axis cs:60605,4763.53309667786)
--(axis cs:61615,4775.87910206565)
--(axis cs:62625,4790.54508657822)
--(axis cs:63635,4804.81515266745)
--(axis cs:64645,4817.09083205944)
--(axis cs:65655,4823.10705592954)
--(axis cs:66666,4831.29362667932)
--(axis cs:67676,4846.98370960589)
--(axis cs:68686,4858.29900260456)
--(axis cs:69696,4858.45031848782)
--(axis cs:70706,4872.73980627765)
--(axis cs:71716,4889.02617782358)
--(axis cs:72726,4900.23916083314)
--(axis cs:73736,4912.24543385844)
--(axis cs:74746,4924.84859033328)
--(axis cs:75756,4935.1265643217)
--(axis cs:76766,4939.18844663797)
--(axis cs:77777,4946.1279549706)
--(axis cs:78787,4951.47237266306)
--(axis cs:79797,4956.74257218308)
--(axis cs:80807,4968.09927809804)
--(axis cs:81817,4980.31182538088)
--(axis cs:82827,4989.08532060161)
--(axis cs:83837,4999.34232200636)
--(axis cs:84847,5008.36788258333)
--(axis cs:85857,5013.17605508467)
--(axis cs:86867,5025.95963990171)
--(axis cs:87877,5033.58583274818)
--(axis cs:88888,5050.74171796446)
--(axis cs:89898,5052.63221826415)
--(axis cs:90908,5060.59454753228)
--(axis cs:91918,5073.12098830592)
--(axis cs:92928,5083.90950212432)
--(axis cs:93938,5089.35402060294)
--(axis cs:94948,5094.63839018961)
--(axis cs:95958,5109.08105720392)
--(axis cs:96968,5118.54558952246)
--(axis cs:97978,5125.47484069235)
--(axis cs:98988,5134.72784294382)
--(axis cs:99999,5139.94358467702)
--(axis cs:99999,8241.3545493078)
--(axis cs:99999,8241.3545493078)
--(axis cs:98988,8236.99220698582)
--(axis cs:97978,8230.93335056926)
--(axis cs:96968,8225.8914917955)
--(axis cs:95958,8218.50932937888)
--(axis cs:94948,8208.77253551681)
--(axis cs:93938,8205.69093817702)
--(axis cs:92928,8200.65054819409)
--(axis cs:91918,8192.90766725182)
--(axis cs:90908,8181.44279509655)
--(axis cs:89898,8175.60653352903)
--(axis cs:88888,8170.55077637251)
--(axis cs:87877,8158.0602804862)
--(axis cs:86867,8151.38084761223)
--(axis cs:85857,8142.68265792263)
--(axis cs:84847,8137.44306396588)
--(axis cs:83837,8125.88342364182)
--(axis cs:82827,8115.20485659097)
--(axis cs:81817,8108.8980125799)
--(axis cs:80807,8100.87324198069)
--(axis cs:79797,8091.70448476602)
--(axis cs:78787,8084.52151912292)
--(axis cs:77777,8079.76965087697)
--(axis cs:76766,8073.80547537877)
--(axis cs:75756,8069.2086261392)
--(axis cs:74746,8058.63103762478)
--(axis cs:73736,8049.60883192943)
--(axis cs:72726,8038.58963225382)
--(axis cs:71716,8026.26202218332)
--(axis cs:70706,8015.34257745637)
--(axis cs:69696,8003.37503599162)
--(axis cs:68686,8001.24661889553)
--(axis cs:67676,7994.7186632502)
--(axis cs:66666,7985.40618928007)
--(axis cs:65655,7978.26363490576)
--(axis cs:64645,7974.86460946419)
--(axis cs:63635,7968.17014222453)
--(axis cs:62625,7954.39583355061)
--(axis cs:61615,7943.73051016088)
--(axis cs:60605,7931.28831004922)
--(axis cs:59595,7921.17763105604)
--(axis cs:58585,7903.78342215663)
--(axis cs:57575,7892.92972236044)
--(axis cs:56565,7879.89380948204)
--(axis cs:55555,7865.81360389451)
--(axis cs:54544,7852.06482830182)
--(axis cs:53534,7842.92971085979)
--(axis cs:52524,7832.85232544237)
--(axis cs:51514,7825.1177182158)
--(axis cs:50504,7810.67156104275)
--(axis cs:49494,7795.43293364062)
--(axis cs:48484,7784.67390573456)
--(axis cs:47474,7770.54820864113)
--(axis cs:46464,7757.27271326784)
--(axis cs:45454,7739.95658483148)
--(axis cs:44444,7727.1101130937)
--(axis cs:43433,7708.20399181253)
--(axis cs:42423,7691.75652048067)
--(axis cs:41413,7677.16670642766)
--(axis cs:40403,7663.28092313942)
--(axis cs:39393,7644.65659846233)
--(axis cs:38383,7627.78662124543)
--(axis cs:37373,7610.15575206617)
--(axis cs:36363,7589.06608086522)
--(axis cs:35353,7569.55571872603)
--(axis cs:34343,7549.2462233338)
--(axis cs:33333,7530.8126081519)
--(axis cs:32322,7509.56994862851)
--(axis cs:31312,7489.88607858621)
--(axis cs:30302,7470.49578425488)
--(axis cs:29292,7450.71997923876)
--(axis cs:28282,7425.45507357495)
--(axis cs:27272,7402.0198328123)
--(axis cs:26262,7373.13096807362)
--(axis cs:25252,7350.22294106085)
--(axis cs:24242,7330.50856900141)
--(axis cs:23232,7309.04365867635)
--(axis cs:22222,7283.77942071909)
--(axis cs:21211,7256.05821342115)
--(axis cs:20201,7224.2515201833)
--(axis cs:19191,7197.26613774129)
--(axis cs:18181,7168.44522092708)
--(axis cs:17171,7135.21038122687)
--(axis cs:16161,7099.62734588717)
--(axis cs:15151,7058.78213538587)
--(axis cs:14141,7017.97920394634)
--(axis cs:13131,6970.40245277935)
--(axis cs:12121,6920.07261081523)
--(axis cs:11111,6868.76387218753)
--(axis cs:10100,6813.00267022947)
--(axis cs:9090,6657.82186515948)
--(axis cs:8080,6454.41730717858)
--(axis cs:7070,6155.02080943785)
--(axis cs:6060,5852.00963202319)
--(axis cs:5050,5469.90329993599)
--(axis cs:4040,4939.31344595442)
--(axis cs:3030,4342.75950386894)
--(axis cs:2020,3413.45035513781)
--(axis cs:1010,2003.75586318984)
--(axis cs:0,2.21792612537202)
--cycle;

\path [fill=forestgreen4416044, fill opacity=0.3]
(axis cs:0,2.40250583151742)
--(axis cs:0,1.09601013821527)
--(axis cs:1010,1094.32788632237)
--(axis cs:2020,2247.41100310568)
--(axis cs:3030,3409.37933061748)
--(axis cs:4040,4554.79724165368)
--(axis cs:5050,5624.18814182473)
--(axis cs:6060,6679.27418390429)
--(axis cs:7070,7743.23255241017)
--(axis cs:8080,8815.06378513604)
--(axis cs:9090,9946.8216536639)
--(axis cs:10100,11131.3593417922)
--(axis cs:11111,12303.8402191226)
--(axis cs:12121,13466.7645998067)
--(axis cs:13131,14620.6659375937)
--(axis cs:14141,15734.3310521365)
--(axis cs:15151,16831.99499572)
--(axis cs:16161,17904.1193845956)
--(axis cs:17171,18986.7484750342)
--(axis cs:18181,20067.6641989814)
--(axis cs:19191,21100.486129376)
--(axis cs:20201,22114.533189346)
--(axis cs:21211,23124.7297442796)
--(axis cs:22222,24133.8883365339)
--(axis cs:23232,25212.7032359052)
--(axis cs:24242,26325.9767611875)
--(axis cs:25252,27430.6914655702)
--(axis cs:26262,28555.976084346)
--(axis cs:27272,29675.2189175571)
--(axis cs:28282,30774.8323971425)
--(axis cs:29292,31878.7024371631)
--(axis cs:30302,32956.9376420792)
--(axis cs:31312,34006.123505782)
--(axis cs:32322,35043.7818965023)
--(axis cs:33333,36035.3384859076)
--(axis cs:34343,37018.9878316741)
--(axis cs:35353,38007.8628938671)
--(axis cs:36363,38997.0771218783)
--(axis cs:37373,40017.4221446739)
--(axis cs:38383,41031.1802693902)
--(axis cs:39393,42038.3252072418)
--(axis cs:40403,43044.4285958382)
--(axis cs:41413,44058.1381987507)
--(axis cs:42423,45080.0669505515)
--(axis cs:43433,46103.2247481186)
--(axis cs:44444,47129.1259146039)
--(axis cs:45454,48136.8470396796)
--(axis cs:46464,49128.6911664779)
--(axis cs:47474,50125.103355617)
--(axis cs:48484,51112.8590320838)
--(axis cs:49494,52094.64006033)
--(axis cs:50504,53068.8992359737)
--(axis cs:51514,54032.3915673497)
--(axis cs:52524,54978.1871117884)
--(axis cs:53534,55898.5144229784)
--(axis cs:54544,56814.6027335803)
--(axis cs:55555,57721.9733925358)
--(axis cs:56565,58619.4780315506)
--(axis cs:57575,59524.3526521583)
--(axis cs:58585,60420.129514201)
--(axis cs:59595,61309.9806435893)
--(axis cs:60605,62211.3466061149)
--(axis cs:61615,63110.7432963844)
--(axis cs:62625,64009.4868826069)
--(axis cs:63635,64900.3924116835)
--(axis cs:64645,65807.7089980114)
--(axis cs:65655,66714.457086237)
--(axis cs:66666,67623.1963850659)
--(axis cs:67676,68540.1807164086)
--(axis cs:68686,69460.4593190888)
--(axis cs:69696,70416.2646091987)
--(axis cs:70706,71362.7171203524)
--(axis cs:71716,72278.4029595981)
--(axis cs:72726,73207.8091547795)
--(axis cs:73736,74145.0329444627)
--(axis cs:74746,75062.622745596)
--(axis cs:75756,75964.3958106626)
--(axis cs:76766,76857.4204675862)
--(axis cs:77777,77755.8011397661)
--(axis cs:78787,78621.3845435071)
--(axis cs:79797,79491.2367458918)
--(axis cs:80807,80360.2596853201)
--(axis cs:81817,81220.8489870562)
--(axis cs:82827,82084.0389517674)
--(axis cs:83837,82910.0736084312)
--(axis cs:84847,83730.0609815232)
--(axis cs:85857,84540.3529729082)
--(axis cs:86867,85339.246879936)
--(axis cs:87877,86123.1427347261)
--(axis cs:88888,86907.2763916776)
--(axis cs:89898,87688.1085752769)
--(axis cs:90908,88461.1714439391)
--(axis cs:91918,89232.7461747111)
--(axis cs:92928,90002.4812092635)
--(axis cs:93938,90765.5901511271)
--(axis cs:94948,91526.6063381963)
--(axis cs:95958,92296.3019864404)
--(axis cs:96968,93064.9491851444)
--(axis cs:97978,93807.7030059914)
--(axis cs:98988,94538.3545627564)
--(axis cs:99999,95257.4249110953)
--(axis cs:99999,164866.923422048)
--(axis cs:99999,164866.923422048)
--(axis cs:98988,163633.864823852)
--(axis cs:97978,162399.990296425)
--(axis cs:96968,161161.134384874)
--(axis cs:95958,159918.221553615)
--(axis cs:94948,158675.266426971)
--(axis cs:93938,157436.631126071)
--(axis cs:92928,156197.920338432)
--(axis cs:91918,154952.06710375)
--(axis cs:90908,153708.344417114)
--(axis cs:89898,152437.624024119)
--(axis cs:88888,151132.009568358)
--(axis cs:87877,149828.327805662)
--(axis cs:86867,148519.16232997)
--(axis cs:85857,147209.682986679)
--(axis cs:84847,145894.089187885)
--(axis cs:83837,144584.986249828)
--(axis cs:82827,143250.213045871)
--(axis cs:81817,141856.030464878)
--(axis cs:80807,140458.704830139)
--(axis cs:79797,139068.987455923)
--(axis cs:78787,137670.308679734)
--(axis cs:77777,136254.758841586)
--(axis cs:76766,134809.527229002)
--(axis cs:75756,133335.373953167)
--(axis cs:74746,131856.785800199)
--(axis cs:73736,130366.780936107)
--(axis cs:72726,128869.612088534)
--(axis cs:71716,127379.293413427)
--(axis cs:70706,125889.90194787)
--(axis cs:69696,124381.951389303)
--(axis cs:68686,122866.835236251)
--(axis cs:67676,121362.528211002)
--(axis cs:66666,119864.737414941)
--(axis cs:65655,118365.178887769)
--(axis cs:64645,116868.006963259)
--(axis cs:63635,115358.845401534)
--(axis cs:62625,113853.258831885)
--(axis cs:61615,112324.025970404)
--(axis cs:60605,110785.348042743)
--(axis cs:59595,109236.273155433)
--(axis cs:58585,107682.151522055)
--(axis cs:57575,106099.959290359)
--(axis cs:56565,104513.761580952)
--(axis cs:55555,102930.490754867)
--(axis cs:54544,101334.04454276)
--(axis cs:53534,99744.0663370786)
--(axis cs:52524,98139.9402194094)
--(axis cs:51514,96525.9816040761)
--(axis cs:50504,94870.6942880071)
--(axis cs:49494,93173.2149143449)
--(axis cs:48484,91467.5723380344)
--(axis cs:47474,89755.3868628924)
--(axis cs:46464,88017.930683979)
--(axis cs:45454,86274.8748722188)
--(axis cs:44444,84519.5860357129)
--(axis cs:43433,82763.4476588837)
--(axis cs:42423,81008.3614923678)
--(axis cs:41413,79297.7410749191)
--(axis cs:40403,77601.4783258623)
--(axis cs:39393,75917.7644682741)
--(axis cs:38383,74242.5282099881)
--(axis cs:37373,72548.8040722443)
--(axis cs:36363,70853.7603886705)
--(axis cs:35353,69172.1533385184)
--(axis cs:34343,67459.202221057)
--(axis cs:33333,65739.472337956)
--(axis cs:32322,63938.8051819156)
--(axis cs:31312,62087.3431368071)
--(axis cs:30302,60234.2969030032)
--(axis cs:29292,58364.6062124062)
--(axis cs:28282,56472.8597150521)
--(axis cs:27272,54590.0694471908)
--(axis cs:26262,52706.4630362092)
--(axis cs:25252,50819.2825844619)
--(axis cs:24242,48936.6544387564)
--(axis cs:23232,47047.3652810127)
--(axis cs:22222,45169.3541596521)
--(axis cs:21211,43305.6764998265)
--(axis cs:20201,41423.4045859245)
--(axis cs:19191,39510.2280752987)
--(axis cs:18181,37590.9078012425)
--(axis cs:17171,35612.8557218574)
--(axis cs:16161,33554.7018675037)
--(axis cs:15151,31501.9097569534)
--(axis cs:14141,29429.3411620095)
--(axis cs:13131,27355.4125620106)
--(axis cs:12121,25262.4398054847)
--(axis cs:11111,23173.9871242168)
--(axis cs:10100,21097.5151873988)
--(axis cs:9090,19027.7570119337)
--(axis cs:8080,16981.6028743457)
--(axis cs:7070,14952.7066387908)
--(axis cs:6060,12919.4094747868)
--(axis cs:5050,10871.4396769691)
--(axis cs:4040,8797.36381255243)
--(axis cs:3030,6623.81172847873)
--(axis cs:2020,4428.77392197599)
--(axis cs:1010,2241.8621936024)
--(axis cs:0,2.40250583151742)
--cycle;

\path [fill=mediumpurple148103189, fill opacity=0.3]
(axis cs:0,2.31488019312847)
--(axis cs:0,0.868262069918526)
--(axis cs:1010,1074.50691217104)
--(axis cs:2020,2164.68235939076)
--(axis cs:3030,3253.78462964337)
--(axis cs:4040,4385.15361945931)
--(axis cs:5050,5546.23013122637)
--(axis cs:6060,6679.99875067075)
--(axis cs:7070,7838.03708649872)
--(axis cs:8080,9021.85081280183)
--(axis cs:9090,10179.818032479)
--(axis cs:10100,11340.1021340096)
--(axis cs:11111,12504.196816144)
--(axis cs:12121,13656.18996044)
--(axis cs:13131,14813.5690605177)
--(axis cs:14141,15974.9303016679)
--(axis cs:15151,17134.7270955831)
--(axis cs:16161,18291.8160533957)
--(axis cs:17171,19459.4282229781)
--(axis cs:18181,20617.1389979397)
--(axis cs:19191,21774.5513001085)
--(axis cs:20201,22922.9465650219)
--(axis cs:21211,24073.2169952323)
--(axis cs:22222,25238.8028788517)
--(axis cs:23232,26422.6095783421)
--(axis cs:24242,27593.7885318184)
--(axis cs:25252,28758.2702704469)
--(axis cs:26262,29869.1676930679)
--(axis cs:27272,30994.9577559549)
--(axis cs:28282,32090.1472118881)
--(axis cs:29292,33189.1677923229)
--(axis cs:30302,34296.0513580758)
--(axis cs:31312,35366.9316971435)
--(axis cs:32322,36420.7643708573)
--(axis cs:33333,37489.2681051405)
--(axis cs:34343,38549.3267579363)
--(axis cs:35353,39614.1495548302)
--(axis cs:36363,40684.2672336878)
--(axis cs:37373,41750.4188899543)
--(axis cs:38383,42870.2790990932)
--(axis cs:39393,44002.5758758641)
--(axis cs:40403,45138.5056750932)
--(axis cs:41413,46261.7423842319)
--(axis cs:42423,47373.8783548463)
--(axis cs:43433,48480.1505444083)
--(axis cs:44444,49595.4366440776)
--(axis cs:45454,50712.4515955396)
--(axis cs:46464,51828.5171754247)
--(axis cs:47474,52911.9183956542)
--(axis cs:48484,53996.3760008443)
--(axis cs:49494,55081.8876187364)
--(axis cs:50504,56182.011926694)
--(axis cs:51514,57270.3213745054)
--(axis cs:52524,58366.8717556537)
--(axis cs:53534,59476.5206373439)
--(axis cs:54544,60584.0786754848)
--(axis cs:55555,61693.5858201193)
--(axis cs:56565,62794.5189090051)
--(axis cs:57575,63893.303744892)
--(axis cs:58585,64986.0666275122)
--(axis cs:59595,66083.5290504707)
--(axis cs:60605,67196.201090627)
--(axis cs:61615,68312.8336529239)
--(axis cs:62625,69420.4188792814)
--(axis cs:63635,70538.3225740496)
--(axis cs:64645,71668.8570017732)
--(axis cs:65655,72814.773514276)
--(axis cs:66666,73958.2686643663)
--(axis cs:67676,75061.8267183637)
--(axis cs:68686,76191.7421469046)
--(axis cs:69696,77310.1216632306)
--(axis cs:70706,78429.4661318491)
--(axis cs:71716,79538.4326997327)
--(axis cs:72726,80631.873365505)
--(axis cs:73736,81716.4008635905)
--(axis cs:74746,82814.1611499951)
--(axis cs:75756,83906.792391873)
--(axis cs:76766,84982.5616052681)
--(axis cs:77777,86054.8652911518)
--(axis cs:78787,87132.6087879457)
--(axis cs:79797,88259.9770923282)
--(axis cs:80807,89438.4994766609)
--(axis cs:81817,90625.9026614734)
--(axis cs:82827,91800.1576186453)
--(axis cs:83837,92955.5015192638)
--(axis cs:84847,94121.0116846145)
--(axis cs:85857,95296.435063673)
--(axis cs:86867,96409.8380684935)
--(axis cs:87877,97534.7606808348)
--(axis cs:88888,98641.2201379843)
--(axis cs:89898,99730.61553998)
--(axis cs:90908,100817.055482735)
--(axis cs:91918,101906.643026647)
--(axis cs:92928,102993.377214276)
--(axis cs:93938,104047.586825433)
--(axis cs:94948,105107.871287712)
--(axis cs:95958,106167.774657225)
--(axis cs:96968,107262.444887348)
--(axis cs:97978,108394.39380905)
--(axis cs:98988,109528.481212154)
--(axis cs:99999,110668.067126993)
--(axis cs:99999,204538.935424241)
--(axis cs:99999,204538.935424241)
--(axis cs:98988,202566.865182552)
--(axis cs:97978,200573.506786671)
--(axis cs:96968,198614.673958442)
--(axis cs:95958,196674.955139456)
--(axis cs:94948,194749.684463354)
--(axis cs:93938,192823.379651334)
--(axis cs:92928,190899.968034403)
--(axis cs:91918,188966.937793263)
--(axis cs:90908,187029.455880743)
--(axis cs:89898,185100.086302876)
--(axis cs:88888,183166.171597092)
--(axis cs:87877,181205.132875555)
--(axis cs:86867,179254.203999374)
--(axis cs:85857,177318.780401793)
--(axis cs:84847,175349.963794063)
--(axis cs:83837,173396.506319965)
--(axis cs:82827,171440.570430485)
--(axis cs:81817,169472.088092118)
--(axis cs:80807,167438.368118242)
--(axis cs:79797,165419.989675646)
--(axis cs:78787,163383.420990391)
--(axis cs:77777,161354.105426116)
--(axis cs:76766,159319.329477085)
--(axis cs:75756,157282.113536381)
--(axis cs:74746,155217.685925765)
--(axis cs:73736,153221.734572229)
--(axis cs:72726,151231.406111857)
--(axis cs:71716,149237.406278931)
--(axis cs:70706,147239.722557601)
--(axis cs:69696,145197.272904016)
--(axis cs:68686,143133.276798567)
--(axis cs:67676,141062.858097044)
--(axis cs:66666,138997.603445032)
--(axis cs:65655,136916.46105209)
--(axis cs:64645,134850.317491678)
--(axis cs:63635,132785.05937779)
--(axis cs:62625,130727.497537562)
--(axis cs:61615,128678.465734852)
--(axis cs:60605,126629.012082553)
--(axis cs:59595,124623.798228742)
--(axis cs:58585,122630.414130275)
--(axis cs:57575,120636.081055314)
--(axis cs:56565,118623.557217732)
--(axis cs:55555,116610.542869922)
--(axis cs:54544,114600.2911919)
--(axis cs:53534,112598.492028073)
--(axis cs:52524,110612.151596504)
--(axis cs:51514,108630.452115934)
--(axis cs:50504,106662.491402103)
--(axis cs:49494,104681.325455714)
--(axis cs:48484,102679.651809898)
--(axis cs:47474,100673.054606364)
--(axis cs:46464,98667.7795019177)
--(axis cs:45454,96641.9540829634)
--(axis cs:44444,94618.7178115733)
--(axis cs:43433,92594.7701387034)
--(axis cs:42423,90583.0044724815)
--(axis cs:41413,88571.9722214619)
--(axis cs:40403,86540.5258663116)
--(axis cs:39393,84500.0813647026)
--(axis cs:38383,82454.1453137458)
--(axis cs:37373,80403.4952557768)
--(axis cs:36363,78345.90563767)
--(axis cs:35353,76288.6733406536)
--(axis cs:34343,74240.1418387082)
--(axis cs:33333,72156.0048678229)
--(axis cs:32322,70071.5417511807)
--(axis cs:31312,68001.2981821227)
--(axis cs:30302,65925.2782617324)
--(axis cs:29292,63812.9429723988)
--(axis cs:28282,61695.5468283693)
--(axis cs:27272,59580.0546849015)
--(axis cs:26262,57456.2252652472)
--(axis cs:25252,55344.5563683753)
--(axis cs:24242,53184.2614489686)
--(axis cs:23232,50994.8756358092)
--(axis cs:22222,48771.0566471432)
--(axis cs:21211,46544.0538812053)
--(axis cs:20201,44344.8480577647)
--(axis cs:19191,42153.3236206795)
--(axis cs:18181,39956.7033980351)
--(axis cs:17171,37755.9761544977)
--(axis cs:16161,35550.3124860752)
--(axis cs:15151,33354.5795334337)
--(axis cs:14141,31157.9349012239)
--(axis cs:13131,28950.5164209854)
--(axis cs:12121,26763.0687583836)
--(axis cs:11111,24573.3974435089)
--(axis cs:10100,22374.3792287065)
--(axis cs:9090,20178.3011792662)
--(axis cs:8080,17979.4755255706)
--(axis cs:7070,15765.8130447654)
--(axis cs:6060,13576.7498859571)
--(axis cs:5050,11383.2181958185)
--(axis cs:4040,9136.57144671786)
--(axis cs:3030,6856.56843075239)
--(axis cs:2020,4570.9076410381)
--(axis cs:1010,2283.53541696726)
--(axis cs:0,2.31488019312847)
--cycle;

\addplot [very thick, steelblue31119180]
table {%
0 1.5582164027159
1010 1629.6736972075
2020 3099.15064880231
3030 4368.58310926405
4040 5363.92258889191
5050 6149.02858445874
6060 6748.54913342926
7070 7192.21983227728
8080 7609.21842710407
9090 8008.62956385105
10100 8349.47263802672
11111 8649.78804983148
12121 8908.35813247249
13131 9138.33721855193
14141 9359.36245190404
15151 9567.07741461841
16161 9766.1676971911
17171 9952.3416063832
18181 10068.211433419
19191 10184.7112791165
20201 10289.2229068649
21211 10350.0682707638
22222 10411.2281999463
23232 10467.8712136465
24242 10519.883960324
25252 10573.7891946092
26262 10623.0020419131
27272 10668.7616620305
28282 10713.2475231384
29292 10755.1387965999
30302 10803.5534758826
31312 10880.5905725603
32322 10920.1118954831
33333 10950.4999962583
34343 10985.7573172644
35353 11025.1345039469
36363 11051.2622502588
37373 11085.3679013134
38383 11112.735772809
39393 11141.5404820281
40403 11177.040516626
41413 11205.7069021871
42423 11234.0585721722
43433 11261.439368245
44444 11290.3659284286
45454 11321.755996561
46464 11348.643281498
47474 11371.7682972418
48484 11398.9778348402
49494 11422.7736028779
50504 11446.8058761498
51514 11469.0272903271
52524 11492.7986319574
53534 11518.3668880853
54544 11543.6342770106
55555 11564.9808020949
56565 11582.4234157224
57575 11608.9735942307
58585 11631.9726906827
59595 11655.148379941
60605 11679.1098025779
61615 11698.7515968186
62625 11719.6506786911
63635 11738.2343902091
64645 11761.3768590456
65655 11774.4177403128
66666 11790.8508806792
67676 11811.4008091603
68686 11830.4999153746
69696 11850.6336804984
70706 11868.5233393014
71716 11887.6557247443
72726 11907.5565633863
73736 11925.0544351212
74746 11938.6650692729
75756 11959.7689346005
76766 11978.364096759
77777 11998.370634578
78787 12017.9852029376
79797 12035.6534197072
80807 12049.1575346656
81817 12066.8234619578
82827 12084.670023684
83837 12101.9441142253
84847 12118.6858459165
85857 12134.7318948315
86867 12150.3867851262
87877 12161.9879000778
88888 12180.7093118394
89898 12194.8031659709
90908 12206.8511352391
91918 12223.8827261978
92928 12240.7221159432
93938 12252.4620166067
94948 12269.263649221
95958 12285.9625668216
96968 12299.6618667601
97978 12314.9884002522
98988 12329.8032052298
99999 12344.5128413621
};
\addlegendentry{\ucbvibf}
\addplot [very thick, darkorange25512714]
table {%
0 1.65034491958437
1010 1555.38350200746
2020 2694.02555265785
3030 3392.11822371736
4040 3818.73645182968
5050 4190.78271642512
6060 4468.29795466436
7070 4692.98427423262
8080 4904.78051378674
9090 5059.61229282979
10100 5167.36584745805
11111 5225.4806610666
12121 5280.89674727566
13131 5332.15941777204
14141 5379.9961986662
15151 5423.3894955093
16161 5475.70683530155
17171 5511.59428247365
18181 5546.24959793448
19191 5576.94085244329
20201 5607.08278326798
21211 5637.1585669405
22222 5664.06661956087
23232 5689.92806935689
24242 5713.45136159789
25252 5734.78050554622
26262 5760.64645753836
27272 5789.62562680046
28282 5814.33974449748
29292 5841.52546326822
30302 5861.39580061139
31312 5880.51019087137
32322 5901.31995856568
33333 5922.10930021429
34343 5942.78359860905
35353 5963.56382567275
36363 5983.77626683859
37373 6004.57208949543
38383 6022.78288091252
39393 6040.751379083
40403 6060.10199276782
41413 6075.20132228128
42423 6092.92019022642
43433 6111.32752522372
44444 6129.97641693605
45454 6144.17702975277
46464 6161.2866637381
47474 6175.88106416598
48484 6189.66459860169
49494 6201.06803971326
50504 6217.48384903373
51514 6230.26056854564
52524 6238.44672196024
53534 6247.59994365169
54544 6259.94260920389
55555 6275.63021602236
56565 6290.18483182129
57575 6304.03300248971
58585 6316.64161940022
59595 6335.25297809712
60605 6347.41070336354
61615 6359.80480611327
62625 6372.47046006442
63635 6386.49264744599
64645 6395.97772076181
65655 6400.68534541765
66666 6408.34990797969
67676 6420.85118642805
68686 6429.77281075005
69696 6430.91267723972
70706 6444.04119186701
71716 6457.64410000345
72726 6469.41439654348
73736 6480.92713289393
74746 6491.73981397903
75756 6502.16759523045
76766 6506.49696100837
77777 6512.94880292379
78787 6517.99694589299
79797 6524.22352847455
80807 6534.48626003936
81817 6544.60491898039
82827 6552.14508859629
83837 6562.61287282409
84847 6572.90547327461
85857 6577.92935650365
86867 6588.67024375697
87877 6595.82305661719
88888 6610.64624716848
89898 6614.11937589659
90908 6621.01867131441
91918 6633.01432777887
92928 6642.28002515921
93938 6647.52247938998
94948 6651.70546285321
95958 6663.7951932914
96968 6672.21854065898
97978 6678.20409563081
98988 6685.86002496482
99999 6690.64906699241
};
\addlegendentry{\ucbvibfi}
\addplot [very thick, forestgreen4416044]
table {%
0 1.74925798486635
1010 1668.09503996238
2020 3338.09246254084
3030 5016.5955295481
4040 6676.08052710305
5050 8247.81390939691
6060 9799.34182934555
7070 11347.9695956005
8080 12898.3333297409
9090 14487.2893327988
10100 16114.4372645955
11111 17738.9136716697
12121 19364.6022026457
13131 20988.0392498021
14141 22581.836107073
15151 24166.9523763367
16161 25729.4106260496
17171 27299.8020984458
18181 28829.286000112
19191 30305.3571023373
20201 31768.9688876353
21211 33215.2031220531
22222 34651.621248093
23232 36130.0342584589
24242 37631.315599972
25252 39124.9870250161
26262 40631.2195602776
27272 42132.6441823739
28282 43623.8460560973
29292 45121.6543247847
30302 46595.6172725412
31312 48046.7333212945
32322 49491.2935392089
33333 50887.4054119318
34343 52239.0950263656
35353 53590.0081161927
36363 54925.4187552744
37373 56283.1131084591
38383 57636.8542396891
39393 58978.0448377579
40403 60322.9534608502
41413 61677.9396368349
42423 63044.2142214596
43433 64433.3362035012
44444 65824.3559751584
45454 67205.8609559492
46464 68573.3109252285
47474 69940.2451092547
48484 71290.2156850591
49494 72633.9274873374
50504 73969.7967619904
51514 75279.1865857129
52524 76559.0636655989
53534 77821.2903800285
54544 79074.3236381703
55555 80326.2320737014
56565 81566.6198062515
57575 82812.1559712586
58585 84051.1405181279
59595 85273.1268995112
60605 86498.3473244288
61615 87717.3846333942
62625 88931.3728572459
63635 90129.6189066087
64645 91337.8579806353
65655 92539.817987003
66666 93743.9669000033
67676 94951.3544637055
68686 96163.6472776698
69696 97399.1079992508
70706 98626.3095341112
71716 99828.8481865126
72726 101038.710621657
73736 102255.906940285
74746 103459.704272898
75756 104649.884881915
76766 105833.473848294
77777 107005.279990676
78787 108145.846611621
79797 109280.112100907
80807 110409.48225773
81817 111538.439725967
82827 112667.125998819
83837 113747.529929129
84847 114812.075084704
85857 115875.017979794
86867 116929.204604953
87877 117975.735270194
88888 119019.642980018
89898 120062.866299698
90908 121084.757930526
91918 122092.406639231
92928 123100.200773848
93938 124101.110638599
94948 125100.936382584
95958 126107.261770028
96968 127113.041785009
97978 128103.846651208
98988 129086.109693304
99999 130062.174166572
};
\addlegendentry{\ucbvich}
\addplot [very thick, mediumpurple148103189]
table {%
0 1.5915711315235
1010 1679.02116456915
2020 3367.79500021443
3030 5055.17653019788
4040 6760.86253308859
5050 8464.72416352243
6060 10128.3743183139
7070 11801.9250656321
8080 13500.6631691862
9090 15179.0596058726
10100 16857.240681358
11111 18538.7971298264
12121 20209.6293594118
13131 21882.0427407515
14141 23566.4326014459
15151 25244.6533145084
16161 26921.0642697355
17171 28607.7021887379
18181 30286.9211979874
19191 31963.937460394
20201 33633.8973113933
21211 35308.6354382188
22222 37004.9297629974
23232 38708.7426070757
24242 40389.0249903935
25252 42051.4133194111
26262 43662.6964791575
27272 45287.5062204282
28282 46892.8470201287
29292 48501.0553823608
30302 50110.6648099041
31312 51684.1149396331
32322 53246.153061019
33333 54822.6364864817
34343 56394.7342983222
35353 57951.4114477419
36363 59515.0864356789
37373 61076.9570728655
38383 62662.2122064195
39393 64251.3286202834
40403 65839.5157707024
41413 67416.8573028469
42423 68978.4414136639
43433 70537.4603415559
44444 72107.0772278254
45454 73677.2028392515
46464 75248.1483386712
47474 76792.4865010089
48484 78338.0139053713
49494 79881.6065372251
50504 81422.2516643984
51514 82950.3867452196
52524 84489.511676079
53534 86037.5063327086
54544 87592.1849336926
55555 89152.0643450208
56565 90709.0380633685
57575 92264.6924001028
58585 93808.2403788937
59595 95353.6636396064
60605 96912.6065865901
61615 98495.649693888
62625 100073.958208422
63635 101661.69097592
64645 103259.587246725
65655 104865.617283183
66666 106477.936054699
67676 108062.342407704
68686 109662.509472736
69696 111253.697283624
70706 112834.594344725
71716 114387.919489332
72726 115931.639738681
73736 117469.06771791
74746 119015.92353788
75756 120594.452964127
76766 122150.945541177
77777 123704.485358634
78787 125258.014889169
79797 126839.983383987
80807 128438.433797452
81817 130048.995376796
82827 131620.364024565
83837 133176.003919614
84847 134735.487739339
85857 136307.607732733
86867 137832.021033934
87877 139369.946778195
88888 140903.695867538
89898 142415.350921428
90908 143923.255681739
91918 145436.790409955
92928 146946.67262434
93938 148435.483238383
94948 149928.777875533
95958 151421.36489834
96968 152938.559422895
97978 154483.95029786
98988 156047.673197353
99999 157603.501275617
};
\addlegendentry{\mvp}
\end{axis}

\end{tikzpicture}
}
% \caption{$H=5, \ K=10^5$.}
% \end{subfigure}
% %\hfill
% \begin{subfigure}[b]{0.37\textwidth}
% \resizebox{.4\linewidth}{!}{% This file was created with tikzplotlib v0.10.1.
\begin{tikzpicture}

\definecolor{crimson2143940}{RGB}{214,39,40}
\definecolor{darkgray176}{RGB}{176,176,176}
\definecolor{darkorange25512714}{RGB}{255,127,14}
\definecolor{forestgreen4416044}{RGB}{44,160,44}
\definecolor{lightgray204}{RGB}{204,204,204}
\definecolor{mediumpurple148103189}{RGB}{148,103,189}
\definecolor{steelblue31119180}{RGB}{31,119,180}

\begin{axis}[
legend cell align={left},
legend style={
  fill opacity=0.8,
  draw opacity=1,
  text opacity=1,
  at={(0.03,0.97)},
  anchor=north west,
  draw=lightgray204,
  font=\small
},
tick align=outside,
tick pos=left,
x grid style={darkgray176},
xmin=0, xmax=1000000,
xtick style={color=black},
y grid style={darkgray176},
ymin=-30000, ymax=1200000,
ytick style={color=black},
grid=both,
xlabel={Episodes},
ylabel={Cumulative Regret},
grid style={dashed,gray!50!white}
]
\path [fill=steelblue31119180, fill opacity=0.3]
(axis cs:0,2.12672796801025)
--(axis cs:0,0.989704837421551)
--(axis cs:10101,6153.14273675563)
--(axis cs:20202,7491.06607230517)
--(axis cs:30303,8045.83228009627)
--(axis cs:40404,8481.54528070426)
--(axis cs:50505,8779.19416797585)
--(axis cs:60606,9024.1144839135)
--(axis cs:70707,9231.79847204198)
--(axis cs:80808,9423.96097965683)
--(axis cs:90909,9587.64517629475)
--(axis cs:101010,9746.38318646813)
--(axis cs:111111,9914.14159256867)
--(axis cs:121212,10059.101169828)
--(axis cs:131313,10185.8549902267)
--(axis cs:141414,10312.870433738)
--(axis cs:151515,10419.2878983704)
--(axis cs:161616,10522.2775441566)
--(axis cs:171717,10620.785785373)
--(axis cs:181818,10704.6929600336)
--(axis cs:191919,10795.1957802875)
--(axis cs:202020,10875.0229927078)
--(axis cs:212121,10962.0166315193)
--(axis cs:222222,11066.2861653902)
--(axis cs:232323,11140.9560671954)
--(axis cs:242424,11209.4747474366)
--(axis cs:252525,11295.5285681141)
--(axis cs:262626,11339.3850081519)
--(axis cs:272727,11400.6626981339)
--(axis cs:282828,11467.4977882198)
--(axis cs:292929,11524.1248481256)
--(axis cs:303030,11577.7722083208)
--(axis cs:313131,11627.4506295)
--(axis cs:323232,11666.1026801056)
--(axis cs:333333,11713.9388640227)
--(axis cs:343434,11740.6801335881)
--(axis cs:353535,11782.2199735615)
--(axis cs:363636,11827.2383783666)
--(axis cs:373737,11852.1287176518)
--(axis cs:383838,11888.930503668)
--(axis cs:393939,11931.8802079454)
--(axis cs:404040,11961.9385945586)
--(axis cs:414141,12005.4643587585)
--(axis cs:424242,12032.3821390565)
--(axis cs:434343,12078.187939601)
--(axis cs:444444,12116.6300870029)
--(axis cs:454545,12132.3868193704)
--(axis cs:464646,12151.6151699443)
--(axis cs:474747,12206.0763359726)
--(axis cs:484848,12242.9560814999)
--(axis cs:494949,12288.4976838428)
--(axis cs:505050,12328.047444038)
--(axis cs:515151,12357.2666390755)
--(axis cs:525252,12391.3908908963)
--(axis cs:535353,12418.9814509369)
--(axis cs:545454,12447.1644868885)
--(axis cs:555555,12476.889827929)
--(axis cs:565656,12521.7064954265)
--(axis cs:575757,12555.9287658033)
--(axis cs:585858,12568.6268985566)
--(axis cs:595959,12570.2913966918)
--(axis cs:606060,12599.5307759642)
--(axis cs:616161,12615.2434555925)
--(axis cs:626262,12645.1008731122)
--(axis cs:636363,12676.8327551288)
--(axis cs:646464,12698.7427545915)
--(axis cs:656565,12720.1992963423)
--(axis cs:666666,12745.5126842102)
--(axis cs:676767,12780.0200475768)
--(axis cs:686868,12794.4039803158)
--(axis cs:696969,12825.6272691469)
--(axis cs:707070,12859.5660435229)
--(axis cs:717171,12873.9521701311)
--(axis cs:727272,12902.5725668766)
--(axis cs:737373,12912.2126795422)
--(axis cs:747474,12931.6734059808)
--(axis cs:757575,12948.7914491713)
--(axis cs:767676,12976.1211155473)
--(axis cs:777777,13009.476688845)
--(axis cs:787878,13030.0007353273)
--(axis cs:797979,13033.8424429241)
--(axis cs:808080,13067.0552662565)
--(axis cs:818181,13066.8715537471)
--(axis cs:828282,13076.5225343575)
--(axis cs:838383,13073.7392751441)
--(axis cs:848484,13105.1272632211)
--(axis cs:858585,13124.7374950863)
--(axis cs:868686,13153.616862338)
--(axis cs:878787,13179.4584609065)
--(axis cs:888888,13207.9727395666)
--(axis cs:898989,13223.5783185111)
--(axis cs:909090,13227.204971605)
--(axis cs:919191,13225.7734303489)
--(axis cs:929292,13242.8626207997)
--(axis cs:939393,13264.3683212309)
--(axis cs:949494,13278.6623540753)
--(axis cs:959595,13297.7754650089)
--(axis cs:969696,13305.4463793526)
--(axis cs:979797,13304.5146027446)
--(axis cs:989898,13333.724113851)
--(axis cs:999999,13348.5970435578)
--(axis cs:999999,19687.243349654)
--(axis cs:999999,19687.243349654)
--(axis cs:989898,19667.2095909637)
--(axis cs:979797,19638.7180887119)
--(axis cs:969696,19636.5857192477)
--(axis cs:959595,19619.1300993014)
--(axis cs:949494,19602.643677131)
--(axis cs:939393,19594.7376523178)
--(axis cs:929292,19589.3106060754)
--(axis cs:919191,19576.9104741291)
--(axis cs:909090,19576.7870296825)
--(axis cs:898989,19564.2015577246)
--(axis cs:888888,19550.8620385724)
--(axis cs:878787,19529.9355767092)
--(axis cs:868686,19506.76303945)
--(axis cs:858585,19488.5037070561)
--(axis cs:848484,19462.0827292558)
--(axis cs:838383,19450.2437044445)
--(axis cs:828282,19442.8864743138)
--(axis cs:818181,19436.6699952228)
--(axis cs:808080,19437.2822461448)
--(axis cs:797979,19409.2335731848)
--(axis cs:787878,19406.6060132123)
--(axis cs:777777,19386.8034773181)
--(axis cs:767676,19355.9643771772)
--(axis cs:757575,19334.5120896973)
--(axis cs:747474,19326.4984330232)
--(axis cs:737373,19300.9232642007)
--(axis cs:727272,19290.4950204411)
--(axis cs:717171,19266.9348238538)
--(axis cs:707070,19253.0203641886)
--(axis cs:696969,19233.1642622986)
--(axis cs:686868,19209.904259254)
--(axis cs:676767,19194.9800645016)
--(axis cs:666666,19174.8385356776)
--(axis cs:656565,19155.6875837218)
--(axis cs:646464,19138.1105476937)
--(axis cs:636363,19116.2102855755)
--(axis cs:626262,19091.2689890971)
--(axis cs:616161,19064.1080147131)
--(axis cs:606060,19049.1506269882)
--(axis cs:595959,19021.6990561812)
--(axis cs:585858,19017.5521631127)
--(axis cs:575757,19003.6823163898)
--(axis cs:565656,18968.576266838)
--(axis cs:555555,18939.6071853372)
--(axis cs:545454,18921.8847788877)
--(axis cs:535353,18896.9563869191)
--(axis cs:525252,18871.341024297)
--(axis cs:515151,18838.3634161937)
--(axis cs:505050,18814.1648894832)
--(axis cs:494949,18787.19268611)
--(axis cs:484848,18760.9394568788)
--(axis cs:474747,18731.707743523)
--(axis cs:464646,18695.9855499617)
--(axis cs:454545,18671.4135150736)
--(axis cs:444444,18646.074419957)
--(axis cs:434343,18614.4506142957)
--(axis cs:424242,18578.2675983627)
--(axis cs:414141,18546.5892431863)
--(axis cs:404040,18506.7701940971)
--(axis cs:393939,18483.7200259628)
--(axis cs:383838,18455.0743575557)
--(axis cs:373737,18432.3875803677)
--(axis cs:363636,18407.8489144962)
--(axis cs:353535,18370.93898293)
--(axis cs:343434,18329.2815945096)
--(axis cs:333333,18299.9991880564)
--(axis cs:323232,18262.7501761263)
--(axis cs:313131,18221.1942049727)
--(axis cs:303030,18181.5905255337)
--(axis cs:292929,18139.4929176513)
--(axis cs:282828,18096.9508440979)
--(axis cs:272727,18042.9407530473)
--(axis cs:262626,17988.2267607435)
--(axis cs:252525,17944.6516943627)
--(axis cs:242424,17877.3211802683)
--(axis cs:232323,17818.1551774401)
--(axis cs:222222,17750.8592102895)
--(axis cs:212121,17676.3851837427)
--(axis cs:202020,17615.8506149926)
--(axis cs:191919,17554.2754944005)
--(axis cs:181818,17471.3128451314)
--(axis cs:171717,17394.3118345971)
--(axis cs:161616,17325.9397707882)
--(axis cs:151515,17241.1454837816)
--(axis cs:141414,17142.66689203)
--(axis cs:131313,17032.3072608738)
--(axis cs:121212,16918.6799856213)
--(axis cs:111111,16791.2960594258)
--(axis cs:101010,16651.07385)
--(axis cs:90909,16495.7337138857)
--(axis cs:80808,16342.0989717687)
--(axis cs:70707,16160.0614882221)
--(axis cs:60606,15971.0628638672)
--(axis cs:50505,15736.3542673781)
--(axis cs:40404,15456.5834692167)
--(axis cs:30303,15083.6116045717)
--(axis cs:20202,14453.7392702564)
--(axis cs:10101,11519.3842488775)
--(axis cs:0,2.12672796801025)
--cycle;

\path [fill=darkorange25512714, fill opacity=0.3]
(axis cs:0,2.2742738915582)
--(axis cs:0,0.802432950451939)
--(axis cs:10101,3727.80622954081)
--(axis cs:20202,4243.59603491653)
--(axis cs:30303,4503.28463101153)
--(axis cs:40404,4690.51749840679)
--(axis cs:50505,4863.30328375734)
--(axis cs:60606,5030.98955879036)
--(axis cs:70707,5161.23742532945)
--(axis cs:80808,5268.40773397188)
--(axis cs:90909,5365.50314850257)
--(axis cs:101010,5439.96788632348)
--(axis cs:111111,5539.03731424288)
--(axis cs:121212,5629.42030603937)
--(axis cs:131313,5702.55719509107)
--(axis cs:141414,5783.9552059273)
--(axis cs:151515,5827.7108476289)
--(axis cs:161616,5879.61163546622)
--(axis cs:171717,5924.61611711453)
--(axis cs:181818,5975.20135811791)
--(axis cs:191919,6018.58461285583)
--(axis cs:202020,6064.147736577)
--(axis cs:212121,6106.4940814919)
--(axis cs:222222,6152.91473373636)
--(axis cs:232323,6177.43060237666)
--(axis cs:242424,6208.5595714958)
--(axis cs:252525,6237.31188557643)
--(axis cs:262626,6270.15254669155)
--(axis cs:272727,6286.48564178306)
--(axis cs:282828,6301.53016453467)
--(axis cs:292929,6336.52791531756)
--(axis cs:303030,6361.1950452498)
--(axis cs:313131,6382.39868891064)
--(axis cs:323232,6409.42802535677)
--(axis cs:333333,6436.30040348839)
--(axis cs:343434,6457.63019721843)
--(axis cs:353535,6492.38107561962)
--(axis cs:363636,6511.06493977686)
--(axis cs:373737,6534.71237533345)
--(axis cs:383838,6541.48434632922)
--(axis cs:393939,6569.08479947521)
--(axis cs:404040,6586.62658147491)
--(axis cs:414141,6587.40486323653)
--(axis cs:424242,6595.79107018932)
--(axis cs:434343,6617.77386399025)
--(axis cs:444444,6642.90816854842)
--(axis cs:454545,6644.14025302844)
--(axis cs:464646,6652.66008566583)
--(axis cs:474747,6685.65678986451)
--(axis cs:484848,6698.42659754511)
--(axis cs:494949,6712.66162990371)
--(axis cs:505050,6712.61697729212)
--(axis cs:515151,6719.14915218798)
--(axis cs:525252,6732.97655170129)
--(axis cs:535353,6753.34035431858)
--(axis cs:545454,6758.92631961298)
--(axis cs:555555,6763.21280853228)
--(axis cs:565656,6773.60399827701)
--(axis cs:575757,6768.95275225989)
--(axis cs:585858,6755.76928205479)
--(axis cs:595959,6781.6460928527)
--(axis cs:606060,6803.75871380109)
--(axis cs:616161,6815.64482598923)
--(axis cs:626262,6809.68115579942)
--(axis cs:636363,6813.70925910352)
--(axis cs:646464,6799.52884076785)
--(axis cs:656565,6791.66499540464)
--(axis cs:666666,6804.42789866322)
--(axis cs:676767,6812.87590408819)
--(axis cs:686868,6814.8126667357)
--(axis cs:696969,6813.75767769755)
--(axis cs:707070,6837.53564487155)
--(axis cs:717171,6842.798725205)
--(axis cs:727272,6841.83584190071)
--(axis cs:737373,6858.25795749405)
--(axis cs:747474,6850.17308531921)
--(axis cs:757575,6849.46358155178)
--(axis cs:767676,6858.89787935061)
--(axis cs:777777,6875.28649690483)
--(axis cs:787878,6893.32050489781)
--(axis cs:797979,6906.71224063982)
--(axis cs:808080,6909.69125165567)
--(axis cs:818181,6920.07596172388)
--(axis cs:828282,6934.03303891258)
--(axis cs:838383,6937.98729301555)
--(axis cs:848484,6952.894370587)
--(axis cs:858585,6948.51597015287)
--(axis cs:868686,6945.86201246919)
--(axis cs:878787,6950.46843026184)
--(axis cs:888888,6964.5056869908)
--(axis cs:898989,6962.62941353009)
--(axis cs:909090,6978.12182274438)
--(axis cs:919191,6989.53347573801)
--(axis cs:929292,6985.0505216117)
--(axis cs:939393,6983.17704310823)
--(axis cs:949494,7010.43692863407)
--(axis cs:959595,7006.40386909863)
--(axis cs:969696,6994.20664935885)
--(axis cs:979797,6999.78427605974)
--(axis cs:989898,7010.37542895415)
--(axis cs:999999,7007.85453931702)
--(axis cs:999999,10060.4286405449)
--(axis cs:999999,10060.4286405449)
--(axis cs:989898,10055.2778673173)
--(axis cs:979797,10043.9557130342)
--(axis cs:969696,10038.6413210536)
--(axis cs:959595,10038.2472187647)
--(axis cs:949494,10036.156818843)
--(axis cs:939393,10019.2902659458)
--(axis cs:929292,10015.6615598401)
--(axis cs:919191,10014.697550026)
--(axis cs:909090,10009.1123344303)
--(axis cs:898989,10012.7762636949)
--(axis cs:888888,10007.7907745498)
--(axis cs:878787,10002.8370631941)
--(axis cs:868686,9998.38182586122)
--(axis cs:858585,9995.53463660562)
--(axis cs:848484,9997.39053936499)
--(axis cs:838383,9993.86508261819)
--(axis cs:828282,9992.57357555624)
--(axis cs:818181,9982.40193250624)
--(axis cs:808080,9973.46906381996)
--(axis cs:797979,9972.1711017517)
--(axis cs:787878,9959.83147823676)
--(axis cs:777777,9948.45226704225)
--(axis cs:767676,9936.35319096232)
--(axis cs:757575,9936.28888582715)
--(axis cs:747474,9931.47528146238)
--(axis cs:737373,9929.25277413608)
--(axis cs:727272,9921.68606782669)
--(axis cs:717171,9920.94860463659)
--(axis cs:707070,9919.45156623025)
--(axis cs:696969,9904.96885168557)
--(axis cs:686868,9907.33139311252)
--(axis cs:676767,9902.66975533968)
--(axis cs:666666,9890.81573968733)
--(axis cs:656565,9885.45323388545)
--(axis cs:646464,9884.47624628862)
--(axis cs:636363,9883.39006421051)
--(axis cs:626262,9890.41161161918)
--(axis cs:616161,9891.5723156528)
--(axis cs:606060,9886.55200495723)
--(axis cs:595959,9873.5890546497)
--(axis cs:585858,9856.80187096598)
--(axis cs:575757,9859.78580161116)
--(axis cs:565656,9854.43921558894)
--(axis cs:555555,9837.89075544971)
--(axis cs:545454,9837.49160169676)
--(axis cs:535353,9831.287605164)
--(axis cs:525252,9815.81341322656)
--(axis cs:515151,9795.27611061902)
--(axis cs:505050,9789.27118187172)
--(axis cs:494949,9784.57470685565)
--(axis cs:484848,9782.97978325055)
--(axis cs:474747,9762.81606738519)
--(axis cs:464646,9742.78488321495)
--(axis cs:454545,9720.58278244457)
--(axis cs:444444,9718.20505215518)
--(axis cs:434343,9699.93147801402)
--(axis cs:424242,9683.10058616989)
--(axis cs:414141,9676.35481065032)
--(axis cs:404040,9670.77825551313)
--(axis cs:393939,9658.64756657076)
--(axis cs:383838,9643.92374301318)
--(axis cs:373737,9640.14663919633)
--(axis cs:363636,9621.81495437437)
--(axis cs:353535,9613.18733729233)
--(axis cs:343434,9586.51007597949)
--(axis cs:333333,9572.825554886)
--(axis cs:323232,9554.26037861743)
--(axis cs:313131,9533.13111840904)
--(axis cs:303030,9520.84605192968)
--(axis cs:292929,9502.7206441466)
--(axis cs:282828,9480.30130675468)
--(axis cs:272727,9461.98843284039)
--(axis cs:262626,9449.19717515348)
--(axis cs:252525,9434.92646512356)
--(axis cs:242424,9404.11995813747)
--(axis cs:232323,9374.21217831969)
--(axis cs:222222,9357.89024138981)
--(axis cs:212121,9323.55473272067)
--(axis cs:202020,9285.23505420568)
--(axis cs:191919,9242.81841706423)
--(axis cs:181818,9194.63507217671)
--(axis cs:171717,9154.29129656719)
--(axis cs:161616,9117.22706010994)
--(axis cs:151515,9077.04536458333)
--(axis cs:141414,9043.87122272424)
--(axis cs:131313,8983.16770820293)
--(axis cs:121212,8932.7885589073)
--(axis cs:111111,8857.48032423897)
--(axis cs:101010,8783.10104755631)
--(axis cs:90909,8722.01020889169)
--(axis cs:80808,8640.95899481639)
--(axis cs:70707,8548.25728141508)
--(axis cs:60606,8432.06252269337)
--(axis cs:50505,8300.8624148771)
--(axis cs:40404,8143.63341096414)
--(axis cs:30303,7965.61844643217)
--(axis cs:20202,7724.80994931705)
--(axis cs:10101,7210.41517757453)
--(axis cs:0,2.2742738915582)
--cycle;

\path [fill=forestgreen4416044, fill opacity=0.3]
(axis cs:0,2.05757879820759)
--(axis cs:0,0.91695444482354)
--(axis cs:10101,11032.0609075429)
--(axis cs:20202,22420.6400423401)
--(axis cs:30303,32855.9823474435)
--(axis cs:40404,43870.9411401152)
--(axis cs:50505,53902.2793095521)
--(axis cs:60606,64038.4169443784)
--(axis cs:70707,73611.2445906766)
--(axis cs:80808,82601.8882870688)
--(axis cs:90909,91790.1403881377)
--(axis cs:101010,100913.027547822)
--(axis cs:111111,109351.819153204)
--(axis cs:121212,117184.353888605)
--(axis cs:131313,124583.746701674)
--(axis cs:141414,131375.385929762)
--(axis cs:151515,137288.30411652)
--(axis cs:161616,142870.889699718)
--(axis cs:171717,148342.019357652)
--(axis cs:181818,153488.256374388)
--(axis cs:191919,158343.370287504)
--(axis cs:202020,162518.942095344)
--(axis cs:212121,166446.799355697)
--(axis cs:222222,170165.80968423)
--(axis cs:232323,173687.30645463)
--(axis cs:242424,177179.714386461)
--(axis cs:252525,180610.911463415)
--(axis cs:262626,184001.207907251)
--(axis cs:272727,187394.719342208)
--(axis cs:282828,190925.297356283)
--(axis cs:292929,194405.335719222)
--(axis cs:303030,197373.659430699)
--(axis cs:313131,199881.356653127)
--(axis cs:323232,202279.352440298)
--(axis cs:333333,204713.893262953)
--(axis cs:343434,207102.778861722)
--(axis cs:353535,209456.666828222)
--(axis cs:363636,211796.200841075)
--(axis cs:373737,214040.731293321)
--(axis cs:383838,216250.441141366)
--(axis cs:393939,218413.378897282)
--(axis cs:404040,220571.782274717)
--(axis cs:414141,222708.832960718)
--(axis cs:424242,224822.817839791)
--(axis cs:434343,226888.204522691)
--(axis cs:444444,228961.262473747)
--(axis cs:454545,231059.746062841)
--(axis cs:464646,233084.306330374)
--(axis cs:474747,235069.175000142)
--(axis cs:484848,237046.646944751)
--(axis cs:494949,239021.051546173)
--(axis cs:505050,240981.218828396)
--(axis cs:515151,242940.450488331)
--(axis cs:525252,244857.323424168)
--(axis cs:535353,246755.545691824)
--(axis cs:545454,248623.980227426)
--(axis cs:555555,250476.018330743)
--(axis cs:565656,252317.419678222)
--(axis cs:575757,254102.431442113)
--(axis cs:585858,255854.209744022)
--(axis cs:595959,257556.972939476)
--(axis cs:606060,259234.194602767)
--(axis cs:616161,260873.134794241)
--(axis cs:626262,262498.354902568)
--(axis cs:636363,264128.713671259)
--(axis cs:646464,265714.383240242)
--(axis cs:656565,267294.263765599)
--(axis cs:666666,268822.689734975)
--(axis cs:676767,270319.506954747)
--(axis cs:686868,271814.733482808)
--(axis cs:696969,273335.390767514)
--(axis cs:707070,274816.83007307)
--(axis cs:717171,276275.350182084)
--(axis cs:727272,277725.606373705)
--(axis cs:737373,279160.17311443)
--(axis cs:747474,280552.059245565)
--(axis cs:757575,281939.346123957)
--(axis cs:767676,283298.514666315)
--(axis cs:777777,284638.467855006)
--(axis cs:787878,285967.474824962)
--(axis cs:797979,287337.337553648)
--(axis cs:808080,288665.687938518)
--(axis cs:818181,290012.394594655)
--(axis cs:828282,291330.545169472)
--(axis cs:838383,292621.054717429)
--(axis cs:848484,293923.139229443)
--(axis cs:858585,295192.263585896)
--(axis cs:868686,296457.356649244)
--(axis cs:878787,297712.697888968)
--(axis cs:888888,298932.792221095)
--(axis cs:898989,300162.913811113)
--(axis cs:909090,301381.656210072)
--(axis cs:919191,302574.089261276)
--(axis cs:929292,303754.649287737)
--(axis cs:939393,304913.13193096)
--(axis cs:949494,306071.125408902)
--(axis cs:959595,307214.676159558)
--(axis cs:969696,308357.809466816)
--(axis cs:979797,309485.561643083)
--(axis cs:989898,310615.192858208)
--(axis cs:999999,311724.838426011)
--(axis cs:999999,682413.633077013)
--(axis cs:999999,682413.633077013)
--(axis cs:989898,679412.704590887)
--(axis cs:979797,676395.70402993)
--(axis cs:969696,673387.154466916)
--(axis cs:959595,670367.304140476)
--(axis cs:949494,667364.653727062)
--(axis cs:939393,664352.58625154)
--(axis cs:929292,661349.395958561)
--(axis cs:919191,658336.394653825)
--(axis cs:909090,655303.467823061)
--(axis cs:898989,652258.61568614)
--(axis cs:888888,649223.22566595)
--(axis cs:878787,646210.791450668)
--(axis cs:868686,643185.734294042)
--(axis cs:858585,640174.941766758)
--(axis cs:848484,637172.583041471)
--(axis cs:838383,634138.538639245)
--(axis cs:828282,631110.190153702)
--(axis cs:818181,628077.263093239)
--(axis cs:808080,625043.165323782)
--(axis cs:797979,622013.504028537)
--(axis cs:787878,618770.843675089)
--(axis cs:777777,614659.138672545)
--(axis cs:767676,610546.925880001)
--(axis cs:757575,606420.660865912)
--(axis cs:747474,602294.202740361)
--(axis cs:737373,598174.616884435)
--(axis cs:727272,594044.512346414)
--(axis cs:717171,589919.709317318)
--(axis cs:707070,585798.163922261)
--(axis cs:696969,581673.890507008)
--(axis cs:686868,577537.701870799)
--(axis cs:676767,573427.606680788)
--(axis cs:666666,569334.548838418)
--(axis cs:656565,565232.742829919)
--(axis cs:646464,561123.719195467)
--(axis cs:636363,557004.776613426)
--(axis cs:626262,552886.618695971)
--(axis cs:616161,548801.105368043)
--(axis cs:606060,544718.361197558)
--(axis cs:595959,540635.633070674)
--(axis cs:585858,536550.422222534)
--(axis cs:575757,532087.32118716)
--(axis cs:565656,526742.928023999)
--(axis cs:555555,521371.166720246)
--(axis cs:545454,516001.662367301)
--(axis cs:535353,510633.572032074)
--(axis cs:525252,505250.875509416)
--(axis cs:515151,499870.021074761)
--(axis cs:505050,494520.328780083)
--(axis cs:494949,489193.071686235)
--(axis cs:484848,483743.002315454)
--(axis cs:474747,478256.999180384)
--(axis cs:464646,472777.233010034)
--(axis cs:454545,467285.229002038)
--(axis cs:444444,461756.803784309)
--(axis cs:434343,456229.233519264)
--(axis cs:424242,450697.118959633)
--(axis cs:414141,445054.4214177)
--(axis cs:404040,439402.227941121)
--(axis cs:393939,433747.444140901)
--(axis cs:383838,428095.220230036)
--(axis cs:373737,422420.661069087)
--(axis cs:363636,416736.701824729)
--(axis cs:353535,411015.852106428)
--(axis cs:343434,405085.872921507)
--(axis cs:333333,399056.978797507)
--(axis cs:323232,392680.322363714)
--(axis cs:313131,386037.00000342)
--(axis cs:303030,379185.702832987)
--(axis cs:292929,371370.385059556)
--(axis cs:282828,363078.350665762)
--(axis cs:272727,354773.175296709)
--(axis cs:262626,346536.378226273)
--(axis cs:252525,337978.881739029)
--(axis cs:242424,329203.188748454)
--(axis cs:232323,320193.35092272)
--(axis cs:222222,310728.565591514)
--(axis cs:212121,301159.774366587)
--(axis cs:202020,291352.927943695)
--(axis cs:191919,281482.538987911)
--(axis cs:181818,271353.922876695)
--(axis cs:171717,260932.664394817)
--(axis cs:161616,250257.984849603)
--(axis cs:151515,238840.265207272)
--(axis cs:141414,227384.022511995)
--(axis cs:131313,215542.65349788)
--(axis cs:121212,203169.875865002)
--(axis cs:111111,190332.983372527)
--(axis cs:101010,176658.751199226)
--(axis cs:90909,161887.7814092)
--(axis cs:80808,146848.434715452)
--(axis cs:70707,131213.669700918)
--(axis cs:60606,114666.694403998)
--(axis cs:50505,97395.6278033781)
--(axis cs:40404,80106.759511882)
--(axis cs:30303,61296.572551472)
--(axis cs:20202,41990.0235069679)
--(axis cs:10101,21273.1512329818)
--(axis cs:0,2.05757879820759)
--cycle;

\path [fill=mediumpurple148103189, fill opacity=0.3]
(axis cs:0,2.15923626783715)
--(axis cs:0,0.97558952455521)
--(axis cs:10101,11319.7256071372)
--(axis cs:20202,22887.4343583965)
--(axis cs:30303,34266.8016879795)
--(axis cs:40404,45093.300140387)
--(axis cs:50505,56135.1160041377)
--(axis cs:60606,67145.0839179331)
--(axis cs:70707,78456.1530271099)
--(axis cs:80808,89771.523791942)
--(axis cs:90909,100980.876884815)
--(axis cs:101010,111420.011825582)
--(axis cs:111111,122601.403864333)
--(axis cs:121212,133938.693909898)
--(axis cs:131313,145587.498034471)
--(axis cs:141414,156555.593311335)
--(axis cs:151515,167036.41227026)
--(axis cs:161616,177857.434587895)
--(axis cs:171717,188659.553064059)
--(axis cs:181818,199628.148583486)
--(axis cs:191919,210375.275418298)
--(axis cs:202020,220964.287451945)
--(axis cs:212121,231132.61920255)
--(axis cs:222222,240834.149855924)
--(axis cs:232323,250372.976939401)
--(axis cs:242424,260048.885171726)
--(axis cs:252525,269726.731307115)
--(axis cs:262626,279672.673905672)
--(axis cs:272727,289026.497451173)
--(axis cs:282828,298217.439960365)
--(axis cs:292929,308730.061147604)
--(axis cs:303030,318777.674457553)
--(axis cs:313131,327917.165327474)
--(axis cs:323232,336510.49567371)
--(axis cs:333333,344889.155937172)
--(axis cs:343434,352839.991532872)
--(axis cs:353535,360923.77815631)
--(axis cs:363636,368392.583883433)
--(axis cs:373737,375367.010870423)
--(axis cs:383838,382165.393772094)
--(axis cs:393939,388177.055638408)
--(axis cs:404040,394003.367532287)
--(axis cs:414141,399439.221448993)
--(axis cs:424242,405060.781395169)
--(axis cs:434343,412325.68969209)
--(axis cs:444444,419868.844297692)
--(axis cs:454545,427385.501054276)
--(axis cs:464646,434871.932650097)
--(axis cs:474747,441977.80501319)
--(axis cs:484848,448578.63209079)
--(axis cs:494949,454584.707658773)
--(axis cs:505050,460818.270467688)
--(axis cs:515151,466916.801306033)
--(axis cs:525252,473451.223364327)
--(axis cs:535353,479953.351719647)
--(axis cs:545454,486789.338115364)
--(axis cs:555555,493524.40699263)
--(axis cs:565656,499883.02584395)
--(axis cs:575757,506035.387443363)
--(axis cs:585858,512256.594699224)
--(axis cs:595959,517890.7825869)
--(axis cs:606060,523453.474781465)
--(axis cs:616161,528935.318778919)
--(axis cs:626262,534202.710695387)
--(axis cs:636363,539024.826129174)
--(axis cs:646464,543271.176094136)
--(axis cs:656565,547075.202574669)
--(axis cs:666666,550950.726351533)
--(axis cs:676767,555126.806221951)
--(axis cs:686868,559512.628147283)
--(axis cs:696969,563776.589798335)
--(axis cs:707070,568121.450183801)
--(axis cs:717171,572444.744606131)
--(axis cs:727272,576791.117724891)
--(axis cs:737373,581384.697449708)
--(axis cs:747474,586165.07510128)
--(axis cs:757575,590947.239242566)
--(axis cs:767676,595733.960322637)
--(axis cs:777777,600528.920587535)
--(axis cs:787878,605317.071436585)
--(axis cs:797979,610355.28392889)
--(axis cs:808080,614794.117073655)
--(axis cs:818181,619209.248286466)
--(axis cs:828282,623643.80474848)
--(axis cs:838383,627968.534999223)
--(axis cs:848484,631511.942739679)
--(axis cs:858585,635309.79699688)
--(axis cs:868686,640058.17488642)
--(axis cs:878787,644907.846617884)
--(axis cs:888888,649792.99480957)
--(axis cs:898989,654711.886440311)
--(axis cs:909090,659625.839712618)
--(axis cs:919191,664313.321220996)
--(axis cs:929292,668248.41008442)
--(axis cs:939393,671589.595195584)
--(axis cs:949494,674916.874150555)
--(axis cs:959595,678807.209566696)
--(axis cs:969696,682784.219500956)
--(axis cs:979797,686941.347433976)
--(axis cs:989898,690736.138211933)
--(axis cs:999999,694043.837922396)
--(axis cs:999999,1215198.40146081)
--(axis cs:999999,1215198.40146081)
--(axis cs:989898,1209242.61647758)
--(axis cs:979797,1202684.91889764)
--(axis cs:969696,1195993.36717319)
--(axis cs:959595,1189362.7611688)
--(axis cs:949494,1182738.37497705)
--(axis cs:939393,1176268.67110546)
--(axis cs:929292,1169742.28905781)
--(axis cs:919191,1163004.87914499)
--(axis cs:909090,1155979.4742674)
--(axis cs:898989,1148251.14946838)
--(axis cs:888888,1140477.05467706)
--(axis cs:878787,1132798.35784769)
--(axis cs:868686,1125307.92840649)
--(axis cs:858585,1117902.03004157)
--(axis cs:848484,1110938.63581192)
--(axis cs:838383,1103977.23770322)
--(axis cs:828282,1096298.14409083)
--(axis cs:818181,1088561.82680921)
--(axis cs:808080,1080861.63984681)
--(axis cs:797979,1073147.18326331)
--(axis cs:787878,1065055.90829642)
--(axis cs:777777,1057008.90132456)
--(axis cs:767676,1048520.87087312)
--(axis cs:757575,1040036.34786964)
--(axis cs:747474,1031523.96498768)
--(axis cs:737373,1023049.72050427)
--(axis cs:727272,1014712.41095991)
--(axis cs:717171,1006786.68635492)
--(axis cs:707070,999355.690328473)
--(axis cs:696969,991934.363862206)
--(axis cs:686868,984726.224889539)
--(axis cs:676767,976913.816900613)
--(axis cs:666666,968388.613902444)
--(axis cs:656565,959940.368627234)
--(axis cs:646464,951382.017132228)
--(axis cs:636363,941893.012570378)
--(axis cs:626262,932137.563249)
--(axis cs:616161,921923.330575938)
--(axis cs:606060,911463.503222151)
--(axis cs:595959,901011.677286233)
--(axis cs:585858,890590.246594774)
--(axis cs:575757,879849.093487201)
--(axis cs:565656,869151.695943441)
--(axis cs:555555,858224.79320728)
--(axis cs:545454,847074.237391639)
--(axis cs:535353,835704.988883213)
--(axis cs:525252,823713.679430183)
--(axis cs:515151,811599.146279451)
--(axis cs:505050,799621.583205413)
--(axis cs:494949,787717.481167892)
--(axis cs:484848,776023.926698866)
--(axis cs:474747,764240.384850125)
--(axis cs:464646,752820.885115374)
--(axis cs:454545,742267.696418762)
--(axis cs:444444,731736.348799852)
--(axis cs:434343,721213.794141067)
--(axis cs:424242,710529.449195991)
--(axis cs:414141,700521.005823953)
--(axis cs:404040,690095.673362848)
--(axis cs:393939,678558.992953591)
--(axis cs:383838,666778.982508372)
--(axis cs:373737,654399.626901276)
--(axis cs:363636,641760.4683802)
--(axis cs:353535,628718.29713819)
--(axis cs:343434,615227.084610468)
--(axis cs:333333,601674.042688708)
--(axis cs:323232,587662.743576668)
--(axis cs:313131,573321.593681555)
--(axis cs:303030,558251.547442308)
--(axis cs:292929,543027.229610891)
--(axis cs:282828,527551.382164834)
--(axis cs:272727,512974.398764039)
--(axis cs:262626,497193.866843424)
--(axis cs:252525,480721.174493512)
--(axis cs:242424,464300.665799358)
--(axis cs:232323,447900.535021129)
--(axis cs:222222,431025.882947308)
--(axis cs:212121,414517.975524277)
--(axis cs:202020,397268.184457046)
--(axis cs:191919,379430.684411141)
--(axis cs:181818,361444.737563185)
--(axis cs:171717,343498.668013985)
--(axis cs:161616,324664.069363696)
--(axis cs:151515,305561.513856052)
--(axis cs:141414,286955.232352696)
--(axis cs:131313,268038.445928015)
--(axis cs:121212,248399.687166946)
--(axis cs:111111,228758.545500937)
--(axis cs:101010,208826.786221779)
--(axis cs:90909,189140.790602648)
--(axis cs:80808,168302.830096298)
--(axis cs:70707,147358.845410848)
--(axis cs:60606,126658.712675913)
--(axis cs:50505,106699.266976651)
--(axis cs:40404,86583.042887876)
--(axis cs:30303,65915.2646339407)
--(axis cs:20202,44334.4615622343)
--(axis cs:10101,22344.4774203549)
--(axis cs:0,2.15923626783715)
--cycle;

\addplot [very thick, steelblue31119180]
table {%
0 1.5582164027159
10101 8836.26349281657
20202 10972.4026712808
30303 11564.721942334
40404 11969.0643749605
50505 12257.774217677
60606 12497.5886738904
70707 12695.929980132
80808 12883.0299757128
90909 13041.6894450902
101010 13198.7285182341
111111 13352.7188259973
121212 13488.8905777246
131313 13609.0811255502
141414 13727.768662884
151515 13830.216691076
161616 13924.1086574724
171717 14007.548809985
181818 14088.0029025825
191919 14174.735637344
202020 14245.4368038502
212121 14319.200907631
222222 14408.5726878399
232323 14479.5556223177
242424 14543.3979638525
252525 14620.0901312384
262626 14663.8058844477
272727 14721.8017255906
282828 14782.2243161588
292929 14831.8088828884
303030 14879.6813669272
313131 14924.3224172364
323232 14964.426428116
333333 15006.9690260395
343434 15034.9808640488
353535 15076.5794782457
363636 15117.5436464314
373737 15142.2581490097
383838 15172.0024306118
393939 15207.8001169541
404040 15234.3543943279
414141 15276.0268009724
424242 15305.3248687096
434343 15346.3192769483
444444 15381.35225348
454545 15401.900167222
464646 15423.800359953
474747 15468.8920397478
484848 15501.9477691894
494949 15537.8451849764
505050 15571.1061667606
515151 15597.8150276346
525252 15631.3659575967
535353 15657.968918928
545454 15684.5246328881
555555 15708.2485066331
565656 15745.1413811323
575757 15779.8055410966
585858 15793.0895308346
595959 15795.9952264365
606060 15824.3407014762
616161 15839.6757351528
626262 15868.1849311046
636363 15896.5215203522
646464 15918.4266511426
656565 15937.9434400321
666666 15960.1756099439
676767 15987.5000560392
686868 16002.1541197849
696969 16029.3957657227
707070 16056.2932038558
717171 16070.4434969925
727272 16096.5337936589
737373 16106.5679718714
747474 16129.085919502
757575 16141.6517694343
767676 16166.0427463622
777777 16198.1400830815
787878 16218.3033742698
797979 16221.5380080544
808080 16252.1687562007
818181 16251.7707744849
828282 16259.7045043357
838383 16261.9914897943
848484 16283.6049962385
858585 16306.6206010712
868686 16330.189950894
878787 16354.6970188078
888888 16379.4173890695
898989 16393.8899381179
909090 16401.9960006438
919191 16401.341952239
929292 16416.0866134376
939393 16429.5529867744
949494 16440.6530156031
959595 16458.4527821551
969696 16471.0160493001
979797 16471.6163457283
989898 16500.4668524073
999999 16517.9201966059
};
\addlegendentry{\ucbvibf}
\addplot [very thick, darkorange25512714]
table {%
0 1.53835342100507
10101 5469.11070355767
20202 5984.20299211679
30303 6234.45153872185
40404 6417.07545468546
50505 6582.08284931722
60606 6731.52604074187
70707 6854.74735337226
80808 6954.68336439414
90909 7043.75667869713
101010 7111.5344669399
111111 7198.25881924093
121212 7281.10443247334
131313 7342.862451647
141414 7413.91321432577
151515 7452.37810610612
161616 7498.41934778808
171717 7539.45370684086
181818 7584.91821514731
191919 7630.70151496003
202020 7674.69139539134
212121 7715.02440710628
222222 7755.40248756309
232323 7775.82139034818
242424 7806.33976481664
252525 7836.11917534999
262626 7859.67486092252
272727 7874.23703731173
282828 7890.91573564467
292929 7919.62427973208
303030 7941.02054858974
313131 7957.76490365985
323232 7981.8442019871
333333 8004.5629791872
343434 8022.07013659896
353535 8052.78420645598
363636 8066.43994707561
373737 8087.42950726489
383838 8092.7040446712
393939 8113.86618302299
404040 8128.70241849402
414141 8131.87983694343
424242 8139.44582817961
434343 8158.85267100213
444444 8180.5566103518
454545 8182.3615177365
464646 8197.72248444039
474747 8224.23642862485
484848 8240.70319039783
494949 8248.61816837968
505050 8250.94407958192
515151 8257.2126314035
525252 8274.39498246392
535353 8292.31397974129
545454 8298.20896065487
555555 8300.551781991
565656 8314.02160693297
575757 8314.36927693552
585858 8306.28557651039
595959 8327.6175737512
606060 8345.15535937916
616161 8353.60857082101
626262 8350.0463837093
636363 8348.54966165701
646464 8342.00254352824
656565 8338.55911464504
666666 8347.62181917528
676767 8357.77282971394
686868 8361.07202992411
696969 8359.36326469156
707070 8378.4936055509
717171 8381.8736649208
727272 8381.7609548637
737373 8393.75536581506
747474 8390.8241833908
757575 8392.87623368947
767676 8397.62553515646
777777 8411.86938197354
787878 8426.57599156729
797979 8439.44167119576
808080 8441.58015773782
818181 8451.23894711506
828282 8463.30330723441
838383 8465.92618781687
848484 8475.14245497599
858585 8472.02530337925
868686 8472.12191916521
878787 8476.65274672798
888888 8486.14823077028
898989 8487.7028386125
909090 8493.61707858736
919191 8502.11551288203
929292 8500.35604072589
939393 8501.23365452701
949494 8523.29687373855
959595 8522.32554393169
969696 8516.42398520621
979797 8521.86999454696
989898 8532.8266481357
999999 8534.14158993094
};
\addlegendentry{\ucbvibfi}
\addplot [very thick, forestgreen4416044]
table {%
0 1.48726662151556
10101 16152.6060702623
20202 32205.331774654
30303 47076.2774494577
40404 61988.8503259986
50505 75648.9535564651
60606 89352.5556741882
70707 102412.457145797
80808 114725.16150126
90909 126838.960898669
101010 138785.889373524
111111 149842.401262865
121212 160177.114876804
131313 170063.200099777
141414 179379.704220879
151515 188064.284661896
161616 196564.43727466
171717 204637.341876235
181818 212421.089625542
191919 219912.954637707
202020 226935.93501952
212121 233803.286861142
222222 240447.187637872
232323 246940.328688675
242424 253191.451567458
252525 259294.896601222
262626 265268.793066762
272727 271083.947319459
282828 277001.824011022
292929 282887.860389389
303030 288279.681131843
313131 292959.178328273
323232 297479.837402006
333333 301885.43603023
343434 306094.325891615
353535 310236.259467325
363636 314266.451332902
373737 318230.696181204
383838 322172.830685701
393939 326080.411519091
404040 329987.005107919
414141 333881.627189209
424242 337759.968399712
434343 341558.719020977
444444 345359.033129028
454545 349172.48753244
464646 352930.769670204
474747 356663.087090263
484848 360394.824630103
494949 364107.061616204
505050 367750.773804239
515151 371405.235781546
525252 375054.099466792
535353 378694.558861949
545454 382312.821297363
555555 385923.592525494
565656 389530.17385111
575757 393094.876314637
585858 396202.315983278
595959 399096.303005075
606060 401976.277900162
616161 404837.120081142
626262 407692.48679927
636363 410566.745142342
646464 413419.051217854
656565 416263.503297759
666666 419078.619286697
676767 421873.556817767
686868 424676.217676804
696969 427504.640637261
707070 430307.496997666
717171 433097.529749701
727272 435885.05936006
737373 438667.394999433
747474 441423.130992963
757575 444180.003494935
767676 446922.720273158
777777 449648.803263775
787878 452369.159250026
797979 454675.420791093
808080 456854.42663115
818181 459044.828843947
828282 461220.367661587
838383 463379.796678337
848484 465547.861135457
858585 467683.602676327
868686 469821.545471643
878787 471961.744669818
888888 474078.008943522
898989 476210.764748626
909090 478342.562016567
919191 480455.241957551
929292 482552.022623149
939393 484632.85909125
949494 486717.889567982
959595 488790.990150017
969696 490872.481966866
979797 492940.632836506
989898 495013.948724547
999999 497069.235751512
};
\addlegendentry{\ucbvich}
\addplot [very thick, mediumpurple148103189]
table {%
0 1.56741289619618
10101 16832.1015137461
20202 33610.9479603154
30303 50091.0331609601
40404 65838.1715141315
50505 81417.1914903942
60606 96901.8982969232
70707 112907.499218979
80808 129037.17694412
90909 145060.833743732
101010 160123.399023681
111111 175679.974682635
121212 191169.190538422
131313 206812.971981243
141414 221755.412832015
151515 236298.963063156
161616 251260.751975795
171717 266079.110539022
181818 280536.443073336
191919 294902.97991472
202020 309116.235954495
212121 322825.297363413
222222 335930.016401616
232323 349136.755980265
242424 362174.775485542
252525 375223.952900313
262626 388433.270374548
272727 401000.448107606
282828 412884.4110626
292929 425878.645379248
303030 438514.61094993
313131 450619.379504514
323232 462086.619625189
333333 473281.59931294
343434 484033.53807167
353535 494821.03764725
363636 505076.526131816
373737 514883.31888585
383838 524472.188140233
393939 533368.024295999
404040 542049.520447568
414141 549980.113636473
424242 557795.11529558
434343 566769.741916578
444444 575802.596548772
454545 584826.598736519
464646 593846.408882735
474747 603109.094931657
484848 612301.279394828
494949 621151.094413333
505050 630219.92683655
515151 639257.973792742
525252 648582.451397255
535353 657829.17030143
545454 666931.787753501
555555 675874.600099955
565656 684517.360893696
575757 692942.240465282
585858 701423.420646999
595959 709451.229936566
606060 717458.489001808
616161 725429.324677429
626262 733170.136972194
636363 740458.919349776
646464 747326.596613182
656565 753507.785600952
666666 759669.670126988
676767 766020.311561282
686868 772119.426518411
696969 777855.47683027
707070 783738.570256137
717171 789615.715480526
727272 795751.764342402
737373 802217.20897699
747474 808844.52004448
757575 815491.793556101
767676 822127.415597879
777777 828768.91095605
787878 835186.489866504
797979 841751.2335961
808080 847827.878460233
818181 853885.537547838
828282 859970.974419653
838383 865972.886351222
848484 871225.2892758
858585 876605.913519223
868686 882683.051646457
878787 888853.102232785
888888 895135.024743316
898989 901481.517954344
909090 907802.656990011
919191 913659.100182994
929292 918995.349571117
939393 923929.13315052
949494 928827.624563801
959595 934084.985367746
969696 939388.793337073
979797 944813.13316581
989898 949989.377344757
999999 954621.119691602
};
\addlegendentry{\mvp}
\end{axis}

\end{tikzpicture}
}
% \caption{$H=5, \ K=10^6$.}
% \end{subfigure}
% %\hspace{1cm}


% %\hspace{0.6cm}
% \begin{subfigure}[b]{0.37\textwidth}
% \resizebox{.4\linewidth}{!}{% This file was created with tikzplotlib v0.10.1.
\begin{tikzpicture}

\definecolor{crimson2143940}{RGB}{214,39,40}
\definecolor{darkgray176}{RGB}{176,176,176}
\definecolor{darkorange25512714}{RGB}{255,127,14}
\definecolor{forestgreen4416044}{RGB}{44,160,44}
\definecolor{lightgray204}{RGB}{204,204,204}
\definecolor{mediumpurple148103189}{RGB}{148,103,189}
\definecolor{steelblue31119180}{RGB}{31,119,180}

\begin{axis}[
legend cell align={left},
legend style={
  fill opacity=0.8,
  draw opacity=1,
  text opacity=1,
  at={(0.03,0.97)},
  anchor=north west,
  draw=lightgray204,
  font=\small
},
tick align=outside,
tick pos=left,
x grid style={darkgray176},
xmin=0, xmax=100000,
xtick style={color=black},
y grid style={darkgray176},
ymin=-20000, ymax=420000,
ytick style={color=black},
grid=both,
xlabel={Episodes},
ylabel={Cumulative Regret},
grid style={dashed,gray!50!white}
]
\path [fill=steelblue31119180, fill opacity=0.3]
(axis cs:0,4.1581568934368)
--(axis cs:0,2.25546459784027)
--(axis cs:1010,2303.31591241326)
--(axis cs:2020,4584.90880440689)
--(axis cs:3030,6783.35818715798)
--(axis cs:4040,8816.14033244131)
--(axis cs:5050,10680.4829805476)
--(axis cs:6060,12277.7780902542)
--(axis cs:7070,13591.5823850484)
--(axis cs:8080,14669.2248182628)
--(axis cs:9090,15572.2900463768)
--(axis cs:10100,16325.6733645506)
--(axis cs:11111,16999.2308814722)
--(axis cs:12121,17568.9382464814)
--(axis cs:13131,18058.4638594471)
--(axis cs:14141,18522.6613711327)
--(axis cs:15151,18940.082813945)
--(axis cs:16161,19343.0243096301)
--(axis cs:17171,19757.996006145)
--(axis cs:18181,20138.9837865732)
--(axis cs:19191,20474.4749183422)
--(axis cs:20201,20768.3676406713)
--(axis cs:21211,21047.1342045717)
--(axis cs:22222,21318.148199968)
--(axis cs:23232,21597.6683633245)
--(axis cs:24242,21832.5799443801)
--(axis cs:25252,22058.2961014575)
--(axis cs:26262,22270.9625324926)
--(axis cs:27272,22482.9827407621)
--(axis cs:28282,22679.4017594352)
--(axis cs:29292,22882.6575509021)
--(axis cs:30302,23110.7175194995)
--(axis cs:31312,23352.2387770739)
--(axis cs:32322,23513.4524639933)
--(axis cs:33333,23662.2593299834)
--(axis cs:34343,23816.8911506271)
--(axis cs:35353,23962.4088296219)
--(axis cs:36363,24104.6373382474)
--(axis cs:37373,24231.9797306712)
--(axis cs:38383,24371.9620223417)
--(axis cs:39393,24515.8949715996)
--(axis cs:40403,24648.4139389666)
--(axis cs:41413,24776.0630475659)
--(axis cs:42423,24897.5028487169)
--(axis cs:43433,25017.0097497513)
--(axis cs:44444,25123.6584305343)
--(axis cs:45454,25229.4983318496)
--(axis cs:46464,25345.491253099)
--(axis cs:47474,25447.8342762288)
--(axis cs:48484,25548.371737063)
--(axis cs:49494,25643.2404333228)
--(axis cs:50504,25723.5028416314)
--(axis cs:51514,25808.9354347327)
--(axis cs:52524,25896.333549495)
--(axis cs:53534,25974.414504957)
--(axis cs:54544,26129.4597915099)
--(axis cs:55555,26241.314270852)
--(axis cs:56565,26315.3520238426)
--(axis cs:57575,26395.6167003849)
--(axis cs:58585,26462.6059115778)
--(axis cs:59595,26533.8008430168)
--(axis cs:60605,26606.0199180481)
--(axis cs:61615,26674.7575569723)
--(axis cs:62625,26746.2834784308)
--(axis cs:63635,26832.0880607148)
--(axis cs:64645,26919.7373233441)
--(axis cs:65655,26993.1612998114)
--(axis cs:66666,27066.4390850798)
--(axis cs:67676,27147.3962985036)
--(axis cs:68686,27221.6966581725)
--(axis cs:69696,27295.4428389279)
--(axis cs:70706,27368.6100163415)
--(axis cs:71716,27448.9357637405)
--(axis cs:72726,27521.9389628889)
--(axis cs:73736,27593.5033024441)
--(axis cs:74746,27661.8225961788)
--(axis cs:75756,27728.9335990855)
--(axis cs:76766,27785.1301611619)
--(axis cs:77777,27855.6390788233)
--(axis cs:78787,27928.7201133273)
--(axis cs:79797,28004.0499701827)
--(axis cs:80807,28068.2945366725)
--(axis cs:81817,28134.7533965653)
--(axis cs:82827,28193.2583615114)
--(axis cs:83837,28255.4019676633)
--(axis cs:84847,28306.4508923279)
--(axis cs:85857,28381.3072763336)
--(axis cs:86867,28444.2733020363)
--(axis cs:87877,28503.9742490195)
--(axis cs:88888,28555.5566876834)
--(axis cs:89898,28609.4379473996)
--(axis cs:90908,28664.6839076824)
--(axis cs:91918,28720.0620528425)
--(axis cs:92928,28773.1986585672)
--(axis cs:93938,28825.3215396172)
--(axis cs:94948,28876.4747323059)
--(axis cs:95958,28936.3572254292)
--(axis cs:96968,28982.7583300168)
--(axis cs:97978,29034.8443737192)
--(axis cs:98988,29083.2472883595)
--(axis cs:99999,29126.2475259261)
--(axis cs:99999,52694.0053450782)
--(axis cs:99999,52694.0053450782)
--(axis cs:98988,52653.0378518839)
--(axis cs:97978,52605.8512440521)
--(axis cs:96968,52558.4070203055)
--(axis cs:95958,52515.1403250516)
--(axis cs:94948,52462.1173558137)
--(axis cs:93938,52412.5322401545)
--(axis cs:92928,52364.5398976523)
--(axis cs:91918,52312.9385924138)
--(axis cs:90908,52260.1770788189)
--(axis cs:89898,52210.8074969316)
--(axis cs:88888,52160.5406560103)
--(axis cs:87877,52111.1335923683)
--(axis cs:86867,52056.1720300173)
--(axis cs:85857,51996.9662135584)
--(axis cs:84847,51935.0841120471)
--(axis cs:83837,51890.7221083416)
--(axis cs:82827,51837.2400319669)
--(axis cs:81817,51786.6399402384)
--(axis cs:80807,51721.6176859054)
--(axis cs:79797,51657.5262297095)
--(axis cs:78787,51586.9911201986)
--(axis cs:77777,51513.411383043)
--(axis cs:76766,51449.4469807117)
--(axis cs:75756,51389.9024303117)
--(axis cs:74746,51317.2751343362)
--(axis cs:73736,51250.2894065708)
--(axis cs:72726,51180.118380993)
--(axis cs:71716,51108.3244346439)
--(axis cs:70706,51036.4678817759)
--(axis cs:69696,50962.7412316123)
--(axis cs:68686,50889.4041199681)
--(axis cs:67676,50815.6265924013)
--(axis cs:66666,50740.8329069336)
--(axis cs:65655,50668.129573346)
--(axis cs:64645,50594.370394508)
--(axis cs:63635,50514.8438869714)
--(axis cs:62625,50437.912424172)
--(axis cs:61615,50328.4443387684)
--(axis cs:60605,50136.3655021815)
--(axis cs:59595,49942.3708043286)
--(axis cs:58585,49752.4294389544)
--(axis cs:57575,49562.9511300267)
--(axis cs:56565,49367.814325012)
--(axis cs:55555,49176.4316657843)
--(axis cs:54544,48976.9309589822)
--(axis cs:53534,48758.4861951236)
--(axis cs:52524,48569.3556313159)
--(axis cs:51514,48374.7083554617)
--(axis cs:50504,48181.3351211476)
--(axis cs:49494,47981.909914678)
--(axis cs:48484,47705.4221408924)
--(axis cs:47474,47398.8959097036)
--(axis cs:46464,47093.7682471751)
--(axis cs:45454,46788.0619360257)
--(axis cs:44444,46488.0276926675)
--(axis cs:43433,46184.0785448188)
--(axis cs:42423,45873.7218262651)
--(axis cs:41413,45560.9809660902)
--(axis cs:40403,45249.6033171392)
--(axis cs:39393,44932.3224836844)
--(axis cs:38383,44617.6990642387)
--(axis cs:37373,44223.3198082781)
--(axis cs:36363,43812.0326198525)
--(axis cs:35353,43395.3080805996)
--(axis cs:34343,42961.3330386104)
--(axis cs:33333,42526.8305604124)
--(axis cs:32322,42086.3002963706)
--(axis cs:31312,41642.2076579558)
--(axis cs:30302,41159.6341269059)
--(axis cs:29292,40682.1330163584)
--(axis cs:28282,40213.9024023981)
--(axis cs:27272,39741.7089896072)
--(axis cs:26262,39255.4097500357)
--(axis cs:25252,38774.1713315573)
--(axis cs:24242,38280.2303085684)
--(axis cs:23232,37773.6913975444)
--(axis cs:22222,37247.4156145848)
--(axis cs:21211,36717.871938716)
--(axis cs:20201,36137.7964359509)
--(axis cs:19191,35364.2603155544)
--(axis cs:18181,34574.8643373737)
--(axis cs:17171,33755.8734470138)
--(axis cs:16161,32897.9844424807)
--(axis cs:15151,32041.9248033863)
--(axis cs:14141,31179.323252695)
--(axis cs:13131,30293.2143126118)
--(axis cs:12121,29328.0881585237)
--(axis cs:11111,28204.2851796272)
--(axis cs:10100,26956.779132721)
--(axis cs:9090,25526.0921465328)
--(axis cs:8080,23882.0320551375)
--(axis cs:7070,22095.9114857642)
--(axis cs:6060,19952.8194031705)
--(axis cs:5050,17544.694379771)
--(axis cs:4040,14755.7889292635)
--(axis cs:3030,11614.1525184886)
--(axis cs:2020,8069.3581074898)
--(axis cs:1010,4179.68155081839)
--(axis cs:0,4.1581568934368)
--cycle;

\path [fill=darkorange25512714, fill opacity=0.3]
(axis cs:0,4.37643033763421)
--(axis cs:0,1.99703793280833)
--(axis cs:1010,2290.02641899044)
--(axis cs:2020,4421.98415952647)
--(axis cs:3030,6148.98096188556)
--(axis cs:4040,7353.42548216193)
--(axis cs:5050,8181.76369670981)
--(axis cs:6060,8819.95423662612)
--(axis cs:7070,9298.57714297233)
--(axis cs:8080,9710.22615842164)
--(axis cs:9090,10127.5066981)
--(axis cs:10100,10445.3514790045)
--(axis cs:11111,10727.5166998888)
--(axis cs:12121,10981.8160937081)
--(axis cs:13131,11198.2937956549)
--(axis cs:14141,11412.1157118638)
--(axis cs:15151,11640.3108080941)
--(axis cs:16161,11818.9331666068)
--(axis cs:17171,11972.0439418942)
--(axis cs:18181,12111.6876655439)
--(axis cs:19191,12248.1278096562)
--(axis cs:20201,12382.5814308259)
--(axis cs:21211,12512.6620776351)
--(axis cs:22222,12633.9474403691)
--(axis cs:23232,12751.6276606409)
--(axis cs:24242,12856.358004854)
--(axis cs:25252,12939.4923207946)
--(axis cs:26262,13021.9845403685)
--(axis cs:27272,13136.9544075125)
--(axis cs:28282,13224.0658699786)
--(axis cs:29292,13295.4690522829)
--(axis cs:30302,13362.8882990437)
--(axis cs:31312,13425.332692733)
--(axis cs:32322,13505.714363151)
--(axis cs:33333,13583.1753860603)
--(axis cs:34343,13654.7949921703)
--(axis cs:35353,13732.4019139655)
--(axis cs:36363,13799.6815216461)
--(axis cs:37373,13868.3434575112)
--(axis cs:38383,13928.313692627)
--(axis cs:39393,13996.3821200451)
--(axis cs:40403,14072.0656023599)
--(axis cs:41413,14135.5263715705)
--(axis cs:42423,14186.8912334953)
--(axis cs:43433,14245.6727926285)
--(axis cs:44444,14299.4194088448)
--(axis cs:45454,14354.6213643774)
--(axis cs:46464,14419.50499807)
--(axis cs:47474,14460.404605612)
--(axis cs:48484,14522.2246802214)
--(axis cs:49494,14566.7571375428)
--(axis cs:50504,14609.2760734996)
--(axis cs:51514,14652.5652466275)
--(axis cs:52524,14693.8042093408)
--(axis cs:53534,14734.1824722723)
--(axis cs:54544,14776.6293812761)
--(axis cs:55555,14820.7208327022)
--(axis cs:56565,14855.6493035399)
--(axis cs:57575,14888.7697351827)
--(axis cs:58585,14922.2015531986)
--(axis cs:59595,14963.434805975)
--(axis cs:60605,15005.0739140404)
--(axis cs:61615,15039.7457858555)
--(axis cs:62625,15082.6009296423)
--(axis cs:63635,15116.74124038)
--(axis cs:64645,15151.052384279)
--(axis cs:65655,15179.8958110966)
--(axis cs:66666,15214.7962653142)
--(axis cs:67676,15244.249452099)
--(axis cs:68686,15280.0862686118)
--(axis cs:69696,15312.328604527)
--(axis cs:70706,15337.7493999603)
--(axis cs:71716,15364.7770615584)
--(axis cs:72726,15396.3665093714)
--(axis cs:73736,15423.1383978458)
--(axis cs:74746,15451.2568555613)
--(axis cs:75756,15483.9278070127)
--(axis cs:76766,15516.9876334461)
--(axis cs:77777,15550.7607543077)
--(axis cs:78787,15582.7840005026)
--(axis cs:79797,15616.3064873948)
--(axis cs:80807,15640.9118890274)
--(axis cs:81817,15679.7968720582)
--(axis cs:82827,15710.1729548376)
--(axis cs:83837,15741.6875961633)
--(axis cs:84847,15753.4984934858)
--(axis cs:85857,15781.6524256505)
--(axis cs:86867,15812.5336382568)
--(axis cs:87877,15840.9055389904)
--(axis cs:88888,15870.8404832831)
--(axis cs:89898,15897.5314297843)
--(axis cs:90908,15915.8303965274)
--(axis cs:91918,15952.5086654775)
--(axis cs:92928,15982.4769269361)
--(axis cs:93938,16012.7247192513)
--(axis cs:94948,16026.8253895483)
--(axis cs:95958,16060.349480842)
--(axis cs:96968,16090.5650367755)
--(axis cs:97978,16107.1578302306)
--(axis cs:98988,16122.6108066538)
--(axis cs:99999,16144.4181874309)
--(axis cs:99999,27981.8046913288)
--(axis cs:99999,27981.8046913288)
--(axis cs:98988,27962.9233119241)
--(axis cs:97978,27948.1998100492)
--(axis cs:96968,27928.7856774145)
--(axis cs:95958,27904.0234955138)
--(axis cs:94948,27872.8300937399)
--(axis cs:93938,27857.1530628741)
--(axis cs:92928,27827.1614067349)
--(axis cs:91918,27802.0105374556)
--(axis cs:90908,27773.7696695751)
--(axis cs:89898,27756.3069627042)
--(axis cs:88888,27730.1830240656)
--(axis cs:87877,27705.7752123852)
--(axis cs:86867,27679.4004312754)
--(axis cs:85857,27650.8887095804)
--(axis cs:84847,27623.8462256823)
--(axis cs:83837,27602.164600945)
--(axis cs:82827,27574.4445318512)
--(axis cs:81817,27538.0975985027)
--(axis cs:80807,27500.2854965052)
--(axis cs:79797,27471.701590495)
--(axis cs:78787,27434.6227163846)
--(axis cs:77777,27397.0929546258)
--(axis cs:76766,27364.1870872289)
--(axis cs:75756,27324.4324577565)
--(axis cs:74746,27289.0106784902)
--(axis cs:73736,27257.6418971588)
--(axis cs:72726,27225.0009207211)
--(axis cs:71716,27186.0837700542)
--(axis cs:70706,27154.7501554531)
--(axis cs:69696,27119.7656480563)
--(axis cs:68686,27082.3109076265)
--(axis cs:67676,27042.6975214286)
--(axis cs:66666,27007.8034759735)
--(axis cs:65655,26973.3434348539)
--(axis cs:64645,26940.0669816508)
--(axis cs:63635,26903.3571706635)
--(axis cs:62625,26863.2194520057)
--(axis cs:61615,26823.4353518221)
--(axis cs:60605,26787.3518357499)
--(axis cs:59595,26741.1056929316)
--(axis cs:58585,26697.0647846831)
--(axis cs:57575,26660.1179185337)
--(axis cs:56565,26624.128604772)
--(axis cs:55555,26588.0098245174)
--(axis cs:54544,26539.9787335984)
--(axis cs:53534,26497.5377901182)
--(axis cs:52524,26461.7589866254)
--(axis cs:51514,26419.6874733958)
--(axis cs:50504,26375.2699368424)
--(axis cs:49494,26331.6945904655)
--(axis cs:48484,26288.0930347431)
--(axis cs:47474,26237.4874489738)
--(axis cs:46464,26196.2082590597)
--(axis cs:45454,26143.7449631266)
--(axis cs:44444,26090.3697368292)
--(axis cs:43433,26043.0042252486)
--(axis cs:42423,25991.3801551527)
--(axis cs:41413,25949.6066782533)
--(axis cs:40403,25883.4976835613)
--(axis cs:39393,25815.4656181846)
--(axis cs:38383,25747.4537571596)
--(axis cs:37373,25681.1136803979)
--(axis cs:36363,25613.2898555403)
--(axis cs:35353,25544.2358094062)
--(axis cs:34343,25465.3307350499)
--(axis cs:33333,25395.3850152067)
--(axis cs:32322,25323.4354701746)
--(axis cs:31312,25244.1199413752)
--(axis cs:30302,25107.1922782123)
--(axis cs:29292,24919.5299557664)
--(axis cs:28282,24727.7405021452)
--(axis cs:27272,24532.8728759084)
--(axis cs:26262,24329.6068982855)
--(axis cs:25252,24138.210397276)
--(axis cs:24242,23907.471046292)
--(axis cs:23232,23601.2246106881)
--(axis cs:22222,23284.1927153876)
--(axis cs:21211,22972.4733927679)
--(axis cs:20201,22659.2624903158)
--(axis cs:19191,22341.738475467)
--(axis cs:18181,21943.7995491934)
--(axis cs:17171,21523.3716902697)
--(axis cs:16161,21087.2425749845)
--(axis cs:15151,20628.23036824)
--(axis cs:14141,20140.587760182)
--(axis cs:13131,19658.6434525497)
--(axis cs:12121,19175.0772020223)
--(axis cs:11111,18662.1398604425)
--(axis cs:10100,18105.0561383217)
--(axis cs:9090,17312.0790836423)
--(axis cs:8080,16463.273864209)
--(axis cs:7070,15611.7168445442)
--(axis cs:6060,14679.1769773599)
--(axis cs:5050,13479.942819467)
--(axis cs:4040,11948.5647929056)
--(axis cs:3030,9971.08548912987)
--(axis cs:2020,7384.25959961112)
--(axis cs:1010,4028.84347883791)
--(axis cs:0,4.37643033763421)
--cycle;

\path [fill=forestgreen4416044, fill opacity=0.3]
(axis cs:0,4.53923343997007)
--(axis cs:0,2.1498627889466)
--(axis cs:1010,2279.17744877492)
--(axis cs:2020,4583.38530173558)
--(axis cs:3030,6907.20234882777)
--(axis cs:4040,9215.29561345543)
--(axis cs:5050,11423.332011701)
--(axis cs:6060,13637.9642471376)
--(axis cs:7070,15919.1696622831)
--(axis cs:8080,18217.5300262269)
--(axis cs:9090,20560.1120724334)
--(axis cs:10100,22892.4141944552)
--(axis cs:11111,25226.5155250895)
--(axis cs:12121,27490.1210675822)
--(axis cs:13131,29726.5501665751)
--(axis cs:14141,31916.4828315366)
--(axis cs:15151,34147.9960056079)
--(axis cs:16161,36457.3827037688)
--(axis cs:17171,38794.643644381)
--(axis cs:18181,41118.3771767553)
--(axis cs:19191,43434.013373142)
--(axis cs:20201,45757.7607306492)
--(axis cs:21211,48073.0034390427)
--(axis cs:22222,50381.9514501766)
--(axis cs:23232,52703.9921905872)
--(axis cs:24242,54974.1420898589)
--(axis cs:25252,57250.877008363)
--(axis cs:26262,59465.7173114236)
--(axis cs:27272,61699.698189046)
--(axis cs:28282,63964.3862926481)
--(axis cs:29292,66224.777367999)
--(axis cs:30302,68477.9780715621)
--(axis cs:31312,70733.3413097397)
--(axis cs:32322,72967.1471892221)
--(axis cs:33333,75210.4919416115)
--(axis cs:34343,77409.6066914118)
--(axis cs:35353,79572.1249258512)
--(axis cs:36363,81768.4447727767)
--(axis cs:37373,83975.5717846421)
--(axis cs:38383,86229.323454749)
--(axis cs:39393,88467.3158828501)
--(axis cs:40403,90733.1070336102)
--(axis cs:41413,93009.7503295182)
--(axis cs:42423,95260.1406398077)
--(axis cs:43433,97521.330979238)
--(axis cs:44444,99776.3594026454)
--(axis cs:45454,102033.565060968)
--(axis cs:46464,104280.079231684)
--(axis cs:47474,106503.028032198)
--(axis cs:48484,108754.984099942)
--(axis cs:49494,110993.539844031)
--(axis cs:50504,113212.723866687)
--(axis cs:51514,115432.884420919)
--(axis cs:52524,117651.043437741)
--(axis cs:53534,119876.194405211)
--(axis cs:54544,122094.914741179)
--(axis cs:55555,124322.391515861)
--(axis cs:56565,126508.758292106)
--(axis cs:57575,128684.527224753)
--(axis cs:58585,130859.295335089)
--(axis cs:59595,133023.520456213)
--(axis cs:60605,135170.235006862)
--(axis cs:61615,137272.323473247)
--(axis cs:62625,139378.274984036)
--(axis cs:63635,141453.437803459)
--(axis cs:64645,143581.80635641)
--(axis cs:65655,145710.36277324)
--(axis cs:66666,147835.681646447)
--(axis cs:67676,149970.994599695)
--(axis cs:68686,152144.244801186)
--(axis cs:69696,154300.096824765)
--(axis cs:70706,156447.679939699)
--(axis cs:71716,158604.578452031)
--(axis cs:72726,160778.375429069)
--(axis cs:73736,162921.785046725)
--(axis cs:74746,165073.250477634)
--(axis cs:75756,167206.854377173)
--(axis cs:76766,169344.322676971)
--(axis cs:77777,171471.679236077)
--(axis cs:78787,173603.484857468)
--(axis cs:79797,175744.765956403)
--(axis cs:80807,177867.971086147)
--(axis cs:81817,180006.007413587)
--(axis cs:82827,182125.182285789)
--(axis cs:83837,184226.330854659)
--(axis cs:84847,186325.812646198)
--(axis cs:85857,188436.391250947)
--(axis cs:86867,190529.879214951)
--(axis cs:87877,192611.759906545)
--(axis cs:88888,194690.777927542)
--(axis cs:89898,196760.60717157)
--(axis cs:90908,198804.016810983)
--(axis cs:91918,200852.17334863)
--(axis cs:92928,202894.703227042)
--(axis cs:93938,204925.432163879)
--(axis cs:94948,206952.273332717)
--(axis cs:95958,208971.373465797)
--(axis cs:96968,210964.106185348)
--(axis cs:97978,212951.589260922)
--(axis cs:98988,214937.900732699)
--(axis cs:99999,216904.681968649)
--(axis cs:99999,364414.982080837)
--(axis cs:99999,364414.982080837)
--(axis cs:98988,361288.479479243)
--(axis cs:97978,358156.184426344)
--(axis cs:96968,355019.820661329)
--(axis cs:95958,351849.081598844)
--(axis cs:94948,348615.652956338)
--(axis cs:93938,345361.977022965)
--(axis cs:92928,342126.297988615)
--(axis cs:91918,338862.554902882)
--(axis cs:90908,335601.374221172)
--(axis cs:89898,332318.133369809)
--(axis cs:88888,328996.483755635)
--(axis cs:87877,325657.88376991)
--(axis cs:86867,322322.920042769)
--(axis cs:85857,318979.734863865)
--(axis cs:84847,315636.421801997)
--(axis cs:83837,312293.529805326)
--(axis cs:82827,308940.574043822)
--(axis cs:81817,305569.998004421)
--(axis cs:80807,302183.234397637)
--(axis cs:79797,298824.882921042)
--(axis cs:78787,295423.599385499)
--(axis cs:77777,292051.049428549)
--(axis cs:76766,288672.991832529)
--(axis cs:75756,285285.506666606)
--(axis cs:74746,281907.341495436)
--(axis cs:73736,278522.936246659)
--(axis cs:72726,275128.733451612)
--(axis cs:71716,271709.340686528)
--(axis cs:70706,268314.339305822)
--(axis cs:69696,264920.578647428)
--(axis cs:68686,261521.998385883)
--(axis cs:67676,258085.092080252)
--(axis cs:66666,254623.311311582)
--(axis cs:65655,251140.700850005)
--(axis cs:64645,247662.547587388)
--(axis cs:63635,244178.492398178)
--(axis cs:62625,240704.813348639)
--(axis cs:61615,237217.471800561)
--(axis cs:60605,233714.242996921)
--(axis cs:59595,230171.627184077)
--(axis cs:58585,226553.878610349)
--(axis cs:57575,222931.951605443)
--(axis cs:56565,219294.948601821)
--(axis cs:55555,215637.267316029)
--(axis cs:54544,211974.525273702)
--(axis cs:53534,208341.597334187)
--(axis cs:52524,204723.009696241)
--(axis cs:51514,201060.48267766)
--(axis cs:50504,197374.786664212)
--(axis cs:49494,193674.618511867)
--(axis cs:48484,189956.241537251)
--(axis cs:47474,186244.855751958)
--(axis cs:46464,182544.138503207)
--(axis cs:45454,178825.350067181)
--(axis cs:44444,175122.062866483)
--(axis cs:43433,171395.392124216)
--(axis cs:42423,167667.149645197)
--(axis cs:41413,163947.884104345)
--(axis cs:40403,160219.575238741)
--(axis cs:39393,156496.637926128)
--(axis cs:38383,152787.911319992)
--(axis cs:37373,149045.114518846)
--(axis cs:36363,145315.692816271)
--(axis cs:35353,141531.691809044)
--(axis cs:34343,137745.848917872)
--(axis cs:33333,133903.750168931)
--(axis cs:32322,129983.761704651)
--(axis cs:31312,126055.77717596)
--(axis cs:30302,122118.185440288)
--(axis cs:29292,118217.449619416)
--(axis cs:28282,114287.271883172)
--(axis cs:27272,110339.959473265)
--(axis cs:26262,106421.816896376)
--(axis cs:25252,102512.665925298)
--(axis cs:24242,98536.994211796)
--(axis cs:23232,94563.898789801)
--(axis cs:22222,90551.3469823005)
--(axis cs:21211,86548.2198539504)
--(axis cs:20201,82563.7722383793)
--(axis cs:19191,78559.3521600222)
--(axis cs:18181,74542.0653557485)
--(axis cs:17171,70533.0241481197)
--(axis cs:16161,66517.4148067835)
--(axis cs:15151,62544.6381566666)
--(axis cs:14141,58557.2697748994)
--(axis cs:13131,54506.4356365706)
--(axis cs:12121,50362.9904440827)
--(axis cs:11111,46203.3733481994)
--(axis cs:10100,42022.4498353988)
--(axis cs:9090,37863.2466967034)
--(axis cs:8080,33696.0116018272)
--(axis cs:7070,29564.9538331225)
--(axis cs:6060,25445.1643441706)
--(axis cs:5050,21305.1513654514)
--(axis cs:4040,17118.0425370349)
--(axis cs:3030,12870.6028229748)
--(axis cs:2020,8583.94365017173)
--(axis cs:1010,4309.48386954513)
--(axis cs:0,4.53923343997007)
--cycle;

\path [fill=mediumpurple148103189, fill opacity=0.3]
(axis cs:0,4.22225284024301)
--(axis cs:0,2.15015629004019)
--(axis cs:1010,2221.75751015613)
--(axis cs:2020,4470.44877667487)
--(axis cs:3030,6752.71144761867)
--(axis cs:4040,9069.04199033526)
--(axis cs:5050,11374.8055034419)
--(axis cs:6060,13681.9099680266)
--(axis cs:7070,15988.1274450788)
--(axis cs:8080,18297.2652004818)
--(axis cs:9090,20600.4341613156)
--(axis cs:10100,22901.7895721689)
--(axis cs:11111,25210.2809320334)
--(axis cs:12121,27527.7741478907)
--(axis cs:13131,29834.1683696554)
--(axis cs:14141,32156.4152978109)
--(axis cs:15151,34445.9278598222)
--(axis cs:16161,36750.7276946684)
--(axis cs:17171,39051.422980898)
--(axis cs:18181,41352.968420989)
--(axis cs:19191,43660.8290161709)
--(axis cs:20201,45982.8848760762)
--(axis cs:21211,48312.0841211311)
--(axis cs:22222,50645.6798284819)
--(axis cs:23232,52959.4908983527)
--(axis cs:24242,55265.4784491635)
--(axis cs:25252,57516.3008405084)
--(axis cs:26262,59771.4481447941)
--(axis cs:27272,62000.6381656105)
--(axis cs:28282,64220.4446865914)
--(axis cs:29292,66447.6210490495)
--(axis cs:30302,68656.4057902035)
--(axis cs:31312,70861.7583147266)
--(axis cs:32322,73064.0480432824)
--(axis cs:33333,75284.1360019448)
--(axis cs:34343,77561.5407676897)
--(axis cs:35353,79834.2052805291)
--(axis cs:36363,82136.5510032759)
--(axis cs:37373,84432.77660192)
--(axis cs:38383,86717.9754833155)
--(axis cs:39393,89002.1369807706)
--(axis cs:40403,91279.0265680568)
--(axis cs:41413,93532.3849332948)
--(axis cs:42423,95801.3093316975)
--(axis cs:43433,98088.6878341552)
--(axis cs:44444,100344.147430762)
--(axis cs:45454,102603.408658549)
--(axis cs:46464,104862.856326877)
--(axis cs:47474,107113.730930728)
--(axis cs:48484,109368.651693106)
--(axis cs:49494,111607.776848132)
--(axis cs:50504,113836.436503655)
--(axis cs:51514,116082.093595839)
--(axis cs:52524,118352.723370289)
--(axis cs:53534,120616.600823834)
--(axis cs:54544,122846.244919296)
--(axis cs:55555,125092.865448721)
--(axis cs:56565,127331.114901986)
--(axis cs:57575,129567.483736335)
--(axis cs:58585,131818.293839123)
--(axis cs:59595,134109.995947972)
--(axis cs:60605,136416.292099138)
--(axis cs:61615,138733.75185536)
--(axis cs:62625,141055.932418117)
--(axis cs:63635,143327.521295786)
--(axis cs:64645,145592.361530774)
--(axis cs:65655,147857.778001782)
--(axis cs:66666,150156.506761241)
--(axis cs:67676,152474.199620972)
--(axis cs:68686,154845.70887904)
--(axis cs:69696,157176.121077978)
--(axis cs:70706,159504.532853715)
--(axis cs:71716,161847.073960355)
--(axis cs:72726,164169.91889426)
--(axis cs:73736,166483.750391542)
--(axis cs:74746,168750.302886329)
--(axis cs:75756,170975.947881251)
--(axis cs:76766,173221.305582868)
--(axis cs:77777,175509.228601287)
--(axis cs:78787,177814.368943353)
--(axis cs:79797,180117.103316813)
--(axis cs:80807,182414.234235088)
--(axis cs:81817,184725.604276736)
--(axis cs:82827,187039.469955234)
--(axis cs:83837,189333.274844497)
--(axis cs:84847,191647.838718635)
--(axis cs:85857,193963.41650873)
--(axis cs:86867,196259.417242533)
--(axis cs:87877,198568.164695088)
--(axis cs:88888,200879.592615954)
--(axis cs:89898,203186.022761742)
--(axis cs:90908,205489.325119148)
--(axis cs:91918,207797.374613031)
--(axis cs:92928,210094.591733712)
--(axis cs:93938,212375.964823939)
--(axis cs:94948,214616.588239811)
--(axis cs:95958,216854.648292623)
--(axis cs:96968,219104.310005522)
--(axis cs:97978,221329.618392668)
--(axis cs:98988,223571.541142636)
--(axis cs:99999,225830.207082255)
--(axis cs:99999,407820.814674424)
--(axis cs:99999,407820.814674424)
--(axis cs:98988,403791.916372865)
--(axis cs:97978,399811.174215051)
--(axis cs:96968,395859.013404198)
--(axis cs:95958,391898.149377089)
--(axis cs:94948,387965.273681817)
--(axis cs:93938,384031.173884046)
--(axis cs:92928,380077.042471432)
--(axis cs:91918,376111.827220005)
--(axis cs:90908,372143.522241055)
--(axis cs:89898,368175.937111881)
--(axis cs:88888,364185.632629191)
--(axis cs:87877,360202.732071578)
--(axis cs:86867,356207.863411127)
--(axis cs:85857,352210.524258995)
--(axis cs:84847,348194.42092991)
--(axis cs:83837,344190.608054205)
--(axis cs:82827,340206.161350239)
--(axis cs:81817,336193.240164376)
--(axis cs:80807,332193.500347546)
--(axis cs:79797,328203.303903547)
--(axis cs:78787,324225.381799042)
--(axis cs:77777,320243.111289261)
--(axis cs:76766,316249.618979059)
--(axis cs:75756,312196.851309708)
--(axis cs:74746,308150.671496564)
--(axis cs:73736,304075.865974542)
--(axis cs:72726,299971.317883317)
--(axis cs:71716,295863.286908068)
--(axis cs:70706,291747.3931993)
--(axis cs:69696,287614.26423121)
--(axis cs:68686,283476.046752651)
--(axis cs:67676,279241.287921315)
--(axis cs:66666,275063.621272764)
--(axis cs:65655,270898.408124349)
--(axis cs:64645,266754.12672421)
--(axis cs:63635,262605.664788446)
--(axis cs:62625,258484.019991416)
--(axis cs:61615,254345.388005596)
--(axis cs:60605,250222.271772302)
--(axis cs:59595,246104.394197078)
--(axis cs:58585,242031.204435902)
--(axis cs:57575,237978.259860098)
--(axis cs:56565,233924.091403027)
--(axis cs:55555,229880.874200015)
--(axis cs:54544,225821.261205404)
--(axis cs:53534,221782.879672913)
--(axis cs:52524,217724.378543858)
--(axis cs:51514,213679.041490735)
--(axis cs:50504,209660.782645786)
--(axis cs:49494,205658.749801563)
--(axis cs:48484,201640.162814938)
--(axis cs:47474,197608.708060236)
--(axis cs:46464,193585.989306667)
--(axis cs:45454,189548.276885345)
--(axis cs:44444,185520.154594185)
--(axis cs:43433,181501.422376904)
--(axis cs:42423,177421.791337575)
--(axis cs:41413,173361.244964383)
--(axis cs:40403,169306.12694399)
--(axis cs:39393,165229.757375258)
--(axis cs:38383,161111.083414193)
--(axis cs:37373,156987.107614934)
--(axis cs:36363,152859.969638411)
--(axis cs:35353,148717.099048843)
--(axis cs:34343,144533.776262077)
--(axis cs:33333,140336.806475461)
--(axis cs:32322,136189.68113605)
--(axis cs:31312,132044.155217394)
--(axis cs:30302,127895.775805456)
--(axis cs:29292,123741.096410419)
--(axis cs:28282,119557.738815046)
--(axis cs:27272,115355.76339266)
--(axis cs:26262,111153.899132332)
--(axis cs:25252,106927.662851183)
--(axis cs:24242,102702.4812514)
--(axis cs:23232,98436.9631514399)
--(axis cs:22222,94169.1240001648)
--(axis cs:21211,89894.5102116012)
--(axis cs:20201,85578.610266003)
--(axis cs:19191,81278.1655191842)
--(axis cs:18181,76995.4381755935)
--(axis cs:17171,72728.2603355537)
--(axis cs:16161,68466.6561222069)
--(axis cs:15151,64199.2301694875)
--(axis cs:14141,59944.7181171763)
--(axis cs:13131,55671.6660123865)
--(axis cs:12121,51401.9572067737)
--(axis cs:11111,47136.5397163591)
--(axis cs:10100,42866.8808562485)
--(axis cs:9090,38611.0217316618)
--(axis cs:8080,34348.6605494483)
--(axis cs:7070,30080.5845914275)
--(axis cs:6060,25827.6067377006)
--(axis cs:5050,21553.4825970014)
--(axis cs:4040,17280.9074771496)
--(axis cs:3030,13020.3757265385)
--(axis cs:2020,8722.70081946868)
--(axis cs:1010,4354.93111112866)
--(axis cs:0,4.22225284024301)
--cycle;

\addplot [very thick, steelblue31119180]
table {%
0 3.20681074563853
1010 3241.49873161582
2020 6327.13345594835
3030 9198.75535282329
4040 11785.9646308524
5050 14112.5886801593
6060 16115.2987467124
7070 17843.7469354063
8080 19275.6284367001
9090 20549.1910964548
10100 21641.2262486358
11111 22601.7580305497
12121 23448.5132025025
13131 24175.8390860294
14141 24850.9923119139
15151 25491.0038086657
16161 26120.5043760554
17171 26756.9347265794
18181 27356.9240619735
19191 27919.3676169483
20201 28453.0820383111
21211 28882.5030716438
22222 29282.7819072764
23232 29685.6798804345
24242 30056.4051264742
25252 30416.2337165074
26262 30763.1861412642
27272 31112.3458651847
28282 31446.6520809167
29292 31782.3952836302
30302 32135.1758232027
31312 32497.2232175149
32322 32799.876380182
33333 33094.5449451979
34343 33389.1120946188
35353 33678.8584551107
36363 33958.33497905
37373 34227.6497694746
38383 34494.8305432902
39393 34724.108727642
40403 34949.0086280529
41413 35168.522006828
42423 35385.612337491
43433 35600.5441472851
44444 35805.8430616009
45454 36008.7801339376
46464 36219.6297501371
47474 36423.3650929662
48484 36626.8969389777
49494 36812.5751740004
50504 36952.4189813895
51514 37091.8218950972
52524 37232.8445904055
53534 37366.4503500403
54544 37553.195375246
55555 37708.8729683181
56565 37841.5831744273
57575 37979.2839152058
58585 38107.5176752661
59595 38238.0858236727
60605 38371.1927101148
61615 38501.6009478703
62625 38592.0979513014
63635 38673.4659738431
64645 38757.0538589261
65655 38830.6454365787
66666 38903.6359960067
67676 38981.5114454525
68686 39055.5503890703
69696 39129.0920352701
70706 39202.5389490587
71716 39278.6300991922
72726 39351.028671941
73736 39421.8963545075
74746 39489.5488652575
75756 39559.4180146986
76766 39617.2885709368
77777 39684.5252309331
78787 39757.8556167629
79797 39830.7880999461
80807 39894.956111289
81817 39960.6966684019
82827 40015.2491967391
83837 40073.0620380024
84847 40120.7675021875
85857 40189.136744946
86867 40250.2226660268
87877 40307.5539206939
88888 40358.0486718468
89898 40410.1227221656
90908 40462.4304932507
91918 40516.5003226282
92928 40568.8692781098
93938 40618.9268898859
94948 40669.2960440598
95958 40725.7487752404
96968 40770.5826751612
97978 40820.3478088857
98988 40868.1425701217
99999 40910.1264355021
};
\addlegendentry{\ucbvibf}
\addplot [very thick, darkorange25512714]
table {%
0 3.18673413522127
1010 3159.43494891418
2020 5903.12187956879
3030 8060.03322550772
4040 9650.99513753377
5050 10830.8532580884
6060 11749.565606993
7070 12455.1469937583
8080 13086.7500113153
9090 13719.7928908711
10100 14275.2038086631
11111 14694.8282801657
12121 15078.4466478652
13131 15428.4686241023
14141 15776.3517360229
15151 16134.2705881671
16161 16453.0878707956
17171 16747.7078160819
18181 17027.7436073687
19191 17294.9331425616
20201 17520.9219605708
21211 17742.5677352015
22222 17959.0700778783
23232 18176.4261356645
24242 18381.914525573
25252 18538.8513590353
26262 18675.795719327
27272 18834.9136417104
28282 18975.9031860619
29292 19107.4995040247
30302 19235.040288628
31312 19334.7263170541
32322 19414.5749166628
33333 19489.2802006335
34343 19560.0628636101
35353 19638.3188616859
36363 19706.4856885932
37373 19774.7285689545
38383 19837.8837248933
39393 19905.9238691149
40403 19977.7816429606
41413 20042.5665249119
42423 20089.135694324
43433 20144.3385089385
44444 20194.894572837
45454 20249.183163752
46464 20307.8566285648
47474 20348.9460272929
48484 20405.1588574822
49494 20449.2258640042
50504 20492.273005171
51514 20536.1263600116
52524 20577.7815979831
53534 20615.8601311952
54544 20658.3040574372
55555 20704.3653286098
56565 20739.8889541559
57575 20774.4438268582
58585 20809.6331689408
59595 20852.2702494533
60605 20896.2128748951
61615 20931.5905688388
62625 20972.910190824
63635 21010.0492055218
64645 21045.5596829649
65655 21076.6196229752
66666 21111.2998706439
67676 21143.4734867638
68686 21181.1985881192
69696 21216.0471262916
70706 21246.2497777067
71716 21275.4304158063
72726 21310.6837150462
73736 21340.3901475023
74746 21370.1337670258
75756 21404.1801323846
76766 21440.5873603375
77777 21473.9268544668
78787 21508.7033584436
79797 21544.0040389449
80807 21570.5986927663
81817 21608.9472352805
82827 21642.3087433444
83837 21671.9260985541
84847 21688.6723595841
85857 21716.2705676155
86867 21745.9670347661
87877 21773.3403756878
88888 21800.5117536743
89898 21826.9191962443
90908 21844.8000330513
91918 21877.2596014666
92928 21904.8191668355
93938 21934.9388910627
94948 21949.8277416441
95958 21982.1864881779
96968 22009.675357095
97978 22027.6788201399
98988 22042.7670592889
99999 22063.1114393799
};
\addlegendentry{\ucbvibfi}
\addplot [very thick, forestgreen4416044]
table {%
0 3.34454811445834
1010 3294.33065916003
2020 6583.66447595365
3030 9888.9025859013
4040 13166.6690752452
5050 16364.2416885762
6060 19541.5642956541
7070 22742.0617477028
8080 25956.770814027
9090 29211.6793845684
10100 32457.432014927
11111 35714.9444366444
12121 38926.5557558325
13131 42116.4929015728
14141 45236.876303218
15151 48346.3170811372
16161 51487.3987552762
17171 54663.8338962503
18181 57830.2212662519
19191 60996.6827665821
20201 64160.7664845142
21211 67310.6116464966
22222 70466.6492162385
23232 73633.9454901941
24242 76755.5681508275
25252 79881.7714668304
26262 82943.7671039
27272 86019.8288311557
28282 89125.8290879102
29292 92221.1134937074
30302 95298.0817559251
31312 98394.5592428498
32322 101475.454446936
33333 104557.121055271
34343 107577.727804642
35353 110551.908367448
36363 113542.068794524
37373 116510.343151744
38383 119508.61738737
39393 122481.976904489
40403 125476.341136176
41413 128478.817216931
42423 131463.645142502
43433 134458.361551727
44444 137449.211134564
45454 140429.457564074
46464 143412.108867445
47474 146373.941892078
48484 149355.612818596
49494 152334.079177949
50504 155293.75526545
51514 158246.68354929
52524 161187.026566991
53534 164108.895869699
54544 167034.720007441
55555 169979.829415945
56565 172901.853446963
57575 175808.239415098
58585 178706.586972719
59595 181597.573820145
60605 184442.239001891
61615 187244.897636904
62625 190041.544166337
63635 192815.965100819
64645 195622.176971899
65655 198425.531811622
66666 201229.496479015
67676 204028.043339974
68686 206833.121593535
69696 209610.337736097
70706 212381.009622761
71716 215156.95956928
72726 217953.55444034
73736 220722.360646692
74746 223490.295986535
75756 226246.180521889
76766 229008.65725475
77777 231761.364332313
78787 234513.542121483
79797 237284.824438723
80807 240025.602741892
81817 242788.002709004
82827 245532.878164805
83837 248259.930329993
84847 250981.117224097
85857 253708.063057406
86867 256426.39962886
87877 259134.821838227
88888 261843.630841589
89898 264539.370270689
90908 267202.695516078
91918 269857.364125756
92928 272510.500607828
93938 275143.704593422
94948 277783.963144528
95958 280410.227532321
96968 282991.963423339
97978 285553.886843633
98988 288113.190105971
99999 290659.832024743
};
\addlegendentry{\ucbvich}
\addplot [very thick, mediumpurple148103189]
table {%
0 3.1862045651416
1010 3288.3443106424
2020 6596.57479807178
3030 9886.5435870786
4040 13174.9747337424
5050 16464.1440502216
6060 19754.7583528636
7070 23034.3560182531
8080 26322.9628749651
9090 29605.7279464887
10100 32884.3352142087
11111 36173.4103241962
12121 39464.8656773322
13131 42752.917191021
14141 46050.5667074936
15151 49322.5790146549
16161 52608.6919084376
17171 55889.8416582259
18181 59174.2032982912
19191 62469.4972676775
20201 65780.7475710396
21211 69103.2971663662
22222 72407.4019143234
23232 75698.2270248963
24242 78983.9798502819
25252 82221.9818458455
26262 85462.6736385633
27272 88678.2007791353
28282 91889.0917508189
29292 95094.3587297342
30302 98276.0907978296
31312 101452.95676606
32322 104626.864589666
33333 107810.471238703
34343 111047.658514883
35353 114275.652164686
36363 117498.260320844
37373 120709.942108427
38383 123914.529448754
39393 127115.947178014
40403 130292.576756023
41413 133446.814948839
42423 136611.550334636
43433 139795.05510553
44444 142932.151012473
45454 146075.842771947
46464 149224.422816772
47474 152361.219495482
48484 155504.407254022
49494 158633.263324847
50504 161748.609574721
51514 164880.567543287
52524 168038.550957074
53534 171199.740248373
54544 174333.75306235
55555 177486.869824368
56565 180627.603152507
57575 183772.871798216
58585 186924.749137513
59595 190107.195072525
60605 193319.28193572
61615 196539.569930478
62625 199769.976204767
63635 202966.593042116
64645 206173.244127492
65655 209378.093063066
66666 212610.064017002
67676 215857.743771144
68686 219160.877815846
69696 222395.192654594
70706 225625.963026507
71716 228855.180434212
72726 232070.618388788
73736 235279.808183042
74746 238450.487191447
75756 241586.399595479
76766 244735.462280963
77777 247876.169945274
78787 251019.875371197
79797 254160.20361018
80807 257303.867291317
81817 260459.422220556
82827 263622.815652736
83837 266761.941449351
84847 269921.129824273
85857 273086.970383863
86867 276233.64032683
87877 279385.448383333
88888 282532.612622572
89898 285680.979936811
90908 288816.423680102
91918 291954.600916518
92928 295085.817102572
93938 298203.569353992
94948 301290.930960814
95958 304376.398834856
96968 307481.66170486
97978 310570.39630386
98988 313681.72875775
99999 316825.510878339
};
\addlegendentry{\mvp}
\end{axis}

\end{tikzpicture}
}
% \caption{$H=10, \ K=10^5$.}
% \end{subfigure}
% %\hfill
% \begin{subfigure}[b]{0.37\textwidth}
% \resizebox{.4\linewidth}{!}{% This file was created with tikzplotlib v0.10.1.
\begin{tikzpicture}

\definecolor{crimson2143940}{RGB}{214,39,40}
\definecolor{darkgray176}{RGB}{176,176,176}
\definecolor{darkorange25512714}{RGB}{255,127,14}
\definecolor{forestgreen4416044}{RGB}{44,160,44}
\definecolor{lightgray204}{RGB}{204,204,204}
\definecolor{mediumpurple148103189}{RGB}{148,103,189}
\definecolor{steelblue31119180}{RGB}{31,119,180}

\begin{axis}[
legend cell align={left},
legend style={
  fill opacity=0.8,
  draw opacity=1,
  text opacity=1,
  at={(0.03,0.97)},
  anchor=north west,
  draw=lightgray204,
  font=\small
},
tick align=outside,
tick pos=left,
x grid style={darkgray176},
xmin=0, xmax=1000000,
xtick style={color=black},
y grid style={darkgray176},
ymin=-100000, ymax=3000000,
ytick style={color=black},
grid=both,
xlabel={Episodes},
ylabel={Cumulative Regret},
grid style={dashed,gray!50!white}
]
\path [fill=steelblue31119180, fill opacity=0.3]
(axis cs:0,4.1581568934368)
--(axis cs:0,2.25546459784027)
--(axis cs:10101,17075.188803749)
--(axis cs:20202,21958.9446441714)
--(axis cs:30303,24431.7457937236)
--(axis cs:40404,26117.394359249)
--(axis cs:50505,27302.6133899681)
--(axis cs:60606,28239.3195922842)
--(axis cs:70707,29015.1494767359)
--(axis cs:80808,29747.9604414048)
--(axis cs:90909,30371.5552865965)
--(axis cs:101010,30911.612517177)
--(axis cs:111111,31427.512112457)
--(axis cs:121212,31849.6034770322)
--(axis cs:131313,32231.8530217123)
--(axis cs:141414,32601.0667869566)
--(axis cs:151515,32936.8425881253)
--(axis cs:161616,33229.0552686342)
--(axis cs:171717,33500.4136685417)
--(axis cs:181818,33790.4310664637)
--(axis cs:191919,34052.5008304188)
--(axis cs:202020,34298.0016458073)
--(axis cs:212121,34556.7657464001)
--(axis cs:222222,34812.5254701219)
--(axis cs:232323,35013.7518242399)
--(axis cs:242424,35248.1271695128)
--(axis cs:252525,35459.8426743929)
--(axis cs:262626,35681.2754215143)
--(axis cs:272727,35884.1099671432)
--(axis cs:282828,36097.8183159121)
--(axis cs:292929,36299.7634793093)
--(axis cs:303030,36459.1518636763)
--(axis cs:313131,36625.6385072999)
--(axis cs:323232,36794.1436523312)
--(axis cs:333333,36988.9901320844)
--(axis cs:343434,37171.7050431696)
--(axis cs:353535,37369.1247986999)
--(axis cs:363636,37546.3806701977)
--(axis cs:373737,37687.6762162163)
--(axis cs:383838,37852.1611318213)
--(axis cs:393939,38036.5027402727)
--(axis cs:404040,38181.393877346)
--(axis cs:414141,38296.7479529439)
--(axis cs:424242,38434.9515007489)
--(axis cs:434343,38599.9750784257)
--(axis cs:444444,38750.8801476991)
--(axis cs:454545,38861.1819322627)
--(axis cs:464646,38969.6076469794)
--(axis cs:474747,39099.1281726381)
--(axis cs:484848,39207.4783521922)
--(axis cs:494949,39318.3732893837)
--(axis cs:505050,39411.3019414908)
--(axis cs:515151,39522.7212070064)
--(axis cs:525252,39623.855200943)
--(axis cs:535353,39774.2481784265)
--(axis cs:545454,39883.2956836982)
--(axis cs:555555,39992.6546492537)
--(axis cs:565656,40126.2481530333)
--(axis cs:575757,40231.0912416713)
--(axis cs:585858,40325.3862491165)
--(axis cs:595959,40423.8927207807)
--(axis cs:606060,40554.2119777762)
--(axis cs:616161,40642.8651422898)
--(axis cs:626262,40734.3528030205)
--(axis cs:636363,40816.250477866)
--(axis cs:646464,40909.5277089179)
--(axis cs:656565,40999.6238901521)
--(axis cs:666666,41097.2161660952)
--(axis cs:676767,41206.1047462413)
--(axis cs:686868,41305.4617306499)
--(axis cs:696969,41403.3840343174)
--(axis cs:707070,41484.0961164635)
--(axis cs:717171,41564.4935883499)
--(axis cs:727272,41635.9969545169)
--(axis cs:737373,41718.0927451142)
--(axis cs:747474,41796.9834848593)
--(axis cs:757575,41873.8348596431)
--(axis cs:767676,41958.5429583684)
--(axis cs:777777,42023.8574027319)
--(axis cs:787878,42081.4963661463)
--(axis cs:797979,42147.1188273474)
--(axis cs:808080,42211.0698664142)
--(axis cs:818181,42252.6110074235)
--(axis cs:828282,42285.6523413838)
--(axis cs:838383,42348.4491649523)
--(axis cs:848484,42394.1650349573)
--(axis cs:858585,42478.1192549405)
--(axis cs:868686,42571.7831178116)
--(axis cs:878787,42611.9692527014)
--(axis cs:888888,42682.4447986449)
--(axis cs:898989,42759.8678162508)
--(axis cs:909090,42834.3536603168)
--(axis cs:919191,42906.6407401747)
--(axis cs:929292,42984.3119143901)
--(axis cs:939393,43037.5580937877)
--(axis cs:949494,43118.9661223713)
--(axis cs:959595,43198.9068040762)
--(axis cs:969696,43236.884983607)
--(axis cs:979797,43326.7175673557)
--(axis cs:989898,43372.8651215068)
--(axis cs:999999,43435.3644067176)
--(axis cs:999999,67256.1786628046)
--(axis cs:999999,67256.1786628046)
--(axis cs:989898,67195.1325870761)
--(axis cs:979797,67152.4227895314)
--(axis cs:969696,67076.037437349)
--(axis cs:959595,67030.5339006717)
--(axis cs:949494,66981.1486083352)
--(axis cs:939393,66916.6649322243)
--(axis cs:929292,66864.5835823427)
--(axis cs:919191,66803.4869527099)
--(axis cs:909090,66746.7918164181)
--(axis cs:898989,66686.4044531918)
--(axis cs:888888,66617.2598026184)
--(axis cs:878787,66565.161183559)
--(axis cs:868686,66516.5824471432)
--(axis cs:858585,66450.8214116389)
--(axis cs:848484,66381.9341297619)
--(axis cs:838383,66333.2502923409)
--(axis cs:828282,66282.3157591784)
--(axis cs:818181,66234.1744506118)
--(axis cs:808080,66201.7779239829)
--(axis cs:797979,66144.077301748)
--(axis cs:787878,66084.7068437811)
--(axis cs:777777,66026.4040140455)
--(axis cs:767676,65976.0454756906)
--(axis cs:757575,65899.0899928953)
--(axis cs:747474,65832.8175748029)
--(axis cs:737373,65772.6082632762)
--(axis cs:727272,65694.9460373425)
--(axis cs:717171,65631.4147094)
--(axis cs:707070,65558.9396488764)
--(axis cs:696969,65485.1358979819)
--(axis cs:686868,65406.7132990909)
--(axis cs:676767,65337.471677957)
--(axis cs:666666,65246.39013288)
--(axis cs:656565,65164.8072134807)
--(axis cs:646464,65096.4545698211)
--(axis cs:636363,65013.1108432706)
--(axis cs:626262,64949.5384903208)
--(axis cs:616161,64864.3317364295)
--(axis cs:606060,64788.2384990032)
--(axis cs:595959,64672.4234422598)
--(axis cs:585858,64581.2694683753)
--(axis cs:575757,64496.9858339696)
--(axis cs:565656,64402.5732696582)
--(axis cs:555555,64301.1754078124)
--(axis cs:545454,64214.9335807992)
--(axis cs:535353,64119.383617673)
--(axis cs:525252,64000.9411561406)
--(axis cs:515151,63913.205712439)
--(axis cs:505050,63815.1616232267)
--(axis cs:494949,63727.0629405579)
--(axis cs:484848,63619.6237248909)
--(axis cs:474747,63513.8594510971)
--(axis cs:464646,63413.5055605972)
--(axis cs:454545,63325.4016638821)
--(axis cs:444444,63220.1463484199)
--(axis cs:434343,63086.8198511359)
--(axis cs:424242,62948.3862702047)
--(axis cs:414141,62834.2741924839)
--(axis cs:404040,62737.2974668179)
--(axis cs:393939,62612.1475171669)
--(axis cs:383838,62459.8205852744)
--(axis cs:373737,62322.957472705)
--(axis cs:363636,62182.60336084)
--(axis cs:353535,62026.8730225482)
--(axis cs:343434,61864.6432420496)
--(axis cs:333333,61704.7293351867)
--(axis cs:323232,61540.3826058245)
--(axis cs:313131,61384.577940297)
--(axis cs:303030,61221.2355250671)
--(axis cs:292929,61067.5221191526)
--(axis cs:282828,60877.5694377332)
--(axis cs:272727,60692.4240415877)
--(axis cs:262626,60500.5386561637)
--(axis cs:252525,60297.3827776549)
--(axis cs:242424,60106.905097113)
--(axis cs:232323,59888.9144031598)
--(axis cs:222222,59679.4667767167)
--(axis cs:212121,59437.8255290928)
--(axis cs:202020,59195.4698974949)
--(axis cs:191919,58965.8552241069)
--(axis cs:181818,58717.242532283)
--(axis cs:171717,58392.9484113245)
--(axis cs:161616,58064.5161040156)
--(axis cs:151515,57718.5762204291)
--(axis cs:141414,57371.5661885264)
--(axis cs:131313,56976.2042916662)
--(axis cs:121212,56578.1281629681)
--(axis cs:111111,56160.4545388492)
--(axis cs:101010,55697.4116982338)
--(axis cs:90909,55191.6152389658)
--(axis cs:80808,54592.7837156692)
--(axis cs:70707,53885.345693417)
--(axis cs:60606,52666.7197939398)
--(axis cs:50505,50586.657397308)
--(axis cs:40404,47501.6407392578)
--(axis cs:30303,43132.7114145608)
--(axis cs:20202,37751.1816834848)
--(axis cs:10101,28022.4238245757)
--(axis cs:0,4.1581568934368)
--cycle;

\path [fill=darkorange25512714, fill opacity=0.3]
(axis cs:0,4.46399887127577)
--(axis cs:0,2.4884629204055)
--(axis cs:10101,10988.1231249621)
--(axis cs:20202,13011.7778125794)
--(axis cs:30303,14075.0582871427)
--(axis cs:40404,14811.9645776719)
--(axis cs:50505,15391.3451913966)
--(axis cs:60606,15872.0373617294)
--(axis cs:70707,16238.131735011)
--(axis cs:80808,16574.5247051889)
--(axis cs:90909,16910.2456621313)
--(axis cs:101010,17204.7406572722)
--(axis cs:111111,17461.6123784346)
--(axis cs:121212,17682.6659414481)
--(axis cs:131313,17894.5623121896)
--(axis cs:141414,18092.8635734043)
--(axis cs:151515,18276.8704889549)
--(axis cs:161616,18431.9201889604)
--(axis cs:171717,18625.3278743936)
--(axis cs:181818,18806.6443183696)
--(axis cs:191919,18986.7432570405)
--(axis cs:202020,19120.7562071022)
--(axis cs:212121,19257.9770449525)
--(axis cs:222222,19410.6458065547)
--(axis cs:232323,19521.5773270099)
--(axis cs:242424,19604.185026784)
--(axis cs:252525,19720.1126648513)
--(axis cs:262626,19810.3577263385)
--(axis cs:272727,19911.1425793029)
--(axis cs:282828,20030.3085456276)
--(axis cs:292929,20117.0219002504)
--(axis cs:303030,20222.2882954054)
--(axis cs:313131,20314.4468198199)
--(axis cs:323232,20404.2704365769)
--(axis cs:333333,20497.955553623)
--(axis cs:343434,20569.0127220001)
--(axis cs:353535,20670.687018517)
--(axis cs:363636,20757.1287307167)
--(axis cs:373737,20856.3792825633)
--(axis cs:383838,20947.856798489)
--(axis cs:393939,21030.92167589)
--(axis cs:404040,21105.066954296)
--(axis cs:414141,21219.8289792219)
--(axis cs:424242,21299.9368402749)
--(axis cs:434343,21368.3939800977)
--(axis cs:444444,21444.4801030993)
--(axis cs:454545,21512.439734932)
--(axis cs:464646,21580.9580640702)
--(axis cs:474747,21613.3052821679)
--(axis cs:484848,21666.9025694415)
--(axis cs:494949,21711.2401634184)
--(axis cs:505050,21771.0495707181)
--(axis cs:515151,21822.2770861377)
--(axis cs:525252,21886.1577126143)
--(axis cs:535353,21951.8708483411)
--(axis cs:545454,22008.5634196321)
--(axis cs:555555,22071.9579619838)
--(axis cs:565656,22135.4313977517)
--(axis cs:575757,22171.3728232612)
--(axis cs:585858,22239.8162830004)
--(axis cs:595959,22314.9663512497)
--(axis cs:606060,22369.2029884277)
--(axis cs:616161,22420.5009624302)
--(axis cs:626262,22478.1373929099)
--(axis cs:636363,22517.057666621)
--(axis cs:646464,22570.6298363996)
--(axis cs:656565,22577.5499052041)
--(axis cs:666666,22626.0125207718)
--(axis cs:676767,22665.9113954406)
--(axis cs:686868,22748.4988749483)
--(axis cs:696969,22818.5071611054)
--(axis cs:707070,22840.665114886)
--(axis cs:717171,22886.6001877198)
--(axis cs:727272,22907.2346172322)
--(axis cs:737373,22970.6681682536)
--(axis cs:747474,23012.0283983454)
--(axis cs:757575,23074.0353353773)
--(axis cs:767676,23108.613193017)
--(axis cs:777777,23133.6964717459)
--(axis cs:787878,23164.5562309903)
--(axis cs:797979,23230.7732436292)
--(axis cs:808080,23259.1238132821)
--(axis cs:818181,23302.9659625825)
--(axis cs:828282,23335.9542281872)
--(axis cs:838383,23382.2845417571)
--(axis cs:848484,23422.2918663682)
--(axis cs:858585,23437.8640835242)
--(axis cs:868686,23458.1146327663)
--(axis cs:878787,23498.1074383645)
--(axis cs:888888,23523.5911075305)
--(axis cs:898989,23582.2550519231)
--(axis cs:909090,23629.1572502409)
--(axis cs:919191,23680.1453027243)
--(axis cs:929292,23721.4406136349)
--(axis cs:939393,23757.4776102957)
--(axis cs:949494,23824.5700845003)
--(axis cs:959595,23852.1534161432)
--(axis cs:969696,23881.0285132494)
--(axis cs:979797,23927.6027292719)
--(axis cs:989898,23958.7327060298)
--(axis cs:999999,24001.4207722068)
--(axis cs:999999,35660.2399066547)
--(axis cs:999999,35660.2399066547)
--(axis cs:989898,35620.2485651952)
--(axis cs:979797,35587.1376696865)
--(axis cs:969696,35540.8670690995)
--(axis cs:959595,35513.9782093504)
--(axis cs:949494,35480.8852762106)
--(axis cs:939393,35428.2336417838)
--(axis cs:929292,35403.5788638651)
--(axis cs:919191,35359.6425092259)
--(axis cs:909090,35312.7992907425)
--(axis cs:898989,35280.9841802985)
--(axis cs:888888,35229.5610561906)
--(axis cs:878787,35210.5209109758)
--(axis cs:868686,35165.8047333617)
--(axis cs:858585,35152.1016380757)
--(axis cs:848484,35126.8293031295)
--(axis cs:838383,35089.7662902619)
--(axis cs:828282,35051.2217738795)
--(axis cs:818181,35022.0605789808)
--(axis cs:808080,34969.5188344162)
--(axis cs:797979,34939.0340187424)
--(axis cs:787878,34888.1111994769)
--(axis cs:777777,34863.6826151756)
--(axis cs:767676,34833.999659394)
--(axis cs:757575,34804.7709342933)
--(axis cs:747474,34747.7881095372)
--(axis cs:737373,34712.4466068326)
--(axis cs:727272,34655.8253422584)
--(axis cs:717171,34631.3440428847)
--(axis cs:707070,34582.8167193823)
--(axis cs:696969,34564.9881220113)
--(axis cs:686868,34507.7955504764)
--(axis cs:676767,34450.8489018399)
--(axis cs:666666,34410.1651771799)
--(axis cs:656565,34365.144041892)
--(axis cs:646464,34345.4865082857)
--(axis cs:636363,34303.1254655594)
--(axis cs:626262,34262.2943510253)
--(axis cs:616161,34225.609209187)
--(axis cs:606060,34177.4322820096)
--(axis cs:595959,34136.697896065)
--(axis cs:585858,34076.9931696105)
--(axis cs:575757,34013.5011767942)
--(axis cs:565656,33987.7269673986)
--(axis cs:555555,33943.231253872)
--(axis cs:545454,33906.0335608561)
--(axis cs:535353,33848.1146049935)
--(axis cs:525252,33800.0207179424)
--(axis cs:515151,33747.8799968717)
--(axis cs:505050,33683.1667024879)
--(axis cs:494949,33635.4128254593)
--(axis cs:484848,33582.0637864674)
--(axis cs:474747,33525.5253112468)
--(axis cs:464646,33490.882593916)
--(axis cs:454545,33425.7646727815)
--(axis cs:444444,33351.2928716142)
--(axis cs:434343,33276.6655433854)
--(axis cs:424242,33227.9221343833)
--(axis cs:414141,33149.0594230413)
--(axis cs:404040,33060.9186672476)
--(axis cs:393939,32990.8811653387)
--(axis cs:383838,32940.0231544116)
--(axis cs:373737,32866.0790524409)
--(axis cs:363636,32789.9213927335)
--(axis cs:353535,32704.0969642244)
--(axis cs:343434,32617.9747846937)
--(axis cs:333333,32549.1062000203)
--(axis cs:323232,32471.9967118915)
--(axis cs:313131,32386.7306516582)
--(axis cs:303030,32308.9991842423)
--(axis cs:292929,32219.116774759)
--(axis cs:282828,32132.5966541428)
--(axis cs:272727,32030.3003396095)
--(axis cs:262626,31931.2690745002)
--(axis cs:252525,31848.8704157307)
--(axis cs:242424,31742.9278039761)
--(axis cs:232323,31653.8003028413)
--(axis cs:222222,31554.016453481)
--(axis cs:212121,31424.7069389686)
--(axis cs:202020,31305.0354762611)
--(axis cs:191919,31180.6109949562)
--(axis cs:181818,31028.9573359778)
--(axis cs:171717,30865.729433111)
--(axis cs:161616,30706.0638642747)
--(axis cs:151515,30555.7935513007)
--(axis cs:141414,30385.4775401167)
--(axis cs:131313,30208.7064365794)
--(axis cs:121212,30014.3295658434)
--(axis cs:111111,29813.1492608031)
--(axis cs:101010,29567.0326474046)
--(axis cs:90909,29291.5338560502)
--(axis cs:80808,28935.5157464705)
--(axis cs:70707,28573.3215900965)
--(axis cs:60606,28200.0779875704)
--(axis cs:50505,27742.6279366045)
--(axis cs:40404,27179.6781572276)
--(axis cs:30303,26241.324031173)
--(axis cs:20202,23684.7181825363)
--(axis cs:10101,18871.211435684)
--(axis cs:0,4.46399887127577)
--cycle;

\path [fill=forestgreen4416044, fill opacity=0.3]
(axis cs:0,4.4218025748378)
--(axis cs:0,2.38314083536624)
--(axis cs:10101,22747.9044182869)
--(axis cs:20202,45507.1292638364)
--(axis cs:30303,68625.1080994069)
--(axis cs:40404,91243.141864535)
--(axis cs:50505,113429.734289215)
--(axis cs:60606,135980.900244987)
--(axis cs:70707,158180.653248037)
--(axis cs:80808,179691.990960519)
--(axis cs:90909,201090.058555465)
--(axis cs:101010,222561.020102346)
--(axis cs:111111,243701.486281811)
--(axis cs:121212,264216.661574447)
--(axis cs:131313,284175.167978294)
--(axis cs:141414,304253.062641991)
--(axis cs:151515,324032.581271215)
--(axis cs:161616,343740.17855034)
--(axis cs:171717,362943.803946603)
--(axis cs:181818,381697.862129413)
--(axis cs:191919,400062.613837421)
--(axis cs:202020,418409.101612721)
--(axis cs:212121,436363.937447003)
--(axis cs:222222,454149.71449662)
--(axis cs:232323,471483.773439477)
--(axis cs:242424,488290.703954612)
--(axis cs:252525,504217.735576556)
--(axis cs:262626,518992.845642759)
--(axis cs:272727,533215.527193603)
--(axis cs:282828,546651.688410665)
--(axis cs:292929,559927.622415)
--(axis cs:303030,572581.605081305)
--(axis cs:313131,584892.412458601)
--(axis cs:323232,597079.982573641)
--(axis cs:333333,609215.147880389)
--(axis cs:343434,621094.347435325)
--(axis cs:353535,632578.402445137)
--(axis cs:363636,643606.517598135)
--(axis cs:373737,654407.662796795)
--(axis cs:383838,664947.334855702)
--(axis cs:393939,674823.946290213)
--(axis cs:404040,684786.645005142)
--(axis cs:414141,694647.889265599)
--(axis cs:424242,704263.090906261)
--(axis cs:434343,713832.729406291)
--(axis cs:444444,723392.147376072)
--(axis cs:454545,732673.92144181)
--(axis cs:464646,741416.269812072)
--(axis cs:474747,750077.453445963)
--(axis cs:484848,758677.047538008)
--(axis cs:494949,767249.803505187)
--(axis cs:505050,775638.476927713)
--(axis cs:515151,783812.506383865)
--(axis cs:525252,792083.482583401)
--(axis cs:535353,800232.883710795)
--(axis cs:545454,808065.023614984)
--(axis cs:555555,815836.102911188)
--(axis cs:565656,823458.806594811)
--(axis cs:575757,831090.673053085)
--(axis cs:585858,838481.831290582)
--(axis cs:595959,845614.409327982)
--(axis cs:606060,852549.305025069)
--(axis cs:616161,859528.517304226)
--(axis cs:626262,866447.329266009)
--(axis cs:636363,873165.606116949)
--(axis cs:646464,879389.408794544)
--(axis cs:656565,885493.587406275)
--(axis cs:666666,891256.282273809)
--(axis cs:676767,897041.98516391)
--(axis cs:686868,902997.165502458)
--(axis cs:696969,908865.286439603)
--(axis cs:707070,914761.374908857)
--(axis cs:717171,920607.160127878)
--(axis cs:727272,926469.084943203)
--(axis cs:737373,932270.512200657)
--(axis cs:747474,938015.77828674)
--(axis cs:757575,943615.126985037)
--(axis cs:767676,949128.746508144)
--(axis cs:777777,954600.347747748)
--(axis cs:787878,959867.267709372)
--(axis cs:797979,965157.516578619)
--(axis cs:808080,970397.781956189)
--(axis cs:818181,975606.650720942)
--(axis cs:828282,980423.947096008)
--(axis cs:838383,985149.20869339)
--(axis cs:848484,989847.084260555)
--(axis cs:858585,994488.24676966)
--(axis cs:868686,999079.654925865)
--(axis cs:878787,1003681.12832836)
--(axis cs:888888,1008308.64194329)
--(axis cs:898989,1012916.47921334)
--(axis cs:909090,1017518.95407433)
--(axis cs:919191,1021741.3992511)
--(axis cs:929292,1025742.06283907)
--(axis cs:939393,1029696.87288905)
--(axis cs:949494,1033643.8635532)
--(axis cs:959595,1037574.24014107)
--(axis cs:969696,1041455.33993931)
--(axis cs:979797,1045313.31967146)
--(axis cs:989898,1049129.81127207)
--(axis cs:999999,1052873.92442964)
--(axis cs:999999,1898888.90806713)
--(axis cs:999999,1898888.90806713)
--(axis cs:989898,1890442.68801245)
--(axis cs:979797,1881924.43714285)
--(axis cs:969696,1873344.78714502)
--(axis cs:959595,1864333.40991992)
--(axis cs:949494,1855211.0145198)
--(axis cs:939393,1845893.26091217)
--(axis cs:929292,1836591.84167305)
--(axis cs:919191,1827250.14046673)
--(axis cs:909090,1817506.91479048)
--(axis cs:898989,1807040.36307772)
--(axis cs:888888,1796611.44675775)
--(axis cs:878787,1786217.42125922)
--(axis cs:868686,1775334.29174439)
--(axis cs:858585,1764339.50667626)
--(axis cs:848484,1753310.76660787)
--(axis cs:838383,1742031.40571578)
--(axis cs:828282,1730720.06559519)
--(axis cs:818181,1719368.92713583)
--(axis cs:808080,1707794.84559817)
--(axis cs:797979,1696229.37551011)
--(axis cs:787878,1684615.96940123)
--(axis cs:777777,1673051.7452052)
--(axis cs:767676,1661380.82103605)
--(axis cs:757575,1649688.29499434)
--(axis cs:747474,1637827.1801999)
--(axis cs:737373,1625743.8193241)
--(axis cs:727272,1613360.40006508)
--(axis cs:717171,1600350.71463817)
--(axis cs:707070,1587294.94253766)
--(axis cs:696969,1574214.67369657)
--(axis cs:686868,1561150.53632575)
--(axis cs:676767,1548043.16688584)
--(axis cs:666666,1534807.85733283)
--(axis cs:656565,1521197.11115422)
--(axis cs:646464,1507704.0995969)
--(axis cs:636363,1494189.13487902)
--(axis cs:626262,1480402.56038491)
--(axis cs:616161,1466558.82852456)
--(axis cs:606060,1452707.02596143)
--(axis cs:595959,1438739.02314791)
--(axis cs:585858,1424510.97372176)
--(axis cs:575757,1410145.08550882)
--(axis cs:565656,1395348.75221079)
--(axis cs:555555,1380169.02129325)
--(axis cs:545454,1364825.42637505)
--(axis cs:535353,1349438.04627593)
--(axis cs:525252,1333746.40999206)
--(axis cs:515151,1317922.82511879)
--(axis cs:505050,1302088.76438853)
--(axis cs:494949,1286191.78811496)
--(axis cs:484848,1269917.37313733)
--(axis cs:474747,1253351.25899476)
--(axis cs:464646,1236773.91971328)
--(axis cs:454545,1220135.94165396)
--(axis cs:444444,1202975.64598989)
--(axis cs:434343,1185157.51466753)
--(axis cs:424242,1166633.34290321)
--(axis cs:414141,1148196.78721917)
--(axis cs:404040,1129604.97903328)
--(axis cs:393939,1110862.14489828)
--(axis cs:383838,1092046.4836159)
--(axis cs:373737,1072405.60130299)
--(axis cs:363636,1052584.49370509)
--(axis cs:353535,1032320.26988724)
--(axis cs:343434,1011787.58890017)
--(axis cs:333333,991182.378082659)
--(axis cs:323232,970601.858451568)
--(axis cs:313131,949929.469776061)
--(axis cs:303030,929168.226847058)
--(axis cs:292929,907822.220291676)
--(axis cs:282828,885444.745693248)
--(axis cs:272727,862082.344192623)
--(axis cs:262626,838053.04925928)
--(axis cs:252525,813485.882090266)
--(axis cs:242424,788421.148900155)
--(axis cs:232323,762770.881242309)
--(axis cs:222222,736420.033066758)
--(axis cs:212121,709701.376946994)
--(axis cs:202020,682783.104280634)
--(axis cs:191919,655533.402844153)
--(axis cs:181818,627363.828383224)
--(axis cs:171717,598049.969776233)
--(axis cs:161616,568390.20798676)
--(axis cs:151515,538174.78403503)
--(axis cs:141414,507616.182563566)
--(axis cs:131313,476456.718458234)
--(axis cs:121212,444970.707298762)
--(axis cs:111111,412100.689087995)
--(axis cs:101010,378515.586248866)
--(axis cs:90909,344584.342956473)
--(axis cs:80808,310118.276544994)
--(axis cs:70707,274308.87204093)
--(axis cs:60606,237615.533876682)
--(axis cs:50505,200490.621715333)
--(axis cs:40404,162540.466150481)
--(axis cs:30303,123123.508488148)
--(axis cs:20202,83202.3611751644)
--(axis cs:10101,42137.1879914179)
--(axis cs:0,4.4218025748378)
--cycle;

\path [fill=mediumpurple148103189, fill opacity=0.3]
(axis cs:0,3.99729752944985)
--(axis cs:0,1.80284616226846)
--(axis cs:10101,22947.0297395378)
--(axis cs:20202,46023.8678686192)
--(axis cs:30303,68677.0010204259)
--(axis cs:40404,91269.6030418348)
--(axis cs:50505,113769.106096968)
--(axis cs:60606,136394.755546272)
--(axis cs:70707,159230.684575185)
--(axis cs:80808,182134.165564521)
--(axis cs:90909,205032.023563839)
--(axis cs:101010,227813.127165455)
--(axis cs:111111,251142.630075851)
--(axis cs:121212,273921.401293382)
--(axis cs:131313,295943.685924206)
--(axis cs:141414,319050.97787122)
--(axis cs:151515,342751.094306396)
--(axis cs:161616,365954.458413876)
--(axis cs:171717,388430.208262011)
--(axis cs:181818,410751.742339579)
--(axis cs:191919,434545.38088989)
--(axis cs:202020,458888.082071172)
--(axis cs:212121,483123.815380851)
--(axis cs:222222,506846.957524279)
--(axis cs:232323,529000.332197627)
--(axis cs:242424,550148.691070484)
--(axis cs:252525,571757.388410278)
--(axis cs:262626,592831.524239981)
--(axis cs:272727,614905.052835207)
--(axis cs:282828,636653.600808853)
--(axis cs:292929,657796.372135916)
--(axis cs:303030,678474.898341041)
--(axis cs:313131,700505.266229078)
--(axis cs:323232,722649.067620993)
--(axis cs:333333,744887.558760198)
--(axis cs:343434,767524.13607421)
--(axis cs:353535,789578.757519272)
--(axis cs:363636,809884.274462932)
--(axis cs:373737,829801.997268993)
--(axis cs:383838,850969.479454525)
--(axis cs:393939,871667.383601339)
--(axis cs:404040,891464.261719817)
--(axis cs:414141,909394.41256047)
--(axis cs:424242,926859.771639536)
--(axis cs:434343,945867.289202513)
--(axis cs:444444,964682.769071319)
--(axis cs:454545,982894.681878673)
--(axis cs:464646,1001056.39331947)
--(axis cs:474747,1019155.97480852)
--(axis cs:484848,1037464.50586863)
--(axis cs:494949,1055661.83350095)
--(axis cs:505050,1073977.20187917)
--(axis cs:515151,1092106.99236131)
--(axis cs:525252,1110178.91572482)
--(axis cs:535353,1128338.52044548)
--(axis cs:545454,1148094.25929845)
--(axis cs:555555,1168173.8008572)
--(axis cs:565656,1187191.46332725)
--(axis cs:575757,1205658.03633972)
--(axis cs:585858,1224653.02936516)
--(axis cs:595959,1242660.34646778)
--(axis cs:606060,1260109.8696782)
--(axis cs:616161,1277195.23825551)
--(axis cs:626262,1293900.96740917)
--(axis cs:636363,1310778.47038874)
--(axis cs:646464,1326889.51809464)
--(axis cs:656565,1341281.80412008)
--(axis cs:666666,1355297.13386389)
--(axis cs:676767,1369029.83730475)
--(axis cs:686868,1382788.34224631)
--(axis cs:696969,1396299.03418438)
--(axis cs:707070,1410289.45252965)
--(axis cs:717171,1424643.00693289)
--(axis cs:727272,1438904.71761775)
--(axis cs:737373,1452287.46882302)
--(axis cs:747474,1465192.68398663)
--(axis cs:757575,1478668.54658265)
--(axis cs:767676,1492148.83364834)
--(axis cs:777777,1505594.19748039)
--(axis cs:787878,1518843.21087594)
--(axis cs:797979,1531728.34179551)
--(axis cs:808080,1545043.0767684)
--(axis cs:818181,1558372.65445415)
--(axis cs:828282,1571730.63278438)
--(axis cs:838383,1585104.99007369)
--(axis cs:848484,1598487.39692149)
--(axis cs:858585,1611937.4965529)
--(axis cs:868686,1625609.9578535)
--(axis cs:878787,1639955.2215852)
--(axis cs:888888,1654831.06806879)
--(axis cs:898989,1669753.39495424)
--(axis cs:909090,1684521.12284635)
--(axis cs:919191,1699071.15520229)
--(axis cs:929292,1714506.98714764)
--(axis cs:939393,1730061.13870909)
--(axis cs:949494,1745162.16991396)
--(axis cs:959595,1759813.15297144)
--(axis cs:969696,1774458.03576828)
--(axis cs:979797,1789081.23835999)
--(axis cs:989898,1802702.05434332)
--(axis cs:999999,1816128.455589)
--(axis cs:999999,2958697.96655953)
--(axis cs:999999,2958697.96655953)
--(axis cs:989898,2937098.01700393)
--(axis cs:979797,2915397.71444535)
--(axis cs:969696,2893554.87263591)
--(axis cs:959595,2871659.48281824)
--(axis cs:949494,2849785.77243388)
--(axis cs:939393,2827825.89232659)
--(axis cs:929292,2805785.43092843)
--(axis cs:919191,2783533.67858138)
--(axis cs:909090,2760984.65632486)
--(axis cs:898989,2738683.11588148)
--(axis cs:888888,2716248.40078687)
--(axis cs:878787,2693844.8571059)
--(axis cs:868686,2671778.98553957)
--(axis cs:858585,2650072.22158653)
--(axis cs:848484,2628501.66509069)
--(axis cs:838383,2606964.81197843)
--(axis cs:828282,2585366.08041906)
--(axis cs:818181,2563068.42766742)
--(axis cs:808080,2540256.2187487)
--(axis cs:797979,2517571.04138388)
--(axis cs:787878,2495173.9317642)
--(axis cs:777777,2472260.15946933)
--(axis cs:767676,2449089.41272306)
--(axis cs:757575,2425899.97309029)
--(axis cs:747474,2402683.06452127)
--(axis cs:737373,2379582.81383537)
--(axis cs:727272,2356149.25681809)
--(axis cs:717171,2332285.98559422)
--(axis cs:707070,2307930.65977808)
--(axis cs:696969,2283728.40553196)
--(axis cs:686868,2260109.34352336)
--(axis cs:676767,2236343.67651772)
--(axis cs:666666,2212612.66219694)
--(axis cs:656565,2188468.07619883)
--(axis cs:646464,2164067.50045799)
--(axis cs:636363,2138968.0251443)
--(axis cs:626262,2111047.25090085)
--(axis cs:616161,2083015.78768801)
--(axis cs:606060,2054384.61649303)
--(axis cs:595959,2025485.98254785)
--(axis cs:585858,1996665.51040772)
--(axis cs:575757,1968004.82605065)
--(axis cs:565656,1940289.73426354)
--(axis cs:555555,1912570.97561214)
--(axis cs:545454,1884651.78203154)
--(axis cs:535353,1856882.44321957)
--(axis cs:525252,1830102.25156556)
--(axis cs:515151,1802647.08966364)
--(axis cs:505050,1775160.03246787)
--(axis cs:494949,1747583.08377875)
--(axis cs:484848,1720056.61521658)
--(axis cs:474747,1692047.85680343)
--(axis cs:464646,1663494.66871895)
--(axis cs:454545,1634709.25707386)
--(axis cs:444444,1605905.63967303)
--(axis cs:434343,1576304.96219863)
--(axis cs:424242,1546686.40990432)
--(axis cs:414141,1517748.51070632)
--(axis cs:404040,1488504.98297271)
--(axis cs:393939,1458303.25127983)
--(axis cs:383838,1427862.1724351)
--(axis cs:373737,1397347.0606573)
--(axis cs:363636,1365860.24407604)
--(axis cs:353535,1332959.62956943)
--(axis cs:343434,1298450.43137587)
--(axis cs:333333,1263562.95786807)
--(axis cs:323232,1229309.99824125)
--(axis cs:313131,1195179.81804824)
--(axis cs:303030,1161720.18521155)
--(axis cs:292929,1127699.30579825)
--(axis cs:282828,1093555.56732817)
--(axis cs:272727,1059184.75655501)
--(axis cs:262626,1023987.66919839)
--(axis cs:252525,989349.680067214)
--(axis cs:242424,954169.735541943)
--(axis cs:232323,918367.972426834)
--(axis cs:222222,881877.496307911)
--(axis cs:212121,842764.212181199)
--(axis cs:202020,803950.344398864)
--(axis cs:191919,764829.411047643)
--(axis cs:181818,726517.840811083)
--(axis cs:171717,689658.271189464)
--(axis cs:161616,652043.068133715)
--(axis cs:151515,613138.916991429)
--(axis cs:141414,572998.20425773)
--(axis cs:131313,533522.265757716)
--(axis cs:121212,493808.276802233)
--(axis cs:111111,453926.214514173)
--(axis cs:101010,413440.98317869)
--(axis cs:90909,373327.765958034)
--(axis cs:80808,333404.795198218)
--(axis cs:70707,292564.664063803)
--(axis cs:60606,250862.403858145)
--(axis cs:50505,209904.323748474)
--(axis cs:40404,169347.458169167)
--(axis cs:30303,127935.587106889)
--(axis cs:20202,85655.5417551661)
--(axis cs:10101,42904.7542056506)
--(axis cs:0,3.99729752944985)
--cycle;

\addplot [very thick, steelblue31119180]
table {%
0 3.20681074563853
10101 22548.8063141624
20202 29855.0631638281
30303 33782.2286041422
40404 36809.5175492534
50505 38944.6353936381
60606 40453.019693112
70707 41450.2475850764
80808 42170.372078537
90909 42781.5852627812
101010 43304.5121077054
111111 43793.9833256531
121212 44213.8658200001
131313 44604.0286566893
141414 44986.3164877415
151515 45327.7094042772
161616 45646.7856863249
171717 45946.6810399331
181818 46253.8367993733
191919 46509.1780272628
202020 46746.7357716511
212121 46997.2956377465
222222 47245.9961234193
232323 47451.3331136998
242424 47677.5161333129
252525 47878.6127260239
262626 48090.907038839
272727 48288.2670043655
282828 48487.6938768226
292929 48683.6427992309
303030 48840.1936943717
313131 49005.1082237985
323232 49167.2631290778
333333 49346.8597336355
343434 49518.1741426096
353535 49697.9989106241
363636 49864.4920155188
373737 50005.3168444607
383838 50155.9908585478
393939 50324.3251287198
404040 50459.3456720819
414141 50565.5110727139
424242 50691.6688854768
434343 50843.3974647808
444444 50985.5132480595
454545 51093.2917980724
464646 51191.5566037883
474747 51306.4938118676
484848 51413.5510385415
494949 51522.7181149708
505050 51613.2317823587
515151 51717.9634597227
525252 51812.3981785418
535353 51946.8158980498
545454 52049.1146322487
555555 52146.915028533
565656 52264.4107113458
575757 52364.0385378204
585858 52453.3278587459
595959 52548.1580815203
606060 52671.2252383897
616161 52753.5984393596
626262 52841.9456466707
636363 52914.6806605683
646464 53002.9911393695
656565 53082.2155518164
666666 53171.8031494876
676767 53271.7882120992
686868 53356.0875148704
696969 53444.2599661496
707070 53521.5178826699
717171 53597.9541488749
727272 53665.4714959297
737373 53745.3505041952
747474 53814.9005298311
757575 53886.4624262692
767676 53967.2942170295
777777 54025.1307083887
787878 54083.1016049637
797979 54145.5980645477
808080 54206.4238951985
818181 54243.3927290176
828282 54283.9840502811
838383 54340.8497286466
848484 54388.0495823596
858585 54464.4703332897
868686 54544.1827824774
878787 54588.5652181302
888888 54649.8523006316
898989 54723.1361347213
909090 54790.5727383674
919191 54855.0638464423
929292 54924.4477483664
939393 54977.111513006
949494 55050.0573653532
959595 55114.7203523739
969696 55156.461210478
979797 55239.5701784436
989898 55283.9988542914
999999 55345.7715347611
};
\addlegendentry{\ucbvibf}
\addplot [very thick, darkorange25512714]
table {%
0 3.47623089584064
10101 14929.6672803231
20202 18348.2479975578
30303 20158.1911591578
40404 20995.8213674497
50505 21566.9865640005
60606 22036.0576746499
70707 22405.7266625538
80808 22755.0202258297
90909 23100.8897590908
101010 23385.8866523384
111111 23637.3808196189
121212 23848.4977536458
131313 24051.6343743845
141414 24239.1705567605
151515 24416.3320201278
161616 24568.9920266175
171717 24745.5286537523
181818 24917.8008271737
191919 25083.6771259984
202020 25212.8958416816
212121 25341.3419919606
222222 25482.3311300178
232323 25587.6888149256
242424 25673.5564153801
252525 25784.491540291
262626 25870.8134004193
272727 25970.7214594562
282828 26081.4525998852
292929 26168.0693375047
303030 26265.6437398238
313131 26350.588735739
323232 26438.1335742342
333333 26523.5308768217
343434 26593.4937533469
353535 26687.3919913707
363636 26773.5250617251
373737 26861.2291675021
383838 26943.9399764503
393939 27010.9014206144
404040 27082.9928107718
414141 27184.4442011316
424242 27263.9294873291
434343 27322.5297617416
444444 27397.8864873568
454545 27469.1022038567
464646 27535.9203289931
474747 27569.4152967074
484848 27624.4831779545
494949 27673.3264944388
505050 27727.108136603
515151 27785.0785415047
525252 27843.0892152784
535353 27899.9927266673
545454 27957.2984902441
555555 28007.5946079279
565656 28061.5791825751
575757 28092.4370000277
585858 28158.4047263055
595959 28225.8321236573
606060 28273.3176352187
616161 28323.0550858086
626262 28370.2158719676
636363 28410.0915660902
646464 28458.0581723427
656565 28471.3469735481
666666 28518.0888489759
676767 28558.3801486403
686868 28628.1472127124
696969 28691.7476415583
707070 28711.7409171341
717171 28758.9721153023
727272 28781.5299797453
737373 28841.5573875431
747474 28879.9082539413
757575 28939.4031348353
767676 28971.3064262055
777777 28998.6895434607
787878 29026.3337152336
797979 29084.9036311858
808080 29114.3213238492
818181 29162.5132707817
828282 29193.5880010333
838383 29236.0254160095
848484 29274.5605847489
858585 29294.9828608
868686 29311.959683064
878787 29354.3141746702
888888 29376.5760818605
898989 29431.6196161108
909090 29470.9782704917
919191 29519.8939059751
929292 29562.50973875
939393 29592.8556260397
949494 29652.7276803554
959595 29683.0658127468
969696 29710.9477911745
979797 29757.3701994792
989898 29789.4906356125
999999 29830.8303394307
};
\addlegendentry{\ucbvibfi}
\addplot [very thick, forestgreen4416044]
table {%
0 3.40247170510202
10101 32442.5462048524
20202 64354.7452195004
30303 95874.3082937774
40404 126891.804007508
50505 156960.178002274
60606 186798.217060835
70707 216244.762644484
80808 244905.133752757
90909 272837.200755969
101010 300538.303175606
111111 327901.087684903
121212 354593.684436605
131313 380315.943218264
141414 405934.622602778
151515 431103.682653122
161616 456065.19326855
171717 480496.886861418
181818 504530.845256319
191919 527798.008340787
202020 550596.102946677
212121 573032.657196999
222222 595284.873781689
232323 617127.327340893
242424 638355.926427383
252525 658851.808833411
262626 678522.94745102
272727 697648.935693113
282828 716048.217051957
292929 733874.921353338
303030 750874.915964182
313131 767410.941117331
323232 783840.920512605
333333 800198.762981524
343434 816440.968167748
353535 832449.336166189
363636 848095.505651612
373737 863406.632049891
383838 878496.909235802
393939 892843.045594246
404040 907195.812019211
414141 921422.338242387
424242 935448.216904736
434343 949495.122036911
444444 963183.896682982
454545 976404.931547887
464646 989095.094762678
474747 1001714.35622036
484848 1014297.21033767
494949 1026720.79581007
505050 1038863.62065812
515151 1050867.66575133
525252 1062914.94628773
535353 1074835.46499336
545454 1086445.22499502
555555 1098002.56210222
565656 1109403.7794028
575757 1120617.87928095
585858 1131496.40250617
595959 1142176.71623794
606060 1152628.16549325
616161 1163043.67291439
626262 1173424.94482546
636363 1183677.37049798
646464 1193546.75419572
656565 1203345.34928025
666666 1213032.06980332
676767 1222542.57602488
686868 1232073.8509141
696969 1241539.98006809
707070 1251028.15872326
717171 1260478.93738303
727272 1269914.74250414
737373 1279007.16576238
747474 1287921.47924332
757575 1296651.71098969
767676 1305254.7837721
777777 1313826.04647648
787878 1322241.6185553
797979 1330693.44604436
808080 1339096.31377718
818181 1347487.78892838
828282 1355572.0063456
838383 1363590.30720458
848484 1371578.92543421
858585 1379413.87672296
868686 1387206.97333513
878787 1394949.27479379
888888 1402460.04435052
898989 1409978.42114553
909090 1417512.9344324
919191 1424495.76985891
929292 1431166.95225606
939393 1437795.06690061
949494 1444427.4390365
959595 1450953.82503049
969696 1457400.06354217
979797 1463618.87840716
989898 1469786.24964226
999999 1475881.41624838
};
\addlegendentry{\ucbvich}
\addplot [very thick, mediumpurple148103189]
table {%
0 2.90007184585916
10101 32925.8919725942
20202 65839.7048118926
30303 98306.2940636574
40404 130308.530605501
50505 161836.714922721
60606 193628.579702208
70707 225897.674319494
80808 257769.48038137
90909 289179.894760936
101010 320627.055172073
111111 352534.422295012
121212 383864.839047808
131313 414732.975840961
141414 446024.591064475
151515 477945.005648912
161616 508998.763273796
171717 539044.239725738
181818 568634.791575331
191919 599687.395968767
202020 631419.213235018
212121 662944.013781025
222222 694362.226916095
232323 723684.152312231
242424 752159.213306213
252525 780553.534238746
262626 808409.596719186
272727 837044.904695109
282828 865104.584068513
292929 892747.838967085
303030 920097.541776295
313131 947842.542138658
323232 975979.532931122
333333 1004225.25831413
343434 1032987.28372504
353535 1061269.19354435
363636 1087872.25926948
373737 1113574.52896315
383838 1139415.82594481
393939 1164985.31744058
404040 1189984.62234627
414141 1213571.4616334
424242 1236773.09077193
434343 1261086.12570057
444444 1285294.20437218
454545 1308801.96947627
464646 1332275.53101921
474747 1355601.91580598
484848 1378760.56054261
494949 1401622.45863985
505050 1424568.61717352
515151 1447377.04101247
525252 1470140.58364519
535353 1492610.48183252
545454 1516373.020665
555555 1540372.38823467
565656 1563740.5987954
575757 1586831.43119519
585858 1610659.26988644
595959 1634073.16450782
606060 1657247.24308561
616161 1680105.51297176
626262 1702474.10915501
636363 1724873.24776652
646464 1745478.50927631
656565 1764874.94015946
666666 1783954.89803042
676767 1802686.75691123
686868 1821448.84288483
696969 1840013.71985817
707070 1859110.05615387
717171 1878464.49626356
727272 1897526.98721792
737373 1915935.14132919
747474 1933937.87425395
757575 1952284.25983647
767676 1970619.1231857
777777 1988927.17847486
787878 2007008.57132007
797979 2024649.69158969
808080 2042649.64775855
818181 2060720.54106078
828282 2078548.35660172
838383 2096034.90102606
848484 2113494.53100609
858585 2131004.85906971
868686 2148694.47169654
878787 2166900.03934555
888888 2185539.73442783
898989 2204218.25541786
909090 2222752.88958561
919191 2241302.41689183
929292 2260146.20903804
939393 2278943.51551784
949494 2297473.97117392
959595 2315736.31789484
969696 2334006.45420209
979797 2352239.47640267
989898 2369900.03567362
999999 2387413.21107426
};
\addlegendentry{\mvp}
\end{axis}

\end{tikzpicture}
}
% \caption{$H=10, \ K=10^6$.}
% \end{subfigure}
% %\hspace{1cm}
% \caption{Performances in terms of cumulative regret in toy environments with $S=3$ states and $A=3$ actions ($10$ runs, mean $\pm$ $95\%$ C.I.).}
% \label{fig:TODO2}
% \end{figure}

\paragraph{Results.}
Figure~\ref{fig:TODO2} represents the cumulative regret of the evaluated algorithms in the first experimental evaluation for different values of $H$ and $K$. From the figures, we can observe that \ucbvi with the Chernoff-Hoeffding bonus and \mvp begin to show a sub-linear regret for $K=10^6$, whereas both versions of \ucbvi with the Bernstein-Freedman bonus greatly outperform the other algorithms in all the evaluated scenarios. In particular, the use of a tighter Bernstein-Freedman bonus (\ucbvibfi) translates into a cumulative regret that is, although of the same order, lower than with the usage of a larger bonus (\ucbvibf), highlighting the importance of lower order terms and constants in empirical performance.

\subsection{RiverSwim}
We now consider the RiverSwim environment~\citep{strehl2008analysis}. This environment emulates a swimmer that has to swim against the current, where the agent has 2 options: \emph{(i)} to try to swim to the other side or \emph{(ii)} to turn back. In this scenario, the rewards and the transition probabilities are designed such that the optimal policy corresponds to trying to swim and reach the other side of the \quotes{river}. This is considered a challenging benchmark for exploration. We consider the scenario with $S=5$ and $H=10$. The reward model and the transition probability are designed such that the suboptimality gap between the optimal action and the other one in the initial state is very low ($\sim\!0.1$, with a scale of the problem in the order of $H=10$).

\paragraph{Results.}
Figure~\ref{fig:river} compares the results when using \mvp and \ucbvi in its original version (\ucbvibf) and the one we propose with tighter bounds (\ucbvibfi). \mvp confirms its poor empirical performance, failing to deliver a sublinear trend for the considered horizon. Instead, \ucbvi, in both versions, shows a clear sublinear trend, with the improved version (\ucbvibfi) with a cumulative regret approximately half of the original one (\ucbvibf).

\begin{figure}[t!]
\centering
\resizebox{0.37\linewidth}{!}{\includegraphics{img/riverswim_s5h10_standalone.pdf}}
\caption{Performances in terms of cumulative regret in the RiverSwim environment with $S=5$ states and horizon $H=10$ ($4$ runs, mean $\pm$ $95\%$ C.I.).}
\label{fig:river}
\end{figure}

\begin{table}[t!]
    \centering
    \begin{tabular}{c|ccc}
        \hline 
        & Bonus ratio & Regret upper bound ratio &  Empirical regret ratio\\ \hline
        \texttt{CH} & $7/2$ & $10$ & - \\ \hline
        \texttt{BF} & $\sqrt{2}$ & $5/4$ & $1.87 \pm 0.03$ \\ \hline
    \end{tabular}
    \caption{Improvement ratios in the bonuses, regret upper bounds, and empirical regret between our analysis and the original of~\citep{azar2017minimax}.}
    \label{tab:ratios}
\end{table}

