\section{Technical Lemmas}
\label{apx:lemmas}

\begin{lemma}[Bernstein inequality for Bernoulli random variables]
\label{lem:bernstein_bernoulli}
    Let $p$ be the parameter of a Bernoulli random variable, and let $\widehat{p}$ be its estimator. Let $\delta > 0$. Then, \wp at least $1-\delta$, it holds that:
    
    \begin{equation*}
        |\widehat{p} - p| \le \sqrt{\frac{2 p (1-p) L}{n}} + \frac{2 L}{3 n},
    \end{equation*}

    where $n$ represents the number of observations, and $L=\ln(2/\delta)$.
\end{lemma}

\begin{proof}
    Let $\{Y_i\}_{i=1 \ldots, n}$ be the set of i.i.d. realizations of a Bernoulli with parameter $p$. Define the auxiliary random variable:

    \begin{equation*}
        X_i = \frac{Y_i}{n}.
    \end{equation*}

    Observe that $X_1, \ldots, X_n$ are independent random variables, and that $0 \le X_i \le 1/n$. Let $S_n$ be their sum, and $E_n$ be the expected value of $S_n$:

    \begin{align}
        S_n &= \sum_{i=1}^n X_i = \widehat{p}, \nonumber \\
        E_n &= \mathbb{E}[S_n] = \sum_{i=1}^n \mathbb{E}[X_i] = p. \nonumber
    \end{align}

    Let $V_n$ be the variance of $S_n$:

    \begin{equation*}
        V_n = \Var [S_n] = \sum_{i=1}^n \Var [X_i] = \sum_{i=1}^n \left( \frac{1(1-p)}{n^2} \right) = \frac{p(1-p)}{n}.
    \end{equation*}

    By applying Bernstein's inequality, we obtain that:

    \begin{equation}
    \label{lem:bernstein_bernoulli:ineq}
        \mathrm{Pr}(|S_n - E_n| > \epsilon) < 2 \exp\left(- \frac{\epsilon^2 / 2}{V_n + C(\epsilon/3)}\right),
    \end{equation}

    where $C$ is the range of values of the addends in $S_n$ (\ie $C=1/n$). By setting this probability to be equal to $\delta$, we can derive that:

    \begin{equation*}
        \frac{\epsilon^2}{2} = V_n \ln\left(\frac{2}{\delta}\right) + \frac{\epsilon}{3n} \ln \left(\frac{2}{\delta}\right).
    \end{equation*}

    Let $L= \ln(2/\delta)$, by solving the second order polynomial we get that:

    \begin{equation*}
        \epsilon = \frac{L}{3n} \pm \sqrt{\frac{L^2}{9n^2} + 2 V_n L}.
    \end{equation*}

    We can discard the equation with the minus, as it would result in $\epsilon < 0$, thus resulting in the inequality in Equation~\eqref{lem:bernstein_bernoulli:ineq} holding \wp $1 - \delta$. As such, we derive that:

    \begin{align*}
        \epsilon &= \frac{L}{3n} + \sqrt{\frac{L^2}{9n^2} + 2 V_n L} \\
        &\le \frac{L}{3n} + \sqrt{\frac{L^2}{9n^2}} + \sqrt{\frac{2p(1-p)L}{n}} \\
        &= \sqrt{\frac{2p(1-p)L}{n}} + \frac{2L}{3n},
    \end{align*}

    thus completing the proof.
\end{proof}

%---------------------------------

\begin{lemma}[Regret decomposition upper bound]
\label{lem:regr_dec}
    Let $k \in \dsb{K}$ and $h \in \dsb{H}$. Assume events $\mathcal{E}$ and $\Omega_{k,h}$ hold. Then the regret from stage $h$ onward of all episodes up to $k$ can be upper bounded as follows:

    \begin{align*}
        \sum_{i=1}^k \Vdiff{i,h}(x_{i,h}) \le \sum_{i=1}^k \Vdifftil{i,h}(x_{i,h}) \le & \ e \sum_{i=1}^k \sum_{j=h}^{H-1} \Big[ \varepsilon_{i,j} + 2 \sqrt{L} \overline{\varepsilon}_{i,j} + \bonus_{i,j}(x_{i,j}, a_{i,j}^{\pi_i}) \\
        &\qquad\qquad\quad + \xi_{i,j}(x_{i,j}, a_{i,j}^{\pi_i}) + \frac{8H^2SL}{3 N_i (x_{i,j}, a_{i,j}^{\pi_i})} \Big].
    \end{align*}
\end{lemma}

\begin{proof}
We begin the proof by considering a single value of $k \in \dsb{K}$. Under $\Omega_{k,h}$, we observe that:
\begin{align}
    \Vdiff{k,h}(x_{k,h}) &= \Vstar{h}(x_{k,h}) - \Vpi{h}(x_{k,h}) \nonumber \\
    &\le \Vest{k,h}(x_{k,h}) - \Vpi{h}(x_{k,h}) \nonumber \\
    &= \Vdifftil{k,h}(x_{k,h}). \nonumber
\end{align}

As such, we bound the pseudo-regret $\Vdifftil{k,h}(x_{k,h})$:

\begin{align}
    \Vdifftil{k,h}(x_{k,h}) &= \Vest{k,h}(x_{k,h}) - \Vpi{h}(x_{k,h}) \nonumber \\
    &= \bonus_{k,h}(x_{k,h}, a_{k,h}^{\pi_k}) + \sum_{y \in \Ss} \Pest_k (y | x_{k,h}, a_{k,h}^{\pi_k}) \Vest{k,h+1}(y) - \sum_{y \in \Ss} P(y | x_{k,h}, a_{k,h}^{\pi_k}) \Vpi{h+1}(y) \nonumber \\
    &= \bonus_{k,h}(x_{k,h}, a_{k,h}^{\pi_k}) + \sum_{y \in \Ss} \left[ \Pest_k (y | x_{k,h}, a_{k,h}^{\pi_k}) - P(y | x_{k,h}, a_{k,h}^{\pi_k}) \right] \Vest{k,h+1}(y) \nonumber \\
    &\quad + \sum_{y \in \Ss} P(y | x_{k,h}, a_{k,h}^{\pi_k}) \left[ \Vest{k,h+1}(y) - \Vpi{h+1}(y) \right] \nonumber \\
    &= \bonus_{k,h}(x_{k,h}, a_{k,h}^{\pi_k}) + \sum_{y \in \Ss} \left[ \Pest_k (y | x_{k,h}, a_{k,h}^{\pi_k}) - P(y | x_{k,h}, a_{k,h}^{\pi_k}) \right] \Vstar{h+1} (y) \nonumber \\
    &\quad + \sum_{y \in \Ss} \left[ \Pest_k (y | x_{k,h}, a_{k,h}^{\pi_k}) - P(y | x_{k,h}, a_{k,h}^{\pi_k}) \right] \left[ \Vest{k, h+1}(y) - \Vstar{h+1}(y) \right] \nonumber \\
    &\quad + \sum_{y \in \Ss} P(y | x_{k,h}, a_{k,h}^{\pi_k}) \Vdifftil{k, h+1}(y) \nonumber \\
    &= \Vdifftil{k,h+1}(x_{k,h+1}) + \bonus_{k,h}(x_{k,h}, a_{k,h}^{\pi_k}) + \varepsilon_{k,h} \label{lem:regr_dec:1} \\
    &\quad + \sum_{y \in \Ss} \left[ \Pest_k (y | x_{k,h}, a_{k,h}^{\pi_k}) - P(y | x_{k,h}, a_{k,h}^{\pi_k}) \right] \Vstar{h+1} (y) \nonumber \\
    &\quad + \sum_{y \in \Ss} \left[ \Pest_k (y | x_{k,h}, a_{k,h}^{\pi_k}) - P(y | x_{k,h}, a_{k,h}^{\pi_k}) \right] \left[ \Vest{k, h+1}(y) - \Vstar{h+1}(y) \right] \nonumber \\
    \begin{split} \label{lem:regr_dec:2}
    &\le \Vdifftil{k,h+1}(x_{k,h+1}) + \bonus_{k,h}(x_{k,h}, a_{k,h}^{\pi_k}) + 
    % \del{c} \left( \VVopt{h+1}(x_{k,h}, a_{k,h}^{\pi_k}), N_k(x_{k,h}, a_{k,h}^{\pi_k}) \right)
    \xi_{k,h}(x_{k,h}, a_{k,h}^{\pi_k}) \\
    &\quad + \varepsilon_{k,h} + \underbrace{\sum_{y \in \Ss} \left[ \Pest_k (y | x_{k,h}, a_{k,h}^{\pi_k}) - P(y | x_{k,h}, a_{k,h}^{\pi_k}) \right] \left[ \Vest{k, h+1}(y) - \Vstar{h+1}(y) \right]}_{(a)},
    \end{split}
\end{align}
where, in Equation~\eqref{lem:regr_dec:1} we apply the definition $\varepsilon_{k,h} \coloneqq P(\cdot | x_{k,h}, a_{k,h}^{\pi_k})^\transpose \Vdifftil{k,h+1}(\cdot) - \Vdifftil{k,h+1}(x_{k,h+1})$, and in Equation~\eqref{lem:regr_dec:2} we apply the definition of $\xi_{k,h}(x_{k,h}, a_{k,h}^{\pi_k})$:

\begin{equation*}
    \xi_{k,h}(x_{k,h}, a_{k,h}^{\pi_k}) = \sum_{y \in \Ss} \left[ \Pest_k (y | x_{k,h}, a_{k,h}^{\pi_k}) - P(y | x_{k,h}, a_{k,h}^{\pi_k}) \right] \Vstar{h+1} (y).
\end{equation*}

Let $\histkh$ be the history of the interactions up to, and including, stage $h$ of episode $k$. Observing that $| \varepsilon_{k,h} | \le H \le + \infty$ and $\mathbb{E}[\varepsilon_{k,h} | \histkh]=0$, we can derive that $\varepsilon_{k,h}$ is a Martingale difference sequence.

We now focus on bounding term $(a)$:

\begin{align}
    (a) &= \sum_{y \in \Ss} \left[ \Pest_k (y | x_{k,h}, a_{k,h}^{\pi_k}) - P(y | x_{k,h}, a_{k,h}^{\pi_k}) \right]\left[ \Vest{k,h+1}(y) - \Vstar{h+1}(y) \right] \nonumber \\
    &\le \sum_{y \in \Ss} \left[ \sqrt{\frac{2 P(y|x_{k,h},a_{k,h}^{\pi_k})(1-P(y|x_{k,h},a_{k,h}^{\pi_k})) L}{N_k(x_{k,h}, a_{k,h}^{\pi_k})}} + \frac{2L}{3 N_k (x_{k,h}, a_{k,h}^{\pi_k})} \right] \Vdifftil{k,h+1}(y) \label{lem:regr_dec:3} \\
    &\le \sum_{y \in \Ss}\sqrt{\frac{2 P(y|x_{k,h},a_{k,h}^{\pi_k})L}{N_k(x_{k,h}, a_{k,h}^{\pi_k})}} \Vdifftil{k,h+1}(y) + \frac{2L}{3N_k(x_{k,h}, a_{k,h}^{\pi_k})} \sum_{y \in \Ss} \Vdifftil{k,h+1}(y) \label{lem:regr_dec:4} \\
    &\le \sqrt{2L} \underbrace{\sum_{y \in \Ss}\sqrt{\frac{P(y|x_{k,h},a_{k,h}^{\pi_k})}{N_k(x_{k,h}, a_{k,h}^{\pi_k})}} \Vdifftil{k,h+1}(y)}_{(b)} + \frac{2SHL}{3N_k(x_{k,h},a_{k,h}^{\pi_k})} \label{lem:regr_dec:5},
\end{align}

where Equation~\eqref{lem:regr_dec:3} is obtained by applying Lemma~\ref{lem:bernstein_bernoulli} to bound $\Pest_k - P$ and by ovserving that $\Vstar{h+1}(y) \ge \Vpi[k]{h+1}(y)$ by definition, Equation~\eqref{lem:regr_dec:4} is obtained by splitting the terms and observing that $1 - P(y|x,a) \le 1$ for every $x,y \in \Ss$ and $a \in \As$, and finally Equation~\eqref{lem:regr_dec:5} is obtained by upper bounding $\Vdifftil{k,h+1}(y)$ with $H$.
To bound term $(b)$, we first need to define the following set of states:

\begin{equation*}
    [y]_{k,h} \coloneqq \{ y \in \Ss : N_k(x_{k,h}, a_{k,h}^{\pi_k}) P(y | x_{k,h}, a_{k,h}^{\pi_k}) \ge 2 H^2L \}.
\end{equation*}

As such, we can rewrite:

\begin{equation}
\label{lem:regr_dec:6}
    (b) = \underbrace{\sum_{y \in [y]_{k,h}}\sqrt{\frac{P(y|x_{k,h},a_{k,h}^{\pi_k})}{N_k(x_{k,h}, a_{k,h}^{\pi_k})}} \Vdifftil{k,h+1}(y)}_{(c)} + \underbrace{\sum_{y \notin [y]_{k,h}}\sqrt{\frac{P(y|x_{k,h},a_{k,h}^{\pi_k})}{N_k(x_{k,h}, a_{k,h}^{\pi_k})}} \Vdifftil{k,h+1}(y)}_{(d)}.
\end{equation}

We now bound term $(c)$ as:

\begin{align}
    (c) &= \sum_{y \in [y]_{k,h}}\sqrt{\frac{P(y|x_{k,h},a_{k,h}^{\pi_k})}{N_k(x_{k,h}, a_{k,h}^{\pi_k})}} \Vdifftil{k,h+1}(y) \nonumber \\
    &= \sum_{y \in [y]_{k,h}} P(y|x_{k,h},a_{k,h}^{\pi_k}) \sqrt{\frac{1}{N_k(x_{k,h},a_{k,h}^{\pi_k})P(y|x_{k,h},a_{k,h}^{\pi_k})}} \Vdifftil{k,h+1}(y) \nonumber \\
    &= \overline{\varepsilon}_{k,h} + \sqrt{\frac{\mathbb{I}(x_{k,h+1}\in[y]_{k,h})}{N_k(x_{k,h},a_{k,h}^{\pi_k})P(x_{k,h+1}|x_{k,h},a_{k,h}^{\pi_k})}}\Vdifftil{k,h+1}(x_{k,h+1}) \label{lem:regr_dec:7} \\
    &\le \overline{\varepsilon}_{k,h} + \sqrt{\frac{1}{2H^2L}}\Vdifftil{k,h+1}(x_{k,h+1}), \label{lem:regr_dec:8}
\end{align}

where Equation~\eqref{lem:regr_dec:7} is obtained by applying the definition of $\overline{\varepsilon}_{k,h}$:

\begin{align*}
    \overline{\varepsilon}_{k,h} &\coloneqq \sum_{y\in\Ss} P(y|x_{k,h},a_{k,h}^{\pi_k}) \sqrt{\frac{\mathbb{I}(y \in [y]_{k,h})}{N_k(x_{k,h},a_{k,h}^{\pi_k})P(y|x_{k,h},a_{k,h}^{\pi_k})}} \Vdifftil{k,h+1}(y) \\
    &\quad - \sqrt{\frac{\mathbb{I}(x_{k,h+1} \in [y]_{k,h})}{N_k(x_{k,h},a_{k,h}^{\pi_k})P(x_{k,h+1}|x_{k,h},a_{k,h}^{\pi_k})}} \Vdifftil{k,h+1}(x_{k,h+1}),
\end{align*}

and Equation~\eqref{lem:regr_dec:8} is obtained by bounding the indicator function with 1, and by applying the definition of $[y]_{k,h}$. With the same reasoning of $\varepsilon_{k,h}$, we can prove that $\overline{\varepsilon}_{k,h}$ is also a Martingale difference sequence.

We can now bound term $(d)$ as follows:

\begin{align}
    (d) &= \sum_{y \notin [y]_{k,h}} \sqrt{\frac{P(y|x_{k,h},a_{k,h}^{\pi_k})}{N_k (x_{k,h},a_{k,h}^{\pi_k})}} \Vdifftil{k,h+1}(y) \nonumber \\
    &= \sum_{y \notin [y]_{k,h}} \sqrt{\frac{N_k (x_{k,h},a_{k,h}^{\pi_k}) P(y|x_{k,h},a_{k,h}^{\pi_k})}{(N_k (x_{k,h},a_{k,h}^{\pi_k}))^2}} \Vdifftil{k,h+1}(y) \nonumber \\
    &\le \frac{H^2S \sqrt{2L}}{N_k (x_{k,h}, a_{k,h}^{\pi_k})}, \label{lem:regr_dec:9}
\end{align}

where Equation~\eqref{lem:regr_dec:9} is obtained by bounding $\Vdifftil{k,h+1}(y)$ with $H$, and by applying the definition of $[y]_{k,h}$. We can now plug the bounds of $(c)$ and $(d)$ into Equation~\eqref{lem:regr_dec:6} to obtain that:

\begin{equation*}
    (b) \le \overline{\varepsilon}_{k,h} + \sqrt{\frac{1}{2H^2L}}\Vdifftil{k,h+1}(x_{k,h+1}) + \frac{H^2S\sqrt{2L}}{N_k(x_{k,h}, a_{k,h}^{\pi_k})}.
\end{equation*}

By plugging the bound of $(b)$ into Equation~\eqref{lem:regr_dec:5}, we obtain that:

\begin{equation*}
    (a) \le \sqrt{2L} \overline{\varepsilon}_{k,h} + \frac{1}{H} \Vdifftil{k,h+1}(x_{k,h+1}) + \frac{8H^2SL}{3N_k(x_{k,h},a_{k,h}^{\pi_k})}.
\end{equation*}

Finally, substituting the bound on $(a)$ into Equation~\eqref{lem:regr_dec:2}, we obtain that:

\begin{align*}
    \Vdifftil{k,h}(x_{k,h}) &\le \left(1 + \frac{1}{H}\right) \Vdifftil{k,h+1}(x_{k,h+1}) + \bonus_{k,h}(x_{k,h}, a_{k,h}^{\pi_k}) + 
    % \del{c} \left( \VVopt{h+1}(x_{k,h}, a_{k,h}^{\pi_k}), N_k(x_{k,h}, a_{k,h}^{\pi_k}) \right) 
    \xi_{k,h} (x_{k,h}, a_{k,h}^{\pi_k})\\
    &\quad + \varepsilon_{k,h} + \sqrt{2L} \overline{\varepsilon}_{k,h} + \frac{8H^2SL}{3N_k(x_{k,h},a_{k,h}^{\pi_k})}.
\end{align*}

We now apply an inductive argument on $\Vdifftil{k,h}(x_{k,h})$ to isolate the term.

Observing that $\Vdifftil{k,H+1}(x_{k,H+1}) = 0$ by definition, we can rewrite:

\begin{align*}
    \Vdifftil{k,h}(x_{k,h}) &\le \sum_{j=h}^{H-1} \gamma_{j-h} \bigg[ \bonus_{k,j}(x_{k,j}, a_{k,j}^{\pi_k}) + 
    % \del{c} \left( \VVopt{j+1}(x_{k,j}, a_{k,j}^{\pi_k}), N_k(x_{k,j}, a_{k,j}^{\pi_k}) \right)
    \xi_{k,j} (x_{k,j}, a_{k,j}^{\pi_k}) \\
    &\quad + \varepsilon_{k,j} + \sqrt{2L} \overline{\varepsilon}_{k,j} + \frac{8H^2SL}{3N_k(x_{k,j},a_{k,j}^{\pi_k})} \bigg],
\end{align*}

where $\gamma_{j-h} = \left(1+\frac{1}{H}\right)^{j-h}$. Notice that the summation is limited to $H-1$. This will be recurrent throughout the paper and is due to the fact that, the reward being deterministic, there is no uncertainty at $h=H$.
As such, we can assume that the policies $\pi_k$ for $k \in \dsb{K}$ always play greedily at the last stage of each episode.

Observing that $1 + \frac{1}{H} > 1$, we trivially derive that $\gamma_{j-h} \le \gamma_H$ for $j \in \dsb{h,H}$. Recalling that $\lim_{x \to +\infty} \left(1 + \frac{1}{x}\right)^x = e$, we can bound $\gamma_H \le e$, and rewrite:

\begin{align}
    \begin{split}\label{lem:regr_dec:10}
    \Vdifftil{k,h}(x_{k,h}) &\le e \sum_{j=h}^{H-1} \bigg[ \bonus_{k,j}(x_{k,j}, a_{k,j}^{\pi_k}) +
    % \del{c} \left( \VVopt{j+1}(x_{k,j}, a_{k,j}), N_k(x_{k,j}, a_{k,j}) \right) 
    \xi_{k,j} (x_{k,j}, a_{k,j}^{\pi_k}) \\
    &\quad + \varepsilon_{k,j} + \sqrt{2L} \overline{\varepsilon}_{k,j} + \frac{8H^2SL}{3N_k(x_{k,j},a_{k,j}^{\pi_k})} \bigg],
    \end{split}
\end{align}

To conclude the proof, we need to show that this holds for any value of $k \in \dsb{K}$. Recalling the definition of $\Omega_{k,h}$:

\begin{equation*}
    \Omega_{k,h} \coloneqq \left\{ \Vest{i,j}(x) \ge \Vstar{j}(x), \forall (i,j) \in [k,h]_{\mathrm{hist}}, x \in \Ss \right\},
\end{equation*}

where $[k,h]_{\mathrm{hist}} \coloneqq \left\{ (i,j) : i \in \dsb{K}, j \in \dsb{H}, (i<k) \vee (i=k, j \ge h) \right\}$, we observe that, if $\Omega_{k,h}$ holds, then also the events $\Omega_{i,j}$ hold for $(i,j)\in[k,h]_{hist}$. As such, we can sum up the previous bound of Equation~\eqref{lem:regr_dec:10} over all the episodes $i \in \dsb{k}$, thus concluding the proof.
\end{proof}

%---------------------------

\begin{lemma}
\label{lem:sum_eps}
    Let $k \in \dsb{K}$ and $h \in \dsb{H}$. Let events $\mathcal{E}$ and $\Omega_{k,h}$ hold. Then the following bounds hold:
    \begin{align*}
        \sum_{i=1}^k \sum_{j=h}^H \varepsilon_{i,j} &\le 2\sqrt{H^2 T_k L}, \\
        \sum_{i=1}^k \sum_{j=h}^H \overline{\varepsilon}_{i,j} &\le 2 \sqrt{T_k},
    \end{align*}
    where $T_k = kH$.
\end{lemma}

\begin{proof}
Let us first recall the definitions of $\varepsilon_{i,j}$ and $\overline{\varepsilon}_{i,j}$:
\begin{align*}
    \varepsilon_{i,j} &\coloneqq P(\cdot | x_{i,j}, a_{i,j}^{\pi_i})^\transpose \Vdifftil{i,j+1}(\cdot) - \Vdifftil{i,j+1}(x_{i,j+1}), \\
    \overline{\varepsilon}_{i,j} &\coloneqq \sum_{y\in\Ss} P(y|x_{i,j},a_{i,j}^{\pi_i}) \sqrt{\frac{\mathbb{I}(y \in [y]_{i,j})}{N_i(x_{i,j},a_{i,j}^{\pi_i})P(y|x_{i,j},a_{i,j}^{\pi_i})}} \Vdifftil{i,j+1}(y) \\
    &\quad - \sqrt{\frac{\mathbb{I}(x_{i,j+1} \in [y]_{i,j})}{N_i(x_{i,j},a_{i,j}^{\pi_i})P(x_{i,j+1}|x_{i,j},a_{i,j}^{\pi_i})}} \Vdifftil{i,j+1}(x_{i,j+1}),
\end{align*}

where:

\begin{equation*}
    [y]_{k,h} \coloneqq \{ y \in \Ss : N_k(x_{k,h}, a_{k,h}^{\pi_k}) P(y | x_{k,h}, a_{k,h}^{\pi_k}) \ge 2 H^2L \}.
\end{equation*}

Under event $\mathcal{E}$ the following events hold:
\begin{equation*}
    \mathcal{E}_{\mathrm{az}}(\mathcal{F}_{\Vdifftil{}, k, h}, H, L), \quad\text{and}\quad
    \mathcal{E}_{\mathrm{az}}(\mathcal{F}_{\Vdifftil{}, k, h}', 1/\sqrt{L}, L).
\end{equation*}

Event $\mathcal{E}_{\mathrm{az}}(\mathcal{F}_{\Vdifftil{}, k, h}, H, L)$ is defined as the event such that:

\begin{align*}
    \sum_{i=1}^k \sum_{j=h}^{H-1} \left[ \sum_{y\in\Ss} P(y|x_{i,j},a_{i,j}^{\pi_i})\Vdifftil{i,j+1}(y) - \Vdifftil{i,j+1}(x_{i,j+1}) \right] &\le 2 \sqrt{k (H-1-h) H^2 L} \\
    &\le 2 \sqrt{H^2 T_k L}.
\end{align*}

Under this event, we can apply the definition of $\varepsilon_{i,j}$ and derive that:

\begin{equation*}
    \sum_{i=1}^k \sum_{j=h}^{H-1} \varepsilon_{i,j} \le 2 \sqrt{H^2 T_k L}.
\end{equation*}

Event $\mathcal{E}_{\mathrm{az}}(\mathcal{F}_{\Vdifftil{}, k, h}', 1/\sqrt{L}, L)$, on the other hand, is defined as the event such that:

\begin{align*}
    \sum_{i=1}^k \sum_{j=h}^H \bigg[ &\sum_{y\in\Ss} P(y|x_{i,j},a_{i,j}^{\pi_i}) \sqrt{\frac{\mathbb{I}(y \in [y]_{i,j})}{N_i(x_{i,j},a_{i,j}^{\pi_i})P(y|x_{i,j},a_{i,j}^{\pi_i})}} \Vdifftil{i,j+1}(y) \bigg] \\
    &\quad - \sqrt{\frac{\mathbb{I}(y \in [y]_{i,j})}{N_i(x_{i,j},a_{i,j}^{\pi_i})P(y|x_{i,j},a_{i,j}^{\pi_i})}} \Vdifftil{i,j+1}(x_{i,j+1}) \\
    &\le 2 \sqrt{k(H-h) \frac{1}{\sqrt{L}^2} L} \\
    &\le 2 \sqrt{T_k}.
\end{align*}

Under this event, we can apply the definition of $\overline{\varepsilon}_{i,j}$ and derive that:

\begin{equation*}
    \sum_{i=1}^k \sum_{j=h}^{H-1} \overline{\varepsilon}_{i,j} \le 2 \sqrt{T_k},
\end{equation*}

thus concluding the proof.
\end{proof}

%--------------------

\begin{lemma}
\label{lem:sum_eps_x}
    Let $k \in \dsb{K}$, $h \in \dsb{H}$, and $x \in \Ss$. Let events $\mathcal{E}$ and $\Omega_{k,h}$ hold. Then the following bounds hold:
    \begin{align*}
        \sum_{i=1}^k \mathbb{I}(x_{i,h}=x) \sum_{j=h}^H \varepsilon_{i,j} &\le 2\sqrt{H^3 N_{k,h}'(x) L}, \\
        \sum_{i=1}^k \mathbb{I}(x_{i,h}=x) \sum_{j=h}^H \overline{\varepsilon}_{i,j} &\le 2 \sqrt{H N_{k,h}'(x)}.
    \end{align*}
\end{lemma}

\begin{proof}
In a similar way to the proof of Lemma~\ref{lem:sum_eps}, we recall the definitions of $\varepsilon_{i,j}$ and $\overline{\varepsilon}_{i,j}$:
\begin{align*}
    \varepsilon_{i,j} &\coloneqq P(\cdot | x_{i,j}, a_{i,j}^{\pi_i})^\transpose \Vdifftil{i,j+1}(\cdot) - \Vdifftil{i,j+1}(x_{i,j+1}), \\
    \overline{\varepsilon}_{i,j} &\coloneqq \sum_{y\in\Ss} P(y|x_{i,j},a_{i,j}^{\pi_i}) \sqrt{\frac{\mathbb{I}(y \in [y]_{i,j})}{N_i(x_{i,j},a_{i,j}^{\pi_i})P(y|x_{i,j},a_{i,j}^{\pi_i})}} \Vdifftil{i,j+1}(y) \\
    &\quad - \sqrt{\frac{\mathbb{I}(x_{i,j+1} \in [y]_{i,j})}{N_i(x_{i,j},a_{i,j}^{\pi_i})P(x_{i,j+1}|x_{i,j},a_{i,j}^{\pi_i})}} \Vdifftil{i,j+1}(x_{i,j+1}),
\end{align*}

where:

\begin{equation*}
    [y]_{k,h} \coloneqq \{ y \in \Ss : N_k(x_{k,h}, a_{k,h}^{\pi_k}) P(y | x_{k,h}, a_{k,h}^{\pi_k}) \ge 2 H^2L \}.
\end{equation*}

Under event $\mathcal{E}$ the following events hold:
\begin{equation*}
    \mathcal{E}_{\mathrm{az}}(\mathcal{F}_{\Vdifftil{}, k, h, x}, H, L), \quad\text{and}\quad
    \mathcal{E}_{\mathrm{az}}(\mathcal{F}'_{\Vdifftil{}, k, h, x}, 1/\sqrt{L}, L).
\end{equation*}

Event $\mathcal{E}_{\mathrm{az}}(\mathcal{F}_{\Vdifftil{}, k, h, x}, H, L)$ is defined as the event such that:

\begin{align*}
    \sum_{i=1}^k \mathbb{I}(x_{i,h}=x) \sum_{j=h}^{H-1} \left[ \sum_{y\in\Ss} P(y|x_{i,j},a_{i,j}^{\pi_i})\Vdifftil{i,j+1}(y) - \Vdifftil{i,j+1}(x_{i,j+1}) \right] \le 2 \sqrt{H^3 N'_{k,h}(x) L}.
\end{align*}

Under this event, we can apply the definition of $\varepsilon_{i,j}$ and derive that:

\begin{equation*}
    \sum_{i=1}^k \mathbb{I}(x_{i,h}=x) \sum_{j=h}^{H-1} \varepsilon_{i,j} \le 2 \sqrt{H^3 N'_{k,h}(x) L}.
\end{equation*}

Event $\mathcal{E}_{\mathrm{az}}(\mathcal{F}'_{\Vdifftil{}, k, h, x}, 1/\sqrt{L}, L)$, on the other hand, is defined as the event such that:

\begin{align*}
    \sum_{i=1}^k \mathbb{I}(x_{i,h}=x) \sum_{j=h}^H \bigg[ &\sum_{y\in\Ss} P(y|x_{i,j},a_{i,j}^{\pi_i}) \sqrt{\frac{\mathbb{I}(y \in [y]_{i,j})}{N_i(x_{i,j},a_{i,j}^{\pi_i})P(y|x_{i,j},a_{i,j}^{\pi_i})}} \Vdifftil{i,j+1}(y) \bigg] \\
    &\quad - \sqrt{\frac{\mathbb{I}(x_{i,j+1} \in [y]_{i,j})}{N_i(x_{i,j},a_{i,j}^{\pi_i})P(x_{i,j+1}|x_{i,j},a_{i,j}^{\pi_i})}} \Vdifftil{i,j+1}(x_{i,j+1}) \\
    &\le 2 \sqrt{N'_{k,h}(x) (H-h) \frac{1}{\sqrt{L}^2} L} \\
    &\le 2 \sqrt{H N'_{k,h}(x)}.
\end{align*}

Under this event, we can apply the definition of $\overline{\varepsilon}_{i,j}$ and derive that:

\begin{equation*}
    \sum_{i=1}^k \mathbb{I}(x_{i,h}=x) \sum_{j=h}^{H-1} \overline{\varepsilon}_{i,j} \le 2 \sqrt{H N'_{k,h}(x)},
\end{equation*}

thus concluding the proof.

\end{proof}

%-------------

\begin{lemma}
\label{lem:sum_next_state_var}
Let $k \in \dsb{K}$ and $h \in \dsb{H}$. Let $\pi_k$ be the policy followed during episode $k$. Under the events $\mathcal{E}$ and $\Omega_{k,h}$, the following holds for every $x \in \Ss$:

\begin{align*}
    \sum_{i=1}^k \sum_{j=h}^{H-1} \VVpi[i]{j+1} (x_{i,j}, a_{i,j}^{\pi_i}) &\le HT_k + 2\sqrt{H^4 T_k L} + \frac{4}{3}H^3L, \\
    \sum_{i=1}^k \mathbb{I}(x_{i,h}=x) \sum_{j=h}^{H-1} \VVpi[i]{j+1} (x_{i,j}, a_{i,j}^{\pi_i}) &\le H^2 N'_{k,h}(x) + 2 \sqrt{H^5 N'_{k,h}(x) L} + \frac{4}{3} H^3 L.
\end{align*}
\end{lemma}

\begin{proof}
We begin the proof by restating the definition of $\VVpi{j+1} (x_{i,j}, a_{i,j}^{\pi_i})$:

\begin{equation*}
    \VVpi{j+1} (x_{i,j}, a_{i,j}^{\pi_i}) \coloneqq \Var_{y \sim P(\cdot | x_{i,j}, a_{i,j}^{\pi_i})}[\Vpi{j+1} (y)]
\end{equation*}

Under event $\mathcal{E}$, the following events hold:

\begin{equation*}
    \mathcal{E}_{\mathrm{fr}} (\mathcal{G}_{\VV, k, h}, H^4 T_k, H^3, L) \quad\text{and}\quad \mathcal{E}_{\mathrm{fr}} (\mathcal{G}_{\VV, k, h, x}, H^5 N'_{k,h}, H^3, L).
\end{equation*}

Event $\mathcal{E}_{\mathrm{fr}} (\mathcal{G}_{\VV, k, h}, H^4 T_k, H^3, L)$ is defined as the event such that:

\begin{equation}
    \sum_{i=1}^k \sum_{j=h}^{H-1} \VVpi{j+1} (x_{i,j}, a_{i,j}^{\pi_i}) - \sum_{i=1}^k \mathbb{E} \left[ \sum_{j=h}^{H-1} \VVpi{j+1} (x_{i,j}, a_{i,j}^{\pi_i}) | \histkh \right] \le 2\sqrt{H^4 T_k L} + \frac{4H^3 L}{3}, \nonumber
\end{equation}

which implies that:

\begin{equation}
    \sum_{i=1}^k \sum_{j=h}^{H-1} \VVpi{j+1} (x_{i,j}, a_{i,j}^{\pi_i}) \le \sum_{i=1}^k \mathbb{E} \left[ \sum_{j=h}^{H-1} \VVpi{j+1} (x_{i,j}, a_{i,j}^{\pi_i}) | \histkh \right] + 2\sqrt{H^4 T_k L} + \frac{4H^3 L}{3}. \label{lem:sum_next_state_var:1}
\end{equation}

On the other hand, event $\mathcal{E}_{\mathrm{fr}} (\mathcal{G}_{\VV, k, h, x}, H^5 N'_{k,h}, H^3, L)$ is defined as the event such that:

\begin{align}
    \sum_{i=1}^k \mathbb{I}(x_{i,h}=x) \sum_{j=h}^{H-1} \VVpi{j+1} (x_{i,j}, a_{i,j}^{\pi_i}) &- \sum_{i=1}^k \mathbb{I}(x_{i,h}=x) \mathbb{E} \left[ \sum_{j=h}^{H-1} \VVpi{j+1} (x_{i,j}, a_{i,j}^{\pi_i}) | \histkh \right] \nonumber \\ 
    &\le 2\sqrt{H^5 N'_{k,h}(x) L} + \frac{4H^3 L}{3}, \nonumber
\end{align}

which implies that:

\begin{align}
    \begin{split}
    \label{lem:sum_next_state_var:2}
    \sum_{i=1}^k \mathbb{I}(x_{i,h}=x) \sum_{j=h}^{H-1} \VVpi{j+1} (x_{i,j}, a_{i,j}^{\pi_i}) &\le \sum_{i=1}^k \mathbb{I}(x_{i,h}=x) \mathbb{E} \left[ \sum_{j=h}^{H-1} \VVpi{j+1} (x_{i,j}, a_{i,j}^{\pi_i}) | \histkh \right] \\ 
    &\quad + 2\sqrt{H^5 N'_{k,h}(x) L} + \frac{4H^3 L}{3}. 
    \end{split}
\end{align}

Observe that by applying the \emph{law of total variance} \citep[LTV, see, \eg Theorem 9.5.5 of][]{blitzstein2019}, we can write:

\begin{align}
    \begin{split}
    \label{lem:sum_next_state_var:3}
    \underset{x_{i,h+1}, \ldots, x_{i, H-1}}{\Var} \left[ \sum_{j=h}^{H-1} R^\pi (x_{i,j}) \right] &= \underbrace{\underset{x_{i,h+1}}{\Var}\left[ \underset{x_{i,h+2}, \ldots, x_{i,H-1}}{\mathbb{E}} \left[ \sum_{j=h}^{H-1} R^\pi (x_{i,j}) \bigg| x_{i,h+1} \right] \right]}_{(a)} \\
    &\quad + \underset{x_{i,h+1}}{\mathbb{E}} \left[ \underbrace{\underset{x_{i,h+2}, \ldots, x_{i,H-1}}{\Var} \left[ \sum_{j=h}^{H-1} R^\pi (x_{i,j}) \bigg| x_{i,h+1} \right]}_{(b)} \right].
    \end{split}
\end{align}

Term $(a)$ can be rewritten as:

\begin{align}
    (a) &= \underset{x_{i,h+1}}{\Var} \left[ R^\pi (x_{i,h}) + \underset{x_{i,h+2}, \ldots, x_{i,H-1}}{\mathbb{E}} \left[ \sum_{j=h+1}^{H-1} R^\pi (x_{i,j}) \bigg| x_{i,h+1} \right] \right] \nonumber \\
    &= \underset{x_{i,h+1}}{\Var} \left[ \Vpi{h+1} (x_{i, h+1}) \right] \label{lem:sum_next_state_var:4} \\
    &= \VVpi{h+1}(x_{i,h}, a_{i,h}^{\pi_i}),
\end{align}

where Equation~\eqref{lem:sum_next_state_var:4} is obtained by observing that $R^\pi (x_{i,h})$ has zero variance \wrt $x_{i,h+1}$, and by applying the definition of value function.

We can then recursively apply the LTV to term $(b)$ and, considering the expectation over the trajectory generated following policy $\pi$ from stage $h$ onward, we can write:

\begin{equation}
\label{lem:sum_next_state_var:5}
    \underset{x_{i,h+1}, \ldots, x_{i, H-1}}{\Var} \left[ \sum_{j=h}^{H-1} R^\pi (x_{i,j}) \right] = \mathbb{E} \left[ \sum_{j=h}^{H-1} \VVpi{j+1} (x_{i,j}, a_{i,j}^{\pi_i}) \right].
\end{equation}

By applying the result of Equation~\eqref{lem:sum_next_state_var:5} to Equations~\eqref{lem:sum_next_state_var:1} and \eqref{lem:sum_next_state_var:2}, we get:

\begin{align}
    \sum_{i=1}^k \mathbb{E} \left[ \sum_{j=h}^{H-1} \VVpi{j+1} (x_{i,j}, a_{i,j}^{\pi_i}) | \histkh \right] &= \sum_{i=1}^k \Var \left[ \sum_{j=h+1}^{H-1} R^\pi (x_{i,j}) \right] \nonumber \\
    &\le k (H-h)^2 \nonumber \\
    &\le H T_k, \label{lem:sum_next_state_var:6}
\end{align}

and:

\begin{align}
    \sum_{i=1}^k \mathbb{I}(x_{i,h}=x) \mathbb{E} \left[ \sum_{j=h}^{H-1} \VVpi{j+1} (x_{i,j}, a_{i,j}^{\pi_i}) | \histkh \right] &= \sum_{i=1}^k \mathbb{I}(x_{i,h}=x) \Var\left[ \sum_{j=h+1}^{H-1} R^\pi (x_{i,j}) \right] \nonumber \\
    &\le N'_{k,h}(x) (H-h)^2 \nonumber \\
    &\le H^2 N'_{k,h}(x). \label{lem:sum_next_state_var:7}
\end{align}

Finally, we can plug Equations~\eqref{lem:sum_next_state_var:6} and \eqref{lem:sum_next_state_var:7} into Equations~\eqref{lem:sum_next_state_var:1} and \eqref{lem:sum_next_state_var:2}, respectively, obtaining:

\begin{align*}
    \sum_{i=1}^k \sum_{j=h}^{H-1} \VVpi{j+1} (x_{i,j}, a_{i,j}^{\pi_i}) &\le HT_k + 2\sqrt{H^4 T_k L} + \frac{4}{3}H^3L, \\
    \sum_{i=1}^k \mathbb{I}(x_{i,h}=x) \sum_{j=h}^{H-1} \VVpi{j+1} (x_{i,j}, a_{i,j}^{\pi_i}) &\le H^2 N'_{k,h}(x) + 2 \sqrt{H^5 N'_{k,h}(x) L} + \frac{4}{3} H^3 L,
\end{align*}

thus concluding the proof.
\end{proof}

%--------------------------

\begin{lemma}
\label{lem:sum_var_diff_opt}
    Let $k \in \dsb{K}$ and $h \in \dsb{H}$. Let $\pi_k$ be the policy played during episode $k$. Under the events $\mathcal{E}$ and $\Omega_{k,h}$, the following holds for every $x \in \Ss$:
    \begin{align*}
        \sum_{i=1}^k \sum_{j=h}^{H-1} \left( \VVopt{j+1} (x_{i,j}, a_{i,j}^{\pi_i}) - \VVpi[i]{j+1} (x_{i,j}, a_{i,j}^{\pi_i}) \right) &\le 2 H \sum_{i=1}^k \sum_{j=h}^{H-1} \Vdifftil{i,j}(x_{i,j}) + 4 H^2 \sqrt{T_k L}, \\
        \sum_{i=1}^k \mathbb{I}(x_{i,h}=x) \sum_{j=h}^{H-1} \left( \VVopt{j+1} (x_{i,j}, a_{i,j}^{\pi_i}) - \VVpi[i]{j+1} (x_{i,j}, a_{i,j}^{\pi_i}) \right) &\le 2 H \sum_{i=1}^k \mathbb{I}(x_{i,h}=x) \sum_{j=h}^{H-1} \Vdifftil{i,h}(x_{i,h}) \\
        &\quad + 4H^2 \sqrt{H N'_{k,h}(x) L}.
    \end{align*}
\end{lemma}

\begin{proof}
We demonstrate the result by providing an upper bound to $\VVopt{j+1} - \VVpi[i]{j+1}$ first, and then bounding its summation over episodes and stages.
We can demonstrate that:

\begin{align}
    \VVopt{j+1} (x_{i,j}, a_{i,j}^{\pi_i}) - \VVpi[i]{j+1} (x_{i,j}, a_{i,j}^{\pi_i}) &= \Var_{y \sim P(\cdot | x_{i,j}, a_{i,j}^{\pi_i})} [\Vstar{j+1}(y)] - \Var_{y \sim P(\cdot | x_{i,j}, a_{i,j}^{\pi_i})} [\Vpi[i]{j+1} (y)] \nonumber \\
    &\le \mathbb{E}_{y \sim P(\cdot | x_{i,j}, a_{i,j}^{\pi_i})} [(\Vstar{j+1}(y))^2 - (\Vpi[i]{j+1}(y))^2] \label{lem:sum_var_diff_opt:1} \\
    &\le 2H \mathbb{E}_{y \sim P(\cdot | x_{i,j}, a_{i,j}^{\pi_i})} [\Vstar{j+1}(y) - \Vpi[i]{j+1}(y)] \label{lem:sum_var_diff_opt:2},
\end{align}

where Equation~\eqref{lem:sum_var_diff_opt:1} is obtained by applying the definition of variance and observing that $\Vstar{j+1}(x) \ge \Vpi[i]{j+1}(x)$ by definition, and Equation~\eqref{lem:sum_var_diff_opt:2} is obtained by expanding the square and by observing that $\Vpi[]{j+1}(x) \le \Vstar{j+1}(x) \le H$.

Using the argument of Equation~\eqref{lem:sum_var_diff_opt:2}, we obtain the following inequalities:

\begin{align}
    \begin{split}
    \label{lem:sum_var_diff_opt:3}
    \sum_{i=1}^k \sum_{j=h}^{H-1} \big( \VVopt{j+1} (x_{i,j}, a_{i,j}^{\pi_i}) &- \VVpi[i]{j+1} (x_{i,j}, a_{i,j}^{\pi_i}) \big) \\
    &\le 2 H \underbrace{\sum_{i=1}^k \sum_{j=h}^{H-1} \mathbb{E}_{y \sim P(\cdot | x_{i,j}, a_{i,j}^{\pi_i})} [\Vdiff{i,j+1}(y)]}_{(a)},
    \end{split} \\
    \begin{split}
    \label{lem:sum_var_diff_opt:4}
    \sum_{i=1}^k \mathbb{I}(x_{i,h}=x) \sum_{j=h}^{H-1} \big( \VVopt{j+1} (x_{i,j}, a_{i,j}^{\pi_i}) &- \VVpi[i]{j+1} (x_{i,j}, a_{i,j}^{\pi_i}) \big) \\
    &\le 2 H \underbrace{\sum_{i=1}^k \mathbb{I}(x_{i,h}=x) \sum_{j=h}^{H-1} \mathbb{E}_{y \sim P(\cdot | x_{i,j}, a_{i,j}^{\pi_i})} [\Vdiff{i,j+1}(y)]}_{(b)}.
    \end{split}
\end{align}

We now bound term $(a)$ as follows:

\begin{align}
    (a) &\le \sum_{i=1}^k \sum_{j=h}^{H-1} \mathbb{E}_{y \sim P(\cdot | x_{i,j}, a_{i,j}^{\pi_i})} [\Vdifftil{i, j+1}(y)] \label{lem:sum_var_diff_opt:5} \\
    &\le 2 \sqrt{H^2 T_k L} + \sum_{i=1}^k \sum_{j=h}^{H-1} \Vdifftil{i,j+1}(x_{i,j+1}) \label{lem:sum_var_diff_opt:6}
\end{align}

where Equation~\eqref{lem:sum_var_diff_opt:5} is obtained because, under $\Omega_{k,h}$, it holds that ${\Vstar{j+1}(y) \le \Vest{i,j+1}(y)}$. Equation~\eqref{lem:sum_var_diff_opt:6} is obtained by considering that, under event $\mathcal{E}$, the event $\mathcal{E}_{\mathrm{az}} (\mathcal{F}_{\Vdifftil{}, k, h}, H, L)$ holds, as shown in Lemma~\ref{lem:sum_eps}.

Following a similar procedure, we bound term $(b)$ by considering event $\mathcal{E}_{\mathrm{az}}(\mathcal{F}_{\Vdifftil{}, k, h, x}, H, L)$, thus obtaining:

\begin{equation}
\label{lem:sum_var_diff_opt:7}
    (b) \le 2H \sqrt{H N'_{k,h}(x) L} + \sum_{i=1}^k \mathbb{I}(x_{i,h}=x) \sum_{j=h}^{H-1} \Vdifftil{i,j+1}(x_{i,j+1}).
\end{equation}

We can then plug Equations~\eqref{lem:sum_var_diff_opt:6} and \eqref{lem:sum_var_diff_opt:7} into Equations~\eqref{lem:sum_var_diff_opt:3} and \eqref{lem:sum_var_diff_opt:4}, respectively, to write:

\begin{align*}
    \sum_{i=1}^k \sum_{j=h}^{H-1} \left( \VVopt{j+1} (x_{i,j}, a_{i,j}^{\pi_i}) - \VVpi[i]{j+1} (x_{i,j}, a_{i,j}^{\pi_i}) \right) &\le 2 H \sum_{i=1}^k \sum_{j=h}^{H-1} \Vdifftil{i,j}(x_{i,j}) + 4 H^2 \sqrt{T_k L}, \\
    \sum_{i=1}^k \mathbb{I}(x_{i,h}=x) \sum_{j=h}^{H-1} \left( \VVopt{j+1} (x_{i,j}, a_{i,j}^{\pi_i}) - \VVpi[i]{j+1} (x_{i,j}, a_{i,j}^{\pi_i}) \right) &\le 2 H \sum_{i=1}^k \mathbb{I}(x_{i,h}=x) \sum_{j=h}^{H-1} \Vdifftil{i,j}(x_{i,j}) \\
    &\quad + 4H^2 \sqrt{H N'_{k,h}(x) L},
\end{align*}

thus concluding the proof.
\end{proof}

%-----------------------

\begin{lemma}
\label{lem:sum_var_diff_est}
Let $k \in \dsb{K}$ and $h \in \dsb{H}$. Let $\pi_k$ denote the policy followed during episode $k$. Under events $\mathcal{E}$ and $\Omega_{k,h}$, the following inequalities hold for every $x \in \Ss$:

\begin{align*}
    \sum_{i=1}^k \sum_{j=h}^{H-1} \VVest{i,j+1}(x_{i,j}, a_{i,j}^{\pi_i}) &- \VVpi[i]{j+1}(x_{i,j}, a_{i,j}^{\pi_i}) \\
    &\le 7 H^2 S \sqrt{A T_k L} + 2H \sum_{i=1}^k \sum_{j=h}^{H-1} \Vdifftil{i,j+1}(x_{i,j+1}),\\
    \sum_{i=1}^k \mathbb{I}(x_{i,j}=x) \sum_{j=h}^{H-1} \VVest{i,j+1}(&x_{i,j}, a_{i,j}^{\pi_i}) - \VVpi[i]{j+1}(x_{i,j}, a_{i,j}^{\pi_i}) \\
    &\le 7 H^2 S \sqrt{H A N'_{k,h}(x) L} + 2H \sum_{i=1}^k \mathbb{I}(x_{i,h}=x) \sum_{j=h}^{H-1} \Vdifftil{i,j+1}(x_{i,j+1}).
\end{align*}
\end{lemma}

\begin{proof}
Similarly to the proof of Lemma~\ref{lem:sum_var_diff_opt}, we demonstrate the result by providing an upper bound to $\VVest{i,j+1} - \VVpi[i]{j+1}$ first, and then bounding its summation over episodes and stages.

\begin{align}
    \VVest{i,j+1}(x_{i,j}, a_{i,j}^{\pi_i}) &- \VVpi[i]{j+1}(x_{i,j}, a_{i,j}^{\pi_i}) \nonumber \\
    &= \Var_{y \sim \Pest_i (\cdot | x_{i,j}, a_{i,j}^{\pi_i})}[\Vest{i,j+1}(y)] - \Var_{y \sim P (\cdot | x_{i,j}, a_{i,j}^{\pi_i})} [\Vpi[i]{j+1} (y)] \nonumber \\
    &= \mathbb{E}_{y \sim \Pest_i (\cdot | x_{i,j}, a_{i,j}^{\pi_i})} [(\Vest{i,j+1}(y))^2] - \mathbb{E}_{y \sim \Pest_i (\cdot | x_{i,j}, a_{i,j}^{\pi_i})} [\Vest{i,j+1}(y)]^2 \nonumber \\
    &\quad - \mathbb{E}_{y \sim P (\cdot | x_{i,j}, a_{i,j}^{\pi_i})} [(\Vpi[i]{j+1}(y))^2] + \mathbb{E}_{y \sim P (\cdot | x_{i,j}, a_{i,j}^{\pi_i})} [\Vpi[i]{j+1}(y)]^2 \nonumber \\
    \begin{split}
    \label{lem:sum_var_diff_est:1}
    &\le \mathbb{E}_{y \sim \Pest_i (\cdot | x_{i,j}, a_{i,j}^{\pi_i})} [(\Vest{i,j+1}(y))^2] - \mathbb{E}_{y \sim P (\cdot | x_{i,j}, a_{i,j}^{\pi_i})} [(\Vpi[i]{j+1}(y))^2] \\
    &\quad + \mathbb{E}_{y \sim P (\cdot | x_{i,j}, a_{i,j}^{\pi_i})} [\Vstar{j+1}(y)]^2 - \mathbb{E}_{y \sim \Pest_i (\cdot | x_{i,j}, a_{i,j}^{\pi_i})} [\Vstar{j+1}(y)]^2
    \end{split} \\
    \begin{split}
    \label{lem:sum_var_diff_est:2}
    &\le \mathbb{E}_{y \sim \Pest_i (\cdot | x_{i,j}, a_{i,j}^{\pi_i})} [(\Vest{i,j+1}(y))^2] - \mathbb{E}_{y \sim P (\cdot | x_{i,j}, a_{i,j}^{\pi_i})} [(\Vest{i,j+1}(y))^2] \\
    &\quad + \mathbb{E}_{y \sim P (\cdot | x_{i,j}, a_{i,j}^{\pi_i})} [(\Vest{i,j+1}(y))^2] - \mathbb{E}_{y \sim P (\cdot | x_{i,j}, a_{i,j}^{\pi_i})} [(\Vpi[i]{j+1}(y))^2] \\
    &\quad + 2H \sum_{y \in \Ss}(P(y|x_{i,j}, a_{i,j}^{\pi_i}) - \Pest_i (y|x_{i,j}, a_{i,j}^{\pi_i})) \Vstar{j+1}(y)
    \end{split} \\
    \begin{split}
    \label{lem:sum_var_diff_est:3}
    &\le \mathbb{E}_{y \sim \Pest_i (\cdot | x_{i,j}, a_{i,j}^{\pi_i})} [(\Vest{i,j+1}(y))^2] - \mathbb{E}_{y \sim P (\cdot | x_{i,j}, a_{i,j}^{\pi_i})} [(\Vest{i,j+1}(y))^2] \\
    &\quad + \mathbb{E}_{y \sim P (\cdot | x_{i,j}, a_{i,j}^{\pi_i})} [(\Vest{i,j+1}(y))^2] - \mathbb{E}_{y \sim P (\cdot | x_{i,j}, a_{i,j}^{\pi_i})} [(\Vpi[i]{j+1}(y))^2] \\
    &\quad + 4 H \sqrt{\frac{H^2 L}{N_i (x_{i,j}, a_{i,j}^{\pi_i})}} 
    \end{split}
\end{align}

where Equation~\eqref{lem:sum_var_diff_est:1} follows from the fact that, under $\Omega_{k,h}$, $\Vest{i,j}(y) \ge \Vstar{j}(y) \ge \Vpi[i]{j}(y)$. Equation~\eqref{lem:sum_var_diff_est:2} is obtained by adding and subtracting $\mathbb{E}_{y \sim P (\cdot | x_{i,j}, a_{i,j}^{\pi_i})} [(\Vest{i,j+1}(y))^2]$, and by observing that $\Vstar{j}(y) \le H$. Equation~\eqref{lem:sum_var_diff_est:3} is obtained by bounding the model error via Hoeffding's inequality.

Putting this result into the double summation, we get:
\begin{align}
\begin{split}
\label{lem:sum_var_diff_est:4}
    \sum_{i=1}^k \sum_{j=h}^{H-1} &\VVest{i,j+1}(x_{i,j}, a_{i,j}^{\pi_i}) - \VVpi[i]{j+1}(x_{i,j}, a_{i,j}^{\pi_i}) \\
    &\le \underbrace{\sum_{i=1}^k \sum_{j=h}^{H-1} \left[ \mathbb{E}_{y \sim \Pest_i (\cdot | x_{i,j}, a_{i,j}^{\pi_i})} [(\Vest{i,j+1}(y))^2] - \mathbb{E}_{y \sim P (\cdot | x_{i,j}, a_{i,j}^{\pi_i})} [(\Vest{i,j+1}(y))^2] \right]}_{(a)} \\
    &\quad + \underbrace{\sum_{i=1}^k \sum_{j=h}^{H-1} \mathbb{E}_{y \sim P (\cdot | x_{i,j}, a_{i,j}^{\pi_i})} [(\Vest{i,j+1}(y))^2] - \mathbb{E}_{y \sim P (\cdot | x_{i,j}, a_{i,j}^{\pi_i})} [(\Vpi[i]{j+1}(y))^2] }_{(b)} \\
    &\quad + \underbrace{\sum_{i=1}^k \sum_{j=h}^{H-1} 4 H \sqrt{\frac{H^2 L}{N_i (x_{i,j}, a_{i,j}^{\pi_i})}}  }_{(c)}.
\end{split}
\end{align}

We begin by bounding term $(a)$:

\begin{align}
    (a) &\le \sum_{i=1}^k \sum_{j=h}^{H-1} H^2 \| \Pest_i (\cdot | x_{i,j}, a_{i,j}^{\pi_i}) - P ( \cdot | x_{i,j}, a_{i,j}^{\pi_i}) \|_1 \nonumber \\
    &\le \sum_{i=1}^k \sum_{j=h}^{H-1} 2H^2 \sqrt{\frac{SL}{N_i (x_{i,j}, a_{i,j}^{\pi_i})}} \label{lem:sum_var_diff_est:5} \\
    &= 2 H^2 \sqrt{SL} \sum_{x\in\Ss} \sum_{a \in \As} \sum_{n=1}^{N_i(x,a)} n^{-1/2} \nonumber \\
    &\le 2 H^2 \sqrt{SL} \sum_{x\in\Ss} \sum_{a \in \As} \sum_{n=1}^{\frac{kH}{SA}} n^{-1/2} \nonumber \\
    &\le H^2 S \sqrt{A T_k L}, \nonumber
\end{align}

where Equation~\eqref{lem:sum_var_diff_est:5} follows by applying the result of Theorem~{2.1} of \cite{weissman2003}, which holds under event $\mathcal{E}$.

We now bound term $(b)$:
\begin{align}
    (b) &= \sum_{i=1}^k \sum_{j=h}^{H-1} \mathbb{E}_{y \sim P (\cdot | x_{i,j}, a_{i,j}^{\pi_i})} [(\Vest{i,j+1}(y) + \Vpi{j+1}(y))(\Vest{i,j+1}(y) - \Vpi{j+1}(y))] \nonumber \\
    &\le 2H \sum_{i=1}^k \sum_{j=h}^{H-1} \mathbb{E}_{y \sim P (\cdot | x_{i,j}, a_{i,j}^{\pi_i})} [\Vdifftil{i,j+1}(y)] \nonumber \\
    &\le 2H(\sum_{i=1}^k \sum_{j=h}^{H-1} \Vdifftil{i,j+1}(x_{i,j+1}) + 2H \sqrt{T_k L}), \label{lem:sum_var_diff_est:6}
\end{align}

where Equation~\eqref{lem:sum_var_diff_est:6} is obtained by observing that, under event $\mathcal{E}$, event $\mathcal{E}_{\mathrm{az}}(\mathcal{F}_{\Vdifftil{}, k, h}, H, L)$ holds.
We now bound term $(c)$:

\begin{align*}
    (c) &\le 4 H^2 \sqrt{L} \sum_{x\in\Ss} \sum_{a\in\As} \sum_{n=1}^{\frac{kH}{SA}} n^{-1/2} \\
    &\le 2 H^2 \sqrt{S A T_k L}.
\end{align*}

Finally, by plugging the bounds of terms $(a)$, $(b)$, and $(c)$ into Equation~\eqref{lem:sum_var_diff_est:4}, we get:

\begin{align*}
    \sum_{i=1}^k \sum_{j=h}^{H-1} \VVest{i,j+1}(x_{i,j}, a_{i,j}^{\pi_i}) &- \VVpi[i]{j+1}(x_{i,j}, a_{i,j}^{\pi_i}) \\
    &\le H^2 S \sqrt{A T_k L} + 2H \sum_{i=1}^k \sum_{j=h}^{H-1} \Vdifftil{i,j+1}(x_{i,j+1}) \\
    &\quad + 4H^2 \sqrt{T_k L} + 2H^2 \sqrt{SAT_kL} \\
    &\le 7 H^2 S \sqrt{A T_k L} + 2H \sum_{i=1}^k \sum_{j=h}^{H-1} \Vdifftil{i,j+1}(x_{i,j+1}).
\end{align*}

Using the same procedure, we can bound the following summation as:

\begin{align*}
    \sum_{i=1}^k \mathbb{I}(x_{i,h}=x) \sum_{j=h}^{H-1} &\VVest{i,j+1}(x_{i,j}, a_{i,j}^{\pi_i}) - \VVpi[i]{j+1}(x_{i,j}, a_{i,j}^{\pi_i}) \\
    &\le 7 H^2 S \sqrt{H A N'_{k,h}(x) L} + 2H \sum_{i=1}^k \mathbb{I}(x_{i,h}=x) \sum_{j=h}^{H-1} \Vdifftil{i,j+1}(x_{i,j+1}),
\end{align*}

thus concluding the proof.
\end{proof}

%--------------------------

\begin{lemma}[Summation over typical episodes of state-action wise model errors]
\label{lem:sum_c}
Let $k \in \dsb{K}$ and $h \in \dsb{H}$. Let $\pi_k$ be the policy followed during episode $k$. Under events $\mathcal{E}$ and $\Omega_{k,h}$, the following inequalities hold for every $x \in \Ss$:

\begin{align}
    \begin{split}
    \label{lem:sum_c:1}
        \sum_{i=1}^k \mathbb{I}(i \in \ktyp) &\sum_{j=h}^{H-1} \left[ \Pest_i(\cdot | x_{i,j}, a_{i,j}^{\pi_i}) - P(\cdot | x_{i,j}, a_{i,j}^{\pi_i}) \right]^\transpose \Vstar{j+1}(\cdot) \\
        &\le \sqrt{6 H S A T_k L^2} + \frac{2}{3} HSAL^2 \\
        &\quad + 2\sqrt{H S A L^2 \sum_{i=1}^k \sum_{j=h}^{H-1} \Vdifftil{i,j}(x_{i,j})},
    \end{split} \\
    \begin{split}
    \label{lem:sum_c:2}
        \sum_{i=1}^k \mathbb{I}(i \in \ktypx, x_{i,h}=x) &\sum_{j=h}^{H-1} \left[ \Pest_i(\cdot | x_{i,j}, a_{i,j}^{\pi_i}) - P(\cdot | x_{i,j}, a_{i,j}^{\pi_i}) \right]^\transpose \Vstar{j+1}(\cdot) \\
        &\le \sqrt{6 H^2 S A N'_{k,h}(x) L^2} + \frac{2}{3} HSAL^2 \\
        &\quad + 2\sqrt{H S A L^2 \sum_{i=1}^k \mathbb{I}(x_{i,h}=x) \sum_{j=h}^{H-1} \Vdifftil{i,j}(x_{i,j})},
    \end{split}
\end{align}

where:

\begin{align*}
    \ktyp &\coloneqq \{ i\in\dsb{k} : (x_{i,h}, a_{i,h}^{\pi_i}) \in [(x,a)]_k, i \ge 250 H S^2 A L, \forall h \in \dsb{H} \}, \\
    \ktypx &\coloneqq \{ i \in \dsb{k} : (x_{i,h}, a_{i,h}^{\pi_i}) \in [(x,a)]_k, N'_{k,h}(x) \ge 250 H S^2 A L, \forall h \in \dsb{H} \}, \\
    [(x,a)]_k &\coloneqq \{ (x,a) \in \Ss \times \As : N_k (x,a) \ge H, N'_{k,h}(x) \ge H, \forall h \in \dsb{H} \}.
\end{align*}
\end{lemma}

\begin{proof}
We begin by demonstrating the bound of Equation~\eqref{lem:sum_c:1}:

\begin{align}
    \sum_{i=1}^k \mathbb{I}(i \in \ktyp) &\sum_{j=h}^{H-1} \big[ \Pest_i(\cdot | x_{i,j}, a_{i,j}^{\pi_i}) - P(\cdot | x_{i,j}, a_{i,j}^{\pi_i}) \big]^\transpose \Vstar{j+1}(\cdot) \nonumber \\
    &\le \sum_{i=1}^k \mathbb{I}(i \in \ktyp) \sum_{j=h}^{H-1} \left[ \sqrt{\frac{2 \VVopt{j+1}(x_{i,j}, a_{i,j}^{\pi_i}) L}{N_i (x_{i,j}, a_{i,j}^{\pi_i})}} + \frac{2HL}{3N_i(x_{i,j}, a_{i,j}^{\pi_i})} \right] \label{lem:sum_c:3} \\
    \begin{split}
    \label{lem:sum_c:4}
        &\le \sqrt{2L} \sqrt{\underbrace{\sum_{i=1}^k \sum_{j=h}^{H-1} \VVopt{j+1}(x_{i,j}, a_{i,j}^{\pi_i})}_{(a)}} \sqrt{\underbrace{\sum_{i=1}^k \mathbb{I}(i \in \ktyp) \sum_{j=h}^{H-1} \frac{1}{N_i (x_{i,j}, a_{i,j}^{\pi_i})}}_{(b)}} \\
        &\quad + \underbrace{\sum_{i=1}^k \mathbb{I}(i \in \ktyp) \sum_{j=h}^{H-1} \frac{2HL}{3 N_i (x_{i,j}, a_{i,j}^{\pi_i})}}_{(c)},
    \end{split}
\end{align}

where Equation~\eqref{lem:sum_c:3} is obtained by applying Bernstein's inequality \citep[see, \eg][]{cesabianchi2006}, and Equation~\eqref{lem:sum_c:4} is obtained by applying Cauchy-Schwarz's inequality. We now bound terms $(a)$, $(b)$, and $(c)$.

By adding and subtracting $\VVpi[i]{j+1}(x_{i,j},a_{i,j}^{\pi_i})$ to term $(a)$, we can rewrite it as:

\begin{equation*}
    (a) = \underbrace{\sum_{i=1}^k \sum_{j=h}^{H-1} \VVpi[i]{j+1}(x_{i,j}, a_{i,j}^{\pi_i})}_{(d)} + \underbrace{\sum_{i=1}^k \sum_{j=h}^{H-1} \left( \VVopt{j+1}(x_{i,j},a_{i,j}^{\pi_i}) -  \VVpi[i]{j+1} (x_{i,j},a_{i,j}^{\pi_i}) \right)}_{(e)}.
\end{equation*}

As events $\mathcal{E}$ and $\Omega_{k,h}$ hold, we can apply Lemmas~\ref{lem:sum_next_state_var} and \ref{lem:sum_var_diff_opt} to bound terms $(d)$ and $(e)$, respectively, thus obtaining: 

\begin{align}
    (a) &\le H T_k + 2 H^2 \sqrt{T_k L} + \frac{4}{3} H^3 L + 2 H \sum_{i=1}^k \sum_{j=h}^{H-1} \Vdifftil{i,j}(x_{i,j}) + 4 H^2 \sqrt{T_k L} \nonumber \\
    &\le 3 T_k H + 2 H \sum_{i=1}^k \sum_{j=h}^{H-1} \Vdifftil{i,j}(x_{i,j}), \label{lem:sum_c:5}
\end{align}

where Equation~\eqref{lem:sum_c:5} holds under the condition of $\ktyp$.

We now bound terms $(b)$ and $(c)$ as follows:

\begin{align}
    (b) &\le \sum_{x \in \Ss} \sum_{a \in \As} \sum_{n=1}^{kH} n^{-1} \nonumber \\
    &\le SAL, \label{lem:sum_c:6} \\
    & \nonumber \\
    (c) &\le \frac{2}{3} HL \sum_{x \in \Ss} \sum_{a \in \As} \sum_{n=1}^{kH} n^{-1} \nonumber \\
    &\le \frac{2}{3} HSAL^2. \label{lem:sum_c:7}
\end{align}

Finally, by plugging the results of Equations~\eqref{lem:sum_c:5}, \eqref{lem:sum_c:6}, and \eqref{lem:sum_c:7} into Equation~\eqref{lem:sum_c:4}, we get:

\begin{align}
    \sum_{i=1}^k \mathbb{I}(i \in \ktyp) &\sum_{j=h}^{H-1} \big[ \Pest_i(\cdot | x_{i,j}, a_{i,j}^{\pi_i}) - P(\cdot | x_{i,j}, a_{i,j}^{\pi_i}) \big]^\transpose \Vstar{j+1}(\cdot) \nonumber \\
    &\le \sqrt{2L}\sqrt{3 T_k H + 2 H \sum_{i=1}^k \sum_{j=h}^{H-1} \Vdifftil{i,j}(x_{i,j})}\sqrt{SAL} + \frac{2}{3} HSAL^2 \nonumber \\
    &\le \sqrt{6 H S A T_k L^2} + 2\sqrt{H S A L^2 (\sum_{i=1}^k \sum_{j=h}^{H-1} \Vdifftil{i,j}(x_{i,j}))} + \frac{2}{3} HSAL^2, \label{lem:sum_c:8}
\end{align}

where Equation~\eqref{lem:sum_c:8} is obtained by computing the product of the square roots and by the subadditivity of the square root.
Following the same procedure, we can obtain the upper bound of Equation~\eqref{lem:sum_c:2} by substituting terms $T_k$ with $H N'_{k,h}(x)$, and terms $\sum_{i=1}^k \sum_{j=h}^{H-1} \Vdifftil{i,j}(x_{i,j})$ with $\sum_{i=1}^k \mathbb{I}(x_{i,h}=x) \sum_{j=h}^{H-1} \Vdifftil{i,j}(x_{i,j})$.
\end{proof}

%------------------------------------

\begin{lemma}[Summation over typical episodes of bonus terms]
\label{lem:sum_b}
Let $k \in \dsb{K}$ and $h \in \dsb{H}$. Let $\pi_k$ be the policy followed during episode $k$. Let the UCB bonus be defined as:

\begin{align*}
    \bonus_{k,h}(x,a) &= \sqrt{\frac{4 L \Var_{y \sim \Pest_k (\cdot | x,a)} [\Vest{k,h+1}(y)]}{N_k(x,a)}} + \frac{7HL}{3 (N_k(x,a) - 1)} \\
    &\quad + \sqrt{\frac{4 \mathbb{E}_{y \sim \Pest_k(\cdot|x,a)} [\min\{ \frac{84^2H^3S^2AL^2}{N'_{k,h+1}(y)} , H^2\}]}{N_k (x,a)}}.
\end{align*}

Under the events $\mathcal{E}$ and $\Omega_{k,h}$ the following inequalities hold for every $x \in \Ss$:

\begin{align}
    \begin{split}
    \label{lem:sum_b:1}
    \sum_{i=1}^k \mathbb{I}(i \in \ktyp) &\sum_{j=h}^{H-1} \bonus_{i,j}(x_{i,j}, a_{i,j}^{\pi_i}) \\
    &\le \sqrt{28 HSAT_kL^2} + \frac{7}{3} HSAL^2 + 2 \sqrt{84^2 H^3 S^4 A^2 L^4}\\
    &\quad + \sqrt{8 HSAL^2 \sum_{i=1}^k \sum_{j=h}^{H-1} \Vdifftil{i,j+1}(x_{i,j+1})},
    \end{split} \\
    \begin{split}
    \label{lem:sum_b:2}
    \sum_{i=1}^k \mathbb{I}(i \in \ktypx, &x_{i,h}=x) \sum_{j=h}^{H-1} \bonus_{i,j}(x_{i,j}, a_{i,j}^{\pi_i}) \\
    &\le \sqrt{28 H^2SA N'_{k,h}(x) L^2} + \frac{7}{3} HSAL^2 + 2 \sqrt{84^2 H^3 S^4 A^2 L^4} \\
    &\quad + \sqrt{8 HSAL^2 \sum_{i=1}^k \mathbb{I}(x_{i,h}=x) \sum_{j=h}^{H-1} \Vdifftil{i,j+1}(x_{i,j+1})},
    \end{split}
\end{align}

where:

\begin{align*}
    \ktyp &\coloneqq \{ i\in\dsb{k} : (x_{i,h}, a_{i,h}^{\pi_i}) \in [(x,a)]_k, i \ge 250 H S^2 A L, \forall h \in \dsb{H} \}, \\
    \ktypx &\coloneqq \{ i \in \dsb{k} : (x_{i,h}, a_{i,h}^{\pi_i}) \in [(x,a)]_k, N'_{k,h}(x) \ge 250 H S^2 A L, \forall h \in \dsb{H} \}, \\
    [(x,a)]_k &\coloneqq \{ (x,a) \in \Ss \times \As : N_k (x,a) \ge H, N'_{k,h}(x) \ge H, \forall h \in \dsb{H} \}.
\end{align*}
\end{lemma}

\begin{proof}
We begin by demonstrating the bound of Equation~\eqref{lem:sum_b:1}. We can rewrite the summation as:

\begin{align}
    \begin{split}
    \label{lem:sum_b:3}
        \sum_{i=1}^k \mathbb{I}(i \in \ktyp) \sum_{j=h}^{H-1} \bonus_{i,j}(x_{i,j}, a_{i,j}^{\pi_i}) &\le \underbrace{\sum_{i=1}^k \mathbb{I}(i \in \ktyp) \sum_{j=h}^{H-1} \sqrt{\frac{4 L \VVest{i,j+1}(x_{i,j}, a_{i,j}^{\pi_i})}{N_i(x_{i,j},a_{i,j}^{\pi_i})}}}_{(a)} \\
        &\quad + \underbrace{\sum_{i=1}^k \mathbb{I}(i \in \ktyp) \sum_{j=h}^{H-1} \frac{7HL}{3 (N_i(x_{i,j},a_{i,j}^{\pi_i}) - 1)}}_{(b)} \\
        &\quad + \underbrace{\sum_{i=1}^k \mathbb{I}(i \in \ktyp) \sum_{j=h}^{H-1} \sqrt{\frac{4 \mathbb{E}_{y \sim \Pest_i(\cdot|x_{i,j},a_{i,j}^{\pi_i})} \bonus'_{i,j+1}(y)}{N_i(x_{i,j}, a_{i,j}^{\pi_i})}}}_{(c)},
    \end{split}
\end{align}

where $\bonus'_{i,j+1}(y) \coloneqq \min \{ \frac{84^2 H^2 S^2 A L^2}{N'_{i,j+1}(y)}, H^2 \}$. First of all, we observe that we can bound term $(b)$ by using a pigeonhole argument as:

\begin{equation}
\label{lem:sum_b:4}
    (b) \le \frac{7}{3} HSAL^2.
\end{equation}

We now bound term $(a)$. By applying Cauchy-Schwarz's inequality, we obtain:

\begin{equation}
\label{lem:sum_b:5}
    (a) \le \sqrt{4L} \sqrt{\underbrace{\sum_{i=1}^k \sum_{j=h}^{H-1} \VVest{i,j+1}(x_{i,j}, a_{i,j}^{\pi_i})}_{(d)}} \sqrt{\underbrace{\sum_{i=1}^k \mathbb{I}(i \in \ktyp) \sum_{j=h}^{H-1} \frac{1}{N_i (x_{i,j}, a_{i,j}^{\pi_i})}}_{(e)}}.
\end{equation}

By applying the same argument as that of Equation~\eqref{lem:sum_c:6} of Lemma~\ref{lem:sum_c}, we bound term $(e)$ with $SAL$.

We can rewrite term $(d)$ as follows:

\begin{equation}
\label{lem:sum_b:6}
    (d) = \underbrace{\sum_{i=1}^k \sum_{j=h}^{H+1} \VVpi[i]{j+1}(x_{i,j}, a_{i,j}^{\pi_i})}_{(f)} + \underbrace{\sum_{i=1}^k \sum_{j=h}^{H-1} [\VVest{i,j+1} (x_{i,j}, a_{i,j}^{\pi_i}) - \VVpi[i]{j+1}(x_{i,j}, a_{i,j}^{\pi_i})]}_{(g)}
\end{equation}

Under events $\mathcal{E}$ and $\Omega_{k,h}$, we can apply Lemmas~\ref{lem:sum_next_state_var} and \ref{lem:sum_var_diff_est} to upper bound terms $(f)$ and $(g)$ respectively, obtaining the following:

\begin{align*}
    (f) &\le HT_k + 2 \sqrt{H^4 T_k L} + \frac{4 H^3 L}{3}, \\
    (g) &\le 7 H^2 S \sqrt{A T_k L} + 2 H \sum_{i=1}^k \sum_{j=h}^{H-1} \Vdifftil{i,j+1}(x_{i,j+1}).
\end{align*}

Plugging the bounds of $(f)$ and $(g)$ into Equation~\eqref{lem:sum_b:6}, we get:

\begin{align}
    (d) &\le HT_k + 2 \sqrt{H^4 T_k L} + \frac{4 H^3 L}{3} + 7 H^2 S \sqrt{A T_k L} + 2 H \sum_{i=1}^k \sum_{j=h}^{H-1} \Vdifftil{i,j+1}(x_{i,j+1}) \nonumber \\
    &\le 4 H T_k + 2 H \sum_{i=1}^k \sum_{j=h}^{H-1} \Vdifftil{i,j+1}(x_{i,j+1}), \label{lem:sum_b:7}
\end{align}

where Equation~\eqref{lem:sum_b:7} holds under the condition of $\ktyp$.
Combining the bounds of terms $(d)$ and $(e)$, we can rewrite Equation~\eqref{lem:sum_b:5} as:

\begin{align}
    (a) &\le \sqrt{4L} \sqrt{4 H T_k + 2 H \sum_{i=1}^k \sum_{j=h}^{H-1} \Vdifftil{i,j+1}(x_{i,j+1})} \sqrt{SAL} \nonumber \\
    &\le \sqrt{16 H S A T_k L^2} + \sqrt{8 HSAL^2 \sum_{i=1}^k \sum_{j=h}^{H-1} \Vdifftil{i,j+1}(x_{i,j+1})}, \label{lem:sum_b:8}
\end{align}

where Equation~\eqref{lem:sum_b:8} is obtained by expanding the products and applying the subadditivity of the square root.

To bound term $(c)$, we apply Cauchy-Schwarz's inequality, obtaining:

\begin{equation}
\label{lem:sum_b:9}
    (c) \le 2 \sqrt{\underbrace{\sum_{i=1}^k \sum_{j=h}^{H-1} \mathbb{E}_{y \sim \Pest_i(\cdot|x_{i,j},a_{i,j}^{\pi_i})} \bonus'_{i,j+1}(y)}_{(h)}} \sqrt{\underbrace{\sum_{i=1}^k \mathbb{I}(i \in \ktyp) \sum_{j=h}^{H-1} \frac{1}{N_i (x_{i,j}, a_{i,j}^{\pi_i})}}_{(i)}}.
\end{equation}

Similar to term $(e)$, we can bound term $(i)$ with $SAL$.
We now bound term $(h)$. We can rewrite the term as:

\begin{align}
    (h) &= \sum_{i=1}^k \sum_{j=h}^{H-1} \sum_{y \in \Ss} \Pest_i (y | x_{i,j}, a_{i,j}^{\pi_i}) \bonus'_{i,j+1}(y) \nonumber \\
    &= \underbrace{\sum_{i=1}^k \sum_{j=h}^{H-1} \sum_{y \in \Ss} (\Pest_i (y | x_{i,j}, a_{i,j}^{\pi_i}) - P (y | x_{i,j}, a_{i,j}^{\pi_i} )) \bonus'_{i,j+1}(y)}_{(j)} + \sum_{i=1}^k \sum_{j=h}^{H-1} \sum_{y \in \Ss} P(y | x_{i,j}, a_{i,j}^{\pi_i}) \bonus'_{i,j+1}(y) \nonumber \\
    &= (j) + \underbrace{\sum_{i=1}^k \sum_{j=h}^{H-1} \mathbb{E}_{y \sim P(\cdot | x_{i,j}, a_{i,j}^{\pi_i})} \bonus'_{i,j+1}(y) - \bonus'_{i,j+1}(x_{i,j+1})}_{(k)} + \underbrace{\sum_{i=1}^k \sum_{j=h}^{H-1} \bonus'_{i,j+1}(x_{i,j+1})}_{(l)}. \label{lem:sum_b:10}
\end{align}

We bound term $(j)$ as follows:

\begin{align}
    (j) &\le H^2 \sum_{i=1}^k \sum_{j=h}^{H-1} \| \Pest_i (\cdot | x_{i,j}, a_{i,j}^{\pi_i}) - P(\cdot | x_{i,j}, a_{i,j}^{\pi_i}) \|_1 \label{lem:sum_b:11} \\
    &\le 2 H^2 \sqrt{SL} \sum_{i=1}^k \sum_{j=h}^{H-1} (N_i (x_{i,j}, a_{i,j}^{\pi_i}))^{-1/2} \label{lem:sum_b:12} \\
    &\le 2 H^2 \sqrt{SL} \sum_{x \in \Ss} \sum_{a \in \As} \sum_{n=1}^{\frac{kH}{SA}} n^{-1/2} \nonumber \\
    &\le H^2 S \sqrt{A T_k L}, \label{lem:sum_b:13}
\end{align}

where Equation~\eqref{lem:sum_b:11} is obtained by bounding $\bonus'_{i,j+1}(y)$ with $H^2$, Equation~\eqref{lem:sum_b:12} follows by applying the result of Theorem~{2.1} of \cite{weissman2003}, which holds under event $\mathcal{E}$, and Equation~\eqref{lem:sum_b:13} follows from a derivation similar to that of term $(a)$ of Lemma~\ref{lem:sum_var_diff_est}.

To bound term $(k)$, we first observe that it is a Martingale difference sequence, and as such we can bound it via the event $\mathcal{E}_{\mathrm{az}}(\mathcal{F}_{\bonus', k, h}, H^2, L)$, which holds under $\mathcal{E}$, obtaining:

\begin{equation*}
    (k) \le 2 H^2 \sqrt{T_k L}.
\end{equation*}

By applying the definition of $\bonus'$, we can bound term $(l)$ as:

\begin{align*}
    (l) &\le 84^2 H^3 S^2 A L^2 \sum_{i=1}^k \sum_{j=h}^{H-1} \frac{1}{N'_{i,j+1}(x_{i,j+1})} \\
    &\le 84^2 H^3 S^2 A L^2 \sum_{x \in \Ss} \sum_{n=1}^T n^{-1} \\
    &\le 84^2 H^3 S^3 A L^3.
\end{align*}

Plugging the bounds of terms $(j)$, $(k)$, and $(l)$ into Equation~\eqref{lem:sum_b:10}, we get:

\begin{align*}
    (h) &\le H^2 S \sqrt{A T_k L} + 2 H^2 \sqrt{T_k L} + 84^2 H^3 S^3 A L^3.
\end{align*}

By applying the bounds of terms $(h)$ and $(i)$ to Equation~\eqref{lem:sum_b:9}, we get:

\begin{align}
    (c) &\le 2 \sqrt{H^2 S \sqrt{A T_k L} + 2 H^2 \sqrt{T_k L} + 84^2 H^3 S^3 A L^3} \sqrt{SAL} \nonumber \\
    &\le 2 \sqrt{3 H S A T_k L} + 2 \sqrt{84^2 H^3 S^4 A^2 L^4}, \label{lem:sum_b:14}
\end{align}

where Equation~\eqref{lem:sum_b:14} is obtained by expanding the products, applying the subadditivity of the square root, and applying the definition of $\ktyp$.

Finally, we can combine the bounds of terms $(a)$, $(b)$, and $(c)$ into Equation~\eqref{lem:sum_b:3}, obtaining the following bound:

\begin{align*}
    \sum_{i=1}^k \mathbb{I}(i \in \ktyp) &\sum_{j=h}^{H-1} \bonus_{i,j}(x_{i,j}, a_{i,j}^{\pi_i}) \\
    &\le \sqrt{16 H S A T_k L^2} + \sqrt{8 HSAL^2 \sum_{i=1}^k \sum_{j=h}^{H-1} \Vdifftil{i,j+1}(x_{i,j+1})} \\
    &\quad + \frac{7}{3} HSAL^2 + 2 \sqrt{3 H S A T_k L} + 2 \sqrt{84^2 H^3 S^4 A^2 L^4} \\
    &\le \sqrt{28 HSAT_kL^2} + \frac{7}{3} HSAL^2 \\
    &\quad + \sqrt{8 HSAL^2 \sum_{i=1}^k \sum_{j=h}^{H-1} \Vdifftil{i,j+1}(x_{i,j+1})} + 2 \sqrt{84^2 H^3 S^4 A^2 L^4},
\end{align*}

thus demonstrating the result of Equation~\eqref{lem:sum_b:1}. By following the same procedure, substituting $T_k$ with $H N'_{k,h}(x)$ and $\sum_{i=1}^k \sum_{j=h}^{H-1} \Vdifftil{i,j}(x_{i,j})$ with $\sum_{i=1}^k \mathbb{I}{(x_{i,h}=x)} \sum_{j=h}^{H-1} \Vdifftil{i,j}(x_{i,j})$, we can obtain an upper bound to Equation~\eqref{lem:sum_b:2} as:

\begin{align*}
    \sum_{i=1}^k \mathbb{I}(i \in \ktypx, &x_{i,h}=x) \sum_{j=h}^{H-1} \bonus_{i,j}(x_{i,j}, a_{i,j}^{\pi_i}) \\
    &\le \sqrt{28 H^2SA N'_{k,h}(x) L^2} + \frac{7}{3} HSAL^2 \\
    &\quad + \sqrt{8 HSAL^2 \sum_{i=1}^k \mathbb{I}(x_{i,h}=x) \sum_{j=h}^{H-1} \Vdifftil{i,j+1}(x_{i,j+1})} + 2 \sqrt{84^2 H^3 S^4 A^2 L^4},
\end{align*}

thus concluding the proof.
% Following the same procedure, we can obtain the upper bound of Equation~\eqref{lem:sum_b:2} by substituting terms $T_k$ with $H N'_{k,h}(x)$ and terms $\sum_{i=1}^k \sum_{j=h}^{H-1} \Vdifftil{i,j}(x_{i,j})$ with $\sum_{i=1}^k \mathbb{I}(x_{i,h}=x) \sum_{j=h}^{H-1} \Vdifftil{i,j}(x_{i,j})$.

\end{proof}