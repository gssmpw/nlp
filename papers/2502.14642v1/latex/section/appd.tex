\label{sec:app}



\subsection{Data Analysis}

\subsubsection{Data Statistics}

\begin{figure}[h]
    \centering
    \includegraphics[width=1\linewidth]{latex/picture/bing.png}
    \caption{The left figure illustrates the publication year distribution of literary works used as raw material from \datasetname. The right figure shows the proportion of Level 1 and Level 2 behaviors within the overall dataset.}
    \label{fig:bing}
\end{figure}



% 行为链数据集

% 总行为数: 15846
% 总人物数:1000
% 最大行为节点数:20
% 最小行为数:10
% 平均节点数:15-16
% 最大input token长度:54
% 最小input token长度:8
% 平均input token长度:18.48
% 语言:英语
% 来源:小说,传记


\begin{table}[h]
\centering
\begin{tabular}{lc}
\toprule
Statistic & Value \\
\midrule
Total Behaviors & 15846 \\
Total Persons & 1001 \\
Max Behavior Nodes & 20 \\
Min Behaviors & 10 \\
Average Nodes & 15-16 \\
Max Input Token Length & 54 \\
Min Input Token Length & 8 \\
Average Input Token Length & 18.48 \\
Language & English \\
Source & Novels, Biographies \\
\bottomrule
\end{tabular}
\caption{Descriptive Statistics of the Behavior Dataset}
\label{tab:behavior_stats} % Optional: Add a label for cross-referencing
\end{table}

\begin{table}[h]
\centering
\small
\begin{tabular}{lc}
\toprule
Statistic & Value \\
\midrule
Style Categories & 25 \\
Genre Categories & 65 \\
Setting Categories & 536 \\
Theme Categories & 115 \\
Lexile Level Categories & 112 \\
Recommended Reading Age Categories & 31 \\
Max Rating Number & 3,380,109 \\
Min Rating Number & 36 \\
Max Ratings Score & 4.7 \\
Min Ratings Score & 3.3 \\
\bottomrule
\end{tabular}
\caption{Book Diversity Statistics}
\label{tab:book_diversity} % Optional: Add a label for cross-referencing
\end{table}

We collect book-related metadata from SuperSummary and Goodreads, including various categorical tags such as Style, Genre, Setting, Theme, Lexile Level, and Recommended Reading Age from SuperSummary. 
Additionally, we record the number of ratings to the rating scores from Goodreads. 
As shown in Table \ref{tab:book_diversity}, the dataset demonstrates a diverse and comprehensive representation of books.

For specific tags, the most prevalent Styles include \textit{Emotional, Mysterious, and Contemplative}. 
Common Genres featured in the dataset are \textit{Historical Fiction, Modern Classic Fiction, Romance, and Mystery \& Crime Fiction}. 
Frequently occurring Settings include \textit{Contemporary, 2010s, and United States}. 
The dataset also highlights recurring Themes such as \textit{Family, Love, Friendship, and Race}. 
The most frequently occurring levels include 830L, 780L, HL690L, and 740L.
Additionally, the Recommended Reading Age ranges from \textit{7 to 18+}.
Together, these attributes demonstrate the richness and diversity of the dataset, ensuring a broad representation of literary elements across various themes, genres, reading levels, and target audiences.

\subsubsection{More Results Analysis}
\label{sec:app_result}
\begin{figure*}[h]
    \centering
    \includegraphics[width=1\linewidth]{latex/picture/style.pdf}
    \caption{Model performance across different book style.}
    \label{fig:style}
\end{figure*}

We select the top 20 Styles based on their frequency and calculated the average metrics for the book subsets under each Style. 
Our analysis shows that the model's ability to predict behavior chains does not exhibit significant differences across different Styles.
% However, relatively speaking, styles such as Informative and Gritty demonstrate slightly higher predictive performance, likely due to their more structured and explicit behavioral patterns. 
% In contrast, styles like Fantastical and Playful exhibit marginally lower performance, potentially due to greater variability and implicit character behaviors. 
% While these differences are not statistically significant, they suggest that narrative structure may influence the model’s ability on the behavior chain task.


\subsection{Implementation Details}

\subsubsection{Dataset Construction Details}
\label{sec:app_con}
We employed Claude-3-5-sonnet-20240620 for behavior chain extraction and context refinement. The ChatGPT-4o-latest model was utilized to generate distractor items for each behavior. For other generation tasks with lower computational requirements, we selected GPT-3.5 as the execution model.

\subsubsection{Model Testing Details}
\label{sec:app_exp}
All evaluated models were post-interaction versions. The history provided to the LLM consisted of summaries from all preceding chapters used for behavior chain extraction.

To prevent data leakage, we systematically substituted all personal entities in the inputs. For conciseness, all analyses in Section \ref{sec:analysis} are based on the multiple-choice task framework.

The single experimental data presented in \ref{sec:analysis} exclusively comprises outputs from the Llama-3p1-70B-Instract model as the target model.

For inputs that exceed the model's context length, we truncated the history from the oldest to the newest.












\subsection{Prompt}
\label{sec:app_prompt}

\begin{tcolorbox}[title = {First Behavior Generation Prompt}, breakable]

You are an expert in Narrative Analysis and Character Behavior Extraction.\\
Please extract the MOST KEY behavior of \{character\} FROM <Paragraphs>.\\
The behavior should have a significant impact on the development of the storyline, reflect character characteristics or emotions.\\

Ensure that the KEY behavior is an objective statement, clearly stated without any vague expressions.\\
DO NOT add subjective interpretations or inferences about the character's behaviors. ONLY describe the KEY behavior itself. DO NOT mention the result in the KEY behavior.\\
Use your own words instead of quoting the original text.\\
DO NOT repeat or imitate <Previous Key Behavior>.\\

The KEY behavior should have a significant impact on the development of the storyline, the characterization of the characters, and the expression of the theme.\\
Ensure that the key behavior is an objective statement, clearly stated without any vague expressions.\\
DO NOT add subjective interpretations or inferences about the character's behaviors. Only describe the behavior itself.\\

The format of your response should be: \{"key behavior": ""\}.\\

<Paragraphs BEGIN>\\
\{parts[0]\}\\
<Paragraphs END>\\

If the behavior of \{character\} cannot be extracted, output "" ONLY.


\end{tcolorbox}

% \vspace{1cm}


\begin{tcolorbox}[title = {Next Behavior Generation Prompt}, breakable]
You are an expert in Narrative Analysis and Character Behavior Extract.\\
Below, I will provide you with <Previous Paragraphs>, <Previous Key Behavior> extracted from <Previous Paragraphs> and <Current Paragraphs>. \\

Please summary the scene change and plot development detailly and naturally after the <Previous Key Behavior> according to the <Previous Paragraphs> and <Current Paragraphs> I give you. The summary should start with "\{examples[-1]['key behavior']\} After that, ".\\

After the summary, you should extract the MOST KEY behavior of \{character\} FROM <Current Paragraphs>, describe in more than 10 words.\\
The behavior should be a non-meaningless behavior taken spontaneously by \{character\}.\\
The behavior should have a significant impact on the development of the storyline or reflect character characteristics or emotions.\\
Ensure that the behavior is objective statements and state the behavior clearly and do not use any vague expressions. \\
DO NOT add subjective interpretations and inference about the character's behaviors. ONLY describe the behavior itself. \\
DO NOT mention the result in the behavior.\\
Ensure use your own words instead of quoting the original text. \\
DO NOT repeat or imitate <Previous Key Behavior>. \\

Please provide a REVISED summary of the scene change and plot development that occurred before the behavior you extracted from <Current Paragraphs>, making sure not to reveal any information about the behavior. Delete the behavior and subsequent plots, and keep only the plots before the behavior.\\
The REVISED summary should end with "After this or in response to this, what behavior did \{character\} take?"\\

<Previous Paragraphs BEGIN>\\
\{marge part[-1]\}\\
<Previous Paragraphs END>\\

<Previous Key Behavior BEGIN>\\
\{examples[-1]['key behavior']\}\\
<Previous Key Behavior END>\\

<Current Paragraphs BEGIN>\\
\{parts[i]\}\\
<Current Paragraphs END>\\

The format of your response should be \{"summary": "","key behavior":"","new summary":""\}.\\
Ensure that the behavior you extract is taken by \{character\} at this moment, rather than behavior of others or past behavior (in <Paragraphs>, \{character\}probable in the first person).\\
If the behavior of \{character\} cannot be extracted, output "None" ONLY.

\end{tcolorbox}





\begin{tcolorbox}[title = {Similirity Check}, breakable]

Please determine whether the following two behaviors refer to the same behavior:\\
Behavior 1: \{sentence1\}\\
Behavior 2: \{sentence2\}\\
If there is a strong possibility that the two behaviors refer to the same behavior, please output 1; otherwise, output 0, Make sure you give me 0/1.
\end{tcolorbox}




\begin{tcolorbox}[title = {Context Refine}, breakable]
<Context Begin>\\
\{relevant part\}\\
<Context End>\\

<Behavior Begin>\\
\{key behavior\}\\
<Behavior End>\\

Your task is to refine the <Context> according to the following requirements:\\

1. If the <Context> explicitly or implicitly suggests the active behaviors of \{character\} in the <Behavior> or discloses the result/reactions of others caused by <Behavior>, delete these from <Context>. Any references to \{character\}' emotions, feelings, psychological states, or internal conflicts should be eliminated from <Context>.\\

2. If < Behavior> include \{character\}'s reaction/response to event/situation/others behavior, then that event/situation/others behavior should to be described intactly and directly in the end of <Context>.\\

3. If the <Behavior> includes any elements such as contextual conditions or encounters rather than purely active behavior of \{character\}, integrate these elements into the <Context>. Pay attention to the clauses in <Behavior> as it often contain contextual information, but do not include it in <Context> if it happens after \{character\}'s behavior.\\

4. Output the refined <Context> directly without other note.

\end{tcolorbox}




\begin{tcolorbox}[title = {Distracter Generation}, breakable]
<Context Start>\\
\{examples[i]['summary refined']\}\\
<Context End>\\

The original subsequent behavior was "\{behavior\}".\\
Estimate what personality trait does this reflect in brief words, and generate 3 behaviors that different personality traits would exhibit, answered in JSON format.\\
\{\\
"original behavior traits":"",\\
"difference": [\\
\{\\
 "trait": "",\\
 "behavior": ""\\
 \},\\
\{\\
 "trait": "",\\
"behavior": ""\\
 \},\\
\{\\
 "trait": "",\\
 "behavior": ""\\
 \}\\
]\\
\}

\end{tcolorbox}




\begin{tcolorbox}[title = {Level Define}, breakable]
<Summary BEGIN>\\
\{summary l\}\\
<Summary END>\\
Please check whether The following behaviors are implied in the summary.\\
In your reply, retain the sequence numbers. The number of key behaviors must be considerably fewer than \{int(n/2)\}.\\

<behaviors BEGIN>\\
\{sentences\}\\
<behaviors END>

\end{tcolorbox}





\subsection{Data instance}
\label{sec:app_instance}
\begin{tcolorbox}[title = {Profile}, breakable]
\{
  "Name": "Jay Gatsby",\\
  "Personality Traits": [
    "Charismatic",
    "Mysterious",
    "Obsessive",
    "Romantic",
    "Wealthy",
    "Idealistic"
  ],\\
  "Motivations and Goals": [\\
    "To reunite with his former lover, Daisy Buchanan",\\
    "To achieve a high social status and wealth",\\
    "To recapture the past and fulfill his ideal vision of life with Daisy"\\
  ],\\
  "Significant Background Events": [\\
    "Born James Gatz to a poor farming family in North Dakota",\\
    "Changed his name to Jay Gatsby and reinvented himself as a wealthy socialite",\\
    "Amassed his fortune through questionable means",\\
    "Became known for his lavish parties at his mansion in West Egg"\\
  ],\\
  "Relationships": \{\\
    "Daisy Buchanan": "Former lover, whom Gatsby is still deeply in love with",\\
    "Tom Buchanan": "Daisy's husband and Gatsby's rival",\\
    "Nick Carraway": "Narrator of the story and Gatsby's neighbor and friend",\\
    "Jordan Baker": "A professional golfer and friend of Daisy, whom Gatsby has a brief romantic interest in",\\
    "George Wilson": "A mechanic and owner of a garage, indirectly involved in Gatsby's downfall"\\
  \},\\
  "Additional Details": \{\\
    "Occupation": "Businessman with mysterious sources of wealth",\\
    "Social Status": "Wealthy and influential, but not born into old money",\\
    "Hobbies": "Throwing extravagant parties, collecting expensive art and cars",\\
    "Residence": "A grand mansion in West Egg, New York"\\
  \}\\
\}\\
\end{tcolorbox}


\begin{tcolorbox}[title = {History}, breakable]
Summary of Chapter 1 \\
Nick, a Yale graduate and World War I veteran, moves to West Egg, Long Island, to work as a bond salesman.  He rents a small house next to the opulent mansion of his mysterious neighbor, Jay Gatsby.

West Egg is separated from the more fashionable East Egg by the Long Island Sound.  Across the water, Nick's cousin Daisy lives with her wealthy and imposing husband, Tom Buchanan, whom Nick knew at Yale.  Tom's wealth and social standing are evident in their "Georgian Colonial mansion."  Also present at their home during Nick's visit is Jordan Baker, a friend of Daisy's and a well-known golfer.

The chapter highlights the Buchanans' superficial and privileged lifestyle.  During dinner, Tom reveals his racist views by referencing a book espousing white supremacist theories, claiming the Nordic race is responsible for civilization.  This disturbs Nick, but no one challenges Tom's comments.

The dinner is interrupted by a phone call, hinting at Tom's infidelity.  Jordan informs Nick that Tom is having an affair with a woman in New York City.  The tension is palpable, but Daisy attempts to deflect attention.  The awkwardness of the situation is further amplified by another phone call.

As the evening progresses, Nick learns that Jordan Baker is a famous golfer with a somewhat scandalous reputation.  He also discovers that Daisy and he, despite being related, are not close.  The Buchanans tease Nick about a rumor of a broken engagement, which he denies.

Returning home, Nick sees Gatsby standing outside his mansion.  Nick considers inviting him over but hesitates, sensing something enigmatic about Gatsby's presence.
\\Summary of Chapter 2 \\
Chapter 2 delves into the "valley of ashes," a desolate area between West Egg and New York City, where industrial ashes are dumped.  Presiding over this wasteland is the faded billboard of Dr. T.J. Eckleburg, featuring giant, spectacled eyes.

This grim setting is significant because it's where George Wilson's struggling auto shop and his wife Myrtle, Tom's mistress, are located.  Tom, on his way to New York City with Nick, stops at Wilson's garage, ostensibly to discuss a car deal.  The real purpose, however, is to rendezvous with Myrtle. Nick finds Myrtle physically unattractive, despite her apparent allure for Tom.

The group, leaving George behind, proceeds to New York City, impulsively buying a puppy along the way.  They gather at a small apartment Tom keeps for his affair.  The other guests include Myrtle's sister, Catherine, and the McKees, neighbors from the building. Mr. McKee is a mediocre photographer.

The atmosphere becomes increasingly tawdry and tense.  A critical moment occurs when Myrtle mentions Daisy's name, provoking Tom to violently strike her, breaking her nose.

Nick, claiming to be drunk (only the second time in his life), leaves the apartment with Mr. McKee amidst the chaos of tending to Myrtle's injury.  He experiences a blackout and then finds himself at Mr. McKee's bedside, observing his amateurish photographs.  Nick manages to escape the apartment and waits at the train station for the 4:00 a.m. train home.
\\Summary of Chapter 3 \\
Chapter 3 depicts one of Gatsby's extravagant parties through Nick's eyes.  Nick receives a formal invitation, a stark contrast to the casual attendance of most guests, who often don't even know Gatsby.  At the party, Nick reconnects with Jordan Baker, who explains she enjoys large gatherings for the anonymity they offer.  A humorous anecdote highlights Gatsby's generosity: he replaced a dress torn at a previous party with an expensive new one.

Rumors about Gatsby's background and wealth circulate among the guests, ranging from accusations of being a German spy to claims of him being a war hero or even a murderer.  While searching for Gatsby, Nick encounters a drunken "owl-eyed" man in Gatsby's impressive library, who is astonished to discover the books are real.  Nick eventually meets Gatsby, but their conversation is cut short by a phone call.  Observing Gatsby later, Nick is struck by his charismatic smile and the contrast between his apparent sobriety and the revelry of his guests.  Gatsby then takes Jordan aside for a private conversation, leaving her visibly shaken.  As the party ends, a drunken car crash involving the owl-eyed man underscores the reckless atmosphere.

Nick then reflects on his experiences, noting that his time in West Egg isn't solely filled with social events but also with work.  He expresses his growing affection for New York City, despite feelings of loneliness, and his burgeoning relationship with Jordan.  He recounts a story about Jordan cheating in a golf tournament, a scandal that, though dismissed, hints at her dishonesty.  Nick rationalizes her behavior as a consequence of her desire for independence clashing with her need for protection.

The chapter concludes with Nick and Jordan becoming romantically involved.  Nick reveals he ended a relationship in the Midwest to pursue Jordan and emphasizes his own sense of honesty.
\\Summary of Chapter 4 \\
Chapter 4 of *The Great Gatsby* opens with a Sunday morning at Gatsby's mansion, where the usual rumors about his past continue to circulate. Nick briefly catalogs the diverse and affluent guests who frequent Gatsby's parties.

One day in late July, Gatsby takes Nick to lunch in his luxurious car.  During the drive, Gatsby seems overly concerned with Nick's opinion of him and his vehicle. He then directly addresses the rumors about his background, claiming to be the son of wealthy, deceased Midwestern parents.  However, he gives the contradictory answer of San Francisco when asked where in the Midwest.  He further elaborates, stating that he inherited his parents' fortune, lived lavishly in Europe, and then became a war hero in World War I, receiving numerous medals.

To support his claims, Gatsby shows Nick a photo of himself at Oxford and a medal from Montenegro.  These items, along with Gatsby's ability to evade a speeding ticket by showing a Christmas card from the police commissioner, convince Nick of the truth of Gatsby's stories.

They have lunch with Meyer Wolfsheim, a shady character who, according to Gatsby, fixed the World Series. Wolfsheim's bizarre cufflinks, made of human molars, further emphasize his connection to the criminal underworld.  Tom Buchanan briefly joins them, but Gatsby disappears abruptly.

Later, Jordan reveals to Nick that she met Gatsby in Louisville in 1917, when Daisy and Gatsby were together.  Jordan recounts how Daisy almost ran off to New York with Gatsby before he went to war, and how she was later deeply unhappy before her wedding to Tom.  Jordan then reveals the reason for her private conversation with Gatsby at the party: Gatsby wants to reunite with Daisy, and he needs Nick's help to arrange a meeting.
\\Summary of Chapter 5 \\
Chapter 5 details the long-awaited reunion between Gatsby and Daisy at Nick's cottage.  Before Daisy arrives, Gatsby's feigned disinterest crumbles as he obsessively prepares, even having Nick's lawn manicured.  Daisy's initial question about the purpose of the invitation and whether Nick harbors romantic feelings for her is met with a reference to *Castle Rackrent*.

The meeting begins awkwardly, and Nick tries to give them space. However, Gatsby, clearly anxious, repeatedly seeks Nick's advice, prompting Nick to reprimand him for his behavior.  Gatsby's reaction suggests he's offended by Nick's criticism.

They then move to Gatsby's mansion. Nick briefly describes the mansion's original owner, a wealthy brewer who, according to rumors, wanted the surrounding area to resemble a medieval village, an idea Nick finds inherently un-American.  The only person present in the vast house is Klipspringer, a freeloading "boarder."  Gatsby proudly displays his extensive and expensive collection of shirts, a display that moves Daisy to tears.  Nick notices a picture of a man on a yacht, whom Gatsby identifies as Dan Cody, his former mentor and close friend.

Gatsby points out the green light on Daisy's dock, visible from his house in clear weather.  After touring the mansion, they again encounter Klipspringer, whom Gatsby compels to play the piano.  Finally, Nick leaves Gatsby and Daisy alone, implying the possibility of their rekindling their past romance.
\\Summary of Chapter 6 \\
Chapter 6 begins with a reporter's visit to Gatsby, seeking a statement regarding vague, unsubstantiated rumors.  This prompts Nick to recount Gatsby's true origins, as told by Gatsby himself.  James Gatz, born to an unremarkable family in South Dakota, reinvented himself as Jay Gatsby after meeting Dan Cody on Lake Superior.  Feeling out of place in his hometown, Gatz left to pursue a grander future.  He encountered Cody's yacht and, through his resourcefulness, became Cody's personal assistant and was even named a potential heir. However, after Cody's death, his mistress, Ella Kaye, intervened, preventing Gatsby from inheriting anything.\\
Weeks after Gatsby and Daisy's reunion, Nick unexpectedly encounters Tom Buchanan at Gatsby's mansion. Tom and two acquaintances have stopped by during a horseback riding trip. Gatsby's overly eager hospitality towards Tom is noticeable. When Tom and his group leave for another gathering, Gatsby almost insists on joining them, despite their clear disinterest.

Later, Tom and Daisy attend one of Gatsby's parties. Nick observes a palpable tension, viewing the West Egg revelry through Daisy's East Egg perspective.  Daisy is clearly repulsed by the spectacle.  During the party, Tom fuels the rumors of Gatsby's bootlegging activities, disparaging him as "newly rich."  Daisy half-heartedly defends Gatsby, claiming the guests are uninvited and that Gatsby is too gracious to turn them away.  Tom openly flirts with other women, while Daisy, in a strange moment, invites Nick to kiss her, an invitation he declines.

After Tom and Daisy depart, Nick stays late. Gatsby confides in Nick about Daisy, revealing his desire for her to leave Tom and be with him. Gatsby then wistfully recalls the pivotal moment, five years prior, when he first kissed Daisy, a memory he cherishes as a defining moment in his life.

\end{tcolorbox}

\begin{tcolorbox}[title = {Behavior Chain}, breakable]
<context 1>\\
Gatsby dismissed every servant in his house and replaced them with new ones who did not interact with the local community. After that, rumors spread in the village about the new servants, and Gatsby explained to Nick that he hired people recommended by Wolfsheim who wouldn't gossip. The next day was extremely hot, and Nick traveled to the Buchanans' house by train. After this or in response to this, what behavior did Jay Gatsby take?
\\<key\_behavior 1>\\
Gatsby called Nick on the phone to invite him to lunch at Daisy's house the next day.
\\<distracters 1>\\
Gatsby shows up unannounced at the Buchanans' house, insisting on seeing Daisy immediately despite the potential awkwardness.\\
Gatsby avoids contact with Nick or Daisy altogether, choosing instead to observe the situation from a distance without direct involvement.\\
Gatsby openly declares his feelings for Daisy in public, ignoring the potential consequences and drawing attention to himself.\\
\\<context 2>\\
Gatsby called Nick on the phone to invite him to lunch at Daisy's house the next day. After that, Nick arrived at the Buchanans' house on a sweltering hot day. The butler answered the phone, and Nick and Gatsby were directed to the salon where Daisy and Jordan were resting on a couch. Tom's voice could be heard on the telephone in the hall, apparently having a heated conversation about selling a car. Tom then entered the room and greeted Gatsby with concealed dislike. The room had a crimson carpet. After this or in response to this, what behavior did Jay Gatsby take?
\\<key\_behavior 2>\\
Gatsby stood in the center of the crimson carpet and gazed around with fascinated eyes.
\\<distracters 2>\\
Gatsby strode toward Tom with a firm handshake and a polite but assertive comment about the weather.\\
Gatsby fidgeted with the edge of his jacket while avoiding making direct eye contact with anyone in the room.\\
Gatsby crossed his arms, gave Tom a steely glare, and responded curtly to any pleasantries.\\
\\<context 3>\\
Gatsby stood in the center of the crimson carpet and gazed around with fascinated eyes. After that, the group moved to the veranda upon Tom's suggestion. They observed the green Sound where a small sail was moving slowly in the heat. After this or in response to this, what behavior did Jay Gatsby take?
\\<key\_behavior 3>\\
Gatsby raised his hand and pointed across the bay to show the location of his house
\\<distracters 3>\\
Gatsby remained silent, simply nodding in acknowledgment of the view without drawing attention to his house.\\
Gatsby spoke poetically about the beauty of the green Sound and the sailboat, avoiding mention of his house entirely.\\
Gatsby excused himself quietly to stand apart, gazing at the water in introspection.\\
\\<context 4>\\
Gatsby raised his hand and pointed across the bay to show the location of his house. After that, the group moved to have lunch in the darkened dining room where they drank cold ale. The atmosphere became tense as Daisy expressed her distress about the heat and confusion, suggesting they all go to town. Tom and Gatsby engaged in small talk about garages and stables, while Daisy persistently pushed for the idea of going to town. When Daisy asked again who wanted to go to town After this or in response to this, what behavior did Jay Gatsby take?
\\<key\_behavior 4>\\
Gatsby's eyes floated toward Daisy when she asked about going to town, and he stared at her, maintaining eye contact
\\<distracters 4>\\
Gatsby casually shrugged and glanced out the window, showing disinterest in Daisy's suggestion.\\
Gatsby immediately stood up and said, 'Let’s go right now!' without waiting for the group’s consensus.\\
Gatsby hesitated, thoughtfully considering whether going to town was practical given the heat and tension in the room.\\
\\<context 5>\\
Gatsby's eyes floated toward Daisy when she asked about going to town, and he stared at her, maintaining eye contact. After that, no one immediately moved to leave. Tom became increasingly agitated at the group's hesitation, and his hand trembled as he finished his drink. Daisy tried to delay their departure by suggesting they smoke first, but Tom dismissed her suggestion. The women went upstairs to prepare while the three men waited outside in the heat. A tense atmosphere developed as they stood shuffling their feet on the hot gravel, with the silver moon already visible in the western sky. Tom stood there with them. After this or in response to this, what behavior did Jay Gatsby take?
\\<key\_behavior 5>\\
Gatsby attempted to speak to Tom but stopped himself, then asked about Tom's stables with visible effort
\\<distracters 5>\\
Gatsby confronted Tom directly, making a pointed comment about Daisy's choices or feelings, his voice unwavering.\\
Gatsby abruptly suggested they leave for town without consulting anyone, his restlessness showing in his tone and actions.\\
Gatsby smirked and made a sly, cutting remark about Tom's stables, using it as a veiled critique of Tom's character.\\
\\<context 6>\\
Gatsby attempted to speak to Tom but stopped himself, then asked about Tom's stables with visible effort. After that, the group prepared to leave for town. Daisy and Jordan went upstairs to get ready while the men waited outside. Tom expressed his displeasure about going to town, and went inside to get whiskey. The narrator made a comment about Daisy's voice, and Gatsby described it as being 'full of money.' Tom returned with a bottle of whiskey, followed by Daisy and Jordan. Tom suggested they take his car. After this or in response to this, what behavior did Jay Gatsby take?
\\<key\_behavior 6>\\
Gatsby objected to Tom's suggestion of driving his car, claiming there wasn't much gas.
\\<distracters 6>\\
Gatsby nodded and agreed that Tom's car would be a good choice, showing a willingness to go along with the plan without objection.\\
Gatsby hesitated, closely examining Tom's car, and suggested they double-check everything to ensure the ride would be safe.\\
Gatsby grinned and suggested they all race both cars to town, adding some excitement to the mundane trip.\\
\\<context 7>\\
Gatsby objected to Tom's suggestion of driving his car, claiming there wasn't much gas. After that, Tom insisted there was plenty of gas and suggested stopping at a drugstore if needed. The group then split up, with Tom driving Gatsby's car with Nick and Jordan, while Gatsby and Daisy followed in Tom's coupé. As they drove, Tom expressed suspicion about Gatsby and mentioned conducting a small investigation into his past. They stopped at Wilson's garage for gas, where Wilson revealed his plans to move west with his wife. The group then continued to the city, eventually ending up in a suite at the Plaza Hotel. The conversation became tense as Tom questioned Gatsby's use of the phrase 'old sport'. In the suite, there was a fallen telephone book with its string parted. After this or in response to this, what behavior did Jay Gatsby take?
\\<key\_behavior 7>\\
Gatsby examined the parted string of the fallen telephone book, muttered 'Hum!' in an interested way, and tossed the book on a chair.
\\<distracters 7>\\
Gatsby ignored the fallen telephone book completely, focusing instead on more immediate concerns in the conversation.\\
Gatsby picked up the fallen telephone book, carefully retied the string, and placed it neatly back where it belonged.\\
Gatsby glanced briefly at the telephone book, shrugged, and dismissed it as unimportant without further reaction.\\
\\<context 8>\\
Gatsby examined the parted string of the fallen telephone book, muttered 'Hum!' in an interested way, and tossed the book on a chair. After that, Tom sharply criticized Gatsby's habit of saying 'old sport,' leading to tension in the room. Daisy attempted to diffuse the situation by requesting ice for mint juleps. The group then heard a wedding march from downstairs, which prompted a discussion about Daisy's wedding and a mysterious guest named Biloxi. As the conversation about Biloxi's conflicting stories continued, Tom suddenly turned to Gatsby with a pointed question about his education. After this or in response to this, what behavior did Jay Gatsby take?
\\<key\_behavior 8>\\
When questioned about his Oxford education, Gatsby first responded evasively by saying 'Not exactly,' then changed his answer to confirm he went there, and finally specified he attended Oxford for five months in 1919.
\\<distracters 8>\\
Gatsby would directly and calmly state he attended Oxford as part of a special program, elaborating on the details without hesitation.\\
Gatsby would respond assertively, challenging Tom by pointing out the validity of his education and questioning why it matters.\\
Gatsby would shrug and dismiss the question, saying something like, 'I don't see why it matters,' and redirecting the conversation elsewhere.\\
\\<context 9>\\
When questioned about his Oxford education, Gatsby first responded evasively by saying 'Not exactly,' then changed his answer to confirm he went there, and finally specified he attended Oxford for five months in 1919. After that, Gatsby explained that his time at Oxford was part of an opportunity given to some officers after the Armistice, allowing them to attend universities in England or France. This explanation seemed to renew the narrator's faith in Gatsby. Meanwhile, Daisy attempted to lighten the mood by suggesting they make mint juleps, but Tom insisted on asking Gatsby one more question, demanding to know what kind of 'row' Gatsby was trying to cause in his house. After this or in response to this, what behavior did Jay Gatsby take?
\\<key\_behavior 9>\\
Gatsby politely invited Tom to continue with his questioning.
\\<distracters 9>\\
Gatsby openly challenges Tom, questioning Tom's motivations and accusing him of hypocrisy in front of everyone.\\
Gatsby avoids eye contact, becomes visibly agitated, and deflects the question by bringing up a completely unrelated topic.\\
Gatsby abruptly raises his voice, passionately declaring his love for Daisy and confronting Tom about his treatment of her.\\
\\<context 10>\\
Gatsby politely invited Tom to continue with his questioning. After that, The tension in the room escalated as Tom accused Gatsby of causing trouble and challenged the modern attitudes towards family life. Daisy attempted to diffuse the situation, but Tom's anger and prejudice became more apparent. Jordan and Nick reacted with discomfort to Tom's outburst. As the argument intensified, Daisy interrupted, pleading for everyone to leave. Nick agreed and suggested they all go home, but Tom insisted on hearing what Gatsby had to say. After this or in response to this, what behavior did Jay Gatsby take?
\\<key\_behavior 10>\\
Gatsby directly told Tom that his wife doesn't love him.
\\<distracters 10>\\
Gatsby calmly acknowledged the tension and suggested a private conversation with Tom to resolve the issue.\\
Gatsby remained silent and avoided escalating the argument further, allowing others to take the lead.\\
Gatsby expressed understanding of Tom's concerns and reassured him that he had no intention to disrupt the family.\\
\\<context 11>\\
Gatsby directly told Tom that his wife doesn't love him. After that, Gatsby revealed that he and Daisy had been in love for five years without Tom's knowledge. Tom became furious and denied this claim, insisting that Daisy loved him when they married and still loves him now. Tom admitted his own infidelities but claimed he always came back to Daisy. Daisy expressed her disgust at Tom and confronted him about their departure from Chicago. After this or in response to this, what behavior did Jay Gatsby take?
\\<key\_behavior 11>\\
Gatsby walked over and stood beside Daisy to show his support when she confronted Tom about their departure from Chicago
\\<distracters 11>\\
Gatsby stayed silent and remained seated, avoiding direct involvement or confrontation.\\
Gatsby aggressively confronted Tom, escalating the conflict and directly challenging his assertions about Daisy.\\
Gatsby distanced himself from the interaction entirely, stepping away from both Daisy and Tom as they argued.\\
\\<context 12>\\
Gatsby walked over and stood beside Daisy to show his support when she confronted Tom about their departure from Chicago. After that, Gatsby urged Daisy to tell Tom the truth about never loving him, insisting that it would erase their past. Daisy hesitated and looked to Jordan and the narrator for support before reluctantly stating she never loved Tom. Tom questioned her about specific moments in their past, but Daisy refused to engage further. She then turned to Gatsby, acknowledging his presence but expressing frustration at his expectations. After this or in response to this, what behavior did Jay Gatsby take?
\\<key\_behavior 12>\\
Gatsby's eyes opened and closed in response to Daisy's confession of loving both him and Tom.
\\<distracters 12>\\
Gatsby smiles softly at Daisy, nodding with quiet acceptance, and assures her that her feelings won't change his regard for her.\\
Gatsby abruptly interrupts, dismissing Daisy's confusion and insisting they leave together immediately regardless of what she just said.\\
Gatsby steps back, averting his gaze, and calmly acknowledges her feelings, signaling that he would respect whatever decision she makes.\\
\\<context 13>\\
Gatsby's eyes opened and closed in response to Daisy's confession of loving both him and Tom. After that, Tom harshly claimed that Daisy never truly knew Gatsby was alive and mentioned his intimate history with Daisy. After this or in response to this, what behavior did Jay Gatsby take?
\\<key\_behavior 13>\\
Gatsby insisted on speaking to Daisy alone and declared to Tom that he would no longer take care of Daisy.
\\<distracters 13>\\
Gatsby emotionally withdrew, avoiding confrontation altogether, and let Tom and Daisy decide their fate without intervention.\\
Gatsby accused Tom of manipulating Daisy and escalated the verbal conflict, refusing to back down and openly challenging Tom in front of everyone.\\
Gatsby tried to comfort Daisy and suggested that she take her time to figure out her feelings, prioritizing her emotional well-being over himself or Tom.\\
\\<context 14>\\
Gatsby insisted on speaking to Daisy alone and declared to Tom that he would no longer take care of Daisy. After that, Tom accused Gatsby of being a common swindler and revealed that he had conducted an investigation into Gatsby's affairs. Tom exposed Gatsby's involvement in illegal activities, including running drug stores that sold grain alcohol. He also mentioned Gatsby's association with Meyer Wolfsheim and accused him of being a bootlegger. After this or in response to this, what behavior did Jay Gatsby take?
\\<key\_behavior 14>\\
Gatsby calmly responded to Tom's accusations about his illegal activities and association with Meyer Wolfsheim.
\\<distracters 14>\\
Gatsby angrily shouted at Tom, denying the accusations and accusing Tom of trying to smear his name.\\
Gatsby avoided eye contact and quietly admitted to parts of the accusations, retreating from any further confrontation.\\
Gatsby immediately launched into a detailed explanation and justification of his activities, attempting to discredit Tom's claims.\\
\\<context 15>\\
Gatsby calmly responded to Tom's accusations about his illegal activities and association with Meyer Wolfsheim. After that, Tom continued his accusations by bringing up Walter Chase and the betting laws, mentioning how Wolfsheim had intimidated Walter into silence. After this or in response to this, what behavior did Jay Gatsby take?
\\<key\_behavior 15>\\
Gatsby displayed an unfamiliar yet recognizable expression on his face when confronted about his more serious criminal activities
\\<distracters 15>\\
Gatsby smirked confidently and dismissed the accusations as baseless lies, daring Tom to present any real evidence.\\
Gatsby maintained a serene expression, calmly stating that his private affairs were none of Tom's concern.\\
Gatsby suddenly raised his voice, vehemently denying the accusations and accusing Tom of hypocrisy and worse behavior.\\
\end{tcolorbox}




