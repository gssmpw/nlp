% This must be in the first 5 lines to tell arXiv to use pdfLaTeX, which is strongly recommended.
% \pdfoutput=1
% In particular, the hyperref package requires pdfLaTeX in order to break URLs across lines.

\documentclass[11pt]{article}

% \usepackage{fontspec}
% \setmainfont{Times New Roman} % 设置主要字体为 Times New Roman 系统字体


% Change "review" to "final" to generate the final (sometimes called camera-ready) version.
% Change to "preprint" to generate a non-anonymous version with page numbers.
\usepackage{acl}
\usepackage{amssymb}
\usepackage{multirow}
% Standard package includes
\usepackage{times}
\usepackage{latexsym}
\usepackage{amsmath}
% \usepackage{xeCJK} % 导入 xeCJK 宏包
% \setCJKmainfont{SimSun} % 设置中文字体
% \usepackage{footmisc} % 引入 footmisc 包
% \renewcommand{\thefootnote}{\fnsymbol{footnote}}

% For proper rendering and hyphenation of words containing Latin characters (including in bib files)
\usepackage[T1]{fontenc}
% For Vietnamese characters
% \usepackage[T5]{fontenc}
% See https://www.latex-project.org/help/documentation/encguide.pdf for other character sets
\usepackage{float}

% This assumes your files are encoded as UTF8
% \usepackage[utf8]{inputenc}

% This is not strictly necessary, and may be commented out,
% but it will improve the layout of the manuscript,
% and will typically save some space.

% This is also not strictly necessary, and may be commented out.
% However, it will improve the aesthetics of text in
% the typewriter font.
\usepackage{inconsolata}

%Including images in your LaTeX document requires adding
%additional package(s)
\usepackage{graphicx}
% \usepackage[backref=page]{hyperref}
\usepackage[capitalise]{cleveref}
\usepackage{url}
\usepackage{booktabs}
\usepackage{bbding}
\usepackage{makecell}
% \usepackage{algorithm}
\usepackage{algpseudocode}
\usepackage{amssymb}
% \usepackage{xcolor}
\usepackage{pifont} 
\usepackage{amsfonts} % for \mathfrak
\usepackage{mathrsfs} % for \mathscr
\usepackage{tabularx}
\usepackage{colortbl}
\usepackage{lscape}
\usepackage{wrapfig}
\usepackage{subfig} 
% \usepackage{emoji}
\usepackage{amsmath}
\usepackage{amssymb}
\usepackage[linesnumbered,ruled]{algorithm2e}
% \usepackage{algorithm2e}

\usepackage{mathptmx} % 引入 Times New Roman 字体

\usepackage[english]{babel}
\usepackage{graphicx}
\usepackage[makeroom]{cancel}
\usepackage{hyperref}


\usepackage{microtype}
% \usepackage{CJKutf8}
\usepackage[utf8]{inputenc}
\usepackage[T1]{fontenc}


\usepackage{colortbl}
\usepackage{booktabs}


\usepackage{tcolorbox}
\tcbuselibrary{breakable}

% \usepackage[table]{xcolor}
% \usepackage[transparent]{xcolor}
\usepackage{xcolor}

% % 200,220,255
% \definecolor{mc-color}{RGB}{245, 250, 255} % 淡蓝色
% \definecolor{gen-color}{RGB}{255, 245, 245} % 淡红色
% \definecolor{mo del-color}{RGB}{255, 245, 245} % 淡红色

\definecolor{mc-color}{RGB}{245, 250, 255} % 淡蓝色
% \definecolor{gen-color}{RGB}{255, 245, 245} % 淡绿色
\definecolor{gen-color}{RGB}{247,254,247}
% {224,251,226} % 淡绿色
% rgb(247,254,247)
\definecolor{model-color}{RGB}{250, 250, 250}

\definecolor{dark-grey}{RGB}{245,245,245}

\definecolor{white}{RGB}{255,255,255}


% 定义带有透明度的颜色

% \definecolor{mc-color}{RGB, opacity=0.5}{173, 216, 230} % 淡蓝色,透明度 50%
% \definecolor{gen-color}{RGB, opacity=0.8}{220,255,220} % 淡绿色,透明度 80%

\newcommand{\datasetname}{\textsc{BehaviorChain}\xspace}
\newcommand{\meone}
{AvgAcc\xspace}
\newcommand{\metwo}
{ChainAcc\xspace}



% Define color for symbols
% \definecolor{red}{rgb}{1.0, 0.0, 0.0}
% \definecolor{green}{rgb}{0.0, 1.0, 0.0}

% \newcommand{\xmark}{\textcolor{red}{\ding{55}}}
% \newcommand{\headercolor}{\rowcolor{gray!15}}




% If the title and author information does not fit in the area allocated, uncomment the following
%
%\setlength\titlebox{<dim>}
%
% and set <dim> to something 5cm or larger.

% Enhancing Uncertainty Estimation for Large Language Models with Instance-level Correctors
% Enhancing Uncertainty Estimation for Large Language Models
% Towards Harmonized Uncertainty Estimation for Large Language Models
\title{How Far are LLMs from Being Our Digital Twins? \\A Benchmark for Persona-Based Behavior Chain Simulation}

% Author information can be set in various styles:
% For several authors from the same institution:
% \author{Author 1 \and ... \and Author n \\
%         Address line \\ ... \\ Address line}
% if the names do not fit well on one line use
%         Author 1 \\ {\bf Author 2} \\ ... \\ {\bf Author n} \\
% For authors from different institutions:
% \author{Author 1 \\ Address line \\  ... \\ Address line
%         \And  ... \And
%         Author n \\ Address line \\ ... \\ Address line}
% To start a separate ``row'' of authors use \AND, as in
% \author{Author 1 \\ Address line \\  ... \\ Address line
%         \AND
%         Author 2 \\ Address line \\ ... \\ Address line \And
%         Author 3 \\ Address line \\ ... \\ Address line}


\author{
    Rui Li\textsuperscript{1},
    Heming Xia\textsuperscript{2},
    Xinfeng Yuan\textsuperscript{3},
    Qingxiu Dong\textsuperscript{1},
    Lei Sha\textsuperscript{4},
    Wenjie Li\textsuperscript{2},
    Zhifang Sui\textsuperscript{1} \\
        \textsuperscript{1}Peking University \quad
        \textsuperscript{2}The Hong Kong Polytechnic University \\
        \textsuperscript{3}Fudan University \quad
        \textsuperscript{4}Beihang University\\
    \texttt{o\_l1ru1@stu.pku.edu.cn, he-ming.xia@connect.polyu.hk}
}

% \textsuperscript{1}State Key Laboratory of Multimedia Information Processing, School of Computer Science, Peking University} \\
% \small{\textsuperscript{2}The Hong Kong Polytechnic University} \\
% \small{\textsuperscript{3}Fudan University} \\
% \small{\textsuperscript{4}Institute of Artificial Intelligence, Beihang University} \\
% \small{\texttt{o\_l1ru1@stu.pku.edu.cn} \texttt{he-ming.xia@connect.polyu.hk}}
% }

% he-ming.xia@connect.polyu.hk

%\author{
%  \textbf{First Author\textsuperscript{1}},
%  \textbf{Second Author\textsuperscript{1,2}},
%  \textbf{Third T. Author\textsuperscript{1}},
%  \textbf{Fourth Author\textsuperscript{1}},
%\\
%  \textbf{Fifth Author\textsuperscript{1,2}},
%  \textbf{Sixth Author\textsuperscript{1}},
%  \textbf{Seventh Author\textsuperscript{1}},
%  \textbf{Eighth Author \textsuperscript{1,2,3,4}},
%\\
%  \textbf{Ninth Author\textsuperscript{1}},
%  \textbf{Tenth Author\textsuperscript{1}},
%  \textbf{Eleventh E. Author\textsuperscript{1,2,3,4,5}},
%  \textbf{Twelfth Author\textsuperscript{1}},
%\\
%  \textbf{Thirteenth Author\textsuperscript{3}},
%  \textbf{Fourteenth F. Author\textsuperscript{2,4}},
%  \textbf{Fifteenth Author\textsuperscript{1}},
%  \textbf{Sixteenth Author\textsuperscript{1}},
%\\
%  \textbf{Seventeenth S. Author\textsuperscript{4,5}},
%  \textbf{Eighteenth Author\textsuperscript{3,4}},
%  \textbf{Nineteenth N. Author\textsuperscript{2,5}},
%  \textbf{Twentieth Author\textsuperscript{1}}
%\\
%\\
%  \textsuperscript{1}Affiliation 1,
%  \textsuperscript{2}Affiliation 2,
%  \textsuperscript{3}Affiliation 3,
%  \textsuperscript{4}Affiliation 4,
%  \textsuperscript{5}Affiliation 5
%\\
%  \small{
%    \textbf{Correspondence:} \href{mailto:email@domain}{email@domain}
%  }
%}

\begin{document}
\maketitle
% \footnotetext{$^\dagger$Equal contribution} % 在 author 环境外添加角注内容

% \footnotetext{$^\dagger$Equal Contributions.}
\begin{abstract}




% Recent advances in large language models (LLMs) have revolutionized the artificial intelligence landscape, 
% enabling seamless integration into human workflows. 
% Given their remarkable capabilities in language processing and human-like cognitive functions, 


% However, this potential contrasts with prevailing research trends, which predominantly emphasize human dialogue simulation while neglecting comprehensive investigations into behavior simulation—a critical dimension of digital twins.
Recently, LLMs have garnered increasing attention across academic disciplines for their potential as human \textbf{\textit{digital twins}}, virtual proxies designed to replicate individuals and autonomously perform tasks such as decision-making, problem-solving, and reasoning on their behalf.
However, current evaluations of LLMs primarily emphasize dialogue simulation while overlooking human behavior simulation, which is crucial for digital twins.
To address this gap, we introduce \datasetname, the first benchmark for evaluating LLMs' ability to simulate continuous human behavior.
\datasetname comprises diverse, high-quality, persona-based behavior chains, totaling 15,846 distinct behaviors across 1,001 unique personas, each with detailed history and profile metadata.
For evaluation, we integrate persona metadata into LLMs and employ them to iteratively infer contextually appropriate behaviors within dynamic scenarios provided by \datasetname. Comprehensive evaluation results demonstrated that even state-of-the-art models struggle with accurately simulating continuous human behavior.
Resources are available at \href{https://github.com/O-L1RU1/BehaviorChain}{https://github.com/O-L1RU1/BehaviorChain}

% For evaluation
% \datasetname provides a robust environment for evaluation, where


 % enabling the simulation of complex behavioral dynamics.
% For evaluation, we integrate persona metadata into LLMs and iteratively infer contextually appropriate behavior nodes in dynamic scenarios. 
% within complex, dynamically evolving scenarios until the end of a chain.
% to iteratively generate contextually appropriate behavior in dynamic scenarios for evaluation.
% It comprises diverse, high-quality, persona-based behavior chains, totaling 15,846 distinct behaviors across 1,001 unique personas, each with detailed history and profile metadata.
% We established an evaluation framework by integrating persona metadata into LLMs and using them to iteratively recognize or generate contextually appropriate behavior node-by-node within complex, dynamically evolving scenarios until the end of a chain.

% Based on their promising performance in language capabilities and human-like cognition, growing research interest has emerged across disciplines to explore LLMs' potential as human \textbf{\textit{digital twins}} - virtual avatars capable of on behalf of specific individuals to perform tasks such as decision-making and reasoning.
% growing research interest has emerged across disciplines to explore LLMs' potential as human \textbf{\textit{digital twins}} - virtual representations on behalf of specific individuals to perform tasks such as decision-making and reasoning.
 % augment decision-making, and replicate complex behaviors.
% provide personalized assistance, 
% conduct a 
% decision-making, or interact autonomously in virtual reality.
% undergo predictive scenario testing,
% However, existing work predominantly focuses on dialogue simulation, overlooking the critical need for temporally consistent behavior modeling in dynamic contexts. 
% To address this gap, we introduce \datasetname, the first benchmark designed to evaluate LLMs’ ability to simulate continuous human behaviors. This benchmark comprises 1,001 diverse, high-quality persona-based ``behavior chains'', encompassing 15,846 distinct behaviors across 1,001 unique personas.  Each persona is accompanied by metadata including detailed history and profile.
% we introduce \datasetname, a novel benchmark featuring 1,001 high-quality persona-based ``behavior chains'', with each chain incorporating a distinct persona enriched with metadata including detailed history and profile. 
% automatically extracted from novels and biographies, contextualized within plot and scene transitions between behaviors.
% Each persona is further characterized by detailed profiles and history.





% Digital twin$-$a virtual replica mirroring a real-world entity$-$holds immense fascination. 
% LLMs' remarkable progress in language capabilities and human-like cognition is driving the vision forward with the potential of LLMs as human digital twins.
% However, this promising potential contrasts sharply with the current work on LLM-powered persona simulation, which neglects the critical dimension of character behavior and its consistency over spatiotemporal ranges.
% % Existing research on LLM-based persona simulation neglects the critical dimension of character behavior and its consistency over spatiotemporal ranges.
% % Related research on LLMs' capabilities in persona-around simulation while overlooks the critical dimension of character behavior and character's consistency across a spatiotemporal range.
% % focusing on dialogue, decision-making, and knowledge within point-in-time context, it
% % , which is a more chanllenging but realistic scenario.
% To fill this gap, we introduce \datasetname, a novel benchmark comprising 1001 high-quality persona-based ``behavior chains'' extracted automatically from novels and biographies, contextualized within plot and scene transitions between behaviors as context and enriched the character with detailed personal histories.
% % detailed character profiles and histories.
% % real-world narratives 
% % novels and biographies
% % This task is designed to provide a comprehensive and longitudinal evaluation of LLMs' ability to discern and maintain character-consistent behaviors.
% We established an evaluation framework by equipping LLMs with character profile and history, and employing them to  gradually recognize\&generate each behavior in a chain until end. The process followed complex and dynamically changing context.
% % Performance is qualified quantifying individual behavior node correctness, and \textit{Chain Accuracy}, assessing continuity across the behavior chain.
% Evaluation on current state-of-the-art models demonstrated that they still fail to accurately simulate human behavior continuously.
% substantial deficiencies in their ability to accurately simulate human behavior, particularly continuously.

% using two metrics: \textit{Average Accuracy}, 
% \textit{Chain Accuracy}

% As digital twin is a critical frontier in the development of LLMs. 
% Our work has uncovered pervasive challenges and offers valuable insights for future research in this domain and provide guidance and foster further exploration of LLMs as digital twins. 
% The dataset and the code are publicly available at [URL].


\end{abstract}

\section{Introduction}
\section{Introduction}

Video generation has garnered significant attention owing to its transformative potential across a wide range of applications, such media content creation~\citep{polyak2024movie}, advertising~\citep{zhang2024virbo,bacher2021advert}, video games~\citep{yang2024playable,valevski2024diffusion, oasis2024}, and world model simulators~\citep{ha2018world, videoworldsimulators2024, agarwal2025cosmos}. Benefiting from advanced generative algorithms~\citep{goodfellow2014generative, ho2020denoising, liu2023flow, lipman2023flow}, scalable model architectures~\citep{vaswani2017attention, peebles2023scalable}, vast amounts of internet-sourced data~\citep{chen2024panda, nan2024openvid, ju2024miradata}, and ongoing expansion of computing capabilities~\citep{nvidia2022h100, nvidia2023dgxgh200, nvidia2024h200nvl}, remarkable advancements have been achieved in the field of video generation~\citep{ho2022video, ho2022imagen, singer2023makeavideo, blattmann2023align, videoworldsimulators2024, kuaishou2024klingai, yang2024cogvideox, jin2024pyramidal, polyak2024movie, kong2024hunyuanvideo, ji2024prompt}.


In this work, we present \textbf{\ours}, a family of rectified flow~\citep{lipman2023flow, liu2023flow} transformer models designed for joint image and video generation, establishing a pathway toward industry-grade performance. This report centers on four key components: data curation, model architecture design, flow formulation, and training infrastructure optimization—each rigorously refined to meet the demands of high-quality, large-scale video generation.


\begin{figure}[ht]
    \centering
    \begin{subfigure}[b]{0.82\linewidth}
        \centering
        \includegraphics[width=\linewidth]{figures/t2i_1024.pdf}
        \caption{Text-to-Image Samples}\label{fig:main-demo-t2i}
    \end{subfigure}
    \vfill
    \begin{subfigure}[b]{0.82\linewidth}
        \centering
        \includegraphics[width=\linewidth]{figures/t2v_samples.pdf}
        \caption{Text-to-Video Samples}\label{fig:main-demo-t2v}
    \end{subfigure}
\caption{\textbf{Generated samples from \ours.} Key components are highlighted in \textcolor{red}{\textbf{RED}}.}\label{fig:main-demo}
\end{figure}


First, we present a comprehensive data processing pipeline designed to construct large-scale, high-quality image and video-text datasets. The pipeline integrates multiple advanced techniques, including video and image filtering based on aesthetic scores, OCR-driven content analysis, and subjective evaluations, to ensure exceptional visual and contextual quality. Furthermore, we employ multimodal large language models~(MLLMs)~\citep{yuan2025tarsier2} to generate dense and contextually aligned captions, which are subsequently refined using an additional large language model~(LLM)~\citep{yang2024qwen2} to enhance their accuracy, fluency, and descriptive richness. As a result, we have curated a robust training dataset comprising approximately 36M video-text pairs and 160M image-text pairs, which are proven sufficient for training industry-level generative models.

Secondly, we take a pioneering step by applying rectified flow formulation~\citep{lipman2023flow} for joint image and video generation, implemented through the \ours model family, which comprises Transformer architectures with 2B and 8B parameters. At its core, the \ours framework employs a 3D joint image-video variational autoencoder (VAE) to compress image and video inputs into a shared latent space, facilitating unified representation. This shared latent space is coupled with a full-attention~\citep{vaswani2017attention} mechanism, enabling seamless joint training of image and video. This architecture delivers high-quality, coherent outputs across both images and videos, establishing a unified framework for visual generation tasks.


Furthermore, to support the training of \ours at scale, we have developed a robust infrastructure tailored for large-scale model training. Our approach incorporates advanced parallelism strategies~\citep{jacobs2023deepspeed, pytorch_fsdp} to manage memory efficiently during long-context training. Additionally, we employ ByteCheckpoint~\citep{wan2024bytecheckpoint} for high-performance checkpointing and integrate fault-tolerant mechanisms from MegaScale~\citep{jiang2024megascale} to ensure stability and scalability across large GPU clusters. These optimizations enable \ours to handle the computational and data challenges of generative modeling with exceptional efficiency and reliability.


We evaluate \ours on both text-to-image and text-to-video benchmarks to highlight its competitive advantages. For text-to-image generation, \ours-T2I demonstrates strong performance across multiple benchmarks, including T2I-CompBench~\citep{huang2023t2i-compbench}, GenEval~\citep{ghosh2024geneval}, and DPG-Bench~\citep{hu2024ella_dbgbench}, excelling in both visual quality and text-image alignment. In text-to-video benchmarks, \ours-T2V achieves state-of-the-art performance on the UCF-101~\citep{ucf101} zero-shot generation task. Additionally, \ours-T2V attains an impressive score of \textbf{84.85} on VBench~\citep{huang2024vbench}, securing the top position on the leaderboard (as of 2025-01-25) and surpassing several leading commercial text-to-video models. Qualitative results, illustrated in \Cref{fig:main-demo}, further demonstrate the superior quality of the generated media samples. These findings underscore \ours's effectiveness in multi-modal generation and its potential as a high-performing solution for both research and commercial applications.

% \begin{figure*}[t]
%     \centering
%     \includegraphics[width=1\textwidth]{latex/picture/digital_twin.pdf}
%     \caption{An overview of behavior chain benchmark framework.}
%     \label{fig:workflow}
% \end{figure*}




\section{Related Work}
\section{Related Work}
\label{sec:relatedwork}

\subsection{Current AI Tools for Social Service}
\label{subsec:relatedtools}
% the title I feel is quite broad

Harnessing technology for social good has always been a grand challenge in social service \cite{berzin_practice_2015}. As early as the 90s, artificial neural networks and predictive models have been employed as tools for risk assessments, decision-making, and workload management in sectors like child protective services and mental health treatment \cite{fluke_artificial_1989, patterson_application_1999}. The recent rise of generative AI is poised to further advance social service practice, facilitating the automation of administrative tasks, streamlining of paperwork and documentation, optimisation of resource allocation, data analysis, and enhancing client support and interventions \cite{fernando_integration_2023, perron_generative_2023}.

Today, AI solutions are increasingly being deployed in both policy and practice \cite{goldkind_social_2021, hodgson_problematising_2022}. In clinical social work, AI has been used for risk assessments, crisis management, public health initiatives, and education and training for practitioners \cite{asakura_call_2020, gillingham2019can, jacobi_functions_2023, liedgren_use_2016, molala_social_2023, rice_piloting_2018, tambe_artificial_2018}. AI has also been employed for mental health support and therapeutic interventions, with conversational agents serving as on-demand virtual counsellors to provide clinical care and support \cite{lisetti_i_2013, reamer_artificial_2023}.
% commercial solutions include Woebot, which simulates therapeutic conversation, and Wysa, an “emotionally intelligent” AI coach, powered by evidenced-based clinical techniques \cite{reamer_artificial_2023}. 
% Non-clinical AI agents like Replika and companion robots can also provide social support and reduce loneliness amongst individuals \cite{ahmed_humanrobot_2024, chaturvedi_social_2023, pani_can_2024, ta_user_2020}.

Present research largely focuses on \textit{\textbf{AI-based decision support tools}} in social service \cite{james_algorithmic_2023, kawakami2022improving}, especially predictive risk models (PRMs) used to predict social service risks and outcomes \cite{gillingham2019can, van2017predicting}, like the Allegheny Family Screening Tool (AFST), which assesses child abuse risk using data from US public systems \cite{chouldechova_case_2018, vaithianathan2017developing}. Elsewhere, researchers have also piloted PRMs to predict social service needs for the homeless using Medicaid data\cite{erickson_automatic_2018, pourat_easy_2023}, and AI-powered algorithms to promote health interventions for at-risk populations, such as HIV testing among Californian homeless \cite{rice_piloting_2018, yadav_maximizing_2017}.

\subsection{Generative AI and Human-AI Collaboration}
\label{subsec:relatedworkhaicollaboration}
Beyond decision-making algorithms and PRMs, advancements in generative AI, such as large language models (LLMs), open new possibilities for human-AI (HAI) collaboration in social services. 
LLMs have been called "revolutionary" \cite{fui2023generative} and a "seismic shift" \cite{cooper2023examining}, offering "content support" \cite{memmert2023towards} by generating realistic and coherent responses to user inputs \cite{cascella2023evaluating}. Their vastly improved capabilities and ubiquity \cite{cooper2023examining} makes them poised to revolutionise work patterns \cite{fui2023generative}. Generative AI is already used in fields like design, writing, music, \cite{han2024teams, suh2021ai, verheijden2023collaborative, dhillon2024shaping, gero2023social} healthcare, and clinical settings \cite{zhang2023generative, yu2023leveraging, biswas2024intelligent}, with promising results. However, the social service sector has been slower in adopting AI \cite{diez2023artificial, kawakami2023training}.

% Yet, the social service sector is one that could perhaps stand to gain the most from AI technologies. As Goldkind \cite{goldkind_social_2021} writes, social service, as a "values-centred profession with a robust code of ethics" (p. 372), is uniquely placed to inform the development of thoughtful algorithmic policy and practice. 
Social service, however, stands to benefit immensely from generative AI. SSPs work in time-poor environments \cite{tiah_can_2024}, often overwhelmed with tedious administrative work \cite{meilvang_working_2023} and large amounts of paperwork and data processing \cite{singer_ai_2023, tiah_can_2024}. 
% As such, workers often work in time-poor environments and are burdened with information overload and administrative tasks \cite{tiah_can_2024, meilvang_working_2023}. 
Generative AI is well-placed to streamline and automate tasks like formatting case notes, formulating treatment plans and writing progress reports, which can free up valuable time for more meaningful work like client engagement and enhance service quality \cite{fernando_integration_2023, perron_generative_2023, tiah_can_2024, thesocialworkaimentor_ai_nodate}. 

Given the immense potential, there has been emerging research interest in HAI collaboration and teamwork in the Human-Computer Interaction and Computer Supported Cooperative Work space \cite{wang_human-human_2020}. HAI collaboration and interaction has been postulated by researchers to contribute to new forms of HAI symbiosis and augmented intelligence, where algorithmic and human agents work in tandem with one another to perform tasks better than they could accomplish alone by augmenting each other's strengths and capabilities  \cite{dave_augmented_2023, jarrahi_artificial_2018}.

However, compared to the focus on AI decision-making and PRM tools, there is scant research on generative AI and HAI collaboration in the social service sector \cite{wykman_artificial_2023}. This study therefore seeks to fill this critical gap by exploring how SSPs use and interact with a novel generative AI tool, helping to expand our understanding of the new opportunities that HAI collaboration can bring to the social service sector.

\subsection{Challenges in AI Use in Social Service}
\label{subsec:relatedworkaiuse}

% Despite the immense potential of AI systems to augment social work practice, there are multiple challenges with integrating such systems into real-life practice. 
Despite its evident benefits, multiple challenges plague the integration of AI and its vast potential into real-life social service practice.
% Numerous studies have investigated the use of PRMs to help practitioners decide on a course of action for their clients. 
When employing algorithmic decision-making systems, practitioners often experience tension in weighing AI suggestions against their own judgement \cite{kawakami2022improving, saxena2021framework}, being uncertain of how far they should rely on the machine. 
% Despite often being instructed to use the tool as part of evaluating a client, 
Workers are often reluctant to fully embrace AI assessments due to its inability to adequately account for the full context of a case \cite{kawakami2022improving, gambrill2001need}, and lack of clarity and transparency on AI systems and limitations \cite{kawakami2022improving}. Brown et al. \cite{brown2019toward} conducted workshops using hypothetical algorithmic tools 
% to understand service providers' comfort levels with using such tools in their work,
and found similar issues with mistrust and perceived unreliability. Furthermore, introducing AI tools can  create new problems of its own, causing confusion and distrust amongst workers \cite{kawakami2022improving}. Such factors are critical barriers to the acceptance and effective use of AI in the sector.

\citeauthor{meilvang_working_2023} (2023) cites the concept of \textit{boundary work}, which explores the delineation between "monotonous" administrative labour and "professional", "knowledge based" work drawing on core competencies of SSPs. While computers have long been used for bureaucratic tasks like client registration, the introduction of decision support systems like PRMs stirred debate over AI "threatening professional discretion and, as such, the profession itself" \cite{meilvang_working_2023}. Such latent concerns arguably drive the resistance to technology adoption described above. Generative AI is only set to further push this boundary, 
% these concerns are only set to grow in tandem with the vast capabilities of generative and other modern AI systems. Compared to the relatively primitive AI systems in past years, perceived as statistical algorithms \cite{brown2019toward} turning preset inputs like client age and behavioural symptoms \cite{vaithianathan2017developing} into simple numerical outputs indicating various risk scores, modern AI systems are vastly more capable: LLMs 
with its ability to formulate detailed reports and assessments that encroach upon the "core" work of SSPs.
% accept unrestricted and unstructured inputs and return a range of verbose and detailed evaluations according to the user's instructions. 
Introducing these systems exacerbate previously-raised issues such as understanding the limitations and possibilities of AI systems \cite{kawakami2022improving} and risk of overreliance on AI \cite{van2023chatgpt}, and requires a re-examination of where users fall on the algorithmic aversion-bias scale \cite{brown2019toward} and how they detect and react to algorithmic failings \cite{de2020case}. We address these critical issues through an empirical, on-the-ground study that to our knowledge is the first of its kind since the new wave of generative AI.

% W 

% Yet, to date, we have limited knowledge on the real-world impacts and implications of human-AI collaboration, and few studies have investigated practitioners’ experiences working with and using such AI systems in practice, especially within the social work context \cite{kawakami2022improving}. A small number of studies have explored practitioner perspectives on the use of AI in social work, including Kawakami et al. \cite{kawakami2022improving}, who interviewed social workers on their experiences using the AFST; Stapleton et al. \cite{stapleton_imagining_2022}, who conducted design workshops with caseworkers on the use of PRMs in child welfare; and Wassal et al. \cite{wassal_reimagining_2024}, who interviewed UK social work professionals on the use of AI. A common thread from all these studies was a general disregard for the context and users, with many practitioners criticising the failure of past AI tools arising from the lack of participation and involvement of social workers and actual users of such systems in the design and development of algorithmic systems \cite{wassal_reimagining_2024}. Similarly, in a scoping review done on decision-support algorithms in social work, Jacobi \& Christensen \cite{jacobi_functions_2023} reported that the majority of studies reveal limited bottom-up involvement and interaction between social workers, researchers and developers, and that algorithms were rarely developed with consideration of the perspective of social workers.
% so the \cite{yang_unremarkable_2019} and \cite{holten_moller_shifting_2020} are not real-world impacts? real-world means to hear practitioner's voice? I feel this is quite important but i didnt get this point in intro!

% why mentioning 'which have largely focused on existing ADS tools (e.g., AFST)'? i can see our strength is more localized, but without basic knowledge of social work i didnt get what's the 'departure' here orz
% the paragraph is great! do we need to also add one in line 20 21?

\subsection{Designing AI for Social Service through Participatory Design}
\label{subsec:relatedworkpd}
% i think it's important! but maybe not a whole subsection? but i feel the strong connection with practitioners is indeed one of our novelties and need to highlight it, also in intro maybe
% Participatory design (PD) has long been used extensively in HCI \cite{muller1993participatory}, to both design effective solutions for a specific community and gain a deep understanding of that community. Of particular interest here is the rich body of literature on PD in the field of healthcare \cite{donetto2015experience}, which in this regard shares many similarities and concerns with social work. PD has created effective health improvement apps \cite{ryu2017impact}, 

% PD offers researchers the chance to gather detailed user requirements \cite{ryu2017impact}...

Participatory design (PD) is a staple of HCI research \cite{muller1993participatory}, facilitating the design of effective solutions for a specific community while gaining a deep understanding of its stakeholders. The focus in PD of valuing the opinions and perspectives of users as experts \cite{schuler_participatory_1993} 
% In recent years, the tech and social work sectors have awakened to the importance of involving real users in designing and implementing digital technologies, developing human-centred design processes to iteratively design products or technologies through user feedback 
has gained importance in recent years \cite{storer2023reimagining}. Responding to criticisms and failures of past AI tools that have been implemented without adequate involvement and input from actual users, HCI scholars have adopted PD approaches to design predictive tools to better support human decision-making \cite{lehtiniemi_contextual_2023}.
% ; accordingly, in social service, a line of research has begun studying and designing for human-AI collaboration with real-world users (e.g. \cite{holten_moller_shifting_2020, kawakami2022improving, yang_unremarkable_2019}).
Section \ref{subsec:relatedworkaiuse} shows a clear need to better understand SSP perspectives when designing and implementing AI tools in the social sector. 
Yet, PD research in this area has been limited. \citeauthor{yang2019unremarkable} (2019), through field evaluation with clinicians, investigated reasons behind the failure of previous AI-powered decision support tools, allowing them to design a new-and-improved AI decision-support tool that was better aligned with healthcare workers’ workflows. Similarly, \citeauthor{holten_moller_shifting_2020} (2020) ran PD workshops with caseworkers, data scientists and developers in public service systems to identify the expectations and needs that different stakeholders had in using ADS tools.

% Indeed, it is as Wise \cite{wise_intelligent_1998} noted so many years ago on the rise of intelligent agents: “it is perhaps when technologies are new, when their (and our) movements, habits and attitudes seem most awkward and therefore still at the forefront of our thoughts that they are easiest to analyse” (p. 411). 
Building upon this existing body of work, we thus conduct a study to co-design an AI tool \textit{for} and \textit{with} SSPs through participatory workshops and focus group discussions. In the process, we revisit many of the issues mentioned in Section \ref{subsec:relatedworkaiuse}, but in the context of novel generative AI systems, which are fundamentally different from most historical examples of automation technologies \cite{noy2023experimental}. This valuable empirical inquiry occurs at an opportune time when varied expectations about this nascent technology abound \cite{lehtiniemi_contextual_2023}, allowing us to understand how SSPs incorporate AI into their practice, and what AI can (or cannot) do for them. In doing so, we aim to uncover new theoretical and practical insights on what AI can bring to the social service sector, and formulate design implications for developing AI technologies that SSPs find truly meaningful and useful.
% , and drive future technological innovations to transform the social service sector not just within [our country], but also on a global scale.

 % with an on-the-ground study using a real prototype system that reflects the state of AI in current society. With the presumption that AI will continue to be used in social work given the great benefits it brings, we address the pressing need to investigate these issues to ensure that any potential AI systems are designed and implemented in a responsible and effective manner.

% Building upon these works, this study therefore seeks to adopt a participatory design methodology to investigate social workers’ perspectives and attitudes on AI and human-AI collaboration in their social work practice, thus contributing to the nascent body of practitioner-centred HCI research on the use of AI in social work. Yet, in a departure from prior work, which have largely focused on existing ADS tools (e.g., AFST) and were situated in a Western context, our paper also aims to expand the scope by piloting a novel generative AI tool that was designed and developed by the researchers in partnership with a social service agency based in Singapore, with aims of generating more insights on wider use cases of AI beyond what has been previously studied.

% i may think 'While the current lacunae of research on applications of AI in social work may appear to be a limitation, it simultaneously presents an exciting opportunity for further research and exploration \cite{dey_unleashing_2023},' this point is already convincing enough, not sure if we need to quote here
% I like this end! it's a good transition to our study design, do we need to mention the localization in intro as well? like we target at singapore

% Given the increasing prominence and acceptance of AI in modern society, 

% These increased capabilities vastly exacerbate the issues already present with a simpler tool like the AFST: the boundaries and limitations of an LLM system are significantly more difficult to understand and its possible use cases are exponentially greater in scope. 

% Put this in discussion section instead?
% Kawakami et al's work "highlights the importance of studying how collaborative decision-making... impacts how people rely upon and make sense of AI models," They conclude by recommending designing tools that "support workers in understanding the boundaries of [an AI system's] capabilities", and implementing design procedures that "support open cultures for critical discussion around AI decision making". The authors outline critical challenges of implementing AI systems, elucidating factors that may hinder their effectiveness and even negatively affect operations within the organisation.


% Is this needed?:
% talk about the strengths of PD in eliciting user viewpoints and knowledge, in particular when it is a field that is novel or where a certain system has not been used or developed or tested before

\begin{figure*}[t]
    \centering
    \includegraphics[width=1\linewidth]{latex/picture/behaviorchain.pdf}
    \caption{The left image illustrates the process of behavior chain construction. The right image shows our multi-choice task setup, where each node on the behavior chain serves as an individual evaluation. Each input includes persona's profile, history, and all context and ground truth behavior preceding the current node.}
    \label{fig:workflow}
\end{figure*}


\section{Benchmark Crafting}
\section{Study Design}
% robot: aliengo 
% We used the Unitree AlienGo quadruped robot. 
% See Appendix 1 in AlienGo Software Guide PDF
% Weight = 25kg, size (L,W,H) = (0.55, 0.35, 06) m when standing, (0.55, 0.35, 0.31) m when walking
% Handle is 0.4 m or 0.5 m. I'll need to check it to see which type it is.
We gathered input from primary stakeholders of the robot dog guide, divided into three subgroups: BVI individuals who have owned a dog guide, BVI individuals who were not dog guide owners, and sighted individuals with generally low degrees of familiarity with dog guides. While the main focus of this study was on the BVI participants, we elected to include survey responses from sighted participants given the importance of social acceptance of the robot by the general public, which could reflect upon the BVI users themselves and affect their interactions with the general population \cite{kayukawa2022perceive}. 

The need-finding processes consisted of two stages. During Stage 1, we conducted in-depth interviews with BVI participants, querying their experiences in using conventional assistive technologies and dog guides. During Stage 2, a large-scale survey was distributed to both BVI and sighted participants. 

This study was approved by the University’s Institutional Review Board (IRB), and all processes were conducted after obtaining the participants' consent.

\subsection{Stage 1: Interviews}
We recruited nine BVI participants (\textbf{Table}~\ref{tab:bvi-info}) for in-depth interviews, which lasted 45-90 minutes for current or former dog guide owners (DO) and 30-60 minutes for participants without dog guides (NDO). Group DO consisted of five participants, while Group NDO consisted of four participants.
% The interview participants were divided into two groups. Group DO (Dog guide Owner) consisted of five participants who were current or former dog guide owners and Group NDO (Non Dog guide Owner) consisted of three participants who were not dog guide owners. 
All participants were familiar with using white canes as a mobility aid. 

We recruited participants in both groups, DO and NDO, to gather data from those with substantial experience with dog guides, offering potentially more practical insights, and from those without prior experience, providing a perspective that may be less constrained and more open to novel approaches. 

We asked about the participants' overall impressions of a robot dog guide, expectations regarding its potential benefits and challenges compared to a conventional dog guide, their desired methods of giving commands and communicating with the robot dog guide, essential functionalities that the robot dog guide should offer, and their preferences for various aspects of the robot dog guide's form factors. 
For Group DO, we also included questions that asked about the participants' experiences with conventional dog guides. 

% We obtained permission to record the conversations for our records while simultaneously taking notes during the interviews. The interviews lasted 30-60 minutes for NDO participants and 45-90 minutes for DO participants. 

\subsection{Stage 2: Large-Scale Surveys} 
After gathering sufficient initial results from the interviews, we created an online survey for distributing to a larger pool of participants. The survey platform used was Qualtrics. 

\subsubsection{Survey Participants}
The survey had 100 participants divided into two primary groups. Group BVI consisted of 42 blind or visually impaired participants, and Group ST consisted of 58 sighted participants. \textbf{Table}~\ref{tab:survey-demographics} shows the demographic information of the survey participants. 

\subsubsection{Question Differentiation} 
Based on their responses to initial qualifying questions, survey participants were sorted into three subgroups: DO, NDO, and ST. Each participant was assigned one of three different versions of the survey. The surveys for BVI participants mirrored the interview categories (overall impressions, communication methods, functionalities, and form factors), but with a more quantitative approach rather than the open-ended questions used in interviews. The DO version included additional questions pertaining to their prior experience with dog guides. The ST version revolved around the participants' prior interactions with and feelings toward dog guides and dogs in general, their thoughts on a robot dog guide, and broad opinions on the aesthetic component of the robot's design. 



\section{Experiments}
\section{Experiments}

\subsection{Setups}
\subsubsection{Implementation Details}
We apply our FDS method to two types of 3DGS: 
the original 3DGS, and 2DGS~\citep{huang20242d}. 
%
The number of iterations in our optimization 
process is 35,000.
We follow the default training configuration 
and apply our FDS method after 15,000 iterations,
then we add normal consistency loss for both
3DGS and 2DGS after 25000 iterations.
%
The weight for FDS, $\lambda_{fds}$, is set to 0.015,
the $\sigma$ is set to 23,
and the weight for normal consistency is set to 0.15
for all experiments. 
We removed the depth distortion loss in 2DGS 
because we found that it degrades its results in indoor scenes.
%
The Gaussian point cloud is initialized using Colmap
for all datasets.
%
%
We tested the impact of 
using Sea Raft~\citep{wang2025sea} and 
Raft\citep{teed2020raft} on FDS performance.
%
Due to the blurriness of the ScanNet dataset, 
additional prior constraints are required.
Thus, we incorporate normal prior supervision 
on the rendered normals 
in ScanNet (V2) dataset by default.
The normal prior is predicted by the Stable Normal 
model~\citep{ye2024stablenormal}
across all types of 3DGS.
%
The entire framework is implemented in 
PyTorch~\citep{paszke2019pytorch}, 
and all experiments are conducted on 
a single NVIDIA 4090D GPU.

\begin{figure}[t] \centering
    \makebox[0.16\textwidth]{\scriptsize Input}
    \makebox[0.16\textwidth]{\scriptsize 3DGS}
    \makebox[0.16\textwidth]{\scriptsize 2DGS}
    \makebox[0.16\textwidth]{\scriptsize 3DGS + FDS}
    \makebox[0.16\textwidth]{\scriptsize 2DGS + FDS}
    \makebox[0.16\textwidth]{\scriptsize GT (Depth)}

    \includegraphics[width=0.16\textwidth]{figure/fig3_img/compare3/gt_rgb/frame_00522.jpg}
    \includegraphics[width=0.16\textwidth]{figure/fig3_img/compare3/3DGS/frame_00522.jpg}
    \includegraphics[width=0.16\textwidth]{figure/fig3_img/compare3/2DGS/frame_00522.jpg}
    \includegraphics[width=0.16\textwidth]{figure/fig3_img/compare3/3DGS+FDS/frame_00522.jpg}
    \includegraphics[width=0.16\textwidth]{figure/fig3_img/compare3/2DGS+FDS/frame_00522.jpg}
    \includegraphics[width=0.16\textwidth]{figure/fig3_img/compare3/gt_depth/frame_00522.jpg} \\

    % \includegraphics[width=0.16\textwidth]{figure/fig3_img/compare1/gt_rgb/frame_00137.jpg}
    % \includegraphics[width=0.16\textwidth]{figure/fig3_img/compare1/3DGS/frame_00137.jpg}
    % \includegraphics[width=0.16\textwidth]{figure/fig3_img/compare1/2DGS/frame_00137.jpg}
    % \includegraphics[width=0.16\textwidth]{figure/fig3_img/compare1/3DGS+FDS/frame_00137.jpg}
    % \includegraphics[width=0.16\textwidth]{figure/fig3_img/compare1/2DGS+FDS/frame_00137.jpg}
    % \includegraphics[width=0.16\textwidth]{figure/fig3_img/compare1/gt_depth/frame_00137.jpg} \\

     \includegraphics[width=0.16\textwidth]{figure/fig3_img/compare2/gt_rgb/frame_00262.jpg}
    \includegraphics[width=0.16\textwidth]{figure/fig3_img/compare2/3DGS/frame_00262.jpg}
    \includegraphics[width=0.16\textwidth]{figure/fig3_img/compare2/2DGS/frame_00262.jpg}
    \includegraphics[width=0.16\textwidth]{figure/fig3_img/compare2/3DGS+FDS/frame_00262.jpg}
    \includegraphics[width=0.16\textwidth]{figure/fig3_img/compare2/2DGS+FDS/frame_00262.jpg}
    \includegraphics[width=0.16\textwidth]{figure/fig3_img/compare2/gt_depth/frame_00262.jpg} \\

    \includegraphics[width=0.16\textwidth]{figure/fig3_img/compare4/gt_rgb/frame00000.png}
    \includegraphics[width=0.16\textwidth]{figure/fig3_img/compare4/3DGS/frame00000.png}
    \includegraphics[width=0.16\textwidth]{figure/fig3_img/compare4/2DGS/frame00000.png}
    \includegraphics[width=0.16\textwidth]{figure/fig3_img/compare4/3DGS+FDS/frame00000.png}
    \includegraphics[width=0.16\textwidth]{figure/fig3_img/compare4/2DGS+FDS/frame00000.png}
    \includegraphics[width=0.16\textwidth]{figure/fig3_img/compare4/gt_depth/frame00000.png} \\

    \includegraphics[width=0.16\textwidth]{figure/fig3_img/compare5/gt_rgb/frame00080.png}
    \includegraphics[width=0.16\textwidth]{figure/fig3_img/compare5/3DGS/frame00080.png}
    \includegraphics[width=0.16\textwidth]{figure/fig3_img/compare5/2DGS/frame00080.png}
    \includegraphics[width=0.16\textwidth]{figure/fig3_img/compare5/3DGS+FDS/frame00080.png}
    \includegraphics[width=0.16\textwidth]{figure/fig3_img/compare5/2DGS+FDS/frame00080.png}
    \includegraphics[width=0.16\textwidth]{figure/fig3_img/compare5/gt_depth/frame00080.png} \\



    \caption{\textbf{Comparison of depth reconstruction on Mushroom and ScanNet datasets.} The original
    3DGS or 2DGS model equipped with FDS can remove unwanted floaters and reconstruct
    geometry more preciously.}
    \label{fig:compare}
\end{figure}


\subsubsection{Datasets and Metrics}

We evaluate our method for 3D reconstruction 
and novel view synthesis tasks on
\textbf{Mushroom}~\citep{ren2024mushroom},
\textbf{ScanNet (v2)}~\citep{dai2017scannet}, and 
\textbf{Replica}~\citep{replica19arxiv}
datasets,
which feature challenging indoor scenes with both 
sparse and dense image sampling.
%
The Mushroom dataset is an indoor dataset 
with sparse image sampling and two distinct 
camera trajectories. 
%
We train our model on the training split of 
the long capture sequence and evaluate 
novel view synthesis on the test split 
of the long capture sequences.
%
Five scenes are selected to evaluate our FDS, 
including "coffee room", "honka", "kokko", 
"sauna", and "vr room". 
%
ScanNet(V2)~\citep{dai2017scannet}  consists of 1,613 indoor scenes
with annotated camera poses and depth maps. 
%
We select 5 scenes from the ScanNet (V2) dataset, 
uniformly sampling one-tenth of the views,
following the approach in ~\citep{guo2022manhattan}.
To further improve the geometry rendering quality of 3DGS, 
%
Replica~\citep{replica19arxiv} contains small-scale 
real-world indoor scans. 
We evaluate our FDS on five scenes from 
Replica: office0, office1, office2, office3 and office4,
selecting one-tenth of the views for training.
%
The results for Replica are provided in the 
supplementary materials.
To evaluate the rendering quality and geometry 
of 3DGS, we report PSNR, SSIM, and LPIPS for 
rendering quality, along with Absolute Relative Distance 
(Abs Rel) as a depth quality metrics.
%
Additionally, for mesh evaluation, 
we use metrics including Accuracy, Completion, 
Chamfer-L1 distance, Normal Consistency, 
and F-scores.




\subsection{Results}
\subsubsection{Depth rendering and novel view synthesis}
The comparison results on Mushroom and 
ScanNet are presented in \tabref{tab:mushroom} 
and \tabref{tab:scannet}, respectively. 
%
Due to the sparsity of sampling 
in the Mushroom dataset,
challenges are posed for both GOF~\citep{yu2024gaussian} 
and PGSR~\citep{chen2024pgsr}, 
leading to their relative poor performance 
on the Mushroom dataset.
%
Our approach achieves the best performance 
with the FDS method applied during the training process.
The FDS significantly enhances the 
geometric quality of 3DGS on the Mushroom dataset, 
improving the "abs rel" metric by more than 50\%.
%
We found that Sea Raft~\citep{wang2025sea}
outperforms Raft~\citep{teed2020raft} on FDS, 
indicating that a better optical flow model 
can lead to more significant improvements.
%
Additionally, the render quality of RGB 
images shows a slight improvement, 
by 0.58 in 3DGS and 0.50 in 2DGS, 
benefiting from the incorporation of cross-view consistency in FDS. 
%
On the Mushroom
dataset, adding the FDS loss increases 
the training time by half an hour, which maintains the same
level as baseline.
%
Similarly, our method shows a notable improvement on the ScanNet dataset as well using Sea Raft~\citep{wang2025sea} Model. The "abs rel" metric in 2DGS is improved nearly 50\%. This demonstrates the robustness and effectiveness of the FDS method across different datasets.
%


% \begin{wraptable}{r}{0.6\linewidth} \centering
% \caption{\textbf{Ablation study on geometry priors.}} 
%         \label{tab:analysis_prior}
%         \resizebox{\textwidth}{!}{
\begin{tabular}{c| c c c c c | c c c c}

    \hline
     Method &  Acc$\downarrow$ & Comp $\downarrow$ & C-L1 $\downarrow$ & NC $\uparrow$ & F-Score $\uparrow$ &  Abs Rel $\downarrow$ &  PSNR $\uparrow$  & SSIM  $\uparrow$ & LPIPS $\downarrow$ \\ \hline
    2DGS&   0.1078&  0.0850&  0.0964&  0.7835&  0.5170&  0.1002&  23.56&  0.8166& 0.2730\\
    2DGS+Depth&   0.0862&  0.0702&  0.0782&  0.8153&  0.5965&  0.0672&  23.92&  0.8227& 0.2619 \\
    2DGS+MVDepth&   0.2065&  0.0917&  0.1491&  0.7832&  0.3178&  0.0792&  23.74&  0.8193& 0.2692 \\
    2DGS+Normal&   0.0939&  0.0637&  0.0788&  \textbf{0.8359}&  0.5782&  0.0768&  23.78&  0.8197& 0.2676 \\
    2DGS+FDS &  \textbf{0.0615} & \textbf{ 0.0534}& \textbf{0.0574}& 0.8151& \textbf{0.6974}&  \textbf{0.0561}&  \textbf{24.06}&  \textbf{0.8271}&\textbf{0.2610} \\ \hline
    2DGS+Depth+FDS &  0.0561 &  0.0519& 0.0540& 0.8295& 0.7282&  0.0454&  \textbf{24.22}& \textbf{0.8291}&\textbf{0.2570} \\
    2DGS+Normal+FDS &  \textbf{0.0529} & \textbf{ 0.0450}& \textbf{0.0490}& \textbf{0.8477}& \textbf{0.7430}&  \textbf{0.0443}&  24.10&  0.8283& 0.2590 \\
    2DGS+Depth+Normal &  0.0695 & 0.0513& 0.0604& 0.8540&0.6723&  0.0523&  24.09&  0.8264&0.2575\\ \hline
    2DGS+Depth+Normal+FDS &  \textbf{0.0506} & \textbf{0.0423}& \textbf{0.0464}& \textbf{0.8598}&\textbf{0.7613}&  \textbf{0.0403}&  \textbf{24.22}& 
    \textbf{0.8300}&\textbf{0.0403}\\
    
\bottomrule
\end{tabular}
}
% \end{wraptable}



The qualitative comparisons on the Mushroom and ScanNet dataset 
are illustrated in \figref{fig:compare}. 
%
%
As seen in the first row of \figref{fig:compare}, 
both the original 3DGS and 2DGS suffer from overfitting, 
leading to corrupted geometry generation. 
%
Our FDS effectively mitigates this issue by 
supervising the matching relationship between 
the input and sampled views, 
helping to recover the geometry.
%
FDS also improves the refinement of geometric details, 
as shown in other rows. 
By incorporating the matching prior through FDS, 
the quality of the rendered depth is significantly improved.
%

\begin{table}[t] \centering
\begin{minipage}[t]{0.96\linewidth}
        \captionof{table}{\textbf{3D Reconstruction 
        and novel view synthesis results on Mushroom dataset. * 
        Represents that FDS uses the Raft model.
        }}
        \label{tab:mushroom}
        \resizebox{\textwidth}{!}{
\begin{tabular}{c| c c c c c | c c c c c}
    \hline
     Method &  Acc$\downarrow$ & Comp $\downarrow$ & C-L1 $\downarrow$ & NC $\uparrow$ & F-Score $\uparrow$ &  Abs Rel $\downarrow$ &  PSNR $\uparrow$  & SSIM  $\uparrow$ & LPIPS $\downarrow$ & Time  $\downarrow$ \\ \hline

    % DN-splatter &   &  &  &  &  &  &  &  & \\
    GOF &  0.1812 & 0.1093 & 0.1453 & 0.6292 & 0.3665 & 0.2380  & 21.37  &  0.7762  & 0.3132  & $\approx$1.4h\\ 
    PGSR &  0.0971 & 0.1420 & 0.1196 & 0.7193 & 0.5105 & 0.1723  & 22.13  & 0.7773  & 0.2918  & $\approx$1.2h \\ \hline
    3DGS &   0.1167 &  0.1033&  0.1100&  0.7954&  0.3739&  0.1214&  24.18&  0.8392& 0.2511 &$\approx$0.8h \\
    3DGS + FDS$^*$ & 0.0569  & 0.0676 & 0.0623 & 0.8105 & 0.6573 & 0.0603 & 24.72  & 0.8489 & 0.2379 &$\approx$1.3h \\
    3DGS + FDS & \textbf{0.0527}  & \textbf{0.0565} & \textbf{0.0546} & \textbf{0.8178} & \textbf{0.6958} & \textbf{0.0568} & \textbf{24.76}  & \textbf{0.8486} & \textbf{0.2381} &$\approx$1.3h \\ \hline
    2DGS&   0.1078&  0.0850&  0.0964&  0.7835&  0.5170&  0.1002&  23.56&  0.8166& 0.2730 &$\approx$0.8h\\
    2DGS + FDS$^*$ &  0.0689 &  0.0646& 0.0667& 0.8042& 0.6582& 0.0589& 23.98&  0.8255&0.2621 &$\approx$1.3h\\
    2DGS + FDS &  \textbf{0.0615} & \textbf{ 0.0534}& \textbf{0.0574}& \textbf{0.8151}& \textbf{0.6974}&  \textbf{0.0561}&  \textbf{24.06}&  \textbf{0.8271}&\textbf{0.2610} &$\approx$1.3h \\ \hline
\end{tabular}
}
\end{minipage}\hfill
\end{table}

\begin{table}[t] \centering
\begin{minipage}[t]{0.96\linewidth}
        \captionof{table}{\textbf{3D Reconstruction 
        and novel view synthesis results on ScanNet dataset.}}
        \label{tab:scannet}
        \resizebox{\textwidth}{!}{
\begin{tabular}{c| c c c c c | c c c c }
    \hline
     Method &  Acc $\downarrow$ & Comp $\downarrow$ & C-L1 $\downarrow$ & NC $\uparrow$ & F-Score $\uparrow$ &  Abs Rel $\downarrow$ &  PSNR $\uparrow$  & SSIM  $\uparrow$ & LPIPS $\downarrow$ \\ \hline
    GOF & 1.8671  & 0.0805 & 0.9738 & 0.5622 & 0.2526 & 0.1597  & 21.55  & 0.7575  & 0.3881 \\
    PGSR &  0.2928 & 0.5103 & 0.4015 & 0.5567 & 0.1926 & 0.1661  & 21.71 & 0.7699  & 0.3899 \\ \hline

    3DGS &  0.4867 & 0.1211 & 0.3039 & 0.7342& 0.3059 & 0.1227 & 22.19& 0.7837 & 0.3907\\
    3DGS + FDS &  \textbf{0.2458} & \textbf{0.0787} & \textbf{0.1622} & \textbf{0.7831} & 
    \textbf{0.4482} & \textbf{0.0573} & \textbf{22.83} & \textbf{0.7911} & \textbf{0.3826} \\ \hline
    2DGS &  0.2658 & 0.0845 & 0.1752 & 0.7504& 0.4464 & 0.0831 & 22.59& 0.7881 & 0.3854\\
    2DGS + FDS &  \textbf{0.1457} & \textbf{0.0679} & \textbf{0.1068} & \textbf{0.7883} & 
    \textbf{0.5459} & \textbf{0.0432} & \textbf{22.91} & \textbf{0.7928} & \textbf{0.3800} \\ \hline
\end{tabular}
}
\end{minipage}\hfill
\end{table}


\begin{table}[t] \centering
\begin{minipage}[t]{0.96\linewidth}
        \captionof{table}{\textbf{Ablation study on geometry priors.}}
        \label{tab:analysis_prior}
        \resizebox{\textwidth}{!}{
\begin{tabular}{c| c c c c c | c c c c}

    \hline
     Method &  Acc$\downarrow$ & Comp $\downarrow$ & C-L1 $\downarrow$ & NC $\uparrow$ & F-Score $\uparrow$ &  Abs Rel $\downarrow$ &  PSNR $\uparrow$  & SSIM  $\uparrow$ & LPIPS $\downarrow$ \\ \hline
    2DGS&   0.1078&  0.0850&  0.0964&  0.7835&  0.5170&  0.1002&  23.56&  0.8166& 0.2730\\
    2DGS+Depth&   0.0862&  0.0702&  0.0782&  0.8153&  0.5965&  0.0672&  23.92&  0.8227& 0.2619 \\
    2DGS+MVDepth&   0.2065&  0.0917&  0.1491&  0.7832&  0.3178&  0.0792&  23.74&  0.8193& 0.2692 \\
    2DGS+Normal&   0.0939&  0.0637&  0.0788&  \textbf{0.8359}&  0.5782&  0.0768&  23.78&  0.8197& 0.2676 \\
    2DGS+FDS &  \textbf{0.0615} & \textbf{ 0.0534}& \textbf{0.0574}& 0.8151& \textbf{0.6974}&  \textbf{0.0561}&  \textbf{24.06}&  \textbf{0.8271}&\textbf{0.2610} \\ \hline
    2DGS+Depth+FDS &  0.0561 &  0.0519& 0.0540& 0.8295& 0.7282&  0.0454&  \textbf{24.22}& \textbf{0.8291}&\textbf{0.2570} \\
    2DGS+Normal+FDS &  \textbf{0.0529} & \textbf{ 0.0450}& \textbf{0.0490}& \textbf{0.8477}& \textbf{0.7430}&  \textbf{0.0443}&  24.10&  0.8283& 0.2590 \\
    2DGS+Depth+Normal &  0.0695 & 0.0513& 0.0604& 0.8540&0.6723&  0.0523&  24.09&  0.8264&0.2575\\ \hline
    2DGS+Depth+Normal+FDS &  \textbf{0.0506} & \textbf{0.0423}& \textbf{0.0464}& \textbf{0.8598}&\textbf{0.7613}&  \textbf{0.0403}&  \textbf{24.22}& 
    \textbf{0.8300}&\textbf{0.0403}\\
    
\bottomrule
\end{tabular}
}
\end{minipage}\hfill
\end{table}




\subsubsection{Mesh extraction}
To further demonstrate the improvement in geometry quality, 
we applied methods used in ~\citep{turkulainen2024dnsplatter} 
to extract meshes from the input views of optimized 3DGS. 
The comparison results are presented  
in \tabref{tab:mushroom}. 
With the integration of FDS, the mesh quality is significantly enhanced compared to the baseline, featuring fewer floaters and more well-defined shapes.
 %
% Following the incorporation of FDS, the reconstruction 
% results exhibit fewer floaters and more well-defined 
% shapes in the meshes. 
% Visualized comparisons
% are provided in the supplementary material.

% \begin{figure}[t] \centering
%     \makebox[0.19\textwidth]{\scriptsize GT}
%     \makebox[0.19\textwidth]{\scriptsize 3DGS}
%     \makebox[0.19\textwidth]{\scriptsize 3DGS+FDS}
%     \makebox[0.19\textwidth]{\scriptsize 2DGS}
%     \makebox[0.19\textwidth]{\scriptsize 2DGS+FDS} \\

%     \includegraphics[width=0.19\textwidth]{figure/fig4_img/compare1/gt02.png}
%     \includegraphics[width=0.19\textwidth]{figure/fig4_img/compare1/baseline06.png}
%     \includegraphics[width=0.19\textwidth]{figure/fig4_img/compare1/baseline_fds05.png}
%     \includegraphics[width=0.19\textwidth]{figure/fig4_img/compare1/2dgs04.png}
%     \includegraphics[width=0.19\textwidth]{figure/fig4_img/compare1/2dgs_fds03.png} \\

%     \includegraphics[width=0.19\textwidth]{figure/fig4_img/compare2/gt00.png}
%     \includegraphics[width=0.19\textwidth]{figure/fig4_img/compare2/baseline02.png}
%     \includegraphics[width=0.19\textwidth]{figure/fig4_img/compare2/baseline_fds01.png}
%     \includegraphics[width=0.19\textwidth]{figure/fig4_img/compare2/2dgs04.png}
%     \includegraphics[width=0.19\textwidth]{figure/fig4_img/compare2/2dgs_fds03.png} \\
      
%     \includegraphics[width=0.19\textwidth]{figure/fig4_img/compare3/gt05.png}
%     \includegraphics[width=0.19\textwidth]{figure/fig4_img/compare3/3dgs03.png}
%     \includegraphics[width=0.19\textwidth]{figure/fig4_img/compare3/3dgs_fds04.png}
%     \includegraphics[width=0.19\textwidth]{figure/fig4_img/compare3/2dgs02.png}
%     \includegraphics[width=0.19\textwidth]{figure/fig4_img/compare3/2dgs_fds01.png} \\

%     \caption{\textbf{Qualitative comparison of extracted mesh 
%     on Mushroom and ScanNet datasets.}}
%     \label{fig:mesh}
% \end{figure}












\subsection{Ablation study}


\textbf{Ablation study on geometry priors:} 
To highlight the advantage of incorporating matching priors, 
we incorporated various types of priors generated by different 
models into 2DGS. These include a monocular depth estimation
model (Depth Anything v2)~\citep{yang2024depth}, a two-view depth estimation 
model (Unimatch)~\citep{xu2023unifying}, 
and a monocular normal estimation model (DSINE)~\citep{bae2024rethinking}.
We adapt the scale and shift-invariant loss in Midas~\citep{birkl2023midas} for
monocular depth supervision and L1 loss for two-view depth supervison.
%
We use Sea Raft~\citep{wang2025sea} as our default optical flow model.
%
The comparison results on Mushroom dataset 
are shown in ~\tabref{tab:analysis_prior}.
We observe that the normal prior provides accurate shape information, 
enhancing the geometric quality of the radiance field. 
%
% In contrast, the monocular depth prior slightly increases 
% the 'Abs Rel' due to its ambiguous scale and inaccurate depth ordering.
% Moreover, the performance of monocular depth estimation 
% in the sauna scene is particularly poor, 
% primarily due to the presence of numerous reflective 
% surfaces and textureless walls, which limits the accuracy of monocular depth estimation.
%
The multi-view depth prior, hindered by the limited feature overlap 
between input views, fails to offer reliable geometric 
information. We test average "Abs Rel" of multi-view depth prior
, and the result is 0.19, which performs worse than the "Abs Rel" results 
rendered by original 2DGS.
From the results, it can be seen that depth order information provided by monocular depth improves
reconstruction accuracy. Meanwhile, our FDS achieves the best performance among all the priors, 
and by integrating all
three components, we obtained the optimal results.
%
%
\begin{figure}[t] \centering
    \makebox[0.16\textwidth]{\scriptsize RF (16000 iters)}
    \makebox[0.16\textwidth]{\scriptsize RF* (20000 iters)}
    \makebox[0.16\textwidth]{\scriptsize RF (20000 iters)  }
    \makebox[0.16\textwidth]{\scriptsize PF (16000 iters)}
    \makebox[0.16\textwidth]{\scriptsize PF (20000 iters)}


    % \includegraphics[width=0.16\textwidth]{figure/fig5_img/compare1/16000.png}
    % \includegraphics[width=0.16\textwidth]{figure/fig5_img/compare1/20000_wo_flow_loss.png}
    % \includegraphics[width=0.16\textwidth]{figure/fig5_img/compare1/20000.png}
    % \includegraphics[width=0.16\textwidth]{figure/fig5_img/compare1/16000_prior.png}
    % \includegraphics[width=0.16\textwidth]{figure/fig5_img/compare1/20000_prior.png}\\

    % \includegraphics[width=0.16\textwidth]{figure/fig5_img/compare2/16000.png}
    % \includegraphics[width=0.16\textwidth]{figure/fig5_img/compare2/20000_wo_flow_loss.png}
    % \includegraphics[width=0.16\textwidth]{figure/fig5_img/compare2/20000.png}
    % \includegraphics[width=0.16\textwidth]{figure/fig5_img/compare2/16000_prior.png}
    % \includegraphics[width=0.16\textwidth]{figure/fig5_img/compare2/20000_prior.png}\\

    \includegraphics[width=0.16\textwidth]{figure/fig5_img/compare3/16000.png}
    \includegraphics[width=0.16\textwidth]{figure/fig5_img/compare3/20000_wo_flow_loss.png}
    \includegraphics[width=0.16\textwidth]{figure/fig5_img/compare3/20000.png}
    \includegraphics[width=0.16\textwidth]{figure/fig5_img/compare3/16000_prior.png}
    \includegraphics[width=0.16\textwidth]{figure/fig5_img/compare3/20000_prior.png}\\
    
    \includegraphics[width=0.16\textwidth]{figure/fig5_img/compare4/16000.png}
    \includegraphics[width=0.16\textwidth]{figure/fig5_img/compare4/20000_wo_flow_loss.png}
    \includegraphics[width=0.16\textwidth]{figure/fig5_img/compare4/20000.png}
    \includegraphics[width=0.16\textwidth]{figure/fig5_img/compare4/16000_prior.png}
    \includegraphics[width=0.16\textwidth]{figure/fig5_img/compare4/20000_prior.png}\\

    \includegraphics[width=0.30\textwidth]{figure/fig5_img/bar.png}

    \caption{\textbf{The error map of Radiance Flow and Prior Flow.} RF: Radiance Flow, PF: Prior Flow, * means that there is no FDS loss supervision during optimization.}
    \label{fig:error_map}
\end{figure}




\textbf{Ablation study on FDS: }
In this section, we present the design of our FDS 
method through an ablation study on the 
Mushroom dataset to validate its effectiveness.
%
The optional configurations of FDS are outlined in ~\tabref{tab:ablation_fds}.
Our base model is the 2DGS equipped with FDS,
and its results are shown 
in the first row. The goal of this analysis 
is to evaluate the impact 
of various strategies on FDS sampling and loss design.
%
We observe that when we 
replace $I_i$ in \eqref{equ:mflow} with $C_i$, 
as shown in the second row, the geometric quality 
of 2DGS deteriorates. Using $I_i$ instead of $C_i$ 
help us to remove the floaters in $\bm{C^s}$, which are also 
remained in $\bm{C^i}$.
We also experiment with modifying the FDS loss. For example, 
in the third row, we use the neighbor 
input view as the sampling view, and replace the 
render result of neighbor view with ground truth image of its input view.
%
Due to the significant movement between images, the Prior Flow fails to accurately 
match the pixel between them, leading to a further degradation in geometric quality.
%
Finally, we attempt to fix the sampling view 
and found that this severely damaged the geometric quality, 
indicating that random sampling is essential for the stability 
of the mean error in the Prior flow.



\begin{table}[t] \centering

\begin{minipage}[t]{1.0\linewidth}
        \captionof{table}{\textbf{Ablation study on FDS strategies.}}
        \label{tab:ablation_fds}
        \resizebox{\textwidth}{!}{
\begin{tabular}{c|c|c|c|c|c|c|c}
    \hline
    \multicolumn{2}{c|}{$\mathcal{M}_{\theta}(X, \bm{C^s})$} & \multicolumn{3}{c|}{Loss} & \multicolumn{3}{c}{Metric}  \\
    \hline
    $X=C^i$ & $X=I^i$  & Input view & Sampled view     & Fixed Sampled view        & Abs Rel $\downarrow$ & F-score $\uparrow$ & NC $\uparrow$ \\
    \hline
    & \ding{51} &     &\ding{51}    &    &    \textbf{0.0561}        &  \textbf{0.6974}         & \textbf{0.8151}\\
    \hline
     \ding{51} &           &     &\ding{51}    &    &    0.0839        &  0.6242         &0.8030\\
     &  \ding{51} &   \ding{51}  &    &    &    0.0877       & 0.6091        & 0.7614 \\
      &  \ding{51} &    &    & \ding{51}    &    0.0724           & 0.6312          & 0.8015 \\
\bottomrule
\end{tabular}
}
\end{minipage}
\end{table}




\begin{figure}[htbp] \centering
    \makebox[0.22\textwidth]{}
    \makebox[0.22\textwidth]{}
    \makebox[0.22\textwidth]{}
    \makebox[0.22\textwidth]{}
    \\

    \includegraphics[width=0.22\textwidth]{figure/fig6_img/l1/rgb/frame00096.png}
    \includegraphics[width=0.22\textwidth]{figure/fig6_img/l1/render_rgb/frame00096.png}
    \includegraphics[width=0.22\textwidth]{figure/fig6_img/l1/render_depth/frame00096.png}
    \includegraphics[width=0.22\textwidth]{figure/fig6_img/l1/depth/frame00096.png}

    % \includegraphics[width=0.22\textwidth]{figure/fig6_img/l2/rgb/frame00112.png}
    % \includegraphics[width=0.22\textwidth]{figure/fig6_img/l2/render_rgb/frame00112.png}
    % \includegraphics[width=0.22\textwidth]{figure/fig6_img/l2/render_depth/frame00112.png}
    % \includegraphics[width=0.22\textwidth]{figure/fig6_img/l2/depth/frame00112.png}

    \caption{\textbf{Limitation of FDS.} }
    \label{fig:limitation}
\end{figure}


% \begin{figure}[t] \centering
%     \makebox[0.48\textwidth]{}
%     \makebox[0.48\textwidth]{}
%     \\
%     \includegraphics[width=0.48\textwidth]{figure/loss_Ignatius.pdf}
%     \includegraphics[width=0.48\textwidth]{figure/loss_family.pdf}
%     \caption{\textbf{Comparison the photometric error of Radiance Flow and Prior Flow:} 
%     We add FDS method after 2k iteration during training.
%     The results show
%     that:  1) The Prior Flow is more precise and 
%     robust than Radiance Flow during the radiance 
%     optimization; 2) After adding the FDS loss 
%     which utilize Prior 
%     flow to supervise the Radiance Flow at 2k iterations, 
%     both flow are more accurate, which lead to
%     a mutually reinforcing effects.(TODO fix it)} 
%     \label{fig:flowcompare}
% \end{figure}






\textbf{Interpretive Experiments: }
To demonstrate the mutual refinement of two flows in our FDS, 
For each view, we sample the unobserved 
views multiple times to compute the mean error 
of both Radiance Flow and Prior Flow. 
We use Raft~\citep{teed2020raft} as our default optical flow model
for visualization.
The ground truth flow is calculated based on 
~\eref{equ:flow_pose} and ~\eref{equ:flow} 
utilizing ground truth depth in dataset.
We introduce our FDS loss after 16000 iterations during 
optimization of 2DGS.
The error maps are shown in ~\figref{fig:error_map}.
Our analysis reveals that Radiance Flow tends to 
exhibit significant geometric errors, 
whereas Prior Flow can more accurately estimate the geometry,
effectively disregarding errors introduced by floating Gaussian points. 

%





\subsection{Limitation and further work}

Firstly, our FDS faces challenges in scenes with 
significant lighting variations between different 
views, as shown in the lamp of first row in ~\figref{fig:limitation}. 
%
Incorporating exposure compensation into FDS could help address this issue. 
%
 Additionally, our method struggles with 
 reflective surfaces and motion blur,
 leading to incorrect matching. 
 %
 In the future, we plan to explore the potential 
 of FDS in monocular video reconstruction tasks, 
 using only a single input image at each time step.
 


\section{Conclusions}
In this paper, we propose Flow Distillation Sampling (FDS), which
leverages the matching prior between input views and 
sampled unobserved views from the pretrained optical flow model, to improve the geometry quality
of Gaussian radiance field. 
Our method can be applied to different approaches (3DGS and 2DGS) to enhance the geometric rendering quality of the corresponding neural radiance fields.
We apply our method to the 3DGS-based framework, 
and the geometry is enhanced on the Mushroom, ScanNet, and Replica datasets.

\section*{Acknowledgements} This work was supported by 
National Key R\&D Program of China (2023YFB3209702), 
the National Natural Science Foundation of 
China (62441204, 62472213), and Gusu 
Innovation \& Entrepreneurship Leading Talents Program (ZXL2024361)

\section{Conclusion}
% We introduce \datasetname, a new benchmark to evaluate LLMs' ability to simulate human behavior in dynamic contexts.  
% Comprising 1001 literary behavior chains with rich persona context, it addresses a key gap in LLM assessment.
% Our findings reveal substantial LLM deficiencies in simulating coherent behavior chains, especially long-range consistency.  We observed a significant gap between individual behavior recognition and chain-level performance, highlighting the challenge of error accumulation.
% Analysis revealed temporal biases, a preference for key behaviors, and complex interactions between history/context length and performance.  These insights underscore the complexities of behavior simulation and suggest future research directions, including improved training data and model architectures.  
% \datasetname offers a valuable resource for robust digital twin development.

We introduced \datasetname, a novel benchmark for evaluating LLMs' ability to simulate continuous human behavior.  Comprising 1,001 persona-based behavior chains extracted from literature, it addresses a critical gap in LLM assessment, particularly given the scarcity of real-world behavioral data.  Our evaluation framework, encompassing behavioral recognition and generation tasks, revealed significant challenges for ten state-of-the-art LLMs.
Further experiments analyzing the influence of key behaviors, history, and context characteristics provided valuable insights into the factors that contribute to or hinder successful persona-based behavior simulation.
\datasetname provides a valuable resource for developing robust digital twins.




\section*{Limitations}

% \datasetname relies on literary works (novels/biographies) for behavior extraction, which may introduce stylistic bias. Fictional narratives often emphasize dramatic or plot-driven actions, while biographies may omit mundane behaviors. This limits generalizability to real-world scenarios where subtle, routine behaviors dominate.  

\datasetname focuses on English-language Western literature, lacking representation of non-Western cultural norms and behavioral expressions. This cultural narrowness limits applicability to global digital twin deployments.
Generation task evaluation via GPT-4o introduces model bias, as the assessor’s cultural/personal biases may influence consistency judgments. Human evaluation, though more reliable, was limited to data validation rather than comprehensive scoring.


Our current dataset, while encompassing literary works from diverse cultural and national backgrounds, is presently limited to English-language materials. We recognize the critical importance of investigating digital twin applications across multilingual contexts. In subsequent phases of this research, we plan to expand the dataset to incorporate multiple languages, thereby enabling comprehensive evaluation of models' capabilities in simulating human behavior through multilingual interactions.
In the current study, we have focused exclusively on data construction and model evaluation while thoroughly examining the model's limitations across multiple dimensions. The aspect of enhancing the model's capability for continuous behavioral simulation remains underexplored, and we plan to address in subsequent research. 

% 虽然囊括了来自多个国家多种文化背景的书籍,但我们的数据集目前只有英文版本,我们认为不同语言背景下的digital twin同样是重要的。因为我们之后会在之后的工作中讲数据集扩充到更多语言,考察模型用多种语言simulate 人类行为的能力。
% 我们关注数据的构建和模型的评估,我们会在以后的工作中继续探索如何在模型如何提升在continous behavior simulating能力,(展望digital twin)




\section*{Ethics Statement}
This paper introduces \datasetname, a benchmark for evaluating LLMs' ability to simulate human behavior, raising several ethical considerations.  While the benchmark itself doesn't directly generate content that could be harmful, the use of LLMs for behavior simulation has the potential for misuse, such as creating deceptive or manipulative content. The dataset construction process, which involves extracting behaviors from literary works, was carefully designed to avoid misrepresenting characters or their actions.  All source materials are publicly available.  Furthermore, the evaluation framework focuses on assessing LLMs' simulation capabilities rather than generating real-world actions.  We acknowledge the potential for bias in the literary sources used and emphasize the importance of responsible use of this benchmark and its findings. 
The datasets used in our experiment are publicly released and labeled through interaction with humans in English. In this process, user privacy is protected, and no personal information is contained in the dataset. The scientific artifacts that we used are available for research with permissive licenses. And the use of these artifacts in this paper is consistent with their intended use. Therefore, we believe that our research work meets the ethics of ACL.



% \section{Engines}

% To produce a PDF file, pdf\LaTeX{} is strongly recommended (over original \LaTeX{} plus dvips+ps2pdf or dvipdf). Xe\LaTeX{} also produces PDF files, and is especially suitable for text in non-Latin scripts.

% \section{Preamble}

% The first line of the file must be
% \begin{quote}
% \begin{verbatim}
% \documentclass[11pt]{article}
% \end{verbatim}
% \end{quote}

% To load the style file in the review version:
% \begin{quote}
% \begin{verbatim}
% \usepackage[review]{acl}
% \end{verbatim}
% \end{quote}
% For the final version, omit the \verb|review| option:
% \begin{quote}
% \begin{verbatim}
% \usepackage{acl}
% \end{verbatim}
% \end{quote}

% To use Times Roman, put the following in the preamble:
% \begin{quote}
% \begin{verbatim}
% \usepackage{times}
% \end{verbatim}
% \end{quote}
% (Alternatives like txfonts or newtx are also acceptable.)

% Please see the \LaTeX{} source of this document for comments on other packages that may be useful.

% Set the title and author using \verb|\title| and \verb|\author|. Within the author list, format multiple authors using \verb|\and| and \verb|\And| and \verb|\AND|; please see the \LaTeX{} source for examples.

% By default, the box containing the title and author names is set to the minimum of 5 cm. If you need more space, include the following in the preamble:
% \begin{quote}
% \begin{verbatim}
% \setlength\titlebox{<dim>}
% \end{verbatim}
% \end{quote}
% where \verb|<dim>| is replaced with a length. Do not set this length smaller than 5 cm.

% \section{Document Body}

% \subsection{Footnotes}

% Footnotes are inserted with the \verb|\footnote| command.\footnote{This is a footnote.}

% \subsection{Tables and figures}

% See Table~\ref{tab:accents} for an example of a table and its caption.
% \textbf{Do not override the default caption sizes.}

% \begin{table}
%   \centering
%   \begin{tabular}{lc}
%     \hline
%     \textbf{Command} & \textbf{Output} \\
%     \hline
%     \verb|{\"a}|     & {\"a}           \\
%     \verb|{\^e}|     & {\^e}           \\
%     \verb|{\`i}|     & {\`i}           \\
%     \verb|{\.I}|     & {\.I}           \\
%     \verb|{\o}|      & {\o}            \\
%     \verb|{\'u}|     & {\'u}           \\
%     \verb|{\aa}|     & {\aa}           \\\hline
%   \end{tabular}
%   \begin{tabular}{lc}
%     \hline
%     \textbf{Command} & \textbf{Output} \\
%     \hline
%     \verb|{\c c}|    & {\c c}          \\
%     \verb|{\u g}|    & {\u g}          \\
%     \verb|{\l}|      & {\l}            \\
%     \verb|{\~n}|     & {\~n}           \\
%     \verb|{\H o}|    & {\H o}          \\
%     \verb|{\v r}|    & {\v r}          \\
%     \verb|{\ss}|     & {\ss}           \\
%     \hline
%   \end{tabular}
%   \caption{Example commands for accented characters, to be used in, \emph{e.g.}, Bib\TeX{} entries.}
%   \label{tab:accents}
% \end{table}

% As much as possible, fonts in figures should conform
% to the document fonts. See Figure~\ref{fig:experiments} for an example of a figure and its caption.

% Using the \verb|graphicx| package graphics files can be included within figure
% environment at an appropriate point within the text.
% The \verb|graphicx| package supports various optional arguments to control the
% appearance of the figure.
% You must include it explicitly in the \LaTeX{} preamble (after the
% \verb|\documentclass| declaration and before \verb|\begin{document}|) using
% \verb|\usepackage{graphicx}|.

% \begin{figure}[t]
%   \includegraphics[width=\columnwidth]{example-image-golden}
%   \caption{A figure with a caption that runs for more than one line.
%     Example image is usually available through the \texttt{mwe} package
%     without even mentioning it in the preamble.}
%   \label{fig:experiments}
% \end{figure}

% \begin{figure*}[t]
%   \includegraphics[width=0.48\linewidth]{example-image-a} \hfill
%   \includegraphics[width=0.48\linewidth]{example-image-b}
%   \caption {A minimal working example to demonstrate how to place
%     two images side-by-side.}
% \end{figure*}

% \subsection{Hyperlinks}

% Users of older versions of \LaTeX{} may encounter the following error during compilation:
% \begin{quote}
% \verb|\pdfendlink| ended up in different nesting level than \verb|\pdfstartlink|.
% \end{quote}
% This happens when pdf\LaTeX{} is used and a citation splits across a page boundary. The best way to fix this is to upgrade \LaTeX{} to 2018-12-01 or later.

% \subsection{Citations}

% \begin{table*}
%   \centering
%   \begin{tabular}{lll}
%     \hline
%     \textbf{Output}           & \textbf{natbib command} & \textbf{ACL only command} \\
%     \hline
%     \citep{Gusfield:97}       & \verb|\citep|           &                           \\
%     \citealp{Gusfield:97}     & \verb|\citealp|         &                           \\
%     \citet{Gusfield:97}       & \verb|\citet|           &                           \\
%     \citeyearpar{Gusfield:97} & \verb|\citeyearpar|     &                           \\
%     \citeposs{Gusfield:97}    &                         & \verb|\citeposs|          \\
%     \hline
%   \end{tabular}
%   \caption{\label{citation-guide}
%     Citation commands supported by the style file.
%     The style is based on the natbib package and supports all natbib citation commands.
%     It also supports commands defined in previous ACL style files for compatibility.
%   }
% \end{table*}

% Table~\ref{citation-guide} shows the syntax supported by the style files.
% We encourage you to use the natbib styles.
% You can use the command \verb|\citet| (cite in text) to get ``author (year)'' citations, like this citation to a paper by \citet{Gusfield:97}.
% You can use the command \verb|\citep| (cite in parentheses) to get ``(author, year)'' citations \citep{Gusfield:97}.
% You can use the command \verb|\citealp| (alternative cite without parentheses) to get ``author, year'' citations, which is useful for using citations within parentheses (e.g. \citealp{Gusfield:97}).

% A possessive citation can be made with the command \verb|\citeposs|.
% This is not a standard natbib command, so it is generally not compatible
% with other style files.

% \subsection{References}

% \nocite{Ando2005,andrew2007scalable,rasooli-tetrault-2015}

% The \LaTeX{} and Bib\TeX{} style files provided roughly follow the American Psychological Association format.
% If your own bib file is named \texttt{custom.bib}, then placing the following before any appendices in your \LaTeX{} file will generate the references section for you:
% \begin{quote}
% \begin{verbatim}
% \bibliography{custom}
% \end{verbatim}
% \end{quote}

% You can obtain the complete ACL Anthology as a Bib\TeX{} file from \url{https://aclweb.org/anthology/anthology.bib.gz}.
% To include both the Anthology and your own .bib file, use the following instead of the above.
% \begin{quote}
% \begin{verbatim}
% \bibliography{anthology,custom}
% \end{verbatim}
% \end{quote}

% Please see Section~\ref{sec:bibtex} for information on preparing Bib\TeX{} files.

% \subsection{Equations}

% An example equation is shown below:
% \begin{equation}
%   \label{eq:example}
%   A = \pi r^2
% \end{equation}

% Labels for equation numbers, sections, subsections, figures and tables
% are all defined with the \verb|\label{label}| command and cross references
% to them are made with the \verb|\ref{label}| command.

% This an example cross-reference to Equation~\ref{eq:example}.

% \subsection{Appendices}

% Use \verb|\appendix| before any appendix section to switch the section numbering over to letters. See Appendix~\ref{sec:appendix} for an example.

% \section{Bib\TeX{} Files}
% \label{sec:bibtex}

% Unicode cannot be used in Bib\TeX{} entries, and some ways of typing special characters can disrupt Bib\TeX's alphabetization. The recommended way of typing special characters is shown in Table~\ref{tab:accents}.

% Please ensure that Bib\TeX{} records contain DOIs or URLs when possible, and for all the ACL materials that you reference.
% Use the \verb|doi| field for DOIs and the \verb|url| field for URLs.
% If a Bib\TeX{} entry has a URL or DOI field, the paper title in the references section will appear as a hyperlink to the paper, using the hyperref \LaTeX{} package.

% \section*{Acknowledgments}

% This document has been adapted
% by Steven Bethard, Ryan Cotterell and Rui Yan
% from the instructions for earlier ACL and NAACL proceedings, including those for
% ACL 2019 by Douwe Kiela and Ivan Vuli\'{c},
% NAACL 2019 by Stephanie Lukin and Alla Roskovskaya,
% ACL 2018 by Shay Cohen, Kevin Gimpel, and Wei Lu,
% NAACL 2018 by Margaret Mitchell and Stephanie Lukin,
% Bib\TeX{} suggestions for (NA)ACL 2017/2018 from Jason Eisner,
% ACL 2017 by Dan Gildea and Min-Yen Kan,
% NAACL 2017 by Margaret Mitchell,
% ACL 2012 by Maggie Li and Michael White,
% ACL 2010 by Jing-Shin Chang and Philipp Koehn,
% ACL 2008 by Johanna D. Moore, Simone Teufel, James Allan, and Sadaoki Furui,
% ACL 2005 by Hwee Tou Ng and Kemal Oflazer,
% ACL 2002 by Eugene Charniak and Dekang Lin,
% and earlier ACL and EACL formats written by several people, including
% John Chen, Henry S. Thompson and Donald Walker.
% Additional elements were taken from the formatting instructions of the \emph{International Joint Conference on Artificial Intelligence} and the \emph{Conference on Computer Vision and Pattern Recognition}.

% % Bibliography entries for the entire Anthology, followed by custom entries
% %\bibliography{anthology,custom}
% % Custom bibliography entries only
\bibliography{custom}

\appendix

\section{Appendix}
% \label{sec:app}
\label{sec:app}



\subsection{Data Analysis}

\subsubsection{Data Statistics}

\begin{figure}[h]
    \centering
    \includegraphics[width=1\linewidth]{latex/picture/bing.png}
    \caption{The left figure illustrates the publication year distribution of literary works used as raw material from \datasetname. The right figure shows the proportion of Level 1 and Level 2 behaviors within the overall dataset.}
    \label{fig:bing}
\end{figure}



% 行为链数据集

% 总行为数: 15846
% 总人物数:1000
% 最大行为节点数:20
% 最小行为数:10
% 平均节点数:15-16
% 最大input token长度:54
% 最小input token长度:8
% 平均input token长度:18.48
% 语言:英语
% 来源:小说,传记


\begin{table}[h]
\centering
\begin{tabular}{lc}
\toprule
Statistic & Value \\
\midrule
Total Behaviors & 15846 \\
Total Persons & 1001 \\
Max Behavior Nodes & 20 \\
Min Behaviors & 10 \\
Average Nodes & 15-16 \\
Max Input Token Length & 54 \\
Min Input Token Length & 8 \\
Average Input Token Length & 18.48 \\
Language & English \\
Source & Novels, Biographies \\
\bottomrule
\end{tabular}
\caption{Descriptive Statistics of the Behavior Dataset}
\label{tab:behavior_stats} % Optional: Add a label for cross-referencing
\end{table}

\begin{table}[h]
\centering
\small
\begin{tabular}{lc}
\toprule
Statistic & Value \\
\midrule
Style Categories & 25 \\
Genre Categories & 65 \\
Setting Categories & 536 \\
Theme Categories & 115 \\
Lexile Level Categories & 112 \\
Recommended Reading Age Categories & 31 \\
Max Rating Number & 3,380,109 \\
Min Rating Number & 36 \\
Max Ratings Score & 4.7 \\
Min Ratings Score & 3.3 \\
\bottomrule
\end{tabular}
\caption{Book Diversity Statistics}
\label{tab:book_diversity} % Optional: Add a label for cross-referencing
\end{table}

We collect book-related metadata from SuperSummary and Goodreads, including various categorical tags such as Style, Genre, Setting, Theme, Lexile Level, and Recommended Reading Age from SuperSummary. 
Additionally, we record the number of ratings to the rating scores from Goodreads. 
As shown in Table \ref{tab:book_diversity}, the dataset demonstrates a diverse and comprehensive representation of books.

For specific tags, the most prevalent Styles include \textit{Emotional, Mysterious, and Contemplative}. 
Common Genres featured in the dataset are \textit{Historical Fiction, Modern Classic Fiction, Romance, and Mystery \& Crime Fiction}. 
Frequently occurring Settings include \textit{Contemporary, 2010s, and United States}. 
The dataset also highlights recurring Themes such as \textit{Family, Love, Friendship, and Race}. 
The most frequently occurring levels include 830L, 780L, HL690L, and 740L.
Additionally, the Recommended Reading Age ranges from \textit{7 to 18+}.
Together, these attributes demonstrate the richness and diversity of the dataset, ensuring a broad representation of literary elements across various themes, genres, reading levels, and target audiences.

\subsubsection{More Results Analysis}
\label{sec:app_result}
\begin{figure*}[h]
    \centering
    \includegraphics[width=1\linewidth]{latex/picture/style.pdf}
    \caption{Model performance across different book style.}
    \label{fig:style}
\end{figure*}

We select the top 20 Styles based on their frequency and calculated the average metrics for the book subsets under each Style. 
Our analysis shows that the model's ability to predict behavior chains does not exhibit significant differences across different Styles.
% However, relatively speaking, styles such as Informative and Gritty demonstrate slightly higher predictive performance, likely due to their more structured and explicit behavioral patterns. 
% In contrast, styles like Fantastical and Playful exhibit marginally lower performance, potentially due to greater variability and implicit character behaviors. 
% While these differences are not statistically significant, they suggest that narrative structure may influence the model’s ability on the behavior chain task.


\subsection{Implementation Details}

\subsubsection{Dataset Construction Details}
\label{sec:app_con}
We employed Claude-3-5-sonnet-20240620 for behavior chain extraction and context refinement. The ChatGPT-4o-latest model was utilized to generate distractor items for each behavior. For other generation tasks with lower computational requirements, we selected GPT-3.5 as the execution model.

\subsubsection{Model Testing Details}
\label{sec:app_exp}
All evaluated models were post-interaction versions. The history provided to the LLM consisted of summaries from all preceding chapters used for behavior chain extraction.

To prevent data leakage, we systematically substituted all personal entities in the inputs. For conciseness, all analyses in Section \ref{sec:analysis} are based on the multiple-choice task framework.

The single experimental data presented in \ref{sec:analysis} exclusively comprises outputs from the Llama-3p1-70B-Instract model as the target model.

For inputs that exceed the model's context length, we truncated the history from the oldest to the newest.












\subsection{Prompt}
\label{sec:app_prompt}

\begin{tcolorbox}[title = {First Behavior Generation Prompt}, breakable]

You are an expert in Narrative Analysis and Character Behavior Extraction.\\
Please extract the MOST KEY behavior of \{character\} FROM <Paragraphs>.\\
The behavior should have a significant impact on the development of the storyline, reflect character characteristics or emotions.\\

Ensure that the KEY behavior is an objective statement, clearly stated without any vague expressions.\\
DO NOT add subjective interpretations or inferences about the character's behaviors. ONLY describe the KEY behavior itself. DO NOT mention the result in the KEY behavior.\\
Use your own words instead of quoting the original text.\\
DO NOT repeat or imitate <Previous Key Behavior>.\\

The KEY behavior should have a significant impact on the development of the storyline, the characterization of the characters, and the expression of the theme.\\
Ensure that the key behavior is an objective statement, clearly stated without any vague expressions.\\
DO NOT add subjective interpretations or inferences about the character's behaviors. Only describe the behavior itself.\\

The format of your response should be: \{"key behavior": ""\}.\\

<Paragraphs BEGIN>\\
\{parts[0]\}\\
<Paragraphs END>\\

If the behavior of \{character\} cannot be extracted, output "" ONLY.


\end{tcolorbox}

% \vspace{1cm}


\begin{tcolorbox}[title = {Next Behavior Generation Prompt}, breakable]
You are an expert in Narrative Analysis and Character Behavior Extract.\\
Below, I will provide you with <Previous Paragraphs>, <Previous Key Behavior> extracted from <Previous Paragraphs> and <Current Paragraphs>. \\

Please summary the scene change and plot development detailly and naturally after the <Previous Key Behavior> according to the <Previous Paragraphs> and <Current Paragraphs> I give you. The summary should start with "\{examples[-1]['key behavior']\} After that, ".\\

After the summary, you should extract the MOST KEY behavior of \{character\} FROM <Current Paragraphs>, describe in more than 10 words.\\
The behavior should be a non-meaningless behavior taken spontaneously by \{character\}.\\
The behavior should have a significant impact on the development of the storyline or reflect character characteristics or emotions.\\
Ensure that the behavior is objective statements and state the behavior clearly and do not use any vague expressions. \\
DO NOT add subjective interpretations and inference about the character's behaviors. ONLY describe the behavior itself. \\
DO NOT mention the result in the behavior.\\
Ensure use your own words instead of quoting the original text. \\
DO NOT repeat or imitate <Previous Key Behavior>. \\

Please provide a REVISED summary of the scene change and plot development that occurred before the behavior you extracted from <Current Paragraphs>, making sure not to reveal any information about the behavior. Delete the behavior and subsequent plots, and keep only the plots before the behavior.\\
The REVISED summary should end with "After this or in response to this, what behavior did \{character\} take?"\\

<Previous Paragraphs BEGIN>\\
\{marge part[-1]\}\\
<Previous Paragraphs END>\\

<Previous Key Behavior BEGIN>\\
\{examples[-1]['key behavior']\}\\
<Previous Key Behavior END>\\

<Current Paragraphs BEGIN>\\
\{parts[i]\}\\
<Current Paragraphs END>\\

The format of your response should be \{"summary": "","key behavior":"","new summary":""\}.\\
Ensure that the behavior you extract is taken by \{character\} at this moment, rather than behavior of others or past behavior (in <Paragraphs>, \{character\}probable in the first person).\\
If the behavior of \{character\} cannot be extracted, output "None" ONLY.

\end{tcolorbox}





\begin{tcolorbox}[title = {Similirity Check}, breakable]

Please determine whether the following two behaviors refer to the same behavior:\\
Behavior 1: \{sentence1\}\\
Behavior 2: \{sentence2\}\\
If there is a strong possibility that the two behaviors refer to the same behavior, please output 1; otherwise, output 0, Make sure you give me 0/1.
\end{tcolorbox}




\begin{tcolorbox}[title = {Context Refine}, breakable]
<Context Begin>\\
\{relevant part\}\\
<Context End>\\

<Behavior Begin>\\
\{key behavior\}\\
<Behavior End>\\

Your task is to refine the <Context> according to the following requirements:\\

1. If the <Context> explicitly or implicitly suggests the active behaviors of \{character\} in the <Behavior> or discloses the result/reactions of others caused by <Behavior>, delete these from <Context>. Any references to \{character\}' emotions, feelings, psychological states, or internal conflicts should be eliminated from <Context>.\\

2. If < Behavior> include \{character\}'s reaction/response to event/situation/others behavior, then that event/situation/others behavior should to be described intactly and directly in the end of <Context>.\\

3. If the <Behavior> includes any elements such as contextual conditions or encounters rather than purely active behavior of \{character\}, integrate these elements into the <Context>. Pay attention to the clauses in <Behavior> as it often contain contextual information, but do not include it in <Context> if it happens after \{character\}'s behavior.\\

4. Output the refined <Context> directly without other note.

\end{tcolorbox}




\begin{tcolorbox}[title = {Distracter Generation}, breakable]
<Context Start>\\
\{examples[i]['summary refined']\}\\
<Context End>\\

The original subsequent behavior was "\{behavior\}".\\
Estimate what personality trait does this reflect in brief words, and generate 3 behaviors that different personality traits would exhibit, answered in JSON format.\\
\{\\
"original behavior traits":"",\\
"difference": [\\
\{\\
 "trait": "",\\
 "behavior": ""\\
 \},\\
\{\\
 "trait": "",\\
"behavior": ""\\
 \},\\
\{\\
 "trait": "",\\
 "behavior": ""\\
 \}\\
]\\
\}

\end{tcolorbox}




\begin{tcolorbox}[title = {Level Define}, breakable]
<Summary BEGIN>\\
\{summary l\}\\
<Summary END>\\
Please check whether The following behaviors are implied in the summary.\\
In your reply, retain the sequence numbers. The number of key behaviors must be considerably fewer than \{int(n/2)\}.\\

<behaviors BEGIN>\\
\{sentences\}\\
<behaviors END>

\end{tcolorbox}





\subsection{Data instance}
\label{sec:app_instance}
\begin{tcolorbox}[title = {Profile}, breakable]
\{
  "Name": "Jay Gatsby",\\
  "Personality Traits": [
    "Charismatic",
    "Mysterious",
    "Obsessive",
    "Romantic",
    "Wealthy",
    "Idealistic"
  ],\\
  "Motivations and Goals": [\\
    "To reunite with his former lover, Daisy Buchanan",\\
    "To achieve a high social status and wealth",\\
    "To recapture the past and fulfill his ideal vision of life with Daisy"\\
  ],\\
  "Significant Background Events": [\\
    "Born James Gatz to a poor farming family in North Dakota",\\
    "Changed his name to Jay Gatsby and reinvented himself as a wealthy socialite",\\
    "Amassed his fortune through questionable means",\\
    "Became known for his lavish parties at his mansion in West Egg"\\
  ],\\
  "Relationships": \{\\
    "Daisy Buchanan": "Former lover, whom Gatsby is still deeply in love with",\\
    "Tom Buchanan": "Daisy's husband and Gatsby's rival",\\
    "Nick Carraway": "Narrator of the story and Gatsby's neighbor and friend",\\
    "Jordan Baker": "A professional golfer and friend of Daisy, whom Gatsby has a brief romantic interest in",\\
    "George Wilson": "A mechanic and owner of a garage, indirectly involved in Gatsby's downfall"\\
  \},\\
  "Additional Details": \{\\
    "Occupation": "Businessman with mysterious sources of wealth",\\
    "Social Status": "Wealthy and influential, but not born into old money",\\
    "Hobbies": "Throwing extravagant parties, collecting expensive art and cars",\\
    "Residence": "A grand mansion in West Egg, New York"\\
  \}\\
\}\\
\end{tcolorbox}


\begin{tcolorbox}[title = {History}, breakable]
Summary of Chapter 1 \\
Nick, a Yale graduate and World War I veteran, moves to West Egg, Long Island, to work as a bond salesman.  He rents a small house next to the opulent mansion of his mysterious neighbor, Jay Gatsby.

West Egg is separated from the more fashionable East Egg by the Long Island Sound.  Across the water, Nick's cousin Daisy lives with her wealthy and imposing husband, Tom Buchanan, whom Nick knew at Yale.  Tom's wealth and social standing are evident in their "Georgian Colonial mansion."  Also present at their home during Nick's visit is Jordan Baker, a friend of Daisy's and a well-known golfer.

The chapter highlights the Buchanans' superficial and privileged lifestyle.  During dinner, Tom reveals his racist views by referencing a book espousing white supremacist theories, claiming the Nordic race is responsible for civilization.  This disturbs Nick, but no one challenges Tom's comments.

The dinner is interrupted by a phone call, hinting at Tom's infidelity.  Jordan informs Nick that Tom is having an affair with a woman in New York City.  The tension is palpable, but Daisy attempts to deflect attention.  The awkwardness of the situation is further amplified by another phone call.

As the evening progresses, Nick learns that Jordan Baker is a famous golfer with a somewhat scandalous reputation.  He also discovers that Daisy and he, despite being related, are not close.  The Buchanans tease Nick about a rumor of a broken engagement, which he denies.

Returning home, Nick sees Gatsby standing outside his mansion.  Nick considers inviting him over but hesitates, sensing something enigmatic about Gatsby's presence.
\\Summary of Chapter 2 \\
Chapter 2 delves into the "valley of ashes," a desolate area between West Egg and New York City, where industrial ashes are dumped.  Presiding over this wasteland is the faded billboard of Dr. T.J. Eckleburg, featuring giant, spectacled eyes.

This grim setting is significant because it's where George Wilson's struggling auto shop and his wife Myrtle, Tom's mistress, are located.  Tom, on his way to New York City with Nick, stops at Wilson's garage, ostensibly to discuss a car deal.  The real purpose, however, is to rendezvous with Myrtle. Nick finds Myrtle physically unattractive, despite her apparent allure for Tom.

The group, leaving George behind, proceeds to New York City, impulsively buying a puppy along the way.  They gather at a small apartment Tom keeps for his affair.  The other guests include Myrtle's sister, Catherine, and the McKees, neighbors from the building. Mr. McKee is a mediocre photographer.

The atmosphere becomes increasingly tawdry and tense.  A critical moment occurs when Myrtle mentions Daisy's name, provoking Tom to violently strike her, breaking her nose.

Nick, claiming to be drunk (only the second time in his life), leaves the apartment with Mr. McKee amidst the chaos of tending to Myrtle's injury.  He experiences a blackout and then finds himself at Mr. McKee's bedside, observing his amateurish photographs.  Nick manages to escape the apartment and waits at the train station for the 4:00 a.m. train home.
\\Summary of Chapter 3 \\
Chapter 3 depicts one of Gatsby's extravagant parties through Nick's eyes.  Nick receives a formal invitation, a stark contrast to the casual attendance of most guests, who often don't even know Gatsby.  At the party, Nick reconnects with Jordan Baker, who explains she enjoys large gatherings for the anonymity they offer.  A humorous anecdote highlights Gatsby's generosity: he replaced a dress torn at a previous party with an expensive new one.

Rumors about Gatsby's background and wealth circulate among the guests, ranging from accusations of being a German spy to claims of him being a war hero or even a murderer.  While searching for Gatsby, Nick encounters a drunken "owl-eyed" man in Gatsby's impressive library, who is astonished to discover the books are real.  Nick eventually meets Gatsby, but their conversation is cut short by a phone call.  Observing Gatsby later, Nick is struck by his charismatic smile and the contrast between his apparent sobriety and the revelry of his guests.  Gatsby then takes Jordan aside for a private conversation, leaving her visibly shaken.  As the party ends, a drunken car crash involving the owl-eyed man underscores the reckless atmosphere.

Nick then reflects on his experiences, noting that his time in West Egg isn't solely filled with social events but also with work.  He expresses his growing affection for New York City, despite feelings of loneliness, and his burgeoning relationship with Jordan.  He recounts a story about Jordan cheating in a golf tournament, a scandal that, though dismissed, hints at her dishonesty.  Nick rationalizes her behavior as a consequence of her desire for independence clashing with her need for protection.

The chapter concludes with Nick and Jordan becoming romantically involved.  Nick reveals he ended a relationship in the Midwest to pursue Jordan and emphasizes his own sense of honesty.
\\Summary of Chapter 4 \\
Chapter 4 of *The Great Gatsby* opens with a Sunday morning at Gatsby's mansion, where the usual rumors about his past continue to circulate. Nick briefly catalogs the diverse and affluent guests who frequent Gatsby's parties.

One day in late July, Gatsby takes Nick to lunch in his luxurious car.  During the drive, Gatsby seems overly concerned with Nick's opinion of him and his vehicle. He then directly addresses the rumors about his background, claiming to be the son of wealthy, deceased Midwestern parents.  However, he gives the contradictory answer of San Francisco when asked where in the Midwest.  He further elaborates, stating that he inherited his parents' fortune, lived lavishly in Europe, and then became a war hero in World War I, receiving numerous medals.

To support his claims, Gatsby shows Nick a photo of himself at Oxford and a medal from Montenegro.  These items, along with Gatsby's ability to evade a speeding ticket by showing a Christmas card from the police commissioner, convince Nick of the truth of Gatsby's stories.

They have lunch with Meyer Wolfsheim, a shady character who, according to Gatsby, fixed the World Series. Wolfsheim's bizarre cufflinks, made of human molars, further emphasize his connection to the criminal underworld.  Tom Buchanan briefly joins them, but Gatsby disappears abruptly.

Later, Jordan reveals to Nick that she met Gatsby in Louisville in 1917, when Daisy and Gatsby were together.  Jordan recounts how Daisy almost ran off to New York with Gatsby before he went to war, and how she was later deeply unhappy before her wedding to Tom.  Jordan then reveals the reason for her private conversation with Gatsby at the party: Gatsby wants to reunite with Daisy, and he needs Nick's help to arrange a meeting.
\\Summary of Chapter 5 \\
Chapter 5 details the long-awaited reunion between Gatsby and Daisy at Nick's cottage.  Before Daisy arrives, Gatsby's feigned disinterest crumbles as he obsessively prepares, even having Nick's lawn manicured.  Daisy's initial question about the purpose of the invitation and whether Nick harbors romantic feelings for her is met with a reference to *Castle Rackrent*.

The meeting begins awkwardly, and Nick tries to give them space. However, Gatsby, clearly anxious, repeatedly seeks Nick's advice, prompting Nick to reprimand him for his behavior.  Gatsby's reaction suggests he's offended by Nick's criticism.

They then move to Gatsby's mansion. Nick briefly describes the mansion's original owner, a wealthy brewer who, according to rumors, wanted the surrounding area to resemble a medieval village, an idea Nick finds inherently un-American.  The only person present in the vast house is Klipspringer, a freeloading "boarder."  Gatsby proudly displays his extensive and expensive collection of shirts, a display that moves Daisy to tears.  Nick notices a picture of a man on a yacht, whom Gatsby identifies as Dan Cody, his former mentor and close friend.

Gatsby points out the green light on Daisy's dock, visible from his house in clear weather.  After touring the mansion, they again encounter Klipspringer, whom Gatsby compels to play the piano.  Finally, Nick leaves Gatsby and Daisy alone, implying the possibility of their rekindling their past romance.
\\Summary of Chapter 6 \\
Chapter 6 begins with a reporter's visit to Gatsby, seeking a statement regarding vague, unsubstantiated rumors.  This prompts Nick to recount Gatsby's true origins, as told by Gatsby himself.  James Gatz, born to an unremarkable family in South Dakota, reinvented himself as Jay Gatsby after meeting Dan Cody on Lake Superior.  Feeling out of place in his hometown, Gatz left to pursue a grander future.  He encountered Cody's yacht and, through his resourcefulness, became Cody's personal assistant and was even named a potential heir. However, after Cody's death, his mistress, Ella Kaye, intervened, preventing Gatsby from inheriting anything.\\
Weeks after Gatsby and Daisy's reunion, Nick unexpectedly encounters Tom Buchanan at Gatsby's mansion. Tom and two acquaintances have stopped by during a horseback riding trip. Gatsby's overly eager hospitality towards Tom is noticeable. When Tom and his group leave for another gathering, Gatsby almost insists on joining them, despite their clear disinterest.

Later, Tom and Daisy attend one of Gatsby's parties. Nick observes a palpable tension, viewing the West Egg revelry through Daisy's East Egg perspective.  Daisy is clearly repulsed by the spectacle.  During the party, Tom fuels the rumors of Gatsby's bootlegging activities, disparaging him as "newly rich."  Daisy half-heartedly defends Gatsby, claiming the guests are uninvited and that Gatsby is too gracious to turn them away.  Tom openly flirts with other women, while Daisy, in a strange moment, invites Nick to kiss her, an invitation he declines.

After Tom and Daisy depart, Nick stays late. Gatsby confides in Nick about Daisy, revealing his desire for her to leave Tom and be with him. Gatsby then wistfully recalls the pivotal moment, five years prior, when he first kissed Daisy, a memory he cherishes as a defining moment in his life.

\end{tcolorbox}

\begin{tcolorbox}[title = {Behavior Chain}, breakable]
<context 1>\\
Gatsby dismissed every servant in his house and replaced them with new ones who did not interact with the local community. After that, rumors spread in the village about the new servants, and Gatsby explained to Nick that he hired people recommended by Wolfsheim who wouldn't gossip. The next day was extremely hot, and Nick traveled to the Buchanans' house by train. After this or in response to this, what behavior did Jay Gatsby take?
\\<key\_behavior 1>\\
Gatsby called Nick on the phone to invite him to lunch at Daisy's house the next day.
\\<distracters 1>\\
Gatsby shows up unannounced at the Buchanans' house, insisting on seeing Daisy immediately despite the potential awkwardness.\\
Gatsby avoids contact with Nick or Daisy altogether, choosing instead to observe the situation from a distance without direct involvement.\\
Gatsby openly declares his feelings for Daisy in public, ignoring the potential consequences and drawing attention to himself.\\
\\<context 2>\\
Gatsby called Nick on the phone to invite him to lunch at Daisy's house the next day. After that, Nick arrived at the Buchanans' house on a sweltering hot day. The butler answered the phone, and Nick and Gatsby were directed to the salon where Daisy and Jordan were resting on a couch. Tom's voice could be heard on the telephone in the hall, apparently having a heated conversation about selling a car. Tom then entered the room and greeted Gatsby with concealed dislike. The room had a crimson carpet. After this or in response to this, what behavior did Jay Gatsby take?
\\<key\_behavior 2>\\
Gatsby stood in the center of the crimson carpet and gazed around with fascinated eyes.
\\<distracters 2>\\
Gatsby strode toward Tom with a firm handshake and a polite but assertive comment about the weather.\\
Gatsby fidgeted with the edge of his jacket while avoiding making direct eye contact with anyone in the room.\\
Gatsby crossed his arms, gave Tom a steely glare, and responded curtly to any pleasantries.\\
\\<context 3>\\
Gatsby stood in the center of the crimson carpet and gazed around with fascinated eyes. After that, the group moved to the veranda upon Tom's suggestion. They observed the green Sound where a small sail was moving slowly in the heat. After this or in response to this, what behavior did Jay Gatsby take?
\\<key\_behavior 3>\\
Gatsby raised his hand and pointed across the bay to show the location of his house
\\<distracters 3>\\
Gatsby remained silent, simply nodding in acknowledgment of the view without drawing attention to his house.\\
Gatsby spoke poetically about the beauty of the green Sound and the sailboat, avoiding mention of his house entirely.\\
Gatsby excused himself quietly to stand apart, gazing at the water in introspection.\\
\\<context 4>\\
Gatsby raised his hand and pointed across the bay to show the location of his house. After that, the group moved to have lunch in the darkened dining room where they drank cold ale. The atmosphere became tense as Daisy expressed her distress about the heat and confusion, suggesting they all go to town. Tom and Gatsby engaged in small talk about garages and stables, while Daisy persistently pushed for the idea of going to town. When Daisy asked again who wanted to go to town After this or in response to this, what behavior did Jay Gatsby take?
\\<key\_behavior 4>\\
Gatsby's eyes floated toward Daisy when she asked about going to town, and he stared at her, maintaining eye contact
\\<distracters 4>\\
Gatsby casually shrugged and glanced out the window, showing disinterest in Daisy's suggestion.\\
Gatsby immediately stood up and said, 'Let’s go right now!' without waiting for the group’s consensus.\\
Gatsby hesitated, thoughtfully considering whether going to town was practical given the heat and tension in the room.\\
\\<context 5>\\
Gatsby's eyes floated toward Daisy when she asked about going to town, and he stared at her, maintaining eye contact. After that, no one immediately moved to leave. Tom became increasingly agitated at the group's hesitation, and his hand trembled as he finished his drink. Daisy tried to delay their departure by suggesting they smoke first, but Tom dismissed her suggestion. The women went upstairs to prepare while the three men waited outside in the heat. A tense atmosphere developed as they stood shuffling their feet on the hot gravel, with the silver moon already visible in the western sky. Tom stood there with them. After this or in response to this, what behavior did Jay Gatsby take?
\\<key\_behavior 5>\\
Gatsby attempted to speak to Tom but stopped himself, then asked about Tom's stables with visible effort
\\<distracters 5>\\
Gatsby confronted Tom directly, making a pointed comment about Daisy's choices or feelings, his voice unwavering.\\
Gatsby abruptly suggested they leave for town without consulting anyone, his restlessness showing in his tone and actions.\\
Gatsby smirked and made a sly, cutting remark about Tom's stables, using it as a veiled critique of Tom's character.\\
\\<context 6>\\
Gatsby attempted to speak to Tom but stopped himself, then asked about Tom's stables with visible effort. After that, the group prepared to leave for town. Daisy and Jordan went upstairs to get ready while the men waited outside. Tom expressed his displeasure about going to town, and went inside to get whiskey. The narrator made a comment about Daisy's voice, and Gatsby described it as being 'full of money.' Tom returned with a bottle of whiskey, followed by Daisy and Jordan. Tom suggested they take his car. After this or in response to this, what behavior did Jay Gatsby take?
\\<key\_behavior 6>\\
Gatsby objected to Tom's suggestion of driving his car, claiming there wasn't much gas.
\\<distracters 6>\\
Gatsby nodded and agreed that Tom's car would be a good choice, showing a willingness to go along with the plan without objection.\\
Gatsby hesitated, closely examining Tom's car, and suggested they double-check everything to ensure the ride would be safe.\\
Gatsby grinned and suggested they all race both cars to town, adding some excitement to the mundane trip.\\
\\<context 7>\\
Gatsby objected to Tom's suggestion of driving his car, claiming there wasn't much gas. After that, Tom insisted there was plenty of gas and suggested stopping at a drugstore if needed. The group then split up, with Tom driving Gatsby's car with Nick and Jordan, while Gatsby and Daisy followed in Tom's coupé. As they drove, Tom expressed suspicion about Gatsby and mentioned conducting a small investigation into his past. They stopped at Wilson's garage for gas, where Wilson revealed his plans to move west with his wife. The group then continued to the city, eventually ending up in a suite at the Plaza Hotel. The conversation became tense as Tom questioned Gatsby's use of the phrase 'old sport'. In the suite, there was a fallen telephone book with its string parted. After this or in response to this, what behavior did Jay Gatsby take?
\\<key\_behavior 7>\\
Gatsby examined the parted string of the fallen telephone book, muttered 'Hum!' in an interested way, and tossed the book on a chair.
\\<distracters 7>\\
Gatsby ignored the fallen telephone book completely, focusing instead on more immediate concerns in the conversation.\\
Gatsby picked up the fallen telephone book, carefully retied the string, and placed it neatly back where it belonged.\\
Gatsby glanced briefly at the telephone book, shrugged, and dismissed it as unimportant without further reaction.\\
\\<context 8>\\
Gatsby examined the parted string of the fallen telephone book, muttered 'Hum!' in an interested way, and tossed the book on a chair. After that, Tom sharply criticized Gatsby's habit of saying 'old sport,' leading to tension in the room. Daisy attempted to diffuse the situation by requesting ice for mint juleps. The group then heard a wedding march from downstairs, which prompted a discussion about Daisy's wedding and a mysterious guest named Biloxi. As the conversation about Biloxi's conflicting stories continued, Tom suddenly turned to Gatsby with a pointed question about his education. After this or in response to this, what behavior did Jay Gatsby take?
\\<key\_behavior 8>\\
When questioned about his Oxford education, Gatsby first responded evasively by saying 'Not exactly,' then changed his answer to confirm he went there, and finally specified he attended Oxford for five months in 1919.
\\<distracters 8>\\
Gatsby would directly and calmly state he attended Oxford as part of a special program, elaborating on the details without hesitation.\\
Gatsby would respond assertively, challenging Tom by pointing out the validity of his education and questioning why it matters.\\
Gatsby would shrug and dismiss the question, saying something like, 'I don't see why it matters,' and redirecting the conversation elsewhere.\\
\\<context 9>\\
When questioned about his Oxford education, Gatsby first responded evasively by saying 'Not exactly,' then changed his answer to confirm he went there, and finally specified he attended Oxford for five months in 1919. After that, Gatsby explained that his time at Oxford was part of an opportunity given to some officers after the Armistice, allowing them to attend universities in England or France. This explanation seemed to renew the narrator's faith in Gatsby. Meanwhile, Daisy attempted to lighten the mood by suggesting they make mint juleps, but Tom insisted on asking Gatsby one more question, demanding to know what kind of 'row' Gatsby was trying to cause in his house. After this or in response to this, what behavior did Jay Gatsby take?
\\<key\_behavior 9>\\
Gatsby politely invited Tom to continue with his questioning.
\\<distracters 9>\\
Gatsby openly challenges Tom, questioning Tom's motivations and accusing him of hypocrisy in front of everyone.\\
Gatsby avoids eye contact, becomes visibly agitated, and deflects the question by bringing up a completely unrelated topic.\\
Gatsby abruptly raises his voice, passionately declaring his love for Daisy and confronting Tom about his treatment of her.\\
\\<context 10>\\
Gatsby politely invited Tom to continue with his questioning. After that, The tension in the room escalated as Tom accused Gatsby of causing trouble and challenged the modern attitudes towards family life. Daisy attempted to diffuse the situation, but Tom's anger and prejudice became more apparent. Jordan and Nick reacted with discomfort to Tom's outburst. As the argument intensified, Daisy interrupted, pleading for everyone to leave. Nick agreed and suggested they all go home, but Tom insisted on hearing what Gatsby had to say. After this or in response to this, what behavior did Jay Gatsby take?
\\<key\_behavior 10>\\
Gatsby directly told Tom that his wife doesn't love him.
\\<distracters 10>\\
Gatsby calmly acknowledged the tension and suggested a private conversation with Tom to resolve the issue.\\
Gatsby remained silent and avoided escalating the argument further, allowing others to take the lead.\\
Gatsby expressed understanding of Tom's concerns and reassured him that he had no intention to disrupt the family.\\
\\<context 11>\\
Gatsby directly told Tom that his wife doesn't love him. After that, Gatsby revealed that he and Daisy had been in love for five years without Tom's knowledge. Tom became furious and denied this claim, insisting that Daisy loved him when they married and still loves him now. Tom admitted his own infidelities but claimed he always came back to Daisy. Daisy expressed her disgust at Tom and confronted him about their departure from Chicago. After this or in response to this, what behavior did Jay Gatsby take?
\\<key\_behavior 11>\\
Gatsby walked over and stood beside Daisy to show his support when she confronted Tom about their departure from Chicago
\\<distracters 11>\\
Gatsby stayed silent and remained seated, avoiding direct involvement or confrontation.\\
Gatsby aggressively confronted Tom, escalating the conflict and directly challenging his assertions about Daisy.\\
Gatsby distanced himself from the interaction entirely, stepping away from both Daisy and Tom as they argued.\\
\\<context 12>\\
Gatsby walked over and stood beside Daisy to show his support when she confronted Tom about their departure from Chicago. After that, Gatsby urged Daisy to tell Tom the truth about never loving him, insisting that it would erase their past. Daisy hesitated and looked to Jordan and the narrator for support before reluctantly stating she never loved Tom. Tom questioned her about specific moments in their past, but Daisy refused to engage further. She then turned to Gatsby, acknowledging his presence but expressing frustration at his expectations. After this or in response to this, what behavior did Jay Gatsby take?
\\<key\_behavior 12>\\
Gatsby's eyes opened and closed in response to Daisy's confession of loving both him and Tom.
\\<distracters 12>\\
Gatsby smiles softly at Daisy, nodding with quiet acceptance, and assures her that her feelings won't change his regard for her.\\
Gatsby abruptly interrupts, dismissing Daisy's confusion and insisting they leave together immediately regardless of what she just said.\\
Gatsby steps back, averting his gaze, and calmly acknowledges her feelings, signaling that he would respect whatever decision she makes.\\
\\<context 13>\\
Gatsby's eyes opened and closed in response to Daisy's confession of loving both him and Tom. After that, Tom harshly claimed that Daisy never truly knew Gatsby was alive and mentioned his intimate history with Daisy. After this or in response to this, what behavior did Jay Gatsby take?
\\<key\_behavior 13>\\
Gatsby insisted on speaking to Daisy alone and declared to Tom that he would no longer take care of Daisy.
\\<distracters 13>\\
Gatsby emotionally withdrew, avoiding confrontation altogether, and let Tom and Daisy decide their fate without intervention.\\
Gatsby accused Tom of manipulating Daisy and escalated the verbal conflict, refusing to back down and openly challenging Tom in front of everyone.\\
Gatsby tried to comfort Daisy and suggested that she take her time to figure out her feelings, prioritizing her emotional well-being over himself or Tom.\\
\\<context 14>\\
Gatsby insisted on speaking to Daisy alone and declared to Tom that he would no longer take care of Daisy. After that, Tom accused Gatsby of being a common swindler and revealed that he had conducted an investigation into Gatsby's affairs. Tom exposed Gatsby's involvement in illegal activities, including running drug stores that sold grain alcohol. He also mentioned Gatsby's association with Meyer Wolfsheim and accused him of being a bootlegger. After this or in response to this, what behavior did Jay Gatsby take?
\\<key\_behavior 14>\\
Gatsby calmly responded to Tom's accusations about his illegal activities and association with Meyer Wolfsheim.
\\<distracters 14>\\
Gatsby angrily shouted at Tom, denying the accusations and accusing Tom of trying to smear his name.\\
Gatsby avoided eye contact and quietly admitted to parts of the accusations, retreating from any further confrontation.\\
Gatsby immediately launched into a detailed explanation and justification of his activities, attempting to discredit Tom's claims.\\
\\<context 15>\\
Gatsby calmly responded to Tom's accusations about his illegal activities and association with Meyer Wolfsheim. After that, Tom continued his accusations by bringing up Walter Chase and the betting laws, mentioning how Wolfsheim had intimidated Walter into silence. After this or in response to this, what behavior did Jay Gatsby take?
\\<key\_behavior 15>\\
Gatsby displayed an unfamiliar yet recognizable expression on his face when confronted about his more serious criminal activities
\\<distracters 15>\\
Gatsby smirked confidently and dismissed the accusations as baseless lies, daring Tom to present any real evidence.\\
Gatsby maintained a serene expression, calmly stating that his private affairs were none of Tom's concern.\\
Gatsby suddenly raised his voice, vehemently denying the accusations and accusing Tom of hypocrisy and worse behavior.\\
\end{tcolorbox}







\end{document}
