% This must be in the first 5 lines to tell arXiv to use pdfLaTeX, which is strongly recommended.
% \pdfoutput=1
% In particular, the hyperref package requires pdfLaTeX in order to break URLs across lines.

\documentclass[11pt]{article}

% \usepackage{fontspec}
% \setmainfont{Times New Roman} % 设置主要字体为 Times New Roman 系统字体


% Change "review" to "final" to generate the final (sometimes called camera-ready) version.
% Change to "preprint" to generate a non-anonymous version with page numbers.
\usepackage{acl}
\usepackage{amssymb}
\usepackage{multirow}
% Standard package includes
\usepackage{times}
\usepackage{latexsym}
\usepackage{amsmath}
% \usepackage{xeCJK} % 导入 xeCJK 宏包
% \setCJKmainfont{SimSun} % 设置中文字体
% \usepackage{footmisc} % 引入 footmisc 包
% \renewcommand{\thefootnote}{\fnsymbol{footnote}}

% For proper rendering and hyphenation of words containing Latin characters (including in bib files)
\usepackage[T1]{fontenc}
% For Vietnamese characters
% \usepackage[T5]{fontenc}
% See https://www.latex-project.org/help/documentation/encguide.pdf for other character sets
\usepackage{float}

% This assumes your files are encoded as UTF8
% \usepackage[utf8]{inputenc}

% This is not strictly necessary, and may be commented out,
% but it will improve the layout of the manuscript,
% and will typically save some space.

% This is also not strictly necessary, and may be commented out.
% However, it will improve the aesthetics of text in
% the typewriter font.
\usepackage{inconsolata}

%Including images in your LaTeX document requires adding
%additional package(s)
\usepackage{graphicx}
% \usepackage[backref=page]{hyperref}
\usepackage[capitalise]{cleveref}
\usepackage{url}
\usepackage{booktabs}
\usepackage{bbding}
\usepackage{makecell}
% \usepackage{algorithm}
\usepackage{algpseudocode}
\usepackage{amssymb}
% \usepackage{xcolor}
\usepackage{pifont} 
\usepackage{amsfonts} % for \mathfrak
\usepackage{mathrsfs} % for \mathscr
\usepackage{tabularx}
\usepackage{colortbl}
\usepackage{lscape}
\usepackage{wrapfig}
\usepackage{subfig} 
% \usepackage{emoji}
\usepackage{amsmath}
\usepackage{amssymb}
\usepackage[linesnumbered,ruled]{algorithm2e}
% \usepackage{algorithm2e}

\usepackage{mathptmx} % 引入 Times New Roman 字体

\usepackage[english]{babel}
\usepackage{graphicx}
\usepackage[makeroom]{cancel}
\usepackage{hyperref}


\usepackage{microtype}
% \usepackage{CJKutf8}
\usepackage[utf8]{inputenc}
\usepackage[T1]{fontenc}


\usepackage{colortbl}
\usepackage{booktabs}


\usepackage{tcolorbox}
\tcbuselibrary{breakable}

% \usepackage[table]{xcolor}
% \usepackage[transparent]{xcolor}
\usepackage{xcolor}

% % 200,220,255
% \definecolor{mc-color}{RGB}{245, 250, 255} % 淡蓝色
% \definecolor{gen-color}{RGB}{255, 245, 245} % 淡红色
% \definecolor{mo del-color}{RGB}{255, 245, 245} % 淡红色

\definecolor{mc-color}{RGB}{245, 250, 255} % 淡蓝色
% \definecolor{gen-color}{RGB}{255, 245, 245} % 淡绿色
\definecolor{gen-color}{RGB}{247,254,247}
% {224,251,226} % 淡绿色
% rgb(247,254,247)
\definecolor{model-color}{RGB}{250, 250, 250}

\definecolor{dark-grey}{RGB}{245,245,245}

\definecolor{white}{RGB}{255,255,255}


% 定义带有透明度的颜色

% \definecolor{mc-color}{RGB, opacity=0.5}{173, 216, 230} % 淡蓝色,透明度 50%
% \definecolor{gen-color}{RGB, opacity=0.8}{220,255,220} % 淡绿色,透明度 80%

\newcommand{\datasetname}{\textsc{BehaviorChain}\xspace}
\newcommand{\meone}
{AvgAcc\xspace}
\newcommand{\metwo}
{ChainAcc\xspace}



% Define color for symbols
% \definecolor{red}{rgb}{1.0, 0.0, 0.0}
% \definecolor{green}{rgb}{0.0, 1.0, 0.0}

% \newcommand{\xmark}{\textcolor{red}{\ding{55}}}
% \newcommand{\headercolor}{\rowcolor{gray!15}}




% If the title and author information does not fit in the area allocated, uncomment the following
%
%\setlength\titlebox{<dim>}
%
% and set <dim> to something 5cm or larger.

% Enhancing Uncertainty Estimation for Large Language Models with Instance-level Correctors
% Enhancing Uncertainty Estimation for Large Language Models
% Towards Harmonized Uncertainty Estimation for Large Language Models
\title{How Far are LLMs from Being Our Digital Twins? \\A Benchmark for Persona-Based Behavior Chain Simulation}

% Author information can be set in various styles:
% For several authors from the same institution:
% \author{Author 1 \and ... \and Author n \\
%         Address line \\ ... \\ Address line}
% if the names do not fit well on one line use
%         Author 1 \\ {\bf Author 2} \\ ... \\ {\bf Author n} \\
% For authors from different institutions:
% \author{Author 1 \\ Address line \\  ... \\ Address line
%         \And  ... \And
%         Author n \\ Address line \\ ... \\ Address line}
% To start a separate ``row'' of authors use \AND, as in
% \author{Author 1 \\ Address line \\  ... \\ Address line
%         \AND
%         Author 2 \\ Address line \\ ... \\ Address line \And
%         Author 3 \\ Address line \\ ... \\ Address line}


\author{
    Rui Li\textsuperscript{1},
    Heming Xia\textsuperscript{2},
    Xinfeng Yuan\textsuperscript{3},
    Qingxiu Dong\textsuperscript{1},
    Lei Sha\textsuperscript{4},
    Wenjie Li\textsuperscript{2},
    Zhifang Sui\textsuperscript{1} \\
        \textsuperscript{1}Peking University \quad
        \textsuperscript{2}The Hong Kong Polytechnic University \\
        \textsuperscript{3}Fudan University \quad
        \textsuperscript{4}Beihang University\\
    \texttt{o\_l1ru1@stu.pku.edu.cn, he-ming.xia@connect.polyu.hk}
}

% \textsuperscript{1}State Key Laboratory of Multimedia Information Processing, School of Computer Science, Peking University} \\
% \small{\textsuperscript{2}The Hong Kong Polytechnic University} \\
% \small{\textsuperscript{3}Fudan University} \\
% \small{\textsuperscript{4}Institute of Artificial Intelligence, Beihang University} \\
% \small{\texttt{o\_l1ru1@stu.pku.edu.cn} \texttt{he-ming.xia@connect.polyu.hk}}
% }

% he-ming.xia@connect.polyu.hk

%\author{
%  \textbf{First Author\textsuperscript{1}},
%  \textbf{Second Author\textsuperscript{1,2}},
%  \textbf{Third T. Author\textsuperscript{1}},
%  \textbf{Fourth Author\textsuperscript{1}},
%\\
%  \textbf{Fifth Author\textsuperscript{1,2}},
%  \textbf{Sixth Author\textsuperscript{1}},
%  \textbf{Seventh Author\textsuperscript{1}},
%  \textbf{Eighth Author \textsuperscript{1,2,3,4}},
%\\
%  \textbf{Ninth Author\textsuperscript{1}},
%  \textbf{Tenth Author\textsuperscript{1}},
%  \textbf{Eleventh E. Author\textsuperscript{1,2,3,4,5}},
%  \textbf{Twelfth Author\textsuperscript{1}},
%\\
%  \textbf{Thirteenth Author\textsuperscript{3}},
%  \textbf{Fourteenth F. Author\textsuperscript{2,4}},
%  \textbf{Fifteenth Author\textsuperscript{1}},
%  \textbf{Sixteenth Author\textsuperscript{1}},
%\\
%  \textbf{Seventeenth S. Author\textsuperscript{4,5}},
%  \textbf{Eighteenth Author\textsuperscript{3,4}},
%  \textbf{Nineteenth N. Author\textsuperscript{2,5}},
%  \textbf{Twentieth Author\textsuperscript{1}}
%\\
%\\
%  \textsuperscript{1}Affiliation 1,
%  \textsuperscript{2}Affiliation 2,
%  \textsuperscript{3}Affiliation 3,
%  \textsuperscript{4}Affiliation 4,
%  \textsuperscript{5}Affiliation 5
%\\
%  \small{
%    \textbf{Correspondence:} \href{mailto:email@domain}{email@domain}
%  }
%}

\begin{document}
\maketitle
% \footnotetext{$^\dagger$Equal contribution} % 在 author 环境外添加角注内容

% \footnotetext{$^\dagger$Equal Contributions.}
\begin{abstract}




% Recent advances in large language models (LLMs) have revolutionized the artificial intelligence landscape, 
% enabling seamless integration into human workflows. 
% Given their remarkable capabilities in language processing and human-like cognitive functions, 


% However, this potential contrasts with prevailing research trends, which predominantly emphasize human dialogue simulation while neglecting comprehensive investigations into behavior simulation—a critical dimension of digital twins.
Recently, LLMs have garnered increasing attention across academic disciplines for their potential as human \textbf{\textit{digital twins}}, virtual proxies designed to replicate individuals and autonomously perform tasks such as decision-making, problem-solving, and reasoning on their behalf.
However, current evaluations of LLMs primarily emphasize dialogue simulation while overlooking human behavior simulation, which is crucial for digital twins.
To address this gap, we introduce \datasetname, the first benchmark for evaluating LLMs' ability to simulate continuous human behavior.
\datasetname comprises diverse, high-quality, persona-based behavior chains, totaling 15,846 distinct behaviors across 1,001 unique personas, each with detailed history and profile metadata.
For evaluation, we integrate persona metadata into LLMs and employ them to iteratively infer contextually appropriate behaviors within dynamic scenarios provided by \datasetname. Comprehensive evaluation results demonstrated that even state-of-the-art models struggle with accurately simulating continuous human behavior.
Resources are available at \href{https://github.com/O-L1RU1/BehaviorChain}{https://github.com/O-L1RU1/BehaviorChain}

% For evaluation
% \datasetname provides a robust environment for evaluation, where


 % enabling the simulation of complex behavioral dynamics.
% For evaluation, we integrate persona metadata into LLMs and iteratively infer contextually appropriate behavior nodes in dynamic scenarios. 
% within complex, dynamically evolving scenarios until the end of a chain.
% to iteratively generate contextually appropriate behavior in dynamic scenarios for evaluation.
% It comprises diverse, high-quality, persona-based behavior chains, totaling 15,846 distinct behaviors across 1,001 unique personas, each with detailed history and profile metadata.
% We established an evaluation framework by integrating persona metadata into LLMs and using them to iteratively recognize or generate contextually appropriate behavior node-by-node within complex, dynamically evolving scenarios until the end of a chain.

% Based on their promising performance in language capabilities and human-like cognition, growing research interest has emerged across disciplines to explore LLMs' potential as human \textbf{\textit{digital twins}} - virtual avatars capable of on behalf of specific individuals to perform tasks such as decision-making and reasoning.
% growing research interest has emerged across disciplines to explore LLMs' potential as human \textbf{\textit{digital twins}} - virtual representations on behalf of specific individuals to perform tasks such as decision-making and reasoning.
 % augment decision-making, and replicate complex behaviors.
% provide personalized assistance, 
% conduct a 
% decision-making, or interact autonomously in virtual reality.
% undergo predictive scenario testing,
% However, existing work predominantly focuses on dialogue simulation, overlooking the critical need for temporally consistent behavior modeling in dynamic contexts. 
% To address this gap, we introduce \datasetname, the first benchmark designed to evaluate LLMs’ ability to simulate continuous human behaviors. This benchmark comprises 1,001 diverse, high-quality persona-based ``behavior chains'', encompassing 15,846 distinct behaviors across 1,001 unique personas.  Each persona is accompanied by metadata including detailed history and profile.
% we introduce \datasetname, a novel benchmark featuring 1,001 high-quality persona-based ``behavior chains'', with each chain incorporating a distinct persona enriched with metadata including detailed history and profile. 
% automatically extracted from novels and biographies, contextualized within plot and scene transitions between behaviors.
% Each persona is further characterized by detailed profiles and history.





% Digital twin$-$a virtual replica mirroring a real-world entity$-$holds immense fascination. 
% LLMs' remarkable progress in language capabilities and human-like cognition is driving the vision forward with the potential of LLMs as human digital twins.
% However, this promising potential contrasts sharply with the current work on LLM-powered persona simulation, which neglects the critical dimension of character behavior and its consistency over spatiotemporal ranges.
% % Existing research on LLM-based persona simulation neglects the critical dimension of character behavior and its consistency over spatiotemporal ranges.
% % Related research on LLMs' capabilities in persona-around simulation while overlooks the critical dimension of character behavior and character's consistency across a spatiotemporal range.
% % focusing on dialogue, decision-making, and knowledge within point-in-time context, it
% % , which is a more chanllenging but realistic scenario.
% To fill this gap, we introduce \datasetname, a novel benchmark comprising 1001 high-quality persona-based ``behavior chains'' extracted automatically from novels and biographies, contextualized within plot and scene transitions between behaviors as context and enriched the character with detailed personal histories.
% % detailed character profiles and histories.
% % real-world narratives 
% % novels and biographies
% % This task is designed to provide a comprehensive and longitudinal evaluation of LLMs' ability to discern and maintain character-consistent behaviors.
% We established an evaluation framework by equipping LLMs with character profile and history, and employing them to  gradually recognize\&generate each behavior in a chain until end. The process followed complex and dynamically changing context.
% % Performance is qualified quantifying individual behavior node correctness, and \textit{Chain Accuracy}, assessing continuity across the behavior chain.
% Evaluation on current state-of-the-art models demonstrated that they still fail to accurately simulate human behavior continuously.
% substantial deficiencies in their ability to accurately simulate human behavior, particularly continuously.

% using two metrics: \textit{Average Accuracy}, 
% \textit{Chain Accuracy}

% As digital twin is a critical frontier in the development of LLMs. 
% Our work has uncovered pervasive challenges and offers valuable insights for future research in this domain and provide guidance and foster further exploration of LLMs as digital twins. 
% The dataset and the code are publicly available at [URL].


\end{abstract}

\section{Introduction}
\section{Introduction}\label{sec:intro}

In computational finance, Monte Carlo simulations are used extensively to estimate the expected value of financial payoffs based on the solution of stochastic differential equations (SDEs) which model the evolution of stock prices, interest rates, exchange rates and other quantities \cite{glasserman04}.  Monte Carlo methods are very general and flexible, but for high accuracy it requires generating a large number of costly SDE path approximations, which has motivated research into a number of variance reduction or, equivalently, cost reduction techniques. One such method is
Multilevel Monte Carlo (MLMC), which was proposed in \cite{GILES2008} and was adapted for various applications that are summarised in \cite{Giles_overview17} and successfully combined with other methods such as quasi-Monte Carlo methods. The main idea of MLMC is to approximate the payoff using different time stepping resolutions when numerically solving the underlying SDE and to generate an optimal number of samples on each level, such that the overall computational cost is minimised subject to the desired bound on the variance. %, such that the total computational cost is minimised. 
The computational savings come from the fact that most samples are computed on the coarser levels and hence are less expensive while only a few samples from the finest levels are required \cite{GILES2008}.


Among the directions in which the computational cost 
of MLMC methods could further be reduced, an important avenue is the use of lower precision calculations, especially for the first Monte Carlo levels where the targeted accuracy is relatively low. 
 An overview of the research on mixed precision for the standard Monte Carlo (MC) framework is provided in \cite{ChowMixedPrecisionStandardMC} but only a few references study the potential of low precision computation in the MLMC framework \cite{Rounding_error_oliver}. To the best of our knowledge, the only MLMC framework with customised precision in the literature is \cite{brugger2014mixed}, but they use a uniform precision for all operations on each Monte Carlo level instead of optimising 
 the precision of each intermediary variable to reduce as much as possible the cost of path generation.
 
An important motivation for an MLMC framework with variable precision would be performing the low precision computations on reconfigurable hardware devices such as Field Programmable Gate Arrays (FPGAs). FPGAs contain customizable logic blocks and connectors that make it easy to adapt the digital circuit architecture for a specific application, leading to a highly parallel and optimised implementation. Therefore they are successfully exploited in applications that require high speed and have high computational workload, such as signal processing \cite{woods2008fpga}, and real time applications like high frequency trading \cite{HFT1,HFT2}. That is why a number of previous works in hardware architecture design implemented the MLMC algorithm to price financial options using FPGAs as accelerators, which resulted in improved speed and power efficiency compared to full CPU architectures \cite{Schryver2013AMM}. The paper \cite{lindsey2016domain} also proposed 
a Domain Specific Language to automate the configuration of FPGAs for this specific application. However, only \cite{brugger2014mixed} proposed a heuristic to reduce the precision in calculations.

In addition, all aforementioned works considered that the random number generation (RNG) is performed in single or double precision. Yet in most cases an important portion of the workload in the overall MLMC simulation comes from the RNG and in \cite{brugger2014mixed} this limited the total computational savings.
To reduce the cost of MLMC simulations in particular those based on the Geometric Brownian Motion (GBM), \cite{approximateICDF_Oliver, NestedOliver} have proposed to use approximate random numbers that are generated by applying an approximation of the inverse CDF to uniform random numbers. In \cite{NestedOliver}, the authors proposed a way to integrate these lower precision random variables into a \textit{nested} MLMC framework and completed a numerical analysis to bound the resulting error at each MC level by a product of the time step and the error in the random number approximation. The same authors show in \cite{approximateICDF_Oliver} that using approximate random variables reduces the cost of path generation by a factor 7.


In this paper we propose a nested MLMC framework that combines the use of approximate random normal variables and lower precision calculations to reduce the computational cost of MLMC even further than \cite{brugger2014mixed,NestedOliver}. We illustrate the efficiency of our framework in Matlab, after making several assumptions on the cost of operations and size of the errors that we carefully justify. We focus on the case of GBM and use the approximate RNG methods presented in \cite{approximateICDF_Oliver} as well as a new slightly modified method that combines CDF inversion and the central limit theorem. To choose the precision of the variables in the low precision path generation, we introduce a novel method to optimise the bit-widths. This optimisation is performed before the main path generation loop is executed and is based on a linear model of the payoff error  
due to rounding when computing in low precision. The error model relies on algorithmic differentiation in a similar manner to \cite{unifying-bwoptim,bitwidth-AD,ADAPT}. The bit-width optimisation procedure can be performed off-line, so this stage can be excluded from the on-line time complexity of our framework. The user specified desired accuracy is then enforced by calculating on-line the number of samples that need to be generated.

In terms of hardware design, we suggest implementing the low precision path generation on FPGAs and the full-precision ones on a CPU or GPU. 
The FPGA offers enough flexibility to define a separate bit-width for every variable in the low precision path generation, and can be reconfigured periodically to update the bit-widths when the market parameters have changed considerably. 


The paper is organized as follows : \Cref{sec:MLMC} introduces MLMC and nested MLMC to make clear the estimator that is implemented in our framework. Then in \Cref{sec:RNG} we detail the methods that could be used to obtain approximate random normally distributed numbers very cheaply for the low precision path generation. In \Cref{sec:error_model} and \Cref{sec:costModel} we propose an error model and a cost model (resp.) that we then use to formulate the optimisation problem that is solved to obtain the optimal bit-widths of fixed point variables in \Cref{sec:optimisation}. Finally we summarise our results and future directions in \Cref{sec:conclusion}.




% \begin{figure*}[t]
%     \centering
%     \includegraphics[width=1\textwidth]{latex/picture/digital_twin.pdf}
%     \caption{An overview of behavior chain benchmark framework.}
%     \label{fig:workflow}
% \end{figure*}




\section{Related Work}
\section{Related Work}
\label{sec:related work}
% In this section, we review the existing literature on point cloud denoising and unsupervised image denoising.
%-------------------------------------------------------------------------
\subsection{Point cloud denoising}

    \subsubsection{Traditional methods}
Traditional approaches to point cloud denoising include statistical methods \cite{computingpointset2003,definingpointset2004,wlop2009HH}, filtering techniques\cite{pointsetsurfaces2001,Robustmoving2005, zaman2017density}, and optimization-based methods \cite{l1sparse2010,clop2014PR,digne2017bilateral,multi-projection2018duan,hu2020featuregraph} . These techniques often rely on handcrafted features and heuristics to distinguish signal from noise. For example, statistical methods may use distribution models to identify and remove outliers. Filtering methods, such as mean or median filters, operate under the assumption that noise is statistically different from the signal. Optimization-based methods formulate denoising as an energy minimization problem, where regularization terms constrain the solution to ensure certain smoothness cirterion or adherence to prior knowledge.

%-------------------------------------------------------------------------
    \subsubsection{Supervised learning based methods}
In recent years, several deep learning-based methods \cite{rakotosaona2020PCN,hermosilla2019TotalDenoising,luo2020DMR,luo_score-based_2021} have been proposed for point cloud denoising. NPD \cite{NPD2019} is the first learning-based point cloud denoising method that directly processes noisy data without requiring noise characteristics or neighboring point definitions. This approach combines local and global information by projecting noisy points onto estimated reference planes, effectively removing noise and enhancing robustness against variations in noise intensity and curvature. PointCleanNet\cite{rakotosaona2020PCN} first removes outlier points and then combines them with residual connectivity to predict the inverse displacement \cite{Guerrero2017PCPNetLL}, and iteratively shifts noisy points to remove noise. Pistilli \etal proposed GPDNet \cite{gpdnet2020}, which is a graph convolutional network to improve denoising robustness at high noise levels. Luo \etal also proposed  DMRDenoise \cite{luo2020DMR}, which filter
points by first downsampling the noisy inputs and reconstructing the local subsurface to perform point upsampling. However, this resampling method is difficult to maintain a good local shape. ScoreDenoise \cite{luo_score-based_2021} is proposed to tackle the aforementioned issues by iteratively updating the point position in implicit gradient fields learned by neural networks. For inference, they follows an iterative procedure with a decaying step size, which stabilizes point movement and prevents over-correction, allowing points to converge gradually toward the underlying geometry. The authors of \cite{de_Silva_Edirimuni_2023_CVPR} proposed IterativePFN - an iterative method that use a novel loss function that utilizes an adaptive ground truth target at each iteration to capture the relationship between intermediate filtering results during training. Zheng \etal proposed a end-to-end network for joint normal filtering-based point cloud denoising \cite{10173632}. They introduce an auxiliary normal filtering task to enhance the network's ability to remove noise while preserving geometric features more accurately.

Supervised methods can achieve impressive results, but are limited by the availability and quality of the training data, as they typically require paired noisy and clean point clouds to train the neural network. The absence of clean data in real-world scenario pose a significant challenge on applicability of these algorithms.

%-------------------------------------------------------------------------
    \subsubsection{Unsupervised learning methods}
Unsupervised learning-based methods for point cloud denoising do not require ground-truth clean data. Instead, these methods leverage the inherent structure or distribution of the point cloud to guide the denoising process. Unsupervised methods show promise in scenarios where clean data is absent or hard to obtain.

Hermosilla \etal first introduced Total Denoising (TotalDn) \cite{hermosilla2019TotalDenoising} as an unsupervised learning approach for point cloud denoising, relying solely on noisy data without requiring clean ground truth. TotalDn approximates the underlying surfaces by regressing points from the distribution of unstructured total noise, utilizing a spatial prior term to refine the estimation of geometry. 

In DMRDenoise \cite{luo2020DMR}, an unsupervised version is proposed which utilizes a loss function that identify local neighborhoods using a probabilistic Gaussian mask on the k-nearest neighbors, which selectively retains points likely to represent the underlying surface. By leveraging an Earth Mover's Distance (EMD) assignment, it achieves a one-to-one correspondence between input and predicted points, aligning them to reduce noise within local neighborhoods.
This approach enhances robustness to noise and adapts well to varied surface geometries. However, the probabilistic masking and EMD calculation add computational complexity, which can slow down inference in dense or noisy point clouds. 

ScoreDenoise \cite{luo_score-based_2021} also introduced an unsupervised version that employs ensemble score function and an adaptive neighborhood-covering loss for model training.  
Score-u is probably the most relevant work to our method. However, the defined score in \cite{luo_score-based_2021} is only an displacement-alike version of the score function and there is no explicit formula relating the estimated score to the final denoising result. Also, the iterative process is computationally expensive, and can suffer from fluctuation, leading to perturbed and unstable solution.

Most recently, Noise4Denoise \cite{noise4Wang2024} method is proposed that use an additional doubly-noisy point cloud to learn noise displacement by comparing the two noise levels. This approach effectively leverages synthetic noise for training, allowing the model to estimate residuals without relying on clean data.
However, in practical applications, noise parameters are often unknown and variable, making it challenging to replicate the exact conditions assumed during training. This reliance on predefined noise characteristics can limit the model's applicability to real-world scenarios where noise distributions may differ significantly from synthetic settings. 
%-------------------------------------------------------------------------
\subsection{Unsupervised image denoising}
Recently unsupervised image denoising has made significant progress. Non-Bayesian methods include PURE \cite{luisier2010image}, SURE \cite{SURE2018} \textit{etc.}, which are based on various unbiased risk estimator under certain noise distribution. Other methods explore self-similarity in natural images \cite{xu2015patch, doi:10.1137/23M1614456} or exploits the statistical properties of noise to achieve denoising effect \cite{gravel2004method}.  

Noise2Noise \cite{2018Noise2NoiseLI} is a pioneering method that does not require clean data, but it requires multiple noisy versions of the same image for training. To address this limitation, methods such as Noise2Void \cite{2018Noise2VoidL}, Noise2Self \cite{2019Noise2SelfBD}, \textit{etc.}, have been developed, which use only a single noisy image. These methods are particularly important for practical applications where obtaining clean images or multiple noisy realizations of the same image is difficult or impossible. Neighbor2Neighbor \cite{huang2021neighbor2neighbor} proposed a two-step method with a a random neighbor sub-sampler that generates training  pairs and a denosing network. Kim \etal proposed Noise2Score\cite{kim_noise2score_2021}, a novel Bayesian framework for self-supervised image denoising without clean data. The core of Noise2Score is the usage of Tweedie's formula, which provides an explicit representation of the denoised image through a score function. Combined with score function estimation, Noise2Score can be applied to image denoising with any exponential family noise. Kim \etal also proposed the Noise Distribution Adaptive Self-Supervised Image Denoising method \cite{kim_noise_2022}, which further extends the application of Noise2Score by combining the Tweedie distribution with score matching. This enables adaptive handling of various noise distributions and dynamically adjusts the denoising process by estimating noise parameters. On the other hand, Xie \etal \cite{scoreXie2024} broadened the denoising scope of Noise2Score by allowing it to handle complex noise models, such as multiplicative and structurally correlated noise, demonstrating broad applicability to diverse noise models.

These development of unsupervised image denoising method motivate us to explore similar ideas in 3D point cloud denoising.





\begin{figure*}[t]
    \centering
    \includegraphics[width=1\linewidth]{latex/picture/behaviorchain.pdf}
    \caption{The left image illustrates the process of behavior chain construction. The right image shows our multi-choice task setup, where each node on the behavior chain serves as an individual evaluation. Each input includes persona's profile, history, and all context and ground truth behavior preceding the current node.}
    \label{fig:workflow}
\end{figure*}


\section{Benchmark Crafting}


\section{Methodology}
\paragraph{Preliminaries.}
We primarily focus on the homologous model merging, in which $\boldsymbol{\theta}_i$ all come from the same base model $\boldsymbol{\theta}_{\rm{base}}$. Given $K$ tasks $\{T_1,T_2,\cdots,T_K\}$ and $K$ corresponding fine-tuned models with parameters $\{\boldsymbol{\theta}_1,\boldsymbol{\theta}_2,\cdots,\boldsymbol{\theta}_K\}$, model merging aims to combine $K$ fine-tuned models into one single model simultaneously performing on $\{T_1,T_2,\cdots,T_K\}$ without post-training~\cite{method_p1_1,method_p1_2}.
Task vector~\cite{ilharco2023editing,yang2024adamerging} is a key element in merging method which could enhances the base model‘s ability or enable the model to handle other tasks. Specifically, for task $T_i$, the task vector $\boldsymbol\tau_i\in \mathbb{R}^D$ is defined as the vector obtained by subtracting the SFT weights $\boldsymbol{\theta}_i$ from the base model weight
$\boldsymbol{\theta}_{\rm{base}}$, \emph{i.e.}, $\boldsymbol\tau_i=\boldsymbol{\theta}_i-\boldsymbol{\theta}_{\rm{base}}$. The merged model could be denoted as $\boldsymbol{\theta}_m=\boldsymbol{\theta}_{\rm{base}}+\sum_i \lambda_i\boldsymbol{\tau}_i$, which $\lambda_i$ is the scaling factor measuring the importance of task vector. For clarification, we also denote the neuron set in $\boldsymbol{\theta}_i$ as $\mathcal{N}_i$, the neuron set in $\boldsymbol{\tau}_i$ as $\mathcal{T}_i$.



\begin{algorithm}[!ht]
    \caption{LED-Merging}
    \label{alg1}
    \begin{algorithmic}[1]
        \REQUIRE  base model $\boldsymbol{\theta}_{\rm{base}}$, SFT models $\{\boldsymbol{\theta}_{i}\mid i\in [K]\}$, mask ratios \{$r_{i} \mid i\in [K]\}$, scaling factors $\{\lambda_i\mid i\in[K]\}$, location datasets $\{\mathcal{X}_{i}\mid i\in[K]\}$
        \ENSURE merged parameter $\boldsymbol{\theta}_{m}$
        \STATE $\mathcal{M}\leftarrow\phi$
        \STATE $\boldsymbol{\theta}_{m}\leftarrow \boldsymbol{\theta}_{\rm{base}}$
        \FOR{$i\in [K]$}
        \STATE $I(\boldsymbol{\theta}_i)=\mathbb{E}_{x\sim \mathcal{X}_i}|\boldsymbol{\theta}_{i}\odot \nabla_{\boldsymbol{\theta}_i}\mathcal{L}(x)|$
        \STATE $I(\boldsymbol{\theta}_{\rm{base}})=\mathbb{E}_{x\sim \mathcal{X}_i}|\boldsymbol{\theta}_{\rm{base}}\odot \nabla_{\boldsymbol{\theta}_{\rm{base}}}\mathcal{L}(x)|$
        
        \STATE calculate $\mathcal{T}^{r_i}_{i}$ following Equation \ref{vote}
        \STATE  $\mathcal{M}\leftarrow \mathcal{M}\cup\{\mathcal{T}^{r_i}_i\}$
       
        
   
        
        
        \ENDFOR  
        \FOR{$i\in [K]$}
        
        \STATE calculate $\text{Disjoint}(\mathcal{T}_i^{r_i})$ use Equation~\ref{disjoint_safety}
        \STATE $\boldsymbol{m}_i \leftarrow \boldsymbol{0}$
        \FOR{$d\in \mathcal{T}_i^{r_i}$}
        \STATE $\boldsymbol{m}_{i,d}=1$
        \ENDFOR
        \STATE $\boldsymbol{\theta}_{m}\leftarrow \boldsymbol{\theta}_{m}+\lambda_i \boldsymbol{\tau}_i\odot \boldsymbol{m}_{i}$
        \ENDFOR
    \end{algorithmic}
\end{algorithm}
    %\vspace{-5pt}
\begin{figure*}[h!]
    \centering
    \includegraphics[width=\linewidth]{figs/pipeline_v2.pdf}
    \vspace{-40mm}
    \caption{Overview of our two-stage training pipeline {\ours}.}
    \label{fig:pipeline}
\end{figure*}


\paragraph{LED-Merging: Location, Election, and Disjoint Merging}
To address the neuron misidentification and interference issues in existing model merging methods, we propose LED-Merging (Location, Election, and Disjoint Merging). Specifically, previous studies \cite{modelstock, ilharco2023editing, tiesmerging} fail to accurately identify safety-related neurons in task vectors with a single magnitude score, namely \textit{neuron misidentification}. Meanwhile, there exists an interference between safety-related and utility-related task vector neurons during the merging process, namely \textit{neuron interference}. To address neuron misidentification, we first locate important neurons both in the base and fine-tuned models and then elect neurons from the task vector considering these two scores together. Subsequently, to mitigate the interference, we introduce a disjoint step, isolating these important neurons so that they influence different base neurons. The whole process is illustrated in Figure~\ref{fig:method}. 




In the location and election step, we consider the importance score from base and fine-tuned models simultaneously to locate task-specific neurons. In this way, it is more accurate than relying on the magnitude score alone because task-specific neurons with high importance score in the fine-tuned model may not necessarily score high in the base model, and vice versa.

{\textbf{Location}}.  We first calculate importance scores for each neuron in a base/fine-tuned model. Given a location dataset $\mathcal{X}_i=\{(x,y)_k\}$, where $x$ is the question and $y$ is the answer, we calculate the importance scores for the weight $\boldsymbol{\theta}_i\in\mathbb{R}^D$ in any  layer as follows~\cite{snip,spareseGPT,sun2024a}:
\begin{equation}
    I(\boldsymbol{\theta}_i)=\mathbb{E}_{x\sim \mathcal{X}_i}[\boldsymbol{\theta}_i\odot \nabla _{\boldsymbol{\theta}_i}\mathcal{L}(x)],
    \label{location}
\end{equation}
which $\mathcal{L}(x)=-\log p(y\mid x)$ is the conditional negative log-likelihood loss. We choose the SNIP score~\cite{snip} because it balances computational efficiency and performance~\cite{cq}. Please refer to Sec.~\ref{sec:ablation} for the comparison between different location methods. After computing importance scores, we choose top-$r_i$ neurons as the important neuron subset $\mathcal{N}_{i}^{r_i}$ from $I(\boldsymbol{\theta}_i)$.
 
 % After computing locating scores, we select the neurons scoring both high in base and fine-tuned models as important neurons in task vectors. Then in the disjoint step,  with preventing  polysemantic neurons  from receiving gradient updates towards different directions,
 % we use set difference to isolate the safety   and utility-related neurons  and construct corresponding masks for merging process,

{\textbf{Election}}. A natural question is how to select important neurons in the task vector $\boldsymbol{\tau}_i$ based on $I(\boldsymbol{\theta}_{\rm{base}})$ and $I(\boldsymbol{\theta}_{i})$. The important neurons in the base model may be different from neurons in the fine-tuned model. Therefore, we introduce the following election strategy to select neurons with high scores in both base and fine-tuned models:
\begin{equation}
    \mathcal{T}_i^{r_i}=\mathcal{N}_i^{r_i}\cap \mathcal{N}_{\rm{base}}^{r_i}.
    \label{vote}
\end{equation}
\emph{Remark}. We compare different choosing methods, including scoring low or high in base or fine-tuned model in Section~\ref{sec:ablation} and find that Equation \ref{vote} achieves the best performance.





{\textbf{Disjoint}}. As important neurons from different task vectors may conflict with each other at the same position, we use the set difference to disjoint the neurons from others to prevent interference:
\begin{equation}
    \text{Disjoint}(\mathcal{T}^{r_i}_{i})=\mathcal{T}^{r_i}_{i}-\mathop{\cup}\limits_{{J}\subsetneqq [K],|J|\geq 2}\mathop{\cap}\limits_{j\in {J}}\mathcal{T}^{r_j}_{j}.
    \label{disjoint_safety}
\end{equation}

Next, we construct a mask $\boldsymbol{m}_i\in\mathbb{R}^D$ to implement disjoint in the merging process. Specifically, this mask $\boldsymbol{m}_i$ is used to select neurons from $\mathcal{T}_i$. The mask ratio is $r_i$, where $r\in(0,1]$. The mask $\boldsymbol{m}_i$ can be derived from:
\begin{equation}
    \boldsymbol{m}_{i,d}=\begin{aligned} &\left\{ \begin{array}{ll} 1, & \text{if } d\in \text{Disjoint}(\mathcal{T}_{i}^{r_i}), \\ 0, & \text{otherwise}. \end{array} \right. \end{aligned}
    \label{mask_safety}
\end{equation}


% \subsection{Merging Models with Masks}
{\textbf{Merging}}. The final
merged task vector $\boldsymbol{\tau}_m$ is as follows:
\begin{equation}
    \boldsymbol{\tau}_m= \sum_i \lambda_i\boldsymbol{\tau}_{i}\odot\boldsymbol{m}_i.
    \label{merged_task_vector}
\end{equation}
We summarize the workflow in Algorithm \ref{alg1}.





\section{Experiments}
\section{Experiments}
\label{sec:experiments}

\begin{figure*}[t]
\vspace{-6mm}
    \centering
    \includegraphics[width=0.8\linewidth]{figs/compare.pdf}
    \vspace{-4mm}
    \caption{\textbf{Qualitative comparison} with the baseline for generating a sequence of novel view images.  
    The results demonstrate that our method synthesizes more consistent multi-view images compared to our baseline model (Zero123). In addition, compared to SyncDreamer, our method visually maintains better similarity to the conditioned image and appears more natural.}
    \label{fig:sota_compare}
\vspace{-5mm}
\end{figure*}

\subsection{Experimental Setups}
\textbf{Dataset.}
Following previous work~\cite{zero123, SyncDreamer}, we evaluate our work on the Google Scanned Object (GSO)~\cite{GSO} dataset to verify the zero-shot novel view image synthesis capability. 
We also provide results for additional datasets in the Supplementary Material.
Specifically, we randomly select 30 objects from the GSO dataset with various object categories. 
Unlike recent approaches~\cite{mvdream, SyncDreamer} that aim to enhance the consistency of novel view synthesis models by generating multiple fixed-view images, our method can generate images from any camera pose and any number of views. Therefore, we conduct experiments under different camera pose settings to validate our approach:
specifically, 
1) \textit{16-views with free camera pose}: for each object, we circularly render 16 views with the elevation angles ranging in $[-10\degree, 40\degree]$ and the azimuth angles are evenly distributed in $[0\degree, 360\degree]$. 
2) \textit{16-views with fixed camera pose}: We maintain a constant elevation angle of $30\degree$ and uniformly sample azimuth angles (same as SyncDreamer~\cite{SyncDreamer}).
3) \textit{32-views with free camera pose}: Similar to the first setting, but we sample 32 views.
It's important to note that our method does not require additional training or fine-tuning on any datasets.

\noindent\textbf{Metrics.}
To validate the effectiveness of our method, we mainly evaluate it based on three criteria:
1) \textit{Quality Score}. We evaluate the image quality of synthesized multi-view images by measuring their similarity with ground truth images. Following prior research~\cite{zero123, sparsefusion}, we report the similarity between the synthesized images and the ground truth images with standard metrics: PSNR, SSIM~\cite{ssim}, and LPIPS~\cite{lpips}.
2) \textit{Multi-view Consistency Score}. As the primary goal of our work is to improve the consistency of generated images, we also employ the 3D consistency score~\cite{3dim} to verify the consistency among the synthesized images. Specifically, we train an Instant-NGP~\cite{instant_ngp} with the input image and part of the synthesized novel view images of our model and evaluate the similarity between the remaining synthesized images and the rendered images of Instant-NGP. For the synthesized multi-view images of each object, we allocate $3/4$ for training and reserve the remaining $1/4$ for validation.
Intuitively, if the consistency of synthesized images is improved, the NeRF-like model will train a better object representation, and the re-rendered images will agree more with the validation images.
3) \textit{Input Consistency Score}. To assess the faithfulness of synthesized images in preserving the identity of the input condition image, we introduce the input consistency score. This score calculates the similarity of each synthesized image with the input condition image, utilizing the LPIPS metric.

In addition, we use synthesized multi-view images to train a neural 3D reconstruction model (NeuS~\cite{neus}) and report commonly used Chamfer Distances (CD) and Volume IoUs between the trained 3D model and the ground truth.

\noindent\textbf{Baselines.}
Given that our main goal is to improve the consistency of the trained baseline model without further fine-tuning, we mainly compare our approach with the used baseline model Zero123~\cite{zero123}. Additionally, we compare our method to the SOTA approaches such as PGD~\cite{tseng2023consistent} and SyncDreamer~\cite{SyncDreamer} using the same Zero123 base model.

\noindent\textbf{Implementation Details.}
We use the official checkpoint provided by Zero123~\cite{zero123}, which is trained on objaverse~\cite{objaverse} for 165,000 steps. We inject our epipolar attention layer after step $T=4$ and layer $L=10$ by default. We find that feature fusion weight $\alpha=0.5$, and the number of context views $M=2$ work better.

\begin{table}[t]
\centering
\caption{Comparison of multi-view consistency, image quality, and input consistency of synthesized multi-view images at the 16-view setting with free camera pose.}
\label{tab:view16_free_compare}
\vspace{-2mm}
\scalebox{0.6}{
\begin{tabular}{c ccc ccc c}
\toprule
              & \multicolumn{3}{c}{Multi-view Consistency} & \multicolumn{3}{c}{Quality Score} & \multicolumn{1}{c}{Input Consis.} \\
              \cmidrule(lr){2-4} \cmidrule(lr){5-7} \cmidrule(lr){8-8}
              & PSNR$\uparrow$  & SSIM$\uparrow$ & LPIPS$\downarrow$ 
              & PSNR$\uparrow$  & SSIM$\uparrow$ & LPIPS$\downarrow$ 
              & LPIPS$\downarrow$ 
              \\ \midrule

Zero123
& 15.225        & 0.645       & 0.408
& 14.255        & 0.747       &	0.208
& 0.303         
\\
SyncDreamer
& 14.830        & 0.626       & 0.434
& 12.650        & 0.713       &	0.254
& 0.317         
\\
Ours 
& \best{18.300}	& \best{0.734}	& \best{0.355}
& \best{14.947}	& \best{0.763}	& \best{0.191}
& \best{0.282}
\\

\bottomrule
\end{tabular}
}
\end{table}

\begin{table}[t]
\vspace{-1mm}
\centering
\caption{Comparison of multi-view consistency, image quality, and input consistency at the 16-view setting with fixed camera pose as SyncDreamer~\cite{SyncDreamer}.}
\label{tab:view16_fxied_compare}
\vspace{-3mm}
\scalebox{0.6}{
\begin{tabular}{c ccc ccc c}
\toprule
              & \multicolumn{3}{c}{Multi-view Consistency} & \multicolumn{3}{c}{Quality Score} & \multicolumn{1}{c}{Input Consis.} \\
              \cmidrule(lr){2-4} \cmidrule(lr){5-7} \cmidrule(lr){8-8}
              & PSNR$\uparrow$  & SSIM$\uparrow$ & LPIPS$\downarrow$ 
              & PSNR$\uparrow$  & SSIM$\uparrow$ & LPIPS$\downarrow$ 
              & LPIPS$\downarrow$ 
              \\ \midrule

Zero123
& 16.556        & 0.682       & 0.378
& 14.592        & 0.750       &	0.207
& 0.305         
\\
SyncDreamer
& \best{22.424}        & \best{0.812}       & \best{0.268}
& 15.269        & 0.749       &	0.196
& 0.300         
\\
Ours 
& 21.151	& 0.780	& 0.302
& \best{15.293}	& \best{0.764}	& \best{0.184}
& \best{0.287}
\\

\bottomrule
\end{tabular}
}
\vspace{-4mm}
\end{table}


\subsection{Comparison With Baseline Models}
The quantitative comparison on three settings are shown in Tab.~\ref{tab:view16_free_compare}, Tab.~\ref{tab:view16_fxied_compare}, and Tab.~\ref{tab:view32_free_compare}. The qualitative comparison is shown in Fig.~\ref{fig:sota_compare}.

\begin{table}[t]
\centering
\caption{Comparison of multi-view consistency and image quality scores of synthesized multi-view images at the 32-view setting with free camera pose.}
\vspace{-3mm}
\label{tab:view32_free_compare}
\scalebox{0.7}{
\begin{tabular}{c ccc ccc}
\toprule
              & \multicolumn{3}{c}{Multi-view Consistency} & \multicolumn{3}{c}{Quality Score} \\
              \cmidrule(lr){2-4} \cmidrule(lr){5-7}
              & PSNR$\uparrow$  & SSIM$\uparrow$ & LPIPS$\downarrow$ 
              & PSNR$\uparrow$  & SSIM$\uparrow$ & LPIPS$\downarrow$ 
              \\ \midrule

Zero123
& 16.515        & 0.694       & 0.378
& 15.142        & 0.733       &	0.211
\\
PGD~\cite{tseng2023consistent}
& 18.481        & 0.720       & 0.343
& 15.281        & 0.739       &	0.205
\\
Ours 
& \best{20.655}	& \best{0.792}	& \best{0.305}
& \best{15.268}	& \best{0.742}	& \best{0.203}
\\

\bottomrule
\end{tabular}
}
\vspace{-3mm}
\end{table}

\begin{table*}
  [t]
  \centering
  \resizebox{\textwidth}{!}{%
  \begin{tabular}{cccccccccccc}
    \toprule \multicolumn{2}{c}{Components}                                                             & \multicolumn{5}{c}{Re-executability Rate (\%)} & \multicolumn{5}{c}{Readability (\#)} \\
    \cmidrule(lr){1-2} \cmidrule(lr){3-7} \cmidrule(lr){8-12}        \hspace{8pt}\labelemoji\hspace{8pt}                                                                & \hspace{8pt}\toolemoji\hspace{8pt}                                      & O0                                 & O1             & O2             & O3             & AVG            & O0             & O1             & O2             & O3             & AVG            \\
    \hline
    \rowcolor[rgb]{0.93,0.93,0.93}\multicolumn{12}{c}{\textbf{Initialize with LLM4Decompile-End-6.7B~\citep{llm4decompile}}}   \\
    \xmark                                                                                              & \xmark                                    & 69.51                              & 46.95          & 50.61          & 46.34          & 53.35          & 3.98 & 3.41 & 3.44 & 3.38 & 3.55 \\
    \cmark                                                                                              & \xmark                                    & 75.61                              & 50.61          & 50.00          & 50.00          & 56.55          & 4.01 & 3.44 & 3.39 & \textbf{3.49} & 3.58 \\
    \xmark                                                                                              & \cmark                                    & 83.54                     & \textbf{56.10}          & 51.22          & 50.61 & 60.37 & 4.05 & 3.51 & 3.51 & 3.42 & 3.62 \\
    \cmark                                                                                              & \cmark                                    & \textbf{85.37}                            & \textbf{56.10}                     & \textbf{51.83} & \textbf{52.43}          & \textbf{61.43} & \textbf{4.13} & \textbf{3.60} & \textbf{3.54} & \textbf{3.49} & \textbf{3.69} \\

    \rowcolor[rgb]{0.93,0.93,0.93}\multicolumn{12}{c}{\textbf{Initialize with Deepseek-Coder-6.7B-base~\citep{deepseekcoder}}} \\
    \xmark                                                                                              & \xmark                                    & 59.15                              & 35.98          & 39.02          & 37.80          & 42.99          & 3.71 & 3.05 & 3.16 & 3.05 & 3.24 \\
    \cmark                                                                                              & \xmark                                    & 66.46                              & 41.46          & 38.41          & 36.59          & 45.73          & 3.76 & 3.17 & \textbf{3.21} & 3.08 & 3.31 \\
    \xmark                                                                                              & \cmark                                    & 70.73                              & 39.63          & 39.02          & 40.24          & 47.41          & 3.90 & 3.17 & 3.08 & 3.11 & 3.31 \\
    \cmark                                                                                              & \cmark                                    & \textbf{79.88}                     & \textbf{45.73} & \textbf{43.90} & \textbf{42.68} & \textbf{53.05} & \textbf{3.96} & \textbf{3.21} & 3.18 & \textbf{3.19} & \textbf{3.38} \\
    \bottomrule
  \end{tabular}%
  }
  \caption{The ablation study of different methods across four optimization levels
  (O0, O1, O2, O3), as well as their average scores (AVG). The results in bold represent the optimal performance. The ~\labelemoji~ and ~\toolemoji~ means Relabedling and Function Call. \textbf{Bold} denotes the best performance.}
  \label{tab:ablation}
\end{table*}



\begin{figure*}[ht]
    \centering
    \begin{minipage}{0.65\textwidth}
        \centering
        \includegraphics[width=0.95\linewidth]{figs/ablation.pdf}
        \vspace{-2mm}
        \captionof{figure}{Qualitative Comparison for different design choices. Our method, employing multi-view epipolar attention, demonstrates the best consistency.}
        \label{fig:ablation}
    \end{minipage}\hfill
    \begin{minipage}{0.33\textwidth}
        \centering
        \includegraphics[width=0.8\linewidth]{figs/neus_ver.pdf}
        \vspace{-3mm}
        \caption{Our method shows better direct 3D reconstruction~\cite{neus}.}
        \label{fig:neus}
    \end{minipage}
    \vspace{-5mm}
\end{figure*}

\noindent\textbf{Multi-view Consistency.}
Tab.~\ref{tab:view16_fxied_compare} presents the 3D consistency scores compared to our baseline model (Zero123) and SyncDreamer. The results indicate a significant improvement across all three metrics achieved by our method when compared with Zero123.
While our method exhibits a marginally lower numerical consistency score compared to SyncDreamer, it enables the synthesis of images with arbitrary camera poses.	
This capability is illustrated in Tab.~\ref{tab:view16_free_compare}, where our method consistently enhances consistency with changes in camera pose settings, whereas SyncDreamer fails to do so and exhibits inferior results compared to Zero123.
Furthermore, our method facilitates the synthesis of multi-view images with any number of camera views. This versatility is demonstrated in Tab.~\ref{tab:view32_free_compare}, where our method continues to achieve significant improvements in consistency scores, while SyncDreamer is unable to operate under such conditions.	

Meanwhile, Fig.~\ref{fig:sota_compare} provides a qualitative comparison with the baseline. While both our method and SyncDreamer enhance consistency, our method visually preserves better similarity to the input image, including color and texture details. The input consistency score further corroborates this.

\noindent\textbf{Image Quality.}
While our primary goal centers around enhancing the consistency of synthesized multi-view images, we also evaluate the image quality by comparing the similarity with the ground truth images. The results shown in Tab.~\ref{tab:view16_free_compare}, Tab.~\ref{tab:view16_fxied_compare}, and Tab.~\ref{tab:view32_free_compare} indicate that our method also enhances the image quality under different settings besides improving the consistency.
Moreover, our method shows better image quality compared with SyncDreamer even in the 16-view setting with fixed camera pose.

\noindent\textbf{Input Consistency.}
Input consistency terms whether the results align with the input image.
Fig.~\ref{fig:sota_compare} illustrates that both our method and SyncDreamer enhance multi-view consistency. However, the color and texture details of SyncDreamer's results diverge from the input image and appear visually unnatural.
This discrepancy is evident in the input consistency score presented in Tab.~\ref{tab:view16_fxied_compare}, indicating lower similarity with the condition image in the SyncDreamer results.	

\subsection{Ablation Study}
The overall quantitative results are shown in Tab.~\ref{tab:ablation}, and the qualitative comparisons are shown in Fig.~\ref{fig:ablation}.

\noindent \textbf{Full Attention \vs Epipolar Attention.}
The results presented in Tab.\ref{tab:ablation} and Fig.\ref{fig:ablation} demonstrate that our epipolar attention mechanism can synthesize more consistent multi-view images compared with full attention. Furthermore, our epipolar attention achieves a greater performance improvement compared to full attention when using multiple reference images. This could be attributed to the fact that our epipolar attention more effectively localizes target information, as depicted in Fig.~\ref{fig:full_attn_compare}, thereby reducing noise from the reference images. In the multi-view setting, where multiple reference images are utilized, this noise reduction becomes particularly crucial.
Moreover, it is noteworthy that the epipolar attention mechanism consumes less GPU memory compared to our baseline, as discussed in Sec.~\ref{sec:attn_analysis}.

\noindent \textbf{Attending Single-View \vs Multi-View.}
Applying the epipolar attention significantly improves the consistency between the input and target views. However, the consistency between different views in the unobserved regions of the input view is not well preserved.
After implementing our epipolar attention in the multi-view setting, the consistency across the generated multi-view images is further improved. The last row in Tab.~\ref{tab:ablation} shows that after applying our multi-view epipolar attention, the consistency score is further improved compared with the single-view setting. Besides, the qualitative result in Fig.~\ref{fig:ablation} also shows better consistency among different target views.



\begin{table}[t]
\centering
\vspace{-1mm}
\caption{Comparison of 3D reconstruction results. Our method significantly improves the reconstruction quality.}
\vspace{-3mm}
\label{tab:neus}
\scalebox{0.7}{
\begin{tabular}{c cc}
\toprule
              &  Chamfer Dist.$\downarrow$  & Volume IoU$\uparrow$
\\ \midrule

            Zero123         & 0.017         & 0.819    \\
            SyncDreamer     & \best{0.013}         & \best{0.847}    \\
            Ours            & 0.014	& 0.842 \\

\bottomrule
\end{tabular}
}
\vspace{-5mm}
\end{table}


\vspace{-2mm}
\subsection{Downstream Application}
\vspace{-2mm}
To demonstrate the effectiveness of our method, we also applied it to the downstream 3D reconstruction task. Specifically, we trained the NeuS model~\cite{neus} directly using images synthesized by our method, Zero123, and SyncDreamer, respectively.
The quantitative results in Tab.~\ref{tab:neus} show that the consistent multi-view images synthesized by our method can significantly improve the 3D reconstruction quality.
Additionally, our method exhibits similar performance to SyncDreamer which requires time-consuming re-training.
The qualitative results in Fig.~\ref{fig:neus} show that it is challenging to train the NeuS model directly due to the lack of consistency in the images generated by Zero123. In contrast, our method generates more consistent multi-view images and, therefore, better reconstructs the geometry and texture details.
We show improvements on other downstream applications such as image-to-3D in the Supplementary Material.



\section{Conclusion}
% We introduce \datasetname, a new benchmark to evaluate LLMs' ability to simulate human behavior in dynamic contexts.  
% Comprising 1001 literary behavior chains with rich persona context, it addresses a key gap in LLM assessment.
% Our findings reveal substantial LLM deficiencies in simulating coherent behavior chains, especially long-range consistency.  We observed a significant gap between individual behavior recognition and chain-level performance, highlighting the challenge of error accumulation.
% Analysis revealed temporal biases, a preference for key behaviors, and complex interactions between history/context length and performance.  These insights underscore the complexities of behavior simulation and suggest future research directions, including improved training data and model architectures.  
% \datasetname offers a valuable resource for robust digital twin development.

We introduced \datasetname, a novel benchmark for evaluating LLMs' ability to simulate continuous human behavior.  Comprising 1,001 persona-based behavior chains extracted from literature, it addresses a critical gap in LLM assessment, particularly given the scarcity of real-world behavioral data.  Our evaluation framework, encompassing behavioral recognition and generation tasks, revealed significant challenges for ten state-of-the-art LLMs.
Further experiments analyzing the influence of key behaviors, history, and context characteristics provided valuable insights into the factors that contribute to or hinder successful persona-based behavior simulation.
\datasetname provides a valuable resource for developing robust digital twins.




\section*{Limitations}

% \datasetname relies on literary works (novels/biographies) for behavior extraction, which may introduce stylistic bias. Fictional narratives often emphasize dramatic or plot-driven actions, while biographies may omit mundane behaviors. This limits generalizability to real-world scenarios where subtle, routine behaviors dominate.  

\datasetname focuses on English-language Western literature, lacking representation of non-Western cultural norms and behavioral expressions. This cultural narrowness limits applicability to global digital twin deployments.
Generation task evaluation via GPT-4o introduces model bias, as the assessor’s cultural/personal biases may influence consistency judgments. Human evaluation, though more reliable, was limited to data validation rather than comprehensive scoring.


Our current dataset, while encompassing literary works from diverse cultural and national backgrounds, is presently limited to English-language materials. We recognize the critical importance of investigating digital twin applications across multilingual contexts. In subsequent phases of this research, we plan to expand the dataset to incorporate multiple languages, thereby enabling comprehensive evaluation of models' capabilities in simulating human behavior through multilingual interactions.
In the current study, we have focused exclusively on data construction and model evaluation while thoroughly examining the model's limitations across multiple dimensions. The aspect of enhancing the model's capability for continuous behavioral simulation remains underexplored, and we plan to address in subsequent research. 

% 虽然囊括了来自多个国家多种文化背景的书籍,但我们的数据集目前只有英文版本,我们认为不同语言背景下的digital twin同样是重要的。因为我们之后会在之后的工作中讲数据集扩充到更多语言,考察模型用多种语言simulate 人类行为的能力。
% 我们关注数据的构建和模型的评估,我们会在以后的工作中继续探索如何在模型如何提升在continous behavior simulating能力,(展望digital twin)




\section*{Ethics Statement}
This paper introduces \datasetname, a benchmark for evaluating LLMs' ability to simulate human behavior, raising several ethical considerations.  While the benchmark itself doesn't directly generate content that could be harmful, the use of LLMs for behavior simulation has the potential for misuse, such as creating deceptive or manipulative content. The dataset construction process, which involves extracting behaviors from literary works, was carefully designed to avoid misrepresenting characters or their actions.  All source materials are publicly available.  Furthermore, the evaluation framework focuses on assessing LLMs' simulation capabilities rather than generating real-world actions.  We acknowledge the potential for bias in the literary sources used and emphasize the importance of responsible use of this benchmark and its findings. 
The datasets used in our experiment are publicly released and labeled through interaction with humans in English. In this process, user privacy is protected, and no personal information is contained in the dataset. The scientific artifacts that we used are available for research with permissive licenses. And the use of these artifacts in this paper is consistent with their intended use. Therefore, we believe that our research work meets the ethics of ACL.



% \section{Engines}

% To produce a PDF file, pdf\LaTeX{} is strongly recommended (over original \LaTeX{} plus dvips+ps2pdf or dvipdf). Xe\LaTeX{} also produces PDF files, and is especially suitable for text in non-Latin scripts.

% \section{Preamble}

% The first line of the file must be
% \begin{quote}
% \begin{verbatim}
% \documentclass[11pt]{article}
% \end{verbatim}
% \end{quote}

% To load the style file in the review version:
% \begin{quote}
% \begin{verbatim}
% \usepackage[review]{acl}
% \end{verbatim}
% \end{quote}
% For the final version, omit the \verb|review| option:
% \begin{quote}
% \begin{verbatim}
% \usepackage{acl}
% \end{verbatim}
% \end{quote}

% To use Times Roman, put the following in the preamble:
% \begin{quote}
% \begin{verbatim}
% \usepackage{times}
% \end{verbatim}
% \end{quote}
% (Alternatives like txfonts or newtx are also acceptable.)

% Please see the \LaTeX{} source of this document for comments on other packages that may be useful.

% Set the title and author using \verb|\title| and \verb|\author|. Within the author list, format multiple authors using \verb|\and| and \verb|\And| and \verb|\AND|; please see the \LaTeX{} source for examples.

% By default, the box containing the title and author names is set to the minimum of 5 cm. If you need more space, include the following in the preamble:
% \begin{quote}
% \begin{verbatim}
% \setlength\titlebox{<dim>}
% \end{verbatim}
% \end{quote}
% where \verb|<dim>| is replaced with a length. Do not set this length smaller than 5 cm.

% \section{Document Body}

% \subsection{Footnotes}

% Footnotes are inserted with the \verb|\footnote| command.\footnote{This is a footnote.}

% \subsection{Tables and figures}

% See Table~\ref{tab:accents} for an example of a table and its caption.
% \textbf{Do not override the default caption sizes.}

% \begin{table}
%   \centering
%   \begin{tabular}{lc}
%     \hline
%     \textbf{Command} & \textbf{Output} \\
%     \hline
%     \verb|{\"a}|     & {\"a}           \\
%     \verb|{\^e}|     & {\^e}           \\
%     \verb|{\`i}|     & {\`i}           \\
%     \verb|{\.I}|     & {\.I}           \\
%     \verb|{\o}|      & {\o}            \\
%     \verb|{\'u}|     & {\'u}           \\
%     \verb|{\aa}|     & {\aa}           \\\hline
%   \end{tabular}
%   \begin{tabular}{lc}
%     \hline
%     \textbf{Command} & \textbf{Output} \\
%     \hline
%     \verb|{\c c}|    & {\c c}          \\
%     \verb|{\u g}|    & {\u g}          \\
%     \verb|{\l}|      & {\l}            \\
%     \verb|{\~n}|     & {\~n}           \\
%     \verb|{\H o}|    & {\H o}          \\
%     \verb|{\v r}|    & {\v r}          \\
%     \verb|{\ss}|     & {\ss}           \\
%     \hline
%   \end{tabular}
%   \caption{Example commands for accented characters, to be used in, \emph{e.g.}, Bib\TeX{} entries.}
%   \label{tab:accents}
% \end{table}

% As much as possible, fonts in figures should conform
% to the document fonts. See Figure~\ref{fig:experiments} for an example of a figure and its caption.

% Using the \verb|graphicx| package graphics files can be included within figure
% environment at an appropriate point within the text.
% The \verb|graphicx| package supports various optional arguments to control the
% appearance of the figure.
% You must include it explicitly in the \LaTeX{} preamble (after the
% \verb|\documentclass| declaration and before \verb|\begin{document}|) using
% \verb|\usepackage{graphicx}|.

% \begin{figure}[t]
%   \includegraphics[width=\columnwidth]{example-image-golden}
%   \caption{A figure with a caption that runs for more than one line.
%     Example image is usually available through the \texttt{mwe} package
%     without even mentioning it in the preamble.}
%   \label{fig:experiments}
% \end{figure}

% \begin{figure*}[t]
%   \includegraphics[width=0.48\linewidth]{example-image-a} \hfill
%   \includegraphics[width=0.48\linewidth]{example-image-b}
%   \caption {A minimal working example to demonstrate how to place
%     two images side-by-side.}
% \end{figure*}

% \subsection{Hyperlinks}

% Users of older versions of \LaTeX{} may encounter the following error during compilation:
% \begin{quote}
% \verb|\pdfendlink| ended up in different nesting level than \verb|\pdfstartlink|.
% \end{quote}
% This happens when pdf\LaTeX{} is used and a citation splits across a page boundary. The best way to fix this is to upgrade \LaTeX{} to 2018-12-01 or later.

% \subsection{Citations}

% \begin{table*}
%   \centering
%   \begin{tabular}{lll}
%     \hline
%     \textbf{Output}           & \textbf{natbib command} & \textbf{ACL only command} \\
%     \hline
%     \citep{Gusfield:97}       & \verb|\citep|           &                           \\
%     \citealp{Gusfield:97}     & \verb|\citealp|         &                           \\
%     \citet{Gusfield:97}       & \verb|\citet|           &                           \\
%     \citeyearpar{Gusfield:97} & \verb|\citeyearpar|     &                           \\
%     \citeposs{Gusfield:97}    &                         & \verb|\citeposs|          \\
%     \hline
%   \end{tabular}
%   \caption{\label{citation-guide}
%     Citation commands supported by the style file.
%     The style is based on the natbib package and supports all natbib citation commands.
%     It also supports commands defined in previous ACL style files for compatibility.
%   }
% \end{table*}

% Table~\ref{citation-guide} shows the syntax supported by the style files.
% We encourage you to use the natbib styles.
% You can use the command \verb|\citet| (cite in text) to get ``author (year)'' citations, like this citation to a paper by \citet{Gusfield:97}.
% You can use the command \verb|\citep| (cite in parentheses) to get ``(author, year)'' citations \citep{Gusfield:97}.
% You can use the command \verb|\citealp| (alternative cite without parentheses) to get ``author, year'' citations, which is useful for using citations within parentheses (e.g. \citealp{Gusfield:97}).

% A possessive citation can be made with the command \verb|\citeposs|.
% This is not a standard natbib command, so it is generally not compatible
% with other style files.

% \subsection{References}

% \nocite{Ando2005,andrew2007scalable,rasooli-tetrault-2015}

% The \LaTeX{} and Bib\TeX{} style files provided roughly follow the American Psychological Association format.
% If your own bib file is named \texttt{custom.bib}, then placing the following before any appendices in your \LaTeX{} file will generate the references section for you:
% \begin{quote}
% \begin{verbatim}
% \bibliography{custom}
% \end{verbatim}
% \end{quote}

% You can obtain the complete ACL Anthology as a Bib\TeX{} file from \url{https://aclweb.org/anthology/anthology.bib.gz}.
% To include both the Anthology and your own .bib file, use the following instead of the above.
% \begin{quote}
% \begin{verbatim}
% \bibliography{anthology,custom}
% \end{verbatim}
% \end{quote}

% Please see Section~\ref{sec:bibtex} for information on preparing Bib\TeX{} files.

% \subsection{Equations}

% An example equation is shown below:
% \begin{equation}
%   \label{eq:example}
%   A = \pi r^2
% \end{equation}

% Labels for equation numbers, sections, subsections, figures and tables
% are all defined with the \verb|\label{label}| command and cross references
% to them are made with the \verb|\ref{label}| command.

% This an example cross-reference to Equation~\ref{eq:example}.

% \subsection{Appendices}

% Use \verb|\appendix| before any appendix section to switch the section numbering over to letters. See Appendix~\ref{sec:appendix} for an example.

% \section{Bib\TeX{} Files}
% \label{sec:bibtex}

% Unicode cannot be used in Bib\TeX{} entries, and some ways of typing special characters can disrupt Bib\TeX's alphabetization. The recommended way of typing special characters is shown in Table~\ref{tab:accents}.

% Please ensure that Bib\TeX{} records contain DOIs or URLs when possible, and for all the ACL materials that you reference.
% Use the \verb|doi| field for DOIs and the \verb|url| field for URLs.
% If a Bib\TeX{} entry has a URL or DOI field, the paper title in the references section will appear as a hyperlink to the paper, using the hyperref \LaTeX{} package.

% \section*{Acknowledgments}

% This document has been adapted
% by Steven Bethard, Ryan Cotterell and Rui Yan
% from the instructions for earlier ACL and NAACL proceedings, including those for
% ACL 2019 by Douwe Kiela and Ivan Vuli\'{c},
% NAACL 2019 by Stephanie Lukin and Alla Roskovskaya,
% ACL 2018 by Shay Cohen, Kevin Gimpel, and Wei Lu,
% NAACL 2018 by Margaret Mitchell and Stephanie Lukin,
% Bib\TeX{} suggestions for (NA)ACL 2017/2018 from Jason Eisner,
% ACL 2017 by Dan Gildea and Min-Yen Kan,
% NAACL 2017 by Margaret Mitchell,
% ACL 2012 by Maggie Li and Michael White,
% ACL 2010 by Jing-Shin Chang and Philipp Koehn,
% ACL 2008 by Johanna D. Moore, Simone Teufel, James Allan, and Sadaoki Furui,
% ACL 2005 by Hwee Tou Ng and Kemal Oflazer,
% ACL 2002 by Eugene Charniak and Dekang Lin,
% and earlier ACL and EACL formats written by several people, including
% John Chen, Henry S. Thompson and Donald Walker.
% Additional elements were taken from the formatting instructions of the \emph{International Joint Conference on Artificial Intelligence} and the \emph{Conference on Computer Vision and Pattern Recognition}.

% % Bibliography entries for the entire Anthology, followed by custom entries
% %\bibliography{anthology,custom}
% % Custom bibliography entries only
\bibliography{custom}

\appendix

\section{Appendix}
% \label{sec:app}
\label{sec:app}



\subsection{Data Analysis}

\subsubsection{Data Statistics}

\begin{figure}[h]
    \centering
    \includegraphics[width=1\linewidth]{latex/picture/bing.png}
    \caption{The left figure illustrates the publication year distribution of literary works used as raw material from \datasetname. The right figure shows the proportion of Level 1 and Level 2 behaviors within the overall dataset.}
    \label{fig:bing}
\end{figure}



% 行为链数据集

% 总行为数: 15846
% 总人物数:1000
% 最大行为节点数:20
% 最小行为数:10
% 平均节点数:15-16
% 最大input token长度:54
% 最小input token长度:8
% 平均input token长度:18.48
% 语言:英语
% 来源:小说,传记


\begin{table}[h]
\centering
\begin{tabular}{lc}
\toprule
Statistic & Value \\
\midrule
Total Behaviors & 15846 \\
Total Persons & 1001 \\
Max Behavior Nodes & 20 \\
Min Behaviors & 10 \\
Average Nodes & 15-16 \\
Max Input Token Length & 54 \\
Min Input Token Length & 8 \\
Average Input Token Length & 18.48 \\
Language & English \\
Source & Novels, Biographies \\
\bottomrule
\end{tabular}
\caption{Descriptive Statistics of the Behavior Dataset}
\label{tab:behavior_stats} % Optional: Add a label for cross-referencing
\end{table}

\begin{table}[h]
\centering
\small
\begin{tabular}{lc}
\toprule
Statistic & Value \\
\midrule
Style Categories & 25 \\
Genre Categories & 65 \\
Setting Categories & 536 \\
Theme Categories & 115 \\
Lexile Level Categories & 112 \\
Recommended Reading Age Categories & 31 \\
Max Rating Number & 3,380,109 \\
Min Rating Number & 36 \\
Max Ratings Score & 4.7 \\
Min Ratings Score & 3.3 \\
\bottomrule
\end{tabular}
\caption{Book Diversity Statistics}
\label{tab:book_diversity} % Optional: Add a label for cross-referencing
\end{table}

We collect book-related metadata from SuperSummary and Goodreads, including various categorical tags such as Style, Genre, Setting, Theme, Lexile Level, and Recommended Reading Age from SuperSummary. 
Additionally, we record the number of ratings to the rating scores from Goodreads. 
As shown in Table \ref{tab:book_diversity}, the dataset demonstrates a diverse and comprehensive representation of books.

For specific tags, the most prevalent Styles include \textit{Emotional, Mysterious, and Contemplative}. 
Common Genres featured in the dataset are \textit{Historical Fiction, Modern Classic Fiction, Romance, and Mystery \& Crime Fiction}. 
Frequently occurring Settings include \textit{Contemporary, 2010s, and United States}. 
The dataset also highlights recurring Themes such as \textit{Family, Love, Friendship, and Race}. 
The most frequently occurring levels include 830L, 780L, HL690L, and 740L.
Additionally, the Recommended Reading Age ranges from \textit{7 to 18+}.
Together, these attributes demonstrate the richness and diversity of the dataset, ensuring a broad representation of literary elements across various themes, genres, reading levels, and target audiences.

\subsubsection{More Results Analysis}
\label{sec:app_result}
\begin{figure*}[h]
    \centering
    \includegraphics[width=1\linewidth]{latex/picture/style.pdf}
    \caption{Model performance across different book style.}
    \label{fig:style}
\end{figure*}

We select the top 20 Styles based on their frequency and calculated the average metrics for the book subsets under each Style. 
Our analysis shows that the model's ability to predict behavior chains does not exhibit significant differences across different Styles.
% However, relatively speaking, styles such as Informative and Gritty demonstrate slightly higher predictive performance, likely due to their more structured and explicit behavioral patterns. 
% In contrast, styles like Fantastical and Playful exhibit marginally lower performance, potentially due to greater variability and implicit character behaviors. 
% While these differences are not statistically significant, they suggest that narrative structure may influence the model’s ability on the behavior chain task.


\subsection{Implementation Details}

\subsubsection{Dataset Construction Details}
\label{sec:app_con}
We employed Claude-3-5-sonnet-20240620 for behavior chain extraction and context refinement. The ChatGPT-4o-latest model was utilized to generate distractor items for each behavior. For other generation tasks with lower computational requirements, we selected GPT-3.5 as the execution model.

\subsubsection{Model Testing Details}
\label{sec:app_exp}
All evaluated models were post-interaction versions. The history provided to the LLM consisted of summaries from all preceding chapters used for behavior chain extraction.

To prevent data leakage, we systematically substituted all personal entities in the inputs. For conciseness, all analyses in Section \ref{sec:analysis} are based on the multiple-choice task framework.

The single experimental data presented in \ref{sec:analysis} exclusively comprises outputs from the Llama-3p1-70B-Instract model as the target model.

For inputs that exceed the model's context length, we truncated the history from the oldest to the newest.












\subsection{Prompt}
\label{sec:app_prompt}

\begin{tcolorbox}[title = {First Behavior Generation Prompt}, breakable]

You are an expert in Narrative Analysis and Character Behavior Extraction.\\
Please extract the MOST KEY behavior of \{character\} FROM <Paragraphs>.\\
The behavior should have a significant impact on the development of the storyline, reflect character characteristics or emotions.\\

Ensure that the KEY behavior is an objective statement, clearly stated without any vague expressions.\\
DO NOT add subjective interpretations or inferences about the character's behaviors. ONLY describe the KEY behavior itself. DO NOT mention the result in the KEY behavior.\\
Use your own words instead of quoting the original text.\\
DO NOT repeat or imitate <Previous Key Behavior>.\\

The KEY behavior should have a significant impact on the development of the storyline, the characterization of the characters, and the expression of the theme.\\
Ensure that the key behavior is an objective statement, clearly stated without any vague expressions.\\
DO NOT add subjective interpretations or inferences about the character's behaviors. Only describe the behavior itself.\\

The format of your response should be: \{"key behavior": ""\}.\\

<Paragraphs BEGIN>\\
\{parts[0]\}\\
<Paragraphs END>\\

If the behavior of \{character\} cannot be extracted, output "" ONLY.


\end{tcolorbox}

% \vspace{1cm}


\begin{tcolorbox}[title = {Next Behavior Generation Prompt}, breakable]
You are an expert in Narrative Analysis and Character Behavior Extract.\\
Below, I will provide you with <Previous Paragraphs>, <Previous Key Behavior> extracted from <Previous Paragraphs> and <Current Paragraphs>. \\

Please summary the scene change and plot development detailly and naturally after the <Previous Key Behavior> according to the <Previous Paragraphs> and <Current Paragraphs> I give you. The summary should start with "\{examples[-1]['key behavior']\} After that, ".\\

After the summary, you should extract the MOST KEY behavior of \{character\} FROM <Current Paragraphs>, describe in more than 10 words.\\
The behavior should be a non-meaningless behavior taken spontaneously by \{character\}.\\
The behavior should have a significant impact on the development of the storyline or reflect character characteristics or emotions.\\
Ensure that the behavior is objective statements and state the behavior clearly and do not use any vague expressions. \\
DO NOT add subjective interpretations and inference about the character's behaviors. ONLY describe the behavior itself. \\
DO NOT mention the result in the behavior.\\
Ensure use your own words instead of quoting the original text. \\
DO NOT repeat or imitate <Previous Key Behavior>. \\

Please provide a REVISED summary of the scene change and plot development that occurred before the behavior you extracted from <Current Paragraphs>, making sure not to reveal any information about the behavior. Delete the behavior and subsequent plots, and keep only the plots before the behavior.\\
The REVISED summary should end with "After this or in response to this, what behavior did \{character\} take?"\\

<Previous Paragraphs BEGIN>\\
\{marge part[-1]\}\\
<Previous Paragraphs END>\\

<Previous Key Behavior BEGIN>\\
\{examples[-1]['key behavior']\}\\
<Previous Key Behavior END>\\

<Current Paragraphs BEGIN>\\
\{parts[i]\}\\
<Current Paragraphs END>\\

The format of your response should be \{"summary": "","key behavior":"","new summary":""\}.\\
Ensure that the behavior you extract is taken by \{character\} at this moment, rather than behavior of others or past behavior (in <Paragraphs>, \{character\}probable in the first person).\\
If the behavior of \{character\} cannot be extracted, output "None" ONLY.

\end{tcolorbox}





\begin{tcolorbox}[title = {Similirity Check}, breakable]

Please determine whether the following two behaviors refer to the same behavior:\\
Behavior 1: \{sentence1\}\\
Behavior 2: \{sentence2\}\\
If there is a strong possibility that the two behaviors refer to the same behavior, please output 1; otherwise, output 0, Make sure you give me 0/1.
\end{tcolorbox}




\begin{tcolorbox}[title = {Context Refine}, breakable]
<Context Begin>\\
\{relevant part\}\\
<Context End>\\

<Behavior Begin>\\
\{key behavior\}\\
<Behavior End>\\

Your task is to refine the <Context> according to the following requirements:\\

1. If the <Context> explicitly or implicitly suggests the active behaviors of \{character\} in the <Behavior> or discloses the result/reactions of others caused by <Behavior>, delete these from <Context>. Any references to \{character\}' emotions, feelings, psychological states, or internal conflicts should be eliminated from <Context>.\\

2. If < Behavior> include \{character\}'s reaction/response to event/situation/others behavior, then that event/situation/others behavior should to be described intactly and directly in the end of <Context>.\\

3. If the <Behavior> includes any elements such as contextual conditions or encounters rather than purely active behavior of \{character\}, integrate these elements into the <Context>. Pay attention to the clauses in <Behavior> as it often contain contextual information, but do not include it in <Context> if it happens after \{character\}'s behavior.\\

4. Output the refined <Context> directly without other note.

\end{tcolorbox}




\begin{tcolorbox}[title = {Distracter Generation}, breakable]
<Context Start>\\
\{examples[i]['summary refined']\}\\
<Context End>\\

The original subsequent behavior was "\{behavior\}".\\
Estimate what personality trait does this reflect in brief words, and generate 3 behaviors that different personality traits would exhibit, answered in JSON format.\\
\{\\
"original behavior traits":"",\\
"difference": [\\
\{\\
 "trait": "",\\
 "behavior": ""\\
 \},\\
\{\\
 "trait": "",\\
"behavior": ""\\
 \},\\
\{\\
 "trait": "",\\
 "behavior": ""\\
 \}\\
]\\
\}

\end{tcolorbox}




\begin{tcolorbox}[title = {Level Define}, breakable]
<Summary BEGIN>\\
\{summary l\}\\
<Summary END>\\
Please check whether The following behaviors are implied in the summary.\\
In your reply, retain the sequence numbers. The number of key behaviors must be considerably fewer than \{int(n/2)\}.\\

<behaviors BEGIN>\\
\{sentences\}\\
<behaviors END>

\end{tcolorbox}





\subsection{Data instance}
\label{sec:app_instance}
\begin{tcolorbox}[title = {Profile}, breakable]
\{
  "Name": "Jay Gatsby",\\
  "Personality Traits": [
    "Charismatic",
    "Mysterious",
    "Obsessive",
    "Romantic",
    "Wealthy",
    "Idealistic"
  ],\\
  "Motivations and Goals": [\\
    "To reunite with his former lover, Daisy Buchanan",\\
    "To achieve a high social status and wealth",\\
    "To recapture the past and fulfill his ideal vision of life with Daisy"\\
  ],\\
  "Significant Background Events": [\\
    "Born James Gatz to a poor farming family in North Dakota",\\
    "Changed his name to Jay Gatsby and reinvented himself as a wealthy socialite",\\
    "Amassed his fortune through questionable means",\\
    "Became known for his lavish parties at his mansion in West Egg"\\
  ],\\
  "Relationships": \{\\
    "Daisy Buchanan": "Former lover, whom Gatsby is still deeply in love with",\\
    "Tom Buchanan": "Daisy's husband and Gatsby's rival",\\
    "Nick Carraway": "Narrator of the story and Gatsby's neighbor and friend",\\
    "Jordan Baker": "A professional golfer and friend of Daisy, whom Gatsby has a brief romantic interest in",\\
    "George Wilson": "A mechanic and owner of a garage, indirectly involved in Gatsby's downfall"\\
  \},\\
  "Additional Details": \{\\
    "Occupation": "Businessman with mysterious sources of wealth",\\
    "Social Status": "Wealthy and influential, but not born into old money",\\
    "Hobbies": "Throwing extravagant parties, collecting expensive art and cars",\\
    "Residence": "A grand mansion in West Egg, New York"\\
  \}\\
\}\\
\end{tcolorbox}


\begin{tcolorbox}[title = {History}, breakable]
Summary of Chapter 1 \\
Nick, a Yale graduate and World War I veteran, moves to West Egg, Long Island, to work as a bond salesman.  He rents a small house next to the opulent mansion of his mysterious neighbor, Jay Gatsby.

West Egg is separated from the more fashionable East Egg by the Long Island Sound.  Across the water, Nick's cousin Daisy lives with her wealthy and imposing husband, Tom Buchanan, whom Nick knew at Yale.  Tom's wealth and social standing are evident in their "Georgian Colonial mansion."  Also present at their home during Nick's visit is Jordan Baker, a friend of Daisy's and a well-known golfer.

The chapter highlights the Buchanans' superficial and privileged lifestyle.  During dinner, Tom reveals his racist views by referencing a book espousing white supremacist theories, claiming the Nordic race is responsible for civilization.  This disturbs Nick, but no one challenges Tom's comments.

The dinner is interrupted by a phone call, hinting at Tom's infidelity.  Jordan informs Nick that Tom is having an affair with a woman in New York City.  The tension is palpable, but Daisy attempts to deflect attention.  The awkwardness of the situation is further amplified by another phone call.

As the evening progresses, Nick learns that Jordan Baker is a famous golfer with a somewhat scandalous reputation.  He also discovers that Daisy and he, despite being related, are not close.  The Buchanans tease Nick about a rumor of a broken engagement, which he denies.

Returning home, Nick sees Gatsby standing outside his mansion.  Nick considers inviting him over but hesitates, sensing something enigmatic about Gatsby's presence.
\\Summary of Chapter 2 \\
Chapter 2 delves into the "valley of ashes," a desolate area between West Egg and New York City, where industrial ashes are dumped.  Presiding over this wasteland is the faded billboard of Dr. T.J. Eckleburg, featuring giant, spectacled eyes.

This grim setting is significant because it's where George Wilson's struggling auto shop and his wife Myrtle, Tom's mistress, are located.  Tom, on his way to New York City with Nick, stops at Wilson's garage, ostensibly to discuss a car deal.  The real purpose, however, is to rendezvous with Myrtle. Nick finds Myrtle physically unattractive, despite her apparent allure for Tom.

The group, leaving George behind, proceeds to New York City, impulsively buying a puppy along the way.  They gather at a small apartment Tom keeps for his affair.  The other guests include Myrtle's sister, Catherine, and the McKees, neighbors from the building. Mr. McKee is a mediocre photographer.

The atmosphere becomes increasingly tawdry and tense.  A critical moment occurs when Myrtle mentions Daisy's name, provoking Tom to violently strike her, breaking her nose.

Nick, claiming to be drunk (only the second time in his life), leaves the apartment with Mr. McKee amidst the chaos of tending to Myrtle's injury.  He experiences a blackout and then finds himself at Mr. McKee's bedside, observing his amateurish photographs.  Nick manages to escape the apartment and waits at the train station for the 4:00 a.m. train home.
\\Summary of Chapter 3 \\
Chapter 3 depicts one of Gatsby's extravagant parties through Nick's eyes.  Nick receives a formal invitation, a stark contrast to the casual attendance of most guests, who often don't even know Gatsby.  At the party, Nick reconnects with Jordan Baker, who explains she enjoys large gatherings for the anonymity they offer.  A humorous anecdote highlights Gatsby's generosity: he replaced a dress torn at a previous party with an expensive new one.

Rumors about Gatsby's background and wealth circulate among the guests, ranging from accusations of being a German spy to claims of him being a war hero or even a murderer.  While searching for Gatsby, Nick encounters a drunken "owl-eyed" man in Gatsby's impressive library, who is astonished to discover the books are real.  Nick eventually meets Gatsby, but their conversation is cut short by a phone call.  Observing Gatsby later, Nick is struck by his charismatic smile and the contrast between his apparent sobriety and the revelry of his guests.  Gatsby then takes Jordan aside for a private conversation, leaving her visibly shaken.  As the party ends, a drunken car crash involving the owl-eyed man underscores the reckless atmosphere.

Nick then reflects on his experiences, noting that his time in West Egg isn't solely filled with social events but also with work.  He expresses his growing affection for New York City, despite feelings of loneliness, and his burgeoning relationship with Jordan.  He recounts a story about Jordan cheating in a golf tournament, a scandal that, though dismissed, hints at her dishonesty.  Nick rationalizes her behavior as a consequence of her desire for independence clashing with her need for protection.

The chapter concludes with Nick and Jordan becoming romantically involved.  Nick reveals he ended a relationship in the Midwest to pursue Jordan and emphasizes his own sense of honesty.
\\Summary of Chapter 4 \\
Chapter 4 of *The Great Gatsby* opens with a Sunday morning at Gatsby's mansion, where the usual rumors about his past continue to circulate. Nick briefly catalogs the diverse and affluent guests who frequent Gatsby's parties.

One day in late July, Gatsby takes Nick to lunch in his luxurious car.  During the drive, Gatsby seems overly concerned with Nick's opinion of him and his vehicle. He then directly addresses the rumors about his background, claiming to be the son of wealthy, deceased Midwestern parents.  However, he gives the contradictory answer of San Francisco when asked where in the Midwest.  He further elaborates, stating that he inherited his parents' fortune, lived lavishly in Europe, and then became a war hero in World War I, receiving numerous medals.

To support his claims, Gatsby shows Nick a photo of himself at Oxford and a medal from Montenegro.  These items, along with Gatsby's ability to evade a speeding ticket by showing a Christmas card from the police commissioner, convince Nick of the truth of Gatsby's stories.

They have lunch with Meyer Wolfsheim, a shady character who, according to Gatsby, fixed the World Series. Wolfsheim's bizarre cufflinks, made of human molars, further emphasize his connection to the criminal underworld.  Tom Buchanan briefly joins them, but Gatsby disappears abruptly.

Later, Jordan reveals to Nick that she met Gatsby in Louisville in 1917, when Daisy and Gatsby were together.  Jordan recounts how Daisy almost ran off to New York with Gatsby before he went to war, and how she was later deeply unhappy before her wedding to Tom.  Jordan then reveals the reason for her private conversation with Gatsby at the party: Gatsby wants to reunite with Daisy, and he needs Nick's help to arrange a meeting.
\\Summary of Chapter 5 \\
Chapter 5 details the long-awaited reunion between Gatsby and Daisy at Nick's cottage.  Before Daisy arrives, Gatsby's feigned disinterest crumbles as he obsessively prepares, even having Nick's lawn manicured.  Daisy's initial question about the purpose of the invitation and whether Nick harbors romantic feelings for her is met with a reference to *Castle Rackrent*.

The meeting begins awkwardly, and Nick tries to give them space. However, Gatsby, clearly anxious, repeatedly seeks Nick's advice, prompting Nick to reprimand him for his behavior.  Gatsby's reaction suggests he's offended by Nick's criticism.

They then move to Gatsby's mansion. Nick briefly describes the mansion's original owner, a wealthy brewer who, according to rumors, wanted the surrounding area to resemble a medieval village, an idea Nick finds inherently un-American.  The only person present in the vast house is Klipspringer, a freeloading "boarder."  Gatsby proudly displays his extensive and expensive collection of shirts, a display that moves Daisy to tears.  Nick notices a picture of a man on a yacht, whom Gatsby identifies as Dan Cody, his former mentor and close friend.

Gatsby points out the green light on Daisy's dock, visible from his house in clear weather.  After touring the mansion, they again encounter Klipspringer, whom Gatsby compels to play the piano.  Finally, Nick leaves Gatsby and Daisy alone, implying the possibility of their rekindling their past romance.
\\Summary of Chapter 6 \\
Chapter 6 begins with a reporter's visit to Gatsby, seeking a statement regarding vague, unsubstantiated rumors.  This prompts Nick to recount Gatsby's true origins, as told by Gatsby himself.  James Gatz, born to an unremarkable family in South Dakota, reinvented himself as Jay Gatsby after meeting Dan Cody on Lake Superior.  Feeling out of place in his hometown, Gatz left to pursue a grander future.  He encountered Cody's yacht and, through his resourcefulness, became Cody's personal assistant and was even named a potential heir. However, after Cody's death, his mistress, Ella Kaye, intervened, preventing Gatsby from inheriting anything.\\
Weeks after Gatsby and Daisy's reunion, Nick unexpectedly encounters Tom Buchanan at Gatsby's mansion. Tom and two acquaintances have stopped by during a horseback riding trip. Gatsby's overly eager hospitality towards Tom is noticeable. When Tom and his group leave for another gathering, Gatsby almost insists on joining them, despite their clear disinterest.

Later, Tom and Daisy attend one of Gatsby's parties. Nick observes a palpable tension, viewing the West Egg revelry through Daisy's East Egg perspective.  Daisy is clearly repulsed by the spectacle.  During the party, Tom fuels the rumors of Gatsby's bootlegging activities, disparaging him as "newly rich."  Daisy half-heartedly defends Gatsby, claiming the guests are uninvited and that Gatsby is too gracious to turn them away.  Tom openly flirts with other women, while Daisy, in a strange moment, invites Nick to kiss her, an invitation he declines.

After Tom and Daisy depart, Nick stays late. Gatsby confides in Nick about Daisy, revealing his desire for her to leave Tom and be with him. Gatsby then wistfully recalls the pivotal moment, five years prior, when he first kissed Daisy, a memory he cherishes as a defining moment in his life.

\end{tcolorbox}

\begin{tcolorbox}[title = {Behavior Chain}, breakable]
<context 1>\\
Gatsby dismissed every servant in his house and replaced them with new ones who did not interact with the local community. After that, rumors spread in the village about the new servants, and Gatsby explained to Nick that he hired people recommended by Wolfsheim who wouldn't gossip. The next day was extremely hot, and Nick traveled to the Buchanans' house by train. After this or in response to this, what behavior did Jay Gatsby take?
\\<key\_behavior 1>\\
Gatsby called Nick on the phone to invite him to lunch at Daisy's house the next day.
\\<distracters 1>\\
Gatsby shows up unannounced at the Buchanans' house, insisting on seeing Daisy immediately despite the potential awkwardness.\\
Gatsby avoids contact with Nick or Daisy altogether, choosing instead to observe the situation from a distance without direct involvement.\\
Gatsby openly declares his feelings for Daisy in public, ignoring the potential consequences and drawing attention to himself.\\
\\<context 2>\\
Gatsby called Nick on the phone to invite him to lunch at Daisy's house the next day. After that, Nick arrived at the Buchanans' house on a sweltering hot day. The butler answered the phone, and Nick and Gatsby were directed to the salon where Daisy and Jordan were resting on a couch. Tom's voice could be heard on the telephone in the hall, apparently having a heated conversation about selling a car. Tom then entered the room and greeted Gatsby with concealed dislike. The room had a crimson carpet. After this or in response to this, what behavior did Jay Gatsby take?
\\<key\_behavior 2>\\
Gatsby stood in the center of the crimson carpet and gazed around with fascinated eyes.
\\<distracters 2>\\
Gatsby strode toward Tom with a firm handshake and a polite but assertive comment about the weather.\\
Gatsby fidgeted with the edge of his jacket while avoiding making direct eye contact with anyone in the room.\\
Gatsby crossed his arms, gave Tom a steely glare, and responded curtly to any pleasantries.\\
\\<context 3>\\
Gatsby stood in the center of the crimson carpet and gazed around with fascinated eyes. After that, the group moved to the veranda upon Tom's suggestion. They observed the green Sound where a small sail was moving slowly in the heat. After this or in response to this, what behavior did Jay Gatsby take?
\\<key\_behavior 3>\\
Gatsby raised his hand and pointed across the bay to show the location of his house
\\<distracters 3>\\
Gatsby remained silent, simply nodding in acknowledgment of the view without drawing attention to his house.\\
Gatsby spoke poetically about the beauty of the green Sound and the sailboat, avoiding mention of his house entirely.\\
Gatsby excused himself quietly to stand apart, gazing at the water in introspection.\\
\\<context 4>\\
Gatsby raised his hand and pointed across the bay to show the location of his house. After that, the group moved to have lunch in the darkened dining room where they drank cold ale. The atmosphere became tense as Daisy expressed her distress about the heat and confusion, suggesting they all go to town. Tom and Gatsby engaged in small talk about garages and stables, while Daisy persistently pushed for the idea of going to town. When Daisy asked again who wanted to go to town After this or in response to this, what behavior did Jay Gatsby take?
\\<key\_behavior 4>\\
Gatsby's eyes floated toward Daisy when she asked about going to town, and he stared at her, maintaining eye contact
\\<distracters 4>\\
Gatsby casually shrugged and glanced out the window, showing disinterest in Daisy's suggestion.\\
Gatsby immediately stood up and said, 'Let’s go right now!' without waiting for the group’s consensus.\\
Gatsby hesitated, thoughtfully considering whether going to town was practical given the heat and tension in the room.\\
\\<context 5>\\
Gatsby's eyes floated toward Daisy when she asked about going to town, and he stared at her, maintaining eye contact. After that, no one immediately moved to leave. Tom became increasingly agitated at the group's hesitation, and his hand trembled as he finished his drink. Daisy tried to delay their departure by suggesting they smoke first, but Tom dismissed her suggestion. The women went upstairs to prepare while the three men waited outside in the heat. A tense atmosphere developed as they stood shuffling their feet on the hot gravel, with the silver moon already visible in the western sky. Tom stood there with them. After this or in response to this, what behavior did Jay Gatsby take?
\\<key\_behavior 5>\\
Gatsby attempted to speak to Tom but stopped himself, then asked about Tom's stables with visible effort
\\<distracters 5>\\
Gatsby confronted Tom directly, making a pointed comment about Daisy's choices or feelings, his voice unwavering.\\
Gatsby abruptly suggested they leave for town without consulting anyone, his restlessness showing in his tone and actions.\\
Gatsby smirked and made a sly, cutting remark about Tom's stables, using it as a veiled critique of Tom's character.\\
\\<context 6>\\
Gatsby attempted to speak to Tom but stopped himself, then asked about Tom's stables with visible effort. After that, the group prepared to leave for town. Daisy and Jordan went upstairs to get ready while the men waited outside. Tom expressed his displeasure about going to town, and went inside to get whiskey. The narrator made a comment about Daisy's voice, and Gatsby described it as being 'full of money.' Tom returned with a bottle of whiskey, followed by Daisy and Jordan. Tom suggested they take his car. After this or in response to this, what behavior did Jay Gatsby take?
\\<key\_behavior 6>\\
Gatsby objected to Tom's suggestion of driving his car, claiming there wasn't much gas.
\\<distracters 6>\\
Gatsby nodded and agreed that Tom's car would be a good choice, showing a willingness to go along with the plan without objection.\\
Gatsby hesitated, closely examining Tom's car, and suggested they double-check everything to ensure the ride would be safe.\\
Gatsby grinned and suggested they all race both cars to town, adding some excitement to the mundane trip.\\
\\<context 7>\\
Gatsby objected to Tom's suggestion of driving his car, claiming there wasn't much gas. After that, Tom insisted there was plenty of gas and suggested stopping at a drugstore if needed. The group then split up, with Tom driving Gatsby's car with Nick and Jordan, while Gatsby and Daisy followed in Tom's coupé. As they drove, Tom expressed suspicion about Gatsby and mentioned conducting a small investigation into his past. They stopped at Wilson's garage for gas, where Wilson revealed his plans to move west with his wife. The group then continued to the city, eventually ending up in a suite at the Plaza Hotel. The conversation became tense as Tom questioned Gatsby's use of the phrase 'old sport'. In the suite, there was a fallen telephone book with its string parted. After this or in response to this, what behavior did Jay Gatsby take?
\\<key\_behavior 7>\\
Gatsby examined the parted string of the fallen telephone book, muttered 'Hum!' in an interested way, and tossed the book on a chair.
\\<distracters 7>\\
Gatsby ignored the fallen telephone book completely, focusing instead on more immediate concerns in the conversation.\\
Gatsby picked up the fallen telephone book, carefully retied the string, and placed it neatly back where it belonged.\\
Gatsby glanced briefly at the telephone book, shrugged, and dismissed it as unimportant without further reaction.\\
\\<context 8>\\
Gatsby examined the parted string of the fallen telephone book, muttered 'Hum!' in an interested way, and tossed the book on a chair. After that, Tom sharply criticized Gatsby's habit of saying 'old sport,' leading to tension in the room. Daisy attempted to diffuse the situation by requesting ice for mint juleps. The group then heard a wedding march from downstairs, which prompted a discussion about Daisy's wedding and a mysterious guest named Biloxi. As the conversation about Biloxi's conflicting stories continued, Tom suddenly turned to Gatsby with a pointed question about his education. After this or in response to this, what behavior did Jay Gatsby take?
\\<key\_behavior 8>\\
When questioned about his Oxford education, Gatsby first responded evasively by saying 'Not exactly,' then changed his answer to confirm he went there, and finally specified he attended Oxford for five months in 1919.
\\<distracters 8>\\
Gatsby would directly and calmly state he attended Oxford as part of a special program, elaborating on the details without hesitation.\\
Gatsby would respond assertively, challenging Tom by pointing out the validity of his education and questioning why it matters.\\
Gatsby would shrug and dismiss the question, saying something like, 'I don't see why it matters,' and redirecting the conversation elsewhere.\\
\\<context 9>\\
When questioned about his Oxford education, Gatsby first responded evasively by saying 'Not exactly,' then changed his answer to confirm he went there, and finally specified he attended Oxford for five months in 1919. After that, Gatsby explained that his time at Oxford was part of an opportunity given to some officers after the Armistice, allowing them to attend universities in England or France. This explanation seemed to renew the narrator's faith in Gatsby. Meanwhile, Daisy attempted to lighten the mood by suggesting they make mint juleps, but Tom insisted on asking Gatsby one more question, demanding to know what kind of 'row' Gatsby was trying to cause in his house. After this or in response to this, what behavior did Jay Gatsby take?
\\<key\_behavior 9>\\
Gatsby politely invited Tom to continue with his questioning.
\\<distracters 9>\\
Gatsby openly challenges Tom, questioning Tom's motivations and accusing him of hypocrisy in front of everyone.\\
Gatsby avoids eye contact, becomes visibly agitated, and deflects the question by bringing up a completely unrelated topic.\\
Gatsby abruptly raises his voice, passionately declaring his love for Daisy and confronting Tom about his treatment of her.\\
\\<context 10>\\
Gatsby politely invited Tom to continue with his questioning. After that, The tension in the room escalated as Tom accused Gatsby of causing trouble and challenged the modern attitudes towards family life. Daisy attempted to diffuse the situation, but Tom's anger and prejudice became more apparent. Jordan and Nick reacted with discomfort to Tom's outburst. As the argument intensified, Daisy interrupted, pleading for everyone to leave. Nick agreed and suggested they all go home, but Tom insisted on hearing what Gatsby had to say. After this or in response to this, what behavior did Jay Gatsby take?
\\<key\_behavior 10>\\
Gatsby directly told Tom that his wife doesn't love him.
\\<distracters 10>\\
Gatsby calmly acknowledged the tension and suggested a private conversation with Tom to resolve the issue.\\
Gatsby remained silent and avoided escalating the argument further, allowing others to take the lead.\\
Gatsby expressed understanding of Tom's concerns and reassured him that he had no intention to disrupt the family.\\
\\<context 11>\\
Gatsby directly told Tom that his wife doesn't love him. After that, Gatsby revealed that he and Daisy had been in love for five years without Tom's knowledge. Tom became furious and denied this claim, insisting that Daisy loved him when they married and still loves him now. Tom admitted his own infidelities but claimed he always came back to Daisy. Daisy expressed her disgust at Tom and confronted him about their departure from Chicago. After this or in response to this, what behavior did Jay Gatsby take?
\\<key\_behavior 11>\\
Gatsby walked over and stood beside Daisy to show his support when she confronted Tom about their departure from Chicago
\\<distracters 11>\\
Gatsby stayed silent and remained seated, avoiding direct involvement or confrontation.\\
Gatsby aggressively confronted Tom, escalating the conflict and directly challenging his assertions about Daisy.\\
Gatsby distanced himself from the interaction entirely, stepping away from both Daisy and Tom as they argued.\\
\\<context 12>\\
Gatsby walked over and stood beside Daisy to show his support when she confronted Tom about their departure from Chicago. After that, Gatsby urged Daisy to tell Tom the truth about never loving him, insisting that it would erase their past. Daisy hesitated and looked to Jordan and the narrator for support before reluctantly stating she never loved Tom. Tom questioned her about specific moments in their past, but Daisy refused to engage further. She then turned to Gatsby, acknowledging his presence but expressing frustration at his expectations. After this or in response to this, what behavior did Jay Gatsby take?
\\<key\_behavior 12>\\
Gatsby's eyes opened and closed in response to Daisy's confession of loving both him and Tom.
\\<distracters 12>\\
Gatsby smiles softly at Daisy, nodding with quiet acceptance, and assures her that her feelings won't change his regard for her.\\
Gatsby abruptly interrupts, dismissing Daisy's confusion and insisting they leave together immediately regardless of what she just said.\\
Gatsby steps back, averting his gaze, and calmly acknowledges her feelings, signaling that he would respect whatever decision she makes.\\
\\<context 13>\\
Gatsby's eyes opened and closed in response to Daisy's confession of loving both him and Tom. After that, Tom harshly claimed that Daisy never truly knew Gatsby was alive and mentioned his intimate history with Daisy. After this or in response to this, what behavior did Jay Gatsby take?
\\<key\_behavior 13>\\
Gatsby insisted on speaking to Daisy alone and declared to Tom that he would no longer take care of Daisy.
\\<distracters 13>\\
Gatsby emotionally withdrew, avoiding confrontation altogether, and let Tom and Daisy decide their fate without intervention.\\
Gatsby accused Tom of manipulating Daisy and escalated the verbal conflict, refusing to back down and openly challenging Tom in front of everyone.\\
Gatsby tried to comfort Daisy and suggested that she take her time to figure out her feelings, prioritizing her emotional well-being over himself or Tom.\\
\\<context 14>\\
Gatsby insisted on speaking to Daisy alone and declared to Tom that he would no longer take care of Daisy. After that, Tom accused Gatsby of being a common swindler and revealed that he had conducted an investigation into Gatsby's affairs. Tom exposed Gatsby's involvement in illegal activities, including running drug stores that sold grain alcohol. He also mentioned Gatsby's association with Meyer Wolfsheim and accused him of being a bootlegger. After this or in response to this, what behavior did Jay Gatsby take?
\\<key\_behavior 14>\\
Gatsby calmly responded to Tom's accusations about his illegal activities and association with Meyer Wolfsheim.
\\<distracters 14>\\
Gatsby angrily shouted at Tom, denying the accusations and accusing Tom of trying to smear his name.\\
Gatsby avoided eye contact and quietly admitted to parts of the accusations, retreating from any further confrontation.\\
Gatsby immediately launched into a detailed explanation and justification of his activities, attempting to discredit Tom's claims.\\
\\<context 15>\\
Gatsby calmly responded to Tom's accusations about his illegal activities and association with Meyer Wolfsheim. After that, Tom continued his accusations by bringing up Walter Chase and the betting laws, mentioning how Wolfsheim had intimidated Walter into silence. After this or in response to this, what behavior did Jay Gatsby take?
\\<key\_behavior 15>\\
Gatsby displayed an unfamiliar yet recognizable expression on his face when confronted about his more serious criminal activities
\\<distracters 15>\\
Gatsby smirked confidently and dismissed the accusations as baseless lies, daring Tom to present any real evidence.\\
Gatsby maintained a serene expression, calmly stating that his private affairs were none of Tom's concern.\\
Gatsby suddenly raised his voice, vehemently denying the accusations and accusing Tom of hypocrisy and worse behavior.\\
\end{tcolorbox}







\end{document}
