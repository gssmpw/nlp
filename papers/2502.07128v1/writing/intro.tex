\section{Introduction}

Computer game development is a creative endeavor that requires significant human effort. To address this, extensive research has been conducted to automate or assist these processes through computational methods~\cite{Gallotta_2024}. Recent advances in Large Language Models (LLMs) have further expanded the possibilities in this domain, enhancing tasks such as game mechanic design~\cite{charity2023conceptualgame}, programming~\cite{wu2024instructiondriven,chatdev}, and game AI development~\cite{yao_react_2023}. More importantly, the robustness of LLMs in processing diverse text inputs shows the potential to integrate these subtasks into a comprehensive game development pipeline, especially for card games that can be represented purely in text. Such integration could streamline workflows and make game development more accessible.
However, LLMs face the following significant challenges~\cite{Gallotta_2024}.

(1) \textbf{Novel design in game mechanic}: 
Game mechanics, defined as the ``actions, behaviors and control mechanisms afforded to the player''~\cite{mda}, are essential game design topics where novelty are often encouraged. 
Prior work used LLMs to compose game mechanics~\cite{charity2023conceptualgame}, which showcased the potential for novel designs through the controllable randomness of LLMs. However, it remains unclear how to encourage a more intentional novelty such that LLM outputs are neither too similar to existing games nor self-repeating.


(2) \textbf{Consistent game generation}: 
Translating game mechanics into consistent game code is critical for prototyping. To improve the consistency, prior work used LLM-generated I/O examples~\cite{liu2023is} to validate the generated code for short functions. This approach is not suitable for long programs with multiple rounds of interactions (such as card games), where the difficulty of generating correct I/O examples increases with the number of interactions.

(3) \textbf{Scalable evaluation with game AI}: 
Evaluating and refining game mechanic designs requires gameplay, which to be automated requires the support of AI agents with sufficient intelligence to explore game dynamics~\cite{Isaksen2015ExploringGS}. However, existing methods, ranging from reinforcement learning to LLM-based agents, come with significant drawbacks, including high hardware requirements~\cite{ma_eureka_2023}, long training times~\cite{shinn_reflexion_2023}, costly inference~\cite{wang2023avalon}, and specialized training needs~\cite{light2024strategist}. These methods are costly and impractical for rapid prototyping of diverse games.


To address these challenges, we propose \ourwork, an LLM-based pipeline designed to assist human designers in card game prototyping. Our work integrates game mechanic design, code generation, and gameplay AI creation into a cohesive framework. The key contributions are as follows:


(1) We propose an indexing method to represent games as game mechanic graphs. By clustering, summarizing, and designing new game mechanics as graph nodes, this approach enables new game mechanics design with a global understanding of existing databases. This enables the generation of novel games distinct from any existing designs.

(2) We employ an LLM-based agent system that generates game code and iteratively reflects on its consistency by self-generated gameplay records. We transfer the reflection techniques on self-reported action trajectories, which are used in robotic task planning~\cite{innermonologue}, into game code generation. It improves the consistency between code and game mechanics, minimizing human labor in game prototyping.

(3) We introduce a scalable method for generating gameplay AI. Using a pool of action-value code functions produced by our framework, stepwise inclusion determines which functions to include based on win rates during self-play. Without calling LLMs in each game decision, it still achieves similar performance as LLM agents in prior work~\cite{yao_react_2023,shinn_reflexion_2023}. Therefore, it can conduct large-scale evaluations on new games generated by our pipeline, completing a cycle that enhances both creativity and efficiency in card game prototyping.
