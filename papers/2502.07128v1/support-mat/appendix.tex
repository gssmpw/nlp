

\section{Code Generation Details}


\subsection{Game engine design}
\label{sup:game-engine}

Instead of generating entire game code from scratch \cite{chatdev}, we designed a card game engine in Python, converting the original generation task into completion tasks for several predefined functions.
To enable maximum compatibility for various card games and easy integration with gameplay AI, we adopt a framework structure where (1) game state updates and gameplay AI are decoupled \cite{zha_douzero_2021}; and (2) dependencies among game logics are minimized by a functional programming design \cite{wu2024instructiondriven}.
The logic of card games are abstracted into 6 functions (initiation, initialize deck, initial dealing, proceed round, get legal actions, and get payoffs) running in a predefined procedure (see Figure \ref{fig:game-engine}). Each function, which modifies the game state in dictionary form, is what our pipeline will generate.

\begin{figure}[ht]
  \centering
  \includegraphics[width=.6\linewidth]{figs/game_engine.pdf}
  \caption{\textbf{Game Engine Framework Design}. The game engine, composed of 6 core functions (in light blue) in a fixed flow, interacts with external gameplay agent (in light green) through game observation and game action in dictionary forms.}\label{fig:game-engine}
\end{figure}


The gameplay AI is integrated into this system by receiving game observations and outputting game actions in dictionaries. While our gameplay AI directly handles dictionaries, for all LLM-based agents we compared against, the game state is converted to the following format. The example below shows the observation information in one game turn. LLM agents may receive concatenated information from the past several turns.
\begin{tcolorbox}[
breakable,
title=Game Observation Example: Boat House Rum,  
colframe=gray, 
colback=white,
]
\begin{lstlisting}[]
# Gameplay since your last decision
Dealing 7 cards to each player
Player 0 decides to: draw-(source: discard_pile)
Player 0 draws cards from discard pile.
Player 0 decides to: discard-(card_idx: 6)
Player 0 discards A-diamonds.
Player 1 decides to: draw-(source: discard_pile)
Player 1 draws cards from discard pile.
Player 1 decides to: discard-(card_idx: 3)
Player 1 discards 9-diamonds.

# Common Information
num_players: 3
current_player: 2
has_drawn_cards_this_turn: False
discard_pile: [{'rank': '9', 'suit': 'diamonds'}]
stock_size: 30

# Player Information

## Player 0
score: 0
melds: []
hand_size: 7
recent_discard_draw_size: 0

## Player 1
score: 0
melds: []
hand_size: 7
recent_discard_draw_size: 0

## Player 2 (Self)
score: 0
current_player: True
melds: []
hand: [{'rank': '3', 'suit': 'diamonds'}, {'rank': 'Q', 'suit': 'spades'}, {'rank': 'Q', 'suit': 'hearts'}, {'rank': '3', 'suit': 'spades'}, {'rank': '4', 'suit': 'hearts'}, {'rank': '4', 'suit': 'spades'}, {'rank': '2', 'suit': 'diamonds'}]
recent_discard_draw: []

# Legal Actions
0: draw-(source: stock)
1: draw-(source: discard_pile)
\end{lstlisting}
\end{tcolorbox}

\subsection{Intermediate Results of LLM-based Validation}

\begin{figure*}[ht]
  \centering
  \includegraphics[width=\linewidth]{figs/agentflow.png}
  \caption{LLM Agent workflow of game code generation.}\label{fig:agentflow}
  % \Description{A woman and a girl in white dresses sit in an open car.}
\end{figure*}

In the code validation process, our method compares the generated gameplay record with the game description, identifying whether there exists an inconsistency. The gameplay record follows the same format as the game observation above. But multiple consecutive observations from different players are concatenated together. Here is an example output of the validation process.

\begin{tcolorbox}[
breakable,
title=Consistency validation results,  
colframe=gray, 
colback=white,
]
\begin{lstlisting}[language=markdown]
***Step by step evaluation***

1. **Turn 0 (Player 0):**
   - Player 0 has cards [K, 10, 10, K, K] and chooses to request from Player 1 the rank 10. This is legal because Player 0 has cards of rank 10.
   - Player 1 has 1 card; no information if it was 10. Assuming it wasn't, Player 0 should "Go Fish," but the stock is empty. The turn ends.

2. **Turn 1 (Player 1):**
   - Player 1 has only card 8 and requests rank 8 from Player 0.
   - Since Player 0 does not have this card (as per observed hand [K, 10, 10, K, K]), the request fails, but again no stock to draw from. The turn ends.

3. **Turn 2 (Player 2):**
   - Player 2 has cards [10, 10] and requests rank 10 from Player 0. This action is valid since they have rank 10 cards.
   - Player 0 has two 10s to pass over-indicating a successful request. Player 2 should take another turn but this is not visible.

4. **Turn 3 (Player 3) and Subsequent Observations:**
   - Player 3 requests rank 8 from Player 1. Since Player 1 does not have the rank (hand size is 1, card 8 not explicitly confirmed), the turn would typically result in "Go Fish."
   - The same cycle continues; player's actions become clearer, but crucial observations like drawn or obtained cards are missing, suggesting errors in turn execution.

5. **Game End:**
   - Legal actions remaining for the player with hand [K, K, K] show remaining possibilities accurately. No correct handling of potential book formation or reattempts captured.

***Analysis Summary***
Summary:
```text
The progression of the turns does not effectively simulate correct sequence dictated by rules, notably in request fulfillment and subsequent actions not following successful card claims due to incomplete turn handling. Data missing for card exchanges leads to incongruity with established gameplay mechanics.
```
Quote (optional):
```markdown
"Each turn consists of: Requesting cards from an opponent; Drawing from the stock if the request results in 'Go Fish'; Optionally revealing a book if completed; Passing the turn after drawing or failing to make a catch. Re-attempt: If a player successfully gets cards, they can continue asking for the same or different rank."
```
\end{lstlisting}
\end{tcolorbox}

Based on the validation results, code edits are proposed by LLM. Same as \cite{agentless}, partial replacement edits are returned to save LLM token usage.

\begin{tcolorbox}[
breakable,
title=Code Edit Proposal Example,  
colframe=gray, 
colback=white,
]
\begin{lstlisting}[]
<<<<<<< SEARCH
def proceed_round(action: dict, game_state: Dict, logger) -> DotDict:
    """Process the action and update the game state."""
    current_player = game_state.common.current_player
    logger.info(f"Player {current_player} takes an action: {action}")
    
    if action['action'] == 'request':
        process_request(action, game_state, logger)
    elif action['action'] == 'draw':
        if game_state.common.facedown_cards.stock:
            card = game_state.common.facedown_cards.stock.pop()
            game_state.players[current_player].facedown_cards.hand.append(card)
            logger.info(f"Player {current_player} draws a card.")
        else:
            logger.info("Stock pile is empty, no card to draw.")

    check_for_books(game_state.players[current_player], game_state, logger)
    
    # Check if the game is over
    if is_game_over(game_state, logger):
        return game_state
    
    game_state.common.current_player = next_player(game_state, logger)
    return game_state
=======
def proceed_round(action: dict, game_state: Dict, logger) -> DotDict:
    """Process the action and update the game state."""
    current_player = game_state.common.current_player
    logger.info(f"Player {current_player} takes an action: {action}")
    
    if action['action'] == 'request':
        if process_request(action, game_state, logger):
            return game_state  # Successful request allows player to continue.
    elif action['action'] == 'draw':
        if game_state.common.facedown_cards.stock:
            card = game_state.common.facedown_cards.stock.pop()
            game_state.players[current_player].facedown_cards.hand.append(card)
            logger.info(f"Player {current_player} draws a card.")
        else:
            logger.info("Stock pile is empty, no card to draw.")

    check_for_books(game_state.players[current_player], game_state, logger)
    
    # Check if the game is over
    if is_game_over(game_state, logger):
        return game_state
    
    game_state.common.current_player = next_player(game_state, logger)
    return game_state
>>>>>>> REPLACE
\end{lstlisting}
\end{tcolorbox}

\subsection{Evaluation on Code Generation}
\label{sup:test-case}
We test our pipeline using two sets of game descriptions: (1) we choose $29$ from the 106 games that we collected. (2) Using the novel game design method in \ref{sec:game-logic}, we mutate each game with $33$ variants and select the one that has the smallest closest cosine similarity to the game database. One of the games cannot be mutated by our method, since none of its extracted game mechanics belongs to a large-enough cluster, where new mechanics are designed. Therefore, we add $28$ mutated games as our second part of testing set.

For the code draft phase and debugging phase, we construct an example database containing $5$ game descriptions and their game code (``crazy-eights", ``california-jack", ``boat-house-rum", ``bull-poker", ``baccarat"). These games are selected from the $106$ games without overlaps to our test set.

It is important to note that LLM-based validation process improves the code alignment, rather than ensuring 100\% correct code. We conducted a manual verification to all generated games where LLM detects no misalignment, where we still find minor inconsistencies. For instance, there are simplified implementation of complex rules: for poker games with a large number of card evaluation standards (such as Straight Flush, Four of a kind, Full House, etc.), there are cases where our method does not implement all the standards mentioned in the game description.
Also, incorrect information visibilities may occur: sometimes, private information for specific players may be leaked to the public by game state outputs.
However, the validation process still minimizes human effort in code development, which benefits the overall prototyping process.


\section{Gameplay AI Details}




\subsection{Evaluation on Gameplay AI}
\label{sup:game-ai-eval}

Although some previous work demonstrates satisfactory performance in several popular games, they are excluded from our comparative analysis as their primary contributions focus on specific game genres \cite{wang2023avalon} or particular attributes of game mechanics \cite{light2024strategist}. For example, methods that learn directly from reward signals at MCTS leaf nodes \cite{light2024strategist} can hardly work on games with extremely unbalanced search trees. Because it is challenging to reach all terminal states with reasonable computational resources. As a result, one has to estimate the reward without reaching the terminal nodes, which introduces more uncertainties to the reward signals. 


Consequently, we focus on prior work that are scalable. We choose Chain-of-Thought~\cite{CoT}, ReAct~\cite{yao_react_2023}, Reflexion~\cite{shinn_reflexion_2023}, and AgentPro~\cite{zhang2024agentpro} as our baselines. As AgentPro is particularly noteworthy for its application of Belief-aware Decision-Making imperfect information game scenarios, we specifically implement the Belief-aware part for comparison.

The evaluation protocol involved training Reflexion agents across all games, with performance measured using a rolling average winning rate calculated over windows of 40 games. This measurement was repeated for 10 distinct windows across all games in our test set. The optimal reflection step was subsequently determined by selecting the iteration that yielded the highest mean winning rate across the entire game suite.




\subsection{Ablation Study}

We demonstrate the effect of the two major components in our gameplay AI method by comparing our method against two ablations: (1) No ensemble: rather than creating an ensemble of action-value functions, we generate only one comprehensive policy with its corresponding function. (2) No optimization: after the first augmentation stage, we skip the rest of the pipeline and use all policy components to create the ensemble. 

The result (see Table~\ref{tab:gameai-adv-ablation}) shows both components are essential for our method. The advantage over random agents cannot be achieved if either of the component is absent.

\begin{table}[ht]
    \centering
    \sisetup{separate-uncertainty= true, table-format=3.1(1), multi-part-units=single}
    \begin{tabular}{lSSS}
        \hline
               & \text{Common} & \text{Mutated} & \text{All Games} \\
        \hline
        - ensemble  & -3.2 \pm 7.0& -2.7 \pm 9.2~$\dagger$ &  -2.9 \pm 7.5~$\dagger$ \\
        - optimize  & 0.1 \pm 9.9 & -2.0 \pm 5.9~$\dagger$ & -0.4 \pm 8.7~$\dagger$ \\
        Ours   &\color{blue} 14.6 \pm 19.3& \color{blue}  12.5 \pm 24.8 & \color{blue} 16.3 \pm 22.6\\
        \hline
    \end{tabular}
    \caption{\textbf{Game AI win rate advantage between our method and its ablations}. Means and standard deviations (shown after $\pm$) of win rate advantages ($\times 10^{-2}$) across all game instances. All metrics between our method and its ablations show statistically significant difference (p-value $\leq 0.05$ in paired t-test). $\dagger$ represents two metrics are unlikely to be different (p-value $\geq 0.7$ in paired t-test).}
    \label{tab:gameai-adv-ablation}
\end{table}

\subsection{Intermediate Results}

During the construction of gameplay AI, our method produces policy components in text form first, which is converted to code during the second augmentation process.
\begin{tcolorbox}[
breakable,
title=Gameplay AI policy component as texts,  
colframe=gray, 
colback=white,
]
\begin{lstlisting}[language=markdown]
**Prioritize Building Melds**
Focus on arranging cards in the hand into matched sets of either three or four of a kind, or sequences of three or more cards of the same suit. Discard cards that do not contribute towards completing a meld.

**Draw from Discard Pile Judiciously**
Only draw from the discard pile when the top two cards can be directly used to form a meld or significantly improve the hand. Otherwise, prefer drawing from the stock pile to maintain unpredictability.

**Reserve Key Cards**
Keep hold of high-value cards like aces and face cards only if they are part of an ongoing strategy to complete a meld or if they are necessary to win the game soon. Otherwise, discard them to minimize penalty points.

**PotentialMeldCreationValue**
A metric that assesses the potential of forming new melds after drawing two cards, either from the stock pile or discard pile. It evaluates the likely improvement in the hand's structure, considering both the draw options and the current meld opportunities.

**UnmatchedCardPenalty**
A metric that evaluates the penalty incurred from holding high-value unmatched cards. This metric helps in deciding the priority of discarding high-value cards from the hand to lower potential score penalties.

**Refined Meld Building Strategy** (Mutually inspired policy)
To refine the strategy for building melds, the AI player should incorporate the following steps: prioritize drawing from the source that maximizes 'PotentialMeldCreationValue.' When choosing cards to discard, evaluate based on 'UnmatchedCardPenalty,' preferably discarding high-penalty cards unless they significantly contribute to potential melds. Use 'MeldCompletionLikelihood' to assess laying off opportunities. Lastly, factor in 'OpponentInterferenceRisk' to prevent assisting opponents with your discards. These criteria ensure strategic decision-making based on the game state, maximizing winning potential and minimizing penalties.
\end{lstlisting}
\end{tcolorbox}

\begin{tcolorbox}[
breakable,
title=Gameplay AI policy component in code,  
colframe=gray, 
colback=white,
]
\begin{lstlisting}[]
def score(state: dict, action: str) -> float:
    # Helper function to calculate the potential meld creation value
    def calculate_potential_meld_creation_value(hand, card):
        value = 0
        # Check for potential sets (three or four of a kind)
        rank_count = sum(1 for c in hand if c['rank'] == card['rank'])
        if rank_count == 2: # possibility to make a set of three
            value += 0.3
        if rank_count == 3: # possibility to make a set of four
            value += 0.5
                    
        # Check for potential sequences
        card_ranks = {'A': 1, '2': 2, '3': 3, '4': 4, '5': 5, '6': 6,
                      '7': 7, '8': 8, '9': 9, '10': 10, 'J': 11, 'Q': 12, 'K': 13}
        hand_ranks = [card_ranks[c['rank']] for c in hand if c['suit'] == card['suit']]
        hand_ranks.append(card_ranks[card['rank']])
        hand_ranks.sort()
        
        # Look for consecutive cards
        for i in range(len(hand_ranks) - 2):
            if hand_ranks[i+1] == hand_ranks[i] + 1 and hand_ranks[i+2] == hand_ranks[i] + 2:
                value += 0.4
                
        return value
        
    # Helper function to calculate unmatched card penalty
    def calculate_unmatched_card_penalty(hand):
        penalty = 0
        for card in hand:
            penalty += {'A': 11, 'K': 10, 'Q': 10, 'J': 10,
                        '10': 10, '9': 9, '8': 8, '7': 7,
                        '6': 6, '5': 5, '4': 4, '3': 3, '2': 2}[card['rank']]
        return penalty

    # Extract relevant information
    current_player_index = state['common']['current_player']
    current_player = state['players'][current_player_index]
    hand = current_player['facedown_cards']['hand']
    discard_pile = state['common']['faceup_cards']['discard_pile']

    # Initialize the result score
    result_score = 0.0

    # Calculate potential meld creation value of the action
    if action == "draw":
        draw_source = state['legal_actions'][0]['args']['source']  # assuming we're provided with actions arguments
        
        # Check both stock and discard pile for card drawing
        if draw_source == "stock":
            # Assuming uncertainty based draw, hence less score boost
            result_score += 0.1  # a relatively small optimistic boost for drawing
        elif draw_source == "discard_pile":
            if discard_pile:
                top_discard_card = discard_pile[-1]
                result_score += calculate_potential_meld_creation_value(hand, top_discard_card)
            else:
                result_score += 0.05  # minor score addition when discard pile is empty

    # Evaluating penalty for unmatched cards
    unmatched_penalty = calculate_unmatched_card_penalty(hand)
    result_score -= unmatched_penalty * 0.01  # reduce score proportionally to the penalty

    # Make sure that the final score is between 0 and 1\n    result_score = max(0, min(1, result_score))

    return result_score
\end{lstlisting}
\end{tcolorbox}





%%%%%%%%%%%%%%%%%%%%%%%%%%%%%%%%%%%%%%%%%%%%%%%%%%%%
\section{LLM System Prompts}
\subsection{Game Mechanics Design}


\begin{tcolorbox}[
breakable,
title=System Prompt for game logic extraction (depth 1),  
colframe=promptcolor, 
colback=white,
]
\begin{lstlisting}[]
You are a wonderful card game designer who extract game logic chains from the game description. You will be given a game description and you need to extract the game logic chains step by step.

# Task

Read the following game description and answer the questions: how to win this card game? Please respond with all direct mechanics that contribute or hinder to this. for example: discard all the cards, get highest hand score. Your output should be a JSON object with the following format:
```json
[
    {
        "name": "Name of the game mechanic",
        "type": "<only choose among: Contribute, Hinder, Mixed>",
        "description": "A concise explanation of how the mechanic works",
        "reasoning": "Explain how this mechanic contributes to the goal"
    },
    ...
]
```

Remember to extract mechanics that DIRECTLY relate to the goal. Examples:
```
input: <a UNO game description>
correct extraction: "Empty the hands"
wrong extraction: "Play a matching card" or "Draw 4 cards".
analysis: "Play a matching card" or "Draw 4 cards" should be extracted in future steps.
```
```
input: The main purpose of the game is to remove all cards from the table, assembling them in the tableau before removing them. Initially, 54 cards are dealt to the tableau in ten piles, face down except for the top cards. The tableau piles build down by rank, and in-suit sequences can be moved together. The 50 remaining cards can be dealt to the tableau ten at a time when none of the piles are empty.
correct extraction: "Remove all cards from the table"
wrong extraction: "Move sequences strategically".
```

# Game Description
{game_description}
\end{lstlisting}
\end{tcolorbox}


\begin{tcolorbox}[
breakable,
title=System Prompt for game logic extraction (depth 2+),  
colframe=promptcolor, 
colback=white,
]
\begin{lstlisting}[]
What are the game mechanics/concept that directly trigger/stop this mechanic: {goal}? Please respond with the direct mechanic that contributes to this. 

- Try your best to extract ALL direct mechanics.
- When evaluating, pay special attention to mechanics that are interesting or unique.
- You can choose from previously extracted mechanics: {previous_mechanics}. But you can also extract new mechanics/concepts.
- Use the same format as before. 
- Usually there are always mechanics or concepts to extract, unless you have reached the end of the chain such as card design or very basic elements.
- If there are no more mechanics or concepts to extract, respond with:
```json
[]
```
\end{lstlisting}
\end{tcolorbox}



\begin{tcolorbox}[
breakable,
title=System Prompt for reflecting on logic extraction,  
colframe=promptcolor, 
colback=white,
]
\begin{lstlisting}[]
Check and refine your results based on the following criteria:
(1) Are there any mechanics that are not DIRECTLY related to the goal?

Respond a complete result in the same format as before.
\end{lstlisting}
\end{tcolorbox}


\begin{tcolorbox}[
breakable,
title=System Prompt for concept summary,  
colframe=promptcolor, 
colback=white,
]
\begin{lstlisting}[]
**Role**: You are an assistant tasked with summarizing game concept from a list of gameplay mechanics.

**Instructions**:
1. **Input**: You will receive a list of gameplay concept instances. These concepts are often centered around a theme or have similar mechanics.
2. **Output**: Your response should include:
```json
{
    "name": "<The name of the game concept>",
    "description":{
        "common": "<A concise statement that generalizes the core concept or theme common across the listed gameplay concept instances.>",
        "variation": "How the description of given instances vary from the shared common themes."
    }
}
```

# Your input
{input_text}
\end{lstlisting}
\end{tcolorbox}


\begin{tcolorbox}[
breakable,
title=System Prompt for designing new logic instances,  
colframe=promptcolor, 
colback=white,
]
\begin{lstlisting}[]
{prefix_prompt}. Now design the novel concept instance based on your summary. 
- The concept should center around the summary you have concluded.
- The concept should be different from all given instances.
- You should specify its name and description. 
- You should follow this format:
```json
{
    "name": "Name of the game concept (e.g., Dynamic Wild Card, Critical Threshold).",
    "description": "A short concise explanation, following the writing style of the previous instances."
}
```
\end{lstlisting}
\end{tcolorbox}

\begin{tcolorbox}[
breakable,
title=Prefix Prompts for designing new logic instances (partially shown),  
colframe=promptcolor, 
colback=white,
]
\begin{lstlisting}[]
"You are tasked with designing a novel card game concept. Reflect on all aspects of this challenge, including understanding core elements, researching existing games, defining game objectives, developing themes, innovating mechanics, player interaction, rule-setting, prototyping, gathering feedback, and iterating on your design. Confirm your comprehension before proceeding with design suggestions.",

"Before diving into the problem, take a moment to clear your mind and approach it with a fresh perspective. With a clear mind, you can create more innovative solutions.\n\nYou are a skilled card game designer. With these fresh insights, please design a novel game concept instance based on your experience and expertise.",

"As an experienced designer of card games, create an innovative game concept inspired by your prior insights.",

"To design a novel card game concept, first ensure you have a clear understanding of the essential elements: game objectives, rules, mechanics, and player dynamics. Identify what information might be missing or unclear, such as game balance issues, potential for player engagement, or themes. Use this review to innovate and develop an improved or entirely new game concept that enhances the player's experience and meets the outlined objectives. Focus on creating unique mechanics or themes that differentiate it from existing games.",

"You are a skilled card game designer. Please design a novel game concept instance, emphasizing teamwork and open communication. Seek input on the implications and potential impact of different design elements on gameplay. Leverage the diverse perspectives and expertise of the group to enhance the creativity and effectiveness of the game design.",   
\end{lstlisting}
\end{tcolorbox}

\subsection{Code Generation}

\begin{figure*}[htbp]
\centering
\begin{lstlisting}[language=Python]
SYS_PROMPT = """You are an expert quiz generator. Given a text passage and a relationship triple, generate specific questions to test knowledge about this relationship based on the context provided.

Input Format:
- Text: A passage containing information about the relationship
- Relationship: A triple containing {'head': entity1, 'type': relation_type, 'tail': entity2}

Task:
Generate up to 5 focused questions that test understanding of the relationship between the head entity and tail entity, considering:
1. Questions should be answerable solely from the given context
2. Questions should be specific enough to have a unique correct answer
3. Questions can ask about the tail entity given the head entity and relationship type
4. Questions can ask about the relationship between the two entities
5. Questions can ask about specific details that establish this relationship

Requirements:
1. Each question must have a clear, unambiguous answer based on the context
2. Avoid overly broad or general questions
3. Focus on the specific relationship provided
4. Use the context to add specific details to questions
5. Ensure questions and answers are factually consistent with the provided text

Response Format:
The response must be a valid JSON object with the following structure:
{
    "1": {
        "question": "Your question text here",
        "reference_answer": "The correct answer based on context"
    },
    "2": {
        "question": "...",
        "reference_answer": "..."
    }
    // ... up to 5 questions
}

Example Input:
Text: "The Greek Orthodox Church observes Lent as a period of fasting and spiritual reflection that begins on Clean Monday and lasts for 40 days. During this time, adherents follow strict dietary restrictions and increase their prayer and attendance at special services."
Relationship: {'head': 'Lent', 'type': 'religion', 'tail': 'Greek Orthodox'}

Example Output:
{
    "1": {
        "question": "Which religious denomination observes Lent beginning on Clean Monday with a 40-day period of fasting and spiritual reflection?",
        "reference_answer": "Greek Orthodox"
    },
    "2": {
        "question": "In the Greek Orthodox tradition, what is the length of the Lent period?",
        "reference_answer": "40 days"
    }
}
"""

USER_PROMPT = """
Please generate questions based on the following input:

Text: {text}
Relationship: {relationship}
"""
\end{lstlisting}
\caption{Our prompt.}
\end{figure*}
    

\subsection{Gameplay AI Generation}
\begin{tcolorbox}[
breakable,
title=Design game policy components (strategy style) in text,  
colframe=promptcolor, 
colback=white,
]
\begin{lstlisting}[]
You are a powerful assistant who designs an AI player for a card game.

# Game rules
{game_description}

# Game state
The AI player knows all cards in its hands, all game play history. But it does not know the content of other players' hands.

# Potential actions
{game_actions}

# Task
Please think in steps to provide me {item_num} useful strategies to win the game.
For each strategy, please describe its definition and how it relates to the game state and a potential action.

# Response format
Please respond in the following JSON format:
{format_instructions}
\end{lstlisting}
\end{tcolorbox}

\begin{tcolorbox}[
breakable,
title=Design game policy components (metric style) in text,  
colframe=promptcolor, 
colback=white,
]
\begin{lstlisting}[]
You are a powerful assistant who designs an AI player for a card game.

# Game rules
{game_description}

# Game state
The AI player knows all cards in its hands, all game play history. But it does not know the content of other players' hands.

# Potential actions
{game_actions}

# Task
To design a good game play policy, we need to design some game state metrics that constitute a reward function.
Now please think in steps to tell me what useful metric can we derive from a game state?
The metric should be correlated with both the game state and the potential action. Provide me with {item_num} metrics.

# Response format
Please respond in the following JSON format:
{format_instructions}
\end{lstlisting}
\end{tcolorbox}


\begin{tcolorbox}[
breakable,
title=Mutually inspire game policy components in text,  
colframe=promptcolor, 
colback=white,
]
\begin{lstlisting}[]
You are a powerful assistant who designs an AI player for a card game.

# Game rules
{game_description}

# Game state
The AI player knows all cards in its hands, all game play history. But it does not know the content of other players' hands.

# Potential actions
{game_actions}

# Task
Given the following strategy of the game:
```json
{game_strategy}
```

Please think in steps to refine the strategy using the following criteria:
(1) If the strategy has anything obscure, for example, if it mentions "strategically use" or "use at critical moments" without specifying what the critical moments are, please clarify what the critical moments are.
(2) If the strategy is conditioned on a game state metric, please describe how such a strategy will be conditioned on the game state. Here are some hints of the game state:
```json
{game_metrics}
```

# Response format
Please respond in the following JSON format:
{format_instructions}
\end{lstlisting}
\end{tcolorbox}



\begin{tcolorbox}[
breakable,
title=Design code for game policy components,  
colframe=promptcolor, 
colback=white,
]
\begin{lstlisting}[]
You are an action-value engineer trying to write action-value functions in python. Your goal is to write an action-value function that will help the agent decide actions in a card game.

# The game
{game_description}

# The policy
In this action-value function, you will focus on the following policy of the game:
{game_policy}

# The input
The function should be able to take a game state and a planned game action as input. The input should be as follows:
{input_description}

# The output
You should return a reward value ranging from 0 to 1. It is an estimate of the probability of winning the game. 
The closer the reward is to 1, the larger chance of winning we will have.
Try to make the output more continuous.
The reward should be calculated based on both the game state and the given game action.

# Response format
You should return a python function in this format:
```python
def score(state: dict, action: str) -> float:
    pass
    return result_score
```
\end{lstlisting}
\end{tcolorbox}


\begin{tcolorbox}[
breakable,
title=Refine code for game policy components,  
colframe=promptcolor, 
colback=white,
]
\begin{lstlisting}[]
Here are some criteria for the code review:
- No TODOs, pass, placeholders, or any incomplete code;
- Include all code in the score function. Don't create custom class or functions outside;
- the last line should be "return result_score", and the result_score should be a float;
- You can only use the following modules: math, numpy (as np), random;
- no potential bugs;

First, you should check the above criteria one by one and review the code in detail. Show your thinking process.
Then, if the codes are perfect, please end your response with the following sentence:
```
Result is good.
```

Otherwise, you should end your response with the full corrected function code. 
\end{lstlisting}
\end{tcolorbox}


To assess our agent's performance, we implemented the baseline opponent agents from previous works. The prompt configurations for each opponent agent were implemented as detailed below.
\begin{tcolorbox}[
breakable,
title=Game-play Prompt for Chain-of-thought Agent,  
colframe=promptcolor, 
colback=white,
]
\begin{lstlisting}[]
You are a player in a card game. Please do your best to beat the other players and win the game.

The card game is as follows:
```
{game_description}
```

Your current state is as follows:
{observation}

Please think in steps and make a decision based on the current state of the game.

You should return an index of the action you want to take from the list of legal actions.
Wrap your response in a dictionary with the key 'action' and the value as the index of the action you want to take.
for example (You MUST return the action index in the following format):

```json
{
    "action": 0
}
```
\end{lstlisting}
\end{tcolorbox}


\begin{tcolorbox}[
breakable,
title=Game-play Prompt for ReAct Agent,  
colframe=promptcolor, 
colback=white,
]
\begin{lstlisting}[]
You are a player in a card game. Please do your best to beat the other players and win the game.

The card game is as follows:
```
{game_description}
```

Previous observation of game history and your current state is as follows:
```
{observation}
```

Your available actions are as follows:
```
{actions}
```

Please think in steps and make a decision based on the current state of the game.
You should return an index of the action you want to take from the list of legal actions.

Instructions:
1. Analyze the query, previous reasoning steps, and observations.
2. Decide on the next action: choose an action or provide a final answer.
3. Respond in the following JSON format:

Remember:
- Be thorough in your reasoning.
- Choose actions when you need more information.
- Always base your reasoning on the actual observations from chosen action.

If you have enough information to answer the query, wrap your response in a dictionary with the key 'action' and the value as the index of the action you want to take.
for example (You MUST return the action index in the following format):

```json
{
    "thought": "Your detailed reasoning about what to do next",
    "action": 0
}
```
\end{lstlisting}
\end{tcolorbox}


\begin{tcolorbox}[
breakable,
title=Game-play Prompt for Reflexion Agent,  
colframe=promptcolor, 
colback=white,
]
\begin{lstlisting}[]
You are a player in a card game. Please do your best to beat the other players and win the game.

The card game is as follows:
```
{game_description}
```

Previous observation of game history and your current state is as follows:
```
{observation}
```

Your available legal actions are as follows:
```
{actions}
```

Your previous reflection based on the game state is as follows:
```
{reflection}
```

Please think in steps and make a decision based on the current state of the game.
You should return an index of the action you want to take from the list of legal actions.

Instructions:
1. Analyze the query, previous reasoning steps, and observations.
2. Decide on the next action: choose an action or provide a final answer.
3. Respond in the following JSON format:

Remember:
- Be thorough in your reasoning.
- Choose actions when you need more information.
- Always base your reasoning on the actual observations from chosen action.

If you have enough information to answer the query, wrap your response in a dictionary with the key 'action' and the value as the index of the action you want to take.
You MUST return the action index in the following format:

```json
{
    "thought": "Your detailed reasoning about what to do next",
    "action": 0
}
```
\end{lstlisting}
\end{tcolorbox}


\begin{tcolorbox}[
breakable,
title=Long-term Reflection Prompt for Reflexion Agent,  
colframe=promptcolor, 
colback=white,
]
\begin{lstlisting}[]
You are a player in a card game. The card game is as follows:
```
{game_description}
```
This is the payoffs for the game:
```
[{payoffs}], you are player {idx} (list index starting from 0).
```
Here is the previous reflections based on the game results:
```
{reflection}
```

The game is over. Please reflect on the game and provide a detailed summarization for WINNING the game including strategies used, reasoning for actions, and any other relevant information.
You should return a string with your reflection on the game, including strategies used, reasoning for actions, and any other relevant information.
Each reflection item should be precised and concise, NEVER repeat the same reflection twice.
You MUST return the reflection in the following format and ONLY add at most 2 most important new summarization to the previous reflection:
```json
{
    "reflection": "1: strategy 1; 2: strategy 2; ..."
}
```
\end{lstlisting}
\end{tcolorbox}


\begin{tcolorbox}[
breakable,
title=Game-play Prompt for Belief Agent,  
colframe=promptcolor, 
colback=white,
]
\begin{lstlisting}[]
You are a player in a card game. Please do your best to beat the other players and win the game.

The card game description is as follows:
```
{game_description}
```

Previous observation of game history and your current state is as follows:
```
{observation}
```

Here is your previous reflection based on the game state is as follows:
```
{reflection}
```

Previous belief is as follows:
```
self-belief: {self_belief}
world-belief: {world_belief}
```


Please read the game decription and observation carefully.
Then you should analyze your own cards and your strategies in Self-belief and then analyze the opponents cards in World-belief.
Lastly, please select your action from:
```
{actions}
```

You MUST return the action index in the following format:
```json
{
    "self-belief": "I have a good hand", 
    "world-belief": "Dealer has a bad hand", 
    "action": 1
}
```
\end{lstlisting}
\end{tcolorbox}
