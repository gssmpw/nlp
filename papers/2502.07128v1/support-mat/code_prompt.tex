\begin{tcolorbox}[
breakable,
title=Prompts for description structurization,  
colframe=promptcolor, 
colback=white,
]
\begin{lstlisting}[]
Design a structured ruleset for implementing a card game system based on the provided input. Ensure the output includes key components as below. The output should be comprehensive, logical, and organized in a format suitable for programming or detailed documentation purposes. Wrap the output in a markdown block.

Include the following sections:
1. **Game State**
   - Define the game state, categorized into common information and player-specific information (grouped into public and private).
2. **Card**
    - Specify card attributes such as rank, suit, and any special abilities or values.
3. **Deck and Initial Dealing**
    - Describe the deck composition, dealing process, and setup at the beginning of the game.
4. **Legal Action Space**
    - List all possible actions players can perform during their turn, specifying the prerequisites of each action.    
5. **Round**
    - Describe the sequence of play and how the game progresses from one player to the next.
    - Elaborate in each players' turn, the order of actions they can take, and the outcomes of each action.
    - Explain how the game ends and the winning conditions. Pay attention to corner cases such as deck exhaustion or all players passing.
6. **Other Game Mechanics & Rules**
    - Detail any additional game mechanics, rules, or special actions that players can take during the game.
7. **Player Observation Information**
    - Specify what information players can observe during the game, such as their hand, the starter pile, declared suits, and opponent actions.
8. **Payoffs**
    - Explain when game ends, how scoring works, including point values for cards.

Ensure clarity and precision to facilitate implementation or usage as a reference for game rules.

# Example
{example}
```
\end{lstlisting}
\end{tcolorbox}


\begin{tcolorbox}[
breakable,
title=System Prompts for code draft,  
colframe=promptcolor, 
colback=white,
]
\begin{lstlisting}[]
You are a card game programmer tasked with implementing a card game based on the given description. Using the provided class templates, your goal is to write only the necessary child classes and implement only the methods indicated with `TODO` comments.

**Instructions:**
- Include only the methods you need to override from the provided class templates.
- Respond with complete, runnable Python code.
- Do **not** include TODOs, placeholders, or explanations; output only the final code.

**Code Environment:**
- These code belongs to a larger game framework. Use them as reference only. Don't include them in your response.
```
{environment_code}
```

**Code Template:**
- Only modify or implement the methods specified with `TODO` comments to complete the game logic.
```
{python_classes}
```

**Examples for Reference:**  
Use these examples as a guide for response format and method implementation.
{examples}

---

### Your Task
Based on the following game description, implement the required classes and methods:

**Game Description:**
```
{game_description}
```

### Note:
- Do **not** raise exceptions (e.g., `ValueError`) when parsing action strings. Instead, ensure the legal action space and action string format are appropriately structured.
\end{lstlisting}
\end{tcolorbox}



\begin{tcolorbox}[
breakable,
title=Prompts for consistency validation,  
colframe=promptcolor, 
colback=white,
]
\begin{lstlisting}[]
You are a card game programmer who verifies code for a card game. You are given a card game description and a part of game play log using the code.

# Task
- You should evaluate step by step to see if the game play log aligns with the rules in the game description.
- Also, examine if the legal action choices in each turn is correct and complete.
- If the game play aligns with the rules, simply return "pass" in the analysis summary.
- If the game play does not align with the rules, you should response in a two-part format: summary and quote(optional). Focus one issue at a time.

# Your game description
```
{game_description}
```

# Your game play log
Note: Only the last several turns of the play log is provided. But if the play log is too short or empty, there might be some errors in the game code.
```
{game_play_log}
```

# Output Format

If the game play log aligns with the rules:
```
***Step by step evaluation***
<your evaluation here>

***Analysis Summary***
```pass```
```

If you doubt the log is too short or empty because of some errors in the game code:
```
***Step by step evaluation***
<your evaluation here>

***Analysis Summary***
```log is too short or empty```
```

Otherwise:
```
***Step by step evaluation***
<your evaluation here>

***Analysis Summary***
Summary:
```text
<summarize the issue>
```
Quote (optional):
```markdown
<quote related game description segment if game play log does not align with the rules>
```
```
\end{lstlisting}
\end{tcolorbox}

