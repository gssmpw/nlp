\begin{tcolorbox}[
breakable,
title=Design game policy components (strategy style) in text,  
colframe=promptcolor, 
colback=white,
]
\begin{lstlisting}[]
You are a powerful assistant who designs an AI player for a card game.

# Game rules
{game_description}

# Game state
The AI player knows all cards in its hands, all game play history. But it does not know the content of other players' hands.

# Potential actions
{game_actions}

# Task
Please think in steps to provide me {item_num} useful strategies to win the game.
For each strategy, please describe its definition and how it relates to the game state and a potential action.

# Response format
Please respond in the following JSON format:
{format_instructions}
\end{lstlisting}
\end{tcolorbox}

\begin{tcolorbox}[
breakable,
title=Design game policy components (metric style) in text,  
colframe=promptcolor, 
colback=white,
]
\begin{lstlisting}[]
You are a powerful assistant who designs an AI player for a card game.

# Game rules
{game_description}

# Game state
The AI player knows all cards in its hands, all game play history. But it does not know the content of other players' hands.

# Potential actions
{game_actions}

# Task
To design a good game play policy, we need to design some game state metrics that constitute a reward function.
Now please think in steps to tell me what useful metric can we derive from a game state?
The metric should be correlated with both the game state and the potential action. Provide me with {item_num} metrics.

# Response format
Please respond in the following JSON format:
{format_instructions}
\end{lstlisting}
\end{tcolorbox}


\begin{tcolorbox}[
breakable,
title=Mutually inspire game policy components in text,  
colframe=promptcolor, 
colback=white,
]
\begin{lstlisting}[]
You are a powerful assistant who designs an AI player for a card game.

# Game rules
{game_description}

# Game state
The AI player knows all cards in its hands, all game play history. But it does not know the content of other players' hands.

# Potential actions
{game_actions}

# Task
Given the following strategy of the game:
```json
{game_strategy}
```

Please think in steps to refine the strategy using the following criteria:
(1) If the strategy has anything obscure, for example, if it mentions "strategically use" or "use at critical moments" without specifying what the critical moments are, please clarify what the critical moments are.
(2) If the strategy is conditioned on a game state metric, please describe how such a strategy will be conditioned on the game state. Here are some hints of the game state:
```json
{game_metrics}
```

# Response format
Please respond in the following JSON format:
{format_instructions}
\end{lstlisting}
\end{tcolorbox}



\begin{tcolorbox}[
breakable,
title=Design code for game policy components,  
colframe=promptcolor, 
colback=white,
]
\begin{lstlisting}[]
You are an action-value engineer trying to write action-value functions in python. Your goal is to write an action-value function that will help the agent decide actions in a card game.

# The game
{game_description}

# The policy
In this action-value function, you will focus on the following policy of the game:
{game_policy}

# The input
The function should be able to take a game state and a planned game action as input. The input should be as follows:
{input_description}

# The output
You should return a reward value ranging from 0 to 1. It is an estimate of the probability of winning the game. 
The closer the reward is to 1, the larger chance of winning we will have.
Try to make the output more continuous.
The reward should be calculated based on both the game state and the given game action.

# Response format
You should return a python function in this format:
```python
def score(state: dict, action: str) -> float:
    pass
    return result_score
```
\end{lstlisting}
\end{tcolorbox}


\begin{tcolorbox}[
breakable,
title=Refine code for game policy components,  
colframe=promptcolor, 
colback=white,
]
\begin{lstlisting}[]
Here are some criteria for the code review:
- No TODOs, pass, placeholders, or any incomplete code;
- Include all code in the score function. Don't create custom class or functions outside;
- the last line should be "return result_score", and the result_score should be a float;
- You can only use the following modules: math, numpy (as np), random;
- no potential bugs;

First, you should check the above criteria one by one and review the code in detail. Show your thinking process.
Then, if the codes are perfect, please end your response with the following sentence:
```
Result is good.
```

Otherwise, you should end your response with the full corrected function code. 
\end{lstlisting}
\end{tcolorbox}