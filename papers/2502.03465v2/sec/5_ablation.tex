\section{Ablation Study} 

% \begin{figure}[t]
% \centering
% \includegraphics[width=1.0\columnwidth]{img/depth-flow-distill.pdf}
% \caption{Qualitative ablation on 2D prior regularization.}
% \label{fig:prior_ablation}
% \vspace{-4mm}
% \end{figure}

% \begin{table}[t]
%   \centering
%   \setlength{\tabcolsep}{8pt}  
%   \renewcommand{\arraystretch}{1.2}  
%   \resizebox{0.9\columnwidth}{!}{\begin{tabular}{l S[table-format=2.2] S[table-format=1.4] S[table-format=1.4]}
%     \toprule
%     \textbf{Method} & {\textbf{PSNR}$\uparrow$} & {\textbf{SSIM}$\uparrow$} & {\textbf{LPIPS}$\downarrow$} \\
%     \midrule
%     w/o Flow Loss & {26.39} & {0.8345} & {0.2482} \\
%     w/o Depth Loss & {28.15} & {0.8751} & {0.1858} \\
%     w/o STAG & {19.58} & {0.6255} & {0.4713} \\
%     Ours ($n = 2$) & {29.18} & {0.9015} & {0.1415} \\
%     Ours ($n = 3$) & {31.15} & {0.9266} & {0.1385} \\
%     \bottomrule
%   \end{tabular}}
%   \vspace{-2mm}
%   \caption{Ablation study on component-wise contribution. $n$ represents the upsampling ratio.} 
%   \label{tab:ablation}
%   \vspace{-4mm}
%   \end{table}


We analyze NutWorld's design choices through ablation studies on the 50 selected video clips. As shown in Table~\ref{tab:ablation}, our experiments demonstrate that discarding any component from the multi-component pipeline leads to significant performance degradation.

% \vspace{-2mm}

\begin{table}[h!]
\setlength{\abovecaptionskip}{0.1cm}
\setlength{\belowcaptionskip}{0cm}
\vspace{-4mm}
  \centering
  \setlength{\tabcolsep}{8pt}  
  \renewcommand{\arraystretch}{1.2}  
  \resizebox{0.75\columnwidth}{!}{\begin{tabular}{l S[table-format=2.2] S[table-format=1.4] S[table-format=1.4]}
    \toprule
    \textbf{Method} & {\textbf{PSNR}$\uparrow$} & {\textbf{SSIM}$\uparrow$} & {\textbf{LPIPS}$\downarrow$} \\
    \midrule
    w/o Flow Loss & {26.39} & {0.8345} & {0.2482} \\
    w/o Depth Loss & {28.15} & {0.8751} & {0.1858} \\
    w/o STAG & {19.58} & {0.6255} & {0.4713} \\
    Ours ($n = 2$) & {29.18} & {0.9015} & {0.1415} \\
    Ours ($n = 3$) & {31.15} & {0.9266} & {0.1385} \\
    \bottomrule
  \end{tabular}}
  \caption{Ablation study on component-wise contribution. $n$ represents the number of upsampler blocks.} 
  \label{tab:ablation}
  \vspace{-4mm}
  \end{table}

\noindent \textbf{Ablations on STAG representation.}
To validate the effectiveness of STAG representation, we perform an ablation by loosening its positional constraints. Following~\cite{tang2025lgm}, we implement a less constrained variant where Gaussian positions are predicted with only $tanh$ activation, limiting their range to $[-1, 1]$. As shown in Table~\ref{tab:ablation}, this loosened constraint leads to significantly degraded performance by \underline{$10dB$} decrease in PSNR, with slower convergence, blurred artifacts and unstable optimization behavior.
These results demonstrate the necessity of structured positional constraints, as the unconstrained Gaussians introduce additional spatial ambiguity during alpha compositing. In contrast, STAG's localized positional constraints provide explicit spatial and temporal correspondence, enabling efficient optimization and high-quality rendering.


\noindent\textbf{Ablations on depth prior}. To evaluate the depth prior (Eq.~\ref{eq:depth_prior}), we trained the NutWorld variant without depth supervision in Figure~\ref{fig:prior_ablation}(a) for comparison, which reveals that the variant w/o depth tend to lost spatial arrangement and converges to a collapsed shortcut instead, i.e. all STAGs are splattered onto a single $z$ plane. Furthermore, quantitative experiments in Table~\ref{tab:ablation} reveal that removing the depth prior leads to a degraded rendering quality, as evidenced by the decrease in PSNR from $29.18 dB$ to $28.15 dB$. These results highlight the necessity of depth prior to address the spatial ambiguity in NutWorld.

\begin{figure}[h!]
\vspace{-4mm}
\setlength{\abovecaptionskip}{0.1cm}
\setlength{\belowcaptionskip}{0cm}
\centering
\includegraphics[width=1.0\columnwidth]{img/depth-flow-distill.pdf}
\caption{Qualitative ablation on 2D prior regularization.}
\label{fig:prior_ablation}
\vspace{-4mm}
\end{figure}


\noindent\textbf{Ablations on flow prior}. To evaluate the flow prior (Eq.~\ref{eq:flow_prior}), we trained a NutWorld variant without flow supervision for comparison. The distribution of the deformation field $\mu(t)$ across $K=6$ frames is visualized in Figure~\ref{fig:prior_ablation}(b) via a violin plot. Without flow supervision, the model exhibits large deformation values with low variance, causing STAGs to deviate from the canonical space in non-reference frames as defined in Eq.~\ref{eq:temporal_align}. This indicates that the variant w/o flow tends to learn an undesirable shortcut by representing each frame with independent STAGs, leading to temporal discontinuity. In contrast, with flow supervision, the distributions are centered near zero with appropriate variance, demonstrating that NutWorld could recover temporal motion through the flow prior, which effectively prevents such shortcut behavior. Additionally, quantitative experiments in Table~\ref{tab:ablation} show that temporal discontinuity leads to inferior reconstruction quality, especially for complex motions.





% \noindent \textbf{Ablations on upsampling ratio.} \yxy{talking about compression and points, compare again with SaV and explain why}

%可视化进一步说明我们不是per-frame gen而是时空连续,点出可能存在的short-cut问题
%其他设计结果的ablation,以 psnr的展示为主

