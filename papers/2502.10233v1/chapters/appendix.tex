\clearpage

\newpage
\appendix
\onecolumn
\renewcommand{\paragraph}[1]{\bigskip\noindent\textbf{#1}\quad} % Makes \paragraph bold


\section{Formal Definition of the MSPRP}
\label{appendix:notation}

The following mathematical model describes the min-max MSPRP cover in this work and table \ref{tab:notation} summarizes the notation used to define the model.

\begin{align}
  \label{eq:obj_fn}
  \textbf{min} \qquad \qquad \qquad \quad Z &= \underset{b \in \NumTours}{\text{max}} \sum_{(i,j) \in \mathcal{E}} D_{ij} \cdot x_{ijb} \\
    \label{eq:flow_cnstr}
    \textbf{s.t.} \qquad \quad \; \; \sum_{(i,j) \in \mathcal{E}} x_{ijb} &=  \sum_{(j,i) \in \mathcal{E}} x_{jib} &&\forall \, i \in \SetOfAllNodes, b \in \NumTours \\
    \label{eq:no_mult_visits}
    \sum_{(i,j) \in \mathcal{E}} x_{ijb} &\leq 1 &&\forall \, i \in \SetOfAllNodes, \, b \in \NumTours \\
    \label{eq:take_only_if_visit}
    BigM \cdot \sum_{i \in \SetOfAllNodes} x_{ijb} &\geq y_{jb} && \forall \, j \in \SetOfStorageLocations , \,b \in \NumTours \\
    \label{eq:include_depot}
    \sum_{h \in \SetOfPackingStations} \sum_{j \in \SetOfStorageLocations} x_{hjb} &= 1 && \forall \, b \in \NumTours \\
     \label{eq:subtour}
    \sum_{i\in S} \sum_{j \in S} x_{ijb} &\leq |S| -1 && \forall \, b \in \NumTours, \,  S \subset \SetOfStorageLocations, |S| \geq 2 \\
    \label{eq:capacity}
     \sum_{i \in \SetOfStorageLocations} y_{ib} &\leq \kappa && \forall b \in \NumTours\\
    \label{eq:meet_demand}
    \sum_{i \in \SetOfLocationsIncludePickItem{p}} \sum_{b \in \NumTours} y_{ib} &= d_p && \forall \, p \in \SetOfSKUs \\
    \label{eq:no_exceed_sup}
    \sum_{b \in \NumTours} y_{ib} &\leq n_{i} && \forall \, i \in \SetOfStorageLocations \\
    \label{eq:def_x}
     x_{ijb} &\in \{0,1\} && \forall \, (i,j) \in \mathcal{E},\, b \in \NumTours \\
    \label{eq:def_y}
     y_{ib} &\geq 0 && \forall \,  i \in \SetOfStorageLocations, \, b \in \NumTours
\end{align}

%% !TeX root = main.tex 


\newcommand{\lnote}{\textcolor[rgb]{1,0,0}{Lydia: }\textcolor[rgb]{0,0,1}}
\newcommand{\todo}{\textcolor[rgb]{1,0,0.5}{To do: }\textcolor[rgb]{0.5,0,1}}


\newcommand{\state}{S}
\newcommand{\meas}{M}
\newcommand{\out}{\mathrm{out}}
\newcommand{\piv}{\mathrm{piv}}
\newcommand{\pivotal}{\mathrm{pivotal}}
\newcommand{\isnot}{\mathrm{not}}
\newcommand{\pred}{^\mathrm{predict}}
\newcommand{\act}{^\mathrm{act}}
\newcommand{\pre}{^\mathrm{pre}}
\newcommand{\post}{^\mathrm{post}}
\newcommand{\calM}{\mathcal{M}}

\newcommand{\game}{\mathbf{V}}
\newcommand{\strategyspace}{S}
\newcommand{\payoff}[1]{V^{#1}}
\newcommand{\eff}[1]{E^{#1}}
\newcommand{\p}{\vect{p}}
\newcommand{\simplex}[1]{\Delta^{#1}}

\newcommand{\recdec}[1]{\bar{D}(\hat{Y}_{#1})}





\newcommand{\sphereone}{\calS^1}
\newcommand{\samplen}{S^n}
\newcommand{\wA}{w}%{w_{\mathfrak{a}}}
\newcommand{\Awa}{A_{\wA}}
\newcommand{\Ytil}{\widetilde{Y}}
\newcommand{\Xtil}{\widetilde{X}}
\newcommand{\wst}{w_*}
\newcommand{\wls}{\widehat{w}_{\mathrm{LS}}}
\newcommand{\dec}{^\mathrm{dec}}
\newcommand{\sub}{^\mathrm{sub}}

\newcommand{\calP}{\mathcal{P}}
\newcommand{\totspace}{\calZ}
\newcommand{\clspace}{\calX}
\newcommand{\attspace}{\calA}

\newcommand{\Ftil}{\widetilde{\calF}}

\newcommand{\totx}{Z}
\newcommand{\classx}{X}
\newcommand{\attx}{A}
\newcommand{\calL}{\mathcal{L}}



\newcommand{\defeq}{\mathrel{\mathop:}=}
\newcommand{\vect}[1]{\ensuremath{\mathbf{#1}}}
\newcommand{\mat}[1]{\ensuremath{\mathbf{#1}}}
\newcommand{\dd}{\mathrm{d}}
\newcommand{\grad}{\nabla}
\newcommand{\hess}{\nabla^2}
\newcommand{\argmin}{\mathop{\rm argmin}}
\newcommand{\argmax}{\mathop{\rm argmax}}
\newcommand{\Ind}[1]{\mathbf{1}\{#1\}}

\newcommand{\norm}[1]{\left\|{#1}\right\|}
\newcommand{\fnorm}[1]{\|{#1}\|_{\text{F}}}
\newcommand{\spnorm}[2]{\left\| {#1} \right\|_{\text{S}({#2})}}
\newcommand{\sigmin}{\sigma_{\min}}
\newcommand{\tr}{\text{tr}}
\renewcommand{\det}{\text{det}}
\newcommand{\rank}{\text{rank}}
\newcommand{\logdet}{\text{logdet}}
\newcommand{\trans}{^{\top}}
\newcommand{\poly}{\text{poly}}
\newcommand{\polylog}{\text{polylog}}
\newcommand{\st}{\text{s.t.~}}
\newcommand{\proj}{\mathcal{P}}
\newcommand{\projII}{\mathcal{P}_{\parallel}}
\newcommand{\projT}{\mathcal{P}_{\perp}}
\newcommand{\projX}{\mathcal{P}_{\mathcal{X}^\star}}
\newcommand{\inner}[1]{\langle #1 \rangle}

\renewcommand{\Pr}{\mathbb{P}}
\newcommand{\Z}{\mathbb{Z}}
\newcommand{\N}{\mathbb{N}}
\newcommand{\R}{\mathbb{R}}
\newcommand{\E}{\mathbb{E}}
\newcommand{\F}{\mathcal{F}}
\newcommand{\var}{\mathrm{var}}
\newcommand{\cov}{\mathrm{cov}}


\newcommand{\calN}{\mathcal{N}}

\newcommand{\jccomment}{\textcolor[rgb]{1,0,0}{C: }\textcolor[rgb]{1,0,1}}
\newcommand{\fracpar}[2]{\frac{\partial #1}{\partial  #2}}

\newcommand{\A}{\mathcal{A}}
\newcommand{\B}{\mat{B}}
%\newcommand{\C}{\mat{C}}

\newcommand{\I}{\mat{I}}
\newcommand{\M}{\mat{M}}
\newcommand{\D}{\mat{D}}
%\newcommand{\U}{\mat{U}}
\newcommand{\V}{\mat{V}}
\newcommand{\W}{\mat{W}}
\newcommand{\X}{\mat{X}}
\newcommand{\Y}{\mat{Y}}
\newcommand{\mSigma}{\mat{\Sigma}}
\newcommand{\mLambda}{\mat{\Lambda}}
\newcommand{\e}{\vect{e}}
\newcommand{\g}{\vect{g}}
\renewcommand{\u}{\vect{u}}
\newcommand{\w}{\vect{w}}
\newcommand{\x}{\vect{x}}
\newcommand{\y}{\vect{y}}
\newcommand{\z}{\vect{z}}
\newcommand{\fI}{\mathfrak{I}}
\newcommand{\fS}{\mathfrak{S}}
\newcommand{\fE}{\mathfrak{E}}
\newcommand{\fF}{\mathfrak{F}}

\newcommand{\Risk}{\mathcal{R}}

\renewcommand{\L}{\mathcal{L}}
\renewcommand{\H}{\mathcal{H}}

\newcommand{\cn}{\kappa}
\newcommand{\nn}{\nonumber}


\newcommand{\Hess}{\nabla^2}
\newcommand{\tlO}{\tilde{O}}
\newcommand{\tlOmega}{\tilde{\Omega}}

\newcommand{\calF}{\mathcal{F}}
\newcommand{\fhat}{\widehat{f}}
\newcommand{\calS}{\mathcal{S}}

\newcommand{\calX}{\mathcal{X}}
\newcommand{\calY}{\mathcal{Y}}
\newcommand{\calD}{\mathcal{D}}
\newcommand{\calZ}{\mathcal{Z}}
\newcommand{\calA}{\mathcal{A}}
\newcommand{\fbayes}{f^B}
\newcommand{\func}{f^U}


\newcommand{\bayscore}{\text{calibrated Bayes score}}
\newcommand{\bayrisk}{\text{calibrated Bayes risk}}

\newtheorem{example}{Example}[section]
\newtheorem{exc}{Exercise}[section]
%\newtheorem{rem}{Remark}[section]

\newtheorem{theorem}{Theorem}[section]
\newtheorem{definition}{Definition}
\newtheorem{proposition}[theorem]{Proposition}
\newtheorem{corollary}[theorem]{Corollary}

\newtheorem{remark}{Remark}[section]
\newtheorem{lemma}[theorem]{Lemma}
\newtheorem{claim}[theorem]{Claim}
\newtheorem{fact}[theorem]{Fact}
\newtheorem{assumption}{Assumption}

\newcommand{\iidsim}{\overset{\mathrm{i.i.d.}}{\sim}}
\newcommand{\unifsim}{\overset{\mathrm{unif}}{\sim}}
\newcommand{\sign}{\mathrm{sign}}
\newcommand{\wbar}{\overline{w}}
\newcommand{\what}{\widehat{w}}
\newcommand{\KL}{\mathrm{KL}}
\newcommand{\Bern}{\mathrm{Bernoulli}}
\newcommand{\ihat}{\widehat{i}}
\newcommand{\Dwst}{\calD^{w_*}}
\newcommand{\fls}{\widehat{f}_{n}}


\newcommand{\brpi}{\pi^{br}}
\newcommand{\brtheta}{\theta^{br}}

% \newcommand{\M}{\mat{M}}
% \newcommand\Mmh{\mat{M}^{-1/2}}
% \newcommand{\A}{\mat{A}}
% \newcommand{\B}{\mat{B}}
% \newcommand{\C}{\mat{C}}
% \newcommand{\Et}[1][t]{\mat{E_{#1}}}
% \newcommand{\Etp}{\Et[t+1]}
% \newcommand{\Errt}[1][t]{\mat{\bigtriangleup_{#1}}}
% \newcommand\cnM{\kappa}
% \newcommand{\cn}[1]{\kappa\left(#1\right)}
% \newcommand\X{\mat{X}}
% \newcommand\fstar{f_*}
% \newcommand\Xt[1][t]{\mat{X_{#1}}}
% \newcommand\ut[1][t]{{u_{#1}}}
% \newcommand\Xtinv{\inv{\Xt}}
% \newcommand\Xtp{\mat{X_{t+1}}}
% \newcommand\Xtpinv{\inv{\left(\mat{X_{t+1}}\right)}}
% \newcommand\U{\mat{U}}
% \newcommand\UTr{\trans{\mat{U}}}
% \newcommand{\Ut}[1][t]{\mat{U_{#1}}}
% \newcommand{\Utinv}{\inv{\Ut}}
% \newcommand{\UtTr}[1][t]{\trans{\mat{U_{#1}}}}
% \newcommand\Utp{\mat{U_{t+1}}}
% \newcommand\UtpTr{\trans{\mat{U}_{t+1}}}
% \newcommand\Utptild{\mat{\widetilde{U}_{t+1}}}
% \newcommand\Us{\mat{U^*}}
% \newcommand\UsTr{\trans{\mat{U^*}}}
% \newcommand{\Sigs}{\mat{\Sigma}}
% \newcommand{\Sigsmh}{\Sigs^{-1/2}}
% \newcommand{\eye}{\mat{I}}
% \newcommand{\twonormbound}{\left(4+\DPhi{\M}{\Xt[0]}\right)\twonorm{\M}}
% \newcommand{\lamj}{\lambda_j}

% \renewcommand\u{\vect{u}}
% \newcommand\uTr{\trans{\vect{u}}}
% \renewcommand\v{\vect{v}}
% \newcommand\vTr{\trans{\vect{v}}}
% \newcommand\w{\vect{w}}
% \newcommand\wTr{\trans{\vect{w}}}
% \newcommand\wperp{\vect{w}_{\perp}}
% \newcommand\wperpTr{\trans{\vect{w}_{\perp}}}
% \newcommand\wj{\vect{w_j}}
% \newcommand\vj{\vect{v_j}}
% \newcommand\wjTr{\trans{\vect{w_j}}}
% \newcommand\vjTr{\trans{\vect{v_j}}}

% \newcommand{\DPhi}[2]{\ensuremath{D_{\Phi}\left(#1,#2\right)}}
% \newcommand\matmult{{\omega}}


The objective function (\ref{eq:obj_fn}) aims to minimize the maximum distance traveled by any picker. Constraints (\ref{eq:flow_cnstr}) ensure that every storage location visited during a picker’s tour is also exited. Constraints (\ref{eq:no_mult_visits}) prevent storage locations from being visited multiple times within a single tour, though multiple visits are allowed across tours if the picker's capacity is insufficient to fulfill the demand in one trip. However, revisiting the same storage location within a single tour is inefficient and therefore disallowed.
Using the Big-M formulation in (\ref{eq:take_only_if_visit}), we ensure that items can only be picked from storage locations included in the respective tour. Since no more than $\kappa$ items can be picked in one tour, setting $BigM=\kappa$ is sufficient.
To guarantee that each tour begins and ends at a packing station, constraints (\ref{eq:include_depot}) enforce that a packing station is exited exactly once per tour. Combined with the network flow constraints (\ref{eq:flow_cnstr}), this ensures that every tour returns to the packing station it initially departed from. Additionally, subtour elimination constraints (\ref{eq:subtour}) ensure that all visited storage locations are connected within a tour.
Constraints (\ref{eq:capacity}) prevent the picker's capacity from being exceeded, while constraints (\ref{eq:meet_demand}) ensure that all customer orders are fulfilled. To avoid exceeding the available stock of items in any storage location during order picking, constraints (\ref{eq:no_exceed_sup}) are enforced. Finally, constraints (\ref{eq:def_x}) and (\ref{eq:def_y}) define the domains of the decision variables $x$ and $y$.
% !TeX root = main.tex 


\newcommand{\lnote}{\textcolor[rgb]{1,0,0}{Lydia: }\textcolor[rgb]{0,0,1}}
\newcommand{\todo}{\textcolor[rgb]{1,0,0.5}{To do: }\textcolor[rgb]{0.5,0,1}}


\newcommand{\state}{S}
\newcommand{\meas}{M}
\newcommand{\out}{\mathrm{out}}
\newcommand{\piv}{\mathrm{piv}}
\newcommand{\pivotal}{\mathrm{pivotal}}
\newcommand{\isnot}{\mathrm{not}}
\newcommand{\pred}{^\mathrm{predict}}
\newcommand{\act}{^\mathrm{act}}
\newcommand{\pre}{^\mathrm{pre}}
\newcommand{\post}{^\mathrm{post}}
\newcommand{\calM}{\mathcal{M}}

\newcommand{\game}{\mathbf{V}}
\newcommand{\strategyspace}{S}
\newcommand{\payoff}[1]{V^{#1}}
\newcommand{\eff}[1]{E^{#1}}
\newcommand{\p}{\vect{p}}
\newcommand{\simplex}[1]{\Delta^{#1}}

\newcommand{\recdec}[1]{\bar{D}(\hat{Y}_{#1})}





\newcommand{\sphereone}{\calS^1}
\newcommand{\samplen}{S^n}
\newcommand{\wA}{w}%{w_{\mathfrak{a}}}
\newcommand{\Awa}{A_{\wA}}
\newcommand{\Ytil}{\widetilde{Y}}
\newcommand{\Xtil}{\widetilde{X}}
\newcommand{\wst}{w_*}
\newcommand{\wls}{\widehat{w}_{\mathrm{LS}}}
\newcommand{\dec}{^\mathrm{dec}}
\newcommand{\sub}{^\mathrm{sub}}

\newcommand{\calP}{\mathcal{P}}
\newcommand{\totspace}{\calZ}
\newcommand{\clspace}{\calX}
\newcommand{\attspace}{\calA}

\newcommand{\Ftil}{\widetilde{\calF}}

\newcommand{\totx}{Z}
\newcommand{\classx}{X}
\newcommand{\attx}{A}
\newcommand{\calL}{\mathcal{L}}



\newcommand{\defeq}{\mathrel{\mathop:}=}
\newcommand{\vect}[1]{\ensuremath{\mathbf{#1}}}
\newcommand{\mat}[1]{\ensuremath{\mathbf{#1}}}
\newcommand{\dd}{\mathrm{d}}
\newcommand{\grad}{\nabla}
\newcommand{\hess}{\nabla^2}
\newcommand{\argmin}{\mathop{\rm argmin}}
\newcommand{\argmax}{\mathop{\rm argmax}}
\newcommand{\Ind}[1]{\mathbf{1}\{#1\}}

\newcommand{\norm}[1]{\left\|{#1}\right\|}
\newcommand{\fnorm}[1]{\|{#1}\|_{\text{F}}}
\newcommand{\spnorm}[2]{\left\| {#1} \right\|_{\text{S}({#2})}}
\newcommand{\sigmin}{\sigma_{\min}}
\newcommand{\tr}{\text{tr}}
\renewcommand{\det}{\text{det}}
\newcommand{\rank}{\text{rank}}
\newcommand{\logdet}{\text{logdet}}
\newcommand{\trans}{^{\top}}
\newcommand{\poly}{\text{poly}}
\newcommand{\polylog}{\text{polylog}}
\newcommand{\st}{\text{s.t.~}}
\newcommand{\proj}{\mathcal{P}}
\newcommand{\projII}{\mathcal{P}_{\parallel}}
\newcommand{\projT}{\mathcal{P}_{\perp}}
\newcommand{\projX}{\mathcal{P}_{\mathcal{X}^\star}}
\newcommand{\inner}[1]{\langle #1 \rangle}

\renewcommand{\Pr}{\mathbb{P}}
\newcommand{\Z}{\mathbb{Z}}
\newcommand{\N}{\mathbb{N}}
\newcommand{\R}{\mathbb{R}}
\newcommand{\E}{\mathbb{E}}
\newcommand{\F}{\mathcal{F}}
\newcommand{\var}{\mathrm{var}}
\newcommand{\cov}{\mathrm{cov}}


\newcommand{\calN}{\mathcal{N}}

\newcommand{\jccomment}{\textcolor[rgb]{1,0,0}{C: }\textcolor[rgb]{1,0,1}}
\newcommand{\fracpar}[2]{\frac{\partial #1}{\partial  #2}}

\newcommand{\A}{\mathcal{A}}
\newcommand{\B}{\mat{B}}
%\newcommand{\C}{\mat{C}}

\newcommand{\I}{\mat{I}}
\newcommand{\M}{\mat{M}}
\newcommand{\D}{\mat{D}}
%\newcommand{\U}{\mat{U}}
\newcommand{\V}{\mat{V}}
\newcommand{\W}{\mat{W}}
\newcommand{\X}{\mat{X}}
\newcommand{\Y}{\mat{Y}}
\newcommand{\mSigma}{\mat{\Sigma}}
\newcommand{\mLambda}{\mat{\Lambda}}
\newcommand{\e}{\vect{e}}
\newcommand{\g}{\vect{g}}
\renewcommand{\u}{\vect{u}}
\newcommand{\w}{\vect{w}}
\newcommand{\x}{\vect{x}}
\newcommand{\y}{\vect{y}}
\newcommand{\z}{\vect{z}}
\newcommand{\fI}{\mathfrak{I}}
\newcommand{\fS}{\mathfrak{S}}
\newcommand{\fE}{\mathfrak{E}}
\newcommand{\fF}{\mathfrak{F}}

\newcommand{\Risk}{\mathcal{R}}

\renewcommand{\L}{\mathcal{L}}
\renewcommand{\H}{\mathcal{H}}

\newcommand{\cn}{\kappa}
\newcommand{\nn}{\nonumber}


\newcommand{\Hess}{\nabla^2}
\newcommand{\tlO}{\tilde{O}}
\newcommand{\tlOmega}{\tilde{\Omega}}

\newcommand{\calF}{\mathcal{F}}
\newcommand{\fhat}{\widehat{f}}
\newcommand{\calS}{\mathcal{S}}

\newcommand{\calX}{\mathcal{X}}
\newcommand{\calY}{\mathcal{Y}}
\newcommand{\calD}{\mathcal{D}}
\newcommand{\calZ}{\mathcal{Z}}
\newcommand{\calA}{\mathcal{A}}
\newcommand{\fbayes}{f^B}
\newcommand{\func}{f^U}


\newcommand{\bayscore}{\text{calibrated Bayes score}}
\newcommand{\bayrisk}{\text{calibrated Bayes risk}}

\newtheorem{example}{Example}[section]
\newtheorem{exc}{Exercise}[section]
%\newtheorem{rem}{Remark}[section]

\newtheorem{theorem}{Theorem}[section]
\newtheorem{definition}{Definition}
\newtheorem{proposition}[theorem]{Proposition}
\newtheorem{corollary}[theorem]{Corollary}

\newtheorem{remark}{Remark}[section]
\newtheorem{lemma}[theorem]{Lemma}
\newtheorem{claim}[theorem]{Claim}
\newtheorem{fact}[theorem]{Fact}
\newtheorem{assumption}{Assumption}

\newcommand{\iidsim}{\overset{\mathrm{i.i.d.}}{\sim}}
\newcommand{\unifsim}{\overset{\mathrm{unif}}{\sim}}
\newcommand{\sign}{\mathrm{sign}}
\newcommand{\wbar}{\overline{w}}
\newcommand{\what}{\widehat{w}}
\newcommand{\KL}{\mathrm{KL}}
\newcommand{\Bern}{\mathrm{Bernoulli}}
\newcommand{\ihat}{\widehat{i}}
\newcommand{\Dwst}{\calD^{w_*}}
\newcommand{\fls}{\widehat{f}_{n}}


\newcommand{\brpi}{\pi^{br}}
\newcommand{\brtheta}{\theta^{br}}

% \newcommand{\M}{\mat{M}}
% \newcommand\Mmh{\mat{M}^{-1/2}}
% \newcommand{\A}{\mat{A}}
% \newcommand{\B}{\mat{B}}
% \newcommand{\C}{\mat{C}}
% \newcommand{\Et}[1][t]{\mat{E_{#1}}}
% \newcommand{\Etp}{\Et[t+1]}
% \newcommand{\Errt}[1][t]{\mat{\bigtriangleup_{#1}}}
% \newcommand\cnM{\kappa}
% \newcommand{\cn}[1]{\kappa\left(#1\right)}
% \newcommand\X{\mat{X}}
% \newcommand\fstar{f_*}
% \newcommand\Xt[1][t]{\mat{X_{#1}}}
% \newcommand\ut[1][t]{{u_{#1}}}
% \newcommand\Xtinv{\inv{\Xt}}
% \newcommand\Xtp{\mat{X_{t+1}}}
% \newcommand\Xtpinv{\inv{\left(\mat{X_{t+1}}\right)}}
% \newcommand\U{\mat{U}}
% \newcommand\UTr{\trans{\mat{U}}}
% \newcommand{\Ut}[1][t]{\mat{U_{#1}}}
% \newcommand{\Utinv}{\inv{\Ut}}
% \newcommand{\UtTr}[1][t]{\trans{\mat{U_{#1}}}}
% \newcommand\Utp{\mat{U_{t+1}}}
% \newcommand\UtpTr{\trans{\mat{U}_{t+1}}}
% \newcommand\Utptild{\mat{\widetilde{U}_{t+1}}}
% \newcommand\Us{\mat{U^*}}
% \newcommand\UsTr{\trans{\mat{U^*}}}
% \newcommand{\Sigs}{\mat{\Sigma}}
% \newcommand{\Sigsmh}{\Sigs^{-1/2}}
% \newcommand{\eye}{\mat{I}}
% \newcommand{\twonormbound}{\left(4+\DPhi{\M}{\Xt[0]}\right)\twonorm{\M}}
% \newcommand{\lamj}{\lambda_j}

% \renewcommand\u{\vect{u}}
% \newcommand\uTr{\trans{\vect{u}}}
% \renewcommand\v{\vect{v}}
% \newcommand\vTr{\trans{\vect{v}}}
% \newcommand\w{\vect{w}}
% \newcommand\wTr{\trans{\vect{w}}}
% \newcommand\wperp{\vect{w}_{\perp}}
% \newcommand\wperpTr{\trans{\vect{w}_{\perp}}}
% \newcommand\wj{\vect{w_j}}
% \newcommand\vj{\vect{v_j}}
% \newcommand\wjTr{\trans{\vect{w_j}}}
% \newcommand\vjTr{\trans{\vect{v_j}}}

% \newcommand{\DPhi}[2]{\ensuremath{D_{\Phi}\left(#1,#2\right)}}
% \newcommand\matmult{{\omega}}


\section{Baselines}
\label{appendix:baselines}

\paragraph{Gurobi \cite{gurobi}.}
We implement the mathematical model described in \Cref{appendix:notation} in the exact solver Gurobi \cite{gurobi} and provide a time budget of 600 and 3600 seconds per test instance. We run the Gurobi solver with activated multi-threading on a machine equipped with two Intel Xeon E5-2690 v4 processors, totaling 28 physical cores and 56 logical threads.

\paragraph{Greedy.}
Due to the absence of heuristics developed for the min-max MSPRP, we develop a greedy heuristic as a simple baseline. The heuristic constructs solutions sequentially by assigning each agent logits for selecting a shelf, weighted inversely by its distance from the agent's current position. Similarly, SKUs are chosen with logits proportional to the number of units an agent could potentially pick. Given the logits, the same sequential action selection as described in \Cref{alg:seq-action-selection} is used to generate actions for all agent. Being a stochastic heuristic, we use it to generate 100 different solutions for each test instance and select the best one. 


\paragraph{Hierarchical Attention Model \cite{luttmann2024neural}.}
The Hierarchical Attention Model (HAM) introduces the idea of a hierarchical decoder to generate actions over the decomposed action space of the MDP formulation of the MSPRP. Although HAM was developed to solve the min-sum MSPRP, creating $\NumTours$ one after another, it can be used to solve the min-max MSPRP as well thanks to our assumption, that there are exactly as many pickers as there are tours. In this work, HAM is trained like all other models on the min-max-based reward defined in \Cref{sec:mdp} using the learning method outlined in \Cref{sec:training}.


\paragraph{2d-Ptr \cite{liu20242d}.}
The 2D Array Pointer network (2d-Ptr) addresses the heterogeneous capacitated vehicle routing problem (HCVRP) by using a dual-encoder setup to map vehicles and customer nodes effectively. This approach facilitates dynamic, real-time decision-making for route optimization. Its decoder employs a 2D array pointer for action selection, prioritizing actions over vehicles. The 2d-Ptr can be adapted to solve the min-max MSPRP by using the 2D pointer hierarchically to select shelves and SKUs and by using pickers instead of vehicles.

\paragraph{Equity Transformer (ET) \cite{son2024equity}.}
The Equity-Transformer (ET) approach  \cite{son2024equity} addresses min-max routing problems by employing a sequential planning approach with sequence generators like the Transformer. It focuses on equitable workload distribution among multiple agents, applying this strategy to challenges like the min-max multi-agent traveling salesman and pickup and delivery problems. In our experiments, we modify the agent context in the decoder to the MSPRP setting


\paragraph{PARCO \cite{berto2024parco}.}
PARCO is a recent NCO framework for solving multi-agent CO problems. It uses a multi-pointer mechanism paired with a conflict handler to generate solutions for multiple agents in parallel. It is a versatile framework, which has been applied to different routing and scheduling problems. 






\section{Model And Training Configuration}

In the following, we detail the model and training parameters as well as the parameters for generating the training / test data. Besides that, to ensure proper reproducibility we provide all training details in our publicly available GitHub repository as configuration files. 


\subsection{Instance Generation}
\label{appendix:instance}
For training and evaluating MAHAM and the baselines described above, we use the same instance generation scheme described in \cite{luttmann2024neural}, who generate instances for three warehouse types that differ in the number of available shelves. They generate instances with 10, 25 and 40 shelves referred to as MSPRP10, MSPRP25 and MSPRP40, respectively. While the number of shelves is fixed, the number of demanded SKUs is altered for each warehouse type.

We randomly select the $|\SetOfStorageLocations|$ storage locations from all $|\SetOfSKUs| \times |\mathcal{V}^{\mathrm{R}}|$ possible SKU-shelf combinations and sample the supply from a discrete uniform distribution with mean $\Bar{n}_i$. Likewise, the demand for each SKU is sampled from a discrete uniform distribution with mean $\Bar{d}_p$. Lastly, we clip the demand of an SKU by the warehouse's total supply for it in order to ensure the feasibility of all generated instances. 
Table \ref{tab:instances} summarizes the parameters of the different instances. 


\begin{table}[h]
\renewcommand{\arraystretch}{1.1} % Increase row height for better readability
\centering
\caption{Parameter values for instance generation}
\label{tab:instances}
\setlength{\tabcolsep}{10pt} % Adjust column spacing
\resizebox{\textwidth}{!}{
\begin{tabular}{p{1.5cm}|ccc|ccc|ccc|ccc}
\toprule
                             & \multicolumn{3}{c|}{MSPRP10}    & \multicolumn{3}{c|}{MSPRP25}   & \multicolumn{3}{c|}{MSPRP40} & \multicolumn{3}{c}{MSPRP50} \\ 
\midrule
$|\mathcal{V}^\mathrm{R}|$   & 10  & 10  & 10  & 25  & 25  & 25  & 40  & 40  & 40 & 50  & 50  & 50  \\
$|\mathcal{V}^\mathrm{S}|$   & 20  & 20  & 20  & 50  & 50  & 50  & 100  & 100  & 100 & 200  & 500  & 1000  \\
$|\SetOfSKUs|$               & 3   & 6   & 9   & 12  & 15  & 18  & 15  & 20  & 30 & 100  & 250  & 500  \\ 
$\kappa$                     & 6   & 9   & 9   & 12  & 12  & 15  & 12  & 15  & 15 & 15  & 15  & 15  \\ 
\bottomrule
\end{tabular}
}
\end{table}





\subsection{Network Hyperparameters}
\label{appendix:network}
To ensure valid and meaningful experiments, the hyperparameters are identical for all models. The size of the embeddings is set to 256 and the number of heads for multi-head attention mechanisms is set to 8. All models use $L=4$ encoder layers, GELU activation functions \cite{hendrycks2016gaussian}, and Layer Normalization \cite{ba2016layernormalization}. To map the different entities of the MSPRP into embedding space, all models use the same features outlined in \Cref{tab:features}.

\begingroup
\begin{spacing}{1.0}
\begin{longtable}{p{3.5cm}p{3.9cm}p{5.5cm}}
    \caption{Key Features for Envisioned Social Media from Participant Interviews} \\
    \toprule
    \textbf{Desired Feature} & \textbf{Key Idea} & \textbf{Example Quote} \\
    \midrule
    \endfirsthead
    \toprule
    \textbf{Desired Feature} & \textbf{Key Idea} & \textbf{Example Quote} \\
    \midrule
    \endhead
    \midrule
    \multicolumn{3}{r}{\textit{Continued on the next page}} \\
    \midrule
    \endfoot
    \bottomrule
    \endlastfoot    
    \multicolumn{3}{c}{
        \parbox{13cm}{\centering 
        \textbf{Immersive Experiences Through Spatial Elements (Section~\ref{lab:4-1})} \\ 
        \tablequote{In 2D, you're just observing\ldots{} But in a 3D setting, even if you don't leave physical footprints, in a way, they're still there.}{01}
        }
    } \\
    \midrule
    {\textbf{[Section~\ref{lab:4-1-1}]}}\newline{}Bringing Realism and Presence Through 3D & Emphasized need for physical presence and embodied interactions beyond flat screens & \tablequote{You turn on your voice, chat\ldots{} walking around a physical world\ldots{} It feels like you're actually in the area of these people; versus traditional social media like Instagram, there's a separation of the screen.}{06} \\
    {\textbf{[Section~\ref{lab:4-1-2}]}}\newline{}Personal Spaces and Community-Centered Landscapes & Desired personal homes and neighborhoods that mirror real-world social structures & \tablequote{You each get your own house\ldots{} anything you want.}{03} \\
    {\textbf{[Section~\ref{lab:4-1-3}]}}\newline{}Holographic Display and Virtual Reality & Sought more natural and immersive interfaces beyond traditional devices & \tablequote{If you're already carrying around like a wand everywhere with you, why would you also want to carry a phone?}{01} \\
    \midrule
    \multicolumn{3}{c}{
        \parbox{13cm}{\centering 
        \textbf{Organic and Intentional Social Interactions (Section~\ref{lab:4-2})} \\ 
        \tablequote{The antonym to pressure, stress-free, community-oriented, wholesome, accessible, and accepting.}{09}
        }
    } \\
    \midrule
    {\textbf{[Section~\ref{lab:4-2-1}]}}\newline{}Flexible and Safe Spaces for Meaningful Conversations & Valued direct conversations in protected spaces & \tablequote{Actual conversations for real connections, individualized chatting\ldots{} to get to know people\ldots{} rather than just posting on a thread.}{22} \\
    {\textbf{[Section~\ref{lab:4-2-2}]}}\newline{}Community Gathering Places & Desired spaces for shared activities and casual encounters & \tablequote{a little card game area, shared experiences, something fun}{06} \\
    {\textbf{[Section~\ref{lab:4-2-2}] (1)}}\newline{}Activity Spaces for Shared Experiences & Valued collaborative activities and shared projects & \tablequote{Part of the fun is not only are we actually in the space, but doing something together\ldots{} we're experiencing something together, almost.}{06} \\
    {\textbf{[Section~\ref{lab:4-2-2}] (2)}}\newline{}(Public) Third Places for Casual Encounters & Desired spaces for serendipitous interactions & \tablequote{You strike a conversation because you're both picking up the same shirt\ldots{} little connections like that happen just because you're in a public space.}{03} \\
    {\textbf{[Section~\ref{lab:4-2-3}]}}\newline{}Ambient Co-Presence & Valued low-intensity connections through shared virtual spaces & \tablequote{At one point, literally, we sat for an hour in silence, listening to a whole playlist\ldots{} we were just sitting there listening, vibing the music.}{19} \\
    {\textbf{[Section~\ref{lab:4-2-4}]}}\newline{}Effortful and Intentional Navigation & Saw physical movement as a meaningful way to engage & \tablequote{Ideally, SMH would have some way that you specifically have to search for stuff\ldots{} where you have to intentionally go in looking for the account or the post versus it just being handed to you.}{05} \\
    \midrule
    \multicolumn{3}{c}{
        \parbox{13cm}{\centering 
        \textbf{Expressive and Lower-Stakes Sharing (Section~\ref{lab:4-3})} \\ 
        \tablequote{I'm into multiple things---you could have a room for Harry Potter, a room for neuroscience, and a room for another fandom or interest.}{01}
        }
    } \\
    \midrule
    {\textbf{[Section~\ref{lab:4-3-1}]}}\newline{}Expression Through Personal Area and Avatar Customization & Wanted customizable avatars and environments & \tablequote{decorat[ing] a house or customizing an avatar allows others to get to know more about my personality\ldots{} through my style.}{16} \\
    {\textbf{[Section~\ref{lab:4-3-2}]}}\newline{}Real-Time Sharing: Capturing States and Changes & Desired ways to share real-time states & \tablequote{what I'm doing at the moment. For example, if I'm at the shops, maybe I'd want to share that}{04} \\
    {\textbf{[Section~\ref{lab:4-3-3}]}}\newline{}Sharing Memories: Multisensory and Emotional Experiences & Wanted to share memories in an immersive and clearer way & \tablequote{jumping into a pool or taking the first bite of an ice cream sundae\ldots{} from your brain, pull it out, and just be able to share that}{04} \\
    \midrule
    \multicolumn{3}{c}{
        \parbox{13cm}{\centering 
        \textbf{Granular, Intuitive, and Fun Privacy Mechanisms (Section~\ref{lab:4-4})}\\ 
        \tablequote{If people are posting more like their genuine real life and their experiences, privacy would be a little bit more complicated.}{11}
        }
    } \\
    \midrule
    {\textbf{[Section~\ref{lab:4-4-1}]}}\newline{}Space-Based Privacy & Wanted privacy boundaries based on spatial metaphors & \tablequote{A bedroom is a respected personal space\ldots{} If you label it as a bedroom, it just feels more intimate.}{13} \\
    {\textbf{[Section~\ref{lab:4-4-2}]}}\newline{}Playful Privacy & Desired engaging privacy mechanisms with elements of fantasy & \tablequote{Secret knock or an object that reveals a secret passage to\ldots{} secret rooms.}{11} \\
    {\textbf{[Section~\ref{lab:4-4-3}]}}\newline{}Contextual Privacy & Wanted context-aware privacy settings & \tablequote{Each of them should have their own setting, because I feel like everybody has different levels of what they want to keep to themselves.}{14} \\
    {\textbf{[Section~\ref{lab:4-4-4}]}}\newline{}Invisibility & Sought robust blocking and visibility control & \tablequote{I don't wanna be able to tell that the person I blocked is even around me.}{22} \\
    {\textbf{[Section~\ref{lab:4-4-5}]}}\newline{}Age-Appropriate Spaces & Emphasized need for age separation & \tablequote{People who are 50 should not be engaging with people who are 13.}{21} \\
    \midrule
    \multicolumn{3}{c}{
        \parbox{13cm}{\centering 
        \textbf{Refocusing Interaction Priorities for Meaningful Engagement (Section~\ref{lab:4-5})} \\ 
        \tablequote{We need old Instagram back in 2010, when my mom was posting me and my brother just for the family and no one else. We didn't need the explore page.}{19}
        }
    } \\
    \midrule
    {\textbf{[Section~\ref{lab:4-5-1}]}}\newline{}Intentional Content Consumption & Desired more control over content exposure & \tablequote{You press which floor you want to go to, and then, when you ride that elevator, it just goes straight to that floor.}{09} \\
    {\textbf{[Section~\ref{lab:4-5-2}]}}\newline{}Reduced Celebrity and Commercial Influence & Wanted focus on real friends over celebrities & \tablequote{If I'm just gonna open the app for like 5 seconds\ldots{} I'm gonna wanna see my best friend's post, not some girl I talked to once 3 years ago.}{07} \\
    {\textbf{[Section~\ref{lab:4-5-3}]}}\newline{}Spatial Representation of Relationships & Sought physical distance to represent emotional closeness & \tablequote{The top 7 people you're closest to would be the 7 houses in your cul-de-sac area.}{15} \\
    {\textbf{[Section~\ref{lab:4-5-4}]}}\newline{}Intentional and Meaningful Use of Time & Valued active engagement over passive consumption & \tablequote{Anytime not scrolling and instead talking to people feels worthwhile, even if it's just light conversation.}{06} \\
\end{longtable}
\end{spacing}
\endgroup
\label{tab:results-overview}

\subsection{Training Hyperparameters}
\label{appendix:training}
We train all models using the self-improvement method described by \cite{pirnay2024selfimprovement}. To ensure consistency, we use identical hyperparameters and training environments for all neural baselines described in \Cref{appendix:baselines}. All models are trained on a single NVIDIA A100 GPU with 40GB of VRAM. Training spans 50 epochs, with each epoch generating $N=5,000$ independent instances. For each instance, $\alpha=100$ candidate solutions are sampled from the reference-policy $\pi_{\text{best}}$, and the best solution is added to the training dataset.
After all instances are solved by the reference policy $\pi_{\text{best}}$, we draw training samples in mini-batches of $B=2.000$ and determine the cross-entropy loss for the pseudo-optimal actions with respect to the target-policy $\pi$. Adam optimizer with a learning rate of 0.0001 is used to update the parameters of the target-policy, and the trainer class from the RL4CO \cite{berto2023rl4co} library is used to guide the learning process.  

The validation dataset consists of 10,000 independently generated instances per epoch. If the target policy outperforms the reference policy on the validation set, the reference policy is updated, and the training dataset is reset. \Cref{algo:learning} provides a detailed breakdown of these steps.
% Define a light gray color for comments
\definecolor{lightgray}{gray}{0.5}

\begin{algorithm}[h]
   \caption{Self-improvement training for neural CO}
   \label{algo:learning}
   \begin{algorithmic}[1]
      \REQUIRE $\mathcal{X}$: distribution over problem instances; $f_*$: objective function
      \REQUIRE $N$: number of instances to sample in each epoch
      \REQUIRE $\alpha$: number of sequences to sample for each instance
      \REQUIRE $\textsc{Validation} \sim \mathcal{X}$: validation dataset
      \STATE Randomly initialize policy $\pi_\theta$
      \STATE $\pi_{\text{best}} \gets \pi_\theta$
      \STATE $\textsc{Dataset} \gets \emptyset$
      \FOR{epoch}
          \STATE Sample set of $n$ problem instances $\textsc{Instances} \sim \mathcal{X}$
          \FOR{each $x \in \textsc{Instances}$}
              
              \STATE \textcolor{lightgray}{// Sample set of $m$ feasible solutions}
              \STATE $A := \{\bm a_{1:T}^{(1)}, \dots, \bm a_{1:T}^{(m)}\} \sim \pi_{\text{best}}$
              \STATE \textcolor{lightgray}{// Add best solution to training dataset}
              \STATE $\textsc{Dataset} \gets \textsc{Dataset} \cup \{(x, \arg\max_{\bm a_{1:T} \in A} f_x(\bm a_{1:T}))\}$ 
          \ENDFOR
          \FOR{batch}
              \STATE \textcolor{lightgray}{// Sample $B$ instances and partial solutions from \textsc{Dataset}}
              \STATE $\{(x_j, \bm{a}^{(j)}_{1:d_j})\}_{j=1}^B \sim \textsc{Dataset}, \qquad \{d_j\}_{j=1}^B \sim \mathcal{U}(1, T-1)$ 
              \STATE \textcolor{lightgray}{// Minimize batch-wise cross entropy loss}
              \STATE $\mathcal{L}_\theta = - \frac{1}{B}\sum_{j=1}^B \log \pi_\theta \left( a^{(j)}_{d_{j+1}} | \bm a^{(j)}_{1:d_j} \right)$
          \ENDFOR
          \IF{greedy performance of $\pi_\theta$ on \textsc{Validation} better than $\pi_{\text{best}}$}
              \STATE \textcolor{lightgray}{// update best policy}
              \STATE $\pi_{\text{best}} \gets \pi_\theta$ 
              \STATE \textcolor{lightgray}{// Empty Training Dataset}
              \STATE $\textsc{Dataset} \gets \emptyset$ 
          \ENDIF
      \ENDFOR
   \end{algorithmic}
\end{algorithm}


