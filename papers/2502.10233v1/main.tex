% This is samplepaper.tex, a sample chapter demonstrating the
% LLNCS macro package for Springer Computer Science proceedings;
% Version 2.21 of 2022/01/12
%
\documentclass[runningheads]{llncs}
%
\usepackage[T1]{fontenc}
% T1 fonts will be used to generate the final print and online PDFs,
% so please use T1 fonts in your manuscript whenever possible.
% Other font encondings may result in incorrect characters.
%
\usepackage{wrapfig}
\usepackage{graphicx}
\usepackage{amsmath}
\usepackage{amssymb}
\usepackage{tabularx}
\usepackage{booktabs}
\usepackage{adjustbox}
\usepackage{xcolor}
\usepackage{bm}

\usepackage{mathtools}
\usepackage{esvect}
\usepackage{longtable}
\usepackage{multirow}
\usepackage{array}
\usepackage{rotating}
\usepackage{subcaption} % For subfigures
\usepackage{threeparttable}
\usepackage{calc}
\usepackage{algorithm}
\usepackage{algorithmic}
\usepackage{setspace}  % Package to control line spacing
\usepackage{bbm}
% for tables
\usepackage{colortbl}
\aboverulesep=0pt
\belowrulesep=0.20pt
% Used for displaying a sample figure. If possible, figure files should
% be included in EPS format.
%
% If you use the hyperref package, please uncomment the following two lines
% to display URLs in blue roman font according to Springer's eBook style:

\usepackage[hidelinks]{hyperref}
\usepackage{color}
\renewcommand\UrlFont{\color{blue}\rmfamily}
%\urlstyle{rm}
%
\usepackage{cleveref}

\newcommand{\SetOfLocationsIncludePickItem}[1]{\mathcal{V}_{\!#1}^\text{S}}
\newcommand{\SetOfSKUs}{\mathcal{P}}
\newcommand{\SetOfAllNodes}%{\mathcal{V}^{\text{L}}}
{\mathcal{V}}
\newcommand{\SetOfPackingStations}{\mathcal{V}^{\text{D}}}
\newcommand{\SetOfStorageLocations}{\mathcal{V}^{\text{S}}}
\newcommand{\NumTours}{\mathcal{B}}
\newcommand{\SetOfShelves}{\mathcal{V}^{\text{R}}}
\newcommand{\SetOfAgents}{\mathcal{M}}

\newcommand{\logits}{L}
\newcommand{\curragent}[1]{{\Omega_{#1}}}
\newcommand{\PermOverL}{{\Omega(\logits)}}


\newcommand{\finalNodeEmb}[1]{\bm{h}^{L}_{#1}}
\newcommand{\shelfPolicy}{g^{\mathcal{V}}_{\theta}}
\newcommand{\SKUPolicy}{g^{\mathcal{P}}_{\theta}}
\newcommand{\subActionSpace}{\mathcal{A}_d}
\newcommand{\numagents}{{M}}
\newcommand{\agentidx}{{m}}

\newcommand{\embdim}{{D}}


\begin{document}
\vspace{-2cm}
\title{Learning to Solve the Min-Max \\ Mixed-Shelves Picker-Routing Problem \\ via Hierarchical and Parallel Decoding}
\authorrunning{Luttmann, Xie}
\titlerunning{Hierarchical and Parallel Decoding for Picker-Routing}
% If the paper title is too long for the running head, you can set
% an abbreviated paper title here
%

%
\author{Laurin Luttmann\inst{1} \and
Lin Xie\inst{2}}
%

% First names are abbreviated in the running head.
% If there are more than two authors, 'et al.' is used.
%
\institute{Leuphana University, Lüneburg, Germany \and
Brandenburg University of Technology, Cottbus, Germany}
%

\maketitle              % typeset the header of the contribution
%
\begin{abstract}
The Mixed-Shelves Picker Routing Problem (MSPRP) is a fundamental challenge in warehouse logistics, where pickers must navigate a mixed-shelves environment to retrieve SKUs efficiently. Traditional heuristics and optimization-based approaches struggle with scalability, while recent machine learning methods often rely on sequential decision-making, leading to high solution latency and suboptimal agent coordination. In this work, we propose a novel hierarchical and parallel decoding approach for solving the min-max variant of the MSPRP via multi-agent reinforcement learning. While our approach generates a joint distribution over agent actions, allowing for fast decoding and effective picker coordination, our method introduces a sequential action selection to avoid conflicts in the multi-dimensional action space. Experiments show state-of-the-art performance in both solution quality and inference speed, particularly for large-scale and out-of-distribution instances. Our code is publicly available at \url{http://github.com/LTluttmann/marl4msprp}


\keywords{Picker Routing \and Mixed-Shelves Warehouses \and Neural Combinatorial Optimization \and Multi-Agent Reinforcement Learning}
\end{abstract}
%
%
%
\section{Introduction}
Order picking, the process of retrieving items from a warehouse to fulfill customer orders, is one of the most labor-intensive and time-consuming operations in warehouse logistics, accounting for up to 65\% of total operating costs \cite{de2007design}. In conventional picker-to-parts warehouses, most of a picker's time is spent traveling between the shelves of the storage area \cite{Tompkins2010}. To reduce travel time, mixed-shelves storage strategies have gained traction in recent years (see \cite{Boysen2017}, \cite{weidingerPickerRoutingMixedshelves2019}, \cite{xie2021introducing}, \cite{xie2023formulating}, and \cite{luttmann2024neural}). Unlike traditional warehouse layouts that allocate a single storage position per Stock-Keeping Unit (SKU), mixed-shelves storage distributes SKUs to multiple shelves of the storage area, potentially decreasing travel distances and improving overall efficiency.

This mixed-shelves approach gives rise to the Mixed-Shelves Picker Routing Problem (MSPRP), which focuses on determining optimal routes for pickers while considering the unique constraints of mixed-shelves warehouses and operations. Despite its practical significance, research on solving the MSPRP remains limited. Existing approaches primarily rely on classical heuristics such as variable neighborhood search \cite{xie2023formulating} and Tabu search \cite{danielsModelWarehouseOrder1998}. While these methods can produce high-quality solutions, they are computationally expensive which makes them impractical for large-scale or real-time applications.
Neural combinatorial optimization (NCO) has emerged as a promising alternative, offering faster solution generation while maintaining high solution quality across various routing problems. However, current NCO applications to the MSPRP are limited to single-picker scenarios that focus on minimizing total travel distance. This represents a significant gap in addressing real-world warehouse operations, where multiple pickers typically work simultaneously and minimizing the longest tour (i.e., the route of the most time-consuming picker) is more critical for maintaining efficient operations. To bridge this gap, we propose a novel NCO approach that integrates hierarchical and parallel decoding to efficiently solve the min-max variant of the MSPRP. Our main contributions are as follows:

\begin{itemize}
    \item We formulate the MSPRP as a cooperative multi-agent problem, aiming to balance workloads among pickers rather than minimizing the total distance.
    \item A Hierarchical and Parallel Decoding framework enables efficient picker coordination in complex multi-dimensional action spaces.
    \item A Sequential Action Selection strategy supports the parallel decoding step by avoiding conflicts while exhibiting strong generalization performance    
    \item We demonstrate state-of-the-art performance in terms of both solution quality and computational efficiency, particularly for large problem instances.
\end{itemize}

\section{Related Work}
\label{subsec:relwork}

\section{Related Works}
\label{sec:related_works}


\noindent\textbf{Diffusion-based Video Generation. }
The advancement of diffusion models \cite{rombach2022high, ramesh2022hierarchical, zheng2022entropy} has led to significant progress in video generation. Due to the scarcity of high-quality video-text datasets \cite{Blattmann2023, Blattmann2023a}, researchers have adapted existing text-to-image (T2I) models to facilitate text-to-video (T2V) generation. Notable examples include AnimateDiff \cite{Guo2023}, Align your Latents \cite{Blattmann2023a}, PYoCo \cite{ge2023preserve}, and Emu Video \cite{girdhar2023emu}. Further advancements, such as LVDM \cite{he2022latent}, VideoCrafter \cite{chen2023videocrafter1, chen2024videocrafter2}, ModelScope \cite{wang2023modelscope}, LAVIE \cite{wang2023lavie}, and VideoFactory \cite{wang2024videofactory}, have refined these approaches by fine-tuning both spatial and temporal blocks, leveraging T2I models for initialization to improve video quality.
Recently, Sora \cite{brooks2024video} and CogVideoX \cite{yang2024cogvideox} enhance video generation by introducing Transformer-based diffusion backbones \cite{Peebles2023, Ma2024, Yu2024} and utilizing 3D-VAE, unlocking the potential for realistic world simulators. Additionally, SVD \cite{Blattmann2023}, SEINE \cite{chen2023seine}, PixelDance \cite{zeng2024make} and PIA \cite{zhang2024pia} have made significant strides in image-to-video generation, achieving notable improvements in quality and flexibility.
Further, I2VGen-XL \cite{zhang2023i2vgen}, DynamicCrafter \cite{Xing2023}, and Moonshot \cite{zhang2024moonshot} incorporate additional cross-attention layers to strengthen conditional signals during generation.



\noindent\textbf{Controllable Generation.}
Controllable generation has become a central focus in both image \citep{Zhang2023,jiang2024survey, Mou2024, Zheng2023, peng2024controlnext, ye2023ip, wu2024spherediffusion, song2024moma, wu2024ifadapter} and video \citep{gong2024atomovideo, zhang2024moonshot, guo2025sparsectrl, jiang2024videobooth} generation, enabling users to direct the output through various types of control. A wide range of controllable inputs has been explored, including text descriptions, pose \citep{ma2024follow,wang2023disco,hu2024animate,xu2024magicanimate}, audio \citep{tang2023anytoany,tian2024emo,he2024co}, identity representations \citep{chefer2024still,wang2024customvideo,wu2024customcrafter}, trajectory \citep{yin2023dragnuwa,chen2024motion,li2024generative,wu2024motionbooth, namekata2024sg}.


\noindent\textbf{Text-based Camera Control.}
Text-based camera control methods use natural language descriptions to guide camera motion in video generation. AnimateDiff \cite{Guo2023} and SVD \cite{Blattmann2023} fine-tune LoRAs \cite{hu2021lora} for specific camera movements based on text input. 
Image conductor\cite{li2024image} proposed to separate different camera and object motions through camera LoRA weight and object LoRA weight to achieve more precise motion control.
In contrast, MotionMaster \cite{hu2024motionmaster} and Peekaboo \cite{jain2024peekaboo} offer training-free approaches for generating coarse-grained camera motions, though with limited precision. VideoComposer \cite{wang2024videocomposer} adjusts pixel-level motion vectors to provide finer control, but challenges remain in achieving precise camera control.

\noindent\textbf{Trajectory-based Camera Control.}
MotionCtrl \cite{Wang2024Motionctrl}, CameraCtrl \cite{He2024Cameractrl}, and Direct-a-Video \cite{yang2024direct} use camera pose as input to enhance control, while CVD \cite{kuang2024collaborative} extends CameraCtrl for multi-view generation, though still limited by motion complexity. To improve geometric consistency, Pose-guided diffusion \cite{tseng2023consistent}, CamCo \cite{Xu2024}, and CamI2V \cite{zheng2024cami2v} apply epipolar constraints for consistent viewpoints. VD3D \cite{bahmani2024vd3d} introduces a ControlNet\cite{Zhang2023}-like conditioning mechanism with spatiotemporal camera embeddings, enabling more precise control.
CamTrol \cite{hou2024training} offers a training-free approach that renders static point clouds into multi-view frames for video generation. Cavia \cite{xu2024cavia} introduces view-integrated attention mechanisms to improve viewpoint and temporal consistency, while I2VControl-Camera \cite{feng2024i2vcontrol} refines camera movement by employing point trajectories in the camera coordinate system. Despite these advancements, challenges in maintaining camera control and scene-scale consistency remain, which our method seeks to address. It is noted that 4Dim~\cite{watson2024controlling} introduces absolute scale but in  4D novel view synthesis (NVS) of scenes.




\section{Problem Formulation}
\label{sec:mathmod}


This work focuses on a min-max variant of the MSPRP with split orders and split deliveries covered in \cite{luttmann2024neural}. The split orders assumption allows items of an order to be picked
within different tours and split deliveries relaxes the assumption that the demand
for an SKU must be satisfied by a single picker tour \cite{xie2021introducing}. A tour is defined by the storage locations visited between two successive visits to a packing station $h \in \mathcal{V}^{\mathrm{D}}$, where picked items are unloaded and commissioned. During a tour, no more than $\kappa$ units can be picked. Further, due to the mixed-shelves storage policy each shelf may consist of multiple storage locations or compartments storing units of different SKUs. Also, the mixed-shelves storage policy allows each SKU $p$ to be retrieved from multiple storage locations $i \in \SetOfLocationsIncludePickItem{p}$. 

The goal of the min-max MSPRP is to pick all $d_p$ demanded units of all requested SKUs $p \in \SetOfSKUs$ and returning them to a packing station $h \in \SetOfPackingStations$ while minimizing the maximum travel distance among the individual pickers $m = 1,...,M$, henceforth also called agents. 
Note that in order to compare our proposed method against baselines \cite{luttmann2024neural}, \cite{liu20242d}, and \cite{son2024equity}, we assume that the number of agents is equal to the number of tours required to collect all demanded items given the picker capacity $\kappa$ (i.e. $\numagents = \left \lceil \frac{\sum_p d_p}{\kappa} \right \rceil$). We provide the mathematical model for the min-max MSPRP in \Cref{appendix:notation}.



\subsection{Markov Decision Process Formulation}
\label{sec:mdp}
\newcommand{\setOfNodesMDP}{\mathcal{V}}
The min-max MSPRP can be modeled as a cooperative Multi-Agent Markov Decision Process (MMDP) with $\numagents$ agents sharing a common reward. An MMDP is defined as $(\mathcal{S}, \SetOfAgents, \{\mathcal{A}_i\}, \Gamma, R)$, where $\mathcal{S}$ and $\SetOfAgents$ are finite sets of states and agents, respectively. Each agent $m$ selects actions from $\mathcal{A}_m$, with the joint action space denoted as $\bm{\mathcal{A}}$. The transition function $\Gamma$ determines state changes based on actions, and $R$ is the shared reward function.  

MMDPs involve sequential decision-making, where agents select and execute actions at each step until a terminal state $s_T$ is reached. The min-max MSPRP is framed as an MMDP, with pickers (agents) visiting warehouse shelves to fulfill SKU demands. A shared $\theta$-parameterized policy determines the next location and SKU to pick. This chapter formally defines the min-max MSPRP as an MMDP, specifying its state, action space, transition rule, and reward function.


\vspace{-4.5mm}
\subsubsection{State.} 
The state $s_t$ of the min-max MSPRP at step $t$ can be represented as a heterogeneous graph $\mathcal{G}=(\setOfNodesMDP,\SetOfSKUs, \SetOfAgents, E_{t})$ with pickers, warehouse locations and SKUs posing different types of nodes in the graph. 
The set of warehouse locations $\setOfNodesMDP$ is the collection of all packing stations $\SetOfPackingStations$ and shelves $\mathcal{V}^\text{R}$. 
%Locations can be represented by the Cartesian coordinates $\mathbf{x}^{\mathcal{V}}_i \in \mathbb{R}^2$. 
The state of an SKU $p \in \SetOfSKUs$ is defined by its remaining demand $d_{pt}$ at step $t$. 
Moreover, edges with weights $E_t$ connect shelf and SKU nodes, specifying the storage quantity $e_{vpt}$ of an item $p$ in the respective shelf $v$ at time $t$. 
Lastly, the state of the pickers $m \in \SetOfAgents$ is defined by their current location $v^m_{t}$, remaining capacity $\kappa^m_{t}$ and the length of their current tour $\tau^m_{1:t}=(v_1^m,\ldots,v_t^m)$, denoted as $dist(\tau^m_{1:t})$.
\vspace{-4.5mm}
\subsubsection{Action.} 
A single agent action $a^m_{t}$ is a tuple $(v,p)$ specifying the next shelf to visit as well as the SKU to pick for agent $m$. Given $s_t$, visiting shelf $v$ is a feasible action if it stores items of at least one SKU currently in demand. Furthermore, given the picking location $v$, the picker may only select an SKU for picking that is both still in demand and available in the current shelf. Note, that the quantity of picked items will be determined heuristically by the transition function $\Gamma$ in order to decrease the complexity of the action space and facilitate policy learning. 

The packing station can always be visited by a picker to unload picked items and thus to restore the capacity. When a picker's capacity is exhausted, visiting the packing station is the only possible action. Moreover, to facilitate agent coordination, a picker may always choose to stay at its current location in order to give other pickers precedence. This way, a hesitant picker may wait and evaluate what the other pickers are doing, before making the next move. 
\vspace{-4.5mm}
\subsubsection{Transition.} 
Given the joint actions $\bm{a}_t = (a_t^1,\ldots,a_t^m)$ of all agents, the transition function $\Gamma(s_t, \bm{a}_t)$  deterministically transits to $s_{t+1}$. The new state consists of the updated agent locations $\bm{v}_{t} = (v^1_{t}, \ldots, v^\numagents_{t})$ and agent tours $\tau^m_{1:t} = \tau^m_{1:t-1} \cup{\{ v^m_{t} \}}$.
To update the remaining demand, supply and picker capacities, the pick quantity $y_t^m$ must be determined. Given the pick locations, SKUs and a permutation $\Omega$ over pickers, we iteratively determine the pick quantity as the minimum of the remaining demand of the selected SKU $p$, the storage quantity at the agent's new location $v$ as well as the agent's remaining capacity:
%
\begin{align}
    y^\curragent{k}_t = \text{min} (\kappa_t^\curragent{k}, \, d_{pt} - \sum_{j=1}^{k-1} y_t^\curragent{j} \cdot \mathbbm{1}_{p^{\curragent{j}}_t = p}, \, e_{vpt} ),
\end{align}
%
where no two agents may select the same shelf-SKU combination in the same decoding step, for which reason the supply $e_{vpt}$ will not be altered by preceding agents. 
%If the above mechanism determines that an agent cannot pick any item from the chosen SKU because its predecessors already retrieved all demanded units, we leave the respective agent in its current position without a picking job, similar to the fallback action defined in \cite{berto2024parco}. 
%Note that permutation $\Omega$ can have a significant influence on the solution quality. We provide our model with a mechanism to learn this permutation, effectively enabling it to generate high quality solutions.
Given the pick quantities $y_t^m$, the transition function updates the demand $d_{pt+1} = d_{pt} - \sum_{m=1}^\numagents y^m_{t} \cdot \mathbbm{1}_{p^m_t = p}$, the supply $e_{vpt+1} = e_{vpt} - \sum_{m=1}^\numagents y^m_{t} \cdot \mathbbm{1}_{v^m_{t} = v; p^m_t = p}$ and the remaining picker capacity $\kappa^m_{t+1} = \kappa^m_{t} - y^m_{t}$. 
%NOTE dont need this condition in this problem since we assume the existance of one agent per tour
%When an agent chooses to return to the packing station, its capacity will be restored to the initial payload $\kappa$.
The problem instance $x$ is solved once the demand for every SKU is met and all pickers have returned to the packing station they were starting from. A feasible solution to $x$, reaching the terminal state $s_T$ in $T$ construction steps, will be denoted as $\bm{a} \coloneqq (\bm{a}_1, \ldots, \bm{a}_{T})$. 
%We use $\tau^m \coloneqq \tau^m_{1:T}$ as a shorthand to denote the full tour of an agent and $\bm{\tau}=(\tau^1, \ldots, \tau^\numagents)$ as a feasible solution.
\vspace{-4.5mm}
\subsubsection{Reward.}
The MMDP formulation of the min-max MSPRP has a sparse reward function, which is only defined for a complete solution $\bm{a}$. We define the reward $R(\bm{a}, x)$ as the negative of the maximum travel distance of any picker, i.e. $R(\bm{a},x) = - \text{max}_{m \in \SetOfAgents} \, dist(\tau^m_{1:T})$, and the goal of our approach is to maximize it.



\section{Method}

% TODOS:

% \textcolor{red}{
% \begin{itemize}
%     \item check notation $\bm{a}$ and $\tau$ (Tau is route, a is solution / all actions)
%     \item describe ranking-based positional encoding 
%     \item remove mha formulas from agent encoder
%     \item perform permutation test w/o communication layer
% \end{itemize}
% }
This section introduces our Multi-Agent Hierarchical Attention Model (MAHAM) -- an extension of the Hierarchical Attention Model (HAM) architecture \cite{luttmann2024neural} -- designed to address the multi-picker min-max variant of the MSPRP. In NCO, the sequential nature of the MDP underlying the CO problem often leads to the adoption of autoregressive (AR) models, which implement a sequential solution generation via an encoder-decoder network, formally represented as:\footnote{henceforth, we use the current problem state $s_t$ instead of the problem instance $x$ and the previous actions $\bm{a}_{1:t-1}$ to condition the models}
\begin{align}
\label{eq:autoregressive_encoding_decoding}
p_\theta(\bm{a}|x) &= \prod_{t=1}^{T} g_\theta(a_{t} | x, \bm{a}_{1:t-1}, H_t) \cdot f_\theta(H_t | x, \bm{a}_{1:t-1})
\end{align}
where $f_\theta$ represents the encoder network, used to construct a hidden representation of the problem instance $x$ given the actions taken so far and $g_\theta$ the decoder, that selects actions based on the problem encoding and its current state. 

MAHAM follows this approach, however the presence of multiple agents and a composite action space $\mathcal{A} \equiv \mathcal{V} \times \mathcal{P}$ introduce special needs which we carefully address with our architecture in \Cref{fig:maham}. While existing approaches tackle multi-agent problems by sequentially generating solutions for one agent after another \cite{son2024equity} or using a separate decoder $\pi_\theta^m$ per agent \cite{zong2022mapdp}, MAHAM poses a shared policy which constructs multiple picker routes in parallel through 1.) a separate agent encoder and 2.) a parallel decoding with sequential action selection scheme.  
% \begin{align}


\subsection{Encoder}


\subsubsection{Problem Encoder.}
\label{sec:encoder}



As defined earlier, the min-max MSPRP can be represented as a heterogeneous graph with agents, packing stations, shelves and SKUs posing different node types. We follow \cite{luttmann2024neural} and first project these different node-types from their distinct feature spaces into a mutual embedding space of dimensionality $\embdim$ using type-specific transformations $W_{\phi_i}$ for node $i$ of type $\phi_i$. The features used to represent agents, stations, shelves and SKUs in the features space are listed in \Cref{tab:features} in \Cref{appendix:network}.

\begin{figure*}[t]
    \centering
    \includegraphics[width=\textwidth]{figures/maham.drawio.pdf}
    \vspace{-5mm}
    \caption{Overview of the MAHAM Architecture}
    \vspace{-5mm}
    \label{fig:maham}
\end{figure*}


Also similar to \cite{luttmann2024neural}, we use several layers of self- and cross-attention between location and SKU nodes. To this end, we treat packing stations as shelves that store zero units for each SKU and concatenate their initial embeddings to those of the shelf nodes, yielding $H^{0}_\mathcal{V} = [H^{0}_{\SetOfPackingStations} || H^{0}_{\SetOfShelves}]$. Likewise, the initial SKU embeddings are denoted as $H^{0}_\mathcal{P}$. While self-attention is applied independently to shelf and SKU embeddings following the Transformer architecture \cite{vaswaniAttentionAllYou2017a}, cross-attention allows shelves and SKUs to influence each other’s embeddings. Consequently, shelf embeddings encode information about the SKUs they store, and SKU embeddings reflect their placement within the storage area -- an essential property for hierarchical action selection. To perform cross-attention we compute a single matrix of attention scores $A$ using shelf embeddings as queries $Q$ and SKU embeddings as keys $K$. This contrasts with the MatNet \cite{kwonMatrixEncodingNetworks2021a} and HAM \cite{luttmann2024neural} architectures, which compute separate attention scores for each node type—once as queries and once as keys. Formally we perform:
\begin{align}
    \label{eq:matnet_dot_score}
    A &= \frac{QK^\top}{\sqrt{d_k}}, \qquad Q = W^Q H_{\mathcal{V}}^{l-1}, \qquad  K = W^K H_{\mathcal{P}}^{l-1}
\end{align}
where $W^Q$ and $W^K \in \mathbb{R}^{d_k \times \embdim}$ are weight matrices learned per attention head\footnote{For succinctness, we omit the layer and head enumeration} and $d_k$ is the per-head embedding dimension. The resulting attention scores $A\in \mathbb{R}^{|\mathcal{V}| \times |\mathcal{P}|}$ can be interpreted as the (learned) influence of an SKU $p$ on the embedding of location $v$. 
Similar to MatNet \cite{kwonMatrixEncodingNetworks2021a} we fuse these learned attention scores with the supply-matrix $E \in \mathbb{R}^{|\mathcal{V}| \times |\mathcal{P}|}$, which specifies how many units of SKU $p$ are stored in location $v$. To this end, we concatenate the attention score and the matrix of storage quantities and feed the resulting score vector through a multi-layer perceptron $\text{MLP}: \mathbb{R}^{|\mathcal{V}| \times |\mathcal{P}| \times 2} \rightarrow \mathbb{R}^{|\mathcal{V}| \times |\mathcal{P}|}$, with a single hidden layer comprising of $\embdim$ units and GELU activation function \cite{hendrycks2016gaussian}. Further, we pass the transpose of the attention scores and of the supply matrix $A^\top, \, E^\top \in \mathbb{R}^{|\mathcal{P}| \times |\mathcal{V}|}$ through a second MLP to obtain the influence $A_{\mathcal{P} \rightarrow \mathcal{V}}$ of locations $v$ on the SKU embeddings $H_{\mathcal{P}}$:
%
\begin{align}
    \label{eq:matnet_mixed_score}
     A_{\mathcal{V} \rightarrow \mathcal{P}} = \mathrm{MLP}_{\mathcal{V}} \big ([A || E ] \big ), \quad  A_{\mathcal{P} \rightarrow \mathcal{V}} = \mathrm{MLP}_{\mathcal{P}} \big ([A^\top || E^\top ] \big ),
\end{align}
%
By avoiding to compute the (computationally expensive) attention scores twice, once to generate shelf embeddings and once for the SKU embeddings, our implementation of the cross-attention mechanism leverages parameter sharing, improving both efficiency and generalization performance, as demonstrated in \Cref{sec:exp}. 
The resulting attention scores are then used to compute the embeddings for the nodes of the respective type: 
%
\begin{align}
     H_{\mathcal{V}}^\prime &= \text{softmax}(A_{\mathcal{V} \rightarrow \mathcal{P}})V_\mathcal{P}, \quad V_\mathcal{P} = W^{V}_\mathcal{P} H_{\mathcal{P}}^{l-1} \\
     H_{\mathcal{P}}^\prime &= \text{softmax}(A_{\mathcal{P} \rightarrow \mathcal{V}})V_\mathcal{V}, \quad V_\mathcal{V} = W^{V}_\mathcal{V} H_{\mathcal{V}}^{l-1}
\end{align}
%
As in \cite{vaswaniAttentionAllYou2017a}, $H_{\mathcal{V}}^\prime$ and $H_{\mathcal{P}}^\prime$  are then augmented through skip connections, layer normalization, and a feed-forward network, yielding the location and SKU embeddings $H_{\mathcal{V}}^l$ and $H_{\mathcal{P}}^l$, respectively, of the current layer $l$. 
%
% \begin{figure*}[!htb]
%     \centering
%     \includegraphics[width=0.8\textwidth]{figures/cntxt_encoder.drawio.pdf}
%     \vspace{-4mm}
%     \caption{Agent Context Encoder}
%     \label{fig:cntxt}
% \end{figure*}



\setlength{\intextsep}{10pt}%
\begin{wrapfigure}{r}{0.6\textwidth}
  \begin{center}
    \vspace{-20pt}
    \includegraphics[width=0.6\textwidth]{figures/cntxt_encoder.drawio.pdf}
    \vspace{-35pt}
  \end{center}
  \caption{Agent Context Encoder}
  \label{fig:cntxt}
\end{wrapfigure}

\subsubsection{Agent Encoder.}
To account for multiple agents, we introduce an Agent Context Encoder, as illustrated in \Cref{fig:cntxt}, into our MAHAM architecture. This encoder leverages the embeddings $H_\mathcal{V}$ and $H_\mathcal{P}$ from the problem encoder, along with the current state $s_t$, to generate embeddings for each picker.  
To facilitate informed decision-making at each decoding step, the agent embeddings incorporate three key types of information. First, spatial information of pickers is captured by using the embedding of a picker's current location. 
Further, the remaining capacity and the length of an agent's current tour are included in the agent encoder, helping the model to determine whether to continue the tour or send the picker to a packing station.
Lastly, the total demand across all SKUs and the average-pooled SKU embeddings provide insights into the remaining workload. 
Each context feature is first projected into a shared embedding space of dimensionality $\embdim$. The resulting embeddings are then concatenated and passed through an MLP, ensuring that the final representation is mapped back to the original embedding space.

Since coordination between pickers is critical in the min-max MSPRP, we add a Multi-Head-Self-Attention (MHSA) layer \cite{vaswaniAttentionAllYou2017a} at the end of the Agent Context Encoder, which enables message passing between agents. As in \cite{vaswaniAttentionAllYou2017a}, we add a positional encoding to the agent embeddings before they enter the MHSA layer. However, given the absence of a natural ordering of pickers, we employ a Ranking-based Position Encoding, where pickers are ranked in descending order of their remaining capacity. This allows the encoder to better prioritize agents based on their current workload, which is crucial for the sequential action selection that will be described in \Cref{sec:seq_action_selection}. We denote the final agent embeddings as $H_\SetOfAgents$ and the set of all embeddings as $H=(H_\mathcal{V},H_\mathcal{P},H_\SetOfAgents)$.




\subsection{Parallel and Hierarchical Decoder}

\chapter{Learning Approaches and Architectures for Neural Decoding} % Neural decoder
\chaptermark{Neural decoders}
%\thispagestyle{empty}
\label{sec:decoder}

Physical layer design encompasses the tasks
of designing the channel coder, the modulator and the
associated algorithms at the receiver side, which includes
synchronization, channel estimation, detection/equalization,
interference/noise mitigation, channel decoding.

In this chapter, we consider the detection/decoding problem, where we aim at developing an optimal neural architecture.
The definition of the optimal criterion is a fundamental step. We propose to use the instantaneous mutual information (MI) of the channel input-output signal pair, which yields to the minimization of the a-posteriori information of the transmitted codeword given the communication channel output observation. 
Since the computation of the a-posteriori information is a formidable task, and for the majority of channels it is unknown, we propose a novel neural estimator based on a discriminative formulation. This leads to the derivation of the mutual information neural decoder (MIND). The developed neural architecture is capable not only to solve the decoding problem in unknown channels, but also to return an estimate of the average MI achieved with the coding scheme, as well as the decoding error probability. Several numerical results are reported and compared with maximum a-posteriori and maximum likelihood decoding strategies. 

The results presented in this chapter are documented in \cite{tonello2022mind}.

\section{Introduction}
\label{sec:mind_introduction}
In digital communication systems, data detection and decoding are fundamental tasks implemented at the receiver. The maximum a-posteriori (MAP) decoding approach is the optimal one to minimize the symbol or sequence error probability \cite{Proakis2001,Bahl1974}. If the channel model is known, the MAP decoder can be realized according to known algorithms, among which the BCJR algorithm \cite{Bahl1974}. For instance, for a linear time invariant (LTI) channel with additive Gaussian noise, the decoder comprises a first stage where the channel impulse response is estimated. Then, a max-log-MAP sequence estimator algorithm is implemented. The decoding metric is essentially related to the Euclidean distance between the received and hypothesized transmitted message filtered by the channel impulse response \cite{Proakis2001}. The task becomes more intriguing for channels that cannot be easily modeled or that are even unknown and for which only data samples are available. In such a case, a learning strategy turns out to be an attractive solution and it can be enabled by recent advances in machine learning for communications \cite{Oshea2017, Nachmani2018, Dorner2018}.

In the following, we derive a neural architecture for optimal decoding in an unknown channel. We propose to exploit the MI of the input-output channel pair as decoding metric. The computation of the MI, unknown in many instances, is per-se a challenging task. Therefore, the MI has to be learned, which can be done in principle with neural architectures \cite{Mine2018, Wunder2019, Song2020, f-DIME}. 

However, they are not directly deployable since in the problem at hand the decoder has a more specific task: it has to learn and minimize the a-posteriori information
\begin{equation}
    i(\mathbf{x|y}) = -\log_2{(p_{X|Y}(\mathbf{x}|\mathbf{y}))}
\end{equation}
of the transmitted codeword $\mathbf{x}$ given the observed/received signal vector $\mathbf{y}$. The direct estimation of the a-posteriori information becomes then the objective. Such an estimation is possible by directly learning the conditional PDF $p_{X|Y}(\mathbf{x}|\mathbf{y})$ with a new estimator that exploits a discriminative model referred to as mutual information neural decoder (MIND). MIND allows not only to perform the decoding task, but also to return an estimate of the achieved rate, the MI, and of the decoding error probability. Finally, in general, the same principle can be used to solve supervised \cite{pmlr-v235-novello24a} and weakly-supervised \cite{novello2024label} classification problems.

\section{Maximal mutual information neural decoder}
\sectionmark{Maximal mutual information neural decoder}
\label{sec:mind}

\subsection{Mutual information decoding principle}
\label{subsec:mind_mid}

The considered communication system comprises an encoder, a channel, and a decoder. The encoder maps a message of $k$ bits into one of the $M=2^k$ encoded messages of $n$ (not necessarily binary) symbols. The uncoded message is denoted with the vector $\mathbf{b}$, while the coded message with the vector $\mathbf{x}$ belonging to the set $\mathcal{A}_x$. Thus, the code rate is $R=k/n$ bits per channel use. The channel conveys the coded message to the receiver. Without loss of generality, we can write that the received message is $\mathbf{y}=H(\mathbf{x},\mathbf{h},\mathbf{n})$, where the CTF $H$ implicitly models the channel internal state $\mathbf{h}$ including stochastic variables, e.g. noise, $\mathbf{n}$. For instance, for a linear time-invariant (LTI) channel with additive noise, we obtain $\mathbf{y}=\mathbf{h}*\mathbf{x}+\mathbf{n}$, where $*$ denotes convolution.

The aim of the decoder is to retrieve the transmitted message so that to minimize the decoding error, or equivalently to maximize the received information for each transmitted message. Indeed, if we consider the MI of the pair $(\mathbf{x},\mathbf{y})$, with joint PDF $p_{XY}(\mathbf{x},\mathbf{y})$, and marginals $p_X(\mathbf{x})$ and $p_Y(\mathbf{y})$, this can be written as

\begin{equation}
i(\mathbf{x};\mathbf{y})=i(\mathbf{x})-i(\mathbf{x}|\mathbf{y})
\end{equation}
where $i(\mathbf{x}) = -\log_2 p_{X}(\mathbf{x})$ is the information of the transmitted message and $i(\mathbf{x}|\mathbf{y})$ is the a-posteriori information (residual uncertainty observing $\mathbf{y}$) \cite{Gallager1968}. It follows that for each transmitted message, the instantaneous mutual information is maximized when the a-posteriori information is minimized. The a-posteriori information is given by
\begin{equation}
\label{eq:MIND_PI}
i(\mathbf{x}|\mathbf{y})=-\log_2 p_{X|Y}(\mathbf{x}|\mathbf{y}) = -\log_2{\frac{p_{XY}(\mathbf{x},\mathbf{y})}{p_{Y}(\mathbf{y})}}
\end{equation}
so that the decoding criterion becomes
\begin{equation}
\mathbf{\hat{x}}=\argmin_{\mathbf{x}\in \mathcal{A}_x}{i(\mathbf{x}|\mathbf{y})} = \argmax_{\mathbf{x}\in \mathcal{A}_x}{\log_2 p_{X|Y}(\mathbf{x}|\mathbf{y})}
\end{equation}
which corresponds to the log-MAP decoding principle \cite{Proakis2001,Bahl1974}.

To compute the a-posteriori information, we need to know the a-posteriori probability $p_{X|Y}(\mathbf{x}|\mathbf{y})$. If the CTF can be modeled, the decoder is realized according to known approaches. In fact, the a-posteriori probability is explicitly calculated as $p_{X|Y}(\mathbf{x}|\mathbf{y})=p_{Y|X}(\mathbf{y}|\mathbf{x})p_{X}(\mathbf{x})/p_Y(\mathbf{y})$, i.e., from the conditional channel PDF, the coded message a-priori probability and the channel output PDF. However, for unknown channels and sources, we have to resort to a learning strategy as explained in the following subsection.

\subsection{Discriminative formulation of MIND}
\label{subsec:mind_ndec}

In the following, we show that it is possible to design a parametric discriminator whose output leads to the estimation of the a-posteriori information. The optimal discriminator is found by maximizing an appropriately defined value function.

Before describing the proposed solution, it should be noted that the adversarial training of GANs (see Sec. \ref{sec:gans}) pushes the discriminator output $D(\mathbf{a})$ towards the optimum value $D^*(\mathbf{a}) = p_{data}(\mathbf{a})/(p_{data}(\mathbf{a})+p_{gen}(\mathbf{a}))$, 
where $p_{data}(\mathbf{a})$ is the real data PDF and $p_{gen}(\mathbf{a})$ is the PDF of the data generated by $G$.
Thus, the output of the optimal discriminator $D^*(\mathbf{a})$ can be used to estimate the PDF ratio 
$p_{gen}(\mathbf{a})/p_{data}(\mathbf{a}) = (1-D^*(\mathbf{a}))/D^*(\mathbf{a}).$
 
Now, the idea is to design a discriminator that estimates the probability density ratio $p_{XY}(\mathbf{x},\mathbf{y})/p_{Y}(\mathbf{y})$ and consequently the a-posteriori information as shown in \eqref{eq:MIND_PI}. 
The following Lemma provides an a-posteriori information estimator that exploits a discriminator trained to maximize a certain value function. At the equilibrium, a transformation of the discriminator output yields the searched density ratio. 

\begin{lemma}
\label{lemma:MIND_Lemma1}
Let $\mathbf{x}\sim p_X(\mathbf{x})$ and $\mathbf{y}\sim p_Y(\mathbf{y})$ be the channel input and output vectors, respectively. Let $H(\cdot)$ be the channel transfer function, in general stochastic, such that $\mathbf{y} = H(\mathbf{x}, \mathbf{h}, \mathbf{n} )$, with $\mathbf{h}$ and $\mathbf{n}$ being internal state and noise variables, respectively. 
Let the discriminator $D(\mathbf{x},\mathbf{y})$ be a scalar function of $(\mathbf{x},\mathbf{y})$.
If $\mathcal{J}_{MIND}(D)$ is the value function defined as
\begin{align}
\label{eq:MIND_discriminator_function}
\mathcal{J}_{MIND}(D) &= \; \mathbb{E}_{(\mathbf{x},\mathbf{y}) \sim p_{U}(\mathbf{x})p_{Y}(\mathbf{y})}\biggl[|\mathcal{T}_x|\log \biggl(D\bigl(\mathbf{x},\mathbf{y}\bigr)\biggr)\biggr] \nonumber \\ 
& + \mathbb{E}_{(\mathbf{x},\mathbf{y}) \sim p_{XY}(\mathbf{x},\mathbf{y})}\biggl[\log \biggl(1-D\bigl(\mathbf{x},\mathbf{y}\bigr)\biggr)\biggr],
\end{align}
where $p_U(\mathbf{x})=1/|\mathcal{T}_x|$ describes a multivariate uniform distribution over the support $\mathcal{T}_x$ of $p_X(\mathbf{x})$ having Lebesgue measure $|\mathcal{T}_x|$, then the optimal discriminator output is
\begin{equation}
\label{eq:MIND_optimal_discriminator_1}
D^*(\mathbf{x},\mathbf{y}) = \arg \max_D \mathcal{J}_{MIND}(D) = \frac{p_{Y}(\mathbf{y})}{p_{Y}(\mathbf{y}) + p_{XY}(\mathbf{x},\mathbf{y})},
\end{equation}
and the a-posteriori information is computed as
\begin{equation}
\label{eq:MIND_i-DMIE}
i(\mathbf{x}|\mathbf{y}) = -\log_2 \biggl(\frac{1-D^*(\mathbf{x},\mathbf{y})}{D^*(\mathbf{x},\mathbf{y})}\biggr).
\end{equation}
\end{lemma}

\begin{proof}
From the hypothesis of the Lemma, the value function can be rewritten as
\begin{align}
\label{eq:MIND_Lebesgue1}
\mathcal{J}_{MIND}(D) &= \int_{\mathcal{T}_x} \int_{\mathcal{T}_y}\biggl[|\mathcal{T}_x|p_{U}(\mathbf{x})p_Y(\mathbf{y}) \log \biggl(D(\mathbf{x},\mathbf{y})\biggr) \nonumber \\ 
&+ p_{XY}(\mathbf{x},\mathbf{y}) \log \biggl(1-D(\mathbf{x},\mathbf{y})\biggr)\biggr] \diff \mathbf{x} \diff \mathbf{y}
\end{align}
where $\mathcal{T}_y$ is the support of $p_Y(\mathbf{y})$.
To maximize $\mathcal{J}_{MIND}(D)$, a necessary and sufficient condition requires to take the derivative of the integrand with respect to $D$ and equating it to $0$, yielding the following equation in $D$
\begin{equation}
\frac{|\mathcal{T}_x|p_{U}(\mathbf{x})p_Y(\mathbf{y})}{D(\mathbf{x},\mathbf{y})} -\frac{p_{XY}(\mathbf{x},\mathbf{y})}{1-D(\mathbf{x},\mathbf{y})} =0.
\end{equation}
The solution of the above equation yields the optimum discriminator $D^*(\mathbf{x},\mathbf{y})$ since: $p_U(\mathbf{x})=\frac{1}{|\mathcal{T}_x|}$ and $\mathcal{J}_{MIND}(D^*)$ is a maximum being the second derivative w.r.t. $D$ a non-positive function.

Finally, at the equilibrium, 
\begin{equation}
p_{X|Y}(\mathbf{x}|\mathbf{y})=\frac{p_{XY}(\mathbf{x},\mathbf{y})}{p_{Y}(\mathbf{y})} = \frac{1-D^*(\mathbf{x},\mathbf{y})}{D^*(\mathbf{x},\mathbf{y})},
\end{equation}
therefore the thesis follows.
\end{proof}

\subsection{Parametric implementation}
\label{subsec:mind_implementation}
The practical implementation of MIND is realized through a neural network-based learning approach. That is, the discriminator $D(\mathbf{x},\mathbf{y})$ with input $(\mathbf{x},\mathbf{y})$ is parameterized by a neural network. In a first phase, training of the discriminator is performed via optimization of \eqref{eq:MIND_discriminator_function}.
Training consists of transmitting repeatedly all coded messages such that at equilibrium the a-posteriori information $i(\mathbf{x}|\mathbf{y})$ is estimated. Then, in the testing phase, decoding can take place for each received new message $\mathbf{y}$ by minimizing the a-posteriori information. 
While the testing phase can be of moderate complexity depending on the neural network dimension, the training phase is more complex since no assumption on the channel and source statistics is made. In the following, we propose two different architectures and implementations of Lemma \ref{lemma:MIND_Lemma1}.

\subsubsection{Unsupervised approach}
The value function in \eqref{eq:MIND_discriminator_function}, as it stands, requires the discriminator to be a scalar parameterized function which takes as input both the transmitted and received vector of samples $(\mathbf{x},\mathbf{y})$. The resulting architecture is presented in Fig. \ref{fig:MIND_unsupervised}. Such formulation is general and can be applied to any coding scheme. In fact, to learn the parameters of $D$, it is sufficient during training to alternate the input joint samples $(\mathbf{x},\mathbf{y})$ with marginal samples $(\mathbf{u},\mathbf{y})$, where $\mathbf{u}$ are realizations of a multivariate uniform distribution with independent components, defined over the support $\mathcal{T}_x$ of $p_X(\mathbf{x})$. Then, during the testing phase, decoding is accomplished by finding $\mathbf{x}$ that minimizes \eqref{eq:MIND_i-DMIE}, for all possible coded messages $\mathbf{x}$. This  method can be thought as an unsupervised learning approach since no known labels are exploited in the loss function for the training process. Hence, the objective is to implicitly estimate the density ratio in \eqref{eq:MIND_optimal_discriminator_1}.
Since in practical cases, the coded message $\mathbf{x}$ has discrete alphabet, another architecture is proposed below.

\begin{figure}
	\centering
	\includegraphics[scale=0.63]{images/decoder/ND_unsupervised_new.pdf}
	\caption{Unsupervised discriminator architecture and relative value function. Input is fed with paired samples from the joint distribution while the output consists of a single probability ratio value.}
	\label{fig:MIND_unsupervised}
\end{figure} 

the single output network, which has to represent a transformation of a probability mass function, tends to collapse the information from previous layers in a unique node and empirically can be verified that it can lead to training problems; lastly, the knowledge of the true label (the transmitted sample) does not explicitly appear in the loss function, thus, it does not guide the training towards optimality. In summary, learning the joint distribution $p_{XY}(\mathbf{x},\mathbf{y})$ may be not convenient for our purpose. All these empirical and heuristic considerations made us rethink about the architecture to use, as explained next. 

\subsubsection{Supervised approach}
It is possible to deploy a discriminator architecture that receives as input only the channel output $\mathbf{y}$. Indeed, if the channel input has discrete nature, e.g., $\mathbf{x}$ belongs to the alphabet $\mathcal{A}_x = \{\mathbf{x}_1, \mathbf{x}_2, \dots, \mathbf{x}_{M}\}$ with PDF 
\begin{equation}
\label{eq:MIND_px}
        p_X(\mathbf{x})=\sum_{\mathbf{x}_i \in \mathcal{A}_x}{P_X(\mathbf{x}_i)\delta(\mathbf{x}-\mathbf{x}_i)},
\end{equation}
then also the multivariate uniform PDF in Lemma \ref{lemma:MIND_Lemma1} reads as
\begin{equation}
\label{eq:MIND_pu}
       p_U(\mathbf{x})=\sum_{\mathbf{x}_i \in \mathcal{A}_x}{P_U(\mathbf{x}_i)\delta(\mathbf{x}-\mathbf{x}_i)}, 
\end{equation}
with 
\begin{equation}
    \label{eq:MIND_1/M}
    P_U(\mathbf{x}_i)=1/M.
\end{equation}
Thus, the value function in \eqref{eq:MIND_discriminator_function}, or its Lebesgue integral expression in \eqref{eq:MIND_Lebesgue1}, can be decomposed as a sum of independent elements for every input codeword $\mathbf{x}_i$, for $i\in \{1,\dots,M\}$, as follows
\begin{equation}
\label{eq:MIND_j_i}
\mathcal{J}_{MIND}(D) = \sum_{i = 1}^{M}{\mathcal{J}_i(D)},
\end{equation}
where 
\begin{align}
\label{eq:MIND_discriminator_function_2}
\mathcal{J}_{i}(D) &= \; \mathbb{E}_{\mathbf{y} \sim p_{Y}(\mathbf{y})}\biggl[\log \biggl(D\bigl(\mathbf{x}_i,\mathbf{y}\bigr)\biggr)\biggr] \\ \nonumber
&+P_{X}(\mathbf{x}_i)\mathbb{E}_{\mathbf{y} \sim p_{Y|X}(\mathbf{y}|\mathbf{x}_i)}\biggl[\log \biggl(1-D\bigl(\mathbf{x}_i,\mathbf{y}\bigr)\biggr)\biggr].
\end{align}
By writing \eqref{eq:MIND_discriminator_function_2} with Lebesgue integrals and following the proof of Lemma \ref{lemma:MIND_Lemma1}, it is simple to verify that the discriminator which maximizes \eqref{eq:MIND_j_i} has to maximize all terms $\mathcal{J}_{i}(D)$, and this happens for 
\begin{equation}
\label{eq:MIND_metric}
D^*(\mathbf{x}_i,\mathbf{y}) = \frac{p_{Y}(\mathbf{y})}{p_{Y}(\mathbf{y}) + P_{X}(\mathbf{x}_i)p_{Y|X}(\mathbf{y}|\mathbf{x}_i)} = \frac{1}{1 + P_{X|Y}(\mathbf{x}_i|\mathbf{y})},
\end{equation}
where $P_{X|Y}(\mathbf{x}_i|\mathbf{y})$ is the PMF of $X|Y$,
having PDF
\begin{equation}
    p_{X|Y}(\mathbf{x}|\mathbf{y})=\sum_{\mathbf{x}_i \in \mathcal{A}_x}{P_{X|Y}(\mathbf{x}_i|\mathbf{y})\delta(\mathbf{x}-\mathbf{x}_i)}.
\end{equation}
More details are offered in Sec. \ref{sec:mind_appendix}.

Now, it is more convenient from an implementation perspective to define the $M$-dimensional vectors
\begin{equation}
\bar{\mathbf{D}}(\mathbf{y}) := 
\begin{bmatrix}
   D(\mathbf{x}_1,\mathbf{y})  \\
   D(\mathbf{x}_2,\mathbf{y}) \\
   \vdots \\
    D(\mathbf{x}_M,\mathbf{y})
\end{bmatrix}
, 
\bar{\mathbf{1}} := 
\begin{bmatrix}
   1  \\
   1 \\
   \vdots \\
    1
\end{bmatrix}
, 
\bar{\mathbf{1}}({\mathbf{x}_i}) := 
\begin{bmatrix}
   0  \\
   \vdots \\
    0 \\
    \underbrace{1}_\text{i-th position} \\ 
    0 \\
    \vdots \\
    0
\end{bmatrix}.
\end{equation}
In fact, the expectations in \eqref{eq:MIND_discriminator_function_2} can be rewritten as
\begin{align}
\label{eq:MIND_disc_vect_1}
    &\mathcal{J}_{MIND}(D) =   \mathbb{E}_{\mathbf{y} \sim p_{Y}(\mathbf{y})}\biggl[\log \bigl(\bar{\mathbf{D}}(\mathbf{y}) \bigr)^T\cdot \bar{\mathbf{1}}_M\biggr] \\ \nonumber 
    &+\sum_{\mathbf{x}_i\in \mathcal{A}_x}{P_X(\mathbf{x}_i) \mathbb{E}_{\mathbf{y} \sim p_{Y|X}(\mathbf{y}|\mathbf{x}_i)}\biggl[\log \bigl(1-\bar{\mathbf{D}}(\mathbf{y})\bigr)^T\cdot \bar{\mathbf{1}}_M({\mathbf{x}_i})\biggr]},
\end{align}
where $\bar{\mathbf{1}}_M$ is a vector of $M$ unitary elements and $\bar{\mathbf{1}}_M({\mathbf{x}_i})$ is the one-hot code of $\mathbf{x}$.
Finally, the value function of the supervised architecture (see Fig. \ref{fig:MIND_supervised}) assumes the following vector form 
\begin{align}
\label{eq:MIND_new_discriminator_function}
 &\mathcal{J}_{MIND}(D) =   \mathbb{E}_{\mathbf{y} \sim p_{Y}(\mathbf{y})}\biggl[\log \bigl(\bar{\mathbf{D}}(\mathbf{y}) \bigr)^T\cdot \bar{\mathbf{1}}_M\biggr] \\ \nonumber
 &+\mathbb{E}_{\mathbf{x} \sim p_{X}(\mathbf{x})} \mathbb{E}_{\mathbf{y} \sim p_{Y|X}(\mathbf{y}|\mathbf{x})}\biggl[\log \bigl(1-\bar{\mathbf{D}}(\mathbf{y})\bigr)^T\cdot \bar{\mathbf{1}}_M(\mathbf{x})\biggr].
\end{align}
The new expression \eqref{eq:MIND_new_discriminator_function} possesses the label information in the scalar product with the one-hot positional code and therefore can be treated as a supervised learning problem. 
\begin{figure}
	\centering
	\includegraphics[scale=0.63]{images/decoder/ND_supervised_new.pdf}
	\caption{Supervised discriminator architecture and relative value function. Input is fed with the received channel samples while the output consists of multiple probability ratio values.}
	\label{fig:MIND_supervised}
\end{figure} 

By comparing the $M$ discriminator outputs $D^*_i = D^*(\mathbf{x}_i,\mathbf{y})$ in \eqref{eq:MIND_metric}, the final decoding stage implements
\begin{equation}
\mathbf{\hat{x}}_i = \argmax_{i \in \{1,...M\}} \log_2 \frac{1-D^*_i}{D^*_i} = \argmin_{\mathbf{x}_i \in \mathcal{A}_x} i(\mathbf{x}_i|\mathbf{y}),
\end{equation}
thus, MIND aims at finding the codeword $\mathbf{x}_i$ which minimizes the residual uncertainty $i(\mathbf{x}_i|\mathbf{y})$ after the observation of $\mathbf{y}$.

\subsection{Estimation of achieved information rate and decoding error probability}
\label{subsec:mind_estimation}
Dealing with an unknown channel, a relevant question is the estimation of the information rate achieved with the used coding scheme. MIND can be exploited for such a goal. In fact, the normalized average mutual information (in bits per channel use) is given by \cite{Gallager1968}
\begin{equation}
    I_n(X;Y) = \frac{1}{n}(H(X)-H(X|Y))
\end{equation}
whose computation requires the entropy of the source $H(X)$ and the conditional entropy $H(X|Y)$:
\begin{equation}
\label{eq:MIND_source_entropy}
H(X) = \mathbb{E}_{\mathbf{x} \sim p_X(\mathbf{x})}[i(\mathbf{x})] = -\sum_{\mathbf{x}_i \in \mathcal{A}_x}{P_X(\mathbf{x}_i)\log_2 P_X(\mathbf{x}_i)}
\end{equation}
\begin{equation}
\label{eq:MIND_conditional_entropy}
H(X|Y) = -\mathbb{E}_{\mathbf{y} \sim p_Y(y)}\biggl[\sum_{\mathbf{x}_i \in \mathcal{A}_x}{P_{X|Y}(\mathbf{x}_i|\mathbf{y})\log_2 P_{X|Y}(\mathbf{x}_i|\mathbf{y})}\biggr].
\end{equation}
The discriminator in MIND, for a given coding scheme, returns an estimate of the a-posteriori probability mass function values $P_{X|Y}(\mathbf{x}_i|\mathbf{y})=(1-D_i)/D_i \; \forall i=\{1,\dots,M\}$, therefore \eqref{eq:MIND_conditional_entropy} can be directly computed. About \eqref{eq:MIND_source_entropy}, it is worth mentioning that the source distribution can be obtained using Monte Carlo integration from the a-posteriori probability obtained with MIND by averaging over $N$ realizations $\mathbf{y}_j$ of $Y$ as follows 
\begin{equation}
\label{eq:MIND_monte-carlo_source}
P_X(\mathbf{x}_i) \stackrel{N\to \infty}{=} \frac{1}{N}\sum_{j=1}^{N}{P_{X|Y}(\mathbf{x}_i|\mathbf{y}_j)}.
\end{equation}
Similarly,
\begin{equation}
H(X|Y) \stackrel{N\to \infty}{=} -\frac{1}{N} \sum_{j=1}^{N}{\sum_{\mathbf{x}_i \in \mathcal{A}_x}{P_{X|Y}(\mathbf{x}_i|\mathbf{y}_j)\log_2 P_{X|Y}(\mathbf{x}_i|\mathbf{y}_j)}}.
\end{equation}

Finally, the average mutual information (in bits) estimator $I_N$ reads as
\begin{align}
I_N(X;Y) & = \;  \frac{1}{N} \sum_{j=1}^{N}{\sum_{\mathbf{x}_i \in \mathcal{A}_x}{p_{X|Y}(\mathbf{x}_i|\mathbf{y}_j)\log_2 p_{X|Y}(\mathbf{x}_i|\mathbf{y}_j)}}  \nonumber \\ 
    & -\frac{1}{N}\sum_{\mathbf{x}_i \in \mathcal{A}_x}{\sum_{j=1}^{N}{p_{X|Y}(x|\mathbf{y}_j)}\log_2 \biggl(\frac{1}{N} \sum_{j=1}^{N}{p_{X|Y}(x|\mathbf{y}_j)}\biggr)},
\end{align}
and it depends only on the a-posteriori probability estimate provided by the output of the discriminator.

MIND can also be used to estimate the decoding error probability $P_e$. In fact, 
\begin{equation}
    P_e = 1-P[\mathbf{x}=\hat{\mathbf{x}}] = 1-\mathbb{E}_{\mathbf{y} \sim p_Y(y)}[P_{X|Y}(\mathbf{x}=\mathbf{\hat{x}}|\mathbf{y})],
\end{equation}
where the probability $P_{X|Y}(\mathbf{x}=\mathbf{\hat{x}}|\mathbf{y})$ is the instantaneous probability of correct decoding obtained from the decision criterion $\max_{\mathbf{x}_i} P_{X|Y}(\mathbf{x}_i|\mathbf{y})$. Hence,
\begin{equation}
    P_e \stackrel{N\to \infty}{=} 1-\frac{1}{N} \sum_{j=1}^{N}{\max_{\mathbf{x}_i \in \mathcal{A}_x} P_{X|Y}(\mathbf{x}_i|\mathbf{y}_j)}.
\end{equation}

We now compare MIND to other well-known decoding criteria in scenarios for which an analytic solution is available. Furthermore, thanks to the network ability to estimate the a-posteriori probability, the average input-output mutual information is also reported.

\section{Numerical results}
\label{sec:mind_results}
The assessment of the proposed decoding approach is done by considering the following three representative scenarios: a) Uncoded transmission of symbols that are produced by a non-uniform source over an additive Gaussian noise channel; b) Uncoded transmission in a non-linear channel with additive Gaussian noise; c) Short block coded transmission in an additive Middleton noise channel.

Details about the architecture and hyper parameters of the neural networks are reported in the GitHub repository \cite{MIND_github}.

\subsection{Non-uniform source}
To show the ability of the decoder to learn and exploit data dependence at the source, we firstly consider 4-PAM transmission with symbols $\mathbf{x}_i \in \{-3,-1,1,3\}$ and mass function generated by a non-uniform source $p(\mathbf{x}) = [(1-P)/2, P/2, (1-P)/2, P/2]$, with $P=0.05$. It should be noted that data dependence at the source can be either created with a coding stage from a source that emits i.i.d. symbols, or intrinsically by the source of data traffic having certain statistics, as in this case. In Fig. \ref{fig:MIND_non-uniform_source}a the symbol error rate (SER) is shown. MIND is compared with the optimal MAP Genie decoder that knows the a-priori distribution of the transmitted data symbols $p_X(\mathbf{x})$ and with the MaxL decoder that only knows the Gaussian nature of the channel and assumes a source with uniform distribution. Moreover, the estimated $P_e$ is also shown. MIND and the MAP Genie decoder perform identically. In Fig. \ref{fig:MIND_non-uniform_source}b, the source entropy, the conditional entropy and the average achieved MI, as estimated by MIND, are reported and compared with the ground truth numerically obtained. It is worth noticing that the average MI increases with the SNR, or alternatively, the residual uncertainty introduced by the channel decreases. In addition, the estimated source entropy tends to stabilize around its real value given by $-P\log_2(P/2)-(1-P)\log_2((1-P)/2)$.

\begin{figure}
\centering
  \includegraphics[scale=0.25]{images/decoder/non-uniform_source_ALL_2.pdf}
  \caption{a) Symbol error rate for a 4-PAM modulation with non-uniform source distribution over an AWGN channel. Comparison among the optimal MAP decoder, MaxL decoder, MIND decoder and the estimated probability of error provided by MIND. b) Estimated source and conditional entropy (top right) and estimated average mutual information (bottom right) using MIND.}
	\label{fig:MIND_non-uniform_source}
 \end{figure}

 
\subsection{Non-linear channel}
As a second example, we consider 4-PAM uncoded transmission with uniform source distribution over a non-linear channel with additive Gaussian noise. The objective of this experiment is to show the ability of MIND to discover such a channel non-linearity. In particular, the channel introduces a non-linearity (for instance because of the presence of non-linear amplifiers) modeled as $y_k=\text{sign}(x_k)\sqrt{|x_k|}+n_k$, where $k$ denotes the $k$-th time instant.
Fig.\ref{fig:MIND_non-linear_channel}a demonstrates how the MIND decoder manages to implicitly learn the non-linear channel model during the training phase and effectively use such information during decoding. A comparison with the MaxL decoder with and without perfect channel state information (CSI) knowledge is also conducted. Results show that MIND exhibits performance identical to the MAP Genie. Fig. \ref{fig:MIND_non-linear_channel}b illustrates both the behaviour of the entropies and the average mutual information as function of the SNR.

\begin{figure}
\centering
   \includegraphics[scale=0.25]{images/decoder/non-linear_channel_ALL_2.pdf}
  \caption{a) Symbol error rate for a 4-PAM modulation with uniform source distribution over a non-linear channel affected by AWGN. Comparison among MaxL decoder with no CSI, MaxL decoder with perfect CSI, MIND decoder and the estimated probability of error. b) Estimated source and conditional entropy (top right) and estimated average mutual information (bottom right) using MIND.}
  \label{fig:MIND_non-linear_channel}
\end{figure}

\subsection{Additive non-gaussian noise channel}
Lastly, and perhaps as most interesting example, we consider a short block coded transmission over an additive non-gaussian channel. The aim is to assess the ability of MIND to learn and exploit the presence of non-gaussian noise. In particular, we suppose that the noise follows a truncated Middleton distribution \cite{Middleton1977}, also called Bernoulli-Gaussian noise model, so that at any given time instant $k$ it is obtained as $n_k = (1-\epsilon_k)n_{1,k}+\epsilon_k n_{2,k}$ where $n_{1,k} \sim \mathcal{N}(0,\sigma_b^2)$ is a zero-mean Gaussian random variable with variance $\sigma_b^2$ and $n_{2,k} \sim \mathcal{N}(0,B\sigma_b^2)$ is also a zero-mean Gaussian random variable but with variance $B$ times larger. Instead, $\epsilon_k$ is a Bernoulli random variable with probability of success $P$. The PDF of the noise samples is then given by
$p_{N}(\mathbf{n}_k) = (1-P)\mathcal{N}(0,\sigma_b^2)+P\mathcal{N}(0,B\sigma_b^2).$
Assuming that the noise model is known and that BPSK symbols are transmitted, two decoding strategies can be devised. The first, denoted with MaxL Middleton, uses maximum likelihood decoding with the known conditional PDF $p(\mathbf{y}|\mathbf{x})=p_{N}(\mathbf{y}-\mathbf{x})$. The second strategy is a genie decoder that knows the outcome of the Bernoulli event for every time instant. That is, it knows whether a received sample is hit by Gaussian noise with variance $\sigma_b^2$ or $B\sigma_b^2$. A third decoding strategy is offered by MIND that learns the channel statistics.
We distinguish among three different types of codes: a) a binary repetition code with length $5$; b) a $(7,4)$ Hamming code; c) a rate $1/2$ convolutional code with memory $2$ and block-length $18$. For each of them, the bit error rate (BER) obtained with the genie, the MaxL Middleton and the MIND decoders is reported. The parameter $B$ was set to $5$ in all the experiments involving Middleton noise, wherein we also set $P=0.05$. 

\begin{figure*}[b]
	\centering
	\includegraphics[scale=0.2]{images/decoder/middleton_info_2.pdf}
	\caption{Bit error rate and mutual information of a short block coded transmission in an additive Middleton noise channel: a) Repetition code; b) Hamming code; c) Convolutional code.}
	\label{fig:MIND_middleton}
\end{figure*} 

Fig.~\ref{fig:MIND_middleton} shows the gain provided by the neural-based decoder scheme over the classical maximum likelihood one that exploits the Middleton distribution. With MIND, the BER performance gets closer to the genie decoder. Fig.\ref{fig:MIND_middleton} also reports an estimate of the average mutual information provided by MIND.

\section{Summary}
\label{sec:MIND_conclusions}
In this chapter, MIND, a neural decoder that uses the mutual information as decoding criterion, has been proposed. Two specific architectures have been described and they are capable of learning the a-posteriori information $i(\mathbf{x}|\mathbf{y})$ of the codeword $\mathbf{x}$ given the channel output observation $\mathbf{y}$ in unknown channels. Additionally, MIND allows the estimation of the achieved information rate with the used coding scheme as well as the decoding error probability. Several numerical results obtained in illustrative channels show that MIND can achieve the performance of the genie MAP decoder that perfectly knows the channel model and source distribution. It outperforms the conventional MaxL decoder that assumes the presence of Gaussian noise. 

\newpage
\section{Mathematical derivations}
\label{sec:mind_appendix}
\subsection{From unsupervised to supervised}
Here we provide extra details on the derivation of the supervised loss function from the supervised one. In particular, given the loss
\begin{equation}
\small
\mathcal{J}_{MIND}(D) =  \mathbb{E}_{(\mathbf{x},\mathbf{y}) \sim p_{U}(\mathbf{x})p_{Y}(\mathbf{y})}\biggl[|\mathcal{T}_x|\log \biggl(D\bigl(\mathbf{x},\mathbf{y}\bigr)\biggr)\biggr] + \mathbb{E}_{(\mathbf{x},\mathbf{y}) \sim p_{XY}(\mathbf{x},\mathbf{y})}\biggl[\log \biggl(1-D\bigl(\mathbf{x},\mathbf{y}\bigr)\biggr)\biggr],
\end{equation}
its Lebesgue integral form has expression
\begin{equation}
    \small
    \mathcal{J}_{MIND}(D) = \int_{\mathcal{T}_x} \int_{\mathcal{T}_y}\biggl[p_{U}(\mathbf{x})p_Y(\mathbf{y})|\mathcal{T}_x| \log \biggl(D(\mathbf{x},\mathbf{y})\biggr) + p_{XY}(\mathbf{x},\mathbf{y}) \log \biggl(1-D(\mathbf{x},\mathbf{y})\biggr)\biggr] \diff \mathbf{x} \diff \mathbf{y}.
\end{equation}

Assuming the channel input has discrete nature, 
then, from \eqref{eq:MIND_px}, the Lebesgue integral can be rewritten as
\begin{align}
    \mathcal{J}_{MIND}(D) &= \int_{\mathcal{T}_x} \int_{\mathcal{T}_y}{\sum_{x_i\in \mathcal{A}_x}{P_U(\mathbf{x}_i)\delta(\mathbf{x}-\mathbf{x}_i)}|\mathcal{T}_x|p_Y(\mathbf{y}) \log \biggl(D(\mathbf{x},\mathbf{y})\biggr)} \diff \mathbf{x} \diff \mathbf{y} \nonumber \\ 
    & +\int_{\mathcal{T}_x} \int_{\mathcal{T}_y}{\sum_{\mathbf{x}_i\in \mathcal{A}_x}{P_X(\mathbf{x}_i)\delta(\mathbf{x}-\mathbf{x}_i)} p_{Y|X}(\mathbf{y}|\mathbf{x}) \log \biggl(1-D(\mathbf{x},\mathbf{y})\biggr)} \diff \mathbf{x} \diff \mathbf{y}
\end{align}
where we used \eqref{eq:MIND_pu}.
By exploiting the sifting property of the delta function
\begin{align}
    \mathcal{J}_{MIND}(D) &=  {\sum_{\mathbf{x}_i\in \mathcal{A}_x}{P_U(\mathbf{x}_i)}{\int_{\mathcal{T}_y}|\mathcal{T}_x|p_Y(\mathbf{y}) \log \biggl(D(\mathbf{x}_i,\mathbf{y})\biggr)} \diff \mathbf{y}}\nonumber \\
    & + {\sum_{\mathbf{x}_i\in \mathcal{A}_x}{P_X(\mathbf{x}_i)\int_{\mathcal{T}_y} p_{Y|X}(\mathbf{y}|\mathbf{x}_i) \log \biggl(1-D(\mathbf{x}_i,\mathbf{y})\biggr)} \diff \mathbf{y}}.
\end{align}
From the hypothesis of the MIND loss function in \eqref{eq:MIND_1/M}, 
\begin{align}
    \mathcal{J}_{MIND}(D) &=  {\sum_{\mathbf{x}_i\in \mathcal{A}_x}{\frac{1}{M}\int_{\mathcal{T}_y}|\mathcal{T}_x|p_Y(\mathbf{y}) \log \biggl(D(\mathbf{x}_i,\mathbf{y})\biggr)} \diff \mathbf{y}}\nonumber \\ 
    & + {\sum_{\mathbf{x}_i\in \mathcal{A}_x}{P_X(\mathbf{x}_i)\int_{\mathcal{T}_y} p_{Y|X}(\mathbf{y}|\mathbf{x}_i) \log \biggl(1-D(\mathbf{x}_i,\mathbf{y})\biggr)} \diff \mathbf{y}},
\end{align}
 and since $|\mathcal{T}_x|=M$, it can be simplified into
\begin{align}
    \mathcal{J}_{MIND}(D) =  & \sum_{\mathbf{x}_i\in \mathcal{A}_x}{\biggl[\int_{\mathcal{T}_y}p_Y(\mathbf{y}) \log \biggl(D(\mathbf{x}_i,\mathbf{y})\biggr)\diff \mathbf{y}} \nonumber \\
     & + P_X(\mathbf{x}_i)\int_{\mathcal{T}_y} p_{Y|X}(\mathbf{y}|\mathbf{x}_i) \log \biggl(1-D(\mathbf{x}_i,\mathbf{y})\biggr) \diff \mathbf{y}\biggr],
\end{align}
or alternatively
\begin{equation}
\small
    \mathcal{J}_{MIND}(D) =  \sum_{\mathbf{x}_i\in \mathcal{A}_x}{\biggl[\int_{\mathcal{T}_y}p_Y(\mathbf{y}) \log \biggl(D(\mathbf{x}_i,\mathbf{y})\biggr)\diff \mathbf{y} + \int_{\mathcal{T}_y} P_{XY}(\mathbf{x}_i,\mathbf{y}) \log \biggl(1-D(\mathbf{x}_i,\mathbf{y})\biggr) \diff \mathbf{y}\biggr]},
\end{equation}
where $P_{XY}(\mathbf{x}_i,\mathbf{y}):=P_X(\mathbf{x}_i)\cdot p_{Y|X}(\mathbf{y}|\mathbf{x}_i)$. The above relation is equivalent to
\begin{equation}
\small
    \mathcal{J}_{MIND}(D) =  \sum_{\mathbf{x}_i\in \mathcal{A}_x}{\biggl[\int_{\mathcal{T}_y}p_Y(\mathbf{y}) \log \biggl(D(\mathbf{x}_i,\mathbf{y})\biggr) + P_{XY}(\mathbf{x}_i,\mathbf{y}) \log \biggl(1-D(\mathbf{x}_i,\mathbf{y})\biggr) \diff \mathbf{y}\biggr]}.
\end{equation}

Finally, the MIND loss function can be rewritten with expectations as
\begin{align}
    \mathcal{J}_{MIND}(D) = \sum_{\mathbf{x}_i\in \mathcal{A}_x}\mathcal{J}_{i}(D) =  & \sum_{\mathbf{x}_i\in \mathcal{A}_x}{\mathbb{E}_{\mathbf{y} \sim p_{Y}(\mathbf{y})}\biggl[\log \biggl(D\bigl(\mathbf{x}_i,\mathbf{y}\bigr)\biggr)\biggr]} \nonumber \\ 
    & + P_{X}(\mathbf{x}_i)\mathbb{E}_{\mathbf{y} \sim p_{Y|X}(\mathbf{y}|\mathbf{x}_i)}\biggl[\log \biggl(1-D\bigl(\mathbf{x}_i,\mathbf{y}\bigr)\biggr)\biggr],
\end{align}
or more conveniently as in \eqref{eq:MIND_new_discriminator_function}.

\subsection{Considerations on the expectation theorem}
In the following we present some other useful results that have either been used in the mathematical derivation or can be used to provide alternative proofs to the main results.

For simplicity, we now consider only the one-dimensional case. Let $p_U(u)$ be a uniform probability density function in $[0,1]$, and let $K(\cdot)$ be a function that maps $x$ into $u$. Then
\begin{equation}
   \int_{\mathcal{T}_x}{p_U(K(a))  \log \bigl(D(a)\bigr)} \diff a = \int_{[0,1]}{p_U(b)  \log \bigl(D(K^{-1}(b))\bigr)} \frac{1}{k(K^{-1}(b))} \diff b
\end{equation}
where $k(\cdot)$ is the derivative of $K(\cdot)$. Notice that $p_U(K(x))=1$, but for notation convenience we leave the entire expression. In particular, a valid choice consists of $K(x)=F_X(x)$ with $F_X(\cdot)$ being the cumulative distribution function of $x$. Hence
\begin{equation}
   \int_{\mathcal{T}_x}{p_U(F_X(a))  \log \bigl(D(a)\bigr)} \diff a = \int_{[0,1]}{p_U(b)  \log \bigl(D(F_{X}^{-1}(b))\bigr)} \frac{1}{p_X(F_{X}^{-1}(b))} \diff b
\end{equation}
which rewritten with an expectation reads as
\begin{align}
\label{eq:MIND_w_den}
   \int_{\mathcal{T}_x}{p_U(F_X(a))  \log \bigl(D(a)\bigr)} \diff a = & \; \mathbb{E}_{u \sim p_{U}(u)}\biggl[\log \biggl( \frac{D(F_{X}^{-1}(u))}{\exp{(p_X(F_{X}^{-1}(u))})} \biggr)\biggr] \nonumber \\
   = & \; \mathbb{E}_{x \sim p_{X}(x)}\biggl[\log \biggl( \frac{D(x)}{\exp{(p_X(x)})} \biggr)\biggr].
\end{align}
Notice that the LHS of \eqref{eq:MIND_w_den} can be interpreted (as done in Lemma \ref{lemma:MIND_Lemma1}) as an integral over a uniform distribution with support $\mathcal{T}_x$, i.e.,
\begin{equation}
    \int_{\mathcal{T}_x}{p_U(F_X(a))  \log \bigl(D(a)\bigr)} \diff a = \int_{\mathcal{T}_x}{\log \bigl(D(a)\bigr)} \diff a = \int_{\mathcal{T}_x}{p_{U_{\mathcal{T}}}(a)  |\mathcal{T}_x|\log \bigl(D(a)\bigr)} \diff a. 
\end{equation}
Without considering the term at the denominator, instead,
\begin{equation}
   \int_{[0,1]}{p_U(b)  \log \bigl(D(F_{X}^{-1}(b))\bigr)} \diff b = \int_{\mathcal{T}_x}{p_U(F_X(a))  \log \bigl(D(a)\bigr)p_X(a)} \diff a,
\end{equation}
and using again expectations
\begin{equation}
\small
\label{eq:MIND_wo_den}
   \int_{\mathcal{T}_x}{p_U(F_X(a))  \log \bigl(D(a)\bigr)p_X(a)} \diff a = \mathbb{E}_{u \sim p_{U}(u)}\biggl[\log \biggl( D(F_{X}^{-1}(u)) \biggr)\biggr] = \mathbb{E}_{x \sim p_{X}(x)}\biggl[\log \biggl( D(x) \biggr)\biggr] .
\end{equation}
For the purposes of MIND, we would like to work with the expression in \eqref{eq:MIND_wo_den} in order to really extract the a-posteriori $p_{X|Y}(x|y)$ from $D$. However, from an implementation perspective, the denominator term inside the expectation operator in \eqref{eq:MIND_w_den} is needed, thus, an estimate of the density $p_X(x)$ is also required, which renders the sampling mechanism challenging. Instead, it is simpler to estimate (with Monte Carlo) the expectation in \eqref{eq:MIND_wo_den} but the term $p_X(x)$ will appear inside the integral, leading to a discriminator having as output the term $p_{XY}(x,y)/p(x)p(y)$.


\subsection{Learning Method}

We compare the performance of agents trained on data from the InSTA pipeline to agents trained on human demonstrations from WebLINX \citep{WebLINX} and Mind2Web \citep{Mind2Web}, two recent and popular benchmarks for web navigation. Recent works that mix synthetic data with real data control the real data sampling probability in the batch $p_{\text{real}}$ independently from data size \citep{DAFusion}. We employ $p_{\text{real}} = 0.5$ in few-shot experiments and $p_{\text{real}} = 0.8$ otherwise. Shown in Figure~\ref{fig:data-statistics}, our data have a wide spread in performance, so we apply several filtering rules to select high-quality training data. First, we require the evaluator to return \texttt{conf} = 1 that the task was successfully completed, and that the agent was on the right track (this selects data where the actions are reliable, and directly caused the task to be solved). Second, we filter data where the trajectory contains at least three actions. Third, we remove data where the agent encountered any type of server error, was presented with a captcha, or was blocked at any point. These steps produce $7,463$ high-quality demonstrations in which agents successfully completed tasks on diverse websites. We sample 500 demonstrations uniformly at random from this pool to create a diverse test set, and employ the remaining $6,963$ demonstrations to train agents on a mix of real and synthetic data.

\subsection{Improving Data-Efficiency}
\label{sec:few-shot}

\begin{wrapfigure}{r}{0.48\textwidth}
    \centering
    \vspace{-0.8cm}
    \includegraphics[width=\linewidth]{assets/few_shot_results_weblinx_mind2web.pdf}
    \vspace{-0.3cm}
    \caption{\small \textbf{Data from InSTA improves efficiency.} Language model agents trained on mixtures of our data and human demonstrations scale faster than agents trained on human data. In a setting with 32 human actions, adding our data improves \textit{Step Accuracy} by +89.5\% relative to human data for Mind2Web, and +122.1\% relative to human data for WebLINX.}
    \vspace{-0.2cm}
    \label{fig:few-shot-results}
\end{wrapfigure}

In a data-limited setting derived from WebLINX \citep{WebLINX} and Mind2Web \citep{Mind2Web}, agents trained on our data \textit{scale faster with increasing data size} than human data alone. Without requiring laborious human annotations, the data produced by our pipeline leads to improvements on Mind2Web that range from +89.5\% in \textit{Step Accuracy} (the rate at which the correct element is selected and the correct action is performed on that element) with 32 human actions, to +77.5\% with 64 human actions, +13.8\% with 128 human actions, and +12.1\% with 256 human actions. For WebLINX, our data improves by +122.1\% with 32 human actions, and +24.6\% with 64 human actions, and +6.2\% for 128 human actions. Adding our data is comparable in performance gained to doubling the amount of human data from 32 to 64 actions. Performance on the original test sets for Mind2Web and WebLINX appears to saturate as the amount of human data increases, but these benchmark only test agent capabilities for a limited set of 150 popular sites.

\subsection{Improving Generalization} 
\label{sec:generalization}

\begin{wrapfigure}{r}{0.48\textwidth}
    \centering
    \vspace{-1.0cm}
    \includegraphics[width=\linewidth]{assets/diverse_results_weblinx_mind2web.pdf}
    \vspace{-0.3cm}
    \caption{\small \textbf{Our data improves generalization.} We train agents with all human data from the WebLINX and Mind2Web training sets, and resulting agents struggle to generalize to more diverse test data. Adding our data improves generalization by +149.0\% for WebLINX, and +156.3\% for Mind2Web.}
    \vspace{-0.3cm}
    \label{fig:generalization-results}
\end{wrapfigure}

To understand how agents trained on data from our pipeline generalize to diverse real-world sites, we construct a more diverse test set than Mind2Web and WebLINX using 500 held-out demonstrations produced by our pipeline. Shown in Figure~\ref{fig:generalization-results}, we train agents using all human data in the training sets for WebLINX and Mind2Web, and compare the performance with agents trained on 80\% human data, and 20\% data from our pipeline. Agents trained with our data achieve comparable performance to agents trained purely on human data on the official test sets for the WebLINX and Mind2Web benchmarks, suggesting that when enough human data are available, synthetic data may not be necessary. However, when evaluated in a more diverse test set that includes 500 sites not considered by existing benchmarks, agents trained purely on existing human data struggle to generalize. Training with our data improves generalization to these sites by +149.0\% for WebLINX agents, and +156.3\% for Mind2Web agents, with the largest gains in generalization \textit{Step Accuracy} appearing for harder tasks.

\section{Experiments}

\section{Experiments}
\label{sec:experiments}
The experiments are designed to address two key research questions.
First, \textbf{RQ1} evaluates whether the average $L_2$-norm of the counterfactual perturbation vectors ($\overline{||\perturb||}$) decreases as the model overfits the data, thereby providing further empirical validation for our hypothesis.
Second, \textbf{RQ2} evaluates the ability of the proposed counterfactual regularized loss, as defined in (\ref{eq:regularized_loss2}), to mitigate overfitting when compared to existing regularization techniques.

% The experiments are designed to address three key research questions. First, \textbf{RQ1} investigates whether the mean perturbation vector norm decreases as the model overfits the data, aiming to further validate our intuition. Second, \textbf{RQ2} explores whether the mean perturbation vector norm can be effectively leveraged as a regularization term during training, offering insights into its potential role in mitigating overfitting. Finally, \textbf{RQ3} examines whether our counterfactual regularizer enables the model to achieve superior performance compared to existing regularization methods, thus highlighting its practical advantage.

\subsection{Experimental Setup}
\textbf{\textit{Datasets, Models, and Tasks.}}
The experiments are conducted on three datasets: \textit{Water Potability}~\cite{kadiwal2020waterpotability}, \textit{Phomene}~\cite{phomene}, and \textit{CIFAR-10}~\cite{krizhevsky2009learning}. For \textit{Water Potability} and \textit{Phomene}, we randomly select $80\%$ of the samples for the training set, and the remaining $20\%$ for the test set, \textit{CIFAR-10} comes already split. Furthermore, we consider the following models: Logistic Regression, Multi-Layer Perceptron (MLP) with 100 and 30 neurons on each hidden layer, and PreactResNet-18~\cite{he2016cvecvv} as a Convolutional Neural Network (CNN) architecture.
We focus on binary classification tasks and leave the extension to multiclass scenarios for future work. However, for datasets that are inherently multiclass, we transform the problem into a binary classification task by selecting two classes, aligning with our assumption.

\smallskip
\noindent\textbf{\textit{Evaluation Measures.}} To characterize the degree of overfitting, we use the test loss, as it serves as a reliable indicator of the model's generalization capability to unseen data. Additionally, we evaluate the predictive performance of each model using the test accuracy.

\smallskip
\noindent\textbf{\textit{Baselines.}} We compare CF-Reg with the following regularization techniques: L1 (``Lasso''), L2 (``Ridge''), and Dropout.

\smallskip
\noindent\textbf{\textit{Configurations.}}
For each model, we adopt specific configurations as follows.
\begin{itemize}
\item \textit{Logistic Regression:} To induce overfitting in the model, we artificially increase the dimensionality of the data beyond the number of training samples by applying a polynomial feature expansion. This approach ensures that the model has enough capacity to overfit the training data, allowing us to analyze the impact of our counterfactual regularizer. The degree of the polynomial is chosen as the smallest degree that makes the number of features greater than the number of data.
\item \textit{Neural Networks (MLP and CNN):} To take advantage of the closed-form solution for computing the optimal perturbation vector as defined in (\ref{eq:opt-delta}), we use a local linear approximation of the neural network models. Hence, given an instance $\inst_i$, we consider the (optimal) counterfactual not with respect to $\model$ but with respect to:
\begin{equation}
\label{eq:taylor}
    \model^{lin}(\inst) = \model(\inst_i) + \nabla_{\inst}\model(\inst_i)(\inst - \inst_i),
\end{equation}
where $\model^{lin}$ represents the first-order Taylor approximation of $\model$ at $\inst_i$.
Note that this step is unnecessary for Logistic Regression, as it is inherently a linear model.
\end{itemize}

\smallskip
\noindent \textbf{\textit{Implementation Details.}} We run all experiments on a machine equipped with an AMD Ryzen 9 7900 12-Core Processor and an NVIDIA GeForce RTX 4090 GPU. Our implementation is based on the PyTorch Lightning framework. We use stochastic gradient descent as the optimizer with a learning rate of $\eta = 0.001$ and no weight decay. We use a batch size of $128$. The training and test steps are conducted for $6000$ epochs on the \textit{Water Potability} and \textit{Phoneme} datasets, while for the \textit{CIFAR-10} dataset, they are performed for $200$ epochs.
Finally, the contribution $w_i^{\varepsilon}$ of each training point $\inst_i$ is uniformly set as $w_i^{\varepsilon} = 1~\forall i\in \{1,\ldots,m\}$.

The source code implementation for our experiments is available at the following GitHub repository: \url{https://anonymous.4open.science/r/COCE-80B4/README.md} 

\subsection{RQ1: Counterfactual Perturbation vs. Overfitting}
To address \textbf{RQ1}, we analyze the relationship between the test loss and the average $L_2$-norm of the counterfactual perturbation vectors ($\overline{||\perturb||}$) over training epochs.

In particular, Figure~\ref{fig:delta_loss_epochs} depicts the evolution of $\overline{||\perturb||}$ alongside the test loss for an MLP trained \textit{without} regularization on the \textit{Water Potability} dataset. 
\begin{figure}[ht]
    \centering
    \includegraphics[width=0.85\linewidth]{img/delta_loss_epochs.png}
    \caption{The average counterfactual perturbation vector $\overline{||\perturb||}$ (left $y$-axis) and the cross-entropy test loss (right $y$-axis) over training epochs ($x$-axis) for an MLP trained on the \textit{Water Potability} dataset \textit{without} regularization.}
    \label{fig:delta_loss_epochs}
\end{figure}

The plot shows a clear trend as the model starts to overfit the data (evidenced by an increase in test loss). 
Notably, $\overline{||\perturb||}$ begins to decrease, which aligns with the hypothesis that the average distance to the optimal counterfactual example gets smaller as the model's decision boundary becomes increasingly adherent to the training data.

It is worth noting that this trend is heavily influenced by the choice of the counterfactual generator model. In particular, the relationship between $\overline{||\perturb||}$ and the degree of overfitting may become even more pronounced when leveraging more accurate counterfactual generators. However, these models often come at the cost of higher computational complexity, and their exploration is left to future work.

Nonetheless, we expect that $\overline{||\perturb||}$ will eventually stabilize at a plateau, as the average $L_2$-norm of the optimal counterfactual perturbations cannot vanish to zero.

% Additionally, the choice of employing the score-based counterfactual explanation framework to generate counterfactuals was driven to promote computational efficiency.

% Future enhancements to the framework may involve adopting models capable of generating more precise counterfactuals. While such approaches may yield to performance improvements, they are likely to come at the cost of increased computational complexity.


\subsection{RQ2: Counterfactual Regularization Performance}
To answer \textbf{RQ2}, we evaluate the effectiveness of the proposed counterfactual regularization (CF-Reg) by comparing its performance against existing baselines: unregularized training loss (No-Reg), L1 regularization (L1-Reg), L2 regularization (L2-Reg), and Dropout.
Specifically, for each model and dataset combination, Table~\ref{tab:regularization_comparison} presents the mean value and standard deviation of test accuracy achieved by each method across 5 random initialization. 

The table illustrates that our regularization technique consistently delivers better results than existing methods across all evaluated scenarios, except for one case -- i.e., Logistic Regression on the \textit{Phomene} dataset. 
However, this setting exhibits an unusual pattern, as the highest model accuracy is achieved without any regularization. Even in this case, CF-Reg still surpasses other regularization baselines.

From the results above, we derive the following key insights. First, CF-Reg proves to be effective across various model types, ranging from simple linear models (Logistic Regression) to deep architectures like MLPs and CNNs, and across diverse datasets, including both tabular and image data. 
Second, CF-Reg's strong performance on the \textit{Water} dataset with Logistic Regression suggests that its benefits may be more pronounced when applied to simpler models. However, the unexpected outcome on the \textit{Phoneme} dataset calls for further investigation into this phenomenon.


\begin{table*}[h!]
    \centering
    \caption{Mean value and standard deviation of test accuracy across 5 random initializations for different model, dataset, and regularization method. The best results are highlighted in \textbf{bold}.}
    \label{tab:regularization_comparison}
    \begin{tabular}{|c|c|c|c|c|c|c|}
        \hline
        \textbf{Model} & \textbf{Dataset} & \textbf{No-Reg} & \textbf{L1-Reg} & \textbf{L2-Reg} & \textbf{Dropout} & \textbf{CF-Reg (ours)} \\ \hline
        Logistic Regression   & \textit{Water}   & $0.6595 \pm 0.0038$   & $0.6729 \pm 0.0056$   & $0.6756 \pm 0.0046$  & N/A    & $\mathbf{0.6918 \pm 0.0036}$                     \\ \hline
        MLP   & \textit{Water}   & $0.6756 \pm 0.0042$   & $0.6790 \pm 0.0058$   & $0.6790 \pm 0.0023$  & $0.6750 \pm 0.0036$    & $\mathbf{0.6802 \pm 0.0046}$                    \\ \hline
%        MLP   & \textit{Adult}   & $0.8404 \pm 0.0010$   & $\mathbf{0.8495 \pm 0.0007}$   & $0.8489 \pm 0.0014$  & $\mathbf{0.8495 \pm 0.0016}$     & $0.8449 \pm 0.0019$                    \\ \hline
        Logistic Regression   & \textit{Phomene}   & $\mathbf{0.8148 \pm 0.0020}$   & $0.8041 \pm 0.0028$   & $0.7835 \pm 0.0176$  & N/A    & $0.8098 \pm 0.0055$                     \\ \hline
        MLP   & \textit{Phomene}   & $0.8677 \pm 0.0033$   & $0.8374 \pm 0.0080$   & $0.8673 \pm 0.0045$  & $0.8672 \pm 0.0042$     & $\mathbf{0.8718 \pm 0.0040}$                    \\ \hline
        CNN   & \textit{CIFAR-10} & $0.6670 \pm 0.0233$   & $0.6229 \pm 0.0850$   & $0.7348 \pm 0.0365$   & N/A    & $\mathbf{0.7427 \pm 0.0571}$                     \\ \hline
    \end{tabular}
\end{table*}

\begin{table*}[htb!]
    \centering
    \caption{Hyperparameter configurations utilized for the generation of Table \ref{tab:regularization_comparison}. For our regularization the hyperparameters are reported as $\mathbf{\alpha/\beta}$.}
    \label{tab:performance_parameters}
    \begin{tabular}{|c|c|c|c|c|c|c|}
        \hline
        \textbf{Model} & \textbf{Dataset} & \textbf{No-Reg} & \textbf{L1-Reg} & \textbf{L2-Reg} & \textbf{Dropout} & \textbf{CF-Reg (ours)} \\ \hline
        Logistic Regression   & \textit{Water}   & N/A   & $0.0093$   & $0.6927$  & N/A    & $0.3791/1.0355$                     \\ \hline
        MLP   & \textit{Water}   & N/A   & $0.0007$   & $0.0022$  & $0.0002$    & $0.2567/1.9775$                    \\ \hline
        Logistic Regression   &
        \textit{Phomene}   & N/A   & $0.0097$   & $0.7979$  & N/A    & $0.0571/1.8516$                     \\ \hline
        MLP   & \textit{Phomene}   & N/A   & $0.0007$   & $4.24\cdot10^{-5}$  & $0.0015$    & $0.0516/2.2700$                    \\ \hline
       % MLP   & \textit{Adult}   & N/A   & $0.0018$   & $0.0018$  & $0.0601$     & $0.0764/2.2068$                    \\ \hline
        CNN   & \textit{CIFAR-10} & N/A   & $0.0050$   & $0.0864$ & N/A    & $0.3018/
        2.1502$                     \\ \hline
    \end{tabular}
\end{table*}

\begin{table*}[htb!]
    \centering
    \caption{Mean value and standard deviation of training time across 5 different runs. The reported time (in seconds) corresponds to the generation of each entry in Table \ref{tab:regularization_comparison}. Times are }
    \label{tab:times}
    \begin{tabular}{|c|c|c|c|c|c|c|}
        \hline
        \textbf{Model} & \textbf{Dataset} & \textbf{No-Reg} & \textbf{L1-Reg} & \textbf{L2-Reg} & \textbf{Dropout} & \textbf{CF-Reg (ours)} \\ \hline
        Logistic Regression   & \textit{Water}   & $222.98 \pm 1.07$   & $239.94 \pm 2.59$   & $241.60 \pm 1.88$  & N/A    & $251.50 \pm 1.93$                     \\ \hline
        MLP   & \textit{Water}   & $225.71 \pm 3.85$   & $250.13 \pm 4.44$   & $255.78 \pm 2.38$  & $237.83 \pm 3.45$    & $266.48 \pm 3.46$                    \\ \hline
        Logistic Regression   & \textit{Phomene}   & $266.39 \pm 0.82$ & $367.52 \pm 6.85$   & $361.69 \pm 4.04$  & N/A   & $310.48 \pm 0.76$                    \\ \hline
        MLP   &
        \textit{Phomene} & $335.62 \pm 1.77$   & $390.86 \pm 2.11$   & $393.96 \pm 1.95$ & $363.51 \pm 5.07$    & $403.14 \pm 1.92$                     \\ \hline
       % MLP   & \textit{Adult}   & N/A   & $0.0018$   & $0.0018$  & $0.0601$     & $0.0764/2.2068$                    \\ \hline
        CNN   & \textit{CIFAR-10} & $370.09 \pm 0.18$   & $395.71 \pm 0.55$   & $401.38 \pm 0.16$ & N/A    & $1287.8 \pm 0.26$                     \\ \hline
    \end{tabular}
\end{table*}

\subsection{Feasibility of our Method}
A crucial requirement for any regularization technique is that it should impose minimal impact on the overall training process.
In this respect, CF-Reg introduces an overhead that depends on the time required to find the optimal counterfactual example for each training instance. 
As such, the more sophisticated the counterfactual generator model probed during training the higher would be the time required. However, a more advanced counterfactual generator might provide a more effective regularization. We discuss this trade-off in more details in Section~\ref{sec:discussion}.

Table~\ref{tab:times} presents the average training time ($\pm$ standard deviation) for each model and dataset combination listed in Table~\ref{tab:regularization_comparison}.
We can observe that the higher accuracy achieved by CF-Reg using the score-based counterfactual generator comes with only minimal overhead. However, when applied to deep neural networks with many hidden layers, such as \textit{PreactResNet-18}, the forward derivative computation required for the linearization of the network introduces a more noticeable computational cost, explaining the longer training times in the table.

\subsection{Hyperparameter Sensitivity Analysis}
The proposed counterfactual regularization technique relies on two key hyperparameters: $\alpha$ and $\beta$. The former is intrinsic to the loss formulation defined in (\ref{eq:cf-train}), while the latter is closely tied to the choice of the score-based counterfactual explanation method used.

Figure~\ref{fig:test_alpha_beta} illustrates how the test accuracy of an MLP trained on the \textit{Water Potability} dataset changes for different combinations of $\alpha$ and $\beta$.

\begin{figure}[ht]
    \centering
    \includegraphics[width=0.85\linewidth]{img/test_acc_alpha_beta.png}
    \caption{The test accuracy of an MLP trained on the \textit{Water Potability} dataset, evaluated while varying the weight of our counterfactual regularizer ($\alpha$) for different values of $\beta$.}
    \label{fig:test_alpha_beta}
\end{figure}

We observe that, for a fixed $\beta$, increasing the weight of our counterfactual regularizer ($\alpha$) can slightly improve test accuracy until a sudden drop is noticed for $\alpha > 0.1$.
This behavior was expected, as the impact of our penalty, like any regularization term, can be disruptive if not properly controlled.

Moreover, this finding further demonstrates that our regularization method, CF-Reg, is inherently data-driven. Therefore, it requires specific fine-tuning based on the combination of the model and dataset at hand.


% \begin{itemize}
%     \item maham vs gurobi, ham and et
%     \item ablation studies: 
%     \begin{itemize}
%         \item compare action selection strategy with agents selecting actions in random order
%         \item compare SL with REINFORCE
%         \item Agent Context Encoder
%         \item MatNet with parameter sharing etc
%     \end{itemize}
%     \item more or less agents then required tours (benchmark against what?)
%     \item extension to multiple depots with balanced packing stations?
%     \item softmax per agent of over all agent-action combinations
% \end{itemize}



\section{Conclusion and Future Work}
In this work, we introduced the first neural solver for the min-max Mixed-Shelves Picker Routing Problem. The core of our approach is the integration of a hierarchical and parallel decoding mechanism capable of efficiently constructing solutions over complex, multi-dimensional action spaces, such as those found in min-max MSPRP. While previous methods relied on sequential solution construction or parallel decision-making prone to conflicts, our approach achieves efficient and effective agent coordination, enabled by a novel Sequential Action Selection algorithm.

Our extensive experimental results, including traditional as well as neural solvers, demonstrate the superiority of MAHAM in both solution quality and inference speed, particularly for large-scale problem instances. These findings highlight the capabilities of neural solvers and prove them as a strong alternative to hand-crafted heuristics.

Future research directions include extending this approach to more dynamic warehouse environments with real-time demand fluctuations and exploring hybrid methods that integrate learning-based techniques with optimization heuristics for further performance improvements. Additionally, our framework could be adapted to other multi-agent combinatorial optimization problems beyond warehouse logistics, such as fleet routing and robotic task allocation.


\bibliographystyle{splncs04}
\bibliography{bib.bib}

\subsection{Lloyd-Max Algorithm}
\label{subsec:Lloyd-Max}
For a given quantization bitwidth $B$ and an operand $\bm{X}$, the Lloyd-Max algorithm finds $2^B$ quantization levels $\{\hat{x}_i\}_{i=1}^{2^B}$ such that quantizing $\bm{X}$ by rounding each scalar in $\bm{X}$ to the nearest quantization level minimizes the quantization MSE. 

The algorithm starts with an initial guess of quantization levels and then iteratively computes quantization thresholds $\{\tau_i\}_{i=1}^{2^B-1}$ and updates quantization levels $\{\hat{x}_i\}_{i=1}^{2^B}$. Specifically, at iteration $n$, thresholds are set to the midpoints of the previous iteration's levels:
\begin{align*}
    \tau_i^{(n)}=\frac{\hat{x}_i^{(n-1)}+\hat{x}_{i+1}^{(n-1)}}2 \text{ for } i=1\ldots 2^B-1
\end{align*}
Subsequently, the quantization levels are re-computed as conditional means of the data regions defined by the new thresholds:
\begin{align*}
    \hat{x}_i^{(n)}=\mathbb{E}\left[ \bm{X} \big| \bm{X}\in [\tau_{i-1}^{(n)},\tau_i^{(n)}] \right] \text{ for } i=1\ldots 2^B
\end{align*}
where to satisfy boundary conditions we have $\tau_0=-\infty$ and $\tau_{2^B}=\infty$. The algorithm iterates the above steps until convergence.

Figure \ref{fig:lm_quant} compares the quantization levels of a $7$-bit floating point (E3M3) quantizer (left) to a $7$-bit Lloyd-Max quantizer (right) when quantizing a layer of weights from the GPT3-126M model at a per-tensor granularity. As shown, the Lloyd-Max quantizer achieves substantially lower quantization MSE. Further, Table \ref{tab:FP7_vs_LM7} shows the superior perplexity achieved by Lloyd-Max quantizers for bitwidths of $7$, $6$ and $5$. The difference between the quantizers is clear at 5 bits, where per-tensor FP quantization incurs a drastic and unacceptable increase in perplexity, while Lloyd-Max quantization incurs a much smaller increase. Nevertheless, we note that even the optimal Lloyd-Max quantizer incurs a notable ($\sim 1.5$) increase in perplexity due to the coarse granularity of quantization. 

\begin{figure}[h]
  \centering
  \includegraphics[width=0.7\linewidth]{sections/figures/LM7_FP7.pdf}
  \caption{\small Quantization levels and the corresponding quantization MSE of Floating Point (left) vs Lloyd-Max (right) Quantizers for a layer of weights in the GPT3-126M model.}
  \label{fig:lm_quant}
\end{figure}

\begin{table}[h]\scriptsize
\begin{center}
\caption{\label{tab:FP7_vs_LM7} \small Comparing perplexity (lower is better) achieved by floating point quantizers and Lloyd-Max quantizers on a GPT3-126M model for the Wikitext-103 dataset.}
\begin{tabular}{c|cc|c}
\hline
 \multirow{2}{*}{\textbf{Bitwidth}} & \multicolumn{2}{|c|}{\textbf{Floating-Point Quantizer}} & \textbf{Lloyd-Max Quantizer} \\
 & Best Format & Wikitext-103 Perplexity & Wikitext-103 Perplexity \\
\hline
7 & E3M3 & 18.32 & 18.27 \\
6 & E3M2 & 19.07 & 18.51 \\
5 & E4M0 & 43.89 & 19.71 \\
\hline
\end{tabular}
\end{center}
\end{table}

\subsection{Proof of Local Optimality of LO-BCQ}
\label{subsec:lobcq_opt_proof}
For a given block $\bm{b}_j$, the quantization MSE during LO-BCQ can be empirically evaluated as $\frac{1}{L_b}\lVert \bm{b}_j- \bm{\hat{b}}_j\rVert^2_2$ where $\bm{\hat{b}}_j$ is computed from equation (\ref{eq:clustered_quantization_definition}) as $C_{f(\bm{b}_j)}(\bm{b}_j)$. Further, for a given block cluster $\mathcal{B}_i$, we compute the quantization MSE as $\frac{1}{|\mathcal{B}_{i}|}\sum_{\bm{b} \in \mathcal{B}_{i}} \frac{1}{L_b}\lVert \bm{b}- C_i^{(n)}(\bm{b})\rVert^2_2$. Therefore, at the end of iteration $n$, we evaluate the overall quantization MSE $J^{(n)}$ for a given operand $\bm{X}$ composed of $N_c$ block clusters as:
\begin{align*}
    \label{eq:mse_iter_n}
    J^{(n)} = \frac{1}{N_c} \sum_{i=1}^{N_c} \frac{1}{|\mathcal{B}_{i}^{(n)}|}\sum_{\bm{v} \in \mathcal{B}_{i}^{(n)}} \frac{1}{L_b}\lVert \bm{b}- B_i^{(n)}(\bm{b})\rVert^2_2
\end{align*}

At the end of iteration $n$, the codebooks are updated from $\mathcal{C}^{(n-1)}$ to $\mathcal{C}^{(n)}$. However, the mapping of a given vector $\bm{b}_j$ to quantizers $\mathcal{C}^{(n)}$ remains as  $f^{(n)}(\bm{b}_j)$. At the next iteration, during the vector clustering step, $f^{(n+1)}(\bm{b}_j)$ finds new mapping of $\bm{b}_j$ to updated codebooks $\mathcal{C}^{(n)}$ such that the quantization MSE over the candidate codebooks is minimized. Therefore, we obtain the following result for $\bm{b}_j$:
\begin{align*}
\frac{1}{L_b}\lVert \bm{b}_j - C_{f^{(n+1)}(\bm{b}_j)}^{(n)}(\bm{b}_j)\rVert^2_2 \le \frac{1}{L_b}\lVert \bm{b}_j - C_{f^{(n)}(\bm{b}_j)}^{(n)}(\bm{b}_j)\rVert^2_2
\end{align*}

That is, quantizing $\bm{b}_j$ at the end of the block clustering step of iteration $n+1$ results in lower quantization MSE compared to quantizing at the end of iteration $n$. Since this is true for all $\bm{b} \in \bm{X}$, we assert the following:
\begin{equation}
\begin{split}
\label{eq:mse_ineq_1}
    \tilde{J}^{(n+1)} &= \frac{1}{N_c} \sum_{i=1}^{N_c} \frac{1}{|\mathcal{B}_{i}^{(n+1)}|}\sum_{\bm{b} \in \mathcal{B}_{i}^{(n+1)}} \frac{1}{L_b}\lVert \bm{b} - C_i^{(n)}(b)\rVert^2_2 \le J^{(n)}
\end{split}
\end{equation}
where $\tilde{J}^{(n+1)}$ is the the quantization MSE after the vector clustering step at iteration $n+1$.

Next, during the codebook update step (\ref{eq:quantizers_update}) at iteration $n+1$, the per-cluster codebooks $\mathcal{C}^{(n)}$ are updated to $\mathcal{C}^{(n+1)}$ by invoking the Lloyd-Max algorithm \citep{Lloyd}. We know that for any given value distribution, the Lloyd-Max algorithm minimizes the quantization MSE. Therefore, for a given vector cluster $\mathcal{B}_i$ we obtain the following result:

\begin{equation}
    \frac{1}{|\mathcal{B}_{i}^{(n+1)}|}\sum_{\bm{b} \in \mathcal{B}_{i}^{(n+1)}} \frac{1}{L_b}\lVert \bm{b}- C_i^{(n+1)}(\bm{b})\rVert^2_2 \le \frac{1}{|\mathcal{B}_{i}^{(n+1)}|}\sum_{\bm{b} \in \mathcal{B}_{i}^{(n+1)}} \frac{1}{L_b}\lVert \bm{b}- C_i^{(n)}(\bm{b})\rVert^2_2
\end{equation}

The above equation states that quantizing the given block cluster $\mathcal{B}_i$ after updating the associated codebook from $C_i^{(n)}$ to $C_i^{(n+1)}$ results in lower quantization MSE. Since this is true for all the block clusters, we derive the following result: 
\begin{equation}
\begin{split}
\label{eq:mse_ineq_2}
     J^{(n+1)} &= \frac{1}{N_c} \sum_{i=1}^{N_c} \frac{1}{|\mathcal{B}_{i}^{(n+1)}|}\sum_{\bm{b} \in \mathcal{B}_{i}^{(n+1)}} \frac{1}{L_b}\lVert \bm{b}- C_i^{(n+1)}(\bm{b})\rVert^2_2  \le \tilde{J}^{(n+1)}   
\end{split}
\end{equation}

Following (\ref{eq:mse_ineq_1}) and (\ref{eq:mse_ineq_2}), we find that the quantization MSE is non-increasing for each iteration, that is, $J^{(1)} \ge J^{(2)} \ge J^{(3)} \ge \ldots \ge J^{(M)}$ where $M$ is the maximum number of iterations. 
%Therefore, we can say that if the algorithm converges, then it must be that it has converged to a local minimum. 
\hfill $\blacksquare$


\begin{figure}
    \begin{center}
    \includegraphics[width=0.5\textwidth]{sections//figures/mse_vs_iter.pdf}
    \end{center}
    \caption{\small NMSE vs iterations during LO-BCQ compared to other block quantization proposals}
    \label{fig:nmse_vs_iter}
\end{figure}

Figure \ref{fig:nmse_vs_iter} shows the empirical convergence of LO-BCQ across several block lengths and number of codebooks. Also, the MSE achieved by LO-BCQ is compared to baselines such as MXFP and VSQ. As shown, LO-BCQ converges to a lower MSE than the baselines. Further, we achieve better convergence for larger number of codebooks ($N_c$) and for a smaller block length ($L_b$), both of which increase the bitwidth of BCQ (see Eq \ref{eq:bitwidth_bcq}).


\subsection{Additional Accuracy Results}
%Table \ref{tab:lobcq_config} lists the various LOBCQ configurations and their corresponding bitwidths.
\begin{table}
\setlength{\tabcolsep}{4.75pt}
\begin{center}
\caption{\label{tab:lobcq_config} Various LO-BCQ configurations and their bitwidths.}
\begin{tabular}{|c||c|c|c|c||c|c||c|} 
\hline
 & \multicolumn{4}{|c||}{$L_b=8$} & \multicolumn{2}{|c||}{$L_b=4$} & $L_b=2$ \\
 \hline
 \backslashbox{$L_A$\kern-1em}{\kern-1em$N_c$} & 2 & 4 & 8 & 16 & 2 & 4 & 2 \\
 \hline
 64 & 4.25 & 4.375 & 4.5 & 4.625 & 4.375 & 4.625 & 4.625\\
 \hline
 32 & 4.375 & 4.5 & 4.625& 4.75 & 4.5 & 4.75 & 4.75 \\
 \hline
 16 & 4.625 & 4.75& 4.875 & 5 & 4.75 & 5 & 5 \\
 \hline
\end{tabular}
\end{center}
\end{table}

%\subsection{Perplexity achieved by various LO-BCQ configurations on Wikitext-103 dataset}

\begin{table} \centering
\begin{tabular}{|c||c|c|c|c||c|c||c|} 
\hline
 $L_b \rightarrow$& \multicolumn{4}{c||}{8} & \multicolumn{2}{c||}{4} & 2\\
 \hline
 \backslashbox{$L_A$\kern-1em}{\kern-1em$N_c$} & 2 & 4 & 8 & 16 & 2 & 4 & 2  \\
 %$N_c \rightarrow$ & 2 & 4 & 8 & 16 & 2 & 4 & 2 \\
 \hline
 \hline
 \multicolumn{8}{c}{GPT3-1.3B (FP32 PPL = 9.98)} \\ 
 \hline
 \hline
 64 & 10.40 & 10.23 & 10.17 & 10.15 &  10.28 & 10.18 & 10.19 \\
 \hline
 32 & 10.25 & 10.20 & 10.15 & 10.12 &  10.23 & 10.17 & 10.17 \\
 \hline
 16 & 10.22 & 10.16 & 10.10 & 10.09 &  10.21 & 10.14 & 10.16 \\
 \hline
  \hline
 \multicolumn{8}{c}{GPT3-8B (FP32 PPL = 7.38)} \\ 
 \hline
 \hline
 64 & 7.61 & 7.52 & 7.48 &  7.47 &  7.55 &  7.49 & 7.50 \\
 \hline
 32 & 7.52 & 7.50 & 7.46 &  7.45 &  7.52 &  7.48 & 7.48  \\
 \hline
 16 & 7.51 & 7.48 & 7.44 &  7.44 &  7.51 &  7.49 & 7.47  \\
 \hline
\end{tabular}
\caption{\label{tab:ppl_gpt3_abalation} Wikitext-103 perplexity across GPT3-1.3B and 8B models.}
\end{table}

\begin{table} \centering
\begin{tabular}{|c||c|c|c|c||} 
\hline
 $L_b \rightarrow$& \multicolumn{4}{c||}{8}\\
 \hline
 \backslashbox{$L_A$\kern-1em}{\kern-1em$N_c$} & 2 & 4 & 8 & 16 \\
 %$N_c \rightarrow$ & 2 & 4 & 8 & 16 & 2 & 4 & 2 \\
 \hline
 \hline
 \multicolumn{5}{|c|}{Llama2-7B (FP32 PPL = 5.06)} \\ 
 \hline
 \hline
 64 & 5.31 & 5.26 & 5.19 & 5.18  \\
 \hline
 32 & 5.23 & 5.25 & 5.18 & 5.15  \\
 \hline
 16 & 5.23 & 5.19 & 5.16 & 5.14  \\
 \hline
 \multicolumn{5}{|c|}{Nemotron4-15B (FP32 PPL = 5.87)} \\ 
 \hline
 \hline
 64  & 6.3 & 6.20 & 6.13 & 6.08  \\
 \hline
 32  & 6.24 & 6.12 & 6.07 & 6.03  \\
 \hline
 16  & 6.12 & 6.14 & 6.04 & 6.02  \\
 \hline
 \multicolumn{5}{|c|}{Nemotron4-340B (FP32 PPL = 3.48)} \\ 
 \hline
 \hline
 64 & 3.67 & 3.62 & 3.60 & 3.59 \\
 \hline
 32 & 3.63 & 3.61 & 3.59 & 3.56 \\
 \hline
 16 & 3.61 & 3.58 & 3.57 & 3.55 \\
 \hline
\end{tabular}
\caption{\label{tab:ppl_llama7B_nemo15B} Wikitext-103 perplexity compared to FP32 baseline in Llama2-7B and Nemotron4-15B, 340B models}
\end{table}

%\subsection{Perplexity achieved by various LO-BCQ configurations on MMLU dataset}


\begin{table} \centering
\begin{tabular}{|c||c|c|c|c||c|c|c|c|} 
\hline
 $L_b \rightarrow$& \multicolumn{4}{c||}{8} & \multicolumn{4}{c||}{8}\\
 \hline
 \backslashbox{$L_A$\kern-1em}{\kern-1em$N_c$} & 2 & 4 & 8 & 16 & 2 & 4 & 8 & 16  \\
 %$N_c \rightarrow$ & 2 & 4 & 8 & 16 & 2 & 4 & 2 \\
 \hline
 \hline
 \multicolumn{5}{|c|}{Llama2-7B (FP32 Accuracy = 45.8\%)} & \multicolumn{4}{|c|}{Llama2-70B (FP32 Accuracy = 69.12\%)} \\ 
 \hline
 \hline
 64 & 43.9 & 43.4 & 43.9 & 44.9 & 68.07 & 68.27 & 68.17 & 68.75 \\
 \hline
 32 & 44.5 & 43.8 & 44.9 & 44.5 & 68.37 & 68.51 & 68.35 & 68.27  \\
 \hline
 16 & 43.9 & 42.7 & 44.9 & 45 & 68.12 & 68.77 & 68.31 & 68.59  \\
 \hline
 \hline
 \multicolumn{5}{|c|}{GPT3-22B (FP32 Accuracy = 38.75\%)} & \multicolumn{4}{|c|}{Nemotron4-15B (FP32 Accuracy = 64.3\%)} \\ 
 \hline
 \hline
 64 & 36.71 & 38.85 & 38.13 & 38.92 & 63.17 & 62.36 & 63.72 & 64.09 \\
 \hline
 32 & 37.95 & 38.69 & 39.45 & 38.34 & 64.05 & 62.30 & 63.8 & 64.33  \\
 \hline
 16 & 38.88 & 38.80 & 38.31 & 38.92 & 63.22 & 63.51 & 63.93 & 64.43  \\
 \hline
\end{tabular}
\caption{\label{tab:mmlu_abalation} Accuracy on MMLU dataset across GPT3-22B, Llama2-7B, 70B and Nemotron4-15B models.}
\end{table}


%\subsection{Perplexity achieved by various LO-BCQ configurations on LM evaluation harness}

\begin{table} \centering
\begin{tabular}{|c||c|c|c|c||c|c|c|c|} 
\hline
 $L_b \rightarrow$& \multicolumn{4}{c||}{8} & \multicolumn{4}{c||}{8}\\
 \hline
 \backslashbox{$L_A$\kern-1em}{\kern-1em$N_c$} & 2 & 4 & 8 & 16 & 2 & 4 & 8 & 16  \\
 %$N_c \rightarrow$ & 2 & 4 & 8 & 16 & 2 & 4 & 2 \\
 \hline
 \hline
 \multicolumn{5}{|c|}{Race (FP32 Accuracy = 37.51\%)} & \multicolumn{4}{|c|}{Boolq (FP32 Accuracy = 64.62\%)} \\ 
 \hline
 \hline
 64 & 36.94 & 37.13 & 36.27 & 37.13 & 63.73 & 62.26 & 63.49 & 63.36 \\
 \hline
 32 & 37.03 & 36.36 & 36.08 & 37.03 & 62.54 & 63.51 & 63.49 & 63.55  \\
 \hline
 16 & 37.03 & 37.03 & 36.46 & 37.03 & 61.1 & 63.79 & 63.58 & 63.33  \\
 \hline
 \hline
 \multicolumn{5}{|c|}{Winogrande (FP32 Accuracy = 58.01\%)} & \multicolumn{4}{|c|}{Piqa (FP32 Accuracy = 74.21\%)} \\ 
 \hline
 \hline
 64 & 58.17 & 57.22 & 57.85 & 58.33 & 73.01 & 73.07 & 73.07 & 72.80 \\
 \hline
 32 & 59.12 & 58.09 & 57.85 & 58.41 & 73.01 & 73.94 & 72.74 & 73.18  \\
 \hline
 16 & 57.93 & 58.88 & 57.93 & 58.56 & 73.94 & 72.80 & 73.01 & 73.94  \\
 \hline
\end{tabular}
\caption{\label{tab:mmlu_abalation} Accuracy on LM evaluation harness tasks on GPT3-1.3B model.}
\end{table}

\begin{table} \centering
\begin{tabular}{|c||c|c|c|c||c|c|c|c|} 
\hline
 $L_b \rightarrow$& \multicolumn{4}{c||}{8} & \multicolumn{4}{c||}{8}\\
 \hline
 \backslashbox{$L_A$\kern-1em}{\kern-1em$N_c$} & 2 & 4 & 8 & 16 & 2 & 4 & 8 & 16  \\
 %$N_c \rightarrow$ & 2 & 4 & 8 & 16 & 2 & 4 & 2 \\
 \hline
 \hline
 \multicolumn{5}{|c|}{Race (FP32 Accuracy = 41.34\%)} & \multicolumn{4}{|c|}{Boolq (FP32 Accuracy = 68.32\%)} \\ 
 \hline
 \hline
 64 & 40.48 & 40.10 & 39.43 & 39.90 & 69.20 & 68.41 & 69.45 & 68.56 \\
 \hline
 32 & 39.52 & 39.52 & 40.77 & 39.62 & 68.32 & 67.43 & 68.17 & 69.30  \\
 \hline
 16 & 39.81 & 39.71 & 39.90 & 40.38 & 68.10 & 66.33 & 69.51 & 69.42  \\
 \hline
 \hline
 \multicolumn{5}{|c|}{Winogrande (FP32 Accuracy = 67.88\%)} & \multicolumn{4}{|c|}{Piqa (FP32 Accuracy = 78.78\%)} \\ 
 \hline
 \hline
 64 & 66.85 & 66.61 & 67.72 & 67.88 & 77.31 & 77.42 & 77.75 & 77.64 \\
 \hline
 32 & 67.25 & 67.72 & 67.72 & 67.00 & 77.31 & 77.04 & 77.80 & 77.37  \\
 \hline
 16 & 68.11 & 68.90 & 67.88 & 67.48 & 77.37 & 78.13 & 78.13 & 77.69  \\
 \hline
\end{tabular}
\caption{\label{tab:mmlu_abalation} Accuracy on LM evaluation harness tasks on GPT3-8B model.}
\end{table}

\begin{table} \centering
\begin{tabular}{|c||c|c|c|c||c|c|c|c|} 
\hline
 $L_b \rightarrow$& \multicolumn{4}{c||}{8} & \multicolumn{4}{c||}{8}\\
 \hline
 \backslashbox{$L_A$\kern-1em}{\kern-1em$N_c$} & 2 & 4 & 8 & 16 & 2 & 4 & 8 & 16  \\
 %$N_c \rightarrow$ & 2 & 4 & 8 & 16 & 2 & 4 & 2 \\
 \hline
 \hline
 \multicolumn{5}{|c|}{Race (FP32 Accuracy = 40.67\%)} & \multicolumn{4}{|c|}{Boolq (FP32 Accuracy = 76.54\%)} \\ 
 \hline
 \hline
 64 & 40.48 & 40.10 & 39.43 & 39.90 & 75.41 & 75.11 & 77.09 & 75.66 \\
 \hline
 32 & 39.52 & 39.52 & 40.77 & 39.62 & 76.02 & 76.02 & 75.96 & 75.35  \\
 \hline
 16 & 39.81 & 39.71 & 39.90 & 40.38 & 75.05 & 73.82 & 75.72 & 76.09  \\
 \hline
 \hline
 \multicolumn{5}{|c|}{Winogrande (FP32 Accuracy = 70.64\%)} & \multicolumn{4}{|c|}{Piqa (FP32 Accuracy = 79.16\%)} \\ 
 \hline
 \hline
 64 & 69.14 & 70.17 & 70.17 & 70.56 & 78.24 & 79.00 & 78.62 & 78.73 \\
 \hline
 32 & 70.96 & 69.69 & 71.27 & 69.30 & 78.56 & 79.49 & 79.16 & 78.89  \\
 \hline
 16 & 71.03 & 69.53 & 69.69 & 70.40 & 78.13 & 79.16 & 79.00 & 79.00  \\
 \hline
\end{tabular}
\caption{\label{tab:mmlu_abalation} Accuracy on LM evaluation harness tasks on GPT3-22B model.}
\end{table}

\begin{table} \centering
\begin{tabular}{|c||c|c|c|c||c|c|c|c|} 
\hline
 $L_b \rightarrow$& \multicolumn{4}{c||}{8} & \multicolumn{4}{c||}{8}\\
 \hline
 \backslashbox{$L_A$\kern-1em}{\kern-1em$N_c$} & 2 & 4 & 8 & 16 & 2 & 4 & 8 & 16  \\
 %$N_c \rightarrow$ & 2 & 4 & 8 & 16 & 2 & 4 & 2 \\
 \hline
 \hline
 \multicolumn{5}{|c|}{Race (FP32 Accuracy = 44.4\%)} & \multicolumn{4}{|c|}{Boolq (FP32 Accuracy = 79.29\%)} \\ 
 \hline
 \hline
 64 & 42.49 & 42.51 & 42.58 & 43.45 & 77.58 & 77.37 & 77.43 & 78.1 \\
 \hline
 32 & 43.35 & 42.49 & 43.64 & 43.73 & 77.86 & 75.32 & 77.28 & 77.86  \\
 \hline
 16 & 44.21 & 44.21 & 43.64 & 42.97 & 78.65 & 77 & 76.94 & 77.98  \\
 \hline
 \hline
 \multicolumn{5}{|c|}{Winogrande (FP32 Accuracy = 69.38\%)} & \multicolumn{4}{|c|}{Piqa (FP32 Accuracy = 78.07\%)} \\ 
 \hline
 \hline
 64 & 68.9 & 68.43 & 69.77 & 68.19 & 77.09 & 76.82 & 77.09 & 77.86 \\
 \hline
 32 & 69.38 & 68.51 & 68.82 & 68.90 & 78.07 & 76.71 & 78.07 & 77.86  \\
 \hline
 16 & 69.53 & 67.09 & 69.38 & 68.90 & 77.37 & 77.8 & 77.91 & 77.69  \\
 \hline
\end{tabular}
\caption{\label{tab:mmlu_abalation} Accuracy on LM evaluation harness tasks on Llama2-7B model.}
\end{table}

\begin{table} \centering
\begin{tabular}{|c||c|c|c|c||c|c|c|c|} 
\hline
 $L_b \rightarrow$& \multicolumn{4}{c||}{8} & \multicolumn{4}{c||}{8}\\
 \hline
 \backslashbox{$L_A$\kern-1em}{\kern-1em$N_c$} & 2 & 4 & 8 & 16 & 2 & 4 & 8 & 16  \\
 %$N_c \rightarrow$ & 2 & 4 & 8 & 16 & 2 & 4 & 2 \\
 \hline
 \hline
 \multicolumn{5}{|c|}{Race (FP32 Accuracy = 48.8\%)} & \multicolumn{4}{|c|}{Boolq (FP32 Accuracy = 85.23\%)} \\ 
 \hline
 \hline
 64 & 49.00 & 49.00 & 49.28 & 48.71 & 82.82 & 84.28 & 84.03 & 84.25 \\
 \hline
 32 & 49.57 & 48.52 & 48.33 & 49.28 & 83.85 & 84.46 & 84.31 & 84.93  \\
 \hline
 16 & 49.85 & 49.09 & 49.28 & 48.99 & 85.11 & 84.46 & 84.61 & 83.94  \\
 \hline
 \hline
 \multicolumn{5}{|c|}{Winogrande (FP32 Accuracy = 79.95\%)} & \multicolumn{4}{|c|}{Piqa (FP32 Accuracy = 81.56\%)} \\ 
 \hline
 \hline
 64 & 78.77 & 78.45 & 78.37 & 79.16 & 81.45 & 80.69 & 81.45 & 81.5 \\
 \hline
 32 & 78.45 & 79.01 & 78.69 & 80.66 & 81.56 & 80.58 & 81.18 & 81.34  \\
 \hline
 16 & 79.95 & 79.56 & 79.79 & 79.72 & 81.28 & 81.66 & 81.28 & 80.96  \\
 \hline
\end{tabular}
\caption{\label{tab:mmlu_abalation} Accuracy on LM evaluation harness tasks on Llama2-70B model.}
\end{table}

%\section{MSE Studies}
%\textcolor{red}{TODO}


\subsection{Number Formats and Quantization Method}
\label{subsec:numFormats_quantMethod}
\subsubsection{Integer Format}
An $n$-bit signed integer (INT) is typically represented with a 2s-complement format \citep{yao2022zeroquant,xiao2023smoothquant,dai2021vsq}, where the most significant bit denotes the sign.

\subsubsection{Floating Point Format}
An $n$-bit signed floating point (FP) number $x$ comprises of a 1-bit sign ($x_{\mathrm{sign}}$), $B_m$-bit mantissa ($x_{\mathrm{mant}}$) and $B_e$-bit exponent ($x_{\mathrm{exp}}$) such that $B_m+B_e=n-1$. The associated constant exponent bias ($E_{\mathrm{bias}}$) is computed as $(2^{{B_e}-1}-1)$. We denote this format as $E_{B_e}M_{B_m}$.  

\subsubsection{Quantization Scheme}
\label{subsec:quant_method}
A quantization scheme dictates how a given unquantized tensor is converted to its quantized representation. We consider FP formats for the purpose of illustration. Given an unquantized tensor $\bm{X}$ and an FP format $E_{B_e}M_{B_m}$, we first, we compute the quantization scale factor $s_X$ that maps the maximum absolute value of $\bm{X}$ to the maximum quantization level of the $E_{B_e}M_{B_m}$ format as follows:
\begin{align}
\label{eq:sf}
    s_X = \frac{\mathrm{max}(|\bm{X}|)}{\mathrm{max}(E_{B_e}M_{B_m})}
\end{align}
In the above equation, $|\cdot|$ denotes the absolute value function.

Next, we scale $\bm{X}$ by $s_X$ and quantize it to $\hat{\bm{X}}$ by rounding it to the nearest quantization level of $E_{B_e}M_{B_m}$ as:

\begin{align}
\label{eq:tensor_quant}
    \hat{\bm{X}} = \text{round-to-nearest}\left(\frac{\bm{X}}{s_X}, E_{B_e}M_{B_m}\right)
\end{align}

We perform dynamic max-scaled quantization \citep{wu2020integer}, where the scale factor $s$ for activations is dynamically computed during runtime.

\subsection{Vector Scaled Quantization}
\begin{wrapfigure}{r}{0.35\linewidth}
  \centering
  \includegraphics[width=\linewidth]{sections/figures/vsquant.jpg}
  \caption{\small Vectorwise decomposition for per-vector scaled quantization (VSQ \citep{dai2021vsq}).}
  \label{fig:vsquant}
\end{wrapfigure}
During VSQ \citep{dai2021vsq}, the operand tensors are decomposed into 1D vectors in a hardware friendly manner as shown in Figure \ref{fig:vsquant}. Since the decomposed tensors are used as operands in matrix multiplications during inference, it is beneficial to perform this decomposition along the reduction dimension of the multiplication. The vectorwise quantization is performed similar to tensorwise quantization described in Equations \ref{eq:sf} and \ref{eq:tensor_quant}, where a scale factor $s_v$ is required for each vector $\bm{v}$ that maps the maximum absolute value of that vector to the maximum quantization level. While smaller vector lengths can lead to larger accuracy gains, the associated memory and computational overheads due to the per-vector scale factors increases. To alleviate these overheads, VSQ \citep{dai2021vsq} proposed a second level quantization of the per-vector scale factors to unsigned integers, while MX \citep{rouhani2023shared} quantizes them to integer powers of 2 (denoted as $2^{INT}$).

\subsubsection{MX Format}
The MX format proposed in \citep{rouhani2023microscaling} introduces the concept of sub-block shifting. For every two scalar elements of $b$-bits each, there is a shared exponent bit. The value of this exponent bit is determined through an empirical analysis that targets minimizing quantization MSE. We note that the FP format $E_{1}M_{b}$ is strictly better than MX from an accuracy perspective since it allocates a dedicated exponent bit to each scalar as opposed to sharing it across two scalars. Therefore, we conservatively bound the accuracy of a $b+2$-bit signed MX format with that of a $E_{1}M_{b}$ format in our comparisons. For instance, we use E1M2 format as a proxy for MX4.

\begin{figure}
    \centering
    \includegraphics[width=1\linewidth]{sections//figures/BlockFormats.pdf}
    \caption{\small Comparing LO-BCQ to MX format.}
    \label{fig:block_formats}
\end{figure}

Figure \ref{fig:block_formats} compares our $4$-bit LO-BCQ block format to MX \citep{rouhani2023microscaling}. As shown, both LO-BCQ and MX decompose a given operand tensor into block arrays and each block array into blocks. Similar to MX, we find that per-block quantization ($L_b < L_A$) leads to better accuracy due to increased flexibility. While MX achieves this through per-block $1$-bit micro-scales, we associate a dedicated codebook to each block through a per-block codebook selector. Further, MX quantizes the per-block array scale-factor to E8M0 format without per-tensor scaling. In contrast during LO-BCQ, we find that per-tensor scaling combined with quantization of per-block array scale-factor to E4M3 format results in superior inference accuracy across models. 


\end{document}
