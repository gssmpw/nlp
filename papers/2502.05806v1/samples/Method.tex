\section{Method}

% 3.1 Preliminaries
% 3.2 Tree Structure Answer Distribution Estimator
    % divide conquer 思想
    % Training the QGen with Policy Gradient:两个reward实现目标
% 3.3 起个名
% 3.4 Cooperate with Multiple Baselines


% We first introduce background about this task and training framework. Then we introduce the proposed TSADE. Finally, we detail the appropriate rewards and the RL implementation. 


% Finally, we provide an explanation of how the Unified-Belief Mapper operates.


\subsection{Background}


% Similarly, in the third round, there are a total of 4 candidate objects, which are divided into two categories based on their color features: 2 green vases and 2 red vases. The questioner asked the question "is it partially in red?" and the oracle gave the answer "No," so the 2 red vases were eliminated from the candidate objects.
% \begin{figure}[h]
%   \centering
%   \includegraphics[width=\linewidth]{images/algorithm.pdf}
%   \caption{binary search algorithm}
%   % \Description{A woman and a girl in white dresses sit in an open car.}
% \end{figure}

% \section{The general case: Proof of \texorpdfstring{\Cref{thm:main-decomp}}{Theorem 1.6}}\label{sec:algo}

First, we show that data structure of \Cref{l:max_min_query} can be used to compute distances witnessed by shortest paths that pass through a constant-size separator.

\begin{lemma}\label{l:single_adhesion}
Fix a constant $k \in \mathbb{N}$. There exists an algorithm which as the input receives an edge-weighted graph $G$ on $n$ vertices and $m$ edges together with a partition of its vertices into three sets $A, B, C$ such that $|B| \leq k$ and there are no edges between $A$ and $C$, and as the output computes $\max_{c \in C} \dist(a, c)$ for every $a \in A$. The running time is $\Oh(m \log n + n \log^{k - 1} n)$.
\end{lemma}

\begin{proof}
Let $B = \{b_1, \ldots, b_k\}$. For any $a \in A, c \in C$, we have $\dist(a, c) = \min_{i \in [k]} \dist(a, b_i) + \dist(c, b_i)$. First, we run Dijkstra's algorithm from every vertex in $B$ to find $\dist(v, b_i)$ for every $v \in V(G)$ and $i \in [k]$. Next, we use \Cref{l:max_min_query} to construct a data structure $\mathbb{D}$ for the point set $\{(\dist(c, b_1), \dots, \dist(c, b_k))\colon c\in C\}\subseteq \mathbb{R}^k$. Now, the value $\max_{c \in C} \dist(a, c)$ for any given $a$ is equal to the answer of $\mathbb{D}$ to the query with argument $(\dist(a, b_1), \dots, \dist(a, b_k))$.
\end{proof}

After computing the distances over a constant-size separator, we will use the following observation to simplify one of the sides of the separation.

\begin{lemma}\label{l:inserting_paths}
Let $G$ be a edge-weighted connected graph and let $A, B, C$ be a partition of its vertices such that there are no edges between $A$ and $C$. For every pair of vertices $u, v \in B$, let $P_{u, v}$ be any shortest path from $u$ to $v$ with all internal vertices in $C$ (assuming such a path exists).

Let $G'$ denote a graph obtained from $G[A \cup B]$ by adding an edge from $u$ to $v$ of weight equal to the length of $P_{u, v}$, for all $u, v \in B$ for which $P_{u, v}$ exists. Then,  $$\dist_G(s, t) = \dist_{G'}(s, t)\qquad\textrm{for all }s,t\in A\cup B.$$
\end{lemma}
\begin{proof}
Let $G''$ be the graph obtained by adding new edges of $G'$ to $G$.
Fix any $s, t \in A \cup B$ and let $P$ denote the shortest path from $s$ to $t$ in $G''$ which minimizes the number of vertices from $C$ visited. Naturally, the weight of $P$ is equal $\dist_G(s, t)$. Assume that such path visits at least one vertex of $C$. Then, the path $P$ is of the form $s \xrightarrow{P_1} x \xrightarrow{P_2} y \xrightarrow{P_3} t$, where $x, y \in B$ and all the internal vertices of $P_2$ are in $C$. By the construction of $G'$, $P_2$ can be replaced with a direct edge from $x$ to $y$ of the same weight. We obtain a same weight path with a smaller number of vertices of $C$ visited, which is a contradiction. Therefore, $P$ is entirely contained in $A \cup B$, hence it exists in $G'$. This shows that $\dist_G(s, t) = \dist_{G'}(s, t)$.
\end{proof}


The next lemma encapsulates the main algorithmic content of the proof of \Cref{thm:main-decomp}. The algorithm will split the tree decomposition provided on input into smaller parts for which the eccentricities are easier to calculate. We use the following lemma to handle a single such part.
\begin{lemma}\label{l:star}
Fix constants $k, g \in \mathbb{N}, 0 < \delta < \frac{1}{54}$. Assume we are given $n \in \mathbb{N}$, an edge-weighted graph $G$ on at most $n$ vertices with a weight function $w \colon E(G) \to \mathbb{N}$, a vertex subset $A$ and a collection of non-empty vertex subsets $V_0, V_1, \dots, V_\ell$ satisfying the following conditions:
\begin{itemize}[nosep]
	\item The sum of weights of all the edges in $G$ is bounded by $\Oh(n)$.
	\item $V(G) \setminus A = V_0 \cup V_1 \cup \dots \cup V_\ell$.
	\item $|A| \leq k$.
	\item For every $i \in [\ell]$, $G[V_i \setminus V_0]$ is connected, $N_G(V_i \setminus V_0) = V_i \cap V_0$, $|V_i| = \Oh(n^\delta)$, and $|V_0 \cap V_i| \leq 4$.
	\item For all $i, j \in [\ell], i \neq j$, $V_i \setminus V_0$ and $V_j \setminus V_0$ are disjoint and non-adjacent in $G$.
	\item Every edge $uv \in E(G)$ with $u, v \not\in A$ is contained in $G[V_i]$ for some $i\in \{0,1,\ldots,\ell\}$.
	\item The graph obtained by taking $G[V_0]$ and adding a clique on $V_0 \cap V_i$ for every $i \in [\ell]$ has Euler genus bounded by $g$.
\end{itemize}
Then, we can compute the eccentricity of every vertex of $G$ in time $\Oh \left( n^{1 + \frac{150 + 54 \delta}{151}} \log^k n \right)$.
\end{lemma}

\begin{proof}
Fix $\delta' = \frac{1 + 97 \delta}{151}$; we have $\delta' - \delta = \frac{1 - 54\delta}{151} > 0$.
Let $E_i$ denote the set of edges with one endpoint in $V_i$ and the other endpoint in $V_i \setminus V_0$. For $i \in [\ell]$, we shall say that $V_i$ is {\em{heavy}} if the sum of weights of $E_i$ is larger than $n^{\delta'}$. Since the sets $E_i$ are pairwise disjoint and the total sum of weights of all the edges is bounded by $\Oh(n)$, the number of heavy subsets is bounded by $\Oh(n^{1 - \delta'})$. Without loss of generality, we may assume that $V_{\ell' + 1}, \dots, V_\ell$ are heavy and $V_1, \dots, V_{\ell'}$ are not, for some $\ell'\in \{0,\ldots,\ell\}$.


For any source vertex $s$, we can calculate distances from $s$ to every vertex of $G$  using breadth first search in time $\Oh(\sum_{e \in E(G)} w(e)) = \Oh(n)$.
In particular, for every $\ell' < i \leq \ell$, we can compute the distances from every vertex of $V_i$ to every vertex of $G$ in total time $\Oh(n^{2 - \delta' + \delta})$, because $$|V_{\ell'+1}\cup \ldots\cup V_{\ell}|\leq n^{1-\delta'}\cdot \Oh(n^\delta)=\Oh(n^{1-\delta'+
\delta}).$$
Additionally, we calculate distances $\dist_G(a, v)$ for every $a \in A, v \in V(G)$ in time $O(n)$.

For every $i \in [\ell]$ and $u,v \in V_0 \cap V_i$, there exists a shortest path $P_{i,u,v}$ from $u$ to $v$ with all internal vertices belonging to $V_i - V_0$ due to the assumption that $G[V_i - V_0]$ is connected and $N_G(V_i - V_0) = V_i \cap V_0$. Therefore, the distance from $u$ to $v$ is bounded by the sum of weights of edges in $E_i$. In particular, for $i \in [\ell']$, $\dist_G(u, v) \leq n^{\delta'}$.

We define $\widetilde{G}$ to be the graph obtained by taking $G[A \cup V_0 \cup \dots \cup V_{\ell'}]$ and applying the following operation for every $i \in \{\ell' + 1, \dots, \ell\}$:
for each pair of vertices $u, v \in A \cup (V_0 \cap V_i)$, add an edge in $\widetilde{G}$ between $u$ and $v$ with weight equal to the total weight of $P_{i,u,v}$. For a fixed $i, u$, we can find $P_{i, u, v}$ for all $v$ using breadth first search in time $\Oh(n)$. Taking a sum over all $i, u$, we get that $\tilde{G}$ can be computed in total time $\Oh(n^{2 - \delta'})$.


\begin{claim}\label{cl:wG}
The sum of the edge weights in $\widetilde{G}$ is $\Oh(n)$. Moreover, for all $u, v \in V(\widetilde{G})$, we have $\dist_{\widetilde{G}}(u, v) = \dist_{G}(u, v)$.
\end{claim}

\begin{proof}
Consider $i \in \{\ell' + 1, \dots, \ell\}$ and any $u, v \in A \cup (V_0 \cap V_i)$ for which we added an edge. Its weight is bounded by the sum of weights of edges in $E_i$. Therefore, the total weight of all edges added is at most
$$
\sum_{i \in \{\ell' + 1, \dots, \ell\}} \left( |A \cup (V_0 \cap V_i)|^2 \sum_{e \in E_i} w(e) \right) \leq (4 + k)^2 \sum_{e \in E(G)} w(e) = \Oh(n).
$$
This proves the first part of the claim.

For the second part of the claim, consider any $i \in \{\ell' + 1, \dots, \ell \}$ and observe that by our assumptions, $A \cup (V_0 \cap V_i)$ separates $(V_0 \cup \dots \cup V_{\ell'} \cup V_{i + 1} \cup \dots \cup V_\ell) \setminus V_i$ from $V_i \setminus V_0$. Hence it suffices to repeatedly apply \Cref{l:inserting_paths}.
\end{proof}

For every $u \in V(\widetilde{G})$, we have $\ecc_G(u) = \max(\ecc_{\widetilde{G}}(v), \max_{v \in V(G) \setminus V(\widetilde{G})} \dist_G(u, v))$. Note, that we already know all the distances $\dist_G(u, v)$ for $v \in V(G) \setminus V(\widetilde{G})$. Similarly, we can already compute $\ecc_G(u)$ for every $u \in V(G) \setminus V(\widetilde{G})$. Therefore, it remains to compute $\ecc_{\widetilde{G}}(v)$ for each $v \in V(\widetilde{G})$. Our goal is to show that this can be done efficiently using \Cref{l:main_ecc}.

Now, let $G'$ be the graph obtained from $\tilde{G}$ by replacing every edge $e$ non-indicent to $A$ with $w(e)\geq 2$ with a path of length $w(e)$ consisting of unit-weight edges. This operation again preserves the distances. Since the sum of edge weights in $\tilde{G}$ is of $\Oh(n)$, the total number of vertices in $G'$ is of $\Oh(n)$. For $0 \leq i \leq \ell'$, we write $V'_i$ to denote the set $V_i$ together with all the vertices added as a part of a path between two endpoints in $V_i$.
As $V_i$ is not heavy for each $i\in [\ell']$, we have
$$
|V'_i \setminus V'_0| \leq |V_i| + \sum_{e \in E_i} w(e) = \Oh(n^{\delta'})\qquad \textrm{for all }i\in [\ell'].
$$

Let $G_0$ denote the graph $G'[V'_0]$ and let $G_0^*$ denote the graph $G'- A$ with $V'_i - V'_0$ contracted to a single vertex $v_i^*$, for each $i \in [\ell']$; note that, all edges of $G_0$ and $G_0^*$ have unit weight.

\begin{claim}
	The graph $G_0^*$ is does not contain $K_{t}$ as a minor, where $t = \Oh(\sqrt{g})$.
\end{claim}

\begin{proof}
Let $\bar{G}_0$ denote the graph obtained by taking $G_0$ and adding a clique on $V_0 \cap V_i$ for every $i \in [\ell']$.
By lemma assumptions and the fact that subdividing edges does not increase the Euler genus, $\bar{G}_0$ has Euler genus at most $g$. In particular, $\bar{G}_0$ is $K_{t'}$-minor-free for some $t' = \Oh(\sqrt{g})$, because the Euler genus of $K_{t'}$ is $\Omega({t'}^2)$.

Similarly, let $\bar{G}_0^*$ be the graph obtained by taking $G_0^*$ and adding a clique on each $V_0 \cap V_i$.
Note, that $\bar{G}_0^* - \{v_1^*, \dots, v_{\ell'}^*\}$ is precisely $\bar{G}_0$. Let $t = \max(t', 6)$.
Recall that a minor model of a clique $K_t$ consists of $t$ pairwise vertex-disjoint connected subgraphs, called
branch sets, such that there is at least one edge between each pair of the branch sets.
Consider a minor model $\varphi$ of $K_{t}$ in $\bar{G}^*_0$.
Note that $\varphi$ cannot contain any singleton branch set of the form $\{v^*_i\}$, for the degree of $v^*_i$ in $\bar{G}^*_0$ is at most $4 < t - 1$. Furthermore, since $N_{\bar{G}^*_0}(v^*_i) = V_0 \cap V_i$, any branch set containing $v^*_i$ and at least one other vertex contains some $u \in V_0 \cap V_i$, and $N_{\bar{G}^*_0}(v^*_i)\subseteq N_{\bar{G}^*_0}(u)$, hence removing $v^*_i$ from this branch set preserves the model. Therefore, we can assume without loss of generality that all branch sets of $\varphi$ are disjoint from $\{v^*_1, \dots, v^*_{\ell'}\}$, hence $\varphi$ is a minor model of $K_{t}$ in $\bar{G}_0$. This is a contradiction, as $t \geq t'$ and $\bar{G}_0$ is $K_{t'}$-minor-free. Therefore, $\bar{G}_0^*$ is $K_t$-minor-free, hence $G_0^*$ also.
\end{proof}

Let $\rho' = \frac{2 - 108 \delta}{151} > 0$. The graph $G^*_0$ is a unit-weight graph and is $K_{t}$-minor-free.
Hence, by applying \Cref{t:r_division} to $G^*_0$ (with $\varepsilon = \rho'/2$)
we obtain an $n^{\rho'}$-division $\mathcal{R}_0$ in time $\Oh(n^{1 + \rho'})$.
We extend it to $G' - A$ by mapping every contracted vertex $v^*_i$ to $N_{G' - A}[V'_i - V'_0] = (V'_i - V'_0) \cup (V_0 \cap V_i)$. Formally, we put $V''_i \coloneqq N_{G' - A}[V'_i - V'_0]$ and 
$$
\mathcal{R} \coloneqq \left\{ (R_0 \cap V'_0) \cup \bigcup_{i \colon v^*_i \in R_0} V''_i \colon R_0 \in \mathcal{R}_0 \right\}.
$$

Now, we argue that $\mathcal{R}$ is a reasonable division of $G' - A$. Clearly, all sets $R \in \mathcal{R}$ are connected in $G' - A$. Pick any $R \in \mathcal{R}$ and let $R_0$ be its corresponding set in $\mathcal{R}_0$.
Every vertex $v^*_i$ is mapped to a set of size $\Oh(n^{\delta'})$, therefore
$$|R| \leq |R_0| \cdot \Oh(n^{\delta'}) = \Oh(n^{\rho' + \delta'}).$$

By our construction, for every $i \in [\ell']$, $R$ is either disjoint from $V'_i - V'_0$ or contains whole $N_{G' - A}[V'_i - V'_0]$. This means that no vertex belonging to any $V'_i - V'_0$ can be in $\partial R$, hence $\partial R \subseteq V'_0$.

Pick any $u \in \partial R \cap R_0$. Assume that $u \not\in \partial R_0$. Then every vertex of $N_{G_0^*}(u)$ must be in $R_0$, hence $N_{G - A'}(u) \subseteq R$, which is a contradiction. This means that $\partial R \cap R_0 \subseteq \partial R_0$.

Pick any $u \in \partial R - R_0$. Then, $u \in V_0 \cap V_i$ for some $i \in [\ell']$ such that $v_i^* \in R_0$. Moreover, $v_i^* \in \partial R_0$ and is adjacent to $u$ in $G_0^*$. The number of such $u$ is bounded by $4 |\partial R_0 \cap \{ v_1^*, \dots, v_{\ell'}^* \}|$.

Putting two cases together, we obtain:
$$
\sum_{R \in \mathcal{R}} |\partial R| = \sum_{R \in \mathcal{R}} \left(|\partial R \cap R_0| + |\partial R - R_0|\right) \leq \sum_{R_0 \in \mathcal{R}_0} \left(|\partial R_0| + 4 |\partial R_0 \cap \{ v_1^*, \dots, v_{\ell'}^* \}|\right) = \Oh(n^{1 - \frac{1}{2}\rho'}).
$$

It remains to show the following claim.

\begin{claim}
Pick any $R \in \mathcal{R}, s_R \in R$. The number of different distance profiles on $R$ relative to $s_R$ in $G' - A$ is of $\Oh(n^{48\rho' + 54\delta'})$.
\end{claim}
\begin{proof}
We look at every vertex $v \in V(G') \setminus A$ and consider three cases: $v \in R$, $v \in V'_0$, and $v \in V'_i \setminus (V'_0 \cup R)$ for some $i \in [\ell']$. By our construction, $R \cap V'_0$ is non-empty, hence w.l.o.g. we can assume that $s_R \in V'_0$ as whether two vertices have the same profile on $R$ is independent of the choice of the pivot vertex.

In the first case, there are at most $|R| = \Oh(n^{\rho' + \delta'})$ such vertices, hence they realise at most that many profiles.

In the second case, we want to observe that profile of any vertex $v \in V'_0$ on $R$ depends only on its profile on $R \cap V'_0$ (relative to $s_R$). Pick any $t \in R - V'_0$. Then $t \in V'_i - V'_0$ for some $i \in [\ell']$, $V_i \cap V_0 \subseteq R \cap V'_0$, and every path from $v$ to $t$ intersects $V_i \cap V_0$. In particular, distances from $v$ to vertices of $V_i \cap V_0$ determine its distance to $t$, which proves the observation.

Let $\tilde{G}_0$ denote the graph obtained by taking $G'[V'_0]$ and for every $i \in [\ell'], u, v \in V_0 \cap V_i$ adding a disjoint path from $u$ to $v$ of length $\dist(u, v)$. Let $P_i$ denote the vertex set of paths added between $V_0 \cap V_i$. For every $t \in V'_0$ we have $\dist_{G' - A}(v, t) = \dist_{\tilde{G}_0}(v, t)$, so it suffices to bound the number of profiles on $R \cap V'_0$ in $\tilde{G}_0$. By our assumptions, $\tilde{G}_0$ has Euler genus bounded by $g$ and all $P_i$ are of size $\Oh(n^{\delta'})$.

Let $R_0$ be the set of $\mathcal{R}_0$ corresponding to $R$. Let $\tilde{R}_0$ denote the set $(R \cap V'_0) \cup \bigcup_{i : v^*_i \in R_0} P_i$. Such set is connected in $\tilde{G}_0$. Moreover, similarly to $R$, its size is $\Oh(n^{\rho' + \delta'})$. Applying \Cref{thm:distprofiles}, we get that the number of distance profiles on $\tilde{R}_0$ in $\tilde{G}_0$ is $\Oh(n^{12(\rho' + \delta')})$, which also bounds the number of profiles on $R$ in $G' - A$ realised by $V'_0$.

For the third case, assume $v \in V'_i \setminus (V'_0 \cup R)$ for some $i\in [\ell']$. Every path from $v$ to any vertex of $R$ in $G' - A$ intersects $V_i \cap V_0$. Let $v_1, \dots v_p$ be the vertices of $V_i \cap V_0$, where $p \leq 4$. The profile of $v$ on $R$ is then determined by the following:
\begin{itemize}[nosep]
 \item[(a)] the profile of each $v_j$ on $R$,
 \item[(b)] $\dist_{G' - A}(v, v_j) - \dist_{G' - A}(v, v_1)$ for each $2 \leq j \leq p$, and
 \item[(c)] $\dist_{G' - A}(s_R, v_j) - \dist_{G' - A}(s_R, v_1)$ for each $2 \leq j \leq p$ where $s_R$ is some pivot vertex of $R$.
\end{itemize}
By the previous case, the number of distance profiles of each $v_j$ is $\Oh(n^{12(\rho' + \delta')})$. The distances between $v$ and $v_j$ are bounded by $|V'_i|$, hence each quantity described in (b) can take $\Oh(n^{\delta'})$ different possible values. Similarly, since $v_1$ and $v_j$ are connected via $V'_i$, $|\dist_{G' - A}(s_R, v_j) - \dist_{G' - A}(s_R, v_1)| \leq \Oh(n^{\delta'})$. The number of different possible profiles of such $v$ is therefore bounded by $\Oh(n^{48(\rho' + \delta') + 6\delta'}) = \Oh(n^{48\rho' + 54\delta'})$. This finishes the proof of the claim.
\end{proof}

Now we can apply \Cref{l:main_ecc} to graph $G'$ with apex set $A$, $X = V(\widetilde{G})$, and the following constants: $$\rho = \rho' + \delta',\qquad \gamma = 1 - \frac{1}{2}\rho',\quad \textrm{and}\quad \alpha = 48\rho' + 54 \delta'.$$ This allows us to calculate all $V(\widetilde{G})$-eccentricities in $G'$ in time
$$
\Oh \left( \left(
	n^{ 2 - \frac{1}{2} \rho' } +
	n^{ 1 + 48\rho' + 54 \delta' }
\right) \log^k n \right) =
\Oh \left( n^{1 + \frac{150 + 54 \delta}{151}} \log^k n \right).
$$
Since for each $v\in V(\widetilde{G})$ we have $\ecc_{\widetilde{G}}(v) = \max_{u \in V(\widetilde{G})} \dist_{\widetilde{G}}(v, u) = \max_{u \in V(\widetilde{G})} \dist_{G'}(v, u)$, this means that we have successfully computed all the eccentricities in $\widetilde{G}$; and as we argued, this is enough to compute all the eccentricities in $G$ as well.

Finally, the total running time of the algorithm is
$$
\Oh \left( n^{1 + \frac{150 + 54 \delta}{151}} \log^k n + n^{2 - \delta' + \delta} \right) =
\Oh \left( n^{1 + \frac{150 + 54 \delta}{151}} \log^k n \right).
$$\qedhere
\end{proof}


\begin{lemma}\label{l:star2}
Fix constants $k, g \in \mathbb{N}, 0 < \delta < \frac{1}{54}$. Assume we are given $n \in \mathbb{N}$, an edge-weighted graph $G$ on at most $n$ vertices with a weight function $w \colon E(G) \to \mathbb{N}$, a vertex subset $A$ and a collection of non-empty vertex subsets $V_0, V_1, \dots, V_\ell$ satisfying the same conditions as in \Cref{l:star} with the following differences:
\begin{itemize}
	\item we don't require $G[V_i - V_0]$ to be connected and $V_i - V_0$ to be adjacent to whole $V_i \cap V_0$;
	\item instead of $|V_0 \cap V_i| \leq 4$, we require $|V_0 \cap V_i| \leq k$.
\end{itemize}
Then, we can compute the eccentricity of every vertex of $G$ in time $\Oh \left( n^{1 + \frac{150 + 54 \delta}{151}} \log^{k + 5g} n \right)$.
\end{lemma}

\begin{proof}
We will reduce our input to one which will satisfy the conditions of \Cref{l:star}. We start by addressing the adhesions $V_0 \cap V_i$ containing too many vertices.

Let $G_0$ denote the graph $G[V_0]$ with cliques placed at $V_0 \cap V_i$ for every $i \in [\ell]$.
For every $i \in [\ell]$ we repeat the following procedure: while $|V_0 \cap V_i| > 4$,
remove arbitrary $5$ vertices from $V_0 \cap V_i$. Since $|V_0 \cap V_i| \leq k$ for each $i\in [\ell]$,
this procedure can be implemented in total time $\Oh(n)$. As a result, at the end we have $|V_0 \cap V_i| \leq 4$ for all $i \in [\ell]$. Let $M$ be the set of all the removed vertices. By our assumptions, $G_0$ has Euler genus bounded by $g$, hence it cannot contain $g + 1$ pairwise disjoint copies of $K_5$
(as the Euler genus of a graph is the sum of the Euler genera of its 2-connected components~\cite{StahlB77} and $K_5$ is not planar). Each removed quintiple of vertices induces a $K_5$ in $G_0$, hence we have $|M| \leq 5g$. We set $A' = A \cup M$ and may thus assume that $V_i$ is disjoint from $A'$ for all $0 \leq i \leq \ell$.

Now, fix $i \in [\ell]$. Let $C^i_1, \dots, C^i_{r_i}$ denote the connected components of $V_i - V_0$ in $G - A'$. We define $W^i_j := N_{G - A'}[C^i_j]$ for every $j \in [r_i]$. Clearly, all $W^i_j$ induce a connected subgraph of $G$ and satisfy $N_{G - A'}(W^i_j - V_0) = W^i_j \cap V_0$. We put $V'_0 := V_0$ and enumerate
$$
\{V'_1, V'_2, \dots V'_{\ell'}\} := \{ W^i_j \colon i \in [\ell], j \in [r_i] \}.
$$
It is easy to verify that the sets $A'$ and $V'_0, V'_1, \dots, V'_{\ell'}$ satisfy the conditions of \Cref{l:star}. We apply said lemma to calculate the eccentricity of every vertex of $G$ in the desired time.
\end{proof}



The next statement is a reformulation of \Cref{thm:main-decomp}.

\begin{theorem}
Fix constants $k, g \in \mathbb{N}$. Assume we are given a graph $G$ on $n$ vertices together with its tree decomposition $(T, \beta)$ and a set of private apices $A_t \subseteq \beta(t)$ for each node $t\in V(T)$ such that the following conditions hold:
\begin{itemize}[nosep]
 \item For every node $t \in V(T)$, we have $|A_t| \leq k$.
 \item For every edge $st \in E(T)$,  we have $|\beta(v) \cap \beta(u)|\leq k$.
 \item For every node $t \in V(T)$, graph obtained by taking $G[\beta(t)] - A_t$ and turning  $(\beta(t) \cap \beta(s))\setminus A_t$ into a clique for every edge $st \in E(T)$ has Euler genus bounded by $g$.
\end{itemize}
Then, we can compute the eccentricity of every vertex of $G$ in time $\Oh \left( n^{1 + \frac{355}{356}} \log^{k + 5g} n \right)$.
\end{theorem}

\begin{proof}
We may assume that $|V(T)|\leq n$, for every tree decomposition with no two bags comparable by inclusion has this property; and adjacent comparable bags can be merged by contracting the edge between them.

For a node $t\in V(T)$, by the {\em{weight}} of $t$ we mean the size of the corresponding bag, that is, $|\beta(t)|$. For any subset of nodes $S \subseteq V(T)$, we define $\beta(S) \coloneqq \bigcup_{t \in S} \beta(t)$ By the {\em{weight}} of $S$, we mean the total weight of the elements of $S$, that is, $\sum_{t\in S} |\beta(t)|$. 

\begin{claim}\label{cl:weight-T}
The weight of $V(T)$ is of $\Oh(n)$.
\end{claim}

\begin{proof}
The sets $\beta'(t) := \beta(t) - \bigcup_{s \in N_T(t)} \beta(s)$ are pairwise disjoint. We have
$$
\sum_{t \in V(T)} |\beta(t)| =
\sum_{t \in V(T)} |\beta'(t)| + 2 \cdot \sum_{st \in E(T)} |\beta(s) \cap \beta(t)| \leq
|V(T)| + 2k|E(T)| = \Oh(n).
$$
\end{proof}

Since every bag induces a graph of bounded Euler genus, the number of edges contained in a bag is linear in its size. In particular, this implies that the total number of edges of $G$ is also bounded by $\Oh(n)$.

We set $$\delta \coloneqq \frac{1}{356}\qquad\textrm{and}\qquad \Delta \coloneqq \frac{355}{356}.$$ Root the tree $T$ in an arbitrarily chosen node; this naturally imposes an ancestor-descendant relation in $T$ (for convenience, every node is considered its own ancestor and descendant).

We start by partitioning $T$ into connected subtrees using the following procedure.
We proceed bottom-up over $T$, processing nodes in any order so that a node is processed after all its strict descendants have been processed. Along the way, we mark some nodes and split the edges of $T$ into heavy and light. Let $t \in V(T)$ be the currently processed non-root node of $T$ and let $e \in E(T)$ be the edge connecting $t$ with its parent. If the total weight of all the unmarked nodes that are descendants of $t$ is at least $n^\delta$ (recall that this includes $t$ itself as well), then we declare $e$ heavy and mark all the descendants of $t$ that were unmarked so far. Otherwise, the edge $e$ is declared light and the procedure proceeds to further nodes of $T$.

Observe that
removing all heavy edges splits $T$ into connected subtrees, say $T'_1, \cdots T'_m$. All of the subtrees, except for possibly the subtree containing the root node, are of weight at least $n^\delta$. In particular, the number of subtrees $m$, and therefore the number of heavy edges, is  bounded by $\Oh(n^{1 - \delta})$. Moreover, in every subtree $T'_i$, removing the node closest to the root splits $T'_i$ into smaller components, each of weight less than $n^\delta$.

Fix a heavy edge $e$ and let $T^e_1$ and $T^e_2$ be the two subtrees into which $T$ splits after removing~$e$. Let $X^e_i = \beta(T^e_i)$ for $i \in \{1, 2\}$. Put $A_e = X^e_1 \setminus X^e_2$, $C_e = X^e_2 \setminus X^e_1$, and $B_e = X^e_1 \cap X^e_2$. By the properties of tree decompositions, such choice of $A_e, B_e, C_e$ satisfies the conditions of \Cref{l:single_adhesion}, hence in time $\Oh(n \log^{k - 1} n)$ we can compute $\max_{v \in X^e_2} \dist_G(u,v)$ for every $u \in X^e_1$, and $\max_{u \in X^e_1} \dist_G(u,v)$ for every $v \in X^e_2$. Computing this for every heavy edge $e$ takes total time $\Oh(n^{2 - \delta} \log^{k - 1} n)$.

Fix any subtree $T'=T'_j$. Let $e_1 = t^{e_1}_1t^{e_1}_2, e_2 = t^{e_2}_1 t^{e_2}_2, \dots, e_\ell = t^{e_\ell}_1 t^{e_\ell}_2$ denote the heavy edges incident to $T'$, where $t^{e_i}_1 \in V(T')$ and $V(T') \subseteq V(T_1^{e_i})$ for every $i \in [\ell]$.
For a vertex $v \in \beta(T')$, let
$$d_0(v) = \max_{u \in \beta(T')} \dist_G(v, u)\qquad\textrm{and}\qquad d_i(v) = \max_{u \in X_2^{e_i}}\dist_G(v,u),\quad\textrm{for } i \in [\ell].$$ We have $\ecc(v) = \max \{ d_i(v)\colon i\in \{0,1,\ldots,\ell\}\}$.The values of $d_i(v)$ are already calculated for all $i\in [\ell]$, hence it remains to compute $d_0(v)$.

For every $i \in [\ell]$ and every pair of vertices $u, v \in \beta(t^{e_i}_1) \cap \beta(t^{e_i}_2)$ we find a shortest path between $u$ and $v$ with all internal vertices inside $X^{e_i}_2$ (or determine that it doesn't exist). For a fixed $u, v$ this can be done in time $\Oh(n)$. Since in total we perform this step at most $2k^2$ times per heavy edge, it takes $\Oh(n^{2 - \delta})$ time in total. Let $P_{i, u, v}$ denote such path, assuming it exists.

Let $G'$ denote the graph obtained from $G[\beta(T')]$ by taking every $i, u, v$ for which $P_{i, u, v}$ exists and adding an edge between $u$ and $v$ of weight equal to the total weight of $P_{i, u, v}$.
The weight of every edge inserted in $\beta(t^{e_i}_1) \cap \beta(t^{e_i}_2)$ is bounded by $|X^{e_i}_2|+1$. The total weight of all edges inserted is therefore at most
$$
\sum_{i \in [\ell]} |\beta(t^{e_i}_1) \cap \beta(t^{e_i}_2)|^2 \cdot (|X^{e_i}_2|+1) \leq
k^2 \sum_{i \in [\ell]} (|X^{e_i}_2|+1) = \Oh(n),
$$
where the last equality follows from the fact that all the trees $T^{e_i}_2$ are pairwise disjoint.
By \Cref{l:inserting_paths}, we have $\dist_{G'}(u, v) = \dist_G(u, v)$ for each $u, v \in \beta(T')$. Hence, computing $d_0(v)$ for every $v \in \beta(T')$ is equivalent to computing the eccentricity of every vertex in $G'$.

If the size of $\beta(T')$ is smaller than $n^\Delta$, we compute the eccentricities naively in time $\Oh(|\beta(T')|^2)$, 
noting that $G'$ has $\Oh(|\beta(T')|)$ edges (thanks to Claim~\ref{cl:weight-T} and bounded genus assumption 
of the last bullet of the theorem statement). Otherwise, we argue that we can use the algorithm in \Cref{l:star} as follows.

Let $t$ be the node of $T'$ closest to the root. Let $s_1, \dots, s_p$ be the children of $t$ in $T$ and let $T''_i$ denote the connected component of $T' - \{t\}$ containing $s_i$. Set $V_0 = \beta(t)$ and $V_i = \beta(T''_i)$ for $i \in [p]$.

It is now easy to verify that $G'$ and sets $A, \{V_i\colon 0\leq i\leq p\}$ selected this way satisfy the assumptions of \Cref{l:star2}. This allows us to use it to compute the eccentricities in $G'$ in time
$$
\Oh \left( n^{1 + \frac{150 + 54\delta}{151}} \log^{k + 5g} n \right) =
\Oh \left( n^{1 + \frac{354}{356}} \log^{k + 5g} n \right).
$$
As we argued, from these eccentricities, we may easily compute all the eccentricities in $G$.

Now, let us analyse the total running time of the whole algorithm. We invoke \Cref{l:star} $\Oh(n^{1 - \Delta})$ times, since we apply it only to subtrees $T'_i$ of size at least $n^\Delta$. The total running time of those applications is hence
$$
\Oh \left( n^{2 + \frac{354}{356} - \Delta} \log^{k + 5g} n \right) =
\Oh \left( n^{1 + \frac{355}{356}} \log^{k + 5g} n \right).
$$
We compute the eccentricities naively for subtrees smaller than $n^\Delta$, hence the total running time of this computation is
$$
\sum_{i \in [m] \colon |\beta(T'_i)| \leq n^\Delta} |\beta(T'_i)|^2 \leq
n^\Delta \cdot \sum_{i \in m} |\beta(T'_i)| = \Oh(n^{1 + \Delta})=\Oh\left(n^{1+\frac{355}{356}}\right).
$$
The rest of computation can be done in $\Oh(n^{2 - \delta} \log^k n)$. Therefore, the whole algorithm runs in time $\Oh \left( n^{1 + \frac{355}{356}} \log^{k + 5g} n \right)$.
\end{proof}



\textbf{Notations.} GuessWhat?! is a guessing game aiming to find the correct object from the image. Each instance of game is denoted as a tuple $\left( I, D, O, o^* \right) $, wherein $I$ represents the observed image, $D$ represents the dialogue consisting of $J$ rounds of Q\&A pairs $\left( q_j,a_j \right) _{j=1}^{J}$, and $O=\left( o_n \right) _{n=1}^{N}$ represents the list of $N$ objects in the image $I$, with $o^*$ referring to the target object. Each question $q_j=\left( w_{m}^{j} \right) _{m=1}^{M_j}$ is a sequence of $M_j$ tokens selected from a predetermined vocabulary $V$. The $V$ is comprised of word tokens, a question stop token $<\mathrm{?}>$, and a dialogue stop token $<\mathrm{End}>$. The answer $a_j\in \left\{ <\mathrm{Yes}>,<\mathrm{No}>,<\mathrm{NA}> \right\} $ can be categorized as either yes, no, or not applicable. 
% Regarding each object $o$, it possesses an object category $c_o\in \{1...C\}$ and a segmentation mask.

\textbf{QGen.} The QGen produces a new question \(q_{j+1}\), given an image \(I\) and a history of \(j\) questions and answers \((q,a)_{1:j}\). It consists of a question encoder, an image encoder, and a question decoder. 
% The question encoder and image encoder are usually recurrent neural network~(RNN)~\cite{schuster1997bidirectional}, convolutional neural network~(CNN)~\cite{gu2018recent}, or vision-language model~(VLM)~\cite{du2022survey}. The question decoder is usually LSTM or transformer~\cite{vaswani2017attention}.

\textbf{Oracle.} The oracle is required to produce a yes-no answer $a_j$ for a target object $o^*$ within an image $I$ given a natural language question. The question is usually represented as the hidden state of an encoder, which is either LSTM or vision-language model~(VLM)~\cite{du2022survey}. 
We then concatenate the question, the bounding box of the target object, and category of the target object into a single vector and feed it as input to a single hidden layer multilayer perceptron~(MLP).
It outputs the final answer distribution using a softmax layer.

\textbf{Guesser.} The guesser takes an image \(I\), a history of questions and answers \((q,a)_{1:J}\), and predicts the correct object \(o^*\) from the set of all objects. The dialogue history is usually represented as the last hidden state of an encoder, which is either LSTM or VLM. The object embeddings are obtained from the categorical and spatial features. 
They are subjected to a dot product, which is then passed through a softmax to obtain a prediction distribution over the objects.


\textbf{SL Training.} The ground truths for QGen, Oracle, and Guesser are respectively the question, answer, and target object label. They are all optimized using cross-entropy loss.

\textbf{RL Training.} The Question Generation process is regarded as a Markov Decision Process (MDP), where the Questioner acts as an agent. At each time step $t$, for the generated dialogue based on the image $I$, the agent's state is defined as the visual information of the image, along with the history of Q\&A and the tokens of the current question generated so far: $S_t=\left( I,(q,a)_{1:j-1},\left( w_{1}^{j},...,w_{m}^{j} \right) \right) $, where $t=\sum_{k=1}^{j-1}{M_k}+m$. The agent's action $A_t$ involves selecting the next output token $w_{m+1}^{j}$ from $V$.
Depending on the agent's actions, the transition between two states falls into the completion of the current question, the end of the dialogue, or a new token.
The dialogue is limited to a maximum number of rounds $J_{\max }$. 
We model the QGen using a stochastic policy $\pi _{\theta}(A\mid S)$, where $\theta $ represents the parameters of the deep neural network utilized in the QGen baseline. These parameters are responsible for generating probability distributions for each state. The objective of policy learning is to estimate the parameter $\theta $.

\begin{figure*}[htbp]
  \centering
  \includegraphics[width=0.9\linewidth]{images/fig4-CVPR.pdf}
  \caption{The framework of the Tree-structured Strategy with Answer Distribution Estimator (TSADE). The red box represents the target object.}
  \label{fig:2}
\end{figure*}

%shwang-checkpoint
\subsection{Tree-structured Strategy with Answer Distribution Estimator (TSADE)}


\textbf{Tree-structured Strategy.} When humans are faced with this guessing game, they hope that each question can provide maximum distinction among the candidate objects. We propose a Tree-Structured strategy to mimic human behavior. As shown in Figure~\ref{fig:1} (c), while ensuring that no misclassification occurs, the Q\&A generated in each round should divide the current candidate objects into two groups as clearer as possible. In other words, half of the answers here should be ``Yes'' and the other half ``No''. The type of this question can be of any aspect, such as category, color, shape, size, location, and so on, as long as it meets the above requirement. Obviously, we also know which group the target belongs to. Therefore, we select this group as the new candidate object for the next round. 
Finally, after multiple rounds of questioning, we will find a single target. For example, in the second round, there are a total of four candidate objects, which can be divided into two groups, two metal things and two coffees. The Questioner asks the question ``Is it a metal thing?'' and the Oracle gives the answer ``No'', so two coffees are eliminated from the candidate objects. 


\textbf{Answer Distribution Estimator (ADE).} As shown in Figure~\ref{fig:2}, ADE contains an Oracle to answer the question and obtains $a_{1:k}^{j}$ based on $C_{1:k}^{j}$ at each round. $C_{1:k}^{j}$ is a set of candidate objects maintained by ADE, which is updated in each round. $k$ refers to the total number of objects in the candidate objects. $a_{1:k}^{j}$ is an answer distribution aimed at $C_{1:k}^{j}$. $C_{1:k}^{j}$ is updated by selecting and only keeping the objects in $a_{1:k}^{j}$ that have the same answer as the target object. In addition, ADE outputs $r_b$ and $r_c$, which are \emph{binary} reward and \emph{candidate-minimization} reward based on the Tree-structured strategy. It measures $r_b$ and $r_c$ with $a_{1:k}^{j}$. It outputs $r_b$ in each round but outputs $r_c$ in the final round. When using the traditional RL paradigm~\cite{strub2017end}, a reward $r_s$ is given at the end. If the Guesser finds the target successfully, $r_s$ is 1, otherwise 0. $r_b$, $r_c$ and $r_s$ are added to the cumulative reward $r^{\left( j \right)}$ as the final reward.




% To realize the above motivation, we apply goal-oriented visual dialogue to an RL paradigm and propose \emph{binary} reward and \emph{candidate-minimization} reward to guide question generation.

\textbf{Binary Reward.} In a dialogue, to implement the Tree-structured strategy, we design the reward to measure whether we can eliminate half of the objects ($k/2$) to the maximum extent in each round of updating $C_{1:k}^{j}$. Given the state $S_t$, where the $<\mathrm{End}>$ token is sampled or the maximum round $J_{\max}$ is reached, the reward of the state-action pair is defined as follows:

\setlength{\abovedisplayskip}{3pt}
\begin{small}
\begin{equation}
r_b\left( S_t,A_t \right) =E\left[ \sum_{j=1}^{J_{end}}{\left( 1-\frac{|l_j-k_j/2|}{k_j/2} \right)} \right] 
\end{equation}
\end{small}
We score each round and calculate the overall expectation. $J_{end}$ refers to the round reaching $J_{\max}$ or the occurrence of $<\mathrm{End}>$ token. $k_j$ refers to the number of objects in $C_{1:k}^{j}$ in the $j$ round. $l_j$ has a value range of $\left[ 0,L_j \right] $, representing the maximum number of ``Yes'' or ``No'' in $a_{1:k}^{j}$. We hope that in each round of the dialogue, we can have $l_j=k_j/2$, where half of the answers are ``Yes'' and half are ``No''. In this case, the reward score for that round of dialogue is 1. When all answers are ``Yes'' or ``No'', the reward score is 0. 
% To ensure correctness, we will select the part of objects where the target is located as the candidate objects for the next round.

\textbf{Candidate-minimization Reward.} Based on the process of finding the target in a dialogue, it is essentially equivalent to continuously narrowing down the scope of candidate objects. When the candidate objects are reduced to only one target, the goal is achieved. In many cases (especially when the dialogue has to stop due to reaching the maximum round limit $J_{\max}$), even if the target is successfully found in a dialogue, the scope of candidate objects has not been narrowed down to only the target. In this case, the result is not of high quality. In order to encourage the generation of higher-quality successful dialogues, we design a reward to measure the quality of those successful dialogues (compared with failed dialogues), as follows:

\setlength{\abovedisplayskip}{3pt}
\begin{small}
\begin{equation}
\begin{aligned}
r_c\left( S_t,A_t \right) =\begin{cases}
	\alpha 
 +\beta  
 \left( 1-\frac{k_{j_{end}}-1}{N-1} \right), &\mathrm{If\ } S_t \in \Omega, \\
	0, &\mathrm{Otherwise}.
\end{cases}
\end{aligned}
\end{equation}
\end{small}
where $\Omega=\{S|\arg\max_o[\mathrm{Guesser}(S)]=o^*\}$, $\alpha $ and $\beta $ are weights used to balance the contribution of the 0-1 reward and the quality of successful dialogues. When $k_{j_{end}}=1$, the dialogue is considered successful and of the highest quality. When $k_{j_{end}}=N$, we consider that even if the Guesser successfully finds the target $o^*$, the quality is not high, and it is likely a lucky guess. As $\alpha $ increases, the contribution of successful dialogues relative to failed dialogues to the overall reward increases. When the target $o^*$ is not successfully found, no reward is given.
When there is no 0-1 reward in this scenario, $r_c =\beta \left( 1-\frac{k_{j_{end}}-1}{N-1} \right) $. 
We encourage the generation of higher-quality successful dialogues by providing higher reward scores to those that can minimize the scope of candidate objects. 

\textbf{Training the QGen with Policy Gradient.} Given state-action pair $\left( S_t,A_t \right) $, we combine two rewards to form the ultimate reward function:

\setlength{\abovedisplayskip}{3pt}
\begin{small}
\begin{equation}
r\left( S_t,A_t \right) =\gamma \cdot r_b\left( S_t,A_t \right) +r_c\left( S_t,A_t \right) 
\end{equation}
\end{small}
where $\gamma$ is a weight to balance $r_b$ and $r_c$. Given the extensive range of actions in the game setting, we employ the policy gradient method~\cite{1999Policy} to train the QGen using the suggested rewards. This training method is similar to the approach used in the FS~\cite{strub2017end}.
% For specific implementation details, please refer to the supplementary materials.
% The objective of the policy gradient is to update the policy parameters through gradient descent based on the anticipated return. As we are operating in an episodic environment, the policy $\pi _{\theta}$, which represents the generative network of the QGen, is considered. In this scenario, the objective function of the policy takes the following structure:
% {\fontsize{8}{12}
% $$
% J(\theta )=E_{\pi _{\theta}}\left[ \sum_{t=1}^T{r}\left( S_t,A_t \right) \right] 
% $$
% }

% The parameters $\theta $ can be optimized by applying the gradient update rule. In the REINFORCE algorithm\cite{1998Reinforcement}, the gradient of $J(\theta )$ can be estimated by sampling a batch of episodes $\tau $ from the policy $\pi _{\theta}$:
% {\fontsize{8}{12}
% $$
% \nabla J(\theta )\approx \left. \left< \sum_{t=1}^T{\sum_{A_t\in V}{\nabla _{\theta}}}\log \pi _{\theta}\left( S_t,A_t \right) \left( Q^{\pi _{\theta}}\left( S_t,A_t \right) -b_{\varphi} \right) \right> \right. _{\tau}
% $$
% }

% Here, $Q^{\pi _{\theta}}\left( S_t,A_t \right) $ represents the state-action value function, which provides the expected cumulative reward at $\left( S_t,A_t \right) $:
% {\fontsize{8}{12}
% $$
% Q^{\pi _{\theta}}\left( S_t,A_t \right) =E_{\pi _{\theta}}\left[ \sum_{t^{'}=t}^T{r}\left( S_{t^{'}},A_{t^{'}} \right) \right] 
% $$
% }

% By substituting the notations with QGen, we obtain the following policy gradient:
% {\fontsize{7}{12}
% $$
% \begin{array}{r}
% 	\nabla J(\theta )\approx \left. \langle \sum_{j=1}^J{\sum_{m=1}^{M_j}{\nabla _{\theta}}}\log \pi _{\theta}\left( w_{m}^{j}\mid I,(q,a)_{1:j-1},w_{1:m-1}^{j} \right) \right.\\
% 	\left. \left( Q^{\pi _{\theta}}\left( I,(q,a)_{1:j-1},w_{1:m-1}^{j},w_{m}^{j} \right) -b_{\varphi} \right) \right. \rangle _{\tau}\\
% \end{array}
% $$
% }

% $b_{\varphi}$ represents a baseline function utilized to mitigate gradient variance, and it can be selected arbitrarily. It is implemented as a one-layer MLP that takes the state $S_t$ as input in QGen and yields the anticipated reward. The baseline $b_{\varphi}$ is trained using mean squared error, given by:
% {\fontsize{8}{12}
% $$
% \min_{\varphi} L(\varphi )=\left< \left. \left[ b_{\varphi}\left( S_t \right) -\sum_{t^{'}=t}^T{r}\left( S_{t^{'}},A_{t^{'}} \right) \right] ^2 \right. \right> _{\tau}
% $$
% }

% It involves the Oracle, which is responsible for generating responses to various questions regarding objects present in an image scene. To represent the spatial feature of the object labeled as $o$, the bounding box information (obtained from the segment mask) is encoded. The box coordinates, width, and height are indicated by $o_{spa}=\left[ x_{\min},y_{\min},x_{\max},y_{\max},x_{\mathrm{center}},y_{\mathrm{center}},w,h \right] $. The category $c_o$ is incorporated using a learned look-up table, and the current question is encoded using an LSTM. These three features are combined into a unified vector, which is then inputted into a one hidden layer MLP. This MLP is followed by a softmax layer, which generates the probability ($p\left( a\mid o_{spa},c_o,q \right) $) for the answer.

% In the given context, the Guesser is tasked with predicting the correct object $o^*$ from a list of objects, given an image $I$ and a series of question-answer pairs. The dialogue, treated as a single sequence of tokens, is encoded using an LSTM, and the last hidden state is extracted as the representation of the dialogue. Additionally, the spatial features and categories of all objects are embedded using an MLP. To generate the final prediction, we compute the dot product between the dialogue features and the object features, followed by a softmax operation.



% \begin{figure}[h]
%   \centering
%   \includegraphics[width=\linewidth]{images/fig3.pdf}
%   \caption{turn-by-turn introduce how binary search works}
%   % \Description{A woman and a girl in white dresses sit in an open car.}
% \end{figure}



% To simulate a complete GuessWhat?! game, we utilize our established models: Oracle, Guesser, and QGen baseline. The process begins with image $I$, from which an initial question $q_1$ is generated. We achieve this by sampling from the QGen baseline model until the stop question token is encountered. Subsequently, the Oracle receives question $q_1$, along with the assigned object category $o^*$ and its spatial information $o_{spa}^{*}$. The Oracle then generates the answer $a_1$, and the question-answer pair $\left( q_1,a_1 \right) $ is added to the dialogue history. This loop continues until either the end of the dialogue token is sampled or the maximum number of questions is reached. Finally, the Guesser employs the entire dialogue history $D$ and the object list $O$ as inputs to predict the object. The game is considered successful if $o^*$ is selected, otherwise it is deemed a failure.


% \subsection{Unified-Belief Mapper (UBM)}

% During the dialogue process, the continuously acquired information by QGen and Guesser greatly contributes to the final decision. However, QGen mainly influences the progression of the dialogue during the intermediate stages, while Guesser mainly influences the result of the task~\cite{le2022multimodal}. Compared with QGen, Guesser's outcome determines the success of the task. So Guesser is more sensitive to numerical changes and has lower fault tolerance. Therefore, as illustrated in Figure 2, our approach is to transfer the information from QGen to Guesser. QGen and Guesser have an Intersection over Union (IoU) overlap relationship with the corresponding bounding boxes of their visual features. Based on this, we construct the Unified-Belief Mapper, a function $f_{ubm}$ that integrates QGen and Guesser information, to get $p_{ubm}$. $p_{ubm}$ refers to the IoU between the bounding boxes of $p_{qgen}\left( O_{1:N} \right) $ and $p_{guesser}\left( O_{1:N} \right) $. By performing a Hadamard Product between $p_{ubm}$ and $p_{qgen}\left( O_{1:N} \right) $ (objects belief distribution in QGen) and then summing all objects, we adjust the dimension to update $p_{guesser}\left( O_{1:N} \right) $ (objects belief distribution in Guesser). Our goal is only to update the internal differences among different objects,thus we perform a normalization operation to ensure that the sum of $p_{guesser}\left( O_{1:N} \right) $ is equal to 1.

% \begin{small}
% \begin{gather}
% p_{ubm}=f_{ubm}\left( p_{qgen}\left( O_{1:N} \right) , p_{guesser}\left( O_{1:N} \right) \right) , \\
% p_{sum}=\sum\nolimits_{i=1}^N{\left( p_{ubm}\odot p_{qgen}\left( O_{1:N} \right) \right)} , \\
% p_{guesser}\left( O_{1:N} \right) =p_{guesser}\left( O_{1:N} \right) +p_{sum} ,
% \end{gather}
% \begin{equation}
%     \begin{split}
%         p_{guesser}\left( O_{1:N} \right) & =norm\left( p_{guesser}\left( O_{1:N} \right) \right) \\
%         & =\frac{p_{guesser}\left( O_{1:N} \right)}{\sum\nolimits_{i=1}^N{p_{guesser}\left( O_{1:N} \right)}}
%     \end{split}
% \end{equation}
% \end{small}
\subsection{Applicability in Two Settings}

Our method can be treated as a plugin that can be applied to different models.
We divide the results into two settings for appropriate comparison.

\textbf{Without $r_s$.} If the models do not cooperate with TSADE, we only train the Oracle, Guesser, and QGen models independently using cross-entropy loss.
% We first independently train the Oracle, Guesser, and QGen models using cross-entropy loss in SL. 
% Then, keeping the Oracle and Guesser models fixed, we train the QGen model in the described RL framework with the rewards TSADE proposed.
Then, if the models cooperate with TSADE, we keep the Oracle and Guesser models fixed and train the QGen model in the described RL framework with the rewards proposed by TSADE.
There is no 0-1 reward, which refers to $r_s$ from FS~\cite{strub2017end}.
FS can be considered as DV+$r_s$.
The purpose of the setting is to compare the model's results in SL with the model's results with TSADE in RL.
% The purpose of the setting is to prove that replacing the 0-1 reward with the reward proposed by TSADE can still improve model performance.
The training approach for QGen in SL follows the same details as DV~\cite{de2017guesswhat}.  %Readers can refer to the supplementary materials for more details. 

\textbf{With $r_s$.} We first independently train the Oracle, Guesser, and QGen models. Then keeping the Oracle and Guesser models fixed, we train the QGen model in the described RL framework using a comprehensive set of rewards. 
If the models follow the 0-1 reward proposed by FS, these rewards only include $r_s$.
On this basis, if the models cooperate with TSADE, these rewards consist of the $r_s$ and the rewards proposed by TSADE.
The purpose of the setting is to demonstrate the improvement results of TSADE on state-of-the-art models based on 0-1 reward in RL.

 % The previous model architecture ADVSE~\cite{2020Answer} can also be trained as baseline under both SL and RL. Therefore, we also apply our method to ADVSE to further demonstrate the effectiveness and generality of our method.
