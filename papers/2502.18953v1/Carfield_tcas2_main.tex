\pdfoutput=1
\documentclass[journal]{IEEEtran}

%%%%%%%%%%%%%%%%%%
%%   PACKAGES   %%
%%%%%%%%%%%%%%%%%%

\RequirePackage[utf8]{inputenc}
\RequirePackage[english]{babel}
\RequirePackage{amsmath,amsfonts}
\RequirePackage{algorithmic}
\RequirePackage[pscoord]{eso-pic}
\RequirePackage{textcomp}
%\RequirePackage{xcolor}
%\RequirePackage[a4paper, total={184mm,239mm}]{geometry}
\PassOptionsToPackage{detect-all, per-mode=symbol, range-units=single, range-phrase=~to~}{siunitx}
\RequirePackage{siunitx}
\RequirePackage{nicefrac}
\RequirePackage[hidelinks]{hyperref}
\RequirePackage{orcidlink}
\RequirePackage[noabbrev,capitalise]{cleveref}
%\RequirePackage{float}
\RequirePackage[caption=false]{subfig}
\RequirePackage{url}
\RequirePackage{lipsum}
\RequirePackage{placeins}
\RequirePackage{xfrac}
\RequirePackage{float}

\RequirePackage{amssymb}
\RequirePackage{pifont}

\RequirePackage{threeparttable}
\RequirePackage{booktabs}
\RequirePackage{colortbl}
\RequirePackage{tabularx}
\RequirePackage{makecell}

\RequirePackage{graphicx}
\RequirePackage{glossaries}
\RequirePackage{xspace}
\RequirePackage{lipsum}
\RequirePackage{microtype}
\RequirePackage{multirow}
%\RequirePackage{subcaption}
%\RequirePackage{tikz}

% Custom commands
\newcommand{\glsf}[1]{\glsreset{#1}\gls{#1}}
\newcommand{\glsplf}[1]{\glsreset{#1}\glspl{#1}}
\newcommand{\Glsf}[1]{\glsreset{#1}\Gls{#1}}
\newcommand{\Glsplf}[1]{\glsreset{#1}\Glspl{#1}}
\newcommand{\glsu}[1]{\glsunset{#1}\gls{#1}}
\newcommand{\glsplu}[1]{\glsunset{#1}\glspl{#1}}
\newcommand{\Glsu}[1]{\glsunset{#1}\Gls{#1}}
\newcommand{\Glsplu}[1]{\glsunset{#1}\Glspl{#1}}
\newcommand{\etal}{\emph{et al.}}
\newcommand{\x}{$\times$}

% for tables
 %\usepackage[table,xcdraw]{xcolor}
 \usepackage{colortbl}

\hyphenation{op-tical net-works semi-conduc-tor IEEE-Xplore}
% updated with editorial comments 8/9/2021

% magic
%\renewcommand{\baselinestretch}{0.98}
\widowpenalty0
\clubpenalty0
\brokenpenalty0

%\setlength{\textfloatsep}{0.2\baselineskip plus 0.2\baselineskip minus 0.5\baselineskip}
%\setlength{\abovecaptionskip}{0.25\baselineskip plus 0.2\baselineskip minus 0.5\baselineskip}

\setlength{\textfloatsep}{10pt} % distance between floats on the top or the bottom and the text
%\setlength{\floatsep}{4pt} % distance between two floats
%\setlength{\intextsep}{4pt} % distance between floats inserted inside the page text (using h) and the text proper
\setlength{\dbltextfloatsep}{10pt} %  distance between a float spanning both columns and the text
%\setlength{\dblfloatsep}{4pt} %  distance between two floats spanning both columns
%\setlength{\abovecaptionskip}{2pt}
%\setlength{\belowcaptionskip}{0pt}

\renewcommand{\subsection}[1]{\paragraph*{\textbf{#1}}}
\renewcommand{\subsubsection}[1]{\paragraph*{\textbf{#1}}}
%%
%% end of the preamble, start of the body of the document 

\microtypesetup{expansion=true, protrusion=true}
\raggedbottom
\tolerance=1000
\hbadness=10000
\sloppy

\begin{document}

\begin{acronym}
\acro{gan}[GANs]{Generative Adversarial Networks}
\acro{rl}[RL]{Reinforcement Learning}
\acro{pae}[PAE]{Periodic Autoencoder}
\acro{fld}[FLD]{Fourier Latent Dynamics}
\acro{ppo}[PPO]{Proximal Policy Optimization}
\acro{fft}[FFT]{Fast Fourier Transform}
\acro{pca}[PCA]{Principal Component Analysis}
\acro{dfm}[DFM]{Deep Fourier Mimic}
\acro{dof}[DoF]{Degrees of Freedom}
\acro{mlp}[MLPs]{Multi-Layer Perceptrons}
\end{acronym}



\title{A Reliable, Time-Predictable Heterogeneous SoC for AI-Enhanced Mixed-Criticality Edge Applications}

\author{
    Angelo Garofalo~\orcidlink{0000-0002-7495-6895}\IEEEauthorrefmark{10}, 
    Alessandro Ottaviano~\orcidlink{0009-0000-9924-3536}\IEEEauthorrefmark{10}, 
    Matteo Perotti~\orcidlink{0000-0003-2413-8592}, 
    Thomas Benz~\orcidlink{0000-0002-0326-9676}, 
    Yvan Tortorella~\orcidlink{0000-0001-8248-5731}, 
    Robert Balas~\orcidlink{0000-0002-7231-9315}, 
    Michael Rogenmoser~\orcidlink{0000-0003-4622-4862}, 
    Chi Zhang~\orcidlink{0009-0003-3243-0558}, 
    Luca Bertaccini~\orcidlink{0000-0002-3011-6368}, 
    Nils Wistoff~\orcidlink{0000-0002-8683-8060}, 
    Maicol Ciani~\orcidlink{0009-0003-7861-9129}, 
    Cyril Koenig~\orcidlink{0009-0007-9771-1691}, 
    Mattia Sinigaglia~\orcidlink{0000-0002-3350-8789}, 
    Luca Valente~\orcidlink{0000-0002-7458-477X}, 
    Paul Scheffler~\orcidlink{0000-0003-4230-1381}, 
    Manuel Eggimann~\orcidlink{0000-0001-8395-7585}, 
    Matheus Cavalcante~\orcidlink{0000-0001-9199-1708}, 
    Francesco Restuccia~\orcidlink{0000-0001-6955-1888}, 
    Alessandro Biondi~\orcidlink{0000-0002-6625-9336}, 
    Francesco Conti~\orcidlink{0000-0002-7924-933X}, 
    Frank K. Gurkaynak~\orcidlink{0000-0002-8476-554X}, 
    Davide Rossi~\orcidlink{0000-0002-0651-5393}, 
    Luca Benini~\orcidlink{0000-0001-8068-3806}
    % stops a space
    \vspace{-0.5cm}
    \IEEEcompsocitemizethanks{%
    \IEEEauthorrefmark{10} Both authors contributed equally to this research.\protect\\
    \IEEEcompsocthanksitem A.~Garofalo, A.~Ottaviano, M.~Perotti, T.~Benz, R.~Balas, M.~Rogenmoser, C.~Zhang, L.~Bertaccini, N.~Wistoff,  C.~Koenig, P.~Scheffler, M.~Eggimann, M.~Cavalcante, F.~K.~Gurkaynak,  and L.~Benini are with the D-ITET department at ETH Zurich, Switzerland.\protect\\
    E-mail: \{agarofalo,aottaviano\}@ethz.ch\protect\\
    Y.~Tortorella, M.~Ciani, M.~Sinigaglia, L.~Valente, F.~Conti, D.~Rossi are with the Department of Electrical, Electronic, and Information Engineering (DEI), University of Bologna, Italy.\protect\\ 
    F.~Restuccia is with the Department of Computer Science and Engineering, University of California in San Diego (UCSD), California, USA.\protect\\ 
    A.~Biondi is with the Department of Excellence in Robotics \& AI, Scuola Superiore Sant'Anna, Pisa, Italy.\protect\\
    }
}
% The paper headers
\markboth{Journal of \LaTeX\ Class Files,~Vol.~14, No.~8, August~2021}%
{Shell \MakeLowercase{\textit{et al.}}: A Sample Article Using IEEEtran.cls for IEEE Journals}

\maketitle


\begin{abstract}
Next-generation mixed-criticality Systems-on-chip (SoCs) for robotics, automotive, and space must execute mixed-criticality AI-enhanced sensor processing and control workloads, ensuring reliable and time-predictable execution of critical tasks sharing resources with non-critical tasks, while also fitting within a sub-2W power envelope. To tackle these multi-dimensional challenges, in this brief, we present a 16nm, reliable, time-predictable heterogeneous SoC with multiple programmable accelerators. Within a 1.2W power envelope, the SoC integrates software-configurable hardware IPs to ensure predictable access to shared resources, such as the on-chip interconnect and memory system, leading to tight upper bounds on execution times of critical applications. To accelerate mixed-precision mission-critical AI, the SoC integrates a reliable multi-core accelerator achieving 304.9 GOPS peak performance at 1.6 TOPS/W energy efficiency. Non-critical, compute-intensive, floating-point workloads are accelerated by a dual-core vector cluster, achieving 121.8 GFLOPS at 1.1 TFLOPS/W and 106.8 GFLOPS/mm$^2$.
\end{abstract}

\begin{IEEEkeywords}
Heterogeneous mixed-critical SoC, time-predictable SoC, hardware acceleration, fault tolerant hardware.
\end{IEEEkeywords}


\section{Introduction}

\begin{figure*}
    \centering
    \includegraphics[width=\textwidth]{figures/Introduction.pdf}
    \caption{Showing the novel problem statement applied to traffic prediction use case. Multiple unstructured observations from the past are used to reconstruct a hidden traffic state from which a full traffic state is forecast with a set of query locations. }
    \label{fig:intro}
\end{figure*}

% Was sagen denn die anderen warum Traffic Prediction gut ist? 
Forecasting the traffic in the near future is an important task for city management.
Data from the near past is used to predict future traffic states with spatio-temporal Graph Neural Networks \cite{bui22}.
Accurate prediction provides the opportunity to optimize traffic flow, reduce traffic jams and increase air quality \cite{Po19}.

% Wieso ist Sparsity in allen Dimensionen wichtig.
While traffic prediction relies on the availability of data from traffic sensors, there exists a plethora of reasons why sensors may stop working temporarily, such as simple errors, energy saving, or overloaded communication systems.
Considering small- or medium-sized cities, the coverage of sensors may be low because the sensors are too expensive or not available.
Also, the sensors are typically static and do not adapt to changes in the traffic flow (e.g. caused by a construction site), which motivates moving sensors that for example could be mounted on cars. 
However, both missing and moving sensors introduce sparsity, since measurements may not be available for all locations at all times.
This sparsity must be explicitly addressed in traffic prediction for a realistic application scenario, which is illustrated in figure \ref{fig:intro}.
From one hour of data on Sunday morning, only few observations of the traffic state are available at each timestep.
The number of observations may differ throughout the observed time and the observation itself can be distributed arbitrarily in the city. 
We assume a relatively low number of sensors to account for resource saving and sensor failure in our proposed framework SUSTeR.
The task is to predict the dense traffic state one timestep after the observations at all possible sensor locations.
We study this problem on the traffic dataset Metr-LA and PEMS-BAY to test our assumption that only a fraction of the sensor values would be enough for good predictions.
By modifying an existing traffic dataset, we are able to compare our results from very sparse observations to the bottom line with all information available.
A successful study will provide insights in how sensors in new cities can be reduced before installing them and further mobile sensors would save more resources and are able to adapt to new traffic situations.
We argue that in order to be adaptable to other cities and changes in traffic flows, prior information like the road network should be neglected and just the sparse observations considered.
This comes with the added benefit of making our solution applicable in regions where no openly available road network is maintained or pathways change frequently (e.g. flood areas, animal observations). 


The aforementioned problem is novel and more challenging than the commonly considered traffic prediction problem, since there exist very few observations in each input sample.
Current works for the traffic prediction problem do not consider any missing values. \cite{Li2021, Shao22}
A common method among state of the art approaches is the usage of Graph Neural Networks on graphs that model the sensor network \cite{bui22}.
The values of a sensor are applied to the same graph node for each timestep which prohibits any non-stationary sensors . 
With fixed sensor locations, the resulting sensor network is highly correlated with the road network.
Streets connecting two intersections with sensors should be also an interesting point for correlations in the sensor network.
However, variable observations and high temporal sparsity rates can not be modeled adequately in a static network.
We show in our experiments that the road network has only a small influence on the traffic predictions.

Besides the traffic prediction for future timesteps, some works explore the field of traffic speed imputation \cite{Cini22, Cuza22} where missing sensor values are predicted.
But the amount of missing values is assumed to be at most 80\%, which on average are still over 40 given sensors in each timestep in the Metr-LA dataset with a total of 207 sensors.
We consider up to 99.9\% missing values which are on average 2.4 observations in each timestep that are used as input.
Such high sparsity rates drastically decrease the chance that multiple values are present in one input sample from the same sensor location, which makes it challenging to recognize and learn temporal correlations for each location on its own.

High sparsity rates (>95\%) result in few sensor values, but if a reconstruction of the traffic state would be possible, we question if spatio-temporal graphs require nodes for each sensor.
In SUSTeR we utilize only a small amount of graph nodes for the encoding of information and do not relate such nodes to the sensor network.
We call this the hidden graph (see figure \ref{fig:intro}), which is still able to reconstruct the complete traffic state.
Due to the reduced number of nodes SUSTeR achieves faster runtimes, as shown in the experiments.
This hidden graph is not embedded directly in the spatial domain, which is why the assignment of observations, as well as the querying of the future traffic, is done with an encoder and a decoder, implemented as neural networks.
The decoding from the hidden graph to future values depends on a set of query locations.
Figure \ref{fig:intro} shows the query locations as given from outside and in combination with the reconstructed traffic state the future values are predicted.

To construct the hidden graph we encode observations from each timestep into from multiple graphs, one for each timestep. 
The graphs are created in a residual style and information is added to the node embeddings from the previous timesteps.
We choose this method to incorporate all timesteps equally into the hidden state because the redundant information along the past is non-existing for high sparsity rates.
From the sequence of graphs where our framework inserted the observations step by step we apply STGCN \cite{Yu18}, an algorithm for traffic prediction to find and learn the spatio-temporal correlations on our small number of graph nodes.
The first future timestep of the STGCN is our hidden graph in which the traffic state is reconstructed. 

% Recent work has an implicit embedding of the graph nodes into the spatial domain as the assignment from the sensor to graph node is fixed one by one.
% Because the graph has the same structure as the road network spatio-temporal correlations can be learned between those sensors.
% We reduce the number of nodes and use a non-linear assignment learned data-driven from the observations.

We find in the experiments that SUSTeR outperforms the plain STGCN and modern traffic prediction frameworks like D2STGNN for high sparsity rates $(\geq 99\%)$.
This is equivalent to only $0.2$ to $2.4$ observation for each timestep on average.
SUSTeR uses fewer parameters than the baselines and can train faster and with less training data.
Our main contributions can be summarized as follows:
\begin{itemize}
    \item We introduce a sparse and unstructured variant of the traffic prediction problem with sparsity in all dimensions. The sensors report only a fraction of their values and are arbitrarily distributed in the spatial domain.
    \item We propose SUSTeR, a framework around the STGCN architecture, which maps sparse observations onto a dense hidden graph to reconstruct the complete traffic state.
    Our code is available at github.\footnote{https://github.com/ywoelker/SUSTeR}
    \item We conducts experiments that show that SUSTeR outperforms the baselines in very sparse situations ($\geq 95\%$) and has a competitive performance in low sparsity rates.
    % \item SUSTeR trains a third faster than the next competitor.
\end{itemize}

\section{SoC Architecture}
% archi overview
Fig.~\ref{fig:soc-archi} shows the SoC architecture, operating across three clock domains, each driven by a dedicated PLL. A \gls{secd} acts as the SoC’s \gls{hwrot}, handling the secure boot and crypto services. 
% hard real-time
For hard real-time and safety-critical tasks, the \gls{safed} features a lockstepped triple-RV32-core for reliability, \gls{ecc}-protected private instruction and data \gls{spm} for deterministic memory access, and an enhanced \gls{rv} \gls{clic} with 6-cycle interrupt latency.

%enhanced with custom ISA extensions, namely \textit{Xfastirq}, providing as low as 6-cycle interrupt latency. 

% soft real-time
Soft real-time tasks with less stringent safety requirements run in the \gls{hostd}, based on the Cheshire platform~\cite{ottavianoCheshireLightweightLinuxCapable2023}, extended with a dual-core RV64GCH processor with hardware-assisted virtualization capabilities: it supports concurrent execution of RTOS and GPOS \glspl{vg} through \gls{rv}-compliant H-extension, and virtual interrupts are managed by per-core \gls{vclic} to reduce interrupt handling and context-switching latency among \glspl{vg}. 
%
The \gls{hostd} cores have private 32KiB L1 \glspl{dc} and share a 128KiB \gls{dpllc}, which interfaces with two external HyperRAM chips via a 400Mb/s deterministic access time HyperBUS memory controller. 
%
The system interconnect is based on a 64b AXI4 bus. A 1MiB on-chip shared L2 \gls{dcspm} is accessible by all domains with 128b/cyc bandwidth. The SoC also includes conventional peripherals, as shown in Fig.~\ref{fig:soc-archi}. Compute-intensive \gls{ai}/\gls{dsp} workloads are offloaded to two domain-specific accelerators: the vector and \gls{amr} clusters. % detailed later in this section. 

\subsubsection{Hardware IPs For Time-Predictability}

\glspl{mct} on the SoC can interfere through the interconnect and memory endpoints - L2 \gls{dcspm} and HyperRAM accessed via the \gls{dpllc}, resulting in non-deterministic behavior and significantly increasing execution time of \glsplf{tct}, as detailed in \looseness=-1 Sec.~\ref{sec:interference-free}.

To ensure predictable communication over AXI4, each initiator is equipped with a software programmable \glsf{tsu}, shown in Fig.~\ref{fig:hw-ips-for-predictability}.a), aimed at reducing execution latency of \glspl{tct} in interference scenarios by controlling each initiator's bandwidth, thereby enforcing a configurable latency upper bound. The \gls{tsu} comprises three components: 
\textbf{1)} The \textit{\gls{gbs}} fragments long AXI4 bursts to a configurable size to ensure fair arbitration between asynchronous initiators with burst capabilities running \glspl{nct} (e.g., \gls{dma} engines paired with \glspl{dsa}) and initiators running higher-priority \glspl{tct}; 
\textbf{2)} The \textit{\gls{wb}} buffers \emph{AW} and \emph{W} channels, forwarding \emph{AW} requests and \emph{W} bursts only when write data is fully within the buffer. This prevents an initiator from holding the \emph{W} channel, avoiding interconnect stalls; 
\textbf{3)} The \textit{\gls{tru}} assigns each initiator a fixed transfer budget within a configurable communication period. 

Favoring \glspl{tct} in the interconnect unavoidably affects the performance of \glspl{nct}. However, \glspl{mct} conflicting on L2 or HyperRAM endpoints can be further isolated by creating interference-free memory paths. 
%
\gls{dcspm} can be accessed via two AXI4 ports and addressed in contiguous mode to isolate its physical banks. This is configurable at runtime with zero additional latency through aliased memory map addresses, as shown in Fig.~\ref{fig:hw-ips-for-predictability}.b). For \glspl{nct} sharing L2 data, the \gls{dcspm} operates in interleaved mode to statistically minimize conflicts. 
%
For \glspl{mct} accessing HyperRAM, the \gls{dpllc} reduces non-deterministic cache misses by creating set-based spatial partitions of configurable sizes, isolated in hardware and assigned to \glspl{vg}' tasks via \textit{part\_id} identifiers linked to AXI4 user signals, as shown in Fig.~\ref{fig:hw-ips-for-predictability}.c). Predictable cache states associated with tasks sharing a partition are maintained by selective partition flushing, preserving the isolation of other \looseness=-1 partitions. 

\subsubsection{Compact, Efficient, \gls{rv} Vector Cluster}
\label{sec:vector}
%\begin{figure}[t]
%    \centering
%    \includegraphics[width=0.95\linewidth]{FIGURES/Vector_cluster_archi.pdf}
%    \caption{\gls{rv} Vector Engine diagram and characterization.}
%    \label{fig:vector-cluster-archi}
%\end{figure}

%To minimize area and power in large \gls{vrf} vector processors designed for \gls{hpc}, which are unsuitable for embedded \glspl{mcs}, 
The proposed SoC integrates a cluster (Fig.~\ref{fig:soc-archi}, top-right) of two compact, energy-efficient \glspl{rvu} controlled by two 32b \gls{rv} scalar cores, forming a clock-gatable \gls{cc}. 
%
To minimize access energy on the \gls{vrf}, each \gls{rvu} instantiate 2KiB latch-based private \gls{vrf}, connected to the 16-banks, 1024b/cyc-bandwidth L1 \gls{spm}, via four independent 64b \gls{vlsu} ports and a low-latency interconnect. This allows vector engines to quickly perform unit-strided, non-unit-strided, and indexed memory accesses, improving compute efficiency for both dense and sparse workloads. A third \gls{rv} core in the cluster manages a 512b/cyc read/write DMA for double-buffered L2-L1 transfers. 

%
Each \gls{rvu} is a VLEN=512b RVZve64d processor, supporting formats from FP8 to FP64, bfloat16, integer, and mixed-precision, including \gls{sdotp} operations. Instructions are fetched by the scalar core, decoded by the vector controller to determine the vector length and element width, and executed by the \gls{vau}, achieving a maximum throughput of 256b/cyc for \gls{fp} operations. The \gls{vrf} is organized in four banks, each with 3 read and 1 write 256b ports to meet 3x256b/cyc input, 256b/cyc output bandwidth needs of the \emph{vfmacc} instruction. 
%
The vector cluster achieves 97.9\% \gls{fpu} utilization at 15.67 DP-FLOP/cyc on edge-sized \glspl{matmul}, up to 121.8 FLOP/cyc on FP8xFP8 \glspl{matmul}, improving performance by 23.8$\times$ to 190.3$\times$ over the \gls{hostd}.

\subsubsection{\gls{amr} Cluster for Mission-Critical \gls{ai}}
\label{sec:int_cluster}
\begin{figure}[t]
    \centering
    \includegraphics[width=0.98\columnwidth]{FIGURES/amr_cluster_features.png}
    \caption{a) Hardware fast recovery (HFR) mechanisms for dual-lockstep mode (DLM) case. b) HFR Finite State Machine. c) \gls{amr} performance.}
    \label{fig:fast-recovery-cluster}
\end{figure}

The \glsf{amr} cluster, shown in bottom right of Fig.~\ref{fig:soc-archi}, includes 12 RV32IMFC cores sharing a 32-banked 256KiB \gls{ecc}-protected L1 \gls{spm}, accessible through a one-cycle latency interconnect with 921Gb/s (@ 900MHz) bandwidth. A 64b/cyc read, 64b/cyc write DMA enables double-buffered L2-L1 data transfers. Mixed-precision integer \gls{dsp}/\gls{ai} tasks are accelerated by the cores through custom \gls{rv} extensions supporting SIMD \gls{sdotp} on data formats ranging from 16b to 2b (all possible mixed permutations). A custom \emph{mac-load} instruction increases MAC unit utilization to 94\% on \glspl{matmul}, overlapping \gls{sdotp} operations with load instructions. 

%
The RV32 cores can be reconfigured through the \gls{amr} hardware to prioritize reliable vs. more performant execution. In the \gls{indip}, all 12 cores operate in MIMD for maximum performance. In \glsf{dlm} /\gls{tlm} mode, six/four main cores have one/two shadow cores, and instructions are committed after a checker/voting mechanism if no error occurs. The \gls{amr} is runtime-programmable; reconfiguration among modes takes 82-183 clock cycles when switching from safety-critical to high-performance sections within application codes (Fig.~\ref{fig:fast-recovery-cluster}.c)). 

%
In case of errors, faulty cores are restored to the nearest reliable state in as few as 24 clock cycles thanks to the \glsf{hfr}, shown in Fig.~\ref{fig:fast-recovery-cluster}.a). 
The \gls{hfr} includes \gls{ecc}-protected recovery registers to back up the internal state of non-faulty cores cycle-by-cycle without extra latency. 
As shown in Fig.~\ref{fig:fast-recovery-cluster}.b), \gls{dlm} with \gls{hfr} prevents rebooting the cluster upon fault detection, while \gls{tlm} with \gls{hfr} is 15$\times$ faster than \gls{tlm} software recovery. 
%
When executing in \gls{dlm} (\gls{tlm}), the performance penalty is limited to 1.89$\times$ (2.85$\times$) compared to \gls{indip}, still achieving 23.1 MAC/cyc (15.3 MAC/cyc) \looseness=-1 on 8b \glspl{matmul}.

% In \gls{dlm}, with half the cores executing compared to \gls{indip}, the performance penalty is limited to 1.89$\times$ (2.85$\times$ in \gls{tlm} mode), still achieving 23.1 MAC/cyc on 8b \glspl{matmul} while ensuring reliability.

\section{Evaluation and Measurements}
\label{sec:evaluation-measurements}
\begin{figure}[t]
    \centering
    \includegraphics[width=0.98\columnwidth]{FIGURES/chip.pdf}
    \caption{a) SoC micro-graph; b) testing setup. }
    \label{fig:chip-die}
\end{figure}
%\begin{figure}[t]
%    \centering
%    \includegraphics[width=\linewidth]{FIGURES/host_safed_sweeps_v2.pdf}
%    \caption{Voltage, Frequency, Power sweeps for HOST domain and SAFED.}
%    \label{fig:host-safed}
%\end{figure}
\begin{figure}[t]
    \centering
    \includegraphics[width=0.98\columnwidth]{FIGURES/sweep_clusters.pdf}
    \caption{Voltage/Frequency/Power and Performance/Energy Efficiency sweeps of \gls{amr}  (\textbf{a}, \textbf{b}) and vector (\textbf{c}, \textbf{d}) clusters.}
    \label{fig:clus-sweeps2}
\end{figure}
\begin{figure}[t]
    \centering
    \includegraphics[width=.98\columnwidth]{FIGURES/interference_aware_plots.pdf}
    \caption{Inteference-aware execution of \glspl{mct} on the SoC: a) \gls{hostd} runs a \gls{tct} accessing HyperRAM, while vector cluster interferes; b) \glspl{mct} running on \gls{amr} and vector clusters in double-buffering, sharing AXI and \gls{dcspm} resources.}
    \label{fig:interference-free}
\end{figure}

Fig.~\ref{fig:chip-die} shows the chip micrograph and the standalone \gls{pcb} designed for testing it. The SoC, fabricated with Intel 16nm FinFet technology, operates from 0.6V to 1.1V, with an overall 1.2W power envelope at a nominal 0.8V. %Both \gls{hostd} and \gls{safed} achieve a maximum operating frequency of 1GHz.

%IT is implemented with the Intel 16nm FinFet technology, and it operates from 0.6V to 1.1V. %To test the prototype, we design a custom-made bring-up PCB shown in Fig.~\ref{fig:chip-die}. This board is connected to external voltage for power delivery, while it integrates a 100MHz oscillator that provides the input to the on-chip PLLs. The board also features a set of peripherals, including HyperRAM chips, and connectors to boot and run applications on the chip.  

%%%%%%%%%%%%%%%%%%%%%%%%%%%%%%%%%%%%%%%%%%%%%%%%%%
%%%% PERF vs E.E. sweeps
%%%%%%%%%%%%%%%%%%%%%%%%%%%%%%%%%%%%%%%%%%%%%%%%%%
\subsubsection{Performance vs Energy Efficiency}
\label{sec:perf-vs-ee}
% : the first runs FP64 \glspl{matmul}, the second the CoreMark benchmark. Both achieve a maximum operating frequency of 1GHz. 
%Fig.~\ref{fig:host-safed} shows the frequency-power sweeps of the \gls{hostd} and \gls{safed}, while 
Fig.~\ref{fig:clus-sweeps2} evaluates the two clusters, in terms of performance and energy efficiency. Measurements are taken at the maximum operating frequency, sweeping the supply voltage from 0.6V to 1.1V. 

The \gls{amr} cluster is benchmarked on integer \glspl{matmul}, the core kernel of \gls{dsp} and \gls{ai} tasks, spanning all supported operands' precisions, from 32b down to 2b, including mixed-precision formats. The cluster's cores are configured to run either in \gls{indip} mode for the best performance or in \gls{dlm} mode for a good trade-off between performance and reliability. The \gls{amr} cluster achieves up to 304.9 GOPS on 2b$\times$2b \glspl{matmul} (161.4 GOPS in \gls{dlm}) at 1.1V, 900 MHz, with a peak energy efficiency of 1.6 TOPS/W at 0.6V, 300 MHz (1.1 TOPS/W in \gls{dlm}).
%
Similarly, we benchmark the vector cluster on FP \glspl{matmul} and \gls{ffts}, spanning all supported precisions, reaching a peak performance of 122 GFLOPS on FP8 \glspl{matmul} at 1.1V and 1GHz, and a peak energy efficiency of 1.1 TOPS/W at 0.6V and 250 MHz.


%%%%%%%%%%%%%%%%%%%%%%%%%%%%%%%%%%%%%%%%%%%%%%%%%%
%%%% INTERFERENCE-FREE EXECUTION
%%%%%%%%%%%%%%%%%%%%%%%%%%%%%%%%%%%%%%%%%%%%%%%%%%
\subsubsection{Interference-aware \glspl{mct} execution}
\label{sec:interference-free}
Fig.~\ref{fig:interference-free} shows how the hardware IPs introduced in Sec.~\ref{fig:hw-ips-for-predictability} enable interference-aware execution of \glsplf{mct}. 
%
In Fig.~\ref{fig:interference-free}.a) the \gls{hostd} runs a \glsf{tct} accessing HyperRAM via the \gls{dpllc} with contiguous stride, while the system DMA interferes by asynchronously transferring data from HyperRAM to the \gls{dcspm} with linear bursts. The measurements show the task latency and jitter, as well as the \gls{dpllc} misses generated by the eviction of cache lines among interfering tasks. In unregulated interference, the \gls{tct} latency degrades by 225$\times$ compared to the isolated (no interference) case (i.e. no interfering DMA). Tuning the \textit{granular burst splitter} and the \textit{traffic regulation unit} of the \glsf{tsu} in software, we regulate the traffic on the interconnect, reducing the latency by 44.4$\times$ compared to the unregulated case. The \gls{tsu} incurs an additional latency of at most 1 clock cycle due to its write buffer. Moreover, by assigning $> 50\%$ spatial partition of the \gls{dpllc} to the \gls{tct}, we reduce cache misses, achieving 75\% of the isolated (no interference) performance.

In a second scenario, the \gls{amr} cluster executes a compute-intensive \gls{tct} in reliable mode, and the vector cluster interferes by executing a \gls{fp} \gls{matmul}. Both accelerators move data in double-buffering from L2 to private L1, overlapping data transfer and computation phases. 
In Fig.~\ref{fig:interference-free}.b), R-E2, the performance of \gls{amr} cluster drops by 12.2$\times$ due to conflicts generated by the vector cluster on the interconnect and the \gls{dcspm}. Programming the \gls{tsu} to regulate the traffic in favor of the \gls{amr} cluster, we reach 95\% of its isolated (no interference) performance (R-E3), degrading the performance of \glsplf{nct}. However, using aliased addresses to access the \gls{dcspm}, we create private memory paths (at zero extra performance overhead) and achieve interference-free execution, matching the isolated (no interference) performance for both tasks \looseness=-1 (R-E4).

We show that these hardware IPs ensure interference-aware concurrent execution of \gls{mct} on shared SoC resources, prioritizing \glspl{tct} over \glspl{nct}  \looseness=-1 with negligible performance overhead.
%Our assessment shows that these hardware IPs ensure interference-aware, concurrent execution of \glspl{mct} on shared resources, prioritizing \glspl{tct} with minimal overhead on \glspl{nct}.%, unlike costly software-based approaches in commodity \gls{mcs} processors. %We discuss these comparisons in \looseness=-1 Sec.~\ref{sec:mcs-soa}.








\section{Comparison with State-of-the-Art}

\begin{figure}[t]
    \centering
    \includegraphics[width=0.98\columnwidth]{FIGURES/SoA_safety_comparison.pdf}
    \caption{Comparison against \gls{soa} heterogeneous SoC for \gls{mcs}.}
    \label{fig:soa-safety}
\end{figure}

\subsection{Comparison with Mixed-Criticality SoCs}
\label{sec:mcs-soa}

Fig.~\ref{fig:soa-safety} compares the proposed SoC against \gls{cots} and \gls{soa} heterogeneous SoC prototypes presented in the literature in the same power class. 
%
\gls{cots} solutions like the \textit{I.MXRT1170} crossover MCUs by NXP~\cite{nxp_industrial_control} mainly address heterogeneous workloads for the average case, relying on ARM Cortex-M cores and \gls{gp} acceleration units. However, they lack hardware mechanisms for reliability, virtualization, and time predictability, limiting their use as a \glspl{mcs}. Renesas' prototype~\cite{otani2728nm600MHz2019} focuses on reliable and virtualized RISC cores, reducing \glspl{vg} context-switch overheads, but it does not enable resource partitioning among concurrent \glspl{vg} in hardware, nor does it provide \gls{ai}/\gls{dsp} hardware acceleration. Another example is the \textit{Stellar} processor by ST~\cite{grossierASILDAutomotivegradeMicrocontroller2023}. Despite integrating reliable and real-time ARM Cortex-R52 cores operating in split-lock and an interconnect with Quality of Service mechanisms, it lacks hardware IPs to observe and partition shared memory endpoints. Moreover, it relies solely on \gls{gp} cores for compute-intensive tasks. 
%
In academia,~\cite{valenteHeterogeneousRISCVBased2024} targets \gls{ai}-enhanced nano-drone applications with a SoC featuring a single-core 64b \gls{rv} processor and an acceleration cluster similar to ours. However, it lacks features for reliability and time-predictability, limiting the~SoC usability in mission-critical scenarios. 

At a comparable power envelope, the proposed SoC is the only one that integrates comprehensive hardware mechanisms for time predictability, enabling software-controlled (e.g., via hypervisors) dynamic partitioning of shared resources like interconnects, caches, and \gls{spm}. It achieves the fastest interrupt latency response, 2$\times$, 3.3$\times$, and 8.3$\times$ lower than~\cite{nxp_industrial_control}, \cite{grossierASILDAutomotivegradeMicrocontroller2023}, and~\cite{valenteHeterogeneousRISCVBased2024}, respectively. Additionally, it supports concurrent execution of GPOS and RTOS, integrates a HW RoT secure domain with extensive security primitives, and a comprehensive set of programmable accelerators for compute-intensive mixed-criticality workloads, achieving \gls{soa} performance and energy \looseness=-1  efficiency.
%The I.MXRT1170 crossover MCUs by NXP~\cite{nxp_industrial_control} target ML-enhanced and motor control applications in industrial, robotics, and automotive domains. Operating below 1W, they integrate 32-b ARM Cortex-M cores and a 2D graphics accelerator but lack support for rich OSs, virtualization, and hardware mechanisms for reliability and time predictability.

%Similarly, despite the \textit{Stellar} processor by ST~\cite{grossierASILDAutomotivegradeMicrocontroller2023} integrates reliable, real-time ARM Cortex-R52 virtualized cores, and interconnect with Quality of Service mechanisms, it still misses comprehensive hardware mechanisms for time-predictability. Morever, it lacks domain-specific accelerators and relies solely on \gls{gp} cores for compute-intensive tasks.

%Renesas' \textbf{prototype}~\cite{otani2728nm600MHz2019} integrates four 32-b, 8-stage out-of-order RISC cores in dual-lockstep mode for high safety integrity, minimizing VM context-switch overheads. Despite its low-power design, it lacks specialized accelerators for AI and floating-point workloads in mixed-criticality edge applications.

%The academic \textbf{SoC in}~\cite{valenteHeterogeneousRISCVBased2024} targets nano-drones and ML-enhanced edge applications in 22nm technology, featuring a 64-b RISC-V host processor and an eight-core 32-b RISC-V programmable cluster with low-bitwidth AI capabilities. While efficient for compute-intensive AI workloads, it lacks hardware mechanisms for reliability and time-predictable execution, limiting its applicability in mission-critical scenarios.



% % NXP processor
% The I.MXRT1170 family of crossover MCUs by NXP~\cite{nxp_industrial_control} targets ML-enhanced and motor control applications in industrial, robotics, and automotive domains. With a power envelope under 1W, these MCUs integrate 32-b ARM Cortex-M based cores and a generic 2D graphics unit acceleration system. However, they lack support for rich OSs like Linux and virtualization. While designed for mixed-criticality application domains, they lack features for reliability and time-predictability in hardware.

% stellar processor
%The Stellar processors by ST~\cite{grossierASILDAutomotivegradeMicrocontroller2023} are tailored for automotive applications. Albeit showing a higher power consumption than~\cite{nxp_industrial_control}, they integrate a homogeneous cluster of six 32-b ARM Cortex-R52 processors, which can operate in split-lock mode for enhanced reliability. 
%These processors are designed for real-time execution of tasks, support virtualization, time-predictable interconnects through Quality of Service mechanisms (not further specified), and a firewall for task isolation. However, they rely on the Cortex-R cores for compute-intensive tasks and do not include domain-specific accelerators.

% Renesas
%Renesas' prototype~\cite{otani2728nm600MHz2019} features a low-power virtual-assisted processor designed to minimize context-switch overheads among different VMs. It consists of four 32-b  8-stage out-of-order RISC cores operating in dual-lockstep mode for high safety integrity levels. Despite the low-power envelope, this prototype works fine for control applications, but it does not provide any specialized accelerators to boost compute-intensive AI and FP workloads typical of modern mixed-criticality edge application pipelines. 

% shaheen
%In academia, the SoC in~\cite{valenteHeterogeneousRISCVBased2024} targets nano-drones and ML-enhanced edge applications in 22nm technology. It includes a single-core 64-b RISC-V host processor accelerated by a programmable cluster of eight 32-b RISC-V cores with floating-point and integer low-bitwidth AI capabilities. Despite this processor being suitable for compute-intensive AI workloads, it lacks hardware features for reliable and time-predictable task execution, limiting its real use in mission-critical scenarios.

% carfield
%At comparable power envelope, the proposed heterogeneous SoC is the only one that integrates, in the same die, comprehensive hardware mechanisms for time-predictability, enabling software-controlled (e.g. by Hypervisors) dynamic partitioning of shared resources like interconnects, caches, and SPM, and showing the fastest interrupt latency response, which is 2x, 3.3x and 8.3x lower than~\cite{nxp_industrial_control}, \cite{grossierASILDAutomotivegradeMicrocontroller2023}, \cite{valenteHeterogeneousRISCVBased2024}, respectively. Moreover, it supports concurrent execution of GPOS and RTOS while offering extensive security primitives within a HW RoT secure domain and an exhaustive set of programmable accelerators for executing AI and FP workloads at state-of-the-art performance and energy efficiency. 


\subsection{Comparison with edge \gls{ai} and Vector Processors}
\label{sec:soa-acc}
\begin{figure}[t]
    \centering
    \includegraphics[width=0.98\columnwidth]{FIGURES/SoA_Accelerators_table_v2.pdf}
    \caption{Comparison against \gls{soa} accelerators for \gls{fp} and edge \gls{ai}.}
    \label{fig:soa-accelerators}
\end{figure}

Fig.~\ref{fig:soa-accelerators} compares the proposed \gls{amr} and vector clusters against \gls{soa} vector and edge \gls{ai} processors. The \gls{amr} cluster compares favorably to a similar parallel cluster that supports low-bit-width integer arithmetic but lacks reliability features~\cite{valenteHeterogeneousRISCVBased2024}, limiting its use in mission-critical applications. In \gls{dlm}, it achieves up to 1.8$\times$ better performance (3.4$\times$ in \gls{indip}) on uniform 8b/4b/2b \glspl{matmul}, with  6.4$\times$ better area efficiency and comparable energy efficiency. In \gls{dlm}, it provides 2.6$\times$ higher performance than the most efficient 8b integer processor~\cite{juSystolicNeuralCPU2023}, which however lacks support for reliable execution modes. 
%
On \gls{fp} workloads, our vector cluster is the only one that operates over the full range of \gls{fp} formats, from 64b down to 8b, achieving the highest computing efficiency due to near-ideal resource utilization. Compared to~\cite{schmidtEightCore144GHzRISCV2022}, implemented in the same technology node, our cluster demonstrates 2.2$\times$ and 3$\times$ higher energy and area efficiency, respectively, on FP16 workloads. 
Additionally, it shows 2.43$\times$ better performance, 2$\times$, and 1.6$\times$ higher energy and area efficiency than~\cite{wang306Vecim28913GOPS2024}, despite the latter leveraging compute-in-memory in the VRF.


%Compare against vector processors: \cite{perottiYunOpenSource64Bit2023}



\section{Conclusion}
 In this brief, we presented a 16nm reliable, time-predictable heterogeneous RISC-V SoC for \gls{ai}-enhanced mixed-criticality applications. To the best of our knowledge, this is the first SoC that combines safety features for \glspl{mcs} with hardware IPs for time-predictable execution of \glspl{mct} and leading-edge domain-specialized programmable accelerators within the same heterogeneous SoC. With a peak performance of 304.9 GOPS at 1.6TOPS/W and 260.7 GOPS/mm$^2$, and 121.8 GFLOPS at 1.1TFLOPS/W and 107GFLOPS/mm$^2$, the proposed SoC offers a comprehensive solution for reliable and deterministic execution of \gls{ai}/\gls{dsp}-enhanced \gls{mc} edge applications, achieving \gls{soa} energy efficiency under \looseness=-1  1.2~W power envelope. %The hardware description of the design is released open-source to foster future research~\footnote{\texttt{\url{https://github.com/pulp-platform/carfield}}}.
%
%To the best of our knowledge, this is the first SoC to combine safety-critical features for MCS with HW IPs for time-predictable execution of MCTs and leading edge domain-specialized programmable processor clusters into the same heterogeneous SoC, providing a comprehensive solution for reliable and deterministic execution of ML-enhanced mixed-criticality edge applications, at state-of-the-art energy efficiency and less than 1.5W power envelope. 

%\textcolor{red}{The platform and all the non-technological IPs integrated in Carfield are released open source under a liberal licence to foster future software and hardware research on reliable, time-predictable, accelerated heterogeneous computing devices for mixed-criticality applications.}


\section*{Acknowledgments}
%Funding, same as DATE
This work was supported by the HORIZON CHIPS-JU TRISTAN (101095947) and ISOLDE (101112274) projects.%, \looseness=-1 within the HORIZON CHIPS-JU program. %The authors thank the Design Center at ETH Zurich, Bryan Casper, and Sirisha Kale from Intel for their support in chip \looseness=-1  fabrication.

\bibliographystyle{ieeetr}
\bibliography{REFERENCES/carfield_soc_bib_brev}


\end{document}


