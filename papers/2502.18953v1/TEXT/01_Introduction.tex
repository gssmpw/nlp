
\section{Introduction}

To meet the growing computational demands of \gls{ai}-enhanced applications in safety-critical domains like automotive, space, and robotics, near-sensors zonal/on-board controllers~\cite{burkacky2023getting} must efficiently handle compute-intensive tasks while ensuring reliable, time-predictable execution of \glspl{tct}~\cite{rehmRoadPredictableAutomotive2021}. To optimize latency, performance, efficiency, and area, Systems-on-chip (SoCs) must be designed as heterogeneous \glspl{mcs}~\cite{jiangReThinkingMixedCriticalityArchitecture2020}, combining \gls{gp} processors and \glspl{dsa}, within a sub-2W power budget typical of high-end microcontrollers~\cite{ojoReviewLowEndMiddleEnd2018}. 

These multi-faceted requirements pose significant design challenges. For instance, achieving time predictability without major performance overhead is hindered by resource conflicts among heterogeneous compute units sharing interconnects and memory endpoints~\cite{majumder2020partaa}, making it difficult to ensure a bounded \gls{wcet}. \textit{Spatial} and \textit{temporal} partitioning of hardware resources~\cite{kloda2019deterministic} are key techniques to mitigate these issues. While recent works partially address this problem in hardware, mainly focusing on interconnects~\cite{jiang2022axi, benz2025axi}, no comprehensive hardware implementation providing observability and controllability of shared resources for predictability has been demonstrated in silicon. As a result, freedom-from-interference or bounded \gls{wcet} on \gls{soa} SoCs~\cite{nxp_industrial_control, valenteHeterogeneousRISCVBased2024, grossierASILDAutomotivegradeMicrocontroller2023} is mainly achieved through software mechanisms, leading to significant performance overhead~\cite{6531078}.

\begin{figure}[!t]
    \centering
    \includegraphics[width=0.98\columnwidth]{FIGURES/carfield_archi_v4_no_frames.png}
    \caption{SoC Architecture.}
    \label{fig:soc-archi}
\end{figure}

\begin{figure*}[!t]
    \centering
    \includegraphics[width=0.99\linewidth]{FIGURES/hw_ips_for_predictability.png}
    \caption{Architectures of the hardware IPs for predictability: a) \gls{tsu}; b) \gls{dcspm}; c) \gls{dpllc}.}
    \label{fig:hw-ips-for-predictability}
\end{figure*}


Another challenge is ensuring that \glspl{dsa} meet both performance and dependability requirements while remaining compact and low-power.
%
On the one hand, traditional high-precision (\gls{fp}) tasks for \gls{dsp} (e.g., radar signal processing) and advanced predictive control require acceleration to meet stringent control-loop bandwidth constraints~\cite{islayem2024hardware, jhung2023hardware}. %While the vector paradigm is promising and flexible, current \gls{soa} solutions rely on area/power-hungry architectures that exceed the cost limits of edge \glspl{mcs}~\cite{schmidtEightCore144GHzRISCV2022}. 
%
On the other hand, a new class of \gls{ai}-intensive low arithmetic precision tasks (e.g. deep neural network inference for object detection, collision avoidance, condition monitoring) require not only tight real-time constraints but also fault tolerance and resiliency~\cite{islayem2024hardware}.

Despite domain-specific acceleration (\gls{dsp}-control, \gls{ai}-perception) being crucial for performance and energy efficiency, software programmability remains a fundamental pillar for keeping pace with rapidly evolving algorithms. The extensible nature of the \gls{rv} \gls{isa} with custom instructions offers a viable approach to specialize instruction processors, striking a balance between efficiency, performance, and programmability~\cite{valenteHeterogeneousRISCVBased2024, UMBRELLA_TCASI_1}. However, while recent research on programmable edge \gls{ai} processors has mainly focused on performance and efficiency metrics~\cite{valenteHeterogeneousRISCVBased2024, xu2023ultra, sun202340nm, juSystolicNeuralCPU2023}, there is a need for silicon-proven, dependable hardware architectures achieving \gls{soa} performance and energy efficiency.%, at the same time offering dependability \looseness=-1  features. 

%Despite domain specific acceleration (\gls{fp}-control, \gls{ai}-perception) is key for performance and energy efficiency, software programmability remains a fundamental pillar for keeping pace with rapidly evolving algorithms. The open-source \gls{rv} \gls{isa} While recent research on programmable edge \gls{ai} processors has focused mainly on performance and efficiency metrics~\cite{valenteHeterogeneousRISCVBased2024, xu2023ultra, sun202340nm}, there is the need for silicon-proven hardware architectures achieving \gls{soa} performance and energy efficiency, at the same time offering dependability \looseness=-1  features. 

%low-precision \gls{ai}-enhanced mission-critical autonomous decision tasks (e.g., object detection, collision avoidance, critical monitoring for ADAS or onboard satellite processing) require not only tight real-time constraints but also fault tolerance and resiliency~\cite{islayem2024hardware}. At the same time, software programmability remains a fundamental pillar for keeping pace with rapidly evolving algorithms. While recent research on programmable edge \gls{ai} processors has focused mainly on functional metrics~\cite{valenteHeterogeneousRISCVBased2024, xu2023ultra}, %there is a need to demonstrate silicon-proven, mission-critical-ready solutions that achieve \gls{soa} performance and energy efficiency.
%there is a need for silicon-proven hardware architectures achieving \gls{soa} performance and energy efficiency while retaining dependability features.



To the best of our knowledge, no SoC addresses these challenges holistically. To close this gap, we present a heterogeneous \gls{rv} SoC implemented in Intel 16nm FinFet technology, featuring three main contributions: 
%
\textbf{1)} end-to-end time-predictable execution of \glspl{mct} enabled by a set of software-programmable hardware IPs for configurable partitioning of shared resources at zero performance overhead, namely: a configurable \glsf{tsu} for interconnect, a \glsf{dpllc}, and the L2 \glsf{dcspm}; 
%
%\textbf{2)} a triple-lockstep safety unit for periodic real-time safety-critical tasks; 
%
\textbf{2)} a 1.07 TFLOPS/W, 107 GFLOPS/mm$^2$-at-FP8 dual-core vector floating-point mixed-precision acceleration cluster for \gls{fp} workloads; 
%
\textbf{3)} a 1.61 TOPS/W, 260.7 GOPS/mm$^2$ at 2b 12-core \gls{rv} integer accelerator cluster with runtime \gls{amr} to trade-off performance with reliability in integer mixed-precision mission-critical \gls{ai} tasks. To enable extensions, benchmarking and comparative analysis, we release the synthesizable hardware description of the SoC design open-source under a liberal license~\footnote{\texttt{\url{https://github.com/pulp-platform/carfield}}}. % to enable extensions, benchmarking and comparative analysis. 
