\documentclass{article}
%\documentclass[review,hidelinks,onefignum,onetabnum]{siamart220329}

%If you turn on SIAM style comment out amsthm and the theorem environments.
\usepackage[margin=1.375in, left=1in, right=2.375in]{geometry}

\usepackage{amsmath}
\usepackage{graphicx}
\usepackage{float}
\usepackage{enumerate}
%\usepackage{subfig}
\usepackage{amssymb}
\usepackage{amsfonts}
\usepackage{mathtools}
\usepackage{xcolor}
\usepackage{amsmath}
\usepackage{algorithm,algpseudocode}
\usepackage{subcaption}
\usepackage{caption}
\usepackage{verbatim}
\usepackage{bm}
\usepackage{amsthm} %Comment if put in SIAM style
\usepackage[toc,page]{appendix}
%\usepackage{refcheck} %Comment out hyperref if you turn on refcheck
\usepackage{hyperref} %Comment if you turn on refcheck
\hypersetup{
colorlinks=true,
}
\usepackage[nameinlink,capitalize]{cleveref}
\crefformat{equation}{(#2#1#3)}
\crefrangeformat{equation}{(#3#1#4) to~(#5#2#6)}
\crefmultiformat{equation}{(#2#1#3)}%
{ and~(#2#1#3)}{, (#2#1#3)}{ and~(#2#1#3)}

\usepackage[export]{adjustbox}
\usepackage{rotating}
\usepackage{multirow}
\usepackage{diagbox}
\usepackage{stmaryrd}
\usepackage[utf8]{inputenc}
\usepackage{xstring}
\usepackage{ifthen}
\usepackage{comment}
\usepackage[normalem]{ulem}

\usepackage{cancel}

\newcommand\hcancel[1]{\setbox0=\hbox{$#1$}%
\rlap{\raisebox{.45\ht0}{\rule{\wd0}{1pt}}}#1}

\newcommand\numberthis{\addtocounter{equation}{1}\tag{\theequation}}

%SPACING
%\usepackage{setspace}
%\onehalfspacing
%\doublespacing



%TIKZ
\usepackage{pgf,tikz,pgfplots}
\usetikzlibrary{decorations.markings}
\pgfplotsset{compat=1.15}
\usepackage{mathrsfs}
\usetikzlibrary{arrows}

%EXAMPLES OF NEWCOMMANDPerturbed
\newcommand{\Raph}[1]{\textcolor{blue}{\bf Raphael: #1}}
\newcommand{\Chris}[1]{\textcolor{blue}{\bf Chris: #1}}
\newcommand{\David}[1]{\textcolor{red}{\bf David: #1}}
\newcommand{\tyler}[1]{\textcolor{blue}{\bf Tyler: #1}}


\newboolean{hidebool}
\setboolean{hidebool}{false}

\ifthenelse{\boolean{hidebool}}
  {% True case
   \newcommand{\hide}[1]{}%
  }
  {% false case
   \newcommand{\hide}[1]{#1}%
  }


%USEFUL NEWCOMMANDS
\DeclareMathOperator{\tr}{tr}
\DeclareMathOperator{\Var}{Var}
\DeclareMathOperator{\rank}{rank}
\DeclareMathOperator{\diag}{diag}
\DeclareMathOperator{\range}{range}
\DeclareMathOperator*{\argmin}{arg\,min}
\renewcommand*{\eqref}[1]{(\ref{#1})}
\newcommand{\T}{\textsf{\textup{T}}}
\newcommand{\F}{\textsf{\textup{F}}}
\newcommand{\ALG}{\textsf{\textup{ALG}}}
\newcommand{\mA}{\bm{A}}
%THEOREM ENVIRONMENTS
%Remove all except for setting environment if you put in SIAM style
\newtheorem{theorem}{Theorem}[section]
\newtheorem{corollary}[theorem]{Corollary}
\newtheorem{lemma}[theorem]{Lemma}
\newtheorem{proposition}{Proposition}[theorem]
\newtheorem{setting}{Setting}[section]
\newtheorem*{remark}{Remark}
\theoremstyle{definition}
\newtheorem{definition}{Definition}[section]
\newtheorem{conjecture}{Conjecture}[section]

\title{Randomized block-Krylov subspace methods for low-rank approximation of matrix functions}

% \author{David Persson\thanks{Courant Institute of Mathematical Sciences, New York University, New York, NY 10012 USA, \href{mailto:dup210@nyu.edu}{dup210@nyu.edu}} \thanks{Center for Computational Mathematics, Flatiron Institute, New York, United States, \href{mailto:dpersson@flatironinstitute.org}{dpersson@flatironinstitute.org}} \and Tyler Chen\thanks{Courant Institute, New York University, 251 Mercer St., New York NY, USA, \href{mailto:tyler.chen@nyu.edu}{tyler.chen@nyu.edu}}~\thanks{Tandon School of Engineering, New York University, 370 Jay St., New York NY, USA. } \and Christopher Musco\thanks{Tandon School of Engineering, New York University, 370 Jay St., New York NY, USA, \href{mailto:cmusco@nyu.edu}{cmusco@nyu.edu} }}

% Chris: I simplified the thanks
\author{David Persson\thanks{New York University \& Flatiron Institute \href{mailto:dup210@nyu.edu}{dup210@nyu.edu}, \href{mailto:dpersson@flatironinstitute.org}{dpersson@flatironinstitute.org}}
\and Tyler Chen\thanks{New York University, \href{mailto:tyler.chen@nyu.edu}{tyler.chen@nyu.edu}} \and Christopher Musco\thanks{New York University, \href{mailto:cmusco@nyu.edu}{cmusco@nyu.edu} }}


%\date{}

\begin{document}

\maketitle

\begin{abstract}
The randomized SVD is a method to compute an inexpensive, yet accurate, low-rank approximation of a matrix. 
The algorithm assumes access to the matrix through matrix-vector products (matvecs).
Therefore, when we would like to apply the randomized SVD to a matrix function, $f(\bm{A})$, one needs to approximate matvecs with $f(\bm{A})$ using some other algorithm, which is typically treated as a black-box. Chen and Hallman (SIMAX 2023) argued that, in the common setting where matvecs with $f(\bm{A})$ are approximated using \emph{Krylov subspace methods} (KSMs), more efficient low-rank approximation is possible if we \emph{open this black-box}. They present an alternative approach that significantly outperforms the naive combination of KSMs with the randomized SVD, although the method lacked theoretical justification.
In this work, we take a closer look at the method, and provide strong and intuitive error bounds that justify its excellent performance for low-rank approximation of matrix functions.


%In the context of trace estimation Chen and Hallman (SIMAX 2023) introduced a version of the Hutch++ algorithm to estimate the trace of a matrix function $f(\bm{A})$. The novelty of their method comes from a more efficient method to compute a randomized low-rank approximation of $f(\bm{A})$ as a variance reduction technique for the stochastic trace estimator. The method constructs a low-rank approximation of $f(\bm{A})$ from a Krylov subspace of $\bm{A}$. Numerical experiments show that their method performs significantly better than similar randomized low-rank approximation methods. However, the theoretical justification was missing. In this work we provide an intuitive analysis of their method and we thus justify its excellent performance. 
\end{abstract}
%\David{We discussed that we should say when we are $(1+\varepsilon)$ away from optimal. Maybe it is better to ignore this, because we would need $\epsilon_1,\epsilon_2,\epsilon_3$ to be small too. }
\section{Introduction}
\label{sec:introduction}
The business processes of organizations are experiencing ever-increasing complexity due to the large amount of data, high number of users, and high-tech devices involved \cite{martin2021pmopportunitieschallenges, beerepoot2023biggestbpmproblems}. This complexity may cause business processes to deviate from normal control flow due to unforeseen and disruptive anomalies \cite{adams2023proceddsriftdetection}. These control-flow anomalies manifest as unknown, skipped, and wrongly-ordered activities in the traces of event logs monitored from the execution of business processes \cite{ko2023adsystematicreview}. For the sake of clarity, let us consider an illustrative example of such anomalies. Figure \ref{FP_ANOMALIES} shows a so-called event log footprint, which captures the control flow relations of four activities of a hypothetical event log. In particular, this footprint captures the control-flow relations between activities \texttt{a}, \texttt{b}, \texttt{c} and \texttt{d}. These are the causal ($\rightarrow$) relation, concurrent ($\parallel$) relation, and other ($\#$) relations such as exclusivity or non-local dependency \cite{aalst2022pmhandbook}. In addition, on the right are six traces, of which five exhibit skipped, wrongly-ordered and unknown control-flow anomalies. For example, $\langle$\texttt{a b d}$\rangle$ has a skipped activity, which is \texttt{c}. Because of this skipped activity, the control-flow relation \texttt{b}$\,\#\,$\texttt{d} is violated, since \texttt{d} directly follows \texttt{b} in the anomalous trace.
\begin{figure}[!t]
\centering
\includegraphics[width=0.9\columnwidth]{images/FP_ANOMALIES.png}
\caption{An example event log footprint with six traces, of which five exhibit control-flow anomalies.}
\label{FP_ANOMALIES}
\end{figure}

\subsection{Control-flow anomaly detection}
Control-flow anomaly detection techniques aim to characterize the normal control flow from event logs and verify whether these deviations occur in new event logs \cite{ko2023adsystematicreview}. To develop control-flow anomaly detection techniques, \revision{process mining} has seen widespread adoption owing to process discovery and \revision{conformance checking}. On the one hand, process discovery is a set of algorithms that encode control-flow relations as a set of model elements and constraints according to a given modeling formalism \cite{aalst2022pmhandbook}; hereafter, we refer to the Petri net, a widespread modeling formalism. On the other hand, \revision{conformance checking} is an explainable set of algorithms that allows linking any deviations with the reference Petri net and providing the fitness measure, namely a measure of how much the Petri net fits the new event log \cite{aalst2022pmhandbook}. Many control-flow anomaly detection techniques based on \revision{conformance checking} (hereafter, \revision{conformance checking}-based techniques) use the fitness measure to determine whether an event log is anomalous \cite{bezerra2009pmad, bezerra2013adlogspais, myers2018icsadpm, pecchia2020applicationfailuresanalysispm}. 

The scientific literature also includes many \revision{conformance checking}-independent techniques for control-flow anomaly detection that combine specific types of trace encodings with machine/deep learning \cite{ko2023adsystematicreview, tavares2023pmtraceencoding}. Whereas these techniques are very effective, their explainability is challenging due to both the type of trace encoding employed and the machine/deep learning model used \cite{rawal2022trustworthyaiadvances,li2023explainablead}. Hence, in the following, we focus on the shortcomings of \revision{conformance checking}-based techniques to investigate whether it is possible to support the development of competitive control-flow anomaly detection techniques while maintaining the explainable nature of \revision{conformance checking}.
\begin{figure}[!t]
\centering
\includegraphics[width=\columnwidth]{images/HIGH_LEVEL_VIEW.png}
\caption{A high-level view of the proposed framework for combining \revision{process mining}-based feature extraction with dimensionality reduction for control-flow anomaly detection.}
\label{HIGH_LEVEL_VIEW}
\end{figure}

\subsection{Shortcomings of \revision{conformance checking}-based techniques}
Unfortunately, the detection effectiveness of \revision{conformance checking}-based techniques is affected by noisy data and low-quality Petri nets, which may be due to human errors in the modeling process or representational bias of process discovery algorithms \cite{bezerra2013adlogspais, pecchia2020applicationfailuresanalysispm, aalst2016pm}. Specifically, on the one hand, noisy data may introduce infrequent and deceptive control-flow relations that may result in inconsistent fitness measures, whereas, on the other hand, checking event logs against a low-quality Petri net could lead to an unreliable distribution of fitness measures. Nonetheless, such Petri nets can still be used as references to obtain insightful information for \revision{process mining}-based feature extraction, supporting the development of competitive and explainable \revision{conformance checking}-based techniques for control-flow anomaly detection despite the problems above. For example, a few works outline that token-based \revision{conformance checking} can be used for \revision{process mining}-based feature extraction to build tabular data and develop effective \revision{conformance checking}-based techniques for control-flow anomaly detection \cite{singh2022lapmsh, debenedictis2023dtadiiot}. However, to the best of our knowledge, the scientific literature lacks a structured proposal for \revision{process mining}-based feature extraction using the state-of-the-art \revision{conformance checking} variant, namely alignment-based \revision{conformance checking}.

\subsection{Contributions}
We propose a novel \revision{process mining}-based feature extraction approach with alignment-based \revision{conformance checking}. This variant aligns the deviating control flow with a reference Petri net; the resulting alignment can be inspected to extract additional statistics such as the number of times a given activity caused mismatches \cite{aalst2022pmhandbook}. We integrate this approach into a flexible and explainable framework for developing techniques for control-flow anomaly detection. The framework combines \revision{process mining}-based feature extraction and dimensionality reduction to handle high-dimensional feature sets, achieve detection effectiveness, and support explainability. Notably, in addition to our proposed \revision{process mining}-based feature extraction approach, the framework allows employing other approaches, enabling a fair comparison of multiple \revision{conformance checking}-based and \revision{conformance checking}-independent techniques for control-flow anomaly detection. Figure \ref{HIGH_LEVEL_VIEW} shows a high-level view of the framework. Business processes are monitored, and event logs obtained from the database of information systems. Subsequently, \revision{process mining}-based feature extraction is applied to these event logs and tabular data input to dimensionality reduction to identify control-flow anomalies. We apply several \revision{conformance checking}-based and \revision{conformance checking}-independent framework techniques to publicly available datasets, simulated data of a case study from railways, and real-world data of a case study from healthcare. We show that the framework techniques implementing our approach outperform the baseline \revision{conformance checking}-based techniques while maintaining the explainable nature of \revision{conformance checking}.

In summary, the contributions of this paper are as follows.
\begin{itemize}
    \item{
        A novel \revision{process mining}-based feature extraction approach to support the development of competitive and explainable \revision{conformance checking}-based techniques for control-flow anomaly detection.
    }
    \item{
        A flexible and explainable framework for developing techniques for control-flow anomaly detection using \revision{process mining}-based feature extraction and dimensionality reduction.
    }
    \item{
        Application to synthetic and real-world datasets of several \revision{conformance checking}-based and \revision{conformance checking}-independent framework techniques, evaluating their detection effectiveness and explainability.
    }
\end{itemize}

The rest of the paper is organized as follows.
\begin{itemize}
    \item Section \ref{sec:related_work} reviews the existing techniques for control-flow anomaly detection, categorizing them into \revision{conformance checking}-based and \revision{conformance checking}-independent techniques.
    \item Section \ref{sec:abccfe} provides the preliminaries of \revision{process mining} to establish the notation used throughout the paper, and delves into the details of the proposed \revision{process mining}-based feature extraction approach with alignment-based \revision{conformance checking}.
    \item Section \ref{sec:framework} describes the framework for developing \revision{conformance checking}-based and \revision{conformance checking}-independent techniques for control-flow anomaly detection that combine \revision{process mining}-based feature extraction and dimensionality reduction.
    \item Section \ref{sec:evaluation} presents the experiments conducted with multiple framework and baseline techniques using data from publicly available datasets and case studies.
    \item Section \ref{sec:conclusions} draws the conclusions and presents future work.
\end{itemize}
\section{Krylov-aware low-rank approximation}

We now describe and motivate the algorithm that we will analyse. Inspired by \cite{chen_hallman_23}, we will call it \textit{Krylov-aware low-rank approximation}. 
%We now motivate and describe the Krylov-aware low-rank approximation algorithm studied in this paper. 
We begin with outlining the block-Lanczos algorithm to approximate matvecs with matrix functions. Next, we present how one would naively implement the randomized SVD on $f(\bm{A})$ using the block-Lanczos method to approximate matvecs. Finally, we present the alternative Krylov-aware algorithm and why this method allows us to gain efficiencies. 

\subsection{The block-Lanczos algorithm}\label{section:block_lanczos}

Given an $n\times \ell$ matrix $\bm{\Omega}$, the block-Lanczos algorithm (\cref{alg:block_lanczos}) can be used to iteratively obtain an orthonormal (graded) basis for $\mathcal{K}_{s}(\bm{A},\bm{\Omega})$. 
In particular, using $s$ block-matrix vector products with $\bm{A}$, the algorithm produces a basis $\bm{Q}_s$ and a block-tridiagonal matrix $\bm{T}_s$ 
\begin{equation}\label{eqn:QTdef}
\bm{Q}_s = 
    \begin{bmatrix} \bm{V}_0 & \cdots & \bm{V}_{s-1} \end{bmatrix}
,\quad
    \bm{T}_s = \bm{Q}_s^\T \bm{A}\bm{Q}_s = 
    \operatorname{tridiag}
    \left(\hspace{-.75em} \begin{array}{c}
        \begin{array}{cccc} \bm{R}_1^\T  & \cdots & \bm{R}_{s-1}^\T \end{array} \\
            \begin{array}{ccccc} \bm{M}_1 & \cdots & \cdots & \bm{M}_{s} \end{array} \\
        \begin{array}{cccc} \bm{R}_1 & \cdots & \bm{R}_{s-1}  \end{array} 
    \end{array} \hspace{-.75em}\right),
    % \begin{bmatrix}
    %     \bm{M}_1 & \bm{R}_1^\T \\
    %     \bm{R}_1 & \ddots & \ddots \\
    %     & \ddots & \ddots & \bm{R}_{q-1}^\T \\
    %     & & \bm{R}_{q-1} & \bm{M}_{q} 
    % \end{bmatrix}
\end{equation}
%\David{I removed the three-term recurrence. We never use it} %These are related by the block-three-term recurrence\begin{equation}\label{eqn:block_krylov}
    %\bm{A} \bm{Q}_s = \bm{Q}_s \bm{T}_s + \bm{V}_s \bm{R}_s \bm{E}_s^\T,
%\end{equation}
where $\bm{R}_0$ is also output by the algorithm and is given by the relation $\bm{\Omega} = \bm{V}_0 \bm{R}_0$. 
\begin{remark}
The block sizes of $\bm{Q}_s$ and $\bm{T}_s$ will never be larger than $\ell$, but may be smaller if the Krylov subspaces $\mathcal{K}_{i}(\bm{A},\bm{\Omega})$ obtained along the way have dimension less than $i\ell$. 
In particular, if $\mathcal{K}_{i}(\bm{A},\bm{\Omega}) = \mathcal{K}_{i-1}(\bm{A},\bm{\Omega})$, then the blocks $\bm{V}_j, \bm{R}_{j-1}, \bm{M}_{j}$ will be empty for all $j \geq i-1$.
The algorithm can terminate at this point, but for analysis it will be useful to imagine that the algorithm is continued and zero-width blocks are produced until it terminates at iteration $s$.
\end{remark}

\begin{algorithm}
\caption{Block-Lanczos Algorithm}
\label{alg:block_lanczos}
\textbf{input:} Symmetric $\bm{A} \in \mathbb{R}^{n \times n}$. Matrix $\bm{\Omega} \in \mathbb{R}^{n \times \ell}$. Number of iterations $s$.\\
\textbf{output:} Orthonormal basis $\bm{Q}_s$ for $\mathcal{K}_{s}(\bm{A}, \bm{\Omega})$,
%\begin{bmatrix} \bm{Q}_1 & \bm{Q}_2 \end{bmatrix}$ for $\mathcal{K}_{q}(\bm{A}, \bm{Z})$, where $\bm{Q}_1$ is orthonormal basis for $\mathcal{K}_{s}(\bm{A}, \bm{Z})$, 
and block tridiagonal $\bm{T}_s$.
\begin{algorithmic}[1]
    \State Compute an orthonormal basis $\bm{V}_0$ for $\range(\bm{\Omega})$ and $\bm{R}_0 = \bm{V}_0^T \bm{\Omega}$.
    \For{$i=1,\ldots, s$}
    \State $\bm{Y} = \bm{A} \bm{V}_{i-1} - \bm{V}_{i-2} \bm{R}_{i-1}^\T$ \Comment{$\bm{Y} = \bm{A} \bm{V}_{i-1}$ if $i=1$}
    \State $\bm{M}_i = \bm{V}_{i-1}^\T \bm{Y}$ 
    % \Chris{Is there a reason to have steps 4 and 5 seperate?} \David{We need to define $\bm{M}_i$ to define $\bm{T}_s$}
    \State $\bm{Y} = \bm{Y} - \bm{V}_{i-1}\bm{M}_i$
    \State $\bm{Y} = \bm{Y} - \sum_{j=0}^{i-1} \bm{V}_j\bm{V}_j^\T \bm{Y}$ \Comment{reorthogonalize (repeat as needed)}
    \State Compute an orthonormal basis $\bm{V}_i$ for $\range(\bm{Y})$ and $\bm{R}_i = \bm{V}_i^T \bm{Y}$. \label{alg:line:qr}
    \EndFor
    \State \Return $\bm{Q}_s$ and $\bm{T}_s$ as in \cref{eqn:QTdef}
\end{algorithmic}
\end{algorithm}
% \Chris{Should we leave reothogonalization out of Algorithm 1? Is it necessary? Or maybe say "optionaly reothogonalize" then add a comment that doing so is only necessary in finite precision?} \David{Yes, will do}

% block-three term recurrence:
% \begin{equation}
%     \bm{A} \overline{\bm{Q}}_i 
%     = \overline{\bm{Q}}_{i-1} \bm{R}_{i}^\T + \overline{\bm{Q}}_i \bm{M}_i + \overline{\bm{Q}}_{i+1} \bm{R}_{i+1}
% \end{equation}

The block Lanczos algorithm can be used to approximate matvecs and quadratic forms with $f(\bm{A})$ using the approximations
\begin{align}\label{eq:lanczos_approximation}
    \bm{Q}_s f(\bm{T}_s)_{:,1:\ell}\bm{R}_0 &\approx f(\bm{A}) \bm{\Omega},
    \\
    \label{eq:lanczos_approximation_qf}
    \bm{R}_0^\T f(\bm{T}_s)_{1:\ell,1:\ell}\bm{R}_0 &\approx \bm{\Omega}^\T f(\bm{A}) \bm{\Omega}. 
\end{align}
If $f$ is a low-degree polynomial, then the approximations \cref{eq:lanczos_approximation,eq:lanczos_approximation_qf} are exact.\footnote{\eqref{eq:lanczos_approximation} and \eqref{eq:lanczos_approximation_qf} are written out under the assumption that $\bm{\Omega}$ has rank $\ell$. If $\rank(\bm{\Omega}) = r <\ell$, then the index set $1:\ell$ should be replaced with $1:r$ in both \eqref{eq:lanczos_approximation} and \eqref{eq:lanczos_approximation_qf}.} 
% \footnote{In fact, a stronger result for multipolynomials holds, see \cite[Theorem 2.7]{frommer_lund_szyld_20}. \Chris{I looked at this paper and didn't see the term ``multipolynomial'' used. I haven't heard it before. Can we remove or just say something more vague?} \David{Perhaps change the word "multipolynomial" to "matrix polynomial", which they use in \cite{frommer_lund_szyld_20}}}
%
\begin{lemma}[{\cite[Lemma 2.1]{chen_hallman_23}}]\label{lemma:block_lanczos_exact}
The approximation \cref{eq:lanczos_approximation} is exact if $f\in\mathbb{P}_{s-1}$, and the approximation \cref{eq:lanczos_approximation_qf} is exact if $f\in \mathbb{P}_{2s-1}$.
\end{lemma}
% \begin{proof}
%     To prove the first result, it suffices to consider $f(x) = x^{j}$ for $j\leq s-1$.
%     Note that $\range(\bm{A}^j \bm{\Omega})\subset \range(\bm{Q}_s)$.
%     Thus, using \eqref{eqn:block_krylov} and $\bm{Q}_s^\T \bm{A} \bm{Q}_s = \bm{T}_s$ we have
%     \[
%     \bm{A}^{j}\bm{\Omega}
%     = \bm{Q}_s\bm{Q}_s^\T \bm{A}^j \bm{\Omega}
%     = \bm{Q}_s\bm{Q}_s^\T \bm{A} \bm{Q}_s \cdots \bm{Q}_s^\T \bm{A}\bm{Q}_s\bm{Q}_s^\T \bm{\Omega}
%     = \bm{Q}_s (\bm{T}_s^j)_{:,\ell} \bm{R}_0.
%     \]
%     The second result follows from the first and the fact that $\bm{Q}_s^\T \bm{A} \bm{Q}_s = \bm{T}_s$. 
% \end{proof}
It follows that \cref{eq:lanczos_approximation,eq:lanczos_approximation_qf} are good approximations if $f$ is well approximated by polynomials.\footnote{We note that for block-size $\ell>1$, the Krylov subspace is not equivalent to $\cup\{\range(p(\bm{A}) \bm{\Omega}) : p \in \mathbb{P}_{s-1}\}$, and bounds based on best approximation may be pessimistic due to this fact.
In fact, deriving stronger bounds is an active area of research; see e.g. \cite{chen_greenbaum_musco_musco,frommer_schweitzer_guttel,frommer_lund_szyld_20,frommer_schweitzer,frommer_simoncini, hochbruck_lubich}. 
% However, in this work we will stick with this simple and well-known bound.
}
In particular, one can obtain bounds in terms of the best polynomial approximation to $f$ on $[\lambda_{\min},\lambda_{\max}]$ \cite{Saad:1992,OrecchiaSachdevaVishnoi:2012,Amsel:2024}.


%\tyler{There used to be a lemma here, but it didn't seem like we used it anywhere but instead use \cref{lemma:2_times_polynomial_approx}.}
% This can be formalized as the following (well-known) lemma. 
% \begin{lemma}\label{thm:lanczos_exact}
% Let $\bm{Q}_s$ and $\bm{T}_s$ be as in \Cref{alg:block_lanczos} and let $\lambda_{\min}$ and $\lambda_{\max}$ be the smallest and largest eigenvalue of $\bm{A}$ respectively. Then, 
% \begin{equation}\label{eqn:pAv_exact}
%     \|f(\bm{A}) \bm{\Omega} - \bm{Q}_s f(\bm{T}_s)_{:,1:\ell}\bm{R}_0\|_2 \leq 2 \|\bm{\Omega}\|_2 \min\limits_{p \in \mathbb{P}_{s-1}}\|f-p\|_{L^{\infty}([\lambda_{\min},\lambda_{\max}])}
% \end{equation}
% and
% \begin{equation}\label{eqn:vpAv_exact}
%     \|\bm{\Omega}^\T f(\bm{A}) \bm{\Omega} - \bm{R}_0^\T f(\bm{T}_s)_{1:\ell,1:\ell}\bm{R}_0\|_2 \leq 2 \|\bm{\Omega}\|_2^2 \min\limits_{p \in \mathbb{P}_{2s-1}}\|f-p\|_{L^{\infty}([\lambda_{\min},\lambda_{\max}])}.
% \end{equation}
% Moreover, if $\bm{V}_s$ is empty (i.e. because the Krylov subspace has become invariant under $\bm{A}$), then in fact \cref{eqn:pAv_exact,eqn:vpAv_exact} hold for polynomials of arbitrary degree.
% \end{lemma}
% In particular, one can obtain bounds in terms of the best polynomial approximation to $f$ on $[\lambda_{\min},\lambda_{\max}]$.
% For block-size $\ell>1$, the Krylov subspace is not equivalent to $\cup\{\range(p(\bm{A}) \bm{\Omega}) : p \in \mathbb{P}_{s-1}\}$, and bounds based on best approxiamtion may be pessimistic due to this fact.
% In fact, deriving stronger bounds is an active area of research; see e.g. \cite{chen_greenbaum_musco_musco,frommer_schweitzer_guttel,frommer_lund_szyld_20,frommer_schweitzer,frommer_simoncini, hochbruck_lubich}. However, in this work we will stick with this simple and well known bound. 



% The key observation which allows the algorithms in this paper to be implemented efficiently is that the Krylov subspace of a Krylov subsapce is a Krylov susbspace. 
% \begin{lemma}\label{lemma:krylov_nested}
%     \Chris{I think it's unclear what it means to write "in the setting of Algortihm 1"? Maybe we should just say "let $Q_1$ and $Z$ be as definied in algorithm 1 or similar". Should we be using $\Omega$ here instead of $\bm{Z}$?} 
%     \tyler{I started a reworded version above.}
%     Assume that $\mathcal{K}_{s+r}(\bm{A},\bm{Z})$ has dimension $\ell (s+r)$. Then, in the setting of \Cref{alg:block_lanczos} we have
%     \begin{equation*}
%         \mathcal{K}_{s+r}(\bm{A}, \bm{Z}) = \mathcal{K}_{r+1}(\bm{A}, \bm{Q}_1),
%     \end{equation*}
%     and running \Cref{alg:block_lanczos} with starting block $\bm{Q}_1$ for $r+1$ iterations and  running \Cref{alg:block_lanczos} with starting block $\bm{Z}$ for $s+r$ iterations yields the same result.
% \end{lemma}

\subsection{The randomized SVD for matrix functions}
The randomized SVD \cite{rsvd} is a simple and efficient algorithm to compute low-rank approximations of matrices that admit accurate low-rank approximations. The basic idea behind the randomized SVD is that if $\bm{\Omega}$ is a standard Gaussian random matrix, i.e. the entries in $\bm{\Omega}$ are independent identically distributed $N(0,1)$ random variables, then $\range(\bm{B} \bm{\Omega})$ should be reasonable approximation to the range of $\bm{B}$'s top singular vectors. Hence, projecting $\bm{B}$ onto $\range(\bm{B}\bm{\Omega})$ should yield a good approximation to $\bm{B}$. \Cref{alg:rsvd} implements the randomized SVD on a symmetric matrix $\bm{B}$. %\David{I added a comment about "to truncate or not to truncate?"}
The algorithm returns either the rank $\ell \geq k$ approximation $\bm{W}\bm{X} \bm{W}^\T$ or the rank $k$ approximation $\bm{W}\llbracket\bm{X}\rrbracket_k \bm{W}^\T$, depending on the needs of the user.

\begin{algorithm}
\caption{Randomized SVD}
\label{alg:rsvd}
\textbf{input:} Symmetric $\bm{B} \in \mathbb{R}^{n \times n}$. Rank $k$. Oversampling parameter $\ell -k$. \\
\textbf{output:} Low-rank approximation to $\bm{B}$: $\bm{W} \bm{X} \bm{W}^{\T}$ or $\bm{W} \llbracket \bm{X} \rrbracket_k \bm{W}^{\T}$
\begin{algorithmic}[1]
    \State Sample a standard Gaussian $n \times \ell $ matrix $\bm{\Omega}$.
    \State Compute $\bm{K} = \bm{B} \bm{\Omega}$.\label{line:K_rsvd}
    \State Compute an orthonormal basis $\bm{W}$ for $\range(\bm{K})$.\label{line:V}
    \State Compute $\bm{X} = \bm{W}^{\T} \bm{B} \bm{W}$. \label{line:X_rsvd}
    \State \Return $\bm{W} \bm{X} \bm{W}^{\T}$ or $\bm{W} \llbracket \bm{X} \rrbracket_k \bm{W}^{\T}$
\end{algorithmic}
\end{algorithm}

Typically, the dominant cost of \Cref{alg:rsvd} is that computation of matvecs $\bm{B}$. We require $2\ell$ such matvecs: $\ell$ in \cref{line:K_rsvd} with $\bm{\Omega}$ and $\ell$ in \cref{line:X_rsvd} with $\bm{W}$. When \Cref{alg:rsvd} is applied to a matrix function $\bm{B} = f(\bm{A})$ these matvecs cannot be performed exactly, but need to be approximated using, for example, the block Lanczos method discussed in the previous section.
\Cref{alg:rsvd_matfun} implements the randomized SVD applied to $f(\bm{A})$ with approximate matvecs using the block Lanczos method. 
The cost is now $(s+r)\ell$ matvecs with $\bm{A}$, where $s$ and $r$ should be set sufficiently large so that the approximations \cref{eq:lanczos_approximation,eq:lanczos_approximation_qf} are accurate.

\begin{algorithm}
\caption{Randomized SVD on a matrix function $f(\bm{A})$}
\label{alg:rsvd_matfun}
\textbf{input:} Symmetric matrix $\bm{A} \in \mathbb{R}^{n \times n}$. Rank $k$. Oversampling parameter $\ell -k$. Matrix function $f: \mathbb{R} \to \mathbb{R}$. Accuracy parameters $s$ and $r$. \\
\textbf{output:} Low-rank approximation to $f(\bm{A})$: $\bm{W}\bm{X} \bm{W}^\T$ or $\bm{W}\llbracket \bm{X}\rrbracket_k \bm{W}^\T$
\begin{algorithmic}[1]
    \State Sample a standard Gaussian $n \times \ell $ matrix $\bm{\Omega}$.
    \State Run \Cref{alg:block_lanczos} for $s$ iterations to obtain an orthonormal basis $\bm{Q}_s$ for $\mathcal{K}_s(\bm{A},\bm{\Omega})$, a block tridiagonal matrix $\bm{T}_s$ and an upper triangular matrix $\bm{R}_0$.
    \State Compute the approximation $\bm{K} = \bm{Q}_s f(\bm{T}_s)_{:,1:\ell} \bm{R}_0 \approx f(\bm{A}) \bm{\Omega}$.
    \State Compute an orthonormal basis $\bm{W}$ for $\range(\bm{K})$. 
    \State Run \Cref{alg:block_lanczos} for $r$ iterations with starting block $\bm{W}$ to obtain the block tridiagonal matrix $\widetilde{\bm{T}}_r$. 
    \State Compute the approximation $\bm{X} = f(\widetilde{\bm{T}}_r)_{1:\ell,1:\ell} \approx \bm{W}^\T f(\bm{A}) \bm{W}$.\label{line:X}
    \State \Return $\bm{W} \bm{X} \bm{W}^{\T}$ or $\bm{W} \llbracket \bm{X} \rrbracket_k \bm{W}^{\T}$
\end{algorithmic}
\end{algorithm}

\subsection{Krylov-aware low-rank approximation}
A key observation in \cite{chen_hallman_23} was that $\range(\bm{W}) \subseteq \range(\bm{Q}_s)$, where $\bm{W}$ and $\bm{Q}_s$ are as in \Cref{alg:rsvd_matfun}. Therefore, by \cite[Lemma 3.3]{funnystrom2} one has
\begin{align*}
    \|f(\bm{A}) - \bm{Q}_s \bm{Q}_s^{\T} f(\bm{A})\bm{Q}_s \bm{Q}_s^{\T}\|_\F &\leq \|f(\bm{A}) - \bm{W} \bm{W}^{\T} f(\bm{A})\bm{W} \bm{W}^{\T}\|_\F,\\
    \|f(\bm{A}) - \bm{Q}_s \llbracket \bm{Q}_s^{\T} f(\bm{A})\bm{Q}_s \rrbracket_k \bm{Q}_s^{\T}\|_\F &\leq \|f(\bm{A}) - \bm{W} \llbracket \bm{W}^{\T} f(\bm{A})\bm{W} \rrbracket_k \bm{W}^{\T}\|_\F.
\end{align*}
Hence, assuming that the quadratic form $\bm{Q}_s^\T f(\bm{A}) \bm{Q}_s$ can be computed accurately, the naive implementation of the randomized SVD outlined in \Cref{alg:rsvd_matfun} will yield a worse error than using $\bm{Q}_s\bm{Q}_s^\T f(\bm{A}) \bm{Q}_s \bm{Q}_s^\T$ as an approximation to $f(\bm{A})$.%, and $\bm{Q}_s$ is a basis that is already computed. 

Since $\bm{Q}_s$ could have as many as $s\ell$ columns, an apparent downside to this approach is that approximating $f(\bm{A})\bm{Q}_s$ might require $rs\ell$ matvecs with $\bm{A}$ if we use $r$ iterations of the block Lanczos method (\Cref{alg:block_lanczos}) to approximately compute each matvec with $f(\bm{A}$.
The key observation which allows Krylov-aware algorithms to be implemented efficiently is the following: 
\begin{lemma}[{\cite[Section 3]{chen_hallman_23}}]\label{lemma:krylovkrylov}
    Suppose that $\bm{Q}_s$ is the output of \Cref{alg:block_lanczos} with starting block $\bm{\Omega}$ and $s$ iterations. Then, running $r+1$ iterations of \Cref{alg:block_lanczos} with starting block $\bm{Q}_s$ yields the same output as running $s+r$ iterations of \Cref{alg:block_lanczos} with starting block $\bm{\Omega}$. 
    %Suppose $\bm{Q}_s$ is a basis for $\mathcal{K}_s(\bm{A},\bm{\Omega})$.
    \end{lemma}
The insight behind \Cref{lemma:krylovkrylov} is that, since $\bm{Q}_s$ is a basis for the Krylov subspace $\mathcal{K}_s(\bm{A},\bm{\Omega})$, we have $\mathcal{K}_{r+1}(\bm{A},\bm{Q}_s) = \mathcal{K}_{s+r}(\bm{A},\bm{\Omega})$. This fact can be verified by explicitly writing out $\mathcal{K}_{r+1}(\bm{A},\bm{Q}_s)$:
\begin{align*}
    \mathcal{K}_{r+1}(\bm{A},\bm{Q}_s) &= \range\big( \big[ \bm{Q}_s \,\bm{A}\bm{Q}_s \, \cdots \, \bm{A}^r\bm{Q}_s \big]\big)\\
    &=  \range\big( \big[\bm{\Omega} \,\hspace{5pt}\bm{A}\bm{\Omega} \, \hspace{5pt}\cdots\hspace{5pt} \, \bm{A}^r\bm{\Omega}\\% First line
    &\hspace{2.2cm} \bm{A} \bm{\Omega} \,\hspace{5pt} \bm{A}^2\bm{\Omega} \, \cdots \, \bm{A}^{r+1} \bm{\Omega}\\
    & \hspace{3.5cm} \ddots\\
    & \hspace{3.5cm} \bm{A}^r \bm{\Omega} \,\hspace{5pt}\bm{A}^{r+1} \bm{\Omega} \,\hspace{5pt} \cdots \, \bm{A}^{s+r-1} \bm{\Omega} \big] \big) = \mathcal{K}_{s+r}(\bm{A},\bm{\Omega}). 
\end{align*}
In addition to the work of Chen and Hallman, this observation was recently used to analyze ``small-block" randomized Krylov methods for low-rank approximation \cite{MeyerMuscoMusco:2024} and a randomized variant of the block conjugate gradient algorithm \cite{BCGPreconditioning}.
    
%     Then,
%     \[
%     \mathcal{K}_{s+r}(\bm{A},\bm{\Omega}) 
%     = \mathcal{K}_{r+1}(\bm{A},\bm{Q}_s).
%     \]
% % Moreover, \cref{alg:line:qr} in \cref{alg:block_lanczos} can be implemented in such a way the output of \cref{alg:block_lanczos} on $\bm{A},\bm{\Omega}$ run for $q=s+r$ iteration is identical to the output of \cref{alg:block_lanczos} run on $\bm{A},\bm{Q}_s$ for $r+1$ iterations.
% \end{lemma}
% \begin{proof}
% Without loss of generality, we can take $\bm{Q}_s = [\bm{\Omega}~\bm{A}\bm{\Omega}~\cdots~\bm{A}^{s-1}\bm{\Omega}]$.
% Then,
% \begin{align*}
%     \mathcal{K}_{r+1}(\bm{A},\bm{Q}_s)
%     &= \range\left(\begin{bmatrix} \bm{Q}_s& \bm{A}\bm{Q}_s & \cdots & \bm{A}^{r} \bm{Q}_s \end{bmatrix}\right)
%     \\&= \range\left(\left[\begin{array}{cccc}
%     \bm{\Omega}& \bm{A}\bm{\Omega} & \cdots & \bm{A}^{s-1} \bm{\Omega} 
%     \end{array}\right.\right. 
%     \\&\hspace{6em}\begin{array}{cccc}
%     \bm{A}\bm{\Omega}& \bm{A}^2\bm{\Omega} & \cdots & \bm{A}^{s} \bm{\Omega} \end{array}
%     \\&\hspace{7em}
%     \left.\left.\begin{array}{cccc}
%     \bm{A}^{r}\bm{\Omega}& \bm{A}^{r+1}\bm{\Omega} & \cdots & \bm{A}^{s+r-1} \bm{\Omega} \end{array}\right]\right)
%     \\&=
%     \range\left(\begin{bmatrix} \bm{\Omega}& \bm{A}\bm{\Omega} & \cdots & \bm{A}^{s+r-1} \bm{\Omega}\end{bmatrix}\right)
%     = \mathcal{K}_{s+r}(\bm{A},\bm{\Omega}).
%     \qedhere
% \end{align*}
% \end{proof}
% \noindent
Notably, \Cref{lemma:krylovkrylov} enables us to approximate $\bm{Q}_s^\T f(\bm{A})\bm{Q}_s$ with just $r\ell$  additional matvecs with $\bm{A}$, even though $\bm{Q}_s$ has $s\ell$ columns! Hence, approximately computing $\bm{Q}_s^\T f(\bm{A}) \bm{Q}_s$ is essentially no more expensive, in terms of the number of matvecs with $\bm{A}$, than approximating $\bm{W}^\T f(\bm{A}) \bm{W}$, as done in \Cref{alg:rsvd_matfun}~\cref{line:X}. As a consequence of \Cref{lemma:krylovkrylov}, we have the following result. 
\begin{lemma}\label{lemma:2_times_polynomial_approx}
Let $\lambda_{\max}$ and $\lambda_{\min}$ denote the largest and smallest eigenvalue of $\bm{A}$. Let $q = s+r$ and let $\bm{T}_q$ and $\bm{Q}_s$ be computed using \Cref{alg:block_lanczos}. Then, 
\begin{align*}
    &\|\bm{Q}_s^\T f(\bm{A})\bm{Q}_s - f(\bm{T}_q)_{1:d_{s,\ell},1:d_{s,\ell}}\|_\F \leq 2\sqrt{\ell s} \inf\limits_{p \in \mathbb{P}_{2r+1}}\|f(x)-p(x)\|_{L^{\infty}([\lambda_{\min},\lambda_{\max}])}.
\end{align*}
\end{lemma}
\begin{proof}
%By \cite[Lemma 3.1]{chen_hallman_23} we know that for any polynomial $p \in \mathbb{P}_{2r+1}$ we have $\bm{Q}_s^\T p(\bm{A}) \bm{Q}_{s} = p(\bm{T}_q)_{1:d_s,1:d_s}$. 
By \Cref{lemma:block_lanczos_exact,lemma:krylovkrylov} we know that for any polynomial $p \in \mathbb{P}_{2r+1}$ we have $\bm{Q}_s^\T p(\bm{A}) \bm{Q}_{s} = p(\bm{T}_q)_{1:d_{s,\ell},1:d_{s,\ell}}$.
Therefore, since $\|\bm{Q}_s\|_\F \leq \sqrt{\ell s}$ and $\|\bm{Q}_s\|_2 \leq 1$ we have
\begin{align*}
    \hspace{5em}&\hspace{-5em}\|\bm{Q}_s^\T f(\bm{A})\bm{Q}_s - f(\bm{T}_q)_{1:d_{s,\ell},1:d_{s,\ell}}\|_\F
    \\&= \|\bm{Q}_s^\T f(\bm{A}) \bm{Q}_s - p(\bm{T}_q)_{1:d_{s,\ell},1:d_{s,\ell}} + p(\bm{T}_q)_{1:d_{s,\ell},1:d_{s,\ell}} - f(\bm{T}_q)_{1:d_{s,\ell},1:d_{s,\ell}} \|_\F 
    \\&\leq \|\bm{Q}_s^\T f(\bm{A}) \bm{Q}_s - \bm{Q}_s^\T p(\bm{A}) \bm{Q}_s \|_\F +  \|(p(\bm{T}_q) - f(\bm{T}_q) )_{1:d_{s,\ell},1:d_{s,\ell}} \|_\F 
    \\&\leq \sqrt{\ell s} \left( \|f(\bm{A}) - p(\bm{A}) \|_2 +  \| p(\bm{T}_q)  - f(\bm{T}_q) \|_2\right)
    \\&\leq 2 \sqrt{\ell s} \|f(x) - p(x)\|_{L^{\infty}([\lambda_{\min},\lambda_{\max}])},
\end{align*}
where the last inequality is due to the fact that the spectrum of $\bm{T}_q$ is contained in $[\lambda_{\min},\lambda_{\max}]$.
Optimizing over $p\in\mathbb{P}_{2r+1}$ gives the result.
%Applying \cref{thm:lanczos_exact} with $\bm{Q}_s$ as the starting block and running block-Lanczos for $r+1$ iterations we have that, for any $p\in\mathbb{P}_{2r+1}$, $\bm{Q}_s^\T p(\bm{A}) \bm{Q}_s = p(\bm{T})_{1:d_s,1:d_s}$.
%Therefore,
%\begin{align*}
    %\varepsilon_r
    %&= \|\bm{Q}_1^\T f(\bm{A}) \bm{Q}_1 - \bm{Q}_1^\T p(\bm{T}) \bm{Q}_1 + p(\bm{T})_{1:d_s,1:d_s} - f(\bm{T})_{1:d_s,1:d_s} \|_\F 
    %\\&\leq \|\bm{Q}_1^\T f(\bm{A}) \bm{Q}_1 - \bm{Q}_1^\T p(\bm{A}) \bm{Q}_1 \|_\F +  \|(p(\bm{T}) - f(\bm{T}) )_{1:d_s,1:d_s} \|_\F 
    %\\&\leq \sqrt{d_s} \left( \|f(\bm{A}) - p(\bm{A}) \|_2 +  \| p(\bm{T}) \bm{E}_1 - f(\bm{T}) \|_2\right)
    %\\&\leq 2 \sqrt{d_s} \|f(x) - p(x)\|_{L^{\infty}([\lambda_{\min},\lambda_{\max}])},
%\end{align*}
%where the last inequality is due to the fact that the spectrum of $\bm{T}$ is contained in $[\lambda_{\min},\lambda_{\max}]$.
%Optimizing over $p\in\mathbb{P}_{2r-1}$ gives the result.
\end{proof}

We can now present the Krylov-aware low-rank approximation algorithm; see \cref{alg:krylow}. The total number of matvecs with $\bm{A}$ is $(s+r)\ell$, the same as \cref{alg:rsvd_matfun}. 
However, as noted above, \cref{alg:krylow} (approximately) projects $f(\bm{A})$ onto a higher dimensional subbspace, ideally obtaining a better approximation.
%The complete algorithm is shown in \cref{alg:krylov_aware_polynomial}, and the total number of matrix-vector products used is $(s+r-1)\ell$, the same as \cref{alg:low-rank_polynomial}. 

%However, the above mentioned fact assumes that one can compute $\bm{Q}_q^T f(\bm{A}) \bm{Q}_q$ accurately. Since $\bm{Q}_q$ has more columns than $\bm{W}$, the potential benefit from using $\bm{Q}_q$ instead of $\bm{W}$ might be lost by the fact that approximating $\bm{Q}_q^\T f(\bm{A})\bm{Q}_q$ is more expensive than $\bm{W}^\T f(\bm{A})\bm{W}$. However, due to the special structure of $\bm{Q}_q$ one can show that approximating $\bm{Q}_q^\T f(\bm{A})\bm{Q}_q$ essentially comes at the same cost as approximating $\bm{W}^\T f(\bm{A})\bm{W}$, even if $\bm{Q}_q$ has more columns than $\bm{W}$; see \cite[Lemma 3.1]{chen_hallman_23}.

\begin{algorithm}
\caption{Krylov-aware low-rank approximation}
\label{alg:krylow}
\textbf{input:} Symmetric $\bm{A} \in \mathbb{R}^{n \times n}$. Rank $k$. Oversampling parameter $\ell -k$. Matrix function $f: \mathbb{R} \to \mathbb{R}$. Number of iterations $q = s + r$.\\
\textbf{output:} Low-rank approximation to $f(\bm{A})$: $\bm{Q}_s \bm{X} \bm{Q}_s^\T$ or $\bm{Q}_s \llbracket \bm{X} \rrbracket_k  \bm{Q}_s^\T$%$\bm{Q}_s \llbracket f(\bm{T}_q)_{1:d_s,1:d_s} \rrbracket_k  \bm{Q}_s^\T$. 
\begin{algorithmic}[1]
    \State Sample a standard Gaussian $n \times \ell $ matrix $\bm{\Omega}$.
    \State Run \cref{alg:block_lanczos} for $q=s+r$ iterations to obtain an orthonormal basis $\bm{Q}_s$ for $\mathcal{K}_{s}(\bm{A},\bm{\Omega})$ and a block tridiagonal matrix $\bm{T}_q$.
    \State Compute $\bm{X} = f(\bm{T}_q)_{1:d_{s,\ell},1:d_{s,\ell}}$. \Comment{$\approx \bm{Q}_s^\T f(\bm{A}) \bm{Q}_s$} 
    %\State Compute the eigenvalue decomposition of $\bm{X} = \bm{V} \bm{D} \bm{V}^\T$, where the eigenvalues in $\bm{D}$ are ordered in descending absolute magnitude. 
    \State \textbf{return} $\textsf{\textup{ALG}}(s,r;f) = \bm{Q}_s \bm{X}\bm{Q}_s^\T$ or $\textsf{\textup{ALG}}_k(s,r;f) = \bm{Q}_s \llbracket \bm{X} \rrbracket_k  \bm{Q}_s^\T$.
\end{algorithmic}
\end{algorithm}
We conclude by noting that the function $f$ in \cref{alg:krylow} does not need to be fixed; one can compute a low-rank approximation for many different functions $f$ at minimal additional cost. 
\section{Error bounds}
In this section, we establish error bounds for \Cref{alg:krylow}. We break the analysis into two parts. First, in \Cref{section:inexact_projections}, we derive general error bounds for approximations to $f(\bm{A})$ when projections $\bm{Q} \bm{Q}^\T f(\bm{A}) \bm{Q} \bm{Q}^\T$ cannot be computed exactly. In \Cref{section:structural} we provide structural bounds for the errors $\|f(\bm{A})- \bm{Q}_s \bm{Q}_s^\T f(\bm{A}) \bm{Q}_s \bm{Q}_s^\T\|_{\F}$ and $\|f(\bm{A})- \bm{Q}_s \llbracket \bm{Q}_s^\T f(\bm{A}) \bm{Q}_s\rrbracket_k \bm{Q}_s^\T\|_{\F}$ that hold with probability $1$, and in \Cref{section:probabilistic} we derive the corresponding probabilistic bounds. Next, in \Cref{section:krylow} we combine the results from \Cref{section:inexact_projections,section:structural,section:probabilistic} to derive end-to-end error bounds for \Cref{alg:krylow}, which involves approximate projection onto $\bm{Q}_s \bm{Q}_s^\T$. %Finally, in \Cref{section:exponential} we apply our bounds to the matrix exponential as an illustrative example.
\subsection{Error bounds for inexact projections}\label{section:inexact_projections}
In this section we will derive error bounds for $\|f(\bm{A}) - \bm{Q} \bm{X} \bm{Q}^\T\|_{\F}$ and $\|f(\bm{A}) - \bm{Q} \llbracket \bm{X} \rrbracket_k \bm{Q}^\T\|_{\F}$ where $\bm{Q}$ is \textit{any} orthonormal basis and $\bm{X}$ is \textit{any} matrix. By \cite[Lemma 3.3]{funnystrom2} we know that the optimal choice of $\bm{X}$ is $\bm{X} = \bm{Q}^\T f(\bm{A}) \bm{Q}$. However, since $\bm{Q}^\T f(\bm{A}) \bm{Q}$ can only be computed approximately, we need to show that the errors $\|f(\bm{A}) - \bm{Q} \bm{Q}^\T f(\bm{A}) \bm{Q} \bm{Q}\|_{\F}$ and $\|f(\bm{A}) - \bm{Q} \llbracket \bm{Q}^\T f(\bm{A}) \bm{Q} \rrbracket_k \bm{Q}\|_{\F}$ are robust against perturbations in $\bm{Q}^\T f(\bm{A}) \bm{Q}$. \Cref{theorem:robust} provides such a result. 
%Note that we consider matrix functions $f(\bm{A})$ when we state the theorem, since this is the focus of this paper. However, \Cref{theorem:robust} remains true for arbitrary square matrices. 
\begin{theorem}\label{theorem:robust}
    Given an orthonormal basis $\bm{Q}$ and an approximation $\bm{X}$ to $\bm{Q}^\T f(\bm{A}) \bm{Q}$. Then,
    \begin{equation}\label{eq:robustness1}
        \|f(\bm{A}) - \bm{Q} \bm{X}\bm{Q}^\T\|_{\F}^2 = \|f(\bm{A}) - \bm{Q}\bm{Q}^\T f(\bm{A})\bm{Q}\bm{Q}^\T\|_{\F}^2 + \|\bm{Q}^\T f(\bm{A})\bm{Q} - \bm{X}\|_{\F}^2,
    \end{equation}
    and
    \begin{equation}\label{eq:robustness2}
        \|f(\bm{A}) - \bm{Q} \llbracket \bm{X} \rrbracket_k \bm{Q}^\T\|_{\F} \leq \|f(\bm{A}) - \bm{Q}\llbracket\bm{Q}^\T f(\bm{A})\bm{Q} \rrbracket_k\bm{Q}^\T\|_{\F} + 2\|\bm{Q}^\T f(\bm{A})\bm{Q} - \bm{X}\|_{\F}.
    \end{equation}
\end{theorem}
% \begin{remark}
%     Note that one can derive a similar bound to \textit{(\ref*{item:truncated})} by using the triangle inequality and obtaining a bound for $\|(f(\bm{T})_{1:d_s,1:d_s})_{(k)}-(\bm{Q}_1^\T f(\bm{A}) \bm{Q}_1)_{(k)}\|_{\F}$. Treating $f(\bm{T})_{1:d_s,1:d_s}$ as a perturbation of $\bm{Q}_1^\T f(\bm{A}) \bm{Q}_1$ would lead to an unfortunate dependence of the singular value gap of $f(\bm{A})$. Our analysis avoids any dependence on gaps. 
% \end{remark}
\begin{proof}
%     We begin with proving \eqref{eq:robustness1}. Note that for any matrix $\bm{B}$ we have $\langle f(\bm{A})-\bm{Q} \bm{Q}^\T f(\bm{A})\bm{Q} \bm{Q}^\T, \bm{Q} \bm{B}\bm{Q} \rangle = 0$. Hence, by the Pythagorean theorem, we have
% \begin{align*}
%     \|f(\bm{A}) - \bm{Q}\bm{X} \bm{Q}^\T\|_{\F}^2 &= \|f(\bm{A}) -\bm{Q} \bm{Q}^\T f(\bm{A})\bm{Q} \bm{Q}^\T + \bm{Q}(\bm{Q}^\T f(\bm{A})\bm{Q} - \bm{X})\bm{Q}^\T\|_{\F}^2 \\&= 
%     \|f(\bm{A}) -\bm{Q} \bm{Q}^\T f(\bm{A})\bm{Q} \bm{Q}^\T\|_{\F}^2 + \|\bm{Q}^\T f(\bm{A})\bm{Q} - \bm{X}\|_{\F}^2,
% \end{align*}
% as required.
\eqref{eq:robustness1} is immediate due to the Pythagorean theorem.

We now proceed with proving \eqref{eq:robustness2} using a similar argument to \cite[Proof of Theorem 5.1]{tropp2016randomized}. Define $\bm{C} = f(\bm{A}) - \bm{Q} \bm{Q}^\T f(\bm{A}) \bm{Q} \bm{Q}^\T + \bm{Q} \bm{X}\bm{Q}^\T$. Note that $\|\bm{C} - f(\bm{A})\|_{\F} = \|\bm{Q}^\T f(\bm{A}) \bm{Q} - \bm{X}\|_{\F}$ and $\bm{Q}^\T \bm{C} \bm{Q} =  \bm{X}$. Hence,
    \begin{align*}
        \|f(\bm{A}) - \bm{Q} \llbracket\bm{X}\rrbracket_k\bm{Q}^\T\|_{\F} &= \|f(\bm{A}) - \bm{Q}\llbracket \bm{Q}^\T \bm{C} \bm{Q}\rrbracket_k\bm{Q}^\T\|_{\F} 
        \\&\leq 
        \|f(\bm{A}) - \bm{C}\|_{\F} + \|\bm{C} - \bm{Q}\llbracket \bm{Q}^\T \bm{C} \bm{Q}\rrbracket_k\bm{Q}^\T\|_{\F} 
        \\&=
        \|\bm{Q}^\T f(\bm{A})\bm{Q}-\bm{X}\|_{\F} + \|\bm{C} - \bm{Q}\llbracket \bm{Q}^\T \bm{C} \bm{Q}\rrbracket_k\bm{Q}^\T\|_{\F}. \numberthis \label{eq:first_ineq}
    \end{align*}
    %Choose an orthogonal projector $\bm{P}$ so that $\range(\bm{P}) \subseteq \range(\bm{Q})$ and $\bm{Q} \llbracket \bm{Q}^\T f(\bm{A})\bm{Q}\rrbracket_k \bm{Q}^\T = \bm{P} f(\bm{A}) \bm{P}$.
    Since $\bm{Q}\llbracket \bm{Q}^\T \bm{C} \bm{Q}\rrbracket_k\bm{Q}^\T$ is the best rank $k$ approximation whose range is contained in $\range(\bm{Q})$ \cite[Lemma 3.3]{funnystrom2} (a similar result is given in \cite[Theorem 3.5]{gu_subspace}), we have
    \begin{align*}
        %\hspace{5em}&\hspace{-5em}
        \|\bm{C} - \bm{Q}\llbracket \bm{Q}^\T \bm{C} \bm{Q}\rrbracket_k\bm{Q}^\T\|_{\F} 
         \leq&\|\bm{C} - \bm{Q}\llbracket \bm{Q}^\T f(\bm{A}) \bm{Q}\rrbracket_k\bm{Q}^\T\|_{\F} \\
         \leq& \|\bm{Q}^\T f(\bm{A})\bm{Q}-\bm{X}\|_{\F} + \|f(\bm{A}) - \bm{Q}\llbracket \bm{Q}^\T f(\bm{A}) \bm{Q}\rrbracket_k\bm{Q}^\T\|_{\F}\numberthis \label{eq:final_ineq},
    \end{align*}
    % \begin{align*}
    %     \hspace{5em}&\hspace{-5em}\|\bm{C} - \bm{Q}\llbracket \bm{Q}^\T \bm{C} \bm{Q}\rrbracket_k\bm{Q}^\T\|_{\F} 
    %     \leq \|\bm{C} - \bm{P} \bm{C} \bm{P}\|_{\F} 
    %     \\&=
    %     \|(f(\bm{A}) - \bm{C}) - \bm{P}(f(\bm{A}) - \bm{C}) \bm{P}\|_{\F} + \|f(\bm{A}) - \bm{P} f(\bm{A}) \bm{P}\|_{\F} \\&\leq
    %     \|f(\bm{A}) - \bm{C}\|_{\F} + \|f(\bm{A}) - \bm{Q}\llbracket \bm{Q}^\T f(\bm{A}) \bm{Q}\rrbracket_k\bm{Q}^\T \|_{\F} 
    %     \\&=
    %     \|\bm{Q}^\T f(\bm{\bm{A}})\bm{Q}-\bm{X}\|_{\F} + \|f(\bm{A}) - \bm{Q}\llbracket \bm{Q}^\T f(\bm{A}) \bm{Q}\rrbracket_k\bm{Q}^\T \|_{\F}, \numberthis \label{eq:final_ineq}
    % \end{align*}
    % where the second inequality is due to the fact that for any matrix $\bm{D}$ one has $\|\bm{D} - \bm{P} \bm{D} \bm{P}\|_{\F} \leq \|\bm{D}\|_{\F}$. 
    where the second inequality is due to the fact that $\|f(\bm{A})-\bm{C}\|_\F =\|\bm{Q}^\T f(\bm{A})\bm{Q}-\bm{X}\|_{\F}$. Combining \eqref{eq:first_ineq} and \eqref{eq:final_ineq} yields the desired inequality.
\end{proof}

A corollary of \Cref{theorem:robust} is that the error of the approximation from \Cref{alg:krylow} will always be bounded from above by the error of the approximation from \Cref{alg:rsvd_matfun}, up to a polynomial approximation error of $f$:
\begin{corollary}
\label{corr:early_bigger_subspace}
    Let $\lambda_{\max}$ and $\lambda_{\min}$ be the largest and smallest eigenvalues of $\bm{A}$. Let $\bm{Q}_s$ and $\bm{X}$ be the output from \Cref{alg:krylow} and let $\bm{W}$ and $\widetilde{\bm{X}}$ be the output from \Cref{alg:rsvd_matfun} with the same input parameters and the same sketch matrix $\bm{\Omega}$. Then,
    \begin{align*}
        &\|f(\bm{A}) - \bm{Q}_s \llbracket\bm{X}\rrbracket_k \bm{Q}_s^\T\|_\F 
        \\&\hspace{2em}
        \leq \|f(\bm{A}) - \bm{W} \llbracket\widetilde{\bm{X}}\rrbracket_k \bm{W}^\T\|_\F + 4\sqrt{\ell s}  \inf\limits_{p \in \mathbb{P}_{2r+1}}\|f(x)-p(x)\|_{L^{\infty}([\lambda_{\min},\lambda_{\max}])}.
    \end{align*}
\end{corollary}
\begin{proof}
    Since $\range(\bm{Q}_s) \subseteq \range(\bm{W})$ we have
    \begin{align*}
        &\|f(\bm{A}) - \bm{Q}_s\llbracket\bm{Q}_s^\T f(\bm{A})\bm{Q}_s \rrbracket_k\bm{Q}_s^\T\|_{\F} 
        \\&\hspace{7em}\leq \|f(\bm{A}) - \bm{W}\llbracket\bm{W}^\T f(\bm{A})\bm{W} \rrbracket_k\bm{W}^\T\|_{\F} 
        \\&\hspace{14em}\leq 
        \|f(\bm{A}) - \bm{W}\llbracket\widetilde{\bm{X}}\rrbracket_k\bm{W}^\T\|_{\F},
    \end{align*}
    where we used that $\bm{W}\llbracket\bm{W}^\T f(\bm{A})\bm{W} \rrbracket_k\bm{W}^\T$ is the best rank $k$ approximation to $f(\bm{A})$ whose range and co-range is contained in $\range(\bm{W})$ \cite[Lemma 3.3]{funnystrom2}. Hence, by \Cref{theorem:robust} we have
    \begin{align*}
        \|f(\bm{A}) - \bm{Q}_s \llbracket \bm{X} \rrbracket_k \bm{Q}_s^\T\|_{\F} 
        &\leq \|f(\bm{A}) - \bm{Q}_s\llbracket\bm{Q}_s^\T f(\bm{A})\bm{Q}_s \rrbracket_k\bm{Q}_s^\T\|_{\F} + 2\|\bm{Q}_s^\T f(\bm{A})\bm{Q}_s - \bm{X}\|_{\F} 
        \\&\leq 
        \|f(\bm{A}) - \bm{W}\llbracket\widetilde{\bm{X}}\rrbracket_k\bm{W}^\T\|_{\F} + 2\|\bm{Q}_s^\T f(\bm{A})\bm{Q}_s - \bm{X}\|_{\F}.
    \end{align*}
    Applying \Cref{lemma:2_times_polynomial_approx} yields the desired result. 
\end{proof}
\Cref{corr:early_bigger_subspace} already shows why we expect the Krylov-aware approach in \Cref{alg:krylow} to outperform a naive combination of Krylov subspace methods and the randomized SVD in \Cref{alg:rsvd_matfun}. In fact, we might expect a major improvement, since $\range(\bm{Q}_s)$ is a significantly larger subspace than $\range(\bm{W})$. In the subsequent sections we will derive stronger bounds for the approximation returned by \Cref{alg:krylow} to better justify this intuition.

\subsection{Structural bounds}\label{section:structural}
We next derive structural bounds for $\|f(\bm{A})- \bm{Q}_s \bm{Q}_s^\T f(\bm{A}) \bm{Q}_s \bm{Q}_s^\T\|_{\F}$ and $\|f(\bm{A})- \bm{Q}_s \llbracket \bm{Q}_s^\T f(\bm{A}) \bm{Q}_s\rrbracket_k \bm{Q}_s^\T\|_{\F}$ that is true for \textit{any} sketch matrix $\bm{\Omega}$ as long as $\bm{\Omega}_k$ defined in \eqref{eq:omega_partition} has rank $k$. These bounds will allow us to obtain probabilistic bounds on the error of approximating $f(\bm{A})$, at least under the assumption that $\bm{Q}_s \bm{Q}_s^\T f(\bm{A}) \bm{Q}_s \bm{Q}_s^\T$ and $\bm{Q}_s \llbracket \bm{Q}_s^\T f(\bm{A}) \bm{Q}_s\rrbracket_k \bm{Q}_s^\T$ are computed exactly. As a reminder, we will remove this assumption in \Cref{section:krylow} using the perturbation bounds from \Cref{section:inexact_projections}.
%\David{I just realized that we need to comment on what happens when $\rank(f(\bm{A})) < k$. This is not a problem, and there is a way to fix it. I just need to think what the best way of doing that is. }

To state our bounds, we introduce the quantity
\begin{equation}
\label{eqn:min_ratio_omega}
    \mathcal{E}_{\bm{\Omega}}(s;f) =
    \min\limits_{p \in \mathbb{P}_{s-1}}\left[\|p(\bm{\Lambda}_{n \setminus k}) \bm{\Omega}_{n \setminus k} \bm{\Omega}_k^{\dagger}\|_{\F}^2\max\limits_{i=1,\ldots,k} \left|\frac{f(\lambda_i)}{p(\lambda_i)}\right|^2\right],
\end{equation}
which quantifies the extent to which an $(s-1)$ degree polynomial can be large (relative to $f$) on the eigenvalues $\lambda_1, \ldots, \lambda_k$ and small on the remaining eigenvalues. Recall that we order $\bm{A}$'s eigenvalues with respect to $f$, so $|f(\lambda_1)| \geq |f(\lambda_1)| \geq \ldots \geq |f(\lambda_n)|$.
\begin{lemma}\label{lemma:structural}
    Consider $\bm{A} \in \mathbb{R}^{n \times n}$ as defined in \eqref{eq:A}. Assuming $\bm{\Omega}_k$ in \eqref{eq:omega_partition} has rank $k$, for all functions $f: \mathbb{R} \to \mathbb{R}$, we have
    \begin{align}\label{eq:structural}
        \begin{split}
        &\|f(\bm{A}) - \bm{Q}_s  \bm{Q}_s^\T f(\bm{A})\bm{Q}_s \bm{Q}_s^\T \|_{\F}^2 
        \\&\hspace{7em}\leq \|f(\bm{A}) - \bm{Q}_s \llbracket \bm{Q}_s^\T f(\bm{A})\bm{Q}_s\rrbracket_k \bm{Q}_s^\T\|_{\F}^2 
        \\&\hspace{14em}\leq
        \|f(\bm{\Lambda}_{n \setminus k})\|_{\F}^2 + 5 \mathcal{E}_{\bm{\Omega}}(s;f).
        \end{split}
    \end{align}
\end{lemma}
From this bound, we can further see why \Cref{alg:krylow} should be preferred over directly applying the randomized SVD to $f(\bm{A})$. As an extreme case, consider when $f$ is a polynomial of degree at most $s-1$. Then the standard randomized SVD bound \cite[Theorem 9.1]{rsvd} essentially replaces $p$ with $f$ in $\mathcal{E}_{\bm{\Omega}}(s;f)$ in \eqref{eq:structural}:
\begin{equation*}
    \|f(\bm{A}) - \bm{W} \llbracket \bm{W}^T f(\bm{A}) \bm{W} \rrbracket_k \bm{W}\|_\F^2 \leq \|f(\bm{\Lambda}_{n \setminus k})\|_\F^2 + 5 \|f(\bm{\Lambda}_{n \setminus k}) \bm{\Omega}_{n \setminus k} \bm{\Omega}_k^{\dagger}\|_\F^2,
\end{equation*}
where $\bm{W}$ is an orthonormal basis for $\range(f(\bm{A}) \bm{\Omega})$. Since $\mathcal{E}_{\bm{\Omega}}(s;f)$ minimizes over \emph{all} degree $s-1$ polynomials, it is always smaller than  $\|f(\bm{\Lambda}_{n \setminus k}) \bm{\Omega}_{n \setminus k} \bm{\Omega}_k^{\dagger}\|_\F^2$. So, the approximation returned by \Cref{alg:krylow} is expected to be more accurate compared to the randomized SVD. When $f$ is not a polynomial, the error of the randomized SVD roughly corresponds to plugging a good polynomial approximation for $f$ into \cref{eqn:min_ratio_omega}, but again there might  be a better choice to minimize $\mathcal{E}_{\bm{\Omega}}(s;f)$.

Indeed, in many cases, we find that the improvement obtained from effectively optimizing over all possible degree $s-1$ polynomials is significant -- there is often a much better choice of polynomial than a direct approximation to $f$ for \cref{eqn:min_ratio_omega}. This is reflected in our experiments (\Cref{sec:experiments}) and we also provide a concrete analysis involving the matrix exponential in \Cref{section:exponential}.


% \begin{lemma}\label{lemma:structural}
%     Consider $\bm{A} \in \mathbb{R}^{n \times n}$ as defined in \eqref{eq:A}. Assuming $\bm{\Omega}_1$ in \eqref{eq:omega_partition} has rank $k$ and let $\bm{Q}_s$ be as in \Cref{alg:krylow}. Then for all functions $f: \mathbb{R} \mapsto \mathbb{R}$ we have
%     \begin{align}\label{eq:structural}
%     \begin{split}
%         \hspace{3em}&\hspace{-3em}\|f(\bm{A}) - \bm{Q}_s \bm{Q}_s^\T f(\bm{A})\bm{Q}_s \bm{Q}_s^\T\|_{\F}^2 \leq \|f(\bm{A}) - \bm{Q}_s \llbracket \bm{Q}_s^\T f(\bm{A})\bm{Q}_s\rrbracket_k \bm{Q}_s^\T\|_{\F}^2 
%         \\&\leq
%         \|f(\bm{\Lambda}_{n \setminus k})\|_{\F}^2 + 5 \min\limits_{p \in \mathbb{P}_{s-1}}\left[\|p(\bm{\Lambda}_{n \setminus k}) \bm{\Omega}_{n \setminus k} \bm{\Omega}_k^{\dagger}\|_{\F}^2 \cdot \max\limits_{i = 1,\ldots,k} \left|\frac{f(\lambda_i)}{p(\lambda_i)}\right|^2\right].
%         \end{split}
%     \end{align}
% \end{lemma}
\begin{proof}[Proof of \Cref{lemma:structural}]
The first inequality is due to the fact that $\bm{Q}_s\bm{Q}_s^{\T}f(\bm{A})\bm{Q}_s\bm{Q}_s^{\T}$ is the nearest matrix to $f(\bm{A})$ in the Frobenius norm whose range and co-range is contained in $\range(\bm{Q}_s)$ \cite[Lemma 3.3]{funnystrom2}. 
We proceed with proving the second inequality. 

To do so, it of course suffices to prove the inequality where $\mathcal{E}_{\bm{\Omega}}(s;f)$ is replaced with $\|p(\bm{\Lambda}_{n \setminus k}) \bm{\Omega}_{n \setminus k} \bm{\Omega}_k^{\dagger}\|_{\F}^2\cdot \max_{i=1,\ldots,k} \left|{f(\lambda_i)}/{p(\lambda_i)}\right|^2$ for any choice of $p \in \mathbb{P}_{s-1}$.
Note that if we choose $p$ so that $p(\lambda_i) = 0$ for some $i = 1,\ldots,k$ then the right hand side of \eqref{eq:structural} is infinite and the bound trivially holds. 
Hence, we may assume that $p(\lambda_i) \neq 0$ for $i = 1,\ldots,k$.
Consequently, $p(\bm{\Lambda}_k)$ is non-singular.
Define $\bm{Z} = p(\bm{A}) \bm{\Omega} \bm{\Omega}_k^{\dagger} p(\bm{\Lambda}_k)^{-1}$ and let $\widetilde{\bm{P}}$ be the orthogonal projector onto $\range(\bm{Z}) \subseteq \range(\bm{Q}_s)$.
Note that $\rank(\bm{Z}) \leq k$ and $\bm{Q}_s \llbracket\bm{Q}_s^\T f(\bm{A})\bm{Q}_s\rrbracket_{k} \bm{Q}_s^\T$ is the best rank $k$ approximation to $f(\bm{A})$ whose range and co-range is contained in $\range(\bm{Q}_s)$ \cite[Lemma 3.3]{funnystrom2}.
Hence,
\begin{equation*}
    \|f(\bm{A}) - \bm{Q}_s \llbracket\bm{Q}_s^\T f(\bm{A})\bm{Q}_s\rrbracket_{k} \bm{Q}_s^\T \|_{\F}^2 \leq \|f(\bm{A}) - \widetilde{\bm{P}}f(\bm{A})\widetilde{\bm{P}} \|_{\F}^2.
\end{equation*}
Now define $\widehat{\bm{P}} = \bm{U}^\T \widetilde{\bm{P}} \bm{U}$, which is the orthogonal projector onto $\range(\bm{U}^\T \bm{Z})$. By the unitary invariance of the Frobenius norm we have
\begin{equation*}
    \|f(\bm{A}) - \widetilde{\bm{P}}f(\bm{A})\widetilde{\bm{P}} \|_{\F}^2 = \|f(\bm{\Lambda}) - \widehat{\bm{P}} f(\bm{\Lambda}) \widehat{\bm{P}}\|_{\F}^2.
\end{equation*}
Furthermore, by Pythagorean theorem,
\begin{equation}\label{eq:two_terms}
    \|f(\bm{\Lambda}) - \widehat{\bm{P}} f(\bm{\Lambda}) \widehat{\bm{P}}\|_{\F}^2 = \|(\bm{I} - \widehat{\bm{P}}) f(\bm{\Lambda})\|_{\F}^2 + \|\widehat{\bm{P}} f(\bm{\Lambda})(\bm{I}-\widehat{\bm{P}})\|_{\F}^2.
\end{equation}
We are going to bound the two terms on the right hand side of \eqref{eq:two_terms} separately. 
Our analysis for the first is similar to the proof of \cite[Theorem 9.1]{rsvd}.

Note that since $\rank(\bm{\Omega}_k) = k$ we have $\bm{\Omega}_k \bm{\Omega}_k^{\dagger} = \bm{I}$. Hence,
\begin{equation*}
    \bm{U}^\T\bm{Z} = \bm{U}^\T p(\bm{A}) \bm{\Omega} \bm{\Omega}_k^{\dagger} p(\bm{\Lambda}_k)^{-1} = \begin{bmatrix} \bm{I} \\ p(\bm{\Lambda}_{n \setminus k}) \bm{\Omega}_{n \setminus k} \bm{\Omega}_k^{\dagger} p(\bm{\Lambda}_k)^{-1}\end{bmatrix} =: \begin{bmatrix} \bm{I} \\ \bm{F} \end{bmatrix}. 
\end{equation*}
Hence, 
\begin{align*}
    \bm{I}-\widehat{\bm{P}} &= \begin{bmatrix} \bm{I} - (\bm{I}+\bm{F}^\T \bm{F})^{-1} & -(\bm{I}+\bm{F}^\T \bm{F})^{-1}\bm{F}^\T \\ -\bm{F}(\bm{I}+\bm{F}^\T \bm{F})^{-1} & \bm{I} - \bm{F}(\bm{I}+\bm{F}^\T \bm{F})^{-1}\bm{F}^\T\end{bmatrix} \\
    &\preceq \begin{bmatrix} \bm{F}^\T \bm{F} & -(\bm{I}+\bm{F}^\T \bm{F})^{-1}\bm{F}^\T \\ -\bm{F}(\bm{I}+\bm{F}^\T \bm{F})^{-1} & \bm{I}\end{bmatrix},
\end{align*}
where the inequality is due to \cite[Proposition 8.2]{rsvd}. Consequently,
\begin{align*}
     \|(\bm{I} - \widehat{\bm{P}}) f(\bm{\Lambda})\|_{\F}^2 &= \tr(f(\bm{\Lambda})(\bm{I} - \widehat{\bm{P}})f(\bm{\Lambda})) 
     \\&\leq
      \|f(\bm{\Lambda}_{n \setminus k})\|_{\F}^2 + \|\bm{F}f(\bm{\Lambda}_k)\|_{\F}^2 \\&\leq
     \|f(\bm{\Lambda}_{n \setminus k})\|_{\F}^2 + \|p(\bm{\Lambda}_{n \setminus k})\bm{\Omega}_{n \setminus k} \bm{\Omega}_k^{\dagger}\|_{\F}^2\|p(\bm{\Lambda}_k)^{-1}f(\bm{\Lambda}_k)\|_2^2  
     \\&= \|f(\bm{\Lambda}_{n \setminus k})\|_{\F}^2 + \|p(\bm{\Lambda}_{n \setminus k})\bm{\Omega}_{n \setminus k} \bm{\Omega}_k^{\dagger}\|_{\F}^2 \max\limits_{i=1,\ldots,k} \left|\frac{f(\lambda_i)}{p(\lambda_i)}\right|^2. \numberthis \label{eq:first_term}
\end{align*}

We proceed with bounding the second term in \eqref{eq:two_terms}.
Our analysis is similar to the proof of \cite[Lemma 3.7]{persson_kressner_23}. 
By the triangle inequality we have 
\begin{equation*}
    \|\widehat{\bm{P}} f(\bm{\Lambda})(\bm{I}-\widehat{\bm{P}})\|_{\F} \leq \left\|\begin{bmatrix} \bm{0} & \\ & f(\bm{\Lambda}_{n \setminus k})\end{bmatrix} \widehat{\bm{P}}\right\|_{\F} + \left\|(\bm{I} - \widehat{\bm{P}}) \begin{bmatrix} f(\bm{\Lambda}_k) & \\ & \bm{0}\end{bmatrix}\right\|_{\F}.
\end{equation*}
Using a similar argument as in \eqref{eq:first_term} we have
\begin{equation}
    \left\|(\bm{I} - \widehat{\bm{P}}) \begin{bmatrix} f(\bm{\Lambda}_k) & \\ & \bm{0}\end{bmatrix} \right\|_{\F} \leq \|p(\bm{\Lambda}_{n \setminus k})\bm{\Omega}_{n \setminus k} \bm{\Omega}_k^{\dagger}\|_{\F}\max\limits_{i=1,\ldots,k} \left|\frac{f(\lambda_i)}{p(\lambda_i)}\right|,\label{eq:second_term}
\end{equation}
and since $\bm{F}(\bm{I}+\bm{F}^\T \bm{F})^{-1}\bm{F}^\T \preceq \bm{F}\bm{F}^\T$ we have
\begin{align*}
    \left\|\begin{bmatrix} \bm{0} & \\ & f(\bm{\Lambda}_{n \setminus k})\end{bmatrix} \widehat{\bm{P}}\right\|_{\F}^2 
    &= \tr\left(\begin{bmatrix} \bm{0} & \\ & f(\bm{\Lambda}_{n \setminus k})\end{bmatrix} \widehat{\bm{P}}\begin{bmatrix} \bm{0} & \\ & f(\bm{\Lambda}_{n \setminus k})\end{bmatrix}\right) 
    \\&=\tr(f(\bm{\Lambda}_{n \setminus k})\bm{F}(\bm{I} + \bm{F}^\T \bm{F})^{-1}\bm{F}^\T f(\bm{\Lambda}_{n \setminus k}))
    \\&\leq
    \tr(f(\bm{\Lambda}_{n \setminus k})\bm{F}\bm{F}^\T f(\bm{\Lambda}_{n \setminus k}))
    =
    \|f(\bm{\Lambda}_{n \setminus k}) \bm{F}\|_{\F}^2 
    \\&\leq \|f(\bm{\Lambda}_{n \setminus k})\|_2^2 \|p(\bm{\Lambda}_k)^{-1}\|_2^2 \|p(\bm{\Lambda}_{n \setminus k}) \bm{\Omega}_{n \setminus k} \bm{\Omega}_k^{\dagger}\|_{\F}^2 
    \\&\leq 
    \|p(\bm{\Lambda}_{n \setminus k})\bm{\Omega}_{n \setminus k} \bm{\Omega}_k^{\dagger}\|_{\F}^2\max\limits_{i=1,\ldots,k} \left|\frac{f(\lambda_i)}{p(\lambda_i)}\right|^2.\numberthis \label{eq:third_term}
\end{align*}
Inserting the bounds \eqref{eq:first_term}, \eqref{eq:second_term}, and \eqref{eq:third_term} into \eqref{eq:two_terms} and optimizing over $\mathbb{P}_{s-1}$ yields the desired inequality. 
\end{proof}

\subsection{Probabilistic bounds}\label{section:probabilistic}
%\David{Explain here that this is why this is much better than the randomized SVD}
With the structural bound available, we are ready to derive probabilistic bounds for $\|f(\bm{A})- \bm{Q}_s \bm{Q}_s^\T f(\bm{A}) \bm{Q}_s \bm{Q}_s^\T\|_{\F}$ and $\|f(\bm{A})- \bm{Q}_s \llbracket\bm{Q}_s^\T f(\bm{A}) \bm{Q}_s\rrbracket_k \bm{Q}_s^\T\|_{\F}$. Note that by \Cref{lemma:structural} it is sufficient to derive a probabilistic bound for $\mathcal{E}_{\bm{\Omega}}(s;f)$ defined in \cref{eqn:min_ratio_omega}. 


We will bound $\mathcal{E}_{\bm{\Omega}}(s;f)$ in terms of a deterministic quantity
\begin{equation}
\label{eqn:min_ratio}
    \mathcal{E}(s;f) =
    \min\limits_{p \in \mathbb{P}_{s-1}}\left[\|p(\bm{\Lambda}_{n \setminus k}) \|_{\F}^2\max\limits_{i=1,\ldots,k} \left|\frac{f(\lambda_i)}{p(\lambda_i)}\right|^2\right] ,
\end{equation}
which again quantifies how large a polynomial can be (relative to $f$) on the eigenvalues $\lambda_1, \ldots, \lambda_k$ and small on the remaining eigenvalues. However, $\mathcal{E}(s;f)$ does not depend on the randomness used by the algorithm.
\begin{lemma}\label{lemma:probabilistic}
    If $\bm{\Omega}$ is a standard Gaussian matrix, $\bm{\Omega}_k$ and $\bm{\Omega}_{n \setminus k}$ are as defined as in \eqref{eq:omega_partition}, and $\mathcal{E}_{\bm{\Omega}}(s;f)$ and $\mathcal{E}(s;f)$ are as defined in \cref{eqn:min_ratio_omega,eqn:min_ratio}, then
    \begin{enumerate}[(i)]
        \item for any $u,t\geq 0$, with probability at least $1-e^{-(u-2)/4} - \sqrt{\pi k} \left(\frac{t}{e}\right)^{-(\ell -k + 1)/2}$,
        \begin{align*}
            \mathcal{E}_{\bm{\Omega}}(s;f)\leq \frac{ut k}{\ell - k + 1} \mathcal{E}(s;f);
        \end{align*}\label{item:tailbound}
        \item if $\ell -k \geq 2$ we have \begin{align*}
            \mathbb{E}[\mathcal{E}_{\bm{\Omega}}(s;f)] 
            \leq \frac{k}{\ell-k-1} \mathcal{E}(s;f).
        \end{align*}\label{item:expectationbound}
    \end{enumerate}
\end{lemma}
% \begin{lemma}\label{lemma:probabilistic}
%     If $\bm{\Omega}$ is a standard Gaussian matrix, and $\bm{\Omega}_k$ and $\bm{\Omega}_{n \setminus k}$ are defined as in \eqref{eq:omega_partition}, then
%     \begin{enumerate}[(i)]
%         \item for any $u,t\geq 0$, with probability at least $1-e^{-(u-2)/4} - \sqrt{\pi k} \left(\frac{t}{e}\right)^{-(\ell -k + 1)/2}$ we have
%         \begin{align*}
%             \hspace{8em}&\hspace{-8em}\min\limits_{p \in \mathbb{P}_{s-1}}\left[\|p(\bm{\Lambda}_{n \setminus k}) \bm{\Omega}_{n \setminus k} \bm{\Omega}_k^{\dagger}\|_{\F}^2\max\limits_{i=1,\ldots,k} \left|\frac{f(\lambda_i)}{p(\lambda_i)}\right|^2\right] 
%             \\&\leq \frac{ut k}{\ell - k + 1} \min\limits_{p \in \mathbb{P}_{s-1}} \left[\|p(\bm{\Lambda}_{n \setminus k})\|_{\F}^2\max\limits_{i=1,\ldots,k} \left|\frac{f(\lambda_i)}{p(\lambda_i)}\right|^2 \right];
%         \end{align*}\label{item:tailbound}
%         \item if $\ell -k \geq 2$ we have \begin{align*}
%             \hspace{8em}&\hspace{-8em}\mathbb{E}\left[\min\limits_{p \in \mathbb{P}_{s-1}}\|p(\bm{\Lambda}_{n \setminus k}) \bm{\Omega}_{n \setminus k} \bm{\Omega}_k^{\dagger}\|_{\F}^2 \max\limits_{i=1,\ldots,k} \left|\frac{f(\lambda_i)}{p(\lambda_i)}\right|^2\right] 
%             \\&\leq \frac{k}{\ell-k-1} \min\limits_{p \in \mathbb{P}_{s-1}} \left[\|p(\bm{\Lambda}_{n \setminus k})\|_{\F}^2\max\limits_{i=1,\ldots,k} \left|\frac{f(\lambda_i)}{p(\lambda_i)}\right|^2\right].
%         \end{align*}\label{item:expectationbound}
%     \end{enumerate}
% \end{lemma}
\begin{proof}
\textit{(\ref*{item:tailbound})}: For any polynomial $p \in \mathbb{P}_{s-1}$ by \cite[Proposition 8.6]{tropp2023randomized} we have with probability at least $1-e^{-(u-2)/4} - \sqrt{\pi k} \left(\frac{t}{e}\right)^{-(\ell -k + 1)/2}$
\begin{equation*}
    \left[\|p(\bm{\Lambda}_{n \setminus k}) \bm{\Omega}_{n \setminus k} \bm{\Omega}_k^{\dagger}\|_{\F}^2\max\limits_{i=1,\ldots,k} \left|\frac{f(\lambda_i)}{p(\lambda_i)}\right|^2\right] \leq \frac{ut k}{\ell - k + 1}\left[\|p(\bm{\Lambda}_{n \setminus k})\|_{\F}^2\max\limits_{i=1,\ldots,k} \left|\frac{f(\lambda_i)}{p(\lambda_i)}\right|^2 \right].
\end{equation*}
The inequality is respected if we minimize both sides over all polynomials.

\textit{(\ref*{item:expectationbound})}: This is proven in an identical fashion utilizing the expectation bound in \cite[Proposition 8.6]{tropp2023randomized}.
\end{proof}

\subsection{Error bounds for Krylov aware low-rank approximation}\label{section:krylow}
% With the results in \Cref{section:inexact_projections}-\Cref{section:probabilistic} we are ready to start deriving a probabilistic error bound for the \Cref{alg:krylow}. We begin with a standard bound on how accurately we can compute the projection $\bm{Q}_s^\T f(\bm{A}) \bm{Q}_s$.
% \begin{lemma}\label{lemma:2_times_polynomial_approx}
% Let $\lambda_{\max}$ and $\lambda_{\min}$ denote the largest and smallest eigenvalue of $\bm{A}$. Let $\bm{T}_q$ and $\bm{Q}_s$ be as in \cref{alg:krylow}. Then, 
% \begin{align*}
%     &\|\bm{Q}_s^\T f(\bm{A})\bm{Q}_s - f(\bm{T}_q)_{1:d_s,1:d_s}\|_{\F} \leq 2\sqrt{d_s} \inf\limits_{p \in \mathbb{P}_{2r+1}}\|f(x)-p(x)\|_{L^{\infty}([\lambda_{\min},\lambda_{\max}])}.
% \end{align*}
% \end{lemma}
% \begin{proof}
% By \cite[Lemma 3.1]{chen_hallman_23} we know that for any polynomial $p \in \mathbb{P}_{2r+1}$ we have $\bm{Q}_s^\T p(\bm{A}) \bm{Q}_{s} = p(\bm{T}_q)_{1:d_s,1:d_s}$. Therefore,
% \begin{align*}
%     \|\bm{Q}_s^\T f(\bm{A})\bm{Q}_s - f(\bm{T}_q)_{1:d_s,1:d_s}\|_{\F}
%     &= \|\bm{Q}_1^\T f(\bm{A}) \bm{Q}_1 - \bm{Q}_1^\T p(\bm{T}) \bm{Q}_1 + p(\bm{T})_{1:d_s,1:d_s} - f(\bm{T})_{1:d_s,1:d_s} \|_{\F} 
%     \\&\leq \|\bm{Q}_1^\T f(\bm{A}) \bm{Q}_1 - \bm{Q}_1^\T p(\bm{A}) \bm{Q}_1 \|_{\F} +  \|(p(\bm{T}) - f(\bm{T}) )_{1:d_s,1:d_s} \|_{\F} 
%     \\&\leq \sqrt{d_s} \left( \|f(\bm{A}) - p(\bm{A}) \|_2 +  \| p(\bm{T})  - f(\bm{T}) \|_2\right)
%     \\&\leq 2 \sqrt{d_s} \|f(x) - p(x)\|_{L^{\infty}([\lambda_{\min},\lambda_{\max}])},
% \end{align*}
% where the last inequality is due to the fact that the spectrum of $\bm{T}$ is contained in $[\lambda_{\min},\lambda_{\max}]$.
% Optimizing over $p\in\mathbb{P}_{2r+1}$ gives the result.
% %Applying \cref{thm:lanczos_exact} with $\bm{Q}_s$ as the starting block and running block-Lanczos for $r+1$ iterations we have that, for any $p\in\mathbb{P}_{2r+1}$, $\bm{Q}_s^\T p(\bm{A}) \bm{Q}_s = p(\bm{T})_{1:d_s,1:d_s}$.
% %Therefore,
% %\begin{align*}
%     %\varepsilon_r
%     %&= \|\bm{Q}_1^\T f(\bm{A}) \bm{Q}_1 - \bm{Q}_1^\T p(\bm{T}) \bm{Q}_1 + p(\bm{T})_{1:d_s,1:d_s} - f(\bm{T})_{1:d_s,1:d_s} \|_{\F} 
%     %\\&\leq \|\bm{Q}_1^\T f(\bm{A}) \bm{Q}_1 - \bm{Q}_1^\T p(\bm{A}) \bm{Q}_1 \|_{\F} +  \|(p(\bm{T}) - f(\bm{T}) )_{1:d_s,1:d_s} \|_{\F} 
%     %\\&\leq \sqrt{d_s} \left( \|f(\bm{A}) - p(\bm{A}) \|_2 +  \| p(\bm{T}) \bm{E}_1 - f(\bm{T}) \|_2\right)
%     %\\&\leq 2 \sqrt{d_s} \|f(x) - p(x)\|_{L^{\infty}([\lambda_{\min},\lambda_{\max}])},
% %\end{align*}
% %where the last inequality is due to the fact that the spectrum of $\bm{T}$ is contained in $[\lambda_{\min},\lambda_{\max}]$.
% %Optimizing over $p\in\mathbb{P}_{2r-1}$ gives the result.
% \end{proof}


%\subsection{Bounds for computing quadratic forms}
%Consider the error 
%\begin{equation}\label{eq:error}
 %   \varepsilon_r := \|\bm{Q}_1^\T f(\bm{A}) \bm{Q}_1 - f(\bm{T})_{1:d_s, 1:d_s}\|_{\F}.
%\end{equation}
%If $f\in\mathbb{P}_{2r+1}$ then $\epsilon_r = 0$. 
%In fact, if $f$ is near to a degree $2r+1$ polynomial, then $\epsilon_r$ is small.

%\begin{remark}
%Suppose that $f(x)^s$ is approximated well with a degree $q$ polynomial $\tilde{p}_q$. Then, define
%\begin{equation*}
    %p_q(x) = f(\lambda_{k}) \frac{\tilde{p}_q(x)}{f(\lambda_{k}) ^s}\approx f(\lambda_{k}) \frac{f(x)^s}{f(\lambda_{k}) ^s}.
%\end{equation*}
%In this case, we can set $\delta_{q}(x) \approx f(\lambda_{k}) \frac{f(x)^s}{f(\lambda_{k}) ^s}$, which would give subspace iteration.

%\end{remark}



%\tyler{if $f$ is operator monotone, can we show $\epsilon_r$ is on the order of $\|f(\bm{\Lambda}_2)\|$}

%\hrulefill

%\begin{proof}
%Note that it suffices to show the statement whenever $f$ is a monomial of degree at most $2r+3$. For any polynomial of degree $0$ the statement is trivial. Hence, we proceed with monomials of degree at least $1$. Note that we have the following recurrence relation
%\begin{equation*}
 %   \bm{A}\bm{Q} = \bm{Q}\bm{T} + \bm{Q}_{q+1}\bm{R}_{q+1} \bm{E}_{q+1}^\T,
%\end{equation*}
%where
%\begin{equation*}
 %   \bm{E}_{q+1} = \begin{bmatrix} \bm{0}_{\ell q \times \ell} \\ \bm{I}_{\ell \times \ell} \end{bmatrix}.
%\end{equation*}
%By induction one can show that 
%\begin{equation*}
    %\bm{A}^d \bm{Q}\bm{E}_i = \bm{Q}\bm{T}^d \bm{E}_i + \sum\limits_{j=0}^{d-1}\bm{A}^{d-j-1}\bm{Q}_{q+1}\bm{R}_{q+1}\bm{E}_{q+1}^\T\bm{T}^j \bm{E}_i.
%\end{equation*}
%Note that $\bm{T}$ is a block tridiagonal matrix with blocks of sizes $\ell \times \ell$. Consequently, the $\ell-$block bandwidth of $\bm{T}^j$ is equal to $j$. Hence, whenever $q+1-i > j$ we have that $\bm{E}_{q+1}^\T\bm{T}^j \bm{E}_i = \bm{0}$. Consequently, whenever $q+1-i \geq d$ we have
%\begin{equation*}
    %\bm{A}^d\bm{Q}_i = \bm{A}^d \bm{Q}\bm{E}_i = \bm{Q}\bm{T}^d \bm{E}_i = \bm{Q}(\bm{T}^r)_{:,((i-1)\ell+1):i\ell}.
%\end{equation*}
%Hence, whenever $q-s+1 \geq d$
%\begin{equation*}
%    \bm{A}^d \bm{Q}_{:,1:\ell s} = \bm{Q}_{:,1:\ell s}(\bm{T}^d)_{:,1:\ell s}.
%\end{equation*}
%Hence,
%\begin{align*}
    %&\bm{Q}^\T_{:,1:ks} \bm{A}^{2d}\bm{Q}^\T_{:,1:ks} = (\bm{T}^{2r})_{1:\ell s,1:\ell s};\\
    %&\bm{Q}^\T_{:,1:ks} \bm{A}^{2d+1}\bm{Q}^\T_{:,1:ks} = (\bm{T}^r)_{:,1:\ell s}^\T\bm{Q}_{:,1:\ell s}^\T \bm{A}\bm{Q}_{:,1:\ell s}(\bm{T}^d)_{:,1:\ell s} = (\bm{T}^{2d+1})_{1:\ell s,1:\ell s},
%\end{align*}
%which implies that for any monomial of degree at most $2(q-s+1)+1 = 2r+3$ the result holds, as required. 

%\end{proof}


% \Chris{This corollary would only be immediate if you know the prior work. My experience in the past is many people in NLA don't know the triangle inequality analysis of Lanczos. I also don't see a Lemma 2.2 in  \cite{chen_hallman_23}. What is the correct number? We should remind them what $d_s$ is, especially since the corollary doesn't define $s$.} \David{REMINDER: check if Lemma 2.2 is in the Arxiv version. }
% \tyler{I think its still immediate even if you don't know it. we will update arxiv whenever the paper is published so that the numbers match. I think we can just use $s$ instead for simplicity since $d_s\leq s$ (and probably is typically equal).}
%We have the following immediate corollary.
%\begin{corollary}\label{corollary:best_poly_approx}
%Let $\lambda_{\max}$ and $\lambda_{\min}$ denote the largest and smallest eigenvalue of $\bm{A}$. Then, under the assumptions of \Cref{lemma:krylov_nested} we have
%\begin{align*}
 %   &\varepsilon_r \leq 2\sqrt{d_s} \inf\limits_{p \in \mathbb{P}_{2r+1}}\|f(x)-p(x)\|_{L^{\infty}([\lambda_{\min},\lambda_{\max}])}.
%\end{align*}
%\end{corollary}
%\begin{proof}
%Note that $\varepsilon_r \leq \sqrt{d_s}\|\bm{Q}_1^\T f(\bm{A})\bm{Q}_1 - f(\bm{T})_{1:\ell s,1:\ell s}\|_2$. The result is obtained by applying \cref{thm:lanczos_exact}.
%Hence, it suffices to show the result for the operator norm. By Lemma~\ref{lemma:polynomial_exactness} we have
%\begin{align*}
 %    &\|\bm{Q}_{:,1:ks}^\T f(\bm{A})\bm{Q}_{:,1:ks} - f(\bm{T})_{1:ks,1:ks}\|_2 =\\ &\|\bm{Q}_{:,1:ks}^\T f(\bm{A})\bm{Q}_{:,1:ks} - \bm{Q}_{:,1:ks}^\T p(\bm{A})\bm{Q}_{:,1:ks} + \bm{Q}_{:,1:ks}^\T p(\bm{A})\bm{Q}_{:,1:ks} - f(\bm{T})_{1:ks,1:ks}\|_2 \leq \\
  %  &\|\bm{Q}_{:,1:ks}^\T f(\bm{A})\bm{Q}_{:,1:ks} - \bm{Q}_{:,1:ks}^\T p(\bm{A})\bm{Q}_{:,1:ks} + p(\bm{T})_{1:ks,1:ks} - f(\bm{T})_{1:ks,1:ks}\|_2 \leq \\
   % &\|\bm{Q}_{:,1:ks}^\T f(\bm{A})\bm{Q}_{:,1:ks} - \bm{Q}_{:,1:ks}^\T p(\bm{A})\bm{Q}_{:,1:ks}\|_2 + \|p(\bm{T})_{1:ks,1:ks} - f(\bm{T})_{1:ks,1:ks}\|_2 \leq \\
    %&\|f(\bm{A}) - p(\bm{A})\|_2 + \|p(\bm{T}) - f(\bm{T})\|_2 \leq 2\mathcal{E}(2r+3, \Lambda(\bm{A}),f).
%\end{align*}
%The result follows since $p$ was arbitrary. 
%\end{proof}
%\subsection{Low rank approximation bounds}
%\begin{remark}
 %   Mention that we want to have bounds when we truncate because we might want to use it  downstream for matvecs and the rank can be quite large since $f$ can be arbitrary. 
%\end{remark}
%\Chris{I also just think we should set up the problem from the beginning that of low-rank approximation of matrix functions, so our goal is to output a fixed rank approximation.}
%\Chris{I would probably seperate out the claims in this Lemma to faciliate easier discussion of why that are interesting claims.}
%\begin{lemma}\label{lemma:structural}
 %    Consider $\bm{A} \in \mathbb{R}^{n \times n}$ as defined in \eqref{eq:A} and a matrix function $f: \mathbb{R} \mapsto \mathbb{R}$. Assuming $\bm{\Omega}_1$ in \eqref{eq:omega_partition} has rank $k$ and assume that $\bm{Q}_1$ and $\bm{T}$ are as in \Cref{alg:krylow}. Then the following hold
     %\begin{enumerate}[(i)]
      %   \item $\|f(\bm{A}) - \bm{Q}_1f(\bm{T})_{1:d_s,1:d_s} \bm{Q}_1^\T\|_{\F}^2 = \|f(\bm{A}) - \bm{Q}_1 \bm{Q}_1^\T f(\bm{A})\bm{Q}_1 \bm{Q}_1^\T\|_{\F}^2 + \varepsilon_r^2$; \label{item:untruncated}
       %  \item $\|f(\bm{A}) - \bm{Q}_1(f(\bm{T})_{1:d_s,1:d_s} )_{(k)}\bm{Q}_1^\T\|_{\F} \leq  \|f(\bm{A}) - \bm{Q}_1 (\bm{Q}_1^\T f(\bm{A})\bm{Q}_1)_{(k)} \bm{Q}_1^\T\|_{\F} + 2\varepsilon_r$; \label{item:truncated}
        % \item $\|f(\bm{A}) - \bm{Q}_1 \bm{Q}_1^\T f(\bm{A})\bm{Q}_1 \bm{Q}_1^\T\|_{\F} \leq \|f(\bm{A}) - \bm{Q}_1 (\bm{Q}_1^\T f(\bm{A})\bm{Q}_1)_{(k)} \bm{Q}_1^\T\|_{\F}$; \label{item:exact_comparison}
       %  \item $\|f(\bm{A}) - \bm{Q}_1 (\bm{Q}_1^\T f(\bm{A})\bm{Q}_1)_{(k)} \bm{Q}_1^\T\|_{\F}^2 \leq \|f(\bm{\Lambda}_2)\|_{\F}^2 + 5 \min\limits_{p \in \mathbb{P}_{s-1}}\left[\|p(\bm{\Lambda}_2) \bm{\Omega}_2 \bm{\Omega}_1^{\dagger}\|_{\F}^2 \cdot \max\limits_{i = 1,\ldots,k} \left|\frac{f(\lambda_i)}{p(\lambda_i)}\right|^2\right]$.\label{item:exact_truncated_bound}
     %\end{enumerate}
%\end{lemma}
%\begin{proof}
%\textit{(\ref*{item:untruncated})} Since $\bm{Q}_1 \bm{Q}_1^\T f(\bm{A})\bm{Q}_1 \bm{Q}_1^\T$ is the nearest matrix to $f(\bm{A})$ whose range and co-range is contained in $\bm{Q}_1$ \textcolor{red}{cite this} we know that $\langle f(\bm{A})-\bm{Q}_1 \bm{Q}_1^\T f(\bm{A})\bm{Q}_1 \bm{Q}_1^\T, \bm{Q}_1 \bm{B}\bm{Q}_1 \rangle = 0$ for any matrix $\bm{B}$. Hence, by Pythagoras theorem we have
%\begin{align*}
 %   &\|f(\bm{A}) - \bm{Q}_1f(\bm{T})_{1:d_s,1:d_s} \bm{Q}_1^\T\|_{\F}^2 = \|f(\bm{A}) -\bm{Q}_1 \bm{Q}_1^\T f(\bm{A})\bm{Q}_1 \bm{Q}_1^\T + \bm{Q}_1(\bm{Q}_1^\T f(\bm{A})\bm{Q}_1 - f(\bm{T})_{1:d_s,1:d_s})\bm{Q}_1^\T\|_{\F}^2 = \\
  %  &\|f(\bm{A}) -\bm{Q}_1 \bm{Q}_1^\T f(\bm{A})\bm{Q}_1 \bm{Q}_1^\T\|_{\F}^2 + \|\bm{Q}_1^\T f(\bm{A})\bm{Q}_1 - f(\bm{T})_{1:d_s,1:d_s}\|_{\F}^2=\|f(\bm{A}) -\bm{Q}_1 \bm{Q}_1^\T f(\bm{A})\bm{Q}_1 \bm{Q}_1^\T\|_{\F}^2 + \varepsilon_r^2,
%\end{align*}
%as required.

%\textit{(\ref*{item:truncated})}: Define $\bm{B} = f(\bm{A}) - \bm{Q}_1 \bm{Q}_1^\T f(\bm{A}) \bm{Q}_1 \bm{Q}_1^\T + \bm{Q}_1 f(\bm{T})_{1:d_s, 1:d_s}\bm{Q}_1^\T$. Note that $\|\bm{B} - f(\bm{A})\|_{\F} = \varepsilon_r$ and $\bm{Q}_1\bm{Q}_1^\T \bm{B} \bm{Q}_1\bm{Q}_1^\T = \bm{Q}_1 f(\bm{T})_{1:d_s, 1:d_s} \bm{Q}_1^\T$.\footnote{This shows that Lanczos exactly computes a quadratic form of a matrix $\bm{B}$ which is near to $f(\bm{A})$. } Hence,
 %   \begin{align*}
  %      &\|f(\bm{A}) - \bm{Q}_{1} (f(\bm{T})_{1:d_s,1:d_s})_{(k)}\bm{Q}_{1}^\T\|_{\F} = \|f(\bm{A}) - (\bm{Q}_{1}\bm{Q}_{1}^\T \bm{B} \bm{Q}_{1}\bm{Q}_{1}^\T)_{(k)}\|_{\F} \leq \\
   %     & \|f(\bm{A}) - \bm{B}\|_{\F} + \|\bm{B} - (\bm{Q}_{1}\bm{Q}_{1}^\T \bm{B} \bm{Q}_{1}\bm{Q}_{1}^\T)_{(k)}\|_{\F} =\\
    %    &\varepsilon_r + \|\bm{B} - (\bm{Q}_{1}\bm{Q}_{1}^\T \bm{B} \bm{Q}_{1}\bm{Q}_{1}^\T)_{(k)}\|_{\F}. \numberthis \label{eq:first_ineq}
    %\end{align*}
    %Choose an orthogonal projector $\bm{P}$ so that $\range(\bm{P}) \subseteq \range(\bm{Q}_1)$ and $(\bm{Q}_1 \bm{Q}_1^\T f(\bm{A})\bm{Q}_1 \bm{Q}_1^\T)_{(k)} = \bm{P} f(\bm{A}) \bm{P}$. Since $(\bm{Q}_{1}\bm{Q}_{1}^\T \bm{B} \bm{Q}_{1}\bm{Q}_{1}^\T)_{(k)}$ is the best rank $k$ approximation whose range is contained in $\range(\bm{Q}_1)$ \textcolor{red}{cite this}, we have
    %\begin{align*}
     %   &\|\bm{B} - (\bm{Q}_{1}\bm{Q}_{1}^\T \bm{B} \bm{Q}_{1}\bm{Q}_{1}^\T)_{(k)}\|_{\F} \leq \|\bm{B} - \bm{P} \bm{B} \bm{P}\|_{\F} = \\
      %  & \|(f(\bm{A}) - \bm{B}) - \bm{P}(f(\bm{A}) - \bm{B}) \bm{P}\|_{\F} + \|f(\bm{A}) - \bm{P} f(\bm{A}) \bm{P}\|_{\F} \leq\\
       % &\|f(\bm{A}) - \bm{B}\|_{\F} + \|f(\bm{A}) - \bm{Q}_1(\bm{Q}_1^\T f(\bm{A}) \bm{Q}_1)_{(k)}\bm{Q}_{1}^\T \|_{\F} =\\
        %&\varepsilon_r + \|f(\bm{A}) - \bm{Q}_1(\bm{Q}_1^\T f(\bm{A}) \bm{Q}_1)_{(k)}\bm{Q}_{1}^\T \|_{\F}. \numberthis \label{eq:final_ineq}
    %\end{align*}
    %Combining \eqref{eq:first_ineq} and \eqref{eq:final_ineq} yields the desired inequality.

    %\textit{(\ref*{item:exact_comparison})}: Let $\bm{P}$ be the proj ector from the proof of \textit{(\ref*{item:truncated})}. Hence, $\range(\bm{P}) \subseteq \range(\bm{Q}_1)$. Since $\bm{Q}_1 \bm{Q}_1^\T f(\bm{A}) \bm{Q}_1 \bm{Q}_1^\T$ is the nearest matrix to $f(\bm{A})$ whose range and co-range is contained in $\range(\bm{Q}_1)$ and $\bm{P} f(\bm{A}) \bm{P}$ is a matrix whose range and co-range is contained in $\range(\bm{Q}_1)$ we have
    %\begin{equation*}
     %   \|f(\bm{A}) - \bm{Q}_1 \bm{Q}_1^\T f(\bm{A})\bm{Q}_1 \bm{Q}_1^\T\|_{\F} \leq \|f(\bm{A}) - \bm{Q}_1 (\bm{Q}_1^\T f(\bm{A})\bm{Q}_1)_{(k)} \bm{Q}_1^\T\|_{\F},
    %\end{equation*}
    %as required.

    %\textit{(\ref*{item:exact_truncated_bound})}: Choose any $p \in \mathbb{P}_{s-1}$. Note that if we choose $p$ so that $p(\lambda_i) = 0$ for some $i = 1,\ldots,k$ then the right hand side of \textit{(\ref*{item:exact_truncated_bound})} is infinity and the bound trivially holds. Hence, we may assume that $p(\lambda_i) \neq 0$ for $i = 1,\ldots,k$. Consequently, $p(\bm{\Lambda}_1)$ is non-singular. Define $\bm{Z} = p(\bm{A}) \bm{\Omega} \bm{\Omega}_1^{\dagger} p(\bm{\Lambda}_1)^{-1}$ and let $\widetilde{\bm{P}}$ be the orthogonal projector onto $\range(\bm{Z}) \subseteq \range(\bm{Q}_1)$. Note that $\rank(\bm{Z}) \leq k$ and $\bm{Q}_1 (\bm{Q}_1^\T f(\bm{A})\bm{Q}_1)_{(k)} \bm{Q}_1^\T$ is the best rank $k$ approximation to $f(\bm{A})$ whose range and co-range is contained in $\range(\bm{Q}_1)$. Hence,
%\begin{equation*}
 %   \|f(\bm{A}) - \bm{Q}_1 (\bm{Q}_1^\T f(\bm{A})\bm{Q}_1)_{(k)} \bm{Q}_1^\T \|_{\F}^2 \leq \|f(\bm{A}) - \widetilde{\bm{P}}f(\bm{A})\widetilde{\bm{P}} \|_{\F}^2.
%\end{equation*}
%Now define $\widehat{\bm{P}} = \bm{U}^\T \widetilde{\bm{P}} \bm{U}$, which is the orthogonal projector onto $\range(\bm{U}^\T \bm{Z})$. By the unitary invariance of the Frobenius norm we have
%\begin{equation*}
 %   \|f(\bm{A}) - \widetilde{\bm{P}}f(\bm{A})\widetilde{\bm{P}} \|_{\F}^2 = \|f(\bm{\Lambda}) - \widehat{\bm{P}} f(\bm{\Lambda}) \widehat{\bm{P}}\|_{\F}^2.
%\end{equation*}
%Furthermore,
%\begin{equation}\label{eq:two_terms}
 %   \|f(\bm{\Lambda}) - \widehat{\bm{P}} f(\bm{\Lambda}) \widehat{\bm{P}}\|_{\F}^2 = \|(\bm{I} - \widehat{\bm{P}}) f(\bm{\Lambda})\|_{\F}^2 + \|\widehat{\bm{P}} f(\bm{\Lambda})(\bm{I}-\widehat{\bm{P}})\|_{\F}^2.
%\end{equation}
%We are going to bound the two terms on the right hand side of \eqref{eq:two_terms} separately. We begin with the first term. 

%Note that 
%\begin{equation*}
 %   \bm{U}^\T\bm{Z} = \bm{U}^\T p(\bm{A}) \bm{\Omega} \bm{\Omega}_1^{\dagger} p(\bm{\Lambda}_1)^{-1} = \begin{bmatrix} \bm{I}_k \\ p(\bm{\Lambda}_2) \bm{\Omega}_2 \bm{\Omega}_1^{\dagger} p(\bm{\Lambda}_1)^{-1}\end{bmatrix} := \begin{bmatrix} \bm{I}_k \\ \bm{F} \end{bmatrix}. 
%\end{equation*}
%Hence, 
%\begin{align*}
 %   \bm{I}-\widehat{\bm{P}} &= \begin{bmatrix} \bm{I}_k - (\bm{I}_k+\bm{F}^\T \bm{F})^{-1} & -(\bm{I}_k+\bm{F}^\T \bm{F})^{-1}\bm{F}^\T \\ -\bm{F}(\bm{I}_k+\bm{F}^\T \bm{F})^{-1} & \bm{I}_{n-k} - \bm{F}(\bm{I}_k+\bm{F}^\T \bm{F})^{-1}\bm{F}^\T\end{bmatrix} \preceq \\
  %  &\begin{bmatrix} \bm{F}^\T \bm{F} & -(\bm{I}_k+\bm{F}^\T \bm{F})^{-1}\bm{F}^\T \\ -\bm{F}(\bm{I}_k+\bm{F}^\T \bm{F})^{-1} & \bm{I}_{n-k}\end{bmatrix},
%\end{align*}
%where the inequality is due to \cite[Proposition 8.2]{rsvd}. Consequently,
%\begin{align*}
 %    &\|(\bm{I} - \widehat{\bm{P}}) f(\bm{\Lambda})\|_{\F}^2 = \tr(f(\bm{\Lambda})(\bm{I} - \widehat{\bm{P}})f(\bm{\Lambda})) \leq\\& \|\bm{F}f(\bm{\Lambda}_1)\|_{\F}^2 + \|f(\bm{\Lambda}_2)\|_{\F}^2 \leq \|f_2(\bm{\Lambda}_2)\|_{\F}^2 + \|p(\bm{\Lambda}_2)\bm{\Omega}_2 \bm{\Omega}_1^{\dagger}\|_{\F}^2\|p(\bm{\Lambda}_1)^{-1}f(\bm{\Lambda}_1)\|_2^2  = \\
  %   & \|f_2(\bm{\Lambda}_2)\|_{\F}^2 + \|p(\bm{\Lambda}_2)\bm{\Omega}_2 \bm{\Omega}_1^{\dagger}\|_{\F}^2 \max\limits_{i=1,\ldots,k} \left|\frac{f(\lambda_i)}{p(\lambda_i)}\right|^2. \numberthis \label{eq:first_term}
%\end{align*}

%We proceed with bounding the second term in \eqref{eq:two_terms}. By the triangle inequality we have 
%\begin{equation*}
 %   \|\widehat{\bm{P}} f(\bm{\Lambda})(\bm{I}-\widehat{\bm{P}})\|_{\F} \leq \|\begin{bmatrix} \bm{0} & \\ & f(\bm{\Lambda}_2)\end{bmatrix} \widehat{\bm{P}}\|_{\F} + \|(\bm{I} - \widehat{\bm{P}}) \begin{bmatrix} f(\bm{\Lambda}_1) & \\ & \bm{0}\end{bmatrix}\|_{\F}.
%\end{equation*}
%Using a similar argument as in \eqref{eq:first_term} we have
%\begin{equation}
 %   \|(\bm{I} - \widehat{\bm{P}}) \begin{bmatrix} f(\bm{\Lambda}_1) & \\ & \bm{0}\end{bmatrix}\|_{\F} \leq \|p(\bm{\Lambda}_2)\bm{\Omega}_2 \bm{\Omega}_1^{\dagger}\|_{\F}\max\limits_{i=1,\ldots,k} \left|\frac{f(\lambda_i)}{p(\lambda_i)}\right|,\label{eq:second_term}
%\end{equation}
%and since $\bm{F}(\bm{I}_k+\bm{F}^\T \bm{F})^{-1}\bm{F}^\T \preceq \bm{F}\bm{F}^\T$ we have
%\begin{align*}
 %   &\|\begin{bmatrix} \bm{0} & \\ & f(\bm{\Lambda}_2)\end{bmatrix} \widehat{\bm{P}}\|_{\F} \leq \|f(\bm{\Lambda}_2) \bm{F}\|_{\F} \leq \|f(\bm{\Lambda}_2)\|_2^2 \|p(\bm{\Lambda}_1)^{-1}\|_2 \|p(\bm{\Lambda}_2) \bm{\Omega}_2 \bm{\Omega}_1^{\dagger}\|_{\F} \leq \\
  %  &\|p(\bm{\Lambda}_2)\bm{\Omega}_2 \bm{\Omega}_1^{\dagger}\|_{\F}\max\limits_{i=1,\ldots,k} \left|\frac{f(\lambda_i)}{p(\lambda_i)}\right|.\numberthis \label{eq:third_term}
%\end{align*}
%Inserting the bounds \eqref{eq:first_term}, \eqref{eq:second_term}, and \eqref{eq:third_term} into \eqref{eq:two_terms} yields the desired inequality. 
%\end{proof}

%To obtain probabilistic bounds we only need to obtain bounds for $\|p(\bm{\Lambda}_2) \bm{\Omega}_2 \bm{\Omega}_1^{\dagger}\|_{\F}^2$, but these are classical results in the randomized low rank approximation literature. \textcolor{red}{say something about Jensen's inequality for expectation bounds.}

%\begin{lemma}\label{lemma:constrained_best_rank_k}
%Let $\bm{Q} \in \mathbb{R}^{n \times \ell}$ be an orthormal basis. Then
%\begin{equation*}
 %   \|\bm{A}-\bm{Q}(\bm{Q}^\T \bm{A}\bm{Q})_k\bm{Q}^\T\|_{\F} = \min\limits_{\bm{C} = \bm{B}^\T, \rank(\bm{C}) \leq k}\|\bm{A} - \bm{Q} \bm{C}\bm{Q}^\T\|_{\F}^2
%\end{equation*}
%\end{lemma}
%\begin{proof}
%We have
%\begin{align*}
 %   &\|\bm{A} - \bm{Q} \bm{C}\bm{Q}^\T \|_{\F}^2 = \|\bm{A}- \bm{Q}\bm{Q}^\T \bm{A}\bm{Q}\bm{Q}^\T + \bm{Q}\bm{Q}^\T \bm{A}\bm{Q}\bm{Q}^\T - \bm{Q}\bm{C}\bm{Q}\|_{\F}^2 = \\
  %  & \|\bm{A} - \bm{Q}\bm{Q}\bm{A}\bm{Q}\bm{Q}^\T\|_{\F}^2 + \|\bm{Q}^\T \bm{A}\bm{Q} - \bm{C}\|_{\F}^2 + 2\tr((\bm{A} - \bm{Q}\bm{Q}\bm{A}\bm{Q}\bm{Q}^\T)(\bm{Q}\bm{Q}\bm{A}\bm{Q}\bm{Q}^\T - \bm{Q}\bm{C}\bm{Q}^\T))=\\
   % = &\|\bm{A} - \bm{Q}\bm{Q}\bm{A}\bm{Q}\bm{Q}^\T\|_{\F}^2 +\|\bm{Q}^\T \bm{A}\bm{Q} - \bm{C}\|_{\F}^2,
%\end{align*}
%The choice of $\bm{C}$ that minimizes this expression is $\bm{C} = (\bm{Q}^\T \bm{A} \bm{Q})_k$. 
%\end{proof}
%\begin{lemma}\label{lemma:exact_constrained}
%We have
 %   \begin{align*}
  %      &\|f(\bm{A}) - \bm{Q}_{:,1:\ell s}(\bm{Q}_{:,1:\ell s}^\T f(\bm{A}) \bm{Q}_{:,1:\ell s})_k\bm{Q}_{:,1:\ell s}^\T \|_{\F}^2 \leq \\
        %&\|f(\bm{\Lambda}_2)\|_{\F}^2 + 5 \min\limits_{p \in \mathbb{P}_{s-1}}\left[\|p(\bm{\Lambda}_2) \bm{\Omega}_2 \bm{\Omega}_1^{\dagger}\|_{\F}^2 \cdot \max\limits_{i = 1,\ldots,k} \left|\frac{f(\lambda_i)}{p(\lambda_i)}\right|^2\right].
    %\end{align*}
%\end{lemma}
%\begin{proof}
%Let $\bm{X} = p(\bm{A}) \bm{\Omega} \bm{\Omega}_1^{\dagger} p(\bm{\Lambda}_1)^{-1}$. We have
%\begin{equation*}
 %   \range(\bm{X}) \subseteq \range(p(\bm{A}) \bm{\Omega}) \subseteq \range(\bm{Q}_{:,1:\ell s}).
%\end{equation*}
%By Lemma~\ref{lemma:constrained_best_rank_k} we know 
%\begin{equation*}
 %   \|f(\bm{A}) - \bm{Q}_{:,1:\ell s}(\bm{Q}_{:,1:\ell s}^\T f(\bm{A}) \bm{Q}_{:,1:\ell s})_k\bm{Q}_{:,1:\ell s}^\T \|_{\F} \leq 
  %  \|f(\bm{A}) - \bm{P}_{\bm{X}} f(\bm{A}) \bm{P}_{\bm{X}}\|_{\F}.
%\end{equation*}
%The rest of the proof follows the proof of Lemma~\ref{lemma:structural}.
%\end{proof}

%\subsection{Krylov aware Nyström approximation}

%\begin{algorithm}
%\caption{Krylov aware Nyström approximation}
%\label{alg:krylow_nystrom}
%\textbf{input:} Symmetric $\bm{A} \in \mathbb{R}^{n \times n}$. Rank $k$. Oversampling parameter $\ell -k$. Number of iterations $q = s + r$. Positive matrix function $f: \mathbb{R} \mapsto \mathbb{R}$.\\
%\textbf{output:} Rank $k$ approximation of %$f(\bm{A})$ in factored form %$\widehat{\bm{U}} \widehat{\bm{D}} %\widehat{\bm{U}}^\T$. 
%\begin{algorithmic}[1]
 %   \State Sample a standard Gaussian $n \times \ell $ matrix $\bm{\Omega}$.
  %  \State Run Algorithm~\ref{alg:block_lanczos} to obtain an orthonormal basis $\bm{Q}_1 \in \mathbb{R}^{n \times d_s}$ for $\mathcal{K}_{s}(\bm{A},\bm{\Omega})$, an orthonormal basis $\bm{Q} = \begin{bmatrix} \bm{Q}_1 & \bm{Q}_2 \end{bmatrix} \in \mathbb{R}^{n \times d_{s+r}}$ of $\mathcal{K}_{s+r}(\bm{A},\bm{\Omega})$, and block tridiagonal matrix $\bm{T} \in \mathbb{R}^{qd_{s+r} \times qd_{s+r}}$.
   % \State Compute $\bm{X} = f(\bm{T})$. 
    %\State Compute the eigenvalue decomposition of $\bm{X}_{1:d_s, 1:d_s} = \bm{V} \bm{D} \bm{V}^\T$.
    %\State Compute $\bm{B} = \bm{Q} \bm{X} \bm{V} (\bm{D}^{1/2})^{\dagger}$.
    %\State Compute the singular value decomposition of $\bm{B} = \bm{W} \bm{S} \bm{Q}^\T$.
    %\State \textbf{return} $\widehat{\bm{U}} = \bm{Q} \bm{W}_{:,1:k}$ and $\widehat{\bm{D}} = \bm{S}^2_{1:k,1:k}$. 
%\end{algorithmic}
%\end{algorithm}

%\begin{remark}
 %   Think about Nystrom versions! Note that if $f$ is positive then $\bm{B} \succeq \bm{0}$ since $\bm{Q}_1^\T \bm{B} \bm{Q}_1^\T = \bm{Q}_1 f(\bm{T})_{1:\ell s, 1:\ell s} \bm{Q}_1^\T \succeq \bm{0}$ and if $\bm{Q}_{\bot}$ is orthogonal to $\bm{Q}_1$ we have $\bm{Q}_{\bot}^\T \bm{B} \bm{Q}_{\bot} = \bm{Q}_{\bot}^\T f(\bm{A}) \bm{Q}_{\bot} \succeq \bm{0}$. 
%\end{remark}

With \Cref{theorem:robust}, \Cref{lemma:structural}, and \Cref{lemma:probabilistic} we can now derive a probabilistic error bound for $\ALG_k(s,r;f)$, the output of \Cref{alg:krylow}. By an almost identical argument one can obtain a similar bound for $\ALG(s,r;f)$, but we omit the details. 


%Before giving our bound, we introduce the quantity
%\begin{equation}\label{eqn:eps1}
%\epsilon_1(r;f) = 4\sqrt{\ell s}  \inf\limits_{p \in \mathbb{P}_{2r+1}}\|f(x)-p(x)\|_{L^{\infty}([\lambda_{\min},\lambda_{\max}])},
%\end{equation}
%which controls the accuracy of the approximation of the quadratic form $\bm{Q}_s^\T f(\bm{A})\bm{Q}_s$.
%Note that if we know that $\|f(\bm{T}_q)_{1:d_s,1:d_s} - \bm{Q}_s^\T f(\bm{A}) \bm{Q}_s\|_{\F} \leq \epsilon$ almost surely then the $\epsilon_1(s;f)$ can be replaced with $2 \epsilon$. However, as is common with bounds for Krylov subspace methods, we opt to provide a bound which does not depend on the quantity $\bm{T}_q$ explicitly. 

\begin{theorem}\label{theorem:krylov_aware}
Consider $\bm{A} \in \mathbb{R}^{n \times n}$ as defined in \eqref{eq:A} with smallest and largest eigenvalues $\lambda_{\min}$ and $\lambda_{\max}$ respectively.
Then, with $\mathcal{E}(s;f)$ %and $\epsilon_1(r;f)$ 
as defined in \cref{eqn:min_ratio} %\cref{eqn:min_ratio,eqn:eps1}
\begin{enumerate}[(i)]
        \item with probability at least $1-e^{-(u-2)/4} - \sqrt{\pi r} \left(\frac{t}{e}\right)^{-(\ell -k + 1)/2}$, %simultaneously for all functions $f: \mathbb{R} \mapsto \mathbb{R}$,
        \begin{align*}
        \|f(\bm{A}) - \ALG_k(s,r;f) \|_{\F} \leq &4\sqrt{\ell s}  \inf\limits_{p \in \mathbb{P}_{2r+1}}\|f(x)-p(x)\|_{L^{\infty}([\lambda_{\min},\lambda_{\max}])}+\\%\epsilon_1(r;f) +
        &\sqrt{\|f(\bm{\Lambda}_{n \setminus k })\|_{\F}^2 +  \frac{5ut k}{\ell - k + 1} \mathcal{E}(s;f)};
        \end{align*}\label{item:krylov_aware_tailbound}
        \item if $\ell -k \geq 2$ that
        \begin{align*}
        \mathbb{E}\|f(\bm{A}) - \ALG_k(s,r;f) \|_{\F} \leq &4\sqrt{\ell s}  \inf\limits_{p \in \mathbb{P}_{2r+1}}\|f(x)-p(x)\|_{L^{\infty}([\lambda_{\min},\lambda_{\max}])}+\\%\epsilon_1(r;f) +
        &\sqrt{\|f(\bm{\Lambda}_{n \setminus k })\|_{\F}^2 +  \frac{5k}{\ell - k - 1} \mathcal{E}(s;f)}.        \end{align*}\label{item:krylov_aware_expectationbound}
    \end{enumerate}
\end{theorem}

% \begin{theorem}\label{theorem:krylov_aware}
% Consider $\bm{A} \in \mathbb{R}^{n \times n}$ as defined in \eqref{eq:A} and fix a function . Let $\bm{Q}_s$ and $\bm{T}_q$ be as in \cref{alg:krylow}. Then, for all functions $f: \mathbb{R} \mapsto \mathbb{R}$ we have
% \begin{enumerate}[(i)]
%         \item with probability at least $1-e^{-(u-2)/4} - \sqrt{\pi r} \left(\frac{t}{e}\right)^{-(\ell -k + 1)/2}$ that
%         \begin{align*}
%         &\|f(\bm{A}) - \bm{Q}_s \llbracket f(\bm{T}_q)_{1:d_s,1:d_s}\rrbracket_{k} \bm{Q}_s^\T\|_{\F} \leq 4\sqrt{d_s} \inf\limits_{p \in \mathbb{P}_{2r+1}}\|f(x)-p(x)\|_{L^{\infty}([\lambda_{\min},\lambda_{\max}])} +\\
%         &\sqrt{\|f(\bm{\Lambda}_{n \setminus k })\|_{\F}^2 +  \frac{5ut k}{\ell - k + 1} \min\limits_{p \in \mathbb{P}_{s-1}} \left[\|p(\bm{\Lambda}_{n \setminus k})\|_{\F}^2\max\limits_{i=1,\ldots,k} \left|\frac{f(\lambda_i)}{p(\lambda_i)}\right|^2 \right]};
%         \end{align*}\label{item:krylov_aware_tailbound}
%         \item if $\ell -k \geq 2$ that
%         \begin{align*}
%         &\mathbb{E}\|f(\bm{A}) - \bm{Q}_s \llbracket f(\bm{T}_q)_{1:d_s,1:d_s}\rrbracket_{k} \bm{Q}_s^\T\|_{\F} \leq 4\sqrt{d_s} \inf\limits_{p \in \mathbb{P}_{2r+1}}\|f(x)-p(x)\|_{L^{\infty}([\lambda_{\min},\lambda_{\max}])} +\\
%         &\sqrt{\|f(\bm{\Lambda}_{n \setminus k })\|_{\F}^2 +  \frac{5k}{\ell - k + 1} \min\limits_{p \in \mathbb{P}_{s-1}} \left[\|p(\bm{\Lambda}_{n \setminus k})\|_{\F}^2\max\limits_{i=1,\ldots,k} \left|\frac{f(\lambda_i)}{p(\lambda_i)}\right|^2 \right]}.        \end{align*}\label{item:krylov_aware_expectationbound}
%     \end{enumerate}
% \end{theorem}
\begin{proof}
\textit{(\ref*{item:krylov_aware_tailbound})}: By applying \Cref{theorem:robust}, \Cref{lemma:structural}, and \Cref{lemma:2_times_polynomial_approx} we obtain the following structural bound %\David{I changed some stuff here. I used the notation that Tyler defined.}  
\begin{align}
\begin{split}
        \|f(\bm{A}) - \bm{Q}_s \llbracket f(\bm{T}_q)_{1:d_{s,\ell},1:d_{s,\ell}}\rrbracket_{k} \bm{Q}_s^\T\|_{\F} \leq &4\sqrt{\ell s}  \inf\limits_{p \in \mathbb{P}_{2r+1}}\|f(x)-p(x)\|_{L^{\infty}([\lambda_{\min},\lambda_{\max}])} +\\
        &\sqrt{\|f(\bm{\Lambda}_{n \setminus k })\|_{\F}^2 +  5 \mathcal{E}_{\bm{\Omega}}(s;f)}.
        \end{split}\label{eq:krylov_aware_structural}
\end{align}
Applying \Cref{lemma:probabilistic} yields \textit{(\ref*{item:krylov_aware_tailbound})} directly. 
\textit{(\ref*{item:krylov_aware_expectationbound})} follows from \Cref{lemma:probabilistic} and an application of Jensen's inequality.
\end{proof}

We conclude this section by commenting on the three terms appearing in the bounds in \Cref{theorem:krylov_aware}. 
The dependence on $4\sqrt{\ell s}  \inf\limits_{p \in \mathbb{P}_{2r+1}}\|f(x)-p(x)\|_{L^{\infty}([\lambda_{\min},\lambda_{\max}])}$ captures the inherent cost of approximating a quadratic form with $f(\bm{A})$ using the Lanczos method, as in done in Line 3 of \Cref{alg:krylow}. A similar term would also arise in an analysis of the Randomized SVD when matvecs are computed approximately using a Krylov subspace method.
The $\|f(\bm{\Lambda}_{n \setminus k})\|_{\F}$ term is due to the fact that our error can never be below the optimal rank $k$ approximation error. Finally, $\mathcal{E}(s;f)$ tells us that $\bm{Q}_s$ is a good orthonormal basis for low-rank approximation if there is a polynomial of degree at most $s-1$ that is large on the eigenvalues $\lambda_1,\ldots,\lambda_k$ (which correspond to the top subspace of $f(\bm{A})$) and is small on the eigenvalues $\lambda_{k+1},\ldots,\lambda_n$. 

As discussed in \Cref{section:structural}, one possible choice of polynomial would be an approximation to $f$, in which case $\mathcal{E}(s;f)$ would be close to $\|f(\bm{\Lambda}_{n \setminus k})\|_{\F}$. In this case, ignoring the polynomial approximation term, we would recover a bound almost identical to the standard Randomized SVD error bound. In particular, when matvecs with $f(\bm{A})$ are implemented exactly, Randomized SVD can be shown to have expected error \cite{rsvd}:
\begin{align*}
\sqrt{\|f(\bm{\Lambda}_{n \setminus k})\|_{\F}^2 + \frac{k}{\ell-k-1}\|f(\bm{\Lambda}_{n \setminus k})\|_{\F}^2}.
\end{align*}
However, other possibly polynomials can also be chosen, so we might have $\mathcal{E}(s;f)  \ll \|f(\bm{\Lambda}_{n \setminus k})\|_{\F}$. In the next section, we give an example involving the matrix exponential to better illustrate why this is often the case. 

% Such a polynomial effectively denoises the contribution from the small eigenvalues of $f(\bm{A})$. A similar intuition underlies standard analyses of randomized Krylov subspace methods for low-rank approximation of a simple matrix $\bm{A}$ \cite{MM15,tropp2023randomized}. 

% Note that if we know that $\|f(\bm{T}_q)_{1:d_{s,\ell},1:d_{s,\ell}} - \bm{Q}_s^\T f(\bm{A}) \bm{Q}_s\|_{\F} \leq \epsilon$ almost surely then this term can be replaced with $2 \epsilon$.

% {\color{blue}\textbf{Tyler: let me try again}}

% In some applications we care not just about a single function, but about an infinite family of functions.
% For example, in quantum thermodynamics, the functions $f(x) = \exp(-\beta x)$, for a range of $\beta>0$ are of interest; see for example \cref{section:quantum_spin}.
% As a corollary to the previous proof, we also obtain \emph{uniform bounds} for certain classes of functions, including monotonically decreasing functions.
% \begin{corollary}
% Consider $\bm{A} \in \mathbb{R}^{n \times n}$ as defined in \eqref{eq:A}.
% Fix a susbset $\Lambda_k$ of $k$ eigenvalues of $\bm{A}$ (repeated eigenvalues treated as distinct elements) and let $\Lambda_{n\setminus k} = \Lambda \setminus \Lambda_k$.
% Define $\mathcal{F}$ as the set of functions $f:\mathbb{R}\to\mathbb{R}$ such that $\forall \lambda\in\Lambda_k, \lambda'\in\Lambda_{n\setminus k}: f(\lambda) \geq f(\lambda')$/ 
% Then, with $\mathcal{E}(s;f)$ and $\epsilon_1(r;f)$ as defined in 
%     with probability at least $1-e^{-(u-2)/4} - \sqrt{\pi r} \left(\frac{t}{e}\right)^{-(\ell -k + 1)/2}$, simultaneously for all functions $f\in\mathcal{F}$,
%         \begin{align*}
%         &\|f(\bm{A}) - \ALG_k(s,r;f) \|_{\F} \leq \epsilon_1(r;f) +
%         \sqrt{\|f(\bm{\Lambda}_{n \setminus k })\|_{\F}^2 +  \frac{5ut k}{\ell - k + 1} \mathcal{E}(s;f)}.
%         \end{align*}
% \end{corollary}

% \begin{proof}
%     For all functions, the structural bound is the same as the preceding proof.
%     By construction, $\bm{\Omega}_k$ and $\bm{\Omega}_{n\setminus k}$ are same for all functions $f\in\mathcal{F}$, so we only need to apply \Cref{lemma:probabilistic}~\textit{(\ref*{item:tailbound})} once.
% \end{proof}
%{\color{blue}{


% -------- MATRIX EXPONENTIAL BOUNDS HERE --------


\subsection{Example bounds for the matrix exponential}\label{section:exponential}
One matrix function for which it is often desirable to obtain a low-rank approximation is the matrix exponential, $\exp(\bm{A})$. This task arises in tasks ranging from network analysis \cite{hpp}, to quantum thermodynamics \cite{chen_hallman_23,epperly2023xtrace}, to solving PDEs \cite{persson_kressner_23}. In this section, we do a deeper dive into our main bounds for Krylov aware low-rank approximation when applied to the matrix exponential to better understand how they compare to bounds obtainable using the more naive approach of directly combining the randomized SVD with a black-box matrix-vector product approximation algorithm for $\exp(\bm{A})$.  For conciseness of exposition, we focus on expectation bounds, and defer details of proofs to \Cref{section:appendix}.

We first demonstrates that, as discussed in the previous section, by choosing $p$ in $\mathcal{E}(s;\exp(x))$ to be a polynomial approximation to $\exp(x)$, we can recover the bounds of the randomized SVD \cite[Theorem 10.5]{rsvd}, up to small factors accounting for the fact that $\exp(x)$ is not a polynomial. %The proof is deferred to \Cref{section:appendix}.



% By constructing particular polynomials of degree $<s$, we can obtain more explicit bounds that depend only on how accurately $f(x)$ can be approximated by polynomials. 
% These bounds are reminiscent of standard bounds that might be obtained if we could do exact matvecs with $f(\bm{A})$, except that they have small error terms accounting for the fact that $f(x)$ might not be a polynomial. We will provide such bounds for $\exp(\bm{A})$. For simplicity, we focus on expectation bounds. However, using an almost identical argument, one can obtain the corresponding tailbounds.

% We begin with the following result, which shows that by \Cref{theorem:krylov_aware} can recover the bounds of the randomized SVD \cite[Theorem 10.5]{rsvd}. The proof of deferred to the appendix for clarity of exposition. 

\begin{corollary}\label{theorem:rsvd_like_bound}
Consider the setting of \Cref{theorem:krylov_aware} with $f(x) = \exp(x)$. Then, if $\ell-k\geq 2$ and $\gamma_{1,n} := \lambda_{\max} - \lambda_{\min} = \lambda_1-\lambda_n$, and $s \geq e\gamma_{1,n}$ we have
\begin{align*}
     &\mathbb{E}\|\exp(\bm{A}) - \ALG_k(s,r;\exp(x))\|_{\F} 
     \\&\hspace{2em}
     \leq  \frac{\sqrt{\ell s}\gamma_{1,n}^{2r+2}}{2^{4r+1}(2r+2)!}\|\exp(\bm{A})\|_2 + \sqrt{1 +  \frac{1}{(1-\frac{\gamma_{1,n}^s}{s!})^2}\frac{5k}{\ell - k - 1}} \|\exp(\bm{\Lambda}_{n \setminus k })\|_{\F}.
\end{align*}
\end{corollary}
%\tyler{if we're doing a specific example idk if we need the previous one as well?}
Note that as $\frac{\sqrt{\ell s}\gamma_{1,n}^{2r+2}}{2^{4r+1}(2r+2)!}\|\exp(\bm{A})\|_2 \to 0$ and $\frac{\gamma_{1,n}^s}{s!} \to 0$, we recover the classical bound for the randomized SVD \cite[Theorem 10.5]{rsvd}. Furthermore, these two terms converge \emph{superexponentially} to $0$ as $s,r \to +\infty$. However, due to the $\frac{5k}{\ell -k -1}$, the above bound only implies a subexponential convergence to the error of the optimal rank $k$ approximation error, $\|\exp(\bm{\Lambda}_{n \setminus k })\|_{\F}$, in terms of the number of required matrix-vector products with $\bm{A}$, which scales as $\ell (r+s)$.

% To prove \Cref{theorem:rsvd_like_bound} we chose $p$ in $\mathcal{E}(s;\exp(x))$ to be a polynomial approximation to $\exp(x)$. 
However, as argued in \Cref{section:structural} there are other polynomials that can achieve a significantly tighter upper bound of $\mathcal{E}(s;\exp(x))$. For example, by choosing $p$ in $\mathcal{E}(s;\exp(x))$ to be a scaled and shifted Chebyshev polynomial we can show the output of \Cref{alg:krylow} converges \emph{exponentially} in $s$ to the optimal low-rank approximation of $\exp(\bm{A})$, even if $\ell$ stays fixed. 
 This demonstrates why \Cref{alg:krylow} is expected to return a better low-rank approximation compared to applying the randomized SVD immediately on $\exp(\bm{A})$. In particular, we have the following result. 

\begin{corollary}\label{theorem:fast_convergence}
Consider the setting of \Cref{theorem:krylov_aware} with $f(x) = \exp(x)$. Define $\gamma_{i,j} = \lambda_{i} - \lambda_{j}$ to be the gap between the $i^{\text{th}}$ and $j^{\text{th}}$ largest eigenvalues. Then, if $\ell-k\geq 2$ and $s \geq 2 + \frac{\gamma_{1,k}}{\log\left(\frac{\gamma_{1,n}}{\gamma_{k,n}}\right)}$%$\gamma = \lambda_{\max} - \lambda_{\min} = \lambda_1 - \lambda_n$, and $\beta = \frac{\lambda_{k} - \lambda_{k+1}}{\lambda_{k} + \lambda_{k+1} - 2\lambda_{\min}}$ we have
\begin{align*}
     &\mathbb{E}\|\exp(\bm{A}) - \ALG_k(s,r;\exp(x))\|_{\F} 
     \leq  \frac{\sqrt{\ell s}\gamma_{1,n}^{2r+2}}{2^{4r+1}(2r+2)!}\|\exp(\bm{A})\|_2 
     \\&\hspace{5em}+ \sqrt{1 +  \exp(2\gamma_{k,n}) \cdot 2^{-2(s-2)\sqrt{\min\left\{1, 2 \frac{\gamma_{k,k+1}}{\gamma_{k+1,n}}\right\}}}\frac{80k}{\ell - k - 1}} \|\exp(\bm{\Lambda}_{n \setminus k })\|_{\F}.
\end{align*}
\end{corollary}
% \begin{proof}
%     Bounding $4\sqrt{\ell s}\inf\limits_{p \in \mathbb{P}_{2r+1}}\|\exp(x)-p(x)\|_{L^{\infty}([\lambda_{\min},\lambda_{\max}])}$ is done identical to as done in \Cref{theorem:rsvd_like_bound}. We proceed with bounding $\mathcal{E}(s;\exp(x))$ by choosing the polynomial $p(x) = (1+x-\lambda_{\min})T_{s-2}\left(\frac{x-\lambda_{\min}}{\lambda_{k+1}-\lambda_{\min}}\right)$, where $T_{s-2}$ is the Chebyshev polynomial of degree $s-2$. Hence,
%     \begin{align*}
%         \mathcal{E}(s;\exp(x)) 
%         &\leq \|p(\bm{\Lambda}_{n \setminus k})\|_{\F}^2 \max\limits_{i=1,\ldots,k} \left|\frac{\exp(\lambda_i)}{p(\lambda_i)}\right|^2 
%         \\&\leq 
%         \|\exp(\bm{\Lambda}_{n \setminus k} - \lambda_{\min}\bm{I})\|_{\F}^2 \left\|T_{s-2}\left(\frac{\bm{\Lambda}_{n \setminus k}-\lambda_{\min}\bm{I}}{\lambda_{k+1}-\lambda_{\min}}\right)\right\|^2 \exp(2\lambda_{\min}) \max\limits_{i=1,\ldots,k} \left|\frac{\exp(\lambda_i-\lambda_{\min})}{p(\lambda_i)}\right|^2 
%         \\&\leq 
%         \|\exp(\bm{\Lambda}_{n \setminus k})\|_{\F}^2\max\limits_{i=1,\ldots,k} \left|\frac{\exp(\lambda_i-\lambda_{\min})}{p(\lambda_i)}\right|^2 
%         \\&\leq 
%         \|\exp(\bm{\Lambda}_{n \setminus k})\|_{\F}^2\max\limits_{i=1,\ldots,k} \left|\frac{\exp(\lambda_i-\lambda_{\min})}{T_{s-2}\left(\frac{x-\lambda_{\min}}{\lambda_{k+1}-\lambda_{\min}}\right)}\right|^2 
%         \\&\leq 
%         \|\exp(\bm{\Lambda}_{n \setminus k})\|_{\F}^2 \frac{e^{2\gamma}}{T_{s-2}\left(\frac{\lambda_k-\lambda_{\min}}{\lambda_{k+1}-\lambda_{\min}}\right)^2} 
%         \\&\leq 
%         4\|\exp(\bm{\Lambda}_{n \setminus k})\|_{\F}^2 e^{2\gamma-4(s-2)\sqrt{\gamma}},
%     \end{align*}
%     where the second inequality uses $0\leq 1+x \leq e^x$ for $x \geq 0$, the third inequality uses $|T_{s-2}(x)| \leq 1$ for $x \in [0,1]$, the fourth inequality uses $\frac{e^x}{1+x} \leq e^x$ for $x \geq 0$, the fifth inequality uses that $T_{s-2}(x)$ is increasing for $x \geq 1$, and the final inequality uses \cite[Lemma 9.3]{tropp2023randomized}. Plugging this inequality into \Cref{theorem:krylov_aware} yields the desired inequality. 
%     \end{proof}


%     \tyler{the inequalities above are very long. how about this which emphasizes what is changing each line so the lines are shorter? Might want to reorder it a bit?}\hrulefill

%     \begin{proof}
%     Bounding $4\sqrt{\ell s}\inf\limits_{p \in \mathbb{P}_{2r+1}}\|\exp(x)-p(x)\|_{L^{\infty}([\lambda_{\min},\lambda_{\max}])}$ is done identical to as done in \Cref{theorem:rsvd_like_bound}. We proceed with bounding $\mathcal{E}(s;\exp(x))$. Define $\gamma(x) = \frac{x-\lambda_{k+1}}{\lambda_{k+1} - \lambda_n}$. Define the polynomial $p(x) = (1+x-\lambda_{n})T_{s-2}\left(1 + 2 \gamma(x)\right)$, where $T_{s-2}$ is the Chebyshev polynomial of degree $s-2$. 
%     Hence, recalling the definition \cref{eqn:min_ratio} of $\mathcal{E}(s;\exp(x))$, since $0\leq 1+x \leq e^x$ for $x\geq 0$ and $|T_{s-2}(x)| \leq 1$ for $x \in [-1,1]$ we have,
%     \[
%     \|p(\bm{\Lambda}_{n \setminus k})\|_{\F}
%     \leq \|\exp(\bm{\Lambda}_{n \setminus k} - \lambda_{n}\bm{I})\|_{\F}\]
%     Hence, using that $\frac{e^x}{1+x} \leq e^x$ for $x \geq 0$ we get
%     \begin{align}
%         \mathcal{E}(s;\exp(x)) 
%         &\leq \|p(\bm{\Lambda}_{n \setminus k})\|_{\F}^2 \max\limits_{i=1,\ldots,k} \left|\frac{e^{\lambda_i}}{p(\lambda_i)}\right|^2 
%         \nonumber\\&\leq 
%         \|\exp(\bm{\Lambda}_{n \setminus k}-\lambda_{n}\bm{I})\|_{\F}^2\max\limits_{i=1,\ldots,k} e^{2\lambda_{n}} \left|\frac{e^{\lambda_i-\lambda_{n}}}{p(\lambda_i)}\right|^2 
%         \nonumber\\&=\|\exp(\bm{\Lambda}_{n \setminus k})\|_{\F}^2\max\limits_{i=1,\ldots,k} \left|\frac{e^{\lambda_i-\lambda_{n}}}{p(\lambda_i)}\right|^2
%         \nonumber\\&\leq\|\exp(\bm{\Lambda}_{n \setminus k})\|_{\F}^2\max\limits_{i=1,\ldots,k} \left|\frac{e^{\lambda_i-\lambda_{n}}}{T_{s-2}\left(1 + 2 \gamma(\lambda_i)\right)}\right|^2.
%         \label{eqn:Esexp_bd}
%     \end{align}
%     First note that for $|x| \geq 1$ we have
%     \begin{equation*}
%         T_{s-2}(x) = \frac{1}{2}\left(\left(x + \sqrt{x^2-1}\right)^{s-2} + \left(x - \sqrt{x^2-1}\right)^{s-2} \right).
%     \end{equation*}
%     Hence, for $\gamma \geq 0$ we have $|T_{s-2}(1+2\gamma)| \geq \frac{1}{2}(1+2\gamma + 2\sqrt{\gamma + \gamma^2})^{s-2} =: \frac{1}{2}h(\gamma)^{s-2}$. Therefore,
%     \begin{align*}
%         \max\limits_{i=1,\ldots,k} \left|\frac{e^{\lambda_i-\lambda_{n}}}{T_{s-2}\left(1 + 2 \gamma(\lambda_i)\right)}\right|^2 &\leq 4 \max\limits_{i=1,\ldots,k} \left|\frac{e^{\lambda_i-\lambda_{n}}}{h(\gamma(\lambda_i))^{s-2}}\right|^2\\
%         &= 4 \max\limits_{i=1,\ldots,k} \left(e^{\lambda_i - \lambda_n - (s-2)\log(h(\gamma(\lambda_i)))}\right)^2.
%     \end{align*}
%     Note that the function 
%     \begin{equation*}
%         g(x) := x - \lambda_n - (s-2)\log(h(\gamma(\lambda_i)))
%     \end{equation*}
%     is convex in $[\lambda_k,\lambda_1]$. Hence, the maximum of $g$ on the interval $[\lambda_k,\lambda_1]$ is attained at either $x = \lambda_k$ or $x = \lambda_1$. The maximum of $g$ is attained at $x = \lambda_k$ whenever $g(\lambda_1) \leq g(\lambda_k)$ which happens whenever
%     \begin{equation}\label{eq:scondition}
%         \frac{\lambda_1 - \lambda_k}{\log\left(\frac{h(\gamma(\lambda_1))}{h(\gamma(\lambda_k))}\right)} + 2 \leq s.
%     \end{equation}
%     Note that we have whenever $x \geq y \geq 0$ we have $\frac{h(x)}{h(y)} \geq \frac{1 + x}{1+y}$. Hence, \eqref{eq:scondition} holds whenever
%     \begin{equation*}
%         \frac{\lambda_1 - \lambda_k}{\log\left(\frac{\lambda_1 - \lambda_{k+1}}{\lambda_k - \lambda_{k+1}}\right)} + 2  \leq s,
%     \end{equation*}
%     which is our assumption on $s$. Hence, 
%     \begin{equation}\label{eq:upperbound}
%          \max\limits_{i=1,\ldots,k} \left|\frac{e^{\lambda_i-\lambda_{n}}}{T_{s-2}\left(1 + 2 \gamma(\lambda_i)\right)}\right|^2 \leq 4 \frac{e^{2(\lambda_k - \lambda_n)}}{h(\gamma(\lambda_k))^{2(s-2)}}.
%     \end{equation}
%     Now we use the argument from \cite[p.21]{MM15}, which shows that $h(\gamma) \geq 2^{\sqrt{\min\{1,2\gamma\}}q}$. Plugging this inequality into \cref{eq:upperbound} and then \cref{eqn:Esexp_bd} into \cref{theorem:krylov_aware} yields the desired inequality. 
%     % Bounding $4\sqrt{\ell s}\inf\limits_{p \in \mathbb{P}_{2r+1}}\|\exp(x)-p(x)\|_{L^{\infty}([\lambda_{\min},\lambda_{\max}])}$ is done identical to as done in \Cref{theorem:rsvd_like_bound}. We proceed with bounding $\mathcal{E}(s;\exp(x))$ by choosing the polynomial $p(x) = (1+x-\lambda_{\min})T_{s-2}\left(\frac{x-\lambda_{\min}}{\lambda_{k+1}-\lambda_{\min}}\right)$, where $T_{s-2}$ is the Chebyshev polynomial of degree $s-2$. 
%     % Hence, recalling the definition \cref{eqn:min_ratio} of $\mathcal{E}(s;\exp(x))$, since $0\leq 1+x \leq e^x$ for $x\geq 0$,
%     % \[
%     % \|p(\bm{\Lambda}_{n \setminus k})\|_{\F}
%     % \leq \|\exp(\bm{\Lambda}_{n \setminus k} - \lambda_{\min}\bm{I})\|_{\F} \left\|T_{s-2}\left(\frac{\bm{\Lambda}_{n \setminus k}-\lambda_{\min}\bm{I}}{\lambda_{k+1}-\lambda_{\min}}\right)\right\|^2.
%     % \]
%     % Hence, using that $|T_{s-2}(x)\leq 1$ for $x\in[0,1]$,
%     % \begin{align}
%     %     \mathcal{E}(s;\exp(x)) 
%     %     &\leq \|p(\bm{\Lambda}_{n \setminus k})\|_{\F}^2 \max\limits_{i=1,\ldots,k} \left|\frac{e^{\lambda_i}}{p(\lambda_i)}\right|^2 
%     %     \nonumber\\&\leq 
%     %     \|\exp(\bm{\Lambda}_{n \setminus k}-\lambda_{\min}\bm{I})\|_{\F}^2\max\limits_{i=1,\ldots,k} e^{2\lambda_{\min}} \left|\frac{e^{\lambda_i-\lambda_{\min}}}{p(\lambda_i)}\right|^2 
%     %     \nonumber\\&=\|\exp(\bm{\Lambda}_{n \setminus k})\|_{\F}^2\max\limits_{i=1,\ldots,k} \left|\frac{e^{\lambda_i-\lambda_{\min}}}{p(\lambda_i)}\right|^2.
%     %     \label{eqn:Esexp_bd}
%     % \end{align}
%     % Finally, using that $\frac{e^x}{1+x} \leq e^x$ for $x \geq 0$,that $T_{s-2}(x)$ is increasing for $x \geq 1$, and \cite[Lemma 9.3]{tropp2023randomized} we have
%     % \begin{align*}
%     %     \max\limits_{i=1,\ldots,k} \left|\frac{e^{\lambda_i-\lambda_{\min}}}{p(\lambda_i)}\right|^2 
%     %     &\leq 
%     %     \max\limits_{i=1,\ldots,k} \left|\frac{e^{\lambda_i-\lambda_{\min}}}{T_{s-2}\big(\frac{x-\lambda_{\min}}{\lambda_{k+1}-\lambda_{\min}}\big)}\right|^2 
%     %     \\&\hspace{5em}\leq \frac{e^{2\gamma}}{T_{s-2}\big(\frac{\lambda_k-\lambda_{\min}}{\lambda_{k+1}-\lambda_{\min}}\big)^2} 
%     %     \leq 
%     %     4 e^{2\gamma-4(s-2)\sqrt{\gamma}},
%     % \end{align*}
%     % Plugging this inequality into \cref{eqn:Esexp_bd} and then \cref{eqn:Esexp_bd} into \cref{theorem:krylov_aware} yields the desired inequality. 
% \end{proof}
%}}

% {\color{red}{
% \subsubsection{Simplified bounds}
% By constructing particular polynomials of degree $<s$, we can obtain more explicit bounds that depend only on how accurately $f(x)$ can be approximated by polynomials. 
% These bounds are reminiscent of standard bounds that might be obtained if we could do exact products with $f(\bm{A})$, except that they have small error terms accounting for the fact that $f(x)$ might not be a polynomial. 
% For example, by taking $p(x)$ as the best approximation to $f(x)$, shifted up to be above $f(x)$, we can obtain a bound like what one might obtain the the randomized SVD. For simplicity, we focus on expectation bounds. However, using an almost identical argument, one can obtain the corresponding tailbounds.

% We will use the quantity
% \begin{equation}\label{eqn:eps2}
% \epsilon_2(s;f) = \min_{p\in\mathbb{P}_{s-1}} \|f(x) - p(x)\|_{L^{\infty}(\Lambda)} 
% \end{equation}
% which relates to how well $f$ can be approximated by polynomials on the \emph{eigenvalues} $\Lambda$ of $\bm{A}$. Such bounds can often be much smaller than the corresponding bounds on an interval containing the eigenvalues.


% \begin{corollary}\label{theorem:rsvd_like_bound}
% Consider the setting of \Cref{theorem:krylov_aware}. Suppose $f(\lambda_i) \geq 0$ for all $i=1, \dots, n$ and define $\Lambda = \{\lambda_1,\ldots,\lambda_n\}$.
% Then, if $\ell-k\geq 2$, with $\epsilon_1(r;f)$ and $\epsilon_2(s;f)$ as defined in \cref{eqn:eps1,eqn:eps2}, we have
% \begin{align*}
%      \mathbb{E}\|f(\bm{A}) - \ALG_k(s,r;f)\|_{\F} 
%      &\leq 
%      \epsilon_1(r;f) + 2\sqrt{\frac{5nk}{\ell - k + 1}} \epsilon_2(s;f)
%      \\&\hspace{6em} + \sqrt{1 +  \frac{5k}{\ell - k + 1}} \|f(\bm{\Lambda}_{n \setminus k })\|_{\F}.
% \end{align*}
% \end{corollary}

% A similar approach allows us to obtain bounds reminiscent of what one might obtain for the randomized block Krylov algorithm with exact products with $f(\bm{A})$.
% If $s$ is chosen sufficiently large and there is a low-degree polynomial that can accurately find the gap between $f(\lambda_k)$ and $f(\lambda_{k+1})$, the \Cref{alg:krylow} satisfies bounds that are similar to \cite[Theorem 9.2]{tropp2023randomized}. 

% % define
% % \[
% % \epsilon_3 = \min_{p\in\mathbb{P}_{\lfloor (s-1)/q \rfloor}} \|f(x) - p(x)\|_{L^{\infty}(\Lambda)}.
% % \]
% %{\color{blue} TODO: double check the statement, the epsilon 2 was missing in the middle and I just put it back in without checking.}\David{We used $q = s+r$ before. I replaced $q$ with $m$.}
% \begin{corollary}
% \label{theorem:rbki_like_bound}
% Consider the setting of \Cref{theorem:rsvd_like_bound}.
% Fix $m \geq 1$ and with $\epsilon_1(r;f)$ and $\epsilon_2(\lfloor (s-1)/m \rfloor;f)$ as defined in \cref{eqn:eps1,eqn:eps2}, define a gap parameter
% \[
% \gamma_{\epsilon}
% = \frac{f(\lambda_k) - (f(\lambda_{k+1})+2\epsilon_2(\lceil (s-1)/m \rceil;f))}{f(\lambda_k) + f(\lambda_{k+1})+2\epsilon_2(\lceil (s-1)/m \rceil;f)},
% \]
% and assume that $\gamma_{\epsilon} > 0$. 
% Then, if $\ell-k\geq 2$, we have
% \begin{align*}
%     \mathbb{E}\|f(\bm{A}) - \ALG_k(s,r;f)\|_{\F} &\leq 
%     \epsilon_1(r;f) + 4 \sqrt{\frac{5nk}{\ell-k+1}}e^{-2(m-1)\sqrt{\gamma_{\epsilon}}} \epsilon_2(\lceil (s-1)/m \rceil;f) 
%     \\&\hspace{6em}+ \sqrt{1+ \frac{20k}{\ell - k + 1} e^{-4(m-1) \sqrt{\gamma_\epsilon}}} \|f(\bm{\Lambda}_{n \setminus k })\|_{\F}.
% \end{align*}
% \end{corollary}

% Note that when $f(x) = x$ we have $\epsilon_1 = \epsilon_2 = 0$ and \Cref{theorem:rbki_like_bound} recovers \cite[Theorem 9.2]{tropp2023randomized}.
% Furthermore, it is worth pointing out that in the special case where For $f(x) = 1/x$ and $\bm{A}$ is positive definite, the optimality of conjugate gradient allows $\epsilon_1$ to also be upgraded to a bound on the eigenvalues of $\bm{A}$. 
% We now proceed with proving \Cref{theorem:rsvd_like_bound,theorem:rbki_like_bound}. 

% \begin{proof}[Proof of \cref{theorem:rsvd_like_bound}]
% Let $\hat{p}(x)$ be a degree $s-1$ polynomial such that $\|f(x) - \hat{p}(x)\|_{L^{\infty}(\Lambda)} = \epsilon_2(s;f)$, and define $p(x) = \hat{p}(x) + \epsilon_2(s;f)$.
% Note that for any $i=1,\ldots, k$, $p(\lambda_i) \leq f(\lambda_i) + 2\epsilon_2(s;f)$ so
% \[
% \|p(\bm{\Lambda}_{n \setminus k})\|_{\F}^2
% \leq \|f(\bm{\Lambda}_{n \setminus k}) + 2\epsilon_2(s;f) \bm{I}\|_{\F}^2.
% \]
% In addition, since $f(\lambda_i) \leq p(\lambda_i)$,
% \[
% \max\limits_{i=1,\ldots,k} \left|\frac{f(\lambda_i)}{p(\lambda_i)}\right|^2
% \leq 1.
% \]
% Combining these results gives the specified bound for a particular degree $s-1$ polynomial $p(x)$ and thus 
% \[
% \min\limits_{p \in \mathbb{P}_{s-1}} \left[\|p(\bm{\Lambda}_{n \setminus k})\|_{\F}^2\max\limits_{i=1,\ldots,k} \left|\frac{f(\lambda_i)}{p(\lambda_i)}\right|^2 \right]
% \leq  \|f(\bm{\Lambda}_{n \setminus k}) + 2\epsilon_2(s;f) \bm{I}\|_{\F}^2.
% \]
% Combining this inequality with \cref{theorem:krylov_aware} and applying the triangle inequality gives the result.
% \end{proof}

% \begin{proof}[Proof of \cref{theorem:rbki_like_bound}]
% Let $\hat{p}(x)$ be a degree $\lfloor (s-1)/m \rfloor = \lceil (s-1)/m \rceil-1$ polynomial such that $\|f(x) - \hat{p}(x)\|_{L^{\infty}(\Lambda)} =\epsilon_2(\lceil (s-1)/m \rceil) :=\epsilon_2$ and define $\tilde{p}(x) = \hat{p}(x) + \epsilon_2$.
% Set $p(x) = \tilde{p}(x)\tilde{T}_{m-1}(\tilde{p}(x))$, where $\tilde{T}_{m-1}(x) = T_{m-1}(x / (f(\lambda_{k+1})+2\epsilon_2))$ is the degree $m-1$ scaled Chebyshev polynomial. Note that for $i=k+1, \ldots, n$, since $\tilde{p}(\lambda_i) \in [0,f(\lambda_{k+1}) + 2\epsilon_2]$ we have by \cite[Chapter 2]{chebyshevpolynomials} that $|\tilde{T}_{m-1}(\tilde{p}(\lambda_{i}))| \leq 1$.
% Thus, using that $p(\lambda_i) \leq f(\lambda_i) + 2\epsilon_2$,
% \[
% \|p(\bm{\Lambda}_{n \setminus k})\|_{\F}^2 
% \leq \|\tilde{p}(\bm{\Lambda}_{n \setminus k}) \|_{\F}^2 \|\tilde{T}_{m-1}(\tilde{p}(\bm{\Lambda}_{n \setminus k}))\|_{2}^2
% \leq \|f(\bm{\Lambda}_{n \setminus k}) + 2\epsilon_2\bm{I}\|_{\F}^2.
% \]
% Now, fix $i=1,\ldots, k$.
% Since $\gamma_{\epsilon}>0$ we have $f(\lambda_i) \geq f(\lambda_{k+1}) + 2\epsilon_2$.
% Thus, since $f(\lambda_i)\leq \tilde{p}(\lambda_i)$ and that $T_{m-1}(x)$ is monotonically increasing for $x \geq 1$ we have that $p(\lambda_i) = \tilde{p}(\lambda_{i})\tilde{T}_{m-1}(\tilde{p}(\lambda_{i})) \geq f(\lambda_{i})\tilde{T}_{m-1}(f(\lambda_{i}))$.
% This implies that
% \[
% \max\limits_{i=1,\ldots,k} \left|\frac{f(\lambda_i)}{p(\lambda_i)}\right|^2
% \leq \max\limits_{i=1,\ldots,k} \left|\frac{f(\lambda_i)}{\tilde{T}_m(f(\lambda_i))}\right|^2
% \leq \tilde{T}_{m-1}(f(\lambda_k))^{-2}.
% \]
% Using basic properties of Chebyshev polynomials \cite[Lemma 9.3]{tropp2023randomized},
% %\tyler{I just tried to pattern match from p46, need to check more carefully}
% \[
% \tilde{T}_{m-1}(f(\lambda_k))
% = T_{m-1}\left(\frac{f(\lambda_k)}{f(\lambda_{k+1}+2\epsilon_2)}\right)
% = T_{m-1}\left( \frac{1+\gamma_\epsilon}{1-\gamma_\epsilon} \right)
% \geq \frac{1}{2}e^{2(m-1) \sqrt{\gamma_\epsilon}}.
% \]
% Combining these results gives the specified bound for a particular degree $\lfloor(s-1)/1\rfloor \leq s-1$ polynomial $p(x)$ and thus
% \[
% \min\limits_{p \in \mathbb{P}_{s-1}} \left[\|p(\bm{\Lambda}_{n \setminus k})\|_{\F}^2\max\limits_{i=1,\ldots,k} \left|\frac{f(\lambda_i)}{p(\lambda_i)}\right|^2 \right]
% \leq  4\|f(\bm{\Lambda}_{n \setminus k}) + 2\epsilon_2\bm{I}\|_{\F}^2 e^{-4(m-1) \sqrt{\gamma_\epsilon}}.
% \]
% Combining this inequality with \cref{theorem:krylov_aware} and applying the triangle inequality gives the result.
% \end{proof}

% % {\color{blue}
% % \tyler{I didn't check these that carefully yet.} \David{I have cleaned up the section a bit. We can uncomment the blue stuff below. }

% % By constructing particular polynomials of degree $<s$, we can obtain more explicit bounds that depend only on how accurately $f(x)$ can be approximated by polynomials. 
% % These bounds are reminiscent of standard bounds that might be obtained if we could do exact products with $f(\bm{A})$, except that they have small error terms accounting for the fact that $f(x)$ might not be a polynomial. 
% % For example, by taking $p(x)$ as the best approximation to $f(x)$, shifted up to be above $f(x)$, we can obtain a bound like what one might obtain the the randomized SVD.
% % \begin{corollary}\label{theorem:rsvd_like_bound2}
% % Suppose $f(\lambda_i) \geq 0$ for all $i=1, \dots, n$ and define
% % \[
% % \epsilon_1 = \min_{p\in\mathbb{P}_{2r+1}} \|f(x) - p(x)\|_{L^{\infty}([\lambda_{\min},\lambda_{\max}])}
% % ,\quad
% % \epsilon_2 = \min_{p\in\mathbb{P}_{s-1}} \|f(x) - p(x)\|_{L^{\infty}(\Lambda)}.
% % \]
% % Then, if $\ell-k\geq 2$, the expected error $\mathbb{E}\|f(\bm{A}) - \bm{Q}_s \llbracket f(\bm{T}_q)_{1:d_s,1:d_s}\rrbracket_{k} \bm{Q}_s^\T\|_{\F}$ is bounded by
% % \[
% % 4\sqrt{d_s} \epsilon_1 +
% % \sqrt{\|f(\bm{\Lambda}_{n \setminus k })\|_{\F}^2 +  \frac{5k}{\ell - k + 1} \|f(\bm{\Lambda}_{n \setminus k}) + 2\epsilon_2 \bm{I}\|_{\F}^2}.
% % \]
% % \end{corollary}
% % A similar approach allows us to obtain bounds reminiscent of what one might obtain for the randomized block Krylov algorithm with exact products with $f(\bm{A})$.
% % For clarity we consider the gaped case where $f(\lambda_k) - f(\lambda_{k+1}) > 0$.
% % \begin{corollary}
% % \label{theorem:rbki_like_bound2}
% % Suppose $f(\lambda_i) \geq 0$ for all $i=1, \dots, n$, fix $q>0$, and define
% % \[
% % \epsilon_1 = \min_{p\in\mathbb{P}_{2r+1}} \|f(x) - p(x)\|_{L^{\infty}([\lambda_{\min},\lambda_{\max}])}
% % ,\quad
% % \epsilon_2 = \min_{p\in\mathbb{P}_{\lfloor (s-1)/q \rfloor}} \|f(x) - p(x)\|_{L^{\infty}(\Lambda)}.
% % \]
% % Define a gap parameter
% % \[
% % \gamma_{\epsilon}
% % = \frac{f(\lambda_k) - (f(\lambda_{k+1})+2\epsilon_2)}{f(\lambda_k) + f(\lambda_{k+1})+2\epsilon_2}.
% % \]
% % Then, if $\ell-k\geq 2$, the expected error $\mathbb{E}\|f(\bm{A}) - \bm{Q}_s \llbracket f(\bm{T}_q)_{1:d_s,1:d_s}\rrbracket_{k} \bm{Q}_s^\T\|_{\F}^2$ is bounded by
% % \[
% % 4\sqrt{d_s} \epsilon_1 +
% % \sqrt{\|f(\bm{\Lambda}_{n \setminus k })\|_{\F}^2 +  \frac{20k}{\ell - k + 1} \|f(\bm{\Lambda}_{n \setminus k}) + 2\epsilon_2\bm{I}\|_{\F}^2 e^{-4(q-1) \sqrt{\gamma_\epsilon}} }.
% % \]
% % \end{corollary}
% % Critically, note that we do not have to choose $q$ a priori. 
% % Thus, we ``automatically'' get the best possible bounds without having to decide how accurately we should approximate products with $f(\bm{A})$.

% % It is worth pointing out that in both \cref{theorem:rsvd_like_bound,theorem:rbki_like_bound}, $\epsilon_2$ depends on the best polynomial approximation to $f(x)$ \emph{on the eigenvalues of $\bm{A}$}.
% % Such bounds can often be much smaller than the corresponding bounds on an interval containing the eigenvalues.
% % In fact, in the special case where For $f(x) = 1/x$ and $\bm{A}$ is positive definite, the optimality of conjugate gradient allows $\epsilon_1$ to also be upgraded to a bound on the eigenvalues of $\bm{A}$.



% % Something about when $f(x) = x$,  we in fact recover standard bounds. 
    

% % \begin{proof}[Proof of \cref{theorem:rsvd_like_bound}]
% % Let $\hat{p}(x)$ be a degree $s-1$ polynomial such that $\|f(x) - \hat{p}(x)\|_{L^{\infty}(\Lambda)} = \epsilon_2$, and define $p(x) = \hat{p}(x) + \epsilon_2$.
% % Note that for any $i=1,\ldots, k$, $p(\lambda_i) \leq f(\lambda_i) + 2\epsilon_2$ so
% % \[
% % \|p(\bm{\Lambda}_{n \setminus k})\|_{\F}^2
% % \leq \|f(\bm{\Lambda}_{n \setminus k}) + 2\epsilon_2 \bm{I}\|_{\F}^2.
% % \]
% % In addition, since $f(\lambda_i) \leq p(\lambda_i)$,
% % \[
% % \max\limits_{i=1,\ldots,k} \left|\frac{f(\lambda_i)}{p(\lambda_i)}\right|^2
% % \leq 1.
% % \]
% % Combining these results gives the specified bound for a particular degree $s-1$ polynomial $p(x)$ and thus 
% % \[
% % \min\limits_{p \in \mathbb{P}_{s-1}} \left[\|p(\bm{\Lambda}_{n \setminus k})\|_{\F}^2\max\limits_{i=1,\ldots,k} \left|\frac{f(\lambda_i)}{p(\lambda_i)}\right|^2 \right]
% % \leq  \|f(\bm{\Lambda}_{n \setminus k}) + 2\epsilon_2 \bm{I}\|_{\F}^2.
% % \]
% % Combining with \cref{theorem:krylov_aware} gives the result.
% % \end{proof}

% % \begin{proof}[Proof of \cref{theorem:rbki_like_bound}]
% % Let $\hat{p}(x)$ be a degree $\lfloor (s-1)/q \rfloor$ polynomial such that $\|f(x) - \hat{p}(x)\|_{L^{\infty}(\Lambda)} = \epsilon_2$ and define $\tilde{p}(x) = \hat{p}(x) + \epsilon_2$.
% % Set $p(x) = \tilde{p}(x)\tilde{T}_{q-1}(\tilde{p}(x))$, where $\tilde{T}_{q-1}(x) = T_{q-1}(x / (f(\lambda_{k+1})+2\epsilon_2))$ is the degree $q-1$ Chebyshev polynomial scaled to start jumping at $f(\lambda_{k+1})+2\epsilon_2$.

% % Note that for $i=k+1, \ldots, n$, since $\tilde{p}(\lambda_i) \in [0,f(\lambda_{k+1}) + 2\epsilon_2]$, $\tilde{T}_{q-1}(\tilde{p}(\lambda_{i})) \leq 1$.
% % Thus, using that $p(\lambda_i) \leq f(\lambda_i) + 2\epsilon_2$,
% % \[
% % \|p(\bm{\Lambda}_{n \setminus k})\|_{\F}^2 
% % \leq \|\tilde{p}(\bm{\Lambda}_{n \setminus k}) \|_{\F}^2 \|\tilde{T}_{q-1}(\tilde{p}(\bm{\Lambda}_{n \setminus k}))\|_{2}^2
% % \leq \|f(\bm{\Lambda}_{n \setminus k}) + 2\epsilon_2\bm{I}\|_{\F}^2.
% % \]
% % Now, fix $i=1,\ldots, k$.
% % If $\gamma_k\in[0,1]$ then $f(\lambda_i) \geq f(\lambda_{k+1}) + 2\epsilon_2$.
% % Thus, since $f(\lambda_i)\leq \tilde{p}(\lambda_i)$, $p(\lambda_i) = \tilde{p}(\lambda_{i})\tilde{T}_{q-1}(\tilde{p}(\lambda_{i})) \geq f(\lambda_{i})\tilde{T}_{q-1}(f(\lambda_{i}))$.
% % This implies that
% % \[
% % \max\limits_{i=1,\ldots,k} \left|\frac{f(\lambda_i)}{p(\lambda_i)}\right|^2
% % \leq \max\limits_{i=1,\ldots,k} \left|\frac{f(\lambda_i)}{\tilde{T}_q(f(\lambda_i))}\right|^2
% % \leq T_{q-1}(f(\lambda_k))^{-2}.
% % \]
% % Using basic properties of Chebyshev polynomials \cite[Lemma 9.3]{tropp2023randomized},
% % \tyler{I just tried to pattern match from p46, need to check more carefully}
% % \[
% % \tilde{T}_{q-1}(f(\lambda_k))
% % = T_{q-1}\left(\frac{f(\lambda_k)}{f(\lambda_{k+1}+2\epsilon_2)}\right)
% % = T_{q-1}\left( \frac{1+\gamma_\epsilon}{1-\gamma_\epsilon} \right)
% % \geq \frac{1}{2}e^{2(q-1) \sqrt{\gamma_\epsilon}}.
% % \]
% % Combining these results gives the specified bound for a particular degree $\lfloor(s-1)/1\rfloor \leq s-1$ polynomial $p(x)$ and thus
% % \[
% % \min\limits_{p \in \mathbb{P}_{s-1}} \left[\|p(\bm{\Lambda}_{n \setminus k})\|_{\F}^2\max\limits_{i=1,\ldots,k} \left|\frac{f(\lambda_i)}{p(\lambda_i)}\right|^2 \right]
% % \leq  4\|f(\bm{\Lambda}_{n \setminus k}) + 2\epsilon_2\bm{I}\|_{\F}^2 e^{-4(q-1) \sqrt{\gamma_\epsilon}}.
% % \]
% % Combining with \cref{theorem:krylov_aware} gives the result.
% % \end{proof}

% % }

% % We proceed with commenting on the three terms appearing in the bounds in \Cref{theorem:krylov_aware}. 
% % The $4\sqrt{d_s} \inf\limits_{p \in \mathbb{P}_{2r+1}}\|f(x)-p(x)\|_{L^{\infty}([\lambda_{\min},\lambda_{\max}])}$ term tells us that the approximation of the quadratic form $\bm{Q}_s^\T f(\bm{A})\bm{Q}_s$ needs to be approximated accurately. If we know that $\|f(\bm{T}_q)_{1:d_s,1:d_s} - \bm{Q}_s^\T f(\bm{A}) \bm{Q}_s\|_{\F} \leq \Delta$ almost surely then the $4\sqrt{d_s} \inf\limits_{p \in \mathbb{P}_{2r+1}}\|f(x)-p(x)\|_{L^{\infty}([\lambda_{\min},\lambda_{\max}])}$ term appearing in \Cref{theorem:krylov_aware} can be replaced with $2 \Delta$. The $\|f(\bm{\Lambda}_{n \setminus k})\|_{\F}$ term tells us that the error can never be below the optimal rank $k$ approximation error. Finally, $\min\limits_{p \in \mathbb{P}_{s-1}} \left[\|p(\bm{\Lambda}_{n \setminus k}) \bm{\Omega}_{n \setminus k} \bm{\Omega}_k^{\dagger}\|_{\F}^2\max\limits_{i=1,\ldots,k} \left|\frac{f(\lambda_i)}{p(\lambda_i)}\right|^2\right]$ tells us that $\bm{Q}_s$ is a good orthonormal basis for low-rank approximation if there is a polynomial of degree at most $s-1$ that is very large on the large eigenvalues $\lambda_1,\ldots,\lambda_k$ and is very small on the small eigenvalues $\lambda_{k+1},\ldots,\lambda_n$, which effectively denoises the contribution from the small eigenvalues of $f(\bm{A})$. A similar intuition was used in \cite{MM15,tropp2023randomized}.

% Furthermore, note that when $f$ is a monotonic function one can derive even stronger bounds by choosing $p$ to be a scaled and shifted Chebyshev polynomial that is small on the eigenvalues in $\bm{\Lambda}_{n \setminus k}$ and large on the eigenvalues in $\bm{\Lambda}_k$. We omit a detailed discussion. }}

% \David{In my opinion, this section is redundant now and should be removed. }
% Furthermore, to obtain explicit bounds one must obtain an upper bound for the optimization problem
% \begin{equation*}
%      \min\limits_{p \in \mathbb{P}_{s-1}} \left[\|p(\bm{\Lambda}_2)\|_{\F}\max\limits_{i=1,\ldots,k} \left|\frac{f(\lambda_i)}{p(\lambda_i)}\right| \right].
% \end{equation*}
% This is a non-trivial task without any assumptions on $f$ and the spectrum of $\bm{A}$. However, when $f$ is a monotonic function that does not change sign one can obtain some explicit bounds. In this case all the eigenvalues in $\bm{\Lambda}_k$ are contained in a closed interval $I_k$ and all eigenvalues in $\bm{\Lambda}_{n \setminus k}$ in a closed interval $I_{n \setminus k}$, where $I_k$ and $I_{n \setminus k}$ can intersect in at most one point, which happens if $f(\lambda_k) = f(\lambda_{k+1})$. In this case one can choose $p$ to be a scaled and shifted Chebyshev polynomial of degree $s-1$ so that $|p(x)| \leq 1$ for $x \in I_{n \setminus k}$ and $p(x)$ grows quickly for $x \in I_{k}$. In this case we obtain 
% \begin{equation}\label{eq:upperbound}
%     \|p(\bm{\Lambda}_2)\|_{\F}\max\limits_{i=1,\ldots,k} \left|\frac{f(\lambda_i)}{p(\lambda_i)}\right| \leq \sqrt{n} \max\limits_{i=1,\ldots,k}\left|\frac{f(\lambda_i)}{p(\lambda_i)}\right|.
% \end{equation}
% While the upper bound is easier to work with, it is still difficult to obtain a tight upper bound without any assumptions on $f$. One can derive an upper bound for special functions, such as $f(x) = \exp(x)$, but we omit details. 

% \subsection{Explicit bounds}
% Deriving tight upper bounds for the objective value of the optimization problem
% \begin{equation}\label{eq:difficult_optim_problem}
%     \min\limits_{p \in \mathbb{P}_{s-1}} \left[\|p(\bm{\Lambda}_2)\|_{\F}\max\limits_{i=1,\ldots,k} \left|\frac{f(\lambda_i)}{p(\lambda_i)}\right| \right]
% \end{equation}
% is a difficult task without any assumptions on $f$ and the spectrum of $\bm{A}$. Instead, we will focus on a certain classes of functions.
% \subsubsection{Functions that are well approximated by polynomials}
% Suppose that there exists a polynomial $p \in \mathbb{P}_{s-1}$ so that
% \begin{equation*}
%     \|p(\bm{A}) - f(\bm{A})\|_2 \leq \delta_{s-1}.
% \end{equation*}
% Then, $f(\bm{A}) \preceq p(\bm{A}) + \delta_{s-1} \bm{I} \preceq f(\bm{A}) + 2 \delta_{s-1} \bm{I}$, and
% \begin{equation*}
%     \min\limits_{p \in \mathbb{P}_{s-1}} \left[\|p(\bm{\Lambda}_2)\|_{\F}\max\limits_{i=1,\ldots,k} \left|\frac{f(\lambda_i)}{p(\lambda_i)}\right| \right] \leq \|f(\bm{\Lambda}_2)\|_{\F} + 2n \delta_{s-1}.
% \end{equation*}
% However, in most cases this will be a significant overestimation of \eqref{eq:difficult_optim_problem}.
% \subsubsection{Monotonic functions that do not change sign}
%  When $f$ is a monotonic function that does not change sign all the eigenvalues in $\bm{\Lambda}_1$ are contained in a closed interval $I_1$ and all eigenvalues in $\bm{\Lambda}_2$ in a closed interval $I_2$ and $I_1 \cap I_2 = \emptyset$. Furthermore, we may also assume without loss of generality that $f$ is positive. If $f$ is negative we consider $g = -f$ and the similar bounds derived in this section still hold. 

% One can upper bound \eqref{eq:difficult_optim_problem} with
% \begin{equation}\label{eq:upper_bound1}
%     \min\limits_{p \in \mathbb{P}_{s-1}} \left[\|p(\bm{\Lambda}_2)\|_{\F}\max\limits_{i=1,\ldots,k} \left|\frac{f(\lambda_i)}{p(\lambda_i)}\right| \right] \leq \sqrt{n} \min\limits_{p \in \mathbb{P}_{s-1}} \left[\|p\|_{L^{\infty}(I_2)} \left\|\frac{f}{p}\right\|_{L^{\infty}(I_1)}\right]. 
% \end{equation}
% By shifting $f$ and the eigenvalues of $\bm{\Lambda}$\footnote{Recall that we ordered the eigenvalues of $\bm{\Lambda}$ so that $\lambda_1 \geq \lambda_2 \geq \ldots \geq \lambda_n$ if $f$ is increasing and $\lambda_1 \leq \lambda_2 \leq \ldots \leq \lambda_n$ if $f$ is decreasing}
% \begin{align*}
%     &g(x) = f(x + \lambda_n) \text{ and } \mu_i = \lambda_i - \lambda_n \quad \text{if } f \text{ is increasing};\\
%     &g(x) = f(\lambda_1 - x) \text{ and } \mu_i = \lambda_1 - \lambda_{i} \quad \text{if } f \text{ is decreasing};
% \end{align*}
% so that $g$ is increasing, $\mu_1 \geq \mu_2 \geq \ldots \geq \mu_n \geq 0$, and $g(\mu_i) = f(\lambda_i)$. Furthermore, we have
% \begin{equation}\label{eq:upper_bound2}
%     \min\limits_{p \in \mathbb{P}_{s-1}} \left[\|p\|_{L^{\infty}(I_2)} \left\|\frac{f}{p}\right\|_{L^{\infty}(I_1)}\right] = \min\limits_{p \in \mathbb{P}_{s-1}} \left[\|p\|_{L^{\infty}([0,\mu_{k+1}])} \left\|\frac{g}{p}\right\|_{L^{\infty}([\mu_{k},\mu_1])}\right],
% \end{equation}
% By setting $p(x) = T_{s-1}\left(\frac{x}{\mu_{k+1}}\right)$ we know that $|p(x)| \leq 1$ for $x \in [0,\mu_{k+1}]$ \textcolor{red}{cite this} and $p(x)$ is increasing for $x \geq \mu_{k+1}$. Hence, 
% \begin{equation}\label{eq:upper_bound3}
%     \min\limits_{p \in \mathbb{P}_{s-1}} \left[\|p\|_{L^{\infty}([0,s_{k+1}])} \left\|\frac{g}{p}\right\|_{L^{\infty}([\mu_{k},\mu_1])}\right] \leq \frac{ \|f(\bm{A})\|_2}{T_{s-1}\left(\frac{\mu_k}{\mu_{k+1}}\right)}.
% \end{equation}
% Note that for $x \geq 1$ we have
% \begin{equation}\label{eq:chebyshev_upper_bound}
%     T_{q}(x) = \frac{1}{2}\left((x + \sqrt{x^2-1})^q + (x + \sqrt{x^2-1})^{-q}\right) \geq \frac{1}{2} (2x-1)^q.
% \end{equation}
% Hence, by combining \eqref{eq:upper_bound1}, \eqref{eq:upper_bound2},\eqref{eq:upper_bound3}, and \eqref{eq:chebyshev_upper_bound} we have
% \begin{equation}\label{eq:upper_bound4}
%     \min\limits_{p \in \mathbb{P}_{s-1}} \left[\|p(\bm{\Lambda}_2)\|_{\F}\max\limits_{i=1,\ldots,k} \left|\frac{f(\lambda_i)}{p(\lambda_i)}\right| \right] \leq \frac{2\sqrt{n}\|f(\bm{A})\|_2}{(2 \frac{\mu_k}{\mu_{k+1}} - 1)^{s-1}}.
% \end{equation}
% \textcolor{red}{maybe we should use a different upper bound for the Chebyshev polynomials.}

% \subsubsection{Monotonic functions that do not change signs and are log-convex functions}
% \eqref{eq:upper_bound4} can still be loose for many functions. Under some additional assumptions on $f$ one can derive tigher bounds for \eqref{eq:upper_bound1}, and one such assumption is that $f$ is log-convex. A function $f$ is log-convex if it can be written as $f(x) = \exp(h(x))$ for a convex function $h$. $f(x) = \exp(\beta x)$ for $\beta \in \mathbb{R}$ and $f(x) = \frac{1}{x^c}$ for $c \geq 0$ are examples of log-convex functions. 

% Using \eqref{eq:upper_bound1}, \eqref{eq:upper_bound2}, and \eqref{eq:chebyshev_upper_bound} we have
% \begin{equation*}
%     \min\limits_{p \in \mathbb{P}_{s-1}} \left[\|p(\bm{\Lambda}_2)\|_{\F}\max\limits_{i=1,\ldots,k} \left|\frac{f(\lambda_i)}{p(\lambda_i)}\right| \right] \leq \sqrt{n}\exp\left(\sup\limits_{x \in [\mu_k,\mu_1]} \left\{h(x) -(s-1) \ln\left(2\frac{x}{\mu_{k+1}}-1\right)\right\}\right).
% \end{equation*}
% Since $h(x) - (s-1)\ln\left(2\frac{x}{\mu_{k+1}}-1\right)$ is a convex function it attains its maximum on the boundary of $[\mu_{k},\mu_1]$. Furthermore, one can also show that if $s\geq \log\left(\frac{g(\mu_1)}{g(\mu_{k})}\right) /\log\left(\frac{2\mu_1 - \mu_{k+1}}{2\mu_k - \mu_{k+1}}\right)+1$ that the maximum is attained at $x = \mu_k$. Hence, in this case we have
% \begin{equation*}
%     \min\limits_{p \in \mathbb{P}_{s-1}} \left[\|p(\bm{\Lambda}_2)\|_{\F}\max\limits_{i=1,\ldots,k} \left|\frac{f(\lambda_i)}{p(\lambda_i)}\right| \right] \leq \sqrt{n} \frac{f(\mu_k)}{\left(\frac{2\mu_k}{\mu_{k+1}} - 1\right)^{s-1}}.
% \end{equation*}
\section{Numerical experiments}
\label{sec:experiments}
In this section we compare the Krylov aware low-rank approximation (\Cref{alg:krylow}) and \Cref{alg:rsvd} (assuming exact matvecs with $f(\bm{A})$) and \Cref{alg:rsvd_matfun} (inexact matvecs with $f(\bm{A})$). All experiments have been performed in MATLAB (version 2020a) and scripts to reproduce the figures are available at \url{https://github.com/davpersson/Krylov_aware_LRA.git}.

\subsection{Test matrices}
We begin with outlining the test matrices and matrix functions used in our examples. 



\subsubsection{Exponential integrator}\label{section:exponential_integrator}
The following example is taken from \cite{persson_kressner_23}. Consider the following parabolic differential equation
\begin{align*}
\begin{split}
    &u_t = \kappa \Delta u + \lambda u \text{ in } [0,1]^2 \times [0,2]\\
    &u(\cdot,0) = \theta \text{ in } [0,1]^2\\
    &u = 0 \text{ on } \Gamma_1\\
    & \frac{\partial u}{\partial \bm{n}} = 0 \text{ on } \Gamma_2
\end{split}
\end{align*}
for $\kappa, \lambda > 0$ and $\Gamma_2 = \{(x,1) \in \mathbb{R}^{2} : x \in [0,1] \}$ and $\Gamma_1 = \partial \mathcal{D} \setminus \Gamma_2$. By discretizing in space using finite differences on a $100 \times 100$ grid we obtain an ordinary differential equation of the form
\begin{align}
\begin{split}
    \dot{\bm{u}}(t) &= \bm{A} \bm{u}(t) \text{ for } t \geq 0,\\
    \bm{u}(0) &= \bm{\theta}, \label{eq:ode}
\end{split}
\end{align}
for symmetric matrix $\bm{A} \in \mathbb{R}^{9900 \times 9900}$. It is well known that the solution to \eqref{eq:ode} is given by $\bm{u}(t) = \exp(t\bm{A})\bm{\theta}$. Suppose that we want to compute the solution for $t \geq 1$. One can verify that 
\begin{equation*}
    \max\limits_{t \geq 1} \frac{\|\exp(t\bm{A}) - \llbracket\exp(t\bm{A})\rrbracket_{k}\|_{\F}}{\|\exp(t\bm{A})\|_{\F}} = \frac{\|\exp(\bm{A}) - \llbracket\exp(\bm{A})\rrbracket_{k}\|_{\F}}{\|\exp(\bm{A})\|_{\F}},
\end{equation*}
and it turns out that $\exp(\bm{A})$ admits a good rank $60$ approximation
\begin{equation*}
    \frac{\|\exp(\bm{A}) - \llbracket\exp(\bm{A})\rrbracket_{60}\|_{\F}}{\|\exp(\bm{A})\|_{\F}} \approx 4 \times 10^{-4}.
\end{equation*}
Hence, we can use \Cref{alg:krylow} to compute $\bm{Q}_s$ and $\bm{T}_q$ and use them to efficiently construct a rank $60$ approximation to $\exp(t\bm{A})$ for any $t$ \emph{with almost no additional cost}. 

In the experiments we set $\kappa = 0.01$ and $\lambda = 1$. 
\subsubsection{Estrada index}\label{section:estrada}
For an (undirected) graph with adjacency matrix $\bm{A}$ the Estrada index is defined as $\tr(\exp(\bm{A}))$. It is used to measure the degree of protein folding \cite{estrada}. One can estimate the Estrada index of a network by the Hutch++ algorithm or its variations \cite{chen_hallman_23,epperly2023xtrace,hpp,ahpp}, which requires computing a low-rank approximation of $\exp(\bm{A})$. Motivated by the numerical experiments in \cite{hpp} we let $\bm{A}$ be the adjacency matrix of Roget’s Thesaurus semantic graph \cite{roget}. 

\begin{figure}[ht]
\begin{subfigure}{.5\textwidth}
  \centering
  \includegraphics[width=.9\linewidth]{imgs/exponential_integrator}  
  \caption{Example from \Cref{section:exponential_integrator}}
\end{subfigure}
\begin{subfigure}{.5\textwidth}
  \centering
  \includegraphics[width=.9\linewidth]{imgs/estrada} 
  \caption{Example from \Cref{section:estrada}}
\end{subfigure}
\begin{subfigure}{.5\textwidth}
  \centering
  \includegraphics[width=.9\linewidth]{imgs/quantum_spin}  
  \caption{Example from \Cref{section:quantum_spin}}
\end{subfigure}
\begin{subfigure}{.5\textwidth}
  \centering
  \includegraphics[width=.9\linewidth]{imgs/synthetic_log}  
  \caption{Example from \Cref{section:synthetic_log}}
\end{subfigure}
\caption{Comparing relative error \eqref{eq:relative_error} for the the approximations returned by \Cref{alg:krylow} without truncation (untruncated), \Cref{alg:krylow} with truncation back to rank $k$ (truncated), \Cref{alg:rsvd_matfun}, and \Cref{alg:rsvd}. The black line shows the optimal rank $k$ approximation relative Frobenius norm error. The rank parameter $k$ is visible as titles in the figures. In all experiments we set $\ell = k$. }
\label{fig:relative_errors}
\end{figure}

\subsubsection{Quantum spin system}\label{section:quantum_spin}
We use an example from \cite[Section 4.3]{epperly2023xtrace}, a similar example is found in \cite{chen_hallman_23}, in which we want to approximate $\exp(-\beta \bm{A})$ where 
\begin{equation*}
    \bm{A} = -\sum\limits_{i=1}^{N-1} \bm{Z}_i\bm{Z}_{i+1} -h\sum\limits_{i=1}^N \bm{X}_i \in \mathbb{R}^{n \times n},
\end{equation*}
where
\begin{equation*}
    \bm{X}_i = \bm{I}_{2^{i-1}} \otimes \bm{X} \otimes \bm{I}_{2^{N-i}}, \quad \bm{Z}_i = \bm{I}_{2^{i-1}} \otimes \bm{Z} \otimes \bm{I}_{2^{N-i}}
\end{equation*}
where $\bm{X}$ and $\bm{Z}$ are the Pauli operators
\begin{equation*}
    \bm{X} = \begin{bmatrix} 0 & 1 \\ 1 & 0 \end{bmatrix}, \quad \bm{Z} = \begin{bmatrix} 1 & 0 \\ 0 & -1 \end{bmatrix}.
\end{equation*}
Estimating the partition function $Z(\beta) = \tr(\exp(-\beta \bm{A}))$ is an important task in quantum mechanics \cite{pfeuty1970one}, which once again can benefit from computing a low-rank approximation of $\exp(-\beta \bm{A})$.

In the experiments we set $N = 14$ so that $n = 2^{14}$, $\beta = 0.3$, and $h = 10$.  



\subsubsection{Synthetic example for the matrix logarithm}\label{section:synthetic_log}
We generate a symmetric matrix $\bm{A} \in \mathbb{R}^{5000 \times 5000}$ with eigenvalues $\lambda_i = \exp(\frac{1}{i^2})$ for $i = 1,\ldots,n$. We let $f(x) = \log(x)$ so that the eigenvalues of $f(\bm{A})$ are $f(\lambda_i) = \frac{1}{i^2}$ for $i = 1,\ldots,n$.
%\subsubsection{Synthetic example for the inverse}
%We generate a matrix $\bm{A} \in \mathbb{R}^{5000 \times 5000}$ with eigenvalues $\lambda_i = i^2$ for $i = 1,\ldots,n$. We let $f(x) = \frac{1}{x}$ so that the eigenvalues of $f(\bm{A})$ are $f(\lambda_i) = \frac{1}{i^2}$ for $i = 1,\ldots,n$.


\begin{figure}[ht]
\begin{subfigure}{.5\textwidth}
  \centering
  \includegraphics[width=.9\linewidth]{imgs/exponential_integrator_p=5}  
  \caption{Example from \Cref{section:exponential_integrator}}
\end{subfigure}
\begin{subfigure}{.5\textwidth}
  \centering
  \includegraphics[width=.9\linewidth]{imgs/estrada_p=5} 
  \caption{Example from \Cref{section:estrada}}
\end{subfigure}
\begin{subfigure}{.5\textwidth}
  \centering
  \includegraphics[width=.9\linewidth]{imgs/quantum_spin_p=5}  
  \caption{Example from \Cref{section:quantum_spin}}
\end{subfigure}
\begin{subfigure}{.5\textwidth}
  \centering
  \includegraphics[width=.9\linewidth]{imgs/synthetic_log_p=5}  
  \caption{Example from \Cref{section:synthetic_log}}
\end{subfigure}
\caption{Comparing relative error  \eqref{eq:relative_error} for the the approximations returned by \Cref{alg:krylow} without truncation (untruncated), \Cref{alg:krylow} with truncation back to rank $k$ (truncated), \Cref{alg:rsvd_matfun}, and \Cref{alg:rsvd}. The black line shows the optimal rank $k$ approximation relative Frobenius norm error. The rank parameter $k$ is visible as titles in the figures. In all experiments we set $\ell = k+5$. }
\label{fig:relative_errors2}
\end{figure}

\subsection{Comparing relative errors}
In this section we compare the error returned by \Cref{alg:krylow},  \Cref{alg:rsvd}, and \Cref{alg:rsvd_matfun}. If $\bm{C}$ is a low-rank approximation returned by one of the algorithms then we compare the relative error
\begin{equation}\label{eq:relative_error}
    \frac{\|f(\bm{A}) - \bm{C}\|_{\F}}{\|f(\bm{A})\|_{\F}}.
\end{equation}
In all experiments we set the parameters in \Cref{alg:krylow} and \Cref{alg:rsvd_matfun} to be $\ell = k$ or $\ell = k + 5$ and $s = r$ so that the total number of matrix vector products with $\bm{A}$ is $2\ell s$. 
When we run \Cref{alg:rsvd_matfun} we compute matvecs with $f(\bm{A})$ exactly, which cannot be done in practice. Hence, the results from this algorithm are only used as a reference for \Cref{alg:krylow} and \Cref{alg:rsvd_matfun}.
The results are presented in \Cref{fig:relative_errors} for $\ell = k$ and \Cref{fig:relative_errors2} for $\ell = k + 5$. All results confirm that \Cref{alg:krylow} returns a more accurate approximation than \Cref{alg:rsvd_matfun}, and can even be more accurate than the \Cref{alg:rsvd}. Note that for the example given in \Cref{section:quantum_spin} the error for the untruncated version of the approximation returned by \Cref{alg:krylow} stagnates. This is because the error from the approximation of the quadratic form dominates the error. In this case, $r$ should be chosen larger than $s$. However, we leave it as an open research question to determine the optimal balancing of $s$ and $r$.  


\section{Conclusion}
In this work, we propose a simple yet effective approach, called SMILE, for graph few-shot learning with fewer tasks. Specifically, we introduce a novel dual-level mixup strategy, including within-task and across-task mixup, for enriching the diversity of nodes within each task and the diversity of tasks. Also, we incorporate the degree-based prior information to learn expressive node embeddings. Theoretically, we prove that SMILE effectively enhances the model's generalization performance. Empirically, we conduct extensive experiments on multiple benchmarks and the results suggest that SMILE significantly outperforms other baselines, including both in-domain and cross-domain few-shot settings.

\bibliographystyle{siam}
\bibliography{bibliography}

\subsection{Lloyd-Max Algorithm}
\label{subsec:Lloyd-Max}
For a given quantization bitwidth $B$ and an operand $\bm{X}$, the Lloyd-Max algorithm finds $2^B$ quantization levels $\{\hat{x}_i\}_{i=1}^{2^B}$ such that quantizing $\bm{X}$ by rounding each scalar in $\bm{X}$ to the nearest quantization level minimizes the quantization MSE. 

The algorithm starts with an initial guess of quantization levels and then iteratively computes quantization thresholds $\{\tau_i\}_{i=1}^{2^B-1}$ and updates quantization levels $\{\hat{x}_i\}_{i=1}^{2^B}$. Specifically, at iteration $n$, thresholds are set to the midpoints of the previous iteration's levels:
\begin{align*}
    \tau_i^{(n)}=\frac{\hat{x}_i^{(n-1)}+\hat{x}_{i+1}^{(n-1)}}2 \text{ for } i=1\ldots 2^B-1
\end{align*}
Subsequently, the quantization levels are re-computed as conditional means of the data regions defined by the new thresholds:
\begin{align*}
    \hat{x}_i^{(n)}=\mathbb{E}\left[ \bm{X} \big| \bm{X}\in [\tau_{i-1}^{(n)},\tau_i^{(n)}] \right] \text{ for } i=1\ldots 2^B
\end{align*}
where to satisfy boundary conditions we have $\tau_0=-\infty$ and $\tau_{2^B}=\infty$. The algorithm iterates the above steps until convergence.

Figure \ref{fig:lm_quant} compares the quantization levels of a $7$-bit floating point (E3M3) quantizer (left) to a $7$-bit Lloyd-Max quantizer (right) when quantizing a layer of weights from the GPT3-126M model at a per-tensor granularity. As shown, the Lloyd-Max quantizer achieves substantially lower quantization MSE. Further, Table \ref{tab:FP7_vs_LM7} shows the superior perplexity achieved by Lloyd-Max quantizers for bitwidths of $7$, $6$ and $5$. The difference between the quantizers is clear at 5 bits, where per-tensor FP quantization incurs a drastic and unacceptable increase in perplexity, while Lloyd-Max quantization incurs a much smaller increase. Nevertheless, we note that even the optimal Lloyd-Max quantizer incurs a notable ($\sim 1.5$) increase in perplexity due to the coarse granularity of quantization. 

\begin{figure}[h]
  \centering
  \includegraphics[width=0.7\linewidth]{sections/figures/LM7_FP7.pdf}
  \caption{\small Quantization levels and the corresponding quantization MSE of Floating Point (left) vs Lloyd-Max (right) Quantizers for a layer of weights in the GPT3-126M model.}
  \label{fig:lm_quant}
\end{figure}

\begin{table}[h]\scriptsize
\begin{center}
\caption{\label{tab:FP7_vs_LM7} \small Comparing perplexity (lower is better) achieved by floating point quantizers and Lloyd-Max quantizers on a GPT3-126M model for the Wikitext-103 dataset.}
\begin{tabular}{c|cc|c}
\hline
 \multirow{2}{*}{\textbf{Bitwidth}} & \multicolumn{2}{|c|}{\textbf{Floating-Point Quantizer}} & \textbf{Lloyd-Max Quantizer} \\
 & Best Format & Wikitext-103 Perplexity & Wikitext-103 Perplexity \\
\hline
7 & E3M3 & 18.32 & 18.27 \\
6 & E3M2 & 19.07 & 18.51 \\
5 & E4M0 & 43.89 & 19.71 \\
\hline
\end{tabular}
\end{center}
\end{table}

\subsection{Proof of Local Optimality of LO-BCQ}
\label{subsec:lobcq_opt_proof}
For a given block $\bm{b}_j$, the quantization MSE during LO-BCQ can be empirically evaluated as $\frac{1}{L_b}\lVert \bm{b}_j- \bm{\hat{b}}_j\rVert^2_2$ where $\bm{\hat{b}}_j$ is computed from equation (\ref{eq:clustered_quantization_definition}) as $C_{f(\bm{b}_j)}(\bm{b}_j)$. Further, for a given block cluster $\mathcal{B}_i$, we compute the quantization MSE as $\frac{1}{|\mathcal{B}_{i}|}\sum_{\bm{b} \in \mathcal{B}_{i}} \frac{1}{L_b}\lVert \bm{b}- C_i^{(n)}(\bm{b})\rVert^2_2$. Therefore, at the end of iteration $n$, we evaluate the overall quantization MSE $J^{(n)}$ for a given operand $\bm{X}$ composed of $N_c$ block clusters as:
\begin{align*}
    \label{eq:mse_iter_n}
    J^{(n)} = \frac{1}{N_c} \sum_{i=1}^{N_c} \frac{1}{|\mathcal{B}_{i}^{(n)}|}\sum_{\bm{v} \in \mathcal{B}_{i}^{(n)}} \frac{1}{L_b}\lVert \bm{b}- B_i^{(n)}(\bm{b})\rVert^2_2
\end{align*}

At the end of iteration $n$, the codebooks are updated from $\mathcal{C}^{(n-1)}$ to $\mathcal{C}^{(n)}$. However, the mapping of a given vector $\bm{b}_j$ to quantizers $\mathcal{C}^{(n)}$ remains as  $f^{(n)}(\bm{b}_j)$. At the next iteration, during the vector clustering step, $f^{(n+1)}(\bm{b}_j)$ finds new mapping of $\bm{b}_j$ to updated codebooks $\mathcal{C}^{(n)}$ such that the quantization MSE over the candidate codebooks is minimized. Therefore, we obtain the following result for $\bm{b}_j$:
\begin{align*}
\frac{1}{L_b}\lVert \bm{b}_j - C_{f^{(n+1)}(\bm{b}_j)}^{(n)}(\bm{b}_j)\rVert^2_2 \le \frac{1}{L_b}\lVert \bm{b}_j - C_{f^{(n)}(\bm{b}_j)}^{(n)}(\bm{b}_j)\rVert^2_2
\end{align*}

That is, quantizing $\bm{b}_j$ at the end of the block clustering step of iteration $n+1$ results in lower quantization MSE compared to quantizing at the end of iteration $n$. Since this is true for all $\bm{b} \in \bm{X}$, we assert the following:
\begin{equation}
\begin{split}
\label{eq:mse_ineq_1}
    \tilde{J}^{(n+1)} &= \frac{1}{N_c} \sum_{i=1}^{N_c} \frac{1}{|\mathcal{B}_{i}^{(n+1)}|}\sum_{\bm{b} \in \mathcal{B}_{i}^{(n+1)}} \frac{1}{L_b}\lVert \bm{b} - C_i^{(n)}(b)\rVert^2_2 \le J^{(n)}
\end{split}
\end{equation}
where $\tilde{J}^{(n+1)}$ is the the quantization MSE after the vector clustering step at iteration $n+1$.

Next, during the codebook update step (\ref{eq:quantizers_update}) at iteration $n+1$, the per-cluster codebooks $\mathcal{C}^{(n)}$ are updated to $\mathcal{C}^{(n+1)}$ by invoking the Lloyd-Max algorithm \citep{Lloyd}. We know that for any given value distribution, the Lloyd-Max algorithm minimizes the quantization MSE. Therefore, for a given vector cluster $\mathcal{B}_i$ we obtain the following result:

\begin{equation}
    \frac{1}{|\mathcal{B}_{i}^{(n+1)}|}\sum_{\bm{b} \in \mathcal{B}_{i}^{(n+1)}} \frac{1}{L_b}\lVert \bm{b}- C_i^{(n+1)}(\bm{b})\rVert^2_2 \le \frac{1}{|\mathcal{B}_{i}^{(n+1)}|}\sum_{\bm{b} \in \mathcal{B}_{i}^{(n+1)}} \frac{1}{L_b}\lVert \bm{b}- C_i^{(n)}(\bm{b})\rVert^2_2
\end{equation}

The above equation states that quantizing the given block cluster $\mathcal{B}_i$ after updating the associated codebook from $C_i^{(n)}$ to $C_i^{(n+1)}$ results in lower quantization MSE. Since this is true for all the block clusters, we derive the following result: 
\begin{equation}
\begin{split}
\label{eq:mse_ineq_2}
     J^{(n+1)} &= \frac{1}{N_c} \sum_{i=1}^{N_c} \frac{1}{|\mathcal{B}_{i}^{(n+1)}|}\sum_{\bm{b} \in \mathcal{B}_{i}^{(n+1)}} \frac{1}{L_b}\lVert \bm{b}- C_i^{(n+1)}(\bm{b})\rVert^2_2  \le \tilde{J}^{(n+1)}   
\end{split}
\end{equation}

Following (\ref{eq:mse_ineq_1}) and (\ref{eq:mse_ineq_2}), we find that the quantization MSE is non-increasing for each iteration, that is, $J^{(1)} \ge J^{(2)} \ge J^{(3)} \ge \ldots \ge J^{(M)}$ where $M$ is the maximum number of iterations. 
%Therefore, we can say that if the algorithm converges, then it must be that it has converged to a local minimum. 
\hfill $\blacksquare$


\begin{figure}
    \begin{center}
    \includegraphics[width=0.5\textwidth]{sections//figures/mse_vs_iter.pdf}
    \end{center}
    \caption{\small NMSE vs iterations during LO-BCQ compared to other block quantization proposals}
    \label{fig:nmse_vs_iter}
\end{figure}

Figure \ref{fig:nmse_vs_iter} shows the empirical convergence of LO-BCQ across several block lengths and number of codebooks. Also, the MSE achieved by LO-BCQ is compared to baselines such as MXFP and VSQ. As shown, LO-BCQ converges to a lower MSE than the baselines. Further, we achieve better convergence for larger number of codebooks ($N_c$) and for a smaller block length ($L_b$), both of which increase the bitwidth of BCQ (see Eq \ref{eq:bitwidth_bcq}).


\subsection{Additional Accuracy Results}
%Table \ref{tab:lobcq_config} lists the various LOBCQ configurations and their corresponding bitwidths.
\begin{table}
\setlength{\tabcolsep}{4.75pt}
\begin{center}
\caption{\label{tab:lobcq_config} Various LO-BCQ configurations and their bitwidths.}
\begin{tabular}{|c||c|c|c|c||c|c||c|} 
\hline
 & \multicolumn{4}{|c||}{$L_b=8$} & \multicolumn{2}{|c||}{$L_b=4$} & $L_b=2$ \\
 \hline
 \backslashbox{$L_A$\kern-1em}{\kern-1em$N_c$} & 2 & 4 & 8 & 16 & 2 & 4 & 2 \\
 \hline
 64 & 4.25 & 4.375 & 4.5 & 4.625 & 4.375 & 4.625 & 4.625\\
 \hline
 32 & 4.375 & 4.5 & 4.625& 4.75 & 4.5 & 4.75 & 4.75 \\
 \hline
 16 & 4.625 & 4.75& 4.875 & 5 & 4.75 & 5 & 5 \\
 \hline
\end{tabular}
\end{center}
\end{table}

%\subsection{Perplexity achieved by various LO-BCQ configurations on Wikitext-103 dataset}

\begin{table} \centering
\begin{tabular}{|c||c|c|c|c||c|c||c|} 
\hline
 $L_b \rightarrow$& \multicolumn{4}{c||}{8} & \multicolumn{2}{c||}{4} & 2\\
 \hline
 \backslashbox{$L_A$\kern-1em}{\kern-1em$N_c$} & 2 & 4 & 8 & 16 & 2 & 4 & 2  \\
 %$N_c \rightarrow$ & 2 & 4 & 8 & 16 & 2 & 4 & 2 \\
 \hline
 \hline
 \multicolumn{8}{c}{GPT3-1.3B (FP32 PPL = 9.98)} \\ 
 \hline
 \hline
 64 & 10.40 & 10.23 & 10.17 & 10.15 &  10.28 & 10.18 & 10.19 \\
 \hline
 32 & 10.25 & 10.20 & 10.15 & 10.12 &  10.23 & 10.17 & 10.17 \\
 \hline
 16 & 10.22 & 10.16 & 10.10 & 10.09 &  10.21 & 10.14 & 10.16 \\
 \hline
  \hline
 \multicolumn{8}{c}{GPT3-8B (FP32 PPL = 7.38)} \\ 
 \hline
 \hline
 64 & 7.61 & 7.52 & 7.48 &  7.47 &  7.55 &  7.49 & 7.50 \\
 \hline
 32 & 7.52 & 7.50 & 7.46 &  7.45 &  7.52 &  7.48 & 7.48  \\
 \hline
 16 & 7.51 & 7.48 & 7.44 &  7.44 &  7.51 &  7.49 & 7.47  \\
 \hline
\end{tabular}
\caption{\label{tab:ppl_gpt3_abalation} Wikitext-103 perplexity across GPT3-1.3B and 8B models.}
\end{table}

\begin{table} \centering
\begin{tabular}{|c||c|c|c|c||} 
\hline
 $L_b \rightarrow$& \multicolumn{4}{c||}{8}\\
 \hline
 \backslashbox{$L_A$\kern-1em}{\kern-1em$N_c$} & 2 & 4 & 8 & 16 \\
 %$N_c \rightarrow$ & 2 & 4 & 8 & 16 & 2 & 4 & 2 \\
 \hline
 \hline
 \multicolumn{5}{|c|}{Llama2-7B (FP32 PPL = 5.06)} \\ 
 \hline
 \hline
 64 & 5.31 & 5.26 & 5.19 & 5.18  \\
 \hline
 32 & 5.23 & 5.25 & 5.18 & 5.15  \\
 \hline
 16 & 5.23 & 5.19 & 5.16 & 5.14  \\
 \hline
 \multicolumn{5}{|c|}{Nemotron4-15B (FP32 PPL = 5.87)} \\ 
 \hline
 \hline
 64  & 6.3 & 6.20 & 6.13 & 6.08  \\
 \hline
 32  & 6.24 & 6.12 & 6.07 & 6.03  \\
 \hline
 16  & 6.12 & 6.14 & 6.04 & 6.02  \\
 \hline
 \multicolumn{5}{|c|}{Nemotron4-340B (FP32 PPL = 3.48)} \\ 
 \hline
 \hline
 64 & 3.67 & 3.62 & 3.60 & 3.59 \\
 \hline
 32 & 3.63 & 3.61 & 3.59 & 3.56 \\
 \hline
 16 & 3.61 & 3.58 & 3.57 & 3.55 \\
 \hline
\end{tabular}
\caption{\label{tab:ppl_llama7B_nemo15B} Wikitext-103 perplexity compared to FP32 baseline in Llama2-7B and Nemotron4-15B, 340B models}
\end{table}

%\subsection{Perplexity achieved by various LO-BCQ configurations on MMLU dataset}


\begin{table} \centering
\begin{tabular}{|c||c|c|c|c||c|c|c|c|} 
\hline
 $L_b \rightarrow$& \multicolumn{4}{c||}{8} & \multicolumn{4}{c||}{8}\\
 \hline
 \backslashbox{$L_A$\kern-1em}{\kern-1em$N_c$} & 2 & 4 & 8 & 16 & 2 & 4 & 8 & 16  \\
 %$N_c \rightarrow$ & 2 & 4 & 8 & 16 & 2 & 4 & 2 \\
 \hline
 \hline
 \multicolumn{5}{|c|}{Llama2-7B (FP32 Accuracy = 45.8\%)} & \multicolumn{4}{|c|}{Llama2-70B (FP32 Accuracy = 69.12\%)} \\ 
 \hline
 \hline
 64 & 43.9 & 43.4 & 43.9 & 44.9 & 68.07 & 68.27 & 68.17 & 68.75 \\
 \hline
 32 & 44.5 & 43.8 & 44.9 & 44.5 & 68.37 & 68.51 & 68.35 & 68.27  \\
 \hline
 16 & 43.9 & 42.7 & 44.9 & 45 & 68.12 & 68.77 & 68.31 & 68.59  \\
 \hline
 \hline
 \multicolumn{5}{|c|}{GPT3-22B (FP32 Accuracy = 38.75\%)} & \multicolumn{4}{|c|}{Nemotron4-15B (FP32 Accuracy = 64.3\%)} \\ 
 \hline
 \hline
 64 & 36.71 & 38.85 & 38.13 & 38.92 & 63.17 & 62.36 & 63.72 & 64.09 \\
 \hline
 32 & 37.95 & 38.69 & 39.45 & 38.34 & 64.05 & 62.30 & 63.8 & 64.33  \\
 \hline
 16 & 38.88 & 38.80 & 38.31 & 38.92 & 63.22 & 63.51 & 63.93 & 64.43  \\
 \hline
\end{tabular}
\caption{\label{tab:mmlu_abalation} Accuracy on MMLU dataset across GPT3-22B, Llama2-7B, 70B and Nemotron4-15B models.}
\end{table}


%\subsection{Perplexity achieved by various LO-BCQ configurations on LM evaluation harness}

\begin{table} \centering
\begin{tabular}{|c||c|c|c|c||c|c|c|c|} 
\hline
 $L_b \rightarrow$& \multicolumn{4}{c||}{8} & \multicolumn{4}{c||}{8}\\
 \hline
 \backslashbox{$L_A$\kern-1em}{\kern-1em$N_c$} & 2 & 4 & 8 & 16 & 2 & 4 & 8 & 16  \\
 %$N_c \rightarrow$ & 2 & 4 & 8 & 16 & 2 & 4 & 2 \\
 \hline
 \hline
 \multicolumn{5}{|c|}{Race (FP32 Accuracy = 37.51\%)} & \multicolumn{4}{|c|}{Boolq (FP32 Accuracy = 64.62\%)} \\ 
 \hline
 \hline
 64 & 36.94 & 37.13 & 36.27 & 37.13 & 63.73 & 62.26 & 63.49 & 63.36 \\
 \hline
 32 & 37.03 & 36.36 & 36.08 & 37.03 & 62.54 & 63.51 & 63.49 & 63.55  \\
 \hline
 16 & 37.03 & 37.03 & 36.46 & 37.03 & 61.1 & 63.79 & 63.58 & 63.33  \\
 \hline
 \hline
 \multicolumn{5}{|c|}{Winogrande (FP32 Accuracy = 58.01\%)} & \multicolumn{4}{|c|}{Piqa (FP32 Accuracy = 74.21\%)} \\ 
 \hline
 \hline
 64 & 58.17 & 57.22 & 57.85 & 58.33 & 73.01 & 73.07 & 73.07 & 72.80 \\
 \hline
 32 & 59.12 & 58.09 & 57.85 & 58.41 & 73.01 & 73.94 & 72.74 & 73.18  \\
 \hline
 16 & 57.93 & 58.88 & 57.93 & 58.56 & 73.94 & 72.80 & 73.01 & 73.94  \\
 \hline
\end{tabular}
\caption{\label{tab:mmlu_abalation} Accuracy on LM evaluation harness tasks on GPT3-1.3B model.}
\end{table}

\begin{table} \centering
\begin{tabular}{|c||c|c|c|c||c|c|c|c|} 
\hline
 $L_b \rightarrow$& \multicolumn{4}{c||}{8} & \multicolumn{4}{c||}{8}\\
 \hline
 \backslashbox{$L_A$\kern-1em}{\kern-1em$N_c$} & 2 & 4 & 8 & 16 & 2 & 4 & 8 & 16  \\
 %$N_c \rightarrow$ & 2 & 4 & 8 & 16 & 2 & 4 & 2 \\
 \hline
 \hline
 \multicolumn{5}{|c|}{Race (FP32 Accuracy = 41.34\%)} & \multicolumn{4}{|c|}{Boolq (FP32 Accuracy = 68.32\%)} \\ 
 \hline
 \hline
 64 & 40.48 & 40.10 & 39.43 & 39.90 & 69.20 & 68.41 & 69.45 & 68.56 \\
 \hline
 32 & 39.52 & 39.52 & 40.77 & 39.62 & 68.32 & 67.43 & 68.17 & 69.30  \\
 \hline
 16 & 39.81 & 39.71 & 39.90 & 40.38 & 68.10 & 66.33 & 69.51 & 69.42  \\
 \hline
 \hline
 \multicolumn{5}{|c|}{Winogrande (FP32 Accuracy = 67.88\%)} & \multicolumn{4}{|c|}{Piqa (FP32 Accuracy = 78.78\%)} \\ 
 \hline
 \hline
 64 & 66.85 & 66.61 & 67.72 & 67.88 & 77.31 & 77.42 & 77.75 & 77.64 \\
 \hline
 32 & 67.25 & 67.72 & 67.72 & 67.00 & 77.31 & 77.04 & 77.80 & 77.37  \\
 \hline
 16 & 68.11 & 68.90 & 67.88 & 67.48 & 77.37 & 78.13 & 78.13 & 77.69  \\
 \hline
\end{tabular}
\caption{\label{tab:mmlu_abalation} Accuracy on LM evaluation harness tasks on GPT3-8B model.}
\end{table}

\begin{table} \centering
\begin{tabular}{|c||c|c|c|c||c|c|c|c|} 
\hline
 $L_b \rightarrow$& \multicolumn{4}{c||}{8} & \multicolumn{4}{c||}{8}\\
 \hline
 \backslashbox{$L_A$\kern-1em}{\kern-1em$N_c$} & 2 & 4 & 8 & 16 & 2 & 4 & 8 & 16  \\
 %$N_c \rightarrow$ & 2 & 4 & 8 & 16 & 2 & 4 & 2 \\
 \hline
 \hline
 \multicolumn{5}{|c|}{Race (FP32 Accuracy = 40.67\%)} & \multicolumn{4}{|c|}{Boolq (FP32 Accuracy = 76.54\%)} \\ 
 \hline
 \hline
 64 & 40.48 & 40.10 & 39.43 & 39.90 & 75.41 & 75.11 & 77.09 & 75.66 \\
 \hline
 32 & 39.52 & 39.52 & 40.77 & 39.62 & 76.02 & 76.02 & 75.96 & 75.35  \\
 \hline
 16 & 39.81 & 39.71 & 39.90 & 40.38 & 75.05 & 73.82 & 75.72 & 76.09  \\
 \hline
 \hline
 \multicolumn{5}{|c|}{Winogrande (FP32 Accuracy = 70.64\%)} & \multicolumn{4}{|c|}{Piqa (FP32 Accuracy = 79.16\%)} \\ 
 \hline
 \hline
 64 & 69.14 & 70.17 & 70.17 & 70.56 & 78.24 & 79.00 & 78.62 & 78.73 \\
 \hline
 32 & 70.96 & 69.69 & 71.27 & 69.30 & 78.56 & 79.49 & 79.16 & 78.89  \\
 \hline
 16 & 71.03 & 69.53 & 69.69 & 70.40 & 78.13 & 79.16 & 79.00 & 79.00  \\
 \hline
\end{tabular}
\caption{\label{tab:mmlu_abalation} Accuracy on LM evaluation harness tasks on GPT3-22B model.}
\end{table}

\begin{table} \centering
\begin{tabular}{|c||c|c|c|c||c|c|c|c|} 
\hline
 $L_b \rightarrow$& \multicolumn{4}{c||}{8} & \multicolumn{4}{c||}{8}\\
 \hline
 \backslashbox{$L_A$\kern-1em}{\kern-1em$N_c$} & 2 & 4 & 8 & 16 & 2 & 4 & 8 & 16  \\
 %$N_c \rightarrow$ & 2 & 4 & 8 & 16 & 2 & 4 & 2 \\
 \hline
 \hline
 \multicolumn{5}{|c|}{Race (FP32 Accuracy = 44.4\%)} & \multicolumn{4}{|c|}{Boolq (FP32 Accuracy = 79.29\%)} \\ 
 \hline
 \hline
 64 & 42.49 & 42.51 & 42.58 & 43.45 & 77.58 & 77.37 & 77.43 & 78.1 \\
 \hline
 32 & 43.35 & 42.49 & 43.64 & 43.73 & 77.86 & 75.32 & 77.28 & 77.86  \\
 \hline
 16 & 44.21 & 44.21 & 43.64 & 42.97 & 78.65 & 77 & 76.94 & 77.98  \\
 \hline
 \hline
 \multicolumn{5}{|c|}{Winogrande (FP32 Accuracy = 69.38\%)} & \multicolumn{4}{|c|}{Piqa (FP32 Accuracy = 78.07\%)} \\ 
 \hline
 \hline
 64 & 68.9 & 68.43 & 69.77 & 68.19 & 77.09 & 76.82 & 77.09 & 77.86 \\
 \hline
 32 & 69.38 & 68.51 & 68.82 & 68.90 & 78.07 & 76.71 & 78.07 & 77.86  \\
 \hline
 16 & 69.53 & 67.09 & 69.38 & 68.90 & 77.37 & 77.8 & 77.91 & 77.69  \\
 \hline
\end{tabular}
\caption{\label{tab:mmlu_abalation} Accuracy on LM evaluation harness tasks on Llama2-7B model.}
\end{table}

\begin{table} \centering
\begin{tabular}{|c||c|c|c|c||c|c|c|c|} 
\hline
 $L_b \rightarrow$& \multicolumn{4}{c||}{8} & \multicolumn{4}{c||}{8}\\
 \hline
 \backslashbox{$L_A$\kern-1em}{\kern-1em$N_c$} & 2 & 4 & 8 & 16 & 2 & 4 & 8 & 16  \\
 %$N_c \rightarrow$ & 2 & 4 & 8 & 16 & 2 & 4 & 2 \\
 \hline
 \hline
 \multicolumn{5}{|c|}{Race (FP32 Accuracy = 48.8\%)} & \multicolumn{4}{|c|}{Boolq (FP32 Accuracy = 85.23\%)} \\ 
 \hline
 \hline
 64 & 49.00 & 49.00 & 49.28 & 48.71 & 82.82 & 84.28 & 84.03 & 84.25 \\
 \hline
 32 & 49.57 & 48.52 & 48.33 & 49.28 & 83.85 & 84.46 & 84.31 & 84.93  \\
 \hline
 16 & 49.85 & 49.09 & 49.28 & 48.99 & 85.11 & 84.46 & 84.61 & 83.94  \\
 \hline
 \hline
 \multicolumn{5}{|c|}{Winogrande (FP32 Accuracy = 79.95\%)} & \multicolumn{4}{|c|}{Piqa (FP32 Accuracy = 81.56\%)} \\ 
 \hline
 \hline
 64 & 78.77 & 78.45 & 78.37 & 79.16 & 81.45 & 80.69 & 81.45 & 81.5 \\
 \hline
 32 & 78.45 & 79.01 & 78.69 & 80.66 & 81.56 & 80.58 & 81.18 & 81.34  \\
 \hline
 16 & 79.95 & 79.56 & 79.79 & 79.72 & 81.28 & 81.66 & 81.28 & 80.96  \\
 \hline
\end{tabular}
\caption{\label{tab:mmlu_abalation} Accuracy on LM evaluation harness tasks on Llama2-70B model.}
\end{table}

%\section{MSE Studies}
%\textcolor{red}{TODO}


\subsection{Number Formats and Quantization Method}
\label{subsec:numFormats_quantMethod}
\subsubsection{Integer Format}
An $n$-bit signed integer (INT) is typically represented with a 2s-complement format \citep{yao2022zeroquant,xiao2023smoothquant,dai2021vsq}, where the most significant bit denotes the sign.

\subsubsection{Floating Point Format}
An $n$-bit signed floating point (FP) number $x$ comprises of a 1-bit sign ($x_{\mathrm{sign}}$), $B_m$-bit mantissa ($x_{\mathrm{mant}}$) and $B_e$-bit exponent ($x_{\mathrm{exp}}$) such that $B_m+B_e=n-1$. The associated constant exponent bias ($E_{\mathrm{bias}}$) is computed as $(2^{{B_e}-1}-1)$. We denote this format as $E_{B_e}M_{B_m}$.  

\subsubsection{Quantization Scheme}
\label{subsec:quant_method}
A quantization scheme dictates how a given unquantized tensor is converted to its quantized representation. We consider FP formats for the purpose of illustration. Given an unquantized tensor $\bm{X}$ and an FP format $E_{B_e}M_{B_m}$, we first, we compute the quantization scale factor $s_X$ that maps the maximum absolute value of $\bm{X}$ to the maximum quantization level of the $E_{B_e}M_{B_m}$ format as follows:
\begin{align}
\label{eq:sf}
    s_X = \frac{\mathrm{max}(|\bm{X}|)}{\mathrm{max}(E_{B_e}M_{B_m})}
\end{align}
In the above equation, $|\cdot|$ denotes the absolute value function.

Next, we scale $\bm{X}$ by $s_X$ and quantize it to $\hat{\bm{X}}$ by rounding it to the nearest quantization level of $E_{B_e}M_{B_m}$ as:

\begin{align}
\label{eq:tensor_quant}
    \hat{\bm{X}} = \text{round-to-nearest}\left(\frac{\bm{X}}{s_X}, E_{B_e}M_{B_m}\right)
\end{align}

We perform dynamic max-scaled quantization \citep{wu2020integer}, where the scale factor $s$ for activations is dynamically computed during runtime.

\subsection{Vector Scaled Quantization}
\begin{wrapfigure}{r}{0.35\linewidth}
  \centering
  \includegraphics[width=\linewidth]{sections/figures/vsquant.jpg}
  \caption{\small Vectorwise decomposition for per-vector scaled quantization (VSQ \citep{dai2021vsq}).}
  \label{fig:vsquant}
\end{wrapfigure}
During VSQ \citep{dai2021vsq}, the operand tensors are decomposed into 1D vectors in a hardware friendly manner as shown in Figure \ref{fig:vsquant}. Since the decomposed tensors are used as operands in matrix multiplications during inference, it is beneficial to perform this decomposition along the reduction dimension of the multiplication. The vectorwise quantization is performed similar to tensorwise quantization described in Equations \ref{eq:sf} and \ref{eq:tensor_quant}, where a scale factor $s_v$ is required for each vector $\bm{v}$ that maps the maximum absolute value of that vector to the maximum quantization level. While smaller vector lengths can lead to larger accuracy gains, the associated memory and computational overheads due to the per-vector scale factors increases. To alleviate these overheads, VSQ \citep{dai2021vsq} proposed a second level quantization of the per-vector scale factors to unsigned integers, while MX \citep{rouhani2023shared} quantizes them to integer powers of 2 (denoted as $2^{INT}$).

\subsubsection{MX Format}
The MX format proposed in \citep{rouhani2023microscaling} introduces the concept of sub-block shifting. For every two scalar elements of $b$-bits each, there is a shared exponent bit. The value of this exponent bit is determined through an empirical analysis that targets minimizing quantization MSE. We note that the FP format $E_{1}M_{b}$ is strictly better than MX from an accuracy perspective since it allocates a dedicated exponent bit to each scalar as opposed to sharing it across two scalars. Therefore, we conservatively bound the accuracy of a $b+2$-bit signed MX format with that of a $E_{1}M_{b}$ format in our comparisons. For instance, we use E1M2 format as a proxy for MX4.

\begin{figure}
    \centering
    \includegraphics[width=1\linewidth]{sections//figures/BlockFormats.pdf}
    \caption{\small Comparing LO-BCQ to MX format.}
    \label{fig:block_formats}
\end{figure}

Figure \ref{fig:block_formats} compares our $4$-bit LO-BCQ block format to MX \citep{rouhani2023microscaling}. As shown, both LO-BCQ and MX decompose a given operand tensor into block arrays and each block array into blocks. Similar to MX, we find that per-block quantization ($L_b < L_A$) leads to better accuracy due to increased flexibility. While MX achieves this through per-block $1$-bit micro-scales, we associate a dedicated codebook to each block through a per-block codebook selector. Further, MX quantizes the per-block array scale-factor to E8M0 format without per-tensor scaling. In contrast during LO-BCQ, we find that per-tensor scaling combined with quantization of per-block array scale-factor to E4M3 format results in superior inference accuracy across models. 



\end{document}

