\section{Related Work}
\label{sec:related-work}

\paragraph{Web Dynamics.} Temporal dynamics in web search are an established research topic. Websites change constantly, often more than hourly____, making them relevant for only a limited time____. This relates directly to the observation that many queries are not unique but frequently reissued____. Even the same users tend to repeat the same queries at different points in time____.

\paragraph{Temporal Information Retrieval.} The observed dynamics motivate Temporal Information Retrieval~(TIR) aiming to use temporal information to improve the ranking quality____, e.g., by using temporal patterns for the term weighting____. While TIR focuses on leveraging temporal properties, our work takes a complementary approach by exploiting past document versions rather than directly addressing their temporal aspects.

\paragraph{Query Rewriting with Keyqueries.} Given a set of target documents, a \emph{keyquery} is the minimal query that retrieves the target documents in the top positions____. We adapt this approach to the web search setting by using previously relevant documents as target documents. Keyqueries use terms generated via RM3 or other query expansion approaches as vocabulary for an efficient enumeration of query candidates____. Beyond other query expansion approaches, the keyquery approach also generates a ranking for each candidate to test if all criteria are fulfilled and thereby fully leveraging historical data. 

\paragraph{Evaluations in Dynamic Settings.} Although temporal dynamics can strongly influence the effectiveness of IR systems, they are rarely  considered during evaluation. Soboroff____ initially investigated how temporal dynamics influence test collection evaluations and hypothesized how they could be maintained. Fr{\"o}be et al.____ studied the case when relevance judgments are re-used between different snapshots of crawled documents. Recently, the LongEval shared task____ provides a test bed of an evolving web search scenario covering over a year. Changes in evolving test collections are described through create, update, and delete operations on documents, topics, and relevance judgments____. While the LongEval dataset contains changes in all types of components, this study is mainly concerned with changing documents and ignore, counterfactually, changes in the relevance label.