\section{Dexterous Safety for Articulated Robots}\label{sec:problem}

In this section, we will formulate the problem of dexterous safety for articulated robots such as humanoids.
We first introduce preliminaries, including system modeling, safety specification, and basic indirect safe control approaches that provide theoretical guarantees.
Then, we formulate dexterous safety control where the robot interacts with the environment under multiple safety constraints.
% In addition, we show that infeasible safe control constraints are inevitable in general conditions and must be handled properly to keep the robot operational.

\subsection{Safe Control Preliminaries}

\paragraph{Robot Dynamics}
We consider control-affine robot dynamics with bounded control.
Let $x\in\cX\subset \RR^{N_x}$ be the system state and $u\in\cU$ be the control input.
Let $\cU\defeq\{u\in\RR^{N_u} \mid u^- \leq u \leq u^+ \}$ where $u^-$ and $u^+$ are the lower and upper control limit respectively.
The dynamics is then given by
\begin{equation}\label{eq:dynamics}
    \dot{x} = f(x) + g(x)u, ~ u \in \cU,
\end{equation}
where $f: \RR^{N_x} \mapsto \RR^{N_x}$ and $g: \RR^{N_x} \mapsto \RR^{N_x\times N_u}$ are both locally Lipschitz continuous.

\paragraph{Constrained Robot Tasks}
Let $\cJ$ denote an arbitrary objective function to be minimized by the robot.
For instance, $\cJ$ may measure the relative distance to some goal location in navigation tasks.
The robot should also satisfy some given constraint by staying within $\cX_S$ (i.e., \textit{spec set}), a subset of the state space $\cX$.
$\cX_S$ is assumed to be the zero sublevel set of some piecewise smooth \textit{energy function} $\phi_0\defeq \cX \mapsto \RR$, i.e., $\cX_S \defeq \{x\in \cX \mid \phi_0(x) \leq 0\}$.
Both $\cX_S$ and $\phi_0$ are task specific.
For instance, $\phi_0 = d_\mathrm{min} - d$ keeps the relative distance $d$  to an obstacle above $d_\textrm{min}$, while $\phi_0=\|\hat{z}-[0, 0, 1]^\top\|_2-\epsilon$ keeps the z-axis of the robot upright up to an error of $\epsilon$.
Hence, we are interested in the following constrained task
\begin{align}\label{prob:constrained_task}
\minimizewrt{{u}}~~ & \cJ(x,u)   \\ \nonumber
\st~~ & \phi_0(x)\leq 0 \\ \nonumber
& u\in\cU 
\end{align}

\paragraph{Safe Control Backbone}

\eqref{prob:constrained_task} has been studied by a broad range of literature on indirect safe control methods such as the safe set algorithm (SSA) \cite{liu2014control} and control barrier functions (CBF) \cite{ames2014control}.
Both SSA and CBF derive safe control laws to restrict $\dot{\phi}_0$ following Lyapunov-like conditions and render a \textit{safe set} $\cX_\mathrm{safe}\subseteq\cX_\cS$ forward invariant.
Namely, if the state $x$ is already within $\cX_\mathrm{safe}$, it should never leave that set, and consequently, stay within $\cX_\cS$.
In some cases, control $u$ does not appear in $\dot\phi_0$ (e.g., $\dot\phi_0 =  - \dot d$ does not depend on the acceleration input for a second-order system), preventing the control input from driving the system to safety.
To solve that issue, the safe set algorithm (SSA) \cite{liu2014control} provides a systematic approach to design an alternative energy function $\phi$ to handle general relative degrees ($>1$) between $\phi_0$ and the control.
SSA designs a continuous and piece-wise smooth energy function $\phi \defeq \cX \mapsto \RR$ (a.k.a. the {\textit{safety index}}).
The general form of an $n^\mathrm{th}$ ($n\geq 0$) order safety index $\phi$ is given as
$\phi = (1+a_1 s)(1+a_2 s)\dots(1+a_n s)\phi_0$ where $s$ is the differentiation operator.
$\phi$ should satisfy that (a) the roots of the characteristic equation $\prod_{i=1}^n(1+a_i s) = 0$ are all negative real (to avoid overshooting of $\phi_0$), (b) $\phi_0^{(n)}$ has relative degree one to the control input.
$\phi$ is alternatively expanded to
\begin{equation}\label{def:phi_root}
    \phi \defeq \phi_0 + \textstyle\sum_{i=1}^{n}k_i \phi^{(i)}_0.
\end{equation}
where $\phi_0^{(i)}$ is the $i^\mathrm{th}$ time derivative of $\phi_0$.
Given $\phi$, SSA derives the following control constraint:
\begin{equation}\label{eq:safe_control_law}
    \dot{\phi}(x,u) \leq -\eta~\mathrm{if}~\phi(x) \geq 0 
\end{equation}
for some constant $\eta>0$.
With that, \eqref{prob:constrained_task} becomes
\begin{subequations}\label{prob:single_safe_control}
\begin{align}
\minimizewrt{{u}}~~ & \cJ(x,u)   \\
\st~~ & \dot{\phi}(x,u) \leq -\eta~\mathrm{if}~\phi(x) \geq 0 \label{eq:single_safe_control_phi_constr} \\
& u\in\cU \label{eq:single_safe_control_control_limit}
\end{align}
\end{subequations}
Solving the above optimization yields safe control $u_\mathrm{safe}$ that enforces the safety constraint, i.e.,  $\phi_0\leq 0$ \cite{liu2014control, chen2023sis}.
Notably, CBF also handles general relative degree between $\phi_0$ and the control \cite{wang2023high} and yields a similar constrained control problem to \eqref{prob:single_safe_control}.
In this paper, we focus on SSA-based approaches without loss of generality, since our contributions are in fact compatible with a family of energy-based safe controllers \cite{wei2019unified}, including but not limited to SSA and CBF.

% By \cite{liu2014control,chen2023sis}, if (a) the roots of the characteristic equation $\prod_{i=1}^n(1+a_i s) = 0$ are all negative real, (b) $\phi_0^{(n)}$ has relative degree one to the control input, and (c) the problem \eqref{eq:safe_control_law} is always feasible, both FI and FTC are guaranteed.
% Note that \eqref{eq:safe_control_law} only considers constraint satisfaction which is compatible with arbitrary control objectives.
% For instance, for reference tracking, we can set $\cJ(u) = \|u-u^r\|_p$ to find $u$ that is minimally invasive to the nominal control $u^r$, presumably generated by a given.

\subsection{Dexterous Safety}

% \ruic{motivate multi-constraint SSA from humanoid tasks in real world}

\eqref{prob:single_safe_control} provides a principled approach to enforce a single constraint $\phi_0$ under input limits $\cU$.
% In this paper, we focus on safe humanoid deployments in cluttered environments, where a single $\phi_0$ can rarely suffice.
As discussed in Section \ref{sec:intro}, a single $\phi_0$ can rarely suffice for dexterous safety in cluttered environments.
% In household scenarios, for example, humanoids need to avoid collisions against multiple objects that can be static (e.g., wall, furniture) or dynamic (e.g., humans).
% The number of constraints may also vary on the air, for instance, when humanoids navigates to different locations.
Hence, we consider one constraint for each robot body and obstacle pair.
We will end up with a set of $M$ constraints, with $\phi_{0,i}$ being the energy function for the $i^\mathrm{th} (i\in[M])$ collision pair and $\cX_{\cS,i}$ the corresponding spec set.
Let $\phi_i$ be the corresponding $n^\mathrm{th}$ order safety index for $\phi_{0,i}$ similar to \eqref{def:phi_root}, \eqref{prob:single_safe_control}
can be naturally extended to a multi-constraint version:
\begin{subequations}\label{prob:multi_safe_control}
\begin{align}
\minimizewrt{{u}}~~ & \cJ(x,u)  \\
\st~~ & \dot{\bphi}(x,u) \leq -\bETA~\mathrm{if}~\bphi(x) \geq 0  \label{eq:multi_safe_control_phi_constr} \\ 
& u\in\cU \label{eq:multi_safe_control_contol_limit}
\end{align}
\end{subequations}
where $\bphi\defeq[\phi_1,\phi_2,\dots,\phi_M]^\top$ and $\bETA\defeq [\eta_1, \eta_2, \dots, \eta_M]^\top$, $\eta_i > 0$.
% Note that \eqref{eq:multi_safe_control_phi_constr} is a collection of $M$ different safe control constraints.
If \eqref{prob:multi_safe_control} can be solved at all times, some safe set $\cX_{\mathrm{safe},i}\subseteq\cX_{\cS,i}$ will be rendered forward invariant for each constraint $i$.
Then, all constraints are enforced and our problem is well solved.
However, for dexterous safety in cluttered environments, \eqref{prob:multi_safe_control} can easily be infeasible, as will be explained next, making our problem particularly challenging.

Without loss of generality, we assume $\cJ$ to be a quadratic objective.
Such an assumption makes the safe control problem a quadratic programming (QP) which can be efficiently solved by off-the-shelf solvers.
In the rest of this paper, we refer to safe control problems like \eqref{prob:single_safe_control} and \eqref{prob:multi_safe_control} as ``QP'' for simplicity.

\subsection{Infeasible Safe Control Problems}\label{sec:infeas_safe_control_problem}

% \ruic{mention sis as one approach to guarantee feasible safe control}

% \ruic{show synthesis under multiple constraints lead to overly restricted operating conditions}

To analyze the feasibility of multi-constraint QPs, we first see how the single-constraint QP case works.
% To achieve safety, the optimization \eqref{prob:single_safe_control_phi_constr} itself must be feasible in the first place.
\eqref{prob:single_safe_control} is feasible when there exists a control within $\cU$ to reduce positive $\phi$ values for all states in $\cX$.
This objective can normally be achieved by properly selecting $k_i$ coefficients in \eqref{def:phi_root} via \textit{Safety Index Synthesis} (SIS) \cite{zhao2023sos, chen2023sis}, formally given by \Cref{problem:synthesis}.
\begin{problem} [Safety Index Synthesis] \label{problem:synthesis}
    Find safety index as $\phi \defeq \phi_0 + \sum_{i=1}^{n}k_i \phi^{(i)}_0$ with parameter $\theta\in\Theta\defeq\{[k_1,k_2,\dots,k_{n}] \mid k_i\in\RR,k_i\geq 0,\forall i \}$, such that
    \begin{equation}\label{eq:synthesis}
        \forall x\in\cX ~ \st ~ \phi(x)\geq 0, \minimizewrt{u\in\cU} \dot{\phi}(x,u) < -\eta
    \end{equation}
\end{problem}
SIS is non-trivial to solve since the condition in \eqref{eq:synthesis} must hold for a continuous state space, leading to an infinite number of constraints.
% SIS solves an energy function $\phi$ (parameterized by $k_i$) that makes \eqref{prob:single_safe_control_phi_constr} always feasible.
In the literature, SIS has been solved via sum-of-square programming \cite{zhao2023sos, chen2023sis} which is already expensive for single-constraint cases, and normally guarantees QP feasibility only in restricted state, control, and task spaces.
Notably, the synthesis of a single feasible energy function $\phi$ primarily concerns about the compatibility between the control constraint \eqref{eq:single_safe_control_phi_constr} and the control limits \eqref{eq:single_safe_control_control_limit}, since single-constraint QPs with unbounded control are always feasible \cite{liu2014control}.
If there are multiple energy functions $\{\phi_i\}$ (more than 200 in our case), SIS needs to additionally handle potential inconsistencies within $\phi_i$'s themselves, which is extremely challenging.
SIS approaches that handle such problems are not found in the literature yet.
% Synthesizing provably safe controllers with multiple constraints has also been studied \cite{breeden2023compositions}.
% However, due to its sampling nature, the approach in \cite{breeden2023compositions} is computationally demanding and spent non-trivial computation time on a low dimensional numerical example.
% To the best of the authors' knowledge, there is no existing synthesis approach that practically handles high-dimensional multi-constraint control problems in general conditions.

In this paper, we argue that, it is intractable to make \eqref{prob:multi_safe_control} or any similar formulations always feasible for dexterous safety in cluttered environments under actuation bounds.
To help illustrations, we explain possible infeasibility of three types: inherent infeasibility, method infeasibility, and kinematics infeasibility (see \cref{fig:infeas_analysis}).

\paragraph{Inherent Infeasibility}
In cluttered environments with dynamic obstacles, there can be cases where collisions are physically inevitable.
For instance, when an obstacle moves towards the base of a humanoid that is already unable to move due to other constraints (see \cref{fig:infeas_1}), there is no control that can prevent collisions, making \eqref{prob:multi_safe_control} infeasible.
Such inherent infeasibility can hardly be avoided without restricting the operation conditions of humanoids.



\begin{figure}[htbp]
    \centering
    % First subfigure
    \begin{subfigure}[b]{0.3\linewidth}
        \centering
        \includegraphics[width=\linewidth]{figure/infeas_1.png}
        \caption{Inherent Infeasibility}
        \label{fig:infeas_1}
    \end{subfigure}
    \hfill
    % First subfigure
    \begin{subfigure}[b]{0.3\linewidth}
        \centering
        \includegraphics[width=\linewidth]{figure/infeas_2.png}
        \caption{Method Infeasibility}
        \label{fig:infeas_2}
    \end{subfigure}
    \hfill
    % First subfigure
    \begin{subfigure}[b]{0.32\linewidth}
        \centering
        \includegraphics[width=\linewidth]{figure/infeas_3.png}
        \caption{Kinematics Infeasibility}
        \label{fig:infeas_3}
    \end{subfigure}
    \caption{Possible scenarios where \eqref{prob:multi_safe_control} can be infeasible. The humanoid should avoid collision with all obstacles (e.g., planes and spheres) in gray.}
    \label{fig:infeas_analysis}
\end{figure}

\paragraph{Method Infeasibility}
Since each $\phi_i$ in \eqref{eq:multi_safe_control_phi_constr} concerns a robot body and obstacle pair, different $\phi_i$'s may generate conflicts in some cases.
Consider the collision avoidance between the humanoid wrist and two planar obstacles.
One can construct an energy function for each wrist-obstacle pair, e.g., $\phi_i = d_\mathrm{min} - d_i$ where $d_i$ is the distance from the surface of the wrist to plane $i$ for $i=1,2$ .
According to \eqref{eq:multi_safe_control_phi_constr}, the corresponding control constraints will be $\dot{\phi}_i = -\dot{d}_i = -v_i \leq -\eta_i \Rightarrow v_i \geq \eta_i$ which simply constrains the wrist to move away from the plane with at least $\eta_i$ velocity ($\eta_i>0$) when $d_i\leq d_\mathrm{min}$.
Such design is mostly effective assuming accurate velocity tracking.
However, in some cases, even those two control constraints may not be compatible.
For example, when the humanoid operates between two planes parallel to each other (e.g., when manipulating objects in shelves) with a total distance to both sides $d_1+d_2$ less than $d_\mathrm{min}$, both $\phi_1$  and $\phi_2$ are non-negative.
Then, each $\phi_i$ would require the wrist to move towards an opposite direction, i.e., $v_1 \geq \eta_1 > 0$ and $v_2 \geq \eta_2 > 0$, which is impossible and renders \eqref{prob:multi_safe_control} infeasible (see \cref{fig:infeas_2}).
However, the humanoid can simply move the arm parallel to the planes until exiting from the openings to be safe (green direction in \cref{fig:infeas_2}).
Hence, different from the previous case, a safe control is not impossible in this case, but rather not found by the QP \eqref{prob:multi_safe_control} with the designed $\phi_i$.
We refer to such failures as method infeasibility.
Although the QP in certain cases (e.g., the example above) can be made feasible by re-designing the control law, it hardly generalizes since we cannot know all possible obstacle configurations beforehand in cluttered environments.

\paragraph{Kinematics Infeasibility}

Infeasibility can also be caused by the complex kinematics chain of humanoids.
Consider a humanoid avoiding hand collision with the sphere nearby in \cref{fig:infeas_3}.
With a similar definition of $\phi_i$ to the above, the control constraints \eqref{eq:multi_safe_control_phi_constr} would require the hand $i$ to move away with a velocity of $v_i \geq \eta_i$ for $i=1,2$ when being too close to the obstacle.
Restricted by the kinematics capability, the maximum velocity at which the hand can move depends on the actual joint configurations.
In \cref{fig:infeas_3}, with a more extended arm pose, the left hand would have a higher velocity limit than the right hand.
Hence, a relatively large $\eta_i$ may work for the left hand but make the QP infeasible when handling the right hand.
While a universally small $\eta_i$ may improve QP feasibility, that would make the robot overly insensitive to potential collisions.
Designing energy functions $\phi_i$ to be compatible with each body on the humanoid in general cases remains a challenge.

The environment configurations shown in \cref{fig:infeas_analysis} are merely toy examples while still making \eqref{prob:multi_safe_control} infeasible.
To address dexterous safety in cluttered environments, we will have significantly more constraints to handle the complex obstacle configurations.
Several scenarios like those in \cref{fig:infeas_analysis} may even be coupling, leading to QP infeasibility that can hardly be mitigated in practice.
In that regard, we do not aim to construct persistently feasible QPs but instead desire a practical method that minimizes violations of control constraints when the QP becomes infeasible.

% We will first show that even with reduced systems and simplified obstacle geometries, formal synthesis is only possible with restrictions on the operating conditions.
% Then, we will show that such reduced cases are in fact representative in general humanoid tasks, rendering \eqref{prob:multi_safe_control} feasible only in highly restricted conditions.

% Consider the scenario in \cref{fig:infeas_sis}, where a dual-link planar robot arm is required to keep its end-effector at least $d_\mathrm{min}$ from each of the two walls.

% \begin{figure}[h]
%   \centering
%   \includegraphics[width=0.6\linewidth]{figure/infeas_example.png}
%   \caption{Illustration of a reduced safe control problem.}
%   \label{fig:infeas_sis}
% \end{figure}

% Assume that the robot arm is governed by first-order integrator dynamics with velocity input.
% We can construct $\phi_i=d_\mathrm{min}-d_i$ to consider collision-avoidance with wall $i$.
% Then, the safe control law is given by
% \begin{equation}\label{eq:infeas_example_control_law}
%     \dot{\phi}_i = -\dot{d}_i \leq -\eta ~ \mathrm{if} ~ d_i \leq d_\mathrm{min}
% \end{equation}
% When both safe control constraints are first active (i.e., $\phi_1 = 0$ and $\phi_2 = 0$), the end-effector needs to escape form both walls to satisfy \eqref{eq:infeas_example_control_law}.
% Assume that the end-effector is to move along the center line between the two walls, the relation between $\dot{\phi}_i$ and the end-effector velocity $v$ can be written as
% \begin{equation}\label{eq:infeas_example_v_phi}
%     \dot{\phi}_i = -\dot{d}_i = -v\sin(\theta/2)
% \end{equation}
% Plugging in the control law, we have
% \begin{equation}\label{eq:infeas_example_v_req}
%     v\sin(\theta/2) \geq \eta
% \end{equation}
% We can see that with a smaller angle formed by the two walls, the end-effector needs to move at inversely proportionally larger velocities to remain safe.
% In other words, with limited joint actuation and end-effector velocities, \eqref{eq:infeas_example_control_law} is feasible only for any planar surface pairs that forms an angle larger than $2\sin^{-1}(\eta/v_\mathrm{max})$.
% Now, consider humanoids in cluttered environments.
% The environment geometry can be highly complex, potentially composed of numerous pairs of intersecting planar surfaces.
% \eqref{prob:multi_safe_control} can only be feasible when all planar surface pairs satisfy the above derived conditions, which greatly limits the curvature of environment geometries that should be avoided.
% Furthermore, obstacle geometries in cluttered environments may not even be connected, which, if considered, would further restrict the feasible environment geometries.

% The above example reveals a hard trade-off between the feasibility of safe control problems and robot operating conditions.
% For humanoids to function effectively in cluttered environments, we must allow generous operating conditions which makes it impractical to enforce the feasibility of \eqref{prob:multi_safe_control}.
% Namely, the safe control constraints in \eqref{prob:multi_safe_control} will frequently be infeasible under bounded actuation limits.
% In the following section, we propose several approaches to robustly generate proper robot control even when the safe control problem becomes infeasible.

% Notably, the infeasibility we handle regards our specific safe control problem \eqref{prob:multi_safe_control} derived from SSA (or any alternative safe control backbone used) only.
% When \eqref{prob:multi_safe_control} is infeasible, it does not necessarily indicate that collision avoidance is physically impossible, but rather that the specific problem cannot return a feasible safe control signal.
% It is possible that alternative safe control approaches generate feasible control in the same condition.
% Nevertheless, investigating such possibility deviates from the focus of this paper.
% % We build on a natural extention of SSA, which provides theoretical safety guarantees \cite{zhao2021zeroviolation, zhao2022provably, zhao2023sos, chen2023sis}, to multi-constraint scenarios.
% In this work, we address the infeasibility of the chosen safe control approach, while whether safety is physically impossible is not of focus.
