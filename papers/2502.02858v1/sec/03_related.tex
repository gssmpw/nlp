\section{Related Works}
\label{sec:related}

% In this section, we review existing works that address safety in robot control which can be divided into two classes: indirect (model-based) methods and direct (model-free) methods.
In this section, we review existing works that address safety in robot control in indirect (model-based) fashion.
Indirect approaches require explicitly modeled robot dynamics and derive safe control laws according to safety specifications.
% In contrast, direct approaches skip the dynamics modeling and directly incorporate both objective and constraints into a single optimization framework.



% \textbf{Indirect safe control.}

Indirect safe control backbones include safe set algorithms (SSA)  \cite{liu2014control, chen2023sis}, control barrier functions (CBFs) \cite{ames2016control, xiao2019control}, and Hamilton–Jacobi (HJ) reachability \cite{choi2021robust}.
While being different in the composition of safe control laws, all the above approaches quantifies safety using an energy function \cite{wei2019unified}.
With modeled system dynamics, optimization-based safe control laws are derived to drive the energy function below (or above if negating the sign) a critical level, satisfying Lyapunov-like conditions which guarantees invariance within a safe set.
In the past decade, indirect safe control methods have been applied to a wide range of applications such as safe quadruped navigation \cite{yun2024safe, he2024agile, molnar2021model}, safe human-robot interaction \cite{liu2023proactive, lin2017real, pandya2024multimodal, liu2022safe}, safe learning \cite{zhao2021zeroviolation}, locomotion \cite{choi2023constraint}, and high-speed drones \cite{singletary2022onboard}.
Prior work also studies more general settings with time-varying factors \cite{glotfelter2019hybrid, chen2023sia}, model uncertainty \cite{taylor2020adaptive, dawson2022safe}, and model mismatch \cite{taylor2020learning}.
All the above works only consider a single safety constraint, making the optimization for safe control normally feasible given adequate control limits.
In dexterous safety in cluttered environments, however, we often need multiple safety constraints to fully capture the safety conditions which also make the optimization sometimes infeasible.
As mentioned in Section \ref{sec:intro}, there is not existing work that handles such tasks with feasibility guarantees.
In a cluttered dynamic environment, there may also exist situations that are physically impossible for the humanoid to escape from (e.g., trapped in a crowd of people).
Prior work usually ignore these situations.
In those cases, the question is indeed not to satisfy the safety specification anymore, but how to achieve minimal violations.
With minimally relaxed control constraints, our approaches automatically fulfill that purpose as well.

% resulting in a lack of feasible robot control.
% Some existing works consider multi-constraint safety, but do not focus on optimization feasibility.
% For example, the constraints in \cite{djeha2023robust, nguyen20163d} are inherently consistent due to connected and closed safe zones.
% \cite{khazoom2022humanoid} assumes large actuation limits.
% \cite{liu2022inputsat} explicitly avoids infeasibility by constructing feasible safety filters, but focuses on a single safety constraint under input saturation instead of multiple safety constraints.
% In our dexterous safety task, the complex humanoid–environment interaction can generate inherently conflicting safety constraints, even without considering input limits.
% Our approach reliably computes safe actions in these challenging conditions, enabling safety in more complex and demanding scenarios than prior methods.

 

% \textbf{Direct safe control.}

% \ruic{brief review of safe RL line of work.}