
\section{Introduction}\label{sec:intro}


% humanoids widely applied;
% good capability, at the same time safety is critical;
% critical performance-safety trade-off;
% this paper: humanoid safety without compromising performance, with flexibility in configuring both functionality (e.g., goal reach, manipulation pose) and safety (e.g., collision).

Humanoid robots are increasingly being deployed in the real world.
Their highly articulated physical structures grant them remarkable dexterity to operate even in cluttered environments; those environments are commonly seen in tasks from factory automation to service and healthcare applications.
Yet, as their capabilities grow, ensuring the safety of the humanoid and its environment becomes both critical and difficult.
% Safety is especially challenging for humanoids because their motions can be highly constrained due to complex environmental geometries.
% In context of collision avoidance, naive approach models robot as sphere, applied in \cite{yun2024safe, he2024agile};
% However, such reduced modeling is over-simplified and inefficient for humanoid since it's conservative.
% Unlock full capability of humanoid needs fine-grained modeling of the geometry to consider more complex interaction w/ environment.
% Refer to as \textbf{dexterous safety}.
% Full model implies:
% \begin{itemize}
%     \item Large number of constraints: multiple controlled rigid bodies, self-collision.
%     \item Joint consideration of task objectives, task constraints, safety constraints.
%     \item High dimension states and control.
% \end{itemize}
A common approach to collision avoidance is to simplify the robot’s geometry—for instance, by wrapping the entire body using a single bounding cylinder \cite{yun2024safe, he2024agile}.
While such geometric reductions enable straightforward algorithms and help in early proof-of-concept implementations, they tend to be overly conservative for humanoid robots, severely restricting feasible motion and task performance.
Realizing the full potential of humanoids calls for more precise and fine-grained modeling of the geometry of both the robot and the environment, especially for tasks involving close-proximity interactions.
We refer to the problem of enforcing safety under limb-level geometry modeling as \textit{dexterous safety}.
In this paper, we aim to address dexterous safety in cluttered environments without compromising performance.

% Given geometry model, safety can be achieved via either direct or indirect safe control.
% \begin{itemize}
%     \item Indirect / Model based: ssa \cite{liu2014control, yun2024safe}, cbf \cite{djeha2023robust, khazoom2022humanoid}, hjb \cite{choi2021robust}.
%     \begin{itemize}
%         \item Pro: Interpretable (decomposing control in objective and constraint oriented), incorporate new constraints without training (directly specify control law), scalable to high dimensional systems.
%         \item Con: Need manual design to jointly consider objective and constraints; prone to model mismatch.
%     \end{itemize}
%     \item Direct / Model free: safe-RL.
%     \begin{itemize}
%         \item Pro: No need for modeling, easier to specify objectives and constraints, direct joint optimization.
%         \item Con: Hard to adapt to unseen constraints (extra training needed).
%     \end{itemize}
% \end{itemize}
% This paper focus on indirect method to facilitate the design and analysis of robot behavior against multi-constraint scenarios.

% \begin{figure}[t]
%   \centering
%   \includegraphics[width=0.8\linewidth]{example-image-a}
%   \caption{teaser.}
%   \label{fig:sample}
% \end{figure}

% \textbf{NOTE}: both handle manul constraints (RL-many reward terms, SSA-multiple lin constraints); both have difficulty satisfying multi constraints (RL-insufficient sample, SSA-conflicting constraints needing relaxation)

Regarding approaches to safe control, researchers have broadly pursued two classes of methods: indirect (model-based) and direct (model-free).
Indirect approaches such as safe set algorithms (SSA)  \cite{liu2014control, chen2023sis}, control barrier functions (CBFs) \cite{ames2016control, xiao2019control}, and Hamilton–Jacobi (HJ) reachability \cite{choi2021robust} explicitly model the robot dynamics and derive safe control laws accordingly.
They allow interpretable decompositions of objectives and constraints and can be adapted to modified problems (e.g., changed safety criteria) without a fine-tuning process.
% Nonetheless, these methods rely on carefully formulated solutions and suffer from inaccurate models even in simulation.
In contrast, direct approaches such safe reinforcement learning bypass robot modeling and incorporate tasks with constraints into a single optimization framework.
Despite their promise, direct methods often require extensive training for high-dimensional problems and retraining when new constraints are introduced.
This paper focuses on indirect methods to accommodate the high dimensionality and flexibility of dexterous safety in cluttered environments.

% Indirect methods often derive safety constraints that are linear in control such that the safe control law can be solved via quadratic programming (QP) \cite{liu2014control, ames2016control}.
% To achieve that, the resulting QP problems must be feasible at every control step.
% When composing the QP, most prior work consider a single safety constraint \cite{liu2022safe, yun2024safe, he2024agile, singletary2022onboard, liu2023proactive, lin2017real, dawson2022safe}.
% In those works, the QP is normally feasible, especially when the control limit is large.
% The single-constraint assumption, however, does not hold in challenging scenarios with multiple external obstacles, potential self-collision, and any other constraints required by task achievement.
% Representative methods with theoretical guarantees include .

Indirect methods are normally energy function-based approaches \cite{wei2019unified} that (a) synthesize an energy function that quantifies safety and (b) derive a control constraint from that energy function to enforce safety.
The difficulty of applying indirect methods can be characterized by two key aspects: the granularity of robot geometry modeling and the distribution of obstacles.
The robot geometry can be modeled either coarsely using a single convex bounding shape such as a box or cylinder (i.e., \textbf{single-body}), or at limb level with each movable link enclosed with an independent shape (i.e., \textbf{multi-body}).
The surrounding obstacles can be distributed either \textbf{sparsely} such that the robot deals with one obstacle at a time, or \textbf{densely} such that the robot has to jointly consider multiple potential collisions.
Prior works mostly focus on the simplest setting with a single rigid body in sparse environments \cite{yun2024safe, he2024agile, singletary2022onboard, pandya2024multimodal, molnar2021model, zhao2021zeroviolation, choi2023constraint}.
Some work could handle single-body safety in dense environments~\cite{chen2021safe, dawson2022safe, zheng2022clustered} by invoking one safety constraint for each surrounding obstacle. However, these approaches are limited to simple 2D problems and do not come with any safety guarantees. 
Regarding multi-body safety, existing literature mostly assumes sparse environments and solves these problems by reducing them to maximizing the closest distance from the robot to the obstacle which could be regulated using one single energy function \cite{liu2023proactive, lin2017real, liu2022safe}.
%For single-body safety in dense environments, \cite{chen2021safe} uses a single energy function considers the closest obstacle only.
%All the above works use a single energy function to fully characterize the safety, resulting in optimizations with a single control constraint other than control limits.

Our problem of dexterous safety for humanoids in cluttered environments falls into the most difficult category: multi-body safety with densely distributed obstacles.
In all three cases above with either single body or sparse obstacles, the safety is quantified with respect to one point on the robot body. In the case of single-body safety, it is quantified with respect to the center of the robot; in the case of multi-body safety, it is quantified with respect to the closest point to the obstacle. Hence only one energy function is needed for the robot. For relatively sparse environments with high safe control frequency, this strategy works for multi-body safety as the clearance is high and safety hazards could be mitigated in time. 
However, the complex interaction between multiple robot links and multiple obstacles in proximity can hardly be treated in the same way.
That is because some motion that decreases this one energy function (e.g., by moving the closest point away from the obstacle) may result in another part on the robot body immediately colliding with another obstacle. 
%In fact, that may even cause undesired behaviors for a single body in cluttered environments due to the lack of collision information at each control step \cite{chen2021safe}.
To the best of the authors' knowledge, there is no existing work that can reliably synthesize one single energy function to achieve high safety performance when multiple collisions with multi-body robots are possible. 
We hypothesize that multi-body safety in dense environments is indeed a \textit{multi-objective optimization problem} which can hardly be captured by a single energy function as one single performance index. 
Hence, we must deploy \textit{multiple energy functions} to more precisely capture the safety.
With each energy function leading to a control constraint, we resort to multi-constraint safe control approaches in this paper.

% Control constraints derived from energy functions can be combined with arbitrary task objectives to formulate an optimization to solve for control.
The aforementioned safety approaches (SSA, CBF, etc.) come with nice theoretical safety guarantees such as forward invariance \cite{liu2014control, chen2023sis, ames2016control}.
They often derive a linear safe control constraint with a quadratic objective (e.g., nominal control tracking), forming quadratic programming (QP) problems to solve online.
In our case, this naturally extends to a QP with multiple constraints.
The solution (if it exists) to this QP would satisfy each control constraint and inherit all safety guarantees of the single-constraint version.
However, as will be shown in this paper, multi-constraint QPs can frequently become infeasible either due to inherently/physically infeasible problems or incompatible energy functions (e.g., derived control constraints), unless certain safety requirements are relaxed (e.g., temporarily allowing contact).
This is because, in dexterous safety, we need to carefully constrain the motion of each rigid body in the robot kinematics chain, which is very different from safety for a single rigid body.
That leads to a highly restricted solution space with or without bounded control.
% To help ensuring QP feasibility, one can indeed deploy a single surrogate constraint to govern all of the above, for example, by only handling the current most critical constraint (e.g., avoiding the closest obstacle).
% However, such an approach would cause racing conditions with multiple critical constraint present, leading to unstable safety behaviors.
% Hence, this paper adopts a more general formulation which considers an arbitrary number of control constraints and handles potential incompatibilities.
Several previous works consider multi-constraint cases but without a focus on QP feasibility, either due to inherently consistent constraints describing non-trivial connected and closed safe zones \cite{djeha2023robust, nguyen20163d} or large actuation limits \cite{khazoom2022humanoid}.
There are works explicitly avoiding QP infeasibility via Safety Index Synthesis but only focus on a single safety constraint, mostly with problem dimensions no more than five \cite{zhao2023sos, chen2023sis, chen2023sia, liu2022inputsat}.
\cite{breeden2023compositions} composes multiple constraints into a feasible controller but suffers poor scalability; it only shows success on a four-dimensional problem with two constraints, which takes over 1000 seconds to compute.
In the task to be considered in this paper, we model a 29-DoF Unitree G1 humanoid with 19 collision volumes, with more than 10 obstacles in proximity at a time in cluttered environments.
Considering self-collision, our QP contains more than 200 constraints in dexterous safety tasks.
There does not exist any method that can synthesize multiple compatible energy functions to make the QP persistently feasible for high-dimensional problems like ours.

% For dexterous safety, the complex robot-environment interaction normally generates multiple inconsistent safety constraints even with unbounded input limits.
% In fact, infeasible constraints are inevitable in general, which will be shown later in this paper.
% Hence, generating proper safe actions under infeasible constraints is critical for safe humanoid deployments.

% Prior work often assumes a single safety condition or large actuation limits that help ensure QP feasibility.
% Such assumptions, however, do not hold in challenging scenarios with multiple potential collisions (e.g., environmental obstacles plus self-collision checks) and restricted control inputs.
% In these cases, the problem may easily become infeasible if multiple constraints conflict.
% Recognizing that infeasibility is inevitable in certain high-dimensional, multi-constraint settings, this paper proposes a safe control methodology with bounded constraint violation.
% We employ weighted slack variables within the QP framework to prioritize key constraints while allowing minor, controlled violations of lower-priority ones. 

% \textbf{Paper proves inevitable inconsistency in general case and propose safe control method with bounded safety violation.}
% Features of new approach:
% \begin{itemize}
% \item Constraint relaxation via weighted slack variables to ensure practical QP feasibility
%     \item  Free adaptation to varying constraints during deployment
%     \begin{itemize}
%         \item regulated constraint satisfaction (dof contribution to safety)
%         \item re-configured constraints (add/remove/change constraints)
%     \end{itemize}
% \end{itemize}

% This paper:
% \begin{itemize}
%     \item proves infeasible QP exists
%     \item analyze diff from single constraint est. multi collision vol
%     \item introduce method to mitigate multi-constraint infeasibility with bounded violation
%     \item compare to baseline multi-constraint safe control
%     \item [deprecated] compare to safe RL on a range of models (reduced to full) in sim
%     \item verify on hardware
% \end{itemize}

% In this paper, we show that constraint incompatibility is unavoidable under general scenarios for humanoid robots, and propose a novel multi-constraint safe control approach with minimal safety violations.
% To support flexible performance-safety trade-off, our approach is also designed to allow tunable regularization on safety behaviors and modifications to safety constraints.

In summary, for dexterous safety for humanoids in cluttered environments, a single energy function can hardly handle multiple safety objectives, while no existing approach could synthesize multiple energy functions that are compatible to guarantee feasible controls.
We argue that under the currently available theoretical tools, the closest to what we desire is to design multiple energy functions for each collision pair of interest and force a multi-constrained QP.
Instead of aiming for guarantees, we desire \textit{a practical method that can be quickly deployed on humanoids with minimal violations of the control constraints when the QP becomes infeasible}, either due to inherently infeasible situations or incompatible energy functions.
To this end, we propose the Relaxed Safe Set Algorithm (r-SSA), which relaxes infeasible control constraints with weighted slack regularization.
We then introduce the Projected Safe Set Algorithm (p-SSA), which improves over r-SSA by removing parameter tuning via decoupled optimization for feasibility and task objectives.
Validation in both simulation and hardware shows that r-SSA can readily compute safe control in challenging dexterous safety tasks with several hundreds of constraints.
At the same time, p-SSA gains top performance across a variety of tasks with zero parameter tuning.
Our contributions can be summarized as follows.
\begin{itemize}
    \item We introduce a novel task of dexterous safety in cluttered environments and analyze the challenges faced by safe control approaches regarding infeasibility from multiple sources.
    \item We propose the Projected Safe Set Algorithm (p-SSA), a novel safe control method for dexterous safety that relaxes conflicting control constraints with minimal violations.
    \item We compare p-SSA to baseline methods in simulated experiments and show top balance between performance and safety across various tasks without parameter tuning.
    \item We verify p-SSA on a Unitree G1 humanoid robot in challenging safe tele-operation tasks.
\end{itemize}

The rest of the paper is organized as follows.
Section \ref{sec:related} reviews related indirect safe control approaches.
Section \ref{sec:problem} formulates the dexterous safe control problem and shows the conditions under which infeasibility arises.
Section \ref{sec:method} presents r-SSA and p-SSA to address infeasible safe control constraints.
In Section \ref{sec:experiment}, we present simulation results and hardware demonstrations of dexterous safety, followed by limitations in Section \ref{sec:limitation} and conclusions in Section \ref{sec:conclusion}.

 

