% 

\begin{figure}[tp]
    \centering
    \renewcommand{\arraystretch}{1} % Adjust row spacing
    \setlength{\tabcolsep}{1pt} % Adjust column spacing
    \begin{tabular}{c c c c c c}
        & Laptop & Person & Dog & Bench & Couch \\ 
        \rotatebox{90}{\hspace{12pt}Image} & 
        \includegraphics[width=0.183\linewidth]{figs/ZS3_img/image1.png} & 
        \includegraphics[width=0.183\linewidth]{figs/ZS3_img/image2.png} & 
        \includegraphics[width=0.183\linewidth]{figs/ZS3_img/image3.png} & 
        \includegraphics[width=0.183\linewidth]{figs/ZS3_img/image4.png} & 
        \includegraphics[width=0.183\linewidth]{figs/ZS3_img/image5.png} \\ 
        \rotatebox{90}{\hspace{12pt}CLIP} & 
        \includegraphics[width=0.183\linewidth]{figs/ZS3_img/clip1.png} & 
        \includegraphics[width=0.183\linewidth]{figs/ZS3_img/clip2.png} & 
        \includegraphics[width=0.183\linewidth]{figs/ZS3_img/clip3.png} & 
        \includegraphics[width=0.183\linewidth]{figs/ZS3_img/clip4.png} & 
        \includegraphics[width=0.183\linewidth]{figs/ZS3_img/clip5.png} \\ 
        \rotatebox{90}{\hspace{12pt} CS} & 
        \includegraphics[width=0.183\linewidth]{figs/ZS3_img/clipsurgery1.png} & 
        \includegraphics[width=0.183\linewidth]{figs/ZS3_img/clipsurgery2.png} & 
        \includegraphics[width=0.183\linewidth]{figs/ZS3_img/clipsurgery3.png} & 
        \includegraphics[width=0.183\linewidth]{figs/ZS3_img/clipsurgery4.png} & 
        \includegraphics[width=0.183\linewidth]{figs/ZS3_img/clipsurgery5.png} \\ 
        \rotatebox{90}{\hspace{1pt}CLIP-VV}& 
        \includegraphics[width=0.183\linewidth]{figs/ZS3_img/clip_vv1.png} & 
        \includegraphics[width=0.183\linewidth]{figs/ZS3_img/clip_vv2.png} & 
        \includegraphics[width=0.183\linewidth]{figs/ZS3_img/clip_vv3.png} & 
        \includegraphics[width=0.183\linewidth]{figs/ZS3_img/clip_vv4.png} & 
        \includegraphics[width=0.183\linewidth]{figs/ZS3_img/clip_vv5.png} \\
        \rotatebox{90}{\hspace{12pt}Ours}& 
        \includegraphics[width=0.183\linewidth]{figs/ZS3_img/ours1.png} & 
        \includegraphics[width=0.183\linewidth]{figs/ZS3_img/ours2.png} & 
        \includegraphics[width=0.183\linewidth]{figs/ZS3_img/ours3.png} & 
        \includegraphics[width=0.183\linewidth]{figs/ZS3_img/ours4.png} & 
        \includegraphics[width=0.183\linewidth]{figs/ZS3_img/ours5.png} \\ 
    \end{tabular}
    \caption{\textbf{Qualitative Comparison on ZS3.} Visualization of zero-shot semantic segmentation (ZS3) results for CLIP \cite{clip}, CLIP Surgery (CS) \cite{clip_surgery}, CLIP-VV \cite{clip_surgery}, and our approach across multiple categories. The heatmaps show activation regions for each queried class, where darker red indicates strongly activated regions. Our method produces more separated activations, demonstrating improved class localization.}
  \label{zs3}
\end{figure}

\begin{figure}[tp]
  \centering
  \includegraphics[width=\linewidth]{figs/sep_miou_mAP.png}
  % \vspace{-8pt}
  \caption{\textbf{Performance vs. MFI Reduction.} Performance of MLR (mAP) and ZS3 (mIoU) on COCO as a function of MFI reduction.}
  \label{fig: mAP_mIOU_MFI_red}
\end{figure}



\textbf{Feature Disentanglement in seen and unseen classes.} 
We pre-train Unmix-CLIP on COCO-14, which contains 80 classes. Despite this, as shown in Sec.\ref{sec: Results}, our approach improves performance even on datasets with previously unseen classes, such as VOC Context. We analyze this improvement by comparing MFI reduction across four datasets: VOC2012 (20 seen classes), COCO-2017 (80 seen classes), Context (59 partially seen classes), and a Context subset (30 unseen classes from COCO-2017). Figure \ref{fig:seperation} shows the self-similarity matrices of class text features from CLIP and Unmix-CLIP, demonstrating the class feature disentanglement. Table \ref{tab:mfi_reduction} quantifies the MFI reduction through the difference in average inter-class similarity between CLIP and Unmix-CLIP. The results show that our framework effectively disentangles representations for both seen and unseen classes, leading to performance improvements.

% \begin{table}[ht]
% \centering
% \caption{Comparison of MFI reduction across datasets.}
% \small
% \begin{tabular}{lcccc}
% \toprule
%  \textbf{Method} & \multicolumn{2}{c}{\textbf{VOC}}  & \multicolumn{2}{c}{\textbf{COCO}}  & \multicolumn{2}{c}{\textbf{Context}} \\ \cmidrule(l){2-3} \cmidrule(lr){3-4} \cmidrule(l){4-5}
%  &Seen & Seen&Complete & Unseen \\
% \midrule
% CLIP        & 0.77 & 0.69 & 0.75 & 0.75 \\
% Ours        & 0.50 & 0.52 & 0.53 & 0.52 \\
% \rowcolor{lightgray!45} $\Delta$ (\%)  &34.8  & 24.8  & 29.8  & 30.4  \\
% \bottomrule
% \end{tabular}
% \label{tab:mfi_reduction}
% \end{table}


\begin{table}[tp]
\centering
\caption{\textbf{Quantitative MFI Reduction.} 
MFI values are reported across different dataset datasets with seen (VOC, COCO), partial (Context), and unseen (Context) classes. Our method significantly reduces MFI for all datasets. }
\small
\begin{tabular}{lcccc} % Changed to six columns
\toprule
\textbf{Method} & \multicolumn{1}{c}{\textbf{VOC}}  & \multicolumn{1}{c}{\textbf{COCO}}  & \multicolumn{2}{c}{\textbf{Context}} \\ 
\cmidrule(lr){2-2} \cmidrule(lr){3-3} \cmidrule(lr){4-5}  
 & Seen  & Seen & Partial & Unseen \\ % Adjusted headers
\midrule
CLIP        & 0.77 & 0.69 & 0.75 & 0.75  \\ % Adjusted for six columns
Ours        & 0.50 & 0.52 & 0.53 & 0.52  \\
\rowcolor{lightgray!45} $\Delta$ (\%)  & 34.8  & 24.9  & 29.8  & 30.4  \\
\bottomrule
\end{tabular}
\label{tab:mfi_reduction}
\end{table}

\textbf{Feature Disentanglement Impact}. Figure \ref{fig: mAP_mIOU_MFI_red} shows how MFI reduction improves performance in both multi-label recognition and zero-shot semantic segmentation on COCO dataset.  We observe that as MFI decreases, the performance of both tasks improves.

%%% [START] 6.1
\begin{figure}[thb!] % 1-column
\footnotesize
\centering 

% Adjust the image width to fit within one column
\newcommand{\imgwidth}{0.235\linewidth} % Set image width to 16% of the line width

% \hspace{-1.9mm}
% \raisebox{0.25in}{\rotatebox{90}{Source}}%
% \hspace{-0.9mm}
% 1st Image
\begin{tikzpicture}[x=1cm, y=1cm]
    \node[anchor=south] (FigA1) at (0,0) {
        \includegraphics[width=\imgwidth]{sec/X_supp/Fig/imgs/loss_ablation/src_img.png}
    };
    \node[anchor=south, yshift=-2.5mm] at (FigA1.south) {\footnotesize Source};
\end{tikzpicture}\hspace{-1mm}%
% \hspace{-2.3mm}
% \raisebox{0.4in}{\rotatebox{90}{Target}}%
% \hspace{-1.3mm}
% 2nd Image
% \begin{tikzpicture}[x=1cm, y=1cm]
%     \node[anchor=south] (FigA2) at (0,0) {
%         \includegraphics[width=\imgwidth]{sec/X_supp/Fig/imgs/loss_ablation/dds.png}
%     };
%     \node[anchor=south, yshift=-2.5mm] at (FigA2.south) {\footnotesize w/o \ FPR};
% \end{tikzpicture}\hspace{-1mm}%
\begin{tikzpicture}[x=1cm, y=1cm]
    \node[anchor=south] (FigB1) at (0,0) {
        \includegraphics[width=\imgwidth]{sec/X_supp/Fig/imgs/loss_ablation/l2_loss.png}
    };
    \node[anchor=south, yshift=-2.5mm] at (FigB1.south) {\footnotesize Euclidean loss};
\end{tikzpicture}\hspace{-1mm}%
% % 3rd Image
\begin{tikzpicture}[x=1cm, y=1cm]
    \node[anchor=south] (FigC1) at (0,0) {
        \includegraphics[width=\imgwidth]{sec/X_supp/Fig/imgs/loss_ablation/l1_loss.png}
    };
    \node[anchor=south, yshift=-2.5mm] at (FigC1.south) {\footnotesize L1 loss};
\end{tikzpicture}\hspace{-1mm}%
% 4th Image
\begin{tikzpicture}[x=1cm, y=1cm]
    \node[anchor=south] (FigD1) at (0,0) {
        \includegraphics[width=\imgwidth]{sec/X_supp/Fig/imgs/loss_ablation/ssim_loss.png}
    };
    \node[anchor=south, yshift=-2.5mm] at (FigD1.south) {\footnotesize SSIM loss};
\end{tikzpicture}

\vspace{-8pt}
\caption{\textbf{Ablation study for loss function.} Edited results of \textit{(first)} the source image from prompt \textit{``a drawing of a cat"} to \textit{``a drawing of a dog"} using \textit{(second)} Euclidean, \textit{(third)} L1, and \textit{(fourth)} SSIM loss function for FPR.}
\label{fig:sup_loss_abl}
\end{figure}



%%% CODE TO GET THE PLOTS %%%%%%%%%%%%%


% \textbf{2. Sensitivity to alpha} \\
% \textbf{3. Some Analysis on ViT would be good} \\

