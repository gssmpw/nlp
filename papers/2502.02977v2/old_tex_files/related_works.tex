\begin{figure*}[tp]
    \centering
    \renewcommand{\arraystretch}{1} % Adjust row spacing
    \setlength{\tabcolsep}{1pt} % Adjust column spacing
    \begin{tabular}{c c c c c}
       \hspace{20pt}VOC (Seen) & \hspace{15pt}COCO (Seen) & \hspace{15pt} Context (Partial) & \hspace{17pt}Context (Unseen) & \multirow{2}{*}{\vspace{-210pt} \includegraphics[width=0.1\linewidth, height=0.425\linewidth, keepaspectratio]{figs/ICML_Sep/colorbar.png}} \\
        \includegraphics[width=0.225\linewidth]{figs/ICML_Sep/voc.png} & 
        \includegraphics[width=0.225\linewidth]{figs/ICML_Sep/coco.png} & 
        \includegraphics[width=0.225\linewidth]{figs/ICML_Sep/context.png} &
        \includegraphics[width=0.225\linewidth]{figs/ICML_Sep/context_no_coco.png} \\ 

         \includegraphics[width=0.225\linewidth]{figs/ICML_Sep/voc_ours.png} & 
         \includegraphics[width=0.225\linewidth]{figs/ICML_Sep/coco_ours.png}& 
         \includegraphics[width=0.225\linewidth]{figs/ICML_Sep/context_ours.png} &  
         \includegraphics[width=0.225\linewidth]{figs/ICML_Sep/context_no_coco_ours.png} \\  
    \end{tabular}
    \caption{\textbf{Class Feature Similarity Analysis.} Comparison of class-text feature similarities between CLIP (top row) and our method (bottom row) across four datasets: VOC, COCO, Context, and Context (unseen). The heatmaps show cosine similarity between class text features, where darker blue indicates higher similarity. Our method achieves higher class feature separation, shown by the reduced off-diagonal similarity values. This improved class separation suggests better discrimination capabilities.} 
    %wyw: VOC (Seen) ... should be aligned center with below images.
    % wyw: The legend bar font size should be decreased.
    %samyak: done!
    \label{fig:seperation}
\end{figure*}



%\subsection{Recoding Information}
\noindent {\bf Recoding information.}
Shannon proposed that optimal information transmission involves designing codes with minimum entropy \cite{shannon1948mathematical}. Barlow extended this idea to neuroscience, suggesting that sensory systems recode information to reduce redundancy with minimal loss \cite{barlow}. This principle has since been applied to many recent works, including image compression \cite{balle2016end} and self-supervised learning \cite{barlowtwins}. Our mutual-information reduction is inspired by the formulation of Barlow Twins \cite{barlowtwins}. We leverage this concept to reduce mutual information among class features in VLMs.

\vspace{1mm}
%\subsection{Vision-Language Models for Fine-grained Tasks}
\noindent {\bf Vision-Language Models for Fine-grained Tasks.}
Vision-language models (VLMs) trained with contrastive losses \cite{clip, align} are challenging to adapt for fine-grained tasks due to two reasons: (1) their reliance on global feature aggregation, which ignores local information. (2) Using the softmax operation in their training loss biases them toward single-object settings.

{\em Recognition.} 
Early efforts to adapt VLMs for recognition centered on learning prompts as classifiers for visual features \cite{coop}. These methods were extended to multi-label settings by learning multiple prompts for each class \cite{dualcoop,dualcoop++,PositiveCoOp}. Building on this, subsequent works incorporated co-occurrence information to make predictions interdependent \cite{scpnet,MLR-GCN}. In contrast, our approach does not rely on prompt learning or co-occurrence modeling during pre-training. Furthermore, our features are adaptable to tasks beyond recognition.

{\em Localization.} Early approaches addressed localization by training image segmentation models and using VLMs to label the segmented regions \cite{sam}. Later methods introduced pre-training setups that combined vision-language alignment with mask distillation to enhance localization \cite{dong2023maskclip}. Recent works adapted features for localization without additional training by leveraging the spatial properties preserved in the value projection of CLIP’s transformer-style aggregation \cite{maskclip}. CLIP Surgery \cite{clip_surgery} identified consistent noisy activations across classes and reduced them by subtracting average features from class-specific features \cite{clip_surgery}, though the cause of these activations remains unclear. GEM generalized this concept to vision transformers \cite{gem_walid}. We use the finding that value projection preserves spatial information. We further improve value projection by disentangling class features during pre-training.

\begin{figure*}[t]
\begin{center}
\includegraphics[width=.85\linewidth]{fig_overview_v3.pdf}
\end{center}
\caption{
FastAtlas Overview: In each frame, we compute charts spanning fully or partially visible triangles (a), determine texture space bounding boxes for the visible portions of the view-space projections of each chart, and tightly pack these boxes into atlases (b, here $2K \times 2K$). We simultaneously bijectively parameterize and shade the charts into their atlas boxes, obtaining high quality texture space shading (c), and use this shading to render the shaded frames (d).}
\label{fig:overview}
\label{fig:alg_overview}
\end{figure*}

\section{Overview}
\label{sec:overview}
Our work has two core contributions: a real-time, GPU-based algorithm for tight packing of general parameterized charts into compact atlases; and a real-time TSS method that
utilizes this packing.  

\paragraph*{FastAtlas Packing.}
FastAtlas runs entirely on the GPU as a series of compute shaders. It takes the bounding boxes of parameterized charts as input, and packs them into an atlas (Fig~\ref{fig:overview}b, Sec.~\ref{sec:pack}). As such, the only input it requires are the dimensions of the bounding boxes.
Its outputs are deterministic; identical input charts are packed into identical atlases. This is critical for TSS and similar applications, as it ensures that consecutive frames taken from the same camera view have the same shading. Even minute shading differences across such frames can cause sampling jitter, leading to undesirable flicker \cite{baker2012rock}. 
While prior methods such as \cite{mueller2018shading,hladky2019tessellated,hladky2021snakebinning,Neff2022MSA} cap the dimensions of the charts that can be packed as-is for a given atlas size, and scale down all charts that exceed these dimensions, we scale all charts by the same factor, and do so only when strictly necessary to achieve packing success (Figs~\ref{fig:atlas},~\ref{fig:sas_issues}). 

\paragraph*{TSS using FastAtlas.}
Our end-to-end TSS atlas generation method combines the packing method above with a novel approach for computing seamless per-frame charts. 
We define our charts as the connected components of the visible surfaces in each frame (Fig.~\ref{fig:overview}a), and efficiently compute them using a parallel union-find algorithm (Sec.~\ref{sec:visible}). Since the boundaries of these charts coincide with the contours of the rendered surface, they are {\em invisible} to the viewer. This approach 
eliminates the artifacts caused by shading discontinuities along visible seams (Fig.~\ref{fig:seams}). 

\begin{parWithWrapFigure}
\begin{wrapfigure}{l}{.27\columnwidth}%
\includegraphics[width=\linewidth]{fig_inset_view_plane.pdf}%
\end{wrapfigure}
We bijectively parametrize the {\em visible portions} of our charts by projecting them to view space (inset). This maps a constant number of texels to each pixel in the final rendered output, evenly distributing residual undersampling error across all image pixels. While conceptually straightforward, efficiently parameterizing charts containing partially visible triangles using viewspace projection is non-trivial, as the visible portions may no longer be triangular (e.g. green triangle in the inset); applying naive projection to triangles with vertices behind the camera may produce ill-posed results. Clipping triangles before projection is both computationally expensive and significantly complicates downstream operations. We avoid explicit clipping by observing that all that is required for atlas packing is the dimensions of, potentially conservative, bounding boxes of these projected visible portions. We compute such bounding boxes without explicit chart clipping by adapting a conservative screen coverage estimator \shortcite{Blinn:CalculatingScreenCoverage} (Sec.~\ref{sec:box}). We then pack the computed boxes using FastAtlas. 
\end{parWithWrapFigure}

Finally, we shade the visible portion of each chart into its corresponding atlas bounding box (Fig~\ref{fig:overview}c). 
The resulting texture is then used during rasterization as a standard texture map (Fig. ~\ref{fig:overview}d). 
Our framework is compatible with all existing approaches for texture space shading, including forward shading, raytraced illumination, or deferred shading in texture space \cite{baker:2016}. In the examples shown, we use the standard forward shading based rendering pipeline included in the G3D Innovation Engine \cite{G3D17}, a commercial grade renderer.
