\subsection{RQ-2 How do different configurations affect the effectiveness of LLMs?}
\label{sec:rq2}


\noindent 
\textbf{Impact of different example numbers.}
As previous studies~\cite{brown2020language,A3CodGen} have shown, the number of examples provided has a significant impact on LLMs' performance. 
To explore this, we adjust the number of examples while keeping other parameters and hyperparameters constant to ensure a fair comparison.
We do not conduct experiments in a zero-shot setting, as LLMs may generate unnormalized outputs without a prompt template, which would hinder automated extraction. 
From Fig.~\ref{fig:ablation}, we observe that as the number of examples increases, both the average token length and time cost rise sharply, while the improvement in Pass@k remains modest.
Based on these findings, we perform our ablation studies (Table~\ref{tab:rq1} and \ref{tab:ablation}) using a one-shot setting in \mytitle.


\noindent 
\textbf{Impact of different selection strategies.}
% Our case study reveals that the RAG method improves the performance of LLMs.
RAG retrieves relevant codes from a retrieval database and supplements this information for code generation~\cite{parvez2021retrieval}. 
To ensure a fair comparison, we set the number of examples to one and evaluated the results of RAG versus random selection on the same LLM (i.e., DeepSeek-V3). From Table~\ref{tab:ablation}, Pass@1 and Compile@1 are higher when RAG is enabled, indicating that it improves the effectiveness of code generation.


\begin{figure}[htbp]
    \centering
    \includegraphics[width=\linewidth]{figs/ablation.pdf}
    \caption{Performance of Qwen2.5-Coder-7B. The x-axis represents the number of shots.}
    \label{fig:ablation}
\end{figure}
\vspace{-0.2cm}

\noindent 
\textbf{Impact of Context Information.}
Since that relevant context typically enhances performance in other programming languages, we conduct an ablation study to examine the influence of context on the quality of LLM-generated contracts. Table~\ref{tab:ablation} shows that providing context information improves both Pass@1 and Compile@1. 
However, there is no clear correlation between gas fees, vulnerability rate, and the presence of context information.


% \vspace{-0.1cm}
\begin{table}[htbp]
    \centering
    \caption{Ablation study on the effect of RAG and Context on DeepSeek-V3's (one-shot) performance.}
    \resizebox{\linewidth}{!}
    {
        \begin{tabular}{cc|cccc}
        \toprule
        RAG & Context & Pass@1 & Compile@1 & Fee & Vul \\
        \midrule
        \ding{51} & \ding{51} & \textbf{21.72\%}& \textbf{53.35\%}&  \textbf{-7525}& 26.61\% \\ 
        \ding{55} & \ding{51} & 20.24\% & 51.08\% & 3828& \textbf{23.68\%}\\ 
        \ding{51} & \ding{55} & 21.28\% & 52.54\% & -708& 26.13\%\\
        \ding{55} & \ding{55} & 20.17\% & 50.32\% & 768& 26.83\%\\   
        \bottomrule
        \end{tabular}
    }
    \label{tab:ablation}
\end{table}
% \vspace{-0.4cm}