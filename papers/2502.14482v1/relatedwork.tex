\section{Related Works}
% PEFT

% LoRA variants
\subsection{LoRA's variants} 
With the introduction of LoRA \citep{hu2021lora}, many derivative methods have emerged. 
AdaLORA \citep{zhang2023adalora} highlights that LoRA ignores the importance of different layer parameters based on a uniform setting of the rank, and proposes an adaptive allocation strategy based on parameter importance to improve fine-tuning efficiency.
DoRA \citep{liu2024dora} introduces a decomposation of weight matrices into magnitude and direction components, leveraging LoRA to update only the directional component, thereby reducing the number of trainable parameters. ReLoRA \citep{lialin2023relora} achieves high-rank training through iterative low-rank updates, periodically merging parameters into the main model. LoRA+ \citep{hayou2024lora+} further improves efficiency by applying different learning rates to the two matrices in LoRA, assigning a higher learning rate to matrix \(B\) to accelerate convergence and enhance performance.   Other works have focused on improving the initialization of the \(AB\) matrix, such as PiSSA \citep{meng2024pissa}, which suggests initializing \(A\) and \(B\) by performing SVD on the pre-trained matrix \(W\) to accelerate the convergence speed. LoRA-GA \citep{wang2024lora} initializes \(A\) and \(B\) using the eigenvectors of the full-gradient matrix, aligning the gradient direction of the low-rank product \(BA\) with the gradient direction of the pretrained weight matrix \(W\).
\begin{figure*}[]
\centering
\includegraphics[width=0.9\textwidth]{nlora_cropped.pdf} 
\caption{The diagram of the Nyström-based initialization}
\label{figure:nlora}
\end{figure*}
% Nyström
\subsection{Nyström-like methods}
Nyström-like methods approximate matrices by sampling a subset of columns, a technique widely used in kernel matrix approximation \citep{baker1979numerical,williams2000using}. Numerous variants have been proposed to enhance the basic Nyström method, including Nyström with k-means clustering \citep{wang2019scalable}, Nyström with spectral problems \citep{vladymyrov2016variational}, randomized Nyström \citep{li2010making,persson2024randomized}, ensemble Nyström method \citep{kumar2009ensemble}, fast-Nys \citep{si2016computationally}.

The Nyström method has also been extended to general matrix approximation beyond symmetric matrices \citep{nemtsov2016matrix}. Some methods \citep{wang2013improving,xiong2021nystromformer} explicitly address general matrix approximation by sampling both rows and columns to reconstruct the full matrix. Inspired by such strategies, we propose NLoRA method by to optimize the approximation for efficient matrix reconstruction.