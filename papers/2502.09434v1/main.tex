\documentclass[lettersize,journal]{IEEEtran}
\newcommand{\CG}{\mathcal{G}\xspace}
\newcommand{\CV}{\mathcal{V}\xspace}
\newcommand{\CE}{\mathcal{E}\xspace}
\newcommand{\CA}{\mathcal{A}\xspace}
\newcommand{\CF}{\mathcal{F}\xspace}
\newcommand{\CR}{\mathcal{R}\xspace}
\newcommand{\CB}{\mathcal{B}\xspace}
\newcommand{\CX}{\mathcal{X}\xspace}
\newcommand{\CK}{\mathcal{K}\xspace}
\newcommand{\CM}{\mathcal{M}\xspace}
\newcommand{\CC}{\mathcal{C}\xspace}
\newcommand{\CL}{\mathcal{L}\xspace}
\newcommand{\CI}{\mathcal{I}\xspace}
\newcommand{\CQ}{\mathcal{Q}\xspace}
\newcommand{\CO}{\mathcal{O}\xspace}
\newcommand{\CP}{\mathcal{P}\xspace}
\newcommand{\CS}{\mathcal{S}\xspace}
\newcommand{\CT}{\mathcal{T}\xspace}
\newcommand{\CJ}{\mathcal{J}\xspace}
\usepackage[para]{footmisc}
\usepackage{subfig}
% \usepackage{subcaption}
% \usepackage{array}
% \usepackage{colortbl}

%before hyperref
\usepackage{amsmath,amsfonts}
\usepackage{algorithmic}
\usepackage[ruled,linesnumbered]{algorithm2e}
% \usepackage{algorithm2e}

% \usepackage{algorithm}
\usepackage{array}
\usepackage[caption=false,font=normalsize,labelfont=sf,textfont=sf]{subfig}
\usepackage{textcomp}
\usepackage{stfloats}
\usepackage{url}
\usepackage{verbatim}
\usepackage{graphicx}
\usepackage{cite}
\usepackage{utfsym}

\hyphenation{op-tical net-works semi-conduc-tor IEEE-Xplore}
\def\BibTeX{{\rm B\kern-.05em{\sc i\kern-.025em b}\kern-.08em
    T\kern-.1667em\lower.7ex\hbox{E}\kern-.125emX}}
\usepackage{balance}
%%%%% new package

\usepackage{times}
\usepackage{fancyhdr,graphicx,amsmath,amssymb}

% \usepackage[ruled,vlined]{algorithm2e}

\usepackage{multirow, bigstrut} 
% \usepackage{arydshln}

\usepackage{url}
\usepackage{graphicx}
\usepackage{booktabs} 
% \usepackage[inkscapelatex=false]{svg}
% \svgpath{{images/}}
\usepackage{xcolor}
\usepackage{hyperref}
\usepackage[capitalize]{cleveref}
\definecolor{deepgreen}{rgb}{0.0, 0.5, 0.0}
\newcommand{\pyh}[1]{\textcolor{deepgreen}{#1}}
\newcommand{\wry}[1]{\textcolor{orange}{\@#1}}
\newcommand{\pyhshan}[1]{\textcolor{red}{#1}}
\newcommand{\darkedred}[1]{\textcolor{red!80!black}{#1}}
\newcommand{\gxlnote}[1]{\textcolor{black}{#1}}
\newcommand{\crnote}[1]{\textcolor{black}{#1}}
\newcolumntype{P}[1]{>{\centering\arraybackslash}p{#1}}
% \newcommand{\pyh}[1]{\textcolor{black}{#1}}
% \newcommand{\wry}[1]{\textcolor{black}{\@#1}}
% \newcommand{\pyhshan}[1]{\textcolor{black}{#1}}



\begin{document}
\include{pythonlisting}
% \title{A Sample Article Using IEEEtran.cls\\ for IEEE Journals and Transactions}
% \bibliographystyle{IEEEtran}


\title{Redistribute Ensemble Training for Mitigating Memorization in Diffusion Models}



% \author{IEEE Publication Technology,~\IEEEmembership{Staff,~IEEE,}

\author{Xiaoliu Guan, Yu Wu, Huayang Huang, Xiao Liu, Jiaxu Miao, Yi Yang
% <-this % stops a space
\thanks{X. Guan,  Y. Wu, H. Huang, and  X. Liu are with the School of Computer Science, Wuhan University, China. E-mail: liuxiaoguan, wuyucs, hyhuang, xiaoliu@whu.edu.cn}% <-this % stops a space
\thanks{J. Miao is with the School of Cyber Science and Technology, Sun Yat-sen University, China. E-mail: miaojx@mail.sysu.edu.cn }
\thanks{Y. Yang is with the College of Computer Science and Technology, Zhejiang University, Hangzhou, Zhejiang, China. E-mail: yangyics@zju.edu.cn}
\thanks{(Corresponding author: Yu Wu.)}}

\maketitle


\begin{abstract}
Diffusion models, known for their tremendous ability to generate high-quality samples, have recently raised concerns due to their data memorization behavior, which poses privacy risks. Recent methods for memory mitigation have primarily addressed the issue within the context of the text modality in cross-modal generation tasks, restricting their applicability to specific conditions. In this paper, we propose a novel method for diffusion models from the perspective of visual modality, which is more generic and fundamental for mitigating memorization. 
Directly exposing visual data to the model increases memorization risk, so we design a framework where models learn through proxy model parameters instead. Specially, the training dataset is divided into multiple shards, with each shard training a proxy model, then aggregated to form the final model.
Additionally, practical analysis of training losses illustrates that the losses for easily memorable images tend to be obviously lower. Thus, we skip the samples with abnormally low loss values from the current mini-batch to avoid memorizing.
However, balancing the need to skip memorization-prone samples while maintaining sufficient training data for high-quality image generation presents a key challenge. Thus, we propose IET-AGC+,  which redistributes highly memorizable samples between shards, to mitigate these samples from over-skipping.  
Furthermore, we dynamically augment samples based on their loss values to further reduce memorization.
Extensive experiments and analysis on four datasets show that our method successfully reduces memory capacity while maintaining performance. Moreover, we fine-tune the pre-trained diffusion models, e.g., Stable Diffusion, and decrease the memorization score by 46.7\%, demonstrating the effectiveness of our method. Code is available in \url{https://github.com/liuxiao-guan/IET_AGC}.
\end{abstract}

\begin{IEEEkeywords}
Diffusion Models, Model Memorization, Data Privacy.
\end{IEEEkeywords}

\section{Introduction}
\IEEEPARstart{R}{ecent} advancements in diffusion models have significantly transformed the landscape of image generation~\cite{croitoru2023diffusion,zhan2023multimodal,zhu2024vision+}. Modern diffusion models, such as Stable Diffusion~\cite{rombach2022high}, Midjourney~\cite{midjourney2022}, and SORA~\cite{sora2024}, can generate realistic images that are hard for humans to distinguish, demonstrating the unparalleled capabilities in producing diverse images. However, recent works~\cite{carlini2023extracting,somepalli2024understanding,wen2023detecting} suggested that diffusion models can memorize images from the training set and reproduce them directly. \gxlnote{This raises privacy concerns, as sensitive information, such as identifiable faces or private documents, may be generated and inadvertently exposed.} To address the critical issue, some works~\cite{zhang2023forget,ni2023degeneration,gandikota2024unified,kumari2023ablating} proposed to make diffusion models ``forget'' specific concepts such as a portrait of a certain celebrity, or the style of a particular artist. However, these works can only blacklist specific content that users want to conceal, but cannot completely cover the privacy-sensitive information that the model might remember, still posing a risk of privacy leakage.

\begin{figure}[tb]
  \centering
  \setlength{\abovecaptionskip}{7pt} % 设置标题上方的间距为 -5pt
  \setlength{\belowcaptionskip}{-7pt} % 设置标题下方的间距为 -5pt
  \includegraphics[width=1.0\linewidth]{imgs/problem.pdf}
  \caption{\gxlnote{Prior methods focus solely on the captions associated with the memorized images, such as caption augmentation. In contrast, our approach takes a more generalizable framework by considering aspects from the visual modality.}}
  \label{fig:problem}
\end{figure}
 
Recently, some works~\cite{somepalli2023diffusion,daras2024ambient,somepalli2024understanding,wen2023detecting} have proposed to mitigate diffusion memorization without specific content limitations, thus reducing the risk of diffusion models leaking privacy-sensitive training data. Most of them focused on tackling the training data memorization in text-to-image diffusion models, and proposed data augmentation for captions/sentences to reduce model memorization.  For instance, Somepalli \MakeLowercase{\textit{et al.}}~\cite{somepalli2024understanding} found that the insufficient diversity in captions easily leads to training data generation and thus utilized random caption replacement, random token replacement, and caption word repetition, \MakeLowercase{\textit{etc.}}, to reduce memorization. 
Based on the discovery that memorized prompts tend to exhibit larger magnitudes, which refers to the difference between the text-conditioned and unconditioned noise prediction, Wen \MakeLowercase{\textit{et al.}}~\cite{wen2023detecting} introduced methods for mitigating memorization through filtering high-magnitude sample during training and minimizing magnitudes during inference. Although these works have made significant progress in understanding the memorization issue in diffusion models, \gxlnote{they only focused on easily memorable images related to specific captions in cross-modal generation tasks as shown in ~\cref{fig:problem}. 
However, they do not directly tackle the memorization problem in image generation. While manipulating captions may reduce the likelihood of memorization being triggered in text-to-image models, the model’s inherent ability to memorize images remains. Memorization can still occur under different conditions~\cite{carlini2023extracting,somepalli2024understanding}.
Therefore, we propose a novel framework for diffusion models from the perspective of the visual modality, which not only mitigates memorization more fundamentally but also provides a more generic approach.}



Following these insights,  in our preliminary ECCV 2024 version~\cite{liu2024iterative}, we propose the first module: \textbf{Iterative Ensemble Training (IET)} framework from the perspective of parameter aggregation as shown in ~\cref{fig:problem}. Transmitting data directly to the model increases the likelihood of memorizing easy samples. \gxlnote{However, if the model learns from parameters of other models, rather than directly from the data, it may help to mitigate the direct memorization}. Specifically, we divide the data into multiple data shards and train several proxy models. These models are then aggregated to form the final model. Inspired by federated learning~\cite{mcmahan2017communication}, we iteratively ensemble the proxy models during training, which helps reduce memorization through multiple aggregations and preserves the generation performance. 
Besides, we suspect that images with varying degrees of memorization might exhibit different behaviors during the training process. Therefore, we analyze the training process and find that the loss of easily memorable images tends to be obviously lower than that of less memorable images. Based on this analysis, we propose the second module: \textbf{Anti-Gradient Control (AGC)} to further reduce memorization of training data. In particular, we skip the samples with abnormally small loss values from the current mini-batch to avoid memorizing these samples. During training, as the diffusion model exhibits varying average loss values across different time steps, we maintain a memory bank to track the average loss at each step. Building on this, we skip samples whose loss ratio—defined as the ratio of the sample's loss to the average loss—falls below a predefined skipping threshold as shown in ~\cref{fig:AGC+_s}.



\gxlnote{
However, the AGC strategy might excessively skip highly memorizable samples, leading to a reduction in available training data and potential degradation of image quality.
This drives us to pursue a better approach that strikes a balance between mitigating memorization and maintaining image quality.
Following these insights, in this paper, we introduce IET-AGC+ building on our ECCV2024 framework~\cite{liu2024iterative}.}
\gxlnote{To address the issue of excessively skipping, we propose a \textbf{Memory Samples Redistribute (MSR)} strategy to ensure that these samples are learned but not easily memorized. In the IET framework, each proxy model learns from its shard, where the same data may be interpreted differently.
\emph{In particular, when a sample is frequently memorized in its original shard, it may not have the same memorization tendency in a new shard}. As the saying goes: One man's meat is another man's poison. This inspires us to exchange easily memorized samples from one shard with another to prevent them from being skipped too frequently as shown in ~\cref{fig:problem}.  Therefore, in the training process, we track the number of times each sample is skipped to identify whether it is most easily memorized. During the interaction, each shard allocates its most frequently skipped samples to the next shard in a circular manner.}


\begin{figure}[tb]
  \centering
  \setlength{\abovecaptionskip}{7pt} % 设置标题上方的间距为 -5pt
  \setlength{\belowcaptionskip}{-10pt} % 设置标题下方的间距为 -5pt
  \includegraphics[width=1.0\linewidth]{imgs/simple_AGC+.pdf}
  \caption{\gxlnote{Threshold-Aware Augmentation (TAA) collaborated with Anti-Gradient Control. We apply three different treatments based on the comparison between the sample's loss ratio and the skipping threshold. }}
  \label{fig:AGC+_s}
\end{figure}
% \vspace{-0.5em}
\gxlnote{
On the other hand, in AGC, images below the threshold are more likely to be memorized, making their exclusion a reasonable choice. However, memorization varies in degree and cannot be simply addressed with a hard threshold. Samples should be dynamically processed based on their level of memorization risk. To address this, we propose a new strategy called \textbf{Threshold-Aware Augmentation (TAA)} collaborated with Anti-Gradient Control as shown in ~\cref{fig:AGC+_s}. For samples that are not skipped but whose loss values are close to the threshold, we apply augmentation to increase their diversity, thereby reducing memorization. 
A lower loss value indicates a higher risk of memorization, so we use dynamic visual augmentation based on sample distance from the threshold. Samples closer to the threshold receive stronger augmentation.}

Extensive experiments on four datasets highlight the importance of our framework. 
Our method significantly reduces the memorized quantity by \gxlnote{90.1\%, 74.6\%, and 91.2\% }compared with the default training (DDPM~\cite{ho2020denoising}) on CIFAR-10~\cite{krizhevsky2009learning} and CIFAR-100~\cite{krizhevsky2009learning} and AFHQ-DOG~\cite{choi2020stargan}, respectively. Furthermore, when fine-tuning the text-conditional diffusion model, Stable Diffusion~\cite{rombach2022high}, our approach decreases the memorization score by \gxlnote{46.7\%} compared to conventional fine-tuning method~\cite{rombach2022high}. In addition, our method can also be applied to existing inference phase mitigation mechanisms~\cite{somepalli2024understanding,wen2023detecting}, further reducing memorization and improving image quality. These results demonstrate the effectiveness of our method.

\gxlnote{Our main contributions are summarized as follows:
\begin{itemize}
\item {
We introduce a generalized method to mitigate memorization from the perspective of the visual modality, which consists of two main parts: leveraging multiple model ensembles for training and skipping easily memorized samples based on the training loss.
}
\item{
We propose Memory Samples Redistribute (MSR), which redistributes easily memorized samples across shards in the above framework while maintaining a balance between memorization reduction and image quality.
}
\item{
We suggest Threshold-Aware Augmentation (TAA), a strategy that adapts the level of augmentation based on the distance between the sample's loss and the skipping threshold, effectively addressing the risk of overlooking memorized samples.
}
\end{itemize}}





\section{Related Work}

\subsection{Memorization in Generative Models}
Several studies have examined the memorization capabilities of the generative model~\cite{wang2024replication,sun2024create}. 
Generative Adversarial Networks (GANs)~\cite{goodfellow2020generative} have been at the forefront of this research area. 
As Webster \MakeLowercase{\textit{et al.}}~\cite{webster2021person} demonstrated when applied to face datasets, GANs can occasionally replicate.
Prior study~\cite{carlini2021extracting} explored an adversarial attack on language models like GPT-2~\cite{radford2019language}, where individual training examples can be recovered, including personally identifiable information and unique text sequences.

Recent studies have shifted their attention toward diffusion models. 
Somepalli \MakeLowercase{\textit{et al.}}~\cite{somepalli2023diffusion} found that diffusion models accurately recall and replicate training images, especially noted with models like the Stable Diffusion model~\cite{rombach2022high}.
Building upon this discovery, Carlini \MakeLowercase{\textit{et al.}}~\cite{carlini2023extracting} developed a tailored black-box attack for diffusion models. They generated images and implemented a membership inference attack to assess density.
Webster \MakeLowercase{\textit{et al.}}~\cite{webster2023reproducible} demonstrated a more efficient extraction attack with fewer network evaluations, identified "template verbatims," and discussed its persistence in newer systems. 
Recent research has shifted towards exploring the theoretical aspects of memory in diffusion models.
Yoon \MakeLowercase{\textit{et al.}}~\cite{yoon2023diffusion} discovered that generalization and memorization are mutually exclusive occurrences and further demonstrated that the dichotomy between memorization and generalization can be apparent at the class level.
Gu \MakeLowercase{\textit{et al.}}~\cite{gu2023memorization} extensively studied how factors like data dimension, model size, time embedding, and class conditions affect the memory capacity of the diffusion model.

\subsection{Memorization Mitigation} 
The mitigation measures have primarily been concerned with filtering inputs and deduplication. 
For example, Stable Diffusion employed well-trained detectors to identify unsuitable generated content. 
However, these temporary solutions can be easily bypassed~\cite{wen2024hard,rando2022red} and do not effectively prevent or lessen copying behavior on a broad scale. 
Kumari \MakeLowercase{\textit{et al.}}~\cite{kumari2023ablating} designed an algorithm to align the image distribution with a specific style, instance, or text prompt they aim to remove, to the distribution related to a core concept. 
This stopped the model from producing target concepts based on its text condition.
\gxlnote{Hintersdorf \MakeLowercase{\textit{et al.}}~\cite{hintersdorf2024finding} localized memorization of individual data samples down to the level of neurons in DMs’ cross-attention layers.}
However, these approaches are inefficient because they necessitate a list of all concepts to be erased, and have not addressed the key issue of how to reduce the memory capacity of the model.
~\cite{dockhorn2022differentially,ghalebikesabi2023differentially} explored the use of differential privacy (DP)~\cite{dwork2006differential} to train diffusion models or fine-tune ImageNet pre-trained models. However, their focus was on ensuring the privacy of the training of diffusion models, not on the privacy of the images generated by the diffusion models. 
\gxlnote{Chen \MakeLowercase{\textit{et al.}}~\cite{chen2024towards} re-guides generation by measuring the similarity between generated and training images, aiming for memorization-free outputs. However, directly relying on the training set during testing is impractical.}
Daras \MakeLowercase{\textit{et al.}}~\cite{daras2024ambient} introduced a technique for training diffusion models utilizing tainted data. By incorporating additional corruption before applying noise, their methodology prevents the model from overfitting to the training data. But their training requires a considerable amount of time. 
~\cite{somepalli2024understanding,wen2023detecting, ren2024unveiling} also suggested a series of recommendations to mitigate copying such as randomly replacing the caption of an image with a random sequence of words, but most of which are limited to text-to-image models. Our work focuses on the nature of memorization in diffusion models, especially for unconditional ones. 
\vspace{-0.1cm}
\gxlnote{\subsection{Data Augmentation Theory and Practice}
Data augmentation is a widely used technique to improve the generalization of machine learning models, particularly in deep learning ~\cite{wang2021regularizing}. It is commonly employed to increase the diversity of training data by applying transformations in image-based tasks. Common data augmentation techniques include pixel erasing ~\cite{zhong2020random,devries2017improved,chen2020gridmask}, image cropping~\cite{chen2016automatic,ciocca2007self}, mixing images~\cite{hendrycks2019augmix,zhang2017mixup}, geometric transformations~\cite{wang2019perspective,jaderberg2015spatial}, kernel filter~\cite{kang2017patchshuffle}, \MakeLowercase{\textit{etc}}.
The use of data augmentation has been widely explored for vision tasks that require extensive annotation. Azizi \MakeLowercase{\textit{et al.}}~\cite{azizi2023synthetic}showed that augmenting the ImageNet training set~\cite{russakovsky2015imagenet} with samples generated by conditional diffusion models results in a significant boost in classification accuracy. Baranchuk \MakeLowercase{\textit{et al.}}~\cite{baranchuk2021label} investigated how diffusion models can be used to augment data for semantic segmentation, leveraging intermediate activations as rich pixel-level representations, especially when labeled data is scarce. Trabucco \MakeLowercase{\textit{et al.}}~\cite{trabucco2023effective} explored methods to augment individual images with a pre-trained diffusion model, showing significant improvements in few-shot scenarios. Other examples include tasks like human motion understanding~\cite{guo2022learning, izadi2011kinectfusion}, optical flow estimation~\cite{dosovitskiy2015flownet, sun2021autoflow}, and physically realistic simulation environments~\cite{de2022next,dosovitskiy2017carla,gan2020threedworld}, etc. Our study uses data augmentation to flexibly enhance model generalization, thereby mitigating memorization.}

\section{Proposed Method}
% \wei{we should highlight pre-training a few more times in this section. it is not until the last subsection that I realize we are developing a pre-trianing framework} 

% \wei{I feel our focus should be pre-training. Other components are serving for enhancing pre-training instead of using pre-training to enhance other components. }

% \wei{Do you think we should first present our pre-training framework and then detail other components? Following this framework, we then describe the challenges in epidemic pre-trainign and introduce the corresponding solutions in each subsections.}

In this section, we introduce our Covariate-Adjusted Pre-training framework for Epidemic forecasting (CAPE). Enhanced by components that capture temporal dependency and infer pseudo-environments (section~\ref{sec: CA} and~\ref{sec: env_estimate}), CAPE aims to learn the intrinsic disease dynamics through pre-training (section~\ref{sec: contrast}).

% to learn broad patterns and diverse environment representations.




\subsection{Model Design}
\subsubsection{Causal Analysis for Epidemic Forecasting}


\begin{figure}[t]
\centering
\includegraphics[scale=0.35]{figures/causal_graph.pdf}
\caption{ Structural causal model (SCM) for epidemic foresting, where $\mathbf{z}^i$ refers to a confounder, and $X_s$ and $X_c$  refers to the spurious and causal factors of the input respectively. }
% \wei{variables connections with epidemics} 
\label{fig: causal_graph}
\end{figure}

% Given the complex interplay between environments and epidemic dynamics, effectively integrating them into epidemic pre-training presents a challenging question. Since the environments impact both historical infection patterns and future disease spread, we draw inspiration from causal inference~\cite{zhou2023causal, jiao2024causal} and consider the environment as a confounder, i.e., a variable that simultaneously affects both the independent variable (e.g., treatment) and the dependent variable (e.g., outcome). To accurately capture these relationships, we incorporate the environment as a distinct component within a Structural Causal Model (SCM), as illustrated in Figure~\ref{fig: causal_graph}. The SCM effectively represents the underlying generative processes by using directed edges to connect each variable (node), thereby depicting the causal pathways among them.
Given the complex interplay between environments and epidemic dynamics, effectively integrating them into epidemic pre-training presents a challenging question. Since environments impact both historical infection patterns and future disease spread, we draw inspiration from causal inference~\cite{zhou2023causal, jiao2024causal} and treat the environment as a confounder—a variable that simultaneously influences both the independent variable (e.g., historical data) and the dependent variable (e.g., future predictions). To capture these relationships, we incorporate the environment as a distinct component within a Structural Causal Model (SCM), as illustrated in Figure~\ref{fig: causal_graph}. The SCM represents the underlying generative processes by using directed edges to connect each variable (node), thereby depicting the causal pathways among them and ensuring that the model accounts for the confounding effects of the environment. Specifically, we consider the environment as a whole as the confounder, which consists of $k$ discrete values from $\mathbf{z}^1$ to $\mathbf{z}^k$. In addition, we adopt an assumption (section~\ref{sec: env_estimate}) that decomposes the input $\mathbf{X}$ to a spurious factor $\mathbf{X}_s$ and a causal factor $\mathbf{X}_c$. \wei{you should probably merge 4.1.1 and 4.1.2 and reorganize the pargarpahs}

% \wei{we need to briefly mention $X_s, Y_s, X_c$... here like what you did in your assumption 4.2}

% \wei{explain the causal a bit, what are nodes/edges; check how other causal inference paper illustrate this concisely. also we should probably put this figure in introduction as we mentioned it in the intro}
% Next, beyond accounting for the environment during pre-training, the key mechanism to disentangle the correlations between diseases and the environment is required. Since we view the environment as the confounder during modeling, we can treat the correlation introduced by the environment as spurious correlations~\cite{ming2022impact}. To mitigate the effects of these spurious correlations, covariate adjustment~\cite{runge2023causal} serves as a standard tool that controls for confounding variables to isolate the true causal effect of interest. Nonetheless, effective covariate adjustment requires a comprehensive set of valid confounders to accurately control for all potential sources of bias, which may not be observed. While this can be hard to achieve given scarce data for a certain disease and region, pre-training enables the model to utilize a large volume of historical data across diverse diseases and regions, which helps the model to account for the underlying environmental factors.


% \subsubsection{Capturing Temporal Correlations}
% \label{sec: temporal_corr}
% Temporal dependency plays an essential role in epidemic time series data as the observed changes often exhibit lagged effects rooted in historical contexts and intervention strategies, shaping current and future trends. In this study, we incorporate self-attention and additional techniques to effectively capture these temporal patterns.

% First, to mitigate the impact of temporal distribution shifts, we employ Reversible Instance Normalization(RevIN)~\cite{kim2021reversible} on the input data. Then, instead of treating each time point as a token, we apply patching~\cite{nie2022time} to efficiently and effectively learn useful representations for forecasting:

% % is then partitioned into \( N \) patches of fixed length \( L \)~\cite{}:

% \[
% \left\{ \mathbf{x}_0, \mathbf{x}_1, \mathbf{x}_2, \ldots, \mathbf{x}_N \right\} = \text{Patching}(\text{RevIN}(\mathbf{X})), \tag{1}
% \]

% where $N$ denotes the number of patches. Each patch \( \mathbf{x}_i \) then undergoes a learnable linear projection and is augmented with a positional embedding:
% \[
% \mathbf{x}_i^{(0)} = \mathbf{W}_{p} \mathbf{x}_i + pos(i), \tag{2}
% \]
% where $\mathbf{W_{p}}$ is the projection matrix and \( pos(i) \) denotes the Sinusoidal Positional Encoding for patch \( i \). 
% The resulting encoded patches, \(\mathbf{X}^{(0)}\), are then passed into the CAPE encoders, with \(\mathbf{X}^{(l)}\) denoting the output of the \(l\)-th encoder. As illustrated in Figure~\ref{fig:CAPE}, each encoder block applies a self-attention mechanism that yields a contextualized representation \(\mathbf{h}_i^{(l)}\), effectively capturing inter-patch correlations:

% % The encoded patches \( \mathbf{X}^{(0)} \) are then fed into the CAPE encoders. We denote the output of the $l$ th encoder as $\mathbf{X}^{(l)}$. As illustrated in Figure~\ref{fig:CAPE}, each encoder block applies a self-attention mechanism to the input, producing a contextualized representation \( \mathbf{h}_i^{(l)} \) that captures inter-patch correlations:

% \[
% \small
% \mathbf{h}_i^{(l)} = \operatorname{Softmax}\left( \frac{(\mathbf{x}_i^{(l)} \mathbf{W}_Q^{(l)}) (\mathbf{X}^{(l)} \mathbf{W}_K^{(l)})^\top}{\sqrt{d_k^{(l)}}} \right) (\mathbf{X}^{(l)} \mathbf{W}_V^{(l)}). \tag{3} 
% \]



% To mitigate the impact of temporal distribution shifts, we employ Reversible Instance Normalization~\cite{kim2021reversible} on the input data:
% \[
% \mathbf{X}_{\text{norm}} = \text{RevIN}(\mathbf{X}). \tag{1}
% \]
% Next, the normalized data is partitioned into \( N \) patches of fixed length \( L \) using a patching strategy:
% \[
% \left\{ \mathbf{x}_0, \mathbf{x}_1, \mathbf{x}_2, \ldots, \mathbf{x}_N \right\} = \text{Patching}(\mathbf{X}_{\text{norm}}). \tag{2}
% \]
% Each patch \( \mathbf{x}_i \) undergoes a learnable linear projection and is augmented with a positional embedding:
% \[
% \mathbf{x}_i^{(0)} = \mathbf{W}_{p} \mathbf{x}_i + pos(i), \tag{3}
% \]
% where $\mathbf{W_{p}}$ is the projection matrix and \( pos(i) \) denotes the positional embedding for patch \( i \).
% The encoded patches \( \mathbf{x}_i^{(0)} \) are then fed into the CAPE encoders. We denote the output of the $l$ th encoder as $\mathbf{x}_i^{(l)}$. As illustrated in Figure~\ref{fig:CAPE}, each encoder block applies a self-attention mechanism to the input, producing a contextualized representation \( \mathbf{h}_i^{(l)} \) that captures inter-patch correlations:

% \[
% \small
% \mathbf{h}_i^{(l)} = \operatorname{Softmax}\left( \frac{(\mathbf{x}_i^{(l)} \mathbf{W}_Q^{(l)}) (\mathbf{X}^{(l)} \mathbf{W}_K^{(l)})^\top}{\sqrt{d_k^{(l)}}} \right) (\mathbf{X}^{(l)} \mathbf{W}_V^{(l)}). \tag{4} 
% \]


\subsubsection{Layer-Wise Covariate Adjustment}
\label{sec: CA}
% \wei{we did not introduce the meaning of $\mathcal{X}$ before.}

Beyond accounting for the environment during pre-training, the key mechanism to disentangle the correlations between diseases and the environment is required. Since we view the environment as the confounder~\cite{ming2022impact}, we can apply covariate adjustment~\cite{runge2023causal}to control for confounding variables and isolate the true causal effect of interest. The adjustment can be expressed as $p(\mathbf{y} | do(\mathbf{X}=\mathbf{x})) = \int p(\mathbf{y} | \mathbf{x}, \mathbf{z}) p(\mathbf{z}) d\mathbf{z}$, where $\mathbf{z}$ refers to the confounder (environment) and $do(\mathbf{X}=\mathbf{x})$ refers to assigning value $\mathbf{x}$ to the input variable $\mathbf{X}$. Nevertheless, it can be hard to identify $\mathbf{z}$ given a limited amount of attributes in the pre-training dataset. Therefore, we adopt the following assumption to further constrain the problem:

% Beyond accounting for the environment during pre-training, it is essential to disentangle the correlations between diseases and the environment. We treat the environment as a confounder, considering its influence as spurious correlations~\cite{ming2022impact}. To mitigate these spurious correlations, covariate adjustment~\cite{runge2023causal} is employed to control confounding variables and isolate the true causal effect
% $p(\mathbf{y} \mid do(\mathcal{X}=\mathbf{x})) = \int p(\mathbf{y} \mid \mathbf{x}, \mathbf{z}) p(\mathbf{z}) \, d\mathbf{z}$,
% where \( \mathbf{z} \) is the confounder and \( do(\mathcal{X}=\mathbf{x}) \) denotes assigning \( \mathbf{x} \) to \( \mathcal{X} \). However, acquiring \( \mathbf{z} \) is challenging due to limited attributes in the pre-training dataset. Therefore, we adopt the following assumption to further constrain the problem:

\begin{assumption}
\label{assumption1}
The epidemic forecasting problem involves a finite set of environments $\mathbf{Z}$, each of which possesses a consistent and distinct representation $\mathbf{z}^i$.
\end{assumption}

Such an assumption is reasonable as the dynamics of epidemics are primarily driven by a limited number of factors, which collectively define distinct environments. By assuming fixed representations for these environments, models can effectively capture and leverage the unique patterns and interactions specific to each environment. Then we are able to reduce the procedure of covariate adjustment to a weighted sum of $p(\mathbf{y}|\mathbf{x},\mathbf{z})$:
\begin{equation}
\label{Eq: 5}
p(\mathbf{y} | \mathbf{x}) = \sum\nolimits_{\mathbf{Z}} p(\mathbf{y} | \mathbf{x}, \mathbf{z}) \cdot p(\mathbf{z})   \tag{1}
\end{equation}
Since multiple layers can be stacked together, we perform the adjustment in a layer-wise manner. Specifically, we process the input $\mathbf{X}$ using patching~\cite{nie2022time} and self-attention to capture the temporal dependency, yielding the contextualized representations $\mathbf{h}_i$ for patch $i$. Given $\mathbf{h}_i^{(l)}$ at the $l$ th layer, we model Eq.~\ref{Eq: 5} as follows:

\vspace{-3mm}
\begin{align}
\label{Eq: 6}
\mathbf{m}_i^{(l)}  = \sum_{k=1}^{K} (\mathbf{h}_i^{(l)} \odot \mathbf{z}^k) \cdot p(\mathbf{z}^k \mid \mathbf{h}_i^{(l)}), \tag{2}
\end{align}
\vspace{-3mm}

where \( \mathbf{Z} = \{ \mathbf{z}^1, \mathbf{z}^2, \ldots, \mathbf{z}^K \} \) represents the set of fixed environment representations, and \( p(\mathbf{y} | \mathbf{x}, \mathbf{z}) \) is modeled by a hadamard product between $\mathbf{h}_i^{(l)}$ and $\mathbf{z}^k$. Finally, a feedforward neural network \( \mathbf{x}_i^{(l+1)} = \sigma(\mathbf{W}_f^{(l)} \mathbf{m}_i^{(l)}) \) is applied to acquire the output representations, which serves as the input for the next block. At the end of the model, we acquire the final representation $\mathbf{X^{(L)}} = g_\theta(\mathbf{X})$. Then, a task-specific head is applied to predict the target variable $\mathbf{y}=\mathbf{h}_\psi(\mathbf{X}^{(L)})$, where $h_\psi$ is a linear transformation.




% In order to estimate the environment and perform the adjustment at the same time, we also use the weighted sum of $\mathbf{Z}$ as the inferred environment, since they are in the same semantic space. Therefore, following the self-attention layer, the environment representation \( \mathbf{z}_i^{(l)} \) for each patch is computed as below:

% \[
% \label{Eq: env_estimator}
% \mathbf{e}_i^{(l)} = g_{\phi}^{(l)}(\mathbf{h}_i^{(l)}, \mathbf{Z}) = \sum_{k=1}^{K} \mathbf{z}^k \cdot p(\mathbf{z}^k \mid \mathbf{h}_i^{(l)}), \tag{6}
% \]

% where \( \mathbf{Z} = \{ \mathbf{z}^1, \mathbf{z}^2, \ldots, \mathbf{z}^K \} \) represents the set of fixed environment representations, and \( p(\mathbf{z}^k \mid \mathbf{h}_i^{(l)}) \) is the probability of environment \( \mathbf{z}^k \) given the contextualized representation \( \mathbf{h}_i^{(l)} \). To perform covariate adjustment, the estimated environment \( \mathbf{e}_i^{(l)} \) is then combined with the encoded input via a hadamard product:
% \[
% \mathbf{m}_i^{(l)} = \mathbf{h}_i^{(l)} \odot \mathbf{e}_i^{(l)} = \sum_{k=1}^{K} (\mathbf{h}_i^{(l)} \odot \mathbf{z}^k) \cdot p(\mathbf{z}^k \mid \mathbf{h}_i^{(l)}). \tag{7}
% \]
% This formulation effectively reduces the original covariate adjustment function to Eq.~\ref{Eq: 6}, where \( p(\mathbf{y} | \mathbf{x}, \mathbf{z}) \) is modeled as \( \mathbf{x} \odot \mathbf{z} \). Finally, a feedforward neural network \( \mathbf{x}_i^{(l+1)} = \sigma(\mathbf{W}_f^{(l)} \mathbf{m}_i^{(l)}) \) is applied to acquire the output representations, which also serves as the input for the next block. At the end of the model, we acquire the final representation $g_\theta(\mathbf{X}) = \mathbf{X^{(L)}}$. Then, a task-specific head is applied to predict the target variable $\mathbf{y}=\mathbf{h}_\psi(\mathbf{X}^{(L)})$.



\subsubsection{Pseudo Environment Estimator}
\label{sec: env_estimate}

\wei{briefly explain why we hope to estimate the environment?} In order to estimate the environment and perform the adjustment at the same time, we use the weighted sum of $\mathbf{Z}$ as the inferred environment. Therefore, following the self-attention layer, the environment representation \( \mathbf{z}_i^{(l)} \) for each patch at the $l$ th layer is computed by an environment estimator $g_{\phi}^{(l)}$:

\vspace{-3mm}
\[
\label{Eq: env_estimator}
\mathbf{e}_i^{(l)} = g_{\phi}^{(l)}(\mathbf{h}_i^{(l)}, \mathbf{Z}) = \sum_{k=1}^{K} \mathbf{z}^k \cdot p(\mathbf{z}^k \mid \mathbf{h}_i^{(l)}), \tag{3}
\]
\vspace{-3mm}

where \( p(\mathbf{z}^k \mid \mathbf{h}_i^{(l)}) \) is the probability of environment \( \mathbf{z}^k \) given the contextualized representation \( \mathbf{h}_i^{(l)} \). To perform covariate adjustment, the estimated environment \( \mathbf{e}_i^{(l)} \) is then combined with the encoded input via $\mathbf{m}_i^{(l)} = \mathbf{h}_i^{(l)} \odot \mathbf{e}_i^{(l)}$ to yield the same output as Eq.~\ref{Eq: 6}.

Next, we detail the modeling of the conditional probability \( p(\mathbf{z}^k \mid \mathbf{h}_i^{(l)}) \). To better model the causal and spurious correlations, we adopt the following decomposition assumption from~\cite{mao2022causal}:
 
\begin{assumption}
As shown in Figure~\ref{fig: causal_graph}, each input $\mathbf{X}$ can be decomposed into the spurious factor $\mathbf{X}_s$ and the causal factor $\mathbf{X}_c$. Consequently, we decompose representation \( \mathbf{h}_i^{(l)} \) into two corresponding representations:  \( \mathbf{h}_{c,i}^{(l)} \) and \( \mathbf{h}_{s,i}^{(l)} \).
\end{assumption} 

Under this assumption and given that the causal factor is independent from the environment, which yeilds \( \mathbf{h}_{c,i}^{(l)} \perp \mathbf{z}^k \), the probability \( p(\mathbf{z}^k \mid \mathbf{h}_i^{(l)}) \) can be expressed as:
\[
p(\mathbf{z}^k \mid \mathbf{h}_i^{(l)}) = p(\mathbf{z}^k \mid \mathbf{h}_{c,i}^{(l)}, \mathbf{h}_{s,i}^{(l)}) = p(\mathbf{z}^k \mid \mathbf{h}_{s,i}^{(l)}). \tag{4}
\]

Then, we model this conditional probability as a cross-attention score using a softmax function:
\[
\label{eq11}
\pi_i^{k(l)} = \operatorname{Softmax} \left( \mathbf{W}_k^{(l)} \mathbf{z}^k \cdot f_{\text{decomp}}(\mathbf{h}_i^{(l)}) \right), \tag{5}
\]

where \( f_{\text{decomp}}(\mathbf{h}_i^{(l)}) \) denotes the decomposition function extracting spurious factors, modeled as a linear transformation \( f_{\text{decomp}}(\mathbf{h}_i) = \mathbf{W}_h^{(l)} \mathbf{h}_i^{(l)} \), and \( \mathbf{W}_k^{(l)} \) maps the environment representation \( \mathbf{z}^k \) to the same dimensionality as \( \mathbf{h}_{s, i}^{(l)} \). Integrating these components, the final model can be expressed as:

\vspace{-3mm}
\[
\mathbf{X}^{(l+1)} = \sigma \left\{ \mathbf{W}_f^{(l)} \sum_{k=1}^{K} \left( [\mathbf{H}^{(l)} \odot \mathbf{z}^k] \cdot \pi_i^{k(l)} \right) \right\}, \tag{6}
\]
\vspace{-3mm}

where \( \sigma \) represents the activation function.

% The CAPE framework integrates environment inference with temporal data processing to enhance epidemic forecasting. By decomposing representations into causal and spurious factors and modeling environment influences through a weighted sum of fixed environment representations, CAPE effectively adjusts for confounders without requiring explicit exogenous covariates. The combination of reversible normalization, patch-based processing, and self-attention mechanisms ensures robust performance against temporal distribution shifts, making CAPE a powerful tool for epidemic forecasting tasks.


\subsection{Pre-training Objectives for Epidemic Forecasting}
\label{sec: contrast}
To capture a wide range of dynamics from the epidemic time series pile, CAPE applies two self-supervised learning strategies for pre-training. 

% \noindent\textbf{Random Masking.} To capture the inherent characteristics from a vast amount of unlabeled epidemic time series data, we employ a masked time series modeling task~\cite{kamarthi2023pems, goswami2024moment} as shown Figure~\ref{fig:CAPE}(c), which randomly masks input patches by setting them to zero with a probability of 30\%. During training, we utilize Mean Squared Error (MSE) as the reconstruction loss to ensure that the reconstructed data closely matches the original input:

% \[
% \mathcal{L}_{recon} = \frac{1}{N}\sum_{i=1}^{N} MSE(\hat{\mathbf{x}_i}, \mathbf{x}_i) , \tag{9}
% \]

% where $\mathbf{x}_i$ is the $i$ the patch of the original time series and $\hat{\mathbf{x}_i}$ is the reconstructed time series patch given by the model.
% Through this process, the model is able to learn robust representations by predicting the masked segments based on their surrounding context. This not only enhances the model's ability to understand temporal dependencies and patterns within the data but also improves its generalization capabilities when applied to unseen epidemic scenarios. 

\noindent\textbf{Random Masking.} To capture characteristics from large unlabeled epidemic time series data, we use a masked time series modeling task~\cite{kamarthi2023pems, goswami2024moment} (Figure~\ref{fig:CAPE}(c)), which randomly masks input patches by setting them to zero with a 30\% probability. During training, we employ Mean Squared Error (MSE) as the reconstruction loss to ensure that reconstructed data closely matches the original: $\mathcal{L}_{\text{recon}} = \frac{1}{N}\sum_{i=1}^{N} \text{MSE}(\hat{\mathbf{x}}_i, \mathbf{x}_i)$.
where \( \mathbf{x}_i \) is the original time series patch and \( \hat{\mathbf{x}}_i \) is its reconstruction by the model.



\noindent\textbf{Hierarchical Environment Contrasting.} 
% To ensure robust and aligned environment representations of the same time steps under different contexts, we employ a hierarchical contrastive loss on the estimated environment representations during pre-training, as shown in Figure~\ref{fig:CAPE}(b). The contrastive loss includes both the instance level and temporal level, which encourages the representations under different contexts to be similar or dissimilar.
% % \textit{Instance-wise Contrastive Loss.} 
% \textit{Instance-wise contrasting} treats environments from different time series samples (a total of $B$ samples) as negative pairs and encourages them to be encoded to be dissimilar~\cite{yue2022ts2vec}, as shown in the second term of Eq.~\ref{Eq: CL}. This term makes the estimated environment representations from different samples far from each other, yielding diverse environments.
% % \textit{Temporal Contrastive Loss.}
% \textit{Temporal contrasting} creates augmented samples with overlapping areas, as shown in the third term of Eq.~\ref{Eq: CL}. 
% Since the environment is independent of the time series in our causal model and should remain consistent throughout the sequence, the representations of the overlapping regions (a total of $\Omega$) are encouraged to be similar, even under varying contexts\cite{yue2022ts2vec}.
% In our framework, the contrastive loss for both the final time series embeddings \( \mathbf{X}^{L} \) and the estimated environment representations $\mathbf{E}^{(l)} = g_{\phi}^{(l)}(\mathbf{H}^{(l)}, \mathbf{Z})$ are computed in a patch-wise manner. As an example, the contrastive loss for the estimated environments is shown below:
To ensure robust and context-aligned environment representations, we apply a hierarchical contrastive loss during pre-training (Figure~\ref{fig:CAPE}(b)), comprising instance-level and temporal-level components. \textit{Instance-wise contrasting} treats environments from different time series (B samples) as negative pairs, encouraging dissimilar representations~\cite{yue2022ts2vec} and promoting diversity (second term of Eq.~\ref{Eq: CL}). \textit{Temporal contrasting} uses augmented samples with overlapping regions ($\Omega$) to maintain consistent environment representations across sequences despite varying contexts~\cite{yue2022ts2vec} (third term of Eq.~\ref{Eq: CL}). In our framework, contrastive loss is computed patch-wise for both final time series embeddings \( \mathbf{X}^{L} \) and estimated environments \( \mathbf{E}^{(l)} = g_{\phi}^{(l)}(\mathbf{H}^{(l)}, \mathbf{Z}) \). An example of the contrastive loss for estimated environments is shown below:


{
\small
\begin{align}
\label{Eq: CL}
&\mathcal{L}_{\text{CL(j, i)}} = - \mathbf{E}_{j,i} \cdot \mathbf{E'}_{j,i} \nonumber \\
&+ \log \left( \sum_{b\in B} \exp \left( \mathbf{E}_{j,i} \cdot \mathbf{E'}_{b,i} \right) + \mathbb{I}_{j \neq b} \exp \left( \mathbf{E}_{j,i} \cdot \mathbf{E}_{b,i} \right) \right) \nonumber \\
&+ \log \left( \sum_{t \in \Omega} \exp \left( \mathbf{E}_{j,i} \cdot \mathbf{E'}_{j,t} \right) + \mathbb{I}_{j \neq t} \exp \left( \mathbf{E}_{j,i} \cdot \mathbf{E}_{j,t} \right) \right)
\tag{7}
\end{align}
}

% \begin{equation}
% \tiny
% \mathcal{L}_{\text{temp}}^{(j, i)}(\mathbf{h}) = -\log \frac{\exp \left( \mathbf{h}_{j,i} \cdot \mathbf{h}'_{j,i} \right)}{\sum_{i' \in \Omega} \left( \exp \left( \mathbf{h}_{j,i} \cdot \mathbf{h}'_{j,i'} \right) + \mathbb{I}_{i \neq i'} \exp \left( \mathbf{h}_{j,i} \cdot \mathbf{h}_{j,i'} \right) \right)}. \tag{13}
% \end{equation}

% in each layer and the final time series representation are defined as \( \mathcal{L}_{\text{CL}(j, i)}(\mathbf{E}^{(l)}) \) and \( \mathcal{L}_{\text{CL}(j, i)}(\mathbf{X}^{(L)}) \). Therefore, the final loss function is given by:

\textbf{Pre-training Loss.} Finally, combining the reconstruction loss and the contrastive loss, we have the final loss function for pre-training:
\begin{align}
\mathcal{L}_{final} &= \mathcal{L}_{recon}(\mathbf{X}, \mathbf{y})  + \alpha  \mathcal{L}_{\text{CL}}(\mathbf{X}^{(L)}) \nonumber \\
&+ \beta [ 1/l \sum_l \mathcal{L}_{\text{CL}}(\mathbf{E}^{(l)}) ], \tag{8}
\end{align}
where $\alpha$ and $\beta$ are hyperparameters used to balance the contrastive loss for the time series and the estimated environments.


\subsection{Optimization}
\subsubsection{Optimization for Pre-Training}
% \wei{can you follow the Aditya's ICML'24 paper to write this section? I feel currently we are describing this procedure in a too straightforward way}

Simply optimizing the likelihood $p_\theta(\mathbf{y}|\mathbf{X})$ will mislead the time series model to capture the shortcut predictive relation between the history input $\mathbf{X}$ and the future predictions $\mathbf{y}$~\cite{wu2024graph}, which is why the environment should be considered during optimization:
\begin{equation}
\small
\theta^* = \arg\min_\theta \mathbb{E}_{\mathbf{e} \sim p(E),\ (\mathbf{X}, \mathbf{y}) \sim p(\mathcal{Y}, \mathcal{X} \mid E=\mathbf{e})} \left[ \left\| \mathbf{y} - h_\psi(g_\theta(\mathbf{X})) \right\|^2 \right].  \tag{9}
\end{equation}

However, we may not be able to directly acquire the environment representations without external materials and the corresponding encoder. To address this, we treat these environments as hidden variables and optimize them in a data-driven way. Unlike previous methods that approximate the latent probability distribution of environments via variational lower bounds~\cite{wu2024graph}, our approach uses the Expectation-Maximization (EM) algorithm to obtain maximum a posteriori (MAP) estimates of environment representations:

\textbf{Expectation Step (E-Step):} We freeze the transformer encoder and environment estimator, then optimize only the learnable environment representations $\mathbf{Z}$. Setting hyperparameters $\alpha, \beta = 0$, we solve:
\begin{equation}
\mathbf{Z}^{t+1} = \underset{\mathbf{Z}}{\arg\min} \left[ \mathcal{L}_{\text{recon}}(\mathbf{X}, \mathbf{V}) \right]. \tag{10}
\end{equation}
\vspace{-3mm}

\textbf{Maximization Step (M-Step):} We fix the updated environment variables $\mathbf{Z}^{t+1}$ and optimize the encoder and environment estimator by minimizing $\mathcal{L}_{\text{final}}$. Thus, the representation $g_\theta$ and task-specific head $h_\psi$ are updated as:
\begin{equation}
\theta^{t+1}, \psi^{t+1} = \underset{\theta, \psi}{\arg\min} \left[ \mathcal{L}_{\text{final}}(\mathbf{X}, \mathbf{V}, \mathbf{Z}^{t+1}) \right]. \tag{11}
\end{equation}
\vspace{-3mm}

Although our strategy for learning explicit representations resembles codebook optimization~\cite{dong2023peco}, we use these representations for covariate adjustment instead of input reconstruction. The detailed pseudo-code for the optimization procedure is presented in Appendix~\ref{Append_B}.



% \subsubsection{Representation Learning}

% \textbf{Self-supervised Learning.} We adopt reconstruction with random masking as our self-supervised learning method, as shown in Fig~\ref{fig:CAPE} (c). The output embedding $x_i^{(L)}$ is transformed into the reconstruction of the original input patch $x_i$, denoted as $\hat{x_i}$. In this case, EM algorithm is applied during pre-training and we use the mean of each patch's MSE loss as our final loss: $\mathcal{L}_{Sup} = 1/N \sum_{i=0} MSE(\hat{x_i}, x_i)$.


% \textbf{Fine-tuning.} During fine-tuning on the downstream datasets, we fine-tune the whole model using the same EM algorithm. We replace the reconstruction head with a prediction head, which concatenates all the latent representations and maps the input to the prediction target $\hat{y}=W[x_1^{(L)}, x_2^{(L)}, ...]$: $\mathcal{L}_{Sup} = MSE(\hat{y}, y)$.



\subsubsection{Optimization for Downstream Tasks}

% \wei{not sure why we put the loss here; it should probably be merged with the contrastive learning section; in this section we focus on describing the optimiation for pre-training and finetuning. }
% \noindent\textbf{Self-supervised Learning.} For pre-training, we employ a self-supervised approach based on reconstruction with random masking, as illustrated in Figure~\ref{fig:CAPE}(c). Given an input patch $\mathbf{x}_i$, its output embedding $\mathbf{x}_i^{(L)}$ is fed into a reconstruction head to produce $\hat{\mathbf{x}_i}$. EM Algorithm is applied during this process.
% We apply the EM algorithm during pre-training, and define the supervised loss as:
% $\mathcal{L}_{Sup} = \frac{1}{N}\sum_{i=1}^{N} MSE(\hat{\mathbf{x}_i}, \mathbf{x}_i).$

\noindent\textbf{Fine-tuning.} For downstream tasks, we fine-tune the entire model using MSE loss, still employing EM for optimization. The difference is that the reconstruction head is replaced by a prediction head that takes the concatenated latent representations $[\mathbf{x}_1^{(L)}, \mathbf{x}_2^{(L)}, \ldots ]$ and maps them to the future prediction:
$\hat{\mathbf{y}} = \mathbf{W}[\mathbf{x}_1^{(L)}, \mathbf{x}_2^{(L)}, \ldots]$.
% with the training objective:
% $\mathcal{L}_{Sup} = MSE(\hat{\mathbf{y}}, \mathbf{y})$.


\noindent\textbf{Zero-shot.} For zero-shot forecasting, the model remains frozen and no parameter is updated. Like the implementation from the Moment model~\cite{goswami2024moment}, we retain the pre-trained reconstruction head and mask the last patch of the input to perform forecasting: $\hat{\mathbf{y}} = \hat{\mathbf{x}_n}$.





\section{Experiments}\label{sec_exp}
%\hp{Accelerating IM simulation~\cite{tang2015influence}}

% \begin{itemize}
%     \item 6.1. Problem setting of three COPs, including the general model and three specific CO problems 
%     \item 6.2. Experiment Setting (hyperparameters, details of training, evaluation, and test) 写在appendix里吧
%     \item 6.3. Performance analysis 这个要占半页
% \end{itemize}

%\hp{need to think of a way to compress these tables / visuals.} 

%\hp{\cancel{Baselines}; hyperparamters; \cancel{metrics}; etc.}

With theoretical guarantees on the existence and convergence of NE for ACCES games, we are also interested in how our proposed algorithm CCDO-RL works empirically. To evaluate this, we conduct experiments of CCDO-RL on three distinct ACCES game instances introduced in Section \ref{sub_exp_ins} and analyze the performance of CCDO-RL in Section \ref{sub_train_eval}. Section 6.2.1 aims to empirically demonstrate the convergence (Figures \ref{fig_exploit_20} and \ref{fig_exploit_50}) of the algorithm CCDO-RL over realistic CO problems, and show its consistency with Theorem \ref{CCDOA}. Section 6.2.2 intends to show the average reward (to seen training graphs) as well as the generalizability (to unseen test graphs) of the combinatorial player in real-world ACCES games (shown in Tables \ref{tab_aver}, and \ref{tab_gene}).

\subsection{Three Instances of ACCES Games} \label{sub_exp_ins}
% \hp{This para does not make much sense. Need to follow the framework in the Preliminaries section.}
% For combinatorial optimization problems in real-world applications, situations are more complicated and intractable due to changeable environmental or physical parameters. The form of parameter sets is very crucial because different types have different solvability and computation complexity. Forms of parameter sets mainly contain discrete sets, interval sets \cite{buchheim2018robust} like polyhedral and ellipsoid, probability distributions \cite{carlsson2018wasserstein}, and variable functions \cite{krause2008robust}.

% In reality, these parameters are often impacted by some common factors, such as conditions of weather, transportation, and individual personalities. \cite{kalimeris2019robust} proposed an assumption that real instances (e.g. demands in CVRP, coverages in CSP) 
%Considering affected or attacked COPs, the real instance $\{\theta_{i}\}$ always relied on the estimated value $\{\hat{\theta}_{i}$\} and the variation determined by independent factors $\{g_{i}\}$ and environment/physical parameters/attacker actions $\{\eta\}$. The concrete parameter influence model is stated as follows:

We consider a certain COP which is parameterized with $\{\theta_{i}\}$, where $i$ is the index of nodes (such as a target in security games) -- e.g., such parameters can be interpreted as attack probability of targets.
%coverage radius, customer's demands, or attack probability of targets. 
In real-world applications, we often need to estimate such parameters before solving the COPs. Unfortunately, the estimation $\{\hat{\theta}_{i}\}$ often bears a gap to the true value $\{\theta_{i}\}$, which derives from e.g. environment (aleatoric) uncertainty, model (epistemic) uncertainty, or an attacker trying to manipulate the defender's utility. We use a generic model to formulate this gap:
\begin{equation}\label{linrob}
    \theta_{i} = \hat{\theta}_{i} + y \cdot \tau_{i},
\end{equation}
where $y$ represents the strategy of the nature/attacker, $\tau_{i}$ is the environment factors like weather and transportation conditions, or human subjective factors like the preference of the attacker. 
Such abstraction can represent a wide range of ACCES games, such as facility location covering problems \cite{an2020battery, TIRKOLAEE2020340}, CVRP \cite{vehiclerouting.ch8,dinh2018exact, FLORIO20231081}, security patrolling (OP) \citep{xu2021robust}, and influence maximization problem \cite{kalimeris2019robust}. We describe three instances of ACCES games based on the model (\ref{linrob}).%Based on this model (\ref{linrob}), we focus on three combinatorial optimization problems with attacks or environmental/physical influence.

% \hp{Hard to follow. We should point out what are the two players, what are X, Y, u etc}

\textbf{Adversarial Covering Salesman Problem (ACSP):} In a map of cities, every city $i$ has a coverage $\theta_{i}$. A salesman finds the shortest path such that all cities are visited or covered, with $\theta_{i}$ influenced by physical factors $\tau_i$ and transportation parameters $y$ based on Eq.(\ref{linrob}). The salesman is Player 1 where $X$ consists of the feasible paths of the salesman. Nature is Player 2 with $Y$ = $[0, 1]^K \ni y, K \in \mathbb{N}$. The utility function of Player 1 $u$ is the opposite of the total traveling distance.

\textbf{Adversarial Capacitated Vehicle Routing Problem (ACVRP):} A vehicle with a constrained capacity of goods finds the shortest path under the worst case with the $i_{th}$ customer's demand $\theta_i$ changed by environmental factors $\tau_i$ and weather parameter $y$ on Eq.(\ref{linrob}). The vehicle is Player 1 where $X$ is the set of the feasible path $x$. Nature is Player 2 where $Y$ is $[0, 1]^K \ni y, K \in \mathbb{N}$. The utility function of Player 1  $u$ is the opposite of total delivery distance satisfying all the demands of customers.


\textbf{Patrolling Game (PG):} The patrolling game is described in the introduction.

For all the problem instances, we run our algorithm on two problem sizes: 20 nodes and 50 nodes. The detailed description and problem parameters of the three game instances are in Appendix \ref{app_ex_para_set}.

% Similarly, in the vehicle route problem (VRP), conditions with correlated parameters arouse broad attention from scholars \cite{vehiclerouting.ch8,dinh2018exact,FLORIO20231081}. \cite{dinh2018exact} considered the demand correlation by geographical proximity of nodes, described by some independent random variables in the fractional form. \cite{FLORIO20231081} utilized 'external factors' to stand for unknown covariates affecting all demands and presented a Bayesian model to learn correlations. Further more, about IM problems, \cite{kalimeris2019robust} combined node features and uncertain hyperparameters to fit the influence probability on each edge.

% \subsection{Training CCDO-RL}

% For all the problems, CCDO-RL adopts the REINFORCE algorithm with an attention-based encoder-decoder framework \cite{kool2018attention} (used as an inductive graph representation component) to learn a (generalizable) COP solver for one player (protagonist), and PPO \cite{schulman2017proximal} to train a policy for the other player (adversary) whose strategy space is continuous. CCDO-RL is trained with 50 epochs on a set of 10,000 graphs (with 20 or 50 nodes). The hyperparameters of CCDO-RL are specified in Appendix \ref{app_ex_para_set} (Table \ref{tab_hyper_ccdorl}). Our code is included as supplementary material for ease of reproduction. 
% % \hp{need to specify hyperparas}

\subsection{Performance of CCDO-RL}\label{sub_train_eval}

Two aspects are evaluated for the performance of CCDO-RL, i.e., i) Convergence to NE (Section \ref{sub_per_conver}) exploring whether CCDO-RL can compute the NE, and ii) Protagonist policy's average reward and generalizability (Section \ref{sub_per_rob}). Generalizability refers to the ability of RL models trained on previously seen graphs (problem instances), to perform well on a new set of unseen test graphs. The model’s usability is enhanced by generalizability, rather than focusing solely on the average reward, which is a critical motivation in the literature on RL for COPs \citep{khalil2017learning, kool2018attention}.

For all the problems, CCDO-RL adopts the REINFORCE algorithm with an attention-based encoder-decoder framework \citep{kool2018attention} (used as an inductive graph representation component) to learn a generalizable COP solver for Player 1 (protagonist), and PPO to train a policy for Player 2 (adversary) whose strategy space is continuous. CCDO-RL is trained on a set of 10,000 graphs (with 20 or 50 nodes). The hyperparameters of CCDO-RL are specified in Appendix \ref{app_ex_para_set} (Table \ref{tab_hyper_ccdorl}). Our code is included as supplementary material and will be open-sourced for ease of reproduction. 

% \textbf{Training.} For all the problems, CCDO-RL adopts the REINFORCE algorithm with attention-based encoder-decoder framework \cite{kool2018attention} (used as an inductive graph representation component) to learn a (generalizable) COP solver for one player (protagonist), and PPO \cite{schulman2017proximal} to train a policy for the other player (adversary) whose strategy space is continuous. CCDO-RL is trained with 50 epochs on a set of 10,000 graphs (with 20 or 50 nodes). 

% \hp{We should first present results about convergence as it is mostly aligned with the theory.}

\subsubsection{Convergence to NE} \label{sub_per_conver}

Exploitability is a common metric to describe the closeness to true NE by calculating the sum of performance distances between each new best response and subgame NE, i.e. $\sum_{i=1,2} U(\pi_{i,k}^{br}, \sigma_{-i,k}) - U(\sigma)$ in the general two-player game. Since our game is zero-sum, the calculation is as follows:
\begin{equation*}
   \text{Exploitability}(\sigma) = \max_{\pi_1 \in \Sigma_1} U(\pi_1, \sigma_{2}) - \min_{\pi_2 \in \Sigma_2} U(\sigma_1, \pi_2).
\end{equation*}
From Figure \ref{fig_exploit_20}, we can see that CCDO-RL can converge to approximate NE in 25 iterations or less (in the PG setting), reaching 0.05 in ACSP, 0.10 in ACVRP, and 0.03 in PG with 20 nodes. Similar results are observed in problems with 50 nodes (see Figure \ref{fig_exploit_50} in Appendix \ref{app_exp}). These results validate the effectiveness of CCDO-RL in finding the NE for various types of games.

%Similarly, the exploitability of three COPs in 50 nodes is provided in the appendix \ref{app_exp}.
\vspace{-\baselineskip}
\begin{figure}[htbp]
	\centering
    \subfigure[ACSP20]{
    \label{csp20_nashconv}
    \includegraphics[scale=0.20]{Figures/nashconv_log_csp20_sm_7.eps}
    }
    \subfigure[ACVRP20]{
    \label{cvrp20_nashconv}%文中引用该图片代号
    \includegraphics[scale=0.20]{Figures/nashconv_log_svrp20_sm_7.eps}
    }
    \subfigure[PG20]{
    \label{opsa20_nashconv}
    \includegraphics[scale=0.20]{Figures/nashconv_log_pg20_sm_7.eps}
    }
    \caption{Exploitability curve of CCDO-RL on three games of 20 nodes}
    \label{fig_exploit_20}
\end{figure}
\vspace{-\baselineskip}
\subsubsection{Average reward and Generalizability of Combinatorial player} \label{sub_per_rob}
% \subsubsection{Robustness and Generalizability of Protagonist Policy} \label{sub_per_rob}
%\hp{CCDO-RL being better in these following metrics is only kind of a by-product.}

% \textbf{Evaluation.} The learned policies are then tested on 200 graphs, where 100 of them are randomly selected from the 10,000 training graphs, and the other 100 are unseen graphs. 
% We use two metrics to evaluate the performance of different policies for the protagonist player: \textbf{Average proportional loss} $R-$ describes the policy overfitting degree \citep{lanctot2017unified}; \textbf{Reward} evaluates the performance of the protagonist with the adversary under three COPs.  
% \begin{eqnarray}
%         &R- = (\hat{D} - \hat{O}) / \hat{D}.
% \end{eqnarray}
% in which $\hat{D}$ is the mean value of the diagonals and $\hat{O}$ is the mean value of the off-diagonals in the payoff matrix provided in the Appendix \ref{app_exp}.

% Because the protagonist policy is trained against a powerful adversary under our ACCES game setting, the obtained policy is naturally robust against adversarial perturbations. This subsection sheds a bit of light on this perspective and quantifies the extent of robustness of CCDO-RL as well as the ability of RL to generalize to unseen test graphs.

\textbf{Evaluation.} The learned policies are tested on 200 graphs, with 100 being randomly selected from the 10,000 training graphs (to show the average reward), and the other 100 being unseen graphs (to test policy generalization). We evaluate the performance of the protagonist with the adversary under three COPs. For each COP, the performance is considered both on the 20-node and 50-node map.
% We use two metrics to evaluate the performance of different policies for the protagonist player: \textbf{Average proportional loss} $R-$ describes the policy overfitting degree \citep{lanctot2017unified}; \textbf{Reward} evaluates the performance of the protagonist with the adversary under three COPs.

\textbf{Baselines.} There are heuristic algorithms for each game instance (Heuristic in Table \ref{tab_aver} and \ref{tab_gene}) and a single-player RL algorithm. For ACVRP, we adopt the Tabu Search algorithm (Tabu) \citep{li2020improved} as the heuristic algorithm, which is widely applied in the routing problem. For ACSP, the common benchmark local search algorithm, LS2 \citep{golden2012generalized}, is used. For PG, we choose the greedy algorithm as the baseline. The "RL against Stoc" algorithm in Tables \ref{tab_aver} and \ref{tab_gene} is identical to the protagonist model in CCDO-RL but trained in environments with stochastic adversarial perturbations.

% \textbf{Baselines.} There are a heuristic algorithms for each game instance {\color{red} (Heuristic mentioned in the Table \ref{tab_aver} and \ref{tab_gene})} and a single-player RL algorithm. For ACVRP, we adopt the Clarke-Wright (CW) algorithm \citep{pichpibul2013heuristic} and the Tabu Search algorithm (Tabu) \citep{li2020improved} as heuristics, which are applied widely in the routing problem. For ACSP, two common benchmark local search algorithms, LS1 and LS2 \citep{golden2012generalized}, are used. For PG, we choose a local search algorithm \citep{vansteenwegen2009iterated} and the greedy algorithm as the heuristic baselines. {\color{red} The "RL  against Stoc" algorithm referred to Tables \ref{tab_aver} and \ref{tab_gene}} is identical to the protagonist model in CCDO-RL {\color{red} but trained on environments with stochastic adversarial perturbations.} 

\textbf{Average Reward.}  As illustrated in Table \ref{tab_aver}, our algorithm achieves a better average reward than baselines (10.08\% improvement on average of all settings against two baselines), regardless of CO instance or problem size, when confronting the adversary trained by CCDO-RL. In the setting of CSP-20 nodes, the average reward is improved by 46.98\% compared to the heuristic and by 7.14\% compared with the RL against Stoc. For the 50-node setting, the improvements are 45.91\% and 5.28\% respectively. Similarly, the improvements in contrast to Heuristic and RL against Stoc are as follows: 1.72\% and 3.01\%  for CVRP-20 nodes, 0.75\% and 4.46\% for CVRP-50 nodes, 4.17\% and 1.48\% for PG-20 nodes, and 10.60\% and 4.38\% for PG-50 nodes.

\textbf{Generalizability.} From Table \ref{tab_gene}, CCDO-RL continues to achieve a better average reward when facing the adversary, demonstrating that the learned RL policies generalize well to unseen graphs. Even though the non-RL baselines do have access to the graph structures and other problem information of the unseen problem instances, CCDO-RL can obtain comparable performances without re-training on the new problem instances. The improvements versus Heuristic and RL against Stoc are 46.61\% and 7.02\% for CSP-20 nodes, 42.24\% and 3.94\% for CSP-50 nodes, 1.12\% and 1.56\% for CVRP-20 nodes, 0.90\% and 5.05\% for CVRP-50 nodes, 5.35\% and 2.40\% for PG-20 nodes, and 12.17\% and 10.33\% for PG-50 nodes. Even when confronting the stochastic adversary, CCDO shows superior generalizability compared to two baselines across three COPs, with average improvements of 6.31\%, 3.42\%, and 3.95\% respectively. Detailed results are provided in Appendix \ref{app_exp} (Tables \ref{tab_csp_full_20} - \ref{tab_op_full_50}). 
% The model’s usability is enhanced by the ability to generalize rather than focusing solely on the average reward, which is a critical motivation of the RL for combinatorial optimization literature \citep{khalil2017learning, kool2018attention}.  

\begin{remark}
    In CO problems (or more broadly, operations research and economics), it is known that achieving solution quality improvements against strong baselines (e.g., the RL methods trained with a stochastic adversary) is very challenging, and the margins are usually small \citep{kool2018attention}, sometimes even less than 1\%. However, these “tiny” marginal improvements in profits keep small business owners in the real world alive. Last, the improvement depends a lot on the problem settings, and we show that sometimes the improvement can be much more significant.
\end{remark}
\vspace{-\baselineskip}
% \textbf{Performance analysis.} The robustness results of CCDO-RL for ACSP are shown in Table \ref{tab_csp}. We have the following observations: 1) On both of the 100 seen/unseen graphs, single-player RL performs better than heuristic algorithms no matter whether attacked or not. (2) When confronting the adversary trained by CCDO-RL, CCDO-RL exceeds RL by 0.25 and 0.24 on the training set, and by 0.25 and 0.18 on the test set, respectively under the 20-node and 50-node graphs. This demonstrates the robustness of CCDO-RL. 3) Compared to the performance of the training set with that of the test set, we can see that RL and CCDO-RL both maintain a certain degree of generalization. Similar results for ACVRP (Table \ref{tab_cvrp}) and SPG (Table \ref{tab_op}) are provided in Appendix \ref{app_exp}. 

\begin{table}[ht]
  \caption{Average reward against CCDO-RL's adversary (on seen graphs)}
  \vspace{\baselineskip}
  \label{tab_aver}
  \centering
  \small
  \begin{tabular}{lllllll}
    \toprule
    \multirow{2}{*}{method} & \multicolumn{2}{c}{ACSP (Mean$\pm$Std)} & \multicolumn{2}{c}{ACVRP (Mean$\pm$Std)} & \multicolumn{2}{c}{PG (Mean$\pm$Std)} \\
    \cmidrule(r){2-3} \cmidrule{4-5} \cmidrule(r){6-7}
                            & 20 nodes & 50 nodes & 20 nodes & 50 nodes & 20 nodes & 50 nodes\\
    \midrule
    Heuristic & 6.13$\pm$1.20 & 7.55$\pm$1.42 & 7.65$\pm$1.23  & 13.38$\pm$1.70 & 2.64$\pm$1.03 & 4.53$\pm$1.84   \\
    RL against Stoc    & 3.50$\pm$0.47  & 4.55$\pm$0.62  & 7.55$\pm$1.16  & 13.90$\pm$1.63 & 2.71$\pm$0.90 & 4.80$\pm$2.18   \\
    CCDO-RL   & $\pmb{3.25}$$\pm$0.42 & $\pmb{4.31}$$\pm$0.51  & $\pmb{7.42}$$\pm$1.21  & $\pmb{13.28}$$\pm$1.52 &  $\pmb{2.75}$$\pm$0.87 & $\pmb{5.01}$$\pm$1.91  \\
    \bottomrule
  \end{tabular}
\end{table}
\vspace{-\baselineskip}

\begin{table}[htp]
  \caption{Generalizability against CCDO-RL's adversary (on unseen graphs)}
  \vspace{\baselineskip}
  \label{tab_gene}
  \centering
  \small
  \begin{threeparttable}
  \begin{tabular}{lllllll}
    \toprule
    \multirow{2}{*}{method} & \multicolumn{2}{c}{ACSP (Mean$\pm$Std)} & \multicolumn{2}{c}{ACVRP (Mean$\pm$Std)} & \multicolumn{2}{c}{PG (Mean$\pm$Std)} \\
    \cmidrule(r){2-3} \cmidrule{4-5} \cmidrule(r){6-7}
                            & 20 nodes & 50 nodes & 20 nodes & 50 nodes & 20 nodes & 50 nodes\\
    \midrule
    Heuristic & 6.20$\pm$1.33 & 7.60$\pm$1.37   & 7.64$\pm$1.30  & 13.27$\pm$1.87 & 2.43$\pm$0.98 & 4.19$\pm$1.69    \\
    RL against Stoc  & 3.56$\pm$0.37  & 4.57$\pm$0.58  & 7.67$\pm$1.30  & 13.85$\pm$1.53 &  2.50$\pm$0.95 & 4.26$\pm$2.17 \\
    CCDO-RL   & $\pmb{3.31}$$\pm$0.35 & $\pmb{4.39}$$\pm$0.52  & $\pmb{7.55}$$\pm$1.28  & $\pmb{13.15}$$\pm$1.59 & $\pmb{2.56}$$\pm$0.92 & $\pmb{4.70}$$\pm$1.94\\

    \bottomrule
  \end{tabular}
  \begin{tablenotes}
      \footnotesize
      \item[1] For the average reward of ACSP and ACVRP, smaller is better while for that of PG larger is better.
  \end{tablenotes}
  \end{threeparttable}
\end{table}
\vspace{-\baselineskip}
% two heuristics and one RL
% \begin{table}[ht]
%   \caption{{\color{red} Average reward of CCDO-RL (on seen graphs). For the value of CSP and CVRP, larger is better while for that of PG smaller is better.}}
%   \label{tab_aver}
%   \centering
%   \small
%   \begin{tabular}{lllllll}
%     \toprule
%     \multirow{2}{*}{method} & \multicolumn{2}{c}{CSP (Mean$\pm$Std)} & \multicolumn{2}{c}{CVRP (Mean$\pm$Std)} & \multicolumn{2}{c}{PG (Mean$\pm$Std)} \\
%     \cmidrule(r){2-3} \cmidrule{4-5} \cmidrule(r){6-7}
%                             & 20 nodes & 50 nodes & 20 nodes & 50 nodes & 20 nodes & 50 nodes\\
%     \midrule
%     Baseline 1 & 4.52$\pm$0.71  & 5.98$\pm$0.94 & 7.64$\pm$1.56  & 13.49$\pm$2.10 & 2.71$\pm$1.10 & 1.82$\pm$1.40   \\
%     Baseline 2 & 6.13$\pm$1.20 & 7.55$\pm$1.42   & 7.65$\pm$1.23  & 13.38$\pm$1.70 & 2.64$\pm$1.03 & 1.47$\pm$0.99  \\
%     RL {\color{red}against Stoc}    & 3.50$\pm$0.47  & 4.55$\pm$0.62  & 7.55$\pm$1.16  & 13.90$\pm$1.63 & 2.71$\pm$0.90 & 1.54$\pm$1.03   \\
%     CCDO-RL   & $\pmb{3.25}$$\pm$0.42 & $\pmb{4.31}$$\pm$0.51  & $\pmb{7.42}$$\pm$1.21  & $\pmb{13.28}$$\pm$1.52 &  $\pmb{2.75}$$\pm$0.87 & $\pmb{1.87}$$\pm$1.22  \\
%     \bottomrule
%   \end{tabular}
% \end{table}


% \begin{table}[htp]
%   \caption{{\color{red}Generalizability of CCDO-RL (on unseen graphs)}}
%   \label{tab_gene}
%   \centering
%   \small
%   \begin{threeparttable}
%   \begin{tabular}{lllllll}
%     \toprule
%     \multirow{2}{*}{method} & \multicolumn{2}{c}{CSP (Mean$\pm$Std)} & \multicolumn{2}{c}{CVRP (Mean$\pm$Std)} & \multicolumn{2}{c}{PG (Mean$\pm$Std)} \\
%     \cmidrule(r){2-3} \cmidrule{4-5} \cmidrule(r){6-7}
%                             & 20 nodes & 50 nodes & 20 nodes & 50 nodes & 20 nodes & 50 nodes\\
%     \midrule
%     Baseline 1 & 4.53$\pm$0.79  & 5.95$\pm$0.96 & 7.55$\pm$1.39  & 13.35$\pm$2.04 & 2.52$\pm$1.08 & $\pmb{1.86}$$\pm$1.44  \\
%     Baseline 2 & 6.20$\pm$1.33 & 7.60$\pm$1.37   & 7.64$\pm$1.3  & 13.27$\pm$1.87 & 2.43$\pm$0.98 & 1.52$\pm$1.20    \\
%     RL {\color{red}against Stoc}  & 3.56$\pm$0.37  & 4.57$\pm$0.58  & 7.67$\pm$1.30  & 13.85$\pm$1.53 &  2.50$\pm$0.95 & 1.03$\pm$5.05 \\
%     CCDO-RL   & $\pmb{3.31}$$\pm$0.35 & $\pmb{4.39}$$\pm$0.52  & $\pmb{7.55}$$\pm$1.28  & $\pmb{13.15}$$\pm$1.59 & $\pmb{2.56}$$\pm$0.92 & 1.35$\pm$5.09\\

%     \bottomrule
%   \end{tabular}
%   \begin{tablenotes}
%       \footnotesize
%       \item[1] For the value of CSP and CVRP, larger is better while for that of PG smaller is better.
%   \end{tablenotes}
%   \end{threeparttable}
% \end{table}





\section{Conclusion}
\gxlnote{This paper presents a novel and effective training method aimed at mitigating the memorization problem in diffusion models. By analyzing the relationship between training loss and memorization, we apply different treatments to samples based on their degree of memorization, minimizing the risk of memorization. Additionally, considering that model directly learning from data can increase the likelihood of memorization and the same data may have
different interpretations on different shards, we employ several data shards to train multiple proxy diffusion models. Through multiple proxy diffusion models aggregation and redistribution of easily memorable samples cross shards, we obtain the final model, achieving a balance between mitigating memorization and maintaining image quality. We experimentally show that our method performs favorably with many existing related methods in different scenarios and datasets.}
We firmly believe that this training strategy has a broad application prospect and great development potential in the field of data privacy protection.


\bibliographystyle{IEEEtran}

\bibliography{arxiv}

% \vspace{-0.2cm}
\begin{IEEEbiography}[{\includegraphics[width=1in,height=1.25in,clip,keepaspectratio]{imgs/author/gxl.jpg}}]{Xiaoliu Guan}
received a bachelor's degree from the School of Computer Science, Wuhan University (Wuhan, China) in 2024. She is currently pursuing a master’s degree in computer science at the School of Computer Science, Wuhan University (Wuhan, China). Her research interests mainly include data privacy in diffusion models.
\end{IEEEbiography} 

\begin{IEEEbiography}[{\includegraphics[width=1in,height=1.25in,clip,keepaspectratio]{imgs/author/wuyu.png}}]{Yu Wu}
 is a Professor with the School of Computer Science at Wuhan University, China. He received his Ph.D. degree from the University of Technology Sydney, Australia in 2021. From 2021 to 2022, he was a postdoc at Princeton University. His research interests are controllable generation and multi-modal perception. He was the recipient of AAAI New Faculty Highlight 2024 and Google PhD Fellowship 2020. He served as the Area Chair for CVPR, ICCV, ECCV, and NeurIPS, and also served as the Workshop Chair of CVPR 2023.
\end{IEEEbiography}
\begin{IEEEbiography}[{\includegraphics[width=1in,height=1.25in,clip,keepaspectratio]{imgs/author/hhy.jpg}}]{Huayang Huang}
 received the master's degree from the School of Cyber Science and Engineering, Wuhan University (Wuhan, China), in 2024. She is currently working toward the PhD degree with the School of Computer Science, Wuhan University (Wuhan, China). Her research interests focus on safe generative AI.
\end{IEEEbiography}
\begin{IEEEbiography}[{\includegraphics[width=1in,height=1.25in,clip,keepaspectratio]{imgs/author/liuxiao.png}}]{Xiao Liu}
received a bachelor's degree from the School of Computer Science, Wuhan University (Wuhan, China) in 2024. He is currently pursuing a master's degree in computer science at the  School of Cyber Science and Engineering, Wuhan University (Wuhan, China).  His research interests mainly include diffusion models and multimodal learning.
\end{IEEEbiography} 
\begin{IEEEbiography}[{\includegraphics[width=1in,height=1.25in,clip,keepaspectratio]{imgs/author/miaojiaxu.png}}]{Jiaxu Miao}
received the PhD degree from the University of Technology Sydney, in 2021. He is an assistant professor with the School of Cyber Science and Technology, Sun Yat-sen University, (Shenzhen, China).
His research interests include visual safety and understanding.
\end{IEEEbiography}
\begin{IEEEbiography}[{\includegraphics[width=1in,height=1.25in,clip,keepaspectratio]{imgs/author/yiyang.png}}]{Yi Yang}
(Senior Member, IEEE) is a distinguished Professor with the College of Computer Science and Technology, Zhejiang University. He has authored over 200 papers in top-tier journals and conferences. His papers have received over 70,000 citations, with an H-index of 128. He has received more than 10 international awards in the field of AI, such as the Zhejiang Provincial Science Award First Prize, the Australian Research Council Discovery Early Career Research Award, the Australian Computer Society Gold Digital Disruptor Award, and the Google Faculty Research Award. His current research interests include machine learning and its applications to multimedia content analysis and computer vision, such as multimedia retrieval and generation understanding.
\end{IEEEbiography}
\clearpage
\newpage
% % \appendices
\appendix
\section*{Loss Analysis on CIFAR-100, AFHQ-DOG, and LAION-10k }
We present the results of loss analysis for CIFAR-100, AFHQ-DOG, and LAION-10k in \cref{fig:cifar100_loss}, \cref{fig:afhq_loss}, and \cref{fig:laion_loss}. 
The results obtained on CIFAR-100 and LAION-10k show similarities to those on CIFAR-10. 
However, on AFHQ-DOG, due to its fewer images, the model exhibits a memorization phenomenon across the entire dataset, resulting in less noticeable differences.
% \vspace{-0.7cm}
\begin{figure}[t]
  \centering
  \setlength{\abovecaptionskip}{-4pt} % 设置标题上方的间距为 -5pt
  \setlength{\belowcaptionskip}{-10pt} % 设置标题下方的间距为 -5pt
  \includegraphics[height=6.5cm]{imgs/cifar100.pdf}
  \caption{
  Comparison of the losses between memorized and non-memorized images on CIFAR-100. 
  }
  \label{fig:cifar100_loss}
\end{figure}
% \clearpage
% \vspace*{0pt}
\begin{figure}[t]
  \centering
  \setlength{\abovecaptionskip}{-4pt} % 设置标题上方的间距为 -5pt
  \setlength{\belowcaptionskip}{-10pt} % 设置标题下方的间距为 -5pt
  \includegraphics[height=6.5cm]{imgs/afhq.pdf}
  \caption{
  Comparison of the losses between memorized and non-memorized images on AFHQ-DOG.
  }
  \label{fig:afhq_loss}
\end{figure}

\vspace*{0pt}
\begin{figure}[t]
  \centering
  \setlength{\abovecaptionskip}{-4pt} % 设置标题上方的间距为 -5pt
  \setlength{\belowcaptionskip}{-5pt} % 设置标题下方的间距为 -5pt
  \includegraphics[height=6.5cm]{imgs/SD_loss_vs_timestep.pdf}
  \caption{
  Comparison of the losses between memorized and non-memorized images on LAION-10k. 
  }
  \label{fig:laion_loss}
\end{figure}

\section*{Experiments of DP-SGD with varying noise multipliers}
We conduct a series of experiments with DP-SGD (Differentially-Private Stochastic Gradient Descent)~\cite{abadi2016deep} changing the noise multipliers $\tau$ to 0.0002, 0.0005, and 0.0008 to compare our method with different noise levels in DP. Results are shown in ~\cref{tab:dp_parameters}.  When the noise multiplier is set to 0.0005, DP-SGD achieves the best scores in terms of MQ and FID. However, all DP-SGD results improve the FID score compared with the baseline model. When $\tau$=0.0005, DP-SGD slightly reduces the memorization (101 v.s. 111), but it is still far from comparable to the memorization reduction capability of our method.
%our method still surpasses it, demonstrating lower memory usage and superior image quality.
% Table generated by Excel2LaTeX from sheet 'Sheet1'
% \vspace{-0.5cm}
\begin{table}[t]
  \centering
  \caption{Performance of DP-SGD across multiple experiments with varying noise multiplier $\tau$.}
  {\fontsize{6.5}{8}\selectfont %
    \begin{tabular}{c|ccc|c}
    \specialrule{\heavyrulewidth}{0pt}{0pt} % 加粗的水平线,位于表格底部
    \hline
    \multirow{2}[4]{*}{Method} & \multicolumn{4}{c}{CIFAR-10} \bigstrut\\
\cline{2-5}         & MQ$_{0.4}$ & MQ$_{0.5}$ & MQ$_{0.6}$$\downarrow$ & FID$\downarrow$  \bigstrut\\
    \hline
    \hline
    Default (DDPM)~\cite{ho2020denoising} & 111   & 465   & 2030  & 8.81 \bigstrut[t]\\
    DP-SGD~\cite{abadi2016deep} $\tau$=0.0002 & 148   & 728   & 3200  & 12.55 \\
    DP-SGD~\cite{abadi2016deep} $\tau$=0.0005 & 101   & 380   & 1716  & 10.02 \\
    DP-SGD~\cite{abadi2016deep} $\tau$=0.0008 & 124   & 549   & 2498  & 13.82 \bigstrut[b]\\
    \hline
    Ours & \textbf{10} & \textbf{73} & \textbf{623} & \textbf{8.33} \bigstrut\\
    \hline
     \specialrule{\heavyrulewidth}{0pt}{0pt} % 加粗的水平线,位于表格底部
    \end{tabular}%
    }
  \label{tab:dp_parameters}%
\end{table}%

\section*{More Ablation Study}
\crnote{We randomly split the data evenly on AFHQ-DOG and LAION-10k, where no class label is available on these datasets.
Experiments demonstrate the effectiveness of our method in this setting.
Additionally, we conduct ablation experiments on CIFAR-10 by randomly splitting the data evenly (without using class labels). Results are shown in ~\cref{tab:nonclass}.
% We believed that a uniformly distributed class was better for the model’s performance.
We find randomly splitting (without class labels) slightly affects the generation quality. }
\begin{table}[t]
  \centering
  \caption{Ablation experiments of randomly splitting the data evenly.}
  % \vspace{-1.1em}
  % \setlength{\abovecaptionskip}{-40pt}
    {\fontsize{6.5}{8}\selectfont %
    \begin{tabular}{P{3.0cm}|P{1.0cm}P{1.0cm}P{1.0cm}|P{0.8cm}}
     \specialrule{\heavyrulewidth}{0pt}{0pt} % 加粗的水平线,位于表格底部
    \hline
    \multirow{2}[4]{*}{Method} & \multicolumn{4}{c}{CIFAR-10}  \bigstrut\\
\cline{2-5}        & MQ$_{0.4}$  & MQ$_{0.5}$ & MQ$_{0.6}$$\downarrow$ & FID$\downarrow$   \bigstrut\\
    \hline
    \hline
    Default (DDPM)~\cite{ho2020denoising} & 111 & 465   & 2030  & 8.81 \bigstrut\\
     
    IET-AGC (w/o class label) & \textbf{10} & \textbf{91} & \textbf{769} & 9.12  \bigstrut\\
    IET-AGC (w/ class label)& 14   & 117  & 839 & \textbf{8.34}  \bigstrut\\
    \hline
    \specialrule{\heavyrulewidth}{0pt}{0pt} % 加粗的水平线,位于表格底部
    \end{tabular}%
    
    }
    \label{tab:nonclass}%
    % \vspace{-1.6em}
\end{table}%
\section*{Implementational Details}
\crnote{
When conducting experiments on training Diffusion models from scratch using CIFAR-10 and CIFAR-100, we set the batch size to 128 and train for 400k and 580k iterations, respectively. 
On AFHQ-DOG, the batch size is set to 60, and we train for 180k iterations. 
In the IET framework, CIFAR-10 and CIFAR-100 are divided into ten shards, each containing the same number of classes and instances.
AFHQ-DOG is divided into five shards based on the number of instances, as it lacks class information. 
On the CIFAR-10 and CIFAR-100 datasets, we set the threshold $\lambda$ to $0.5$, indicating that data with loss less than half of the average loss is skipped. 
For the AFHQ-DOG dataset, due to its smaller size and pronounced memory phenomena, we adjust the threshold $\lambda$ to $0.714$.
To demonstrate the effectiveness of our method in text-conditioned Diffusion models, we fine-tune Stable Diffusion on LAION-10k. 
The IET framework divides the LAION-10k dataset into 4 shards, with the threshold $\lambda$ set to 0.8. 
We set the batch size to 8 and fine-tune Stable Diffusion for 200k iterations.
On all datasets, the smoothing factor $\gamma$ for the memory bank is set to 0.8. 
}
\section*{\gxlnote{The IID and non-IID Setting in IET framework}}
In addition, in our approach, the dataset is evenly distributed, meaning each data shard is set in an IID (independently and identically distributed) manner. To validate the reasonableness of this dataset configuration, we also establish an experiment where each data shard is set up in a non-independent and identically distributed (non-IID) manner on CIFAR-10. Similar to Hsu \MakeLowercase{\textit{et al.}}~\cite{hsu2019measuring}, we employ the Dirichlet distribution to generate data, thereby establishing such a setting.
% Table generated by Excel2LaTeX from sheet 'Sheet1'
\begin{table}[t]
  \centering
  \caption{The results under the IID and non-IID setting in the IET framework.}
    \begin{tabular}{c|ccc|c}
    \specialrule{\heavyrulewidth}{0pt}{0pt} % 加粗的水平线,位于表格底部
    \hline
    \multirow{2}[4]{*}{Method} & \multicolumn{4}{c}{CIFAR-10} \bigstrut\\
\cline{2-5}          & MQ$_{0.4}$ & MQ$_{0.5}$ & MQ$_{0.6}$$\downarrow$  & FID↓ \bigstrut\\
    \hline
    \hline
    Default (DDPM)~\cite{ho2020denoising} & 111   & 465   & 2030  & 8.81 \bigstrut[t]\\
    IET$_\mathrm{non-IID}$ & \textbf{69} & \textbf{315} & \textbf{1439} & 12.5 \\
    IET$_\mathrm{IID}$ & 89    & 382   & 1783  & \textbf{7.61} \bigstrut[b]\\
    \hline
    \specialrule{\heavyrulewidth}{0pt}{0pt} % 加粗的水平线,位于表格底部
    \end{tabular}%
  \label{tab:ablation_IID}%
  \vspace{-8pt}
\end{table}%

Results in ~\cref{tab:ablation_IID} show that when the data shards are set in non-IID, Iterative Ensemble Training (IET) successfully reduces the memorization.
However, the non-IID splitting reduces the quality of performance since the divergent optimization of different models and aggregating these models affects the performance. 
Differently, when the data shards are set in IID,  IET$_\mathrm{IID}$ can reduce both the memorization and FID score. 
Thus in this paper, we choose to use the IID splitting strategy. 


\section*{Algorithm}
A pseudo-code implementation is provided in ~\cref{algorithm}.
\vspace{-10pt}
\begin{algorithm}[htp]
\caption{The Proposed Framework}
\SetAlgoLined
\LinesNotNumbered
\KwIn{Dataset $D$, the number of dataset $N$, training rounds $M$, training epochs per interaction period $E$, number of shards $K$, initial model $\theta^0$, skipping threshold $\lambda$, smoothing factor $\eta$, memory bank $l$,  augmentation Range $R$,  skip time $s$ }
\KwOut{Diffusion model $\theta^M$}
Divide dataset $D$ into an equal number of shards $D_i$, $i\in [1,...,K]$\;
Initialize memory bank $l$ with all elements as $0$\;
Initialize skip time $s_i^j$ with all samples as $0$, $j\in [1,...,\frac{N}{K}]$, $i\in [1,...,K]$;

\For{$m=1$ \KwTo $M$}{
    \For{$i=1$ \KwTo $K$}{
        Initialize model $\theta_{i}^m \leftarrow \theta^{m-1}$\;
        \For{$e=1$ \KwTo $E$}{            
            $x \in D_i \quad \text{and} \quad \text{is} \quad  j\text{th} \quad \text{sample}$
            
            $\epsilon  \sim \mathit{N} (0, I) $ , $t$ $\sim$ Uniform $({1, ..., T})$\;       
            $Loss = \mathcal{L} (x, t, \epsilon; \theta_{i}^m)$ \\
% \tcp{Threshold-Aware Control}\\
            \If{$\frac{Loss}{l_t} < \lambda$}
            {
                $Loss \leftarrow 0$;
                $s_i^j=s_i^j +1$
            }
            \If{$\lambda<\frac{Loss}{l_t} < R\lambda$}
            {
                $Loss \leftarrow \mathcal{L}(\mathbf{Aug}(x, e^{-5\parallel \frac{ r - \lambda}{\lambda}\parallel}))$;
            }
            $l_{t} \leftarrow \eta \cdot l_t + (1 - \eta) \cdot \mathcal{L} (x, t, \epsilon; \theta_{i}^m)$ \\
             Then update $\theta_{i}^m$ by $Loss$
             % \leftarrow \theta_{i}^m - \eta \nabla Loss$
        }
    }
    \tcp{Model Aggregation}
    \If{$m \mod 2 == 0$}
    {
       $\theta^m\leftarrow \frac{1}{K}\sum_{i=1}^{K}\theta_{i}^m$
    }
    \tcp{Memory Samples Redistribute}
    \Else
    {
        \For{$i=1$ \KwTo $K$}{
                $Top_P \gets$ $Top_P(s_i^1, s_i^2, \dots, s_i^\frac{N}{K})$\;
                $D_i^{\text{easy}} \gets \emptyset$\;
        
        \For{$j = 1$ \KwTo $\frac{N}{K}$}{
            \If{$s_i^j \geq Top_P$}{
                $D_i^{\text{easy}} \cup \{x_j\} \rightarrow D_i^{\text{easy}}$ \;
            }
            $s_i^j \gets 0$
        }
         $D_{i + 1} \cup D_i^{\text{easy}}  \rightarrow D_{i + 1}$
        }
    }   
}
\label{algorithm}
\end{algorithm}







\end{document}