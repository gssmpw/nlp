\documentclass[lettersize,journal]{IEEEtran}
%
% --- inline annotations
%
\newcommand{\red}[1]{{\color{red}#1}}
\newcommand{\todo}[1]{{\color{red}#1}}
\newcommand{\TODO}[1]{\textbf{\color{red}[TODO: #1]}}
% --- disable by uncommenting  
% \renewcommand{\TODO}[1]{}
% \renewcommand{\todo}[1]{#1}



\newcommand{\VLM}{LVLM\xspace} 
\newcommand{\ours}{PeKit\xspace}
\newcommand{\yollava}{Yo’LLaVA\xspace}

\newcommand{\thisismy}{This-Is-My-Img\xspace}
\newcommand{\myparagraph}[1]{\noindent\textbf{#1}}
\newcommand{\vdoro}[1]{{\color[rgb]{0.4, 0.18, 0.78} {[V] #1}}}
% --- disable by uncommenting  
% \renewcommand{\TODO}[1]{}
% \renewcommand{\todo}[1]{#1}
\usepackage{slashbox}
% Vectors
\newcommand{\bB}{\mathcal{B}}
\newcommand{\bw}{\mathbf{w}}
\newcommand{\bs}{\mathbf{s}}
\newcommand{\bo}{\mathbf{o}}
\newcommand{\bn}{\mathbf{n}}
\newcommand{\bc}{\mathbf{c}}
\newcommand{\bp}{\mathbf{p}}
\newcommand{\bS}{\mathbf{S}}
\newcommand{\bk}{\mathbf{k}}
\newcommand{\bmu}{\boldsymbol{\mu}}
\newcommand{\bx}{\mathbf{x}}
\newcommand{\bg}{\mathbf{g}}
\newcommand{\be}{\mathbf{e}}
\newcommand{\bX}{\mathbf{X}}
\newcommand{\by}{\mathbf{y}}
\newcommand{\bv}{\mathbf{v}}
\newcommand{\bz}{\mathbf{z}}
\newcommand{\bq}{\mathbf{q}}
\newcommand{\bff}{\mathbf{f}}
\newcommand{\bu}{\mathbf{u}}
\newcommand{\bh}{\mathbf{h}}
\newcommand{\bb}{\mathbf{b}}

\newcommand{\rone}{\textcolor{green}{R1}}
\newcommand{\rtwo}{\textcolor{orange}{R2}}
\newcommand{\rthree}{\textcolor{red}{R3}}
\usepackage{amsmath}
%\usepackage{arydshln}
\DeclareMathOperator{\similarity}{sim}
\DeclareMathOperator{\AvgPool}{AvgPool}

\newcommand{\argmax}{\mathop{\mathrm{argmax}}}     

%before hyperref
\usepackage{amsmath,amsfonts}
\usepackage{algorithmic}
\usepackage[ruled,linesnumbered]{algorithm2e}
% \usepackage{algorithm2e}

% \usepackage{algorithm}
\usepackage{array}
\usepackage[caption=false,font=normalsize,labelfont=sf,textfont=sf]{subfig}
\usepackage{textcomp}
\usepackage{stfloats}
\usepackage{url}
\usepackage{verbatim}
\usepackage{graphicx}
\usepackage{cite}
\usepackage{utfsym}

\hyphenation{op-tical net-works semi-conduc-tor IEEE-Xplore}
\def\BibTeX{{\rm B\kern-.05em{\sc i\kern-.025em b}\kern-.08em
    T\kern-.1667em\lower.7ex\hbox{E}\kern-.125emX}}
\usepackage{balance}
%%%%% new package

\usepackage{times}
\usepackage{fancyhdr,graphicx,amsmath,amssymb}

% \usepackage[ruled,vlined]{algorithm2e}

\usepackage{multirow, bigstrut} 
% \usepackage{arydshln}

\usepackage{url}
\usepackage{graphicx}
\usepackage{booktabs} 
% \usepackage[inkscapelatex=false]{svg}
% \svgpath{{images/}}
\usepackage{xcolor}
\usepackage{hyperref}
\usepackage[capitalize]{cleveref}
\definecolor{deepgreen}{rgb}{0.0, 0.5, 0.0}
\newcommand{\pyh}[1]{\textcolor{deepgreen}{#1}}
\newcommand{\wry}[1]{\textcolor{orange}{\@#1}}
\newcommand{\pyhshan}[1]{\textcolor{red}{#1}}
\newcommand{\darkedred}[1]{\textcolor{red!80!black}{#1}}
\newcommand{\gxlnote}[1]{\textcolor{black}{#1}}
\newcommand{\crnote}[1]{\textcolor{black}{#1}}
\newcolumntype{P}[1]{>{\centering\arraybackslash}p{#1}}
% \newcommand{\pyh}[1]{\textcolor{black}{#1}}
% \newcommand{\wry}[1]{\textcolor{black}{\@#1}}
% \newcommand{\pyhshan}[1]{\textcolor{black}{#1}}



\begin{document}
\include{pythonlisting}
% \title{A Sample Article Using IEEEtran.cls\\ for IEEE Journals and Transactions}
% \bibliographystyle{IEEEtran}


\title{Redistribute Ensemble Training for Mitigating Memorization in Diffusion Models}



% \author{IEEE Publication Technology,~\IEEEmembership{Staff,~IEEE,}

\author{Xiaoliu Guan, Yu Wu, Huayang Huang, Xiao Liu, Jiaxu Miao, Yi Yang
% <-this % stops a space
\thanks{X. Guan,  Y. Wu, H. Huang, and  X. Liu are with the School of Computer Science, Wuhan University, China. E-mail: liuxiaoguan, wuyucs, hyhuang, xiaoliu@whu.edu.cn}% <-this % stops a space
\thanks{J. Miao is with the School of Cyber Science and Technology, Sun Yat-sen University, China. E-mail: miaojx@mail.sysu.edu.cn }
\thanks{Y. Yang is with the College of Computer Science and Technology, Zhejiang University, Hangzhou, Zhejiang, China. E-mail: yangyics@zju.edu.cn}
\thanks{(Corresponding author: Yu Wu.)}}

\maketitle


\begin{abstract}
Diffusion models, known for their tremendous ability to generate high-quality samples, have recently raised concerns due to their data memorization behavior, which poses privacy risks. Recent methods for memory mitigation have primarily addressed the issue within the context of the text modality in cross-modal generation tasks, restricting their applicability to specific conditions. In this paper, we propose a novel method for diffusion models from the perspective of visual modality, which is more generic and fundamental for mitigating memorization. 
Directly exposing visual data to the model increases memorization risk, so we design a framework where models learn through proxy model parameters instead. Specially, the training dataset is divided into multiple shards, with each shard training a proxy model, then aggregated to form the final model.
Additionally, practical analysis of training losses illustrates that the losses for easily memorable images tend to be obviously lower. Thus, we skip the samples with abnormally low loss values from the current mini-batch to avoid memorizing.
However, balancing the need to skip memorization-prone samples while maintaining sufficient training data for high-quality image generation presents a key challenge. Thus, we propose IET-AGC+,  which redistributes highly memorizable samples between shards, to mitigate these samples from over-skipping.  
Furthermore, we dynamically augment samples based on their loss values to further reduce memorization.
Extensive experiments and analysis on four datasets show that our method successfully reduces memory capacity while maintaining performance. Moreover, we fine-tune the pre-trained diffusion models, e.g., Stable Diffusion, and decrease the memorization score by 46.7\%, demonstrating the effectiveness of our method. Code is available in \url{https://github.com/liuxiao-guan/IET_AGC}.
\end{abstract}

\begin{IEEEkeywords}
Diffusion Models, Model Memorization, Data Privacy.
\end{IEEEkeywords}

\section{Introduction}
\IEEEPARstart{R}{ecent} advancements in diffusion models have significantly transformed the landscape of image generation~\cite{croitoru2023diffusion,zhan2023multimodal,zhu2024vision+}. Modern diffusion models, such as Stable Diffusion~\cite{rombach2022high}, Midjourney~\cite{midjourney2022}, and SORA~\cite{sora2024}, can generate realistic images that are hard for humans to distinguish, demonstrating the unparalleled capabilities in producing diverse images. However, recent works~\cite{carlini2023extracting,somepalli2024understanding,wen2023detecting} suggested that diffusion models can memorize images from the training set and reproduce them directly. \gxlnote{This raises privacy concerns, as sensitive information, such as identifiable faces or private documents, may be generated and inadvertently exposed.} To address the critical issue, some works~\cite{zhang2023forget,ni2023degeneration,gandikota2024unified,kumari2023ablating} proposed to make diffusion models ``forget'' specific concepts such as a portrait of a certain celebrity, or the style of a particular artist. However, these works can only blacklist specific content that users want to conceal, but cannot completely cover the privacy-sensitive information that the model might remember, still posing a risk of privacy leakage.

\begin{figure}[tb]
  \centering
  \setlength{\abovecaptionskip}{7pt} % 设置标题上方的间距为 -5pt
  \setlength{\belowcaptionskip}{-7pt} % 设置标题下方的间距为 -5pt
  \includegraphics[width=1.0\linewidth]{imgs/problem.pdf}
  \caption{\gxlnote{Prior methods focus solely on the captions associated with the memorized images, such as caption augmentation. In contrast, our approach takes a more generalizable framework by considering aspects from the visual modality.}}
  \label{fig:problem}
\end{figure}
 
Recently, some works~\cite{somepalli2023diffusion,daras2024ambient,somepalli2024understanding,wen2023detecting} have proposed to mitigate diffusion memorization without specific content limitations, thus reducing the risk of diffusion models leaking privacy-sensitive training data. Most of them focused on tackling the training data memorization in text-to-image diffusion models, and proposed data augmentation for captions/sentences to reduce model memorization.  For instance, Somepalli \MakeLowercase{\textit{et al.}}~\cite{somepalli2024understanding} found that the insufficient diversity in captions easily leads to training data generation and thus utilized random caption replacement, random token replacement, and caption word repetition, \MakeLowercase{\textit{etc.}}, to reduce memorization. 
Based on the discovery that memorized prompts tend to exhibit larger magnitudes, which refers to the difference between the text-conditioned and unconditioned noise prediction, Wen \MakeLowercase{\textit{et al.}}~\cite{wen2023detecting} introduced methods for mitigating memorization through filtering high-magnitude sample during training and minimizing magnitudes during inference. Although these works have made significant progress in understanding the memorization issue in diffusion models, \gxlnote{they only focused on easily memorable images related to specific captions in cross-modal generation tasks as shown in ~\cref{fig:problem}. 
However, they do not directly tackle the memorization problem in image generation. While manipulating captions may reduce the likelihood of memorization being triggered in text-to-image models, the model’s inherent ability to memorize images remains. Memorization can still occur under different conditions~\cite{carlini2023extracting,somepalli2024understanding}.
Therefore, we propose a novel framework for diffusion models from the perspective of the visual modality, which not only mitigates memorization more fundamentally but also provides a more generic approach.}



Following these insights,  in our preliminary ECCV 2024 version~\cite{liu2024iterative}, we propose the first module: \textbf{Iterative Ensemble Training (IET)} framework from the perspective of parameter aggregation as shown in ~\cref{fig:problem}. Transmitting data directly to the model increases the likelihood of memorizing easy samples. \gxlnote{However, if the model learns from parameters of other models, rather than directly from the data, it may help to mitigate the direct memorization}. Specifically, we divide the data into multiple data shards and train several proxy models. These models are then aggregated to form the final model. Inspired by federated learning~\cite{mcmahan2017communication}, we iteratively ensemble the proxy models during training, which helps reduce memorization through multiple aggregations and preserves the generation performance. 
Besides, we suspect that images with varying degrees of memorization might exhibit different behaviors during the training process. Therefore, we analyze the training process and find that the loss of easily memorable images tends to be obviously lower than that of less memorable images. Based on this analysis, we propose the second module: \textbf{Anti-Gradient Control (AGC)} to further reduce memorization of training data. In particular, we skip the samples with abnormally small loss values from the current mini-batch to avoid memorizing these samples. During training, as the diffusion model exhibits varying average loss values across different time steps, we maintain a memory bank to track the average loss at each step. Building on this, we skip samples whose loss ratio—defined as the ratio of the sample's loss to the average loss—falls below a predefined skipping threshold as shown in ~\cref{fig:AGC+_s}.



\gxlnote{
However, the AGC strategy might excessively skip highly memorizable samples, leading to a reduction in available training data and potential degradation of image quality.
This drives us to pursue a better approach that strikes a balance between mitigating memorization and maintaining image quality.
Following these insights, in this paper, we introduce IET-AGC+ building on our ECCV2024 framework~\cite{liu2024iterative}.}
\gxlnote{To address the issue of excessively skipping, we propose a \textbf{Memory Samples Redistribute (MSR)} strategy to ensure that these samples are learned but not easily memorized. In the IET framework, each proxy model learns from its shard, where the same data may be interpreted differently.
\emph{In particular, when a sample is frequently memorized in its original shard, it may not have the same memorization tendency in a new shard}. As the saying goes: One man's meat is another man's poison. This inspires us to exchange easily memorized samples from one shard with another to prevent them from being skipped too frequently as shown in ~\cref{fig:problem}.  Therefore, in the training process, we track the number of times each sample is skipped to identify whether it is most easily memorized. During the interaction, each shard allocates its most frequently skipped samples to the next shard in a circular manner.}


\begin{figure}[tb]
  \centering
  \setlength{\abovecaptionskip}{7pt} % 设置标题上方的间距为 -5pt
  \setlength{\belowcaptionskip}{-10pt} % 设置标题下方的间距为 -5pt
  \includegraphics[width=1.0\linewidth]{imgs/simple_AGC+.pdf}
  \caption{\gxlnote{Threshold-Aware Augmentation (TAA) collaborated with Anti-Gradient Control. We apply three different treatments based on the comparison between the sample's loss ratio and the skipping threshold. }}
  \label{fig:AGC+_s}
\end{figure}
% \vspace{-0.5em}
\gxlnote{
On the other hand, in AGC, images below the threshold are more likely to be memorized, making their exclusion a reasonable choice. However, memorization varies in degree and cannot be simply addressed with a hard threshold. Samples should be dynamically processed based on their level of memorization risk. To address this, we propose a new strategy called \textbf{Threshold-Aware Augmentation (TAA)} collaborated with Anti-Gradient Control as shown in ~\cref{fig:AGC+_s}. For samples that are not skipped but whose loss values are close to the threshold, we apply augmentation to increase their diversity, thereby reducing memorization. 
A lower loss value indicates a higher risk of memorization, so we use dynamic visual augmentation based on sample distance from the threshold. Samples closer to the threshold receive stronger augmentation.}

Extensive experiments on four datasets highlight the importance of our framework. 
Our method significantly reduces the memorized quantity by \gxlnote{90.1\%, 74.6\%, and 91.2\% }compared with the default training (DDPM~\cite{ho2020denoising}) on CIFAR-10~\cite{krizhevsky2009learning} and CIFAR-100~\cite{krizhevsky2009learning} and AFHQ-DOG~\cite{choi2020stargan}, respectively. Furthermore, when fine-tuning the text-conditional diffusion model, Stable Diffusion~\cite{rombach2022high}, our approach decreases the memorization score by \gxlnote{46.7\%} compared to conventional fine-tuning method~\cite{rombach2022high}. In addition, our method can also be applied to existing inference phase mitigation mechanisms~\cite{somepalli2024understanding,wen2023detecting}, further reducing memorization and improving image quality. These results demonstrate the effectiveness of our method.

\gxlnote{Our main contributions are summarized as follows:
\begin{itemize}
\item {
We introduce a generalized method to mitigate memorization from the perspective of the visual modality, which consists of two main parts: leveraging multiple model ensembles for training and skipping easily memorized samples based on the training loss.
}
\item{
We propose Memory Samples Redistribute (MSR), which redistributes easily memorized samples across shards in the above framework while maintaining a balance between memorization reduction and image quality.
}
\item{
We suggest Threshold-Aware Augmentation (TAA), a strategy that adapts the level of augmentation based on the distance between the sample's loss and the skipping threshold, effectively addressing the risk of overlooking memorized samples.
}
\end{itemize}}





\section{Related Work}

\subsection{Memorization in Generative Models}
Several studies have examined the memorization capabilities of the generative model~\cite{wang2024replication,sun2024create}. 
Generative Adversarial Networks (GANs)~\cite{goodfellow2020generative} have been at the forefront of this research area. 
As Webster \MakeLowercase{\textit{et al.}}~\cite{webster2021person} demonstrated when applied to face datasets, GANs can occasionally replicate.
Prior study~\cite{carlini2021extracting} explored an adversarial attack on language models like GPT-2~\cite{radford2019language}, where individual training examples can be recovered, including personally identifiable information and unique text sequences.

Recent studies have shifted their attention toward diffusion models. 
Somepalli \MakeLowercase{\textit{et al.}}~\cite{somepalli2023diffusion} found that diffusion models accurately recall and replicate training images, especially noted with models like the Stable Diffusion model~\cite{rombach2022high}.
Building upon this discovery, Carlini \MakeLowercase{\textit{et al.}}~\cite{carlini2023extracting} developed a tailored black-box attack for diffusion models. They generated images and implemented a membership inference attack to assess density.
Webster \MakeLowercase{\textit{et al.}}~\cite{webster2023reproducible} demonstrated a more efficient extraction attack with fewer network evaluations, identified "template verbatims," and discussed its persistence in newer systems. 
Recent research has shifted towards exploring the theoretical aspects of memory in diffusion models.
Yoon \MakeLowercase{\textit{et al.}}~\cite{yoon2023diffusion} discovered that generalization and memorization are mutually exclusive occurrences and further demonstrated that the dichotomy between memorization and generalization can be apparent at the class level.
Gu \MakeLowercase{\textit{et al.}}~\cite{gu2023memorization} extensively studied how factors like data dimension, model size, time embedding, and class conditions affect the memory capacity of the diffusion model.

\subsection{Memorization Mitigation} 
The mitigation measures have primarily been concerned with filtering inputs and deduplication. 
For example, Stable Diffusion employed well-trained detectors to identify unsuitable generated content. 
However, these temporary solutions can be easily bypassed~\cite{wen2024hard,rando2022red} and do not effectively prevent or lessen copying behavior on a broad scale. 
Kumari \MakeLowercase{\textit{et al.}}~\cite{kumari2023ablating} designed an algorithm to align the image distribution with a specific style, instance, or text prompt they aim to remove, to the distribution related to a core concept. 
This stopped the model from producing target concepts based on its text condition.
\gxlnote{Hintersdorf \MakeLowercase{\textit{et al.}}~\cite{hintersdorf2024finding} localized memorization of individual data samples down to the level of neurons in DMs’ cross-attention layers.}
However, these approaches are inefficient because they necessitate a list of all concepts to be erased, and have not addressed the key issue of how to reduce the memory capacity of the model.
~\cite{dockhorn2022differentially,ghalebikesabi2023differentially} explored the use of differential privacy (DP)~\cite{dwork2006differential} to train diffusion models or fine-tune ImageNet pre-trained models. However, their focus was on ensuring the privacy of the training of diffusion models, not on the privacy of the images generated by the diffusion models. 
\gxlnote{Chen \MakeLowercase{\textit{et al.}}~\cite{chen2024towards} re-guides generation by measuring the similarity between generated and training images, aiming for memorization-free outputs. However, directly relying on the training set during testing is impractical.}
Daras \MakeLowercase{\textit{et al.}}~\cite{daras2024ambient} introduced a technique for training diffusion models utilizing tainted data. By incorporating additional corruption before applying noise, their methodology prevents the model from overfitting to the training data. But their training requires a considerable amount of time. 
~\cite{somepalli2024understanding,wen2023detecting, ren2024unveiling} also suggested a series of recommendations to mitigate copying such as randomly replacing the caption of an image with a random sequence of words, but most of which are limited to text-to-image models. Our work focuses on the nature of memorization in diffusion models, especially for unconditional ones. 
\vspace{-0.1cm}
\gxlnote{\subsection{Data Augmentation Theory and Practice}
Data augmentation is a widely used technique to improve the generalization of machine learning models, particularly in deep learning ~\cite{wang2021regularizing}. It is commonly employed to increase the diversity of training data by applying transformations in image-based tasks. Common data augmentation techniques include pixel erasing ~\cite{zhong2020random,devries2017improved,chen2020gridmask}, image cropping~\cite{chen2016automatic,ciocca2007self}, mixing images~\cite{hendrycks2019augmix,zhang2017mixup}, geometric transformations~\cite{wang2019perspective,jaderberg2015spatial}, kernel filter~\cite{kang2017patchshuffle}, \MakeLowercase{\textit{etc}}.
The use of data augmentation has been widely explored for vision tasks that require extensive annotation. Azizi \MakeLowercase{\textit{et al.}}~\cite{azizi2023synthetic}showed that augmenting the ImageNet training set~\cite{russakovsky2015imagenet} with samples generated by conditional diffusion models results in a significant boost in classification accuracy. Baranchuk \MakeLowercase{\textit{et al.}}~\cite{baranchuk2021label} investigated how diffusion models can be used to augment data for semantic segmentation, leveraging intermediate activations as rich pixel-level representations, especially when labeled data is scarce. Trabucco \MakeLowercase{\textit{et al.}}~\cite{trabucco2023effective} explored methods to augment individual images with a pre-trained diffusion model, showing significant improvements in few-shot scenarios. Other examples include tasks like human motion understanding~\cite{guo2022learning, izadi2011kinectfusion}, optical flow estimation~\cite{dosovitskiy2015flownet, sun2021autoflow}, and physically realistic simulation environments~\cite{de2022next,dosovitskiy2017carla,gan2020threedworld}, etc. Our study uses data augmentation to flexibly enhance model generalization, thereby mitigating memorization.}

\begin{figure}[t]
  \centering
  \setlength{\abovecaptionskip}{7pt} % 设置标题上方的间距为 -5pt
  \setlength{\belowcaptionskip}{-7pt} % 设置标题下方的间距为 -5pt
  \includegraphics[height=6.5cm]{imgs/line_plot.pdf}
  \caption{
  Comparison of the training losses between memorized and non-memorized images. 
  }
  \label{fig:LossAnalysis}
\end{figure}
\section{\gxlnote{Exploring Training Loss and Memorization in Diffusion Models}}
To reduce memorization of training data, we delve into the causes of memorization phenomena, specifically analyzing it through the lens of the training loss, \gxlnote{because we suspect that images with varying degrees of memorization might exhibit different behaviors during the training process.}
We begin by establishing the fundamental notation linked with diffusion models.
Diffusion models \cite{ho2020denoising} originate from the non-equilibrium statistical physics \cite{sohl2015deep}.
They are essentially straightforward: they operate as image denoisers.
During the training process, when given a clean image $x$, time-step $t$ is sampled from the interval [$0$, $T$], along with a Gaussian noise vector $\epsilon \sim \mathit{N} (0, I)$,
resulting in a noised image $x_t$:
\begin{equation}\label{eq:noised_data}
    x_t = \sqrt{\alpha_t}x + \sqrt{1-\alpha_t} \epsilon, 
\end{equation}
where the scheduled variance $\alpha_t$ varies between $0$ and $1$, with $\alpha_0 = 1$ and $\alpha_T = 0$. 
The diffusion model then removes the noise to reconstruct the original image $x$ by predicting the noise introduced, achieved through stochastic minimization of the objective function
$\frac{1}{N} \sum_{i} \mathbb{E}_{t,\epsilon} \mathcal{L} (x_i, t, \epsilon; \theta)$, where
\begin{equation}\label{eq:loss}
  \mathcal{L} (x_i, t, \epsilon; \theta) = \| \epsilon - \epsilon_\theta(\sqrt{\alpha_t}x_i + \sqrt{1-\alpha_t}\epsilon, t) \|^2.
\end{equation}

\begin{figure*}[t]
  \centering
  \setlength{\abovecaptionskip}{7pt} % 设置标题上方的间距为 -5pt
  \setlength{\belowcaptionskip}{-7pt} % 设置标题下方的间距为 -5pt
  \includegraphics[width=1.0\linewidth]{imgs/method.pdf}
  \caption{Framework overview of our method. During the training stage, we train multiple proxy models on several data shards. Besides, we selectively skip samples based on their training loss and track how often each sample is skipped in each shard. During the interaction stage, there are two main parts: first, the proxy models are aggregated into a new model, and its weights are distributed as initial weights for the next training phase; second, each shard redistributes its top $P$ skipped samples to the next shard, assigning the last shard to the first. In the next training stage, each shard resumes training with the updated data and model.}
  \label{fig:method}
\end{figure*}
To analyze the correlation between losses and image memorization, \gxlnote{We identify memorized images on CIFAR-10 by generating 65,536 images using a pre-trained model (DDPM)~\cite{ho2020denoising}  and selecting the top 256 training images with the highest similarity to their nearest generated neighbors.} Then we calculate their loss functions at each time step. 
Similarly, we sample 256 non-memorized images from the remaining training data and compute their losses at each time step.
\cref{fig:LossAnalysis} shows the comparisons of the losses.
% when the time step is in the interval [0, 600] $(T=1000)$. 
Memorized images exhibit significantly smaller loss values during this period, indicating that the model tends to reconstruct noise into such images.


\section{Method}
In this section, we present our methodology for mitigating the memorization in diffusion models, without sacrificing excessive image quality.
\subsection{Framework Overview}
\gxlnote{
As shown in ~\cref{fig:method}, our method trains the model by the following two steps iteratively: 
1) training proxy models on each data shard, and 2) conducting two rounds of interaction: proxy model aggregation and shard data redistribution.
Specifically, during the training stage, we divide the dataset into multiple data shards ($D_1, D_2,..., D_K$) and train corresponding proxy diffusion models ($\theta_1, \theta_2,...,\theta_K$). Additionally, we selectively skip certain samples based on their training loss and keep track of the number of times each sample is skipped in each shard. During the interaction stage, the proxy diffusion models  ($\theta_1, \theta_2,...,\theta_K$) from different shards are aggregated into a new model $\hat{\theta}$ through averaging, which serves as the initial model for the next training phase. Meanwhile, each shard identifies and redistributes its top $P$ most easily skipped sample sets to the next shard, updating the data of each shard accordingly. During the next training, each shard resumes training with the updated data shard ($D_1', D_2',..., D_K'$) and model $\hat{\theta}$.
}
\subsection{\gxlnote{Threshold-Aware Control}}
\gxlnote{We first introduce the model updating step.}
In this subsection, we elaborate on how to utilize the aforementioned loss analysis to devise a training strategy to alleviate the occurrence of memorization.
\subsubsection{Anti-Gradient Control}
\textbf{Memory Bank:}
To identify images with exceptionally low loss values that are prone to memorization during training, we need to maintain the average losses for each time step. 
However, computing the average loss at each time step entails substantial computational expenses, as it necessitates evaluating the losses for all images using the model at each time step. Thus, we propose a memory bank to store and update losses during mini-batch training without increasing the time cost. However, 
the losses generally decrease with the training step growing. To address this, when calculating the average loss in the memory bank, we adjust the aggregation process by assigning higher weights to losses that are closer to the current update, rather than simply averaging all losses at the current time step.
Specifically, we initialize an array of length $T$ with zeros, termed the memory bank. 
After calculating the loss for a mini-batch, we update the memory bank using the Exponential Moving Average (EMA) ~\cite{polyak1992acceleration} method based on the loss and the sampled time step, thereby better reflecting the current state of the model:
\begin{equation}\label{eq:ema}
  l_{t} \leftarrow  \eta \cdot l_{t} + (1 - \eta) \cdot \mathcal{L} (x, t, \epsilon; \theta),
\end{equation}
where $\eta$ represents the smoothing factor, and $l_{t}$ represents the averaged loss in the memory bank at time step $t$. 

\textbf{ Loss Ratio-Based Selection:}
In previous observations, if the model exhibits memorization of a certain sample, the loss value of the model on that sample tends to be abnormally small.
Thus, we use the ratio of the training loss of a certain sample to the mean loss in the memory bank at the time step $t$ as a measure to mitigate memorization:
\begin{equation}\label{eq:ratio}
  r(x) = \frac{\mathcal{L} (x, t, \epsilon; \theta)}{l_t}.
\end{equation}
A smaller value of $r(x)$ may indicate a higher likelihood of the image being memorized. 
Then we establish a configurable threshold denoted as $\lambda$. 
If the loss ratio $r(x)$ falls below this threshold $\lambda$, we will skip the image in the mini-batch.
\begin{figure}[tb]
% \vspace{-1.5cm}
  \centering
  \setlength{\abovecaptionskip}{7pt} % 设置标题上方的间距为 -5pt
  \setlength{\belowcaptionskip}{-7pt} % 设置标题下方的间距为 -5pt
  \includegraphics[width=1.0\linewidth]{imgs/AGC+.pdf}
  \caption{\gxlnote{The proposed model update procedure (AGC with TAA). During training, we dynamically update and maintain a memory bank of losses at each timestep. For each sample's loss ratio $\frac{Loss}{l_t}$, we compare it with 
$\lambda$ and $R\lambda$  to update the loss, considering three cases: for losses less than $\lambda$, we skip the sample and update its skip times; for losses between $\lambda$ and $R\lambda$, we augment the sample and retrain to obtain a new loss; for losses greater than $R\lambda$, we keep the loss unchanged.}}
  \label{fig:AGC+}
\end{figure}


\gxlnote{\subsubsection{Threshold-Aware Augmentation}In AGC, images below the threshold are more likely to be memorized, making their exclusion a reasonable choice. However, memorization varies in degree and samples should be dynamically processed based on their level of memorization risk. Therefore, we design this strategy, dynamically enhancing samples to increase their diversity and thus mitigate memorization.}

\gxlnote{Specifically, for samples not skipped,  if their ratio $r$  does not exceed a specific value, that is, $R$ times the threshold, we apply augmentation to them as follows:
\begin{equation}
\mathcal{L}(\mathbf{Aug}(x, \rho(x), t, \epsilon; \theta) \quad \operatorname{if} \quad \lambda < r(x) <  R\lambda, 
\end{equation}
where $R$ is a multiplier with $R>1$, and $\rho(x)$ represents the relative augmentation strength. 
 For the augmentation function, we choose RandAugment\cite{cubuk2020randaugment} which introduces a vastly simplified search space for data augmentation. 
At the same time, we believe that the lower the sample’s loss value is, the higher its risk of memorization is. Therefore, we apply varying levels of augmentation based on its distance from the threshold—the closer it is, the stronger the augmentation. First, we calculate the relative distance between the loss ratio and the skip threshold:
\begin{equation}
    d(x) = \parallel \frac{ r(x) - \lambda}{\lambda} \parallel.
\end{equation}
Then we choose $e^{-Ax}$ as our negatively correlated function between the distance and the augmentation strength:
\begin{equation}
    \rho(x) =e^{-Ad(x)},
\end{equation}
where $A$ is set  as a constant value of 5.}
\gxlnote{\subsubsection{Threshold-Aware Control}With threshold-aware augmentation, the overall model updating is the following function:
\begin{equation}\label{eq:update_loss_plus}
    \mathcal{L}(x) = 
    \begin{cases} 
    0 & \operatorname{if } r(x) < \lambda \\
    % \mathcal{L}(\mathbf{Aug}(x, e^{-5\parallel \frac{ r - \lambda}{\lambda} \parallel})) & \text{if } \lambda < r(x) < R\lambda \\
    \mathcal{L}(\mathbf{Aug}(x, \rho(x)) & \operatorname{if } \lambda < r(x) < R\lambda \\
    \mathcal{L}(x) & \operatorname{otherwise},
    \end{cases}
\end{equation}
where we re-purpose it by expressing as $\mathcal{L}(x) \propto \mathcal{L}(x, t, \epsilon; \theta)$, omitting $t, \epsilon, \theta$ for simplicity. The overall process is in ~\cref{fig:AGC+}.
}

\subsection{Iterative Ensemble Training}
% \subsubsection{Iterative Ensemble Training}
\gxlnote{In traditional training approaches, directly transmitting the entire training data to the model increases the likelihood of easy samples being memorized. However, if the model learns from parameters of other models, rather than directly from the data, it may help to mitigate memorization. Thus,  we propose a framework that trains multiple proxy diffusion models on different data shards of a dataset. }


\textbf{Training on Different Data Shards.} Unlike the training methods of previous diffusion models, which train a single model on the entire dataset once, in this paper, we divide the dataset into multiple data shards and then train the corresponding proxy diffusion models on each separate part. 
\gxlnote{Specially, we suppose the dataset $D$ contains $N$ samples and $C$ classes.  We divide the dataset into $K$ parts in the IID (Independently and Identically Distributed) setting in which each data shard is randomly assigned a uniform distribution over $C$ classes. 
If the dataset does not contain class information, we divide the dataset into $K$ equal parts. In summary, each data shard contains $\frac{N}{K}$ samples.
}
Then, each shard $i$ trains a separate proxy diffusion model $\theta_{i}$ on its own dataset.

\textbf{Aggregating the Multiple Diffusion Models.} After a period of training, each shard develops a distinct proxy diffusion model. We simply average the weights of all proxy models $\theta_{i}$ to obtain a final model $\hat{\theta}$ as
\begin{equation}\label{eq:average}
    \frac{1}{K}\sum_{i = 1}^{K}\theta_{i} \rightarrow \hat{\theta}.
\end{equation}

Then, we repeat the two stages of training on separate shards of the data and aggregate proxy models, using the obtained final model as the initial model for the first stage.

\textbf{Training Time Analysis.}
As each shard contains only $\frac{1}{K}$ of the total data, the training time for each proxy model is proportionally reduced,
maintaining the overall computational cost \emph{nearly constant} compared to training a single model on the entire dataset. 
The only additional computational cost arises from periodically merging the proxy models, which is minimal and has little impact on overall training efficiency.

\vspace{0.5cm}
\subsection{Memory Samples Redistribute}
\gxlnote{Although AGC effectively mitigates memorization by skipping easily memorized samples, this exclusion may result in reducing the available training data, potentially leading to a decrease in image quality.
To address this issue, we integrated Memory Samples Redistribute (MSR) to ensure that these
samples are learned but not easily memorized.  In the IET framework, each proxy model learns from its shard, where the same data may be interpreted differently. A sample frequently memorized in its original shard may not have the same memorization tendency in a new shard. 
Thus, we allow each shard to redistribute samples that are most easily memorized to the next shard during training, which in practice corresponds to the samples that are most frequently skipped.
}

\gxlnote{Specifically, during the training process, we keep track of the number of times each sample is skipped. We define \( s_i^j \) as skip count for the $j$th sample in the \( i \)th shard's dataset and $s_i = \{s_i^1, s_i^2,..., s_i^\frac{N}{K}\}$ represents the set of skip counts for the $i$th shard.  Then each shard identifies the top $P$ of samples that are most likely to be skipped $s_i^{top}=\{\tilde{s}_i^1, \tilde{s}_i^2,..., \tilde{s}_i^{P*\frac{N}{K}} \}$, where $P$ represents the redistributed proportion of the total samples.  The dataset of most easily memorized samples is defined as:
\begin{equation}
    D_i^{\operatorname{easy}} = \{x^j | s_i^j \in s_i^{top} \}.
\end{equation}
Next, each shard distributes these samples to the next shard in a circular manner, as shown in the following function: 
\begin{equation}
    D_{i + 1} \cup D_i^{\operatorname{easy}}  \rightarrow D_{i + 1}^{\prime}, 
\end{equation}
where $i = 1, 2, \ldots, K.$ }
\gxlnote{
As is shown in ~\cref{fig:method},  the top $P$ most skipped samples from $D_1$ are redistributed to $D_2$, the samples from $D_2$ are assigned to $D_3$, and so on, with the samples from $D_K$ being assigned to $D_1$. In the next training phase, each shard’s dataset is updated accordingly.
}



\section{Experiments}
\label{sec:exp}
Following the settings in Section \ref{sec:existing}, we evaluate \textit{NovelSum}'s correlation with the fine-tuned model performance across 53 IT datasets and compare it with previous diversity metrics. Additionally, we conduct a correlation analysis using Qwen-2.5-7B \cite{yang2024qwen2} as the backbone model, alongside previous LLaMA-3-8B experiments, to further demonstrate the metric's effectiveness across different scenarios. Qwen is used for both instruction tuning and deriving semantic embeddings. Due to resource constraints, we run each strategy on Qwen for two rounds, resulting in 25 datasets. 

\subsection{Main Results}

\begin{table*}[!t]
    \centering
    \resizebox{\linewidth}{!}{
    \begin{tabular}{lcccccccccc}
    \toprule
    \multirow{3}*{\textbf{Diversity Metrics}} & \multicolumn{10}{c}{\textbf{Data Selection Strategies}} \\
    \cmidrule(lr){2-11}
    & \multirow{2}*{\textbf{K-means}} & \multirow{2}*{\vtop{\hbox{\textbf{K-Center}}\vspace{1mm}\hbox{\textbf{-Greedy}}}}  & \multirow{2}*{\textbf{QDIT}} & \multirow{2}*{\vtop{\hbox{\textbf{Repr}}\vspace{1mm}\hbox{\textbf{Filter}}}} & \multicolumn{5}{c}{\textbf{Random}} & \multirow{2}{*}{\textbf{Duplicate}} \\ 
    \cmidrule(lr){6-10}
    & & & & & \textbf{$\mathcal{X}^{all}$} & ShareGPT & WizardLM & Alpaca & Dolly &  \\
    \midrule
    \rowcolor{gray!15} \multicolumn{11}{c}{\textit{LLaMA-3-8B}} \\
    Facility Loc. $_{\times10^5}$ & \cellcolor{BLUE!40} 2.99 & \cellcolor{ORANGE!10} 2.73 & \cellcolor{BLUE!40} 2.99 & \cellcolor{BLUE!20} 2.86 & \cellcolor{BLUE!40} 2.99 & \cellcolor{BLUE!0} 2.83 & \cellcolor{BLUE!30} 2.88 & \cellcolor{BLUE!0} 2.83 & \cellcolor{ORANGE!20} 2.59 & \cellcolor{ORANGE!30} 2.52 \\    
    DistSum$_{cosine}$  & \cellcolor{BLUE!30} 0.648 & \cellcolor{BLUE!60} 0.746 & \cellcolor{BLUE!0} 0.629 & \cellcolor{BLUE!50} 0.703 & \cellcolor{BLUE!10} 0.634 & \cellcolor{BLUE!40} 0.656 & \cellcolor{ORANGE!30} 0.578 & \cellcolor{ORANGE!10} 0.605 & \cellcolor{ORANGE!20} 0.603 & \cellcolor{BLUE!10} 0.634 \\
    Vendi Score $_{\times10^7}$ & \cellcolor{BLUE!30} 1.70 & \cellcolor{BLUE!60} 2.53 & \cellcolor{BLUE!10} 1.59 & \cellcolor{BLUE!50} 2.23 & \cellcolor{BLUE!20} 1.61 & \cellcolor{BLUE!30} 1.70 & \cellcolor{ORANGE!10} 1.44 & \cellcolor{ORANGE!20} 1.32 & \cellcolor{ORANGE!10} 1.44 & \cellcolor{ORANGE!30} 0.05 \\
    \textbf{NovelSum (Ours)} & \cellcolor{BLUE!60} 0.693 & \cellcolor{BLUE!50} 0.687 & \cellcolor{BLUE!30} 0.673 & \cellcolor{BLUE!20} 0.671 & \cellcolor{BLUE!40} 0.675 & \cellcolor{BLUE!10} 0.628 & \cellcolor{BLUE!0} 0.591 & \cellcolor{ORANGE!10} 0.572 & \cellcolor{ORANGE!20} 0.50 & \cellcolor{ORANGE!30} 0.461 \\
    \midrule    
    \textbf{Model Performance} & \cellcolor{BLUE!60}1.32 & \cellcolor{BLUE!50}1.31 & \cellcolor{BLUE!40}1.25 & \cellcolor{BLUE!30}1.05 & \cellcolor{BLUE!20}1.20 & \cellcolor{BLUE!10}0.83 & \cellcolor{BLUE!0}0.72 & \cellcolor{ORANGE!10}0.07 & \cellcolor{ORANGE!20}-0.14 & \cellcolor{ORANGE!30}-1.35 \\
    \midrule
    \midrule
    \rowcolor{gray!15} \multicolumn{11}{c}{\textit{Qwen-2.5-7B}} \\
    Facility Loc. $_{\times10^5}$ & \cellcolor{BLUE!40} 3.54 & \cellcolor{ORANGE!30} 3.42 & \cellcolor{BLUE!40} 3.54 & \cellcolor{ORANGE!20} 3.46 & \cellcolor{BLUE!40} 3.54 & \cellcolor{BLUE!30} 3.51 & \cellcolor{BLUE!10} 3.50 & \cellcolor{BLUE!10} 3.50 & \cellcolor{ORANGE!20} 3.46 & \cellcolor{BLUE!0} 3.48 \\ 
    DistSum$_{cosine}$ & \cellcolor{BLUE!30} 0.260 & \cellcolor{BLUE!60} 0.440 & \cellcolor{BLUE!0} 0.223 & \cellcolor{BLUE!50} 0.421 & \cellcolor{BLUE!10} 0.230 & \cellcolor{BLUE!40} 0.285 & \cellcolor{ORANGE!20} 0.211 & \cellcolor{ORANGE!30} 0.189 & \cellcolor{ORANGE!10} 0.221 & \cellcolor{BLUE!20} 0.243 \\
    Vendi Score $_{\times10^6}$ & \cellcolor{ORANGE!10} 1.60 & \cellcolor{BLUE!40} 3.09 & \cellcolor{BLUE!10} 2.60 & \cellcolor{BLUE!60} 7.15 & \cellcolor{ORANGE!20} 1.41 & \cellcolor{BLUE!50} 3.36 & \cellcolor{BLUE!20} 2.65 & \cellcolor{BLUE!0} 1.89 & \cellcolor{BLUE!30} 3.04 & \cellcolor{ORANGE!30} 0.20 \\
    \textbf{NovelSum (Ours)}  & \cellcolor{BLUE!40} 0.440 & \cellcolor{BLUE!60} 0.505 & \cellcolor{BLUE!20} 0.403 & \cellcolor{BLUE!50} 0.495 & \cellcolor{BLUE!30} 0.408 & \cellcolor{BLUE!10} 0.392 & \cellcolor{BLUE!0} 0.349 & \cellcolor{ORANGE!10} 0.336 & \cellcolor{ORANGE!20} 0.320 & \cellcolor{ORANGE!30} 0.309 \\
    \midrule
    \textbf{Model Performance} & \cellcolor{BLUE!30} 1.06 & \cellcolor{BLUE!60} 1.45 & \cellcolor{BLUE!40} 1.23 & \cellcolor{BLUE!50} 1.35 & \cellcolor{BLUE!20} 0.87 & \cellcolor{BLUE!10} 0.07 & \cellcolor{BLUE!0} -0.08 & \cellcolor{ORANGE!10} -0.38 & \cellcolor{ORANGE!30} -0.49 & \cellcolor{ORANGE!20} -0.43 \\
    \bottomrule
    \end{tabular}
    }
    \caption{Measuring the diversity of datasets selected by different strategies using \textit{NovelSum} and baseline metrics. Fine-tuned model performances (Eq. \ref{eq:perf}), based on MT-bench and AlpacaEval, are also included for cross reference. Darker \colorbox{BLUE!60}{blue} shades indicate higher values for each metric, while darker \colorbox{ORANGE!30}{orange} shades indicate lower values. While data selection strategies vary in performance on LLaMA-3-8B and Qwen-2.5-7B, \textit{NovelSum} consistently shows a stronger correlation with model performance than other metrics. More results are provided in Appendix \ref{app:results}.}
    \label{tbl:main}
    \vspace{-4mm}
\end{table*}


\begin{table}[t!]
\centering
\resizebox{\linewidth}{!}{
\begin{tabular}{lcccc}
\toprule
\multirow{2}*{\textbf{Diversity Metrics}} & \multicolumn{3}{c}{\textbf{LLaMA}} & \textbf{Qwen}\\
\cmidrule(lr){2-4} \cmidrule(lr){5-5} 
& \textbf{Pearson} & \textbf{Spearman} & \textbf{Avg.} & \textbf{Avg.} \\
\midrule
TTR & -0.38 & -0.16 & -0.27 & -0.30 \\
vocd-D & -0.43 & -0.17 & -0.30 & -0.31 \\
\midrule
Facility Loc. & 0.86 & 0.69 & 0.77 & 0.08 \\
Entropy & 0.93 & 0.80 & 0.86 & 0.63 \\
\midrule
LDD & 0.61 & 0.75 & 0.68 & 0.60 \\
KNN Distance & 0.59 & 0.80 & 0.70 & 0.67 \\
DistSum$_{cosine}$ & 0.85 & 0.67 & 0.76 & 0.51 \\
Vendi Score & 0.70 & 0.85 & 0.78 & 0.60 \\
DistSum$_{L2}$ & 0.86 & 0.76 & 0.81 & 0.51 \\
Cluster Inertia & 0.81 & 0.85 & 0.83 & 0.76 \\
Radius & 0.87 & 0.81 & 0.84 & 0.48 \\
\midrule
NovelSum & \textbf{0.98} & \textbf{0.95} & \textbf{0.97} & \textbf{0.90} \\
\bottomrule
\end{tabular}
}
\caption{Correlations between different metrics and model performance on LLaMA-3-8B and Qwen-2.5-7B.  “Avg.” denotes the average correlation (Eq. \ref{eq:cor}).}
\label{tbl:correlations}
\vspace{-2mm}
\end{table}

\paragraph{\textit{NovelSum} consistently achieves state-of-the-art correlation with model performance across various data selection strategies, backbone LLMs, and correlation measures.}
Table \ref{tbl:main} presents diversity measurement results on datasets constructed by mainstream data selection methods (based on $\mathcal{X}^{all}$), random selection from various sources, and duplicated samples (with only $m=100$ unique samples). 
Results from multiple runs are averaged for each strategy.
Although these strategies yield varying performance rankings across base models, \textit{NovelSum} consistently tracks changes in IT performance by accurately measuring dataset diversity. For instance, K-means achieves the best performance on LLaMA with the highest NovelSum score, while K-Center-Greedy excels on Qwen, also correlating with the highest NovelSum. Table \ref{tbl:correlations} shows the correlation coefficients between various metrics and model performance for both LLaMA and Qwen experiments, where \textit{NovelSum} achieves state-of-the-art correlation across different models and measures.

\paragraph{\textit{NovelSum} can provide valuable guidance for data engineering practices.}
As a reliable indicator of data diversity, \textit{NovelSum} can assess diversity at both the dataset and sample levels, directly guiding data selection and construction decisions. For example, Table \ref{tbl:main} shows that the combined data source $\mathcal{X}^{all}$ is a better choice for sampling diverse IT data than other sources. Moreover, \textit{NovelSum} can offer insights through comparative analyses, such as: (1) ShareGPT, which collects data from real internet users, exhibits greater diversity than Dolly, which relies on company employees, suggesting that IT samples from diverse sources enhance dataset diversity \cite{wang2024diversity-logD}; (2) In LLaMA experiments, random selection can outperform some mainstream strategies, aligning with prior work \cite{xia2024rethinking,diddee2024chasing}, highlighting gaps in current data selection methods for optimizing diversity.



\subsection{Ablation Study}


\textit{NovelSum} involves several flexible hyperparameters and variations. In our main experiments, \textit{NovelSum} uses cosine distance to compute $d(x_i, x_j)$ in Eq. \ref{eq:dad}. We set $\alpha = 1$, $\beta = 0.5$, and $K = 10$ nearest neighbors in Eq. \ref{eq:pws} and \ref{eq:dad}. Here, we conduct an ablation study to investigate the impact of these settings based on LLaMA-3-8B.

\begin{table}[ht!]
\centering
\resizebox{\linewidth}{!}{
\begin{tabular}{lccc}
\toprule
\textbf{Variants} & \textbf{Pearson} & \textbf{Spearman} & \textbf{Avg.} \\
\midrule
NovelSum & 0.98 & 0.96 & 0.97 \\
\midrule
\hspace{0.10cm} - Use $L2$ distance & 0.97 & 0.83 & 0.90\textsubscript{↓ 0.08} \\
\hspace{0.10cm} - $K=20$ & 0.98 & 0.96 & 0.97\textsubscript{↓ 0.00} \\
\hspace{0.10cm} - $\alpha=0$ (w/o proximity) & 0.79 & 0.31 & 0.55\textsubscript{↓ 0.42} \\
\hspace{0.10cm} - $\alpha=2$ & 0.73 & 0.88 & 0.81\textsubscript{↓ 0.16} \\
\hspace{0.10cm} - $\beta=0$ (w/o density) & 0.92 & 0.89 & 0.91\textsubscript{↓ 0.07} \\
\hspace{0.10cm} - $\beta=1$ & 0.90 & 0.62 & 0.76\textsubscript{↓ 0.21} \\
\bottomrule
\end{tabular}
}
\caption{Ablation Study for \textit{NovelSum}.}
\label{tbl:ablation}
\vspace{-2mm}
\end{table}

In Table \ref{tbl:ablation}, $\alpha=0$ removes the proximity weights, and $\beta=0$ eliminates the density multiplier. We observe that both $\alpha=0$ and $\beta=0$ significantly weaken the correlation, validating the benefits of the proximity-weighted sum and density-aware distance. Additionally, improper values for $\alpha$ and $\beta$ greatly reduce the metric's reliability, highlighting that \textit{NovelSum} strikes a delicate balance between distances and distribution. Replacing cosine distance with Euclidean distance and using more neighbors for density approximation have minimal impact, particularly on Pearson's correlation, demonstrating \textit{NovelSum}'s robustness to different distance measures.










\section{Conclusion}
\gxlnote{This paper presents a novel and effective training method aimed at mitigating the memorization problem in diffusion models. By analyzing the relationship between training loss and memorization, we apply different treatments to samples based on their degree of memorization, minimizing the risk of memorization. Additionally, considering that model directly learning from data can increase the likelihood of memorization and the same data may have
different interpretations on different shards, we employ several data shards to train multiple proxy diffusion models. Through multiple proxy diffusion models aggregation and redistribution of easily memorable samples cross shards, we obtain the final model, achieving a balance between mitigating memorization and maintaining image quality. We experimentally show that our method performs favorably with many existing related methods in different scenarios and datasets.}
We firmly believe that this training strategy has a broad application prospect and great development potential in the field of data privacy protection.


\bibliographystyle{IEEEtran}

\bibliography{arxiv}

% \vspace{-0.2cm}
\begin{IEEEbiography}[{\includegraphics[width=1in,height=1.25in,clip,keepaspectratio]{imgs/author/gxl.jpg}}]{Xiaoliu Guan}
received a bachelor's degree from the School of Computer Science, Wuhan University (Wuhan, China) in 2024. She is currently pursuing a master’s degree in computer science at the School of Computer Science, Wuhan University (Wuhan, China). Her research interests mainly include data privacy in diffusion models.
\end{IEEEbiography} 

\begin{IEEEbiography}[{\includegraphics[width=1in,height=1.25in,clip,keepaspectratio]{imgs/author/wuyu.png}}]{Yu Wu}
 is a Professor with the School of Computer Science at Wuhan University, China. He received his Ph.D. degree from the University of Technology Sydney, Australia in 2021. From 2021 to 2022, he was a postdoc at Princeton University. His research interests are controllable generation and multi-modal perception. He was the recipient of AAAI New Faculty Highlight 2024 and Google PhD Fellowship 2020. He served as the Area Chair for CVPR, ICCV, ECCV, and NeurIPS, and also served as the Workshop Chair of CVPR 2023.
\end{IEEEbiography}
\begin{IEEEbiography}[{\includegraphics[width=1in,height=1.25in,clip,keepaspectratio]{imgs/author/hhy.jpg}}]{Huayang Huang}
 received the master's degree from the School of Cyber Science and Engineering, Wuhan University (Wuhan, China), in 2024. She is currently working toward the PhD degree with the School of Computer Science, Wuhan University (Wuhan, China). Her research interests focus on safe generative AI.
\end{IEEEbiography}
\begin{IEEEbiography}[{\includegraphics[width=1in,height=1.25in,clip,keepaspectratio]{imgs/author/liuxiao.png}}]{Xiao Liu}
received a bachelor's degree from the School of Computer Science, Wuhan University (Wuhan, China) in 2024. He is currently pursuing a master's degree in computer science at the  School of Cyber Science and Engineering, Wuhan University (Wuhan, China).  His research interests mainly include diffusion models and multimodal learning.
\end{IEEEbiography} 
\begin{IEEEbiography}[{\includegraphics[width=1in,height=1.25in,clip,keepaspectratio]{imgs/author/miaojiaxu.png}}]{Jiaxu Miao}
received the PhD degree from the University of Technology Sydney, in 2021. He is an assistant professor with the School of Cyber Science and Technology, Sun Yat-sen University, (Shenzhen, China).
His research interests include visual safety and understanding.
\end{IEEEbiography}
\begin{IEEEbiography}[{\includegraphics[width=1in,height=1.25in,clip,keepaspectratio]{imgs/author/yiyang.png}}]{Yi Yang}
(Senior Member, IEEE) is a distinguished Professor with the College of Computer Science and Technology, Zhejiang University. He has authored over 200 papers in top-tier journals and conferences. His papers have received over 70,000 citations, with an H-index of 128. He has received more than 10 international awards in the field of AI, such as the Zhejiang Provincial Science Award First Prize, the Australian Research Council Discovery Early Career Research Award, the Australian Computer Society Gold Digital Disruptor Award, and the Google Faculty Research Award. His current research interests include machine learning and its applications to multimedia content analysis and computer vision, such as multimedia retrieval and generation understanding.
\end{IEEEbiography}
\clearpage
\newpage
% %%%%%%%% ICML 2025 EXAMPLE LATEX SUBMISSION FILE %%%%%%%%%%%%%%%%%
\documentclass{article}

% Recommended, but optional, packages for figures and better typesetting:
\usepackage{microtype}
\usepackage{graphicx}
\usepackage{subfig} % Retained subfig
\usepackage{booktabs} % for professional tables

% hyperref makes hyperlinks in the resulting PDF.
% If your build breaks (sometimes temporarily if a hyperlink spans a page)
% please comment out the following usepackage line and replace
% \usepackage{icml2025} with \usepackage[nohyperref]{icml2025} above.
\usepackage{hyperref}

% Attempt to make hyperref and algorithmic work together better:
\newcommand{\theHalgorithm}{\arabic{algorithm}}

% Use the following line for the initial blind version submitted for review:
% \usepackage{icml2025}

% If accepted, instead use the following line for the camera-ready submission:
\usepackage[accepted]{icml2025}

% For revision
\providecommand{\yulun}[1]{\textcolor{red}{[{\bf #1}]}}

% For theorems and such
\usepackage{amsmath}
\usepackage{amssymb}
\usepackage{mathtools}
\usepackage{amsthm}
\usepackage{arydshln} % For dashed lines in tables
\usepackage{xcolor}
\usepackage{pifont}
\usepackage{colortbl}
\definecolor{colorTab}{rgb}{0.9,0.9,0.98}
\definecolor{color3}{gray}{0.95}
\usepackage{enumitem}
\usepackage{multicol}
% \usepackage{multirow}
\usepackage{algorithm}
\usepackage{algpseudocode}
\usepackage{graphicx}  % For resizing the algorithm


% if you use cleveref..
\usepackage[capitalize,noabbrev]{cleveref}
\definecolor{css}{rgb}{0.7529, 0, 0}
\definecolor{fss}{rgb}{0, 0.7, 0.3}
\definecolor{pbp}{rgb}{0.2, 0.2, 0.6}
\definecolor{reasonable}{HTML}{92D050}
\newcommand{\mycomment}[1]{\hfill\small\texttt{$\triangleright$ #1}}


%%%%%%%%%%%%%%%%%%%%%%%%%%%%%%%%
% THEOREMS
%%%%%%%%%%%%%%%%%%%%%%%%%%%%%%%%
\theoremstyle{plain}
\newtheorem{theorem}{Theorem}[section]
\newtheorem{proposition}[theorem]{Proposition}
\newtheorem{lemma}[theorem]{Lemma}
\newtheorem{corollary}[theorem]{Corollary}
\theoremstyle{definition}
\newtheorem{definition}[theorem]{Definition}
\newtheorem{assumption}[theorem]{Assumption}
\theoremstyle{remark}
\newtheorem{remark}[theorem]{Remark}

% Todonotes is useful during development; simply uncomment the next line
%    and comment out the line below the next line to turn off comments
%\usepackage[disable,textsize=tiny]{todonotes}
\usepackage[textsize=tiny]{todonotes}
% \newcommand{\mycomment}[1]{\hfill\small\texttt{$\triangleright$ #1}}r.
% Therefore, a short form for the running title is supplied here:
\icmltitlerunning{Submission and Formatting Instructions for ICML 2025}

\begin{document}
\twocolumn[
\icmltitle{Progressive Binarization with Semi-Structured Pruning for LLMs}

% It is OKAY to include author information, even for blind
% submissions: the style file will automatically remove it for you
% unless you've provided the [accepted] option to the icml2025
% package.

% List of affiliations: The first argument should be a (short)
% identifier you will use later to specify author affiliations
% Academic affiliations should list Department, University, City, Region, Country
% Industry affiliations should list Company, City, Region, Country

% You can specify symbols, otherwise they are numbered in order.
% Ideally, you should not use this facility. Affiliations will be numbered
% in order of appearance and this is the preferred way.
\icmlsetsymbol{equal}{*}

\begin{icmlauthorlist}
\icmlauthor{Xianglong Yan}{equal,sjtu}
\icmlauthor{Tianao Zhang}{equal,sjtu}
\icmlauthor{Zhiteng Li}{sjtu}
\icmlauthor{Yulun Zhang\textsuperscript{\textdagger}}{sjtu}

%\icmlauthor{}{sch}
%\icmlauthor{}{sch}
\end{icmlauthorlist}

\icmlaffiliation{sjtu}{Shanghai Jiao Tong University}
% \icmlaffiliation{comp}{Company Name, Location, Country}
% \icmlaffiliation{sch}{School of ZZZ, Institute of WWW, Location, Country}

\icmlcorrespondingauthor{Yulun Zhang}{yulun100@gmail.com}
% You may provide any keywords that you
% find helpful for describing your paper; these are used to populate
% the "keywords" metadata in the PDF but will not be shown in the document
\icmlkeywords{Machine Learning, ICML}

\vskip 0.3in
]

% It is OKAY to include author information, even for blind
% submissions: the style file will automatically remove it for you
% unless you've provided the [accepted] option to the icml2025
% package.

% List of affiliations: The first argument should be a (short)
% identifier you will use later to specify author affiliations
% Academic affiliations should list Department, University, City, Region, Country
% Industry affiliations should list Company, City, Region, Country

% You can specify symbols, otherwise they are numbered in order.
% Ideally, you should not use this facility. Affiliations will be numbered
% in order of appearance and this is the preferred way.
\icmlsetsymbol{equal}{*}



% You may provide any keywords that you
% find helpful for describing your paper; these are used to populate
% the "keywords" metadata in the PDF but will not be shown in the document
\icmlkeywords{Machine Learning, ICML}

\vskip 0.3in


% this must go after the closing bracket ] following \twocolumn[ ...

% This command actually creates the footnote in the first column
% listing the affiliations and the copyright notice.
% The command takes one argument, which is text to display at the start of the footnote.
% The \icmlEqualContribution command is standard text for equal contribution.
% Remove it (just {}) if you do not need this facility.

%\printAffiliationsAndNotice{}  % leave blank if no need to mention equal contribution
\printAffiliationsAndNotice{\icmlEqualContribution} % otherwise use the standard text.
% \tableofcontents


\section{Analysis of Binarization Difficulty}
After the standard binarization process, we define the quantization error \( \mathcal{L}_1 \) after binarization as:
\begin{equation}
    \mathcal{L}_1 = \|\mathbf{W} - \widehat{\mathbf{W}}\|_F^2, \quad \text{where} \, \widehat{\mathbf{W}} = \alpha \mathbf{B} + \mu \,.
\end{equation}
$\mathcal{L}_1$ represents the difference between the binarized matrix and the full-precision matrix. It can be observed that during the binarization process, the row-wise scaling factor $\alpha$ is used to approximate the magnitude of elements in each row, and the quantization error $\mathcal{L}_1$ varies depending on the weight distribution. Intuitively, when using the row-wise scaling factor approach for binarization, weight matrices with more dispersed row-wise distributions tend to face greater binarization difficulty, which in turn results in larger quantization errors.

\textbf{Binarization Difficulty Score.$\quad$}We revisit the binarization process and propose a new metric, the Binarization Difficulty (BD) score, which quantifies the inherent difficulty of binarizing a given weight matrix. This score not only provides a measure of how challenging the binarization task is, but also serves as an indicator of the potential quantization error that may arise as a result of the binarization process. The definition of BD score is as follows:
\begin{equation}
    \textit{BD} = \frac{1}{n} \sum_{i=1}^{n} \text{Var}\left( |\mathbf{W}_{i.} - \mu_i| \right),
\end{equation}
where $\mathbf{W}$ represents the given weight matrix. We take into account the redistribution of the weight matrix caused by standard binarization in the calculation of the BD score, and apply absolute value processing before computing the variance. BD score quantifies the difficulty of binarization, where higher values of BD score indicate more dispersed weight distributions and thus greater binarization challenges. 

To validate the effectiveness of the BD score, we performed binarization experiments on different weight matrices and analyzed the relationship between the BD score and quantization error. The results are shown in Figure~\ref{fig:fig4} left. As observed, there is a positive correlation between the BD score and quantization error. We hope that the BD score can serve as a guiding metric in the process of binarization.
\begin{figure}[t]
\centering
\includegraphics[width=\columnwidth]{figs/fig4_v2.pdf}
\vspace{-5mm}
\caption{\textbf{Left}: The left side shows the relationship between quantization error and BD score, where they exhibit a positive correlation. \textbf{Right}: The right side shows the BD score of the same weight matrix at different pruning ratios, where the BD score only decreases at an appropriate pruning ratio.}
\label{fig:fig4}
\vspace{-5mm}
\end{figure}



% \begin{equation}\label{eq1}
% %\vspace{-0.1in}
%     \widehat{\mathbf{W}} = \alpha \cdot \mathbf{B} + \mu
% \end{equation}

\section{Impact of Pruning on Binarization Difficulty}
\textbf{Pruning Enhances Binarization.$\quad$}
We then attempt to apply the $N:M$ sparsity, a semi-structured pruning method, to prune the model. After pruning, we calculate the BD score of the remaining weights in the weight matrix. Figure~\ref{fig:fig4} right shows the BD score of a weight matrix at different pruning ratios. We observe that an appropriate pruning ratio can reduce the difficulty of binarization, leading to lower quantization errors in the binarization phase. Of course, pruning introduces additional pruning errors. In our experiments, we recorded the errors from both pruning error and binarization error, as well as their total error. It is evident that when we use an appropriate pruning ratio, the total error is lower than the binarization error without pruning. This is a crucial finding, suggesting that by combining pruning and binarization techniques, we can effectively reduce errors during model compression, thereby preserving both model performance and efficiency.

\section {SPBO Implementation}\label{func}
Here, we provide the function for splitting the pruning mask. The SPBO algorithm is already largely implemented in the main text. For simplicity, a simplified version of splitting mask is used in the main text. In the detailed implementation, while searching for the pruning mask, we simultaneously perform segmentation on the mask, ultimately returning a pruning mask list. The specific function implementation is shown in Algorithm~\ref{split}.

% \begin{algorithm}[!h]
% \caption{Main Framework of BiLLM: Inner details of each function are shown in Algorithm \ref{alg2} }
% \label{alg1}
% \func{$\operatorname{SplitMask}$}$(\mathbf{W}, N, M, \mathbf{scores})$
% \begin{algorithmic}[1]
% \STATE \textbf{Input:} $\mathbf{W}$ (weight matrix), $N$, $M$, $\mathbf{scores}$ (score matrix)
% \STATE \textbf{Output:} Mask list

% \STATE $\text{rows}, \text{cols} \leftarrow \text{shape of } \mathbf{W}$
% \STATE $\text{num\_groups} \leftarrow \frac{\text{cols}}{M}$
% \STATE $\text{mask\_list} \leftarrow []$
% \STATE $\mathbf{scores\_grouped} \leftarrow \text{reshape}(\mathbf{scores}, \text{rows}, \text{num\_groups}, M)$

% \FOR{$i = 1, 2, \ldots, M - N$}
%     \STATE $\text{top\_indices} \leftarrow \text{torch.topk}(\mathbf{scores\_grouped}, M-i-1, \text{dim}=-1).\text{indices}$
%     \STATE $\text{batch\_indices} \leftarrow \text{torch.arange(rows)}.\text{view}(-1, 1, 1).\text{expand\_as}(\text{top\_indices})$
%     \STATE $\text{group\_indices} \leftarrow \text{torch.arange(num\_groups)}.\text{view}(1, -1, 1).\text{expand\_as}(\text{top\_indices})$
%     \STATE $\mathbf{mask\_grouped} \leftarrow \text{torch.zeros\_like}(\mathbf{scores\_grouped}, \text{dtype}=\text{torch.bool})$
%     \STATE $\mathbf{mask\_grouped}[\text{batch\_indices}, \text{group\_indices}, \text{top\_indices}] \leftarrow \text{True}$
%     \STATE $\mathbf{mask} \leftarrow \text{reshape}(\mathbf{mask\_grouped}, \text{rows}, \text{cols})$
%     \STATE $\text{mask\_list}.append(\mathbf{mask})$
% \ENDFOR
% \STATE \textbf{return} $\text{mask\_list}$
% \end{algorithmic}
% \end{algorithm}
\begin{algorithm*}[!h]
    \caption{Detailed algorithm for splitting pruning mask into group}
% \vspace{-4.2mm}
% \begin{multicols}{2}
\label{split}
\func{$\operatorname{SplitMask}$}$(\mathbf{W}, N, M, \mathbf{scores})$
\begin{algorithmic}[1]
\STATE \textbf{Input:} $\mathbf{W}$ (weight matrix), $N$, $M$, $\mathbf{scores}$ (score matrix)
\STATE \textbf{Output:} Mask list

\STATE $\text{rows}, \text{cols} \leftarrow \text{shape of } \mathbf{W}$
\STATE $\text{num\_groups} \leftarrow \frac{\text{cols}}{M}$
\STATE $\text{mask\_list} \leftarrow []$
\STATE $\mathbf{scores\_grouped} \leftarrow \text{reshape}(\mathbf{scores}, \text{rows}, \text{num\_groups}, M)$

\FOR{$i = 1, 2, \ldots, M - N$}
    \STATE $\text{top\_indices} \leftarrow \text{torch.topk}(\mathbf{scores\_grouped}, M-i-1, \text{dim}=-1).\text{indices}$
    \STATE $\text{batch\_indices} \leftarrow \text{torch.arange(rows)}.\text{view}(-1, 1, 1).\text{expand\_as}(\text{top\_indices})$
    \STATE $\text{group\_indices} \leftarrow \text{torch.arange(num\_groups)}.\text{view}(1, -1, 1).\text{expand\_as}(\text{top\_indices})$
    \STATE $\mathbf{mask\_grouped} \leftarrow \text{torch.zeros\_like}(\mathbf{scores\_grouped}, \text{dtype}=\text{torch.bool})$
    \STATE $\mathbf{mask\_grouped}[\text{batch\_indices}, \text{group\_indices}, \text{top\_indices}] \leftarrow \text{True}$
    \STATE $\mathbf{mask} \leftarrow \text{reshape}(\mathbf{mask\_grouped}, \text{rows}, \text{cols})$
    \STATE $\text{mask\_list}.append(\mathbf{mask})$
\ENDFOR
\STATE \textbf{return} $\text{mask\_list}$
\end{algorithmic}

% \end{multicols}
% \vspace{-3mm}
\end{algorithm*}


% \begin{algorithm}[!h]
% % \begin{multicols}{2}
% \caption{Main Framework of BiLLM: Inner details of each function are shown in Algorithm \ref{alg2} }
% \label{alg1}

% % \func{$\operatorname{Mask2Group}$}$(\mathbf{M})$
% % \begin{algorithmic}[1]
% % \FOR{$b = 1, 2, \ldots, B$}
% %     \FOR{$g = 1, 2, \ldots, G$}
% %         \STATE  $\mathbf{M}_\text{group}[b,g] = \mathbf{M}[b, g \cdot M : (g+1) \cdot M]$
% %     \ENDFOR
% % \ENDFOR
% % \STATE \textbf{return} $\mathbf{M}_\text{group}$

% \func {$\operatorname{Mask2Group}$}$(\mathbf{W}, N, M, \mathbf{Scores})$
% \begin{algorithmic}[1]
%     \STATE Let rows, cols = dimensions of W
%     \STATE Let num_groups = cols // M
%     \STATE Initialize empty mask_list

%     \STATE Reshape scores into groups: scores_grouped = reshape(scores, (rows, num_groups, M))

%     \For {i from 0 to (M - N - 1)}
%         \STATE Find the top M-i-1 indices in each group: top_indices = topk(scores_grouped)
%         \STATE Generate batch and group indices
%         \STATE Initialize mask_grouped with zeros (shape: scores_grouped)
%         \STATE Set mask_grouped elements corresponding to top_indices to True
%         \STATE Reshape mask_grouped back to the original shape (rows, cols)
%         \STATE Append the mask to mask_list
%     \ENDFOR
% \STATE \textbf{return} $\mathbf{M}_\text{group}$



% \end{algorithmic}
% \end{algorithm}

% \begin{algorithm*}[!h]
%     \caption{aaa}
% % \vspace{-4.2mm}
% % \begin{multicols}{2}
% \label{mainalg}
% func $\operatorname{SPBO}$($\mathbf{W}$, $\mathbf{M}_\text{p}$, $\mathbf{X}$ ,$T$)\\ 
% {\bf Input:} $\mathbf{W} \in \mathbb{R}^{n\times m}$ - full-precision weight \\
% \hspace*{0.43in}$ \mathbf{M}_\text{p} \in \mathbb{R}^{n\times m}$ - pruning mask \\
% \hspace*{0.43in}$ \mathbf{X} \in \mathbb{R}^{B\times L \times m}$ - calibration data \\
% \hspace*{0.43in}$T$ - binarized parameter optimization steps\\
% {\bf Output:} $ \mathbf{\widehat{W}} \in \mathbb{R}^{n\times m}$
% \begin{algorithmic}[1]
% \STATE $\mathbf{S} \coloneqq  \operatorname{X2S}(\mathbf{X})$ \mycomment{decouple $\mathbf{W}$ and $\mathbf{X}$}
% \STATE $\widehat{\mathbf{W}},\alpha,\mathbf{B},\mu \coloneqq \operatorname{binary}(\mathbf{W})$
% \STATE $\textbf{M}_\text{group} \coloneqq \operatorname{split\_mask}(\mathbf{M}_\text{p})$\mycomment{pruning mask group}
% \STATE $\mathbf{M}_\text{k} \coloneqq \mathbf{M}_\text{group}[0] $
% \FOR{$\textbf{M} \text{ in } \mathbf{M}_\text{group}$}
%     \STATE $\mathbf{B} \leftarrow \mathbf{B}\odot \mathbf{M}$\mycomment{prune binarized matrix}
%     \STATE $\mathbf{W} \leftarrow \mathbf{W}\odot \mathbf{M}$
%     \STATE $\mathbf{M}_k \leftarrow \mathbf{M}_k \cup \mathbf{M}$ \mycomment{current pruning mask}

%     \FOR{$iter = 1, 2, \ldots, T$}
   
%        % \STATE $\mathbf{P}\coloneqq \operatorname{einsum}(\mathbf{R})$
%     % \STATE $\mathcal{L} \coloneqq \sum_k\sum_l (\mathbf{S}\odot  \mathbf{P})_{kl}$
%         \STATE $\mu \leftarrow \operatorname{update\_\mu}(\mathbf{S},\mathbf{W},\mathbf{B},\alpha)$\mycomment{update $\mu$}
%         \STATE $\alpha \leftarrow \operatorname{update\_\alpha}(\mathbf{S},\mathbf{W},\mu,\mathbf{B})$\mycomment{update $\alpha$}
%         \STATE $\widehat{\mathbf{W}} \leftarrow (\alpha\cdot\mathbf{B}+\mu)\odot\mathbf{M}_k$
%     \ENDFOR
% \ENDFOR

% \STATE {\bf return}  $\mathbf{\widehat{W}}$

% \end{algorithmic}

% func $\operatorname{update\_\alpha}$ $(\mathbf{S},\mathbf{W},\mathbf{\mu},\mathbf{B})$
% \begin{algorithmic}[1]
% \STATE $\widetilde{\mathbf{W}}\coloneqq \mathbf{W}-\mu$
% \FOR{$i = 1, 2, \ldots, n$}
%     \FOR{$k = 1, 2, \ldots, m;l = 1, 2, \ldots, m$}
%         \STATE $\mathbf{U}_{kl}\coloneqq \mathbf{B}_{ik}\widetilde{\mathbf{W}}_{il}$
%         \STATE $\mathbf{V}_{kl}\coloneqq\mathbf{B}_{ik}\mathbf{B}_{il}$
%     \ENDFOR
%     \STATE $num \coloneqq \sum_k\sum_l(\mathbf{S}\odot \mathbf{U})_{kl}$
%     \STATE $den \coloneqq \sum_k\sum_l(\mathbf{S}\odot\mathbf{V})_{kl}+\epsilon$
%     \STATE $\alpha_i \coloneqq \frac{num}{den}$\mycomment{refined $\alpha$}
% \ENDFOR
% \STATE {\bf return}  $\alpha$
% \end{algorithmic}

% func $\operatorname{update\_\mu}$ $(\mathbf{S},\mathbf{W},\mathbf{B},\alpha)$
% \begin{algorithmic}[1]
% \FOR{$i = 1, 2, \ldots, n$}
%     \FOR{$k = 1, 2, \ldots, m;l = 1, 2, \ldots, m$}
%         \STATE $\mathbf{P}_{kl}\coloneqq\mathbf{W}_{ik}-\alpha_i\mathbf{B}_{il}$
%     \ENDFOR
%     \STATE $num \coloneqq \sum_k\sum_l(\mathbf{S}\odot\mathbf{P})_{kl}$
%     \STATE $den \coloneqq \sum_k\sum_l\mathbf{S}_{kl}+\epsilon$
%     \STATE $\mu_i \coloneqq \frac{num}{den}$\mycomment{refined $\mu$}
% \ENDFOR
% \STATE {\bf return}  $\mu$
% \end{algorithmic}
% % \vfill  % 填充垂直空间
% % \columnbreak  % 强制换列

% func $\operatorname{binary}$ $(\mathbf{W})$
% \begin{algorithmic}[1]
% \STATE $\mu \coloneqq \frac{1}{m}\sum_{j=1}^m\mathbf{W}_{.j}$
% \STATE $\widetilde{\mathbf{W}}\coloneqq \mathbf{W}-\mu$\mycomment{row-wise redistribution}
% % \STATE $\alpha \coloneqq \dfrac{||\widetilde{\mathbf{W}}||_{\ell1}}{n\times m}$
% \STATE $\alpha \coloneqq \frac{1}{m}\sum_{j=1}^m|\widetilde{\mathbf{W}}_{.j}|$\mycomment{row-wise scaling factor}
% \STATE $\mathbf{B} \coloneqq  \operatorname{sign}(\widetilde{\mathbf{W}})$
% \STATE $\mathbf{\widehat{W}} \coloneqq \alpha\cdot \mathbf{B} + \mu $\mycomment{binarized output}
% \STATE {\bf return}  $\mathbf{\widehat{W}},\alpha,\mathbf{B},\mu$
% \end{algorithmic}


% func $\operatorname{X2S}$ $(\mathbf{X})$
% \begin{algorithmic}[1]
% \FOR{$b = 1, 2, \ldots B$}
%     \FOR{$k = 1, 2, \ldots, m;l = 1, 2, \ldots, m$}
%         \STATE  $\mathbf{S}_{kl} = \sum_b\sum_i(\mathbf{X}_b)_{ik}(\mathbf{X}_b)_{il}$
%     \ENDFOR
% \ENDFOR
% \STATE {\bf return}  $\mathbf{S}$
% \end{algorithmic}






% \end{multicols}
% \vspace{-3mm}
% \end{algorithm*}

\section{Derivation of the Optimization Formulas in SPBO}
Since simultaneously optimizing the pruning mask and binarization parameters is an NP-hard problem, we use a greedy algorithm to solve this issue. First, we derive the optimization formula for the binarization parameters without the pruning mask. The current definition of quantization error is:
\begin{equation}
    \mathcal{L} = ||\mathbf{W}\mathbf{X}-\widehat{\mathbf{W}}\mathbf{X}||^2_{F}.
\end{equation}

\textbf{Rewritten quantization error to decouple $\mathbf{W}$ and $\mathbf{X}$}\quad
We first rewrite the quantization error $\mathcal{L}$ to decouple $\mathbf{W}$ and $\mathbf{X}$, reducing the computational cost when calculating the quantization error. We define $\widetilde{\mathbf{W}}$ as $\widetilde{\mathbf{W}} = \mathbf{W}-\mu$. Then we rewrite the quantization error as:
\begin{align} \label{eq2}
    \mathcal{L} 
    &= ||\mathbf{W}\mathbf{X}-\widehat{\mathbf{W}}\mathbf{X}||_F^2\\
    &= ||\mathbf{X}(\widetilde{\mathbf{W}}-\alpha\mathbf{B})^\top||_F^2\\
    &= \sum_i\sum_j(\sum_b\sum_k(\mathbf{X}_b)_{ik}(\widetilde{\mathbf{W}}_{jk}-\alpha_j\mathbf{B}_{jk}))^2.
\end{align}

The residual matrix is defined as $\mathbf{R}=\mathbf{W}-\mu-\alpha\mathbf{B}$ and further simplify $\mathcal{L}$:
\begin{align} \label{eq2}
    \mathcal{L} 
    &= \sum_i\sum_j(\sum_b\sum_k(\mathbf{X}_b)_{ik}\mathbf{R}_{jk})^2\\
    &= \sum_i\sum_j(\sum_b\sum_k\sum_l(\mathbf{X}_b)_{ik}(\mathbf{X}_b)_{il}\mathbf{R}_{jk}\mathbf{R}_{jl})\\
    &= \sum_k\sum_l(\sum_b\sum_i(\mathbf{X}_b)_{ik}(\mathbf{X}_b)_{il})(\sum_j\mathbf{R}_{jk}\mathbf{R}_{jl}).
\end{align}

After that, we define the matrix $\mathbf{S}$ using the following formula:
\begin{equation}\label{eq10}
    \mathbf{S}_{kl} = \sum_b\sum_i(\mathbf{X}_b)_{ik}(\mathbf{X}_b)_{il},
\end{equation}
where $k=1,2,\dots,m$, $l=1,2,\dots,m$. Then we obtain the final simplified $\mathcal{L}$ as
\begin{equation}
    \mathcal{L} = \langle \mathbf{S}, \mathbf{R^\top R} \rangle_F = \text{Tr}(\mathbf{R}\mathbf{S}\mathbf{R}^\top).
\end{equation}

\textbf{Parameter Optimization Formula}\quad
We use the quantization error $\mathcal{L}$ to update $\mu$:
\begin{align} \label{eq2}
    \mathcal{L} 
    &=\sum_k\sum_l\mathbf{S}_{kl}\sum_j\mathbf{R}_{jk}\mathbf{R}_{jl}\\
    &= \sum_k\sum_l\mathbf{S}_{kl}\sum_j(\widetilde{\mathbf{W}}_{jk}\widetilde{\mathbf{W}}_{jl}\\
    &-\alpha_j(\mathbf{B}_{jk}\widetilde{\mathbf{W}}_{jl}+\mathbf{B}_{jl}\widetilde{\mathbf{W}}_{jk})+\alpha_j^2\mathbf{B}_{jk}\mathbf{B}_{jl})\\
    &=\sum_k\sum_l\mathbf{S}_{kl}\sum_j((\mathbf{W}_{jk}-\mu_j)(\mathbf{W}_{jl}-\mu_j)\\
    &-\alpha_j(\mathbf{B}_{jk}(\mathbf{W}_{jl}-\mu_j)\\
    &+\mathbf{B}_{jl}(\mathbf{W}_{jk}-\mu_j))+\alpha_j^2\mathbf{B}_{jk}\mathbf{B}_{jl}).
\end{align}


To obtain the optimal solution for $\mu$, we take the partial derivative of $\mathcal{L}$ with respect to $\mu_j$, where $j=1,2,\dots,n$:
\begin{equation} \label{eq2}
    \frac{\partial \mathcal{L}}{\partial \mu_j} = \sum_k\sum_l\mathbf{S}_{kl}(-\mathbf{W}_{jl}-\mathbf{W}_{jk}+2\mu_j+\alpha_j\mathbf{B}_{jk}+\alpha_j\mathbf{B}_{jl}).
\end{equation}

We set $\frac{\partial \mathcal{L}}{\partial \mu_j} = 0$ to get the optimal solution for $\mu_j$:
\begin{align} \label{eq2}
    \mu_j 
    &= \frac{\sum_k\sum_l\mathbf{S}_{kl}(\mathbf{W}_{jk}-\alpha_j\mathbf{B}_{jk}+\mathbf{W}_{jl}-\alpha_j\mathbf{B}_{jl})}{2\sum_k\sum_l\mathbf{S}_{kl}},\\
    &\text{where $j=1,2,\dots,n$}.
\end{align}

Then, we define the matrix $\mathbf{P}$ as:
\begin{align} \label{eq2}
   &\mathbf{P}_{kl} = \mathbf{W}_{jk}-\alpha_j\mathbf{B}_{jl},\\
   &\text{where $k=1,2,\dots,m$, $l=1,2,\dots,m$}.
\end{align}

After that, we can simplify $\mu_j$ as
\begin{align} \label{eq2}
    &\mu_j = \frac{\sum_k\sum_l(\mathbf{S}\odot(\mathbf{P}+\mathbf{P}^\top))_{kl}}{2\sum_k\sum_l\mathbf{S}_{kl}},\\
    &\text{where $j=1,2,\dots,n$}.
\end{align}

Since $\mathbf{S}$ is symmetric, we can further simplify the above equation as:
\begin{align} \label{eq2}
    \mu_j 
    &= \frac{\sum_k\sum_l(\mathbf{S}\odot\mathbf{P})_{kl}}{\sum_k\sum_l\mathbf{S}_{kl}},\quad \text{where $j=1,2,\dots,n$}.
\end{align}

We can also express $\mu$ in a more compact vector form:
\begin{align}
\mu = \frac{\mathbf{1}^\top \mathbf{S} (\mathbf{W} - \alpha \mathbf{B})^\top}{\mathbf{1}^\top \mathbf{S} \mathbf{1}}.
\end{align}

Similarly, we use the same quantization error to update $\alpha$:
\begin{align} \label{eq2}
    \mathcal{L} 
    &= \sum_k\sum_l\mathbf{S}_{kl}\sum_j\mathbf{R}_{jk}\mathbf{R}_{jl}\\
    &= \sum_k\sum_l\mathbf{S}_{kl}\sum_j(\widetilde{\mathbf{W}}_{jk}\widetilde{\mathbf{W}}_{jl}\\
    &-\alpha_j(\mathbf{B}_{jk}\widetilde{\mathbf{W}}_{jl}+\mathbf{B}_{jl}\widetilde{\mathbf{W}}_{jk})+\alpha_j^2\mathbf{B}_{jk}\mathbf{B}_{jl}).
\end{align}

To obtain the optimal solution for $\alpha$, we take the partial derivative of $\mathcal{L}$ with respect to $\alpha_j$, where $j=1,2,\dots,n$:
\begin{equation} \label{eq2}
    \frac{\partial \mathcal{L}}{\partial \alpha_j} = \sum_k\sum_l\mathbf{S}_{kl}(2\mathbf{B}_{jk}\mathbf{B}_{jl}\alpha_j-(\mathbf{B}_{jk}\widetilde{\mathbf{W}}_{jl}+\mathbf{B}_{jl}\widetilde{\mathbf{W}}_{jk})).
\end{equation}

We set $\frac{\partial \mathcal{L}}{\partial \alpha_j} = 0$ to get the optimal solution for $\alpha_j$:
\begin{align} \label{eq2}
   & \alpha_j 
    = \frac{\sum_k\sum_l\mathbf{S}_{kl}(\mathbf{B}_{jk}\widetilde{\mathbf{W}}_{jl}+\mathbf{B}_{jl}\widetilde{\mathbf{W}}_{jk})}{2\sum_k\sum_l\mathbf{S}_{kl}(\mathbf{B}_{jk}\mathbf{B}_{jl})},\\
    &\text{where $j=1,2,\dots,n$}.
\end{align}

Then, we define the matrix $\mathbf{U}$ and $\mathbf{V}$ as:
\begin{equation} \label{eq2}
   \mathbf{U}_{kl} = \mathbf{B}_{jk}\widetilde{\mathbf{W}}_{jl},\,\mathbf{V}_{kl}=\mathbf{B}_{jk}\mathbf{B}_{jl},
\end{equation}
where $k=1,2,\dots,m$, $l=1,2,\dots,m$. After that, we simplify $\alpha_j$ using $\mathbf{U}$ and $\mathbf{V}$:
\begin{equation} \label{eq2}
   \alpha_j= \frac{\sum_k\sum_l(\mathbf{S}\odot(\mathbf{U}+\mathbf{U}^\top))_{kl}}   {2\sum_k\sum_l(\mathbf{S} \odot \mathbf{V})_{kl}}.
\end{equation}

Since $\mathbf{S}$ is symmetric, we can further simplify the above equation as
\begin{equation} \label{eq2}
   \alpha_j= \frac{\sum_k\sum_l(\mathbf{S}\odot\mathbf{U})_{kl}}   {\sum_k\sum_l(\mathbf{S} \odot \mathbf{V})_{kl}}.
\end{equation}

We can also express $\alpha$ in a more compact vector form:
\begin{equation}
    \alpha= \frac{\operatorname{diag}(\mathbf{B}\mathbf{S}(\mathbf{W}-\mu)^\top)}{\operatorname{diag}(\mathbf{B}\mathbf{S}\mathbf{B}^\top)}.
\end{equation}
This is the parameter optimizing formula in the absence of a pruning mask. In the presence of a pruning mask, we directly prune the $\mathbf{W}$ and $\mathbf{B}$ matrices and then apply the optimizing formula. This is a simplified approach, where only the retained weights undergo binary approximation.




\section{More Experimental Results}
\subsection{Perplexity results} 
We present more experimental results related to perplexity. In Table~\ref{tab:ptb_series}, we show the PTB test results of our method PBS$^2$P compared to other baseline methods. It is evident that our method outperforms the existing SOTA binarization methods. Table~\ref{tab:c4_series} displays the results on the C4 dataset, where our method continues to perform the best, and we also achieve a lower average bit count.
\subsection{Zero-shot results}
We also provide the comparison of 7 zero-shot QA datasets on the OPT family and LLaMA family, as shown in~\Cref{tab:zero_shot_acc_opt} and in ~\Cref{tab:zero_shot_acc_llama}.
\subsection{Ablation Study}
We also conducted our ablation experiments on the LLaMA-2-7B model, and observed the same trends and phenomena as in the ablation experiments on LLaMA-1-7B in the main text. This further strengthens the conclusions drawn from our ablation study and validates the effectiveness of our method. Table~\ref{tab:Progressive_Strategy} to Table~\ref{tab:split_points} present the results of our ablation experiments on LLaMA-2-7B.
% \subsection{Experimental Results for All Pruning Ratios}
\section{Dialog Examples} \label{ap_dialog}
In Figure~\ref{fig:dialog}, we present several dialogue results, including methods such as BiLLM, ARB-LLM, and PBSP. These methods are applied to the LLaMA-13B and LLaMA-2-13B models, and the corresponding dialogue results are shown.

\newpage
\begin{table*}[t]
\vspace{-2mm}
\renewcommand{\arraystretch}{1.0}
\footnotesize
\centering
% \setlength{\tabcolsep}{1.90mm}
\caption{Perplexity comparison of RTN, GPTQ, BiLLM, ARB-LLM, and PBS$^2$P on the LLaMA and OPT families. The evaluation results demonstrate the perplexity performance on the PTB dataset across various model sizes. }
\label{tab:ptb_series}
\vspace{2mm}
\resizebox{1.0\textwidth}{!}{
\begin{tabular}{lccrrrrrrcrrr}
    \toprule
    \rowcolor{color3}
    \multicolumn{3}{c}{\textbf{Settings}} & \multicolumn{4}{c}{\textbf{ LLaMA-1}}  & \multicolumn{2}{c}{\textbf{ LLaMA-2}} & \multicolumn{1}{c}{\textbf{ LLaMA-3}}  &
    \multicolumn{3}{c}{\textbf{ OPT}}\\
    \midrule
   Method & \#Block & \multicolumn{1}{p{4.19em}}{W-Bits} & 7B    & 13B   & 30B   & 65B   & 7B    & 13B   & 8B & 1.3B & 2.7B & 30B\\
    \midrule
    FP16 & -     & 16    & 41.15  & 28.09   &  23.51   & 25.06  & 37.91 &50.93    & 11.18  & 20.29&17.97  &14.03 \\
    \cdashline{1-13}
    \addlinespace[0.2em]

    RTN    &   -  & 3  & 3.2e2 & 64.52 & 80.45 & 81.56 & 1.6e3 &2.2e2 &1.8e3 & 8.9e3 & 9.0e3 & 1.0e3\\
    GPTQ     & 128  & 3 & 84.88 & 26.40 &20.21 & 19.54
    & 4.8e3  & 40.33 & 18.83 &17.54 & 15.15 &11.28 \\ 
    RTN    &   -  & 2  &1.2e5 &8.4e4 & 3.2e4& 2.1e4&2.4e4& 5.1e4 & 6.3e5& 8.0e3& 5.9e3 & 1.0e5\\
    GPTQ     & 128  & 2 &1.4e3 & 2.2e2 &69.46 &47.70 & 5.5e3   &4.1e2 & 717.23  &1.1e2 & 58.38&14.18\\ 
    \cdashline{1-13}

    \addlinespace[0.2em]
    RTN    &   -  & 1  & 1.5e5 & 1.9e6 & 1.4e4& 6.8e4 & 9.9e4 &3.8e4  &7.6e5 & 1.1e4 & 2.8e4 & 5.4e3\\
    GPTQ     & 128  & 1 &1.2e5 &1.0e5 &1.0e4&2.0e4& 6.6e4 &2.7e4  & 9.7e5 & 6.5e3 & 8.4e3 & 7.1e3\\ 
    BiLLM & 128   & 1.11  & 3.7e2 & 84.87  & 43.10  & 44.68   & 5.2e3  &3.0e2 & 87.25  &1.1e2  & 88.52  &21.41  \\
    ARB-LLM & 128   & 1.11  &1.9e2   &54.38   & 34.65  & 32.20   & 389.59  & 1.9e2  &45.49  & 43.34 &31.77 & 16.88 \\
    \cdashline{1-13}
    \addlinespace[0.2em]
    \rowcolor{colorTab}
    PBS$^2$P & 128   & 0.80  & \textbf{80.27} & \textbf{34.01} & \textbf{26.68} & \textbf{27.54} & \textbf{67.74} & \textbf{68.90} & \textbf{16.15} & \textbf{39.40} & \textbf{29.40} & \textbf{13.04}\\
    \rowcolor{colorTab}
    PBS$^2$P & 128   & 0.70  & \textbf{87.69} & \textbf{38.11} & \textbf{28.38} & \textbf{29.12} & \textbf{75.22} & \textbf{79.59} & \textbf{18.65} & \textbf{46.98} & \textbf{31.71} & \textbf{17.14}\\
    \rowcolor{colorTab}
    PBS$^2$P & 128   & 0.55   & \textbf{134.63} & \textbf{54.06} & \textbf{32.47} & \textbf{33.36} & \textbf{119.30} & \textbf{114.40} & \textbf{25.86} & \textbf{76.55} & \textbf{42.85} & \textbf{18.73}\\
    \bottomrule
    \end{tabular}
}
\vspace{-5mm}
\end{table*}

\newpage
\begin{table*}[!t]
\vspace{-2mm}
\renewcommand{\arraystretch}{1.0}
\footnotesize
\centering
% \setlength{\tabcolsep}{1.90mm}
\caption{Perplexity comparison of RTN, GPTQ, BiLLM, ARB-LLM, and PBS$^2$P on the LLaMA and OPT families. The evaluation results demonstrate the perplexity performance on the C4 dataset across various model sizes. }
\label{tab:c4_series}
\vspace{2mm}
\resizebox{1.0\textwidth}{!}{
\begin{tabular}{lccrrrrrrcrrr}
    \toprule
    \rowcolor{color3}
    \multicolumn{3}{c}{\textbf{Settings}} & \multicolumn{4}{c}{\textbf{ LLaMA-1}}  & \multicolumn{2}{c}{\textbf{ LLaMA-2}} & \multicolumn{1}{c}{\textbf{ LLaMA-3}}  &
    \multicolumn{3}{c}{\textbf{ OPT}}\\
    \midrule
   Method & \#Block & \multicolumn{1}{p{4.19em}}{W-Bits} & 7B    & 13B   & 30B   & 65B   & 7B    & 13B   & 8B & 1.3B & 2.7B & 30B\\
    \midrule
    FP16 & -     & 16    & 7.34   & 6.80   &  6.13   & 5.81   & 7.26  &6.73    & 9.45   & 16.07 &14.34  &11.45 \\
    \cdashline{1-13}
    \addlinespace[0.2em]

    RTN    &   -  & 3  & 28.24 & 13.24 & 28.58 & 12.76 & 3.8e2 & 12.50 & 5.7e2& 5.0e3& 1.1e4 & 1.0e3\\
    GPTQ     & 128  & 3 & N/A & 7.15  &6.51 & 6.03
    & 7.94  & 7.05 & 17.68 & 16.11 & 14.16 & 10.91\\ 
    RTN    &   -  & 2  &1.1e5 & 5.8e4 &2.7e4& 2.2e4 & 3.0e4 & 5.1e4& 7.7e5& 7.4e3& 7.3e3 & 6.1e4\\
    GPTQ     & 128  & 2 &79.06 &18.97&14.86 &10.23&  35.26  &19.65  & 3.9e2 & 63.05 & 35.80&12.92 \\ 
    \cdashline{1-13}

    \addlinespace[0.2em]
    RTN    &   -  & 1  & 1.9e5 & 9.94 & 1.3e4 & 1.3e5 &1.1e5&4.6e4  &1.4e6 &1.0e4& 2.3e4 &5.0e3 \\
    GPTQ     & 128  & 1 &1.8e5&1.0e5 &9.5e3 &2.3e4 & 6.7e4  & 1.9e4 & 1.1e6 &6.3e3 & 6.7e3 &7.9e3 \\ 
    BiLLM & 128   & 1.11  &46.96  &16.83   & 12.11  &  11.09  &39.38 & 25.87  &61.04   &64.14   & 44.77  &16.17  \\
    ARB-LLM & 128   & 1.11  & 17.92  & 12.48  & 10.09  &  8.91  & 20.12  & 14.29  &35.70  &28.19  & 21.46  &13.34  \\
    \cdashline{1-13}
    \addlinespace[0.2em]
    \rowcolor{colorTab}
    PBS$^2$P & 128   & 0.80  & \textbf{9.52} & \textbf{8.11} & \textbf{7.16} & \textbf{6.63} & \textbf{9.43} & \textbf{8.44} & \textbf{15.62} & \textbf{25.62} & \textbf{20.78} & \textbf{12.69}\\
    \rowcolor{colorTab}
    PBS$^2$P & 128   & 0.70  & \textbf{10.35} & \textbf{8.96} & \textbf{7.83} & \textbf{7.04} & \textbf{10.43} & \textbf{9.20} & \textbf{17.48} & \textbf{30.12} & \textbf{22.73} & \textbf{13.04}\\
    \rowcolor{colorTab}
    PBS$^2$P & 128   & 0.55   & \textbf{13.06} & \textbf{11.44} & \textbf{9.33} & \textbf{8.52} & \textbf{13.12} & \textbf{11.19} & \textbf{22.90} & \textbf{46.07} & \textbf{30.33} & \textbf{14.21}\\
    \bottomrule
    \end{tabular}
}
\end{table*}







\newpage
\begin{table*}[h]
  \centering
    \caption{
  Accuracy(\%) of 7 QA datasets on \textbf{OPT} family. We compare the results among GPTQ, PB-LLM, BiLLM, ARB-LLM, and PBSP$^2$P to validate the quantization effect.}
\vspace{1mm}
\resizebox{1.0\textwidth}{!}{
  \setlength{\tabcolsep}{5.5pt}
    \begin{tabular}{llccccccccc}
    \toprule
    \rowcolor{color3}
    \textbf{Models} & \textbf{Method} &\textbf{W-Bits}& \textbf{PIQA} $\uparrow$& \textbf{BoolQ} $\uparrow$& \textbf{OBQA} $\uparrow$& \textbf{Winogrande} $\uparrow$& \textbf{ARC-e} $\uparrow$& \textbf{ARC-c} $\uparrow$ & \textbf{Hellaswag} $\uparrow$ & \textbf{Average} $\uparrow$ \\
    \midrule
    \multirow
          & FP16 &16 &71.71 &57.74 &23.20 &59.75 &57.11 &23.38 &41.49 &47.77 \\
          \cdashline{2-11}
    \addlinespace[0.2em]
          & GPTQ & 2.00  & 59.47 & 42.66 & 15.80 & 50.04 & 37.21 & 21.42 & 30.92 & 36.79\\
          & PB-LLM & 1.70  & 54.57 & 61.77 & 13.00 & 50.99 & 28.79 & 20.56 & 26.55 & 36.60\\
          \textbf{OPT-1.3B}& BiLLM & 1.11  & 59.52 & 61.74 & 14.80 & 52.17 & 36.53 & 17.83 & 29.64 & 38.89  \\
          & ARB-LLM& 1.11  & 65.45& 60.31 & 15.40 & 53.04 & 48.27 & 19.37 & 33.44 & 42.18 \\
          \cdashline{2-11}
    \addlinespace[0.2em]
          & \cellcolor{colorTab}PBS$^2$P &\cellcolor{colorTab}0.80& \cellcolor{colorTab}\textbf{65.72} & \cellcolor{colorTab}\textbf{61.93} & \cellcolor{colorTab}\textbf{16.00} & \cellcolor{colorTab}\textbf{56.04} & \cellcolor{colorTab}\textbf{47.35} & \cellcolor{colorTab}\textbf{22.10} & \cellcolor{colorTab}\textbf{34.51} & \cellcolor{colorTab}\textbf{43.38}  \\
          & \cellcolor{colorTab}PBS$^2$P&\cellcolor{colorTab}0.55 &\cellcolor{colorTab}\textbf{62.62} & \cellcolor{colorTab}\textbf{62.02} & \cellcolor{colorTab}\textbf{13.80} & \cellcolor{colorTab}\textbf{53.04} & \cellcolor{colorTab}\textbf{39.60} & \cellcolor{colorTab}\textbf{19.11} & \cellcolor{colorTab}\textbf{31.13} & \cellcolor{colorTab}\textbf{40.19}  \\
    \midrule
    \multirow 
          & FP16 &16&73.78 &60.28 &25.00 &61.01 &60.77 &26.88 &45.86 & 50.51   \\
          \cdashline{2-11}
    \addlinespace[0.2em]
          & GPTQ & 2.00  & 61.81 & 54.43 & 15.40 & 52.33 & 40.15 & 20.56 & 32.55 & 39.60\\
          & PB-LLM & 1.70  & 56.42 & 62.23 & 12.80 & 50.12 & 31.61 & 18.60 & 27.61 & 37.06\\
          \textbf{OPT-2.7B}& BiLLM & 1.11  & 62.57 & 62.20 & 15,40 & 52.57 & 39.65 & 19.80 & 30.88 & 40.44  \\
          & ARB-LLM & 1.11  & 68.50 & 61.99 & 21.60 & 58.33 & 52.82 & 22.27 & 37.50 & 46.14 \\
          \cdashline{2-11}
    \addlinespace[0.2em]
          & \cellcolor{colorTab}PBS$^2$P &\cellcolor{colorTab}0.80 & \cellcolor{colorTab}\textbf{69.04} & \cellcolor{colorTab}\textbf{62.69} & \cellcolor{colorTab}\textbf{19.40} & \cellcolor{colorTab}\textbf{56.75} & \cellcolor{colorTab}\textbf{54.38} & \cellcolor{colorTab}\textbf{22.44} & \cellcolor{colorTab}\textbf{37.74} & \cellcolor{colorTab}\textbf{46.06} \\
          & \cellcolor{colorTab}PBS$^2$P &\cellcolor{colorTab}0.55 & \cellcolor{colorTab}\textbf{66.65} & \cellcolor{colorTab}\textbf{62.45} & \cellcolor{colorTab}\textbf{14.40} & \cellcolor{colorTab}\textbf{54.85} & \cellcolor{colorTab}\textbf{48.02} & \cellcolor{colorTab}\textbf{20.39} & \cellcolor{colorTab}\textbf{33.90} & \cellcolor{colorTab}\textbf{42.95} \\
    \midrule
    \multirow
          & FP16 &16 & 76.28  & 65.99  & 27.60  & 65.39  & 65.66  & 30.63  & 50.51  & 54.57  \\
          \cdashline{2-11}
    \addlinespace[0.2em]
          & GPTQ & 2.00  & 69.37 & 55.05 & 21.20 & 55.80 & 56.06 & 23.38 & 41.29 & 46.02\\
          & PB-LLM & 1.70  & 56.47 & 55.57 & 13.20 & 50.28 & 29.97 & 18.69 & 27.50 & 35.95\\
          \textbf{OPT-6.7B}& BiLLM &1.11 & 58.60 & 62.14 & 13.20 & 53.12 & 33.75 & 18.26 & 28.83 & 38.27 \\
          & ARB-LLM & 1.11 & 72.47 &62.87 & 22.20 & 60.62 & 59.09 & 26.79 & 42.08 & 49.45 \\
          \cdashline{2-11}
    \addlinespace[0.2em]
          & \cellcolor{colorTab}PBS$^2$P &\cellcolor{colorTab}0.80 & \cellcolor{colorTab}\textbf{73.39} & \cellcolor{colorTab}\textbf{62.48} & \cellcolor{colorTab}\textbf{24.40} & \cellcolor{colorTab}\textbf{61.88} & \cellcolor{colorTab}\textbf{62.04} & \cellcolor{colorTab}\textbf{26.11} & \cellcolor{colorTab}\textbf{44.07} & \cellcolor{colorTab}\textbf{50.62}  \\
          & \cellcolor{colorTab}PBS$^2$P &\cellcolor{colorTab}0.55 & \cellcolor{colorTab}\textbf{70.02} & \cellcolor{colorTab}\textbf{62.14} & \cellcolor{colorTab}\textbf{20.40} & \cellcolor{colorTab}\textbf{58.64} & \cellcolor{colorTab}\textbf{55.81} & \cellcolor{colorTab}\textbf{23.98} & \cellcolor{colorTab}\textbf{39.53} & \cellcolor{colorTab}\textbf{47.22} \\
    \midrule
    \multirow
          & FP16 &16 & 77.64 &70.43 &30.20  &68.19 &70.12 &34.56 & 54.30  & 57.92  \\
          \cdashline{2-11}
    \addlinespace[0.2em]
          & GPTQ & 2.00  & 73.88 & 63.94 & 24.20 & 62.19 & 60.77 & 28.24 & 47.88 & 51.59\\
          & PB-LLM & 1.70  & 66.76 & 62.29 & 17.40 & 51.07 & 49.33 & 22.53 & 36.53 & 43.70\\
          \textbf{OPT-30B}& BiLLM &1.11 & 72.74 & 62.35 & 21.00 & 60.14 & 60.69 & 27.56 & 42.81 & 49.61 \\
          & ARB-LLM & 1.11  & 75.08 & 65.78 & 26.40 & 65.43 & 64.81 & 29.69 & 48.59 & 53.68 \\
          \cdashline{2-11}
    \addlinespace[0.2em]
          & \cellcolor{colorTab}PBS$^2$P &\cellcolor{colorTab}0.80 & \cellcolor{colorTab}\textbf{76.44} & \cellcolor{colorTab}\textbf{64.07} & \cellcolor{colorTab}\textbf{26.40} & \cellcolor{colorTab}\textbf{66.30} & \cellcolor{colorTab}\textbf{67.38} & \cellcolor{colorTab}\textbf{32.34} & \cellcolor{colorTab}\textbf{49.68} & \cellcolor{colorTab}\textbf{54.66}  \\
          & \cellcolor{colorTab}PBS$^2$P &\cellcolor{colorTab}0.55 & \cellcolor{colorTab}\textbf{75.95} & \cellcolor{colorTab}\textbf{62.57} & \cellcolor{colorTab}\textbf{23.00} & \cellcolor{colorTab}\textbf{64.33} & \cellcolor{colorTab}\textbf{64.90} & \cellcolor{colorTab}\textbf{29.69} & \cellcolor{colorTab}\textbf{47.14} & \cellcolor{colorTab}\textbf{52.51} \\

    \bottomrule
    \end{tabular}%
}
\vspace{-3mm}
\label{tab:zero_shot_acc_opt}
\end{table*}

\begin{table*}[ht]
  \centering
    \caption{
  Accuracies (\%) for 7 zero-shot tasks from semi-structured binarized LLaMA families with PSB$^2$P.}
\vspace{1mm}
\resizebox{1.0\textwidth}{!}{
  \setlength{\tabcolsep}{5.5pt}
    \begin{tabular}{llccccccccc}
    \toprule
    \rowcolor{color3}
    \textbf{Models} & \textbf{Method} &\textbf{W-Bits}& \textbf{PIQA} $\uparrow$& \textbf{BoolQ} $\uparrow$& \textbf{OBQA} $\uparrow$& \textbf{Winogrande} $\uparrow$& \textbf{ARC-e} $\uparrow$& \textbf{ARC-c} $\uparrow$ & \textbf{Hellaswag} $\uparrow$ & \textbf{Average} $\uparrow$\\
    \midrule
    \multirow
          & FP16 &16 & 78.67  & 75.05  & 34.20  & 70.01  & 75.34  & 41.89  & 56.93  & 61.70  \\
          \cdashline{2-11}
    \addlinespace[0.2em]
          & BiLLM&1.11 & 61.53  & 60.12  & 13.60  & 56.83  & 38.47  & 21.42  & 31.68  & 40.50  \\
          \textbf{LLaMA-1-7B}& ARB-LLM &1.11 & 68.23  &69.17  & 21.60  & 62.43  & 53.66  & 25.68  & 38.96  & 48.50  \\
          \cdashline{2-11}
    \addlinespace[0.2em]
          & \cellcolor{colorTab}PBS$^2$P &\cellcolor{colorTab}0.80& \cellcolor{colorTab}\textbf{75.95} & \cellcolor{colorTab}\textbf{67.49} & \cellcolor{colorTab}\textbf{27.60} & \cellcolor{colorTab}\textbf{68.11} & \cellcolor{colorTab}\textbf{68.01} & \cellcolor{colorTab}\textbf{34.90} & \cellcolor{colorTab}\textbf{50.51} & \cellcolor{colorTab}\textbf{56.08}  \\
          & \cellcolor{colorTab}PBS$^2$P&\cellcolor{colorTab}0.55 &\cellcolor{colorTab}\textbf{70.95} & \cellcolor{colorTab}\textbf{65.54} & \cellcolor{colorTab}\textbf{23.00} & \cellcolor{colorTab}\textbf{64.96} & \cellcolor{colorTab}\textbf{60.69} & \cellcolor{colorTab}\textbf{28.75} & \cellcolor{colorTab}\textbf{43.78} & \cellcolor{colorTab}\textbf{51.10}  \\
    \midrule
    \multirow
          & FP16&16 & 78.07 &77.68  &31.40  &68.98  &76.30  &43.34  &57.16 &61.80  \\
          \cdashline{2-11}
    \addlinespace[0.2em]
          & BiLLM &1.11 & 59.90 & 54.37 & 16.20 & 53.04 & 41.92 & 20.73 & 30.37 & 39.50 \\
          \textbf{LLaMA-2-7B}& ARB-LLM &1.11 & 66.16 & 66.82 & 20.80 & 59.27 & 51.56 & 24.32 & 37.61 & 46.60 \\
          \cdashline{2-11}
    \addlinespace[0.2em]
          & \cellcolor{colorTab}PBS$^2$P &\cellcolor{colorTab}0.80 & \cellcolor{colorTab}\textbf{75.57} & \cellcolor{colorTab}\textbf{70.28} & \cellcolor{colorTab}\textbf{29.20} & \cellcolor{colorTab}\textbf{67.09} & \cellcolor{colorTab}\textbf{70.88} & \cellcolor{colorTab}\textbf{37.37} & \cellcolor{colorTab}\textbf{50.67} & \cellcolor{colorTab}\textbf{57.29}  \\
          & \cellcolor{colorTab}PBS$^2$P &\cellcolor{colorTab}0.55 & \cellcolor{colorTab}\textbf{70.78} & \cellcolor{colorTab}\textbf{63.73} & \cellcolor{colorTab}\textbf{23.80} & \cellcolor{colorTab}\textbf{65.59} & \cellcolor{colorTab}\textbf{62.37} & \cellcolor{colorTab}\textbf{31.48} & \cellcolor{colorTab}\textbf{43.90} & \cellcolor{colorTab}\textbf{51.66} \\
    \bottomrule
    \end{tabular}%
}
\vspace{-3mm}
\label{tab:zero_shot_acc_llama}
\end{table*}



\begin{table*}[t]
% \label{tab:ablation}
\vspace{-3mm}
\caption{Ablation study on LLaMA-2-7B, where all PBS$^2$P is applied an $N$:$M$ sparsity of 4:8. Results are measured by perplexity on the Wikitext2 and C4 datasets. Our results are highlighted in \textbf{bold}.}
\label{tab:ablations} 
\vspace{1mm}
\hspace{22mm}
% 子表格1
\subfloat[\small Ablation for SPBO Strategy. \label{tab:Progressive_Strategy}]{\hspace{-2mm}\vspace{-2mm}
\scalebox{0.9}
{\begin{tabular}{c c c}
\toprule
\rowcolor{color3}
\textbf{SPBO Strategy} & \textbf{Wikitext2$\downarrow$} & \textbf{C4$\downarrow$} \\
\midrule
\ding{55} & 13.14 & 15.60 \\
\ding{51} & \textbf{10.64} & \textbf{13.12} \\
\bottomrule
\end{tabular}}
}\hspace{10mm}\vspace{-2mm}
% 子表格2
\subfloat[\small Ablation for Group Size \label{tab:group_size}]
{\scalebox{0.9}
{\hspace{-1.5mm}
\begin{tabular}{c c c c c c}
\toprule
\rowcolor{color3}
\textbf{Group Size} & \textbf{64} & \textbf{128} & \textbf{256} &\textbf{512}\\
\midrule
\textbf{Wikitext2}$\downarrow$& 10.17  & \textbf{10.64} & 11.01 & 11.77  \\
\textbf{C4}$\downarrow$ & 12.29  & \textbf{13.12} & 14.11 & 14。76  \\
\bottomrule
\end{tabular}
}
}

\hspace{6.5mm}
% 子表格3
\subfloat[\small Ablation for Metric in Coarse-Stage Search (CSS)\label{tab:Coarse-Stage_Search}]{\hspace{-0mm}\vspace{-2mm}
\scalebox{0.85}
{\begin{tabular}{c c c c}
\toprule
\rowcolor{color3}
\textbf{Coarse-Stage Search} & \textbf{Metric} & \textbf{Wikitext2$\downarrow$} & \textbf{C4$\downarrow$} \\
\midrule
\ding{55} & — & 11.53 & 14.58 \\
\ding{51}  & RI & 14.85 & 18.39 \\
\ding{51} & LR & \textbf{10.64} & \textbf{13.12}\\
\bottomrule
\end{tabular}}
}\hspace{3mm}\vspace{-2mm}
% 子表格4
\subfloat[\small Ablation for Pruning Type\label{tab:Prune_Type}]
{\scalebox{0.85}
{\hspace{-1.5mm}\begin{tabular}{c c c c}
\toprule
\rowcolor{color3}
\textbf{Prune Type} & \textbf{Hardware-friendly} & \textbf{Wikitext2$\downarrow$} & \textbf{C4$\downarrow$} \\
\midrule
Structured & \ding{51} & 516.18 & 251.02 \\
Unstructured & \ding{55} & 8.72 & 11.29 \\
Semi-Structured & \ding{51} & \textbf{10.64} & \textbf{13.12} \\
\bottomrule
\end{tabular}}
}


\hspace{10mm}
\subfloat[\small Ablation for Metric in Fine-Stage Search\label{tab:Fine-Stage_Metric}]{\hspace{0mm}\vspace{-5mm}
%\resizebox{0.5\textwidth}{\height}
\scalebox{0.9}
{\begin{tabular}{c c c c c c}
\toprule
\rowcolor{color3}
\textbf{Metric} & \textbf{Random} & \textbf{Magnitude} & \textbf{Wanda} & \textbf{SI} & \textbf{Ours}\\
\midrule
\textbf{Wikitext2$\downarrow$} & 7,569.45 & 80.23 & 11.02 & 99.68 & \textbf{10.64} \\
\textbf{C4$\downarrow$} & 4,893.59 & 14.25 & 13.77 & 92.23 & \textbf{13.12} \\
\bottomrule
\end{tabular}}
}\hspace{5mm}\vspace{-0mm}
% 子表格6
\subfloat[\small Ablation for Number of Split Points\label{tab:split_points}]
{\scalebox{0.9}
{\hspace{-1.5mm}\begin{tabular}{c c c c}
\toprule
\rowcolor{color3}
\textbf{\#Split Points} &\textbf{1} & \textbf{2} & \textbf{3} \\
\midrule
\textbf{Wikitext2}$\downarrow$ & 13.23 & \textbf{10.64} & 10.18\\
\textbf{C4}$\downarrow$ & 15.91 & \textbf{13.12} & 12.77 \\
\bottomrule
\end{tabular}}
}
\hspace{3mm}

\label{tab:ablations}
\vspace{-8mm}
\end{table*}






\begin{figure*}[h]
% \label{fig:dialogue}
  \centering
  \vspace{-1mm}
  % \fbox{\rule{0pt}{2in} \rule{0.9\linewidth}{0pt}}
  % \fbox{\parbox[c][9cm]{\linewidth}{Abstract}}
   \includegraphics[width=1.0\textwidth]{figs/dialog.pdf}
   \vspace{-8mm}
   \caption{Conversation examples on LLaMA-13B (language supplementary) and LLaMA-2-13B(Q\&A). We compare our best method PBS$^2$P with BiLLM and ARB-LLM. \textcolor{red}{Inappropiate} and \textcolor{reasonable}{reasonable} responses are shown in corresponding colors.}
   \label{fig:dialog}
   \vspace{-5mm}
\end{figure*}
% \section{}

\end{document}


% This document was modified from the file originally made available by
% Pat Langley and Andrea Danyluk for ICML-2K. This version was created
% by Iain Murray in 2018, and modified by Alexandre Bouchard in
% 2019 and 2021 and by Csaba Szepesvari, Gang Niu and Sivan Sabato in 2022.
% Modified again in 2023 and 2024 by Sivan Sabato and Jonathan Scarlett.
% Previous contributors include Dan Roy, Lise Getoor and Tobias
% Scheffer, which was slightly modified from the 2010 version by
% Thorsten Joachims & Johannes Fuernkranz, slightly modified from the
% 2009 version by Kiri Wagstaff and Sam Roweis's 2008 version, which is
% slightly modified from Prasad Tadepalli's 2007 version which is a
% lightly changed version of the previous year's version by Andrew
% Moore, which was in turn edited from those of Kristian Kersting and
% Codrina Lauth. Alex Smola contributed to the algorithmic style files.


\end{document}