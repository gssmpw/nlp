\section{Related Work}
\label{sec:related-work}
The need for clear terminology when it comes to communication about ``risk'' and ``safety'' is not particular to the field of automated driving.
A core concern in the fields of risk research and risk communication \parencite{renn1998} is to establish well-grounded, ideally commonly understandable \parencite{sellnow2009}, terminology between different stakeholders, e.g., the general public, regulatory bodies, policymakers, industry, or public institutions such as non-governmental organizations (NGOs) \parencite{fischhoff1984,renn1998,christensen2003}.
In this context, \citeauthor{fischhoff1984} demand ``[\ldots] an explicit and accepted definition of the term `risk'[\ldots]'' \parencite[123]{fischhoff1984}, noting that the definition of ``risk'' is ``inherently controversial'' \parencite[124]{fischhoff1984}.

\citeauthor{christensen2003} note that deviating terminology between stakeholders can derail discussions from their ``core issue(s)'' \parencite[182]{christensen2003}.
They specifically include terminology that is related to ``identifying, estimating, regulating, and communicating risk'' \parencite[182]{christensen2003}, hence including all the above-mentioned stakeholders in their argument.
The authors analyze several references from different regulatory bodies and NGOs such as the European Commission, the UN/OECD, the US-EPA, or ISO/IEC (specifically ISO/IEC Guide 51 \parencite{iso51}).
\citeauthor{christensen2003} \parencite{christensen2003} discuss, explain, and clarify applications of terms and views related to risk associated sciences to facilitate communication between stakeholder groups.
By not providing a fully consolidated terminology, they acknowledge that, while there must be a fundamental consensus among stakeholders, communication always has to be tailored to the communicating parties.
In the following sections of the paper, we will mainly consider regulators communicating to industry who is implementing automated driving technology.

The need for clear and consistent communication about risks related to autonomous systems in general is highlighted in \parencite{wmg2023}.
The report emphasizes that it is crucial to identify \emph{who} communicates \emph{how} about \emph{what} related to the communication about safety-critical autonomous systems.
As \parencite{christensen2003}, the authors explicitly stress that messages related to safety assurance and their content should be tailored to the relevant audiences.
According to \parencite{wmg2023}, this is particularly important for calibrating the expectations of different stakeholders and for raising awareness for the limitations of autonomous systems' capabilities.

Different concepts of ``risk'' and ``safety'' for automated driving systems have, e.g., been discussed in \parencite{koopman2024} or \parencite{salem2024}.
\citeauthor{koopman2024} \parencite{koopman2024} review automotive safety standards (ISO~26262:2018 \parencite{iso2018}, ISO~21448:2022 \parencite{iso21448}, ANSI/UL4600 \parencite{ul4600}) and additional resources from the German Ethics Commission \parencite{difabio2017} to the US National Highway Traffic Safety Administration for their conceptualizations of ``safety''.
They provide additional examples of ``safety problems'' related to automated driving systems which are not covered by the purely technical definitions assumed in ISO~26262 and ISO~21448.
The authors discuss that risk and safety for automated vehicles should be discussed in a more nuanced way than only considering technical definitions grounded in ``net (physical or monetary) harm''.
However, the discussion related to ``risk'' falls slightly short, not acknowledging that existing risk definitions (e.g., \parencite{fischhoff1984,renn1998}) already allow for broader discussions beyond ``net harm''.
While the newly proposed risk definition does include the influence of ``importance for stakeholders'', which allows a more interdisciplinary view on risk, the risk definition is slightly derailed by the introduction of the additional term ``loss''.
``Loss'' is defined in a slightly broader sense, compared to harm in ISO Guide~51 (see below), including ``negative societal externalities'' and ``injury and death of animals''.
While this extends technical notions of risk, it mainly shifts complexity in the terminology, addressing rather the effects of inflicted harm than the harm (violated stakeholder needs and values) itself.

In previous publications (\parencite{salem2024, nolte2024}), we discuss possible conceptualizations of risk and safety, considering a broader view of defining ``safety'' and ``risk'', particularly addressing ethical questions and stakeholder values.
While we provide a similar assessment of technical standards as \citeauthor{koopman2024}, we include general safety and risk management standards such as IEC~61508 \parencite{iec61508}, ISO~31000 \parencite{iso31000}, or ISO Guide 51 \parencite{iso51} to avoid a narrow technical focus.
Particularly, we consider ethical questions for defining ``risk'' and ``safety'' by relating harm to stakeholder values\footnote{When compared to \citeauthor{koopman2024}, harmed stakeholder values are a potential root cause for negative societal externalities that express non-acceptance of the technology.}.

Considering this body of related work, several distinctions can be made when assessing definitions for ``safety'' and ``risk'':
Notions of ``safety'' and ``risk'' can be separated by underlying ``risk sources''\footnote{E.g., E/E-failures as per ISO~26262 or functional insufficiencies as per ISO~21448.} (as per \parencite{christensen2003}; we'll argue in \cref{sec:analysis}, why we prefer the term ``hazard sources'') or the considered types of ``harm''\footnote{Such as physical or monetary harm.}
In the following, we will give definitions and perspectives on ``safety'' and ``risk'' that will be used for assessing the terminology provided in the regulatory documents.