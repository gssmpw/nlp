\section{Conclusion \& Recommendations}
\label{sec:conclusion}
This paper reviewed UN, EU, German, and UK regulation for the type approval of SAE-Level-$\geq$3 automated driving systems regarding the underlying notions of safety and risk.
Our analysis has shown that the considered acts vary significantly in the consistency of the used concepts.
The UK AV Act appeared as the regulation with the most consistent, yet generic, conceptualization of safety and risk.
The UN and EU regulations adopt a very technical notion of safety and risk, based on the absence of unreasonable risk, while mostly focusing on harm to the life and limb of humans.
Compared to the other acts, the German AFGBV provides ambiguous notions of risk and safety.
These notions range from the assumption of a \emph{complete} absence of risk, to more technical notions considering the absence of \emph{unreasonable} risk.

In summary, UN and EU regulations as well as the UK AV Act provide frameworks that permit traditional technical interpretations of risk management.
Moreover, the UK AV Act could provide a legal framework that allows adopting more holistic perspectives on safety, as discussed by \parencite{salem2024} or \parencite{koopman2024}.

The German StVG and AFGBV introduce significant legal uncertainty for designers and developers of automated vehicles due to ambiguous or even contradictory wording and requirements.
This creates additional effort for engineers to validate assumptions regarding the requirements with the regulators.
However, as the regulatory statutes stand as they are, even such clarification can entail the need for court cases, before legal uncertainty is settled.
This can unnecessarily impede the market introduction of automated vehicles, where clear regulation could have helped from the beginning.

Finally, specifically as the German acts are subject to regular evaluation and adaption, we recommend regulators
\begin{enumerate}
    \item[a)] to issue clarifying regulation to fix contradictions and ambiguity, particularly regarding open signifiers such as ``safety'' and ``risk'',
    \item[b)] to consider how a broader notion of risk connected to different harm types can force developers to consider negative impacts such as ``inconvenience for the public'' (see \parencite{koopman2024} for more examples), and
    \item[c)] to find wordings that encourage designers and developers to diligently document the rationale for development decisions that are involved in the required tradeoffs between stakeholder needs and values. 
\end{enumerate}

The UK AV Act could serve as an interesting example in this regard, despite only providing enabling regulation and despite the lack of explicit definitions for risk and safety.









