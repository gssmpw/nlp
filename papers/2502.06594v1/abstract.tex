``Safety'' and ``Risk'' are key concepts for the design and development of automated vehicles.
For the market introduction or large-scale field tests, both concepts are not only relevant for engineers developing the vehicles, but for all stakeholders (e.g., regulators, lawyers, or the general public) who have stakes in the technology.
In the communication between stakeholder groups, common notions of these abstract concepts are key for efficient communication and setting mutual expectations.

In the European market, automated vehicles require Europe-wide type approval or at least operating permits in the individual states.
For this, a central means of communication between regulators and engineers are regulatory documents.
Flawed terminology regarding the safety expectations for automated vehicles can unnecessarily complicate relations between regulators and manufacturers and thus hinder the introduction of the technology.
In this paper, we review relevant documents at the UN- and EU-level, for the UK, and Germany regarding their (implied) notions of safety and risk.
We contrast the regulatory notions with established and more recently developing notions of safety and risk in the field of automated driving.
Based on the analysis, we provide recommendations on how explicit definitions of safety and risk in regulatory documents can support rather than hinder the market introduction of automated vehicles.