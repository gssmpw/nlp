\section{Introduction}

Visual blending is a powerful graphic design and communication technique, offering a creative means to convey novel and groundbreaking ideas~\cite{Cunha2020LetsFT, 10.1145/3290605.3300402}.
Visual blends strategically combine two distinct objects into a unified composition, achieving a harmonious effect through unique and compelling designs~\cite{10.1145/3290605.3300402, 10.1007/s00354-020-00107-x, 10.1145/3411764.3445089}.
For instance, blending a dumbbell with an orange might symbolize the relationship between fitness and nutrition, highlighting how physical exercise complements proper nutritional intake in maintaining good health (refer to Figure~\ref{fig:teaser} on the left).
Such representations are ubiquitous in our daily lives, communicating symbolic messages from simple graphic design to intricate visual arts~\cite{10.1086/209396, benczes2009visual}.
The widespread use of visual blends stems from their capacity to engage audiences and effectively communicate complex or abstract concepts in a memorable and impactful manner.

While visual blends excel at grabbing audience attention, creating good ones presents a non-trivial design challenge.
Key design considerations include selecting appropriate objects, determining suitable blending attributes, and ensuring harmonious integration of blended elements.
A major conceptual challenge lies in identifying symbolic objects that effectively convey the intended message while preserving their individual recognizability~\cite{10.1145/3290605.3300402}.
From a technical standpoint, creating visually appealing and coherent blends requires meticulous attention to blending techniques and adherence to established design principles, making the process both skill-intensive and time-consuming.
To convey abstract concepts or experiences, visual blends often employ metaphors~\cite{MetaCLUE_10204033,grady1999blending,10.1145/3290605.3300402}, necessitating the selection of suitable source-domain objects that are both semantically relevant and visually compatible.





In the realm of Artificial Intelligence Generated Content (AIGC), the generation of images is a rapidly evolving field. 
Existing text-to-image (T2I) generation techniques, while capable of producing images from textual input, also provide an alternative tool for accelerating and simplifying the process of creating visual blends.
% have been developed to produce images according to the given textual input.
However, current T2I models encounter challenges when handling abstract descriptions, as they often lead to images with distorted or nonsensical textual elements~\cite{Liao_Chen_Fu_Du_He_Wang_Han_Zhang_2024}.
This limitation highlights the need for a deeper understanding of human cognition in transforming abstract content into visual elements in images.
One intriguing aspect of addressing this problem involves incorporating Conceptual Metaphor Theory (CMT) into the image-generation process~\cite{MetaCLUE_10204033,lakoff2008metaphors}.
By grounding abstract concepts or experiences in concrete objects and their interrelations, CMT posits that metaphor, beyond being a rhetorical device, offers a way to transform disparate concepts into more relatable and comprehensible information for users.
However, current research has not yet fully explored how to represent abstract concepts in image generation, particularly when employing metaphors to combine multiple objects and their relationships.




Our research pioneers the intersection of visual metaphors and image generation, offering a unique lens for understanding the cross-modal relationships underpinning creative ideation in visual blends.
% introducing a novel framework for systematic visual blend ideation.
We explore the potential of AIGC techniques for producing visually blended content and develop a system for semantically comparing the resulting design options.
Our key innovation integrates metaphors with commonsense reasoning at both object and attribute levels, enabling the creation of visually diverse and semantically rich blends that resonate with human metaphorical cognition.
By leveraging diverse visual-textual correspondences, we aim to unlock new creative possibilities and generate blended visuals that are not confined by predefined visual paradigms (e.g., specific shapes and styles) or literal representations. 
Inspired by Lakoff and Johnson's foundational work on metaphors~\cite{lakoff2008metaphors}, we consider language a proxy for connecting abstract ideas with visual elements, establishing a robust foundation for conceptual understanding and creative expression~\cite{10.1145/3106625}.




In line with our research, we introduce \sysname, an AI-powered system that assists users in generating visual blend ideas by incorporating metaphors derived from user input.
Informed by interviews with eight design practitioners, we identified their needs and obstacles when creating visual blends with concrete objects, especially in drafting design options for abstract concepts.
Here, ``concepts'' are keywords extracted from user-provided expressions, while ``objects'' are tangible visual elements representing these ideas.
Our method leverages metaphors and image generation techniques, complemented by large language models (LLMs) and commonsense knowledge bases, to enrich the diversity and metaphoricity of design outcomes.
A within-subject study with 24 participants compared \sysname{} to a baseline that combined ChatGPT (built with GPT-3.5 integrated with DALL·E 3) and Google Search.
Results showed that \sysname{} significantly enhances creativity, metaphoricity, and user experience in generating visually compelling ideas for abstract concepts.
Participants appreciated its ability to generate diverse outputs and inspire innovative ideas through metaphorical blends.
This research underscores the value of specialized AI tools in unlocking the creative potential of generative models, bridging gaps in user understanding of AI capabilities, and expanding the possibilities of metaphor-informed visual design.
The main contributions of this work are three-fold:
\begin{itemize}[leftmargin=*]
    \item We propose a metaphor-inspired approach to visual blending that identifies conceptually relevant objects through commonsense reasoning and combines them based on attribute similarity; 
    \item We introduce \sysname, an AI-assisted creativity support system that transforms user-provided textual input into visually blended concepts, facilitating the exploration of diverse visual-textual correspondences;
    \item Through a comprehensive evaluation, we provide insights into designing abstract concepts using metaphors within the context of AIGC, showcasing the system’s potential to advance creative workflows and expand the boundaries of visual ideation.
\end{itemize}



