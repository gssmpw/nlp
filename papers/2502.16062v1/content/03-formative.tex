\section{Formative Study}

% In order to understand the process and pain points of users utilizing metaphors for visual creation, we invited eight participants with different design backgrounds to participate in semi-structured interviews. Both amateur and professional designers are included, differentiated by whether or not they are engaged in a full-time design-related job. And they all have experience using visual metaphors in their design work. Semi-structured interviews were designed to have a direct conversation with the designers and to investigate their needs when designing visually in real-life scenarios. 
% Moreover, we incorporated a user research methodology called ``Think Aloud'' into our interviews. In this process, participants were invited to report their thoughts, steps, and feelings in detail. Although the design process of the participants was not strictly controlled, this method can help to understand the real and detailed conceptualization process as well as the execution process of the users in making visual designs.



To understand how designers create visual blends and uncover potential challenges, we conducted in-depth interviews with eight practitioners, four from advertising and four from graphic design.
% backgrounds \zy{(e.g. advertising design, architectural design, clothing design, etc.)}.
The participants included both amateur and professional designers, distinguished by whether they were engaged in full-time design-related work.
We employed a semi-structured interview format to engage in open-ended discussions about a specific design task. 
Additionally, we used the think-aloud protocol, encouraging participants to verbalize their thoughts, concerns, and plans in detail throughout the design process to gain deeper insights.


% All participants had prior experience incorporating visual metaphors into their design projects.

% To gain insights into how users employ metaphors for visual creation and to identify their challenges, we conducted semi-structured interviews with eight participants who had diverse design backgrounds. 
% The participants consisted of both amateur and professional designers, with the distinction being whether or not they were engaged in full-time design-related work. 
% All participants had prior experience incorporating visual metaphors into their design projects.
% The semi-structured interviews were designed to facilitate direct conversations with the designers, enabling us to explore their requirements when it comes to visually designing in real-life scenarios. Additionally, we implemented the think-aloud protocol, which encouraged participants to articulate their thoughts, steps, and feelings in detail throughout the design process. This approach enabled us to obtain a comprehensive understanding of the participants' authentic and intricate conceptualization and execution processes when creating visual designs.

% Additionally, we incorporated the think-aloud protocol to encourage participants to vocalize their thoughts, steps, and emotions in detail throughout the design process, which allowed us to gain a comprehensive understanding of the users' actual and intricate conceptualization and execution processes in creating visual designs.

% \zs{The semi-structured interviews facilitated direct conversations with the designers, allowing us to explore their needs under a specific design task.







\subsection{Interview Setup and Process}


We recruited eight participants (5 female, 3 male) for the study, with ages ranging from 21 to 31 years ($Mean = 24, SD = 4.04$). 
All participants had prior experience incorporating visual metaphors into their design work and held formal education in visual design. Three participants were full-time design professionals, while the remaining five were design students who integrated visual metaphors into their personal projects.
During the study, participants first shared their previous experiences designing visual metaphors and were tasked with creating visual blends focused on a specific topic (i.e., ``\textit{Exercise fuels your body like vitamins}'', a theme chosen for its relatability and clarity). 
% \zy{The topic was inspired by the work of Chilton et al. ~\cite{10.1145/3290605.3300402} and was chosen for its closeness to life and representativeness, making it easy for interviewers to understand and accept.}
We emphasized the design process over the final product, requesting participants to detail the evolution of their ideas.
Interview questions covered the major design activities, exploring how participants conceptualized their ideas, identified sources of inspiration, and transformed concepts into prototypes. 
Toward the end of the interviews, participants were asked to reflect on the most challenging aspects of the design process and to articulate their specific needs for AI tools that could streamline their workflow and enhance their creative output in the context of visual blend design.


% All had prior experience incorporating visual metaphors into their design projects and had received education in the visual design field.
% Among them, three were full-time professional designers, while the rest were amateur designers who incorporated visual metaphors into their projects.
% During the study, participants first shared previous experience when working with visual metaphors and were tasked with designing visual blends centered around a specific topic (i.e., ``\textit{Exercise fuels your body like vitamins}'').
% Participants were required to describe how their ideas evolved, with a focus on the design process rather than the final outcome.
% \zs{Interview questions covered the major design activities, including concept development, sources of inspiration, and the prototyping and implementation of their ideas. 
% Toward the end, participants identified the most difficult and time-consuming parts of the design process and shared their requirements for AI-assisted tools to support visual blend design.}

% The whole interview lasted
% \sz{@Yue, please complete the time of the study.}
%  XX minutes.

% Additionally, they were tasked with creating a live design based on a given design topic (``Exercise fuels your body like vitamins''). Whether discussing their previous work or the on-site design activity, participants were required to describe their conceptualization process and the execution of their ideas. It is important to note that the on-site design activity focused on describing the creative process rather than producing a final design outcome. By examining the detailed creative process, we aimed to gain a better understanding of the participant's needs and the challenges they encountered.
% The interview questions addressed various aspects, including participants' design backgrounds, the development of their original concepts, sources of inspiration, and the prototyping and implementation of their ideas. Toward the end of the interview, participants were asked to identify the most difficult and time-consuming parts of the entire design process. They were also given the opportunity to express their requirements for AI-assisted tools that could support them in designing visual metaphors.


% We recruited a total of eight participants (5 female, 3 male; age ranged 21-31, $Mean = 24, SD = 4.04$) with at least some prior experience in visual metaphor design. 
% All participants had previous education related to visual design. Three of them are currently full-time professional designers. There were also a number of amateur designers who used visual metaphors in their own design work. 
% Participants were asked not only to give examples from their previous design work that contained design metaphors but also to create a live design based on a design topic (``Exercise fuels your body like vitamins'') given in the field. Whether it was about their previous design work or about their work on site, they needed to describe their conceptualization process and the execution of their ideas. However, it is worth noting that site design is only required to describe the creative process, not to produce a design outcome. 
% By studying the detailed creative process, we can better understand the needs and difficulties encountered by the participants. Interview questions focus on their design background, how they developed their original concepts, where they found inspiration, and how their ideas were prototyped and put into practice. At the end of the interview, they are asked to point out the most difficult and time-consuming parts of the whole design process and to present their needs for AI-assisted tools that support them in designing visual metaphors.

\subsection{Findings}

The feedback from participants revealed that they followed three main iterative processes when designing visual blends: ideation, gathering materials, and implementation. Among them, the ideation stage was cited by most participants (6 out of 8) as the most challenging and time-consuming phase. 
One participant (E5, Female, 21) emphasized the time-consuming nature of ideation, stating, ``\textit{...ideation is a major bottleneck. I need to brainstorm broadly to find the best solution, but this process is limited by my personal knowledge and requires constant iteration}''. 
% \att{I need to brainstorm broadly to find the best solution, but this process is limited by my personal knowledge and requires constant iteration.}
According to her, the primary difficulties stemmed from selecting appropriate visual elements, identifying their commonalities, and determining how to distribute and arrange them effectively.
To better understand the ideation process of creating visual blends and address the challenges and needs voiced by participants, we further subdivided ideation into three sub-steps: idea generation, concept visualization, and visual integration.

% According to participant feedback, there were three main iterative processes involved in designing visual blends: ideation, gathering materials, and implementation. The ideation stage was highlighted by most participants (6 out of 8) as the most challenging and time-consuming phase.
% One participant (E5, Female, Age 21) stressed how time-consuming the ideation process was, stating "Ideation is a major bottleneck. You need to brainstorm widely to find the best solution, but it's limited by personal knowledge and requires constant iteration." She indicated the primary difficulties stemmed from identifying commonalities between visual elements and determining how to effectively arrange and distribute them.
% To gain deeper insights into the ideation process and better address the challenges and needs voiced by participants, we further subdivided ideation into three sub-steps: generating initial ideas, developing conceptual directions, and visually integrating concepts.

% Based on the feedback provided by the participants, it was evident that all interviewees followed three main processes in an iterative manner: ideation, creation or collection of elements, and implementation. The ideation stage was identified by almost all participants (6 out of 8) as both the most challenging and time-consuming phase. One participant (E5, Female, Age = 21) highlighted the time-consuming nature of conceptualization, stating, ``Conceptualization is super time-consuming.'' According to her, the difficulties primarily arose from the task of finding commonalities among the elements and determining how to distribute and arrange them effectively.
% To gain a deeper understanding of the ideation process and address the participants' challenges and requirements, we further subdivided it into four sub-steps: idea acquisition, materialization of abstract concepts, material collection, and blending.


% From the participants' feedback, we found that all interviewees would go through three main processes in an iterative manner: ideation, creation or collection of elements, and implementation. Almost all participants (6/8) identified the ideation stage as the most difficult as well as the most time-consuming stage. As one participant said, ``Conceptualization is super time-consuming'' (U5, F, Age = 21). Because, in her opinion, ``It is particularly difficult to find commonalities between the elements and to think about the distribution of the arrangement of the elements'' (U5, F, Age = 21). We further break down the ideation process and address their difficulties and needs in four sub-steps: idea acquisition, materialization of abstract concepts, material collection, and blending. 


% \subsubsection{Idea Acquisition}


\subsubsection{Idea Generation}

Participants primarily concentrated on identifying key terms, conceptualizing the desired visual message, and choosing figurative objects that can effectively represent the expression.
For all participants, keyword extraction emerged as the universal initial step when presented with a design topic.
After that, they attempted to identify appropriate expansions or elaborations of these keywords.
% \att{After that, they attempted to identify appropriate expansions or elaborations of these keywords.}
However, several participants, including professional designers, highlighted the challenges in generating innovative design concepts, especially when dealing with ambiguous or abstract ones.
One participant (E1, Female, 31) emphasized the difficulty of initiating the creative process even for experienced professionals and reflected that ``\textit{...initial concepts immediately flashed in my mind, but they often default to clichés, lacking novelty and engagement}''.
Another participant (E4, Female, 28) echoed this sentiment and expressed a specific need for AI assistance in overcoming this hurdle. 
She envisioned an AI tool capable of generating a wide range of conceptual options as a starting point for further refinement.

\subsubsection{Concept Visualization}


Participants reported significant challenges in transforming abstract concepts into appropriate concepts related to visual representation.
Identifying concrete representations for these abstract ideas was particularly time-consuming and labor-intensive, as one participant (E3, Female, 27) highlighted.
This challenge intensified when design topics involved multiple abstract concepts, requiring considerable effort to visually translate and integrate them cohesively.
% \att{This challenge intensified when design topics involved multiple abstract concepts, requiring significant effort to visually translate and integrate them harmoniously.}
To address this, participants expressed the need for tools or methods that could systematically examine the input abstract concepts from diverse perspectives and viewpoints.
Our observations revealed that participants assessed how connected the selected concepts were when designing.
Therefore, we prioritize the development of features that facilitate semantic relevance analysis for designers.
% \att{Therefore, we prioritize developing features that facilitate semantic relevance analysis for designers.}
This would enable a more comprehensive exploration and understanding of the concepts, better equipping the designer to create meaningful visual representations that capture the essence of the original abstract expressions.




% Two primary factors emerged as crucial considerations: semantic relevance and visual structural similarity.

% participants consistently encountered difficulties in translating abstract concepts into appropriate concepts related to visual representation. 
% Identifying concrete representations for these abstract ideas was particularly time-consuming, as one participant (E3, Female, Age 27) highlighted. 
% This challenge intensified when design topics incorporated multiple abstract concepts, demanding considerable effort to visually translate and integrate them harmoniously. 
% The complexity stemmed from the need to accurately portray each abstract concept while maintaining a cohesive overall visual composition. 
% To address this challenge, a participant suggested the need for a tool or method capable of analyzing abstract concepts from various angles, facilitating a deeper understanding and enabling the creation of more effective visual metaphors. 

% When dealing with descriptions containing abstract expressions, users mentioned that finding relevant visual metaphors is an extremely challenging task. 
% One participant (E3, Female, Age 27) noted that the time-consuming aspect lies in identifying concrete objects that can encapsulate abstract concepts. This becomes especially difficult when a design topic involves multiple abstract concepts, requiring significant effort to translate them visually in a cohesive manner. 
% The complexity arises from the need to effectively represent each abstract concept while integrating them into a unified whole.
% Another participant (E6, Male, Age 21) expressed the need for an approach or tool that could analyze the input abstract concepts from diverse perspectives. 
% This would enable a more comprehensive exploration and understanding of the concepts, better equipping the designer to create meaningful visual representations.

% Participants reported significant challenges in transforming abstract concepts into visual metaphors. 
% The process of identifying tangible representations for abstract ideas was particularly time-consuming, as noted by one participant. 
% This challenge was amplified when design topics encompassed multiple abstract concepts, necessitating careful consideration of visual translation and integration. 
% Effectively representing each abstract concept within a cohesive visual composition presented a complex task. 
% To address this, participants suggested the need for tools or methods capable of analyzing abstract concepts from various viewpoints, thereby facilitating a deeper understanding and enabling the creation of more meaningful visual representations.

% When working with descriptions that contained abstract expressions or concepts, users indicated that finding relevant visual metaphors to represent those abstractions was an extremely challenging undertaking. 
% One participant, E3, a 27-year-old female, pointed out that the most time-consuming aspect was identifying concrete, tangible objects or visuals that could effectively encapsulate and symbolize the intended abstract concepts. 
% She noted that this process becomes especially difficult and laborious when the design topic or goal involves multiple abstract concepts that need to be visually translated. 
% In such cases, significant effort is required to find a cohesive way to represent each individual abstract notion visually while also integrating them together into a unified, coherent whole.
% Another user, E6, a 21-year-old male, expressed the need for an approach, methodology or tool that could systematically analyze and examine the inputted abstract concepts from diverse perspectives and viewpoints. 
% Having such a multi-faceted analysis capability would enable designers to more comprehensively explore and develop a deeper understanding of the abstract concepts they are trying to visualize. 
% This enhanced conceptual understanding would better equip them to ultimately create visual representations that are meaningful and successfully capture the essence of the original abstract ideas.


% In this step, the designer is tasked with choosing figurative objects that can effectively represent the abstract concepts acquired in the previous sub-stage. 
% One participant pointed out that the time-consuming aspect lies in finding concrete objects that can encapsulate abstract concepts (E3, Female, Age = 27). This becomes particularly challenging when a design topic involves multiple abstract concepts, requiring a considerable workload on how to translate them visually.
% This complexity arises from the need to represent each abstract concept effectively in a cohesive manner.
% Another participant (U6, Male, Age = 21) expressed the need for an approach or tool that could analyze the abstract concepts entered from diverse perspectives. 
% This would enable a more comprehensive exploration and understanding of the concepts, aiding the designer in creating meaningful visual representations.


% When encountering descriptions containing abstract expressions, users also mentioned that finding relevant visual metaphors is a significantly challenging task.
% One participant pointed out that the time-consuming aspect lies in finding concrete objects that can encapsulate abstract concepts (E3, Female, Age = 27). This becomes particularly challenging when a design topic involves multiple abstract concepts, requiring a considerable workload on how to translate them visually.
% This complexity arises from the need to represent each abstract concept effectively in a cohesive manner.
% Another participant (U6, Male, Age = 21) expressed the need for an approach or tool that could analyze the abstract concepts entered from diverse perspectives. 
% This would enable a more comprehensive exploration and understanding of the concepts, aiding the designer in creating meaningful visual representations.


% In this stage, designers must translate abstract concepts into concrete visual representations. 
% Finding suitable visual metaphors for these abstract ideas is a time-consuming challenge, especially when multiple concepts are involved. 
% Participants expressed the need for tools that can analyze abstract concepts from various perspectives, aiding in the discovery of meaningful visual correspondences.

% focused on extracting keywords and contemplating the overall concept. 
% When presented with a design topic, the initial step taken by all participants was to extract keywords. 
% However, a professional designer (E1, Female, Age 31) among the participants emphasized the challenge of initiating a design idea, particularly when it needs to be interesting. She stated, ``Even for professional designers, it can be very difficult to get started with a design idea, especially the one with diverse meanings or abstract descriptions''. 
% Another professional designer  shared the same difficulty. It is worth noting that her requirement for an AI-assisted tool is relevant in this context. She expressed a desire for a tool that could provide numerous ideas, which she could then refine.

% In the initial phase of the design process, participants primarily concentrated on identifying key terms and developing a comprehensive conceptual framework. Across all participants, keyword extraction emerged as the universal starting point when presented with a design prompt.
% However, several participants, including professional designers, highlighted the significant challenge of generating innovative design concepts, especially when dealing with ambiguous or abstract topics. 
% One designer emphasized the difficulty of initiating the creative process even for experienced professionals. 

% This stage is for extracting keywords and thinking about the concept. The first thing all participants did when given a design topic was to extract keywords. But a participant who is a professional designer noted that ``Even for professional designers, it can be very difficult to get started with a design idea, especially one that is interesting'' (U1, F, Age = 31). Another professional designer expressed the same difficulty (U4, F, Age = 28). Her need for an AI-assisted tool is also relevant. She wanted the tool to give her lots of ideas on the basis of which she could refine them (U4, F, Age = 28).


% \subsubsection{Materialization of Abstract Concepts}

% This stage requires the designer to select figurative objects to represent the abstract concepts acquired in the previous sub-stage. A participant mentioned that finding concrete objects that can embody abstract concepts is the time-consuming part (U3, F, Age = 27). Especially when a design topic contains multiple abstract concepts, she spends a lot of time thinking about how to translate the abstract concepts. Another participant said his ideal AI-assisted tool would be one that could be analyzed from multiple perspectives based on the abstract concepts entered (U6, M, Age = 21).

% \subsubsection{Material Collection}


Furthermore, participants universally recognized the importance of this step in bringing their ideas to fruition.
At this step, designers are required to move beyond textual descriptions and manifest concepts visually.
Two participants (E2, Female, 21; E6, Male, 21) mentioned that they would boost their creativity by taking reference for existing design solutions, as E6 reflected ``\textit{...benchmarking against existing work sparks innovation and allows for iterative refinement}''.
The visual output of a designer's ideas relies on pre-defined concepts and is also a means of optimizing their design.
However, designers encountered instances where suitable presentation materials for their envisioned design were unavailable, thereby restricting their creative potential.
To overcome this obstacle, participants expressed a need for rapid access to a diverse range of reference images.


% a tool capable of rapidly presenting the reference images.

% during the study, participants also mentioned the power of AIGC which could mitigate the gap of missing design material

% A visual representation of the designers’ ideas relies on pre-established concepts while also serving as a means for enhancing their design solutions.
% The selection of materials depended on the pre-established set of concepts, while simultaneously serving as a means for designers to refine their choices. 

% Another participant (E3, Female, Age 27) also pointed out the difficulty in accumulating all the desired material at once.

% They commonly collect a variety of materials to inspire creative exploration.
% The materials gathered served as both a resource pool for design elements and a reference point for existing design solutions. 

% emphasized the importance of collecting a substantial amount of material, as it provided them with a wider range of options during the design process.


% Such materials can serve as inspirations for designers from visual aspects

% On the one hand, these materials are determined by the concept range selected in the previous step, and on the other hand, these materials can also help users optimize their choices. However, users find that sometimes there is no suitable presentation material for the picture in their mind, which limits the possibility of users' creation to a certain extent.

% Another participant (E3, Female, Age 27) also pointed out the difficulty in accumulating all the desired material at once.
% To overcome this obstacle, participants expressed a need for a tool capable of rapidly collecting or generating a large volume of reference images.


% All participants universally acknowledged this step as an essential step in bringing their ideas to life. 
% Half of the participants (4 out of 8) identified it as the most time-consuming phase. 
% The materials gathered during this stage encompass both the resources participants will utilize in their designs and examples of existing designs for reference.
% Two participants (E2, Female, Age 21; E6, Male, Age 21) emphasized the significance of collecting a substantial amount of material, as it provided them with a wider range of options during the design process. 
% However, despite its importance, one participant (E3, Female, Age 27) expressed difficulty in accumulating all the desired material at once. 
% Consequently, she expressed a need for an assistive tool that could rapidly generate a large number of reference images, as it would greatly assist her in overcoming this challenging obstacle.

% All participants unanimously agreed on the critical importance of material collection in transforming their ideas into tangible designs. This phase was particularly time-consuming for half of the participants. 
% The materials gathered served as both a resource pool for design elements and a reference point for existing design solutions. 
% Several participants highlighted the value of extensive material collection in providing a broad spectrum of design options. 
% However, one participant encountered challenges in efficiently accumulating a comprehensive collection of materials. 
% To overcome this obstacle, she expressed a need for a tool capable of rapidly generating a large volume of reference images.

% Every participant unanimously recognized the material collection step as a crucial phase in bringing their conceptual ideas to fruition. Half of the participants (4 out of 8) identified this as the most time-consuming stage of the process. 
% The materials gathered during this step include both the resources and assets the participants intended to directly utilize in their final designs, as well as existing design examples to use as references.
% Two participants (E2, Female, Age 21; E6, Male, Age 21) stressed the importance of amassing a substantial collection of materials, as having a wider array of options available aided them during the design process. 
% However, one participant (E3, Female, Age 27) voiced difficulties in accumulating all the desired materials at once.
% As a result, she expressed a need for an assistive tool capable of rapidly generating a large number of reference images, as this would greatly help overcome that particular challenge.

% This stage was recognized by all participants as a necessary step in getting the ideas in their heads off the ground. It was also the step considered most time-consuming by the largest number of participants (4/8). The materials include both the materials they will use in their designs and examples of other people's designs for reference. As two participants noted, ``It's only by collecting a lot of material that I have more options in the design process'' (U2, F, Age = 21; U6, M, Age = 21). Despite the fact that material collection is so important, ``It's hard to collect all the material I want at once'' (U3, F, Age = 27). Therefore, she expressed that an assistive tool that can quickly generate lots of reference photos would help her out with a very difficult problem (U3, F, Age = 27).

% \subsubsection{Blending}


\begin{figure*}[t]
  \centering
  \includegraphics[width=\linewidth]{figure/pipeline-hor.pdf}
  \caption{\sysname\ operates through a multi-stage pipeline.
  % follows a pipeline comprising several key stages. 
  The initial stage involves concept inference to identify relevant objects and their attributes. Subsequently, a similarity-based selection process empowers users to choose suitable object and attribute combinations.
  % a similarity calculation aids users in selecting suitable object and attribute combinations. 
  The system then explores potential blending schemes and synthesizes corresponding prompts for the T2I model, culminating in iterative image generation based on the selected prompts to support the ideation process.}
  % Finally, the system generates an image for ideation reference.}
  % \Description{The research pipeline.}
     \label{fig:pipeline}
\end{figure*}



\subsubsection{Visual Integration}

This step involves the merging of two objects to create a cohesive and creative design. 
Participants identified the primary challenge, which is determining the optimal points of connection between the two objects. 
One participant (E8, Male, 22) commented, ``\textit{Joining two things aesthetically isn't random; it needs a thoughtful and organized process.}''.
The complexity arises from the need to effectively represent individual concepts visually while maintaining a cohesive overall visual composition.
By observing the general approaches designers use to merge two objects, we found that ensuring a seamless blend requires careful consideration of factors such as composition, color, perspective, texture, and scale.
The importance of this step in assessing a designer's overall skill set is particularly emphasized in terms of combining disparate elements. 
Moreover, the process of creating visually appealing blends often involves an iterative approach of trial and error, exploring and evaluating various options throughout the creative process.
Participants suggested expanding the design solution space through the divergent exploration of conceptual attributes followed by convergent selection based on attribute similarity to reduce cognitive load during the design process.




In summary, the creation of visual blends extends beyond the mere combination of two images. It requires creativity and a solid understanding of related concepts and their interconnected attributes to achieve effective and aesthetically pleasing visual communication.
Based on the findings from the formative study, we summarized four design requirements as follows:
\begin{enumerate}[leftmargin=*, label=\textbf{R\arabic*}]
    \item Assist with innovative idea generation while possessing clear connections to the message;
    \item Facilitate semantic relevance analysis and visual integration based on attribute similarity;
    \item Enable rapid access to diverse reference materials or design examples; 
    \item Allow users to try multiple ideas and compare them.
\end{enumerate}

