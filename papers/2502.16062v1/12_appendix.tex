\begin{figure}[b]
  \centering
  \includegraphics[width=\linewidth]{figure/study-supp-t1.pdf}
  \caption{Visual blends for the topic ``smoking is like a warm welcome to death'' (T1).
  }
  % \Description{The procedure of the user study.}
  \label{fig:study-supp-01}
\end{figure}


% \appendix
% \section{Appendix Section}
% \label{sec:appendix_section}
\section*{Appendix}
\section{User Study Results}

% We present our user study results in this section.
% For the topic of ``smoking is like a warm welcome to death'' (T1), participants generated results with the \sysname\ and the baseline system are listed as below:


% For the topic of ``knowledge guides the hope of our life'' (T2), participants generated results with the \sysname\ and the baseline system are like:


% In this section, we share the results collected from our user study, which compared the outcomes generated by participants using both our \sysname\ system and the baseline system.
This section presents the user study results, comparing \sysname\ outcomes to those of a baseline system.
We focus on two specific topics:

\begin{itemize}
    \item ``Smoking is like a warm welcome to death'' (Figure~\ref{fig:study-supp-01})
\end{itemize}

\begin{itemize}
    \item ``Knowledge guides the hope of our life'' (Figure~\ref{fig:study-supp-02})
\end{itemize}

% From the results, we can see that the results generated by users using the baseline system are more inclined to express the meaning of abstract descriptions through rich picture elements and by introducing the interaction with human characters.

% \begin{figure}[ht]
%   \centering
%   \includegraphics[width=\linewidth]{figure/study-supp-t1.pdf}
%   \caption{Visual blends for the topic ``smoking is like a warm welcome to death'' (T1).
%   }
%   % \Description{The procedure of the user study.}
%   \label{fig:study-supp-01}
% \end{figure}

From the results, it is evident that the baseline system tends to convey the meaning of abstract descriptions through the use of intricate visual elements and by incorporating interaction with human characters.
When encountering abstract content that cannot be directly described with images, text elements are often incorporated within the image itself.
In contrast, the \sysname\ system employs a more imaginative approach, generating visual blends and representing abstract ideas through a variety of objects that are creatively merged to convey metaphorical meaning.

% generates visual blends and represents abstract concepts through diverse objects, creatively merging them together to convey metaphorical significance.

% When encountering abstract content that cannot be directly described with images, There will be text content appearing as elements in the screen content. 
% \sysname\ takes visual fusion as the generation goal, usually expressing abstract concepts in ideograms through different entities, and integrating them together in a creative way.








% \begin{figure}[h]
%   \centering
%   \includegraphics[width=\linewidth]{figure/study-supp-t1.pdf}
%   \caption{The participants of our user study generated examples based on the concept of ``smoking is like a warm welcome to death''.
%   }
%   \Description{The procedure of the user study.}
%   \label{fig:study-supp-01}
% \end{figure}




\begin{figure}[b]
  \centering
  \includegraphics[width=\linewidth]{figure/study-supp-t2.pdf}
  \caption{Visual blends for the topic ``knowledge guides the hope of our life'' (T2).
  }
  % \Description{The procedure of the user study.}
  \label{fig:study-supp-02}
\end{figure}











% \section{More \sysname: Additional Cases}

% In this section, we present a demonstration of the \sysname\ system's adaptability and versatility by showcasing a collection of 20 distinct sets of examples generated by the system.
% Through this showcase, we aim to emphasize the system's capacity for generating a wide range of diverse and creative outcomes.


% % list other 20 examples generated from \sysname\ as below:

% % \begin{figure}
% %     \centering
% %     \subfigure[]{\includegraphics[width=\textwidth]{figure/results-supp-01.pdf}} 
% %     \subfigure[]{\includegraphics[width=\textwidth]{figure/results-supp-01.pdf}} 
% %     \subfigure[]{\includegraphics[width=\textwidth]{figure/results-supp-01.pdf}}
% %     \subfigure[]{\includegraphics[width=\textwidth]{figure/results-supp-01.pdf}}
% %     \caption{(a) blah (b) blah (c) blah (d) blah}
% %     \label{fig:foobar}
% % \end{figure}





% % \begin{figure}[H]

% % \subfloat[Fig6a.pdf]{%
% %   \includegraphics[clip,width=\columnwidth]{figure/results-supp-01.pdf}%
% % }

% % \subfloat[Fig6b.pdf]{%
% %   \includegraphics[clip,width=\columnwidth]{figure/results-supp-01.pdf}%
% % }

% % \subfloat[Fig6b.pdf]{%
% %   \includegraphics[clip,width=\columnwidth]{figure/results-supp-01.pdf}%
% % }

% % \subfloat[Fig6b.pdf]{%
% %   \includegraphics[clip,width=\columnwidth]{figure/results-supp-01.pdf}%
% % }

% % \subfloat[Fig6b.pdf]{%
% %   \includegraphics[clip,width=\columnwidth]{figure/results-supp-01.pdf}%
% % }

% % \subfloat[Fig6b.pdf]{%
% %   \includegraphics[clip,width=\columnwidth]{figure/results-supp-01.pdf}%
% % }

% % \subfloat[Fig6b.pdf]{%
% %   \includegraphics[clip,width=\columnwidth]{figure/results-supp-01.pdf}%
% % }

% % \caption{main caption}

% % \end{figure}


% \begin{figure*}[h]
%   \centering
%   \includegraphics[width=\linewidth]{figure/supp.pdf}
%   \caption{The Sample ideas generated by \sysname. These examples are randomly selected from the topics used in previous research or commonly used in our daily lives.
%   }
%   % \Description{The procedure of the user study.}
%   \label{fig:supp}
% \end{figure*}
% % \vspace{105mm}



% \begin{figure}[H]
%   \centering
%   \includegraphics[width=\linewidth]{figure/results-supp-1.pdf}
%   \caption{The example of ``word can hurt people''.
%   }
%   \Description{The procedure of the user study.}
%   \label{fig:supp-01}
% \end{figure}

% \begin{figure}[H]
%   \centering
%   \includegraphics[width=\linewidth]{figure/results-supp-2.pdf}
%   \caption{The example of ``take care of your mental health''.
%   }
%   \Description{The procedure of the user study.}
%   \label{fig:supp-01}
% \end{figure}

% \begin{figure}[H]
%   \centering
%   \includegraphics[width=\linewidth]{figure/results-supp-3.pdf}
%   \caption{The example of ``calling for deliciousness''.
%   }
%   \Description{The procedure of the user study.}
%   \label{fig:supp-01}
% \end{figure}

% \begin{figure}[H]
%   \centering
%   \includegraphics[width=\linewidth]{figure/results-supp-04.pdf}
%   \caption{The example of ``carry fresh''.
%   }
%   \Description{The procedure of the user study.}
%   \label{fig:supp-01}
% \end{figure}

% \begin{figure}[H]
%   \centering
%   \includegraphics[width=\linewidth]{figure/results-supp-5.pdf}
%   \caption{The example of ``follow the sound to dream''.
%   }
%   \Description{The procedure of the user study.}
%   \label{fig:supp-01}
% \end{figure}

% \begin{figure}[H]
%   \centering
%   \includegraphics[width=\linewidth]{figure/results-supp-6.pdf}
%   \caption{The example of ``knowledge guides the hope of our life''.
%   }
%   \Description{The procedure of the user study.}
%   \label{fig:supp-01}
% \end{figure}

% \begin{figure}[H]
%   \centering
%   \includegraphics[width=\linewidth]{figure/results-supp-07.pdf}
%   \caption{The example of ``where will time go''.
%   }
%   \Description{The procedure of the user study.}
%   \label{fig:supp-01}
% \end{figure}

% \begin{figure}[H]
%   \centering
%   \includegraphics[width=\linewidth]{figure/results-supp-8.pdf}
%   \caption{The example of ``find healing within a book''.
%   }
%   \Description{The procedure of the user study.}
%   \label{fig:supp-01}
% \end{figure}

% \begin{figure}[H]
%   \centering
%   \includegraphics[width=\linewidth]{figure/results-supp-9.pdf}
%   \caption{The example of ``exercise energizes us''.
%   }
%   \Description{The procedure of the user study.}
%   \label{fig:supp-01}
% \end{figure}

% \begin{figure}[H]
%   \centering
%   \includegraphics[width=\linewidth]{figure/results-supp-10.pdf}
%   \caption{The example of ``sweet love''.
%   }
%   \Description{The procedure of the user study.}
%   \label{fig:supp-01}
% \end{figure}

% \begin{figure}[H]
%   \centering
%   \includegraphics[width=\linewidth]{figure/results-supp-11.pdf}
%   \caption{The example of ``difficulty fosters our growth''.
%   }
%   \Description{The procedure of the user study.}
%   \label{fig:supp-01}
% \end{figure}

% \begin{figure}[H]
%   \centering
%   \includegraphics[width=\linewidth]{figure/results-supp-12.pdf}
%   \caption{The example of ``sickness warns you''.
%   }
%   \Description{The procedure of the user study.}
%   \label{fig:supp-01}
% \end{figure}

% \begin{figure}[H]
%   \centering
%   \includegraphics[width=\linewidth]{figure/results-supp-13.pdf}
%   \caption{The example of ``youth: a fleeting summer''.
%   }
%   \Description{The procedure of the user study.}
%   \label{fig:supp-01}
% \end{figure}

% \begin{figure}[H]
%   \centering
%   \includegraphics[width=\linewidth]{figure/results-supp-14.pdf}
%   \caption{The example of ``let courage be the force that propels you''.
%   }
%   \Description{The procedure of the user study.}
%   \label{fig:supp-01}
% \end{figure}

% \begin{figure}[H]
%   \centering
%   \includegraphics[width=\linewidth]{figure/results-supp-15.pdf}
%   \caption{The example of ``knowledge nourish your spirit''.
%   }
%   \Description{The procedure of the user study.}
%   \label{fig:supp-01}
% \end{figure}

% \begin{figure}[H]
%   \centering
%   \includegraphics[width=\linewidth]{figure/results-supp-16.pdf}
%   \caption{The example of ``nature at your fingertips''.
%   }
%   \Description{The procedure of the user study.}
%   \label{fig:supp-01}
% \end{figure}

% \begin{figure}[H]
%   \centering
%   \includegraphics[width=\linewidth]{figure/results-supp-17.pdf}
%   \caption{The example of ``spray delicious''.
%   }
%   \Description{The procedure of the user study.}
%   \label{fig:supp-01}
% \end{figure}

% \begin{figure}[H]
%   \centering
%   \includegraphics[width=\linewidth]{figure/results-supp-18.pdf}
%   \caption{The example of ``thoughts fly''.
%   }
%   \Description{The procedure of the user study.}
%   \label{fig:supp-01}
% \end{figure}

% \begin{figure}[H]
%   \centering
%   \includegraphics[width=\linewidth]{figure/results-supp-19.pdf}
%   \caption{The example of ``coffee wakes up the morning''.
%   }
%   \Description{The procedure of the user study.}
%   \label{fig:supp-01}
% \end{figure}

% \begin{figure}[H]
%   \centering
%   \includegraphics[width=\linewidth]{figure/results-supp-20.pdf}
%   \caption{The example of ``hope grows in the dark''.
%   }
%   \Description{The procedure of the user study.}
%   \label{fig:supp-01}
% \end{figure}

% \begin{figure}[H]
%   \centering
%   \includegraphics[width=\linewidth]{figure/results-supp-01.pdf}
%   \caption{The example of ``word can hurt people''.
%   }
%   \Description{The procedure of the user study.}
%   \label{fig:supp-01}
% \end{figure}


% \pagebreak
% \clearpage %Page break

% \newpage

\section{Prompts}




%这里按照prompt的功能分块来写,或者也可以和图3系统界面的模块相呼应
\subsection{Metaphor Generation}


\begin{Verbatim}[breaklines=true]
Prompt = f'''
    system_setup:
        """
        I hope to set you as an assistant with strong reasoning ability and creativity.
        """
    task_definition:
        """
        Our task is to find the hidden meaning in the sentence ({Input}).
        """
    user_input:
        """
        {Input}.
        """
    task_execution:
        """
        Summarize the hidden meaning of the sentence ({Input}) in one sentence.
        """
    result_return:
        """    
        Return the result of your calculation as a valid JSON object. 
        Should be in the following JSON format:
        ###
        {{
        "result": xxx
        }}
        ###
        """
    '''
\end{Verbatim}
\vspace{-8px}

\subsection{Objects Generation}
\begin{Verbatim}[breaklines=true]
Prompt = f'''
system_setup:
    """
    I hope to set you as an assistant with strong reasoning ability and creativity.
    """
task_definition:
    """
    Our task is to support querying a concept, identify multiple physical objects (nouns) associated with it, and reason about the relationships between these physical objects (nouns) and the concept. Then, only return a valid JSON object.
    """
user_input:
    """
    The user input is {INPUT}.
    """
Points to Note:
     """
    1. The content of the answer can not include activities, such as "running" and "swimming".
    2. The content of the answer can not include categories, such as "fruits" and "vegetables".
    3. The content of the answer can not include abstract concepts, such as "exercise" and "beauty".
    """
task_execution:
    """
    1. First, list five specific valid physical objects (nouns), and each of these physical objects (nouns) should have a metaphorical relationship with the {INPUT}.
    The physical object and the {INPUT} can be connected with the "is like" structure, such as "hope is like seeds", and "knowledge is like books". 
    These objects should refer to physical entities (nouns), such as stones and leaves. You can select suitable physical objects (nouns) from external knowledge, or you can reason on your own.
    External knowledge may not necessarily meet the requirements of the task; it requires you to make a selection. 
    External knowledge includes: ###{Related_Concepts}###.
    2. Second, give the reasons why such an object (noun) has a metaphorical relationship with the {INPUT}.
    """
result_return:
    """    
    Return the result of your calculation as a valid JSON object.
    It should be in the following JSON format:
    ###
    {{
    "result": [
    ["physical object(noun) 1", "reason 1"],
    ["physical object(noun) 2", "reason 2"],
    ["physical object(noun) 3", "reason 3"],
    ["physical object(noun) 4", "reason 4"],
    ["physical object(noun) 5", "reason 5"]
    ]
    }}
    ###
    For example, return
    ###
    {{
    "result": [
    ["Dumbbells", "Because they can be used for weightlifting exercises to enhance muscle strength"],
    ["Running Shoes", "Because they are suitable for running exercises to improve cardiovascular function"],
    ["Swimming Goggles", "Because they can protect the eyes and aid in swimming exercises underwater"],
    ["Sportswear", "Because they provide a comfortable wearing experience, making exercise smoother"],
    ["Yoga Mat", "Because they offer a comfortable mat, helping with yoga and balance exercises"]
    ]
    }}
    ###
    """
'''
\end{Verbatim}
\vspace{-10px}

\subsection{Attributes Generation}
\begin{Verbatim}[breaklines=true]
Prompt = f'''
    system_setup:
        """
        I hope to set you as an assistant with strong reasoning ability and creativity.
        """
    task_definition:
        """
        Our task is to receive a list of entities and then show the five most important visible physical attributes of every object.
        Then, only return a valid JSON object.
        """
    user_input:
        """
        The user input is {INPUT}.
        """
    task_execution:
        """
        Please list five attributes for each object. These attributes should be the most important and belong to the physical or tangible properties of the object, such as shape, size, color, or any property that directly describes the object's appearance and structure. 
        Additionally, ensure these attributes are visible to the naked eye, meaning they can be directly observed by human eyes.
        You cannot answer with general terms like "size", "color", or "shape". For example, you should answer with "red" or "round".
        You can select suitable physical attributes from external knowledge, or you can reason on your own.
        External knowledge may not necessarily meet the requirements of the task; it requires you to make a selection.
        External knowledge includes:
        ###{Related_Concepts}###
        """
    result_return:
        """    
        Return the result of your calculation as a valid JSON object. 
        For each object, return [1 object and  5 most important visible physical attributes, total 6 elements].
        It should be in the following JSON format:
        ###
        {{
        "result": [
        ["physical object 1", "<Visible physical attribute 1>", "<Visible physical attribute 2>", "<Visible physical attribute 3>", "<Visible physical attribute 4>", "<Visible physical attribute 5>"],
        ["physical object 2", "<Visible physical attribute 1>", "<Visible physical attribute 2>", "<Visible physical attribute 3>", "<Visible physical attribute 4>", "<Visible physical attribute 5>"],
        ["physical object 3", "<Visible physical attribute 1>", "<Visible physical attribute 2>", "<Visible physical attribute 3>", "<Visible physical attribute 4>", "<Visible physical attribute 5>"],
        ["physical object 4", "<Visible physical attribute 1>", "<Visible physical attribute 2>", "<Visible physical attribute 3>", "<Visible physical attribute 4>", "<Visible physical attribute 5>"],
        ["physical object 5", "<Visible physical attribute 1>", "<Visible physical attribute 2>", "<Visible physical attribute 3>", "<Visible physical attribute 4>", "<Visible physical attribute 5>"]
        ]
        }}
        ###
        For example, return
        ###
        {{
        "result": [
        ["yoga", "stretching", "meditative", "flexible", "focused", "balance"], 
        ["jump rope", "cardiovascular", "coordination", "rhythmic", "agility", "endurance"],
        ["bike", "pedaling", "two-wheeled", "outdoor", "transportation", "gears"], 
        ["hiking", "trail", "scenery", "backpack", "exploration", "nature"], 
        ["hula hoop", "rotating", "waist", "colorful", "spinning", "fun"]]
        }}
        ###
        """
    '''
\end{Verbatim}

% Our task is to merge two entities by selecting a common point to create a new entity that incorporates elements from both entities.

\subsection{Scheme Generation}
\begin{Verbatim}[breaklines=true]
Prompt = f'''
system_setup:
    """
    You are a creative assistant for a designer.
    """
task_definition:
    """
    Combine two objects by identifying shared attributes of connection, thereby creating a new object that integrates components from both.
    """
user_input:
    """
    The first object is {Object A}, the shared connecting attribute of {Object A} could be {Attribute 1}.
    The second object is {Object B}, the shared connecting attribute of {Object B} could be {Attribute 2}.
    """
task_execution:
    """
    First, thinking about how to merge {Object A} and {Object B} into one object by utilizing their commonalities: {Attribute 1} of {Object A} and {Attribute 2} of {Object B}.
    Second, justify the rationale for such combination.
    Third, iterate this process to produce (NUM) distinct combinations.
    """
result_return:
    """
    Return the results as a valid JSON object in the following JSON format:
    ###
    {{
    "result":  [
        ["<scheme 1>", "<reason 1>"],
        ["<scheme 2>", "<reason 2>"],
        ...
        ]
    }}
    ###
    For example, return
    ###
    {{
    "result":   [
    ["Merge the shapes of an orange and a dumbbell plate together.", "<Because both the shape of an orange and a dumbbell plate are circular.>"],
    ["<scheme 2>", "<reason 2>"],
    ...
    ]   
    }}
    ###
    """
'''
\end{Verbatim}
\vspace{-10px}


\subsection{Image Generation}
\begin{Verbatim}[breaklines=true]
Prompt = f'''
"Generate an image that creatively blends {Object A} with {Object B}, they should be blended into a single object that has elements from both." + "Highlight the results of blending {Attribute 1} of {Object A} with {Attribute 2} of {Object B} in the resulting blended image." + {selectedScheme} + "The blended image symbolizes a {METAPHORICAL THEME}.The image should have a plain, solid-color background and no text or words."
'''
\end{Verbatim}


% The image cannot contain any text. There are no words. Use a solid color monotone background.

% \subsection{Target-to-Source Inference}
% \begin{verbatim}
% 在这里粘贴你的Python代码
% \end{verbatim}

% \subsection{Objects}


% \subsection{Attributes}


% \subsection{Design Scheme}


% \subsection{Design Restrictions}











% \section{Research Methods}

% \subsection{Part One}

% Lorem ipsum dolor sit amet, consectetur adipiscing elit. Morbi
% malesuada, quam in pulvinar varius, metus nunc fermentum urna, id
% sollicitudin purus odio sit amet enim. Aliquam ullamcorper eu ipsum
% vel mollis. Curabitur quis dictum nisl. Phasellus vel semper risus, et
% lacinia dolor. Integer ultricies commodo sem nec semper.

% \subsection{Part Two}

% Etiam commodo feugiat nisl pulvinar pellentesque. Etiam auctor sodales
% ligula, non varius nibh pulvinar semper. Suspendisse nec lectus non
% ipsum convallis congue hendrerit vitae sapien. Donec at laoreet
% eros. Vivamus non purus placerat, scelerisque diam eu, cursus
% ante. Etiam aliquam tortor auctor efficitur mattis.

% \section{Online Resources}

% Nam id fermentum dui. Suspendisse sagittis tortor a nulla mollis, in
% pulvinar ex pretium. Sed interdum orci quis metus euismod, et sagittis
% enim maximus. Vestibulum gravida massa ut felis suscipit
% congue. Quisque mattis elit a risus ultrices commodo venenatis eget
% dui. Etiam sagittis eleifend elementum.

% Nam interdum magna at lectus dignissim, ac dignissim lorem
% rhoncus. Maecenas eu arcu ac neque placerat aliquam. Nunc pulvinar
% massa et mattis lacinia.

% \newpage

\section{Self-designed Scales for User Study}

To measure outcome satisfaction, aesthetic qualities, and metaphoricity, we use self-developed 7-point Likert scales.
The specific questions are as follows.

% We designed a 7-point Likert scale to measure Outcome Satisfaction, Aesthetic Qualities, and Metaphoricity. The questions are as follows:

\subsection{Outcome Satisfaction}
Higher scores on this measure indicate greater outcome satisfaction.

% \begin{itemize}
%     \item Regarding the amount of outcome generated, your satisfaction level is
%     \item Regarding the diversity of outcomes generated, your satisfaction level is
%     \item Regarding the creativity of outcome generated, your satisfaction level is
%     \item Regarding the outcome generated, your overall satisfaction level is
% \end{itemize}

Q1. To what extent are you satisfied with the \textbf{amount} of outcomes produced?

\begin{figure}[H]
  \centering
  \includegraphics[width=0.85\linewidth]{figure/scale.pdf}
  \label{fig:supp-01}
\end{figure}

Q2. To what extent are you satisfied with the \textbf{diversity} of outcomes produced?

\begin{figure}[H]
  \centering
  \includegraphics[width=0.85\linewidth]{figure/scale.pdf}
  \label{fig:supp-02}
\end{figure}

% Q3. To what extent are you satisfied with the \textbf{creativity} of outcomes produced?

% \begin{figure}[H]
%   \centering
%   \includegraphics[width=0.85\linewidth]{figure/scale.pdf}
%   \label{fig:supp-01}
% \end{figure}

Q3. \textbf{Overall}, to what extent are you satisfied with the outcomes produced?

\begin{figure}[H]
  \centering
  \includegraphics[width=0.85\linewidth]{figure/scale.pdf}
  \label{fig:supp-03}
\end{figure}



\subsection{Aesthetic Qualities}
Higher scores on this measure indicate greater aesthetic quality.

% \begin{itemize}
%     \item Regarding the aesthetic quality of the outcome generated, your rate is 
% \end{itemize}

Q5. How would you rate the overall \textbf{aesthetic quality} of the generated outcome?

\begin{figure}[H]
  \centering
  \includegraphics[width=0.85\linewidth]{figure/scale-q5.pdf}
  \label{fig:supp-04}
\end{figure}

\subsection{Metaphoricity}
Higher scores on this measure indicate greater metaphoricity.
% \begin{itemize}
%     \item Please rate the similarity between the source and target in the outcome generated
% \end{itemize}

Q6. How would you rate the \textbf{metaphoricity} between the \textit{source} and the \textit{target} in the generated outcomes?

\begin{figure}[H]
  \centering
  \includegraphics[width=0.95\linewidth]{figure/scale-q6.pdf}
  \label{fig:supp-05}
\end{figure}






\section{More \sysname: Additional Cases}

In this section, we present a demonstration of the \sysname\ system's adaptability and versatility by showcasing a collection of 20 distinct sets of examples generated by the system (Figure~\ref{fig:supp}).
Through this showcase, we aim to emphasize the system's capacity for generating a wide range of diverse and creative outcomes.


% list other 20 examples generated from \sysname\ as below:

% \begin{figure}
%     \centering
%     \subfigure[]{\includegraphics[width=\textwidth]{figure/results-supp-01.pdf}} 
%     \subfigure[]{\includegraphics[width=\textwidth]{figure/results-supp-01.pdf}} 
%     \subfigure[]{\includegraphics[width=\textwidth]{figure/results-supp-01.pdf}}
%     \subfigure[]{\includegraphics[width=\textwidth]{figure/results-supp-01.pdf}}
%     \caption{(a) blah (b) blah (c) blah (d) blah}
%     \label{fig:foobar}
% \end{figure}





% \begin{figure}[H]

% \subfloat[Fig6a.pdf]{%
%   \includegraphics[clip,width=\columnwidth]{figure/results-supp-01.pdf}%
% }

% \subfloat[Fig6b.pdf]{%
%   \includegraphics[clip,width=\columnwidth]{figure/results-supp-01.pdf}%
% }

% \subfloat[Fig6b.pdf]{%
%   \includegraphics[clip,width=\columnwidth]{figure/results-supp-01.pdf}%
% }

% \subfloat[Fig6b.pdf]{%
%   \includegraphics[clip,width=\columnwidth]{figure/results-supp-01.pdf}%
% }

% \subfloat[Fig6b.pdf]{%
%   \includegraphics[clip,width=\columnwidth]{figure/results-supp-01.pdf}%
% }

% \subfloat[Fig6b.pdf]{%
%   \includegraphics[clip,width=\columnwidth]{figure/results-supp-01.pdf}%
% }

% \subfloat[Fig6b.pdf]{%
%   \includegraphics[clip,width=\columnwidth]{figure/results-supp-01.pdf}%
% }

% \caption{main caption}

% \end{figure}


\begin{figure*}[h]
  \centering
  \includegraphics[width=\linewidth]{figure/supp.pdf}
  \caption{The Sample ideas generated by \sysname. These examples are randomly selected from the topics used in previous research or commonly used in our daily lives.
  }
  % \Description{The procedure of the user study.}
  \label{fig:supp}
\end{figure*}
% \vspace{105mm}

