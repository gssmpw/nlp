\documentclass[conference]{IEEEtran}
\usepackage{times}

% numbers option provides compact numerical references in the text. 
\usepackage[numbers]{natbib}
\usepackage{multicol}
\usepackage[bookmarks=true]{hyperref}

\usepackage{amssymb}
\usepackage{multirow}
\usepackage{makecell}
\usepackage{graphicx}
\usepackage{caption}
\usepackage{subcaption}
\usepackage{amsfonts, amsmath, amsthm}
\usepackage[table]{xcolor}
\usepackage{algorithmic}
\usepackage{algorithm}
\usepackage{marvosym}

\pdfinfo{
   /Author (Yuhui Chen, Shuai Tian, Shugao Liu, Yingting Zhou, Haoran Li and Dongbin Zhao)
   /Title (ConRFT: A Reinforced Fine-tuning Method for VLA Models via Consistency Policy)
   /CreationDate (D:20250413080000)
   /Subject (Robotics)
   /Keywords (Robotics;Reinforcement-learning)
}

\begin{document}

% paper title
\title{ConRFT: A Reinforced Fine-tuning Method for VLA Models via Consistency Policy}

\renewcommand{\authorrefmark}[1]{\textsuperscript{#1}}
% You will get a Paper-ID when submitting a pdf file to the conference system
% \author{Author Names Omitted for Anonymous Review. Paper-ID [394]}

\author{\authorblockN{Yuhui Chen\authorrefmark{1}\authorrefmark{2},
Shuai Tian\authorrefmark{1}\authorrefmark{2},
Shugao Liu\authorrefmark{1}\authorrefmark{2}, 
Yingting Zhou\authorrefmark{1}\authorrefmark{2}, 
Haoran Li\authorrefmark{1}\authorrefmark{2}\textsuperscript{\Letter}, and
Dongbin Zhao\authorrefmark{1}\authorrefmark{2}\textsuperscript{\Letter}}
\authorblockA{\authorrefmark{1}SKL-MAIS, Institute of Automation, Chinese Academy of Sciences, Beijing, China}
\authorblockA{\authorrefmark{2}School of Artificial Intelligence, University of Chinese Academy of Sciences, Beijing, China\\
Email: \{chenyuhui2022, tianshuai2023, liushugao2023, zhouyingting2025, lihaoran2015, dongbin.zhao\}@ia.ac.cn}}

\maketitle

\begin{abstract}

Vision-Language-Action (VLA) models have shown substantial potential in real-world robotic manipulation. However, fine-tuning these models through supervised learning struggles to achieve robust performance due to limited, inconsistent demonstrations, especially in contact-rich environments. In this paper, we propose a reinforced fine-tuning approach for VLA models, named ConRFT, which consists of offline and online fine-tuning with a unified consistency-based training objective, to address these challenges. In the offline stage, our method integrates behavior cloning and Q-learning to effectively extract policy from a small set of demonstrations and stabilize value estimating. In the online stage, the VLA model is further fine-tuned via consistency policy, with human interventions to ensure safe exploration and high sample efficiency. We evaluate our approach on eight diverse real-world manipulation tasks. It achieves an average success rate of 96.3\% within 45–90 minutes of online fine-tuning, outperforming prior supervised methods with a 144\% improvement in success rate and 1.9x shorter episode length. This work highlights the potential of integrating reinforcement learning to enhance the performance of VLA models for real-world robotic applications. Videos and code are available at our project website \href{https://cccedric.github.io/conrft/}{https://cccedric.github.io/conrft/}.

\end{abstract}

\IEEEpeerreviewmaketitle

\section{Introduction}

Recent advancements in training generalist robotic policies using Vision-Language-Action (VLA) models have demonstrated remarkable capabilities in understanding and executing various manipulation tasks. These successes are primarily attributed to large-scale imitation-style pre-training and grounding with robot actions \cite{o2023open, brohan2023rt, black2024pi_0}. While pre-trained policies capture powerful representations, they often fall short when handling the complexities of real-world scenarios \cite{jones2025beyond}. Fine-tuning with domain-specific data is essential to optimize model performance for downstream tasks \cite{black2024pi_0, wang2024scaling}. While Supervised Fine-Tuning (SFT) of the VLA model using human teleoperation data remains the predominant adaptation approach, this process faces significant challenges: the model's performance heavily relies on the quality and quantity of task-specific data. However, these human-collected datasets may not consistently provide optimal trajectories due to inherent issues such as sub-optimal data and inconsistent action \cite{xu2024rldg}.
% Human teleoperation, while accessible, introduces inconsistencies in task execution and stylistic variations in demonstrations, which makes it difficult for VLA models to learn effective policies. These issues become particularly pronounced in contact-rich manipulation tasks that require precise control and dexterity \cite{xu2024rldg}. Imitation learning alone often proves inadequate in addressing these challenges, leading to sub-optimal success rates, especially in unseen or complex scenarios.

Significant progress in Large-Language-Models (LLMs) and Vision-Language-Models (VLMs) have highlighted the value of reinforcement learning as a powerful tool for bridging the gap between policy capabilities and human preference \cite{christiano2017deep, ouyang2022training, luong2024reft} or improving model reasoning \cite{pang2024iterative}. In addition, deploying reinforcement learning (RL) with task-specific reward functions to learn from online interaction data is also a promising direction \cite{ramamurthy2022reinforcement, bai2024digirl, carta2023grounding}. However, extending these insights to VLA models presents unique challenges because, unlike LLMs, VLA models necessitate direct physical interaction in real-world robotic tasks. The safety and cost constraints of collecting data in contact-rich environments demand high sample efficiency and risk-aware exploration, making a straightforward implementation of RL infeasible. Recent work has attempted to leverage RL to address the challenges faced in SFT \cite{xu2024rldg, mark2024policy}, while these methods primarily focus on utilizing RL for data augmentation or quality improvement rather than directly optimizing VLA models through RL objectives. This limits the policy’s ability to explore states out of the demonstration dataset, thus undermining the potential benefits of RL-based fine-tuning in real-world settings. 

To leverage the benefits of RL-based techniques for efficiently fine-tuning VLA models with online interaction data, we propose a reinforced fine-tuning (RFT) approach consisting of offline and online stages with a unified consistency-based training objective. While this design is similar to offline-to-online methods \cite{lee2022offline, nakamoto2024cal, zhou2024efficient}, we found that expert demonstrations' scarcity constrains their offline training performance. Motivated by insights from CPQL \cite{chen2023boosting}, we propose a unified training objective that integrates supervised learning with Q-learning in the offline stage and further fine-tunes the VLA model via consistency policy through online RL. During offline training, our approach leverages prior demonstrations and handles out-of-distribution (OOD) states, effectively extracting the policy and value function before interacting with real-world environments. In the subsequent online stage, we solve two challenges of sample efficiency and real-world safety requirements by exploiting task-specific rewards with CPQL \cite{chen2023boosting} under human interventions through Human-in-the-Loop (HIL) learning \cite{kelly2019interactive, luo2024precise}. 

Our contributions are summarized as follows: 
\begin{enumerate}
    \item We present a \textbf{Con}sistency-based \textbf{R}einforced \textbf{F}ine-\textbf{T}uning (\textbf{ConRFT}) approach, a novel pipeline with the unified training objective both for offline and online fine-tuning.
    \item By integrating offline RL with a consistency-based behavior cloning (BC) loss, we propose Cal-ConRFT, which focuses on extracting an efficient policy and value function to provide a stable initialization with a small set of demonstrations.
    \item During online fine-tuning, we propose HIL-ConRFT, which retains the same loss structure from the offline stage for rapid policy adaption while leveraging human interventions to ensure safe exploration and high sample efficiency in real-world environments.
\end{enumerate}

We evaluate our approach on eight real-world manipulation tasks, demonstrating its ability to outperform state-of-the-art (SOTA) methods. Our framework achieves an average success rate of 96.3\% after 45–90 minutes of online fine-tuning, showcasing high sample efficiency. Additionally, it outperforms SFT methods that are trained on either human data or RL policy data, with an average success rate improvement of 144\% and an episode length of 1.9x shorter. 

\section{Related Work}

\subsection{Reinforced Fine-tuning for Large Models}

RL has been widely adopted for fine-tuning LLMs and VLMs. Early works have primarily focused on RL incorporating human feedback \cite{christiano2017deep, ouyang2022training, luong2024reft, casper2023open, zhai2024fine} by learning from human preferences or by integrating task-specific rewards without explicit human preference \cite{ramamurthy2022reinforcement, bai2024digirl, carta2023grounding, kimin2021pebbla}. While many of these approaches employ on-policy algorithms (e.g., PPO \cite{schulman2017proximal}) to fine-tune pre-trained policies \cite{bai2024digirl, gupta2019relay, shao2024deepseekmath}, they typically demand large amounts of interaction data to achieve desirable performance \cite{ball2023efficient, li2024selu}. While RL has demonstrated success in many domains, it typically learns within self-generated synthetic environments rather than real-world environments. This gap prevents direct transfer for VLA models, which require real-world interaction. Our work addresses this discrepancy by developing RL frameworks tailored for efficient real-world VLA fine-tuning.
% While RL has demonstrated remarkable success across various domains, a critical distinction warrants attention when applying RL to fine-tune LLMs and VLMs. Those RL paradigms typically involve agents interacting with self-generated synthetic environments rather than authentic real-world settings. This fundamental discrepancy precludes direct experience transfer when adapting VLA models, as their fine-tuning inherently requires physical environment interactions. Our research primarily addresses this paradigm shift by developing principled RL frameworks that enable efficient VLA model fine-tuning in real-world deployment scenarios.
% In contrast to LLM or VLM fine-tuning, whose learning objectives primarily involve text or image-based tasks, VLA fine-tuning operates on physical robots, where interactions with real-world environments can be expensive and potentially unsafe. Consequently, our work focus on off-policy actor-critic RL methods \cite{chen2023boosting} that offer higher sample efficiency, making them more practical in real-world scenarios with strict safety and resource constraints. We aim to address the fine-tuning challenges of adapting VLA models to complex, safety-critical tasks involving direct robotic manipulation.

\subsection{Real-world RL Systems}

Real-world robotic RL systems require algorithms that are both sample-efficient in handling high-dimensional inputs and flexible enough to accommodate practical considerations like reward specification and environment resets \cite{luo2024precise}. Several previous methods have effectively demonstrated policy learning directly in physical environments \cite{riedmiller2009reinforcement, johannink2019residual, luo2024serl, luo2024precise}, using both off-policy \cite{tony2022offline, luo2023rlif, hu2023reboot, russell2024continuously}, on-policy \cite{zhu2019dexterous, zhuang2023robot} methods, or posing "RL as supervised learning" \cite{mark2024policy, jan2007reinforcement}. Despite this progress, many real-world RL systems still demand prolonged training sessions or require large amounts of interaction data \cite{henery2020ingredients}, which can be impractical and risk-prone in contact-rich tasks. In contrast to previous methods that train from scratch, our work focuses on utilizing pre-trained VLA models to provide high-quality policy initialization. This approach effectively mitigates unnecessary exploratory behaviors in early RL phases, thereby optimizing both policy learning efficiency and operational safety in the training process.
% In contrast to these real-world RL setups that often train smaller models for quick policy convergence, our framework builds upon a pre-trained VLA model, providing an effective initialization to shorten the online training time. By integrating both offline and online RFT, we balance data efficiency with policy optimization while ensuring safe exploration under human interventions in real-world environments. 

\subsection{Offline-to-online Methods}

Offline-to-online RL aims to leverage offline datasets to initialize a policy, which is then fine-tuned via online interactions for improved sample efficiency \cite{lee2022offline}. Existing works commonly adopt an offline pre-training stage followed by an online fine-tuning stage \cite{lee2022offline, agarwal2022reincarnating, rafailov2023moto, nakamoto2024cal}, mixing offline and online data as training proceeds. This offline-to-online pipeline is similar to our proposed two-stage fine-tuning approach that exploits pre-collected data to bootstrap policy training and then fine-tunes the policy in the real-world tasks \cite{tony2022offline}. Most offline-to-online methods assume the availability of large-scale, diverse datasets with sufficient state coverage \cite{rajeswaran2017learning, nair2018overcoming}, a condition rarely met in real-world deployments. We explore leveraging pre-trained VLA models as the base policy to enable sample-efficient policy refinement, achieving superior fine-tuning performance even under stringent demonstration data constraints.
% Most offline-to-online methods typically presume the availability of large-scale, diverse datasets to ensure sufficient state coverage \cite{rajeswaran2017learning, nair2018overcoming}, which is a condition rarely satisfied in practical deployments. Our work addresses this fundamental mismatch. We explore leveraging pre-trained VLA models as the base policy to enable sample-efficient policy refinement, achieving superior fine-tuning performance even under stringent demonstration data constraints.
% However, while most Offline2Online methods assume large-scale offline datasets or well-aligned state distributions \cite{rajeswaran2017learning, nair2018overcoming} which is often collected in simulation, our real-world scenario is far more constrained: we have only a handful of demonstrations with insufficient state coverage for conventional offline RL approaches.

\section{Problem Setup and Preliminaries}

We focus on fine-tuning a pre-trained VLA model for downstream tasks. Specifically, we assume access to a pre-trained VLA model $\pi_{\phi_{\mathrm{pre}}}$, which encodes high-level representations from both visual inputs (e.g., RGB images) and language instructions. In supervised fine-tuning (SFT), we aim to adapt $\phi_{\mathrm{pre}}$ to $\phi$ on the target task using a small set of labeled demonstrations while preserving the model’s general feature-extraction capability. Formally, let $\tau = (s_0, a_0, \dots, s_H)$ be a trajectory for the target task, then the VLA model fine-tuning aims to solve $\min_{\phi}\mathcal{L}(\tau, \phi)$ where $\mathcal{L}$ may be a negative log-likelihood (NLL) or a mean-squared error (MSE) measuring the discrepancy between the predicted actions and those in the demonstration. This procedure allows us to effectively leverage compressed knowledge in robotic tasks while steering the VLA model to the downstream environment.

Since demonstrations are often limited, inconsistent, and sub-optimal, preventing the policy from covering diverse states, SFT struggles in real-world, contact-rich robotic tasks. To address these issues, we formulate each robotic task as a Markov Decision Process (MDP), where the goal of RL is to find the optimal policy in the MDP, $\mathcal{M} = (\mathcal{S}, \mathcal{A}, \mathcal{P}, r, \rho, \gamma)$, where $s \in \mathcal{S}$ denotes the state space and $a \in \mathcal{A}$ denotes the action space. $\mathcal{P}(s'|s, a)$ is the environmental transition probabilities that depend on the system dynamics, and $\rho(s)$ denotes the initial state distribution. $r(s, a)$ and $\gamma \in (0, 1)$ are the reward function and the reward discount factor. The policy $\pi$ is estimated by maximizing the cumulative expected value of the reward, denoted as $V^{\pi}(s)=\mathbb{E}_{\pi}[\sum_{t=0}^{H}\gamma^tr(s_t,a_t)|s_0=s,a_t\sim\pi(s_t),s_{t+1}\sim p(\cdot|s_t,a_t)]$. The Q-function of a given policy $\pi$ is denoted as $Q^{\pi}(s,a)=\mathbb{E}_{\pi}[\sum_{t=0}^{H}\gamma^tr(s_t,a_t)|s_0=s,a_0=a,s_{t+1}\sim p(\cdot|s_t,a_t)]$. $H$ represents the maximum episode step of a trajectory. By coupling the VLA policy with the learned Q-function, RFT allows the VLA model to refine its behavior based on trial-and-error interactions and task-specific feedback.

\begin{figure*}[ht]
\centering
\includegraphics[width=\linewidth]{method.png}
\caption{\textbf{Overview of ConRFT.} This figure illustrates the architecture of our reinforced fine-tuning approach for a pre-trained VLA model, which comprises two stages: the offline Cal-ConRFT and the online HIL-ConRFT. Both stages use a unified consistency-based training objective. During the offline stage, we only use pre-collected demonstrations for fine-tuning. During the online stage, a human operator can intervene in the robot policy via teleoperation tools(e.g. a SpaceMouse). And we use both pre-collected demonstrations, policy transitions, and human interventions for fine-tuning. }
\label{fig:method}
\end{figure*}

\section{Method}

The proposed pipline ConRFT consists of two stages: offline fine-tuning followed by online fine-tuning to optimize robotic policies, as shown in Fig. \ref{fig:method}. In the following sections, we provide a detailed description of the two stages, with the pipeline illustration in Appendix \ref{apx:algorithm}. 

\begin{figure*}[ht]
\centering
\includegraphics[width=\linewidth]{tasks.jpg}
\caption{\textbf{Overview of all real-world experimental tasks.} The real-world tasks include picking and placing (a) banana, (b) spoon, (d) and (f) bread, operating with (c) drawer and (e) toaster, assembling complex objects such as (g) chair wheel and (h) Chinese Knot. }
\label{fig:tasks}
\end{figure*}

\subsection{Stage I: Offline Fine-tuning with Cal-ConRFT}
Since pre-trained VLA models often lack zero-shot generalizability to novel robotic configurations, in the offline stage, we focus on training the policy using a small, pre-collected offline dataset (20–30 demonstrations) before transitioning to online reinforcement learning.
% This offline stage is essential because pre-trained VLA models often lack zero-shot performance on new robots or tasks, even a handful of demonstrations can align the model to the specific robotic setup and environment.
We initialize the policy with the pre-trained VLA model for reinforcement learning, reducing both the exploration burden and the overall online training time. Considering the ability to utilize offline data effectively, we choose the Calibrated Q-Learning (Cal-QL) \cite{nakamoto2024cal} as our base offline RL method since we want the Q-function to be robust to out-of-distribution (OOD) actions. Specifically, Cal-QL trains the Q-function on a pre-collected dataset by reducing temporal difference (TD) error and an additional regularizer. This regularizer penalizes Q-values for OOD actions when they exceed the value of the reference policy $V^{\mu}(s)$, while compensating for this penalization on actions observed within the offline dataset. The Cal-QL training objective for the critic is given by:
\begin{equation}
    \begin{aligned}
    \mathcal{L}_{Q}^{offline}(\theta)=&\alpha(\mathbb{E}_{s\sim\mathcal{D},a\sim\pi(\cdot|s)}[\max(Q_{\theta}(s,a),V^{\mu}(s))]\\
    &-\mathbb{E}_{s,a\sim\mathcal{D}}[Q_{\theta}(s,a)])\\
    &+\frac{1}{2}\mathbb{E}_{(s,a,s')\sim\mathcal{D}}[(Q_{\theta}(s,a)-\mathcal{B}^{\pi}\overline{Q}_{\overline{\theta}}(s,a))^2] \\
    \end{aligned}
    \label{eq:calql_offline}
\end{equation}
\noindent where $Q_{\theta}$ is the learned Q-function parameterized by $\theta$, $\overline{Q}_{\overline{\theta}}$ is the delayed target Q-function parameterized by $\overline{\theta}$. $\mathcal{B}^{\pi}\overline{Q}(s,a)=r(s,a)+\gamma\mathbb{E}_{a'\sim\pi(\cdot|s')}(\overline{Q}(s',a'))$ is the Bellman backup operator. $\alpha$ is a hyper-parameter to control the conservative penalty. And $\mathcal{D}$ is the demo buffer that stores demonstrations.

However, while Cal-QL is generally efficient at leveraging offline datasets, it struggles to train an effective policy when only small set of demonstrations (e.g., 20–30) are available. In such cases, limited state coverage leads to poor value estimates, making it difficult for the policy to generalize to unseen states. By contrast, typical offline RL datasets are often collected from multiple behavior policies, providing a broad range of state coverage to reduce the distribution shift. Lacking this breadth, the Cal-QL loss alone may not adequately guide the learning process, resulting in poor performance.

To address this issue, we propose augmenting the offline training process by incorporating a BC loss. The BC loss directly minimizes the difference between the actions generated by the policy and those from the demonstrations. By incorporating BC loss, we encourage the model to imitate the behaviors from the demonstrations, providing additional supervisory signals during the offline stage. This helps the VLA model to learn a more effective policy and initialize a stable Q function with few demonstrations, especially in the case of contact-rich manipulation tasks where control precision is critical. 

Motivated by combining the BC loss with Q guidance under a consistency-based objective \cite{chen2023boosting}, we introduce Cal-ConRFT in the offline stage. This approach employs a consistency policy as the action head for fine-tuning the VLA model, addressing two key concerns: 1) it helps leverage inconsistent and sub-optimal demonstrations that often arise in pre-collected data, and 2) compared to diffusion-based action head, the consistency-based action head remains computationally lightweight for efficient inference \cite{chen2023boosting, xing2025consistency, prasad2024consistency}. The consistency policy is a diffusion-model-based policy \cite{li2024stabilizing} that learns to map random actions sampled from the unit Gaussian to generate actions drawn from the expert action distribution conditioned on the current state. For the consistency policy, we discretize the diffusion horizon $[\epsilon,K]$ into $M$ sub-intervals with boundaries $k_1=\epsilon \le k_2 \le \cdots \le k_m = K$ and $\epsilon = 0.002$. Specifically, the VLA model with a consistency policy as the action head is given by:
\begin{equation}
    \begin{aligned}
    \pi_{\psi}(a|s) &= f_{\psi}(a^k,k|E_{\phi}(s))\\
    \end{aligned}
    \label{eq:cp}
\end{equation}
\noindent where $f$ denotes the consistency policy parameterized with $\psi$, subscripts $k$ denoted the diffusion step, $a^k\sim\mathcal{N}(0,kI)$ and $E_{\phi}(s)$ denotes the encoded state of the pre-trained VLA model parameterized with $\phi$. The consistency-based training objective for VLA model fine-tuning is given by:
\begin{equation}
    \begin{aligned}
    &\mathcal{L}_{\pi}^{offline}(\psi) = \beta\mathcal{L}_{\pi}^{BC} + \eta\mathcal{L}_{\pi}^{Q}\\
    \end{aligned}
    \label{eq:cpql_offline}
\end{equation}

% \begin{equation}
%     \begin{aligned}
%     &\mathcal{L}_{\pi}^{offline}(\psi) = -\eta\mathbb{E}_{s\sim\mathcal{D},a\sim\pi_{\psi}}[Q(s,a)] \\ +& \beta\mathbb{E}_{(s,a)\sim\mathcal{D},m\sim\mathcal{U}[1,M-1]}[d(f_{\psi}(a+k_mz,k_m|E(s)),a)]\\
%     \end{aligned}
%     \label{eq:cpql_offline}
% \end{equation}
\noindent where BC loss $\mathcal{L}_{\pi}^{BC}=\mathbb{E}_{(s,a)\sim\mathcal{D},m\sim\mathcal{U}[1,M-1]}[d(f_{\psi}(a+k_mz,k_m|E(s)),a)]$, $z\sim\mathcal{N}(0,I)$, $d$ stands for the Euclidean distance $d(x,y)=||x-y||_2$, and Q loss $\mathcal{L}_{\pi}^{Q}=-\mathbb{E}_{s\sim\mathcal{D},a\sim\pi_{\psi}}[Q(s,a)]$. $\beta$ and $\eta$ are two hyper-parameters to balance the BC loss and Q loss. This combination enables efficient policy learning and stable value estimation, even with a small set of demonstrations, by aligning value estimates with expert actions and improving policy performance during offline training. Moreover, it provides a reliable initialization for the online stage, facilitating safe and effective exploration.

\begin{table*}
    \centering
    \scalebox{0.87}{
    \begin{tabular}{r|c|ccccc|ccccc}
    &\multirow{2}{*}{\makecell[c]{\textbf{Training}\\\textbf{Time}\\\textbf{(mins)}}} &\multicolumn{5}{c}{\textbf{Success Rate (\%)}} &\multicolumn{5}{c}{\textbf{Episode length}}  \\
    \textbf{Task} & &\makecell[c]{SFT\cite{team2024octo}} &\makecell[c]{Cal-\\ConRFT} &\makecell[c]{HG-\\DAgger\cite{kelly2019interactive}} &PA-RL\cite{mark2024policy} &\makecell[c]{HIL-\\ConRFT} &\makecell[c]{SFT\cite{team2024octo}} &\makecell[c]{Cal-\\ConRFT} &\makecell[c]{HG-\\DAgger\cite{kelly2019interactive}} &PA-RL\cite{mark2024policy} &\makecell[c]{HIL-\\ConRFT} \\ 
    \hline
    Pick Banana       &45   &40   &50   &60 (+50\%)   &80 (+100\%)  &\textbf{90} (+80\%)    &63.7 &57.8 &67.5 (0.9x) &56.1 (1.1x) &\textbf{51.2} (1.1x) \\
    Put Spoon         &45   &50   &55   &90 (+80\%)   &80 (+60\%)   &\textbf{100} (+82\%)   &49.9 &57.2 &50.9 (1.0x) &45.3 (1.1x) &\textbf{22.6} (2.5x) \\
    Open Drawer       &15   &35   &30   &80 (+129\%)  &60 (+71\%)   &\textbf{100} (+233\%)  &63.6 &61.7 &48.4 (1.3x) &57.1 (1.1x) &\textbf{32.4} (1.8x) \\
    Pick Bread        &45   &65   &55   &65 (+0\%)    &80 (+23\%)   &\textbf{100} (+82\%)   &53.2 &49.1 &65.6 (0.8x) &51.7 (1.0x) &\textbf{31.6} (1.6x) \\
    Open Toaster      &30   &30   &30   &75 (+116\%)  &100 (+233\%) &\textbf{100} (+233\%)  &51.2 &50.7 &43.4 (1.2x) &34.3 (1.5x) &\textbf{22.1} (2.3x) \\
    Put Bread         &60   &5    &20   &60 (+1100\%) &75 (+1400\%) &\textbf{100} (+400\%)  &102  &84.8 &74.2 (1.4x) &72.1 (1.4x) &\textbf{36.6} (2.3x) \\
    Insert Wheel      &60   &35   &35   &40 (+14\%)   &30 (-14\%)   &\textbf{80} (+129\%)   &42.7 &43.4 &53.0 (0.8x) &47.4 (0.9x) &\textbf{21.9} (2.0x) \\
    Hang Chinese Knot &90   &55   &40   &50 (-10\%)   &65 (+18\%)   &\textbf{100} (+150\%)  &52.6 &54.9 &47.5 (1.1x) &44.4 (1.3x) &\textbf{26.8} (2.0x) \\
    \hline
    \rowcolor{gray!20} \textbf{Average}  &48.8 &39.4 &39.4 &65 (+65\%) &71.3 (+81\%) &\textbf{96.3} (+144\%) &59.9 &57.5 &56.3 (1.1x) &51.1 (1.2x) &\textbf{30.7} (1.9x) \\
    \end{tabular}}
    \caption{\textbf{All experiment results for various offline and online fine-tuning methods.} We report the policy performance against various baselines after offline fine-tuning (SFT \cite{team2024octo} and Cal-ConRFT) and after online fine-tuning (HG-DAgger \cite{kelly2019interactive}, PA-RL \cite{mark2024policy} and HIL-ConRFT), including success rates and average episode lengths for various tasks. Specifically, for online fine-tuning, HG-DAgger, and PA-RL training starts from the SFT baseline, and HIL-ConRFT training starts from the Cal-ConRFT baseline. The performance improvement is relative to corresponding offline results. Policies are trained using the same number of online episodes with human interventions for all methods. All metrics are reported over 20 trials per task. }
    \label{tab:exp_result}
\end{table*}

\subsection{Stage II: Online Fine-tuning with HIL-ConRFT}

While the offline stage provides an initial policy from a small set of demonstration data, its performance is limited by the scope and quality of the pre-collected demonstrations. Therefore, we have the online stage with HIL-ConRFT, where the VLA model is further fine-tuned online via the consistency policy through interacting with the real-world environment. During online training process, the demo buffer $\mathcal{D}$ for offline stage is remained. Furthermore, we have a replay buffer $\mathcal{R}$ to store online data, then implement symmetric sampling \cite{ball2023efficient}, whereby for each batch, we sample equally between these two buffers to form each training batch. Since the VLA model continuously gathers new transitions based on its current policy, the data distribution naturally evolves with the policy. This ongoing interaction reduces the distribution-shift problem that the offline stage faces. As a result, we use a standard Q loss for online critic updating: 
\begin{equation}
    \begin{aligned}
    \mathcal{L}_{Q}^{online}(\theta)&=\mathbb{E}_{(s,a,s')\sim(\mathcal{D}\cup\mathcal{R})}[(Q_{\theta}(s,a)-\mathcal{B}^{\pi}\overline{Q}(s,a))^2] \\
    \end{aligned}
    \label{eq:ql_online}
\end{equation}
The consistency-based training objective for VLA model fine-tuning is given by:
\begin{equation}
    \begin{aligned}
    &\mathcal{L}_{\pi}^{online}(\psi) = \beta\mathcal{L}_{\pi}^{BC} + \eta\mathcal{L}_{\pi}^{Q}
    \end{aligned}
    \label{eq:cpql_online}
\end{equation}
\noindent where BC loss $\mathcal{L}_{\pi}^{BC}=\mathbb{E}_{(s,a)\sim(\mathcal{D}\cup\mathcal{R}),m\sim\mathcal{U}[1,M-1]}[d(f_{\psi}(a+k_mz,k_m|E(s)),a)]$, $z\sim\mathcal{N}(0,I)$, $d$ stands for the Euclidean distance $d(x,y)=||x-y||_2$, and Q loss $\mathcal{L}_{\pi}^{Q}=-\mathbb{E}_{s\sim(\mathcal{D}\cup\mathcal{R}),a\sim\pi_{\psi}}[Q(s,a)]$. Note that this objective closely mirrors Equation \ref{eq:cpql_offline} from the offline stage, enabling a quick adaption to online fine-tuning.
% \begin{equation}
%     \begin{aligned}
%     &\mathcal{L}_{\pi}^{online}(\psi) = -\eta\mathbb{E}_{s\sim(\mathcal{D}\cup\mathcal{R}),a\sim\pi_{\psi}}[Q(s,a)] \\
%     +& \beta\mathbb{E}_{(s,a)\sim(\mathcal{D}\cup\mathcal{R}),m\sim\mathcal{U}[1,M-1]}[d(f_{\psi}(a+k_mz,k_m|E(s)),a)]\\
%     \end{aligned}
%     \label{eq:cpql_online}
% \end{equation}

Typically, we decrease the BC loss weight $\beta$ while increasing the Q loss weight $\eta$ during the online stage, yet we keep the BC loss for two main reasons. 1) Firstly, it ensures the policy continues to align with the demonstration data, preventing drastic deviations that could lead to performance collapse. This is important for maintaining the quality of actions in contact-rich manipulation tasks, where sudden changes in the policy can result in unsafe or inefficient behaviors. 2) Secondly, since reinforcement learning inherently involves exploration, it can become unstable in high-dimensional state-action spaces. By providing a stabilizing effect on exploration \cite{li2024generalizing}, the BC loss prevents the policy from deviating too far from its offline baseline, thereby reducing the risk of inefficient or unsafe behaviors. This aspect is important in real-world robotic training, especially in physical environments where unsafe actions can lead to damage or other hazards.

Also, we integrate human interventions into the online stage through Human-in-the-Loop learning. Specifically, HIL learning allows for timely interventions by a human operator who can provide corrective actions during the exploration process, which will then take over the control of the robot from the VLA model. These human corrections are added to the demo buffer $\mathcal{D}$, offering high-level guidance that steers exploration in a safer and more efficient direction \cite{huihan2023robot}. Human interventions are essential when the robot engages in destructive behaviors, such as colliding with obstacles, applying excessive force, or damaging the environment. In addition to ensuring safe exploration, human interventions accelerate policy convergence. In scenarios where the policy leads the robot into an unrecoverable or undesirable state or when the robot becomes stuck in a local optimum that would otherwise require significant time and steps to overcome without external assistance, the human operator can step in to correct the robot’s actions and guide it towards safer and more effective behavior. This results in a stable learning process, where the VLA model is fine-tuned quicker and more safely than it would through autonomous exploration alone. 

\section{Experiment and Results}

In this section, we validate the proposed fine-tuning framework through real-world experiments. We first present the experimental setup and the results for various baselines and then discuss these results and their implications.

\begin{figure*}[ht]
\centering
\includegraphics[width=\linewidth]{online_result.png}
\caption{\textbf{Learning curves during online training.} This figure presents the success rates, intervention rates, and episode lengths for HIL-SERL \cite{luo2024precise}, HG-DAgger \cite{kelly2019interactive}, PA-RL \cite{mark2024policy} and our method across five representative real-world tasks, displayed as a running average over 20 episodes. PA-RL is implemented without human intervention. Note that human interventions may lead the policy to successful outcomes, and thus, the actual policy success rate when interventions exist might be lower than the curve suggests. }
\label{fig:online_result}
\end{figure*}

\subsection{Overview of Experiments}

Our experiments aim to evaluate our approach's effectiveness and efficiency for fine-tuning VLA models in real-world scenarios. To this end, we perform real-world experiments across eight diverse manipulation tasks, as illustrated in Figure \ref{fig:tasks}. These tasks are designed to reflect a variety of manipulation challenges, including object placement tasks (e.g., placing bread into a toaster and putting bread on a white plate), precise and contact-rich manipulation (e.g., aligning and inserting a wheel into the chair base), and dynamic object handling (e.g., hanging a Chinese Knot). To validate our fine-tuning approach, we select the Octo-small model \cite{team2024octo} for its balance of performance and inference efficiency, and employ a consistency policy \cite{prasad2024consistency} as the action head on a 7-DoF Franka Emika robot arm.

For all tasks, the state observation includes two RGB images captured from a wrist-mounted camera (128 × 128) and a side camera (256 × 256), in combination with the robot's proprioceptive state of the robot arm, including end-effector poses, twists, forces/torques, and gripper status. The action space is defined as either a 6-dimensional end-effector delta pose for the downstream impedance controller or a 7-dimensional target that includes 1-dimensional binary gripper action, additionally for tasks that involve grasping. Data collection and policies command actions at 10Hz. Before training, positive and negative demonstrations are collected from human operators to train a binary classifier that gives binary feedback on whether the corresponding task is done successfully or not for each task. Additionally, each task's initial state is randomized using either a scripted robot motion or manual resets by a human operator. We present descriptions of each task in our real-world experiments and more details on the experiment tasks, training, and evaluation procedure in the Appendix \ref{apx:tasks}.

\subsection{Experimental Results}

In this section, we provide the experimental results for all tasks as shown in Figure \ref{fig:tasks}. For each task, we report result metrics, including the success rate, episode length, and total training time in Table \ref{tab:exp_result}. The training time includes the duration of scripted motions, policy rollouts, and onboard computations, all of which are conducted using an NVIDIA RTX A6000 GPU. For the offline stage, we compare Cal-ConRFT and SFT, where the SFT uses an NLL loss for behavior cloning \cite{team2024octo}. For the online stage, we compared HIL-ConRFT with multiple baselines, including HG-DAgger \cite{kelly2019interactive} that incorporates human corrections to fine-tune the policy through supervised learning, PA-RL \cite{mark2024policy} that optimized actions through a policy-agnostic Q-function and fine-tune the policy through supervised learning with the optimized actions. We also compare HIL-SERL \cite{luo2024precise} that trains a RL policy with human interventions from scratch, and RLDG \cite{xu2024rldg} that fine-tunes the VLA model using SFT \cite{team2024octo} with demonstrations collected by RL policy. 

\subsubsection{ConRFT Outperforms Supervised Methods}

We compare different approaches for supervised and reinforced methods in Table \ref{tab:exp_result} and present the corresponding online learning curves in Figure \ref{fig:online_result}. Our approach, ConRFT, achieves the highest average success rate of 96.3\% after 45 to 90 minutes of real-world training across all tasks, representing a 144\% improvement over the supervised baseline. It outperforms SOTA methods such as HG-DAgger and PA-RL, with average success rates of 65\% and 71.3\%. While HG-DAgger leverages human corrections to fine-tune the VLA model through supervised learning, it fails to achieve significant policy improvement and even experiences a performance drop on some tasks due to the sub-optimality and inconsistency of human corrections. For example, we observe that for contact-rich tasks that require precise, careful manipulation, such as Insert Wheel and Hang Chinese Knot, HG-DAgger has limited policy improvement after online fine-tuning. Specifically, in the Hang Chinese Knot task, the careful handling of soft objects demands consistent and precise controls. The inherent variability in human corrections, such as differences in the angle of insertion, introduces noise and conflicting information into the training process. This inconsistency prevents the policy's ability to learn precise and dexterous behaviors. Additionally, the complexity of contact dynamics means that minor deviations in the policy can result in significant performance drops, further exacerbating the challenges posed by inconsistent human corrections. 

In the absence of human corrections, PA-RL offers a direct action optimization using a policy-agnostic Q-function trained through Cal-QL. By optimizing actions based on reward signals, PA-RL overcomes the sub-optimality of human corrections and demonstrates more stable policy improvement in simpler tasks such as Pick Banana and Put Spoon. However, it fails to improve the policy performance in contact-rich tasks that require precise, careful manipulation, such as Insert Wheel. Precise alignment and controlled insertion forces are critical in the Insert Wheel task. However, due to the limited state coverage in the demo buffer and replay buffer, the policy-agnostic Q-function is unable to generalize effectively to different wheel and slot positions. This limits the policy's ability to handle the slight variations in state transitions required for successful insertion, leading to sub-optimal performance in complex manipulation scenarios. Consequently, while PA-RL shows promise in simple environments, it struggles to scale to complex tasks demanding high precision and dexterity.

These observations underscore the advantages of our proposed approach, which effectively mitigates the issues associated with inconsistent human corrections and limited state coverage by reinforcement learning. Our method, ConRFT, effectively and safely explores a broad range of states and directly optimizes the policy using task-specific rewards, thereby demonstrating high sample efficiency and mitigating the impact of inconsistent human corrections. This stability and performance highlight the effectiveness of our approach in overcoming the limitations of existing fine-tuning methods in real-world robotic applications.

\begin{table}[t]
    \centering
    \scalebox{0.82}{
    \begin{tabular}{r|c|cc|cc}
         &\multirow{2}{*}{\makecell[c]{\textbf{Training}\\\textbf{Time}\\\textbf{(mins)}}} &\multicolumn{2}{c}{\textbf{Success Rate (\%)}} &\multicolumn{2}{c}{\textbf{Episode length}}  \\
        \multirow{2}{*}{\textbf{Task}} & &\makecell[c]{HIL-\\SERL\cite{luo2024precise}} &\makecell[c]{HIL-\\ConRFT} &\makecell[c]{HIL-\\SERL\cite{luo2024precise}} &\makecell[c]{HIL-\\ConRFT} \\ 
        \hline
        Pick Banana       &45   &0 $\rightarrow$ 15           &50 $\rightarrow$ \textbf{90}  &\textbf{30.6} &51.2 \\
        Put Spoon         &45   &0 $\rightarrow$ 60           &55 $\rightarrow$ \textbf{100} &56.1          &\textbf{22.6} \\
        Open Drawer       &15   &0 $\rightarrow$ 10           &30 $\rightarrow$ \textbf{100} &67.5          &\textbf{32.4} \\
        Pick Bread        &45   &0 $\rightarrow$ 45           &55 $\rightarrow$ \textbf{100} &\textbf{22.0} &31.6 \\
        Open Toaster      &30   &0 $\rightarrow$ \textbf{100} &30 $\rightarrow$ \textbf{100} &28.1          &\textbf{22.1} \\
        Put Bread         &60   &0 $\rightarrow$ 5            &20 $\rightarrow$ \textbf{100} &62.0          &\textbf{36.6} \\
        Insert Wheel      &60   &0 $\rightarrow$ 5            &35 $\rightarrow$ \textbf{80}  &42.0          &\textbf{21.9} \\
        Hang Chinese Knot &90   &0 $\rightarrow$ 15           &40 $\rightarrow$ \textbf{100} &57.3          &\textbf{26.8} \\
        \hline
        \rowcolor{gray!20} \textbf{Average}  &48.8 &0 $\rightarrow$ 31.9 &39.4 $\rightarrow$ \textbf{96.3} &45.7 &\textbf{30.7} \\
    \end{tabular}}
    % \vspace{1em}
    \caption{\textbf{Experiment results for training from scratch (HIL-SERL \cite{luo2024precise}) and fine-tuning VLA (HIL-ConRFT).} Policies are trained using the same number of episodes with human interventions. All metrics are reported over 20 trials per task.}
    \label{tab:offline_rl_result}
\end{table}

Another critical metric for evaluating policy performance is the episode length, which represents the total number of steps the policy takes to complete a task. As shown in Table \ref{tab:exp_result}, the VLA model fine-tuned with HIL-ConRFT achieves an average episode length of 30.7 steps, demonstrating a 1.9x shorter than the offline baselines. In contrast, HG-DAgger achieves an average episode length of 56.3 steps, which is only 1.1x shorter than the offline baseline. Similarly, PA-RL attains an average episode length of 51.1 steps. It lacks policy exploration due to the conservative nature of the policy-agnostic Q-function, preventing it from effectively optimizing how to complete the task more quickly or trying more efficient behaviors. 

These results illustrate that ConRFT can effectively exploit the dynamic characteristics of MDPs to optimize the VLA model via consistency policy for maximizing the discounted sum of rewards. They also show the limitations of supervised methods in handling sub-optimal data and efficient policy exploration. By encouraging policies to obtain rewards more quickly, our approach results in shorter episode lengths than supervised methods relying solely on imitating demonstrations. This enhanced sample efficiency and reduced episode length highlight the advantages of ConRFT for fine-tuning VLA models in real-world robotic applications.

\subsubsection{Fine-tuning VLA Outperforms Training From Scratch}

Reinforcement learning from scratch typically demands extensive interaction with the environment and frequent human interventions, which can lead to a lengthy training process and high safety risks. For instance, HIL-SERL \cite{luo2024precise}, an approach that trains policies through RL from scratch with human interventions, fails to converge to an effective policy within the same training duration as our approach, reaching an average success rate of only 31.9\% as shown in Table \ref{tab:offline_rl_result}. The learning curves in Figure \ref{fig:online_result} reveal that HIL-ConRFT consistently improves policy performance during the online stage. While HIL-SERL can achieve optimal policies eventually, it usually requires over two hours of online training with a higher intervention rate for each task, resulting in more destructive behaviors during exploring (e.g., collisions with the environment), especially in the early stage of training.

In contrast, starting from a pre-trained VLA model and performing offline fine-tuning reduces the online training time and improves sample efficiency. Building upon offline initialized policy, ConRFT accelerates the policy convergence and enhances the final performance. As a result, fine-tuning VLA models via consistency policy enables them to reach higher success rates more quickly and with fewer interventions compared to training entirely from scratch, demonstrating the benefits of leveraging pre-trained VLA models in real-world robotic applications.

\begin{figure}[t]
\centering
\includegraphics[width=\linewidth]{abla_bc.png}
\caption{\textbf{Learning curves for HIL-ConRFT online fine-tuning from SFT \cite{team2024octo} and Cal-ConRFT baselines.} This figure presents success and intervention rates across two representative tasks, displayed as a running average over 20 episodes. }
\label{fig:abla_bc}
\end{figure}

\subsubsection{Analysis}

\paragraph{Why fine-tuning from Cal-ConRFT rather than SFT or Cal-QL?}

As illustrated in Table \ref{tab:exp_result}, we observe that during the offline stage, the performance of Cal-ConRFT is similar to that of the SFT baseline. This observation raises the question of why Q loss should be introduced during the offline stage. The reason is that when the offline stage relies solely on SFT, the fine-tuned policy benefits from imitation learning but may require substantial online fine-tuning to handle states and actions not covered by the offline dataset. In contrast, incorporating Q loss during the offline stage allows the early Q-value estimations to provide initialization for policy improvement, facilitating quicker adaptation during online fine-tuning. This approach helps address potential biases and ensures more stable learning. Moreover, in scenarios with a small set of demonstrations, we find that relying on Cal-QL alone is insufficient to train an effective policy, resulting in a 0\% success rate on all tasks. The lack of data affects the policy's ability to accurately estimate Q-values, resulting in weak performance after the offline stage and longer training time in the online stage. 

We compare the online fine-tuning curves starting from Cal-ConRFT and the SFT baseline on two representative tasks to investigate further the impact of introducing Q loss, as shown in Figure \ref{fig:abla_bc}. Although both curves begin with similar success rates, the higher intervention rate observed during training from the SFT baseline indicates that the SFT-trained policy experiences severe policy forgetting in the early stages of online training. This suggests that Cal-ConRFT enables quicker adaptation of the online learning process by leveraging the Q loss during the offline stage, allowing more effective and stable policy improvement with a small set of demonstration data. 

% \paragraph{Does our two-stage approach face performance drop when transitioning to online stage?}

% Yes, we still observe a performance drop when transitioning to online stage, since the performance is evaluated with sparse reward signals that only indicate success or failure of the whole task. Human interventions can correct the policy and lead to successful outcomes, thus the success rate in Figure \ref{fig:online_result} does not actually reflect the policy’s real performance. The policy have to experience both successes and failures in order to estimate the accurate value function. For instance, in the object placement tasks such as Pick Banana and Put Spoon, where the policy needs to (1) pick the object, (2) move it to a target position, and (3) place it, any sub-goal failure results in an overall task failure. During online training, we observe that while the initial policy can often successfully perform the initial pick, the policy in the early-stage training usually explores actions to finish the task quicker, such as prematurely closing the gripper before reaching the object. This leads to more frequent failures in the pick phase and, consequently, a drop in overall success rate. Nonetheless, these failure cases are essential for the policy to learn a better value function. Over repeated interactions, the policy uses these transitions for training and eventually surpass the initial performance, discovering a more robust policy for completing the task.

\begin{table}[t]
    \centering
    \scalebox{0.88}{
    \begin{tabular}{r|ccccc}
          &\multicolumn{5}{c}{\textbf{Success Rate (\%)}}\\
        \textbf{Task}     &DP\cite{chi2023diffusion} &SFT\cite{team2024octo} &RLDG\cite{xu2024rldg} &Cal-ConRFT &HIL-ConRFT\\ 
        \hline
        Put Spoon         &60   &70   &\textbf{100} &55         &\textbf{100} \\
        Put Bread         &30   &65   &\textbf{100} &20         &\textbf{100} \\
        Insert Wheel      &35   &40   &50           &35         &\textbf{80}  \\
        \hline
        \rowcolor{gray!20} \textbf{Average} &41.7 &58.3 &83.3 &36.7 &\textbf{93.3} \\
    \end{tabular}}
    \caption{\textbf{Experimental comparisons with various demonstrations.} Diffusion Policy (DP) \cite{chi2023diffusion} and SFT \cite{team2024octo} are trained with 150 demonstrations collected by human teleoperation, while RLDG \cite{xu2024rldg} is trained with 150 demonstrations collected by RL policy. Cal-ConRFT is trained with 20 demonstrations collected by human teleoperation, and HIL-ConRFT is trained with 20 demonstrations as well as 80-120 policy-generated rollout trajectories. All metrics are reported over 20 trials per task.}
    \label{tab:offline_bc_result}
\end{table}

\paragraph{Does increasing the number of demonstrations enhance policy performance for SFT?}

Typically, during a 45-60 minutes online fine-tuning stage, the policy collects approximately 80 to 120 successful and failed trajectories. To ensure a fair comparison between our approach and supervised training methods, we further compare training Diffusion Policy (DP) \cite{chi2023diffusion} and supervised fine-tuning VLA \cite{team2024octo} using 150 demonstrations on three representative tasks, which aligns with the total number of demonstrations utilized by our approach. Additionally, we compare RLDG \cite{xu2024rldg} with fine-tuning using 150 demonstrations collected by RL policy. As shown in Table \ref{tab:offline_bc_result}, even though DP and SFT benefit from a larger quantity of demonstrations, their success rates still fail to match the performance of our method, especially on contact-rich tasks such as Insert Wheel. This indicates that simply adding more human-collected demonstrations with supervised learning does not necessarily guarantee higher performance due to the inconsistent and sub-optimal actions inherent in human-collected data. Meanwhile, RLDG achieves higher success rates using optimal data collected from RL policies, suggesting that the consistency of these RL-collected data can improve the final performance. On the other hand, our method directly fine-tune the policy by optimizing the consistency-based training objective, achieving the highest success rate.

\begin{table}[t]
\centering
\scalebox{1.0}{
\begin{tabular}{r|cc}
     &\multicolumn{2}{c}{\textbf{Success Rate (\%)}} \\
    \textbf{Task}     &Kosmos-2(1.6B)     &PaliGemma(3B) \\
    \hline
    Pick Banana       &60$\rightarrow$100 &65$\rightarrow$100 \\
    Put Spoon         &55$\rightarrow$100 &30$\rightarrow$100 \\
    Hang Chinese Knot &45$\rightarrow$100 &60$\rightarrow$100 \\
    \hline
    \rowcolor{gray!20} \textbf{Average} &53.3$\rightarrow$100 &51.7$\rightarrow$100 \\
\end{tabular}
}
\caption{\textbf{Experimental results of ConRFT on different VLA models.} We fine-tune RoboVLM \cite{li2024towards} with two VLM backbones using our method. Specifically, we fine-tune only the action head while keeping the visual encoders and transformer backbone frozen. All metrics are reported over 20 trials per task.}
\label{tab:more_vlas}
\end{table}

\paragraph{Practicality of ConRFT across Various VLA Models}

ConRFT is highly versatile and can be applied to any VLM-based architecture with an action head. This flexibility stems from its ability to optimize the action generation process independently of the underlying visual encoder, making it adaptable to various VLA frameworks. To further validate its applicability generalization, we test out approach on fine-tuning RoboVLM \cite{li2024towards} with two distinct VLM backbones. As shown in Table \ref{tab:more_vlas}, the results indicate that ConRFT can effectively enhance the performance of various VLAs, improving the success rates across multiple robotic tasks. This ability to fine-tune the action generation while leveraging the pretrained visual components underscores the broad applicability of ConRFT.

\section{Limitations}

Although our approach demonstrates strong performance and sample efficiency for fine-tuning VLA models in real-world manipulation tasks, several limitations remain.

\subsection{Sensitivity to Reward Engineering}

In this work, we implement a task-specific binary classifier to calculate the reward for RL. However, the inherent distributional shift between the classifier's training data and the state-action distributions generated during RL exploration creates a critical vulnerability, as it can lead the learned policy to engage in reward hacking, exploiting unintended behaviors where the classifier provides inaccurate rewards. For instance, the robot might position its end-effector at a specific location that triggers a false positive, causing the policy to converge to an incorrect behavior. Since these reward classifiers typically provide only sparse feedback, the policy may learn slowly, even with the help of human interventions. On the other hand, this reward-driven approach leads to highly specialized policies that are closely tied to the specific conditions of the task, limiting their ability to generalize to new environments. While introducing multi-task dense reward signals could improve sample efficiency and accelerate policy convergence, it would also demand more sophisticated reward engineering for real-world applications.

\subsection{Frozen Encoders and Transformer Backbone}

Our current implementation runs the interaction and policy learning processes in separate threads, fine-tuning only the action head network with consistent policy while keeping the visual encoders and transformer backbone frozen. While this design choice boosts real-time performance, it constrains the policy’s ability to refine its perception and representation modules during online training, especially for unseen scenarios. Allowing partial or complete updates of these frozen components, potentially with efficient techniques such as parameter-efficient fine-tuning (e.g., LoRA \cite{hu2021lora}), could enhance final task performance and adaptability without sacrificing safety or speed.

\section{Conclusion} 

We presented a two-stage approach, ConRFT, for reinforced fine-tuning VLA models in real-world robotic applications. By first performing offline fine-tuning (Cal-ConRFT) with a small set of demonstrations, we initialize a reliable policy and value function via a unified training objective that integrates Q loss and BC loss in a consistency-based framework. We then leveraged task-specific rewards and human interventions in the online fine-tuning stage (HIL-ConRFT) to fine-tune the VLA model via consistency policy. Experiments on eight diverse real-world tasks demonstrated that our approach outperforms SOTA methods in terms of success rate, sample efficiency, and episode length. Overall, this work showcases a practical way to use reinforcement learning for safe and efficient VLA model fine-tuning.

\section*{Acknowledgments}
This work is supported by the National Natural Science Foundation of China (NSFC) under Grant No. 62136008 and in part by the International Partnership Program of the Chinese Academy of Sciences under Grant 104GJHZ2022013GC.

%% Use plainnat to work nicely with natbib. 
% \bibliographystyle{plainnat}
\bibliographystyle{unsrtnat}
% \bibliography{references}
\documentclass{MITstyle}

%\usepackage[table]{xcolor}
\usepackage{chngcntr}
\usepackage{hyperref}
\usepackage{microtype}

\title{A Lightweight and Extensible Cell Segmentation and Classification Model for Whole Slide Images}

\author{Nikita Shvetsov~$^{1, }$\footnote{Correspondence e-mail: nikita.shvetsov@uit.no}, Thomas K. Kilvaer~$^{2, 3}$, Masoud Tafavvoghi~$^{4}$, Anders Sildnes~$^{1}$, \\ Kajsa Møllersen~$^{4}$, Lill-Tove Rasmussen Busund~$^{5, 6}$, Lars Ailo Bongo~$^{1}$ \\
%
\vspace{1em} % Space between authors and afilliations
%
\normalfont{\small $^{1}$Department of Computer Science, UiT The Arctic University of Norway}\\
\normalfont{\small $^{2}$Department of Oncology, University Hospital of North Norway}\\
\normalfont{\small $^{3}$Department of Clinical Medicine, UiT The Arctic University of Norway}\\
\normalfont{\small $^{4}$Department of Community Medicine, UiT The Arctic University of Norway}\\
\normalfont{\small $^{5}$Department of Medical Biology, UiT The Arctic University of Norway} \\
\normalfont{\small $^{6}$Department of Clinical Pathology, University Hospital of North Norway} %\vspace{2em}
}

\begin{document}
\maketitle

\section*{Abstract}

% \begin{abstract}
% Developing clinically useful cell-level analysis tools in digital pathology remains challenging due to limitations in dataset granularity, inconsistent annotations, computational demands of advanced models, and difficulties in integrating new technologies into clinical workflows. To address these challenges, we propose a multi-faceted solution that enhances data quality, model performance, and usability to create a lightweight and extensible cell segmentation and classification model.

% First, we update data labels by employing a cross-relabeling process that refines the labels of two existing datasets, PanNuke and MoNuSAC, to create a new unified dataset with enhanced granularity, encompassing seven distinct cell types. Second, we leverage the H-Optimus foundation model as a fixed encoder to improve feature representation for simultaneous cell segmentation and classification tasks. Third, to address the computational demands of foundation models, we employ knowledge distillation to reduce model size and complexity while maintaining comparable performance. Finally, to facilitate integration into clinical workflows, we integrate the distilled model into the QuPath software, a widely used open-source platform in digital pathology.

% Our results demonstrate improvements in cell segmentation and classification performance using the H‑Optimus-based model compared to a CNN-based model. Specifically, the average $R^2$ improved from 0.575 to 0.871, and the average $PQ$ score improved from 0.450 to 0.492, indicating better alignment with actual cell counts and enhanced segmentation and classification quality. Furthermore, the distilled student model maintains performance comparable to the larger foundation model while reducing the parameter count by a factor of 48.
% Overall, by reducing computational complexity and integrating it into existing workflows, the proposed approach may significantly impact diagnostic processes, reduce the workload of pathologists, and contribute to improved patient outcomes. Though our approach shows potential enhancements in efficiency and usability of cell segmentation and classification models in digital pathology, extensive validation is needed to deploy these models in clinical practice.
% \end{abstract}

%%% shortened abstract
\begin{abstract}
Developing clinically useful cell-level analysis tools in digital pathology remains challenging due to limitations in dataset granularity, inconsistent annotations, high computational demands, and difficulties integrating new technologies into workflows. To address these issues, we propose a solution that enhances data quality, model performance, and usability by creating a lightweight, extensible cell segmentation and classification model. 

First, we update data labels through cross-relabeling to refine annotations of PanNuke and MoNuSAC, producing a unified dataset with seven distinct cell types. Second, we leverage the H-Optimus foundation model as a fixed encoder to improve feature representation for simultaneous segmentation and classification tasks. Third, to address foundation models' computational demands, we distill knowledge to reduce model size and complexity while maintaining comparable performance. Finally, we integrate the distilled model into QuPath, a widely used open-source digital pathology platform. 

Results demonstrate improved segmentation and classification performance using the H-Optimus-based model compared to a CNN-based model. Specifically, average $R^2$ improved from 0.575 to 0.871, and average $PQ$ score improved from 0.450 to 0.492, indicating better alignment with actual cell counts and enhanced segmentation quality. The distilled model maintains comparable performance while reducing parameter count by a factor of 48. By reducing computational complexity and integrating into workflows, this approach may significantly impact diagnostics, reduce pathologist workload, and improve outcomes. Although the method shows promise, extensive validation is necessary prior to clinical deployment.
\end{abstract}
\clearpage

\section{Introduction}
In digital pathology, accurate segmentation and classification of cells are crucial for many diagnostic, prognostic, and predictive analyses \cite{Jaber_Beziaeva_etal._2019,Lin_Pan_etal._2022,Park_Ock_etal._2022,Shen_Choi_etal._2024}. Nowadays, developments in computational pathology offer multiple solutions \cite{H._Qu_P._Wu_etal._2020,Javed_Mahmood_etal._2020} to utilize cell-level datasets to train machine learning models that solve these problems. The quality and specificity of training datasets are critical for robust and accurate models. Adhering to the principle of "garbage in, garbage out", it is essential to ensure that these datasets are extensively and accurately labeled with distinct classes that reflect the diverse biological characteristics of different cell types. Unfortunately, the number of open-source datasets comprising such high-quality annotations is limited. Existing cell segmentation datasets \cite{Gamper_Koohbanani_etal._2019,Graham_Vu_etal._2019,Verma_Kumar_etal._2021} may offer extensive annotations for certain cell types while providing more general labels for others. For example, in PanNuke, which is one of the largest open-source datasets comprising labeled cells, various types of morphologically and functionally different inflammatory cells like macrophages and lymphocytes are clustered in a broad "inflammatory" class. Consequently, these classes are frequently omitted from analyses or aggregated into broader meta-classes \cite{Gamper_Koohbanani_etal._2020} and likely interfere with other cell classes included in the dataset. This and similar inconsistencies in annotation granularity limit the ability of machine learning models to learn the comprehensive and nuanced features necessary for accurate cell segmentation and classification. To address these challenges, methods for refining and standardizing dataset annotations are essential to enhance the quality of training data.

A complementary approach to mitigate the absence of high-quality training data is the use of foundation models. Foundation models as encoders are defined as large-scale, versatile networks pre-trained on vast, diverse datasets using self-supervised learning, contrasting with convolutional neural network (CNN) pre-trained encoders that rely on supervised learning with labeled data. In practice, foundation models leverage enormous amounts of weakly or unlabeled data from millions of whole slide images (WSIs) and employ self-attention mechanisms to capture long-range dependencies and global context \cite{Chen_Ding_etal._2024,Saillard_Jenatton_etal._2024,Vorontsov_Bozkurt_etal._2024,Xu_Usuyama_etal._2024}. As a consequence, foundation models are able to produce transferable feature representations across different cell types and tissue environments. The feature representations can be leveraged by decoder networks to produce segmentation masks and pixel-level classifications. Because foundation models have comprehensive feature representations, they can be effectively fine-tuned using much smaller amounts of cell-level data compared to the large datasets needed to train models from scratch. Furthermore, foundation models incorporate adversarial training elements or contrastive learning \cite{Chen_Ding_etal._2024,Xu_Usuyama_etal._2024}, enhancing their resilience and adaptability by exposing them to challenging and varied scenarios during training. This may result in more generalizable models, often making them well-suited for diverse and complex tasks in digital pathology.

Despite the inherent advantages of foundation models, their deployment for practical use faces its own obstacles. In particular, they require substantial computational power, financial investments and rigorous testing to ensure reliability and efficacy for a given task \cite{Akkus_Dangott_etal._2022,Dragomir_Cocuz_etal._2022,Go_2022,Jafri_Farooqui_etal._2024}. Moreover, while foundation models enhance feature representation and performance, they depend on the quality of available annotations for decoder fine-tuning and, like any other model, cannot resolve existing inconsistencies or ambiguities in data labels. Therefore, there remains a critical need for solutions that address both data quality and practical deployment considerations.
Further, integrating new technologies into existing clinical workflows often encounters resistance, as it necessitates adjustments to established diagnostic processes. So, there is a need to develop solutions that could be integrated into current practices, minimizing the burden on medical professionals to adopt new tools \cite{King_Williams_etal._2023}.

Existing solutions \cite{Goldsborough_Philps_etal._2024,Hörst_Rempe_etal._2024}, while addressing some aspects of these challenges, fall short in providing a comprehensive approach. To address the data quality and clinical deployment issues, we propose a multi-faceted solution that encompasses data refinement, model optimization, and integration with existing pathology tools (\hyperref[fig:fig1]{Figure 1}). The outcome is a lightweight cell segmentation and classification model that can be integrated into digital pathology workflows for practical clinical use.

\begin{figure}[h!]
    \centering
    \includegraphics[width=\textwidth, height=0.82\textheight, keepaspectratio]{images/Figure_1.pdf}
    \caption{Overview of the proposed solution, including 1) Data refinement using cross-relabeling, 2) Teacher model development and fine tuning, 3) Student model optimization with knowledge distillation and 4) Student model and QuPath integration}
    \label{fig:fig1}
\end{figure}
\clearpage

Our approach begins with preparing the data for the fine-tuning and training of the machine learning models. We create a refined dataset, acquired via cross-relabeling two cell-level datasets, enhancing annotation specificity and consistency of the labeled data. Subsequently, we create a cell segmentation and classification model based on the foundation model. We leverage the foundation model as a fixed encoder and fine-tune a decoder using the refined dataset to improve generalization across diverse tissue- and cell types.
To ensure that the model remains lightweight and deployable in a possibly resource-constrained environment, we employ knowledge distillation to approximate the functionality of the foundation model. Finally, to facilitate the practical application of our model in digital pathology workflows, we integrate it with the QuPath \cite{Bankhead_Loughrey_etal._2017} application. Each methodological component contributes to the overarching goal of enhancing model performance, generalizability, and usability in clinical settings.

The primary contributions of this paper are:
\begin{enumerate}
    \item \textit{Data labels refinement through cross-relabeling:}
    
    We propose a new method for refining labels of cell-level datasets through cross-relabeling. This method employs classification models to re-label broad and ambiguous instances, resulting in a more diverse dataset. Our evaluation demonstrates that these classification models achieve high accuracy on test subsets, indicating the reliability of the method for label refinement.

    \item \textit{Enhanced model performance via foundation models:}
    
    We employ a foundation model as a feature extractor for the cell segmentation and classification task. In comparison with training a CNN model from scratch, the foundation model backbone only needs fine-tuning, which significantly reduces training time, computational resources and data requirements. We show that using a foundation model encoder leads to better performance in cell segmentation and classification networks than using a CNN-based encoder. This improvement may enable the model to generalize more effectively across various tissue types and imaging methods.
    
    \item \textit{Model optimization through knowledge distillation:}
    
    We show that a smaller student model trained using knowledge distillation on the refined dataset obtained via our cross-relabeling approach from a foundation model achieves comparable performance in cell segmentation and quantification tasks. As a result, this model is more suitable for deployment in environments without high-performance computing resources.
    
    \item \textit{Integration with QuPath:}
    
    We integrate the distilled cell segmentation and classification model into QuPath, a widely used open-source digital pathology platform, to accelerate clinical adaptation by enabling pathologists to more easily incorporate advanced computational tools into their existing workflows.
\end{enumerate}

Through these methodological steps, we aim to bridge the gap between advanced machine learning techniques and practical clinical applications, making accurate and efficient digital pathology accessible in a broader range of healthcare settings.

\section{Refining Existing Datasets Using Cross-Relabeling}
To address the limitations of sparse and ambiguous labeling of cell-level datasets, we propose a generalizable cross-relabeling strategy that can be applied to any dataset containing broadly categorized or imprecisely labeled cell types. This approach involves training and subsequently leveraging classification models to refine broad categories into more specific or biologically relevant classes.
When applied to cell-level data, the methodology includes extracting individual cell images from the dataset patches, preprocessing these images to standardize the size and accommodate partial cells, and then training deep learning classifiers capable of distinguishing between the finer cell subtypes within the coarser categories. 
To illustrate our approach, we focus on the PanNuke \cite{Gamper_Koohbanani_etal._2020, Gamper_Koohbanani_etal._2019} and MoNuSAC \cite{Verma_Kumar_etal._2021} datasets that we have used to train models for cell quantification in our previous works \cite{Shvetsov_Grønnesby_etal._2022,Shvetsov_Sildnes_etal._2024}. We find that for better cell differentiation we have to introduce more granular labels. PanNuke includes a broad classification of "inflammatory" cells, encompassing lymphocytes, macrophages, and neutrophils. Each cell type differs significantly in structure, function, and clinical relevance. Conversely, MoNuSAC uses the label "epithelial" for a class that comprises both benign epithelial cells and malignant neoplastic cells. This practice makes it challenging to differentiate between benign and malignant epithelial cells in the dataset, which is a critical distinction when identifying tumor areas within tissue samples. To address these issues, we implement a cross-relabeling strategy as shown in \hyperref[fig:fig2]{Figure 2}. The key components are two classification models: one is trained on singular cell images from PanNuke data to classify the epithelial meta-class into epithelial and neoplastic classes. The other is trained on MoNuSAC to refine the inflammatory class into lymphocytes, neutrophils, and macrophages.

\begin{figure}[h!]
    \centering
    \includegraphics[width=\textwidth]{images/Figure_2.pdf}
    \caption{Refined dataset generation via cross relabeling}
    \label{fig:fig2}
\end{figure}

The refining approach consists of three consecutive steps. The first is the preprocessing step, in which we extract individual cells from both datasets (\hyperref[fig:fig3]{Figure 3}). The specifics of PanNuke and MoNuSAC patch preparation before cell preprocessing are provided in \hyperref[chap:S1]{Appendix S1}.

\begin{figure}[h!]
    \centering
    \includegraphics[width=\textwidth]{images/Figure_3.pdf}
    \caption{Cell instances preprocessing including (1) cell map extraction, (2) bounding box delineation, (3) adjusting cell boxes and (4) cropping and resizing of cell images}
    \label{fig:fig3}
\end{figure}

During preprocessing, we extract cell type maps from the ground truth label mask and calculate bounding boxes around each cell instance. To accommodate partial cells at patch borders, a common issue in cropped patch images, we employ mirror padding and extend the field of view of the cell label by 15 pixels to capture adjacent cells. We then crop and resize the identified regions to $64 \times 64$ pixels using bicubic interpolation.

The preprocessed PanNuke dataset comprises 68,031 neoplastic and 23,207 epithelial cell images, while MoNuSAC comprises  33,104 lymphocytes, 1,252 neutrophils, and 1,695 macrophages, which we subsequently use in training cell classification models and classifying the cell image data \hyperref[fig:S2]{Appendix Figure S2 (1)}. 

The next step is to train two distinct ResNet50-based classifiers tailored to address the specific labeling challenges inherent in each dataset. We use ResNet50 for classification models due to its proven effectiveness for image classification tasks in histopathology \cite{pan2022reviewmachinelearningapproaches}, and its compatibility with small images. For the PanNuke dataset, we design the classifier, trained on MoNuSAC data, to disaggregate the heterogeneous "inflammatory" cell category into distinct subtypes: lymphocytes, macrophages, and neutrophils. Similarly, for the MoNuSAC dataset, the classifier is trained on PanNuke data and distinguishes between benign and malignant epithelial cells within the overarching "epithelial" label. By applying these targeted classifiers to their respective datasets, we assign more specific labels to individual cell instances, thus enabling us to create a unified dataset.
To ensure a balanced representation of classes, we train both models on datasets that had been equalized to match the size of the least represented class. Thus, we obtain datasets comprising 23,207 samples per class for PanNuke and 1,252 samples per class for MoNuSAC data. Next, we partition both of them into training (70\%), validation (20\%), and testing (10\%) subsets. To mitigate the risk of overfitting, we use a single dropout layer with a rate of p=0.5 in both models and data augmentation using randomized color perturbations, rotation, and horizontal and vertical flipping. We employ AdamW optimizer and the cross-entropy loss function for the training criterion.

To evaluate the two trained models, we measure the classification accuracy on the respective test subsets. The accuracies on the test subset for both classifiers are presented in \hyperref[tab:1]{Table 1}. The PanNuke model achieves an average accuracy of 93.57\%, with higher accuracy for neoplastic cells (96.06\%) compared to epithelial cells (86.26\%). The confusion matrix in Figure A3.1 shows that the model predominantly distinguishes accurately between epithelial and neoplastic tissues, with a substantial number of correct classifications and relatively few misclassifications. The MoNuSAC model demonstrates an average accuracy of 98.92\%, excelling in classifying lymphocytes (99.67\%) and macrophages (94.12\%), with lower performance for neutrophils (85.71\%). The confusion matrix in Figure A3.2 shows that the model identifies lymphocytes and performs reasonably well with macrophages and neutrophils.

\begin{table}[h!]
\renewcommand{\arraystretch}{1.5}
  \centering
  \caption{Cell classification results for PanNuke and MoNuSAC trained models (CI 95\%).}
  \label{tab:1}
  \begin{tabular}{|l|c|c|}
   \hline
   %\rowcolor{gray!30}
    Accuracy               & PanNuke model              & MoNuSAC model              \\
    \hline
    Average      & 0.936 (0.931--0.941)         & 0.989 (0.986--0.993)        \\
    \hline
    Neoplastic   & 0.961 (0.956--0.965)         & -                          \\
    \hline
    Epithelial   & 0.863 (0.849--0.877)         & -                          \\
    \hline
    Lymphocytes  & -                          & 0.997 (0.995--0.999)        \\
    \hline
    Neutrophils  & -                          & 0.857 (0.796--0.918)        \\
    \hline
    Macrophages  & -                          & 0.941 (0.906--0.976)        \\
    \hline
  \end{tabular}
\end{table}

Finally, during the last step, we use the model trained on PanNuke data for epithelial cells in MoNuSAC and the model trained on MoNuSAC for the inflammatory cells class in PanNuke. Specifically, we use classifier models to relabel epithelial cells in MoNuSAC and inflammatory cells in PanNuke data. Then we combine cells with refined labels and the rest of the cells in both datasets to create a refined dataset (\hyperref[fig:S2]{Appendix Figure S2 (2)}). The process of relabeling cells and visualizing them on a patch is shown in \hyperref[fig:fig4]{Figure 4}. The cell counts in the refined dataset are provided in \hyperref[tab:S4]{Appendix Table S4}.

\begin{figure}[h!]
    \centering
    \includegraphics[width=\textwidth, height=0.42\textheight, keepaspectratio]{images/Figure_4.pdf}
    \caption{Cell relabeling procedure for epithelial and inflammatory cell classes}
    \label{fig:fig4}
\end{figure}

%\hfill

Relabeling and combining datasets have been explored in a prior study \cite{Parulekar_Kanwat_etal._2023}, where consecutive fine-tuning on multiple datasets was employed to account for hierarchical class label structures. While the method presented in \cite{Parulekar_Kanwat_etal._2023} is intuitive, it often lacks consistency and requires multiple fine-tuning runs, which can be cumbersome and time-consuming. 
In contrast, cross-relabeling simplifies this process by using specialized classification models tailored to each dataset's specific labeling challenges. This approach provides better transparency and produces a unified dataset encompassing seven distinct cell types across multiple tissue samples, enhancing data diversity for further model training or fine-tuning.

Despite these improvements, cross-relabeling does not entirely resolve issues related to poor labeling quality or the amount of labeled data. Specifically, our results show lower accuracies persist for underrepresented classes, such as macrophages, which may stem from a limited sample availability and intrinsic challenges in distinguishing these cells based solely on H\&E staining. Furthermore, while our method enhances label specificity, it relies on the initial quality of the broad labels; thus, any fundamental inaccuracies in the original annotations can propagate through the relabeling process. Addressing the overall problem of limited data labels may require integrating additional data sources or utilizing complementary immunohistochemical staining methods.
Although the reported performance metrics are obtained from evaluations on the native test sets of each dataset, it is important to note that the primary application of these classifiers is to perform cross-relabeling, where a model trained on one dataset (e.g., PanNuke) is applied to another (e.g., MoNuSAC) and vice versa. We acknowledge that a more systematic evaluation of cross-dataset generalization is needed and could be performed in future work.

Overall, the refined dataset produced by our approach can enhance the supervised training or fine-tuning of cell segmentation and classification models, especially those that utilize pre-trained foundation models to improve feature extraction robustness. In addition, these models can detect nuanced classes that enable researchers to conduct more detailed analyses of biological processes in computational pathology.

\section{Foundation models for robust cell segmentation and classification}

Accurate cell segmentation and classification in digital pathology are hindered by limited labeled data and the fact that conventional CNNs are unable to capture global contextual information due to their local receptive field constraints \cite{Gheflati_Rivaz_2022,Yang_Marcus_etal.}. Traditional approaches in cell quantification have predominantly relied on CNN encoders, such as ResNet50, given their proven effectiveness in semantic segmentation tasks \cite{Deshmane_2023,Graham_Vu_etal._2019,Mukasheva_Koishiyeva_etal._2024,Stringer_Wang_etal._2021}. However, approaches that include fine-tuning of pretrained CNNs, data augmentation, and stain normalization to partially increase data variability and address staining differences often fail to achieve the necessary generalization and robustness across diverse tissue types and staining conditions \cite{G._Wang_W._Li_etal._2018,Gao_Bagci_etal._2018,Karim_El_Khoury_Martin_Fockedey_etal._2021}.

To overcome these challenges, we leverage an encoder-decoder network that uses a foundation model as the encoder and a CNN upsampling decoder (\hyperref[fig:fig5]{Figure 5}) for simultaneous cell segmentation and classification in 2D patches extracted from WSIs. Foundation models with transformer-based architectures are viable alternatives to CNN-based encoders \cite{Shamshad_Khan_etal._2023,Sourget_2023}. They enable the creation of more advanced architectures that can decode or transform learned features more effectively \cite{Chen_Duan_etal._2023,Cheng_Misra_etal._2022,Xie_Wang_etal._2021}.

\begin{figure}[h!]
    \centering
    \includegraphics[width=\textwidth]{images/Figure_5.pdf}
    \caption{UNETR-like model with foundational model as backbone}
    \label{fig:fig5}
\end{figure}

By utilizing a transformer-based encoder, we incorporate global contextual information into the feature extraction process, which is a key advantage of such architectures \cite{Chen_Lu_etal._2021}. This foundation model integration facilitates accurate pixel-wise segmentation and classification without the need for extensive encoder training, thereby potentially improving generalization across varied cellular structures and tissue types.
In our implementation, we employ a modified UNETR \cite{Hatamizadeh_Tang_etal._2021} architecture that combines a vision transformer (ViT) \cite{Dosovitskiy_Beyer_etal._2021} encoder with a CNN-based decoder. The encoder utilizes the pretrained H-Optimus foundation model, which contains 1.1 billion parameters and is trained on over 500,000 H\&E stained WSIs \cite{Saillard_Jenatton_etal._2024}. We extract outputs from four evenly spaced transformer blocks $Z_i$, where $i \in [1, 14, 26, 38]$, to serve as residual connections for the CNN decoder. We select these blocks based on our observation that features from non-adjacent levels of the encoder lead to better overall performance on the test subset.

The CNN decoder upsamples the feature representations, acquired from the transformer blocks, to generate an intermediate vector that is handled by two task-specific layers that generate cell segmentation and classification masks. The first task-specific layer is the ‘Cellpose head’,  which is used to delineate cell instances. The layer generates horizontal and vertical gradient maps to form vector fields that are refined through gradient tracking in a post-processing step using the Cellpose algorithm \cite{Stringer_Wang_etal._2021}, known for its efficacy in cell segmentation tasks and generalizability across multiple domains \cite{Pachitariu_Stringer_2022,Stringer_Pachitariu_2024}. The second task-specific layer is the "Cell type head", which assigns labels to individual pixels. In the post-processing step, we determine the output classification label of each segmented cell instance by majority voting over the labeled pixels that comprise the cell in the segmentation map.

To evaluate model performance and measure the impact of adding a foundation model as backbone, we compare it to a ResNet50-based model. ResNet50 is a widely used solution for encoders in segmentation architectures in the medical domain \cite{Deshmane_2023,Graham_Vu_etal._2019,Mukasheva_Koishiyeva_etal._2024,Stringer_Wang_etal._2021}. For the H-Optimus-based model, we utilize frozen weights for the encoder and only fine-tune the decoder to take advantage of the extensive pre-training of the foundation model. For the ResNet50-based model we start with ImageNet \cite{Deng_Dong_etal.} weights and train both encoder and decoder parts. Hyperparameters for the training step are set to be identical, where possible, for comparable evaluation. 
For this evaluation, we deliberately use the PanNuke dataset to provide a standardized and controlled comparison between the H‑Optimus and ResNet50-based models (\hyperref[fig:S2]{Appendix Figure S2 (3)}). Specifically, we use two of the default PanNuke dataset splits (66\%) for training and validation, and reserve the third split (33\%) for testing.

To address the challenge of cell class imbalance in the PanNuke dataset, which is a common characteristic in most cell-level H\&E patch datasets, both models’ training processes employ a weighted loss function comprising cross-entropy and focal loss \cite{Lin_Goyal_etal._2018}. The focal loss component is adjusted with coefficients derived from each cell class' instance frequency, emphasizing learning from underrepresented classes and enhancing the model's sensitivity to rare but significant cellular patterns. The cross-entropy loss is augmented with spectral decoupling regularization \cite{Pezeshki_Kaba_etal._2021,Pohjonen_Stürenberg_etal._2022} and spatially varying label smoothing \cite{Islam_Glocker_2021}, which potentially stabilizes training and improves generalization in case of complex tissue morphologies. For optimization, we employ the \textit{AdamW} \cite{Loshchilov_Hutter_2019} to counter unbalanced class scenarios, with cosine annealing learning rate scheduler.

We utilize the scikit-learn library \cite{Van_der_Walt_Schönberger_etal._2014} and HoVer-Net \cite{Graham_Vu_etal._2019} implementations of $R^2$ (the coefficient of determination) and $PQ$ (panoptic quality) to evaluate our experiments. Complete mathematical formulations and detailed explanations of these metrics are provided in \hyperref[chap:S5]{Appendix S5}. To compute confidence intervals, we use nonparametric bootstrapping, where after calculating the metric on the full sample, we generated 1000 bootstrap replicates by resampling with replacement and then determined the 95\% confidence intervals as the 2.5th and 97.5th percentiles of the resulting empirical distribution.

%\hfill

The model comparisons are summarized in \hyperref[tab:2]{Table 2}. The H‑Optimus-based model achieves higher $R^2$ across all cell classes compared to the ResNet50-based model, which means that its predictions are more closely aligned with the PanNuke cell counts, indicating a stronger correlation with the observed data. Notably, the improvement of $R^2_{dead}$ may be an indicator of better global contextual representations provided by the foundation model backbone. In terms of segmentation and classification quality combined, measured by the PQ score, the H‑Optimus-based model demonstrates notable improvements across most cell classes. Overall, the average $R^2$ improved from 0.575 to 0.871, while the average $PQ$ score improved from 0.450 to 0.492, demonstrating better performance of the H-Optimus-based model.

\begin{table}[h!]
\renewcommand{\arraystretch}{1.5}
  \centering
  \caption{Cell quantification metrics for baseline and proposed models (CI 95\%).}
  \label{tab:2}
  \begin{tabular}{|l|c|c|}
    \hline
    %\rowcolor{gray!30}
    Metric             & Resnet50-based            & H-optimus-based              \\
    \hline
    $R^2_{neoplastic}$    & 0.681 (0.576--0.769)       & \textbf{0.941 (0.917--0.960)} \\
    \hline
    $R^2_{inflammatory}$  & 0.863 (0.778--0.903)       & \textbf{0.949 (0.918--0.966)} \\
    \hline
    $R^2_{connective}$    & 0.600 (0.488--0.698)       & 0.609 (0.436--0.772)          \\
    \hline
    $R^2_{dead}$          & 0.097 (-11.389--0.669)     & 0.925 (0.404--0.982)          \\
    \hline
    $R^2_{epithelial}$    & 0.635 (0.490--0.747)       & \textbf{0.930 (0.886--0.964)} \\
    \hline
    $PQ_{neoplastic}$       & 0.517 (0.499--0.535)       & \textbf{0.589 (0.575--0.604)} \\
    \hline
    $PQ_{inflammatory}$     & 0.455 (0.429--0.482)       & \textbf{0.528 (0.507--0.549)} \\
    \hline
    $PQ_{connective}$       & 0.416 (0.400--0.431)       & \textbf{0.451 (0.436--0.465)} \\
    \hline
    $PQ_{dead}$             & 0.374 (0.342--0.408)       & 0.292 (0.209--0.365)          \\
    \hline
    $PQ_{epithelial}$       & 0.488 (0.460--0.519)       & \textbf{0.599 (0.579--0.618)} \\
    \hline
  \end{tabular}
\end{table}

Our results  show that integrating the H‑Optimus foundation model within the UNETR architecture enhances the model's ability to segment and classify cells across diverse tissues from PanNuke data. The pretrained transformer encoder provides robust feature representations, resulting in higher average $R^2$ and $PQ$ scores compared to the CNN-based model. This leads to more reliable cell quantification and more accurate downstream analysis. Additionally, the streamlined fine-tuning process reduces computational overhead and training time, making the model more adaptable for new data.

Despite these advancements, the foundation model-based approach does not fully resolve all challenges related to cell segmentation and classification. We observe lower metric scores for underrepresented classes in the training data. Furthermore, foundation models typically encompass billions of parameters, resulting in substantial computational and memory requirements. It therefore poses challenges for deployment in resource-constrained environments, limiting their practical applicability in certain clinical settings.

\section{Model optimization via Knowledge Distillation}

To address the limitations posed by the extensive size of foundation models, we implement knowledge distillation — a model compression technique that leverages the teacher-student paradigm \cite{Hinton_Vinyals_etal._2015}. By training a smaller, more efficient student model to replicate the output of a larger, pre-trained teacher model, we retain performance while significantly reducing the model's complexity and resource requirements (\hyperref[fig:fig6]{Figure 6}).

\begin{figure}[h!]
    \centering
    \includegraphics[width=\textwidth, height=0.45\textheight, keepaspectratio]{images/Figure_6.pdf}
    \caption{Knowledge distillation framework for training a student model using a pre-trained teacher}
    \label{fig:fig6}
\end{figure}

We employ knowledge distillation to compress the H‑Optimus-based teacher model into a more efficient student model. The teacher model is the modified UNETR architecture with the H‑Optimus foundation model described in the previous chapter. The student model is based on a UNet architecture augmented with residual connections and incorporates a smaller ViT encoder with 9 million parameters \cite{Steiner_Kolesnikov_etal._2022,Wightman_2019}. 

First, we fine-tune the teacher model using the refined dataset from the cross-relabeling procedure (Section 2). Initially we train the decoder of the teacher model while keeping the encoder weights frozen. We split the refined dataset into train (70\%), validation (20\%) and test (10\%) subsets (\hyperref[fig:S2]{Appendix Figure S2 (4)}). During fine-tuning, we use the train and validation subsets, while leaving the test subset for model evaluation. We set the training procedure and model hyperparameters to be identical to those that were used to demonstrate the utility of foundation models for the simultaneous cell segmentation and classification task.

Next, we perform knowledge distillation from teacher to student using the refined dataset used to fine-tune the teacher model. The student model is trained to replicate the teacher model's outputs. We utilize a specialized loss function that aligns the student's predicted probability distribution with the teacher's, incorporating the teacher's class probability distribution derived from the output. Following the methodology of Hinton et al. \cite{Hinton_Vinyals_etal._2015}, we experiment with various hyperparameter settings for the temperature ($T$) and the balancing coefficients ($\alpha$ and $\beta$) in the loss function. We vary $T$ from 1 to 20 and adjust $\alpha$ and $\beta$ to balance the distillation and student losses. Through iterative tuning and evaluation, we identify that setting $T=14$, $\alpha=0.3$, and $\beta=0.7$ yields a configuration that converges and closely approximates the teacher model's performance during training.

Finally, we assess the performance of both models using the $R^2$ and $PQ$ (defined in \hyperref[chap:S5]{Appendix S5}) on the test set of the refined dataset (\hyperref[tab:3]{Table 3}). We observe that the 95\% confidence intervals overlap for most cell types, so we cannot claim statistically significant performance differences between the teacher and student models. One exception appears in the neoplastic class. The teacher model produces an $R^2$ of 0.919, while the student model shows an $R^2$ of 0.852. In addition, the student model achieves higher $PQ$ values for the neoplastic and connective classes, though the confidence intervals show overlap.

\begin{table}[h!]
\renewcommand{\arraystretch}{1.5}
  \centering
  \caption{Cell quantification metrics for teacher and distilled student models (CI 95\%).}
  \label{tab:3}
  \begin{tabular}{|l|c|c|}
    \hline
    %\rowcolor{gray!30}
    Metric & Teacher & Student \\
    \hline
    $R^2_{neoplastic}$    & \textbf{0.919} (0.898--0.939) & 0.852 (0.800--0.891) \\
    \hline
    $R^2_{lymphocyte}$    & 0.969 (0.956--0.977)         & 0.969 (0.956--0.978) \\
    \hline
    $R^2_{connective}$    & 0.694 (0.548--0.809)         & 0.618 (0.469--0.741) \\
    \hline
    $R^2_{dead}$          & 0.755 (0.400--0.908)         & 0.424 (0.100--0.731) \\
    \hline
    $R^2_{epithelial}$    & 0.922 (0.870--0.958)         & 0.843 (0.738--0.917) \\
    \hline
    $R^2_{macrophage}$    & 0.384 (-0.369--0.724)        & 0.704 (0.352--0.859) \\
    \hline
    $R^2_{neutrofil}$     & 0.854 (0.578--0.929)         & 0.833 (0.502--0.925) \\
    \hline
    $PQ_{neoplastic}$       & 0.581 (0.569--0.593)         & 0.601 (0.588--0.613) \\
    \hline
    $PQ_{lymphocyte}$       & 0.536 (0.520--0.553)         & 0.563 (0.544--0.579) \\
    \hline
    $PQ_{connective}$       & 0.436 (0.421--0.451)         & 0.457 (0.441--0.474) \\
    \hline
    $PQ_{dead}$             & 0.272 (0.235--0.315)         & 0.279 (0.201--0.369) \\
    \hline
    $PQ_{epithelial}$       & 0.522 (0.500--0.545)         & 0.530 (0.506--0.555) \\
    \hline
    $PQ_{macrophage}$       & 0.524 (0.459--0.588)         & 0.474 (0.405--0.543) \\
    \hline
    $PQ_{neutrofil}$        & 0.541 (0.490--0.592)         & 0.565 (0.522--0.607) \\
    \hline
  \end{tabular}
\end{table}


We further decompose the $PQ$ metric into its $SQ$ and $DQ$ components (\hyperref[tab:S6]{Appendix Table S6}). Both models produce nearly identical $SQ$ values, which indicates that they predict instance boundaries with similar precision. Although the student model shows some improvement in $DQ$ scores for certain classes, the confidence intervals overlap and do not confirm a statistically significant difference.

We observe that the student and teacher models yield comparable detection performance despite the student model using a much smaller and simpler architecture. A model with fewer parameters reduces the risk of overfitting when training data are scarce relative to the model’s complexity \cite{Farias_Ludermir_etal._2022}. The knowledge distillation process also encourages the student model to focus on the most generalizable detection features learned from the teacher. These factors enable the student model to achieve similar detection performance across different cell types.

Additionally, considering the model sizes reported in \hyperref[tab:4]{Table 4}, the distilled model achieves a significant reduction compared to the teacher model, with a 48-fold decrease in parameter count and a 5.5-fold reduction in on-disk size. In inference mode, the teacher model requires 16 GB of VRAM for a batch size of 32, while the distilled model only needs 3 GB of VRAM for the same batch size. These reductions make the distilled model significantly more practical for fine-tuning and deployment in resource-constrained environments.

\begin{table}[h!]
\renewcommand{\arraystretch}{1.5}
  \centering
  \caption{Parameter counts and size of teacher and distilled model}
  \label{tab:4}
  \adjustbox{max width=\textwidth}{%
  \begin{tabular}{|l|c|c|c|}
    \hline
    %\rowcolor{gray!30}
    Metric & H-optimus-based (Teacher) & mobileViT-based (Student) & Magnitude of difference \\
    \hline
    Parameters count       & 1,158,917,906   & \textbf{24,093,393}   & \textbf{48x}  \\
    \hline
    Estimated Total Size (MB) & 87,912       & \textbf{15,935}    & \textbf{5.5x} \\
    \hline
  \end{tabular}%
}
\end{table}

%\hfill

With recent advancements in complex network architectures and the use of pretrained encoders to achieve state-of-the-art performance \cite{Baumann_Dislich_etal._2024,Hörst_Rempe_etal._2024} in cell segmentation and classification tasks, model size, computational complexity, and processing times have increased. This limits the scalability and accessibility of these models. As we demonstrate, this may be mitigated using knowledge distillation. Studies in the field of natural language processing have demonstrated the efficacy of knowledge distillation in retaining the capabilities of the teacher model while achieving significant reductions in size and complexity \cite{Huangpu_Gao_2024,Sun_Yu_etal.}. 

We demonstrate the feasibility of knowledge distillation in digital pathology, specifically for cell segmentation and classification tasks. Moreover, we achieve this performance while also significantly reducing the parameter count. In addressing the challenge of knowledge transfer, we found that distillation from a transformer-based model to a smaller transformer is more straightforward than attempting to map transformer features to CNN blocks. In our experiments, using a CNN-based network as a student results in worse cell quantification performance due to the structural constraints of CNN feature space dimensions. 

Although our primary approach relies on a transformer-based student model that performs well, it can be further optimized to incorporate advantages from CNN architectures. For example, employing alternative techniques such as using ViT adapters \cite{Chen_Duan_etal._2023} or $1 \times 1$ convolutions to adjust feature map sizes may be beneficial for harnessing CNN advantages like enhanced local feature extraction. Moreover, if additional performance improvements are desired, the process can be further enhanced by applying supplementary knowledge distillation techniques, such as self-distillation \cite{Zhang_Song_etal._2019} or online distillation \cite{Houyon_Cioppa_etal._2023}.

Despite these promising results, further validation on independent datasets is necessary to fully understand the model's limitations. Underrepresented classes may pose challenges when addressing complex cases. Pathologists need to validate these models to adopt them in clinical settings. While the distilled models are smaller and more deployable, a technological gap persists because pathologists traditionally rely on established methods for inspecting WSIs and diagnosing diseases. Addressing the complexities involved in deploying models for inference and supporting pathologists in adopting new tools is essential for integrating these models into clinical workflows.

\section{Model integration with QuPath}
Digital pathology tools with graphical user interfaces are essential for visualizing and analyzing WSIs. To make our student model useful in clinical pathology workflows, it needs to be integrated into a tool that enables inspecting regions, creating annotations, and providing quantitative analyses of biomarkers. Therefore, we integrate the trained student model from the previous chapter into the QuPath open‑source platform \cite{Bankhead_Loughrey_etal._2017}. QuPath provides the required annotation, visualization, and analysis tools to interpret complex histological data, including workflows for cell segmentation, classification, and quantification (\hyperref[fig:fig7]{Figure 7}). 

\begin{figure}[h!]
    \centering
    \includegraphics[width=\textwidth]{images/Figure_7.pdf}
    \caption{Visualization of model-generated cell quantification annotations (left) and the corresponding unannotated slide (right) in QuPath}
    \label{fig:fig7}
\end{figure}

To identify the regions in a WSI critical for prognosticating tumor development, such as specific tumor areas or border regions without overlapping healthy tissue, the pathologist uses QuPath to outline these regions. Then, the pathologist initiates a cell segmentation and classification script through the QuPath interface for the selected regions. The resulting annotations and quantified cell information are then directly overlaid onto the WSI in the QuPath interface. Additional design and implementation details are in \hyperref[chap:S7]{Appendix S7}. 

Two common approaches for integrating deep learning models into QuPath are Java‑based native QuPath extensions \cite{Goldsborough_Philps_etal._2024} and the execution of RESTful API requests to a model server coupled with handling the response via an extension, as demonstrated in the application of cell segmentation models applied to immunofluorescence images \cite{Sugawara_2023}. While the community is actively working on these integration strategies, there is currently no universal solution that fully addresses all integration and performance requirements.

Extensions may offer better integration with QuPath, allowing slightly improved performance and more widespread usage of the built-in QuPath models, but they lack the flexibility to customize models and modify their behavior. For example, the newest version of QuPath includes models such as StarDist \cite{Weigert_Schmidt} and InstanSeg \cite{Goldsborough_Philps_etal._2024} that can perform cell segmentation. Both models pose limitations when applied to simultaneous cell segmentation and classification. StarDist performs well only on convex, round shapes by design, whereas some neoplastic, inflammatory, and connective cells exhibit complex and non-convex shapes. InstanSeg provides only semantic segmentation without assigning classes to the segmented cells.

%\hfill

In contrast, our approach offers an alternative integration strategy. It utilizes the paquo library to directly interact with QuPath’s internal application programming interface from within Python. This enables data exchange and processing without the need for intermediate conversion steps and provides greater control over model customization, retraining, and the incorporation of custom processing steps.

The integration of our custom model with QuPath underscores its potential to significantly enhance the diagnostic process by reducing the time burden on pathologists and enabling them to focus on more complex interpretative tasks using familiar software. Leveraging a tool that is already well-established among pathologists increases the likelihood of its adoption into daily clinical workflows. The quantitative data generated through the automated workflow is critical for both clinical decision-making and research, facilitating more accurate biomarker analysis, enabling robust statistical evaluations, and supporting hypothesis generation and testing. Additionally, by streamlining cell segmentation and classification, the tool enhances the scalability and reproducibility of pathological assessments, ultimately contributing to improved diagnostic accuracy and patient outcomes.

\section{Conclusion and future work}

In this study, we address critical challenges in digital pathology and tackle the usability and deployment issues of the developed models in standard computing environments without the need for high-performance computing systems. Our multi-faceted approach encompasses data refinement through cross-relabeling, leveraging foundation models for robust cell segmentation and classification, optimizing model performance via knowledge distillation, and integrating the optimized model into the QuPath software for practical application. This approach is used to construct a capable, versatile, and adjustable model for cell segmentation and classification, with enhanced performance and usability.

\begin{sloppypar}
While our approach shows potential in the field of computational pathology, certain limitations persist. 
For example, our implementation currently exhibits lower performance in detecting macrophages. 
This serves as an instance of the broader challenge of accurately identifying complex cell types. In order to address this issue, extending our approach to incorporate additional data sources, exploring alternative modeling approaches, and integrating other imaging modalities such as immunohistochemical staining may help improve detection accuracy. Moreover, although the distilled model reduces computational demands, integrating advanced deep learning models into clinical practice requires addressing technological gaps and potential resistance to adopting new tools within established diagnostic processes.
\end{sloppypar}

Future work could focus on several key areas to refine the proposed approach and facilitate its adoption in clinical environments. Enhancing the cell-relabeling process with additional datasets \cite{Graham_Jahanifar_etal._2021} could improve the representation of underrepresented cell types and enhance overall model performance. Also, incorporating additional data sources, such as multi-modal imaging or complementary staining methods, may address limitations related to cell type differentiation and class imbalance. Exploring other foundation models \cite{Vorontsov_Bozkurt_etal._2024,Zimmermann_Vorontsov_etal._2024} or introducing additional modalities \cite{Ding_Wagner_etal._2024,Vaidya_Zhang_etal._2025} may provide alternative architectures better suited to specific tasks or offer improved efficiency. Implementing more complex knowledge distillation techniques \cite{Houyon_Cioppa_etal._2023,Zhang_Song_etal._2019} could further optimize the model's performance and adaptability. Additionally, deeper integration with QuPath or other digital pathology software could provide pathologists more control over cell quantification analysis directly within the QuPath interface, thereby increasing accessibility and usability. Such enhancements would not only refine model performance but also ensure greater adaptability and scalability within various clinical environments. Finally, extensive validation of the model by pathologists and benchmarking against independent datasets are essential steps toward establishing the model's reliability and fostering confidence in its clinical utility.

\section*{Acknowledgments} 
This work was funded in part by the Research Council of Norway grant no. 309439 SFI Visual Intelligence, and the North Norwegian Health Authority grant no. HNF1521-20.

\bibliographystyle{IEEEtran}
\begin{sloppypar}
\begin{thebibliography}{99}

\bibitem{chaplot2020neural} Chaplot, Devendra Singh, et al. "Neural topological slam for visual navigation." Proceedings of the IEEE/CVF conference on computer vision and pattern recognition. 2020.

\bibitem{maksymets2021thda} Maksymets, Oleksandr, et al. "Thda: Treasure hunt data augmentation for semantic navigation." Proceedings of the IEEE/CVF International Conference on Computer Vision. 2021.

\bibitem{mezghan2022memory} Mezghan, Lina, et al. "Memory-augmented reinforcement learning for image-goal navigation." 2022 IEEE/RSJ International Conference on Intelligent Robots and Systems (IROS). IEEE, 2022.

\bibitem{al2022zero} Al-Halah, Ziad, Santhosh Kumar Ramakrishnan, and Kristen Grauman. "Zero experience required: Plug \& play modular transfer learning for semantic visual navigation." Proceedings of the IEEE/CVF Conference on Computer Vision and Pattern Recognition. 2022.

\bibitem{ye2021auxiliary} Ye, Joel, et al. "Auxiliary tasks and exploration enable objectgoal navigation." Proceedings of the IEEE/CVF international conference on computer vision. 2021.

\bibitem{chaplot2020object} Chaplot, Devendra Singh, et al. "Object goal navigation using goal-oriented semantic exploration." Advances in Neural Information Processing Systems 33 (2020)

\bibitem{ramakrishnan2022poni} Ramakrishnan, Santhosh Kumar, et al. "Poni: Potential functions for objectgoal navigation with interaction-free learning." Proceedings of the IEEE/CVF Conference on Computer Vision and Pattern Recognition. 2022.

\bibitem{ramrakhya2022habitat} Ramrakhya, Ram, et al. "Habitat-web: Learning embodied object-search strategies from human demonstrations at scale." Proceedings of the IEEE/CVF Conference on Computer Vision and Pattern Recognition. 2022.

\bibitem{mousavian2019visual} Mousavian, Arsalan, et al. "Visual representations for semantic target driven navigation." 2019 International Conference on Robotics and Automation (ICRA). IEEE, 2019.

\bibitem{dhariwal2021diffusion} Dhariwal, Prafulla, and Alexander Nichol. "Diffusion models beat gans on image synthesis." Advances in neural information processing systems 34 (2021)

\bibitem{ho2022classifier} Ho, Jonathan, and Tim Salimans. "Classifier-free diffusion guidance." arXiv preprint arXiv:2207.12598 (2022).

\bibitem{nichol2021glide} Nichol, Alex, et al. "Glide: Towards photorealistic image generation and editing with text-guided diffusion models." arXiv preprint arXiv:2112.10741 (2021)

\bibitem{brooks2023instructpix2pix} Brooks, Tim, Aleksander Holynski, and Alexei A. Efros. "Instructpix2pix: Learning to follow image editing instructions." Proceedings of the IEEE/CVF Conference on Computer Vision and Pattern Recognition. 2023.

\bibitem{fu2023guiding} Fu, Tsu-Jui, et al. "Guiding instruction-based image editing via multimodal large language models." arXiv preprint arXiv:2309.17102 (2023).

\bibitem{geng2024instructdiffusion} Geng, Zigang, et al. "Instructdiffusion: A generalist modeling interface for vision tasks." Proceedings of the IEEE/CVF Conference on Computer Vision and Pattern Recognition. 2024.

\bibitem{zhou2024minedreamer} Zhou, Enshen, et al. "Minedreamer: Learning to follow instructions via chain-of-imagination for simulated-world control." arXiv preprint arXiv:2403.12037 (2024).

\bibitem{zhou2023esc} Zhou, Kaiwen, et al. "Esc: Exploration with soft commonsense constraints for zero-shot object navigation." International Conference on Machine Learning. PMLR, 2023.

\bibitem{yu2023l3mvn} Yu, Bangguo, Hamidreza Kasaei, and Ming Cao. "L3mvn: Leveraging large language models for visual target navigation." 2023 IEEE/RSJ International Conference on Intelligent Robots and Systems (IROS). IEEE, 2023.

\bibitem{gadre2023cows} Gadre, Samir Yitzhak, et al. "Cows on pasture: Baselines and benchmarks for language-driven zero-shot object navigation." Proceedings of the IEEE/CVF Conference on Computer Vision and Pattern Recognition. 2023.

\bibitem{shah2023navigation} Shah, Dhruv, et al. "Navigation with large language models: Semantic guesswork as a heuristic for planning." Conference on Robot Learning. PMLR, 2023.

\bibitem{cai2024bridging} Cai, Wenzhe, et al. "Bridging zero-shot object navigation and foundation models through pixel-guided navigation skill." 2024 IEEE International Conference on Robotics and Automation (ICRA). IEEE, 2024.

\bibitem{yu2023co} Yu, Bangguo, Hamidreza Kasaei, and Ming Cao. "Co-NavGPT: Multi-robot cooperative visual semantic navigation using large language models." arXiv preprint arXiv:2310.07937 (2023).

\bibitem{wu2024voronav} Wu, Pengying, et al. "Voronav: Voronoi-based zero-shot object navigation with large language model." arXiv preprint arXiv:2401.02695 (2024).

\bibitem{qin2023mp5} Qin, Yiran, et al. "Mp5: A multi-modal open-ended embodied system in minecraft via active perception." arXiv preprint arXiv:2312.07472 (2023).

\bibitem{du2024learning} Du, Yilun, et al. "Learning universal policies via text-guided video generation." Advances in Neural Information Processing Systems 36 (2024).

\bibitem{ajay2024compositional} Ajay, Anurag, et al. "Compositional foundation models for hierarchical planning." Advances in Neural Information Processing Systems 36 (2024).

\bibitem{liang2024skilldiffuser} Liang, Zhixuan, et al. "Skilldiffuser: Interpretable hierarchical planning via skill abstractions in diffusion-based task execution." Proceedings of the IEEE/CVF Conference on Computer Vision and Pattern Recognition. 2024.

\bibitem{heusel2017gans} Heusel, Martin, et al. "Gans trained by a two time-scale update rule converge to a local nash equilibrium." Advances in neural information processing systems 30 (2017).

\bibitem{zhang2018unreasonable} Zhang, Richard, et al. "The unreasonable effectiveness of deep features as a perceptual metric." Proceedings of the IEEE conference on computer vision and pattern recognition. 2018.

\bibitem{brown2020language} Brown, Tom B. "Language models are few-shot learners." arXiv preprint arXiv:2005.14165 (2020).

\bibitem{podell2023sdxl} Podell, Dustin, et al. "Sdxl: Improving latent diffusion models for high-resolution image synthesis." arXiv preprint arXiv:2307.01952 (2023).

\bibitem{brohan2022rt} Brohan, Anthony, et al. "Rt-1: Robotics transformer for real-world control at scale." arXiv preprint arXiv:2212.06817 (2022).

\bibitem{brohan2023rt} Brohan, Anthony, et al. "Rt-2: Vision-language-action models transfer web knowledge to robotic control." arXiv preprint arXiv:2307.15818 (2023).

\bibitem{li2024manipllm} Li, Xiaoqi, et al. "Manipllm: Embodied multimodal large language model for object-centric robotic manipulation." Proceedings of the IEEE/CVF Conference on Computer Vision and Pattern Recognition. 2024.

\bibitem{shah2023vint} Shah, Dhruv, et al. "ViNT: A foundation model for visual navigation." arXiv preprint arXiv:2306.14846 (2023).

\bibitem{liu2024visual} Liu, Haotian, et al. "Visual instruction tuning." Advances in neural information processing systems 36 (2024).

\bibitem{hu2021lora} Hu, Edward J., et al. "Lora: Low-rank adaptation of large language models." arXiv preprint arXiv:2106.09685 (2021).

\bibitem{qin2023supfusion} Qin, Yiran, et al. "SupFusion: Supervised LiDAR-camera fusion for 3D object detection." Proceedings of the IEEE/CVF International Conference on Computer Vision. 2023.

\bibitem{qin2024worldsimbench} Qin, Yiran, et al. "Worldsimbench: Towards video generation models as world simulators." arXiv preprint arXiv:2410.18072 (2024).

\bibitem{yu2025gamefactory} Yu, Jiwen, et al. "GameFactory: Creating New Games with Generative Interactive Videos." arXiv preprint arXiv:2501.08325 (2025).

\bibitem{zhou2024code} Zhou, Enshen, et al. "Code-as-Monitor: Constraint-aware Visual Programming for Reactive and Proactive Robotic Failure Detection." arXiv preprint arXiv:2412.04455 (2024).

\bibitem{zhang2024ad} Zhang, Zaibin, et al. "AD-H: Autonomous Driving with Hierarchical Agents." arXiv preprint arXiv:2406.03474 (2024).

\bibitem{wang2024toward} Wang, Chaoqun, et al. "Toward Accurate Camera-based 3D Object Detection via Cascade Depth Estimation and Calibration." arXiv preprint arXiv:2402.04883 (2024).

\bibitem{huang2024story3d} Huang, Yuzhou, et al. "Story3d-agent: Exploring 3d storytelling visualization with large language models." arXiv preprint arXiv:2408.11801 (2024).

\bibitem{savinov2018semi} Savinov, Nikolay, Alexey Dosovitskiy, and Vladlen Koltun. "Semi-parametric topological memory for navigation." arXiv preprint arXiv:1803.00653 (2018).

\bibitem{majumdar2022zson} Majumdar, Arjun, et al. "Zson: Zero-shot object-goal navigation using multimodal goal embeddings." Advances in Neural Information Processing Systems 35 (2022): 32340-32352.

\bibitem{yadav2023offline} Yadav, Karmesh, et al. "Offline visual representation learning for embodied navigation." Workshop on Reincarnating Reinforcement Learning at ICLR 2023. 2023.

\bibitem{yadav2023ovrl} Yadav, Karmesh, et al. "Ovrl-v2: A simple state-of-art baseline for imagenav and objectnav." arXiv preprint arXiv:2303.07798 (2023).

\bibitem{sun2024fgprompt} Sun, Xinyu, et al. "FGPrompt: fine-grained goal prompting for image-goal navigation." Advances in Neural Information Processing Systems 36 (2024).

\bibitem{zhu2017target} Zhu, Yuke, et al. "Target-driven visual navigation in indoor scenes using deep reinforcement learning." 2017 IEEE international conference on robotics and automation (ICRA). IEEE, 2017.

\bibitem{koh2024generating} Koh, Jing Yu, Daniel Fried, and Russ R. Salakhutdinov. "Generating images with multimodal language models." Advances in Neural Information Processing Systems 36 (2024).

\bibitem{krantz2022instance} Krantz, Jacob, et al. "Instance-specific image goal navigation: Training embodied agents to find object instances." arXiv preprint arXiv:2211.15876 (2022).

\bibitem{schulman2017proximal} Schulman, John, et al. "Proximal policy optimization algorithms." arXiv preprint arXiv:1707.06347 (2017).

\bibitem{anderson2018evaluation} Anderson, Peter, et al. "On evaluation of embodied navigation agents." arXiv preprint arXiv:1807.06757 (2018).

\bibitem{lin2024navcot} Lin, Bingqian, et al. "NavCoT: Boosting LLM-Based Vision-and-Language Navigation via Learning Disentangled Reasoning." arXiv preprint arXiv:2403.07376 (2024).

\bibitem{NavGPT} Zhou, Gengze, Yicong Hong, and Qi Wu. "Navgpt: Explicit reasoning in vision-and-language navigation with large language models." Proceedings of the AAAI Conference on Artificial Intelligence.

\bibitem{hahn2021no} Hahn, Meera, et al. "No rl, no simulation: Learning to navigate without navigating." Advances in Neural Information Processing Systems 34 (2021): 26661-26673.

\bibitem{li2025t2isafety} Li, Lijun, et al. "T2ISafety: Benchmark for Assessing Fairness, Toxicity, and Privacy in Image Generation." arXiv preprint arXiv:2501.12612 (2025).

\bibitem{an2024agfsync} An, Jingkun, et al. "AGFSync: Leveraging AI-Generated Feedback for Preference Optimization in Text-to-Image Generation." arXiv preprint arXiv:2403.13352 (2024).


\end{thebibliography}
\end{sloppypar}

\clearpage
\beginsupplement
\section*{Appendix}
\renewcommand{\thesubsection}{S\arabic{subsection}}

\subsection{\label{chap:S1}PanNuke and MoNuSAC preprocessing}
The PanNuke dataset comprises a set of 7,901 RGB patches, each with dimensions of $256 \times 256$ pixels, which we set as the standard patch size for our analysis. In contrast, the MoNuSAC dataset encompasses 294 images of heterogeneous dimensions. To standardize the MoNuSAC images with our experiments, we implement a standardization protocol. Specifically, for images exceeding the dimensions of $256 \times 256$ pixels, we segment them into equal-sized patches and apply mirror padding to the remaining portions to avoid information loss at the peripherals. Patches with dimensions less than $128 \times 128$ pixels are excluded from the dataset due to the insufficient resolution to capture relevant cellular details. For patches where either dimension falls between 128 and 256 pixels, we employ upsampling to achieve the standard patch size. As a result, we obtain a total of 2,823 RGB patches derived from the MoNuSAC dataset for subsequent analysis. For additional details on the MoNuSAC data preparation process, refer to the source code \cite{Shvetsov_2025a}.
\clearpage

\subsection{\label{chap:S2}Data usage for the methodology}

\counterwithin{figure}{subsection}
\renewcommand{\thefigure}{S\arabic{subsection}}

\begin{figure}[h!]
    \centering
    \includegraphics[width=\textwidth, height=0.85\textheight, keepaspectratio]{images/A2.pdf}
    \caption{Overview of the methodology for cross-labeling, dataset refinement, and model comparison. (1) Cross-relabeling - training and testing cell classification models, (2) Cross-relabeling - using cell classification models to create refined dataset, (3) Fine-tuning and training models for comparison, (4) Student knowledge distillation with refined dataset}
    \label{fig:S2}
\end{figure}
\clearpage

\subsection{\label{chap:S3}Confusion matrices for classification models}
\counterwithin{figure}{subsection}
\renewcommand{\thefigure}{S\arabic{subsection}.\arabic{figure}}

\begin{figure}[h!]
    \centering
    \includegraphics[width=\textwidth, height=0.4\textheight, keepaspectratio]{images/A3_1.pdf}
    \caption{Confusion matrix for PanNuke trained model}
    \label{fig:S3.1}
\end{figure}

\begin{figure}[h!]
    \centering
    \includegraphics[width=\textwidth, height=0.4\textheight, keepaspectratio]{images/A3_2.pdf}
    \caption{Confusion matrix for MoNuSAC trained model}
    \label{fig:S3.2}
\end{figure}

\clearpage

\subsection{\label{chap:S4}Datasets cell counts}

\counterwithin{table}{subsection}
\renewcommand{\thetable}{S\arabic{subsection}}

\begin{table}[h!]
\renewcommand{\arraystretch}{2.0}
\centering
\caption{\label{tab:S4}Cell counts for PanNuke, MoNuSAC and refined datasets. Numbers in parentheses indicate preprocessed cell counts for cell classifier models training and testing.}
%\adjustbox{max width=\textwidth}{%
\begin{tabular}{|l|c|c|c|}
\hline
%\rowcolor{gray!30}
Cell type & PanNuke & MoNuSAC & Refined \\
\hline
Neoplastic & 77,403 (68,031) & - & 105,451 \\
\hline
Epithelial & 26,572 (23,207) & - & 29,926 \\
\hline
Epithelial (benign and malignant) & - & 31,402 & - \\
\hline
Inflammatory & 32,276 & - & - \\
\hline
Lymphocytes & - & 37,045 (33,104) & 65,275 \\
\hline
Neutrophils & - & 1,355 (1,252) & 3,833 \\
\hline
Macrophage & - & 1,842 (1,695) & 3,410 \\
\hline
Dead & 2,908 & - & 2,908 \\
\hline
Connective & 50,585 & - & 50,585 \\
\hline
\end{tabular}
%
%}
\end{table}



\clearpage

\subsection{\label{chap:S5}Definition of validation metrics}
\counterwithin{equation}{subsection}
\renewcommand{\theequation}{\arabic{equation}}

\subsubsection{\label{chap:S5.1}R\textsuperscript{2}}
The coefficient of determination, denoted as $R^2$, is a statistical measure that represents the proportion of variance in the dependent variable that is predictable from the independent variables. In the context of cell quantification in pathology, $R^2$ is used to assess how well the predicted quantities of different cell types in a patch align with the actual quantities observed in the ground truth data, with higher values representing more accurate quantification. $R^2$ is defined as
\begin{equation*}
R^2 = 1 - \frac{\sum_{i=1}^n (y_i - \hat{y}_i)^2}{\sum_{i=1}^n (y_i - \bar{y})^2},
\end{equation*}
where $y_i$ represents the actual number of cells of a specific type in the $i$-th image, $\hat{y}_i$ represents the predicted number of cells of that type in the $i$-th image, $\bar{y}$ is the mean of the actual numbers across all images, and $n$ is the total number of images in the dataset.

The $R^2$ metric has a range of $(-\infty, 1]$. An $R^2$ of 1 indicates perfect prediction, where all predicted values exactly match the actual values. An $R^2$ of 0 suggests that the model explains none of the variability of the response data around its mean. If $R^2$ is negative, it indicates that the model performs worse than a model that simply predicts the mean of the actual values for all observations.

\subsubsection{\label{chap:S5.2}PQ}
Panoptic Quality ($PQ$) is a comprehensive metric used to evaluate the performance of segmentation models in tasks that require both instance segmentation and classification. $PQ$ provides a single score that encapsulates both the detection accuracy (i.e., how many objects were correctly identified) and the segmentation quality (i.e., how accurately the objects' boundaries were delineated). This metric is particularly useful in multiclass scenarios where each pixel is classified into distinct categories, such as different cell types in pathology images.

$PQ$ is calculated as the product of two terms: Detection Quality ($DQ$) and Segmentation Quality ($SQ$). It can be expressed as
\begin{equation*}
PQ = DQ \cdot SQ,
\end{equation*}
where
\begin{equation*}
DQ = \frac{TP}{TP + 0.5\, FP + 0.5\, FN},
\end{equation*}
\begin{equation*}
SQ = \frac{\sum_{(p, g) \in \mathcal{M}} IoU(p, g)}{TP}.
\end{equation*}
In these formulas, $TP$ denotes the number of correctly matched instances between ground truth and prediction, $FP$ denotes the predicted instances that have no corresponding ground truth, $FN$ denotes the ground truth instances that were not detected, $IoU(p, g)$ is the Intersection over Union for a pair of matched instances $p$ (prediction) and $g$ (ground truth), and $\mathcal{M}$ is the set of matched pairs.

The $PQ$ metric is calculated for each class and is averaged across classes to provide a global performance measure.

The $PQ$ score has a range of $[0, 1.0]$, where a higher score indicates better performance in both detecting and segmenting the instances correctly. A $PQ$ of 1 signifies perfect identification and segmentation of all instances, whereas a $PQ$ of 0 indicates that no instances were correctly identified and segmented.

\clearpage

\subsection{\label{chap:S6}Segmentation and Detection quality metrics for teacher and student models}

\begin{table}[h!]
\renewcommand{\arraystretch}{2.0}
\centering
\caption{Segmentation and detection quality for student and teacher models (CI 95\%)}
\label{tab:S6}
%\adjustbox{max width=\textwidth}{%
\begin{tabular}{|l|c|c|}
\hline
%\rowcolor{gray!30}
Metric & Teacher & Student \\
\hline
$SQ_{neoplastic}$ & 0.819 (0.815--0.823) & 0.824 (0.819--0.828) \\
\hline
$SQ_{lymphocyte}$ & 0.795 (0.788--0.802) & 0.790 (0.783--0.796) \\
\hline
$SQ_{connective}$ & 0.770 (0.762--0.776) & 0.780 (0.772--0.786) \\
\hline
$SQ_{dead}$ & 0.659 (0.623--0.688) & 0.657 (0.624--0.695) \\
\hline
$SQ_{epithelial}$ & 0.780 (0.770--0.790) & 0.788 (0.779--0.797) \\
\hline
$SQ_{macrophage}$ & 0.788 (0.760--0.810) & 0.757 (0.730--0.783) \\
\hline
$SQ_{neutrofil}$ & 0.782 (0.761--0.801) & 0.775 (0.759--0.792) \\
\hline
$DQ_{neoplastic}$ & 0.706 (0.692--0.719) & 0.727 (0.712--0.741) \\
\hline
$DQ_{lymphocyte}$ & 0.675 (0.656--0.698) & 0.713 (0.691--0.734) \\
\hline
$DQ_{connective}$ & 0.566 (0.546--0.584) & 0.583 (0.565--0.602) \\
\hline
$DQ_{dead}$ & 0.410 (0.361--0.465) & 0.435 (0.306--0.561) \\
\hline
$DQ_{epithelial}$ & 0.668 (0.639--0.694) & 0.673 (0.644--0.702) \\
\hline
$DQ_{macrophage}$ & 0.657 (0.583--0.727) & 0.615 (0.531--0.703) \\
\hline
$DQ_{neutrofil}$ & 0.691 (0.625--0.753) & 0.729 (0.679--0.778) \\
\hline
\end{tabular}
%
%}
\end{table}

\clearpage

\subsection{\label{chap:S7}QuPath integration method}
We adopt an integration strategy leveraging the paquo \cite{Bayer_AG} library, a Python package that enables direct interaction with QuPath’s internal API, thereby facilitating seamless data exchange without intermediate conversion steps. The data processing pipeline (\hyperref[fig:S7]{Appendix Figure S7}) begins with the acquisition of WSIs and their associated annotations from QuPath, which are represented as Shapely \cite{Gillies_Wel_etal._2024} polygons. Utilizing paquo, we directly read, create, and modify these annotations and detections within a QuPath project in the Python environment. Images are then cropped using these polygons and processed by cell segmentation and classification models employing standard vision processing toolkits such as OpenCV, pyvips, and PyTorch. Additionally, QuPath employs Groovy scripts to initiate a Python process that starts the entire pipeline from QuPath graphical interface: fetching polygons, extracting images from them, and running deep learning model inference on the cropped images. 
The results are returned to QuPath, leveraging paquo's Python bindings to manipulate QuPath data while minimizing the computational overhead typically associated with cross-environment communication.

\counterwithin{figure}{subsection}
\renewcommand{\thefigure}{S\arabic{subsection}}

\begin{figure}[h!]
    \centering
    \includegraphics[width=\textwidth]{images/A7.pdf}
    \caption{QuPath integration workflow using Python environment}
    \label{fig:S7}
\end{figure}

Compared to traditional workflows that involve exporting annotations as GeoJSON, classifying them in Python, and reimporting them into QuPath, our approach offers several advantages. We eliminate the need to switch between programming languages, providing a cohesive and streamlined development process entirely within QuPath software and removing the necessity to use other tools. Meanwhile, we avoid storing annotations as intermediate JSON files unless required for external use or archiving. By conducting the entire inference and post-processing workflow within the Python environment, we leverage the power and flexibility of Python libraries for image processing and machine learning. This approach also enables adjustments to any set of labels and models, thereby improving its applicability.

%\hfill

The distilled model and QuPath integration code are packaged into a Docker container, enabling streamlined execution with the Docker engine. Detailed integration code and deployment instructions can be found in the GitHub repository \cite{Shvetsov_2025b}.

Despite these benefits, we acknowledge that the paquo library is a proof‑of‑concept project in its early development stage and has not been tested across all versions of QuPath.

\clearpage

\subsection{\label{chap:S8}Data and code availability statement}
All datasets, models, and code used in this study are publicly available and can be obtained from the repositories listed below. 
The PanNuke \cite{Gamper_Koohbanani_etal._2019} and MoNuSAC \cite{Verma_Kumar_etal._2021} datasets are publicly accessible, and download information along with detailed descriptions can be found in their respective articles. Preprocessing scripts for PanNuke and MoNuSAC data, as well as individual cell extraction scripts, are available on GitHub \cite{Shvetsov_2025a}. The H-Optimus foundation model used in our experiments can be downloaded from the HuggingFace repository \cite{hoptimus2024}, and model information is available on GitHub \cite{Saillard_Jenatton_etal._2024}. In addition, the integration code for QuPath and the distilled model packaged in a Docker container are provided in the repository \cite{Shvetsov_2025b}, and paquo Python library is available from the authors GitHub repository \cite{Bayer_AG}.
\clearpage

\end{document}


\clearpage
\appendix

\subsection{Algorithm Illustration}
\label{apx:algorithm}
The whole pipeline of conRFT is outlined in Algorithm \ref{alg:conrft}.

\begin{algorithm}[ht]
    \caption{Procedure of ConRFT } 
    \begin{algorithmic}
        \REQUIRE A pre-trained VLA model $\pi_{\theta, \psi}$ with VLA parameter $\phi$ and a consistency head parameter $\psi$. A critic model $Q$ with parameter $\theta$. A pre-collected dataset $\mathcal{D}$ including 20-30 demonstrations. 
        Initialize batch size $B$.
        \STATE Randomly initialize the action head $\psi$ and the critic model $\theta$
        \STATE \textcolor{gray!90}{\# Stage I: Offline fine-tuning with Cal-ConRFT}
        \FOR{each offline training step}
            \STATE Sample $(s_t, a_t, r_t, s_{t+1})$ of $batch\_size$ from $\mathcal{D}$
            \STATE Update the action head $\psi$ and the critic model $\theta$ by Equation \ref{eq:calql_offline} and Equation \ref{eq:cpql_offline}.
        \ENDFOR
        \STATE \textcolor{gray!90}{\# Stage II: Online fine-tuning with HIL-ConRFT}
        \STATE \textcolor[RGB]{18,220,168}{\textbf{\# Start Policy Learning Thread}:}
        \STATE \textbf{Wait until} The number of transitions in $\mathcal{R}$ is at least 100
        \FOR{each online training step}
            \STATE Sample $(s_t, a_t, r_t, s_{t+1})$ of $\frac{B}{2}$ from $\mathcal{D}$ and $\mathcal{R}$
            \STATE Combine both minibatches to form batch of size $B$
            \STATE Update the action head $\psi$ and the critic model $\theta$ by Equation \ref{eq:ql_online} and Equation \ref{eq:cpql_online}.
        \ENDFOR
        \STATE \textcolor[RGB]{26,205,230}{\textbf{\# Start Interaction Thread}:}
        \FOR{each interaction step}
            \IF{no human intervention} 
                \STATE Take action $a_t \sim \pi_{\psi}(\cdot|s_t)$
                \STATE Store transition $(s_t,a_t,r_t,s_{t+1})$ in $\mathcal{R}$
            \ELSE 
                \STATE Take action $a_{intv}$
                \STATE Store transition $(s_t,a_{intv},r_t,s_{t+1})$ in $\mathcal{D}$
            \ENDIF
        \ENDFOR
    \end{algorithmic} 
    \label{alg:conrft}
\end{algorithm}

\subsection{Task Description, Setup and Policy Training Details}

\label{apx:tasks}
In this section, we provide hardware setup and training details for each task. The 6-dimensional action space refers to the 6-dimensional end-effector delta pose, and the 7-dimensional action space includes the 6-dimensional end-effector delta pose and 1-dimensional gripper control action. The learning rate is 3e-4, and the batch size is 256 for all tasks. 

For the consistency policy utilized for fine-tuning VLA models, we set $k \in [0.002, 80.0]$ and the number of sub-intervals $M=40$ where the sub-interval boundaries are determined with the formula $k_i=(\epsilon^{\frac{1}{\rho}}+\frac{i-1}{M-1}(T^{\frac{1}{\rho}}-\epsilon^{\frac{1}{\rho}}))^{\rho}$, where $\rho=7$. The network is based on a 2-layer multi-layer perceptron (MLP), with a hidden size of 256 and the Mish function serving as the activation function. 

For the diffusion policy, we use diffusion steps $K=5$, a cosine beta schedule, Resnet 18, and the $ LN_Resnet$ architecture, with a hidden size of 256 and $n=3$ blocks.  

For the reward of all tasks, we give $+10$ reward when the task is completed and a $-0.05$ reward on each step. For HIL-SERL, which uses a DQN network for gripper control, we give a $-0.2$ reward every time the policy opens/closes the gripper. 

\paragraph{Pick Banana}

This task involves picking up a banana in the basket and placing it on a green plate, which requires control of the gripper to move the fruit, as shown in Figure \ref{fig:tasks_detail}. It requires the policy to grasp and place the banana, ensuring it remains intact while avoiding collisions with the surrounding environment, such as the basket. We report more specific details of the policy training for this task in Table \ref{tab:task1_detail}. The task description for the VLA model is "Put the yellow banana on the green plate." 

\begin{table}[htbp]
    \centering
    \begin{tabular}{c|c}
        Parameter &Value\\
        \hline 
        Action space &7-dimensional\\
        Initial offline demonstrations &20\\
        Max episode length &100\\
        Reset method &Human reset\\
        Randomization range & 3 cm in x and y\\
        $(\alpha, \beta, \eta)$ for offline fine-tuning & $(0.01, 1.0, 0.1)$\\
        $(\beta, \eta)$ for online fine-tuning & $(0.5, 1.0)$\\
    \end{tabular}
    \caption{\textbf{Policy training details for the Pick Banana task.} }
    \label{tab:task1_detail}
\end{table}

\paragraph{Put Spoon}

This task involves picking up a spoon and placing it on a blue table linen, which requires the gripper to grasp and put the spoon, as shown in Figure \ref{fig:tasks_detail}. The challenge lies in the control needed to grasp the spoon. We report more specific details of the policy training for this task in Table \ref{tab:task2_detail}. The task description for the VLA model is "Put the spoon on the blue towel."

\begin{table}[htbp]
    \centering
    \begin{tabular}{c|c}
        Parameter &Value\\
        \hline 
        Action space &7-dimensional\\
        Initial offline demonstrations &20\\
        Max episode length &100\\
        Reset method &Human reset\\
        Randomization range & 3 cm in x and y\\
        $(\alpha, \beta, \eta)$ for offline fine-tuning & $(0.01, 1.0, 0.1)$\\
        $(\beta, \eta)$ for online fine-tuning & $(0.5, 1.0)$\\
    \end{tabular}
    \caption{\textbf{Policy training details for the Put Spoon task.} }
    \label{tab:task2_detail}
\end{table}

\paragraph{Open Drawer}

This task involves opening a drawer by grasping the handle and pulling it outward, as shown in Figure \ref{fig:tasks_detail}. It requires the policy to securely grip the handle and apply the correct force to open the drawer without damaging the hinges or surrounding area. We report more specific details of the policy training for this task in Table \ref{tab:task3_detail}. The task description for the VLA model is "Open the drawer."

\begin{table}[htbp]
    \centering
    \begin{tabular}{c|c}
        Parameter &Value\\
        \hline 
        Action space &6-dimensional\\
        Initial offline demonstrations &20\\
        Max episode length &100\\
        Reset method &Script reset\\
        Randomization range & 3 cm in y and x\\
        $(\alpha, \beta, \eta)$ for offline fine-tuning & $(0.01, 1.0, 0.1)$\\
        $(\beta, \eta)$ for online fine-tuning & $(0.5, 1.0)$\\
    \end{tabular}
    \caption{\textbf{Policy training details for the Open Drawer task.} }
    \label{tab:task3_detail}
\end{table}

\paragraph{Pick Bread}

This task involves picking up a slice of bread and placing it into a toaster, which requires control of the gripper to position the bread accurately without damaging it, as shown in Figure \ref{fig:tasks_detail}. The challenge lies in aligning the bread with the toaster's slot and lowering it, avoiding collisions with the toaster or the surrounding environment. We report more specific details of the policy training for this task in Table \ref{tab:task4_detail}. The task description for the VLA model is "Put the bread in the grey toaster."

\begin{table}[htbp]
    \centering
    \begin{tabular}{c|c}
        Parameter &Value\\
        \hline 
        Action space &7-dimensional\\
        Initial offline demonstrations &30\\
        Max episode length &100\\
        Reset method &Human reset\\
        Randomization range & 2 cm in x and y\\
        $(\alpha, \beta, \eta)$ for offline fine-tuning & $(0.01, 1.0, 0.1)$\\
        $(\beta, \eta)$ for online fine-tuning & $(0.5, 1.0)$\\
    \end{tabular}
    \caption{\textbf{Policy training details for the Pick Bread task.} }
    \label{tab:task4_detail}
\end{table}

\paragraph{Open Toaster}

This task involves pressing the button on a toaster to start the toasting process, as shown in Figure \ref{fig:tasks_detail}. It requires precise control of the gripper to avoid slipping or applying excessive force while ensuring that the button is pressed in a controlled and consistent manner. We report more specific details of the policy training for this task in Table \ref{tab:task5_detail}. The task description for the VLA model is "Press the black button and open the toaster."

\begin{table}[htbp]
    \centering
    \begin{tabular}{c|c}
        Parameter &Value\\
        \hline 
        Action space &6-dimensional\\
        Initial offline demonstrations &20\\
        Max episode length &100\\
        Reset method &Script reset\\
        Randomization range & 2 cm in y and z\\
        $(\alpha, \beta, \eta)$ for offline fine-tuning & $(0.01, 1.0, 0.1)$\\
        $(\beta, \eta)$ for online fine-tuning & $(0.5, 1.0)$\\
    \end{tabular}
    \caption{\textbf{Policy training details for the Open Toaster task.} }
    \label{tab:task5_detail}
\end{table}

\paragraph{Put Bread}

This task involves picking up a slice of toasted bread from the toaster and placing it on a white plate, as shown in Figure \ref{fig:tasks_detail}. The challenge lies in the precision required to grasp the toast without crushing or damaging it. The gripper must carefully move the toast from the toaster slot while avoiding contact with the toaster's edges or other objects nearby. We report more specific details of the policy training for this task in Table \ref{tab:task6_detail}. The task description for the VLA model is "Put the bread on the white plate."

\begin{table}[htbp]
    \centering
    \begin{tabular}{c|c}
        Parameter &Value\\
        \hline 
        Action space &7-dimensional\\
        Initial offline demonstrations &30\\
        Max episode length &120\\
        Reset method &Human reset\\
        Randomization range & 2 cm in x and y\\
        $(\alpha, \beta, \eta)$ for offline fine-tuning & $(0.01, 1.0, 0.1)$\\
        $(\beta, \eta)$ for online fine-tuning & $(0.5, 1.0)$\\
    \end{tabular}
    \caption{\textbf{Policy training details for the Put Bread task.} }
    \label{tab:task6_detail}
\end{table}

\paragraph{Insert Wheel}

This task involves installing wheels on the chair base by inserting pins into their corresponding slots, as shown in Figure \ref{fig:tasks_detail}. It is a contact-rich task requiring precise control to ensure the pins align correctly with the slots. The complexity of this task increases due to the tight tolerances and complex contact dynamics between the pin and the slot, making it a highly demanding task that requires precision and control. We report more specific details of the policy training for this task in Table \ref{tab:task8_detail}. The task description for the VLA model is "Insert the black wheel into the grey chair base."

\begin{table}[htbp]
    \centering
    \begin{tabular}{c|c}
        Parameter &Value\\
        \hline 
        Action space &7-dimensional\\
        Initial offline demonstrations &30\\
        Max episode length &100\\
        Reset method &Human reset\\
        Randomization range & 2 cm in x and y\\
        $(\alpha, \beta, \eta)$ for offline fine-tuning & $(0.01, 1.0, 0.1)$\\
        $(\beta, \eta)$ for online fine-tuning & $(0.5, 1.0)$\\
    \end{tabular}
    \caption{\textbf{Policy training details for the Insert Wheel task.} }
    \label{tab:task8_detail}
\end{table}

\paragraph{Hang Chinese Knot}

This task involves hanging a Chinese knot on a hook, which requires careful manipulation of a soft and dynamic object, as shown in Figure \ref{fig:tasks_detail}. The task requires fine dexterity to handle the knot's soft body and to maintain its structure while attaching it to the hook. The task involves dealing with the dynamics of soft object manipulation, where maintaining consistent contact and proper tension is critical for success. We report more specific details of the policy training for this task in Table \ref{tab:task9_detail}. The task description for the VLA model is "Hang the Chinese knot on the hook."

\begin{table}[ht]
    \centering
    \begin{tabular}{c|c}
        Parameter &Value\\
        \hline 
        Action space &7-dimensional\\
        Initial offline demonstrations &30\\
        Max episode length &100\\
        Reset method &Human reset\\
        Randomization range & 3 cm in y and z\\
        $(\alpha, \beta, \eta)$ for offline fine-tuning & $(0.01, 1.0, 0.1)$\\
        $(\beta, \eta)$ for online fine-tuning & $(0.5, 1.0)$\\
    \end{tabular}
    \caption{\textbf{Policy training details for the Hang Chinese Knot task.} }
    \label{tab:task9_detail}
\end{table}

\begin{figure*}[htbp]
    \centering
    \includegraphics[width=\linewidth]{tasks_detail.jpg}
    \caption{\textbf{Hardware setup and illustrations of camera views.} We give the illustrations of hardware setup and the corresponding camera views for all real-world tasks in this paper, including a) Pick Banana, b) Put Spoon, c) Open Drawer, d) Pick Bread, e) Open Toaster, f) Put Bread, g) Insert Wheel, h) Hand Chinese Knot.}
    \label{fig:tasks_detail}
\end{figure*}

\subsection{More experiment results}

\label{apx:more_exp}
In this section, we provide all policy learning curves for HIL-ConRFT on all tasks in Figure \ref{fig:all_result}. 

\begin{figure*}[ht]
\centering
\includegraphics[width=\linewidth]{all_result.png}
\caption{\textbf{Learning curves during online training for all tasks.} This figure presents the success rates, intervention rates, and episode lengths, displayed as a running average of over 20 episodes.}
\label{fig:all_result}
\end{figure*}

\end{document}


