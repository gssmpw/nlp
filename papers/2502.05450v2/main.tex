\documentclass[conference]{IEEEtran}
\usepackage{times}

% numbers option provides compact numerical references in the text. 
\usepackage[numbers]{natbib}
\usepackage{multicol}
\usepackage[bookmarks=true]{hyperref}

\usepackage{amssymb}
\usepackage{multirow}
\usepackage{makecell}
\usepackage{graphicx}
\usepackage{caption}
\usepackage{subcaption}
\usepackage{amsfonts, amsmath, amsthm}
\usepackage[table]{xcolor}
\usepackage{algorithmic}
\usepackage{algorithm}
\usepackage{marvosym}

\pdfinfo{
   /Author (Yuhui Chen, Shuai Tian, Shugao Liu, Yingting Zhou, Haoran Li and Dongbin Zhao)
   /Title (ConRFT: A Reinforced Fine-tuning Method for VLA Models via Consistency Policy)
   /CreationDate (D:20250413080000)
   /Subject (Robotics)
   /Keywords (Robotics;Reinforcement-learning)
}

\begin{document}

% paper title
\title{ConRFT: A Reinforced Fine-tuning Method for VLA Models via Consistency Policy}

\renewcommand{\authorrefmark}[1]{\textsuperscript{#1}}
% You will get a Paper-ID when submitting a pdf file to the conference system
% \author{Author Names Omitted for Anonymous Review. Paper-ID [394]}

\author{\authorblockN{Yuhui Chen\authorrefmark{1}\authorrefmark{2},
Shuai Tian\authorrefmark{1}\authorrefmark{2},
Shugao Liu\authorrefmark{1}\authorrefmark{2}, 
Yingting Zhou\authorrefmark{1}\authorrefmark{2}, 
Haoran Li\authorrefmark{1}\authorrefmark{2}\textsuperscript{\Letter}, and
Dongbin Zhao\authorrefmark{1}\authorrefmark{2}\textsuperscript{\Letter}}
\authorblockA{\authorrefmark{1}SKL-MAIS, Institute of Automation, Chinese Academy of Sciences, Beijing, China}
\authorblockA{\authorrefmark{2}School of Artificial Intelligence, University of Chinese Academy of Sciences, Beijing, China\\
Email: \{chenyuhui2022, tianshuai2023, liushugao2023, zhouyingting2025, lihaoran2015, dongbin.zhao\}@ia.ac.cn}}

\maketitle

\begin{abstract}

Vision-Language-Action (VLA) models have shown substantial potential in real-world robotic manipulation. However, fine-tuning these models through supervised learning struggles to achieve robust performance due to limited, inconsistent demonstrations, especially in contact-rich environments. In this paper, we propose a reinforced fine-tuning approach for VLA models, named ConRFT, which consists of offline and online fine-tuning with a unified consistency-based training objective, to address these challenges. In the offline stage, our method integrates behavior cloning and Q-learning to effectively extract policy from a small set of demonstrations and stabilize value estimating. In the online stage, the VLA model is further fine-tuned via consistency policy, with human interventions to ensure safe exploration and high sample efficiency. We evaluate our approach on eight diverse real-world manipulation tasks. It achieves an average success rate of 96.3\% within 45–90 minutes of online fine-tuning, outperforming prior supervised methods with a 144\% improvement in success rate and 1.9x shorter episode length. This work highlights the potential of integrating reinforcement learning to enhance the performance of VLA models for real-world robotic applications. Videos and code are available at our project website \href{https://cccedric.github.io/conrft/}{https://cccedric.github.io/conrft/}.

\end{abstract}

\IEEEpeerreviewmaketitle

\section{Introduction}

Recent advancements in training generalist robotic policies using Vision-Language-Action (VLA) models have demonstrated remarkable capabilities in understanding and executing various manipulation tasks. These successes are primarily attributed to large-scale imitation-style pre-training and grounding with robot actions \cite{o2023open, brohan2023rt, black2024pi_0}. While pre-trained policies capture powerful representations, they often fall short when handling the complexities of real-world scenarios \cite{jones2025beyond}. Fine-tuning with domain-specific data is essential to optimize model performance for downstream tasks \cite{black2024pi_0, wang2024scaling}. While Supervised Fine-Tuning (SFT) of the VLA model using human teleoperation data remains the predominant adaptation approach, this process faces significant challenges: the model's performance heavily relies on the quality and quantity of task-specific data. However, these human-collected datasets may not consistently provide optimal trajectories due to inherent issues such as sub-optimal data and inconsistent action \cite{xu2024rldg}.
% Human teleoperation, while accessible, introduces inconsistencies in task execution and stylistic variations in demonstrations, which makes it difficult for VLA models to learn effective policies. These issues become particularly pronounced in contact-rich manipulation tasks that require precise control and dexterity \cite{xu2024rldg}. Imitation learning alone often proves inadequate in addressing these challenges, leading to sub-optimal success rates, especially in unseen or complex scenarios.

Significant progress in Large-Language-Models (LLMs) and Vision-Language-Models (VLMs) have highlighted the value of reinforcement learning as a powerful tool for bridging the gap between policy capabilities and human preference \cite{christiano2017deep, ouyang2022training, luong2024reft} or improving model reasoning \cite{pang2024iterative}. In addition, deploying reinforcement learning (RL) with task-specific reward functions to learn from online interaction data is also a promising direction \cite{ramamurthy2022reinforcement, bai2024digirl, carta2023grounding}. However, extending these insights to VLA models presents unique challenges because, unlike LLMs, VLA models necessitate direct physical interaction in real-world robotic tasks. The safety and cost constraints of collecting data in contact-rich environments demand high sample efficiency and risk-aware exploration, making a straightforward implementation of RL infeasible. Recent work has attempted to leverage RL to address the challenges faced in SFT \cite{xu2024rldg, mark2024policy}, while these methods primarily focus on utilizing RL for data augmentation or quality improvement rather than directly optimizing VLA models through RL objectives. This limits the policy’s ability to explore states out of the demonstration dataset, thus undermining the potential benefits of RL-based fine-tuning in real-world settings. 

To leverage the benefits of RL-based techniques for efficiently fine-tuning VLA models with online interaction data, we propose a reinforced fine-tuning (RFT) approach consisting of offline and online stages with a unified consistency-based training objective. While this design is similar to offline-to-online methods \cite{lee2022offline, nakamoto2024cal, zhou2024efficient}, we found that expert demonstrations' scarcity constrains their offline training performance. Motivated by insights from CPQL \cite{chen2023boosting}, we propose a unified training objective that integrates supervised learning with Q-learning in the offline stage and further fine-tunes the VLA model via consistency policy through online RL. During offline training, our approach leverages prior demonstrations and handles out-of-distribution (OOD) states, effectively extracting the policy and value function before interacting with real-world environments. In the subsequent online stage, we solve two challenges of sample efficiency and real-world safety requirements by exploiting task-specific rewards with CPQL \cite{chen2023boosting} under human interventions through Human-in-the-Loop (HIL) learning \cite{kelly2019interactive, luo2024precise}. 

Our contributions are summarized as follows: 
\begin{enumerate}
    \item We present a \textbf{Con}sistency-based \textbf{R}einforced \textbf{F}ine-\textbf{T}uning (\textbf{ConRFT}) approach, a novel pipeline with the unified training objective both for offline and online fine-tuning.
    \item By integrating offline RL with a consistency-based behavior cloning (BC) loss, we propose Cal-ConRFT, which focuses on extracting an efficient policy and value function to provide a stable initialization with a small set of demonstrations.
    \item During online fine-tuning, we propose HIL-ConRFT, which retains the same loss structure from the offline stage for rapid policy adaption while leveraging human interventions to ensure safe exploration and high sample efficiency in real-world environments.
\end{enumerate}

We evaluate our approach on eight real-world manipulation tasks, demonstrating its ability to outperform state-of-the-art (SOTA) methods. Our framework achieves an average success rate of 96.3\% after 45–90 minutes of online fine-tuning, showcasing high sample efficiency. Additionally, it outperforms SFT methods that are trained on either human data or RL policy data, with an average success rate improvement of 144\% and an episode length of 1.9x shorter. 

\section{Related Work}

\subsection{Reinforced Fine-tuning for Large Models}

RL has been widely adopted for fine-tuning LLMs and VLMs. Early works have primarily focused on RL incorporating human feedback \cite{christiano2017deep, ouyang2022training, luong2024reft, casper2023open, zhai2024fine} by learning from human preferences or by integrating task-specific rewards without explicit human preference \cite{ramamurthy2022reinforcement, bai2024digirl, carta2023grounding, kimin2021pebbla}. While many of these approaches employ on-policy algorithms (e.g., PPO \cite{schulman2017proximal}) to fine-tune pre-trained policies \cite{bai2024digirl, gupta2019relay, shao2024deepseekmath}, they typically demand large amounts of interaction data to achieve desirable performance \cite{ball2023efficient, li2024selu}. While RL has demonstrated success in many domains, it typically learns within self-generated synthetic environments rather than real-world environments. This gap prevents direct transfer for VLA models, which require real-world interaction. Our work addresses this discrepancy by developing RL frameworks tailored for efficient real-world VLA fine-tuning.
% While RL has demonstrated remarkable success across various domains, a critical distinction warrants attention when applying RL to fine-tune LLMs and VLMs. Those RL paradigms typically involve agents interacting with self-generated synthetic environments rather than authentic real-world settings. This fundamental discrepancy precludes direct experience transfer when adapting VLA models, as their fine-tuning inherently requires physical environment interactions. Our research primarily addresses this paradigm shift by developing principled RL frameworks that enable efficient VLA model fine-tuning in real-world deployment scenarios.
% In contrast to LLM or VLM fine-tuning, whose learning objectives primarily involve text or image-based tasks, VLA fine-tuning operates on physical robots, where interactions with real-world environments can be expensive and potentially unsafe. Consequently, our work focus on off-policy actor-critic RL methods \cite{chen2023boosting} that offer higher sample efficiency, making them more practical in real-world scenarios with strict safety and resource constraints. We aim to address the fine-tuning challenges of adapting VLA models to complex, safety-critical tasks involving direct robotic manipulation.

\subsection{Real-world RL Systems}

Real-world robotic RL systems require algorithms that are both sample-efficient in handling high-dimensional inputs and flexible enough to accommodate practical considerations like reward specification and environment resets \cite{luo2024precise}. Several previous methods have effectively demonstrated policy learning directly in physical environments \cite{riedmiller2009reinforcement, johannink2019residual, luo2024serl, luo2024precise}, using both off-policy \cite{tony2022offline, luo2023rlif, hu2023reboot, russell2024continuously}, on-policy \cite{zhu2019dexterous, zhuang2023robot} methods, or posing "RL as supervised learning" \cite{mark2024policy, jan2007reinforcement}. Despite this progress, many real-world RL systems still demand prolonged training sessions or require large amounts of interaction data \cite{henery2020ingredients}, which can be impractical and risk-prone in contact-rich tasks. In contrast to previous methods that train from scratch, our work focuses on utilizing pre-trained VLA models to provide high-quality policy initialization. This approach effectively mitigates unnecessary exploratory behaviors in early RL phases, thereby optimizing both policy learning efficiency and operational safety in the training process.
% In contrast to these real-world RL setups that often train smaller models for quick policy convergence, our framework builds upon a pre-trained VLA model, providing an effective initialization to shorten the online training time. By integrating both offline and online RFT, we balance data efficiency with policy optimization while ensuring safe exploration under human interventions in real-world environments. 

\subsection{Offline-to-online Methods}

Offline-to-online RL aims to leverage offline datasets to initialize a policy, which is then fine-tuned via online interactions for improved sample efficiency \cite{lee2022offline}. Existing works commonly adopt an offline pre-training stage followed by an online fine-tuning stage \cite{lee2022offline, agarwal2022reincarnating, rafailov2023moto, nakamoto2024cal}, mixing offline and online data as training proceeds. This offline-to-online pipeline is similar to our proposed two-stage fine-tuning approach that exploits pre-collected data to bootstrap policy training and then fine-tunes the policy in the real-world tasks \cite{tony2022offline}. Most offline-to-online methods assume the availability of large-scale, diverse datasets with sufficient state coverage \cite{rajeswaran2017learning, nair2018overcoming}, a condition rarely met in real-world deployments. We explore leveraging pre-trained VLA models as the base policy to enable sample-efficient policy refinement, achieving superior fine-tuning performance even under stringent demonstration data constraints.
% Most offline-to-online methods typically presume the availability of large-scale, diverse datasets to ensure sufficient state coverage \cite{rajeswaran2017learning, nair2018overcoming}, which is a condition rarely satisfied in practical deployments. Our work addresses this fundamental mismatch. We explore leveraging pre-trained VLA models as the base policy to enable sample-efficient policy refinement, achieving superior fine-tuning performance even under stringent demonstration data constraints.
% However, while most Offline2Online methods assume large-scale offline datasets or well-aligned state distributions \cite{rajeswaran2017learning, nair2018overcoming} which is often collected in simulation, our real-world scenario is far more constrained: we have only a handful of demonstrations with insufficient state coverage for conventional offline RL approaches.

\section{Problem Setup and Preliminaries}

We focus on fine-tuning a pre-trained VLA model for downstream tasks. Specifically, we assume access to a pre-trained VLA model $\pi_{\phi_{\mathrm{pre}}}$, which encodes high-level representations from both visual inputs (e.g., RGB images) and language instructions. In supervised fine-tuning (SFT), we aim to adapt $\phi_{\mathrm{pre}}$ to $\phi$ on the target task using a small set of labeled demonstrations while preserving the model’s general feature-extraction capability. Formally, let $\tau = (s_0, a_0, \dots, s_H)$ be a trajectory for the target task, then the VLA model fine-tuning aims to solve $\min_{\phi}\mathcal{L}(\tau, \phi)$ where $\mathcal{L}$ may be a negative log-likelihood (NLL) or a mean-squared error (MSE) measuring the discrepancy between the predicted actions and those in the demonstration. This procedure allows us to effectively leverage compressed knowledge in robotic tasks while steering the VLA model to the downstream environment.

Since demonstrations are often limited, inconsistent, and sub-optimal, preventing the policy from covering diverse states, SFT struggles in real-world, contact-rich robotic tasks. To address these issues, we formulate each robotic task as a Markov Decision Process (MDP), where the goal of RL is to find the optimal policy in the MDP, $\mathcal{M} = (\mathcal{S}, \mathcal{A}, \mathcal{P}, r, \rho, \gamma)$, where $s \in \mathcal{S}$ denotes the state space and $a \in \mathcal{A}$ denotes the action space. $\mathcal{P}(s'|s, a)$ is the environmental transition probabilities that depend on the system dynamics, and $\rho(s)$ denotes the initial state distribution. $r(s, a)$ and $\gamma \in (0, 1)$ are the reward function and the reward discount factor. The policy $\pi$ is estimated by maximizing the cumulative expected value of the reward, denoted as $V^{\pi}(s)=\mathbb{E}_{\pi}[\sum_{t=0}^{H}\gamma^tr(s_t,a_t)|s_0=s,a_t\sim\pi(s_t),s_{t+1}\sim p(\cdot|s_t,a_t)]$. The Q-function of a given policy $\pi$ is denoted as $Q^{\pi}(s,a)=\mathbb{E}_{\pi}[\sum_{t=0}^{H}\gamma^tr(s_t,a_t)|s_0=s,a_0=a,s_{t+1}\sim p(\cdot|s_t,a_t)]$. $H$ represents the maximum episode step of a trajectory. By coupling the VLA policy with the learned Q-function, RFT allows the VLA model to refine its behavior based on trial-and-error interactions and task-specific feedback.

\begin{figure*}[ht]
\centering
\includegraphics[width=\linewidth]{method.png}
\caption{\textbf{Overview of ConRFT.} This figure illustrates the architecture of our reinforced fine-tuning approach for a pre-trained VLA model, which comprises two stages: the offline Cal-ConRFT and the online HIL-ConRFT. Both stages use a unified consistency-based training objective. During the offline stage, we only use pre-collected demonstrations for fine-tuning. During the online stage, a human operator can intervene in the robot policy via teleoperation tools(e.g. a SpaceMouse). And we use both pre-collected demonstrations, policy transitions, and human interventions for fine-tuning. }
\label{fig:method}
\end{figure*}

\section{Method}

The proposed pipline ConRFT consists of two stages: offline fine-tuning followed by online fine-tuning to optimize robotic policies, as shown in Fig. \ref{fig:method}. In the following sections, we provide a detailed description of the two stages, with the pipeline illustration in Appendix \ref{apx:algorithm}. 

\begin{figure*}[ht]
\centering
\includegraphics[width=\linewidth]{tasks.jpg}
\caption{\textbf{Overview of all real-world experimental tasks.} The real-world tasks include picking and placing (a) banana, (b) spoon, (d) and (f) bread, operating with (c) drawer and (e) toaster, assembling complex objects such as (g) chair wheel and (h) Chinese Knot. }
\label{fig:tasks}
\end{figure*}

\subsection{Stage I: Offline Fine-tuning with Cal-ConRFT}
Since pre-trained VLA models often lack zero-shot generalizability to novel robotic configurations, in the offline stage, we focus on training the policy using a small, pre-collected offline dataset (20–30 demonstrations) before transitioning to online reinforcement learning.
% This offline stage is essential because pre-trained VLA models often lack zero-shot performance on new robots or tasks, even a handful of demonstrations can align the model to the specific robotic setup and environment.
We initialize the policy with the pre-trained VLA model for reinforcement learning, reducing both the exploration burden and the overall online training time. Considering the ability to utilize offline data effectively, we choose the Calibrated Q-Learning (Cal-QL) \cite{nakamoto2024cal} as our base offline RL method since we want the Q-function to be robust to out-of-distribution (OOD) actions. Specifically, Cal-QL trains the Q-function on a pre-collected dataset by reducing temporal difference (TD) error and an additional regularizer. This regularizer penalizes Q-values for OOD actions when they exceed the value of the reference policy $V^{\mu}(s)$, while compensating for this penalization on actions observed within the offline dataset. The Cal-QL training objective for the critic is given by:
\begin{equation}
    \begin{aligned}
    \mathcal{L}_{Q}^{offline}(\theta)=&\alpha(\mathbb{E}_{s\sim\mathcal{D},a\sim\pi(\cdot|s)}[\max(Q_{\theta}(s,a),V^{\mu}(s))]\\
    &-\mathbb{E}_{s,a\sim\mathcal{D}}[Q_{\theta}(s,a)])\\
    &+\frac{1}{2}\mathbb{E}_{(s,a,s')\sim\mathcal{D}}[(Q_{\theta}(s,a)-\mathcal{B}^{\pi}\overline{Q}_{\overline{\theta}}(s,a))^2] \\
    \end{aligned}
    \label{eq:calql_offline}
\end{equation}
\noindent where $Q_{\theta}$ is the learned Q-function parameterized by $\theta$, $\overline{Q}_{\overline{\theta}}$ is the delayed target Q-function parameterized by $\overline{\theta}$. $\mathcal{B}^{\pi}\overline{Q}(s,a)=r(s,a)+\gamma\mathbb{E}_{a'\sim\pi(\cdot|s')}(\overline{Q}(s',a'))$ is the Bellman backup operator. $\alpha$ is a hyper-parameter to control the conservative penalty. And $\mathcal{D}$ is the demo buffer that stores demonstrations.

However, while Cal-QL is generally efficient at leveraging offline datasets, it struggles to train an effective policy when only small set of demonstrations (e.g., 20–30) are available. In such cases, limited state coverage leads to poor value estimates, making it difficult for the policy to generalize to unseen states. By contrast, typical offline RL datasets are often collected from multiple behavior policies, providing a broad range of state coverage to reduce the distribution shift. Lacking this breadth, the Cal-QL loss alone may not adequately guide the learning process, resulting in poor performance.

To address this issue, we propose augmenting the offline training process by incorporating a BC loss. The BC loss directly minimizes the difference between the actions generated by the policy and those from the demonstrations. By incorporating BC loss, we encourage the model to imitate the behaviors from the demonstrations, providing additional supervisory signals during the offline stage. This helps the VLA model to learn a more effective policy and initialize a stable Q function with few demonstrations, especially in the case of contact-rich manipulation tasks where control precision is critical. 

Motivated by combining the BC loss with Q guidance under a consistency-based objective \cite{chen2023boosting}, we introduce Cal-ConRFT in the offline stage. This approach employs a consistency policy as the action head for fine-tuning the VLA model, addressing two key concerns: 1) it helps leverage inconsistent and sub-optimal demonstrations that often arise in pre-collected data, and 2) compared to diffusion-based action head, the consistency-based action head remains computationally lightweight for efficient inference \cite{chen2023boosting, xing2025consistency, prasad2024consistency}. The consistency policy is a diffusion-model-based policy \cite{li2024stabilizing} that learns to map random actions sampled from the unit Gaussian to generate actions drawn from the expert action distribution conditioned on the current state. For the consistency policy, we discretize the diffusion horizon $[\epsilon,K]$ into $M$ sub-intervals with boundaries $k_1=\epsilon \le k_2 \le \cdots \le k_m = K$ and $\epsilon = 0.002$. Specifically, the VLA model with a consistency policy as the action head is given by:
\begin{equation}
    \begin{aligned}
    \pi_{\psi}(a|s) &= f_{\psi}(a^k,k|E_{\phi}(s))\\
    \end{aligned}
    \label{eq:cp}
\end{equation}
\noindent where $f$ denotes the consistency policy parameterized with $\psi$, subscripts $k$ denoted the diffusion step, $a^k\sim\mathcal{N}(0,kI)$ and $E_{\phi}(s)$ denotes the encoded state of the pre-trained VLA model parameterized with $\phi$. The consistency-based training objective for VLA model fine-tuning is given by:
\begin{equation}
    \begin{aligned}
    &\mathcal{L}_{\pi}^{offline}(\psi) = \beta\mathcal{L}_{\pi}^{BC} + \eta\mathcal{L}_{\pi}^{Q}\\
    \end{aligned}
    \label{eq:cpql_offline}
\end{equation}

% \begin{equation}
%     \begin{aligned}
%     &\mathcal{L}_{\pi}^{offline}(\psi) = -\eta\mathbb{E}_{s\sim\mathcal{D},a\sim\pi_{\psi}}[Q(s,a)] \\ +& \beta\mathbb{E}_{(s,a)\sim\mathcal{D},m\sim\mathcal{U}[1,M-1]}[d(f_{\psi}(a+k_mz,k_m|E(s)),a)]\\
%     \end{aligned}
%     \label{eq:cpql_offline}
% \end{equation}
\noindent where BC loss $\mathcal{L}_{\pi}^{BC}=\mathbb{E}_{(s,a)\sim\mathcal{D},m\sim\mathcal{U}[1,M-1]}[d(f_{\psi}(a+k_mz,k_m|E(s)),a)]$, $z\sim\mathcal{N}(0,I)$, $d$ stands for the Euclidean distance $d(x,y)=||x-y||_2$, and Q loss $\mathcal{L}_{\pi}^{Q}=-\mathbb{E}_{s\sim\mathcal{D},a\sim\pi_{\psi}}[Q(s,a)]$. $\beta$ and $\eta$ are two hyper-parameters to balance the BC loss and Q loss. This combination enables efficient policy learning and stable value estimation, even with a small set of demonstrations, by aligning value estimates with expert actions and improving policy performance during offline training. Moreover, it provides a reliable initialization for the online stage, facilitating safe and effective exploration.

\begin{table*}
    \centering
    \scalebox{0.87}{
    \begin{tabular}{r|c|ccccc|ccccc}
    &\multirow{2}{*}{\makecell[c]{\textbf{Training}\\\textbf{Time}\\\textbf{(mins)}}} &\multicolumn{5}{c}{\textbf{Success Rate (\%)}} &\multicolumn{5}{c}{\textbf{Episode length}}  \\
    \textbf{Task} & &\makecell[c]{SFT\cite{team2024octo}} &\makecell[c]{Cal-\\ConRFT} &\makecell[c]{HG-\\DAgger\cite{kelly2019interactive}} &PA-RL\cite{mark2024policy} &\makecell[c]{HIL-\\ConRFT} &\makecell[c]{SFT\cite{team2024octo}} &\makecell[c]{Cal-\\ConRFT} &\makecell[c]{HG-\\DAgger\cite{kelly2019interactive}} &PA-RL\cite{mark2024policy} &\makecell[c]{HIL-\\ConRFT} \\ 
    \hline
    Pick Banana       &45   &40   &50   &60 (+50\%)   &80 (+100\%)  &\textbf{90} (+80\%)    &63.7 &57.8 &67.5 (0.9x) &56.1 (1.1x) &\textbf{51.2} (1.1x) \\
    Put Spoon         &45   &50   &55   &90 (+80\%)   &80 (+60\%)   &\textbf{100} (+82\%)   &49.9 &57.2 &50.9 (1.0x) &45.3 (1.1x) &\textbf{22.6} (2.5x) \\
    Open Drawer       &15   &35   &30   &80 (+129\%)  &60 (+71\%)   &\textbf{100} (+233\%)  &63.6 &61.7 &48.4 (1.3x) &57.1 (1.1x) &\textbf{32.4} (1.8x) \\
    Pick Bread        &45   &65   &55   &65 (+0\%)    &80 (+23\%)   &\textbf{100} (+82\%)   &53.2 &49.1 &65.6 (0.8x) &51.7 (1.0x) &\textbf{31.6} (1.6x) \\
    Open Toaster      &30   &30   &30   &75 (+116\%)  &100 (+233\%) &\textbf{100} (+233\%)  &51.2 &50.7 &43.4 (1.2x) &34.3 (1.5x) &\textbf{22.1} (2.3x) \\
    Put Bread         &60   &5    &20   &60 (+1100\%) &75 (+1400\%) &\textbf{100} (+400\%)  &102  &84.8 &74.2 (1.4x) &72.1 (1.4x) &\textbf{36.6} (2.3x) \\
    Insert Wheel      &60   &35   &35   &40 (+14\%)   &30 (-14\%)   &\textbf{80} (+129\%)   &42.7 &43.4 &53.0 (0.8x) &47.4 (0.9x) &\textbf{21.9} (2.0x) \\
    Hang Chinese Knot &90   &55   &40   &50 (-10\%)   &65 (+18\%)   &\textbf{100} (+150\%)  &52.6 &54.9 &47.5 (1.1x) &44.4 (1.3x) &\textbf{26.8} (2.0x) \\
    \hline
    \rowcolor{gray!20} \textbf{Average}  &48.8 &39.4 &39.4 &65 (+65\%) &71.3 (+81\%) &\textbf{96.3} (+144\%) &59.9 &57.5 &56.3 (1.1x) &51.1 (1.2x) &\textbf{30.7} (1.9x) \\
    \end{tabular}}
    \caption{\textbf{All experiment results for various offline and online fine-tuning methods.} We report the policy performance against various baselines after offline fine-tuning (SFT \cite{team2024octo} and Cal-ConRFT) and after online fine-tuning (HG-DAgger \cite{kelly2019interactive}, PA-RL \cite{mark2024policy} and HIL-ConRFT), including success rates and average episode lengths for various tasks. Specifically, for online fine-tuning, HG-DAgger, and PA-RL training starts from the SFT baseline, and HIL-ConRFT training starts from the Cal-ConRFT baseline. The performance improvement is relative to corresponding offline results. Policies are trained using the same number of online episodes with human interventions for all methods. All metrics are reported over 20 trials per task. }
    \label{tab:exp_result}
\end{table*}

\subsection{Stage II: Online Fine-tuning with HIL-ConRFT}

While the offline stage provides an initial policy from a small set of demonstration data, its performance is limited by the scope and quality of the pre-collected demonstrations. Therefore, we have the online stage with HIL-ConRFT, where the VLA model is further fine-tuned online via the consistency policy through interacting with the real-world environment. During online training process, the demo buffer $\mathcal{D}$ for offline stage is remained. Furthermore, we have a replay buffer $\mathcal{R}$ to store online data, then implement symmetric sampling \cite{ball2023efficient}, whereby for each batch, we sample equally between these two buffers to form each training batch. Since the VLA model continuously gathers new transitions based on its current policy, the data distribution naturally evolves with the policy. This ongoing interaction reduces the distribution-shift problem that the offline stage faces. As a result, we use a standard Q loss for online critic updating: 
\begin{equation}
    \begin{aligned}
    \mathcal{L}_{Q}^{online}(\theta)&=\mathbb{E}_{(s,a,s')\sim(\mathcal{D}\cup\mathcal{R})}[(Q_{\theta}(s,a)-\mathcal{B}^{\pi}\overline{Q}(s,a))^2] \\
    \end{aligned}
    \label{eq:ql_online}
\end{equation}
The consistency-based training objective for VLA model fine-tuning is given by:
\begin{equation}
    \begin{aligned}
    &\mathcal{L}_{\pi}^{online}(\psi) = \beta\mathcal{L}_{\pi}^{BC} + \eta\mathcal{L}_{\pi}^{Q}
    \end{aligned}
    \label{eq:cpql_online}
\end{equation}
\noindent where BC loss $\mathcal{L}_{\pi}^{BC}=\mathbb{E}_{(s,a)\sim(\mathcal{D}\cup\mathcal{R}),m\sim\mathcal{U}[1,M-1]}[d(f_{\psi}(a+k_mz,k_m|E(s)),a)]$, $z\sim\mathcal{N}(0,I)$, $d$ stands for the Euclidean distance $d(x,y)=||x-y||_2$, and Q loss $\mathcal{L}_{\pi}^{Q}=-\mathbb{E}_{s\sim(\mathcal{D}\cup\mathcal{R}),a\sim\pi_{\psi}}[Q(s,a)]$. Note that this objective closely mirrors Equation \ref{eq:cpql_offline} from the offline stage, enabling a quick adaption to online fine-tuning.
% \begin{equation}
%     \begin{aligned}
%     &\mathcal{L}_{\pi}^{online}(\psi) = -\eta\mathbb{E}_{s\sim(\mathcal{D}\cup\mathcal{R}),a\sim\pi_{\psi}}[Q(s,a)] \\
%     +& \beta\mathbb{E}_{(s,a)\sim(\mathcal{D}\cup\mathcal{R}),m\sim\mathcal{U}[1,M-1]}[d(f_{\psi}(a+k_mz,k_m|E(s)),a)]\\
%     \end{aligned}
%     \label{eq:cpql_online}
% \end{equation}

Typically, we decrease the BC loss weight $\beta$ while increasing the Q loss weight $\eta$ during the online stage, yet we keep the BC loss for two main reasons. 1) Firstly, it ensures the policy continues to align with the demonstration data, preventing drastic deviations that could lead to performance collapse. This is important for maintaining the quality of actions in contact-rich manipulation tasks, where sudden changes in the policy can result in unsafe or inefficient behaviors. 2) Secondly, since reinforcement learning inherently involves exploration, it can become unstable in high-dimensional state-action spaces. By providing a stabilizing effect on exploration \cite{li2024generalizing}, the BC loss prevents the policy from deviating too far from its offline baseline, thereby reducing the risk of inefficient or unsafe behaviors. This aspect is important in real-world robotic training, especially in physical environments where unsafe actions can lead to damage or other hazards.

Also, we integrate human interventions into the online stage through Human-in-the-Loop learning. Specifically, HIL learning allows for timely interventions by a human operator who can provide corrective actions during the exploration process, which will then take over the control of the robot from the VLA model. These human corrections are added to the demo buffer $\mathcal{D}$, offering high-level guidance that steers exploration in a safer and more efficient direction \cite{huihan2023robot}. Human interventions are essential when the robot engages in destructive behaviors, such as colliding with obstacles, applying excessive force, or damaging the environment. In addition to ensuring safe exploration, human interventions accelerate policy convergence. In scenarios where the policy leads the robot into an unrecoverable or undesirable state or when the robot becomes stuck in a local optimum that would otherwise require significant time and steps to overcome without external assistance, the human operator can step in to correct the robot’s actions and guide it towards safer and more effective behavior. This results in a stable learning process, where the VLA model is fine-tuned quicker and more safely than it would through autonomous exploration alone. 

\section{Experiment and Results}

In this section, we validate the proposed fine-tuning framework through real-world experiments. We first present the experimental setup and the results for various baselines and then discuss these results and their implications.

\begin{figure*}[ht]
\centering
\includegraphics[width=\linewidth]{online_result.png}
\caption{\textbf{Learning curves during online training.} This figure presents the success rates, intervention rates, and episode lengths for HIL-SERL \cite{luo2024precise}, HG-DAgger \cite{kelly2019interactive}, PA-RL \cite{mark2024policy} and our method across five representative real-world tasks, displayed as a running average over 20 episodes. PA-RL is implemented without human intervention. Note that human interventions may lead the policy to successful outcomes, and thus, the actual policy success rate when interventions exist might be lower than the curve suggests. }
\label{fig:online_result}
\end{figure*}

\subsection{Overview of Experiments}

Our experiments aim to evaluate our approach's effectiveness and efficiency for fine-tuning VLA models in real-world scenarios. To this end, we perform real-world experiments across eight diverse manipulation tasks, as illustrated in Figure \ref{fig:tasks}. These tasks are designed to reflect a variety of manipulation challenges, including object placement tasks (e.g., placing bread into a toaster and putting bread on a white plate), precise and contact-rich manipulation (e.g., aligning and inserting a wheel into the chair base), and dynamic object handling (e.g., hanging a Chinese Knot). To validate our fine-tuning approach, we select the Octo-small model \cite{team2024octo} for its balance of performance and inference efficiency, and employ a consistency policy \cite{prasad2024consistency} as the action head on a 7-DoF Franka Emika robot arm.

For all tasks, the state observation includes two RGB images captured from a wrist-mounted camera (128 × 128) and a side camera (256 × 256), in combination with the robot's proprioceptive state of the robot arm, including end-effector poses, twists, forces/torques, and gripper status. The action space is defined as either a 6-dimensional end-effector delta pose for the downstream impedance controller or a 7-dimensional target that includes 1-dimensional binary gripper action, additionally for tasks that involve grasping. Data collection and policies command actions at 10Hz. Before training, positive and negative demonstrations are collected from human operators to train a binary classifier that gives binary feedback on whether the corresponding task is done successfully or not for each task. Additionally, each task's initial state is randomized using either a scripted robot motion or manual resets by a human operator. We present descriptions of each task in our real-world experiments and more details on the experiment tasks, training, and evaluation procedure in the Appendix \ref{apx:tasks}.

\subsection{Experimental Results}

In this section, we provide the experimental results for all tasks as shown in Figure \ref{fig:tasks}. For each task, we report result metrics, including the success rate, episode length, and total training time in Table \ref{tab:exp_result}. The training time includes the duration of scripted motions, policy rollouts, and onboard computations, all of which are conducted using an NVIDIA RTX A6000 GPU. For the offline stage, we compare Cal-ConRFT and SFT, where the SFT uses an NLL loss for behavior cloning \cite{team2024octo}. For the online stage, we compared HIL-ConRFT with multiple baselines, including HG-DAgger \cite{kelly2019interactive} that incorporates human corrections to fine-tune the policy through supervised learning, PA-RL \cite{mark2024policy} that optimized actions through a policy-agnostic Q-function and fine-tune the policy through supervised learning with the optimized actions. We also compare HIL-SERL \cite{luo2024precise} that trains a RL policy with human interventions from scratch, and RLDG \cite{xu2024rldg} that fine-tunes the VLA model using SFT \cite{team2024octo} with demonstrations collected by RL policy. 

\subsubsection{ConRFT Outperforms Supervised Methods}

We compare different approaches for supervised and reinforced methods in Table \ref{tab:exp_result} and present the corresponding online learning curves in Figure \ref{fig:online_result}. Our approach, ConRFT, achieves the highest average success rate of 96.3\% after 45 to 90 minutes of real-world training across all tasks, representing a 144\% improvement over the supervised baseline. It outperforms SOTA methods such as HG-DAgger and PA-RL, with average success rates of 65\% and 71.3\%. While HG-DAgger leverages human corrections to fine-tune the VLA model through supervised learning, it fails to achieve significant policy improvement and even experiences a performance drop on some tasks due to the sub-optimality and inconsistency of human corrections. For example, we observe that for contact-rich tasks that require precise, careful manipulation, such as Insert Wheel and Hang Chinese Knot, HG-DAgger has limited policy improvement after online fine-tuning. Specifically, in the Hang Chinese Knot task, the careful handling of soft objects demands consistent and precise controls. The inherent variability in human corrections, such as differences in the angle of insertion, introduces noise and conflicting information into the training process. This inconsistency prevents the policy's ability to learn precise and dexterous behaviors. Additionally, the complexity of contact dynamics means that minor deviations in the policy can result in significant performance drops, further exacerbating the challenges posed by inconsistent human corrections. 

In the absence of human corrections, PA-RL offers a direct action optimization using a policy-agnostic Q-function trained through Cal-QL. By optimizing actions based on reward signals, PA-RL overcomes the sub-optimality of human corrections and demonstrates more stable policy improvement in simpler tasks such as Pick Banana and Put Spoon. However, it fails to improve the policy performance in contact-rich tasks that require precise, careful manipulation, such as Insert Wheel. Precise alignment and controlled insertion forces are critical in the Insert Wheel task. However, due to the limited state coverage in the demo buffer and replay buffer, the policy-agnostic Q-function is unable to generalize effectively to different wheel and slot positions. This limits the policy's ability to handle the slight variations in state transitions required for successful insertion, leading to sub-optimal performance in complex manipulation scenarios. Consequently, while PA-RL shows promise in simple environments, it struggles to scale to complex tasks demanding high precision and dexterity.

These observations underscore the advantages of our proposed approach, which effectively mitigates the issues associated with inconsistent human corrections and limited state coverage by reinforcement learning. Our method, ConRFT, effectively and safely explores a broad range of states and directly optimizes the policy using task-specific rewards, thereby demonstrating high sample efficiency and mitigating the impact of inconsistent human corrections. This stability and performance highlight the effectiveness of our approach in overcoming the limitations of existing fine-tuning methods in real-world robotic applications.

\begin{table}[t]
    \centering
    \scalebox{0.82}{
    \begin{tabular}{r|c|cc|cc}
         &\multirow{2}{*}{\makecell[c]{\textbf{Training}\\\textbf{Time}\\\textbf{(mins)}}} &\multicolumn{2}{c}{\textbf{Success Rate (\%)}} &\multicolumn{2}{c}{\textbf{Episode length}}  \\
        \multirow{2}{*}{\textbf{Task}} & &\makecell[c]{HIL-\\SERL\cite{luo2024precise}} &\makecell[c]{HIL-\\ConRFT} &\makecell[c]{HIL-\\SERL\cite{luo2024precise}} &\makecell[c]{HIL-\\ConRFT} \\ 
        \hline
        Pick Banana       &45   &0 $\rightarrow$ 15           &50 $\rightarrow$ \textbf{90}  &\textbf{30.6} &51.2 \\
        Put Spoon         &45   &0 $\rightarrow$ 60           &55 $\rightarrow$ \textbf{100} &56.1          &\textbf{22.6} \\
        Open Drawer       &15   &0 $\rightarrow$ 10           &30 $\rightarrow$ \textbf{100} &67.5          &\textbf{32.4} \\
        Pick Bread        &45   &0 $\rightarrow$ 45           &55 $\rightarrow$ \textbf{100} &\textbf{22.0} &31.6 \\
        Open Toaster      &30   &0 $\rightarrow$ \textbf{100} &30 $\rightarrow$ \textbf{100} &28.1          &\textbf{22.1} \\
        Put Bread         &60   &0 $\rightarrow$ 5            &20 $\rightarrow$ \textbf{100} &62.0          &\textbf{36.6} \\
        Insert Wheel      &60   &0 $\rightarrow$ 5            &35 $\rightarrow$ \textbf{80}  &42.0          &\textbf{21.9} \\
        Hang Chinese Knot &90   &0 $\rightarrow$ 15           &40 $\rightarrow$ \textbf{100} &57.3          &\textbf{26.8} \\
        \hline
        \rowcolor{gray!20} \textbf{Average}  &48.8 &0 $\rightarrow$ 31.9 &39.4 $\rightarrow$ \textbf{96.3} &45.7 &\textbf{30.7} \\
    \end{tabular}}
    % \vspace{1em}
    \caption{\textbf{Experiment results for training from scratch (HIL-SERL \cite{luo2024precise}) and fine-tuning VLA (HIL-ConRFT).} Policies are trained using the same number of episodes with human interventions. All metrics are reported over 20 trials per task.}
    \label{tab:offline_rl_result}
\end{table}

Another critical metric for evaluating policy performance is the episode length, which represents the total number of steps the policy takes to complete a task. As shown in Table \ref{tab:exp_result}, the VLA model fine-tuned with HIL-ConRFT achieves an average episode length of 30.7 steps, demonstrating a 1.9x shorter than the offline baselines. In contrast, HG-DAgger achieves an average episode length of 56.3 steps, which is only 1.1x shorter than the offline baseline. Similarly, PA-RL attains an average episode length of 51.1 steps. It lacks policy exploration due to the conservative nature of the policy-agnostic Q-function, preventing it from effectively optimizing how to complete the task more quickly or trying more efficient behaviors. 

These results illustrate that ConRFT can effectively exploit the dynamic characteristics of MDPs to optimize the VLA model via consistency policy for maximizing the discounted sum of rewards. They also show the limitations of supervised methods in handling sub-optimal data and efficient policy exploration. By encouraging policies to obtain rewards more quickly, our approach results in shorter episode lengths than supervised methods relying solely on imitating demonstrations. This enhanced sample efficiency and reduced episode length highlight the advantages of ConRFT for fine-tuning VLA models in real-world robotic applications.

\subsubsection{Fine-tuning VLA Outperforms Training From Scratch}

Reinforcement learning from scratch typically demands extensive interaction with the environment and frequent human interventions, which can lead to a lengthy training process and high safety risks. For instance, HIL-SERL \cite{luo2024precise}, an approach that trains policies through RL from scratch with human interventions, fails to converge to an effective policy within the same training duration as our approach, reaching an average success rate of only 31.9\% as shown in Table \ref{tab:offline_rl_result}. The learning curves in Figure \ref{fig:online_result} reveal that HIL-ConRFT consistently improves policy performance during the online stage. While HIL-SERL can achieve optimal policies eventually, it usually requires over two hours of online training with a higher intervention rate for each task, resulting in more destructive behaviors during exploring (e.g., collisions with the environment), especially in the early stage of training.

In contrast, starting from a pre-trained VLA model and performing offline fine-tuning reduces the online training time and improves sample efficiency. Building upon offline initialized policy, ConRFT accelerates the policy convergence and enhances the final performance. As a result, fine-tuning VLA models via consistency policy enables them to reach higher success rates more quickly and with fewer interventions compared to training entirely from scratch, demonstrating the benefits of leveraging pre-trained VLA models in real-world robotic applications.

\begin{figure}[t]
\centering
\includegraphics[width=\linewidth]{abla_bc.png}
\caption{\textbf{Learning curves for HIL-ConRFT online fine-tuning from SFT \cite{team2024octo} and Cal-ConRFT baselines.} This figure presents success and intervention rates across two representative tasks, displayed as a running average over 20 episodes. }
\label{fig:abla_bc}
\end{figure}

\subsubsection{Analysis}

\paragraph{Why fine-tuning from Cal-ConRFT rather than SFT or Cal-QL?}

As illustrated in Table \ref{tab:exp_result}, we observe that during the offline stage, the performance of Cal-ConRFT is similar to that of the SFT baseline. This observation raises the question of why Q loss should be introduced during the offline stage. The reason is that when the offline stage relies solely on SFT, the fine-tuned policy benefits from imitation learning but may require substantial online fine-tuning to handle states and actions not covered by the offline dataset. In contrast, incorporating Q loss during the offline stage allows the early Q-value estimations to provide initialization for policy improvement, facilitating quicker adaptation during online fine-tuning. This approach helps address potential biases and ensures more stable learning. Moreover, in scenarios with a small set of demonstrations, we find that relying on Cal-QL alone is insufficient to train an effective policy, resulting in a 0\% success rate on all tasks. The lack of data affects the policy's ability to accurately estimate Q-values, resulting in weak performance after the offline stage and longer training time in the online stage. 

We compare the online fine-tuning curves starting from Cal-ConRFT and the SFT baseline on two representative tasks to investigate further the impact of introducing Q loss, as shown in Figure \ref{fig:abla_bc}. Although both curves begin with similar success rates, the higher intervention rate observed during training from the SFT baseline indicates that the SFT-trained policy experiences severe policy forgetting in the early stages of online training. This suggests that Cal-ConRFT enables quicker adaptation of the online learning process by leveraging the Q loss during the offline stage, allowing more effective and stable policy improvement with a small set of demonstration data. 

% \paragraph{Does our two-stage approach face performance drop when transitioning to online stage?}

% Yes, we still observe a performance drop when transitioning to online stage, since the performance is evaluated with sparse reward signals that only indicate success or failure of the whole task. Human interventions can correct the policy and lead to successful outcomes, thus the success rate in Figure \ref{fig:online_result} does not actually reflect the policy’s real performance. The policy have to experience both successes and failures in order to estimate the accurate value function. For instance, in the object placement tasks such as Pick Banana and Put Spoon, where the policy needs to (1) pick the object, (2) move it to a target position, and (3) place it, any sub-goal failure results in an overall task failure. During online training, we observe that while the initial policy can often successfully perform the initial pick, the policy in the early-stage training usually explores actions to finish the task quicker, such as prematurely closing the gripper before reaching the object. This leads to more frequent failures in the pick phase and, consequently, a drop in overall success rate. Nonetheless, these failure cases are essential for the policy to learn a better value function. Over repeated interactions, the policy uses these transitions for training and eventually surpass the initial performance, discovering a more robust policy for completing the task.

\begin{table}[t]
    \centering
    \scalebox{0.88}{
    \begin{tabular}{r|ccccc}
          &\multicolumn{5}{c}{\textbf{Success Rate (\%)}}\\
        \textbf{Task}     &DP\cite{chi2023diffusion} &SFT\cite{team2024octo} &RLDG\cite{xu2024rldg} &Cal-ConRFT &HIL-ConRFT\\ 
        \hline
        Put Spoon         &60   &70   &\textbf{100} &55         &\textbf{100} \\
        Put Bread         &30   &65   &\textbf{100} &20         &\textbf{100} \\
        Insert Wheel      &35   &40   &50           &35         &\textbf{80}  \\
        \hline
        \rowcolor{gray!20} \textbf{Average} &41.7 &58.3 &83.3 &36.7 &\textbf{93.3} \\
    \end{tabular}}
    \caption{\textbf{Experimental comparisons with various demonstrations.} Diffusion Policy (DP) \cite{chi2023diffusion} and SFT \cite{team2024octo} are trained with 150 demonstrations collected by human teleoperation, while RLDG \cite{xu2024rldg} is trained with 150 demonstrations collected by RL policy. Cal-ConRFT is trained with 20 demonstrations collected by human teleoperation, and HIL-ConRFT is trained with 20 demonstrations as well as 80-120 policy-generated rollout trajectories. All metrics are reported over 20 trials per task.}
    \label{tab:offline_bc_result}
\end{table}

\paragraph{Does increasing the number of demonstrations enhance policy performance for SFT?}

Typically, during a 45-60 minutes online fine-tuning stage, the policy collects approximately 80 to 120 successful and failed trajectories. To ensure a fair comparison between our approach and supervised training methods, we further compare training Diffusion Policy (DP) \cite{chi2023diffusion} and supervised fine-tuning VLA \cite{team2024octo} using 150 demonstrations on three representative tasks, which aligns with the total number of demonstrations utilized by our approach. Additionally, we compare RLDG \cite{xu2024rldg} with fine-tuning using 150 demonstrations collected by RL policy. As shown in Table \ref{tab:offline_bc_result}, even though DP and SFT benefit from a larger quantity of demonstrations, their success rates still fail to match the performance of our method, especially on contact-rich tasks such as Insert Wheel. This indicates that simply adding more human-collected demonstrations with supervised learning does not necessarily guarantee higher performance due to the inconsistent and sub-optimal actions inherent in human-collected data. Meanwhile, RLDG achieves higher success rates using optimal data collected from RL policies, suggesting that the consistency of these RL-collected data can improve the final performance. On the other hand, our method directly fine-tune the policy by optimizing the consistency-based training objective, achieving the highest success rate.

\begin{table}[t]
\centering
\scalebox{1.0}{
\begin{tabular}{r|cc}
     &\multicolumn{2}{c}{\textbf{Success Rate (\%)}} \\
    \textbf{Task}     &Kosmos-2(1.6B)     &PaliGemma(3B) \\
    \hline
    Pick Banana       &60$\rightarrow$100 &65$\rightarrow$100 \\
    Put Spoon         &55$\rightarrow$100 &30$\rightarrow$100 \\
    Hang Chinese Knot &45$\rightarrow$100 &60$\rightarrow$100 \\
    \hline
    \rowcolor{gray!20} \textbf{Average} &53.3$\rightarrow$100 &51.7$\rightarrow$100 \\
\end{tabular}
}
\caption{\textbf{Experimental results of ConRFT on different VLA models.} We fine-tune RoboVLM \cite{li2024towards} with two VLM backbones using our method. Specifically, we fine-tune only the action head while keeping the visual encoders and transformer backbone frozen. All metrics are reported over 20 trials per task.}
\label{tab:more_vlas}
\end{table}

\paragraph{Practicality of ConRFT across Various VLA Models}

ConRFT is highly versatile and can be applied to any VLM-based architecture with an action head. This flexibility stems from its ability to optimize the action generation process independently of the underlying visual encoder, making it adaptable to various VLA frameworks. To further validate its applicability generalization, we test out approach on fine-tuning RoboVLM \cite{li2024towards} with two distinct VLM backbones. As shown in Table \ref{tab:more_vlas}, the results indicate that ConRFT can effectively enhance the performance of various VLAs, improving the success rates across multiple robotic tasks. This ability to fine-tune the action generation while leveraging the pretrained visual components underscores the broad applicability of ConRFT.

\section{Limitations}

Although our approach demonstrates strong performance and sample efficiency for fine-tuning VLA models in real-world manipulation tasks, several limitations remain.

\subsection{Sensitivity to Reward Engineering}

In this work, we implement a task-specific binary classifier to calculate the reward for RL. However, the inherent distributional shift between the classifier's training data and the state-action distributions generated during RL exploration creates a critical vulnerability, as it can lead the learned policy to engage in reward hacking, exploiting unintended behaviors where the classifier provides inaccurate rewards. For instance, the robot might position its end-effector at a specific location that triggers a false positive, causing the policy to converge to an incorrect behavior. Since these reward classifiers typically provide only sparse feedback, the policy may learn slowly, even with the help of human interventions. On the other hand, this reward-driven approach leads to highly specialized policies that are closely tied to the specific conditions of the task, limiting their ability to generalize to new environments. While introducing multi-task dense reward signals could improve sample efficiency and accelerate policy convergence, it would also demand more sophisticated reward engineering for real-world applications.

\subsection{Frozen Encoders and Transformer Backbone}

Our current implementation runs the interaction and policy learning processes in separate threads, fine-tuning only the action head network with consistent policy while keeping the visual encoders and transformer backbone frozen. While this design choice boosts real-time performance, it constrains the policy’s ability to refine its perception and representation modules during online training, especially for unseen scenarios. Allowing partial or complete updates of these frozen components, potentially with efficient techniques such as parameter-efficient fine-tuning (e.g., LoRA \cite{hu2021lora}), could enhance final task performance and adaptability without sacrificing safety or speed.

\section{Conclusion} 

We presented a two-stage approach, ConRFT, for reinforced fine-tuning VLA models in real-world robotic applications. By first performing offline fine-tuning (Cal-ConRFT) with a small set of demonstrations, we initialize a reliable policy and value function via a unified training objective that integrates Q loss and BC loss in a consistency-based framework. We then leveraged task-specific rewards and human interventions in the online fine-tuning stage (HIL-ConRFT) to fine-tune the VLA model via consistency policy. Experiments on eight diverse real-world tasks demonstrated that our approach outperforms SOTA methods in terms of success rate, sample efficiency, and episode length. Overall, this work showcases a practical way to use reinforcement learning for safe and efficient VLA model fine-tuning.

\section*{Acknowledgments}
This work is supported by the National Natural Science Foundation of China (NSFC) under Grant No. 62136008 and in part by the International Partnership Program of the Chinese Academy of Sciences under Grant 104GJHZ2022013GC.

%% Use plainnat to work nicely with natbib. 
% \bibliographystyle{plainnat}
\bibliographystyle{unsrtnat}
% \bibliography{references}
% This must be in the first 5 lines to tell arXiv to use pdfLaTeX, which is strongly recommended.
\pdfoutput=1
% In particular, the hyperref package requires pdfLaTeX in order to break URLs across lines.

\documentclass[11pt]{article}

% Change "review" to "final" to generate the final (sometimes called camera-ready) version.
% Change to "preprint" to generate a non-anonymous version with page numbers.
\usepackage{acl}

% Standard package includes
\usepackage{times}
\usepackage{latexsym}

% Draw tables
\usepackage{booktabs}
\usepackage{multirow}
\usepackage{xcolor}
\usepackage{colortbl}
\usepackage{array} 
\usepackage{amsmath}

\newcolumntype{C}{>{\centering\arraybackslash}p{0.07\textwidth}}
% For proper rendering and hyphenation of words containing Latin characters (including in bib files)
\usepackage[T1]{fontenc}
% For Vietnamese characters
% \usepackage[T5]{fontenc}
% See https://www.latex-project.org/help/documentation/encguide.pdf for other character sets
% This assumes your files are encoded as UTF8
\usepackage[utf8]{inputenc}

% This is not strictly necessary, and may be commented out,
% but it will improve the layout of the manuscript,
% and will typically save some space.
\usepackage{microtype}
\DeclareMathOperator*{\argmax}{arg\,max}
% This is also not strictly necessary, and may be commented out.
% However, it will improve the aesthetics of text in
% the typewriter font.
\usepackage{inconsolata}

%Including images in your LaTeX document requires adding
%additional package(s)
\usepackage{graphicx}
% If the title and author information does not fit in the area allocated, uncomment the following
%
%\setlength\titlebox{<dim>}
%
% and set <dim> to something 5cm or larger.

\title{Wi-Chat: Large Language Model Powered Wi-Fi Sensing}

% Author information can be set in various styles:
% For several authors from the same institution:
% \author{Author 1 \and ... \and Author n \\
%         Address line \\ ... \\ Address line}
% if the names do not fit well on one line use
%         Author 1 \\ {\bf Author 2} \\ ... \\ {\bf Author n} \\
% For authors from different institutions:
% \author{Author 1 \\ Address line \\  ... \\ Address line
%         \And  ... \And
%         Author n \\ Address line \\ ... \\ Address line}
% To start a separate ``row'' of authors use \AND, as in
% \author{Author 1 \\ Address line \\  ... \\ Address line
%         \AND
%         Author 2 \\ Address line \\ ... \\ Address line \And
%         Author 3 \\ Address line \\ ... \\ Address line}

% \author{First Author \\
%   Affiliation / Address line 1 \\
%   Affiliation / Address line 2 \\
%   Affiliation / Address line 3 \\
%   \texttt{email@domain} \\\And
%   Second Author \\
%   Affiliation / Address line 1 \\
%   Affiliation / Address line 2 \\
%   Affiliation / Address line 3 \\
%   \texttt{email@domain} \\}
% \author{Haohan Yuan \qquad Haopeng Zhang\thanks{corresponding author} \\ 
%   ALOHA Lab, University of Hawaii at Manoa \\
%   % Affiliation / Address line 2 \\
%   % Affiliation / Address line 3 \\
%   \texttt{\{haohany,haopengz\}@hawaii.edu}}
  
\author{
{Haopeng Zhang$\dag$\thanks{These authors contributed equally to this work.}, Yili Ren$\ddagger$\footnotemark[1], Haohan Yuan$\dag$, Jingzhe Zhang$\ddagger$, Yitong Shen$\ddagger$} \\
ALOHA Lab, University of Hawaii at Manoa$\dag$, University of South Florida$\ddagger$ \\
\{haopengz, haohany\}@hawaii.edu\\
\{yiliren, jingzhe, shen202\}@usf.edu\\}



  
%\author{
%  \textbf{First Author\textsuperscript{1}},
%  \textbf{Second Author\textsuperscript{1,2}},
%  \textbf{Third T. Author\textsuperscript{1}},
%  \textbf{Fourth Author\textsuperscript{1}},
%\\
%  \textbf{Fifth Author\textsuperscript{1,2}},
%  \textbf{Sixth Author\textsuperscript{1}},
%  \textbf{Seventh Author\textsuperscript{1}},
%  \textbf{Eighth Author \textsuperscript{1,2,3,4}},
%\\
%  \textbf{Ninth Author\textsuperscript{1}},
%  \textbf{Tenth Author\textsuperscript{1}},
%  \textbf{Eleventh E. Author\textsuperscript{1,2,3,4,5}},
%  \textbf{Twelfth Author\textsuperscript{1}},
%\\
%  \textbf{Thirteenth Author\textsuperscript{3}},
%  \textbf{Fourteenth F. Author\textsuperscript{2,4}},
%  \textbf{Fifteenth Author\textsuperscript{1}},
%  \textbf{Sixteenth Author\textsuperscript{1}},
%\\
%  \textbf{Seventeenth S. Author\textsuperscript{4,5}},
%  \textbf{Eighteenth Author\textsuperscript{3,4}},
%  \textbf{Nineteenth N. Author\textsuperscript{2,5}},
%  \textbf{Twentieth Author\textsuperscript{1}}
%\\
%\\
%  \textsuperscript{1}Affiliation 1,
%  \textsuperscript{2}Affiliation 2,
%  \textsuperscript{3}Affiliation 3,
%  \textsuperscript{4}Affiliation 4,
%  \textsuperscript{5}Affiliation 5
%\\
%  \small{
%    \textbf{Correspondence:} \href{mailto:email@domain}{email@domain}
%  }
%}

\begin{document}
\maketitle
\begin{abstract}
Recent advancements in Large Language Models (LLMs) have demonstrated remarkable capabilities across diverse tasks. However, their potential to integrate physical model knowledge for real-world signal interpretation remains largely unexplored. In this work, we introduce Wi-Chat, the first LLM-powered Wi-Fi-based human activity recognition system. We demonstrate that LLMs can process raw Wi-Fi signals and infer human activities by incorporating Wi-Fi sensing principles into prompts. Our approach leverages physical model insights to guide LLMs in interpreting Channel State Information (CSI) data without traditional signal processing techniques. Through experiments on real-world Wi-Fi datasets, we show that LLMs exhibit strong reasoning capabilities, achieving zero-shot activity recognition. These findings highlight a new paradigm for Wi-Fi sensing, expanding LLM applications beyond conventional language tasks and enhancing the accessibility of wireless sensing for real-world deployments.
\end{abstract}

\section{Introduction}

In today’s rapidly evolving digital landscape, the transformative power of web technologies has redefined not only how services are delivered but also how complex tasks are approached. Web-based systems have become increasingly prevalent in risk control across various domains. This widespread adoption is due their accessibility, scalability, and ability to remotely connect various types of users. For example, these systems are used for process safety management in industry~\cite{kannan2016web}, safety risk early warning in urban construction~\cite{ding2013development}, and safe monitoring of infrastructural systems~\cite{repetto2018web}. Within these web-based risk management systems, the source search problem presents a huge challenge. Source search refers to the task of identifying the origin of a risky event, such as a gas leak and the emission point of toxic substances. This source search capability is crucial for effective risk management and decision-making.

Traditional approaches to implementing source search capabilities into the web systems often rely on solely algorithmic solutions~\cite{ristic2016study}. These methods, while relatively straightforward to implement, often struggle to achieve acceptable performances due to algorithmic local optima and complex unknown environments~\cite{zhao2020searching}. More recently, web crowdsourcing has emerged as a promising alternative for tackling the source search problem by incorporating human efforts in these web systems on-the-fly~\cite{zhao2024user}. This approach outsources the task of addressing issues encountered during the source search process to human workers, leveraging their capabilities to enhance system performance.

These solutions often employ a human-AI collaborative way~\cite{zhao2023leveraging} where algorithms handle exploration-exploitation and report the encountered problems while human workers resolve complex decision-making bottlenecks to help the algorithms getting rid of local deadlocks~\cite{zhao2022crowd}. Although effective, this paradigm suffers from two inherent limitations: increased operational costs from continuous human intervention, and slow response times of human workers due to sequential decision-making. These challenges motivate our investigation into developing autonomous systems that preserve human-like reasoning capabilities while reducing dependency on massive crowdsourced labor.

Furthermore, recent advancements in large language models (LLMs)~\cite{chang2024survey} and multi-modal LLMs (MLLMs)~\cite{huang2023chatgpt} have unveiled promising avenues for addressing these challenges. One clear opportunity involves the seamless integration of visual understanding and linguistic reasoning for robust decision-making in search tasks. However, whether large models-assisted source search is really effective and efficient for improving the current source search algorithms~\cite{ji2022source} remains unknown. \textit{To address the research gap, we are particularly interested in answering the following two research questions in this work:}

\textbf{\textit{RQ1: }}How can source search capabilities be integrated into web-based systems to support decision-making in time-sensitive risk management scenarios? 
% \sq{I mention ``time-sensitive'' here because I feel like we shall say something about the response time -- LLM has to be faster than humans}

\textbf{\textit{RQ2: }}How can MLLMs and LLMs enhance the effectiveness and efficiency of existing source search algorithms? 

% \textit{\textbf{RQ2:}} To what extent does the performance of large models-assisted search align with or approach the effectiveness of human-AI collaborative search? 

To answer the research questions, we propose a novel framework called Auto-\
S$^2$earch (\textbf{Auto}nomous \textbf{S}ource \textbf{Search}) and implement a prototype system that leverages advanced web technologies to simulate real-world conditions for zero-shot source search. Unlike traditional methods that rely on pre-defined heuristics or extensive human intervention, AutoS$^2$earch employs a carefully designed prompt that encapsulates human rationales, thereby guiding the MLLM to generate coherent and accurate scene descriptions from visual inputs about four directional choices. Based on these language-based descriptions, the LLM is enabled to determine the optimal directional choice through chain-of-thought (CoT) reasoning. Comprehensive empirical validation demonstrates that AutoS$^2$-\ 
earch achieves a success rate of 95–98\%, closely approaching the performance of human-AI collaborative search across 20 benchmark scenarios~\cite{zhao2023leveraging}. 

Our work indicates that the role of humans in future web crowdsourcing tasks may evolve from executors to validators or supervisors. Furthermore, incorporating explanations of LLM decisions into web-based system interfaces has the potential to help humans enhance task performance in risk control.






\section{Related Work}
\label{sec:relatedworks}

% \begin{table*}[t]
% \centering 
% \renewcommand\arraystretch{0.98}
% \fontsize{8}{10}\selectfont \setlength{\tabcolsep}{0.4em}
% \begin{tabular}{@{}lc|cc|cc|cc@{}}
% \toprule
% \textbf{Methods}           & \begin{tabular}[c]{@{}c@{}}\textbf{Training}\\ \textbf{Paradigm}\end{tabular} & \begin{tabular}[c]{@{}c@{}}\textbf{$\#$ PT Data}\\ \textbf{(Tokens)}\end{tabular} & \begin{tabular}[c]{@{}c@{}}\textbf{$\#$ IFT Data}\\ \textbf{(Samples)}\end{tabular} & \textbf{Code}  & \begin{tabular}[c]{@{}c@{}}\textbf{Natural}\\ \textbf{Language}\end{tabular} & \begin{tabular}[c]{@{}c@{}}\textbf{Action}\\ \textbf{Trajectories}\end{tabular} & \begin{tabular}[c]{@{}c@{}}\textbf{API}\\ \textbf{Documentation}\end{tabular}\\ \midrule 
% NexusRaven~\citep{srinivasan2023nexusraven} & IFT & - & - & \textcolor{green}{\CheckmarkBold} & \textcolor{green}{\CheckmarkBold} &\textcolor{red}{\XSolidBrush}&\textcolor{red}{\XSolidBrush}\\
% AgentInstruct~\citep{zeng2023agenttuning} & IFT & - & 2k & \textcolor{green}{\CheckmarkBold} & \textcolor{green}{\CheckmarkBold} &\textcolor{red}{\XSolidBrush}&\textcolor{red}{\XSolidBrush} \\
% AgentEvol~\citep{xi2024agentgym} & IFT & - & 14.5k & \textcolor{green}{\CheckmarkBold} & \textcolor{green}{\CheckmarkBold} &\textcolor{green}{\CheckmarkBold}&\textcolor{red}{\XSolidBrush} \\
% Gorilla~\citep{patil2023gorilla}& IFT & - & 16k & \textcolor{green}{\CheckmarkBold} & \textcolor{green}{\CheckmarkBold} &\textcolor{red}{\XSolidBrush}&\textcolor{green}{\CheckmarkBold}\\
% OpenFunctions-v2~\citep{patil2023gorilla} & IFT & - & 65k & \textcolor{green}{\CheckmarkBold} & \textcolor{green}{\CheckmarkBold} &\textcolor{red}{\XSolidBrush}&\textcolor{green}{\CheckmarkBold}\\
% LAM~\citep{zhang2024agentohana} & IFT & - & 42.6k & \textcolor{green}{\CheckmarkBold} & \textcolor{green}{\CheckmarkBold} &\textcolor{green}{\CheckmarkBold}&\textcolor{red}{\XSolidBrush} \\
% xLAM~\citep{liu2024apigen} & IFT & - & 60k & \textcolor{green}{\CheckmarkBold} & \textcolor{green}{\CheckmarkBold} &\textcolor{green}{\CheckmarkBold}&\textcolor{red}{\XSolidBrush} \\\midrule
% LEMUR~\citep{xu2024lemur} & PT & 90B & 300k & \textcolor{green}{\CheckmarkBold} & \textcolor{green}{\CheckmarkBold} &\textcolor{green}{\CheckmarkBold}&\textcolor{red}{\XSolidBrush}\\
% \rowcolor{teal!12} \method & PT & 103B & 95k & \textcolor{green}{\CheckmarkBold} & \textcolor{green}{\CheckmarkBold} & \textcolor{green}{\CheckmarkBold} & \textcolor{green}{\CheckmarkBold} \\
% \bottomrule
% \end{tabular}
% \caption{Summary of existing tuning- and pretraining-based LLM agents with their training sample sizes. "PT" and "IFT" denote "Pre-Training" and "Instruction Fine-Tuning", respectively. }
% \label{tab:related}
% \end{table*}

\begin{table*}[ht]
\begin{threeparttable}
\centering 
\renewcommand\arraystretch{0.98}
\fontsize{7}{9}\selectfont \setlength{\tabcolsep}{0.2em}
\begin{tabular}{@{}l|c|c|ccc|cc|cc|cccc@{}}
\toprule
\textbf{Methods} & \textbf{Datasets}           & \begin{tabular}[c]{@{}c@{}}\textbf{Training}\\ \textbf{Paradigm}\end{tabular} & \begin{tabular}[c]{@{}c@{}}\textbf{\# PT Data}\\ \textbf{(Tokens)}\end{tabular} & \begin{tabular}[c]{@{}c@{}}\textbf{\# IFT Data}\\ \textbf{(Samples)}\end{tabular} & \textbf{\# APIs} & \textbf{Code}  & \begin{tabular}[c]{@{}c@{}}\textbf{Nat.}\\ \textbf{Lang.}\end{tabular} & \begin{tabular}[c]{@{}c@{}}\textbf{Action}\\ \textbf{Traj.}\end{tabular} & \begin{tabular}[c]{@{}c@{}}\textbf{API}\\ \textbf{Doc.}\end{tabular} & \begin{tabular}[c]{@{}c@{}}\textbf{Func.}\\ \textbf{Call}\end{tabular} & \begin{tabular}[c]{@{}c@{}}\textbf{Multi.}\\ \textbf{Step}\end{tabular}  & \begin{tabular}[c]{@{}c@{}}\textbf{Plan}\\ \textbf{Refine}\end{tabular}  & \begin{tabular}[c]{@{}c@{}}\textbf{Multi.}\\ \textbf{Turn}\end{tabular}\\ \midrule 
\multicolumn{13}{l}{\emph{Instruction Finetuning-based LLM Agents for Intrinsic Reasoning}}  \\ \midrule
FireAct~\cite{chen2023fireact} & FireAct & IFT & - & 2.1K & 10 & \textcolor{red}{\XSolidBrush} &\textcolor{green}{\CheckmarkBold} &\textcolor{green}{\CheckmarkBold}  & \textcolor{red}{\XSolidBrush} &\textcolor{green}{\CheckmarkBold} & \textcolor{red}{\XSolidBrush} &\textcolor{green}{\CheckmarkBold} & \textcolor{red}{\XSolidBrush} \\
ToolAlpaca~\cite{tang2023toolalpaca} & ToolAlpaca & IFT & - & 4.0K & 400 & \textcolor{red}{\XSolidBrush} &\textcolor{green}{\CheckmarkBold} &\textcolor{green}{\CheckmarkBold} & \textcolor{red}{\XSolidBrush} &\textcolor{green}{\CheckmarkBold} & \textcolor{red}{\XSolidBrush}  &\textcolor{green}{\CheckmarkBold} & \textcolor{red}{\XSolidBrush}  \\
ToolLLaMA~\cite{qin2023toolllm} & ToolBench & IFT & - & 12.7K & 16,464 & \textcolor{red}{\XSolidBrush} &\textcolor{green}{\CheckmarkBold} &\textcolor{green}{\CheckmarkBold} &\textcolor{red}{\XSolidBrush} &\textcolor{green}{\CheckmarkBold}&\textcolor{green}{\CheckmarkBold}&\textcolor{green}{\CheckmarkBold} &\textcolor{green}{\CheckmarkBold}\\
AgentEvol~\citep{xi2024agentgym} & AgentTraj-L & IFT & - & 14.5K & 24 &\textcolor{red}{\XSolidBrush} & \textcolor{green}{\CheckmarkBold} &\textcolor{green}{\CheckmarkBold}&\textcolor{red}{\XSolidBrush} &\textcolor{green}{\CheckmarkBold}&\textcolor{red}{\XSolidBrush} &\textcolor{red}{\XSolidBrush} &\textcolor{green}{\CheckmarkBold}\\
Lumos~\cite{yin2024agent} & Lumos & IFT  & - & 20.0K & 16 &\textcolor{red}{\XSolidBrush} & \textcolor{green}{\CheckmarkBold} & \textcolor{green}{\CheckmarkBold} &\textcolor{red}{\XSolidBrush} & \textcolor{green}{\CheckmarkBold} & \textcolor{green}{\CheckmarkBold} &\textcolor{red}{\XSolidBrush} & \textcolor{green}{\CheckmarkBold}\\
Agent-FLAN~\cite{chen2024agent} & Agent-FLAN & IFT & - & 24.7K & 20 &\textcolor{red}{\XSolidBrush} & \textcolor{green}{\CheckmarkBold} & \textcolor{green}{\CheckmarkBold} &\textcolor{red}{\XSolidBrush} & \textcolor{green}{\CheckmarkBold}& \textcolor{green}{\CheckmarkBold}&\textcolor{red}{\XSolidBrush} & \textcolor{green}{\CheckmarkBold}\\
AgentTuning~\citep{zeng2023agenttuning} & AgentInstruct & IFT & - & 35.0K & - &\textcolor{red}{\XSolidBrush} & \textcolor{green}{\CheckmarkBold} & \textcolor{green}{\CheckmarkBold} &\textcolor{red}{\XSolidBrush} & \textcolor{green}{\CheckmarkBold} &\textcolor{red}{\XSolidBrush} &\textcolor{red}{\XSolidBrush} & \textcolor{green}{\CheckmarkBold}\\\midrule
\multicolumn{13}{l}{\emph{Instruction Finetuning-based LLM Agents for Function Calling}} \\\midrule
NexusRaven~\citep{srinivasan2023nexusraven} & NexusRaven & IFT & - & - & 116 & \textcolor{green}{\CheckmarkBold} & \textcolor{green}{\CheckmarkBold}  & \textcolor{green}{\CheckmarkBold} &\textcolor{red}{\XSolidBrush} & \textcolor{green}{\CheckmarkBold} &\textcolor{red}{\XSolidBrush} &\textcolor{red}{\XSolidBrush}&\textcolor{red}{\XSolidBrush}\\
Gorilla~\citep{patil2023gorilla} & Gorilla & IFT & - & 16.0K & 1,645 & \textcolor{green}{\CheckmarkBold} &\textcolor{red}{\XSolidBrush} &\textcolor{red}{\XSolidBrush}&\textcolor{green}{\CheckmarkBold} &\textcolor{green}{\CheckmarkBold} &\textcolor{red}{\XSolidBrush} &\textcolor{red}{\XSolidBrush} &\textcolor{red}{\XSolidBrush}\\
OpenFunctions-v2~\citep{patil2023gorilla} & OpenFunctions-v2 & IFT & - & 65.0K & - & \textcolor{green}{\CheckmarkBold} & \textcolor{green}{\CheckmarkBold} &\textcolor{red}{\XSolidBrush} &\textcolor{green}{\CheckmarkBold} &\textcolor{green}{\CheckmarkBold} &\textcolor{red}{\XSolidBrush} &\textcolor{red}{\XSolidBrush} &\textcolor{red}{\XSolidBrush}\\
API Pack~\cite{guo2024api} & API Pack & IFT & - & 1.1M & 11,213 &\textcolor{green}{\CheckmarkBold} &\textcolor{red}{\XSolidBrush} &\textcolor{green}{\CheckmarkBold} &\textcolor{red}{\XSolidBrush} &\textcolor{green}{\CheckmarkBold} &\textcolor{red}{\XSolidBrush}&\textcolor{red}{\XSolidBrush}&\textcolor{red}{\XSolidBrush}\\ 
LAM~\citep{zhang2024agentohana} & AgentOhana & IFT & - & 42.6K & - & \textcolor{green}{\CheckmarkBold} & \textcolor{green}{\CheckmarkBold} &\textcolor{green}{\CheckmarkBold}&\textcolor{red}{\XSolidBrush} &\textcolor{green}{\CheckmarkBold}&\textcolor{red}{\XSolidBrush}&\textcolor{green}{\CheckmarkBold}&\textcolor{green}{\CheckmarkBold}\\
xLAM~\citep{liu2024apigen} & APIGen & IFT & - & 60.0K & 3,673 & \textcolor{green}{\CheckmarkBold} & \textcolor{green}{\CheckmarkBold} &\textcolor{green}{\CheckmarkBold}&\textcolor{red}{\XSolidBrush} &\textcolor{green}{\CheckmarkBold}&\textcolor{red}{\XSolidBrush}&\textcolor{green}{\CheckmarkBold}&\textcolor{green}{\CheckmarkBold}\\\midrule
\multicolumn{13}{l}{\emph{Pretraining-based LLM Agents}}  \\\midrule
% LEMUR~\citep{xu2024lemur} & PT & 90B & 300.0K & - & \textcolor{green}{\CheckmarkBold} & \textcolor{green}{\CheckmarkBold} &\textcolor{green}{\CheckmarkBold}&\textcolor{red}{\XSolidBrush} & \textcolor{red}{\XSolidBrush} &\textcolor{green}{\CheckmarkBold} &\textcolor{red}{\XSolidBrush}&\textcolor{red}{\XSolidBrush}\\
\rowcolor{teal!12} \method & \dataset & PT & 103B & 95.0K  & 76,537  & \textcolor{green}{\CheckmarkBold} & \textcolor{green}{\CheckmarkBold} & \textcolor{green}{\CheckmarkBold} & \textcolor{green}{\CheckmarkBold} & \textcolor{green}{\CheckmarkBold} & \textcolor{green}{\CheckmarkBold} & \textcolor{green}{\CheckmarkBold} & \textcolor{green}{\CheckmarkBold}\\
\bottomrule
\end{tabular}
% \begin{tablenotes}
%     \item $^*$ In addition, the StarCoder-API can offer 4.77M more APIs.
% \end{tablenotes}
\caption{Summary of existing instruction finetuning-based LLM agents for intrinsic reasoning and function calling, along with their training resources and sample sizes. "PT" and "IFT" denote "Pre-Training" and "Instruction Fine-Tuning", respectively.}
\vspace{-2ex}
\label{tab:related}
\end{threeparttable}
\end{table*}

\noindent \textbf{Prompting-based LLM Agents.} Due to the lack of agent-specific pre-training corpus, existing LLM agents rely on either prompt engineering~\cite{hsieh2023tool,lu2024chameleon,yao2022react,wang2023voyager} or instruction fine-tuning~\cite{chen2023fireact,zeng2023agenttuning} to understand human instructions, decompose high-level tasks, generate grounded plans, and execute multi-step actions. 
However, prompting-based methods mainly depend on the capabilities of backbone LLMs (usually commercial LLMs), failing to introduce new knowledge and struggling to generalize to unseen tasks~\cite{sun2024adaplanner,zhuang2023toolchain}. 

\noindent \textbf{Instruction Finetuning-based LLM Agents.} Considering the extensive diversity of APIs and the complexity of multi-tool instructions, tool learning inherently presents greater challenges than natural language tasks, such as text generation~\cite{qin2023toolllm}.
Post-training techniques focus more on instruction following and aligning output with specific formats~\cite{patil2023gorilla,hao2024toolkengpt,qin2023toolllm,schick2024toolformer}, rather than fundamentally improving model knowledge or capabilities. 
Moreover, heavy fine-tuning can hinder generalization or even degrade performance in non-agent use cases, potentially suppressing the original base model capabilities~\cite{ghosh2024a}.

\noindent \textbf{Pretraining-based LLM Agents.} While pre-training serves as an essential alternative, prior works~\cite{nijkamp2023codegen,roziere2023code,xu2024lemur,patil2023gorilla} have primarily focused on improving task-specific capabilities (\eg, code generation) instead of general-domain LLM agents, due to single-source, uni-type, small-scale, and poor-quality pre-training data. 
Existing tool documentation data for agent training either lacks diverse real-world APIs~\cite{patil2023gorilla, tang2023toolalpaca} or is constrained to single-tool or single-round tool execution. 
Furthermore, trajectory data mostly imitate expert behavior or follow function-calling rules with inferior planning and reasoning, failing to fully elicit LLMs' capabilities and handle complex instructions~\cite{qin2023toolllm}. 
Given a wide range of candidate API functions, each comprising various function names and parameters available at every planning step, identifying globally optimal solutions and generalizing across tasks remains highly challenging.



\section{Preliminaries}
\label{Preliminaries}
\begin{figure*}[t]
    \centering
    \includegraphics[width=0.95\linewidth]{fig/HealthGPT_Framework.png}
    \caption{The \ourmethod{} architecture integrates hierarchical visual perception and H-LoRA, employing a task-specific hard router to select visual features and H-LoRA plugins, ultimately generating outputs with an autoregressive manner.}
    \label{fig:architecture}
\end{figure*}
\noindent\textbf{Large Vision-Language Models.} 
The input to a LVLM typically consists of an image $x^{\text{img}}$ and a discrete text sequence $x^{\text{txt}}$. The visual encoder $\mathcal{E}^{\text{img}}$ converts the input image $x^{\text{img}}$ into a sequence of visual tokens $\mathcal{V} = [v_i]_{i=1}^{N_v}$, while the text sequence $x^{\text{txt}}$ is mapped into a sequence of text tokens $\mathcal{T} = [t_i]_{i=1}^{N_t}$ using an embedding function $\mathcal{E}^{\text{txt}}$. The LLM $\mathcal{M_\text{LLM}}(\cdot|\theta)$ models the joint probability of the token sequence $\mathcal{U} = \{\mathcal{V},\mathcal{T}\}$, which is expressed as:
\begin{equation}
    P_\theta(R | \mathcal{U}) = \prod_{i=1}^{N_r} P_\theta(r_i | \{\mathcal{U}, r_{<i}\}),
\end{equation}
where $R = [r_i]_{i=1}^{N_r}$ is the text response sequence. The LVLM iteratively generates the next token $r_i$ based on $r_{<i}$. The optimization objective is to minimize the cross-entropy loss of the response $\mathcal{R}$.
% \begin{equation}
%     \mathcal{L}_{\text{VLM}} = \mathbb{E}_{R|\mathcal{U}}\left[-\log P_\theta(R | \mathcal{U})\right]
% \end{equation}
It is worth noting that most LVLMs adopt a design paradigm based on ViT, alignment adapters, and pre-trained LLMs\cite{liu2023llava,liu2024improved}, enabling quick adaptation to downstream tasks.


\noindent\textbf{VQGAN.}
VQGAN~\cite{esser2021taming} employs latent space compression and indexing mechanisms to effectively learn a complete discrete representation of images. VQGAN first maps the input image $x^{\text{img}}$ to a latent representation $z = \mathcal{E}(x)$ through a encoder $\mathcal{E}$. Then, the latent representation is quantized using a codebook $\mathcal{Z} = \{z_k\}_{k=1}^K$, generating a discrete index sequence $\mathcal{I} = [i_m]_{m=1}^N$, where $i_m \in \mathcal{Z}$ represents the quantized code index:
\begin{equation}
    \mathcal{I} = \text{Quantize}(z|\mathcal{Z}) = \arg\min_{z_k \in \mathcal{Z}} \| z - z_k \|_2.
\end{equation}
In our approach, the discrete index sequence $\mathcal{I}$ serves as a supervisory signal for the generation task, enabling the model to predict the index sequence $\hat{\mathcal{I}}$ from input conditions such as text or other modality signals.  
Finally, the predicted index sequence $\hat{\mathcal{I}}$ is upsampled by the VQGAN decoder $G$, generating the high-quality image $\hat{x}^\text{img} = G(\hat{\mathcal{I}})$.



\noindent\textbf{Low Rank Adaptation.} 
LoRA\cite{hu2021lora} effectively captures the characteristics of downstream tasks by introducing low-rank adapters. The core idea is to decompose the bypass weight matrix $\Delta W\in\mathbb{R}^{d^{\text{in}} \times d^{\text{out}}}$ into two low-rank matrices $ \{A \in \mathbb{R}^{d^{\text{in}} \times r}, B \in \mathbb{R}^{r \times d^{\text{out}}} \}$, where $ r \ll \min\{d^{\text{in}}, d^{\text{out}}\} $, significantly reducing learnable parameters. The output with the LoRA adapter for the input $x$ is then given by:
\begin{equation}
    h = x W_0 + \alpha x \Delta W/r = x W_0 + \alpha xAB/r,
\end{equation}
where matrix $ A $ is initialized with a Gaussian distribution, while the matrix $ B $ is initialized as a zero matrix. The scaling factor $ \alpha/r $ controls the impact of $ \Delta W $ on the model.

\section{HealthGPT}
\label{Method}


\subsection{Unified Autoregressive Generation.}  
% As shown in Figure~\ref{fig:architecture}, 
\ourmethod{} (Figure~\ref{fig:architecture}) utilizes a discrete token representation that covers both text and visual outputs, unifying visual comprehension and generation as an autoregressive task. 
For comprehension, $\mathcal{M}_\text{llm}$ receives the input joint sequence $\mathcal{U}$ and outputs a series of text token $\mathcal{R} = [r_1, r_2, \dots, r_{N_r}]$, where $r_i \in \mathcal{V}_{\text{txt}}$, and $\mathcal{V}_{\text{txt}}$ represents the LLM's vocabulary:
\begin{equation}
    P_\theta(\mathcal{R} \mid \mathcal{U}) = \prod_{i=1}^{N_r} P_\theta(r_i \mid \mathcal{U}, r_{<i}).
\end{equation}
For generation, $\mathcal{M}_\text{llm}$ first receives a special start token $\langle \text{START\_IMG} \rangle$, then generates a series of tokens corresponding to the VQGAN indices $\mathcal{I} = [i_1, i_2, \dots, i_{N_i}]$, where $i_j \in \mathcal{V}_{\text{vq}}$, and $\mathcal{V}_{\text{vq}}$ represents the index range of VQGAN. Upon completion of generation, the LLM outputs an end token $\langle \text{END\_IMG} \rangle$:
\begin{equation}
    P_\theta(\mathcal{I} \mid \mathcal{U}) = \prod_{j=1}^{N_i} P_\theta(i_j \mid \mathcal{U}, i_{<j}).
\end{equation}
Finally, the generated index sequence $\mathcal{I}$ is fed into the decoder $G$, which reconstructs the target image $\hat{x}^{\text{img}} = G(\mathcal{I})$.

\subsection{Hierarchical Visual Perception}  
Given the differences in visual perception between comprehension and generation tasks—where the former focuses on abstract semantics and the latter emphasizes complete semantics—we employ ViT to compress the image into discrete visual tokens at multiple hierarchical levels.
Specifically, the image is converted into a series of features $\{f_1, f_2, \dots, f_L\}$ as it passes through $L$ ViT blocks.

To address the needs of various tasks, the hidden states are divided into two types: (i) \textit{Concrete-grained features} $\mathcal{F}^{\text{Con}} = \{f_1, f_2, \dots, f_k\}, k < L$, derived from the shallower layers of ViT, containing sufficient global features, suitable for generation tasks; 
(ii) \textit{Abstract-grained features} $\mathcal{F}^{\text{Abs}} = \{f_{k+1}, f_{k+2}, \dots, f_L\}$, derived from the deeper layers of ViT, which contain abstract semantic information closer to the text space, suitable for comprehension tasks.

The task type $T$ (comprehension or generation) determines which set of features is selected as the input for the downstream large language model:
\begin{equation}
    \mathcal{F}^{\text{img}}_T =
    \begin{cases}
        \mathcal{F}^{\text{Con}}, & \text{if } T = \text{generation task} \\
        \mathcal{F}^{\text{Abs}}, & \text{if } T = \text{comprehension task}
    \end{cases}
\end{equation}
We integrate the image features $\mathcal{F}^{\text{img}}_T$ and text features $\mathcal{T}$ into a joint sequence through simple concatenation, which is then fed into the LLM $\mathcal{M}_{\text{llm}}$ for autoregressive generation.
% :
% \begin{equation}
%     \mathcal{R} = \mathcal{M}_{\text{llm}}(\mathcal{U}|\theta), \quad \mathcal{U} = [\mathcal{F}^{\text{img}}_T; \mathcal{T}]
% \end{equation}
\subsection{Heterogeneous Knowledge Adaptation}
We devise H-LoRA, which stores heterogeneous knowledge from comprehension and generation tasks in separate modules and dynamically routes to extract task-relevant knowledge from these modules. 
At the task level, for each task type $ T $, we dynamically assign a dedicated H-LoRA submodule $ \theta^T $, which is expressed as:
\begin{equation}
    \mathcal{R} = \mathcal{M}_\text{LLM}(\mathcal{U}|\theta, \theta^T), \quad \theta^T = \{A^T, B^T, \mathcal{R}^T_\text{outer}\}.
\end{equation}
At the feature level for a single task, H-LoRA integrates the idea of Mixture of Experts (MoE)~\cite{masoudnia2014mixture} and designs an efficient matrix merging and routing weight allocation mechanism, thus avoiding the significant computational delay introduced by matrix splitting in existing MoELoRA~\cite{luo2024moelora}. Specifically, we first merge the low-rank matrices (rank = r) of $ k $ LoRA experts into a unified matrix:
\begin{equation}
    \mathbf{A}^{\text{merged}}, \mathbf{B}^{\text{merged}} = \text{Concat}(\{A_i\}_1^k), \text{Concat}(\{B_i\}_1^k),
\end{equation}
where $ \mathbf{A}^{\text{merged}} \in \mathbb{R}^{d^\text{in} \times rk} $ and $ \mathbf{B}^{\text{merged}} \in \mathbb{R}^{rk \times d^\text{out}} $. The $k$-dimension routing layer generates expert weights $ \mathcal{W} \in \mathbb{R}^{\text{token\_num} \times k} $ based on the input hidden state $ x $, and these are expanded to $ \mathbb{R}^{\text{token\_num} \times rk} $ as follows:
\begin{equation}
    \mathcal{W}^\text{expanded} = \alpha k \mathcal{W} / r \otimes \mathbf{1}_r,
\end{equation}
where $ \otimes $ denotes the replication operation.
The overall output of H-LoRA is computed as:
\begin{equation}
    \mathcal{O}^\text{H-LoRA} = (x \mathbf{A}^{\text{merged}} \odot \mathcal{W}^\text{expanded}) \mathbf{B}^{\text{merged}},
\end{equation}
where $ \odot $ represents element-wise multiplication. Finally, the output of H-LoRA is added to the frozen pre-trained weights to produce the final output:
\begin{equation}
    \mathcal{O} = x W_0 + \mathcal{O}^\text{H-LoRA}.
\end{equation}
% In summary, H-LoRA is a task-based dynamic PEFT method that achieves high efficiency in single-task fine-tuning.

\subsection{Training Pipeline}

\begin{figure}[t]
    \centering
    \hspace{-4mm}
    \includegraphics[width=0.94\linewidth]{fig/data.pdf}
    \caption{Data statistics of \texttt{VL-Health}. }
    \label{fig:data}
\end{figure}
\noindent \textbf{1st Stage: Multi-modal Alignment.} 
In the first stage, we design separate visual adapters and H-LoRA submodules for medical unified tasks. For the medical comprehension task, we train abstract-grained visual adapters using high-quality image-text pairs to align visual embeddings with textual embeddings, thereby enabling the model to accurately describe medical visual content. During this process, the pre-trained LLM and its corresponding H-LoRA submodules remain frozen. In contrast, the medical generation task requires training concrete-grained adapters and H-LoRA submodules while keeping the LLM frozen. Meanwhile, we extend the textual vocabulary to include multimodal tokens, enabling the support of additional VQGAN vector quantization indices. The model trains on image-VQ pairs, endowing the pre-trained LLM with the capability for image reconstruction. This design ensures pixel-level consistency of pre- and post-LVLM. The processes establish the initial alignment between the LLM’s outputs and the visual inputs.

\noindent \textbf{2nd Stage: Heterogeneous H-LoRA Plugin Adaptation.}  
The submodules of H-LoRA share the word embedding layer and output head but may encounter issues such as bias and scale inconsistencies during training across different tasks. To ensure that the multiple H-LoRA plugins seamlessly interface with the LLMs and form a unified base, we fine-tune the word embedding layer and output head using a small amount of mixed data to maintain consistency in the model weights. Specifically, during this stage, all H-LoRA submodules for different tasks are kept frozen, with only the word embedding layer and output head being optimized. Through this stage, the model accumulates foundational knowledge for unified tasks by adapting H-LoRA plugins.

\begin{table*}[!t]
\centering
\caption{Comparison of \ourmethod{} with other LVLMs and unified multi-modal models on medical visual comprehension tasks. \textbf{Bold} and \underline{underlined} text indicates the best performance and second-best performance, respectively.}
\resizebox{\textwidth}{!}{
\begin{tabular}{c|lcc|cccccccc|c}
\toprule
\rowcolor[HTML]{E9F3FE} &  &  &  & \multicolumn{2}{c}{\textbf{VQA-RAD \textuparrow}} & \multicolumn{2}{c}{\textbf{SLAKE \textuparrow}} & \multicolumn{2}{c}{\textbf{PathVQA \textuparrow}} &  &  &  \\ 
\cline{5-10}
\rowcolor[HTML]{E9F3FE}\multirow{-2}{*}{\textbf{Type}} & \multirow{-2}{*}{\textbf{Model}} & \multirow{-2}{*}{\textbf{\# Params}} & \multirow{-2}{*}{\makecell{\textbf{Medical} \\ \textbf{LVLM}}} & \textbf{close} & \textbf{all} & \textbf{close} & \textbf{all} & \textbf{close} & \textbf{all} & \multirow{-2}{*}{\makecell{\textbf{MMMU} \\ \textbf{-Med}}\textuparrow} & \multirow{-2}{*}{\textbf{OMVQA}\textuparrow} & \multirow{-2}{*}{\textbf{Avg. \textuparrow}} \\ 
\midrule \midrule
\multirow{9}{*}{\textbf{Comp. Only}} 
& Med-Flamingo & 8.3B & \Large \ding{51} & 58.6 & 43.0 & 47.0 & 25.5 & 61.9 & 31.3 & 28.7 & 34.9 & 41.4 \\
& LLaVA-Med & 7B & \Large \ding{51} & 60.2 & 48.1 & 58.4 & 44.8 & 62.3 & 35.7 & 30.0 & 41.3 & 47.6 \\
& HuatuoGPT-Vision & 7B & \Large \ding{51} & 66.9 & 53.0 & 59.8 & 49.1 & 52.9 & 32.0 & 42.0 & 50.0 & 50.7 \\
& BLIP-2 & 6.7B & \Large \ding{55} & 43.4 & 36.8 & 41.6 & 35.3 & 48.5 & 28.8 & 27.3 & 26.9 & 36.1 \\
& LLaVA-v1.5 & 7B & \Large \ding{55} & 51.8 & 42.8 & 37.1 & 37.7 & 53.5 & 31.4 & 32.7 & 44.7 & 41.5 \\
& InstructBLIP & 7B & \Large \ding{55} & 61.0 & 44.8 & 66.8 & 43.3 & 56.0 & 32.3 & 25.3 & 29.0 & 44.8 \\
& Yi-VL & 6B & \Large \ding{55} & 52.6 & 42.1 & 52.4 & 38.4 & 54.9 & 30.9 & 38.0 & 50.2 & 44.9 \\
& InternVL2 & 8B & \Large \ding{55} & 64.9 & 49.0 & 66.6 & 50.1 & 60.0 & 31.9 & \underline{43.3} & 54.5 & 52.5\\
& Llama-3.2 & 11B & \Large \ding{55} & 68.9 & 45.5 & 72.4 & 52.1 & 62.8 & 33.6 & 39.3 & 63.2 & 54.7 \\
\midrule
\multirow{5}{*}{\textbf{Comp. \& Gen.}} 
& Show-o & 1.3B & \Large \ding{55} & 50.6 & 33.9 & 31.5 & 17.9 & 52.9 & 28.2 & 22.7 & 45.7 & 42.6 \\
& Unified-IO 2 & 7B & \Large \ding{55} & 46.2 & 32.6 & 35.9 & 21.9 & 52.5 & 27.0 & 25.3 & 33.0 & 33.8 \\
& Janus & 1.3B & \Large \ding{55} & 70.9 & 52.8 & 34.7 & 26.9 & 51.9 & 27.9 & 30.0 & 26.8 & 33.5 \\
& \cellcolor[HTML]{DAE0FB}HealthGPT-M3 & \cellcolor[HTML]{DAE0FB}3.8B & \cellcolor[HTML]{DAE0FB}\Large \ding{51} & \cellcolor[HTML]{DAE0FB}\underline{73.7} & \cellcolor[HTML]{DAE0FB}\underline{55.9} & \cellcolor[HTML]{DAE0FB}\underline{74.6} & \cellcolor[HTML]{DAE0FB}\underline{56.4} & \cellcolor[HTML]{DAE0FB}\underline{78.7} & \cellcolor[HTML]{DAE0FB}\underline{39.7} & \cellcolor[HTML]{DAE0FB}\underline{43.3} & \cellcolor[HTML]{DAE0FB}\underline{68.5} & \cellcolor[HTML]{DAE0FB}\underline{61.3} \\
& \cellcolor[HTML]{DAE0FB}HealthGPT-L14 & \cellcolor[HTML]{DAE0FB}14B & \cellcolor[HTML]{DAE0FB}\Large \ding{51} & \cellcolor[HTML]{DAE0FB}\textbf{77.7} & \cellcolor[HTML]{DAE0FB}\textbf{58.3} & \cellcolor[HTML]{DAE0FB}\textbf{76.4} & \cellcolor[HTML]{DAE0FB}\textbf{64.5} & \cellcolor[HTML]{DAE0FB}\textbf{85.9} & \cellcolor[HTML]{DAE0FB}\textbf{44.4} & \cellcolor[HTML]{DAE0FB}\textbf{49.2} & \cellcolor[HTML]{DAE0FB}\textbf{74.4} & \cellcolor[HTML]{DAE0FB}\textbf{66.4} \\
\bottomrule
\end{tabular}
}
\label{tab:results}
\end{table*}
\begin{table*}[ht]
    \centering
    \caption{The experimental results for the four modality conversion tasks.}
    \resizebox{\textwidth}{!}{
    \begin{tabular}{l|ccc|ccc|ccc|ccc}
        \toprule
        \rowcolor[HTML]{E9F3FE} & \multicolumn{3}{c}{\textbf{CT to MRI (Brain)}} & \multicolumn{3}{c}{\textbf{CT to MRI (Pelvis)}} & \multicolumn{3}{c}{\textbf{MRI to CT (Brain)}} & \multicolumn{3}{c}{\textbf{MRI to CT (Pelvis)}} \\
        \cline{2-13}
        \rowcolor[HTML]{E9F3FE}\multirow{-2}{*}{\textbf{Model}}& \textbf{SSIM $\uparrow$} & \textbf{PSNR $\uparrow$} & \textbf{MSE $\downarrow$} & \textbf{SSIM $\uparrow$} & \textbf{PSNR $\uparrow$} & \textbf{MSE $\downarrow$} & \textbf{SSIM $\uparrow$} & \textbf{PSNR $\uparrow$} & \textbf{MSE $\downarrow$} & \textbf{SSIM $\uparrow$} & \textbf{PSNR $\uparrow$} & \textbf{MSE $\downarrow$} \\
        \midrule \midrule
        pix2pix & 71.09 & 32.65 & 36.85 & 59.17 & 31.02 & 51.91 & 78.79 & 33.85 & 28.33 & 72.31 & 32.98 & 36.19 \\
        CycleGAN & 54.76 & 32.23 & 40.56 & 54.54 & 30.77 & 55.00 & 63.75 & 31.02 & 52.78 & 50.54 & 29.89 & 67.78 \\
        BBDM & {71.69} & {32.91} & {34.44} & 57.37 & 31.37 & 48.06 & \textbf{86.40} & 34.12 & 26.61 & {79.26} & 33.15 & 33.60 \\
        Vmanba & 69.54 & 32.67 & 36.42 & {63.01} & {31.47} & {46.99} & 79.63 & 34.12 & 26.49 & 77.45 & 33.53 & 31.85 \\
        DiffMa & 71.47 & 32.74 & 35.77 & 62.56 & 31.43 & 47.38 & 79.00 & {34.13} & {26.45} & 78.53 & {33.68} & {30.51} \\
        \rowcolor[HTML]{DAE0FB}HealthGPT-M3 & \underline{79.38} & \underline{33.03} & \underline{33.48} & \underline{71.81} & \underline{31.83} & \underline{43.45} & {85.06} & \textbf{34.40} & \textbf{25.49} & \underline{84.23} & \textbf{34.29} & \textbf{27.99} \\
        \rowcolor[HTML]{DAE0FB}HealthGPT-L14 & \textbf{79.73} & \textbf{33.10} & \textbf{32.96} & \textbf{71.92} & \textbf{31.87} & \textbf{43.09} & \underline{85.31} & \underline{34.29} & \underline{26.20} & \textbf{84.96} & \underline{34.14} & \underline{28.13} \\
        \bottomrule
    \end{tabular}
    }
    \label{tab:conversion}
\end{table*}

\noindent \textbf{3rd Stage: Visual Instruction Fine-Tuning.}  
In the third stage, we introduce additional task-specific data to further optimize the model and enhance its adaptability to downstream tasks such as medical visual comprehension (e.g., medical QA, medical dialogues, and report generation) or generation tasks (e.g., super-resolution, denoising, and modality conversion). Notably, by this stage, the word embedding layer and output head have been fine-tuned, only the H-LoRA modules and adapter modules need to be trained. This strategy significantly improves the model's adaptability and flexibility across different tasks.


\section{Experiment}
\label{s:experiment}

\subsection{Data Description}
We evaluate our method on FI~\cite{you2016building}, Twitter\_LDL~\cite{yang2017learning} and Artphoto~\cite{machajdik2010affective}.
FI is a public dataset built from Flickr and Instagram, with 23,308 images and eight emotion categories, namely \textit{amusement}, \textit{anger}, \textit{awe},  \textit{contentment}, \textit{disgust}, \textit{excitement},  \textit{fear}, and \textit{sadness}. 
% Since images in FI are all copyrighted by law, some images are corrupted now, so we remove these samples and retain 21,828 images.
% T4SA contains images from Twitter, which are classified into three categories: \textit{positive}, \textit{neutral}, and \textit{negative}. In this paper, we adopt the base version of B-T4SA, which contains 470,586 images and provides text descriptions of the corresponding tweets.
Twitter\_LDL contains 10,045 images from Twitter, with the same eight categories as the FI dataset.
% 。
For these two datasets, they are randomly split into 80\%
training and 20\% testing set.
Artphoto contains 806 artistic photos from the DeviantArt website, which we use to further evaluate the zero-shot capability of our model.
% on the small-scale dataset.
% We construct and publicly release the first image sentiment analysis dataset containing metadata.
% 。

% Based on these datasets, we are the first to construct and publicly release metadata-enhanced image sentiment analysis datasets. These datasets include scenes, tags, descriptions, and corresponding confidence scores, and are available at this link for future research purposes.


% 
\begin{table}[t]
\centering
% \begin{center}
\caption{Overall performance of different models on FI and Twitter\_LDL datasets.}
\label{tab:cap1}
% \resizebox{\linewidth}{!}
{
\begin{tabular}{l|c|c|c|c}
\hline
\multirow{2}{*}{\textbf{Model}} & \multicolumn{2}{c|}{\textbf{FI}}  & \multicolumn{2}{c}{\textbf{Twitter\_LDL}} \\ \cline{2-5} 
  & \textbf{Accuracy} & \textbf{F1} & \textbf{Accuracy} & \textbf{F1}  \\ \hline
% (\rownumber)~AlexNet~\cite{krizhevsky2017imagenet}  & 58.13\% & 56.35\%  & 56.24\%& 55.02\%  \\ 
% (\rownumber)~VGG16~\cite{simonyan2014very}  & 63.75\%& 63.08\%  & 59.34\%& 59.02\%  \\ 
(\rownumber)~ResNet101~\cite{he2016deep} & 66.16\%& 65.56\%  & 62.02\% & 61.34\%  \\ 
(\rownumber)~CDA~\cite{han2023boosting} & 66.71\%& 65.37\%  & 64.14\% & 62.85\%  \\ 
(\rownumber)~CECCN~\cite{ruan2024color} & 67.96\%& 66.74\%  & 64.59\%& 64.72\% \\ 
(\rownumber)~EmoVIT~\cite{xie2024emovit} & 68.09\%& 67.45\%  & 63.12\% & 61.97\%  \\ 
(\rownumber)~ComLDL~\cite{zhang2022compound} & 68.83\%& 67.28\%  & 65.29\% & 63.12\%  \\ 
(\rownumber)~WSDEN~\cite{li2023weakly} & 69.78\%& 69.61\%  & 67.04\% & 65.49\% \\ 
(\rownumber)~ECWA~\cite{deng2021emotion} & 70.87\%& 69.08\%  & 67.81\% & 66.87\%  \\ 
(\rownumber)~EECon~\cite{yang2023exploiting} & 71.13\%& 68.34\%  & 64.27\%& 63.16\%  \\ 
(\rownumber)~MAM~\cite{zhang2024affective} & 71.44\%  & 70.83\% & 67.18\%  & 65.01\%\\ 
(\rownumber)~TGCA-PVT~\cite{chen2024tgca}   & 73.05\%  & 71.46\% & 69.87\%  & 68.32\% \\ 
(\rownumber)~OEAN~\cite{zhang2024object}   & 73.40\%  & 72.63\% & 70.52\%  & 69.47\% \\ \hline
(\rownumber)~\shortname  & \textbf{79.48\%} & \textbf{79.22\%} & \textbf{74.12\%} & \textbf{73.09\%} \\ \hline
\end{tabular}
}
\vspace{-6mm}
% \end{center}
\end{table}
% 

\subsection{Experiment Setting}
% \subsubsection{Model Setting.}
% 
\textbf{Model Setting:}
For feature representation, we set $k=10$ to select object tags, and adopt clip-vit-base-patch32 as the pre-trained model for unified feature representation.
Moreover, we empirically set $(d_e, d_h, d_k, d_s) = (512, 128, 16, 64)$, and set the classification class $L$ to 8.

% 

\textbf{Training Setting:}
To initialize the model, we set all weights such as $\boldsymbol{W}$ following the truncated normal distribution, and use AdamW optimizer with the learning rate of $1 \times 10^{-4}$.
% warmup scheduler of cosine, warmup steps of 2000.
Furthermore, we set the batch size to 32 and the epoch of the training process to 200.
During the implementation, we utilize \textit{PyTorch} to build our entire model.
% , and our project codes are publicly available at https://github.com/zzmyrep/MESN.
% Our project codes as well as data are all publicly available on GitHub\footnote{https://github.com/zzmyrep/KBCEN}.
% Code is available at \href{https://github.com/zzmyrep/KBCEN}{https://github.com/zzmyrep/KBCEN}.

\textbf{Evaluation Metrics:}
Following~\cite{zhang2024affective, chen2024tgca, zhang2024object}, we adopt \textit{accuracy} and \textit{F1} as our evaluation metrics to measure the performance of different methods for image sentiment analysis. 



\subsection{Experiment Result}
% We compare our model against the following baselines: AlexNet~\cite{krizhevsky2017imagenet}, VGG16~\cite{simonyan2014very}, ResNet101~\cite{he2016deep}, CECCN~\cite{ruan2024color}, EmoVIT~\cite{xie2024emovit}, WSCNet~\cite{yang2018weakly}, ECWA~\cite{deng2021emotion}, EECon~\cite{yang2023exploiting}, MAM~\cite{zhang2024affective} and TGCA-PVT~\cite{chen2024tgca}, and the overall results are summarized in Table~\ref{tab:cap1}.
We compare our model against several baselines, and the overall results are summarized in Table~\ref{tab:cap1}.
We observe that our model achieves the best performance in both accuracy and F1 metrics, significantly outperforming the previous models. 
This superior performance is mainly attributed to our effective utilization of metadata to enhance image sentiment analysis, as well as the exceptional capability of the unified sentiment transformer framework we developed. These results strongly demonstrate that our proposed method can bring encouraging performance for image sentiment analysis.

\setcounter{magicrownumbers}{0} 
\begin{table}[t]
\begin{center}
\caption{Ablation study of~\shortname~on FI dataset.} 
% \vspace{1mm}
\label{tab:cap2}
\resizebox{.9\linewidth}{!}
{
\begin{tabular}{lcc}
  \hline
  \textbf{Model} & \textbf{Accuracy} & \textbf{F1} \\
  \hline
  (\rownumber)~Ours (w/o vision) & 65.72\% & 64.54\% \\
  (\rownumber)~Ours (w/o text description) & 74.05\% & 72.58\% \\
  (\rownumber)~Ours (w/o object tag) & 77.45\% & 76.84\% \\
  (\rownumber)~Ours (w/o scene tag) & 78.47\% & 78.21\% \\
  \hline
  (\rownumber)~Ours (w/o unified embedding) & 76.41\% & 76.23\% \\
  (\rownumber)~Ours (w/o adaptive learning) & 76.83\% & 76.56\% \\
  (\rownumber)~Ours (w/o cross-modal fusion) & 76.85\% & 76.49\% \\
  \hline
  (\rownumber)~Ours  & \textbf{79.48\%} & \textbf{79.22\%} \\
  \hline
\end{tabular}
}
\end{center}
\vspace{-5mm}
\end{table}


\begin{figure}[t]
\centering
% \vspace{-2mm}
\includegraphics[width=0.42\textwidth]{fig/2dvisual-linux4-paper2.pdf}
\caption{Visualization of feature distribution on eight categories before (left) and after (right) model processing.}
% 
\label{fig:visualization}
\vspace{-5mm}
\end{figure}

\subsection{Ablation Performance}
In this subsection, we conduct an ablation study to examine which component is really important for performance improvement. The results are reported in Table~\ref{tab:cap2}.

For information utilization, we observe a significant decline in model performance when visual features are removed. Additionally, the performance of \shortname~decreases when different metadata are removed separately, which means that text description, object tag, and scene tag are all critical for image sentiment analysis.
Recalling the model architecture, we separately remove transformer layers of the unified representation module, the adaptive learning module, and the cross-modal fusion module, replacing them with MLPs of the same parameter scale.
In this way, we can observe varying degrees of decline in model performance, indicating that these modules are indispensable for our model to achieve better performance.

\subsection{Visualization}
% 


% % 开始使用minipage进行左右排列
% \begin{minipage}[t]{0.45\textwidth}  % 子图1宽度为45%
%     \centering
%     \includegraphics[width=\textwidth]{2dvisual.pdf}  % 插入图片
%     \captionof{figure}{Visualization of feature distribution.}  % 使用captionof添加图片标题
%     \label{fig:visualization}
% \end{minipage}


% \begin{figure}[t]
% \centering
% \vspace{-2mm}
% \includegraphics[width=0.45\textwidth]{fig/2dvisual.pdf}
% \caption{Visualization of feature distribution.}
% \label{fig:visualization}
% % \vspace{-4mm}
% \end{figure}

% \begin{figure}[t]
% \centering
% \vspace{-2mm}
% \includegraphics[width=0.45\textwidth]{fig/2dvisual-linux3-paper.pdf}
% \caption{Visualization of feature distribution.}
% \label{fig:visualization}
% % \vspace{-4mm}
% \end{figure}



\begin{figure}[tbp]   
\vspace{-4mm}
  \centering            
  \subfloat[Depth of adaptive learning layers]   
  {
    \label{fig:subfig1}\includegraphics[width=0.22\textwidth]{fig/fig_sensitivity-a5}
  }
  \subfloat[Depth of fusion layers]
  {
    % \label{fig:subfig2}\includegraphics[width=0.22\textwidth]{fig/fig_sensitivity-b2}
    \label{fig:subfig2}\includegraphics[width=0.22\textwidth]{fig/fig_sensitivity-b2-num.pdf}
  }
  \caption{Sensitivity study of \shortname~on different depth. }   
  \label{fig:fig_sensitivity}  
\vspace{-2mm}
\end{figure}

% \begin{figure}[htbp]
% \centerline{\includegraphics{2dvisual.pdf}}
% \caption{Visualization of feature distribution.}
% \label{fig:visualization}
% \end{figure}

% In Fig.~\ref{fig:visualization}, we use t-SNE~\cite{van2008visualizing} to reduce the dimension of data features for visualization, Figure in left represents the metadata features before model processing, the features are obtained by embedding through the CLIP model, and figure in right shows the features of the data after model processing, it can be observed that after the model processing, the data with different label categories fall in different regions in the space, therefore, we can conclude that the Therefore, we can conclude that the model can effectively utilize the information contained in the metadata and use it to guide the model for classification.

In Fig.~\ref{fig:visualization}, we use t-SNE~\cite{van2008visualizing} to reduce the dimension of data features for visualization.
The left figure shows metadata features before being processed by our model (\textit{i.e.}, embedded by CLIP), while the right shows the distribution of features after being processed by our model.
We can observe that after the model processing, data with the same label are closer to each other, while others are farther away.
Therefore, it shows that the model can effectively utilize the information contained in the metadata and use it to guide the classification process.

\subsection{Sensitivity Analysis}
% 
In this subsection, we conduct a sensitivity analysis to figure out the effect of different depth settings of adaptive learning layers and fusion layers. 
% In this subsection, we conduct a sensitivity analysis to figure out the effect of different depth settings on the model. 
% Fig.~\ref{fig:fig_sensitivity} presents the effect of different depth settings of adaptive learning layers and fusion layers. 
Taking Fig.~\ref{fig:fig_sensitivity} (a) as an example, the model performance improves with increasing depth, reaching the best performance at a depth of 4.
% Taking Fig.~\ref{fig:fig_sensitivity} (a) as an example, the performance of \shortname~improves with the increase of depth at first, reaching the best performance at a depth of 4.
When the depth continues to increase, the accuracy decreases to varying degrees.
Similar results can be observed in Fig.~\ref{fig:fig_sensitivity} (b).
Therefore, we set their depths to 4 and 6 respectively to achieve the best results.

% Through our experiments, we can observe that the effect of modifying these hyperparameters on the results of the experiments is very weak, and the surface model is not sensitive to the hyperparameters.


\subsection{Zero-shot Capability}
% 

% (1)~GCH~\cite{2010Analyzing} & 21.78\% & (5)~RA-DLNet~\cite{2020A} & 34.01\% \\ \hline
% (2)~WSCNet~\cite{2019WSCNet}  & 30.25\% & (6)~CECCN~\cite{ruan2024color} & 43.83\% \\ \hline
% (3)~PCNN~\cite{2015Robust} & 31.68\%  & (7)~EmoVIT~\cite{xie2024emovit} & 44.90\% \\ \hline
% (4)~AR~\cite{2018Visual} & 32.67\% & (8)~Ours (Zero-shot) & 47.83\% \\ \hline


\begin{table}[t]
\centering
\caption{Zero-shot capability of \shortname.}
\label{tab:cap3}
\resizebox{1\linewidth}{!}
{
\begin{tabular}{lc|lc}
\hline
\textbf{Model} & \textbf{Accuracy} & \textbf{Model} & \textbf{Accuracy} \\ \hline
(1)~WSCNet~\cite{2019WSCNet}  & 30.25\% & (5)~MAM~\cite{zhang2024affective} & 39.56\%  \\ \hline
(2)~AR~\cite{2018Visual} & 32.67\% & (6)~CECCN~\cite{ruan2024color} & 43.83\% \\ \hline
(3)~RA-DLNet~\cite{2020A} & 34.01\%  & (7)~EmoVIT~\cite{xie2024emovit} & 44.90\% \\ \hline
(4)~CDA~\cite{han2023boosting} & 38.64\% & (8)~Ours (Zero-shot) & 47.83\% \\ \hline
\end{tabular}
}
\vspace{-5mm}
\end{table}

% We use the model trained on the FI dataset to test on the artphoto dataset to verify the model's generalization ability as well as robustness to other distributed datasets.
% We can observe that the MESN model shows strong competitiveness in terms of accuracy when compared to other trained models, which suggests that the model has a good generalization ability in the OOD task.

To validate the model's generalization ability and robustness to other distributed datasets, we directly test the model trained on the FI dataset, without training on Artphoto. 
% As observed in Table 3, compared to other models trained on Artphoto, we achieve highly competitive zero-shot performance, indicating that the model has good generalization ability in out-of-distribution tasks.
From Table~\ref{tab:cap3}, we can observe that compared with other models trained on Artphoto, we achieve competitive zero-shot performance, which shows that the model has good generalization ability in out-of-distribution tasks.


%%%%%%%%%%%%
%  E2E     %
%%%%%%%%%%%%


\section{Conclusion}
In this paper, we introduced Wi-Chat, the first LLM-powered Wi-Fi-based human activity recognition system that integrates the reasoning capabilities of large language models with the sensing potential of wireless signals. Our experimental results on a self-collected Wi-Fi CSI dataset demonstrate the promising potential of LLMs in enabling zero-shot Wi-Fi sensing. These findings suggest a new paradigm for human activity recognition that does not rely on extensive labeled data. We hope future research will build upon this direction, further exploring the applications of LLMs in signal processing domains such as IoT, mobile sensing, and radar-based systems.

\section*{Limitations}
While our work represents the first attempt to leverage LLMs for processing Wi-Fi signals, it is a preliminary study focused on a relatively simple task: Wi-Fi-based human activity recognition. This choice allows us to explore the feasibility of LLMs in wireless sensing but also comes with certain limitations.

Our approach primarily evaluates zero-shot performance, which, while promising, may still lag behind traditional supervised learning methods in highly complex or fine-grained recognition tasks. Besides, our study is limited to a controlled environment with a self-collected dataset, and the generalizability of LLMs to diverse real-world scenarios with varying Wi-Fi conditions, environmental interference, and device heterogeneity remains an open question.

Additionally, we have yet to explore the full potential of LLMs in more advanced Wi-Fi sensing applications, such as fine-grained gesture recognition, occupancy detection, and passive health monitoring. Future work should investigate the scalability of LLM-based approaches, their robustness to domain shifts, and their integration with multimodal sensing techniques in broader IoT applications.


% Bibliography entries for the entire Anthology, followed by custom entries
%\bibliography{anthology,custom}
% Custom bibliography entries only
\bibliography{main}
\newpage
\appendix

\section{Experiment prompts}
\label{sec:prompt}
The prompts used in the LLM experiments are shown in the following Table~\ref{tab:prompts}.

\definecolor{titlecolor}{rgb}{0.9, 0.5, 0.1}
\definecolor{anscolor}{rgb}{0.2, 0.5, 0.8}
\definecolor{labelcolor}{HTML}{48a07e}
\begin{table*}[h]
	\centering
	
 % \vspace{-0.2cm}
	
	\begin{center}
		\begin{tikzpicture}[
				chatbox_inner/.style={rectangle, rounded corners, opacity=0, text opacity=1, font=\sffamily\scriptsize, text width=5in, text height=9pt, inner xsep=6pt, inner ysep=6pt},
				chatbox_prompt_inner/.style={chatbox_inner, align=flush left, xshift=0pt, text height=11pt},
				chatbox_user_inner/.style={chatbox_inner, align=flush left, xshift=0pt},
				chatbox_gpt_inner/.style={chatbox_inner, align=flush left, xshift=0pt},
				chatbox/.style={chatbox_inner, draw=black!25, fill=gray!7, opacity=1, text opacity=0},
				chatbox_prompt/.style={chatbox, align=flush left, fill=gray!1.5, draw=black!30, text height=10pt},
				chatbox_user/.style={chatbox, align=flush left},
				chatbox_gpt/.style={chatbox, align=flush left},
				chatbox2/.style={chatbox_gpt, fill=green!25},
				chatbox3/.style={chatbox_gpt, fill=red!20, draw=black!20},
				chatbox4/.style={chatbox_gpt, fill=yellow!30},
				labelbox/.style={rectangle, rounded corners, draw=black!50, font=\sffamily\scriptsize\bfseries, fill=gray!5, inner sep=3pt},
			]
											
			\node[chatbox_user] (q1) {
				\textbf{System prompt}
				\newline
				\newline
				You are a helpful and precise assistant for segmenting and labeling sentences. We would like to request your help on curating a dataset for entity-level hallucination detection.
				\newline \newline
                We will give you a machine generated biography and a list of checked facts about the biography. Each fact consists of a sentence and a label (True/False). Please do the following process. First, breaking down the biography into words. Second, by referring to the provided list of facts, merging some broken down words in the previous step to form meaningful entities. For example, ``strategic thinking'' should be one entity instead of two. Third, according to the labels in the list of facts, labeling each entity as True or False. Specifically, for facts that share a similar sentence structure (\eg, \textit{``He was born on Mach 9, 1941.''} (\texttt{True}) and \textit{``He was born in Ramos Mejia.''} (\texttt{False})), please first assign labels to entities that differ across atomic facts. For example, first labeling ``Mach 9, 1941'' (\texttt{True}) and ``Ramos Mejia'' (\texttt{False}) in the above case. For those entities that are the same across atomic facts (\eg, ``was born'') or are neutral (\eg, ``he,'' ``in,'' and ``on''), please label them as \texttt{True}. For the cases that there is no atomic fact that shares the same sentence structure, please identify the most informative entities in the sentence and label them with the same label as the atomic fact while treating the rest of the entities as \texttt{True}. In the end, output the entities and labels in the following format:
                \begin{itemize}[nosep]
                    \item Entity 1 (Label 1)
                    \item Entity 2 (Label 2)
                    \item ...
                    \item Entity N (Label N)
                \end{itemize}
                % \newline \newline
                Here are two examples:
                \newline\newline
                \textbf{[Example 1]}
                \newline
                [The start of the biography]
                \newline
                \textcolor{titlecolor}{Marianne McAndrew is an American actress and singer, born on November 21, 1942, in Cleveland, Ohio. She began her acting career in the late 1960s, appearing in various television shows and films.}
                \newline
                [The end of the biography]
                \newline \newline
                [The start of the list of checked facts]
                \newline
                \textcolor{anscolor}{[Marianne McAndrew is an American. (False); Marianne McAndrew is an actress. (True); Marianne McAndrew is a singer. (False); Marianne McAndrew was born on November 21, 1942. (False); Marianne McAndrew was born in Cleveland, Ohio. (False); She began her acting career in the late 1960s. (True); She has appeared in various television shows. (True); She has appeared in various films. (True)]}
                \newline
                [The end of the list of checked facts]
                \newline \newline
                [The start of the ideal output]
                \newline
                \textcolor{labelcolor}{[Marianne McAndrew (True); is (True); an (True); American (False); actress (True); and (True); singer (False); , (True); born (True); on (True); November 21, 1942 (False); , (True); in (True); Cleveland, Ohio (False); . (True); She (True); began (True); her (True); acting career (True); in (True); the late 1960s (True); , (True); appearing (True); in (True); various (True); television shows (True); and (True); films (True); . (True)]}
                \newline
                [The end of the ideal output]
				\newline \newline
                \textbf{[Example 2]}
                \newline
                [The start of the biography]
                \newline
                \textcolor{titlecolor}{Doug Sheehan is an American actor who was born on April 27, 1949, in Santa Monica, California. He is best known for his roles in soap operas, including his portrayal of Joe Kelly on ``General Hospital'' and Ben Gibson on ``Knots Landing.''}
                \newline
                [The end of the biography]
                \newline \newline
                [The start of the list of checked facts]
                \newline
                \textcolor{anscolor}{[Doug Sheehan is an American. (True); Doug Sheehan is an actor. (True); Doug Sheehan was born on April 27, 1949. (True); Doug Sheehan was born in Santa Monica, California. (False); He is best known for his roles in soap operas. (True); He portrayed Joe Kelly. (True); Joe Kelly was in General Hospital. (True); General Hospital is a soap opera. (True); He portrayed Ben Gibson. (True); Ben Gibson was in Knots Landing. (True); Knots Landing is a soap opera. (True)]}
                \newline
                [The end of the list of checked facts]
                \newline \newline
                [The start of the ideal output]
                \newline
                \textcolor{labelcolor}{[Doug Sheehan (True); is (True); an (True); American (True); actor (True); who (True); was born (True); on (True); April 27, 1949 (True); in (True); Santa Monica, California (False); . (True); He (True); is (True); best known (True); for (True); his roles in soap operas (True); , (True); including (True); in (True); his portrayal (True); of (True); Joe Kelly (True); on (True); ``General Hospital'' (True); and (True); Ben Gibson (True); on (True); ``Knots Landing.'' (True)]}
                \newline
                [The end of the ideal output]
				\newline \newline
				\textbf{User prompt}
				\newline
				\newline
				[The start of the biography]
				\newline
				\textcolor{magenta}{\texttt{\{BIOGRAPHY\}}}
				\newline
				[The ebd of the biography]
				\newline \newline
				[The start of the list of checked facts]
				\newline
				\textcolor{magenta}{\texttt{\{LIST OF CHECKED FACTS\}}}
				\newline
				[The end of the list of checked facts]
			};
			\node[chatbox_user_inner] (q1_text) at (q1) {
				\textbf{System prompt}
				\newline
				\newline
				You are a helpful and precise assistant for segmenting and labeling sentences. We would like to request your help on curating a dataset for entity-level hallucination detection.
				\newline \newline
                We will give you a machine generated biography and a list of checked facts about the biography. Each fact consists of a sentence and a label (True/False). Please do the following process. First, breaking down the biography into words. Second, by referring to the provided list of facts, merging some broken down words in the previous step to form meaningful entities. For example, ``strategic thinking'' should be one entity instead of two. Third, according to the labels in the list of facts, labeling each entity as True or False. Specifically, for facts that share a similar sentence structure (\eg, \textit{``He was born on Mach 9, 1941.''} (\texttt{True}) and \textit{``He was born in Ramos Mejia.''} (\texttt{False})), please first assign labels to entities that differ across atomic facts. For example, first labeling ``Mach 9, 1941'' (\texttt{True}) and ``Ramos Mejia'' (\texttt{False}) in the above case. For those entities that are the same across atomic facts (\eg, ``was born'') or are neutral (\eg, ``he,'' ``in,'' and ``on''), please label them as \texttt{True}. For the cases that there is no atomic fact that shares the same sentence structure, please identify the most informative entities in the sentence and label them with the same label as the atomic fact while treating the rest of the entities as \texttt{True}. In the end, output the entities and labels in the following format:
                \begin{itemize}[nosep]
                    \item Entity 1 (Label 1)
                    \item Entity 2 (Label 2)
                    \item ...
                    \item Entity N (Label N)
                \end{itemize}
                % \newline \newline
                Here are two examples:
                \newline\newline
                \textbf{[Example 1]}
                \newline
                [The start of the biography]
                \newline
                \textcolor{titlecolor}{Marianne McAndrew is an American actress and singer, born on November 21, 1942, in Cleveland, Ohio. She began her acting career in the late 1960s, appearing in various television shows and films.}
                \newline
                [The end of the biography]
                \newline \newline
                [The start of the list of checked facts]
                \newline
                \textcolor{anscolor}{[Marianne McAndrew is an American. (False); Marianne McAndrew is an actress. (True); Marianne McAndrew is a singer. (False); Marianne McAndrew was born on November 21, 1942. (False); Marianne McAndrew was born in Cleveland, Ohio. (False); She began her acting career in the late 1960s. (True); She has appeared in various television shows. (True); She has appeared in various films. (True)]}
                \newline
                [The end of the list of checked facts]
                \newline \newline
                [The start of the ideal output]
                \newline
                \textcolor{labelcolor}{[Marianne McAndrew (True); is (True); an (True); American (False); actress (True); and (True); singer (False); , (True); born (True); on (True); November 21, 1942 (False); , (True); in (True); Cleveland, Ohio (False); . (True); She (True); began (True); her (True); acting career (True); in (True); the late 1960s (True); , (True); appearing (True); in (True); various (True); television shows (True); and (True); films (True); . (True)]}
                \newline
                [The end of the ideal output]
				\newline \newline
                \textbf{[Example 2]}
                \newline
                [The start of the biography]
                \newline
                \textcolor{titlecolor}{Doug Sheehan is an American actor who was born on April 27, 1949, in Santa Monica, California. He is best known for his roles in soap operas, including his portrayal of Joe Kelly on ``General Hospital'' and Ben Gibson on ``Knots Landing.''}
                \newline
                [The end of the biography]
                \newline \newline
                [The start of the list of checked facts]
                \newline
                \textcolor{anscolor}{[Doug Sheehan is an American. (True); Doug Sheehan is an actor. (True); Doug Sheehan was born on April 27, 1949. (True); Doug Sheehan was born in Santa Monica, California. (False); He is best known for his roles in soap operas. (True); He portrayed Joe Kelly. (True); Joe Kelly was in General Hospital. (True); General Hospital is a soap opera. (True); He portrayed Ben Gibson. (True); Ben Gibson was in Knots Landing. (True); Knots Landing is a soap opera. (True)]}
                \newline
                [The end of the list of checked facts]
                \newline \newline
                [The start of the ideal output]
                \newline
                \textcolor{labelcolor}{[Doug Sheehan (True); is (True); an (True); American (True); actor (True); who (True); was born (True); on (True); April 27, 1949 (True); in (True); Santa Monica, California (False); . (True); He (True); is (True); best known (True); for (True); his roles in soap operas (True); , (True); including (True); in (True); his portrayal (True); of (True); Joe Kelly (True); on (True); ``General Hospital'' (True); and (True); Ben Gibson (True); on (True); ``Knots Landing.'' (True)]}
                \newline
                [The end of the ideal output]
				\newline \newline
				\textbf{User prompt}
				\newline
				\newline
				[The start of the biography]
				\newline
				\textcolor{magenta}{\texttt{\{BIOGRAPHY\}}}
				\newline
				[The ebd of the biography]
				\newline \newline
				[The start of the list of checked facts]
				\newline
				\textcolor{magenta}{\texttt{\{LIST OF CHECKED FACTS\}}}
				\newline
				[The end of the list of checked facts]
			};
		\end{tikzpicture}
        \caption{GPT-4o prompt for labeling hallucinated entities.}\label{tb:gpt-4-prompt}
	\end{center}
\vspace{-0cm}
\end{table*}
% \section{Full Experiment Results}
% \begin{table*}[th]
    \centering
    \small
    \caption{Classification Results}
    \begin{tabular}{lcccc}
        \toprule
        \textbf{Method} & \textbf{Accuracy} & \textbf{Precision} & \textbf{Recall} & \textbf{F1-score} \\
        \midrule
        \multicolumn{5}{c}{\textbf{Zero Shot}} \\
                Zero-shot E-eyes & 0.26 & 0.26 & 0.27 & 0.26 \\
        Zero-shot CARM & 0.24 & 0.24 & 0.24 & 0.24 \\
                Zero-shot SVM & 0.27 & 0.28 & 0.28 & 0.27 \\
        Zero-shot CNN & 0.23 & 0.24 & 0.23 & 0.23 \\
        Zero-shot RNN & 0.26 & 0.26 & 0.26 & 0.26 \\
DeepSeek-0shot & 0.54 & 0.61 & 0.54 & 0.52 \\
DeepSeek-0shot-COT & 0.33 & 0.24 & 0.33 & 0.23 \\
DeepSeek-0shot-Knowledge & 0.45 & 0.46 & 0.45 & 0.44 \\
Gemma2-0shot & 0.35 & 0.22 & 0.38 & 0.27 \\
Gemma2-0shot-COT & 0.36 & 0.22 & 0.36 & 0.27 \\
Gemma2-0shot-Knowledge & 0.32 & 0.18 & 0.34 & 0.20 \\
GPT-4o-mini-0shot & 0.48 & 0.53 & 0.48 & 0.41 \\
GPT-4o-mini-0shot-COT & 0.33 & 0.50 & 0.33 & 0.38 \\
GPT-4o-mini-0shot-Knowledge & 0.49 & 0.31 & 0.49 & 0.36 \\
GPT-4o-0shot & 0.62 & 0.62 & 0.47 & 0.42 \\
GPT-4o-0shot-COT & 0.29 & 0.45 & 0.29 & 0.21 \\
GPT-4o-0shot-Knowledge & 0.44 & 0.52 & 0.44 & 0.39 \\
LLaMA-0shot & 0.32 & 0.25 & 0.32 & 0.24 \\
LLaMA-0shot-COT & 0.12 & 0.25 & 0.12 & 0.09 \\
LLaMA-0shot-Knowledge & 0.32 & 0.25 & 0.32 & 0.28 \\
Mistral-0shot & 0.19 & 0.23 & 0.19 & 0.10 \\
Mistral-0shot-Knowledge & 0.21 & 0.40 & 0.21 & 0.11 \\
        \midrule
        \multicolumn{5}{c}{\textbf{4 Shot}} \\
GPT-4o-mini-4shot & 0.58 & 0.59 & 0.58 & 0.53 \\
GPT-4o-mini-4shot-COT & 0.57 & 0.53 & 0.57 & 0.50 \\
GPT-4o-mini-4shot-Knowledge & 0.56 & 0.51 & 0.56 & 0.47 \\
GPT-4o-4shot & 0.77 & 0.84 & 0.77 & 0.73 \\
GPT-4o-4shot-COT & 0.63 & 0.76 & 0.63 & 0.53 \\
GPT-4o-4shot-Knowledge & 0.72 & 0.82 & 0.71 & 0.66 \\
LLaMA-4shot & 0.29 & 0.24 & 0.29 & 0.21 \\
LLaMA-4shot-COT & 0.20 & 0.30 & 0.20 & 0.13 \\
LLaMA-4shot-Knowledge & 0.15 & 0.23 & 0.13 & 0.13 \\
Mistral-4shot & 0.02 & 0.02 & 0.02 & 0.02 \\
Mistral-4shot-Knowledge & 0.21 & 0.27 & 0.21 & 0.20 \\
        \midrule
        
        \multicolumn{5}{c}{\textbf{Suprevised}} \\
        SVM & 0.94 & 0.92 & 0.91 & 0.91 \\
        CNN & 0.98 & 0.98 & 0.97 & 0.97 \\
        RNN & 0.99 & 0.99 & 0.99 & 0.99 \\
        % \midrule
        % \multicolumn{5}{c}{\textbf{Conventional Wi-Fi-based Human Activity Recognition Systems}} \\
        E-eyes & 1.00 & 1.00 & 1.00 & 1.00 \\
        CARM & 0.98 & 0.98 & 0.98 & 0.98 \\
\midrule
 \multicolumn{5}{c}{\textbf{Vision Models}} \\
           Zero-shot SVM & 0.26 & 0.25 & 0.25 & 0.25 \\
        Zero-shot CNN & 0.26 & 0.25 & 0.26 & 0.26 \\
        Zero-shot RNN & 0.28 & 0.28 & 0.29 & 0.28 \\
        SVM & 0.99 & 0.99 & 0.99 & 0.99 \\
        CNN & 0.98 & 0.99 & 0.98 & 0.98 \\
        RNN & 0.98 & 0.99 & 0.98 & 0.98 \\
GPT-4o-mini-Vision & 0.84 & 0.85 & 0.84 & 0.84 \\
GPT-4o-mini-Vision-COT & 0.90 & 0.91 & 0.90 & 0.90 \\
GPT-4o-Vision & 0.74 & 0.82 & 0.74 & 0.73 \\
GPT-4o-Vision-COT & 0.70 & 0.83 & 0.70 & 0.68 \\
LLaMA-Vision & 0.20 & 0.23 & 0.20 & 0.09 \\
LLaMA-Vision-Knowledge & 0.22 & 0.05 & 0.22 & 0.08 \\

        \bottomrule
    \end{tabular}
    \label{full}
\end{table*}




\end{document}


\clearpage
\appendix

\subsection{Algorithm Illustration}
\label{apx:algorithm}
The whole pipeline of conRFT is outlined in Algorithm \ref{alg:conrft}.

\begin{algorithm}[ht]
    \caption{Procedure of ConRFT } 
    \begin{algorithmic}
        \REQUIRE A pre-trained VLA model $\pi_{\theta, \psi}$ with VLA parameter $\phi$ and a consistency head parameter $\psi$. A critic model $Q$ with parameter $\theta$. A pre-collected dataset $\mathcal{D}$ including 20-30 demonstrations. 
        Initialize batch size $B$.
        \STATE Randomly initialize the action head $\psi$ and the critic model $\theta$
        \STATE \textcolor{gray!90}{\# Stage I: Offline fine-tuning with Cal-ConRFT}
        \FOR{each offline training step}
            \STATE Sample $(s_t, a_t, r_t, s_{t+1})$ of $batch\_size$ from $\mathcal{D}$
            \STATE Update the action head $\psi$ and the critic model $\theta$ by Equation \ref{eq:calql_offline} and Equation \ref{eq:cpql_offline}.
        \ENDFOR
        \STATE \textcolor{gray!90}{\# Stage II: Online fine-tuning with HIL-ConRFT}
        \STATE \textcolor[RGB]{18,220,168}{\textbf{\# Start Policy Learning Thread}:}
        \STATE \textbf{Wait until} The number of transitions in $\mathcal{R}$ is at least 100
        \FOR{each online training step}
            \STATE Sample $(s_t, a_t, r_t, s_{t+1})$ of $\frac{B}{2}$ from $\mathcal{D}$ and $\mathcal{R}$
            \STATE Combine both minibatches to form batch of size $B$
            \STATE Update the action head $\psi$ and the critic model $\theta$ by Equation \ref{eq:ql_online} and Equation \ref{eq:cpql_online}.
        \ENDFOR
        \STATE \textcolor[RGB]{26,205,230}{\textbf{\# Start Interaction Thread}:}
        \FOR{each interaction step}
            \IF{no human intervention} 
                \STATE Take action $a_t \sim \pi_{\psi}(\cdot|s_t)$
                \STATE Store transition $(s_t,a_t,r_t,s_{t+1})$ in $\mathcal{R}$
            \ELSE 
                \STATE Take action $a_{intv}$
                \STATE Store transition $(s_t,a_{intv},r_t,s_{t+1})$ in $\mathcal{D}$
            \ENDIF
        \ENDFOR
    \end{algorithmic} 
    \label{alg:conrft}
\end{algorithm}

\subsection{Task Description, Setup and Policy Training Details}

\label{apx:tasks}
In this section, we provide hardware setup and training details for each task. The 6-dimensional action space refers to the 6-dimensional end-effector delta pose, and the 7-dimensional action space includes the 6-dimensional end-effector delta pose and 1-dimensional gripper control action. The learning rate is 3e-4, and the batch size is 256 for all tasks. 

For the consistency policy utilized for fine-tuning VLA models, we set $k \in [0.002, 80.0]$ and the number of sub-intervals $M=40$ where the sub-interval boundaries are determined with the formula $k_i=(\epsilon^{\frac{1}{\rho}}+\frac{i-1}{M-1}(T^{\frac{1}{\rho}}-\epsilon^{\frac{1}{\rho}}))^{\rho}$, where $\rho=7$. The network is based on a 2-layer multi-layer perceptron (MLP), with a hidden size of 256 and the Mish function serving as the activation function. 

For the diffusion policy, we use diffusion steps $K=5$, a cosine beta schedule, Resnet 18, and the $ LN_Resnet$ architecture, with a hidden size of 256 and $n=3$ blocks.  

For the reward of all tasks, we give $+10$ reward when the task is completed and a $-0.05$ reward on each step. For HIL-SERL, which uses a DQN network for gripper control, we give a $-0.2$ reward every time the policy opens/closes the gripper. 

\paragraph{Pick Banana}

This task involves picking up a banana in the basket and placing it on a green plate, which requires control of the gripper to move the fruit, as shown in Figure \ref{fig:tasks_detail}. It requires the policy to grasp and place the banana, ensuring it remains intact while avoiding collisions with the surrounding environment, such as the basket. We report more specific details of the policy training for this task in Table \ref{tab:task1_detail}. The task description for the VLA model is "Put the yellow banana on the green plate." 

\begin{table}[htbp]
    \centering
    \begin{tabular}{c|c}
        Parameter &Value\\
        \hline 
        Action space &7-dimensional\\
        Initial offline demonstrations &20\\
        Max episode length &100\\
        Reset method &Human reset\\
        Randomization range & 3 cm in x and y\\
        $(\alpha, \beta, \eta)$ for offline fine-tuning & $(0.01, 1.0, 0.1)$\\
        $(\beta, \eta)$ for online fine-tuning & $(0.5, 1.0)$\\
    \end{tabular}
    \caption{\textbf{Policy training details for the Pick Banana task.} }
    \label{tab:task1_detail}
\end{table}

\paragraph{Put Spoon}

This task involves picking up a spoon and placing it on a blue table linen, which requires the gripper to grasp and put the spoon, as shown in Figure \ref{fig:tasks_detail}. The challenge lies in the control needed to grasp the spoon. We report more specific details of the policy training for this task in Table \ref{tab:task2_detail}. The task description for the VLA model is "Put the spoon on the blue towel."

\begin{table}[htbp]
    \centering
    \begin{tabular}{c|c}
        Parameter &Value\\
        \hline 
        Action space &7-dimensional\\
        Initial offline demonstrations &20\\
        Max episode length &100\\
        Reset method &Human reset\\
        Randomization range & 3 cm in x and y\\
        $(\alpha, \beta, \eta)$ for offline fine-tuning & $(0.01, 1.0, 0.1)$\\
        $(\beta, \eta)$ for online fine-tuning & $(0.5, 1.0)$\\
    \end{tabular}
    \caption{\textbf{Policy training details for the Put Spoon task.} }
    \label{tab:task2_detail}
\end{table}

\paragraph{Open Drawer}

This task involves opening a drawer by grasping the handle and pulling it outward, as shown in Figure \ref{fig:tasks_detail}. It requires the policy to securely grip the handle and apply the correct force to open the drawer without damaging the hinges or surrounding area. We report more specific details of the policy training for this task in Table \ref{tab:task3_detail}. The task description for the VLA model is "Open the drawer."

\begin{table}[htbp]
    \centering
    \begin{tabular}{c|c}
        Parameter &Value\\
        \hline 
        Action space &6-dimensional\\
        Initial offline demonstrations &20\\
        Max episode length &100\\
        Reset method &Script reset\\
        Randomization range & 3 cm in y and x\\
        $(\alpha, \beta, \eta)$ for offline fine-tuning & $(0.01, 1.0, 0.1)$\\
        $(\beta, \eta)$ for online fine-tuning & $(0.5, 1.0)$\\
    \end{tabular}
    \caption{\textbf{Policy training details for the Open Drawer task.} }
    \label{tab:task3_detail}
\end{table}

\paragraph{Pick Bread}

This task involves picking up a slice of bread and placing it into a toaster, which requires control of the gripper to position the bread accurately without damaging it, as shown in Figure \ref{fig:tasks_detail}. The challenge lies in aligning the bread with the toaster's slot and lowering it, avoiding collisions with the toaster or the surrounding environment. We report more specific details of the policy training for this task in Table \ref{tab:task4_detail}. The task description for the VLA model is "Put the bread in the grey toaster."

\begin{table}[htbp]
    \centering
    \begin{tabular}{c|c}
        Parameter &Value\\
        \hline 
        Action space &7-dimensional\\
        Initial offline demonstrations &30\\
        Max episode length &100\\
        Reset method &Human reset\\
        Randomization range & 2 cm in x and y\\
        $(\alpha, \beta, \eta)$ for offline fine-tuning & $(0.01, 1.0, 0.1)$\\
        $(\beta, \eta)$ for online fine-tuning & $(0.5, 1.0)$\\
    \end{tabular}
    \caption{\textbf{Policy training details for the Pick Bread task.} }
    \label{tab:task4_detail}
\end{table}

\paragraph{Open Toaster}

This task involves pressing the button on a toaster to start the toasting process, as shown in Figure \ref{fig:tasks_detail}. It requires precise control of the gripper to avoid slipping or applying excessive force while ensuring that the button is pressed in a controlled and consistent manner. We report more specific details of the policy training for this task in Table \ref{tab:task5_detail}. The task description for the VLA model is "Press the black button and open the toaster."

\begin{table}[htbp]
    \centering
    \begin{tabular}{c|c}
        Parameter &Value\\
        \hline 
        Action space &6-dimensional\\
        Initial offline demonstrations &20\\
        Max episode length &100\\
        Reset method &Script reset\\
        Randomization range & 2 cm in y and z\\
        $(\alpha, \beta, \eta)$ for offline fine-tuning & $(0.01, 1.0, 0.1)$\\
        $(\beta, \eta)$ for online fine-tuning & $(0.5, 1.0)$\\
    \end{tabular}
    \caption{\textbf{Policy training details for the Open Toaster task.} }
    \label{tab:task5_detail}
\end{table}

\paragraph{Put Bread}

This task involves picking up a slice of toasted bread from the toaster and placing it on a white plate, as shown in Figure \ref{fig:tasks_detail}. The challenge lies in the precision required to grasp the toast without crushing or damaging it. The gripper must carefully move the toast from the toaster slot while avoiding contact with the toaster's edges or other objects nearby. We report more specific details of the policy training for this task in Table \ref{tab:task6_detail}. The task description for the VLA model is "Put the bread on the white plate."

\begin{table}[htbp]
    \centering
    \begin{tabular}{c|c}
        Parameter &Value\\
        \hline 
        Action space &7-dimensional\\
        Initial offline demonstrations &30\\
        Max episode length &120\\
        Reset method &Human reset\\
        Randomization range & 2 cm in x and y\\
        $(\alpha, \beta, \eta)$ for offline fine-tuning & $(0.01, 1.0, 0.1)$\\
        $(\beta, \eta)$ for online fine-tuning & $(0.5, 1.0)$\\
    \end{tabular}
    \caption{\textbf{Policy training details for the Put Bread task.} }
    \label{tab:task6_detail}
\end{table}

\paragraph{Insert Wheel}

This task involves installing wheels on the chair base by inserting pins into their corresponding slots, as shown in Figure \ref{fig:tasks_detail}. It is a contact-rich task requiring precise control to ensure the pins align correctly with the slots. The complexity of this task increases due to the tight tolerances and complex contact dynamics between the pin and the slot, making it a highly demanding task that requires precision and control. We report more specific details of the policy training for this task in Table \ref{tab:task8_detail}. The task description for the VLA model is "Insert the black wheel into the grey chair base."

\begin{table}[htbp]
    \centering
    \begin{tabular}{c|c}
        Parameter &Value\\
        \hline 
        Action space &7-dimensional\\
        Initial offline demonstrations &30\\
        Max episode length &100\\
        Reset method &Human reset\\
        Randomization range & 2 cm in x and y\\
        $(\alpha, \beta, \eta)$ for offline fine-tuning & $(0.01, 1.0, 0.1)$\\
        $(\beta, \eta)$ for online fine-tuning & $(0.5, 1.0)$\\
    \end{tabular}
    \caption{\textbf{Policy training details for the Insert Wheel task.} }
    \label{tab:task8_detail}
\end{table}

\paragraph{Hang Chinese Knot}

This task involves hanging a Chinese knot on a hook, which requires careful manipulation of a soft and dynamic object, as shown in Figure \ref{fig:tasks_detail}. The task requires fine dexterity to handle the knot's soft body and to maintain its structure while attaching it to the hook. The task involves dealing with the dynamics of soft object manipulation, where maintaining consistent contact and proper tension is critical for success. We report more specific details of the policy training for this task in Table \ref{tab:task9_detail}. The task description for the VLA model is "Hang the Chinese knot on the hook."

\begin{table}[ht]
    \centering
    \begin{tabular}{c|c}
        Parameter &Value\\
        \hline 
        Action space &7-dimensional\\
        Initial offline demonstrations &30\\
        Max episode length &100\\
        Reset method &Human reset\\
        Randomization range & 3 cm in y and z\\
        $(\alpha, \beta, \eta)$ for offline fine-tuning & $(0.01, 1.0, 0.1)$\\
        $(\beta, \eta)$ for online fine-tuning & $(0.5, 1.0)$\\
    \end{tabular}
    \caption{\textbf{Policy training details for the Hang Chinese Knot task.} }
    \label{tab:task9_detail}
\end{table}

\begin{figure*}[htbp]
    \centering
    \includegraphics[width=\linewidth]{tasks_detail.jpg}
    \caption{\textbf{Hardware setup and illustrations of camera views.} We give the illustrations of hardware setup and the corresponding camera views for all real-world tasks in this paper, including a) Pick Banana, b) Put Spoon, c) Open Drawer, d) Pick Bread, e) Open Toaster, f) Put Bread, g) Insert Wheel, h) Hand Chinese Knot.}
    \label{fig:tasks_detail}
\end{figure*}

\subsection{More experiment results}

\label{apx:more_exp}
In this section, we provide all policy learning curves for HIL-ConRFT on all tasks in Figure \ref{fig:all_result}. 

\begin{figure*}[ht]
\centering
\includegraphics[width=\linewidth]{all_result.png}
\caption{\textbf{Learning curves during online training for all tasks.} This figure presents the success rates, intervention rates, and episode lengths, displayed as a running average of over 20 episodes.}
\label{fig:all_result}
\end{figure*}

\end{document}


