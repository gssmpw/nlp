\section{Related work}
This section consists of a review of the literature; first, on haptic feedback at locations other than the wrist and hand, and then, on previous studies of touch perception on the face.
\subsection{Alternative feedback sites}
\label{sc:elsewhere}
Feedback away from the hand and face has been mentioned in the literature in three different contexts: simulation of touch in VR, transmission of information when the hands are otherwise engaged, and sensory substitution for assistive devices. We summarize each of these below.

VR offers strong motivation for developing the ability to provide haptic feedback on areas other than the hand and forearm, to create more realistic experiences in simulated worlds (where objects may touch any part of the body). Since the circumference of the head extends beyond the face, in this work, devices modeled as headbands are treated separately from those designed for the ``face" -- while acknowledging that the forehead is an important site for haptic feedback, and is a subset of the skin area targeted by most head-worn devices. The head and torso have both been used for VR haptics____. Previous work has also explored creative ways to provide a variety of stimuli to the face, using a manipulator arm mounted on the visor____, ultrasonic arrays____, and multi-modal feedback, including vibrotactile and thermal sensations, on the contact surface between the skin and the visor____. While effective for VR experiences, such devices are difficult to apply to the other potential use cases of facial haptics, as they are designed to be attached to a visor that covers the eyes, or to some other off-board mounting location.

Another reason to display information via haptic feedback away from the hands is so that the hands may remain unencumbered. The abdomen____, front____ and back____ of the torso, whole torso____, head____, and earlobes____ have all been studied for localization and pattern recognition performance, distributing vibrotactile actuators over their surface and quantifying recognition accuracy. Specific applications such as navigation guidance have been targeted through such methods, for example, for pilots____ and visually impaired users____.

Finally, as explained in Section~\ref{sc:intro}, the loss of a limb, or loss of sensation in the limb, can make it necessary to provide haptic feedback at the remaining available locations. It has been proposed to use the cheek for haptic feedback in robot teleoperation by possibly repurposing a pneumatically-actuated haptic feedback device designed for the forearm, to apply pressure to the cheek____. This work made a strong argument in favor of facial haptics, based on the high sensory innervation of facial skin____ and the proximity of neural regions mapping to the face and hands____. To these points, we add that the face is of particular interest for upper-limb amputees, as, in many cases, there exists a phantom map of the fingertips on the lower cheek____. However, to the best of our knowledge, there exist no experimental results relating to haptic feedback on the face with a technology \emph{mounted directly on the face} that is feasible to develop for everyday use. 

\subsection{Studies of facial touch sensitivity}
While facial haptics has not been studied comprehensively from the point of view of device design, there exists considerable literature on the sensitivity of the face to touch. The sensory innervation of the face is by the three branches of the trigeminal nerve, innervating the ophthalmic, maxillary, and mandibular areas respectively____. These areas have been studied, both, from a neuroscience perspective, to identify the functional representation of the face in the somatosensory cortex____, and from a clinical perspective, to restore sensation to healthy levels after surgical intervention____.

Numerous studies report two-point discrimination tests, both static and dynamic, by various instruments, finding that resolution increases generally from superior to inferior regions, especially near the lips____. In these works, moving points could be distinguished from each other at approximately a centimeter of separation on the cheek____. However, as noted in previous works, these tests are not sufficient to characterize the response to vibratory stimuli____. Studies on the detection of vibration on facial skin (obtained for the study of speech production)____ have shown that displacement thresholds on the face exceed those on the fingertip, and are comparable to those on the forearm____. It was also shown that, on the face, these thresholds are not very sensitive to changes in frequency____.

From the point of view of device design, it is important to note that the aforementioned two-point discrimination tests are manually administered, without the ability to program stimuli into a device anchored to the face. Similarly, in works studying the correlation between facial touch and phantom fingertip sensation in amputees____, the skin was stroked by an experimenter (or held against a vibration source or wet object). One branch of research has sought to design specialized pneumatic devices, compatible with Magnetic Resonance (MR) environments, in order to provide more consistent vibrotactile stimuli for functional brain imaging studies____. Being aimed at MR-compatibility, these designs are not suited for use in other settings; yet, as examples of device design for facial haptic feedback in the absence of a visor, they illuminate some important design considerations, such as the highly-variable curvature of faces which makes standardized devices difficult to attach.