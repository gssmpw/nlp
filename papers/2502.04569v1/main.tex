\documentclass[conference]{IEEEtran}
\IEEEoverridecommandlockouts
% The preceding line is only needed to identify funding in the first footnote. If that is unneeded, please comment it out.
%Template version as of 6/27/2024

\usepackage{cite}
\usepackage{amsmath,amssymb,amsfonts}
\usepackage{algorithmic}
\usepackage{graphicx}
\usepackage{textcomp}
\usepackage{xcolor}
\usepackage{hyperref}

% \newcommand{\sgc}[1]{\color{blue}SG: #1\color{black}}
\newcommand{\sgc}[1]{}

\def\BibTeX{{\rm B\kern-.05em{\sc i\kern-.025em b}\kern-.08em
    T\kern-.1667em\lower.7ex\hbox{E}\kern-.125emX}}
\begin{document}

\title{Localization of Vibrotactile Stimuli on the Face\\
%{\footnotesize \textsuperscript{*}Note: Sub-titles are not captured for https://ieeexplore.ieee.org  and
%should not be used}
% \thanks{Identify applicable funding agency here. If none, delete this.}
}

\author{\IEEEauthorblockN{Shivani Guptasarma}
\IEEEauthorblockA{\textit{Department of Mechanical Engineering} \\
\textit{Stanford University}\\
Stanford USA \\
shivanig@stanford.edu}
\and
\IEEEauthorblockN{Allison M. Okamura}
\IEEEauthorblockA{\textit{Department of Mechanical Engineering} \\
\textit{Stanford University}\\
Stanford, USA \\
aokamura@stanford.edu}
\and
\IEEEauthorblockN{Monroe Kennedy III}
\IEEEauthorblockA{\textit{Department of Mechanical Engineering} \\
\textit{Stanford University}\\
Stanford, USA \\
monroek@stanford.edu}
% \and
% \IEEEauthorblockN{4\textsuperscript{th} Given Name Surname}
% \IEEEauthorblockA{\textit{dept. name of organization (of Aff.)} \\
% \textit{name of organization (of Aff.)}\\
% City, Country \\
% email address or ORCID}
% \and
% \IEEEauthorblockN{5\textsuperscript{th} Given Name Surname}
% \IEEEauthorblockA{\textit{dept. name of organization (of Aff.)} \\
% \textit{name of organization (of Aff.)}\\
% City, Country \\
% email address or ORCID}
% \and
% \IEEEauthorblockN{6\textsuperscript{th} Given Name Surname}
% \IEEEauthorblockA{\textit{dept. name of organization (of Aff.)} \\
% \textit{name of organization (of Aff.)}\\
% City, Country \\
% email address or ORCID}
}

\maketitle

\begin{abstract}
The face remains relatively unexplored as a target region for haptic feedback, despite providing a considerable surface area consisting of highly sensitive skin. There are promising applications for facial haptic feedback, especially in cases of severe upper limb loss or spinal cord injury, where the face is typically less impacted than other body parts. Moreover, the neural representation of the face is adjacent to that of the hand, and phantom maps have been discovered between the fingertips and the cheeks. However, there is a dearth of compact devices for facial haptic feedback, and vibrotactile stimulation, a common modality of haptic feedback, has not been characterized for localization acuity on the face. We performed a localization experiment on the cheek, with an arrangement of off-the-shelf coin vibration motors. The study follows the methods of prior work studying other skin regionswhich participants attempt to identify the sites of discrete vibrotactile stimuli. We intend for our results to inform the future development of systems using vibrotactile feedback to convey information via the face.
%This document is a model and instructions for \LaTeX.
%This and the IEEEtran.cls file define the components of your paper [title, text, heads, etc.]. *CRITICAL: Do Not Use Symbols, Special Characters, Footnotes, 
%or Math in Paper Title or Abstract.
\end{abstract}

\begin{IEEEkeywords}
Haptic interfaces, Human computer interaction, Sensory aids.
\end{IEEEkeywords}

\section{Introduction}
\label{sc:intro}
The hand and forearm are the most widely-studied regions for haptic feedback, but are not always available or suitable for applications where such feedback might be useful. Upper limb prostheses, for example, may be hugely improved by conveying haptic feedback from sensors on the prosthetic hand to the upper arm~\cite{dey2023}, but not when there is shoulder disarticulation, or forequarter amputation. Users of assistive devices with spinal cord injury might be able to control their devices better with haptic feedback (e.g.,~\cite{deo2021effects}), but might have lost sensitivity in the hand or forearm.   

Facial skin is usually available, with unaffected sensitivity, in both these scenarios; yet, it has not been used for sensory substitution, perhaps because designing any device for the face brings up unique concerns: it must not be bulky or uncomfortable, must not occlude the eyes, ears, nose or jaw, and must be socially acceptable. Design of haptic interfaces for the face typically focuses on Virtual Reality (VR) applications rather than everyday tasks. In these applications, the concerns mentioned above do not arise, as the goal is not to routinely convey information from elsewhere, but to simulate touch on the face itself, for a short period of time.  

We argue that there are several reasons to seriously consider the face as a target for haptic feedback, and to put efforts into designing facial haptic interfaces, wearable directly on the face, that account for its specific challenges. First, as mentioned, there exist applications for haptic feedback where the hand and forearm cannot be used. Second, there exist, already, several devices and accessories worn on the face, which have come to be regarded as ordinary, and there is reason to imagine that a well-designed haptic interface might do the same. Third, there has been fundamental research relating the neural representations of the fingertips and the face, and the plasticity of the brain regions responsible for sensation on the face and hands~\cite{ramachandran1998ThePO} -- which opens up immense potential for sensory substitution via the face in neural and myoelectric prostheses. 

In this work, we report a study on the localization accuracy of an array of coin vibration motors placed on the cheek using a skin-safe adhesive. We begin with a discussion of the design considerations and our approach towards choosing the vibrotactile modality. Then, we present the responses of~$10$ volunteers to vibrotactile stimulation of $12$~sites, actuated one at a time. We placed the actuators on the front and side of the cheek, taking care to cover the lower cheek, where phantom maps occur in some amputees~\cite{ramachandran1998ThePO}. Participants were asked to identify the location of each stimulus and to give free-form feedback about the interface. 

The contributions of this work are: (a) a discussion of the motivations and design considerations for facial haptic feedback for everyday applications, and (b) preliminary experimental results on the localization acuity of vibrotactile stimuli on the cheek. Based on our results, we propose future studies that can further inform the design of vibrotactile interfaces for the cheek.

% Haptic feedback can be used to convey touch from prosthetic limbs~\cite{dey2023}, biofeedback during the operation of assistive technology~\cite{deo2021effects} or performance of tasks and activities~\cite{jafari2016}, end-effector forces for robotic teleoperation~\cite{patel2022review}, and simulated touch in virtual interactions~\cite{shi2024review}. In several of these scenarios, the hand and forearm are either unavailable or unsuitable as target regions for feedback. 

% In this work, we motivate the study of the face as a target for haptic feedback, 

% and study localization accuracy for a cheek array of coin vibration motors.

\section{Related work}
This section consists of a review of the literature; first, on haptic feedback at locations other than the wrist and hand, and then, on previous studies of touch perception on the face.
\subsection{Alternative feedback sites}
\label{sc:elsewhere}
Feedback away from the hand and face has been mentioned in the literature in three different contexts: simulation of touch in VR, transmission of information when the hands are otherwise engaged, and sensory substitution for assistive devices. We summarize each of these below.

VR offers strong motivation for developing the ability to provide haptic feedback on areas other than the hand and forearm, to create more realistic experiences in simulated worlds (where objects may touch any part of the body). Since the circumference of the head extends beyond the face, in this work, devices modeled as headbands are treated separately from those designed for the ``face" -- while acknowledging that the forehead is an important site for haptic feedback, and is a subset of the skin area targeted by most head-worn devices. The head and torso have both been used for VR haptics~\cite{adilkhanoc2022review, kaul2016haptichead}. Previous work has also explored creative ways to provide a variety of stimuli to the face, using a manipulator arm mounted on the visor~\cite{wilberz2020}, ultrasonic arrays~\cite{shen2022, lan2024}, and multi-modal feedback, including vibrotactile and thermal sensations, on the contact surface between the skin and the visor~\cite{wolf2019face}. While effective for VR experiences, such devices are difficult to apply to the other potential use cases of facial haptics, as they are designed to be attached to a visor that covers the eyes, or to some other off-board mounting location.

Another reason to display information via haptic feedback away from the hands is so that the hands may remain unencumbered. The abdomen~\cite{cholewiakVibrotactileLocalizationAbdomen2004b}, front~\cite{vanerpAbsoluteLocalizationVibrotactile2008} and back~\cite{jones2008, jones2009} of the torso, whole torso~\cite{kim2023}, head~\cite{gilliland1994}, and earlobes~\cite{lee2019activearring} have all been studied for localization and pattern recognition performance, distributing vibrotactile actuators over their surface and quantifying recognition accuracy. Specific applications such as navigation guidance have been targeted through such methods, for example, for pilots~\cite{cholewiakVibrotactileLocalizationAbdomen2004b} and visually impaired users~\cite{katzschmann2018blind}.

Finally, as explained in Section~\ref{sc:intro}, the loss of a limb, or loss of sensation in the limb, can make it necessary to provide haptic feedback at the remaining available locations. It has been proposed to use the cheek for haptic feedback in robot teleoperation by possibly repurposing a pneumatically-actuated haptic feedback device designed for the forearm, to apply pressure to the cheek~\cite{guptasarma2024}. This work made a strong argument in favor of facial haptics, based on the high sensory innervation of facial skin~\cite{corniani2020tactile} and the proximity of neural regions mapping to the face and hands~\cite{leemhuis2022rethinking}. To these points, we add that the face is of particular interest for upper-limb amputees, as, in many cases, there exists a phantom map of the fingertips on the lower cheek~\cite{ramachandran1998ThePO}. However, to the best of our knowledge, there exist no experimental results relating to haptic feedback on the face with a technology \emph{mounted directly on the face} that is feasible to develop for everyday use. 

\subsection{Studies of facial touch sensitivity}
While facial haptics has not been studied comprehensively from the point of view of device design, there exists considerable literature on the sensitivity of the face to touch. The sensory innervation of the face is by the three branches of the trigeminal nerve, innervating the ophthalmic, maxillary, and mandibular areas respectively~\cite{gray1918}. These areas have been studied, both, from a neuroscience perspective, to identify the functional representation of the face in the somatosensory cortex~\cite{kikkert2023}, and from a clinical perspective, to restore sensation to healthy levels after surgical intervention~\cite{siemionow2011}.

Numerous studies report two-point discrimination tests, both static and dynamic, by various instruments, finding that resolution increases generally from superior to inferior regions, especially near the lips~\cite{costasNormalSensationHuman1994, fogacaEvaluationCutaneousSensibility2005, vriensExtensionNormalValues2009b}. In these works, moving points could be distinguished from each other at approximately a centimeter of separation on the cheek~\cite{fogacaEvaluationCutaneousSensibility2005}. However, as noted in previous works, these tests are not sufficient to characterize the response to vibratory stimuli~\cite{cholewiakVibrotactileLocalizationAbdomen2004b}. Studies on the detection of vibration on facial skin (obtained for the study of speech production)~\cite{barlow1987} have shown that displacement thresholds on the face exceed those on the fingertip, and are comparable to those on the forearm~\cite{morioka2008vibrotactile}. It was also shown that, on the face, these thresholds are not very sensitive to changes in frequency~\cite{barlow1987}.

From the point of view of device design, it is important to note that the aforementioned two-point discrimination tests are manually administered, without the ability to program stimuli into a device anchored to the face. Similarly, in works studying the correlation between facial touch and phantom fingertip sensation in amputees~\cite{ramachandran1998ThePO}, the skin was stroked by an experimenter (or held against a vibration source or wet object). One branch of research has sought to design specialized pneumatic devices, compatible with Magnetic Resonance (MR) environments, in order to provide more consistent vibrotactile stimuli for functional brain imaging studies~\cite{kikkert2023}. Being aimed at MR-compatibility, these designs are not suited for use in other settings; yet, as examples of device design for facial haptic feedback in the absence of a visor, they illuminate some important design considerations, such as the highly-variable curvature of faces which makes standardized devices difficult to attach. 
\section{Methods}
In this section, we first describe the design considerations behind to our decision to use the vibrotactile modality for this first experimental study, then report the materials and protocol for the experiment.

\subsection{Modality and mechanism}
For prosthetic and assistive applications, various types of information can be conveyed via haptics, including contact made or lost at different locations, the magnitude of normal forces, magnitude and direction of shear and torsion, as well as warmth, edges, and texture. Both discrete and continuous stimuli may be useful. In this study, we focus on discrete stimuli.

Haptic stimulation can be provided by skin deformation (poking~\cite{guptasarma2024}, pinching, twisting, or shearing~\cite{jayatilake2012stretch}), as well as vibration, thermal, or electrical stimulation. Because skin on the face is often compliant, the effective receptive field depends not only upon the mechanoreceptors stimulated, but also the size of the skin region that experiences displacements when a localized stimulus is delivered. The various stimuli can be capable of providing either continuous or discrete information, through switching on/off, or activating in spatiotemporal patterns (discrete) or the modulation of amplitude or frequency (continuous).

Skin deformation mechanisms have an advantage in that they often convey sensations through the same modality as originally intended (for example, mapping touch on the face to touch on the fingertip~\cite{ramachandran1998ThePO}), whereas vibration feedback is typically a form of sensory substitution. However, skin deformation mechanisms pose a considerable challenge because they require successful attachment and leverage on a body part that varies widely in shape, size, stiffness, texture, and skin health. Previous works describing such designs have faced this challenge by relying on custom face masks~\cite{kikkert2023}, since they were needed only inside MR scanners, or have attempted, with uncertain success, to use tight straps against bony protuberances~\cite{guptasarma2024}. Being made of soft materials, these were typically pneumatically driven, requiring noisy and bulky off-board compressors with a power supply. The prominent non-pneumatic designs have been the VR devices mentioned in Section~\ref{sc:elsewhere}, mounted on the headset visor.

In comparison to mechanisms for skin deformation feedback, vibrotactile actuators can be selected to be far more compact, such that even if the information that may be conveyed per actuator site is less (which is not known), more sites may be used over the available area. Being ubiquitous in smartphones, they have been developed to a point where they are widely available and economically accessible. Since their moving components are within their casing, they are also easy to mount on most surfaces.

In addition to mechanical considerations is a crucial logistical and social one: to be acceptable for continuous widespread use, a device designed for the face must not interfere with communication in any way, whether verbal or nonverbal, and must not impede mastication. It must not be physically uncomfortable or perceived as undesirable. In this context, it is encouraging to note that wearable items for the face and head are common in societies worldwide: including spectacles, monocles, hearing aids, earphones, headphones, headbands, over-the-ear microphones, earrings and noserings, among others. This list includes both active and passive devices, but with the active ones never undergoing large-scale external motions. Based on these observations, we hypothesize that, at present, vibrotactile feedback is the most promising choice for facial haptics, with actuators being relatively small, lightweight, and appearing far more passive (externally) than existing mechanisms for applying skin deformation. Thermal and electrical stimulation are also attractive for their potentially low profile configurations, but these actuators are less readily available.

\subsection{Vibrotactile actuators}
The three choices available for generating vibration are piezoelectric actuators, Eccentric Rotating Mass (ERM) motors and Linear Resonant Actuators (LRA). Piezoelectric actuators are taken out of consideration for this application by the high voltages required to drive them. The remaining two types of actuators are both suitable for haptic feedback. Frequency and amplitude are coupled in ERM motors, while LRAs offer greater control over the amplitude over a narrow range of frequency. LRAs are used with drivers to provide the required low-voltage alternating current, whereas ERM motors work with direct current (DC). As ERM is the more mature technology, these motors are more easily available and at lower cost. For this first study, we choose to use one stimulus at a fixed amplitude, frequency and pulse duration across all sites, hence, ERM motors are sufficient.

For this experiment, we used~$12$~Tatako~$12000$~RPM ($200$~Hz) coin motors of~$10$~mm diameter and~$3$~mm thickness, purchased for under~$0.70$~USD per motor. An appropriate frequency range was determined through a pilot test, concluding that very low frequencies were mechanically uncomfortable while very high frequencies created a shrill sound, undesirable so close to the ears. We also found that placing the motors on the nose occluded vision and created an unpleasant sensation (sometimes prompting sneezing), and that, on the forehead and tip of the chin, there was excessive transfer of vibration to the bone, which was also unpleasant. The decision was thus made to place the motors on the cheeks.

In order to achieve a medium frequency vibration, the motors (rated for~$3$~V~DC/$80$~mA) were actuated with a Pulse-Width Modulated signal with peak~$5$~V at a~$59\%$ duty cycle from the digital pins of an Arduino (Elegoo) Mega 2560 microcontroller board. The true amplitude of vibration for a face-mounted device (unlike when measuring thresholds with an externally-fixed device, e.g.,~\cite{barlow1987}) is dependent upon the mechanical response of the skin. The number of motors was selected as a balance between adequately covering the area of a single cheek and avoiding overburdening participants cognitively. The motors were placed at gaps of approximately~$10$~mm (one diameter, which is also the order of magnitude of the two-point discrimination threshold on the cheek~\cite{fogacaEvaluationCutaneousSensibility2005}), varying slightly with location. As vibration for prolonged periods can be uncomfortable, each individual stimulus lasted for~$200$~ms. The voltage across, and current through, the motors during operation were~$2.8$~V and $63$~mA respectively. %The acceleration of the motors when measured using an ADXL377 accelerometer (weighing~$1.18$~g, while each motor weighed~$0.94$~g) had an amplitude of approximately~$16~g$ at~$11$~Hz. \sgc{I think the accelerometer is picking up some other frequency -- my setup is probably inadequate to get an accurate reading. An FFT is attached. I did see a second peak at 124 Hz.}

\subsection{Protocol}
We recruited a total of~$10$~volunteers ($6$~female and~$4$~male) with no current injuries on the face and little to no facial hair on the cheeks. All participants gave informed consent. Participant~7 was cross-dominant, using his right hand for writing and fine motor control. The other nine participants were right-handed. All participants had uncorrected hearing and either uncorrected or fully-corrected vision.

For consistency, we placed the motors in a~$4\times3$ grid drawn using a flexible stencil made by punching holes in a polyethylene sheet. The stencil was oriented with reference to a line drawn from the base of the right nostril to the top of the right ear, with the top row of motors falling above this line and the others below, as shown on the left in Fig.~\ref{fig:cheek}. To account for the different ways in which this standard grid might lie on participants' faces, photographs and measurements of facial dimensions were recorded for each participant, with consent.

Seating participants before a mirror and using wet wipes to prepare the skin, motors were attached to the right cheek, using a skin-safe adhesive (we used glue made for false eyelashes rather than for theater prosthetics, in the hope that it might be hypoallergenic for a greater part of the population). Adhesive tape was avoided, as were fabric masks, so as to minimize unnecessary stimulation of areas surrounding the attachment site, as well as to avoid the transmission of vibration between sites, through materials other than the cheek itself.

\begin{figure}[htbp]
\centerline{\includegraphics[width=0.5\textwidth]{face_gui.png}}
\caption{Left: the array of actuators on a participant's cheek; Right: the interface shown to participants for collecting responses.}
\label{fig:cheek}
\end{figure}

An array of buttons representing motor attachment sites (Fig.~\ref{fig:cheek}, right) was displayed on the touchscreen of a Microsoft Surface Pro~($2.50$~GHz processor,~$60$~Hz display). Due to the variation in face size and shape, and therefore, the placement relative to facial landmarks, a regular rectangular grid was shown on the screen, rather than a schematic of the face. A Python script running on the same device sent stimuli from a constrained random sequence, including each site~$3$ times, to the microcontroller. No two directly neighboring sites (within the same row or column) were actuated consecutively. Two such sessions were conducted without a break, placing the screen on the front-left or front-right of the participant respectively, so that the participant used their left and their right hand an equal number of times to touch the buttons.

As a calibration round prior to each of the two sessions, participants were asked to close their eyes and all sites from~$1$-$12$ were actuated while being verbally identified by the experimenter. During this time, participants were allowed to request the repetition of any stimuli until they were satisfied.

During the sessions, to ensure a consistent gaze direction, participants were asked to gaze at a certain point on the image displayed on the screen, while the buttons were hidden. The buttons reappeared~$0.5$~s after each stimulus was delivered, disappearing after each response. Each stimulus was delivered~$3$~s after the previous response.

Participants wore noise-canceling headphones, playing static, during sessions; however, it is impossible to isolate the effect of auditory cues on the behavior of hearing participants in such a study, since any vibrations on the cheek are transmitted to the bones, and can be heard internally.

Finally, each participant was given a brief verbal interview consisting of the following questions: (a) "Was the method of attachment uncomfortable?", and (b) "Was the sensation of vibration on the face unpleasant?". They were also asked if they had any other comments or observations. Selected insights from these comments are mentioned in Section~\ref{sc:results}.

\section{Results}
\label{sc:results}
Confusion matrices for all~$10$ participants are shown in Fig.~\ref{fig:conf}. Each stimulus was delivered a total of~$6$ times during the two sessions. 
\begin{figure*}[htbp]
\centerline{\includegraphics[width=\textwidth]{confusion_matrices_pooled.png}}
\caption{Confusion matrices showing correct responses along the diagonal, and misclassifications off the diagonal.}
\label{fig:conf}
\vspace{-1em}
\end{figure*}
The number of accurate responses across the~$12$ sites, and overall accuracy for each participant, are shown in Fig.~\ref{fig:accgrid}. Overall, it appears that sites near the extremes of the grid are the easiest to correctly identify. %\sgc{Check if variation is due to dimensions}
\begin{figure}[htbp]
\centerline{\includegraphics[width=0.55\textwidth]{accuracy_grids_pooled.png}}
\caption{The number of correct responses per site, for each participant. L-R is medial-lateral in each grid. }
\label{fig:accgrid}
\vspace{-1em}
\end{figure}

Aggregating errors at all sites, the shift from stimulus to response site was classified as medial or lateral along one axis, or inferior or superior along the other (Fig.~\ref{fig:shift}). Several participants perceived vibrations to be originating higher than their true origins (superior shift), albeit with a few exceptions. Meanwhile, there is a marked lateral shift (away from nose, towards ear) in the responses recorded from all participants.

\begin{figure}[htbp]
\centerline{\includegraphics[width=0.6\textwidth]{shift_correct_ratios.png}}
\caption{The directionality of errors by each participant; errors along columns and rows shown separately.}
\label{fig:shift}
\end{figure}

This lateral shift is also evident in Fig.\ref{fig:hist}, which shows the frequencies of responses given for each stimulus, aggregated over all participants. While stimuli close to the ear (3, 6, 9, and 12) and close to the nose and mouth (1, 4, and 7) were overwhelmingly localized to the correct column, those in the middle column were often misjudged to be closer to the ear. 
\begin{figure}[htbp]
\centerline{\includegraphics[width=0.4\textwidth, trim={1em 0 1em 0}]{response_2d_histograms.png}}
\caption{The frequencies of responses for each stimulus site, aggregated over participants. L-R is medial-lateral.}
\label{fig:hist}
\vspace{-1em}
\end{figure}

Subjective feedback from all participants confirmed that the attachment was not uncomfortable and vibration was not unpleasant. Two participants declared unprompted, during sessions, that it was a pleasant feeling. However, participants noted that if vibration was felt without warning (when the device was first connected), it was very startling. Participants reported being able to hear some vibrations through their bones, although opinions varied on whether this might have influenced their responses. A few participants mentioned that the softer part of the lower cheek, near the lip, was weighed down somewhat by the motors. Several participants remarked that the task was more difficult than expected, and that they wished the face were more sensitive.

\section{Discussion}
From the above results, it appears that while localization acuity on the cheek may be lower for vibrotactile stimuli than for light touch or pressure, localization is likely to be feasible with a smaller number of actuators placed at greater distances from each other, e.g., at the corners of the grid. 

A limitation of the setup was that some of the connecting wires could not be kept from touching the cheek (in real applications, cables may be tucked away behind the ear). While this was not uncomfortable for participants, it may have delivered additional vibrations at unintended locations. Motors placed very close to the eyes were also noticeable in the peripheral vision, which may or may not be acceptable in a real-world application.

The lateral shift found in the responses is reminiscent of similar phenomena in auditory localization~\cite{Lewald1996473} and in tactile localization on the waist~\cite{HO2007136}. In the present work, we did not study the relationship of head or eye position with this effect, but it is worthy of further study. 

Future work comparing the dominant and non-dominant side, and investigating the effect of head pose and eye gaze, is needed before vibrotactile stimulation can be adopted for haptic feedback applications on the face. It is necessary to identify the highest number of sites -- perhaps~$4$-$6$ -- that can be reliably distinguished. It is also of interest to experiment with the identification of spatiotemporal patterns, for example, simultaneous or sequential actuations at more than one site.

\errorcontextlines=99
\bibliography{main}{}
\bibliographystyle{IEEEtran}

% \vspace{12pt}
% \color{red}
% IEEE conference templates contain guidance text for composing and formatting conference papers. Please ensure that all template text is removed from your conference paper prior to submission to the conference. Failure to remove the template text from your paper may result in your paper not being published.

\end{document}

% 
This document is a model and instructions for \LaTeX.
Please observe the conference page limits. For more information about how to become an IEEE Conference author or how to write your paper, please visit   IEEE Conference Author Center website: https://conferences.ieeeauthorcenter.ieee.org/.

\subsection{Maintaining the Integrity of the Specifications}

The IEEEtran class file is used to format your paper and style the text. All margins, 
column widths, line spaces, and text fonts are prescribed; please do not 
alter them. You may note peculiarities. For example, the head margin
measures proportionately more than is customary. This measurement 
and others are deliberate, using specifications that anticipate your paper 
as one part of the entire proceedings, and not as an independent document. 
Please do not revise any of the current designations.

\section{Prepare Your Paper Before Styling}
Before you begin to format your paper, first write and save the content as a 
separate text file. Complete all content and organizational editing before 
formatting. Please note sections \ref{AA} to \ref{FAT} below for more information on 
proofreading, spelling and grammar.

Keep your text and graphic files separate until after the text has been 
formatted and styled. Do not number text heads---{\LaTeX} will do that 
for you.

\subsection{Abbreviations and Acronyms}\label{AA}
Define abbreviations and acronyms the first time they are used in the text, 
even after they have been defined in the abstract. Abbreviations such as 
IEEE, SI, MKS, CGS, ac, dc, and rms do not have to be defined. Do not use 
abbreviations in the title or heads unless they are unavoidable.

\subsection{Units}
\begin{itemize}
\item Use either SI (MKS) or CGS as primary units. (SI units are encouraged.) English units may be used as secondary units (in parentheses). An exception would be the use of English units as identifiers in trade, such as ``3.5-inch disk drive''.
\item Avoid combining SI and CGS units, such as current in amperes and magnetic field in oersteds. This often leads to confusion because equations do not balance dimensionally. If you must use mixed units, clearly state the units for each quantity that you use in an equation.
\item Do not mix complete spellings and abbreviations of units: ``Wb/m\textsuperscript{2}'' or ``webers per square meter'', not ``webers/m\textsuperscript{2}''. Spell out units when they appear in text: ``. . . a few henries'', not ``. . . a few H''.
\item Use a zero before decimal points: ``0.25'', not ``.25''. Use ``cm\textsuperscript{3}'', not ``cc''.)
\end{itemize}

\subsection{Equations}
Number equations consecutively. To make your 
equations more compact, you may use the solidus (~/~), the exp function, or 
appropriate exponents. Italicize Roman symbols for quantities and variables, 
but not Greek symbols. Use a long dash rather than a hyphen for a minus 
sign. Punctuate equations with commas or periods when they are part of a 
sentence, as in:
\begin{equation}
a+b=\gamma\label{eq}
\end{equation}

Be sure that the 
symbols in your equation have been defined before or immediately following 
the equation. Use ``\eqref{eq}'', not ``Eq.~\eqref{eq}'' or ``equation \eqref{eq}'', except at 
the beginning of a sentence: ``Equation \eqref{eq} is . . .''

\subsection{\LaTeX-Specific Advice}

Please use ``soft'' (e.g., \verb|\eqref{Eq}|) cross references instead
of ``hard'' references (e.g., \verb|(1)|). That will make it possible
to combine sections, add equations, or change the order of figures or
citations without having to go through the file line by line.

Please don't use the \verb|{eqnarray}| equation environment. Use
\verb|{align}| or \verb|{IEEEeqnarray}| instead. The \verb|{eqnarray}|
environment leaves unsightly spaces around relation symbols.

Please note that the \verb|{subequations}| environment in {\LaTeX}
will increment the main equation counter even when there are no
equation numbers displayed. If you forget that, you might write an
article in which the equation numbers skip from (17) to (20), causing
the copy editors to wonder if you've discovered a new method of
counting.

{\BibTeX} does not work by magic. It doesn't get the bibliographic
data from thin air but from .bib files. If you use {\BibTeX} to produce a
bibliography you must send the .bib files. 

{\LaTeX} can't read your mind. If you assign the same label to a
subsubsection and a table, you might find that Table I has been cross
referenced as Table IV-B3. 

{\LaTeX} does not have precognitive abilities. If you put a
\verb|\label| command before the command that updates the counter it's
supposed to be using, the label will pick up the last counter to be
cross referenced instead. In particular, a \verb|\label| command
should not go before the caption of a figure or a table.

Do not use \verb|\nonumber| inside the \verb|{array}| environment. It
will not stop equation numbers inside \verb|{array}| (there won't be
any anyway) and it might stop a wanted equation number in the
surrounding equation.

\subsection{Some Common Mistakes}\label{SCM}
\begin{itemize}
\item The word ``data'' is plural, not singular.
\item The subscript for the permeability of vacuum $\mu_{0}$, and other common scientific constants, is zero with subscript formatting, not a lowercase letter ``o''.
\item In American English, commas, semicolons, periods, question and exclamation marks are located within quotation marks only when a complete thought or name is cited, such as a title or full quotation. When quotation marks are used, instead of a bold or italic typeface, to highlight a word or phrase, punctuation should appear outside of the quotation marks. A parenthetical phrase or statement at the end of a sentence is punctuated outside of the closing parenthesis (like this). (A parenthetical sentence is punctuated within the parentheses.)
\item A graph within a graph is an ``inset'', not an ``insert''. The word alternatively is preferred to the word ``alternately'' (unless you really mean something that alternates).
\item Do not use the word ``essentially'' to mean ``approximately'' or ``effectively''.
\item In your paper title, if the words ``that uses'' can accurately replace the word ``using'', capitalize the ``u''; if not, keep using lower-cased.
\item Be aware of the different meanings of the homophones ``affect'' and ``effect'', ``complement'' and ``compliment'', ``discreet'' and ``discrete'', ``principal'' and ``principle''.
\item Do not confuse ``imply'' and ``infer''.
\item The prefix ``non'' is not a word; it should be joined to the word it modifies, usually without a hyphen.
\item There is no period after the ``et'' in the Latin abbreviation ``et al.''.
\item The abbreviation ``i.e.'' means ``that is'', and the abbreviation ``e.g.'' means ``for example''.
\end{itemize}
An excellent style manual for science writers is \cite{b7}.

\subsection{Authors and Affiliations}\label{AAA}
\textbf{The class file is designed for, but not limited to, six authors.} A 
minimum of one author is required for all conference articles. Author names 
should be listed starting from left to right and then moving down to the 
next line. This is the author sequence that will be used in future citations 
and by indexing services. Names should not be listed in columns nor group by 
affiliation. Please keep your affiliations as succinct as possible (for 
example, do not differentiate among departments of the same organization).

\subsection{Identify the Headings}\label{ITH}
Headings, or heads, are organizational devices that guide the reader through 
your paper. There are two types: component heads and text heads.

Component heads identify the different components of your paper and are not 
topically subordinate to each other. Examples include Acknowledgments and 
References and, for these, the correct style to use is ``Heading 5''. Use 
``figure caption'' for your Figure captions, and ``table head'' for your 
table title. Run-in heads, such as ``Abstract'', will require you to apply a 
style (in this case, italic) in addition to the style provided by the drop 
down menu to differentiate the head from the text.

Text heads organize the topics on a relational, hierarchical basis. For 
example, the paper title is the primary text head because all subsequent 
material relates and elaborates on this one topic. If there are two or more 
sub-topics, the next level head (uppercase Roman numerals) should be used 
and, conversely, if there are not at least two sub-topics, then no subheads 
should be introduced.

\subsection{Figures and Tables}\label{FAT}
\paragraph{Positioning Figures and Tables} Place figures and tables at the top and 
bottom of columns. Avoid placing them in the middle of columns. Large 
figures and tables may span across both columns. Figure captions should be 
below the figures; table heads should appear above the tables. Insert 
figures and tables after they are cited in the text. Use the abbreviation 
``Fig.~\ref{fig}'', even at the beginning of a sentence.

\begin{table}[htbp]
\caption{Table Type Styles}
\begin{center}
\begin{tabular}{|c|c|c|c|}
\hline
\textbf{Table}&\multicolumn{3}{|c|}{\textbf{Table Column Head}} \\
\cline{2-4} 
\textbf{Head} & \textbf{\textit{Table column subhead}}& \textbf{\textit{Subhead}}& \textbf{\textit{Subhead}} \\
\hline
copy& More table copy$^{\mathrm{a}}$& &  \\
\hline
\multicolumn{4}{l}{$^{\mathrm{a}}$Sample of a Table footnote.}
\end{tabular}
\label{tab1}
\end{center}
\end{table}

\begin{figure}[htbp]
\centerline{\includegraphics{fig1.png}}
\caption{Example of a figure caption.}
\label{fig}
\end{figure}

Figure Labels: Use 8 point Times New Roman for Figure labels. Use words 
rather than symbols or abbreviations when writing Figure axis labels to 
avoid confusing the reader. As an example, write the quantity 
``Magnetization'', or ``Magnetization, M'', not just ``M''. If including 
units in the label, present them within parentheses. Do not label axes only 
with units. In the example, write ``Magnetization (A/m)'' or ``Magnetization 
\{A[m(1)]\}'', not just ``A/m''. Do not label axes with a ratio of 
quantities and units. For example, write ``Temperature (K)'', not 
``Temperature/K''.

\section*{Acknowledgment}

The preferred spelling of the word ``acknowledgment'' in America is without 
an ``e'' after the ``g''. Avoid the stilted expression ``one of us (R. B. 
G.) thanks $\ldots$''. Instead, try ``R. B. G. thanks$\ldots$''. Put sponsor 
acknowledgments in the unnumbered footnote on the first page.

\section*{References}

Please number citations consecutively within brackets \cite{b1}. The 
sentence punctuation follows the bracket \cite{b2}. Refer simply to the reference 
number, as in \cite{b3}---do not use ``Ref. \cite{b3}'' or ``reference \cite{b3}'' except at 
the beginning of a sentence: ``Reference \cite{b3} was the first $\ldots$''

Number footnotes separately in superscripts. Place the actual footnote at 
the bottom of the column in which it was cited. Do not put footnotes in the 
abstract or reference list. Use letters for table footnotes.

Unless there are six authors or more give all authors' names; do not use 
``et al.''. Papers that have not been published, even if they have been 
submitted for publication, should be cited as ``unpublished'' \cite{b4}. Papers 
that have been accepted for publication should be cited as ``in press'' \cite{b5}. 
Capitalize only the first word in a paper title, except for proper nouns and 
element symbols.

For papers published in translation journals, please give the English 
citation first, followed by the original foreign-language citation \cite{b6}.

\begin{thebibliography}{00}
\bibitem{b1} G. Eason, B. Noble, and I. N. Sneddon, ``On certain integrals of Lipschitz-Hankel type involving products of Bessel functions,'' Phil. Trans. Roy. Soc. London, vol. A247, pp. 529--551, April 1955.
\bibitem{b2} J. Clerk Maxwell, A Treatise on Electricity and Magnetism, 3rd ed., vol. 2. Oxford: Clarendon, 1892, pp.68--73.
\bibitem{b3} I. S. Jacobs and C. P. Bean, ``Fine particles, thin films and exchange anisotropy,'' in Magnetism, vol. III, G. T. Rado and H. Suhl, Eds. New York: Academic, 1963, pp. 271--350.
\bibitem{b4} K. Elissa, ``Title of paper if known,'' unpublished.
\bibitem{b5} R. Nicole, ``Title of paper with only first word capitalized,'' J. Name Stand. Abbrev., in press.
\bibitem{b6} Y. Yorozu, M. Hirano, K. Oka, and Y. Tagawa, ``Electron spectroscopy studies on magneto-optical media and plastic substrate interface,'' IEEE Transl. J. Magn. Japan, vol. 2, pp. 740--741, August 1987 [Digests 9th Annual Conf. Magnetics Japan, p. 301, 1982].
\bibitem{b7} M. Young, The Technical Writer's Handbook. Mill Valley, CA: University Science, 1989.
\bibitem{b8} D. P. Kingma and M. Welling, ``Auto-encoding variational Bayes,'' 2013, arXiv:1312.6114. [Online]. Available: https://arxiv.org/abs/1312.6114
\bibitem{b9} S. Liu, ``Wi-Fi Energy Detection Testbed (12MTC),'' 2023, gitHub repository. [Online]. Available: https://github.com/liustone99/Wi-Fi-Energy-Detection-Testbed-12MTC
\bibitem{b10} ``Treatment episode data set: discharges (TEDS-D): concatenated, 2006 to 2009.'' U.S. Department of Health and Human Services, Substance Abuse and Mental Health Services Administration, Office of Applied Studies, August, 2013, DOI:10.3886/ICPSR30122.v2
\bibitem{b11} K. Eves and J. Valasek, ``Adaptive control for singularly perturbed systems examples,'' Code Ocean, Aug. 2023. [Online]. Available: https://codeocean.com/capsule/4989235/tree
\end{thebibliography}
