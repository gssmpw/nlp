\section*{Limitations}

\subsection*{Dataset} 

(1) \textit{Limited data size.} In this work, we utilize data from human players in the GeoGuessr game to train Vision Language Models for performing geographic reasoning on images. The copyright and usage rights of the images are subject to that of Google Street View. However, the size of \dataname is limited due to the scarcity of available data on YouTube and the data noise. 

(2) \textit{Panoramic street view images.} To simulate the perspective of players in the GeoGuessr game, we use stitched panoramic images as the input to the model. Furthermore, nearly all images in the data from GeoGuessr are street views, despite our efforts to ensure a geographically balanced distribution of data across countries. This limits its distribution, as there's more weather, street, car, and vegetation information in street views than in other images. Models trained with \dataname might be weak at images with less street-level information. 

(3) \textit{Future work} could consider expanding the training dataset by incorporating images of different sizes and types, including more detailed annotations to create dataset s more than street views, to further enhance the performance of image \geoloc tasks with better reasoning.

\subsection*{Models} 

(1) \textit{Limited model sizes.} Due to cost constraints, we are unable to train larger models and conduct our experiments using top-performing, medium-sized open-source models (around 7B parameters). While this choice may result in performance that is not as competitive as larger models, it ensures a practical balance between computational feasibility and model efficacy. We also refrain from using closed-source models, as their lack of transparency regarding training data and inability to be trained on \dataname make them unsuitable for fair comparison. 

(2) \textit{Limited tool sets.} We evaluated only a limited set of tools and grounding words in \micname. Identifying more geographic features such as cars, road markings, and poles would require more precise recognition methods and more sophisticated model designs, which could potentially improve performance. 

(3) \textit{Complexity of subsystems.} We employ a pipeline approach to construct our model, aiming to maximize the performance of each component at every stage. However, this process introduces knowledge from different resources, which might conflict with each other. Currently, we implement a Guesser to handle the potential conflict and show the contribution of each ablated subsystems. We also examine the reasoning from \macname to show the necessity of \micname.

(4) \textit{Future works} can focus on including larger backbone models to further improve the performance, adding more tools, and conduct end-to-end training to better integrate the information, or add another fact-checking module to better discern information. 


\section*{Ethical Considerations}

\subsection*{Data Collection}

In this work, we use the data from GeoGuessr players on YouTube to train our models. We carefully process the data and remove the personal information of the players, using all data for academic and non-commercial purposes, and giving appropriate credit to them in this paper. We make sure the use of our data is acceptable under YouTube's copyright policies and the Fair Use guidelines. 

\subsection*{Model Usage}

While the task of image \geoloc has the potential to enable innovative applications in fields such as navigation and tourism, the misuse of these models could also lead to risks such as privacy breaches and surveillance. In our work, we ensured that all training and testing data came from publicly available sources, with no involvement of private or personal images or location data. Currently, as shown in our experiments, these models have not yet reached a level of precision to accurately predict coordinates-level locations. For the future development of this field, it is crucial for researchers to ensure that these models are used within appropriate boundaries to prevent the leakage of private information. 