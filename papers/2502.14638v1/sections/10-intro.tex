\section{Introduction}

Image \geoloc---the task of predicting the location where an image
was taken~\cite{hays2008im2gps}---remains a challenging multimodal problem.
%
For example, to say Figure~\ref{fig:task} is a picture from
Darlington (in England) requires reading the name of the hotel to
determine possible candidates and excluding---for instance---the Croft hotel in Ontario based on architecture.
%
Directly predicting the exact location of an
image~\cite{weyand2016planet, haas2023learning, vivanco2024geoclip} is
difficult for computer vision models and requires extensive training on large
image-location datasets.  

In contrast, human experts infer locations by reasoning.
%
For example, in a GeoGuessr\footnote{\url{http://www.geoguessr.com}} game video, an expert player, \textit{zi8gzag}, explained how he identified a location in Korea: the presence of single yellow road lines and the language on the road signs suggest an Asian region; large spikes atop concrete poles narrow it down to Japan and Korea, and the black and yellow guardrails rule out Japan.
%
While recent research integrates textual knowledge~\cite{luo2022g} and explicit clues~\cite{zhang2024can, mendes2024granular, ligeoreasoner} with Vision Language Models (\textsc{vlm}s) to enhance accuracy, the reasoning in these models is often limited to a few words related to landmarks and does not provide a concrete analysis, as human experts would.


\begin{figure}[t]
  \includegraphics[width=\linewidth]{images/intro.pdf} \caption{In
  image \geoloc, models need to find both cultural and
  geographical clues to infer correct locations. External tools like
  maps and guidebooks can also be helpful by providing extra
  knowledge.}  \label{fig:task}
\end{figure}


To date, these models' reasoning remains more superficial than humans' for two reasons: (1)~\textbf{Lack of high-quality reasoning datasets:} Existing geo-tagged datasets lack linguistic reasoning elements, while constructing a dataset that involves reasoning based on image details is resource-intensive. (2)~\textbf{Complexity of diverse information retrieval:} Images often contain rich details, such as road signs, texts, and building styles, requiring additional tools for accurate retrieval and interpretation.

To address these questions, we introduce \dataname, a detailed and high-quality reasoning dataset for image \geoloc, and \modelname, a framework that combines both visual analysis and external knowledge to perform analytical reasoning. Inspired by the popular game GeoGuessr, \dataname has over $2000$ instances from five experienced YouTubers, recording their process of analyzing image details to infer locations, which trains \textsc{vlm}s to generate reasoning that mimics professional human players. With tools like public maps and expert-written guidebooks, we design a pipeline that dives into fine-grained details and retrieves relevant information to further enhance accuracy. We evaluate \modelname against state-of-the-art models on two open benchmarks using five levels of prediction and ablate each component to investigate their contributions. \modelname outperforms previous state-of-the-art models by a 14\% reduction in average distance error while using less than $1000$ training samples. We further illustrate the reasoning of \modelname by providing examples of both successful and challenging cases. We release our dataset and framework to advance the use of reasoning in the field of image \geoloc.