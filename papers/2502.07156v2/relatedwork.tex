\section{Related Work}
In medical imaging, where trust and accountability are paramount, counterfactuals have been applied to 2D radiology modalities such as chest X-ray \cite{Singla2021BlackBoxSmoothly, Seah2019GVR_CFs}. These employed counterfactuals to explain classifications, pinpointing regions critical for diagnosing conditions like pneumonia and plural effusion.

Despite these advances, applying counterfactual explanations to 3D imaging modalities like CT scans remains largely unexplored, inspiring our work to address this gap.
Existing work on 3D MRIs employed 2D slice based CF generation using causal modeling \cite{Ribeiro2023SCM_CF} and conditional generation \cite{Kumar2022MRICF}. 
Work by \citet{Peng2024BrainMRI3D} focused on 3D MRI CF generation, also using a VQ-GAN but combined with causal modeling to generate the CF volumes. 

These works rely on conditionally generating CF images rather than directly explaining an arbitrary 3D classification model. The generated images may not align with the classifiers we aim to explain, which is why we chose the Latent Shift approach, as it modifies image features based directly on the classifier. This also allows arbitrary classifiers to be explained using the same latent variable model used in our research.