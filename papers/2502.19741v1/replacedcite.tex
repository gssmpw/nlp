\section{Related Works}
\label{related works}
\textbf{Classic Causal Inference} has been studied in two languages: the graphical models ____ and the potential outcome framework ____. The most related method is the propensity score method in the potential outcome framework, e.g., IPW method ____, which is widely applied to many scenarios ____. There are also many outcome regression models, including meta-learners ____, neural networks-based works ____. By incorporating them, one can construct a doubly robust estimator ____, i.e., the effect estimator is consistent as either the propensity model or the outcome repression model is consistent. 

\textbf{Causal Inference without SUTVA} has drawn increasing attention recently. ____ extend the traditional propensity score to account for neighbors’ treatments and features and propose a generalized Inverse Probability Weighting (IPW) estimator. ____ define the joint propensity score and then propose a subclassification-based method. Drawing upon previous works, ____ consider two IPW estimators and derive a closed-form estimator for the asymptotic variance. 
Based on the representation learning, ____ add neighborhood exposure and neighbors' features as additional input variables and applies HSIC to learn balanced representations. ____ use adversarial learning to learn balanced representations for better effect estimation. ____ propose a framework to learn causal effects on a hypergraph. ____ propose a reweighted representation learning method to learn balanced representations.
Under networked interference, ____ propose an estimator to achieve DR property. However, these works do assume the unconfoundedness assumption, which might not hold in real-world scenarios. Different from them, we explore the problem of networked effect estimation without the unconfoundedness assumption.

\textbf{Causal Inference without Uncounfoundedness Assumption} is an important problem since the unconfoundedness assumption is usually violated in observational studies.
Classic methods to solve this problem usually assume there exist additional variables, e.g., instrumental variable ____, proximal variable ____.
Another effective way to address this problem is to recover the latent confounder using representation learning methods.
CEVAE ____ assumes that latent confounders can be recovered by their proxies and applies VAE to learn confounders.
As a follow-up work, TEDVAE ____ 
decouples the learned latent confounders into several factors to achieve a more accurate estimation of treatment effects.
In the mediation analysis, DMAVAE ____ proposes to recover latent confounders using the VAE similar to CEVAE.
Our work is closely related to these works.
Different from them, we focus on the causal effect 
without the unconfoundedness assumption in the presence of networked interference. We also provide theoretical guarantees for the latent confounder identifiability, which ensures the effectiveness of our estimator.