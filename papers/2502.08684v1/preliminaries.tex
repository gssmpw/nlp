\section{Preliminaries}
\label{sec:preliminaries}

\textbf{Job Shop Scheduling Problem.} The JSSP is a classical combinatorial optimization problem defined as follows. Let $J = \{J_1, J_2, \dots, J_n\}$ represent a set of $n$ jobs, and let $M = \{M_1, M_2, \dots, M_m\}$ denote a set of $m$ machines. Each job $J_i$ consists of a sequence of $k_i$ operations, $\{O_{i1}, O_{i2}, \dots, O_{ik_i}\}$, where each operation $O_{ij}$ requires exclusive use of a specific machine $M_k \in M$ for a fixed processing time $p_{ij} > 0$. Operations within a job must follow a predefined order, meaning that $O_{ij}$ must complete before $O_{i(j+1)}$ begins for all $j = 1, \dots, k_i - 1$. Additionally, each machine $M_k$ can process at most one operation at a time, and once an operation starts on its assigned machine, it must run without interruption until completion.

The objective is to determine a feasible schedule $\mathcal{S}$ that minimizes the \emph{makespan}, $C_{\max}$, defined as the total time required to complete all jobs:
\begin{align}
C_{\max} = \max_{i=1, \dots, n} C_i,
\end{align}
where $C_i$ is the completion time of job $J_i$.

\textbf{Self-Evaluation in Large Language Models.} Breaking down complex tasks into intermediate steps, often referred to as reasoning chains, has proven highly effective for enhancing the performance of LLMs on multi-step problems \cite{brown2020language, wei2022chain}. By structuring tasks into logical sub-components, these models can process and solve problems more systematically. However, as reasoning chains grow longer, errors can accumulate across intermediate steps, significantly reducing the accuracy of the final outcomes \cite{chen2024self}.

Self-evaluation provides a mechanism to address these challenges by enabling models to critique and refine their intermediate reasoning. This approach involves evaluating the correctness and coherence of each step in the reasoning chain, allowing for dynamic adjustments and reducing the risk of error propagation \cite{madaan2024self}. By integrating explicit feedback into the reasoning process, self-evaluation ensures more consistent and robust outputs, particularly in tasks requiring logical precision and multi-step decision-making.
