\section{Example of Action Space Transition and State Transitions}
\label{app:example}

This section provides an illustrative example of the transitions in the action space and state representations for a JSSP instance. 

% Figures \ref{fig:sol} and \ref{fig:optsol} compare a non-optimal and an optimal solution to the same instance. The key difference between the solutions lies in how the operations of each job are parallelized across the available machines, demonstrating the impact of scheduling decisions on the makespan.

\begin{table}[h]
\caption{JSSP instance with 4 jobs and 4 machines where the processing times of each operation are indicated.}
\label{tab:jssp_example}% Correct caption for minipage
\begin{tabularx}{\columnwidth}{X X X X X X X}
\toprule
Jobs  & Operations & \(m_1\) & \(m_2\) & \(m_3\) & \(m_4\) \\ \midrule
\(j_1\) & \(o_{11}\) & - & - & - & 5 \\
       & \(o_{12}\) & - & 6 & - & - \\
       & \(o_{13}\) & 3 & - & - & - \\
       & \(o_{14}\) & - & - & 2 & - \\
\hline
\(j_2\) & \(o_{21}\) & - & - & - & 8 \\ %d
       & \(o_{22}\) & 3 & - & - & - \\ %d
\hline
\(j_3\) & \(o_{31}\) & - & - & 3 & - \\
       & \(o_{32}\) & 4 & - & - & - \\
       & \(o_{33}\) & - & - & - & 5 \\
\hline
\(j_4\) & \(o_{41}\) & - & 6 & - & - \\ %d
       & \(o_{42}\) & - & - & - & 4 \\ %d
       & \(o_{43}\) & - & - & 5 & - \\ \bottomrule
\end{tabularx}
\end{table}

The set of possible assignments in the initial step is given by:
\[
\mathcal{JM} = \{(j_1, m_4), (j_2, m_4), (j_3, m_3), (j_4, m_2)\}.
\]
The corresponding power set, ensuring that no machine is assigned more than one job simultaneously, defines the action space:
\begin{multline}
\mathcal{A} = \{ \{(j_1, m_4)\}, \{(j_2, m_4)\}, \dots, \{(j_1, m_4), (j_3, m_3)\}, \{(j_1, m_4), (j_4, m_2)\}, \dots, \\
\{(j_1, m_4), (j_3, m_3), (j_4, m_2)\} \}.
\end{multline}

An element of $\mathcal{A}$ is selected in each step, and the actions within this subset are executed. The corresponding operations are then removed from the graph, and the features of the remaining nodes are updated. This process repeats until all operations have been assigned.

As suggested in \cite{echeverria2024multi}, the action space of possible assignments is constrained to the maximum number of assignments (i.e., the number of machines), ordered by their minimum start times, to provide a sufficiently broad set of actions. This reduces the action space size, particularly for larger instances, enabling the policy to focus on a more manageable subset of options, which is especially advantageous in complex scenarios.


% \begin{figure*}[h]
%   \centering
%   \begin{minipage}[b]{0.49\textwidth}
%     \includegraphics[width=\textwidth]{appendix/no_opt.png}
%     \caption{A solution to the JSSP instance with a makespan of 35.}
%     \label{fig:sol}
%   \end{minipage}
%   \hfill
%   \begin{minipage}[b]{0.49\textwidth}
%     \includegraphics[width=\textwidth]{appendix/opt.png}
%     \caption{An optimal solution with a makespan of 22.}
%     \label{fig:optsol}
%   \end{minipage}
% \end{figure*}
