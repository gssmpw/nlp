% CVPR 2025 Paper Template; see https://github.com/cvpr-org/author-kit

\documentclass[10pt,twocolumn,letterpaper]{article}

%%%%%%%%% PAPER TYPE  - PLEASE UPDATE FOR FINAL VERSION
% \usepackage{cvpr}              % To produce the CAMERA-READY version
% \usepackage[review]{cvpr}      % To produce the REVIEW version
\usepackage[pagenumbers]{cvpr} % To force page numbers, e.g. for an arXiv version
\usepackage{pifont}
\usepackage{multirow}
\newcommand{\workname}{Movie Weaver }

% Import additional packages in the preamble file, before hyperref
%
% --- inline annotations
%
\newcommand{\red}[1]{{\color{red}#1}}
\newcommand{\todo}[1]{{\color{red}#1}}
\newcommand{\TODO}[1]{\textbf{\color{red}[TODO: #1]}}
% --- disable by uncommenting  
% \renewcommand{\TODO}[1]{}
% \renewcommand{\todo}[1]{#1}



\newcommand{\VLM}{LVLM\xspace} 
\newcommand{\ours}{PeKit\xspace}
\newcommand{\yollava}{Yo’LLaVA\xspace}

\newcommand{\thisismy}{This-Is-My-Img\xspace}
\newcommand{\myparagraph}[1]{\noindent\textbf{#1}}
\newcommand{\vdoro}[1]{{\color[rgb]{0.4, 0.18, 0.78} {[V] #1}}}
% --- disable by uncommenting  
% \renewcommand{\TODO}[1]{}
% \renewcommand{\todo}[1]{#1}
\usepackage{slashbox}
% Vectors
\newcommand{\bB}{\mathcal{B}}
\newcommand{\bw}{\mathbf{w}}
\newcommand{\bs}{\mathbf{s}}
\newcommand{\bo}{\mathbf{o}}
\newcommand{\bn}{\mathbf{n}}
\newcommand{\bc}{\mathbf{c}}
\newcommand{\bp}{\mathbf{p}}
\newcommand{\bS}{\mathbf{S}}
\newcommand{\bk}{\mathbf{k}}
\newcommand{\bmu}{\boldsymbol{\mu}}
\newcommand{\bx}{\mathbf{x}}
\newcommand{\bg}{\mathbf{g}}
\newcommand{\be}{\mathbf{e}}
\newcommand{\bX}{\mathbf{X}}
\newcommand{\by}{\mathbf{y}}
\newcommand{\bv}{\mathbf{v}}
\newcommand{\bz}{\mathbf{z}}
\newcommand{\bq}{\mathbf{q}}
\newcommand{\bff}{\mathbf{f}}
\newcommand{\bu}{\mathbf{u}}
\newcommand{\bh}{\mathbf{h}}
\newcommand{\bb}{\mathbf{b}}

\newcommand{\rone}{\textcolor{green}{R1}}
\newcommand{\rtwo}{\textcolor{orange}{R2}}
\newcommand{\rthree}{\textcolor{red}{R3}}
\usepackage{amsmath}
%\usepackage{arydshln}
\DeclareMathOperator{\similarity}{sim}
\DeclareMathOperator{\AvgPool}{AvgPool}

\newcommand{\argmax}{\mathop{\mathrm{argmax}}}     



\makeatletter
\renewcommand{\thefootnote}{\fnsymbol{footnote}}
\makeatother
% It is strongly recommended to use hyperref, especially for the review version.
% hyperref with option pagebackref eases the reviewers' job.
% Please disable hyperref *only* if you encounter grave issues, 
% e.g. with the file validation for the camera-ready version.
%
% If you comment hyperref and then uncomment it, you should delete *.aux before re-running LaTeX.
% (Or just hit 'q' on the first LaTeX run, let it finish, and you should be clear).
\definecolor{cvprblue}{rgb}{0.21,0.49,0.74}
\usepackage[pagebackref,breaklinks,colorlinks,allcolors=cvprblue]{hyperref}

%%%%%%%%% PAPER ID  - PLEASE UPDATE
\def\paperID{5034} % *** Enter the Paper ID here
\def\confName{CVPR}
\def\confYear{2025}

\begin{document}


\twocolumn[{

%%%%%%%%% TITLE - PLEASE UPDATE
\title{Movie Weaver: Tuning-Free Multi-Concept Video Personalization with Anchored Prompts
\vspace{-1em}
}

%%%%%%%%% AUTHORS - PLEASE UPDATE

\author{Feng Liang\footnotemark[1]~~\textsuperscript{\rm 1},
        Haoyu Ma\textsuperscript{\rm 2}, 
        Zecheng He\textsuperscript{\rm 2},
        Tingbo Hou\textsuperscript{\rm 2},
        Ji Hou\textsuperscript{\rm 2},
        Kunpeng Li\textsuperscript{\rm 2}, 
        Xiaoliang Dai\textsuperscript{\rm 2}, \\
        Felix Juefei-Xu\textsuperscript{\rm 2}, 
        Samaneh Azadi\textsuperscript{\rm 2},
        Animesh Sinha\textsuperscript{\rm 2},
        Peizhao Zhang\textsuperscript{\rm 2}, 
        Peter Vajda\textsuperscript{\rm 2}, 
        Diana Marculescu\textsuperscript{\rm 1}\\
\normalsize{\textsuperscript{\rm 1}The University of Texas at Austin, Chandra Family Department of Electrical and Computer Engineering \textsuperscript{\rm 2}Meta GenAI}\\
\texttt{\{jeffliang,dianam\}@utexas.edu}, \texttt{\{haoyuma,zechengh,stzpz\}@meta.com}\\
\texttt{\href{https://jeff-liangf.github.io/projects/movieweaver}{https://jeff-liangf.github.io/projects/movieweaver}}
}

\maketitle


\vspace{-3em}

\begin{center}
    \centering
    \includegraphics[width=0.8\textwidth]{figs/teaser.pdf}
    \vspace{-0.5em}
    \captionof{figure}{
    We introduce \textbf{Movie Weaver}, a video diffusion model for personalized multi-concept video creation. 
    Besides text prompts, our model allows users to input different combinations of reference images, \eg, face, body, and animal images, to customize videos in a tuning-free manner. The left column displays different types of reference images, while the right column shows the generated videos, with \emph{anchored prompt} listed beneath each video. We encourage readers to check our video results in the supplementary materials.
    }
    \label{fig:teaser}
\end{center}
}]

\footnotetext[1]{Work partially done during an internship at Meta GenAI.}
\begin{abstract}


The choice of representation for geographic location significantly impacts the accuracy of models for a broad range of geospatial tasks, including fine-grained species classification, population density estimation, and biome classification. Recent works like SatCLIP and GeoCLIP learn such representations by contrastively aligning geolocation with co-located images. While these methods work exceptionally well, in this paper, we posit that the current training strategies fail to fully capture the important visual features. We provide an information theoretic perspective on why the resulting embeddings from these methods discard crucial visual information that is important for many downstream tasks. To solve this problem, we propose a novel retrieval-augmented strategy called RANGE. We build our method on the intuition that the visual features of a location can be estimated by combining the visual features from multiple similar-looking locations. We evaluate our method across a wide variety of tasks. Our results show that RANGE outperforms the existing state-of-the-art models with significant margins in most tasks. We show gains of up to 13.1\% on classification tasks and 0.145 $R^2$ on regression tasks. All our code and models will be made available at: \href{https://github.com/mvrl/RANGE}{https://github.com/mvrl/RANGE}.

\end{abstract}

    
\section{Introduction}
Backdoor attacks pose a concealed yet profound security risk to machine learning (ML) models, for which the adversaries can inject a stealth backdoor into the model during training, enabling them to illicitly control the model's output upon encountering predefined inputs. These attacks can even occur without the knowledge of developers or end-users, thereby undermining the trust in ML systems. As ML becomes more deeply embedded in critical sectors like finance, healthcare, and autonomous driving \citep{he2016deep, liu2020computing, tournier2019mrtrix3, adjabi2020past}, the potential damage from backdoor attacks grows, underscoring the emergency for developing robust defense mechanisms against backdoor attacks.

To address the threat of backdoor attacks, researchers have developed a variety of strategies \cite{liu2018fine,wu2021adversarial,wang2019neural,zeng2022adversarial,zhu2023neural,Zhu_2023_ICCV, wei2024shared,wei2024d3}, aimed at purifying backdoors within victim models. These methods are designed to integrate with current deployment workflows seamlessly and have demonstrated significant success in mitigating the effects of backdoor triggers \cite{wubackdoorbench, wu2023defenses, wu2024backdoorbench,dunnett2024countering}.  However, most state-of-the-art (SOTA) backdoor purification methods operate under the assumption that a small clean dataset, often referred to as \textbf{auxiliary dataset}, is available for purification. Such an assumption poses practical challenges, especially in scenarios where data is scarce. To tackle this challenge, efforts have been made to reduce the size of the required auxiliary dataset~\cite{chai2022oneshot,li2023reconstructive, Zhu_2023_ICCV} and even explore dataset-free purification techniques~\cite{zheng2022data,hong2023revisiting,lin2024fusing}. Although these approaches offer some improvements, recent evaluations \cite{dunnett2024countering, wu2024backdoorbench} continue to highlight the importance of sufficient auxiliary data for achieving robust defenses against backdoor attacks.

While significant progress has been made in reducing the size of auxiliary datasets, an equally critical yet underexplored question remains: \emph{how does the nature of the auxiliary dataset affect purification effectiveness?} In  real-world  applications, auxiliary datasets can vary widely, encompassing in-distribution data, synthetic data, or external data from different sources. Understanding how each type of auxiliary dataset influences the purification effectiveness is vital for selecting or constructing the most suitable auxiliary dataset and the corresponding technique. For instance, when multiple datasets are available, understanding how different datasets contribute to purification can guide defenders in selecting or crafting the most appropriate dataset. Conversely, when only limited auxiliary data is accessible, knowing which purification technique works best under those constraints is critical. Therefore, there is an urgent need for a thorough investigation into the impact of auxiliary datasets on purification effectiveness to guide defenders in  enhancing the security of ML systems. 

In this paper, we systematically investigate the critical role of auxiliary datasets in backdoor purification, aiming to bridge the gap between idealized and practical purification scenarios.  Specifically, we first construct a diverse set of auxiliary datasets to emulate real-world conditions, as summarized in Table~\ref{overall}. These datasets include in-distribution data, synthetic data, and external data from other sources. Through an evaluation of SOTA backdoor purification methods across these datasets, we uncover several critical insights: \textbf{1)} In-distribution datasets, particularly those carefully filtered from the original training data of the victim model, effectively preserve the model’s utility for its intended tasks but may fall short in eliminating backdoors. \textbf{2)} Incorporating OOD datasets can help the model forget backdoors but also bring the risk of forgetting critical learned knowledge, significantly degrading its overall performance. Building on these findings, we propose Guided Input Calibration (GIC), a novel technique that enhances backdoor purification by adaptively transforming auxiliary data to better align with the victim model’s learned representations. By leveraging the victim model itself to guide this transformation, GIC optimizes the purification process, striking a balance between preserving model utility and mitigating backdoor threats. Extensive experiments demonstrate that GIC significantly improves the effectiveness of backdoor purification across diverse auxiliary datasets, providing a practical and robust defense solution.

Our main contributions are threefold:
\textbf{1) Impact analysis of auxiliary datasets:} We take the \textbf{first step}  in systematically investigating how different types of auxiliary datasets influence backdoor purification effectiveness. Our findings provide novel insights and serve as a foundation for future research on optimizing dataset selection and construction for enhanced backdoor defense.
%
\textbf{2) Compilation and evaluation of diverse auxiliary datasets:}  We have compiled and rigorously evaluated a diverse set of auxiliary datasets using SOTA purification methods, making our datasets and code publicly available to facilitate and support future research on practical backdoor defense strategies.
%
\textbf{3) Introduction of GIC:} We introduce GIC, the \textbf{first} dedicated solution designed to align auxiliary datasets with the model’s learned representations, significantly enhancing backdoor mitigation across various dataset types. Our approach sets a new benchmark for practical and effective backdoor defense.



\section{Related Work}

\subsection{Large 3D Reconstruction Models}
Recently, generalized feed-forward models for 3D reconstruction from sparse input views have garnered considerable attention due to their applicability in heavily under-constrained scenarios. The Large Reconstruction Model (LRM)~\cite{hong2023lrm} uses a transformer-based encoder-decoder pipeline to infer a NeRF reconstruction from just a single image. Newer iterations have shifted the focus towards generating 3D Gaussian representations from four input images~\cite{tang2025lgm, xu2024grm, zhang2025gslrm, charatan2024pixelsplat, chen2025mvsplat, liu2025mvsgaussian}, showing remarkable novel view synthesis results. The paradigm of transformer-based sparse 3D reconstruction has also successfully been applied to lifting monocular videos to 4D~\cite{ren2024l4gm}. \\
Yet, none of the existing works in the domain have studied the use-case of inferring \textit{animatable} 3D representations from sparse input images, which is the focus of our work. To this end, we build on top of the Large Gaussian Reconstruction Model (GRM)~\cite{xu2024grm}.

\subsection{3D-aware Portrait Animation}
A different line of work focuses on animating portraits in a 3D-aware manner.
MegaPortraits~\cite{drobyshev2022megaportraits} builds a 3D Volume given a source and driving image, and renders the animated source actor via orthographic projection with subsequent 2D neural rendering.
3D morphable models (3DMMs)~\cite{blanz19993dmm} are extensively used to obtain more interpretable control over the portrait animation. For example, StyleRig~\cite{tewari2020stylerig} demonstrates how a 3DMM can be used to control the data generated from a pre-trained StyleGAN~\cite{karras2019stylegan} network. ROME~\cite{khakhulin2022rome} predicts vertex offsets and texture of a FLAME~\cite{li2017flame} mesh from the input image.
A TriPlane representation is inferred and animated via FLAME~\cite{li2017flame} in multiple methods like Portrait4D~\cite{deng2024portrait4d}, Portrait4D-v2~\cite{deng2024portrait4dv2}, and GPAvatar~\cite{chu2024gpavatar}.
Others, such as VOODOO 3D~\cite{tran2024voodoo3d} and VOODOO XP~\cite{tran2024voodooxp}, learn their own expression encoder to drive the source person in a more detailed manner. \\
All of the aforementioned methods require nothing more than a single image of a person to animate it. This allows them to train on large monocular video datasets to infer a very generic motion prior that even translates to paintings or cartoon characters. However, due to their task formulation, these methods mostly focus on image synthesis from a frontal camera, often trading 3D consistency for better image quality by using 2D screen-space neural renderers. In contrast, our work aims to produce a truthful and complete 3D avatar representation from the input images that can be viewed from any angle.  

\subsection{Photo-realistic 3D Face Models}
The increasing availability of large-scale multi-view face datasets~\cite{kirschstein2023nersemble, ava256, pan2024renderme360, yang2020facescape} has enabled building photo-realistic 3D face models that learn a detailed prior over both geometry and appearance of human faces. HeadNeRF~\cite{hong2022headnerf} conditions a Neural Radiance Field (NeRF)~\cite{mildenhall2021nerf} on identity, expression, albedo, and illumination codes. VRMM~\cite{yang2024vrmm} builds a high-quality and relightable 3D face model using volumetric primitives~\cite{lombardi2021mvp}. One2Avatar~\cite{yu2024one2avatar} extends a 3DMM by anchoring a radiance field to its surface. More recently, GPHM~\cite{xu2025gphm} and HeadGAP~\cite{zheng2024headgap} have adopted 3D Gaussians to build a photo-realistic 3D face model. \\
Photo-realistic 3D face models learn a powerful prior over human facial appearance and geometry, which can be fitted to a single or multiple images of a person, effectively inferring a 3D head avatar. However, the fitting procedure itself is non-trivial and often requires expensive test-time optimization, impeding casual use-cases on consumer-grade devices. While this limitation may be circumvented by learning a generalized encoder that maps images into the 3D face model's latent space, another fundamental limitation remains. Even with more multi-view face datasets being published, the number of available training subjects rarely exceeds the thousands, making it hard to truly learn the full distibution of human facial appearance. Instead, our approach avoids generalizing over the identity axis by conditioning on some images of a person, and only generalizes over the expression axis for which plenty of data is available. 

A similar motivation has inspired recent work on codec avatars where a generalized network infers an animatable 3D representation given a registered mesh of a person~\cite{cao2022authentic, li2024uravatar}.
The resulting avatars exhibit excellent quality at the cost of several minutes of video capture per subject and expensive test-time optimization.
For example, URAvatar~\cite{li2024uravatar} finetunes their network on the given video recording for 3 hours on 8 A100 GPUs, making inference on consumer-grade devices impossible. In contrast, our approach directly regresses the final 3D head avatar from just four input images without the need for expensive test-time fine-tuning.


\section{Study Design}
% robot: aliengo 
% We used the Unitree AlienGo quadruped robot. 
% See Appendix 1 in AlienGo Software Guide PDF
% Weight = 25kg, size (L,W,H) = (0.55, 0.35, 06) m when standing, (0.55, 0.35, 0.31) m when walking
% Handle is 0.4 m or 0.5 m. I'll need to check it to see which type it is.
We gathered input from primary stakeholders of the robot dog guide, divided into three subgroups: BVI individuals who have owned a dog guide, BVI individuals who were not dog guide owners, and sighted individuals with generally low degrees of familiarity with dog guides. While the main focus of this study was on the BVI participants, we elected to include survey responses from sighted participants given the importance of social acceptance of the robot by the general public, which could reflect upon the BVI users themselves and affect their interactions with the general population \cite{kayukawa2022perceive}. 

The need-finding processes consisted of two stages. During Stage 1, we conducted in-depth interviews with BVI participants, querying their experiences in using conventional assistive technologies and dog guides. During Stage 2, a large-scale survey was distributed to both BVI and sighted participants. 

This study was approved by the University’s Institutional Review Board (IRB), and all processes were conducted after obtaining the participants' consent.

\subsection{Stage 1: Interviews}
We recruited nine BVI participants (\textbf{Table}~\ref{tab:bvi-info}) for in-depth interviews, which lasted 45-90 minutes for current or former dog guide owners (DO) and 30-60 minutes for participants without dog guides (NDO). Group DO consisted of five participants, while Group NDO consisted of four participants.
% The interview participants were divided into two groups. Group DO (Dog guide Owner) consisted of five participants who were current or former dog guide owners and Group NDO (Non Dog guide Owner) consisted of three participants who were not dog guide owners. 
All participants were familiar with using white canes as a mobility aid. 

We recruited participants in both groups, DO and NDO, to gather data from those with substantial experience with dog guides, offering potentially more practical insights, and from those without prior experience, providing a perspective that may be less constrained and more open to novel approaches. 

We asked about the participants' overall impressions of a robot dog guide, expectations regarding its potential benefits and challenges compared to a conventional dog guide, their desired methods of giving commands and communicating with the robot dog guide, essential functionalities that the robot dog guide should offer, and their preferences for various aspects of the robot dog guide's form factors. 
For Group DO, we also included questions that asked about the participants' experiences with conventional dog guides. 

% We obtained permission to record the conversations for our records while simultaneously taking notes during the interviews. The interviews lasted 30-60 minutes for NDO participants and 45-90 minutes for DO participants. 

\subsection{Stage 2: Large-Scale Surveys} 
After gathering sufficient initial results from the interviews, we created an online survey for distributing to a larger pool of participants. The survey platform used was Qualtrics. 

\subsubsection{Survey Participants}
The survey had 100 participants divided into two primary groups. Group BVI consisted of 42 blind or visually impaired participants, and Group ST consisted of 58 sighted participants. \textbf{Table}~\ref{tab:survey-demographics} shows the demographic information of the survey participants. 

\subsubsection{Question Differentiation} 
Based on their responses to initial qualifying questions, survey participants were sorted into three subgroups: DO, NDO, and ST. Each participant was assigned one of three different versions of the survey. The surveys for BVI participants mirrored the interview categories (overall impressions, communication methods, functionalities, and form factors), but with a more quantitative approach rather than the open-ended questions used in interviews. The DO version included additional questions pertaining to their prior experience with dog guides. The ST version revolved around the participants' prior interactions with and feelings toward dog guides and dogs in general, their thoughts on a robot dog guide, and broad opinions on the aesthetic component of the robot's design. 


\section{Dataset}
\label{sec:dataset}

\subsection{Data Collection}

To analyze political discussions on Discord, we followed the methodology in \cite{singh2024Cross-Platform}, collecting messages from politically-oriented public servers in compliance with Discord's platform policies.

Using Discord's Discovery feature, we employed a web scraper to extract server invitation links, names, and descriptions, focusing on public servers accessible without participation. Invitation links were used to access data via the Discord API. To ensure relevance, we filtered servers using keywords related to the 2024 U.S. elections (e.g., Trump, Kamala, MAGA), as outlined in \cite{balasubramanian2024publicdatasettrackingsocial}. This resulted in 302 server links, further narrowed to 81 English-speaking, politics-focused servers based on their names and descriptions.

Public messages were retrieved from these servers using the Discord API, collecting metadata such as \textit{content}, \textit{user ID}, \textit{username}, \textit{timestamp}, \textit{bot flag}, \textit{mentions}, and \textit{interactions}. Through this process, we gathered \textbf{33,373,229 messages} from \textbf{82,109 users} across \textbf{81 servers}, including \textbf{1,912,750 messages} from \textbf{633 bots}. Data collection occurred between November 13th and 15th, covering messages sent from January 1st to November 12th, just after the 2024 U.S. election.

\subsection{Characterizing the Political Spectrum}
\label{sec:timeline}

A key aspect of our research is distinguishing between Republican- and Democratic-aligned Discord servers. To categorize their political alignment, we relied on server names and self-descriptions, which often include rules, community guidelines, and references to key ideologies or figures. Each server's name and description were manually reviewed based on predefined, objective criteria, focusing on explicit political themes or mentions of prominent figures. This process allowed us to classify servers into three categories, ensuring a systematic and unbiased alignment determination.

\begin{itemize}
    \item \textbf{Republican-aligned}: Servers referencing Republican and right-wing and ideologies, movements, or figures (e.g., MAGA, Conservative, Traditional, Trump).  
    \item \textbf{Democratic-aligned}: Servers mentioning Democratic and left-wing ideologies, movements, or figures (e.g., Progressive, Liberal, Socialist, Biden, Kamala).  
    \item \textbf{Unaligned}: Servers with no defined spectrum and ideologies or opened to general political debate from all orientations.
\end{itemize}

To ensure the reliability and consistency of our classification, three independent reviewers assessed the classification following the specified set of criteria. The inter-rater agreement of their classifications was evaluated using Fleiss' Kappa \cite{fleiss1971measuring}, with a resulting Kappa value of \( 0.8191 \), indicating an almost perfect agreement among the reviewers. Disagreements were resolved by adopting the majority classification, as there were no instances where a server received different classifications from all three reviewers. This process guaranteed the consistency and accuracy of the final categorization.

Through this process, we identified \textbf{7 Republican-aligned servers}, \textbf{9 Democratic-aligned servers}, and \textbf{65 unaligned servers}.

Table \ref{tab:statistics} shows the statistics of the collected data. Notably, while Democratic- and Republican-aligned servers had a comparable number of user messages, users in the latter servers were significantly more active, posting more than double the number of messages per user compared to their Democratic counterparts. 
This suggests that, in our sample, Democratic-aligned servers attract more users, but these users were less engaged in text-based discussions. Additionally, around 10\% of the messages across all server categories were posted by bots. 

\subsection{Temporal Data} 

Throughout this paper, we refer to the election candidates using the names adopted by their respective campaigns: \textit{Kamala}, \textit{Biden}, and \textit{Trump}. To examine how the content of text messages evolves based on the political alignment of servers, we divided the 2024 election year into three periods: \textbf{Biden vs Trump} (January 1 to July 21), \textbf{Kamala vs Trump} (July 21 to September 20), and the \textbf{Voting Period} (after September 20). These periods reflect key phases of the election: the early campaign dominated by Biden and Trump, the shift in dynamics with Kamala Harris replacing Joe Biden as the Democratic candidate, and the final voting stage focused on electoral outcomes and their implications. This segmentation enables an analysis of how discourse responds to pivotal electoral moments.

Figure \ref{fig:line-plot} illustrates the distribution of messages over time, highlighting trends in total messages volume and mentions of each candidate. Prior to Biden's withdrawal on July 21, mentions of Biden and Trump were relatively balanced. However, following Kamala's entry into the race, mentions of Trump surged significantly, a trend further amplified by an assassination attempt on him, solidifying his dominance in the discourse. The only instance where Trump’s mentions were exceeded occurred during the first debate, as concerns about Biden’s age and cognitive abilities temporarily shifted the focus. In the final stages of the election, mentions of all three candidates rose, with Trump’s mentions peaking as he emerged as the victor.
\section{Experimental Methodology}\label{sec:exp}
In this section, we introduce the datasets, evaluation metrics, baselines, and implementation details used in our experiments. More experimental details are shown in Appendix~\ref{app:experiment_detail}.

\textbf{Dataset.}
We utilize various datasets for training and evaluation. Data statistics are shown in Table~\ref{tab:dataset}.

\textit{Training.}
We use the publicly available E5 dataset~\cite{wang2024improving,springer2024repetition} to train both the LLM-QE and dense retrievers. We concentrate on English-based question answering tasks and collect a total of 808,740 queries. From this set, we randomly sample 100,000 queries to construct the DPO training data, while the remaining queries are used for contrastive training. During the DPO preference pair construction, we first prompt LLMs to generate expansion documents, filtering out queries where the expanded documents share low similarity with the query. This results in a final set of 30,000 queries.

\textit{Evaluation.}
We evaluate retrieval effectiveness using two retrieval benchmarks: MS MARCO \cite{bajaj2016ms} and BEIR \cite{thakur2021beir}, in both unsupervised and supervised settings.

\textbf{Evaluation Metrics.}
We use nDCG@10 as the evaluation metric. Statistical significance is tested using a permutation test with $p<0.05$.

\textbf{Baselines.} We compare our LLM-QE model with three unsupervised retrieval models and five query expansion baseline models.
% —

Three unsupervised retrieval models—BM25~\cite{robertson2009probabilistic}, CoCondenser~\cite{gao2022unsupervised}, and Contriever~\cite{izacard2021unsupervised}—are evaluated in the experiments. Among these, Contriever serves as our primary baseline retrieval model, as it is used as the backbone model to assess the query expansion performance of LLM-QE. Additionally, we compare LLM-QE with Contriever in a supervised setting using the same training dataset.

For query expansion, we benchmark against five methods: Pseudo-Relevance Feedback (PRF), Q2Q, Q2E, Q2C, and Q2D. PRF is specifically implemented following the approach in~\citet{yu2021improving}, which enhances query understanding by extracting keywords from query-related documents. The Q2Q, Q2E, Q2C, and Q2D methods~\cite{jagerman2023query,li2024can} expand the original query by prompting LLMs to generate query-related queries, keywords, chains-of-thought~\cite{wei2022chain}, and documents.


\textbf{Implementation Details.} 
For our query expansion model, we deploy the Meta-LLaMA-3-8B-Instruct~\cite{llama3modelcard} as the backbone for the query expansion generator. The batch size is set to 16, and the learning rate is set to $2e-5$. Optimization is performed using the AdamW optimizer. We employ LoRA~\cite{hu2022lora} to efficiently fine-tune the model for 2 epochs. The temperature for the construction of the DPO data varies across $\tau \in \{0.8, 0.9, 1.0, 1.1\}$, with each setting sampled eight times. For the dense retriever, we utilize Contriever~\cite{izacard2021unsupervised} as the backbone. During training, we set the batch size to 1,024 and the learning rate to $3e-5$, with the model trained for 3 epochs.

\section{Conclusion}
We introduce a novel approach, \algo, to reduce human feedback requirements in preference-based reinforcement learning by leveraging vision-language models. While VLMs encode rich world knowledge, their direct application as reward models is hindered by alignment issues and noisy predictions. To address this, we develop a synergistic framework where limited human feedback is used to adapt VLMs, improving their reliability in preference labeling. Further, we incorporate a selective sampling strategy to mitigate noise and prioritize informative human annotations.

Our experiments demonstrate that this method significantly improves feedback efficiency, achieving comparable or superior task performance with up to 50\% fewer human annotations. Moreover, we show that an adapted VLM can generalize across similar tasks, further reducing the need for new human feedback by 75\%. These results highlight the potential of integrating VLMs into preference-based RL, offering a scalable solution to reducing human supervision while maintaining high task success rates. 

\section*{Impact Statement}
This work advances embodied AI by significantly reducing the human feedback required for training agents. This reduction is particularly valuable in robotic applications where obtaining human demonstrations and feedback is challenging or impractical, such as assistive robotic arms for individuals with mobility impairments. By minimizing the feedback requirements, our approach enables users to more efficiently customize and teach new skills to robotic agents based on their specific needs and preferences. The broader impact of this work extends to healthcare, assistive technology, and human-robot interaction. One possible risk is that the bias from human feedback can propagate to the VLM and subsequently to the policy. This can be mitigated by personalization of agents in case of household application or standardization of feedback for industrial applications. 


{
    \small
    \bibliographystyle{ieeenat_fullname}
    \bibliography{main}
}

% WARNING: do not forget to delete the supplementary pages from your submission 
\clearpage
\pagenumbering{gobble}
\maketitlesupplementary

\section{Additional Results on Embodied Tasks}

To evaluate the broader applicability of our EgoAgent's learned representation beyond video-conditioned 3D human motion prediction, we test its ability to improve visual policy learning for embodiments other than the human skeleton.
Following the methodology in~\cite{majumdar2023we}, we conduct experiments on the TriFinger benchmark~\cite{wuthrich2020trifinger}, which involves a three-finger robot performing two tasks: reach cube and move cube. 
We freeze the pretrained representations and use a 3-layer MLP as the policy network, training each task with 100 demonstrations.

\begin{table}[h]
\centering
\caption{Success rate (\%) on the TriFinger benchmark, where each model's pretrained representation is fixed, and additional linear layers are trained as the policy network.}
\label{tab:trifinger}
\resizebox{\linewidth}{!}{%
\begin{tabular}{llcc}
\toprule
Methods       & Training Dataset & Reach Cube & Move Cube \\
\midrule
DINO~\cite{caron2021emerging}         & WT Venice        & 78.03     & 47.42     \\
DoRA~\cite{venkataramanan2023imagenet}          & WT Venice        & 81.62     & 53.76     \\
DoRA~\cite{venkataramanan2023imagenet}          & WT All           & 82.40     & 48.13     \\
\midrule
EgoAgent-300M & WT+Ego-Exo4D      & 82.61    & 54.21      \\
EgoAgent-1B   & WT+Ego-Exo4D      & \textbf{85.72}      & \textbf{57.66}   \\
\bottomrule
\end{tabular}%
}
\end{table}

As shown in Table~\ref{tab:trifinger}, EgoAgent achieves the highest success rates on both tasks, outperforming the best models from DoRA~\cite{venkataramanan2023imagenet} with increases of +3.32\% and +3.9\% respectively.
This result shows that by incorporating human action prediction into the learning process, EgoAgent demonstrates the ability to learn more effective representations that benefit both image classification and embodied manipulation tasks.
This highlights the potential of leveraging human-centric motion data to bridge the gap between visual understanding and actionable policy learning.



\section{Additional Results on Egocentric Future State Prediction}

In this section, we provide additional qualitative results on the egocentric future state prediction task. Additionally, we describe our approach to finetune video diffusion model on the Ego-Exo4D dataset~\cite{grauman2024ego} and generate future video frames conditioned on initial frames as shown in Figure~\ref{fig:opensora_finetune}.

\begin{figure}[b]
    \centering
    \includegraphics[width=\linewidth]{figures/opensora_finetune.pdf}
    \caption{Comparison of OpenSora V1.1 first-frame-conditioned video generation results before and after finetuning on Ego-Exo4D. Fine-tuning enhances temporal consistency, but the predicted pixel-space future states still exhibit errors, such as inaccuracies in the basketball's trajectory.}
    \label{fig:opensora_finetune}
\end{figure}

\subsection{Visualizations and Comparisons}

More visualizations of our method, DoRA, and OpenSora in different scenes (as shown in Figure~\ref{fig:supp pred}). For OpenSora, when predicting the states of $t_k$, we use all the ground truth frames from $t_{0}$ to $t_{k-1}$ as conditions. As OpenSora takes only past observations as input and neglects human motion, it performs well only when the human has relatively small motions (see top cases in Figure~\ref{fig:supp pred}), but can not adjust to large movements of the human body or quick viewpoint changes (see bottom cases in Figure~\ref{fig:supp pred}).

\begin{figure*}
    \centering
    \includegraphics[width=\linewidth]{figures/supp_pred.pdf}
    \caption{Retrieval and generation results for egocentric future state prediction. Correct and wrong retrieval images are marked with green and red boundaries, respectively.}
    \label{fig:supp pred}
\end{figure*}

\begin{figure*}[t]
    \centering
    \includegraphics[width=0.9\linewidth]{figures/motion_prediction.pdf}
    \vspace{-0.5mm}
    \caption{Motion prediction results in scenes with minor changes in observation.}
    \vspace{-1.5mm}
    \label{fig:motion_prediction}
\end{figure*}

\subsection{Finetuning OpenSora on Ego-Exo4D}

OpenSora V1.1~\cite{opensora}, initially trained on internet videos and images, produces severely inconsistent results when directly applied to infer future videos on the Ego-Exo4D dataset, as illustrated in Figure~\ref{fig:opensora_finetune}.
To address the gap between general internet content and egocentric video data, we fine-tune the official checkpoint on the Ego-Exo4D training set for 50 epochs.
OpenSora V1.1 proposed a random mask strategy during training to enable video generation by image and video conditioning. We adopted the default masking rate, which applies: 75\% with no masking, 2.5\% with random masking of 1 frame to 1/4 of the total frames, 2.5\% with masking at either the beginning or the end for 1 frame to 1/4 of the total frames, and 5\% with random masking spanning 1 frame to 1/4 of the total frames at both the beginning and the end.

As shown in Fig.~\ref{fig:opensora_finetune}, despite being trained on a large dataset, OpenSora struggles to generalize to the Ego-Exo4D dataset, producing future video frames with minimal consistency relative to the conditioning frame. While fine-tuning improves temporal consistency, the moving trajectories of objects like the basketball and soccer ball still deviate from realistic physical laws. Compared with our feature space prediction results, this suggests that training world models in a reconstructive latent space is more challenging than training them in a feature space.


\section{Additional Results on 3D Human Motion Prediction}

We present additional qualitative results for the 3D human motion prediction task, highlighting a particularly challenging scenario where egocentric observations exhibit minimal variation. This scenario poses significant difficulties for video-conditioned motion prediction, as the model must effectively capture and interpret subtle changes. As demonstrated in Fig.~\ref{fig:motion_prediction}, EgoAgent successfully generates accurate predictions that closely align with the ground truth motion, showcasing its ability to handle fine-grained temporal dynamics and nuanced contextual cues.

\section{OpenSora for Image Classification}

In this section, we detail the process of extracting features from OpenSora V1.1~\cite{opensora} (without fine-tuning) for an image classification task. Following the approach of~\cite{xiang2023denoising}, we leverage the insight that diffusion models can be interpreted as multi-level denoising autoencoders. These models inherently learn linearly separable representations within their intermediate layers, without relying on auxiliary encoders. The quality of the extracted features depends on both the layer depth and the noise level applied during extraction.


\begin{table}[h]
\centering
\caption{$k$-NN evaluation results of OpenSora V1.1 features from different layer depths and noising scales on ImageNet-100. Top1 and Top5 accuracy (\%) are reported.}
\label{tab:opensora-knn}
\resizebox{0.95\linewidth}{!}{%
\begin{tabular}{lcccccc}
\toprule
\multirow{2}{*}{Timesteps} & \multicolumn{2}{c}{First Layer} & \multicolumn{2}{c}{Middle Layer} & \multicolumn{2}{c}{Last Layer} \\
\cmidrule(r){2-3}   \cmidrule(r){4-5}  \cmidrule(r){6-7}  & Top1           & Top5           & Top1            & Top5           & Top1           & Top5          \\
\midrule
32        &  6.10           & 18.20             & 34.04               & 59.50             & 30.40             & 55.74             \\
64        & 6.12              & 18.48              & 36.04               & 61.84              & 31.80         & 57.06         \\
128       & 5.84             & 18.14             & 38.08               & 64.16              & 33.44       & 58.42 \\
256       & 5.60             & 16.58              & 30.34               & 56.38              &28.14          & 52.32        \\
512       & 3.66              & 11.70            & 6.24              & 17.62              & 7.24              & 19.44  \\ 
\bottomrule
\end{tabular}%
}
\end{table}

As shown in Table~\ref{tab:opensora-knn}, we first evaluate $k$-NN classification performance on the ImageNet-100 dataset using three intermediate layers and five different noise scales. We find that a noise timestep of 128 yields the best results, with the middle and last layers performing significantly better than the first layer.
We then test this optimal configuration on ImageNet-1K and find that the last layer with 128 noising timesteps achieves the best classification accuracy.

\section{Data Preprocess}
For egocentric video sequences, we utilize videos from the Ego-Exo4D~\cite{grauman2024ego} and WT~\cite{venkataramanan2023imagenet} datasets.
The original resolution of Ego-Exo4D videos is 1408×1408, captured at 30 fps. We sample one frame every five frames and use the original resolution to crop local views (224×224) for computing the self-supervised representation loss. For computing the prediction and action loss, the videos are downsampled to 224×224 resolution.
WT primarily consists of 4K videos (3840×2160) recorded at 60 or 30 fps. Similar to Ego-Exo4D, we use the original resolution and downsample the frame rate to 6 fps for representation loss computation.
As Ego-Exo4D employs fisheye cameras, we undistort the images to a pinhole camera model using the official Project Aria Tools to align them with the WT videos.

For motion sequences, the Ego-Exo4D dataset provides synchronized 3D motion annotations and camera extrinsic parameters for various tasks and scenes. While some annotations are manually labeled, others are automatically generated using 3D motion estimation algorithms from multiple exocentric views. To maximize data utility and maintain high-quality annotations, manual labels are prioritized wherever available, and automated annotations are used only when manual labels are absent.
Each pose is converted into the egocentric camera's coordinate system using transformation matrices derived from the camera extrinsics. These matrices also enable the computation of trajectory vectors for each frame in a sequence. Beyond the x, y, z coordinates, a visibility dimension is appended to account for keypoints invisible to all exocentric views. Finally, a sliding window approach segments sequences into fixed-size windows to serve as input for the model. Note that we do not downsample the frame rate of 3D motions.

\section{Training Details}
\subsection{Architecture Configurations}
In Table~\ref{tab:arch}, we provide detailed architecture configurations for EgoAgent following the scaling-up strategy of InternLM~\cite{team2023internlm}. To ensure the generalization, we do not modify the internal modules in InternML, \emph{i.e.}, we adopt the RMSNorm and 1D RoPE. We show that, without specific modules designed for vision tasks, EgoAgent can perform well on vision and action tasks.

\begin{table}[ht]
  \centering
  \caption{Architecture configurations of EgoAgent.}
  \resizebox{0.8\linewidth}{!}{%
    \begin{tabular}{lcc}
    \toprule
          & EgoAgent-300M & EgoAgent-1B \\
          \midrule
    Depth & 22    & 22 \\
    Embedding dim & 1024  & 2048 \\
    Number of heads & 8     & 16 \\
    MLP ratio &    8/3   & 8/3 \\
    $\#$param.  & 284M & 1.13B \\
    \bottomrule
    \end{tabular}%
    }
  \label{tab:arch}%
\end{table}%

Table~\ref{tab:io_structure} presents the detailed configuration of the embedding and prediction modules in EgoAgent, including the image projector ($\text{Proj}_i$), representation head/state prediction head ($\text{MLP}_i$), action projector ($\text{Proj}_a$) and action prediction head ($\text{MLP}_a$).
Note that the representation head and the state prediction head share the same architecture but have distinct weights.

\begin{table}[t]
\centering
\caption{Architecture of the embedding ($\text{Proj}_i$, $\text{Proj}_a$) and prediction ($\text{MLP}_i$, $\text{MLP}_a$) modules in EgoAgent. For details on module connections and functions, please refer to Fig.~2 in the main paper.}
\label{tab:io_structure}
\resizebox{\linewidth}{!}{%
\begin{tabular}{lcl}
\toprule
       & \multicolumn{1}{c}{Norm \& Activation} & \multicolumn{1}{c}{Output Shape}  \\
\midrule
\multicolumn{3}{l}{$\text{Proj}_i$ (\textit{Image projector})} \\
\midrule
Input image  & -          & 3$\times$224$\times$224 \\
Conv 2D (16$\times$16) & -       & Embedding dim$\times$14$\times$14    \\
\midrule
\multicolumn{3}{l}{$\text{MLP}_i$ (\textit{State prediction head} \& \textit{Representation head)}} \\
\midrule
Input embedding  & -          & Embedding dim \\
Linear & GELU       & 2048          \\
Linear & GELU       & 2048          \\
Linear & -          & 256           \\
Linear & -          & 65536     \\
\midrule
\multicolumn{3}{l}{$\text{Proj}_a$ (\textit{Action projector})} \\
\midrule
Input pose sequence  & -          & 4$\times$5$\times$17 \\
Conv 2D (5$\times$17) & LN, GELU   & Embedding dim$\times$1$\times$1    \\
\midrule
\multicolumn{3}{l}{$\text{MLP}_a$ (\textit{Action prediction head})} \\
\midrule
Input embedding  & -          & Embedding dim$\times$1$\times$1 \\
Linear & -          & 4$\times$5$\times$17     \\
\bottomrule
\end{tabular}%
}
\end{table}


\subsection{Training Configurations}
In Table~\ref{tab:training hyper}, we provide the detailed training hyper-parameters for experiments in the main manuscripts.

\begin{table}[ht]
  \centering
  \caption{Hyper-parameters for training EgoAgent.}
  \resizebox{0.86\linewidth}{!}{%
    \begin{tabular}{lc}
    \toprule
    Training Configuration & EgoAgent-300M/1B \\
    \midrule
    Training recipe: &  \\
    optimizer & AdamW~\cite{loshchilov2017decoupled} \\
    optimizer momentum & $\beta_1=0.9, \beta_2=0.999$ \\
    \midrule
    Learning hyper-parameters: &  \\
    base learning rate & 6.0E-04 \\
    learning rate schedule & cosine \\
    base weight decay & 0.04 \\
    end weight decay & 0.4 \\
    batch size & 1920 \\
    training iters & 72,000 \\
    lr warmup iters & 1,800 \\
    warmup schedule & linear \\
    gradient clip & 1.0 \\
    data type & float16 \\
    norm epsilon & 1.0E-06 \\
    \midrule
    EMA hyper-parameters: &  \\
    momentum & 0.996 \\
    \bottomrule
    \end{tabular}%
    }
  \label{tab:training hyper}%
\end{table}%

\clearpage


\end{document}
