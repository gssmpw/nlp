\section{Related Work} \label{Sec:Related}
\subsection{Circuit Representation Learning}
Circuit representation learning has emerged as an attractive direction in the field of EDA, focusing on training models to obtain general circuit embeddings that can be applied to various downstream tasks~\cite{chen2024large}. The first circuit representation learning framework, DeepGate \cite{li2022deepgate}, proposes supervising model training using logic-1 probability under random simulation and 
achieves substantial improvements in tasks like testability analysis~\cite{shi2022deeptpi} and power estimation~\cite{khan2024deepseq}.  Its successor, DeepGate2 \cite{shi2023deepgate2}, further refines this approach by separating functional and structural embeddings for different applications. Additionally, Gamora \cite{wu2023gamora} and HOGA \cite{deng2024less} leverage sub-circuit identification as pre-training tasks, while FGNN \cite{wang2022functionality} trains models in an unsupervised contrastive manner by distinguishing between equivalent and non-equivalent circuits. Despite these advancements, existing models primarily focus on learning representations of AIG netlists. There is still no available solution capable of learning post-mapping netlists with arbitrary logic cells and addressing the practical applications on post-mapping stages. 

While these advancements primarily address representation learning for intermediate formats such as AIGs, the challenges posed by post-mapping netlists remain largely unexplored. Functional ECO represents one such critical post-mapping application, where effective representation learning could significantly enhance performance and efficiency.

\subsection{Functional ECO}
ECO are a critical component in the VLSI design process, used to rectify design problems after tape-out or chip fabrication. ECO involve making modifications to correct these errors, and they are indispensable in avoiding the high expenses associated with design re-spin ~\cite{jaeger2007virtually}.
% ECO research has been started since the 1990 s, the necessity for automated ECO tools has become more apparent. The management of engineering changes has been a focus of academic and industry research. In the early days, fault models were used to describe the design errors. These algorithms modify the defect design according to the defect model. These fault models are usually not sufficient to represent all functional faults in the defect design, so they can only correct specific errors.
For functional ECO, the purpose is to generate patch so that the original circuit is equal to the golden circuit, while minimizing the resource cost of the generated patches and making the running time as short as possible. Synthesis-based ECO algorithms are good at solving this problem~\cite{huang2013match}. It relies on a diagnostic strategy to identify internal rectifier signals, and then applies a resynthesis technique to generate patch functions for functional differences. These algorithms have been able to automate the process of functional ECO.

% However, their main disadvantage is that they all depend on a single fixed signal, which can cause patches to contain a lot of redundant logic. 

% However, when such algorithms face the Multierror Rectification ECO problem, the generated patches tend to have more redundancy, and the search for candidate patch signals takes a long time.
