\section{Related Works}
\subsection{Information Cascade Modeling}
    Information cascade modeling can be primarily categorized into two types: micro-level____ and macro-level____.
    The former focuses on predicting next affected user, while the latter concentrates on the overall trends, such as its popularity or outbreak status.
    In this paper, we focus on the cascade popularity prediction and categorize existing methods into three types:
    
    \noindent \textbf{Feature-based methods.}
    These works focus on making hand-crafted features for cascade and conducting popularity prediction via traditional machine learning approaches____.
    However, feature-based methods heavily rely on the expert knowledge, has high customization costs, and exhibits limited generalization and suboptimal performance____. 

    \noindent \textbf{Statistics-based methods.}
    These studies assume that information diffusion follow a specific probability statistical model, such as the Poisson process____, Hawkes process____. 
    Statistics-based methods are interpretable but have strong parametric assumptions, making them unsuitable for real-word applications____.
    
    \noindent \textbf{Deep leaning methods.}
    These methods adopt deep learning techniques to promote cascade popularity prediction.
    Early representative works, such as DeepCas____ and DeepHawkes____, focus on capturing the temporal dynamics of cascades via RNN or LSTM.
    Considering the topology in information cascade, GNNs have been introduced to capture local structural patterns. 
    For example, CasCN____ and CoupledGNN____ adopt variant GNNs to model the interactions between users and the spread influence.
    Apart from local structure, CasFlow____ introduces the social network as global context to enhance popularity prediction.
    Advanced techniques, such as VAEs____, Transformers____, and Neural ODEs____, have been further explored in cascade modeling.
    For more comprehensive reviews, please refer to____.

    \noindent \textbf{Towards LLM-based methods.}
    Due to the strong generalization capabilities, various fields such as vision____ and time-series____ are renovates by the general frameworks based on LLMs. 
    To our best knowledge, this is the first attempt to introduce an LLM-based method in information cascade modeling.

\subsection{Autoregressive Modeling}
    Autoregression is a fundamental concept in sequence modeling, which uses observations from previous time steps to predict the next value____.
    This paradigm, which provides fine-grained supervision, has become the best practice for training LLMs ____ and has also inspired other fields____.
    Here, we briefly categorize existing works into three types:
    
    \noindent \textbf{RNN-based methods.}
    Early studies perform the autoregressive modeling based on RNN variants, and achieve success across various domains____.
    However, these approaches come with imitations of RNNs____, including low computational efficiency and limited capability in long-distance dependencies.
    
    \noindent \textbf{Transformer-based methods.}
    Following the introduction of Transformer____, the potential of autoregressive modeling has been further explored, with representative works including iGPT____, Autoformer____ and VAR____.
    Currently, transform-based methods dominate the field of autoregressive modeling.
    
    \noindent \textbf{LLM-based methods.}
    Built upon Transformer architecture, LLMs with large-scale parameters are pretrained on massive datasets, demonstrating superior capabilities in autoregressive modeling.
    Therefore, researchers attempt to investigate the feasibility of reusing LLMs for autoregressive modeling.
    For example, Toto____ treats videos as sequences of visual tokens and reuses the LLMs as backbones to autoregressively predict future tokens.
    Similar ideas have also been applied in the field of time-series forecasting____.

\subsection{Prompt Learning}
    Prompt learning____ has emerged as a novel learning paradigm to adapt LLMs to specific tasks by designing textual prompts.
    Due to the widespread application of LLMs, designing sophisticated textual prompts for specific tasks has become the research hotspot across various fields____.
    For example, 
    LLM4NG____ designs the class-level semantic prompt templates based on text-attribute on graphs to facilitate the node classification in few-shot scenarios.
    Autotimes____  introduces textual timestamps of time-series to enhance LLM-based forecasting.
    Building on advanced prompting techniques, deft textual prompts ____ for time-series are further explored.
    Since LLMs have not yet fully entered the field of information cascade modeling, studies on prompt learning for cascade data are still lacking. 
    However, textual information is prevalent in cascade diffusion process, so designing cascade prompt templates based on textual information for cascade popularity prediction holds significant promise.