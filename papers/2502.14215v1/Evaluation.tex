\section{Evaluation}
\label{sec:evaluation}
\subsection{Implementation}
We implemented our approach in \tool in around 4,000 lines of Python for partition generation and 500 lines of C++ for equivalence verification built on SolSEE~\cite{lin2022solsee}. 
We used Z3 solver~\cite{de2008z3}, version 4.13.0, to check equivalence relationship between the original and partitioned function versions.
\tool is empowered by \textit{GPT-4o mini} developed by OpenAI, where its default model setting is reused.
We also employed the embedding model \textit{text-embedding-ada-002} developed by OpenAI  to score LLM-written program partitions compared to human-written ones in terms of embedding similarity.
We used the Levenshtein distance to measure the edit distance between the original functions and its partitioned output.
Furthermore, we used Slither~\cite{slither}, version 0.10.4, for PDG construction and taint analysis for Solidity smart contracts.

\subsection{Research Question}
In the evaluation, we aim to answer the following research questions.
\begin{itemize}[leftmargin=*]
    \item  \textbf{RQ1}. How accurately does \tool generate fine-grained program partitions for smart contracts?
    \item  \textbf{RQ2}. How useful is \tool to mitigate real-world smart contracts from attacks? 
    \item  \textbf{RQ3}. What is the run-time gas overhead of \tool?
    \item  \textbf{RQ4}. How do different base LLMs affect the performance of \tool?
\end{itemize}

\paragraph{Benchmark and ground truth}
To answer RQ1, we collected real-world confidential smart contracts tagged with sensitive annotations from previous studies on fully homomorphic encryption-based solution fhEVM~\cite{fhEVM} and MPC-based computation-based solution COTI~\cite{COTI}.
Specifically, 
we initially obtained 10 cases from fhevm-examples~\cite{fhevm-examples}, 6 cases from COTI-examples~\cite{COTI-examples}, and additionally crawled from GitHub 6 smart contract applications participating in the competition using the fhEVM. 
We manually analyzed these contracts, and excluded 4 cases involving only the encryption of input data and the use of TEE-generated random number because they do not apply to program partitioning.
Finally, we built a confidential smart contract benchmark encompassing 18 smart contracts with sensitive annotations as shown in~\Cref{tab:gen}.
Specifically, there are 99 sensitive contract functions related to the \secrete variables.
To obtain the ranking coefficients (c.f.~\Cref{sec:ranking}), 
we manually partitioned these contract functions to construct a human-written dataset.
To avoid manual mistakes, we validate program partitions by performing equivalence checking using the proposed formal verification tool.

To answer RQ2, we searched all the price manipulation attacks recorded in the well-studied DeFi hack repository~\cite{DeFiHacks} as of September 2024.
There are 63 incidents labeled with price manipulation attacks.
We investigate the root cause of the attack, and found the root cause analysis reports for 25 cases are missing, and 15 of the rest cases are actually due to other vulnerability issues such as permission bugs or not open source.
For the remaining 23 cases,
13 cases were attacked because of the use of vulnerable API \textit{getAmmountIn/getAmmountOut} by Uniswap V2, while the rest cases are caused by other customized API uses.
To avoid bias in the evaluation, we selected two case of \textit{getAmmountIn/getAmmountOut}, i.e., Nmbplatform and SellToken, and select five out of the remaining 11 cases in the manipulation-related victim contract benchmark.
Beside manipulation attacks recorded in DeFi hack repository, we include two reported randomness manipulation attacks\footnote{https://owasp.org/www-project-smart-contract-top-10/2023/en/src/SC08-insecure-randomness.html}. 
Finally, to answer RQ2, we curated 9 attacks, leading to a total loss of about 25 million dollars.  

\paragraph{Experiment setup} All the experiments are conducted on a server computer equipped with Ubuntu 22.04.5 LTS, 504 GB RAM, and 95 AMD 7643 Cores. 
We used the commercial OpenAI API to access \textit{GPT-4o mini} and \textit{text-embedding-ada-002}. Moreover, we used the advanced open source LLMs including Gemma2:27b, Llama3.1:8b, Qwen2.5:32b supported by Ollama\footnote{https://github.com/ollama/ollama} for the sensitivity study in RQ4. 
We allowed the symbolic execution unrolls up to five times for loop statements and the overall time budget for the equivalence checking was capped at ten minutes.
The evaluation artifact is available on~\website.  
% \subsection{Experiment Setup}
% \ye{list experiment setting}
%Please add the following packages if necessary:
%\usepackage{booktabs, multirow} % for borders and merged ranges
%\usepackage{soul}% for underlines
%\usepackage{xcolor,colortbl} % for cell colors
%\usepackage{changepage,threeparttable} % for wide tables
%If the table is too wide, replace \begin{table}[!htp]...\end{table} with
%\begin{adjustwidth}{-2.5 cm}{-2.5 cm}\centering\begin{threeparttable}[!htb]...\end{threeparttable}\end{adjustwidth}
% \begin{table*}[t]\centering
% \caption{The experiment result on partition generation.}\label{tab:gen}
% \scriptsize
% \begin{tabular}{lp{0.16\linewidth}rrr|rrrrrrrrr}\toprule
% \multirow{2}{*}{contracts} &\multirow{2}{*}{Secrete data} &\multirow{2}{*}{LoC} &\multirow{2}{*}{\#Func} &\multirow{2}{*}{\parbox{.7cm}{\raggedleft Ground Truth}} &\multirow{2}{*}{\parbox{.8cm}{\raggedleft \#Critical Func}} &\multirow{2}{*}{\parbox{.8cm}{\raggedleft \#Partitions of Funcs}} &\multicolumn{6}{c}{ Top-2} \\\cmidrule{8-13}
% & & & & & & &TP &FP &FN &Precision &Recall &F1 \\\midrule
% ConfidentialID &identities &164 &17 &8 &8 &65 &12 &2 &\cmark &0.86 &0.92 &0.89 \\
% ConfidentialERC20 &balances, allowances &117 &9 &5 &4 &32 &8 &0 &\cmark &1.00 &0.89 &0.94 \\
% TokenizedAssets &assets &21 &3 &3 &3 &25 &6 &0 &0 &1.00 &1.00 &1.00 \\
% EncryptedERC20 &balances, allowances &109 &16 &6 &4 &40 &6 &2 &0 &0.75 &1.00 &0.86 \\
% DarkPool &orders, balances &84 &8 &7 &6 &38 &4 &4 &3 &0.50 &0.57 &0.53 \\
% IdentityRegistry &identities &167 &17 &8 &7 &54 &12 &2 &\cmark &0.86 &0.92 &0.89 \\
% BlindAuction &bids &190 &13 &5 &5 &42 &6 &4 &0 &0.60 &1.00 &0.75 \\
% ConfidentialAuction &bids &188 &14 &6 &6 &58 &9 &3 &0 &0.75 &1.00 &0.86 \\
% Battleship &player1/2-Board &78 &3 &2 &2 &10 &2 &0 &\cmark &1.00 &0.67 &0.80 \\
% VickreyAuction &bids &232 &9 &5 &5 &38 &7 &\cmark &\cmark &0.88 &0.88 &0.88 \\
% Comp &balances, allowances &136 &9 &6 &2 &20 &4 &0 &4 &1.00 &0.50 &0.67 \\
% GovernorZama &proposals &203 &17 &10 &9 &67 &11 &5 &\cmark &0.69 &0.92 &0.79 \\
% Suffragium &\_votes &155 &21 &7 &4 &32 &7 &\cmark &3 &0.88 &0.70 &0.78 \\
% AuctionInstance &Biddinglist &149 &8 &3 &3 &13 &0 &5 &0 &NA &NA &NA \\
% Leaderboard &players &23 &3 &3 &3 &28 &2 &4 &0 &0.33 &1.00 &0.50 \\
% NFTExample &\_tokenURIs &602 &34 &2 &2 &15 &3 &\cmark &0 &0.75 &1.00 &0.86 \\
% CipherBomb &cards, roles &280 &16 &7 &8 &55 &9 &3 &2 &0.75 &0.82 &0.78 \\
% EncryptedFunds & totalSupplyPerToken, etc. &168 &16 &6 &5 &40 &4 &4 &2 &0.50 &0.67 &0.57 \\
% \midrule
% Overall & &170 & 233 &99 &86 &672 & & & &0.77 &0.85 &0.78 \\
% \bottomrule
% \end{tabular}
% \end{table*}
%Please add the following packages if necessary:
%\usepackage{booktabs, multirow} % for borders and merged ranges
%\usepackage{soul}% for underlines
%\usepackage{xcolor,colortbl} % for cell colors
%\usepackage{changepage,threeparttable} % for wide tables
%If the table is too wide, replace \begin{table}[!htp]...\end{table} with
%\begin{adjustwidth}{-2.5 cm}{-2.5 cm}\centering\begin{threeparttable}[!htb]...\end{threeparttable}\end{adjustwidth}
\begin{table*}[t]\centering
    \caption{The experiment result on partition generation. Note for fair comparison, \#Partitions$\star$ excludes those partition results of the identified sensitive functions beyond the ground truth, $Precision = \frac{TP}{TP + FP}$, and $Success\; rate = \frac{\#Hit} { \#Ground Truth}$ indicates the proportion of functions in the ground truth for which \tool successfully produce correct program partitions.}\label{tab:gen}
    \scriptsize
    \resizebox{\linewidth}{!}{
    \begin{tabular}{lp{0.16\linewidth}rrr|rr|rrrrrrr}\toprule
    \multirow{2}{*}{Contract} &\multirow{2}{*}{Secrete data} &\multirow{2}{*}{LoC} &\multirow{2}{*}{\parbox{.4cm}{\#Func}} &\multirow{2}{*}{\parbox{.7cm}{\raggedleft \#Ground Truth}} &\multirow{2}{*}{\parbox{.8cm}{\raggedleft \#Sensitive Func}} &\multirow{2}{*}{\parbox{1cm}{\raggedleft \#Partitions$\star$}} &\multirow{2}{*}{TP} &\multirow{2}{*}{FP} &\multirow{2}{*}{\#Hit} &\multirow{2}{*}{Precision} &\multirow{2}{*}{Success rate}  \\
    & & & & & & & & & & &\\\midrule
    ConfidentialID &identities &164 &17 &8 &8 &67 &16 &8 &7 &0.67 &0.88  \\
    ConfidentialERC20 &balances, allowances &117 &9 &5 &5 &42 &15 &0 &5 &1.00 &1.00 \\
    TokenizedAssets &assets &21 &3 &3 &3 &30 &9 &0 &3 &1.00 &1.00  \\
    EncryptedERC20 &balances, allowances &109 &16 &6 &6 &60 &15 &3 &6 &0.83 &1.00  \\
    DarkPool &orders, balances &84 &8 &7 &7 &50 &9 &6 &3 &0.60 &0.43 \\
    IdentityRegistry &identities &167 &17 &8 &8 &73 &18 &6 &7 &0.75 &0.88  \\
    BlindAuction &bids &190 &13 &5 &5 &45 &9 &6 &3 &0.60 &0.60  \\
    ConfidentialAuction &bids &188 &14 &6 &6 &57 &12 &6 &4 &0.67 &0.67  \\
    Battleship &player1/2-Board &78 &3 &2 &2 &13 &3 &3 &2 &0.50 &1.00  \\
    VickreyAuction &bids &232 &9 &5 &5 &49 &8 &7 &3 &0.53 &0.60  \\
    Comp &balances, allowances &136 &9 &7 &7 &62 &13 &6 &5 &0.68 &0.71  \\
    GovernorZama &proposals &203 &17 &10 &10 &90 &18 &12 &6 &0.60 &0.60  \\
    Suffragium &\_votes &155 &21 &6 &6 &57 &17 &1 &6 &0.94 &1.00  \\
    AuctionInstance &Biddinglist &149 &8 &3 &3 &18 &NA &NA &NA &NA &NA  \\
    Leaderboard &players &23 &3 &3 &3 &28 &9 &0 &3 &1.00 &1.00  \\
    NFTExample &\_tokenURIs &602 &34 &2 &3 &19 &4 &2 &2 &0.67 &1.00  \\
    EncryptedFunds &SupplyPerToken, etc. &168 &16 &6 &7 &60 &18 &0 &6 &1.00 &1.00  \\
    CipherBomb &cards, roles &280 &16 &7 &8 &57 &15 &3 &5 &0.83 &0.71  \\
    \midrule
    Overall & &170&233 &99 &102 &877 &208 &69 &76 &0.76 &0.78 \\
    \bottomrule
    \end{tabular}
    }
\end{table*}


\subsection{RQ1: Partition Generation}
We evaluated \tool on the aforementioned 18 confidential smart contracts encompassing 99 sensitive functions. 
\Cref{tab:gen} shows the experiment results under the top-3 setting.
The first five columns on the left demonstrate the name, \secrete variables, line of codes, number of public functions, and number of sensitive functions of smart contracts, respectively.
The middle two columns present the number of sensitive functions identified by \tool and the number of generated program partitions for those identified sensitive functions that match with the ground truth.  
The rest columns demonstrate the verification results of the resulting partitions after equivalence checking.
TP and FP refer to the number of correct and incorrect program partitions that pass and fail the equivalence checking, respectively.
\#Hit displays the number of ground truth functions for which \tool successfully produce compilable, secure and functionally equivalent partitions, and we use \emph{Success rate} to indicate its proportions in ground truth functions.

\Cref{tab:gen} clearly shows that \tool is able to generate a reasonably high accuracy of program partitioning for smart contracts.
Overall, \tool is able to generate 877 program partitions for 99 ground truth functions out of which 76 functions have been successfully partitioned, achieving success rate of 0.78, with relatively good precision 0.76.
Particularly, \tool failed to perform equivalence checking for \texttt{AuctionInstance} due to timeout. 
This is because AuctionInstance has nested loops in its sensitive functions, and although we cap the loop iteration count to five during symbolic execution, there remains a state explosion problem.
\tool wrongly flagged the function \texttt{open} of CipherBomb as sensitive function because this function can write a data variable \texttt{turnDealNeeded} that a sensitive function \texttt{checkDeal} explicitly declassified from the known \secrete \texttt{cards, roles} as shown in \Cref{tab:gen}, while \tool currently does not support the declassification annotation.  
Nevertheless, \tool also has the potential to identify other sensitive functions beyond the human-specified ground truth functions.
% \ye{To fix}
% Specifically, \tool has detected one more potentially sensitive functions for contract \texttt{NFTExample}, \texttt{EncryptedFunds}, and \texttt{CipherBomb}.
\tool flagged the function \texttt{mint} of NFTExample because mint writes values to one of the sensitive data variable, and
successfully discover a neglected view function \texttt{getEncryptedTokenID} involving sensitive data variables.

\begin{figure}[t]
    \centering
    \includegraphics[trim=1cm 1cm 0cm 1cm, width=\columnwidth]{figure/top-k.pdf}
    \caption{The impact of Top-K selection.}
    \label{fig:topk_selection}
\end{figure}

We also investigate the impact of different top-K setting.
\Cref{fig:topk_selection} demonstrates the averaged precision, success rate, and the ratio of the size of isolated privileged (trusted) codebase (TCB) by \tool compared to the size of original function code. 
We use this ratio metric to reflect the TCB minimization in comparison with isolating the whole sensitive functions into secure partitioning~\cite{brumley2004privtrans, liu2017ptrsplit, wu2013automatically}.
\Cref{fig:topk_selection} delineates that the success rate jumps from top-1 to top-3, and then remains relatively stable. 
But, TCB size decreases a bit from top-1 to top-2 and then increases slightly, and generally the trusted codebase encompassing the privileged operations reduces around 30\% code.
We argue that \tool aims to generate not only correct program partitions but also fine-grained  partitions that could reduce the size of trusted codebase compared to the coarse-grained, namely function-level program partitioning.
Therefore, we suggest using top-3 as the default setting for \tool.

We studied all the 877 program partitions yielded by \tool.
Excluding AuctionInstance's partitions, there are in total 675 true positives and 184 false positives.
There are four causes of false positives.
We found 76 cases failing due to the state change inequivalence and 38 cases failing due to the unmatched return value, and 24 cases are caused by incomparable partition results that differ significantly from the original function, while the remaining 46 cases are caused by the internal error of the equivalence checker such as incorrect Z3 array conversion from complicated structure variables of Solidity code.
Besides, we highlighted that since our equivalence checker is conservative and some reported false positives may not be true. 
For instance, we found for \texttt{BlindAuction}, \tool generates 5 partitions for its \texttt{bid} function (c.f.,~\Cref{lst:bid}) which involve complicated external contract calls beyond the capability of the developed dedicated equivalence checker.
Although \tool reported all the five partitions are false positives, we manually checked and confirmed that three of them are actually true positives.



\begin{tcolorbox}[size=title, opacityfill=0.1]
	\textbf{Answer to RQ1}: 
	\tool is able to generate correct program partitions for smart contracts with reasonably high accuracy. Under top-3 setting, among 99 functions to partition, \tool identified all the sensitive functions while their generated partitions gain a success rate of 78\% and reduce around 30\% code compared to function-level partitioning.  
\end{tcolorbox}
% \begin{figure}
%     \includegraphics[width=\linewidth]{figure/CaseStudy.pdf}
%     \caption{Case study of bad partition result of function \textit{bid}.}
% \end{figure}

% % \ye{Should we interpret more? As we found that some false positives are not \texttt{true} false positives.}




% \paragraph{Introduce experiment result.}

% \paragraph{Investigation of False Positives}

% \paragraph{Effectiveness of LLM Repairs}




%Please add the following packages if necessary:
%\usepackage{booktabs, multirow} % for borders and merged ranges
%\usepackage{soul}% for underlines
%\usepackage{xcolor,colortbl} % for cell colors
%\usepackage{changepage,threeparttable} % for wide tables
%If the table is too wide, replace \begin{table}[!htp]...\end{table} with
%\begin{adjustwidth}{-2.5 cm}{-2.5 cm}\centering\begin{threeparttable}[!htb]...\end{threeparttable}\end{adjustwidth}
\begin{table*}[t]\centering
    \caption{The experiment result on manipulation attack mitigation.}\label{tab:vul_compare}
    \small
    \begin{tabular}{lrrrrrrrr}\toprule
    Victim &Date &Loss &DefiTainter &DeFort &DefiRanger &GPTScan &\tool \\\midrule
    Nmbplatform &14-Dec-2022 &\$76k &\cmark &\cmark &\cmark &\xmark  &\cmark \\
    SellToken &11-Jun-2023 &\$109k &\xmark  &\xmark  &\cmark &\xmark &\cmark \\
    Indexed Finance &15-Oct-2021 &\$16M &\xmark &\xmark &\cmark &\xmark &\cmark \\
    Jimbo &29-May-2023 &\$8M &\xmark &\xmark &NA &\cmark &\cmark \\
    NeverFall &3-May-2023 &\$74K &\cmark &\xmark &\xmark &\xmark &\cmark \\
    BambooAI &4-Jul-2023 &\$138K &\xmark &\xmark &\cmark &\cmark &\cmark \\
    BelugaDex &13-Oct-2023 &\$175K &\xmark &\xmark &NA &\xmark &\xmark \\
    Roast Football &5-Dec-2022 &\$8K &\xmark &\xmark &NA &\xmark &\cmark \\
    FFIST &20-Jul-2023 &\$110K &\xmark &\xmark &NA &\cmark &\cmark \\
    \midrule
    Overall & & &2 &1 &4 &3 &8 \\
    \bottomrule
    \end{tabular}
    \end{table*}
\subsection{RQ2: Application in Real-world Victim Contracts}
We evaluated \tool on nine victim contracts vulnerable to price and randomness manipulation. 
Recall \tool is able to identify all the sensitive functions related to given \secrete that we believe play pivotal role in the manipulation attack where attackers interfere with the \secrete without any protection.
In typical manipulation incidents, attackers first manipulate the \secrete and then earn a profit.
Most attacks like the well-studied price manipulation will incur at least two function calls where the first function call will alter \secrete variables about liquidity, and the second function call will make a profit with the use of the manipulated data.
% This ``write-read'' access pattern of \secrete can be exploited to detect potential manipulation attacks.
% Specifically, we will report a manipulation attack if a sensitive information flow exists in any two sensitive functions within attacker transactions.

We compare \tool with the existing manipulation detection tools DefiTainter~\cite{kong2023defitainter}, DeFort~\cite{xie2024defort}, DefiRanger~\cite{wu2023defiranger}, and GPTScan~\cite{sun2024gptscan}.
Because DefiRanger is not open source, we take their reported results~\cite{wu2023defiranger} in~\Cref{tab:vul_compare}. 
Different from DefiTainter and DeFort demanding function-related annotations, \tool needs developers to provide a set of annotations on \secrete variables of smart contracts.
We manually analyze the source code of smart contracts and identify security-critical related data variables as the \secrete variables to protect. 
For instance, the data variables ``\_liability'' and ``\_totalSupply'' of BelugaDex are critical since they determine the amount of tokens to burn, which is vulnerable to arbitrary manipulation by attackers.
\Cref{tab:vul_compare} shows the comparison results.
\tool is able to mitigate eight attacks by reducing the attack surface, followed by DefiRanger and GPTScan.
% \tool failed to detect the manipulation attack of \texttt{Roast Football} because in this case, manipulation of \secrete and profit making happen during a single function call, which is beyond the detection capability of \tool. 
\tool failed to defend against \textbf{BelugaDex}'s attack because this attack arises from a flawed token withdrawal logic rather than manipulating the constant token exchange rate between two stable coins: USDT and USDCE. The withdrawal amount depends on the two statuses of the underlying asset token: \textit{\_liability} equaling to the sum of deposit and dividend, and \textit{totalSupply} equaling to the sum of token distribution. 
In this exploit, the attacker deposited USDT tokens and then used swap function between USDT and USDCE to update asset liability. 
Consequently, the attacker spent less USDT asset tokens to withdraw the same original deposit tokens.
Although \tool could hide \textit{\_liability} and \textit{totalSupply} within a secure environment, the abovementioned attack vector still exists.

We detail how \tool mitigates the following attacks.
\begin{itemize}[leftmargin=*]
  \item \textbf{Nmbplatform} and \textbf{SellToken} use Uniswap V2-based liquidity pools to swap token A for another token B, but these pools are vulnerable to price manipulation because the reserves of two tokens within the pools are visible to the public. Therefore, malicious users can easily create a significant liquidity imbalance by depositing a certain amount of tokens  borrowed using flashloan technique. \tool mitigates these attacks through marking the reserves of two tokens within the pools as sensitive and hiding all operations on these reserves in a privileged partition inside a secure environment.   
  \item \textbf{Indexed Finance} uses a custom price model to calculate token exchange rate based on a critical data structure called \textit{\_records}. Because one of its fields named \textit{denorm} is not affected by the change of the liquidity within the contract, attacker is able to manipulate the contract to forge a flawed token price that creates an arbitrage opportunity. \tool treated \textit{\_record} as sensitive data variables, and created privileged partition for the sensitive operations. Therefore, \textit{denorm} of \textit{\_record} is invisible to malicious users, reducing the chance of gaining a profit.   
  \item \textbf{Jimbo}'s price calculation relies on the complicated states of bins comprising \textit{activeBin}, \textit{triggerBin}, and so on (c.f.~\Cref{sec:motivation}). By observing and interfering with these bins, malicious users are able to initiate manipulation attacks. \tool places all sensitive operations related to these bins into privileged partition for executing within a secure environment, making bins' states always sealed to mitigate such manipulation. 
  \item \textbf{NeverFall} and \textbf{BambooAI} are liquidity pool token contracts implementing ERC20 standard~\cite{eip20}, where NeverFall permits users to sell and buy liquidity tokens, and BambooAI allows users to update the pool's parameters when performing token transfers. In both of the two attacks, malicious users manipulate a bookkeeping variable \textit{balances} that manages the distribution of tokens among users and is responsible for the liquidity price calculation, through selling, buying, and transferring a certain amount of tokens. \tool defends against these attacks by isolating all operations related to \textit{balances} in a privileged partition, making it hard for attackers to predicate the amount of tokens they need to carry out attacks. 
  % \item \textbf{BelugaDex}'s attack arises from a flawed token withdrawal logic rather than manipulating the constant token exchange rate between two stable coins: USDT and USDCE. The withdrawal amount depends on the two statuses of the underlying asset token: \textit{\_liability} equaling to the sum of deposit and dividend, and \textit{totalSupply} equaling to the sum of token distribution. In this exploit, the attacker deposited USDT tokens and then used swap function between USDT and USDCE to update asset liability. Consequently, the attacker spent less USDT asset tokens to withdraw the same original deposit tokens.        
  \item \textbf{Roast Football} and \textbf{FFIST} are two contracts on BSC that use on-chain data for random number generation for lottery and token airdrop purpose, respectively. For instance, as shown in~\Cref{lst:roast}, the lottery of Roast Football leverage a function \textit{randMod} to generate random number (\Cref{randnum:gen}) from the current block number and timestamp, user-given input \textit{buyer}, and contract-managed data \textit{\_balances} recording token distributions (\Cref{balance:decl}). As the sensitive random seed \textit{\_balances} is known to the public, malicious users could easily manipulate the input with an elected buyer address to generate a desired random number that bypasses any of the branch constraints in this function. Considering \textit{\_balances} as sensitive data variable, \tool isolates all the data and operations (in \textcolor{academicred}{orange} box, Lines \ref{balance:decl} and \ref{randnum:gen}) about random number generation, thus disallowing attackers to predicate random number output. We also highlight that some approach~\cite{luu2016making, atzei2017survey} could complain the use of block-related data like \textit{block.number} and \textit{block.timestamp} since they may be manipulated by blockchain miners, and such manipulation can be prevented through decentralized governance mechanisms such as proof of stakes, which is out of our research scope. FFIST generates random number with a previously airdropped address stored in the contract. The way \tool protects this generation is similar to \textit{Roast Football}, and we elide it for saving space. 
\end{itemize}

\begin{figure}[t]
      \begin{lstlisting}[basicstyle=\scriptsize,  numberstyle=\scriptsize, language=Solidity, caption={The \textit{randMod} function of Roast Football.}, label=lst:roast]
contract RFB{
  @\tikzmark{starta}@mapping (address => uint256) _balances;@\tikzmark{enda}@ @\label{balance:decl}@
  function randMod(address buyer,uint256 buyamount) internal  returns(uint){
    @\tikzmark{startb}@uint randnum = uint(keccak256(abi.encodePacked(block.number,block.timestamp,buyer,@\textbf{\_balances[pair]))}@);@\tikzmark{endb}@ @\label{randnum:gen}@
    uint256 buyBNBamount = buyamount.div(10**_decimals).mul(getPrice());
    // increase nonce
    if(randnum%(10000*luckyMultiplier) == 8888 && buyBNBamount > (0.1 ether)){
        distributor.withdrawDistributor(buyer, 79);
        distributor.withdrawDistributor(marketingFeeReceiver,9);
    }else if(randnum%(1000*luckyMultiplier) == 888){
        ...
    }else if(randnum%(100*luckyMultiplier) == 88){
        ...
    }else if(randnum%(10*luckyMultiplier) == 8){
        ...
    }
    return randnum;
  }
}
\end{lstlisting}
\begin{tikzpicture}[remember picture,overlay]
  \draw[academicred,thick,rounded corners]
    ([shift={(-3pt,1.3ex)}]pic cs:starta) 
      rectangle 
    ([shift={(3pt,-0.65ex)}]pic cs:enda);
  \draw[academicred,thick,rounded corners]
    ([shift={(-3pt,1.3ex)}]pic cs:startb) 
      rectangle 
    ([shift={(20pt,-0.65ex)}]pic cs:endb);
\end{tikzpicture}
\end{figure}



Therefore, we argue that \tool is able to effectively mitigate real-world manipulation attacks through secure program partitioning.

% \ma{
% % \textit{Defense against integer overflow:} \\
% % \textit{Defense against reentrancy:} \\
% \textit{Defense against front-running:} If a malicious user has the ability to obtain the pubilc transaction details, it will cause a front-running attack. 
% For example, in XXX contract XXX is sensitive transaction information that determines which transactions can be added to the chain. 
% Under the protection of PartitionGPT, no one could steal this data, since all sensitive transaction information will be processed in TEE.\\ 
% \textit{Defense against user-control randomness:} In smart contracts, pseudo-random numbers generated based on public on-chain information are predictable, allowing attackers to exploit controllable randomness for transaction fraud. 
% However, after partitioning with PartitionGPT, random numbers can be securely generated by TEE. 
% For instance, XXX in XXX contract is a random number that cannot be collected or tampered with by external entities.\\
% \textit{Defense against time manipulation:} In XXX contract, transactions need to be confirmed based on the system time. 
% Since system time in untrusted normal side can be tampered with by attackers, they can interfere with the transaction process. 
% After partitioning, the time in the contract is securely generated by TEE, such attacks can be effectively prevented.\\
% }


\begin{tcolorbox}[size=title, opacityfill=0.1]
	\textbf{Answer to RQ2}: 
	\tool is able to defend against eight of nine real-world manipulation attacks beyond the existing detection-only tools, underscoring the potential of secure program partitioning in preventing sensitive data leakage to mitigate manipulation attacks.
\end{tcolorbox}

% \paragraph{Summary of application results}

% \paragraph{One-by-one explanation.}

% \ye{Perform a literature review to search all the vulnerable functions in economic attacks}
% \ye{Point: enhance security of smart contracts}

% \ye{price manipluation - getAmountOut}

% \ye{randomness -- this may not be good example}
\begin{figure}[t]
    \centering
    \includegraphics[trim=0cm 1.5cm 0cm 1cm, width=\columnwidth]{figure/RuntimeOverhead.pdf}
    \caption{The runtime overhead of deploying partitions.}
    \label{fig:runtime_overhead}
    % \vspace{-0.8cm}
\end{figure}

\subsection{RQ3: Runtime Overhead}
The partitions produced by \tool should be deployed to secure execution environment to protect sensitive functionality.
Trusted execution environment (TEE) has been employed to protect the privacy of smart contracts~\cite{cheng2019ekiden, russinovich2019ccf}.
TEE is able to execute general instructions but with minimal time cost.
% , which has advantages over the alogirithm-based privacy protection solutions such as MPC~\cite{} and homomorphic encryption~\cite{}.
There have been several TEE infrastructures designed for smart contracts such as open-source CCF~\cite{russinovich2019ccf} in which the whole Ethereum Virtual Machine is running at SGX enclave mode supporting TEE devices like Intel SGX.

To conduct the experiments, we select eight functions from two contracts listed in~\Cref{fig:runtime_overhead}.
The first four functions belong to \texttt{BlindAuction}, and they are responsible for the biding, querying and post-biding processing, while the latter four functions of \texttt{EncryptedERC20} represent the most common ERC20 functionalities. 
For each contract, it includes two executable instances: one for the public part deployed in a Ethereum test network powered by Ganache~\cite{Ganache} of version 6.12.2, and the other for the privileged part deployed in the CCF network~\cite{russinovich2019ccf} with a simulated TEE environment. 
For the cross-chain communications between Ethereum-side and TEE-side instance, we replace the call statements, i.e., XXX\_priv(), with message events logged in the normal test network, which will then be passed to a third-party router to trigger the execution of privileged function within the TEE-side instance.
Also, we replace the call statements, i.e., XXX\_callback(), with message events logged in the CCF network and then the third-party router processes the events to trigger the callback function of the Ethereum-side instance.
To orchestrate the communication, we formulate their call orders into scheduling policies.
We highlight that the aforementioned substitution rules can be automated by using a few code transformation templates.
Furthermore, we implemented the third-party router as a listener thread to automatically monitor new events from the normal test network and the CCF network and then deal with the events according to the abovementioned scheduling policies.
For each function, we manually crafted five test cases to execute and then collected their runtime gas overheads.
We clarify that currently we do not add encryption and decryption methods for the data communicated between the Ethereum-side and TEE-side instance.
Developers of smart contracts should be responsible for this particular setting.
For instance, developers may need to exchange their keys through transactions or smart contracts where the data will be decoded in the TEE-side~\cite{ren2022cloak,scrt}.

\Cref{fig:runtime_overhead} plots the runtime gas overhead of the original contract and the TEE-powered post-partition contract.
Note that the execution of the TEE-side instance will not incur gas overhead for the CCF network.
Apparently, after partitioning, six out of eight contract functions will take more gas overhead, increasing between 61\% and 103\%. 
The main reason is that these functions not only communicate message from the Ethereum-side instance with TEE-side instance but also deal with callback message from TEE-side instance to require  Ethereum-side instance to proceed with normal operation execution, leading to two more transactions. 
In comparison, for the two functions: \texttt{withdraw} and \texttt{mint}, callback-related communications are not needed, thus reducing the gas consumption.
Note each communication will take one transaction, and 21,000 gas is charged for any transaction on Ethereum as a ``base fee'' so that $n$ transactions will need to consume no less than n$\times$21,000 gas.
Therefore, we argue that although deploying the partitions into TEEs could incur more gas overhead, developers could still benefit in twofold.
First, sensitive functions usually take a small proportion of all the public functions (c.f., 99 out of 233 in~\Cref{tab:gen}), resulting in a moderate gas overhead increase for users.     
Second, the attack surface of smart contracts is largely minimized, and it will be difficult for attackers to manipulate the \secrete variables within a secure environment like TEE.

\begin{table}[t]\centering
    \caption{Effectiveness with different LLMs.}\label{tab:llm_compare}
    \scriptsize
    \begin{tabular}{lrrrrr}\toprule
      LLM &\#Partitions &TP &FP &Precision \\\midrule
      GPT-4o mini &907 &675 &184 &0.78 \\
      Gemma2:27b &621 &436 &149 &0.75 \\
      llama3.1:8b &227 &98 &121 &0.44 \\
      Qwen2.5:32b &758 &476 &157 &0.75 \\
      \bottomrule
    \end{tabular}
\end{table}

% \begin{figure}[t]
%     \centering
%     \includegraphics[trim=1cm 1.2cm 0cm 0.7cm,width=.7\linewidth]{figure/Repair.pdf}
%     \caption{The distribution of secure-repair count.}
%     \label{fig:repair_count}
% \end{figure}
\begin{figure}[t]
    \centering
    \includegraphics[trim=1cm 1.2cm 0cm 0.7cm,width=\columnwidth]{figure/fix.pdf}
    \caption{The distribution of the times that different LLMs need to generate compilable smart contracts.}
    \label{fig:fix_count}
\end{figure}
\begin{tcolorbox}[size=title, opacityfill=0.1]
	\textbf{Answer to RQ3}: 
	The runtime gas overhead is moderate by deploying partitions to a TEE-based blockchain infrastructure, where except for two cases, after partitioning by \tool, six of eight functions incur 61\% to 101\% more gas mostly paid for additional communication transactions.  
\end{tcolorbox}

\subsection{RQ4: Sensitivity Study}
To investigate the effectiveness of different base LLMs,
we conducted a sensitivity study on partition generation for the aforementioned 18 confidential smart contracts (c.f.~\Cref{tab:gen}) with four LLMs: (1) GPT-4o mini, (2) Gemma2:27b, (3) Llama3.1:8b, and (4) Qwen2.5:32b. 

\Cref{tab:llm_compare} shows the comparison results.
We evaluated LLMs on the overall number of generated partitions, true positives, false positives, and precision.
Recall that we excluded AuctionInstance's partitions in the precision evaluation.
Note \#Partitions also includes partitions beyond function scope of the ground truth.
It is evident that GPT-4o mini substantially outperforms over the rest LLMs. 
GPT-4o mini yields 907 partitions in total, achieving a precision score of 0.78, followed by Qwen2.5:32b having 758 partitions and precision of 0.75.
Llama3.1:8b performs the worst, rendering the least number of partitions and the poorest precision (0.44).
To have an in-depth analysis of such significant difference in the number of resulting partitions,
we studied the capability of different LLMs in generating compilable program partitions for smart contracts, which plays a vital role in our approach.
\Cref{fig:fix_count} depicts the distribution of times that different LLMs need to generate compilable partitions.
\Cref{fig:fix_count} indicates that 83\% partition candidates initially generated by GPT-4o mini are compilable, followed by Qwen2.5:32b having 70\%.
It is surprising that all the partition candidates generated by Llama3.1:8b must be fixed at least once (using \textcircled{5} \textbf{Revise} of \tool).
The main reason, we believe, may be that Llama3.1 has not been trained on Solidity smart contract datasets.
Therefore, we recommend using the closed-source GPT-4o mini for efficiency and the open-source Qwen2.5:32b to facilitate smart contract partitioning with budget consideration. 

%Please add the following packages if necessary:
%\usepackage{booktabs, multirow} % for borders and merged ranges
%\usepackage{soul}% for underlines
%\usepackage{xcolor,colortbl} % for cell colors
%\usepackage{changepage,threeparttable} % for wide tables
%If the table is too wide, replace \begin{table}[!htp]...\end{table} with
%\begin{adjustwidth}{-2.5 cm}{-2.5 cm}\centering\begin{threeparttable}[!htb]...\end{threeparttable}\end{adjustwidth}
% \section{Case Study}
% list bad partition cases/false positives.

% list partitions for real-world smart contracts. 
% Discuss what should be done to enforce partitions within TEE-based confidential computing solutions.
\begin{tcolorbox}[size=title, opacityfill=0.1]
	\textbf{Answer to RQ4}: 
	The selection of base LLMs could affect the performance of \tool. The sensitivity study found that GPT-4o mini generates the largest number of partitions while achieving the highest precision of 0.78, followed by Qwen2.5:32b.
\end{tcolorbox}
\subsection{Threats to Validity}
\paragraph{Lack of ground truth}
We manually selected a set of annotated smart contracts and real-world smart contracts vulnerable to manipulation attacks for evaluating the effectiveness of \tool.
To address this threat,
we systematically crawled well-studied confidential smart contracts in which developers have explicitly labeled sensitive data, as well as victim smart contracts from known real-world attacks with root causes scrutinized by security experts.  

\paragraph{External Validity}
Our findings in program partitioning may not generalize to other large language models. 
To mitigate this threat, we have evaluated \tool with state-of-the-art LLMs developed by four vendors, namely closed-source GPT-4o mini by OpenAI, open-source Llama3.1 by Meta, Qwen2.5 by Alibaba, and Gemma2 by Google.
% Albeit in the future, more powerful LLMs are predicated, we believe that the proposed LLM-driven approach still remains effective to perform secure program partitioning for smart contracts.     