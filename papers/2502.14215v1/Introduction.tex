\section{Introduction}
Smart contracts are script programs deployed and executed on blockchain, facilitating the customization and processing of complicated business logic within transactions.
Most smart contracts are developed in Turing-complete programming languages such as Solidity~\cite{solidity}. 
These smart contracts have empowered a wider range of applications across fungible tokens, non-fungible tokens (NFT), decentralized exchanges, and predication markets on different blockchain platforms including Ethereum~\cite{Ethereum}, BSC~\cite{bsc}, and Solana~\cite{solana}.
However, smart contracts may contain design and implementation flaws, making them vulnerable to different types of security attacks. These include common vulnerabilities, such as integer overflow~\cite{overflow} and reentrancy~\cite{reentrancy}, as well as manipulation vulnerabilities, such as  front-running~\cite{frontrunning}, user-control randomness~\cite{randomness}, and price manipulation~\cite{manipulation}. 
%\wei{maybe better to say common vulnerabilites are studied well and have had many detection methods but the later lack good approach to defend. Current complex attacks resulting huge loss belongs to manipulation vulnerabilites. It can help enhance the connection to the following paragraph. }
% \rain{I do not fully comprehend smart contracts; however, the introduction lists various other issues and then abruptly shifts to manipulation. 
% I think it is difficult to understand how this differs from other vulnerabilities.}
% {While common vulnerabilities have been extensively studied and well addressed, being free of manipulation vulnerabilities remains a big challenge. }
{While common vulnerabilities have been extensively studied and well addressed, defending against manipulation vulnerabilities remains a big challenge. }


Manipulation vulnerabilities are caused by sensitive information leakage because of the inherent transparency feature of blockchains.
For instance, in 2023, Jimbo was attacked due to the manipulation of unprotected price-related internal states called bins, leading to a loss of about eight million dollars~\cite{Jimbo}.
% \rain{Are there any partial examples that illustrate this issue? I believe that smart contracts exist on public chains, so what aspects need to be concealed? There is a lack of clear explanations on this matter.}
% To address manipulation vulnerabilities, we first examine the limitations of prior work in the domain of smart contract security. 
While static analysis~\cite{luu2016making,securify,tikhomirov2018smartcheck,brent2020ethainter,feng2019precise,sun2024gptscan} and dynamic analysis techniques~\cite{2018contractfuzzer,nguyen2020sfuzz,echidna,xie2024defort,wu2023defiranger,kong2023defitainter} have demonstrated impressive capabilities in vulnerability detection, they primarily target implementation bugs rather than addressing the inherent risk of data transparency in blockchains. 
% \rain{implementation bug and misuse patterns are not mentioned in the first paragraph. Is this a type of inherent risk?}
Particularly, Ethainter~\cite{brent2020ethainter} detects information flow problems in smart contracts, but it is limited to only the detection of unrestricted data write due to poor access control.
The price manipulation attacks can be detected by some existing methods~\cite{xie2024defort,wu2023defiranger,kong2023defitainter} through rule-based transaction analysis, but such incomplete rules still leave a room for attackers to adjust their activities to avoid being detected.
Similarly, formal verification approaches~\cite{manticore,mythril,liu2024propertygpt,Certora,hildenbrandt2018kevm} excel at proving correctness properties but are less effective in scenarios where the root cause lies in unintended exposure of sensitive contract states.
Runtime verification techniques~\cite{magazzeni2017validation, rodler2018sereum, li2020securing, chen2024demystifying} offer the potential to halt harmful executions dynamically. However, these methods often require modification of the execution environment or rely on predefined rules that are inadequate for addressing information leakage, as they focus on transaction execution anomalies. Furthermore, program repair solutions~\cite{yu2020smart, so2023smartfix, nguyen2021sguard, gao2024sguard+, zhang2024acfix, tolmach2022property, zhang2020smartshield, jin2021aroc} concentrate on patching common vulnerabilities or fixing implementation errors. They do not account for design-specific issues related to data transparency, leaving contracts vulnerable to manipulation attacks such as \textit{front-running}, \textit{user-controlled randomness}, and \textit{price manipulation}. 
%\wei{from my understanding, the aforementioned approaches are detection methods but \tool is a defend mechanism. maybe better to highlight that we are defending the manipulaiton vulnerabilites instead of detection. Defending is more useful than detection considering the difficulties and limitaions of detecting such vulnerabilites. It can enhance our motivaton.}
{Beyond the aforementioned program analyses, a more practical solution is to defend against manipulation vulnerabilities through privacy-aware software development.}

% The primary research focus has been on smart contract security. 
% Static analysis~\cite{luu2016making,securify,tikhomirov2018smartcheck,brent2020ethainter,feng2019precise,sun2024gptscan}, dynamic analysis~\cite{2018contractfuzzer,nguyen2020sfuzz,echidna,xie2024defort,wu2023defiranger,kong2023defitainter}, and formal verification~\cite{manticore,mythril,liu2024propertygpt,Certora,hildenbrandt2018kevm} have achieved great success in the vulnerability detection.
% To mitigate the effect of smart contract vulnerabilities,
% runtime verification~\cite{magazzeni2017validation, rodler2018sereum, li2020securing, chen2024demystifying} have been introduced to secure smart contract execution on the fly. 
% They monitor the past and current execution of smart contracts and automatically terminate or raise an alarm of abnormal transaction execution if violations arise herein.
% Even though, the existing runtime verification approaches may undermine the decentralized trust because they require the modification of the execution environment.
% Meanwhile, program repair techniques~\cite{yu2020smart, so2023smartfix, nguyen2021sguard, gao2024sguard+, zhang2024acfix, tolmach2022property, zhang2020smartshield} have been proposed to automatically revise vulnerable contract code before contract deployment or even for on-chain smart contracts~\cite{jin2021aroc}, but currently they are limited to cover common vulnerability types and their patches usually address implementation bugs.
% However, all the aforementioned works cannot prevent sensitive information leakage problem, which we believe is pivotal to many real-world manipulation-related attacks including~\textit{front-run}, \textit{user-control randomness}, and \textit{price manipulation}.

% Thus, confidential smart contracts have recently garnered research interest to protect sensitive information of smart contracts~\cite{qi2024sok, li2022sok}.
% The existing approaches either make smart contract always residing in TEE-based secure environments~\cite{cheng2019ekiden, russinovich2019ccf, yan2020confidentiality, xiao2020privacyguard, yuan2018shadoweth, yin2019phala}, or employ zero-knowledge algorithm-based solutions to protect smart contract execution that seal only sensitive operations~\cite{kosba2016hawk, bunz2020zether}.
% On the other hand, the principle of least privilege~\cite{saltzer1975protection} is one of important principles in practice to secure information flow in programs that could significantly minimize attack surface.
% But, in the aforementioned approach, the sensitive operations are intertwined with non-sensitive operations in the same codebase that makes it hard to modularly enforce the security protection and verify the validity accordingly. 
% More importantly, with the maturity of secure infrastructures customized for smart contract execution, applications, e.g., games that are driven by high-quality and secure randomness, have benefited a lot by deploying the codebase of non-sensitive operations on normal blockchains like Ethereum and the codebase of sensitive operations such as randomness generation on TEE-based blockchains like Secrete Network~\cite{scrt}.
% Yet, manually partitioning contract could be time-consuming and error-prone. 

The primary limitation of existing approaches to protecting sensitive information in smart contracts lies in their coarse-grained handling of sensitive operations.
{These approaches can be broadly categorized into hardware-supported and non-hardware-supported solutions.
While the hardware-based solutions~\cite{cheng2019ekiden, russinovich2019ccf, yan2020confidentiality, xiao2020privacyguard, yuan2018shadoweth, yin2019phala} provide a trusted execution environment (TEE) that could hide all the execution and data information when running the entire smart contract, they lack flexibility and increase dependency on specific infrastructures. }
% \rain{I think we can emphasize that running the entire smart contract may lead to performance issues, as the operation of TEEs is resource-constrained.
% }
%\wei{It seems that `` protecting sensitive information'' is not consistent to the previous paragraph ``vulnerability detection''. Maybe better, to think how to orgnize the introduction. The current is - manipulation vulnerabilities -> reason: sensitive information leakage -> detection methods -> protecting sensitive information. }
%\wei{It is better to simply tell what is TEE here for the readers without the related background.}
% \rain{I think we should to position non-hardware-supported solutions before hardware-supported ones, as this would create a better connection with our approach, which also emphasizes hardware support.
% Besides, we also mentioned other TEEs here. 
% I am concerned that the reviewers might have questions regarding this and seek comparative experiments.
% }
Similarly, the non-hardware-supported solutions, e.g., empowered by zero-knowledge proofs~\cite{kosba2016hawk, bunz2020zether}, seal sensitive operations during execution but often come with computational overhead and integration complexity. These methods fail to leverage the principle of least privilege~\cite{saltzer1975protection} effectively, as they intermingle sensitive and non-sensitive operations within a single codebase, making it difficult to enforce modular security or validate protection measures in a scalable manner.
This intertwining of sensitive and non-sensitive operations not only complicates the security enforcement but also restricts developers from optimally utilizing secure infrastructures. For example, while applications like games can benefit from high-quality randomness by isolating sensitive operations on TEE-based blockchains such as Secret Network~\cite{scrt}, manually partitioning the smart contract code is a labor-intensive and error-prone process that introduces potential risks and inefficiencies.

Our work addresses these limitations by introducing an automated and modular framework for smart contract partitioning {that is applicable to mitigate manipulation vulnerabilities caused by sensitive information leakage.}
% \rain{Should we emphasize TEEs?}
%\wei{I still prefer to emphasize \tool is a defense mechanism which is more practical and uselful compared with the detecion approches.} 
By combining program slicing and in-context learning capability of large language models (LLMs), we decouple sensitive operations from non-sensitive ones, enabling developers to deploy sensitive operations on secure infrastructures while maintaining non-sensitive operations on standard blockchains like Ethereum. 
% This not only enforces the principle of least privilege but also simplifies the validation and enforcement of security protections. Moreover, the automated nature of our approach minimizes manual intervention, reducing errors and significantly improving development efficiency, thereby supporting the creation of more secure and robust decentralized applications.
% Our work directly tackles the sensitive information leakage problem by designing a fine-grained program partitioning approach tailored to segregate sensitive and non-sensitive components within smart contracts. By leveraging techniques such as program slicing, dynamic instrumentation, and privilege separation, our method ensures that sensitive state variables are protected and isolated from exposure. This approach fundamentally differs from existing methods by focusing on preventing the exploitation of information transparency rather than merely detecting or patching vulnerabilities post hoc. Through this paradigm shift, we aim to mitigate critical attack vectors and strengthen the overall security of smart contract ecosystems.
% To address this problem, in this paper, we propose to automatically partition smart contracts with LLM's in-context learning.
% Like traditional secure program partitioning~\cite{zdancewic2002secure},
%  relies on user-provided security annotations to identify sensitive operations in the code. However, incomplete or inaccurate annotations can lead to information leakage, posing a significant risk. To address this, 
\tool was designed with two purposes: (1) executing sensitive operations on secure infrastructure prevents the sensitive data to be exposed; (2) proper partitioning minimizes the size of sensitive code and runtime overhead.
Firstly, we utilize taint analysis to automatically trace all sensitive operations associated with user-defined secret data in smart contracts, reducing reliance on manual annotations.
% Coarse-grained partitioning techniques~\cite{brumley2004privtrans, liu2017ptrsplit, wu2013automatically} extract entire functions containing sensitive operations without further subdivision. While straightforward, this approach often results in oversized sensitive partitions due to the inclusion of insensitive dependencies, limiting its practical application. Conversely, fine-grained partitioning methods~\cite{tsai2020civet, liu2019program, zheng2003using} divide sensitive functions into smaller units but require intensive and iterative developer input, making the process laborious and less scalable.
Secondly, our solution combines program slicing with the capabilities of LLMs to automatically 
% \rain{Perhaps a brief description should be added to explain how the LLMs work in partition.}
generate fine-grained program partitions that are both compilable and likely secure. 
% \rain{I am uncertain whether it is appropriate to use "secure." 
% Perhaps "confidentiality" would be more suitable?
% Our objective is to prevent leakage, correct?}
This reduces developer effort while ensuring precise partitioning. 
Additionally, we have developed a dedicated checker to formally verify the functional equivalence between the original and partitioned code, ensuring that the transformed code remains consistent with the intended behavior of the smart contract. 
This approach effectively addresses the limitations of the existing methods by enhancing security, usability, and reliability. 
% \wei{This paragraph reads more like the background or the related work, talking about the limitations of the current partition methods and why we use LLM to partition. But I am not sure if we should hilighlight our motivation, nolvelty and idea for how we think and what idea we use to solve manipulation attacks, what are the insights of our approach.}

We implement our approach in \tool that is able to generate \textit{compilable}, and \textit{verified} program partitions.
% \rain{These three concepts have not been mentioned previously. 
% I think should be purposed at the beginning of above paragraph, specifically outlining the objectives of our tool: to implement compilable, secure?confidentiality?, and functionally equivalent solutions. 
% "Compilable" means xxx...}
We evaluate \tool on 18 annotated confidential smart contracts having 99 sensitive functions in total and nine real-world victim smart contracts of manipulation attacks.
The experimental results show that \tool is able to generate secure program partitions, achieving success rate of 78\%, reducing around 30\% code compared to function-level partitioning that does not split sensitive functions.
Additionally, \tool can be applied to detect eight out of nine manipulation attacks, indicating its applicability for protecting real-world smart contracts. 
We deployed the program partitions within a TEE-based execution environment~\cite{russinovich2019ccf}, and it shows the runtime gas overhead for sensitive functions to partition is moderate, resulting in 61\% to 103\% gas increase for carrying out communications between isolated sensitive code within TEE and normal code on Ethereum.
Moreover, our study also compares the performance of four different LLMs. 
It shows that \emph{GPT-4o mini} used by \tool outperforms the selected three open-source LLMs where \emph{Qwen2.5:32b} is recommended as the alternative LLM for \tool.

We summarize the following main contributions.
\begin{itemize}
    \item To the best that we know, we are the first to propose an LLM-driven framework, \tool, for secure program partitioning, and apply it to smart contracts. \tool combines program slicing with LLM's in-context learning for fine-grained partition generation. 
    % \rain{"Repairing" is mentioned for the first time.}
    \item We devise a dedicated equivalence checker for Solidity smart contracts to formally verify the functional equivalence between the original and partitioned code, ensuring their functionality conformance.
    \item We evaluate~\tool on 18 annotated confidential smart contracts that have 99 sensitive functions. The results show \tool achieved success rate of 78\%, reducing 30\% code compared to function-level partitioning. Moreover, our evaluation also indicates that \tool is able to effectively defend against real-world manipulation attacks, and the runtime overhead of \tool is moderate by deploying the partitions into a TEE-based secure environment.
    \item All the benchmarks, source code, and experimental results are available on \website.
\end{itemize}

\paragraph*{Organization}
The rest of the paper is organized as follows.
%\Cref{sec:background} provides the background and motivates the work.
\Cref{sec:background} provides the necessary background and justifies the motivation for this work.
\Cref{sec:approach} details our fine-grained program partitioning approach, and we discuss the equivalence checking in~\Cref{sec:verification}, followed by the implementation and evaluation in~\Cref{sec:evaluation}.
The related work is discussed in~\Cref{sec:relate}, and \Cref{sec:conclude} concludes the paper and mentions future work. 
