% \documentclass[acmsmall,screen,review]{acmart}
\documentclass[journal]{IEEEtran}



%
\setlength\unitlength{1mm}
\newcommand{\twodots}{\mathinner {\ldotp \ldotp}}
% bb font symbols
\newcommand{\Rho}{\mathrm{P}}
\newcommand{\Tau}{\mathrm{T}}

\newfont{\bbb}{msbm10 scaled 700}
\newcommand{\CCC}{\mbox{\bbb C}}

\newfont{\bb}{msbm10 scaled 1100}
\newcommand{\CC}{\mbox{\bb C}}
\newcommand{\PP}{\mbox{\bb P}}
\newcommand{\RR}{\mbox{\bb R}}
\newcommand{\QQ}{\mbox{\bb Q}}
\newcommand{\ZZ}{\mbox{\bb Z}}
\newcommand{\FF}{\mbox{\bb F}}
\newcommand{\GG}{\mbox{\bb G}}
\newcommand{\EE}{\mbox{\bb E}}
\newcommand{\NN}{\mbox{\bb N}}
\newcommand{\KK}{\mbox{\bb K}}
\newcommand{\HH}{\mbox{\bb H}}
\newcommand{\SSS}{\mbox{\bb S}}
\newcommand{\UU}{\mbox{\bb U}}
\newcommand{\VV}{\mbox{\bb V}}


\newcommand{\yy}{\mathbbm{y}}
\newcommand{\xx}{\mathbbm{x}}
\newcommand{\zz}{\mathbbm{z}}
\newcommand{\sss}{\mathbbm{s}}
\newcommand{\rr}{\mathbbm{r}}
\newcommand{\pp}{\mathbbm{p}}
\newcommand{\qq}{\mathbbm{q}}
\newcommand{\ww}{\mathbbm{w}}
\newcommand{\hh}{\mathbbm{h}}
\newcommand{\vvv}{\mathbbm{v}}

% Vectors

\newcommand{\av}{{\bf a}}
\newcommand{\bv}{{\bf b}}
\newcommand{\cv}{{\bf c}}
\newcommand{\dv}{{\bf d}}
\newcommand{\ev}{{\bf e}}
\newcommand{\fv}{{\bf f}}
\newcommand{\gv}{{\bf g}}
\newcommand{\hv}{{\bf h}}
\newcommand{\iv}{{\bf i}}
\newcommand{\jv}{{\bf j}}
\newcommand{\kv}{{\bf k}}
\newcommand{\lv}{{\bf l}}
\newcommand{\mv}{{\bf m}}
\newcommand{\nv}{{\bf n}}
\newcommand{\ov}{{\bf o}}
\newcommand{\pv}{{\bf p}}
\newcommand{\qv}{{\bf q}}
\newcommand{\rv}{{\bf r}}
\newcommand{\sv}{{\bf s}}
\newcommand{\tv}{{\bf t}}
\newcommand{\uv}{{\bf u}}
\newcommand{\wv}{{\bf w}}
\newcommand{\vv}{{\bf v}}
\newcommand{\xv}{{\bf x}}
\newcommand{\yv}{{\bf y}}
\newcommand{\zv}{{\bf z}}
\newcommand{\zerov}{{\bf 0}}
\newcommand{\onev}{{\bf 1}}

% Matrices

\newcommand{\Am}{{\bf A}}
\newcommand{\Bm}{{\bf B}}
\newcommand{\Cm}{{\bf C}}
\newcommand{\Dm}{{\bf D}}
\newcommand{\Em}{{\bf E}}
\newcommand{\Fm}{{\bf F}}
\newcommand{\Gm}{{\bf G}}
\newcommand{\Hm}{{\bf H}}
\newcommand{\Id}{{\bf I}}
\newcommand{\Jm}{{\bf J}}
\newcommand{\Km}{{\bf K}}
\newcommand{\Lm}{{\bf L}}
\newcommand{\Mm}{{\bf M}}
\newcommand{\Nm}{{\bf N}}
\newcommand{\Om}{{\bf O}}
\newcommand{\Pm}{{\bf P}}
\newcommand{\Qm}{{\bf Q}}
\newcommand{\Rm}{{\bf R}}
\newcommand{\Sm}{{\bf S}}
\newcommand{\Tm}{{\bf T}}
\newcommand{\Um}{{\bf U}}
\newcommand{\Wm}{{\bf W}}
\newcommand{\Vm}{{\bf V}}
\newcommand{\Xm}{{\bf X}}
\newcommand{\Ym}{{\bf Y}}
\newcommand{\Zm}{{\bf Z}}

% Calligraphic

\newcommand{\Ac}{{\cal A}}
\newcommand{\Bc}{{\cal B}}
\newcommand{\Cc}{{\cal C}}
\newcommand{\Dc}{{\cal D}}
\newcommand{\Ec}{{\cal E}}
\newcommand{\Fc}{{\cal F}}
\newcommand{\Gc}{{\cal G}}
\newcommand{\Hc}{{\cal H}}
\newcommand{\Ic}{{\cal I}}
\newcommand{\Jc}{{\cal J}}
\newcommand{\Kc}{{\cal K}}
\newcommand{\Lc}{{\cal L}}
\newcommand{\Mc}{{\cal M}}
\newcommand{\Nc}{{\cal N}}
\newcommand{\nc}{{\cal n}}
\newcommand{\Oc}{{\cal O}}
\newcommand{\Pc}{{\cal P}}
\newcommand{\Qc}{{\cal Q}}
\newcommand{\Rc}{{\cal R}}
\newcommand{\Sc}{{\cal S}}
\newcommand{\Tc}{{\cal T}}
\newcommand{\Uc}{{\cal U}}
\newcommand{\Wc}{{\cal W}}
\newcommand{\Vc}{{\cal V}}
\newcommand{\Xc}{{\cal X}}
\newcommand{\Yc}{{\cal Y}}
\newcommand{\Zc}{{\cal Z}}

% Bold greek letters

\newcommand{\alphav}{\hbox{\boldmath$\alpha$}}
\newcommand{\betav}{\hbox{\boldmath$\beta$}}
\newcommand{\gammav}{\hbox{\boldmath$\gamma$}}
\newcommand{\deltav}{\hbox{\boldmath$\delta$}}
\newcommand{\etav}{\hbox{\boldmath$\eta$}}
\newcommand{\lambdav}{\hbox{\boldmath$\lambda$}}
\newcommand{\epsilonv}{\hbox{\boldmath$\epsilon$}}
\newcommand{\nuv}{\hbox{\boldmath$\nu$}}
\newcommand{\muv}{\hbox{\boldmath$\mu$}}
\newcommand{\zetav}{\hbox{\boldmath$\zeta$}}
\newcommand{\phiv}{\hbox{\boldmath$\phi$}}
\newcommand{\psiv}{\hbox{\boldmath$\psi$}}
\newcommand{\thetav}{\hbox{\boldmath$\theta$}}
\newcommand{\tauv}{\hbox{\boldmath$\tau$}}
\newcommand{\omegav}{\hbox{\boldmath$\omega$}}
\newcommand{\xiv}{\hbox{\boldmath$\xi$}}
\newcommand{\sigmav}{\hbox{\boldmath$\sigma$}}
\newcommand{\piv}{\hbox{\boldmath$\pi$}}
\newcommand{\rhov}{\hbox{\boldmath$\rho$}}
\newcommand{\upsilonv}{\hbox{\boldmath$\upsilon$}}

\newcommand{\Gammam}{\hbox{\boldmath$\Gamma$}}
\newcommand{\Lambdam}{\hbox{\boldmath$\Lambda$}}
\newcommand{\Deltam}{\hbox{\boldmath$\Delta$}}
\newcommand{\Sigmam}{\hbox{\boldmath$\Sigma$}}
\newcommand{\Phim}{\hbox{\boldmath$\Phi$}}
\newcommand{\Pim}{\hbox{\boldmath$\Pi$}}
\newcommand{\Psim}{\hbox{\boldmath$\Psi$}}
\newcommand{\Thetam}{\hbox{\boldmath$\Theta$}}
\newcommand{\Omegam}{\hbox{\boldmath$\Omega$}}
\newcommand{\Xim}{\hbox{\boldmath$\Xi$}}


% Sans Serif small case

\newcommand{\Gsf}{{\sf G}}

\newcommand{\asf}{{\sf a}}
\newcommand{\bsf}{{\sf b}}
\newcommand{\csf}{{\sf c}}
\newcommand{\dsf}{{\sf d}}
\newcommand{\esf}{{\sf e}}
\newcommand{\fsf}{{\sf f}}
\newcommand{\gsf}{{\sf g}}
\newcommand{\hsf}{{\sf h}}
\newcommand{\isf}{{\sf i}}
\newcommand{\jsf}{{\sf j}}
\newcommand{\ksf}{{\sf k}}
\newcommand{\lsf}{{\sf l}}
\newcommand{\msf}{{\sf m}}
\newcommand{\nsf}{{\sf n}}
\newcommand{\osf}{{\sf o}}
\newcommand{\psf}{{\sf p}}
\newcommand{\qsf}{{\sf q}}
\newcommand{\rsf}{{\sf r}}
\newcommand{\ssf}{{\sf s}}
\newcommand{\tsf}{{\sf t}}
\newcommand{\usf}{{\sf u}}
\newcommand{\wsf}{{\sf w}}
\newcommand{\vsf}{{\sf v}}
\newcommand{\xsf}{{\sf x}}
\newcommand{\ysf}{{\sf y}}
\newcommand{\zsf}{{\sf z}}


% mixed symbols

\newcommand{\sinc}{{\hbox{sinc}}}
\newcommand{\diag}{{\hbox{diag}}}
\renewcommand{\det}{{\hbox{det}}}
\newcommand{\trace}{{\hbox{tr}}}
\newcommand{\sign}{{\hbox{sign}}}
\renewcommand{\arg}{{\hbox{arg}}}
\newcommand{\var}{{\hbox{var}}}
\newcommand{\cov}{{\hbox{cov}}}
\newcommand{\Ei}{{\rm E}_{\rm i}}
\renewcommand{\Re}{{\rm Re}}
\renewcommand{\Im}{{\rm Im}}
\newcommand{\eqdef}{\stackrel{\Delta}{=}}
\newcommand{\defines}{{\,\,\stackrel{\scriptscriptstyle \bigtriangleup}{=}\,\,}}
\newcommand{\<}{\left\langle}
\renewcommand{\>}{\right\rangle}
\newcommand{\herm}{{\sf H}}
\newcommand{\trasp}{{\sf T}}
\newcommand{\transp}{{\sf T}}
\renewcommand{\vec}{{\rm vec}}
\newcommand{\Psf}{{\sf P}}
\newcommand{\SINR}{{\sf SINR}}
\newcommand{\SNR}{{\sf SNR}}
\newcommand{\MMSE}{{\sf MMSE}}
\newcommand{\REF}{{\RED [REF]}}

% Markov chain
\usepackage{stmaryrd} % for \mkv 
\newcommand{\mkv}{-\!\!\!\!\minuso\!\!\!\!-}

% Colors

\newcommand{\RED}{\color[rgb]{1.00,0.10,0.10}}
\newcommand{\BLUE}{\color[rgb]{0,0,0.90}}
\newcommand{\GREEN}{\color[rgb]{0,0.80,0.20}}

%%%%%%%%%%%%%%%%%%%%%%%%%%%%%%%%%%%%%%%%%%
\usepackage{hyperref}
\hypersetup{
    bookmarks=true,         % show bookmarks bar?
    unicode=false,          % non-Latin characters in AcrobatÕs bookmarks
    pdftoolbar=true,        % show AcrobatÕs toolbar?
    pdfmenubar=true,        % show AcrobatÕs menu?
    pdffitwindow=false,     % window fit to page when opened
    pdfstartview={FitH},    % fits the width of the page to the window
%    pdftitle={My title},    % title
%    pdfauthor={Author},     % author
%    pdfsubject={Subject},   % subject of the document
%    pdfcreator={Creator},   % creator of the document
%    pdfproducer={Producer}, % producer of the document
%    pdfkeywords={keyword1} {key2} {key3}, % list of keywords
    pdfnewwindow=true,      % links in new window
    colorlinks=true,       % false: boxed links; true: colored links
    linkcolor=red,          % color of internal links (change box color with linkbordercolor)
    citecolor=green,        % color of links to bibliography
    filecolor=blue,      % color of file links
    urlcolor=blue           % color of external links
}
%%%%%%%%%%%%%%%%%%%%%%%%%%%%%%%%%%%%%%%%%%%




%%
%% end of the preamble, start of the body of the document source.
\begin{document}

%%
%% The "title" command has an optional parameter,
%% allowing the author to define a "short title" to be used in page headers.
\title{Towards Secure Program Partitioning for Smart Contracts with LLM's In-Context Learning}

\author{
\IEEEauthorblockN{
  Ye Liu$^1$, 
  Yuqing Niu$^1$, 
  Chengyan Ma$^1$,
  Ruidong Han$^1$, 
  Wei Ma$^1$, 
  Yi Li$^2$, 
  Debin Gao$^1$, 
  and David Lo$^1$,~\IEEEmembership{Fellow,~IEEE}}\\
\IEEEauthorblockA{
  $^1$Singapore Management University\\
  $^2$Nanyang Technological University\\
}
}

% \author{Ye Liu,
%         Yuqing Niu,
%         Chengyan Ma~\textsuperscript{\orcidlink{0000-0001-9256-6930}},
%         Ruidong Han~\textsuperscript{\orcidlink{0000-0001-6859-60057}},
%         Wei Ma,
%         Yi Li,
%         Debin Gao,
%         David Lo~\textsuperscript{\orcidlink{0000-0002-4367-7201}},~\IEEEmembership{Fellow,~IEEE},
% \thanks{
%           Ye Liu, Yuqing Niu, Chengyan Ma, Ruidong Han, Wei Ma, and David Lo are with the School of Computing and Information Systems, Singapore Management University(e-mail:xxxEMAIL).
%           Yi Li is with the School of Computer Science and Engineering, Nanyang Technological University, Singapore.
%           Debin Gao is with the School of Computer Science and Engineering, Beihang University, China.
%         }
% }



% % First author
% \author{Ye Liu}
% \affiliation{%
%   \institution{Singapore Management University}
%   \country{Singapore}
% }
% \email{yeliu@smu.edu.sg}

% % Second author
% \author{Yuqing Niu}
% \affiliation{%
%   \institution{Singapore Management University}
%   \country{Singapore}
% }
% \email{yuqingniu@smu.edu.sg}

% % Third author
% \author{Chengyan Ma}
% \affiliation{%
%   \institution{Singapore Management University}
%   \country{Singapore}
% }
% \email{chengyanma@smu.edu.sg}

% \author{Ruidong Han}
% \affiliation{%
%   \institution{Singapore Management University}
%   \country{Singapore}
% }
% \email{rdhan@smu.edu.sg}


% \author{Wei Ma}
% \affiliation{%
%   \institution{Singapore Management University}
%   \country{Singapore}
% }
% \email{weima@smu.edu.sg}

% % \author{Juantao Zhong}
% % \affiliation{%
% %   \institution{The Hong Kong University of Science and Technology}
% %   % \city{Shenzhen}
% %   \country{Hong Kong}
% % }
% % \email{jzhong012@e.ntu.edu.sg}

% \author{Yi Li}
% \affiliation{%
%   \institution{Nanyang Technological University}
%   \country{Singapore}
% }
% \email{yi_li@ntu.edu.sg}


% \author{David Lo}
% \affiliation{%
%   \institution{Singapore Management University}
%   \country{Singapore}
% }
% \email{davidlo@smu.edu.sg}


%%
%% By default, the full list of authors will be used in the page
%% headers. Often, this list is too long, and will overlap
%% other information printed in the page headers. This command allows
%% the author to define a more concise list
%% of authors' names for this purpose.

%% This command processes the author and affiliation and title
%% information and builds the first part of the formatted document.

%%
%% The abstract is a short summary of the work to be presented in the
%% article.
% Smart contracts are highly susceptible to manipulation attacks due to the leakage of sensitive information. Addressing such vulnerabilities is particularly challenging because they stem from inherent data confidentiality issues rather than straightforward implementation bugs. 
% To tackle this, we present \tool, a novel approach that combines static analysis with the in-context learning capabilities of large language models (LLMs) to partition smart contracts into privileged and normal codebases, guided by a few annotated \secrete variables.
% We evaluated \tool on 18 annotated smart contracts containing 99 sensitive functions. 
% The results demonstrate that \tool successfully generates \emph{compilable}, \emph{secure}, and \emph{functionally equivalent} partitions for 78\% of the subject functions with high accuracy. Additionally, \tool reduces the size of the trusted codebase by approximately 30\% compared to traditional function-level partitioning. 
% In testing on nine real-world victim contracts affected by manipulation attacks, \tool effectively identified eight cases, highlighting its potential for broad applicability and the emerging need for secure program partitioning in smart contract development.

\maketitle
\begin{abstract}
Smart contracts are highly susceptible to manipulation attacks due to the leakage of sensitive information. Addressing manipulation vulnerabilities is particularly challenging because they stem from inherent data confidentiality issues rather than straightforward implementation bugs. 
To tackle this by preventing sensitive information leakage, we present \tool, the first LLM-driven approach that combines static analysis with the in-context learning capabilities of large language models (LLMs) to partition smart contracts into privileged and normal codebases, guided by a few annotated \secrete variables.
We evaluated \tool on 18 annotated smart contracts containing 99 sensitive functions. 
The results demonstrate that \tool successfully generates \emph{compilable}, and \emph{verified} partitions for 78\% of the sensitive functions while reducing approximately 30\% code compared to function-level partitioning approach. 
Furthermore, we evaluated \tool on nine real-world manipulation attacks that lead to a total loss of 25 million dollars, \tool effectively prevents eight cases, highlighting its potential for broad applicability and the necessity for secure program partitioning during smart contract development to diminish manipulation vulnerabilities.
% Smart contracts are vulnerable to manipulation attacks due to sensitive information leakage.
% Prevention of manipulation vulnerabilities from smart contracts are challenging because they are inherent data confidentiality issues rather than easy-to-fix implementation bugs.
% To preserve confidentiality, in this paper, we propose \tool to accept a few annotated \secrete variables and combine static analysis with LLM's in-context learning to partition smart contracts into privileged and normal codebases.
% We evaluate \tool on 18 annotated smart contracts with 99 sensitive functions.
% The evaluation indicates that \tool could successfully generate \emph{compilable}, \emph{secure}, and \emph{functionally equivalent} partitions for 78\% subject functions with a reasonably high accuracy, resulting in a decrease of around 30\% codes in the trusted codebase compared to the function-level partitioning.
% Of nine real-world victim smart contracts suffering manipulation attacks, \tool can be applied to identify eight cases, indicating the potential wide applicability and emerging need for secure program partitioning.
\end{abstract}


\section{Introduction}

Large language models (LLMs) have achieved remarkable success in automated math problem solving, particularly through code-generation capabilities integrated with proof assistants~\citep{lean,isabelle,POT,autoformalization,MATH}. Although LLMs excel at generating solution steps and correct answers in algebra and calculus~\citep{math_solving}, their unimodal nature limits performance in plane geometry, where solution depends on both diagram and text~\citep{math_solving}. 

Specialized vision-language models (VLMs) have accordingly been developed for plane geometry problem solving (PGPS)~\citep{geoqa,unigeo,intergps,pgps,GOLD,LANS,geox}. Yet, it remains unclear whether these models genuinely leverage diagrams or rely almost exclusively on textual features. This ambiguity arises because existing PGPS datasets typically embed sufficient geometric details within problem statements, potentially making the vision encoder unnecessary~\citep{GOLD}. \cref{fig:pgps_examples} illustrates example questions from GeoQA and PGPS9K, where solutions can be derived without referencing the diagrams.

\begin{figure}
    \centering
    \begin{subfigure}[t]{.49\linewidth}
        \centering
        \includegraphics[width=\linewidth]{latex/figures/images/geoqa_example.pdf}
        \caption{GeoQA}
        \label{fig:geoqa_example}
    \end{subfigure}
    \begin{subfigure}[t]{.48\linewidth}
        \centering
        \includegraphics[width=\linewidth]{latex/figures/images/pgps_example.pdf}
        \caption{PGPS9K}
        \label{fig:pgps9k_example}
    \end{subfigure}
    \caption{
    Examples of diagram-caption pairs and their solution steps written in formal languages from GeoQA and PGPS9k datasets. In the problem description, the visual geometric premises and numerical variables are highlighted in green and red, respectively. A significant difference in the style of the diagram and formal language can be observable. %, along with the differences in formal languages supported by the corresponding datasets.
    \label{fig:pgps_examples}
    }
\end{figure}



We propose a new benchmark created via a synthetic data engine, which systematically evaluates the ability of VLM vision encoders to recognize geometric premises. Our empirical findings reveal that previously suggested self-supervised learning (SSL) approaches, e.g., vector quantized variataional auto-encoder (VQ-VAE)~\citep{unimath} and masked auto-encoder (MAE)~\citep{scagps,geox}, and widely adopted encoders, e.g., OpenCLIP~\citep{clip} and DinoV2~\citep{dinov2}, struggle to detect geometric features such as perpendicularity and degrees. 

To this end, we propose \geoclip{}, a model pre-trained on a large corpus of synthetic diagram–caption pairs. By varying diagram styles (e.g., color, font size, resolution, line width), \geoclip{} learns robust geometric representations and outperforms prior SSL-based methods on our benchmark. Building on \geoclip{}, we introduce a few-shot domain adaptation technique that efficiently transfers the recognition ability to real-world diagrams. We further combine this domain-adapted GeoCLIP with an LLM, forming a domain-agnostic VLM for solving PGPS tasks in MathVerse~\citep{mathverse}. 
%To accommodate diverse diagram styles and solution formats, we unify the solution program languages across multiple PGPS datasets, ensuring comprehensive evaluation. 

In our experiments on MathVerse~\citep{mathverse}, which encompasses diverse plane geometry tasks and diagram styles, our VLM with a domain-adapted \geoclip{} consistently outperforms both task-specific PGPS models and generalist VLMs. 
% In particular, it achieves higher accuracy on tasks requiring geometric-feature recognition, even when critical numerical measurements are moved from text to diagrams. 
Ablation studies confirm the effectiveness of our domain adaptation strategy, showing improvements in optical character recognition (OCR)-based tasks and robust diagram embeddings across different styles. 
% By unifying the solution program languages of existing datasets and incorporating OCR capability, we enable a single VLM, named \geovlm{}, to handle a broad class of plane geometry problems.

% Contributions
We summarize the contributions as follows:
We propose a novel benchmark for systematically assessing how well vision encoders recognize geometric premises in plane geometry diagrams~(\cref{sec:visual_feature}); We introduce \geoclip{}, a vision encoder capable of accurately detecting visual geometric premises~(\cref{sec:geoclip}), and a few-shot domain adaptation technique that efficiently transfers this capability across different diagram styles (\cref{sec:domain_adaptation});
We show that our VLM, incorporating domain-adapted GeoCLIP, surpasses existing specialized PGPS VLMs and generalist VLMs on the MathVerse benchmark~(\cref{sec:experiments}) and effectively interprets diverse diagram styles~(\cref{sec:abl}).

\iffalse
\begin{itemize}
    \item We propose a novel benchmark for systematically assessing how well vision encoders recognize geometric premises, e.g., perpendicularity and angle measures, in plane geometry diagrams.
	\item We introduce \geoclip{}, a vision encoder capable of accurately detecting visual geometric premises, and a few-shot domain adaptation technique that efficiently transfers this capability across different diagram styles.
	\item We show that our final VLM, incorporating GeoCLIP-DA, effectively interprets diverse diagram styles and achieves state-of-the-art performance on the MathVerse benchmark, surpassing existing specialized PGPS models and generalist VLM models.
\end{itemize}
\fi

\iffalse

Large language models (LLMs) have made significant strides in automated math word problem solving. In particular, their code-generation capabilities combined with proof assistants~\citep{lean,isabelle} help minimize computational errors~\citep{POT}, improve solution precision~\citep{autoformalization}, and offer rigorous feedback and evaluation~\citep{MATH}. Although LLMs excel in generating solution steps and correct answers for algebra and calculus~\citep{math_solving}, their uni-modal nature limits performance in domains like plane geometry, where both diagrams and text are vital.

Plane geometry problem solving (PGPS) tasks typically include diagrams and textual descriptions, requiring solvers to interpret premises from both sources. To facilitate automated solutions for these problems, several studies have introduced formal languages tailored for plane geometry to represent solution steps as a program with training datasets composed of diagrams, textual descriptions, and solution programs~\citep{geoqa,unigeo,intergps,pgps}. Building on these datasets, a number of PGPS specialized vision-language models (VLMs) have been developed so far~\citep{GOLD, LANS, geox}.

Most existing VLMs, however, fail to use diagrams when solving geometry problems. Well-known PGPS datasets such as GeoQA~\citep{geoqa}, UniGeo~\citep{unigeo}, and PGPS9K~\citep{pgps}, can be solved without accessing diagrams, as their problem descriptions often contain all geometric information. \cref{fig:pgps_examples} shows an example from GeoQA and PGPS9K datasets, where one can deduce the solution steps without knowing the diagrams. 
As a result, models trained on these datasets rely almost exclusively on textual information, leaving the vision encoder under-utilized~\citep{GOLD}. 
Consequently, the VLMs trained on these datasets cannot solve the plane geometry problem when necessary geometric properties or relations are excluded from the problem statement.

Some studies seek to enhance the recognition of geometric premises from a diagram by directly predicting the premises from the diagram~\citep{GOLD, intergps} or as an auxiliary task for vision encoders~\citep{geoqa,geoqa-plus}. However, these approaches remain highly domain-specific because the labels for training are difficult to obtain, thus limiting generalization across different domains. While self-supervised learning (SSL) methods that depend exclusively on geometric diagrams, e.g., vector quantized variational auto-encoder (VQ-VAE)~\citep{unimath} and masked auto-encoder (MAE)~\citep{scagps,geox}, have also been explored, the effectiveness of the SSL approaches on recognizing geometric features has not been thoroughly investigated.

We introduce a benchmark constructed with a synthetic data engine to evaluate the effectiveness of SSL approaches in recognizing geometric premises from diagrams. Our empirical results with the proposed benchmark show that the vision encoders trained with SSL methods fail to capture visual \geofeat{}s such as perpendicularity between two lines and angle measure.
Furthermore, we find that the pre-trained vision encoders often used in general-purpose VLMs, e.g., OpenCLIP~\citep{clip} and DinoV2~\citep{dinov2}, fail to recognize geometric premises from diagrams.

To improve the vision encoder for PGPS, we propose \geoclip{}, a model trained with a massive amount of diagram-caption pairs.
Since the amount of diagram-caption pairs in existing benchmarks is often limited, we develop a plane diagram generator that can randomly sample plane geometry problems with the help of existing proof assistant~\citep{alphageometry}.
To make \geoclip{} robust against different styles, we vary the visual properties of diagrams, such as color, font size, resolution, and line width.
We show that \geoclip{} performs better than the other SSL approaches and commonly used vision encoders on the newly proposed benchmark.

Another major challenge in PGPS is developing a domain-agnostic VLM capable of handling multiple PGPS benchmarks. As shown in \cref{fig:pgps_examples}, the main difficulties arise from variations in diagram styles. 
To address the issue, we propose a few-shot domain adaptation technique for \geoclip{} which transfers its visual \geofeat{} perception from the synthetic diagrams to the real-world diagrams efficiently. 

We study the efficacy of the domain adapted \geoclip{} on PGPS when equipped with the language model. To be specific, we compare the VLM with the previous PGPS models on MathVerse~\citep{mathverse}, which is designed to evaluate both the PGPS and visual \geofeat{} perception performance on various domains.
While previous PGPS models are inapplicable to certain types of MathVerse problems, we modify the prediction target and unify the solution program languages of the existing PGPS training data to make our VLM applicable to all types of MathVerse problems.
Results on MathVerse demonstrate that our VLM more effectively integrates diagrammatic information and remains robust under conditions of various diagram styles.

\begin{itemize}
    \item We propose a benchmark to measure the visual \geofeat{} recognition performance of different vision encoders.
    % \item \sh{We introduce geometric CLIP (\geoclip{} and train the VLM equipped with \geoclip{} to predict both solution steps and the numerical measurements of the problem.}
    \item We introduce \geoclip{}, a vision encoder which can accurately recognize visual \geofeat{}s and a few-shot domain adaptation technique which can transfer such ability to different domains efficiently. 
    % \item \sh{We develop our final PGPS model, \geovlm{}, by adapting \geoclip{} to different domains and training with unified languages of solution program data.}
    % We develop a domain-agnostic VLM, namely \geovlm{}, by applying a simple yet effective domain adaptation method to \geoclip{} and training on the refined training data.
    \item We demonstrate our VLM equipped with GeoCLIP-DA effectively interprets diverse diagram styles, achieving superior performance on MathVerse compared to the existing PGPS models.
\end{itemize}

\fi 

\section{Related Work}\label{sec:related_works}
\gls{bp} estimation from \gls{ecg} and \gls{ppg} waveforms has received significant attention due to its potential for continuous, unobtrusive monitoring. Earlier work relied on classical machine learning with handcrafted features, but deep learning methods have since emerged as more robust alternatives. Convolutional or recurrent architectures designed for \gls{ecg}/\gls{ppg} have shown strong performance, including ResUNet with self-attention~\cite{Jamil}, U-Net variants~\cite{Mahmud_2022}, and hybrid \gls{cnn}--\gls{rnn} models~\cite{Paviglianiti2021ACO}. These architectures often outperform traditional feature-engineering approaches, particularly when both \gls{ecg} and \gls{ppg} signals are used~\cite{Paviglianiti2021ACO}.

Nevertheless, many existing methods train solely on \gls{ecg}/\gls{ppg} data, which, while plentiful~\cite{mimiciii,vitaldb,ptb-xl}, often exhibit significant variability in signal quality and patient-specific characteristics. This variability poses challenges for achieving robust generalization across populations. Recent work has explored transfer learning to overcome these issues; for example, Yang \emph{et~al.}~\cite{yang2023cross} studied the transfer of \gls{eeg} knowledge to \gls{ecg} classification tasks, achieving improved performance and reduced training costs. Joshi \emph{et~al.}~\cite{joshi2021deep} also explored the transfer of \gls{eeg} knowledge using a deep knowledge distillation framework to enhance single-lead \gls{ecg}-based sleep staging. However, these studies have largely focused on within-modality or narrow domain adaptations, leaving open the broader question of whether an \gls{eeg}-based foundation model can serve as a versatile starting point for generalized biosignal analysis.

\gls{eeg} has become an attractive candidate for pre-training large models not only because of the availability of large-scale \gls{eeg} repositories~\cite{TUEG} but also due to its rich multi-channel, temporal, and spectral dynamics~\cite{jiang2024large}. While many time-series modalities (for example, voice) also exhibit rich temporal structure, \gls{eeg}, \gls{ecg}, and \gls{ppg} share common physiological origins and similar noise characteristics, which facilitate the transfer of temporal pattern recognition capabilities. In other words, our hypothesis is that the underlying statistical properties and multi-dimensional dynamics in \gls{eeg} make it particularly well-suited for learning robust representations that can be effectively adapted to \gls{ecg}/\gls{ppg} tasks. Our work is the first to validate the feasibility of fine-tuning a transformer-based model initially trained on EEG (CEReBrO~\cite{CEReBrO}) for arterial \gls{bp} estimation using \gls{ecg} and \gls{ppg} data.

Beyond accuracy, real-world deployment of \gls{bp} estimation models calls for efficient inference. Traditional deep networks can be computationally expensive, motivating recent interest in quantization and other compression techniques~\cite{nagel2021whitepaperneuralnetwork}. Few studies have combined large-scale pre-training with post-training quantization for \gls{bp} monitoring. Hence, our method integrates these two aspects: leveraging a potent \gls{eeg}-based foundation model and applying quantization for a compact, high-accuracy cuffless \gls{bp} solution.
\section{Motivation}
\begin{figure*}
    \centering
    \setlength{\abovecaptionskip}{0cm}
    \includegraphics[width=1.0\linewidth]{Figure/motivation_sample.pdf}
    \caption{(a), (b), and (c) show real code snippets from an early Airpush version, a later Airpush version, and the Hiddad adware family. Airpush's core behavior includes: (1) get ad data from a specific URL, (2) asynchronous execution to avoid user interruptio, and (3) push ads continuously through the background service. (a) and (b) demonstrate that both Airpush versions share invariant behaviors, with similar API calls and permissions despite implementation differences. Hiddad, while skipping step (2) for simpler ad display, shares steps (1) and (3) with Airpush, especially the newer version.}
    \label{fig:motivation_sample}
\end{figure*}

\subsection{Invariance in Malware Evolution}
% Malware families commonly evolve to circumvent new detection techniques and security measures, resulting in constant changes in their code implementations or API calls. These changes cause the feature space, extracted from applications, to gradually deviate from the initial decision boundaries of malware detectors, thereby significantly degrading detection performance. However, we have identified that during the evolution of malware, core malicious behaviors and execution logic exhibit a certain degree of invariance. This invariance is reflected in the training set through specific intents, permissions, and function calls, which are captured by feature extraction techniques. We categorize this invariance into two types: intra-family invariance and inter-family invariance. 
Malware families commonly evolve to circumvent new detection techniques and security measures, resulting in constant changes in their code implementations or API calls to bypass detection. This leads to drifts in the feature space and a decline in detection accuracy. Yet, we argue that during the evolution of malware, core malicious behaviors and execution patterns remain partially invariant, captured in training data through intents, permissions, and function calls. We define these invariances as intra- and inter-family invariance.
\begin{itemize}
    \item Intra-family invariance: While versions within a malware family may vary in implementation, their core malicious intent remains relatively stable.
    \item Inter-family invariance: Certain malicious behavior patterns are consistent across different malware families. As malware trends shift, even new families emerging after detector training may share malicious intents with families in the training set.
\end{itemize}
To illustrate invariant malicious behaviors in drift scenarios, we select APKs from Androzoo\footnote{https://androzoo.uni.lu} and decompile them using JADX\footnote{https://github.com/skylot/jadx} to obtain .java files. Our analysis focuses on core malicious behaviors in the source code. For intra-family invariance, we use versions of the Airpush family, known for intrusive ad delivery, from different periods. For inter-family invariance, we examine the Hiddad family, which shares aggressive ad delivery and tracking tactics but uses broader permissions, increasing privacy risks. Figure.\ref{fig:motivation_sample} shows code snippets with colored boxes highlighting invariant behaviors across samples. While Airpush uses asynchronous task requests, Hiddad relies on background services and scheduled tasks to evade detection.

% To highlight the invariance of malicious behaviors in drift scenarios, we selected real APK files from Androzoo~\footnote{https://androzoo.uni.lu} platform and decompiled them using the jadx tool~\footnote{https://github.com/skylot/jadx} to obtain their corresponding .java files. The invariance analysis is conducted on the core malicious behaviors represented in the source code. For intra-family invariance, we used the long-standing Airpush malware family as an example, selecting versions from different periods, which is known for its intrusive ad delivery. For inter-family invariance, we chose the Hiddad adware family, which emerged later. Airpush and Hiddad rely on aggressive ad delivery and user tracking, but Hiddad uses broader permissions, violating more privacy. Figure.\ref{fig:motivation_sample} presents code snippets from these malware families, with colored boxes highlighting the invariant malicious behaviors shared across different family samples. Since Hiddad prioritizes ad delivery via background services and scheduled tasks to avoid suspicion and detection, it omits the asynchronous task requests used in step two by Airpush.

Figure~\ref{fig:motivation_sample}(a)\footnote{MD5: 17950748f9d37bed2f660daa7a6e7439} and (b)\footnote{MD5: ccc833ad11c7c648d1ba4538fe5c0445} show core code from this family in 2014 and later years, respectively. The 2014 version uses \verb|NotifyService| and \verb|TimerTask| to notify users every 24 hours, maintaining ad exposure. The later version, adapting to Android 8.0’s restrictions, triggers \verb|NotifyService| via \verb|BroadcastReceiver| with \verb|WAKE_LOCK| to sustain background activity. In Drebin’s~\cite{Arpdrebin} feature space, these invariant behaviors are captured through features like \verb|android_app_NotificationManager;notify|, \verb|permission_READ_PHONE_STATE| and so on. Both implementations also use \verb|HttpURLConnection| for remote communication, asynchronously downloading ads and tracking user activity, and sharing Drebin features such as \verb|java/net/HttpURLConnection| and \verb|android_permission_INTERNET|.

Similarly, Figure.~\ref{fig:motivation_sample}(c)\footnote{MD5: 84573e568185e25c1916f8fc575a5222} shows a real sample from the Hiddad family, which uses HTTP connections for ad delivery, along with \verb|AnalyticsServer| and \verb|WAKE_LOCK| for continuous background services. Permissions like \verb|android_permission_WAKE_LOCK| and API calls such as \verb|getSystemService| reflect shared, cross-family invariant behaviors, whose learning would enhance model detection across variants.

Capturing the core malicious behaviors of Airpush aids in detecting both new Airpush variants and the Hiddad family, as they share similar malicious intents. These stable behaviors form consistent indicators in the feature space. However, detectors with high validation performance often fail to adapt to such variants, underscoring the need to investigate root causes and develop a drift-robust malware detector.

% Based on this analysis, learning features that represent the core malicious behaviors of the Airpush family not only aids in detecting new Airpush variants but also helps in identifying the Hiddad family. However, malware detectors trained on historical data often fail to effectively detect these malicious samples. Even if the model achieves near-perfect performance on the validation set, its performance deteriorates significantly over time, prompting us to further investigate the root causes and build a drift-robust malware detector.

\begin{center}
\fcolorbox{black}{gray!10}{\parbox{.9\linewidth}{\textit{\textbf{Take Away}: The feature space of training samples contains invariance within and among malware families to be learned.}}}
\end{center}

% The analysis suggests that learning invariant features from the Airpush family not only aids in detecting newer versions of Airpush but also improves the detection of Hiddad. However, in practice, detectors trained on these features fail to effectively identify such samples, with performance degrading over time despite near-perfect validation set results. This motivates further our exploration of an ideal malware detector that can learn these invariant behaviours and remain robust in its discrimination throughout malware evolution.

% We present several representative pseudo-code implementations of malicious software to demonstrate invariance visually. For intra-family invariance, we selected Rootkit, a common malware family present in both the training and testing phases. This family exploits system vulnerabilities to obtain root privileges, enabling high-privilege operations such as modifying system files or the kernel, and intercepting system calls to mask its behavior, showcasing a deep confrontation with the operating system’s security mechanisms. Rootkit execution can be simplified into three main steps: file hiding, root privilege acquisition, and system call hooking. With the introduction of stricter permission management and security measures in Android 6.0, this family had to rely on more complex kernel-level attacks to maintain stealth. Figure X illustrates two simplified executions of Rootkit before and after this update, with (a) representing the earlier version. While (b) introduces more complex system calls like openat and fork for file and process hiding, it retains the core semantics from (a). Both versions use the \textit{interceptSystemCall()} function to intercept system calls. The earlier version intercepted \textit{readdir} to hide malicious files, whereas later versions achieved more sophisticated file hiding by intercepting \textit{openat}. Additionally, both versions implement system-level privilege escalation. The only difference in the new version is that it intercepts the fork system call, returning an error code to hide the malicious process \textit{com.malicious.app}. 

% Privilege escalation and process hiding are also commonly used by other malware families. Figure (c) shows a pseudo-code from the Spyware family, which specializes in stealing sensitive user information while maintaining stealth. Thus, in terms of execution logic, Spyware shares root privilege escalation and process hiding with Rootkit, but additionally calls \textit{TelephonyManager} to collect SIM card serial numbers and send them to a remote server. This is understandable, as despite the diversity of malware families, core malicious behaviors can be categorized into a limited number of types\cite{malradar}. Therefore, we conclude that stable patterns indicative of malicious behavior exist in the training samples. While new functionalities will inevitably emerge, once these patterns are learned, the malware detector will exhibit some robustness against drift.








% \subsection{The Contribution of Features to Detectors}
% \subsection{Create Ideal Drift-robust Malware Detector}
\subsection{Failure of Learning Invariance}
% \subsubsection{Vanilla Malware Detector}
Let $f_r \in \mathcal{R}$ be a sample in the data space with label $y \in \mathcal{Y} = {0, 1}$, where 0 represents benign software and 1 represents malware. The input feature vector $x \in \mathcal{X}$ includes features $\mathcal{F}$ extracted from $f_r$ according to predefined rules. The goal of learning-based malware detection is to train a model $\mathcal{M}$ based on $\mathcal{F}$, mapping these features into a latent space $\mathcal{H}$ and passing them to a classifier for prediction. The process is formally described as follows:

\begin{equation}
\arg \min _{\theta} R_{erm}\left(\mathcal{F}\right)
\end{equation}
where $\theta$ is the model parameter to be optimized and $R_{erm}(\mathcal{F})$ represents the expected loss based on features space $\mathcal{F}$, defined as:
\begin{equation}
R_{erm}\left(\mathcal{F}\right)=\mathbb{E}[\ell(\hat{y}, y)].
\end{equation}
$\ell$ is a loss function. By minimizing the loss function, $\mathcal{M}$ achieves the lowest overall malware detection error. 

% Let $r \in \mathcal{R}$ represent a sample in the original data space, where the corresponding label is denoted as $y \in \mathcal{Y} = \{0, 1\}$, with 0 indicating benign software and 1 indicating malware. The input feature vector $x \in \mathcal{X}$ comprises features $\mathcal{F}$ extracted from $r$ according to specific rules. The objective of learning-based malware detection schemes is to learn a model $\mathcal{M}$ based on the feature set $\mathcal{F}$, which maps the features into a latent space $\mathcal{H}$ and feeds them into a classifier to generate predictions. The process is formally described as follows:

% However, as malware evolves, the model's performance gradually degrades. This motivates us to explore an ideal malware detector that can consistently maintain strong discriminative power throughout malware evolution.

% To explore the robustness of various features in response to malware evolution and their contribution to the performance of the detector, we define two key properties of the features used for training: stability and discriminability. Therefore, the aforementioned objective encourages the model to learn discriminative features that can minimize the loss function. However, as we know, this objective may fail during the evolution of malware. 

% This optimization objective encourages the model to learn discriminative features that minimize the loss function. However, the aforementioned objective function may fail during the evolution of malware. Therefore, to explore the robustness of various features in response to malware evolution and their contribution to the performance of the detector, we define two key properties of the features used for training: stability and discriminability. Intuitively, an ideal drift-robust malware detector would be designed to learn features that are both stable and discriminative. 


\subsubsection{Stability and Discriminability of Features}
\label{active ratio}
To investigate the drift robustness in malware evolution from the feature perspective, we introduce two key properties of features: stability and discriminability. Stability refers to a feature's ability to maintain consistent relevance across different distributions, while discriminability reflects a feature's capacity to distinguish different categories effectively. Typically, feature analysis relies on model performance and architecture, which may introduce bias in defining these feature properties. Therefore, we propose a modelless formal definition, making it applicable across various model architectures. 

Let $f_j$ represent the $j$-th feature in the feature set $\mathcal{F}$, and $S$ denote the set of all samples. To capture the behavior of feature $f_j$ under different conditions, we compute its active ratio over a subset $S^{\prime} \subseteq S$, representing how frequently or to what extent the feature is ``active'' within that subset. Specifically, for a binary feature space, feature $f_j$ takes values 0 or 1 (indicating the absence or presence of the feature, respectively), the active ratio of $f_j$ in the subset $S^{\prime}$ is defined as the proportion of samples where $f_j$ is present, which is defined as Eq.~\ref{active ratio}:
\begin{equation}
\label{active ratio}
r\left(f_j, S^{\prime}\right)=\frac{1}{\left|S^{\prime}\right|} \sum_{s \in S^{\prime}} f_j(s) 
\end{equation}
The ratio measures how frequently the feature is activated within the subset $S^{\prime}$ relative to the total number of samples in the subset. At this point, we can define the stability and discriminability of features.

\begin{myDef} 
\textbf{Stable Feature}: A feature $f_j$ is defined as stable if, for any sufficiently large subset of samples $S^{\prime} \subseteq S$, the active ratio $r\left(f_j, S^{\prime}\right)$ remains within an $\epsilon$-bound of the overall active ratio $r\left(f_j, S\right)$ across the entire sample set, regardless of variations in sample size or composition. Formally, $f_j$ is stable if:
\begin{equation}
\forall S^{\prime} \subseteq S,\left|S^{\prime}\right| \geq n_0, \quad\left|r\left(f_j, S^{\prime}\right)-r\left(f_j, S\right)\right| \leq \epsilon    
\end{equation}
where $\epsilon>0$ is a small constant, and $n_0$ represents a minimum threshold for the size of $S^{\prime}$ to ensure the stability condition holds.
\end{myDef}

When we consider discriminability, there is a need to focus on the category to which the sample belongs. Thus, let $C=\left\{C_1, C_2, \ldots, C_k\right\}$ be a set of $k$ classes, and $S_k \subseteq S$ be the subset of samples belonging to class $C_k$. The active ratio of feature $f_j$ in class $C_k$ is given by:
\begin{equation}
  r\left(f_j, S_k\right)=\frac{1}{\left|S_k\right|} \sum_{s \in S_k} f_j(s)  
\end{equation}

\begin{myDef}
\textbf{Discriminative Feature}: A feature $f_j$ is discriminative if its active ratio differs significantly between at least two classes, $C_p$ and $C_q$. Specifically, there exists a threshold $\delta > 0$ such that:
\begin{equation}
  \exists C_p, C_q \in C, p \neq q, \quad\left|r\left(f_j, S_p\right)-r\left(f_j, S_q\right)\right| \geq \delta  
\end{equation}
\end{myDef}
Furthermore, the discriminative nature of the feature should be independent of the relative class sizes, meaning that the difference in activation should remain consistent despite variations in the proportion of samples in different classes. Mathematically, for any subset $\tilde{S}_p \subseteq S_p$ and $\tilde{S}_q \subseteq S_q$, where $\left|\tilde{S}_p\right| \neq\left|S_p\right|$ or $\left|\tilde{S}_q\right| \neq\left|S_q\right|$, the discriminative property still holds:
\begin{equation}
    \left|r\left(f_j, \tilde{S}_p\right)-r\left(f_j, \tilde{S}_q\right)\right| \geq \delta
\end{equation}

% \begin{figure}
%     \centering
%     % \setlength{\abovecaptionskip}{0.5cm}
%     \includegraphics[width=\linewidth]{Figure/feature_diff_10.pdf}
%     \caption{Discriminative change of Top 10 discriminative features in the training set during the test phase}
%     \label{fig:f1_family}
% \end{figure}
\begin{figure*}[htbp]
    \centering
    \begin{subfigure}[t]{0.48\textwidth}  
        \centering
        \includegraphics[width=1.0\textwidth]{Figure/feature_diff_10.pdf}
        \caption{}
        \label{fig:diff}
    \end{subfigure}
    \hfill
    \begin{subfigure}[t]{0.48\textwidth}
        \centering
        \includegraphics[width=1.0\textwidth]{Figure/feature_importance_10.pdf}
        \caption{}
        \label{fig:importance}
    \end{subfigure}
    
    \caption{(a) and (b) illustrate changes in the Discriminability of the top 10 discriminative training features and the top 10 important testing features, respectively. ``Discriminability'' is defined as the absolute difference in active ratios between benign and malicious samples. The grey dotted line indicates the start of the testing phase, with preceding values representing each feature's discriminability across months in the training set.}
    \label{fig:feature_discrimination}
\end{figure*}


\subsubsection{Failure Due to Learning Unstable Discriminative Features}
\label{motivation: failure}
The high test set performance of the malware detector within the same period suggests that it effectively learns discriminative features to distinguish benign software from malware. However, our analysis reveals that performance degradation over time is mainly due to the model's inability to capture stable discriminative features from the training set. To illustrate this, we sample 110,723 benign and 20,790 malware applications from the Androzoo\footnote{https://androzoo.uni.lu} platform (2014-2021). Applications are sorted by release date, with 2014 samples used for training and subsequent data divided into 30 equally spaced test intervals. We extract DREBIN~\cite{Arpdrebin} features, covering nine behavioral categories such as hardware components, permissions, and restrict API calls, and select the top 10 discriminative features based on active ratio differences to track over time.

The model configuration follows DeepDrebin~\cite{Grossedeepdrebin}, a three-layer fully connected neural network with 200 neurons per layer. We evaluate performance in each interval using macro-F1 scores. As shown in Figure~\ref{fig:feature_discrimination}, although the top 10 discriminative features maintain stable active ratios, the detector’s performance consistently declines. We further examine feature importance over time using Integrated Gradients (IG) with added binary noise, averaging results across five runs to ensure robustness, as recommended by Warnecke et al.~\cite{IG_explain}.

Figure~\ref{fig:feature_discrimination} presents the top 10 discriminative (a) and important features (b) identified by the model and their active ratio changes. While stable, highly discriminative features from the training set persist through the test phase, the ERM-based detector often relied on unstable features whose discriminative power fluctuated over time. This reliance leads to inconsistent model performance, stabilizing only when the feature discriminative power remains steady. Thus, we attribute the failure of ERM-based malware detectors in drift scenarios to their over-reliance on these unstable features and under-learning of already existing stable discriminative features, limiting its generalization to new samples.

% Figure.\ref{fig:importance} presents the top 10 important features identified by the model and their active ratio changes. The results show that while the highly discriminative features in the training set remained relatively stable during the test phase, the detector trained under empirical risk minimization (ERM) tended to rely on transient discriminative features. The discriminative power of certain features that the model focused on significantly fluctuated during the test phase, and does not seem to be significant enough in the training set. Moreover, when the discriminative power of features remained stable, the model’s performance also stabilized. Therefore, we attribute the failure of ERM-based malware detectors in drift scenarios to their over-reliance on these unstable features and under-learning of already existing stable discriminative features, limiting its generalization to new samples.

Moreover, we observe that highly discriminative features are often associated with high-permission operations and indicate potential malicious activity. For instance, features like \verb|api_calls::java/lang/Runtime;->exec| and \verb|GET_TASKS| are rarely used in legitimate applications. This aligns with malware invariance over time, where core malicious intents remain stable even as implementation details evolve.

\begin{center}
\fcolorbox{black}{gray!10}{\parbox{.9\linewidth}{\textit{\textbf{Take Away}: There are stable and highly discriminative features representing invariance in the training samples, yet current malware detectors fail to learn these features leading to decaying models' performance.}}}
\end{center}


\subsection{Create Model to Learn Invariance}
\label{learn_invariant_feature}
Our discussion highlights the importance of learning stable, discriminative features for drift-robust malware detection. ERM captures features correlated with the target variable, including both stable and unstable information~\cite{understanding}. When unstable information is highly correlated with the target, the model tends to rely on it. Thus, the key challenge is to isolate and enhance stable features, aligning with the goals of invariant learning outlined in Section~\ref{invariant_learning}.
% Our preceding discussion emphasized the importance of learning stable discriminative features for building drift-robust malware detectors. The goal of Empirical Risk Minimization (ERM) is to capture features closely related to the target variable, including both stable and unstable information, and the model is more inclined to rely on transient information when it is more relevant to the target~\cite{understanding}. The main challenge is therefore to isolate the stable component, which is consistent with the invariant learning goal described in Section~\ref{invariant_learning}. 

However, applying invariant learning methods is challenging. Its effectiveness presupposes firstly that the environment segmentation can expose the unstable information that the model needs to forget~\cite{environment_label, env_label}. In malware detection, it is uncertain which application variants will trigger distribution changes. Effective invariant learning requires the encoder to produce rich and diverse representations that provide valuable information for the invariant predictor~\cite{yang2024invariant}. Without high-quality representations, invariant learning may fail. This is also reflected in Figure~\ref{fig:feature_discrimination}, where even in the training phase, the learnt features are still deficient in discriminating between goodware and malware, and hard to fully represent the execution purpose of malware, relying instead on easily confusing features.

Thus, given arbitrary malware detectors, our intuition is to use time-aware environment segmentation to naturally expose the instability in malware distribution drift. Within each environment, ERM assumptions guide associations with the target variable, while the encoder provides both stable and unstable features for the invariant predictor. By minimizing invariant risk, unstable elements are filtered, thereby enhancing the detector's generalization capability.

% Thus, given arbitrary malware detectors, our scheme aims to further extend the learning capability of the encoder to provide more learnable information to the invariant predictor by relying only on the time-aware environment segmentation. Minimizing the invariant risk, in turn, filters out instability in the feature representation, thus further enhancing the generalization ability of the malware detector.

\begin{center}
\fcolorbox{black}{gray!10}{\parbox{.9\linewidth}{\textit{\textbf{Take Away}: Invariant learning helps to learn temporal stable features, but it is necessary to ensure that the training set can expose unstable information and the encoder can learn rich and good representations.}}}
\end{center}



\section{Proposed Approach}
\label{sec:approach}

In this section, we introduce the proposed approaches in this work, including the Gait-frequency Network, the Gait-Net-augmented 
kino-dynamics MPC formulation, and the sequential MPC solving mechanism.


\subsection{Gait-frequency Network}
\label{subsec:gaitnet}

Instead of relying on heuristics or solving an additional optimization problem to determine step frequency from the desired foot location, we propose a lightweight Gait-frequency Network that maps current state feedback and desired foot placement to the preferred step duration for the upcoming stride, seamlessly integrating with the MPC framework while co-optimizing the target variables. An illustration of the Gait-Net is shown in Fig.  \ref{fig:gaitnet}. 

\subsubsection{Data collection}
We begin by using a whole-body MPC as our baseline controller to collect variable-frequency walking data. Besides a more accurate representation of kinematics and dynamics of the robot model, this approach is chosen for two key reasons: first, whole-body MPC outperforms other simplified-model-based control methods under strong unknown disturbances \cite{dantec2024centroidal}. Second, we utilize a MATLAB-based high-fidelity simulation framework, bypassing real-time hardware constraints. This enables the use of a more precise and robust interior-point-method-based nonlinear programming solver at higher control frequencies. The primary objective of this stage is to gather high-resolution data using a robust control algorithm under controlled disturbances in simulation.

We ran 15 different sets of walking simulations that have a total period of 600 seconds of walking data with a small-size humanoid model in the simulation. In each set, we command the robot to walk at a constant velocity in the range of $[0,\:1]$ \unit{m/s}. At the beginning of each stride, we generate a randomized step duration between $[150,\:400]$ \unit{ms}. In the first half of the walking simulation, the robot is commanded to walk without any disturbances, while in the second half, we apply randomized external impulses to the CoM of the robot every 2 seconds with the magnitude of $[10,\:100]$ \unit{N} for a duration of 0.2 seconds. 

Despite the relatively small training dataset, the Gait-Net and MPC combination effectively handles various scenarios with predicted step durations, as demonstrated in Sec. \ref{sec:Results}.

\subsubsection{Latent space feature reduction}
The collected data initially includes the robot's floating-base state variables and world-frame foot locations for both legs ($\mathbb R^{16}$). However, not all features/states are equally deterministic in the CoM space to determine the output of the network. To streamline the feature space for more efficient inference in each iteration while maintaining accurate predictions of variable MPC sampling time, we apply Principal Component Analysis (PCA) to reduce the input features. Specifically, we select one feature with a high absolute loading from each of the top six principal axes, as these features are minimally correlated and capture the most significant variance in the feature space. Fig.  \ref{fig:PCA} illustrates the PCA loadings of all features across six principal axes.

%%%%%%%%%%%%%%%%%%%%%%%%%%%%%%%%%%%%%%%%%%%
\subsection{Gait-Net-augmented Kino-dynamic MPC}
\label{subsec:gaitnetmpc}

\subsubsection{Motivation}
Given a fixed periodic contact sequence in MPC, one can vary the duration of each swing phase by altering the MPC sampling time $dt$ for the entire swing duration of each footstep made up of $h'$ MPC time steps. Hence the swing time $\Delta t = h'dt$. To achieve a variable-frequency walking MPC, we need to optimize the contact wrenches, contact locations, and MPC $dt$ all together.
In the discrete-time CD at time step $k$, 
\begin{equation}
\begin{aligned}
\label{eq:cd_discrete}
    \begin{bmatrix}
        \bm l_{G,k+1} \\ \bm k_{G,k+1}
    \end{bmatrix}
     = \begin{bmatrix}
        \bm l_{G,k} \\ \bm k_{G,k}
    \end{bmatrix} +
    \left[\begin{array}{c} 
    \sum_{i = 0}^{n_i}\bm f_i \\
    \sum_{i = 0}^{n_i}(\bm p_{f,i}-\bm p_c) \times \bm f_i + \bm \tau_i
    \end{array} \right]dt_k.
\end{aligned}
\end{equation}
Nonlinearity arises from the bilinear and multi-linear terms formed by the (cross) products of the three optimization variables.

In this section, we introduce a novel NN-augmented solving mechanism inspired by the popular SQP approach, allowing the MPC $dt$ to be concurrently determined by the Gait-Net alongside the optimization of other variables. 

\subsubsection{Implicit kinematics assurance in trajectory reference}
By selecting spatial momenta $\bm h$ and their primitive $\bm H$ (centroidal pose) as optimization variables, CD-MPC avoids including joint angles in the optimization process. However, generating these spatial reference trajectories still requires a corresponding whole-body joint space trajectory, which is a significant part of ensuring the kinematics feasibility of the optimization results. 
\begin{align}
    \bm h^\text{ref}_k = \bm A_G(\mathbf q^\text{ref}_k)\dot{\mathbf q}^\text{ref}_k,
\end{align}
\begin{equation}
\begin{aligned}
        \bm H^\text{ref}_k
        \approx \bm A_G(\mathbf q^\text{ref}_k)\mathbf q^\text{ref}_k - \sum^{k-1}_{i = 0} \dot{\bm A}_G(\mathbf q^\text{ref}_i,\dot{\mathbf q}^\text{ref}_i)\mathbf q^\text{ref}_i\:dt.
\end{aligned}
\end{equation}

Additionally, after obtaining the local foot location and CoM position trajectory solutions in each iteration, we perform a fast analytical inverse kinematics (IK) $f_\text{IK}$ to compute the corresponding joint trajectories for spatial momenta updates. The complete analytical IK and the approximation of $\bm H$ are provided in the Appendix. 

\remark{The spatial momentum and centroidal pose reference trajectories are updated in each sequential iteration to align with foot position updates, implicitly enforcing kinematic consistency in the reference trajectory. }


\subsubsection{Convex MPC Subproblem}
\label{subsubsec:cmpc}
To address the weak correlation between foot location and swing duration in the CD formulation, we integrate the proposed Gait-Net (Sec. \ref{subsec:gaitnet}) to predict the MPC sampling time at the beginning of each step based on the local foot and CoM location solutions at each sequential iteration. This process continues until the convergence condition is met. This also translates $dt$ from the optimization variable space to the parameter space for computation efficiency. 

\begin{figure}[!t]
\vspace{0.2cm}
    \center
    \includegraphics[clip, trim=4.5cm 12cm 4.1cm 12cm, width=1\columnwidth]{figures/bilinear_surf.pdf}
    \caption{{\bfseries Bilinear Envelope Approximation by Neglecting Search Direction Product $\delta a\cdot\delta b$.}}
    \label{fig:bisurf}
    \vspace{-0.2cm}
\end{figure}

To linearize the bilinearly constrained dynamics constraint (\ref{eq:cd_discrete}) in the kino-dynamics MPC problem. We take inspiration from the SQP solving mechanism and solve the search direction $\delta$ of the bilinear variables, 
\begin{align}
    \bm f_i = \bm f^j_i +  \delta \bm f_i, \:\: \bm \tau_i = \bm \tau^j_i +  \delta \bm \tau_i, \nonumber\\
    \bm p_{f,i} = \bm p^j_{f,i} +\delta \bm p_{f,i}, \:\: \bm p_{c} = \bm p^j_{c} +\delta \bm p_{c},
\end{align}
where the $j$ superscript denotes the total solution from the previous sequential iteration $j$. 
The dynamics evolution of the angular momentum can  be simplified with the assumption:
\begin{assumption}
    \textit{With reasonable warm-start/initialization of the bilinear variables $a_0,\: b_0$, the bilinear product of the search directions are minimal and negligible as the sequential iteration $j$ increases, $\delta a^j \cdot \delta b^j \approx 0$, illustrated in Fig.  \ref{fig:bisurf}.} 
    \label{assump2}
\end{assumption}

Therefore, at iteration $j+1$, 
\begin{align}
\label{eq:lG}
    \bm l_{G,k+1} & = \bm l_{G,k} + \sum_{i = 0}^{n_i}\bm f^j_i dt_k +  \delta \bm f_i dt_k
\end{align}
\begin{equation}
\allowdisplaybreaks
\begin{aligned}
    \label{eq:kG}
    \bm k_{G,k+1} & = \bm k_{G,k} + \sum_{i = 0}^{n_i}(\bm p^j_{f,i} - \bm p^j_c+\delta \bm p_{f,i}-\delta \bm p_c) \\ & \quad 
    \times (\bm f^j_i +  \delta \bm f_i)dt_k  + (\bm \tau^j_i + \delta \bm \tau_i)dt_k \\
    & = \bm k_{G,k} + \sum_{i = 0}^{n_i} \delta \bm \tau_i dt_k  -\bm f^j_i \times(\delta \bm p_{f,i}-\delta \bm p_c) dt_k \\ & \quad  
    +(\bm p^j_{f,i} - \bm p^j_c)\times \delta \bm f_i dt_k + 
    \underbrace{(\delta \bm p_{f,i}-\delta \bm p_c) \times \delta \bm f_i dt_k}_{\approx 0} \\ & \quad
     + \underbrace{(\bm p^j_{f,i} - \bm p^j_c)\times \bm f^j_i dt_k + \bm \tau^j_i dt_k}_\text{const.} 
\end{aligned}
\end{equation}
\remark{The linearization w.r.t. search directions $\delta$ and with \textit{Assumption} \ref{assump2} arrives mathematically identical to 1st-order Taylor expansion of bilinear constraints with the SQP algorithm \cite{boggs1995sequential}. }


\paragraph{Optimization Variables}
We choose the state variable to be the spatial momentum vector $\bm h$ and centroidal pose $\bm H$, and the control input variables to be the search directions of ground contact wrench, foot location, and MPC sampling time over a finite horizon $h$,
\begin{align}
    \bm {x} = \bigl\{ \bm H_{k};\: {\bm h}_{k}\bigr\}^{h}_{k=0}, \quad\quad\quad\quad\quad\\
    \bm {u} = \bigl\{\{\delta \bm f_{i,k},\: \delta \bm \tau_{i,k},\: \delta \bm p_{f,i,k}\}^{n_i}_{i=0},\:\delta \bm p_{c} \bigr\}^{h-1}_{k=0},
\end{align}

\paragraph{Objective Function}
The linearized finite horizon optimization problem can be formulated as 
\begin{alignat}{3}
    \label{eq:CDMPCcost2}
    \underset{\bm x^j, \bm {u}^j}{\text{min}} \: & \sum_{k = 0}^{h-1} \Bigg\| 
    \begin{bmatrix}
        \bm h_{k} \\ \bm H_{k} \\ \bm p^{j-1}_{f,k} +\delta \bm p_{f,k} \\ \bm p^{j-1}_{c,k} +\delta \bm p_{c,k}
    \end{bmatrix} -
    \begin{bmatrix}
        \bm h_{k}^{\text{ref}} \\ \bm H_{k}^{\text{ref}} \\ \bm p_{f,k}^{\text{ref}}\\ \bm p_{c,k}^{\text{ref}}
    \end{bmatrix}
    \Bigg\|^2 _{\bm L_1} \\ \nonumber & \quad +
    \Bigg\| \begin{bmatrix} 
    \bm f^{j-1}_i + \delta \bm f_i \\ \bm \tau^{j-1}_i  + \delta \bm \tau_i 
    \end{bmatrix} \Bigg\|^2 _{\bm L_2}
\end{alignat}
The objectives are to track spatial trajectories, foot reference trajectory, and CoM position trajectory while minimizing ground reaction wrenches. Note that the foot reference is constructed based on a preferred foot location through Gait-Net with a nominal MPC $dt$. This trajectory will be updated with each sequential iteration $j$ until convergence.
% \subsubsection{Constraints}
\paragraph{Linearized dynamics}
With \textit{Assumption} \ref{assump2} and search direction linearization in equation (\ref{eq:lG}-\ref{eq:kG}), we obtain the CD as a linear form with respect to our choice of control input and state variables. At time step $k$, the discrete dynamics is 
\begin{align}
\label{eq:cdlinear}
    \bm x_{k+1} = \mathbf A_k \bm x_{k} + \mathbf B_k \bm u_k + \mathbf C_k,
\end{align}
where $\mathbf A_k$ and $\mathbf B_k$ are the state space matrices. $\mathbf C_k$ contains only the constant terms shown in equation (\ref{eq:lG}-\ref{eq:kG}) in terms of optimization results obtained from the last sequential iteration $j-1$. By including a constant $1$ at the end of the state variables, $\bm x'_{k} = [\bm x_{k};\:1]$, we can then obtain the discrete dynamics in a linearized state-space form, as similarly outlined in \cite{di2018dynamic,chen2024learning}.

\paragraph{Linear momentum and CoM position}
The dynamics evolution of CoM position $\bm p_c$ is embedded in the linear momentum equation as an equality constraint:
\begin{align}
    \bm p^{j-1}_{c,k+1} +\delta \bm p_{c,k+1}  = \bm p^{j-1}_{c,k} +\delta \bm p_{c,k} + \frac{\bm l_{G,k}}{m}\:dt_k.
\end{align}

\paragraph{Friction pyramid}
The friction pyramid is a conservative approximation of the friction cone when the pyramid is inscribed, such that the pyramid friction coefficient is then approximated as 
\begin{align}
    \mu_{\Box} = \frac{\sqrt{2}}{2}\mu,
\end{align}
where $\mu$ is the actual ground friction coefficient. The linear inequality constraint is
\begin{align}
    \nonumber 
    -\mu_{\Box}  ({f}_{i,z}^{j-1}+\delta f_{i,z}) \leq  \big\{({f}_{i,x}^{j-1}+\delta f_{i,x}), 
    \: ({f}_{i,y}^{j-1}+\delta f_{i,y})\bigr\} \\
    \leq \mu_{\Box}  ({f}_{i,x}^{j-1}+\delta f_{i,x}) \quad \quad
    \label{eq:frictionCons}
\end{align}

\paragraph{Contact-switching Constraints}
For periodic walking gait, we use a binary contact schedule $\bm \sigma \in \mathbb R^{n_i\times h}$ to describe the contact switches for each contact point $i$. Hence, the ground reaction wrench can be switched on or off based on whether a leg is in swing or stance, 
\begin{align}
    \label{eq:contactConstraint1}
     \sigma_{i,k}
    \begin{bmatrix}
        \bm f_\text{min} \\ \bm \tau_\text{min} 
    \end{bmatrix} \leq
    \begin{bmatrix}
        \bm f^{j-1}_i + \delta \bm f_i \\ \bm \tau^{j-1}_i  + \delta \bm \tau_i
    \end{bmatrix} \leq 
    \sigma_{i,k}
    \begin{bmatrix}
        \bm f_\text{max} \\ \bm \tau_\text{max} 
    \end{bmatrix}  
\end{align}
In addition to the control input saturation, we also enforce stationary foot location for each footstep while the contact schedule $\sigma_{i,k} = 1$ for contact point $i$.

\paragraph{Dynamic Range of Foot location}
With the future foot location as part of the optimization variables, the constraints can be directly applied with only one-step preview data in discrete terrain - the bounds of the foot location are dependent on the position of the robot,  
\begin{align}
    \bm p_{f,\text{min}}(\bm p_c) \leq \bm p^{j-1}_{f,k} +\delta \bm p_{f,k} \leq \bm p_{f,\text{max}}(\bm p_c)
    \label{eq:KDfoot}
\end{align}
A visualization of the constraint-triggering conditions is illustrated in Fig. \ref{fig:footlocation}. 

\begin{algorithm}[h!]
\caption{Gait-Net-augmented Sequential CMPC}
\label{alg:gaitMPC}
\begin{algorithmic}[1]
\Require $\mathbf q, \: \dot{\mathbf q}, \: \mathbf q^\text{cmd}, \: \dot{\mathbf q}^\text{cmd}$
\State \textbf{intialize} $\bm x_0 = f_\text{j2m}(\mathbf q, \: \dot{\mathbf q}), \: \bm u^0 =\bm u_\text{IG}, \: dt^0 = 0.05$ 
\State $\{ \mathbf q^\text{ref},\:\dot{\mathbf q}^\text{ref},\:\bm p_f^\text{ref}\} = f_\text{ref} \big(\mathbf q, \: \dot{\mathbf q}, \: \mathbf q^\text{cmd}, \: \dot{\mathbf q}^\text{cmd} \big)$
\State $\bm x^\text{ref} = f_\text{j2m}(\mathbf q^\text{ref},\:\dot{\mathbf q}^\text{ref},\:\bm p_f^\text{ref})$
\State $ j = 0$ 
\While{$j \leq j_\text{max} \:\text{and}\: \bm \eta \leq \delta \bm u  $} 
\State $\delta \bm u^{j} = \texttt{cmpc}(\bm x^\text{ref},\:\bm p_f^\text{ref},\:\bm p_c^\text{ref},\: \bm x_0,\: dt^j, \: \bm u^j)$
\State $\bm u^{j+1} = \bm u^j + \delta \bm u^j$ 
\State $dt^{j+1} = \Pi_\text{GN}(\mathbf q, \: \dot{\mathbf q},\: \bm p_f^{j})$
\State $\{ \bm x^\text{ref},\:\bm p_f^\text{ref}\}= f_\text{IK}(\bm p_f^{j},\:\bm p_c^{j},\: dt^{j+1})$
\State $j=j+1$
\EndWhile \\
\Return $\bm u^{j+1} $
\end{algorithmic}
\end{algorithm}
\subsubsection{Gait-Net-augmented Sequential MPC Algorithm}
Algorithm \ref{alg:gaitMPC} outlines the procedure for solving the proposed kino-dynamic MPC with sequential CMPC subproblems and Gait-Net. 

In the initialization stage (1-4), $f_\text{j2m}$ describes the mapping from joint-space general-coordinate states to spatial momenta. $f_\text{ref}$ construct a reference trajectory in generalized coordinate with a nominal sampling duration $dt^0$. Within the sequential CMPC iterations (5-11), the iteration is terminated until reached max iteration $j_\text{max}$ or the search direction reaches the desired tolerance $\bm \eta$. In each iteration, the CMPC subproblem described in \ref{subsubsec:cmpc} is solved via QP; the MPC sampling time is updated through Gait-Net policy $\Pi_\text{GN}$. Subsequently, the reference trajectories are updated to reflect the latest foot location and MPC $dt$. 

%%%%%%%%%%%%%% performance comparison %%%%%%%%%%%%%%%%
\begin{figure}[!t]
\vspace{0.2cm}
     \centering
     \begin{subfigure}[b]{0.5\textwidth}
         \centering
	   \includegraphics[clip, trim=0cm 10.4cm 7.4cm 0cm, width=1\columnwidth]{figures/comparison1_3.pdf}
          \caption{Baseline 1: Fixed step duration for every step.}
          \vspace{0.2cm}
          \label{fig:comp1_3}
     \end{subfigure}
     \begin{subfigure}[b]{0.5\textwidth}
        \includegraphics[clip, trim=0cm 11cm 7.4cm 0cm, width=1\columnwidth]{figures/comparison1_1.pdf}
	\caption{Baseline 2: Solving step duration as part of optimization variables in NMPC.}
        \vspace{0.2cm}
	\label{fig:comp1_1}
     \end{subfigure}
     \begin{subfigure}[b]{0.5\textwidth}
         \centering
	   \includegraphics[clip, trim=0cm 10cm 7.4cm 0cm, width=1\columnwidth]{figures/comparison1_2.pdf}
          \caption{Proposed: Gait-Net-augmented Kino-dynamic MPC.}
          \vspace{0.4cm}
          \label{fig:comp1_2}
     \end{subfigure}
     \begin{subfigure}[b]{0.48\textwidth}
         \centering
	   \includegraphics[clip, trim=0cm 3cm 0cm 0cm, width=1\columnwidth]{figures/mpcdt_comparison.pdf}
          \caption{Comparison of MPC $dt$ (interpreted as step duration).}
          % \vspace{0.1cm}
          \label{fig:dt_comp}
     \end{subfigure}
     \caption{{\bfseries{Comparison of Discrete Terrain Locomotion Performance in 2D Simulation.}} }
        \label{fig:comp1}
\end{figure}

%%%%%%%%%%%%%%%% 3D simulation %%%%%%%%%%%%%
\begin{figure*}[!t]
\vspace{0.2cm}
		\center
		\includegraphics[clip, trim=0cm 12cm 0.2cm 0cm, width=2\columnwidth]{figures/foot_location_snap.pdf}
		\caption{{\bfseries Locomotion over 3-D Stepping-stone Terrain.} Simulation snapshots (left) and plot of measured foot locations (right). In the plot, only foot locations that are on the ground are visualized. The green dashed-line bounding box represents the CoM position threshold that triggers the foot location constraints for the corresponding stepping stone patch.}
		\label{fig:footlocation}
		\vspace{-0.2cm}
\end{figure*}
\begin{figure*}[!t]
\vspace{-0.1cm}
		\center
		\includegraphics[clip, trim=0cm 11.3cm 0.2cm 0cm, width=1.9\columnwidth]{figures/h_tracking2.pdf}
		\caption{{\bfseries Spatial Momenta Measurement vs. MPC Prediction along $l_{G,x},\:k_{G,y},\:$and $k_{G,z}$ of 3-D Stepping-stone Simulation Results.}}
		\label{fig:h_tracking}
		\vspace{-0.2cm}
\end{figure*}

\remark{As the sequential iteration progresses, the reference trajectories $\{ \bm x^\text{ref},\:\bm p_f^\text{ref}\}$ are continuously updated to match closely to the real spatial momentum and pose trajectories based on the latest kinematics results. This process inherently embeds an \textit{implicit kinematics assurance} within the framework.}

\remark{The Gait-Net-augmented kino-dynamic MPC is run at the beginning of each footstep to determine a local step duration in terms of MPC $dt$. The rest of this footstep will incorporate the same $dt$ without the inference of Gait-Net and solve only the contact location and wrenches.}
\subsection{Implementation} \label{impl}
%
\TN is developed with 5249 lines of code in Rust, utilizing \textit{rustc} and fully integrating with \textit{Cargo}, Rust's official package manager. \TN focuses on target files that can be compiled into an executable or a library \cite{cargotarget54online}. Using \textit{Cargo}, we address dependency issues prior to compilation and identify all targets in the package suitable for analysis. Compilation of these target files is done through \textit{rustc}. Upon completion, \TN is activated within the \texttt{after\_analysis} callback function of the \textit{rustc} driver, which is triggered by \textit{rustc} following the generation of Rust compiler's MIR, allowing us to employ the resulting MIR data as input for \tyanalyzer to start the analysis.


The workflow of \TN{} can be divided into two phases: 1) detecting if the type conversion generates a problematic \texttt{dst\_ty}, and 2) checking if the problematic type is accessed. With pairs of type sets generated by \analysisone, we can find a problematic type conversion even when a generic type is involved. With the alias graph built by \analysistwo, we can track how the pointer can be accessed. The type conversion pairs and alias graph are stored in \pcg, which accelerates the interprocedural analysis to obtain the information of external functions. For interprocedural analysis, we introduced a depth limitation to avoid the path explosion problem. We set the path length to 1 (tracing only the immediate caller or callee function), aligning with that in Rupta~\cite{rupta}.

\section{Evaluation}

% Our proposed framework was compared with Apollo \cite{b7Apollo1, b7Apollo2}, which demonstrates that it can model analytic operators using data content. Two loss functions were utilized, the root-mean-square deviation error (RMSE), and the mean absolute error (MAE). The selection of these two loss functions is because they fulfil the disadvantages of each other, while RMSE is sensitive to outlier MAE is not and the MAE cannot take into account the direction of the error while the RMSE can achieve it. Speedup was computed to determine how quickly our framework can model the operator $\Phi$. We utilised the \textit{Speedup} and \textit{Amortized Speedup}, which assesses the require time to approximate each operator in comparison to exhaustively executing them on all datasets (more is better). Particularly, the speedup is equalled $\frac{T{^{(i)}_{op}}}{T{^{(i)}_{SimOp} + T_{vec} + T_{sim} + T_{pred}}}$, where $T{^{(i)}_{op}}$ is the execution time for operator $i$, across all the datasets, $T{^{(i)}_{SimOp}}$ is the time needed to model the operator with the datasets selected from the similarity search, $T_{vec}$  is the time needed to compute the vector embedding for each dataset, $T_{sim}$, is the time needed to perform similarity search, and $T_{pred}$ is the time needed to predict on the dataset $D_o$. In addition to the dataset vectorisation, which is done once for each data lake, we calculate amortised speedup. Furthermore, an experimental evaluation of our proposed model for dataset vectorization NumTabData2Vec has been performed to show that our approach can transform a dataset to a vector embedding representation space $z$. For the evaluation experiments, three different NumTabData2Vec were built to project the dataset representation with vector sizes of $100$, $200$, and $300$. Each model has eight transformer layers and is trained parallel using four NVIDIA A10s GPUs, and trained for fifty epochs.
We compared our framework with Apollo \cite{b7Apollo1, b7Apollo2}, which models analytic operators using data content. Two loss functions to measure prediction accuracy are employed: root-mean-square error (RMSE) and mean absolute error (MAE). RMSE is sensitive to outliers, while MAE is not; conversely, RMSE accounts for error direction, which MAE cannot. Speedup metrics are also used to evaluate how efficiently our framework models operator $\Phi$. Specifically, \textit{Speedup} and \textit{Amortized Speedup} measure the time required to approximate each operator versus exhaustively executing them on all datasets. Speedup is defined as $\frac{T{^{(i)}_{op}}}{T{^{(i)}_{SimOp} + T_{vec} + T_{sim} + T_{pred}}}$, where $T{^{(i)}_{op}}$ is the time to execute operator $i$ on all datasets, $T{^{(i)}_{SimOp}}$ is the time to model the operator with datasets from similarity search, $T_{vec}$ is the vector embedding computation time, $T_{sim}$ is the similarity search time, and $T_{pred}$ is the prediction time for $D_o$. Amortized speedup includes dataset vectorization, performed once per data lake for multiple operators (in our case two operators).
We also evaluate our dataset vectorization model, NumTabData2Vec, which projects datasets into vector embedding space $z$. Three versions were built with vector sizes of $100$, $200$, and $300$, each featuring eight transformer layers. The models were trained for 50 epochs on four NVIDIA A10 GPUs in parallel.

\subsection{Evaluation Setup}
Our framework is deployed over an AWS EC2 virtual machine server running with 48 VCPUs of AMD EPYC 7R32 processors at 2.40GHz, and four A10s GPUs with 24GB of memory each, $192GB$ of RAM memory, and $2TB$ of storage, running over Ubuntu 24.4 LTS. Our code is written in Python (v.3.9.1) and PyTorch modules (v.2.4.0). Apollo was deployed in a virtual machine with 8 VCPUs Intel Xeon E5-2630 @ 2.30GHz, $64GB$ of RAM memory, and $250GB$ of storage, running Ubuntu 24.4 LTS like in their experimental evaluation. 

\subsection{Datasets}
\begin{table}[!ht]
    \centering
    \setlength\doublerulesep{0.5pt}
    \caption{Dataset properties for experimental evaluation}
    \label{tab:table-evaluation-datasets}
    \begin{tabular}{||c|c|c|c||}
        \hline
         \makecell{Dataset Name}& \makecell{\# Files} & \makecell{\# Tuples} & \makecell{\# Columns}\\ \hline\hline
         Household Power & & & \\
         Consumption \cite{b21HPCdataset} & $401$ & $2051$ & 7\\
         \hline
         Adult \cite{b22AdultDataset} & $100$ & $228$ & 14\\
         \hline
         Stocks \cite{b23StockMarketDataset} & $508$ & $1959 - 13$ & 7 \\
         \hline
         Weather \cite{b23WeatherDataset} & $49$ & $516$ & 7 \\ \hline
    \end{tabular}

\end{table}

We evaluated our framework using four diverse datasets to represent real-world scenarios. Table \ref{tab:table-evaluation-datasets} summarizes these datasets and their attributes. The vectorization module, NumTabData2Vec, was trained on data separate from the experimental evaluation data, split $60\%$ for training and $40\%$ for testing.
The Household Power Consumption (HPC) dataset \cite{b21HPCdataset} contains electric power usage measurements from a household in Sceaux, France. It includes $401$ datasets, each with $2051$ tuples and seven features recorded at one-minute intervals. The Adult dataset \cite{b22AdultDataset}, commonly used for binary classification, predicts whether an individual earns more or less than $50K$ annually. It comprises $100$ datasets, each with $228$ individuals and various socio-economic features.
The Stock Market dataset \cite{b23StockMarketDataset} includes daily NASDAQ stock prices obtained from Yahoo Finance, with $508$ datasets. Each dataset contains $13$ to $1959$ tuples, each describing seven feature attributes. The Weather dataset \cite{b23WeatherDataset} provides hourly weather measurements from $36$ U.S. cities between $2012$ and $2017$, split into $49$ datasets, each with $516$ tuples and seven features.


\begin{figure}[!t]
     \centering
     \begin{subfigure}[b]{0.24\textwidth}
         \centering
         \includegraphics[width=\textwidth]{Figures/Results/Sim_Search/HPC/HPC_LR_RMSE_Loss_fig.pdf}
         \caption{Linear Regression RMSE error loss}
         \label{fig:HPC-LR-RMSE}
     \end{subfigure}
     \hfill 
     \begin{subfigure}[b]{0.24\textwidth}
         \centering
         \includegraphics[width=\textwidth]{Figures/Results/Sim_Search/HPC/HPC_LR_MAE_Loss_fig.pdf}
         \caption{Linear Regression MAE error loss}
         \label{fig:HPC-LR-MAE}
     \end{subfigure}
        
     \begin{subfigure}[b]{0.24\textwidth}
         \centering
         \includegraphics[width=\textwidth]{Figures/Results/Sim_Search/HPC/HPC_MLP_RMSE_Loss_fig.pdf}
         \caption{MLP for Regression RMSE error loss}
         \label{fig:HPC-MLP-RMSE}
     \end{subfigure}
     \hfill 
     \begin{subfigure}[b]{0.24\textwidth}
         \centering
         \includegraphics[width=\textwidth]{Figures/Results/Sim_Search/HPC/HPC_MLP_MAE_Loss_fig.pdf}
         \caption{MLP for Regression MAE error loss}
         \label{fig:HPC-MLP-MAE}
     \end{subfigure}
        \caption{Household power consumption dataset prediction error loss}
        \label{fig:HPC-EVAL-RES}
\end{figure}

Our framework was evaluated by registering the accuracy of predicting the output of various ML operators over multiple datasets in $D$ without actually executing the operator on them. To evaluate our scheme and its parameters, we use all four datasets, ranging the size of the produced vectors as well as the similarity functions used.
We project all datasets into $k$-dimensional spaces with varying vector dimensions ($100$, $200$, and $300$). For each dataset in Table \ref{tab:table-evaluation-datasets}, we model different operators: For the regression datasets (Household Power Consumption and Stock Market), we model Linear Regression (LR) and Multi-Layer Perceptron (MLP) operators; for the classification datasets (Weather and Adult), we model the Support Vector Machine (SVM) and MLP classifier operators. Each experiment has been executed $10$ times and we report the average of the error loss, as well as the speedup. 

\begin{figure}[!t]
     \centering
     \begin{subfigure}[b]{0.24\textwidth}
         \centering
         \includegraphics[width=\textwidth]{Figures/Results/Sim_Search/Stocks/Stocks_LR_RMSE_Loss_fig.pdf}
         \caption{Linear Regression RMSE error loss}
         \label{fig:Stock-LR-RMSE}
     \end{subfigure}
     \hfill 
     \begin{subfigure}[b]{0.24\textwidth}
         \centering
         \includegraphics[width=\textwidth]{Figures/Results/Sim_Search/Stocks/Stocks_LR_MAE_Loss_fig.pdf}
         \caption{Linear Regression MAE error loss}
         \label{fig:Stock-LR-MAE}
     \end{subfigure}
        
     \begin{subfigure}[b]{0.24\textwidth}
         \centering
         \includegraphics[width=\textwidth]{Figures/Results/Sim_Search/Stocks/Stocks_MLP_RMSE_Loss_fig.pdf}
         \caption{MLP for Regression RMSE error loss}
         \label{fig:Stock-MLP-RMSE}
     \end{subfigure}
     \hfill 
     \begin{subfigure}[b]{0.24\textwidth}
         \centering
         \includegraphics[width=\textwidth]{Figures/Results/Sim_Search/Stocks/Stocks_MLP_MAE_Loss_fig.pdf}
         \caption{MLP for Regression MAE error loss}
         \label{fig:Stock-MLP-MAE}
     \end{subfigure}
        \caption{Stock market dataset prediction error loss}
        \label{fig:Stock-EVAL-RES}
\end{figure}
\subsection{Evaluation Results}



Figures \ref{fig:HPC-EVAL-RES}, \ref{fig:Stock-EVAL-RES}, \ref{fig:Weather-EVAL-RES}, and \ref{fig:Adult-EVAL-RES} present the evaluation results for each method, comparing the performance of different similarity search techniques across various vector embedding representation spaces. The red (with hatches), brown, and blue bars correspond to vector embeddings of size 100, 200, and 300 respectively. In each sub-figure, the y-axis represents the error loss value, while the x-axis displays the similarity search method applied over the vector embeddings. Figures \ref{fig:HPC-EVAL-RES} and \ref{fig:Stock-EVAL-RES} show the results for the Stock market and Household power consumption datasets, where the bottom sub-figure demonstrates the MLP regression model, and the top sub-figure presents the LR model. Figures \ref{fig:Weather-EVAL-RES} and \ref{fig:Adult-EVAL-RES} depict the evaluation results for the Weather and Adult datasets. In these Figures, the top sub-figure shows the SVM with SGD results, while the bottom sub-figure shows the MLP classifier. The left sub-figures in all Figures use the RMSE loss function, whereas the right sub-figures use the MAE loss function. 



Figure \ref{fig:HPC-EVAL-RES}, we show, for the HPC dataset, shows as increase the vector dimension size there is slightly lower prediction error for all the operator modelling. While for different similarity methods did not result in any significant differences in the prediction error loss for all the operator modelling. This suggests that, regardless the similarity selection method, our framework effectively selects the most optimal subset of data to improve model predictions on the unseen input dataset $D_o$. Additionally, we observe higher error loss with a vector size of 100, which can be attributed to the reduced representation capacity of lower-dimensional vectors. This limitation results in fewer ``right" datasets being selected.

For the stock market dataset, Figure \ref{fig:Stock-EVAL-RES} depicts that a vector embedding representation of size $300$ models more accurate operators, with cosine similarity performing best in the similarity search and modelling the most optimal operator. However, due to the inherent volatility in Stock market data from different days, all models in the stock market dataset experiments exhibit high loss values. 

In the weather dataset, the SVM operator results from sub-figures \ref{fig:Weather-SVM-RMSE} and \ref{fig:Weather-SVM-MAE} show that using $300$ vectors in the representation space consistently led to more accurate operator models across all similarity methods. Specifically, cosine similarity in combination with the $300$-dimensional vector embedding reduced the error rate in operator predictions, demonstrating that projecting datasets into this representation space and applying cosine similarity improves the prediction accuracy on the modelled operator. For the MLP classifier from sub-figures \ref{fig:Weather-MLP-RMSE} and \ref{fig:Weather-MLP-MAE}, the results illustrate that using vector embeddings of size $200$ and K-Means clustering produced the most accurate MLP classifier operators.

% Overall, we observe that the error loss was minimized 
% (** what do you mean, minimized? In general, here you should comment on the effect of similarity function, the effect of vector size and the effect of different operators to the accuracy of prediction. E.g., in Household dataset shows little effect in all bars, but in Stock, the cosine seems better and larger size of vectors leads to better performance etc. **)
% in most cases, indicating that our framework effectively selects the most relevant datasets from the data lake $D$, thereby improving data quality and reducing $\Phi$ prediction errors on the target dataset $D_o$. This demonstrates that the datasets are accurately transformed into the vector embedding representation space, allowing for the selection of datasets most similar to $D_o$. 

%Adult


%Weather
\begin{figure}[t!]
     \centering
     \begin{subfigure}[b]{0.24\textwidth}
         \centering
         \includegraphics[width=\textwidth]{Figures/Results/Sim_Search/Weather/Weather_SVM_RMSE_Loss_fig.pdf}
         \caption{SVM with SGD RMSE error loss}
         \label{fig:Weather-SVM-RMSE}
     \end{subfigure}
     \hfill 
     \begin{subfigure}[b]{0.24\textwidth}
         \centering
         \includegraphics[width=\textwidth]{Figures/Results/Sim_Search/Weather/Weather_SVM_MAE_Loss_fig.pdf}
         \caption{SVM with SGD MAE error loss}
         \label{fig:Weather-SVM-MAE}
     \end{subfigure}
        
     \begin{subfigure}[b]{0.24\textwidth}
         \centering
         \includegraphics[width=\textwidth]{Figures/Results/Sim_Search/Weather/Weather_MLP_RMSE_Loss_fig.pdf}
         \caption{MLP RMSE error loss}
         \label{fig:Weather-MLP-RMSE}
     \end{subfigure}
     \hfill 
     \begin{subfigure}[b]{0.24\textwidth}
         \centering
         \includegraphics[width=\textwidth]{Figures/Results/Sim_Search/Weather/Weather_MLP_MAE_Loss_fig.pdf}
         \caption{MLP MAE error loss}
         \label{fig:Weather-MLP-MAE}
     \end{subfigure}
        \caption{Weather dataset prediction error loss}
        \label{fig:Weather-EVAL-RES}
\end{figure}

On the other hand, the Adult dataset shows the lowest error rates, with error loss values consistently below $0.5$ across all vector embedding dimensions and similarity search methods (see Figure \ref{fig:Adult-EVAL-RES}). The Adult dataset, besides exhibiting a high number of rows, also has a higher number of columns, which demonstrates that our framework performs consistently well even with larger datasets.
Additionally, we observe that the lowest prediction error across all datasets occurs when using higher-dimensional vector embeddings. With a trade-off between accuracy and execution time as the difference to generate all data lake available datasets vector embedding representation between $100$ and $300$ size dimension in the vector representation space to be less than $60$ seconds. This confirms that a higher number of vector dimensions leads to more accurate predictions, consistent with findings in previous research \cite{b8Word2Vec}.


\begin{figure}[!t]
     \centering
     \begin{subfigure}[b]{0.24\textwidth}
         \centering
         \includegraphics[width=\textwidth]{Figures/Results/Sim_Search/Adult/Adult_MLP_RMSE_Loss_fig.pdf}
         \caption{SVM with SGD RMSE error loss}
         \label{fig:Adult-LR-RMSE}
     \end{subfigure}
     \hfill 
     \begin{subfigure}[b]{0.24\textwidth}
         \centering
         \includegraphics[width=\textwidth]{Figures/Results/Sim_Search/Adult/Adult_MLP_RMSE_Loss_fig.pdf}
         \caption{SVM with SGD MAE error loss}
         \label{fig:Adult-LR-MAE}
     \end{subfigure}
     
     \begin{subfigure}[b]{0.24\textwidth}
         \centering
         \includegraphics[width=\textwidth]{Figures/Results/Sim_Search/Adult/Adult_MLP_RMSE_Loss_fig.pdf}
         \caption{MLP RMSE error loss}
         \label{fig:Adult-MLP-RMSE}
     \end{subfigure}
     \hfill 
     \begin{subfigure}[b]{0.24\textwidth}
         \centering
         \includegraphics[width=\textwidth]{Figures/Results/Sim_Search/Adult/Adult_MLP_MAE_Loss_fig.pdf}
         \caption{MLP MAE error loss}
         \label{fig:Adult-MLP-MAE}
     \end{subfigure}
        \caption{Adult dataset prediction error loss}
        \label{fig:Adult-EVAL-RES}
\end{figure}





We conducted an experimental evaluation using the Sampling Ratio (SR) approach, similar to Apollo \cite{b7Apollo1}, but employed neural networks built from the vector embeddings of each dataset. The SR approach involves a unified random selection of $l\%$ datasets from the vector representation space, using this subset to construct a neural network for predicting operator outputs. We tested SR values of $0.1$, $0.2$, and $0.4$, as well as vector embedding dimensions of $100$, $200$, and $300$, across all datasets. 
Figure \ref{fig:SR-EVAL-RES} presents the sampling ratio results for the Adult dataset using MLP (sub-figure \ref{fig:Adult-SR-RMSE}) and for the Weather dataset using LR (sub-figure \ref{fig:Weather-SR-SVM-MAE}). In each sub-figure the y-axis represents the RMSE prediction error loss while the x-axis denotes the vector dimension



\begin{figure}[htpb!]
     \centering
     \begin{subfigure}[b]{0.24\textwidth}
         \centering
         \includegraphics[width=\textwidth]{Figures/Results/SR/Adult/Adult_MLP_SR_RMSE_Loss_fig.pdf}
         \caption{Adult Dataset MLP Operator RMSE error loss}
         \label{fig:Adult-SR-RMSE}
     \end{subfigure}
     \hfill 
     \begin{subfigure}[b]{0.24\textwidth}
         \centering
         \includegraphics[width=\textwidth]{Figures/Results/SR/HPC/HPC_LR_SR_RMSE_Loss_fig.pdf}
         \caption{HPC dataset LR Operator RMSE error loss}
         \label{fig:Weather-SR-SVM-MAE}
     \end{subfigure}
     \caption{Sampling Ratio prediction results}
        \label{fig:SR-EVAL-RES}
\end{figure}

Both experiments demonstrate that as the vector embedding dimension increases, coupled with a larger sampling ratio (SR) value, there is a slight decrease in the prediction error loss. This improvement occurs because higher-dimensional vector embeddings provide a more accurate representation of the datasets in k-dimensions, with better dataset selection leading to enhanced prediction accuracy. Comparing the SR approach to our similarity search method for the HPC dataset, the SR approach was approximately $15\%$ less accurate in operator prediction across all vector embedding dimensions. A similar trend was observed in the Weather dataset. However, the Stock dataset exhibited a much larger discrepancy, with the SR approach performing about $70\%$ worse in prediction accuracy across all vector embedding dimensions. Likewise, in the Adult dataset, the SR approach delivered the poorest performance, with nearly $90\%$ lower prediction accuracy compared to the similarity search methods.

\begin{table*}[htbp]
    \centering
        \caption{Evaluation results of our framework exported analytic operator with lowest prediction error in comparison with Apollo}
    \label{tab:table-eval-res}
    % \scalebox{0.8}{
    \setlength\doublerulesep{0.5pt}
    % \begin{adjustbox}{width=\linewidth,center}
    \begin{tabular}{|c|c|c|c|c|c|c|}
    \hline
         \makecell{Dataset\\Name} & Method & Operator & RMSE &  MAE & Speedup  & Amortized Speedup \\
         \hline\hline
         \multirow{7}{*}{\makecell{Household\\Power\\Consumption}}& \makecell{$300$V Cosine} & LR & $\mathbf{6.61}$ & $\mathbf{5.42}$ & $0.0017$ & $\mathbf{1.99}$ \\ \cline{2-7}
                  & \makecell{$300$V SR-$0.2$} & LR & $7.77$ & $6.66$ &  $0.0018$  & $1.42$\\ \cline{2-7} 
        & \makecell{Apollo-SR $0.1$} & LR & $2968.01$ &  $2352.55$ & $\mathbf{0.015}$ & $0.024$ \\ \cline{2-7}
         & \makecell{Apollo-SR $0.2$} & LR & $2811.49$ &  $2229.50$ & $0.015$ & $0.024$ \\ \cline{2-7}\cline{2-7}
         & \makecell{$300$V K-Means} & MLP Regr. & $\mathbf{6.70}$ & $\mathbf{3.38}$ &  $0.9249$  & $\mathbf{1.99}$\\ \cline{2-7}
         & \makecell{Apollo-SR $0.1$} & MLP Regr. & $3322.05$ &  $2606.99$ & $2.38$ & $1.74$ \\ \cline{2-7}
         & \makecell{Apollo-SR $0.2$} & MLP Regr. & $3850.01$ &  $2609.36$ & $\mathbf{2.38}$ & $1.74$\\ \cline{1-7} \cline{1-7} 
         % Stock
         % \multirow{5}{*}{\makecell{Stock}}& \multirow{1}{*}{ \makecell{$100$V Euclidean}} & LR & $229388.93$ & $193066.03$ \\ \cline{2-5}
        \multirow{7}{*}{\makecell{Stock}} &  \makecell{$300$V Cosine} & LR & $306382.28$ & $125335.65$ & $0.00085$ & $\mathbf{1.91}$\\ \cline{2-7}
        & \makecell{$300$V SR-$0.4$} & LR & $21861625.91$ & $5674215.265$ &  $0.00087$  & $0.33$\\ \cline{2-7}
        & \makecell{Apollo-SR $0.1$} & LR & $\mathbf{153665.92}$ &  $\mathbf{118236.48}$ & $\mathbf{0.00093}$ & $0.00096$\\ \cline{2-7}
         & \makecell{Apollo-SR $0.2$} & LR & $166844.95$ &  $133306.68$ & $0.00093$ & $0.00096$\\ \cline{2-7}\cline{2-7}
         &  \makecell{$300$V Cosine} & MLP Regr. & $\mathbf{140236.47}$ & $\mathbf{123571.12}$ & $0.63$ & $\mathbf{1.91}$\\ \cline{2-7}
         & \makecell{Apollo-SR $0.1$} & MLP Regr. &  $175150.82$ &  $145123.09$ & $\mathbf{0.93}$ & $0.96$\\ \cline{2-7}
         & \makecell{Apollo-SR $0.2$} & MLP Regr. & $174390.81$ &  $146338.73$ & $0.93$ & $0.96$\\ \cline{1-7} \cline{1-7}
         % Weather
         \multirow{7}{*}{\makecell{Weather}}& \multirow{1}{*}{ \makecell{$300$V Cosine}} & \makecell{SVM SGD}& $\mathbf{14.13}$ & $\mathbf{7.63}$ & $1.06$ & $\mathbf{22.8}$ \\ \cline{2-7}
               & \makecell{Apollo-SR $0.1$} & SVM & $69.51$ &  $25.52$ & $\mathbf{2.10}$ &  $1.16$\\ \cline{2-7}
                        & \makecell{Apollo-SR $0.2$} & SVM & $68.70$ &  $22.81$ & $2.10$ & $1.16$\\ \cline{2-7} \cline{2-7}
       &  \multirow{1}{*}{ \makecell{$200$V Cosine}}& MLP & $\mathbf{14.29}$ & $\mathbf{4.03}$ & $1.03$  & $\mathbf{22.8}$\\ \cline{2-7}
        &  \multirow{1}{*}{ \makecell{$200$V SR-$0.4$}}& MLP & $15.95$ & $13.31$ & $1.02$  & $1.77$\\ \cline{2-7}
         & \makecell{Apollo-SR $0.1$} & MLP & $69.62$ &  $23.10$ & $\mathbf{1.34}$ & $1.14$ \\ \cline{2-7}
         & \makecell{Apollo-SR $0.2$} & MLP & $673.56$ &  $\mathbf{84.70}$ & $1.32$ & $1.14$\\ \cline{1-7} \cline{1-7}
         
         % Adult
         \multirow{7}{*}{\makecell{Adult}}& \multirow{1}{*}{ \makecell{$300$V Cosine}} & \makecell{SVM SGD}& $\mathbf{0.36}$ & $\mathbf{0.2}$ & $0.37$   & $\mathbf{2.78}$\\ \cline{2-7}
                  & \makecell{Apollo-SR $0.1$} & SVM & $68.32$ &  $22.95$ & $\mathbf{0.75}$ & $0.85$ \\ \cline{2-7}
                 & \makecell{Apollo-SR $0.2$} & SVM & $68.88$ &  $22.88$ & $0.74$ & $0.85$\\ \cline{2-7} \cline{2-7}

         &  \multirow{1}{*}{ \makecell{$300$V K-Means}}& MLP & $\mathbf{0.36}$ & $\mathbf{0.19}$ & $0.30$ & $2.78$ \\ \cline{2-7}
        & \makecell{$300$V SR-$0.2$} & MLP & $6.01$ & $6.00$ &  $0.54$  & $\mathbf{3.54}$\\ \cline{2-7}
         & \makecell{Apollo-SR $0.1$} & MLP & $71.11$ &  $26.51$ & $\mathbf{1.07}$ & $1.31$\\ \cline{2-7}
         & \makecell{Apollo-SR $0.2$} & MLP & $70.16$ &  $25.74$ & $1.05$ & $1.31$\\ \cline{1-7}
         
    \end{tabular}
    % }
\end{table*}

% Table \ref{tab:table-eval-res} illustrates the model operators for each dataset and each loss function, amortized speedup and speedup from our framework in comparison with the same model operators from the Apollo \cite{b7Apollo1, b7Apollo2} framework with SR of $0.1$ and $0.2$. The values $100$V, $200$V, and $300$V in the method column correspond to the dimensions of the vector embedding used for each dataset. The lowest prediction error for each modelled operator in each dataset is highlighted in the method that is used in the similarity search step from our pipeline. Apollo outperforms our framework only on the stock dataset for SR equal with $0.1$ in the LR analytic operator for both RMSE and MAE loss function which performs $50\%$ and $6\%$ better on each loss function equivalent. While our framework for the MLP for Regression outperforms the Apollo modelled operator for $20\%$ and $84\%$ for RMSE and MAE loss functions. However, this difference in the Stock dataset for LR operator modelling is not significant. In the remaining datasets, our framework illustrates that it can outperform Apollo for different values of SR. This makes us confirm that our similarity search using similarity functions selects the most similar datasets $D_r$ from data lake directory $D$, increasing data quality and minimising $\Phi$ prediction errors on the dataset $D_o$. For the Adult dataset, our model operators also perform better, which indicates our method's advantage with an increased number of dataset features (columns). In term of speedup we can see that Apollo outperformed our framework of all modelled operators. In terms of speedup we can see that Apollo outperformed our framework of all modelled operators. This is due to the vectorisation method of our framework which consists of big complexity time. Furthermore, in amortized speedup in most of the amortized speedup in which the vectorization is not counted because it is executed only one time and can be reused our framework surpasses Apollo framework in most of the operators with a big difference with our framework to be between $10\%$ and $60\%$ faster than Apollo. Additionally, most datasets demonstrate better amortized speedup when using the SR approach within our framework. This is because the prediction process relies solely on the vector representation, rather than leveraging all dataset tuples as done in the similarity search method for operator modelling. However, in terms of prediction accuracy, the SR approach does not perform as well as the similarity search method, which achieves superior results.

Table \ref{tab:table-eval-res} compares model operators, loss functions, and speedup metrics for our framework and Apollo at SR values of $0.1$ and $0.2$. Methods $100$V, $200$V, and $300$V denote vector embedding dimensions. The lowest prediction errors align with our pipeline's similarity search method.
Apollo outperforms our framework on the Stock dataset for the LR analytic operator at SR equals with $0.1$ (with $50\%$ and $6\%$ improvements for RMSE and MAE, respectively). However, our framework excels with the MLP regression operator, improving RMSE and MAE by $20\%$ and $17\%$, respectively. The LR operator's performance gap on the Stock dataset is minor.
For other datasets, our framework consistently surpasses Apollo across different SR values. This demonstrates the effectiveness of our similarity search approach, which enhances data quality and reduces $\Phi$ prediction errors by identifying relevant datasets $D_r$ from the data lake directory $D$. The Adult dataset also highlights our framework's advantage with increasing feature dimensions.
Although Apollo achieves better raw speedup due to the higher complexity of our framework's vectorization step, our framework outperforms it in amortized speedup. By excluding the reusable vectorization process, it achieves speed gains of $10\%$ to $60\%$ for most operators.
The SR approach, leveraging vector embedding representations, enhances operator prediction compared to Apollo and achieves greater amortized speedup. However, the similarity search method outperforms both Apollo and the SR approach in prediction accuracy and amortized speedup, establishing its clear superiority across most datasets and operator scenarios.

\subsection{NumTabData2Vec Evaluation Results}

\begin{figure}[!ht]
    \centering
    \includegraphics[width=0.4\textwidth]{Figures/Results/Representation/V200_representation.pdf}
    \caption{Vector representation for each dataset from NumTabData2Vec}
    \label{fig:eval-data-repr}
\end{figure}


\begin{table}[!htp]
    \centering
    \caption{Similarity between vectors of different datasets scenarios}
    \label{tab:vec-rep-sim}
    \setlength\doublerulesep{0.5pt}
    \begin{tabular}{||c|c||}
    \hline
    Model Name & Similarity \\
    \hline\hline
     \makecell{NumTabData2Vec\\$100$ Vector size} & $0.54$\\
     \hline
      \makecell{NumTabData2Vec\\$200$ Vector size}   & $0.18$\\
      \hline
       \makecell{NumTabData2Vec\\$300$ Vector size}  & $0.16$\\ \hline
    \end{tabular}
\end{table}

% Our proposed model, \textit{NumTabData2Vec}, for dataset vectorization is compared between all the available dataset scenarios to determine whether it can effectively distinguish between them based on qualitative differences. The comparison involves selecting $n$ random datasets for each detaset scenario and projecting them into their respective vector embedding representations. Then for each dataset scenario, it gains the average vector embedding representation by the average vector embedding representation of the $n$ random datasets. The vector embedding representation for each dataset scenario depicted in Figure \ref{fig:eval-data-repr} in from the $k$-dimensional space (size of $200$) transformed to the 3d space using the PCA. Figure \ref{fig:eval-data-repr} demonstrates that each dataset occupies a distinct dimension, with non-overlapping or clustering closely together. This indicates that \textit{NumTabData2Vec} can identify the datasets from various situations and does not have a close representation like previous methods achieved it with the same accuracy but on different data types (such as word, and graphs) \cite{b8Word2Vec, b9Graph2Vec} and not in an entire dataset. Table \ref{tab:vec-rep-sim}, further illustrates the average cosine similarity between the vector embeddings of all datasets, demonstrating how dissimilar are the datasets in their vector representation. As the size dimension of the vector embedding representation increases, the model's ability to distinguish across datasets improves as their average similarity decreases. Furthermore, this indicates that larger vector dimension sizes are unneeded since between $100$ and $300$ is sufficient.

Our proposed model, \textit{NumTabData2Vec}, was evaluated to determine its ability to distinguish dataset scenarios based on qualitative differences. For each scenario, $n$ random datasets were selected, and their vector embeddings averaged to represent the scenario. These embeddings, initially in a 200-dimensional space, were projected into 3D using PCA and are shown in Figure \ref{fig:eval-data-repr}. The figure illustrates that each dataset scenario occupies a distinct space, with minimal overlap or clustering. This demonstrates that \textit{NumTabData2Vec} effectively distinguishes datasets, outperforming prior methods like Word2Vec and Graph2Vec \cite{b8Word2Vec, b9Graph2Vec}, which achieved similar accuracy but on different data types (e.g., words, graphs) rather than entire datasets. Table \ref{tab:vec-rep-sim} further highlights the average cosine similarity between dataset embeddings, showing greater dissimilarity as vector dimensions increase. However, results suggest that dimensions between $100$ and $300$ are sufficient for accurate distinction, avoiding the need for larger vector sizes.

\begin{figure}[!ht]
    \centering
    \includegraphics[width=0.4\textwidth]{Figures/Results/Representation/plot_representation_200Vectors.pdf}
    \caption{Synthetic data vector embedding representation}
    \label{fig:eval-sd-data-repr}
\end{figure}

To evaluate \textit{NumTabData2Vec}'s ability to distinguish datasets with varying row and column counts, we generated synthetic numerical tabular datasets of different dimensions and vectorized them. Figure \ref{fig:eval-sd-data-repr} shows datasets with columns ranging from three to thirty and rows from ten to one thousand, projected from a $200$-dimensional space to 2D using PCA. Each bullet caption c and r corresponds to the columns and rows of the dataset, respectively. Datasets with the same number of columns cluster closely in the representation space, and a similar pattern is observed for datasets with the same number of rows. These results indicate that our method effectively distinguishes datasets based on size during vectorization.

\begin{table}[!htp]
    \centering
    \caption{NumTabData2Vec execution time for different dataset dimensions and different vector sizes }

    \begin{adjustbox}{width=\columnwidth,center}
    \label{tab:vec-exec-time}
    \setlength\doublerulesep{0.5pt}
    \begin{tabular}{||c|c|c|c|c||}
    \hline
     \makecell{\# of columns} & \makecell{\# of rows} & \makecell{$50$ Vectors\\Execution time} & \makecell{$100$ Vectors\\Execution time} & \makecell{$200$ Vectors\\Execution time} \\
    \hline\hline
     $3$ & $100$ & $0.0004$ sec & $0.00042$ sec & $0.00051$ sec\\ \hline
     $3$ & $500$ & $0.0004$ sec & $0.00041$ sec & $0.00049$ sec\\ \hline
     $3$ & $1000$ & $0.0004$ sec & $0.00041$ sec & $0.00049$ sec\\ \hline
     $3$ & $1500$ & $0.0004$ sec & $0.00041$ sec & $0.00055$ sec\\ \hline
     $3$ & $1800$ & $0.0004$ sec & $0.00041$ sec & $0.00055$ sec\\ \hline
     \hline
     $10$ & $100$ & $0.0004$ sec & $0.0004$ sec & $0.00057$ sec\\ \hline
     $10$ & $500$ & $0.00039$ sec & $0.0004$ sec & $0.00051$ sec\\ \hline
     $10$ & $1000$ & $0.00041$ sec & $0.00042$ sec & $0.00052$ sec\\ \hline
     $10$ & $1500$ & $0.00041$ sec & $0.00042$ sec & $0.00055$ sec\\ \hline
     $10$ & $1800$ & $0.00041$ sec & $0.00042$ sec & $0.00052$ sec\\ \hline
     \hline
     $20$ & $100$ & $0.0004$ sec & $0.00042$ sec & $0.0005$ sec\\ \hline
     $20$ & $500$ & $0.0004$ sec & $0.00042$ sec & $0.0005$ sec\\ \hline
     $20$ & $1000$ & $0.00042$ sec & $0.00043$ sec & $0.00052$ sec\\ \hline
     $20$ & $1500$ & $0.00043$ sec & $0.00044$ sec & $0.00054$ sec\\ \hline
     $20$ & $1800$ & $0.00044$ sec & $0.00044$ sec & $0.00054$ sec\\ \hline    
     \hline\hline
    \end{tabular}
    \end{adjustbox}
\end{table}

To evaluate how dataset dimensions affect the execution time of \textit{NumTabData2Vec}, we created synthetic datasets with varying numbers of rows ($100$, $500$, $1000$, $1500$, and $1800$) and columns ($3$, $10$, and $20$). These datasets were vectorized into different dimensions, and the execution times were recorded. Table \ref{tab:vec-exec-time} shows that increasing the k-dimension requires approximately $20\%$ more time to generate the vector embeddings. This is expected, as a higher k-dimension involves more hyperparameters, which naturally increases computation time.

Interestingly, varying the number of columns did not significantly impact execution time. However, increasing the number of rows resulted in approximately $5\%$ additional execution time. This is because larger datasets require the extraction of more features, which has a modest impact on the model's execution time.

\begin{figure}[!ht]
    \centering
    \includegraphics[width=0.4\textwidth]{Figures/Results/Representation/plot_representation_noise_data_200Vectors.pdf}
    \caption{HPC Dataset vector embedding representation with addition of Noise}
    \label{fig:eval-nd-data-repr}
\end{figure}

To evaluate \textit{NumTabData2Vec}'s ability to distinguish datasets based on different properties like distribution and order, we introduced Gaussian noise to random $l\%$ of data tuples in an HPC dataset. Figure \ref{fig:eval-nd-data-repr} visualises the original and noise-modified datasets, projected from a 200-dimensional space to 2D using PCA. Each bullet caption g denotes the percentage of Gaussian noise added in the dataset. As noise increases, the representation space shifts further from the original dataset, indicating that \textit{NumTabData2Vec} effectively captures distribution differences. Additionally, since the HPC dataset has an inherent order, the model's sensitivity to noise demonstrates its ability to distinguish datasets based on ordering as well.

\begin{figure}[!ht]
    \centering
    \includegraphics[width=0.4\textwidth]{Figures/Results/Representation/plotrepresentationnoisedata1col200Vectors.pdf}
    \caption{HPC Dataset vector embedding representation with addition of Noise in the first column}
    \label{fig:eval-nd-data-repr-1col}
\end{figure}

To evaluate how fine-grained as distinction can be, we introduced noise into a single column and repeated the previous experiment, with the difference being that noise was added exclusively to the first column. Figure \ref{fig:eval-nd-data-repr-1col} visualizes the dataset's 2D vector space. The amount of Gaussian noise added to the dataset's first column is indicated by g in the bullet caption. The results show that as more noise is introduced to the column, the vector representation moves further away from the original dataset. In contrast to the previous experiment shown in Figure \ref{fig:eval-nd-data-repr}, the noisy dataset's representation stays closest to the original when only a single column is modified. Also in this experiment the dataset points in the 2-dimension are more grouped between them instead the previous experiment. 

%closely grouped compared to the previous experiment.

\section{Related Work}
\label{sec:RelatedWork}

Within the realm of geophysical sciences, super-resolution/downscaling is a challenge that scientists continue to tackle. There have been several works involved in downscaling applications such as river mapping \cite{Yin2022}, coastal risk assessment \cite{Rucker2021}, estimating soil moisture from remotely sensed images \cite{Peng2017SoilMoisture} and downscaling of satellite based precipitation estimates \cite{Medrano2023PrecipitationDownscaling} to name a few. We direct the reader to \cite{Karwowska2022SuperResolutionSurvey} for a comprehensive review of satellite based downscaling applications and methods. Pertaining to our objective of downscaling \acp{WFM}, we can draw comparisons with several existing works. 
In what follows, we provide a brief review of functionally adjacent works to contrast the novelty of our proposed model and its role in addressing gaps in literature. 

When it comes to downscaling \ac{WFM}, several works use statistical downscaling techniques. These works downscale images by using statistical techniques that utilize relationships between neighboring water fraction pixels. For instance, \cite{Li2015SRFIM} treat the super-resolution task as a sub-pixel mapping problem, wherein the input fraction of inundated pixels must be exactly mapped to the output patch of inundated pixels. 
% In doing so, they are able to apply a discrete particle swarm optimization method to maximize the Flood Inundation Spatial Dependence Index (FISDI). 
\cite{Wang2019} improved upon these approaches by including a spectral term to fully utilize spectral information from multi spectral remote sensing image band. \cite{Wang2021} on the other hand also include a spectral correlation term to reduce the influence of linear and non-linear imaging conditions. All of these approaches are applied to water fraction obtained via spectral unmixing \cite{wang2013SpectralUnmixing} and are designed to work with multi spectral information from MODIS. However, we develop our model with the intention to be used with water fractions directly derived from the output of satellites. One such example is NOAA/VIIRS whose water fraction extraction method is described in \cite{Li2013VIIRSWFM}. \cite{Li2022VIIRSDownscaling} presented a work wherein \ac{WFM} at 375-m flood products from VIIRS were downscaled 30-m flood event and depth products by expressing the inundation mechanism as a function of the \ac{DEM}-based water area and the VIIRS water area.

On the other hand, the non-linear nature of the mapping task lends itself to the use of neural networks. Several models have been adapted from traditional single image digital super-resolution in computer vision literature \cite{sdraka2022DL4downscalingRemoteSensing}. Existing deep learning models in single image super-resolution are primarily dominated by \ac{CNN} based models. Specifically, there has been an upward trend in residual learning models. \acp{RDN} \cite{Zhang2018ResidualDenseSuperResolution} introduced residual dense blocks that employed a contiguous memory mechanism that aimed to overcome the inability of very deep \acp{CNN} to make full use of hierarchical features. 
\acp{RCAN} \cite{Zhang2018RCANSuperResolution} introduced an attention mechanism to exploit the inter-channel dependencies in the intermediate feature transformations. There have also been some works that aim to produce more lightweight \ac{CNN}-based architectures \cite{Zheng2019IMDN,Xiaotong2020LatticeNET}. Since the introduction of the vision transformer \cite{Vaswani2017Attention} that utilized the self-attention mechanism -- originally used for modeling text sequences -- by feeding a sequence 2D sub-image extracted from the original image. Using this approach \cite{LuESRT2022} developed a light-weight and efficient transformer based approach for single image super-resolution. 


For the task of super-resolution of \acp{WFM}, we discuss some works whose methodology is similar to ours even though they differ in their problem setting. \cite{Yin2022} presented a cascaded spectral spatial model for super-resolution of MODIS imagery with a scaling factor 10. Their architecture consists of two stages; first multi-spectral MODIS imagery is converted into a low-resolution \ac{WFM} via spectral unmixing by passing it through a deep stacked residual \ac{CNN}. The second stage involved the super-resolution mapping of these \acp{WFM} using a nested multi-level \ac{CNN} model. Similar to our work, the input fraction images are obtained with zero errors which may not be reflective of reality since there tends to be sensor noise, the spatial distribution of whom cannot be easily estimated. We also note that none of these works directly tackle flood inundation since they've been trained with river map data during non-flood circumstance and \textit{ergo} do not face a data scarcity problem as we do. 
% In this work, apart from the final product of \acp{WFM}, we are not presented with any additional spectral information about the low resolution image. This was intended to work directly with products that can generate \ac{WFM} either directly (VIIRS) or indirectly (Landsat).
\cite{Jia2019} used a deep \ac{CNN} for land mapping that consists of several classes such as building, low vegetation, background and trees. 
\cite{Kumar2021} similarly employ a \ac{CNN} based model for downscaling of summer monsoon rainfall data over the Indian subcontinent. Their proposed Super-Resolution Convolutional Neural Network (SRCNN) has a downscaling factor of 4. 
\cite{Shang2022} on the other hand, proposed a super-resolution mapping technique using Generative Adversarial Networks (GANs). They first generate high resolution fractional images, somewhat analogous to our \ac{WFM}, and are then mapped to categorical land cover maps involving forest, urban, agriculture and water classes. 
\cite{Qin2020} interestingly approach lake area super-resolution for Landsat and MODIS data as an unsupervised problem using a \ac{CNN} and are able to extend to other scaling factors. \cite{AristizabalInundationMapping2020} performed flood inundation mapping using \ac{SAR} data obtained from Sentinel-1. They showed that \ac{DEM}-based features helped to improve \ac{SAR}-based predictions for quadratic discriminant analysis, support vector machines and k-nearest neighbor classifiers. While almost all of the aforementioned works can be adapted to our task. We stand out in the following ways (i) We focus on downscaling of \acp{WFM} directly, \textit{i.e.,} we do not focus on the algorithm to compute the \ac{WFM} from multi-channel satellite data and (ii) We focus on producing high resolution maps only for instances of flood inundation. The latter point produces a data scarcity issue which we seek to remedy with synthetic data. 


%%%%%%%%%%%%%%%%% Additional unused information %%%%%%%%%%%%%%%%


%     \item[\cite{Wang2021}] Super-Resolution Mapping Based on Spatial–Spectral Correlation for Spectral Imagery
%     \begin{itemize}
%         \item Not a deep neural network approach. SRM based on spatial–spectral correlation (SSC) is proposed in order to overcome the influence of linear and nonlinear imaging conditions and utilize more accurate spectral properties.
%         \item (fig 1) there are two main SRM types: (1) the initialization-then-optimization SRM, where the class labels are allocated randomly to subpixels, and the location of each subpixel is optimized to obtain the final SRM result. and (2)soft-then-hard SRM, which involves two steps: the subpixel sharpening and the class allocation.  
%         \item SSC procedures: (1) spatial correlation is performed by the MSAM to reduce the influences of linear imaging conditions on image quality. (2) A spectral correlation that utilizes spectral properties based on the nonlinear KLD is proposed to reduce the influences of nonlinear imaging conditions. (3) spatial and spectral correlations are then combined to obtain an optimization function with improved linear and nonlinear performances. And finally (4) by maximizing the optimization function, a class allocation method based on the SA is used to assign LC labels to each subpixel, obtaining the final SRM result.
%         \item (Comparable) 
%     \end{itemize}
%     %--------------------------------------------------------------------
% \cite{Wang2021} account for the influence of linear and non-linear imaging conditions by involving more accurate spectral properties. 
%     %--------------------------------------------------------------------
%     \item[\cite{Yin2022}] A Cascaded Spectral–Spatial CNN Model for Super-Resolution River Mapping With MODIS Imagery
%     \begin{itemize}
%         \item produce  Landsat-like  fine-resolution (scale of 10)  river  maps  from  MODIS images. Notice the original coarse-resolution remotely sensed images, not the river fraction images.
%         \item combined  CNN  model that  contains  a spectral  unmixing  module  and  an  SRM  module, and the SRM module is made up of an encoder and a decoder that are connected through a series of convolutional blocks. 
%         \item With an adaptive cross-entropy loss function to address class imbalance.	
%         \item The overall accuracy, the omission error, the  commission  error,  and  the  mean  intersection  over  union (MIOU)  calculated  to  assess  the results.
%         \item partially comparable with ours, only the SRM module part
%     %--------------------------------------------------------------------

% To decouple the description of the objective and the \ac{ML} model architecture, the motivation for the model architecture is described in \secref{sec:Methodology}. 


%     \item[\cite{Wang2019}] Improving Super-Resolution Flood Inundation Mapping for Multi spectral Remote Sensing Image by Supplying More Spectral 
%     \begin{itemize}
%         \item proposed the SRFIM-MSI,where a new spectral term is added to the traditional SRFIM to fully utilize the spectral information from multi spectral remote sensing image band. 
%         \item The original SRFIM \cite{Huang2014, Li2015} obtains the sub pixel spatial distribution of flood inundation within mixed pixels by maximizing their spatial correlation while maintaining the original proportions of flood inundation within the mixed pixels. The SRFIM is formulated as a maximum combined optimization issue according to the principle of spatial correlation.
%         \item follow the terminology in \cite{Wang2021}, this is an initialization-then-optimization SRM. 
%         \item (Comparable) 
%     \end{itemize}
%     %--------------------------------------------------------------------


%--------------------------------------------------------------------
%     \item[\cite{Jia2019}] Super-Resolution Land Cover Mapping Based on the Convolutional Neural Network
%     \begin{itemize}
%         \item SRMCNN (Super-resolution mapping CNN) is proposed to obtain fine-scale land cover maps from coarse remote sensing images. Specifically, an encoder-decoder CNN is used to determined the labels (i.e., land cover classes) of the subpixels within mixed pixels.
%         \item There were three main parts in SRMCNN. The first part was a three-sequential convolutional layer with ReLU and pooling. The second part is up-sampling, for which a multi transposed-convolutional layer was adopted. To keep the feature learned in the previous layer, a skip connection was used to concatenate the output of the corresponding convolution layer. The last part was the softmax classifier, in which the feature in the antepenultimate layer was classified and class probabilities are obtained.
%         \item The loss: the optimal allocation of classes to the subpixels of mixed pixel is achieved by maximizing the spatial dependence between neighbor pixels under constraint that the class proportions within the mixed pixels are preserved.
%         \item (Preferred), this paper is designed to classify background, Building, Low Vegetation, or Tree in the land. But we can easily adapt to our problem and should compare with this paper.
%     \end{itemize}
%     %--------------------------------------------------------------------

%     \item[\cite{Kumar2021}] Deep learning–based downscaling of summer monsoon rainfall data over Indian region
%     \begin{itemize}
%         \item down-scaling (scale of 4) rainfall data. The output image is not binary image.
%         \item three algorithms: SRCNN, stacked SRCNN, and DeepSD are employed, based on \cite{Vandal2019}
%         \item mean square error and pattern correlation coefficient are used as evaluation metrics.
%         \item SRCNN: super-resolution-based convolutional neural networks (SRCNN) first upgrades the low-resolution image to the higher resolution size by using bicubic interpolation. Suppose the interpolated image is referred to as Y; SRCNNs’task is to retrieve from Y an image F(Y) which is close to the high-resolution ground truth image X.
%         \item stacked SRCNN: stack 2 or more SRCNN blocks to increasing the scaling factor.
%         \item DeepSD: uses topographies as an additional input to stacked SRCNN.
%         \item These algorithms are not designed for binary output images, but if prefer, the ``modified'' stacked SRCNN or DeepSD can be used as baseline algorithms.
%     
%     \item[\cite{Shang2022}] Super resolution Land Cover Mapping Using a Generative Adversarial Network
%     \begin{itemize}
%         \item propose an end-to-end SRM model based on a generative adversarial network (GAN), that is, GAN-SRM, to improve the two-step learning-based SRM methods. 
%         \item Two-step SRM method: The first step is fraction-image super-resolution (SR), which reconstructs a high-spatial-resolution fraction image from the low input, methods like SVR, or CNN has been widely adopted. The second step is converting the high-resolution fraction images to a categorical land cover map, such as with a soft-max function to assign each high-resolution pixel to a unique category value.
%         \item The proposed GAN-SRM model includes a generative network and a discriminative network, so that both the fraction-image SR and the conversion of the fraction images to categorical map steps are fully integrated to reduce the resultant uncertainty. 
%         \item applied to the National Land Cover Database (NLCD), which categorized land into four typical classes:forest, urban, agriculture,and water. scale factor of 8. 
%         \item (Preferred), we should compare with this work.
%     \end{itemize}
%     %--------------------------------------------------------------------

%   \item[\cite{Qin2020}] Achieving Higher Resolution Lake Area from Remote Sensing Images Through an Unsupervised Deep Learning Super-Resolution Method
%   \begin{itemize}
%       \item propose an unsupervised deep gradient network (UDGN) to generate a higher resolution lake area from remote sensing images.
%       \item UDGN models the internal recurrence of information inside the single image and its corresponding gradient map to generate images with higher spatial resolution. 
%       \item A single image super-resolution approach, not comparable
%   \end{itemize}
%     %--------------------------------------------------------------------




%     \item[\cite{Demiray2021}] D-SRGAN: DEM Super-Resolution with Generative Adversarial Networks
%     \begin{itemize}
%         \item A GAN based model is proposed to increase the spatial resolution of a given DEM dataset up to 4 times without additional information related to data.
%         \item Rather than processing each image in a sequence independently, our generator architecture uses a recurrent layer to update the state of the high-resolution reconstruction in a manner that is consistent with both the previous state and the newly received data. The recurrent layer can thus be understood as performing a Bayesian update on the ensemble member, resembling an ensemble Kalman filter. 
%         \item A single image super-resolution approach, not comparable
%     \end{itemize}
%     %--------------------------------------------------------------------
%     \item[\cite{Leinonen2021}] Stochastic Super-Resolution for Downscaling Time-Evolving Atmospheric Fields With a Generative Adversarial Network
%     \begin{itemize}
%         \item propose a super-resolution GAN that operates on sequences of two-dimensional images and creates an ensemble of predictions for each input. The spread between the ensemble members represents the uncertainty of the super-resolution reconstruction.
%         \item for sequence of input images, not comparable with ours.
%     \end{itemize} 
%     %--------------------------------------------------------------------

% \end{itemize}





We present RiskHarvester, a risk-based tool to compute a security risk score based on the value of the asset and ease of attack on a database. We calculated the value of asset by identifying the sensitive data categories present in a database from the database keywords. We utilized data flow analysis, SQL, and Object Relational Mapper (ORM) parsing to identify the database keywords. To calculate the ease of attack, we utilized passive network analysis to retrieve the database host information. To evaluate RiskHarvester, we curated RiskBench, a benchmark of 1,791 database secret-asset pairs with sensitive data categories and host information manually retrieved from 188 GitHub repositories. RiskHarvester demonstrates precision of (95\%) and recall (90\%) in detecting database keywords for the value of asset and precision of (96\%) and recall (94\%) in detecting valid hosts for ease of attack. Finally, we conducted an online survey to understand whether developers prioritize secret removal based on security risk score. We found that 86\% of the developers prioritized the secrets for removal with descending security risk scores.
% \cleardoublepage

\bibliographystyle{IEEEtranN}
% \bibliographystyle{ACM-Reference-Format}
\bibliography{reference}

\end{document}
\endinput
%%
%% End of file `sample-acmsmall-conf.tex'.
