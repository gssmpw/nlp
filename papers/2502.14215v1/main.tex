% \documentclass[acmsmall,screen,review]{acmart}
\documentclass[journal]{IEEEtran}


\newcommand{\thought}[1]{{\color[rgb]{0.2,0.39,0.66}(#1)}}
\newcommand{\todo}[1]{{\color[rgb]{1.0,0.0,0.0}(#1)}}
\newcommand{\hsh}[1]{{\color{green!50!black} Henrik: #1}}
\newcommand{\st}[1]{{\color{red!50!black} Sebastian: #1}}

\newcommand{\ulm}[1]{_{\scaleto{\mathrm{#1}}{3pt}}}
\newcommand\at[2]{\left.#1\right|_{#2}}











\newtheorem{assumption}{Assumption}

\DeclareMathOperator*{\argmax}{arg\,max}
\DeclareMathOperator*{\argmin}{arg\,min}

\newcommand{\swname}[1]{\texttt{#1}}
\newcommand{\ie}{i\/.\/e\/.,\/~}
\newcommand{\eg}{e\/.\/g\/.,\/~}
\newcommand{\cf}{cf\/.\/~}

\newcommand{\fig}{Fig\/.\/~}
\newcommand{\defn}{Def\/.\/~}
\newcommand{\sect}{Sec\/.\/~}
\newcommand{\tabl}{Tab\/.\/~}
\newcommand{\algo}{Algorithm~}
\newcommand{\theo}{Theorem~}

\newcommand{\bnnl}{3 hidden layers}
\newcommand{\bnnn}{50 neurons}
\newcommand{\bnna}{tanh activations}

\newcommand{\capt}[1]{\mdseries{\emph{#1}}}

\newcommand{\videolink}{at \url{https://youtu.be/_d7AqTRjz6g}}
\newcommand{\codelink}{\url{https://github.com/wheelbot/mini-wheelbot}}

\newcommand{\fakepar}[1]{\vspace{0mm}\noindent\textbf{#1.}}

\newcommand{\needref}{\textcolor{red}{[REF]}}

\newcommand{\plotfontsize}{9pt}



%%
%% end of the preamble, start of the body of the document source.
\begin{document}

%%
%% The "title" command has an optional parameter,
%% allowing the author to define a "short title" to be used in page headers.
\title{Towards Secure Program Partitioning for Smart Contracts with LLM's In-Context Learning}

\author{
\IEEEauthorblockN{
  Ye Liu$^1$, 
  Yuqing Niu$^1$, 
  Chengyan Ma$^1$,
  Ruidong Han$^1$, 
  Wei Ma$^1$, 
  Yi Li$^2$, 
  Debin Gao$^1$, 
  and David Lo$^1$,~\IEEEmembership{Fellow,~IEEE}}\\
\IEEEauthorblockA{
  $^1$Singapore Management University\\
  $^2$Nanyang Technological University\\
}
}

% \author{Ye Liu,
%         Yuqing Niu,
%         Chengyan Ma~\textsuperscript{\orcidlink{0000-0001-9256-6930}},
%         Ruidong Han~\textsuperscript{\orcidlink{0000-0001-6859-60057}},
%         Wei Ma,
%         Yi Li,
%         Debin Gao,
%         David Lo~\textsuperscript{\orcidlink{0000-0002-4367-7201}},~\IEEEmembership{Fellow,~IEEE},
% \thanks{
%           Ye Liu, Yuqing Niu, Chengyan Ma, Ruidong Han, Wei Ma, and David Lo are with the School of Computing and Information Systems, Singapore Management University(e-mail:xxxEMAIL).
%           Yi Li is with the School of Computer Science and Engineering, Nanyang Technological University, Singapore.
%           Debin Gao is with the School of Computer Science and Engineering, Beihang University, China.
%         }
% }



% % First author
% \author{Ye Liu}
% \affiliation{%
%   \institution{Singapore Management University}
%   \country{Singapore}
% }
% \email{yeliu@smu.edu.sg}

% % Second author
% \author{Yuqing Niu}
% \affiliation{%
%   \institution{Singapore Management University}
%   \country{Singapore}
% }
% \email{yuqingniu@smu.edu.sg}

% % Third author
% \author{Chengyan Ma}
% \affiliation{%
%   \institution{Singapore Management University}
%   \country{Singapore}
% }
% \email{chengyanma@smu.edu.sg}

% \author{Ruidong Han}
% \affiliation{%
%   \institution{Singapore Management University}
%   \country{Singapore}
% }
% \email{rdhan@smu.edu.sg}


% \author{Wei Ma}
% \affiliation{%
%   \institution{Singapore Management University}
%   \country{Singapore}
% }
% \email{weima@smu.edu.sg}

% % \author{Juantao Zhong}
% % \affiliation{%
% %   \institution{The Hong Kong University of Science and Technology}
% %   % \city{Shenzhen}
% %   \country{Hong Kong}
% % }
% % \email{jzhong012@e.ntu.edu.sg}

% \author{Yi Li}
% \affiliation{%
%   \institution{Nanyang Technological University}
%   \country{Singapore}
% }
% \email{yi_li@ntu.edu.sg}


% \author{David Lo}
% \affiliation{%
%   \institution{Singapore Management University}
%   \country{Singapore}
% }
% \email{davidlo@smu.edu.sg}


%%
%% By default, the full list of authors will be used in the page
%% headers. Often, this list is too long, and will overlap
%% other information printed in the page headers. This command allows
%% the author to define a more concise list
%% of authors' names for this purpose.

%% This command processes the author and affiliation and title
%% information and builds the first part of the formatted document.

%%
%% The abstract is a short summary of the work to be presented in the
%% article.
% Smart contracts are highly susceptible to manipulation attacks due to the leakage of sensitive information. Addressing such vulnerabilities is particularly challenging because they stem from inherent data confidentiality issues rather than straightforward implementation bugs. 
% To tackle this, we present \tool, a novel approach that combines static analysis with the in-context learning capabilities of large language models (LLMs) to partition smart contracts into privileged and normal codebases, guided by a few annotated \secrete variables.
% We evaluated \tool on 18 annotated smart contracts containing 99 sensitive functions. 
% The results demonstrate that \tool successfully generates \emph{compilable}, \emph{secure}, and \emph{functionally equivalent} partitions for 78\% of the subject functions with high accuracy. Additionally, \tool reduces the size of the trusted codebase by approximately 30\% compared to traditional function-level partitioning. 
% In testing on nine real-world victim contracts affected by manipulation attacks, \tool effectively identified eight cases, highlighting its potential for broad applicability and the emerging need for secure program partitioning in smart contract development.

\maketitle
\begin{abstract}
Smart contracts are highly susceptible to manipulation attacks due to the leakage of sensitive information. Addressing manipulation vulnerabilities is particularly challenging because they stem from inherent data confidentiality issues rather than straightforward implementation bugs. 
To tackle this by preventing sensitive information leakage, we present \tool, the first LLM-driven approach that combines static analysis with the in-context learning capabilities of large language models (LLMs) to partition smart contracts into privileged and normal codebases, guided by a few annotated \secrete variables.
We evaluated \tool on 18 annotated smart contracts containing 99 sensitive functions. 
The results demonstrate that \tool successfully generates \emph{compilable}, and \emph{verified} partitions for 78\% of the sensitive functions while reducing approximately 30\% code compared to function-level partitioning approach. 
Furthermore, we evaluated \tool on nine real-world manipulation attacks that lead to a total loss of 25 million dollars, \tool effectively prevents eight cases, highlighting its potential for broad applicability and the necessity for secure program partitioning during smart contract development to diminish manipulation vulnerabilities.
% Smart contracts are vulnerable to manipulation attacks due to sensitive information leakage.
% Prevention of manipulation vulnerabilities from smart contracts are challenging because they are inherent data confidentiality issues rather than easy-to-fix implementation bugs.
% To preserve confidentiality, in this paper, we propose \tool to accept a few annotated \secrete variables and combine static analysis with LLM's in-context learning to partition smart contracts into privileged and normal codebases.
% We evaluate \tool on 18 annotated smart contracts with 99 sensitive functions.
% The evaluation indicates that \tool could successfully generate \emph{compilable}, \emph{secure}, and \emph{functionally equivalent} partitions for 78\% subject functions with a reasonably high accuracy, resulting in a decrease of around 30\% codes in the trusted codebase compared to the function-level partitioning.
% Of nine real-world victim smart contracts suffering manipulation attacks, \tool can be applied to identify eight cases, indicating the potential wide applicability and emerging need for secure program partitioning.
\end{abstract}


% 
% 
The widespread integration of communication networks and smart devices in modern control systems has increased the vulnerability of industrial systems to online cyber-attacks, e.g., Industroyer, Blackenergy, etc \citep{osti_1505628}.
% Modern control systems have seen a large push to include communication networks and smart devices to increase performance, made possible by improvements in communication device cost and energy consumption. This trend has been coupled with the usage of open-standard communication protocols among industrial control systems, making them vulnerable to online cyber-attacks such as Industroyer, Blackenergy, etc \citep{osti_1505628}. 
To counter this, methods have been developed to improve security by achieving attack detection, mitigation, and monitoring, among others \citep{sandberg2022secure}. This paper focuses on active attack diagnosis to mitigate stealthy attacks. 
%
%\subsection{Literature review}

Active diagnosis techniques rely on the inclusion of additional moduli to control systems
% inclusion within the control system of additional moduli 
to alter the behavior of the system compared to information known by the attacker. 
For instance, the concept of additive watermarking was introduced in \cite{mo2015physical}, where noise signals of known mean and variance are added at the plant and compensated for it at the controller. 
This compensation, however, is not exact, causing some performance degradation. Thus, trade-offs between performance and detectability  are necessary \citep{zhu2023detection}.
% A later work \citep{zhu2023detection} designs the watermark signal by trading performance for detection. Thus, although additive watermarking serves as a good detection scheme, they endure performance losses even in the nominal case. 

In encrypted control \citep{darup2021encrypted}, the sensor data is encrypted, sent to the controller, and then operated on directly. Encrypted input signals are sent back to the plant for decryption. Although encryption is widespread in IT security, in control systems it presents some concerns, such as the introduction of time delays \citep{stabile2024verifiable}, while it may present inherent weaknesses \citep{alisic2023model}.
% they are not preferred as they introduce time delays \citep{stabile2024verifiable} which can cause instability, and some encryption schemes can be very weak  \citep{alisic2023model}. 

In moving target defense \citep{griffioen2020moving}, the plant is augmented with fictitious dynamics, known to the controller. The plant output is transmitted to the controller along with the fictitious states over a network under attack. 
The additional measurements then aide in the detection of attacks. 
This comes at the cost of higher communication bandwidth needs, which increases rapidly with the dimension of the augmented systems.
% Since the dynamics of the fictitious dynamics are exactly known to the controller, the attack is detected easily. However, when the scale of the system increases, the communication bandwidth used by moving the target defense approach increases rapidly. 

Other recently proposed works include two-way coding \citep{fang2019two}, a weak encryuption technique, and dynamic masking \citep{abdalmoaty2023privacy}, which enhances privacy as well as security, have been shown to be effective against zero-dynamics attacks.
% Two-way coding \citep{fang2019two} and dynamic masking \citep{abdalmoaty2023privacy} are other recently proposed approaches. Two-way coding is another form of weak encryption technique whilst dynamic masking proposes an architecture that enhances both privacy and security. These schemes are shown to be effective against zero dynamics attacks but remain to be studied for other classes of attacks. 
% Recent extensions include \citep{mukherjee2021secure,ramos2024privacy}.
% Some other works which are related are \citep{mukherjee2021secure}, an extension of \cite{fang2019two}. The work \citep{ramos2024privacy} is an extension of moving target defense for multi-agent systems. 
Furthermore, filtering techniques for attack detection are proposed by \cite{murguia2020security,hashemi2022codesign,escudero2023safety}, while not focusing on stealthy attacks.
% The works \citep{murguia2020security,hashemi2022codesign,escudero2023safety} develop filtering techniques to guarantee safety, without being focused on stealthy covert attacks.

Multiplicative watermarking (mWM) has been proposed by the authors as a diagnosis technique \citep{ferrari2020switching}. mWM consists of a pair of filters on each communication channel between the plant and its controller; the scheme is affine to weak encryption, whereby ``encoding'' and ``decoding'' are done by changing signals' dynamic characteristics through inverse pairs of filters. This enables original signals to be recovered exactly, and thus does not lead to performance degradation.
% A multiplicative watermark is an affine to a weak encryption technique, through which the signal is ``encoded'' by a filter, changing its dynamic behavior. The use of inverse pairs means that the original signal can be recovered, through ``decoding'' via an inverse filter. As such, differently to techniques based on additive watermarking, no performance is lost due to the injection of noise, and there are no bandwidth limitations.

%\subsection{Contributions}
One of the critical features of multiplicative watermarking is that to detect stealthy attacks, the mWM filter parameters must be switched over time. In this paper, an algorithm to optimally design the mWM parameters after a switching event is presented, enhancing detection performance, without changing the switching time.
% This is done without changing the switching time, which is taken as given.

\textcolor{black}{
To formalize the filter design problem, we suppose the defender is interested in optimal performance against adversaries injecting covert attacks with matched system parameters \citep{smith2015covert}, including the mWM parameters prior to the switch. This scenario represents a worst case where malicious agents can take full control of the system while remaining undetected.
Thus, the attack strategy is explicitly included within the formulation of the closed-loop system, and the mWM filters are chosen by solving an optimization problem minimizing the attack-energy-constrained output-to-output gain (AEC-OOG) \citep{anand2023risk}, a variation of the output-to-output gain proposed in  \cite{teixeira2015strategic}.
}
The main contributions of this paper are:
% We consider an adversary injecting a covert attack with matched system parameters \citep{smith2015covert}, i.e., an attacker with full knowledge of the control system parameters, including those of the mWM filters before the switch. This scenario is taken as a worst case, as it has been shown that this class of attacks can be made stealthy. To quantitatively define a cost, the output-to-output gain (OOG) \citep{teixeira2015strategic} is leveraged,
% a metric introduced to evaluate the impact of an additive attack in a control system. %Specifically, OOG evaluates the worst-case performance loss that an attacker injecting an undetectable attack can obtain. 
% Here, the maximum performance loss caused by a stealthy adversary with limited energy is taken, the attack-energy-constrained OOG (AEC-OOG) \citep{anand2023risk}. The main contributions of this paper are:
\begin{enumerate}
%[label=\alph*.]
\item The problem of optimally designing the switching mWM filters is formulated as an optimization problem, with the AEC-OOG is taken as the objective;%where the AEC-OOG is taken as the impact metric; 
\item The worst-case scenario of a covert attack with exact knowledge of plant and mWM filter parameters is embedded within the design problem;
% The optimization problem is defined to incorporate the worst-case scenario of a covert attack with exact knowledge of plant and mWM filter parameters;
\item The feasibility of the optimization problem is shown to be dependent only on stability conditions; 
\item A solution scheme is proposed to promote randomization of the mWM filter parameters such that an eavesdropping adversary cannot remain stealthy.
\end{enumerate} 

This builds on the results of \cite{ferrari2020switching}, where the focus was on the design of the switching protocols, rather than the parameters themselves.
Compared to previous work \citep{gallo2021design}, this paper introduces an optimization problem which is always feasible (thanks to the use of AEC-OOG in the objective), while also considering a more sophisticated class of covert attacks, where the presence of watermark is known to the adversary. 
Moreover, this paper poses a different objective than \citep{zhang2023hybrid}; indeed, while \citep{zhang2023hybrid} provided a design strategy to ensure certain privacy properties, in this paper we address the problem of optimal parameter design following a switching event.


%\subsection{Organization}
The rest of the paper is organized as follows. 
After formulating the problem in Section~\ref{sec:PF}, we propose our design algorithm in Section~\ref{sec:main}, and analyze its properties. It is then evaluated through a numerical example in Section~\ref{sec:NE}, and concluding remarks are given Section~\ref{sec:Con}.
% We provide the problem background in Section~\ref{sec:PF}. We formulate the design problem in Section~\ref{sec:main}, together with an analysis of its properties. The proposed algorithm is evaluated through a numerical example in Section \ref{sec:NE}. Concluding remarks are offered in Section \ref{sec:Con}.
\section{Mobile Networks Powered by \glspl{LLM}}
\label{sec:LLM_enabled_MNs}
\begin{figure*}[t!]
\centering
\includegraphics[width=.99\textwidth]{Fig1.eps}
    \caption{Possible architectural designs for integrated \gls{LLM} and \gls{MNO} ecosystem.}
    \label{fig:LLM_possible_architectures}
\end{figure*}
The historical data of the \gls{MNO}, archived over years of expertise, constitutes a solid foundation for training the \gls{LLM} using structured and unstructured multi-modal inputs (as illustrated in Fig.~\ref{fig:LLM_possible_architectures}a) such as user intents, network logs, alarm descriptions, trouble tickets, \gls{PCAP} files (e.g. from Wireshark or tcpdump), dashboard screenshots, audio recordings (e.g. from \gls{IVR} systems), video feeds (e.g. from infrastructure surveillance), and \gls{NWDAF} analytics. To this end, a separate collection framework aggregates data from various sources into a centralized repository, and extracts most informative features such as warnings, error codes, timestamps, and user/gNB/session/bearer/\gls{QoS} flow/slice IDs. The extracted features are then converted into unified embeddings that are combined into a common vector space with suitable metadata (e.g. to differentiate data formats). The resulting vector store is used to fine-tune the \gls{LLM} to deeply internalize \gls{MNO}-specific knowledge \cite{Bariah2023understanding}. This allows the \gls{LLM} to learn patterns, sequences, and deviations that correlate with normal or faulty network operations. This is made possible using a timestamp-based cross-referencing to link different entries from several data sources, allowing detailed description and context for each flagged event as well as the resolution workflow for the spotted anomalies.

In live mobile networks, fresh multi-modal data is continuously fed into the \gls{LLM}, either uploaded in batches or streamed in real-time. The \gls{LLM} analyzes this data and identifies potential anomalous behaviors in light of its accumulated learning. In case of new anomalies not covered during the fine-tuning stage, the \gls{LLM} can rely on clustering techniques to group similar patterns and flag outliers as suspected behaviors. The \gls{LLM} is also capable of using \gls{RAG}-enabled external knowledge databases such as \gls{3GPP} documents \cite{Said2024instruct}, \gls{IEEE} standards, \gls{IETF} RFCs and vendors documentation \cite{soman2023observations} to compare the actual network behavior with the expected one to identify misconfigurations and spot unusual trends in protocols and communication flows. Well-crafted prompts, on the other hand, can guide the \gls{LLM} responses to provide focused solutions. Paradigms such as the \gls{CoT} reasoning can be used to break down the \gls{LLM} insights into a series of simplified and actionable sub-tasks. It can be extended by the \gls{ToT} technique to explore different reasoning paths and identify the most optimal solution. The \gls{LLM} can naturally produce stepwise reasoning if datasets used for fine-tuning contain \gls{CoT} and \gls{ToT} examples, or through creative prompting \cite{Zhou2024survey}. In parallel, \gls{NOC} engineers can intervene to confirm, guide or reject the \gls{LLM} findings, if needed, e.g. using its intuitive conversational interface. Through continuous self-learning, the \gls{LLM} will dynamically adapt to evolving network conditions, optimizing its performance over time \cite{Chaparadza2023optimization}.

%For instance, when a network experiences latency issues, the \gls{LLM} seamlessly analyze multi-modal information from diverse origins to identify the root cause, e.g. overloaded \gls{UPF} due to insufficient capacity, and then suggest a solution, e.g. step-by-step instructions including suitable code scripts for the involved \glspl{NF} to autonomously reroute traffic or modify policies. Conventional 5G networks can only alert about anomalies using suitable \gls{NWDAF} analytics that track the violated thresholds and notify the \gls{OAM} center to display the details on complex dashboards.

By incorporating \glspl{LLM} (e.g. as \glspl{NF}) into upcoming 6G networks, expected to be designed with \gls{SbD} principles \cite{Khaloopour2024Resilience}, \glspl{LLM} will naturally inherit the same built-in security safeguards rather than adding them as an afterthought. This design-driven approach focuses on proactive threat management, end-to-end encryption, authentication, network slicing isolation, \gls{AI}-driven threat detection with automated reactions, and stateless designs, fostering a resilient \gls{LLM}.
%The design-driven security in 5G and upcoming 6G networks ensures that security is natively integrated at every layer of the architecture rather than added as an afterthought. This approach focuses on proactive threat management, end-to-end encryption, authentication, network slicing, and \gls{AI}-driven threat detection and automated reactions to counter evolving cyber threats.



%!TEX root=main.tex
\section{Motivating Example}\label{sec:motivating-example}
%
\sloppy We motivate and illustrate our technique with a well-known example of  a distributed system, Dijkstra's \emph{dining philosophers} \cite{Dijkstra71}.   There are $n$ philosophers sitting around a table and sharing $n$ forks.  Each philosopher has exactly two forks available, one to each side, shared with the adjacent  philosophers. On the table, there is a bowl of pasta, and the philosophers alternate between thinking and eating.  
\begin{wrapfigure}[17]{r}{.40\textwidth}
 \centering
 \vspace{-0.7cm}
 \includegraphics[scale=0.3]{Figs/philsEx.pdf} % first figure itself
  \caption{LTS $T^i$ modelling philosopher $i$}\label{fig:philsEx}
\end{wrapfigure}
Each philosopher needs both forks to eat, and the access to a fork is (of course) mutually exclusive between its two users.  We want to design a  distributed protocol for this problem  that guarantees deadlock freedom. %, as well as overall \emph{fairness}, i.e., every philosopher is allowed to eat infinitely many times. 
This problem admits many  solutions. For instance,  a way to avoid deadlock is implementing a policy where each philosopher picks up a fork only when her both forks are free.  Another possible solution is obtained  if  each philosopher first takes her left fork and then her right one,  excepting  one philosopher that takes the forks in the opposite order. These are distributed protocols, i.e., they do not depend on a centralised architecture. These are the kind of solutions we expect our technique to synthesize. 
\begin{wrapfigure}[25]{hr}{.25\textwidth}
    \vspace{-0.7cm}
    \includegraphics[scale=0.45]{Figs/TracePhil.pdf} % second figure itself
    \caption{A Path in $T^0 \parallel T^1 \parallel T^2$ leading to deadlock}\label{fig:philsTrace}
\end{wrapfigure}
Let us describe how this problem is specified in our setting.  The behavior of  philosopher's $i$ (for $0 \leq i < n$) is modeled  via the following set $\mathit{PS}^i$ of formulas:
\begin{enumerate}[(1)]
\item \label{phils-form-1} $\forall s : I(s) \Rightarrow \mathit{thk}^i(s) \wedge \neg \mathit{ownLeft}^i(s) \wedge \neg \mathit{ownRight}^i(s)$,
\item \label{phils-form-2}  $\forall s,s' : \neg \mathit{eat}^i(s) \Rightarrow \neg \textit{getThk}^i(s,s')$,
\item \label{phils-form-3} $\forall s,s' : \textit{getThk}^i(s,s') \wedge \mathit{eat}^i(s) \Rightarrow \mathit{thk}^i(s') \wedge \neg \mathit{ownRight}^i(s') \wedge \neg \mathit{ownLeft}^{i }(s') $,
\item  \label{phils-form-4} $\forall s,s': \textit{getHgr}^i(s,s') \Rightarrow \mathit{hgr}^i(s')$,
\item \label{phils-form-5} $\forall s, s':  \neg (\mathit{hgr}^i(s) \land \mathit{ownRight}^i(s) \land \mathit{ownLeft}^i(s)) \Rightarrow \neg \mathit{getEat}^i(s,s')$,
\item \label{phils-form-6} $\forall s,s': \textit{getEat}^i(s,s') \wedge \mathit{hgr}^i(s) \wedge \mathit{ownRight}^i(s) \wedge \mathit{ownLeft}^i(s) \Rightarrow \mathit{eat}^i(s')$,
\item \label{phils-form-7} $\forall s,s': \mathit{getRight}^i(s,s') \wedge \mathit{avFork}_{i+1}(s) \wedge \mathit{hgr}^i(s) \Rightarrow \mathit{ownRight}^i(s')$,
\item  \label{phils-form-8}$\forall s,s': \mathit{getLeft}^i(s,s') \wedge \mathit{avFork}_{i}(s) \wedge \mathit{hgr}^i(s) \Rightarrow \mathit{ownLeft}^i(s')$,
\item \label{phils-form-9}$\forall s \in S : I(s) \Rightarrow (\exists s' \in S: \tcpost{s}{s'} \wedge \mathit{eat}^i(s'))$,
\item \label{phils-form-10}$\forall s :  \mathit{avFork}_i(s)  \Rightarrow (\exists s' : \mathit{chFork_{i}}(s,s') \wedge \neg \mathit{avFork}_i(s'))$,
\item \label{phils-form-11}$\forall s :  \mathit{avFork}_{i+1}(s)  \Rightarrow (\exists s' : \mathit{chFork_{i+1}}(s,s') \wedge \neg \mathit{avFork}_{i+1}(s'))$.
\end{enumerate}

Thus, $\mathit{PS}^i$ describes the \emph{local} specification of philosopher $i$. These formulas may originate from a specification in a domain specific language, referring to actions (e.g., $\mathit{getEat}$), their corresponding guards (e.g., formula \ref{phils-form-5}) and effects (e.g., formula \ref{phils-form-6}). We do not deal with the design of a specification language in this paper, and directly refer to the set of formulas capturing the components behaviors. Notice how the local view of the system state is captured. Each philosopher uses Boolean variables $\mathit{ownLeft}^i$ and $\mathit{ownRight}^i$ for signaling the acquisition of the right and left fork, respectively. Variables $\mathit{avFork}_{i}$ and $\mathit{avFork}_{i+1}$ are used to indicate whether a fork is busy or not (being $+$ the addition modulo $n$).  The actions for philosopher $i$ are: $\mathit{getThk}^i$ (the philosopher goes to the thinking state), $\mathit{getHgr}^i$ (the philosopher gets hungry), $\mathit{getEat}^i$ (the philosopher goes to the eating state). Philosophers also have actions to obtain the forks: $\mathit{getRight}^i$ (obtain the right fork if available), and $\mathit{getLeft}^{i}$ (obtain the left fork of available). %The action definitions are relatively straightforward, as shown above. Note that
Variables $\mathit{avFork}_{i}$ and $\mathit{avFork}_{i+1}$ are shared with other philosophers. 
%\begin{figure}[t!]
%    \centering
%    \begin{minipage}{0.45\textwidth}
%        \centering
%        \includegraphics[scale=0.3,width=0.9\textwidth]{Figs/philsEx.pdf} % first figure itself
%        \caption{LTS $T^i$ modelling philosopher $i$}\label{fig:philsEx}
%    \end{minipage}\hfill
%    \begin{minipage}{0.45\textwidth}
%        \centering
%        \includegraphics[scale=0.4]{Figs/TracePhil.pdf} % second figure itself
%        \caption{A Path in $T^0 \parallel T^1 \parallel T^2$ leading to deadlock}\label{fig:philsTrace}
%    \end{minipage}
%\end{figure}
%
Formula \ref{phils-form-9} in $\mathit{PS}^i$ is a reachability (local) constraint, requiring that philosopher $i$ can  reach the eating state from the initial state. In this example, the forks act as locks: they are shared variables, and thus can be changed by the ``environment'' (from a local philosopher's perspective). This is captured through actions $\mathit{chFork}_{i}$ and $\mathit{chFork}_{i+1}$ that model the acquisition of the locks by the environment. Constraints \ref{phils-form-10}-\ref{phils-form-11} establish that, if a fork is free, it can be grabbed by another philosopher. 

The inputs for our technique include a \emph{global} temporal requirement, constraining the components interaction.  For instance,  in the case  $n=3$ we will have the local specifications $\mathit{PS}^0, \mathit{PS}^1$, and $\mathit{PS}^2$ together with the  {\LTL}  formula:
\begin{multline*}
   \Box \neg (\mathit{ownRight}^0 \wedge \mathit{ownRight}^1 \wedge \mathit{ownRight}^2) 
   \wedge \neg (\mathit{ownLeft}^0 \wedge \mathit{ownLeft}^1 \wedge \mathit{ownLeft}^2),
\end{multline*}
which states that  the system is deadlock-free.  

Note that, for each specification $\mathit{PS}^i$, one can employ a model finder to build an LTS $T^i$ that satisfies specification $\mathit{PS}^i$. Given the expressiveness of the logic, this can only be done up to a given bound on the size of the LTSs (the logic is undecidable). We use the Alloy Analyzer~\cite{AlloyBook}, a bounded model finder for relational logic (which subsumes first-order logic with transitive closure), for this task. Fig.~\ref{fig:philsEx} shows an LTS satisfying $\mathit{PS}^i$, obtained in this way. The dotted arrows represent environment transitions (which, once components are composed, will become actions performed by other philosophers). The global system behavior is obtained by the (asynchronous) parallel composition of the obtained LTSs, denoted  $T^0 \parallel T^1 \parallel T^2$. Now, we can verify whether the global system satisfies the global properties. In our example, if the LTSs $T^0$, $T^1$ and $T^2$ are as shown in Fig.~\ref{fig:philsEx}, then the global system will contain an execution leading to deadlock. 
The trace in Fig.~\ref{fig:philsTrace}, in which the three philosophers get hungry in turns, and then take their corresponding right fork, leads to deadlock. %Since all the forks have been taken, no one of the philosophers is able to pick the left hand side fork to eat. Thus, the system reach a deadlock situation, violating the global property. 

Our approach exploits counterexamples to global properties to guide the search for local implementations.  For instance,  we force at least one of the local implementations to avoid the consecutive execution of $\textit{getHgr}^i$ and $\mathit{getRight}^i$. Assuming that under this new constraint we obtain an LTS $T'^{0}$ for philosopher $0$, that takes the left fork before taking the right one, then the global system $T'^{0} \parallel T^1 \parallel T^2$ is deadlock-free and satisfies the global temporal requirement.


%An example of execution in 	$T_1 \parallel T_2 \parallel T_3$ is show:
%\[
%	(s^0_0, s^1_0, s^2_0) \rightarrow (s^0_4,s^1_0,s^2_0) \rightarrow (s^0_4,s^1_4,s^2_0)  \rightarrow (s^0_4,s^1_4,s^2_4)  \rightarrow (s^0_8,s^1_1,s^2_7) \rightarrow
%\]

%!TEX root=main.tex

\section{Our Model of Concurrency}\label{sec:approach}
We consider finite-state concurrent programs of the form $P_0 \parallel \dots \parallel P_k$ consisting of a finite collection of processes or threads running in parallel, which communicate with each other via shared variables. We use \emph{try-locks} (i.e., non-blocking locks \cite{OaksWong2004}) as the main synchronization mechanism. Try-locks allow for a fine-grained concurrency. Furthermore, we assume that  locks are used to protect shared variables, and then variables can only be written by the processes ``owning'' the corresponding locks. This is a common practice in languages like {\JAVA}, where \emph{synchronize} clauses and locks are used  to provide process synchronization. 

%	We consider finite-state concurrent programs of the form $P_1 \parallel \dots \parallel P_n$ consisting of a finite collection of processes $P_0, \dots, P_n$ or threads running in parallel, which 
%communicate with each other via shared variables. The main mechanism of synchronization used by the processes are \emph{try-locks}, i.e., non-blocking locks \cite{JavaConcurrency,POSIX}. These kinds of locks allow for a fine-grained concurrency. Furthermore, we assume that our programs follow a well-known discipline of synchronization: locks are used to protect shared variables, and then variables can only be written by the process owning the corresponding lock. This is a common practice for instance in \textsf{Java},  where \emph{synchronize} clauses can be used in combination with locks to provide mutual exclusion. 

Formally, a process with shared variables $\mathit{Sh}$ and locks $\mathcal{L}$ will be modeled as a transition system $P = \langle S, \mathit{Act},  \rightarrow, s, \mathit{AP}, L \rangle$, where $\mathit{Act} = \mathit{Int} \cup \mathit{Env}$, $\mathit{AP} = \mathit{Sh} \cup \mathit{Loc} \cup \mathcal{L}$, and $\mathit{Sh}, \mathit{Loc}, \mathcal{L}, \mathit{Int}, \mathit{Env}$ are mutually disjoint finite sets. 
Intuitively, $\mathit{Sh}$ is the collection of shared variables available to the process, $\mathit{Loc}$ is the collection of the local variables of the process, $\mathcal{L}$ is the set of locks used by the process, $\mathit{Int}$ is a set of internal actions and $\mathit{Env}$ is a set of environmental actions. From now on, we assume that every shared variable $g$ has a corresponding lock (named $\ell_g$), and therefore we have that $\{\ell_g \mid g \in Sh\} \subseteq \mathcal{L}$. %When useful we use the expression $\mathit{assoc}(\ell)$ to denote the variable associated to lock $\ell$, if it exists. 

The environmental actions are defined taking into account the shared variables and the locks, i.e.:
$
	\textit{Env} = \bigcup_{\ell \in \mathcal{L}}\{\textit{ch}_\ell\} \cup  \bigcup_{g \in \textit{Sh}}\{\textit{ch}_g\} \cup \{\textit{env}\}.
$
Intuitively,  $\textit{ch}_\ell$ is an environmental action that performs changes on lock $\ell$, and $\textit{ch}_g$ is similar but for shared variables. The additional action $\textit{env}$ represents the possible changes that the environment may perform over the shared variables not owned by the current process.

%Furthermore, for each lock $\ell$ we consider two additional atomic propositions, named $\mathit{own}_\ell$ and $\mathit{av}_\ell$; intuitively, the former signals the acquisition of lock $\ell$ and the latter becomes true when the lock is available. Also, we provide an environmental action $ch_\ell \in Env$ that changes the state of the lock and its (possible) associated value.
%		We provide actions to acquire the locks in boththe process and the environment; so, for any lock $\ell$, $aqc_\ell, rel_\ell \in Int$ are actions for acquiring and releasing $\ell$, and $ch_\ell \in Env$ is an environmental action   
For the sake of simplicity, we write $s \overset{a}{\dashrightarrow} s'$ instead of $s \xrightarrow{a} s'$ when $a \in Env$, and we write $s \dashrightarrow s'$ if $s \overset{a}{\dashrightarrow} s'$ for some $a$.

We impose the following restrictions on processes to ensure that they behave according to our model of concurrency:
%		Formally, a process is a tuple $P = \langle S, Sh, Loc, \mathcal{L}, Act, Env, \rightarrow, L \rangle$ where $S$ is a finite set of states; $Loc$ and $Sh$ are finite sets of local and shared variable names, respectively. We assume that each  variable $x$ range over a domain $D_x$ being $D$ the union of all these domains; $\mathcal{L} = \{\ell_0, \dots, \ell_n\}$ is a finite set of lock names, we assume $D_\mathcal{L} = \{true, false\}$; $Act$ and $Env$ are finite sets of internal and external action, respectively. 		
%		$\rightarrow \subseteq (Act \cup Env) \times S \times S$ is a transition relation, we write $s \xrightarrow{a} s'$ instead of $(a, s, s') \in \rightarrow$, and when $a \in Env$ we write
%$s \overset{a}{\dashrightarrow} s'$; $L: S \rightarrow D^{(Sh \cup Loc) \cup \mathcal{L}}$ is a valuation function returning the value of each variable in a given state. We assume that every shared variable $g$ has a corresponding lock (named $\ell_g$), and therefore we have that $\{\ell_g \mid g \in Sh\} \subseteq \mathcal{L}$.	Note that any process is basically a transition system, for a state $s \in S$ we denote $Post(s) = \{s' \mid \exists a : s \xrightarrow{a} s'\}$ the sets of successors of state $s$, and $Post_{Env}(s) = \{s' \mid \exists a : s \overset{a}{\dashrightarrow} s'\}$, the successors of $s$ reached via environmental transitions.
% locks can only be acquired when they are free
\begin{description}[font=\normalfont]
	%\item[P1.] $\forall s \in S : \neg L(s)(av_\ell) \Rightarrow \neg (\exists s' : s \rightarrow s' \wedge L(s')(own_\ell))$
	\item[P1.] $\bigwedge_{\ell \in \mathcal{L}}(\forall s \in S :  \textit{av}_\ell \notin L(s) \Rightarrow \neg (\exists s' : s \rightarrow s' \wedge \textit{own}_\ell \in L(s')))$
	%\item[P2.] $\forall s,s' \in S : s \rightarrow s' \wedge \neg L(s)(own_\ell) \wedge L(s')(own_\ell) \Rightarrow$\\
	%		\hspace*{5cm}  $\bigwedge_{\ell' \in \mathcal{L}\setminus\{\ell\}} (L(s)(own_{\ell'}) \equiv L(s')(own_{\ell'}))$
	\item[P2.] $\bigwedge_{\ell \in \mathcal{L}}(\forall s,s' \in S : s \rightarrow s' \wedge \textit{own}_\ell \notin L(s) \wedge \textit{own}_\ell \in L(s') \Rightarrow$\\
			\hspace*{5cm}  $\bigwedge_{\ell' \in \mathcal{L}\setminus\{\ell\}} (own_{\ell'} \in L(s) \equiv \textit{own}_{\ell'} \in L(s')))$
	%\item[P3.] $\forall s \in S: L(s)(own_\ell) \Rightarrow \neg L(s)(av_\ell)$
	\item[P3.] $\bigwedge_{\ell \in \mathcal{L}}(\forall s \in S: \textit{own}_\ell \in L(s) \Rightarrow   \textit{av}_\ell \notin L(s))$
	%\item[P4.] $\forall s \in S : \neg L(s)(own_\ell) \Rightarrow \exists s' \in S : s \overset{ch_\ell}{\dashrightarrow} s'$
	\item[P4.] $\bigwedge_{\ell \in \mathcal{L}}(\forall s \in S : \textit{own}_\ell \notin L(s) \Rightarrow \exists s' \in S : s \overset{\textit{ch}_\ell}{\dashrightarrow} s')$
	%\item[P5.] $\forall s,s' \in S : s \overset{ch_\ell}{\dashrightarrow}{s'} \Rightarrow L(s)(av_\ell) \equiv \neg L(s')(av_\ell)$ 
	\item[P5.] $\bigwedge_{\ell \in \mathcal{L}}(\forall s,s' \in S : s \overset{\textit{ch}_\ell}{\dashrightarrow}{s'} \Rightarrow \textit{av}_\ell \in L(s) \equiv  \textit{av}_\ell \notin L(s'))$ 	
	%\item[P6.] $\forall  s,s' \in S : s \overset{ch_\ell}{\dashrightarrow} s' \Rightarrow \bigwedge_{x \in Sh \cup Loc \setminus \{av_{\ell}, own_{\ell}\}\cup assoc(\ell)} (L(s)(x) \equiv L(s')(x))$
	%\item[P6.] $\bigwedge_{\ell \in \mathcal{L}}(\forall  s,s' \in S : s \overset{\textit{ch}_\ell}{\dashrightarrow} s' \Rightarrow \bigwedge_{x \in Sh \cup Loc \setminus \{\textit{av}_{\ell}, \textit{own}_{\ell}\}} (x \in L(s) \equiv x \in L(s')))$
	\item[P6.] $\bigwedge_{\ell \in \mathcal{L}}(\forall  s,s' \in S : s \overset{\textit{ch}_\ell}{\dashrightarrow} s' \Rightarrow \bigwedge_{x \in Sh \cup Loc \setminus \{\textit{ch}_\ell, \textit{av}_\ell\}}(x \in L(s) \equiv x \in L(s')))$

	\item[P7.] $\bigwedge_{g  \in \textit{Sh}}(\forall  s,s' \in S : s \overset{\textit{ch}_g}{\dashrightarrow} s' \Rightarrow \bigwedge_{x \in (Sh \cup Loc) \setminus \{g\}}(x \in L(s) \equiv x \in L(s')))$
	
	%\item[P7.] $\forall s \in S : \neg L(s)(own_{\ell_g})  \Rightarrow$ \\
	%		\hspace*{3cm}$(\exists s': s \dashrightarrow s' : L(s')(g)) \wedge  (\exists s': s \dashrightarrow s' :  \neg L(s')(g)))$
	\item[P8.] $\bigwedge_{g \in \textit{Sh}}(\forall s \in S :  \textit{own}_{\ell_g} \notin L(s) \wedge \textit{av}_{\ell_g} \notin L(s) \Rightarrow$ \\
			\hspace*{3cm}$(\exists s': s \overset{ch_g}{\dashrightarrow} s' : g \in L(s')) \wedge  (\exists s': s \overset{ch_g}{\dashrightarrow} s' : g \notin L(s')))$
	
	\item[P9.] $\textit{env} = (\bigcup_{g \in \textit{Sh}}\{\textit{ch}_g\})^+$
	%\item[P8.] $\forall s,s': s \dashrightarrow^* s' \wedge (\bigwedge_{\ell \in \mathcal{L}}  \textit{own}_{\ell} \in L(s) \equiv \textit{own}_{\ell} \in L(s')) \Rightarrow s \dashrightarrow s'$
\end{description}
%Note that it is straightforward to express these formulas in \textsf{FORL}.
Intuitively, $\text{P1}$ states that, if a lock is not available, then it cannot be acquired. $\text{P2}$ says that only one lock per time unit can be acquired by the process. $\text{P3}$
says that, if a lock is owned by the process, then it is not available. $\text{P4}$ states that, if the process does not own a lock, then the environment can acquire it. $\text{P5}$ expresses the fact that 
action $\textit{ch}_\ell$ changes the state of the lock. $\text{P6}$ states that the environmental action $\textit{ch}_\ell$ produces changes only in variables $\textit{own}_\ell$ and $\textit{av}_\ell$. $\text{P7}$ is similar but for shared variables.
$\text{P8}$ says that, if the lock corresponding to a variable is not owned by the process, then any behavior of the environment is possible. Finally, $\text{P9}$ states that $\textit{env}$ models the possible changes that the environment may perform over shared variables. This condition makes it possible to promote local properties to the global system (Theorem \ref{th:prom} below).%	Some additional  comments about our definition of process are useful. Note that 
%Intuitively, the states of a process can be identified with parts of its source code, where each of these parts may execute a sequential code which we abstract away. To perform a given transition some locks may be needed (i.e., the proposition $own_\ell$ is true in the source of the transition); this is the synchronization part.  

%We also consider all possible behaviors of the environment ($\text{P8}$-$\text{P9}$). This is similar to the assumptions made in \cite{PnueliRosner89} for the synthesis of reactive controllers over open systems. Here, we restricted ourselves to boolean variables, other datatypes can be dealt with in a similar way.  
Note that $\text{P8}$-$\text{P9}$ are similar to the assumptions made in \cite{PnueliRosner89} for the synthesis of reactive controllers over open systems. Here, we restricted ourselves to boolean variables, other datatypes can be dealt with in a similar way. 

	Let us define the transition structure corresponding to the concurrent execution of a collection of processes.
\begin{definition}
 Given processes $P_0,\dots,P_k$ 
where $P_i = \langle S_i,\mathit{Env}_i \cup \mathit{Int}_i, \mathit{Sh} \cup \mathit{Loc}_i \cup \mathcal{L},  \rightarrow_i, s_i, L_i \rangle$ for $i \in [0,k]$ with shared variables $\mathit{Sh}$
 and locks $\mathcal{L}$, with $\mathit{Loc}_i, \mathit{Loc}_j, \mathit{Act}_i, \mathit{Act}_j$ being pairwise disjoint for $i \neq j$, we define 
%the structure $P_0 \parallel \dots \parallel P_n = \langle S, \bigcup_{i \in [0,k]} Act_i, Sh \cup \bigcup_{\ell \in \mathcal{L}}\{own^i_\ell, av^i_\ell \} \setminusbigcup_{\ell \in \mathcal{L}}\{own_\ell, av_\ell\} \cup \bigcup_{i \in [0,k]}Loc_i , \rightarrow, L \rangle$  as follows:
the structure $P_0 \parallel \dots \parallel P_n = \langle S^\parallel, \mathit{Act}^\parallel, \rightarrow^\parallel, s^\parallel , \mathit{AP}^\parallel , L^\parallel, \rangle$  as follows:
\begin{enumerate}	
	\item $S^\parallel = \{s \in \Pi_{i \in [0,k]} S_i \mid \forall g \in \mathit{Sh} : \forall i,j \in [0,k] : L_i(s \proj i)(g) = L_j(s \proj j)(g) \}$\\
			\hspace*{0.8cm}$ \cap \{ s \in \Pi_{i \in [0,k]} S_i \mid \forall \ell \in \mathcal{L} : \#\{i \in [0,k] \mid L_i(s \proj i)(\mathit{own}_\ell)\} \leq 1 \}$,
	\item $\mathit{Act}^\parallel =  \bigcup_{i \in [0,k]} \mathit{Act}_i$,
	\item $\rightarrow^\parallel  = \{ (s,a,s') \mid \exists i \in [0,k] : ((s \proj i \xrightarrow{a}_i s' \proj i) \wedge (\forall j \neq i : s \proj j = s \proj j)) \}$,
	\item $s^\parallel  = \langle s_0, \dots, s_k \rangle$,	
	\item $\mathit{AP}^\parallel = ((\mathit{Sh} \cup \bigcup_{\ell \in \mathcal{L},i\in [0,k]}\{\mathit{own}^i_\ell, av^i_\ell \}) \setminus \bigcup_{\ell \in \mathcal{L}}\{\mathit{own}_\ell, av_\ell\}) \cup \bigcup_{i \in [0,k]}Loc_i$,
	\item $L^\parallel(s)(x)  =  L_0(s \proj 0)(x) \mbox{ if } x \in \mathit{Sh} \setminus \{\mathit{own}_\ell, \mathit{av}_\ell\}$,
	\item $L^\parallel(s)(x) =  L_i(s \proj i)(x) \mbox{ if } x \in Loc_i$,
	\item $L^\parallel(s)(\mathit{own}^i_\ell)  =  L_i(s \proj i)(\mathit{own}_\ell)$,
	\item $L^\parallel(s)(\mathit{av}^i_\ell)  =  L_i(s \proj i)(\mathit{av}_\ell)$.
\end{enumerate}
\end{definition}
Roughly speaking, the global system is an asynchronous product of the participating processes, where  the valuation of shared variables coincides for every process, and each lock is mapped to its owner. It is easy to see that this construction suffers from the state explosion problem, thus the explicit construction of this structure is unfeasible in practice.

One interesting point about our formalization  is that certain properties can be promoted from a given process to the concurrent program where this process resides, this enables modular reasoning over asynchronic products.
%which can be addressed with techniques such as symbolic model checking or bounded model checking.
%However, in practice, one can characterise the executions of the asynchronous product using the definitions of $S^\parallel$ and $\rightarrow^\parallel$, and so bounded model checking can be used to efficiently model check \textsf{LTL} properties over this structure.
		
% TRY TO PROVE THE THEOREM FOR ACTL-X
% Granularity of locks: Only one lock can be aqcuired in one step and only one shared variable can be update in one step, assume that there are
% local actions and environmental actions to acquire and change shared variables
%	Interestingly, the next theorem states   that  local properties without the next operator are promoted from processes to global systems, this makes possible modular reasoning over the asynchronous product.
\begin{theorem}\label{th:prom} Given processes $P_0,\dots, P_k$ such that $P_0 \parallel \dots \parallel P_k$ is well formed and a \textsf{LTL-X} property $\phi$, if $P_i \vDash \phi$, then we have: $P_0 \parallel \dots \parallel P_n \vDash_\mathcal{F} \phi$. 
%\begin{proof}  First, we prove that for any trace $\pi' \in Tr(P_0 \parallel \dots \parallel P_k)(s)$ we have that the projection of this trace to process $P_i$ is stutter equivalent to some trace $\pi \in Tr(P_i)(s \proj i)$. Since stuttering equivalence preserves path formulas without the next operator,  the result follows. 
%
%	Let $\pi' \in Tr(P_0 \parallel \dots \parallel P_k)(s)$, then we prove that the trace $\pi \proj i = \pi[0] \proj i \rightarrow \pi[1] \proj i \rightarrow \dots$ is stutter equivalent to a trace 
%$\pi \in Tr(P_i)(s \proj i )$. $\pi$ is just defined by removing all the transitions in $\pi[j] \proj i \rightarrow \pi[j+1] \proj i$  that
%do not correspond to transitions in $P_i$. It is simple to see that the removed transitions are stuttering steps, that is $L_i(\pi[j] \proj i)  = L_i(\pi[j+1] \proj i)$, otherwise if $L_i(\pi[j] \proj i)  \neq L_i(\pi[j+1] \proj i)$
%the transition must correspond to the transition of another process (say $P_j$). Furthermore, $L_i(\pi[j] \proj i)$ and $L_i(\pi[j+1] \proj i)$ can only differ on their valuation of shared variables (they must coincide on the valuation of $Int_i$), but by condition \textbf{P7} we have a matching transition in $P_i$, which is a contradiction. Also note that removing these transitions keeps the trace infinite, since $\pi$ is fair.
%	Since we have removed only stuttering steps, the resulting execution is stuttering equivalent to $\pi \proj i$. Using this property and an induction on \textsf{ACTL-X} formulas, the result follows.
%\end{proof}
\end{theorem}
%
%Intuitively, this property states that universal local properties without the next operator are promoted from processes to global systems. As noted in \cite{Lamport83}, the next operator is unsuitable  to state global properties since stuttering steps during the execution of the global system may falsify them. Furthermore, 
Note that local properties are preserved only by fair executions, otherwise one could devise some global execution that prevents the process from progressing. 
\subsection{Obtaining Guarded-Command Programs.}
	Interestingly, given a structure $P_0 \parallel \dots \parallel P_k$  we can define a corresponding program in the guarded-command notation, written $Prog(P_0 \parallel \dots \parallel P_k)$, as follows. The shared variables are those in $\textit{Sh}$ plus an additional shared variable $\ell$ for each lock, with domain $[0,k]\cup\{\bot\}$ (where $\bot$ is a value used to indicate that the lock is available). Additionally, for each $\langle S_i, \textit{Env}_i \cup \textit{Int}_i, \textit{Sh} \cup \textit{Loc}_i \cup \mathcal{L},  \rightarrow_i, s_i, L_i \rangle$ we define a corresponding process. To do so, we introduce for each $s \in S_i$  the equivalence class: $[ s ]  = \{ s' \in S_i \mid (s \dashrightarrow^* s') \vee (s'  \dashrightarrow^* s)\}$. 
That is, it is the set of states connected to $s$ via environmental transitions. It is direct to prove that it is already an equivalence class. The collection of all equivalence classes
is denoted $S_i /_{\dashrightarrow^*}$.The local variables of the process are those in $\textit{Loc}_i$. Additionally, a fresh variable $\textit{state}$, with domain $S_i/_{\dashrightarrow^*}$, is considered. 
It is worth noting that the programs we synthesize follow the discipline of acquiring a lock before modifying the corresponding variable. When a lock is not available, the process may continue executing other branches. However, note that a process could get blocked when all its guards are false, thus other synchronization mechanisms such as blocking locks, semaphores and condition variables can be expressed by these programs. 
Finally, given states $s,s' \in S_i$ with $s \xrightarrow{a} s'$ and  $[s] \neq [s']$, we consider the following guarded command:
\[
 [a] \left( \begin{array}{l} 
 			 state = [s] \\
			 \wedge \bigwedge \{x = (x \in L(s)) \mid x \in \mathit{Sh} \cup \mathit{Loc} \}\\ 
			\wedge \bigwedge \{ \ell = i \mid \ell \in \mathcal{L} : \mathit{own}_\ell \in L(s)\}	 \\ 
			 \wedge \bigwedge \{ \ell = \bot \mid \ell \in \mathcal{L} : \mathit{av}_\ell \in L(s)\}  \end{array} \right) \rightarrow \left( \begin{array}{l} 
 																					  			  \{x {:=} x \in L(s') \mid x \in \mathit{Loc}\cup \mathit{Sh}\} \\ 
																					 			  \cup \{\mathit{state} {:=} [s'] \} \\
																								  \cup \{ \ell := i \mid \ell \in \mathcal{L} : \mathit{own}_\ell \in L(s)\}  \\
																								  \cup \{ \ell := \bot \mid \ell \in \mathcal{L} : \mathit{av}_\ell \in L(s)\}
																					  \end{array} \right)
\]
%	Note that this definition may result in programs with redundant branches, which can be simplified in different ways. 
We can prove that our translation from transition systems to programs is correct. That is, the executions of the program satisfy the same temporal properties as the 
asynchronous product $P_0 \parallel \dots \parallel P_n$.
\begin{theorem}\label{th:proppreservation} Given a program $P_0 \parallel \dots \parallel P_n$ and \textsf{LTL} property $\phi$, then we have that:
$
	P_0 \parallel \dots \parallel P_n \vDash \phi \Leftrightarrow Prog(P_0 \parallel \dots \parallel P_n) \vDash \phi
$
\end{theorem}
\begin{example}[Mutex]  
\label{ex:mutex-ts}
Consider a system composed of two processes (it can straightforwardly be generalized to $n$ processes) both with non-critical, waiting and critical sections. The global property  is mutual exclusion: the two processes cannot be in their critical sections simultaneously. We consider one lock $\ell$, actions: $\mathit{getNCS}$ (the process enters to the non-critical section), $\mathit{getCS}$ (the process enters to the critical section), $\mathit{getLock}$ (the process acquires the lock), $\mathit{getTry}$ (the process goes to the try state), and the corresponding propositions $\mathit{ncs}, \mathit{cs}, \mathit{try}, \mathit{own}_\ell, \mathit{av}_\ell$.  

In Fig.~\ref{fig:mutex} the transition system and the program corresponding to this example are shown. It is interesting to observe that,  the coherence conditions \text{P1}-\text{P9} imply that, for any action $B \rightarrow C$  containing a $x := E$ in $C$ for a shared variable $x$, we have that the statement $\ell_x=i$ is implied by $B$ (if $x:=E$ is different from the skip statement), i.e., the lock corresponding to $x$ is acquired by the process before executing the assignment. Similarly, note that locks are acquired only if they are available.
\end{example}
\begin{figure}[t!]
\begin{minipage}[b]{0.40\linewidth}
\centering
\includegraphics[scale=0.6]{Figs/mutex-resized.pdf}
\end{minipage}
\begin{minipage}[b]{0.60\linewidth}
\centering
\begin{lstlisting}[style=Unity]
Program Mutex
 var m:Lock;
  Process $P_i$ with $i \in \{0,1\}$
   var $\text{try}_i, \text{ncs}_i, \text{cs}_i$:boolean
   var st:$\{\text{S0},\text{S1},\text{S2},\text{S3},\text{S4},\text{S5}\}$
   initial: $\text{ncs}_i\wedge \neg \text{cs}_i \wedge \neg \text{try}_i$ 
   begin
    [getTry]st=S5$\wedge$Av$_m\rightarrow$st:=S3,$\text{try}_i$:=true,$\text{ncs}_i$:=false
    [getLock]st=S3$\wedge$Av$_m \rightarrow$st:=S1,$\text{try}_i$:=true,own$_m$:=true
    [getLock]st=S3$\rightarrow$st:=S3
    [getCS]st=S1$\rightarrow$st:=S0,$\text{cs}_i$:=true,$\text{try}_i$:=false
    [getNCS]st=S0$\rightarrow$st:=S5,$\text{ncs}_i$:=true,$\text{cs}_i$:=false
   end
end
\end{lstlisting}
\end{minipage}
\caption{Transition Structure and Program with a lock $m$ for Mutex}\label{fig:mutex}
\end{figure}
%
%	We can prove that our translation from transition systems to programs is correct. That is, the executions of the program satisfy the same temporal properties as the 
%asynchronous product $P_0 \parallel \dots \parallel P_n$.
%\begin{theorem}\label{th:proppreservation} Given a program $P_0 \parallel \dots \parallel P_n$ and \textsf{LTL} property $\phi$, then we have that:
%$
%	P_0 \parallel \dots \parallel P_n \vDash \phi \Leftrightarrow Prog(P_0 \parallel \dots \parallel P_n) \vDash \phi
%$
%\end{theorem}
%such that, for every $\pi \in Tr(P_0 \parallel \dots \parallel P_n)$ and $i \geq 0$ we have that: $\pi[i] \vDash \phi$ iff $f(\pi)[i] \vDash \theta(\phi)$, being $\phi$ any boolean formula. 
% In order to prove this, we need to consider a translation (named $\theta$) between boolean formulas built up from the transition structure and the  program's boolean expressions.
%It is defined recursively as follows: $\theta(own_\ell) = (\ell = i)$, $\theta(av_\ell) = (\ell = \bot)$, $\theta(p) = p$ (for any $p \in Loc \cup Sh$), and $\theta(\varphi \vee \psi) = \theta(\varphi) \vee \theta(\psi)$, $\theta(\varphi \wedge \psi) = \theta(\varphi) \wedge \theta(\psi)$, $\theta(\neg \varphi) = \neg \theta(\varphi)$. Then, we can prove the following theorem:
%\begin{theorem}\label{th:proppreservation} Given a program $P_0 \parallel \dots \parallel P_n$, we have a one-to-one function $f: Tr(P_0 \parallel \dots \parallel P_n) \rightarrow Tr(Prog(P_0 \parallel \dots \parallel P_n))$,
%such that, for every $\pi \in Tr(P_0 \parallel \dots \parallel P_n)$ and $i \geq 0$ we have that: $\pi[i] \vDash \phi$ iff $f(\pi)[i] \vDash \theta(\phi)$, being $\phi$ any boolean formula. 
%\begin{proof} 
%	Let us define for each $\pi \in Tr(P_0 \parallel \dots \parallel P_k)$ a corresponding execution $f(\pi) \in Tr(Prog(P_0 \parallel \dots \parallel P_k))$. The definition of $f(\pi)$ is by induction.
%$\pi[0]$ corresponds to the initial state of the program, by definition the two satisfy the same boolean formulas. Assume that $f(\pi[0]) \dots f(\pi[n])$ are defined, and consider $\pi[n+1]$. We know that there is a transition $\pi[i] \xrightarrow{a} \pi[i+1]$, then by definition of the asynchronous product we have a process $P_j$ with a transition
%$\pi[i] \proj j \xrightarrow{a} \pi[i+1] \proj j$, if $[\pi[i]] = [\pi[i+1]]$ then we define $f(\pi[i+1]) = f(\pi[i])$, otherwise we have a corresponding guarded command $B \rightarrow C$ in the process 
%corresponding to $P_j$ such that, by induction, $f(\pi[i]) \vDash B$  and we define $f(\pi[i+1])$ as the state obtained after executing $C$, which by definition of the program satisfies the same
%boolean properties as $\pi[i+1]$. In a similar way we can define the inverse of $f$.
%\end{proof}
%\end{theorem}
\vspace{-0.5cm}
\subsection{Defining Processes in \textsf{FORL}.}
	Given a process as defined above, we can easily describe it in \textsf{FORL}. A type $\mathit{Node}$ is used for modeling the states, the transition relation is described using binary relations, propositions are captured using a type $\mathit{Prop}$, and the valuation function is defined via a relation $\mathit{val}:\mathit{Node} \rightarrow \mathit{Prop}$. Finally, formulas are used to describe the transitions. Let us describe this by means of an example. Consider the process of Fig.\ref{fig:mutex},  the corresponding $\textsf{FORL}$ specification is shown in Fig.\ref{fig:forl-spec}.
\begin{figure}[t!]
$%\scriptsize % finish this
\begin{array}{l}
	\textit{succs},\textit{local}, \textit{env}:\textit{Node} \times \textit{Node}\\
	\textit{val}: \textit{Node} \times \mathit{Prop} \\
	s_0,s_1,s_2,s_3,s_4,s_5:\textit{Node}\\
	\textit{val} = \{ (s_5 ,  \textit{ncs}_i), (s_5, \textit{av}_m), (s_2, \mathit{ncs}_i), (s_4, \mathit{try}_i) 
	, (s_3, \mathit{try}_i), (s_3, \mathit{av}_m) \\ 
	\hspace*{0.7cm},  (s_1, \mathit{try}_i), (s_1,\mathit{own}_m),  (s_0, \mathit{cs}_i), (s_0, \mathit{own}_m) \}\\
	\textit{getTry} = \{(s_5, s_3),(s_2, s_4) \}\\
	\textit{getNCS} = \{s_0,s_5 \} \\ 
	\textit{getCS} = \{ (s_1, s_0)\}\\
	\textit{getLock} = \{ (s_3, s_1), (s_3, s_3),(s_4,s_4) \} \\
	\textit{ch}_{\ell} = \{ (s_3, s_4),(s_4, s_3),(s_5,s_2), (s_2,s_5) \}\\
	\textit{local} =  \mathit{getTry} \cup \mathit{getNCS}  \cup \mathit{getCS} \cup \mathit{getLock}\\
	%\textit{env} = \textit{ch}_{\ell} \\
	%\textit{succs} =   local \cup env \\
\end{array}
$
%\begin{array}{l}
%	\mathit{succs},\mathit{local}, \mathit{env}:\mathit{Node} \rightarrow \mathit{Node}\\
%	\mathit{val}: \mathit{Node} \rightarrow \mathit{Prop} \\
%	s_0,s_1,s_2,s_3,s_4,s_5:\mathit{Node}\\
%	\mathit{val} = (s_5 {\rightarrow} \textit{ncs}_i) + (s_5 \rightarrow \textit{av}_m) + (s_2 \rightarrow \mathit{ncs}_i) + (s_4 \rightarrow \mathit{try}_i) 
%	+ (s_3 \rightarrow \mathit{try}_i) + (s_3 \rightarrow \mathit{av}_m) \\ 
%	\hspace*{0.7cm}+  (s_1 \rightarrow \mathit{try}_i) +  (s_1 \rightarrow \mathit{own}_m)	+  (s_0 \rightarrow \mathit{cs}_i) +  (s_0 \rightarrow \mathit{own}_m) \\
%	\mathit{getTry} = (s_5 \rightarrow s_3)  + (s_2 \rightarrow s_4)\\
%	\mathit{getNCS} = s_0 \rightarrow s_5 \\ 
%	\mathit{getCS} = s_1 \rightarrow s_0\\
%	\mathit{getLock} = (s_3 \rightarrow s_1) + (s_3 \rightarrow s_3) + (s_4 \rightarrow s_4) \\
%	\mathit{ch}_{\ell} = (s_3 \rightarrow s_4) + (s_4 \rightarrow s_3) + (s_5 \rightarrow s_2) + (s_2 \rightarrow s_5)\\
%	\mathit{local} =  \mathit{getTry} + \mathit{getNCS}  + \mathit{getCS} + \mathit{getLock}\\
%	\mathit{env} = \mathit{ch}_{\ell} \\
%	\mathit{succs} =   local + env 
%\end{array}
\caption{\textsf{FORL} specification for Mutex}\label{fig:forl-spec}
\end{figure}
Note that we use variables $\mathit{succs}$ and $\mathit{env}$ to capture the relations $\rightarrow$ and $\dashrightarrow$, respectively. Given a transition structure $\mathit{TS}$ we call $\mathcal{A}(\mathit{TS})$ its corresponding specification. Note that, if we replace $\mathit{getTry} = \{ (s_5,s_3), (s_2,s_4)\}$ by $\mathit{getTry} \subseteq \{ (s_5,s_3), (s_2,s_4)\}$  in the given specification, we obtain a weaker specification. Performing this for all the internal actions returns a specification which, intuitively, 
allows the SAT solver to disable some local transitions, thus the non-determinism may be reduced. This weaker theory is denoted by $\mathcal{A}_{Ref}(TS)$.

%	We also can consider a weaker specification if $\mathit{getCS} = s_1 \rightarrow s_0$ is replaced
%by $\mathit{getCS} \subseteq s_1 \rightarrow s_0$, and similarly for the other local actions. Intuitively, this specification allows one to disable some local transitions, and in some cases to reduce the non-determinism. We name $\mathcal{A}_{Ref}(TS)$  this weaker theory.
%\section{Process Specifications}\label{sec:spec}
%	In this section we describe the main way in which we specify distributed programs.
%\paragraph{Theory Presentations, Specifications and Kripke Structures.} 
%	The main vehicle to express process specifications are \textsf{FO(TC)} theory presentations. 
	%A theory presentation is a tuple $T=\langle \tau, \Gamma \rangle$ where $\tau$ is a vocabulary and $\Gamma$ is a set of \textsf{FO(TC)}  formulas. We say that $T$ is finite when $\Gamma$ is finite. Given a theory $\langle \tau, \Gamma \rangle$, we say that $\mathcal{A} \vDash T$ iff $\mathcal{A} \vDash \Gamma$.
\subsection{Specifying Distributed Programs} 
	Distributed programs are specified by means of a  collection of \textsf{FORL} specifications that share  some symbols (locks and shared variables) plus a global temporal property.
\begin{definition} A specification of a distributed program is a tuple: $\langle Sh, \mathcal{L}, \{ S_i \}_{i \in I}, \phi \rangle$ where $Sh=\{g_0,\dots, g_p\}$ is a finite collection
of shared variable names, $\mathcal{L} =\{\ell_0,\dots,\ell_q\}\cup \{\ell_{g_i} \mid 0 \leq i \leq p \}$ is a finite collection of lock names including names for the shared variables and $I$ is a finite index set. Furthermore, each specification $S_i$ contains in its declaration part:
\begin{itemize}
	\item Types $\mathit{Node}, \textit{Prop}$ characterising the process' states, and the set of propositions, respectively.
	\item Variables $\mathit{init}{:}\mathit{Node}$ and $p,q,\dots{:}Prop$ characterizing the initial state and the propositions, respectively,
	\item Relations  $a_0, \dots, a_m{:} \mathit{Node} \rightarrow \mathit{Node}$ and  ${\mathit{ch}_g}_0, \dots, {\mathit{ch}_g}_0{:} \mathit{Node} \rightarrow \mathit{Node}$, representing the 
	internal actions, and the environmental actions, respectively.
%	representing the relations associated to internal actions,
%	\item Relations  ${\mathit{ch}_g}_0, \dots, {\mathit{ch}_g}_0{:} \mathit{Node} \rightarrow \mathit{Node}$ representing the relations associated to environmental actions that change the shared variables,
	\item Relations  $p_0,\dots,p_t{:}\mathit{Node} \rightarrow \mathit{Prop}$  corresponding to the local variables, and ${\mathit{ch}_g}_0, \dots, {\mathit{ch}_g}_0{:} \mathit{Node} \rightarrow \mathit{Node}$ corresponding to the shared variables,
%	\item Relations $g_0, \dots, g_p{:} \mathit{Node} \rightarrow \mathit{Prop}$ corresponding to the shared variables,
	\item Relations $\{\mathit{av}_\ell \mid \ell \in \mathcal{L}\} \cup \{\mathit{own}_\ell \mid \ell \in \mathcal{L}\}$ of type $\mathit{Node} \rightarrow \mathit{Prop}$  representing the lock mechanisms associated to the shared variables. 
%	\item Unary relation symbols $\text{av}^1_{\ell_0}, \dots, \text{av}^1_{\ell_p}, \text{own}^1_{\ell_0}, \dots, \text{own}^1_{\ell_p}$ representing the lock mechanisms associated to the shared variables. 
 \end{itemize}
	$S_i$ contains axioms $\text{P1}$-$\text{P9}$.
Furthermore, $\phi$ is a {\LTLX} formula  containing propositional variables declared in the $S_i's$.
\end{definition}
	Given a program specification $\langle  Sh, \mathcal{L}, \{S_i\}, \phi  \rangle$, an instance of it is a collection of  environments 
$\{ e_i  \}_{i \in I}$ such that: $e_i \vDash S_i$. Furthermore, let $TS_{e_i}$ be the transition structure corresponding to  $e_i$, then we have $TS_{e_0} \parallel \dots \parallel TS_{e_k} \vDash \phi$. 
Typically, a process specification contains axioms for the actions in the style of pre and postconditions, plus frame axioms (i.e., the list of variables that are not changed by the actions), in addition to  axioms \text{P1}-\text{P9}. 
\begin{example}[Mutex cont.] Let us consider a specification for the mutex example with two processes. The specification is given by a tuple $\langle \emptyset, \{m\}, \{ S_0, S_1 \}, \phi \rangle$ where there is no shared variables, a lock $m$, specifications $S_0$ and $S_1$ corresponding to the two processes, whose variable
declarations are those in Example \ref{ex:mutex-ts}. The formulas of the specification define the actions in terms of pre and post-conditions, e.g.:
\[
\begin{array}{l}
	\forall s,s' : \textit{Node} \mid s' \in \textit{enterCS}_i[s] \Rightarrow \\
	 \hspace*{3cm} (((\textit{try}_i \in \textit{val}[s]) \wedge ({\textit{own}_m}_i \in \textit{val}_i[s])) \Rightarrow cs \in \textit{val}_i[s']))
\end{array}
%\begin{array}{l}
%	\mathbf{all} \ s,s' {:} \mathit{Node} \mid s' \ \mathbf{in} \ \mathit{enterCS}_i[s] \ \mathbf{implies} \\
%	\hspace*{1cm}	 (((try_i \ \mathbf{in} \ val[s]) \ \mathbf{and} \ ({own_\ell}_i \ \mathbf{in} \ val[s])) \ \mathbf{implies} \ cs \ \mathbf{in} \ val[s']))
%\end{array}
\]
states the pre and postcondition of action $\mathit{enterCS}$ (and similarly for the other actions). The global property $\phi$ is \emph{mutual exclusion}, which can be easily characterised in {\LTL}.
%\[
%\( \exists s: \textit{Node} \mid \textit{cs}_i \in \textit{val}_i[s] \wedge \textit{cs}_j \in \textit{val}_j[s] \wedge j \neq i )
%\neg ( \exists s: \textit{Node} \mid \textit{cs}_i \in \textit{val}_i[s] \wedge \textit{cs}_j \in \textit{val}_j[s] \wedge j \neq i )
%\]
%\vspace*{-0.1cm}
%\[
%\mathbf{not} \ ( \mathbf{some}\ s: \mathit{Node} \mid \mathit{enterCS}_i  \ \mathbf{in} \ val[s] \land \mathit{enterCS}_j \ \mathbf{in} \ val[s] \land j \neq i )
%\]
% Due to space restrictions, we do not include the complete specification in this paper, but it can be found in \cite{}.
%\begin{example}[Mutex]\label{ex:mutex-spec} Consider a system composed of two processes (it can easily be generalized to $n$ processes) both with non-critical, waiting and critical sections. The global property  is mutual exclusion: the two processes cannot be in their critical sections simultaneously. The \textsf{FORL} specification is shown in Fig.~\ref{fig:mutex-spec}.
%For specifying this system we consider no shared variables ($Sh =\emptyset$) and one lock ($\mathcal{L}=\{\ell\}$). Furthermore, for each process $P_i$ (where $i \in \{0,1\}$)
% we consider a theory $\langle \tau_i, \Gamma_i \rangle$, defined as follows: $\tau_i$ contains unary relations $init, ncs_i, try_i, {av_\ell}_i, {own_\ell}_i$, and binary relations $enterTry_i, enterCS_i, enterNCS_i, getLock_i$. $\Gamma_i$ consists of the following axioms:
%\begin{figure*}[ht!]
%$
%\begin{array}{l}
% \text{init}, \text{ncs}_i, \text{try}_i, \text{av}_\ell, \text{own}_\ell:\text{Node}  \\
% \text{enterTry}_i, \text{enterCS}_i, \text{enterNCS}_i, \text{getLock}_i:\text{Node}\rightarrow \text{Node}  \\
% \all \ s:\text{Node} \mid (s \ \inc \ \text{init}_i) \ \imp \ (s  \ \inc \ ncs_i) \ \conj \ \nega \ s \ \inc \ \text{try}_i \ \conj \ s \ \inc \ {\text{av}_\ell}_i \\
% \bm{\wedge} \wedge
%\end{array}
%$
%\end{figure*}
% 
% 
%\begin{itemize}
%	\item The initial condition for the processes: 
%	        \[
%		\forall s: init_i(s) \Rightarrow  ncs_i(s) \wedge \neg cs_i(s) \wedge \neg try_i(s) \wedge {av_\ell}_i(s),
%		\]
%	\item The axioms for the actions, in a pre/post condition style:
%		\[
%		\begin{array}{l}
%			\forall s:  \forall s': enterTry_i(s,s') \Rightarrow ((ncs_i(s) \vee (try_i(s) \wedge \neg {own_\ell}_i(s))) \Rightarrow try(s'))\\
%		 	\forall s:  \forall s': enterCS_i(s,s') \Rightarrow ((try_i(s) \wedge {own_\ell}_i(s)) \Rightarrow cs(s'))\\
%		 	\forall s:  \forall s': enterNCS_i(s,s') \Rightarrow (cs_i(s)  \Rightarrow (ncs_i(s') \wedge \neg {own_\ell}_i(s')))\\
%			\forall s:  \forall s': getLock_i(s,s') \Rightarrow (try(s) \wedge av_\ell(s)  \Rightarrow own_\ell(s))\\
%		\end{array}
%		\]
%	\item Assumptions about the occurrence of actions:
%		\[
%		\begin{array}{l}
%			\forall s : \neg cs_i(s) \Rightarrow (\neg \exists s': getNCS(s,s')) \\
%			\forall s : \neg try_i(s) \wedge (\neg ncs_i(s) \vee {own_\ell}_i(s)) \Rightarrow (\neg \exists s' : getTry(s')) \\
%			\forall s : \neg try_i(s) \vee \neg {own_\ell}_i(s) \Rightarrow (\neg \exists s' : getCS(s,s'))  \\
%			\forall s : \neg try_i(s) \vee \neg {av_\ell}_i \Rightarrow (\neg \exists s' : getLock(s,s'))
%		\end{array}
%		\]
%	\item Frame axioms for the actions:
%		\[
%		\begin{array}{l}
%			\forall s : \forall s' : getNCS_i(s,s') \Rightarrow ({av_\ell}_i(s) \equiv {av_\ell}_i(s'))\\
%			\forall s : \forall s' : getTRY_i(s,s') \Rightarrow ({own_\ell}_i(s) \equiv {own_\ell}_i(s')) \wedge ({av_\ell}_i(s) \equiv {av_\ell}_i(s')) \\
%			\forall s : \forall s' : getCS_i(s,s') \Rightarrow ({own_\ell}_i(s) \equiv {own_\ell}_i(s')) \wedge ({av_\ell}_i(s) \equiv {av_\ell}_i(s')) \\
%			\forall s : \forall s' : getLock_i(s,s') \Rightarrow ({cs}_i(s) \equiv {cs}_i(s')) \wedge ({ncs}_i(s) \equiv {ncs}_i(s')) \\
%			\forall s : \forall s' : {ch_\ell}_i(s,s') \Rightarrow ({cs}_i(s) \equiv {cs}_i(s')) \wedge ({ncs}_i(s) \equiv {ncs}_i(s')) \wedge (try_i(s) \equiv try_i(s'))
%		\end{array}
%		\]
%%	 all s:ProcessMeta.nodes | 
%%					((not Prop_TRYING[ProcessMeta, s]) or (not Own_S[ProcessMeta, s])) implies (no ProcessMeta.ACTgetCS[s])
%%	
%%-- pre -> execution is possible
%%    all s:ProcessMeta.nodes | 
%%				(Prop_TRYING[ProcessMeta, s] and Own_S[ProcessMeta, s])   implies (some ProcessMeta.ACTgetCS[s])
%	\item $ncs, cs, try$ are mutually disjoint: 
%	\[
%		 \forall s : \neg( cs(s) \wedge ncs(s)) \wedge \neg( try(s) \wedge ncs(s)) \wedge \neg( try(s) \wedge cs(s))
%	\] 	
%	\item Axioms \textbf{P1}-\textbf{P8}
%\end{itemize}
%Finally, the global property is $\A \G (\neg cs_0 \wedge \neg cs_1)$. 
%It is worth noting that many of these axioms (e.g., the frame axioms) can be automatically generated. We have implemented a  prototype tool that given pre and post conditions for actions, plus temporal formulas for local and global properties, it automatically generates the corresponding specification in $\textsf{FO(TC)}$.% this facilitates the task of producing this kind of specifications.
\end{example}	
%
%Given $AP=\{p_0,\dots,p_n\}$ and $Act=\{a_0,\dots,a_m\}$, and a transition system $M=\langle S, R^M_{a_0},\dots,R^M_{a_m}, v^M \rangle$, we define the theory $Th(M) = \langle \tau_M, \Gamma_M \rangle$ in the following way: 
% $\tau_M = \langle R^2_{a_0}, \dots, R^2_{a_m}, s_0, \dots, s_k, p^1_0, \dots, p^1_n \rangle$ and:
%\[
%\begin{array} {l l} \Gamma^M = & \{\bigwedge_{i \neq j} s_i \neq s_j, \forall s : \bigvee_{0\leq i \leq k}s=s_i \} \\
%										       &\cup \bigcup_{0\leq i \leq m} \{ R_{a_i}(s_k,s_{k'}) \mid R_{a_i}^M(s_k,s_{k'})\} \\
%										        & \cup \bigcup_{0\leq i \leq m} \{ \neg R_{a_i}(s_k,s_{k'}) \mid \neg R_{a_i}^M(s_k,s_{k'}) \}\\
%										        &\cup \bigcup_{0\leq i \leq n} \{ p_i(s_j) \mid p_i \in v^M(s_j) \}
%										        \cup \bigcup_{0\leq i \leq n} \{ \neg p_i(s_j) \mid p_i \notin v^M(s_j) \}
%\end{array}
%\]
%\end{definition}

%It is worth noting that, given any TS structure $M=\langle S, Act, \rightarrow, s, AP, L \rangle$, we can define a theory $Th(TS)$ that characterizes $TS$ up to isomorphism. The definition of $Th(TS)$ is straightforward: constants are used to explicitly represent the states of the structure, and  axioms are added to  describe the relation. Formally:
%\begin{definition}\label{def:th(M)} Given a transition system $TS=\langle S, Act, \rightarrow, s, AP, L \rangle$ with  
%$Act = \{a_0,\dots,a_m\}$ and $AP=\{p_0,\dots,p_t\}$, we define the  theory $Th(TS) = \langle \tau_{TS}, \Gamma_{TS} \rangle$,  where $\tau_{TS}$ is the vocabulary corresponding to $TS$ (see Section~\ref{sec:background})
% and:
% \[\displaystyle
%\begin{array} {l l} \Gamma^{TS} = & \{\bigwedge_{i \neq j} s_i \neq s_j, \forall s : \bigvee_{0\leq i \leq k}s=s_i \} 
%										       \cup \bigcup_{0\leq i \leq m} \{ R_{a_i}(s_k,s_{k'}) \mid s_k \xrightarrow{a_i},s_{k'} \} \\
%										        & \cup \bigcup_{0\leq i \leq m} \{ \neg R_{a_i}(s_k,s_{k'}) \mid \neg (s_k \xrightarrow{a_i} s_{k'}) \}\\
%										        &\cup \bigcup_{0\leq i \leq n} \{ p_i(s_j) \mid p_i \in L(s_j) \}
%										        \cup \bigcup_{0\leq i \leq n} \{ \neg p_i(s_j) \mid p_i \notin L(s_j) \}
%\end{array}
%\]

%
%The following theorem is straightforward:
%\begin{theorem} Let  $M=\langle S, R^M_{a_0},\dots,R^M_{a_n}, v^M \rangle$ be a Kripke structure. For any relational structure $\mathcal{A}$ it holds that:
% $
% \mathcal{A} \vDash Th(M) \text{ iff } \mathcal{A}|_{\tau_M} \cong \mathcal{A}_M,
% $
% where $\mathcal{A}|_{\tau_M}$ is the reduct of $\mathcal{A}$ to vocabulary $\tau_M$ \cite{ChangKeisler90}.
%\end{theorem}
% SAY SOMETHING ABOUT ISOMORPHISM??
%\subsubsection{Refinements}
%	Given a theory $\langle \tau, \Gamma \rangle$ specifying a process and an instance of it $\mathcal{A} \vDash \Gamma$, we will be interested in those transition structures that refine $TS_{\mathcal{A}}$, obtained by disabling some local non-determinism present in $TS_{\mathcal{A}}$. The collection of these refinements can be characterized by means of a \textsf{FO(TC)} theory, as follows: 	
%\begin{definition} Given a transition system $TS=\langle S, Act, \rightarrow, s, AP, L \rangle$ where $S =\{s,s_0,\dots, s_n\}$, 
%$Act = \{a_0,\dots,a_m\}$ and $AP=\{p_0,\dots,p_t\}$, and a set $Loc \subseteq Act$, we define $Ref_{Loc}(TS) = \langle \tau_{TS}, \Gamma_{Ref(TS)} \rangle$ where $\tau_{TS}$ is
%defined as in Section \ref{sec:background}, and:
%\[
%\begin{array} {l l} \Gamma^{TS} = & \{\bigwedge_{i \neq j} s_i \neq s_j, \forall s : \bigvee_{0\leq i \leq k}s=s_i \} 	
%											\cup \bigcup_{0\leq i \leq n} \{ p_i(s_j) \mid p_i \in L(s_j) \} \\
%											& \cup \bigcup_{0\leq i \leq m} \{ R_{a_i}(s_k,s_{k'}) \mid a_i \notin Loc \wedge s_k \xrightarrow{a_i},s_{k'} \} \\						
%										        & \cup \bigcup_{0\leq i \leq m} \{ \neg R_{a_i}(s_k,s_{k'}) \mid \neg (s_k \xrightarrow{a_i} s_{k'}) \}
%										        	   \cup \bigcup_{0\leq i \leq t} \{ \neg p_i(s_j) \mid p_i \notin L(s_j) \}
%\end{array}
%\]
%\end{definition}
%	Note that in this definition the local actions are those belonging to the set $Loc$. Roughly speaking, the axioms of this theory state that: (i) all the states of $TS$ must be present, (i) no new transitions must be added, (iii) environmental transitions must be preserved.
%	One interesting property of $Ref_{Loc}(TS)$ is that it preserves \textsf{CTL} universal properties.
%\begin{theorem}\label{theorem:preservation} Given a transition structure $TS$, and  a $\textsf{ACTL}$ formula $\phi$ over the vocabulary of $TC$, we have that:
%$TS \vDash \phi$ implies  $Ref(TS) \vDash \phi^*$.
%\begin{proof} First, note that for any $\mathcal{A} \vDash Ref(TS)$, we have a simulation relation $R$ such that $TS_\mathcal{A} \; R \; TS$, since
%simulation relations preserve \textsf{ACTL} properties, and the translation from transition structures to first-order structures is an isomorphism, the result follows.
%\end{proof}
%\end{theorem}
%Summing up, specifications are theory presentations, and transition structures/programs and their refinements can be captured as models of these theories.
	



\section{Implementation}

We implement the proposed On-device Sora on iPhone 15 Pro~\cite{apple2023}, leveraging its GPU of 2.15 TFLOPS and 3.3 GB of available memory, with the two methods proposed in Sec. \ref{sec:ours1} and \ref{sec:ours2}. In addition, to execute large video generative models (\ie, T5 \cite{raffel2020exploring} and STDiT \cite{opensora}) with the limited device memory, we devise and implement Concurrent Inference with Dynamic Loading (CI-DL), which partitions the models into smaller blocks that can be loaded into the memory and executed concurrently. The details of CI-DL is described in \Cref{sec:ours3}. The model components—T5~\cite{raffel2020exploring}, STDiT~\cite{opensora}, and VAE~\cite{doersch2016tutorial}—in PyTorch~\cite{paszke2019pytorch} are converted to MLPackage, an Apple’s CoreML framework~\cite{sahin2021introduction} for machine learning apps. Since current version of CoreML \cite{apple2023} lacks support for certain diffusion-related operations in text-to-video generation, we develop custom solutions like xFormer \cite{xFormers2022} and cache-based acceleration. We implement denoising scheduling, sampling pipeline, and tensor-to-video conversion in Swift~\cite{swift} using Apple-provided libraries. To optimize models, T5~\cite{raffel2020exploring}, the largest in video generation, is quantized to int8, while others models  (STDiT~\cite{opensora} and VAE~\cite{doersch2016tutorial}) run in float32; we found that they are challenging to quantize due to sensitivity and performance degradation.%Our future implementations of On-device Sora will explore additional optimization to further enhance model efficiency.

%\jjm{In addition to the two key challenges mentioned in \Cref{sec:challenges}, there is one more additional challenge, which is a high memory requirement.}
%To execute large video generative models (i.e., T5 \cite{raffel2020exploring} and STDiT \cite{opensora}) with the limited device memory, we propose Concurrent Inference with Dynamic Loading, which partitions the models into smaller blocks that can be loaded into the memory and executed concurrently. By parallelizing model execution and block loading, it effectively accelerates iterative model inference, e.g., multiple denoising steps. Also, it improves memory utilization while minimizing the block loading overhead by retaining specific model blocks in memory dynamically based on the available runtime memory.
%\jjm{For a more detailed explanation of CI-DL, please refer to \Cref{sec:ours3}.}
% 20 20 20 -> 16 16 16
% 10 10 10 -> 27 * 7 189 210 ->189 
\section{Evaluation}
% In light of experiments of CacheBlend (\S\ref{eval:1}) and EPIC (\S\ref{eval:2}), we design our experiments (\S\ref{eval:3}).

% \noindent\textbf{LLM Dataset.} 2WikiMQA, MuSiQue, HotpotQA, SAMSum, MultiNews.

% \noindent\textbf{LLM Baselines.} Full KV recompute, Prefix caching, Full KV reuse, CacheBlend, EPIC.
% \subsection{CacheBlend evaluation}\label{eval:1}
% \begin{itemize}
%     \item TTFT-Score Comparison.
%     \item RPS-TTFT Comparison.
%     \item Sensitivity Analysis. (1) chunk number; (2) chunk length; (3) batch size; (4) recompute ratio; (5) storage device (CPU RAM / slower Disk).
% \end{itemize}
% \subsection{EPIC evaluation}\label{eval:2}
% \begin{itemize}
%     \item TTFT-Score Comparison.
%     \item (CCR+RPS)-TTFT/Throughput Comparison.
%     \item Context length-TTFT Comparison.
%     \item Semantic-based / fixed-token-based splitting.
% \end{itemize}
% \subsection{\sys}\label{eval:3}\
% \noindent\textbf{VLM Model.} InternVL 2.5-8B \cite{chen2024internvl}, Qwen2-VL-7B \cite{wang2024qwen2vl}, LLaVA-1.6-vicuna-7B, LLaVA-1.6-Mistral-7B \cite{liu2024llavanext}.

% \noindent\textbf{VLM Dataset.} SparklesDialogueCC, SparklesDialogueVG \cite{huang2024sparkles}, MMDU \cite{liu2024mmdu}.

% \noindent\textbf{VLM Baselines.} CacheBlend, Prefix caching, Full KV reuse, \sys.

% \begin{itemize}
%     \item TTFT-Score Comparison.
%     \item RPS-TTFT/Throughput Comparison.
%     \item Sensitivity Analysis: Image number.
%     \item Why does CacheBlend fail to work when serving MLLM?
% \end{itemize}

In this section, we evaluate \sys~in terms of response time and generation quality. We also investigate whether \sys~is applicable when the number of images is large.
\subsection{Experimental settings}
We select two prevalent MLLMs in the experiments: LLaVA-1.6-vicuna-7B and LLaVA-1.6-mistral-7B \cite{liu2024llavanext}. All experiments are run on a server with 1 NVIDIA H800-80 GB GPU, 20-core Intel(R) Xeon(R) Platinum CPUs, and 100GB DRAM.

Two datasets are used in our evaluation. (1) \textbf{MMDU} \cite{liu2024mmdu}: This dataset aims to evaluate MLLMs' abilities in multi-turn and multi-image conversations. Each conversation stitches together multiple images and sentence-level text (e.g., ``IMAGE\#1, IMAGE\#2. Can you describe these images as detailed as possible?"). (2) \textbf{SparklesEval} \cite{huang2024sparkles}: This is also a dataset for assessing MLLMs' conversational competence across multiple images and conversation turns. Unlike MMDU, SparklesEval integrates multiple images at word level (e.g., ``Can you link the celebration occurring in IMAGE\#1 and the dirt bike race in IMAGE\#2 ?"). As shown in the examples, the prompts of two datasets are open questions. Previous works adopt GPT score to evaluate the quality of MLLMs' responses to the open questions \cite{liu2024mmdu, huang2024sparkles}. GPT score is a GPT-assisted evaluation that uses a powerful judge model (e.g., GPT-4o, Qwen, etc.) to assess the answers. We also employ this metric and their evaluation prompt, as listed in Appendix~\ref{prompt}.
% (3) \textbf{V*Bench} \cite{wu2024v}:  A dataset specifically designed to evaluate
% MLLMs in their ability to process high-resolution images and focus on visual details. Each sample contains a high-resolution image, a question, and four options.
% We select 100 samples from each of the above datasets for testing, each including 1 to 5 images.

% We use the following metrics to measure the performance of algorithms. (1) Time-To-First-Token (TTFT) refers to the time it takes for LLMs, to generate and return the first token after receiving an request. This metric is designed to measure the time spent in the prefill stage, which can be optimized by addressing the PIC problem. (2) GPT score \cite{liu2024mmdu, huang2024sparkles} is a GPT-assited evaluation  that uses a judge model (e.g., GPT-4o, Qwen, etc.) to assess the quality of model-generated responses. We employ this metric to assess the quality of MLLMs' responses to the open questions in MMDU and SparklesEval. We apply the evaluation prompts in MMDU \cite{liu2024mmdu} to guide the judge model for scoring in the range of 10. 

% (3) F1 score is a metric used to evaluate the similarity between MLLMs’ output and the groundtruth answer. We employ this metric to assess the accuracy of the MLLMs' answers to the multiple-choice questions in V*Bench.

We compare \sys-$k$ with three existing CC algorithms: prefix caching, full reuse, and CacheBlend \cite{yao2024cacheblend}. CacheBlend is also a position-independent algorithm designed for RAG system. It recomputes $r$\% of total tokens with largest KV deviation, so we denote it as CacheBlend-$r$. The primary focus of CacheBlend is the KV deviation, while the \sys's selection process involves the identification of tokens that exhibit both high attention scores and significant KV deviation. We implement the four CC algorithms based on vLLM 0.6.4 \cite{kwon2023efficient}.

% (1) Prefix Caching: This algorithm merely stores and reuses the KV chche of the prefix. And the KV cache of non-prefix tokens needs to be computed during prefill. (2) Full Reuse: This algorithm reduces TTFT by fully reusing the entire KV cache regardless of the position of multimodal data. (3) CacheBlend \cite{yao2024cacheblend}: This is a state-of-the-art partial reuse algorithm that achieves a trade-off between TTFT and generation quality by dynamically selecting partial tokens to recompute.
% Additionally, we evaluate various variants of CacheBlend, denoted as CacheBlend-r, where $r$ represents the ratio of tokens recomputed. Similarly, we test different variants of InfoBlend, denoted as InfoBlend-k, where $k$ indicates the number of tokens recomputed at each chunk boundary.

\subsection{Effectiveness of \sys}
Based on vLLM offline inference, we compare the performance of all algorithms. Specifically, we process all requests sequentially and evaluate their generation quality and processing time for prefill. The workflow initiates with the precomputation of the relevant KV cache for images. Subsequently, we send the user's query along with the cache\_ids of the images to the serving system. Prefix caching will process the query with the KV cache of system prompt only. \sys~concatenates the dummy cache and stored cache, and computes the first output token using selective attention mechanism in single step. Full reuse and CacheBlend first compute the KV cache of text, and then produce the first output token with the concatenated KV cache. We record the processing time of the algorithms and finally score for each response.
\begin{figure}[t]
    \centering
    \includegraphics[width=\columnwidth]{figs/legend_result.pdf}
    % \vskip -0.2in
    \includegraphics[width=\columnwidth]{figs/results.pdf}
    \caption{Comparison of TTFT ($\downarrow$ Better) and Score ($\uparrow$ Better) using different models on different datasets. }
    \label{fig:ttft-score}
    % \vskip -0.2in
\end{figure}

\figurename~\ref{fig:ttft-score} presents the experimental results of all algorithms across different models and datasets. The results indicate that \sys~consistently outperforms CacheBlend in terms of both TTFT and score across various configurations. \sys-32 reduces TTFT by up to 54.1\% while maintaining a loss of score within 13.6\% compared to prefix caching. Additionally, it is clear that \sys~exhibits a slight decrease in TTFT compared to full reuse, since \sys~is a single-step process. Overall, compared to other algorithms, \sys~achieves the best trade-off between TTFT and score.

\subsection{Sensitivity analysis}
In order to achieve a more profound comprehension of \sys, a subsequent analysis is necessary to ascertain how the number of images impacts overall performance. We divide the dataset of MMDU into 10 groups in terms of the number of images. We evaluate the TTFT and score of \sys~and baselines on each group. The average value of results are shown in \figurename~\ref{fig:10}. The TTFT of \sys~is consistently shorter than that of prefix caching. When the number of images is 10, \sys~achieves 54.7\% reduction in TTFT. Furthermore, the performance of \sys~remains unaffected by the number of images, exhibiting negligible or no accuracy degradation.
\begin{figure}[t]
    \centering
    \includegraphics[width=0.9\columnwidth]{figs/legend_image_num.pdf}
    \vskip -0.2in
    \subfloat[]{
        \includegraphics[width=0.42\columnwidth]{figs/TTFT_all.pdf}
        \label{fig:10a}
    }
    \subfloat[]{
        \includegraphics[width=0.4\columnwidth]{figs/Score_all.pdf}
        \label{fig:10b}
    }
    \caption{The performance of \sys~as the number of images increases. For clarity, we only present the results of \sys-32. Other variants of \sys~show similar patterns.}
    \label{fig:10}
    % \vskip -0.2in
\end{figure}

% \subsection{Latency and throughput performance of InfoBlend}
% To assess Infoblend's latency and throughput performance, we leverage VLLM's OpenAI-compatible API server to simulate real-world user request patterns. We first select $n$ samples from MMDU and pre-generate KV caches for their contexts. Subsequently, we simulate user request behavior by repeatedly sending the user queries along with the cache\_ids of these 
% $n$ samples at a specified request rate over a period of time. by varying the request rate, We measure the latency and throughput across different experimental conditions.

% In Figure, we present a comparison of latency and throughput between InfoBlend and CacheBlend at varying request rates.  Compared to CacheBlend, InfoBlend achieves up to 80\% reduction in TTFT and 2-3 $\times$ improvement in throughput. This gap increases as the request rate rises.

\section{Related Work}

\subsection{View-Dependent Control}
View-dependent representations have a long history in computer graphics.
In his pioneering work, Rademacher proposed interpolating between \textit{key viewpoints} and associated \textit{key deformations} to manipulate 3D models~\cite{rademacher1999view}.
Other researchers have extended the idea to create 3D animation systems~\cite{10.1111:j.1467-8659.2004.00772.x}, streamline the modeling process~\cite{DBLP:journals/corr/abs-2103-15472}, and integrate physical simulation\cite{koyama2013view}.
Of particular note, Rivers et al.~\cite{rivers25Dcartoonmodels} introduced \textit{2.5D Cartoon Models}, a combination of planar meshes transformed, based upon view angle, so as to appears three dimensional.
Our work draws upon these works but is, to our knowledge, the first work to attempt to use view-dependent techniques to retarget 3D motion onto 2D characters.   

\subsection{Animation from 2D Images}

% output is still 2D
Many researchers have proposed different methods for creating animations from 2D images. Hornung et al.~\cite{Hornung2007anim2Dpicmotion} presented a method to deform a character from a photograph given user-provided joint annotations.
\textit{Toonsynth}~\cite{Dvoroznak18-SIG} and \textit{Neural Puppet}~\cite{poursaeed2020neural} both present methods to create new images of hand-drawn characters from examples.
% output is 3D model
Other researchers have proposed methods of obtaining 3D geometry from 2D sketches~\cite{igarashi2006teddy, Dvoroznak20-SA} and images~\cite{ArtiSketch,weng2019photo}.
% focus on sketches specifically
A number of works have specifically focused on childlike drawings.
Lingens et al.~\cite{lingens2020towards} proposed an evolutionary algorithm for animating children's drawings. 
\textit{MagicToon}~\cite{feng2017magictoon} creates a 3D model from childlike drawings for AR applications.
Similar to our work, Smith et al.~\cite{SmithHodgins} proposed a method for animating childlike drawings using 3D skeletal motion. 
However, the resulting animations are only suitable for use in 2D applications and the type of motions it supports are limited.

Unlike these previous works, our solution can be used in 3D contexts but does not create a 3D model. We instead relying upon a view-dependent formulation of the animated character.

Software development is increasingly conceived as a collaboration activity between developers and AIs. Indeed, IDEs already implement features to enable interactive development, with AI suggesting implementations that are reused by developers.

Although multiple studies show this interaction can be successful, there is still limited understanding of how the models must be configured and used in the context of code generation tasks. This study addresses this gap, systematically investigating the impact of several key parameters, including the repeated submission of a prompt to accommodate for the non-deterministic nature of the models.

Our study reveals several key findings about the usage of ChatGPT. In particular, we discovered how creativity, although up to a limited extent, is useful to increase the range of methods whose code can be generated correctly. A major role is played by parameter top-p, which is commonly underrated, and instead has a major impact on the correctness of the results, with lower values producing better results. Finally, prompts should be submitted multiple times, with $5$ repetitions combined with a temperature of $1.2$ resulting in an effective configuration in our experiments.  

Future work concerns two main research directions. One is about replicating this experiment with other AI assistants, to validate our findings in multiple contexts. The second research direction concerns finding strategies to deal with the need to submit the same prompt multiple times to obtain a useful result, and thus developing approaches able to select or merge multiple responses automatically. 
% \cleardoublepage

\bibliographystyle{IEEEtranN}
% \bibliographystyle{ACM-Reference-Format}
\bibliography{reference}

\end{document}
\endinput
%%
%% End of file `sample-acmsmall-conf.tex'.
