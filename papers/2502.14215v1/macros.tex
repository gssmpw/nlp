%%% macros.tex
%%% Import packages, define utility commands and define macros
%%% Version 1.1.0

%%%----------
%%% Imports
%\IEEEoverridecommandlockouts
% The preceding line is only needed to identify funding in the first footnote. If that is unneeded,
%please comment it out.
% \usepackage[table]{xcolor}  %rowcolors and rowcolor; already in table?
\usepackage{tikz}          %for numbered cycles
\usepackage{amsmath}
% \usepackage[ruled]{algorithm2e}
\usepackage{graphicx}
\usepackage{booktabs}
\usepackage{latexsym}   % \Diamond
\usepackage{array}      % p{}
\usepackage{multirow}
%\usepackage{subfigure}
\usepackage{subcaption}
\usepackage{algorithm}
\usepackage[noend]{algpseudocode}
% \usepackage{algorithmicx}
%\usepackage[font={small,bf}]{caption}  % must before threeparttable
\usepackage{threeparttable}
\usepackage{paralist}   % compactitem
\usepackage{xspace}
\usepackage{color}
\usepackage{adjustbox}
\usepackage{balance}
\usepackage{wasysym}
\usepackage{rotating}   % turn
\usepackage{listings}
%\usepackage{flushend}  %TODO add in the future
\usepackage{tabu}
\usepackage{amssymb}
\usepackage{pifont}
\usepackage{framed}
\let\labelindent\relax
\usepackage[shortlabels]{enumitem}
\setenumerate[1]{itemsep=0pt,partopsep=0pt,parsep=\parskip,topsep=2pt}
\setitemize[1]{itemsep=0pt,partopsep=0pt,parsep=\parskip,topsep=2pt}
\setdescription{itemsep=0pt,partopsep=0pt,parsep=\parskip,topsep=2pt}
\usepackage{multirow}
%\usepackage[hyphens]{url}  % old ACM
\usepackage[hyphens]{url}
%\usepackage[pdfborder={0 0 0}, citecolor=blue, linkcolor=blue, urlcolor=black, colorlinks=true]{hyperref}
\usepackage{hyperref}
%\usepackage[draft]{hyperref}

%\usepackage{float}
% sout
\usepackage[normalem]{ulem}

\usepackage{cleveref}

\usepackage{multicol}
\usepackage{orcidlink}
%a fix for making tabu and threeparttable work together
%http://tex.stackexchange.com/a/56524
\usepackage{xpatch}
% \usepackage{minted}
% fix mint error when Upload paper to ArXiv 
% source: https://tex.stackexchange.com/questions/280590/work-around-for-minted-code-highlighting-in-arxiv
% Step 1: finalize the mint cached data, as a result the folder ``_minted-main'' will be created
% \usepackage[finalizecache,cachedir=.]{minted}
% Step 2: change it to frozencache, and then upload every thing to arXiv including ``_minted-main''
% \usepackage[frozencache,cachedir=.]{minted}

% \newsavebox{\mintedbox}

\usepackage{ragged2e}

% \usepackage[skip=2pt]{caption}
% \newcommand{\distance}{4pt}
% \setlength{\textfloatsep}{\distance}%set distance between figure/tables on the top/bottom with text
% \setlength{\floatsep}{\distance}%set distance between figures or tables
% \setlength{\intextsep}{\distance}%set distance between figures/tables in text with text
% \setlength{\dbltextfloatsep}{\distance} %distance between a figure/table spanning both columns and the text;
% \setlength{\dblfloatsep}{\distance} %distance between two figures/tables spanning both columns.

% checkmark
% \usepackage{bbding}

% citeauthors
\usepackage[numbers]{natbib}

% \usepackage{amssymb}
% \usepackage{cite}

%code block style
\newcommand{\red}[1]{\textcolor[rgb]{1.00,0.00,0.00}{#1}}
\newcommand{\blue}[1]{\textcolor[rgb]{0.00,0.00,1.00}{#1}}
\newcommand{\green}[1]{\textcolor[rgb]{0.00,0.60,0.00}{#1}}
\newcommand{\darkblue}[1]{\textcolor[rgb]{0.00,0.00,0.65}{#1}}
\newcommand{\darkred}[1]{\textcolor[RGB]{139,0,0}{#1}}
\newcommand{\lightred}[1]{\textcolor[RGB]{255,204,203}{#1}}
\newcommand{\myred}[1]{\textcolor[rgb]{1.00,0.00,0.00}{#1}}

%https://en.wikibooks.org/wiki/LaTeX/Special_Characters#Other_symbols
\newcommand{\cha}{\red{\ding{55}}\xspace}
\newcommand{\gou}{\green{\ding{52}}\xspace}
\newcommand{\ling}{\darkblue{\RIGHTcircle}\xspace}

\newcommand*\rot{\rotatebox{75}}
\newcommand*\numcircledtikz[1]{\tikz[baseline=(char.base)]{
            \node[shape=circle,draw,inner sep=0.4pt] (char) {#1};}}

%--http://www.iam.uni-bonn.de/~alt/latex/rgb.tex--
%for table highlight
\definecolor{wheat1}{rgb}{1.000000,0.905882,0.729412}
%for table header
\definecolor{LightGray}{rgb}{0.827451,0.827451,0.827451}

%http://tex.stackexchange.com/questions/94799/how-do-i-color-table-columns
\newcolumntype{a}{>{\columncolor{wheat1}}l}

\definecolor{mygreen}{rgb}{0,0.6,0}
\definecolor{mygray}{rgb}{0.5,0.5,0.5}
\definecolor{mymauve}{rgb}{0.58,0,0.82}
\definecolor{darkblue}{rgb}{0.0,0.0,0.6}
\definecolor{maroon}{RGB}{102, 0, 0}
\definecolor{Maroon}{cmyk}{0,0.87,0.68,0.32}
\definecolor{darkred}{RGB}{139, 0, 0}
\definecolor{forestgreen}{RGB}{34, 139, 34}

%https://goo.gl/GVcUco
\lstset{ %
  backgroundcolor=\color{white},   % choose the background color
  basicstyle=\footnotesize,        % size of fonts used for the code
  breaklines=true,                 % automatic line breaking only at whitespace
  captionpos=t,                    % sets the caption-position to bottom
  commentstyle=\color{mygreen},    % comment style
  escapeinside={\%*}{*)},          % if you want to add LaTeX within your code
  keywordstyle=\color{blue},       % keyword style
  stringstyle=\color{mymauve},     % string literal style
}

%http://tex.stackexchange.com/a/140242
%http://texblog.org/2012/08/29/changing-the-font-size-in-latex/
\lstdefinelanguage{XML}
{
  basicstyle=\ttfamily\small,   %\small is less than \footnote?
  morestring=[b]",
  moredelim=[s][\color{darkblue}]{<}{\ },
  moredelim=[s][\color{darkblue}]{</}{>},
  moredelim=[l][\color{darkblue}]{/>},
  moredelim=[l][\color{darkblue}]{>},
  morecomment=[s]{<?}{?>},
  morecomment=[s]{<!--}{-->},
  stringstyle=\color{darkred},
  identifierstyle=\color{mymauve}
}

%https://en.wikibooks.org/wiki/LaTeX/Source_Code_Listings#Settings
%https://en.wikibooks.org/wiki/LaTeX/Source_Code_Listings#Style_definition
\lstdefinestyle{customJava}{
  breaklines=true,
  keepspaces=true,
  frame=single,
  language=Java,
  showstringspaces=false,
  %moredelim=**[is][\color{orange}]{@}{@},
  basicstyle=\footnotesize\ttfamily,
  keywordstyle=\color{blue},
  otherkeywords={+, getIntent},
  numbers=left,
  numbersep=5pt,
  numberstyle=\scriptsize\color{black},
  rulecolor=\color{black},
  stepnumber=1,
  tabsize=1,
  commentstyle=\itshape\color{green!40!black},
  %identifierstyle=\color{blue},
  stringstyle=\color{orange},
  emph=[1]  %http://tex.stackexchange.com/a/148194
  {
        do,
        try,
        new,
        catch,
        while,
        SecProvider,
        SecReceiver,
        SecService,
        SecActivity,
        SecSink,
  },
  emphstyle=[1]{\color{darkred}},
  emph=[2]  %http://tex.stackexchange.com/a/148194
  {
        @Override,
  },
  emphstyle=[2]{\color{purple!40!black}},
  %belowcaptionskip=-2em, %http://tex.stackexchange.com/a/61559
  belowskip=-1em, %http://tex.stackexchange.com/a/50108
}


\newif\ifANNOYMIZE
\ANNOYMIZEtrue

\newif\ifACM
%\ACMtrue  %ACM
\ACMfalse %IEEE

\ifACM
%\usepackage{authblk}   %http://tex.stackexchange.com/questions/9594/adding-more-than-one-author-with-different-affiliation
\fi

\ifACM
\newcommand{\myfig}{Figure\xspace}
\else
\newcommand{\myfig}{Fig.\xspace}
\fi

\ifACM
\newcommand{\mysec}{\S}
\else
\newcommand{\mysec}{\S}
\fi

\renewcommand{\algorithmicrequire}{\textbf{Input:}}
\renewcommand{\algorithmicensure}{\textbf{Output:}}

\newcommand{\TODO}{\textbf{\textcolor[rgb]{1.00,0.00,0.00}{[TODO]}}\xspace}
\newcommand{\MARK}[1]{\textbf{\textcolor[rgb]{1.00,0.00,0.00}{{#1}}}\xspace}
\newcommand{\code}[1]{{\fontfamily{cmtt}\fontseries{m}\fontshape{n}\selectfont\small{#1}}}

% with comment
\newcommand{\yl}[1]{{\footnotesize{\textcolor{blue}{[YL: {#1}]}}}\xspace}
\newcommand{\yq}[1]{{\footnotesize{\textcolor{blue}{[YQ: {#1}]}}}\xspace}
\newcommand{\wei}[1]{{\footnotesize{\textcolor{blue}{[Wei: {#1}]}}}\xspace}
\newcommand{\rain}[1]{{\footnotesize{\textcolor{blue}{[rain: {#1}]}}}\xspace}
\newcommand{\ye}[1]{{\footnotesize{\textcolor{purple}{[Liu Ye: {#1}]}}}\xspace}
\newcommand{\ma}[1]{{\footnotesize{\textcolor{red}{[ma: {#1}]}}}\xspace}
\newcommand{\review}[1]{{\footnotesize{\textcolor{green}{[Review: {#1}]}}}\xspace}
\definecolor{cadmiumgreen}{rgb}{0.0, 0.42, 0.24}
\newcommand{\solvedreview}[1]{{\footnotesize{\textcolor{cadmiumgreen}{[\checkmark Review: {#1}]}}}\xspace}
\usepackage{tikz}
\usetikzlibrary{tikzmark}
% without comment
%\renewcommand{\yq}[1]{}
%renewcommand{\dao}[1]{}
%\renewcommand{\review}[1]{}
%\renewcommand{\solvedreview}[1]{}

\newcommand{\fixme}[1]{{\color{red}{#1}}}

\newcommand\y{$\checkmark$\xspace}
\newcommand\x{\textcolor[rgb]{1.00,0.00,0.00}{$\times$}\xspace}


\newcommand{\tool}{\textsc{PartitionGPT}\xspace} %\textsc{Pgpt}
\newsavebox{\bigimage} % for positioning subfigures
\newcommand{\secrete}{sensitive data\xspace}

\newcommand{\cmark}{\ding{51}}
\newcommand{\xmark}{\ding{55}}


\newcommand{\CNumber}{23\xspace}
\newcommand{\ContestNumber}{XX\xspace}

\usepackage{array}

\newcolumntype{L}[1]{>{\raggedright\let\newline\\\arraybackslash\hspace{0pt}}m{#1}}
\newcolumntype{C}[1]{>{\centering\let\newline\\\arraybackslash\hspace{0pt}}m{#1}}
\newcolumntype{R}[1]{>{\raggedleft\let\newline\\\arraybackslash\hspace{0pt}}m{#1}}
%\newcommand{\website}{\url{https://sites.google.com/view/propertygpt}\xspace}
\newcommand{\website}{\url{https://github.com/academic-starter/PartitionGPT}\xspace}

\newcommand{\etal}{\textit{et al.}\xspace}
\newcommand{\Certora}{Certora\xspace}
\newcommand{\prepost}{pre-/post-conditions\xspace}
\makeatletter
\chardef\TPT@@@asteriskcatcode=\catcode`*
\catcode`*=11
\xpatchcmd{\threeparttable}
  {\TPT@hookin{tabular}}
  {\TPT@hookin{tabular}\TPT@hookin{tabu}}
  {}{}
\catcode`*=\TPT@@@asteriskcatcode
\makeatother
\usepackage{tcolorbox}
\tcbuselibrary{breakable, skins}
\tcbset{%
  label begin/.style={label={#1}},%          just for symmetry
  label end/.style={after upper=\label{#1}}% new end label
}
\newtcolorbox[%
auto counter]{mybox}[2][]{%
  enhanced jigsaw,
  breakable,
  #1}
  \lstdefinelanguage{Solidity}{
    keywords={contract, function, external, internal, payable, emit, return, if, else},
    keywordstyle=\color{blue}\bfseries,
    ndkeywords={bool, uint64},
    ndkeywordstyle=\color{darkgray}\bfseries,
    identifierstyle=\color{black},
    sensitive=false,
    comment=[l]{//},
    morecomment=[s]{/*}{*/},
    commentstyle=\color{purple}\ttfamily,
    stringstyle=\color{red}\ttfamily,
    morestring=[b]',
    morestring=[b]"
  }
  
  % Define Style
  \lstset{
     language=Solidity,
     backgroundcolor=\color{white},
     extendedchars=true,
     basicstyle=\footnotesize\ttfamily,
     showstringspaces=false,
     showspaces=false,
     numbers=left,
     xleftmargin=12pt,
     numberstyle=\footnotesize,
     numbersep=7pt,
     tabsize=1,
     breaklines=true,
     showtabs=false,
     captionpos=b,
     escapechar=@
  }
  
% % Define styles for TikZ highlighting
% \tikzset{highlight/.style={rectangle, draw=red, fill=yellow!20, thick, inner sep=0pt}}
\definecolor{academicblue}{RGB}{33, 113, 181} % Deep Blue
\definecolor{academicred}{RGB}{213, 94, 0} % Bright Orange-Red
\definecolor{academicgreen}{RGB}{0, 128, 96} % Professional Green
% Define a custom style to highlight specific lines
% Custom highlighting style using TikZ overlays
% Define colors for highlighting
\newcommand{\Highlight}{\makebox[0pt][l]{\color{yellow!30}\rule[-2pt]{0.95\linewidth}{7.5pt}}}
% \newcommand{\HighTwolight}{\makebox[0pt][l]{\color{yellow!30}\rule[-10pt]{0.95\linewidth}{15pt}}}
% \newcommand{\highlight}[2]{%
%   \draw[yellow,line width=14pt,opacity=0.3]%
%     ([yshift=4pt]#1) -- ([yshift=4pt]#2);%
% }

\newtheorem{definition}{Definition}
