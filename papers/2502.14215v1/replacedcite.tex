\section{Related Work}
\label{sec:relate}
% Our work is generally related to smart contract security and secure program partitioning.
\subsection{Smart Contract Security}
The detection of vulnerabilities in smart contracts has been a central focus of blockchain security research. Symbolic execution tools like Oyente____, Manticore____, and Mythril____ pioneered the detection of critical vulnerabilities such as reentrancy, mishandled exceptions, and integer overflow/underflow. These tools systematically explore execution paths to identify potential exploits but are limited by path explosion and incomplete semantic coverage. Static analysis tools, such as Slither____ and SmartCheck____, leverage data-flow and control-flow analyses to identify a wide range of vulnerabilities, including bad coding practices and information-flow issues. Securify____ and Ethainter____ use rule-based pattern matching to detect vulnerabilities, while SmartScopy____ introduces summary-based symbolic evaluation for attack synthesis.
Dynamic analysis techniques, including fuzzing tools like ContractFuzzer____, sFuzz____, and Echidna____, analyze runtime behaviors by generating test inputs to uncover exploitable bugs. However, these tools often suffer from limited state coverage and rely heavily on predefined oracles. Formal verification tools, such as KEVM____ and Certora____, as well as semi-automated tools like Echidna____, require user-provided specifications, such as invariants or pre-/post-conditions, which can be challenging to define for complex contracts.

With the advent of decentralized finance (DeFi), the scope of vulnerabilities has expanded to include governance attacks, oracle price manipulation, and front-running____. Tools like DeFort____, DeFiRanger____ and DeFiTainter____ address DeFi-specific vulnerabilities using transaction analysis and inter-contract taint tracking. Despite these advancements, existing tools often rely on predefined security patterns, limiting their ability to capture sophisticated or emergent vulnerabilities.
Recently, LLM-based security analysis tools like GPTScan____ and PropertyGPT____ have been proposed to leverage the in-context learning of LLM in the field of static analysis and formal verification for smart contracts.
% \ye{Distinguish \tool from the existing approaches.}
\tool complements the existing security efforts through performing secure program partitioning to preserving confidentiality for smart contracts.

\subsection{Secure Program Partitioning}
% \ye{yuqing can help}
Secure program partitioning has been introduced since 2001 by~Zdancewic et al.____ for protecting confidential data during computation in distributed systems containing mutually untrusted host.
They developed a program splitter that accept Java program and its security annotations on data variables, and assign according statements to the given hosts.

The program partitioning works can be broadly categorized based on the granularity of code splitting.
Function-level program partitioning takes the entire functions as units for separation, and it has been extensively studied ____.
Brumley and Song____ accept a few annotations on variables and functions, and then partitions the input source into two programs: the monitor and the slave.
They apply inter-procedural static analysis, i.e., taint analysis and C-to-C translation.
The workflow of our approach is in general similar to them. But, \tool harnesses LLM's in-context learning for Solidity-to-Solidity translation, and we propose a dedicated equivalence checker to formally verify the correctness of the resulting program partitions. 
Subsequently, 
Liu et al.____ proposes a set of techniques for supporting general pointer in the automatic program partitioning by employing a parameter-tree approach for representing data of pointer types that exist in languages like C. 
To balance the security and performance and select appropriate partitions, Wu et al.____ define a quantitative measure of the security
and performance of privilege separation.
Particularly, they measure the security by the size of code executed in unprivileged process, where the smaller the privileged part is,
the more secure the program is.
Our work also follows the insight about the security measure during the partition ranking but \tool uses the edit distance compared to the original program code rather than runtime overhead to select the appropriate partition that could compete with the human-written.
There are also program partitioning techniques specialized for creating and deploying partitioning to TEE-based secure environments____.
For instance, Glamdring____ uses static program slicing and backward slicing to isolate security-related functions involving sensitive operations in C/C++ programs for execution within Intel SGX enclaves. 
% Many early tools focus on function-level partitioning, treating entire functions as units of separation. For example, Privtrans uses static analysis to partition C programs into privileged and non-privileged components that communicate via RPC ____. 
% Similarly, ProgramCutter employs dynamic dependency graphs to balance security and performance ____. These methods are relatively straightforward but struggle to handle sensitive operations intertwined within functions. This limitation can lead to over-partitioning, where entire functions are unnecessarily moved to sensitive partitions, or under-partitioning, where sensitive operations remain exposed. Other function-level partitioning frameworks include SeCage, which combines static and dynamic analysis to divide programs into secret compartments with hardware-based isolation ____, and GoTEE, which compiles Go functions into enclaves with a lightweight runtime and API shielding ____. 
% Additionally, TLR focuses on partitioning Android applications by leveraging ARM TrustZone to separate secure and normal operations ____. 
% Rubinov et al. also address Android program partitioning, generating communication logic for sensitive code executed in TrustZone environments ____. 
% Some tools extend function-level partitioning to optimize security and performance. For instance, SOAAP helps developers reason about the compartmentalization of applications by providing source-level annotations and a performance simulator ____. 
% However, these annotations can be heavyweight, requiring developers to specify the partition boundary and what global state can be accessed by each security domain. Similarly, Kenali applies privilege separation to OS kernel code by enforcing data-flow integrity, but it lacks general applicability to user-space applications ____. 
% In summary, these approaches demonstrate the utility of function-level partitioning in reducing the trusted computing base (TCB), but their coarse granularity often hinders their ability to effectively isolate sensitive logic.
Different from the abovementioned approach, \tool creates program partitions for Solidity smart contracts, and deploys partitions to a virtualized TEE-based secure environment tailored for blockchain context that could support different TEE devices. 

To achieve finer-grained security, tools like Civet____ extend partitioning to the statement level.
Civet____, designed for Java, combines dynamic taint tracking and type-checking to partition Java applications while securing sensitive operations inside enclaves. Additionally, Program-mandering (PM)____ introduces a quantitative approach by modeling privilege separation as an integer programming problem. 
PM optimizes partitioning boundaries to balance security and performance trade-offs based on user-defined constraints. By quantifying information flow between sensitive and non-sensitive domains, PM helps developers refine partitions iteratively, but it still relies on significant user input, such as defining budgets and goals.
Other statement-level partitioning tools include Jif/split____, which leverages security annotations to enforce information flow policies and partitions Java programs into trusted and untrusted components. 
Similarly, Swift____ partitions web applications to ensure that security-critical data remains on the trusted server while client-side operations handle non-sensitive data. DataShield____ takes a similar approach by separating sensitive and non-sensitive data in memory and enforcing logical separation, though it does not physically split code into separate domains.
While these tools offer finer-grained partitioning and improved optimization, they often require significant developer input, such as specifying trust relationships, or iteratively refining partitions.
In contrast, \tool leverage LLM's in-context capability to automate the generation and improvement of program partitions, and the resulting program partitions are verified using a dedicated prover.

% \wei{Do we need a Threat to Validity Section?}