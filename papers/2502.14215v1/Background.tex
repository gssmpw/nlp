\section{Background and Motivation}
\label{sec:background}
\subsection{Blockchain and Smart Contracts, and Their Privacy Issues}
Blockchain technology was first introduced in Bitcoin~\cite{nakamoto2008bitcoin} and has emerged as a transformative innovation, enabling decentralized systems that eliminate the need for intermediaries in transactions and applications. At its core, blockchain provides a distributed, immutable ledger that ensures transparency and security. Smart contracts, programmable scripts executed on the blockchain, have further expanded its utility by automating processes such as financial transactions, supply chain management, and governance. These contracts are deterministic and operate transparently, allowing all participants in the network to verify their behavior. This has been pivotal in the success of decentralized finance (DeFi) and other blockchain-based applications, where trust is derived from the open and verifiable nature of smart contracts.

However, privacy is one of the major concerns for blockchains as most of these systems store and log everything viewable to the public~\cite{almashaqbeh2022sok}.
Sensitive information, such as user account details, transaction data, and contract-specific logic, is often exposed on-chain, creating risks of data leakage. For instance, adversaries can analyze public transaction histories and state variables of smart contracts to infer private information in order to exploit vulnerabilities. 
In DeFi space, attackers have leveraged publicly accessible contract states to orchestrate complex exploits, such as oracle price manipulations and front-running attacks. The lack of mechanisms to distinguish and safeguard sensitive data from non-sensitive data exacerbates these risks, making smart contracts an attractive target for malicious actors.

While efforts to mitigate data leakage risks exist, they are often inadequate. Techniques like homomorphic encryption~\cite{solomon2023smartfhe}, zero knowledge proofs~\cite{kosba2016hawk}, and multiparty computation~\cite{ren2022cloak} can obscure sensitive data, but these approaches may conflict with the principles of decentralization or introduce inefficiencies. This highlights the urgent need for innovative solutions that preserve the decentralized and transparent nature of blockchain systems while providing robust privacy protections. 
Strategies such as privilege separation could address these challenges, enabling smart contracts to securely handle sensitive information. 
% Ensuring data privacy is crucial not only for safeguarding user assets and sensitive operations but also for enhancing trust in blockchain systems, fostering broader adoption across industries.
% \subsection{Program Partitioning}

% \subsection{LLM for Code Refactoring}
% Large Language Models (LLMs) have recently shown remarkable potential in code refactoring by leveraging their ability to understand and generate natural language and programming code. 
% These models, trained on extensive datasets of code and text,  can comprehend program structures, semantics, and intention of a given codebase to perform code transformations that enhance readability, maintainability, and performance. Unlike traditional refactoring tools that rely on predefined patterns or static analysis, with an impressive capability of in-context learning, LLMs can adapt to diverse coding styles and complex contexts, offering more nuanced and context-aware suggestions~\cite{shirafuji2023refactoring}. 
% The use of LLMs for code refactoring is transforming software development practices by automating tedious tasks, reducing human error, and enabling developers to focus on higher-level design and problem-solving. 
% However, challenges remain in ensuring correctness, and aligning with specific tasks, underscoring the need for further research in this domain.