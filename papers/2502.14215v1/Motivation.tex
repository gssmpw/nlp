
% \ye{List a table to illustrate what attacks can be mitigated by confidential smart contracts}

% \ye{Introduce a real-world price manipulation attack incident as an example to highlight the importance of hiding sensitive information using confidential smart contracts.}

% \ye{What's the research challenge towards confidential smart contracts?}
% \begin{itemize}
%     \item \textbf{Challenge-I}: Identifying sensitive information is generally subjective. How to resolve it.
%     \item \textbf{Challenge-II}: Distributed execution environment makes it non-trivial for partitioning smart contracts for the use of confidential execution. 
%     Specifically, two new contracts, obtained from the partition of an original contract, have to run on two different blockchains -- one for privacy-free purpose and the other for confidential execution. 
%     The state consistency with the original contract has to be enforced and its correctness must be guaranteed across the contracts.
%     However, traditional application partitioning for confidential execution environment, e.g., Intel SGX, does not apply in smart contracts.
%     First, traditional application partitioning assumes devices capable of confidential execution environment, but this usually does not hold in nodes of popular blockchain networks, e.g., Ethereum, BSC, etc.
    
% \end{itemize}

\subsection{A Motivating Example}
\label{sec:motivation}
% \ye{for juantao to update}
% \paragraph{Manipulation Attack}
Jimbo suffered from a manipulation attack on May 29, 2023, where the attacker gained a profit of 8 million dollars by devising a highly-complicated transaction sequences with crafted inputs~\cite{Jimbo}.
Jimbo is a Self-Market Making Liquidity Bin Tokens (SMMLBTs)~\cite{SMMLBTs}, where ``bin'' represents a range of prices, with positions further to the right denoting higher prices.  In the Jimbo protocol, rebalancing of asset in the liquidity pool is defined by the different states of the bin including \textit{active bin}, \textit{floor bin} and \textit{trigger bin}. 
Specifically,
\textit{floor bin} denotes the minimum price of Jimbo tokens; \textit{active bin} represents the price currently traded on; and \textit{trigger bin} refers to the price that triggers the rebalancing.
One key point is that when \textit{active bin} is above \textit{trigger bin}, users can call a \textit{shift()} function of Jimbo to increase \textit{floor bin}.

The success of the attack relies on the attacker obtaining exact value of the bins.
In this exploit, the attacker first initiated a flashloan of 10,000 Ether to add liquidity to the rightmost bin as shown in~\Cref{fig:jimbo:initial}, where the current the \textit{active bin} is 8,387,711 and \textit{trigger bin} is 8,387,715.
Next, the attacker bought Jimbo's token to make \textit{active bin} above \textit{trigger bin}, where active bin moved from 8,387,711 to 8,387,716 as depicted in~\Cref{fig:jimbo:manipulate}.
Subsequently, the attacker rebalanced the liquidity with \textit{shift()} that increased the value of \textit{floor bin}.
Hence, the attacker creates a huge arbitrage space, and more details above the following profit earning operations can refer to the security analysis report\footnote{https://x.com/yicunhui2/status/1663793958781353985}.
Consequently, the attacker sold Jimbo's token at a significantly high price indicated by the increasing \textit{floor bin}.

However, this delicate attack vector is non-trivial for the existing price manipulation detection tools~\cite{xie2024defort,kong2023defitainter, wu2023defiranger} to detect and the existing runtime verification techniques~\cite{rodler2018sereum,chen2024demystifying} to defend against. 
First, the manipulation detection tools heavily rely on the correct extraction of the token exchange rate. But, in this case, the exchange rate between Ether and Jimbo is defined over the complicated states of bin, which is challenging to automatically infer.
Second, the manipulation of bin's states is carefully-designed (\textit{active bin} is close to \textit{trigger bin}), making runtime verification tools hard to differentiate between normal transactions and abnormal attacks.
% \rain{Perhaps we could provide more information about how isolating the floor/active/trigger bins can mitigate this attack while not affecting the usage of other users.}

Running privileged code inside a secure environment makes it impossible for an attacker to read such values and therefore the attack will be unsuccessful.
To mitigate such manipulation attacks, 
in this work, 
we propose to conduct a fine-grained secure program partitioning for smart contracts to isolate the execution of the operations related to \secrete variables, e.g., \textit{floor/active/trigger bin} of Jimbo, within a secure environment, where sensitive information leakage problem could be largely mitigated.  
We highlight that migrating and running the entire piece of smart contract into the secure environment is not practical, as that will lead to prohibitively high runtime overhead.
%  which guarantees the minimum price and is set to the fifth bin to the left of active bin, and trigger bin refers to the price that triggers the rebalancing and is set to 5 bins right above to the active bin.
% To exploit, The attacker started by initiating a flashloan of 10,000 Ether from Aave and then add liquidity to the rightmost bin, which is 896 bins right above the current active bin and cannot be touched in normal cases. After buying specific amount of Jimbo, which pushed the active bin right above trigger bin, the attacker called \texttt{shift()} function to trigger the rebalancing, which made the floor bin to a significantly high position, and sold Jimbo at this manipulated high price. Sequentially, the attacker triggered rebalancing again to bring Jimbo back to its normal price. This process was repeated multiple times to enhance malicious gains.

\begin{figure}[t]
    \begin{subfigure}[b]{0.48\columnwidth} % Adjust width as needed
        \centering
        % \includegraphics[trim=0 4em 0 0,width=\textwidth]{figure/jimbo-attack-initial.pdf} % 
        \includegraphics[width=\textwidth]{figure/jimbo-attack-initial.pdf} % Replace with your image filename
        \caption{Initial liquidity setting.}
        \label{fig:jimbo:initial}
    \end{subfigure}
    \hfill
    % Second subfigure
    \begin{subfigure}[b]{0.48\columnwidth} % Adjust width as needed
        \centering
        \includegraphics[width=\textwidth]{figure/jimbo-attack-manipulate.pdf} % Replace with your image filename
        \caption{Manipulated liquidity setting.}
        \label{fig:jimbo:manipulate}
    \end{subfigure}
    % Main caption for the figure
    \caption{Illustration of Jimbo's key attack step.}
    \label{fig:mainfigure}
\end{figure}

% First, the attacker flashloaned 10000 ether and addLiquidity to the rightmost bin, which cannot be touched forever in normal cases Next, the attacker buy \$JIMBO to make the activeBin above triggerBin, and call `shift` to increase the floorBin
% (before swap, activeBin is 8387711, triggerBin is 8387715; attacker buy Jimbo to make activeBin = 8387716)  After that, it bounght all \$Jimbo in the 'normal' liquidity range(red regions in above protocol images, 51 bins in total), and now the activeBin is the rightmost bin, where the attacker add liquidity before Now by calling shift again(obviously activeBin > triggerBin), anchorBins and floorBin are built based on this really high price   Next step is depleting the anchorBins to achieve the floorBin. I don't know how hacker calculate the exact token needed for this swap (cannot use swapTokenForExactNATIVE since getSwapIn will return the wrong number) but you can check my PoC for a method easier to understand When achieving the floorBin, I can call `reset` to 'flatten' the liquidity above(mentioned in 4), and buy all available cheaper \$Jimbo Finally, calling `shift` to rise the price and sell all \$JIMBO to get profit.

% \paragraph{Limitation of Existing Tools}

% \paragraph{Our approach}