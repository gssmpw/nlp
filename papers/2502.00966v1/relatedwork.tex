\section{Related Work}
\subsection{Algorithmic \& Generative Percussion Music}

Percussion music originated from instinct, with early percussionists relying on innate musicality. Over time, techniques were codified and passed down over generations \cite{hartenberger2016cambridge, benzon1993stages}, with rhythmic repetition and rule-based phrasing forming a foundation for intricate patterns \cite{adler1999mathematics}. This enabled the creation of new, complex patterns derived from traditional phrases, much like certain procedural approaches to music composition \cite{langston1989six}, which range from medieval to contemporary works \cite{mcalpine1999making}. Additionally, contemporary percussion composers sometimes move away from strict rules, introducing elements of randomness to add further musical complexity \cite{tucker2017emergence,popoff2011indeterminate}.

Steve Reich is known for repetitive rhythms and phasing \cite{reich}, where phrases are played at different tempos to create desynchronized textures, as seen in  \textit{Drumming} and \textit{Clapping Music} \cite{clapping, hartenberger2016performance}. Similarly, Philip Glass uses overlapping rhythms to build complex layers from simple patterns, as in his \textit{String Quartet 6} \cite{isac2020repetitive, glass}. John Cage, in works like his \textit{Composed Improvisations}, pioneered rule-driven compositions \cite{feisst2009john} that use randomness, allowing performers to improvise within structured frameworks---such as having musicians roll dice to decide their next pattern \cite{cage}.

Our work takes inspiration from contemporary percussion compositions by incorporating phasing, rhythmic layering, and randomness, implemented by our robotic music system. These techniques, pioneered by composers like Reich, Glass, and Cage, are particularly well-suited for our approach emphasizing unique robot strengths, as robots can easily execute random or complex patterns requiring difficult coordination, like phasing and overlapping rhythms, which would require extensive training for human musicians to perform accurately. 

\highlight{\subsection{Robot Musicianship}}

As robots become \highlight{increasingly prevalent} \cite{cone2019robots,jimeno2019fewer}, their \highlight{applications} in creative fields \highlight{extend} beyond automation \cite{gemeinboeck2013creative,jeon2017robotic,thorn2020human,savery2021shimon}. \highlight{Musical robots have been developed for various instruments \cite{bretan2016survey, kapur2005history}, ranging from string \cite{kusuda2008toyota, jorda2002afasia}, to wind \cite{dannenberg2011mcblare,solis2006waseda}, to percussion \cite{hoffman2010shimon,weinberg2006robot}.} 

Recent \highlight{developments in} robotic music \highlight{tend to emphasize \textit{anthropomorphism}, with systems mimicking human appearance} or trained on human \highlight{movements} \cite{savery2021shimon,hoffman2010shimon,kapur2007integrating}. \highlight{Examples include the Waseda Flutist Robot \cite{solis2006waseda}, Toyota's violin-playing robot \cite{kusuda2008toyota}, and others \cite{uchiyama2023development,zhang2011musical, wu2010towards}. Robot systems also leverage physical presence and visual cues---important elements to live performance \cite{schutz2008seeing}---which digital musicians cannot \cite{weinberg2007robotic, pessanha2021virtual}.}

%, which use anthropomorphism as a vehicle toward more familiar and ``human'' music performances

\highlight{While robotic music systems are often viewed as novel instruments \cite{solis2011musical}, musicians emphasize that human control---either through programming or real-time interaction---is essential for classification as a new instrument \cite{weinberg2007robotic}. Systems like Shimon \cite{hoffman2010shimon, hoffman2011interactive} or GuitarBot, in performance with violinist Mari Kimura \cite{auslander2009lucille}, demonstrate real-time interaction through their ability to synchronize and adapt to human musicians.}

\highlight{Robots also possess capabilities that transcend human limitations, such as increased accuracy \cite{zhuo2021human}, improved ability to follow} instructions while introducing \highlight{unpredictable} variations \cite{weinberg2007robotic, bruun2020human}, \highlight{and the ability to perform feats of speed or scale that would require complex coordination of multiple humans \cite{weinberg2006toward, bretan2016survey}. For example, Haile, a robot that plays a Native American drum, exceeds human speed \cite{weinberg2006toward}, while other systems utilize three-dimensional space to achieve orchestral effects \cite{bretan2016survey, flo2015doppelganger}. However, these systems typically employ instrument actuation methods that mimic human movements, still aiming to produce music that sounds as human-like as possible.}

% Our work diverges from prior approaches by explicitly leveraging robots' unique capabilities rather than emulating human musicianship. Through the Beatbots system, we explore new artistic possibilities enabled by algorithmic percussion music and novel actuation methods. While this approach may produce unconventional musical output, we believe it will advance the development of distinctly robotic music systems. Throughout our design process, we prioritized input from human musicians [51] to ensure our system maintains artistic integrity while pushing the boundaries of traditional musicianship.

\highlight{Our work diverges from prior approaches by explicitly leveraging robots' unique capabilities rather than emulating humans. Through the \textit{Beatbots}, we explore new artistic possibilities enabled by our algorithmic percussion music and novel actuation method. While this approach may produce unconventional musical output, we believe it will lead to new insights toward distinctly robotic music systems and musical culture of the future \cite{rowe2004machine}. Given this divergence, we prioritized input from human musicians \cite{vear2024jess} throughout our design process to ensure our system maintains artistic integrity while pushing the boundaries of traditional musicianship.}