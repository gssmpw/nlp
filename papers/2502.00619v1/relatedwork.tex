\section{Related Work}
\subsection{Fairness Learning in Medical Image Segmentation}
Fairness-oriented medical image segmentation datasets with demographic information \cite{tschandl2018ham10000, tianfairseg} have enabled research on bias mitigation, driving the development of robust fairness training strategies. Advanced generative approaches \cite{li2024fairdiff, ktena2024generative} aim to generate diverse samples from skewed distributions to mitigate bias. However, challenges remain in addressing the computational demands and achieving high-quality sample generation for high-dimensional medical images, such as 3D whole-body computed tomography (CT), for use in augmentation datasets. On the other hand, fairness-focused loss function modifications, such as distributionally robust optimization (DRO) \cite{sagawa2019distributionally} and fair error-bound scaling (FEBS) \cite{tianfairseg}, have been proposed to integrate fairness considerations directly into the optimization process. While partially effective, these methods are vulnerable to the data distribution within the training batch, limiting their applicability in 3D medical image segmentation, where large batch sizes are constrained. 
%Furthermore, existing benchmarks and studies primarily focus on demographic attributes \cite{tian2025fairdomain, jin2024fairmedfm}, often overlooking critical \blue{clinical} factors such as tumor progression and metastasis, which are essential for medical image analysis. In this study, we propose a general method that can be broadly applied to mitigate bias arising from both demographic and \blue{clinical} factors. 
\blue{
Furthermore, existing benchmarks and studies primarily focus on demographic attributes \cite{tian2025fairdomain, jin2024fairmedfm}, often overlooking critical clinical factors, such as tumor progression and metastasis, which contribute to regional variations in clinical practice patterns due to inherent biases. In this study, we propose a general method to mitigate bias arising from both demographic and clinical factors.}
%, including those shaping regional practice differences.}
 
\subsection{Mixture of Expert in Multi-distribution Learning}
Recent advances in the MoE framework \cite{shazeer2017outrageously} have demonstrated remarkable potential for adapting AI models to diverse data distributions, particularly within continual learning paradigms. MoE enhances robustness and adaptability when confronted with previously unseen data patterns \cite{rypesc2024divide, yu2024boosting}. In the medical domain, MoE has been effectively extended to address challenges such as multimodal integration \cite{jiang2024m4oe}, scanning modality heterogeneity \cite{zhang2024foundation}, and catastrophic forgetting issues in continual learning \cite{chen2024low, wang2024sam}, unifying these approaches into a cohesive framework that enhances performance. However, theoretical insights into how MoE facilitates the adaptation of disparate distributions to a target distribution remain limited. {This study clarifies the underlying mechanism of MoE as dynamic parameter selection,  and integrates environmental attributes into its gating mechanism, enabling distributional adaptation for medical image segmentation. }

\subsection{Training Neural Networks as Optimal Control}
Neural networks, especially those with shortcut connections, perform complex transformations through successive modifications of a hidden state. These networks can be conceptualized as undergoing a continuous dynamic process, which can be described using ordinary differential equations (ODEs) \cite{weinan2017proposal, lu2018beyond}. Training these networks resembles solving an optimal control problem, where the objective is to adjust the network parameters, i.e., weights, to minimize a loss function \cite{chen2018neural, sun2024layer}. %{This approach is analogous to determining the optimal control path from an input  to its segmented result.}

{Fixed-architecture feedforward neural networks can be regarded as operating under a non-feedback control. Employing a consistent strategy, however, can be restrictive in dynamic environments \cite{aastrom1995adaptive}.}
To address this limitation, feedback control mechanisms \cite{doyle2013feedback, aastrom2021feedback} offer an alternative by enabling continuous monitoring and adjustment based on the system's outputs and desired targets.
Furthermore, mode-switching control \cite{yamaguchi1996mode, boskovic2000multi} introduces additional flexibility by enabling the policy to alternate between multiple optimized operational modes in response to external inputs. This capacity is essential for managing complex systems across different conditions and offers a robust response to diverse operational challenges \cite{yu2017sliding}. %In the context of image segmentation, it is similarly beneficial to adopt varying strategies based on the type of image, which motivates our method.
Likewise, in the context of image segmentation, adopting varying strategies— such as leveraging distributional attributes of the input—proves beneficial and serves as the motivation for our method.





\begin{figure*}[ht]
\vskip 0.2in
\begin{center}
\centerline{\includegraphics[width=2\columnwidth]{fig/main.pdf}}
\caption{(a) Schematic of the dMoE segmentation network for fairness learning, and (b) its interpretation through a control system.}
\label{fig_main}
\end{center}
\vskip -0.2in
\end{figure*}