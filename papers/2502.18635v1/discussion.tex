\section{Discussion}
\label{sec:discussion}
We frame our discussion as takeaways for industry practitioners that aim to optimize configurations of a RAG pipeline in a multi-objective setting. The first consideration
is what we call \emph{objective relationships}. In our experiments, we found that safety and alignment are often positively correlated with each other, and similarly for cost and latency. However, these two sets of objectives involve conflicting parameters, which makes it challenging to set a suitable trade-off between reliability (safety and alignment) and efficiency (cost and latency). Resolving conflicts between objectives is inherently challenging and remains an open question; we recommend that practitioners be thoughtful about latent relationships between objectives when choosing which objectives to optimize over. %In our experiments, alignment is significantly affected by the choice of parameters, and likely needs to be controlled by prompting.

The second consideration is \emph{task dependence}, meaning that an optimal configuration for a RAG pipeline on a task in one setting may not generalize to another. While we observe configurations that work well across both tasks with respect to certain objectives (\eg high chunk size correlates with higher safety and alignment%\evancomment{Dangerous implication to include here - perhaps `correlates with' rather than `results in'}
), there is no optimal configuration that is shared across the two tasks. Practitioners should be aware that the optimal configuration will be highly dependent on the task they are building for, including the way their system will be used, the domain, and the stakeholders. 

The third consideration is \emph{objective dependence}, where objective evaluations follow different trends (or have no trend) across different configurations. For example, we see high objective dependence for temperature, since high temperatures (\eg $>1.0$) consistently reduced all four objectives. However, it is harder to discern a relationship between chunk overlap and the objectives, indicating low objective dependence. Task and objective dependence can also compound, highlighting the challenge of collectively optimizing the configuration of a RAG system.

\paragraph{Future Work} Our results demonstrate that end-to-end optimization of RAG systems is a promising avenue for research. We highlight two key areas for future exploration: first, improving the efficiency of Algorithm~\ref{alg:1}. A potential direction is ``decoupled evaluation'', where not all objectives are assessed at every iteration, towards a cost-aware optimization strategy. For instance, evaluating safety and alignment is significantly more expensive than measuring cost and latency.

Second, improvements can be made to the framework itself. Our current approach does not account for prompt engineering, despite its significant impact on response quality. Additionally, our safety and alignment metrics considered in this work remain limited in scope, excluding aspects such as toxicity and data leakage. However, our framework is inherently flexible and can be extended with adapted objectives or additional parameters. Another open challenge is configuration selection from the Pareto frontier. As Figure \ref{fig:pareto-front} illustrates, multiple configurations are Pareto-optimal, making the selection of the most suitable trade-off for a given application non-trivial.

