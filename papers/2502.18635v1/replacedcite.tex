\section{Related work}
\label{sec:related}
There has been a mixture of work separately addressing multi-objective optimization and hyperparameter optimization in LLM and RAG systems, which we summarize here. We provide further comparison with fine-tuning and model-merging approaches in Appendix ~\ref{appendix:related}.

\paragraph{Hyperparameter optimization (HO).} As many pieces of LLM and RAG pipelines have hyperparameters that must be tuned before deployment, there is a large body of work testing the efficacy of using HO to tune these systems. %As mentioned earlier, HO in these settings is particularly difficult due to the high number of parameters; evaluation during each iteration of the HO process can be very expensive and time-intensive, and evaluations are often noisy.

____ propose a cost-based pruning strategy to hyperparameter tune LLM systems under budget constraints. They focus on hyperparameters like the type of model (\eg text-davinci-003, gpt-3.5-turbo, or gpt-4), the maximum number of tokens that can be generated in a response, the model temperature\footnote{A parameter for which low or high values sharpen or soften the probability distribution of a token being outputted by an LLM, respectively.}, and the model top-$p$.\footnote{A parameter that restricts the domain of tokens that can be outputted by an LLM to those whose cumulative probability is greater than $p$.} They use a search method called BlendSearch____, which combines Bayesian optimization and local search____, to find the optimal combinations of parameters, and measure performance on the tasks APPS____, XSum____, MATH____, and HumanEval____. In comparison with our work, ____ do not consider a RAG system.

Most related to our work, ____ proposed AutoRAG, an open-source framework designed for RAG experimentation and hyperparameter optimization. They use a a greedy algorithm for selecting the hyperparameters governing RAG modules like query expansion, retrieval, passage augmentation, passage re-ranking, prompt making, and generating (the LLM). In concurrent work, ____ proposed AutoRAG-HP, which frames hyperparameter selection as an online multi-armed bandit (MAB) problem. To carry out HO, they introduce a novel two-level Hierarchical MAB (Hier-MAB) method, where a high-level MAB guides the optimization of modules, and several low-level MABs search for optimal settings within each module. Significantly, our work is distinct from both ____ and ____ in that they do not consider multi-objective settings.

\paragraph{Multi-objective alignment.} Several researchers have proposed methods for incorporating multiple objectives directly into the LLM fine-tuning and alignment processes. ____ developed an approach for multi-objective alignment from human feedback using scalar linearization. ____ expanded on that approach by developing an algorithm that finds a diverse set of Pareto-optimal solutions that maximize the hypervolume, given a set of objectives.____ proposed a reward-function free extension called Multi-Objective Direct Preference Optimization (MODPO). The latter showed that MODPO can effectively find a Pareto-optimal frontier of fine-tuned models, trading off objectives like ``helpfulness'' and ``harmlessness''. While these works have demonstrated success in multi-objective LLM alignment, we focus on RAG pipelines and avoid aligning and fine-tuning models altogether. 
\begin{table}[t]
\caption{Common choices, parameters, and hyperparameters that are often made/tuned when implementing RAG pipelines. \textbf{Bold} indicates a parameter that was optimized over in our experiments.}
\label{tab:system_params}
\begin{tabular}{p{2cm}p{3.4cm}p{7.1cm}}
\toprule
\textbf{Domain} & \textbf{Parameters} & \textbf{Notes} \\
\midrule
\multirow{2}{*}{System-level} & \textbf{LLM model} & \eg gpt-4, llama-3.1-8b, llama-3.1-70b \\
 & \textbf{Embedding model} & For RAG pipelines, \eg text-embedding-ada-002 \\
\midrule
\multirow{4}{*}{LLM controls} & System prompt &  \\
 & \textbf{Temperature} &  \\
 & Top-$k$, top-$p$ &  \\
 & Max length of output &  \\
\midrule
\multirow{9}{*}{Fine-tuning} & Preference-tuning approach & \eg RLHF, DPO, KTO; these methods may also introduce hyperparameters that must be tuned, \eg DPO uses $\beta$, KTO uses $\beta, \lambda_U, \lambda_D$ \\
 & Parameter-Efficient Fine-Tuning (PEFT) & \eg LoRA, adapter modules~\citep{houlsby2019parameter}; these methods may also introduce hyperparameters, \eg LoRA uses rank and scaling $\alpha$ \\
 & Dropout rate &  \\
 & Learning rate &  \\
 & Training epochs &  \\
\midrule
\multirow{7}{*}{RAG controls}
& Modules & Approaches for query expansion, retrieval, passage augmentation and re-ranking, and prompt making%\evancomment{List seems weirdly written here? Is `passage augmentation and re-ranking' one thing or should the `and' be before `prompt making' instead?}
\\
& \textbf{Chunk size} &  \\
& \textbf{Number of chunks} & \\
& \textbf{Chunk overlap} & \\
& \textbf{Re-rank threshold} \\
& Retrieval size (top-$k$) & \\
\bottomrule
\end{tabular}
\end{table}