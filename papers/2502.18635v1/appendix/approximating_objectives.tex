\section{Approximating Objective Functions}
\label{sec:approx_objective}
Equation \ref{eq:objective} defines our proposed task as the maximization of the $\mathcal{HV}$ across multiple objectives. Each objective function evaluates a property of the RAG system for a given workload. We define a workload as a probability distribution across all possible queries, $P(q)$, where $q \in \mathcal{Q}$ is a user query for the RAG system. Further, we use $f_m^q: \mathcal{X}, \mathcal{Q} \mapsto \mathbb{R}$  to denote an evaluation function for an individual query and objective $m$. Using these definitions, we can write down the objective function as the expectation across queries:
\begin{equation}
    f_m(\mathbf{x}) = \sum_{q \in \mathcal{Q}} f_m^q(\mathbf{x}, q) P(q)
\end{equation}

Assuming we can sample queries from the workload, $q \sim P(q)$, we can use a Monte Carlo approximation for each objective function using a sample set $\mathcal{Q'} \subset \mathcal{Q}$:
\begin{equation}
\label{eq:mc}
    f_m(\mathbf{x}) \approx \frac{1}{|\mathcal{Q'}|}\sum_{q \in \mathcal{Q'}} f_m^q(\mathbf{x}, q)
\end{equation}
We assume that generating data synthetically using an LLM (Section \ref{sec:synthetic_data}) is equivalent to sampling queries $q \sim P(q)$ to obtain $\mathcal{Q'}$.
Our algorithm uses Equation \ref{eq:mc} as a tractable approximation of the objective functions in Equation \ref{eq:moo}.