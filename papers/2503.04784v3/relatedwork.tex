\section{Related work}
Since the introduction of the Generative Pretrained Transformer (GPT), the field of large language models (LLMs) has witnessed a remarkable surge in innovation, fundamentally reshaping the landscape of natural language processing. This section offers an in-depth review of recent technical developments, spotlighting the key advancements that have influenced the direction of our research.

OpenAI's GPT series has been a trailblazer in the development of LLMs. GPT-1\cite{r2}  introduced the concept of unsupervised pre-training followed by fine-tuning, thereby setting a new benchmark for language understanding. GPT-2\cite{r3} further expanded on this approach by increasing the model's scale and training data, demonstrating remarkable zero-shot learning capabilities. GPT-3\cite{r4}, with its massive 175 billion parameters, revolutionized few-shot and one-shot learning. The subsequent releases of GPT-3.5 and GPT-4 have continued to push the boundaries, delivering even more impressive performance across a wide range of natural language tasks\cite{r9}.

Meta's LLaMA\cite{r10} has gained significant traction due to its focus on efficiency and scalability. It offers a more streamlined architecture compared to some of its predecessors while maintaining competitive performance. LLaMA has been widely adopted and adapted for various applications, highlighting its versatility and potential in different research and practical scenarios.
ByteDance's Doubao\cite{r22} is another powerful model. It is trained on extensive datasets, enabling it to understand and generate text across a diverse range of topics. Doubao excels in providing accurate and contextually relevant responses, making it a valuable tool for users seeking information and engaging in conversations. Alibaba's Qwen\cite{r11} is known for its high-performance natural language processing capabilities. It supports wide array of tasks, including text generation, question answering, and translation. Qwen is trained on a vast corpus of multi-modal data, which enhances its ability to handle complex language scenarios and provide comprehensive answers. DeepSeek\cite{r12} has also made a mark with its high-quality language generation and understanding. The model is designed to handle large-scale datasets and complex language structures, offering precise and coherent responses. DeepSeek's architecture is optimized for efficient training and inference, making it suitable for a variety of applications.Google's PaLM\cite{r13} is trained on massive datasets to capture a broad spectrum of language knowledge and semantic understanding. It has demonstrated excellent performance in tasks such as language translation, text summarization, and dialogue systems.

The Transformer architecture, the foundation of many modern LLMs, has undergone continuous refinement. Researchers have explored ways to enhance the self-attention mechanism. For instance, some studies have introduced multi-scale attention\cite{r14}, allowing the model to capture dependencies at different granularities. This approach enables the model to better handle long-range and short-range semantic relationships in text.Hybrid architectures combining Transformers with other neural network components have also been proposed. The integration of convolutional layers with Transformers\cite{r15} can capture local linguistic patterns more effectively. Convolutional filters can extract local features such as n - grams, complementing the global attention mechanism of Transformers.


To train large language models more efficiently, advanced training strategies have been developed. Curriculum learning\cite{r16} has been applied to LLMs, where the model is first trained on simpler tasks or data and gradually exposed to more complex ones. This approach can accelerate convergence and improve the model's generalization ability.Another significant development is the use of efficient optimization algorithms. Adaptive learning rate schedulers, such as AdamW\cite{r17}, have been widely adopted. AdamW adjusts the learning rate during training, helping the model converge faster and avoid overfitting. Additionally, techniques like gradient accumulation\cite{r18} allow for larger effective batch sizes when memory is limited, improving the stability of training.


Tokenization methods have evolved to better represent text for LLMs. Byte - Pair Encoding (BPE)\cite{r19} is a popular approach that splits words into subword units, reducing the vocabulary size while retaining the ability to represent rare or out-of-vocabulary words. Some recent works have proposed enhanced tokenization techniques that consider semantic and syntactic information\cite{r20}, resulting in more meaningful token representations.In terms of embeddings, researchers have explored ways to incorporate contextual information. Contextualized word embeddings, such as ELMo\cite{r21}, provide representations that vary depending on the surrounding text. This allows the model to capture the polysemous nature of words more effectively, improving performance in tasks that require fine - grained semantic understanding.In summary, the technical landscape of large language models has been rapidly evolving. 

Our proposed framework builds on these advancements, introducing novel mechanisms to further enhance the performance and capabilities of LLMs.






\begin{figure}[htbp]
    \centering
    \includegraphics[width=\textwidth]{1.pdf}
    \caption{The architecture of KunlunBaize}
    \label{fig:your_label}
\end{figure}

\begin{figure}[htbp]
    \centering
    \includegraphics[width=\textwidth]{2.pdf}
    \caption{Densely leanable residual connected TransformerX blocks}
    \label{fig:your_label}
\end{figure}

\begin{figure}[htbp]
    \centering
    \includegraphics[width=\textwidth]{3.pdf}
    \caption{TransformerX}
    \label{fig:your_label}
\end{figure}