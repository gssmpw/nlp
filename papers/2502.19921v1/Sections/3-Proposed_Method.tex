\section {Method}
\label{sec:Prop_method}

\subsection{Notations}
We use bold lowercase symbols $(\boldsymbol{\mathrm{x}})$ for time series. 
The parametric mappings are represented as $f_{\theta}(.)$ where $\theta$ is the parameter.
The discrete Fourier transformation of a time series is denoted as $\mathcal{F}(\mathrm{x})$, yielding a complex variable $|X(e^{j \omega})| e^{j {\phi(\omega)}}$ which contains magnitude and phase information of each harmonic (sinusoidal). 
$\phi(\omega_k)$ and $\mathrm{T}_k$ represent the phase angle and period of the $k$-th harmonic with frequency $\omega_k$.
We mainly used the textbook notations~\citep{Oppenheim} throughout the script, providing a comprehensive list of notations and detailed definitions in Appendix~\ref{appen:notations}.

\subsection{Objective}
Given a dataset $\mathcal{D} = \{  (\boldsymbol{\mathrm{x}}(t)_i, \mathbf{y}_i) \}_{i=1}^K$ where each $\boldsymbol{\mathrm{x}} \in X$ consists of uniformly sampled real-valued values and each $\mathbf{y} \in Y$ represents the corresponding labels,
the objective is to have consistent and accurate outputs for all variants of a sample that are subjected to shifts\footnote{We represent a time shift ($t^{\prime}$) for a sample $\boldsymbol{\mathrm{x}}$ as $\boldsymbol{\mathrm{x}}(t-t^{\prime})$, similar to~\citet{Oppenheim}. All the time shifts throughout the paper imply circular shift, i.e., $(t-t^{\prime}) = (t-t^{\prime})_{\% t}$ where \% is the modulus.} such that when a parametric model $f_{\theta}: X \rightarrow Y$ is evaluated on the set $\mathcal{D}_{\text{test}} = \{  (\boldsymbol{\mathrm{x}}(t)_i, \mathbf{y}_i) \}_{i=1}^L$, the output will be the same and true $\mathbf{y}_i$ for all $t^{\prime}$ to be shift-invariant, i.e., $\mathbf{y}_i=f_{\theta}(\boldsymbol{\mathrm{x}}(t-t^{\prime})_i),  \forall t^{\prime} \in \mathbb{R}$.


We propose a diffeomorphism that maps randomly shifted time series samples to the same point in data space, preserving all relevant information to ensure shift-invariance.
The motivation and theoretical derivation of our method are presented in the following steps.


% |X(e^{j \omega})| e^{j \phi(\omega)} \equiv |X(e^{j \omega})| e^{j \phi(\omega)} e^{-j \omega t^{\prime}} e^{-j \omega \Delta t}

% \begin{proposition}[Consistency Under Shift]\label{prop:shift_change}
% The task relevant information that raw time series data conveys about the labels  (i.e., $\mathcal{I}(\mathbf{y}; \mathbf{x})$) remains constant despite a random time shift $t^{\prime}$.
% \begin{gather}\label{eq:prop_shift_change}
%         \mathcal{I}(\mathbf{y}; \mathbf{x}) = \mathcal{I}(\mathbf{y}; \mathbf{x}(t-t^{\prime})) \implies
%          \mathbf{x} \equiv \mathbf{x}(t-t^{\prime}) \equiv \mathbf{x}(t-t^{\prime}-\Delta t) \\
%          \mathcal{I}(\mathbf{y}; \mathbf{x}) = \mathcal{I}(\mathbf{y}; \mathbf{x}(t-t^{\prime} - \Delta t))
% \end{gather}
% \end{proposition}
% \begin{proof}
% From shift-consistency,
% \begin{equation}
%     \mathbf{y}=f_{\theta}(\boldsymbol{\mathrm{x}}) = f_{\theta}(\boldsymbol{\mathrm{x}}(t-t^{\prime})) \implies \mathcal{I}(\mathbf{y}; \mathbf{x}) = \mathcal{I}(\mathbf{y}; \mathbf{x}(t-t^{\prime}))
% \end{equation}
% $\text{yields}\ \mathbf{x} \equiv \mathbf{x}(t-t^{\prime}),\ \text{where}\ \forall t^{\prime} \in \mathbb{R} \hspace{2mm} \therefore \hspace{2mm} \mathbf{x} \equiv \mathbf{x}(t-t^{\prime}-\Delta t)$
% % Therefore, $\mathbf{x} \equiv \mathbf{x}(t-t^{\prime}-\Delta t)$.
% \end{proof}

\begin{proposition}[Time shift as a Group Operation]\label{prop:shift_change}
Shift operation in time domain defines an Abelian Group of phase angles in the frequency domain for each harmonic with frequency $\omega_k$.
\begin{gather}\label{eq:prop_shift_change}
        (\Phi_k, + \hspace{-2mm} \mod 2\pi), \hspace{2mm} \text{where} \hspace{2mm} \Phi_k = \{ \phi \mid \phi = (\phi(\omega_k) + \omega_k t^{\prime}) \text{ mod } 2\pi,  t^{\prime} \in \mathbb{R} \}
\end{gather}
\end{proposition}
\begin{proof}
Using $\mathcal{F}(x(t+t^{\prime})) = |X(e^{j \omega})| e^{j {\phi(\omega)}} e^{j \omega t^{\prime}}$, and the multiplication of complex numbers 
\begin{equation} \label{eq:prop_1}
    \exists t^{\prime} \in \mathbb{R},\ \forall \phi \in (-\pi,\pi], \hspace{2mm} \phi = (\phi(\omega_k) + \omega_k t^{\prime} )\text{ mod } 2\pi
\end{equation}
See Appendix~\ref{appen:proof} for detailed proof with group axioms.
\end{proof}


\begin{wrapfigure}[22]{t}{8.4cm}
\vspace{-5mm}
    \centering
    \includegraphics[width=0.52\columnwidth]{Figures/proof_cover.pdf}
    \caption{\textbf{(a)} Frequency domain representation of a harmonic at frequency $\omega_0$ with different phase angles in unit circle.
    \textbf{(b)} Time domain representation of a signal $\mathrm{x}(t)$ and its shifted version $\mathrm{x}(t-t^{\prime})$.
    The phase angle of the harmonic can cover all (i.e., surjective $\mathcal{T}(\mathbf{x}, \phi)$) potential shifts.
    Moreover, shifts in the time domain correspond to unique (i.e., injective $\mathcal{T}(\mathbf{x}, \phi)$) angle rotations in the frequency domain for the sinusoidal with periodicity $\mathrm{T}_0$.
    Therefore, the proposed transformation function $\mathcal{T}(\mathbf{x}, \phi)$ is bijective.}
    \label{fig:cover_proof}
\end{wrapfigure}

Proposition~\ref{prop:shift_change} states that the shift variants of a sequence define a group of phase angles, known as circle group~\citep{fuchs1960abelian} $\mathbb{T}$.
An important observation from Equation~\ref{eq:prop_1} is that different shift values ($t^{\prime}$) can map to the same phase angle ($\phi$) due to modulo operation with $2\pi$.

% \begin{wrapfigure}[22]{t}{8.4cm}
% \vspace{-5mm}
%     \centering
%     \includegraphics[width=0.52\columnwidth]{Figures/proof_cover.pdf}
%     \caption{\textbf{(a)} Frequency domain representation of a sinusoidal at frequency $\omega_0$ with different phase angles in unit circle.
%     \textbf{(b)} Time domain representation of a signal $\mathrm{x}(t)$ and its shifted version $\mathrm{x}(t-t^{\prime})$.
%     The phase angle of the sinusoidal can cover all (i.e., surjective $\mathcal{T}(\mathbf{x}, \phi)$) potential shifts.
%     Moreover, shifts in the time domain correspond to unique (i.e., injective $\mathcal{T}(\mathbf{x}, \phi)$) angle rotations in the frequency domain for the sinusoidal with periodicity $T_0$.
%     Therefore, the proposed transformation function $\mathcal{T}(\mathbf{x}, \phi)$ is bijective.}
%     \label{fig:cover_proof}
% \end{wrapfigure}


However, a closer look reveals that this mapping can be defined uniquely for specific harmonics using the circular shift.
Specifically, we can represent every point in the shift space uniquely with the phase angle of a harmonic whose period is equal to or longer than the length of sample, i.e., $\mathrm{T}_0 \leq t$.
In the remainder of this section, we explain how this observation is framed as a novel diffeomorphism.
We denote the frequency, period, and phase of this specific harmonic as $\omega_0$, $\mathrm{T}_0$, and $\phi(\omega_0)$, respectively.

% Since shifts in finite-length time series wrap around when they exceed the total duration---shifting a 2-second sample by four seconds results the same signal---we can represent every point in the shift space uniquely with the phase angle of a harmonic whose period is equal to or longer than the length of sample, i.e., $\mathrm{T}_0 \leq t$.


% In the rest of the section, which explains how this observation is framed as a novel diffeormophism, we denote the frequency, period and phase of this specific harmonic as \(\omega_0\), \(\mathrm{T}_0\) and \(\phi(\omega_0)\), respectively.

% Additionally, since time series can be decomposed into harmonics using the Fourier transform, we define each sample as lying on a manifold \(M^{\phi}\) according to the phase angle \(\phi(\omega_0)\) of this harmonic for that sample.




% Proposition~\ref{prop:shift_change} states that if a shifted version of a time series sample contains all necessary information for the downstream task, further shifting it by a random value will not result in information loss, provided they belong to the same closed set when considering the label space.
% While this proof may seem intuitive, it forms the foundational basis for our proposed method.

The proposed transformation function, $\mathcal{T}(\mathbf{x}, \phi)$, takes a sample $\mathbf{x}$ and an angle $\phi \in (-\pi, \pi]$.
It then applies a linear phase shift to each harmonic, mapping the time series to a new variant where the phase angle of the harmonic with frequency $\omega_0$ matches the desired angle $\phi$.
The proposed transformation, which converts a time series to another shifted variant, is defined as in Equations~\ref{eq:phase_shift} and~\ref{eq:phase_shift2}.
\begin{gather}\label{eq:phase_shift}
   \mathbf{x}(t) \xrightarrow[]{\mathcal{T}(\mathbf{x},\phi)} \mathcal{F}^{-1}(|X(e^{j \omega})| e^{j \phi(\omega)} e^{-j \omega \Delta \phi}) \hspace{3mm} \text{where}
   \\
   \begin{split}
       \Delta \phi = 
    \begin{cases}
      \frac{(\theta - 2\pi) * T_0}{2 \pi}, & \text{if } \theta > \pi \\    
      \frac{\theta * T_0}{2 \pi}, & \text{else}
    \end{cases}
    \hspace{3mm} \text{and} \hspace{2mm}
    \theta = [\phi(\omega_0) -\phi]\ \%\ 2\pi \hspace{3mm}
    \end{split} 
   \label{eq:phase_shift2}
\end{gather}
% Specifically, the transformation represents every possible point in the shift space by using the phase of a sinusoidal, with a period longer or equal to the length of the sample, thereby ensuring coverage of all shifts.
Mainly, the transformation first decomposes a signal to its harmonics, then it calculates the phase difference, denoted as $\Delta \phi$, between the harmonic with frequency $\omega_0$ and the desired angle $\phi$.
Finally, It returns to the time domain by taking the inverse Fourier transform, $\mathcal{F}^{-1} (.)$, while applying a linear phase shift to all harmonics to preserve the waveform morphology.
In the end, the transformation matches the phase angle of the harmonic at frequency $\omega_0$ with the desired angle $\phi$.
We first demonstrate that the proposed transformation is a bijective function, as shown in Theorem~\ref{theorem:guarentees_shifting}.
\begin{theorem}[Covering the Entire Time Space Injectively]\label{theorem:guarentees_shifting}
Given a sample $\mathbf{x}$, the defined function $\mathcal{T}(\mathbf{x}, \phi): \Phi \times \mathbb{R}^d \rightarrow \mathbb{R}^d \times \Delta \Phi$ is bijective such that all shift variants of a sample can be covered with the unique phase angle of a harmonic whose period is longer or equal to the length of $\mathbf{x}$. 
\begin{gather*}
\forall \phi_a, \phi_b \in \Phi,\  \mathcal{T}(\mathbf{x}, \phi_a) = \mathcal{T}(\mathbf{x}, \phi_b) \implies \phi_a = \phi_b \\
\forall t^{\prime} \in \mathbb{R},\ \exists \phi \in \Phi,\ \mathcal{T}(\mathbf{x},\phi) = \left(\mathbf{x}(t - t^{\prime}), \Delta \phi\right),
% \\
% \mathcal{T}(\mathbf{x}, \phi) = (\mathbf{x}(t - t^{\prime}), \Delta \phi)
\end{gather*}
where the first and second equations represent the injection and surjection, respectively.
\end{theorem}
% \begin{proof}
%     \begin{gather}
%     \mathbf{x}(t-t^{\prime}) \xrightarrow[]{\mathcal{F(.)}} |X(e^{j \omega})|  e^{j ( \phi(\omega) - \omega t^{\prime})} \\
%     \phi_{\mathbf{x}(t)} - \phi_{\mathbf{x}(t-t^{\prime})} = \left(\frac{2 \pi}{T_0} t^{\prime}\right)_{\%\ 2 \pi} \hspace{-1mm} \text{if} \hspace{2mm} T_0 \geq t^{\prime} 
%     \end{gather}
% \end{proof}
% Theorem~\ref{theorem:guarentees_shifting} shows that the presented transformation is a bijective function.
We provide an intuitive demonstration in Figure~\ref{fig:cover_proof}, with a detailed mathematical proof in Appendix~\ref{appen:proof}.
Since each point in the shift space can be uniquely defined by the phase angle of a harmonic with period $\mathrm{T}_0$, we use the angle of this harmonic to define manifolds~\footnote{The manifold is defined as a $\mathit{d}$-dimensional Euclidean space, matching the data's dimension, to better explain the abstract transformation. There is no manifold learning of low-dimensional space in our transformation.}, $\mathcal{M}^{\phi}$, on which the samples lie.
Specifically, we apply the proposed transformation $\mathcal{T}(\mathbf{x}, \phi)$ for each sample to map it to a manifold defined by the angle, i.e., $\mathcal{T}(\mathbf{x}, \phi_a) \in \mathcal{M}^{\phi_a}$, $\mathcal{T}(\mathbf{x}, \phi_b) \in \mathcal{M}^{\phi_b}$, and $\bigcap_{i=0}^{2\pi} \mathcal{M}^{\phi_i} = \emptyset$ (See Appendices~\ref{sec:Diffeomorphisms} and~\ref{appen:notation_list} for detailed definition of manifolds and notations).
\begin{figure*}[b]
    \centering
    \includegraphics[width=\textwidth]{Figures/overall_NIPS.pdf}
    \caption{\textbf{(a)} An input signal in the time domain and complex plane representation of its decomposed sinusoidal of frequency $\omega_0 = \frac{2\pi}{T_0}$ with the phase angle $\phi_0$.
    \textbf{(b)} Guiding the diffeomorphism to map samples between manifolds.
    \textbf{(c)} The obtained waveform with a phase shift applied to all frequencies linearly, calculated by the angle difference, as in Equation~\ref{eq:phase_shift2}, without altering the waveform.
    \textbf{(d)} The loss functions for optimizing networks with the cross-entropy and the variance of possible manifolds.}
    \label{fig:overall_idea}
\end{figure*}
We, therefore, can map a sample and its randomly shifted variants to the same point in the space, which is sufficient for providing shift-invariancy as demonstrated in Theorem~\ref{theorem:guarentees_transformation} with a detailed proof in Appendix~\ref{appen:proof}.
\begin{theorem}[Guarantees for Shift-Invariancy]\label{theorem:guarentees_transformation}
Given $\mathbf{x}$ and a randomly shifted variant of it $\mathbf{x}(t-t^{\prime})$, if $\mathcal{T}(\mathbf{x}, \phi)$ is applied to both samples with the same angle $\phi_a$, the resulting samples will be the same.
\begin{gather*}\label{eq:guarentees_transformation}
% \mathcal{T}(\mathbf{x}(t), \phi_a) = \mathcal{T}(\mathbf{x}(t-t^{\prime}), \phi_a)
\mathcal{T}(\mathbf{x}(t), \phi_a) = \left( \Tilde{\mathbf{x}}(t),\ \Delta \phi_{\mathbf{x}(t)} \right), \hspace{3mm} \mathcal{T}(\mathbf{x}(t - t^{\prime}), \phi_a) = \left( \Tilde{\mathbf{x}}(t),\ \Delta \phi_{\mathbf{x}(t - t^{\prime})} \right)
\end{gather*}
\end{theorem}
\begin{proof}
    \begin{gather}
        \phi_{\mathbf{x}(t)} = \phi(\omega), \hspace{1mm} \phi_{\mathbf{x}(t-t^{\prime})} = \phi(\omega) - \omega t^{\prime}, \hspace{3mm}
    \Delta \phi_{\mathbf{x}(t-t^{\prime})} - \Delta \phi_{\mathbf{x}(t)} =  -\omega_0 \frac{T_0}{2\pi} t^{\prime}  \\
    \phi_{\mathcal{T}(\mathbf{x}(t-t^{\prime}), \phi_a)} - \phi_{\mathcal{T}(\mathbf{x}, \phi_a)} = \left[ \frac{T_0}{T} \omega_0 - \omega \right] t^{\prime}, \hspace{3mm}
    \phi_{\mathcal{T}(\mathbf{x}, \phi_a)} = \phi_{\mathcal{T}(\mathbf{x}(t-t^{\prime}), \phi_a)}
    \end{gather}
    Therefore, the output time series samples will be the same after applying the transformation.
    % $\mathcal{T}(\mathbf{x}, \phi_a) = \mathcal{T}(\mathbf{x}(t-t^{\prime}), \phi_a)$.
\end{proof}
The proof concludes by demonstrating that the harmonics retain the same phase and magnitude after transformation, despite an unknown shift applied to the sample.
Moreover, since the transformation only contains exponentials with Fourier transform, it is fully differentiable, allowing optimization with neural networks.
Therefore, we use a guidance network $f_{G_{\theta}}: \mathbb{R}^d \rightarrow \Phi$ with a shift-invariant input, absolute Fourier transform of samples, to generate an angle in radians for mapping.
Simultaneously, the main classifier $f_{C_{\theta}}: X \rightarrow Y$ maps the transformed samples to the label space.
Both networks are optimized with cross-entropy loss.
The optimizer for the guidance network $(\mathcal{L}_G)$ has an additional loss term to reduce variations in a batch ($\mathcal{B}$) of angles, as given in Equation~\ref{eq:lossess}.
\begin{equation}\label{eq:lossess}
    \mathcal{L}_C = -\sum_{i=1}^{N} \sum_{j=1}^{C} \mathrm{y}_{ij} \log f_{C_{j}}(\mathcal{T}(\mathbf{x}_i, \phi_i)) \hspace{4mm} \mathcal{L}_G = \mathcal{L}_C + \sqrt{\text{Var}_{\mathbf{x} \sim \mathcal{B}} \left( f_{\theta_G}\left(|\mathcal{F}(\mathbf{x})|\right) \right)}
    % \sqrt{\mathbb{V}_{\mathbf{x} \sim \mathcal{B}} \left( f_{\theta_G}\left(|\mathcal{F}(\mathcal{T}(\mathbf{x}, \boldsymbol{\phi}))|\right) \right)}
    % \sqrt{\text{Var}(f_{\theta_G}(|\mathcal{F}(\mathcal{T}(\mathbf{x}, \boldsymbol{\phi}))|))} 
\end{equation}
% The guidance network which is optimized by the proposed loss works as an adaptive linear dimensionality reduction in the entire data space.
The guidance network, optimized by the proposed loss, works as an adaptive linear constraint that limits the regions in the original data space where samples can be found.
In other words, if we conceptualize the data space of samples as expanding with shift variants, as illustrated in Figure~\ref{fig:overall_idea}, the model learns to reduce the potential points where samples can be found in the data space.

Additionally, for real-world samples where optimal phase shift values are unavailable, applying trivial phase shifting may lead to suboptimal data space representations.
To address this, we transform the data space using the proposed diffeomorphism for the downstream tasks using the guidance network (see Appendix~\ref{appendix:visual_examples} for a detailed analysis of the guidance network).
Moreover, unlike traditional manifold learning methods~\citep{Riemann_manifold, adaptive_manifold}, which project data into lower-dimensional spaces, our approach operates directly within the original data space.
In our ablation studies, we thoroughly examine the impact of loss terms on the performance and present the findings.

