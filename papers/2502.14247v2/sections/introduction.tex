\section{Introduction}

Automated generation of high-quality digital 3D assets has drawn more and more attention in recent years. Digital 3D assets have become deeply ingrained in modern life and production. These assets vividly express the imaginations of creators across various fields, including gaming and film, bringing joy and creating immersive experiences for both players and audiences alike. Meanwhile, 3D assets also serve as essential building blocks in the domains of physical simulation and embodied AI, enabling machines and robots to understand the elements in the real world. However, the creation of 3D assets is far from simple; it is often a complex, time-consuming, and expensive process. Taking text prompts or an image as input, the digital 3D asset production pipeline commonly involves stages of 3D shape generation and texture generation, each requiring a high level of expertise and proficiency in digital content creation software.

In this report, we present Pandora3D, a framework designed for high-quality 3D shape and texture generation. The framework consists of two main components: 3D shape generation and texture generation.

\begin{itemize}
    \item 3D Shape Generation: The 3D shape generation pipeline utilizes a Variational Autoencoder (VAE) to encode implicit 3D geometries into a latent space. A diffusion network is then used to generate latents conditioned on input prompts, with modifications aimed at enhancing the model's capacity. We also explore an alternative Artist-Created Mesh (AM) generation approach, which shows promising results for simpler geometries.

    \item Texture Generation: The texture generation process is multi-staged, starting with the generation of frontal images, followed by multi-view images generation, RGB-to-PBR texture conversion, and high-resolution multi-view texture refinement. A novel consistency scheduler is integrated into every stage of this process to ensure pixel-wise consistency among multi-view textures during inference, leading to seamless integration.
\end{itemize}

The pipeline demonstrates effective handling of diverse input formats, leverages advanced neural architectures, and incorporates novel methodologies to produce high-quality 3D content. This report details the system architecture, experimental results, and potential future directions to improve and expand the framework.