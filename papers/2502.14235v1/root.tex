%%%%%%%%%%%%%%%%%%%%%%%%%%%%%%%%%%%%%%%%%%%%%%%%%%%%%%%%%%%%%%%%%%%%%%%%%%%%%%%%
%2345678901234567890123456789012345678901234567890123456789012345678901234567890
%        1         2         3         4         5         6         7         8

\documentclass[letterpaper, 10 pt, conference]{ieeeconf}  % Comment this line out if you need a4paper

%\documentclass[a4paper, 10pt, conference]{ieeeconf}      % Use this line for a4 paper

\IEEEoverridecommandlockouts                              % This command is only needed if 
                                                          % you want to use the \thanks command

\overrideIEEEmargins                                      % Needed to meet printer requirements.

%In case you encounter the following error:
%Error 1010 The PDF file may be corrupt (unable to open PDF file) OR
%Error 1000 An error occurred while parsing a contents stream. Unable to analyze the PDF file.
%This is a known problem with pdfLaTeX conversion filter. The file cannot be opened with acrobat reader
%Please use one of the alternatives below to circumvent this error by uncommenting one or the other
%\pdfobjcompresslevel=0
\pdfminorversion=4

% See the \addtolength command later in the file to balance the column lengths
% on the last page of the document

% The following packages can be found on http:\\www.ctan.org
%\usepackage{graphics} % for pdf, bitmapped graphics files
%\usepackage{epsfig} % for postscript graphics files
%\usepackage{mathptmx} % assumes new font selection scheme installed
%\usepackage{times} % assumes new font selection scheme installed
%\usepackage{amsmath} % assumes amsmath package installed
%\usepackage{amssymb}  % assumes amsmath package installed

\title{\LARGE \bf
OG-Gaussian: Occupancy Based Street Gaussians for Autonomous Driving
}

\author{Yedong Shen$^{1}$, Xinran Zhang$^{1}$, Yifan Duan$^{1}$, Shiqi Zhang$^{2}$, Heng Li$^{1}$, Yilong Wu$^{1}$\\ Jianmin Ji$^{1}$ and Yanyong Zhang$^{2}$*~\IEEEmembership{Fellow,~IEEE}
\thanks{*The corresponding author.}% <-this % stops a space
\thanks{$^{1}$~School of Computer Science and Technology, University of Science and Technology of China, Hefei 230026, China {\tt\small \{sydong2002, zxrr, dyf0202, li\_heng, elonwu\}@mail.ustc.edu.cn, jianmin@ustc.edu.cn}}
%         University of Twente, 7500 AE Enschede, The Netherlands
%         {\tt\small albert.author@papercept.net}}%
\thanks{$^{2}$~School of Artificial  Intelligence and Data Science, University of Science and Technology of China, Hefei 230026, China {\tt\small zhangshiqi\_1127@mail.ustc.edu.cn, yanyongz@ustc.edu.cn} }
%         Dayton, OH 45435, USA
%         {\tt\small b.d.researcher@ieee.org}}%
% }
}% <-this % stops a space


\usepackage{amsmath}
\usepackage{amssymb}
\usepackage[utf8]{inputenc}

\usepackage{booktabs}  % 加载 booktabs 宏包
\usepackage{graphicx}
\usepackage{multirow}
\usepackage{xcolor}
% \usepackage{hyperref}
\makeatletter
\let\NAT@parse\undefined
\makeatother
\usepackage[colorlinks,
            urlcolor=blue,
            linkcolor=blue,       %%修改此处为你想要的颜色
            anchorcolor=blue,  %%修改此处为你想要的颜色
            citecolor=green        %%修改此处为你想要的颜色,例如修改blue为red
            ]{hyperref}
\begin{document}

\maketitle
\thispagestyle{empty}
\pagestyle{empty}

%%%%%%%%%%%%%%%%%%%%%%%%%%%%%%%%%%%%%%%%%%%%%%%%%%%%%%%%%%%%%%%%%%%%%%%%%%%%%%%%
\begin{abstract}
Accurate and realistic 3D scene reconstruction enables the lifelike creation of autonomous driving simulation environments. With advancements in 3D Gaussian Splatting (3DGS), previous studies have applied it to reconstruct complex dynamic driving scenes. These methods typically require expensive LiDAR sensors and pre-annotated datasets of dynamic objects. To address these challenges, we propose OG-Gaussian, a novel approach that replaces LiDAR point clouds with Occupancy Grids (OGs) generated from surround-view camera images using Occupancy Prediction Network (ONet). Our method leverages the semantic information in OGs to separate dynamic vehicles from static street background, converting these grids into two distinct sets of initial point clouds for reconstructing both static and dynamic objects. Additionally, we estimate the trajectories and poses of dynamic objects through a learning-based approach, eliminating the need for complex manual annotations. Experiments on Waymo Open dataset demonstrate that OG-Gaussian is on par with the current state-of-the-art in terms of reconstruction quality and rendering speed, achieving an average PSNR of 35.13 and a rendering speed of 143 FPS, while significantly reducing computational costs and economic overhead.
% 最后一句介绍总体结果
\end{abstract}


%%%%%%%%%%%%%%%%%%%%%%%%%%%%%%%%%%%%%%%%%%%%%%%%%%%%%%%%%%%%%%%%%%%%%%%%%%%%%%%%
\section{INTRODUCTION}
Reconstructing realistic, geometrically accurate 3D scenes has long been a key objective in computer vision. Thanks to advancements in technologies like Neural Radiance Fields (NeRF)~\cite{mildenhall2021NeRF} and 3D Gaussian Splatting (3DGS)~\cite{kerbl20233d}, generating high-precision 3D models is now more attainable. These technologies greatly enhance the realism of virtual environments and have significant applications in various fields, such as medical imaging~\cite{wang2021birds}, surgical navigation~\cite{liu2024endogaussian} and virtual reality~\cite{jiang2024vr, abdal2024gaussian}. In autonomous driving, the reconstruction technologies can provide precise 3D models of the surroundings, including street, building, and even dynamic object. This capability improves the autonomous system's navigation ability and enables simulation of extreme scenarios, expanding the boundaries of reality while digitizing it. 

% 第一段不用自动驾驶, 不用介绍NeRF, 直接和3dgs在一起介绍
% 背景(1)- 现在技术(3dgs现在在研究什么 有问题 引入:在自动驾驶场景中, 有哪些工作, 有哪些问题(要解决的, 得有数据支持))- 自己方法, 转到自动驾驶, 给出数据 

% \textcolor{orange}{(The labels for figures and tables need to be properly filled out, and each object must be referenced in the main text. The label here can be "fig:intro")}} % 图片标题

\begin{figure}[h]
    \centering
    \includegraphics[width=\columnwidth]{jieshao.pdf} % 插入图片, 宽度为单列宽度    
    \caption{\textbf{OG-Gaussian reconstruction example. }(a) and (b) visualize the performance of our method in reconstructing distant dynamic objects and the scene under rainy and low-light conditions. (c) and (d) present the reconstruction results of original 3DGS from the same viewpoint. Red bounding boxes indicate the location of dynamic objects in the ground truth.}
    \label{fig:intro} 
    \vspace{-20pt}
\end{figure}

% Noticing these important applications, many researchers have begun to focus on this area of work. With the significant progress NeRF technology has made in indoor and single-object reconstruction, researchers have started applying it to outdoor scene reconstruction. \textcolor{blue}{The sentence above is not necessary. I suggest deleting it. Too naive.}To address the challenges \textcolor{blue}{Xinran: what challeng0es? We should tell the story clearly, then the challenge, and then the method. So, this paragraph can be removed.}in autonomous driving, where there are numerous dynamic objects with varying movement trajectories, subsequent researchers turned their attention to individually modeling multiple dynamic objects. Nevertheless, due to the large number of parameters in NeRF networks, the training costs are high, and the rendering speed does not meet the requirements. \textcolor{blue}{I think we do not need to mention the disadvantage of NeRF here}

To achieve the high-precision reconstruction of autonomous driving scenes, NeRF is used as the foundational technology, representing the scene as a continuous 3D volume through neural network~\cite{xie2023s, wu2023mars, yang2023emernerf}. While this method generates high-quality outdoor scenes, it comes with the cost of requiring extensive training resources and slower rendering speed. With the emergence of 3DGS, the low-cost, fast-rendering 3D scene reconstruction method quickly captured widespread attention. The native 3DGS is not well suited for handling large outdoor scene with dynamic objects. To adapt this technology for autonomous driving scene reconstruction, existing works of 3DGS put their attention on integrating LiDAR-generated point clouds and using annotated 3D bounding boxes to reconstruct street scenes with dynamic objects~\cite{luiten2023dynamic, zhou2024drivinggaussian, yan2024street}. They have successfully separated dynamic objects from static background, achieving promising reconstruction results under low training cost\cite{liu2025dvlo}. 

% \textcolor{orange}{Add a footnote with the citation, but do not use a soft link here. Tesla introduces Occupancy Network, a new representation in autonomous driving perception, which **explain occupancy network or characteristics**  } 

% \textcolor{orange}{Xinran: The last sentence is a little simple, say nothing. } \textcolor{orange}{Xinran: Tesla proposes Occ network, and applies it to autonomous driving, what about your challenges? Something difficult to solve? Applying it to 3DGS is not a challenge, just a project. It seems we miss a challenge here. }

% Tesla introduces Occupancy Network (ONet)\footnote{https://www.youtube.com/@tesla}, a new representation in autonomous driving perception which is based on previous work~\cite{mescheder2019occupancy}, it models the real world as voxel grids with semantic information. 

% 五张图片, 得介绍一下occ, 雷达点贵, 所以occ应运而生
However, these techniques need expensive LiDAR to generate point clouds and datasets with pre-labeled dynamic vehicle boundaries and trajectories. To mitigate this constraint, we bring Occupancy Prediction Network(ONet)~\cite{mescheder2019occupancy}, a new approach in autonomous driving perception into the field of 3D scene reconstruction. Since ONet models the real world as voxel grids with semantic information directly, we can eliminate the need for costly LiDAR and resolve the problem of bounding boxes failing to capture unannotated objects\cite{liu2024difflow3d}. 

Based on the considerations, we propose OG-Gaussian, a new solution for autonomous driving scene reconstruction. Our method starts by capturing surround-view images using cameras mounted on the vehicle. Then, we use a Occupancy Prediction Network (ONet) to obtain Occupancy Grid (OG) information about the surrounding world. By leveraging the semantic information of the Occupancy Grid, we separate the original scene into street scene and dynamic vehicles. 

Afterward, we convert the Occupancy Grids of background street into point clouds and transform dynamic vehicle grids into an initial set of point clouds via 2D image projection. Instead of relying on expensive LiDAR point clouds as initial point clouds (prior), the point clouds obtained from Occupancy Grids can serve as low-cost alternative prior in our approach. These point clouds will be converted into optimizable sets of Gaussian ellipsoids. To track dynamic vehicles, we define their initial points' positions and rotation matrices as learnable parameters. This allows us to optimize the vehicles' poses and trajectories, which describe the operation of dynamic vehicles in the real world. In this way, our method does not require pre-labeled trajectories or bounding boxes of dynamic objects. The final Optimized Gaussian ellipsoids will be projected into 2D space to render the reconstructed autonomous driving scenes in the real world.

We conduct experiments on several representative scenes in Waymo Open dataset~\cite{sun_2020_cvpr}, which includes a wide variety of driving scenes collected from diverse urban and suburban areas. Experiments demonstrate that our method is comparable to the current state-of-the-art methods using LiDAR in reconstruction quality and rendering speed. Additionally, we carry out ablation study to confirm the effectiveness of using processed Occupancy Grids as prior in reconstructing autonomous driving scenes. We provide fast, low-cost 3D scene reconstruction method for downstream tasks.

% 我们提出了OG-Gaussian, 将Occupancy Grid引入自动驾驶场景重建任务中, 从此不再需要昂贵的激光雷达来产生初始点云, 只需要图像的输入, 大大降低了3D场景重建的成本。
% 我们利用Occupancy Grid具有语义的特性对动静态物体进行分离, 并在训练过程中对动态物体的姿态进行估计, 不再需要像以前的方法一样进行复杂的人工标注, 获得动态物体的边界框来实现动静态物体的分离。
% 经过了综合的实验, 我们的方法在重建质量和渲染速度上与需要雷达点云输入的方法几乎相同, 并大大领先于之前的基于辐射场的方法。这项工作为下游任务提供了快速、低成本的3D重建场景。

The main contributions of this paper are as follows:
\begin{itemize}

\item We introduce OG-Gaussian, incorporating Occupancy Grid into autonomous driving scenes reconstruction. This approach eliminates the need for expensive LiDAR to generate initial point clouds, requiring only image input and significantly reducing the cost of 3D scene reconstruction.
\item We leverage the semantic properties of Occupancy Grids to separate dynamic vehicles from static background and estimate their poses, removing the need for manual labeling of dynamic objects.
\item Through extensive experiments, our method achieves an average PSNR of 35.13 and 143 FPS, without relying on LiDAR or any annotations. It's comparable to state-of-the-art methods in terms of reconstruction quality and rendering speed. 

\end{itemize}

\section{RELATED WORKS}

\noindent\textbf{Occupancy Network and applications. }Obtaining accurate semantic 3D Occupancy Grids is crucial for downstream tasks, making research on the Occupancy Prediction Network highly valuable\cite{duan2024cellmap}. MonoScene~\cite{cao2022monoscene} completes dense 3D grids from a single RGB image, it ensures spatial consistency via 2D-3D projection and 3D context prior. By extending BEV representation to 3D, TPVFormer~\cite{huang2023tri} proposes a new representation named TPV and combines it with a transformer to predict 3D semantic Occupancy Grids. SurroundOcc~\cite{wei2023surroundocc} utilizes an innovative method to generate dense occupancy labels and employed a 2D-3D attention mechanism to predict Occupancy Grids from multi-camera images. To facilitate future research, Occ3D~\cite{tian2024occ3d} introduces two benchmark datasets (Occ3D-Waymo and Occ3D-nuScenes) and a new model (CTF-Occ network). As the accuracy of occupancy prediction networks improves, many related applications are also emerging. OccWorld~\cite{zheng2023occworld} introduces a world model framework based on 3D occupancy space. It demonstrates an effective capability in modeling scene evolution on the nuScenes benchmark. OCC-VO~\cite{li2024occ} transforms 2D camera images into 3D semantic occupancy to address the depth information challenge in Visual Odometry(VO). 

\noindent\textbf{3D scenes reconstruction for Autonomous Driving. } 
Simulating real-world street scene is essential for developing and testing autonomous driving systems. A prime example is CARLA~\cite{dosovitskiy17}, a well-known open-source driving simulator that's widely used for creating complex 3D environments. But the scenes created by these simulators often lack the realism needed to fully immerse users in real-world environments. 
% Consequently, NeRF and 3DGS have gained attention for reconstructing detailed 3D scenes from 2D images. 
To adapt NeRF for unbounded and dynamic field like autonomous driving, ~\cite{martin2021NeRF, turki2022mega} improves NeRF to model multi-scale urban scenarios. ~\cite{xu2023grid} combines compact multi-resolution ground features with NeRF to achieve high-fidelity rendering. Research in ~\cite{ost2021neural, song2022towards} begin to explore handling dynamic scenes with multiple objects, leading to Suds~\cite{turki2023suds} processing scenes into static backgrounds and dynamic objects. While S-NeRF~\cite{xie2023s} introduces LiDAR as a form of supervision, MARS and EmerNeRF~\cite{wu2023mars, yang2023emernerf} further optimize NeRF for outdoor scene reconstruction to enhance its performance. However, NeRF is limited by its high training costs and slow rendering speed, so attention of industry is gradually shifting toward 3DGS due to its lower training costs and faster rendering speed~\cite{kerbl20233d}. \cite{yang2024deformable} uses a transformer to model Gaussian motion from monocular images. ~\cite{luiten2023dynamic} parametrizes the entire scene using a set of variable dynamic Gaussians. DrivingGaussian~\cite{zhou2024drivinggaussian} firstly introduces LiDAR point clouds as a prior and incrementally reconstructs the entire scene. Street Gaussians~\cite{yan2024street} reconstructs dynamic scenes with separate rendering and LiDAR. S3 Gaussian~\cite{huang2024textit} improves this using self-supervised vehicle bounding boxes. 
% Street Gaussians~\cite{yan2024street} achieves precise reconstruction of dynamic scenes by separately rendering the background model and dynamic objects, they also use LiDAR. Building on~\cite{yan2024street}, S3 Gaussian~\cite{huang2024textit} effectively reconstructs outdoor scenes by leveraging self-supervised vehicle bounding boxes.  

% , each of which contains 59 parameters that adequately represent the state of each Gaussian. Of these 59, 3 parameters are used to represent the center position of the Gaussian, 7 parameters are used to represent the covariance matrix, 1 parameter is used to represent the opacity, and 48 parameters are used to represent the spherical harmonic coefficients.~\textcolor{orange}{Xinran: If you want to introduce the 59 parameters, give the formula of 3DGS, which including these parameters.}
\begin{figure*}[h]
\centering
\includegraphics[width=0.95\textwidth]{jiagou_compressed.pdf}
\vspace{-3pt}
\caption{\textbf{Overview of OG-Gaussian. }OG-Gaussian utilizes a trained 3D Occupancy Prediction network to obtain Occupancy Grid data for the scene. It separates static and dynamic objects into different initial point cloud models using semantic information. After the separated reconstruction, we globally render both static and dynamic objects, producing 3D scenes, depth maps and so on.}
\label{fig:overview}
% Fig.~\ref{fig:overview}
\vspace{-14pt}
\end{figure*}

\section{PRELIMINARY}

% Before introducing the core components of our OG-Gaussian, 
Firstly, we will introduce the classical 3DGS algorithm and 3D Occupancy Grid in the preliminary section.

\noindent\textbf{Classical 3DGS algorithm. }In the first step of reconstruction, we initialize the 3DGS point clouds by placing a Gaussian at each point, other parameters randomly initialized except the center point position. COLMAP~\cite{schoenberger2016sfm} is used to generate the Structure from Motion (SfM) point clouds as the prior. After initialization, each Gaussian ellipsoid can be represented by Eq.~\ref{1}:
\begin{equation}
    \begin{aligned}
     G(x)&=e^{-\frac{1}{2}(x)^{T} \Sigma^{-1}(x)} \\
     \Sigma&=R S S^{T} R^{T}
    \end{aligned}
    \label{1}
\end{equation}

$G(x)$ is the probability distribution function of the Gaussian, where $x$ is a three-dimensional vector representing a point in space. $\Sigma$ is the covariance matrix that describes the shape, orientation, and size of the Gaussian in 3D space. This can be written in $R S S^{T} R^{T}$, so the covariance matrix can be constructed by rotation and scaling matrix (R and S).

After representing the 3D Gaussians, we can rasterize the rendering by $ \alpha 
 $ blending\cite{wu2024mm}. From near to far, each ellipsoid projects onto the image plane, and the overlapping areas can be fused through rasterization to generate a new image. Eq.~\ref{2} is the loss function used in 3D Gaussian, where $\mathcal{L}_{1}$ is used to measure the pixels’ difference and $\mathcal{L}_{\text {D-SSIM }}$ is used to measure the structural similarity between two images.
 
 \begin{equation}
      \mathcal{L}=(1-\lambda) \mathcal{L}_{1}+\lambda \mathcal{L}_{\text {D-SSIM }}
      \label{2}
 \end{equation}

The properties of the 3D Gaussian sphere can be updated through gradient back propagation, the cloning and splitting of the Gaussian spheres can also be controlled. The gradient back propagation is divided into two branches, the upper branch updates the properties of the 3D Gaussian ellipsoid, and the lower branch realizes the cloning and splitting of the 3D Gaussian ellipsoid. 
% The main specifics of cloning and splitting can be expressed as follows: 
% \begin{itemize}

% \item The under-reconstructed region has a Gaussian ellipsoid with small variance, it's hard to describe the shape of this geometry with a single Gaussian sphere, so the Gaussian sphere will be cloned.
% \item The Gaussian ellipsoid in the over-reconstructed region has a large variance and the fit covers all the shapes, but there are too many shapes that don't belong to this geometry, so it has to be split.
% \item After a fixed number of iterations, a culling operation is performed, removing almost transparent Gaussian ellipsoids and Gaussian ellipsoids with too much variance.
% \end{itemize}

\noindent\textbf{3D Occupancy Grid.} Occupancy Prediction Network~\cite{mescheder2019occupancy} generates 3D grids $V\in\mathbb{R}^{H\times W\times D}$, where $H, W, D$ represent the height, width, and depth of the grid. Each element $V(i, j, k)$ in the grid corresponds to the grid cell at position $(i, j, k)$. A threshold $\tau$ is applied to classify whether each grid is occupied or not: 

\begin{equation}    
  V(i, j, k)=\begin{cases}
  1&\mathrm{if~}p(i, j, k)\geq\tau\\
  0&\mathrm{if~}p(i, j, k)<\tau
  \end{cases}
  \label{3}
\end{equation}

Assigning $N$ semantic categories to the grid (such as buildings, roads, and vehicles), the state of grid can be represented by $V(i, j, k)$ and the semantic category $C(i, j, k)$. $C(i, j, k)$ can be acquired by choosing the category $c$ with the highest probability using Eq.~\ref{4}: 

\begin{equation}
    C(i, j, k)=\arg\max_c[p_1(i, j, k)\ldots p_N(i, j, k)]
    \label{4} 
\end{equation}

% ~\textcolor{orange}{Xinran: In my overleaf, there are some errors in the two above equations when compiling, check them. }
In this way, we can predict the occupancy state of each grid and get detailed information about its semantic category.

\section{METHOD} \label{sec:method}
% In preliminary, we introduce 3DGS which performs well in purely static scenes. In the following section, 
Next, we will provide a detailed explanation of the OG-Gaussian method, as shown in Fig.~\ref{fig:overview}

% Previous reconstruction methods have high training cost and slow rendering speed, 3DGS method is difficult to represent dynamic scenes directly. Improved dynamic object reconstruction methods based on 3DGS usually rely on expensive LiDAR point clouds as a prior. To solve these problems, we propose a novel scene representation named OG-Gaussian, which is built upon 3DGS and Occupancy Grid.

\subsection{OG-Gaussian}
In this section we focus on the basic structure of the OG-Gaussian and introduce how we represent street scene and dynamic vehicles with two different sets of  point clouds. We will provide a detailed explanation of them below.

\noindent\textbf{Street model. }The initial point clouds of the Street model is a set of points in the world coordinate system. According to the information in the preliminary, the parameters of 3D Gaussians can be expressed as the covariance matrix $\Sigma_{s}$ and the position vector $\mu_{s} \in \mathbb{R}^3$. The matrix $\Sigma_{s}$ can be split into the rotation matrix $R_s$ and the scaling matrix $S_s$, this recovery process is:

\begin{equation}
    \Sigma_{s} = R_sS_sS_s^TR_s^T
    \label{5}
\end{equation}


\begin{figure}[h]
    % \vspace{-12pt}
    \centering
    \includegraphics[width=0.85\columnwidth]{dongtai_compressed.pdf} % 插入图片, 宽度为单列宽度
    \vspace{-10pt}
    \caption{\textbf{Initial point cloud generation process. }We extract dynamic vehicles from the street scene, then upsample and project them to obtain dense, colorized dynamic vehicle point cloud prior. The street scene can be converted into an initial background point clouds directly.} % 图片标题
    \label{fig:dynamic} % 图片标签, 方便引用
    \vspace{-19pt}
\end{figure}

In addition to the covariance and position matrices, each Gaussian contains a parameter $\alpha$ to represent opacity and a set of spherical harmonics coefficients (Eq.~\ref{6}) to represent the appearance of scenes. $l, m$ in Eq.~\ref{6} are referred to the degree and order that define the specific spherical harmonic function. In order to obtain color information of the original view, we also need to multiply the spherical harmonics coefficients with the spherical harmonics basis functions projected from the view direction. To get the semantic information for each Gaussian, we add logit $\beta_{s} \in \mathbb{R}^N$ to each point, where $N$ stands for the total number of semantic categories.

\begin{equation}
    \mathbf{z}_{s} = (z_{m, l})_{l:0\leq\ell\leq\ell_{max}}^{m:-\ell\leq m\leq\ell}
    \label{6}
\end{equation}

\noindent\textbf{Dynamic Vehicle model. }Autonomous driving scene contains multiple moving vehicles, and we also need to represent them with a set of optimizable point clouds. Observing dynamic vehicles from limited views, their movements result in significant changes in the surrounding space, so it's difficult to reconstruct them directly using 3DGS. We use the well-established detection and segmentation model~\cite{wu2019detectron2, cheng2021mask2former} to extract dynamic vehicle objects at the pixel level, and extract initial dynamic points cloud based on the semantic information of Occupancy grids, noting that the position of the point cloud is under the vehicle coordinate system.

The Gaussian properties of dynamic vehicles and streets are similar and they have the same meaning for the opacity $\alpha_{d}$ and the scale matrix $\mathbf{S}_d$. However, as we have mentioned above, their positions and rotations are under the vehicle coordinate system, which is different from the street scene. To avoid using the ground truth pose value, we represent the actual state of the dynamic vehicle by tracking its pose. The vehicle's pose can be represented by a set of rotation matrices $\mathbf{R}_t$ and translation vectors $\mathbf{T}_t$, as:

\begin{equation}
    \begin{aligned}
    &\boldsymbol{\mu}_w=\mathbf{R}_t\boldsymbol{\mu}_o+\mathbf{T}_t\\&\mathbf{R}_w=\mathbf{R}_o\mathbf{R}_t^T  
    \end{aligned}
    \label{7}
\end{equation}
Where $\boldsymbol{\mu}_w$ and $\mathbf{R}_w$ are the position matrix and rotation matrix of each Gaussian in the world coordinate system, $\boldsymbol{\mu}_o$ and $\mathbf{R}_o$ are the object position and rotation with respect to the car. From the prior knowledge we can get the covariance matrix of the dynamic vehicle by $\mathbf{R}_w$ and $\mathbf{S}_d$. In order to get more accurate vehicle pose, we replace the rotation matrix and position matrix of each frame as parameters like Eq.~\ref{8}, then we use them to get vehicle position and trajectory without ground truth of the vehicle trajectory.

\begin{equation}
\begin{aligned}
&\mathbf{R}_t^{\prime}=\mathbf{R}_t\Delta\mathbf{R}_t\\&\mathbf{T}_t^{\prime}=\mathbf{T}_t+\Delta\mathbf{T}_t
\end{aligned}
\label{8}
\end{equation}

The semantic representation of the dynamic vehicle model is also different from the street model, semantics in the street model is an N-dimensional vector (N is the number of semantic categories), the semantics of the vehicle model has only two categories for the vehicle semantic class (from the Occupancy prediction results) and non-vehicle class, so it's a one-dimensional scalar.

In the street model, we use spherical harmonics coefficients to represent the appearance of the scene. But when dealing with dynamic vehicles, their positions change with time. Therefore, it's wasteful to use multiple consecutive spherical harmonics coefficients to represent dynamic objects at each timestamp. Instead, we replace each SH coefficient $z_{m, l}$ with a set of Fourier transform coefficients $f\in\mathbb{R}^k$, constructing the 4D spherical harmonics coefficients (including the time dimension) in such a way that the real-valued inverse discrete Fourier transform can be performed to recover $z_{m, l}$ at a given time step.

\subsection{Occupancy Prior with Surrounding Views}

The original 3DGS generates sparse point clouds as prior with the help of structure-from-motion (SfM). In the task of reconstructing unbounded scenes of streets, it would be difficult to accurately represent dynamic objects and complex street scenes using SfM point clouds directly, this approach would produce some obvious geometric errors and incomplete recoveries. In order to provide accurate initialized point clouds for 3DGS, we convert the results predicted by ONet into an initialized point cloud to obtain accurate geometric information and maintain multi-camera consistency in the surround view alignment.

Specifically, we extract the vehicle point clouds based on the semantic information of OGs, and we define the vehicle location information for each timestamp as $\boldsymbol{\mu}_t$, if $\boldsymbol{\mu}_{t+1}-\boldsymbol{\mu}_t\geq\boldsymbol{\mu}_{th}$, we can label this vehicle as dynamic, where $\boldsymbol{\mu}_{th}$ denotes the threshold of the positional offset that determines it to be a dynamic object.

In order to generate densified point clouds to represent the dynamic vehicles, we upsample the point clouds of the dynamic objects with a voxel size of 0.05m. We project these points to the corresponding image planes and assign colors to them by querying pixel values. For each initial point $o$ of the dynamic vehicle, we transform its coordinates relative to the camera coordinate system, followed by the projection step described in Eq.~\ref{9}, where $x_p^q$ is the 2D pixel of the image, and $K \in \mathbb{R}^{3*3}$ is the internal reference matrix of each camera. $R_t$ and $T_t$ represent the orthogonal rotation matrix and translation vector.

\begin{equation}
    x_p^q=K[R_t\cdot o_d+T_t]
    \label{9}
\end{equation}

% \begin{table}[b]
%     \centering \vspace{-2em}
%     \caption{\textnormal{\textbf{Overview of the scene. }}}
%     \vspace{-10pt}
%     \begin{tabular}{ccccc}
%     \toprule
%          Scene & Weather & Lighting & Area & Traffic flow \\
%     \midrule
%          00 & Rainy & dark & urban & low\\
%          01 & Cloudy & ordinary & suburbia & high \\
%          02 & Sunny & bright & downtown & middle\\        
%     \bottomrule
%     \end{tabular}
%     % \vspace{-5pt}
%     \label{tab:scene}
% \end{table}

Finally, we convert the remaining Occupancy Grids into dense point clouds with position taken from their center coordinates. The specific process for generating initial point clouds of static and dynamic objects can be seen in Fig. ~\ref{fig:dynamic}. In addition to this, we also aggregate the dense point clouds with the point clouds generated by COLMAP in order to deal with distant buildings.

\subsection{Global Rendering via Gaussian Splatting}

To render the entire OG-Gaussian, we aggregate the contribution of each Gaussian to produce the final image. Previous methods represent scenes using neural fields, which require accounting for factors like lighting complexity when composing the scene. Our OG-Gaussian rendering approach is based on 3DGS, enabling high-fidelity rendering of autonomous driving scenes by projecting the Gaussians of all point clouds into 2D image space.

Given a rendering timestamp \(t\), we first calculate the spherical harmonics coefficients using Eq.~\ref{6}. After transforming point clouds from the vehicle coordinate system to the world coordinate system, we combine the street model and the dynamic model into a global model. Using the camera's extrinsic parameters \(W\) and intrinsic parameters \(K\), we project point clouds onto a 2D plane and calculate each point's parameters in the 2D space. In the Eq.~\ref{10}, \(J\) is the Jacobian matrix of \(K\), while \(\mu'\) and \(\Sigma'\) represent the position and covariance matrix in the 2D image space.

\begin{equation}    
\begin{aligned}
\mu^{\prime}&=\mathbf{KW}\boldsymbol{\mu}\\\Sigma^{\prime}&=\mathbf{JW}\Sigma\mathbf{W}^T\mathbf{J}^T
\end{aligned}
\label{10}
\end{equation}

After this, we can calculate the appearance color \(a\) of each pixel based on the opacity of the points. In Eq.~\ref{11}, \(\alpha_i\) is the product of opacity \(\alpha\) and the probability of the 2D Gaussian, while \(a_i\) is the color derived from the spherical harmonic \(z\) with a specific view direction.

\begin{equation}
\begin{aligned}
\mathbf{a}=\sum_{i\in N}\mathbf{a}_i\alpha_i\prod_{j=1}^{i-1}(1-\alpha_j) 
\end{aligned}
\label{11}
\end{equation}



\section{EXPERIMENT}

\begin{table}[b]
    \centering \vspace{-2em}
    \caption{\textnormal{\textbf{Overview of the scene. }}}
    \vspace{-10pt}
    \begin{tabular}{ccccc}
    \toprule
         Scene & Weather & Lighting & Area & Traffic flow \\
    \midrule
         00 & Rainy & dark & urban & low\\
         01 & Cloudy & ordinary & suburbia & high \\
         02 & Sunny & bright & downtown & middle\\        
    \bottomrule
    \end{tabular}
    % \vspace{-5pt}
    \label{tab:scene}
\end{table}

\begin{table*}[t]
\vspace{+6pt}
\centering
\caption{\textnormal{\textbf{Comparison of overall performance between OG-Gaussian and baselines on the Waymo dataset} Here, PSNR-dy refers to the PSNR value specifically for rendering dynamic vehicles. The result with bold black font is the best result for this scene.}}
\vspace{-10pt}
\begin{tabular}{lcccccccc}
\toprule
Methods & Inputs & Modal & Scene & PSNR[dB]~$\uparrow$ & PSNR-dym[dB]~$\uparrow$ & SSIM~$\uparrow$ & LPIPS~$\downarrow$ & FPS~$\uparrow$ \\  
\midrule
\multirow{3}{*}{MARS~\cite{wu2023mars}} & \multirow{3}{*}{Images} & \multirow{3}{*}{Camera} & 00 & 31.07 & 21.24 & 0.901 & 0.243 & 0.67\\
 &  & & 01 & 30.11 & 20.96 & 0.906 & 0.251 & 0.62\\
 &  & & 02 & 31.54 & 21.63 & 0.917 & 0.239 & 0.73\\
 \hline
\multirow{3}{*}{EmerNeRF~\cite{yang2023emernerf}}  & \multirow{3}{*}{Images+LiDAR points} & \multirow{3}{*}{Camera+LiDAR} & 00 & 33.24 & 23.21 & 0.911 & 0.221 & 1.94\\
 &  & & 01 & 31.18 & 22.86 & 0.904 & 0.226 & 1.62\\
 &  & & 02 & 33.07 & 23.09 & 0.916 & 0.241 & 1.96\\
\hline
\multirow{3}{*}{3DGS~\cite{kerbl20233d}} & \multirow{3}{*}{Images+SfM points} & \multirow{3}{*}{Camera}& 00 & 30.21 & / & 0.916 & 0.181 & \textbf{209}\\
 &  & & 01 & 29.16 & / & 0.907 & 0.192 & \textbf{204}\\
 &  & & 02 & 32.46 & / & 0.923 & 0.179 & \textbf{218}\\
\hline
\multirow{3}{*}{Street Gaussian~\cite{yan2024street}} & \multirow{3}{*}{Images+LiDAR points} & \multirow{3}{*}{Camera+LiDAR} & 00 & \textbf{35.92} & 24.96 & 0.961 & 0.096 & 133\\
 &  & & 01 & 32.91 & \textbf{24.53} & 0.943 & \textbf{0.094} & 129\\
 &  & & 02 & 36.14 & 26.14 & 0.959 & 0.091 & 137\\
\hline
\multirow{3}{*}{Ours} & \multirow{3}{*}{Images} & \multirow{3}{*}{Camera} & 00 & 35.87 & \textbf{25.17} & \textbf{0.965} & \textbf{0.093} & 143\\
 &  & & 01 & \textbf{33.24} & 24.28 & \textbf{0.958} & 0.097 & 139\\
 &  & & 02 & \textbf{36.28} & \textbf{26.33} & \textbf{0.962} & \textbf{0.089} & 148\\
\bottomrule
\end{tabular}
\vspace{-5pt}
\label{tab:overall}
\end{table*}

\subsection{Experiment Setups}
We perform 30, 000 iterations on OG-Gaussian, using the same learning rate settings as the original 3DGS. The learning rates for the rotational transformation $\Delta R_t$ and translational transformation $\Delta T_t$ are set to 0.001 and 0.005. The Adam optimizer is used for optimization. All experiments are conducted on a machine equipped with an Nvidia A100 GPU.

% This surround-view capability is essential for accurate perception, obstacle detection, and safe navigation, making it a valuable resource for training models in complex driving scenarios.

\noindent \textbf{Dataset. } The Waymo Open Dataset~\cite{sun_2020_cvpr} is widely used in autonomous driving research due to its high-quality 360-degree panoramic imagery, which offers comprehensive coverage of the vehicle's surrounding\cite{10.1145/3636534.3649386}. The dataset's quality, diversity, and scale make it a valuable asset for researchers and developers in autonomous driving. It has been widely adopted due to its realistic and challenging dataset, which enables the development and testing of algorithms in conditions that closely resemble real-world driving. 
% Our OG-Gaussian also relies on occupancy prediction results provided by the Occ3D framework. Occ3D is designed for 3D occupancy prediction and includes a labeling process that generates dense, visibility-aware labels for any given scene. This process involves three stages: voxel densification, occlusion reasoning, and image-guided voxel refinement. It offers rich scene data and high-resolution 3D voxel occupancy representations for the Waymo Open Dataset. The Occ3D-Waymo dataset consists of 798 training scenes and 202 validation scenes, totaling 200, 000 frames. It covers a wide range of outdoor scenarios with high diversity, and we utilize its high-fidelity 3D voxel occupancy representations for our research.

\begin{figure*}[!h]
\centering
\includegraphics[width=0.94\textwidth]{shiyan.pdf} \vspace{-1em}
\caption{\textbf{Qualitative comparison of different reconstruction methods.} Each column in the figure represents the qualitative results of different reconstruction methods, with each row showing the same viewpoint.}
\label{fig:experiment} 
\vspace{-15pt}
\end{figure*}

\noindent\textbf{Baselines. }For selecting baselines, we choose several representative outdoor scene reconstruction methods, including both implicit and explicit representations. We will introduce these methods in detail below.
\begin{itemize}
    \item MARS~\cite{wu2023mars} is an autonomous driving simulator using NeRF to independently model dynamic objects and static backgrounds.
    \item EmerNeRF~\cite{yang2023emernerf} enhances dynamic scene reconstruction through self-supervised separation of scene elements and a flow field that aggregates multi-frame features.
    \item 3DGS~\cite{kerbl20233d} uses Gaussian ellipsoids for high-quality 3D scene representation, excelling in rendering speed.
    \item Street Gaussians~\cite{yan2024street} extends 3DGS for urban scenes, combining LiDAR point clouds and dynamic object segmentation for view synthesis at impressive speeds.
\end{itemize}
% (1)MARS is an autonomous driving simulator using Neural Radiance Fields (NeRF) to independently model dynamic objects and static backgrounds, offering flexibility in NeRF backbones, sampling methods, and input modes. (2)EmerNeRF enhances dynamic scene reconstruction through self-supervised separation of scene elements and a flow field that aggregates multi-frame features. (3)3DGS uses Gaussian ellipsoids for efficient, high-quality 3D scene representation, excelling in rendering speed. (4)Street Gaussian extends 3DGS for urban scenes, combining LiDAR point clouds and dynamic object segmentation for high-fidelity view synthesis at impressive speeds.~\textcolor{orange}{Xinran: Use item or something, it is not clear use (1)(2)...}

% \begin{itemize}

% \item MARS is an autonomous driving simulator based on Neural Radiance Fields (NeRF). It uses separate networks to model foreground instances and background environments independently, allowing for implicit control over dynamic objects and static scenes. Additionally, MARS enables seamless switching between different NeRF backbone models, sampling methods, and input modes. 

% \item EmerNeRF is a method for spatiotemporal representation of dynamic driving scenes based on NeRF. It uses self-supervision to separate dynamic and static elements in a scene. By parameterizing an induced flow field from the dynamic parts and leveraging this flow field to aggregate multi-frame features, EmerNeRF enhances the reconstruction accuracy of dynamic objects.

% \item 3DGS is an advanced 3D scene representation and rendering technique that models scenes using a set of Gaussian ellipsoids. It achieves high-quality new view synthesis through efficient rasterization techniques. This method offers significant advantages in rendering speed while also maintaining a high level of rendering quality.

% \item Street Gaussian is an autonomous driving scene reconstruction framework based on 3DGS. Leveraging radar point clouds and dynamic object segmentation tools, it achieves high-fidelity view synthesis of urban street scenes while maintaining impressive rendering speeds.

% \end{itemize}

\subsection{Comparisons with Baselines}

In terms of experimental scene settings, we choose three representative sequences from the Waymo Open Dataset~\cite{sun_2020_cvpr}. These sequences vary in weather conditions, traffic flow and street types as shown in Tab.~\ref{tab:scene}, allowing us to evaluate the reconstruction performance across different scenarios.

Tab.~\ref{tab:overall} presents a quantitative comparison of our method with the baselines in terms of rendering quality and speed. We use Peak Signal-to-Noise Ratio(PSNR), Structural Similarity Index Measure(SSIM)~\cite{wang2004image} and Learned Perceptual Image Patch Similarity(LPIPS)~\cite{zhang2018unreasonable} as metrics to assess the reconstruction quality. To specifically evaluate the reconstruction of dynamic vehicles, we render the objects identified as dynamic and use their PSNR (PSNR-dym) as the metric to measure the effect of dynamic object reconstruction. The experiment results show that the average PSNR for Mars~\cite{wu2023mars} and EmerNeRF~\cite{yang2023emernerf}, dynamic-static separation reconstruction methods based on NeRF, are 30.91 and 32.50, respectively. In comparison, our method achieves a higher average of 35.13. Regarding rendering speed, our method is two orders of magnitude faster than the aforementioned approaches. Street Gaussian~\cite{yan2024street}, which is based on LiDAR point clouds, performs similarly to ours in both rendering quality and speed. But this method requires expensive LiDAR data, while our method operates in a purely visual modality. The original 3DGS~\cite{kerbl20233d} method is faster than ours, but it does not separately reconstruct dynamic objects, which result in blurry artifacts in regions involving dynamic objects. As a consequence, it only achieved a PSNR of 30.61 and an SSIM of 0.954, since it does not separate dynamic objects, we don't calculate its PSNR-dym. We visualized each reconstruction method across three sets of scenes to evaluate their qualitative performance, as shown in Fig.~\ref{fig:experiment}.
% The specific visualization results are shown in Fig.~\ref{fig:experiment}.
% ~\textcolor{orange}{Xinran: Though you cite the references in the table, cite them again in the main body.}~\textcolor{orange}{Xinran: When writing this section, always remember to include the quantitative data from the experiments and provide numbers rather than adjectives like 'better' or 'fast.' Besides these cases, avoid criticizing others' methods unnecessarily.}

\begin{table}[b]
\centering \vspace{-15pt}
\caption{\textnormal{\textbf{Ablation study results. }The results of each experimental setup.}}
\vspace{-1em}
\begin{tabular}{lcccc}
\toprule
Initial Method & PSNR~$\uparrow$ & PSNR-dym~$\uparrow$ & SSIM~$\uparrow$ & LPIPS~$\downarrow$ \\ 
\midrule
w/o OG,SfM & 26.01   & 19.08   & 0.895   & 0.247   \\
w/o OG   & 32.62   & 19.94  & 0.923  & 0.180  \\
Ours  & \textbf{35.13} & \textbf{25.26} & \textbf{0.962} & \textbf{0.093}  \\
\bottomrule
\end{tabular}
% \vspace{-5pt}
\label{tab:ablation}
\end{table}

\subsection{Ablations and Analysis}
To validate the significance of OG-based point clouds as prior, we designed the following ablation experiment. Three experimental setups were configured, all sharing identical settings except for the point clouds prior. In the first setup, we use the point clouds combined by OGs and SfM points from COLMAP as prior. In the second setup, only the SfM point clouds are used as prior, while in the third setup, randomly generated point clouds are employed as prior.

\begin{figure}[h]
    \centering \vspace{-10pt}
    \includegraphics[width=0.9\columnwidth]{xiaorong.pdf} % 插入图片, 宽度为单列宽度
    \vspace{-0.8em}
    \caption{\textbf{Visual ablation results for specific scenes.} The results of different initialization methods.} % 图片标题
    \label{fig:ablation} % 图片标签, 方便引用
    \vspace{-2pt}
\end{figure}

Tab.~\ref{tab:ablation} shows that adding the Occupancy Grid improves the overall PSNR of the reconstructed scene (35.13) compared to using only the SFM point cloud (32.62), and PSNR-dym increases significantly. On the other hand, using randomly generated point clouds based on spatial size performs worse than the other two methods. This experiment indicate the importance of initial point clouds in 3D scene reconstruction. The qualitative comparison of different initialization methods can be seen in Fig. ~\ref{fig:ablation}. Here, each metric represents the average value across the three experimental scenarios.

\section{CONCLUSIONS}
We introduce OG-Gaussian, an efficient method that incorporates the Occupancy Grids into 3DGS for reconstructing outdoor autonomous driving scenes. Our approach leverages the prior provided by the Occupancy Grid for scene reconstruction while also separating and reconstructing dynamic vehicles from static street scenes. We achieve performance on par with LiDAR-dependent SOTA while relying solely on camera images. Our method will enable future researchers to reconstruct their autonomous driving scenes quickly and at a low cost, contributing to the advancement of autonomous driving technology.

\section*{acknowledgment} This work is supported by the National Natural Science Foundation of China (No. 62332016) and the Chinese Academy of Sciences Frontier Science Key Research Project ZDBS-LY-JSC001.
% 最后展望一下
% \addtolength{\textheight}{-12cm}   % This command serves to balance the column lengths
                                  % on the last page of the document manually. It shortens
                                  % the textheight of the last page by a suitable amount.
                                  % This command does not take effect until the next page
                                  % so it should come on the page before the last. Make
                                  % sure that you do not shorten the textheight too much.

%%%%%%%%%%%%%%%%%%%%%%%%%%%%%%%%%%%%%%%%%%%%%%%%%%%%%%%%%%%%%%%%%%%%%%%%%%%%%%%%



%%%%%%%%%%%%%%%%%%%%%%%%%%%%%%%%%%%%%%%%%%%%%%%%%%%%%%%%%%%%%%%%%%%%%%%%%%%%%%%%



%%%%%%%%%%%%%%%%%%%%%%%%%%%%%%%%%%%%%%%%%%%%%%%%%%%%%%%%%%%%%%%%%%%%%%%%%%%%%%%%
% \section*{APPENDIX}

% Appendixes should appear before the acknowledgment.

% \section*{ACKNOWLEDGMENT}

% The preferred spelling of the word ÒacknowledgmentÓ in America is without an ÒeÓ after the ÒgÓ. Avoid the stilted expression, ÒOne of us (R. B. G.) thanks . . .Ó  Instead, try ÒR. B. G. thanksÓ. Put sponsor acknowledgments in the unnumbered footnote on the first page.



%%%%%%%%%%%%%%%%%%%%%%%%%%%%%%%%%%%%%%%%%%%%%%%%%%%%%%%%%%%%%%%%%%%%%%%%%%%%%%%%

% References are important to the reader; therefore, each citation must be complete and correct. If at all possible, references should be commonly available publications.



% \begin{thebibliography}{99}
% \bibitem{c1} Martin-Brualla, R., Radwan, N., Sajjadi, M. S., Barron, J. T., Dosovitskiy, A., & Duckworth, D. (2021). NeRF in the wild: Neural radiance fields for unconstrained photo collections. In Proceedings of the IEEE/CVF conference on computer vision and pattern recognition (pp. 7210-7219).
% \bibitem{c2} Turki, H., Ramanan, D., & Satyanarayanan, M. (2022). Mega-NeRF: Scalable construction of large-scale NeRFs for virtual fly-throughs. In Proceedings of the IEEE/CVF Conference on Computer Vision and Pattern Recognition (pp. 12922-12931).
% \bibitem{c3} Xu, L., Xiangli, Y., Peng, S., Pan, X., Zhao, N., Theobalt, C., ... & Lin, D. (2023). Grid-guided neural radiance fields for large urban scenes. In Proceedings of the IEEE/CVF Conference on Computer Vision and Pattern Recognition (pp. 8296-8306).
% \bibitem{c4} Ost, J., Mannan, F., Thuerey, N., Knodt, J., & Heide, F. (2021). Neural scene graphs for dynamic scenes. In Proceedings of the IEEE/CVF Conference on Computer Vision and Pattern Recognition (pp. 2856-2865).
% \bibitem{c5}
% \bibitem{c6}
% \bibitem{c7}
% \bibitem{c8}
% \bibitem{c20} J. P. Wilkinson, ÒNonlinear resonant circuit devices (Patent style), Ó U.S. Patent 3 624 12, July 16, 1990. 
% \end{thebibliography}
% \bibliographystyle{unsrt}
% \bibliography{reference}

\bibliographystyle{IEEEtran}
\bibliography{IEEEabrv, reference}


\end{document}
