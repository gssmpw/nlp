\section{Related Work}
Our paper brings together multiple fields often studied separately. In the following, we outline numerous connections between byzantine elements (and, more specifically, our selection problem) and other areas studying related notions and problems. Outlining these connections can lead to fruitful interdisciplinary future insights.

\noindent \textbf{Fault-Tolerant Distributed Computing.} The fault-tolerant distributed computing literature often considers a setup with $n$ parties, out of which at most $t$ are corrupted and may exhibit arbitrary deviations from the intended behavior. Following the tradition of the field, such parties are called \emph{byzantine}, while the other (at least) $n - t$ parties are referred to as \emph{honest} and will follow the intended behavior. One of the most prominently-studied problems is binary \emph{Byzantine Agreement} (BA), where each party $i$ has an input $x_i \in \{0, 1\}$, and must eventually provide an output $y_i \in \{0, 1\}$, such that any two honest parties give the same output (the \emph{agreement} condition) and this output is moreover the input of some honest party (the \emph{validity} condition). 
Depending on additional assumptions (e.g., synchronous communication, cryptographic setup, randomization), the difficulty of designing a correct BA protocol varies, with protocols resilient against $t < n/2$ corruptions existing in certain settings.
The core challenge here lies in the global lack of trust: nobody can be sure who are the corrupted parties, and hence protocols resilient against even a small number of corruptions can be highly non-trivial.
In contrast, assuming the existence of a \emph{trusted third party (TTP)} that is guaranteed to be honest makes BA essentially trivial: all parties send their $x_i$'s to the TTP, which takes a majority vote and then sends back the outcome to all parties. Consequently, efforts in the literature boil down to simulating the behavior of the TTP in an untrusted setting. The keen reader might have already noted that the simple TTP protocol above fails to work past the $t < n/2$ corruption threshold (if too many parties are byzantine, they can outnumber the honest votes), and, in fact, no TTP protocol can surpass this bound, implying that $t < n/2$ is best possible in the distributed setting too.
Hence, from a fresh point of view, understanding the difficulty of problems in fault tolerance begins with understanding their difficulty in the TTP setting. While this was mostly straightforward for binary BA (and is hence rarely considered), the story becomes interesting again for agreement problems over other domains:

When the inputs are real numbers, \emph{Honest-Range Validity} ____ requires that the agreed-upon value is between the smallest and largest honest inputs. A stronger requirement stands in variations of \emph{Median Validity} ____, asking that the output is close to the median of the honest inputs. An orthogonal generalization of \emph{Honest-Range Validity} is when the inputs are $D$-dimensional real vectors, where \emph{Convex Validity} ____ requires the output vector to be in the convex hull of the honest input vectors. This has been further generalized to abstract convexity spaces, including graphs and lattices ____. When the inputs are linear orders over a set of alternatives, also known as \emph{rankings} in social-choice theoretic terms, ____ considers \emph{Pareto Validity}, requiring that whenever all honest inputs rank $a$ above $b$, so does the output ranking. The same paper also studies a second problem, requiring an output ranking that is close to the \emph{Kemeny median} of the honest rankings. Another notion inspired by social-choice is that of \emph{Voting Validity} ____.

Note how all these problems are non-trivial even in the TTP setting: how should the TTP aggregate $n$ votes to yield a valid outcome when $t$ of the votes might be corrupted and should not be considered? The problem we study is precisely of this flavor: the TTP receives the box values $v_1, \dots, v_n$ and must output a set of boxes to open to minimize the loss from empty boxes\footnote{Technically, maximize the win from non-empty boxes (the sum of the two is non-constant, so there is a distinction).} (possibly using randomization). In general, recent results indicate that the difficulty of solving the ``centralized'' TTP problem often matches that of its decentralized version, at least when considering BA with non-trivial validity conditions ____. Hence, it is imperative to understand the difficulty of the centralized problem in more settings, and our paper sets out to do just this for our selection problem.

\noindent\textbf{Malice in Game Theory.} While most of game theory considers rational (selfish) agents, a few works consider the presence of malicious agents, who derive their utility from making the system perform poorly or from others' disutility. This has been particularly investigated in distributed settings, where considering malicious actors is the norm. An excellent survey of work at the interface of rational and malicious behavior can be found in ____, which also introduces the \emph{Price of Malice}, quantifying the degradation of the performance of a system of rational agents with the introduction of malicious actors.\footnote{The notion is inspired by the celebrated \emph{Price of Anarchy}, which instead targets the difference between selfish and collaborating agents.} The same paper analyzes the price of malice in a virus inoculation game, and subsequent papers study it and other notions of malice in other classes of games ____. Particularly relevant to the distributed computing field is the concept of \emph{BAR fault tolerance} ____, which requires distributed protocols to withstand \textbf{B}yzantine, \textbf{A}ltruistic, and \textbf{R}ational (BAR) behavior. Specifically, they must tolerate a constant fraction of Byzantine agents, as is standard, while ensuring that the remaining agents --- who are rational --- have sufficient incentives to follow the protocol. Subsequent work adopted the notion and applied it in a variety of distributed settings, spanning theory and practice ____. A number of works define fault-tolerant solution concepts, e.g., fault-tolerant Nash Equilibria, and apply them to study the fault tolerance of various games ____. Also relevant are the appealing results in ____, which find that ``large'' games are naturally fault-tolerant. Malice can also manifest as \emph{spite}. In this context, ____ examine auctions with spiteful agents, i.e., agents who derive utility from others' disutility. Conversely, several works consider altruistic agents (which, in the previous context, referred to agents who follow the protocol regardless of incentives, but this is not a strict requirement), modeled as agents for whom larger utilities of others translate to larger utilities for themselves ____.

\noindent \textbf{Bribery in Voting.} The social choice literature considers election \emph{bribery} ____: changing the outcome 
by bribing a subset of voters to change their ballots. The aggregation problem that the TTP faces in distributed computing is of a similar flavour: the byzantine votes can be thought of as bought votes, and the bound on the number of corruptions $t$ as the budget of the briber. However, there is a notable difference: in voting the aggregation mechanism is fixed and the aim is to quantify the damage that can be done for a given profile of votes given a budget $t$. In contrast, our goal is designing a robust aggregation mechanism that is aware of the presence of at most $t$ ill-intended votes (which should be rightfully nullified if ever identified, although no signals to this end exist in our setting).

\noindent \textbf{Stackelberg and Security Games.} Our problem can be seen as a \emph{zero-sum Stackelberg (maxi-min) game} where the leader commits to a (potentially randomized) strategy of picking $\ell$ boxes, and then the follower (i.e., the adversary), knowing the leader's choice, chooses $t$ boxes to nullify. The leader (follower) attempts to maximize (minimize) the expected sum of the $\ell$ selected boxes post-nullification.
Related is the class of \emph{security games}, which have received extensive attention in recent years ____. In one variant ____, there are $n$ potential targets, an attacker, and a defender. The defender moves first, chooses $t$ targets to protect (using a potentially randomized strategy), and then the attacker chooses $\ell$ targets to attack, knowing the strategy of the defender. The utilities of the two players additively depend solely on attacked targets, with the goal of the attacker being to attack undefended targets and that of the defender being to protect attacked
targets, but the game is not necessarily zero-sum. The existing literature largely concerns the non-zero-sum case with $\ell = 1$ attacked targets. On the computational front, security games are largely amenable to techniques from combinatorial optimization ____, often linear programming ____. Results become scarcer when seeking more efficient algorithms: ____ for the $\ell = 1$ case, and ____ for the general-$\ell$ case if instead of Stackelberg Equilibria we require Nash Equilibria, the former becoming computationally demanding. Our problem corresponds to a zero-sum variant of the previously described security game with the order the attacker and defender play in reversed. We note that this does not fundamentally change the game since for zero-sum games, the maxi-min and mini-max values coincide by von Neumann's theorem ____. Stackleberg and Nash Equilibria are also closely tied in the zero-sum case, so the algorithm of ____ can be used for our setting, but it is arguably more complicated than our approach (and quadratic instead of linear).

\noindent \textbf{Robust Combinatorial Optimization.} An active area of operations research concerns optimization under uncertainty____: when there is uncertainty in the constraints or objective. 
For our purposes, let us restrict ourselves to a problem template where only the objective is uncertain: the (not necessarily continuous) feasible region is known, denoted by $\mathcal{F} \subseteq \mathbb{R}^n$, and we are interested in $\max_{x \in \mathcal{F}} c^T x$. Instead of knowing $c$, we only know an uncertainty set $\mathcal{U}$ such that $c \in \mathcal{U}$. No probability distribution over $\mathcal{U}$ is supplied: we seek a solution $x$ that maximizes $c^T x$ in the worst-case, making for the \emph{maxi-min objective} $\max_{x \in \mathcal{F}} \min_{x \in \mathcal{U}} c^T x$. This setup is very flexible, as $\mathcal{F}$ can range from a polytope in the continuous case to the set of $s$-$t$ paths or spanning trees of a graph in the discrete one. The maxi-min objective can be replaced with regret-inspired variants. The uncertainty set $\mathcal{U}$ can take various shapes, with the two most prominent ones being \emph{discrete uncertainty}: $\mathcal{U}_D = \{c_1, \dots, c_k\}$ and \emph{interval uncertainty}: $\mathcal{U}_I = [a_1, b_1] \times \dots \times [a_n, b_n]$. Interval uncertainty admits a variant in the spirit of byzantine fault-tolerance, introduced in ____: given a threshold $\Gamma$ define \emph{$\Gamma$-interval uncertainty} $\mathcal{U}_I^\Gamma$ such that $c \in \mathcal{U}_I^\Gamma$ if $c \in \mathcal{U}_I$ and $|\{i : c_i \neq b_i\}| \leq \Gamma$. See ____ for excellent surveys of results and techniques. One of the basic cases considered in the robust optimization literature concerns $\mathcal{F} = \{x \in \{0, 1\}^n : \sum_{i = 1}^n x_i = \ell\}$, the so-called \emph{Selection Problem}: see ____ for a compilation of results under various objective and uncertainty set assumptions. Of particular interest to us is the case with the normal maxi-min objective and $\Gamma=t$-interval uncertainty for $\mathcal{U}_I = [0, v_1] \times \dots \times [0, v_n]$: this corresponds exactly with choosing to open $\ell$ boxes, out of which the adversary can nullify $t$. Note, however, that this only models the deterministic part of our paper (which is straightforward): it cannot model committing to a randomized strategy of which $\ell$ boxes to open to which the adversary replies by nullifying $t$ so as to minimize the expectation. One of the few papers considering such randomized strategies is ____, where the authors show that for discrete and interval uncertainty sets, it is possible to optimize the regret objective in polynomial time as long as (non-robust) optimization over $\mathcal{F}$ is feasible polynomially (which trivially holds for the Selection Problem). Note that their result does not target $\Gamma$-interval uncertainty and is for the regret objective, hence not applicable to us. 
The paper ____ also refers to an unpublished (and not publicly available) 2012 manuscript of Bertsimas, Nasrabadi, and Orlin, titled ``On the power of nature in robust discrete optimization,'' claiming a similar result for our maxi-min objective, this time more generally applying to any pair $(\mathcal{F}, \mathcal{U})$ such that (non-robust) optimization over both $\mathcal{F}$ and $\mathcal{U}$ is feasible polynomially (which is the case in our setting). Modulo the fact that their paper cannot be reasonably retrieved, their result implies polynomial solvability for our problem (most likely using continuous optimization techniques like LP, and hence not strongly-polynomial). Our approach to the problem will be different, leading to a better, linear-time, algorithm.

\noindent \textbf{Statistical Learning with Adversarial Noise.} Learning the underlying distribution or statistics about a dataset is a fundamental problem in statistical learning. However, real-world data is often ill-behaved, including a fraction of adversarially corrupted/byzantine data. An extensive line of work has been dedicated to robust learning in this setting ____, including but not limited to learning high-dimensional Gaussians with adversarial noise.

\noindent \textbf{Adversarial Bandits and Experts.} \emph{Bandit} and \emph{Expert Learning} are two related cornerstone models of decision-making under uncertainty, with applications ranging from machine learning to theoretical computer science and economics ____. In the former, a decision-maker faces a slot machine with $n$ arms with unknown, possibly different, reward distributions. At each time-step, having observed past behavior, the decision-maker must select an arm to pull, gaining an instantaneous reward sampled from that arm's distribution independently of other time-steps.
The standard goal is regret minimization across a (possibly not fixed) time horizon $T$. Adversarial variants of the model have been explored, where instead of independent stochastic rewards, the rewards in each time slot are chosen by an adversary, bringing the setup closer to ours. Our problem can be seen as a single-shot ($T = 1$) variant of adversarial bandits. Here, each arm gives a known reward of either $0$ or $v_i$, and the number of zeros is limited by $t$ (the number of Byzantine arms), but it is not known which $t$ arms give zeros. The goal also changes from regret minimization to value maximization. Some work also explored models in-between stochastic and adversarial bandits ____, closer to our byzantine setting.
More broadly, there is interest in such in-between models ____.

\noindent \textbf{Secretary Problem.} In the classical \emph{Secretary Problem} ____, a stream of $n$ items arrive one by one in an online fashion. After each arrival, the algorithm should either commit to that item or discard it permanently. Each item has a value, and common goals include maximizing the probability of picking a highest-value item or maximizing the expected value of the picked item. A plethora of variations of the problem have been explored, including picking $\ell$ items instead of a single one ____, connecting the problem to our paper. In the standard secretary model, the adversary picks the values of the items knowing the algorithm, and then the items are presented to the algorithm in an order chosen uniformly at random (the \emph{random-order} model). Most developed algorithms are not robust to even small adversarial perturbations of the random order assumption. To study this effect, ____ introduces a semi-random model, where all values stay adversarial, but the arrival times of $t$ items are adversarial, chosen before the other $n - t$ items' arrival times are randomly generated. Like us, inspired by the distributed computing literature, they call this the \emph{Byzantine Secretary} model. We note that the problem in ____ is very different from ours: they consider an online setting where all values are adversarial, but the arrival times of only $t$ are adversarial, while our setting is offline, with all but $t$ of the values being honest. More subtly, given the values $v_1, \dots, v_n$, we want 
the best possible solution for a given instance, instead of the best achievable for a worst-case instance as in their setting (making our problem harder in this regard).

\noindent \textbf{Prophet Inequalities.} A commonly studied online stopping problem, related to the Secretary Problem, but also to Bandit Learning due to its online flavor, goes under the name of \emph{Prophet Inequalities} ____. The basic setup considers $n$ random variables with known distributions $X_1, \dots, X_n$. One by one, the realization of these random variables is revealed to the decision-maker, who has the option between committing to the current value or passing to the next round. The goal is to maximize the expectation of the selected value, and the standard result is that a policy achieving an expectation of at least half of the ex-post maximum value exists. Variants where $\ell$ values are to be selected have been studied, but once again, the combination of the stochastic and online aspects makes this setup very different from ours.