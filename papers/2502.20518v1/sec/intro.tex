\section{Introduction}\label{sec:intro}
% Bullet Point
% - What is IAA
% Image aesthetic assessment evaluates the aesthetic score, or preference, of images using AI models.

% - What are GIAA and PIAA
% GIAA and PIAA evaluate the mean score (or score distribution) of groups and the personal score of images, respectively.

% - Why are they difficult
% IAA is challenging due to individual subjectivity, as the same image can yield diverse aesthetic scores across individuals. IAA must therefore address both image diversity and the varied personal preferences due to demography.

% - What is the problem with current methods
% Current approaches divide the modeling of image diversity and individual subjectivity into GIAA and PIAA, first learning mean scores (without individual subjectivity) across images with GIAA, then fine-tuning the GIAA models for personal scores in PIAA. 
% Problem:
% (1) Existing PIAA approaches learn personal scores based on GIAA models, making aesthetic differences between groups and individuals feasible only through parameter shifts, which can not explicitly reflect how the aesthetic changes as the demography changes from group to individual.
% (2) The existing approach oversimplifies the composition of groups, consisting of demography and the number of individuals in groups. Averaging the score may still bias the score due to these factors.
% (3) The model performance on unseen users, as well as how it correlates to the demographic difference between those in the trainset, is unclear

% - What you propose: What is your hypothesis and how do you test it? 

% (1) How to model both GIAA and PIAA in such a way that the aesthetic between groups and individual is explicitly conditioned on their demography?
% We propose a unified modeling of GIAA and PIAA that encodes all demography in a distributional manner. This provides a theory to illustrate the transfer learning between GIAA and PIAA. Our theory indicates a high zero-shot GIAA performance of a PIAA model, and the experiment shows that it achieves comparable performance to the GIAA baseline. 

% (2) Can the individual subjectivity of image aesthetics be removed by averaging the individual score? 
% If so, we should observe an invariant performance, of GIAA models, upon various groups' composition of the test users. Also, its transfer learning to PIAA should also be invariant to the group's composition.
% We start by varying the demography and the group size under the same user sets of train and test sets. Our results show that by using our proposed sub-sampled GIAA (sGIAA) to augment the group's composition, the zero-shot PIAA performance of our models greatly increases by 13.6 SROCC while keeping the GIAA performance. 

% We then test the model on disjoint users for train and test sets, split according to demography. Our results demonstrate a the GIAA model performance can vary by at most 41.8\% and 60.9\% on photo and artwork datasets, respectively. The significant variation in performance indicates that the mean score does not completely remove the individual subjectivity but a large score bias caused by demography is dominant. On the other hand, the PIAA model performance can vary by at most 24.1\% and 80.6\% on photo and artwork datasets, respectively. 


% IAA is complex because of image diversity and individual subjectivity. GIAA -> PIAA
Assessing the aesthetics of images, known as Image Aesthetics Assessment (IAA), is a challenging task due to the inherent complexity of image diversity and individual subjectivity~\cite{talebi2018nima, yang2022personalized, ke2021musiq, yang2023multi, zhu2022personalized, zhu2020personalized, strafforello2024backflip}. 
Extensive research~\cite{van2021cross, kossmann2023composition, van2023order, wagemans2012century, damiano2023role} has explored how image aesthetics correlate with various image attributes, \eg spatial composition~\cite{van2021cross,kossmann2023composition,van2023order}, figure-ground organization~\cite{wagemans2012century}, and symmetry~\cite{damiano2023role}. Image aesthetics is also influenced by individual differences in perception~\cite{van2021individual, de2014individual, samaey2020individual}, contributing to individual subjectivity that correlates with demographic factors. 
%Moreover, the subjectivity in image aesthetics can vary with image types, as the aesthetics of artworks are generally considered more subjective than those of photographs~\cite{vessel2018stronger}. 
This adds further challenges for modeling IAA.

Current IAA approaches address these challenges in two stages. First, Generic IAA (GIAA) models~\cite{talebi2018nima,yang2022personalized,ke2021musiq} %are used to 
estimate averaged user aesthetic scores or score distributions across a broad range of images, aiming to capture a mean score without individual subjectivity. Subsequently, Personal IAA (PIAA) models~\cite{yang2022personalized,yang2023multi,zhu2022personalized,zhu2020personalized,li2020personality,zhu2021learning,li2022transductive,yan2024hybrid,shi2024personalized} adapt these generic models, fine-tuning them with a small amount of data (\ie few-shot learning), with or without incorporating personal traits to handle the subjectivity. This process represents a form of transfer learning~\cite{zhuang2020comprehensive} from GIAA to PIAA, even though it is not explicitly defined as such in the existing PIAA literature~\cite{yang2022personalized,yang2023multi,zhu2022personalized,zhu2020personalized,li2020personality,zhu2021learning,li2022transductive,yan2024hybrid,shi2024personalized}.

% Problem:
However, the existing approaches present several limitations. %challenges:
% \begin{itemize}
%     \item Existing PIAA approaches~\cite{yang2022personalized,yang2023multi,zhu2022personalized,zhu2020personalized,li2020personality,zhu2021learning,li2022transductive,yan2024hybrid,shi2024personalized} make it difficult to analyze aesthetic differences between groups and individuals, as well as their correlation with demographic factors, since these differences can only be inferred through parameter shifts observed during fine-tuning of PIAA. 
%     \item Although existing GIAA approaches~\cite{talebi2018nima,yang2022personalized,ke2021musiq} assume that individual subjectivity can be minimized by averaging scores, the bias caused by demography may persist within group averages. Furthermore, these methods often oversimplify group composition by overlooking variations in demographic factors and group size, which can introduce bias and affect the fine-tuning of PIAA. This challenge remains largely unexplored.
%     \item Furthermore, the existing GIAA~\cite{talebi2018nima,yang2022personalized,ke2021musiq} and PIAA~\cite{yang2022personalized,yang2023multi,zhu2022personalized,zhu2020personalized,li2020personality,zhu2021learning,li2022transductive,yan2024hybrid,shi2024personalized} approaches do not adequately address generalization to unseen users. Given the high subjectivity of image aesthetics, it is crucial to investigate model generalization on unseen test users and how it correlates with demographic differences between training and test users.
% \end{itemize}
\textit{1)} Existing PIAA approaches~\cite{yang2022personalized,yang2023multi,zhu2022personalized,zhu2020personalized,li2020personality,zhu2021learning,li2022transductive,yan2024hybrid,shi2024personalized} make it difficult to analyze aesthetic differences between groups and individuals, as well as their correlation with demographic factors, since these differences can only be inferred through parameter shifts observed during fine-tuning of PIAA. 
\textit{2)} Although existing GIAA approaches~\cite{talebi2018nima,yang2022personalized,ke2021musiq} assume that individual subjectivity can be minimized by averaging scores, the bias caused by demography may persist within group averages. Furthermore, these methods often oversimplify group composition by overlooking variations in demographic factors and group size, which can introduce bias and affect the fine-tuning of PIAA. This challenge remains largely unexplored.
\textit{3)} Furthermore, the existing GIAA~\cite{talebi2018nima,yang2022personalized,ke2021musiq} and PIAA~\cite{yang2022personalized,yang2023multi,zhu2022personalized,zhu2020personalized,li2020personality,zhu2021learning,li2022transductive,yan2024hybrid,shi2024personalized} approaches do not adequately address generalization to unseen users, \ie zero-shot learning. Given the high subjectivity of image aesthetics, it is crucial to investigate model generalization on unseen test users and how it correlates with demographic differences between training and test users.

%
For the first limitation, we %first 
propose a novel IAA approach by encoding personal traits in a distributional format that accounts for both individual and group characteristics. %We input the distributional trait encoding to a single model that performs both GIAA and PIAA.
Our method is capable of inferencing both GIAA and PIAA with a single model by receiving the corresponding trait distribution as input. 
This approach reveals the geometry of the IAA domain, where the input space (personal traits) and output space (aesthetic scores) form distinct convex hulls based on personal data for given images, as depicted in Figure~\ref{fig:convex_hull}. We refer to these convex hulls as the trait convex hull and the score convex hull, respectively.
In this context, GIAA maps the average trait distribution located at the inner regime of the trait convex hull %center of the convex hull 
to the average score distribution located at the inner regime of the score convex hull. %center of the convex hull 
In contrast, PIAA maps each vertex of the trait convex hull to corresponding points in the score convex hull. 
This geometric insight leads us to propose a theory suggesting that transfer learning from GIAA to PIAA represents an extrapolation within the characteristic domain, whereas the reverse direction constitutes an interpolation—a generally more effective approach for machine learning models. It follows that PIAA models can perform well on GIAA data without GIAA pre-training. 

For experimental validation, we extend the GIAA and PIAA baseline models by conditioning them on a %to incorporat 
distributional trait encoding and demonstrate that our models achieve performance comparable to the unconditioned GIAA and PIAA baselines. %, confirming the effectiveness of this encoding strategy.
%
%Second, 
Then, we modify PIAA training by omitting few-shot sampling and instead use the full training data to ensure a fair comparison between GIAA and PIAA on the same images and users. 
Direct training on PIAA data achieves performance comparable to the GIAA baseline, whereas training on GIAA data underperforms relative to the PIAA baseline. Additionally, we demonstrate that models trained directly on PIAA data match the performance of the GIAA baseline and even exceed the PIAA baseline.
These results support our proposed theory regarding interpolation and extrapolation. Furthermore, this framework provides a basis for explaining observed aesthetic differences between groups and individuals.

\begin{figure*}[h]
    \centering
    \includegraphics[width=1.0\linewidth]{figures/convex_hull.pdf}
    \caption{
    %The geometry of IAA. %The input convex hull is in the space of demographic distribution, \eg age, gender, and education. The output convex hull is in the space of scores distribution. The vertices denote the data for individuals and the center denotes the mean of groups.
    %
    The geometry of IAA. For a given input image, the input trait and output score of the IAA are represented by two convex hulls.
    Left: The input convex hull in the space of trait distribution, \eg age, gender, and education; Right: The output convex hull in the space of aesthetic scores distribution.
    %
    The averaged personal traits distribution of different subsets of individuals all lie within the convex hull formed by the individual traits (\textit{Trait Domain}). 
    These traits can be provided as input to the IAA model, in addition to the input image.
    %
    Similarly, the average aesthetic score distribution given by a group of individuals all lie within the convex hull formed by the aesthetic scores given by single individuals (\textit{Score Domain}).  For both convex hulls, each point $P_i$ represents an individual data point, where $i = 1, 2, 3, \dots$, and $G$ represents the averaged data.
}
    \label{fig:convex_hull}
\end{figure*}

%Moreover, 
% Third, 
We address the second limitation by showing the impact of the group's composition in the training set on GIAA models, confirming that averaging individual scores does not necessarily eliminate individual subjectivity.
%
%Second, averaging individual scores does not necessarily eliminate the individual subjectivity in our theory. 
%For the second limitation, We further demonstrate the impact of the group's composition in the training set on GIAA models. 
To this end, we introduce \textit{sGIAA}, a data augmentation method that sub-samples GIAA data, increasing diversity in both demography and group size. 
% 
By sampling between 2 and the maximum number of users per image, we show that this approach maintains GIAA performance while significantly improving zero-shot PIAA performance by 20.9\%. This improvement arises from increased demographic variation and 
by pulling the training domain of the group closer to individual users, when sampling smaller group sizes.
%
%closer exploration of GIAA data at an individual level by sampling smaller group sizes. 
%This also underscores the advantage of our approach in inferring both GIAA and PIAA performance. While the effect on GIAA is negligible, the significant increase in zero-shot PIAA performance impacts subsequent PIAA fine-tuning, highlighting the role of group composition and addressing the second issue.
%
While this approach has negligible effects on GIAA, the significant increase in zero-shot PIAA performance impacts subsequent PIAA fine-tuning. This observation highlights the role of group composition, disregarded in previous approaches. 

%Lastly, 
For the third limitation, we assess the model's ability to generalize to new users by dividing training and test users into distinct groups, according to demography such as gender and education level.
We begin by analyzing the score distribution within each demographic category (\eg, male and female) and calculate the Gini index~\cite{lerman1984note} across these groups to evaluate the effectiveness of demographic-based separation in score distribution. Our findings reveal that, for both photos and artworks, education is the primary factor driving distinct aesthetic judgments. Additionally, experience with photography and art emerges as the second most influential factor for the photo and artwork datasets, respectively. Furthermore, the lower Gini index in the artwork dataset compared to the photo dataset indicates that individual subjectivity is stronger for artworks than for photos. 

We then evaluate the proposed IAA model %models 
on users with distinct demographic groups, calculating the Earth Mover's Distance (EMD)~\cite{rubner2000earth} between their score distributions. Consistent with the Gini index trends, the largest variations in EMD are observed for education, photo experience, and art experience. 
%Additionally, a strong correlation is found between EMD and GIAA model performance, with a Pearson Linear Correlation Coefficient (PLCC) of -0.980 on the photo dataset and -0.721 on the artwork dataset. The lower PLCC for the artwork dataset is attributed to demographic imbalance. 
%Furthermore, the GIAA model performance can vary by at most 41.8\% and 60.9\% on photo and artwork datasets, respectively. On the other hand, the PIAA model performance can vary by at most 24.1\% and 80.6\% on photo and artwork datasets, respectively. 
%
% The greater variation in model performance on the artwork dataset again highlights the individual subjectivity for artworks compared to photographs for both GIAA and PIAA. 
Additionally, we observe a greater variation in model performance on the artwork dataset compared to the photo dataset, which again indicates higher individual subjectivity for artworks compared to photographs, for both GIAA and PIAA. 
%This investigates the third issue and 
This further emphasizes the challenge of PIAA when generalizing to unseen users with varying demographic profiles.

To our knowledge, ours is the first model enabling both GIAA and PIAA. It matches the performance of the GIAA baseline and even surpasses state-of-the-art PIAA models that require GIAA pre-training. Furthermore, it is the first theoretical framework to address aesthetic differences between groups and individuals, account for diverse demographic factors and group size, and support model generalization to unseen users.
