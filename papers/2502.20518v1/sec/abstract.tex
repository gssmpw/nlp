\begin{abstract}
Image aesthetic assessment (IAA) evaluates image aesthetics, a task complicated by image diversity and user subjectivity. Current approaches address this in two stages: Generic IAA (GIAA) models estimate mean aesthetic scores, while Personal IAA (PIAA) models adapt GIAA using transfer learning to incorporate user subjectivity. 
However, a theoretical understanding of transfer learning between GIAA and PIAA, particularly concerning the impact of group composition, group size, aesthetic differences between groups and individuals, and demographic correlations, is lacking. 
%
This work establishes a theoretical foundation for IAA, proposing a unified model that encodes individual characteristics in a distributional format for both individual and group assessments. We show that transferring from GIAA to PIAA involves extrapolation, while the reverse involves interpolation, which is generally more effective for machine learning. 
%
Experiments with varying group compositions, including sub-sampling by group size and disjoint demographics, reveal significant performance variation even for GIAA, indicating that mean scores do not fully eliminate individual subjectivity. Performance variations and Gini index analysis reveal education as the primary factor influencing aesthetic differences, followed by photography and art experience, with stronger individual subjectivity observed in artworks than in photos.
Our model uniquely supports both GIAA and PIAA, enhancing generalization across demographics.

\end{abstract}