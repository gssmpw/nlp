\section{Conclusion}
%In summary, 
We propose the first model capable of supporting both GIAA and PIAA, matching GIAA baseline performance and even surpassing state-of-the-art PIAA models that require GIAA pre-training. Additionally, our model introduces the first theoretical framework for addressing aesthetic differences between groups and individuals, accounting for diverse demographics and group size.

Our comprehensive experiments investigate the transfer learning between GIAA and PIAA under these factors. The results support our theory that transferring from GIAA to PIAA involves extrapolation, while the reverse—interpolation—is generally more effective for machine learning. Additionally, sub-sampled GIAA (sGIAA) improves zero-shot PIAA performance by 20.9\%, underscoring the importance of group size variation, especially for PIAA fine-tuning in the current scenario.
%
For unseen users with diverse demographics, significant performance variations %—up to 60.9\% and 80.6\% fluctuations on artworks datasets for GIAA and PIAA, respectively—
underscore the challenges faced by existing IAA approaches in model generalizability and the heightened subjectivity in artworks. Additionally, Gini index analysis reveals education as the primary factor influencing aesthetic differences, followed by photography and art experience.

% Our proposed method is limited by the need of personal traits as input, which might not always be available.
% However, we intend our approach to be used as a theoretical framework to study the role of individual differences in IAA, which can guide the development of practical solutions to this task.
While our method requires personal traits as input, which may not always be available, it provides a theoretical framework for studying individual differences in IAA, guiding the development of practical solutions.
% \clearpage