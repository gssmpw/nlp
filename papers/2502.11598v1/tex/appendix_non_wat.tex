\section{Impact of Inference-Time Watermark Neutralization (WN) on Knowledge Preservation under Non-watermarked Setting}
Previous experiments have demonstrated that when teacher model is watermarked, inference-time watermark neutralization (WN) effectively enables the student model to bypass the watermark while acquiring knowledge from the teacher model's outputs that is comparable to what would be learned without any attack. In this section, we conducted additional experiments to examine whether applying WN would affect the knowledge acquired by the student model in cases where the teacher model itself is not watermarked (noting that the student model has no access to the detector and thus cannot determine the presence of watermarks).

Consistent with the settings in our main experiments, we generated QA pairs using GLM-4-9b-chat. After filtering and deduplication, we obtained 200,000 non-watermarked samples. We trained both Llama-7b and Llama-3.2-1b models using this dataset, and then applied WN for watermark removal and evaluated the performance changes across various benchmarks, with results presented in Table \ref{tab:knowledge_non_wat}. It can be concluded that WN \textit{does not} exert a substantial negative impact on knowledge preservation under non-watermarked setting.

\begin{table*}[h!]
    \caption{Impact of WN on knowledge preservation under non-watermarked setting.}
    \centering
    \resizebox{0.78\textwidth}{!}{
    \begin{tabular}{ccccc}
        \toprule
        \textbf{Benchmark} & \textbf{Student Model} &
        \textbf{Ori. SM} & \textbf{Trained SM} & \textbf{Trained SM + WN} \\
        \midrule
        \multirow{2}{*}{\makecell{ARC Challenge \\ (ACC)}} & Llama-7b & 0.4181 & 0.4454 & 0.4488 \textcolor[rgb]{0.0,0.5,0.0}{(+0.7\%)}\\
        & Llama-3.2-1b & 0.3166 & 0.3524 & 0.3507 \textcolor[rgb]{0.5,0.0,0.0}{(-0.5\%)}\\
        \midrule
        \multirow{2}{*}{\makecell{TruthfulQA \\ Multiple Choice (ACC)}} & Llama-7b & 0.3407 & 0.4254 & 0.4507 \textcolor[rgb]{0.0,0.5,0.0}{(+5.9\%)}\\
        & Llama-3.2-1b & 0.3768 & 0.4052 & 0.3951 \textcolor[rgb]{0.5,0.0,0.0}{(-2.5\%)}\\
        \midrule
        \multirow{2}{*}{\makecell{MTBench \\ (Full Score: 10)}} & Llama-7b & 2.64 & 3.99 & 4.12 \textcolor[rgb]{0.0,0.5,0.0}{(+3.2\%)}\\
        & Llama-3.2-1b & 2.78 & 3.15 & 3.08 \textcolor[rgb]{0.5,0.0,0.0}{(-2.2\%)}\\
        \bottomrule
    \end{tabular}
    }
    \label{tab:knowledge_non_wat}
\end{table*}

