\section{Adaptive Control for Inverse Watermark Strength}
\label{sec:adaptive}
Throughout all previous experiments, we consistently used an inverse watermark strength of $\delta'=2.5$ for WN, which achieved complete watermark removal in all cases. As detailed in Appendix \ref{sec:schemes}, the chosen watermark strength of the teacher model represents a notably high intensity that remains practical for deployment in LLM services, suggesting that $\delta'=2.5$ is sufficient for the vast majority of scenarios.

However, to account for potential extreme cases, we also explored strategies for adaptive control of inverse watermark strength. Our approach is to estimate the required inverse watermark strength $\delta'$ by detecting the watermark intensity inherited by the student model. Since the student model holder does not have access to the watermark detector, we employed Water-Probe \cite{liu2024can} to measure watermark intensity. Water-Probe is a recently proposed identification algorithm that tests for watermarks by comparing the model's responses to specially crafted prompts, where higher similarity in responses to crafted prompts pairs indicates a higher likelihood (or strength) of watermarking.

We conducted experiments using KGW with $n=1$. Table \ref{tab:waterprobe} shows the cosine similarity scores detected by Water-Probe-v2\footnote{There are two versions of Water-Probe, with version 2 demonstrating more stable performance in our experiments.} for Llama-7b student models trained with different watermark strengths $\delta$, compared with a student model trained on unwatermarked data. Based on this reference table, we can estimate the watermark strength $\delta$ used in the teacher model by examining the WaterProbe-v2 cosine similarity score of the trained student model. This estimation enables us to adaptively select an appropriate inverse watermark strength $\delta'$ for removal. 

Here is a practical example: Suppose the teacher model is watermarked using KGW with $n=1, \delta=5.0$. The trained student model's detected cosine similarity is 0.1694, which is slightly higher than the reference value of 0.1366 for $\delta=3.0$. Given our prior knowledge that $\delta'=2.5$ can completely remove watermarks with $\delta=3.0$, we should proportionally increase the inverse watermark strength. Therefore, we set $\delta'=3.0$ for this case. The removal results are shown in Table \ref{tab:kgw_delta5}.

\begin{table}[t]
\caption{Water-Probe-v2 cosine similarity scores for student models under different watermark strength settings.}
\centering
\begin{tabular}{lcccccc}
\toprule
\multirow{2}{*}{\textbf{Settings}} & \multirow{2}{*}{Unw.} & \multicolumn{5}{c}{KGW with Different $\delta$} \\
\cmidrule(lr){3-7}
 & & $\delta=1.0$ & $\delta=2.0$ & $\delta=3.0$ & $\delta=4.0$ & $\delta=5.0$ \\
\midrule
\textbf{Cosine Similarity} & 0.0065 & 0.0853 & 0.1032 & 0.1366 & 0.1544 & 0.1694 \\
\bottomrule
\end{tabular}
\label{tab:waterprobe}
\end{table}
\begin{table}[t]
\caption{Median p-values for watermark detection using WN under different inverse watermark strengths. The watermark used in teacher model is KGW, with $\delta=5.0$, the student model is Llama-7b.}
\centering
\resizebox{0.55\textwidth}{!}{
\begin{tabular}{cccc}
\toprule
\textbf{Token Number} & \textbf{No Attack} & \textbf{WN $\delta'=2.5$} & \textbf{WN $\delta'=3.0$} \\
\midrule
10k & \cellcolor{blue!30}7.75e-59 & \cellcolor{blue!10}2.81e-03 & 9.17e-02 \\
20k & \cellcolor{blue!30}7.48e-115 & \cellcolor{blue!10}6.91e-05 & 4.29e-02 \\
30k & \cellcolor{blue!30}1.14e-170 & \cellcolor{blue!30}6.31e-07 & 1.13e-02 \\
\bottomrule
\end{tabular}
}
\label{tab:kgw_delta5}
\end{table}

We acknowledge that the current estimation method is relatively rough. However, it's important to emphasize that in practical LLM services, it would be unrealistic to use such strong watermarks as $\delta=5.0$, as this would significantly degrade the output quality. In most cases, selecting an inverse watermark strength of $\delta'=2.5$ is already sufficient.