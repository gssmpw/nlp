\section{Details of Training Data Collection}
\label{sec:training-data}
\subsection{Prompt Used for Training Data Collection} 
\label{sec:prompt}
Following the work by \citet{sander2024watermarking}, we prompted the teacher model to generate question-answering samples consisting of instruction, input and answer, as shown in the prompt template in Figure \ref{fig:prompt}.
\begin{figure*}[h!]
\begin{tcolorbox}[colback=gray!10, colframe=black, rounded corners]
You are asked to come up with a set of 20 diverse task instructions and their answers. These instructions will be given to large
language model and we will evaluate it for completing the instructions. Here are the requirements:

\vspace{5pt}

1. Try not to repeat the verb for each instruction to maximize diversity.

2. The language used for the instruction also should be diverse. For example, you should combine questions
with imperative instructions.

3. The type of instructions should be diverse. The list should include diverse types of tasks like open-ended
generation, classification, math reasoning, code generation, world common knowledge, reading comprehension, editing, etc.

2. A language model should be able to complete the instruction. For example, do not ask the assistant to create any visual or audio output. For another example, do not ask the assistant to wake you up at 5pm or set a reminder because it cannot perform any action.

3. The instructions should be in English.

4. The instructions should be 1 to 2 sentences long. Either an imperative sentence or a question is permitted.

5. You should generate an appropriate input to the instruction. It should contain a specific example for the instruction when needed.

6. Most of the instructions do not require input. In this case, you must simply put "None" in the input field.

7. The input should not be the answer to the instruction. For example, if the instruction is "Write a headline for ...", the input should not be the said headline. Instead, the input should be "None".

8. The output should be an appropriate response to the instruction and the input. Make sure the output is less than 100 words.

9. Do not repeat instructions.

10. Use [[, ]] to flag the start and end of each example.

\vspace{5pt}

Examples of generating 3 task instruction and its answer:

\vspace{5pt}

[[Instuction: Give a detailed description of a method for brainstorming an essay.

Input:

Answer: Mind Mapping is one of the methods for brainstorming an essay. Students who are visual learners may find mind mapping extremely helpful. In order to Mind Mapping: Take a paper and draw a circle in the center Write your topic inside it. Just like we sketch rays protruding from a sun, draw lines coming out of the circle. Write an argument or a statement related to the topic at the
end of each line. Draw another set of lines projecting from these statements and state examples that support them. Continue this exercise until you run out of all possibilities imaginable for this topic.]]

\vspace{5pt}

[[Instuction: Write a creative or imaginative response to the given prompt.

Input: You are a time traveler who has just arrived in ancient Egypt. What do you see?

Answer: Wow! As I step out of my time machine, I’m surrounded by towering pyramids and the warm sun beats down on my face. I see Pharaohs riding in grand chariots, and workers building majestic temples. It’s like something out of a dream!]]

\vspace{5pt}

[[Instuction: Solve a math problem.

Input: A snail is at the bottom of a 20-foot well. Each day, it climbs up 3 feet, but at night, it slips back 2 feet. How many days will it take for the snail to reach the top of the well?

Answer: To solve this problem, we first calculate the net distance the snail covers each day: 3 feet up minus 2 feet down, 
resulting in 1 foot of progress daily. Since the well is 20 feet deep, we can set up the equation: number of days $\times$ 1 foot = 20 feet.  Solving this, we find it takes 20 days for the snail to reach the top. We can verify this by noting that after 19 days, the snail will 
have climbed 19 feet. On the 20th day, it will climb the final 3 feet, reaching the top before any night-time slipping occurs. 
Therefore, the snail will reach the top of the well in 20 days.]]
\end{tcolorbox}
\caption{Prompt used for acquiring training data.}
\label{fig:prompt}
\end{figure*}

\subsection{Example Samples of Training Data}
\label{sec:examples}
Example samples of training data produced using different watermarking schemes and $n$ are shown in Figure \ref{fig:training_data}.
\begin{figure*}[h!]
\begin{tcolorbox}[colback=gray!10, colframe=black, rounded corners]
\textbf{KGW $n=1$} \\[0.5em]
Instruction: How does photosynthesis work? \\
Input: None \\
Answer: Photosynthesis is the process by which green plants, algae, and some bacteria use light to synthesize foods with the help of chlorophyll. \\

\textbf{KGW $n=2$} \\[0.5em]
Instruction: Explain how the Internet has transformed the way people learn.\\
Input: None\\
Answer: The Internet has transformed learning by providing access to vast information, allowing flexible learning times, and facilitating interactive communication between learners and instructors.\\

\textbf{KGW $n=3$} \\[0.5em]
Instruction: Provide the definition of a simile.\\
Input: None.\\
Answer: A simile is a figure of speech in which two things are compared using like or as, often used to enhance the vividness of a comparison.\\

\textbf{SynthID-Text $n=1$} \\[0.5em]
Instruction: How do you brew a cup of green tea? \\
Input: None \\
Answer: 1. Boil water: Start by boiling water to the ideal temperature of 160 degrees Fahrenheit (70 degrees Celsius). 2. Steep the tea: Place a green tea bag or loose-leaf green tea in a tea pot or cup. 3. Add hot water: Pour the hot water directly over the tea. 4. Steep for the appropriate time: Green tea typically needs to steep for about 2-3 minutes. 5. Remove the tea: Take the tea bag or strain the loose leaves from the water. 6. Sweeten if desired: Add honey, sugar, or a sweetener of your choice to taste. 7. Serve: Pour the brewed tea into a cup and enjoy.\\

\textbf{SynthID-Text $n=2$} \\[0.5em]
Instruction: Provide a list of three common kitchen tools and their uses. \\
Input: None \\
Answer: 1. Cutting board: A cutting board is used for chopping, slicing, and dicing various ingredients, such as vegetables, fruits, and meats. 2. Measuring cups and spoons: These tools are used for accurately measuring ingredients, ensuring consistent and precise measurements in recipes. 3. Kitchen knife: A kitchen knife is versatile and used for slicing, dicing, chopping, and mincing a wide range of ingredients.\\

\textbf{SynthID-Text $n=3$} \\[0.5em]
Instruction: Analyze the given quote and identify the underlying theme.\\
Input: The only way to do great work is to love what you do.\\
Answer: 1. Passion and joy in work 2. Importance of personal fulfillment in work 3. The necessity of loving one's job\\

\end{tcolorbox}
\caption{Selected examples in training dataset.}
\label{fig:training_data}
\vspace{-15pt}
\end{figure*}