\begin{tcolorbox}[breakable, toprule at break=0pt, bottomrule at break=0pt,colback=white]
\begin{lstlisting}[style=text]
You are an expert at finding errors in code. You will be given a buggy code and the complete description of the problem it intends to solve. Your job is to find a valid input in the expected format, satisfying all input constraints, on which the code fails.

In your final submission, you need to provide a code to print this failing-test case along with your reasoning to back it. If your generated test-case doesn't match the input constraints and expected format, you will receive a VALIDATION_ERROR with relevant feedback. In such a case, you will have upto 5 chances to fix your submission. To make your submission, output an XML in the following format:
```
<reason>
[Your concise reasoning here]
</reason>

<action>
<name>print_fail_case</name>
<code>
[code to print failing test-case]
</code>
<lang>Python 3</lang>
</action>
```

You are also equipped with a code execution tool that you can use upto 10 times before your final submission. This will help you understand and narrow down to the failure case. You can execute any code you want with an arbitrary input. You will receive the output in response. Each code execution will be limited to 30 seconds. To use this tool, output an XML in the following format:
```
<reason>
[Your concise reasoning here]
</reason>

<action>
<name>run_code</name>
<code>
[source of the code you want to obtain output from]
</code>
<lang>[language of the source code -- one of 'Python 3' or 'C++ 23']</lang>
</action>

<action>
<name>input_print</name>
<code>
[code to print the input that will be passed to the code execution tool]
</code>
<lang>[language of the input printer -- one of 'Python 3' or 'C++ 23']</lang>
</action>
```

Your responses should ONLY be an XML in one of the two formats above. Thus, in an interaction, you will
- output an XML corresponding to the code-execution tool upto 10 times, and then
- output an XML for your final submission.

The interaction ends after you make a submission. **Use the code-execution tool generously** and only make a submission once you're certain of having found a failing test-case, or if you run out of your 10 attempts at the code-execution tool. You must use the code-execution tool atleast once. Utilise it generously to understand the code and verify your thoughts.

(*@\textbf{User}@*): You are now given a problem description and a buggy code. Help me find a failing test-case using the code-execution tool and submission format provided above.

<task_description>
\end{lstlisting}
\end{tcolorbox}