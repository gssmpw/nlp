\chapter{\textcolor{black}{Compacting algorithm}}\label{app: SPIC compacting}

The compacting algorithm $g$ is a crucial component of the proposed \gls{spic} framework, enabling the removal of non-relevant parts of the residual for class-specific applications. Operating only with a mask $\m$, the algorithm eliminates redundant spaces, resulting in a series of bounding box transformations that can be applied to any image.

The process begins by extracting a binary mask $\m$ from the \gls{ssm} $\s$. For example, if the relevant class is "traffic sign," the mask $\m$ will contain 1s at positions corresponding to traffic signs and 0s elsewhere.

This mask is then processed using the standard two-pass binary connected-component labeling algorithm \cite{Shapiro2001connectedCOmp}, which identifies all distinct connected components in the frame and assigns each a unique label. For every connected component $c_i$, a bounding box $b_i$ is defined. Each bounding box is the smallest rectangle that entirely contains its connected component, specified by the coordinates of its top-left and bottom-right corners.

With the bounding boxes identified the iterative process of removing empty regions can start. The algorithm repositions the bounding boxes toward the top-left corner of the frame while preventing any overlap between them. It does this through iterative upward and leftward shifts, continuing until no further movement is possible.

During the upward shift phase, bounding boxes move toward the top edge of the frame until they either contact the top border or another bounding box. Importantly, a bounding box is moved only after all bounding boxes positioned above it have been moved. The leftward shift operates similarly, moving bounding boxes toward the left edge of the frame. By alternating between these two shifts, the algorithm incrementally relocates all bounding boxes to the top-left corner, ensuring that no overlaps occur.

Throughout the process, the algorithm records the original and new coordinates of each bounding box. This information is essential for applying the transformations to other images. After completion, the algorithm outputs pairs of original and final coordinates for every bounding box. These pairs coordinates enable the identification and repositioning of corresponding objects in the residual or original image.

Moreover, since the algorithm relies solely on the \gls{ssm} to generate the mask $\m$ and reposition the bounding boxes, it can be executed at the receiver's end as well. In this context, the coordinate pairs are used to restore the compacted bounding boxes to their original positions. This inverse operation is referred to as $g^{-1}$, the inverse of the compacting algorithm.