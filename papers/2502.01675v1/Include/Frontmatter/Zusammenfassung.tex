\thispagestyle{plain}			% Supress header 
\setlength{\parskip}{0pt plus 1.0pt}
\section*{Zusammenfassung}
Mit der ständigen Weiterentwicklung digitaler Technologien stehen moderne Kommunikationsnetzwerke vor der beispiellosen Herausforderung, die riesigen Datenmengen zu bewältigen, die täglich von vernetzten intelligenten Geräten erzeugt werden. Autonome Fahrzeuge, intelligente Sensoren, IoT-Systeme usw. gewinnen zunehmend an Interesse und es werden neue Kommunikationsparadigmen benötigt. Diese Arbeit befasst sich mit diesen Problemen, indem ein neues Kommunikationsparadigma eingesetzt wird, das semantische Kommunikation mit generativen Modellen kombiniert, um die Bildkompression und Ressourcenallokation in Edge-Netzwerken zu optimieren. Im Gegensatz zu traditionellen bitzentrierten Kommunikationssystemen priorisiert die semantische Kommunikation die Übertragung von bedeutungsvollen Daten, die speziell ausgewählt werden, um den Inhalt zu vermitteln, anstatt eine getreue Darstellung der Originaldaten zu erreichen. Die Kommunikationsinfrastruktur profitiert von dem Fokus auf relevante Datenanteile durch signifikante Verbesserungen in der Bandbreiteneffizienz und Reduzierung der Latenz.

Zentraler Bestandteil dieser Arbeit ist die Entwicklung semantik-erhaltender Bildkompressionsalgorithmen unter Verwendung fortschrittlicher generativer Modelle wie Generative Adversarial Networks und Denoising Diffusion Probabilistic Models. Diese Algorithmen komprimieren Bilder, indem sie nur semantisch relevante Merkmale kodieren und die generative Leistung auf der Empfängerseite nutzen. Dies ermöglicht die genaue Rekonstruktion von qualitativ hochwertigen Bildern bei minimaler Datenübertragung. Die Arbeit stellt außerdem ein zielorientiertes Optimierungsframework für Edge-Netzwerke vor, das auf dem Information-Bottleneck-Problem und stochastischer Optimierung basiert und sicherstellt, dass Kommunikationsressourcen dynamisch zugewiesen werden, um Effizienz und Aufgabenleistung zu maximieren.

Durch die Integration semantischer Kommunikation in Edge-Netzwerke erreicht das vorgeschlagene System ein Gleichgewicht zwischen rechnerischer Effizienz und Kommunikationseffektivität, was es besonders für Echtzeitanwendungen geeignet macht. Die Arbeit vergleicht die Leistung dieser semantischen Kommunikationsmodelle mit herkömmlichen Bildkompressionstechniken, wobei sowohl klassische als auch semantikbewusste Bewertungsmetriken verwendet werden. Die Ergebnisse zeigen das Potenzial der Kombination von generativer KI und semantischer Kommunikation, um effizientere, semantisch zielorientierte Kommunikationsnetzwerke zu schaffen, die den Anforderungen moderner, datengetriebener Anwendungen gerecht werden.

\thispagestyle{empty}
\mbox{}