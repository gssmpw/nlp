\thispagestyle{plain}			% Supress header 
\setlength{\parskip}{0pt plus 1.0pt}
\section*{Abstract}
As digital technologies continue to advance, modern communication networks face unprecedented challenges in handling the vast amounts of data produced daily by connected intelligent devices. Autonomous vehicles, smart sensors, IoT systems etc., are gaining more and more interest and new communication paradigms are needed. This thesis addresses these challenges by combining semantic communication with generative models to optimize image compression and resource allocation in edge networks. Unlike traditional bit-centric communication systems, semantic communication prioritizes the transmission of meaningful data specifically selected to convey the meaning rather than obtain a faithful representation of the original data. The communication infrastructure can benefit of the focus solely on the relevant parts of the data due to significant improvements in  bandwidth efficiency and latency reduction.

Central to this work is the design of semantic-preserving image compression algorithms, utilizing advanced generative models such as Generative Adversarial Networks and Denoising Diffusion Probabilistic Models. These algorithms compress images by encoding only semantically relevant features and exploiting the generative power at the receiver side. This allows for the accurate reconstruction of high-quality images with minimal data transmission. The thesis also introduces a Goal-Oriented edge network optimization framework based on the Information Bottleneck problem and stochastic optimization, ensuring that communication resources are dynamically allocated to maximize efficiency and task performance.

By integrating semantic communication into edge networks, the proposed system achieves a balance between computational efficiency and communication effectiveness, making it particularly suited for real-time applications. The thesis compares the performance of these semantic communication models with conventional image compression techniques, using both classical and semantic-aware evaluation metrics. The results demonstrate the potential of combining generative AI and semantic communication to create more efficient semantic-goal-oriented communication networks that meet the demands of modern data-driven applications.


% The exponential growth of digital technology has revolutionized information generation, transmission, and consumption, particularly within the realms of the Internet of Things (IoT), autonomous vehicles, and smart cities. Traditional communication systems, rooted in Shannon's information theory, primarily emphasize the accurate transmission of raw data without accounting for the contextual significance or relevance of the information. This approach leads to inefficiencies, especially as the volume of data continues to surge. Addressing this challenge, this thesis explores the paradigms of \gls{sc} and \gls{goc}, which prioritize the transmission of meaningful and contextually relevant information to enhance communication efficiency and effectiveness.

% Central to this exploration is the integration of advanced generative models, including Generative Adversarial Networks (GANs), Variational Autoencoders (VAEs), and Diffusion Models (DDPMs), within semantic communication frameworks. The research introduces several novel models: the Semantic-Preserving Image Coding (SPIC) framework employs diffusion models to achieve high-quality image reconstruction from compressed semantic representations; the Class-Specific SPIC (CSPIC) enhances the reconstruction of detailed objects without necessitating additional fine-tuning; and the Semantic-Aware GAN (SQ-GAN) utilizes masked vector quantization to selectively encode semantically relevant regions, significantly reducing redundancy and improving semantic preservation at ultra-low bit rates.

% Furthermore, the thesis delves into goal-oriented resource allocation within edge networks, leveraging the Information Bottleneck (IB) principle and stochastic optimization techniques to dynamically balance compression efficiency and semantic integrity based on real-time network conditions. Comparative analyses demonstrate that the proposed models outperform classical compression algorithms such as BPG and JPEG2000 in terms of both traditional metrics and semantic relevance, underscoring their potential in next-generation communication systems. By marrying semantic understanding with intelligent resource management, this research paves the way for more efficient and meaningful data transmission in increasingly data-intensive environments.

% KEYWORDS (MAXIMUM 10 WORDS)
\vfill
Keywords: Semantic Communication, Generative AI, Goal-Oriented Communication, Edge Network Optimization, Image Compression.

\thispagestyle{empty}
\mbox{}