\chapter{\textcolor{black}{Conclusion}}
\label{ch: Conclusion}
\thispagestyle{plain}

In this thesis, the potential of \gls{sc} and \gls{goc} paradigms within modern digital networks has been explored and exploited. The rapid proliferation of data driven technologies such as the \gls{iot}, autonomous vehicles and smart cities has underscored the limitations of traditional bit-centric communication systems. These systems, grounded in Shannon's information theory, focus primarily on the accurate transmission of raw data without considering the contextual significance of the information being conveyed. This fundamental mismatch between data production and communication infrastructure capabilities has necessitated the exploration of more efficient and intelligent communication frameworks.

\cref{ch: SEMCOM} discussed how the core of this thesis focused on integrating of \gls{sc} principles with generative models and their potential applications in the context of edge computing. By focusing on the conveyance of relevant meaning rather than exact data reproduction, \gls{sc} reduces unnecessary bandwidth consumption and inefficiencies. In all those cases where it is possible and reasonable to discuss the semantics, then the faithful representation of the original data is unnecessary as long as the meaning has been conveyed. This paradigm also aligns with \gls{goc}, where the transmitted data is tailored to meet specific objectives, further reducing the communication overhead. The goal of the communication can either be the classical syntactic data transmission or the semantic preservation of the data. By focusing on the goal of the communication, it is possible to transmit only the most pertinent information, thereby reducing the load in communication networks and optimizing resource utilization.

In \cref{ch: SPIC}, the \gls{spic} framework was introduced as a novel method for semantic-aware image compression. The framework demonstrated the potential for high-fidelity image reconstruction from compressed semantic representations. The proposed modular transmitter-receiver architecture is based on a doubly conditioned \gls{ddpm} model, the \gls{semcore}, specifically designed to perform \gls{sr} under the conditioning of the \gls{ssm}. By doing so the reconstructed images preserve their semantic features at a fraction of the \gls{bpp} compared to classical methods such as \gls{bpg} and \gls{jpeg2000}.

Furthermore, the enhancement introduced by \gls{cspic} addressed a critical aspect in image reconstruction: the accurate representation of small and detailed objects. Without requiring extensive retraining of the underlying \gls{semcore} model, \gls{cspic} improved the preservation of important semantic classes, such as traffic signs.  The modular design at the core of the \gls{spic} and \gls{cspic} showcased the flexibility and adaptability of the system in different contexts.

The integration of \gls{sc} principles continued in \cref{ch: SQGAN}, where the \gls{sqgan} model was proposed. This architecture employed vector quantization in tandem with a semantic-aware masking mechanism, enabling selective transmission of semantically important regions of the image and the \gls{ssm}. By prioritizing critical semantic classes and utilizing techniques such as Semantic Relevant Classes Enhancement or the Semantic-Aware discriminator, the model excelled at maintaining high reconstruction quality even at very low bit rates, further emphasizing the efficiency gains of the proposed approach.

Finally, in \cref{ch: Goal_oriented}, the thesis was extended to include the \gls{goc} for resource allocation in \glspl{en}. By adopting the \gls{ib} principle to perform \gls{goc} was developed a framework to dynamically adjust compression and transmission parameters based on network conditions and resource constraints. This dynamic adaptation was crucial in balancing compression efficiency with semantic preservation, optimizing the use of computational and communication resources in edge networks.

By leveraging the \gls{sqgan} within the \gls{en}, the research demonstrated the synergy between \gls{sc} and \gls{goc}. Real-time network conditions informed adjustments to the masking process, enabling the edge network to operate autonomously and efficiently. This approach validated the potential of \gls{sgoc} to enhance resource utilization in modern network infrastructures.



% In this thesis, the potential of \gls{sc} and \gls{goc} paradigms within modern digital networks has been explored and exploited. The rapid proliferation of data driven technologies such as the \gls{iot}, autonomous vehicles, and smart cities has underscored the limitations of traditional bit-centric communication systems. These systems, grounded in Shannon's information theory, focus primarily on the accurate transmission of raw data without considering the contextual significance of the information being conveyed. This fundamental mismatch between data production and communication infrastructure capabilities has necessitated the exploration of more efficient and intelligent communication frameworks.

% As explained in \cref{ch: SEMCOM} at the core of this thesis lies the integration of \gls{sc} principles with generative models, particularly within the context of edge computing. \gls{sc}, which emphasizes the conveyance of meaning rather than mere symbol reconstruction, offers a pathway to significantly reduce bandwidth usage and enhance the efficiency of data transmission. This approach aligns seamlessly with the objectives of \gls{goc}, which prioritizes the transmission of information that is directly relevant to achieving specific goals. By focusing on the semantic content of the data, it becomes possible to transmit only the most pertinent information, thereby reducing the load in  communication networks and optimizing resource utilization.

% In \cref{ch: SPIC} the development and implementation of the \gls{spic} framework marked a significant stride in bridging \gls{sc} with practical image compression techniques. By leveraging diffusion models, \glspl{ddpm} were employed to reconstruct high-resolution images from compressed semantic representations. This modular approach, consisting of a transmitter and receiver architecture, facilitated the efficient encoding and decoding of both the low-resolution original image and the associated \gls{ssm}. The \gls{spic} framework demonstrated the capability to maintain high levels of semantic preservation while achieving substantial compression rates, thereby showcasing its potential as a viable alternative to classical image compression algorithms such as \gls{bpg} and \gls{jpeg2000}.

% Building upon the foundational work of \gls{spic}, the introduction of the \gls{cspic} further refined the approach by addressing the reconstruction of small and detailed objects within images. This enhancement was achieved without necessitating additional fine-tuning or retraining of the underlying \gls{semcore} model, thereby exploiting the framework's modularity and flexibility. The \gls{cspic} model underscored the importance of preserving critical semantic classes, ensuring that essential details (i.e. "traffic signs") are preserved. These level of semantic preservation was evaluated by the Traffic signs classification accuracy presented in \sref{sec: GM evaluation metrics}.

% In \cref{ch: SQGAN} the \gls{sqgan} model represented a novel integration of vector quantization and \gls{sc} principles. The \gls{sqgan} architecture incorporated a \gls{samm} to selectively transmit semantically relevant regions of the data. This selective encoding process significantly reduced redundancy and enhanced communication efficiency, particularly at extremely low \gls{bpp} values. The introduction of the \gls{samm} and the \gls{spe} facilitated the prioritization of latent vectors associated with critical semantic classes, thereby improving the overall reconstruction quality of important objects within images. Additionally, the designed Semantic Relevant Classes Enhancement data augmentation technique and the Semantic Aware Discriminator further refined the model's ability to preserve critical semantic information.

% In \cref{ch: Goal_oriented} asignificant contribution of this research was the exploration of goal-oriented resource allocation within \glspl{en}. By leveraging the \gls{ib} principle, the thesis addressed the challenge of dynamically adjusting compression parameters to balance the trade-off between compression efficiency and semantic preservation. The application of stochastic optimization techniques facilitated the optimal allocation of computational and communication resources, ensuring that the \gls{en} operates efficiently under varying network conditions and resource constraints. This integration of \gls{goc} principles with resource optimization strategies underscored the importance of adaptive and intelligent network management in modern communication infrastructures.

% Additionally, by employing the \gls{sqgan} model within the \gls{en} framework, the research demonstrated the potential of \gls{sgoc}. The integration of the \gls{sqgan} model within the \gls{en} architecture enabled the dynamic adjustment of the masking fractions based on real-time network conditions and resource availability. This approach ensured that the \gls{en} could autonomously optimize its operations, thereby enhancing communication efficiency and resource utilization in a goal-oriented fashion with focus on \gls{sc}.

% Throughout the research, the importance of modular and flexible framework design was emphasized. The proposed models, \gls{spic}, \gls{cspic}, and \gls{sqgan}, were designed to be easily integrated into existing communication systems without the need for extensive modifications. This design philosophy ensures that the advancements in \gls{sc} can be readily adopted in practical applications, facilitating the transition from traditional to intelligent communication paradigms.

% The comparative analysis of the proposed models against classical compression algorithms highlighted the superiority of semantic-aware approaches in preserving critical information at low bit rates. While traditional algorithms excel in minimizing pixel-level distortions, they fall short in maintaining the semantic integrity of the data. In contrast, the proposed semantic and goal-oriented models demonstrated enhanced performance in preserving meaningful content, thereby offering a more effective solution for applications where semantic accuracy is crucial.

