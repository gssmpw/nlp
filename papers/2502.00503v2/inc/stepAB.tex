% file: stepAB.tex
%	Taken from estimateR.tex (stepA and stepB)
%	Taken from case-negative.tex (delta-bounde end-enclosure)

	
%%%%%%%%%%%%%%%%%%%%%%%%%%%%%%%%%%%%%%%%%%%%%%%%%%
\sectL[stepAB]{Steps A and B}
%%%%%%%%%%%%%%%%%%%%%%%%%%%%%%%%%%%%%%%%%%%%%%%%%%
\ssect{Step A}
	Recall that $\stepA(E_0,\veps,H)$ returns an
	$\veps$-admissible pair (see \refeq{stepA}) for $E_0$.
	For this purpose, we need the following:

	\bleml[ad]\ \\
	Let $ k \geq 1 $, $ H > 0 $, $ \veps > 0 $, and 
	$ E_0 \ib \RR^n $. Define  
	\[
		\ol{B} \as \sum_{i=0}^{k-1} [0, H]^i \bff^{[i]}(E_0)
				+ 	[-\veps, \veps]^n, \qquad
		M \as \sup_{\bfp \in \ol{B}} \| \bff^{[k]}(\bfp) \|_2.
	\]  
	If
	\[
		h = \min \left\{ H, \left( \frac{\veps}{M} 	\right)^{\frac{1}{k}}
				\right\},  \qquad
		F_1 = \sum_{i=0}^{k-1} [0, h]^i \bff^{[i]}(E_0) + 
			[-\veps, \veps]^n.
	\]  
	then the pair $(h, F_1)$ is $\veps$-admissible for $E_0$.
	\eleml
	\savespace{
	\bpf
	To verify \refeq{tay},
	 we only need to verify
		\[[0,h]^k\bff^{[k]}(F_1)\ib [-\veps,\veps]^n.\]
	We have:
	\beqarrys
		h^k\bff^{[k]}(F_1)
		    &\ib& h^k[-M,M]^n
		          &\text{(by the definition of $M$  )}\\
			&\ib& [-\veps,\veps]^n.
		           &\text{(by the definition of $h$  )}
	\eeqarrys
	\epf
	}% savespace
	
	\refLem{ad} justifies the following implementation of \stepA:
	
	\Ldent\progb{
	\lline[-3] \stepA$(  E_0,\veps,H)$$\ssa(h,F_1)$
	\lline[7] INPUT: $\bff$, $H>0, \veps>0, k\ge 1$ and $E_0\ib\RR^n$
	\lline[7] OUTPUT:  an $\veps$-admissible pair for
			$E_0$ such that $h\le H$.
	\lline[2] $\ol{B}_0\ass\sum_{i=0}^{k-1}[0,H]^i\bff^{[i]}(E_0)
					+[-\veps,\veps]^n$.
	\lline[2] $M\ass \|\bff^{[k]}(\ol{B}_0)\|_2$.
	\lline[2] $h\ass\min\{(\frac{\veps}{M})^{\frac{1}{k}},H\}$.

	\lline[2]  $F_1\ass \sum_{i=0}^{k-1}[0,h]^i \bff^{[i]}(E_0)
				+[-\veps,\veps]^n.$
	\lline[2] Return $(h,F_1)$
	}
	
	
%%%%%%%%%%%%%%%%%%%%%%%%%%%%%%%%%%%%%%%%%%%%%%%%%%
\ssect{Step B}
%%%%%%%%%%%%%%%%%%%%%%%%%%%%%%%%%%%%%%%%%%%%%%%%%%
	For \stepb, there are several methods such as
	the Direct Method \cite{nedialkov+2:validated-ode:99},
	Lohner's method \cite{lohner:thesis},
	and $C^1$-Lohner method \cite{wilczak-zgliczynski:lohner:11}.
	Our implementation uses the Direct Method:

	\beql{stepBdirect}
		\includegraphics[width=0.86\linewidth]{figs/stepBdirect}
	\eeql
	
%%%%%%%%%%%%%%%%%%%%%%%%%%%%%%%%%%%%%%%%%%%%%%%%%%
\ssect{$\delta$-bounded End-enclosure}\label{sec-ne}
	In this section, let $(E_0, h, F_1)$ be an admissible triple.
	For any $\bfx \in \IVP(E_0, h)$ and
	$\delta>0$, the \dt{$\delta$-tube} of $\bfx$ is
	the set
		$$\Tube_\delta(\bfx) \as
				\set{(t,\bfp): \|\bfp-\bfx(t)\|_2 \le \delta, 0\le t\le h}
				\quad \big(\ib [0,h]\times\RR^n\big)$$
	We say that a function $\ell: [0,h]\to\RR^n$ belongs to the
	$\delta$-tube of $\bfx$ is for all $t\in[0,h]$,
	$(t,\ell(t)) \in \Tube_\delta(\bfx)$,see graph \ref{fig:tube} for
	illustration.
	
	% TODO: \usepackage{graphicx} required
	\begin{figure}
		\centering
		\includegraphics[width=0.4\linewidth]{figs/Tube}
		\caption{The dashed lines in the figure form a \(\delta\)-tube
		around the red solid curve representing \(\bfx(t)\). The segment
		\(l(t)\) is a line segment inside this \(\delta\)-tube.}
		\label{fig:tube}
	\end{figure}
	
	\bleml[delta-distance]\ \\
		Consider 
			$\ivp(Ball(\bfp_0,r_0),h,F_1)$
		and $\olmu = \mu_2(J_\bff(F_1))$.  
		Let $\bfx_c \in\ivp(\bfp_0,h)$
			and $\bfq_0=\bfx_c(h)$.
		If the linear function
			$$\ell(t)\as (1-t/h)\bfp_0+(t/h)\bfq_0$$
		lies in the $\delta$-tube of $\bfx_c$ then
		\benum[(i)]
		\item  $Ball(\bfq_0, r_0 e^{\olmu h}+\delta)$ is an
				end-enclosure for $\ivp(Ball(\bfp_0,r_0),h)$.
		\item  Let $r'=\max(r_0 e^{\olmu h},r_0)$ then
			$Box(Ball(\bfp_0, r'+\delta),
					Ball(\bfq_0, r'+\delta))$
				is a full-enclosure for $\ivp(Ball(\bfp_0,r_0),h)$.
		\eenum
	\eleml
	\savespace{
	\bpf
	By \refLem{endenclosure}, we have
		$E_1 =  Ball(\bfq_0, r_0e^{\olmu h}+\delta)$ is the
	end-enclosure for $\ivp(E_0, h)$.  
	
	Next, we prove $(ii)$.  
	We show that for any $T \in [0, h]$,  the end-enclosure of 
	$\ivp(E_0,T)$ is a subset of $Box(Ball(\bfp_0, r' +\delta),
	Ball(\bfq_0, r' +\delta))$.
	Note that by \refLem{endenclosure}, we have $E_1 =  Ball(l(T),
	r_0e^{\olmu T}+\delta)$ is the
	end-enclosure for $\ivp(E_0, T)$.
	
	Let $l(T)_i$ denote the $i$-th component of $l(T)$ and $r(T)\as
	r_0e^{\olmu T}+\delta$. Then, we only need to
	prove that for any $i=1\dd n$, the interval
		$l(T)_i \pm r(T)$ satisfies  
	\[
	l(T)_i \pm r(T) \subseteq Box((\bfp_0)_i\pm (r'+\delta),
	(\bfq_0)_i\pm (r'+\delta)),
	\]  
	where $(\bfp_0)_i $ and $(\bfq_0)_i $ are the $i$-th components of
	$\bfp_0$ and $\bfq_0$,
	respectively.  
	
	Since $l(T)$ is a line segment, it follows that  
	\[
		\min((\bfq_0)_i, (\bfp_0)_i)
			\leq l(T)_i \leq \max((\bfq_0)_i, (\bfp_0)_i).
	\]  
	Additionally, we have $r(T) \leq r' + \delta$.  
	
	Combining these observations, we conclude that 
	\[
	l(T)_i \pm r(T) \subseteq Box((\bfp_0)_i\pm (r'+\delta),
	(\bfq_0)_i\pm (r'+\delta)).
	\] 

	\epf
	}% savespace
	
	\savespace{
	% TODO: \usepackage{graphicx} required
	\begin{figure}[h]
		\centering
		\includegraphics[width=0.7\linewidth]{figs/delta-distance}
	
		\caption{The red curve $\bfx_c(t)$ represents the solution to  
			$\ivp(\bfp_0, h)$. The blue curve $l(t)$ lies within the
			$\delta$-tube of $\bfx_c$. According to
			\refLem{delta-distance}, the ball $B = Ball(l(h), r_0e^{\olmu
			h} + \delta)$ provides an end-enclosure for
			$\ivp(Ball(\bfp_0, r_0), h)$, as shown in the left graph. In
			the right graph, the green box, $Box(Ball(\bfp_0, r' +
			\delta), Ball(\bfq_0, r' + \delta))$, projected onto the
			vertical axis, represents the full-enclosure.}  

		\label{fig:delta-distance}
	\end{figure}
	}% savespace
	
	\savespace{
	Next, we show that the Euler method with a step size
	$0<h_1=h_1(\delta)\le h$ will be able to compute a series of  points
	$P=(\bfp_0\dd \bfp_{N+1})$ such that for any $i=1\dd N+1$ 
	the line segment $\ol{\bfp_{i-1}\bfp_i}$ lies in the $\delta$-tube of
	$\bfx_c$.
	
	The classic Euler method for $\ivp(\bfp_0, h)$
	produces a polygonal path of the form
			$$[(0,\bfp_0), (t_1,\bfp_1) \dd (t_N,\bfp_N), (h,\bfp_{N+1})]$$
	of length $N+1$.  In case the time step is uniformly $h_1$,
	then $t_i = i h_1$ for $i=0\dd N\as \floor{h/h_1}$ and
	we can represent the path\footnote{
		Allowing the degenerate case where $t_N=h$.
	} as
			$P=[\bfp_0, \bfp_1 \dd \bfp_N, \bfp_{N+1}]$.

	We now consider a variant of this classical subroutine
	to compute a polygonal approximation $P$ to 
			\beql{bfxp0}
				\bfx =\ivp_\bff(\bfp_0,h,F_1)\eeql
	so that $P$ lies in the $\delta$-tube of $\bfx$,
	for any user chosen $\delta>0$. 
	
	\Ldent\progb{
		\lline[0]  ${\sc Euler}_\bff(\bfp_0,h,\delta)\ssa (h_1,P)$
		\lline[5] INPUT: $\bff, \bfp_0, h, F_1, \delta$ as described
			above subject to \refeq{lognormneg}.
		\lline[5] OUTPUT:  a polygonal path $P$ with step size $h_1$
			from time $0$ to $h$.  
		\lline[25]	Morever, $P$ lies in the
				$\delta$-tube of $\bfx$ in \refeq{bfxp0}.
		\lline
		\lline[10] $h_1 \ass
				\min\set{h,
					\frac{2\olmu\delta}{\|\bff^{[2]}(F_1)\|
						(e^{\olmu h}-1)}}$.
		\lline[10] $N\ass \floor{h/h_1}$
		\lline[10] $\wtbfy\ass \bfp_0$
		\lline[10] Initialize a list $P\ass [\bfp_0]$
		\lline[10] For $i=1$ to $N$
		\lline[15] 		$\wtbfy\ass \wtbfy + h_1\cdot \bff(\wtbfy)$
		\lline[15]		$P.append(\wtbfy)$
		\lline[10] $P.append(\wtbfy + (h-Nh_1)\cdot \bff(\wtbfy))$ 
		\lline[10] Return $(h_1, P)$
	}

	\bthmT[negative]{Correctness of {\sc Euler}}\ \\
		The algorithm ${\sc Euler}_\bff(\bfp_0,h,\delta)$
		is correct, i.e.,
		the output $P=[\bfp_0\dd \bfp_{N+1}]$
		lies within the $\delta$-tube of the solution
		$\bfx$ in \refeq{bfxp0}.
	\ethmT
	\bpf
		 Let 
		 	$t_i=
		 \begin{cases}
		 	ih_1, & i<N+1\\
		 	h,    & i=N+1
		 \end{cases}.$
		 
		 For any  \(i = 1 \dd N+1\) and  $t\in [t_{i-1},t_{i}]$,
		 let $\bfq=\bfp_i+(t-t_i)\bff(\bfp_i)$.
		 Then the theorem follows if we prove
		 	$\|\bfq - \bfx(t)\| \le \delta $.
		By \refLem{Euler-step},
		\beqarrys
		\|\bfq - \bfx(t)\|  
		&\leq& 
			\frac{\|\bff^{[2]}(F_1)\|h_1}{2\olmu}(e^{\olmu_1 t} - 1)  
		 \\  
		&\leq& \frac{(e^{\olmu_1 t} - 1)\delta}
			{e^{\olmu_1 	h} - 1} 
				& \text{(by the definition of \(h_1\))}\\  
		&\leq& \delta  
			& \text{(since \(t \le h\))}
		\eeqarrys
			
	\epf
	}%savespace
	





%%%%%%%%%%%%%%%%%%%%%%%%%%%%%%%%%%%%%%%%%%%%%%
% THIS IS THE OLD METHOD, using ball geometry.
%%%%%%%%%%%%%%%%%%%%%%%%%%%%%%%%%%%%%%%%%%%%%%
%%%%%%%%%%%%%%%%%%%%%%%%%%%%%%%%%%%%%%%%%%%%%%%%%%
\ignore{
\sect{Computing an Admissible Pair $(h,R)$ }
%%%%%%%%%%%%%%%%%%%%%%%%%%%%%%%%%%%%%%%%%%%%%%%%%%
	Let
			$$M(\bfp_0,r) \as
				\sup_{\bfp\in Box(\bfp_0,r)} \| \bff(\bfp)\|_2.$$
	\cheeX{We defined $Box(\bfp_0,r)$ with $2$-norm.}
	%%%
	If $r=0$, let $M(\bfp_0,0)$ denote $\| \bff(\bfp_0\|_2$.
	Theorem M tells us that if the pair ($h,r$)
	satisfies the condition $h\le r/M(\bfp_0,r)$,
	then the set $\IVP(\bfp_0,h)$ is a singleton, and the true range of
	the unique solution in $\IVP(\bfp_0,h)$
	is contained in $Ball(\bfp_0,r)$.
	\ignore{
			Bw said tht $\bfx$ is unique here.
			But which is the unique solution in the example
				$x'=f(x)=\sqrt{|x|}$
			and
				$x(0)=0$?
			Then $IVP_f(0,h)=\set{x_a: a\in (0,h]}$
			and 
				$$x_a(t)=\clauses{0 &\rmif\ t\le a,\\
								(x-a)^2 & \rmif\ t>a.}$$
			\cored{Bingwei says $f(x)$ is not Lipshitz continuous
			at $x=0$)}
	}
	%%
	We want to generalize this result to say that for all
			$\bfx\in \IVP(Ball(\bfp_0,r_0),h)$
	the true range of $\bfx$ is contained
	in $Ball(\bfp_0,R)$ for some $R$.
	For this purpose, let us define the function
		\beql{H}
			H(\bfp_0,R) \as \frac{R-r_0}{M(\bfp_0,R)}
		\eeql
	and a pair of reals
			$(h, R)$
	is called an \bf{admissible pair for $Ball(\bfp_0,r_0)$}
	if $h>0$, $R>r_0$ and $h\le H(\bfp_0,R)$.
	We have the following lemma:

	\blem
	Let $(h, R)$ be an admissible pair for $Ball(\bfp_0,r_0)$.
	Then for all 
		$\bfx \in \IVP(Ball(\bfp_0, r_0), h)$
	the true range of $\bfx$ is contained in $Ball(\bfp_0, R)$.
	\elem
	\bpf
	For any $\bfq \in Ball(\bfp_0, r_0)$, let $r = R - r_0$ and
	$h_{\bfq} = \frac{r}{M(\bfq, r)}$. Then, by Theorem M,
	we have
	$\bfx_{\bfq} =\IVP(\bfq, h_{\bfq})$
	and the true range of $\bfx_{\bfq}$ is contained in
				$Ball(\bfq, r))$.
		\beqarrys
			h_{\bfq} &=& \frac{R - r_0}{M(\bfq, R-r_0)} \\
					&\ge& \frac{R - r_0}{M(\bfp_0, R)}
				& \text{(since $Ball(\bfq, r) \ib Ball(\bfp_0, R)$}\\
					&\ge& h 
				& \text{(by admissibility of $(h,R)$).}
		\eeqarrys
	Since $h_\bfq\ge h$, it is clear that there exists a
	solution $\bfx\in IVP(Ball(\bfp_0,r_0),h)$ such that
	$\bfx$ is obtained by restricting the domain of $\bfx_\bfq$
	from $[0,h_\bfq]$ to $[0,h]$.  Since the true range of $\bfx_\bfq$
	is contained in $Ball(\bfp_0,R)$, we conclude that the
	true range of $\bfx$ is also contained in $Ball(\bfp_0,R)$.
	\epf
	
%	We now consider the case where $\bff$ are polynomials.
%	Let $\deg(\bff)$ be the maximum total degree of any $f_i\in\bff$.
%	If $\bff$ is linear, i.e., $\deg(\bff)=1$, the validated
%	IVP problem can be considered solved (or solvable).
%	In the nonlinear case, we can use the following lemma:
%	
%	\blem Let $\bff=(f_1\dd f_n)$ be a polynomial system with
%		$\deg(\bff)=d\ge 2$.
%		Then the function $M(\bfp_0,r)=\Theta(r^{d})$.
%	\elem
%	\bpf
%		Clearly,
%			$M(\bfp_0,r) = \|\bff(\bfq)\|$ for some
%				$\bfq\in Ball(\bfp_0,r)$.
%		(In this proof, $\|\cdot\|$ refers to the $2$-norm.)
%		Write $\|\bff(\bfq)\| = \sqrt{F(\bfq-\bfp_0)}$
%		where
%			$$F(\bfx)\as \sum_{i=1}^n f_i(\bfp_0+\bfx)^2.$$
%		Since $F(\bfx)$ has total degree $2d$, we may rewrite it as
%				$F(\bfx)= \sum_{i=0}^{2d} F^i(\bfx)$
%		where each $F^i$ is a homogeneous polynomial of
%		total degree $i$, and $F^{2d}(\bfx)$ does not vanish.
%
%		Consider the unit ball $Ball(\0,1)$ centered at the origin $\0$.
%		Let $S^{n-1}=\partial Ball(\0,1)$ denote the boundary of
%		the unit ball.  For any $\bfv\in Ball(\0,1)$, 
%		it follows from the homogeneity of $F^i$ that for all $r>0$,
%				$F^i(r \bfv)= r^i F^i(\bfv)$.  For
%				$\bfv\in S^{n-1}$, let 
%			$$g_{\bfv}(r) \as F(r\bfv)= \sum_{i=0}^{2d} r^i F^i(\bfv)$$
%		is a polynomial of degree $\le 2d$. 
%
%		\dt{Lower bound on $M(\bfp_0,r)$:}
%		let
%			$$\bfv^*=\argmax_{\|\bfv\|=1}  \|F^{2d}(\bfv)\|.$$
%		Therefore the leading coefficient of
%			$g_{\bfv^*}(r)$ is $F^{2d}(\bfv^*)$ which is non-zero.
%			Hence $g_{\bfv^*}(r)$ is a polynomial of degree $2d$.
%			It follows that
%		$M(\bfp_0,r)\ge \sqrt{g_{\bfv^*}(r)}=\Omega(r^d)$.
%
%		\dt{Upper bound on $M(\bfp_0,r)$:}
%		\chee{Bingwei, please complete this argument}
%		Let $a_i \as \sup\set{| F^i(\bfv)|: \bfv\in S^{n-1}}$
%		for $i=0\dd 2d$.  Note that $a_{2d}\ne 0$ since $F^{2d}$ does not
%		vanish.  Let 
%			$$g_*(r) \as \sum_{i=0}^{2d} a_i r^i.$$
%		Then 
%			\beqarrys
%			M(\bfp_0,r) &=& \sup\set{ \|\bff(\bfp_0+r\bfv)\|:
%								\bfv\in Ball(\0,1)}\\
%					&=& \sup\set{\sqrt{F(r\bfv)}:
%								\bfv\in Ball(\0,1)}\\
%					&\le& \sup\set{\sqrt{g_{\bfv}(s)}:
%								\bfv\in S^{n-1}, 0<s\le r}\\
%					&\le & \sqrt{g_*(r)}
%								& \text{(???)}\\
%			\eeqarrys
%			
%	\epf	
	Given $\olh$, $\bff$, and $Ball(\bfx_0, r_0)$,
	we aim to compute an admissible pair $(h, r)$
	such that $h \le \olh$.
%	We use the unbounded search method to get a lower bound
%	of a local maximum of $\frac{r - r_0}{M(\bfp_0, r)}$.
		FIRST, I describe an "ideal" algorithms where $h(r,r_0,\bfp_0)$ is
	computed exactly.
	We allow users to input $ R>r_0 $ to meet the required control precision.
	\Ldent\progb{
			\lline[0] \estimateR$(\olh, \bfp_0, r_0, R)$
			\lline[10]		INPUT: $\olh>0, r_0>0, \bfp_0\in\RR^n, R$
			\lline[10]		OUTPUT: $(h,r)$, an admissible pair for
								$Ball(\bfp_0,r_0)$ such that $h\le \olh$.
			\lline[5]	If $ R $ is provided as input, then
					$ r_0 \ass R $; otherwise, $r\ass 2r_0$.
			\lline[5]	$h\ass h(r, Ball(\bfp_0,r_0))$
			\lline[5]	If ($h>\olh$)
			\lline[10]		$h\ass \olh$
			\lline[5]	Return $(h,r)$
		}%progb
	\blem
		Let $h(r)= h(r,Ball(\bfp_0,r_0))$ and assume that
		$h(r)\to 0$ as $r\to\infty$. 
		Then the algorithm \estimateR\ is correct.
	\elem
	\bpf
	The termination of this algorithm is obviously. 
	 The output $(h,r)$ is clearly
		correct by the previous lemma.
	%	Let us write $h(r)$ for $h(r,Ball(\bfp_0,r_0))$.
	%	Suppose the does not terminate.
	%	That means that the sequence $h(2r_0), h(4r_0)\dd h(2^i r_0) \dd$
	%	is bounded below by $\olh$ for all $i\ge 1$.  This
	%	contradicts the previous lemma 
	\epf
}%ignore
	
	
%	\blem
%	If $h(r)$ in \refeq{h(r)} satisfies $\lim_{r\to\infty} h(r)= 0$
%	then Algorithm \ref{alg-initialenclosure} is correct.
%	\elem
%	\bpf
%	We first prove the termination of the algorithm. 
%	We only have to show that the while loop terminates.
%	If the while loop cannot terminate, then $ h(nr_0) $ must increase
%	for positive integers $ n $. However, since $\lim_{r \to \infty}
%	h(r) = 0$, there must exist an integer $ N > 0 $ such that for all
%	$ n > N $, $ h(nr_0) < h(2r_0) $. Thus, the while loop can
%	terminate.
%	
%	Next, we prove the correctness of the algorithm's output. We can see
%	that the algorithm only returns in the third step; therefore, if
%	$K_2\le K_1$	 then  $K_1= h(\frac{r}{2}) $ so $(K_1,\frac{r}{2})$
%	is an admissible pair.
%	If $K_2>K_1$, that means $K_2=h(r)>\olh$, then $(\olh,r)$ is an
%	admissible pair.  
%	
%	
%	
%	\epf
	
%	Here are some of our examples and data, where $h$ and $r $ are the
%	outputs of our algorithm \ref{alg-initialenclosure}, and $h_1, h_2$,
%	$h_3$ represent the step sizes obtained by Nedialko's method for
%	orders $1, 2,$ and $3$, respectively.
%
%%%%%%%%%%%%%%%%%%%%%%%%%%%%%%%%%%%%%%%%%%%%%%%
%\ssect{Some Data from Bw}
%	\cored{
%		Here are some of our examples and data, where $h$ and $r $ are
%		the outputs of our algorithm \ref{alg-initialenclosure}, and
%		$h_1, h_2$, $h_3$ represent the step sizes obtained by Nedialko's
%		method for orders $1, 2,$ and $3$, respectively.
%	}
%
%	\btableL[l | c | c c | r | r |r|r|r| l]{admiss}{Admissible Pairs}
%	{ Example &
%		Parameters & \multicolumn{2}{c|}{Initial Values}
%			& \multicolumn{2}{c|}{Admissible Pair}
%			& \multicolumn{3}{c|}{Nedialkov Estimates}
%			& Notes/Refs.\\
%		& $(a,b,c,d)$ & $(x_0,y_0)$ & $r_0$ & $h$ & $r$
%			& $h_1$ &$h_2$ & $h_3$
%			& Notes/Refs.\\\hline\hline
%	 E1 Predator-Prey 
%		& $(1,1,1,1)$ & $(1,1)$ & $1$ & $0.075$ & $4.000$
%			& $0.009$ & $0.033$ & $0.045$
%			& $(x_0,y_0)=(1,1)$ is a node.\\
%		& $(1.17,0.9,2.1,1.2)$ & $(1,1)$ & $1$ & $0.065$ & $4.000$
%			&$0.006$ & $0.030$  & $0.037$
%			& Youtube. See \refFig{prey}(left).\\
%		& $(1,1,0.666,1.333)$ & $(0.9,0.9)$ &$1$ & $0.066$ & $4.000$
%			& $0.007$ & $0.029$ & $0.039$
%			& wikipedia. See Figure \refFig{prey}(middle).\\
%		& $(1,1,2,2)$ & $(1,2)$ & $1$ & $0.041$ & $4.000$
%			&$0.004$ & $0.020$ & $0.029$
%			& medium. See \refFig{prey}(right).
%	\\\hline
%	N1 Neumaier
%		& & $(1,-1)$ & $0.01$ & $0.079$ & $1.80\times10^{14}$
%			& $0.006$ & $0.027$ & $0.044$
%			& \\
%	N1 Neumaier
%		& & $(1,-1)$ & $0.1$ & $0.079$ & $2.25\times10^{14}$
%			& $0.006$ &$0.027$ & $0.044$
%			& \\
%	\hline
%	N2 Neumaier
%		& & $(1,-1)$ & $0.01$ & $0.024$ & $5.280$
%			&$0.0007$& $0.0034$& $0.0063$
%			& \\
%	N2 Neumaier
%		& & $(1,-1)$ & $0.1$ & $0.023$ & $6.400$
%			& $0.0009$ & $0.0028$ & $0.0050$ 
%			& \\
%	\hline
%	}


