%file: intro-new.tex
%%%%%%%%%%%%%%%%%%%%%%%%%%%%%%%%%%%%%%%%%%%%%%%%%%
\section{Introduction}
%%%%%%%%%%%%%%%%%%%%%%%%%%%%%%%%%%%%%%%%%%%%%%%%%%
	We consider the following system of
	first order ordinary differential equations (ODEs)
	\beql{bfx'}
		\bfx' = \bff(\bfx)
	\eeql
	where $\bfx=[x_1\dd x_n]\in\RR^n$
	are functions of time and $\bfx'=[x_1'\dd x_n']$
	indicate differentiation with respect to time,
	and $\bff=[f_1\dd f_n]:\RR^n\to\RR^n$.
	Our ODE is autonomous since $\bff$ does not
	depend on time, and so we can assume an initial time
	$t=0$.  Up to time scaling, we can also assume that the end time
	is $h=1$.  

	Given $\bfp_0\in\RR^n$ and $h>0$,
	the \dt{initial value problem} (IVP) for $(\bfp_0,h)$ is the
	mathematical problem of finding a \dt{solution},
	i.e.,
	a continuous function $\bfx: [0,h]\to\RR^n$
	that satisfies \refeq{bfx'}, 
	subject to $\bfx(0)=\bfp_0$.
	Let $\ivp_\bff(\bfp_0,h)$ denote the set of all such solutions.
	%%
    Since $\bff$ is usually fixed or understood,
	we normally omit $\bff$ in our notations.
	%%
	We say that $(\bfp_0,h)$ is \dt{well-defined} if
	the solution exists and is unique, i.e.,
	$\ivp(\bfp_0,h)=\set{\bfx_0}$ is a singleton. 
	In this case, we write $\bfx_0= \ivp(\bfp_0,h)$.
	%% Move away from intro:
	It is convenient to write
			$\bfx(t;\bfp_0)$ for $\bfx_0(t)$.
	%
	See \refFig{Volterra-21-13} for the solution to the
	Volterra system (see Eg1 in \refTab{problems}).
	% described as Eg1 in our list of examples below.
	%%
	The IVP problem has numerous applications including
	in modeling physical, chemical or biological systems,
	%%
    

	%\FigEPS{Volterra-21-13}{0.1}{
  	%	Volterra system (see Eg1) with $\bfp_0=[1,3]$;
	%	the negative lognorm region is above green parabola.
	%}

	% Scaling does not work in \includegraphics! 
	% New macro: FigEPSissac 

	\FigEPSissac{Volterra-21-13}{0.5}{
  		Volterra system (see Eg1) with $\bfp_0=[1,3]$;
		the negative \lognorm\ region is above green parabola.}


	The mathematical IVP gives rise to a several
	algorithmic formulations since we cannot explicitly represent
	$\bfx_0= \ivp(\bfp_0,h)$.
	We choose the simplest algorithm form of IVP,
	that of computing an enclosure for $\bfx(h;\bfp_0)$.
	In any real world applications, we only have 
	approximate values for $\bfp_0$; so we replace $\bfp_0$ by a
    region $B_0\ib\RR^n$ and let
    $\ivp(B_0,h)\as \bigcup_{\bfp\in B_0}\ivp(p_0,h)$.  
	Call $B_1\ib \RR^n$ an \dt{end-enclosure} for $\ivp(B_0,h)$
	if we have the inclusion
		$\set{\bfx(h) : \bfx\in \ivp(B_0,h)}\ib B_1$.
	\savespace{
		Corliss \cite[Section 3]{corliss:survey-ode-intvl:89}
		confirms that this is the interval viewpoint.
		}%savespace
	So our formal formal algorithmic problem is
	the following \dt{End Enclosure Problem}:

	% use image to get equation number!!!
		\beql{endEncProb}
			\includegraphics[width=0.75\columnwidth]{figs/endEncIVP}
		\eeql


	\ignore{% KEEP THIS TO GENERATE THE IMAGE!!!
		\Ldent\progb{
			% \label{Problem}
			\lline[-2] \endEncIVP($B_0,\veps$) $\ssa (\ulB_0,\olB_1)$
			\lline[0] INPUT: $\veps>0, B_0\ib \RR^n$ is a box.
			\lline[5] such that $\ivp(B_0,1)$ is valid.
			\lline[0] OUTPUT: $\ulB_0, \olB_1$ are boxes in $\RR^n$
			\lline[5] with $\ulB_0\ib B_0$, $\wmax(\olB_1)<\eps$
			\lline[5] and $\olB_1$ is an end-enclosure of
						$\ivp(\ulB_0,1)$.
		}
	}%
	
	\issacArxiv{
		This is called the \dt{Reachability Problem} in
		the non-linear control systems and verification literature
		(e.g.,	\cite{shen+2:tight-reach:21}).
			% (e.g., \cite{fan+3:simulation-reach:17}),
		}%
    The input to $\endEncIVP(B_0,\veps)$ is assumed to be
	valid;  see the precise definition in
%%% SHOULD turn into an automatic label:	
	\refSSec{Notations}
	below.  Note that we also allow $B_0$ to be shrunk to some
	$\ulB_0$ in order to satisfy a user-specified
	error bound of $\veps$.  This is a novel feature that
	will prove very useful.
    
%\ssect{New Ideas?}
%%%%%%%%%%%%%%%%%%%%%%%%%%%%%%%%%%%%%%%%%%%%%%%%%%
\ssect{How we exploit negative Logarithmic Norms}
%%%%%%%%%%%%%%%%%%%%%%%%%%%%%%%%%%%%%%%%%%%%%%%%%%
	Many authors have tried to exploit properties
	of contractive systems;
	see Sontag \cite{sontag:contractive:10} for an overview.
	An early paper is
	Adams and Ames \cite{adams-ames:contracting:78}.
	%Adams and Ames \cite{adams-ames:contracting:78} show that
	%when the approximate differential operator is quasi-monotone,
	%one can exploit contraction properties of iterations.
	A \dt{logNorm bound} of $B_1\ib \RR^n$ is any upper bound on
		%%
		\beql{lognormbound}
		\mu_2(J_\bff(B_1))\as \sup \set{
					\mu_2(J_\bff(\bfp)) : {\bfp\in B_1}}
		\eeql
		%%
	We call $B_1$ a \dt{contraction zone} if it has a
	negative \lognorm\ bound.
	Here, $J_\bff$ is the Jacobian of $\bff$
	and $\mu_2$ is the \lognorm\ function
	% (\cored{Section 2.4}).
	(\refSSec{lognorm}).
	We exploit the fact that
		$$\|\bfx(t;\bfp_0)-\bfx(t;\bfp_1)\|
			\le \|\bfp_0-\bfp_1\| e^{t\olmu}$$
	(\refThm{ne} in \refSSec{lognorm})
	In \refFig{Volterra-21-13}, the exact contraction zone is the
	region above the green parabola.  In tracing a solution
	$\bfx(t;\bfp_0)$ for $t\in [0,h]$ through a contraction zone,
	we can compute a end-enclosure $B$
	for $\IVP(B_0,h,B_1)$ with $\wmax(B)<\wmax(B_0)$
	(i.e., the end-enclosure is ``shrinking'').
	This is superior to any Taylor bounds such as
		$\bfx(t; B_0)= B_0+ \bff(B_0)t + \cdots$
	which cannot shrink.

	We take this idea a step further:
	if $B_1$ is a non-contraction zone,
	we introduce an ``radical map'' $\pi:\RR^n\to\RR^n$
	with $\bfy=\pi(\bfx)$ such that
	the system $\bfx'=\bff(\bfx)$ transforms to
	the system $\bfy'=\bfg(\bfy)$ such that $\pi(B_0)$
	is a contraction zone in the $(\bfy,\bfg)$-space.
	By computing a good end-endclosure in the $(\bfy,\bfg)$-space,
	we infer a corresponding end-enclosure in the $(\bfx,\bff)$-space.
	\refSSec{comparison} gives numerical
	evidence that such an approach is quite effective.

	%Following the lead of Neuemair 
	%\cite{neumaier:theoryI:94}, we want to exploit the
	%theory of logarithm norms (log norm for short)
	%\cite{strom:log-norm:75, soderlind:log-norm-history}.



%%%%%%%%%%%%%%%%%%%%%%%%%%%%%%%%%%%%%%%%%%%%%%%%%%
% END
%%%%%%%%%%%%%%%%%%%%%%%%%%%%%%%%%%%%%%%%%%%%%%%%%%



%\ssect{What is new?}
%%%%%%%%%%%%%%%%%%%%%%%%%%%%%%%%%%%%%%%%%%%%%%%%%%
\subsection{What is Achieved in This Paper}
%%%%%%%%%%%%%%%%%%%%%%%%%%%%%%%%%%%%%%%%%%%%%%%%%%
	We present an algorithm
		$\ourAlgo(B_0,\veps)\ssa(\ulB_0,\olB_1)$
	that solves the above $\endEncIVP(B_0,\veps)$ problem.  It is
	\dt{complete} in that sense that if the input is valid,
	then \dt{(C1)} the algorithm halts, and
	\dt{(C2)} the output $(\ulB_0,\olB_1)$ specifications of \\
	$\endEncIVP(B_0,\veps)$.  Our algorithm do not
	need any extra ``hyperparameters'': we literally
	implement what is described in this paper.
	Algorithms that only satisfy \dt{(C2)}
	are said\footnote{
		Completeness and partial completeness are standard
		terms in theoretical computer science.
	} to be \dt{partially correct}.
	To our knowledge, all current validated IVP algorithms 
	are only partially correct.

	Our preliminary implementation shows the
	viability of \\ \ourAlgo, and
	its ability to do certain computations where
	current validated IVP software fails.

	% Validity of
	% input should ideally be only a syntactic requirement.
	Our $\endEncIVP$ Problem requires the validity 
	of the input $(B_0,\veps)$.  This makes it a 
	\dt{promise problem}
	\cite{goldreich:promise-problems:06}).
	It raises a separate computational issue to check validity.
	To our knowledge, this is an open problem although 
	Some version of such questions are
	undecidable in the analytic complexity framework
	\cite{graca+2:max-intvl-ivp:09, ko:real:bk}.
	%In addition, our IVP algorithm is the first algorithm 
	%to support user-specified error bound.
	%	In this age of AI, the word ``algorithm''  
	%	often refers to heuristics or techniques,
	%	with no pre-determined concept of correctness.
	%The computational complexity literature sometimes
	%appear to claim complete IVP algorithms, but
	%we shall review these claims below.
	Hans Stetter \cite{stetter:validatedODE:90}
	summarized the state-of-the-art over 30 years ago as follows:
	{\em To date, no programs that could be truly called `scientific
	software' have been produced. 
	AWA is state-of-art, and can be used by a sufficiently expert user --
	it requires selection of step-size, order and suitable choice of
	inclusion set represention.}  
	Corliss \cite[Section 10]{corliss:survey-ode-intvl:89}
	made similar remarks.
	% there is no validated IVP software that
	% {\em is ready to be placed
	% in the hands of scientists and engineers for routine use}.
	Unfortunately, the situation has not changed in 30 years.
	The gold standard for Validated IVP software is based on
	Lohner's algorithm
	% remove lohner:thesis to save space
	%\cite{lohner:thesis,moore:diffEqn:09,corliss:survey-ode-intvl:89}.
	\cite{moore:diffEqn:09,corliss:survey-ode-intvl:89}.
	called AWA\footnote{
		AWA is the German term for IVP.
	}. 
	%
	Bunger's recent implementation \cite{bunger:taylorODE:20}
	of AWA in INTLAB/MATLAB \cite{intlab-ref} has greatly  
	improved its usability.  Besides the requisite IVP
	inputs, this software allow input options
	% like Taylor order (which
	% do not affect correctness) but it also has
	such as min-step size \ttt{hmin},
	the absolute and relative error tolerances \ttt{AbsTol}
	and \ttt{RelTol}, and it automatically generates
	a time grid $t_1<\cdots<t_N$
	\cite[p.~8]{bunger:taylorODE:20}.
	These are well-known \dt{hyperparameters} of
	numerical algorithms which
	experienced users may set to
	coax a recalcitrant instance to converge, but no
	guarantees.
	% (e.g., Bunger's AWA provide options such as
	% such as min-step size, max-error tolerance, etc.)

%%%%%%%%%%%%%%%%%%%%%%%%%%%%%%%%%%%%%%%%%%%%%%%%%%
% END
%%%%%%%%%%%%%%%%%%%%%%%%%%%%%%%%%%%%%%%%%%%%%%%%%%



%\ssect{Review}
\tikzstyle{edge}=[-latex',draw=black!90,shorten <=1pt,shorten >=1pt]
\tikzstyle{redge}=[latex'-,draw=black!90,shorten <=1pt,shorten >=1pt]
\tikzstyle{dedge}=[latex'-latex',draw=black!90,shorten <=1pt,shorten >=1pt]

\tikzstyle{block}=[draw, text width=5em,align=center,shape=rectangle, rounded corners, , align=center]
\tikzstyle{nobox}=[align=center]
\definecolor{emb}{RGB}{209,228,252}
\definecolor{hidden-blue}{RGB}{194,232,247}
% \definecolor{hidden-orange}{RGB}{243,202,120}
\definecolor{hidden-orange}{RGB}{224,224,224}
% \definecolor{hidden-orange}{RGB}{10,128,122}
\definecolor{hidden-yellow}{RGB}{242,244,193}
\definecolor{output-purple}{RGB}{219,203,231}
\definecolor{output-green}{RGB}{204,231,207}
\definecolor{output-blue}{RGB}{44,169,225}

% \definecolor{hiddendraw}{RGB}{205, 44, 36}
\definecolor{output-black}{RGB}{0,0,0}
\definecolor{output-white}{RGB}{255,255,255}
\definecolor{myblue}{RGB}{137,195,235}
\definecolor{hiddendraw}{RGB}{137,195,235}

\tikzstyle{leaf}=[draw=hiddendraw,
    rounded corners, minimum height=1em,
    fill=myblue!40,text opacity=1, 
    fill opacity=.5,  text=black,align=left,font=\scriptsize,
    inner xsep=3pt,
    inner ysep=1pt,
]
\tikzstyle{middle}=[draw=hiddendraw,
    rounded corners, minimum height=1.5em,
    fill=output-white!40,text opacity=1, 
    fill opacity=.5, text=black, align=center, font=\small,
    inner xsep=7pt,
    inner ysep=1pt,
]

\begin{figure*}[htbp!]
\centering
\begin{forest}
  for tree={
      forked edges,
      grow=east,
      reversed=true,
      anchor=base west,
      parent anchor=east,
      child anchor=west,
      base=middle,
      font=\scriptsize,
      rectangle,
      line width=0.8pt,
      draw=output-black,
      rounded corners,align=left,
      minimum width=2em, s sep=6pt, l sep=8pt,
  },
  where level=1{text width=0.2\linewidth}{},
  where level=2{text width=0.2\linewidth,font=\scriptsize}{},
  where level=3{font=\scriptsize}{},
  where level=4{font=\scriptsize}{},
  where level=5{font=\scriptsize}{},
  [LLMs for Foreign Language Education, middle,rotate=90,anchor=north,edge=output-black
      [LLM as Data Enhancer\\(Section \ref{sec:enhancer}),middle,anchor=west,edge=output-black, text width=0.18\linewidth
        [Data Creation, middle, text width=0.13\linewidth, edge=output-black
            [{PFQS~\cite{li-zhang-2024-planning}, MCQG-SRefine~\cite{yao2024mcqg}, \cite{lee2024few},\\BF-TC~\cite{liu2024personality}, MathVC~\cite{yue2024mathvc}, EduAgent~\cite{xu2024eduagent},\\Generative Students~\cite{lu2024generative}}, leaf, text width=0.525\linewidth, edge=output-black]
        ]
        [Data Reformation, middle, text width=0.13\linewidth, edge=output-black
            [{Book2Dial~\cite{wang-etal-2024-book2dial}, Slide2Lecture~\cite{zhang2024awaking},\\WikiDomains~\cite{asthana-etal-2024-evaluating}, \cite{freyer2024easy}, \cite{day2025evaluating},\\Anthropological Prompting~\cite{alkhamissi-etal-2024-investigating}, \cite{liu2024culturally}}, leaf, text width=0.525\linewidth, edge=output-black]
        ]
        [Data Annotation, middle, text width=0.13\linewidth, edge=output-black
            [{EXCGEC~\cite{ye2024excgec}, EDEN~\cite{li-etal-2024-eden}, MEGAnno+~\cite{kim-etal-2024-meganno},\\PDFChatAnnotator~\cite{tang2024pdfchatannotator}, Coannotating~\cite{li-etal-2023-coannotating}}, leaf, text width=0.525\linewidth, edge=output-black]
        ]
      ]
      [LLM as Task Predictor\\(Section \ref{sec:predictor}),middle,anchor=west,edge=output-black, text width=0.18\linewidth
        [Discriminative, middle, text width=0.13\linewidth, edge=output-black
            [{TOREE~\cite{zhuang-etal-2024-toree}, LLM-KT~\cite{zhan2024knowledge}, CLST~\cite{jung2024clst},\\Diallogue-KT~\cite{scarlatos2024exploring}, \cite{neshaei2024towards},\\\cite{mizumoto2023exploring}, \cite{sessler2024can}, \cite{syamkumar2024improving}}, leaf, text width=0.525\linewidth, edge=output-black]
        ]
        [Generative, middle, text width=0.13\linewidth, edge=output-black
            [{EXPECT~\cite{fei-etal-2023-enhancing}, GEE~\cite{song-etal-2024-gee}, EXCGEC~\cite{ye2024excgec}\\FELT~\cite{borges-etal-2024-teach}, LLM-as-a-tutor~\cite{han-etal-2024-llm}, \cite{stamper2024enhancing}\\SocraticLM~\cite{liusocraticlm}, \cite{favero2024enhancing}, \cite{xiao2024humanaicollaborativeessayscoring}}, leaf, text width=0.525\linewidth, edge=output-black]
        ]
        [Mixed, middle, text width=0.13\linewidth, edge=output-black
            [{FABRIC~\cite{han2023fabric}, \cite{stahl2024exploring},\\ReaLMistake~\cite{kamoi2024evaluating}, \cite{lu2023error}}, leaf, text width=0.525\linewidth, edge=output-black]
        ]
      ]
      [LLM-empowered Agent\\(Section \ref{sec:agent}),middle,anchor=west,edge=output-black, text width=0.18\linewidth
        [Abilities, middle, text width=0.13\linewidth, edge=output-black
            [{FOKE~\cite{hu2024foke}, KnowEdit~\cite{zhang2024comprehensive}, \cite{zhang2024comprehensive},\\ PedCoT~\cite{jiang2024llms}, LHP~\cite{sonkar-etal-2024-pedagogical}, \cite{razafinirina2024pedagogical},\\LessonPlanner~\cite{fan2024lessonplanner}, MemoryBank~\cite{zhong2024memorybank}, \cite{hu2024teaching},\\vNMF~\cite{jiang2024ai}, ChatTutor~\cite{chen2024empowering}, \cite{zheng2024automatic}}, leaf, text width=0.525\linewidth, edge=output-black]
        ]
        [Applications, middle, text width=0.13\linewidth, edge=output-black
            [{SimClass~\cite{zhang2024simulating}, MathVC~\cite{yue2024mathvc}, BIPED~\cite{kwon-etal-2024-biped},\\LLM-as-a-tutor~\cite{han-etal-2024-llm}, KORLINGS~\cite{lee2024developing}, \cite{liu2024personality}}, leaf, text width=0.525\linewidth, edge=output-black]
        ]
      ]
]
\end{forest}
\caption{An overview of LLM-centric research of FLE.}
\label{fig:review}
\end{figure*}

	
% Remarks about our new notations
One of the barriers to the validated IVP literature
	is its cumbersome notations and lack of precise
	input/output criteria for its algorithms.
	We provide a streamlined notation by exploiting
	the autonomous nature of our ODE, and introducing
	high-level data structures such as the scaffold.

\ssect{Paper Overview}
	%%
	The remainder of the paper is organized as follows: 
	\dt{Section 2} introduces some key concepts and computational tools. 
	\dt{Section 3} gives an overview of our algorithm.
	\dt{Section 4} describes our \stepA\ and \stepB\ subroutines.
	\dt{Section 5} compares our transform approach to
		the classic Euler in a suitable local neighborhood.
	\dt{Section 6} describes the \Extend\ and \Refine\ subroutines.
	\dt{Section 7} presents our end-enclosure algorithm and some experiments.
	We conclude in \dt{Section 8}.
	\issacArxiv[
		\dt{Appendix A} gives some critical proofs. Due
		to space limitation, the remaining proofs
		in found in arXiv \cite{zhang-yap:ivp:25arxiv}.
	]{
		\dt{Appendix A} gives all the proofs.
		\dt{Appendix B} provide details of the affine
		transform $\olpi$.
	}
	

	\ignore{
	In Sections 8 and 9, we present the theoretical details of our
		refinement method, with Section 8 addressing the case where the
		logarithmic norm is negative and Section 9 covering the case where
		the logarithmic norm is non-negative. 
	Finally, in Section 10, we provide the algorithmic details of our
		method and prove that it is guaranteed to terminate.

	Appendix???
	}
	

	%%%



%%%%%%%%%%%%%%%%%%%%%%%%%%%%%%%%%%%%%%%%%%%%%%%%%%%
%	
%	A common scheme in Validated IVP is the 2-step scheme:
%	Step A is finds an initial step size
%	and an domain of existence and uniqueness of solution;
%	Step B refinds the domain of Step A (possibly
%	reducing the step size).
%
%	Besides certified computation (or interval methods),
%		this is of interest for verification research.
%	E.g., Kellison-Appel \cite{kellison-appel:verified-ode:22}
%	developed a framework based on the Coq Proof assistant to
%	verify the accuracy and correctness of ODEs.
%	
%
%%%%%%%%%%%%%%%%%%%%%%%%%%%%%%%%%%%%%%%%%%%%%%%%%%%
