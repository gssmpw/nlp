%%%%%%%%%%%%%%%%%%%%%%%%%%%%%%%%%%%%%%%%%%%%%%%%%%
\newpage
\appendix
%\section*{Appendix A: Proofs}
\section{Appendix A: Proofs}
%%%%%%%%%%%%%%%%%%%%%%%%%%%%%%%%%%%%%%%%%%%%%%%%%%
	%\ssect{The relevant proofs of the conclusions in the paper }
	Note that the numberings of lemmas and theorems
	in this Appendix are the same as corresponding
	results in the text, and have hyperlinks to the text.
	
	
	\dt{ \refCor{cor-1}}
		\issacArxiv[]{
			{\em  \ \\
			Let $\bfx_1\in \ivp(\bfp_1,h,Ball(\bfp_0,r))$ 
			and $\bfx_2\in \ivp(\bfp_2,h,Ball(\bfp_0,r))$.
			\\ If $\olmu\ge \mu_2(\frac{\partial \bff}
			{\partial \bfx}(Ball(\bfp_0,r)))$
			then for all $t\in [0,h]$
			\beq
			\|\bfx_1(t)-\bfx_2(t)\|_2
			\le \|\bfp_1-\bfp_2\|_2 e^{\olmu t}.\eeq
			}
	}
		\bpf
		Note that $\bfx_1$ and $\bfx_2$ are solutions of \refeq{bfx'}
		with different initial values. Therefore, we have $\bfx'_1 =
		\bff(\bfx_1)$ and $\bfx'_2 = \bff(\bfx_2)$. This implies that
		
		\[
		\bfx'_1(t) - \bff(\bfx_1(t))
		= \bff(\bfx_1(t)) - \bff(\bfx_1(t)) = 0 = \veps.
		\]
		If $\olmu\ne 0$, then \refeq{bfx12} is the first case
		of \refeq{xibfx} since $\delta=\|\bfp_1-\bfp_2\|_2$.
		If $\olmu=0$, it comes from the second case since $\veps=0$.
		\epf
	
	The following is a useful lemma:

	\blemT[eulerOneStep]{Full-enclosure for One Euler Step}\ \\
		Let		$(B_0,H,B_1)$
		be an admissible triple with 
		$\olmu\ge \mu_2(J_{\bff}(B_1))$,
		and
		$ M\ge \|\bff^{[2]}(B_1)\|$.
		Denote the Euler step at $\bfq_0\in B_0$
		by the linear function
		$$\ell(t;\bfq_0) = \bfq_0 + t \bff(\bfq_0).$$
		Then for all $t\in [0,H]$,
		$$\|\bfx(t;\bfp_0)-\ell(t;\bfq_0)\|
		\le \|\bfp_0-\bfq_0\| e^{\olmu t}+ \half Mt^2$$
	\elemT	
	\bpf
		By \refCor{cor-1},
		\beql{localcor1}
		\|\bfx(t;\bfp_0)-\bfx(t;\bfq_0\| \le 
		\|\bfp_0-\bfq_0\|e^{\olmu t}
		\eeql
		We also have
		\beqarrys
		\bfx(t;\bfq_0) &=& \bfq_0 + t\cdot \bff(q_0)
		+ \half t^2 \bfx''(\tau)
		& \text{(for some $\tau\in [0,t]$)}\\
		\lefteqn{\|\bfx(t;\bfq_0) - (\bfq_0 + t\cdot \bff(q_0))\|} \\
		&\le& \| \half t^2 \bfx''(\tau)\| \\
		&=& \| \half t^2 \bff\supn[2](\bfx(\tau;\bfq_0)\| \\
		&\le& \| \half t^2 M\|
		& \text{(since $M\ge \bff\supn[x](B_1)$)}
		\eeqarrys
		Combined with \refeq{localcor1},
		the triangular inequality shows our desired bound.
		\epf
		
	
	\dt{	\refLem{eulerStep}}\ \\
	{\em
		Let $(B_0,H,B_1)$ be admissible triple,
	$\olmu\ge \mu_2(J_{\bff}(B_1))$ and
	$ M\ge \|\bff^{[2]}(B_1)\|$.
	%%
	If $h_1>0$ is given by
	\beq
	h_1\ass h(H,M,\olmu,\veps) \as
	\begin{cases}
		\min\set{H, \frac{2\olmu\veps}
			{M \cdot (e^{\olmu H}-1)}}
		&\rmif\ \olmu\ge 0\\
		\min\set{H, \frac{2\olmu\veps}
			{M \cdot (e^{\olmu H}-1)-\olmu^2\veps}}
		&\rmif\ \olmu<0
	\end{cases}
	\eeq
	consider the path $Q_{h_1}=(\bfq_0,\bfq_1\dd \bfq_m)$ 
	from the Euler method with step-size $h_1$.
	If each $\bfq_i\in B_1$ ($i=0\dd m$),
	then for all $t\in [0,H]$, we have
	\beq
	\|Q_{h_1}(t)-\bfx(t;\bfq_0)\|\le \veps.
	\eeq
	I.e., $Q_{h_1}(t)$ lies inside the $\veps$-tube of $\bfx(t;\bfq_0)$.
	}
	\bpf
	For simplicity, we only prove the lemma
	when $H/h_1$ is an integer.
	We first show that the Euler method with step size $h_1>0$ 
	has the following error bound:
	\beql{gm}
	\|\bfq-\bfx(H)\|\le 	
	\begin{cases}
		\frac{Mh_1}{2\olmu}(e^{\olmu H}-1) & \olmu\ge 0,\\
		\frac{Mh_1}{2\olmu+\olmu^2 h_1}(e^{\olmu H}-1) & 
		\olmu<0.
	\end{cases}
	\eeql
	To show \refeq{gm}, assume
	$(\bfp_0=\bfx(0),\bfp_1\dd \bfp_m=\bfq)$  are obtained by the
	Euler method corresponding to $t_0=0,t_1\dd t_m=H$. 
	Let $g_i=\|\bfp_i-\bfx(t_i)\|_2$ be the error bound.  Then we have 
	\beqarrys
	g_m &\le&  g_{m-1}e^{\olmu h_1}+
	\frac{Mh_1^2}{2}
	& \text{(by Taylor formula)}\\
	&\le& g_{m-2}e^{\olmu h_1}+
	\frac{Mh_1^2}{2}
	e^{\olmu h_1} +	\frac{Mh_1^2}{2}
	& \text{(by expanding $g_{m-1}$)}\\
	&\vdots& \\
	&\le& \frac{Mh_1^2}{2}(1+e^{\olmu 		
		h_1}+\cdots e^{\olmu h_1 (m-1)})
	& \text{(since $g_0=0$)}\\
	&\le&   \frac{Mh_1^2}{2}\frac{e^{\olmu H}-1}
	{e^{\olmu h_1}-1}\\	
	&\le& \clauses{
		\frac{Mh_1}{2\olmu}(e^{\olmu H}-1)
		& \rmif\ \olmu\ge 0,\\
		%%%%%	
		\frac{Mh_1}{2\olmu+\olmu^2 h_1}(e^{\olmu H}-1)
		& \rmif\ \olmu< 0.}
	\eeqarrys
	If $\olmu\ge 0$, then the last formula is justified by
	$e^{\olmu h_1}-1\ge \olmu h_1$, and
	so $g_m\le \frac{Mh_1}{2\olmu}(e^{\olmu H}-1)$.
	If $\olmu<0$, then the formula is justified by
	$e^{\olmu h_1}-1\le \olmu h_1+ \half\olmu^2 h^2_1$
	(use the fact that $f(x)=e^x - 1-x-\half x^2<0$
	when $x<0$; check that $f'(x)=e^x-1-x>0$ for all $x<0$).
	This proves \refeq{gm}.
	
	Focusing on the case $\olmu<0$:
	we claim that 
	$$\delta>\frac{Mh_1}{2\olmu+\olmu^2 h_1}(e^{\olmu H}-1)$$
	is equivalent to 
	$$h_1 < \frac{2\olmu\delta}
	{M \cdot (e^{\olmu H}-1)-\olmu^2\delta}.$$
	This is verified by direct algebraic manipulation.
	\epf	
	
	\dt{ \refLem{ad}}
		\issacArxiv[]{{\em \ \\
			Let $ k \geq 1 $, $ H > 0 $, $ \veps > 0 $, and 
			$ E_0 \ib \RR^n $. Define  
			\[
			\ol{B} = \sum_{i=0}^{k-1} [0, H]^i \bff^{[i]}(E_0) + 	[-\veps,
			\veps]^n, \qquad
			M := \sup_{\bfp \in \ol{B}} \| \bff^{[k]}(\bfp) \|_2.
			\]  
			Then, the pair $(h, F_1)$ is $\veps$-admissible for $E_0$, where  
			\[
			h = \min \left\{ H, \left( \frac{\veps}{M} 	\right)^{\frac{1}{k}}
			\right\},  \qquad
			F_1 = \sum_{i=0}^{k-1} [0, h]^i \bff^{[i]}(E_0) + 
			[-\veps, \veps]^n.
			\]  
	}}
		\bpf
		To verify \refeq{tay},
		we only need to verify
		\[[0,h]^k\bff^{[k]}(F_1)\ib [-\veps,\veps]^n.\]
		We have:
		\beqarrys
		h^k\bff^{[k]}(F_1)
		&\ib& h^k[-M,M]^n
		&\text{(by the definition of $M$  )}\\
		&\ib& [-\veps,\veps]^n.
		&\text{(by the definition of $h$  )}
		\eeqarrys
		\epf
		
		
		%\blemT[endenclosure]{Triangular Inequality} \ \\
		\blemDIY[Lemma A.1. \text{(Triangular Inequality)}]
			{\label{lem:endenclosure} \ \\
		Let
		$\bfx_c \in \ivp(\bfp_0, h,F_1)$,
		and 
		$\olmu= \mu_2(J_\bff(F_1))$ (not necessarily negative).
		Suppose $\bfq_0$ satisfies
		\[
		\|\bfq_0 - \bfx_c(h)\|_2 \leq \delta,
		\]
		then for any $\bfx \in \ivp(E_0, h)$ where $E_0=Ball(\bfp_0,r)$, 
		we have $\bfx(h) \in Ball(\bfq_0, r)$,
		where  $r = r_0 e^{\olmu h} + \delta$. 
		}
		%\elemT

		\bpf
		By \refCor{cor-1} we have for any $\bfx\in\ivp(E_0,h)$,
		$\|\bfx(h)-\bfx_c(h)\|\le r_0e^{\olmu h}$.
		Since $\|\bfq_0 - \bfx_c(h)\|_2 \leq \delta$, 
		then by the triangular inequality 
		we have 
		$$\|\bfq_0 - \bfx(h)\|_2\le \|\bfx(h)-\bfx_c(h)\|+\|\bfq_0 -
		\bfx_c(h)\|_2\le r.$$
		So, $\bfx(h) \in  Ball(\bfq_0, r)$.
		\epf
	
	\dt{\refLem{delta-distance}}
	\issacArxiv[]{{\em\ \\
			Consider 
			$\ivp(Ball(\bfp_0,r_0),h,F_1)$
			and $\olmu = \mu_2(J_\bff(F_1))$.  
			Let $\bfx_c \in\ivp(\bfp_0,h)$
			and $\bfq_0=\bfx_c(h)$.
			If the linear function
			$$\ell(t)\as (1-t/h)\bfp_0+(t/h)\bfq_0$$
			lies in the $\delta$-tube of $\bfx_c$ then
			\benum[(i)]
			\item  $Ball(\bfq_0, r_0 e^{\olmu h}+\delta)$ is an
			end-enclosure for $\ivp(Ball(\bfp_0,r_0),h)$.
			\item  Let $r'=\max(r_0 e^{\olmu h},r_0)$ then
			$Box(Ball(\bfp_0, r'+\delta),
			Ball(\bfq_0, r'+\delta))$
			is a full-enclosure for $\ivp(Ball(\bfp_0,r_0),h)$.
			\eenum
	}}
		\bpf
		% By \refLem{endenclosure},
		% dangerous:
		by {\bf Lemma A.1},
		we have $E_1 =  Ball(\bfq_0, r_0e^{\olmu h}+\delta)$ is the
		end-enclosure for $\ivp(E_0, h)$.  
		
		Next, we prove $(ii)$.  
		We show that for any $T \in [0, h]$,  the end-enclosure of 
		$\ivp(E_0,T)$ is a subset of $Box(Ball(\bfp_0, r' +\delta),
		Ball(\bfq_0, r' +\delta))$.
		Note that
		% by \refLem{endenclosure},
		% dangerous:
		by {\bf Lemma A.1},
		we have $E_1 =  Ball(l(T),
		r_0e^{\olmu T}+\delta)$ is the
		end-enclosure for $\ivp(E_0, T)$.
		
		Let $l(T)_i$ denote the $i$-th component of $l(T)$ and $r(T)\as
		r_0e^{\olmu T}+\delta$. Then, we only need to
		prove that for any $i=1\dd n$, the interval $l(T)_i \pm r(T)$ satisfies  
		\[
		l(T)_i \pm r(T) \subseteq Box((\bfp_0)_i\pm (r'+\delta),
		(\bfq_0)_i\pm (r'+\delta)),
		\]  
		where $(\bfp_0)_i $ and $(\bfq_0)_i $ are the $i$-th components of
		$\bfp_0$ and $\bfq_0$,
		respectively.  
		
		Since $l(T)$ is a line segment, it follows that  
		\[
		\min((\bfq_0)_i, (\bfp_0)_i) \leq l(T)_i \leq \max((\bfq_0)_i, (\bfp_0)_i).
		\]  
		Additionally, we have $r(T) \leq r' + \delta$.  
		
		Combining these observations, we conclude that 
		\[
		l(T)_i \pm r(T) \subseteq Box((\bfp_0)_i\pm (r'+\delta),
		(\bfq_0)_i\pm (r'+\delta)).
		\] 
		
		\epf
	
	%\bleml[Jbfg]
	\blemDIY[Lemma A.2]{\ \\
		\label{lem:Jbfg}
	\benum
	\item $\bfg(\bfy)=J_{\whpi}(\whpi\inv(\bfy))
		\Bigcdot \ol\bfg(\whpi\inv(\bfy))$\\
		$=\diag(-d_i y_i^{1+\tfrac{1}{d_i}}: i=1\dd n)
		\Bigcdot \ol\bfg(\whpi\inv(\bfy)) $.
	\item The Jacobian matrix of $\bfg$ with respect to
		$\bfy=(y_1\dd y_n)$ is:
		\beql{Jbfg}
		J_{\bfg}(\bfy)=A(\bfy)
		+ 
		P\inv(\bfy)
		\Bigcdot 
		J_{\ol\bfg}(\whpi\inv(\bfy))\Bigcdot P(\bfy),
		\eeql
		where 
		$$A(\bfy)=\diag\Big(-(d_i+1)y_i^{\frac{1}{d_i}}\cdot
		(\wt\bfg(\wtpi\inv(\bfy)))_1): i=1\dd n\Big) $$
		and 
		$$P(\bfy)=\diag\Big(\tfrac{\wtpi\inv(\bfy)_i^{d_i+1}}{d_i}:
		i=1\dd n \Big).$$
	\eenum
	}%DIY
	%\eleml
	\issacArxiv[]{
	\bpf
	\benum
	\item 
	For each $ i = 1\dd n $, we have from \refeq{bfy'} that
	$y_i'= g_i(\bfy)$ where $\bfy=(y_1\dd y_n)$,
	$\bfg=(g_1\dd g_n)$, i.e.,
	\beqarrys
	g_i(\bfy) = y_i' 
	&=& \left(\frac{1}{\wty_i^{d_i}}\right)'
	\qquad\text{(by \refeq{bfy'} and $y_i=\wty_i^{-d_i}$)}\\
	&=& -d_i \wty_i^{-(d_i+1)} \wty_i' \\
	&=& -d_i y_i^{1 + \tfrac{1}{d_i}} \Big(\wt\bfg
	\big( y_1^{-\tfrac{1}{d_1}}\dd y_n^{-\tfrac{1}{d_n}} \big)\Big)_i\\
	&=& -d_i y_i^{1+\tfrac{1}{d_i}} (\wt\bfg(\wtpi\inv(\bfy)))_i.
	\eeqarrys
	Thus, $$\bfg(\bfy)=
	(g_1(\bfy)\dd g_n(\bfy))
	=\diag(-d_i y_i^{1+\tfrac{1}{d_i}}, i=1\dd n)
	\Bigcdot \wt\bfg(\wtpi\inv(\bfy))$$
	\item 
	By plugging 
	$g_i(\bfy) = -d_i y_i^{1+\tfrac{1}{d_i}} (\wt\bfg(\wtpi\inv(\bfy)))_i$
	into the Jacobian, we get
	{\tiny
		\beqarray
		J_\bfg(\bfy)&=& \mmat{\nabla( g_1(\bfy) ) \\ \vdots \\
			\nabla( g_n(\bfy) } 
		=
		\mmat{\nabla( -d_i y_1^{1+\tfrac{1}{d_i}}
			(\wt\bfg(\wtpi\inv(\bfy)))_1) \\ \vdots \\
			\nabla (-d_n y_n^{1+\tfrac{1}{d_n}} (\wt\bfg(\wtpi\inv(\bfy)))_n)}
		\nonumber\\
		\nonumber\\
		&=& \mmat{\nabla(-d_1 y_1^{1+\tfrac{1}{d_1}})
			(\wt\bfg(\wtpi\inv(\bfy)))_1
			\\ \vdots \\
			\nabla (-d_n y_n^{1+\tfrac{1}{d_n}})
			(\wt\bfg(\wtpi\inv(\bfy)))_n}
		+ \mmat{ -d_1 y_1^{1+\tfrac{1}{d_1}}
			\nabla((\wt\bfg(\wtpi\inv(\bfy)))_1)
			\\ \vdots \\
			-d_n y_n^{1+\tfrac{1}{d_n}} 
			\nabla ((\wt\bfg(\wtpi\inv(\bfy)))_n)}
		\label{eq:splitjacobian}
		\eeqarray
	}
	Note that for any $i=1\dd n$, 
	{\scriptsize
		$$	\nabla (-d_i y_i^{1+\tfrac{1}{d_i}})	(\wt\bfg(\wtpi\inv(\bfy)))_i
		=  
		\big(0\dd 0, -d_i\left(1 + \tfrac{1}{d_i}\right)y_i^{\frac{1}{d_i}}
		(\wt\bfg(\wtpi\inv(\bfy)))_i\dd 0 \big)
		$$}
	and
	{\tiny
		\beqarrys
		-d_i y_i^{1+\tfrac{1}{d_i}}
		\nabla((\wt\bfg(\wtpi\inv(\bfy)))_i)
		&=&  \Big( -d_i y_i^{1+\tfrac{1}{d_i}} 
		\frac{\partial (\wt\bfg(\bfx))_i}{\partial x_j}
		(\wtpi\inv(\bfy)) \frac{\partial \wtpi\inv(\bfy)}{\partial \bfy} : j=1\dd n
		\Big)   \\
		&=& 
		\Big(\tfrac{d_i}{d_j} \left(\tfrac{y_i^{1 + \frac{1}{d_i}} }{y_j^{1 + \frac{1}{d_j}} }\right)
		\frac{\partial (\wt\bfg(\bfx))_i}{\partial x_j}
		(\wtpi\inv(\bfy)) : j=1\dd n \Big)\\
		&=&
		\Big( \tfrac{d_i}{d_j}  \wtpi\inv(\bfy)_j^{d_j+1}
		\frac{\partial (\wt\bfg(\bfx))_i}{\partial x_j}
		(\wtpi\inv(\bfy)) \wtpi\inv(\bfy)_i^{-d_i-1} : j=1\dd n \Big).
		\eeqarrys}
	Thus, $$J_{\bfg}(\bfy)=A(\bfy)+P\inv(\bfy)
	\Bigcdot
	J_{\wt\bfg}(\wtpi\inv(\bfy))\Bigcdot P(\bfy),$$
	where $$A(\bfy)=\diag(-(d_1+1)y_1^{\frac{1}{d_1}} (\wt\bfg(\wtpi\inv(\bfy)))_1)\dd 
	-(d_n+1)y_n^{\frac{1}{d_n}} (\wt\bfg(\wtpi\inv(\bfy)))_n) $$
	and 	$$P(\bfy)=\diag(\tfrac{\wtpi\inv(\bfy)_1^{d_1+1}}{d_1}\dd \tfrac{\wtpi\inv(\bfy)_n^{d_n+1}}{d_n}).$$
	\eenum
	\epf}
	
	
	\dt{\refThm{keylemma}}
	\issacArxiv[]{
		{\em \ \\
			\benum[(a)]
			\item
			\beqarrys
			\mu_2 \big(J_{\bfg} (\pi(F_1))\big)
			&\le&
			\max\set{\tfrac{-(d_i+1)}{\chb_i}:
				i=1\dd n}\\
			&&
			+ \max_{i=1}^n \set{d_i} 
			\cdot
			\|J_{\olbfg}(\olpi(F_1))\|_2
			\cdot
			\max_{i=1}^n \set{\tfrac{(\chb_i)^{d_i+1}}{d_i}}.
			\eeqarrys
			\item If $d_1=\cdots=d_n=d$ then
			$$\mu_2 \left(J_{\bfg} (\pi(F_1))\right)
			\le
			-(d+1)\tfrac{1}{\chb_{\max}}
			+(\chb_{\max})^{d+1}
			\|J_{\olbfg}(\olpi(F_1))\|_2.
			$$
			\eenum
	}}
	\bpf
	% From \refLem{Jbfg}(b)
	% Dangerous:
	From {\bf Lemma A.2(b)}
	%%%
	we have for any $\bfp=(p_1\dd p_n) \in \olpi(F_1)$,
	\beql{splitJacobian1}
	J_{\bfg}(\whpi(\bfp))
	= A(\bfp) + P\inv(\bfp)
	\frac{\partial \olbfg}{\partial \bfx}(\bfp)P(\bfp)
	\eeql
	where
	$P(\bfp)=\diag\big(
	\tfrac{p_i^{d_i+1}}{d_i}: i=1\dd n \big)$
	and
	$A(\bfp) = \diag(a_1\dd a_n)$ with
	\beql{aid1}
	a_i \as -d_i(1 + \tfrac{1}{d_i}) p_i\inv
	\cdot (\olbfg(\bfp))_i. \eeql
	Thus, $A, P$ are diagonal matrices and $p_i\inv$ is well-defined
	since $\bfp\in B_2\ge \1$, \refeq{translation}.
	
	By \refLem{lognorm}(b) and \refeq{aid1}, we conclude that
	the form
	\beql{mudiag}
	\mu_2(A(\bfp))=	\mu_2(\diag(a_1\dd a_n))
	= \max\set{a_i: i=1\dd n}.
	\eeql
	
	From \refeq{olbfg}, we conclude that
	{\small
		\beqarrys
		\mu_2\big(J_{\bfg}(\whpi(\bfp))\big) 
		&=& \mu_2\left(A(\bfp) + P\inv(\bfp)\frac{\partial \olbfg}{\partial
			\bfx}(\bfp)P(\bfp)\right)\\
		&& \text{(by \refeq{splitJacobian1})} \\
		&\le& \mu_2(A(\bfp))
		+ \mu_2\left(P\inv(\bfp)\frac{\partial \olbfg}{\partial
			\bfx}(\bfp)P(\bfp)\right)\\
		&& \text{(by \refLem{lognorm}(a))} \\
		&\le& \mu_2(A(\bfp)) 
		+ \left\|P\inv(\bfp)\frac{\partial \olbfg}{\partial
			\bfx}(\bfp)P(\bfp)\right\|_2 \\
		&& \text{(by \refLem{lognorm}(b))} \\
		&\le& \max\set{\tfrac{-(d_i+1)}{\chb_i}:
			i=1\dd n}\\
		&&
		+\left\|P\inv(\bfp)\right\|\left\|\frac{\partial \olbfg}{\partial
			\bfx}(\bfp)\right\|\left\|P(\bfp)\right\|\\
		&& \text{(by \refeq{matrixnorm})}\\
		&\le& \max\set{\tfrac{-(d_i+1)}{\chb_i}:
			i=1\dd n}\\
		&&
		+ \max_{i=1}^n \set{d_i} 
		\cdot
		\|J_{\olbfg}(\olpi(F_1))\|_2
		\cdot
		\max_{i=1}^n \set{\tfrac{(\chb_i)^{d_i+1}}{d_i}}.
		\eeqarrys
	}
	\epf
	
	
	\dt{\refLem{Set-d}}
	\issacArxiv[]{
		{\em \
			Assuming $d$ satisfies \refeq{d}, we have:
			\benum[(a)]
			\item
			$\mu_2 \left(J_{\bfg} (\pi(F_1))\right)
			\le	 (-2+(\chb_{\max})^{d+2})
			\cdot \frac{\|J_{\olbfg}(\olpi(F_1))\|_2}{\chb_{\max}}.$
			\item
			If
			$\log_2(\chb_{\max}) < \tfrac{1}{d+2}$
			then
			$\mu_2 \left(J_{\bfg} (\pi(F_1))\right)< 0$.
			\eenum
	}}
	\bpf
	\benum[(a)]
	\item
	By \refThm{keylemma} we have
	{\small
		\beqarrays
		\mu_2\left(J_{\bfg} (\pi(F_1))\right)
		&\le& -(d+1)\frac{1}{\chb_{\max}}
		+(\chb_{\max})^{d+1}\|J_{\olbfg}(\olpi(F_1))\|_2\\
		&=& \Big(\tfrac{-(d+1)}
		{\|J_{\olbfg}(\olpi(F_1))\|_2}+(\chb_{\max})^{d+2} \Big)
		\cdot \tfrac{\|J_{\olbfg}(\olpi(F_1))\|_2}{\chb_{\max}}\\ 
		&&	 \text{(by factoring)} \\
		&\le&  \Big(-2+(\chb_{\max})^{d+2} \Big)
		\cdot \tfrac{\|J_{\olbfg}(\olpi(F_1))\|_2}{\chb_{\max}} \\
		&&
		\text{(By eqn.\refeq{d}, we have $(d+1)
			\ge 2(
			\|J_{\olbfg}(\olpi(F_1))\|_2)$)}. 
		\eeqarrays
	}
	\item
	Since $(\chb_{\max})^{d+2}<2$
	is equivalent to 
	$\log_2(\chb_{\max}) < \tfrac{1}{d+2}$,
	we conclude that
	$\mu_2 \left(J_{\bfg} (\pi(F_1))\right)< 0$.
	\eenum
	\epf
	
	%\bleml[error-bound-under-phi]
	\blemDIY[Lemma A.3]{\ \\
		\label{lem:error-bound-under-phi}
	Let $\bfp,\bfq\in B\ib\RR^n$ and $\phi\in C^1(F_1\to \RR^n)$, then 
	$\|\phi(\bfp)-\phi(\bfq)\|_2
	\le \|J_{\phi}(B)\|_2 \cdot \|\bfp-\bfq\|_2$
	}% endDIY
	%\eleml
	\issacArxiv[]{
	\bpf
	\beqarrys
	\|\phi(\bfp)-\phi(\bfq)\|_2
	&\le&
	\|\phi(\bfq)+J_{\phi}(\xi)\Bigcdot(\bfp-\bfq) -\phi(\bfq)\|_2\\
	&& \text{(by Taylor expansion of $\phi(\bfp)$ at $\bfq$)}\\
	&=&\|J_{\phi}(\xi)\Bigcdot(\bfp-\bfq)\|_2\\
	&\le& \|J_{\phi}(\xi)\|_2 \cdot \|(\bfp-\bfq)\|_2\\
	&\le& \|J_{\phi}(B)\|_2   \cdot \|(\bfp-\bfq)\|_2,
	\eeqarrys
	where $\xi\in B$.
	\epf}
	
	\dt{ \refLem{error-bound-ode}}
	\issacArxiv[]{
		{\em
			\ \\
			Let $\bfy=\pi(\bfx)$ and 
			$$
			\mmatx[rcl]{
				\bfx 	&\in&
				\ivp_{\bff}(\bfx_0,h,F_1),\\
				\bfy 	&\in&
				\ivp_{\bfg}(\pi(\bfx_0),h,\pi(F_1)).
			}$$
			%%%
			For any $\delta>0$ and any point $\bfp\in \RR^n$ satisfying 
			\beq
			\|\pi(\bfp)-\bfy(h)\|_2
			\le \frac{\delta}{\|J_{\pi\inv}(\pi(F_1))\|_2},
			\eeq
			we have 
			$$\|\bfp-\bfx(h)\|_2\le \delta.$$ 
	}}
	\bpf
	\beqarrys
	\|\bfp-\bfx(h)\|_2
	&=& \|\pi\inv(\pi(\bfp))-\pi\inv(\pi(\bfx(h)))\|_2\\
	&=& \|\pi\inv(\pi(\bfp))-\pi\inv(\bfy(h))\|_2\\
	&\le&  			  
	\|J_{\pi\inv}(\pi(F_1))\|_2
	\cdot \|\pi(\bfp)-\bfy(h)\|_2\\
	%  && \text{(by \refLem{error-bound-under-phi})}\\
	&& \text{(by {\bf Lemma A.3})}\\
	&\le& \delta\qquad
	\text{(by condition \refeq{delta1}.)}
	\eeqarrys
	\epf
	
	
	
	\dt{\refLem{refine}}
	\issacArxiv[]{
		{\em
			\benum[(a)]
			\item 	(Partial Correctness)
			Given $\veps>0$ and $\delta$-bounded $m$-stage
			scaffold $\calD$, if 
			the subroutine \Refine($\calD, \delta,\veps$) terminates,
			it is correct.
			\item (Termination) The subroutine \Refine'\
			will halt for any valid input
			$(\calD, \delta,\veps)$.
			\eenum
	}}
	\bpf
	\benum[(a)]
	\item If the subroutine terminates, this conclusion is hold by  the
	loop termination condition.
	More precisely: if the value of $r_0$ in (Line 1:) is
	less than $\veps$, then we terminate without entering
	the while-loop, and the result hold.
	If we enter the while-loop, then termination implies
	that the last statement is the assignment of $r_0$ in (Line 4:).
	Again this is correct.
	\item 
	Let $\calD$ be an $m$-stage scaffold and $\calD'$ be the
	modified scaffold.
	Let
	$r' = \wmax(E_m(\calD')) $.
	By the (Line 2:) %\textbf{Set radius}
	and (Line 3:) %\textbf{Set end-enclosure} 
	in the subroutine, we have $r' \leq 2r(m) $,
	here $r(k)$ (for $k=0\dd m$) is defined by
	\[
	r(k) = \clauses{ r_0 & \rmif\ k=0,\\
		r(k-1)e^{h_k \mu_k} + \delta' & \rmif\ k\ge 1,}
	\]
	where $h_k=t_k-t_{k-1}$,
	$\mu_k=\mu_2(J_{\bff}(F_k(\calD)))$ and $Ball_\bfp(r_0)$ be the circumball of $E_{0}(\calD)$.
	
	Note that in each iteration of the loop, $ E_0(\calD) $ is
	halved.  Thus, $r_0$ is also halved if $ \delta $ is sufficiently small. Consequently, we obtain 
	$ r(m) < \veps $, which implies termination.
	\eenum
	\epf
	
	\dt{\refThm{correct-main}}
	\issacArxiv[]{
		{\em
			\refAlg{endEnc} is correct.
	}}
	\bpf
	
	If the algorithm terminates, its correctness is ensured by the
	conclusions in \refSec{stepAB}.
	
	We now proceed to prove the termination of the algorithm. Specifically,
	we need to show that the loop in the algorithm can terminate, which
	means that the time variable $ t $ can reach $ 1 $. 
	
	To establish this, it suffices to demonstrate that for given inputs $
	B_0 $ and $ \veps > 0 $, there exists a positive lower bound for the
	step size, denoted by $ \ul h > 0 $, such that for any $ i > 0 $ in the
	loop, the step size $ h_i = t_{i+1} - t_i $ satisfies $ h_i \geq \ul h
	$.
	
	Consider the state of the scaffold $ \calD $ after the $ i $-th
	iteration of the loop. At this stage, $ \calD $ is an $ i $-step
	scaffold. By the design of the $\Extender$ algorithm, we have 
	\[
	(h_i, F_i) \ass \stepA(E_{i-1}, \veps, 1 - t_{i-1}).
	\]
	
	
	Since $ \wmax(E_{m-1}) \leq \veps $, let $ \bfx_c \in \ivp(m(B_0), 1)
	$. Then, it holds that 
	\[
	E_{m-1} \ib \bfx_c(T_i) + [-2\veps, 2\veps]^n.
	\]
	
	By \reflem{ad}, if 
	\[
	(h(t), F(t)) \ass \stepA\big(\bfx_c(t) + [-2\veps, 2\veps]^n, \veps, 1
	- t\big),
	\] 
	then $ h(t_i) \leq h_i $.
	
	Define 
	\[
	\ul h \ass \min_{t \in [0, 1]} h(t).
	\]
	Since $ h(t) > 0 $ for all $ t \in [0, 1] $,  we conclude that $ \ul h
	> 0 $. 
	
	Thus, for any $ i > 0 $ in the loop, the step size satisfies $ h_i =
	t_{i+1} - t_i \geq \ul h $. This ensures that the loop will terminate
	in a finite number of iterations.
	\epf

%%%%%%%%%%%%%%%%%%%%%%%%%%%%%%%%%%%%%%%%%%%%%%%%%%
%\sect{Appendix}
%%%%%%%%%%%%%%%%%%%%%%%%%%%%%%%%%%%%%%%%%%%%%%%%%%
%%%%%%%%%%%%%%%%%%%%%%%%%%%%%%%%%%%%%%%%%%%%%%%%
%\section*{Appendix B: The  affine map $\ol\pi$}
\section{Appendix B: The  affine map $\ol\pi$}
%%%%%%%%%%%%%%%%%%%%%%%%%%%%%%%%%%%%%%%%%%%%%%%%%%
	Consider the condition \refeq{0ninolB1}.
	Let 
	Without loss of generality, assume
	$0 \notin \olI_1$.  To further simplify our
	notations, we assume
	\beql{assume1}  
	\olI_1 >0.
	\eeql  
	In case $\olI_1<0$, we shall indicate the necessary changes
	to the formulas.
	We first describe an invertible linear map
	$\wt\pi:\RR^n\to\RR^n$
	such that
	\beql{pos}
	\wt\pi(\bff(B_1))>\1=(1\dd 1)
	\qquad\text{(Greater-than-One Property of $\wt\pi$)}
	\eeql
	Note that \refeq{pos} means that for each $i=1\dd n$,
	the $i$th component $(\wt\pi(\bff(B_1)))_i$ is greater than one.
	
	To define $\wt\pi$, we first introduce the box $\wtB_1$:
	\beql{olB_1}
	\mmatx{
		\wtB_1
		&\as& Box(\bff(B_1))\\
		&=& \prod_{i=1}^n \olI_i
		& \text{(implicit definition of $\olI_i$)}\\
		&=& \prod_{i=1}^n [\ola_i,\olb_i]
		& \text{(implicit definition of $\ola_i, \olb_i$)}
	} \eeql
	where $Box(S)\in\intbox\RR^n$ is the smallest
	box containing a set $S\ib\RR^n$.
	For instance, $\olI_i=f_i(B_1)$ where $\bff=(f_1\dd f_n)$.
	The assumption \refeq{assume1} says that
	$ \olI_1>0$, i.e., either $\ola_1>0$.
	%Also property (P3) implies $\ola_i<\olb_i$ for all $i$.
	
	We now define the map $\wt\pi:\RR^n\to\RR^n$ as follows:
	$\wt\pi(x_1\dd x_n) = (\wtx_1\dd \wtx_n)$ where
	\beql{wtpi}
	\wtx_i  \as  \clauses{
		\frac{x_i}{\ola_i}	& \rmif\ \ola_i>0,
		& \text{(i.e., $f_i(B_1)>0$)}\\
		\frac{x_i}{\olb_i}	& \eliF\ \olb_i<0,
		& \text{(i.e., $f_i(B_1)<0$)}\\
		x_i+ x_1\big(
		\tfrac{1+\olb_i-\ola_i}{\ola_1}\big)
		& \elsE
		& \text{(i.e., $0\in f_i(B_1)$).}
	}\\
	\eeql
	Note that if \( \olI_1 < 0 \), we only need to modify the third
	clause in \refeq{wtpi} to \( x_i + x_1 \big( \tfrac{1 + \olb_i -
		\ola_i}{\olb_1} \big) \).
	
	
	Observe that $\wt\pi(\wtB_1)$ is generally a parallelopiped,
	not a box.  Even for $n=2$, $\wt\pi(\wtB_1)$ is a parallelogram.
	So we are interested in the box $Box(\wt\pi(\wtB_1))$:
	\beql{B'1}
	\mmatx{
		B'_1 \as Box(\wt\pi(\wtB_1))
		&=& \prod_{i=1}^n I'_i
		& \text{(implicit definition of $I'_i$)}\\
		&=& \prod_{i=1}^n [a'_i,b'_i]
		& \text{(implicit definition of $a'_i, b'_i$)}
	}
	\eeql
	Then we have the following results:
	
	\bleml[B'1] \ 
	\benum[(a)]  
	\item  $\wt\pi$ is an invertible linear map given by
	\beql{olpiolA}
	\wt\pi(\bfx) = \olA\Bigcdot\bfx  \eeql
	$ \frac{1}{\ola_i}$, $ \frac{1}{\olb_i}$ or $1$ along
	the diagonal and other non-zero entries in column 1 only,
	Here’s the revised version with improved clarity and formatting:
	
	\[
	\mmat{
		v_1 & & & & \\
		c_2 & v_2 & & & \\
		c_3 & & v_3 & & \\
		\vdots & & & \ddots & \\
		c_n & & & & v_n
	}
	\label{olA}
	\]
	where
	$$v_i=
	\clauses{
		\frac{1}{\ola_i} & \rmif\ \ola_i>0,\\
		\frac{1}{\olb_i}	& \eliF\ \olb_i<0,\\
		1	& \elsE }$$
	$$c_i=
	\clauses{
		0 &\rmif\ 0\nin f_i(B_1),\\
		\tfrac{ 1+ \olb_i-\ola_i}{\ola_1} &\elsE. }$$
	\item The box $Box(\wt\pi(\wtB_1)) = \prod_{i=1}^n I'_i$
	is explicitly given by
	\beql{I'i}
	I'_i = 
	\begin{cases}
		\left[1, \frac{\olb_i}{\ola_i}\right] & \text{if } \ola_i > 0, \\[10pt]
		\left[1, \frac{\ola_i}{\olb_i}\right] & \text{else if } \olb_i < 0, \\[10pt]
		\left[1+\olb_i, \frac{\olb_1}{\ola_1} 
		\big(1 + \olb_i\big(1 + \frac{\ola_1}{\olb_1}\big) - \ola_i\big)\right] 
		& \text{else  }.
	\end{cases}
	\eeql
	
	\item
	The map $\wt\pi$ has the
	positivity property of \refeq{pos}.
	\eenum
	\eleml
	\bpf
	\benum[(a)] 
	\item
	From the definition of $\wtpi$ in \refeq{wtpi},
	we see that the matrix $\olA$ matrix 
	the form described in the lemma.  
	This matrix is clearly invertible.
	\item
	We derive explicit formulas for $I'_i$
	in each of the 3 cases:
	\bitem
	\item If $\ola_i>0$, then it is clear that
	$(\wt\pi(B_1))_i= \left[1, \frac{\olb_i}{\ola_i}\right]$.
	\item Else if $\olb_i<0$, it is also clear that
	$(\wt\pi(B_1))_i= 	\left[1, \frac{\ola_i}{\olb_i}\right]$.
	\item Else, we consider an arbitrary 
	point $\bfx=(x_1\dd x_n)\in \wtB_1$:
	{\small \beqarrys
		(\wt\pi(\bfx))_i &=& x_i+ x_1
		\big(\tfrac{1+\olb_i-\ola_i}{\ola_1}\big)
		&\text{(by definition)}\\
		&\ge& \ola_i +\ola_1\big(
		\tfrac{1+\olb_i-\ola_i}{\ola_1}\big)\\
		&&\text{($x_j\in[\ola_j,\olb_j]$ ($\forall~ j$)
			\& $(1+\olb_i-\ola_i)/\ola_1 > 0$))}\\
		&=& 1+\olb_i.\\
		(\wt\pi(\bfx))_i &=&
		x_i+ x_1\big(
		\tfrac{1+\olb_i-\ola_i}{\ola_1}\big)\\
		&\le& \olb_i +\olb_1\big(
		\tfrac{1+\olb_i-\ola_i}{\ola_1}\big)\\
		&&\text{($x_j\in[\ola_j,\olb_j]$ 
			and $(1+\olb_i-\ola_i)/\ola_1 > 0$)}\\
		&=& \tfrac{\olb_1}{\ola_1}
		\big(1+\olb_i(1+\tfrac{\ola_1}{\olb_1})-\ola_i\big).
		\eeqarrys}
	Since both the upper and lower bounds are
	attainable, they determine the interval $I'_i$
	as claimed.
	\eitem
	\item It is sufficient to show that
	$I'_i\ge1$.
	This is clearly true for the first two clauses of \refeq{I'i}.
	For the last two clauses, we have $I'_i\ge 1+\olb_i$
	by part(b).  The result follows since $0\le \olb_i$.
	\eenum
	\epf
	
	
	Let 	\beql{wtB_1}
	\mmatx{
		B^{*}_1
		&\as& Box(\wt\pi(B_1))\\
		&=& \prod_{i=1}^n I^{*}_i
		& \text{(implicit definition of $I*_i$)}\\
		&=& \prod_{i=1}^n [a^{*}_i,b^{*}_i]
		& \text{(implicit definition of $a^{*}_i, b^{*}_i$)}
	} \eeql
	We now define the affine map
	$\ol\pi:\RR^n\to\RR^n$:
	\beql{olpi}
	\mmatx{
		\ol\pi(\bfx) &=& (\olpi_1(x_1),\olpi_2(x_2)\dd\olpi_n(x_n)) \\
		&& \qquad \text{ where } \bfx=(x_1\dd x_n) 
		\text{ and } \\
		\olpi_i(x)
		&\as& \wtpi(x)-a^{*}_i +1.
	}
	\eeql
	Then we have the following results, which is property (Q2):
	\bleml[olB1] \ \\
	
	
	$\olpi(B_1)>\bf1.$
	%\item %(c)
	%$ \wmax(Box(\wtpi(\wtB_1))) $ depends on $ \wmax(B_1) $, and and approaches zero as $ \wmax(B_1) \to 0 $.
	
	\eleml
	\bpf
	The conclusion follows from the fact that 
	$\wtpi(B_1)\ib \prod_{i=1}^n [a^{*}_i,b^{*}_i]$ and 
	$\olpi(B_1)=\wtpi(B_1)-(a^{*}_1\dd a^{*}_n)+1$.
	
	\epf
	
%%%%%%%%%%%%%%%%%%%%%%%%%%%%%%%%%%%%%%%%%%%%%%%%%%
% END
%%%%%%%%%%%%%%%%%%%%%%%%%%%%%%%%%%%%%%%%%%%%%%%%%%

