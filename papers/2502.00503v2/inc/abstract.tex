%%%%%%%%%%%%%%%%%%%%%%%%%%%%%%%%%%%%%%%%%%%%%%%%%%
\begin{abstract}
	The Initial Value Problem (IVP) is concerned with finding
	solutions to a system of autonomous
	ordinary differential equations (ODE)
				\begin{equation}
				\textbf{x}' = \textbf{f}(\textbf{x})
					\label{eq:abs1}
				\end{equation}
	with given initial condition $\textbf{x}(0)\in B_0$
	for some box
	$B_0\subseteq \mathbb{R}^n$.
	Here $\textbf{f}:\mathbb{R}^n\to\mathbb{R}^n$
	and $\textbf{x}:[0,1]\to\mathbb{R}^n$
	where $\textbf{f}$ and $\textbf{x}$ are $C^1$-continuous.
	Let $\texttt{IVP}_\textbf{f}(B_0)$ denote the set of all
	such solutions $\textbf{x}$.
	Despite over 40 years of development to design
	a validated algorithm for the
	IVP problem, no complete algorithm currently exists.

	In this paper, we introduce a novel
	way to exploit the theory of \textbf{logarithmic norms}:
	we introduce the concept of a \textbf{radical transform}
	$\pi:\mathbb{R}^n\to\mathbb{R}^n$
	to convert the $(\textbf{x},\textbf{f})$-system 
	in (\ref{eq:abs1}) into another
	system $\textbf{y}' = \textbf{g}(\textbf{y})$ so that
	the $(\textbf{y},\textbf{g})$-space has negative
	logarithmic norm in any desired small enough neighborhood.
	
	Based on such radical transform steps,
	we construct a complete validated algorithm for the
	following \textbf{End-Enclosure Problem}:
	\begin{center}
			INPUT:\hspace*{5mm} $(\textbf{f}, B_0,\varepsilon)$,\\
			OUTPUT:\hspace{2mm} $(\underline{B}_0,B_1)$,
	\end{center}
	where $B_0\subseteq \mathbb{R}^n$ is a box, $\varepsilon>0$,
	and 
			$(\underline{B}_0,B_1)$,
	such that
		$\underline{B}_0\subseteq B_0$, 
		the diameter of $B_1$ is at most $\varepsilon$,
		and
		$B_1$ is an end-enclosure for $\texttt{IVP}(\underline{B}_0)$,
		i.e., for all
			$\textbf{x}\in \texttt{IVP}(\underline{B}_0)$,
			$\textbf{x}(1)\in B_1$.
	This is a ``promise'' algorithm in the
	sense that the input contains a promise that for each
	$\textbf{p}_0\in B_0$,
	there is a unique solution
	$\textbf{x}\in \texttt{IVP}(B_0)$ with $\textbf{x}(0)=\textbf{p}_0$.

	A preliminary implementation of our algorithm shows promise.
\end{abstract}

