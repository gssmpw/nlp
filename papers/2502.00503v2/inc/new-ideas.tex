%%%%%%%%%%%%%%%%%%%%%%%%%%%%%%%%%%%%%%%%%%%%%%%%%%
\ssect{How we exploit negative Logarithmic Norms}
%%%%%%%%%%%%%%%%%%%%%%%%%%%%%%%%%%%%%%%%%%%%%%%%%%
	Many authors have tried to exploit properties
	of contractive systems;
	see Sontag \cite{sontag:contractive:10} for an overview.
	An early paper is
	Adams and Ames \cite{adams-ames:contracting:78}.
	%Adams and Ames \cite{adams-ames:contracting:78} show that
	%when the approximate differential operator is quasi-monotone,
	%one can exploit contraction properties of iterations.
	A \dt{logNorm bound} of $B_1\ib \RR^n$ is any upper bound on
		%%
		\beql{lognormbound}
		\mu_2(J_\bff(B_1))\as \sup \set{
					\mu_2(J_\bff(\bfp)) : {\bfp\in B_1}}
		\eeql
		%%
	We call $B_1$ a \dt{contraction zone} if it has a
	negative \lognorm\ bound.
	Here, $J_\bff$ is the Jacobian of $\bff$
	and $\mu_2$ is the \lognorm\ function
	% (\cored{Section 2.4}).
	(\refSSec{lognorm}).
	We exploit the fact that
		$$\|\bfx(t;\bfp_0)-\bfx(t;\bfp_1)\|
			\le \|\bfp_0-\bfp_1\| e^{t\olmu}$$
	(\refThm{ne} in \refSSec{lognorm})
	In \refFig{Volterra-21-13}, the exact contraction zone is the
	region above the green parabola.  In tracing a solution
	$\bfx(t;\bfp_0)$ for $t\in [0,h]$ through a contraction zone,
	we can compute a end-enclosure $B$
	for $\IVP(B_0,h,B_1)$ with $\wmax(B)<\wmax(B_0)$
	(i.e., the end-enclosure is ``shrinking'').
	This is superior to any Taylor bounds such as
		$\bfx(t; B_0)= B_0+ \bff(B_0)t + \cdots$
	which cannot shrink.

	We take this idea a step further:
	if $B_1$ is a non-contraction zone,
	we introduce an ``radical map'' $\pi:\RR^n\to\RR^n$
	with $\bfy=\pi(\bfx)$ such that
	the system $\bfx'=\bff(\bfx)$ transforms to
	the system $\bfy'=\bfg(\bfy)$ such that $\pi(B_0)$
	is a contraction zone in the $(\bfy,\bfg)$-space.
	By computing a good end-endclosure in the $(\bfy,\bfg)$-space,
	we infer a corresponding end-enclosure in the $(\bfx,\bff)$-space.
	\refSSec{comparison} gives numerical
	evidence that such an approach is quite effective.

	%Following the lead of Neuemair 
	%\cite{neumaier:theoryI:94}, we want to exploit the
	%theory of logarithm norms (log norm for short)
	%\cite{strom:log-norm:75, soderlind:log-norm-history}.



%%%%%%%%%%%%%%%%%%%%%%%%%%%%%%%%%%%%%%%%%%%%%%%%%%
% END
%%%%%%%%%%%%%%%%%%%%%%%%%%%%%%%%%%%%%%%%%%%%%%%%%%

