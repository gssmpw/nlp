% file: keylemma.tex
%    from new-approach-KeyLemma.tex


%%%%%%%%%%%%%%%%%%%%%%%%%%%%%%%%%%%%%%%%%%%%%%%%%%
\ssect{Key Lemma}
	Consider an admissible triple $(E_0, h, F_1)$.
	By the validity of $\ivp(E_0,h)$, the following	
	condition can be achieved if $E_0$ sufficiently shrunk:

		\beql{0ninolB1}
			\0\nin \olF_1 \as Box(\bff(F_1)) = \prod_{i=1}^n \olI_i.
		\eeql  

	This implies that there exists some $i =1\dd n$ such that
	$0 \notin \olI_i$. 

	\ignore{% MOVE BELOW
		Wlog, say
			$0 \notin \olI_1$.  To further simplify our
				notations, we assume
			\beql{assume1}  
				\olI_1 >0.
			\eeql  
			In case $\olI_1<0$, we shall indicate the necessary changes
			to the formulas.
	}%
	Recall in \refSSec{keyIdea}
	that in the hard case, we compute the
	map $\pi = \whpi \circ \olpi$.
	Define the box $B_2$ and $\chb_{\max}$
		\beql{translation}
			B_2 \as Box(\olpi(F_1)) = \prod_{i=1}^n [1,\chb_i].
					\quad
              \chb_{\max} \as \max_{i=1\dd n} \chb_i.
		\eeql

	Using $\olpi$, we can introduce an intermediate ODE system
		with new differential variables
			\beql{olbfy}
					\olbfy \as \olpi(\bfx)
			\eeql
		and algebraic function
			\beql{olbfg1}
				\olbfg(\olbfy) \as J_{\olpi} \Bigcdot
							\bff(\olpi\inv(\olbfy))
			\eeql
		satisfying the ODE
				\beql{olbfy'}
					\olbfy' = \olbfg(\olbfy)
				\eeql
		and
			\beql{olbfg}
				\olbfg(\olpi(F_1))\ge \1= [1\dd 1].
			\eeql
	% Note that $J_{\olpi}$ is the constant matrix $ \olA$,
	% and $J_{\olpi\inv}$ is $(\olA)\inv$.
	Note that $(\pi(E_0), h, \pi(F_1))$ is
	an admissible triple in the $(\bfy,\bfg)$-space.



	\bthmT[keylemma]{Key Lemma} \ \\
	\benum[(a)]
	\item
		\beqarrys
	 	\mu_2 \big(J_{\bfg} (\pi(F_1))\big)
			&\le&
 			\max\set{\tfrac{-(d_i+1)}{\chb_i}:
							i=1\dd n}\\
		&&
			+ \max_{i=1}^n \set{d_i} 
        		\cdot
					\|J_{\olbfg}(\olpi(F_1))\|_2
				\cdot
				\max_{i=1}^n \set{\tfrac{(\chb_i)^{d_i+1}}{d_i}}.
		\eeqarrys
	\item If $d_1=\cdots=d_n=d$ then
			$$\mu_2 \left(J_{\bfg} (\pi(F_1))\right)
				\le
					-(d+1)\tfrac{1}{\chb_{\max}}
						+(\chb_{\max})^{d+1}
								\|J_{\olbfg}(\olpi(F_1))\|_2.
			$$
	\eenum
	\ethml

	\savespace{
	\bpf
	From \refLem{Jbfg}(b)
	we have for any $\bfp=(p_1\dd p_n) \in \olpi(F_1)$,
		\beql{splitJacobian1}
			J_{\bfg}(\whpi(\bfp))
			= A(\bfp) + P\inv(\bfp)
				\frac{\partial \olbfg}{\partial \bfx}(\bfp)P(\bfp)
		\eeql
	where
       $P(\bfp)=\diag\big(
	   			\tfrac{p_i^{d_i+1}}{d_i}: i=1\dd n \big)$
        and
		$A(\bfp) = \diag(a_1\dd a_n)$ with
        \beql{aid1}
				a_i \as -d_i(1 + \tfrac{1}{d_i}) p_i\inv
					\cdot (\olbfg(\bfp))_i. \eeql
	Thus, $A, P$ are diagonal matrices and $p_i\inv$ is well-defined
	since $\bfp\in B_2\ge \1$, \refeq{translation}.
	
	By \refLem{lognorm}(b) and \refeq{aid1}, we conclude that
	the form
		\beql{mudiag}
			\mu_2(A(\bfp))=	\mu_2(\diag(a_1\dd a_n))
					= \max\set{a_i: i=1\dd n}.
		\eeql

	From \refeq{olbfg}, we conclude that
	{\small
	\beqarrys
    \mu_2\big(J_{\bfg}(\whpi(\bfp))\big) 
		&=& \mu_2\left(A(\bfp) + P\inv(\bfp)\frac{\partial \olbfg}{\partial
			\bfx}(\bfp)P(\bfp)\right)\\
			&& \text{(by \refeq{splitJacobian1})} \\
		&\le& \mu_2(A(\bfp))
			+ \mu_2\left(P\inv(\bfp)\frac{\partial \olbfg}{\partial
				\bfx}(\bfp)P(\bfp)\right)\\
			&& \text{(by \refLem{lognorm}(a))} \\
		&\le& \mu_2(A(\bfp)) 
			+ \left\|P\inv(\bfp)\frac{\partial \olbfg}{\partial
				\bfx}(\bfp)P(\bfp)\right\|_2 \\
			&& \text{(by \refLem{lognorm}(b))} \\
        &\le& \max\set{\tfrac{-(d_i+1)}{\chb_i}:
							i=1\dd n}\\
		&&
			+\left\|P\inv(\bfp)\right\|\left\|\frac{\partial \olbfg}{\partial
				\bfx}(\bfp)\right\|\left\|P(\bfp)\right\|\\
					&& \text{(by \refLem{matrixnorm}(b))}\\
        &\le& \max\set{\tfrac{-(d_i+1)}{\chb_i}:
							i=1\dd n}\\
		&&
			+ \max_{i=1}^n \set{d_i} 
        		\cdot
					\|J_{\olbfg}(\olpi(F_1))\|_2
				\cdot
				\max_{i=1}^n \set{\tfrac{(\chb_i)^{d_i+1}}{d_i}}.
	\eeqarrys
	}
	\epf
	}% savespace
	
	Until now, the value of $\bfd$ in the radical map $\whpi$ was 
	arbitrary. We now specify $\bfd = \bfd(F_1)$.
	The definition of $\bfd$ is motivated by \refThm{keylemma}.
	The optimal choice of $\bfd$ is not obvious. So we make
	a simple choice by restricting
	$d_1=\cdots=d_n =d$.  In this case, we could choose:

	\beql{d}
	d=d(F_1) \as\max\Big\{1,
			\quad 2 \| J_{\olbfg}(\olpi(F_1)) \|_2 -1 \Big\}.
	\eeql
	Although the theory allows $d\ge 1$ to be an arbitrary real,
	in practice, we use the ceiling of $d$.
	
	\bleml[Set-d]\
		Assuming $d$ satisfies \refeq{d}, we have:
	\benum[(a)]
	\item
		$\mu_2 \left(J_{\bfg} (\pi(F_1))\right)
				\le	 (-2+(\chb_{\max})^{d+2})
				\cdot \frac{\|J_{\olbfg}(\olpi(F_1))\|_2}{\chb_{\max}}.$
	\item
		If
			$\log_2(\chb_{\max}) < \tfrac{1}{d+2}$
		then
			$\mu_2 \left(J_{\bfg} (\pi(F_1))\right)< 0$.
	\eenum
	\eleml
	
	\savespace{
	\bpf
	%\benum[(a)]
	\benum
	\item
	By \refThm{keylemma} we have
	{\small
	\beqarrays
		\mu_2\left(J_{\bfg} (\pi(F_1))\right)
		&\le& -(d+1)\frac{1}{\chb_{\max}}
					+(\chb_{\max})^{d+1}\|J_{\olbfg}(\olpi(F_1))\|_2\\
		&=& \Big(\tfrac{-(d+1)}
				{\|J_{\olbfg}(\olpi(F_1))\|_2}+(\chb_{\max})^{d+2} \Big)
				\cdot \tfrac{\|J_{\olbfg}(\olpi(F_1))\|_2}{\chb_{\max}}\\ 
		&&	 \text{(by factoring)} \\
		&\le&  \Big(-2+(\chb_{\max})^{d+2} \Big)
				\cdot \tfrac{\|J_{\olbfg}(\olpi(F_1))\|_2}{\chb_{\max}} \\
		&&
			\text{(By eqn.\refeq{d}, we have $(d+1)
				\ge 2(
				\|J_{\olbfg}(\olpi(F_1))\|_2)$)}. 
	\eeqarrays
	}
	\item
		Since $(\chb_{\max})^{d+2}<2$
			is equivalent to 
			$\log_2(\chb_{\max}) < \tfrac{1}{d+2}$,
		we conclude that
			$\mu_2 \left(J_{\bfg} (\pi(F_1))\right)< 0$.
	\eenum
	\epf

	}% savespace

	Given an admissible triple $(E_0,h,F_1)$,
	we introduce a subroutine called \Transform$(\bff,F_1)$ to
	convert the diffential equation
	$\bfx'=\bff(\bfx)$ to $\bfy'=\bfg(\bfy)$ according to above map
	$\pi$.   However, this transformation depends on the
	condition $\refeq{0ninolB1}$.  So we first define the
	following predicate \AvoidsZero($\bff,F_1$):

	\Ldent\progb{
		\lline[0] 
		\AvoidsZero$(\bff,F_1) \ssa$ \true\ or \false.  
		\lline[5] INPUT: $F_1 \ib \intbox\RR^n$.
		\lline[5] OUTPUT: \true\ if and only if
				$\bf0 \notin \text{Box}(\bff(F_1))$.
		% \lline
		% \lline[10] For $i=1\dd n$
		% \lline[15]	If $0 \nin f_i(F_1)$, return \true.
		% \lline[15] Return \false
	}

	Now we may define the transformation subroutine:

	\Ldent\progb{\label{alg:transform}
		\lline[0] 
			\Transform$(\bff,F_1,\olmu^1) \ssa (\pi, \bfg, \olmu)$ 
		\lline[5] INPUT: $F_1\ib \intbox\RR^n$
					where $\0\nin Box(\bff(F_1)$, $\olmu_1$.
		\lline[5] OUTPUT: $(\pi, \mu,\bfg)$ where
		\lline[10] $\pi$ and $\bfg$ satisfy
				 \refeq{circ} and \refeq{bfy'}.
		\lline[10] If (\AvoidsZero($\bff,F_1$=\false \& $ \olmu_1\ge 0$)
		\lline[15] Return ($\id,\bff, \olmu$)
		\lline[10] Compute $\olpi$ to satisfy \refeq{olbfg}
		\lline[10] Compute $\pi$ and $\bfg$ 
				 according \refeq{circ} and \refeq{bfy'}.
		\lline[10] $\olmu \ass \mu_2(J_{\bfg}(\pi(B)))$.
		\lline[10] Return ($\pi,\bfg, \olmu$)
		}
%%%%%%%%%%%%%%%%%%%%%%%%%%%%%%%%%%%%%%%%%%%%%%
\ssect{Transformation of Error Bounds}

	We want to compute a transformation
		$$\delta_x\mapsto \delta_y$$
	such that
	if $B$ is a $\delta_y$-bounded end-enclosure
		for $(\pi(E_0),h, \pi(F_1))$
		in the $(\bfy,\bfg)$-space, then
	then $\pi\inv(B)$ is 
		a $\delta_x$-bounded end-enclosure
	of $(E_0,h,F_1)$ in the $(\bfx,\bff)$-space.
	The following lemma achieves this:
	
	\savespace{
	\bleml[error-bound-under-phi]
		Let $\bfp,\bfq\in B\ib\RR^n$ and $\phi\in C^1(F_1\to \RR^n)$, then 
		$\|\phi(\bfp)-\phi(\bfq)\|_2
		\le \|J_{\phi}(B)\|_2 \cdot \|\bfp-\bfq\|_2$
	\eleml
	\bpf
		\beqarrys
		\|\phi(\bfp)-\phi(\bfq)\|_2
		&\le&
		 \|\phi(\bfq)+J_{\phi}(\xi)\Bigcdot(\bfp-\bfq) -\phi(\bfq)\|_2\\
		 	&& \text{(by Taylor expansion of $\phi(\bfp)$ at $\bfq$)}\\
		&=&\|J_{\phi}(\xi)\Bigcdot(\bfp-\bfq)\|_2\\
		&\le& \|J_{\phi}(\xi)\|_2 \cdot \|(\bfp-\bfq)\|_2\\
		&\le& \|J_{\phi}(B)\|_2   \cdot \|(\bfp-\bfq)\|_2,
		\eeqarrys
	where $\xi\in B$.
	\epf
	}% savespace
	

	\bleml[error-bound-ode]
	\ \\
	Let $\bfy=\pi(\bfx)$ and 
		$$
		\mmatx[rcl]{
			\bfx 	&=&
								\ivp_{\bff}(\bfx_0,h,F_1),\\
			\bfy 	&=&
								\ivp_{\bfg}(\pi(\bfx_0),h,\pi(F_1)).
			}$$
	%%%
	For any $\delta_x>0$ and any point $\bfp\in \RR^n$ satisfying 
		\beql{delta1}
			\|\pi(\bfp)-\bfy(h)\|_2
				\le \delta_y
				\as \frac{\delta_x}{\|J_{\pi\inv}(\pi(F_1))\|_2},
		\eeql
	we have 
			$$\|\bfp-\bfx(h)\|_2\le \delta_x.$$ 
	\eleml
	\savespace{
		\bpf
	\beqarrys
		\|\bfp-\bfx(h)\|_2
			&=& \|\pi\inv(\pi(\bfp))-\pi\inv(\pi(\bfx(h)))\|_2\\
			&=& \|\pi\inv(\pi(\bfp))-\pi\inv(\bfy(h))\|_2\\
			&\le&  			  
				\|J_{\pi\inv}(\pi(F_1))\|_2
					\cdot \|\pi(\bfp)-\bfy(h)\|_2\\
			&& \text{(by \refLem{error-bound-under-phi})}\\
			&\le& \delta\qquad
				\text{(by condition \refeq{delta1}.)}
	\eeqarrys
	\epf
	}% savespace
	
	% It follows that
	% 	$$\delta_y = \clauses{
	% 			\delta_x & \rmif\ \pi=id,\\
	% 			\frac{\delta}{\|J_{\pi\inv}(\pi(F_1))\|_2}.$$


	\Ldent\progb{
		\lline[0]  $\TransformBound(\delta,\pi,F_1) \ssa \delta'$
		\lline[5]	INPUT: $\delta>0$, $\pi, F_1$ as above.
		\lline[5]	OUTPUT: $\delta'>0$ satisfying the corollary.
		\lline[10]	If ($\pi$ is the identity map)
		\lline[15]		Return $\delta$.
		\lline[10]	Else
		\lline[15]		Return 
				$\frac{\delta}{\|J_{\pi\inv}(\pi(F_1))\|_2}$.
	}
\ignore{
%%%%%%%%%%%%%%%%%%%%%%%%%%%%%%%%%%%%%%%%%%%%%%%%%%
\sect{Comparing our Transform Method with Euler's Method}
	Let $(E_0, h, F_1)$ be an admissible triple, and 
	$\olmu^1 = \mu_2(J_{\bff}(F_1)) \geq 0$.
	
	Assume that $\pi$ and $\bfg$ are derived from 
	\refeq{pi} and \refeq{bfg}, respectively, and that 
	$\olmu^2 = \mu_2(J_{\bfg}(\pi(F_1))) < 0$.
	
	Let $Ball_{\bfp}(r_0) $ be the circumball of $E_0$. For a given
	precision $\delta > 0$, 
	we can compute a point $\bfQ$ by tracing 
	$\bfx \in \ivp(\bfp, h)$ such that 
	$\|\bfQ - \bfx(h)\| < \delta$. 
	The resulting end-enclosure for $\ivp(E_0, h)$ is
	$Ball_\bfQ\left(r_0 e^{\olmu^1 h} 
	+ \delta\right)$. 
	
	Using our transform method (illustrated in Figure 
	$\ref{fig:computerenclosure}$), we can obtain a point 
	$\bfq$ such that 
	$\|\pi\inv(\bfq) - \bfx(h)\| \leq \delta$. 
	The corresponding end-enclosure for $\ivp(E_0, h)$ is 
	$Ball_{\pi\inv(\bfq)}\left(r_0e^{\olmu^1 h} 
	+\delta\right)$.
	
	Furthermore,  let $Ball_{\pi(E_0)}(r'_0)$ be the circumball
	of $\pi(E_0)$ and $d_m\as \|\pi(m(E_0))-m(\pi(E_0))\|$ then 
	\beql{anotherend-enclosure}
		F\as \pi\inv\left(Ball_{\bfq}((r'_0+d_m) e^{\olmu^2 h} 
		+  \delta_1)\right)
	\eeql
	is another end-enclosure for $\ivp(E_0, h)$. Therefore, by 
	taking the intersection of enclosures, we can always 
	obtain a tighter end-enclosure.  
	
	We have the following lemma:
	
	\blem  \ \\
	The diameter of  $F$ \refeq{anotherend-enclosure}
	is less than  
	$
		2 J_{\pi\inv}(\pi(F_1)) J_{\pi}(F_1) (r'_0+d_m)
		e^{\olmu^2 h} + 2 \delta.
	$  
	\elem  
	
	%\bpf
\begin{pf}
	\beqarrays  
			\pi(E_0)
		&\ib& 	
			\pi(\bfp+[-r_0,r_0]^n)
		\\
		&\ib&
			\pi(\bfp)+ J_{\pi}(F_1)\Bigcdot [-r_0,r_0]^n
		\\
		&& \text{(by  mean value theorem.)}			
	\eeqarrays
	Thus, 
	$r'_0\le \|J_{\pi}(F_1)\|r_0.$
	\beqarrays  
		F
	&\ib& 	
		\pi\inv(\bfq)+ J_{\pi\inv}(\pi(F_1))\Bigcdot
		Box((r'_0+d_m)e^{\olmu^2h}+\delta_1).\\
	&& \text{(by mean value theorem)}			
	\eeqarrays
	Thus
	\beqarrays  
		diameter(F)
	&\le& 
		2\|J_{\pi\inv}(\pi(F_1))\|((r'_0+d_m)e^{\olmu^2h}+\delta_1)
	\\
%	&\le&  
	%	2 \|J_{\pi\inv}(\pi(F_1))\| \|J_{\pi}(F_1)\|  (r_0+d_m)
	%	e^{\olmu^2 h} + 2 \|J_{\pi\inv}(\pi(F_1))\| \delta_1
%	\\	
	&\le& 
		2 \|J_{\pi\inv}(\pi(F_1))\| \|J_{\pi}(F_1)\|  (r_0+d_m)
		e^{\olmu^2 h} +2\delta 
	\\
	&&  \text{(by \refeq{delta1})}	
	\eeqarrays
	%\epf  
\end{pf}
	%
	%		{\scriptsize
		%		\btable[c |c| c| c | c | c | c|c|c]{
			%			\hline
			%			Cases &  $E_0$ & $F_1$ & $h$ & $\delta$ &$\olmu^1$ &$\olmu^2$ & Len1 & Len2
			%			\\\hline
			%			Example 1& $Box_{(3,3)}(10^{-4})$ &$[2.9, 3.0]\times [2.9, 3.0]$ & $2.37\times 10^{-4}$&0.001 & 5.90 & -62.70 & 0.002 & 0.0009\\
			%			\hline
			%			Example 2 & $Box_{(3,3)}(10^{-4})$ & $[2.9, 3.0]\times [2.9, 3.0]$ & $5.34\times 10^{-4}$&0.001 & 7.00 & -20.28 &0.002 & 0.0002\\
			%			\hline
			%			Example 3 &$Box_{(3,3)}(10^{-4})$ & $[2.9, 3.0]\times [2.9, 3.0]$ & $2.23\times 10^{-4}$&0.001 & 5.39 & -52.66 &0.002 & 0.0004\\
			%			\hline
			%		}
		%	}%
	%	

%	Next, we compare the Euler step size:
	%	We have the following result:
	%
	%	
	%	
	%	Let $ M_{\bff}=\|\bff^{[2]}(F_1)\|, M_{\bfg}=\|\bfg^{[2]}(\pi(F_1))\|$
	%	then we have:
	%	\blem
	%	$M_{\bfg}\le J^2_{\pi}(F_1)J_{\pi\inv}(\pi(F_1))M_{\bff}$
	%	\elem
	%	
	%	We can compute the point $\bfQ$ by Euler method with
	%	step  size
	%             \beql{euler-size}
	%	       h\le \frac{2\olmu^1\delta}{M_{\bff}(e^{\olmu^1h}-1)},
	%            \eeql
	%	and
	%	we can similarly compute the point $\bfq$ using 
	%	the step size:
	%            \begin{align}
	%                h_1 &\le \frac{2\olmu^2\delta'}{M_{\bfg}   
	%                        (e^{\olmu^2h}-1)} 
	%                        \label{eq:transform-size}\\
	%                &\le \frac{2\olmu^2\delta}{M_{\bff}    
	%                    (e^{\olmu^2h}-1)}\frac{1}{ J^2_{\pi}
	%							(F_1)J^2_{\pi\inv}(\pi(F_1))} 
	%                    & \text{(by above lemma and \refeq{delta1})}.
	%                         \label{eq:transform-size2}
	%            \end{align}
	%
	%	The relative sizes of $h$ and $h_1$ 
	%    in \refeq{euler-size} and \refeq{transform-size} determine the
	% efficiency of the two methods.  A more direct but cruder comparison
	% is to  use \refeq{euler-size} and \refeq{transform-size2}, but in
	% this case, the transform method appears worse.
	%	

    Here are some examples and experimental data:
	The table lists the parameter $\veps$ from \stepa, where $\olmu^1$
	represents the logarithmic norm of the original IVP. The parameters
	$d_1$ and $d_2$ correspond to the radical map used in the Transform,
	with $d_1$ defined by equation \refeq{d}.  

\ssect{Some Examples}
    We will use the following running examples in our experiments:

		\begin{Example}[ 0.]
            This is the trivial $x'=ax$ with solution $x(t)=e^{at}$.
            Clearly, the logarithmic norm of this system is
            negative iff $a<0$.
        \end{Example}

	\begin{Example}[ 1. Volterra System]
            This is the Predator-Prey equation for
            $\bfx\in \ivp(E_0,1)$ where
            \beql{volterra}
			\bff=\mmat{x \\ y} '
			= \mmat{\phantom{-}ax(1-y) \\ -by(1-x)} ,
                    \qquad E_0 = Box_{(1,3)}( 0.1),
			\eeql
		with $a>1$ and $0<b<1$.  We choose $(a,b)=(2,1)$ and
		$E_0=Box_{(1,3)}(0.1)$ to closely
    track the AWA examples in \cite{moore:diffEqn:09} and
    \cite[p.13]{bunger:taylorODE:20}. 
	Thus $J_{\bff} =\mmat{2(1-y) & -2x\\ y & x-1}$ and the eigenvalues of
	$\half(J_{\bff}+ J_{\bff}^{\tr})$ are
            $$\half(1+x-2y \pm \sqrt{\Delta(x,y)}), \qquad \text{where } 
            \Delta(x,y)\as 5(x^2+y^2)-6(x+2y)+9$$
	  Note that
	  $\Delta(x,y)=5(x-\tfrac{3}{5})^2 + 5(y-\tfrac{6}{5})^2$ is
	  positive. Thus $\sqrt{\Delta(\bfq)}=\sqrt{5}\|\bfq-\bfp_0\|$
	  where $\bfp_0=(\tfrac{3}{5},\tfrac{6}{5})$.
	  Let $\ell(x,y)\as 1+x-2y$.  Thus $\ell(\bfq)> 0$ iff
	  $\bfq$ lies below the line $1+x-2y=0$.
	  \refFig{Volterra-21-13} shows
	  the trace of $\bfx(t)$ with $\bfx(0)=(3,1)$ and $t\in [0, 5.48]$
	  representing one complete
      cycle as computed by MATLAB; Bunger \cite{bunger:taylorODE:20}
      said that AWA cannot complete this computation 

      \FigEPS{Volterra-21-13}{0.2}{Solution to the Volterra system
	  	with $(a,b)=(1,2)$, and $\bfx(0)=(1,3)$ is shown in blue.
		The points $\bfq$ with negative
		$\mu_2(J_\bff(\bfq))$ lie above the green parabola as defined by
		the red line and red dot.}
 
	We have $\mu_2(J_\bff(\bfq)) = \half(\ell(\bfq)+\sqrt{\Delta(\bfq)})$,
	and this can only be negative when $\ell(\bfq)<0$, i.e., when $\bfq$
	lies above the line $\ell(x,y)=0$.  The algebraic locus of the curve
	$\mu_2(J_\bff(\bfq))=0$ is given by the equation
	$|\ell(\bfq)|^2=\Delta(\bfq)$.
    \ignore{
      We have $\sqrt{5}|\ell(\bfq)|$ is the distance
	  of $\bfq$ from the line $\ell=0$.}   This equation is a parabola
	  defined by the line $\ell(x,y)=0$ and the point $\bfp_0$.
 
    \end{Example}

    \begin{Example}[ 2. Van der Pol System]
		\beql{vanderpol}
			\bff=\mmat{x \\ y} '
			= \mmat{y \\ -x^2y + y-x} ,
                    \qquad E_0 = Box_{(-3,3)}( 0.1).
			\eeql
		It could be checked that the logarithmic norm of this system is
		always positive.

        \NOignore{
              $J_f = \mmat{0 & 1 \\ -2xy -1 & 1-x^2}$
              and the two eigenvalues
              of $(J_f+ J_f\tr)/2$ are
              
              $\mmat{ 1-x^2 + \sqrt{(1-x^2)^2 + 4x^2y^2}\\
                     1-x^2 - \sqrt{(1-x^2)^2 + 4x^2y^2}
                         }$,
            and the larger eigenvalue is always positive.
        }
	\end{Example}
	
	\begin{Example}[ 3.]
		Consider the system  
		\[
		\bff=\mmat{x \\ y} '
		= \mmat{x^2 \\ -y^2 + 	7x} , 
            \qquad E_0 = [-2.0, -1.0] \times [8.0, 9.0].
		\]
		The logarithmic norm of this system is initially negative but
		becomes positive. The solution to the $x$-component is $x(t)=
		\tfrac{1}{x_0\inv -t}$ which is negative for all $t>0$ (assuming
		$x_0<0$). Thus the $y$-axis is an asymptote of the solution curve
		(red curve in \refFig{neg-pos-system}). If the end enclosure
		$F_1=Box_{\bfp_1}(w_x, w_y)$ then $w_y/w_x\to \infty$. We have
		$J_\bff=\mmat{2x& 0\\ 7&-2y}$, and the eigenvalues of
		$\half(J_\bff+J_\bff{\tr})$ are
        $(x-y) + \sqrt{(x+y)^2 + (7/2)^2}$.
		        The phase portrait of this system is seen in
		\refFig{neg-pos-system}. The solution
		$\bfx$ with $\bfx(0)=(-1.5, 8.5)$ is shown in red.
	
		\FigEPS{neg-pos-system}{0.3}{Phase Portrait of Example 4}
	
    We check that for any point $\bfp\in\RR^2$ outside
    of the second quadrant $\set{x<0, y>0}$, we have
    the property
    $\mu_2(J_\bff(\bfp))>0$.   Moreover, for
     any solution $\bfx(t)\in \ivp(E_0,1)$ there is a moment 
        $0< t_1<1$ such that
        $\mu_2(J_\bff(x(t)))$ is negative if $t<t_1$
        and positive if $t>t_1$.  
	\end{Example}

     \begin{Example}[ 4. Lorenz System]

     {\scriptsize
     	\beql{lorenz}
       \bff = \mmat{x\\y\\z}' = 
            \mmat{\sigma(y-x)\\ 
                    x(\rho-z)-y\\
                    xy-\beta z},
        E_0 = Box_{(15,15,36)}(0.1),\\
        (\sigma,\rho,\beta)=(10,28,\tfrac{8}{3}).  
          \eeql}

     We have
     {\scriptsize $J_\bff=\mmat{-10 & 10 & 0\\
                    28-z & -1 & -x\\
                    y  & x  & -8/3}
     $} and
     
    {\scriptsize $\half(J_\bff+J_\bff{\tr})=\mmat{-10 & 19-\half z & \half y\\
                    19-\half z & -1 & 0\\
                    \half y  & 0  & -8/3}
     $}.  
	 We could no longer easily compute global bounds on log norms,
	 but only obtain bounds on $\olmu =\mu_2(J_\bff(F_1))$ for given
     admissible pair $(h,F_1)$ for $E_0$
 	E.g. for {\small $(h,F_1)=(0.01, 
        [14.80, 15.12]\times [13.34, 15.10] \times [35.75, 37.45])$}
    we obtain the bound
    $\olmu\le 4.33$.
     \end{Example}

	\ignore{\begin{Example}[ 4.]
	 	Consider the IVP where   
	 	\[
	 	\bff=\mmat{x \\ y} '
	 	= \mmat{x^2-2x+4 \\ -y^2 + 2y-7x-4} ,
                 \qquad
                E_0 = [-2.1, -1.9] \times [6.9, 7.1].
	 	\]
        \end{Example}
    }%ignore
	
     
%%%%%%%%%%%%%%%%%%%%%%%%%%%%%%%%%%%%%%%%%%%%%%%%%%
\ssect{Experimental Comparison of Transformation with Euler} 
%%%%%%%%%%%%%%%%%%%%%%%%%%%%%%%%%%%%%%%%%%%%%%%%%%
		The transformed logarithmic norms is $\olmu^2$, respectively. Let
		$B_E$ denote an end-enclosure obtained using the Euler method
		directly on the original system, and $B_T$ denote an end-enclosure
		obtained using our transform method.
        The following table compares $B_E$ and $B_T$
        with respect to the ratios of the lengths
        and widths, repectively:
	\beql{rho}
            \rho(B_E, B_T) \as 
               \big( \tfrac{\wmax(B_E)}{\wmax(B_T)}, 
                    \tfrac{\wmin(B_E)}{ \wmin(B_T))}\big).
        \eeql

      \refTab{transform} define 
     a set of experiments, where 
     each experiment is based on the above examples,
     but with different numerical parameters:
     these parameters are $(E_0, \veps, F_1, h, \delta)$.  The goal is to compare
     $B_E$ and $B_T$ by using $\rho(B_E, B_T)$.
	\begin{table*}[]
      \centering

	{\scriptsize
	\btable[c| c | c | c | c | c | c | c | c ]{
	\hline
	E.g. & $E_0$ & $F_1$ & $h$ & $\delta$ & $\olmu^1$
		& $d_1$ & $\olmu^2$ & $\rho(B_E,B_T)$ \\\hline
	\hline 
	Eg1 & $Box_{(1,3)}(10^{-4})$ & $[0.9,1.0]\times[2.9,3.0]$
		& 0.0028 & $10^{-7}$ & 0.07 & 17  & -68.30 
			& $(1.0000,1.0000)$  \\
	    & & & & & & 1 & -5.82 &  $(1.0000,1.0000)$ \\\hline
	
	Eg2-a & $Box_{(3,-3)}(10^{-4})$ & $[2.9, 3.0]\times [-3.0,-2.9]$
		& $0.000018$ & $10^{-7}$ & 5.90 & 23
			& -142.73 & $(1.0000,1.0002)$ \\
	      & & & & & & 1 & -11.28 & $(1.0000,1.0003)$ \\\hline
	      
	Eg2-b & $Box_{(3,-3)}(10^{-4})$ & $[2.9, 3.0]\times [-3.0,-2.9]$
		& 0.00086 & $10^{-7}$ & 5.90 & 23 & -140.80
			& $(1.0000,1.0002)$ \\
	      & & & & & & 1 & -11.00 & $(1.0000,1.0007)$ \\\hline
	
	Eg2-c & $Box_{(3,-3)}(10^{-4})$ & $[2.9, 3.0]\times [-3.0,-2.7]$
		& 0.01 & $10^{-7}$ & 5.93 & 23 & -370.14 & $(1.0000,1.0321)$ \\
	      & & & & & & 1 & -9.20 & $(1.0000,1.0458)$ \\\hline
	
	Eg3-a & $Box_{(3,-3)}(10^{-4})$ & $[2.9, 3.0]\times [-3.0,-2.9]$
		& $0.00003$ & $10^{-7}$ & 9.75 & 20 & -180.71
			& $(1.0000,1.0000)$ \\
	      & & & & & & 1 & -10.08 & $(1.0001,1.0003)$ \\\hline
	
	Eg3-b & $Box_{(3,-3)}(10^{-4})$ & $[2.9, 3.0]\times [-3.0,-2.9]$
		& 0.001 & $10^{-7}$ & 9.75 & 20 & -177.05 & $(1.0000,1.0000)$ \\
	      & & & & & & 1 & -9.87 & $(1.0002,1.0008)$ \\\hline
	
	Eg3-c & $Box_{(3,-3)}(10^{-4})$ & $[2.9, 3.2]\times [-3.0,-2.6]$
		& 0.02 & $10^{-7}$ & 10.64 & 20 & -163.12 &  $(1.0000,1.0035)$\\
	      & & & & & & 1 & -9.39 & $(1.0000,1.0665)$ \\\hline
	Eg4-a & $Box_{(1.0,3.0,1.0)}(10^{-4})$
		& $[0.9, 1.0]\times [2.9,3.0]\times[0.9,1.0]$ & $0.001$
			& $10^{-7}$ & 13.60 & 58 & -22.10
			& $(1.0000,1.0000,1.0102)$\\
	      & & & & & & 1 & -2.97 & $(1.0000,1.0000,1.0186)$ \\\hline
	Eg4-b & $Box_{(1.0,3.0,1.0)}(10^{-4})$
		& $[0.9, 1.5]\times [2.9,3.7]\times[0.9,1.0]$ & $0.02$
			& $10^{-7}$ & 13.62 & 10 & -6.01 & $(1.0000,1.0000,1.3286)$ \\
	      & & & & & & 1 & -3.01 & $(1.0000,1.0000,1.3933)$ \\\hline
	}
	}
	\caption{Comparison our transform method with Euler's method}
	\label{tab:transform}
	\end{table*}


	From the experimental results, we can conclude the following:  
	\\ 1. Applying the transformation consistently yields a tighter
	end-enclosure.  
	\\ 2. The benefit of the transformation becomes increasingly
	significant as the step size grows larger.  
	\\ 3. Decreasing the value of $d$ further tightens the end-enclosure.  
	\\ 4. When the given \ivp exhibits significantly faster growth along
	one coordinate axis compared to another, the transformation method
	demonstrates superior performance. This is because the Euler method is
	constrained to using square boxes, whereas the transformation method
	can adaptively use elongated boxes.  


%%%%%%%%%%%%%%%%%%%%%%%%%%%%%%%%%%%%%%%%%%%%%%%%%%
%%%%%%%%%%%%%%%%%%%%%%%%%%%%%%%%%%%%%%%%%%%%%%%%%%
}
