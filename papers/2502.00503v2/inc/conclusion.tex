%%%%%%%%%%%%%%%%%%%%%%%%%%%%%%%%%%%%%%%%%%%%%%%%%%
\sect{Conclusion}
%%%%%%%%%%%%%%%%%%%%%%%%%%%%%%%%%%%%%%%%%%%%%%%%%%
	This paper introduces a novel way to exploit
	contractive systems locally using \lognorm s.
	We presented the first complete validated IVP 
	algorithm with the unique ability to pre-specify the
	desired precision $\vareps$ of the output. 
	Preliminary implementations shows promise.

\savespace{
	It is interesting to contrast our approach to the
	Lohner-type algorithms 
	\cite{nedialkov+2:validated-ode:99}
	where the original differential
	equation $\bfx'=\bff(\bfx)$ is successively transformed into
    $\bfx_1'=\bff_1(\bfx_1)$,
    $\bfx_2'=\bff_2(\bff_2)$, etc.
    The successive computations are
    done in the transformed space of $\bfx_m$ (after $m$ transforms).
    In contrast
    
	Let us give the precise description of the $\bff_i,\bfx_i$
	$\bfx_i'=\bff_i(\bfx_i)$
	for $i=1,2,\ldots$.
	We thus avoid  his iterative sequence of matrix products
	$A_1A_2\cdots A_i$.
}
    
	In \cite[Section 10]{nedialkov+2:validated-ode:99}, 
		% also: Corliss \cite{corliss:survey-ode-intvl:89}
	``{\em Some Directions for Future Research}'',
	the authors
	presented a list of challenges that remain relevant today.
	Our algorithm is one answer to their call for automatic
	step sizes, order control (interpreted as error control)
	and tools for quasi-monotone problems (i.e., contractive systems).
	% the section ``Some Directions for Future Research'',
	% There are still many challenges to address, such as 
		%%
	% 	greater efficiency,
	% 	automatic step sizes, and termination
	% 	Stetter \cite{stetter:validatedODE:90},

\dt{Limitations}:
	\ignore{
	    1. We have only described an abstract algorithm
		and wrote small test examples.
		A further development 
		along the lines of the AIE framework
		\cite{xu-yap:real-zero-system:19}).
    }
	1. As our approach is not a Lohner-type algorithm, it is 
	impacted by the wrapping effect.  This is seen in the growth of
	number of stages $\#(\calD)$ as $\veps$ decreases
	(particularly in Eg3).
	
	2. The efficiency of our algorithm can be further
	improved by carefully
	selecting appropriate parameter values $(\veps, H)$ in \stepA\ and
	optimizing the choice of $\bfd$ in the transform.
	We offer them as options for the user, but such options
	do not affect the correctness of our algorithm.
	



