%%%%%%%%%%%%%%%%%%%%%%%%%%%%%%%%%%%%%%%%%%%%%%%%%%
\sectL[comparison]{Transform Method versus Euler Method}
		Let $(E_0, h, F_1)$ be an admissible triple
	that has been transformed into $(\pi(E_0),h,\pi(F_1))$.
	Let $\olmu^1$ be the \lognorm\ bound for $(\bff,F_1)$
	and
	$\olmu^2$ be the corresponding bound for $(\bfg,\pi(F_1))$.
	Given $\delta_x>0$, we can use the Euler Method to
	compute a point $\bfQ$ such that 
	$\|\bfQ-\bfx(h)\|\le \delta_x$ and
	\beql{end-enclosure0}
	E_1^{euler} \as Box_\bfQ\left(r_0 e^{\olmu^1 h} 
	+ \delta_x\right). \eeql 
	as an end-enclosure for $\ivp(E_0,h,F_1)$, $r_0$ radius
	of the circumball of $E_0$.  Using our $\pi$-transform we can
	first compute a point $\bfq$ such that 
	$\|\bfq-\bfy(h)\|\le \delta_y$ 
	and take its inverse, or we can take the inverse of
	the end-enclosure in $(\bfy,\bfg)$-space.  These 2 methods gives
	the following end-enclosures:
	\beql{end-enclosure1}
	E_1^{xform1}
	\as Box_{\pi\inv(\bfq)}\left(r_0e^{\olmu^1 h} 
	+\delta_x\right).
	\eeql
	\beql{end-enclosure2}
	E_1^{xform2}
	\as \pi\inv\left(Box_{\bfq}((r'_0+d_m) e^{\olmu^2 h} 
	+  \delta_y)\right),
	\eeql
	where $r'_0$ radius of the circumball of 
	$\pi(E_0)$ and $d_m\ge \|\pi(m(E_0))-m(\pi(E_0))\|$.
	Let $E_1^{xform}\as E_1^{xform1}\cap E_1^{xform2}$. 
	We can compare $E_1^{euler}$ and $E_1^{xform}$ using two
	indenpendent ratios:
	% We have
	% 		$\wmin(E_1^{xform})\le \wmax(E_1^{xform})
	% 			\le \wmin(E_1^{euler})= \wmax(E_1^{euler})$.
	\beql{rho}
	\rho(E_1^{euler}, E_1^{xform}) \as 
	\Big( \tfrac{\wmax(E_1^{euler})}{\wmax(E_1^{xform})}, 
	\tfrac{\wmin(E_1^{euler})}{ \wmin(E_1^{xform}))}\Big).
	\eeql
	Our current experiments shows that the first ratio in 
	$\rho(E_1^{euler}, E_1^{xform})$ is always less than
	the second ratio, and for simplicity, we only show the
	second ratio, which is denoted by
	$\sigma(E_1^{euler}, E_1^{xform})$ in the
	last column of \refTab{transform}.
	
	\ignore{
	Let $Ball_{\bfp}(r_0) $ be the circumball of $E_0$. For a given
	precision $\delta > 0$, 
	we can compute a point $\bfQ$ by tracing 
	$\bfx \in \ivp(\bfp, h)$ such that 
	$\|\bfQ - \bfx(h)\| < \delta$. 
	The resulting end-enclosure for $\ivp(E_0, h)$ is
	$Ball_\bfQ\left(r_0 e^{\olmu^1 h} 
	+ \delta\right)$. 
	
	Using our transform method (illustrated in Figure 
	$\ref{fig:computerenclosure}$), we can obtain a point 
	$\bfq$ such that 
	$\|\pi\inv(\bfq) - \bfx(h)\| \leq \delta$. 
	The corresponding end-enclosure for $\ivp(E_0, h)$ is 
		\beql{end-enclosure1}
			Ball_{\pi\inv(\bfq)}\left(r_0e^{\olmu^1 h} 
				+\delta\right).
		\eeql
	
	Furthermore,  let $Ball_{\pi(E_0)}(r'_0)$ be the circumball
	of $\pi(E_0)$ and $d_m\as \|\pi(m(E_0))-m(\pi(E_0))\|$ then 
		\beql{end-enclosure2}
			F\as \pi\inv\left(Ball_{\bfq}((r'_0+d_m) e^{\olmu^2 h} 
			+  \delta_1)\right)
		\eeql
	is another end-enclosure for $\ivp(E_0, h)$. Therefore, by 
	taking the intersection of enclosures, we can always 
	obtain a tighter end-enclosure.  
	}
	
	\savespace{
	We have the following lemma:
	
	\blem  \ \\
	The diameter of  $F$ \refeq{anotherend-enclosure}
	is less than  
	$
		2 J_{\pi\inv}(\pi(F_1)) J_{\pi}(F_1) (r'_0+d_m)
		e^{\olmu^2 h} + 2 \delta.
	$  
	\elem  
	}

	\savespace{
	\bpf
	\beqarrays  
			\pi(E_0)
		&\ib& 	
			\pi(\bfp+[-r_0,r_0]^n)
		\\
		&\ib&
			\pi(\bfp)+ J_{\pi}(F_1)\Bigcdot [-r_0,r_0]^n
		\\
		&& \text{(by  mean value theorem.)}			
	\eeqarrays
	Thus, 
	$r'_0\le \|J_{\pi}(F_1)\|r_0.$
	\beqarrays  
		F
	&\ib& 	
		\pi\inv(\bfq)+ J_{\pi\inv}(\pi(F_1))\Bigcdot
		Box((r'_0+d_m)e^{\olmu^2h}+\delta_1).\\
	&& \text{(by mean value theorem)}			
	\eeqarrays
	Thus
	\beqarrays  
		diameter(F)
	&\le& 
		2\|J_{\pi\inv}(\pi(F_1))\|((r'_0+d_m)e^{\olmu^2h}+\delta_1)
	\\
	%	&\le&  
		%	2 \|J_{\pi\inv}(\pi(F_1))\| \|J_{\pi}(F_1)\|  (r_0+d_m)
		%	e^{\olmu^2 h} + 2 \|J_{\pi\inv}(\pi(F_1))\| \delta_1
	%	\\	
	&\le& 
		2 \|J_{\pi\inv}(\pi(F_1))\| \|J_{\pi}(F_1)\|  (r_0+d_m)
		e^{\olmu^2 h} +2\delta 
	\\
	&&  \text{(by \refeq{delta1})}	
	\eeqarrays
	\epf  
	}

%%%%%%%%%%%%%%%%%%%%%%%%%%%%%%%%%%%%%%%%%%%%%%%%%%
\ssectL[comparison]{Experimental Comparison} 
%%%%%%%%%%%%%%%%%%%%%%%%%%%%%%%%%%%%%%%%%%%%%%%%%%
    \refTab{transform}
	compares a single step of our transform method with 
	the classic Euler method.


	Each row represents a single experiment.
    The columns under $(E_0, F_1, h)$ represent an admissible triple.
	The column under $\olmu^1$ (resp.~ $\olmu^2$) represents
	the \lognorm\ bound of $F_1$ in the $(\bfx, \bff)$-space
	(resp.~ $\pi(F_1)$ in the $(\bfy,\bfg)$-space).
	The $\delta$ column ensures that the end-enclosure is
	$\delta$-bounded in the sense of
	\refeqs{end-enclosure0}{end-enclosure1}.
	The $d$ column refers to uniform exponent $\bfd=(d\dd d)$ 
	of our radical transform.  The last column
	$\sigma(E_1^{euler},E_1^{xform})$
	is the most significant, showing the relative improvement
	of our method over Euler.
	\savespace{
	Each experiment such as Eg1 or Eg1-a has two
	rows corresponding to two variants of the experiment
	with different values of $d$.
	}

	\newcommand*{\myalign}[2]{\multicolumn{1}{#1}{#2}}
	%%%
	\begin{table}[] \centering
	{\tiny
		\btable[l| l | l | l | l | l ]{
			Eg* & \myalign{c|}{\dt{Name}}
			& \myalign{c|}{$\bff(\bfx)$}
			& \myalign{c|}{\dt{Parameters}}
			& \myalign{c|}{\dt{Box} $B_0$}
			& \myalign{c}{\dt{Reference}}
			\\[1mm] \hline \hline
			Eg1 & Volterra 
			& $\mmatP{\phantom{-}ax(1-y) \\ -by(1-x)}$ % ($a>1, 0<b<1$)
			&  $\mmatP{a\\ b}=\mmatP{2\\ 1}$
			& $Box_{(1,3)}( 0.1)$
			& \cite{moore:diffEqn:09},
			\cite[p.13]{bunger:taylorODE:20}
			\\[2mm] \hline 
			Eg2 & Van der Pol 
			& $\mmatP{y \\ -c(1-x^2)y -x}$
			& $c=1$
			& $Box_{(-3,3)}( 0.1)$
			& \cite[p.2]{bunger:taylorODE:20}
			\\[1mm] \hline
			Eg3 & Asymptote
			& $\mmatP{x^2 \\ -y^2 + 	7x}$
			& N/A
			& $ Box_{(-1.5,8.5)}(0.5)$
			& N/A
			\\[1mm] \hline
			Eg4 & Lorenz
			& $\mmatP{\sigma(y-x)\\ x(\rho-z)-y\\ xy-\beta z}$
			&
			$\mmatP{\sigma\\\rho\\\beta}=\mmatP{10\\ 28\\8/3}$
			& $ Box_{(15,15,36)}(0.1)$
			& \cite[p.11]{bunger:taylorODE:20}
			\\[1mm] \hline
	} }
	\caption{List of IVP Problems}
	\label{tab:problems}
	\end{table}


	\begin{table*}[]
      \centering
	{\tiny
	\btable[c| c | c | c | c | c | c | c | c ]{
	Eg* & $E_0$ & $F_1$ & $h$ & $\delta$ & $\olmu^1$
		& $d$ & $\olmu^2$ & $\sigma(B_E,B_T)$ \\\hline
	\hline
	Eg1 & $Box_{(1,3)}(10^{-4})$ & $[0.9,1.0]\times[2.9,3.0]$
		& 0.0028 & $10^{-7}$ & 0.07 & 17  & -68.30 
			& $1.0000$
		\\
	    & & & & & & 1 & -5.82 &  $1.0000$
		\\\hline
	Eg2-a & $Box_{(3,-3)}(10^{-4})$ & $[2.9, 3.0]\times [-3.0,-2.9]$
		& $0.000018$ & $10^{-7}$ & 5.90 & 23
			& -142.73 & $1.0002$
		\\
	      & & & & & & 1 & -11.28 & $1.0003$
		\\\hline
	      
	Eg2-b & $Box_{(3,-3)}(10^{-4})$ & $[2.9, 3.0]\times [-3.0,-2.9]$
		& 0.00086 & $10^{-7}$ & 5.90 & 23 & -140.80
			& $1.0002$
		\\
	     & & & & & & 1 & -11.00 & $1.0007$
		\\\hline
	
	Eg2-c & $Box_{(3,-3)}(10^{-4})$ & $[2.9, 3.0]\times [-3.0,-2.7]$
		& 0.01 & $10^{-7}$ & 5.93 & 23 & -370.14 & $1.0321$
		\\
	      & & & & & & 1 & -9.20 & $1.0458$
		\\\hline
	
	Eg3-a & $Box_{(3,-3)}(10^{-4})$ & $[2.9, 3.0]\times [-3.0,-2.9]$
		& $0.00003$ & $10^{-7}$ & 9.75 & 20 & -180.71
			& $1.0000$
		\\
	      & & & & & & 1 & -10.08 & $1.0003$
		\\\hline
	
	Eg3-b & $Box_{(3,-3)}(10^{-4})$ & $[2.9, 3.0]\times [-3.0,-2.9]$
		& 0.001 & $10^{-7}$ & 9.75 & 20 & -177.05 & $1.0000$
		\\
	      & & & & & & 1 & -9.87 & $1.0008$
		\\\hline
	
	Eg3-c & $Box_{(3,-3)}(10^{-4})$ & $[2.9, 3.2]\times [-3.0,-2.6]$
		& 0.02 & $10^{-7}$ & 10.64 & 20 & -163.12 &  $1.0035$
		\\
	      & & & & & & 1 & -9.39 & $1.0665$
		\\\hline
	Eg4-a & $Box_{(1.0,3.0,1.0)}(10^{-4})$
		& $[0.9, 1.0]\times [2.9,3.0]\times[0.9,1.0]$ & $0.001$
			& $10^{-7}$ & 13.60 & 58 & -22.10
			& $1.0102$
		\\
	      & & & & & & 1 & -2.97 & $1.0186$
		\\\hline
	Eg4-b & $Box_{(1.0,3.0,1.0)}(10^{-4})$
		& $[0.9, 1.5]\times [2.9,3.7]\times[0.9,1.0]$ & $0.02$
			& $10^{-7}$ & 13.62 & 10 & -6.01 & $1.3286$
		\\
	      & & & & & & 1 & -3.01 & $1.3933$
		\\\hline
	}
	}
	\caption{Comparison our transform method with Euler's method}
	\label{tab:transform}
	\end{table*}


	From the experimental results, we can conclude the following:  
	\\ 1. Applying the transformation consistently yields a tighter
	end-enclosure.  
	\\ 2. The benefit of the transformation becomes increasingly
	significant as the step size grows larger.  
	\\ 3. Decreasing the value of $d$ further tightens the end-enclosure.  
	\\ 4. When the given \ivp\ exhibits significantly faster growth along
	one coordinate axis compared to another, the transformation method
	demonstrates superior performance. This is because the Euler method is
	constrained to using square boxes, whereas the transformation method
	can adaptively use elongated boxes.  


%%%%%%%%%%%%%%%%%%%%%%%%%%%%%%%%%%%%%%%%%%%%%%%%%%
%% EXTRA STUFF
%%%%%%%%%%%%%%%%%%%%%%%%%%%%%%%%%%%%%%%%%%%%%%%%%%
\ignore{%
\ssect{Some Examples}
    We will use the following running examples in our experiments:

		\begin{Example}[ 0.]
            This is the trivial $x'=ax$ with solution $x(t)=e^{at}$.
            Clearly, the logarithmic norm of this system is
            negative iff $a<0$.
        \end{Example}

	\begin{Example}[ 1. Volterra System]
            This is the Predator-Prey equation for
            $\bfx\in \ivp(E_0,1)$ where
            \beql{volterra}
			\bff=\mmat{x \\ y} '
			= \mmat{\phantom{-}ax(1-y) \\ -by(1-x)} ,
                    \qquad E_0 = Box_{(1,3)}( 0.1),
			\eeql
		with $a>1$ and $0<b<1$.  We choose $(a,b)=(2,1)$ and
		$E_0=Box_{(1,3)}(0.1)$ to closely
    track the AWA examples in
	\cite{moore:diffEqn:09} and
    \cite[p.13]{bunger:taylorODE:20}. 
	Thus $J_{\bff} =\mmat{2(1-y) & -2x\\ y & x-1}$ and the eigenvalues of
	$\half(J_{\bff}+ J_{\bff}^{\tr})$ are
            $$\half(1+x-2y \pm \sqrt{\Delta(x,y)}), \qquad \text{where } 
            \Delta(x,y)\as 5(x^2+y^2)-6(x+2y)+9$$
	  Note that
	  $\Delta(x,y)=5(x-\tfrac{3}{5})^2 + 5(y-\tfrac{6}{5})^2$ is
	  positive. Thus $\sqrt{\Delta(\bfq)}=\sqrt{5}\|\bfq-\bfp_0\|$
	  where $\bfp_0=(\tfrac{3}{5},\tfrac{6}{5})$.
	  Let $\ell(x,y)\as 1+x-2y$.  Thus $\ell(\bfq)> 0$ iff
	  $\bfq$ lies below the line $1+x-2y=0$.
	  \refFig{Volterra-21-13} shows
	  the trace of $\bfx(t)$ with $\bfx(0)=(3,1)$ and $t\in [0, 5.48]$
	  representing one complete
      cycle as computed by MATLAB; Bunger \cite{bunger:taylorODE:20}
      said that AWA cannot complete this computation 

      \FigEPS{Volterra-21-13}{0.2}{Solution to the Volterra system
	  	with $(a,b)=(1,2)$, and $\bfx(0)=(1,3)$ is shown in blue.
		The points $\bfq$ with negative
		$\mu_2(J_\bff(\bfq))$ lie above the green parabola as defined by
		the red line and red dot.}
 
	We have $\mu_2(J_\bff(\bfq)) = \half(\ell(\bfq)+\sqrt{\Delta(\bfq)})$,
	and this can only be negative when $\ell(\bfq)<0$, i.e., when $\bfq$
	lies above the line $\ell(x,y)=0$.  The algebraic locus of the curve
	$\mu_2(J_\bff(\bfq))=0$ is given by the equation
	$|\ell(\bfq)|^2=\Delta(\bfq)$.
    \ignore{
      We have $\sqrt{5}|\ell(\bfq)|$ is the distance
	  of $\bfq$ from the line $\ell=0$.}   This equation is a parabola
	  defined by the line $\ell(x,y)=0$ and the point $\bfp_0$.
 
    \end{Example}

    \begin{Example}[ 2. Van der Pol System]
		\beql{vanderpol}
			\bff=\mmat{x \\ y} '
			= \mmat{y \\ -x^2y + y-x} ,
                    \qquad E_0 = Box_{(-3,3)}( 0.1).
			\eeql
		It could be checked that the logarithmic norm of this system is
		always positive.

        \NOignore{
              $J_f = \mmat{0 & 1 \\ -2xy -1 & 1-x^2}$
              and the two eigenvalues
              of $(J_f+ J_f\tr)/2$ are
              
              $\mmat{ 1-x^2 + \sqrt{(1-x^2)^2 + 4x^2y^2}\\
                     1-x^2 - \sqrt{(1-x^2)^2 + 4x^2y^2}
                         }$,
            and the larger eigenvalue is always positive.
        }
	\end{Example}
	
	\begin{Example}[ 3.]
		Consider the system  
		\[
		\bff=\mmat{x \\ y} '
		= \mmat{x^2 \\ -y^2 + 	7x} , 
            \qquad E_0 = [-2.0, -1.0] \times [8.0, 9.0].
		\]
		The logarithmic norm of this system is initially negative but
		becomes positive. The solution to the $x$-component is $x(t)=
		\tfrac{1}{x_0\inv -t}$ which is negative for all $t>0$ (assuming
		$x_0<0$). Thus the $y$-axis is an asymptote of the solution curve
		(red curve in \refFig{neg-pos-system}). If the end enclosure
		$F_1=Box_{\bfp_1}(w_x, w_y)$ then $w_y/w_x\to \infty$. We have
		$J_\bff=\mmat{2x& 0\\ 7&-2y}$, and the eigenvalues of
		$\half(J_\bff+J_\bff{\tr})$ are
        $(x-y) + \sqrt{(x+y)^2 + (7/2)^2}$.
		        The phase portrait of this system is seen in
		\refFig{neg-pos-system}. The solution
		$\bfx$ with $\bfx(0)=(-1.5, 8.5)$ is shown in red.
	
		\FigEPS{neg-pos-system}{0.3}{Phase Portrait of Example 4}
	
    We check that for any point $\bfp\in\RR^2$ outside
    of the second quadrant $\set{x<0, y>0}$, we have
    the property
    $\mu_2(J_\bff(\bfp))>0$.   Moreover, for
     any solution $\bfx(t)\in \ivp(E_0,1)$ there is a moment 
        $0< t_1<1$ such that
        $\mu_2(J_\bff(x(t)))$ is negative if $t<t_1$
        and positive if $t>t_1$.  
	\end{Example}

     \begin{Example}[ 4. Lorenz System]

     {\small\beql{lorenz}
       \bff = \mmat{x\\y\\z}' = 
            \mmat{\sigma(y-x)\\ 
                    x(\rho-z)-y\\
                    xy-\beta z},
        \qquad
        E_0 = Box_{(15,15,36)}(0.1),
        \quad
        (\sigma,\rho,\beta)=(10,28,8/3).  
          \eeql}

     We have
     {\small $J_\bff=\mmat{-10 & 10 & 0\\
                    28-z & -1 & -x\\
                    y  & x  & -8/3}
     $} and
    {\small $\half(J_\bff+J_\bff{\tr})=\mmat{-10 & 19-\half z & \half y\\
                    19-\half z & -1 & 0\\
                    \half y  & 0  & -8/3}
     $}.  
	 We could no longer easily compute global bounds on log norms,
	 but only obtain bounds on $\olmu =\mu_2(J_\bff(F_1))$ for given
     admissible pair $(h,F_1)$ for $E_0$
 	E.g. for {\small $(h,F_1)=(0.01, 
        [14.80, 15.12]\times [13.34, 15.10] \times [35.75, 37.45])$}
    we obtain the bound
    $\olmu\le 4.33$.
     \end{Example}

	\ignore{\begin{Example}[ 4.]
	 	Consider the IVP where   
	 	\[
	 	\bff=\mmat{x \\ y} '
	 	= \mmat{x^2-2x+4 \\ -y^2 + 2y-7x-4} ,
                 \qquad
                E_0 = [-2.1, -1.9] \times [6.9, 7.1].
	 	\]
        \end{Example}
    }%ignore
}%	
