\ssect{Brief Literature Review} 

The validated IVP literature appeared almost from the start of
	interval analysis, pioneered
	by Moore, Eijgenraam, Rihm and others
	\cite{moore:autoAnalysis:65,moore-v1-autoAnalysis,moore:bk,
	eijgenraam:ivp:81,rihm:interval-ivp:94}.
	Corliss \cite{corliss:survey-ode-intvl:89} surveys this early period.
	Approaches based on Taylor expansion is dominant as they benefit
	from techniques such automatic differentiation
	and data structures such as the \dt{Taylor model}.  The latter,
	developed and popularized by Makino and Berz
	\cite{makino-berz:taylorModels:17, bunger:taylorODE:20,
		bunger:taylor-precond:21},
	has proven to be very effective. 
	%Additionally, the method's order can be easily adjusted by adding or
	%removing terms, and the step size can be modified without the need to
	%recompute the series coefficients.  
	%moore:localCoord:65,moore-v2-localCoord,moore:interval:bk66
	A major activity is the development of techniques
	to control the ``wrapping effect'', first described by Moore.
	Here Lohner's approach \cite{lohner:thesis,adams:enclosure:87}
	has been most influential. 
	%
	Another advancement is the $C^r$-Lohner method
	developed by Zgliczyński et al. \cite{zgliczynski:lohner:02}. This
	approach involves %formulating and
	solving auxiliary IVP systems to estimate
	higher order terms in the Taylor expansion.
	The field of validated methods, including IVP,
	underwent great development in the decades of 1980-2000.
	% By doing so, it effectively mitigates error accumulation caused by
	% interval arithmetic.  
	Nedialkov et al provide an excellent survey of
	the various subroutines of validated IVP
	\cite{jackson+nedialkov:advances-ode:02, nedialkov:thesis:99,
	nedialkov-jackson-pryce:HOI:01, nedialkov+2:validated-ode:99}.

	%%
	%The problem enjoyed the attention of researchers 
	%from the beginning of interval analysis.
	% E.g., Rihm \cite{rihm:interval-ivp:94} identified
	% \cite{moore:localCoord:65} as one of the 3 earliest work.
	%%
In Nonlinear Control Theory
	(e.g., \cite{scott-barton:bounds:13, fan+3:simulation-reach:18}),
	the End-Enclosure Problem is studied under various
	\dt{Reachability} problems.
In complexity theory, Ker-i Ko \cite{ko:real:bk} has shown
	that IVP is PSPACE-complete.
	This result makes the very strong assumption
	that the search space is the unit square ($n=1$).
	Bournez et al \cite{bournez+2:ode-unbounded:11}
	avoided this restriction by assuming that $\bff$
	has analytic extension  to $\CC^d$.
	
In this paper, we exploit the concept
	of \dt{logarithmic norm}\footnote{
		This concept goes by other names, including
			{logarithmic derivative},
			{matrix measure} and
			{Lozinski\u{i} measure}.
	}
	(or \dt{\lognorm} for short). It was independently introduced
	by Germund Dahlquist and Sergei M.~Lozinski\u{i} in 1958
	\cite{soderlind:log-norm-history}.
	It is interesting to note that their motivation
	was to improve bound errors in IVP.
	%	{by using differential inequalities that could
	%	distinguish between forward and reverse time integration.}
	Neumaier \cite{neumaier:theoryI:94}
	is one of the first to use \lognorm s in validated IVP.
	The earliest survey is
	T.~Str\"om (1975) \cite{strom:log-norm:75}.
		% title="On Logarithmic Norms"
	% \cite{neumaier:theoryI:94}
	% ``The logarithmic norm. History and modern theory'',
	% BIT Numerical Mathematics, 46(3):631-652, 2006.
	\savespace{
		M.~Vidyasagar (1978) \cite{vidyasagar:matrix-measure:78}
		introduced a modification of matrix measure which 
		%was generalized in a 14-page paper by
		has most of the properties
		needed in the applications. % of matrix measures.
	}% savespace
	%such as
	% the error bounds in the numerical integrations of ordinary
	% differential equations. 
	%   He also connects it with Lyapunov functions.
	The survey of Gustaf S\"{o}derlind
	\cite{soderlind:log-norm-history}
	extends the classical theory of \lognorm s to the general
	setting of {functional analytic} via Banach spaces.
	% There are numerous applications of this concept:
	% As noted by S\"oderlind, % abstract
	%The numerous applications include stability theory of IVPs and
	%BVPs, to solvability of algebraic, nonlinear, operator
	%and functional equations.
	%E.g., 
	%W. A. COPPEL, ``Stability and Asymptotic Behaviour of Differential
	%Equations,'' Heath, Boston, 1965. 

	% The error bound grows exponentially in the log norm
	% (like in the classical theory), but when the log norm is negative,
	% negative, the true solution's range contracts. 
	% Our paper will show how to turn the positive log norm
	% case into the negative case.
	

%%%%%%%%%%%%%%%%%%%%%%%%%%%%%%%%%%%%%%%%%%%%%%%%%%
% END
%%%%%%%%%%%%%%%%%%%%%%%%%%%%%%%%%%%%%%%%%%%%%%%%%%
%	An excellent historical overview is Neumaier
%	\cite{neumaier:taylor-forms:03} but see
%	\cite{makino-berz:multidimTaylor:01}.
%	Such algorithms have been studied for more than 50 years.
%	According to Rihm \cite{rihm:interval-ivp:94},
%	the earliest work on interval methods for ODEs is from Moore
%	\cite{moore:interval:bk66,moore-v1-autoAnalysis,moore-v2-localCoord}.
%	Cite other surveys quoted by Rihm...
%
%	
%	CHECK: Review Rihm
%	\cite{rihm:interval-ivp:94} which is quite comprehensive.
%	Nedialkov cites Rihm for the Admissibility Theorem
%	that justifies the standard technique of STEP A.

%	\bitem
%	\item This is important to cite:
%		Check out this 1996 by Corliss and 5 others (including
%		Griewank):
%		\cite{corliss+5:stiff-ode:96}
%	\item "Validated ODE" paper by Stetter 1990:
%		\cite{stetter:validatedODE:90}.
%	\item ``Solving Analytic Differential Equations in
%			Polynomial Time over Unbounded Domains''
%			by Olivier Bournez, Daniel S. Gra\'{c}a, and Amaury Pouly.
%			\cite{bournez+2:ode-unbounded:11}.
%			They give a "complete" algorithm but in a rather
%			limited setting.
%			Note that Ker I Ko did not do anything to
%			guarantee that the standard "admissible pair" $(h,B)$
%			is a pure existential statement with no lower
%			bound on $h>0$, and therefore cannot guarantee
%			that we can eventually reach some target time (say $t=1$).
%	\eitem
%

