%file: radical-xform.tex

%%%%%%%%%%%%%%%%%%%%%%%%%%%%%%%%%%%%%%%%%%%%%%%%%%
\ssectL[keyIdea]{Key Idea: The Radical Transform method}
%%%%%%%%%%%%%%%%%%%%%%%%%%%%%%%%%%%%%%%%%%%%%%%%%%
	Consider an admissible triple $(E_0, h, F_1)$.
	Given $\delta>0$,
	our goal is to compute an $\delta$-bounded
	end-enclosure $E_1\ib F_1$, \refeq{deltabound}.
	We can compute $\olmu$ that is a \lognorm\ bound for $(\bff, F_1)$
	(see \refeq{lognormbound}).  We have 2 cases:

	\dt{Easy Case: $\olmu\le 0$.}
	So $F_1$ is a contraction zone and we
	can use \refLem{eulerStep} to compute
	a point $\bfq$ such that $Ball(\bfq,\delta)$ is a shrinking
	end-enclosure for $\ivp(E_0,h,F_1)$.
	
	\dt{Hard Case: $\olmu >0$.}
	The key idea is to construct an invertible transformation
			$\pi:\RR^n\to\RR^n$
	We introduce $\bfy =(y_1\dd y_n) \as \pi(\bfx)$ and consider the
	differential equation
		\beql{bfy'}
			\bfy' = \bfg(\bfy), \qquad
				\bfg(\bfy) \as J_{\pi}(\pi\inv(\bfy))\Bigcdot
				\bff(\pi\inv(\bfy)).
		\eeql
	with admissible triple $(\pi(E_0),h,\pi(F_1))$.
	We define 
		\beql{circ}
			\pi=\whpi\circ \olpi
		\eeql
	as a composition of two maps $\olpi, \whpi :\RR^n\to\RR^n$ where
	$\olpi$ is an affine map (see Appendix B)
	and 
		\beql{whpibfx}
			\whpi(\bfx)=(x_1^{-d_1}\dd x_n^{-d_n})
		\eeql
	for some $\bfd=(d_1\dd d_n)$.  In view of \refeq{whpibfx}, we
	call $\pi$ the \dt{radical transform}. 
	Assuming $\ivp(B_0,h,B_1)$ is valid (\cored{Section 2.1}),
	if $B_1$ is small enough, we show that $\pi(B_1)$
	is a contraction zone for $(\pi(B_1), \bfg)$.  This reduces
	to the easy case, and after computing a shrunk end-enclosure,
	we can pull-back to a shrunk end-enclosure for $\ivp(B_0,h,B_1)$.

	For consistency, we define $(\pi,\bfg)$ to be $(\id, \bff)$
	in the easy case.
