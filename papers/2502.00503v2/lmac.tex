% file: lmac.tex
%	for p.tex in "talk23-notes" for ODE
% Oct23:

%%%%%%%%%%%%%%%%%%%%%%%%%%%%%%%%%%%%%%%%%%%%%%%%%%
% Converting between arXiv and issac version of paper:
%%%%%%%%%%%%%%%%%%%%%%%%%%%%%%%%%%%%%%%%%%%%%%%%%%
% please comment out one of the following 2 lines: 
\newcommand{\issacArxiv}[2][]{#2}	% arXiv version
%\newcommand{\issacArxiv}[2][]{}		% issac version


%%%%%%%%%%%%%%%%%%%%%%%%%%%%%%%%%%%%%%%%%%%%%%%%%%
% proofreading...
%%%%%%%%%%%%%%%%%%%%%%%%%%%%%%%%%%%%%%%%%%%%%%%%%%
% \bw[replacement text]{original text}
\newcommand{\bw}[2][--]{\cored{Bingwei says: #2}\comagenta{New: #1}}
\newcommand{\bwX}[2][--]{#1}
\newcommand{\chee}[2][--]{\coblue{Chee says: #2}\cocyan{New: #1}}
\newcommand{\cheeX}[2][--]{#1}

%% CONCEPTS RELATED TO SOLUTIONS of ODES:
\newcommand{\IVP}{\mbox{\text{IVP}}} %		IVP(B_0) or IVP_\bff(B_0)

\newcommand{\Sol}{\mbox{Sol}} %=solution
%%
%\newcommand{\Term}{\mbox{Term}} %=terminal
\newcommand{\End}{\mbox{End}} % alternative to Term
%\newcommand{\Reach}{\mbox{Reach}} % alternative to Term
%%
\newcommand{\Slice}{\mbox{Slice}} %= Slice at time t
\newcommand{\Tube}{\mbox{Tube}} %= Tube around a solution.
%%
\newcommand{\dilate}{\text{dilate}} %= Tube around a solution.
\newcommand{\wmax}{w_{\max}} 
\newcommand{\wmin}{w_{\min}} 
%%
%\newcommand{\Imag}{\mbox{Image}} % 
%\newcommand{\Trac}{\mbox{Trace}} % alternative to Image
%\newcommand{\Range}{\mbox{Range}} % alternative to Image
\newcommand{\Locus}{\mbox{Locus}} % alternative to Image

%%% NORM:  I am slightly undecided about max-norm or 2-norm for Balls.
%%%		Seems like best to use max-norm for Lipschitz bound...
%%%		WAIT: it should be called "infty" norm.  max-norm is
%%%			for functions x:[0,h]\to\RR^n.
%%%		
%% "nnorm" because it is easy to type nn...
%%		NOTE: the arg is optional!
\newcommand{\nnorm}[1][\cdot]%
	{\left\lVert #1 \right\rVert_{\max}}% double ||

%% consistent short hand: logNorm or lognorm?
\def\lognorm{{logNorm}}

%%% Note: need \mbox to protect the font in math mode?
\def\IVP{\mbox{{\textbf{IVP}}}}	% DO NOT USE	 \IVP(B_0,h)
\def\ivp{\mbox{{\ttt{IVP}}}}		% USE 			\ivp(B_0,h)
\def\ttIVP{\mbox{\ttt{IVP\_Algorithm}}} % $\ttivp(\bff,B_0,\veps)$
\def\stepA{{\mbox{\tt StepA}}}
\def\stepa{{\mbox{\tt Step A}}} % when to use this instead of \stepA?
\def\stepB{{\mbox{\tt StepB}}}
\def\stepb{{\mbox{\tt Step B}}} % when to use this instead of \stepB?

\def\standardIVP{\mbox{{\tt standard\_IVP}}}
\def\simpleIVP{\mbox{{\tt simple\_IVP}}}
\def\simpleIVPdirect{\mbox{{\tt Direct}}}
\def\simpleIVPlohner{\mbox{{\tt Lohner}}}
\def\endEncIVP{\mbox{{\tt EndEnclosure\_IVP}}} % the problem
%\def\endEncAlgo{\mbox{{\tt EndEnclosure\_Algorithm}}}
\def\endEncAlgo{\mbox{{\tt EndEnclosure\_Algo}}} % the algorithm
\def\ourAlgo{{\endEncAlgo}}
\def\capdCr{\mbox{\tt CAPD}}


%%%
\newcommand{\Stepivp}{\mbox{\tt StepIVP}}
%\newcommand{\Refine}{\mbox{\tt Refine}}
\def\Refine{\mbox{\ttt{Refine}}}
\def\Refiner{\mbox{\ttt{Refine$^+$}}}
\def\Extend{\mbox{\ttt{Extend}}}
\def\Extender{\mbox{\ttt{Extend$^+$}}}

%%%
\def\bigStep{\mbox{\ttt{bigStep}}}
\def\smallStep{\mbox{\ttt{smallStep}}}

\def\Verify{{\ttt{Verify}}} % obsolete
% \def\IsPositive{{\ttt{IsPositive}}}
\def\AvoidsZero{{\ttt{AvoidsZero}}}
\def\Transform{{\ttt{Transform}}}

\def\TransformBound{{\ttt{TransformBound}}}

%\newcommand{\estimateR}{\mbox{\tt EstimateR}}
\newcommand{\admiss}{\mbox{\tt AdmissiblePair}}

%%%
%%%
\newcommand{\sqib}{\;\sqsubseteq\;}

% btableL:  variant of btable, with label and caption
% E.g.  \btable [c|c|c|] {--label--}{..caption..}{..table contents..}
% Chee: change name to btableLscript to avoid conflict with myMac.

\newcommand{\btableLscript}[4][ccccccccccccccc]{
	    \begin{center}
			\begin{table}[h]
			{\scriptsize
				\begin{tabular}{#1}
				#4
				\end{tabular}
				\label{tab:#2}
				\caption{#3}
			}%size
			\end{table}
	    \end{center}}

\newcommand{\estimateR}{{\rm estimateR}}


%%%%%%%%%%%%%% from mac-algo.tex
	\newcommand{\Bisect}{\mbox{\texttt{bisect}}}%should be \sc Bisect, mathOp

