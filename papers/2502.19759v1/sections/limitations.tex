\section*{Limitations}
Our study highlights the overlooked issue of history recall in voice interaction models and introduces a benchmark for systematic evaluation. We focus on open-source multi-round voice interaction models, analyzing them with additional results in Appendix \ref{subsubapp:additional_models}. However, other open-source models not covered in our analyses may exist. Additionally, while most recent models enhance semantic modeling by jointly generating spoken and text responses, some still generate speech directly without relying on intermediate text. Future research could extend our analysis to these models.

Another limitation of our retrieval-based analyses is its focus on text-based retrieval-augmented generation (RAG). Currently, no well-established speech retriever modules exist for open-source models, and open-source voice interaction models struggle with speech-based prompting for RAG, restricting our analysis to text prompts. Furthermore, we do not consider the latency introduced by RAG. Developing low-latency speech retrievers that efficiently integrate spoken information—including linguistic and non-verbal cues—remains crucial for real-time conversational applications.

Finally, our benchmark addresses only the simplest form of questions requiring past information, those directly retrieving and utilizing prior context. While our analysis shows that current open-source voice interaction models struggle even with basic recall, more advanced benchmarks will be necessary as these models evolve. For instance, future benchmarks could move beyond simple retrieval-based responses to questions requiring deeper reasoning over past context. Additionally, a benchmark focusing on memory capabilities in common voice interaction scenarios—such as handling fragmented information (e.g., a customer providing a phone number in segments)—would be valuable for assessing more complex recall abilities.

\section*{Ethical Considerations}

Our analysis highlights the recall capabilities of voice interaction models, particularly in personalized voice assistants that rely on past interactions for customized services. However, this capability inherently raises security and privacy concerns, as stored conversational data may be vulnerable to unauthorized access. As voice assistants become more deeply integrated into daily life, ensuring they retain necessary context while safeguarding user data is crucial. Therefore, alongside advancements in memory retention and utilization, developing robust mechanisms to protect stored history must remain a parallel research priority.