\begin{figure*}[!t]
    \centering
    \includegraphics[width=\linewidth]{02_images/dataset_visualization/dataset_visualization_v7_zoom10.pdf}
    % \vspace{1pt}
    \caption{Overview of a number of representative instances of the Extended PaintNet dataset, featuring realistic spray painting demonstrations designed by experts. The dataset is divided into four categories of growing complexity.
    Different colors represent separate paths.
    }
    \label{fig:dataset_overview}
    \vspace{-6pt}
\end{figure*}

PaintNet~\cite{tiboni2023paintnet} was the first dataset of expert demonstrations introduced to support the study of motion generation conditioned on free-form 3D objects, and was specifically designed for the spray painting task. In this work, we expand PaintNet more than threefold, resulting in a new version that contains 3088 samples.

Every sample is a pair of a 3D object and its corresponding spray painting paths. Each object is represented as a triangle mesh, with vertex coordinates expressed in real-world millimeter scale. The meshes are provided in an aligned and smoothed watertight~\cite{Huang_Robust_2018} format with any private information (\eg, engraved logos) accurately anonymized.
%
The number of spray painting paths associated to an object varies according to the geometry of the object. Each path is encoded as a sequence of end-effector configurations in task space, \ie, positions and orientations of the gun nozzle. More precisely, we record 3D positions as the ideal paint deposition point, 12cm away from the gun nozzle, and 3D gun orientations as Euler angles.
%
The data pairs represent realistic spray painting demonstrations that a real robot could directly execute. In other words, they encode feasible trajectories designed by experts to reach near-complete coverage of the whole surface of the 3D objects. Each waypoint is collected by sampling the end-effector pose at a rate of 250Hz during execution. 
%
Ground-truth paths are produced ad-hoc for each object category by custom heuristics based on long-standing experience in the field.

The four object categories composing the dataset are presented below, ordered by increasing complexity:

\begin{itemize}[leftmargin=*,itemsep=2pt]
 \item \textbf{Cuboids}: a basic class of 1000 rectangular cuboids tailored for testing models under minimal generalization requirements and relatively simple path patterns. Cuboids are sampled with varying height and depth uniformly from $0.5m$ to $1.5m$, while having a fixed width of $1m$. Their volume ranges from $0.25m^3$ to $2.25m^3$. A fixed number of six raster-like paths is associated to each cuboid to paint the exterior faces, with gun orientations that are perpendicular to the surface at all times.

\item \textbf{Windows}: a set of 1000 window-like data pairs, with width and height varying uniformly from $0.4\,m$ to $1.8\,m$, and a fixed thickness of $4\,cm$. For each sample, up to $3$ horizontal cross sections and up to $1$ vertical cross section are randomly selected. Windows introduce harder challenges for motion generation, such as predicting a variable, high number of paths while handling non-trivial gun orientations and geometric patterns.

\item \textbf{Shelves}: a set of 1000 shelves featuring highly concave surfaces. The instances differ significantly in volume and number of inner compartments. Their volume ranges from $18dm^3$ to $160dm^3$, with up to 6 inner compartments for the larger samples.

\item \textbf{Containers}: a set of 88 industrial containers including meshes with highly heterogeneous global and local geometric properties (\eg, wavy and grated surfaces). Here, the manually-guided painting paths designed by experts show evident irregularities across instances. This class of objects is particularly challenging due to the limited number of samples.
\end{itemize}
%
An overview of the Extended PaintNet dataset is depicted in Fig.~\ref{fig:dataset_overview}.
The dataset is publicly available at \url{https://gabrieletiboni.github.io/MaskPlanner/}.
%