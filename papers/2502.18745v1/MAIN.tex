\documentclass[lettersize,journal]{IEEEtran}

\usepackage{amsmath,amsfonts}
\usepackage{algorithmic}
\usepackage{algorithm}
\usepackage{array}
\usepackage{textcomp}
\usepackage{stfloats}
\usepackage{url}
\usepackage{verbatim}
\usepackage{graphicx}
\usepackage{cite}
\hyphenation{op-tical net-works semi-conduc-tor IEEE-Xplore}

\usepackage[dvipsnames,table,xcdraw]{xcolor}
\usepackage{amsmath}
\usepackage{amsfonts}
\usepackage{nicefrac}
\usepackage{comment}
\usepackage{hyperref}
\usepackage{xspace}
\usepackage{multirow}
\usepackage{subcaption}
\usepackage{graphbox}
\usepackage[normalem]{ulem}
\useunder{\uline}{\ul}{}

\usepackage{makecell}
\usepackage{graphicx}
\usepackage{nicematrix}

\usepackage{enumitem}  % lists with noindent

\usepackage{amssymb}% http://ctan.org/pkg/amssymb
\usepackage{pifont}% http://ctan.org/pkg/pifont
\usepackage{dsfont} % for indicator function \mathds{1}
\usepackage{balance}
\newcommand{\cmark}{\ding{51}}%
\newcommand{\xmark}{\ding{55}}%


% referencing objects
\newcommand{\figref}[1]{fig.\,\ref{fig:#1}}
\newcommand{\Figref}[1]{Fig.\,\ref{fig:#1}}
\newcommand{\tabref}[1]{tab.\,\ref{tab:#1}}
\newcommand{\Tabref}[1]{Tab.\,\ref{tab:#1}}
\newcommand{\secref}[1]{sec.\,\ref{sec:#1}}
\newcommand{\Secref}[1]{Sec.\,\ref{sec:#1}}

% complexity classes
\newcommand{\nc}[0]{NC$^1$\xspace}
\newcommand{\tc}[0]{TC$^0$\xspace}
\newcommand{\ac}[0]{AC$^0$\xspace}
\newcommand{\AND}[0]{\texttt{AND}}
\newcommand{\OR}[0]{\texttt{OR}}
\newcommand{\NOT}[0]{\texttt{NOT}}
\newcommand{\MAJ}[0]{\texttt{MAJORITY}}

% math
\newcommand{\expect}[1]{\mathbb{E}\left[ #1 \right]}
\newcommand{\transpose}{^\intercal}
\newcommand{\func}[1]{\left( #1 \right)}
\newcommand{\step}[1]{^{(#1)}}
\newcommand{\R}[1]{\mathbb{R}^{#1}}
\newcommand{\C}[1]{\mathbb{C}^{#1}}
\newcommand{\Order}[1]{\mathcal{O}\left(#1\right)}
\newcommand{\fp}{^{*}}
\newcommand{\iter}[1]{^{\left( #1 \right)}}
\newcommand{\loss}{\mathcal{L}}
\newcommand{\partialdiff}[3]{\left. \frac{\partial #1}{\partial #2}\right|_{#3}}
\newcommand{\distribution}[1]{$\mathcal{D}_{#1}$}

% variables
\newcommand{\bfM}{\mathbf{M}}
\newcommand{\bfX}{\mathbf{X}}
\newcommand{\bfZ}{\mathbf{Z}}

\usepackage{xspace} 
\newcommand{\ours}{MaskPlanner\xspace}

\usepackage{bbm}


\begin{document}

\title{MaskPlanner: Learning-Based Object-Centric Motion Generation from 3D Point Clouds}

\author{Gabriele Tiboni$^{1}$, Raffaello Camoriano$^{1, 2}$, Tatiana Tommasi$^{1}$% <-this % stops a space
\thanks{$^{1}$Department of Control and Computer Engineering, Politecnico di Torino, Turin, Italy {\tt\small first.last@polito.it}}%
\thanks{$^{2}$Istituto Italiano di Tecnologia, Genoa, Italy}%
}

\markboth{Under review}{}

\maketitle

\begin{abstract}
\begin{abstract}
  In this work, we present a novel technique for GPU-accelerated Boolean satisfiability (SAT) sampling. Unlike conventional sampling algorithms that directly operate on conjunctive normal form (CNF), our method transforms the logical constraints of SAT problems by factoring their CNF representations into simplified multi-level, multi-output Boolean functions. It then leverages gradient-based optimization to guide the search for a diverse set of valid solutions. Our method operates directly on the circuit structure of refactored SAT instances, reinterpreting the SAT problem as a supervised multi-output regression task. This differentiable technique enables independent bit-wise operations on each tensor element, allowing parallel execution of learning processes. As a result, we achieve GPU-accelerated sampling with significant runtime improvements ranging from $33.6\times$ to $523.6\times$ over state-of-the-art heuristic samplers. We demonstrate the superior performance of our sampling method through an extensive evaluation on $60$ instances from a public domain benchmark suite utilized in previous studies. 


  
  % Generating a wide range of diverse solutions to logical constraints is crucial in software and hardware testing, verification, and synthesis. These solutions can serve as inputs to test specific functionalities of a software program or as random stimuli in hardware modules. In software verification, techniques like fuzz testing and symbolic execution use this approach to identify bugs and vulnerabilities. In hardware verification, stimulus generation is particularly vital, forming the basis of constrained-random verification. While generating multiple solutions improves coverage and increases the chances of finding bugs, high-throughput sampling remains challenging, especially with complex constraints and refined coverage criteria. In this work, we present a novel technique that enables GPU-accelerated sampling, resulting in high-throughput generation of satisfying solutions to Boolean satisfiability (SAT) problems. Unlike conventional sampling algorithms that directly operate on conjunctive normal form (CNF), our method refines the logical constraints of SAT problems by transforming their CNF into simplified multi-level Boolean expressions. It then leverages gradient-based optimization to guide the search for a diverse set of valid solutions.
  % Our method specifically takes advantage of the circuit structure of refined SAT instances by using GD to learn valid solutions, reinterpreting the SAT problem as a supervised multi-output regression task. This differentiable technique enables independent bit-wise operations on each tensor element, allowing parallel execution of learning processes. As a result, we achieve GPU-accelerated sampling with significant runtime improvements ranging from $10\times$ to $1000\times$ over state-of-the-art heuristic samplers. Specifically, we demonstrate the superior performance of our sampling method through an extensive evaluation on $60$ instances from a public domain benchmark suite utilized in previous studies.

\end{abstract}

\begin{IEEEkeywords}
Boolean Satisfiability, Gradient Descent, Multi-level Circuits, Verification, and Testing.
\end{IEEEkeywords}
\end{abstract}
\begin{IEEEkeywords}
Motion Generation, Deep Learning, 3D Learning, Imitation Learning.
\end{IEEEkeywords}


\section{Introduction}
\label{sec:introduction}
\IEEEPARstart{M}{otion} generation conditioned on 3D objects is fundamental to numerous industrial robotic applications, including spray painting, welding, sanding, and cleaning.
Despite the different objectives, these tasks share key challenges arising from the complexity of free-form 3D inputs and the high-dimensional outputs required to define complete robot programs.
In particular, the offline generation of long-horizon motions demands substantial computational resources for both optimization and planning.
Additionally, encoding human expert behavior into explicit optimization objectives remains extremely challenging for such complex tasks.
%
To address these difficulties, robotics practitioners often rely on task-specific knowledge, impose strong simplifying assumptions regarding the object shapes, and develop heuristic algorithms to render each task individually more tractable.
%
However, these solutions necessitate extensive re-engineering for each new product, making the process time-consuming, costly, and unable to efficiently adapt to new scenarios. 
%
In this context, establishing a unifying paradigm to tackle these tasks is a crucial step for the community to transition from handcrafted strategies to techniques that are scalable and capable of generalization.

\begin{figure}[t]
    \centering
    \includegraphics[width=\linewidth]{02_images/intro_figure/intro_figure_v11.pdf}
    \caption{Several object-centric robotic applications may be unified under a single problem formulation, as they share common assumptions on the desired output paths---referred to as unstructured paths.}
    \vspace{-6pt}
    \label{fig:intro_figure}
\end{figure}

To this end, we formalize the \emph{Object-Centric Motion Generation} (OCMG) framework, a novel problem setting that unifies robotic tasks aiming to generate \emph{multiple}, \emph{long-horizon} paths based on \emph{static}, \emph{free-form} 3D objects.
For the sake of a general formulation, we consider output paths to be \emph{unstructured}: no predefined path order is assumed, and paths can generally vary in number and length according to the input object.
Notably, these properties are shared by a wide range of robotic applications---as illustrated in Figure~\ref{fig:intro_figure}---where negligible dynamic interactions with the 3D objects are involved
and global object geometric information is given.

Robotic spray painting is an industrially-relevant example of the OCMG problem:
multiple paths are necessary to paint a single object, and the resulting paint coverage is invariant with respect to the order of execution---\eg, separate paths can be executed in parallel across multiple robots.
Furthermore, motion generation can occur offline, and no reactive planning is needed since global information of the surface geometry is available.
Note that the pattern of the spray painting paths varies significantly with each object instance, making it difficult to codify general rules---\eg, human experts in the field rely on sophisticated, high-level reasoning and costly trial-and-error to determine robot programs based on the 3D geometry of the objects.
%
%
%
Existing research studies resort to decoupling the spray painting task in (i)~3D object partitioning into convex surfaces, and (ii)~offline trajectory optimization through either domain-specific heuristics~\cite{Sheng_Automated_2000,Chen_Automated_2008,Li_Automatic_2010,Andulkar_Incremental_2015,atkar24uniform,gleeson2022generating}, or reinforcement learning-based policies~\cite{Kiemel_PaintRL_2019}.
Yet, such approaches make simplified premises on the structure of output trajectories, require expensive optimization routines, and are heavily tailored to specific shapes and convex surfaces only.
%
These issues leave robotic spray painting solutions largely limited in flexibility and generalization capabilities, despite their relevance in product manufacturing.


Recently, interest in purely data-driven approaches has grown for extrapolating path patterns without the need to explicitly encode optimization objectives and task-specific constraints.
Assuming that expert data is available, learning-based methods pave the way for solutions that are scalable, cheap to deploy at inference time, and generalizable to unseen scenarios.
%
A number of successful applications demonstrate the potential of data-driven methods in related problems, such as motion forecasting in autonomous driving~\cite{yuan2021agentformer,ngiam2022scenetr,varadarajan2022multipath++,nayakanti2023wayformer}, multi-agent imitation learning~\cite{SRINIVASAN2021598}, or socially-compliant robot navigation~\cite{kretzschmar2013featuretrajpred,kretzschmar2014learningsocialnavigation,pfeiffer2016predicting}.
Yet, these solutions consider a short prediction horizon, assume a fixed number of paths, and do not deal with 3D input data.
%
%
Tiboni \etal~\cite{tiboni2023paintnet} recently proposed the first approach to handle unstructured paths conditioned on 3D objects, predicting disconnected path segments across the object surface rather than directly inferring long-horizon paths. 
This approach allows for accurate local predictions, but crucially lacks a way to organize predicted segments into separate paths, and finally concatenate them to generate long-horizon paths.


In this work, we propose \emph{\ours}, a novel deep learning method to address OCMG tasks directly from expert data, building on~\cite{tiboni2023paintnet}.
Our pipeline breaks down the motion generation problem into the joint prediction of (1)~path segments and (2)~path masks, in a single forward pass. Particularly, we propose learning binary masks over predicted segments to identify which path each segment belongs to.
This strategy allows the network to simultaneously make local (segment predictions) and global (mask predictions) planning decisions in one step.
In turn, our method effectively infers the required \emph{number} of paths and the \emph{length} of each path according to a given input object.


When tested in the field of robotic spray painting, \ours is capable of predicting segments and masks for 40 paths in only 100ms, spanning a total length of 70 meters and 8 minutes of execution time, and achieving near-optimal paint coverage on held-out 3D objects in simulation.
%
Moreover, we successfully execute the generated paths on a real 6-DoF specialized painting robot for previously unseen object instances, achieving qualitative results that are indistinguishable from those obtained via human-expert trajectories. 
%
%
Overall, we make significant progress in addressing the OCMG problem, and focus on spray painting as a representative application to conduct a thorough experimental evaluation.\\

Our novel contributions can be summarized as follows:
\begin{itemize}[leftmargin=*,itemsep=2pt]
    \item \textbf{Mask predictions}:
        we propose \ours, a novel deep learning method that predicts path segments along with a set of masks, identifying which segments belong to the same path.
    \item \textbf{Improved segment predictions}:
        we formalize a multi-component loss function based on the Chamfer Distance and tailored to segment prediction. We study the effect of each component with an extensive ablation analysis.
    \item \textbf{Segment concatenation}:
        a novel post-processing step is designed to filter and concatenate the segments clustered within the same mask into long-horizon paths.
    \item \textbf{Improved benchmarking}:
        we release a new public dataset extending that in~\cite{tiboni2023paintnet} by more than three times in size.
        Two novel baselines are implemented ad-hoc for comparison with methods that perform na\"ive autoregressive or one-shot predictions for object-centric motion generation. 
    \item \textbf{Real-world validation}:
        we assess the performance of \ours by executing the predicted paths on a 6-DoF spray painting robot, achieving expert-level painting quality on previously unseen object instances. 
\end{itemize}

\section{Related Work}
\label{sec:related_work}
In this section, we provide a review of the literature related to the OCMG problem (Sec.~\ref{sec:motion_plan_gen}-\ref{sec:3dpointclouds}) with an in-depth focus on previous works in robotic spray painting (Sec.~\ref{sec:painting}). 

\subsection{Learning-based Motion Planning and Generation}
\label{sec:motion_plan_gen}
\emph{Motion planning} involves finding low-cost, goal-conditioned trajectories in a given environment while accounting for task-specific constraints---such as enforcing kinematic or dynamic feasibility, safety, or smoothness~\cite{karur2021surveypathplanning}.
%
Conventional methods based on search or sampling over a discrete representation of the state space tend to be computationally expensive~\cite{kicki2023fasttateo,ichnowski2020deepmotionplanning}, hence unsuitable for real-world applications where planning has tight time requirements or spans high-dimensional spaces. 
%
Learning-based approaches for motion planning have been proposed to speed up planning times~\cite{surveylearningrobotmotionplanning2021} by predicting the sampling distribution for sampling-based methods~\cite{wang2020neuralrrt,cheng2020learningpathplanning}, warm-starting traditional solvers~\cite{ichnowski2020deepmotionplanning}, promoting the feasibility of planned trajectories~\cite{kicki2023fasttateo}, performing end-to-end planning~\cite{pfeiffer2017perception,bency2019neuraloraclenet}, or training Reinforcement Learning policies~\cite{tsounis2020deepgaitrl,kim2020motion} to predict short-term actions that maximize cumulative rewards.
%
Motion planning solutions yet focus on reaching a specified goal state from a predefined start location, hence they are not designed to generate complex path patterns that mimic expert behavior. 

%
\emph{Motion generation}~\cite{bekris2024motiongensurvey} encompasses a broader scope than traditional planning, as it may not involve start and goal states and often necessitates adherence to task-specific motion patterns. 
% 
Among learning-based approaches, Sasagawa et al.~\cite{sasagawa2021motionbilateral} train a recurrent neural network to tackle long-term motion generation in complex tasks such as writing letters, which requires separate sequential strokes.
Saito et al.~\cite{saito2023structured} adopt supervised deep learning to tackle long-horizon manipulation tasks 
and breaking the motion generation problem down into subgoals prediction. 
Neural networks also proved effective in generating human-like whole-body trajectories by learning from human motion capture data, in both humanoid robotics~\cite{viceconte2022adherent} and character control~\cite{zhang2018mode}.
%
%
%%%%%%%%%%
Imitation Learning~(IL)~\cite{ross2011reductionbc,behav_cloning,ho2016gail,panilautonomousdriving,Ze2024DP3diffusionpolicy} tackles motion generation assuming that a reward function is described implicitly through expert demonstrations, hence solving the task by learning from data.
%
In particular, Behavioral Cloning (BC)~\cite{ross2011reductionbc,behav_cloning} consists of supervised learning techniques that directly find a mapping from the current state to the optimal action, \eg., through regression methods.
%%%%%%%%%%
Alternatively, Inverse Reinforcement Learning (IRL)~\cite{abbeel2004apprenticeshipirl,ziebart2008maximumirl} aims at learning a representation of the underlying reward function the human experts used to generate their actions. IRL has been successfully deployed to learn parking lot navigation strategies~\cite{abbeel2008apprenticeshipirlplanning}, human-like driving behavior~\cite{wulfmeier2016watchirlpathplanning}, and long-term motion forecasting~\cite{shkurti2018modelpursuit}. 
%

Notably, BC and IRL methods typically frame motion generation as a sequential decision making problem in an unknown dynamic environment with the Markov property---a Markov Decision Process. While our work is also fully data-driven, we address the challenge of global, long-horizon motion generation with complete state information.
%
Furthermore, although adaptations of IL to multiple agents~\cite{SRINIVASAN2021598} and global trajectory learning~\cite{osa2017guiding,duan2024structured,behav_cloning} were proposed, no method can manage unstructured paths---namely, they fail to model scenarios where the number of agents/paths is unknown, and no temporal correlation exists among separate paths.
%
Finally, our work focuses on learning paths that are generalizable across complex 3D shapes directly, a setting which has been rarely addressed before in robot imitation learning~\cite{schulman2016,Ze2024DP3diffusionpolicy}.



%%%%%%%%%%%%%%%%%%%%%%%%%%
\subsection{Motion Prediction}
Motion prediction aims at anticipating the motion of multiple agents ahead in the future.
To solve this problem, existing methods generally employ supervised learning techniques from observed trajectories, with applications to autonomous driving~\cite{yuan2021agentformer,ngiam2022scenetr,varadarajan2022multipath++,nayakanti2023wayformer}, human motion forecasting~\cite{alahi2016social,gupta2018social,rudenko2020humanpredictionsurvey}, and socially-compliant robot navigation~\cite{kretzschmar2013featuretrajpred,kretzschmar2014learningsocialnavigation,pfeiffer2016predicting}. 
%
Pfeiffer et al.~\cite{pfeiffer2016predicting} leverage the maximum entropy principle to learn a joint probability distribution over the future trajectories of all agents in the scene from data, including the controllable robot.
Gupta et al.~\cite{gupta2018social} predict socially-plausible human motion paths using a recurrent model and generative adversarial networks.
Nayakanti et al.~\cite{nayakanti2023wayformer} adopt a family of attention-based networks for motion forecasting in autonomous driving, investigating the most effective ways to fuse scene information including agents'~history, road configuration, and traffic light state.
%

Notably, motion prediction deals with output paths that are jointly executed through decentralized agents that move simultaneously over time, from a given starting state.
In turn, numerous strategies were proposed to aggregate information across agents and across time, such as
pooling layers~\cite{alahi2016social,gupta2018social}, independent self-attention for each axis~\cite{yu2020spatiotemporal}, or joint attention mechanisms on multiple axes~\cite{yuan2021agentformer,ngiam2022scenetr,nayakanti2023wayformer}.
%
In contrast, the OCMG problem considers learning a set of disjoint paths that are uncorrelated in time---\eg., they may be executed separately at different times, in an arbitrary order, and from unknown starting states.
%
In addition, the motion prediction literature assumes a known, fixed number of agents in the scene. Recently, Gu et al.~\cite{gu2023vip3d} introduced the first end-to-end approach to couple motion prediction with object detection and tracking, effectively dealing with a varying number of agents that is automatically inferred at test time.
%%%%%%%%%%%%%%%
Yet, adapting these works to heterogeneous path lengths and long-horizon motions is an open problem---the literature focuses on fixed prediction horizons of only 3-8 seconds.
Overall, the temporal correlation of predicted paths and assumptions on fixed, short-horizon forecasting render motion prediction methods unsuitable for direct application to the OCMG setting.
%%%%%%%%%%%%%%%%%%%%%%%%%%

\subsection{Set Prediction}
Canonical deep learning models are not designed to directly predict sets, \ie collections of permutation-invariant elements with varying cardinality.
%
Early works addressed this issue in the context of multi-label classification~\cite{rezatofighi2017deepsetnet}, where an unknown number of labels must be associated with a given input. 
For regression tasks of set prediction (\eg, Object Detection), the difficulty is instead to avoid generating near-duplicate outputs (\ie, near-identical bounding boxes) due to an unknown number of output elements.
%
This was originally mitigated via postprocessing techniques such as non-maximal suppression~\cite{erhan2014scalable,redmon2016yolo}. 
Later, auto-regressive recurrent models were proposed for sequentially predicting output sets~\cite{vinyalsseqtoseq,stewart2016end}, but these approaches were eventually outperformed by transformer-based architectures~\cite{carion2020detr,cheng2021maskformer}.
Transformers excel in such tasks by leveraging attention mechanisms to decode output elements in parallel and capture long-range dependencies, resulting in more robust and accurate set predictions.

Regardless of the architecture, the loss function designed for set prediction must always be invariant by a permutation of the predictions, or the ground truth.
This can be achieved by \emph{matching} predictions with ground truths before the loss computation, either implicitly, leveraging on a moving window across the input image~\cite{redmon2016yolo}, or explicitly, by solving a bipartite matching problem~\cite{erhan2014scalable,stewart2016end,cheng2021maskformer}.
The latter approach has been widely adopted in Object Detection~\cite{carion2020detr} and Panoptic Segmentation~\cite{cheng2021maskformer} tasks using the Hungarian algorithm~\cite{kuhn1955hungarian}.
%%%%%%%%%%%%%%%%

\subsection{3D Deep Learning from Point Clouds}
\label{sec:3dpointclouds}
3D deep learning architectures apply predictive models to process free-form 3D data~\cite{ahmed2018survey}, typically represented as voxel grids, meshes, or point clouds.
Particularly, point cloud representations describe objects as unstructured sets of 3D points, and were successfully proposed to perform tasks such as 3D object classification~\cite{Qi_Pointnet_2017} and segmentation~\cite{Qi_Pointnet++_2017}, and shape completion~\cite{Yuan_Pcn_2018,Alliegro_Denoise_2021}.
The latter task involves reconstructing missing parts of a 3D object or scene from incomplete input data and has shown to be relevant for robotics applications \cite{completionHumanoids,3dsgrasp}.
Similarly to the OCMG framework, the input is a free-form 3D shape and the output is unstructured, \ie, the unordered set of points that fill missing input regions.
%
Inspired by these methods, our work leverages the expressive power of 3D deep learning architectures and adapts them to predict unstructured robotic paths that generalize to new object instances.
%%%%%%%%%%%%%

\begin{table*}[t]
\centering
\caption{Literature review with a selected number of exemplary works in fields of applications related to OCMG. We dissect each work to shed light on the differences and similarities with our problem setting.}
\label{tab:literature_review}
\def\arraystretch{1.25}%
\resizebox{\linewidth}{!}{
\begin{NiceTabular}{|c|c|c|c|c|c|c|c|c|c|l|}
\CodeBefore
  \rectanglecolor[gray]{0.9}{1-12}{2-0}
\Body
\hline 
%\rowcolor[HTML]{EFEFEF}
\multirow{4}{*}{Tasks} & \multirow{4}{*}{Works}  & \multirow{4}{*}{Input} & \multicolumn{4}{c}{Output} & \multirow{4}{*}{Method} & \multicolumn{3}{c|}{Pros (+) and Cons (-)}\\ %\cline{4-7} \cline{9-11}
& & & \makecell{Multiple \\ paths} & \makecell{Variable \\ num. of \\paths} & \makecell{Variable \\ path \\ lengths} & \makecell{Long \\ horizon \\ paths} & & \makecell{Fast \\ Inference \\ (+)} & \makecell{Ability to \\ Generalize \\ (+)} & \multicolumn{1}{c|}{For Painting Applications} \\ \hline

\multirow{4}{*}{Spray Painting} & 
\cite{atkar24uniform,Andulkar_Incremental_2015} & \multirow{2}{*}{\makecell[c]{3D \\ (convex only)}}&
  \cmark &
  \xmark &
  \xmark  &
  \cmark &
  %Task-specific &
  \multirow{2}{*}{\makecell[c]{Task-specific \\ Heuristics}} &
  \multirow{2}{*}{\xmark} &
  \multirow{2}{*}{\xmark} & 
  \multirow{2}{*}{\makecell[l]{(+) High paint coverage \\ (-) High design costs and manual tuning}} \\
  %(+) High paint coverage \\ 
\cline{2-2} 
\cline{4-7}

  & \cite{Sheng_Automated_2000,Biegelbauer_Inverse_2005,Chen_Automated_2008,Li_Automatic_2010} & & %(convex only) & 
  \cmark &
  \cmark &
  \cmark  &
  \cmark & %Heuristics 
  & & & %(-) High design costs and manual tuning
  \\ 
  \cline{2-11}

& \cite{Kiemel_PaintRL_2019,jonnarth2024learningcoverageicml} &
  2D &
  \xmark &
  \xmark &
  \xmark  &
  \cmark &
  \makecell{Reinforcement\\ Learning} & 
  \xmark  &
  \xmark &
  \makecell[l]{(+) Explicit paint coverage optimization\\ (-) Requires accurate simulation} \\ \hline

\makecell{Multi-Agent \\ Visual Inspection} & 
\cite{jing2020multi,multiUAV_2023} & 3D &
  \cmark &
  \xmark &
  \cmark  &
  \cmark &
\makecell{Coverage \\ Path Planning} &
  \xmark &
  \xmark & 
  \makecell[l]{(+) High inspection coverage\\ (-) Sample-specific hyperparameters\\ (-) Unable to model painting patterns}
  \\ \hline

\multirow{3}{*}{\makecell{Multi-Agent \\ Motion Prediction}} & \multirow{2}{*}{\cite{yuan2021agentformer,ngiam2022scenetr,varadarajan2022multipath++,nayakanti2023wayformer}} & \multirow{2}{*}{\makecell{2D map \\ + agent features }}  & 
\multirow{2}{*}{\cmark} &
\multirow{2}{*}{\xmark} &
\multirow{2}{*}{\xmark} &
\multirow{2}{*}{\xmark} &
\multirow{7}{*}{\makecell{Imitation and \\ Supervised Learning}} &
\multirow{7}{*}{\cmark} &
\multirow{7}{*}{\cmark} &
\multirow{7}{*}{\makecell[l]{(+) Learns painting patterns from data\\ (+) Little domain knowledge required\\ (-) Implicit paint coverage optimization}} \\ 

& & %+ agent features 
& & & & & & & & \\ \cline{2-7}

& \cite{gu2023vip3d} & 3D &
\cmark &
\cmark &
\xmark & 
\xmark &
 & &&\\ \cline{1-7}

\makecell{3D Shape\\ Completion} &
\cite{Yuan_Pcn_2018,Alliegro_Denoise_2021}
& 3D & \multicolumn{4}{c|}{$\sim$ Point-wise predictions}
&
& &&\\ \cline{1-7}

\makecell{\textbf{Object Centric} \\ \textbf{Motion Generation}} & \textbf{Ours} & 3D &
\cmark &
\cmark &
\cmark & 
\cmark
&
& &&
\\ \hline




\end{NiceTabular}%
}
\end{table*}

%%%%%%%%%%%%%



\subsection{Robotic Spray Painting}
\label{sec:painting}
Autonomous robotic spray painting is an instance of the NP-hard \emph{Coverage Path Planning} (CPP) problem with additional challenges arising from the non-linear dynamics of paint deposition and hard-to-model engineering requirements. 
Due to its complexity, the landscape of robotic spray painting is dominated by heuristic methods operating under simplifying assumptions about the output path structure (\eg, raster patterns) and object geometry (\eg, convex surfaces)~\cite{Sheng_Automated_2000,Chen_Automated_2008,Li_Automatic_2010,Andulkar_Incremental_2015,atkar24uniform,Chen_Trajectory_2020,gleeson2022generating}.
Most methods further require a 3D mesh or the full CAD model of the object, while point clouds---which are easier to obtain in real-world scenarios through laser scanning---are only considered in~\cite{Chen_Trajectory_2020}.
% 
Critically, all existing heuristics yet assume to work with objects that can be partitioned into convex or low-curvature surfaces. This renders them inapplicable for painting concave objects such as shelves and containers, where global reasoning and more complex path patterns are required. 
%
Other works rely on matching the objects with a combination of hand-designed elementary geometric components collected in a database~\cite{Biegelbauer_Inverse_2005}.
Matching components are associated with local painting strokes, which are then merged to form painting paths.
Despite its merits, this method requires costly work by experts to explicitly codify object parts and their corresponding painting procedures for each object family and is unable to generalize to arbitrary free-form objects.
%
Recently, Gleeson et al.~\cite{gleeson2022generating} proposed a trajectory optimization procedure for spray painting that targets the adaptation of an externally provided trajectory candidate, without directly handling motion generation. 

\emph{Reinforcement Learning} (RL) has alternatively been employed to train path generators by directly optimizing objectives such as paint coverage~\cite{Kiemel_PaintRL_2019} or total variation~\cite{jonnarth2024learningcoverageicml}, but these efforts have so far been confined to planar domains.
RL-based stroke sequencing has also shown success in reconstructing 2D images~\cite{Huang_Learning_2019}. Although promising, RL is yet to be demonstrated successful for long-horizon 3D object planning due to the high dimensionality of the state and action spaces. The need for an accurate simulator and low generalizability of RL agents to novel objects also stand out as major issues.

Overall, we remark that all the aforementioned works only show results on a few proprietary object instances. They do not release either the data or the method implementation to allow a fair comparison, besides lacking a discussion on the generalization to new object instances and categories.

Within the CPP literature, robotic applications for multi-agent visual inspection of 3D objects also share important similarities to the OCMG setting.
Recent works proposed optimization-based methods for planning multiple paths and demonstrated their effectiveness in multi-UAV missions on large structures~\cite{jing2020multi,multiUAV_2023}. While these methods can effectively generate long-horizon paths around both convex and concave surfaces, 
they rely on sample-specific hyperparameters, incur high computational costs, and are unable to replicate expert painting patterns.

Table \ref{tab:literature_review} provides a summary of the most relevant publications showing how existing settings and tasks in the literature relate to the OCMG problem.

\section{Method}
\label{sec:method}
\begin{figure}[!t]
    \centering
    \includegraphics[width=\linewidth]{02_images/notation_helper/notation_helper_v2.pdf}
    \vspace{-16pt}
    \caption{Schematic illustration of a data sample $(\bO, \bY)$ describing input and output of the OCMG problem. The output paths are unstructured as they vary in number and length depending on the input object and can be executed in arbitrary order.}
    \label{fig:notation_helper}
\end{figure}


\subsection{Problem statement}
We formalize the OCMG problem as finding a mapping from a 3D object point cloud to a set of unstructured output paths (see Fig.~\ref{fig:notation_helper}). 
Given expert demonstrations to learn from, we aim to generate accurate paths for previously unseen objects. 

Let $\bO$ represent the object geometry as a point cloud consisting of an arbitrary number of points in 3D space.
%
Each object $\bO$ is associated with a ground truth set of paths $\bY {=} \{\by^i \}_{i=1}^{n(\bO)}$, with object-dependent cardinality $n(\bO)$. 
Every path $\by^i = (\bp^i_1, \dots, \bp^i_T) \in \mathcal{Y} \subseteq \mathbb{R}^{6 \cdot T}$ is encoded as a sequence of 6D poses $\bp \in \mathbb{R}^6$. For simplicity, we consider the dimension $T$ fixed, with shorter paths zero-padded to reach that maximum length.  

Under this formulation, we consider the problem of finding a function $f : 2^{\mathbb{R}^{3}} \rightarrow 2^{\mathcal{Y}}$ mapping the set of points $\bO$ describing the object geometry to the set of desired paths $\bY$\footnote{For a set \( \mathcal{S} \), the notation \( 2^{\mathcal{S}} \) here denotes the powerset of \( \mathcal{S} \), \ie, the set of all subsets of \( \mathcal{S} \).}.
To do so, we parametrize $f$ using a deep neural network, and train it through empirical risk minimization~\cite{vapnik1991principleserm}.
%
Specifically, we minimize a loss function $\mathcal{L}(\hat{\bY}, \bY)$, which quantifies the discrepancy between the predicted paths $\hat{\bY}=f(\bO)$ and the ground truth paths $\bY$ on the training data, using gradient descent to optimize the network parameters.

%
We highlight that this formulation does not make task-specific assumptions related to the spray painting problem, making our contribution applicable to a broad range of object-centric motion generation tasks (\eg, welding or cleaning).
%
%

\subsection{Method Overview}
We tackle object-centric motion generation with a tailored deep learning model that copes with unstructured input---3D point clouds---and unstructured output paths.
More precisely, instead of directly predicting a set of paths, we decompose the problem into the prediction of unordered path segments, \ie, short sequences of 6D end-effector poses (or \emph{waypoints}).
%
In addition, we concurrently predict a set of probability masks that identify which segments belong to the same path.
We denote our method as \emph{\ours}, and emphasize that all required path segments and masks are predicted by our network in parallel, with a single forward pass.
Such approach induces our model to learn end-to-end global embeddings of the input object that allow for both local (segments) and global (masks) planning decisions in one step.
Overall, by designing a joint segment and mask prediction pipeline, we conveniently simplify the problem of dealing with unstructured paths and address 

\begin{itemize}[leftmargin=*,itemsep=2pt]
\item \textbf{long-horizon paths:} we do not impose constraints on the shape or length of each path.
Diverse path configurations can be handled just as effectively, as exemplified in Fig.~\ref{fig:temporal_correlation}.
%
\item \textbf{unordered paths:} within a set prediction framework, the order of the predicted path segments is irrelevant. Therefore, concatenating segments that belong to the same path naturally yields a set of output paths that are invariant by permutation.
%
\item \textbf{variable length and number of paths:} we predict a conservatively large number of path segments, allowing for the generation of redundant overlapping segments that can be easily filtered out.
\end{itemize}

Our method takes inspiration from the Panoptic Segmentation (PanSeg) literature~\cite{carion2020detr,cheng2021maskformer}, where a variable number of class instances shall be predicted given a static environment with global information (an RGB image).
Notably, works in the field of PanSeg shifted towards one-shot predictors over the years as opposed to multiple-stage or autoregressive approaches.
Similarly, we depart from sequential methods for OCMG, as we aim to reach real-time inference capabilities and avoid compounding errors on long-horizon predictions.


\begin{figure}[!t]
    \centering
    \includegraphics[width=\linewidth]{02_images/temporal_correlation/fig_global_v3.pdf}
    \vspace{-8pt}
    \caption{Example of two arbitrary configurations of ground truth paths $\bY$ on an L-shaped 2D object.
    Notice how \ours can easily manage both cases by breaking down the learning problem into the prediction of path-agnostic segments and their associated path masks.
    }
    \label{fig:temporal_correlation}
    \vspace{-6pt}
\end{figure}

Throughout this work, we demonstrate that \ours is capable of predicting a large number of long-horizon paths with a single forward pass, that implicitly incorporate task-specific requirements without domain knowledge.
We remark that ad-hoc trajectory optimization methods~\cite{gleeson2022generating} may still be applied to impose task-specific kinodynamic constraints (\eg  reachability and collision avoidance), which we consider subsequent and complementary to the scope of this work. 
Here, we directly interpolate the predicted sequence of 6D waypoints for execution on a real robot, resulting in feasible trajectories out of the box.

In the following part of this section we describe each step of our method in detail, breaking it down into segment predictions (Sec.~\ref{sec:segments_prediction}), mask predictions (Sec.~\ref{sec:masks_prediction}), and postprocessing (Sec.~\ref{sec:postprocessing}).
%

\begin{figure*}[!t]
    \centering
    \includegraphics[width=\linewidth]{02_images/method/method_v9_withNotation.pdf}
    \caption{Overview of the training pipeline of our method (\ours). Global features are learned from a point cloud representation of the input object, and used to concurrently predict path segments and path masks, in a single forward pass.
    }
    \label{fig:method}
    \vspace{-6pt}
\end{figure*}

%%%%%%%%%%%%%%%%%%%%%%%%%%%%%%%%%%%%%%%%%%%%%%%%


\subsection{Segment Predictions}
\label{sec:segments_prediction}

Let $\bS$ be the set of ground truth path segments $\bS {=} \{\bs^j\}_{j=1}^{k(\bY)}$ of an object $\bO$ and its associated paths $\bY$.
We define a segment as a sequence of $\lambda \in \mathbb{N}^+$ poses obtained from a path $\by^i \in \bY$.  
Namely, $\bs^j = (\bp^i_t, \dots, \bp^i_{t+\lambda-1}) \in \mathbb{R}^{6 \cdot\lambda}$ for some timestep $t{=}1, \dots , T-\lambda$.
We derive all segments in \(\bS\) by striding along all paths with a step of $\lambda-1$, \ie considering an overlap of one pose between consecutive segments.
Notice that the total number of resulting segments $k(\bY)$ depends on the number of paths and their lengths.

%
We design our model to take the object point cloud $\bO$ as input, and predict a set of path segments $\bShat{=}\{\bshat^j\}_{j=1}^{K}$ that approximate the true segments $\bS$. 
We adopt the PointNet++ architecture~\cite{Qi_Pointnet++_2017} as basic backbone for global feature extraction from $\bO$, followed by a fully connected 3-layer decoder that jointly outputs all path segments. 
A fixed number of segments $K = \max k(\bY)$ is predicted to ensure all ground truth segments are recovered for all objects.

Let $\bP$ be the set of unordered ground truth waypoints, 
formally described as 
$\{\bp^i_t {\in} \mathbb{R}^6 \ | \ i {\in} [1, \dots, n(\bO)], \ t{\in} [1, \dots, T] \}$
---equivalent to the set of segments for $\lambda{=}1$. Analogously, we define  
$\bPhat{=}\{ \bshat^j_t {\in} \mathbb{R}^6 \ | \ j {\in} [1, \dots, K], \ t {\in} [1, \dots, \lambda] \}$
as the set of individual predicted waypoints, obtained by interpreting $\bShat$ as an unordered collection of waypoints.

\ours is trained with a novel loss function aimed at driving the prediction of path segments $\bShat$ close to the ground truth segments $\bS$ by means of Euclidean distances in $\mathbb{R}^{6\cdot\lambda}$ space.
To do this, our loss includes auxiliary point-wise terms that penalize prediction errors in the lower dimensional space $\mathbb{R}^{6}$, by disregarding how waypoints are arranged into segments.
Overall, our \emph{Point-to-Segment Chamfer Distance} (P2S-CD) is defined as:
\begin{equation}
\label{eq:pscd}
\begin{split}
    \mathcal{L}_{p2s}(\bPhat ,\bP,\bShat ,\bS) & = \\
    & w^{f}_{p} \cdot d_{ACD}(\bPhat , \bP) + w^{f}_{s} \cdot d_{ACD}(\bShat , \bS) \ + \\
    & \underbrace{\raisebox{10pt}{$w^{b}_{p} \cdot d_{ACD}( \bP,\bPhat)$}}_{\text{Point-wise}} \raisebox{10pt}{$+$} \underbrace{\raisebox{10pt}{$w^{b}_{s} \cdot d_{ACD}(\bS,\bShat)$~.}}_{\text{Segment-wise}}
\end{split}
\end{equation} 
Here, the \emph{Asymmetric Chamfer Distance} (ACD) \emph{from} set $A$ \emph{to} set $B$ is: 
\begin{equation}
    d_{ACD}(A, B) = \frac{1}{|A|} \sum _{a\in A}\min_{b\in B} \| a-b\| _{2}^{2}~,
\end{equation}
and the parameters $w^f_p, w^f_s, w^b_p, w^b_s \in \mathbb{R}^{+}$ weight the computation of the four ACD terms ($f$: forward, $b$: backward, $p$: point-wise, $s$: segment-wise). 

%%%%%%%%%%%%%%%%%%%%%%%%%%%%%%%%%%%%%%%%%%%%%%%%%%%%%%%%%%%%%
Based on this formulation, we propose an auxiliary point-to-segment curriculum that (1) first focuses on matching waypoints in some random permutation (point-wise terms), and then (2) gradually promotes local structure by computing distances among sequences of poses (segment-wise terms).
%
To achieve this, we start the training with a dominant point-wise term ($w^b_p {\gg} w^b_s$) and progressively converge to equal point-wise and segment-wise contributions ($w^b_p{=}1, w^b_s{=}1$).
Notably, this procedure is asymmetric and only applied on the backward terms: this ensures that predictions are pulled sufficiently close to the ground truth poses, which in turn enables effective global coverage of the desired poses, and limits the generation of clusters of poses~\cite{densityawarecd}.
%
On the other hand, we keep the forward weights fixed to $w^f_p{=}0$, $w^f_s{=}1$ throughout the training, promoting the generation of segments that are smooth and locally accurate.
%
We refer to this process as the \emph{Asymmetric Point-to-Segment} (AP2S) curriculum, and report a schematic illustration of our overall loss function in Fig.~\ref{fig:apscurriculum}.

\begin{figure}[tb]
    \centering
    \includegraphics[width=\linewidth]{02_images/asymmCurrChamferDistance/asymm_curr_CD_v7_blu.pdf}
    \caption{
    Illustration of our Asymmetric Point-to-Segment curriculum for segment predictions ($\lambda{=}3$). The parameters $w^b_p,w^b_s$ weighting the backward point-wise and segment-wise ACD terms vary during training.}
    \vspace{-6pt }
    \label{fig:apscurriculum}
\end{figure}

\begin{figure*}[!t]
    \centering
    \includegraphics[width=\linewidth]{02_images/postprocessing/postprocessing_v2.pdf}
    \caption{
    Postprocessing: concatenation of the set of predicted segments belonging to the same path mask.
    In step (1) the figure depicts raw network predictions with $\lambda{=}4$, with separate segments differentiated by color. Step (2) shows the effect of segment filtering. In step (3) and (4), where the path is identified and further refined, the ordered sequence of waypoints is shown with a color gradient.}
    \label{fig:postprocessing}
    % \vspace{-12pt}
\end{figure*}

%%%%%%%%%%%%%%%%%%%%%%%%%%%%%%%%%%%%%%%%%%%%%%%%%%%%
\subsection{Mask Predictions}
\label{sec:masks_prediction}

Our model concurrently predicts a set of probability masks over predicted segments, indicating which segments belong to the same path. 
Defining a supervised learning objective for this task is not trivial, as it requires comparing predicted masks on generated segments with ground truth masks on generated segments. However, the latter do not exist.
%
A reasonable choice is to match each predicted segment with the closest ground truth segment---as already proposed in Sec.\ref{sec:segments_prediction}---and construct target masks accordingly.
We refer to this strategy as \emph{nearest-neighbour label association}, and use it to project masks over ground truth segments onto predicted segments.
In particular, let $\bM{=}\{\bm^i\}_{i=1}^{n(\bO)}$ be the set of \emph{target path masks} for some input pair $(\bO,\bY)$.
Each element $\bm^i \in \{0,1\}^K$ encodes the set of predicted segments that belong to path $i$, that is
%
\begin{equation}
\begin{split}
    & \bm^{i}_{j} =\begin{cases}
1 & \mathrm{if\ } \mathrm{NN}(\bshat^j) \ \mathrm{belongs \ to \ path} \ i\\
0 & \mathrm{otherwise},
\end{cases} \\
& \forall \ j = 1, \dots, K
\end{split}
\end{equation}
where $\mathrm{NN}(\bshat) {=} \argmin_{\bs \in \bS} \| \bs-\bshat \|^2$. 
%
Our model is designed to predict a set of $N\geq n(\bO)$ probability masks $\bMhat{=}\{ \bmhat^{i} \}_{i=1}^{N}$, where $N{=}\max n(\bO)$ is the maximum number of paths across objects $\bO$ in the training set. 
All masks are predicted in parallel via a 3-layer MLP decoder from the global features of the object, followed by a sigmoid activation.
Note that, as in~\cite{cheng2021maskformer}, we do not force mask predictions to be mutually exclusive for a given segment $\bshat^j$. Hence we avoid using a softmax activation.

We drive the predicted path masks $\bMhat$ towards the target path masks $\bM$ through the Binary Cross Entropy (BCE) loss:
\begin{equation}
\label{eq:bce_loss}
    \mathcal{L}_{bce}(\bmhat, \bm) = - \sum_{j=1}^{K}\Bigl[\bm_j\cdot\log(\bmhat_j) + (1-\bm_j)\cdot\log(1-\bmhat_j)\Bigl].
\end{equation}
Importantly, we must consider the prediction of permutation invariant masks to cope with a set of unordered paths.
Therefore, we assign each predicted path mask $\bmhat$ to a target mask $\bm$ by finding a bijection $ \sigma : \bMhat \rightarrow \bM$.
%
Similarly to~\cite{carion2020detr,cheng2021maskformer}, we do this by solving a bipartite matching problem between the two sets, where the assignment costs are computed using $\mathcal{L}_{bce}$.
%
Particularly, we pad the target masks with a ``no path'' token $\varnothing$ to allow one-to-one matching.
Ultimately, the training loss for mask prediction is as follows:
\begin{equation}
\begin{split}
    % \mathds{1}_{4+3}
    \mathcal{L}_{mask}(\bMhat, \bM) = \\ \sum_{i=1}^{N}
    \Bigl[ 
        % \mathcal{L}_{bce}(c^i, \mathds{1}_{\bm^i \neq \varnothing})
        & c^{\sigma(i)}\cdot\log(\hat{c}^i) + (1-c^{\sigma(i)})\cdot\log(1-\hat{c}^i) \\
        & + c^{\sigma(i)} \cdot
         \mathcal{L}_{bce}(\bmhat^i, \bm^{\sigma(i)})
    \Bigl]~,
\end{split}
\end{equation}
where $\hat{c}^i \in [0,1]$ is a learned confidence score for each mask, and $c^i{=}\mathds{1}_{\bm^i \neq \varnothing}$ indicates the true masks.
%%%%%%%%%%%%%%%%%%%%%%%%
Overall, we train our model to minimize $\mL = \mL_{p2s} + \mL_{mask}$. 
We display a schematic description of our training pipeline in Fig.~\ref{fig:method}.

At inference time, we derive the groups of segments belonging to the same path by assigning each predicted segment $\bshat^j$ to one of the $N$ predicted masks. Formally, the assignment of segment $\bshat^j$ occurs as follows:
\begin{equation}
\label{eq:path_mask_assignment}
    \begin{split}
        \underset{i \in [1, \dots, N]}{\argmax} \ \bmhat^i_j \ \ \text{s.t.} \ \ \hat{c}^i \geq 0.5~.
    \end{split}
\end{equation}
In other words, segments are assigned to the path mask with the highest predicted probability, discarding path masks that are predicted as ``no path''. Assigning all predicted segments according to Eq.~(\ref{eq:path_mask_assignment}) results in a final number of $\hat{n}(\bO)\leq N$ paths predicted by our model given the input object, ideally equal to $n(\bO)$.


%%%%%%%%%%%%%%%%%%%%%%
\subsection{Postprocessing: Segment Concatenation}
\label{sec:postprocessing}
%
A final postprocessing step is applied to concatenate the subset of predicted segments $\bShat^i \subseteq \bShat$ that are assigned to the same path mask $\bmhat^i$, and produce an ordered sequence of 6D waypoints, \ie the executable path.

Viable solutions may include solving an open Traveling Salesman Problem (TSP) among segments or employing learning-based approaches for ranking. Here, we adopt a simple and effective concatenation strategy based on segment proximity and alignment. 

First, predicted segments in excess are removed by discarding segment pairs whose distance falls below a predefined threshold, proceeding in ascending order of pairwise distances\footnote{Note that our loss $\mathcal{L}_{p2s}$ implicitly promotes overlapping if the number of predicted segments $K$ is higher than the ground truth segments $k(\bY)$.}. 
%
Then, we find an optimal path that connects the retained segments. Consider the set of segments in $\bShat^i$ as nodes of a directed graph. An edge among two segments is weighted based on the proximity in space and orientation between the starting and ending poses of the two segments, as well as the similarity in segment directions.
More formally, the assigned cost to the edge from $\bshat^j$ to $\bshat^k$ is 
\begin{equation}
\label{eq:cost}
    C(\bshat^j,\bshat^k) = \| \bshat^j_\lambda - \bshat^k_1 \|_2^2 + w_v \cdot \| (\bshat^j_\lambda - \bshat^j_{\lambda-1}) - (\bshat^k_2 - \bshat^k_1)  \|_2^2~.
\end{equation}
Here, $w_v \in \mathbb{R}^+$ is a trade-off weight between the two terms, and the segments' subscripts specify the index of a particular pose among the $\lambda$ poses that make up each segment.

We find the optimal concatenation by employing the Edmonds' algorithm~\cite{edmonds1967optimum} to the $k$-nearest neighbor graph of segments $\bShat^i$ constructed using $C(\cdot, \cdot)$ and $k{=}5$.
Ultimately, we extract the longest path from the resulting Directed Acyclic Graph (DAG) and obtain the final \emph{ordered} sequence of predicted segments.
At this point, filtering techniques such as interpolation, upsampling, and smoothing may be conveniently applied.
The same process is repeated independently for all predicted path masks.
For clarity, we illustrate each step of the postprocessing in Fig.~\ref{fig:postprocessing}.

\section{The Extended PaintNet Dataset}
\label{sec:ext_paintnet_dataset}
\begin{figure*}[!t]
    \centering
    \includegraphics[width=\linewidth]{02_images/dataset_visualization/dataset_visualization_v7_zoom10.pdf}
    % \vspace{1pt}
    \caption{Overview of a number of representative instances of the Extended PaintNet dataset, featuring realistic spray painting demonstrations designed by experts. The dataset is divided into four categories of growing complexity.
    Different colors represent separate paths.
    }
    \label{fig:dataset_overview}
    \vspace{-6pt}
\end{figure*}

PaintNet~\cite{tiboni2023paintnet} was the first dataset of expert demonstrations introduced to support the study of motion generation conditioned on free-form 3D objects, and was specifically designed for the spray painting task. In this work, we expand PaintNet more than threefold, resulting in a new version that contains 3088 samples.

Every sample is a pair of a 3D object and its corresponding spray painting paths. Each object is represented as a triangle mesh, with vertex coordinates expressed in real-world millimeter scale. The meshes are provided in an aligned and smoothed watertight~\cite{Huang_Robust_2018} format with any private information (\eg, engraved logos) accurately anonymized.
%
The number of spray painting paths associated to an object varies according to the geometry of the object. Each path is encoded as a sequence of end-effector configurations in task space, \ie, positions and orientations of the gun nozzle. More precisely, we record 3D positions as the ideal paint deposition point, 12cm away from the gun nozzle, and 3D gun orientations as Euler angles.
%
The data pairs represent realistic spray painting demonstrations that a real robot could directly execute. In other words, they encode feasible trajectories designed by experts to reach near-complete coverage of the whole surface of the 3D objects. Each waypoint is collected by sampling the end-effector pose at a rate of 250Hz during execution. 
%
Ground-truth paths are produced ad-hoc for each object category by custom heuristics based on long-standing experience in the field.

The four object categories composing the dataset are presented below, ordered by increasing complexity:

\begin{itemize}[leftmargin=*,itemsep=2pt]
 \item \textbf{Cuboids}: a basic class of 1000 rectangular cuboids tailored for testing models under minimal generalization requirements and relatively simple path patterns. Cuboids are sampled with varying height and depth uniformly from $0.5m$ to $1.5m$, while having a fixed width of $1m$. Their volume ranges from $0.25m^3$ to $2.25m^3$. A fixed number of six raster-like paths is associated to each cuboid to paint the exterior faces, with gun orientations that are perpendicular to the surface at all times.

\item \textbf{Windows}: a set of 1000 window-like data pairs, with width and height varying uniformly from $0.4\,m$ to $1.8\,m$, and a fixed thickness of $4\,cm$. For each sample, up to $3$ horizontal cross sections and up to $1$ vertical cross section are randomly selected. Windows introduce harder challenges for motion generation, such as predicting a variable, high number of paths while handling non-trivial gun orientations and geometric patterns.

\item \textbf{Shelves}: a set of 1000 shelves featuring highly concave surfaces. The instances differ significantly in volume and number of inner compartments. Their volume ranges from $18dm^3$ to $160dm^3$, with up to 6 inner compartments for the larger samples.

\item \textbf{Containers}: a set of 88 industrial containers including meshes with highly heterogeneous global and local geometric properties (\eg, wavy and grated surfaces). Here, the manually-guided painting paths designed by experts show evident irregularities across instances. This class of objects is particularly challenging due to the limited number of samples.
\end{itemize}
%
An overview of the Extended PaintNet dataset is depicted in Fig.~\ref{fig:dataset_overview}.
The dataset is publicly available at \url{https://gabrieletiboni.github.io/MaskPlanner/}.
%

\section{Experimental Evaluation}
\label{sec:experiments}
\section{Experimental Evaluation}\label{section:experiments}
We already achieved our primary objective of deriving time-series-specific subsampling guarantees for DP-SGD adapted to forecasting.
Our main goal for this section is to investigate the trade-offs we discovered in discussing these guarantees.
In addition, we train common probabilistic forecasting architectures on standard datasets to verify the feasibility of training deep differentially private forecasting models while retaining meaningful utility.
The full experimental setup  is described in~\cref{appendix:experimental_setup}.
%An implementation will be made available upon publication.

\subsection{Trade-Offs in Structured Subsampling}

\begin{figure}
    \vskip 0.2in
    \centering
        \includegraphics[width=0.99\linewidth]{figures/experiments/eval_pld_deterministic_vs_random_top_level/daily_20_32_main.pdf}
        \vskip -0.3cm
        \caption{Top-level deterministic iteration (\cref{theorem:deterministic_top_level_wr}) vs top-level WOR sampling (\cref{theorem:wor_top_level_wr}) for $\numinstances=1$.
        Sampling is more private despite requiring more compositions.}
        \label{fig:deterministic_vs_random_top_level_daily_main}
    \vskip -0.2in
\end{figure}




For the following experiments, we assume that we have $N=320$ sequences, batch size $\batchsize = 32$, and noise scale $\sigma = 1$.
We further assume $L=10  (L_F + L_C) + L_F - 1$, so that 
the chance of bottom-level sampling a subsequence containing any specific element is 
$r=0.1$ when choosing $\numinstances = 1$ as the number of subsequences.
In~\cref{appendix:extra_experiments_eval_pld}, we repeat all experiments with a wider range of parameters.
All results are consistent with the ones shown here.

\textbf{Number of Subsequences $\bm{\numinstances}$.}
Let us begin with a trade-off inherent to bi-level subsampling:
We can achieve the same batch size $\batchsize$ with different $\numinstances$, each
leading to different top- and bottom-level amplification.
We claim that $\numinstances = 1$ (i.e., maximum bottom-level amplification) is preferable.
For a fair comparison, we compare our provably tight guarantee for $\numinstances=1$ (\cref{theorem:wor_top_level_wr})
with optimistic lower bounds for $\numinstances > 1$ (\cref{theorem:wor_top_wr_bottom_upper})
instead of our sound upper bounds (\cref{theorem:wor_top_level_wr_general}), i.e.,
we make the competitors stronger.
As shown in~\cref{fig:monotonicity_daily_main}, $\numinstances = 1$ only has smaller $\delta(\epsilon)$ for $\epsilon \geq 10^{-1}$ when considering a single training step.
However, after $100$-fold composition, $\numinstances = 1$ achieves smaller $\delta(\epsilon)$ even in $[10^{-3}, 10^{-1}]$ (see~\cref{fig:monotonicity_composed_daily_main}).
Our explanation is that $\numinstances > 1$ results in larger $\delta(\epsilon)$ for large $\epsilon$, i.e., is more likely to have a large privacy loss.
Because the privacy loss of a composed mechanism is the sum of component privacy losses~\cite{sommer2018privacy}, this is problematic when performing multiple training steps.
We shall thus later use $\numinstances=1$ for training.

%Intuitively, $\delta(\epsilon)$ can be interpreted as the probability that the log-likelihood ratio of $M_x$ and $M_{x'}$ (``privacy loss'') exceeds $\epsilon$.\footnote{For the formal relation between privay loss and privacy profiles, see~\cref{lemma:profile_from_pld} taken from~\cite{balle2018improving}}


\textbf{Step- vs Epoch-Level Accounting.}
Next, we show the benefit of top-level sampling sequences (\cref{theorem:wor_top_level_wr}) instead of deterministically iterating over them (\cref{theorem:deterministic_top_level_wr}), even though we risk privacy leakage at every training step.
For our parameterization and $\numinstances=1$, top-level sampling with replacement requires $10$ compositions per epoch.
As shown in~\cref{fig:deterministic_vs_random_top_level_daily_main}, the resultant epoch-level profile is nevertheless smaller, and remains so after $10$ epochs.
This is consistent with any work on DP-SGD (e.g., \cite{abadi2016deep}) that uses subsampling instead of deterministic iteration.

\textbf{Epoch Privacy vs Length.} In~\cref{appendix:extra_experiments_epoch_length} we additionally explore the fact that, if we wanted to use deterministic top-level iteration, 
the number of subsequences 
$\numinstances$ would affect epoch length.
As expected, we observe that composing many private mechanisms ($\numinstances=1$) is preferable to composing few much less private mechanisms ($\numinstances > 1$) 
when considering a fixed number of training steps.

\begin{figure}
    \vskip 0.2in
    \centering
        \includegraphics[width=0.99\linewidth]{figures/experiments/eval_pld_label_noise/daily_30_32_main.pdf}
        \vskip -0.3cm
        \caption{Varying label noise $\sigma_F$ for top-level WOR and bottom-level WR  (\cref{theorem:data_augmentation_general}) with $\sigma_C = 0, \numinstances=1$.
        Increasing $\sigma_F$ is equivalent to decreasing forecast length.
        }
        \label{fig:label_noise_daily_main}
    \vskip -0.2in
\end{figure}

\textbf{Amplification by Label Perturbation.}
Finally, because the way in which adding Gaussian noise to the context and/or forecast window 
amplifies privacy (\cref{theorem:data_augmentation_general}) 
may be somewhat opaque, let us consider top-level sampling without replacement, bottom-level sampling with replacement,
$\numinstances=1$, $\sigma_C=0$, and varying label noise standard deviations $\sigma_F$. 
As shown in~\cref{fig:label_noise_daily_main}, increasing $\sigma_F$ has the same effect as letting the forecast length $L_C$ go to zero, i.e., eliminates the risk of leaking private information if it appears in the forecast window.
Of course, this data augmentation 
will have an effect on model utility, which we investigate shortly.

\begin{figure*}
\centering
\vskip 0.2in
    \begin{subfigure}{0.49\textwidth}
        \includegraphics[]{figures/experiments/eval_pld_monotonicity_composed/daily_20_32_1_main.pdf}
        \caption{Training step $1$}\label{fig:monotonicity_daily_main}
    \end{subfigure}
    \hfill
    \begin{subfigure}{0.49\textwidth}
        \includegraphics[]{figures/experiments/eval_pld_monotonicity_composed/daily_20_32_100_main.pdf}
        \caption{Training step $100$}\label{fig:monotonicity_composed_daily_main}
    \end{subfigure}\caption{
    Top-level WOR and bottom-level WR sampling under varying number of subsequences.
    Under composition, even optimistic lower bounds (\cref{theorem:wor_top_wr_bottom_upper}) 
    indicate worse privacy for $\numinstances > 1$ than 
    our tight upper bound for $\numinstances=1$ (\cref{theorem:wor_top_level_wr}).}
    \label{fig:monotonicity_daily_main_container}
\vskip -0.2in
\end{figure*}


\subsection{Application to Probabilistic Forecasting}
While the contribution of our work lies in formally analyzing the privacy of DP-SGD adapted to forecasting, 
training models with this algorithm can serve as a sanity-check to verify that the guarantees are sufficiently strong to retain meaningful utility under non-trivial privacy budgets.


\begin{table}[b]
\vskip -0.38cm
\caption{Average CRPS on \texttt{traffic} for $\delta=10^{-7}$. Seasonal, AutoETS, and models with $\epsilon=\infty$ are without noise.}
\label{table:1_event_training_traffic_main}
\vskip 0.18cm
\begin{center}
\begin{small}
\begin{sc}
\begin{tabular}{lcccc}
\toprule
Model & $\epsilon = 0.5$ & $\epsilon = 1$ & $\epsilon = 2$ &  $\epsilon = \infty$ \\
\midrule
SimpleFF & $0.207$ & $0.195$ & $0.193$ & $0.136$ \\ 
DeepAR & $\mathbf{0.157}$ & $\mathbf{0.145}$ & $\mathbf{0.142}$ & $\mathbf{0.124}$ \\
iTransf. & $0.211$ & $0.193$ & $0.188$ & $0.135$ \\
DLinear & $0.204$ & $0.192$ & $0.188$ & $0.140$ \\
\midrule
Seasonal   & $0.251$ & $0.251$ & $0.251$ & $0.251$\\
AutoETS   & $0.407$ & $0.407$ & $0.407$ & $0.407$\\
\bottomrule
\end{tabular}
\end{sc}
\end{small}
\end{center}
\vskip -0.1in
\end{table}

\textbf{Datasets, Models, and Metrics.}
We consider three standard benchmarks: \texttt{traffic}, \texttt{electricity}, and \texttt{solar\_10\_minutes} as used in~\cite{Lai2018modeling}.
We further consider four common architectures: 
A two-layer feed-forward neural network (``SimpleFeedForward''), a recurrent neural network (``DeepAR''~\cite{salinas2020deepar}),
an encoder-only transformer (``iTransformer''~\cite{liu2024itransformer}), and a refined feed-forward network proposed to compete with attention-based models (``DLinear''~\cite{zeng2023transformers}).
We let these architectures parameterize elementwise $t$-distributions to obtain probabilistic forecasts.
We measure the quality of these probabilistic forecasts using continuous ranked probability scores (CRPS), which we approximate via mean weighted quantile losses (details in~\cref{appendix:metrics}).
As a reference for what constitutes ``meaningful utility'', we compare against seasonal na\"{i}ve forecasting and exponential smoothing (``AutoETS'') without introducing any noise.
All experiments are repeated with $5$ random seeds.


\textbf{Event-Level Privacy.} \cref{table:1_event_training_traffic_main} shows CRPS of all models on the \texttt{traffic} test set 
when setting $\delta=10^{-7}$, and training on the training set until reaching a pre-specified $\epsilon$
with $1$-event-level privacy. For the other datasets and standard deviations, see~\cref{appendix:privacy_utility_tradeoff_event_level_privacy}.
The column $\epsilon=\infty$ indicates non-DP training.
As can be seen, models can retain much of their utility and outperform the baselines, even for $\epsilon \leq 1$ which is generally considered a small privacy budget~\cite{ponomareva2023dp}.
For instance, the average CRPS of DeepAR on the traffic dataset is $0.124$ with non-DP training and $0.157$ for $\epsilon=0.5$.
Note that, since all models are trained using  our tight privacy analysis,
which specific model performs best  on which specific dataset is orthogonal to our contribution. 

\textbf{Other results.}
In~\cref{appendix:privacy_utility_tradeoff_user_level_privacy} we additionally train with $w$-event and $w$-user privacy.
In~\cref{appendix:privacy_utility_tradeoff_label_privacy}, we demonstrate that label perturbations can offer an improved privacy--utility trade-off. 
All results confirm that our guarantees for DP-SGD adapted to forecasting are strong enough to enable provably private training while retaining utility.


\section{Conclusions}
\label{sec:conclusions}
In this paper, we address the core problem of robot motion generation conditioned on free-form 3D objects. Particularly, we formalize the Object-Centric Motion Generation (OCMG) problem setting, aiming to unify several robotic applications under a common framework for the prediction of long-horizon, unstructured paths.
To tackle this problem, we introduce \ours, a novel deep learning method capable of inferring smooth and accurate paths directly from expert data.
%
In simple terms, \ours breaks down motion generation into the joint prediction of local path segments and probability masks, followed by a postprocessing step that concatenates the segments predicted within the same mask.
Our approach demonstrates generalization across both convex and concave 3D objects (Cuboids, Windows, Shelves), after being trained for only 6 hours on 800 samples on a single NVIDIA GeForce RTX 4080 GPU.
We validate \ours in the context of robotic spray painting, demonstrating its ability to achieve near-complete paint coverage on previously unseen objects, both in simulation and in real-world experiments.
Our method remains task-agnostic and makes minimal assumptions on both the object geometry and the output path patterns, paving the way for future applications of deep learning to OCMG tasks beyond spray painting---such as welding, sanding, cleaning, or visual inspection.
In fact, our findings showcase the potential of data-driven methods to achieve solutions to OCMG tasks that are scalable, generalizable, and cheap to deploy: models can (1) be pre-trained with data from similar tasks, (2) boosted in performance when more samples become available, (3) leverage heuristics-based demonstrations, and (4) be deployed at 10Hz on large objects.
Remarkably, we believe this approach can shine even for high-precision tasks when used in combination with task-specific trajectory optimization techniques that leverage \ours to promptly generate good starting solutions for novel objects.
%%%%%%%%%%
Furthermore, \ours enables the generation of unstructured paths from data modalities beyond 3D objects by replacing the feature encoder, opening new directions of work for applications such as agricultural coverage path planning from aerial images and multi-UAV search-and-rescue missions.

\noindent\textbf{Limitations.}  Our current pipeline relies on a simple postprocessing strategy to concatenate predicted segments, which is not able to recover from erroneous predictions and may lead to inaccurate paths in such cases. Learning-based modules for concatenation will be investigated in future work to be robust to misplaced segment predictions.
We also reckon that the choice of the segment length $\lambda{=}4$ may not work equally well across different tasks and objects.
This hyperparameter should be tuned accordingly to learn accurate isolated patterns that generalize well across the object surface.
%
Alternatively, tuning the granularity of the sampled waypoints from the expert trajectories to achieve the desired degree of sparsity in the predicted poses would serve a similar purpose.
Finally, our practical implementation simplifies gun orientations to unit vectors (from 3-DoF to 2-DoF), making the sole assumption that the task is invariant to gun rotations around the approach axis.
%
Future work can extend the pipeline to the prediction of full orientation representations---\eg, Euler angles---to tackle tasks where the additional degree of freedom is necessary, such as motion generation for object grasping.



\section*{Acknowledgments}
The authors acknowledge the EFORT group's support, which provided domain knowledge, object meshes, expert trajectory data, and access to specialized painting robot hardware for our real-world experimental evaluation.
This study was carried out within the FAIR - Future Artificial Intelligence Research and received funding from the European Union Next-GenerationEU (PIANO NAZIONALE DI RIPRESA E RESILIENZA (PNRR) – MISSIONE 4 COMPONENTE 2, INVESTIMENTO 1.3 – D.D. 1555 11/10/2022, PE00000013). This manuscript reflects only the authors’ views and opinions, neither the European Union nor the European Commission can be considered responsible for them. 


%%% APPENDIX
{\appendices
\section{Asymmetric Point-to-Segment Curriculum}
\label{appendix:pointtosegment_ablation}

When training \ours to minimize the proposed loss $\mL_{p2s}$ (see Sec.~\ref{sec:segments_prediction}), various weighting schemes can be applied to the point-wise and segment-wise Chamfer Distance (CD) terms.
In principle, an ideal learning algorithm would drive both point-wise and segment-wise matching simultaneously.
In practice, however, gradient-based optimization can converge to substantially different local optima---\eg while the Chamfer Distance is computationally efficient, it's known to be sensitive to outliers and insensitive to local mismatches in density~\cite{densityawarecd}.
In our ablation study (see Tab.~\ref{tab:loss_ablation_v2}), we evaluate four main weighting configurations:
\noindent\begin{itemize}
    \item \textbf{\ours w/out AP2S}: a baseline with a fully segment-wise loss function that leverages no auxiliary point-wise CD terms. This weighting scheme also resembles the loss function introduced in~\cite{tiboni2023paintnet}.
    
    \item \textbf{(1)~Asymmetric}: keeps a segment-wise forward CD term, but uses a point-wise loss function for the backward CD term.

    \item \textbf{(2)~P2S curriculum}: uses a coarse-to-fine schedule to progressively assign more weight to segment-wise predictions, but does so symmetrically for both forward and backward terms.

    \item \textbf{(1)~+~(2)~\ours}: our full asymmetric point-to-segment curriculum that starts training with a higher weight on the backward point-wise term and gradually balances both point-wise and segment-wise backward terms. The forward term is fully segment-wise throughout training.
\end{itemize}

Quantitative results show that \ours converges to lower PCD scores when both asymmetric CD terms and a gradual point-to-segment curriculum are included.
In particular, we observe that including the computation of auxiliary point-wise terms only helps if used in combination with the AP2S curriculum.
A variety of additional configurations have also been tried, but led to no improvements.
Moreover, we remind that na\"ively optimizing for fully point-wise CD terms yields qualitative sparse predictions that fail in capturing detailed path structures (see Point-wise baseline in Fig.~\ref{fig:qualitatives_segments_prediction}).
%
In turn, we conclude that the AP2S curriculum is crucial to promote effective convergence while ensuring smoothness and local consistency in the final predictions.

\begin{table}[]
\centering
\caption{PCD on the test set of all object categories when models are trained on $\mL_{p2s}$ with different weighting configurations (10 seeds). The ``$a\rightarrow b$" notation indicates the value varies from $a$ to $b$ across training.}
\label{tab:loss_ablation_v2}
\resizebox{\columnwidth}{!}{%
\def\arraystretch{1.25}%
\begin{tabular}{l|cl|cl|cl|cl|}
\cline{2-9}
 &
  \multicolumn{2}{c|}{\begin{tabular}[c]{@{}c@{}}\ours\\ w/out AP2S\end{tabular}} &
  \multicolumn{2}{c|}{\begin{tabular}[c]{@{}c@{}}(1)\\ Asymmetric\end{tabular}} &
  \multicolumn{2}{c|}{\begin{tabular}[c]{@{}c@{}}(2)\\ P2S Curr.\end{tabular}} &
  \multicolumn{2}{c|}{\textbf{\begin{tabular}[c]{@{}c@{}}(1) + (2)\\ \ours\end{tabular}}} \\ \cline{2-9} 
 &
  \multicolumn{2}{c|}{\begin{tabular}[c]{@{}c@{}}$w^f_p=0$\\ $w^f_s=1$\\ $w^b_p=0$\\ $w^b_s=1$\end{tabular}} &
  \multicolumn{2}{c|}{\begin{tabular}[c]{@{}c@{}}$w^f_p=0$\\ $w^f_s=1$\\ $w^b_p=1$\\ $w^b_s=0$\end{tabular}} &
  \multicolumn{2}{c|}{\begin{tabular}[c]{@{}c@{}}$w^f_p=100\rightarrow 1$\\ $w^f_s=0.01\rightarrow 1$\\ $w^b_p=100\rightarrow 1$\\ $w^b_s=0.01\rightarrow 1$\end{tabular}} &
  \multicolumn{2}{c|}{\begin{tabular}[c]{@{}c@{}}$w^f_p=0$\\ $w^f_s=1$\\ $w^b_p=100\rightarrow 1$\\ $w^b_s=0.01\rightarrow 1$\end{tabular}} \\ \hline
\multicolumn{1}{|l|}{Cuboids}    & \multicolumn{2}{c|}{7.79}            & \multicolumn{2}{c|}{11.81}  & \multicolumn{2}{c|}{9.24}   & \multicolumn{2}{c|}{\textbf{6.52}} \\ \hline
\multicolumn{1}{|l|}{Windows}    & \multicolumn{2}{c|}{7.18}            & \multicolumn{2}{c|}{12.55}  & \multicolumn{2}{c|}{11.89}  & \multicolumn{2}{c|}{\textbf{6.83}} \\ \hline
\multicolumn{1}{|l|}{Shelves}    & \multicolumn{2}{c|}{11.27}           & \multicolumn{2}{c|}{30.65}  & \multicolumn{2}{c|}{34.84}  & \multicolumn{2}{c|}{\textbf{7.43}} \\ \hline
\multicolumn{1}{|l|}{Containers} & \multicolumn{2}{c|}{\textbf{220.66}} & \multicolumn{2}{c|}{242.07} & \multicolumn{2}{c|}{240.19} & \multicolumn{2}{c|}{248.19}        \\ \hline
\end{tabular}%
}
\end{table}

} %%% END OF APPENDIX

%%% BIBLIOGRAPHY
\bibliographystyle{IEEEtran}
\balance{}
\bibliography{bibliography}

\end{document}


