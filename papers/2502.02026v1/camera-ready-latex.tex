%File: anonymous-submission-latex-2025.tex
\documentclass[letterpaper]{article} % DO NOT CHANGE THIS
\usepackage{aaai25}  % DO NOT CHANGE THIS
\usepackage{times}  % DO NOT CHANGE THIS
\usepackage{helvet}  % DO NOT CHANGE THIS
\usepackage{courier}  % DO NOT CHANGE THIS
\usepackage[hyphens]{url}  % DO NOT CHANGE THIS
\usepackage{graphicx} % DO NOT CHANGE THIS
\urlstyle{rm} % DO NOT CHANGE THIS
\def\UrlFont{\rm}  % DO NOT CHANGE THIS
\usepackage{natbib}  % DO NOT CHANGE THIS AND DO NOT ADD ANY OPTIONS TO IT
\usepackage{caption} % DO NOT CHANGE THIS AND DO NOT ADD ANY OPTIONS TO IT
\frenchspacing  % DO NOT CHANGE THIS
\setlength{\pdfpagewidth}{8.5in} % DO NOT CHANGE THIS
\setlength{\pdfpageheight}{11in} % DO NOT CHANGE THIS
\usepackage{cuted}

\usepackage{amsmath, amssymb, amsthm}
\usepackage{booktabs}
\usepackage{mathtools}
\allowdisplaybreaks

\theoremstyle{definition}
\newtheorem{definition}{Definition}

\theoremstyle{plain}
\newtheorem{theorem}{Theorem}
\newtheorem{lemma}{Lemma}

\newcommand{\argmax}{\mathop{\rm arg\,max}\limits}
\newcommand{\argmin}{\mathop{\rm arg\,min}\limits}

%
% These are recommended to typeset algorithms but not required. See the subsubsection on algorithms. Remove them if you don't have algorithms in your paper.
\usepackage{algorithm}
\usepackage{algorithmic}

% \usepackage{balance}
%
% These are are recommended to typeset listings but not required. See the subsubsection on listing. Remove this block if you don't have listings in your paper.
\usepackage{newfloat}
\usepackage{listings}
\DeclareCaptionStyle{ruled}{labelfont=normalfont,labelsep=colon,strut=off} % DO NOT CHANGE THIS
\lstset{%
	basicstyle={\footnotesize\ttfamily},% footnotesize acceptable for monospace
	numbers=left,numberstyle=\footnotesize,xleftmargin=2em,% show line numbers, remove this entire line if you don't want the numbers.
	aboveskip=0pt,belowskip=0pt,%
	showstringspaces=false,tabsize=2,breaklines=true}
\floatstyle{ruled}
\newfloat{listing}{tb}{lst}{}
\floatname{listing}{Listing}
%
% Keep the \pdfinfo as shown here. There's no need
% for you to add the /Title and /Author tags.
\pdfinfo{
/TemplateVersion (2025.1)
}

% DISALLOWED PACKAGES
% \usepackage{authblk} -- This package is specifically forbidden
% \usepackage{balance} -- This package is specifically forbidden
% \usepackage{color (if used in text)
% \usepackage{CJK} -- This package is specifically forbidden
% \usepackage{float} -- This package is specifically forbidden
% \usepackage{flushend} -- This package is specifically forbidden
% \usepackage{fontenc} -- This package is specifically forbidden
% \usepackage{fullpage} -- This package is specifically forbidden
% \usepackage{geometry} -- This package is specifically forbidden
% \usepackage{grffile} -- This package is specifically forbidden
% \usepackage{hyperref} -- This package is specifically forbidden
% \usepackage{navigator} -- This package is specifically forbidden
% (or any other package that embeds links such as navigator or hyperref)
% \indentfirst} -- This package is specifically forbidden
% \layout} -- This package is specifically forbidden
% \multicol} -- This package is specifically forbidden
% \nameref} -- This package is specifically forbidden
% \usepackage{savetrees} -- This package is specifically forbidden
% \usepackage{setspace} -- This package is specifically forbidden
% \usepackage{stfloats} -- This package is specifically forbidden
% \usepackage{tabu} -- This package is specifically forbidden
% \usepackage{titlesec} -- This package is specifically forbidden
% \usepackage{tocbibind} -- This package is specifically forbidden
% \usepackage{ulem} -- This package is specifically forbidden
% \usepackage{wrapfig} -- This package is specifically forbidden
% DISALLOWED COMMANDS
% \nocopyright -- Your paper will not be published if you use this command
% \addtolength -- This command may not be used
% \balance -- This command may not be used
% \baselinestretch -- Your paper will not be published if you use this command
% \clearpage -- No page breaks of any kind may be used for the final version of your paper
% \columnsep -- This command may not be used
% \newpage -- No page breaks of any kind may be used for the final version of your paper
% \pagebreak -- No page breaks of any kind may be used for the final version of your paperr
% \pagestyle -- This command may not be used
% \tiny -- This is not an acceptable font size.
% \vspace{- -- No negative value may be used in proximity of a caption, figure, table, section, subsection, subsubsection, or reference
% \vskip{- -- No negative value may be used to alter spacing above or below a caption, figure, table, section, subsection, subsubsection, or reference

\setcounter{secnumdepth}{0} %May be changed to 1 or 2 if section numbers are desired.

% The file aaai25.sty is the style file for AAAI Press
% proceedings, working notes, and technical reports.
%

% Title

% Your title must be in mixed case, not sentence case.
% That means all verbs (including short verbs like be, is, using,and go),
% nouns, adverbs, adjectives should be capitalized, including both words in hyphenated terms, while
% articles, conjunctions, and prepositions are lower case unless they
% directly follow a colon or long dash
\title{ContinuouSP: Generative Model for Crystal Structure Prediction 
\\ with Invariance and Continuity}
\author{
    %Authors
    % All authors must be in the same font size and format.
    \textsuperscript{\rm 1} Yuji Tone,
    \textsuperscript{\rm 1} Masatoshi Hanai,
    \textsuperscript{\rm 1} Mitsuaki Kawamura,
    \textsuperscript{\rm 1} Kenjiro Taura,
    \textsuperscript{\rm 1} Toyotaro Suzumura
}
\affiliations{
    %Afiliations
    \textsuperscript{\rm 1}The University of Tokyo,
    % If you have multiple authors and multiple affiliations
    % use superscripts in text and roman font to identify them.
    % For example,
    % Sunil Issar\textsuperscript{\rm 2},
    % J. Scott Penberthy\textsuperscript{\rm 3},
    % George Ferguson\textsuperscript{\rm 4},
    % Hans Guesgen\textsuperscript{\rm 5}
    % Note that the comma should be placed after the superscript
    Tokyo, Japan\\
    % email address must be in roman text type, not monospace or sans serif
    tone@eidos.ic.i.u-tokyo.ac.jp,
    hanai@ds.itc.u-tokyo.ac.jp,
    mitsuaki.kawamura@riken.jp, \\
    % mkawamura@ds.itc.u-tokyo.ac.jp, \\
    tau@eidos.ic.i.u-tokyo.ac.jp,
    suzumura@ds.itc.u-tokyo.ac.jp
%
% See more examples next
}

%Example, Single Author, ->> remove \iffalse,\fi and place them surrounding AAAI title to use it
\iffalse
\title{My Publication Title --- Single Author}
\author {
    Author Name
}
\affiliations{
    Affiliation\\
    Affiliation Line 2\\
    name@example.com
}
\fi

\iffalse
%Example, Multiple Authors, ->> remove \iffalse,\fi and place them surrounding AAAI title to use it
\title{My Publication Title --- Multiple Authors}
\author {
    % Authors
    First Author Name\textsuperscript{\rm 1},
    Second Author Name\textsuperscript{\rm 2},
    Third Author Name\textsuperscript{\rm 1}
}
\affiliations {
    % Affiliations
    \textsuperscript{\rm 1}Affiliation 1\\
    \textsuperscript{\rm 2}Affiliation 2\\
    firstAuthor@affiliation1.com, secondAuthor@affilation2.com, thirdAuthor@affiliation1.com
}
\fi


% REMOVE THIS: bibentry
% This is only needed to show inline citations in the guidelines document. You should not need it and can safely delete it.
\usepackage{bibentry}
% END REMOVE bibentry

\begin{document}

\maketitle

\begin{abstract}
% The discovery of new materials using generative machine learning models has become a significant research topic in recent years. However, even for the relatively simple task of crystal structure prediction (CSP), methods that satisfy the essential requirements of invariance and continuity have not yet been established. In this study, we propose a CSP model called ContinuouSP, which meets these requirements using an Energy-Based Model. We demonstrated the effectiveness of this model using both datasets from previous research and newly prepared datasets.

The discovery of new materials using crystal structure prediction (CSP) based on generative machine learning models has become a significant research topic in recent years. 
In this paper, we study invariance and continuity in the generative machine learning for CSP.
We propose a new model, called ContinuouSP, which effectively handles symmetry and periodicity in crystals.
We clearly formulate the invariance and the continuity, and construct a model based on the energy-based model. 
Our preliminary evaluation demonstrates the effectiveness of this model with the CSP task.
\end{abstract}

% Uncomment the following to link to your code, datasets, an extended version or similar.
%
% \begin{links}
%     \link{Code}{https://aaai.org/example/code}
%     \link{Datasets}{https://aaai.org/example/datasets}
%     \link{Extended version}{https://aaai.org/example/extended-version}
% \end{links}

\section{Introduction}
Recently, there has been a growing trend toward incorporating computers into the discovery of new materials, a process that traditionally relied heavily on human intuition. 
Particularly for solid materials, recent successes using generative machine learning models provide a rational approach from the perspective of ``emulating human intuition''. 
However, theoretically, a solid material consists of an atomic spatial arrangement which is not necessarily finite, making it challenging to model correctly in a generative machine learning framework.

When discussing machine learning models for materials, \emph{invariance} and \emph{continuity} are critical aspects. 
Consider a function with materials as its domain; for this function to hold physical meaning, it must be invariant under operations such as ``10 cm translation along the x-axis'' or ``60° rotation around the y-axis''. 
Additionally, if some atoms in the material are displaced by a small amount, the resulting change should also be small, meaning that the continuity must be preserved. 
In generative models, two functions are considered: (i) the mapping from training data to the training outcome, and (ii) the probability density function determined by this outcome. 
A ``physically correct'' model can only be obtained when all these requirements are satisfied.

In this paper, we focus on crystal structure prediction (CSP), which predicts crystal structures from given atom species vector in one period. 
% Crystals are solid materials that can be described with finite information because of their periodicity, making them computationally manageable.
% For example, to satisfy invariance, one might consider normalizing crystal descriptions using methods such as those defined in \cite{ICUr}. However, while this may satisfy invariance, it is difficult to simultaneously achieve continuity. Indeed, this normalized description changes drastically with small variations in atomic positions. When the description changes drastically, it becomes unavoidable that the learning outcome and probability density also undergo drastic changes. On the other hand, recent diffusion-based CSP models and crystal generation models without given compositions have achieved invariance of probability density without normalization by employing equivariant score predictors incorporating graph neural networks \cite{CDVAE} \cite{DiffCSP} \cite{SyMat}. However, these models still fail to meet continuity.
Recent CSP models have achieved the invariance in their probability density by employing equivariant score predictors that incorporate graph neural networks (GNNs) \cite{CDVAE,DiffCSP,SyMat,DiffCSP++}.
However, it is still difficult to achieve the continuity simultaneously.
% In fact, the description of these methods changes drastically with small variations in atomic positions. 
% When the description changes drastically, it becomes unavoidable that the learning outcome and probability density also undergo drastic changes. 

Therefore, we develop a CSP model called \emph{ContinuouSP}, which simultaneously meets the invariance and the continuity by using an energy-based model (EBM). \footnote{Code: \url{https://github.com/packer-jp/ContinuouSP}}
% EBM is the theoretical background for diffusion models.
We employ a modified version of the crystal graph convolutional neural networks (CGCNN) \cite{CGCNN} as the energy predictor. 
CGCNN is an invariant property prediction model that can also ensure continuity with straightforward modifications.

The main contributions of this paper are as follows:
\begin{itemize}
    \item The concepts of invariance and continuity in solid materials and periodic units of crystals are mathematically formulated.
    \item A CSP model, ContinuouSP, is designed within the EBM framework to satisfy the invariance and the continuity.
    \item Through performance evaluation, it is demonstrated that while ContinuouSP does not achieve state-of-the-art performance, it surpasses traditional machine learning approaches and performs comparably to several existing generative models.
    \item We clarify the advantages and challenges of ContinuouSP through supplementary experiments and theoretical analyses.
\end{itemize}

\section{Related Work}

\subsection{Crystal Property Prediction}
Modern machine learning schemes to predict material properties were first developed to accelerate molecular dynamics simulation~\cite{PhysRevLett.98.146401}.
For this purpose, the radial and angular distribution functions were convoluted with artificial neural networks inside each atom, and then the total energy was output.
The GNN is also employed to construct a universal model including various kinds of elements~\cite{CGCNN,SchNet,MEGNet,PotNet}.
This GNN model can handle periodicity and rotation, permutation, and translation invariance and be extended to the hypergraph to treat the bond angle directly~\cite{ALIGNN}.
Recently, a more accurate graph-transformer-based model~\cite{Matformer} and infinitely fully connected neural network for crystal systems~\cite{Crystalformer} have been proposed.
These models can be assumed as an extension of the GNN for crystals by using the attention mechanism.
This study uses these GNN models to describe the logarithmic probability because that function is expected to behave as the total energy, and we can easily modify that model to guarantee the continuity of the lattice deformation.

\subsection{CSP and Crystal Generation}
CSP and crystal generation using machine learning have traditionally developed under the predict-optimize paradigm \cite{Predict_Optimize}, based on formation energy predictors and optimization methods. On the other hand, in recent years, significant research efforts have been dedicated to exploring the potential of generative models. FTCP \cite{FTCP}, proposed alongside the implementation of a crystal generative model using VAE, serves as a reversible feature representation for crystals. Among crystal generative models utilizing GANs is CrystalGAN \cite{CrystalGAN}, which targets crystals with specific compositions and enables efficient crystal generation by partitioning the search space. Furthermore, inspired by the success of diffusion models in the field of computer vision, numerous studies have reported advances using diffusion models. One advantage of diffusion models lies in their ability to guarantee the invariance of probability density functions by adopting equivariant score predictors for translations, rotations, and permutations of atomic orders. CDVAE \cite{CDVAE}, a pioneering work in this approach, explicitly handles atomic coordinates. DiffCSP \cite{DiffCSP} employs fractional coordinates and Fourier-transformed features, while SyMat \cite{SyMat} utilizes interatomic distances. EquiCSP \cite{EquiCSP} explores the invariance concerning the permutation of lattice vector orders. DiffCSP++ \cite{DiffCSP++} further extends the DiffCSP method by incorporating space group considerations. However, we believe that, although these approaches partially ensure certain invariances, they still fall short of fully satisfying all the necessary properties that should be met.

\subsection{Continuity of Machine Learning Models}
In the field of computer vision, the continuity of machine learning models has been a subject of discussion. Specifically, it has been pointed out that using classical representations such as quaternions or Euler angles for 3D rotations in point clouds or joints results in a lack of the continuity \cite{Continuity_Rotation}. In general, for 3D rotations, representations with 4 or less dimensions are insufficient from the perspective of the continuity, and it has been shown that representations of 5 or higher dimensions should be used, along with practical representation example. Additionally, self-selecting ensembles \cite{Self_Selecting_Ensambles} provide an approach to address the topological complexity arising from the rotational symmetry of the target.

\subsection{Energy-Based Models}
Energy-based models (EBMs), which trace their origins back to the Boltzmann machine \cite{Boltzman_Machine}, continue to be actively explored across various domains. For example, \cite{EBM_OpenAI} demonstrates performance comparable to then state-of-the-art methods in tasks such as image generation and corrupted data restoration. Also, Generative PointNet \cite{Generative_PointNet} leverages EBMs for point cloud generation, achieving high performance while ensuring invariance. In the realm of physics applications, protein conformation prediction stands out as a prominent example \cite{EBM_Meta}. Moreover, EBMs form the theoretical foundation of diffusion models, and their indirect impact in this area is immeasurable.

\section{Preliminaries}

\subsection{Solid Materials}
A \emph{solid material} refers to an atomic arrangement in 3D space. Its mathematical definition is as follows:
\begin{definition}[Solid Materials]
Let $\mathbb{A}$ be the set of atomic species consisting of H, He, Li, and so on. The set of all solid materials, $\mathcal{S}$, is the set of all countable subsets of $\mathbb{A} \times \mathbb{R}^{3 \times 1}$. Namely, $(a, x) \in S$ ($\in \mathcal{S}$) represents a pair of an atomic species and a coordinate in the solid material $S$.
\end{definition}

Next, we define each kind of \emph{invariance} in solid materials. Physical properties such as the formation energy, the band gap, and the bulk modulus, which take scalar values, are examples of invariance under translation or rotation. Below, $\mathcal{Y}$ represents an arbitrary set, and elements of groups are identified with their standard representations.
\begin{definition}[Translation Invariance on Solid Materials]
A partial mapping $f: \mathcal{S} \rightharpoonup \mathcal{Y}$ is said to be translation invariant if, for any solid material $S \in \mathrm{Dom}(f)$ and any translation operation $b \in \mathrm{T}(3)$, the equation $f(\{(a, x + b) \mid (a, x) \in S\}) = f(S)$ holds.
\end{definition}
\begin{definition}[Rotation Invariance on Solid Materials]
A partial mapping $f: \mathcal{S} \rightharpoonup \mathcal{Y}$ is said to be rotation invariant if, for any solid material $S \in \mathrm{Dom}(f)$ and any rotation operation $Q \in \mathrm{O}(3)$, the equation $f(\{(a, Qx) \mid (a, x) \in S \}) = f(S)$ holds.
\end{definition}

Finally, we define the \emph{continuity} of solid materials. Continuity here means that when each atom is displaced by a small amount, the changes in the values corresponding to the solid material, such as physical properties, are also small.
\begin{definition}[Continuity on Solid Materials]
A partial mapping $f: \mathcal{S} \rightharpoonup \mathbb{R}^m$ is said to be continuous at a solid material $S \in \mathcal{S}$ if, for any $\varepsilon > 0$, there exists a $\delta > 0$ such that the following conditions hold for any solid material $S' \in \mathrm{Dom}(f)$. 
% If there exists a bijection $g: S \to S'$ such that for any atom $(a, x) \in S$, letting $(a', x') = g(a, x)$, we have $a' = a$ and $\|x' - x\| < \delta$, then $\|f(S') - f(S)\| < \varepsilon$. 
There exists a bijection $\phi: S \to S'$, $\phi(a, x) = (a', x')$, and if $a' = a$ and $\|x' - x\| < \delta$, then $\|f(S') - f(S)\| < \varepsilon$. In what follows, when we simply say that $f$ is continuous, it means it is continuous for all solid materials in its domain.
\end{definition}

\subsection {Crystals}
Let us formulate the \emph{periodic unit} commonly used to describe crystals. In general, a crystal can be represented by (i) a \emph{species vector} corresponding to the species of atoms in one period, (ii) a \emph{coordinate matrix} corresponding to the coordinates of atoms in one period, and (iii) a \emph{lattice basis} corresponding to the shape of the unit cell.

\begin{definition}[Periodic Units]
The set of all periodic units with $n$ atoms per period is denoted as $\mathcal{P}_n$. This consists of $(A, (x_1, \ldots, x_n), L) \in \mathbb{A}^n \times \mathbb{R}^{3 \times n} \times \mathbb{R}^{3 \times 3}$ such that for any $i, j \in \{1, \ldots, n\}$ and $k \in \mathbb{Z}^{3 \times 1}$ with $i \neq j$ or $k \neq 0$, $x_j + Lk \neq x_i$.
Namely, in a periodic unit $(A, X, L) \in \mathcal{P}_n$, $A$ represents the species vector, $X$ represents the coordinate matrix, and $L$ represents the lattice basis.
\end{definition}

A periodic unit can be converted into a solid material as follows:

\begin{definition}[Periodic Units to Solid Materials]
The conversion from a periodic unit to a solid material, $\mathrm{PtoS}: \bigcup_{n=1}^\infty \mathcal{P}_n \to \mathcal{S}$ is represented as follows:
\begin{multline}
\mathrm{PtoS}((a_1, \ldots, a_n), (x_1, \ldots, x_n), L) = \\ \left\{
(a_i, x_i + Lk)
\mid
i \in \{1, \ldots, n\}, k \in \mathbb{Z}^{3 \times 1}
\right\}.
\end{multline}
Through this conversion $\mathrm{PtoS}$, the entire set of crystals can be expressed as the image $\mathrm{PtoS}(\bigcup_{n=1}^\infty \mathcal{P}_n)$. Namely, a crystal is a solid material where atoms are periodically arranged according to lattice basis.
\end{definition}

It should be noted that $\mathrm{PtoS}$ is not injective. In other words, a periodic unit can be re-described without changing the crystal it indicates. The replacement of lattice basis, encompassing both cases where the number of atoms per period changes and does not change, as well as the rearrangement of atomic order, corresponds to a re-description. To address this, we introduce the concept of \emph{re-description invariance}. For example, a function that takes a periodic unit as input and calculates a physical property of the crystal it indicates should reproduce the same output even if the input periodic unit is re-described. Only when this is guaranteed, the function can be interpreted as a mapping from a crystal to its physical property.

\begin{definition}[Strong Re-description Invariance]
A mapping $g: \bigcup_{n=1}^\infty \mathcal{P}_n \to \mathcal{Y}$ is said to be strongly re-description invariant if, for any periodic units $P, P' \in \bigcup_{n=1}^\infty \allowbreak \mathcal{P}_n$, $\mathrm{PtoS}(P') = \mathrm{PtoS}(P)$ implies $g(P') = g(P)$. In this case, there exists a unique mapping $f: \mathrm{PtoS}(\bigcup_{n=1}^\infty \mathcal{P}_n) \to \mathcal{Y}$ such that $f \circ \mathrm{PtoS} = g$. When $g$ is strongly re-description invariant, the notation $g \circ \mathrm{PtoS}^{-1}$ is expediently allowed and is considered equal to $f$.
\end{definition}

Additionally, a version with a fixed number of atoms per period, which is a weaker form of the above definition, is also defined.

\begin{definition}[Weak Re-description Invariance]
A mapping $g: \bigcup_{n=1}^\infty \mathcal{P}_n \to \mathcal{Y}$ is said to be $n$-weakly re-description invariant if, for any periodic units with $n$ atoms $P, P' \in \mathcal{P}_n$, $\mathrm{PtoS}(P') = \mathrm{PtoS}(P)$ implies $g(P') = g(P)$. In this case, there exists a unique mapping $f: \mathrm{PtoS}(\mathcal{P}_n) \to \mathcal{Y}$ such that $f \circ \mathrm{PtoS}|_{\mathcal{P}_n} = g|_{\mathcal{P}_n}$. When $g$ is $n$-weakly re-description invariant, the notation $g \circ \mathrm{PtoS}|_{\mathcal{P}_n}^{-1}$ is expediently allowed and is considered equal to $f$.

\end{definition}
\subsection {Crystal Graph Convolutional Neural Networks}
The \emph{crystal graph convolutional neural networks (CGCNN)} \cite{CGCNN} is a machine learning model designed to predict physical properties of crystals. By converting periodic units into graphs, it achieves any kind of invariance.

In the process of graph construction, atoms within one periodic unit of the crystal are interpreted as nodes, and edges are drawn between nodes corresponding to atomic pairs within a cutoff distance. Each edge is associated with a feature determined by the interatomic distance. Notably, this can be multi-edges.

\begin{definition}[Graph Construction in CGCNN]
For a graph including $n$ nodes corresponding to a periodic unit $P = (A, (x_1, \ldots, x_n), L) \in \mathcal{P}_n$, 
the neighborhood $\mathrm{N}_P(i)$ of a node $i \in \{1, \ldots, n\}$ is defined as follows:
\begin{equation}
\mathrm{N}_P(i) 
= \left\{(j, k) \;\middle|\;  
\begin{aligned}
&j \in \{1, \ldots, n\}, \\
&k \in \mathbb{Z}^{3 \times 1}, \\
&\|x_j + Lk - x_i\| < D
\end{aligned}
\right\},
\end{equation}
where $D$ is the cutoff distance.
When $(j, k) \in \mathrm{N}_P(i)$, an edge is drawn between nodes $i$ and $j$, carrying a feature $e_{i, j, k} \in \mathbb{R}^{d_\mathrm{e} \times 1}$ determined by the distance 
$\|x_j + Lk - x_i\|$. The resulting graph is undirected.
\end{definition}

Next, the node features are updated through graph convolution.

\begin{definition}[Graph Convolution in CGCNN]
The feature $v_i \in \mathbb{R}^{d_\mathrm{v} \times 1}$ of a node $i$, initialized based on the atomic species, is updated as follows:
\begin{equation}
v_i \leftarrow v_i + \sum_{(j, k) \in \mathrm{N}_P(i)} \psi_\theta(v_i, v_j, e_{i, j, k}).
\end{equation}
Here, $\theta$ represents the learning parameters, and $\psi_\theta$ is a continuous mapping for each column vector of the input.
\end{definition}

The resulting node features are then aggregated using mean pooling, and the output is obtained via a multi-layer perceptron (MLP). This process constitutes the CGCNN.

\begin{definition}[CGCNN]
$\mathrm{CGCNN}_\theta: \bigcup_{n=1}^\infty \mathcal{P}_n \to \mathbb{R}$ is defined by the following steps:
\begin{enumerate}
 \item Convert the input crystal into a graph.
 \item Update node features through graph convolution.
 \item Aggregate the graph features using mean pooling.
 \item Pass the aggregated features through an MLP to obtain the output.
\end{enumerate}
\end{definition}

CGCNN defined as above satisfies all the desired invariances, although it should be noted that this does not satisfy the continuity as it stands.
\begin{theorem}[Invariance of CGCNN]
\label{thm:invariance_of_CGCNN}
$\mathrm{CGCNN}_\theta$ is strongly re-description invariant. Also, $\mathrm{CGCNN}_\theta \circ \mathrm{PtoS}^{-1}$ is invariant to translation and rotation.
\end{theorem}

\subsection {Energy-Based Models}

The \emph{energy-based model (EBM)} is one of the generative models designed to learn the underlying probability distribution of a given dataset. Unlike other generative models such as VAEs or GANs, the NN in an EBM does not directly output samples. Instead, it outputs a scalar value called \emph{energy} associated with the input, which is tied to the probability density.

\begin{definition}[Probability Density Function in EBMs]
The goal is to learn a probability distribution over a set $\mathcal{X}$. For a point $x \in \mathcal{X}$, the parameterized probability density is denoted as $p_\theta(x)$. A relationship is established between $p_\theta(x)$ and an NN $H_\theta: \mathcal{X} \to \mathbb{R}$ as follows:
\begin{gather}
p_\theta(x) = \frac{\exp(-\beta H_\theta(x))}{Z(\theta, \beta)}, \\
\text{where } Z(\theta, \beta) = \int_{x \in \mathcal{X}} \exp(-\beta H_\theta(x)) \mathrm{d}x.
\end{gather}
Here, $\beta > 0$ is the inverse temperature. By analogy with statistical mechanics, $H_\theta$ is referred to as the \emph{energy function}.
\end{definition}

A sampling method using the energy function is Markov-chain Monte Carlo (MCMC). In MCMC, a point is initially chosen at random and then iteratively updated. A fundamental MCMC algorithm is the Metropolis-Hastings (MH) algorithm, which proposes transitions to neighboring states and decides whether to accept these transitions based on changes in energy.

\begin{definition}[MH]
In MH, by denoting the transition probability from a point $x \in \mathcal{X}$ to another point $x' \in \mathcal{X}$ as $r(x' \mid x)$, the acceptance probability is defined as follows:
\begin{equation}
\min\left\{1, \frac{r(x \mid x') p_\theta(x')}{r(x' \mid x) p_\theta(x)}\right\}.
\end{equation}
\end{definition}

This ensures that the detailed balance condition is satisfied, and after sufficient iterations, the samples follow the probability distribution $p_\theta$.

If $H_\theta$ is differentiable with respect to the input, another MCMC method, Langevin Monte Carlo (LMC) algorithm, can be applied.

\begin{definition}[LMC]
In LMC, a point $x \in \mathcal{X} = \mathbb{R}^m$ is updated as follows:
\begin{equation}
x \leftarrow x - \alpha \beta \nabla_x H_\theta(x) + \sqrt{2\alpha} u,
\text{where } u \sim \mathcal{N}(0, I)
\end{equation}
If the step size $\alpha$ is sufficiently small, no acceptance decision like in MH is required. An LMC variant that includes acceptance decisions is called the Metropolis-adjusted Langevin algorithm (MALA).
\end{definition}

The EBM is typically trained via maximum likelihood estimation.

\begin{definition}[Loss Function in EBMs]
In an EBM, given a dataset $(x_1, \ldots, x_n) \in \mathcal{X}^n$, learning minimizes the following loss $J(\theta)$:
\begin{equation}
J(\theta) = -\sum_{i=1}^n \log p_\theta(x_i).
\end{equation}
\end{definition}

Although directly computing the loss function is difficult, its gradient can be calculated.

\begin{theorem}[Gradient of Loss Function in EBMs]
\label{thm:gradient_of_loss_function_in_EBMs}
The gradient of the loss $J(\theta)$ is given by:
\begin{equation}
\nabla_\theta J(\theta) 
= \frac{\beta}{n} \sum_{i=1}^n \left( \nabla_\theta H_\theta(x_i) - \mathbb{E}_{p_\theta(x)} \left[ \nabla_\theta H_\theta(x) \right] \right).
\end{equation}
Here, the subscript $p_\theta(x)$ of the expectation operator $\mathbb{E}$ represents sampling $x$ from the probability distribution corresponding to $p_\theta$.
\end{theorem}

In essence, learning progresses to minimize the energy for data points while maximizing the energy for sample points.

\section{Proposed Method}

\begin{figure*}[tb]
\centering
\includegraphics[width=0.95\textwidth]{training} % Reduce the figure size so that it is slightly narrower than the column.
\caption{Diagram illustrating the training workflow of ContinuouSP: For each periodic unit of the crystal included in the training data points, the species vector is $\mathrm{Reduce}$-d to the composition, then $\mathrm{Expand}$-ed using a geometric distribution to create a new species vector. This vector is used to condition the sampling performed via MCMC. The resulting new crystal and the original crystal are compared by calculating the energy difference, which is used as a pseudo-loss.}
\label{fig:training}
\end{figure*}

We propose a CSP model, \emph{ContinuouSP}, that is both invariant and continuous. By leveraging the EBM framework, we can retain the invariance properties guaranteed by CGCNN.

First, we modify CGCNN slightly to ensure the continuity.

\begin{definition}[Graph Convolution in CGCNN']
$\mathrm{CGCNN}_\theta'$ is defined as a modified version of the update formula of $\mathrm{CGCNN}_\theta$, as follows:
\begin{multline}
v_i \leftarrow v_i + \\ \sum_{(j, k) \in \mathrm{N}_P(i)} \cos^2\Bigg(\frac{\pi\|x_j + Lk - x_i\|}{2D}\Bigg) \psi_\theta(v_i, v_j, e_{i, j, k}).
\end{multline}
Here, $\psi_\theta$ is a continuous mapping for each column vector of its inputs.
\end{definition}

Next, we use this directly as the energy function.

\begin{definition}[Energy Function in ContinuouSP]
The energy function $H_\theta: \bigcup_{n=1}^\infty \mathcal{P}_n \to \mathbb{R}$ is defined as $H_\theta = \mathrm{CGCNN}_\theta'$
\end{definition}

For the CSP task, which is a conditional generation problem based on the species vector, we define the relationship between the energy and the probability density.

\begin{definition}[Probability Density Function in ContinuouSP]
For a periodic unit $(A, X, L) \in \mathcal{P}_n$, the conditional probability density $p_\theta(X, L \mid A)$ is expressed using the energy function $H_\theta$ as:
\begin{gather}
\label{eq:probability_distribution_function_in_ContinuouSP}
p_\theta(X, L \mid A) = \frac{\exp(-\beta H_\theta(A, X, L))}{Z(\theta, \beta, A)}, \\
\begin{multlined}
\label{eq:partition_function_in_ContinuouSP}
\text{where } Z(\theta, \beta, A) = \\ \int_{\substack{X \in \mathbb{R}^{3 \times n} \\ L \in \mathbb{R}^{3 \times 3}}} \exp(-\beta H_\theta(A, X, L)) \mathrm{d}X \mathrm{d}L. 
\end{multlined}
\end{gather}
\end{definition}

The energy function $H_\theta$ satisfies the following properties:

\begin{theorem}[Properties of Energy Function in ContinuouS\-P]
\label{thm:properties_of_energy_function_in_ContinuouSP}
$H_\theta$ is strongly re-description invariant. Also, $H_\theta \circ \mathrm{PtoS}^{-1}$ is invariant to translation and rotation, and is continuous.
\end{theorem}

\begin{theorem}[Properties of Probability Density Function in ContinuouSP]
\label{thm:properties_of_probability_density_function_in_ContinuouSP}
For any positive integer $n$, $p_\theta$ is $n$-weakly re-description invariant. Also, $p_\theta \circ \mathrm{PtoS}|_{\mathcal{P}_n}^{-1}$ is invariant to translation and rotation, and is continuous.
\end{theorem}

In the standard EBM, the parameters $\theta$ are determined via maximum likelihood estimation over the dataset. However, in the CSP task, the generation condition, species vector, is not inherently unique to the crystal (even ignoring permutation symmetry). Thus, directly using it in the loss function is inappropriate. In ContinuouSP, the loss function is defined prescriptively as follows:

\begin{definition}[Loss Function in ContinuouSP]
The loss function $J$ in ContinuouSP on the dataset $((A_1, X_1, L_1), \allowbreak \ldots, (A_m, X_m, L_m)) \in \left(\bigcup_{n=1}^\infty \mathcal{P}_n\right)^m$ is defined as:
\begin{multline}
J(\theta) = \frac{1}{m} \sum_{i=1}^m \Bigg(\beta H_\theta(A_i, X_i, L_i) + \\ \sum_{j=1}^\infty (1-q)^{j-1} q \log Z(\theta, \beta, A_i')\Bigg),
\end{multline}
where: 
\begin{itemize}
    \item $(1-q)^{j-1}q$ is the probability mass of geometric distribution with success probability $q$ and $j$ trials.
    \item $A_i'=\mathrm{Expand}(\mathrm{Reduce}(A_i), j)$.
    \item $\mathrm{Reduce}$ is a function that constructs the simplest integer ratio of atomic counts (so-called composition) from the species vector.
    \item $\mathrm{Expand}$ is a function that constructs the species vector by multiplying the composition by an integer.
\end{itemize}
Due to the re-description invariance of $H_\theta$, the computation outcome does not depend on the order of atoms in $A_i'$. Therefore, this loss function is well-defined.
\end{definition}

Then, the gradient of the loss function can be expressed in a form that corresponds well to that of a general EBMs.
\begin{theorem}[Gradient of Loss Function in ContinuouSP]
\label{thm:gradient_of_loss_function_in_ContinuouSP}
The gradient of the loss $J(\theta)$ is given by:
\begin{multline}
\nabla_\theta J(\theta) = \frac{\beta}{m} \sum_{i=1}^m \Bigg( \nabla_\theta H_\theta(A_i, X_i, L_i) - \\ \mathbb{E}_{\substack{j \sim \mathrm{Geom}(q) \\ p_\theta(X, L \mid A_i')}}\left[ \nabla_\theta H_\theta(A_i', X, L) \right] \Bigg), 
\end{multline}
where $\mathrm{Geom}(q)$ is geometric distribution with success probability $q$.
\end{theorem}

Thus, the pseudo-loss is computed through the flow illustrated in Fig. \ref{fig:training}, guiding the training process.

\begin{definition}[Loss Function for Single Datum]
For the mapping \(\mathcal{L}: \bigcup_{n=1}^\infty \mathcal{P}_n \to \mathbb{R}\) from a single datum to the loss, the following holds:
\begin{multline}
\mathcal{L}(A, X, L) = \frac{1}{m}\Bigg(\beta H_\theta(A, X, L) + \\ \sum_{j=1}^\infty (1-q)^{j-1} q \log Z(\theta, \beta, A')\Bigg) + R(\theta),
\end{multline}
where $R(\theta)$ accounts for the contributions of other data points.
\end{definition}

The following properties hold for $\mathcal{L}$:

\begin{theorem}[Properties of Loss Function in ContinuouSP]
\label{thm:properties_of_loss_function_in_ContinuouSP}
The mapping $\mathcal{L}$ is strongly re-description invariant. Also, the mapping $\mathcal{L} \circ \mathrm{PtoS}^{-1}$ is invariant to translation and rotation, and is continuous.
\end{theorem}

If the mapping from a datum to the loss is invariant and continuous, it is reasonable to expect that the training results will also be invariant and continuous with respect to the datum. Therefore, we conclude that in ContinuouSP, both the training results regarding the training data and the probability density functions determined by the training results satisfy all requirements.

\section{Experiment}

\subsection{Experimental Setup}
\subsubsection{Crystal Structure Prediction}

We evaluate the performance of ContinuouSP on the CSP task using datasets that were utilized in previous researches such as \cite{CDVAE} and \cite{DiffCSP}. After training as usual, we generate species vectors for each crystal in the test dataset, use them as input conditions for the model, and compare the actual structures with the sampled structures.

\paragraph{Datasets}
We evaluate our methods on three datasets with varying complexity. \emph{Perov-5} \cite{Perov_5_0}\cite{Perov_5_1} consists of 18,928 crystals with perovskite structures. Each crystal has a cubic unit cell containing 5 atoms, characterized by similar structural configurations. This dataset is relatively simple. \emph{MP-20} comprises 45,231 inorganic crystals and is considered a standard dataset. Each crystal contains up to 20 atoms per unit cell. \emph{MPTS-52} is a more complex extension of MP-20, consisting of 40,476 inorganic crystals. Each crystal also contains up to 20 atoms per unit cell. Perov-5 and MP-20 datasets are split into 60-20-20 ratios for training, validation, and testing. For MPTS-52, the dataset is split into 27,380/5,000/8,096 entries based on the order of the earliest published year.

\paragraph{Baselines}
We compare our approach with several baseline methods to evaluate its effectiveness. \emph{RS}, \emph{BO}, and \emph{PSO} \cite{Predict_Optimize} represent traditional methods based on the predict-optimize paradigm. Specifically, a formation energy predictor is first constructed using MEGNet \cite{MEGNet}, followed by the discovery of structures that minimize formation energy through various optimization methods. Each of these methods employs over 5,000 iteration steps. \emph{P-cG-SchNet} \cite{DiffCSP} is an autoregressive generative model adapted for CSP by incorporating lattice basis predictions into SchNet \cite{SchNet}, which was originally designed for molecular tasks. \emph{CDVAE} \cite{CDVAE} is a crystal generative model based on diffusion and can be applied to CSP. \emph{DiffCSP} \cite{DiffCSP} is another diffusion-based model explicitly designed for CSP.

\paragraph{Metrics}
For evaluation, 20 samples are generated for each crystal in the test dataset, and each kind of metric value is computed by using \texttt{StructureMatcher(stol=0.5, angle\_tol=10, ltol=0.3)} from pymatgen library. \emph{Match Rate} is the percentage of test crystals for which at least one sampled structure matches the actual structure. A match is determined if the \texttt{get\_rms\_dist} method of \texttt{StructureMatcher} returns a value other than \texttt{None}. \emph{RMSE} is the mean of the smallest atomic RMS displacements among matched structures for each test crystal. The \texttt{float} values returned by \texttt{get\_rms\_dist} method are used for this calculation.

\paragraph{Implementation Details}
In the $\psi_\theta$ of CGCNN, the sum of the outputs of linear layers applied to $v_i$ and $v_j$, and the Hadamard product of the output of a linear layer applied to $e_{i, j, k}$, are prepared twice in the same way. One of them is passed through a sigmoid function, and the two are multiplied. After each graph convolutional layer, the vertex features are passed through a softplus activation function. The edge features are computed using Gaussian smearing with a maximum value of 20 Å and a standard deviation coefficient of 3.0. The cutoff distance $D$ for graph construction is set to 3.0 times the cube root of the reciprocal of the atomic density. For Perov-5, both the vertex feature dimension $d_\mathrm{v}$ and the edge feature dimension $d_\mathrm{e}$ are set to 32, with 3 graph convolutional layers and a 2-layer MLP with the softplus activation function. On the other hand, for MP-20 and MPTS-52, both $d_\mathrm{v}$ and $d_\mathrm{e}$ are set to 64, with 6 graph convolutional layers and a 4-layer MLP also with the softplus activation function. To prevent excessively low or high atomic densities, a penalty is added to the energy output by the neural network, which is the logarithm of the atomic density divided by the reference atomic density of 0.05 $\text{\AA}^{-3}$. The geometric distribution parameter $q$ in the loss function is set to 0.5, and sampling is performed using 1000 iterations of MALA. During the iterations, the inverse temperature $\beta$ changes exponentially from 1 to 1000, and the step size $\alpha$ changes exponentially from 0.5 to 0.0005. The lattice basis is initialized with a normal distribution such that the expected atomic density is 0.05 $\text{\AA}^{-3}$, and the atomic coordinates are initialized with a uniform distribution in fractional coordinates. At each sampling step, the lattice basis is reduced using Niggli reduction. The training optimization algorithm uses Adam with a learning rate of 0.001, $\beta_1 = 0.9$, and $\beta_2 = 0.999$. Training is performed for 5 epochs on Perov-5 and 1 epoch on MP-20 and MPTS-52, using a single NVIDIA A100 GPU.
 
\subsubsection{Displacement vs. Energy}
As a supplementary experiment to verify the validity of the model trained for the CSP task, the change in energy with respect to atomic displacement from stable structures is examined. Specifically, for crystals with stable structures included in the test dataset, Gaussian noise with a standard deviation of 0.1 Å is added to the coordinates of a single atom within the periodic unit, and the resulting structure is passed through the energy predictor. By repeating this process a sufficient number of times, the relation between the displacement and the energy can be obtained. For this experiment, five crystals are randomly selected from each of Perov-5, MP-20, and MPTS-52. From the periodic unit of each selected crystal, one atom is randomly chosen. Then, 1000 pairs of displacement magnitudes and energy values are obtained.

\subsection{Experimental Result}
\subsubsection{Crystal Structure Prediction}

\begin{table*}[t]
\centering
\caption{The results of performance evaluation on the CSP task: Metrics such as Match Rate and RMSE are reported for datasets like Perov-5, MP-20, and MPTS-50, alongside results from existing methods based on the predict-optimize paradigm and generative models. ContinuouSP generally outperforms methods based on the predict-optimize paradigm and demonstrates performance comparable to several existing generative models.}
\label{tab:result_csp}
\setlength{\tabcolsep}{10pt}
\renewcommand{\arraystretch}{1.5}
\begin{tabular}{c c c c c c c}
\toprule
 & \multicolumn{2}{c}{Perov-5} & \multicolumn{2}{c}{MP-20} & \multicolumn{2}{c}{MPTS-52} \\
 & Match rate$\uparrow$ & RMSE$\downarrow$ & Match rate$\uparrow$ & RMSE$\downarrow$ & Match rate$\uparrow$ & RMSE$\downarrow$ \\
\midrule \midrule
RS & 29.22 & 0.2924 & 8.73 & 0.2501 & 2.05 & 0.3329 \\
BO & 21.03 & 0.2830 & 8.11 & 0.2816 & 2.05 & 0.3024 \\
PSO & 20.90 & 0.0836 & 4.05 & 0.1670 & 1.06 & 0.2339 \\
\midrule \midrule
P-cG-SchNet & 97.94 & 0.3463 & 32.64 & 0.3018 & 12.96 & 0.3942 \\
CDVAE & 88.51 & 0.0464 & 66.95 & 0.1026 & 20.79 & 0.2085 \\
DiffCSP & 98.60 & \textbf{0.0128} & \textbf{77.93} & \textbf{0.0492} & \textbf{34.02} & \textbf{0.1749} \\
\midrule \midrule
ContinuouSP & \textbf{99.16} & 0.0893 & 44.21 & 0.1757 & 11.77 & 0.2647 \\
\bottomrule
\end{tabular}
\end{table*}

The experimental results are presented in Table \ref{tab:result_csp}. Overall, while the ContinuouSP does not surpass DiffCSP, generally outperforms the methods based on the predict-optimize paradigm, and exhibits comparable performance to several existing generative models. Notably, the Match Rate on the Perov-5 dataset achieves the highest value among all methods. It is unfortunate that the relative performance decreases as the complexity of the datasets increases. However, this is primarily attributed to the current inability to identify optimal hyperparameter settings for these cases, rather than any fundamental issues in the core methodology causing such tendencies. Regarding training time, it remained within 10 hours for all datasets. This is neither significantly longer nor shorter compared to prior models. Nevertheless, the drawback of EBMs --- the long time required for a single epoch — has become apparent. Addressing this issue, such as by employing parallelizable sampling methods, will be a focus for future improvements.

\subsubsection{Displacement vs. Energy}

\begin{figure*}[tb]
\centering
\includegraphics[width=0.95\textwidth]{displacement_vs_energy} % Reduce the figure size so that it is slightly narrower than the column.
\caption{The results of the displacement vs. energy experiment: For crystals with stable structures, an atom is selected from the periodic unit, and Gaussian noise is added to its coordinates. The diagram shows the relationship between the displacement and the energy output by the model. From top to bottom, the results correspond to crystals picked from the Perov-5, MP-20, and MPTS-52 datasets, respectively. Note that the format of IDs representing crystals varies across datasets. The relationship between displacement and energy is divided into two types: those where energy monotonically increases with displacement and those where both increases and decreases are observed.}
\label{fig:displacement_vs_energy}
\end{figure*}

We show the experimental result in Fig. \ref{fig:displacement_vs_energy}. Across all datasets, the behavior can be categorized into two patterns: one where the energy increases monotonically with displacement from the stable position, and another where both increases and decreases in energy are observed. For the cases of monotonic energy increase, the slope also increases from horizontal as displacement grows, which aligns well with physical behavior, such as the harmonic force field approximation. Notably, these phenomena are reproduced despite the fact that no explicit constraints, such as those used in loss functions, are incorporated into the model, as is the case like \cite{CDVAE}. There are also instances where a range of energy values is observed for the same displacement magnitude, which can be interpreted as the model capturing the anisotropy of energy changes. On the other hand, cases where energy decreases with displacement indicate instances where the model fails to correctly identify the energy of the stable structure as a true local minimum. Thus, while our model can automatically emulate correct physical behavior in some cases, it also faces challenges in consistently handling all scenarios.

\section{Discussion}
\subsection{Relation with Data Augmentation}
One method to address arbitrariness in a training dataset is data augmentation. For instance, this involves applying random rotations to image data before feeding it into the model. So far, we have discussed the perspective of ensuring various invariances in crystals at the model level. However, ContinuouSP actually incorporates elements of data augmentation. Specifically, the process of doubling the number of atoms per period through $\mathrm{Reduce}$ and $\mathrm{Expand}$ serves this purpose, absorbing the arbitrariness related to the size of the period. While addressing other forms of arbitrariness through data augmentation could have been a viable approach, ensuring the invariances at the model level is preferable. This is because data augmentation struggles to handle the replacement of lattice basis that does not change the number of atoms per period. Unlike rotation or translational symmetry in periodicity, lattice basis replacement is unbounded and cannot be sampled correctly due to its symmetry and the inherent arbitrariness unrelated to the number of atoms per period. Taking these factors into consideration, we aimed to maximize the invariances at the model level by utilizing an EBM with CGCNN as its backbone.

\subsection{Comparison with Other Methods}
Here, we compare ContinuouSP with other methods from a theoretical perspective.

First, CSP using EBMs is similar to the predict-optimize paradigm. This is because the neural network in an EBM functions as an energy predictor, and the behavior of MCMC closely resembles optimization---particularly at low temperatures, where it can essentially be regarded as optimization. However, the reason we adopted EBM, despite the similarities, lies in the inherent challenges of the predict-optimize paradigm. Specifically, the paradigm's reliance on constructing an energy predictor from actual formation energy data makes it less robust to data biases. The pairs of crystal structures and formation energies in the training data must conform to the probability distribution derived from those energies. If the training data consists only of stable crystals, for instance, the energy predictor might overestimate the energy of unstable structures during optimization. Preparing a suitable dataset that satisfies this requirement is not straightforward. In contrast, EBMs or general generative models learn the probability distribution based solely on the presence or absence of data. This means that a dataset consisting only of stable crystals would suffice. Nonetheless, constructing formation energy predictors or other property predictors and utilizing them for transfer learning in EBMs or similar models could be promising for performance improvement.

The relationship with diffusion models is also crucial. Diffusion models can be seen as an evolution of EBMs, and their efficiency arises from not requiring MCMC sampling during training. Instead, they explicitly assume a probability distribution from the training data and train a score predictor to learn its gradient. We attribute the success of diffusion models in the computer vision domain partly to the validity of assuming simple distributions, such as mixtures of Gaussians, for objects in Euclidean space like images. In contrast, the topological space of crystals is more complex due to periodicity. This complexity is exemplified by the fact that even for the same crystal, the degrees of freedom can vary depending on the method of description per period. From these observations, we consider the use of EBMs instead of diffusion models in CSP to be reasonably justified. However, from the perspective of training efficiency, diffusion models have a clear advantage, as reflected in the performance evaluation results of this study. Improving the training efficiency of EBMs remains an important challenge for future work.

\section{Conclusion}
In this study, we proposed the CSP model ContinuouSP. We began by formalizing key concepts, including invariance to translation and rotation and continuity, both at the solid material level, as well as the conversion from periodic units to solid materials and the associated invariance to re-description. Using the property prediction model CGCNN and the EBM framework, we successfully fulfilled these requirements. In performance evaluation, while the model did not achieve state-of-the-art on ordinary datasets, it outperformed traditional methods and demonstrated comparable performance to several existing generative models. Moreover, a unique feature of ContinuouSP was its ability to maintain performance on extraordinary datasets, a characteristic not observed in existing models. Additionally, we discussed the relationship with data augmentation, principled comparisons with existing methods such as the predict-optimize paradigm and diffusion models, and the theoretical underpinnings of these approaches. Through this work, we reaffirm the potential of EBM in CSP tasks. Future work will focus on improving performance to further solidify its advantages.

% \section{Acknowledgments}
% AAAI is especially grateful to Peter Patel Schneider for his work in implementing the original aaai.sty file, liberally using the ideas of other style hackers, including Barbara Beeton. We also acknowledge with thanks the work of George Ferguson for his guide to using the style and BibTeX files --- which has been incorporated into this document --- and Hans Guesgen, who provided several timely modifications, as well as the many others who have, from time to time, sent in suggestions on improvements to the AAAI style. We are especially grateful to Francisco Cruz, Marc Pujol-Gonzalez, and Mico Loretan for the improvements to the Bib\TeX{} and \LaTeX{} files made in 2020.

% The preparation of the \LaTeX{} and Bib\TeX{} files that implement these instructions was supported by Schlumberger Palo Alto Research, AT\&T Bell Laboratories, Morgan Kaufmann Publishers, The Live Oak Press, LLC, and AAAI Press. Bibliography style changes were added by Sunil Issar. \verb+\+pubnote was added by J. Scott Penberthy. George Ferguson added support for printing the AAAI copyright slug. Additional changes to aaai25.sty and aaai25.bst have been made by Francisco Cruz and Marc Pujol-Gonzalez.

% \bigskip
% \noindent Thank you for reading these instructions carefully. We look forward to receiving your electronic files!

\section{Acknowledgments}
% The numerical calculations were performed on the supercomputers at Information Technology Center of The University of Tokyo. 
This research was conducted using the Wisteria / BDEC-01 at The University of Tokyo.
This work was supported by MEXT ``Advanced Research Infrastructure for Materials and Nanotechnology in Japan (ARIM) '', MEXT ``Developing a Research Data Ecosystem for the Promotion of Data-Driven Science'' (2024-10) and JSPS KAKENHI (22K17899, 20K15012).
We used ChatGPT (GPT-4o) for grammar checks and writing revisions.

\bibliography{aaai25}


\newpage

\section{Appendix}
\subsection{Invariance of CGCNN}
Let us prove Theorem \ref{thm:invariance_of_CGCNN}. First, in order to prove the strong re-description invariance of $\mathrm{CGCNN}_\theta$, consider constructing a graph corresponding to a solid material.
\begin{definition}[Graph Construction for Solid Materials]
For a graph including infinite nodes corresponding to a solid material $S \in \mathcal{S}$, the neighborhood $\mathrm{N}_S(a, x)$ of a vertex $(a, x) \in S$ is defined as follows:
$$
\mathrm{N}_S(a, x) = \{(a', x') \mid (a', x') \in S, \|x' - x\| < D\}.
$$
For $(a', x') \in \mathrm{N}_S(a, x)$, an edge is formed between vertices $(a, x)$ and $(a', x')$, with a feature $e_{a, x, a', x'} \in \mathbb{R}^{d_\mathrm{e} \times 1}$ determined by the distance $\|x' - x\|$. Due to symmetry, the resulting graph is undirected.
\end{definition}

\begin{definition}[Graph Convolution for Solid Materials]
The feature $u_{a, x} \in \mathbb{R}^{d_\mathrm{v} \times 1}$ of a vertex $(a, x)$ is initialized based on its atomic species and is updated as follows:
$$
u_{a, x} \leftarrow u_{a, x} + \sum_{(a', x') \in \mathrm{N}_S(a, x)} \psi_\theta(u_{a, x}, u_{a', x'}, e_{a, x, a', x'}).
$$
\end{definition}

By using these, the strong re-description invariance of $\mathrm{CGCNN_\theta}$ can be established as follows:

\begin{lemma}[Strong Re-description Invariance of CGCNN]
\label{lem:strong_redescription_invariance_of_CGCNN}
$\mathrm{CGCNN_\theta'}$ is strongly re-description invarant.
\end{lemma}

\begin{proof}[Proof of Lemma \ref{lem:strong_redescription_invariance_of_CGCNN}]
Consider a periodic unit $((a_1, \ldots, a_n), \allowbreak (x_1, \ldots, x_n), L) \in \mathcal{P}_n$ corresponding to a solid material $S = \mathrm{PtoS}(P)$. By mathematical induction, after graph construction and a fixed number of graph convolution steps, for any $i \in \{1, \ldots, n\}$ and $k \in \mathbb{Z}^{3 \times 1}$, we have $u_{a_i, x_i + Lk} = v_{i, k}$. The resulting node features are passed through mean pooling and an MLP, yielding the same result.
\end{proof}

Next, we show proofs of the translation and rotation invariance of $\mathrm{CGCNN}_\theta \circ \mathrm{PtoS}^{-1}$

\begin{lemma}[Translation invariance of CGCNN]
\label{lem:translation_invariance_of_CGCNN}
$\mathrm{CGCNN_\theta} \circ \mathrm{PtoS}^{-1}$ is translation invariant.
\end{lemma}
\begin{lemma}[Rotation invariance of CGCNN]
\label{lem:rotation_invariance_of_CGCNN}
$\mathrm{CGCNN_\theta} \circ \mathrm{PtoS}^{-1}$ is rotation invariant.
\end{lemma}

\begin{proof}[Proof of Lemma \ref{lem:translation_invariance_of_CGCNN}]
Let a crystal be $C \in \mathrm{PtoS}(\bigcup_{n=1}^\infty \mathcal{P}_n)$ and a periodic unit $P = (A, (x_1, \ldots, x_n), L)$ such that $\mathrm{PtoS}(P) = C$. For any translation $b \in \mathrm{T}(3)$, define a new periodic unit $P' = (A, (x_1 + b, \ldots, x_n + b), L)$. The graph and initial node features constructed from $P$ and $P'$ are identical. Thus, $\mathrm{CGCNN}_\theta(P') = \mathrm{CGCNN}_\theta(P)$.
Hence:
\begin{align*}
&\mathrm{CGCNN}_\theta \circ \mathrm{PtoS}^{-1}(\{(a, x + b) \mid (a, x) \in C\}) \\
= &\mathrm{CGCNN}_\theta(P') \\
= &\mathrm{CGCNN}_\theta(P) \\
= &\mathrm{CGCNN}_\theta \circ \mathrm{PtoS}^{-1}(C).
\end{align*}
\end{proof}
\begin{proof}[Proof of Lemma \ref{lem:rotation_invariance_of_CGCNN}]
It can be proven in a similar manner to Lemma \ref{lem:translation_invariance_of_CGCNN}.
\end{proof}

Using the above, Theorem \ref{thm:invariance_of_CGCNN} follows.
\begin{proof}[Proof of Theorem \ref{thm:invariance_of_CGCNN}]
It follows from Lemmas \ref{lem:strong_redescription_invariance_of_CGCNN}, \ref{lem:translation_invariance_of_CGCNN}, \ref{lem:rotation_invariance_of_CGCNN}.
\end{proof}

\subsection{Gradient of Loss function in EBMs}
Theorem \ref{thm:gradient_of_loss_function_in_EBMs} is well-known and can be derived through the following transformations:
\begin{proof}[Proof of Theorem \ref{thm:gradient_of_loss_function_in_EBMs}]
\begin{align}
    & \nabla_\theta J(\theta) \\
    = & \frac{\beta}{n} \sum_{i=1}^n \nabla_\theta H_\theta(x_i) + \nabla_\theta \log Z(\theta, \beta) \\
    = & \frac{\beta}{n} \sum_{i=1}^n \nabla_\theta H_\theta(x_i) + \frac{\nabla_\theta Z(\theta, \beta)}{Z(\theta, \beta)} \\
    = & \frac{\beta}{n} \sum_{i=1}^n \nabla_\theta H_\theta(x_i) - \notag \\
    &\qquad\frac{\beta}{Z(\theta, \beta)} \int_{x \in \mathcal{X}} \exp(-\beta H_\theta(x)) \nabla_\theta H_\theta(x) \mathrm{d}x \\
    = & \frac{\beta}{n} \sum_{i=1}^n \nabla_\theta H_\theta(x_i) - \beta \mathbb{E}_{p_\theta(x)}\left[\nabla_\theta H_\theta(x)\right] \\
    = & \frac{\beta}{n} \sum_{i=1}^n (\nabla_\theta H_\theta(x_i) - \mathbb{E}_{p_\theta(x)}\left[\nabla_\theta H_\theta(x)\right]).
\end{align}
\end{proof}

\subsection{Properties of Energy Function in ContinuouSP}
Let us prove Theorem \ref{thm:properties_of_energy_function_in_ContinuouSP}. By $H_\theta = \mathrm{CGCNN}_\theta'$, it suffices to demonstrate the properties of $\mathrm{CGCNN}_\theta'$. Since most of the proof is almost identical to that of Theorem \ref{thm:invariance_of_CGCNN}, we focus here on proving the continuity of $\mathrm{CGCNN}_\theta' \circ \mathrm{PtoS}^{-1}$. First, consider the following lemma:
\begin{lemma}[continuity on periodic units to solid materials]
\label{lem:continuity_on_periodic_units_to_solid_materials}
Let $g: \bigcup_{n=1}^\infty \mathcal{P}_n \to \mathbb{R}^m$ be strongly permutation-invariant. Then, if the mapping $h_A$ satisfying $h_A(X, L) = g(A, X, L)$, is continuous for any $A$ with respect to each column vector of the input, then $g \circ \mathrm{PtoS}^{-1}$is continuous.
\end{lemma}
\begin{proof}[Proof of Lemma \ref{lem:continuity_on_periodic_units_to_solid_materials}]
Let $C \in \mathrm{PtoS}(\bigcup_{n=1}^\infty \mathcal{P}_n)$ and $\varepsilon > 0$. Consider $P = (A, X, L) = (A, (x_1, \ldots, x_n), (l_1, l_2, l_3))$ such that $\mathrm{PtoS}(P) = C$ and $n$ has sufficiently many divisors. By the continuity of $h_A$, there exists $\delta > 0$ such that for any periodic unit $P' = (A, X', L')$, if the norms of each column of $X' - X$ and $L' - L$ are less than $\delta$, then $\|h_A(X', L') - h_A(X, L)\| < \varepsilon$. For this $\delta$, consider any $C' \in \mathrm{PtoS}(\bigcup_{n=1}^\infty \mathcal{P}_n)$. There exists a bijection $\phi: C \to C'$ such that for any atom $(a, x) \in C$, if $\phi(a, x) = (a', x')$, then $a' = a$ and $\|x' - x\| < \delta$. Thus, there exists $P' = (A, X', L') = (A, (x_1', \ldots, x_n'), (l_1', l_2', l_3'))$ such that $\mathrm{PtoS}(P') = C'$ and for any $i \in \{1, \ldots, n\}$ and $k, k' \in \mathbb{Z}^{3 \times 1}$, $\|x_i' + L'k' - x_i - Lk\| < \delta$. By considering $k = k' = 0$, $\|x_i' - x_i\| < \delta$ stands. Also, by considering $k = k' = (\pm 1, 0, 0)^\mathrm{T}$, $\|l_1' - l_1\| < \delta$ stands. Similarly, $\|l_2' - l_2\| < \delta$ and $\|l_3' - l_3\| < \delta$. Hence, $\|h_A(X', L') - h_A(X, L)\| = \|g(P') - g(P)\| = \|g \circ \mathrm{PtoS}^{-1}(C') - g \circ \mathrm{PtoS}^{-1}(C)\| < \varepsilon$.
\end{proof}

By using this lemma, we can prove the continuity of $\mathrm{CGCNN}_\theta' \circ \mathrm{PtoS}^{-1}$ as follows:
\begin{lemma}[Continuity of CGCNN']
\label{lem:continuity_of_CGCNN'}
$\mathrm{CGCNN}_\theta' \circ \mathrm{PtoS}^{-1}$ is continuous.
\end{lemma}
\begin{proof}[Proof of Lemma \ref{lem:continuity_of_CGCNN'}]
The mapping $\omega_k$ defined as 
\begin{multline}
\omega_k(x_i, x_j, L) = \\
\left\{\begin{array}{ll}
\cos^2\left(\frac{\pi\|x_j + Lk - x_i\|}{2D}\right) & (\|x_j + Lk - x_i\| < D) \\
0 & (\|x_j + Lk - x_i\| \geq D)
\end{array}\right.
\end{multline}
is continuous with respect to each column vector of the input.Using this, the update rule for $\mathrm{CGCNN}_\theta'$ is expressed as:
\begin{equation}
v_i \leftarrow v_i + \sum_{\substack{j \in \{1, \ldots, n\} \\ k \in \mathbb{Z}^{3 \times 1}}} \omega_k(x_i, x_j, L)\psi_\theta(v_i, v_j, e_{i, j, k}).
\end{equation}
This update rule relies solely on mappings that are continuous with respect to each column vector of $X$ and $L$. Additionally, since mean pooling and MLP are continuous mappings, $\mathrm{CGCNN}_\theta'$ satisfies the conditions of Lemma \ref{lem:continuity_on_periodic_units_to_solid_materials}.
\end{proof}

Thus, the proof of Theorem \ref{thm:properties_of_energy_function_in_ContinuouSP} is complete.
\begin{proof}[Proof of Theorem \ref{thm:properties_of_energy_function_in_ContinuouSP}]
The content corresponding to Theorem \ref{thm:invariance_of_CGCNN} and Lemma \ref{lem:continuity_of_CGCNN'} implies properties concerning $\mathrm{CGCNN_\theta'}$. These properties also hold for $H_\theta = \mathrm{CGCNN_\theta'}$.
\end{proof}

\subsection{Properties of Probability Density Function in ContinuouSP}
\begin{proof}[Proof of Theorem \ref{thm:properties_of_probability_density_function_in_ContinuouSP}]
Considering the substitution that simultaneously changes the order of $X'$ in Eq. \ref{eq:partition_function_in_ContinuouSP} along with the order of $A$, the value of $Z$, the denominator in Eq. \ref{eq:probability_distribution_function_in_ContinuouSP}, remains unchanged due to the strong re-description invariance of $H_\theta$. Therefore, revisiting the strong re-description invariance of $H_\theta$ in the numerator of Eq. \ref{eq:probability_distribution_function_in_ContinuouSP}, we can conclude that $p_\theta$ remains unchanged even when the unit cell is altered without changing the crystal or the number of atoms per period. Furthermore, due to the invariance and the continuity of $H_\theta \circ \mathrm{PtoS}^{-1}$, it follows immediately that $p_\theta \circ \mathrm{PtoS}|_{\mathcal{P}_n}^{-1}$ also exhibits the same properties.
\end{proof}

\subsection{Gradient of Loss Function in ContinuouSP}
\begin{proof}[Proof of Theorem \ref{thm:gradient_of_loss_function_in_ContinuouSP}]
\begin{align}
    & \nabla_\theta J(\theta) \\
    =&\frac{1}{m} \sum_{i=1}^m \Bigg(\beta \nabla_\theta H_\theta(A_i, X_i, L_i) + \notag \\ 
    & \qquad\qquad\quad \sum_{j=1}^\infty (1-q)^{j-1} q \nabla_\theta \log Z(\theta, \beta, A_i')\Bigg) \\
    = &\frac{1}{m} \sum_{i=1}^m \Bigg(\beta \nabla_\theta H_\theta(A_i, X_i, L_i) + \notag \\
    & \qquad\qquad\qquad\sum_{j=1}^\infty (1-q)^{j-1} q \frac{\nabla_\theta Z(\theta, \beta, A_i')}{Z(\theta, \beta, A_i')} \Bigg) \\
    = &\frac{1}{m} \sum_{i=1}^m \Bigg(\beta \nabla_\theta H_\theta(A_i, X_i, L_i) + \sum_{j=1}^\infty \frac{\beta(1-q)^{j-1} q}{Z(\theta, \beta, A_i')} \notag \\
    &\!\!\! \int_{\substack{X \in \mathbb{R}^{3 \times n} \\ L \in \mathbb{R}^{3 \times 3}}} \exp(-\beta H_\theta(A_i', X, L)) \nabla_\theta H_\theta(A_i', X, L) \mathrm{d}X \mathrm{d}L \Bigg) \\
    = & \frac{1}{m} \sum_{i=1}^m \Bigg(\beta \nabla_\theta H_\theta(A_i, X_i, L_i) - \notag \\
    & \qquad\qquad\qquad \beta\mathbb{E}_{\substack{j \sim \mathrm{Geom}(q) \\ p_\theta(X, L \mid A_i')}}\left[ \nabla_\theta H_\theta(A_i', X, L) \right] \Bigg) \\
    = & \frac{\beta}{m} \sum_{i=1}^m \Bigg( \nabla_\theta H_\theta(A_i, X_i, L_i) - \notag \\ 
    & \qquad\qquad\qquad \mathbb{E}_{\substack{j \sim \mathrm{Geom}(q) \\ p_\theta(X, L \mid A_i')}}\left[ \nabla_\theta H_\theta(A_i', X, L) \right] \Bigg).
\end{align}
\end{proof}

\subsection{Properties of Loss Function in ContinuouSP}
\begin{proof}[Proof of Theorem \ref{thm:properties_of_loss_function_in_ContinuouSP}]
Since $\mathrm{Reduce}(A)$ is the composition which is intrinsic to the crystal, the summation doesn't depend on re-description. Thus, also with the strong re-description invariance of $H_\theta$, the strong re-description invariance of $\mathcal{L}$ stands. Furthermore, due to the invariance and the continuity of $H_\theta \circ \mathrm{PtoS}^{-1}$, it follows immediately that $\mathcal{L} \circ \mathrm{PtoS}^{-1}$ also exhibits the same properties.
\end{proof}

% \section{Preparing an Anonymous Submission}

% This document details the formatting requirements for anonymous submissions. The requirements are the same as for camera ready papers but with a few notable differences:

% \begin{itemize}
%     \item Anonymous submissions must not include the author names and affiliations. Write ``Anonymous Submission'' as the ``sole author'' and leave the affiliations empty.
%     \item The PDF document's metadata should be cleared with a metadata-cleaning tool before submitting it. This is to prevent leaked information from revealing your identity.
%     \item References must be anonymized whenever the reader can infer that they are to the authors' previous work.
%     \item AAAI's copyright notice should not be included as a footer in the first page.
%     \item Only the PDF version is required at this stage. No source versions will be requested, nor any copyright transfer form.
% \end{itemize}

% You can remove the copyright notice and ensure that your names aren't shown by including \texttt{submission} option when loading the \texttt{aaai25} package:

% \begin{quote}\begin{scriptsize}\begin{verbatim}
% \documentclass[letterpaper]{article}
% \usepackage[submission]{aaai25}
% \end{verbatim}\end{scriptsize}\end{quote}

% The remainder of this document are the original camera-
% ready instructions. Any contradiction of the above points
% ought to be ignored while preparing anonymous submis-
% sions.

% \section{Camera-Ready Guidelines}

% Congratulations on having a paper selected for inclusion in an AAAI Press proceedings or technical report! This document details the requirements necessary to get your accepted paper published using PDF\LaTeX{}. If you are using Microsoft Word, instructions are provided in a different document. AAAI Press does not support any other formatting software.

% The instructions herein are provided as a general guide for experienced \LaTeX{} users. If you do not know how to use \LaTeX{}, please obtain assistance locally. AAAI cannot provide you with support and the accompanying style files are \textbf{not} guaranteed to work. If the results you obtain are not in accordance with the specifications you received, you must correct your source file to achieve the correct result.

% These instructions are generic. Consequently, they do not include specific dates, page charges, and so forth. Please consult your specific written conference instructions for details regarding your submission. Please review the entire document for specific instructions that might apply to your particular situation. All authors must comply with the following:

% \begin{itemize}
% \item You must use the 2025 AAAI Press \LaTeX{} style file and the aaai25.bst bibliography style files, which are located in the 2025 AAAI Author Kit (aaai25.sty, aaai25.bst).
% \item You must complete, sign, and return by the deadline the AAAI copyright form (unless directed by AAAI Press to use the AAAI Distribution License instead).
% \item You must read and format your paper source and PDF according to the formatting instructions for authors.
% \item You must submit your electronic files and abstract using our electronic submission form \textbf{on time.}
% \item You must pay any required page or formatting charges to AAAI Press so that they are received by the deadline.
% \item You must check your paper before submitting it, ensuring that it compiles without error, and complies with the guidelines found in the AAAI Author Kit.
% \end{itemize}

% \section{Copyright}
% All papers submitted for publication by AAAI Press must be accompanied by a valid signed copyright form. They must also contain the AAAI copyright notice at the bottom of the first page of the paper. There are no exceptions to these requirements. If you fail to provide us with a signed copyright form or disable the copyright notice, we will be unable to publish your paper. There are \textbf{no exceptions} to this policy. You will find a PDF version of the AAAI copyright form in the AAAI AuthorKit. Please see the specific instructions for your conference for submission details.

% \section{Formatting Requirements in Brief}
% We need source and PDF files that can be used in a variety of ways and can be output on a variety of devices. The design and appearance of the paper is strictly governed by the aaai style file (aaai25.sty).
% \textbf{You must not make any changes to the aaai style file, nor use any commands, packages, style files, or macros within your own paper that alter that design, including, but not limited to spacing, floats, margins, fonts, font size, and appearance.} AAAI imposes requirements on your source and PDF files that must be followed. Most of these requirements are based on our efforts to standardize conference manuscript properties and layout. All papers submitted to AAAI for publication will be recompiled for standardization purposes. Consequently, every paper submission must comply with the following requirements:

% \begin{itemize}
% \item Your .tex file must compile in PDF\LaTeX{} --- (you may not include .ps or .eps figure files.)
% \item All fonts must be embedded in the PDF file --- including your figures.
% \item Modifications to the style file, whether directly or via commands in your document may not ever be made, most especially when made in an effort to avoid extra page charges or make your paper fit in a specific number of pages.
% \item No type 3 fonts may be used (even in illustrations).
% \item You may not alter the spacing above and below captions, figures, headings, and subheadings.
% \item You may not alter the font sizes of text elements, footnotes, heading elements, captions, or title information (for references and mathematics, please see the limited exceptions provided herein).
% \item You may not alter the line spacing of text.
% \item Your title must follow Title Case capitalization rules (not sentence case).
% \item \LaTeX{} documents must use the Times or Nimbus font package (you may not use Computer Modern for the text of your paper).
% \item No \LaTeX{} 209 documents may be used or submitted.
% \item Your source must not require use of fonts for non-Roman alphabets within the text itself. If your paper includes symbols in other languages (such as, but not limited to, Arabic, Chinese, Hebrew, Japanese, Thai, Russian and other Cyrillic languages), you must restrict their use to bit-mapped figures. Fonts that require non-English language support (CID and Identity-H) must be converted to outlines or 300 dpi bitmap or removed from the document (even if they are in a graphics file embedded in the document).
% \item Two-column format in AAAI style is required for all papers.
% \item The paper size for final submission must be US letter without exception.
% \item The source file must exactly match the PDF.
% \item The document margins may not be exceeded (no overfull boxes).
% \item The number of pages and the file size must be as specified for your event.
% \item No document may be password protected.
% \item Neither the PDFs nor the source may contain any embedded links or bookmarks (no hyperref or navigator packages).
% \item Your source and PDF must not have any page numbers, footers, or headers (no pagestyle commands).
% \item Your PDF must be compatible with Acrobat 5 or higher.
% \item Your \LaTeX{} source file (excluding references) must consist of a \textbf{single} file (use of the ``input" command is not allowed.
% \item Your graphics must be sized appropriately outside of \LaTeX{} (do not use the ``clip" or ``trim'' command) .
% \end{itemize}

% If you do not follow these requirements, your paper will be returned to you to correct the deficiencies.

% \section{What Files to Submit}
% You must submit the following items to ensure that your paper is published:
% \begin{itemize}
% \item A fully-compliant PDF file.
% \item Your \LaTeX{} source file submitted as a \textbf{single} .tex file (do not use the ``input" command to include sections of your paper --- every section must be in the single source file). (The only allowable exception is .bib file, which should be included separately).
% \item The bibliography (.bib) file(s).
% \item Your source must compile on our system, which includes only standard \LaTeX{} 2020 TeXLive support files.
% \item Only the graphics files used in compiling paper.
% \item The \LaTeX{}-generated files (e.g. .aux,  .bbl file, PDF, etc.).
% \end{itemize}

% Your \LaTeX{} source will be reviewed and recompiled on our system (if it does not compile, your paper will be returned to you. \textbf{Do not submit your source in multiple text files.} Your single \LaTeX{} source file must include all your text, your bibliography (formatted using aaai25.bst), and any custom macros.

% Your files should work without any supporting files (other than the program itself) on any computer with a standard \LaTeX{} distribution.

% \textbf{Do not send files that are not actually used in the paper.} Avoid including any files not needed for compiling your paper, including, for example, this instructions file, unused graphics files, style files, additional material sent for the purpose of the paper review, intermediate build files and so forth.

% \textbf{Obsolete style files.} The commands for some common packages (such as some used for algorithms), may have changed. Please be certain that you are not compiling your paper using old or obsolete style files.

% \textbf{Final Archive.} Place your source files in a single archive which should be compressed using .zip. The final file size may not exceed 10 MB.
% Name your source file with the last (family) name of the first author, even if that is not you.


% \section{Using \LaTeX{} to Format Your Paper}

% The latest version of the AAAI style file is available on AAAI's website. Download this file and place it in the \TeX\ search path. Placing it in the same directory as the paper should also work. You must download the latest version of the complete AAAI Author Kit so that you will have the latest instruction set and style file.

% \subsection{Document Preamble}

% In the \LaTeX{} source for your paper, you \textbf{must} place the following lines as shown in the example in this subsection. This command set-up is for three authors. Add or subtract author and address lines as necessary, and uncomment the portions that apply to you. In most instances, this is all you need to do to format your paper in the Times font. The helvet package will cause Helvetica to be used for sans serif. These files are part of the PSNFSS2e package, which is freely available from many Internet sites (and is often part of a standard installation).

% Leave the setcounter for section number depth commented out and set at 0 unless you want to add section numbers to your paper. If you do add section numbers, you must uncomment this line and change the number to 1 (for section numbers), or 2 (for section and subsection numbers). The style file will not work properly with numbering of subsubsections, so do not use a number higher than 2.

% \subsubsection{The Following Must Appear in Your Preamble}
% \begin{quote}
% \begin{scriptsize}\begin{verbatim}
% \documentclass[letterpaper]{article}
% % DO NOT CHANGE THIS
% \usepackage[submission]{aaai25} % DO NOT CHANGE THIS
% \usepackage{times} % DO NOT CHANGE THIS
% \usepackage{helvet} % DO NOT CHANGE THIS
% \usepackage{courier} % DO NOT CHANGE THIS
% \usepackage[hyphens]{url} % DO NOT CHANGE THIS
% \usepackage{graphicx} % DO NOT CHANGE THIS
% \urlstyle{rm} % DO NOT CHANGE THIS
% \def\UrlFont{\rm} % DO NOT CHANGE THIS
% \usepackage{graphicx}  % DO NOT CHANGE THIS
% \usepackage{natbib}  % DO NOT CHANGE THIS
% \usepackage{caption}  % DO NOT CHANGE THIS
% \frenchspacing % DO NOT CHANGE THIS
% \setlength{\pdfpagewidth}{8.5in} % DO NOT CHANGE THIS
% \setlength{\pdfpageheight}{11in} % DO NOT CHANGE THIS
% %
% % Keep the \pdfinfo as shown here. There's no need
% % for you to add the /Title and /Author tags.
% \pdfinfo{
% /TemplateVersion (2025.1)
% }
% \end{verbatim}\end{scriptsize}
% \end{quote}

% \subsection{Preparing Your Paper}

% After the preamble above, you should prepare your paper as follows:
% \begin{quote}
% \begin{scriptsize}\begin{verbatim}
% \begin{document}
% \maketitle
% \begin{abstract}
% %...
% \end{abstract}\end{verbatim}\end{scriptsize}
% \end{quote}

% \noindent If you want to add links to the paper's code, dataset(s), and extended version or similar this is the place to add them, within a \emph{links} environment:
% \begin{quote}%
% \begin{scriptsize}\begin{verbatim}
% \begin{links}
%   \link{Code}{https://aaai.org/example/guidelines}
%   \link{Datasets}{https://aaai.org/example/datasets}
%   \link{Extended version}{https://aaai.org/example}
% \end{links}\end{verbatim}\end{scriptsize}
% \end{quote}
% \noindent Make sure that you do not de-anonymize yourself with these links.

% \noindent You should then continue with the body of your paper. Your paper must conclude with the references, which should be inserted as follows:
% \begin{quote}
% \begin{scriptsize}\begin{verbatim}
% % References and End of Paper
% % These lines must be placed at the end of your paper
% \bibliography{Bibliography-File}
% \end{document}
% \end{verbatim}\end{scriptsize}
% \end{quote}

% \begin{quote}
% \begin{scriptsize}\begin{verbatim}
% \begin{document}\\
% \maketitle\\
% ...\\
% \bibliography{Bibliography-File}\\
% \end{document}\\
% \end{verbatim}\end{scriptsize}
% \end{quote}

% \subsection{Commands and Packages That May Not Be Used}
% \begin{table*}[t]
% \centering

% \begin{tabular}{l|l|l|l}
% \textbackslash abovecaption &
% \textbackslash abovedisplay &
% \textbackslash addevensidemargin &
% \textbackslash addsidemargin \\
% \textbackslash addtolength &
% \textbackslash baselinestretch &
% \textbackslash belowcaption &
% \textbackslash belowdisplay \\
% \textbackslash break &
% \textbackslash clearpage &
% \textbackslash clip &
% \textbackslash columnsep \\
% \textbackslash float &
% \textbackslash input &
% \textbackslash input &
% \textbackslash linespread \\
% \textbackslash newpage &
% \textbackslash pagebreak &
% \textbackslash renewcommand &
% \textbackslash setlength \\
% \textbackslash text height &
% \textbackslash tiny &
% \textbackslash top margin &
% \textbackslash trim \\
% \textbackslash vskip\{- &
% \textbackslash vspace\{- \\
% \end{tabular}
% %}
% \caption{Commands that must not be used}
% \label{table1}
% \end{table*}

% \begin{table}[t]
% \centering
% %\resizebox{.95\columnwidth}{!}{
% \begin{tabular}{l|l|l|l}
%     authblk & babel & cjk & dvips \\
%     epsf & epsfig & euler & float \\
%     fullpage & geometry & graphics & hyperref \\
%     layout & linespread & lmodern & maltepaper \\
%     navigator & pdfcomment & pgfplots & psfig \\
%     pstricks & t1enc & titlesec & tocbind \\
%     ulem
% \end{tabular}
% \caption{LaTeX style packages that must not be used.}
% \label{table2}
% \end{table}

% There are a number of packages, commands, scripts, and macros that are incompatable with aaai25.sty. The common ones are listed in tables \ref{table1} and \ref{table2}. Generally, if a command, package, script, or macro alters floats, margins, fonts, sizing, linespacing, or the presentation of the references and citations, it is unacceptable. Note that negative vskip and vspace may not be used except in certain rare occurances, and may never be used around tables, figures, captions, sections, subsections, subsubsections, or references.


% \subsection{Page Breaks}
% For your final camera ready copy, you must not use any page break commands. References must flow directly after the text without breaks. Note that some conferences require references to be on a separate page during the review process. AAAI Press, however, does not require this condition for the final paper.


% \subsection{Paper Size, Margins, and Column Width}
% Papers must be formatted to print in two-column format on 8.5 x 11 inch US letter-sized paper. The margins must be exactly as follows:
% \begin{itemize}
% \item Top margin: 1.25 inches (first page), .75 inches (others)
% \item Left margin: .75 inches
% \item Right margin: .75 inches
% \item Bottom margin: 1.25 inches
% \end{itemize}


% The default paper size in most installations of \LaTeX{} is A4. However, because we require that your electronic paper be formatted in US letter size, the preamble we have provided includes commands that alter the default to US letter size. Please note that using any other package to alter page size (such as, but not limited to the Geometry package) will result in your final paper being returned to you for correction.


% \subsubsection{Column Width and Margins.}
% To ensure maximum readability, your paper must include two columns. Each column should be 3.3 inches wide (slightly more than 3.25 inches), with a .375 inch (.952 cm) gutter of white space between the two columns. The aaai25.sty file will automatically create these columns for you.

% \subsection{Overlength Papers}
% If your paper is too long and you resort to formatting tricks to make it fit, it is quite likely that it will be returned to you. The best way to retain readability if the paper is overlength is to cut text, figures, or tables. There are a few acceptable ways to reduce paper size that don't affect readability. First, turn on \textbackslash frenchspacing, which will reduce the space after periods. Next, move all your figures and tables to the top of the page. Consider removing less important portions of a figure. If you use \textbackslash centering instead of \textbackslash begin\{center\} in your figure environment, you can also buy some space. For mathematical environments, you may reduce fontsize {\bf but not below 6.5 point}.


% Commands that alter page layout are forbidden. These include \textbackslash columnsep,  \textbackslash float, \textbackslash topmargin, \textbackslash topskip, \textbackslash textheight, \textbackslash textwidth, \textbackslash oddsidemargin, and \textbackslash evensizemargin (this list is not exhaustive). If you alter page layout, you will be required to pay the page fee. Other commands that are questionable and may cause your paper to be rejected include \textbackslash parindent, and \textbackslash parskip. Commands that alter the space between sections are forbidden. The title sec package is not allowed. Regardless of the above, if your paper is obviously ``squeezed" it is not going to to be accepted. Options for reducing the length of a paper include reducing the size of your graphics, cutting text, or paying the extra page charge (if it is offered).


% \subsection{Type Font and Size}
% Your paper must be formatted in Times Roman or Nimbus. We will not accept papers formatted using Computer Modern or Palatino or some other font as the text or heading typeface. Sans serif, when used, should be Courier. Use Symbol or Lucida or Computer Modern for \textit{mathematics only. }

% Do not use type 3 fonts for any portion of your paper, including graphics. Type 3 bitmapped fonts are designed for fixed resolution printers. Most print at 300 dpi even if the printer resolution is 1200 dpi or higher. They also often cause high resolution imagesetter devices to crash. Consequently, AAAI will not accept electronic files containing obsolete type 3 fonts. Files containing those fonts (even in graphics) will be rejected. (Authors using blackboard symbols must avoid packages that use type 3 fonts.)

% Fortunately, there are effective workarounds that will prevent your file from embedding type 3 bitmapped fonts. The easiest workaround is to use the required times, helvet, and courier packages with \LaTeX{}2e. (Note that papers formatted in this way will still use Computer Modern for the mathematics. To make the math look good, you'll either have to use Symbol or Lucida, or you will need to install type 1 Computer Modern fonts --- for more on these fonts, see the section ``Obtaining Type 1 Computer Modern.")

% If you are unsure if your paper contains type 3 fonts, view the PDF in Acrobat Reader. The Properties/Fonts window will display the font name, font type, and encoding properties of all the fonts in the document. If you are unsure if your graphics contain type 3 fonts (and they are PostScript or encapsulated PostScript documents), create PDF versions of them, and consult the properties window in Acrobat Reader.

% The default size for your type must be ten-point with twelve-point leading (line spacing). Start all pages (except the first) directly under the top margin. (See the next section for instructions on formatting the title page.) Indent ten points when beginning a new paragraph, unless the paragraph begins directly below a heading or subheading.


% \subsubsection{Obtaining Type 1 Computer Modern for \LaTeX{}.}

% If you use Computer Modern for the mathematics in your paper (you cannot use it for the text) you may need to download type 1 Computer fonts. They are available without charge from the American Mathematical Society:
% http://www.ams.org/tex/type1-fonts.html.

% \subsubsection{Nonroman Fonts.}
% If your paper includes symbols in other languages (such as, but not limited to, Arabic, Chinese, Hebrew, Japanese, Thai, Russian and other Cyrillic languages), you must restrict their use to bit-mapped figures.

% \subsection{Title and Authors}
% Your title must appear centered over both text columns in sixteen-point bold type (twenty-four point leading). The title must be written in Title Case according to the Chicago Manual of Style rules. The rules are a bit involved, but in general verbs (including short verbs like be, is, using, and go), nouns, adverbs, adjectives, and pronouns should be capitalized, (including both words in hyphenated terms), while articles, conjunctions, and prepositions are lower case unless they directly follow a colon or long dash. You can use the online tool \url{https://titlecaseconverter.com/} to double-check the proper capitalization (select the "Chicago" style and mark the "Show explanations" checkbox).

% Author's names should appear below the title of the paper, centered in twelve-point type (with fifteen point leading), along with affiliation(s) and complete address(es) (including electronic mail address if available) in nine-point roman type (the twelve point leading). You should begin the two-column format when you come to the abstract.

% \subsubsection{Formatting Author Information.}
% Author information has to be set according to the following specification depending if you have one or more than one affiliation. You may not use a table nor may you employ the \textbackslash authorblk.sty package. For one or several authors from the same institution, please separate them with commas and write all affiliation directly below (one affiliation per line) using the macros \textbackslash author and \textbackslash affiliations:

% \begin{quote}\begin{scriptsize}\begin{verbatim}
% \author{
%     Author 1, ..., Author n\\
% }
% \affiliations {
%     Address line\\
%     ... \\
%     Address line\\
% }
% \end{verbatim}\end{scriptsize}\end{quote}


% \noindent For authors from different institutions, use \textbackslash textsuperscript \{\textbackslash rm x \} to match authors and affiliations. Notice that there should not be any spaces between the author name (or comma following it) and the superscript.

% \begin{quote}\begin{scriptsize}\begin{verbatim}
% \author{
%     AuthorOne\equalcontrib\textsuperscript{\rm 1,\rm 2},
%     AuthorTwo\equalcontrib\textsuperscript{\rm 2},
%     AuthorThree\textsuperscript{\rm 3},\\
%     AuthorFour\textsuperscript{\rm 4},
%     AuthorFive \textsuperscript{\rm 5}}
% }
% \affiliations {
%     \textsuperscript{\rm 1}AffiliationOne,\\
%     \textsuperscript{\rm 2}AffiliationTwo,\\
%     \textsuperscript{\rm 3}AffiliationThree,\\
%     \textsuperscript{\rm 4}AffiliationFour,\\
%     \textsuperscript{\rm 5}AffiliationFive\\
%     \{email, email\}@affiliation.com,
%     email@affiliation.com,
%     email@affiliation.com,
%     email@affiliation.com
% }
% \end{verbatim}\end{scriptsize}\end{quote}

% You can indicate that some authors contributed equally using the \textbackslash equalcontrib command. This will add a marker after the author names and a footnote on the first page.

% Note that you may want to  break the author list for better visualization. You can achieve this using a simple line break (\textbackslash  \textbackslash).

% \subsection{\LaTeX{} Copyright Notice}
% The copyright notice automatically appears if you use aaai25.sty. It has been hardcoded and may not be disabled.

% \subsection{Credits}
% Any credits to a sponsoring agency should appear in the acknowledgments section, unless the agency requires different placement. If it is necessary to include this information on the front page, use
% \textbackslash thanks in either the \textbackslash author or \textbackslash title commands.
% For example:
% \begin{quote}
% \begin{small}
% \textbackslash title\{Very Important Results in AI\textbackslash thanks\{This work is
%  supported by everybody.\}\}
% \end{small}
% \end{quote}
% Multiple \textbackslash thanks commands can be given. Each will result in a separate footnote indication in the author or title with the corresponding text at the botton of the first column of the document. Note that the \textbackslash thanks command is fragile. You will need to use \textbackslash protect.

% Please do not include \textbackslash pubnote commands in your document.

% \subsection{Abstract}
% Follow the example commands in this document for creation of your abstract. The command \textbackslash begin\{abstract\} will automatically indent the text block. Please do not indent it further. {Do not include references in your abstract!}

% \subsection{Page Numbers}

% Do not print any page numbers on your paper. The use of \textbackslash pagestyle is forbidden.

% \subsection{Text}
% The main body of the paper must be formatted in black, ten-point Times Roman with twelve-point leading (line spacing). You may not reduce font size or the linespacing. Commands that alter font size or line spacing (including, but not limited to baselinestretch, baselineshift, linespread, and others) are expressly forbidden. In addition, you may not use color in the text.

% \subsection{Citations}
% Citations within the text should include the author's last name and year, for example (Newell 1980). Append lower-case letters to the year in cases of ambiguity. Multiple authors should be treated as follows: (Feigenbaum and Engelmore 1988) or (Ford, Hayes, and Glymour 1992). In the case of four or more authors, list only the first author, followed by et al. (Ford et al. 1997).

% \subsection{Extracts}
% Long quotations and extracts should be indented ten points from the left and right margins.

% \begin{quote}
% This is an example of an extract or quotation. Note the indent on both sides. Quotation marks are not necessary if you offset the text in a block like this, and properly identify and cite the quotation in the text.

% \end{quote}

% \subsection{Footnotes}
% Use footnotes judiciously, taking into account that they interrupt the reading of the text. When required, they should be consecutively numbered throughout with superscript Arabic numbers. Footnotes should appear at the bottom of the page, separated from the text by a blank line space and a thin, half-point rule.

% \subsection{Headings and Sections}
% When necessary, headings should be used to separate major sections of your paper. Remember, you are writing a short paper, not a lengthy book! An overabundance of headings will tend to make your paper look more like an outline than a paper. The aaai25.sty package will create headings for you. Do not alter their size nor their spacing above or below.

% \subsubsection{Section Numbers.}
% The use of section numbers in AAAI Press papers is optional. To use section numbers in \LaTeX{}, uncomment the setcounter line in your document preamble and change the 0 to a 1. Section numbers should not be used in short poster papers and/or extended abstracts.

% \subsubsection{Section Headings.}
% Sections should be arranged and headed as follows:
% \begin{enumerate}
% \item Main content sections
% \item Appendices (optional)
% \item Ethical Statement (optional, unnumbered)
% \item Acknowledgements (optional, unnumbered)
% \item References (unnumbered)
% \end{enumerate}

% \subsubsection{Appendices.}
% Any appendices must appear after the main content. If your main sections are numbered, appendix sections must use letters instead of arabic numerals. In \LaTeX{} you can use the \texttt{\textbackslash appendix} command to achieve this effect and then use \texttt{\textbackslash section\{Heading\}} normally for your appendix sections.

% \subsubsection{Ethical Statement.}
% You can write a statement about the potential ethical impact of your work, including its broad societal implications, both positive and negative. If included, such statement must be written in an unnumbered section titled \emph{Ethical Statement}.

% \subsubsection{Acknowledgments.}
% The acknowledgments section, if included, appears right before the references and is headed ``Acknowledgments". It must not be numbered even if other sections are (use \texttt{\textbackslash section*\{Acknowledgements\}} in \LaTeX{}). This section includes acknowledgments of help from associates and colleagues, credits to sponsoring agencies, financial support, and permission to publish. Please acknowledge other contributors, grant support, and so forth, in this section. Do not put acknowledgments in a footnote on the first page. If your grant agency requires acknowledgment of the grant on page 1, limit the footnote to the required statement, and put the remaining acknowledgments at the back. Please try to limit acknowledgments to no more than three sentences.

% \subsubsection{References.}
% The references section should be labeled ``References" and must appear at the very end of the paper (don't end the paper with references, and then put a figure by itself on the last page). A sample list of references is given later on in these instructions. Please use a consistent format for references. Poorly prepared or sloppy references reflect badly on the quality of your paper and your research. Please prepare complete and accurate citations.

% \subsection{Illustrations and  Figures}

% \begin{figure}[t]
% \centering
% \includegraphics[width=0.9\columnwidth]{figure1} % Reduce the figure size so that it is slightly narrower than the column. Don't use precise values for figure width.This setup will avoid overfull boxes.
% \caption{Using the trim and clip commands produces fragile layers that can result in disasters (like this one from an actual paper) when the color space is corrected or the PDF combined with others for the final proceedings. Crop your figures properly in a graphics program -- not in LaTeX.}
% \label{fig1}
% \end{figure}

% \begin{figure*}[t]
% \centering
% \includegraphics[width=0.8\textwidth]{figure2} % Reduce the figure size so that it is slightly narrower than the column.
% \caption{Adjusting the bounding box instead of actually removing the unwanted data resulted multiple layers in this paper. It also needlessly increased the PDF size. In this case, the size of the unwanted layer doubled the paper's size, and produced the following surprising results in final production. Crop your figures properly in a graphics program. Don't just alter the bounding box.}
% \label{fig2}
% \end{figure*}

% % Using the \centering command instead of \begin{center} ... \end{center} will save space
% % Positioning your figure at the top of the page will save space and make the paper more readable
% % Using 0.95\columnwidth in conjunction with the


% Your paper must compile in PDF\LaTeX{}. Consequently, all your figures must be .jpg, .png, or .pdf. You may not use the .gif (the resolution is too low), .ps, or .eps file format for your figures.

% Figures, drawings, tables, and photographs should be placed throughout the paper on the page (or the subsequent page) where they are first discussed. Do not group them together at the end of the paper. If placed at the top of the paper, illustrations may run across both columns. Figures must not invade the top, bottom, or side margin areas. Figures must be inserted using the \textbackslash usepackage\{graphicx\}. Number figures sequentially, for example, figure 1, and so on. Do not use minipage to group figures.

% If you normally create your figures using pgfplots, please create the figures first, and then import them as pdfs with proper bounding boxes, as the bounding and trim boxes created by pfgplots are fragile and not valid.

% When you include your figures, you must crop them \textbf{outside} of \LaTeX{}. The command \textbackslash includegraphics*[clip=true, viewport 0 0 10 10]{...} might result in a PDF that looks great, but the image is \textbf{not really cropped.} The full image can reappear (and obscure whatever it is overlapping) when page numbers are applied or color space is standardized. Figures \ref{fig1}, and \ref{fig2} display some unwanted results that often occur.

% If your paper includes illustrations that are not compatible with PDF\TeX{} (such as .eps or .ps documents), you will need to convert them. The epstopdf package will usually work for eps files. You will need to convert your ps files to PDF in either case.

% \subsubsection {Figure Captions.}The illustration number and caption must appear \textit{under} the illustration. Labels and other text with the actual illustration must be at least nine-point type. However, the font and size of figure captions must be 10 point roman. Do not make them smaller, bold, or italic. (Individual words may be italicized if the context requires differentiation.)

% \subsection{Tables}

% \subsection{Tables}

% Tables should be presented in 10 point roman type. If necessary, they may be altered to 9 point type. You must not use \texttt{\textbackslash resizebox} or other commands that resize the entire table to make it smaller, because you can't control the final font size this way.
% If your table is too large you can use \texttt{\textbackslash setlength\{\textbackslash tabcolsep\}\{1mm\}} to compress the columns a bit or you can adapt the content (e.g.: reduce the decimal precision when presenting numbers, use shortened column titles, make some column duble-line to get it narrower).

% Tables that do not fit in a single column must be placed across double columns. If your table won't fit within the margins even when spanning both columns and using the above techniques, you must split it in two separate tables.

% \subsubsection {Table Captions.} The number and caption for your table must appear \textit{under} (not above) the table.  Additionally, the font and size of table captions must be 10 point roman and must be placed beneath the figure. Do not make them smaller, bold, or italic. (Individual words may be italicized if the context requires differentiation.)



% \subsubsection{Low-Resolution Bitmaps.}
% You may not use low-resolution (such as 72 dpi) screen-dumps and GIF files---these files contain so few pixels that they are always blurry, and illegible when printed. If they are color, they will become an indecipherable mess when converted to black and white. This is always the case with gif files, which should never be used. The resolution of screen dumps can be increased by reducing the print size of the original file while retaining the same number of pixels. You can also enlarge files by manipulating them in software such as PhotoShop. Your figures should be 300 dpi when incorporated into your document.

% \subsubsection{\LaTeX{} Overflow.}
% \LaTeX{} users please beware: \LaTeX{} will sometimes put portions of the figure or table or an equation in the margin. If this happens, you need to make the figure or table span both columns. If absolutely necessary, you may reduce the figure, or reformat the equation, or reconfigure the table.{ \bf Check your log file!} You must fix any overflow into the margin (that means no overfull boxes in \LaTeX{}). \textbf{Nothing is permitted to intrude into the margin or gutter.}


% \subsubsection{Using Color.}
% Use of color is restricted to figures only. It must be WACG 2.0 compliant. (That is, the contrast ratio must be greater than 4.5:1 no matter the font size.) It must be CMYK, NOT RGB. It may never be used for any portion of the text of your paper. The archival version of your paper will be printed in black and white and grayscale. The web version must be readable by persons with disabilities. Consequently, because conversion to grayscale can cause undesirable effects (red changes to black, yellow can disappear, and so forth), we strongly suggest you avoid placing color figures in your document. If you do include color figures, you must (1) use the CMYK (not RGB) colorspace and (2) be mindful of readers who may happen to have trouble distinguishing colors. Your paper must be decipherable without using color for distinction.

% \subsubsection{Drawings.}
% We suggest you use computer drawing software (such as Adobe Illustrator or, (if unavoidable), the drawing tools in Microsoft Word) to create your illustrations. Do not use Microsoft Publisher. These illustrations will look best if all line widths are uniform (half- to two-point in size), and you do not create labels over shaded areas. Shading should be 133 lines per inch if possible. Use Times Roman or Helvetica for all figure call-outs. \textbf{Do not use hairline width lines} --- be sure that the stroke width of all lines is at least .5 pt. Zero point lines will print on a laser printer, but will completely disappear on the high-resolution devices used by our printers.

% \subsubsection{Photographs and Images.}
% Photographs and other images should be in grayscale (color photographs will not reproduce well; for example, red tones will reproduce as black, yellow may turn to white, and so forth) and set to a minimum of 300 dpi. Do not prescreen images.

% \subsubsection{Resizing Graphics.}
% Resize your graphics \textbf{before} you include them with LaTeX. You may \textbf{not} use trim or clip options as part of your \textbackslash includegraphics command. Resize the media box of your PDF using a graphics program instead.

% \subsubsection{Fonts in Your Illustrations.}
% You must embed all fonts in your graphics before including them in your LaTeX document.

% \subsubsection{Algorithms.}
% Algorithms and/or programs are a special kind of figures. Like all illustrations, they should appear floated to the top (preferably) or bottom of the page. However, their caption should appear in the header, left-justified and enclosed between horizontal lines, as shown in Algorithm~\ref{alg:algorithm}. The algorithm body should be terminated with another horizontal line. It is up to the authors to decide whether to show line numbers or not, how to format comments, etc.

% In \LaTeX{} algorithms may be typeset using the {\tt algorithm} and {\tt algorithmic} packages, but you can also use one of the many other packages for the task.

% \begin{algorithm}[tb]
% \caption{Example algorithm}
% \label{alg:algorithm}
% \textbf{Input}: Your algorithm's input\\
% \textbf{Parameter}: Optional list of parameters\\
% \textbf{Output}: Your algorithm's output
% \begin{algorithmic}[1] %[1] enables line numbers
% \STATE Let $t=0$.
% \WHILE{condition}
% \STATE Do some action.
% \IF {conditional}
% \STATE Perform task A.
% \ELSE
% \STATE Perform task B.
% \ENDIF
% \ENDWHILE
% \STATE \textbf{return} solution
% \end{algorithmic}
% \end{algorithm}

% \subsubsection{Listings.}
% Listings are much like algorithms and programs. They should also appear floated to the top (preferably) or bottom of the page. Listing captions should appear in the header, left-justified and enclosed between horizontal lines as shown in Listing~\ref{lst:listing}. Terminate the body with another horizontal line and avoid any background color. Line numbers, if included, must appear within the text column.

% \begin{listing}[tb]%
% \caption{Example listing {\tt quicksort.hs}}%
% \label{lst:listing}%
% \begin{lstlisting}[language=Haskell]
% quicksort :: Ord a => [a] -> [a]
% quicksort []     = []
% quicksort (p:xs) = (quicksort lesser) ++ [p] ++ (quicksort greater)
% 	where
% 		lesser  = filter (< p) xs
% 		greater = filter (>= p) xs
% \end{lstlisting}
% \end{listing}

% \subsection{References}
% The AAAI style includes a set of definitions for use in formatting references with BibTeX. These definitions make the bibliography style fairly close to the ones  specified in the Reference Examples appendix below. To use these definitions, you also need the BibTeX style file ``aaai25.bst," available in the AAAI Author Kit on the AAAI web site. Then, at the end of your paper but before \textbackslash end{document}, you need to put the following lines:

% \begin{quote}
% \begin{small}
% \textbackslash bibliography\{bibfile1,bibfile2,...\}
% \end{small}
% \end{quote}

% Please note that the aaai25.sty class already sets the bibliographystyle for you, so you do not have to place any \textbackslash bibliographystyle command in the document yourselves. The aaai25.sty file is incompatible with the hyperref and navigator packages. If you use either, your references will be garbled and your paper will be returned to you.

% References may be the same size as surrounding text.
% However, in this section (only), you may reduce the size to {\em \textbackslash small} (9pt) if your paper exceeds the allowable number of pages. Making it any smaller than 9 point with 10 point linespacing, however, is not allowed.

% The list of files in the \textbackslash bibliography command should be the names of your BibTeX source files (that is, the .bib files referenced in your paper).

% The following commands are available for your use in citing references:
% \begin{quote}
% {\em \textbackslash cite:} Cites the given reference(s) with a full citation. This appears as ``(Author Year)'' for one reference, or ``(Author Year; Author Year)'' for multiple references.\smallskip\\
% {\em \textbackslash shortcite:} Cites the given reference(s) with just the year. This appears as ``(Year)'' for one reference, or ``(Year; Year)'' for multiple references.\smallskip\\
% {\em \textbackslash citeauthor:} Cites the given reference(s) with just the author name(s) and no parentheses.\smallskip\\
% {\em \textbackslash citeyear:} Cites the given reference(s) with just the date(s) and no parentheses.
% \end{quote}
% You may also use any of the \emph{natbib} citation commands.


% \section{Proofreading Your PDF}
% Please check all the pages of your PDF file. The most commonly forgotten element is the acknowledgements --- especially the correct grant number. Authors also commonly forget to add the metadata to the source, use the wrong reference style file, or don't follow the capitalization rules or comma placement for their author-title information properly. A final common problem is text (expecially equations) that runs into the margin. You will need to fix these common errors before submitting your file.

% \section{Improperly Formatted Files }
% In the past, AAAI has corrected improperly formatted files submitted by the authors. Unfortunately, this has become an increasingly burdensome expense that we can no longer absorb). Consequently, if your file is improperly formatted, it will be returned to you for correction.

% \section{Naming Your Electronic File}
% We require that you name your \LaTeX{} source file with the last name (family name) of the first author so that it can easily be differentiated from other submissions. Complete file-naming instructions will be provided to you in the submission instructions.

% \section{Submitting Your Electronic Files to AAAI}
% Instructions on paper submittal will be provided to you in your acceptance letter.

% \section{Inquiries}
% If you have any questions about the preparation or submission of your paper as instructed in this document, please contact AAAI Press at the address given below. If you have technical questions about implementation of the aaai style file, please contact an expert at your site. We do not provide technical support for \LaTeX{} or any other software package. To avoid problems, please keep your paper simple, and do not incorporate complicated macros and style files.

% \begin{quote}
% \noindent AAAI Press\\
% 1101 Pennsylvania Ave, NW Suite 300\\
% Washington, DC 20004 USA\\
% \textit{Telephone:} 1-202-360-4062\\
% \textit{E-mail:} See the submission instructions for your particular conference or event.
% \end{quote}

% \section{Additional Resources}
% \LaTeX{} is a difficult program to master. If you've used that software, and this document didn't help or some items were not explained clearly, we recommend you read Michael Shell's excellent document (testflow doc.txt V1.0a 2002/08/13) about obtaining correct PS/PDF output on \LaTeX{} systems. (It was written for another purpose, but it has general application as well). It is available at www.ctan.org in the tex-archive.

% \appendix
% \section{Reference Examples}
% \label{sec:reference_examples}

% \nobibliography*
% Formatted bibliographies should look like the following examples. You should use BibTeX to generate the references. Missing fields are unacceptable when compiling references, and usually indicate that you are using the wrong type of entry (BibTeX class).

% \paragraph{Book with multiple authors~\nocite{em:86}} Use the \texttt{@book} class.\\[.2em]
% \bibentry{em:86}.

% \paragraph{Journal and magazine articles~\nocite{r:80, hcr:83}} Use the \texttt{@article} class.\\[.2em]
% \bibentry{r:80}.\\[.2em]
% \bibentry{hcr:83}.

% \paragraph{Proceedings paper published by a society, press or publisher~\nocite{c:83, c:84}} Use the \texttt{@inproceedings} class. You may abbreviate the \emph{booktitle} field, but make sure that the conference edition is clear.\\[.2em]
% \bibentry{c:84}.\\[.2em]
% \bibentry{c:83}.

% \paragraph{University technical report~\nocite{r:86}} Use the \texttt{@techreport} class.\\[.2em]
% \bibentry{r:86}.

% \paragraph{Dissertation or thesis~\nocite{c:79}} Use the \texttt{@phdthesis} class.\\[.2em]
% \bibentry{c:79}.

% \paragraph{Forthcoming publication~\nocite{c:21}} Use the \texttt{@misc} class with a \texttt{note="Forthcoming"} annotation.
% \begin{quote}
% \begin{footnotesize}
% \begin{verbatim}
% @misc(key,
%   [...]
%   note="Forthcoming",
% )
% \end{verbatim}
% \end{footnotesize}
% \end{quote}
% \bibentry{c:21}.

% \paragraph{ArXiv paper~\nocite{c:22}} Fetch the BibTeX entry from the "Export Bibtex Citation" link in the arXiv website. Notice it uses the \texttt{@misc} class instead of the \texttt{@article} one, and that it includes the \texttt{eprint} and \texttt{archivePrefix} keys.
% \begin{quote}
% \begin{footnotesize}
% \begin{verbatim}
% @misc(key,
%   [...]
%   eprint="xxxx.yyyy",
%   archivePrefix="arXiv",
% )
% \end{verbatim}
% \end{footnotesize}
% \end{quote}
% \bibentry{c:22}.

% \paragraph{Website or online resource~\nocite{c:23}} Use the \texttt{@misc} class. Add the url in the \texttt{howpublished} field and the date of access in the \texttt{note} field:
% \begin{quote}
% \begin{footnotesize}
% \begin{verbatim}
% @misc(key,
%   [...]
%   howpublished="\url{http://...}",
%   note="Accessed: YYYY-mm-dd",
% )
% \end{verbatim}
% \end{footnotesize}
% \end{quote}
% \bibentry{c:23}.

% \vspace{.2em}
% For the most up to date version of the AAAI reference style, please consult the \textit{AI Magazine} Author Guidelines at \url{https://aaai.org/ojs/index.php/aimagazine/about/submissions#authorGuidelines}

% \section{Acknowledgments}
% AAAI is especially grateful to Peter Patel Schneider for his work in implementing the original aaai.sty file, liberally using the ideas of other style hackers, including Barbara Beeton. We also acknowledge with thanks the work of George Ferguson for his guide to using the style and BibTeX files --- which has been incorporated into this document --- and Hans Guesgen, who provided several timely modifications, as well as the many others who have, from time to time, sent in suggestions on improvements to the AAAI style. We are especially grateful to Francisco Cruz, Marc Pujol-Gonzalez, and Mico Loretan for the improvements to the Bib\TeX{} and \LaTeX{} files made in 2020.

% The preparation of the \LaTeX{} and Bib\TeX{} files that implement these instructions was supported by Schlumberger Palo Alto Research, AT\&T Bell Laboratories, Morgan Kaufmann Publishers, The Live Oak Press, LLC, and AAAI Press. Bibliography style changes were added by Sunil Issar. \verb+\+pubnote was added by J. Scott Penberthy. George Ferguson added support for printing the AAAI copyright slug. Additional changes to aaai25.sty and aaai25.bst have been made by Francisco Cruz and Marc Pujol-Gonzalez.

% \bigskip
% \noindent Thank you for reading these instructions carefully. We look forward to receiving your electronic files!

% \bibliography{aaai25}

\end{document}
