

\section{Evaluation}
\label{sec: user study}

Based on the students' data collected from our workshop, we explored the simulation performance in not only straightforward accuracy comparison, but also the dynamic patterns of students' learning performance at fine-grained levels.



\subsection{Simulation Settings}
The simulation settings were similar with those in the EduAgent dataset. 
We split the dataset into training and testing set by following a individual-wise manner with 0.7 ratio. 
We set the first five questions as past questions of the student history and other questions as future questions for prediction.
Both of the model input and output were the same as the EduAgent simulation experiment (Fig. \ref{framework:prompt} and Fig. \ref{framework:finetune}).
For LLMs-based models, we obtained the results with or without our TIR module for both prompting-based models (standard and CoT prompt) and finetuning-based models (BertKT), which were compared with deep learning models. 
All LLMs-based models used both GPT-4o and GPT-4o mini. 


\begin{figure*}
\centering
\includegraphics[width=1\linewidth]{figures/model_compare_student.pdf}
\caption{
\mytextcolor{
Heatmap to show the average simulation accuracy (each cell) for each individual student using each model.}}
\Description{
This figure shows the average simulation accuracy for each individual student using each model.
We found a significant effect of the model on simulation accuracy ($F(5, 85) = 5.16, \, p = 0.0004, \, \eta_{p}^{2} = 0.18$), indicating a large effect size. For pairwise comparisons, significant differences were observed between the following model pairs:
%
AKT vs. DKT: $t(17) = 2.29, \, p = 0.035, \, \eta_{p}^{2} = 0.16$.
BertKT with TIR vs. ATKT: $t(17) = 3.42, \, p = 0.003, \, \eta_{p}^{2} = 0.24$.
BertKT with TIR vs. DKT: $t(17) = 3.24, \, p = 0.005, \, \eta_{p}^{2} = 0.30$.
BertKT with TIR vs. DKVMN: $t(17) = 3.83, \, p = 0.001, \, \eta_{p}^{2} = 0.24$.
BertKT with TIR vs. SimpleKT: $t(17) = 2.45, \, p = 0.025, \, \eta_{p}^{2} = 0.14$.
These results indicate that BertKT (with TIR) exhibited significantly better performance compared to most deep learning models. Although we did not find significance between the BertKT (with TIR) and AKT, it still showed better simulation accuracy of BertKT (with TIR) than AKT for most individual students.
}
\label{model compare student}
\end{figure*}

\mytextcolor{
\subsection{TIR Makes LLMs Superior than Deep Learning Models}
We first compared the accuracy and F1-score of the simulated performance of various models by comparing with the real students' performance. 
As depicted in Fig. \ref{model compare accuracy}(a), without the integration of our TIR module, the best model is SimpleKT with 0.656 accuracy, which was better than all LLMs-based models. However, with our TIR module, all LLMs-based models increased the simulation accuracy and all had larger accuracy than the SimpleKT model. The F1 score results were similar as well in Fig. \ref{model compare accuracy}(c), i.e. all TIR-augmented LLMs held better F1 score than all deep learning models. \mytextcolor{These results are encouraging because the selected deep learning models are proven educational models widely accepted in the educational domain \cite{10.1145/3394486.3403282,DBLP:conf/iclr/0001L0H023,10.1145/3474085.3475554,piech2015deepknowledgetracing,zhang2017dynamickeyvaluememorynetworks} to inform teaching practices and support adaptive learning strategies \cite{scholtz2021systematic}.
Therefore, the superior predictive capability of our model indicates a strong potential for real-world applicability.}
To further demonstrate our model's impact, we selected BertKT with TIR as an example to perform statistical analysis by comparing with the five deep learning models in individual-level, lecture-level, and question-level. The effect of different models on student simulation performance was measured by repeated-measures ANOVA with paired t-tests for pair-wise comparisons.
}

% \subsection{Benchmarking Results}
% We first compared the absolute accuracy and F1-score of the simulated performance of various models by comparing with the real students' performance. 
% As depicted in Fig. \ref{model compare accuracy}(a), without the integration of our TIR module, the best model is SimpleKT with 0.656 accuracy, which was better than all LLMs-based models. However, with our TIR module, all LLMs-based models increased the simulation accuracy and all had larger accuracy than the SimpleKT model. The F1 score results were similar as well in Fig. \ref{model compare accuracy}(c), i.e. all TIR-integrated LLMs-based models held better F1 score than all deep learning models. 
% Furthermore, to showcase the impact of the integration of our TIR module, for BertKT models, we showed the simulation performance (accuracy, F1 score) in individual student level (Fig. \ref{individual student accuracy}),  question level (Fig. \ref{question-level accuracy}),lecture level (Fig. \ref{lecture-level accuracy}), with or without TIR module. All levels of analysis showed that the integration of the TIR module could improve the simulation performance in most cases.
% These results demonstrate that the integration of our TIR module apparently improved the student learning performance simulation of LLMs-based models. 





\mytextcolor{
\textbf{Individual-Level}: We calculated the average simulation accuracy for each individual student, as depicted in Fig. \ref{model compare student}.
We found a significant effect of the model on simulation accuracy ($F(5, 85) = 5.16, \, p = 0.0004, \, \eta_{p}^{2} = 0.18$), indicating a large effect size. For pairwise comparisons, significant differences were observed between the following model pairs:
%
\begin{itemize}
    \item AKT vs. DKT: $t(17) = 2.29, \, p = 0.035, \, \eta_{p}^{2} = 0.16$.
    \item BertKT with TIR vs. ATKT: $t(17) = 3.42, \, p = 0.003, \, \eta_{p}^{2} = 0.24$.
    \item BertKT with TIR vs. DKT: $t(17) = 3.24, \, p = 0.005, \, \eta_{p}^{2} = 0.30$.
    \item BertKT with TIR vs. DKVMN: $t(17) = 3.83, \, p = 0.001, \, \eta_{p}^{2} = 0.24$.
    \item BertKT with TIR vs. SimpleKT: $t(17) = 2.45, \, p = 0.025, \, \eta_{p}^{2} = 0.14$.
\end{itemize}
These results indicate that BertKT (with TIR) exhibited significantly better performance compared to most deep learning models. Although we did not find significance between the BertKT (with TIR) and AKT, Fig. \ref{model compare student} still showed better simulation accuracy of BertKT (with TIR) than AKT for most individual students.
}


\mytextcolor{
\textbf{Lecture-Level}: We then calculated the average simulation accuracy for each specific course lecture ID, as depicted in Fig. \ref{model compare lecture}. Results showed a significant effect of the model on simulation accuracy ($F(5, 55) = 4.53, \, p = 0.002, \, \eta_{p}^{2} = 0.20$), indicating a large effect size. Significant differences were observed between the following model pairs:
%
\begin{itemize}
    \item AKT vs. DKVMN: $t(11) = 2.49, \, p = 0.03, \, \eta_{p}^{2} = 0.21$.
    \item BertKT with TIR vs. ATKT: $t(11) = 3.06, \, p = 0.01, \, \eta_{p}^{2} = 0.28$.
    \item BertKT with TIR vs. DKT: $t(11) = 2.91, \,p = 0.014, \, \eta_{p}^{2} = 0.27$.
    \item BertKT with TIR vs. DKVMN: $t(11) = 4.27, \, p = 0.001, \, \eta_{p}^{2} = 0.35$.
\end{itemize} 
Similar with individual-level results, these results indicate the superiority of the BertKT (with TIR) than these deep learning models. Although we did not find significance between the BertKT (with TIR) and AKT/SimpleKT, Fig. \ref{model compare lecture} still showed better simulation accuracy of BertKT (with TIR) than AKT/SimpleKT for most lectures.
}





\mytextcolor{
\textbf{Question-Level}: We then calculated the average simulation accuracy for each specific question ID in post-test, as depicted in Fig. \ref{model compare question}. 
Results did not find a significant effect of the model on simulation accuracy ($F(5, 30) = 1.09, \, p = 0.387, \, \eta_{p}^{2} = 0.14$). However, significant differences were observed between the following model pairs:
%
\begin{itemize}
    \item BertKT with TIR vs. DKVMN: $t(6) = 3.69, \, p = 0.01, \, \eta_{p}^{2} = 0.46$.
    \item BertKT with TIR vs. simpleKT: $t(6) = 2.59, \, p = 0.041, \, \eta_{p}^{2} = 0.32$.
\end{itemize}
Although we did not find significance between the BertKT (with TIR) and AKT/ATKT/DKT, Fig. \ref{model compare question} still showed better simulation accuracy of BertKT (with TIR) than AKT/ATKT/DKT for most question IDs.
}




%xyz: The title here reads weird (revised)
\subsection{TIR Enhances Model Learning Efficiency}
Although the training of BertKT+TIR model used all training students, it is worth noting that, for other prompt-based models (Standard and CoT), we only used four example students as the contextual example demonstration instead of all students in the training set. 
However, after integrating our TIR module, both of prompt-based models (Standard and CoT) achieved better simulation performance than deep learning models (Fig. \ref{model compare accuracy}), which used all students in the training set for model training. This demonstrates that our TIR module could enhance the exploitation efficiency of prompt-based models to achieve comparable or even more realistic student simulation within more limited training data.

\begin{figure}
\centering
\includegraphics[width=1\linewidth]{figures/model_compare_lecture.pdf}
\caption{
\mytextcolor{Heatmap to show the average simulation accuracy (each cell) for each specific lecture using each model.}}
\Description{
This figure shows the average simulation accuracy for each specific course lecture ID. Results showed a significant effect of the model on simulation accuracy ($F(5, 55) = 4.53, \, p = 0.002, \, \eta_{p}^{2} = 0.20$), indicating a large effect size. Significant differences were observed between the following model pairs:
%
AKT vs. DKVMN: $t(11) = 2.49, \, p = 0.03, \, \eta_{p}^{2} = 0.21$.
BertKT with TIR vs. ATKT: $t(11) = 3.06, \, p = 0.01, \, \eta_{p}^{2} = 0.28$.
BertKT with TIR vs. DKT: $t(11) = 2.91, \,p = 0.014, \, \eta_{p}^{2} = 0.27$.
BertKT with TIR vs. DKVMN: $t(11) = 4.27, \, p = 0.001, \, \eta_{p}^{2} = 0.35$.
Similar with individual-level results, these results indicate the superiority of the BertKT (with TIR) than these deep learning models. Although we did not find significance between the BertKT (with TIR) and AKT/SimpleKT, it still showed better simulation accuracy of BertKT (with TIR) than AKT/SimpleKT for most lectures.
}
\label{model compare lecture}
\end{figure}





\subsection{TIR Empowers Smaller LLMs}
We also compared the simulation performance using both GPT-4o and GPT-4o mini.
We found that the integration of our TIR module increased all of the GPT-4o mini based simulation models (Standard, CoT, BertKT), which were even better than the simulation models using GPT-4o without the TIR module. Note that GPT-4o mini is a much smaller model than GPT-4o\footnote{https://openai.com/index/gpt-4o-mini-advancing-cost-efficient-intelligence/}. Without TIR, GPT-4o had apparently better simulation performance than GPT-4o mini in Standard and CoT models, as depicted in Fig. \ref{model compare accuracy}(b). However, after integrating the TIR module, both Standard and CoT models in GPT-4o mini outperformed GPT-4o in an obvious margin (Fig. \ref{model compare accuracy}(b)). These results demonstrate that the TIR module could improve the smaller LLMs to learn from example students in the training set more effectively. As a result, smaller LLMs could achieve comparable or even better performance, eliminating the need of using larger size LLMs. 





\subsection{TIR Captures Individual Differences Better}
We then examined whether the models could capture the individual differences and correlation among simulated and real students. Specifically, we used the BertKT with or without the TIR module \mytextcolor{(baseline)} for simulation and compared with the 
label (real students' groundtruth). 
%
This was quantitatively measured by the Pearson correlation between the simulated students' test accuracy sequence along with student IDs and the real students'.    
As depicted in Fig. \ref{individual student correlation}, we found that the integration of the TIR module better captured the correlation between simulated and real students regarding the learning performance sequence along with student IDs than the no TIR case and apparently improved the Pearson correlation from $r=0.02$ (No TIR) to $r=0.42$ (With TIR).
These results demonstrate that our TIR module enabled more realistic simulation by better capturing the individual differences of student learning performance.

\begin{figure}
\centering
\includegraphics[width=1\linewidth]{figures/model_compare_question.pdf}
\caption{\mytextcolor{Heatmap to show the average simulation accuracy (each cell) for each post-test question ID using each model.}}
\Description{
The figure shows the average simulation accuracy for each specific question ID in post-test. 
Results did not find a significant effect of the model on simulation accuracy ($F(5, 30) = 1.09, \, p = 0.387, \, \eta_{p}^{2} = 0.14$). However, significant differences were observed between the following model pairs:
%
BertKT with TIR vs. DKVMN: $t(6) = 3.69, \, p = 0.01, \, \eta_{p}^{2} = 0.46$.
BertKT with TIR vs. simpleKT: $t(6) = 2.59, \, p = 0.041, \, \eta_{p}^{2} = 0.32$.
Although we did not find significance between the BertKT (with TIR) and AKT/ATKT/DKT, it still showed better simulation accuracy of BertKT (with TIR) than AKT/ATKT/DKT for most question IDs.
}
\label{model compare question}
\end{figure}

\mytextcolor{
We further checked the statistical difference of the average simulation accuracy per student with or without the TIR module (baseline).
The normality of the differences between both was assessed using the Shapiro-Wilk test, which indicated no significant deviation from normality ($W = 0.9235, \, p = 0.1493$; df = 17). A paired t-test was then conducted to evaluate the impact of the TIR module on prediction performance. The results showed a statistically significant improvement in accuracy with the TIR module compared to the baseline ($t = 2.4139,\, p = 0.0273$; df = 17). A Bland-Altman analysis revealed a mean difference (bias) of 0.0881 (95\% CI: 0.0186 to 0.1577), with the limits of agreement ranging from -0.2069 (95\% CI: -0.5100 to 0.0962) to 0.3832 (95\% CI: 0.0800 to 0.6863). 
The Bland-Altman plot (Fig. \ref{individual student correlation}(d)) visualizes these findings, showing the mean difference as a dashed red line and the limits of agreement as dashed blue lines. The scatter of points around the mean difference is relatively consistent, suggesting that the agreement between the two models does not vary substantially across the range of predicted accuracy values. 
These results indicate a consistent positive effect of the TIR module on prediction performance, while maintaining acceptable levels of agreement with the baseline model.
}

\begin{figure*}
\centering
\includegraphics[width=1\linewidth]{figures/correlation_student_crop.pdf}
\caption{
\mytextcolor{
Individual-level results: simulations using BertKT with and without TIR across different students.
(a). Correlation between student ID and simulated/real post-test question answer accuracy (vertical bar: standard deviation). (b,c,d,e). The distribution of simulation accuracy differences (b), boxplot (c), Bland-Altman plot to show the mean differences (d), and barplot (e) (error bar: 95\% confidence interval) between two models. (f). Average simulation accuracy and F1 score for each individual student using two models.
}}
\Description{
This figure contains six subplots (labeled a–f) presenting the performance comparison of BertKT with and without TIR for question answer accuracy in simulations. The subplots highlight various statistical and comparative insights based on the dataset.
%
(a) A line plot showing "Question Answer Accuracy" for different student IDs. Three sets of data are represented: the label (ground truth), simulations with TIR (orange solid line with stars), and simulations without TIR (green solid line with stars). Each point represents the average accuracy for a student, with vertical bars indicating the standard deviation. Pearson correlation coefficients for simulations with TIR (r = 0.42) and without TIR (r = 0.02) against the labels are noted on the plot.
%
(b) A histogram showing the distribution of differences between TIR and baseline (no TIR). The x-axis represents the difference in accuracy, while the y-axis represents the frequency of occurrences. A smooth curve overlays the histogram, indicating the approximate probability density. It shows that the data is normally distributed.
%
(c) A boxplot comparing "Accuracy per Student" for simulations with TIR and without TIR. The boxes represent the interquartile range, with the median shown as a line within each box, and whiskers extending to represent the range. It shows that the TIR case has better accuracy.
%
(d) A Bland-Altman plot visualizing the agreement between TIR and baseline performance. The x-axis represents the mean of the two methods (baseline and TIR), while the y-axis shows the difference. Dotted lines indicate the mean difference, upper, and lower limits of agreement (95\% confidence interval). Points are scattered across the plot to show individual observations.
%
(e) A bar chart showing mean performance for simulations with TIR and without TIR, along with error bars representing the 95\% confidence interval. It shows that the TIR case has better accuracy.
%
(f) Two grouped bar plots comparing simulation accuracy (left) and F1 score (right) for each student ID. The bars are color-coded: blue for simulations with TIR and orange for those without TIR. Each student ID is labeled on the y-axis. It shows that the TIR case has better accuracy.
}
\label{individual student correlation}
\end{figure*}



\subsection{TIR Captures Lecture Correlation Better}



% \mytextcolor{The statistical analysis of the data was conducted to assess the impact of the TIR module on student learning performance. The Shapiro-Wilk test for normality showed that the distribution of differences between the baseline and TIR models was not significantly different from a normal distribution (W-statistic = 0.9736, p-value = 0.9444). Given the normal distribution, a paired t-test was performed, yielding a t-statistic of 2.5173 and a p-value of 0.0286, indicating a statistically significant improvement in performance with the TIR module. The degrees of freedom for the paired t-test were 11. Additionally, a Bland-Altman analysis was conducted to assess the agreement between the two models. The mean difference (bias) was found to be 0.0764, with a 95\% confidence interval (CI) of 0.0195 to 0.1334. The limits of agreement were -0.1209 (95\% CI: -0.3263 to 0.0845) for the lower limit and 0.2738 (95\% CI: 0.0684 to 0.4792) for the upper limit, suggesting variability in the differences between the two models, but the positive mean difference indicates that the TIR model generally outperforms the baseline.}

We then examined whether the models could capture the lecture correlation and differences. Specifically, we still used the BertKT with or without the TIR module \mytextcolor{(baseline)} for simulation and compared with the 
%xyz: What's the label? (revised)
label (real students' groundtruth). Since different lectures had their own difficulty, students therefore had different learning performance (post-test question accuracy) across different lectures. Therefore, by comparing the trend of simulated and real students' learning performance along with the lectures, we could see whether the simulation models could capture such variations of lecture difficulty and cross-lecture correlation. 
%
This trend was quantitatively measured by the Pearson correlation between the simulated students'  test accuracy sequence along with lectures and the real students' sequence.    
As depicted in Fig. \ref{lecture correlation}(a), we found that the integration of the TIR module better captured the correlation between simulated and real students regarding the learning performance sequence along with lectures than the no TIR case and apparently improved the Pearson correlation from $r=0.42$ (No TIR) to $r=0.52$ (With TIR).
For more intuitive visualization in individual students, we showed the average question answering accuracy in each lecture for each specific simulated and real student, as depicted in Fig. \ref{lecture correlation}(b,c,d). This visualization also revealed larger similarity between simulated students (with TIR) and real students, compared with the simulation similarity without TIR.
These results demonstrate that our TIR module enables more realistic simulation by better capturing the lecture correlation in student learning performance.

\begin{figure*}
\centering
\includegraphics[width=1\linewidth]{figures/aggre_lecture.pdf}
\caption{
\mytextcolor{
Lecture-level results: simulations using BertKT with and without TIR across different lecture IDs.
(a). Correlation between lecture ID and simulated/real post-test question answer accuracy (vertical bar: standard deviation). (b)(c)(d). Heatmaps of label, BertKT with TIR, and BertKT without TIR question answer accuracy across all students and lectures. Each cell shows the average question answer accuracy of a student answering questions in a specific lecture. Darker color represents higher question answer accuracy. (e,f,g,h). The distribution of simulation accuracy differences (e), boxplot (f), Bland-Altman plot to show the mean differences (g), and barplot (h) (error bar: 95\% confidence interval) between two models. (i). Average simulation accuracy and F1 score for each lecture ID using two models.
}}
\Description{
This figure contains nine subplots (labeled a–i).
(a) shows Question Answer Accuracy Across Lectures: This line plot compares the question answer accuracy of BertKT with TIR (orange line with stars), without TIR (green line with triangles), and the label trend (dotted line) across 12 lectures. Vertical bars represent the standard deviation. The Pearson correlation coefficients are noted: 0.52 (with TIR) and 0.42 (without TIR).
(a) shows that BertKT with TIR more closely aligns with the label accuracy trend compared to the baseline (no TIR), demonstrating higher consistency. The Pearson correlation confirms a stronger association with TIR integration.
(b), (c), (d) show the Heatmaps of Accuracy per Lecture and Student:
Three heatmaps visualize question answer accuracy for the label (b), with TIR (c), and without TIR (d) across 12 lectures and all students (u1 to u76). Darker colors represent higher accuracy.
They show that TIR integration produces patterns closer to the label accuracy heatmap. Without TIR, the accuracy distribution appears more dispersed, indicating weaker alignment with ground truth.
(e) shows the Distribution of Differences (TIR - Baseline):
It shows a histogram showing the distribution of accuracy differences between TIR and baseline (no TIR). The plot includes a smooth density curve for visualizing probability. It shows that the difference is normally distributed.
(f) shows the Performance Comparison by Model (Boxplot):
It presents a boxplot comparing the accuracy per lecture for simulations with and without TIR. The median and interquartile ranges are shown, with whiskers for variability.
It shows that BertKT with TIR achieves consistently higher median accuracy and a narrower interquartile range, signifying both improved and more stable performance.
(g) is a Bland-Altman Plot (Agreement Between Methods):
This plot assesses the agreement between baseline and TIR methods. The x-axis represents the mean accuracy of both methods, and the y-axis shows their difference. Dotted lines indicate the mean difference and the limits of agreement.
It reveals that most points lie within the limits of agreement, indicating reasonable consistency between the methods. However, the positive mean difference suggests that TIR consistently outperforms the baseline.
(h) shows the Mean Performance With Error Bars:
It presents a bar chart compares the mean accuracy across all lectures for TIR and baseline methods, with error bars representing the 95\% confidence interval.
It shows that BertKT with TIR exhibits higher mean accuracy and less overlap between confidence intervals, suggesting a statistically significant improvement.
(i) shows the Simulation Accuracy and F1 Score Comparison (Grouped Bar Charts):
It presents Two bar charts compare simulation accuracy and F1 scores across 12 lectures for TIR and baseline methods.
It shows that TIR achieves higher accuracy and F1 scores in most lectures, highlighting its efficacy. 
}
\label{lecture correlation}
\end{figure*}



\mytextcolor{
We further checked the statistical difference of the average simulation accuracy per lecture with or without the TIR module. The normality of the accuracy differences between them was evaluated using the Shapiro-Wilk test, which indicated no significant deviation from normality ($W = 0.9736, \, p = 0.9444$; df=11). A paired t-test was then performed to assess the impact of the TIR module on simulation performance. The analysis revealed a statistically significant improvement in accuracy with the TIR module compared to the no TIR case ($t = 2.5173, \, p = 0.0286$; df=11). Bland-Altman analysis (Fig. \ref{lecture correlation}(g)) showed a mean difference (bias) of 0.0764 (95\% CI: 0.0195 to 0.1334), with limits of agreement ranging from -0.1209 (95\% CI: -0.3263 to 0.0845) to 0.2738 (95\% CI: 0.0684 to 0.4792). 
% The Bland-Altman plot   visualizes these findings, showing the mean difference as a dashed red line and the limits of agreement as dashed blue lines. The scatter of points around the mean difference is relatively consistent, suggesting that the agreement between the two models does not vary substantially across the range of predicted accuracy values. 
These findings suggest that the TIR module consistently enhances prediction performance while demonstrating acceptable agreement with the baseline model.
}



\subsection{TIR Captures Question Differences Better}
Moreover, we further explored whether the models could capture the different questions' correlation using the BertKT with or without the TIR module. Different questions corresponded to specific knowledge concepts of course materials. Therefore, by comparing the trend of simulated and real students' test accuracy along with different questions, we could see whether the simulation models could capture students' learning performance across fine-grained and varying knowledge concepts. 
This trend was also quantitatively measured by the Pearson correlation between the simulated student question accuracy sequence along with question ID and the real students'.    
As depicted in Fig. \ref{question correlation}(a), we found that the integration of the TIR module better captured the correlation with question ID compared with real cases (label) than the no TIR case and apparently improved the Pearson correlation from $r=-0.50$ (No TIR) to $r=0.37$ (With TIR).
For more intuitive visualization in individual students, we showed the average question answering accuracy in each question for each specific simulated and real student, as depicted in Fig. \ref{question correlation}(b,c,d). This visualization also revealed larger similarity between simulated students (with TIR) and real students, compared with the simulation similarity without TIR.
These results demonstrate that our TIR module enables more realistic simulation by better capturing the question correlation (i.e. knowledge concept correlation) in student learning performance.

\mytextcolor{
We then checked the statistical difference of the average simulation accuracy per question with or without the TIR module (baseline).
The normality of the differences between both was assessed using the Shapiro-Wilk test, which indicated a significant deviation from normality ($W = 0.7451,\, p = 0.0113$; df = 6). Therefore, a Wilcoxon signed-rank test (instead of a paired t-test) was performed to evaluate the impact of the TIR module on prediction performance. The test revealed a statistically significant improvement in accuracy with the TIR module compared to the baseline ($p = 0.0277$; df = 6). A Bland-Altman analysis (Fig. \ref{question correlation}(g)) showed a mean difference (bias) of 0.1437 (95\% CI: 0.0306 to 0.2568), with the limits of agreement ranging from -0.1555 (95\% CI: -0.4754 to 0.1644) to 0.4429 (95\% CI: 0.1230 to 0.7628). These results demonstrate a significant positive effect of the TIR module on prediction performance, with an acceptable level of agreement between the two models.
}

\begin{figure*}
\centering
\includegraphics[width=1\linewidth]{figures/aggre_question.pdf}
\caption{
\mytextcolor{
Question-level results: simulations using BertKT with and without TIR across different post-test question IDs.
(a). Correlation between question ID and simulated/real post-test question answer accuracy (vertical bar: standard deviation). (b)(c)(d). Heatmaps of label, BertKT with TIR, and BertKT without TIR question answer accuracy across all students and questions. Each cell shows the average question answer accuracy of a student answering a specific question. Darker color represents higher question answer accuracy. (e,f,g,h). The distribution of simulation accuracy differences (e), boxplot (f), Bland-Altman plot to show the mean differences (g), and barplot (h) (error bar: 95\% confidence interval) between two models. (i). Average simulation accuracy and F1 score for each post-test question ID using two models.
}}
\Description{
This figure contains nine subplots (labeled a–i).
(a) shows Question Answer Accuracy Across Question IDs: This line plot compares the question answer accuracy of BertKT with TIR (orange line with stars), without TIR (green line with triangles), and the label trend (dotted line) across 7 question IDs. Vertical bars represent the standard deviation. The Pearson correlation coefficients are noted: 0.37 (with TIR) and -0.50 (without TIR).
(a) shows that BertKT with TIR more closely aligns with the label accuracy trend compared to the baseline (no TIR), demonstrating higher consistency. The Pearson correlation confirms a stronger association with TIR integration.
(b), (c), (d) show the Heatmaps of Accuracy per Question ID and Student:
Three heatmaps visualize question answer accuracy for the label (b), with TIR (c), and without TIR (d) across 7 questions and all students (u1 to u76). Darker colors represent higher accuracy.
They show that TIR integration produces patterns closer to the label accuracy heatmap. Without TIR, the accuracy distribution appears more dispersed, indicating weaker alignment with ground truth.
(e) shows the Distribution of Differences (TIR - Baseline):
It shows a histogram showing the distribution of accuracy differences between TIR and baseline (no TIR). The plot includes a smooth density curve for visualizing probability. It shows that the difference is not normally distributed.
(f) shows the Performance Comparison by Model (Boxplot):
It presents a boxplot comparing the accuracy per question ID for simulations with and without TIR. The median and interquartile ranges are shown, with whiskers for variability.
It shows that BertKT with TIR achieves consistently higher median accuracy and a narrower interquartile range, signifying both improved and more stable performance.
(g) is a Bland-Altman Plot (Agreement Between Methods):
This plot assesses the agreement between baseline and TIR methods. The x-axis represents the mean accuracy of both methods, and the y-axis shows their difference. Dotted lines indicate the mean difference and the limits of agreement.
It reveals that most points lie within the limits of agreement, indicating reasonable consistency between the methods. However, the positive mean difference suggests that TIR consistently outperforms the baseline.
(h) shows the Mean Performance With Error Bars:
It presents a bar chart compares the mean accuracy across all question IDs for TIR and baseline methods, with error bars representing the 95\% confidence interval.
It shows that BertKT with TIR exhibits higher mean accuracy and less overlap between confidence intervals, suggesting a statistically significant improvement.
(i) shows the Simulation Accuracy and F1 Score Comparison (Grouped Bar Charts):
It presents Two bar charts compare simulation accuracy and F1 scores across 7 question IDs for TIR and baseline methods.
It shows that TIR achieves higher accuracy and F1 scores in most question IDs, highlighting its efficacy. 
}
\label{question correlation}
\end{figure*}


\subsection{Dynamism of Skill Levels in Learning Path}
\label{subsec: dynamism}
Furthermore, we explored whether the simulation captured the dynamism of students' skill levels in the learning path. Here the learning path referred to the chronological learning process from the first slide to the last slide in the lecture. 
%As mentioned before, each lecture was delivered by an instructor who taught the course based on slides we prepared in advance. 
In our online education system, students' skill levels were represented by the average question answering accuracy where the questions were corresponding to a specific slide. This enabled us to measure to what extent the students mastered the knowledge concepts per slide.         
We still used the BertKT with or without the TIR module for simulation and compared with the label. Then we compared the trend of simulated and real students' skill levels along with the slide ID. 
As depicted in Fig. \ref{slide course correlation}(a), we found that the integration of the TIR module better captured the dynamism of skill levels in the whole learning path from the first slide to the last slide compared with real cases (label) than the no TIR case.
For more intuitive visualization in individual students, we also showed the average question answering accuracy in each slide for each specific simulated and real student, as depicted in Fig. \ref{slide course correlation}(b,c,d). This visualization also revealed larger similarity between simulated students (with TIR) and real students, compared with the simulation similarity without TIR.
This trend was also quantitatively measured by the Pearson correlation between the simulated skill level sequence along with slide ID and the real student sequence along with slide ID. However, since different lectures had different slides, we analyzed the Pearson correlation in each lecture. As depicted in Fig. \ref{slide course correlation}(e), the integration of our TIR module captured better correlation in most lectures.
These results demonstrate that our TIR module enables more realistic simulation by better capturing the dynamism of students' skill levels across the learning path.

\begin{figure*}
\centering
\includegraphics[width=1\linewidth]{figures/aggre_slide.pdf}
\caption{\mytextcolor{(a). Question answer accuracy of simulations using BertKT with and without TIR against the labels on our newly collected dataset across questions related to each slide. Each star point depicts the average question answer accuracy of questions related to that slide while the vertical bar represents the standard deviation. The dotted line represents the label question answer accuracy trend. The solid lines represent the predicted question answer accuracy trend. (b)(c)(d). Heatmaps of label, BertKT with TIR, and BertKT without TIR question answer accuracy across all students and slides. Each cell shows the average student's answer accuracy of questions that are related to the slide. Darker color represents larger question answer accuracy. (e). Pearson correlation between the simulated (BertKT with or without TIR) skill level sequence along with slide ID and the real student sequence along with slide ID in each lecture on our newly collected dataset.}}
\Description{Figure (a) shows the question answer accuracy of simulations using BertKT with and without TIR against the labels on our newly collected dataset across questions related to each slide. It shows that the integration of the TIR module better captured the dynamism of skill levels in the whole learning path from the first slide to the last slide compared with real cases (label) than the no TIR case. Figure (b,c,d) show heatmaps of label, BertKT with TIR, and BertKT without TIR question answer accuracy across all students and slides. Each cell shows the average question answer accuracy related to a specific slide of a student. They show larger similarity between simulated students (with TIR) and real students, compared with the simulation similarity without TIR. Figure (e) shows the Pearson correlation between the simulated (BertKT with or without TIR) skill level sequence along with slide ID and the real student sequence along with slide ID in each lecture on our newly collected dataset. It shows that the integration of our TIR module captured better such correlation in most lectures.}
\label{slide course correlation}
\end{figure*}


\subsection{Fine-Grained Inter-Student Correlation}
One important aspect of contextual simulation was to not only capture the individual differences in learning but also the individual correlations in the same course. Fig. \ref{graph} showed the inter-student correlation for both simulated students (BertKT with or without TIR) and real students. Each node represented one student and two nodes were connected if both students took the same lecture. The color depth of each node represented the average question answering correctness of all lectures that the individual student attended. The weight of the edge connected by two nodes represented the inter-student correlation, which was calculated by the Pearson correlation between the question correctness sequences of two students corresponding to the questions from the overlapped lectures between two students. Note that each student attended multiple lectures. But two students might not attend the same lectures. Therefore, the edge weight only considered the overlapped lectures between two students. However, the color depth of each node considered all lectures that one student had attended. That was why the edge weight might be 1 but the two nodes had different color depth. This meant that the students had the same correctness sequence for the overlapped lectures but their overall accuracy for all lectures attended by each student was different.
%  
As depicted in Fig. \ref{graph}, we found that the integration of the TIR module better captured both individual student learning performance (average question answering correctness represented by the color depth of each node) and inter-student correlation (Pearson correlation of question correctness sequences between two students, represented by the edge weight between two nodes), which were more similar with real students (label), compared with the model without the TIR module.
These results demonstrate that our TIR module enables more realistic and finer-grained simulation by better capturing the inter-student correlation in student learning performance.





% \begin{figure}%[tbhp]
% \centering
% \includegraphics[width=1\linewidth]{figures/dynamism_slide_lecture_crop.pdf}
% \caption{Pearson correlation between the simulated (BertKT with or without TIR) skill level sequence along with slide ID and the real student sequence along with slide ID in each lecture on our newly collected dataset.}
% \Description{This figure shows the Pearson correlation between the simulated (BertKT with or without TIR) skill level sequence along with slide ID and the real student sequence along with slide ID in each lecture on our newly collected dataset. It shows that the integration of our TIR module captured better such correlation in most lectures.}
% \label{slide correlation per lecture}
% \end{figure}