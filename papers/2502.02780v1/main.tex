
\documentclass[sigconf]{acmart}

\usepackage[noend]{algpseudocode}
\usepackage{algorithmicx,algorithm}
\usepackage{url}
\usepackage{multirow}
\usepackage{todonotes}
\usepackage{xcolor}

\usepackage{xspace}
\def\deg{\circ\xspace}
\def\ie{\textit{i.e.}\xspace}
\def\etal{\textit{et al.}\xspace}
\def\etc{\textit{etc.}\xspace}
\def\eg{\textit{e.g.}\xspace}
\def\wrt{\textit{w.r.t.}\xspace}

%%
%% \BibTeX command to typeset BibTeX logo in the docs
\AtBeginDocument{%
  \providecommand\BibTeX{{%
    Bib\TeX}}}

\copyrightyear{2025}
\acmYear{2025}
\setcopyright{cc}
\setcctype{by}
\acmConference[CHI '25]{CHI Conference on Human Factors in Computing Systems}{April 26-May 1, 2025}{Yokohama, Japan}
\acmBooktitle{CHI Conference on Human Factors in Computing Systems (CHI '25), April 26-May 1, 2025, Yokohama, Japan}\acmDOI{10.1145/3706598.3713773}
\acmISBN{979-8-4007-1394-1/25/04}

\newcommand{\mytextcolor}[1]{\textcolor{black}{#1}}

\begin{document}

\title{Classroom Simulacra: Building Contextual Student Generative Agents in Online Education for Learning Behavioral Simulation}

\author{Songlin Xu}
% \authornote{Both authors contributed equally to this research.}
\email{soxu@ucsd.edu}
% \orcid{1234-5678-9012}
% \authornotemark[1]
% \email{webmaster@marysville-ohio.com}
\affiliation{%
  \institution{University of California, San Diego}
  \city{San Diego}
  \state{California}
  \country{USA}
}

\author{Hao-Ning Wen}
% \authornote{Both authors contributed equally to this research.}
% \email{soxu@ucsd.edu}
% \orcid{1234-5678-9012}
% \authornotemark[1]
% \email{webmaster@marysville-ohio.com}
\affiliation{%
  \institution{University of California, San Diego}
  \city{San Diego}
  \state{California}
  \country{USA}
}

\author{Hongyi Pan}
% \authornote{Both authors contributed equally to this research.}
% \email{soxu@ucsd.edu}
% \orcid{1234-5678-9012}
% \authornotemark[1]
% \email{webmaster@marysville-ohio.com}
\affiliation{%
  \institution{University of California, San Diego}
  \city{San Diego}
  \state{California}
  \country{USA}
}

\author{Dallas Dominguez}
% \authornote{Both authors contributed equally to this research.}
% \email{soxu@ucsd.edu}
% \orcid{1234-5678-9012}
% \authornotemark[1]
% \email{webmaster@marysville-ohio.com}
\affiliation{%
  \institution{University of California, San Diego}
  \city{San Diego}
  \state{California}
  \country{USA}
}

\author{Dongyin Hu}
% \authornote{Both authors contributed equally to this research.}
% \email{soxu@ucsd.edu}
% \orcid{1234-5678-9012}
% \authornotemark[1]
% \email{webmaster@marysville-ohio.com}
\affiliation{%
  \institution{University of Pennsylvania}
  \city{Philadelphia}
  \state{Pennsylvania}
  \country{USA}
}

\author{Xinyu Zhang}
% \authornote{Both authors contributed equally to this research.}
\email{xyzhang@ucsd.edu}
% \orcid{1234-5678-9012}
% \authornotemark[1]
% \email{webmaster@marysville-ohio.com}
\affiliation{%
  \institution{University of California, San Diego}
  \city{San Diego}
  \state{California}
  \country{USA}
}

\renewcommand{\shortauthors}{Xu, et al.}


% 150 word max
\begin{abstract}
Student simulation supports educators to improve teaching by interacting with virtual students. However, most existing approaches ignore the modulation effects of course materials because of two challenges: the lack of datasets with granularly annotated course materials, and the limitation of existing simulation models in processing extremely long textual data. 
To solve the challenges, we first run a 6-week education workshop from N = 60 students to collect fine-grained data using a custom built online education system, which logs students' learning behaviors as they interact with lecture materials over time. Second, we propose a transferable iterative reflection (TIR) module that augments both prompting-based and finetuning-based large language models (LLMs) for simulating learning behaviors. Our comprehensive experiments show that TIR enables the LLMs to perform more accurate student simulation than classical deep learning models, even with limited demonstration data. Our TIR approach better captures the granular dynamism of learning performance and inter-student correlations in classrooms, paving the way towards a ``digital twin'' for online education.

\end{abstract}

\begin{CCSXML}
<ccs2012>
   <concept>
       <concept_id>10003120.10003121</concept_id>
       <concept_desc>Human-centered computing~Human computer interaction (HCI)</concept_desc>
       <concept_significance>500</concept_significance>
       </concept>
 </ccs2012>
\end{CCSXML}

\ccsdesc[500]{Human-centered computing~Human computer interaction (HCI)}

\keywords{Student Simulation, Generative Agents, Classroom Digital Twin}

% \received{20 February 2007}
% \received[revised]{12 March 2009}
% \received[accepted]{5 June 2009}

\begin{teaserfigure}
  \includegraphics[width=\textwidth]{figures/teaser.pdf}
  \caption{Overview of our Classroom Simulacra framework which uses a transferable iterative reflection method to effectively learn from students' history and capture how course materials modulate learning behaviors, so as to enable realistic student simulation.}
  \Description{
  This figure shows the overview of our Classroom Simulacra framework which uses the transferable iterative reflection to effectively learn from students' history and perform realistic student simulation by capturing the modulation effect of course materials on student learning.
  In learning history, course stimuli are presented to students which incurs a learning process and results in a learning outcome. In transferable iterative reflection, we feed the learning history to the module. Then, we use the iterative interaction between a reflective agent and a novice agent to enable effective reflections from students' learning history and use that reflection to improve student simulation. Finally, in student simulation, new course stimuli are presented to virtual students to simulate the real student's learning outcome with the help of transferable iterative reflection results.}
  \label{teaser}
\end{teaserfigure}

%%
%% This command processes the author and affiliation and title
%% information and builds the first part of the formatted document.
\maketitle

\section{Introduction}
Backdoor attacks pose a concealed yet profound security risk to machine learning (ML) models, for which the adversaries can inject a stealth backdoor into the model during training, enabling them to illicitly control the model's output upon encountering predefined inputs. These attacks can even occur without the knowledge of developers or end-users, thereby undermining the trust in ML systems. As ML becomes more deeply embedded in critical sectors like finance, healthcare, and autonomous driving \citep{he2016deep, liu2020computing, tournier2019mrtrix3, adjabi2020past}, the potential damage from backdoor attacks grows, underscoring the emergency for developing robust defense mechanisms against backdoor attacks.

To address the threat of backdoor attacks, researchers have developed a variety of strategies \cite{liu2018fine,wu2021adversarial,wang2019neural,zeng2022adversarial,zhu2023neural,Zhu_2023_ICCV, wei2024shared,wei2024d3}, aimed at purifying backdoors within victim models. These methods are designed to integrate with current deployment workflows seamlessly and have demonstrated significant success in mitigating the effects of backdoor triggers \cite{wubackdoorbench, wu2023defenses, wu2024backdoorbench,dunnett2024countering}.  However, most state-of-the-art (SOTA) backdoor purification methods operate under the assumption that a small clean dataset, often referred to as \textbf{auxiliary dataset}, is available for purification. Such an assumption poses practical challenges, especially in scenarios where data is scarce. To tackle this challenge, efforts have been made to reduce the size of the required auxiliary dataset~\cite{chai2022oneshot,li2023reconstructive, Zhu_2023_ICCV} and even explore dataset-free purification techniques~\cite{zheng2022data,hong2023revisiting,lin2024fusing}. Although these approaches offer some improvements, recent evaluations \cite{dunnett2024countering, wu2024backdoorbench} continue to highlight the importance of sufficient auxiliary data for achieving robust defenses against backdoor attacks.

While significant progress has been made in reducing the size of auxiliary datasets, an equally critical yet underexplored question remains: \emph{how does the nature of the auxiliary dataset affect purification effectiveness?} In  real-world  applications, auxiliary datasets can vary widely, encompassing in-distribution data, synthetic data, or external data from different sources. Understanding how each type of auxiliary dataset influences the purification effectiveness is vital for selecting or constructing the most suitable auxiliary dataset and the corresponding technique. For instance, when multiple datasets are available, understanding how different datasets contribute to purification can guide defenders in selecting or crafting the most appropriate dataset. Conversely, when only limited auxiliary data is accessible, knowing which purification technique works best under those constraints is critical. Therefore, there is an urgent need for a thorough investigation into the impact of auxiliary datasets on purification effectiveness to guide defenders in  enhancing the security of ML systems. 

In this paper, we systematically investigate the critical role of auxiliary datasets in backdoor purification, aiming to bridge the gap between idealized and practical purification scenarios.  Specifically, we first construct a diverse set of auxiliary datasets to emulate real-world conditions, as summarized in Table~\ref{overall}. These datasets include in-distribution data, synthetic data, and external data from other sources. Through an evaluation of SOTA backdoor purification methods across these datasets, we uncover several critical insights: \textbf{1)} In-distribution datasets, particularly those carefully filtered from the original training data of the victim model, effectively preserve the model’s utility for its intended tasks but may fall short in eliminating backdoors. \textbf{2)} Incorporating OOD datasets can help the model forget backdoors but also bring the risk of forgetting critical learned knowledge, significantly degrading its overall performance. Building on these findings, we propose Guided Input Calibration (GIC), a novel technique that enhances backdoor purification by adaptively transforming auxiliary data to better align with the victim model’s learned representations. By leveraging the victim model itself to guide this transformation, GIC optimizes the purification process, striking a balance between preserving model utility and mitigating backdoor threats. Extensive experiments demonstrate that GIC significantly improves the effectiveness of backdoor purification across diverse auxiliary datasets, providing a practical and robust defense solution.

Our main contributions are threefold:
\textbf{1) Impact analysis of auxiliary datasets:} We take the \textbf{first step}  in systematically investigating how different types of auxiliary datasets influence backdoor purification effectiveness. Our findings provide novel insights and serve as a foundation for future research on optimizing dataset selection and construction for enhanced backdoor defense.
%
\textbf{2) Compilation and evaluation of diverse auxiliary datasets:}  We have compiled and rigorously evaluated a diverse set of auxiliary datasets using SOTA purification methods, making our datasets and code publicly available to facilitate and support future research on practical backdoor defense strategies.
%
\textbf{3) Introduction of GIC:} We introduce GIC, the \textbf{first} dedicated solution designed to align auxiliary datasets with the model’s learned representations, significantly enhancing backdoor mitigation across various dataset types. Our approach sets a new benchmark for practical and effective backdoor defense.



\section{Related Work}

\subsection{Large 3D Reconstruction Models}
Recently, generalized feed-forward models for 3D reconstruction from sparse input views have garnered considerable attention due to their applicability in heavily under-constrained scenarios. The Large Reconstruction Model (LRM)~\cite{hong2023lrm} uses a transformer-based encoder-decoder pipeline to infer a NeRF reconstruction from just a single image. Newer iterations have shifted the focus towards generating 3D Gaussian representations from four input images~\cite{tang2025lgm, xu2024grm, zhang2025gslrm, charatan2024pixelsplat, chen2025mvsplat, liu2025mvsgaussian}, showing remarkable novel view synthesis results. The paradigm of transformer-based sparse 3D reconstruction has also successfully been applied to lifting monocular videos to 4D~\cite{ren2024l4gm}. \\
Yet, none of the existing works in the domain have studied the use-case of inferring \textit{animatable} 3D representations from sparse input images, which is the focus of our work. To this end, we build on top of the Large Gaussian Reconstruction Model (GRM)~\cite{xu2024grm}.

\subsection{3D-aware Portrait Animation}
A different line of work focuses on animating portraits in a 3D-aware manner.
MegaPortraits~\cite{drobyshev2022megaportraits} builds a 3D Volume given a source and driving image, and renders the animated source actor via orthographic projection with subsequent 2D neural rendering.
3D morphable models (3DMMs)~\cite{blanz19993dmm} are extensively used to obtain more interpretable control over the portrait animation. For example, StyleRig~\cite{tewari2020stylerig} demonstrates how a 3DMM can be used to control the data generated from a pre-trained StyleGAN~\cite{karras2019stylegan} network. ROME~\cite{khakhulin2022rome} predicts vertex offsets and texture of a FLAME~\cite{li2017flame} mesh from the input image.
A TriPlane representation is inferred and animated via FLAME~\cite{li2017flame} in multiple methods like Portrait4D~\cite{deng2024portrait4d}, Portrait4D-v2~\cite{deng2024portrait4dv2}, and GPAvatar~\cite{chu2024gpavatar}.
Others, such as VOODOO 3D~\cite{tran2024voodoo3d} and VOODOO XP~\cite{tran2024voodooxp}, learn their own expression encoder to drive the source person in a more detailed manner. \\
All of the aforementioned methods require nothing more than a single image of a person to animate it. This allows them to train on large monocular video datasets to infer a very generic motion prior that even translates to paintings or cartoon characters. However, due to their task formulation, these methods mostly focus on image synthesis from a frontal camera, often trading 3D consistency for better image quality by using 2D screen-space neural renderers. In contrast, our work aims to produce a truthful and complete 3D avatar representation from the input images that can be viewed from any angle.  

\subsection{Photo-realistic 3D Face Models}
The increasing availability of large-scale multi-view face datasets~\cite{kirschstein2023nersemble, ava256, pan2024renderme360, yang2020facescape} has enabled building photo-realistic 3D face models that learn a detailed prior over both geometry and appearance of human faces. HeadNeRF~\cite{hong2022headnerf} conditions a Neural Radiance Field (NeRF)~\cite{mildenhall2021nerf} on identity, expression, albedo, and illumination codes. VRMM~\cite{yang2024vrmm} builds a high-quality and relightable 3D face model using volumetric primitives~\cite{lombardi2021mvp}. One2Avatar~\cite{yu2024one2avatar} extends a 3DMM by anchoring a radiance field to its surface. More recently, GPHM~\cite{xu2025gphm} and HeadGAP~\cite{zheng2024headgap} have adopted 3D Gaussians to build a photo-realistic 3D face model. \\
Photo-realistic 3D face models learn a powerful prior over human facial appearance and geometry, which can be fitted to a single or multiple images of a person, effectively inferring a 3D head avatar. However, the fitting procedure itself is non-trivial and often requires expensive test-time optimization, impeding casual use-cases on consumer-grade devices. While this limitation may be circumvented by learning a generalized encoder that maps images into the 3D face model's latent space, another fundamental limitation remains. Even with more multi-view face datasets being published, the number of available training subjects rarely exceeds the thousands, making it hard to truly learn the full distibution of human facial appearance. Instead, our approach avoids generalizing over the identity axis by conditioning on some images of a person, and only generalizes over the expression axis for which plenty of data is available. 

A similar motivation has inspired recent work on codec avatars where a generalized network infers an animatable 3D representation given a registered mesh of a person~\cite{cao2022authentic, li2024uravatar}.
The resulting avatars exhibit excellent quality at the cost of several minutes of video capture per subject and expensive test-time optimization.
For example, URAvatar~\cite{li2024uravatar} finetunes their network on the given video recording for 3 hours on 8 A100 GPUs, making inference on consumer-grade devices impossible. In contrast, our approach directly regresses the final 3D head avatar from just four input images without the need for expensive test-time fine-tuning.


\section{Model}
\begin{figure*}[t]
  \centering
  \includegraphics[width=\textwidth]{figures/framework_fig2.pdf}
   \caption{
   The pipeline of our \Model framework. We first generate an initial task instruction using LLMs with in-context learning and sample trajectories aligned with the initial language instructions in the environment. Next, we use the LLM to summarize the sampled trajectories and generate refined task instructions that better match these trajectories. We then modify specific actions within the trajectories to perform new actions in the environment, collecting negative trajectories in the process. Using the refined task instructions, along with both positive and negative trajectories, we train a lightweight reward model to distinguish between matching and non-matching trajectories. The learned reward model can then collaborate with various LLM agents to improve task planning.
   }
   \label{fig:pipeline}
\end{figure*}

In this section, we provide a detailed introduction to our framework, autonomous Agents from automatic Reward Modeling And Planning (\Model). The framework includes automated reward data generation in section~\ref{sec:data}, reward model design in section~\ref{sec:model}, and planning algorithms in section~\ref{sec:plan}.

\subsection{Background}
The planning tasks for LLM agents can be typically formulated as a Partially Observable Markov Decision Process (POMDP): $(\mathcal{X}, \mathcal{S}, \mathcal{A}, \mathcal{O}, \mathcal{T})$, where:
\begin{itemize}
    \item $\mathcal{X}$ is the set of text instructions;
    \item $\mathcal{S}$ is the set of environment states;
    \item $\mathcal{A}$ is the set of available actions at each state;
    \item $\mathcal{O}$ represents the observations available to the agents, including text descriptions and visual information about the environment in our setting;
    \item $\mathcal{T}: \mathcal{S} \times \mathcal{A} \rightarrow \mathcal{S}$ is the transition function of states after taking actions, which is given by the environment in our settings. 
\end{itemize}

Given a task instruction $\mathit{x} \in \mathcal{X}$ and the initial environment state $\mathit{s_0} \in \mathcal{S}$, planning tasks require the LLM agents to propose a sequence of actions ${\{a_n\}_{n=1}^{N}}$ that aim to complete the given task, where $a_n \in \mathcal{A}$ represents the action taken at time step $n$, and $N$ is the total number of actions executed in a trajectory.
Following the $n$-th action, the environment transitions to state $\mathit{s_{n}}$, and the agent receives a new observation $\mathit{o_{n}}$. Based on the accumulated state and action histories, the task evaluator determines whether the task is completed.

An important component of our framework is the learned reward model $\mathcal{R}$, which estimates whether a trajectory $h$ has successfully addressed the task:
\begin{equation}
    r = \mathcal{R}(\mathit{x}, h),
\end{equation}
where $h = \{\{a_n\}_{n=1}^N, \{o_n\}_{n=0}^{N}\}$, $\{a_n\}_{n=1}^N$ are the actions taken in the trajectory, $\{o_n\}_{n=0}^{N}$ are the corresponding environment observations, and $r$ is the predicted reward from the reward model.
By integrating this reward model with LLM agents, we can enhance their performance across various environments using different planning algorithms.

\subsection{ Automatic Reward Data Generation.}
\label{sec:data}
To train a reward model capable of estimating the reward value of history trajectories, we first need to collect a set of training language instructions $\{x_m\}_{m=1}^M$, where $M$ represents the number of instruction goals. Each instruction corresponds to a set of positive trajectories $\{h_m^+\}_{m=1}^M$ that match the instruction goals and a set of negative trajectories $\{h_m^-\}_{m=1}^M$ that fail to meet the task requirements. This process typically involves human annotators and is time-consuming and labor-intensive~\citep{christiano2017deep,rafailov2024direct}. As shown in Fig.~\ref{fig:instruction_generation_sciworld} of the Appendix. we automate data collection by using Large Language Model (LLM) agents to navigate environments and summarize the navigation goals without human labels.

\noindent\textbf{Instruction Synthesis.} The first step in data generation is to propose a task instruction for a given observation. We achieve this using the in-context learning capabilities of LLMs. The prompt for instruction generation is shown in Fig.~\ref{fig:instruction_refinement_sciworld} of the Appendix. Specifically, we provide some few-shot examples in context along with the observation of an environment state to an LLM, asking it to summarize the observation and propose instruction goals. In this way, we collect a set of synthesized language instructions $\{x_m^{raw}\}_{m=1}^M$, where $M$ represents the total number of synthesized instructions.

\noindent\textbf{Trajectory Collection.} Given the synthesized instructions $x_m^{raw}$ and the environment, an LLM-based agent is instructed to take actions and navigate the environment to generate diverse trajectories $\{x_m^{raw}, h_m\}_{m=0}^M$ aimed at accomplishing the task instructions. Here, $h_m$ represents the $m$-th history trajectory, which consists of $N$ actions $\{a_n\}_{n=1}^N$ and $N+1$ environment observations $\{o_n\}_{n=0}^N$.
Due to the limited capabilities of current LLMs, the generated trajectories $h_m$ may not always align well with the synthesized task instructions $x_m$. To address this, we ask the LLM to summarize the completed trajectory $h_m$ and propose a refined goal $x_m^r$. This process results in a set of synthesized demonstrations $\{x_m^r, h_m\}_{m=0}^{M_r}$, where $M_r$ is the number of refined task instructions.

\noindent\textbf{Pairwise Data Construction.} 
To train a reward model capable of distinguishing between good and poor trajectories, we also need trajectories that do not satisfy the task instructions. To create these, we sample additional trajectories that differ from $\{x_m^r, h_m\}$ and do not meet the task requirements by modifying actions in $h_m$ and generating corresponding negative trajectories $\{h_m^-\}$. For clarity, we refer to the refined successful trajectories as $\{x_m, h_m^+\}$ and the unsuccessful ones as $\{x_m, h_m^-\}$. These paired data will be used to train the reward model described in Section~\ref{sec:model}, allowing it to estimate the reward value of any given trajectory in the environment.

\subsection{ Reward Model Design.} 
\label{sec:model}
\noindent\textbf{Reward Model Architectures.}
Theoretically, we can adopt any vision-language model that can take a sequence of visual and text inputs as the backbone for the proposed reward model. In our implementation, we use the recent VILA model~\citep{lin2023vila} as the backbone for reward modeling since it has carefully maintained open-source code, shows strong performance on standard vision-language benchmarks like~\citep{fu2023mme,balanced_vqa_v2,hudson2018gqa}, and support multiple image input. 

The goal of the reward model is to predict a reward score to estimate whether the given trajectory $(x_m, h_m)$  has satisfied the task instruction or not, which is different from the original goal of VILA models that generate a series of text tokens to respond to the task query. To handle this problem, we additionally add a fully-connected layer for the model, which linearly maps the hidden state of the last layer into a scalar value. 

\noindent\textbf{Optimazation Target.}
Given the pairwise data that is automatically synthesized from the environments in Section~\ref{sec:data}, we optimize the reward model by distinguishing the good trajectories $(x_m, h^+_m)$ from bad ones $(x_m, h^-_m)$. Following standard works of reinforcement learning from human feedback~\citep{bradley1952rank,sun2023salmon,sun2023aligning}, we treat the optimization problem of the reward model as a binary classification problem and adopt a cross-entropy loss. Formally, we have 
\begin{equation}
    \mathcal{L(\theta)} = -\mathbf{E}_{(x_m,h_m^+,h_m^-)}[\log\sigma(\mathcal{R}_\theta(x_m, h_m^+)-\mathcal{R}_\theta(x_m, h_m^-))],
\end{equation}
where $\sigma$ is the sigmoid function and $\theta$ are the learnable parameters in the reward model $\mathcal{R}$.
By optimizing this target, the reward model is trained to give higher value scores to the trajectories that are closer to the goal described in the task instruction. 

\subsection{ Planning with Large Vision-Langauge Reward Model.}
After getting the reward model to estimate how well a sampled trajectory match the given task instruction, we are able to combine it with different planning algorithms to improve LLM agents' performance. Here, we summarize the typical algorithms we can adopt in this paper.

\noindent\textbf{Best of N.} This is a simple algorithm that we can adopt the learned reward model to improve the LLM agents' performances. We first prompt the LLM agent to generate $n$ different trajectories independently and choose the one with the highest predicted reward score as the prediction for evaluation. Note that this simple method is previously used in natural language generation~\citep{zhang2024improving} and we adopt it in the context of agent tasks to study the effectiveness of the reward model for agent tasks.

\noindent\textbf{Reflexion.} Reflexion~\citep{shinn2024reflexion} is a planning framework that enables large language models (LLMs) to learn from trial-and-error without additional fine-tuning. Instead of updating model weights, Reflexion agents use verbal feedback derived from task outcomes. This feedback is converted into reflective summaries and stored in an episodic memory buffer, which informs future decisions. Reflexion supports various feedback types and improves performance across decision-making, coding, and reasoning tasks by providing linguistic reinforcement that mimics human self-reflection and learning. %This approach yields significant gains over baseline methods in several benchmarks.

\noindent\textbf{MCTS.} 
We also consider tree search-based planning algorithms like Monte Carlo Tree Search (MCTS)~\citep{coulom2006efficient,silver2017mastering} to find the optimal policy. 
There is a tree structure constructed by the algorithm, where each node represents a state and each edge signifies an action.
Beginning at the initial state of the root node, the algorithm navigates the state space to identify action and state trajectories with high rewards, as predicted by our learned reward model. 

The algorithm tracks 1) the frequency of visits to each node and 2) a value function that records the maximum predicted reward obtained from taking action ${a}$ in state ${s}$.
MCTS would visit and expand nodes with either higher values (as they lead to high predicted reward trajectory) or with smaller visit numbers (as they are under-explored).
We provide more details in the implementation details and the appendix section.


\label{sec:plan}


\section{Simulation Study}
\label{sec: eduagent results}



We first explored the feasibility of our framework compared with baseline models in the public dataset named EduAgent\cite{xu2024eduagent}. 

\subsection{The EduAgent Dataset}
The EduAgent dataset was collected from N = 301 students, who were asked to watch 5-min online course videos. After that, students were prompted to finish a post test which comprises 10-12 questions. The dataset contained students' correctness on each post test question, as well as corresponding question contents and course materials which were specifically related to each question. More details about this dataset could be obtained from \cite{xu2024eduagent}. 


\subsection{Experiment Settings}
We split the dataset into training and testing set by following a individual-wise manner with 0.8 ratio. Specifically, all post test performance of 80\% students were used as the training set and all post test performance of another 20\% students were testing set. The training set was further divided into model training and model validation set following the same individual-wise manner with 0.8 ratio as well. 
We set the first five questions as past questions of the student history and other questions as future questions for prediction.
As depicted in Fig. \ref{framework:prompt}, the simulation model input included the correctness of past questions of real students, as well as corresponding past questions contents and course materials, which were specifically related to each corresponding past question. Moreover, the model input also included future question contents and course materials which are specifically related to each future question. The model output was the correctness of each future question for predictions. 

As depicted in Section~\ref{sec:model}, our TIR module could augment both prompting-based simulation (standard prompt, CoT prompt) and finetuning-based (BertKT) simulation performance. Therefore, in the experiment, we show results of both simulation types with or without the integration of our TIR module. All LLMs-based models used GPT4o-mini. We also compared with five state-of-the-art knowledge tracing models based on deep learning, as depicted in Section. \ref{sec:model}.



\subsection{Results and Analysis}

Results were depicted in Table. \ref{tab:result_eduagent}. We found that the integration of the TIR module improved the simulation performance so that both the simulation accuracy and f1 score were better than all deep learning baseline models. Specifically, the best deep learning model was SimpleKT with 0.6772 accuracy and 0.6698 F1 score. Without the TIR module, the best LLMs-based model was CoT-based prompting with 0.6222 accuracy and 0.5610 F1 score. However, after integrating the TIR module, the best LLMs-based model was finetuning-based BertKT model with 0.7012 accuracy and 0.6880 F1 score, which was superior than the best deep learning model. 

Moreover, we found that the integration of the TIR module could improve all LLMs-based models including standard prompting, CoT prompting, and BertKT, as supported by Table. \ref{tab:result_eduagent}. Although the accuracy in CoT slightly decreased, its F1 score was however obviously improved. 

These results demonstrate the feasibility and effectiveness of our TIR module to enhance existing LLMs-based approaches for more realistic student simulation, which were even better than deep learning models.


\begin{figure}%[tbhp]
\centering
\includegraphics[width=1\linewidth]{figures/system.pdf}
%xyz: Too long (revised)
\caption{
\mytextcolor{
(a). Illustration of our CogEdu system. 
(b). Our action prompt strategy for instructors based on attention ratio and knowledge ratio.
(c,d). The UI of the user end (c) and server end (d) of CogEdu. 
}
}
\Description{Figure (a) shows N student clients on the right side and one teacher client on the left. The teacher client has access to the three modules of feedback: area of interest (AoI) feedback, general cognitive feedback, and action prompt. The bounding boxes of AoI in the middle of the screen represent the area where the students were looking at. The CogBar at the top left corner summarizes the general cognitive feedback. The knowledge ratio, shown as a horizontal bar, is calculated by the ratio of students that were not confused about the contents over all students. The attention ratio, also shown as a horizontal bar below the knowledge ratio, is calculated by the ratio of students that were paying attention over all students. On the top right, the action prompt is the suggestion given to the instructor based on the values of CogBar. In the figure, the action prompt is "Draw attention!" in red. Figure (b) shows our action prompt strategy based on attention ratio and knowledge ratio. Low knowledge ratio triggers "Repeat the current content" recommendation. Low attention ratio triggers "Draw attention" recommendation. The horizontal axis denotes the attention ratio while the vertical axis denotes the knowledge ratio. Each ratio is separated into 3 buckets: low, medium, and high. When both attention and knowledge ratio are high, no feedback was given. When both attention and knowledge ratio are low, "draw attention" was given as the feedback. Other than that, when the knowledge ratio is lower, "repeat" was given as the feedback. When the attention ratio is lower, "draw attention" was given as the feedback to instructors. Figure (c) shows the UI of the user end of CogEdu. There are a few entries for the user to enter: identity (student/teacher), passcode, and name. There is also a lecture title and abstract on the bottom of the page. Figure (d) shows the UI of the server end of CogEdu. There are some entries for the system administrator to enter: lecture title, lecture abstract, lecture instructor, lecture time, lecture Zoom ID, gaze information on/off, and cognitive information on/off. There is also a participant list. 
}
\label{cogedu system}
\end{figure}

\section{Online Education Workshop and Dataset}
\label{sec:workshop}

Although our simulation experiment on the EduAgent dataset demonstrated the effectiveness of our framework compared with baseline models, the EduAgent dataset itself only contains 5-min lectures. Such short duration may not capture the fine-grained effect of course stimuli on student learning performance.
Therefore, it is necessary to examine the student simulation in lectures with longer duration to reveal further insights. 


\begin{figure}%[tbhp]
\centering
\includegraphics[width=1\linewidth]{figures/course_demo.pdf}
\caption{A real online education example of our live CogEdu system shown in Fig. \ref{cogedu system}}
\Description{This figure shows a screenshot of a live example of the online education system shown in Fig. \ref{cogedu system}. The system is in a browser that integrates the Zoom interface. In the middle, there is a course slide. In addition, there are two bounding boxes (AoI) on the slide. In the bottom there are two bars: knowledge ratio at 33.3\% and attention ratio at 100\%.}
\label{course system demo}
\end{figure}




\subsection{Workshop Design}

To this end, we conducted a 6-week online education workshop to deliver 12 lectures, where each lecture lasted 1 hour. The long-duration lectures could not only verify the simulation results, but also reveal new insights about how the simulation models can capture students' learning performance variations across the whole lecture (depicted in Section \ref{subsec: dynamism}). \mytextcolor{The workshop syllabus is depicted in Appendix Table. \ref{tab:workshop syllabus}}.

\subsubsection{\textbf{Participants}}

We recruited 30 elementary school students, 30 high school students, and 8 instructors from high schools and universities in the local area using emails and social media. 
We removed the demographic information (such as age and gender) for privacy concerns.
Our data collection was approved by the Institutional Review Board (IRB). All participated students and instructors were informed of the experiment form and then signed consent forms. For participants under 18 years old, we obtained the written consent form from both participants and their parents.







\subsubsection{\textbf{Task and Procedure}}
We first prompted the students and instructors to watch an introduction video about how to use our online education system to facilitate learning and teaching, as well as the detailed procedures of our data collection (Fig.~\ref{system procedure}).
%
After that, students were required to first go through a gaze calibration process (depicted in Section. \ref{subsubsec: student client}) for accurate gaze collection.
%
Then students were prompted to perform facial expressions (including confused and neutral expressions) in order to train a model for cognitive information detection (more details in Section \ref{subsubsec: student client}). 
%
After that, both students and instructors were in the same online video conference system (Section \ref{subsec: cogedu}) and instructors presented the course materials to the students. The lecture materials were slides prepared by our research team. During the lectures, our online education system provided visual feedback to the instructors about the students' learning status, and the instructors could adapt their teaching strategies accordingly (depicted in Section. \ref{subsubsec: teacher client}).
%
After the lecture, students were required to finish a post-test composed of 10-12 questions related to each specific lecture to measure their learning outcome. 

\begin{figure}%[tbhp]
\centering
\includegraphics[width=1\linewidth]{figures/system_process.pdf}
\caption{
\mytextcolor{The procedure to use our online learning system. (a)(b)(c). Gaze calibration process for gaze tracking. (d)(e). Facial expression model training data collection process. (f)(g). Students and teachers join in the online video calling from their own clients in class.}}
\Description{Figure (a). A popup window saying "[Before start: 1/2] Please calibrate the GazeCloud for gaze collection." (b). A cartoon image of a face on the top. A button saying "Start Gaze Calibration" in the middle. 4 sentences describing what the user must make sure to check at the bottom. (c). A sentence saying "Look at dot". A dot in the middle of the screen. (d). A window saying "Please make no expression." (e). A window saying "Please make confused expression." (f). A window with the word "student" on top of the button "Join Audio by Computer." (g). A window with the word "teacher" on top of the button "Join Audio by Computer."}
\label{system procedure}
\end{figure}


\subsubsection{\textbf{Experiment Design}}
Our six-week workshop was composed of a series of 12 lectures about the basics of artificial intelligence, covering different topics such as basic concepts in machine learning, computer vision, natural language processing, reinforcement learning, etc. Each lecture lasted one hour. The difficulty of the lectures was tailored to match the knowledge level of elementary and high school students, respectively.
%
For each week, the instructors delivered two lectures. Students were encouraged to select the same time slot for real-time and synchronous teaching among all students together (Fig. \ref{cogedu system}(a)). If students had time conflicts with the instructors, we made new time arrangements for these students for additional data collection.
%
Each student was encouraged to attend as many lectures as they could. Overall, each student attended 3 lectures on average. 



\subsubsection{\textbf{Measurement}}
As depicted above, for students, we mainly collected their gaze, facial expressions, and post-test answers. The gaze and facial expressions were mainly used to generate student status feedback to instructors so that the instructors could take specific actions to increase the students' engagement and improve the quality of collected data. The post-test answers were used to measure the student learning outcomes.


\begin{figure*}%[tbhp]
\centering
\includegraphics[width=1\linewidth]{figures/bertkt_train_loss.pdf}
\caption{
\mytextcolor{
(a)(b). BertKT training curve without (a) and with (b) TIR across different epochs. Vertical axis on the left is the loss value (solid lines). Vertical axis on the right is the metrics (accuracy and F1 score) value (dotted lines).}}
\Description{Figure (a)(b) show the BertKT training curve (training/validation loss, accuracy, and F1 score) with(a) and without(b) TIR across different epochs. Vertical axis on the left is the loss value. Vertical axis on the right is the metrics (accuracy and F1 score) value. The dotted lines represent metrics (accuracy and F1 score) while the solid lines represent training and validation losses. On the top (a), BertKT without TIR training loss curve starts at around 0.69 and ends at around 0.59 with 100 epochs. On the bottom (b), BertKT with TIR training curve starts at around 0.69 and ends at around 0.52 with 100 epochs. On the top (a), BertKT without TIR validation loss curve starts at around 0.73 and ends at around 0.67 with 100 epochs. On the bottom (b), BertKT with TIR validation loss curve starts at around 0.72 and ends at around 0.60 with 100 epochs. On the top (a), BertKT without TIR accuracy curve starts at around 0.45 and ends at around 0.62 with 100 epochs. On the bottom (b), BertKT with TIR accuracy curve starts at around 0.42 and ends at around 0.71 with 100 epochs. F1 scores of BertKT with and without TIR follow similar trends as accuracy scores of BertKT with and without TIR. All the curves are more smooth for BertKT with TIR compared to BertKT without TIR.}
\label{loss}
\end{figure*}

\subsection{Online Education System}
\label{subsec: cogedu}




Although existing video conference software such as Zoom\footnote{https://zoom.us/} provided a stable solution for online education, the subtle student behaviors may not be captured to provide insights to teachers. Moreover, research showed that students' learning performance might become worse compared with in-person instructions\cite{nguyen2015effectiveness}. As a result, the quality of our collected data could be severely compromised. 
%
Therefore, to solve this problem and facilitate subtle communication between students and instructors, we developed an online education system named \textbf{CogEdu} that could support synchronized teaching between students and teachers in a client on the computer, while also providing real-time student status feedback to teachers to enhance the education process.
%
Based on the ubiquitous webcams on laptops, we collected the gaze information and facial expressions of students. By analyzing the collected data, we provided real-time feedback to the instructors about the understanding of current contents, the attention status, and a fine-grained visualization of contents that students were concerned about. Understanding about contents (or confusion) and attention were referred to as \textit{cognitive information}. To further assist instructors, the system also provided teaching strategy suggestions based on the collected data. 

As a result, this system could augment student learning engagement and teaching effectiveness to enable high-quality data collection. More details were depicted below.



\subsubsection{\textbf{System Implementation}}

We implemented the CogEdu system on a cloud server. Users (students and instructors) could access the system using their browsers (Fig. \ref{cogedu system}). Considering that most users were more familiar with Zoom, a video teleconferencing software program, we implemented the video conference function based on Zoom APIs\footnote{https://developers.zoom.us/docs/api/}. All the feedback was overlaid over the embedded Zoom interface. To support the large flow of facial expression data before and during the lecture, and to enhance the robustness of the system, we adopted Kubernetes on the google cloud\footnote{https://cloud.google.com/kubernetes-engine} to manage the deployment, scaling, and management. Instances scaled up when the load was growing to reduce latency and achieve satisfying real-time performance.



\subsubsection{\textbf{Student Client}}
\label{subsubsec: student client}
On the student’s side, students were required to first go through a gaze calibration process and then collected facial expressions for cognitive information detection. To collect gaze information from the students, we used the service from GazeRecorder\footnote{https://gazerecorder.com/}. Around 28 gaze positions were provided from the service per second, which were then labeled as fixations or saccades using a velocity-based method \cite{engbert2003microsaccades}. Meanwhile, the system sent facial expressions to the server for cognitive information detection every second. The students' side uploaded all gaze information and cognitive information every five seconds.

The algorithm we used to transform raw gaze into fixations was from \cite{engbert2003microsaccades}. The basic idea was to calculate a velocity threshold, and gaze points with velocity below were labeled as fixations. 
%
The confusion information of students was detected using a support vector machine (SVM). Before the lecture started, students were asked to make confused expressions and neutral expressions. Collected data were cropped to focus on the eyebrow-eye region and then fed to principal component analysis (PCA) to extract features. An SVM was trained based on the features to classify either confused or neutral expressions.

Attention detection was facilitated by gaze detection, confusion detection, and browser built-in properties. When the user switched to another tab or application, \textit{document.visibilityState} in the browser became hidden. This property was checked together with confusion detection. When the confusion detection algorithm on the server failed to detect a face, which meant the student's face was out of the camera, the system then asserted the student to be not attentive. Thirdly, when the gaze of the student fell out of the screen, the student was labeled as not attentive as well. 

%xyzr1: Use hatch patterns in the bars to make it better for black-and-white printing 
\begin{figure*}
\centering
\includegraphics[width=0.8\linewidth]{figures/model_compare_accuracy.pdf}
\caption{
\mytextcolor{
Model accuracy and F1 score comparison on our newly collected dataset. Left (a,c) shows comparison (a: accuracy, c: F1 score) among deep learning models and LLMs-based models using GPT-4o. Right (b,d) shows comparison (b: accuracy, d: F1 score) of LLMs-based models using GPT-4o v.s. GPT-4o mini.}}
\Description{Figure (a) illustrates that large language model based models get improved with TIR and outperform all deep learning baseline models (akt, atkt, dkt, dkvmn, and simpleKT). The accuracy of deep learning models (akt, atkt, dkt, dkvmn, and simpleKT) is 0.625, 0.527, 0.551, 0.606, and 0.656 respectively. The accuracy of large language model based models (Standard, CoT, BertKT) is 0.629, 0.633, and 0.602 respectively. The accuracies of large language model based models with TIR (Standard+TIR, CoT+TIR, BertKT+TIR) are 0.668, 0.676, and 0.684 respectively. Figure (b) illustrates that in most cases (standard and CoT prompts), without TIR, GPT-4o outperforms GPT-4o mini. With TIR, GPT-4o mini outperforms GPT-4o without TIR. Figure (c) illustrates that LLMs-based models get improved with TIR and outperform all deep learning baseline models (akt, atkt, dkt, dkvmn, and simpleKT). The F1 scores of deep learning models (akt, atkt, dkt, dkvmn, and simpleKT) are 0.6, 0.529, 0.553, 0.606, and 0.649 respectively. The F1 scores of LLMs-based models (Standard, CoT, BertKT) are 0.678, 0.682, and 0.651 respectively. The F1 scores of LLMs-based models with TIR (Standard+TIR, CoT+TIR, BertKT+TIR) are 0.706, 0.713, and 0.703 respectively. Figure (d) illustrates the comparison of LLMs-based models with and without TIR using GPT-4o v.s. GPT-4o mini and shows that GPT-4o mini with TIR outperforms GPT-4o without TIR.}
\label{model compare accuracy}
\end{figure*}

\subsubsection{\textbf{Teacher Client}}
\label{subsubsec: teacher client}
On the instructor's side, the system fetched all information that students uploaded to the server every five seconds and then processed the information to provide the instructor with feedback. The feedback provided to instructors consisted of three modules: area of interest (AoI) feedback, general cognitive feedback, and action prompt.


\textbf{Area of Interest Feedback}: This module took as input the gaze and cognitive information and visualized feedback as bounding boxes on the ongoing lecture. These bounding boxes (depicted in Fig. \ref{cogedu system} and Fig. \ref{course system demo}) were clustered from gazes that were labeled as fixations using a spectral clustering algorithm. The bounded area was where students were looking. The opacity of the bounding box represented the ratio of students looking at this area over all students, and the color represented the ratio of students that were confused about the contents in the area over students looking at this area. More details about the spectral clustering method were depicted in Appendix \ref{appendix sub sec: gaze cluster}.


\textbf{General Cognitive Feedback}: This module took as input the cognitive information, and visualized feedback as a summarized bar chart (depicted in Fig. \ref{cogedu system} and Fig. \ref{course system demo}). We displayed the ratio of students that were not confused about the contents over all students (knowledge ratio), and the ratio of students that were paying attention over all students (attention ratio). 

\textbf{Action Prompt}: Based on the general cognitive information, the system provided teaching suggestions to the instructor (depicted in Fig. \ref{cogedu system} and Fig. \ref{course system demo}). When the attention ratio fell lower, instructors were prompted to draw attention from students. When the knowledge ratio fell lower, repeating the current contents was recommended (Fig.~\ref{cogedu system}(b)).



% \begin{figure*}
% \centering
% \includegraphics[width=1\linewidth]{figures/simulation_student_crop.pdf}
% \caption{Simulation accuracy (top) and F1 score (bottom) of BertKT with and without TIR on our new dataset in individual student level.}
% \Description{This figure contain two bar graphs that compare the performance of BertKT with and without TIR in individual student level. At the top it shows the comparison of BertKT with and without TIR in terms of their simulation accuracy in individual student level. BertKT with TIR performs better or similarly with BertKT without TIR in 16 out of 18 individual students simulation accuracy. At the bottom it shows the comparison of BertKT with and without TIR in terms of their simulation F1 scores in individual student level. BertKT with TIR performs better or similarly with BertKT without TIR in 13 out of 18 individual students F1 scores.}
% \label{individual student accuracy}
% \end{figure*}









\section{Evaluation}
\label{sec: user study}

Based on the students' data collected from our workshop, we explored the simulation performance in not only straightforward accuracy comparison, but also the dynamic patterns of students' learning performance at fine-grained levels.



\subsection{Simulation Settings}
The simulation settings were similar with those in the EduAgent dataset. 
We split the dataset into training and testing set by following a individual-wise manner with 0.7 ratio. 
We set the first five questions as past questions of the student history and other questions as future questions for prediction.
Both of the model input and output were the same as the EduAgent simulation experiment (Fig. \ref{framework:prompt} and Fig. \ref{framework:finetune}).
For LLMs-based models, we obtained the results with or without our TIR module for both prompting-based models (standard and CoT prompt) and finetuning-based models (BertKT), which were compared with deep learning models. 
All LLMs-based models used both GPT-4o and GPT-4o mini. 


\begin{figure*}
\centering
\includegraphics[width=1\linewidth]{figures/model_compare_student.pdf}
\caption{
\mytextcolor{
Heatmap to show the average simulation accuracy (each cell) for each individual student using each model.}}
\Description{
This figure shows the average simulation accuracy for each individual student using each model.
We found a significant effect of the model on simulation accuracy ($F(5, 85) = 5.16, \, p = 0.0004, \, \eta_{p}^{2} = 0.18$), indicating a large effect size. For pairwise comparisons, significant differences were observed between the following model pairs:
%
AKT vs. DKT: $t(17) = 2.29, \, p = 0.035, \, \eta_{p}^{2} = 0.16$.
BertKT with TIR vs. ATKT: $t(17) = 3.42, \, p = 0.003, \, \eta_{p}^{2} = 0.24$.
BertKT with TIR vs. DKT: $t(17) = 3.24, \, p = 0.005, \, \eta_{p}^{2} = 0.30$.
BertKT with TIR vs. DKVMN: $t(17) = 3.83, \, p = 0.001, \, \eta_{p}^{2} = 0.24$.
BertKT with TIR vs. SimpleKT: $t(17) = 2.45, \, p = 0.025, \, \eta_{p}^{2} = 0.14$.
These results indicate that BertKT (with TIR) exhibited significantly better performance compared to most deep learning models. Although we did not find significance between the BertKT (with TIR) and AKT, it still showed better simulation accuracy of BertKT (with TIR) than AKT for most individual students.
}
\label{model compare student}
\end{figure*}

\mytextcolor{
\subsection{TIR Makes LLMs Superior than Deep Learning Models}
We first compared the accuracy and F1-score of the simulated performance of various models by comparing with the real students' performance. 
As depicted in Fig. \ref{model compare accuracy}(a), without the integration of our TIR module, the best model is SimpleKT with 0.656 accuracy, which was better than all LLMs-based models. However, with our TIR module, all LLMs-based models increased the simulation accuracy and all had larger accuracy than the SimpleKT model. The F1 score results were similar as well in Fig. \ref{model compare accuracy}(c), i.e. all TIR-augmented LLMs held better F1 score than all deep learning models. \mytextcolor{These results are encouraging because the selected deep learning models are proven educational models widely accepted in the educational domain \cite{10.1145/3394486.3403282,DBLP:conf/iclr/0001L0H023,10.1145/3474085.3475554,piech2015deepknowledgetracing,zhang2017dynamickeyvaluememorynetworks} to inform teaching practices and support adaptive learning strategies \cite{scholtz2021systematic}.
Therefore, the superior predictive capability of our model indicates a strong potential for real-world applicability.}
To further demonstrate our model's impact, we selected BertKT with TIR as an example to perform statistical analysis by comparing with the five deep learning models in individual-level, lecture-level, and question-level. The effect of different models on student simulation performance was measured by repeated-measures ANOVA with paired t-tests for pair-wise comparisons.
}

% \subsection{Benchmarking Results}
% We first compared the absolute accuracy and F1-score of the simulated performance of various models by comparing with the real students' performance. 
% As depicted in Fig. \ref{model compare accuracy}(a), without the integration of our TIR module, the best model is SimpleKT with 0.656 accuracy, which was better than all LLMs-based models. However, with our TIR module, all LLMs-based models increased the simulation accuracy and all had larger accuracy than the SimpleKT model. The F1 score results were similar as well in Fig. \ref{model compare accuracy}(c), i.e. all TIR-integrated LLMs-based models held better F1 score than all deep learning models. 
% Furthermore, to showcase the impact of the integration of our TIR module, for BertKT models, we showed the simulation performance (accuracy, F1 score) in individual student level (Fig. \ref{individual student accuracy}),  question level (Fig. \ref{question-level accuracy}),lecture level (Fig. \ref{lecture-level accuracy}), with or without TIR module. All levels of analysis showed that the integration of the TIR module could improve the simulation performance in most cases.
% These results demonstrate that the integration of our TIR module apparently improved the student learning performance simulation of LLMs-based models. 





\mytextcolor{
\textbf{Individual-Level}: We calculated the average simulation accuracy for each individual student, as depicted in Fig. \ref{model compare student}.
We found a significant effect of the model on simulation accuracy ($F(5, 85) = 5.16, \, p = 0.0004, \, \eta_{p}^{2} = 0.18$), indicating a large effect size. For pairwise comparisons, significant differences were observed between the following model pairs:
%
\begin{itemize}
    \item AKT vs. DKT: $t(17) = 2.29, \, p = 0.035, \, \eta_{p}^{2} = 0.16$.
    \item BertKT with TIR vs. ATKT: $t(17) = 3.42, \, p = 0.003, \, \eta_{p}^{2} = 0.24$.
    \item BertKT with TIR vs. DKT: $t(17) = 3.24, \, p = 0.005, \, \eta_{p}^{2} = 0.30$.
    \item BertKT with TIR vs. DKVMN: $t(17) = 3.83, \, p = 0.001, \, \eta_{p}^{2} = 0.24$.
    \item BertKT with TIR vs. SimpleKT: $t(17) = 2.45, \, p = 0.025, \, \eta_{p}^{2} = 0.14$.
\end{itemize}
These results indicate that BertKT (with TIR) exhibited significantly better performance compared to most deep learning models. Although we did not find significance between the BertKT (with TIR) and AKT, Fig. \ref{model compare student} still showed better simulation accuracy of BertKT (with TIR) than AKT for most individual students.
}


\mytextcolor{
\textbf{Lecture-Level}: We then calculated the average simulation accuracy for each specific course lecture ID, as depicted in Fig. \ref{model compare lecture}. Results showed a significant effect of the model on simulation accuracy ($F(5, 55) = 4.53, \, p = 0.002, \, \eta_{p}^{2} = 0.20$), indicating a large effect size. Significant differences were observed between the following model pairs:
%
\begin{itemize}
    \item AKT vs. DKVMN: $t(11) = 2.49, \, p = 0.03, \, \eta_{p}^{2} = 0.21$.
    \item BertKT with TIR vs. ATKT: $t(11) = 3.06, \, p = 0.01, \, \eta_{p}^{2} = 0.28$.
    \item BertKT with TIR vs. DKT: $t(11) = 2.91, \,p = 0.014, \, \eta_{p}^{2} = 0.27$.
    \item BertKT with TIR vs. DKVMN: $t(11) = 4.27, \, p = 0.001, \, \eta_{p}^{2} = 0.35$.
\end{itemize} 
Similar with individual-level results, these results indicate the superiority of the BertKT (with TIR) than these deep learning models. Although we did not find significance between the BertKT (with TIR) and AKT/SimpleKT, Fig. \ref{model compare lecture} still showed better simulation accuracy of BertKT (with TIR) than AKT/SimpleKT for most lectures.
}





\mytextcolor{
\textbf{Question-Level}: We then calculated the average simulation accuracy for each specific question ID in post-test, as depicted in Fig. \ref{model compare question}. 
Results did not find a significant effect of the model on simulation accuracy ($F(5, 30) = 1.09, \, p = 0.387, \, \eta_{p}^{2} = 0.14$). However, significant differences were observed between the following model pairs:
%
\begin{itemize}
    \item BertKT with TIR vs. DKVMN: $t(6) = 3.69, \, p = 0.01, \, \eta_{p}^{2} = 0.46$.
    \item BertKT with TIR vs. simpleKT: $t(6) = 2.59, \, p = 0.041, \, \eta_{p}^{2} = 0.32$.
\end{itemize}
Although we did not find significance between the BertKT (with TIR) and AKT/ATKT/DKT, Fig. \ref{model compare question} still showed better simulation accuracy of BertKT (with TIR) than AKT/ATKT/DKT for most question IDs.
}




%xyz: The title here reads weird (revised)
\subsection{TIR Enhances Model Learning Efficiency}
Although the training of BertKT+TIR model used all training students, it is worth noting that, for other prompt-based models (Standard and CoT), we only used four example students as the contextual example demonstration instead of all students in the training set. 
However, after integrating our TIR module, both of prompt-based models (Standard and CoT) achieved better simulation performance than deep learning models (Fig. \ref{model compare accuracy}), which used all students in the training set for model training. This demonstrates that our TIR module could enhance the exploitation efficiency of prompt-based models to achieve comparable or even more realistic student simulation within more limited training data.

\begin{figure}
\centering
\includegraphics[width=1\linewidth]{figures/model_compare_lecture.pdf}
\caption{
\mytextcolor{Heatmap to show the average simulation accuracy (each cell) for each specific lecture using each model.}}
\Description{
This figure shows the average simulation accuracy for each specific course lecture ID. Results showed a significant effect of the model on simulation accuracy ($F(5, 55) = 4.53, \, p = 0.002, \, \eta_{p}^{2} = 0.20$), indicating a large effect size. Significant differences were observed between the following model pairs:
%
AKT vs. DKVMN: $t(11) = 2.49, \, p = 0.03, \, \eta_{p}^{2} = 0.21$.
BertKT with TIR vs. ATKT: $t(11) = 3.06, \, p = 0.01, \, \eta_{p}^{2} = 0.28$.
BertKT with TIR vs. DKT: $t(11) = 2.91, \,p = 0.014, \, \eta_{p}^{2} = 0.27$.
BertKT with TIR vs. DKVMN: $t(11) = 4.27, \, p = 0.001, \, \eta_{p}^{2} = 0.35$.
Similar with individual-level results, these results indicate the superiority of the BertKT (with TIR) than these deep learning models. Although we did not find significance between the BertKT (with TIR) and AKT/SimpleKT, it still showed better simulation accuracy of BertKT (with TIR) than AKT/SimpleKT for most lectures.
}
\label{model compare lecture}
\end{figure}





\subsection{TIR Empowers Smaller LLMs}
We also compared the simulation performance using both GPT-4o and GPT-4o mini.
We found that the integration of our TIR module increased all of the GPT-4o mini based simulation models (Standard, CoT, BertKT), which were even better than the simulation models using GPT-4o without the TIR module. Note that GPT-4o mini is a much smaller model than GPT-4o\footnote{https://openai.com/index/gpt-4o-mini-advancing-cost-efficient-intelligence/}. Without TIR, GPT-4o had apparently better simulation performance than GPT-4o mini in Standard and CoT models, as depicted in Fig. \ref{model compare accuracy}(b). However, after integrating the TIR module, both Standard and CoT models in GPT-4o mini outperformed GPT-4o in an obvious margin (Fig. \ref{model compare accuracy}(b)). These results demonstrate that the TIR module could improve the smaller LLMs to learn from example students in the training set more effectively. As a result, smaller LLMs could achieve comparable or even better performance, eliminating the need of using larger size LLMs. 





\subsection{TIR Captures Individual Differences Better}
We then examined whether the models could capture the individual differences and correlation among simulated and real students. Specifically, we used the BertKT with or without the TIR module \mytextcolor{(baseline)} for simulation and compared with the 
label (real students' groundtruth). 
%
This was quantitatively measured by the Pearson correlation between the simulated students' test accuracy sequence along with student IDs and the real students'.    
As depicted in Fig. \ref{individual student correlation}, we found that the integration of the TIR module better captured the correlation between simulated and real students regarding the learning performance sequence along with student IDs than the no TIR case and apparently improved the Pearson correlation from $r=0.02$ (No TIR) to $r=0.42$ (With TIR).
These results demonstrate that our TIR module enabled more realistic simulation by better capturing the individual differences of student learning performance.

\begin{figure}
\centering
\includegraphics[width=1\linewidth]{figures/model_compare_question.pdf}
\caption{\mytextcolor{Heatmap to show the average simulation accuracy (each cell) for each post-test question ID using each model.}}
\Description{
The figure shows the average simulation accuracy for each specific question ID in post-test. 
Results did not find a significant effect of the model on simulation accuracy ($F(5, 30) = 1.09, \, p = 0.387, \, \eta_{p}^{2} = 0.14$). However, significant differences were observed between the following model pairs:
%
BertKT with TIR vs. DKVMN: $t(6) = 3.69, \, p = 0.01, \, \eta_{p}^{2} = 0.46$.
BertKT with TIR vs. simpleKT: $t(6) = 2.59, \, p = 0.041, \, \eta_{p}^{2} = 0.32$.
Although we did not find significance between the BertKT (with TIR) and AKT/ATKT/DKT, it still showed better simulation accuracy of BertKT (with TIR) than AKT/ATKT/DKT for most question IDs.
}
\label{model compare question}
\end{figure}

\mytextcolor{
We further checked the statistical difference of the average simulation accuracy per student with or without the TIR module (baseline).
The normality of the differences between both was assessed using the Shapiro-Wilk test, which indicated no significant deviation from normality ($W = 0.9235, \, p = 0.1493$; df = 17). A paired t-test was then conducted to evaluate the impact of the TIR module on prediction performance. The results showed a statistically significant improvement in accuracy with the TIR module compared to the baseline ($t = 2.4139,\, p = 0.0273$; df = 17). A Bland-Altman analysis revealed a mean difference (bias) of 0.0881 (95\% CI: 0.0186 to 0.1577), with the limits of agreement ranging from -0.2069 (95\% CI: -0.5100 to 0.0962) to 0.3832 (95\% CI: 0.0800 to 0.6863). 
The Bland-Altman plot (Fig. \ref{individual student correlation}(d)) visualizes these findings, showing the mean difference as a dashed red line and the limits of agreement as dashed blue lines. The scatter of points around the mean difference is relatively consistent, suggesting that the agreement between the two models does not vary substantially across the range of predicted accuracy values. 
These results indicate a consistent positive effect of the TIR module on prediction performance, while maintaining acceptable levels of agreement with the baseline model.
}

\begin{figure*}
\centering
\includegraphics[width=1\linewidth]{figures/correlation_student_crop.pdf}
\caption{
\mytextcolor{
Individual-level results: simulations using BertKT with and without TIR across different students.
(a). Correlation between student ID and simulated/real post-test question answer accuracy (vertical bar: standard deviation). (b,c,d,e). The distribution of simulation accuracy differences (b), boxplot (c), Bland-Altman plot to show the mean differences (d), and barplot (e) (error bar: 95\% confidence interval) between two models. (f). Average simulation accuracy and F1 score for each individual student using two models.
}}
\Description{
This figure contains six subplots (labeled a–f) presenting the performance comparison of BertKT with and without TIR for question answer accuracy in simulations. The subplots highlight various statistical and comparative insights based on the dataset.
%
(a) A line plot showing "Question Answer Accuracy" for different student IDs. Three sets of data are represented: the label (ground truth), simulations with TIR (orange solid line with stars), and simulations without TIR (green solid line with stars). Each point represents the average accuracy for a student, with vertical bars indicating the standard deviation. Pearson correlation coefficients for simulations with TIR (r = 0.42) and without TIR (r = 0.02) against the labels are noted on the plot.
%
(b) A histogram showing the distribution of differences between TIR and baseline (no TIR). The x-axis represents the difference in accuracy, while the y-axis represents the frequency of occurrences. A smooth curve overlays the histogram, indicating the approximate probability density. It shows that the data is normally distributed.
%
(c) A boxplot comparing "Accuracy per Student" for simulations with TIR and without TIR. The boxes represent the interquartile range, with the median shown as a line within each box, and whiskers extending to represent the range. It shows that the TIR case has better accuracy.
%
(d) A Bland-Altman plot visualizing the agreement between TIR and baseline performance. The x-axis represents the mean of the two methods (baseline and TIR), while the y-axis shows the difference. Dotted lines indicate the mean difference, upper, and lower limits of agreement (95\% confidence interval). Points are scattered across the plot to show individual observations.
%
(e) A bar chart showing mean performance for simulations with TIR and without TIR, along with error bars representing the 95\% confidence interval. It shows that the TIR case has better accuracy.
%
(f) Two grouped bar plots comparing simulation accuracy (left) and F1 score (right) for each student ID. The bars are color-coded: blue for simulations with TIR and orange for those without TIR. Each student ID is labeled on the y-axis. It shows that the TIR case has better accuracy.
}
\label{individual student correlation}
\end{figure*}



\subsection{TIR Captures Lecture Correlation Better}



% \mytextcolor{The statistical analysis of the data was conducted to assess the impact of the TIR module on student learning performance. The Shapiro-Wilk test for normality showed that the distribution of differences between the baseline and TIR models was not significantly different from a normal distribution (W-statistic = 0.9736, p-value = 0.9444). Given the normal distribution, a paired t-test was performed, yielding a t-statistic of 2.5173 and a p-value of 0.0286, indicating a statistically significant improvement in performance with the TIR module. The degrees of freedom for the paired t-test were 11. Additionally, a Bland-Altman analysis was conducted to assess the agreement between the two models. The mean difference (bias) was found to be 0.0764, with a 95\% confidence interval (CI) of 0.0195 to 0.1334. The limits of agreement were -0.1209 (95\% CI: -0.3263 to 0.0845) for the lower limit and 0.2738 (95\% CI: 0.0684 to 0.4792) for the upper limit, suggesting variability in the differences between the two models, but the positive mean difference indicates that the TIR model generally outperforms the baseline.}

We then examined whether the models could capture the lecture correlation and differences. Specifically, we still used the BertKT with or without the TIR module \mytextcolor{(baseline)} for simulation and compared with the 
%xyz: What's the label? (revised)
label (real students' groundtruth). Since different lectures had their own difficulty, students therefore had different learning performance (post-test question accuracy) across different lectures. Therefore, by comparing the trend of simulated and real students' learning performance along with the lectures, we could see whether the simulation models could capture such variations of lecture difficulty and cross-lecture correlation. 
%
This trend was quantitatively measured by the Pearson correlation between the simulated students'  test accuracy sequence along with lectures and the real students' sequence.    
As depicted in Fig. \ref{lecture correlation}(a), we found that the integration of the TIR module better captured the correlation between simulated and real students regarding the learning performance sequence along with lectures than the no TIR case and apparently improved the Pearson correlation from $r=0.42$ (No TIR) to $r=0.52$ (With TIR).
For more intuitive visualization in individual students, we showed the average question answering accuracy in each lecture for each specific simulated and real student, as depicted in Fig. \ref{lecture correlation}(b,c,d). This visualization also revealed larger similarity between simulated students (with TIR) and real students, compared with the simulation similarity without TIR.
These results demonstrate that our TIR module enables more realistic simulation by better capturing the lecture correlation in student learning performance.

\begin{figure*}
\centering
\includegraphics[width=1\linewidth]{figures/aggre_lecture.pdf}
\caption{
\mytextcolor{
Lecture-level results: simulations using BertKT with and without TIR across different lecture IDs.
(a). Correlation between lecture ID and simulated/real post-test question answer accuracy (vertical bar: standard deviation). (b)(c)(d). Heatmaps of label, BertKT with TIR, and BertKT without TIR question answer accuracy across all students and lectures. Each cell shows the average question answer accuracy of a student answering questions in a specific lecture. Darker color represents higher question answer accuracy. (e,f,g,h). The distribution of simulation accuracy differences (e), boxplot (f), Bland-Altman plot to show the mean differences (g), and barplot (h) (error bar: 95\% confidence interval) between two models. (i). Average simulation accuracy and F1 score for each lecture ID using two models.
}}
\Description{
This figure contains nine subplots (labeled a–i).
(a) shows Question Answer Accuracy Across Lectures: This line plot compares the question answer accuracy of BertKT with TIR (orange line with stars), without TIR (green line with triangles), and the label trend (dotted line) across 12 lectures. Vertical bars represent the standard deviation. The Pearson correlation coefficients are noted: 0.52 (with TIR) and 0.42 (without TIR).
(a) shows that BertKT with TIR more closely aligns with the label accuracy trend compared to the baseline (no TIR), demonstrating higher consistency. The Pearson correlation confirms a stronger association with TIR integration.
(b), (c), (d) show the Heatmaps of Accuracy per Lecture and Student:
Three heatmaps visualize question answer accuracy for the label (b), with TIR (c), and without TIR (d) across 12 lectures and all students (u1 to u76). Darker colors represent higher accuracy.
They show that TIR integration produces patterns closer to the label accuracy heatmap. Without TIR, the accuracy distribution appears more dispersed, indicating weaker alignment with ground truth.
(e) shows the Distribution of Differences (TIR - Baseline):
It shows a histogram showing the distribution of accuracy differences between TIR and baseline (no TIR). The plot includes a smooth density curve for visualizing probability. It shows that the difference is normally distributed.
(f) shows the Performance Comparison by Model (Boxplot):
It presents a boxplot comparing the accuracy per lecture for simulations with and without TIR. The median and interquartile ranges are shown, with whiskers for variability.
It shows that BertKT with TIR achieves consistently higher median accuracy and a narrower interquartile range, signifying both improved and more stable performance.
(g) is a Bland-Altman Plot (Agreement Between Methods):
This plot assesses the agreement between baseline and TIR methods. The x-axis represents the mean accuracy of both methods, and the y-axis shows their difference. Dotted lines indicate the mean difference and the limits of agreement.
It reveals that most points lie within the limits of agreement, indicating reasonable consistency between the methods. However, the positive mean difference suggests that TIR consistently outperforms the baseline.
(h) shows the Mean Performance With Error Bars:
It presents a bar chart compares the mean accuracy across all lectures for TIR and baseline methods, with error bars representing the 95\% confidence interval.
It shows that BertKT with TIR exhibits higher mean accuracy and less overlap between confidence intervals, suggesting a statistically significant improvement.
(i) shows the Simulation Accuracy and F1 Score Comparison (Grouped Bar Charts):
It presents Two bar charts compare simulation accuracy and F1 scores across 12 lectures for TIR and baseline methods.
It shows that TIR achieves higher accuracy and F1 scores in most lectures, highlighting its efficacy. 
}
\label{lecture correlation}
\end{figure*}



\mytextcolor{
We further checked the statistical difference of the average simulation accuracy per lecture with or without the TIR module. The normality of the accuracy differences between them was evaluated using the Shapiro-Wilk test, which indicated no significant deviation from normality ($W = 0.9736, \, p = 0.9444$; df=11). A paired t-test was then performed to assess the impact of the TIR module on simulation performance. The analysis revealed a statistically significant improvement in accuracy with the TIR module compared to the no TIR case ($t = 2.5173, \, p = 0.0286$; df=11). Bland-Altman analysis (Fig. \ref{lecture correlation}(g)) showed a mean difference (bias) of 0.0764 (95\% CI: 0.0195 to 0.1334), with limits of agreement ranging from -0.1209 (95\% CI: -0.3263 to 0.0845) to 0.2738 (95\% CI: 0.0684 to 0.4792). 
% The Bland-Altman plot   visualizes these findings, showing the mean difference as a dashed red line and the limits of agreement as dashed blue lines. The scatter of points around the mean difference is relatively consistent, suggesting that the agreement between the two models does not vary substantially across the range of predicted accuracy values. 
These findings suggest that the TIR module consistently enhances prediction performance while demonstrating acceptable agreement with the baseline model.
}



\subsection{TIR Captures Question Differences Better}
Moreover, we further explored whether the models could capture the different questions' correlation using the BertKT with or without the TIR module. Different questions corresponded to specific knowledge concepts of course materials. Therefore, by comparing the trend of simulated and real students' test accuracy along with different questions, we could see whether the simulation models could capture students' learning performance across fine-grained and varying knowledge concepts. 
This trend was also quantitatively measured by the Pearson correlation between the simulated student question accuracy sequence along with question ID and the real students'.    
As depicted in Fig. \ref{question correlation}(a), we found that the integration of the TIR module better captured the correlation with question ID compared with real cases (label) than the no TIR case and apparently improved the Pearson correlation from $r=-0.50$ (No TIR) to $r=0.37$ (With TIR).
For more intuitive visualization in individual students, we showed the average question answering accuracy in each question for each specific simulated and real student, as depicted in Fig. \ref{question correlation}(b,c,d). This visualization also revealed larger similarity between simulated students (with TIR) and real students, compared with the simulation similarity without TIR.
These results demonstrate that our TIR module enables more realistic simulation by better capturing the question correlation (i.e. knowledge concept correlation) in student learning performance.

\mytextcolor{
We then checked the statistical difference of the average simulation accuracy per question with or without the TIR module (baseline).
The normality of the differences between both was assessed using the Shapiro-Wilk test, which indicated a significant deviation from normality ($W = 0.7451,\, p = 0.0113$; df = 6). Therefore, a Wilcoxon signed-rank test (instead of a paired t-test) was performed to evaluate the impact of the TIR module on prediction performance. The test revealed a statistically significant improvement in accuracy with the TIR module compared to the baseline ($p = 0.0277$; df = 6). A Bland-Altman analysis (Fig. \ref{question correlation}(g)) showed a mean difference (bias) of 0.1437 (95\% CI: 0.0306 to 0.2568), with the limits of agreement ranging from -0.1555 (95\% CI: -0.4754 to 0.1644) to 0.4429 (95\% CI: 0.1230 to 0.7628). These results demonstrate a significant positive effect of the TIR module on prediction performance, with an acceptable level of agreement between the two models.
}

\begin{figure*}
\centering
\includegraphics[width=1\linewidth]{figures/aggre_question.pdf}
\caption{
\mytextcolor{
Question-level results: simulations using BertKT with and without TIR across different post-test question IDs.
(a). Correlation between question ID and simulated/real post-test question answer accuracy (vertical bar: standard deviation). (b)(c)(d). Heatmaps of label, BertKT with TIR, and BertKT without TIR question answer accuracy across all students and questions. Each cell shows the average question answer accuracy of a student answering a specific question. Darker color represents higher question answer accuracy. (e,f,g,h). The distribution of simulation accuracy differences (e), boxplot (f), Bland-Altman plot to show the mean differences (g), and barplot (h) (error bar: 95\% confidence interval) between two models. (i). Average simulation accuracy and F1 score for each post-test question ID using two models.
}}
\Description{
This figure contains nine subplots (labeled a–i).
(a) shows Question Answer Accuracy Across Question IDs: This line plot compares the question answer accuracy of BertKT with TIR (orange line with stars), without TIR (green line with triangles), and the label trend (dotted line) across 7 question IDs. Vertical bars represent the standard deviation. The Pearson correlation coefficients are noted: 0.37 (with TIR) and -0.50 (without TIR).
(a) shows that BertKT with TIR more closely aligns with the label accuracy trend compared to the baseline (no TIR), demonstrating higher consistency. The Pearson correlation confirms a stronger association with TIR integration.
(b), (c), (d) show the Heatmaps of Accuracy per Question ID and Student:
Three heatmaps visualize question answer accuracy for the label (b), with TIR (c), and without TIR (d) across 7 questions and all students (u1 to u76). Darker colors represent higher accuracy.
They show that TIR integration produces patterns closer to the label accuracy heatmap. Without TIR, the accuracy distribution appears more dispersed, indicating weaker alignment with ground truth.
(e) shows the Distribution of Differences (TIR - Baseline):
It shows a histogram showing the distribution of accuracy differences between TIR and baseline (no TIR). The plot includes a smooth density curve for visualizing probability. It shows that the difference is not normally distributed.
(f) shows the Performance Comparison by Model (Boxplot):
It presents a boxplot comparing the accuracy per question ID for simulations with and without TIR. The median and interquartile ranges are shown, with whiskers for variability.
It shows that BertKT with TIR achieves consistently higher median accuracy and a narrower interquartile range, signifying both improved and more stable performance.
(g) is a Bland-Altman Plot (Agreement Between Methods):
This plot assesses the agreement between baseline and TIR methods. The x-axis represents the mean accuracy of both methods, and the y-axis shows their difference. Dotted lines indicate the mean difference and the limits of agreement.
It reveals that most points lie within the limits of agreement, indicating reasonable consistency between the methods. However, the positive mean difference suggests that TIR consistently outperforms the baseline.
(h) shows the Mean Performance With Error Bars:
It presents a bar chart compares the mean accuracy across all question IDs for TIR and baseline methods, with error bars representing the 95\% confidence interval.
It shows that BertKT with TIR exhibits higher mean accuracy and less overlap between confidence intervals, suggesting a statistically significant improvement.
(i) shows the Simulation Accuracy and F1 Score Comparison (Grouped Bar Charts):
It presents Two bar charts compare simulation accuracy and F1 scores across 7 question IDs for TIR and baseline methods.
It shows that TIR achieves higher accuracy and F1 scores in most question IDs, highlighting its efficacy. 
}
\label{question correlation}
\end{figure*}


\subsection{Dynamism of Skill Levels in Learning Path}
\label{subsec: dynamism}
Furthermore, we explored whether the simulation captured the dynamism of students' skill levels in the learning path. Here the learning path referred to the chronological learning process from the first slide to the last slide in the lecture. 
%As mentioned before, each lecture was delivered by an instructor who taught the course based on slides we prepared in advance. 
In our online education system, students' skill levels were represented by the average question answering accuracy where the questions were corresponding to a specific slide. This enabled us to measure to what extent the students mastered the knowledge concepts per slide.         
We still used the BertKT with or without the TIR module for simulation and compared with the label. Then we compared the trend of simulated and real students' skill levels along with the slide ID. 
As depicted in Fig. \ref{slide course correlation}(a), we found that the integration of the TIR module better captured the dynamism of skill levels in the whole learning path from the first slide to the last slide compared with real cases (label) than the no TIR case.
For more intuitive visualization in individual students, we also showed the average question answering accuracy in each slide for each specific simulated and real student, as depicted in Fig. \ref{slide course correlation}(b,c,d). This visualization also revealed larger similarity between simulated students (with TIR) and real students, compared with the simulation similarity without TIR.
This trend was also quantitatively measured by the Pearson correlation between the simulated skill level sequence along with slide ID and the real student sequence along with slide ID. However, since different lectures had different slides, we analyzed the Pearson correlation in each lecture. As depicted in Fig. \ref{slide course correlation}(e), the integration of our TIR module captured better correlation in most lectures.
These results demonstrate that our TIR module enables more realistic simulation by better capturing the dynamism of students' skill levels across the learning path.

\begin{figure*}
\centering
\includegraphics[width=1\linewidth]{figures/aggre_slide.pdf}
\caption{\mytextcolor{(a). Question answer accuracy of simulations using BertKT with and without TIR against the labels on our newly collected dataset across questions related to each slide. Each star point depicts the average question answer accuracy of questions related to that slide while the vertical bar represents the standard deviation. The dotted line represents the label question answer accuracy trend. The solid lines represent the predicted question answer accuracy trend. (b)(c)(d). Heatmaps of label, BertKT with TIR, and BertKT without TIR question answer accuracy across all students and slides. Each cell shows the average student's answer accuracy of questions that are related to the slide. Darker color represents larger question answer accuracy. (e). Pearson correlation between the simulated (BertKT with or without TIR) skill level sequence along with slide ID and the real student sequence along with slide ID in each lecture on our newly collected dataset.}}
\Description{Figure (a) shows the question answer accuracy of simulations using BertKT with and without TIR against the labels on our newly collected dataset across questions related to each slide. It shows that the integration of the TIR module better captured the dynamism of skill levels in the whole learning path from the first slide to the last slide compared with real cases (label) than the no TIR case. Figure (b,c,d) show heatmaps of label, BertKT with TIR, and BertKT without TIR question answer accuracy across all students and slides. Each cell shows the average question answer accuracy related to a specific slide of a student. They show larger similarity between simulated students (with TIR) and real students, compared with the simulation similarity without TIR. Figure (e) shows the Pearson correlation between the simulated (BertKT with or without TIR) skill level sequence along with slide ID and the real student sequence along with slide ID in each lecture on our newly collected dataset. It shows that the integration of our TIR module captured better such correlation in most lectures.}
\label{slide course correlation}
\end{figure*}


\subsection{Fine-Grained Inter-Student Correlation}
One important aspect of contextual simulation was to not only capture the individual differences in learning but also the individual correlations in the same course. Fig. \ref{graph} showed the inter-student correlation for both simulated students (BertKT with or without TIR) and real students. Each node represented one student and two nodes were connected if both students took the same lecture. The color depth of each node represented the average question answering correctness of all lectures that the individual student attended. The weight of the edge connected by two nodes represented the inter-student correlation, which was calculated by the Pearson correlation between the question correctness sequences of two students corresponding to the questions from the overlapped lectures between two students. Note that each student attended multiple lectures. But two students might not attend the same lectures. Therefore, the edge weight only considered the overlapped lectures between two students. However, the color depth of each node considered all lectures that one student had attended. That was why the edge weight might be 1 but the two nodes had different color depth. This meant that the students had the same correctness sequence for the overlapped lectures but their overall accuracy for all lectures attended by each student was different.
%  
As depicted in Fig. \ref{graph}, we found that the integration of the TIR module better captured both individual student learning performance (average question answering correctness represented by the color depth of each node) and inter-student correlation (Pearson correlation of question correctness sequences between two students, represented by the edge weight between two nodes), which were more similar with real students (label), compared with the model without the TIR module.
These results demonstrate that our TIR module enables more realistic and finer-grained simulation by better capturing the inter-student correlation in student learning performance.





% \begin{figure}%[tbhp]
% \centering
% \includegraphics[width=1\linewidth]{figures/dynamism_slide_lecture_crop.pdf}
% \caption{Pearson correlation between the simulated (BertKT with or without TIR) skill level sequence along with slide ID and the real student sequence along with slide ID in each lecture on our newly collected dataset.}
% \Description{This figure shows the Pearson correlation between the simulated (BertKT with or without TIR) skill level sequence along with slide ID and the real student sequence along with slide ID in each lecture on our newly collected dataset. It shows that the integration of our TIR module captured better such correlation in most lectures.}
% \label{slide correlation per lecture}
% \end{figure}
\section{Discussion}
\label{sec: discussion}

In this work, we run a 6-week online education workshop to collect fine-grained annotations of both course materials and student learning performance. This enabled a contextual student simulation to consider the effect of course materials' modulation on student learning. We further improved the student simulation by proposing a transferable iterative reflection module that augmented both prompting-based LLMs' simulation and finetuning-based LLMs' simulation, which achieved even better performance than deep learning models. This was also verified in another public dataset.

\begin{figure*}%[tbhp]
\centering
\includegraphics[width=1\linewidth]{figures/graph.pdf}
\caption{Inter-student correlation graphs of the label (a) and simulations using BertKT with (b) and without TIR (c) on our newly collected dataset where a node is a student and an edge connects two students if both students took the same lecture. The node color depth represents the average question answering correctness of all lectures that the individual student attended. The edge weight connected by two nodes represents the inter-student correlation, which was calculated by the Pearson correlation between the question correctness sequences of two students corresponding to the questions from the overlapped lectures between two students.}
\Description{This figure shows the inter-student correlation graphs of the label (a) and simulations using BertKT with (b) and without TIR (c) on our newly collected dataset where a node is a student and an edge connects two students if both students took the same lecture. The node color depth represents the average question answering correctness of all lectures that the individual student attended. The edge weight connected by two nodes represents the inter-student correlation, which was calculated by the Pearson correlation between the question correctness sequences of two students corresponding to the questions from the overlapped lectures between two students. It shows that the integration of the TIR module better captured both individual student learning performance (average question answering correctness represented by the color depth of each node) and inter-student correlation (Pearson correlation of question correctness sequences between two students, represented by the edge weight between two nodes), which were more similar with real students (label), compared with the model without the TIR module.
}
\label{graph}
\end{figure*}

\subsection{Application Scenarios}

With the increasing importance of AI-assisted education and intelligent tutoring systems, our work could serve as the important groundwork to support a list of applications in educational context.

% cite related papers to support why our simulation is needed and how our simulation can work.

\subsubsection{Student: Enhancing Self-Learning}
The classroom simulacra could create digital twins about a specific student based on the past learning histories. This digital twin further emulates the student's learning performance in the future course materials and tests. This could support the self-assessment and reflections of students to set personalized goals based on their learning pace. Specifically, students could track their learning trends and progress and see how their skills have developed. As a result, it could help students reflect on their strengths and weakness to be improved during learning. Students can also set more appropriate learning goals, milestones, and study priorities based on the simulated results so that they can be motivated to achieve clearer learning targets. Such decision making and reflection in learning are also related to the metacognitive skills of students to develop learning strategies and study habits.

\subsubsection{Instructor: Adapting Teaching Strategy}
Teachers utilizing this system will be able to analyze predictions across an entire class or cohort, allowing them to test pedagogical approaches or curriculum structures before they even deploy them on a class of real human students. Having access to such a representative digital class can allow teachers to use simulation results to tailor the learning experience, presenting students with course contents that truly match their skill levels.
With the classroom simulacra, the instructors could optimize the curriculum design by identifying the common areas of difficulty, which is achieved by analyzing simulation results across the whole class.
For instance, students who turn out to learn fast can accept accelerated course pace while students who may struggle with upcoming tests or course contents can be delivered more related course resources to build personalized learning path. This could also improve the teaching resource allocation by enabling instructors to allocate teaching resources more effectively by identifying skills that may require more time for students to master. Last but not least, this system could provide insights for adaptive interventions of instructors so that students could receive personalized interventions (like verbal reminder or one-on-one tutoring) when they are identified to be at risk of falling behind. 


\subsubsection{Parents: Home Support}
In the home environment, the classroom simulacra can simulate specific children's learning performance based on their past learning history. As a result, parents can have insightful information about how well their children will do in certain curriculum (strengths and weakness), and they could make better decision-making for tasks like choosing the right school, extracurricular activities, choosing advanced courses, and wisely investing in a tutoring service that can target specific areas where their children may struggle. Parents can also support learning outside of school through informed insights about necessary home activities and resources which align with their children's predicted needs and create a conducive study environment tailored to their children’s learning style according to the  performance simulation insights.  



\subsection{Limitations and Future Work}
We also acknowledge the potential limitations in this work.

\subsubsection{Population Diversity}
The first limitation is the population diversity in our online education workshop. We decide to recruit students from elementary schools and high schools because these students usually do not have prior knowledge about our course materials related to Artificial Intelligence. As a result, we could better capture the learning performance and learning outcomes of students when they learn new knowledge. However, we also acknowledge that the data collection with more diverse populations (such as different age groups) could better support and further extend the findings of our experiments.

% \subsubsection{Simulation Resolution}
% Second, our classroom simulacra predicts students' question answering correctness instead of specific question answering choices. This design follows the common practice of existing knowledge tracing approaches \cite{piech2015deepknowledgetracing,10.1145/3474085.3475554,10.1145/3394486.3403282} for student simulation. 
% Because predicting student correctness could directly reveal the skill levels of students on specific course concepts, which can be used for adaptive teaching interventions\cite{hacker2000test}.
% Moreover, student choice prediction is more challenging compared with student correctness prediction. Because students' correctness can be reflected by their past skill levels. If students make a correct choice, it is easy to infer that students may have mastered this related concept. However, if students make a wrong choice, it is hard to obtain the relevant information that may lead to a specific wrong choice by students. 
% However, we acknowledge that simulating students' specific question answering choices could provide further insights about why students make a wrong choice and better support students' learning improvement.
% We leave such exploration for our future work.

\subsubsection{Simulated Behavioral Types}
In our work, we use the students' question answering correctness to represent their learning performance, which can be mapped to students' skill levels on related course concepts. By further mapping them into specific slide IDs in the course (such as Section. \ref{subsec: dynamism}), this simulation can reflect the students' learning behaviors during the course. 
%
\mytextcolor{However, we acknowledge that student behaviors are not limited to question correctness. \mytextcolor{We clarify that the cognitive states information was only used for teachers to obtain students' learning states during data collection, and was not used for behavioral simulation.}
There are also more diverse learning behaviors in the real world scenarios such as reasoning processes, learning reflections, personal preferences, learning styles, etc. For example, students' cognitive states (such as attention, confusion) during the course could directly reflect their learning styles and personal preferences in the course. The simulation for such additional learning behaviors could provide further insights and evidence support. We also believe that LLMs have great potential in simulating such diverse behaviors due to the strong in-context learning ability \cite{wei2023larger} and large knowledge base \cite{alkhamissi2022review}. 
Therefore, such simulation on more diverse learning behavioral types could be the future directions for exploration. 
}

\mytextcolor{
%xyzr1: This paragraph needs to be rewritten. It's too abrupt. Unclear why suddently you talk about this. 
\subsubsection{Generalization and Cost Differences}
We clarify that our goal is to show that TIR can augment LLMs to achieve better performance than themselves without TIR and deep learning models. We have a fair comparison within each LLMs-based model with or without TIR using the same training/testing data. But we do not intend to directly compare prompting-based LLMs with finetuning-based LLMs. Because both have their own pros and cons. For example, prompting-based LLMs need less training data but finetuning-based LLMs have better simulation performance. 
However, for future potential applications to extend the fine-tuned models in external datasets, it is also necessary to fine-tune models again in such new datasets, which is similar to deep learning models that use training data to update model weights. 
As such, it is not comparable/applicable to directly compare both regarding the generality or training time/computational resources. 
%
When comparing with deep learning, we mainly use finetuning-based LLMs for a fair comparison because they use the same training data. However, using much less training data, the TIR-augmented prompting-based LLMs can also achieve comparable or even better performance than deep learning. This also demonstrates the effectiveness of our TIR module. 
%
}

\mytextcolor{
\subsubsection{Insights for Educators}
Our current classroom simulacra serves as a student simulation model. Integrating it into an end-to-end intelligent teaching system entails non-trivial efforts. Nevertheless, the classroom simulacra is grounded in a real-world student-educator interaction dataset. 
\mytextcolor{Its predictive capability aligns with proven educational models that have been utilized to successfully inform teaching practices and support adaptive learning strategies \cite{scholtz2021systematic}. The foundational accuracy of our model indicates a strong potential for real-world applicability, as seen in similar simulations that have influenced educational strategies even before empirical testing \cite{xing2019dropout}}.
%
The simulator can support educators by delivering actionable insights that enhance personalized interventions, curriculum design, and evidence-based teaching practices. It can identify specific knowledge gaps for individual students, enabling targeted interventions, and allows educators to explore hypothetical scenarios to optimize teaching strategies for diverse learner profiles. Additionally, the simulator aids in curriculum optimization by simulating student responses to different teaching methodologies, helping to refine pacing and content sequencing. Therefore, the simulator provides a research-backed tool for testing the impact of instructional methods and predicting long-term outcomes. 
\mytextcolor{Beyond learning analytics, it integrates behavioral insights to detect learning issues and offers a safe experimental environment for innovative teaching approaches. Case studies in our work illustrate its practical utility, such as identifying impactful topics for exam preparation based on students' different learning performance on different questions (question-level) and guiding classroom time allocation based on students' learning performance across different slides (slide-level). 
In conclusion, while real-world educator experiences would strengthen our findings, the current study offers a solid foundation that demonstrates the simulator’s predictive power and practical potential. Future work that includes educator feedback will further bolster our understanding and validation of its effectiveness in real classroom settings.
% We recognize the importance of direct educator input for validating the practical value of our tool. 
% To that end, we plan to initiate a pilot study involving educators who can test the simulator within real educational contexts and provide firsthand feedback on its utility and impact. This will help refine our model further and build a more comprehensive understanding of its real-world effectiveness.
}
% It is worth noting that similar educational tools and simulations have been developed and adopted without immediate real-world data, yet they have significantly contributed to teaching practices upon subsequent use. For example, adaptive learning platforms and predictive analytics tools have frequently been integrated into educational settings after their predictive reliability has been demonstrated through controlled studies.
% Even without immediate real-world data, our study contributes to the field by providing a proof of concept that educators can use as a starting point to incorporate predictive data into their teaching. By leveraging insights from the simulator, educators may be better equipped to understand student learning patterns and make data-driven decisions that enhance educational outcomes.
}

% \mytextcolor{
% \subsubsection{Insights for Educators}
% Our student simulator addresses key educational research challenges by offering predictive insights grounded in validated algorithms and a real-world student-educator interaction dataset. 
% While we acknowledge the current limitation of not having direct educator feedback using our student simulator, the simulator demonstrates significant potential for enhancing teaching practices through its sophisticated predictive capabilities. 
% The model can anticipate student learning performance and knowledge gaps, enabling proactive instructional adjustments across various educational contexts. By supporting educators in identifying individual student learning needs, exploring hypothetical teaching scenarios, optimizing curriculum design, and simulating the impact of different instructional methods, the simulator provides a robust tool for data-driven pedagogical strategies. 
% Furthermore, this predictive capability aligns with proven educational models that have been utilized to inform teaching practices and support adaptive learning strategies. The foundational accuracy of our model indicates a strong potential for real-world applicability, as seen in similar simulations that have influenced educational strategies even before empirical testing \cite{xing2019dropout}.
% The approach is anchored in established educational theories and has shown promising results in preliminary case studies, such as identifying impactful exam preparation topics and guiding classroom time allocation based on detailed learning performance analysis. 
% Despite the absence of immediate real-world educator experiences, our methodology follows a precedent in educational technology development, where similar adaptive learning platforms and predictive analytics tools have been successfully integrated into educational settings after demonstrating predictive reliability through controlled studies. 
% To further validate the simulator's practical utility, we propose a pilot study that will engage educators in testing the tool within actual classroom contexts, collect direct feedback, and refine the model based on real-world insights. In essence, our study provides a compelling proof of concept that represents a promising step toward more personalized, data-driven educational strategies that have the potential to transform teaching and learning experiences.
% }
\section{Conclusion \& Future Work}\label{conclusion}
This work presents XAMBA, the first framework optimizing SSMs on COTS NPUs, removing the need for specialized accelerators. XAMBA mitigates key bottlenecks in SSMs like CumSum, ReduceSum, and activations using ActiBA, CumBA, and ReduBA, transforming sequential operations into parallel computations. These optimizations improve latency, throughput (Tokens/s), and memory efficiency. Future work will extend XAMBA to other models, explore compression, and develop dynamic optimizations for broader hardware platforms.



% This work introduces XAMBA, the first framework to optimize SSMs on COTS NPUs, eliminating the need for specialized hardware accelerators. XAMBA addresses key bottlenecks in SSM execution, including CumSum, ReduceSum, and activation functions, through techniques like ActiBA, CumBA, and ReduBA, which restructure sequential operations into parallel matrix computations. These optimizations reduce latency, enhance throughput, and improve memory efficiency. 
% Experimental results show up to 2.6$\times$ performance improvement on Intel\textregistered\ Core\texttrademark\ Ultra Series 2 AI PC. 
% Future work will extend XAMBA to other models, incorporate compression techniques, and explore dynamic optimization strategies for broader hardware platforms.


% This work presents XAMBA, an optimization framework that enhances the performance of SSMs on NPUs. Unlike transformers, SSMs rely on structured state transitions and implicit recurrence, which introduce sequential dependencies that challenge efficient hardware execution. XAMBA addresses these inefficiencies by introducing CumBA, ReduBA, and ActiBA, which optimize cumulative summation, ReduceSum, and activation functions, respectively, significantly reducing latency and improving throughput. By restructuring sequential computations into parallelizable matrix operations and leveraging specialized hardware acceleration, XAMBA enables efficient execution of SSMs on NPUs. Future work will extend XAMBA to other state-space models, integrate advanced compression techniques like pruning and quantization, and explore dynamic optimization strategies to further enhance performance across various hardware platforms and frameworks.
% This work presents XAMBA, an optimization framework that enhances the performance of SSMs on NPUs. Key techniques, including CumBA, ReduBA, and ActiBA, achieve significant latency reductions by optimizing operations like cumulative summation, ReduceSum, and activation functions. Future work will focus on extending XAMBA to other state-space models, integrating advanced compression techniques, and exploring dynamic optimization strategies to further improve performance across various hardware platforms and frameworks.

% This work introduces XAMBA, an optimization framework for improving the performance of Mamba-2 and Mamba models on NPUs. XAMBA includes three key techniques: CumBA, ReduBA, and ActiBA. CumBA reduces latency by transforming cumulative summation operations into matrix multiplication using precomputed masks. ReduBA optimizes the ReduceSum operation through matrix-vector multiplication, reducing execution time. ActiBA accelerates activation functions like Swish and Softplus by mapping them to specialized hardware during the DPU’s drain phase, avoiding sequential execution bottlenecks. Additionally, XAMBA enhances memory efficiency by reducing SRAM access, increasing data reuse, and utilizing Zero Value Compression (ZVC) for masks. The framework provides significant latency reductions, with CumBA, ReduBA, and ActiBA achieving up to 1.8X, 1.1X, and 2.6X reductions, respectively, compared to the baseline.
% Future work includes extending XAMBA to other state-space models (SSMs) and exploring further hardware optimizations for emerging NPUs. Additionally, integrating advanced compression techniques like pruning and quantization, and developing adaptive strategies for dynamic optimization, could enhance performance. Expanding XAMBA's compatibility with other frameworks and deployment environments will ensure broader adoption across various hardware platforms.



%%
%% The acknowledgments section is defined using the "acks" environment
%% (and NOT an unnumbered section). This ensures the proper
%% identification of the section in the article metadata, and the
%% consistent spelling of the heading.
\begin{acks}
This work is supported by National Science Foundation CNS-2403124, CNS-2312715, CNS-2128588 and the University of California San Diego Center for Wireless Communications.
\end{acks}

%%
%% The next two lines define the bibliography style to be used, and
%% the bibliography file.
\bibliographystyle{ACM-Reference-Format}
\bibliography{sample-base}


%%
%% If your work has an appendix, this is the place to put it.
\appendix

\subsection{Lloyd-Max Algorithm}
\label{subsec:Lloyd-Max}
For a given quantization bitwidth $B$ and an operand $\bm{X}$, the Lloyd-Max algorithm finds $2^B$ quantization levels $\{\hat{x}_i\}_{i=1}^{2^B}$ such that quantizing $\bm{X}$ by rounding each scalar in $\bm{X}$ to the nearest quantization level minimizes the quantization MSE. 

The algorithm starts with an initial guess of quantization levels and then iteratively computes quantization thresholds $\{\tau_i\}_{i=1}^{2^B-1}$ and updates quantization levels $\{\hat{x}_i\}_{i=1}^{2^B}$. Specifically, at iteration $n$, thresholds are set to the midpoints of the previous iteration's levels:
\begin{align*}
    \tau_i^{(n)}=\frac{\hat{x}_i^{(n-1)}+\hat{x}_{i+1}^{(n-1)}}2 \text{ for } i=1\ldots 2^B-1
\end{align*}
Subsequently, the quantization levels are re-computed as conditional means of the data regions defined by the new thresholds:
\begin{align*}
    \hat{x}_i^{(n)}=\mathbb{E}\left[ \bm{X} \big| \bm{X}\in [\tau_{i-1}^{(n)},\tau_i^{(n)}] \right] \text{ for } i=1\ldots 2^B
\end{align*}
where to satisfy boundary conditions we have $\tau_0=-\infty$ and $\tau_{2^B}=\infty$. The algorithm iterates the above steps until convergence.

Figure \ref{fig:lm_quant} compares the quantization levels of a $7$-bit floating point (E3M3) quantizer (left) to a $7$-bit Lloyd-Max quantizer (right) when quantizing a layer of weights from the GPT3-126M model at a per-tensor granularity. As shown, the Lloyd-Max quantizer achieves substantially lower quantization MSE. Further, Table \ref{tab:FP7_vs_LM7} shows the superior perplexity achieved by Lloyd-Max quantizers for bitwidths of $7$, $6$ and $5$. The difference between the quantizers is clear at 5 bits, where per-tensor FP quantization incurs a drastic and unacceptable increase in perplexity, while Lloyd-Max quantization incurs a much smaller increase. Nevertheless, we note that even the optimal Lloyd-Max quantizer incurs a notable ($\sim 1.5$) increase in perplexity due to the coarse granularity of quantization. 

\begin{figure}[h]
  \centering
  \includegraphics[width=0.7\linewidth]{sections/figures/LM7_FP7.pdf}
  \caption{\small Quantization levels and the corresponding quantization MSE of Floating Point (left) vs Lloyd-Max (right) Quantizers for a layer of weights in the GPT3-126M model.}
  \label{fig:lm_quant}
\end{figure}

\begin{table}[h]\scriptsize
\begin{center}
\caption{\label{tab:FP7_vs_LM7} \small Comparing perplexity (lower is better) achieved by floating point quantizers and Lloyd-Max quantizers on a GPT3-126M model for the Wikitext-103 dataset.}
\begin{tabular}{c|cc|c}
\hline
 \multirow{2}{*}{\textbf{Bitwidth}} & \multicolumn{2}{|c|}{\textbf{Floating-Point Quantizer}} & \textbf{Lloyd-Max Quantizer} \\
 & Best Format & Wikitext-103 Perplexity & Wikitext-103 Perplexity \\
\hline
7 & E3M3 & 18.32 & 18.27 \\
6 & E3M2 & 19.07 & 18.51 \\
5 & E4M0 & 43.89 & 19.71 \\
\hline
\end{tabular}
\end{center}
\end{table}

\subsection{Proof of Local Optimality of LO-BCQ}
\label{subsec:lobcq_opt_proof}
For a given block $\bm{b}_j$, the quantization MSE during LO-BCQ can be empirically evaluated as $\frac{1}{L_b}\lVert \bm{b}_j- \bm{\hat{b}}_j\rVert^2_2$ where $\bm{\hat{b}}_j$ is computed from equation (\ref{eq:clustered_quantization_definition}) as $C_{f(\bm{b}_j)}(\bm{b}_j)$. Further, for a given block cluster $\mathcal{B}_i$, we compute the quantization MSE as $\frac{1}{|\mathcal{B}_{i}|}\sum_{\bm{b} \in \mathcal{B}_{i}} \frac{1}{L_b}\lVert \bm{b}- C_i^{(n)}(\bm{b})\rVert^2_2$. Therefore, at the end of iteration $n$, we evaluate the overall quantization MSE $J^{(n)}$ for a given operand $\bm{X}$ composed of $N_c$ block clusters as:
\begin{align*}
    \label{eq:mse_iter_n}
    J^{(n)} = \frac{1}{N_c} \sum_{i=1}^{N_c} \frac{1}{|\mathcal{B}_{i}^{(n)}|}\sum_{\bm{v} \in \mathcal{B}_{i}^{(n)}} \frac{1}{L_b}\lVert \bm{b}- B_i^{(n)}(\bm{b})\rVert^2_2
\end{align*}

At the end of iteration $n$, the codebooks are updated from $\mathcal{C}^{(n-1)}$ to $\mathcal{C}^{(n)}$. However, the mapping of a given vector $\bm{b}_j$ to quantizers $\mathcal{C}^{(n)}$ remains as  $f^{(n)}(\bm{b}_j)$. At the next iteration, during the vector clustering step, $f^{(n+1)}(\bm{b}_j)$ finds new mapping of $\bm{b}_j$ to updated codebooks $\mathcal{C}^{(n)}$ such that the quantization MSE over the candidate codebooks is minimized. Therefore, we obtain the following result for $\bm{b}_j$:
\begin{align*}
\frac{1}{L_b}\lVert \bm{b}_j - C_{f^{(n+1)}(\bm{b}_j)}^{(n)}(\bm{b}_j)\rVert^2_2 \le \frac{1}{L_b}\lVert \bm{b}_j - C_{f^{(n)}(\bm{b}_j)}^{(n)}(\bm{b}_j)\rVert^2_2
\end{align*}

That is, quantizing $\bm{b}_j$ at the end of the block clustering step of iteration $n+1$ results in lower quantization MSE compared to quantizing at the end of iteration $n$. Since this is true for all $\bm{b} \in \bm{X}$, we assert the following:
\begin{equation}
\begin{split}
\label{eq:mse_ineq_1}
    \tilde{J}^{(n+1)} &= \frac{1}{N_c} \sum_{i=1}^{N_c} \frac{1}{|\mathcal{B}_{i}^{(n+1)}|}\sum_{\bm{b} \in \mathcal{B}_{i}^{(n+1)}} \frac{1}{L_b}\lVert \bm{b} - C_i^{(n)}(b)\rVert^2_2 \le J^{(n)}
\end{split}
\end{equation}
where $\tilde{J}^{(n+1)}$ is the the quantization MSE after the vector clustering step at iteration $n+1$.

Next, during the codebook update step (\ref{eq:quantizers_update}) at iteration $n+1$, the per-cluster codebooks $\mathcal{C}^{(n)}$ are updated to $\mathcal{C}^{(n+1)}$ by invoking the Lloyd-Max algorithm \citep{Lloyd}. We know that for any given value distribution, the Lloyd-Max algorithm minimizes the quantization MSE. Therefore, for a given vector cluster $\mathcal{B}_i$ we obtain the following result:

\begin{equation}
    \frac{1}{|\mathcal{B}_{i}^{(n+1)}|}\sum_{\bm{b} \in \mathcal{B}_{i}^{(n+1)}} \frac{1}{L_b}\lVert \bm{b}- C_i^{(n+1)}(\bm{b})\rVert^2_2 \le \frac{1}{|\mathcal{B}_{i}^{(n+1)}|}\sum_{\bm{b} \in \mathcal{B}_{i}^{(n+1)}} \frac{1}{L_b}\lVert \bm{b}- C_i^{(n)}(\bm{b})\rVert^2_2
\end{equation}

The above equation states that quantizing the given block cluster $\mathcal{B}_i$ after updating the associated codebook from $C_i^{(n)}$ to $C_i^{(n+1)}$ results in lower quantization MSE. Since this is true for all the block clusters, we derive the following result: 
\begin{equation}
\begin{split}
\label{eq:mse_ineq_2}
     J^{(n+1)} &= \frac{1}{N_c} \sum_{i=1}^{N_c} \frac{1}{|\mathcal{B}_{i}^{(n+1)}|}\sum_{\bm{b} \in \mathcal{B}_{i}^{(n+1)}} \frac{1}{L_b}\lVert \bm{b}- C_i^{(n+1)}(\bm{b})\rVert^2_2  \le \tilde{J}^{(n+1)}   
\end{split}
\end{equation}

Following (\ref{eq:mse_ineq_1}) and (\ref{eq:mse_ineq_2}), we find that the quantization MSE is non-increasing for each iteration, that is, $J^{(1)} \ge J^{(2)} \ge J^{(3)} \ge \ldots \ge J^{(M)}$ where $M$ is the maximum number of iterations. 
%Therefore, we can say that if the algorithm converges, then it must be that it has converged to a local minimum. 
\hfill $\blacksquare$


\begin{figure}
    \begin{center}
    \includegraphics[width=0.5\textwidth]{sections//figures/mse_vs_iter.pdf}
    \end{center}
    \caption{\small NMSE vs iterations during LO-BCQ compared to other block quantization proposals}
    \label{fig:nmse_vs_iter}
\end{figure}

Figure \ref{fig:nmse_vs_iter} shows the empirical convergence of LO-BCQ across several block lengths and number of codebooks. Also, the MSE achieved by LO-BCQ is compared to baselines such as MXFP and VSQ. As shown, LO-BCQ converges to a lower MSE than the baselines. Further, we achieve better convergence for larger number of codebooks ($N_c$) and for a smaller block length ($L_b$), both of which increase the bitwidth of BCQ (see Eq \ref{eq:bitwidth_bcq}).


\subsection{Additional Accuracy Results}
%Table \ref{tab:lobcq_config} lists the various LOBCQ configurations and their corresponding bitwidths.
\begin{table}
\setlength{\tabcolsep}{4.75pt}
\begin{center}
\caption{\label{tab:lobcq_config} Various LO-BCQ configurations and their bitwidths.}
\begin{tabular}{|c||c|c|c|c||c|c||c|} 
\hline
 & \multicolumn{4}{|c||}{$L_b=8$} & \multicolumn{2}{|c||}{$L_b=4$} & $L_b=2$ \\
 \hline
 \backslashbox{$L_A$\kern-1em}{\kern-1em$N_c$} & 2 & 4 & 8 & 16 & 2 & 4 & 2 \\
 \hline
 64 & 4.25 & 4.375 & 4.5 & 4.625 & 4.375 & 4.625 & 4.625\\
 \hline
 32 & 4.375 & 4.5 & 4.625& 4.75 & 4.5 & 4.75 & 4.75 \\
 \hline
 16 & 4.625 & 4.75& 4.875 & 5 & 4.75 & 5 & 5 \\
 \hline
\end{tabular}
\end{center}
\end{table}

%\subsection{Perplexity achieved by various LO-BCQ configurations on Wikitext-103 dataset}

\begin{table} \centering
\begin{tabular}{|c||c|c|c|c||c|c||c|} 
\hline
 $L_b \rightarrow$& \multicolumn{4}{c||}{8} & \multicolumn{2}{c||}{4} & 2\\
 \hline
 \backslashbox{$L_A$\kern-1em}{\kern-1em$N_c$} & 2 & 4 & 8 & 16 & 2 & 4 & 2  \\
 %$N_c \rightarrow$ & 2 & 4 & 8 & 16 & 2 & 4 & 2 \\
 \hline
 \hline
 \multicolumn{8}{c}{GPT3-1.3B (FP32 PPL = 9.98)} \\ 
 \hline
 \hline
 64 & 10.40 & 10.23 & 10.17 & 10.15 &  10.28 & 10.18 & 10.19 \\
 \hline
 32 & 10.25 & 10.20 & 10.15 & 10.12 &  10.23 & 10.17 & 10.17 \\
 \hline
 16 & 10.22 & 10.16 & 10.10 & 10.09 &  10.21 & 10.14 & 10.16 \\
 \hline
  \hline
 \multicolumn{8}{c}{GPT3-8B (FP32 PPL = 7.38)} \\ 
 \hline
 \hline
 64 & 7.61 & 7.52 & 7.48 &  7.47 &  7.55 &  7.49 & 7.50 \\
 \hline
 32 & 7.52 & 7.50 & 7.46 &  7.45 &  7.52 &  7.48 & 7.48  \\
 \hline
 16 & 7.51 & 7.48 & 7.44 &  7.44 &  7.51 &  7.49 & 7.47  \\
 \hline
\end{tabular}
\caption{\label{tab:ppl_gpt3_abalation} Wikitext-103 perplexity across GPT3-1.3B and 8B models.}
\end{table}

\begin{table} \centering
\begin{tabular}{|c||c|c|c|c||} 
\hline
 $L_b \rightarrow$& \multicolumn{4}{c||}{8}\\
 \hline
 \backslashbox{$L_A$\kern-1em}{\kern-1em$N_c$} & 2 & 4 & 8 & 16 \\
 %$N_c \rightarrow$ & 2 & 4 & 8 & 16 & 2 & 4 & 2 \\
 \hline
 \hline
 \multicolumn{5}{|c|}{Llama2-7B (FP32 PPL = 5.06)} \\ 
 \hline
 \hline
 64 & 5.31 & 5.26 & 5.19 & 5.18  \\
 \hline
 32 & 5.23 & 5.25 & 5.18 & 5.15  \\
 \hline
 16 & 5.23 & 5.19 & 5.16 & 5.14  \\
 \hline
 \multicolumn{5}{|c|}{Nemotron4-15B (FP32 PPL = 5.87)} \\ 
 \hline
 \hline
 64  & 6.3 & 6.20 & 6.13 & 6.08  \\
 \hline
 32  & 6.24 & 6.12 & 6.07 & 6.03  \\
 \hline
 16  & 6.12 & 6.14 & 6.04 & 6.02  \\
 \hline
 \multicolumn{5}{|c|}{Nemotron4-340B (FP32 PPL = 3.48)} \\ 
 \hline
 \hline
 64 & 3.67 & 3.62 & 3.60 & 3.59 \\
 \hline
 32 & 3.63 & 3.61 & 3.59 & 3.56 \\
 \hline
 16 & 3.61 & 3.58 & 3.57 & 3.55 \\
 \hline
\end{tabular}
\caption{\label{tab:ppl_llama7B_nemo15B} Wikitext-103 perplexity compared to FP32 baseline in Llama2-7B and Nemotron4-15B, 340B models}
\end{table}

%\subsection{Perplexity achieved by various LO-BCQ configurations on MMLU dataset}


\begin{table} \centering
\begin{tabular}{|c||c|c|c|c||c|c|c|c|} 
\hline
 $L_b \rightarrow$& \multicolumn{4}{c||}{8} & \multicolumn{4}{c||}{8}\\
 \hline
 \backslashbox{$L_A$\kern-1em}{\kern-1em$N_c$} & 2 & 4 & 8 & 16 & 2 & 4 & 8 & 16  \\
 %$N_c \rightarrow$ & 2 & 4 & 8 & 16 & 2 & 4 & 2 \\
 \hline
 \hline
 \multicolumn{5}{|c|}{Llama2-7B (FP32 Accuracy = 45.8\%)} & \multicolumn{4}{|c|}{Llama2-70B (FP32 Accuracy = 69.12\%)} \\ 
 \hline
 \hline
 64 & 43.9 & 43.4 & 43.9 & 44.9 & 68.07 & 68.27 & 68.17 & 68.75 \\
 \hline
 32 & 44.5 & 43.8 & 44.9 & 44.5 & 68.37 & 68.51 & 68.35 & 68.27  \\
 \hline
 16 & 43.9 & 42.7 & 44.9 & 45 & 68.12 & 68.77 & 68.31 & 68.59  \\
 \hline
 \hline
 \multicolumn{5}{|c|}{GPT3-22B (FP32 Accuracy = 38.75\%)} & \multicolumn{4}{|c|}{Nemotron4-15B (FP32 Accuracy = 64.3\%)} \\ 
 \hline
 \hline
 64 & 36.71 & 38.85 & 38.13 & 38.92 & 63.17 & 62.36 & 63.72 & 64.09 \\
 \hline
 32 & 37.95 & 38.69 & 39.45 & 38.34 & 64.05 & 62.30 & 63.8 & 64.33  \\
 \hline
 16 & 38.88 & 38.80 & 38.31 & 38.92 & 63.22 & 63.51 & 63.93 & 64.43  \\
 \hline
\end{tabular}
\caption{\label{tab:mmlu_abalation} Accuracy on MMLU dataset across GPT3-22B, Llama2-7B, 70B and Nemotron4-15B models.}
\end{table}


%\subsection{Perplexity achieved by various LO-BCQ configurations on LM evaluation harness}

\begin{table} \centering
\begin{tabular}{|c||c|c|c|c||c|c|c|c|} 
\hline
 $L_b \rightarrow$& \multicolumn{4}{c||}{8} & \multicolumn{4}{c||}{8}\\
 \hline
 \backslashbox{$L_A$\kern-1em}{\kern-1em$N_c$} & 2 & 4 & 8 & 16 & 2 & 4 & 8 & 16  \\
 %$N_c \rightarrow$ & 2 & 4 & 8 & 16 & 2 & 4 & 2 \\
 \hline
 \hline
 \multicolumn{5}{|c|}{Race (FP32 Accuracy = 37.51\%)} & \multicolumn{4}{|c|}{Boolq (FP32 Accuracy = 64.62\%)} \\ 
 \hline
 \hline
 64 & 36.94 & 37.13 & 36.27 & 37.13 & 63.73 & 62.26 & 63.49 & 63.36 \\
 \hline
 32 & 37.03 & 36.36 & 36.08 & 37.03 & 62.54 & 63.51 & 63.49 & 63.55  \\
 \hline
 16 & 37.03 & 37.03 & 36.46 & 37.03 & 61.1 & 63.79 & 63.58 & 63.33  \\
 \hline
 \hline
 \multicolumn{5}{|c|}{Winogrande (FP32 Accuracy = 58.01\%)} & \multicolumn{4}{|c|}{Piqa (FP32 Accuracy = 74.21\%)} \\ 
 \hline
 \hline
 64 & 58.17 & 57.22 & 57.85 & 58.33 & 73.01 & 73.07 & 73.07 & 72.80 \\
 \hline
 32 & 59.12 & 58.09 & 57.85 & 58.41 & 73.01 & 73.94 & 72.74 & 73.18  \\
 \hline
 16 & 57.93 & 58.88 & 57.93 & 58.56 & 73.94 & 72.80 & 73.01 & 73.94  \\
 \hline
\end{tabular}
\caption{\label{tab:mmlu_abalation} Accuracy on LM evaluation harness tasks on GPT3-1.3B model.}
\end{table}

\begin{table} \centering
\begin{tabular}{|c||c|c|c|c||c|c|c|c|} 
\hline
 $L_b \rightarrow$& \multicolumn{4}{c||}{8} & \multicolumn{4}{c||}{8}\\
 \hline
 \backslashbox{$L_A$\kern-1em}{\kern-1em$N_c$} & 2 & 4 & 8 & 16 & 2 & 4 & 8 & 16  \\
 %$N_c \rightarrow$ & 2 & 4 & 8 & 16 & 2 & 4 & 2 \\
 \hline
 \hline
 \multicolumn{5}{|c|}{Race (FP32 Accuracy = 41.34\%)} & \multicolumn{4}{|c|}{Boolq (FP32 Accuracy = 68.32\%)} \\ 
 \hline
 \hline
 64 & 40.48 & 40.10 & 39.43 & 39.90 & 69.20 & 68.41 & 69.45 & 68.56 \\
 \hline
 32 & 39.52 & 39.52 & 40.77 & 39.62 & 68.32 & 67.43 & 68.17 & 69.30  \\
 \hline
 16 & 39.81 & 39.71 & 39.90 & 40.38 & 68.10 & 66.33 & 69.51 & 69.42  \\
 \hline
 \hline
 \multicolumn{5}{|c|}{Winogrande (FP32 Accuracy = 67.88\%)} & \multicolumn{4}{|c|}{Piqa (FP32 Accuracy = 78.78\%)} \\ 
 \hline
 \hline
 64 & 66.85 & 66.61 & 67.72 & 67.88 & 77.31 & 77.42 & 77.75 & 77.64 \\
 \hline
 32 & 67.25 & 67.72 & 67.72 & 67.00 & 77.31 & 77.04 & 77.80 & 77.37  \\
 \hline
 16 & 68.11 & 68.90 & 67.88 & 67.48 & 77.37 & 78.13 & 78.13 & 77.69  \\
 \hline
\end{tabular}
\caption{\label{tab:mmlu_abalation} Accuracy on LM evaluation harness tasks on GPT3-8B model.}
\end{table}

\begin{table} \centering
\begin{tabular}{|c||c|c|c|c||c|c|c|c|} 
\hline
 $L_b \rightarrow$& \multicolumn{4}{c||}{8} & \multicolumn{4}{c||}{8}\\
 \hline
 \backslashbox{$L_A$\kern-1em}{\kern-1em$N_c$} & 2 & 4 & 8 & 16 & 2 & 4 & 8 & 16  \\
 %$N_c \rightarrow$ & 2 & 4 & 8 & 16 & 2 & 4 & 2 \\
 \hline
 \hline
 \multicolumn{5}{|c|}{Race (FP32 Accuracy = 40.67\%)} & \multicolumn{4}{|c|}{Boolq (FP32 Accuracy = 76.54\%)} \\ 
 \hline
 \hline
 64 & 40.48 & 40.10 & 39.43 & 39.90 & 75.41 & 75.11 & 77.09 & 75.66 \\
 \hline
 32 & 39.52 & 39.52 & 40.77 & 39.62 & 76.02 & 76.02 & 75.96 & 75.35  \\
 \hline
 16 & 39.81 & 39.71 & 39.90 & 40.38 & 75.05 & 73.82 & 75.72 & 76.09  \\
 \hline
 \hline
 \multicolumn{5}{|c|}{Winogrande (FP32 Accuracy = 70.64\%)} & \multicolumn{4}{|c|}{Piqa (FP32 Accuracy = 79.16\%)} \\ 
 \hline
 \hline
 64 & 69.14 & 70.17 & 70.17 & 70.56 & 78.24 & 79.00 & 78.62 & 78.73 \\
 \hline
 32 & 70.96 & 69.69 & 71.27 & 69.30 & 78.56 & 79.49 & 79.16 & 78.89  \\
 \hline
 16 & 71.03 & 69.53 & 69.69 & 70.40 & 78.13 & 79.16 & 79.00 & 79.00  \\
 \hline
\end{tabular}
\caption{\label{tab:mmlu_abalation} Accuracy on LM evaluation harness tasks on GPT3-22B model.}
\end{table}

\begin{table} \centering
\begin{tabular}{|c||c|c|c|c||c|c|c|c|} 
\hline
 $L_b \rightarrow$& \multicolumn{4}{c||}{8} & \multicolumn{4}{c||}{8}\\
 \hline
 \backslashbox{$L_A$\kern-1em}{\kern-1em$N_c$} & 2 & 4 & 8 & 16 & 2 & 4 & 8 & 16  \\
 %$N_c \rightarrow$ & 2 & 4 & 8 & 16 & 2 & 4 & 2 \\
 \hline
 \hline
 \multicolumn{5}{|c|}{Race (FP32 Accuracy = 44.4\%)} & \multicolumn{4}{|c|}{Boolq (FP32 Accuracy = 79.29\%)} \\ 
 \hline
 \hline
 64 & 42.49 & 42.51 & 42.58 & 43.45 & 77.58 & 77.37 & 77.43 & 78.1 \\
 \hline
 32 & 43.35 & 42.49 & 43.64 & 43.73 & 77.86 & 75.32 & 77.28 & 77.86  \\
 \hline
 16 & 44.21 & 44.21 & 43.64 & 42.97 & 78.65 & 77 & 76.94 & 77.98  \\
 \hline
 \hline
 \multicolumn{5}{|c|}{Winogrande (FP32 Accuracy = 69.38\%)} & \multicolumn{4}{|c|}{Piqa (FP32 Accuracy = 78.07\%)} \\ 
 \hline
 \hline
 64 & 68.9 & 68.43 & 69.77 & 68.19 & 77.09 & 76.82 & 77.09 & 77.86 \\
 \hline
 32 & 69.38 & 68.51 & 68.82 & 68.90 & 78.07 & 76.71 & 78.07 & 77.86  \\
 \hline
 16 & 69.53 & 67.09 & 69.38 & 68.90 & 77.37 & 77.8 & 77.91 & 77.69  \\
 \hline
\end{tabular}
\caption{\label{tab:mmlu_abalation} Accuracy on LM evaluation harness tasks on Llama2-7B model.}
\end{table}

\begin{table} \centering
\begin{tabular}{|c||c|c|c|c||c|c|c|c|} 
\hline
 $L_b \rightarrow$& \multicolumn{4}{c||}{8} & \multicolumn{4}{c||}{8}\\
 \hline
 \backslashbox{$L_A$\kern-1em}{\kern-1em$N_c$} & 2 & 4 & 8 & 16 & 2 & 4 & 8 & 16  \\
 %$N_c \rightarrow$ & 2 & 4 & 8 & 16 & 2 & 4 & 2 \\
 \hline
 \hline
 \multicolumn{5}{|c|}{Race (FP32 Accuracy = 48.8\%)} & \multicolumn{4}{|c|}{Boolq (FP32 Accuracy = 85.23\%)} \\ 
 \hline
 \hline
 64 & 49.00 & 49.00 & 49.28 & 48.71 & 82.82 & 84.28 & 84.03 & 84.25 \\
 \hline
 32 & 49.57 & 48.52 & 48.33 & 49.28 & 83.85 & 84.46 & 84.31 & 84.93  \\
 \hline
 16 & 49.85 & 49.09 & 49.28 & 48.99 & 85.11 & 84.46 & 84.61 & 83.94  \\
 \hline
 \hline
 \multicolumn{5}{|c|}{Winogrande (FP32 Accuracy = 79.95\%)} & \multicolumn{4}{|c|}{Piqa (FP32 Accuracy = 81.56\%)} \\ 
 \hline
 \hline
 64 & 78.77 & 78.45 & 78.37 & 79.16 & 81.45 & 80.69 & 81.45 & 81.5 \\
 \hline
 32 & 78.45 & 79.01 & 78.69 & 80.66 & 81.56 & 80.58 & 81.18 & 81.34  \\
 \hline
 16 & 79.95 & 79.56 & 79.79 & 79.72 & 81.28 & 81.66 & 81.28 & 80.96  \\
 \hline
\end{tabular}
\caption{\label{tab:mmlu_abalation} Accuracy on LM evaluation harness tasks on Llama2-70B model.}
\end{table}

%\section{MSE Studies}
%\textcolor{red}{TODO}


\subsection{Number Formats and Quantization Method}
\label{subsec:numFormats_quantMethod}
\subsubsection{Integer Format}
An $n$-bit signed integer (INT) is typically represented with a 2s-complement format \citep{yao2022zeroquant,xiao2023smoothquant,dai2021vsq}, where the most significant bit denotes the sign.

\subsubsection{Floating Point Format}
An $n$-bit signed floating point (FP) number $x$ comprises of a 1-bit sign ($x_{\mathrm{sign}}$), $B_m$-bit mantissa ($x_{\mathrm{mant}}$) and $B_e$-bit exponent ($x_{\mathrm{exp}}$) such that $B_m+B_e=n-1$. The associated constant exponent bias ($E_{\mathrm{bias}}$) is computed as $(2^{{B_e}-1}-1)$. We denote this format as $E_{B_e}M_{B_m}$.  

\subsubsection{Quantization Scheme}
\label{subsec:quant_method}
A quantization scheme dictates how a given unquantized tensor is converted to its quantized representation. We consider FP formats for the purpose of illustration. Given an unquantized tensor $\bm{X}$ and an FP format $E_{B_e}M_{B_m}$, we first, we compute the quantization scale factor $s_X$ that maps the maximum absolute value of $\bm{X}$ to the maximum quantization level of the $E_{B_e}M_{B_m}$ format as follows:
\begin{align}
\label{eq:sf}
    s_X = \frac{\mathrm{max}(|\bm{X}|)}{\mathrm{max}(E_{B_e}M_{B_m})}
\end{align}
In the above equation, $|\cdot|$ denotes the absolute value function.

Next, we scale $\bm{X}$ by $s_X$ and quantize it to $\hat{\bm{X}}$ by rounding it to the nearest quantization level of $E_{B_e}M_{B_m}$ as:

\begin{align}
\label{eq:tensor_quant}
    \hat{\bm{X}} = \text{round-to-nearest}\left(\frac{\bm{X}}{s_X}, E_{B_e}M_{B_m}\right)
\end{align}

We perform dynamic max-scaled quantization \citep{wu2020integer}, where the scale factor $s$ for activations is dynamically computed during runtime.

\subsection{Vector Scaled Quantization}
\begin{wrapfigure}{r}{0.35\linewidth}
  \centering
  \includegraphics[width=\linewidth]{sections/figures/vsquant.jpg}
  \caption{\small Vectorwise decomposition for per-vector scaled quantization (VSQ \citep{dai2021vsq}).}
  \label{fig:vsquant}
\end{wrapfigure}
During VSQ \citep{dai2021vsq}, the operand tensors are decomposed into 1D vectors in a hardware friendly manner as shown in Figure \ref{fig:vsquant}. Since the decomposed tensors are used as operands in matrix multiplications during inference, it is beneficial to perform this decomposition along the reduction dimension of the multiplication. The vectorwise quantization is performed similar to tensorwise quantization described in Equations \ref{eq:sf} and \ref{eq:tensor_quant}, where a scale factor $s_v$ is required for each vector $\bm{v}$ that maps the maximum absolute value of that vector to the maximum quantization level. While smaller vector lengths can lead to larger accuracy gains, the associated memory and computational overheads due to the per-vector scale factors increases. To alleviate these overheads, VSQ \citep{dai2021vsq} proposed a second level quantization of the per-vector scale factors to unsigned integers, while MX \citep{rouhani2023shared} quantizes them to integer powers of 2 (denoted as $2^{INT}$).

\subsubsection{MX Format}
The MX format proposed in \citep{rouhani2023microscaling} introduces the concept of sub-block shifting. For every two scalar elements of $b$-bits each, there is a shared exponent bit. The value of this exponent bit is determined through an empirical analysis that targets minimizing quantization MSE. We note that the FP format $E_{1}M_{b}$ is strictly better than MX from an accuracy perspective since it allocates a dedicated exponent bit to each scalar as opposed to sharing it across two scalars. Therefore, we conservatively bound the accuracy of a $b+2$-bit signed MX format with that of a $E_{1}M_{b}$ format in our comparisons. For instance, we use E1M2 format as a proxy for MX4.

\begin{figure}
    \centering
    \includegraphics[width=1\linewidth]{sections//figures/BlockFormats.pdf}
    \caption{\small Comparing LO-BCQ to MX format.}
    \label{fig:block_formats}
\end{figure}

Figure \ref{fig:block_formats} compares our $4$-bit LO-BCQ block format to MX \citep{rouhani2023microscaling}. As shown, both LO-BCQ and MX decompose a given operand tensor into block arrays and each block array into blocks. Similar to MX, we find that per-block quantization ($L_b < L_A$) leads to better accuracy due to increased flexibility. While MX achieves this through per-block $1$-bit micro-scales, we associate a dedicated codebook to each block through a per-block codebook selector. Further, MX quantizes the per-block array scale-factor to E8M0 format without per-tensor scaling. In contrast during LO-BCQ, we find that per-tensor scaling combined with quantization of per-block array scale-factor to E4M3 format results in superior inference accuracy across models. 


\end{document}
\endinput
%%
%% End of file `sample-sigconf.tex'.
