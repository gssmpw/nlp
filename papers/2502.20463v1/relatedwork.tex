\section{Related work}
\subsection{Avoiding political discussions}

Over decades, scholars have suggested different explanations for why individuals may avoid engaging in political discussions. Eliasoph in \textit{Avoiding politics} argues that rather than an aversion to politics, lack of political talk is more because of a fragile public sphere where ``intelligence, curiosity, and generosity have evaporated'' \cite{eliasoph1998avoiding}. Noelle-Neumann’s spiral of silence theory suggests that individuals avoid expressing their opinions when they are perceived to be unpopular \cite{noelle1974spiral}. Others attribute avoidance to individual predispositions such as conflict avoidance \cite{ulbig1999conflict} and the Big Five personality traits \cite{hibbing2011personality}. Hostility \cite{matthes2013hostile} and disagreement \cite{gerber2012disagreement} have also been identified as factors in political discussion avoidance. More recently, through a series of studies, Carlson and Settle highlight the importance of political knowledge in the decision to not engage in political discussions \cite{carlson2022goes}. Specifically, in one study, they find that one of the most commonly cited reasons for political discussion avoidance was the respondent's lack of accurate political information or knowledge \cite{carlson2022goes}.

While a large majority avoid or only rarely engage in political discussions, a small but vocal minority of individuals who are deeply involved in politics dominate political discussions \cite{krupnikov2022other}. These individuals are highly politically engaged and are often extreme partisans who are hostile towards those they disagree with \cite{krupnikov2022other}. Thus, most political discussions that we observe online are invariably between these deeply involved individuals. A public sphere with largely the deeply involved engaging in discussions is especially problematic. Interactions predominantly between deeply involved individuals are likely to result in more extreme views and more polarization \cite{sunstein2001republic}. Further, the visibility of deeply involved individuals engaging in hostile discussions, coupled with an increased media focus on them, often distorts perceptions of polarization and who is the median partisan \cite{levendusky2016mis,bail2022breaking,krupnikov2022other}. Finally, as Berelson et al. \cite{berelson1954voting} in their foundational work on opinion formation note, a mass democracy cannot function if all individuals are deeply involved. A wider distribution of individuals based on political involvement may facilitate compromise, avert extreme partisanship and provide room for consensus and stability in democratic decision-making \cite{berelson1954voting}. Thus, it is crucial that political discussions include individuals who do not usually engage with politics. We argue that personal narratives can facilitate a wider engagement among the politically disinclined.


\subsection{Personal narratives in political discussions}

Personal narratives are usually first-person accounts that recount individual experiences. Engaging with personal narratives requires minimal prior knowledge or training, allowing individuals from diverse backgrounds to participate meaningfully in political discourse \cite{young2002inclusion}. Personal narratives also make political discussions more accessible and create a safe space for dialogue, encouraging participation even among those hesitant to express their views publicly \cite{patterson1998narrative}. Moreover, personal narratives help situate individual stories within the context of larger political systems, enabling individuals to see how their experiences relate to broader societal issues \cite{riessman2003analysis}. Political apathy, another key reason for political inactivity, can also be countered through personal narratives as they create connections across ideological differences, encourage engagement via relatable storytelling, and highlight the impact of individual experiences within broader political and social systems \cite{jones2012contesting}. Finally, personal narratives help build inclusive spaces by emphasizing individual experiences as valid forms of knowledge, which is crucial for engaging PDIs who may feel excluded from traditional political discourses \cite{stivers1993reflections}. 

Personal narratives may positively impact the substance of the political discussions as well. These narratives often humanize the narrator, either by showcasing their positive qualities or revealing their vulnerabilities, creating a sense of connection with readers \cite{robinson1981personal}. This connection is further strengthened by the empathy personal narratives can evoke, making readers more open to considering opposing viewpoints \cite{shuman2006entitlement}. Personal narratives can also inspire political action by turning personal stories into powerful tools for raising awareness and building solidarity \cite{robinson2016sharing}. Similarly, employing these narratives in political discourse also grants users tools for persuasion in political contexts, as these narratives have the ability to influence attitudes and beliefs, unlike data-heavy arguments, as stories often appeal directly to emotions \cite{green2000role}. Somewhat counterintuitively, personal narratives also enhance perceptions of rationality in political discussions. When people use personal narratives to express their political views, they are often seen as more rational, garnering greater respect even from opposing partisan groups \cite{kubin2021personal}.  Finally, personal narratives can amplify the voices of underrepresented groups, challenge mainstream narratives, and inspire hope while advocating for change \cite{10.1093/oso/9780190851712.003.0004}. These studies suggest that personal narratives may not only encourage PDIs to participate in political discussions but may also substantively improve the quality of the discussions.

\subsection{Identifying personal narratives}

Research on identifying and extracting personal narratives from textual data has evolved over the years. Gordon and Swanson laid the groundwork by creating a standard corpus of personal narratives from blog posts and using statistical models to classify them \cite{gordon2009identifying}. Subsequent research explored various statistical methods for narrative identification, focusing on different linguistic characteristics. Yao and Huang examined temporal characteristics \cite{yao2018temporal}, while Ceran et al. investigated the density of part-of-speech tags and named entities \cite{ceran2012semantic}.
Researchers then expanded their approaches to incorporate semantic information. Eisenberg and Finlayson utilized verb and character features \cite{eisenberg2017simpler}, and Dirkson et al. leveraged psycholinguistic features to improve narrative identification \cite{dirkson2019narrative}. 
Recent studies have demonstrated the potential of transformer-based models in personal narrative identification. Ganti et al. conducted a study where they documented the performance of various transformer-based models \cite{ganti2023narrative}, while Anotoniak et al. successfully fine-tuned these models to identify narratives at both document and span levels across different domains \cite{antoniak2023people}. Falk and Lapesa further validated the robustness of transformer-based models in identifying narratives in argumentation settings \cite{falk-lapesa-2022-reports}. Generative models, including large language models,  have also been used to identify personal narratives \cite{ganti2023narrative}, but they performed worse than fine-tuning a transformer-based model.