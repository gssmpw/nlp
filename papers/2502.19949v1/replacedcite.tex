\section{Related Work}
\label{related-work}}

\textbf{Machine Learning Approaches.} Typically, all approaches for
clinical prediction tasks based on PPG data rely on a combination of
signal processing and machine learning, where the precise focus varies
considerably across different approaches. At one end of the spectrum,
approaches relying on clinically interpretable features focus heavily on
signal processing to extract meaningful features and typically use
comparably simple classification/regression models to perform the
prediction. At the other end of the spectrum, deep learning methods
using raw time series as input with as little signal processing
as possible rely on complex prediction models to extract and process
meaningful features by themselves. In between, there are classifiers
based on image-representations, that leverage signal processing tools to
turn raw waveforms into image representations and then most commonly
also rely on deep learning models to perform the prediction.

\textbf{Time Series Models.} Recent advancements in deep learning have
significantly impacted healthcare by enabling complicated analysis of
raw physiological data. These models are particularly effective in
estimating BP and detecting AF, among other applications. The key
advantage of deep learning is its ability to recognize complicated
patterns directly from raw data such as electrocardiogram (ECG) and PPG
signals, eliminating the need for extensive manual feature development
____. They provide an effective means to learn the complex,
nonlinear underlying relationship between PPG signals and various
physiological parameters, without the need to define a convenient
analytical form for such a transformation. Consequently, deep learning
architectures such as convolutional neural networks (CNNs) and recurrent
neural networks (RNNs) have shown remarkable performance in BP
estimation, by capturing the temporal and spatial nuances inherent in
raw physiological signals ____. Similarly, deep learning models have
demonstrated significant potential in detecting AF from raw ECG signals.
The authors of ____ developed a deep learning model employing a CNN to
detect AF from single-lead ECG recordings. Their model demonstrated high
accuracy, underscoring the potential of deep learning in the detection
of arrhythmias. Similarly, ____ implemented deep learning approaches
using CNNs on PPG signals for AF detection. These approaches yielded
results that were competitive with ECG-based methods, thereby
demonstrating the feasibility of utilizing PPG signals for AF detection.
End-to-end deep learning frameworks that process raw ECG data to
generate AF predictions have simplified and improved the accuracy of AF
detection.

\textbf{Feature-Based Models.} Besides working on raw data, another
possibility to solve classification or regression tasks on PPG data is
to establish machine learning models operating on clinically
interpretable PPG features. These features include features based on
pulse morphology, e.g. PPG pulse wave features such as the systolic
peak, the diastolic peak or pulse duration and PPG derivative features
(6,17), and irregularity features based on measures of randomness,
variability and complexity in the inter-beat-intervals that can be
determined from the PPG ____. These clinically interpretable PPG
features are used for BP estimation (see below), AF detection (see
below) and other questions related to the cardiovascular system, e.g.
the assessment of arterial stiffness ____. In addition, PPG signals show
sometimes a strict periodic behaviour or, in general, a quasi-periodic
behaviour. This motivates the use of features from Fourier- ____,
Wavelet- ____ or Hilbert-Huang- ____ Transformations for training
models. Such models are defined for BP estimation and AF detection. In
related work, features from Fourier-Transformation are used to define
models for the detection of aneurysms ____ or stenoses ____. As PPG
signals are in general not periodic, but quasiperiodic, one might expect
better results with Wavelet- or Hilbert-Huang-transformations.

\textbf{Image-Based Models.} Image-based models, such as those using the
Continuous Wavelet Transform (CWT), convert physiological signals into
visual representations, enabling deep learning to analyze, classify, and
estimate physiological parameters. The CWT is a powerful tool for
analyzing localized variations of power within a time series signal.
Unlike the Fourier Transform, which provides a global frequency
representation, CWT can provide a time-frequency representation that
preserves the temporal localization of features. CWT based scalograms
have already been used for PPG signal transformation to classify BP
(Normal, Prehypertension, Stage 1 hypertension and Stage 2 hypertension)
____, estimate heart rate variability (HRV) and signal quality ____
as well as to detect atrial fibrillation ____.

\hypertarget{materials-and-methods}{%