\begin{table}[htbp]
    \centering
    \caption{Example of Unreliable LLM Annotation with Very Low Reliability Score}
    \begin{tabular}{|p{3cm}|p{9cm}|}
        \hline
        \multicolumn{2}{|c|}{\textbf{Paper Information}} \\
        \hline
        \textbf{Title} & Setting health care priorities in Oregon. Cost-effectiveness meets the rule of rescue \\
        \hline
        \textbf{Abstract} & The Oregon Health Services Commission recently completed work on its principal charge: creation of a prioritized list of health care services, ranging from the most important to the least important. Oregon's draft priority list was criticized because it seemed to favor minor treatments over lifesaving ones. This reaction reflects a fundamental and irreconcilable conflict between cost-effectiveness analysis and the powerful human proclivity to rescue endangered life: the "Rule of Rescue." Oregon's final priority list was generated without reference to costs and is, therefore, more intuitively sensible than the initial list. However, the utility of the final list is limited by its lack of specificity with regard to conditions and treatments. An alternative approach for setting health care priorities would circumvent the Rule of Rescue by carefully defining necessary indications for treatment. Such an approach might be applied to Oregon's final list in order to achieve better specificity. \\
        \hline
        \multicolumn{2}{|c|}{\textbf{Annotation Comparison}} \\
        \hline
        \textbf{Expert Annotation} & Interesting Hypothesis \\
        \hline
        \textbf{LLM Annotation} & \textit{Llama-3.1-8B probability distribution (assessed independently and then normalized):} \\
        & Interesting Hypothesis: 0.326 \\
        & Technical Advance: 0.337 \\
        & New Finding: 0.337 \\
        \hline
        \textbf{Reliability Score} & 0.00 (Very Low Reliability) \\
        \hline
        
        \hline
    \end{tabular}
    \label{tab:unreliable_example}
\end{table}