\begin{table}[htbp]
    \centering
    \caption{Example of Low Reliablilty LLM Annotation}
    \begin{tabular}{|p{3cm}|p{9cm}|}
        \hline
        \multicolumn{2}{|c|}{\textbf{Paper Information}} \\
        \hline
        \textbf{Title} & Dynamics of age-structured and spatially structured predator-prey interactions: individual-based models and population-level formulations. \\
        \hline
        \textbf{Abstract} & In this article, we investigate the spatial and temporal dynamics of predator and prey populations using an individual-based modeling approach. In our models, the individual is the fundamental unit, and the dynamics are governed by individual rules for growth, movement, reproduction, feeding, and mortality. We first establish the congruence between age-structured predator-prey population models and the corresponding individual-based population model under homogeneous spatial conditions. Given the agreement between the formalisms, we then use the individual-based model to investigate the dynamics of spatially structured predator-prey systems. In particular, we contrast the dynamics of predator-prey systems in which predators adopt either an ``ambush'' or a ``cruising'' strategy. We show that the stability of the spatially structured predator-prey system depends on the relative mobility of prey and predators and that prey mobility, in particular, has a strong effect on stability. Local density dependence in prey reproduction can quantitatively alter the asymmetrical influence of prey mobility on stability, but we show that the asymmetry exists when local density dependence is removed. We hypothesize that this asymmetrical response is due to prey ``escape'' in space caused by differences in rates of spread of prey and predator populations that arise because of fundamental differences between prey and predator reproduction. \\
        \hline
        \multicolumn{2}{|c|}{\textbf{Annotation Comparison}} \\
        \hline
        \textbf{Expert Annotation} & New Finding \\
        \hline
        \textbf{LLM Annotation} & \textit{Llama-3.1-405B probability distribution(assessed independently and then normalized):} \\
        & Interesting Hypothesis: 0.117 \\
        & Technical Advance: 0.384 \\
        & New Finding: 0.498 \\
        \hline
        \textbf{Reliability Score} & 0.13 (Low Reliability) \\
        \hline
        
    \end{tabular}
    \label{tab:moderate_reliability_example}
\end{table}