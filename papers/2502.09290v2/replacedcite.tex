\section{Literature Review}
\label{LR}

Recent research has explored methods for optimizing the scheduling of EV charging and discharging to enhance their value through interactions with households, buildings, local energy markets, and the grid. For example, Li et al. ____ presented an electric-hydrogen integrated energy system that utilize V2G to alleviate peak demand and bridge energy gaps, thus reducing dependency on the utility grid. In ____, Bijan et al. investigated the impact of V2G operations and energy transactions on the local grid's reliability, costs, and emissions, taking into account the variability of renewable energy sources. Shurrab et al. ____ proposed a V2V energy-sharing model designed to optimize the social welfare while meeting charging requirements. Al-Obaidi et al. ____ proposed an EV scheduling strategy for P2P energy trading and ancillary services. Thompson et al. ____ developed a comprehensive framework to assess the economic benefits of V2X, including V2B, V2V, and V2G interactions. Nonetheless, the effectiveness of these multiple value streams from EVs could be hindered by local network constraints. Without considering these constraints, the findings might lack credibility. 

Recent investigations into EV charging scheduling and coordination have increasingly taken into account the constraints of distribution networks. For example, a framework for managing residential EVs was proposed in ____ to leverage V2G capabilities in mitigating grid overloads and maintaining grid voltage. Feng et al. ____ presented a P2P energy trading framework using a generalized fast dual ascent method, incorporating power network constraints to protect the distribution network's security. Affolabi et al. ____ designed a dual-level market structure for facilitating energy transactions among EVs in charging stations, taking into account the voltage constraints of the power network. Hoque et al. ____ proposed an EV energy trading framework that incorporates network export-import limits, known as the dynamic operating envelope (DOE), considering both network constraints and prosumer preferences. However, the energy system's uncertainties, like renewable energy generation, fluctuating loads, and diverse EV arrival times, pose significant challenges to scheduling EV charging. Such uncertainties can result in EV scheduling decisions that may not be viable or violate network constraints. Despite improved forecasts for renewable energy output and load demands, prediction inaccuracies remain, and the impact on the effectiveness of EV scheduling and coordination is underexplored. 

Real-time scheduling and coordination of EV charging can mitigate the adverse impacts of uncertainties, with RHO emerging as an effective method. RHO has been applied to handle uncertainties in energy system management. For example, the study in ____ introduced an RHO strategy for EVs in electricity exchange markets to handle volatile electricity prices. Nimalsiri et al. ____ developed an RHO-based approach for managing the uncertainties related to EV charging and discharging schedules, ensuring voltage levels remain safe. Hou et al. ____ presented a chance-constrained programming-based rolling horizon method to address the uncertainties of EV charging demand in the energy hub system. Zaneti et al. ____ utilized RHO to minimize the operational costs of a charging station considering uncertainties in the electric bus arrival time and solar PV generation. Mohy-ud-din et al. ____ presented a forward-looking energy management approach using RHO for virtual power plant operations, effectively countering the variability in renewable energy sources and demand. Wang et al. ____ employed RHO to adjust power usage, taking into account the variability of renewables, temperature predictions, and consumer behavior in a tri-level distribution market model. Wu et al. ____ developed a two-layer control framework that incorporates RHO and a multi-uncertainty sampling method aiming to reduce forecasting errors for EV arrivals and minimize power exchange between the main network and the microgrid. Nevertheless, prediction inaccuracies are unavoidable and often adversely affect the performance of V2X services. These studies did not specifically analyze how various uncertainties impact system operations and the benefits of V2X value stacking. 

A preliminary investigation into the impact of uncertainties on the EV value-stacking model was conducted by ____, considering factors like residential household electricity consumption and rooftop PV generation. However, many existing studies above treated EV arrivals as deterministic elements in the system, neglecting the impact of uncertainty in EV arrivals on system operations. In contrast, this work aims to bridge the research gap by exploring V2X value stacking via EV coordination that satisfies power network constraints and quantifying the impact of prediction errors in building load, renewable energy, and EV arrivals within residential communities. Through a comprehensive evaluation, our work seeks to advance the understanding of V2X value stacking and highlight the significance of various uncertainties in V2X value-stacking.