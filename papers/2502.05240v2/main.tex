\documentclass[lettersize,journal]{IEEEtran}
\usepackage{amsmath,amsfonts}
\usepackage{algorithmic}
\usepackage{algorithm}
\usepackage{array}
\usepackage[caption=false,font=normalsize,labelfont=sf,textfont=sf]{subfig}
\usepackage{textcomp}
\usepackage{stfloats}
\usepackage{url}
\usepackage{verbatim}
\usepackage{graphicx}
\usepackage{lipsum} 
\usepackage{cite}
\hyphenation{op-tical net-works semi-conduc-tor IEEE-Xplore}

\usepackage{times}
\usepackage{microtype}
\usepackage{epsfig}
\usepackage[table,xcdraw]{xcolor}
\usepackage{graphicx}
\usepackage{caption}
\usepackage{float}
\usepackage{placeins}
\usepackage{color, colortbl}
\usepackage{stfloats}
\usepackage{enumitem}
\usepackage{tabularx}
\usepackage{xstring}
\usepackage{multirow}
\usepackage{xspace}
\usepackage{url}
\usepackage{subcaption}
\usepackage{xcolor}
\usepackage{tcolorbox}
\usepackage[hang,flushmargin]{footmisc}
\usepackage{color}
\usepackage{tikz}
\usepackage{bbding}
\usepackage{makecell}

\usepackage[edges]{forest}
\definecolor{hidden-draw}{RGB}{20,68,106}
\definecolor{hidden-pink}{RGB}{255,245,247}
\newcommand{\etal}{{\emph{et al.}}}
\definecolor{lightgrey}{gray}{0.92}
\definecolor{light\_double\_grey}{gray}{0.95}
\definecolor{lightred}{RGB}{251,49,153}
\definecolor{lightorange}{RGB}{244,230,217}

\definecolor{LightRed}{rgb}{1,0.92,0.92}
\definecolor{LightBlue}{rgb}{0.9,0.94,1}
\definecolor{LightGreen}{rgb}{0.9,1.0,0.88}
\newcommand{\lightgraytext}[1]{\textcolor[rgb]{0.5,0.5,0.5}{#1}}

\newcommand{\paratitle}[1]{\vspace{1.5ex}\noindent\textbf{#1}}
\newcommand{\definedsection}[1]{\noindent\textit{#1}\vspace{1.0ex}}

\newcommand{\thickcline}[1]{%
    \arrayrulecolor{black}\Xcline{#1}{2.5pt}%
    \arrayrulecolor{black}% 
}

\usepackage{hyperref}       % hyperlinks
\usepackage{url}            % simple URL typesetting
\usepackage{booktabs}       % professional-quality tables
\usepackage{amsfonts}       % blackboard math symbols
\usepackage{nicefrac}       % compact symbols for 1/2, etc.
\usepackage{microtype}      % microtypography
\usepackage{xcolor}         % colors

\newcommand{\zekun}[1]{\textcolor{blue}{[Zekun: #1]}}
\newcommand{\issue}[1]{\textcolor{red}{[Issue: #1]}}


\begin{document}


\title{Survey on AI-Generated Media Detection: From Non-MLLM to MLLM}

\author{Yueying Zou, Peipei Li*, Zekun Li, Huaibo Huang, Xing Cui, \\Xuannan Liu, Chenghanyu Zhang, Ran He,~\IEEEmembership{Fellow,~IEEE}
        % <-this % stops a space  
\thanks{Yueying Zou, Peipei Li, Xing Cui, and Xuannan Liu are with the School of Artificial Intelligence, and Chenghanyu Zhang is with the School of Science, all at Beijing University of Posts and Telecommunications, Beijing 100876, China. E-mail: {zouyueying2001, lipeipei, cuixing, liuxuannan, zhangchenghanyu}@bupt.edu.cn.}
\thanks{Zekun Li is with the School of Computer Science, University of California, Santa Barbara, USA. E-mail: zekunli@cs.ucsb.edu.}
\thanks{Huaibo Huang and Ran He are with the State Key Laboratory of Multimodal Artificial Intelligence Systems, CASIA, New Laboratory of Pattern Recognition, CASIA, and School of Artificial Intelligence, University of Chinese Academy of Sciences, Beijing 100190, China. E-mail: {huaibo.huang, rhe}@cripac.ia.ac.cn.}
\thanks{Peipei Li* is the corresponding author. E-mail: lipeipei@bupt.edu.cn.}
}
%\thanks{Huaibo Huang and Ran He are with the State Key Laboratory of Multimodal Artificial Intelligence Systems, CASIA, New Laboratory of Pattern Recognition, CASIA, and School of Artificial Intelligence, University of Chinese Academy of Sciences, Beijing 100190, China. E-mail: {huaibo.huang, rhe}@cripac.ia.ac.cn.}

% <-this % stops a space

%\thanks{Manuscript received April 19, 2021; revised August 16, 2021.}

% The paper headers
\markboth{Journal of \LaTeX\ Class Files,~Vol.~14, No.~8, August~2021}%
{Shell \MakeLowercase{\textit{et al.}}: A Sample Article Using IEEEtran.cls for IEEE Journals}

%\IEEEpubid{0000--0000/00\$00.00~\copyright~2021 IEEE}
% Remember, if you use this you must call \IEEEpubidadjcol in the second
% column for its text to clear the IEEEpubid mark.

\maketitle

\begin{abstract}
The proliferation of AI-generated media poses significant challenges to information authenticity and social trust, making reliable detection methods highly demanded. Methods for detecting AI-generated media have evolved rapidly, paralleling the advancement of Multimodal Large Language Models (MLLMs). Current detection approaches can be categorized into two main groups: Non-MLLM-based and MLLM-based methods. The former employs high-precision, domain-specific detectors powered by deep learning techniques, while the latter utilizes general-purpose detectors based on MLLMs that integrate authenticity verification, explainability, and localization capabilities. Despite significant progress in this field, there remains a gap in literature regarding a comprehensive survey that examines the transition from domain-specific to general-purpose detection methods. This paper addresses this gap by providing a systematic review of both approaches, analyzing them from single-modal and multi-modal perspectives. We present a detailed comparative analysis of these categories, examining their methodological similarities and differences. Through this analysis, we explore potential hybrid approaches and identify key challenges in forgery detection, providing direction for future research. Additionally, as MLLMs become increasingly prevalent in detection tasks, ethical and security considerations have emerged as critical global concerns. We examine the regulatory landscape surrounding Generative AI (GenAI) across various jurisdictions, offering valuable insights for researchers and practitioners in this field.

\end{abstract}

\begin{IEEEkeywords}
AI-generated Media detection, MLLM, deep learning, literarture survey
%\LaTeX, paper, template, typesetting.
\end{IEEEkeywords}

\section{Introduction}
\label{sec:intro}

\begin{figure*}[tb]
    \centering
    \includegraphics[width=0.848\linewidth]{figs/circuitnn.pdf} 
    \caption{Illustration of differentiable CircuitNN. CircuitNN is designed based on differentiable NAND gates. After DAS is guided by PI and PO pairs of the truth table, CircuitNN can get the precise circuit architecture logic equivalent to the truth table.}
    \label{fig:circuitnn}
\end{figure*}

% 1. Describe the importance of logic synthesis
% 2. Existing Problems
% (a) Neural Architecture Search: Unstable, Predefined Setting, etc.
% (b) Circuit Generation: Probabilistic Model, Logic Equivalence

With the rapid advancement of technology, the scale of integrated circuits (ICs) has expanded exponentially. 
This expansion has introduced significant challenges in chip manufacturing, particularly concerning power and area metrics.
A primary objective in IC design is achieving the same circuit function with fewer transistors, thereby reducing power usage and area occupancy.

Logic synthesis~\cite{hachtel2005logicsynth}, a critical step in electronic design automation (EDA), transforms behavioral-level circuit designs into optimized gate-level circuits, ultimately yielding the final IC layout. 
The primary goal of logic synthesis is to identify the physical implementation with the fewest gates for a given circuit function. 
This task constitutes a challenging NP-hard combinatorial optimization problem. 
Current logic synthesis tools~\cite{brayton2010abc, wolf2013yosys} rely on human-designed heuristics, often leading to sub-optimal outcomes.

Differentiable architecture search (DAS) techniques~\cite{liu2018darts, chu2020darts} offer novel perspectives on addressing challenges in this problem.
Circuit functions can be represented through truth tables, which map binary inputs to their corresponding outputs. 
Truth tables provide a precise representation of input-output relationships, ensuring the design of functionally equivalent circuits.
Inspired by this, researchers~\cite{deepmind2024ai4sys, wang2024tnet} have begun exploring the application of DAS to synthesize circuits directly from truth tables.
Specifically, \citet{deepmind2024ai4sys} proposed CircuitNN, a framework that learns differentiable connection structures with logic gates, enabling the automatic generation of logic circuits from truth tables.
This approach significantly reduces the complexity of traditional circuit generation. 
Building on this, \citet{wang2024tnet} introduced T-Net, a triangle-shaped variant of CircuitNN, incorporating regularization techniques to enhance the efficiency of DAS.

Despite these advancements, several challenges remain. 
The computational complexity of DAS grows quadratically with the number of gates, posing scalability issues.
Although triangle-shaped architecture~\cite{wang2024tnet} partially mitigates this problem, redundancy persists. 
%Additionally, DAS is susceptible to converging to local optima, limiting the ability to search architectures that satisfy the given truth tables~\cite{liu2018darts}. 
%Furthermore, hyperparameters (network depth and layer width) require extensive searches, introducing complexity and prolonging the synthesis process. 
Additionally, DAS is susceptible to converging to local optima~\cite{liu2018darts} and hyperparameters (network depth and layer width) require extensive searches. 
The challenges arise from the vast search space in DAS. 
% Even with predefined settings for CircuitNN, finding a configuration that meets the truth table requires extensive trial and error during the DAS process. 
Intuitively, limiting the search space through predefined parameters (network depth, gates per layer, and connection probabilities) can significantly reduce the complexity.

Recent advances~\cite{openai2023gpt4, abramson2024alphafold3, esser2024sd3, li2024mar} in conditional generative models have demonstrated remarkable performance across language, vision, and graph generation tasks. 
Motivated by these developments, we propose a novel approach to circuit generation that generates preliminary circuit structures to guide DAS in generating refined circuits matching specified truth tables. 
Firstly, we introduce CircuitVQ, a tokenizer with a discrete codebook for circuit tokenization. 
Built upon our Circuit AutoEncoder framework~\cite{hou2022graphmae,li2023maskgae,wu2025mgvga}, CircuitVQ is trained through a circuit reconstruction task. 
Specifically, the CircuitVQ encoder encodes input circuits into discrete tokens using a learnable codebook, while the decoder reconstructs the circuit adjacency matrix based on these tokens.
Subsequently, the CircuitVQ encoder serves as a circuit tokenizer for CircuitAR pretraining, which employs a masked autoregressive modeling paradigm~\cite{chang2022maskgit, li2023mage}. 
In this process, the discrete codes function as supervision signals. 
After training, CircuitAR can generate discrete tokens progressively, which can be decoded into initial circuit structures by the decoder of the CircuitVQ. 
These prior insights can guide DAS in producing refined circuits that match the target truth tables precisely.

Our key contributions can be summarized as follows:
\begin{itemize}
\item We introduce CircuitVQ, a circuit tokenizer that facilitates graph autoregressive modeling for circuit generation, based on our Circuit AutoEncoder framework;
\item Develop CircuitAR, a model trained using masked autoregressive modeling, which generates initial circuit structures conditioned on given truth tables;
\item Propose a refinement framework that integrates differentiable architecture search to produce functionally equivalent circuits guided by target truth tables;
\item Comprehensive experiments demonstrating the scalability and capability emergence of our CircuitAR and the superior performance of the proposed circuit generation approach.
\end{itemize}

% Motivation
% (a) Diffusion (Vision, Graph), Autoregressive (Language, Vision)
% (b) Circuit Generation for Predefined Setting
% (c) Neural Architecture Search for Strict Logic Equivalence

% Contribution
% (a) Circuit Tokenizer (new transformer arch, training strategy)
% (b) CircuitAR (train and gen strategies, post-ar strategy)
% (c) Extensive Evaluation including BitD (Bit Distance) for Scalability

\putsec{back}{Background}

This section first provides the background on 3D graphics rendering. It then
describes the state-of-the-art radiance field rendering method: 3D Gaussian
splatting.

%%%%%%%%%%%%%%%%%%%%%%%%%%%%%%%%%%%%%%%%%%%%%%%%%%%%%%%%%%%%%%%%%%%%%%%%%%%%%%%%%%
\begin{figure}[t]
  \centering
  \includegraphics[width=0.95\columnwidth]{figures/opengl-pipeline.pdf}
  \caption{OpenGL rendering pipeline.}
  \vspace{-0.20in}
  \label{fig:opengl-pipeline}
\end{figure}

\putssec{gr}{Preliminaries on 3D Graphics Rendering}

\noindent \textbf{Graphics Pipeline.}
%
A 3D scene is rendered into a 2D image through a series of stages, which is
referred to as a \emph{graphics pipeline} or a \emph{rendering pipeline}.
%
To run the graphics pipeline, graphics software conventionally builds on
standard graphics APIs such as OpenGL~\cite{opengl}, Direct3D~\cite{d3d},
Vulkan~\cite{vulkan}, and Metal~\cite{metal}. Each API defines a set of
functions that process the operations in the rendering pipeline on graphics
hardware. 

\figref{opengl-pipeline} illustrates a high-level overview of the OpenGL
rendering pipeline. Other graphics APIs also employ a similar pipeline model.
%
The OpenGL pipeline can be largely divided into five stages: vertex shading,
vertex post-processing, rasterization, fragment shading, and per-fragment
processing. 
%
In hardware-based graphics rendering, each pipeline stage maps to either
programmable shader cores or fixed-function units in GPUs. 
%
Note that in software-based rendering, the operations in each stage are
executed entirely on the shader cores without using fixed-function hardware.

When a draw call is invoked with input vertices, the vertex shader\footnote{A
shader is a small program that runs on the shader cores.} transforms the
position of each vertex from 3D world space into clip space coordinates, which
will be further transformed into 2D screen positions and depth by
fixed-function hardware.
%
In the vertex post-processing stage, the vertices are then assembled into
primitives (e.g., triangles). In this stage, primitives outside the visible
space are excluded through a process known as \emph{view frustum culling},
and only the visible part of a primitive remains if part of the primitive is
outside the screen space.

The visible primitives are fed into a hardware rasterizer to identify the
pixels that overlap with them. The rasterizer produces \emph{fragments} for
each primitive; if a pixel is covered by multiple primitives, there will be
more than one fragment for the pixel.
%
Also, vertex attributes computed by the vertex shader are interpolated for each
fragment in this stage. 

Using the per-fragment data (e.g., pixel position, interpolated features) and
shared data (e.g., textures), the fragment shader computes and outputs a color
and an opacity (i.e., an RGBA value) for each fragment. 
%
In the final per-fragment processing stage, raster operations perform depth and
stencil tests. For the fragments that pass the tests, their RGBA colors are
blended or stored into the color buffer to generate the final pixel color.
%
It should be noted that the rendering pipeline can be implemented in
hardware with various optimizations, as long as the final pixel colors are
correctly produced.

%%%%%%%%%%%%%%%%%%%%%%%%%%%%%%%%%%%%%%%%%%%%%%%%%%%%%%%%%%%%%%%%%%%%%%%%%%%%%%%%%%
\begin{figure}[t]
  \centering
  \includegraphics[width=0.95\columnwidth]{figures/nvidia-gpu-arch.pdf}
  \caption{NVIDIA Ampere GPU architecture~\cite{ampere}.}
  \vspace{-0.20in}
  \label{fig:nvidia-gpu-arch}
\end{figure}

\myparagraph{Graphics-Specific Hardware in GPUs.}
%
As previously discussed, modern GPUs employ programmable shader cores that
execute different types of shader programs. Today, the shader cores are not
only accessible from graphics software but are also exposed to run
general-purpose programs through software frameworks such as CUDA and OpenCL.
%
Still, GPUs also feature graphics-specific hardware that facilitates the
execution of certain parts of the graphics pipeline, which is \emph{not}
accessible via general-purpose computing frameworks.

As shown in~\figref{nvidia-gpu-arch}, for example, an NVIDIA GPU includes
several special-purpose graphics units in addition to the programmable shaders
(i.e., Streaming Multiprocessor; SM).
%
Each Graphics Processing Cluster (GPC) includes a number of Texture Processing
Clusters (TPCs), each of which contains a PolyMorph Engine. The PolyMorph
Engine performs operations such as vertex fetching and viewport transformation,
and forwards the results to the Raster Engine~\cite{wit:kil11}.

The Raster Engine (rasterizer) sets up triangle edges using input vertex
positions and computes the pixel coverage of each triangle, a process called
rasterization. 
%
The fragments produced by the rasterizer are sent to the depth ($z$) test unit
(ZROP) if an early $z$-test is enabled.
%
This unit compares the depth of each fragment with the value in the $z$-buffer at
the same pixel position, discarding fragments that would ultimately fail the
late $z$-test conducted after fragment shading. By doing so, it prevents
unnecessary fragment shading computations.
%
After fragment shading in the shader cores (SM), the render output units
(ROPs), also known as raster operation units, perform blending or storing
operations while ensuring the proper ordering of fragments for the same pixel
location.

\myparagraph{Tile-Based Rendering.}
%
Most contemporary GPUs, including NVIDIA RTX, AMD Radeon, Intel Gen, and ARM
Mali, now use some variant of tile-based rendering (TBR).
%
When rendering an image using the hardware graphics pipeline, the screen space
is divided into a grid of screen tiles, each containing a block of pixels.
These tiles are assigned to the shader cores in the form of warps or thread
blocks.
%
For instance, NVIDIA GPUs split the screen space into a grid of
16$\times$16-pixel tiles, each of which is assigned to a specific GPC. 
%
This improves cache locality and reduces off-chip memory access during
rendering.
%
To achieve this, GPUs perform tile binning in hardware~\cite{gen11,lin:mor09}.
The fragments produced by the hardware rasterizer are grouped into bins based
on their tile IDs. These bins are then flushed to the shader cores when certain
conditions are met (e.g., a bin is full, a timeout occurs, or there is a lack
of available bins for new fragments with different tile IDs).
%
\secref{analysis} provides further analysis and discussion of fixed-function
units and tile-based rendering, based on microbenchmarking of modern GPUs.


%%%%%%%%%%%%%%%%%%%%%%%%%%%%%%%%%%%%%%%%%%%%%%%%%%%%%%%%%%%%%%%%%%%%%%%%%%%%%%%%%%
\putssec{3dgs}{Radiance Field Rendering with Gaussian Splatting}

3D Gaussian splatting~\cite{ker:kop23} introduces a novel method that achieves
state-of-the-art rendering performance and quality by \emph{explicitly}
representing a scene with a set of anisotropic 3D Gaussians.
%
Each Gaussian is characterized by geometric properties, such as a position
(mean) coordinate $\mu$ and a 3$\times$3 covariance matrix $\Sigma$, as well as
visual properties, such as opacity $o$ and spherical harmonic (SH) coefficients
$sh$, to represent the view-dependent color of the Gaussian.

For training, given a sparse set of 2D images, an initial set of 3D points is
generated using a Structure-from-Motion (SfM) technique. These points serve as
the centers for the initial isotropic Gaussians.
%
During the training phase, the features of the Gaussians are updated
continuously based on their computed gradients. To better represent the fine
geometric details of the scene, the number of Gaussians increases as they are
cloned and split into smaller ones.

While a 3D Gaussian is mathematically defined as a continuous function
over the entire 3D space, Gaussian splatting models each Gaussian as an
ellipsoid for practical purposes.
%
During rendering, these 3D Gaussians are projected onto the 2D image plane as
ellipses, referred to as \emph{2D splats}. 
%
The splats are sorted by depth, from nearest to farthest relative to the given
viewpoint. The final pixel color ($\mathbf{{C}}$) is then computed using
$\alpha$-blending (\eqnref{volumerender}), which combines the colors
($\mathrm{\mathbf{c}_i}$) of overlapping splats in front-to-back order:
%
\begin{equation}
\small
\begin{aligned}
  \mathbf{{C}} = \mathrm{\sum\limits_{i=1}^{N}} &\mathrm{\alpha_i\mathbf{c}_i} \mathrm{\prod_{j=1}^{i-1}} \mathrm{(1-\alpha_j)}, \\
  \textrm{with}~~\mathrm{\alpha_i} = o_\mathrm{i} \cdot \mathrm{exp}(-\frac{1}{2}&({p'}-\mu')^T\Sigma'^{-1}({p'}-\mu')),
  \label{eqn:volumerender} 
\end{aligned}
\end{equation}
%
where ${p'}$ denotes the pixel position, and $\mu'$ and $\Sigma'$ represent the
mean and the covariance matrix of the 2D splat, respectively.

\section{MLLM-based Detector}
\label{sec:mllm}
\begin{figure*}[!ht]
  \centering
    \includegraphics[width=1.0\linewidth]{Fig/MLLM-text.pdf}
    \caption{Illustrating of MLLM-based detection methodologies for AI-generated text}
    \label{fig:MLLM-text}
\end{figure*}
This paper primarily focuses on MLLM-based methods for detecting AI-generated media. Therefore, we first introduce relevant MLLM-based approaches. Before diving into these methods, it is worth noting that previous works~\cite{lin2024detecting, deng2024survey, yu2024fake} have reviewed some Non-MLLM-based methods.

As a product of advancements in Natural Language Processing (NLP) and Computer Vision (CV), MLLMs represent a significant milestone in AI. Compared to traditional Non-MLLM detection methods, MLLMs leverage their multimodal nature and reasoning abilities to offer several distinct advantages. First, their human-like cognitive abilities, enabled by Chain-of-Thought (CoT) and In-Context Learning (ICL), allow MLLMs not only to detect potential forgery traces in AI-generated media but also to reason about and explain their decision-making processes. Additionally, textual input and output empower MLLMs to support flexible query formats and provide human-interpretable contextual explanations. In terms of forgery analysis potential, MLLMs excel at identifying and describing visual forgery cues, conducting adaptive analyses driven by textual prompts, and validating authenticity through causal reasoning. These capabilities make MLLMs highly effective in supporting forgery detection in AI-generated media, particularly in identifying and describing forgery traces, performing flexible, text-driven analyses, and verifying authenticity through causal reasoning. In contrast, traditional Non-MLLM detection methods primarily focus on single-modal feature extraction and classification, often lacking interpretability and causal analysis capabilities. By addressing these limitations, MLLMs demonstrate their effectiveness in supporting AI-generated media detection. In the following sections, we will analyze the underlying technologies and methodologies in detail.
\subsection{Text}
 % 替换为你的图像文件
%\input{Table/attack_taxonomy}
\subsubsection{\textbf{Authenticity}}
MLLMs can be used in judgment of the authenticity of AI-generated text. The methods can be divided into five types: Statistical-based methods, Prompt-engineering, Self-consistency, Multi-Author, and Watermarking, all of which leverage the capability of MLLMs, as shown in Fig.~\ref{fig:MLLM-text} (a).
% \zekun{Statistical-based methods, prompt engineering, and self-consistency seem to be methods, whereas multi-author and watermarking appear to be issues. Placing all five at the same level might not be accurate.}
\begin{itemize}
\item \textbf{Statistical-based}
By examining statistical differences in language use, such as probability distributions or specific features, zero-shot methods can distinguish human writing from GPT-generated text, leveraging both shallow and deep characteristics. For shallow features, HowkGPT~\cite{vasilatos2023howkgpt} computes perplexity scores, establishing thresholds to distinguish their origins. 
DNA-GPT~\cite{yang2023dna} uses N-gram analysis or probability divergence. In the context of deep features, DetectLLM~\cite{su2023detectllm} introduces two methods DetectLLM-LRR and DetectLLM-NRR both leveraging log-rank information. DetectLLM-NRR focuses on accuracy with fewer perturbations, while DetectLLM-LRR emphasizes speed and efficiency. DetectGPT~\cite{mitchell2023detectgpt} leverages the negative curvature regions of the model's log probability function, without requiring additional training. Subsequently, Fast-DetectGPT~\cite{bao2023fast} introduces the concept of conditional probability curvature, which improves upon DetectGPT by replacing the computationally intensive perturbation step with a faster sampling step.

\item \textbf{Prompt Engineering}
Some researchers leverage MLLMs to detect In the LOKI study ~\cite{ye2024loki}, results show that MLLMs achieve only 61.5\% accuracy in judgment tasks asking, `Is the provided text generated by AI?'. However, accuracy increases to 89.2\% when the task is reformulated into a multiple-choice format, such as `Which of the following text is generated?'. The improvement stems from MLLMs' strength in contrastive analysis, as binary choice tasks allow direct comparison of subtle differences, unlike isolated judgment tasks relying solely on internal feature detection. Bhattacharjee et al.~\cite{bhattacharjee2024fighting} find that even though ChatGPT struggles to detect AI-generated text, it performs well in identifying human-written text. Zhang et al.~\cite{zhang2024detection} design various prompts, such as Base task-specific prompts, Style-specific prompts, and Evasion-optimized prompts to show the vulnerability of detectors.

\item \textbf{Self-consistency}
The self-consistency hypothesis suggests that, within a given input context, machine-generated text tends to make more predictable choices in words or tokens compared to humans. DetectGPT-SC~\cite{wang2023detectgpt} masks a portion of the input text and uses LLM to predict the masked words or tokens. It measures the consistency between the predictions and the original text to determine whether the text was generated by the LLMs. Additionally, numerous studies~\cite{nguyen2024simllm,zhu2023beat,mao2024raidar, hao2024learning} focus on utilizing LLMs to revise or rewrite sentences or phrases and then calculate the similarity between the original and the rewritten versions. SimLLM~\cite{nguyen2024simllm} uses candidate LLMs to proofread an input text, generating multiple versions and comparing their similarity to the original text to determine if the text was generated by an LLM. Zhu et al.~\cite{zhu2023beat} use ChatGPT to revise and analyze the similarity. Moreover, Raidar~\cite{mao2024raidar} prompts LLMs to rewrite the text, calculate the editing distance of the output, and exhibit high robustness in new content and multi-domain applications. Rewritelearning~\cite{hao2024learning} trains an LLM to rewrite input text, minimizing edits for AI-generated media while applying more edits to human-written text.


\item \textbf{Multi-Author}
Multi-Author core idea is to distinguish different authors (\textit{varying degrees of LLM intervention, e.g., partly written by AI, polished by AI}) rather than simply classify text as human-written or AI-generated. 
MIXSET~\cite{zhang2024llm} is the first dataset comprising human-written, machine-generated, and
human/LLM-refined machine-generated texts (MGTs) and focuses on multi-author binary classification. From then on, LLM-DetectAIve~\cite{abassy2024llm} provides a four-way classification task with the addition of three labels: ``human-written/machine-written", ``machine-written, then machine-humanized", ``human-written, then machine-polished". Beemo~\cite{artemova2024beemo} is a benchmark designed to evaluate AI-generated text detection in multi-author scenarios. LLMDetect~\cite{cheng2024beyond} introduces two tasks: LLM Role Recognition (LLM-RR) for multi-class classification and LLM Influence Measurement (LLM-IM) for quantifying LLM involvement, showing fine-tuned PLM-based models outperform advanced LLMs in detecting their outputs. 

\item \textbf{Watermarking}
To watermark LLMs, Kirchenbauer et al.~\cite{kirchenbauer2023watermark, kirchenbauerreliability} propose a method involving inserting signatures during the decoding stage. These methods categorize the vocabulary into ``red" and ``green" lists, restricting the LLM to decoding tokens from the green list. Subsequently, Christ et al.~\cite{christ2024undetectable} and Unigram-Watermark~\cite{zhaoprovable} suggest various algorithms for splitting the red and green lists or sampling tokens from the green list's probabilistic distribution to enhance the interpretability and robustness of watermarking mechanisms during the inference process. PersonaMark~\cite{zhang2024personamark} is a personalized text watermarking method that leverages sentence structure and user-specific hashing. By embedding unique watermarks, it guarantees copyright protection and user tracking of generated text while maintaining the text's naturalness and generation quality.
\end{itemize}
\begin{figure}[t!]
  \centering
  \includegraphics[width=\linewidth]{Figures/4_asr_eval.pdf}
  \caption{Performance of different attack methods. Surprisingly, simply intervening information from the template region (i.e., \textsc{TempPatch}) can significantly increase attack success rates.}
  \label{fig:asr_eval}
\end{figure}


\subsubsection{\textbf{Explainability}}
Traditionally, detecting LLM-generated text is often framed as a binary classification task. Methods are shown in Fig.~\ref{fig:MLLM-text} (b). However, there is also an ``undecided" category~\cite{ji2024detecting}, which is used to represent ambiguous texts that may originate from either humans or AI. This category is crucial for enhancing the explainability of detection results. By incorporating it, the system not only improves its reliability but also allows ordinary users to better understand the detection outcomes. Ji et al.~\cite{ji2024detecting} construct a dataset containing LLMs-generated text and human-generated text. Three human annotators are tasked with producing ternary labels along with explanation notes. They identify eight categories of explanations provided by human annotators, including spelling errors, grammatical errors, perplexity, logical errors, and unnecessary repetition.


\subsubsection{\textbf{Localization}}
Methods of localization are shown in Fig.~\ref{fig:MLLM-text} (a).
Gruda et al.~\cite{gruda2024three} have proposed three ways that ChatGPT can assist in academic writing. Similar to ``Multi-Author", LLMs play different roles based on varying user needs, from creating and drafting to polishing. The text totally written by AI is easier to detect than human-collaborated text. Some researchers quantify the involvement ratio of LLMs in content creation and localize which part of a phrase is written by AI.  LLMDetect~\cite{cheng2024beyond} offers an involvement ratio strategy. GigaCheck~\cite{tolstykh2024gigacheck} combines fine-tuned general-purpose LLMs to distinguish human-written texts from LLM-generated texts. Additionally, it employs a DETR-like model to localize AI-generated intervals in human-machine collaborative texts.

\begin{figure*}[!ht]
  \centering
    \includegraphics[width=1.0\linewidth]{Fig/MLLM-Image.pdf}
    \caption{Illustrating of MLLM-based detection methodologies for AI-generated images. ``Mask + Image → Text" approach is reproduced from~\cite{li2024forgerygpt}, ``Text + Image → Mask" approach is reproduced from~\cite{huang2024sida}, and Independent Mask Localization method is adapted from~\cite{lian2024large}}
    \label{fig:MLLM-image}
\end{figure*}

\subsection{Image}
\subsubsection{\textbf{Authenticity}}
For assessing image authenticity using MLLMs, we divide the approach into three categories: Prompt engineering, Fine-tuning, and Integration with external detectors, as shown in Fig.~\ref{fig:MLLM-image} (a).
\begin{itemize}
\item \textbf{Prompt-engineering}
Prompt engineering can be categorized into four types: Judgment prompts, Multiple-choice prompts, Score prompts, and In-context prompts. 
For \textbf{Judgment prompts}, the model is directly queried with questions (\textit{e.g., `Is the provided image generated by AI?'}~\cite{ye2024loki} \textit{, `Is this an example of a real image?'}~\cite{shi2024shield, huang2024visualcritic}). However, variations in phrasing, such as replacing ``real" with ``bonafide" or ``spoof"~\cite{shi2024shield}. LOKI~\cite{ye2024loki} shows that MLLMs may not be good at judging whether the input image is generated by AI. Mantis-8B shows the best performance only achieving 54.6\% accuracy, compared to 80.1\% for human evaluators. Nevertheless, Jia et al.~\cite{jia2024can} suggest that guiding MLLMs to focus on regions of an image likely to contain forgery clues (\textit{e.g., `Analyze the eye area'}) can enhance detection effectiveness. About \textbf{Multiple-choice prompts}, it gives MLLMs some choice (\textit{e.g., `Which of the following image is the generated image?'}~\cite{ye2024loki}). LOKI shows that MLLMs perform better in multiple-choice tasks compared to judgment tasks. GPT-4o achieves the best results, with an overall accuracy of 80.8\%, which is close to the human accuracy of 84.5\%. 
Also for \textbf{Score prompts}, MLLMs are tasked with providing a probability score for their judgments. Jia et al.~\cite{jia2024can} observe that such requests result in a 100\% rejection rate by GPT-4V.
In addition, \textbf{In-context prompts}, also referred to as one-shot questions, MLLMs are provided with examples to guide their detection (\textit{eg., The first set of images is of a real face, is the second set of images a real
face or a spoof face? Please answer `this image is a real face'})~\cite{shi2024shield}. It shows that MLLMs may give more accurate answers. Prompt engineering enhances the performance of MLLMs in detecting AI-generated images through flexible prompt design. However, it is highly sensitive to the specific design choices, with task formats and phrasing significantly impacting effectiveness. Additionally, its robustness may be limited in complex scenarios, particularly when faced with diverse or shifting data distributions.

\item \textbf{Fine-tuning}
To improve the MLLMs’ detection capabilities, fine-tuning involves adjusting model parameters using targeted datasets. $\textit{X}^2$-DFD~\cite{chen2024textit} comprises three modules: Model Feature Assessment (MFA), Strong Feature Strengthening (SFS) and Weak Feature Supplementing (WFS). MFA evaluates and ranks forgery-related features, while SFS leverages the top-ranked features to create an explainable training dataset. This dataset is used to fine-tune the MLLM, enhancing both detection accuracy and explainability. Similarly, Fakeshield~\cite{xu2024fakeshield} includes two key components. The Domain Tagging-Enhanced Forgery Detection Module generates domain-specific tags (\textit{e.g., Photoshop, DeepFake, AIGC}) and integrates image features with instruction-based textual inputs to produce tampering detection results and explanations. Lightweight LoRA fine-tuning techniques are employed to improve detection efficiency and maintain strong explainability.


\item \textbf{External detectors}
From the experiment results of ~\cite{ye2024loki}, we can find that MLLMs are not good at directly judging whether the image is generated by AI. Researchers have proposed integrating MLLMs with external detectors to enhance their feature discrimination capabilities. For instance, $\textit{X}^2$-DFD~\cite{chen2024textit} evaluates forgery-related features and ranks them based on detection performance, utilizing external detectors (\textit{e.g., blending-based detectors}~\cite{lin2025fake}) to strengthen the handling of weak feature areas. These external prediction scores are then incorporated into the MLLMs. Additionally, FFAA~\cite{huang2024ffaa} introduces a multi-answer intelligent decision system, which combines a cross-modal fusion module and a classification module to identify the best answer that aligns with an image's authenticity. This integration significantly enhances the accuracy and reliability of detection.

\end{itemize}

\subsubsection{\textbf{Explainability}}
The explainability of MLLMs is a remarkable feature, and recent studies have increasingly explored its potential. The methods are illustrated in Fig.~\ref{fig:MLLM-image} (b). Some works~\cite{jia2024can,shi2024shield,lian2024large,huang2024sida} directly query MLLMs with prompts such as `explain what the artifacts are'. However, prior investigations~\cite{jia2024can, shi2024shield} reveal that directly generating textual explanations often leads to hallucinations or overthinking, producing inaccurate outcomes or refusal to respond. Moreover, MLLMs often struggle to comprehensively perceive all relevant features, limiting their effectiveness in explainability. To address these limitations, researchers have employed approaches such as fine-tuning MLLMs~\cite{chen2024textit, huang2024ffaa, xu2024fakeshield} or integrating external modules~\cite{sun2024forgerysleuth}. These approaches aim to establish a comprehensive evaluation framework by categorizing features into three levels: low-level pixel features (\textit{e.g., noise, color, texture, sharpness, and AI-generated fingerprints}), middle-level visual features (\textit{e.g., traces of tampered regions or boundaries, lighting inconsistencies, perspective relationships, and physical constraints}), and high-level semantic anomalies (\textit{e.g., content that contradicts common sense, incites, or misleads}). This multi-level feature evaluation provides a holistic approach to enhancing the detection capabilities and explainability of MLLMs.


\subsubsection{\textbf{Localization}}
Binary classification tasks in forgery detection cannot inherently provide detailed insights into tampered regions. This limitation becomes more pronounced as modern generative models employ increasingly sophisticated forgery techniques, such as localized modifications (\textit{e.g., altering facial features like eyes or mouths}) or holistic image synthesis. To address this challenge, mask localization has emerged as a more flexible and effective approach, effectively capturing subtle forgeries and adapting to diverse scenarios. Existing methods can be categorized into two primary approaches: \textbf{Image-Text-Mask Alignment Localization} and \textbf{Independent Mask Localization}. The methods are illustrated in Fig.~\ref{fig:MLLM-image} (b).

\begin{itemize}
\item \textbf{Image-Text-Mask Alignment Localization}
In this approach, ``image" refers to the input image, ``text" represents the explainable textual output about forgery, and ``mask" indicates the localized forgery region. Further, methods in this category can be divided into two subcategories: ``Mask + Image → Text" and ``Text + Image → Mask". For \textbf{``Mask + Image → Text"}, Forgerygpt~\cite{li2024forgerygpt} employs a Mask Extraction Module to capture pixel-level features of tampered regions, using the FL-Expert to generate precise forgery masks and the Mask Encoder to transform mask features into tokens compatible with the MLLM. These mask, image, and text features are then fused and input into the MLLM, enabling accurate localization of tampered regions along with explainable outputs. 
About \textbf{``Text + Image → Mask"}, Fakeshield~\cite{xu2024fakeshield} introduces a tamper comprehension module to enhance the detection of forgery regions by aligning descriptive features of tampered areas with visual attributes. By integrating segmentation techniques based on the Segment Anything Model, it generates precise forgery masks. Similarly, SIDA~\cite{huang2024sida} extends MLLM with specialized tokens and leverages multi-head attention for the precise fusion of detection and segmentation features. Editscout~\cite{nguyen2024editscout} combines an MLLM-based reasoning query generation module and a segmentation model, where the [SEG] token bridges user prompts and images to produce binary masks for edited regions with minimal fine-tuning.

\item \textbf{Independent Mask Localization}
ForgeryTalker~\cite{lian2024large} proposes a method that employs an independent mask decoder to directly generate mask predictions, offering a more modular approach to forgery detection. This approach offers a modular method for forgery detection and sends tokens to LLMs to generate explainable text outputs.
\end{itemize}

\begin{figure*}[!ht]
  \centering
    \includegraphics[width=1.0\linewidth]{Fig/MLLM-Video_Audio.pdf}
    \caption{Illustrating of MLLM-based detection methodologies for AI-generated Video and Audio}
    \label{fig:MLLM-Video&Audio}
\end{figure*}

\subsection{Video}
MLLMs integrate linguistic and visual data to process videos by leveraging LLMs and connecting them with modality-specific encoders through interfaces like Q-former. Notable open-source Video-LLMs include: \textbf{VideoChat}~\cite{li2023videochat}: a chat-centric interactive system primarily designed for video content understanding and multimodal generation; \textbf{VideoChatGPT}~\cite{maaz2023video}: combines visual encoders with LLMs for video-based conversational analysis; ~\textbf{Video-LLaMA}~\cite{zhang2023video}: integrates audio and visual signals from videos using Q-former, enabling efficient handling of multimodal tasks; \textbf{LLaMA-VID}~\cite{li2025llama}: represents video frames as tokens containing contextual and content information, significantly improving video processing efficiency.

Currently, the primary focus of Video Anomaly Detection (VAD) tasks using MLLMs lies in identifying anomalies in real-world scenarios, such as criminal behavior and abnormal incidents. However, detecting AI-generated videos necessitates addressing specific artifacts, including violations of natural physics and frame flickering. The methods are illustrated in Fig.~\ref{fig:MLLM-Video&Audio} (a). Chang et al.~\cite{chang2024matters} provide a comprehensive summary of the common defects observed in generated videos, offering valuable insights into this emerging challenge.

    \subsubsection{\textbf{Authenticity}}
    The detector of AI-generated video can be divided into two categories: Frame-Level detector and Video-Level detector. Frame-Level detector primarily focuses on studying forgery traces at the image level, while Video-Level detector focuses on detecting forged videos, such as through temporal and frequency domain analysis. Existing methods that use MLLMs as detectors are mostly frame-level detection approaches combined with a consistency detector.
    
    \begin{itemize}
    \item \textbf{Frame-Level detector}
    LOKI~\cite{ye2024loki} also shows the video modality result of judgment and multiple-choice tasks of LLMs, both accuracy respectively 71.3\% and 77.3\% by GPT-4o. 
    MM-Det~\cite{song2024learning} leverages MLLMs for frame-level forgery detection and to generate explainable text. It also uses Vector Quantised-Variational AutoEncoder (VQ-VAE) to reconstruct video content, by comparing the residuals between the reconstructed video and the original video to amplify diffusion forgery features. Finally, it introduces an innovative attention mechanism in the Transformer network to balance the detection of intra-frame and inter-frame forgery traces, integrating global and local features. VANE-Bench~\cite{bharadwaj2024vane} is a benchmark that uses MLLMs to detect AI-generated anomalies, including sudden appearance and disappearant objects, violating natural physics. 

    \item \textbf{Watermarking}
    Li et al.~\cite{li2024video} propose a multi-modal video watermarking approach. They embed imperceptible watermarks into strategically selected keyframes using a flow-based mechanism, ensuring minimal visual disruption. Additionally, the approach uses multiple loss functions to balance watermark robustness and video content integrity, effectively preventing unauthorized access by video-based LLMs.
    
    \end{itemize}

    \subsubsection{\textbf{Explainability}}
    Despite the growing interest in utilizing MLLMs for AI-generated video detection, current research has yet to address the explainability of these methods. Future work could focus on developing frameworks that integrate MLLMs with interpretable visual analysis techniques to provide clear and actionable explanations.
    \subsubsection{\textbf{Localization}}
    Similarly, the localization of manipulated regions in AI-generated videos using MLLMs remains an unexplored area. Research in this direction could explore the potential of MLLMs to combine temporal and spatial features for precise localization, which is particularly challenging in dynamic video content.


    \subsection{Audio}
    Currently, both open-source and proprietary MLLMs offering audio input support remain limited. Moreover, most existing models primarily emphasize audio content comprehension, with relatively little focus on analyzing acoustic characteristics. The methods are illustrated in Fig.~\ref{fig:MLLM-Video&Audio} (b).
    \subsubsection{\textbf{Authenticity}}
    
    \begin{itemize}
    \item \textbf{Prompt-engineering}
    LOKI~\cite{ye2024loki} selects open-source models supporting audio input, such as Qwen-Audio~\cite{chu2023qwen}, SALMONN-7B~\cite{sun2024video} and GPT-4o. For judgment tasks, the accuracy of SALMONN-7B is only 51.2\%. Additionally, some models lack support for multiple-choice tasks. Among those that do, the highest accuracy is achieved by AnyGPT, reaching 50.3\%. Research on distinguishing real and fake audio using MLLMs and acoustic cues remains limited. However, datasets such as those introduced by LOKI~\cite{ye2024loki} and SONICS~\cite{rahman2024sonics} focus on detecting fake voices or music. The field of AI-generated audio detection with Multimodal foundational models is still in its early stages.
    \end{itemize}
    
    \subsubsection{\textbf{Explainability}}
    To date, no research has explored the explainability of audio MLLM-based methods. This represents a significant gap, as explainability is crucial for understanding the decision-making process of these models, particularly in identifying subtle acoustic forgeries. Future studies could focus on developing frameworks that incorporate interpretable audio analysis techniques, thereby improving the transparency and trustworthiness of MLLM-based methods.

    \subsubsection{\textbf{Localization}}
    Currently, there is no published research addressing localization capabilities in audio MLLM-based methods. Localization is critical for pinpointing specific manipulated segments within audio signals, especially in cases of partial or layered forgeries. Further research could investigate how multimodal alignment or segment-wise attention mechanisms might enhance localization accuracy in MLLM-based frameworks.

\subsection{Multimodal}
Having explored text-guided detection methods for individual modalities such as text, image, video, and audio, we now turn our focus to multimodal collaboration. These methods leverage language to guide MLLMs in understanding and processing features from other modalities, demonstrating strong cross-modal adaptability. By integrating features from image, video, and audio modalities, we aim to explore how the intrinsic connections among multimodal content can further enhance the accuracy and robustness of AI-generated media detection.
\subsubsection{\textbf{Authenticity}}
\begin{itemize}
    \item \textbf{Text-Image}
    A key focus in this domain is evaluating image-text consistency and providing explanations for MLLM judgments. Out-of-context (OOC) media misuse involves cases where individuals are required to assess the accuracy of the accompanying statement and evaluate whether the image and caption correspond to the same event. This form of misuse, in which authentic images are paired with false text, represents one of the simplest yet most effective ways to mislead audiences. SNIFFER~\cite{qi2024sniffer} is an MLLM specifically designed for detecting and interpreting OOC misinformation, combining image-text consistency analysis, external knowledge retrieval, and fine-grained instruction tuning. \cite{wu2023cheap} integrates GPT-3.5 to enhance the contextual understanding capabilities of the traditional COSMOS model, leveraging IoU, Sentence BERT, and Prompt Engineering to fuse multimodal information effectively. Fka-owl~\cite{liu2024fka} through knowledge-augmented Large Vision-Language Models(LVLMs) to detect fake news.
    For \textbf{watermarking tasks}, text-image integration necessitates incorporating metadata from the text component and the generation context. Liu et al.~\cite{liu2023t2iw} propose the T2IW framework, which seamlessly embeds a binary watermark into generated images using a joint generation process that combines text encoding and noise. VLPMarker~\cite{tang2023watermarking}, a watermarking method based on backdoor injection, utilizes orthogonal transformation techniques to protect CLIP model copyrights while maintaining model efficiency and accuracy.
    \item \textbf{Visual-Audio}
    %用LLM的方法:与end-to-end的方法不同,因为LLMs能够识别跨模态哪或跨模态中可能存在的空间和时空伪造痕迹不一样
    ~\cite{shahzad2024good} integrates visual frames, audio speech, and text prompts into ChatGPT to generate outputs encompassing audiovisual analysis, interpretation, and authenticity prediction. Their approach involves designing various prompts, including binary classification prompts, probability prediction prompts, and tasks to identify synthetic artifacts. Unlike end-to-end learning-based methods, ChatGPT can effectively detect spatial and spatiotemporal artifacts and inconsistencies within or across modalities. For \textbf{watermarking tasks}, V²A-Mark~\cite{zhang2024v2a} embeds localization and copyright watermarks into video frames and audio samples, which employs a temporal alignment and fusion module and a degradation prompt learning mechanism for visual data, along with a sample-level versatile watermark for the audio. 
    
\end{itemize}

\section{Non-LLM-based Detector}
\label{sec:non-mllm}

In addition to methods that use MLLMs, there are various traditional techniques to detect AI-generated media. These approaches employ specialized algorithms and can be categorized into modalities such as text, image, audio, and video, based on the type of data processed.
\begin{figure}[t]
  \centering
  \includegraphics[width=\linewidth]{Figures/7_Meta-Llama-3-8B-Instruct_temp-4_probe_w_resp.pdf}
  \caption{Harmful probes from middle layers (i.e., layer 14 in Llama-3-8B-Instruct) can be transferred to response generation while maintaining high accuracy.}
  \label{fig:probe_in_resp}
\end{figure}


\subsection{Text}
\subsubsection{\textbf{Authenticity}}
Text content detection methods primarily fall into three categories: stylistic-based, linguistics features-based methods, and watermarking. These approaches determine whether a text is AI-generated by analyzing stylistic features, linguistic structures, and watermarking respectively.
\begin{itemize}
    \item \textbf{Stylistic-based} Unlike traditional binary classification problems, stylistic-based methods focus on distinguishing the writing styles of different authors. Each AI model has its unique writing style, and identifying these distinct styles proves to be more effective than a simple binary classification task.
    DeTeCtive~\cite{guo2024detective} is a multi-task, multi-level contrastive learning framework that demonstrates superior performance in detecting AI-generated text across in-distribution and out-of-distribution scenarios. It also introduces a novel feature, Training-Free Incremental Adaptation, which enables adaptation to new data without retraining.
    Shah et al.~\cite{shah2023detecting} propose a novel approach combining features like vocabulary diversity, readability metrics, and semantic distribution with machine learning models for classification. Kumarage et al.~\cite{kumarage2023stylometric} leverage stylometric features with a PLM embedding to enhance the detection of AI-generated text.
    
    \item \textbf{Linguistics-based}
    Hamed et al.~\cite{hamed2023improving} employ an unsupervised approach using repetition patterns of higher-order n-grams as textual characteristics, achieving notable results. Gallé et al.~\cite{galle2021unsupervised} innovatively utilize bigram networks from authentic scientific articles as a benchmark for comparison with ChatGPT-generated content, attaining high accuracy. Both methods cleverly account for the relationships between words.
    
    \item \textbf{Watermarking}
    To watermark existing text, some researchers~\cite{yoo2023robust}~\cite{munyer2024deeptextmark}~\cite{yang2023watermarking} use synonym replacement or syntactic transformations while maintaining overall meaning. However, these methods often rely on specific rules that can lead to unnatural modifications, degrading text quality and making it easier for attackers to detect. To overcome these issues, AWT~\cite{abdelnabi2021adversarial} employs a transformer encoder to encode sentences and merge them with message embeddings, which are then processed by a transformer decoder to generate watermarked text. Detection involves analyzing the watermarked text via transformer encoder layers to extract hidden messages. Then, REMARK-LLM~\cite{zhang2024remark} utilizes a pretrained LLM for watermark insertion and includes a reparameterization step to create sparser token distributions, enabling it to embed twice as many signatures as AWT while still ensuring effective detection, thereby enhancing watermark payload capacity.
    
\end{itemize}

\subsubsection{\textbf{Explainability}}
%当前解释模块对普通用户(lay users)的可理解性仍然较差。现有系统往往难以用直观方式解释复杂的检测逻辑
GPTZero~\cite{gptzero} is an online closed-source detector, which relies on six features for explainability: readability, percent SAT,
simplicity, perplexity, burstiness, and average sentence length. However, it does not provide clarity
on how these features influence its final judgments. Mitrovic et al.~\cite{mitrovic2023chatgpt} use implemented Shapley Additive Explanations to reveal how features of ChatGPT-generated text (such as formality, politeness, and impersonality) influence the classification decisions of detection models.
Ji et al.~\cite{ji2024detecting} introduce a ternary classification framework consisting of human-writing text (HWT), MGT, and an ``undecided" category. Human annotators relabel the text with the newly added ``uncertain" category and provide explanations for their decisions. Current explanation modules still fail to provide intuitive understandability for non-expert users. Existing systems often struggle to intuitively explain the complex detection logic.

\subsubsection{\textbf{Localization}} Zhang et al.~\cite{zhang2024machine} leverage contextual information to analyze multiple sentences simultaneously, and divide the text into chunks and extracting features using fixed-parameter detection models, avoiding additional training. MFD~\cite{tao2024unveiling} framework identifies specific paragraphs or sentences generated by LLMs by combining low-level structural features, high-level semantic features, and deep linguistic features. It enhances robustness through contrastive learning.

\subsection{Image}
\subsubsection{\textbf{Authenticity}}

\begin{figure}[!ht]
  \centering
    \includegraphics[width=1.0\linewidth]{Fig/Non-MLLM-Image.pdf}
    \caption{Illustrating of Non-MLLM-based authenticity detection methodologies for AI-generated images. The methods are categorized into: (a)~\textit{Low-level} (b)~\textit{High-level} (c)~\textit{Reconstruction error} (d)~\textit{Watermarking}, (d) is reproduced from~\cite{luo2025digital}}
    \label{fig:Non-MLLM-Image}
\end{figure}

Image detection methods can be broadly categorized into four types: high-level, low-level approaches, reconstruction error-based methods, and watermarking methods. High-level methods analyze geometric information, such as abnormal lighting, shadows, and reflections. They also examine human anatomy, including pupil reflections and body abnormalities in images. In contrast, low-level~\cite{yang2021mtd} feature methods rely on spatial and frequency domain analysis, as well as identifying artificial fingerprints. Reconstruction error-based methods utilize the reconstruction capabilities of diffusion models, identifying anomalies by comparing differences between the original and reconstructed images.
Watermarking methods involve embedding watermarks either before or after image generation, enabling the detection of AI-generated images through dedicated watermark detectors. The methods are illustrated in Fig.~\ref{fig:Non-MLLM-Image}.
~\begin{itemize}
    \item ~\textbf{High-Level}
   High-level methods primarily analyze \textbf{geometric} information, such as abnormal lighting, shadows, and reflections, as well as \textbf{human anatomy}, including pupil shape reflection and abnormalities in the human body within images. FHAD~\cite{wang2024generated} detects fine-grained human body abnormalities and proposes solutions for missing or redundant body parts through reconstruction. Fraid~\cite{farid2022lighting,farid2022perspective} examines the geometric consistency of vanishing points, shadows, and reflections in generated images, as well as lighting consistency, using these inconsistencies for detection. Sarkar et al.~\cite{sarkar2024shadows} propose three classifiers based on object-shadow relationships, perspective fields, and line segment analysis, achieving good results. AIDE~\cite{yan2024sanity} employs a mixture of expert approach, combining low-level pixel statistics with high-level semantic features, effectively identifying various AI-generated images.
    
    \item ~\textbf{Low-Level}
    Low-level methods primarily focus on spatial and frequency domain information. In the \textbf{spatial} domain, PatchCraft~\cite{zhong2024patchcraft} enhances texture features through image scrambling and reconstruction, examining pixel correlations for detection with robustness to perturbations. LGrad~\cite{tan2023learning} utilizes CNNs to convert images into gradient representations, performing well in cross-model and cross-category tests.
    For \textbf{frequency} domain analysis, AUSOME~\cite{poredi2023ausome} employs discrete Fourier and cosine transforms to analyze diffusion model-generated images, identifying specific patterns in DALL-E 2 outputs. Wolter et al.~\cite{wolter2022wavelet} propose a wavelet packet-based multi-scale time-frequency analysis method, preserving spatial and frequency information. Synthbuster~\cite{bammey2023synthbuster} leverages frequency artifacts in diffusion model-generated images for detection. Frank et al.~\cite{frank2020leveraging} analyze artificial traces in GAN-generated images using discrete cosine transforms.
    Researchers have also examined \textbf{artificial fingerprints} in images. Corvi et al.~\cite{corvi2023intriguing} discover that various generators leave specific traces in images. SeDID~\cite{ma2023exposing} cleverly utilizes the deterministic reverse process of diffusion models, introducing the concept (\textit{e.g., time step, stride error}) to distinguish between real and synthetic images by analyzing error patterns at specific timesteps. Moreover, the E3~\cite{azizpour2024e3} framework uses transfer learning to create specialized expert embedders for different synthetic image generators, allowing accurate detection with minimal data. It combines embeddings from multiple experts through an Expert Knowledge Fusion Network to enhance detection performance, particularly for newly emerged generators. 
    
    \item ~\textbf{Reconstruction Error} 
    With the reconstruction capability of Diffusion models, researchers identify abnormal regions by comparing the differences between the original and reconstructed images. DIRE~\cite{wang2023dire} was the first detector proposed for diffusion-generated images. AEROBLADE~\cite{ricker2024aeroblade} utilizes autoencoder reconstruction errors from LDMs in a train-free method. FIRE~\cite{chu2024fire} detects diffusion-generated images by analyzing frequency-based reconstruction errors. DRCT~\cite{chendrct} builds on the aforementioned observation and employs contrastive learning to improve generalization by generating hard samples during the reconstruction process. In addition, SemGIR~\cite{yu2024semgir} utilizes an image-to-text approach followed by text-to-image regeneration, calculating the similarity between the original and re-generated images to distinguish AI-generated images.

    \item ~\textbf{Watermarking}
    EditGuard~\cite{zhang2024editguard} embeds dual invisible watermarks in images to achieve copyright protection and tamper localization. This method trains a unified Image-Bit Steganography Network (IBSN), which decouples the training process from specific tampering types, enhancing the model's generalizability and allowing it to operate effectively without labeled data for particular tampering scenarios.
    %watermarking for generative models
    Additionally, watermarks can be integrated into diffusion models. The watermarks embedded in generative models are static, meaning that they do not adjust based on changes in the generated content. 
    %during静态水印
    DiffusionShield~\cite{cui2023diffusionshield} generates watermarks in generative diffusion models (GDMs) using a blockwise strategy that segments the watermark into basic patches. Each user has a unique sequence of patches that encodes copyright information across their images. The method also utilizes joint optimization to improve efficiency and accuracy, allowing for the easy addition of new users without retraining.
    %latent diffusion models 水印
    Moreover, Latent Diffusion Models (LDMs) generate the image in the latent space of a pre-trained autoencoder. We argue that this latent space can be used to integrate watermarking into the generation process.
    ZoDiac~\cite{zhang2024robust} injects watermarks into the latent space of stable diffusion models during noise sampling, enhancing the invisibility and robustness of the watermarked images. LaWa~\cite{rezaei2024lawa} modifies latent features of pre-trained LDM to embed watermarks during image generation
    %插件动态水印
    However, some researchers have found ways to design watermarks that can be dynamically adjusted according to the context. WMAdapter~\cite{ci2024wmadapter} is a plugin that seamlessly integrates watermarking into the diffusion models in the diffusion process, enabling dynamic watermarking without the need for individual fine-tuning for each watermark. 

\end{itemize}

Moreover, a recent study~\cite{tan2024c2p} has found CLIP model does not truly understand the concepts of ``real" and ``forged". Instead, it detects deepfake content by identifying similar concepts or features. Therefore, C2P-CLIP~\cite{tan2024c2p} integrates category-related concepts (\textit{e.g., DeepFake, Camera}) into CLIP's image encoder through a text encoder, through the use of image-text contrastive learning techniques. Also, some researchers~\cite{kim2024correlation, song2024quality} have found that existing methods typically train detection models by mixing deepfake data with varying levels of forgery quality. These approaches may cause the model to overly rely on easily identifiable forgery traces in low-quality samples, which can negatively affect its generalization ability. To address this, FreDA~\cite{song2024quality} proposes improving the facial structure of low-quality samples by combining the low-frequency features of real images with the high-frequency features of forged images, thereby enhancing their realism.

\subsubsection{\textbf{Explainability}}
For Non-MLLM methods, explainability tends to focus more on interpretability, which involves explaining the internal decision-making mechanisms of the model, rather than producing human-understandable explanatory content.
Cifake~\cite{bird2024cifake} employs Gradient Class Activation Mapping (Grad-CAM) technology, revealing that the model primarily relies on subtle visual defects in the image background, rather than the features of the objects themselves, to differentiate between real and synthetic images. ASAP~\cite{huang2024asap} uses gradient-based methods to identify pixel groups that have the greatest impact on classification results, revealing key falsified patterns in AI-generated images.

\subsubsection{\textbf{Localization}}
The main methods for localizing AI-generated forgery regions extract diverse features and employ various feature fusion modules to improve detection accuracy. They also utilize different strategies to enhance tampered edge traces, enabling high-precision localization of forgery regions.
DA-HFNet~\cite{liu2024hfnet} extracts RGB features, noise fingerprint features, and frequency domain features. It employs a dual-attention fusion mechanism for multimodal features and a multi-scale feature interaction strategy, along with edge loss optimization, to accurately localize forged regions. DiffForensics~\cite{yu2024diffforensics} trains a module that can simultaneously extract both high-level and low-level features and proposes an Edge Cue Enhancement Module to strengthen the edge features of the tampered region. MoNFAP~\cite{miao2024mixture} framework integrates both detection and localization tasks while incorporating various noise features to enhance the clues for forgery detection.
Also, HiFi-Net++~\cite{guo2024language} categorizes forgery attributes into multiple levels, such as fully synthetic, diffusion models, conditional generation, etc. It employs multi-level classification learning to comprehensively represent forgery features. By capturing the contextual dependencies between forgery attributes through hierarchical relationships, the method outputs both forgery detection and localization results.
SAFIRE~\cite{kwon2024safire} addresses the image forgery localization problem from a more fundamental perspective. The approach divides an image into different source regions based on its origin. Each source region represents an independent part of the image, which may be captured, AI-generated, or tampered with through other means. SAFIRE uses a point-based hint mechanism, where a point in the image is utilized to segment the source region that contains it, thereby enabling the division of the image into distinct source regions.


\subsection{Video}
\subsubsection{\textbf{Authenticity}}
~\cite{chang2024matters} identifies three main issues in AI-generated videos: appearance, motion, and geometry. Appearance refers to the inconsistency in color and texture, often resulting in distortions, especially during transitions between video frames. Motion indicates that the motion trajectories of objects may not comply with physical laws. Geometry highlights that objects in generated videos frequently violate real-world geometric rules, such as spatial proportions, scale, and occlusion order. We observe that methods for detecting AI-generated videos can be categorized into two types: \textbf{Frame-level}, and \textbf{Video-level} approaches. Each of these methods is suited to different detection scenarios and requirements, enabling effective identification across various video authentication tasks.

\begin{itemize}
    \item \textbf{Frame-Level} 
    Similar to the classification approach used in MLLM detectors, frame-level detection primarily focuses on identifying forgery traces by extracting individual video frames. Bohacek~\cite{bohacek2024human} detects AI-generated human motion in videos by utilizing multi-modal embeddings, including CLIP-based models, to map the visual information of video frames to their corresponding textual descriptions within the same semantic space. Each frame is first classified as real or fake using an SVM. Then, the authenticity of the entire video is determined based on the majority of the frame predictions. AIGVDet~\cite{bai2024ai} extracts features and performs classification on the spatial and optical flow of each frame. The results from each frame are combined through a decision fusion module to determine whether the video is AI-generated.

    \item \textbf{Video-Level}
    In video-level analysis, the focus is on the unique characteristics of videos, such as temporal and spatial features. 
    For \textbf{temporal-based} methods, DIVID~\cite{liu2024turns} combines CNN and LSTM architectures to capture both spatial and temporal features by leveraging DIRE ~\cite{wang2023dire} values. This approach improves accuracy by incorporating explicit knowledge from reconstructed frames and temporal dependencies, thereby enhancing the detector's generalizability on OOD video datasets. In addition, He et al.~\cite{he2024exposing} find that temporal dependencies in real and generated videos differ significantly: Real videos are captured by camera devices, with very short time intervals between frames, resulting in high temporal redundancy. In contrast, AI video generation models generate videos by controlling the temporal continuity between frames in latent space. To address this, they leverage local motion information and global appearance variations through representation learning. The model combines these features using a channel attention mechanism for effective feature fusion.
    However, other approaches focus on the \textbf{spatial-temporal consistency}. 
    Yan et al.~\cite{yan2024generalizing} propose a Video-level Blending method to simulate inconsistencies in facial features across consecutive frames in deepfake videos. Additionally, they introduce a lightweight Spatio-temporal Adapter, a plugin that enhances CNN or ViT architectures to simultaneously capture both spatial and temporal features.
    DuB3D~\cite{ji2024distinguish} adopts a dual-branch architecture, with one branch processing the raw spatio-temporal data and the other handling optical flow data.
    Demamba~\cite{chen2024demamba} is a plug-and-play detector, which processes the spatial and temporal dimensions of features, modeling the spatio-temporal consistency between features through grouping and scanning. By aggregating global and local features, it utilizes an MLP to classify the video, outputting the probability of whether the video is real or fake.
    %video-fingerprint
    Moreover, generated videos leave distinct traces, similar to image \textbf{fingerprints}, which can be learned and detected after performing a Fourier transform. Vahdati et al.~\cite{vahdati2024beyond} find video generators leave different traces than image generators, combining frame and video-level analysis for classifier training.

    \item \textbf{Watermarking}
    Similar to image watermarking, video watermarking can be implemented frame by frame using image watermarking techniques. Additionally, it is crucial to consider temporal correlations and the robustness of the watermark in video watermarking. DVMark~\cite{luo2023dvmark} uses an end-to-end trainable multi-scale network for robust watermark embedding and extraction across various spatial-temporal clues. REVMark~\cite{zhang2023novel} focuses on improving the robustness against H.264/AVC compression via the temporal alignment module and DiffH264 distortion layer.
    \end{itemize}

\subsubsection{\textbf{Explainability}}
At present, there is no existing research that specifically explores the explainability of AI-generated video detection using a Non-MLLM detector, leaving this area open for future investigation.

\subsubsection{\textbf{Localization}}
Currently, no research paper specifically addresses the Localization of detecting AI-generated videos for Non-MLLM detectors.


\subsection{Audio}
\subsubsection{\textbf{Authenticity}}
\begin{itemize}
    \item \textbf{Fingerprint}
    Traditional audio detection methods often rely on handcrafted features that encompass both perceptual and physical attributes. Salvi et al.~\cite{salvi2024listening} suggest that each TTS model may have a unique ``fingerprint", which is derived from background noise and high-frequency components. 
    \item\textbf{Watermarking}
    %在生成音频的时候加入水印
    Deep-learning audio watermarking methods focus on multi-bit watermarking and follow a generator or detector framework.
    %Multi-Bit Watermarking:嵌入多个比特的信息来实现,可以传递更多的内容
    DeAR~\cite{liu2023dear} is designed to counter audio re-recording (AR) distortions by modeling these distortions through a pipeline of environmental reverberation, band-pass filtering, and Gaussian noise. The approach employs a differential time-frequency transform for optimal watermark embedding, allowing end-to-end training of the encoder and decoder without relying on predefined rules.
    AudioSeal~\cite{roman2024proactive} is a localized watermarking that jointly trains a generator and a detector to embed and robustly detect watermarks. The approach enhances detection accuracy by masking the watermark in random sections of the audio signal and extends to multi-bit watermarking, enabling the attribution of audio to specific models or versions without compromising the detection process.
    %Zero-Bit Watermarking:是一种不携带具体信息的水印方法,在音频信号中嵌入特定的模式或特征来表明某个音频片段是水印的
    Other researchers have explored zero-bit watermarking, which is better adapted for the detection of AI-generated media. 
    Wu et al.~\cite{wu2023adversarial} introduce small, imperceptible perturbations to the original audio, directing its deep features towards specific watermark characteristics. To ensure practical robustness, they utilize data augmentation and error-correcting coding techniques.
    \end{itemize}
    

\subsubsection{\textbf{Explainability}} About interpretability features, SLIM~\cite{zhu2024slim} addresses audio deepfake detection by exploiting the Style-Linguistics Mismatch between real and fake speech, where real speech exhibits a natural dependency between linguistic content and vocal style, while deepfakes break this dependency. It learns this dependency in two stages: first by contrasting the style and linguistic representations of real speech, and then by using these learned features to classify audio as real or fake.
SFAT-Net-3~\cite{cuccovillo2024audio} combines amplitude and phase encoding and introduces a more complex decoder to predict the F0, F1, and F2 phoneme trajectories.
Pascu et al.~\cite{pascu2024easy} use scalar features, such as Mean Unvoiced Segment Length, through the classifier to detect and offer interpretability in the process. 

\subsubsection{\textbf{Localization}}
%定位
For localization of AI-generated segments,
HarmoNet~\cite{liu2024harmonet} combines multi-scale harmonic F0 features with self-supervised learning representations and an attention mechanism and also introduces a new Partial Loss function to focus on the boundary between real and fake regions.
CFPRF~\cite{wu2024coarse} combines frame-level detection network and proposal refinement network with difference-aware feature learning and boundary-aware feature enhancement modules.

%绿色安全的检测
What's more, Green AI is important to protect users' rights.
Safeear~\cite{li2024safeear} develops a neural audio code that decouples semantic and acoustic information, providing a novel privacy-preserving approach for deepfake detection. 



\subsection{Multimodal}
\subsubsection{\textbf{Authenticity}}
\begin{itemize}
    \item \textbf{Text-visual}
    HAMMER~\cite{Shao2023CVPR}, based on hierarchical manipulation reasoning, integrates unimodal encoders, multimodal aggregators, and dedicated detection heads. It captures inter-modal interactions through manipulation-aware contrastive learning and modality-aware cross-attention for content detection. 
    
    \item \textbf{Audio-visual}
    AI-generated audio-visual detection often relies on content consistency detection methods~\cite{li2024zero}, while other researchers employ graph-based multimodal fusion strategies~\cite{yin2024fine} to enhance the detection process.
    Li et al.~\cite{li2024zero} propose a zero-shot detection method based on content consistency, which utilizes Automatic Speech Recognition and Visual Speech Recognition models to decode audio and video content, respectively, generating content sequences for both modalities. Then it calculates the edit distance between these two content sequences as a metric to measure the consistency between the audio and video modalities.
    Yin et al.~\cite{yin2024fine} constructs heterogeneous graphs using positional encoding, capturing intra- and inter-modal relationships through cross-modal graph interaction and dehomogenized graph pooling modules. 

    \item \textbf{Trimodal}
    For trimodal fusion detection methods, there is a notable fusion strategy that effectively integrates the three modalities.
    Yoon et al.~\cite{yoon2024triple} propose a trimodal deepfake detection method using zero-shot identity and one-shot deepfake baselines, implementing visual, auditory, and linguistic feature interaction through a two-stage approach, with residual connections and late fusion to prevent information loss.
    
\end{itemize}

\subsubsection{\textbf{Localization}}
There are only localization methods for visual-audio.
DiMoDif~\cite{koutlis2024dimodif} detects forged content by calculating the differences between audio and video signals and using these differences to identify forgeries. Additionally, it optimizes the localization accuracy of the forged regions by calculating the overlap between the predicted forged intervals and the ground truth annotations.
MMMS-BA~\cite{katamneni2024contextual} framework effectively captures the interaction between audio and video signals using a cross-modal attention mechanism across multiple modalities and sequences. Additionally, it performs deepfake detection and localization through classification and regression heads.
\section{Evaluation}


\begin{table}[t]
    \centering
    % \vspace{-0.1in}
    \scalebox{0.78}{
    % \begin{small}
        \begin{tabular}{lccc}
            \toprule
            \multirow{2}*{\textbf{MoE Models}} & \textbf{Parameters} & \textbf{Experts Per Layer} & \textbf{Num. of} \\
            & \textbf{(active / total)} & \textbf{(active / total)} & \textbf{Layers} \\
            \otoprule 
            \mixtral~\cite{jiang2024mixtral} & 12.9B / 46.7B & 2 / 8 & 32 \\
            % \hline
            \qwen~\cite{yang2024qwen2} & 2.7B / 14.3B & 4 / 60 & 24 \\
            \phimoe~\cite{abdin2024phi} & 6.6B / 42B & 2 / 16 & 32 \\
            \bottomrule 
        \end{tabular}
    % \end{small}
    }
    \caption{Characteristics of three \MoE models in evaluation.}
    \vspace{-0.2in}
    \label{table:eval-moe-models}
\end{table}








\subsection{Experimental Setup}
\label{subsec:eval-setup}


% \begin{figure*}[t]
%     \centering
%     \begin{subfigure}[t]{0.48\textwidth}
%         \centering
%         \includegraphics[width=.9\linewidth]{figs/eval-overall-lmsys.pdf}
%         \caption{Serving three \MoE models with LMSYS-Chat-1M dataset.}
%     \end{subfigure}
%     \begin{subfigure}[t]{0.48\textwidth}
%         \centering
%         \includegraphics[width=.9\linewidth]{figs/eval-overall-sharegpt.pdf}
%         \caption{Serving three \MoE models with ShareGPT dataset.}
%     \end{subfigure}
%     \caption{Overall performance of prefill and decode stages for \sys and other four baselines.}
%     \label{fig:eval-overall.pdf}
% \end{figure*}


\noindent \textbf{Testbed.}
We conduct all experiments on a six-GPU testbed, where each GPU is an NVIDIA GeForce RTX 3090 with 24 GB GPU memory. 
%
All GPUs are inter-connected using pairwise NVLinks and connected to the CPU memory using PCIe 4.0 with 32GB/s bandwidth. 
%
Additionally, the testbed has a total of 32 AMD Ryzen Threadripper PRO 3955WX CPU cores and 480 GB CPU memory.


\noindent \textbf{Models.}
We employ three popular \MoE-based \LLMs in our evaluation: \mixtral~\cite{jiang2024mixtral}, \qwen~\cite{yang2024qwen2}, and \phimoe~\cite{abdin2024phi}.
Table~\ref{table:eval-moe-models} describes the parameters, number of \MoE layers, and number of experts per layer for the three models.
Following the evaluation of existing works~\cite{song2024promoe}, we profile the models to set the optimal prefetch distance $d$ to three before evaluation.
% We set $d$ of \mixtral, \qwen, and \phimoe to \todo{$xxx$}, \todo{$xxx$}, and \todo{$xxx$}, respectively.


\noindent \textbf{Datasets and traces.}
We employ two real-world prompt datasets commonly used for \LLM evaluation: LMSYS-Chat-1M~\cite{zheng2023lmsys} and ShareGPT~\cite{sharegpt}.
%
For most experiments, we split the sampled datasets in a standard 7:3 ratio, where 70\% of the prompts' context data (\ie, semantic embeddings and expert maps) are stored in \sys's Expert Map Store, and 30\% of the prompts are used for testing. 
%
For online serving experiments, we empty the Expert Map Store and use real-world \LLM inference traces~\cite{patel2024splitwise,stojkovic2025dynamollm} released by Microsoft Azure to set input and generation lengths and drive invocations.

\noindent \textbf{Baselines.}
We compare \sys against four \SOTA \MoE serving baselines:
1) \textbf{MoE-Infinity}~\cite{xue2024moe} uses coarse-grained request-level expert activation patterns and synchronous expert prediction and prefetching for \MoE serving. 
We prepare the expert activation matrix collection for MoE-Infinity before evaluation for a fair comparison.
%
% However, the open-sourced MoE-Infinity codebase~\cite{moe-infinity-code} lacks some features described in its original paper, we had to modify
%y 
2) \textbf{ProMoE}~\cite{song2024promoe} employs a stride-based speculative expert prefetching approach for \MoE serving. Since the codebase of ProMoE is not open-sourced and requires training predictors for each \MoE model, we reproduced a prototype of ProMoE on top of MoE-Infinity in our best effort.
%
3) \textbf{Mixtral-Offloading}~\cite{eliseev2023fast} combines a layer-wise speculative expert prefetching and a \LRU-based expert cache. 
%
4) \textbf{DeepSpeend-Inference} employs an expert-agnostic layer-wise parameter offloading approach, which uses pure on-demand loading and does not support prefetching. 
%
We implement the offloading logic of DeepSpeed-Inference in the MoE-Infinity codebase and add an expert cache for a fair comparison.
We enable all baselines to serve \MoE models from HuggingFace Transformer~\cite{wolf2020huggingface}. 


\noindent \textbf{Metrics.}
Following the standard evaluation methodology of existing works~\cite{song2024promoe,xue2024moe,zhong2024distserve,agrawal2024taming} on \LLM serving, we report the performance of the prefill and decode stages separately. 
We measure Time-to-First-Token (TTFT) for the prefill stage and Time-Per-Output-Token (TPOT) for the decode stage.
Additionally, we also report other system metrics, such as expert hit rate and overheads, for detailed evaluation.


% \noindent \textbf{\sys's setting.}
% The hyperparameters of \sys containing the prefetch distance $d$ for each \MoE model, Expert Map Store capacity $C$, and Expert Cache memory limit $M$.
% For most experiments, we profile the \MoE models and set the prefetch distance $d$ to their optimal values. The Expert Map Store capacity $C$ is set to \todo{$xxx$} expert maps. We configure the Expert Cache memory limit to \todo{$xxx$} GB.
% The hyperparameter sensitivity is analyzed in \S\ref{subsec:eval-sensitivity}.


\begin{figure}[t]
  \centering
  \includegraphics[width=.95\linewidth]{figs/eval-overall-arxiv.pdf}
  \vspace{-0.15in}
  \caption{Overall performance of prefill and decode stages for \sys and other four baselines.}
  \vspace{-0.2in}
  \label{fig:eval-overall}
\end{figure}


\subsection{Overall Performance}
\label{subsec:eval-overall}



We first evaluate the performance of prefill and decode stages when running \sys and other baselines with the three \MoE models, where we measure Time-To-First-Token (TTFT) and Time-Per-Output-Token (TPOT) for each stage.
Note that the inference latency with expert offloading tends to be higher than no offloading due to two reasons: 
1) During inference, an excessive amount of parameters in \MoE models are loaded and offloaded, which prolongs the inference latency.
2) All baselines and \sys are implemented on top of the MoE-Infinity codebase~\cite{moe-infinity-code}, whose inference latency is inherently impacted by MoE-Infinity's implementation.
Nevertheless, comparing \sys and baselines is fair with the same experimental setup.

Figure~\ref{fig:eval-overall} shows the \TTFT, \TPOT, and expert hit rate of \sys and other four baselines when serving three \MoE models with LMSYS-Chat-1M and ShareGPT datasets, respectively.
DeepSpeed has both the worst \TTFT and \TPOT due to expert-agnostic offloading and lacking expert prefetching.
While Mixtral-Offloading, ProMoE, and MoE-Infinity perform better than DeepSpeed-Inference, they are underperformed by \sys because of coarse-grained offloading designs.
Compared to DeepSpeed-Inference, Mixtral-Offloading, ProMoE, and MoE-Infinity, our \sys reduces the average \TTFT by 44\%, 35\%, 33\%, 30\%, and reduces the average \TPOT by 70\%, 61\%, 55\%, 48\%, across three \MoE models.
%
% Figure~\ref{fig:eval-overall} also reports the expert hit rate of \sys and each baseline. 
For expert hit rate, Mixtral-Offloading achieves a higher hit rate than the other three baselines because of its synchronous speculative prefetching with a prefetch distance of 1. However, due to synchronous prefetching, its \TTFT and \TPOT are worse than others except DeepSpeed-Inference.
\sys improves the average expert hit rate by 147\%, 11\%, 34\%, and 63\% over DeepSpeed-Inference, Mixtral-Offloading, ProMoE, and MoE-Infinity, respectively.

% \begin{figure}[t]
%   \centering
%   \includegraphics[width=.9\linewidth]{figs/eval-overall-sharegpt.pdf}
%   % \vspace{-0.15in}
%   \caption{}
%   % \vspace{-0.25in}
%   \label{fig:eval-overall-sharegpt.pdf}
% \end{figure}




\subsection{Online Serving Performance}
\label{subsec:eval-online}


Except for the offline evaluation (\ie, Expert Map Store in full capacity before serving), we also evaluate \sys against other baselines in online serving settings.
We empty the Expert Map Store of \sys and the expert activation matrix collection of MoE-Infinity for the online serving experiment.
%
The request traces are derived from Azure \LLM inference traces~\cite{patel2024splitwise,stojkovic2025dynamollm}, with 64 requests randomly sampled to drive LMSYS-Chat-1M prompts for each \MoE model serving. 
To ensure consistency, \sys and all baselines input and generate the exact number of tokens specified in the traces.
%
Figure~\ref{fig:eval-online-serve} illustrates the CDF of end-to-end request latency across three \MoE models. The results demonstrate that \sys significantly reduces overall request latency compared to other baselines in online serving scenarios.


\begin{figure}[t]
  \centering
  \includegraphics[width=.95\linewidth]{figs/eval-online-serve-arxiv.pdf}
  \vspace{-0.15in}
  \caption{CDF of request latency for \MoE online serving.}
  \vspace{-0.2in}
  \label{fig:eval-online-serve}
\end{figure}



\subsection{Impact of Expert Cache Limits}



We measure the \TPOT of \sys and other baselines by limiting the expert cache memory budget to investigate their performance in the latency-memory trade-off (\S\ref{subsec:bg-latency-memory-tradeoff}).
We mainly focus on \TPOT to show the end-to-end performance impacted by varying cache limits.
Figure~\ref{fig:eval-cache-limit.pdf} shows the \TPOT of \sys and other four baselines when serving three \MoE models under different expert cache limits.
We gradually increase the GPU memory allocated for caching experts from 6 GB to 96 GB while employing the same experimental setting in \S\ref{subsec:eval-overall}.
Similarly, DeepSpeed-Inference has the worst \TPOT due to being expert-agnostic.
\sys consistently outperforms Mixtral-Offloading, ProMoE, and MoE-Infinity under varying expert cache limits.
Especially for limited GPU memory sizes (\eg, 6GB), \sys reduces the \TPOT by 32\%, 24\%, 18\%, and 18\%, compared to DeepSpeed-Inference, Mixtral-Offloading, ProMoE, and MoE-Infinity, across three \MoE models, respectively.
With fine-grained expert offloading, \sys significantly reduces the expert on-demand loading latency while maintaining a lower GPU memory footprint, therefore achieving a better spot in the latency-memory trade-off of \MoE serving.

% \subsection{Impact of Inference Batch Size}

\subsection{Ablation Study}
\label{subsec:eval-ablation}


% \begin{figure}[t]
%   \centering
%   \includegraphics[width=.95\linewidth]{figs/eval-expert-tracking.pdf}
%   % \vspace{-0.15in}
%   \caption{Expert hit rate of different expert pattern tracking approaches.}
%   % \vspace{-0.25in}
%   \label{fig:eval-expert-tracking}
% \end{figure}



We present the ablation study of \sys's design.


\textbf{Effectiveness of expert map search.}
One of \sys's key designs is the expert map, which tracks expert selection preferences in fine granularity.
We evaluate the effectiveness of the expert map against five expert pattern-tracking approaches as follows.
%
1) \textbf{Speculate}: speculative prediction used by Mixtral-Offloading~\cite{eliseev2023fast} and ProMoE~\cite{song2024promoe}, 
%
2) \textbf{Hit count}: request-level expert hit count used by MoE-Infinity~\cite{xue2024moe}, 
%
3) \textbf{Map (T)}: expert map with only trajectory similarity search,
4) \textbf{Map (T+S)}: expert map with both trajectory and semantic similarity search,
%
and
5) \textbf{Map (T+S+$\delta$)}: expert map with full features enabled, including trajectory and semantic similarity search (\S\ref{subsec:design-similarity-match}) and dynamic expert selection (\S\ref{subsec:design-expert-prefetch}).
%
We implement the above methods in \sys's Expert Map Matcher for a fair comparison.
Figure~\ref{fig:eval-expert-tracking} shows the expert hit rate of the above expert pattern tracking methods.
%
Speculative prediction is effective due to the widespread presence of residual connections in Transformer blocks. However, its effectiveness decreases drastically as prefetch distance increases~\cite{song2024promoe}.
%
The request-level expert activation count has the worst performance due to coarse granularity.
%
As features are incrementally restored to \sys's expert map, the expert hit rate gradually increases, demonstrating its effectiveness.

% \textbf{Effectiveness of asynchronous map matching.}




\begin{figure}[t]
  \centering
  \includegraphics[width=.9\linewidth]{figs/eval-cache-limit-arxiv.pdf}
  \vspace{-0.15in}
  \caption{Performance of \sys and other four baselines under varying expert cache limits.}
  \vspace{-0.1in}
  \label{fig:eval-cache-limit.pdf}
\end{figure}

\begin{figure}[!t]
    \centering
    \begin{subfigure}[t]{0.585\linewidth}
        \centering
        \includegraphics[width=\linewidth]{figs/eval-expert-tracking.pdf}
        \caption{Expert pattern tracking approaches.}
        \label{fig:eval-expert-tracking}
    \end{subfigure}
    % \hspace{0.02in}
    \begin{subfigure}[t]{0.385\linewidth}
        \centering
        \includegraphics[width=\linewidth]{figs/eval-prefetch-and-cache-arxiv.pdf}
        \caption{Prefetch and caching.}
        \label{fig:eval-prefetch-and-cache}
    \end{subfigure}
    \vspace{-0.1in}
    \caption{Ablation study of \sys.}
    \label{fig:eval-ablation}
    \vspace{-0.2in}
\end{figure}

\textbf{Effectiveness of expert prefetching and caching.}
We evaluate \sys's expert prefetching and caching against two caching algorithms:
1) \textbf{\LRU} used by Mixtral-Offloading~\cite{eliseev2023fast}
and 
2) \textbf{\LFU} used by MoE-Infinity~\cite{xue2024moe}.
%
Figure~\ref{fig:eval-prefetch-and-cache} depicts the expert hit rate of \sys and two baselines.
The results show that \LRU performs poorly in expert offloading scenarios. Though \LFU achieves a higher hit rate than \LRU, \sys surpasses both, achieving the highest expert hit rate.

\subsection{Sensitivity Analysis}
\label{subsec:eval-sensitivity}


\begin{figure}[t]
  \centering
  \includegraphics[width=.9\linewidth]{figs/eval-prefetch-distance.pdf}
  \vspace{-0.15in}
  \caption{Performance of \sys serving \MoE models with different prefetch distances.}
  \vspace{-0.1in}
  \label{fig:eval-prefetch-distance}
\end{figure}

% \begin{figure}[t]
%   \centering
%   \includegraphics[width=.9\linewidth]{figs/eval-store-capacity.pdf}
%   % \vspace{-0.15in}
%   \caption{Semantic and trajectory similarity lower bounds in \sys's serving with different Expert Map Store capacity.}
%   % \vspace{-0.25in}
%   \label{fig:eval-store-capacity}
% \end{figure}

\begin{figure}[t]
    \centering
    \begin{subfigure}[t]{0.55\linewidth}
        \centering
        \includegraphics[width=\linewidth]{figs/eval-store-capacity.pdf}
        \caption{Expert Map Store capacity.}
        \label{fig:eval-store-capacity}
    \end{subfigure}
    % \hspace{0.02in}
    \begin{subfigure}[t]{0.435\linewidth}
        \centering
        \includegraphics[width=\linewidth]{figs/eval-batch-size-arxiv.pdf}
        \caption{Inference batch size.}
        \label{fig:eval-batch-size}
    \end{subfigure}
    \vspace{-0.1in}
    \caption{Sensitivity analysis of \sys.}
    \vspace{-0.2in}
    \label{fig:eval-sensitivity}
\end{figure}


We analyze the sensitivity of three hyperparameters: prefetch distance of \MoE models, the capacity of Expert Map Store, and inference batch size.


\textbf{Prefetch distance of \MoE models.}
Figure~\ref{fig:eval-prefetch-distance} shows the \TTFT and \TPOT of \sys when serving three \MoE models with different prefetch distances.
%
We have demonstrated that the expert hit rate decreases when gradually increasing the prefetch distance (Figure~\ref{fig:bg-hit-distance}).
%
When the prefetch distance is small ($<3$), \sys cannot perfectly hide its system delay from the inference process, such as the map matching and expert prefetching, leading to the increase of inference latency.
%
With larger prefetch distances ($>3$), \sys has worse expert hit rates that also degrade the performance. 
Therefore, we set the prefetch distance $d$ to 3 for evaluating \sys.


\textbf{Capacity of Expert Map Store.}
We measure the mean semantic and trajectory similarity scores searched in \sys's expert map matching for \MoE model serving.
%
Figure~\ref{fig:eval-store-capacity} presents the mean semantic and trajectory similarity scores of \sys with different Expert Map Store capacity sizes.
%
Both semantic and trajectory similarity scores improve as the store capacity increases.
%
While the similarity scores exhibit a significant increase with capacities below 1K, further capacity expansion yields diminishing similarity gains. 
To minimize \sys's memory overhead, we set \sys's Expert Map Store capacity to 1K in evaluation.


\textbf{Inference batch size.}
We investigate the impact of inference batch size on \sys and three baselines using \mixtral with LMSYS-Chat-1M.
%
Figure~\ref{fig:eval-batch-size} presents the performance of \sys, Mixtral-Offloading, ProMoE, and MoE-Infinity as the batch size increases from one to four. \sys achieves the lowest \TTFT and \TPOT in most cases.


% \textbf{Inference batch size.}


% \subsection{Scalability}
% \label{subsec:eval-scalability}
% From one to six GPUs


\begin{figure}[t]
  \centering
  \includegraphics[width=.92\linewidth]{figs/eval-overhead-latency.pdf}
  \vspace{-0.15in}
  \caption{Latency breakdown of \sys's one inference iteration with three \MoE models.}
  \vspace{-0.1in}
  \label{fig:eval-overhead-latency.pdf}
\end{figure}





\subsection{System Overheads}
\label{subsec:eval-overhead}


\noindent \textbf{Latency overheads of \sys's operations.}
Figure~\ref{fig:eval-overhead-latency.pdf} shows the latency breakdown of one inference iteration in \sys when serving the three \MoE models.
We report any operations of \sys in \S\ref{subsec:eval-overall} that may incur a significant latency delay, including context collection, map matching, expert on-demand loading, expert prefetching, and map update after the iteration completes.
\qwen has lower end-to-end iteration latency than \mixtral and \phimoe because of significantly fewer parameters.
Note that expert prefetching, map matching, and map update tasks are executed asynchronously, aside from the inference process. Hence, they do not contribute to the end-to-end iteration latency.
Excluding three asynchronous tasks, the total delay incurred by other operations is consistently less than 30ms (5\% of the iteration) across three \MoE models, which is negligible compared to the inference latency.


\noindent \textbf{Memory overheads of \sys's Expert Map Store.}
Figure~\ref{fig:eval-overhead-memory.pdf} shows the CPU memory footprint of \sys's Expert Map Store when varying the store capacity from 1K to 32K maps.
The memory needed to store expert maps for \qwen is more than \mixtral and \phimoe because it has more experts per layer over the other two models, which increases the map shape.
Even for the largest capacity (32K), the Expert Map Store requires less than 200MB of memory to store the maps, which is trivial since modern GPU servers usually have abundant CPU memory (\eg, p4d.24xlarge on AWS EC2~\cite{aws-ec2} has over 1100 GB of CPU memory).
In the evaluation, \sys's map store capacity with 1K maps is sufficient for maintaining performance (\S\ref{subsec:eval-sensitivity}), resulting in minimal memory overhead.



\begin{figure}[t]
  \centering
  \includegraphics[width=.85\linewidth]{figs/eval-overhead-memory.pdf}
  % \vspace{-0.1in}
  \caption{CPU memory footprint of \sys's Expert Map Store with different capacity.}
  \vspace{-0.1in}
  \label{fig:eval-overhead-memory.pdf}
\end{figure}

\section{Regulation}
\label{sec:reg}

\begin{table*}[!ht]
    \centering
        \renewcommand{\arraystretch}{1.4}
        \caption{Comparison of AI Governance Approaches in the \textbf{EU}, \textbf{USA}, and \textbf{China} across four dimensions: Risk Management Frameworks, Transparency Requirements, Technical Neutrality, and Industry Participation. This table highlights the unique priorities and methodologies each region adopts in addressing AI-generated content detection and governance.}
        \label{tab:ai_governance}

        \resizebox{\linewidth}{!}{
        \begin{tabular}
        %{m{4cm}|m{5cm}|m{5cm}|m{5cm}}\hline 
        {p{4cm}<{\centering} | p{5cm}<{\centering} | p{5cm}<{\centering} | p{5cm}<{\centering}}\hline
        \rowcolor{lightgrey} 
\textbf{Aspect} & 
\textbf{EU} &
\textbf{USA} &
\textbf{China} \\ \hline



\textbf{Risk Management Framework} 
& Four risk levels (minimal risk, limited risk, high risk, and unacceptable risk)
& Non-binding guidance
& A classification and grading approach is adopted, emphasizing inclusive and prudent regulation. \\ \hline

\textbf{Transparency Requirements} 
& \begin{itemize}
    \item AI-generated content must be clearly labeled.
    \item Record model training data sources and decision processes for external audits.
    \item Mandate explainability modules to help users understand AI decision logic.
\end{itemize}
& \begin{itemize}
    \item Encourage companies to voluntarily use watermarks or labels in generated content.
    \item Promote the development of transparency standards, such as industry collaboration on transparency APIs.
\end{itemize}
& \begin{itemize}
    \item Establish legal obligations for identifying generative AI content.
    \item Require generative AI platforms to regularly disclose algorithm models, training data, and technical documentation.
\end{itemize} \\ \hline

\textbf{Technology Neutrality Principle} 
& Less emphasis on technological neutrality, favoring a risk-oriented approach
& Emphasizes technological neutrality to safeguard innovation freedom.
& Combines technological neutrality with a risk-oriented approach. \\ \hline

\textbf{Degree of Industry Participation} 
& \begin{itemize}
    \item Prefers mandatory legal regulations to ensure industry participation.
    \item Establishes a unified regulatory framework to ensure compliance by both multinational corporations and SMEs.
\end{itemize}
& \begin{itemize}
    \item Encourages industry-led initiatives with voluntary participation in regulation.
\end{itemize}
& \begin{itemize}
    \item Industry participation is guided primarily by policy, with the government fostering collaboration across the industrial chain.
    \item Require generative AI platforms to regularly disclose algorithm models, training data, and technical documentation.
\end{itemize} \\ \hline
        

%_____________________________________
\end{tabular}
        }
    \label{regulation}
\end{table*}

In recent years, the rapid development of GenAI technologies has not only driven technological innovation and industrial advancement but also raised societal concerns, including the spread of misinformation, data privacy breaches, and ethical controversies. The rapid dissemination and difficult-to-monitor nature of AI-generated media have prompted governments and research institutions worldwide to focus on effectively regulating the applications and potential impacts of generative AI. Against this backdrop, we examine AI-generated media detection policies from four perspectives~\cite{shi2024large}: risk management frameworks, transparency requirements, technical neutrality, and industry participation. Risk management frameworks~\cite{novelli2024taking, zeng2024ai} evaluate how different countries identify, classify, and mitigate the potential risks of AI systems through policy and technical measures. Transparency requirements examine the implementation of policies on data source disclosure, algorithm transparency, and external audits.
The technical neutrality perspective explores whether AI regulations are enforced in a technology-neutral manner to avoid stifling innovation and industrial growth.
Industry participation analyzes the depth and breadth of collaboration between governments and enterprises in AI-generated media detection, including the interplay of legal mandates and voluntary contributions.
Analyzing these dimensions reveals differences in governance priorities across nations while providing valuable insights for researchers and policymakers to foster global collaboration and advancement in AI-generated media detection.

In 2024, the European Union (EU) passed the world’s first comprehensive artificial intelligence regulation, the Artificial Intelligence Act (AIA)~\cite{ArtificialIntelligenceAct}. It adopts a risk-based tiered regulatory approach, categorizing AI systems into four levels: minimal risk, limited risk, high risk, and unacceptable risk. Generative AI systems are generally classified as limited risk, requiring basic transparency obligations. The United States (US) emphasizes technical neutrality and industry self-regulation. The National Institute of Standards and Technology (NIST) introduced the AI Risk Management Framework (AI RMF) to guide developers in identifying and mitigating risks. Meanwhile, several legislative initiatives, such as the No AI Fraud Act and the COPIED Act, aim to protect intellectual property and combat deepfakes. China~\cite{ChinaAIGovernance2023} focuses on safety controls and ethical use within its governance framework. Policies like the Generative AI Service Management Provisions adopt an inclusive, risk-sensitive classification and grading approach, encouraging AI integration into national governance. A detailed comparison is presented in Table~\ref{tab:ai_governance}.

Looking ahead, global AI governance must balance innovation with regulation. Combining the EU’s tiered framework, the US’s technical neutrality and self-regulation model, and China’s classification-based oversight can promote multilateral collaboration and standardization. Policies should strengthen the integration of technology and ethics, enhancing governance flexibility and responsiveness. Industry stakeholders should actively participate in policy formulation, leveraging dynamic monitoring and transparency requirements to ensure AI safety and social responsibility, achieving a win-win for innovation and compliance.




%当前AI风险的定义与分类,指出风险的复杂性。随着人工智能技术的进步,与其部署相关风险也在增加,因此有必要对主要监管机构采取不同方法进行比较。因此我们探讨了美国、欧盟和中国采取的政策的区别。AIA是全球第一部全面的人工智能法案,采用基于风险的方法将人工智能风险分为四类,相比之下,美国的ai监管技术以技术中立和行业自律为主。中国实施了强调安全控制和道德使用的ai治理框架,鼓励人工智能融入国家治理体系,并对风险进行了列举。事实上,现在还没有。人工智能的治理不仅需要各国的努力还需要全球协调应对措施,以在促进创新的同时降低风险。正如越来越多的学者所强调的,单一静态分散的法规是不够管理复杂全球人工智能生态系统。事实上,现在急需一个全球人工智能风险分类法对ai risk进行基于具体场景的分类和集成式的风险评估模型,将AI风险管理从静态分类转向动态情境评估。因为随着通用人工智能GPAI的多功能性和应用的不可预测性,并且静态的风险评估方法可能低估或高估某些实际风险,使得传统的基于应用领域的风险评估框架难以有效使用。需要强调的是,不仅是需要明确的法律法规对AIrisk进行监管,还需要对应的AI监管技术以确保人工智能系统在整个生命周期中安全、负责和透明,并遵守道德和安全标准。AI法案和监管技术在发展的同时还需要满足以人为本、可持续性的人工智能治理模型。

% !TEX root = main.tex

\section{Future research}\label{sec:future}
Below we list a few research questions, which we find interesting and
particularly promising directions after our contribution.

\para{Exact complexity for $3$-VASS}
We have shown that shortest paths in binary $3$-VASS are of at most triply-exponential length.
It is tempting to conjecture that actually the upper bound for the length of the paths is shorter,
at most doubly-exponential. We conjecture so and leave this conjecture to the future research.

\para{Example of a $3$-VASS with doubly-exponential path}
We have shown that shortest paths in binary $3$-VASS are of at most triple-exponential length.
However, currently we still do not know any example in which even a path of doubly-exponential length is needed,
it might be that paths of exponential length are sufficient leading to \pspace-completeness for binary $3$-VASS.
It would be very interesting to find an example of a binary $3$-VASS with shortest path between two configurations
being doubly exponential. An example of binary $4$-VASS of doubly-exponential shortest path is known (see Section 5 in~\cite{DBLP:conf/concur/Czerwinski0LLM20}). Maybe some modification of this $4$-VASS would allow to design a $3$-VASS with similar properties.

\para{Reachability for $d$-VASS with $d \geq 4$}
It is a natural question whether our techniques extend to higher dimensions.
The answer is: possibly yes, but we would need a few other structural results for $3$-VASS
to make a similar approach to $4$-VASS possible. In the proof of Lemma~\ref{lem:main} we do not only
use $2$-VASS reachability as a black box, but we use a deep understanding of the reachability relation in $2$-VASS
from~\cite{DBLP:conf/focs/0001CMOSW24}. Probably a similar understanding of the reachability relation for $3$-VASS would be needed
to advance understanding of $4$-VASS along our lines. 

In general it is very interesting to determine the complexity of the reachability problem for $d$-VASS.
We have excluded that for each $d \geq 3$ the problem is $\F_d$-completely, but it is still possible that
the problem is $\F_{d-C}$-complete for some constant $C \in \N$ and $d$ big enough.
Recall that in~\cite{DBLP:conf/fsttcs/CzerwinskiJ0LO23}
it was shown that the reachability problem for $(2d+4)$-VASS is $\F_d$-hard for any $d \geq 3$ and this
is the best currently known lower bound for arbitrary dimension.
Therefore the other natural possibility is that the reachability problem for $(2d+C)$-VASS is $\F_d$-complete for some
constant $C \in \N$. 

\para{Applications of the approximation technique}
Another natural research direction is to search for other applications of the technique of approximating the reachability sets,
which allows to lower the complexity down, below the size of the reachability set.
One particular case, which seems to be prone to such techniques is the $2$-VASS with some number of $\Z$-counters, namely counters, which can take values below zero.
The best complexity lower bound for the reachability problem in this model is \pspace-hardness inherited from~\cite{BlondinFGHM15},
while the best upper bound is Ackermann membership inherited from VASS reachability~\cite{LS19}.
The reachability sets for that systems are not necessarily semilinear.
This disqualifies most of the techniques relying on the semilinearity of reachability sets, but our techniques
seem to be promising for that model.





In this paper we have described our efforts in mechanizing the strand spaces framework~\cite{FHG98} in Coq.
To assess the flexibility of the approach and the usability of the library and of the proofs we have analyzed a variety of examples: a basic authentication protocol and some of its variants, the classical Needham-Schroeder-Lowe authentication protocol, and a recent key management API equipped with a key management policy.

Wherever possible, our mechanization remains faithful to the original pen-and-paper development of strand spaces.
At the same time, we put a lot of engineering effort to make the code and the proofs reusable.
For that, we have made the framework modular and parametric in the terms and the penetrator.
Additionally, we have developed a number of strands-specific tactics whose goal is to make the life of the protocol's analyst easier by removing some of the burden of these kinds of proofs.
Indeed, the tactics automate a number of intermediate steps enabling, in some cases, easy proof reuse.
For instance, the proof of the NSL responder's nonce secrecy
 required just one hour of work using the initiator's nonce secrecy.
The mechanization
gives the freedom to experiment with protocols and their properties, while retaining the unique ability of strand spaces-based analyses to give interesting insights on the inner workings of protocols.
With our experiments, we uncovered
and fixed issues, discarded
redundant or unused requirements, and significantly improved previous results on the analysis of key management policies, making it possible to formally prove the security of the \emph{secure templates} policy from \cite{BCFS-ccs10} (\cref{sec:casestudies}).

\cref{tab:simpleauth,tab:nsl} in \cref{sec:summary}  summarize the premises for each security property across the analyzed protocol variants. These premises are essential for our security proofs and offer important insights into the assumptions required to make a security protocol correct. The strand spaces model highlights this aspect, and the use of Coq and the \easystrands{} library further clarifies the minimal and necessary nature of these assumptions, reinforcing the model's ability to accurately capture security requirements.
With the insights from these experiments we also developed a new proof technique which we call \emph{protected predicate} technique that, in certain situations, simplifies the proofs making some previously challenging cases trivial.


Another advantage of having this mechanized platform is that it opens up new and interesting avenues of research.
\ifdefined\COLORDIFF
    \color{cbred}
\else
\fi
For instance, an intriguing enhancement to our framework would be the inclusion of algebraic intruders. We believe they can be implemented using at least two approaches, which we briefly outline below.

Given an equational theory $E$ over a signature $\mathit{FS}$, the first approach requires implementing $E$ as a (terminating and confluent) rewriting system \lstinline{rew_E}, and allow penetrators to use \lstinline{rew_E} to manipulate terms containing symbols of $\mathit{FS}$.
More concretely, we first need to create an instance of \easystrands{} terms with support for function symbols in $\mathit{FS}$, then we can extend the penetrator as:
\begin{lstlisting}
Inductive penetrator_strand : Σ -> Prop := ...
| PT_Eqn : forall (g h : 𝔸) i, replace g h rew_E  -> penetrator_strand (i, [⊖ g; ⊕ h]).
\end{lstlisting}
where \lstinline{replace g h rew_E} holds iff \lstinline{g} can be rewritten as \lstinline{h} under \lstinline{rew_E}.
This approach is inspired by that of Tamarin \cite{MSCB13}.

The second approach aligns  with the method used in DY*~\cite{DY}, where cryptographic primitives are modeled as functions that symbolically represent the actual primitives, e.g., \lstinline{dec (c, k) = (if c = enc (m, k) then m else Error)}.
With these definitions, the equational theory $E$ could be defined using Coq Setoids and used for terms in place of Leibniz equality.
This has the advantage to allow both honest parties and the intruder to transparently use the equational theory.
However, as observed by~\citet{DY}, this approach requires proving (at least) that $E$ is an equivalence relation respected by all functions, predicates, and protocol specifications which can be lengthy and tedious.
\ifdefined\COLORDIFF
    \color{black}
\else
\fi

Despite their age, strand spaces have been a catalyst for extensive research, leading to notable extensions that include authentication tests~\cite{guttman2000authentication}, process algebraic-style choice operators~\cite{YEMMS16},
 compositionality \cite{StrandComposition,StrandIndependence,StrandMixed}, and stateful protocols \cite{J12}.
Many of these advancements are crucial for enhancing the expressiveness and usability of the model.
Our plan is to enhance \easystrands{} by integrating these extensions, thereby enabling scalability to more realistic protocols.
Ultimately, this will help narrow the gap with state-of-the-art tools such as DY* \cite{DY}.
In terms of foundational research, an intriguing avenue involves closely examining the relationship between Paulson's inductive method \cite{Paulson94} and strand spaces. We plan to mechanize Paulson's method in Coq and conduct a comparative analysis to assess the relative merits of these two inductive methods.

Finally, we defined a maximal penetrator as the set of strands that do not violate sensitive cryptographic operations required for protocol security. This method is inspired by the approach in \cite{banaSymbolic} to achieve computational soundness and, to our knowledge, has not been explored in a purely symbolic context before. It allows for proving injective agreement without explicitly defining the Dolev-Yao attacker, which we showed to be \diff{strictly} subsumed by the maximal penetrator. Notably, this approach facilitates the composition of protocols proven secure under their respective maximal penetrators, provided they adhere to each other's constraints. We are currently extending this technique to protocols like NSL, where security relies on decryption capabilities.






%\vfill
\bibliographystyle{unsrt}
\bibliography{mainbib}
\end{document}
\renewcommand\refnamesec{参考文献} 
% \bibliographystylesec{plain}
\bibliographystylesec{unsrt}
\bibliographysec{引用}

%\newpage

% \section{Biography Section}
% If you have an EPS/PDF photo (graphicx package needed), extra braces are
%  needed around the contents of the optional argument to biography to prevent
%  the LaTeX parser from getting confused when it sees the complicated
% \begin{IEEEbiographynophoto}{John Doe}
% Use $\backslash${\tt{begin\{IEEEbiographynophoto\}}} and the author name as the argument followed by the biography text.
% \end{IEEEbiographynophoto}
% \end{document}



