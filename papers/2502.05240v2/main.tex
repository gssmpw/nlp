\documentclass[lettersize,journal]{IEEEtran}
\usepackage{amsmath,amsfonts}
\usepackage{algorithmic}
\usepackage{algorithm}
\usepackage{array}
\usepackage[caption=false,font=normalsize,labelfont=sf,textfont=sf]{subfig}
\usepackage{textcomp}
\usepackage{stfloats}
\usepackage{url}
\usepackage{verbatim}
\usepackage{graphicx}
\usepackage{lipsum} 
\usepackage{cite}
\hyphenation{op-tical net-works semi-conduc-tor IEEE-Xplore}

\usepackage{times}
\usepackage{microtype}
\usepackage{epsfig}
\usepackage[table,xcdraw]{xcolor}
\usepackage{graphicx}
\usepackage{caption}
\usepackage{float}
\usepackage{placeins}
\usepackage{color, colortbl}
\usepackage{stfloats}
\usepackage{enumitem}
\usepackage{tabularx}
\usepackage{xstring}
\usepackage{multirow}
\usepackage{xspace}
\usepackage{url}
\usepackage{subcaption}
\usepackage{xcolor}
\usepackage{tcolorbox}
\usepackage[hang,flushmargin]{footmisc}
\usepackage{color}
\usepackage{tikz}
\usepackage{bbding}
\usepackage{makecell}

\usepackage[edges]{forest}
\definecolor{hidden-draw}{RGB}{20,68,106}
\definecolor{hidden-pink}{RGB}{255,245,247}
\newcommand{\etal}{{\emph{et al.}}}
\definecolor{lightgrey}{gray}{0.92}
\definecolor{light\_double\_grey}{gray}{0.95}
\definecolor{lightred}{RGB}{251,49,153}
\definecolor{lightorange}{RGB}{244,230,217}

\definecolor{LightRed}{rgb}{1,0.92,0.92}
\definecolor{LightBlue}{rgb}{0.9,0.94,1}
\definecolor{LightGreen}{rgb}{0.9,1.0,0.88}
\newcommand{\lightgraytext}[1]{\textcolor[rgb]{0.5,0.5,0.5}{#1}}

\newcommand{\paratitle}[1]{\vspace{1.5ex}\noindent\textbf{#1}}
\newcommand{\definedsection}[1]{\noindent\textit{#1}\vspace{1.0ex}}

\newcommand{\thickcline}[1]{%
    \arrayrulecolor{black}\Xcline{#1}{2.5pt}%
    \arrayrulecolor{black}% 
}

\usepackage{hyperref}       % hyperlinks
\usepackage{url}            % simple URL typesetting
\usepackage{booktabs}       % professional-quality tables
\usepackage{amsfonts}       % blackboard math symbols
\usepackage{nicefrac}       % compact symbols for 1/2, etc.
\usepackage{microtype}      % microtypography
\usepackage{xcolor}         % colors

\newcommand{\zekun}[1]{\textcolor{blue}{[Zekun: #1]}}
\newcommand{\issue}[1]{\textcolor{red}{[Issue: #1]}}


\begin{document}


\title{Survey on AI-Generated Media Detection: From Non-MLLM to MLLM}

\author{Yueying Zou, Peipei Li*, Zekun Li, Huaibo Huang, Xing Cui, \\Xuannan Liu, Chenghanyu Zhang, Ran He,~\IEEEmembership{Fellow,~IEEE}
        % <-this % stops a space  
\thanks{Yueying Zou, Peipei Li, Xing Cui, and Xuannan Liu are with the School of Artificial Intelligence, and Chenghanyu Zhang is with the School of Science, all at Beijing University of Posts and Telecommunications, Beijing 100876, China. E-mail: {zouyueying2001, lipeipei, cuixing, liuxuannan, zhangchenghanyu}@bupt.edu.cn.}
\thanks{Zekun Li is with the School of Computer Science, University of California, Santa Barbara, USA. E-mail: zekunli@cs.ucsb.edu.}
\thanks{Huaibo Huang and Ran He are with the State Key Laboratory of Multimodal Artificial Intelligence Systems, CASIA, New Laboratory of Pattern Recognition, CASIA, and School of Artificial Intelligence, University of Chinese Academy of Sciences, Beijing 100190, China. E-mail: {huaibo.huang, rhe}@cripac.ia.ac.cn.}
\thanks{Peipei Li* is the corresponding author. E-mail: lipeipei@bupt.edu.cn.}
}
%\thanks{Huaibo Huang and Ran He are with the State Key Laboratory of Multimodal Artificial Intelligence Systems, CASIA, New Laboratory of Pattern Recognition, CASIA, and School of Artificial Intelligence, University of Chinese Academy of Sciences, Beijing 100190, China. E-mail: {huaibo.huang, rhe}@cripac.ia.ac.cn.}

% <-this % stops a space

%\thanks{Manuscript received April 19, 2021; revised August 16, 2021.}

% The paper headers
\markboth{Journal of \LaTeX\ Class Files,~Vol.~14, No.~8, August~2021}%
{Shell \MakeLowercase{\textit{et al.}}: A Sample Article Using IEEEtran.cls for IEEE Journals}

%\IEEEpubid{0000--0000/00\$00.00~\copyright~2021 IEEE}
% Remember, if you use this you must call \IEEEpubidadjcol in the second
% column for its text to clear the IEEEpubid mark.

\maketitle

\begin{abstract}
The proliferation of AI-generated media poses significant challenges to information authenticity and social trust, making reliable detection methods highly demanded. Methods for detecting AI-generated media have evolved rapidly, paralleling the advancement of Multimodal Large Language Models (MLLMs). Current detection approaches can be categorized into two main groups: Non-MLLM-based and MLLM-based methods. The former employs high-precision, domain-specific detectors powered by deep learning techniques, while the latter utilizes general-purpose detectors based on MLLMs that integrate authenticity verification, explainability, and localization capabilities. Despite significant progress in this field, there remains a gap in literature regarding a comprehensive survey that examines the transition from domain-specific to general-purpose detection methods. This paper addresses this gap by providing a systematic review of both approaches, analyzing them from single-modal and multi-modal perspectives. We present a detailed comparative analysis of these categories, examining their methodological similarities and differences. Through this analysis, we explore potential hybrid approaches and identify key challenges in forgery detection, providing direction for future research. Additionally, as MLLMs become increasingly prevalent in detection tasks, ethical and security considerations have emerged as critical global concerns. We examine the regulatory landscape surrounding Generative AI (GenAI) across various jurisdictions, offering valuable insights for researchers and practitioners in this field.

\end{abstract}

\begin{IEEEkeywords}
AI-generated Media detection, MLLM, deep learning, literarture survey
%\LaTeX, paper, template, typesetting.
\end{IEEEkeywords}

\section{Introduction}


\begin{figure}[t]
\centering
\includegraphics[width=0.6\columnwidth]{figures/evaluation_desiderata_V5.pdf}
\vspace{-0.5cm}
\caption{\systemName is a platform for conducting realistic evaluations of code LLMs, collecting human preferences of coding models with real users, real tasks, and in realistic environments, aimed at addressing the limitations of existing evaluations.
}
\label{fig:motivation}
\end{figure}

\begin{figure*}[t]
\centering
\includegraphics[width=\textwidth]{figures/system_design_v2.png}
\caption{We introduce \systemName, a VSCode extension to collect human preferences of code directly in a developer's IDE. \systemName enables developers to use code completions from various models. The system comprises a) the interface in the user's IDE which presents paired completions to users (left), b) a sampling strategy that picks model pairs to reduce latency (right, top), and c) a prompting scheme that allows diverse LLMs to perform code completions with high fidelity.
Users can select between the top completion (green box) using \texttt{tab} or the bottom completion (blue box) using \texttt{shift+tab}.}
\label{fig:overview}
\end{figure*}

As model capabilities improve, large language models (LLMs) are increasingly integrated into user environments and workflows.
For example, software developers code with AI in integrated developer environments (IDEs)~\citep{peng2023impact}, doctors rely on notes generated through ambient listening~\citep{oberst2024science}, and lawyers consider case evidence identified by electronic discovery systems~\citep{yang2024beyond}.
Increasing deployment of models in productivity tools demands evaluation that more closely reflects real-world circumstances~\citep{hutchinson2022evaluation, saxon2024benchmarks, kapoor2024ai}.
While newer benchmarks and live platforms incorporate human feedback to capture real-world usage, they almost exclusively focus on evaluating LLMs in chat conversations~\citep{zheng2023judging,dubois2023alpacafarm,chiang2024chatbot, kirk2024the}.
Model evaluation must move beyond chat-based interactions and into specialized user environments.



 

In this work, we focus on evaluating LLM-based coding assistants. 
Despite the popularity of these tools---millions of developers use Github Copilot~\citep{Copilot}---existing
evaluations of the coding capabilities of new models exhibit multiple limitations (Figure~\ref{fig:motivation}, bottom).
Traditional ML benchmarks evaluate LLM capabilities by measuring how well a model can complete static, interview-style coding tasks~\citep{chen2021evaluating,austin2021program,jain2024livecodebench, white2024livebench} and lack \emph{real users}. 
User studies recruit real users to evaluate the effectiveness of LLMs as coding assistants, but are often limited to simple programming tasks as opposed to \emph{real tasks}~\citep{vaithilingam2022expectation,ross2023programmer, mozannar2024realhumaneval}.
Recent efforts to collect human feedback such as Chatbot Arena~\citep{chiang2024chatbot} are still removed from a \emph{realistic environment}, resulting in users and data that deviate from typical software development processes.
We introduce \systemName to address these limitations (Figure~\ref{fig:motivation}, top), and we describe our three main contributions below.


\textbf{We deploy \systemName in-the-wild to collect human preferences on code.} 
\systemName is a Visual Studio Code extension, collecting preferences directly in a developer's IDE within their actual workflow (Figure~\ref{fig:overview}).
\systemName provides developers with code completions, akin to the type of support provided by Github Copilot~\citep{Copilot}. 
Over the past 3 months, \systemName has served over~\completions suggestions from 10 state-of-the-art LLMs, 
gathering \sampleCount~votes from \userCount~users.
To collect user preferences,
\systemName presents a novel interface that shows users paired code completions from two different LLMs, which are determined based on a sampling strategy that aims to 
mitigate latency while preserving coverage across model comparisons.
Additionally, we devise a prompting scheme that allows a diverse set of models to perform code completions with high fidelity.
See Section~\ref{sec:system} and Section~\ref{sec:deployment} for details about system design and deployment respectively.



\textbf{We construct a leaderboard of user preferences and find notable differences from existing static benchmarks and human preference leaderboards.}
In general, we observe that smaller models seem to overperform in static benchmarks compared to our leaderboard, while performance among larger models is mixed (Section~\ref{sec:leaderboard_calculation}).
We attribute these differences to the fact that \systemName is exposed to users and tasks that differ drastically from code evaluations in the past. 
Our data spans 103 programming languages and 24 natural languages as well as a variety of real-world applications and code structures, while static benchmarks tend to focus on a specific programming and natural language and task (e.g. coding competition problems).
Additionally, while all of \systemName interactions contain code contexts and the majority involve infilling tasks, a much smaller fraction of Chatbot Arena's coding tasks contain code context, with infilling tasks appearing even more rarely. 
We analyze our data in depth in Section~\ref{subsec:comparison}.



\textbf{We derive new insights into user preferences of code by analyzing \systemName's diverse and distinct data distribution.}
We compare user preferences across different stratifications of input data (e.g., common versus rare languages) and observe which affect observed preferences most (Section~\ref{sec:analysis}).
For example, while user preferences stay relatively consistent across various programming languages, they differ drastically between different task categories (e.g. frontend/backend versus algorithm design).
We also observe variations in user preference due to different features related to code structure 
(e.g., context length and completion patterns).
We open-source \systemName and release a curated subset of code contexts.
Altogether, our results highlight the necessity of model evaluation in realistic and domain-specific settings.





\section{Background}
\label{sec:back}

\subsection{Generative Approaches for Different Modalities}
This section examines the various types of content generated by generative models, including text, image, video, audio, and multimodal content, along with the methods used in each domain.
% \begin{figure*}[!ht]
%   \centering
%     \includegraphics[width=1.0\linewidth]{Fig/illustration.pdf}
%     \caption{Illustrating different multimedia generation processes using large AI models}
%     \label{fig:illustration}
% \end{figure*}

\textbf{Text:} 
%介绍常用模型+multimodal的时候作为prompt
In AI-generated media, text generation is primarily achieved using Large Language Models (LLMs) like GPT-4o~\cite{openai2024}, LLaMA3~\cite{dubey2024llama}, and Claude 3.5 Sonnet~\cite{anthropic2023claude3}. These models leverage vast datasets to perform complex language tasks, including news creation~\cite{kreps2022all}, code generation~\cite{idrisov2024program}, and script drafting~\cite{buschek2024collage}.
Furthermore, text serves as a foundational input for generating other modalities. For instance, in text-to-image generation, models translate descriptive text prompts into corresponding visual content, bridging the gap between textual descriptions and visual representations. 

\textbf{Image:} 
%Text-guide Image Generation with Diffusion Models // LLMs
In the past two years, research powered by MLLMs has increasingly focused on achieving a more intuitive and interactive image generation process. As their foundation, diffusion models (DMs) are the dominant technology in image generation. Current research on diffusion models primarily revolves around three key formulations: denoising diffusion probabilistic models (DDPMs)~\cite{ho2020denoising}, score-based generative models (SGMs)~\cite{song2019generative}, and stochastic differential equations (SDEs)~\cite{song2020score}. More advanced models guided by text have also emerged, including Stable Diffusion V2~\cite{rombach2022high}, Google Imagen2~\cite{deepmind2023imagen2}, and Midjourney~\cite{midjourney2023}. Notably, DALL·E 3~\cite{betker2023improving}, which integrates with GPT-4 and leverages the powerful text understanding capabilities of GPT-4. GPT-4 first processes and interprets the text, generating a structured semantic representation that is then used by DALL·E 3 for image generation. Users can interact with GPT-4 to modify aspects of the generated image, such as colors, styles, elements, or details. Additionally, MLLMs play a crucial role in image generation by unifying textual and visual modalities to create more dynamic outputs. Important examples include MiniGPT-4~\cite{zhu2023minigpt}, LLaVA~\cite{liu2024visual}, and Qwen-VL~\cite{bai2023qwen}.
%写一段生成的特点

\textbf{Video:} 
%Text-guide video Generation with Diffusion Models // LLMs
Intuitively, a video is an expansion of a series of images over time. Recently, DMs have emerged as the leading framework for Text-to-Video (TTV) generation. Within the DMs framework, there are two main categories: (1) frame-wise DMs and (2) spatio-temporal diffusion models. Frame-wise DMs, such as Meta’s Make-A-Video~\cite{singer2022make}, and DALL·E 2~\cite{openai2023dalle2} (\textit{when adapted for video}), apply the diffusion process to each individual frame. However, they must carefully address challenges related to maintaining consistency and smooth transitions between consecutive frames to avoid flickering or object deformation. On the other hand, spatio-temporal DMs, like SORA~\cite{brooks2024video}, Google DeepMind’s Veo~\cite{deepmind2023veo}, and Stable Video Diffusion~\cite{blattmann2023stable}, focus on capturing both spatial and temporal coherence across the entire video sequence. Additionally, similar to the previously introduced Image MLLMs, Video MLLMs also leverage the exceptional comprehension capabilities of LLMs to enhance video realism. Recent successful examples, such as LLaMA-VID~\cite{li2025llama} and VideoChat2~\cite{li2024mvbench}, through extensive use of diverse multi-modal data, including text, image, and video, and multi-stage alignment training, have achieved improved video understanding based on LLMs.

\textbf{Audio:}  
Most deep learning-based speech synthesis systems typically consist of two main components: (1) a Text-to-Speech (TTS) model that converts text into an acoustic feature, such as a mel-spectrogram, and (2) a vocoder that generates a time-domain speech waveform from this acoustic feature. Notably, DDPMs~\cite{ho2020denoising} can also be applied to audio generation~\cite{jeong2021diff}. Jeong et al. were the first to apply DDPMs for mel-spectrogram generation, where noise is transformed into a mel-spectrogram through diffusion time steps. Models like AudioLDM~\cite{liu2023audioldm}, Make-An-Audio~\cite{huang2023make}, and TANGO~\cite{ghosal2023text} all leverage the Latent Diffusion Model (LDM). Particularly, TANGO~\cite{ghosal2023text} uses LLMs as a frozen, instruction-tuned text encoder to provide strong text representation capabilities. Meanwhile, WavJourney~\cite{liu2023wavjourney} focuses on generating structured scripts and enabling user interaction for storytelling audio creation, UniAudio~\cite{yang2023uniaudio} emphasizes tokenization and sequence processing for various audio types, aiming to build a robust, adaptable universal audio generation model. The growing use of LLMs in audio generation—whether as conditioners for specific tasks~\cite{ghosal2023text}, sources of inspiration~\cite{yang2023uniaudio}, or interactive agents~\cite{liu2023wavjourney}—is transforming how we interact with sound and music.

\textbf{Multimodal:}
Multimodal generation represents the culmination of advancements across individual modalities, integrating text, image, video, and audio into cohesive and context-aware outputs. For example, 
Text-to-Image (TTI)~\cite{rombach2022high, deepmind2023imagen2, midjourney2023, bai2023qwen, liu2024visual, zhu2023minigpt}, Text-to-Video (TTV)~\cite{singer2022make,openai2023dalle2,brooks2024video, deepmind2023veo, blattmann2023stable} and Text-to-Speech (TTS)~\cite{ghosal2023text, liu2023wavjourney, yang2023uniaudio} tasks are multimodal systems that extend text-only generation by using textual prompts to guide visual content generation. Multimodal generation acts as an integrative framework, combining the specialized capabilities of single-modal systems to achieve a holistic understanding of content.


\subsection{Definition and Formulation}
%介绍一下有些有可解释性
\begin{enumerate}
    \item \textbf{Authenticity Detection}:
    Authenticity detection is a binary classification task that determines whether a given piece of media $X$ is authentic or AI-generated. Formally, the task is defined as:
$D = \{(X_i, Y_i)\}_{i=1}^N$
where $X_i$ represents the media sample (\textit{e.g., an image, video, or text}), and $Y_i\in \{real, fake\}$ indicates its authenticity. The detection model $F_\theta$, parameterized by $\theta$, maps input data to authenticity labels:
$F_\theta : X \rightarrow \{real, fake\}$.
The training objective is to optimize $\theta$ by minimizing a predefined loss function:
    \begin{equation}\label{eqn-1}
    \theta = \arg \min_{\theta} \frac{1}{N} \sum_{i=1}^{N} \text{Loss}(X_i, Y_i, \theta)
    \end{equation}
Extensions of this task may involve embedding watermarks during or after the generation process for post-verification, supporting forgery authentication, and copyright protection.


    \item \textbf{Explainability}:
    Explainability aims to provide human-interpretable reasoning for detection decisions, typically presented as natural language explanations or visual representations of salient features~\cite{dang2024explainable, zhao2024explainability}. The task can be further categorized into three levels: direct explanation (\textit{direct identification of forgery clues with few-shot in-context examples}), reasoning-based explanation (\textit{multi-hop reasoning and logical consistency evaluation}), and free-form fine-grained analysis (\textit{fine-grained analysis of forgery aspects, aligned with a predefined taxonomy of forgery cues}). For a given input $X$, generate an explanation $E$ that: (1) identifies relevant forgery clues $\mathcal{C} = \{c_1, c_2, \dots, c_k\}$; and (2) supports multi-layer forgery analysis (\textit{low-level, mid-level, high-level}). Formally, the task is defined as: 
    \begin{equation}\label{eqn-2}
    g(f(X; \theta), X; \phi) = E, \\
    \mathcal{L}_{\text{exp}} = \text{KL}(p(E \mid X, Y) \| q(C))
    \end{equation}
    where $p(E \mid X, Y)$ is the generated explanation distribution, and $f(X; \theta)$ is the detection model output.
   

    \item \textbf{Localization}:
    Forgery localization identifies specific regions or segments within the input that are manipulated or generated. This task is commonly framed as: Pixel-wise classification (\textit{for images, this involves predicting a forgery heatmap where each pixel indicates the likelihood of forgery}); Segment-wise classification (\textit{for videos, this extends to identifying forged regions across multiple frames with temporal consistency}); Bounding box detection (\textit{for coarse localization, bounding boxes can be used to enclose suspected forged regions}). Given an input $X\in \mathbb{R}^{H \times W \times C}$ (\textit{e.g., an image}), the localization model $h$, parameterized by $psi$, outputs one or more of the following: A forgery heatmap: $M \in [0, 1]^{H \times W} $, where $M(i, j)$ indicates the likelihood of forgery at pixel $(i, j)$. A binary mask: $\hat{M} \in \{0, 1\}^{H \times W} $, derived by applying a threshold $\tau$ to the heatmap. A set of bounding boxes: $ B=\{b_1, b_2, \dots, b_k\} $, where each $b_i=[x_{\text{min}}, y_{\text{min}}, x_{\text{max}}, y_{\text{max}}]$ specifies the coordinates of a forged region. The model can be represented as: 
    \begin{equation}\label{eqn-3}
        \begin{aligned}
        h(X; \psi) &= \{ \hat{M}, M, B \}
        \end{aligned}
    \end{equation}
        where $M \in [0, 1]^{H \times W}$, $\hat{M} \in \{0, 1\}^{H \times W}$, $B \in \mathbb{R}^{k \times 4}$.
\end{enumerate}






\section{MLLM-based Detector}
\label{sec:mllm}
\begin{figure*}[!ht]
  \centering
    \includegraphics[width=1.0\linewidth]{Fig/MLLM-text.pdf}
    \caption{Illustrating of MLLM-based detection methodologies for AI-generated text}
    \label{fig:MLLM-text}
\end{figure*}
This paper primarily focuses on MLLM-based methods for detecting AI-generated media. Therefore, we first introduce relevant MLLM-based approaches. Before diving into these methods, it is worth noting that previous works~\cite{lin2024detecting, deng2024survey, yu2024fake} have reviewed some Non-MLLM-based methods.

As a product of advancements in Natural Language Processing (NLP) and Computer Vision (CV), MLLMs represent a significant milestone in AI. Compared to traditional Non-MLLM detection methods, MLLMs leverage their multimodal nature and reasoning abilities to offer several distinct advantages. First, their human-like cognitive abilities, enabled by Chain-of-Thought (CoT) and In-Context Learning (ICL), allow MLLMs not only to detect potential forgery traces in AI-generated media but also to reason about and explain their decision-making processes. Additionally, textual input and output empower MLLMs to support flexible query formats and provide human-interpretable contextual explanations. In terms of forgery analysis potential, MLLMs excel at identifying and describing visual forgery cues, conducting adaptive analyses driven by textual prompts, and validating authenticity through causal reasoning. These capabilities make MLLMs highly effective in supporting forgery detection in AI-generated media, particularly in identifying and describing forgery traces, performing flexible, text-driven analyses, and verifying authenticity through causal reasoning. In contrast, traditional Non-MLLM detection methods primarily focus on single-modal feature extraction and classification, often lacking interpretability and causal analysis capabilities. By addressing these limitations, MLLMs demonstrate their effectiveness in supporting AI-generated media detection. In the following sections, we will analyze the underlying technologies and methodologies in detail.
\subsection{Text}
 % 替换为你的图像文件
%\input{Table/attack_taxonomy}
\subsubsection{\textbf{Authenticity}}
MLLMs can be used in judgment of the authenticity of AI-generated text. The methods can be divided into five types: Statistical-based methods, Prompt-engineering, Self-consistency, Multi-Author, and Watermarking, all of which leverage the capability of MLLMs, as shown in Fig.~\ref{fig:MLLM-text} (a).
% \zekun{Statistical-based methods, prompt engineering, and self-consistency seem to be methods, whereas multi-author and watermarking appear to be issues. Placing all five at the same level might not be accurate.}
\begin{itemize}
\item \textbf{Statistical-based}
By examining statistical differences in language use, such as probability distributions or specific features, zero-shot methods can distinguish human writing from GPT-generated text, leveraging both shallow and deep characteristics. For shallow features, HowkGPT~\cite{vasilatos2023howkgpt} computes perplexity scores, establishing thresholds to distinguish their origins. 
DNA-GPT~\cite{yang2023dna} uses N-gram analysis or probability divergence. In the context of deep features, DetectLLM~\cite{su2023detectllm} introduces two methods DetectLLM-LRR and DetectLLM-NRR both leveraging log-rank information. DetectLLM-NRR focuses on accuracy with fewer perturbations, while DetectLLM-LRR emphasizes speed and efficiency. DetectGPT~\cite{mitchell2023detectgpt} leverages the negative curvature regions of the model's log probability function, without requiring additional training. Subsequently, Fast-DetectGPT~\cite{bao2023fast} introduces the concept of conditional probability curvature, which improves upon DetectGPT by replacing the computationally intensive perturbation step with a faster sampling step.

\item \textbf{Prompt Engineering}
Some researchers leverage MLLMs to detect In the LOKI study ~\cite{ye2024loki}, results show that MLLMs achieve only 61.5\% accuracy in judgment tasks asking, `Is the provided text generated by AI?'. However, accuracy increases to 89.2\% when the task is reformulated into a multiple-choice format, such as `Which of the following text is generated?'. The improvement stems from MLLMs' strength in contrastive analysis, as binary choice tasks allow direct comparison of subtle differences, unlike isolated judgment tasks relying solely on internal feature detection. Bhattacharjee et al.~\cite{bhattacharjee2024fighting} find that even though ChatGPT struggles to detect AI-generated text, it performs well in identifying human-written text. Zhang et al.~\cite{zhang2024detection} design various prompts, such as Base task-specific prompts, Style-specific prompts, and Evasion-optimized prompts to show the vulnerability of detectors.

\item \textbf{Self-consistency}
The self-consistency hypothesis suggests that, within a given input context, machine-generated text tends to make more predictable choices in words or tokens compared to humans. DetectGPT-SC~\cite{wang2023detectgpt} masks a portion of the input text and uses LLM to predict the masked words or tokens. It measures the consistency between the predictions and the original text to determine whether the text was generated by the LLMs. Additionally, numerous studies~\cite{nguyen2024simllm,zhu2023beat,mao2024raidar, hao2024learning} focus on utilizing LLMs to revise or rewrite sentences or phrases and then calculate the similarity between the original and the rewritten versions. SimLLM~\cite{nguyen2024simllm} uses candidate LLMs to proofread an input text, generating multiple versions and comparing their similarity to the original text to determine if the text was generated by an LLM. Zhu et al.~\cite{zhu2023beat} use ChatGPT to revise and analyze the similarity. Moreover, Raidar~\cite{mao2024raidar} prompts LLMs to rewrite the text, calculate the editing distance of the output, and exhibit high robustness in new content and multi-domain applications. Rewritelearning~\cite{hao2024learning} trains an LLM to rewrite input text, minimizing edits for AI-generated media while applying more edits to human-written text.


\item \textbf{Multi-Author}
Multi-Author core idea is to distinguish different authors (\textit{varying degrees of LLM intervention, e.g., partly written by AI, polished by AI}) rather than simply classify text as human-written or AI-generated. 
MIXSET~\cite{zhang2024llm} is the first dataset comprising human-written, machine-generated, and
human/LLM-refined machine-generated texts (MGTs) and focuses on multi-author binary classification. From then on, LLM-DetectAIve~\cite{abassy2024llm} provides a four-way classification task with the addition of three labels: ``human-written/machine-written", ``machine-written, then machine-humanized", ``human-written, then machine-polished". Beemo~\cite{artemova2024beemo} is a benchmark designed to evaluate AI-generated text detection in multi-author scenarios. LLMDetect~\cite{cheng2024beyond} introduces two tasks: LLM Role Recognition (LLM-RR) for multi-class classification and LLM Influence Measurement (LLM-IM) for quantifying LLM involvement, showing fine-tuned PLM-based models outperform advanced LLMs in detecting their outputs. 

\item \textbf{Watermarking}
To watermark LLMs, Kirchenbauer et al.~\cite{kirchenbauer2023watermark, kirchenbauerreliability} propose a method involving inserting signatures during the decoding stage. These methods categorize the vocabulary into ``red" and ``green" lists, restricting the LLM to decoding tokens from the green list. Subsequently, Christ et al.~\cite{christ2024undetectable} and Unigram-Watermark~\cite{zhaoprovable} suggest various algorithms for splitting the red and green lists or sampling tokens from the green list's probabilistic distribution to enhance the interpretability and robustness of watermarking mechanisms during the inference process. PersonaMark~\cite{zhang2024personamark} is a personalized text watermarking method that leverages sentence structure and user-specific hashing. By embedding unique watermarks, it guarantees copyright protection and user tracking of generated text while maintaining the text's naturalness and generation quality.
\end{itemize}
\begin{figure}[t!]
  \centering
  \includegraphics[width=\linewidth]{Figures/4_asr_eval.pdf}
  \caption{Performance of different attack methods. Surprisingly, simply intervening information from the template region (i.e., \textsc{TempPatch}) can significantly increase attack success rates.}
  \label{fig:asr_eval}
\end{figure}


\subsubsection{\textbf{Explainability}}
Traditionally, detecting LLM-generated text is often framed as a binary classification task. Methods are shown in Fig.~\ref{fig:MLLM-text} (b). However, there is also an ``undecided" category~\cite{ji2024detecting}, which is used to represent ambiguous texts that may originate from either humans or AI. This category is crucial for enhancing the explainability of detection results. By incorporating it, the system not only improves its reliability but also allows ordinary users to better understand the detection outcomes. Ji et al.~\cite{ji2024detecting} construct a dataset containing LLMs-generated text and human-generated text. Three human annotators are tasked with producing ternary labels along with explanation notes. They identify eight categories of explanations provided by human annotators, including spelling errors, grammatical errors, perplexity, logical errors, and unnecessary repetition.


\subsubsection{\textbf{Localization}}
Methods of localization are shown in Fig.~\ref{fig:MLLM-text} (a).
Gruda et al.~\cite{gruda2024three} have proposed three ways that ChatGPT can assist in academic writing. Similar to ``Multi-Author", LLMs play different roles based on varying user needs, from creating and drafting to polishing. The text totally written by AI is easier to detect than human-collaborated text. Some researchers quantify the involvement ratio of LLMs in content creation and localize which part of a phrase is written by AI.  LLMDetect~\cite{cheng2024beyond} offers an involvement ratio strategy. GigaCheck~\cite{tolstykh2024gigacheck} combines fine-tuned general-purpose LLMs to distinguish human-written texts from LLM-generated texts. Additionally, it employs a DETR-like model to localize AI-generated intervals in human-machine collaborative texts.

\begin{figure*}[!ht]
  \centering
    \includegraphics[width=1.0\linewidth]{Fig/MLLM-Image.pdf}
    \caption{Illustrating of MLLM-based detection methodologies for AI-generated images. ``Mask + Image → Text" approach is reproduced from~\cite{li2024forgerygpt}, ``Text + Image → Mask" approach is reproduced from~\cite{huang2024sida}, and Independent Mask Localization method is adapted from~\cite{lian2024large}}
    \label{fig:MLLM-image}
\end{figure*}

\subsection{Image}
\subsubsection{\textbf{Authenticity}}
For assessing image authenticity using MLLMs, we divide the approach into three categories: Prompt engineering, Fine-tuning, and Integration with external detectors, as shown in Fig.~\ref{fig:MLLM-image} (a).
\begin{itemize}
\item \textbf{Prompt-engineering}
Prompt engineering can be categorized into four types: Judgment prompts, Multiple-choice prompts, Score prompts, and In-context prompts. 
For \textbf{Judgment prompts}, the model is directly queried with questions (\textit{e.g., `Is the provided image generated by AI?'}~\cite{ye2024loki} \textit{, `Is this an example of a real image?'}~\cite{shi2024shield, huang2024visualcritic}). However, variations in phrasing, such as replacing ``real" with ``bonafide" or ``spoof"~\cite{shi2024shield}. LOKI~\cite{ye2024loki} shows that MLLMs may not be good at judging whether the input image is generated by AI. Mantis-8B shows the best performance only achieving 54.6\% accuracy, compared to 80.1\% for human evaluators. Nevertheless, Jia et al.~\cite{jia2024can} suggest that guiding MLLMs to focus on regions of an image likely to contain forgery clues (\textit{e.g., `Analyze the eye area'}) can enhance detection effectiveness. About \textbf{Multiple-choice prompts}, it gives MLLMs some choice (\textit{e.g., `Which of the following image is the generated image?'}~\cite{ye2024loki}). LOKI shows that MLLMs perform better in multiple-choice tasks compared to judgment tasks. GPT-4o achieves the best results, with an overall accuracy of 80.8\%, which is close to the human accuracy of 84.5\%. 
Also for \textbf{Score prompts}, MLLMs are tasked with providing a probability score for their judgments. Jia et al.~\cite{jia2024can} observe that such requests result in a 100\% rejection rate by GPT-4V.
In addition, \textbf{In-context prompts}, also referred to as one-shot questions, MLLMs are provided with examples to guide their detection (\textit{eg., The first set of images is of a real face, is the second set of images a real
face or a spoof face? Please answer `this image is a real face'})~\cite{shi2024shield}. It shows that MLLMs may give more accurate answers. Prompt engineering enhances the performance of MLLMs in detecting AI-generated images through flexible prompt design. However, it is highly sensitive to the specific design choices, with task formats and phrasing significantly impacting effectiveness. Additionally, its robustness may be limited in complex scenarios, particularly when faced with diverse or shifting data distributions.

\item \textbf{Fine-tuning}
To improve the MLLMs’ detection capabilities, fine-tuning involves adjusting model parameters using targeted datasets. $\textit{X}^2$-DFD~\cite{chen2024textit} comprises three modules: Model Feature Assessment (MFA), Strong Feature Strengthening (SFS) and Weak Feature Supplementing (WFS). MFA evaluates and ranks forgery-related features, while SFS leverages the top-ranked features to create an explainable training dataset. This dataset is used to fine-tune the MLLM, enhancing both detection accuracy and explainability. Similarly, Fakeshield~\cite{xu2024fakeshield} includes two key components. The Domain Tagging-Enhanced Forgery Detection Module generates domain-specific tags (\textit{e.g., Photoshop, DeepFake, AIGC}) and integrates image features with instruction-based textual inputs to produce tampering detection results and explanations. Lightweight LoRA fine-tuning techniques are employed to improve detection efficiency and maintain strong explainability.


\item \textbf{External detectors}
From the experiment results of ~\cite{ye2024loki}, we can find that MLLMs are not good at directly judging whether the image is generated by AI. Researchers have proposed integrating MLLMs with external detectors to enhance their feature discrimination capabilities. For instance, $\textit{X}^2$-DFD~\cite{chen2024textit} evaluates forgery-related features and ranks them based on detection performance, utilizing external detectors (\textit{e.g., blending-based detectors}~\cite{lin2025fake}) to strengthen the handling of weak feature areas. These external prediction scores are then incorporated into the MLLMs. Additionally, FFAA~\cite{huang2024ffaa} introduces a multi-answer intelligent decision system, which combines a cross-modal fusion module and a classification module to identify the best answer that aligns with an image's authenticity. This integration significantly enhances the accuracy and reliability of detection.

\end{itemize}

\subsubsection{\textbf{Explainability}}
The explainability of MLLMs is a remarkable feature, and recent studies have increasingly explored its potential. The methods are illustrated in Fig.~\ref{fig:MLLM-image} (b). Some works~\cite{jia2024can,shi2024shield,lian2024large,huang2024sida} directly query MLLMs with prompts such as `explain what the artifacts are'. However, prior investigations~\cite{jia2024can, shi2024shield} reveal that directly generating textual explanations often leads to hallucinations or overthinking, producing inaccurate outcomes or refusal to respond. Moreover, MLLMs often struggle to comprehensively perceive all relevant features, limiting their effectiveness in explainability. To address these limitations, researchers have employed approaches such as fine-tuning MLLMs~\cite{chen2024textit, huang2024ffaa, xu2024fakeshield} or integrating external modules~\cite{sun2024forgerysleuth}. These approaches aim to establish a comprehensive evaluation framework by categorizing features into three levels: low-level pixel features (\textit{e.g., noise, color, texture, sharpness, and AI-generated fingerprints}), middle-level visual features (\textit{e.g., traces of tampered regions or boundaries, lighting inconsistencies, perspective relationships, and physical constraints}), and high-level semantic anomalies (\textit{e.g., content that contradicts common sense, incites, or misleads}). This multi-level feature evaluation provides a holistic approach to enhancing the detection capabilities and explainability of MLLMs.


\subsubsection{\textbf{Localization}}
Binary classification tasks in forgery detection cannot inherently provide detailed insights into tampered regions. This limitation becomes more pronounced as modern generative models employ increasingly sophisticated forgery techniques, such as localized modifications (\textit{e.g., altering facial features like eyes or mouths}) or holistic image synthesis. To address this challenge, mask localization has emerged as a more flexible and effective approach, effectively capturing subtle forgeries and adapting to diverse scenarios. Existing methods can be categorized into two primary approaches: \textbf{Image-Text-Mask Alignment Localization} and \textbf{Independent Mask Localization}. The methods are illustrated in Fig.~\ref{fig:MLLM-image} (b).

\begin{itemize}
\item \textbf{Image-Text-Mask Alignment Localization}
In this approach, ``image" refers to the input image, ``text" represents the explainable textual output about forgery, and ``mask" indicates the localized forgery region. Further, methods in this category can be divided into two subcategories: ``Mask + Image → Text" and ``Text + Image → Mask". For \textbf{``Mask + Image → Text"}, Forgerygpt~\cite{li2024forgerygpt} employs a Mask Extraction Module to capture pixel-level features of tampered regions, using the FL-Expert to generate precise forgery masks and the Mask Encoder to transform mask features into tokens compatible with the MLLM. These mask, image, and text features are then fused and input into the MLLM, enabling accurate localization of tampered regions along with explainable outputs. 
About \textbf{``Text + Image → Mask"}, Fakeshield~\cite{xu2024fakeshield} introduces a tamper comprehension module to enhance the detection of forgery regions by aligning descriptive features of tampered areas with visual attributes. By integrating segmentation techniques based on the Segment Anything Model, it generates precise forgery masks. Similarly, SIDA~\cite{huang2024sida} extends MLLM with specialized tokens and leverages multi-head attention for the precise fusion of detection and segmentation features. Editscout~\cite{nguyen2024editscout} combines an MLLM-based reasoning query generation module and a segmentation model, where the [SEG] token bridges user prompts and images to produce binary masks for edited regions with minimal fine-tuning.

\item \textbf{Independent Mask Localization}
ForgeryTalker~\cite{lian2024large} proposes a method that employs an independent mask decoder to directly generate mask predictions, offering a more modular approach to forgery detection. This approach offers a modular method for forgery detection and sends tokens to LLMs to generate explainable text outputs.
\end{itemize}

\begin{figure*}[!ht]
  \centering
    \includegraphics[width=1.0\linewidth]{Fig/MLLM-Video_Audio.pdf}
    \caption{Illustrating of MLLM-based detection methodologies for AI-generated Video and Audio}
    \label{fig:MLLM-Video&Audio}
\end{figure*}

\subsection{Video}
MLLMs integrate linguistic and visual data to process videos by leveraging LLMs and connecting them with modality-specific encoders through interfaces like Q-former. Notable open-source Video-LLMs include: \textbf{VideoChat}~\cite{li2023videochat}: a chat-centric interactive system primarily designed for video content understanding and multimodal generation; \textbf{VideoChatGPT}~\cite{maaz2023video}: combines visual encoders with LLMs for video-based conversational analysis; ~\textbf{Video-LLaMA}~\cite{zhang2023video}: integrates audio and visual signals from videos using Q-former, enabling efficient handling of multimodal tasks; \textbf{LLaMA-VID}~\cite{li2025llama}: represents video frames as tokens containing contextual and content information, significantly improving video processing efficiency.

Currently, the primary focus of Video Anomaly Detection (VAD) tasks using MLLMs lies in identifying anomalies in real-world scenarios, such as criminal behavior and abnormal incidents. However, detecting AI-generated videos necessitates addressing specific artifacts, including violations of natural physics and frame flickering. The methods are illustrated in Fig.~\ref{fig:MLLM-Video&Audio} (a). Chang et al.~\cite{chang2024matters} provide a comprehensive summary of the common defects observed in generated videos, offering valuable insights into this emerging challenge.

    \subsubsection{\textbf{Authenticity}}
    The detector of AI-generated video can be divided into two categories: Frame-Level detector and Video-Level detector. Frame-Level detector primarily focuses on studying forgery traces at the image level, while Video-Level detector focuses on detecting forged videos, such as through temporal and frequency domain analysis. Existing methods that use MLLMs as detectors are mostly frame-level detection approaches combined with a consistency detector.
    
    \begin{itemize}
    \item \textbf{Frame-Level detector}
    LOKI~\cite{ye2024loki} also shows the video modality result of judgment and multiple-choice tasks of LLMs, both accuracy respectively 71.3\% and 77.3\% by GPT-4o. 
    MM-Det~\cite{song2024learning} leverages MLLMs for frame-level forgery detection and to generate explainable text. It also uses Vector Quantised-Variational AutoEncoder (VQ-VAE) to reconstruct video content, by comparing the residuals between the reconstructed video and the original video to amplify diffusion forgery features. Finally, it introduces an innovative attention mechanism in the Transformer network to balance the detection of intra-frame and inter-frame forgery traces, integrating global and local features. VANE-Bench~\cite{bharadwaj2024vane} is a benchmark that uses MLLMs to detect AI-generated anomalies, including sudden appearance and disappearant objects, violating natural physics. 

    \item \textbf{Watermarking}
    Li et al.~\cite{li2024video} propose a multi-modal video watermarking approach. They embed imperceptible watermarks into strategically selected keyframes using a flow-based mechanism, ensuring minimal visual disruption. Additionally, the approach uses multiple loss functions to balance watermark robustness and video content integrity, effectively preventing unauthorized access by video-based LLMs.
    
    \end{itemize}

    \subsubsection{\textbf{Explainability}}
    Despite the growing interest in utilizing MLLMs for AI-generated video detection, current research has yet to address the explainability of these methods. Future work could focus on developing frameworks that integrate MLLMs with interpretable visual analysis techniques to provide clear and actionable explanations.
    \subsubsection{\textbf{Localization}}
    Similarly, the localization of manipulated regions in AI-generated videos using MLLMs remains an unexplored area. Research in this direction could explore the potential of MLLMs to combine temporal and spatial features for precise localization, which is particularly challenging in dynamic video content.


    \subsection{Audio}
    Currently, both open-source and proprietary MLLMs offering audio input support remain limited. Moreover, most existing models primarily emphasize audio content comprehension, with relatively little focus on analyzing acoustic characteristics. The methods are illustrated in Fig.~\ref{fig:MLLM-Video&Audio} (b).
    \subsubsection{\textbf{Authenticity}}
    
    \begin{itemize}
    \item \textbf{Prompt-engineering}
    LOKI~\cite{ye2024loki} selects open-source models supporting audio input, such as Qwen-Audio~\cite{chu2023qwen}, SALMONN-7B~\cite{sun2024video} and GPT-4o. For judgment tasks, the accuracy of SALMONN-7B is only 51.2\%. Additionally, some models lack support for multiple-choice tasks. Among those that do, the highest accuracy is achieved by AnyGPT, reaching 50.3\%. Research on distinguishing real and fake audio using MLLMs and acoustic cues remains limited. However, datasets such as those introduced by LOKI~\cite{ye2024loki} and SONICS~\cite{rahman2024sonics} focus on detecting fake voices or music. The field of AI-generated audio detection with Multimodal foundational models is still in its early stages.
    \end{itemize}
    
    \subsubsection{\textbf{Explainability}}
    To date, no research has explored the explainability of audio MLLM-based methods. This represents a significant gap, as explainability is crucial for understanding the decision-making process of these models, particularly in identifying subtle acoustic forgeries. Future studies could focus on developing frameworks that incorporate interpretable audio analysis techniques, thereby improving the transparency and trustworthiness of MLLM-based methods.

    \subsubsection{\textbf{Localization}}
    Currently, there is no published research addressing localization capabilities in audio MLLM-based methods. Localization is critical for pinpointing specific manipulated segments within audio signals, especially in cases of partial or layered forgeries. Further research could investigate how multimodal alignment or segment-wise attention mechanisms might enhance localization accuracy in MLLM-based frameworks.

\subsection{Multimodal}
Having explored text-guided detection methods for individual modalities such as text, image, video, and audio, we now turn our focus to multimodal collaboration. These methods leverage language to guide MLLMs in understanding and processing features from other modalities, demonstrating strong cross-modal adaptability. By integrating features from image, video, and audio modalities, we aim to explore how the intrinsic connections among multimodal content can further enhance the accuracy and robustness of AI-generated media detection.
\subsubsection{\textbf{Authenticity}}
\begin{itemize}
    \item \textbf{Text-Image}
    A key focus in this domain is evaluating image-text consistency and providing explanations for MLLM judgments. Out-of-context (OOC) media misuse involves cases where individuals are required to assess the accuracy of the accompanying statement and evaluate whether the image and caption correspond to the same event. This form of misuse, in which authentic images are paired with false text, represents one of the simplest yet most effective ways to mislead audiences. SNIFFER~\cite{qi2024sniffer} is an MLLM specifically designed for detecting and interpreting OOC misinformation, combining image-text consistency analysis, external knowledge retrieval, and fine-grained instruction tuning. \cite{wu2023cheap} integrates GPT-3.5 to enhance the contextual understanding capabilities of the traditional COSMOS model, leveraging IoU, Sentence BERT, and Prompt Engineering to fuse multimodal information effectively. Fka-owl~\cite{liu2024fka} through knowledge-augmented Large Vision-Language Models(LVLMs) to detect fake news.
    For \textbf{watermarking tasks}, text-image integration necessitates incorporating metadata from the text component and the generation context. Liu et al.~\cite{liu2023t2iw} propose the T2IW framework, which seamlessly embeds a binary watermark into generated images using a joint generation process that combines text encoding and noise. VLPMarker~\cite{tang2023watermarking}, a watermarking method based on backdoor injection, utilizes orthogonal transformation techniques to protect CLIP model copyrights while maintaining model efficiency and accuracy.
    \item \textbf{Visual-Audio}
    %用LLM的方法:与end-to-end的方法不同,因为LLMs能够识别跨模态哪或跨模态中可能存在的空间和时空伪造痕迹不一样
    ~\cite{shahzad2024good} integrates visual frames, audio speech, and text prompts into ChatGPT to generate outputs encompassing audiovisual analysis, interpretation, and authenticity prediction. Their approach involves designing various prompts, including binary classification prompts, probability prediction prompts, and tasks to identify synthetic artifacts. Unlike end-to-end learning-based methods, ChatGPT can effectively detect spatial and spatiotemporal artifacts and inconsistencies within or across modalities. For \textbf{watermarking tasks}, V²A-Mark~\cite{zhang2024v2a} embeds localization and copyright watermarks into video frames and audio samples, which employs a temporal alignment and fusion module and a degradation prompt learning mechanism for visual data, along with a sample-level versatile watermark for the audio. 
    
\end{itemize}

\section{Non-LLM-based Detector}
\label{sec:non-mllm}

In addition to methods that use MLLMs, there are various traditional techniques to detect AI-generated media. These approaches employ specialized algorithms and can be categorized into modalities such as text, image, audio, and video, based on the type of data processed.
\arrayrulecolor{black}
\begin{table*}[!t]
    \centering
        \renewcommand{\arraystretch}{1.3}
        \caption{Non-MLLM detectors for AI-generated media, spanning from unimodal to multimodal content. \textbf{Au} means Authenticity detection, \textbf{Ex} means Explainability, \textbf{Lo} means Localization.}
        \resizebox{\linewidth}{!}{
        \begin{tabular}{c|c|ccc|c|l}
\hline 
&  & \multicolumn{3}{c|}{\textbf{Task}}                                      &                                                                             \\ 
\cline{3-5} 
\multirow{-2}{*}{\textbf{Method}} & \multirow{-2}{*}{\textbf{Venue}} & \textbf{Au} & \textbf{Ex} &\textbf{Lo} & \multirow{-2}{*}{\textbf{Category}}  & \makecell[c]{\multirow{-2}{*}{\textbf{Highlight}}}                                    \\ \hline 
\rowcolor{lightorange}
\multicolumn{7}{c}{\textbf{Text}}\\ 
DeTeCtive~\cite{guo2024detective}                                             & \lightgraytext{{[}ArXiv'24{]}}                                            
& \CheckmarkBold      %Au           
& -      %Ex                
& -       %Lo                    
& Stylistic-based  
& Learn distinct writing styles\\
Shah et al.~\cite{shah2023detecting}                                             & \lightgraytext{{[}IJACSA'23{]}}                                            
& \CheckmarkBold      %Au           
& -      %Ex                
& -       %Lo                  
& Stylistic-based
& Discuss various factors that need to be considered while detecting AI-generated text                              \\
Kumarage et al.~\cite{kumarage2023stylometric}                            & \lightgraytext{{[}Arxiv'23{]}}                                   
& \CheckmarkBold      %Au           
& -      %Ex                
& -       %Lo                  
& Stylistic-based            
& Use stylometric signals                                   \\
Hamed et al.~\cite{hamed2023improving}                            & \lightgraytext{{[}Preprint'23{]}}                                         
& \CheckmarkBold      %Au           
& -      %Ex                
& -       %Lo                  
& Linguistics-based   
& Extract the TF-IDF bigrams to train supervised Machine Learning algorithm           \\
Gallé et al.~\cite{galle2021unsupervised}                            & \lightgraytext{{[}Arxiv'21{]}}                                         
& \CheckmarkBold      %Au           
& -      %Ex                
& -       %Lo                  
& Linguistics-based             
& Leveraging repeated higher-order n-grams as detection signal           \\
Yoo et al.~\cite{yoo2023robust}                            & \lightgraytext{{[}Arxiv'23{]}}                                          
& \CheckmarkBold      %Au           
& -      %Ex                
& -       %Lo                  
& Watermarking               
& Use invariant features of natural language to embed robust watermarks to corruptions        \\
DeepTextMark~\cite{munyer2024deeptextmark}                            & \lightgraytext{{[}IEEE'24{]}}                                         
& \CheckmarkBold      %Au           
& -      %Ex                
& -       %Lo                  
& Watermarking         
& Use Word2Vec, Sentence Encoding, and transformer-based classifier for watermark insertion and detection          \\
Yang et al.~\cite{yang2023watermarking}                            & \lightgraytext{{[}Arxiv'23{]}}                                      
& \CheckmarkBold      %Au           
& -      %Ex                
& -       %Lo                  
& Watermarking                
& Inject watermarks by replacing synonyms with different hash values.      \\
AWT~\cite{abdelnabi2021adversarial}                            & \lightgraytext{{[}IEEE'21{]}}                                         
& \CheckmarkBold      %Au           
& -      %Ex                
& -       %Lo                  
& Watermarking                     
&  Learn word substitutions along with their locations to hide watermarks          \\
REMARK-LLM~\cite{zhang2024remark}                            & \lightgraytext{{[}USENIX'24{]}}                              
& \CheckmarkBold      %Au           
& -      %Ex                
& -       %Lo                  
& Watermarking       
&  Insert watermarks into LLM-generated texts without compromising the semantic integrity          \\
\iffalse
GPTZero~\cite{gptzero}                            & \lightgraytext{{[}-{]}}                                           & Text           & Ex          & -    
& \CheckmarkBold      %txt           
& -      %img                
& -       %vid             
& -       %aud                    
& (找不到论文)            \\
\fi
Mitrovic et al.~\cite{mitrovic2023chatgpt}                            & \lightgraytext{{[}Arxiv'23{]}}                                         
& -      %Au           
& \CheckmarkBold      %Ex                
& -       %Lo                  
& -                
&  Apply Shapley Additive Explanations to uncover the detection model's reasoning         \\
Ji et al.~\cite{ji2024detecting}                            & \lightgraytext{{[}Arxiv'24{]}}                                          
& -      %Au           
& \CheckmarkBold      %Ex                
& -       %Lo                  
& -     
&  Introduce novel ternary text classification scheme to enhance explainability          \\
Zhang et al.~\cite{zhang2024machine}                            & \lightgraytext{{[}Arxiv'24{]}}                                       
& -      %Au           
& -      %Ex                
& \CheckmarkBold       %Lo                  
& -                      
&  Provide additional context by including multiple sentences at once but predict each one individually        \\
MFD~\cite{tao2024unveiling}                            & \lightgraytext{{[}Arxiv'24{]}}                                       
& -      %Au           
& -      %Ex                
& \CheckmarkBold       %Lo                  
& -               
&  Integrate low-level structural, high-level semantic, and deep-level linguistic features          \\
\rowcolor{lightorange}
\multicolumn{7}{c}{\textbf{Image}}\\ 
FHAD~\cite{wang2024generated}                            & \lightgraytext{{[}Arxiv'24{]}}                                           
& \CheckmarkBold       %Au           
& -      %Ex                
& -      %Lo                  
& High-Level                  
&  Use correlation of body parts to detect absent abnormalities     \\
Farid~\cite{farid2022lighting}                            & \lightgraytext{{[}Arxiv'22{]}}                                 
& \CheckmarkBold       %Au           
& -      %Ex                
& -      %Lo                  
& High-Level              
& Explore if physics-based forensic analyses will prove fruitful in detecting synthetic media           \\
Sarkar et al.~\cite{sarkar2024shadows}                            & \lightgraytext{{[}CVPR'24{]}}                 
& \CheckmarkBold       %Au           
& -      %Ex                
& -      %Lo                  
& High-Level              
& Use geometric properties         \\
AIDE~\cite{yan2024sanity}                            & \lightgraytext{{[}Arxiv'24{]}}                                        
& \CheckmarkBold       %Au           
& -      %Ex                
& -      %Lo                  
& High-Level                   
& Use multiple experts to simultaneously extract visual artifacts and noise patterns           \\
\iffalse
PatchCraft~\cite{zhong2024patchcraft}                            & \lightgraytext{{[}Arxiv'24{]}}                           
& Image           & Au          & Low-Level       
& -       %txt           
& \CheckmarkBold       %img                
& -       %vid             
& -       %aud       
& 论文只有引用           \\
\fi
LGrad~\cite{tan2023learning}                            & \lightgraytext{{[}CVPR'23{]}}                                       
& \CheckmarkBold       %Au           
& -      %Ex                
& -      %Lo                  
& Low-Level                
& Use gradients as the representation of artifacts in GAN-generated images           \\
AUSOME~\cite{poredi2023ausome}                            & \lightgraytext{{[}SPIE'23{]}}                                      
& \CheckmarkBold       %Au           
& -      %Ex                
& -      %Lo                  
& Low-Level  
&   Use spectral analysis and machine learning        \\
Wolter et al.~\cite{wolter2022wavelet}                            & \lightgraytext{{[}ML'22{]}}                    
& \CheckmarkBold       %Au           
& -      %Ex                
& -      %Lo                  
& Low-Level 
&  Use wavelet-packet-based analysis and boundary wavelets       \\
Synthbuster~\cite{bammey2023synthbuster}                            & \lightgraytext{{[}IEEE'23{]}}                            
& \CheckmarkBold       %Au           
& -      %Ex                
& -      %Lo                  
& Low-Level 
&  Use spectral analysis to highlight the artifacts in the Fourier transform of a residual image        \\
Frank et al.~\cite{frank2020leveraging}                            & \lightgraytext{{[}ICML'20{]}}                             
& \CheckmarkBold       %Au           
& -      %Ex                
& -      %Lo                  
& Low-Level   
&  Employ frequency representations for detecting         \\
Corvi et al.~\cite{corvi2023intriguing}                            & \lightgraytext{{[}CVPR'23{]}}                   
& \CheckmarkBold       %Au           
& -      %Ex                
& -      %Lo                  
& Low-Level   
&    Consider second-order statistics both in the spatial domain and in the frequency domains         \\
SeDID~\cite{ma2023exposing}                            & \lightgraytext{{[}Arxiv'23{]}}                            
& \CheckmarkBold       %Au           
& -      %Ex                
& -      %Lo                  
& Low-Level      
&    Exploit diffusion models' deterministic reverse and deterministic to denoise computation errors         \\
E3~\cite{azizpour2024e3}                            & \lightgraytext{{[}CVPR'24{]}}                            
& \CheckmarkBold       %Au           
& -      %Ex                
& -      %Lo                  
& Low-Level    
&   Create a set of expert embedders to accurately capture traces from each new target generator       \\
DIRE~\cite{wang2023dire}                            & \lightgraytext{{[}ICCV'23{]}}                           
& \CheckmarkBold       %Au           
& -      %Ex                
& -      %Lo                  
& Reconstruction Error    
&   Measure error between the input image and its reconstruction counterpart by pre-trained diffusion model         \\
AEROBLADE~\cite{ricker2024aeroblade}                            & \lightgraytext{{[}CVPR'24{]}}            
& \CheckmarkBold       %Au           
& -      %Ex                
& -      %Lo                  
& Reconstruction Error         
&   Compute images' AE reconstruction error         \\
FIRE~\cite{chu2024fire}                            & \lightgraytext{{[}Arxiv'24{]}}                       
& \CheckmarkBold       %Au           
& -      %Ex                
& -      %Lo                  
& Reconstruction Error        
&   Investigate the influence of frequency decomposition on reconstruction error         \\
DRCT~\cite{chendrct}                            & \lightgraytext{{[}ICML'24{]}}                           
& \CheckmarkBold       %Au           
& -      %Ex                
& -      %Lo                  
& Reconstruction Error    
&   Generate hard samples and adopt contrastive training to guide the learning of diffusion artifacts         \\
SemGIR~\cite{yu2024semgir}                            & \lightgraytext{{[}MM'24{]}}                      
& \CheckmarkBold       %Au           
& -      %Ex                
& -      %Lo                  
& Reconstruction Error    
&   Compel detector to focus on the inherent characteristic of the model expressed within them         \\
EditGuard~\cite{zhang2024editguard}                            & \lightgraytext{{[}CVPR'24{]}}                           
& \CheckmarkBold       %Au           
& -      %Ex                
& -      %Lo                  
& Watermarking         
& Train united Image-Bit Steganography Network to embed dual invisible watermarks into original images      \\
DiffusionShield~\cite{cui2023diffusionshield}                            & \lightgraytext{{[}Arxiv'23{]}}             
& \CheckmarkBold       %Au           
& -      %Ex                
& -      %Lo                  
& Watermarking  
&   Protect images from infringement by encoding the ownership message into an imperceptible watermark        \\
ZoDiac~\cite{zhang2024robust}                            & \lightgraytext{{[}Arxiv'24{]}}           
& \CheckmarkBold       %Au           
& -      %Ex                
& -      %Lo                  
& Watermarking    
&  Inject watermarks into trainable latent space for protection  \\
LaWa~\cite{rezaei2024lawa}                            & \lightgraytext{{[}Arxiv'24{]}}                  
& \CheckmarkBold       %Au           
& -      %Ex                
& -      %Lo                  
& Watermarking  
&  Change latent feature of pre-trained LDMs to integrate watermarking into the generation process  \\
WMAdapter~\cite{ci2024wmadapter}                            & \lightgraytext{{[}Arxiv'24{]}}               
& \CheckmarkBold       %Au           
& -      %Ex                
& -      %Lo                  
& Watermarking    
&  Use pretrained watermark decoder and minimal training pipeline to design a lightweight structure \\
Cifake~\cite{bird2024cifake}                            & \lightgraytext{{[}IEEE'24{]}}                           
& -       %Au           
& \CheckmarkBold      %Ex                
& -      %Lo                  
& -
&   Benchmarks of mirroring ten classes of the already available CIFAR-10 dataset with latent diffusion     \\
ASAP~\cite{huang2024asap}                            & \lightgraytext{{[}Arxiv'24{]}}              
& -       %Au           
& \CheckmarkBold      %Ex                
& -      %Lo                  
& -  
&  Extract distinct patterns and allow users to interactively explore them using various views.          \\
DA-HFNet~\cite{liu2024hfnet}                            & \lightgraytext{{[}Arxiv'24{]}}                    
& -       %Au           
& -      %Ex                
& \CheckmarkBold      %Lo                  
& -
&   Use dual-attention mechanism for deeper feature fusion and multi-scale feature interaction       \\
DiffForensics~\cite{yu2024diffforensics}                            & \lightgraytext{{[}CVPR'24{]}}                
& -       %Au           
& -      %Ex                
& \CheckmarkBold      %Lo                  
& -      
&   Propose a two-stage learning framework for IFDL tasks combining macro-features and micro-features        \\
MoNFAP~\cite{miao2024mixture}                            & \lightgraytext{{[}Arxiv'24{]}}            
& -       %Au           
& -      %Ex                
& \CheckmarkBold      %Lo                  
& -
&    Integrate detection and localization processing into a single predictor for face manipulation localization        \\
HiFi-Net++~\cite{guo2024language}                            & \lightgraytext{{[}IJCV'24{]}}                    
& -       %Au           
& -      %Ex                
& \CheckmarkBold      %Lo                  
& -    
&   Use additional language-guided forgery localization enhancer       \\
SAFIRE~\cite{kwon2024safire}                            & \lightgraytext{{[}Arxiv'24{]}}                             
& -       %Au           
& -      %Ex                
& \CheckmarkBold      %Lo                  
& -    
&   Capitalize on SAM’s point prompting capability to distinguish each source when an image has been forged        \\
\rowcolor{lightorange}
\multicolumn{7}{c}{\textbf{Video}}\\ 
Bohacek et al.~\cite{bohacek2024human}                            & \lightgraytext{{[}Arxiv'24{]}}                     
& \CheckmarkBold       %Au           
& -      %Ex                
& -      %Lo                  
& Frame-Level  
&   Leverage multi-modal semantic embedding to make it robust to the types of laundering      \\
AIGVDet~\cite{bai2024ai}                            & \lightgraytext{{[}Arxiv'24{]}}                      
& \CheckmarkBold       %Au           
& -      %Ex                
& -      %Lo                  
& Frame-Level  
&   Capture the forensic traces with a two-branch spatio-temporal convolutional neural network      \\
DIVID~\cite{liu2024turns}                            & \lightgraytext{{[}Arxiv'24{]}}                         
& \CheckmarkBold       %Au           
& -      %Ex                
& -      %Lo                  
& Video-Level    
&   Use CNN and LSTM to capture different levels of abstraction features and temporal dependencies     \\
He et al.~\cite{he2024exposing}                            & \lightgraytext{{[}Arxiv'24{]}}                        
& \CheckmarkBold       %Au           
& -      %Ex                
& -      %Lo                  
& Video-Level      
&   Design channel attention-based feature fusion by combining local and global temporal clues adaptively  \\
Yan et al.~\cite{yan2024generalizing}                            & \lightgraytext{{[}Arxiv'24{]}}               
& \CheckmarkBold       %Au           
& -      %Ex                
& -      %Lo                  
& Video-Level   
&    Blend original image and its warped version frame-by-frame to implement Facial Feature Drift   \\
DuB3D~\cite{ji2024distinguish}                            & \lightgraytext{{[}Arxiv'24{]}}                    
& \CheckmarkBold       %Au           
& -      %Ex                
& -      %Lo                  
& Video-Level  
&    Use a dual-branch architecture that adaptively leverages and fuses raw spatio-temporal data and optical flows      \\
Demamba~\cite{chen2024demamba}                            & \lightgraytext{{[}Arxiv'24{]}}                             
& \CheckmarkBold       %Au           
& -      %Ex                
& -      %Lo                  
& Video-Level   
&    Leverage a structured state space model to capture spatial-temporal inconsistencies across different regions      \\
Vahdati et al.~\cite{vahdati2024beyond}                            & \lightgraytext{{[}CVPR'24{]}}            
& \CheckmarkBold       %Au           
& -      %Ex                
& -      %Lo                  
& Video-Level    
&    Use synthetic video traces to perform reliable synthetic video detection or generator source attribution     \\
DVMark~\cite{luo2023dvmark}                            & \lightgraytext{{[}IEEE'23{]}}               
& \CheckmarkBold       %Au           
& -      %Ex                
& -      %Lo                  
& Watermarking
&   Use multi-scale design to make watermarks distributed across multiple spatial-temporal scales  \\
REVMark~\cite{zhang2023novel}                            & \lightgraytext{{[}MM'23{]}}              
& \CheckmarkBold       %Au           
& -      %Ex                
& -      %Lo                  
& Watermarking 
&  Use encoder/decoder structure with pre-processing block to extract temporal-associated features on aligned frames  \\
\rowcolor{lightorange}
\multicolumn{7}{c}{\textbf{Audio}}\\ 
Salvi et al.~\cite{salvi2024listening}                            & \lightgraytext{{[}Arxiv'24{]}}          
& \CheckmarkBold       %Au           
& -      %Ex                
& -      %Lo                  
& Fingerprint       
&    Indicate that analyzing the background noise alone leads to better classification results across diverse scenarios   \\
DeAR~\cite{liu2023dear}                            & \lightgraytext{{[}AAAI'23{]}}                                 
& \CheckmarkBold       %Au           
& -      %Ex                
& -      %Lo                  
& Watermarking   
&    Resist AR distortion at different distances in the real world   \\
AudioSeal~\cite{roman2024proactive}                            & \lightgraytext{{[}ICML'24{]}}                   
& \CheckmarkBold       %Au           
& -      %Ex                
& -      %Lo                  
& Watermarking       
&    Jointly train generator and detector for localized speech watermarking \\
Wu et al.~\cite{wu2023adversarial}                            & \lightgraytext{{[}ICME'23{]}}                          
& \CheckmarkBold       %Au           
& -      %Ex                
& -      %Lo                  
& Watermarking      
&    Embed a watermark into a feature domain mapped by a deep neural network   \\
SLIM~\cite{zhu2024slim}                            & \lightgraytext{{[}Arxiv'24{]}}                         
& -       %Au           
& \CheckmarkBold      %Ex                
& -      %Lo                  
& -   
&    Use style-linguistics mismatch in fake speech to separate style and linguistics contents from real speech   \\
SFAT-Net-3~\cite{cuccovillo2024audio}                            & \lightgraytext{{[}CVPR'24{]}}        
& -       %Au           
& \CheckmarkBold      %Ex                
& -      %Lo                  
& -   
&    Encode magnitude and phase of input speech to predict the trajectory of first phonetic formants\\
Pascu et al.~\cite{pascu2024easy}                            & \lightgraytext{{[}Arxiv'24{]}}                   
& -       %Au           
& \CheckmarkBold      %Ex                
& -      %Lo                  
& -         
&    Demonstrate that attacks can be identified with surprising accuracy using small subset of simplistic features  \\
HarmoNet~\cite{liu2024harmonet}                            & \lightgraytext{{[}ISCA'24{]}}                                   
& -       %Au           
& -      %Ex                
& \CheckmarkBold      %Lo                  
& -      
&  Use latent representations extraction capability of SSL along with harmonic F0 characteristic of speech\\
CFPRF~\cite{wu2024coarse}                            & \lightgraytext{{[}MM'24{]}}            
& -       %Au           
& -      %Ex                
& \CheckmarkBold      %Lo                  
& -      
&  Mine temporal inconsistency cues\\ %by perceiving subtle differences between different frames and capturing contextual information of multiple transition boundaries\\
\rowcolor{lightorange}
\multicolumn{7}{c}{\textbf{Multimodal}}\\ 
HAMMER~\cite{Shao2023CVPR}                            & \lightgraytext{{[}CVPR'23{]}}                     
& \CheckmarkBold        %Au           
& -      %Ex                
& -     %Lo                  
& Text-Image       
&    Capture interaction of image-texts based on embeddings alignment and multi-modal embedding aggregation\\
Li et al.~\cite{li2024zero}                            & \lightgraytext{{[}Arxiv'24{]}}                           
& \CheckmarkBold        %Au           
& -      %Ex                
& -     %Lo                  
& Visual-Audio    
&    Employ pre-trained ASR and VSR models to edit distance between audio and video sequences\\
Yoon et al.~\cite{yoon2024triple}                            & \lightgraytext{{[}IF'24{]}}            
& \CheckmarkBold        %Au           
& -      %Ex                
& -     %Lo                  
& Visual-Audio        
&    Propose a baseline approach based on zero-shot identity and one-shot deepfake detection with limited data\\
DiMoDif~\cite{koutlis2024dimodif}                            & \lightgraytext{{[}Arxiv'24{]}}                 
& -        %Au           
& -      %Ex                
& \CheckmarkBold     %Lo                  
& -
&    Exploit inter-modality differences in machine perception of speech \\
MMMS-BA~\cite{katamneni2024contextual}                            & \lightgraytext{{[}IJCB'24{]}}     
& -        %Au           
& -      %Ex                
& \CheckmarkBold     %Lo                  
& -   
&   Leverage attention from neighboring sequences and multi-modal representations \\ \hline
\end{tabular}
        }
    \label{table:non-mllm-detector}
\end{table*}

% \\
% ImgTrojan~\cite{tao2024imgtrojan}                                                 & \lightgraytext{{[}arXiv'24{]}}                                           & White           & Generator      & Tuning-based              & \cellcolor{LightRed}I + T → T           & Inject poisoned image-text pairs into the training of VLMs as triggers.                             

\subsection{Text}
\subsubsection{\textbf{Authenticity}}
Text content detection methods primarily fall into three categories: stylistic-based, linguistics features-based methods, and watermarking. These approaches determine whether a text is AI-generated by analyzing stylistic features, linguistic structures, and watermarking respectively.
\begin{itemize}
    \item \textbf{Stylistic-based} Unlike traditional binary classification problems, stylistic-based methods focus on distinguishing the writing styles of different authors. Each AI model has its unique writing style, and identifying these distinct styles proves to be more effective than a simple binary classification task.
    DeTeCtive~\cite{guo2024detective} is a multi-task, multi-level contrastive learning framework that demonstrates superior performance in detecting AI-generated text across in-distribution and out-of-distribution scenarios. It also introduces a novel feature, Training-Free Incremental Adaptation, which enables adaptation to new data without retraining.
    Shah et al.~\cite{shah2023detecting} propose a novel approach combining features like vocabulary diversity, readability metrics, and semantic distribution with machine learning models for classification. Kumarage et al.~\cite{kumarage2023stylometric} leverage stylometric features with a PLM embedding to enhance the detection of AI-generated text.
    
    \item \textbf{Linguistics-based}
    Hamed et al.~\cite{hamed2023improving} employ an unsupervised approach using repetition patterns of higher-order n-grams as textual characteristics, achieving notable results. Gallé et al.~\cite{galle2021unsupervised} innovatively utilize bigram networks from authentic scientific articles as a benchmark for comparison with ChatGPT-generated content, attaining high accuracy. Both methods cleverly account for the relationships between words.
    
    \item \textbf{Watermarking}
    To watermark existing text, some researchers~\cite{yoo2023robust}~\cite{munyer2024deeptextmark}~\cite{yang2023watermarking} use synonym replacement or syntactic transformations while maintaining overall meaning. However, these methods often rely on specific rules that can lead to unnatural modifications, degrading text quality and making it easier for attackers to detect. To overcome these issues, AWT~\cite{abdelnabi2021adversarial} employs a transformer encoder to encode sentences and merge them with message embeddings, which are then processed by a transformer decoder to generate watermarked text. Detection involves analyzing the watermarked text via transformer encoder layers to extract hidden messages. Then, REMARK-LLM~\cite{zhang2024remark} utilizes a pretrained LLM for watermark insertion and includes a reparameterization step to create sparser token distributions, enabling it to embed twice as many signatures as AWT while still ensuring effective detection, thereby enhancing watermark payload capacity.
    
\end{itemize}

\subsubsection{\textbf{Explainability}}
%当前解释模块对普通用户(lay users)的可理解性仍然较差。现有系统往往难以用直观方式解释复杂的检测逻辑
GPTZero~\cite{gptzero} is an online closed-source detector, which relies on six features for explainability: readability, percent SAT,
simplicity, perplexity, burstiness, and average sentence length. However, it does not provide clarity
on how these features influence its final judgments. Mitrovic et al.~\cite{mitrovic2023chatgpt} use implemented Shapley Additive Explanations to reveal how features of ChatGPT-generated text (such as formality, politeness, and impersonality) influence the classification decisions of detection models.
Ji et al.~\cite{ji2024detecting} introduce a ternary classification framework consisting of human-writing text (HWT), MGT, and an ``undecided" category. Human annotators relabel the text with the newly added ``uncertain" category and provide explanations for their decisions. Current explanation modules still fail to provide intuitive understandability for non-expert users. Existing systems often struggle to intuitively explain the complex detection logic.

\subsubsection{\textbf{Localization}} Zhang et al.~\cite{zhang2024machine} leverage contextual information to analyze multiple sentences simultaneously, and divide the text into chunks and extracting features using fixed-parameter detection models, avoiding additional training. MFD~\cite{tao2024unveiling} framework identifies specific paragraphs or sentences generated by LLMs by combining low-level structural features, high-level semantic features, and deep linguistic features. It enhances robustness through contrastive learning.

\subsection{Image}
\subsubsection{\textbf{Authenticity}}

\begin{figure}[!ht]
  \centering
    \includegraphics[width=1.0\linewidth]{Fig/Non-MLLM-Image.pdf}
    \caption{Illustrating of Non-MLLM-based authenticity detection methodologies for AI-generated images. The methods are categorized into: (a)~\textit{Low-level} (b)~\textit{High-level} (c)~\textit{Reconstruction error} (d)~\textit{Watermarking}, (d) is reproduced from~\cite{luo2025digital}}
    \label{fig:Non-MLLM-Image}
\end{figure}

Image detection methods can be broadly categorized into four types: high-level, low-level approaches, reconstruction error-based methods, and watermarking methods. High-level methods analyze geometric information, such as abnormal lighting, shadows, and reflections. They also examine human anatomy, including pupil reflections and body abnormalities in images. In contrast, low-level~\cite{yang2021mtd} feature methods rely on spatial and frequency domain analysis, as well as identifying artificial fingerprints. Reconstruction error-based methods utilize the reconstruction capabilities of diffusion models, identifying anomalies by comparing differences between the original and reconstructed images.
Watermarking methods involve embedding watermarks either before or after image generation, enabling the detection of AI-generated images through dedicated watermark detectors. The methods are illustrated in Fig.~\ref{fig:Non-MLLM-Image}.
~\begin{itemize}
    \item ~\textbf{High-Level}
   High-level methods primarily analyze \textbf{geometric} information, such as abnormal lighting, shadows, and reflections, as well as \textbf{human anatomy}, including pupil shape reflection and abnormalities in the human body within images. FHAD~\cite{wang2024generated} detects fine-grained human body abnormalities and proposes solutions for missing or redundant body parts through reconstruction. Fraid~\cite{farid2022lighting,farid2022perspective} examines the geometric consistency of vanishing points, shadows, and reflections in generated images, as well as lighting consistency, using these inconsistencies for detection. Sarkar et al.~\cite{sarkar2024shadows} propose three classifiers based on object-shadow relationships, perspective fields, and line segment analysis, achieving good results. AIDE~\cite{yan2024sanity} employs a mixture of expert approach, combining low-level pixel statistics with high-level semantic features, effectively identifying various AI-generated images.
    
    \item ~\textbf{Low-Level}
    Low-level methods primarily focus on spatial and frequency domain information. In the \textbf{spatial} domain, PatchCraft~\cite{zhong2024patchcraft} enhances texture features through image scrambling and reconstruction, examining pixel correlations for detection with robustness to perturbations. LGrad~\cite{tan2023learning} utilizes CNNs to convert images into gradient representations, performing well in cross-model and cross-category tests.
    For \textbf{frequency} domain analysis, AUSOME~\cite{poredi2023ausome} employs discrete Fourier and cosine transforms to analyze diffusion model-generated images, identifying specific patterns in DALL-E 2 outputs. Wolter et al.~\cite{wolter2022wavelet} propose a wavelet packet-based multi-scale time-frequency analysis method, preserving spatial and frequency information. Synthbuster~\cite{bammey2023synthbuster} leverages frequency artifacts in diffusion model-generated images for detection. Frank et al.~\cite{frank2020leveraging} analyze artificial traces in GAN-generated images using discrete cosine transforms.
    Researchers have also examined \textbf{artificial fingerprints} in images. Corvi et al.~\cite{corvi2023intriguing} discover that various generators leave specific traces in images. SeDID~\cite{ma2023exposing} cleverly utilizes the deterministic reverse process of diffusion models, introducing the concept (\textit{e.g., time step, stride error}) to distinguish between real and synthetic images by analyzing error patterns at specific timesteps. Moreover, the E3~\cite{azizpour2024e3} framework uses transfer learning to create specialized expert embedders for different synthetic image generators, allowing accurate detection with minimal data. It combines embeddings from multiple experts through an Expert Knowledge Fusion Network to enhance detection performance, particularly for newly emerged generators. 
    
    \item ~\textbf{Reconstruction Error} 
    With the reconstruction capability of Diffusion models, researchers identify abnormal regions by comparing the differences between the original and reconstructed images. DIRE~\cite{wang2023dire} was the first detector proposed for diffusion-generated images. AEROBLADE~\cite{ricker2024aeroblade} utilizes autoencoder reconstruction errors from LDMs in a train-free method. FIRE~\cite{chu2024fire} detects diffusion-generated images by analyzing frequency-based reconstruction errors. DRCT~\cite{chendrct} builds on the aforementioned observation and employs contrastive learning to improve generalization by generating hard samples during the reconstruction process. In addition, SemGIR~\cite{yu2024semgir} utilizes an image-to-text approach followed by text-to-image regeneration, calculating the similarity between the original and re-generated images to distinguish AI-generated images.

    \item ~\textbf{Watermarking}
    EditGuard~\cite{zhang2024editguard} embeds dual invisible watermarks in images to achieve copyright protection and tamper localization. This method trains a unified Image-Bit Steganography Network (IBSN), which decouples the training process from specific tampering types, enhancing the model's generalizability and allowing it to operate effectively without labeled data for particular tampering scenarios.
    %watermarking for generative models
    Additionally, watermarks can be integrated into diffusion models. The watermarks embedded in generative models are static, meaning that they do not adjust based on changes in the generated content. 
    %during静态水印
    DiffusionShield~\cite{cui2023diffusionshield} generates watermarks in generative diffusion models (GDMs) using a blockwise strategy that segments the watermark into basic patches. Each user has a unique sequence of patches that encodes copyright information across their images. The method also utilizes joint optimization to improve efficiency and accuracy, allowing for the easy addition of new users without retraining.
    %latent diffusion models 水印
    Moreover, Latent Diffusion Models (LDMs) generate the image in the latent space of a pre-trained autoencoder. We argue that this latent space can be used to integrate watermarking into the generation process.
    ZoDiac~\cite{zhang2024robust} injects watermarks into the latent space of stable diffusion models during noise sampling, enhancing the invisibility and robustness of the watermarked images. LaWa~\cite{rezaei2024lawa} modifies latent features of pre-trained LDM to embed watermarks during image generation
    %插件动态水印
    However, some researchers have found ways to design watermarks that can be dynamically adjusted according to the context. WMAdapter~\cite{ci2024wmadapter} is a plugin that seamlessly integrates watermarking into the diffusion models in the diffusion process, enabling dynamic watermarking without the need for individual fine-tuning for each watermark. 

\end{itemize}

Moreover, a recent study~\cite{tan2024c2p} has found CLIP model does not truly understand the concepts of ``real" and ``forged". Instead, it detects deepfake content by identifying similar concepts or features. Therefore, C2P-CLIP~\cite{tan2024c2p} integrates category-related concepts (\textit{e.g., DeepFake, Camera}) into CLIP's image encoder through a text encoder, through the use of image-text contrastive learning techniques. Also, some researchers~\cite{kim2024correlation, song2024quality} have found that existing methods typically train detection models by mixing deepfake data with varying levels of forgery quality. These approaches may cause the model to overly rely on easily identifiable forgery traces in low-quality samples, which can negatively affect its generalization ability. To address this, FreDA~\cite{song2024quality} proposes improving the facial structure of low-quality samples by combining the low-frequency features of real images with the high-frequency features of forged images, thereby enhancing their realism.

\subsubsection{\textbf{Explainability}}
For Non-MLLM methods, explainability tends to focus more on interpretability, which involves explaining the internal decision-making mechanisms of the model, rather than producing human-understandable explanatory content.
Cifake~\cite{bird2024cifake} employs Gradient Class Activation Mapping (Grad-CAM) technology, revealing that the model primarily relies on subtle visual defects in the image background, rather than the features of the objects themselves, to differentiate between real and synthetic images. ASAP~\cite{huang2024asap} uses gradient-based methods to identify pixel groups that have the greatest impact on classification results, revealing key falsified patterns in AI-generated images.

\subsubsection{\textbf{Localization}}
The main methods for localizing AI-generated forgery regions extract diverse features and employ various feature fusion modules to improve detection accuracy. They also utilize different strategies to enhance tampered edge traces, enabling high-precision localization of forgery regions.
DA-HFNet~\cite{liu2024hfnet} extracts RGB features, noise fingerprint features, and frequency domain features. It employs a dual-attention fusion mechanism for multimodal features and a multi-scale feature interaction strategy, along with edge loss optimization, to accurately localize forged regions. DiffForensics~\cite{yu2024diffforensics} trains a module that can simultaneously extract both high-level and low-level features and proposes an Edge Cue Enhancement Module to strengthen the edge features of the tampered region. MoNFAP~\cite{miao2024mixture} framework integrates both detection and localization tasks while incorporating various noise features to enhance the clues for forgery detection.
Also, HiFi-Net++~\cite{guo2024language} categorizes forgery attributes into multiple levels, such as fully synthetic, diffusion models, conditional generation, etc. It employs multi-level classification learning to comprehensively represent forgery features. By capturing the contextual dependencies between forgery attributes through hierarchical relationships, the method outputs both forgery detection and localization results.
SAFIRE~\cite{kwon2024safire} addresses the image forgery localization problem from a more fundamental perspective. The approach divides an image into different source regions based on its origin. Each source region represents an independent part of the image, which may be captured, AI-generated, or tampered with through other means. SAFIRE uses a point-based hint mechanism, where a point in the image is utilized to segment the source region that contains it, thereby enabling the division of the image into distinct source regions.


\subsection{Video}
\subsubsection{\textbf{Authenticity}}
~\cite{chang2024matters} identifies three main issues in AI-generated videos: appearance, motion, and geometry. Appearance refers to the inconsistency in color and texture, often resulting in distortions, especially during transitions between video frames. Motion indicates that the motion trajectories of objects may not comply with physical laws. Geometry highlights that objects in generated videos frequently violate real-world geometric rules, such as spatial proportions, scale, and occlusion order. We observe that methods for detecting AI-generated videos can be categorized into two types: \textbf{Frame-level}, and \textbf{Video-level} approaches. Each of these methods is suited to different detection scenarios and requirements, enabling effective identification across various video authentication tasks.

\begin{itemize}
    \item \textbf{Frame-Level} 
    Similar to the classification approach used in MLLM detectors, frame-level detection primarily focuses on identifying forgery traces by extracting individual video frames. Bohacek~\cite{bohacek2024human} detects AI-generated human motion in videos by utilizing multi-modal embeddings, including CLIP-based models, to map the visual information of video frames to their corresponding textual descriptions within the same semantic space. Each frame is first classified as real or fake using an SVM. Then, the authenticity of the entire video is determined based on the majority of the frame predictions. AIGVDet~\cite{bai2024ai} extracts features and performs classification on the spatial and optical flow of each frame. The results from each frame are combined through a decision fusion module to determine whether the video is AI-generated.

    \item \textbf{Video-Level}
    In video-level analysis, the focus is on the unique characteristics of videos, such as temporal and spatial features. 
    For \textbf{temporal-based} methods, DIVID~\cite{liu2024turns} combines CNN and LSTM architectures to capture both spatial and temporal features by leveraging DIRE ~\cite{wang2023dire} values. This approach improves accuracy by incorporating explicit knowledge from reconstructed frames and temporal dependencies, thereby enhancing the detector's generalizability on OOD video datasets. In addition, He et al.~\cite{he2024exposing} find that temporal dependencies in real and generated videos differ significantly: Real videos are captured by camera devices, with very short time intervals between frames, resulting in high temporal redundancy. In contrast, AI video generation models generate videos by controlling the temporal continuity between frames in latent space. To address this, they leverage local motion information and global appearance variations through representation learning. The model combines these features using a channel attention mechanism for effective feature fusion.
    However, other approaches focus on the \textbf{spatial-temporal consistency}. 
    Yan et al.~\cite{yan2024generalizing} propose a Video-level Blending method to simulate inconsistencies in facial features across consecutive frames in deepfake videos. Additionally, they introduce a lightweight Spatio-temporal Adapter, a plugin that enhances CNN or ViT architectures to simultaneously capture both spatial and temporal features.
    DuB3D~\cite{ji2024distinguish} adopts a dual-branch architecture, with one branch processing the raw spatio-temporal data and the other handling optical flow data.
    Demamba~\cite{chen2024demamba} is a plug-and-play detector, which processes the spatial and temporal dimensions of features, modeling the spatio-temporal consistency between features through grouping and scanning. By aggregating global and local features, it utilizes an MLP to classify the video, outputting the probability of whether the video is real or fake.
    %video-fingerprint
    Moreover, generated videos leave distinct traces, similar to image \textbf{fingerprints}, which can be learned and detected after performing a Fourier transform. Vahdati et al.~\cite{vahdati2024beyond} find video generators leave different traces than image generators, combining frame and video-level analysis for classifier training.

    \item \textbf{Watermarking}
    Similar to image watermarking, video watermarking can be implemented frame by frame using image watermarking techniques. Additionally, it is crucial to consider temporal correlations and the robustness of the watermark in video watermarking. DVMark~\cite{luo2023dvmark} uses an end-to-end trainable multi-scale network for robust watermark embedding and extraction across various spatial-temporal clues. REVMark~\cite{zhang2023novel} focuses on improving the robustness against H.264/AVC compression via the temporal alignment module and DiffH264 distortion layer.
    \end{itemize}

\subsubsection{\textbf{Explainability}}
At present, there is no existing research that specifically explores the explainability of AI-generated video detection using a Non-MLLM detector, leaving this area open for future investigation.

\subsubsection{\textbf{Localization}}
Currently, no research paper specifically addresses the Localization of detecting AI-generated videos for Non-MLLM detectors.


\subsection{Audio}
\subsubsection{\textbf{Authenticity}}
\begin{itemize}
    \item \textbf{Fingerprint}
    Traditional audio detection methods often rely on handcrafted features that encompass both perceptual and physical attributes. Salvi et al.~\cite{salvi2024listening} suggest that each TTS model may have a unique ``fingerprint", which is derived from background noise and high-frequency components. 
    \item\textbf{Watermarking}
    %在生成音频的时候加入水印
    Deep-learning audio watermarking methods focus on multi-bit watermarking and follow a generator or detector framework.
    %Multi-Bit Watermarking:嵌入多个比特的信息来实现,可以传递更多的内容
    DeAR~\cite{liu2023dear} is designed to counter audio re-recording (AR) distortions by modeling these distortions through a pipeline of environmental reverberation, band-pass filtering, and Gaussian noise. The approach employs a differential time-frequency transform for optimal watermark embedding, allowing end-to-end training of the encoder and decoder without relying on predefined rules.
    AudioSeal~\cite{roman2024proactive} is a localized watermarking that jointly trains a generator and a detector to embed and robustly detect watermarks. The approach enhances detection accuracy by masking the watermark in random sections of the audio signal and extends to multi-bit watermarking, enabling the attribution of audio to specific models or versions without compromising the detection process.
    %Zero-Bit Watermarking:是一种不携带具体信息的水印方法,在音频信号中嵌入特定的模式或特征来表明某个音频片段是水印的
    Other researchers have explored zero-bit watermarking, which is better adapted for the detection of AI-generated media. 
    Wu et al.~\cite{wu2023adversarial} introduce small, imperceptible perturbations to the original audio, directing its deep features towards specific watermark characteristics. To ensure practical robustness, they utilize data augmentation and error-correcting coding techniques.
    \end{itemize}
    

\subsubsection{\textbf{Explainability}} About interpretability features, SLIM~\cite{zhu2024slim} addresses audio deepfake detection by exploiting the Style-Linguistics Mismatch between real and fake speech, where real speech exhibits a natural dependency between linguistic content and vocal style, while deepfakes break this dependency. It learns this dependency in two stages: first by contrasting the style and linguistic representations of real speech, and then by using these learned features to classify audio as real or fake.
SFAT-Net-3~\cite{cuccovillo2024audio} combines amplitude and phase encoding and introduces a more complex decoder to predict the F0, F1, and F2 phoneme trajectories.
Pascu et al.~\cite{pascu2024easy} use scalar features, such as Mean Unvoiced Segment Length, through the classifier to detect and offer interpretability in the process. 

\subsubsection{\textbf{Localization}}
%定位
For localization of AI-generated segments,
HarmoNet~\cite{liu2024harmonet} combines multi-scale harmonic F0 features with self-supervised learning representations and an attention mechanism and also introduces a new Partial Loss function to focus on the boundary between real and fake regions.
CFPRF~\cite{wu2024coarse} combines frame-level detection network and proposal refinement network with difference-aware feature learning and boundary-aware feature enhancement modules.

%绿色安全的检测
What's more, Green AI is important to protect users' rights.
Safeear~\cite{li2024safeear} develops a neural audio code that decouples semantic and acoustic information, providing a novel privacy-preserving approach for deepfake detection. 



\subsection{Multimodal}
\subsubsection{\textbf{Authenticity}}
\begin{itemize}
    \item \textbf{Text-visual}
    HAMMER~\cite{Shao2023CVPR}, based on hierarchical manipulation reasoning, integrates unimodal encoders, multimodal aggregators, and dedicated detection heads. It captures inter-modal interactions through manipulation-aware contrastive learning and modality-aware cross-attention for content detection. 
    
    \item \textbf{Audio-visual}
    AI-generated audio-visual detection often relies on content consistency detection methods~\cite{li2024zero}, while other researchers employ graph-based multimodal fusion strategies~\cite{yin2024fine} to enhance the detection process.
    Li et al.~\cite{li2024zero} propose a zero-shot detection method based on content consistency, which utilizes Automatic Speech Recognition and Visual Speech Recognition models to decode audio and video content, respectively, generating content sequences for both modalities. Then it calculates the edit distance between these two content sequences as a metric to measure the consistency between the audio and video modalities.
    Yin et al.~\cite{yin2024fine} constructs heterogeneous graphs using positional encoding, capturing intra- and inter-modal relationships through cross-modal graph interaction and dehomogenized graph pooling modules. 

    \item \textbf{Trimodal}
    For trimodal fusion detection methods, there is a notable fusion strategy that effectively integrates the three modalities.
    Yoon et al.~\cite{yoon2024triple} propose a trimodal deepfake detection method using zero-shot identity and one-shot deepfake baselines, implementing visual, auditory, and linguistic feature interaction through a two-stage approach, with residual connections and late fusion to prevent information loss.
    
\end{itemize}

\subsubsection{\textbf{Localization}}
There are only localization methods for visual-audio.
DiMoDif~\cite{koutlis2024dimodif} detects forged content by calculating the differences between audio and video signals and using these differences to identify forgeries. Additionally, it optimizes the localization accuracy of the forged regions by calculating the overlap between the predicted forged intervals and the ground truth annotations.
MMMS-BA~\cite{katamneni2024contextual} framework effectively captures the interaction between audio and video signals using a cross-modal attention mechanism across multiple modalities and sequences. Additionally, it performs deepfake detection and localization through classification and regression heads.


\section{Implementation and Evaluation}
\label{sec:evaluation}

We prototype our proposal into a tool \toolName, using approximately 5K lines of OCaml (for the program analysis) and 5K lines of Python code (for the repair). 
In particular, we employ Z3~\cite{DBLP:conf/tacas/MouraB08} as the SMT solver, clingo~\cite{DBLP:books/sp/Lifschitz19} as the ASP solver, and Souffle~\cite{scholz2016fast} as the Datalog engine. %, respectively.
To show the effectiveness, 
we design the experimental evaluation to answer the 
following research questions (RQ):
(Experiments ran on a server with an Intel® Xeon® Platinum 8468V, 504GB RAM, and 192 cores. All the dataset are publicly available from \cite{zenodo_benchmark})

\begin{itemize}[align=left, leftmargin=*,labelindent=0pt]
\item \textbf{RQ1:} How effective is \toolName in verifying CTL properties for relatively small but complex programs, compared to the state-of-the-art tool  \function~\cite{DBLP:conf/sas/UrbanU018}?


\item \textbf{RQ2:} What is the effectiveness of \toolName in detecting real-world bugs, which can be encoded using both CTL and linear temporal logic (LTL), such as non-termination gathered from GitHub \cite{DBLP:conf/sigsoft/ShiXLZCL22} and unresponsive behaviours in protocols  \cite{DBLP:conf/icse/MengDLBR22}, compared with \ultimate~\cite{DBLP:conf/cav/DietschHLP15}?

\item \textbf{RQ3:} How effective is \toolName in repairing CTL violations identified in RQ1 and RQ2? which has not been achieved by any existing tools. 


 

\end{itemize}



% \begin{itemize}[align=left, leftmargin=*,labelindent=0pt]
% \item \textbf{RQ1:} What is the effectiveness of \toolName in verifying CTL properties in a set of relatively small yet challenging programs, compared to the state-of-the-art tools, T2~\cite{DBLP:conf/fmcad/CookKP14},  \function~\cite{DBLP:conf/sas/UrbanU018}, and \ultimate~\cite{DBLP:conf/cav/DietschHLP15}?


% \item \textbf{RQ2:} What is the effectiveness of \toolName in finding  real-world bugs, which can be encoded using CTL properties, such as non-termination 
% gathered from GitHub \cite{DBLP:conf/sigsoft/ShiXLZCL22} and unresponsive behaviours in protocol implementations \cite{DBLP:conf/icse/MengDLBR22}?

% \item \textbf{RQ3:} What is the effectiveness of \toolName in repairing CTL bugs from RQ1--2?

% \end{itemize}

%The benchmark programs are from various sources. More specifically, termination bugs from real-world projects \cite{DBLP:conf/sigsoft/ShiXLZCL22} and CTL analysis \cite{DBLP:conf/fmcad/CookKP14} \cite{DBLP:conf/sas/UrbanU018}, and temporal bugs in real-world protocol implementations \cite{DBLP:conf/icse/MengDLBR22}. 



% \ly{are termination bugs ok? Do we need to add new CTL bugs?}
\subsection{RQ1: Verifying CTL Properties}

% Please add the following required packages to your document preamble:
%  \Xhline{1.5\arrayrulewidth}

\hide{\begin{figure}[!h]
\vspace{-8mm}
\begin{lstlisting}[xleftmargin=0.2em,numbersep=6pt,basicstyle=\footnotesize\ttfamily]
(*@\textcolor{mGray}{//$EF(\m{resp}{\geq}5)$}@*)
int c = *; int resp = 0;
int curr_serv = 5; 
while (curr_serv > 0){ 
 if (*) {  
   c--; 
   curr_serv--;
   resp++;} 
 else if (c<curr_serv){
   curr_serv--; }}
\end{lstlisting} 
\vspace{-2mm}
\caption{A possibly terminating loop} 
\label{fig:terminating_loop}
\vspace{-2mm}
\end{figure}}


%loses precision due to a \emph{dual widening} \cite{DBLP:conf/tacas/CourantU17}, and 

The programs listed in \tabref{tab:comparewithFuntionT2} were obtained from the evaluation benchmark of \function, which includes a total of 83 test cases across over 2,000 lines of code. We categorize these test cases into six groups, labeled according to the types of CTL properties. 
These programs are short but challenging, as they often involve complex loops or require a more precise analysis of the target properties. The \function tends to be conservative, often leading it to return ``unknown" results, resulting in an accuracy rate of 27.7\%. In contrast, \toolName demonstrates advantages with improved accuracy, particularly in \ourToolSmallBenchmark. 
%achieved by the novel loop summaries. 
The failure cases faced by \toolName highlight our limitations when loop guards are not explicitly defined or when LRFs are inadequate to prove termination. 
Although both \function and \toolName struggle to obtain meaningful invariances for infinite loops, the benefits of our loop summaries become more apparent when proving properties related to termination, such as reachability and responsiveness.  




\begin{table}[!t]
\vspace{1.5mm}
\caption{Detecting real-world CTL bugs.}
\normalsize
\label{tab:comparewithCook}
\renewcommand{\arraystretch}{0.95}
\setlength{\tabcolsep}{4pt}  
\begin{tabular}{c|l|c|cc|cc}
\Xhline{1.5\arrayrulewidth}
\multicolumn{1}{l|}{\multirow{2}{*}{\textbf{}}} & \multirow{2}{*}{\textbf{Program}}        & \multirow{2}{*}{\textbf{LoC}} & \multicolumn{2}{c|}{\textbf{\ultimateshort}}   & \multicolumn{2}{c}{\textbf{\toolName}}             \\ \cline{4-7} 
  \multicolumn{1}{l|}{}                           &                                          &                               & \multicolumn{1}{c|}{\textbf{Res.}} & \textbf{Time} & \multicolumn{1}{c|}{\textbf{Res.}} & \textbf{Time} \\ \hline
  1 \xmark                                      & \multirow{2}{*}{\makecell[l]{libvncserver\\(c311535)}}   & 25                            & \multicolumn{1}{c|}{\xmark}      & 2.845         & \multicolumn{1}{c|}{\xmark}      & 0.855         \\  
  1 \cmark                                      &                                          & 27                            & \multicolumn{1}{c|}{\cmark}      & 3.743         & \multicolumn{1}{c|}{\cmark}      & 0.476         \\ \hline
  2 \xmark                                      & \multirow{2}{*}{\makecell[l]{Ffmpeg\\(a6cba06)}}         & 40                            & \multicolumn{1}{c|}{\xmark}      & 15.254        & \multicolumn{1}{c|}{\xmark}      & 0.606         \\  
  2 \cmark                                      &                                          & 44                            & \multicolumn{1}{c|}{\cmark}      & 40.176        & \multicolumn{1}{c|}{\cmark}      & 0.397         \\ \hline
  3 \xmark                                      & \multirow{2}{*}{\makecell[l]{cmus\\(d5396e4)}}           & 87                            & \multicolumn{1}{c|}{\xmark}      & 6.904         & \multicolumn{1}{c|}{\xmark}      & 0.579         \\  
  3 \cmark                                      &                                          & 86                            & \multicolumn{1}{c|}{\cmark}      & 33.572        & \multicolumn{1}{c|}{\cmark}      & 0.986         \\ \hline
  4 \xmark                                      & \multirow{2}{*}{\makecell[l]{e2fsprogs\\(caa6003)}}      & 58                            & \multicolumn{1}{c|}{\xmark}      & 5.952         & \multicolumn{1}{c|}{\xmark}      & 0.923         \\  
  4 \cmark                                      &                                          & 63                            & \multicolumn{1}{c|}{\cmark}      & 4.533         & \multicolumn{1}{c|}{\cmark}      & 0.842         \\ \hline
  5 \xmark                                      & \multirow{2}{*}{\makecell[l]{csound-an...\\(7a611ab)}} & 43                            & \multicolumn{1}{c|}{\xmark}      & 3.654         & \multicolumn{1}{c|}{\xmark}      & 0.782         \\  
  5 \cmark                                      &                                          & 45                            & \multicolumn{1}{c|}{TO}          & -             & \multicolumn{1}{c|}{\cmark}      & 0.648         \\ \hline
  6 \xmark                                      & \multirow{2}{*}{\makecell[l]{fontconfig\\(fa741cd)}}     & 25                            & \multicolumn{1}{c|}{\xmark}      & 3.856         & \multicolumn{1}{c|}{\xmark}      & 0.769         \\  
  6 \cmark                                      &                                          & 25                            & \multicolumn{1}{c|}{Error}       & -             & \multicolumn{1}{c|}{\cmark}      & 0.651         \\ \hline
  7 \xmark                                      & \multirow{2}{*}{\makecell[l]{asterisk\\(3322180)}}       & 22                            & \multicolumn{1}{c|}{\unk}        & 12.687        & \multicolumn{1}{c|}{\unk}        & 0.196         \\  
  7 \cmark                                      &                                          & 25                            & \multicolumn{1}{c|}{\unk}        & 11.325        & \multicolumn{1}{c|}{\unk}        & 0.34          \\ \hline
  8 \xmark                                      & \multirow{2}{*}{\makecell[l]{dpdk\\(cd64eeac)}}          & 45                            & \multicolumn{1}{c|}{\xmark}      & 3.712         & \multicolumn{1}{c|}{\xmark}      & 0.447         \\  
  8 \cmark                                      &                                          & 45                            & \multicolumn{1}{c|}{\cmark}      & 2.97          & \multicolumn{1}{c|}{\unk}        & 0.481         \\ \hline
  9 \xmark                                      & \multirow{2}{*}{\makecell[l]{xorg-server\\(930b9a06)}}   & 19                            & \multicolumn{1}{c|}{\xmark}      & 3.111         & \multicolumn{1}{c|}{\xmark}      & 0.581         \\  
  9 \cmark                                      &                                          & 20                            & \multicolumn{1}{c|}{\cmark}      & 3.101         & \multicolumn{1}{c|}{\cmark}      & 0.409         \\ \hline
  10 \xmark                                      & \multirow{2}{*}{\makecell[l]{pure-ftpd\\(37ad222)}}      & 42                            & \multicolumn{1}{c|}{\cmark}      & 2.555         & \multicolumn{1}{c|}{\xmark}      & 0.933         \\  
  10 \cmark                                      &                                          & 49                            & \multicolumn{1}{c|}{\cmark}        & 2.286         & \multicolumn{1}{c|}{\cmark}      & 0.383         \\ \hline
  11 \xmark  & \multirow{2}{*}{\makecell[l]{live555$_a$\\(181126)}} & 34  & \multicolumn{1}{c|}{\cmark} &  2.715         & \multicolumn{1}{c|}{\xmark}    & 0.513   \\  
  11 \cmark  &     &   37    & \multicolumn{1}{c|}{\cmark} &  2.837         & \multicolumn{1}{c|}{\cmark}      & 0.341 \\ \hline
  12 \xmark  & \multirow{2}{*}{\makecell[l]{openssl\\(b8d2439)}} & 88  & \multicolumn{1}{c|}{\xmark} &  4.15          & \multicolumn{1}{c|}{\xmark}    & 0.78   \\
  12 \cmark  &     &  88     & \multicolumn{1}{c|}{\cmark} &  3.809         & \multicolumn{1}{c|}{\cmark}      & 0.99 \\ \hline
  13 \xmark  & \multirow{2}{*}{\makecell[l]{live555$_b$\\(131205)}} & 83  & \multicolumn{1}{c|}{\xmark} & 2.838         & \multicolumn{1}{c|}{\xmark}    & 0.602     \\  
  13 \cmark  &    &   84     & \multicolumn{1}{c|}{\cmark} &  2.393         & \multicolumn{1}{c|}{\cmark}      & 0.565 \\ \Xhline{1.5\arrayrulewidth}
                                                   & {\bf{Total}}                                  & 1249  & \multicolumn{1}{c|}{\bestBaseLineReal}          & $>$180       & \multicolumn{1}{c|}{\ourToolRealBenchmark}              & 16.01        \\ \Xhline{1.5\arrayrulewidth}
  \end{tabular}
  \end{table}

\subsection{RQ2: CTL Analysis on  Real-world Projects}




Programs in \tabref{tab:comparewithCook} are from real-world repositories, each associated with a Git commit number where developers identify and fix the bug manually. 
In particular, the property used for programs 1-9 (drawn from \cite{DBLP:conf/sigsoft/ShiXLZCL22}) is  \code{AF(Exit())}, capturing non-termination bugs. The properties used for programs 10-13 (drawn from \cite{DBLP:conf/icse/MengDLBR22}) are of the form \code{AG(\phi_1{\rightarrow}AF(\phi_2))}, capturing unresponsive behaviours from the protocol implementation. 
We extracted the main segments of these real-world bugs into smaller programs (under 100 LoC each), preserving features like data structures and pointer arithmetic. Our evaluation includes both buggy (\eg 1\,\xmark) and developer-fixed (\eg 1\,\cmark) versions.
After converting the CTL properties to LTL formulas, we compared our tool with the latest release of UltimateLTL (v0.2.4), a regular participant in SV-COMP \cite{svcomp} with competitive performance. 
Both tools demonstrate high accuracy in bug detection, while \ultimateshort often requires longer processing time. 
This experiment indicates that LRFs can effectively handle commonly seen real-world loops, and \toolName performs a more lightweight summary computation without compromising accuracy. 



%Following the convention in \cite{DBLP:conf/sigsoft/ShiXLZCL22}, t
%Prior works \cite{DBLP:conf/sigsoft/ShiXLZCL22} gathered such examples by extracting 
%\toolName successfully identifies the majority of buggy and correct programs, with the exception of programs 7 and 8. 







{
\begin{table*}[!h]
  \centering
\caption{\label{tab:repair_benchmark}
{Experimental results for repairing CTL bugs. Time spent by the ASP solver is separately recorded. 
}
}
\small
\renewcommand{\arraystretch}{0.95}
  \setlength{\tabcolsep}{9pt}
\begin{tabular}{l|c|c|c|c|c|c|c|c}
  \Xhline{1.5\arrayrulewidth}
  \multicolumn{1}{c|}{\multirow{2}{*}{\textbf{Program}}} & \multicolumn{1}{c|}{\multirow{2}{*}{\shortstack{\textbf{LoC}\\\textbf{(Datalog)}}}} & \multicolumn{3}{c|}{\textbf{Configuration}}                                 & \multicolumn{1}{c|}{\multirow{2}{*}{\textbf{Fixed}}} & \multicolumn{1}{c|}{\multirow{2}{*}{\textbf{\#Patch}}} & \multicolumn{1}{c|}{\multirow{2}{*}{\textbf{ASP(s)}}} & \multirow{2}{*}{\textbf{Total(s)}} \\ \cline{3-5}

  \multicolumn{1}{c|}{}                                  & \multicolumn{1}{c|}{}                              & \multicolumn{1}{c|}{\textbf{Symbols}} & \multicolumn{1}{c|}{\textbf{Facts}} & \multicolumn{1}{c|}{\textbf{Template}} & \multicolumn{1}{c|}{} & \multicolumn{1}{c|}{} & \multicolumn{1}{c|}{}  &                                      \\ \hline

AF\_yEQ5 (\figref{fig:first_Example})                                           & 115                           & 3+0                   & 0+1                & Add                & \cmark     & 1                   & 0.979                              & 1.593                                \\
test\_until.c                                         & 101                            & 0+3                   & 1+0                & Delete                & \cmark     & 1                   & 0.023                              & 0.498                                \\
next.c                                                & 87                            & 0+4                   & 1+0                & Delete                & \cmark     & 1                   & 0.023                              & 0.472                                \\
libvncserver                                          & 118                            & 0+6                   & 1+0                & Delete                & \cmark     & 3                   & 0.049                              & 1.081                                \\
Ffmpeg                                                & 227                           & 0+12                  & 1+0                & Delete                & \cmark     & 4                   & 13.113                              & 13.335                                \\
cmus                                                  & 145                           & 0+12                  & 1+0                & Delete                & \cmark     & 4                   & 0.098                              & 2.052                                \\
e2fsprogs                                             & 109                           & 0+8                   & 1+0                & Delete                & \cmark     & 2                   & 0.075                              & 1.515                                \\
csound-android                                        & 183                           & 0+8                   & 1+0                & Delete                & \cmark     & 4                   & 0.076                              & 1.613                                \\
fontconfig                                            & 190                           & 0+11                  & 1+0                & Delete                & \cmark     & 6                   & 0.098                              & 2.507                                \\
dpdk                                                  & 196                           & 0+12                  & 1+0                & Delete                & \cmark     & 1                   & 0.091                              & 2.006                                \\
xorg-server                                           & 118                            & 0+2                   & 1+0                & Delete                & \cmark     & 2                   & 0.026                              & 0.605                                \\
pure-ftpd                                             & 258                           & 0+21                  & 1+0                & Delete                & \cmark     & 2                   & 0.069                              & 3.590                               \\
live$_a$                                              & 112                            & 3+4                   & 1+1                & Update                & \cmark     & 1                   & 0.552                              & 0.816                                \\
openssl                                               & 315                           & 1+0                   & 0+1                & Add.                & \cmark     & 1                   & 1.188                              & 2.277                                \\
live$_b$                                              & 217                           & 1+0                   & 0+1                & Add                & \cmark     & 1                   & 0.977                              & 1.494                                 \\
  \Xhline{1.5\arrayrulewidth}
\textbf{Total}                                                 & 2491                          &                       &                    &                   &           &                     & 17.437                              & 35.454                               \\ 
  \Xhline{1.5\arrayrulewidth}           
\end{tabular}

\vspace{-2mm}
\end{table*}
}


\subsection{RQ3: Repairing CTL Property Violations} 


\tabref{tab:repair_benchmark} gathers all the program instances (from \tabref{tab:comparewithFuntionT2} and \tabref{tab:comparewithCook}) that violate their specified CTL properties and are sent to \toolName for repair.   
The \textbf{Symbols} column records the number of symbolic constants + symbolic signs, while the \textbf{Facts} column records the number of facts allowed to be removed + added. 
We gradually increase the number of symbols and the maximum number of facts that can be added or deleted. 
The \textbf{Configuration} column shows the first successful configuration that led to finding patches, and we record the total searching time till reaching such configurations. 
We configure \toolName to apply three atomic templates in a breadth-first manner with a depth limit of 1, \ie, \tabref{tab:repair_benchmark} records the patch result after one iteration of the repair. 
The templates are applied sequentially in the order: delete, update, and add. The repair process stops when a correct patch is found or when all three templates have been attempted. 
%without success. 
% Because of this configuration, \toolName only finds one patch for Program 1 (AF\_yEQ5). 
% The patch inserting \plaincode{if (i>10||x==y) \{y=5; return;\}} mentioned in \figref{fig:Patched-program} cannot be found in current configuration, as it requires deleting facts then adding new facts on the updated program.
% The `Configuration' column in \tabref{tab:repair_benchmark} shows the number of symbolic constants and signs, the number of facts allowed to be removed and added, and the template used when a patch is found.

Due to the current configuration, \toolName only finds patch (b) for Program 1 (AF\_yEQ5), while the patch (a) shown in \figref{fig:Patched-program} can be obtained by allowing two iterations of the repair: the first iteration adds the conditional than a second iteration to add a new assignment on the updated program. 
Non-termination bugs are resolved within a single iteration by adding a conditional statement that provides an earlier exit. 
For instance, \figref{fig:term-Patched-program} illustrates the main logic of 1\,\xmark, which enters an infinite loop when \code{\m{linesToRead}{\leq}0}. 
\toolName successfully 
provides a fix that prevents \code{\m{linesToRead}{\leq}0} from occurring before entering the loop. Note that such patches are more desirable which fix the non-termination bug without dropping the loops completely. 
%much like the example shown in  \figref{fig:term-Patched-program}. At the same time, 
Unresponsive bugs involve adding more function calls or assignment modifications. 
%Most repairs were completed within seconds. 

On average, the time taken to solve ASP accounts for 49.2\% (17.437/35.454) of the total repair time. We also keep track of the number of patches that successfully eliminate the CTL violations. More than one patch is available for non-termination bugs, as some patches exit the entire program without entering the loop. 
While all the patches listed are valid, those that intend to cut off the main program logic can be excluded based on the minimum change criteria. 
After a manual inspection of each buggy program shown in \tabref{tab:repair_benchmark}, we confirmed that at least one generated patch is semantically equivalent to the fix provided by the developer. 
As the first tool to achieve automated repair of CTL violations, \toolName successfully resolves all reported bugs. 



\begin{figure}[!t]
\begin{lstlisting}[xleftmargin=6em,numbersep=6pt,basicstyle=\footnotesize\ttfamily]
void main(){ //AF(Exit())
  int lines ToRead = *;
  int h = *;
  (*@\repaircode{if ( linesToRead <= 0 )  return;}@*)
  while(h>0){
    if(linesToRead>h)  
        linesToRead=h; 
    h-=linesToRead;} 
  return;}
\end{lstlisting}
\caption{Fixing a Possible Hang Found in libvncserver \cite{LibVNCClient}}
\label{fig:term-Patched-program}
\end{figure}


\section{Regulation}
\label{sec:reg}

\begin{table*}[!ht]
    \centering
        \renewcommand{\arraystretch}{1.4}
        \caption{Comparison of AI Governance Approaches in the \textbf{EU}, \textbf{USA}, and \textbf{China} across four dimensions: Risk Management Frameworks, Transparency Requirements, Technical Neutrality, and Industry Participation. This table highlights the unique priorities and methodologies each region adopts in addressing AI-generated content detection and governance.}
        \label{tab:ai_governance}

        \resizebox{\linewidth}{!}{
        \begin{tabular}
        %{m{4cm}|m{5cm}|m{5cm}|m{5cm}}\hline 
        {p{4cm}<{\centering} | p{5cm}<{\centering} | p{5cm}<{\centering} | p{5cm}<{\centering}}\hline
        \rowcolor{lightgrey} 
\textbf{Aspect} & 
\textbf{EU} &
\textbf{USA} &
\textbf{China} \\ \hline



\textbf{Risk Management Framework} 
& Four risk levels (minimal risk, limited risk, high risk, and unacceptable risk)
& Non-binding guidance
& A classification and grading approach is adopted, emphasizing inclusive and prudent regulation. \\ \hline

\textbf{Transparency Requirements} 
& \begin{itemize}
    \item AI-generated content must be clearly labeled.
    \item Record model training data sources and decision processes for external audits.
    \item Mandate explainability modules to help users understand AI decision logic.
\end{itemize}
& \begin{itemize}
    \item Encourage companies to voluntarily use watermarks or labels in generated content.
    \item Promote the development of transparency standards, such as industry collaboration on transparency APIs.
\end{itemize}
& \begin{itemize}
    \item Establish legal obligations for identifying generative AI content.
    \item Require generative AI platforms to regularly disclose algorithm models, training data, and technical documentation.
\end{itemize} \\ \hline

\textbf{Technology Neutrality Principle} 
& Less emphasis on technological neutrality, favoring a risk-oriented approach
& Emphasizes technological neutrality to safeguard innovation freedom.
& Combines technological neutrality with a risk-oriented approach. \\ \hline

\textbf{Degree of Industry Participation} 
& \begin{itemize}
    \item Prefers mandatory legal regulations to ensure industry participation.
    \item Establishes a unified regulatory framework to ensure compliance by both multinational corporations and SMEs.
\end{itemize}
& \begin{itemize}
    \item Encourages industry-led initiatives with voluntary participation in regulation.
\end{itemize}
& \begin{itemize}
    \item Industry participation is guided primarily by policy, with the government fostering collaboration across the industrial chain.
    \item Require generative AI platforms to regularly disclose algorithm models, training data, and technical documentation.
\end{itemize} \\ \hline
        

%_____________________________________
\end{tabular}
        }
    \label{regulation}
\end{table*}

In recent years, the rapid development of GenAI technologies has not only driven technological innovation and industrial advancement but also raised societal concerns, including the spread of misinformation, data privacy breaches, and ethical controversies. The rapid dissemination and difficult-to-monitor nature of AI-generated media have prompted governments and research institutions worldwide to focus on effectively regulating the applications and potential impacts of generative AI. Against this backdrop, we examine AI-generated media detection policies from four perspectives~\cite{shi2024large}: risk management frameworks, transparency requirements, technical neutrality, and industry participation. Risk management frameworks~\cite{novelli2024taking, zeng2024ai} evaluate how different countries identify, classify, and mitigate the potential risks of AI systems through policy and technical measures. Transparency requirements examine the implementation of policies on data source disclosure, algorithm transparency, and external audits.
The technical neutrality perspective explores whether AI regulations are enforced in a technology-neutral manner to avoid stifling innovation and industrial growth.
Industry participation analyzes the depth and breadth of collaboration between governments and enterprises in AI-generated media detection, including the interplay of legal mandates and voluntary contributions.
Analyzing these dimensions reveals differences in governance priorities across nations while providing valuable insights for researchers and policymakers to foster global collaboration and advancement in AI-generated media detection.

In 2024, the European Union (EU) passed the world’s first comprehensive artificial intelligence regulation, the Artificial Intelligence Act (AIA)~\cite{ArtificialIntelligenceAct}. It adopts a risk-based tiered regulatory approach, categorizing AI systems into four levels: minimal risk, limited risk, high risk, and unacceptable risk. Generative AI systems are generally classified as limited risk, requiring basic transparency obligations. The United States (US) emphasizes technical neutrality and industry self-regulation. The National Institute of Standards and Technology (NIST) introduced the AI Risk Management Framework (AI RMF) to guide developers in identifying and mitigating risks. Meanwhile, several legislative initiatives, such as the No AI Fraud Act and the COPIED Act, aim to protect intellectual property and combat deepfakes. China~\cite{ChinaAIGovernance2023} focuses on safety controls and ethical use within its governance framework. Policies like the Generative AI Service Management Provisions adopt an inclusive, risk-sensitive classification and grading approach, encouraging AI integration into national governance. A detailed comparison is presented in Table~\ref{tab:ai_governance}.

Looking ahead, global AI governance must balance innovation with regulation. Combining the EU’s tiered framework, the US’s technical neutrality and self-regulation model, and China’s classification-based oversight can promote multilateral collaboration and standardization. Policies should strengthen the integration of technology and ethics, enhancing governance flexibility and responsiveness. Industry stakeholders should actively participate in policy formulation, leveraging dynamic monitoring and transparency requirements to ensure AI safety and social responsibility, achieving a win-win for innovation and compliance.




%当前AI风险的定义与分类,指出风险的复杂性。随着人工智能技术的进步,与其部署相关风险也在增加,因此有必要对主要监管机构采取不同方法进行比较。因此我们探讨了美国、欧盟和中国采取的政策的区别。AIA是全球第一部全面的人工智能法案,采用基于风险的方法将人工智能风险分为四类,相比之下,美国的ai监管技术以技术中立和行业自律为主。中国实施了强调安全控制和道德使用的ai治理框架,鼓励人工智能融入国家治理体系,并对风险进行了列举。事实上,现在还没有。人工智能的治理不仅需要各国的努力还需要全球协调应对措施,以在促进创新的同时降低风险。正如越来越多的学者所强调的,单一静态分散的法规是不够管理复杂全球人工智能生态系统。事实上,现在急需一个全球人工智能风险分类法对ai risk进行基于具体场景的分类和集成式的风险评估模型,将AI风险管理从静态分类转向动态情境评估。因为随着通用人工智能GPAI的多功能性和应用的不可预测性,并且静态的风险评估方法可能低估或高估某些实际风险,使得传统的基于应用领域的风险评估框架难以有效使用。需要强调的是,不仅是需要明确的法律法规对AIrisk进行监管,还需要对应的AI监管技术以确保人工智能系统在整个生命周期中安全、负责和透明,并遵守道德和安全标准。AI法案和监管技术在发展的同时还需要满足以人为本、可持续性的人工智能治理模型。

\section{Theoretical Analysis}\label{sec:theoretical}

\textbf{Different correct answers are competitor.}\quad For any LLM trained with cross-entropy loss, different correct answers are competitors in terms of probability \footnote{The ``same question'' refers to questions that are semantically equivalent but do not need to be identical.}. Continuing with the example of proposing a president, suppose $\tau^{a}$ (``\texttt{Barack}'') is the label of a sample whose $\bm{q}$ is ``\texttt{[INST]Could you give me one name of president?[\textbackslash INST]}'' and a generated token vector $\bm{a}_{t-1}$  can be decoded into ``\texttt{Sure, here is a historical American president:**}'', the loss of the next token at this position during supervised fine-tuning can be written as:
\begin{equation}
\begin{aligned}
 &L^{\tau^a} = - \log \frac{\exp(\mathcal{M}({\tau^a}|\bm{q},\bm{a}_{t-1}))}{\sum_{m=1}^{|\bm{Y}|} \exp(\mathcal{M}(\tau^{m}|\bm{q},\bm{a}_{t-1}))} ,
 % \\   &L^{\tau^b} = - \log \frac{\exp(\mathcal{M}(\tau^b|\bm{q},\bm{a}_{t-1}))}{\sum_{m=1}^{|\bm{Y}|} \exp(\mathcal{M}(\tau^{m}|\bm{q},\bm{a}_{t-1}))} ,
\end{aligned}
\end{equation}
where $L^{\tau^a}$ is the loss on the sample with the next token label $\tau^{a}$.
Consider cases where multiple distinct answers to the same question appear in the training set, the situation becomes different. For example, $\tau^{b}$ (``\texttt{George}'') is the label in another sample with the same question. When the model is simultaneously fine-tuned on both samples, the gradient update for the model will be:
\begin{equation}
\begin{aligned}
 & \nabla_{\mathcal{M}} (L^{\tau^a} + L^{\tau^b}) = \nabla_{\mathcal{M}} L^{\tau^a} + \nabla_{\mathcal{M}} L^{\tau^b} \\
% &= -y_a^{\tau^a}\frac{1}{\Omega_a^{\tau^a}}\nabla_{\mathcal{M}}\Omega_a^{\tau^a}-\sum_{m \neq a}^{|\bm{Y}|} y_a^{\tau^m}\frac{1}{\Omega_a^{\tau^m}}\nabla_{\mathcal{M}}\Omega_a^{\tau^m}
% \\
% &\quad -y_b^{\tau^b}\frac{1}{\Omega_b^{\tau^b}}\nabla_{\mathcal{M}}\Omega_b^{\tau^b}-\sum_{m \neq b}^{|\bm{Y}|} y_b^{\tau^m}\frac{1}{\Omega_b^{\tau^m}}\nabla_{\mathcal{M}}\Omega_b^{\tau^m}
% \\
&\quad= \underbrace{-y_a^{\tau^a}\frac{1}{\Omega_a^{\tau^a}}\nabla_{\mathcal{M}}\Omega_a^{\tau^a}-y_b^{\tau^b}\frac{1}{\Omega_b^{\tau^b}}\nabla_{\mathcal{M}}\Omega_b^{\tau^b}}_{\text{(1) maximizing the probability of annotated answer}}\\& \quad \underbrace{-y_a^{\tau^b}\frac{1}{\Omega_a^{\tau^b}}\nabla_{\mathcal{M}}\Omega_a^{\tau^b}-y_b^{\tau^a}\frac{1}{\Omega_b^{\tau^a}}\nabla_{\mathcal{M}}\Omega_b^{\tau^a}}_{{\text{\textbf{(2)} minimizing the probability of the other annotated answer}}}\\& \quad \underbrace{-\sum_{m \neq a,b}^{|\bm{Y}|}y_{a,b}^{\tau^m} \left[ \frac{1}{\Omega_a^{\tau^m}}\nabla_{\mathcal{M}}\Omega_a^{\tau^m} + \frac{1}{\Omega_b^{\tau^m}}\nabla_{\mathcal{M}}\Omega_b^{\tau^m} \right]}_{\text{(3) minimizing the probability of incorrect answers}},
\end{aligned}\label{eq:competitor}
\end{equation}
where $\Omega_a^{\tau^a}=\frac{\exp(\mathcal{M}(\tau^a|\bm{q},\bm{a}_{t-1}))}{\sum_{m=1}^{|\bm{Y}|} \exp(\mathcal{M}(\tau^{m}|\bm{q},\bm{a}_{t-1}))}$, and $y_a^{\tau^m}$ indicates the next token label of a training sample with ground-truth label ${\tau^a}$, that is, we have $y_a^{\tau^a}=1$ and $y_a^{\tau^b}=0$. In particular, when $\mathcal{M}$ is in a certain state during training, we have $\Omega_a^{\tau^a}=\Omega_b^{\tau^a}$, and we make distinctions to facilitate the reader's understanding here. As we can see, for scenarios with multiple answers, the training objective can be divided into three parts:
(1) For each sample, increase the probability of its own annotation in the output distribution.
(2) For each sample, decrease the probability of another sample's annotation in the output distribution. \textit{\textbf{Note:}} This part leads to the issue where probability cannot anymore capture the reliability of LLM responses, as different correct answers tend to reduce the probability of other correct answers, making low probabilities cannot indicates low reliability.
(3) For both samples, decrease the probability of other outputs not present in the annotations in the output distribution.





\section{CONCLUSION}
\label{sec:concl}
The rapid rise of AI-generated media challenges information authenticity and societal trust, necessitating robust detection mechanisms. This survey examines the evolution of AI-generated media detection, focusing on the shift from Non-MLLM-based domain-specific detectors to MLLM-based general-purpose approaches. We compare these methods across authenticity, explainability, and localization tasks from both single-modal and multi-modal perspectives. Additionally, we review datasets, methodologies, and evaluation metrics, identifying key limitations and research challenges.
Beyond technical concerns, MLLM-based detection raises ethical and security issues. As GenAI sees broader deployment, regulatory frameworks vary significantly across jurisdictions, complicating governance. By summarizing these regulations, we provide insights for researchers navigating legal and ethical challenges.
While many challenges remain, We hope this survey sparks further discussion, informs future research, and contributes to a more secure and trustworthy AI ecosystem.




%\vfill
\bibliographystyle{unsrt}
\bibliography{mainbib}
\end{document}
\renewcommand\refnamesec{参考文献} 
% \bibliographystylesec{plain}
\bibliographystylesec{unsrt}
\bibliographysec{引用}

%\newpage

% \section{Biography Section}
% If you have an EPS/PDF photo (graphicx package needed), extra braces are
%  needed around the contents of the optional argument to biography to prevent
%  the LaTeX parser from getting confused when it sees the complicated
% \begin{IEEEbiographynophoto}{John Doe}
% Use $\backslash${\tt{begin\{IEEEbiographynophoto\}}} and the author name as the argument followed by the biography text.
% \end{IEEEbiographynophoto}
% \end{document}



