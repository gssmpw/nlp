%%%%%%%% ICML 2025 EXAMPLE LATEX SUBMISSION FILE %%%%%%%%%%%%%%%%%

\documentclass{article}
\usepackage{amsmath} 
\let\iint\relax
% Recommended, but optional, packages for figures and better typesetting:
\usepackage{microtype}
\usepackage{graphicx, caption, subcaption}
\usepackage{wrapfig}
%\usepackage{subfigure}
\usepackage{wasysym}
\usepackage{booktabs} % for professional tables

% hyperref makes hyperlinks in the resulting PDF.
% If your build breaks (sometimes temporarily if a hyperlink spans a page)
% please comment out the following usepackage line and replace
% \usepackage{icml2025} with \usepackage[nohyperref]{icml2025} above.
\usepackage{hyperref}


% Attempt to make hyperref and algorithmic work together better:
\newcommand{\theHalgorithm}{\arabic{algorithm}}
% Use the following line for the initial blind version submitted for review:
% \usepackage{icml2025}

% If accepted, instead use the following line for the camera-ready submission:
%\usepackage{icml2025}
\usepackage[accepted]{icml2025}

% For theorems and such
%\usepackage{amsmath}
\usepackage{amssymb}
\usepackage{mathtools}
\usepackage{amsthm}

\usepackage{url}
\usepackage{makecell}
\usepackage{array}
\usepackage{caption}
\usepackage{subcaption}
\usepackage{wrapfig}
\usepackage{graphicx}
\usepackage{multirow}
\usepackage{xfrac}
\usepackage{enumitem}
\usepackage{scalerel}
%\usepackage{algpseudocode}
%\let\algorithmic\relax


% if you use cleveref..
\usepackage[capitalize,noabbrev]{cleveref}

%%%%%%%%%%%%%%%%%%%%%%%%%%%%%%%%
% THEOREMS
%%%%%%%%%%%%%%%%%%%%%%%%%%%%%%%%
\theoremstyle{plain}
\newtheorem{theorem}{Theorem}[section]
\newtheorem{proposition}[theorem]{Proposition}
\newtheorem{lemma}[theorem]{Lemma}
\newtheorem{corollary}[theorem]{Corollary}
\theoremstyle{definition}
\newtheorem{definition}[theorem]{Definition}
\newtheorem{assumption}[theorem]{Assumption}
\theoremstyle{remark}
\newtheorem{remark}[theorem]{Remark}


% Todonotes is useful during development; simply uncomment the next line
%    and comment out the line below the next line to turn off comments
%\usepackage[disable,textsize=tiny]{todonotes}
\usepackage[textsize=tiny]{todonotes}
\renewcommand{\thesubfigure}{\alph{subfigure}}

% The \icmltitle you define below is probably too long as a header.
% Therefore, a short form for the running title is supplied here:
\icmltitlerunning{Field Matching: an Electrostatic Paradigm to Generate and Transfer Data}

\begin{document}

\twocolumn[
\icmltitle{Field Matching: an Electrostatic Paradigm to Generate and Transfer Data}

% It is OKAY to include author information, even for blind
% submissions: the style file will automatically remove it for you
% unless you've provided the [accepted] option to the icml2025
% package.

% List of affiliations: The first argument should be a (short)
% identifier you will use later to specify author affiliations
% Academic affiliations should list Department, University, City, Region, Country
% Industry affiliations should list Company, City, Region, Country

% You can specify symbols, otherwise they are numbered in order.
% Ideally, you should not use this facility. Affiliations will be numbered
% in order of appearance and this is the preferred way.
\icmlsetsymbol{equal}{*}

\begin{icmlauthorlist}
\icmlauthor{Alexander Kolesov}{equal,skoltech,airi}
\icmlauthor{Manukhov Stepan}{equal,skoltech,msu}
\icmlauthor{Vladimir V. Palyulin}{skoltech}
\icmlauthor{Alexander Korotin}{skoltech,airi}
\end{icmlauthorlist}
\icmlaffiliation{airi}{Artificial Intelligence Research Institute, Moscow, Russia}
\icmlaffiliation{skoltech}{Skolkovo Institute of Science and Technology, Moscow, Russia}
\icmlaffiliation{msu}{Lomonosov Moscow State University, Faculty of Physics, Moscow, Russia}


\icmlcorrespondingauthor{Alexander Kolesov}{a.kolesov@skoltech.ru}

% You may provide any keywords that you
% find helpful for describing your paper; these are used to populate
% the "keywords" metadata in the PDF but will not be shown in the document
\icmlkeywords{Machine Learning, ICML}

\vskip 0.3in
]

% this must go after the closing bracket ] following \twocolumn[ ...

% This command actually creates the footnote in the first column
% listing the affiliations and the copyright notice.
% The command takes one argument, which is text to display at the start of the footnote.
% The \icmlEqualContribution command is standard text for equal contribution.
% Remove it (just {}) if you do not need this facility.

%\printAffiliationsAndNotice{}  % leave blank if no need to mention equal contribution
\printAffiliationsAndNotice{\icmlEqualContribution} % otherwise use the standard text.

\begin{abstract}
We propose Electrostatic Field Matching (EFM), a novel method that is suitable for both generative modeling and distribution transfer tasks. Our approach is inspired by the physics of an electrical capacitor. We place source and target distributions on the capacitor plates and assign them positive and negative charges, respectively. We then learn the electrostatic field of the capacitor using a neural network approximator. To map the distributions to each other, we start at one plate of the capacitor and move the samples along the learned electrostatic field lines until they reach the other plate. We theoretically justify that this approach provably yields the distribution transfer. In practice, we demonstrate the performance of our EFM in toy and image data experiments.

% We propose electrostatic field matching (EFM), a novel method suitable both for generative modeling and distribution transfer. Our approach is inspired by the physics of an electric capacitor. We place source and target distributions on the capacitor plates and assign positive and negative charges to them, respectively. Then we learn a neural network approximator for the electrostatic field of the capacitor. To map the distributions to each other, starting on one plate of the capacitor, we move the samples along the learned electrostatic field lines until they reach the other plate. We theoretically justify that this approach indeed yields the distribution transfer. In practice, we  showcase the performance of our EFM in toy and image data experiments.

% Generative modeling is frequently inspired by physics. Concepts from non-equilibrium physics are at the heart of transformations in diffusion generative models. However, this physical methodology is not applicable for solving unconditional generation data  and unpaired translation problems simultaneously. We propose a novel scalable methodology Field Matching (\textbf{FiM}) based on  electrostatics theory. Our method allows to solve both problems. We also establish corresponding theoretical results of the method and demonstrate its applicability and effectiveness in illustrative scenarios and image data setups. Our source code is available at ...
\end{abstract}

%======================= Introduction =============================%
\section{Introduction}
\label{introduction}
The basic task of generative modeling is to learn a transformation between two distributions accessible by i.i.d. samples. The typical scenarios considered are \textbf{noise-to-data} \cite{goodfellow2014generative} and \textbf{data-to-data} \cite{zhu2017unpaired}. These are usually referred to as the unconditional data generation and data translation, respectively.
%The solution of the first task is a transformation from a simple distribution $\mathbb{P}(\cdot)$ to untractable distribution $\mathbb{Q}(\cdot)$, while a map in the second converts one complex distribution to another one. 
 
%ML researcher community often refers to the general physics for finding appropriate transformation between distributions.
Physics is often at the heart of the principles of generative modeling. The first attempt to link generative models and physics was made in Energy-Based models \citep[EBM]{lecun2005loss}. They parameterize data distributions using the
Gibbs-Boltzmann distribution density and generate data through simulation of Langevin dynamics \cite{du2019implicit, song2021train}.

\textbf{Diffusion Models} \citep[DM]{sohl2015deep, ho2020denoising, kingma2021variational} are a widely popular generative models' class which is inspired by \textit{nonequilibrium thermodynamics}. The diffusion models consist of forward and backward stochastic processes \cite{song2021score}. While the forward process corrupts the data via injection of Gaussian noise, the backward process reverses the forward process and recovers the data. 

\textbf{Poisson Flow Generative Models} \citep[PFGM]{xu2022poissonflowgenerativemodels,xu2023pfgm++} use ideas from the \textit{electrostatic} theory for the data generation process, recovering an electric field between a hyperplane of the data and a hemisphere of large radius.

Both DM and PFGM use physical principles to corrupt data, simplifying the data distribution to a tractable one. As a result, they are only used directly for \textbf{noise-to-data} tasks.
%Since PFGM and DM evolve data in degenerative process, noising data distribution to a simple distribution at infinity, they are directly applied only for noise-to-data tasks. 
  
%Thermodynamics \cite{sohl2015deep} and Electrostatics \cite{xu2022poissonflowgenerativemodels} is already at the heart of some generative models, while the theory of wave equations and Yukawa's interactions wait for its application in machine learning \cite{liu2023genphys}.

%Nonetheless, the aforementioned approaches are applicable only to data generation problem and are not generalized to data-to-data tasks.


 


\begin{figure}[t]
\vskip 0.2in
\begin{center}
\centerline{\includegraphics[width=\columnwidth]{electroBridge5.pdf}}
\caption{Our Electrostatic field matching (EFM) method. Two data distributions $\mathbb{P}(\textbf{x}^{+})$ and $\mathbb{Q}(\textbf{x}^-),\; \textbf{x}^\pm\in\mathbb{R}^D$ are placed in the space $\mathbb{R}^{D+1}$ in the planes $z=0$ and $z=L$, respectively. The distribution $ \mathbb{P}(\textbf{x}^{+})$ is assigned a positive charge, and the distribution $\mathbb{Q}(\textbf{x}^-)$ a negative charge. These charges create an electric field ${\textbf{E}}(\widetilde{\textbf{x}})$, where $\widetilde{\textbf{x}} = (\textbf{x}, z)\in\mathbb{R}^{D+1}$. The lines of the field begin at positive charges and end at negative charges. Movement along the electric field lines provably (see our theorem \ref{main_theorem}) transforms the distribution $\mathbb{P}(\textbf{x}^{+})$ into the distribution $\mathbb{Q}(\textbf{x}^-)$.}
\label{electro_bridge1}
\vspace{-6 mm}
\end{center}
\vskip -0.2in
\end{figure}

More recently, modifications of DM have appeared that can learn diffusion in a data-to-data scenario. Diffusion Bridge Matching \citep[BM]{shi2024diffusion, albergo2023building, gushchin2024adversarial} is an SDE-based method that recovers the continuous-time Markovian process between data distributions. Flow matching \citep[FM]{lipmanflow,liu2022flow,klein2024equivariant,chen2024flow,xie2024reflected} is the limiting case of BM that learns ODE-based transformation between distributions.

%Slight modifications of DM are able to recover transformation for data-to-data tasks. Flow matching (FM) \cite{lipman2022flow,liu2022flow} is ODE-based approach that learns velocity as a dynamic transformation between data distributions. Bridge Matching (BM) \cite{shi2024diffusion, albergo2022building, gushchin2024adversarial} is SDE-based methodology for which FM is the limiting  case. 
 
However, there is no method based on electrostatic theory that can be applied to \textbf{data-to-data} translation tasks.

\textbf{Contributions.} We propose and theoretically justify a new paradigm for generative modeling called \textit{Electrostatic Field Matching} (EFM). It is based on the electrostatic theory and suitable for both noise-to-data and data-to-data generative scenarios. We provide proof-of-concept experiments on low- and high-dimensional generative modeling tasks.
 

 
%======================= Introduction =============================%

%=======================  BackGround =============================%
\section{Background and Related Works}
\label{background} 

\subsection{Basic physics}
\label{base_phys}

To understand the physics behind the electrostatic field matching method, let us give some basic background from standard Maxwell's 3D-electrostatics and then generalize it to the case of $D$ dimensions. Information on Maxwell's electrostatics can be found in any electricity textbook, for instance \citep[Chapter 5]{LandauLifshitz2}.

\subsubsection{Maxwell's electrostatics\footnote{ All formulas are written in the Heaviside–Lorentz system of units, where Planck's constant $\hbar = 1$, the speed of light $c = 1$, and the electric constant, which stands as a multiplier in Coulomb's law (see below), is $k = 1/(4\pi)$. This system of units is convenient for our purposes because it eliminates unnatural physical constants, and also because some formulas look particularly simple in this system of units  (the Gauss's theorem and the circulation theorem).}}

\textbf{The field and the potential of a point charge. } Let the point charge $q\in\mathbb{R}$ be located at the point $\textbf{x}'\in\mathbb{R}^3$. At the point $\textbf{x}\in\mathbb{R}^3$ it creates an electric field $\textbf{E}(\textbf{x})\in\mathbb{R}^3$ equal to:
\begin{equation}
    \textbf{E}(\textbf{x}) = \frac{1}{4\pi}\frac{q}{||\textbf{x} - \textbf{x}'||^3}(\textbf{x} - \textbf{x}').
    \label{3D_field}
\end{equation}



%It is known that two point charges $q_1$ and $q_2$ located at a distance $\textbf{r} = \textbf{r}_1 - \textbf{r}_2$ will interact with the Coulomb force:

%\begin{equation}
%    \textbf{F} = \frac{1}{4\pi}\frac{q_1q_2}{r^2}\textbf{n} = \frac{1}{4\pi}\frac{q_1q_2}{r^3}\textbf{r}
%    \label{3D_coulomb}
%\end{equation}

%Charges do not interact by themselves, but through the electric field. To determine the electric field strength at any point in space, one should place a test charge $q_0$ (i.e., a charge whose magnitude is small compared to all other charges in the system) at that point, measure the force $\textbf{F}$ acting on the electric field side, and then the strength:

%\begin{equation}
%    \textbf{E} = \frac{\textbf{F}}{q_0}
%    \label{3D_field_def}
%\end{equation}

%In particular, the field strength of a point charge can be calculated from (\ref{3D_coulomb}) if we take $q_1 = q, q_2 = q_0$:

Note that the electric field\footnote{The meaning of electric field is as follows. If a charge $q_0$ is placed in an electric field, then the force acting on $q_0$ equals to $\textbf{F} = q_0\textbf{E}$. Using Eq.  (\ref{3D_field}) we obtain Coulomb's law of interaction of point charges: $\textbf{F} = k\frac{qq_0}{||\textbf{x} - \textbf{x}'||^3}(\textbf{x} - \textbf{x}')$, where $k = \frac{1}{4\pi}$. } is a potential field, i.e., it can be expressed as a gradient of a scalar function $\varphi(\textbf{x})$:
\begin{equation}
    \textbf{E}(\textbf{x}) = - \nabla \varphi(\textbf{x}).
    \label{E_nabla_phi}
\end{equation}

The function $\varphi$ is called the electric field potential. From (\ref{3D_field}), and $\nabla\frac{1}{||\textbf{x}||} = - \frac{\textbf{x}}{||\textbf{x}||^3}$, we obtain that the potential of the point charge is equal:

\begin{equation}
    \varphi(\textbf{x}) = \frac{1}{4\pi}\frac{q}{||\textbf{x} - \textbf{x}'||}.
    \label{3D_potential}
\end{equation}
\textbf{The superposition principle. } If point charges $q_1, q_2, ..., q_N$ are located at points $\textbf{x}_1, \textbf{x}_2, ..., \textbf{x}_N$, they create independent fields $\textbf{E}_1(\textbf{x}), \textbf{E}_2(\textbf{x}), ..., \textbf{E}_N(\textbf{x}),$ and potentials $\varphi_1(\textbf{x}), \varphi_2(\textbf{x}), ..., \varphi_N(\textbf{x})$ at a given point $\textbf{x}\in \mathbb{R}^3$. All these charges together create the following field and potential:

\begin{equation}
    \label{3D_superposition_discrete}
    \begin{split}
        \textbf{E}(\textbf{x}) = \sum_{n = 1}^N\textbf{E}_n(\textbf{x}) = \sum_{n=1}^N \frac{q_n}{4\pi}\frac{(\textbf{x} - \textbf{x}_n)}{||\textbf{x} - \textbf{x}_n||^3}, \\
    \varphi (\textbf{x}) = \sum_{n = 1}^N \varphi_n(\textbf{x}) = \sum_{n = 1}^N \frac{1}{4\pi}\frac{q_n}{||\textbf{x} - \textbf{x}_n||}.
    \end{split}
\end{equation}

In the general case we are dealing with a continuously distributed charge $q(\textbf{x})$. Then the superposition principle can be written as: 

\begin{equation}
\label{3D_superposition}
\begin{split}
        \textbf{E}(\textbf{x}) = \int \frac{1}{4\pi}\frac{(\textbf{x} - \textbf{x}')}{||\textbf{x} - \textbf{x}'||^3}q(\textbf{x}')d\textbf{x}', \\
        \varphi(\textbf{x}) = \int \frac{1}{4\pi}\frac{1}{||\textbf{x} - \textbf{x}'||}q(\textbf{x}')d\textbf{x}'.
\end{split}  
\end{equation}

%Let us now consider the case of a continuously distributed charge. Such charges are described by the electric charge density $\rho = dq/dV$ given in the whole space. Let us break the charged body into infinitesimal charges $dq = \rho(\textbf{r}')dV' \equiv \rho(\textbf{r}')d^3\textbf{r}'$ located at the points $\textbf{r}'$. Then, because of (\ref{3D_superposition}) and (\ref{3D_potential}), the potential at the point $\textbf{r}$:

%\begin{equation}
%    \varphi(\textbf{r}) = \int \frac{1}{4\pi}\frac{\rho(\textbf{r}')d^3\textbf{r}'}{|\textbf{r} - \textbf{r}'|}
%\end{equation}

%The electric field can then be directly calculated by the formula (\ref{E_nabla_phi}), or via the superposition principle (see (\ref{3D_field}),(\ref{3D_superposition})):

%\begin{equation}
%    \textbf{E}(\textbf{r}) = \int \frac{1}{4\pi}\frac{\rho(\textbf{r}')}{|\textbf{r} - \textbf{r}'|^3}(\textbf{r} - \textbf{r}')d^3\textbf{r}'
%    \label{3D_field_continious}
%\end{equation}

%The key concept for our problem is electric field strength lines. 

Note that the charge distribution $q(\textbf{x})$ can have values greater than zero (positive charge) or less than zero (negative charge). 

    \textbf{An electric field strength line} is a curve $\textbf{x}(\tau)\in\mathbb{R}^3,\;\tau\in[a,b]\subset\mathbb{R}$ whose tangent to each point is parallel to the electric field at that point. In other words:
    \begin{equation}
        \frac{d\textbf{x}(\tau)}{d\tau} = \textbf{E}(\textbf{x}), \text{where }\tau\in[a,b]\subset\mathbb{R}.
        \label{3D_electric_line}
    \end{equation}


The electrostatic field satisfies two theorems which define the most important properties of these lines. 

\textbf{Gauss's theorem} \citep[\wasyparagraph 31]{LandauLifshitz2}. For any closed two-dimensional surface $\partial M$, which bounds the set $M\subset \mathbb{R}^3$ (see Fig. \ref{fig_gauss}), the electric field flux is equal to the total charge enclosed by this surface:
    \begin{equation}
        \iint_{\partial M} \textbf{E}\cdot\textbf{dS} = \int_M q(\textbf{x})d\textbf{x}.
        \label{3D_gauss}
    \end{equation}

\begin{figure}[ht]
\vskip 0.2in
\begin{center}
\centerline{\includegraphics[width=\columnwidth]{gauss.pdf}}
\caption{An illustration of the Gauss's theorem.}
\label{fig_gauss}
\end{center}
\vskip -0.2in
\end{figure}


In particular, it follows from Gauss's theorem that the electric field line must begin at a positive charge (or at infinity) and end at a negative charge (or at infinity). The lines cannot simply terminate in a space where there are no charges. %Eventually this will allow us to correctly move (without loss of information) from one set of pictures (which will be assigned by a positive charge) to another (which will be negatively “charged”). 

%(otherwise we could find a surface near the break point where the left side of the equation (\ref{3D_gauss}) would not be zero, but the right side would be zero)

\textbf{Theorem of electric field circulation}.\citep[\wasyparagraph 26]{LandauLifshitz2} For any closed loop $\ell$ (Fig. \ref{fig_circulation}) the electric field circulation is equal to zero:

    \begin{equation}
        \oint_{\ell} \textbf{E}\cdot \textbf{dl} = 0. 
        \label{3D_circ}
    \end{equation}
    
\begin{figure}[ht]
\vskip 0.2in
\begin{center}
\centerline{\includegraphics[width=\columnwidth]{circulation.pdf}}
\caption{An illustration of the electric field circulation theorem.}
\label{fig_circulation}
\end{center}
\vskip -0.2in
\end{figure}

It follows from the circulation theorem that there are no field lines which form closed loops. %If this were not the case, then when teaching $\textbf{E}$ to bridge between sets of pictures, there would be situations where the motion along the lines of intensity would occur in a circle and never end. The circulation theorem rules out this possibility. 



\subsubsection{$D$-dimensional electrostatics}
\label{ND_electrostatics}

The generalization of electrostatic equations for higher dimensions appears in discussions related to the influence of extra dimensions on the physics \cite{Ehrenfest17}, \cite{GUREVICH1971201}, \cite{caruso2023still}. The generalization modifies Eqs. (\ref{3D_gauss}) and (\ref{3D_circ}) by replacing $\mathbb{R}^3$ with $\mathbb{R}^D$ and replacing dimensionality 2 of $\partial M$ to $D-1$ in the Gauss's theorem. The definitions in Eqs. (\ref{E_nabla_phi}), (\ref{3D_electric_line}) and the superposition principle remain unchanged. The differences affect only the explicit expressions for the potential and the electric field.

The potential at the point $\textbf{x}\in \mathbb{R}^D$ of a point charge $q$, which is located at $\textbf{x}'\in\mathbb{R}^D$ equals to:

\begin{equation}
    \varphi(\textbf{x}) = \frac{1}{(D-2)S_{D-1}}\frac{q}{||\textbf{x} - \textbf{x}'||^{D-2}},
    \label{ND_potential}
\end{equation}
where $S_{D-1}$ is the surface area of an $(D-1)$-dimensional sphere with radius $1$. Then the field of the point charge is:
\begin{equation}
    \textbf{E}(\textbf{x}) = -\nabla\varphi(\textbf{x}) = \frac{q}{S_{D-1}}\frac{\textbf{x} - \textbf{x}'}{||\textbf{x} - \textbf{x}'||^{D}}
    \label{ND_field}.
\end{equation}
The field of a distributed charge $q(\textbf{x})$ can be obtained by the principle of superposition as in Eq. (\ref{3D_superposition}) for 3D case:

\begin{equation}
    \textbf{E}(\textbf{x}) = \int \frac{1}{S_{D-1}}\frac{\textbf{x} - \textbf{x}'}{||\textbf{x} - \textbf{x}'||^{D}}q(\textbf{x}') d\textbf{x}'.
    \label{ND_field_continious}
\end{equation}

Together Gauss's theorem and the circulation theorem in a $D$-dimensional space ensure the following \textbf{principal characteristics} of electric field lines: 

\textbf{(i)} Electric field lines cannot terminate in points where there are no charges;

\textbf{(ii)} For a system having zero total charge ($\int q(\textbf{x})d\textbf{x} = 0$), electric field lines almost surely start at positive charge and end at negative charge; 

\textbf{(iii)} There are no electric field lines that form closed loops.

For convenience of the reader, these properties are proven in Appendix \ref{ND_lines}

%=======================  BackGround =============================%




%====================== PFGM =========================%
    
\subsection{Poisson Flow Generative Model (PFGM)}
The first attempt to couple electrostatic theory and generative modeling is proposed by \cite{xu2022poissonflowgenerativemodels,xu2023pfgm++}. The authors work with a $D$-dimensional data distribution.  %$\textbf{x}_{1},...,\textbf{x}_{N}: \forall i \mapsto \textbf{x}_{i} \in \mathbb{R}^{D}$ from original data distribution $\mathbb{P}(\textbf{x})$. They augment samples with spatial coordinate $z=0$ and place them in the hyperplane $z=0$ in $D+1$-dimensional space. These supplemented data points are indicated as
They embed this distribution into ${(D+1)}$-dimensional space by applying the transformation
$\textbf{x} \to \tilde{\textbf{x}} = (\textbf{x},0) $ , i.e., place the data  $\textbf{x}$ on a hyperplane $z=0$ in ${(D+1)}$-dimensional space. The data distribution is interpreted then as a positive electrostatic charge distribution.

%\begin{equation}
%\tilde{\textbf{x}} :=  \{ (\textbf{x}_{1},z), ...., %(\textbf{x}_{N},z) \}. \vspace{-1 mm}  
%\end{equation}


 
 
%point charges and describe them by distribution $\mathbb{P}_{0}(\tilde{\textbf{x}}):=\mathbb{P}(\textbf{x})\delta(z)$. This distribution is connected to flux through the hyperplane.





\begin{figure*}[!h]
\centering

\begin{subfigure}[b]{0.44\linewidth}
\centering
\includegraphics[width=0.995\linewidth]{PFGM1.pdf}
\caption{\centering Near the plate.}
\label{fig:PFGM1}
\end{subfigure}
\begin{subfigure}[b]{0.442\linewidth}
\centering
\includegraphics[width=0.995\linewidth]{PFGM2.pdf}
\caption{\centering Away from the plate.}
\label{fig:PFGM2}
\end{subfigure}

\caption{\centering\small  PFGM concept. The original data have a distribution $\mathbb{P}_0(\widetilde{\textbf{x}})$, which is assigned a positive charge that produces an electric field $\textbf{E}(\widetilde{\textbf{x}})$. Near the plate (Fig. \ref{fig:PFGM1}), the field lines can have a complex structure, while away from the plate (Fig. \ref{fig:PFGM2}) the charge looks like a point, and therefore the electric field is uniformly distributed: $\mathbb{P}_{\infty}(\widetilde{\textbf{x}}) = \mathcal{U}(\mathcal{S}^+_{\infty})$}
\vspace{-2mm}
\label{fig:translation}

\end{figure*}



 %\begin{figure}[ht]
%\vskip 0.2in
%\begin{center}
%\centerline{\includegraphics[width=\columnwidth]{PFGM.pdf}}
%\caption{PFGM concept. The original data have a distribution $\mathbb{P}_0(\widetilde{\textbf{x}})$, which is assigned a positive charge that produces an electric field $\widetilde{\textbf{E}}(\widetilde{\textbf{x}})$. Near the plate (top figure), the field lines can have a complex structure, while away from the plate (bottom figure) the charge looks like a point, and therefore the electric field is uniformly distributed: $\mathbb{P}_{\infty}(\widetilde{\textbf{x}}) = \mathcal{U}(\mathcal{S}_{\infty})$}
%\label{pfgm}
%\end{center}
%\vskip -0.2in
%\end{figure}
 % \citep[\wasyparagraph 40-41]{LandauLifshitz2}
 
The intuition of the method is that the charged points $\tilde{\textbf{x}}$ in the hyperplane $z=0$ generate the electric field $ \textbf{E}(\cdot)$ which behaves at infinity  as the field of a point charge. If a point charge is placed inside a sphere $\mathcal{S}_{\infty}$ with an infinite radius, then the flux density $\mathbb{P}_{\infty}(\cdot)$ through the surface of the sphere is distributed uniformly. For the simplicity the authors consider the hemisphere  $\mathcal{S}^{+}_{\infty}$ (Fig.~\ref{fig:PFGM2}). Then:
\begin{equation}
\mathbb{P}_{\infty}(\cdot) = \mathcal{U}(\mathcal{S}_{\infty}^{+})\vspace{-2mm}.  
\end{equation}

Thus, the electric field lines define the correspondence between uniformly distributed charges on the surface of the hemisphere $\mathcal{S}^{+}_{\infty}$ and the original data samples from $\mathbb{P}_{0}(\tilde{\textbf{x}})$  in the hyperplane $z=0$. 

If a massless point charge is placed in the electric field $ \textbf{E}(\cdot)$ with field lines directed from $\mathbb{P}_{0}(\cdot)$ to $\mathbb{P}_{\infty}(\cdot)$, then the charge moves along the lines to $\mathbb{P}_{\infty}(\cdot)$. This movement transforms the data samples from the complex distribution to the simple distribution on the hemisphere along the field lines. The corresponding inverse transformation generates the data samples from uniformly distributed samples on the hemisphere. The inverse map is a movement along these field lines in the backward direction and is defined by the following ODE with electric field $ -\textbf{E}(\cdot)$.
\begin{equation}
\label{main_odde}
\frac{\widetilde{\textbf{x}}(t)}{dt} = - \textbf{E}(\widetilde{\textbf{x}}).  
\end{equation}
To recover the electric field $\bf{E}$$(\cdot)$ in the extended ${(D+1)}$-dimensional space, the authors propose to approximate it with a neural network $f_{\theta}(\cdot) : \mathbb{R}^{D+1} \to \mathbb{R}^{D+1}$. 

First, they compute the ground truth electric field $\textbf{E}(\widetilde{\textbf{x}})$ empirically at a set of arbitrary ${(D+1)}$-dimensional points $\widetilde{\textbf{x}}$ inside the hemisphere $\mathcal{S}^{+}_{\infty}$ through samples from $\mathbb{P}_{0}(\cdot)$ according to Eq. (\ref{ND_field_continious}). Second, the electric field is learned at $\widetilde{\textbf{x}}$ by minimizing the difference between the predicted $f_{\theta}(\widetilde{\textbf{x}})$ and the ground-truth $\textbf{E}(\widetilde{\textbf{x}})$. Having learned the electric field $\textbf{E}(\cdot)$ in the ${(D+1)}$-dimensional space, they simulate the ODE (\ref{main_odde}) with initial samples from $\mathbb{P}_{\infty}(\cdot)$ until the spatial  coordinate $z$ reaches 0. Finally, they get samples  $\widetilde{\textbf{x}}_{T}\sim \mathbb{P}_{0}(\cdot)$, where $T$ is the end time of the ODE simulation.
 
%====================== PFGM =========================%


\section{Electrostatic Field Matching (EFM)}
\label{main} 

This section introduces Electrostatic Field Matching (EFM), a novel generative modeling paradigm applicable to both noise-to-data and data-to-data generation grounded in electrostatic theory.  The \wasyparagraph\ref{EFM_into} gives an intuitive description of the method. The \wasyparagraph\ref{EFM_theorem} gives a theoretical formulation of the method and the main theorem. In the \wasyparagraph\ref{prac_implementation}, the learning and inference algorithms are formulated. 


\subsection{Intuitive explanation of the method}
\label{EFM_into}



%The data generation either can work from randomly distributed vectors (noise) or by a direct transport from one distribution to another. A method for generating data from noise using an electric field was suggested in \cite{xu2022poissonflowgenerativemodels}. We propose to utilize electric field lines for a direct mapping between distributions. 

Our idea is to consider distributions as electric charge densities. One could assign positive charge values to the first distribution and negative charges to the second one, i.e., the charge density follows the distributions up to a sign. We then place these distributions on two $D$-dimensional planes at  distance $L$ from each other (Fig. \ref{electro_bridge1}). This will produce an electric field with lines starting at one density and finishing at another. We prove theorem \ref{main_theorem} that movement along the lines allows one to make a transition from one distribution to another almost surely.


\subsection{Formal theoretical justification}
\label{EFM_theorem}

Let $\mathbb{P}(\textbf{x}^{+})$ and $\mathbb{Q}(\textbf{x}^-), \textbf{x}^\pm\in\mathbb{R}^D$ be two data distributions. Let us assign to the first distribution a positive charge $q^{+}(\textbf{x}^{+}) = \mathbb{P}(\textbf{x}^{+})$, and to the second distribution a negative charge $q^{-}(\textbf{x}^{-}) = -\mathbb{Q}(\textbf{x}^{-})$. Note that the charge distributions are normalized to $\int q^{+}(\textbf{x}^{+}) d\textbf{x}^+ = 1, \; \int q^-(\textbf{x}^-)d\textbf{x}^- = -1$ and therefore the total charge of the system is zero. 

Let us now place these distributions in $(D+1)$-dimensional space. The new point in this space can be written as:
\begin{equation}
    (x_1, x_2,...,x_D,z) = (\textbf{x}, z) = \widetilde{\textbf{x}} \in \mathbb{R}^{D+1}.
\end{equation}
 
 We place $q^+(\textbf{x}^+)$ in the hyperplane $z=0$, and $q^-(\textbf{x}^-)$ in the hyperplane $z = L$(Fig. \ref{electro_bridge1}). One can think of it as a $(D+1)$-dimensional \textbf{capacitor}. The distributions would be written as:
 \begin{equation}
 \begin{split}
     q^+(\widetilde{\textbf{x}}) = q^+(\textbf{x}, z) = q^+(\textbf{x})\delta(z),\;\;\;\;\;\;\;\\ 
     q^-(\widetilde{\textbf{x}}) = q^-(\textbf{x}, z) = q^-(\textbf{x})\delta(z-L),
 \end{split}
 \end{equation}
where $\delta(\cdot)$ denotes Dirac delta function. 

%Let $M\subset \mathbb{R}^\mathcal{D}$ be the first dataset with distribution $P(\textbf{x}^{(1)}),\textbf{x}^{(1)}\in M $, and $R\subset \mathbb{R}^\mathcal{D}$  be the second dataset with distribution $Q(\textbf{x}^{(2)}), \textbf{x}^{(2)}\in R $. We assign $M$ a positive charge with density $\rho_{+}(\textbf{x}^{(1)})) = P(\textbf{x}^{(1)})$, and $R$ a negative charge with density $\rho_{-}(\textbf{x}^{(2)})) = - Q(\textbf{x}^{(2)})$.
%Let $M\subset \mathbb{R}^\mathcal{D}$ be the manifold describing the first set of pictures, $R\subset \mathbb{R}^\mathcal{D}$ the second. Let $P(\textbf{x})\geq 0, \;\textbf{x}\in M$ define the distribution of pictures on $M$. We also interpret $P(\textbf{x})$ as the density of positive “charge” normalized to 1: $\int_M P(\textbf{x})d\textbf{x} = 1$. Similarly, let $Q(\textbf{y})\leq 0,\;\textbf{y}\in R$ be the density distribution of the negative “charge”\; on $R$.  $\int_R Q(\textbf{y})d\textbf{y} = -1$. 

The electric field produced at the point $\widetilde{\textbf{x}}\in \mathbb{R}^{D+1}$ between plates will consist of two summands:
\begin{equation}
    \textbf{E}(\widetilde{\textbf{x}}) = \textbf{E}_{+}(\widetilde{\textbf{x}}) + \textbf{E}_{-}(\widetilde{\textbf{x}}),
    \label{main_field_bridge}
\end{equation}
where $\textbf{E}_{+}(\textbf{x})$ and $\textbf{E}_{-}(\widetilde{\textbf{x}})$ are the fields defined by the $q^+(\widetilde{\textbf{x}}^+)$ and $q^-(\widetilde{\textbf{x}}^-)$, respectively.  The exact expression for these fields is given by Eq. (\ref{ND_field_continious}) with replacement of $D$ by $D+1$:
\begin{equation}
        \textbf{E}_\pm(\widetilde{\textbf{x}}) = \int \frac{1}{S_{D}}\frac{\widetilde{\textbf{x}} - \widetilde{\textbf{x}}'}{||\widetilde{\textbf{x}} - \widetilde{\textbf{x}}'||^{D+1}}q^\pm(\widetilde{\textbf{x}}') d\widetilde{\textbf{x}}'. 
\end{equation}

Finally, let us define the map between the distributions $T: \text{supp}(\mathbb{P}(\textbf{x}^+))\rightarrow \text{supp}(\mathbb{Q}(\textbf{x}^-))$ using electric field lines. Consider a point $\widetilde{\textbf{x}}^+ = (\textbf{x}^+,0)$ in the support of the first distribution. Let us denote the field line $\widetilde{\textbf{x}}(\tau), (\tau\in[a,b])$ starting at this point. From the properties of electric field lines formulated in Section \ref{ND_electrostatics}, it must end almost surely at the point $\widetilde{\textbf{x}}^- = (\textbf{x}^-,L)$ in the support of the second distribution $q^-(\widetilde{\textbf{x}}^-)$ with negative charge. Thus, $\widetilde{\textbf{x}}(a) = (\textbf{x}^+,0)= \widetilde{\textbf{x}}^+,\;\;\widetilde{\textbf{x}}(b)= (\textbf{x}^-,L) = \widetilde{\textbf{x}}^-$. Using this, we define $T(\textbf{x}^+) = \textbf{x}^-$.



Then, for this map we prove the following theorem:



\begin{theorem}[\textbf{Electrostatic Field Matching}]
\label{main_theorem}
    Let $\textbf{x}^+$ is distributed over $\mathbb{P}(\textbf{x}^+)$. Then $\textbf{x}^-=T(\textbf{x}^+)$ is distributed over $\mathbb{Q}(\textbf{x}^-)$ almost surely:
    \begin{equation}
        \text{If}\;\; \textbf{x}^+\sim \mathbb{P}(\textbf{x}^+) \Rightarrow T(\textbf{x}^+) = \textbf{x}^-\sim \mathbb{Q}(\textbf{x}^-).
    \end{equation}
\end{theorem}


%\begin{theorem}
%    \label{main_theorem}
%    Let $\{\textbf{x}_i\}_{i=1}^n$ be a set of points from manifold $M$ distributed over $P(\textbf{x})$. We denote the electric field strength lines starting at these points as $\{\textbf{r}_i(\tau)\}_{i= 1}^n, \tau\in[0,T]$. By the properties of field lines (see appendix \ref{ND_lines}) they must end on the manifold $R$, that is:
%    \begin{equation}
%        \frac{d\textbf{r}_i}{d\tau} = \textbf{E}(\textbf{r}_i), \; \textbf{r}_i(0) = \textbf{x}_i\in M,\;\textbf{r}_i(T) = \textbf{y}_i\in R
%    \end{equation}
%    where $\textbf{E}(\textbf{r})$ is the field between manifolds $M$ and $R$ (see (\ref{main_field_bridge})). 
%    
%    Then the set $\{\textbf{y}_i\}_{i = 1}^n$ forms a distribution $\widetilde{Q}_n(\textbf{y})$ on manifold $R$, which converges to the true distribution $Q(\textbf{y})$ with probability 1 (almost surely):
%    \begin{equation}
%        \mathbb{P}\big(\lim_{n\rightarrow \infty} \widetilde{Q}_n = Q\big) = 1
%    \end{equation}
%\end{theorem}

In other words, the map $T$ defined by electric field lines does transfer $\mathbb{P}(\textbf{x}^+)$ into $\mathbb{Q}(\textbf{x}^-)$ indeed, as we had intended.
%bridge built between manifolds $M$ and $R$ using electric field lines does indeed translate the distribution $P(\textbf{x})$ into $Q(\textbf{y})$.

The proof of the theorem is given in Appendix \ref{proof_main_theorem}. 


%================= Prac Implementation ==================%
\subsection{Learning and Inference Algorithm}
\label{prac_implementation}

We consider samples $\textbf{X}^{+} = \{\textbf{x}^{+}_{1},...,\textbf{x}^{+}_{N}\}$ and $\textbf{X}^{-} = \{\textbf{x}^{-}_{1},...,\textbf{x}^{-}_{M}\}$ distributed by $\mathbb{P}(\textbf{x}^{+})$ and $\mathbb{Q}(\textbf{x}^{-})$, respectively. We then extend the space to $(D+1)$ by placing the first sample at $z=0$ and the second sample at $z=L$. Thus $\textbf{X}^{+} \rightarrow\widetilde{\textbf{X}}^{+} = \{(\textbf{x}^{+}_{1},0),...,(\textbf{x}^{+}_{N},0)\} = \{\widetilde{\textbf{x}}^+_1,...,\widetilde{\textbf{x}}^+_N\}$ and $\textbf{X}^{-} \rightarrow\widetilde{\textbf{X}}^- = \{(\textbf{x}^{-}_{1},L),...,(\textbf{x}^{-}_{M},L)\}= \{\widetilde{\textbf{x}}^-_1,...,\widetilde{\textbf{x}}^-_M\}$

%We consider distributions $\mathbb{P}(\textbf{x}^{+})$ and $\mathbb{Q}(\textbf{x}^{-})$ accessible by samples. $D$-dimensional samples $\textbf{X}^{+} = \{\textbf{x}^{+}_{1},...,\textbf{x}^{+}_{N}\}$ and $\textbf{X}^{-} = \{\textbf{x}^{-}_{1},...,\textbf{x}^{-}_{N}\}$ are extended with spatial coordinates $z=0$ and $z=L$ respectively. That is, the extended samples $\widetilde{\textbf{X}}^{+} = \{(\textbf{x}^{+}_{1},0),...,(\textbf{x}^{+}_{N},0\}$ and $\widetilde{\textbf{X}}^{-} = \{(\textbf{x}^{-}_{1},L),...,(\textbf{x}^{-}_{N},L) \}$ are placed in $D$-dimensional hyperplanes located at the distance $L$. 
% These $D$-dimensional hyperplanes are situated in  an extended $D+1$-dimensional space. 

\textbf{Training.} To recover the electric field $\textbf{E}(\cdot)$ in ${(D+1)}$-dimensional points between the hyperplanes, we approximate it with a neural network $f_{\theta}(\cdot): \mathbb{R}^{D+1} \to \mathbb{R}^{D+1}$. We sample the value $t$  from the uniform distribution $\mathcal{U}(0,L)$ and take two random samples $\widetilde{\textbf{x}}^{+}$ and $\widetilde{\textbf{x}}^{-}$. Then, we get a new point $\widetilde{\textbf{x}}$ between the planes as follows:
\begin{equation}
\label{middle_point_sample}
    \widetilde{\textbf{x}} = t\widetilde{\textbf{x}}^{+} + (1-t)\widetilde{\textbf{x}}^{-} + \widetilde{\epsilon},
\end{equation} 
where random $\widetilde{\epsilon}$ is sampled by using the following scheme. Sampling the noise $\epsilon$ from $\mathcal{N}(0, \sigma^{2}I_{D+1\times D+1})$ and calculating the Euclidean norm $||\epsilon||_{2}$, we multiply this norm by a unit Gaussian vector and get $\widetilde{\epsilon}$:
\begin{equation}
\label{noise-gauss}
 \widetilde{\epsilon} = ||\epsilon||_{2}\frac{m}{||m||_{2}}, \quad m \sim \mathcal{N}(0,I_{D+1\times D+1}).
 \end{equation}

The ground-truth $\textbf{E}({\widetilde{\textbf{x}}})$ is estimated with Eq. (\ref{main_field_bridge}). Specifically, the integral is approximated via Monte Carlo sampling of Eq. (\ref{ND_field_continious}) by using samples from $\mathbb{P}(\textbf{x}^{+})$ and $\mathbb{Q}(\textbf{x}^{-})$. Then we use a neural network approximation $f_{\theta}( {\widetilde{\textbf{x}}})$ to learn the ground-truth electric field $ \textbf{E}({\widetilde{\textbf{x}}})$ at points from the extended ${(D+1)}$-dimensional space.
%We consider the electric field $\widetilde{\textbf{E}}_{-}(\cdot)$ with negative sign in (\ref{main_field_bridge}).
% We use a neural network approximation $f_{\theta}( {\widetilde{\textbf{x}}})$ to estimate the ground-truth electric field $ \textbf{E}({\widetilde{\textbf{x}}})$ in points from the extended ${(D+1)}$-dimensional space. 
We learn  $f_{\theta}( \cdot)$ by minimising the squared error difference between the ground truth  $\textbf{E}({\widetilde{\textbf{x}}})$  and  the predictions  $f_{\theta}({\widetilde{\textbf{x}}})$  over the parameters of the neural network  with SGD, i.e., the learning objective is
\begin{equation}
  \mathbb{E}_{\widetilde{\textbf{x}}}|| f_{\theta}({\widetilde{\textbf{x}}}) -  \textbf{E}({\widetilde{\textbf{x}}}) ||^{2}_{2} \to  \min_{\theta}.  
\end{equation}

\textbf{Inference.} Having learned the vector field $\textbf{E}(\cdot)$ in the extended space with $f_{\theta}(\cdot)$, we simulate the movement between hyperplanes to transfer data from $\mathbb{P}(\textbf{x}^{+})$ to $\mathbb{Q}(\textbf{x}^{-})$. For this, we run an ODE solver for Eq. \eqref{main_odde}.
 
One needs a right stopping time for the ODE solver. In order to find it, we follow the idea of \cite{xu2022poissonflowgenerativemodels} and use an equivalent ODE solver with $\widetilde{\textbf{x}}$ evolving with the extended variable z:
\begin{equation}
\label{augment_ode}
    d\widetilde{\textbf{x}} = d(\textbf{x},z) = \big(\frac{d\textbf{x}}{dt}\frac{dt}{dz}dz,dz\big) = (\textbf{E}_{x}(\widetilde{\textbf{x}})\textbf{E}_{z}^{-1}(\widetilde{\textbf{x}}),1)dz,
\end{equation}
where we denote $\textbf{E}(\widetilde{\textbf{x}})=(\textbf{E}_{x}(\widetilde{\textbf{x}}),\textbf{E}_{z}(\widetilde{\textbf{x}}))$. In the new ODE (\ref{augment_ode}), we replace the time variable $t$ with the physically meaningful variable $z$, allowing explicit start ($z=0$) and end  ($z=L$) conditions. We start with samples from $\mathbb{Q}(\textbf{x}^{-})$, i.e., when $z=L$. Then, we arrive at the data distribution $\mathbb{P}(\textbf{x}^{+})$ when $z$ reaches 0 during the ODE simulation.

% We can freely choose a large distance between planes $L$ as the starting point in the ODE (\ref{augment_ode}). Thus, having run ODE (\ref{augment_ode}), we transfer samples from one data distribution to another.

All the ingredients for training and inference in our method are described in Algorithm \ref{algorithm:EFM}, where we summarize the learning and inference procedures. 
% \linebreak

\begin{algorithm}[h!]
         
        \textbf{Input:} Distributions accessible by samples:\\
         \hspace*{5mm} $\mathbb{P}(\textbf{x}^{+})\delta(z)$
          and $\mathbb{Q}(\textbf{x}^{-})\delta(z-L)$;\\ 
        \hspace*{5mm} NN approximator $f_{\theta}(\cdot) :\mathbb{R}^{D+1} \to \mathbb{R}^{D+1}$; \\
 
        { \textbf{Repeat until  converged :} }{
        
            \hspace*{5mm}Sample batch $\widetilde{\textbf{x}}^{+} \sim \mathbb{P}(\textbf{x}^{+})\delta(z)$;\\
            \hspace*{5mm}Sample batch $\widetilde{\textbf{x}}^{-} \sim  \mathbb{Q}(\textbf{x}^{-})\delta(z-L)$;\\
            \hspace*{5mm}Sample $t \sim \mathcal{U}(0,L)$;\\
            \hspace*{5mm}Compute noise $\widetilde{\epsilon}$ by (\ref{noise-gauss});\\
            \hspace*{5mm}Calculate batch $\widetilde{\textbf{x}} = t\widetilde{\textbf{x}}^{+} + (1-t)\widetilde{\textbf{x}}^{-} + \widetilde{\epsilon}$;\\
             \hspace*{5mm}Calculate $\textbf{E}_{+}(\widetilde{\textbf{x}})$  and $\textbf{E}_{-}(\widetilde{\textbf{x}})$ through (\ref{ND_field_continious});\\
             \hspace*{5mm}Calculate $\textbf{E}({\widetilde{\textbf{x}}})$ with (\ref{main_field_bridge});\\
            \hspace*{5mm}Compute $\mathcal{L} = \mathbb{E}_{\widetilde{\textbf{x}}}|| f_{\theta}({\widetilde{\textbf{x}}}) - \textbf{E}({\widetilde{\textbf{x}}}) ||^{2}_{2} \to  \min_{\theta}$;\\
            \hspace*{5mm}Update $\theta$ by using $\frac{\partial \mathcal{L} }{\partial \theta}$;\;
            
           
        }
        \caption{Electrostatic field matching}
        \label{algorithm:EFM}
\end{algorithm}



%======================= Exps ==================%
\section{Experimental Illustrations}

\begin{figure*}[!h]
\begin{subfigure}[b]{0.247\linewidth}
\centering
\includegraphics[width=0.995\linewidth]{pics/input.png}
\caption{\centering\scriptsize Samples from  $\mathbb{P}(\textbf{x}^{+})$, which are placed on the left hyperplane $z=0$ .}
\label{fig:3d_data_init}
%\vspace{2.8mm}
\end{subfigure}
\begin{subfigure}[b]{0.232\linewidth}
\centering
\includegraphics[width=0.995\linewidth]{pics/output.png}
\caption{\centering\scriptsize Samples from  $\mathbb{Q}(\textbf{x}^{-})$, which are placed on the right hyperplane $z=L$.}
\label{fig:3d_data_target}
\end{subfigure}
\begin{subfigure}[b]{0.232\linewidth}
\includegraphics[width=0.995\linewidth]{pics/mapped.png}
\caption{\centering\scriptsize Mapped samples by $T(\textbf{x}^{+})$ for the distance $L=6$. }
\label{fig:3d_data_mapped}
\end{subfigure}
\begin{subfigure}[b]{0.232\linewidth}
\centering
\includegraphics[width=0.995\linewidth]{pics/mapped_long.png}
\caption{\centering\scriptsize  Mapped samples for $T(\textbf{x}^{+})$
for the distance $L=30$. }
\label{fig:3d_mapped_long}
%\vspace{3mm}
\end{subfigure}
\caption{\centering \textit{Illustrative 2D Gaussian$\rightarrow$Swiss Roll experiment}: input and target distributions $\mathbb{P}(\textbf{x}^{+})$ and $\mathbb{Q}(\textbf{x}^{-})$ together with the result of the distribution transfer learned with our EFM method  for distances $L=6$ and $L=30$ between the capacitor plates. }
\label{fig:toy}
\end{figure*}



\label{experiments} 
In this section, we demonstrate the proof-of-concept experiments with our proposed EFM method. We show a 2-dimensional illustrative experiment (\S\ref{3d_exp}), image-to-image translation experiment (\S\ref{transfer_exp}) and image generation experiment (\S\ref{generation_exp}) with the colored MNIST dataset.  We describe details of the aforementioned experiments in Appendix \ref{app:experiments_details}.

\subsection{Gaussian to Swiss Roll Experiment}
\label{3d_exp}
An intuitive first test to validate the method is to transfer between distributions whose densities can be visualized for comparison.
We consider the 2-dimensional zero-centered Gaussian distribution with the identity covariance matrix as $\mathbb{P}(\textbf{x}^{+})$ and the Swiss Roll distribution as $\mathbb{Q}(\textbf{x}^{-})$, see their visualizations in Figs. \ref{fig:3d_data_init} and \ref{fig:3d_data_target}, respectively.

To show the effect of hyperparameter $L$ in our EFM method, we do two experiments. In the first one, the samples from $\mathbb{Q}(\textbf{x}^{-})$ are placed on the hyperplane $L=6$ (see Fig. \ref{fig:3d_data_mapped}), while in the second one, we use $L=30$ (see Fig. \ref{fig:3d_mapped_long}). We show the learned trajectories of samples' movement along the electrostatic field in Figs. \ref{fig:3d_close} and \ref{fig:3d_long}, respectively.  

When $L$ is small, the electric field lines are rather straight, see Fig. \ref{fig:3d_close}. The learned electric field $f_{\theta}(\cdot)\approx\textbf{E}(\cdot)$ allows one to accurately perform the distribution transfer, see Fig. \ref{fig:3d_data_mapped}. 

When the distance $L$ between the hyperplanes is large, the learned map $T(\textbf{x}^{+})$ recovers the target density $\mathbb{Q}(\textbf{x}^{-})$ poorly (see Fig. \ref{fig:3d_mapped_long}). Presumably, this is because it is more difficult to perfectly recover the electrical field $\textbf{E}(\cdot)$ by a neural network between plates with a large distance $L$. 
\vspace{-1mm}
\begin{figure}[!hb]
\begin{center}
\centerline{\includegraphics[width=\columnwidth]{pics/traj_1_new.png}}
\vspace{-3mm}
\caption{\centering The sampling trajectories of our \textbf{EFM} method in image-to-image translation experiment, see \S\ref{transfer_exp}.}
\label{fig:traj_1}
\end{center}
\vspace{-8mm}
\end{figure}
% We also observe the sampling curves $\textbf{x}(\tau)$, whose tangent at each point is parallel to the electric field. We see that these trajectories have a high curvature and their tube has a high dispersion. 

% Thus, our approach successfully recovers  $\mathbb{Q}(\textbf{x}^{-})$ (see Fig.\ref{fig:3d_data_mapped}), generating samples from the the distribution on the right hyperplane. Since our setup is symmetric, we demonstrate mapping from the left to the right hyperplane. However, the ODE simulation is applicable in both directions.
 

%Instead of training a neural network, we construct the grid in the space between hyperplanes. We empirically define the electric field $\textbf{E}(\cdot)$ at each point of the grid  by (\ref{ND_field_continious}) and use it in the ODE simulation (\ref{main_odde}).
 


\begin{figure}[h]
\begin{subfigure}[b]{0.15\textwidth}
\centering\includegraphics[width=1.6\linewidth]{pics/3d_close.png}
\caption{\centering \scriptsize $L=6$} 
\label{fig:3d_close}
\end{subfigure}
\hspace{19mm}
\begin{subfigure}[b]{0.2\linewidth}
\centering\includegraphics[width=2.5\linewidth]{pics/3d_long.png}
\caption{\centering \scriptsize $L=30$}
\label{fig:3d_long}
\end{subfigure}
\caption{Electric field line structure for the Gaussian$\rightarrow$Swiss Roll experiment with $L=6$ and $L=30$. It can be seen that at large
distances, the field lines are more curved than at small distances.}
\label{ris:image1}
\end{figure}


%======================= Toy Exps =======================%

 


\begin{figure}
\begin{center}
\vspace{-4mm}
\centerline{\includegraphics[width=\columnwidth]{pics/traj_gener_new.png}}
\caption{\centering The Sampling trajectories of our \textbf{EFM} method in noise-to-image translation experiment, see \S\ref{generation_exp}.}
\label{fig:traj_gener}
\end{center}
\end{figure} 
\begin{figure*}
\begin{subfigure}{0.33\linewidth}
\includegraphics[width=0.995\linewidth]{pics/cm_long_init.png}
\caption{\centering\scriptsize \centering\scriptsize Samples from  $\mathbb{P}(\textbf{x}^{+})$, which are placed on the left plate $z=0$.}
\label{fig:translation-init}
\end{subfigure}
\begin{subfigure}{0.33\linewidth}
\includegraphics[width=0.995\linewidth]{pics/cm_long_our.png}
\caption{\centering\scriptsize \centering\scriptsize Samples from \textbf{our} approximation of $\mathbb{Q}(\textbf{x}^{-})$, located on the right plate $z=10$.}
\label{fig:translation-target}
\end{subfigure}
\begin{subfigure}{0.33\linewidth}
\includegraphics[width=0.995\linewidth]{pics/cm_long_fm.png}
\caption{\centering\scriptsize  Samples from \textbf{FM}'s approximation of $\mathbb{Q}(\textbf{x}^{-})$, located on the right plate $z=10$. }
\label{fig:translation-target_}
\end{subfigure}
\caption{\centering \small \textit{Image-to-Image translation}. Pictures from the initial distribution, the result of applying our EFM method as well as the Flow Matching method are presented.}
\vspace{-2mm}
\label{fig:translation2}
\end{figure*}
%Pictures from the initial distribution (Fig. \ref{fig:translation-init}), the result of applying our EFM method (Fig. \ref{fig:translation-target}) as\\ well as the Flow Matching (Fig. \ref{fig:translation-target_}) method are presented.

\vspace{2mm}\begin{figure*}[h]
\begin{subfigure}[b]{0.33\linewidth}
\centering
\includegraphics[width=0.995\linewidth]{pics/noise.png}
\caption{\centering\scriptsize White noise samples from  $\mathbb{P}(\textbf{x}^{+})$, which are placed on the left plate $z=0$.}
\label{fig:generation-init}
\end{subfigure}
\begin{subfigure}[b]{0.33\linewidth}
\centering
\includegraphics[width=0.995\linewidth]{pics/efm.png}
\caption{\centering\scriptsize Samples from \textbf{our} approximation of $\mathbb{Q}(\textbf{x}^{-})$, located on the right plate $z=30$.}
\label{fig:generation-our}
\end{subfigure}
\begin{subfigure}[b]{0.33\linewidth}
\centering
\includegraphics[width=0.995\linewidth]{pics/pfgm.png}
\caption{\centering\scriptsize Samples from \textbf{PFGM}'s approximation of $\mathbb{Q}(\textbf{x}^{-})$, simulated from hemisphere with the learned field.}
\label{fig:generation-pfgm}
\end{subfigure}
\caption{\centering\small\textit{Noise-to-Image generation}. Pictures from the initial distribution (Fig. \ref{fig:generation-init}), the result of our EFM method (Fig. \ref{fig:generation-our}) as well as the PFGM method (Fig. \ref{fig:generation-pfgm}) are presented.}
\vspace{-2mm}
\label{fig:translation1}
\end{figure*}
%Noise-to-Image generation}. Pictures from the initial distribution (Fig. \ref{fig:generation-init}), the result of our EFM method (Fig. \ref{fig:generation-our})as well as the PFGM method (Fig. \ref{fig:generation-pfgm}) are presented
%=========================== Img-to-Img experiment =============%
\subsection{Image-to-Image Translation Experiment}
\label{transfer_exp}
Here we consider the image-to-image translation task for transforming colored digits 3 to colored digits 2 \citep[\wasyparagraph 5.3]{gushchin2024entropic}. The data is based on the conventional $32 \times 32$ MNIST images dataset but the digits are randomly colored. We consider $\textit{unpaired}$ translation task, i.e., there is no pre-defined correspondence between digits. In other words, one colored digit 3 can be mapped to many possible digits 2 and vice versa.
 
We place colored digits 3 on the left hyperplane $z=0$ and colored digits 2 on the right plate $z=10$. We learn the electric field $\textbf{E}(\cdot)$ between plates and show how the translation happens, see Fig. \ref{fig:traj_1}. For more examples of input-translated pairs, see Fig. \ref{fig:translation2}.
% and \ref{fig:translation-target}.

% Having approximated the electric field with a neural network, we run the simulation ODE \ref{main_odde} from the left hyperplane of \textit{test} digits 3 to the right of digits 2. We demonstrate that our method perfectly preserves style and content of digits, recovering the target distribution $\mathbb{Q}(\textbf{x}^{-})$ of colored digits 2 (see Fig.\ref{fig:translation-target}). We detect that we perfectly translate color from digits 3 to generated digits 2. Also, we see that thin digits 3 are transformed to thin digits 2 and vice-versa for thick digits. 


For comparison, we add the results of the translation of the the popular ODE-based Flow Matching (FM) method \citep{liu2022flow,lipmanflow,tong2023conditional}. The key difference between our method and FM is that FM matches to a \textit{time}-conditional transformation (velocity), whereas our method matches to a \textit{space}-conditional transformation (electric field). Interestingly, FM does not always accurately translate the shape and color of the initial digits 3, see Fig.\ref{fig:translation-target_}. 

 



% Also, we showcase the sampling trajectories for our method when we translate digits 3 to digits 2 (see Fig.\ref{fig:traj_1}). Since we see that intermediate images are realistic, the electrical field lines pass through manifold of data. 
 
%Our proposed method perfectly recovers $\mathbb{Q}(\textbf{x}^{-})$. Our map $T$ clearly preserves the color of digits. As previously, we detect the declaining of performance with the increasing distance between planes. Color and shape of digits preserve worse. 

 


\subsection{Image Generation Experiment}
\label{generation_exp}
We also consider the task of generating $32 \times 32$ colored digits 2 from the MNIST dataset. For this task, we place white noise on the left hyperplane $z=0$ and colored digits 2 on the right plate $z=30$. We learn the electric field $\textbf{E}(\cdot)$ between the plates and demonstrate recovering the distribution $\mathbb{Q}(\textbf{x}^{-})$. Also, we show the sampling trajectories for our EFM (see Fig. \ref{fig:traj_gener}). We qualitatively see that our method recovers the target distribution $\mathbb{Q}(\textbf{x}^{-})$ of colored digits 2 (see Fig. \ref{fig:generation-our}). 

Also, for completeness and comparison, we show the results of generation of PFGM method which is also based on the electrostatic theory \citep[PFGM]{xu2022poissonflowgenerativemodels}, see Fig. \ref{fig:generation-pfgm}.  We run the PFGM method with hyper parameters, which are described in Appendix \ref{app:experiments_details}.




% \section{Related works}
% \label{related} 
 


% \textbf{Noise-to-Data.} The first attempt to bind generative models and thermodynamics is Energy-Based models(EBM)\cite{lecun2005loss}. These models find energy state function for data and generate new samples through simulation of Langevin Dynamics \cite{du2019implicit}. Another class of generative models \cite{sohl2015deep} inspired by thermodynamics is Diffusion Models(DM), which are extremely popular nowadays. These models are composed of forward and backward stochastic processes \cite{song2020score}. While the forward process corrupts data, injecting Gaussian noise simultaneously, the backward process reverses the forward, recovering data. The more recent Poisson Flow Generation Models (PFGM) \cite{xu2022poissonflowgenerativemodels}  is inspired by the electrostatic theory. PFGM regards the data as charges in a hyperplane and approximates electric field in augmented space between the hyperplane and upper semi-sphere, whose flux is uniform. Then, PFGM simulates backward ODE along field lines , generating  data samples. 

% Nonetheless, DM and PFGM is only applied for \textbf{Noise-to-data} setting and can't build transformation between untractable distributions. 

% \textbf{Data-to-Data.} There are some generative models, allowing build maps between complex data distributions. Instead of adding and removing noise as DM, Flow matching (FM) \cite{lipman2022flow,liu2022flow} is ODE-based approach that transforms probability distributions. FM learns the time derivative of this transformation(a.k.a flow) as a time-dependent vector field (velocity), approximating  this by neural networks with $\mathcal{L}$2 loss function. Bridge Matching (BM) \cite{shi2024diffusion, albergo2022building, gushchin2024adversarial}  is SDE-based methodology for which FM is the limiting  case.

 



\section{Discussions \& Limitations}

% \subsection{Limitations}

\textbf{Influence of dimensionality}. In high dimensions, our algorithm may require working with small numbers. Specifically, the multiplier $1/||\textbf{x} - \textbf{x}'||^D$ in the electric field formula Eq. (\ref{ND_field_continious}) may produce values comparable to machine precision as the dimensionality of $D$ increases. As a result, the training of our method may become less stable.

\textbf{The impact of inter-plate distance $L$ on the field estimation.}  The larger the inter-plate distance $L$ is, the more curved and disperse the electric field lines become, see, e.g., Fig. \ref{fig:3d_long}. Also, with an increase of this distance the electric field has to be accurately learned in a larger volume between the plates. A careful selection of the hyperparameter $L$ may be important when applying our method.

\textbf{Defining the optimal training volume.} Our training approach involves sampling points $\widetilde{\textbf{x}}^+$ and $\widetilde{\textbf{x}}^-$ from the distributions, interpolating them with Eq. (\ref{middle_point_sample}) and noising them with Eq. (\ref{noise-gauss}). This allows us to consider an intermediate point $\widetilde{\textbf{x}}$ between the plates (\ref{middle_point_sample}) to learn the electrostatic field. 
% The effective electric field training volume turns out to be restricted, although, strictly speaking, the electric field is not zero at any point in space. 
There may exist smarter schemes to choose such points; it is a promising question of further work. 


 

\section{Impact Statement} This paper presents work whose goal is to advance the field of Machine Learning. There are many potential societal consequences 
of our work, none of which we feel must be specifically highlighted here.


\label{discussions}  



 

 

 
 
 
 

 
 

\bibliography{references}
\bibliographystyle{icml2025}


%%%%%%%%%%%%%%%%%%%%%%%%%%%%%%%%%%%%%%%%%%%%%%%%%%%%%%%%%%%%%%%%%%%%%%%%%%%%%%%
%%%%%%%%%%%%%%%%%%%%%%%%%%%%%%%%%%%%%%%%%%%%%%%%%%%%%%%%%%%%%%%%%%%%%%%%%%%%%%%
% APPENDIX
%%%%%%%%%%%%%%%%%%%%%%%%%%%%%%%%%%%%%%%%%%%%%%%%%%%%%%%%%%%%%%%%%%%%%%%%%%%%%%%
%%%%%%%%%%%%%%%%%%%%%%%%%%%%%%%%%%%%%%%%%%%%%%%%%%%%%%%%%%%%%%%%%%%%%%%%%%%%%%%
\newpage
\appendix
\onecolumn

\section{Properties of D-dimensional electric field lines}
\label{ND_lines}


\begin{figure}[ht]
\vskip 0.2in
\begin{center}
\centerline{\includegraphics[width=\columnwidth]{appendixA2.pdf}}
\caption{Electric field flux. (a) Through an arbitrary stream tube, (b) through a stream tube located infinitely close to the charged plane, (c) placed at a large distance from the charged plane. }
\label{appendixA2}
\end{center}
\vskip -0.2in
\end{figure}



In this Appendix the basic properties of electric field lines in $D$-dimensions are formulated and proved. 

%In order to do this, we introduce some additional concepts and prove auxiliary statements. 

\begin{definition}
\label{flux_def}
    The flux of electric field with a strength $\textbf{E}$ through an area $\textbf{dS}$ is called $d\Phi = \textbf{E}\cdot\textbf{dS}$. The flux through a finite surface is defined as an integral:

    \begin{equation}
        \Phi = \int d\Phi = \iint \textbf{E}\cdot\textbf{dS}.
    \end{equation}
\end{definition}

\begin{definition}
    Consider a closed piecewise smooth curve $\Gamma$ placed in an electric field. A field line passes through each point of this contour. The set of these lines is called a stream surface or a stream tube (Fig.\ref{appendixA2}, (a))
\end{definition}

\begin{lemma}
\label{lemma_constant_flux}
    The electric field flux is conserved along a stream surface if there are no charges inside that surface. 
\end{lemma}

\begin{proof}
    Consider an arbitrary stream tube $\partial M$ (Fig.  \ref{appendixA2},(a)). Note that the normal for closed surfaces is directed outwards. Near the right end of the tube, the normal and the electric field are co-directional, and near the left boundary, they have opposite directions. Therefore, $\Phi_1 = -\Phi_1'$. It is required to prove that $\Phi_1' = \Phi_2$. 

    The full flux is a sum of fluxes through the ends of the tube and through its lateral surface:
    
    \begin{equation}
        \Phi_{full} = \Phi_1+\Phi_2+\Phi_{lat}.
    \end{equation}

    The flux through the lateral surface, by the definition of a stream tube, must be zero: $\Phi_{lat} = 0$. Thus, $\Phi_{full} = \Phi_1 + \Phi_2 = \Phi_2 - \Phi_1'$.
    
    From the Gauss's theorem (\ref{3D_gauss}):

    \begin{equation}
        \Phi_{full} = \iint_{\partial M} \textbf{E}\cdot\textbf{dS} = \int_M q(\textbf{x})d\textbf{x} = 0. 
    \end{equation}

    From where it follows that $\Phi_1' = \Phi_2$.
    
\end{proof}

\begin{corollary}
    An electric field line cannot terminate in empty space.
\end{corollary}

\begin{lemma}
\label{dPhi_lemma}
    Consider a charge distribution with density $q(\textbf{x})$ on an $D$-dimensional hyperplane embedded in $\mathbb{R}^{D+1}$.  Let $dS$ denote an element of $D$-dimensional surface area.  For a stream tube (Fig.(\ref{appendixA2}, (c)) with area $dS$ as its base, the electric flux through this tube remains constant and is given by:

    \begin{equation}
        d\Phi = \frac{q(\textbf{x})dS}{2}.
    \end{equation}
    
\end{lemma}

\begin{proof}
    Consider a stream tube in form of a cylinder and the charged surface dividing it in two equal halves. The axis of the cylinder is perpendicular to the surface. %infinitely close to the surface (Fig.(\ref{appendixA2}, b)), which is equally spaced up and down with respect to the plane. 
    The flux through this surface will consist of three contributions: The flux through the upper base $d\Phi_1 = EdS$, the flux through the lower base $d\Phi_2 = EdS = d\Phi_1\equiv d\Phi$, and the flux through the lateral surface $d\Phi_{lat} = 0$:

    \begin{equation}
        d\Phi_{full} = d\Phi_1 + d\Phi_2 + d\Phi_{lat} = d\Phi+d\Phi +0= 2d\Phi.
    \end{equation}

    From the Gauss's theorem (\ref{3D_gauss}):

    \begin{equation}
        d\Phi_{full} = dq_{in} = q(\textbf{x})dS.
    \end{equation}

    Hence, near the surface the electric field flux is $d\Phi = \frac{1}{2}q(\textbf{x})dS$. Since the flux is conserved along the stream tube (lemma \ref{lemma_constant_flux}), $d\Phi = \frac{1}{2}q(\textbf{x})dS$ at any distance from the surface, not only infinitely close to the plane. 
    
\end{proof}


\begin{figure}[ht]
\vskip 0.2in
\begin{center}
\centerline{\includegraphics[width=60mm]{appendixA1.pdf}}
\caption{The field flux of a point charge $q_0$ through an arbitrary surface $\Sigma$ seen at solid angle $\Omega$.}
\label{appendixA1}
\end{center}
\vskip -0.2in
\end{figure}

\begin{lemma}
    Let us assume that a surface $\Sigma$ can be seen at a solid angle $\Omega$ from a point charge $q_0$ (Fig.\ref{appendixA1}). The electric field flux through this surface is equal to:

\begin{equation}
    \Phi = \frac{q_0\Omega}{S_{D-1}}.
    \label{flux_through_sigma}
\end{equation}
    
\end{lemma}

\begin{proof}
    Divide $\Sigma$ into small surface elements $dS$. The total flux is the integral over the entire surface, $\Phi = \int_\Sigma d\Phi$. By the definition of flux (\ref{flux_def}), and according to Eq. (\ref{ND_field}) we have:

    \begin{equation}
        d\Phi = \textbf{E}\cdot\textbf{dS}= E dS \cos\alpha = E dS_{\perp} = \frac{q_0}{S_{D-1}}\frac{dS_{\perp}}{r^{D-1}} = \frac{q_0d\Omega}{S_{D-1}}
    \end{equation}

    Then after integration over the solid angle, we obtain (\ref{flux_through_sigma}).
\end{proof}

\begin{lemma}
\label{lines_main_lemma}
    Let there be an electrically neutral system $(\int q(\textbf{x})d\textbf{x}  = q_{+} - |q_-|= 0)$ bounded in space. Then the electric field lines must begin at positive charges and end at negative charges, except perhaps for the number of lines of zero measure. 
\end{lemma}

\begin{proof}
    Let us assume the opposite and consider the electric field lines that start at the positive charges of the system and end at infinity (if there are no such lines, consider the lines that come from infinity and end at the negative charges). Let us denote the size of the system by $\ell = \max_{\textbf{x},\textbf{y}\in\text{supp}(q) }(|\textbf{x} - \textbf{y}|)$. Consider moving a distance $L\gg \ell$ along these lines from the initial charge system. Let $\xi = \ell/L\ll 1$. We define the surface $\Sigma$ such that it is intersected by the lines. Using the multipole decomposition of the electric field \citep[\wasyparagraph 40-41]{LandauLifshitz2} with a first-order accuracy, we obtain:

    \begin{equation}
        \textbf{E}|_{\Sigma} = \textbf{E}^{(0)} + \textbf{E}^{(1)} + ... = \textbf{E}_{\text{point}} + O\left(\frac{\xi}{L^{D-2}}\right),
    \end{equation}

    where $\textbf{E}^{(i)}$ is the $i$-th order of multipole expansion, $\textbf{E}^{(0)} = \textbf{E}_{\text{point}}$ is the field of a point charge. At large distance from the system the contribution of the point charge becomes the major one. All other contributions can be neglected.  And therefore, in the limiting case $\xi\rightarrow 0, L\rightarrow\infty$ the formula (\ref{flux_through_sigma}) can be used. Since there is no limit on the increase of $L$, one can achieve an approximation accuracy as high as one needs.

    Then $\Phi = \int \textbf{E}\cdot\textbf{dS} = (q_{+} - |q_-|)\Omega/S_{D-1}(1) = 0$ by convention. Hence,

    \begin{equation}
        \int_{\Sigma}\textbf{E}\cdot\textbf{dS} = 0.
    \end{equation}

    Therefore, if the lines that start (end) at the charges of the system and end (start) at infinity exist, their measure is zero and they do not create any flux.     
    
\end{proof}


\begin{figure}[ht]
\vskip 0.2in
\begin{center}
\centerline{\includegraphics[width=60mm]{appendixA3.pdf}}
\caption{Closed loop of an electric field line. This situation is impossible due to the circulation theorem. }
\label{appendixA3}
\end{center}
\vskip -0.2in
\end{figure}

\begin{lemma}
    Electric field lines cannot form closed loops (as in Fig. \ref{appendixA3}).
\end{lemma}

\begin{proof}
    Assume that there exists a closed loop $\ell$ along which $\oint_{\ell} \textbf{E}\cdot\textbf{dl}>0$. At the same time, by virtue of the circulation theorem (\ref{3D_circ}) $\oint_{\ell} \textbf{E}\cdot\textbf{dl}=0$.Since the two expressions contradict to each other there can be no such thing.
\end{proof}


\section{Proof of the Electrostatic Field Matching theorem}
\label{proof_main_theorem}


\begin{theorem}[\textbf{Electrostatic Field Matching}]
    Let $\textbf{x}^+$ be distributed over $\mathbb{P}(\textbf{x}^+)$. Then $\textbf{x}^-=T(\textbf{x}^+)$ is distributed over $\mathbb{Q}(\textbf{x}^-)$ almost surely:
    \begin{equation}
        \text{If}\;\; \textbf{x}^+\sim \mathbb{P}(\textbf{x}^+) \Rightarrow T(\textbf{x}^+) = \textbf{x}^-\sim \mathbb{Q}(\textbf{x}^-).
    \end{equation}
\end{theorem}

\begin{proof}

    Let $\{\widetilde{\textbf{x}}^+_i\}_{i=1}^n$ be a set of points distributed over the distribution $q^+(\widetilde{\textbf{x}}^+) = q^+(\textbf{x}^+)\delta(z) = \mathbb{P}(\textbf{x}^+)\delta(z)$. We denote the electric field lines starting at these points as $\{\widetilde{\textbf{x}}_i(\tau)\}_{i= 1}^n, \tau\in[a,b]\subset \mathbb{R}$. From the properties of electric field lines (see \ref{lines_main_lemma}) they must end on the support of the second distribution $q^-(\widetilde{\textbf{x}}^-)= q^-(\textbf{x}^-)\delta(z-L) = \mathbb{Q}(\textbf{x}^-)\delta(z-L)$, that is:
    \begin{equation}
        \frac{d\widetilde{\textbf{x}}_i}{d\tau} = \textbf{E}(\widetilde{\textbf{x}}_i), \; \widetilde{\textbf{x}}_i(a) = \widetilde{\textbf{x}}^+_i,\;\widetilde{\textbf{x}}_i(b) = \widetilde{\textbf{x}}^-_i,
    \end{equation}
    where $\textbf{E}(\widetilde{\textbf{x}})$ is the field between plates (see (\ref{main_field_bridge})). 
    
    Let us denote by $\hat{\mathbb{Q}}_n(\widetilde{\textbf{x}}^-) = \hat{\mathbb{Q}}_n(\textbf{x}^-)\delta(z-L)$ the effective distribution of points $\{\widetilde{\textbf{x}}^-_i\}_{i = 1}^n$ from the second data set, which were obtained by moving along electric field lines. Then we have to prove that $\hat{\mathbb{Q}}_n(\textbf{x}^-)$ converges to the true distribution $\mathbb{Q}(\textbf{x}^-)$ with probability 1 (almost surely):
    
    \begin{equation}
        P\big(\lim_{n\rightarrow \infty} \hat{\mathbb{Q}}_n = \mathbb{Q}\big) = 1,
    \end{equation}

where $P(\cdot)$ is probability of a given event.

\begin{figure}[ht]
\vskip 0.2in
\begin{center}
\centerline{\includegraphics[width=60mm]{appendixB1.pdf}}
\caption{To the proof of the Electrostatic Field Matching theorem. Here $\mathbb{P}(\textbf{x}^+)$ and $\mathbb{Q}(\textbf{x}^-)$ are distributions of two sets of data. $dn, dn'$ are the number of points $\textbf{x}^+_i, \textbf{x}^-_i$ that fall in the volumes $dS, dS'$, respectively. A stream tube starting at $dS$ and ending at $dS'$ is also shown.}
\label{appendixB1}
\end{center}
\vskip -0.2in
\end{figure}


    Let us select an element of $D$-dimensional volume $dS$ on the first distribution. Let $dn$ be the number of points $\textbf{x}^+_i\in\mathbb{R}^D$ that are in this volume. Consider an electric field stream tube with $dS$ as a base. Let $dS'$ be the element of the volume into which $dS$ has passed, and let $dn'$ be the number of points $\textbf{x}^-_i\in\mathbb{R}^D$ which have entered into this volume (Fig. \ref{appendixB1}).

    An effective distribution $\hat{\mathbb{Q}}_n(\textbf{x}^-)$ is:
    
    \begin{equation}
        \hat{\mathbb{Q}}_n(\textbf{x}^-) = \frac{d(\text{probability})}{d(\text{volume})} = \frac{dn'}{n dS'},
    \end{equation}

    where $d(\text{probability}) = \frac{dn'}{n}$ is the probability of points $\textbf{x}^-_i$ falling in the $d(\text{volume}) = dS'$.

    By definition, the field lines do not cross the stream tube, so $dn' = dn$.

    Since the points $\textbf{x}^+_i$ are distributed over $\mathbb{P}(\textbf{x})$, due to the strong law of large numbers, the ratio $dn/n$ converges to $\mathbb{P}(\textbf{x})dS$ with probability one (almost surely):

    \begin{equation}
        P\big(\lim_{n\rightarrow \infty }\frac{dn}{n} = \mathbb{P}(\textbf{x})dS\big) = 1 \Longleftrightarrow \frac{dn}{n} \xrightarrow[n\rightarrow\infty]{\text{almost surely}}\mathbb{P}(\textbf{x})dS
    \end{equation}

    The electric field flux is conserved along the stream tube and is equal to (\ref{dPhi_lemma}):

    \begin{equation}
        d\Phi = \frac{\mathbb{P}(\textbf{x}^+)dS}{2} = \frac{\mathbb{Q}(\textbf{x}^-)dS'}{2}.
    \end{equation}

    From whence we get:

    \begin{equation}
        \hat{\mathbb{Q}}_n(\textbf{x}^+)dS' = \frac{dn'}{n} = \frac{dn}{n}\xrightarrow[n\rightarrow\infty]{\text{almost surely}}\mathbb{P}(\textbf{x}^+)dS = \mathbb{Q}(\textbf{x}^-)dS',
    \end{equation}
    
    which proves the theorem.

    
\end{proof}


\section{Experimental details}
\label{app:experiments_details}

We aggregate the hyper-parameters of our Algorithm~\ref{algorithm:EFM} for different experiments in the Table \ref{table:hyperparams}. We base our code for the experiments on PFGM's code \url{https://github.com/Newbeeer/Poisson_flow}.

In the 2D illustrative example (see \S\ref{3d_exp} ), we make the inference by constructing the ODE Euler solver for the equation \ref{main_odde} with the following iterative scheme:
\begin{equation}
    \widetilde{\textbf{x}}_{t+1} = \widetilde{\textbf{x}}_{t} + \lambda\textbf{E}(\widetilde{\textbf{x}}_{t}). 
\end{equation}
We use the learning rate $\lambda=2e-3$ and number of steps $NFE=20$ to reach the right hyperplane $z=6$.

In the case of the Image experiments (see \S\ref{transfer_exp} and \S\ref{generation_exp}),  we use the RK45 ODE solver provided by \url{https://docs.scipy.org/doc/scipy/reference/generated/scipy.integrate.RK45.html} for the inference process with the hyper-parameters rtol=1e-4 , atol=1e-4 and number of steps (NFE) equals to 100. Also, we use Exponential Moving Averaging (EMA) technique with the ema rate decay equals to 0.99 . As for the optimization procedure, we use Adam optimizer \cite{kingma2015adam} with the learning rate $\lambda=2e-4$ and weight decay equals to 1e-4. 

Evaluation of the training time for our solver on the image's experiments (see \S\ref{transfer_exp} and \S\ref{generation_exp})takes less than 10 hours on a single GPU GTX 1080ti (11 GB VRAM).


\begin{table}[!h]
\centering
\begin{tabular}{|l|l|l|l|l|l|l|l|}

\hline
Experiment            & D    & Batch Size   & $L$    & $NFE$,Num Steps & $\lambda$,LR   &  Weight Decay & $\sigma$ \\ \hline
Gaussian   Swiss-roll & 2    & 1024 & 6 & 20        & 2e-3 & 0.           & 0.001  \\ \hline
Colored MNIST Translation (3$\rightarrow$2)     & 3072 & 64   & 10   & 100       & 2e-4 & 1e-4         & 0.01  \\ \hline
Colored MNIST Generation                    & 3072 & 64   & 30   & 100       & 2e-4 & 1e-4         & 0.01\\ \hline
\end{tabular}
\caption{\centering\scriptsize Hyper-parameters of Algorithm~\ref{algorithm:EFM} for different experiments, where $D$ is the dimensionality of task, $L$ is the distance betwenn plates and $\sigma$ is used for the definition points between plates (see \S\ref{prac_implementation}).}
\label{table:hyperparams}
\end{table}

We use the source code \url{https://github.com/Newbeeer/pfgmpp} for running \textbf{PFGM} in our experiments. We found the following values of hyper parameters are appropriate for us: $\gamma=5, \tau=0.3, \epsilon = 1e-3$, see \cite{xu2022poissonflowgenerativemodels} for details.

Also, we utilize the source code of Flow Matching (\textbf{FM}) from the github page \url{https://github.com/atong01/conditional-flow-matching/tree/main} in the experiment \S\ref{transfer_exp}. 



%%%%%%%%%%%%%%%%%%%%%%%%%%%%%%%%%%%%%%%%%%%%%%%%%%%%%%%%%%%%%%%%%%%%%%%%%%%%%%%
%%%%%%%%%%%%%%%%%%%%%%%%%%%%%%%%%%%%%%%%%%%%%%%%%%%%%%%%%%%%%%%%%%%%%%%%%%%%%%%


\end{document}


% This document was modified from the file originally made available by
% Pat Langley and Andrea Danyluk for ICML-2K. This version was created
% by Iain Murray in 2018, and modified by Alexandre Bouchard in
% 2019 and 2021 and by Csaba Szepesvari, Gang Niu and Sivan Sabato in 2022.
% Modified again in 2023 and 2024 by Sivan Sabato and Jonathan Scarlett.
% Previous contributors include Dan Roy, Lise Getoor and Tobias
% Scheffer, which was slightly modified from the 2010 version by
% Thorsten Joachims & Johannes Fuernkranz, slightly modified from the
% 2009 version by Kiri Wagstaff and Sam Roweis's 2008 version, which is
% slightly modified from Prasad Tadepalli's 2007 version which is a
% lightly changed version of the previous year's version by Andrew
% Moore, which was in turn edited from those of Kristian Kersting and
% Codrina Lauth. Alex Smola contributed to the algorithmic style files.
