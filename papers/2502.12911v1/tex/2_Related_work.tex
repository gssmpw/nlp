\section{Related Work}
\label{sec:Related_work}
Schema linking is essential for aligning natural language queries with database components, playing a pivotal role in the overall text-to-SQL process. Recent approaches utilize large language models (LLMs) to enhance schema linkage through methods such as semantic retrieval, historical query integration, and bidirectional linking~\citep{pourreza2024dtssql, cao2024rsl,zhang2024structure}. Techniques like DAIL-SQL and E-SQL refine linkage by leveraging LLMs' contextual understanding and query enrichment capabilities~\citep{gao2023dailsql, caferouglu2024sql}. Retrieval-based strategies focus on retrieve information from database to improve SQL generation~\citep{pourreza2024dtssql}. Chess~\citep{talaei2024chess} and CodeS~\citep{li2024codes} utilize semantic retrieval to improve schema linking by accessing relevant database information.
Despite these advancements, many state-of-the-art text-to-SQL models emphasize fine-tuning strategy. DTS-SQL~\citep{pourreza2024dtssql} fine-tunes a local LLM to select relevant tables from user queries, while Solid-SQL~\citep{liu-etal-2025-solid} introduces a data augmentation method for enhancing schema linking. DELLM~\citep{hong2024knowledge} finetuned a specialized data expert LLM to provide additional knowledge for schema linking and SQL generation. Approaches like TA-SQL~\citep{qu2024before} use in-context learning to guide LLMs, generating dummy SQL for linked element abstraction. Similarly, RSL-SQL~\citep{cao2024rsl} employs a bidirectional approach for schema simplification. Meanwhile, E-SQL~\citep{qu2024before} and  \citep{maamari2024death} note that schema linking challenges persist, leading some models to use the entire schema directly as input for SQL generation. 

Different with the previous research, our KaSLA utilize a knapsack optimization framework to avoid the relevant elements missing and the inclusion of redundant elements, optimizing schema linking to enhance SQL generation accuracy effectively.
