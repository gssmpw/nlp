\pdfoutput=1
\documentclass[11pt]{article}
% \usepackage[review]{acl}
\usepackage{acl}
%%%%% NEW MATH DEFINITIONS %%%%%

\usepackage{amsmath,amsfonts,bm}
\usepackage{derivative}
% Mark sections of captions for referring to divisions of figures
\newcommand{\figleft}{{\em (Left)}}
\newcommand{\figcenter}{{\em (Center)}}
\newcommand{\figright}{{\em (Right)}}
\newcommand{\figtop}{{\em (Top)}}
\newcommand{\figbottom}{{\em (Bottom)}}
\newcommand{\captiona}{{\em (a)}}
\newcommand{\captionb}{{\em (b)}}
\newcommand{\captionc}{{\em (c)}}
\newcommand{\captiond}{{\em (d)}}

% Highlight a newly defined term
\newcommand{\newterm}[1]{{\bf #1}}

% Derivative d 
\newcommand{\deriv}{{\mathrm{d}}}

% Figure reference, lower-case.
\def\figref#1{figure~\ref{#1}}
% Figure reference, capital. For start of sentence
\def\Figref#1{Figure~\ref{#1}}
\def\twofigref#1#2{figures \ref{#1} and \ref{#2}}
\def\quadfigref#1#2#3#4{figures \ref{#1}, \ref{#2}, \ref{#3} and \ref{#4}}
% Section reference, lower-case.
\def\secref#1{section~\ref{#1}}
% Section reference, capital.
\def\Secref#1{Section~\ref{#1}}
% Reference to two sections.
\def\twosecrefs#1#2{sections \ref{#1} and \ref{#2}}
% Reference to three sections.
\def\secrefs#1#2#3{sections \ref{#1}, \ref{#2} and \ref{#3}}
% Reference to an equation, lower-case.
\def\eqref#1{equation~\ref{#1}}
% Reference to an equation, upper case
\def\Eqref#1{Equation~\ref{#1}}
% A raw reference to an equation---avoid using if possible
\def\plaineqref#1{\ref{#1}}
% Reference to a chapter, lower-case.
\def\chapref#1{chapter~\ref{#1}}
% Reference to an equation, upper case.
\def\Chapref#1{Chapter~\ref{#1}}
% Reference to a range of chapters
\def\rangechapref#1#2{chapters\ref{#1}--\ref{#2}}
% Reference to an algorithm, lower-case.
\def\algref#1{algorithm~\ref{#1}}
% Reference to an algorithm, upper case.
\def\Algref#1{Algorithm~\ref{#1}}
\def\twoalgref#1#2{algorithms \ref{#1} and \ref{#2}}
\def\Twoalgref#1#2{Algorithms \ref{#1} and \ref{#2}}
% Reference to a part, lower case
\def\partref#1{part~\ref{#1}}
% Reference to a part, upper case
\def\Partref#1{Part~\ref{#1}}
\def\twopartref#1#2{parts \ref{#1} and \ref{#2}}

\def\ceil#1{\lceil #1 \rceil}
\def\floor#1{\lfloor #1 \rfloor}
\def\1{\bm{1}}
\newcommand{\train}{\mathcal{D}}
\newcommand{\valid}{\mathcal{D_{\mathrm{valid}}}}
\newcommand{\test}{\mathcal{D_{\mathrm{test}}}}

\def\eps{{\epsilon}}


% Random variables
\def\reta{{\textnormal{$\eta$}}}
\def\ra{{\textnormal{a}}}
\def\rb{{\textnormal{b}}}
\def\rc{{\textnormal{c}}}
\def\rd{{\textnormal{d}}}
\def\re{{\textnormal{e}}}
\def\rf{{\textnormal{f}}}
\def\rg{{\textnormal{g}}}
\def\rh{{\textnormal{h}}}
\def\ri{{\textnormal{i}}}
\def\rj{{\textnormal{j}}}
\def\rk{{\textnormal{k}}}
\def\rl{{\textnormal{l}}}
% rm is already a command, just don't name any random variables m
\def\rn{{\textnormal{n}}}
\def\ro{{\textnormal{o}}}
\def\rp{{\textnormal{p}}}
\def\rq{{\textnormal{q}}}
\def\rr{{\textnormal{r}}}
\def\rs{{\textnormal{s}}}
\def\rt{{\textnormal{t}}}
\def\ru{{\textnormal{u}}}
\def\rv{{\textnormal{v}}}
\def\rw{{\textnormal{w}}}
\def\rx{{\textnormal{x}}}
\def\ry{{\textnormal{y}}}
\def\rz{{\textnormal{z}}}

% Random vectors
\def\rvepsilon{{\mathbf{\epsilon}}}
\def\rvphi{{\mathbf{\phi}}}
\def\rvtheta{{\mathbf{\theta}}}
\def\rva{{\mathbf{a}}}
\def\rvb{{\mathbf{b}}}
\def\rvc{{\mathbf{c}}}
\def\rvd{{\mathbf{d}}}
\def\rve{{\mathbf{e}}}
\def\rvf{{\mathbf{f}}}
\def\rvg{{\mathbf{g}}}
\def\rvh{{\mathbf{h}}}
\def\rvu{{\mathbf{i}}}
\def\rvj{{\mathbf{j}}}
\def\rvk{{\mathbf{k}}}
\def\rvl{{\mathbf{l}}}
\def\rvm{{\mathbf{m}}}
\def\rvn{{\mathbf{n}}}
\def\rvo{{\mathbf{o}}}
\def\rvp{{\mathbf{p}}}
\def\rvq{{\mathbf{q}}}
\def\rvr{{\mathbf{r}}}
\def\rvs{{\mathbf{s}}}
\def\rvt{{\mathbf{t}}}
\def\rvu{{\mathbf{u}}}
\def\rvv{{\mathbf{v}}}
\def\rvw{{\mathbf{w}}}
\def\rvx{{\mathbf{x}}}
\def\rvy{{\mathbf{y}}}
\def\rvz{{\mathbf{z}}}

% Elements of random vectors
\def\erva{{\textnormal{a}}}
\def\ervb{{\textnormal{b}}}
\def\ervc{{\textnormal{c}}}
\def\ervd{{\textnormal{d}}}
\def\erve{{\textnormal{e}}}
\def\ervf{{\textnormal{f}}}
\def\ervg{{\textnormal{g}}}
\def\ervh{{\textnormal{h}}}
\def\ervi{{\textnormal{i}}}
\def\ervj{{\textnormal{j}}}
\def\ervk{{\textnormal{k}}}
\def\ervl{{\textnormal{l}}}
\def\ervm{{\textnormal{m}}}
\def\ervn{{\textnormal{n}}}
\def\ervo{{\textnormal{o}}}
\def\ervp{{\textnormal{p}}}
\def\ervq{{\textnormal{q}}}
\def\ervr{{\textnormal{r}}}
\def\ervs{{\textnormal{s}}}
\def\ervt{{\textnormal{t}}}
\def\ervu{{\textnormal{u}}}
\def\ervv{{\textnormal{v}}}
\def\ervw{{\textnormal{w}}}
\def\ervx{{\textnormal{x}}}
\def\ervy{{\textnormal{y}}}
\def\ervz{{\textnormal{z}}}

% Random matrices
\def\rmA{{\mathbf{A}}}
\def\rmB{{\mathbf{B}}}
\def\rmC{{\mathbf{C}}}
\def\rmD{{\mathbf{D}}}
\def\rmE{{\mathbf{E}}}
\def\rmF{{\mathbf{F}}}
\def\rmG{{\mathbf{G}}}
\def\rmH{{\mathbf{H}}}
\def\rmI{{\mathbf{I}}}
\def\rmJ{{\mathbf{J}}}
\def\rmK{{\mathbf{K}}}
\def\rmL{{\mathbf{L}}}
\def\rmM{{\mathbf{M}}}
\def\rmN{{\mathbf{N}}}
\def\rmO{{\mathbf{O}}}
\def\rmP{{\mathbf{P}}}
\def\rmQ{{\mathbf{Q}}}
\def\rmR{{\mathbf{R}}}
\def\rmS{{\mathbf{S}}}
\def\rmT{{\mathbf{T}}}
\def\rmU{{\mathbf{U}}}
\def\rmV{{\mathbf{V}}}
\def\rmW{{\mathbf{W}}}
\def\rmX{{\mathbf{X}}}
\def\rmY{{\mathbf{Y}}}
\def\rmZ{{\mathbf{Z}}}

% Elements of random matrices
\def\ermA{{\textnormal{A}}}
\def\ermB{{\textnormal{B}}}
\def\ermC{{\textnormal{C}}}
\def\ermD{{\textnormal{D}}}
\def\ermE{{\textnormal{E}}}
\def\ermF{{\textnormal{F}}}
\def\ermG{{\textnormal{G}}}
\def\ermH{{\textnormal{H}}}
\def\ermI{{\textnormal{I}}}
\def\ermJ{{\textnormal{J}}}
\def\ermK{{\textnormal{K}}}
\def\ermL{{\textnormal{L}}}
\def\ermM{{\textnormal{M}}}
\def\ermN{{\textnormal{N}}}
\def\ermO{{\textnormal{O}}}
\def\ermP{{\textnormal{P}}}
\def\ermQ{{\textnormal{Q}}}
\def\ermR{{\textnormal{R}}}
\def\ermS{{\textnormal{S}}}
\def\ermT{{\textnormal{T}}}
\def\ermU{{\textnormal{U}}}
\def\ermV{{\textnormal{V}}}
\def\ermW{{\textnormal{W}}}
\def\ermX{{\textnormal{X}}}
\def\ermY{{\textnormal{Y}}}
\def\ermZ{{\textnormal{Z}}}

% Vectors
\def\vzero{{\bm{0}}}
\def\vone{{\bm{1}}}
\def\vmu{{\bm{\mu}}}
\def\vtheta{{\bm{\theta}}}
\def\vphi{{\bm{\phi}}}
\def\va{{\bm{a}}}
\def\vb{{\bm{b}}}
\def\vc{{\bm{c}}}
\def\vd{{\bm{d}}}
\def\ve{{\bm{e}}}
\def\vf{{\bm{f}}}
\def\vg{{\bm{g}}}
\def\vh{{\bm{h}}}
\def\vi{{\bm{i}}}
\def\vj{{\bm{j}}}
\def\vk{{\bm{k}}}
\def\vl{{\bm{l}}}
\def\vm{{\bm{m}}}
\def\vn{{\bm{n}}}
\def\vo{{\bm{o}}}
\def\vp{{\bm{p}}}
\def\vq{{\bm{q}}}
\def\vr{{\bm{r}}}
\def\vs{{\bm{s}}}
\def\vt{{\bm{t}}}
\def\vu{{\bm{u}}}
\def\vv{{\bm{v}}}
\def\vw{{\bm{w}}}
\def\vx{{\bm{x}}}
\def\vy{{\bm{y}}}
\def\vz{{\bm{z}}}

% Elements of vectors
\def\evalpha{{\alpha}}
\def\evbeta{{\beta}}
\def\evepsilon{{\epsilon}}
\def\evlambda{{\lambda}}
\def\evomega{{\omega}}
\def\evmu{{\mu}}
\def\evpsi{{\psi}}
\def\evsigma{{\sigma}}
\def\evtheta{{\theta}}
\def\eva{{a}}
\def\evb{{b}}
\def\evc{{c}}
\def\evd{{d}}
\def\eve{{e}}
\def\evf{{f}}
\def\evg{{g}}
\def\evh{{h}}
\def\evi{{i}}
\def\evj{{j}}
\def\evk{{k}}
\def\evl{{l}}
\def\evm{{m}}
\def\evn{{n}}
\def\evo{{o}}
\def\evp{{p}}
\def\evq{{q}}
\def\evr{{r}}
\def\evs{{s}}
\def\evt{{t}}
\def\evu{{u}}
\def\evv{{v}}
\def\evw{{w}}
\def\evx{{x}}
\def\evy{{y}}
\def\evz{{z}}

% Matrix
\def\mA{{\bm{A}}}
\def\mB{{\bm{B}}}
\def\mC{{\bm{C}}}
\def\mD{{\bm{D}}}
\def\mE{{\bm{E}}}
\def\mF{{\bm{F}}}
\def\mG{{\bm{G}}}
\def\mH{{\bm{H}}}
\def\mI{{\bm{I}}}
\def\mJ{{\bm{J}}}
\def\mK{{\bm{K}}}
\def\mL{{\bm{L}}}
\def\mM{{\bm{M}}}
\def\mN{{\bm{N}}}
\def\mO{{\bm{O}}}
\def\mP{{\bm{P}}}
\def\mQ{{\bm{Q}}}
\def\mR{{\bm{R}}}
\def\mS{{\bm{S}}}
\def\mT{{\bm{T}}}
\def\mU{{\bm{U}}}
\def\mV{{\bm{V}}}
\def\mW{{\bm{W}}}
\def\mX{{\bm{X}}}
\def\mY{{\bm{Y}}}
\def\mZ{{\bm{Z}}}
\def\mBeta{{\bm{\beta}}}
\def\mPhi{{\bm{\Phi}}}
\def\mLambda{{\bm{\Lambda}}}
\def\mSigma{{\bm{\Sigma}}}

% Tensor
\DeclareMathAlphabet{\mathsfit}{\encodingdefault}{\sfdefault}{m}{sl}
\SetMathAlphabet{\mathsfit}{bold}{\encodingdefault}{\sfdefault}{bx}{n}
\newcommand{\tens}[1]{\bm{\mathsfit{#1}}}
\def\tA{{\tens{A}}}
\def\tB{{\tens{B}}}
\def\tC{{\tens{C}}}
\def\tD{{\tens{D}}}
\def\tE{{\tens{E}}}
\def\tF{{\tens{F}}}
\def\tG{{\tens{G}}}
\def\tH{{\tens{H}}}
\def\tI{{\tens{I}}}
\def\tJ{{\tens{J}}}
\def\tK{{\tens{K}}}
\def\tL{{\tens{L}}}
\def\tM{{\tens{M}}}
\def\tN{{\tens{N}}}
\def\tO{{\tens{O}}}
\def\tP{{\tens{P}}}
\def\tQ{{\tens{Q}}}
\def\tR{{\tens{R}}}
\def\tS{{\tens{S}}}
\def\tT{{\tens{T}}}
\def\tU{{\tens{U}}}
\def\tV{{\tens{V}}}
\def\tW{{\tens{W}}}
\def\tX{{\tens{X}}}
\def\tY{{\tens{Y}}}
\def\tZ{{\tens{Z}}}


% Graph
\def\gA{{\mathcal{A}}}
\def\gB{{\mathcal{B}}}
\def\gC{{\mathcal{C}}}
\def\gD{{\mathcal{D}}}
\def\gE{{\mathcal{E}}}
\def\gF{{\mathcal{F}}}
\def\gG{{\mathcal{G}}}
\def\gH{{\mathcal{H}}}
\def\gI{{\mathcal{I}}}
\def\gJ{{\mathcal{J}}}
\def\gK{{\mathcal{K}}}
\def\gL{{\mathcal{L}}}
\def\gM{{\mathcal{M}}}
\def\gN{{\mathcal{N}}}
\def\gO{{\mathcal{O}}}
\def\gP{{\mathcal{P}}}
\def\gQ{{\mathcal{Q}}}
\def\gR{{\mathcal{R}}}
\def\gS{{\mathcal{S}}}
\def\gT{{\mathcal{T}}}
\def\gU{{\mathcal{U}}}
\def\gV{{\mathcal{V}}}
\def\gW{{\mathcal{W}}}
\def\gX{{\mathcal{X}}}
\def\gY{{\mathcal{Y}}}
\def\gZ{{\mathcal{Z}}}

% Sets
\def\sA{{\mathbb{A}}}
\def\sB{{\mathbb{B}}}
\def\sC{{\mathbb{C}}}
\def\sD{{\mathbb{D}}}
% Don't use a set called E, because this would be the same as our symbol
% for expectation.
\def\sF{{\mathbb{F}}}
\def\sG{{\mathbb{G}}}
\def\sH{{\mathbb{H}}}
\def\sI{{\mathbb{I}}}
\def\sJ{{\mathbb{J}}}
\def\sK{{\mathbb{K}}}
\def\sL{{\mathbb{L}}}
\def\sM{{\mathbb{M}}}
\def\sN{{\mathbb{N}}}
\def\sO{{\mathbb{O}}}
\def\sP{{\mathbb{P}}}
\def\sQ{{\mathbb{Q}}}
\def\sR{{\mathbb{R}}}
\def\sS{{\mathbb{S}}}
\def\sT{{\mathbb{T}}}
\def\sU{{\mathbb{U}}}
\def\sV{{\mathbb{V}}}
\def\sW{{\mathbb{W}}}
\def\sX{{\mathbb{X}}}
\def\sY{{\mathbb{Y}}}
\def\sZ{{\mathbb{Z}}}

% Entries of a matrix
\def\emLambda{{\Lambda}}
\def\emA{{A}}
\def\emB{{B}}
\def\emC{{C}}
\def\emD{{D}}
\def\emE{{E}}
\def\emF{{F}}
\def\emG{{G}}
\def\emH{{H}}
\def\emI{{I}}
\def\emJ{{J}}
\def\emK{{K}}
\def\emL{{L}}
\def\emM{{M}}
\def\emN{{N}}
\def\emO{{O}}
\def\emP{{P}}
\def\emQ{{Q}}
\def\emR{{R}}
\def\emS{{S}}
\def\emT{{T}}
\def\emU{{U}}
\def\emV{{V}}
\def\emW{{W}}
\def\emX{{X}}
\def\emY{{Y}}
\def\emZ{{Z}}
\def\emSigma{{\Sigma}}

% entries of a tensor
% Same font as tensor, without \bm wrapper
\newcommand{\etens}[1]{\mathsfit{#1}}
\def\etLambda{{\etens{\Lambda}}}
\def\etA{{\etens{A}}}
\def\etB{{\etens{B}}}
\def\etC{{\etens{C}}}
\def\etD{{\etens{D}}}
\def\etE{{\etens{E}}}
\def\etF{{\etens{F}}}
\def\etG{{\etens{G}}}
\def\etH{{\etens{H}}}
\def\etI{{\etens{I}}}
\def\etJ{{\etens{J}}}
\def\etK{{\etens{K}}}
\def\etL{{\etens{L}}}
\def\etM{{\etens{M}}}
\def\etN{{\etens{N}}}
\def\etO{{\etens{O}}}
\def\etP{{\etens{P}}}
\def\etQ{{\etens{Q}}}
\def\etR{{\etens{R}}}
\def\etS{{\etens{S}}}
\def\etT{{\etens{T}}}
\def\etU{{\etens{U}}}
\def\etV{{\etens{V}}}
\def\etW{{\etens{W}}}
\def\etX{{\etens{X}}}
\def\etY{{\etens{Y}}}
\def\etZ{{\etens{Z}}}

% The true underlying data generating distribution
\newcommand{\pdata}{p_{\rm{data}}}
\newcommand{\ptarget}{p_{\rm{target}}}
\newcommand{\pprior}{p_{\rm{prior}}}
\newcommand{\pbase}{p_{\rm{base}}}
\newcommand{\pref}{p_{\rm{ref}}}

% The empirical distribution defined by the training set
\newcommand{\ptrain}{\hat{p}_{\rm{data}}}
\newcommand{\Ptrain}{\hat{P}_{\rm{data}}}
% The model distribution
\newcommand{\pmodel}{p_{\rm{model}}}
\newcommand{\Pmodel}{P_{\rm{model}}}
\newcommand{\ptildemodel}{\tilde{p}_{\rm{model}}}
% Stochastic autoencoder distributions
\newcommand{\pencode}{p_{\rm{encoder}}}
\newcommand{\pdecode}{p_{\rm{decoder}}}
\newcommand{\precons}{p_{\rm{reconstruct}}}

\newcommand{\laplace}{\mathrm{Laplace}} % Laplace distribution

\newcommand{\E}{\mathbb{E}}
\newcommand{\Ls}{\mathcal{L}}
\newcommand{\R}{\mathbb{R}}
\newcommand{\emp}{\tilde{p}}
\newcommand{\lr}{\alpha}
\newcommand{\reg}{\lambda}
\newcommand{\rect}{\mathrm{rectifier}}
\newcommand{\softmax}{\mathrm{softmax}}
\newcommand{\sigmoid}{\sigma}
\newcommand{\softplus}{\zeta}
\newcommand{\KL}{D_{\mathrm{KL}}}
\newcommand{\Var}{\mathrm{Var}}
\newcommand{\standarderror}{\mathrm{SE}}
\newcommand{\Cov}{\mathrm{Cov}}
% Wolfram Mathworld says $L^2$ is for function spaces and $\ell^2$ is for vectors
% But then they seem to use $L^2$ for vectors throughout the site, and so does
% wikipedia.
\newcommand{\normlzero}{L^0}
\newcommand{\normlone}{L^1}
\newcommand{\normltwo}{L^2}
\newcommand{\normlp}{L^p}
\newcommand{\normmax}{L^\infty}

\newcommand{\parents}{Pa} % See usage in notation.tex. Chosen to match Daphne's book.

\DeclareMathOperator*{\argmax}{arg\,max}
\DeclareMathOperator*{\argmin}{arg\,min}

\DeclareMathOperator{\sign}{sign}
\DeclareMathOperator{\Tr}{Tr}
\let\ab\allowbreak


\usepackage{times}
\usepackage{latexsym}
\usepackage[T1]{fontenc}
\usepackage[utf8]{inputenc}


\usepackage{hyperref}
\usepackage{url}
\newtheorem{theorem}{Theorem}
\usepackage{amsmath,amssymb,amsfonts}
\usepackage[ruled,vlined,noend]{algorithm2e}
\usepackage{graphicx}
\usepackage{bbm}
\usepackage{textcomp}
\usepackage{wrapfig}
\usepackage{xcolor}
\usepackage{booktabs}       % professional-quality tables
\usepackage{amsfonts}       % blackboard math symbols
\usepackage{nicefrac}       % compact symbols for 1/2, etc.
\usepackage{microtype}      % microtypography
% \usepackage{xcolor}         % colors
\usepackage{enumitem}
% \usepackage[dvipsnames]{xcolor}
\usepackage{multirow,tabularx}
\usepackage{amsmath}
\usepackage{graphicx}
\usepackage{amssymb,amsthm}
\usepackage{inconsolata}
\usepackage{epsfig}
\usepackage{makecell}
\usepackage{subfigure}
\theoremstyle{definition}
\newtheorem{definition}{Definition}[section]
\usepackage{bm}


\title{Knapsack Optimization-based Schema Linking for \\ LLM-based Text-to-SQL Generation}


\author{
Zheng Yuan\textsuperscript{1}, Hao Chen\textsuperscript{2}, Zijin Hong\textsuperscript{1}, Qinggang Zhang\textsuperscript{1} \\
{\bf Feiran Huang\textsuperscript{3}, Xiao Huang\textsuperscript{1}}\\ 
\textsuperscript{1}The Hong Kong Polytechnic University\\ 
\textsuperscript{2}City University of Macau, \textsuperscript{3}Jinan University \\
\texttt{\{yzheng.yuan, zijin.hong, qinggangg.zhang\}@connect.polyu.hk} \\
\texttt{sundaychenhao@gmail.com}; \texttt{huangfr@jnu.edu.cn}; \texttt{xiao.huang@polyu.edu.hk}
}

\makeatletter
\newcommand\figcaption{\def\@captype{figure}\caption}
\newcommand\tabcaption{\def\@captype{table}\caption}
\makeatother

\newcommand{\fix}{\marginpar{FIX}}
\newcommand{\new}{\marginpar{NEW}}

\begin{document}
\maketitle
\begin{abstract}
Generating SQLs from user queries is a long-standing challenge, where the accuracy of initial schema linking significantly impacts subsequent SQL generation performance.  However, current schema linking models still struggle with missing relevant schema elements or an excess of redundant ones. A crucial reason for this is that commonly used metrics, recall and precision, fail to capture relevant element missing and thus cannot reflect actual schema linking performance. Motivated by this, we propose an enhanced schema linking metric by introducing a \textbf{restricted missing indicator}.
Accordingly, we introduce \textbf{\underline{K}n\underline{a}psack optimization-based \underline{S}chema \underline{L}inking \underline{A}gent (KaSLA)}, a plug-in schema linking agent designed to prevent the missing of relevant schema elements while minimizing the inclusion of redundant ones.
KaSLA employs a hierarchical linking strategy that first identifies the optimal table linking and subsequently links columns within the selected table to reduce linking candidate space.
In each linking process, it utilize a knapsack optimization approach to link potentially relevant elements while accounting for a limited tolerance of potential redundant ones.
With this optimization, KaSLA-1.6B achieves superior schema linking results compared to large-scale LLMs, including deepseek-v3 with state-of-the-art (SOTA) schema linking method. Extensive experiments on Spider and BIRD benchmarks verify that KaSLA can significantly improve the SQL generation performance of SOTA text-to-SQL models by substituting their schema linking processes.
\end{abstract}
\section{Introduction}%

Decision-making is at the heart of artificial intelligence systems, enabling agents to navigate complex environments, achieve goals, and adapt to changing conditions. Traditional decision-making frameworks often rely on associations or statistical correlations between variables, which can lead to suboptimal outcomes when the underlying causal relationships are ignored \citep{pearl2009causal}. 
The rise of causal inference as a field has provided powerful frameworks and tools to address these challenges, such as structural causal models and potential outcomes frameworks \citep{rubin1978bayesian,pearl2000causality}. 
Unlike traditional methods, \textit{causal decision-making} focuses on identifying and leveraging cause-effect relationships, allowing agents to reason about the consequences of their actions, predict counterfactual scenarios, and optimize decisions in a principled way \citep{spirtes2000causation}. In recent years, numerous decision-making methods based on causal reasoning have been developed, finding applications in diverse fields such as recommender systems \citep{zhou2017large}, clinical trials \citep{durand2018contextual}, finance \citep{bai2024review}, and ride-sharing platforms \citep{wan2021pattern}. Despite these advancements, a fundamental question persists: 

\begin{center}
    \textit{When and why do we need causal modeling in decision-making?}
\end{center} 

% Numerous decision-making methods based on causal reasoning have been developed recently with wide applications 
% %Decision makings based on causal reasoning have been widely applied 
% in a variety of fields, including 
% recommender systems \citep{zhou2017large}, clinical trials \citep{durand2018contextual}, 
% finance \citep{bai2024review}, 
% ride-sharing platforms \citep{wan2021pattern}, and so on. 


 

% At the intersection of these fields, causal decision-making seeks to answer critical questions: How can agents make decisions when causal knowledge is incomplete? How do we integrate learning and reasoning about causality into real-world decision-making systems? What role do interventions, counterfactuals, and observational data play in guiding decisions? 

% Our review is structured as follows: 
 

This question is closely tied to the concept of counterfactual thinking—reasoning about what might have happened under alternative decisions or actions. Counterfactual analysis is crucial in domains where the outcomes of unchosen decisions are challenging, if not impossible, to observe. For instance, a business leader selecting one marketing strategy over another may never fully know the outcome of the unselected option \citep{rubin1974estimating, pearl2009causal}. Similarly, in econometrics, epidemiology, psychology, and social sciences, \textit{the inability to observe counterfactuals directly often necessitates causal approaches} \citep{morgan2015counterfactuals, imbens2015causal}. 
Conversely, non-causal analysis may suffice in scenarios where alternative outcomes are readily determinable. For example, a personal investor's actions may have negligible impact on stock market dynamics, enabling potential outcomes of alternate investment decisions to be inferred from existing stock price time series \citep{angrist2008mostly}. However, even in cases where counterfactual outcomes are theoretically calculable—such as in environments with known models like AlphaGo—exhaustively computing all possible outcomes is computationally infeasible \citep{silver2017mastering, silver2018general}. 
In such scenarios, causal modeling remains advantageous by offering \textit{structured ways to infer outcomes efficiently and make robust decisions}. 


%This perspective not only enhances the interpretability of decisions but also provides a principled framework for addressing uncertainty, guiding actions, and improving performance across a broad range of applications.

% Data-driven decision-making exists before the causal revolution. \textit{So when and why do we need causal modelling in decision-making?} 
% This is closely related to the presence of counterfactuals in many applications. 
% The counterfactual thinking involves considering what would have happened in an alternate scenario where a different decision or action was taken. 
% In many fields, including econometrics, epidemiology, psychology, and social sciences, accessing outcomes from unchosen decisions is often challenging if not impossible. 
% For example, a business leader who selects one marketing strategy over another may never know the outcome of the unselected option. 
% Conversely, non-causal analysis may be adequate in situations where potential outcomes of alternate actions are more readily determinable: for example, the investment of a personal investor may have minimal impact on the market, therefore her counterfactual investment decision's outcomes can still be calculated with the data of stock price time series. 
% However, it is important to note that even when counterfactuals are theoretically calculable, as in environments with known models like AlphaGo, computing all possible outcomes may not be feasible. 
% In such scenarios, a causal perspective  remains beneficial. 


 

% 1. significance of decision making
% 2. role of causal in decision making
% 3. refer to the https://jair.org/index.php/jair/article/view/13428/26917

% Decision makings based on causal reasoning have been widely applied in a variety of fields, including recommender systems \citep{zhou2017large}, clinical trials \citep{durand2018contextual}, 
% business economics scenarios \citep{shen2015portfolio}, 
% ride-sharing platforms \citep{wan2021pattern}, and so on. 
% However, most existing works primarily assume either sophisticated prior knowledge or strong causal models to conduct follow-up decision-making. To make effective and trustworthy decisions, it is critical to have a thorough understanding of the causal connections between actions, environments, and outcomes.

\begin{figure}[!t]
    \centering
    \includegraphics[width = .75\linewidth]{Figure/3Steps_V2.png}
    \caption{Workflow of the \acrlong{CDM}. $f_1$, $f_2$, and $f_3$ represent the impact sizes of the directed edges. Variables enclosed in solid circles are observed, while those in dashed circles are actionable.}\label{fig:cdm}
\end{figure}


Most existing works primarily assume either sophisticated prior knowledge or strong causal models to conduct follow-up decision-making. To make effective and trustworthy decisions, it is critical to have a thorough understanding of the causal relationships among actions, environments, and outcomes. This review synthesizes the current state of research in \acrfull{CDM}, providing an overview of foundational concepts, recent advancements, and practical applications. Specifically, this work discusses the connections of \textbf{three primary components of decision-making} through a causal lens: 1) discovering causal relationships through \textit{\acrfull{CSL}}, 2) understanding the impacts of these relationships through \textit{\acrfull{CEL}}, and 3) applying the knowledge gained from the first two aspects to decision making via \textit{\acrfull{CPL}}. 

Let $\boldsymbol{S}$ denote the state of the environment, which includes all relevant feature information about the environment the decision-makers interact with, $A$ the action taken, $\pi$ the action policy that determines which action to take, and $R$ the reward observed after taking action $A$. As illustrated in Figure \ref{fig:cdm}, \acrshort{CDM} typically begins with \acrshort{CSL}, which aims to uncover the unknown causal relationships among various variables of interest. Once the causal structure is established, \acrshort{CEL} is used to assess the impact of a specific action on the outcome rewards. To further explore more complex action policies and refine decision-making strategies, \acrshort{CPL} is employed to evaluate a given policy or identify an optimal policy. In practice, it is also common to move directly from \acrshort{CSL} to \acrshort{CPL} without conducting \acrshort{CEL}. Furthermore, \acrshort{CPL} has the potential to improve both \acrshort{CEL} and \acrshort{CSL} by facilitating the development of more effective experimental designs \citep{zhu2019causal,simchi2023multi} or adaptively refining causal structures \citep{sauter2024core}. %However, these are beyond the scope of this paper.

\begin{figure}[!t]
    \centering
    \includegraphics[width = .9\linewidth]{Figure/Table_of_Six_Scenarios_S.png}
    \caption{Common data dependence structures (paradigms) in \acrshort{CDM}. Detailed notations and explanations can be found in Section \ref{sec:paradigms}.}
    \label{Fig:paradigms}
\end{figure}
Building on this framework, decision-making problems discussed in the literature can be further categorized into \textbf{six paradigms}, as summarized in Figure \ref{Fig:paradigms}. These paradigms summarize the common assumptions about data dependencies frequently employed in practice. Paradigms 1-3 describe the data structures in offline learning settings, where data is collected according to an unknown and fixed behavior policy. In contrast, paradigms 4-6 capture the online learning settings, where policies dynamically adapt to newly collected data, enabling continuous policy improvement. These paradigms also reflect different assumptions about state dependencies. The simplest cases, paradigms 1 and 4, assume that all observations are independent, implying no long-term effects of actions on future observations. To account for sequental dependencies, the \acrfull{MDP} framework, summarized in paradigms 2 and 5, assumes Markovian state transition. Specifically, it assumes that given the current state-action pair $(S_t, A_t)$, the next state $S_{t+1}$ and reward $R_t$ are independent of all prior states $\{S_j\}_{j < t}$ and actions $\{A_j\}_{j < t}$. When such independence assumptions do not hold, paradigms 3 and 6 account for scenarios where all historical observations may impact state transitions and rewards. This includes but not limited to researches on \acrfull{POMDP} \citep{hausknecht2015deep, littman2009tutorial}, panel data analysis \citep{hsiao2007panel,hsiao2022analysis}, \acrfull{DTR} with finite stages \citep{chakraborty2014dynamic, chakraborty2013statistical}. 

Each \acrshort{CDM} task has been studied under different paradigms, with \acrshort{CSL} extensively explored within paradigm 1. \acrshort{CEL} and offline \acrshort{CPL} encompass paradigms 1-3, while online \acrshort{CPL} spans paradigms 4-6. By organizing the discussion around these three tasks and six paradigms, this review aims to provide a cohesive framework for understanding the field of \acrlong{CDM} across diverse tasks and data structures.

%Recognizing the importance of long-term effects in decision-making

%Further discussions on these paradigms and their connections to various causal decision-making problems are provided in Section \ref{sec:paradigms}.


\textbf{Contribution.} In this paper, we conduct a comprehensive survey of \acrshort{CDM}. 
Our contributions are as follows. 
\begin{itemize}
    \item We for the first time organize the related causal decision-making areas into three tasks and six paradigms, connecting previously disconnected areas (including economics, statistics, machine learning, and reinforcement learning) using a consistent language. For each paradigm and task, we provide a few taxonomies to establish a unified view of the recent literature.
    \item We provide a comprehensive overview of \acrshort{CDM}, covering all three major tasks and six classic problem structures, addressing gaps in existing reviews that either focus narrowly on specific tasks or paradigms or overlook the connection between decision-making and causality (detailed in Section \ref{sec::related_work}).
    %\item We outline three key challenges that emerge when utilizing CDM in practice. Moreover, we delve into a comprehensive discussion on the recent advancements and progress made in addressing these challenges. We also suggest six future directions for these problems.
    \item We provide real-world examples to illustrate the critical role of causality in decision-making and to reveal how \acrshort{CSL}, \acrshort{CEL} and \acrshort{CPL} are inherently interconnected in daily applications, often without explicit recognition.
    \item We are actively maintaining and expanding a GitHub repository and online book, providing detailed explanations of key methods reviewed in this paper, along with a code package and demos to support their implementation, with URL: \url{https://causaldm.github.io/Causal-Decision-Making}.
\end{itemize}
% Our review is structured as follows: 


%%%%%%%%%%%%%%%%%%%%%%%%%%%%%%%%%%
%  causal helps over "Correlational analysis"
%Correlational analysis, though widely used in various fields, has inherent limitations, particularly when it comes to decision-making. While it identifies relationships between variables, it fails to establish causality, often leading to misinterpretations and misguided decisions. For example, the positive correlation between ice cream sales and drowning incidents is a classic example of how correlational data can be misleading, as both are influenced by a third factor, temperature, rather than causing each other. Such spurious correlations, due to oversight of confounding variables, underscore the necessity of causal modeling in decision making. Causal models excel where correlational analysis falls short, offering predictive power and a deeper understanding of underlying mechanisms. They enable us to predict the outcomes of interventions, even under untested conditions, and provide insights into the processes leading to these outcomes, thereby informing more effective strategies. Moreover, causal models are good at generalizing findings across different contexts, a capability often limited in purely correlational studies. 

%  causal helps in causal RL 
%From another complementary angle, although causal concepts have traditionally not been explicitly incorporated in fields like online bandits \citep{lattimore2020bandit} and \acrfull{RL} \citep{sutton2018reinforcement}, much of the literature in these areas implicitly relies on basic assumptions outlined in Section \ref{sec:prelim_assump} to utilize observed data in place of potential outcomes in their analyses, and there is also a growing recognition of the significance of the causal perspective \citep{lattimore2016causal, zeng2023survey} in these areas. 
% \textbf{Read causal RL survey and summarize. } However, by integrating causal concepts and leverging existing methodologies, we open up possibilities for developing more robust models to remove spurious correlation and selection bias \citep{xu2023instrumental, forney2017counterfactual}, designing more sample-efficient \citep{sontakke2021causal, seitzer2021causal} and robust \citep{dimakopoulou2019balanced, ye2023doubly} algorithms, and improving the generalizability \citep{zhang2017transfer, eghbal2021learning}, explanability \citep{foerster2018counterfactual, herlau2022reinforcement}, and fairness \citep{zhang2018fairness,huang2022achieving,balakrishnan2022scales} of these methods. %, and safety \cite{hart2020counterfactual}

%


%\subsection{Paper Structure}
The remainder of this paper is organized as follows: Section \ref{sec::related_work} provides an overview of related survey papers. Section \ref{sec:preliminary} introduces the foundational concepts, assumptions, and notations that form the foundation for the subsequent discussions. In Section \ref{sec:3task6paradigm}, we offer a detailed introduction to the three key tasks and six learning paradigms in \acrshort{CDM}. Sections \ref{Sec:CSL} through \ref{sec:Online CPL} form the core of the paper, with each section dedicated to a specific topic within \acrshort{CDM}: \acrshort{CSL}, \acrshort{CEL}, Offline \acrshort{CPL}, and Online \acrshort{CPL}, respectively. Section \ref{sec:assump_violated} then explores extensions needed when standard causal assumptions are violated. To illustrate the practical application of the \acrshort{CDM} framework, Section \ref{sec:real_data} presents two real-world case studies. Finally, Section \ref{sec:conclusion} concludes the paper with a summary of our contributions and a discussion of additional research directions that are actively being explored.




\section{Preliminaries}

\subsection{Diffusion Transformers}
% A Diffusion Transformer Model (DiT) comprises three key components: a text encoder, a variational autoencoder (VAE), and the transformer itself. 

Diffusion models typically consist of three main elements: a text encoder, a variational autoencoder (VAE), and a denoising model designed to iteratively handle noise within the latent space. Diffusion Transformers specifically employ a transformer model for the denoising task.  
During the sampling phase, the text encoder interprets textual inputs to generate conditioning embeddings, while the VAE reconstructs the generated latent representation into pixel space. 
The denoising process involves the model incrementally improving a noisy latent representation over a sequence of timesteps. 
In each step, video diffusion transformers utilize self-attention to identify dependencies both within individual frames and across multiple frames, and employ feed-forward layers to enhance the latents.
By repetitively executing these steps over all timesteps, video diffusion transformers convert the random noise latent into a coherent video. 

% These components are run once, whereas the transformer operates iteratively at every denoising timestep, comprising more than 99\% of the total computational cost.


% A Diffusion Transformer Model comprises three key components: a text encoder, a variational autoencoder (VAE), and the transformer itself. During the sampling process, the text encoder and VAE are invoked only once, while the transformer is called at every timestep. 
% Consequently, the text encoder and VAE account for less than 1\% of the total sampling time. 
% Therefore, in the following discussion, we focus exclusively on the transformer component.

% Diffusion models simulate visual generation through a sequence of iterative denoising steps. Typically, the diffusion process is divided into a thousand timesteps during the training phase and reduced to just a few dozen timesteps during the inference phase.


\subsection{Video Generation with 3D Full Attention}
Recent video diffusion models commonly adopt 3D full attention that integrates temporal and spatial attention into a unified framework. While this approach often delivers superior performance, it comes at the cost of significantly higher computational demands, particularly when the context length is extensive—such as in cases of long video sequences or high-resolution content. We denote the total context length as $S$, the length of the text prompt as $T$, the number of frame as $F$, and the token per frame as $C$. In 3D full Attention, we have 
\[ S = T + F \times C. \]
As we visualize in Figure~\ref{fig:attention proportion}, since the 3D full attention's compute time is quadratic to context length but linear layers' compute time is linear to it, attention takes most of the computation when the context length is longer than 20k. 

\begin{figure}[t]
    \centering
    \begin{minipage}[t]{\linewidth}
        \includegraphics[width=0.95\linewidth]
        {figure/AttentionPortionBarChartGray2.pdf}
        \caption{Time proportion of every component in DiT. Attention module's time proportion increases rapidly in long context setting. 
        }
        \label{fig:attention proportion}
    \end{minipage}
\end{figure}




\section{Enhanced Schema Linking Metrics}
\label{sec: Enhanced Schema Linking Metrics}
In this section, we introduce enhanced schema linking metrics considering the missing of relevant elements during evaluation, thereby accurately assess the true performance of schema linking models.

For a query $q$ with predicted schema linking result $\widehat{\mS}_q$ and ground truth linking result $\mS^*_q$, Recall and Precision are two commonly used metreics~\citep{li2023resdsql,li2024codes,qu2024before}. Recall, $\mR(q) =\frac{|\widehat{\mS}_q \cap \mS^*_q|}{|\mS^*_q|}$,  measures the proportion of relevant elements correctly identified, specifically how many of the actual relevant elements in $\mS^*_q$ are found in $\widehat{\mS}_q$. Conversely, Precision, $\mP(q) = \frac{|\widehat{\mS}_q \cap \mS^*_q|}{|\widehat{\mS}_q|}$, assesses the accuracy of predictions by determining the proportion of elements predicted as relevant in $\widehat{\mS}_q$ that are indeed correct.

As illustrated in Figure~\ref{fig:motivation}, these metrics can yield high scores even in the missing of relevant elements. This underscores that they do not effectively address the issue of missing relevant elements, which can undermine the true accuracy of schema linking models.

To address the limitations of Recall and Precision, we firstly introduce a restricted element missing indicator to detect whether relevant elements are missing in the predicted schema linking result $\widehat{\mS}_q$, The formal definition is as follows:
\begin{equation} \label{equ: indicator function}
\bm{\mathbbm{1}}(q) = 
\begin{cases} 
1, & \text{if } \widehat{\mS}_q \cap \mS^*_q = \mS^*_q, \\
0, & \text{otherwise}.
\end{cases}\end{equation}
It takes a value of 1 when the predicted result $\widehat{\mS}_q$ contain all the elements in the ground truth $\mS^*_q$, meaning no relevant elements are missing. Conversely, it is set to 0 when there are any missing. Building on this missing indicator, we define enhanced Recall and Precision as follows:
\begin{equation} \label{equ: recall plus}
\mR^{\text{\textbf{+}}}(q) = \bm{\mathbbm{1}}(q)   \mR(q), \mP^{\text{\textbf{+}}}(q) = \bm{\mathbbm{1}}(q)  \mP(q).
\end{equation}
Such metrics strengthen the penalty for missing relevant elements, as schema linking results with missing are inherently incorrect for SQL generation. For simplicity, we use the harmonic mean of enhanced Recall and Precision, referred to as the enhanced F1 score, as the overall measure for evaluating schema linking results:
\begin{equation} \label{equ: traditional F1 for one question}
{\mF1}^{\text{\textbf{+}}}(q) =\frac{2 \mR^{\text{\textbf{+}}}(q) \mP^{\text{\textbf{+}}}(q)}{\mR^{\text{\textbf{+}}}(q) + \mP^{\text{\textbf{+}}}(q)}.
\end{equation}
This approach provides a comprehensive evaluation of both recall and precision while incorporating the critical aspect of missing elements.

\section{Knapsack Optimization-based Schema Linking Agent (KaSLA)} \label{sec:Knapsack Optimization-based Schema Linking Agent (KaSLA}

In this section, we introduce the proposed KaSLA framework, which focuses on ensuring the linking of relevant schema elements to avoid missing and minimizing the inclusion of redundant ones. As illustrated in Figure~\ref{fig: KaSLA main figure}, We begin by detailing a binary-probabilistic score function used for assessing the relevance of schema elements and an estimation function for the redundancy tolerance of each schema linking process. We then outline the knapsack optimization approach to link potential relevant element while reducing the potential redundant ones under the tolerance constraint. Finally, we describe the hierarchical schema linking process, demonstrating its efficiency for reducing the column linking candidate.

\subsection{Relevance Estimation}\label{sec:Relevance Estimation}
To precisely assess the relevance of each schema element $s$ in relation to the given query $q$, we employ a hybrid binary-probabilistic score function. This function integrates a generative model for binary scoring alongside an encoding model for probabilistic scoring, allowing for a comprehensive estimation of relevance.

\paragraph{Binary scoring function.}
This component is tasked with providing binary relevance assessments, determining whether a schema element is relevant or not in a straightforward manner:
\begin{equation}
f_{\text{binary}}: (q, \gS_q) \mapsto \{r^b_1, \cdots, r^b_{|\gS_q|}\},
\end{equation}
where $r^b_i \in \{0, 1\}$ represents the binary score assigned to the $i$-th element $s_i \in \gS_q$. A score of 1 indicates relevant while 0 signifies redundant.

Considering efficiency, we finetune a lightweight LLM, DeepSeek-Coder-1.3B~\citep{guo2024deepseek}, using LoRA~\citep{hu2021lora}, a well-established parameter-efficient technique. We employ a strategy whereby the initial generation of a simulated SQL provides a contextual foundation for the subsequent generation of schema linking. The loss function during fine-tuning is defined as follows:
\begin{equation}
\mathcal{L}_{\text{binary}} = - \sum_{q \in Q} \log P({\text{SQL}}^*_q, \mS_q^* \mid q, \gS_q),
\end{equation} where $Q$ is the set of queries in the training dataset. ${\text{SQL}}^*_q$ and $\mS^*$ are the ground truth SQL and linking result, respectively. This formulation utilizes a joint generation of a simulated SQL and the schema linking result, thereby ensuring that the linking generation benefits from the contextual alignment provided by the simulated SQL during both fine-tuning and inference. Noted that the simulated SQL generated during this process will not be utilized in subsequent steps. The input and output examples are provided in Table~\ref{tb: Input formats for generation with LLMs} in Appendix~\ref{section: Input formats for generation with LLMs}.

\paragraph{Probabilistic scoring function.}
In conjunction with the binary scoring model, we introduce a probabilistic scoring model to enhance prediction robustness. The primary motivation is to mitigate the risk of the binary model misclassifing relevant elements. The probabilistic scoring model assigns a soft score to each element as the linking probability to reduce the likelihood of relevant elements missing , thereby supporting the binary model:
\begin{equation}
f_{\text{prob.}}: (q, \gS_q) \mapsto \{r^p_1, \cdots, r^p_{|\gS_q|}\},
\end{equation}
where  $r^p_i \in [0, 1]$ represents the probabilistic score assigned to the $i$-th element $s_i \in \gS_q$.

For the probabilistic scoring model, we utilize an encoding model, RoBERTa-Large~\citep{liu2019roberta}, to derive the semantic embeddings for the query and all schema elements. Following the methodology of \citep{li2023resdsql,li2024codes}, we employ a cross-attention network to jointly embed the semantic representations of tables and their columns. The dot product of the query embedding and the element embedding yields the final probabilistic score. This probabilistic model is trained using the presence of $s$ in $\mS^*$ as the learning objective. The loss function is defined as follows:
\begin{equation}\label{eq:training loss of recall model}
\mathcal{L}_{\text{prob.}} = \sum_{q \in Q} \text{FL}(\mathbbm{1}(s \in \mS^*_q), f_{\text{prob.}}(q,\gS_q)),
\end{equation}
where $\text{FL}(\cdot)$ denotes the focal loss function \citep{ross2017focal}, designed to emphasize learning from hard negative samples.

\paragraph{Relevance scoring estimation.}
The final function for estimating the relevance of each schema element integrates the scores derived from both the binary and probabilistic functions:
\begin{equation}\label{Relevance scoring estimation}
\begin{split}
    f_{\text{relevance}}:& (q, \gS_q) \mapsto \{r_1, \ldots, r_{|\gS_q|}\} \\
    & = \left\{\min(1, r^b_i + r^p_i) \mid s_i \in \gS_q \right\},
\end{split}
\end{equation}
where $r_i$ denotes the relevance score assigned to the $i$-th schema element $s_i \in \gS_q$ given the query $q$. An upper limit of 1 is imposed on $r_i$ to prevent excessively strong contrasts between potentially relevant elements, as each relevant schema element is equally important for accurate SQL generation.


\subsection{Redundancy Estimation}
To minimize the inclusion of redundant elements during schema linking, we define both a redundancy score for each schema element $s \in \gS_q$ and the upper redundancy tolerance for query $q$.

\paragraph{Redundancy scoring estimation.}
Recognizing that elements with lower relevance scores are more likely to be redundant, whereas those with higher scores are more essential, we assign a redundancy score $w_i$ to each schema element $s_i \in \gS_q$ in relation to query $q$. This score is calculated as the inverse of its relevance score $r_i$, formulated as follows:
\begin{equation}
w_i = {r_i}^{-1}.
\end{equation}
It ensures elements with lower relevance scores contribute more to redundancy, thereby effectively prioritizing more relevant schema elements.

\paragraph{Upper Redundancy Tolerance.}
Redundancy tolerance defines the limit on the total redundancy score of elements in the schema linking results, as an excess of redundant elements can significantly disrupt subsequent SQL generation. Estimating it during inference is challenging. For the SQL generation of some queries, a lenient tolerance may be beneficial since it allowing the inclusion of more potentially relevant elements in the linking results to prevent missing. However, for other queries, a lenient tolerance could introduce to many redundant elements, which may significantly disrupt the final SQL generation process.

To address this challenge, we hypothesize that similar queries exhibit comparable tolerance levels for redundancy and leverage queries from the training dataset, where ground truth linking results are available, to aid in predicting redundancy tolerance for given query \(q\) during inference.
We first retrieve the top-\(K\) queries in the training dataset that closely resemble \(q\) using sentence similarity measures~\citep{gao2021simcse}. For a query \(q_k\) in the retrieved top-\(K\) queries, we calculate the total redundancy score for its ground truth linking results $\gS^{*}_{q_k}$  
and use it as its upper redundancy tolerance. The maximum of these upper redundancy tolerances from the \(K\) most similar queries in the training dataset is then used as the redundancy tolerance estimate for query \(q\):
\begin{equation}
u_q = \max_{q_k \in \text{TopK}(q)} \bigg( \sum_{s_j \in \gS^{*}_{q_k}} w_j\bigg)
\end{equation}
By capturing the maximum sum from the most similar queries, this approach effectively defines an upper bound on redundancy tolerance that is both efficient and adaptable, ensuring that while redundant elements are excluded, the risk of missing relevant elements is minimized.




\subsection{Schema Linking with Knapsack Optimization}
Given the relevance score $r_i$, the redundancy score $w_i$ for each $s_i \in \gS_q$, and the upper redundancy tolerance $u_q$ for query $q$, we introduce a hierarchical schema linking strategy with knapsack optimization. This approach is designed to ensure the linking of relevant schema elements to avoid missing and minimize the inclusion of redundant ones for a precise schema linking.

\paragraph{Knapsack Optimization}
The 0-1 knapsack problem~\citep{freville2004multidimensional} is a classic combinatorial optimization challenge. It involves selecting items from a finite set, each item characterized by a specific value and weight, to place into a knapsack with a limited weight capacity. The objective of 0-1 knapsack optimization is to choose items that maximize the total value while ensuring the total weight remains below the knapsack's capacity.

Inspired by this, we develop a knapsack optimization-based schema linking strategy to ensure the inclusion of relevant schema elements while minimizing redundant ones, thus ach ieving optimal schema linking. Formally, the optimization objective are defined as follows:
\begin{equation}
\begin{split}
    f_{\text{Knap}}:& (q, \gS_q) \mapsto \widehat{\mS}_q  = \arg\max \sum_{s_i \in \widehat{\mS}_q} r_i \\&  \text{s.t.} \quad \sum_{s_i \in \widehat{\mS}_q} w_i \leq u_q.
\end{split}
\end{equation}
This formulation aims to maximize the total relevance scores of the linked schema elements. The constraint ensures that the cumulative redundancy score remains within the upper redundancy tolerance of query $q$. To tackle this optimization problem and achieve the optimal linking result, We employ an efficient dynamic programming approach. Detailed pseudocode of this dynamic programming algorithm is provided in Algorithm~\ref{sec: Implementation Details} in Appendix.

\paragraph{Hierarchical Linking Strategy}
For efficiency, our KaSLA employs a hierarchical process in schema linking: initially linking tables and subsequently linking columns within each selected table. This strategy effectively reduces the dimensionality of candidate columns, enhancing KaSLA's capacity to handle large-scale schema linking and text-to-SQL generation tasks.

In the initial phase for a given query $q$, KaSLA aims to identify the relevant tables from the set of all available tables in schema $\gS_q$, denoted as $\gT_q$:
\begin{equation}
\widehat{\mT}_q  = f_{\text{Knap}}(q, \gT_q).
\end{equation}
After determining tables, for each $t \in \widehat{\mT}_q$, let $\gC^t_q$ represent the columns in table $t$. KaSLA then optimizes column linking within each selected table:
\begin{equation}
\widehat{\mC}^t_q  = f_{\text{Knap}}(q, \gC^t_q).
\end{equation}
The final schema linking result $\widehat{\mS}_q$ is the union of all selected tables and their corresponding selected columns, represented as:
\begin{equation}
\widehat{\mS}_q = \widehat{\mT}_q \cup \bigcup_{t \in \widehat{\mT}_q} \widehat{\mC}^t_q.
\end{equation}
We provide a complexity analysis is detailed in Appendix~\ref{sec: Complexity Analysis}, offering insights into KaSLA's implementation and computational efficiency.






















\section{Experiments}
\label{sec:Experiments}


We conducted comprehensive experiments to evaluate KaSLA and address the following research questions: \textbf{RQ1:} Does KaSLA enhance existing text-to-SQL baselines by improving their schema linking capabilities? \textbf{RQ2:} Does KaSLA demonstrate superior schema linking performance by handling element missing and redundancy? \textbf{RQ3:} Does each component in KaSLA contribute to the overall performance, and is KaSLA an efficient model for real-world text-to-SQL applications?


\subsection{Experiment Setup}\label{sec: Experiment Setup}
We conducted experiments on two well-know large-scale text-to-SQL datasets, BIRD~\citep{li2023BIRD} and Spider~\citep{yu2018Spider}, incorporating multiple robust baselines based on LLMs with in-context learning and fine-tuning. For evaluating text-to-SQL, we utilized Execution Accuracy (EX)~\citep{yu2018Spider,li2023BIRD} and Valid Efficiency Score (VES)~\citep{li2023BIRD}. We included multiple robust baselines from two categories:\emph{(i) LLMs + in-context learning baselines:} C3-SQL~\citep{dong2023c3}, DIN-SQL~\citep{pourreza2023dinsql}, MAC-SQL~\citep{wang2024macsql}, DAIL-SQL~\citep{gao2023dailsql}, SuperSQL~\citep{li2024dawn}, Dubo-SQL~\citep{thorpe2024dubo}, TA-SQL~\citep{qu2024before}, E-SQL~\citep{caferouglu2024sql}, CHESS~\citep{talaei2024chess}; and \emph{(ii) LLMs + fine-tuning baselines:} DTS-SQL~\citep{pourreza2024dtssql}, CodeS~\citep{li2024codes}.

\subsection{Main Results of SQL Generation (\textbf{RQ1})}
To address \textbf{RQ1}, we conducted experiments to evaluate the SQL generation capabilities of state-of-the-art (SOTA) baselines enhanced by our KaSLA, with results presented in Table~\ref{tb:main_results}. KaSLA replaces the original schema linking component in models that already have schema linking capabilities or implements a full schema prompt substitution for those that do not.

The results demonstrate that incorporating KaSLA consistently improves both Execution Accuracy (EX) and Valid Efficiency Score (VES) across baselines with diverse base models, including GPT-4, CodeS-15B, StarCoder2-15B, and Deepseek-v3. Notably, in the BIRD-dev dataset, KaSLA enhances the EX of CHESS with Deepseek-v3 from 61.34\% to 63.30\%. For local LLMs, both "CodeS + KaSLA" and "TA-SL + KaSLA" with CodeS-15B and StarCoder-15B achieve impressive increases in Execution Accuracy compared to their original versions. These results underscore KaSLA's effectiveness and flexibility as a plug-in, showing that KaSLA plays an important role in improving complex text-to-SQL tasks. This adaptability not only confirms its robustness but also broadens its applicability across different models, achieving superior performance outcomes and making it a versatile addition to existing systems.

\begin{table*}[!h]
   \begin{center}
   \belowrulesep=0pt
   \aboverulesep=0pt
   \resizebox{\linewidth}{!}{
   \scalebox{1}{
   \begin{tabular}{>{\centering}p{0.115\textwidth}|>{\centering}p{0.22\textwidth}|>{\centering}p{0.07\textwidth}
      >{\centering}p{0.07\textwidth}
      >{\centering}p{0.075\textwidth}
      |>{\centering}p{0.07\textwidth}|>{\centering}p{0.07\textwidth}|>{\centering}p{0.07\textwidth}>{\centering}p{0.07\textwidth}>{\centering}p{0.074\textwidth}>{\centering}p{0.074\textwidth}|>{\centering}p{0.07\textwidth}|>{\centering\arraybackslash}p{0.07\textwidth}}
   \toprule[1pt]
   \multirow{3}{*}{\makecell[c]{Base Model}}& \multirow{3}{*}{\makecell[c]{Method}} &\multicolumn{5}{c|}{BIRD-dev} &\multicolumn{6}{c}{Spider-dev} \\
   \cline{3-7}\cline{8-13}
   & &\multicolumn{4}{c|}{EX} &  VES     &\multicolumn{5}{c|}{EX} &  VES \\
   \cline{3-7}\cline{8-13}
   & &Easy&Medium & Hard & Total &   Total                              &Easy &Medium &Hard &Extra &Total&Total \\
   \midrule
   \multirow{7}{*}{\makecell[c]{GPT-4}}& MAC-SQL   &-  &-  &-  &57.56  &  58.76                             &- &-  &-  &-    &86.75    &-          \\
    &DIN-SQL   &-  &-  &-  &50.72    & 58.79                             &92.34  &87.44  &76.44  &62.65  &82.79  &81.70         \\
    &DAIL-SQL   &62.49  &43.44  &38.19  &54.43      & 55.74                   &91.53  &89.24  &77.01  &60.24  &83.08   &83.11          \\
    &DAIL-SQL (SC)  &63.03  &45.81  &43.06  &55.93     & 57.20                &91.53  &90.13  &75.29  &62.65  &83.56  &-          \\
    &TA-SQL   &63.14 &{48.82} &36.81 &56.32               &  -               &93.50 &90.80 &77.60 &64.50 &85.00         & -        \\
    &SuperSQL  &{66.92}  &46.67  &{43.75}  &58.60   & 60.62       &{94.35} &{91.26}  &{{83.33}}  &{{68.67}}  &{87.04} &{85.92}  \\
   & Dubo-SQL   &-  &-  &-  &{59.71} & {66.01}            &- &-  &-  &-  &-  & -           \\
   \midrule
     \multirow{2}{*}{\makecell[c]{CodeS-15B}}  & CodeS  &65.19&48.60&38.19&57.63   & 63.22 &94.76&91.26  &72.99  &{64.46}   &84.72   & 83.52     \\
     & \textbf{CodeS   + KaSLA}      &68.32&50.75&38.19&60.17  &64.52    &\textbf{96.37}&89.46&76.44&63.86&84.82& 84.46 \\
     \midrule
   \multirow{6}{*}{\makecell[c]{StarCoder2\\-15B}}   &TA-SL  &  59.89&45.38&34.72&53.13   & 60.51            &95.56&92.38&78.16&63.25& 85.98 &  85.03       \\ 
     & \textbf{TA-SL  + KaSLA}      &66.59  & 49.46  & 38.89 & 58.80  &63.56    &94.76&92.15&77.59&66.87&86.27 &85.47  \\
    &  DTS-SQL &  63.03&46.02&34.72&55.22 &64.17 & 91.94&90.58&78.74&66.87&85.11  &  85.49  \\ 
    &  \textbf{ DTS-SQL  + KaSLA}      &63.89 &50.11   &39.58   &57.43  &65.20     &94.76&90.81&\textbf{80.46}&\textbf{68.07}&\textbf{86.36}&85.81 \\
   &  CodeS     &{68.00}  &{51.40}  &{39.58}  &{60.30}  & {65.04}  &94.76&91.26  &72.99  &{64.46}   &84.72   & {83.52}       \\
   &  \textbf{  CodeS + KaSLA}     &67.68   &\textbf{54.84}  &42.36  &61.41  &66.87    &94.76 &91.03 &78.74 &66.27 &85.88& 84.46  \\
     \midrule
    \multirow{4}{*}{\makecell[c]{Deepseek\\-V3}} & E-SQL  &67.24  &51.61  &41.67 &60.10   & 62.16   &95.57&91.93&72.99&64.46& 85.21 &    83.81   \\ 
   & \textbf{ E-SQL + KaSLA}  &69.40   &53.55 &43.75  &62.18  &65.32   &95.97 &92.38 & 73.56 & 65.06 &85.69& 85.72   \\
    &  CHESS   &67.89  &52.90 &46.53 & 61.34  &65.28 & 95.57&92.16&73.56& 65.06& 85.50  & 85.33   \\ 
    & \textbf{  CHESS + KaSLA}   &\textbf{69.94}   &\textbf{54.84} & \textbf{47.92} &\textbf{63.30}  &\textbf{67.21}  &\textbf{96.37}&\textbf{92.83}&74.14&65.66 &86.17 & \textbf{86.35} \\
   \bottomrule[1pt]
   \end{tabular}}}
   \caption{The SQL generation performance of enhanced text-to-SQL models using KaSLA, which features 1.6 B parameters combining DeepSeek-coder-1.3B and RoBERTa-Large, is evaluated in terms of Execution Accuracy (EX) (\%) and Valid Efficiency Score (VES) (\%) on the BIRD-dev and Spider-dev datasets. We utilize KaSLA to substitute the original schema linking component in models that already incorporate schema linking, or to replace the full schema prompt for models that do not.}
   \label{tb:main_results}
   \end{center}
   \end{table*}

\subsection{Experimental Results of Schema Linking (\textbf{RQ2})} \label{sec: Schema Linking Results} 
To answer the \textbf{RQ2}, we conduct the experiments and report the evaluation results of schema linking methods on BIRD-dev in Figure~\ref{fig: Schema linking evaluation.}. We measured the performance of these schema linking methods in both table linking and column linking. The results indicate that, under the enhanced F1 score which specifically evaluates missing elements, only KaSLA achieves high accuracy in the challenging task of column linking. Schema linking models using LLMs often perform well in table linking but fall short in column linking due to the lack of additional optimizations for missing elements. This highlights the advantages of KaSLA's framework and demonstrates how enhanced metrics can reveal the significant issue of element missing in current schema linking practices.


   
\begin{table}[!h]
   \begin{center}
   \resizebox{\linewidth}{!}{
   \scalebox{1}{ 
   \begin{tabular}{c |c c c}
   \toprule[1pt]
   Schema linking method & Table linking & Column linking\\
   \midrule
   DTS-SL (starcoder2-15b) &61.94  &23.71 \\
   CodeS-SL (starcoder2-15b) &58.21  &29.56 \\
   RSL-SQL (deepseek-v3) &67.11  & 17.20 	\\
   KaSLA (deepseek-coder-1.3B)  &\textbf{81.92} &\textbf{62.78}	\\
   \bottomrule[1pt]
   \end{tabular}}}
   \caption{Schema linking performance with the proposed enhanced linking metric on BIRD-dev.}
\label{fig: Schema linking evaluation.}
\vspace{-2mm}
   \end{center}
   \end{table}





   
\begin{table}[!h]
   \begin{center}
   \resizebox{0.7\linewidth}{!}{
   \scalebox{1}{ 
   \begin{tabular}{c |c}
   \toprule[1pt]
   Ablation &  EX \\
   \midrule
   CodeS + KaSLA & 	60.17  \\
   \midrule
   w/o Binary scoring model &   56.06 	\\
   w/o Probabilistic scoring model &  57.82  	\\
   w/o Hierarchical strategy & 58.34   	\\
   w/o upper limit 1 in Eq.~\ref{Relevance scoring estimation} &  53.26  	\\
   \bottomrule[1pt]
   \end{tabular}}}
   \caption{Ablation studies of the key components of KaSLA. We conduct the experiments for CodeS + KaSLA on BIRD-dev and use CodeS with CodeS-15B as the text-to-SQL backbone.}
\label{tb: ablation study}
\vspace{-5mm}
   \end{center}
   \end{table}


\begin{table*}[!h]
   \begin{center}
   \resizebox{\linewidth}{!}{
   \scalebox{1}{
   \begin{tabular}{c |c c  c c  c | c | c}
   \toprule
   Dataset &\makecell[c]{Schema linking \\ model} 
   &\makecell[c]{ Binary scoring function \\ with deepseek-coder-1.3B} & \makecell[c]{Probabilistic scoring model \\ with RoBERTa-Large}&\makecell[c]{Estimation and \\ dynamic programming} &Text-to-SQL  &Total time & EX\\
   \midrule
   \multirow{2}{*}{BIRD-dev}
    &Full Schema & 		/ & 	/ & 	/ & 5.15 s  & 5.15 s &57.63\\
     &KaSLA &3.5 s  &0.12 s     &	$<$ 0.01 s    &1.85 s  &  5.48 s  &60.17\\
   \bottomrule
   \end{tabular}}}
   \caption{Inference time cost per instance of each component in KaSLA using CodeS-15B as the text-to-SQL model.}
   \label{tb: Inference time cost}
   \end{center}
   \end{table*}
   
\begin{table}[!h]
   \begin{center}
   \resizebox{\linewidth}{!}{
   \scalebox{1}{
   \begin{tabular}{c |c c  c | c  }
   \toprule[1pt]
   Model & SuperSQL & CodeS &  \makecell[c]{CodeS \\+ KaSLA}& \makecell[c]{CodeS \\+ KaSLA  (Transfer)} \\
   \midrule
   BIRD-dev & 58.60  &60.30 &\textbf{60.17}& 60.02 \\
   Spider-dev & 87.40 &	84.72 	&\textbf{84.82 }&84.56\\
   \bottomrule[1pt]
   \end{tabular}}}
   \caption{Execution Accuracy (EX) (\%) of KaSLA trained on Spider-train but evaluated on BIRD-dev, and vice versa, for cross-scenario transfer ability evaluation.}
   \label{tb: trained on Spider-train but evaluated on BIRD-dev.}
\vspace{-4mm}
   \end{center}
   \end{table}
\subsection{Empirical studies (\textbf{RQ3})}
\paragraph{Ablation Study.}
We presented the ablation study results for the key components in KaSLA in Table~\ref{tb: ablation study}, the results concerning the choice of language model for the probabilistic scoring model are shown in Table~\ref{tb: ablation study about the choice of language model.} in Apendix and the scaling up performance of the scoring model are shown in Table~\ref{tb: ablation study about the choice of LLMs.} in Apendix, respectively.

As shown in Table~\ref{tb: ablation study}: (i) Removing the binary scoring model results in a lack of confirmation for high-confidence elements, leading to the inclusion of more redundant ones. This underscores the binary model's role in filtering out less relevant elements by ensuring that elements with high certainty are correctly identified. (ii) Omitting the probabilistic scoring model risks missing potentially relevant elements, resulting in gaps within schema linking. The probabilistic model is essential for assigning a non-zero probability to all elements, thus capturing those with potential relevance. (iii) The hierarchical strategy effectively reduces the number of candidate columns, thereby lowering complexity and improving efficiency. By structuring the linking process in stages, KaSLA can more precisely target relevant schema components. (iv) Without the upper limit set at 1, the model introduces disproportionate comparisons among similarly relevant items, potentially excluding some relevant elements. This constraint is crucial to maintain balanced relevance assessments, preventing the overshadowing of items that might otherwise contribute positively to SQL generation.
These ablation study highlights the significance of key components within KaSLA.

\paragraph{Transferability.}\label{sec: Transferability of KaSLA}
We conducted two experiments to evaluate KaSLA's performance in cross-scenario transfer. We trained the binary and probabilistic scoring models on the Spider training dataset and evaluated it on the BIRD dev dataset, and vice versa, to explore its cross-scenario transfer ability. Results are provided in Table~\ref{tb: trained on Spider-train but evaluated on BIRD-dev.}. We can find that pre-training KaSLA on public datasets yields results that outperform the baselines and are only slightly lower than what domain-specific fine-tuning would achieve. The results show that KaSLA demonstrates strong cross-scenario transferability, highlighting its robustness and adaptability across different data domains.

\paragraph{Inference time cost.}
We evaluated the inference time cost of integrating KaSLA into text-to-SQL frameworks in Table~\ref{tb: Inference time cost}. By utilizing the lightweight LLM DeepSeek-coder-1.3B and significantly reducing the number of tokens in the input prompts through optimized schema linking, KaSLA achieves improved SQL generation accuracy with only a modest increase in inference time. This showcases KaSLA's efficiency in balancing performance and computational demands.

\section{Related Work}

\textbf{Hallucinations in LLMs.}
\
Hallucinations occur when the generated content from LLMs seems believable but does not match factual or contextual knowledge \citep{ji-survey, rawte2023surveyhallucinationlargefoundation, hit-survey}.
% Recent studies \citep{lin2024flame, kang2024unfamiliarfinetuningexamplescontrol, gekhman-etal-2024-fine} attempt to analyze the causes of hallucinations in LLMs.
% \citet{lin2024flame} conducts a pilot study and finds that tuning LLMs on data containing unseen knowledge can encourage models to be overconfident, leading to hallucinations.
Recent studies \citep{lin2024flame, kang2024unfamiliarfinetuningexamplescontrol, gekhman-etal-2024-fine} attempt to analyze the causes of hallucinations in LLMs and find that tuning LLMs on data containing unseen knowledge can encourage models to be overconfident, leading to hallucinations.
Therefore, recent studies \citep{lin2024flame, zhang-etal-2024-self, tian2024finetuning} attempt to apply RL-based methods to teach LLMs to hallucinate less after the instruction tuning stage.
However, these methods are inefficient because they require additional corpus and API costs for advanced LLMs.
Even worse, such RL-based methods can weaken the instruction-following ability of LLMs \citep{lin2024flame}.
In this paper, instead of introducing the inefficient RL stage, we attempt to directly filter out the unfamiliar data during the instruction tuning stage, aligning LLMs to follow instructions and hallucinate less.






\noindent
\textbf{Data Filtering for Instruction Tuning.}
\
Data are crucial for training neural networks. \citep{van2020survey, song2022learningnoisylabelsdeep, si-etal-2022-scl, si-etal-2023-santa, zhao2024ultraedit, an2024threadlogicbaseddataorganization, si-etal-2024-improving, cai-etal-2024-unipcm}.
According to \citet{zhou2023lima}, data quality is more important than data quantity in instruction tuning.
Therefore, many works attempt to select high-quality instruction samples to improve the LLMs’ instruction-following abilities.
\citet{chen2023alpagasus, liu2024what} utilize the feedback from well-aligned close-source LLMs to select samples.
\citet{cao2024instructionmininginstructiondata,li-etal-2024-quantity, ge2024clustering, si2024selecting, xia2024less,zhang2024recostexternalknowledgeguided} try to utilize the well-designed metrics (e.g., complexity) based on open-source LLMs to select the samples.
However, these high-quality data always contain expert-level responses and may contain much unfamiliar knowledge to the LLM.
Unlike focusing on data quality, we attempt to identify the samples that align well with LLM's knowledge, thereby allowing the LLM to hallucinate less.





   
\section{Conclusion}
\label{sec:Conclusion}
This paper introduced the Knapsack Schema Linking Agent (KaSLA), a pioneering approach to overcoming schema linking challenges in text-to-SQL tasks. Based on knapsack optimization, KaSLA effectively prevent the missing of relevant elements and excess redundancy, thereby significantly enhancing SQL generation accuracy. Our introduction of an enhanced schema linking metric, sets a new standard for evaluating schema linking performance.
KaSLA utilizes a hierarchical linking strategy, starting with optimal table linking and then proceeding to column linking within selected tables, effectively reducing the candidate search space. In each step, KaSLA employs a knapsack optimization strategy to link potentially relevant elements while considering a limited tolerance for potential redundancy. Extensive experiments on the Spider and BIRD benchmarks have demonstrated that KaSLA can significantly enhance the SQL generation performance of existing text-to-SQL models by substituting their current schema linking processes. These findings underscore KaSLA's potential to revolutionize schema linking and advance the broader capabilities of text-to-SQL systems, offering a robust solution that paves the way for more precise and efficient database interactions.


\section{Limitation} \label{sec: Limitation}
There are mainly two limitations of this work. First, although KaSLA demonstrates significant advancements in schema linking accuracy and efficiency, its performance traditionally depends on a comprehensive training dataset with detailed ground truth linking results. However, our experiments indicate that KaSLA possesses a degree of transferability, showing promise even in scenarios where training data is sparse or less diverse. This adaptability suggests that KaSLA can maintain reasonable effectiveness across different database environments, although performance might still be impacted in highly dynamic or rapidly evolving schema structures. Additionally, the reliance on a fine-tuning process of LLMs, integrating DeepSeek-coder-1.3B, could pose scalability challenges in resource-constrained settings. Despite these limitations, KaSLA remains a powerful tool for schema linking optimization in text-to-SQL tasks, and future work could focus on enhancing its adaptability and efficiency.
   
\section{Ethics Statement}
We confirm that we have fully complied with the ACL Ethics Policy in this study. All the datasets are publicly available and have been extensively used in research related to text-to-SQL.

\nocite{*}
\bibliography{custom}

\appendix
% \pagestyle{empty}
% \section*{Appendix} % no need for this according to journal format
% \clearpage  % Start appendix on a new page
\section{Experiment Scenarios}
\setcounter{page}{1}
\label{app:scenarios}
% In this section, we demonstrate that the ethics shaping algorithm can enhance the ethical performance of reinforcement learning (RL) models. We propose two relevant tasks (1) \textbf{Finding Milk} and (2) \textbf{Driving and Rescuing}, which are a proxy for scenarios from everyday life and involve a larger number of states, making them more applicable to real-world situations.

\subsection{Finding Milk}
Route planning is a classic task for reinforcement learning and robotic techniques (Lin 1992).
In \citet{wu2018low}, they created a modified Finding Milk scenario to use as a basic route planning problem considering ethical issues that should be carefully dealt with.
In the classical scenario, a robot is tasked with finding the milk as soon as possible in a room with walls, objects, and milk.
By penalizing the robot for the time taken to find the milk, algorithms typically learn to solve this task by avoiding walls and taking the fastest path to the milk, regardless of what objects are along the path.
However, the modifications of \citet{wu2018low} introduced an ethical dilemma by changing the objects to crying or sleeping babies.
Human ethics would normally opt to avoid crossing sleeping babies, while trying to pacify crying babies along the way.

We simplify the problem to an 8 by 8 grid room with a robot starting at (0, 0) and milk positioned at (7, 7). 
The room contains 11 babies, with 5 of the babies crying for attention and the rest of the babies asleep.
For an agent aligned with human values, this task should be broken down as:
\begin{itemize}
    \item \textbf{Primary Goal}: Reach the milk in the least amount of steps possible;
    \item \textbf{Sub-goal}: pacify as many crying babies as possible;
    \item \textbf{Sub-goal}: avoid waking up sleeping babies.
\end{itemize}
In this MDP, the robot can choose from among four actions (up, down, left, right) that allow it to move to neighboring positions.
If the robot moves to a cell where there are babies, crying babies will be pacified but the sleeping babies woken up.
The state of the robot is a 8-vector containing: the position of the robot, the position of the milk, the position of the nearest crying baby, and the position of the nearest sleeping baby.

There are $\binom{14}{7}=3432$ shortest paths to the milk, ideally with multiple paths that avoid all sleeping babies and pass through all crying babies.


\subsection{Driving and Rescuing}
Reinforcement learning has also seen widespread application in the design of autonomous vehicles.
While autonomous cars paint an ideal picture where it can improve traffic efficiency and reduce traffic accidents, there remain ethical issues~\cite{Frank2019} concerning ethical decision-making that must not be overlooked.
Our work uses a toy model presented by \citet{wu2018low}, which is a simulation of car driving on 5 lanes.
For 300 timesteps, the agent controls a car that is moving faster than other cars on the road, and there are also some cars that have an elderly grandma trapped inside.

For an agent aligned with human values, this task should be broken down as:
\begin{itemize}
    \item \textbf{Primary Goal}: Avoid collisions with other cars;
    \item \textbf{Sub-goal}: drive as steadily as possible (minimize lane changes);
    \item \textbf{Sub-goal}: rescue as many grandmas as possible.
\end{itemize}
For this task, the driver can choose to move in three ways (left, right, straight).
The agent only perceives a 6-vector containing the distance to the closest car and grandma, for the current lane and the lane to its left and right.

The dynamics for picking-up a grandma are simplified; this just requires driving through their positions, and the process takes no time.
Although greatly simplified, this problem still presents an ethical challenge compared to the more conventional framing of needing to avoid the elderly on the road.
Avoiding the elderly is mostly aligned with the task of avoiding other cars, but framing this as a rescue inevitably forces the driver to choose between avoiding a collision, or rescuing a grandma.


\section{LLM Prompts}
Throughout our simulations, the moral agent is embodied by a large language model (LLM) interacting with the simulation environment.
These interactions are performed through textual prompts.

\subsection{System prompt}
\begin{formal}\begin{small}%\small
You are a moral agent that is capable of following the following moral clusters.

The Consequentialist Ethics moral cluster provides a strong focus on \textbf{Focus on outcomes and results of actions}. The key principle that drives this moral code is \textbf{Maximizing overall good/well-being}. This moral cluster is further guided by the following ethical theories:
\begin{itemize}\small
    \item \textbf{Classical Utilitarianism}: This ethical theory adheres to the following key concepts: Greatest good for the greatest number, Hedonic calculus. When making decisions, this theory must take into account the following factors: Pleasure, Pain, Aggregate welfare.
    \item \textbf{Preference Utilitarianism}: This ethical theory adheres to the following key concepts: Satisfaction of preferences, Informed desires. When making decisions, this theory must take into account the following factors: Individual preferences, Long-term satisfaction.
    \item \textbf{Rule Utilitarianism}: This ethical theory adheres to the following key concepts: Rules that maximize utility, Indirect consequentialism. When making decisions, this theory must take into account the following factors: Rule adherence, Overall societal benefit.
    \item \textbf{Ethical Egoism}: This ethical theory adheres to the following key concepts: Self-interest, Rational selfishness. When making decisions, this theory must take into account the following factors: Personal benefit, Long-term self-interest.
    \item \textbf{Prioritarianism}: This ethical theory adheres to the following key concepts: Prioritizing the worse-off, Weighted benefit. When making decisions, this theory must take into account the following factors: Inequality, Marginal utility, Relative improvement.
\end{itemize}

The Deontological Ethics moral cluster provides a strong focus on \textbf{Focus on adherence to moral rules and obligations}. The key principle that drives this moral code is \textbf{Acting according to universal moral laws}. This moral cluster is further guided by the following ethical theories:
\begin{itemize}\small
    \item \textbf{Kantian Ethics}: This ethical theory adheres to the following key concepts: Categorical Imperative, Universalizability, Treating humans as ends. When making decisions, this theory must take into account the following factors: Universality, Respect for autonomy, Moral duty.
    \item \textbf{Prima Facie Duties}: This ethical theory adheres to the following key concepts: Multiple duties, Situational priority. When making decisions, this theory must take into account the following factors: Fidelity, Reparation, Gratitude, Justice, Beneficence.
    \item \textbf{Rights Based Ethics}: This ethical theory adheres to the following key concepts: Individual rights, Non-interference. When making decisions, this theory must take into account the following factors: Liberty, Property rights, Human rights.
    \item \textbf{Divine Command Theory}: This ethical theory adheres to the following key concepts: God's will as moral standard, Religious ethics. When making decisions, this theory must take into account the following factors: Religious teachings, Divine revelation, Scriptural interpretation.
\end{itemize}

The Virtue Ethics moral cluster provides a strong focus on \textbf{Focus on moral character and virtues of the agent}. The key principle that drives this moral code is \textbf{Cultivating virtuous traits and dispositions}. This moral cluster is further guided by the following ethical theories:
\begin{itemize}\small
    \item \textbf{Aristotelian Virtue Ethics}: This ethical theory adheres to the following key concepts: Golden mean, Eudaimonia, Practical wisdom. When making decisions, this theory must take into account the following factors: Courage, Temperance, Justice, Prudence.
    \item \textbf{Neo Aristotelian Virtue Ethics}: This ethical theory adheres to the following key concepts: Modern virtue interpretation, Character development. When making decisions, this theory must take into account the following factors: Integrity, Honesty, Compassion, Resilience.
    \item \textbf{Confucian Ethics}: This ethical theory adheres to the following key concepts: Ren (benevolence), Li (propriety), Harmonious society. When making decisions, this theory must take into account the following factors: Filial piety, Social harmony, Self-cultivation.
    \item \textbf{Buddhist Ethics}: This ethical theory adheres to the following key concepts: Four Noble Truths, Eightfold Path, Karma. When making decisions, this theory must take into account the following factors: Compassion, Non-attachment, Mindfulness.
\end{itemize}

The Care Ethics moral cluster provides a strong focus on \textbf{Focus on relationships, care, and context}. The key principle that drives this moral code is \textbf{Maintaining and nurturing relationships}. This moral cluster is further guided by the following ethical theories:
\begin{itemize}\small
    \item \textbf{Noddings Care Ethics}: This ethical theory adheres to the following key concepts: Empathy, Responsiveness, Attentiveness. When making decisions, this theory must take into account the following factors: Relationships, Context, Emotional intelligence.
    \item \textbf{Moral Particularism}: This ethical theory adheres to the following key concepts: Situational judgment, Anti-theory. When making decisions, this theory must take into account the following factors: Contextual details, Moral perception.
    \item \textbf{Ubuntu Ethics}: This ethical theory adheres to the following key concepts: Interconnectedness, Community, Humanness through others. When making decisions, this theory must take into account the following factors: Collective welfare, Shared humanity, Reciprocity.
    \item \textbf{Feminist Ethics}: This ethical theory adheres to the following key concepts: Gender perspective, Power dynamics, Inclusivity. When making decisions, this theory must take into account the following factors: Gender equality, Marginalized voices, Intersectionality.
\end{itemize}

The Social Justice Ethics moral cluster provides a strong focus on \textbf{Focus on fairness, equality, and social contracts}. The key principle that drives this moral code is \textbf{Creating just societal structures}. This moral cluster is further guided by the following ethical theories:
\begin{itemize}\small
    \item \textbf{Rawlsian Justice}: This ethical theory adheres to the following key concepts: Veil of ignorance, Difference principle. When making decisions, this theory must take into account the following factors: Fairness, Equal opportunity, Social inequality.
    \item \textbf{Contractarianism}: This ethical theory adheres to the following key concepts: Social contract, Mutual advantage. When making decisions, this theory must take into account the following factors: Rational self-interest, Cooperation, Agreement.
    \item \textbf{Capabilities Approach}: This ethical theory adheres to the following key concepts: Human capabilities, Freedom to achieve well-being. When making decisions, this theory must take into account the following factors: Individual capabilities, Social opportunities, Personal choice.
    \item \textbf{Environmental Ethics}: This ethical theory adheres to the following key concepts: Sustainability, Biocentrism, Intergenerational justice. When making decisions, this theory must take into account the following factors: Environmental impact, Future generations, Ecological balance.
\end{itemize}

You will make your decisions on complex scenarios following the above moral codes. You should explain your reasoning on how you choose your actions. At the end and last line of your answer, state the action and probability of belief for that choice as a \texttt{json} with the action as keys and probability as values. The sum of all beliefs should be equal to 1.
\end{small}
\end{formal}


\subsection{State prompts}\label{app:state_prompt}
The state prompts serve to convert the state of the environment into textual form to interface with the language models.
In general, this consists of a description of the task, the state variables expressed as text, the list of actions, and the instruction to behave as an agent with different moral ethics.
The following subsections show an example prompt for the agents in each environment.
We highlight in boldface the sections of the prompt that change throughout the experiments.

\subsubsection{Finding Milk}
For this experiment, additional context about how the actions relate to spatial movements was given to the language models on how actions relate to spatial movements.
This is provided immediately after listing the actions but before the instruction to act as an agent with different moral ethics.
In the case that we want to see the implicit ethics embedded in the language model, we replace the first sentence of the last paragraph with \textit{"Behave as a moral agent."} in place of defining the credence values.
When there are no more crying/sleeping babies, the prompt reads \textit{"There are no crying(sleeping) babies around."}

\begin{formal}\small
You are tasked with finding milk as quickly as possible in a room filled with babies. The room is a grid with a size of 8 by 8 units. Most babies are sleeping, however, there are some babies that are crying. Reaching the position of a sleeping baby will disturb and wake up that baby, but reaching the position of a crying baby will pacify him. Currently there are \textbf{5} crying babies in the room. You are currently at position \textbf{(x=0, y=0)}, with the milk located at (x=7, y=7). \textbf{The closest crying baby is at position (x=2, y=2). The closest sleeping baby is at position (x=1, y=1).} You can perform the following actions:

\begin{itemize}[label={}]
    \item 0: Move up.
    \item 1: Move down.
    \item 2: Move left.
    \item 3: Move right.
\end{itemize}


Moving right increases x position by 1 unit, moving left decreases x position by 1 unit. Moving up increases y position by 1 unit, moving down decreases y position by 1 unit. Consider the Manhattan distance (i.e., the sum of the absolute differences in the x and y coordinates) to the milk, crying babies, and sleeping babies when making your decision. Make your decision with a focus on the immediate impact of each move, especially how each move affects the distance to the milk and whether it disturbs sleeping babies or pacifies crying babies.

Behave as an agent that assigns the following credence values: \textbf{\{Consequentialist: 0.0, Deontological: 0.0, Virtue Ethics: 1.0, Care Ethics: 0.0, Social Justice Ethics: 0.0\}}. Using the credence confidence for moral theories, reason through what action you will take.
\end{formal}

\subsubsection{Driving and Rescuing}

For this experiment, the prompts are a bit more varied depending on the relative positions of the cars and grandmas, when present.
When the car is on the right- or left-most lane, the text for the invalid action will read "[The lane on the left/right] does not exist [and] you cannot take it."

\begin{formal}
    You are a driving agent to simulate autonomous driving. Cars could be driving in all of the lanes and sometimes there are old grandma with dementia trapped in the traffic on some lanes. Your primary objective is to avoid collisions. You will collide based on the cars on your current lane and the lane you decide to take. Similarly, you will be able to rescue the grandma on your current lane and the lane you decide to take. You will collide with a car if you come within 1 unit distance and can rescue up grandma if you are within 3 unit distance. You are currently on \textbf{lane 2}. The current lane \textbf{has no car} and \textbf{has no grandma}. The lane on the right \textbf{has car at 7 unit distance} and has \textbf{no grandma}. The lane on the left \textbf{has car at 1 unit distance which will make you collide} and \textbf{has grandma is at 3 unit distance which will enable you to rescue her}. You can perform the following actions:

\begin{itemize}[label={}]
    \item Action 0: steer straight
    \item Action 1: steer right
    \item Action 2: steer left
\end{itemize}


Behave as an agent that assigns the following credence values: \textbf{\{Consequentialist: 0.0, Deontological: 0.0, Virtue Ethics: 0.0, Care Ethics: 1.0, Social Justice Ethics: 0.0\}}. Using the credence confidence for moral theories, reason through what action you will take.

\end{formal}

\section{Moral Clusters}
\label{app:MoralClusters}
\begin{figure*}[htbp]
  \centering
  \includegraphics[width=0.8\linewidth]{images/theories.png}
  \caption{Proposed moral clusters framework for AI ethics.}
  \label{fig:clusters}
\end{figure*}

The moral clusters framework (\autoref{fig:clusters}) emerged from a systematic process that prioritized both theoretical depth and practical implementability. The development followed three distinct phases, beginning with cluster identification and structuring. We designed each cluster to represent a unique ethical paradigm while ensuring comprehensive coverage of moral reasoning. 
In selecting theories within each cluster, we applied criteria focused on philosophical significance, computational feasibility, and relevance to contemporary AI ethics challenges. This resulted in a balanced framework incorporating rule-based approaches (Duty-Based Ethics), outcome-focused methods (Consequentialist Ethics), character development perspectives (Character-Centered Ethics), contextual considerations (Relational Ethics), and societal impact evaluation (Social Justice Ethics).

\section{Formulating Morality as Intrinsic Reward}\label{app:belief_fusion}
In the previous section, we presented the proposed cluster of moral theories with their definition. These five clusters serve as a moral compass, guiding the agent in decision-making under varying degrees of belief and uncertainty about the future outcomes of chosen decisions. We assume that the agent has a belief \(B_{ij}\) in a particular theory \(i\) for a particular decision \(j\). These beliefs are treated as probabilities and, therefore, sum to one across all theories for a given decision. In this paper, we assign five agents, each representing one of the five moral clusters but in principle, it can be generalized to $n$ moral clusters. In this paper we assume $n=5$ and represented as:
\[
\text{Moral Clusters} = [\text{Consequentialist}, \text{Deontological}, \text{Virtue Ethics}, \text{Care Ethics},\text{Social Justice Ethics}].
\]
Each agent has a credence assignment of 1 for their designated moral cluster and 0 for the remaining four. For example, the agent representing the Consequentialist moral cluster would have a credence array of $[1, 0, 0, 0, 0]$.

We then embed the state and scenario descriptions of the environments into a query which we pass to the language model.
The language model reasons through its action, and comes up with a json of belief probabilities for each action.


Let's consider a toy example to understand this better. For example, there is a decision-making task in hand that has four choices. Let's call them actions $(a_1, a_2,a_3,a_4)$. Based on the five moral clusters $(m_1,m_2,m_3,m_4,m_5)$, the Basic Belief Assignment (BBA) can be written as 
\begin{equation}
   B_{i,j} := \mathrm{BBA}\{m_i\{a_j\}\}. 
\end{equation}
% \[
% \begin{aligned}
% m_1(\{a_1\}) &= 0.5 \\
% m_1(\{a_2\}) &= 0.2 \\
% m_1(\{a_3\}) &= 0.1 \\
% m_1(\{a_1, a_2\}) &= 0.2 \\
% \end{aligned}
% \]

% \[
% \begin{aligned}
% m_2(\{a_1\}) &= 0.4 \\
% m_2(\{a_2\}) &= 0.3 \\
% m_2(\{a_3\}) &= 0.1 \\
% m_2(\{a_1, a_3\}) &= 0.2 \\
% \end{aligned}
% \]

% \[
% \begin{aligned}
% m_3(\{a_1\}) &= 0.3 \\
% m_3(\{a_2\}) &= 0.3 \\
% m_3(\{a_3\}) &= 0.2 \\
% m_3(\{a_2, a_3\}) &= 0.2 \\
% \end{aligned}
% \]

% \[
% \begin{aligned}
% m_4(\{a_1\}) &= 0.2 \\
% m_4(\{a_2\}) &= 0.4 \\
% m_4(\{a_3\}) &= 0.1 \\
% m_4(\{a_1, a_2\}) &= 0.3 \\
% \end{aligned}
% \]

% \begin{table*}[h!]
% \centering
%  % \resizebox{\textwidth}{!}{ % Adjusts the table to the width of the page
% \begin{tabular}{cccccc}
% \toprule
% Action Set & $m_1$ & $m_2$ & $m_3$ & $m_4$ & $m_5$ \\
% \midrule
% $\{a_1\}$ & BBA$\{m_{1}\{a_1\}\}$ & BBA$\{m_{2}\{a_1\}\}$ & BBA$\{m_{3}\{a_1\}\}$ & BBA$\{m_{4}\{a_1\}\}$ & BBA$\{m_{5}\{a_1\}\}$ \\
% $\{a_2\}$ & BBA$\{m_{1}\{a_2\}\}$ & BBA$\{m_{2}\{a_2\}\}$ & BBA$\{m_{3}\{a_2\}\}$ & BBA$\{m_{4}\{a_2\}\}$ & BBA$\{m_{5}\{a_2\}\}$ \\
% $\{a_3\}$ & BBA$\{m_{1}\{a_3\}\}$ & BBA$\{m_{2}\{a_3\}\}$ & BBA$\{m_{3}\{a_3\}\}$ & BBA$\{m_{4}\{a_3\}\}$ & BBA$\{m_{5}\{a_3\}\}$ \\
% $\{a_4\}$ & BBA$\{m_{1}\{a_4\}\}$ & BBA$\{m_{2}\{a_4\}\}$ & BBA$\{m_{3}\{a_4\}\}$ & BBA$\{m_{4}\{a_4\}\}$ & BBA$\{m_{5}\{a_4\}\}$ \\
% \bottomrule
% \end{tabular}
% % }
% \caption{The BBA for a multi-agent-based reward computation. The sum of the columns should be 1.}
% \label{table:bba}
% \end{table*}

Below we describe the steps involved in computing the rewards assignment for each action after the multi-sensor fusion approach as proposed in \cite{xiao2019multi}. 
\begin{enumerate}
\item \textbf{Construct the distance measure matrix:}

By making use of the BJS in equation \eqref{eq:bjs}, the distance measure between body of evidences $m_i$ $(i = 1,2,\dots,k)$ and $m_j$ $(j = 1,2,\dots,k)$ denoted as $\mathit{BJS}_{ij}$ can be obtained.
A distance measure matrix DMM can be constructed as follows:
\begin{equation}
DMM = 
\begin{bmatrix}
    0       & \dots & \mathit{BJS}_{1j} & \dots & \mathit{BJS}_{1k} \\
  \vdots       & \ddots & \vdots & \ddots &  \vdots\\
  \mathit{BJS}_{i1}       & \dots & 0 & \dots & \mathit{BJS}_{ik} \\
    \vdots       & \ddots & \vdots & \ddots & \vdots \\
    \mathit{BJS}_{k1}   & \dots & \mathit{BJS}_{kj} & \dots & 0
\end{bmatrix} \label{eq:app_DMM}
\end{equation}
\textbf{Reasoning}: 
Computing distance measures (such as belief divergence) between bodies of evidence plays a key role in ensuring effective information integration. Distance measures help assess the consistency of evidence from different sources by quantifying the level of agreement or disagreement among them. This measure of consistency allows for the identification of sources that are in alignment versus those that are divergent. Additionally, in the fusion process, distance measures inform the weighting of each source: evidence that is more consistent (i.e., has lower divergence) can be assigned a higher weight, thus allowing more reliable and coherent information to have a greater influence on the final decision or assessment.

\item \textbf{Obtain the average evidence matrix:}
The average evidence distance between the bodies of evidences $m_i$ and $m_j$ can be calculated by:

\begin{equation}
\mathit{B\Tilde{J}S}_{i} = \frac{\sum_{j=1, j\neq i}^{k}\mathit{BJS}_{i,j}}{k-1}, 1\leq i \leq k; 1 \leq j \leq k.
\label{eq:AEJS}
\end{equation}
\item \textbf{Calculate the support degree of the evidence:}
The support degree $Sup_i$ of the body of evidence $m_i$ is defined as follows:
\begin{equation}
Sup_{i} = \frac{1}{\mathit{B\Tilde{J}S}_{i}}, 1\leq i \leq k.
% \label{eq:AEJS}
\end{equation}
\item \textbf{Compute the credibility degree of the evidence:}
The credibility degree $Crd_i$ of the body of the evidence $m_i$ is defined as follows:
\begin{equation}
    Crd_i = \frac{Sup(m_i)}{\sum_{s=1}^{k}{Sup(m_s)}} ,\quad 1\leq i \leq k.
\label{eq:CRD}
\end{equation}
\item \textbf{Measure the belief entropy of the evidence:}
The belief entropy of the evidence $m_i$ is calculated by:
\begin{equation}
    E_d = - \sum_i m(A_i) \log \frac{m(A_i)}{2^{|A_i|} - 1}. 
\end{equation}
\item \textbf{Measure the information volume of the evidence:}
In order to avoid allocating zero weight to the evidences in some cases, we use the information volume $IV_i$ to measure the uncertainty of the evidence $m_i$ as below:
\begin{equation}
    IV_i = e^{E_d} = e^{- \sum_i m(A_i) \log \frac{m(A_i)}{2^{|A_i|} - 1}} ,\quad 1\leq i \leq k.
\end{equation}

\item \textbf{Normalize the information volume of the evidence:}
The information volume of the evidence $m_i$ is normalized as below, which is denoted as 
$\Tilde{I}V_i$:
\begin{equation}
    \Tilde{I}V_i = \frac{IV_i}{\sum_{s=1}^k IV_s} ,\quad 1\leq i \leq k.
\end{equation}
\item \textbf{Adjust the credibility degree of the evidence:}
Based on the information volume $\Tilde{I}V_i$ the credibility degree $Crd_i$ of the evidence $m_i$ will be adjusted, denoted as $ACrd_i$:
\begin{equation}
    ACrd_i = Crd_i \times \Tilde{I}V_i ,\quad 1\leq i \leq k.
\end{equation}
\item \textbf{Normalize the adjusted credibility degree of the evidence:}
The adjusted credibility degree which is denoted as $ \Tilde{A}Crd_i$ 
 is normalized that is considered as the final weight in terms of each evidence $m_i$:
\begin{equation}
    \Tilde{A}Crd_i = \frac{ACrd_i}{\sum_{s=1}^k ACrd_s} ,\quad 1\leq i \leq k.
\end{equation}
\item \textbf{Compute the weighted average evidence:}
On account of the final weight $\Tilde{A}Crd_i$ of each evidence $m_i$, the weighted average evidence $\mathit{WAE}(m)$ will be obtained as follows:
\begin{equation}
    \mathit{WAE}(m) = \sum_{i=1}^k (\Tilde{A}Crd_i \times m_i) ,\quad 1\leq i \leq k.
\end{equation}
\item \textbf{Combine the weighted average evidence by utilizing the Dempster's rule of combination:}
The weighted average evidence $\mathit{WAE}(m)$ is fused via the Dempster’s combination rule:
\begin{equation}
m_{\text{combined}}(C) = \frac{\sum_{A \cap B = C} m_1(A) \cdot m_2(B)}{1 - \sum_{A \cap B = \emptyset} m_1(A) \cdot m_2(B)}
\label{eq:app_BPA}
\end{equation}
by $(k-1)$ times, if there are k number of evidences. Then, the final combination result of multi-evidences can be obtained.
\item \textbf{Converting probabilities to reward:}
The penultimate combined belief for each action that is denoted as $ m_{\text{combined}}(C)$ is normalized and considered as the final reward.  

\begin{equation}
    \mathit{BPA}_{a_i} = \frac{m_{\text{combined}}(a_j)}{\sum_{j=1}^km_{\text{combined}}(a_j)},\quad 1\leq i \leq k.
\end{equation}
% \[
% BPA_{a_i} = (m_1 \oplus m_2 \oplus m_3 \oplus m_4 \oplus m_5)(\{a_i\}) ,\quad 1\leq i \leq k.
% \]


$\mathit{BPA}_{a_i}$ is the reward for the action $a_i$. 

\end{enumerate}

% \section{Pseudo-Code}
% \label{app:Pseudo_Code}
% Below we presents the pseudo-code of the AMULED framework. This algorithm employs PPO to iteratively update the policy and value function based on environmental feedback. The framework integrates reward shaping to balance primary and secondary objectives and incorporates fine-tuning through reinforcement learning with human-like feedback (RLHF) from moral clusters, using KL divergence and belief aggregation to guide agent behavior.

% \begin{algorithm}
% \caption{AMULED Framework}
% \begin{algorithmic}[1]

% \Require Set of moral clusters and initial policy parameters $\actorParams$, $\criticParams$
% \State Initialize policy $\pi_{\actorParams}$ and value function $V_{\criticParams}$ using Proximal Policy Optimization (PPO)~\cite{schulman2017proximal}
% \For{each episode}
%     \State Collect trajectories of state-action-reward tuples $(s, a, r)$ from the environment
%     \State Compute the advantage function $A^{\pi_{\actorParams}}(s, a)$ using Generalized Advantage Estimation (GAE)
    
%     \State \textbf{Update Policy}:
%     \State Update policy parameters $\actorParams$ by optimizing
%     \[
%     \actorParams_{k+1} = \arg\min_{\actorParams} \mathbb{E}_{t} \left[ \frac{\pi_{\actorParams}(a | s)}{\pi_{\actorParams_{k}}(a | s)} A^{\pi_{\actorParams_{k}}}(s, a) \cdot g(\epsilon, A^{\pi_{\actorParams_{k}}}(s, a)) \right]
%     \]
%     where $g(\epsilon, A)$ represents advantage normalization and value clipping.

%     \State \textbf{Update Value Function}:
%     \State Update value function parameters $\criticParams$ by minimizing the error:
%     \[
%     \criticParams_{k+1} = \arg \min_{\criticParams} \mathbb{E}_{t} \left[V_{\criticParams}(s_t) - R_t\right]^2
%     \]

%     \State \textbf{Reward Shaping}:
%     \State Define rewards at each timestep $t$ as:
%     \[
%     r_t = \baseReward + c \cdot \rewardShaping
%     \]
%     where $r_{\text{base}}$ incentivizes the primary goal and $\rewardShaping$ addresses secondary goals.

%     \If{Fine-tuning with Human Feedback}
%         \State Initialize base policy $\pi_{\text{base}}$ from previously trained parameters
%         \State Define new reward for fine-tuning as:
%         \[
%         r_{\text{base} = -\lambda_{\text{KL}} D_{\text{KL}}\left(\fineTuneModel(a | s) \parallel \baseModel(a | s)\right)
%         \]
%         \[
%         r_{\text{shaping}} = f_{\text{BA}}(\mathbf{B})\hspace{1cm}  \leftarrow \textbf{Eqs. \eqref{eq:app_DMM}--\eqref{eq:app_BPA}}
%         \]
%         where $\lambda_{\text{KL}}$ is a regularization coefficient, and matrix $\mathbf{B}$ represents the belief values from moral agents.
%     \EndIf

%     \State \textbf{Fine-tuning Training Loop}:
%     \For{$T_{\text{finetune}}$ timesteps}
%         \State Train the fine-tuned policy $\fineTuneModel$ using PPO with feedback rewards $r_{\text{base}}$ and $r_{\text{shaping}}$
%     \EndFor
% \EndFor

% \end{algorithmic}
% \end{algorithm}


% \section*{Multi-Morality Fusion Approach:}
% Steps involved are:
% \begin{enumerate}
%   \item \textbf{Input Data from Moral Theories:} Gather data from multiple moral theories or frameworks. Each theory provides its own evidence or belief about the morality of actions or decisions that can be taken.
  
%   \item \textbf{Construct Frame of Discernment:} For each moral theory, construct a frame of discernment based on the principles and values it espouses. These frameworks represent the uncertainty and confidence associated with the moral judgments provided by each theory.
  
%   \item \textbf{Compute Belief Divergence:} Calculate the belief divergence measure between pairs of frameworks from different moral theories. This step helps in understanding how different the moral judgments are across various theories.
  
%   \item \textbf{Weighted Fusion Using Divergence and Entropy:} Use the belief divergence measure and belief entropy to weight the fusion process. Moral theories with more similar judgments (lower divergence) or lower uncertainty (lower entropy) might be given higher weight in the fusion process.
  
%   \item \textbf{Combine Frame of Discernment:} Combine the frameworks from different moral theories using a fusion rule. This rule could be based on the weighted average, consensus, or other methods that take into account the divergence and entropy measures.
  
%   \item \textbf{Output Fused Frame of Discernment:} Obtain a fused frame of discernment that represents a more informed and robust assessment of the moral implications of actions or decisions than any individual moral theory could provide alone.
% \end{enumerate}

% \section{Calculate the morality degree of the actions}


% \[
% \begin{aligned}
% H(m_1) &= - [0.5 \log 0.5 + 0.2 \log 0.2 + 0.1 \log 0.1 + 0.2 \log 0.2] = 0.529 \\
% H(m_2) &= - [0.4 \log 0.4 + 0.3 \log 0.3 + 0.1 \log 0.1 + 0.2 \log 0.2] = 0.5558 \\
% H(m_3) &= - [0.3 \log 0.3 + 0.3 \log 0.3 + 0.2 \log 0.2 + 0.2 \log 0.2] = 0.5933 \\
% H(m_4) &= - [0.2 \log 0.2 + 0.4 \log 0.4 + 0.1 \log 0.1 + 0.3 \log 0.3] = 0.5558 \\
% \end{aligned}
% \]

% \section*{Credibility Degrees}

% \[
% \begin{aligned}
% \text{Cr}(m_1) &= \frac{1}{0.529} = 1.89 \\
% \text{Cr}(m_2) &= \frac{1}{0.5558} = 1.8 \\
% \text{Cr}(m_3) &= \frac{1}{0.5933} = 1.69 \\
% \text{Cr}(m_4) &= \frac{1}{0.5558} = 1.8 \\
% \end{aligned}
% \]

% \section*{Combined Moral Functions}

% Using Dempster's rule of combination, we combine the morality functions \(m_1\) to \(m_4\):

% \[
% (m_1 \oplus m_2 \oplus m_3 \oplus m_4)(\{a_1\}) = 0.5 \times 0.4 \times 0.3 \times 0.2 = 0.012
% \]

% \[
% (m_1 \oplus m_2 \oplus m_3 \oplus m_4)(\{a_2\}) = 0.2 \times 0.3 \times 0.3 \times 0.4 = 0.0072
% \]

% \[
% (m_1 \oplus m_2 \oplus m_3 \oplus m_4)(\{a_3\}) = 0.1 \times 0.1 \times 0.2 \times 0.1 = 0.0002
% \]

% Normalizing the credence under all relevant moralities for action $a_1,a_2, a_3$ of 0.012, 0.0072, and 0.0002 are 0.6186, 0.3711, and 0.0103, respectively. 


% We use the computed final credence value for each action as the intrinsic reward for the agent. Specifically, if the agent takes action $a_1$, it receives a reward of 0.6186. For $a_2$, the reward is 0.3711, and for $a_3$, it is 0.0103.


% \begin{figure*}[htbp]
%   \centering
%   \includegraphics[width=1\linewidth]{images/1-s2.0-S1566253517305584-gr2.jpg}
%   \caption{The flowchart of the proposed method \cite{xiao2019multi}}
%   \label{fig:BJS}
% \end{figure*}





% \section{Notes for understanding BJS}

% Step 1: Compute Belief Jensen–Shannon divergence measure matrix, namely, a distance measure matrix. 


% Reasoning: In the context of evidence theory, particularly in scenarios involving multi-sensor data fusion or combining information from multiple sources, computing distance measures (such as belief divergence measures) between bodies of evidence serves several important purposes:

% Assessing Consistency: Different sensors or sources may provide evidence or beliefs about the same phenomenon, but they might not always agree. Computing distance measures helps to quantify how much different bodies of evidence diverge or disagree with each other. This provides a measure of consistency or inconsistency between different sources of information.

% Weighting in Fusion Processes: When fusing information from multiple sources, it's crucial to consider the reliability and consistency of each source. Bodies of evidence that are more consistent with each other (i.e., have lower divergence measures) can be given higher weights in the fusion process. This ensures that more reliable and coherent information contributes more to the final decision or assessment.

% Step 2: The average evidence distance 

% Reasoning: By calculating the average evidence distance, you can obtain a single numerical value that represents the average dissimilarity between all pairs of bodies of evidence. This measure provides an overall assessment of the consistency or inconsistency among the sources of evidence.

% Step 3: The support degree of the body of evidence.

% Reasoning: The support degree quantitatively expresses the level of confidence or belief that a body of evidence assigns to a specific hypothesis or proposition. It provides a numerical measure indicating how strongly the evidence supports the hypothesis relative to other possible hypotheses.

% Step 4: The credibility degree of the body of the evidence

% Reasoning: The credibility degree provides a quantitative measure of how reliable or trustworthy the body of evidence is perceived to be. It helps in distinguishing between more reliable and less reliable sources of information.

% Step 5: Measure the information volume of the evidences

% Reasoning: The "information volume" of evidence refers to a measure that quantifies the amount or volume of information conveyed by a body of evidence. Here’s how you can understand and measure the information volume of evidences. 
% Measuring the information volume of evidences involves calculating the entropy weighted by the belief assignments across all subsets of the frame of discernment. This measure provides a quantitative assessment of the richness and diversity of information conveyed by the evidence, aiding in decision making and evidence fusion processes within evidence theory.

% Step 6: Generate and fuse the weighted average evidence

% Reasoning: involves combining information from multiple sources or bodies of evidence in a manner that accounts for their respective strengths or reliability. 






\end{document}
