\pdfoutput=1
\documentclass[11pt]{article}
% \usepackage[review]{acl}
\usepackage{acl}
%%%%% NEW MATH DEFINITIONS %%%%%

% \usepackage{amsmath,amsfonts,bm}
\usepackage{amsmath,amsfonts}

\usepackage{pifont}


\newcommand{\R}{\mathbb{R}}


\def\va{{\mathbf{a}}}
\def\vg{{\mathbf{g}}}

% Sets
\def\sR{\mathbb{R}}
\def\sC{\mathbb{C}}
\def\sZ{\mathbb{Z}}
\def\sN{\mathbb{N}}
\def\sQ{\mathbb{Q}}

\def\sS{\mathcal{S}}



% Vectors
\def\vzero{{\mathbf{0}}}
\def\vone{{\mathbf{1}}}
\def\vmu{{\mathbf{\mu}}}
\def\vtheta{{\mathbf{\theta}}}
\def\va{{\mathbf{a}}}
\def\vb{{\mathbf{b}}}
\def\vc{{\mathbf{c}}}
\def\vd{{\mathbf{d}}}
\def\ve{{\mathbf{e}}}
\def\vf{{\mathbf{f}}}
\def\vg{{\mathbf{g}}}
\def\vh{{\mathbf{h}}}
\def\vi{{\mathbf{i}}}
\def\vj{{\mathbf{j}}}
\def\vk{{\mathbf{k}}}
\def\vl{{\mathbf{l}}}
\def\vm{{\mathbf{m}}}
\def\vn{{\mathbf{n}}}
\def\vo{{\mathbf{o}}}
\def\vp{{\mathbf{p}}}
\def\vq{{\mathbf{q}}}
\def\vr{{\mathbf{r}}}
\def\vs{{\mathbf{s}}}
\def\vt{{\mathbf{t}}}
\def\vu{{\mathbf{u}}}
\def\vv{{\mathbf{v}}}
\def\vw{{\mathbf{w}}}
\def\vx{{\mathbf{x}}}
\def\vy{{\mathbf{y}}}
\def\vz{{\mathbf{z}}}
\def\vzeta{{\mathbf{\zeta}}}

% Matrix
\def\mA{{\mathbf{A}}}
\def\mB{{\mathbf{B}}}
\def\mC{{\mathbf{C}}}
\def\mD{{\mathbf{D}}}
\def\mE{{\mathbf{E}}}
\def\mF{{\mathbf{F}}}
\def\mG{{\mathbf{G}}}
\def\mH{{\mathbf{H}}}
\def\mI{{\mathbf{I}}}
\def\mJ{{\mathbf{J}}}
\def\mK{{\mathbf{K}}}
\def\mL{{\mathbf{L}}}
\def\mM{{\mathbf{M}}}
\def\mN{{\mathbf{N}}}
\def\mO{{\mathbf{O}}}
\def\mP{{\mathbf{P}}}
\def\mQ{{\mathbf{Q}}}
\def\mR{{\mathbf{R}}}
\def\mS{{\mathbf{S}}}
\def\mT{{\mathbf{T}}}
\def\mU{{\mathbf{U}}}
\def\mV{{\mathbf{V}}}
\def\mW{{\mathbf{W}}}
\def\mX{{\mathbf{X}}}
\def\mY{{\mathbf{Y}}}
\def\mZ{{\mathbf{Z}}}
\def\mBeta{{\mathbf{\beta}}}
\def\mPhi{{\mathbf{\Phi}}}
\def\mLambda{{\mathbf{\Lambda}}}
\def\mSigma{{\mathbf{\Sigma}}}


% Expectation
% \def\eE{\mathop{\mathbb{E}}\limits}
\def\eE{\mathbb{E}}

% Probability
\def\pP{\mathbb{P}}

% Tilde
\def\tf{\tilde{f}}
\def\tS{\tilde{S}}
\def\wtF{\widetilde{\mathcal{F}}}
\def\whR{\widehat{R}}
\def\tvx{\tilde{\mathbf{x}}}
\def\ty{\tilde{y}}


\def\defeq{\overset{\textup{def}}{=}}
% \def\defeq{\overset{.}{=}}
\def\defone{\overset{\text{\ding{172}}}{=}}
\def\deftwo{\overset{\text{\ding{173}}}{=}}
\def\leqone{\overset{\text{\ding{172}}}{\leq}}
\def\leqtwo{\overset{\text{\ding{173}}}{\leq}}
\def\leqthree{\overset{\text{\ding{174}}}{\leq}}
\def\leqfour{\overset{\text{\ding{175}}}{\leq}}
\def\eqone{\overset{\text{\ding{172}}}{=}}
\def\eqtwo{\overset{\text{\ding{173}}}{=}}
\def\eqthree{\overset{\text{\ding{174}}}{=}}
\def\eqfour{\overset{\text{\ding{175}}}{=}}
\def\geqfive{\overset{\text{\ding{176}}}{\geq}}

\usepackage{times}
\usepackage{latexsym}
\usepackage[T1]{fontenc}
\usepackage[utf8]{inputenc}


\usepackage{hyperref}
\usepackage{url}
\newtheorem{theorem}{Theorem}
\usepackage{amsmath,amssymb,amsfonts}
\usepackage[ruled,vlined,noend]{algorithm2e}
\usepackage{graphicx}
\usepackage{bbm}
\usepackage{textcomp}
\usepackage{wrapfig}
\usepackage{xcolor}
\usepackage{booktabs}       % professional-quality tables
\usepackage{amsfonts}       % blackboard math symbols
\usepackage{nicefrac}       % compact symbols for 1/2, etc.
\usepackage{microtype}      % microtypography
% \usepackage{xcolor}         % colors
\usepackage{enumitem}
% \usepackage[dvipsnames]{xcolor}
\usepackage{multirow,tabularx}
\usepackage{amsmath}
\usepackage{graphicx}
\usepackage{amssymb,amsthm}
\usepackage{inconsolata}
\usepackage{epsfig}
\usepackage{makecell}
\usepackage{subfigure}
\theoremstyle{definition}
\newtheorem{definition}{Definition}[section]
\usepackage{bm}


\title{Knapsack Optimization-based Schema Linking for \\ LLM-based Text-to-SQL Generation}


\author{
Zheng Yuan\textsuperscript{1}, Hao Chen\textsuperscript{2}, Zijin Hong\textsuperscript{1}, Qinggang Zhang\textsuperscript{1} \\
{\bf Feiran Huang\textsuperscript{3}, Xiao Huang\textsuperscript{1}}\\ 
\textsuperscript{1}The Hong Kong Polytechnic University\\ 
\textsuperscript{2}City University of Macau, \textsuperscript{3}Jinan University \\
\texttt{\{yzheng.yuan, zijin.hong, qinggangg.zhang\}@connect.polyu.hk} \\
\texttt{sundaychenhao@gmail.com}; \texttt{huangfr@jnu.edu.cn}; \texttt{xiao.huang@polyu.edu.hk}
}

\makeatletter
\newcommand\figcaption{\def\@captype{figure}\caption}
\newcommand\tabcaption{\def\@captype{table}\caption}
\makeatother

\newcommand{\fix}{\marginpar{FIX}}
\newcommand{\new}{\marginpar{NEW}}

\begin{document}
\maketitle
\begin{abstract}
Generating SQLs from user queries is a long-standing challenge, where the accuracy of initial schema linking significantly impacts subsequent SQL generation performance.  However, current schema linking models still struggle with missing relevant schema elements or an excess of redundant ones. A crucial reason for this is that commonly used metrics, recall and precision, fail to capture relevant element missing and thus cannot reflect actual schema linking performance. Motivated by this, we propose an enhanced schema linking metric by introducing a \textbf{restricted missing indicator}.
Accordingly, we introduce \textbf{\underline{K}n\underline{a}psack optimization-based \underline{S}chema \underline{L}inking \underline{A}gent (KaSLA)}, a plug-in schema linking agent designed to prevent the missing of relevant schema elements while minimizing the inclusion of redundant ones.
KaSLA employs a hierarchical linking strategy that first identifies the optimal table linking and subsequently links columns within the selected table to reduce linking candidate space.
In each linking process, it utilize a knapsack optimization approach to link potentially relevant elements while accounting for a limited tolerance of potential redundant ones.
With this optimization, KaSLA-1.6B achieves superior schema linking results compared to large-scale LLMs, including deepseek-v3 with state-of-the-art (SOTA) schema linking method. Extensive experiments on Spider and BIRD benchmarks verify that KaSLA can significantly improve the SQL generation performance of SOTA text-to-SQL models by substituting their schema linking processes.
\end{abstract}
%具身智能体在复杂场景下 manipulation 的 performance robustness 和泛化能力始终是一个广受关注的研究方向。其中,visuomotor imitation learning 是具身智能体 Policy 的主流范式之一,它允许 agent 从高维视觉观察和机器人本体感知中 effectively 学习 manipulation skills。
%然而,增加场景的复杂度和 visual distraction,会导致在简单场景下表现良好的决策模型性能下降。实际上,不仅是 simple imitation learning policy,先进的多模态 foundation models such as GPT-4o 或 vision language action models (VLA),也不能很好地关注一张语义丰富的图片中的特定的局部问题。对于 robot control or 多模态大模型,其往往侧重于 action prediction, observation mapping or 多模态 alignment,而缺少直观的视觉感知增强。模型需要隐性地或遵循 high-level text instruction 从相关的视觉区域中获得面向任务语义的定位知识。
%To tackle this challenge problem, we introduce Imit Diff, a diffusion transformer imitation learning framework with dual resolution enhancement guided by fine-grained semantics information。具体来说,our work 有三个关键组成部分。
%1) Semanstic Injection. Imit Diff 通过 vision language models (VLM) 和 vision foundation models 的 pretrain knowledge 将面向任务的语义信息和高层文本指导转化为显式的 pixel-level 视觉定位标签,注入到 environment observation中。
%2) Dual Res Fusion。 我们构建了双分辨率图像观测流,使用双分辨率视觉编码器分别提取全局和细粒度视觉特征。多尺度视觉信息随后在 attention block 中进行融合,在保证计算 effiency 的前提下,为全局视觉观测引入多尺度细粒度信息,提升场景理解能力。
%3) Consistency policy on diffusion transformer。Diffusion based imitation policies 通常受到 denoise times 的困扰。我们建立了基于 consistency policy 的 DiT action head。Policy 的决策层可以通过 single step denoise 实现系统高频响应。额外地,受益于较快的 inference time,我们引入 temperal ensemble 改善预测动作的平滑性。
%我们设计了四个在 manipulation 精细度上具有挑战性的现实世界任务来评估 Imit Diff,并通过增加场景复杂度和 visual distraction 来测试模型的场景理解能力。额外地,我们设计了 visual distraction 和 category generalization 的 zero shot 实验来验证模型是否受益于 dual res enhancement framework and fine-grained semantics injection。实验结果表明,Imit Diff outperforms 现有的 strong baselines。
%In summary, the contributions of our work are three-fold:
%1) We propose Imit Diff, a DiT architecture imitation learning framework with dual res enhancement guied by fine-grained semantics information.
%2) 我们构建了 open-set vision foundation models pipeline 来获得显式视觉遮罩。该方法能够有效处理机器人控制场景的运动模糊、遮挡、物体丢失情况。并将其作为 fine-grained 语义信息引导 policy decision。
%3) 我们在DiT上实现了consistency policy,显著减少了模型推理时间。通过异步控制框架,实现了 open-set vision foundation models 工作流下的实时控制。
%The code will be publicly available soon。

\section{Introduction}


\label{Intro}
The performance robustness and generalization capabilities of embodied agents in complex manipulation scenarios have long been a focus of significant research interest \citep{ju2025robo, yuan2024learning}. Visuomotor imitation learning is one of the mainstream paradigms of robot manipulation policy \citep{chi2023diffusion, shridhar2023perceiver, ze2023gnfactor, florence2022implicit, hansen2022pre}. This approach enables agents to derive state estimation and decision-making capabilities from expert demonstrations that incorporate high-dimensional visual observations and robot proprioception \citep{ze20243d}.

However, as scene complexity and visual distractions increase, the performance of decision models that excel in simpler environments tends to degrade \citep{zheng2024instruction, liurobustness}. Not only do simple imitation learning policies face challenges, but even advanced multimodal foundation models, such as GPT-4o \citep{hurst2024gpt} or vision language action models (VLA) \citep{liu2024rdt, brohan2022rt, brohan2023rt, o2023open, kim2024openvla, wen2024diffusion}, struggle to accurately focus on specific details within semantically complex images. In fact, in robot control and embodied multimodal foundation models, the focus is often on action prediction, observation mapping, or multimodal alignment. Therefore, intuitive visual perception enhancement is typically lacking. Models can only acquire task-oriented semantic localization knowledge from relevant visual regions either implicitly or when guided by high-level text instructions \citep{reuss2023multimodal}.

To tackle this challenge problem, we introduce \textbf{Imit Diff}, a diffusion transformer imitation learning framework with dual resolution enhancement guided by fine-grained semantics information. Specifically, our work has three key components:

\begin{enumerate}

\item \textbf{Semanstic injection.} Imit Diff transforms task-oriented semantic information and high-level textual guidance into explicit pixel-level visual localization labels through the pretrain knowledge of vision language models (VLM) and vision foundation models, and injects them into the policy observation.

\item \textbf{Dual resolution (dual res) fusion.} We develop a dual res image observation stream and employed a dual res vision encoder to extract global and fine-grained visual features. The extracted multi-scale visual information is subsequently fused within an attention block, integrating fine-grained details into the global visual feature. This approach enhances scene understanding while maintaining computational efficiency.

\item \textbf{Consistency policy on diffusion transformer (DiT).} Diffusion-based imitation policies often suffer from inefficiencies due to the required denoising steps. To address this, we design a DiT \citep{peebles2023scalable} action head incorporating a consistency policy \citep{song2023consistency}, enabling the decision layer to achieve high-frequency system responses through single-step denoising. Furthermore, leveraging faster inference times, we introduce temperal ensemble to enhance the smoothness of predicted actions.

\end{enumerate}

We design four real-world tasks with challenging manipulation precision to evaluate Imit Diff and test the model's scene understanding capabilities by introducing increased scene complexity and visual distractions. Additionally, we conducted zero-shot experiments on visual distraction and category generalization to assess the benefits of the dual res enhancement framework and fine-grained semantic injection. Experimental results demonstrate that Imit Diff significantly outperforms existing strong baselines. 

In summary, the contributions of our work are three-fold:

\begin{enumerate}

\item We propose Imit Diff, a DiT architecture imitation learning framework with dual res enhancement guied by fine-grained semantics information.

\item We developed an open-set vision foundation model pipeline to generate explicit visual masks. This approach effectively addresses challenges such as motion blur, occlusion, and object loss in robot control scenarios, leveraging the generated masks as fine-grained semantic information to guide policy decisions.

\item We implemented a consistency policy on DiT, which significantly reduced the model inference time. Through the asynchronous control framework, we achieved real-time control under the workflow of open-set vision foundation models.

\end{enumerate}

The code will be made publicly available soon.


\section{Preliminaries}
\label{sec:preliminary}
\newcommand{\trg}{\mathbf{z}^{\text{trg}}}
\newcommand{\trgt}{\mathbf{z}_{t}^{\text{trg}}}
\newcommand{\src}{\mathbf{z}^{\text{src}}}
\newcommand{\srct}{\mathbf{z}_{t}^{\text{src}}}
\newcommand{\srctopt}{\mathbf{z}_{t}^{\text{src}\ast}}
\newcommand{\trgtopt}{\mathbf{z}_{t}^{\text{trg}\ast}}
\newcommand{\txts}{y^{\text{src}}} % source prompt
\newcommand{\txtt}{y^{\text{trg}}} % target prompt
\newcommand{\dds}{\text{DDS}}
\newcommand{\sds}{\text{SDS}}
\newcommand{\ids}{\text{IDS}}
\newcommand{\loss}[1]{\mathcal{L}_{#1}}
\newcommand{\grad}[1]{\nabla_{#1}}
\newcommand{\pardiff}[2]{\frac{\partial #1}{\partial #2}}
\newcommand{\postmean}{\tilde{\mathbf{z}}_{0}}

\subsection{Diffusion Model and Sampling Guidance}
Text-to-image diffusion models $\epsilon_\phi(\cdot)$ are based on diffusion probabilistic models (DPMs)~\cite{ho2020denoising, song2020score, rombach2022high}. %, which are latent variable models based on Markov chain with Gaussian distribution to synthesize the data from the distribution of the given training datasets. 
The models are trained to estimate the denoising score when the original image $\mathbf{z}_0$ and the text condition $y$ are given:
\begin{equation*}
    \mathcal{L}(\phi) = \mathbb{E}_{t, \epsilon}
    [\lVert 
    \epsilon_\phi(\mathbf{z}_t, y, t) - \epsilon 
    \rVert_2^2],
\end{equation*}
where $\epsilon\sim\mathcal{N}(0, \mathbf{I})$ and $t\sim\mathcal{U}(0, 1)$. $\mathbf{z}_t$ refers to the stochastic latent of $\mathbf{z}_0$ via the forward diffusion process as follows:
\begin{equation} \label{eq:forward}
    \mathbf{z}_t=\sqrt{\alpha_t}\mathbf{z}_0+\sqrt{1-\alpha_t}\epsilon,
\end{equation}
where $\alpha_t$ is noise schedule. With the trained $\epsilon_\phi(\cdot)$, high-quality samples can be generated using the classifier-free guidance (CFG) \cite{ho2021classifier} by subtracting unconditioned denoising score from the conditioned score with guidance scale $\omega$:
\begin{equation} \label{eq:cfg}
    \epsilon_\phi^\omega(\mathbf{z}_t, y, t) 
    = (1+\omega)\epsilon_\phi(\mathbf{z}_t, y, t) 
    -\omega\epsilon_\phi(\mathbf{z}_t, \varnothing, t).
\end{equation}

\subsection{Score Distillation Sampling (SDS)}
% \label{sec:3.2}
With pretrained text-to-image diffusion models $\epsilon_\phi(\cdot)$, SDS \cite{poole2022dreamfusion} synthesizes 3D data $\mathbf{z}$ for a given text prompt $y$ by optimizing the differentiable rendering function paremetrized by $\theta$, where $\mathbf{z}=g(\theta)$:
{\small
\begin{align}
    \grad{\theta}\loss{\sds}(\mathbf{z}, y)
    &= \mathbb{E}_{t, \epsilon} 
    \left[ \omega(t)(\epsilon_\phi^\omega(\mathbf{z}_t, y, t)-\epsilon)\pardiff{\mathbf{z}}{\theta}\right].
    \label{eq:sds}
\end{align}
}
%where $t\sim\mathcal{U}(0, 1)$ and $\epsilon\sim\mathcal{N}(0, \mathbf{I})$. 
The optimized parameters $\theta^*$ provide the text-conditioned 3D volume that follows the diffusion prior~\cite{poole2022dreamfusion}. % distribution of pretrained diffusion models. 
However, a single text prompt $y$ can refer to many different 3D volumes, each with diverse backgrounds or structural details of the object. Therefore, an inherent limitation of SDS \cite{poole2022dreamfusion} is that the score conditioned by the prompt $y$ does not always provide the diffusion prior to the identical object during the optimization process, leading to blurry and unclear results. 
%When it comes to image editing, the SDS loss in \eqref{eq:sds} is computed with respect to the data itself $\mathbf{z}=\trg$ and the target text prompt $\txtt$. In particular, $\trg$ is first set as the input image $\src$. Note that the shortcomings of SDS in image editing are even more severe, as there is no guidance regarding the source image $\src$ during the updates. As shown in Fig. \ref{fig:inversion}, the edited image $\trg$ is not only blurred, but also contains the heavily altered background and structure details that are not included in the prompt $\txtt$.

\subsection{Delta Denoising Score (DDS)}
DDS \cite{hertz2023delta} is proposed to synthesize the image $\trg$ from the given source image $\src$ and its corresponding prompt $\txts$, which is aligned to the target prompt $\txtt$. 
Based on the insight that the gradient should be zero if $\txtt$ matches $\txts$, DDS minimizes the identity change of $\src$ by simple replacing $\epsilon$ in \eqref{eq:sds} with the score $\epsilon_\phi^\omega(\srct, \txts, t)$] as follows:
{\small
\begin{equation} \label{eq:dds}
% \begin{split}
    \grad{\theta}\loss{\dds}
     = \mathbb{E}_{t, \epsilon} 
    \left[ 
      (\epsilon_\phi^\omega(\trgt, \txtt, t) - {\epsilon}_{\phi}^\omega(\srct, \txts, t)) \pardiff{\trg}{\theta}\right]. %\nonumber\\
    % = \grad{\theta}\loss{\sds}(\trg, \txtt) - \grad{\theta}\loss{\sds}(\src, \txts).
% \end{split}
\end{equation}
}
%where the same $\epsilon$ is used to generate $\trgt$ and $\srct$. 
For simplicity, we denote $\epsilon_\phi^{\text{trg}}=\epsilon_\phi^\omega(\trg, \txtt, t)$ and $\epsilon_\phi^{\text{src}}=\epsilon_\phi^\omega(\src, \txts, t)$. 
Here, $\epsilon_\phi^{\text{trg}}$ and  ${\epsilon}_\phi^{\text{src}}$ can be interpreted as the gradients representing the direction from $\trgt$ to $\trg$ and the direction from $\srct$ to $\src$, respectively.
$\theta = \trg$ is thus gradually optimized along the direction from $\src$ to $\trg$, as shown in \cref{fig:algorithm}.
It is worth noting that the guidance of the update can be calculated at the same point $\trgt$, thanks to the shared $\epsilon$.
%DDS reduces the uncertainty caused by SDS loss and provides more clear results.
However, the slight error in the gradient caused by the score ${\epsilon}_\phi^{\text{src}}$ still leads to the incorrect direction for the optimization. 


\subsection{Fixed-point Iteration}

% \add{Explanation of fixed-point iteration and applications of fixed-point iteration to the diffusion model}
In numerical analysis, a fixed-point iteration \cite{parikh2014proximal} is an iterative method to find fixed points of a function $f$, where $f(x) = x.$ Given an initial point $x_0$, the iteration is defined as: 
\begin{equation*}
x_{n+1} = f(x_n), \quad n = 0, 1, 2, \ldots
\end{equation*}
Under appropriate conditions, this sequence converges to a fixed point. Thanks to its applicability to non-linear problems with low computational costs, fixed-point iteration is widely used in optimization, including applications in the context of diffusion models \cite{meiri2023fixed}. 

%These can also be applied to text-guided diffusion models, potentially enhancing their performance. Null-text Inversion \cite{mokady2023null} replaces the traditional approach of mapping all noise vectors to a single image using random noise for each optimization iteration with a more local optimization that employs a single noise vector. 
%Specifically, drawing inspiration from GAN literature, it leverages the sequence of noised latent codes obtained from an initial DDIM inversion as a pivot. The optimization is then performed around this pivot, leading to a more refined and accurate inversion. However, since this method relies on a fixed pivot, it struggles to fully preserve the complex structural consistency of the original image. We propose a novel model that utilizes fixed-point iteration to correct the gradient direction aligned with the source prompt, thereby ensuring that the structure of the source image is preserved while selectively editing the target image.



\section{Enhanced Schema Linking Metrics}
\label{sec: Enhanced Schema Linking Metrics}
In this section, we introduce enhanced schema linking metrics considering the missing of relevant elements during evaluation, thereby accurately assess the true performance of schema linking models.

For a query $q$ with predicted schema linking result $\widehat{\mS}_q$ and ground truth linking result $\mS^*_q$, Recall and Precision are two commonly used metreics~\citep{li2023resdsql,li2024codes,qu2024before}. Recall, $\mR(q) =\frac{|\widehat{\mS}_q \cap \mS^*_q|}{|\mS^*_q|}$,  measures the proportion of relevant elements correctly identified, specifically how many of the actual relevant elements in $\mS^*_q$ are found in $\widehat{\mS}_q$. Conversely, Precision, $\mP(q) = \frac{|\widehat{\mS}_q \cap \mS^*_q|}{|\widehat{\mS}_q|}$, assesses the accuracy of predictions by determining the proportion of elements predicted as relevant in $\widehat{\mS}_q$ that are indeed correct.

As illustrated in Figure~\ref{fig:motivation}, these metrics can yield high scores even in the missing of relevant elements. This underscores that they do not effectively address the issue of missing relevant elements, which can undermine the true accuracy of schema linking models.

To address the limitations of Recall and Precision, we firstly introduce a restricted element missing indicator to detect whether relevant elements are missing in the predicted schema linking result $\widehat{\mS}_q$, The formal definition is as follows:
\begin{equation} \label{equ: indicator function}
\bm{\mathbbm{1}}(q) = 
\begin{cases} 
1, & \text{if } \widehat{\mS}_q \cap \mS^*_q = \mS^*_q, \\
0, & \text{otherwise}.
\end{cases}\end{equation}
It takes a value of 1 when the predicted result $\widehat{\mS}_q$ contain all the elements in the ground truth $\mS^*_q$, meaning no relevant elements are missing. Conversely, it is set to 0 when there are any missing. Building on this missing indicator, we define enhanced Recall and Precision as follows:
\begin{equation} \label{equ: recall plus}
\mR^{\text{\textbf{+}}}(q) = \bm{\mathbbm{1}}(q)   \mR(q), \mP^{\text{\textbf{+}}}(q) = \bm{\mathbbm{1}}(q)  \mP(q).
\end{equation}
Such metrics strengthen the penalty for missing relevant elements, as schema linking results with missing are inherently incorrect for SQL generation. For simplicity, we use the harmonic mean of enhanced Recall and Precision, referred to as the enhanced F1 score, as the overall measure for evaluating schema linking results:
\begin{equation} \label{equ: traditional F1 for one question}
{\mF1}^{\text{\textbf{+}}}(q) =\frac{2 \mR^{\text{\textbf{+}}}(q) \mP^{\text{\textbf{+}}}(q)}{\mR^{\text{\textbf{+}}}(q) + \mP^{\text{\textbf{+}}}(q)}.
\end{equation}
This approach provides a comprehensive evaluation of both recall and precision while incorporating the critical aspect of missing elements.

\section{Knapsack Optimization-based Schema Linking Agent (KaSLA)} \label{sec:Knapsack Optimization-based Schema Linking Agent (KaSLA}

In this section, we introduce the proposed KaSLA framework, which focuses on ensuring the linking of relevant schema elements to avoid missing and minimizing the inclusion of redundant ones. As illustrated in Figure~\ref{fig: KaSLA main figure}, We begin by detailing a binary-probabilistic score function used for assessing the relevance of schema elements and an estimation function for the redundancy tolerance of each schema linking process. We then outline the knapsack optimization approach to link potential relevant element while reducing the potential redundant ones under the tolerance constraint. Finally, we describe the hierarchical schema linking process, demonstrating its efficiency for reducing the column linking candidate.

\subsection{Relevance Estimation}\label{sec:Relevance Estimation}
To precisely assess the relevance of each schema element $s$ in relation to the given query $q$, we employ a hybrid binary-probabilistic score function. This function integrates a generative model for binary scoring alongside an encoding model for probabilistic scoring, allowing for a comprehensive estimation of relevance.

\paragraph{Binary scoring function.}
This component is tasked with providing binary relevance assessments, determining whether a schema element is relevant or not in a straightforward manner:
\begin{equation}
f_{\text{binary}}: (q, \gS_q) \mapsto \{r^b_1, \cdots, r^b_{|\gS_q|}\},
\end{equation}
where $r^b_i \in \{0, 1\}$ represents the binary score assigned to the $i$-th element $s_i \in \gS_q$. A score of 1 indicates relevant while 0 signifies redundant.

Considering efficiency, we finetune a lightweight LLM, DeepSeek-Coder-1.3B~\citep{guo2024deepseek}, using LoRA~\citep{hu2021lora}, a well-established parameter-efficient technique. We employ a strategy whereby the initial generation of a simulated SQL provides a contextual foundation for the subsequent generation of schema linking. The loss function during fine-tuning is defined as follows:
\begin{equation}
\mathcal{L}_{\text{binary}} = - \sum_{q \in Q} \log P({\text{SQL}}^*_q, \mS_q^* \mid q, \gS_q),
\end{equation} where $Q$ is the set of queries in the training dataset. ${\text{SQL}}^*_q$ and $\mS^*$ are the ground truth SQL and linking result, respectively. This formulation utilizes a joint generation of a simulated SQL and the schema linking result, thereby ensuring that the linking generation benefits from the contextual alignment provided by the simulated SQL during both fine-tuning and inference. Noted that the simulated SQL generated during this process will not be utilized in subsequent steps. The input and output examples are provided in Table~\ref{tb: Input formats for generation with LLMs} in Appendix~\ref{section: Input formats for generation with LLMs}.

\paragraph{Probabilistic scoring function.}
In conjunction with the binary scoring model, we introduce a probabilistic scoring model to enhance prediction robustness. The primary motivation is to mitigate the risk of the binary model misclassifing relevant elements. The probabilistic scoring model assigns a soft score to each element as the linking probability to reduce the likelihood of relevant elements missing , thereby supporting the binary model:
\begin{equation}
f_{\text{prob.}}: (q, \gS_q) \mapsto \{r^p_1, \cdots, r^p_{|\gS_q|}\},
\end{equation}
where  $r^p_i \in [0, 1]$ represents the probabilistic score assigned to the $i$-th element $s_i \in \gS_q$.

For the probabilistic scoring model, we utilize an encoding model, RoBERTa-Large~\citep{liu2019roberta}, to derive the semantic embeddings for the query and all schema elements. Following the methodology of \citep{li2023resdsql,li2024codes}, we employ a cross-attention network to jointly embed the semantic representations of tables and their columns. The dot product of the query embedding and the element embedding yields the final probabilistic score. This probabilistic model is trained using the presence of $s$ in $\mS^*$ as the learning objective. The loss function is defined as follows:
\begin{equation}\label{eq:training loss of recall model}
\mathcal{L}_{\text{prob.}} = \sum_{q \in Q} \text{FL}(\mathbbm{1}(s \in \mS^*_q), f_{\text{prob.}}(q,\gS_q)),
\end{equation}
where $\text{FL}(\cdot)$ denotes the focal loss function \citep{ross2017focal}, designed to emphasize learning from hard negative samples.

\paragraph{Relevance scoring estimation.}
The final function for estimating the relevance of each schema element integrates the scores derived from both the binary and probabilistic functions:
\begin{equation}\label{Relevance scoring estimation}
\begin{split}
    f_{\text{relevance}}:& (q, \gS_q) \mapsto \{r_1, \ldots, r_{|\gS_q|}\} \\
    & = \left\{\min(1, r^b_i + r^p_i) \mid s_i \in \gS_q \right\},
\end{split}
\end{equation}
where $r_i$ denotes the relevance score assigned to the $i$-th schema element $s_i \in \gS_q$ given the query $q$. An upper limit of 1 is imposed on $r_i$ to prevent excessively strong contrasts between potentially relevant elements, as each relevant schema element is equally important for accurate SQL generation.


\subsection{Redundancy Estimation}
To minimize the inclusion of redundant elements during schema linking, we define both a redundancy score for each schema element $s \in \gS_q$ and the upper redundancy tolerance for query $q$.

\paragraph{Redundancy scoring estimation.}
Recognizing that elements with lower relevance scores are more likely to be redundant, whereas those with higher scores are more essential, we assign a redundancy score $w_i$ to each schema element $s_i \in \gS_q$ in relation to query $q$. This score is calculated as the inverse of its relevance score $r_i$, formulated as follows:
\begin{equation}
w_i = {r_i}^{-1}.
\end{equation}
It ensures elements with lower relevance scores contribute more to redundancy, thereby effectively prioritizing more relevant schema elements.

\paragraph{Upper Redundancy Tolerance.}
Redundancy tolerance defines the limit on the total redundancy score of elements in the schema linking results, as an excess of redundant elements can significantly disrupt subsequent SQL generation. Estimating it during inference is challenging. For the SQL generation of some queries, a lenient tolerance may be beneficial since it allowing the inclusion of more potentially relevant elements in the linking results to prevent missing. However, for other queries, a lenient tolerance could introduce to many redundant elements, which may significantly disrupt the final SQL generation process.

To address this challenge, we hypothesize that similar queries exhibit comparable tolerance levels for redundancy and leverage queries from the training dataset, where ground truth linking results are available, to aid in predicting redundancy tolerance for given query \(q\) during inference.
We first retrieve the top-\(K\) queries in the training dataset that closely resemble \(q\) using sentence similarity measures~\citep{gao2021simcse}. For a query \(q_k\) in the retrieved top-\(K\) queries, we calculate the total redundancy score for its ground truth linking results $\gS^{*}_{q_k}$  
and use it as its upper redundancy tolerance. The maximum of these upper redundancy tolerances from the \(K\) most similar queries in the training dataset is then used as the redundancy tolerance estimate for query \(q\):
\begin{equation}
u_q = \max_{q_k \in \text{TopK}(q)} \bigg( \sum_{s_j \in \gS^{*}_{q_k}} w_j\bigg)
\end{equation}
By capturing the maximum sum from the most similar queries, this approach effectively defines an upper bound on redundancy tolerance that is both efficient and adaptable, ensuring that while redundant elements are excluded, the risk of missing relevant elements is minimized.




\subsection{Schema Linking with Knapsack Optimization}
Given the relevance score $r_i$, the redundancy score $w_i$ for each $s_i \in \gS_q$, and the upper redundancy tolerance $u_q$ for query $q$, we introduce a hierarchical schema linking strategy with knapsack optimization. This approach is designed to ensure the linking of relevant schema elements to avoid missing and minimize the inclusion of redundant ones for a precise schema linking.

\paragraph{Knapsack Optimization}
The 0-1 knapsack problem~\citep{freville2004multidimensional} is a classic combinatorial optimization challenge. It involves selecting items from a finite set, each item characterized by a specific value and weight, to place into a knapsack with a limited weight capacity. The objective of 0-1 knapsack optimization is to choose items that maximize the total value while ensuring the total weight remains below the knapsack's capacity.

Inspired by this, we develop a knapsack optimization-based schema linking strategy to ensure the inclusion of relevant schema elements while minimizing redundant ones, thus ach ieving optimal schema linking. Formally, the optimization objective are defined as follows:
\begin{equation}
\begin{split}
    f_{\text{Knap}}:& (q, \gS_q) \mapsto \widehat{\mS}_q  = \arg\max \sum_{s_i \in \widehat{\mS}_q} r_i \\&  \text{s.t.} \quad \sum_{s_i \in \widehat{\mS}_q} w_i \leq u_q.
\end{split}
\end{equation}
This formulation aims to maximize the total relevance scores of the linked schema elements. The constraint ensures that the cumulative redundancy score remains within the upper redundancy tolerance of query $q$. To tackle this optimization problem and achieve the optimal linking result, We employ an efficient dynamic programming approach. Detailed pseudocode of this dynamic programming algorithm is provided in Algorithm~\ref{sec: Implementation Details} in Appendix.

\paragraph{Hierarchical Linking Strategy}
For efficiency, our KaSLA employs a hierarchical process in schema linking: initially linking tables and subsequently linking columns within each selected table. This strategy effectively reduces the dimensionality of candidate columns, enhancing KaSLA's capacity to handle large-scale schema linking and text-to-SQL generation tasks.

In the initial phase for a given query $q$, KaSLA aims to identify the relevant tables from the set of all available tables in schema $\gS_q$, denoted as $\gT_q$:
\begin{equation}
\widehat{\mT}_q  = f_{\text{Knap}}(q, \gT_q).
\end{equation}
After determining tables, for each $t \in \widehat{\mT}_q$, let $\gC^t_q$ represent the columns in table $t$. KaSLA then optimizes column linking within each selected table:
\begin{equation}
\widehat{\mC}^t_q  = f_{\text{Knap}}(q, \gC^t_q).
\end{equation}
The final schema linking result $\widehat{\mS}_q$ is the union of all selected tables and their corresponding selected columns, represented as:
\begin{equation}
\widehat{\mS}_q = \widehat{\mT}_q \cup \bigcup_{t \in \widehat{\mT}_q} \widehat{\mC}^t_q.
\end{equation}
We provide a complexity analysis is detailed in Appendix~\ref{sec: Complexity Analysis}, offering insights into KaSLA's implementation and computational efficiency.






















\section{Experiments}
\label{sec:Experiments}


We conducted comprehensive experiments to evaluate KaSLA and address the following research questions: \textbf{RQ1:} Does KaSLA enhance existing text-to-SQL baselines by improving their schema linking capabilities? \textbf{RQ2:} Does KaSLA demonstrate superior schema linking performance by handling element missing and redundancy? \textbf{RQ3:} Does each component in KaSLA contribute to the overall performance, and is KaSLA an efficient model for real-world text-to-SQL applications?


\subsection{Experiment Setup}\label{sec: Experiment Setup}
We conducted experiments on two well-know large-scale text-to-SQL datasets, BIRD~\citep{li2023BIRD} and Spider~\citep{yu2018Spider}, incorporating multiple robust baselines based on LLMs with in-context learning and fine-tuning. For evaluating text-to-SQL, we utilized Execution Accuracy (EX)~\citep{yu2018Spider,li2023BIRD} and Valid Efficiency Score (VES)~\citep{li2023BIRD}. We included multiple robust baselines from two categories:\emph{(i) LLMs + in-context learning baselines:} C3-SQL~\citep{dong2023c3}, DIN-SQL~\citep{pourreza2023dinsql}, MAC-SQL~\citep{wang2024macsql}, DAIL-SQL~\citep{gao2023dailsql}, SuperSQL~\citep{li2024dawn}, Dubo-SQL~\citep{thorpe2024dubo}, TA-SQL~\citep{qu2024before}, E-SQL~\citep{caferouglu2024sql}, CHESS~\citep{talaei2024chess}; and \emph{(ii) LLMs + fine-tuning baselines:} DTS-SQL~\citep{pourreza2024dtssql}, CodeS~\citep{li2024codes}.

\subsection{Main Results of SQL Generation (\textbf{RQ1})}
To address \textbf{RQ1}, we conducted experiments to evaluate the SQL generation capabilities of state-of-the-art (SOTA) baselines enhanced by our KaSLA, with results presented in Table~\ref{tb:main_results}. KaSLA replaces the original schema linking component in models that already have schema linking capabilities or implements a full schema prompt substitution for those that do not.

The results demonstrate that incorporating KaSLA consistently improves both Execution Accuracy (EX) and Valid Efficiency Score (VES) across baselines with diverse base models, including GPT-4, CodeS-15B, StarCoder2-15B, and Deepseek-v3. Notably, in the BIRD-dev dataset, KaSLA enhances the EX of CHESS with Deepseek-v3 from 61.34\% to 63.30\%. For local LLMs, both "CodeS + KaSLA" and "TA-SL + KaSLA" with CodeS-15B and StarCoder-15B achieve impressive increases in Execution Accuracy compared to their original versions. These results underscore KaSLA's effectiveness and flexibility as a plug-in, showing that KaSLA plays an important role in improving complex text-to-SQL tasks. This adaptability not only confirms its robustness but also broadens its applicability across different models, achieving superior performance outcomes and making it a versatile addition to existing systems.

\begin{table*}[!h]
   \begin{center}
   \belowrulesep=0pt
   \aboverulesep=0pt
   \resizebox{\linewidth}{!}{
   \scalebox{1}{
   \begin{tabular}{>{\centering}p{0.115\textwidth}|>{\centering}p{0.22\textwidth}|>{\centering}p{0.07\textwidth}
      >{\centering}p{0.07\textwidth}
      >{\centering}p{0.075\textwidth}
      |>{\centering}p{0.07\textwidth}|>{\centering}p{0.07\textwidth}|>{\centering}p{0.07\textwidth}>{\centering}p{0.07\textwidth}>{\centering}p{0.074\textwidth}>{\centering}p{0.074\textwidth}|>{\centering}p{0.07\textwidth}|>{\centering\arraybackslash}p{0.07\textwidth}}
   \toprule[1pt]
   \multirow{3}{*}{\makecell[c]{Base Model}}& \multirow{3}{*}{\makecell[c]{Method}} &\multicolumn{5}{c|}{BIRD-dev} &\multicolumn{6}{c}{Spider-dev} \\
   \cline{3-7}\cline{8-13}
   & &\multicolumn{4}{c|}{EX} &  VES     &\multicolumn{5}{c|}{EX} &  VES \\
   \cline{3-7}\cline{8-13}
   & &Easy&Medium & Hard & Total &   Total                              &Easy &Medium &Hard &Extra &Total&Total \\
   \midrule
   \multirow{7}{*}{\makecell[c]{GPT-4}}& MAC-SQL   &-  &-  &-  &57.56  &  58.76                             &- &-  &-  &-    &86.75    &-          \\
    &DIN-SQL   &-  &-  &-  &50.72    & 58.79                             &92.34  &87.44  &76.44  &62.65  &82.79  &81.70         \\
    &DAIL-SQL   &62.49  &43.44  &38.19  &54.43      & 55.74                   &91.53  &89.24  &77.01  &60.24  &83.08   &83.11          \\
    &DAIL-SQL (SC)  &63.03  &45.81  &43.06  &55.93     & 57.20                &91.53  &90.13  &75.29  &62.65  &83.56  &-          \\
    &TA-SQL   &63.14 &{48.82} &36.81 &56.32               &  -               &93.50 &90.80 &77.60 &64.50 &85.00         & -        \\
    &SuperSQL  &{66.92}  &46.67  &{43.75}  &58.60   & 60.62       &{94.35} &{91.26}  &{{83.33}}  &{{68.67}}  &{87.04} &{85.92}  \\
   & Dubo-SQL   &-  &-  &-  &{59.71} & {66.01}            &- &-  &-  &-  &-  & -           \\
   \midrule
     \multirow{2}{*}{\makecell[c]{CodeS-15B}}  & CodeS  &65.19&48.60&38.19&57.63   & 63.22 &94.76&91.26  &72.99  &{64.46}   &84.72   & 83.52     \\
     & \textbf{CodeS   + KaSLA}      &68.32&50.75&38.19&60.17  &64.52    &\textbf{96.37}&89.46&76.44&63.86&84.82& 84.46 \\
     \midrule
   \multirow{6}{*}{\makecell[c]{StarCoder2\\-15B}}   &TA-SL  &  59.89&45.38&34.72&53.13   & 60.51            &95.56&92.38&78.16&63.25& 85.98 &  85.03       \\ 
     & \textbf{TA-SL  + KaSLA}      &66.59  & 49.46  & 38.89 & 58.80  &63.56    &94.76&92.15&77.59&66.87&86.27 &85.47  \\
    &  DTS-SQL &  63.03&46.02&34.72&55.22 &64.17 & 91.94&90.58&78.74&66.87&85.11  &  85.49  \\ 
    &  \textbf{ DTS-SQL  + KaSLA}      &63.89 &50.11   &39.58   &57.43  &65.20     &94.76&90.81&\textbf{80.46}&\textbf{68.07}&\textbf{86.36}&85.81 \\
   &  CodeS     &{68.00}  &{51.40}  &{39.58}  &{60.30}  & {65.04}  &94.76&91.26  &72.99  &{64.46}   &84.72   & {83.52}       \\
   &  \textbf{  CodeS + KaSLA}     &67.68   &\textbf{54.84}  &42.36  &61.41  &66.87    &94.76 &91.03 &78.74 &66.27 &85.88& 84.46  \\
     \midrule
    \multirow{4}{*}{\makecell[c]{Deepseek\\-V3}} & E-SQL  &67.24  &51.61  &41.67 &60.10   & 62.16   &95.57&91.93&72.99&64.46& 85.21 &    83.81   \\ 
   & \textbf{ E-SQL + KaSLA}  &69.40   &53.55 &43.75  &62.18  &65.32   &95.97 &92.38 & 73.56 & 65.06 &85.69& 85.72   \\
    &  CHESS   &67.89  &52.90 &46.53 & 61.34  &65.28 & 95.57&92.16&73.56& 65.06& 85.50  & 85.33   \\ 
    & \textbf{  CHESS + KaSLA}   &\textbf{69.94}   &\textbf{54.84} & \textbf{47.92} &\textbf{63.30}  &\textbf{67.21}  &\textbf{96.37}&\textbf{92.83}&74.14&65.66 &86.17 & \textbf{86.35} \\
   \bottomrule[1pt]
   \end{tabular}}}
   \caption{The SQL generation performance of enhanced text-to-SQL models using KaSLA, which features 1.6 B parameters combining DeepSeek-coder-1.3B and RoBERTa-Large, is evaluated in terms of Execution Accuracy (EX) (\%) and Valid Efficiency Score (VES) (\%) on the BIRD-dev and Spider-dev datasets. We utilize KaSLA to substitute the original schema linking component in models that already incorporate schema linking, or to replace the full schema prompt for models that do not.}
   \label{tb:main_results}
   \end{center}
   \end{table*}

\subsection{Experimental Results of Schema Linking (\textbf{RQ2})} \label{sec: Schema Linking Results} 
To answer the \textbf{RQ2}, we conduct the experiments and report the evaluation results of schema linking methods on BIRD-dev in Figure~\ref{fig: Schema linking evaluation.}. We measured the performance of these schema linking methods in both table linking and column linking. The results indicate that, under the enhanced F1 score which specifically evaluates missing elements, only KaSLA achieves high accuracy in the challenging task of column linking. Schema linking models using LLMs often perform well in table linking but fall short in column linking due to the lack of additional optimizations for missing elements. This highlights the advantages of KaSLA's framework and demonstrates how enhanced metrics can reveal the significant issue of element missing in current schema linking practices.


   
\begin{table}[!h]
   \begin{center}
   \resizebox{\linewidth}{!}{
   \scalebox{1}{ 
   \begin{tabular}{c |c c c}
   \toprule[1pt]
   Schema linking method & Table linking & Column linking\\
   \midrule
   DTS-SL (starcoder2-15b) &61.94  &23.71 \\
   CodeS-SL (starcoder2-15b) &58.21  &29.56 \\
   RSL-SQL (deepseek-v3) &67.11  & 17.20 	\\
   KaSLA (deepseek-coder-1.3B)  &\textbf{81.92} &\textbf{62.78}	\\
   \bottomrule[1pt]
   \end{tabular}}}
   \caption{Schema linking performance with the proposed enhanced linking metric on BIRD-dev.}
\label{fig: Schema linking evaluation.}
\vspace{-2mm}
   \end{center}
   \end{table}





   
\begin{table}[!h]
   \begin{center}
   \resizebox{0.7\linewidth}{!}{
   \scalebox{1}{ 
   \begin{tabular}{c |c}
   \toprule[1pt]
   Ablation &  EX \\
   \midrule
   CodeS + KaSLA & 	60.17  \\
   \midrule
   w/o Binary scoring model &   56.06 	\\
   w/o Probabilistic scoring model &  57.82  	\\
   w/o Hierarchical strategy & 58.34   	\\
   w/o upper limit 1 in Eq.~\ref{Relevance scoring estimation} &  53.26  	\\
   \bottomrule[1pt]
   \end{tabular}}}
   \caption{Ablation studies of the key components of KaSLA. We conduct the experiments for CodeS + KaSLA on BIRD-dev and use CodeS with CodeS-15B as the text-to-SQL backbone.}
\label{tb: ablation study}
\vspace{-5mm}
   \end{center}
   \end{table}


\begin{table*}[!h]
   \begin{center}
   \resizebox{\linewidth}{!}{
   \scalebox{1}{
   \begin{tabular}{c |c c  c c  c | c | c}
   \toprule
   Dataset &\makecell[c]{Schema linking \\ model} 
   &\makecell[c]{ Binary scoring function \\ with deepseek-coder-1.3B} & \makecell[c]{Probabilistic scoring model \\ with RoBERTa-Large}&\makecell[c]{Estimation and \\ dynamic programming} &Text-to-SQL  &Total time & EX\\
   \midrule
   \multirow{2}{*}{BIRD-dev}
    &Full Schema & 		/ & 	/ & 	/ & 5.15 s  & 5.15 s &57.63\\
     &KaSLA &3.5 s  &0.12 s     &	$<$ 0.01 s    &1.85 s  &  5.48 s  &60.17\\
   \bottomrule
   \end{tabular}}}
   \caption{Inference time cost per instance of each component in KaSLA using CodeS-15B as the text-to-SQL model.}
   \label{tb: Inference time cost}
   \end{center}
   \end{table*}
   
\begin{table}[!h]
   \begin{center}
   \resizebox{\linewidth}{!}{
   \scalebox{1}{
   \begin{tabular}{c |c c  c | c  }
   \toprule[1pt]
   Model & SuperSQL & CodeS &  \makecell[c]{CodeS \\+ KaSLA}& \makecell[c]{CodeS \\+ KaSLA  (Transfer)} \\
   \midrule
   BIRD-dev & 58.60  &60.30 &\textbf{60.17}& 60.02 \\
   Spider-dev & 87.40 &	84.72 	&\textbf{84.82 }&84.56\\
   \bottomrule[1pt]
   \end{tabular}}}
   \caption{Execution Accuracy (EX) (\%) of KaSLA trained on Spider-train but evaluated on BIRD-dev, and vice versa, for cross-scenario transfer ability evaluation.}
   \label{tb: trained on Spider-train but evaluated on BIRD-dev.}
\vspace{-4mm}
   \end{center}
   \end{table}
\subsection{Empirical studies (\textbf{RQ3})}
\paragraph{Ablation Study.}
We presented the ablation study results for the key components in KaSLA in Table~\ref{tb: ablation study}, the results concerning the choice of language model for the probabilistic scoring model are shown in Table~\ref{tb: ablation study about the choice of language model.} in Apendix and the scaling up performance of the scoring model are shown in Table~\ref{tb: ablation study about the choice of LLMs.} in Apendix, respectively.

As shown in Table~\ref{tb: ablation study}: (i) Removing the binary scoring model results in a lack of confirmation for high-confidence elements, leading to the inclusion of more redundant ones. This underscores the binary model's role in filtering out less relevant elements by ensuring that elements with high certainty are correctly identified. (ii) Omitting the probabilistic scoring model risks missing potentially relevant elements, resulting in gaps within schema linking. The probabilistic model is essential for assigning a non-zero probability to all elements, thus capturing those with potential relevance. (iii) The hierarchical strategy effectively reduces the number of candidate columns, thereby lowering complexity and improving efficiency. By structuring the linking process in stages, KaSLA can more precisely target relevant schema components. (iv) Without the upper limit set at 1, the model introduces disproportionate comparisons among similarly relevant items, potentially excluding some relevant elements. This constraint is crucial to maintain balanced relevance assessments, preventing the overshadowing of items that might otherwise contribute positively to SQL generation.
These ablation study highlights the significance of key components within KaSLA.

\paragraph{Transferability.}\label{sec: Transferability of KaSLA}
We conducted two experiments to evaluate KaSLA's performance in cross-scenario transfer. We trained the binary and probabilistic scoring models on the Spider training dataset and evaluated it on the BIRD dev dataset, and vice versa, to explore its cross-scenario transfer ability. Results are provided in Table~\ref{tb: trained on Spider-train but evaluated on BIRD-dev.}. We can find that pre-training KaSLA on public datasets yields results that outperform the baselines and are only slightly lower than what domain-specific fine-tuning would achieve. The results show that KaSLA demonstrates strong cross-scenario transferability, highlighting its robustness and adaptability across different data domains.

\paragraph{Inference time cost.}
We evaluated the inference time cost of integrating KaSLA into text-to-SQL frameworks in Table~\ref{tb: Inference time cost}. By utilizing the lightweight LLM DeepSeek-coder-1.3B and significantly reducing the number of tokens in the input prompts through optimized schema linking, KaSLA achieves improved SQL generation accuracy with only a modest increase in inference time. This showcases KaSLA's efficiency in balancing performance and computational demands.


\section{Related Work}

\begin{figure}[bt!]
    \centering
    % First row
    \begin{subfigure}[t]{0.48\linewidth}
        \centering
        \includegraphics[width=\textwidth]{figure/rep11.png}
        \caption{Objects with the prompt: \textit{A white truck that is stationary in the same direction.} \cite{nuprompt}}
    \end{subfigure}
    \hfill
    \begin{subfigure}[t]{0.48\linewidth}
        \centering
        \includegraphics[width=\textwidth]{figure/rep21.png}
        \caption{Frame-based object expression using numerical coordinates \cite{drivelm}.}
    \end{subfigure}
    
    
    % Second row
    \begin{subfigure}[t]{0.48\linewidth}
        \centering
        \includegraphics[width=\textwidth]{figure/TrafficQA-Object_Representation_rep12.jpg}
        \caption{Object referring in \cite{vidstg} with prompt: \textit{What is beneath the adult}.}
    \end{subfigure}
    \hfill
    \begin{subfigure}[t]{0.48\linewidth}
        \centering
        % \includegraphics[width=\textwidth]{figure/rep22.jpg}
        \includegraphics[width=\textwidth]{figure/TrafficQA-Object_Representation_22.jpg}
        \caption{Location of the green bus \textit{[(c1,0.0,0.5,0.4)]} in the video. (Ours)}
        \label{fig:objct_ref4}
    \end{subfigure}
    
    \caption{Different methods for describing objects in images and videos using language expressions. We adopt a tuple-based spatio-temporal object representation for the unique object reference, as shown in (d). }
    \label{fig:object_representation}
\end{figure}


% \begin{table}[htb]
% \centering
% \resizebox{0.5\textwidth}{!}{%
% \begin{tabular}{cccccccc}
% % \hline
% \midrule
% \makecell{\textbf{Dataset}} & \makecell{\textbf{Tasks}} & \makecell{\textbf{QA Gen.}} & \makecell{\textbf{\# Videos}\\\textbf{/Scenes}} & \makecell{\textbf{\# QAs}}  & \makecell{\textbf{\# Grounds.}} & \makecell{\textbf{Domain}} \\

% % \\\textbf{/Capts.}
% % \hline
% \midrule

% HAD \cite{had}         & Video QA & Manual & 5.6k & 45k & - & Driving \\

% DRAMA \cite{malla2023drama}         & Video QA& Manual & 18k  & 102k  & - & Driving \\

% LingoQA \cite{marcu2024lingoqavisualquestionanswering}  & Video QA & Manual & 28k & 419k & - & Driving \\

% NuScenes-QA \cite{qian2024nuscenes}         & Image QA & Template & 850 & 460k &  - & Driving \\

% DriveLM \cite{sima2023drivelm}         & Image QA & Temp. + Man. & 188k  & 4.2M & - & Driving \\

% City-3DQA \cite{sun20243dquestionansweringcity} & Scene QA & Temp + Man. & 193 & 450k & - &  City \\

% \midrule
% HC-STVG \cite{hc-stvg} & Video Grounding & Manual &5.6k & - & 5.6k&General\\

% DVD-ST \cite{dvd-st} & Video Grounding & Manual & 2.7k & - &5.7k & General  \\

% Refer-KITTI \cite{referkitti} & Referred-MOT & Manual & 18 & - & 818 & Driving \\

% NuPrompt \cite{nuprompt}         & Referred-MOT & LLM & 850 & - & 35k  & Driving \\

% \midrule


% \textbf{TUMTraffic-VideoQA (Ours)} & \makecell{Video QA, \\ST Grounding} & Temp. + LLM  &1k & 88k  & 5.7k &  Roadside \\


% % \hline
% \midrule
% \end{tabular}%
% }
% \caption{Summary and comparison of language datasets in the traffic domain for question answering, video grounding, and referred multi-object tracking.}
% \label{tab:related_datasets}
% \end{table}



\begin{table*}[thb!]
\centering
\caption{Summary of visual-language datasets in the traffic domain for question answering, video grounding, and referred multi-object tracking. The table’s upper section presents QA tasks, while the lower section covers grounding and referring tasks. We introduce the first roadside video understanding dataset and unify the tasks in one benchmark. }
\resizebox{\textwidth}{!}{%
\begin{tabular}{c|ccccccccccc}
% \hline
\midrule
\textbf{Dataset} & \textbf{Venue} & \textbf{Tasks} & \textbf{QA Gen.} & \textbf{\# Videos/Scenes} & \textbf{\# QAs/Captions}  & \textbf{\# Grounding} & \textbf{Domain} \\
% \hline
\midrule


% BDD-X \cite{kim2018textual}         & ECCV18 & video-level & Manual & $\sim$7k (v) & $\sim$26k & $\sim$3.7 & $\sim$77h & - & 1 & 4 & Driving \\

% HAD \cite{had}         & CVPR'19 & Video QA & Manual & 5.6k & 45k & - & Driving \\

% SUTD \cite{xu2021sutd}         & CVPR 2021 & video-level & Manual & $\sim$10k (v) & $\sim$63k & $\sim$6.3 & - & 70s & - & - & D + T \\

DRAMA \cite{malla2023drama}         & WACV'23 & Video QA& Manual & 18k  & 102k  & - & Driving \\
LingoQA \cite{marcu2024lingoqavisualquestionanswering}  & ECCV'24 & Video QA & Manual & 28k & 419k & - & Driving & \\

NuScenes-QA \cite{qian2024nuscenes}         & AAAI'24 & Image QA & Template & 850 & 460k &  - & Driving \\

DriveLM \cite{drivelm}         & ECCV'24 & Image QA & Temp. + Man. & 188k  & 4.2M & - & Driving \\

% ELM \cite{zhou2024embodied}         & ECCV'24 & Video-Level & Temp. + LLM & - & $\sim$9M & - &  Driving \\


% SQA-3D \cite{sqa3d}  & ICLR'23 & Scene QA & Manual & 650 & 33.4k & - & Indoor\\

City-3DQA \cite{sun20243dquestionansweringcity} & ACM MM'24& Scene QA & Temp. + Man. & 193 & 450k & - &  City \\

\midrule
HC-STVG \cite{hc-stvg} & ACM MM'22 & Video Grounding & Manual &5.6k & - & 5.6k&General\\

DVD-ST \cite{dvd-st} & -  & Video Grounding & Manual & 2.7k & - &5.7k & General  \\

Refer-KITTI \cite{referkitti} & CVPR'23  & Referred-MOT & Manual & 18 & - & 818 & Driving \\

NuPrompt \cite{nuprompt}         & AAAI'25 & Referred-MOT & LLM & 850 & - & 35k  & Driving \\

% STPR && Video-Level &&5.2k&-&30k &General \\

% VD-STG\cite{vidstg} &&&&\\

\midrule

\textbf{TUMTraffic-VideoQA (Ours)} & - & Video QA, ST Grounding & Temp. + LLM  &1k & 87.3k  & 5.7k &  Roadside \\

% \hline
\midrule
\end{tabular}%
}

\label{tab:related_datasets}
\end{table*}

% Granularity in thousands?

% Can we trust an information about a dataset which was found only in another paper?  

% Modality ?


% \begin{table}[htb]
% \centering
% \resizebox{0.5\textwidth}{!}{%
% \begin{tabular}{c|ccccc}
% % \hline
% \midrule
% \textbf{Dataset} & \textbf{Task} & \textbf{\#Scenes} & \textbf{\#QA}  & \textbf{\#Grounding} & \textbf{Domain} \\
% % \hline
% \midrule

% HAD \cite{had}         & Video QA & $\sim$5.6k & $\sim$45k & - & Driving \\

% DRAMA \cite{malla2023drama}         & Video QA & $\sim$18k  & $\sim$102k  & - & Driving \\

% NuScenes-QA \cite{qian2024nuscenes}         & Image QA & 850 & $\sim$460k &  - & Driving \\

% DriveLM \cite{sima2023drivelm}         & Image QA & $\sim$188k  & $\sim$4.2M & - & Driving \\

% SQA-3D \cite{sqa3d}  & Scene QA & 650 & 33.4k & - & Indoor \\

% City-3DQA \cite{sun20243dquestionansweringcity} & Scene QA & 193 & 450k & - & City \\

% \midrule
% Refer-KITTI\cite{referkitti} & Referred-MOT & 18 & - & 818 & Driving \\

% NuPrompt \cite{nuprompt}         & Referred-MOT & 850 & - & 35k  & Driving \\

% DVD-ST\cite{dvd-st} & Video Grounding & 2.7k & - &5.7k & General \\

% HC-STVG\cite{hc-stvg} & Video Grounding & 5.6k & - & 5.6k & General \\

% \midrule

% \textbf{TUMTraffic-VideoQA(Ours)} & Video QA, Grounding & 1k & 88k  & 5.7k & Traffic \\

% % \hline
% \midrule
% \end{tabular}%
% }
% \caption{Related datasets}
% \label{tab:related_datasets}
% \end{table}


% [x] connect table 1 with introductions. 

\subsection{Vision-Language Datasets in Traffic Scenes}
% DriveLM\cite{drivelm},
% HAD \cite{had} and 
With the rapid advancements in LLMs, significant efforts have been made to integrate language into the development of vision-language foundation models. As summarized in Table \ref{tab:related_datasets}, several pioneering datasets have been introduced for traffic scenarios, particularly focusing on vehicle-centric environments \cite{addatasetseurvey}. NuScenes-QA \cite{qian2024nuscenes} provides a question-answering benchmark tailored for driving scenes. Meanwhile, DRAMA \cite{malla2023drama} is designed for video-level open-ended tasks aimed at evaluating driving instructions and assessing the importance of objects within their environments. Besides, referring to specific traffic participants through natural language—commonly known as referred object grounding and tracking—is a crucial task in traffic scene understanding. Some works \cite{referkitti,nuprompt} extend the KITTI \cite{kitti} and nuScenes \cite{caesar2020nuscenesmultimodaldatasetautonomous} datasets, by associating natural language descriptions with specific vehicles and pedestrians. This facilitates fine-grained identification and tracking of traffic participants, allowing for precise object localization based on language descriptions in complex driving environments. However, most existing efforts primarily focus on driving scenarios and are typically constrained to individual tasks such as question answering, video grounding, or referred multi-object tracking. A significant research gap also remains in the availability of large-scale datasets designed specifically for roadside surveillance scenarios. Our work aims to bridge this gap by providing a comprehensive dataset tailored for multiple tasks in roadside traffic understanding within a unified framework.
% is also an important aspect of traffic scene understanding
% introducing a standardized object representation and 


\subsection{Fine-Grained Video Understanding}

Fine-grained video understanding centers on the precise analysis of intricate video content, targeting tasks that demand nuanced reasoning across spatial and temporal dimensions. Some representative tasks include spatio-temporal grounding \cite{vidstg,hc-stvg}, mapping specific objects or events to precise locations and times within a video based on a given query; video object referring \cite{mevis,referkitti,nuprompt}, which involves tracking objects through space and time given text prompts; video temporal grounding \cite{UniVTG,huang2024vtimellm}, identifying specific moments or intervals in a video that align with a provided textual query. These tasks require high precision, nuanced multimodal alignment, and the ability to capture subtle temporal and spatial dynamics. It is particularly challenging due to the difficulty of properly representing fine-grained video details and the inherent cross-modality misalignment. With the advancement of visual LLMs, recent advancements enhance the capabilities of fine-grained video understanding \cite{videunderstandingsurvey} and facilitate understanding across abstract and detailed levels. 

% , with advanced visual embedding techniques and modality alignment strategies to bridge the gap between textual and visual semantics, significantly





\subsection{Language-Based Object Referring}


Referring objects in visual data, such as images and videos, is typically achieved by associating them with predefined definitions or language descriptions. Figure \ref{fig:object_representation} illustrates four commonly used methods for representing objects through language expressions. The inherent ambiguity of natural language, coupled with the modality gap between visual and linguistic representations, presents significant challenges. Object representation in tasks such as object referring often necessitates careful dataset curation to ensure that linguistic expressions uniquely or collectively correspond to specific objects in videos. For example, some datasets include only scenarios with uniquely identifiable objects \cite{hc-stvg}, while others contain expressions that jointly refer to multiple objects \cite{dvd-st}. However, in complex real-world applications such as autonomous driving, textual descriptions alone are often insufficient to uniquely specify an object. To address this challenge, DriveLM \cite{drivelm} introduces a structured tuple representation, $\textless c, CAM, x, y \textgreater$, where  c  denotes the object identifier,  CAM  specifies the camera, and $\textless x, y \textgreater$ represents the 2D center coordinates within the camera’s coordinate system. Alternatively, ELM \cite{zhou2024embodied} simplifies the problem by converting temporal video tasks into frame-level questions, using a tuple $\textless c, x, y \textgreater$ to identify objects within individual frames without temporal dependencies. Despite the advancements, formulating a unified, precise, and unique language representation for objects in video remains open challenges. 




In this work, we design a spatio-temporal object representation in videos with a four-element tuple format $(c, f_n, x, y)$, where c denotes a unique object identifier, $f_n$ indicates the normalized frame timestamp, and $(x, y)$ corresponds to the object’s normalized spatial coordinates within the frame.  The same object is consistently assigned the identifier  c  throughout the video, while its spatial position changes over time. This formulation enables precise tracking and referencing of objects across both spatial and temporal dimensions, facilitating robust language-based interaction in dynamic environments. Besides, it provides a standardized interface for fine-grained video understanding, enabling more detailed and structured analysis.

 




   
\section{Conclusion}
\label{sec:Conclusion}
This paper introduced the Knapsack Schema Linking Agent (KaSLA), a pioneering approach to overcoming schema linking challenges in text-to-SQL tasks. Based on knapsack optimization, KaSLA effectively prevent the missing of relevant elements and excess redundancy, thereby significantly enhancing SQL generation accuracy. Our introduction of an enhanced schema linking metric, sets a new standard for evaluating schema linking performance.
KaSLA utilizes a hierarchical linking strategy, starting with optimal table linking and then proceeding to column linking within selected tables, effectively reducing the candidate search space. In each step, KaSLA employs a knapsack optimization strategy to link potentially relevant elements while considering a limited tolerance for potential redundancy. Extensive experiments on the Spider and BIRD benchmarks have demonstrated that KaSLA can significantly enhance the SQL generation performance of existing text-to-SQL models by substituting their current schema linking processes. These findings underscore KaSLA's potential to revolutionize schema linking and advance the broader capabilities of text-to-SQL systems, offering a robust solution that paves the way for more precise and efficient database interactions.


\section{Limitation} \label{sec: Limitation}
There are mainly two limitations of this work. First, although KaSLA demonstrates significant advancements in schema linking accuracy and efficiency, its performance traditionally depends on a comprehensive training dataset with detailed ground truth linking results. However, our experiments indicate that KaSLA possesses a degree of transferability, showing promise even in scenarios where training data is sparse or less diverse. This adaptability suggests that KaSLA can maintain reasonable effectiveness across different database environments, although performance might still be impacted in highly dynamic or rapidly evolving schema structures. Additionally, the reliance on a fine-tuning process of LLMs, integrating DeepSeek-coder-1.3B, could pose scalability challenges in resource-constrained settings. Despite these limitations, KaSLA remains a powerful tool for schema linking optimization in text-to-SQL tasks, and future work could focus on enhancing its adaptability and efficiency.
   
\section{Ethics Statement}
We confirm that we have fully complied with the ACL Ethics Policy in this study. All the datasets are publicly available and have been extensively used in research related to text-to-SQL.

\nocite{*}
\bibliography{custom}

\appendix
\section{Appendix}
\subsection{Data - Survey on current passenger behaviour regarding POIs}

\begin{center}
    \begin{minipage}{\textwidth}

        \centering
        \captionof{table}{Usage of various navigation methods by our survey participants.}
        \begin{tabular}{r|cc|cc|cc|cc|cc}
            \toprule
            & \multicolumn{2}{c|}{AppleCar/AndroidAuto} & \multicolumn{2}{c|}{InVehicleSystems} & \multicolumn{2}{c|}{SmartphoneApps} & \multicolumn{2}{c|}{Compass} & \multicolumn{2}{c}{PaperMaps} \\
            & N & \% & N & \% & N & \% & N & \% & N & \% \\
            \midrule
            Never & 39 & 36.4\% & 13 & 12.1\% & 2 & 1.9\% & 101 & 94.4\% & 79 & 73.8\% \\
            Rarely & 19 & 17.8\% & 19 & 17.8\% & 15 & 14.0\% & 2 & 1.9\% & 26 & 24.3\% \\
            Sometimes & 18 & 16.8\% & 22 & 20.6\% & 33 & 30.8\% & 2 & 1.9\% & 2 & 1.9\% \\
            Often & 22 & 20.6\% & 34 & 31.8\% & 32 & 29.9\% & 0 & 0\% & 0 & 0\% \\
            Always & 9 & 8.4\% & 19 & 17.8\% & 25 & 23.4\% & 2 & 1.9\% & 0 & 0\% \\
            \bottomrule
        \end{tabular}

        \vspace{\baselineskip}
        
        \captionof{table}{The types of information drivers and passengers need to find a missed point of interest. Multiple choice was possible.}
        \begin{tabular}{r|cc|cc}
            \toprule
            \textbf{Type of information} & \multicolumn{2}{c|}{Drivers} & \multicolumn{2}{c}{Passengers} \\
            & N & \% & N & \% \\
            \midrule
            Name & 71 & 89\% & 65 & 90\% \\
            Perspective Picture & 39 & 49\% & 38 & 53\% \\
            Web Picture & 46 & 58\% & 44 & 61\% \\
            Description Text & 45 & 56\% & 49 & 68\% \\
            Category & 45 & 56\% & 49 & 68\% \\
            \bottomrule
        \end{tabular}

        \vspace{\baselineskip}

        \captionof{table}{Percentages on how and when passengers and drivers look for a missed point of interest.}
        \begin{tabular}{r|c|c}
            \toprule
            \textbf{Action}             & \textbf{Drivers}   & \textbf{Passengers} \\
            \midrule
            Nothing                     & 5\%                 & 4\% \\
            System saves Automatically  & 0\%                 & 3\% \\
            Stop the Car                & 3\%                 & 0\% \\
            Search later on Smartphone  & 46\%                & 8\% \\
            Search immediately on Smartphone & 11\%           & 49\% \\
            Search later on Navigation system & 24\%          & 5\% \\
            Search immediately on Navigation system & 11\%    & 31\% \\
            \bottomrule
        \end{tabular}

    \end{minipage}
\end{center}



\clearpage
\subsection{Data - Pre-Study on Eye-Gaze Interaction}

\begin{center}
    \begin{minipage}{\textwidth}
      
    \centering
    \captionof{table}{Descriptive Statistics for the pre-study Raw NASA Task Load Index.}
    \begin{tabular}{l|c|c|c|c|c|c|c}
        \toprule
        & \textbf{Mental} & \textbf{Physical} & \textbf{Temporal} & \textbf{Performance} & \textbf{Effort} & \textbf{Frustration} & \textbf{Score} \\
        \midrule
        Mean & 15.0 & 19.5 & 40.0 & 33.5 & 28.0 & 13.5 & 24.8 \\
        Median & 12.5 & 15.0 & 42.5 & 25.0 & 32.5 & 15.0 & 25.0 \\
        Standard Deviation & 10.5 & 19.9 & 23.1 & 21.4 & 18.1 & 11.6 & 9.47 \\
        Shapiro-Wilk W & 0.942 & 0.782 & 0.961 & 0.848 & 0.898 & 0.882 & 0.959 \\
        Shapiro-Wilk p & 0.573 & 0.009 & 0.794 & 0.055 & 0.206 & 0.139 & 0.769 \\
        \bottomrule
    \end{tabular}  

    \vspace{\baselineskip}

    \captionof{table}{Descriptive Statistics for the pre-study System Usability Scale.}
    \begin{tabular}{l|c|c|c|c|c}
        \toprule
        \textbf{Question} & \textbf{Mean} & \textbf{Median} & \textbf{Std. Deviation} & \textbf{Shapiro-Wilk W} & \textbf{Shapiro-Wilk p} \\
        \midrule
        Frequent Use & 4.00 & 4.00 & 0.943 & 0.841 & 0.045 \\
        Unnecessary Complex & 1.40 & 1.00 & 0.516 & 0.640 & < .001 \\
        Easy to Use & 4.60 & 5.00 & 0.516 & 0.640 & < .001 \\
        Support of Technical & 1.50 & 1.00 & 0.972 & 0.603 & < .001 \\
        Well Integrated & 4.40 & 4.00 & 0.516 & 0.640 & < .001 \\
        Inconsistency & 1.80 & 2.00 & 0.789 & 0.820 & 0.025 \\
        Learn Quickly & 4.60 & 5.00 & 0.699 & 0.650 & < .001 \\
        Cumbersome & 1.60 & 1.00 & 0.966 & 0.678 & < .001 \\
        Confident & 4.20 & 4.00 & 0.632 & 0.794 & 0.012 \\
        Learn a Lot Before & 1.10 & 1.00 & 0.316 & 0.366 & < .001 \\
        Score & 86.0 & 87.5 & 8.01 & 0.925 & 0.398 \\
        \bottomrule
    \end{tabular}

    \end{minipage}
\end{center}



\clearpage
\subsection{Data - Study on Visualizing Passed and Upcoming POIs}

\begin{center}
    \begin{minipage}{\textwidth}

        \centering
        \captionof{table}{Descriptive Statistics for the visualization-study Raw NASA Task Load Index, grouped by condition.}
        \begin{tabular}{llccccccc}
            \toprule
            \textbf{Statistic} & \textbf{Condition} & \textbf{Mental} & \textbf{Physical} & \textbf{Temporal} & \textbf{Performance} & \textbf{Effort} & \textbf{Frustration} & \textbf{Score} \\
            \midrule
            \multirow{3}{*}{Mean} & List & 28.6 & 25.5 & 20.7 & 14.8 & 29.5 & 19.5 & 23.1 \\
            & Timeline & 29.8 & 27.1 & 23.3 & 20.0 & 27.4 & 28.6 & 26.0 \\
            & Minimap & 46.7 & 36.4 & 42.4 & 42.6 & 56.9 & 50.2 & 45.9 \\
            \midrule
            \multirow{3}{*}{Median} & List & 25 & 20 & 15 & 5 & 25 & 15 & 25.0 \\
            & Timeline & 25 & 20 & 15 & 5 & 20 & 20 & 22.5 \\
            & Minimap & 40 & 30 & 45 & 40 & 60 & 55 & 49.2 \\
            \midrule
            \multirow{3}{1.5cm}{Std. Deviation} & List & 19.8 & 21.3 & 17.0 & 19.5 & 25.3 & 18.5 & 14.6 \\
            & Timeline & 22.8 & 21.3 & 19.6 & 29.0 & 26.6 & 26.0 & 19.3 \\
            & Minimap & 21.5 & 24.7 & 22.3 & 22.8 & 23.5 & 22.6 & 15.8 \\
            \midrule
            \multirow{3}{1.5cm}{Shapiro-Wilk W} & List & 0.917 & 0.879 & 0.909 & 0.767 & 0.835 & 0.851 & 0.941 \\
            & Timeline & 0.868 & 0.904 & 0.911 & 0.683 & 0.821 & 0.874 & 0.927 \\
            & Minimap & 0.947 & 0.886 & 0.920 & 0.938 & 0.954 & 0.946 & 0.927 \\
            \midrule
            \multirow{3}{1.5cm}{Shapiro-Wilk p} & List & 0.076 & 0.014 & 0.052 & < .001 & 0.002 & 0.004 & 0.229 \\
            & Timeline & 0.009 & 0.043 & 0.058 & < .001 & 0.001 & 0.011 & 0.118 \\
            & Minimap & 0.297 & 0.019 & 0.088 & 0.201 & 0.399 & 0.290 & 0.118 \\
            \bottomrule
        \end{tabular}


        \vspace{\baselineskip}

    
        \captionof{table}{Descriptive Statistics for the visualization-study System Usability Scale scores, grouped by condition.}
        \begin{tabular}{lccccc}
            \toprule
            \textbf{Condition} & \textbf{Mean} & \textbf{Median} & \textbf{Std. Deviation} & \textbf{Shapiro-Wilk W} & \textbf{Shapiro-Wilk p} \\
            \midrule
            List & 78.5 & 77.5 & 10.2 & 0.885 & 0.018 \\
            Minimap & 61.1 & 60.0 & 14.3 & 0.962 & 0.563 \\
            Timeline & 71.5 & 75.0 & 15.9 & 0.921 & 0.091 \\
            \bottomrule
        \end{tabular}


        \vspace{\baselineskip}

        
        \caption{Descriptive Statistics for the visualization-study Motion Sickness Questionnaire, grouped by time of completion.}
        \begin{tabular}{lccccc}
            \toprule
            \textbf{Study Order} & \textbf{Mean} & \textbf{Median} & \textbf{Std. Deviation} & \textbf{Shapiro-Wilk W} & \textbf{Shapiro-Wilk p} \\
            \midrule
            Pre-study & 0.381 & 0 & 0.669 & 0.617 & < .001 \\
            After First Condition & 0.857 & 0 & 1.15 & 0.732   & < .001 \\
            After Second Condition & 1.19 & 1 & 1.36 & 0.822   & 0.001  \\
            After Third Condition & 1.00 & 1 & 1.05 & 0.826    & 0.002  \\
            \bottomrule
        \end{tabular}


    \end{minipage}
\end{center}


\begin{table*}[ht]
    \centering
    \caption{Descriptive Statistics for the visualization-study User Experience Questionnaire, grouped by condition.}
    \begin{tabular}{llcccccc}
    \toprule
    \textbf{} & \textbf{Condition} & \textbf{Attractiveness} & \textbf{Perspicuity} & \textbf{Efficiency} & \textbf{Dependability} & \textbf{Stimulation} & \textbf{Novelty} \\
    \midrule
    \multirow{3}{*}{\textbf{M}} & Timeline & 1.26 & 1.42 & 1.10 & 1.12 & 1.11 & 1.37 \\
    & Minimap & 0.952 & 0.964 & 0.619 & 0.821 & 1.00 & 1.57 \\
    & List & 1.78 & 2.05 & 1.79 & 1.68 & 1.54 & 1.31 \\
    \midrule
    \multirow{3}{*}{\textbf{Mdn}} & Timeline & 1.50 & 1.75 & 1.25 & 1.25 & 1.25 & 1.50 \\
    & Minimap & 1.00 & 0.750 & 0.500 & 0.500 & 1.00 & 1.25 \\
    & List & 1.83 & 2.25 & 1.50 & 1.50 & 1.50 & 1.25 \\
    \midrule
    \multirow{3}{*}{\textbf{SD}} & Timeline & 0.921 & 1.03 & 0.937 & 0.883 & 0.986 & 1.11 \\
    & Minimap & 0.972 & 1.01 & 0.993 & 0.946 & 0.939 & 1.13 \\
    & List & 0.642 & 0.692 & 0.755 & 0.717 & 0.704 & 0.798 \\
    \midrule
    \multirow{3}{*}{\textbf{W}} & Timeline & 0.938 & 0.833 & 0.951 & 0.912 & 0.970 & 0.933 \\
    & Minimap & 0.972 & 0.975 & 0.980 & 0.948 & 0.967 & 0.877 \\
    & List & 0.976 & 0.893 & 0.847 & 0.926 & 0.942 & 0.982 \\
    \midrule
    \multirow{3}{*}{\textbf{p}} & Timeline & 0.200 & 0.002 & 0.359 & 0.059 & 0.737 & 0.160 \\
    & Minimap & 0.783 & 0.844 & 0.918 & 0.313 & 0.661 & 0.013 \\
    & List & 0.866 & 0.026 & 0.004 & 0.112 & 0.235 & 0.953 \\
    \bottomrule
    \end{tabular}
\end{table*}

\begin{table*}[ht]
    \centering
    \caption{User ranking values for the visualization-study conditions.}
    \begin{tabular}{l|cc|cc|cc}
    \toprule
    \multirow{2}{*}{\textbf{Condition}}& \multicolumn{2}{c|}{\textbf{Least Favorite}} & \multicolumn{2}{c|}{\textbf{Middle}} & \multicolumn{2}{c}{\textbf{Favorite}} \\
                & N     & \%             & N    & \%                  & N       & \%                 \\
    \midrule
    List        & 2     & 9.5\%          & 4    & 19.0\%              & 15      & 71.4\%              \\
    Minimap     & 11    & 52.4\%         & 7    & 33.3\%              & 3       & 14.3\%             \\
    Timeline    & 8     & 38.1\%         & 10   & 47.6\%              & 3       & 14.3\%             \\
    \bottomrule
    \end{tabular}
\end{table*}




\end{document}
