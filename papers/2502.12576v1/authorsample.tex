%%%%%%%%%%%%%%%%%%%% author.tex %%%%%%%%%%%%%%%%%%%%%%%%%%%%%%%%%%%
%
% sample root file for your "contribution" to a contributed volume
%
% Use this file as a template for yoown input.
%
%%%%%%%%%%%%%%%% Springer %%%%%%%%%%%%%%%%%%%%%%%%%%%%%%%%%%


% RECOMMENDED %%%%%%%%%%%%%%%%%%%%%%%%%%%%%%%%%%%%%%%%%%%%%%%%%%%
\documentclass[graybox]{svmult}

% choose options for [] as required from the list
% in the Reference Guide

\usepackage{type1cm}        % activate if the above 3 fonts are
                            % not available on your system
%
\usepackage{makeidx}         % allows index generation
\usepackage{graphicx}        % standard LaTeX graphics tool
\usepackage{orcidlink}

                             % when including figure files
\usepackage{multicol}        % used for the two-column index
\usepackage[bottom]{footmisc}% places footnotes at page bottom
% \usepackage{emoji}

% \setemojifont{Apple Color Emoji}
\usepackage{newtxtext}       % 
\usepackage{newtxmath}       % selects Times Roman as basic font
\usepackage{hyperref}
\usepackage{nameref}

% see the list of further useful packages
% in the Reference Guide

\makeindex             % used for the subject index
                       % please use the style svind.ist with
                       % your makeindex program

%%%%%%%%%%%%%%%%%%%%%%%%%%%%%%%%%%%%%%%%%%%%%%%%%%%%%%%%%%%%%%%%%%%%%%%%%%%%%%%%%%%%%%%%%

\begin{document}

\title*{A Fuzzy Evaluation of Sentence Encoders on Grooming Risk Classification}
% Use \titlerunning{Short Title} for an abbreviated version of
% your contribution title if the original one is too long
\author{Geetanjali Bihani\orcidlink{0000-0001-8352-7948}, Julia Rayz\orcidlink{0000-0003-3786-2416}}
% Use \authorrunning{Short Title} for an abbreviated version of
% your contribution title if the original one is too long
\institute{Geetanjali Bihani \at Purdue University, USA, \email{gbihani@purdue.edu},
\and Julia Rayz \at  Purdue University, USA,  \email{jtaylor1@purdue.edu}}
%
% Use the package "url.sty" to avoid
% problems with special characters
% used in your e-mail or web address
%
\maketitle

\abstract*{}

\abstract{
With the advent of social media, children are becoming increasingly vulnerable to the risk of grooming in online settings. Detecting grooming instances in an online conversation poses a significant challenge as the interactions are not necessarily sexually explicit, since the predators take time to build trust and a relationship with their victim. Moreover, predators evade detection using indirect and coded language. While previous studies have fine-tuned Transformers to automatically identify grooming in chat conversations, they overlook the impact of coded and indirect language on model predictions, and how these align with human perceptions of grooming.
In this paper, we address this gap and evaluate bi-encoders on the task of classifying different degrees of grooming risk in chat contexts, for three different participant groups, i.e. law enforcement officers, real victims, and decoys. Using a fuzzy-theoretic framework presented in \cite{bihanirayz24_regress}, we map human assessments of grooming behaviors to estimate the actual degree of grooming risk. Our analysis reveals that fine-tuned models fail to tag instances where the predator uses indirect speech pathways and coded language to evade detection. Further, we find that such instances are characterized by a higher presence of out-of-vocabulary (OOV) words in samples, causing the model to misclassify. Our findings highlight the need for more robust models to identify coded language from noisy chat inputs in grooming contexts.} 

\section{Introduction}
\label{sec:1}
Grooming refers to the insidious process where predators forge a relationship with children and build up trust with the intention of sexual exploitation \cite{ring21}.  In \cite{bihanirayz24_regress}, we noted that models assign much lower risk scores to high-risk chat contexts because these contexts contain implicit and coded language. In this work, we follow up on this initial investigation and evaluate Transformer-based classifiers on the task of detecting different degrees of grooming risk in online grooming conversations. We approach this task with a non-binary perspective, recognizing the necessity to differentiate between different levels of risk, because it can inform targeted preventive measures to safeguard individuals in realistic scenarios.

Grooming has been described as a complex multi-stage phenomenon, with the severity of grooming varying as the chat progresses\cite{ring21, olson2007}. Yet, works on automated grooming detection usually frame it as a binary classification problem \cite{preub2021, vogt2021}, categorizing entire chat instances as either grooming or non-grooming. Existing methods for grooming detection primarily rely on decoy conversations to train and fine-tune neural language models \cite{vogt2021}. Whether decoy conversations can be used to approximate real grooming scenarios has been contested in linguistic and cyberforensic analyses of online grooming conversations \cite{chiang19, ring21_2, ring22}. This disparity in data distribution can have an impact on the generalization of such models when applied out of distribution (OOD). 

We aim to address these gaps in this paper by evaluating Transformer sentence encoders on the task of classifying degrees of grooming risk expressed in natural language. We evaluate these models across diverse participant groups, specifically for real victims, law enforcement officers (LEO), and decoys. To that end, we use the fuzzy-theoretic framework that maps human assessments of grooming behaviors present within chats to the severity of grooming risk, as given in \cite{bihanirayz24_regress}. Our analysis reveals that fine-tuned sentence encoder classifiers show an increased rate of errors in identifying high-risk chat contexts, which is caused by the indirect speech pathways used by predators to manipulate and coerce victims. We find that Transformer classifiers fail to flag cases that contain coded language and lack sexually explicit content. Further, we find that the proportion of coded language used in grooming chats across different participant groups varies, causing the models to have different performance across populations. 

This finding underscores the importance of robust modeling of indirect speech acts by language models, especially those utilized by predators. Notably, to the best of our knowledge, no prior work has incorporated human assessments of risk to compare fine-tuning results, making our study unique in its comprehensive evaluation of language model performance in grooming risk classification.

% \subsection{Uncertainty in Natural Language}
% Natural language utterances are typically imprecise, containing varying degrees of vagueness. 

% Despite notable progress in this field, recent findings highlight the current inadequacies of pre-trained language models in terms of reliability in decision-making \cite{bihanirayz2024}. 


% \subsection{Online Grooming}


\section{Categorizing Grooming Risk}
\label{sec:2}

The concept of risk, much like other subjective notions, embodies fuzziness. This fuzziness becomes apparent when considering grooming strategies, where what constitutes risky behavior and what does not is often blurry. These degrees of severity are influenced by the presence of grooming strategies within a given context \cite{ring21}. Moreover, grooming strategies employed by individuals with malicious intent, can vary widely in their subtlety and perceived harm \cite{ring21}. Similar to the concept of risk, the assessment of grooming strategies itself lacks a precise boundary, as it depends on various factors such as individual perceptions, cultural norms, and situational contexts.

\begin{figure}
\centering
\includegraphics[scale=.2]{figures/pipeline.jpg}
\caption{Presence of grooming strategies present within a given chat context used to create grooming risk categories}
\label{fig:pipeline}       % Give a unique label
\end{figure}

% Yet, grooming risk has primarily been treated as a binary variable in prior literature on automated grooming detection \cite{preub2021, vogt2021}. This binary representation oversimplifies the complexity of grooming behaviors, leading to a loss of valuable information regarding the degree of risk present in a given chat context. By categorizing grooming risk solely as either present or absent, nuances in the severity of grooming behaviors are disregarded, potentially resulting in inadequate risk judgments. Previous research in this area has not adequately addressed this limitation, highlighting the need for more nuanced approaches that capture the continuum of grooming risk within chat contexts. 

In this work, we use grooming strategies present within a given text to create grooming risk categories using membership functions defined in \cite{bihanirayz24_regress}. Cyberforensic analyses of grooming conversations have shown a higher presence of grooming strategies within more advanced stages of grooming \cite{ring21}. Based on these findings, we utilize human annotations for the presence of twelve grooming strategies, as described in \cite{ring21} to estimate the degree of risk in a chat context. For a given chat context $c$, we define the total number of observed grooming strategies $o(c)$ as the sum of individual strategy scores $s_i$, where each $s_i \in \{0, 0.5, 1\}$ represents the absence (0), partial presence (0.5), or full presence (1) of the $i^{th}$ strategy. This is shown in Equation~\ref{eq:1}, also described in prior work \cite{bihanirayz24_regress}. Ideally, the maximum number of strategies assigned to a single chat line can be 12. However, we find that in our data, such cases do not occur, with the highest number of strategies used together limited to 5. This can be attributed to the limited context that the predator has to work with, where they test out different strategies during different stages of the conversation \cite{kloess2017}. For a detailed description of the grooming strategies used in this work, please refer to Table~\ref{app_tab:1} in the Appendix.

\begin{equation}
\label{eq:1}
o(c)=\sum_{i=1}^{N} s_{i}(c)
% o = \sum_{strategies}{\mu_{strategy}(c)} 
\end{equation}


% For figures use
%

\begin{equation}
\label{eq:2}
\mu_{risk}(c) = \varphi\left(o(c)-m\right) \text{, where }\varphi(z)=\frac{e^{-z^{2} / 2}}{\sqrt{2 \pi}}
\end{equation}
\begin{equation}
\label{eq:2.1}
\mu_{mod}(c) = \mu^{0.2}_{risk}(c) \text{ for $m=0.2$}
\end{equation}
\begin{equation}
\label{eq:2.2}
\mu_{sig}(c) = \mu_{risk}(c) \text{ for $m=1$}
\end{equation}
\begin{equation}
\label{eq:2.3}
\mu_{sev}(c) = \mu^2_{risk}(c) \text{ for $m=2$}
\end{equation}


We employ a Gaussian membership function to map the overall strategy membership $(o(c))$ to various degrees of grooming risk, as outlined in Equation~\ref{eq:2}. While this equation was originally introduced in \cite{bihanirayz24_regress} for fuzzy evaluation of grooming risk scoring, we leverage this framework to delineate grooming risk into three distinct severity levels, which are not necessarily mutually exclusive. These categories are defined as \textit{moderate}, \textit{significant}, and \textit{severe} levels of grooming risk, arranged in ascending order of risk severity. The respective fuzzy membership functions for moderate, significant, and severe risk are defined in Equations~\ref{eq:2.1}, \ref{eq:2.2} and \ref{eq:2.3}. Using this method, chat contexts with fewer grooming strategies are assigned higher membership values within the \textit{moderate} risk category. Conversely, those with a greater number of strategies present receive higher memberships within the \textit{significant} and \textit{severe} grooming risk categories. For our classifier fine-tuning process, we defuzzify the risk degree using an $\alpha$-cut$=0.5$, selecting the highest degree of risk membership exceeding the $\alpha$-cut. Thus, if a chat context has memberships surpassing the $\alpha$-cut in \textit{moderate} and \textit{severe} categories, it is categorized as a \textit{severe} risk chat context.



% \begin{equation}
% \label{eq:3}
% \mu(moderate) = \mu_{risk}(0.2)
% \end{equation}


% \begin{equation}
% \label{eq:4}
% \mu(significant) = \mu_{risk}(1)
% \end{equation}

% \begin{equation}
% \label{eq:5}
% \mu(severe) = \mu_{risk}(2)
% \end{equation}


\section{Methodology}
\label{sec:3}
This section details our approach to fine-tuning bi-encoders specifically for grooming risk classification, utilizing human assessments of grooming strategy behaviors in chat contexts. Our aim in this investigation is to evaluate the potential of fine-tuning models from the Transformer bi-encoder family \cite{sbert2019}, to estimate the extent of grooming risk present in chat interactions. Our focus extends beyond conventional language cues, as we aim to discern the limitations of these bi-encoders, particularly in scenarios where higher-risk situations may not always manifest through explicit language. Drawing from prior research indicating variations in language usage across different groups in grooming behaviors \cite{ring21}, we fine-tuned and evaluated our bi-encoders separately for distinct grooming contexts. Specifically, we focused on grooming conversations involving predators interacting with law enforcement officers (LEO), victims, and decoys. This analysis underscores the oversight of current automated models of grooming classification in accounting for nuanced differences across different participant conversations. 

\subsection{Task Definition}
\label{subsec:3.1}

\textbf{Definition 1 - Chat window length $(w)$}: Number of messages in an exchange between two participants in a grooming conversation.

\noindent \textbf{Definition 2 - Chat Context $(c)$}: A sequence comprising the current and last $w-1$ messages in a grooming conversation. We fix $w=4$ for our analysis.

\noindent \textbf{Definition 3 - Grooming risk category $(C_{groom})$}: Denotes the severity of grooming within a chat context within the following three categories: \{\textit{moderate}, \textit{significant}, \textit{severe}\}.

\subsection{Models Studied}
\label{subsec:2}
We conducted our analysis on pre-trained bi-encoder models. These encoders leverage siamese and triplet network structures to derive sentence embeddings with semantically meaningful representations. These representations are optimized to capture semantic similarity between sentences in a vector space, making them suitable for various downstream tasks such as semantic search and clustering. The models we analyze include Sentence-BERT (SBERT) \cite{sbert2019}, MPNET \cite{mpnet}, and RoBERTa \cite{roberta}. The base models, as given within the names, are BERT, MPNET and RoBERTa respectively. We choose BERT and RoBERTa based on their application in prior work, and MPNET based on the quality of embeddings it generates.

\subsection{Fine-tuning Details}
\label{subsec:3}
The process of fine-tuning involves adapting pre-trained models to specific downstream tasks using task-specific data. We fine-tune bi-encoders to predict grooming risk class for a given chat context $c$ by optimizing on minimizing the cross-entropy loss between the predicted and actual grooming risk classes. For model fine-tuning, we use an Adam optimizer \cite{kingma2014}, learning rate of $2.10^{-5}$, over $5$ epochs, with a batch size of $4$.

To understand the performance differences of models fine-tuned on interactions of predators with different participant groups, we fine-tuned three separate models. Thus, one model underwent fine-tuning on interactions of predators with law enforcement officers (LEO), another on interactions with real victims (Victim), and a third on interactions with decoys (Decoy).

\section{Results}
We examined how well fine-tuned bi-encoder predictions fare on the task of identifying varying degrees of grooming risk in predatory chat contexts, across different participant groups. We consider chat contexts with a window size of $w=4$. Our analysis reveals variations in model performance across different levels of grooming risk, with the model achieving higher accuracy in \textit{moderate} contexts but showing poorer performance in \textit{significant} and \textit{severe} risk scenarios. 

% Notably, models trained on real-victim chats perform worse than those trained on law enforcement officers (LEO) and decoy chats, highlighting the importance of accounting for linguistic variations across participant groups in training automated grooming risk estimation models.

% summarize findings

% \subsection{Model Performance}

We report macro-averaged F1 scores on predictions in Table~\ref{tab:3}, and note a stark difference between F1 scores on \textit{moderate}, \textit{significant} and \textit{severe} risk contexts. We find that across the three participant groups, \textit{moderate} and \textit{severe} risk contexts receive higher coverage as compared to \textit{significant} risk contexts. These results highlight the limitations of fine-tuned bi-encoder models in detecting grooming behaviors where human evaluators have indicated a higher presence of grooming strategies. Moreover, all three models show similar error rates across different degrees of risk. These findings can be attributed to their inability to capture indirect communication pathways and adversarial grooming entrapment language (e.g. misspelled words, abbreviations, emojis, etc.) utilized by predators \cite{Lykousas2021}. We also find that the models show the worst performance for \textit{significant} risk scenarios for decoy chats while showing a ~$10\%$-$20\%$ better F1 score for law enforcement and real victim chats. This can be attributed to the differences in the way grooming conversations progress for real victims versus decoys \cite{chiang19}. These findings highlight the differences in grooming detection model performance across different participant groups and question the inherent assumption made by the prior work regarding training and finetuning automated models of grooming risk estimation using unrepresentative decoy and law enforcement chats.


\begin{table}[!t]
\centering
\caption{F1 score reported for different participant groups}
\label{tab:3}       % Give a unique label
%
% Follow this input for your own table layout
%
\begin{tabular}{p{0.2\textwidth}p{0.2\textwidth}p{0.1\textwidth}p{0.1\textwidth}p{0.1\textwidth}}
\hline\noalign{\smallskip}
\textbf{Model} & \textbf{Grooming Risk}  & \textbf{LEO}  & \textbf{Victim}  & \textbf{Decoy}  \\
\noalign{\smallskip}\hline\noalign{\smallskip}
$S_{\text {\textbf{\textit{BERT-base}}}}$ & Moderate &  0.83 & 0.90 & 0.84 \\
& Significant & 0.47 & 0.39 & 0.26 \\
& Severe & 0.53 & 0.74 &  0.67 \\
\noalign{\smallskip}\hline\noalign{\smallskip}
& Overall & 0.70 & 0.68 & 0.59  \\
\noalign{\smallskip}\hline\noalign{\smallskip}

$S_{\text {\textbf{\textit{RoBERTa-base}}}}$ & Moderate & 0.84  & 0.89 & 0.83 \\
& Significant & 0.49 & 0.43 & 0.27 \\
& Severe & 0.80 & 0.68 & 0.71 \\
\noalign{\smallskip}\hline\noalign{\smallskip}
& Overall & 0.71 & 0.67 & 0.60 \\
\noalign{\smallskip}\hline\noalign{\smallskip}

$S_{\text {\textbf{\textit{MPNet}}}}$ & Moderate & 0.83 & 0.90 & 0.86 \\
& Significant & 0.46 & 0.42 & 0.27 \\
& Severe & 0.82 & 0.71 & 0.67 \\
\noalign{\smallskip}\hline\noalign{\smallskip}
& Overall & 0.71 & 0.68 & 0.60  \\
\noalign{\smallskip}\hline\noalign{\smallskip}
\end{tabular}
% $^a$ Table foot note (with superscript)
\end{table}


% For figures use
%
\begin{figure}[htbp]
\centering
\includegraphics[scale=.12]{figures/all_heatmap_new.jpg}
\caption{Confusion matrices illustrating the classification performance of the models across different participant groups: law enforcement officers (LEO), victims, and decoys. Each heatmap represents the comparison between actual and predicted labels for grooming risk categories (moderate, significant, severe). The top label refers to the participant group on which the model was fine-tuned. The color intensity denotes the count of instances falling into each category, with blue shades indicating higher counts, and red shades indicating lower counts. }
\label{fig:heatmap}       % Give a unique label
\end{figure}

We also analyzed model predictions across different participant groups, reporting the classification performance as depicted in Figure~\ref{fig:heatmap}. We find that for law enforcement conversations, the model's classification performance has high coverage across all severities of grooming risk. On the other hand, the model incorrectly classifies more than~$40\%$ of \textit{significant} and \textit{severe} risk samples for decoy chats. Similar to the findings in \cite{bihanirayz24_regress}, our model is unable to detect \textit{significant} and \textit{severe} grooming risk contexts due to an absence of explicit sexual and predatory language in many cases. Instead, predators adopt the use of adversarial grooming entrapment language (e.g. misspelled words, abbreviations, emojis, etc.), which lends to a higher occurrence of out-of-vocabulary (OOV) tokens in the fine-tuning set. Refer to Table ~\ref{tab:oov} for a comparison of the presence of OOV tokens across different participant groups. An increased proportion of OOV tokens in chat instances leads to the deterioration of model performance, causing the model to not flag severe grooming risk instances.  

\begin{table}[htbp]
\caption{OOV tokens across chats from different participant groups}
\label{tab:oov}       % Give a unique label

\begin{tabular}{cccc}
\hline\noalign{\smallskip}
 & \textbf{LEO} & \textbf{Victim} & \textbf{Decoy} \\
\noalign{\smallskip}\hline\noalign{\smallskip}
OOV \% & $26.70$ & $20.68$ & $30.80$ \\
OOV words per message chunk & $4.85$ & $4.88$ & $6.48$ \\
\noalign{\smallskip}\hline\noalign{\smallskip}
\end{tabular}
% $^a$ Table foot note (with superscript)
\end{table}



\section{Discussion and Conclusion}
    This paper investigates whether bi-encoders can learn to classify varying degrees of grooming risk inherent in online grooming conversations. We fine-tune and evaluate a transformer bi-encoder on the task of grooming detection across different participant groups, for increasing degrees of risk. Our analysis highlights that fine-tuned sentence encoders still exhibit higher Detecting coded language in harmful contexts is a non-trivial task, particularly in domains like online grooming, where automated models of detection cannot discount precision.  While Transformer bi-encoders like SBERT show promise, they rely on surface-level features and are fine-tuned on decoy conversations instead of real victim chats, leaving questions about their effectiveness in realistic scenarios unanswered. Additionally, previous work has not compared model fine-tuning results with human assessments of grooming risk. This study addresses this gap by investigating the ability of bi-encoders to predict the level of grooming risk in chat contexts, for three different participant groups, i.e. law enforcement officers, real victims, and decoys. in categorizing more severe risk contexts, misclassifying them as the lowest severity defined in this work. We find that such discrepancies are tied to cases where surface form text does not contain explicit identifiers of grooming, but rather uses indirect speech pathways to manipulate victims. Our results align with recent research which shows that many predatory behaviors can evade detection from current filters through tactics like modifying explicit identifiers, introducing typos, using emojis, and out-of-vocabulary words \cite{Lykousas2021}. In such cases, fine-tuning sentence embedding models does not help the model learn useful indicators of grooming within natural language and is a trivial approach to solving a non-trivial task. The sole reliance on word form and incentivizing training loss lead to learning shortcuts while ignoring nuance \cite{bihanirayz2024}.  Even with the integration of long-range context, the task of encoding intricate lexical semantic phenomena to enhance natural language understanding continues to be a challenge \cite{bihanirayz21, vulic2021}. This finding underscores how the current fine-tuning methods for transformer bi-encoders still need to be improved, to be deployed in real-world scenarios and calls for the need for robust modeling of indirect speech acts employed in grooming contexts by language models. We plan to address these findings in future work. 


\begin{acknowledgement}
This work is supported by the DoJ grants 15PJDP-21-GK-03269-MECP and 15PJDP-22-GK-03107-MECP. 
\end{acknowledgement}


% Template for ISBI paper; to be used with:
%          spconf.sty  - ICASSP/ICIP LaTeX style file, and
%          IEEEbib.bst - IEEE bibliography style file.
% --------------------------------------------------------------------------
\documentclass{article}
\usepackage{spconf,amsmath,graphicx}

% It's fine to compress itemized lists if you used them in the
% manuscript
\usepackage{enumitem}
\usepackage{multirow}
\setlist{nosep, leftmargin=14pt}
\usepackage{booktabs}
\usepackage{caption}
\usepackage{subcaption}
\usepackage{mwe} % to get dummy images
\usepackage{url}

% Example definitions.
% --------------------
\def\x{{\mathbf x}}
\def\L{{\cal L}}

% Title.  
% ------
\title{Anatomical Grounding Pre-training for Medical Phrase Grounding}
%
% Single address.
% ---------------
% \name{Wenjun Zhang, Shakes Chandra, Aaron Nicolson}
% \address{University of Queensland}
%
% For example:
% ------------
%\address{School\\
%	Department\\
%	Address}
%
% Two addresses (uncomment and modify for two-address case).
% ----------------------------------------------------------
%\twoauthors
%  {A. Author-one, B. Author-two\sthanks{Some author footnote.}}
%	{School A-B\\
%	Department A-B\\
%	Address A-B}
%  {C. Author-three, D. Author-four\sthanks{The fourth author performed the work
%	while at ...}}
%	{School C-D\\
%	Department C-D\\
%	Address C-D}
%
% More than two addresses
% -----------------------
\name{Wenjun Zhang$^{\star}$ \qquad Shekhar S. Chandra$^{\star}$ \qquad Aaron Nicolson$^{\dagger}$}

\address{$^{\star}$University of Queensland\\$^{\dagger}$Australian e-Health Research Centre, CSIRO Health and Biosecurity, Brisbane, Australia}
% \address{$^{\dagger}$}Australian e-Health Research Centre, CSIRO Health and Biosecurity, Brisbane, Australia \\}

\begin{document}
%\ninept
%
\maketitle

\begin{abstract}

Medical Phrase Grounding (MPG) maps radiological findings described in medical reports to specific regions in medical images. The primary obstacle hindering progress in MPG is the scarcity of annotated data available for training and validation. We propose anatomical grounding as an in-domain pre-training task that aligns anatomical terms with corresponding regions in medical images, leveraging large-scale datasets such as Chest ImaGenome. Our empirical evaluation on MS-CXR demonstrates that anatomical grounding pre-training significantly improves performance in both a zero-shot learning and fine-tuning setting, outperforming state-of-the-art MPG models. Our fine-tuned model achieved state-of-the-art performance on MS-CXR with an mIoU of 61.2, demonstrating the effectiveness of anatomical grounding pre-training for MPG.

% Phrase grounding models maps phrases to specific regions in an image, while for medical phrase grounding, the phrase 

\end{abstract}

\section{Introduction}
MPG involves mapping a descriptive phrase containing a radiological finding to a specific region in a medical image \cite{10.1007/978-3-031-43990-2_35}. An MPG model could be used to visually connect findings in a radiologist report---whether produced by radiologist or by automatic report generation model---to the corresponding regions in the images. Findings accompanied by their associated bounding boxes are easier to verify, enhancing the reliability of reported information \cite{bernstein_can_2023, 10204026, doi:10.1148/ryai.2020190043}.

MPG is a specialised application within the broader field of phrase grounding. State-of-the-art general-domain phrase grounding models are pre-trained on large-scale phrase-to-region datasets and demonstrate strong zero-shot learning and few-shot transferability on downstream localisation tasks \cite{9879567, 9710994, 10.1007/978-3-031-72970-6_3}. However, despite their success in general-domain tasks, these models struggle to generalise to MPG, especially in a zero-shot learning setting. One possible reason is the significant domain shift from general-domain to medical-domain data \cite{zhao-titov-2023-transferability}. Furthermore, large-scale pre-training is challenging in the medical domain due to the scarcity of annotated MPG datasets, with only a small public benchmark dataset available \cite{10.1007/978-3-031-20059-5_1}. 

To overcome the challenges of limited MPG training data and the large domain gap between MPG and the general phrase grounding data, we propose to leverage anatomical grounding as an in-domain pre-training task for MPG, as demonstrated in Figure \ref{fig:concept} (middle). Anatomical grounding involves aligning text describing an anatomical region with the corresponding region within a medical image. This approach leverages the extensive anatomical text-to-region data available in datasets such as Chest ImaGenome \cite{wu2021chestimagenomedatasetclinical}, enabling effective fine-tuning or zero-shot learning for MPG tasks, where data is more limited \cite{ 10.1007/978-3-031-20059-5_1}. This pre-training step equips the model to recognise common anatomical landmarks, which radiologists frequently reference when describing findings in radiological reports. For instance, by learning to localise the \textit{right apical zone} with the Chest ImaGenome dataset, the model is more capable of localising findings such as a \textit{small right apical pneumothorax}.


\begin{figure}
    \centering
    \includegraphics[width=1\linewidth]{concept.png}
    \caption{Anatomical grounding as an in-domain pre-training task for Medical Phrase Grounding (MPG).}
    \label{fig:concept}
\end{figure}

We evaluated the effectiveness of anatomical grounding pre-training on MS-CXR, a MPG dataset, using two pre-trained general-domain phrase grounding models, TransVG \cite{9710016} and MDETR \cite{9710994}. We also evaluate it in both a zero-shot learning and a fine-tuning setting. Figure \ref{fig:concept} describes the training process; TransVG or MDETR is first pre-trained on anatomical grounding. They are then fine-tuned on MPG (if they are not evaluated in a zero-shot learning setting). Our empirical evaluation demonstrates that anatomical grounding pre-training significantly improves performance in a zero-shot learning setting, and significantly improves the performance of MDETR in a fine-tuning setting. We compare our anatomically grounded pre-trained models to state-of-the-art MPG models from the literature, and demonstrate that our models achieve an improvement in performance. The pre-trained models, and demo for this work are available at: \url{https://github.com/Claire1217/AGPT}.


\section{Related Work}
\subsection{General-domain Phrase Grounding}
Vision-language models pre-trained on large-scale image-text datasets, such as CLIP, have shown strong zero-shot learning and few-shot learning capabilities on global image understanding tasks \cite{pmlr-v139-radford21a}. GLIP extends this by pre-training on large-scale phrase grounding data \cite{9879567}. The learned representations demonstrate strong transferability to various local-level recognition tasks. Current pre-trained general-domain phrase grounding models are typically applied to two primary tasks: phrase localisation and referring expression comprehension. Phrase localisation focuses on identifying and locating multiple objects mentioned in a sentence. MDETR is a phrase localisation model, associating sub-phrases within a sentence with multiple object queries \cite{9710994}. In contrast, TransVG is a referring expression comprehension model---it detects a single object or region in an image for a whole sentence \cite{9710016}.

\subsection{Medical Phrase Grounding}
Due to the scarcity of annotated data, MPG has received limited attention in the literature. Boecking \textit{et al.} introduced MS-CXR, a phrase grounding chest X-ray benchmark dataset \cite{10.1007/978-3-031-20059-5_1}. Their objective with the dataset was to evaluate the grounding performance of their self-supervised biomedical vision-language model (BioViL). BioViL demonstrates strong zero-shot learning capabilities, given that it is not trained for MPG. Recently, Chen \textit{et al.} directly fine-tuned TransVG on a split of MS-CXR in order to directly learn MPG, forming MedRPG \cite{10.1007/978-3-031-43990-2_35}. Here, a bounding box supervised loss and a specific contrastive loss were leveraged. Unlike these models, we pre-train on large-scale anatomical grounding data using Chest ImaGenome, in order to provide in-domain pre-training.

\subsection{Anatomical Information in Medical Imaging}
Anatomical information has been effectively used in tasks like pathology detection and classification to improve accuracy and localisation. For example, the Anatomy-Driven Pathology Detection (ADPD) model \cite{muller_anatomy-driven_2023} used easy-to-annotate anatomical regions as proxies for pathologies, helping to locate disease locations without detailed pathology-specific bounding boxes. AnaXNet \cite{agu_anaxnet_2021} used anatomical relationships to improve classification by identifying the exact regions where findings occur. Despite these successes, no work has applied anatomical information to medical phrase grounding. 

\section{Methodology}\label{sec:methodology}
Our work addresses \textbf{medical phrase grounding} (MPG),  which involves mapping a descriptive phrase containing radiological finding to a specific
region in a medical image. This can be defined as learning a function  \( f: P \times I \rightarrow B \), where \( P \) represents the set of medical phrases, \( I \) represents the set of medical images, and \( B \) represents the set of bounding boxes. Given a phrase \( p \in P \) and an image \( i \in I \), the model predicts a bounding box \( b \in B \) such that $b = f(p, i)$. Our approach introduces a novel training framework for MPG, which involves extending the pre-training of general phrase grounding models with an anatomical grounding pre-training. 

Anatomical grounding involves predicting bounding boxes for anatomical structures using textual descriptions of their locations. The task can be formulated as 
\( f_{\text{anat}}: A \times I \rightarrow B \). Specifically, for each anatomical term \( a \in A \) and image \( i \in I \), the model predicts a bounding box \( b \in B \) such that $b = f_{\text{anat}}(a, i; \theta_{\text{gen}})$, 
where \( \theta_{\text{gen}} \) are the initial general-domain pre-trained weights. Through anatomical grounding pre-training, we refine the weights to create anatomy-specific parameters  \( \theta_{\text{anat}} \). 

To enhance generalisation and robustness, we leverage GPT-4 to generate four additional synonymous variations for each anatomical location in the Chest ImaGenome dataset. This aligns with clinical practice, where radiologists frequently use interchangeable terms to describe the same region. For example, ``left lung base” might also be referred to as ``left basal lung” or ``left lower lung base”. The detailed augmentation of anatomical regions is included in the aforementioned code repository. 

\section{Datasets} \label{sec:dataset}

\paragraph*{Chest ImaGenome \cite{wu2021chestimagenomedatasetclinical}}
We use the Chest ImaGenome dataset for anatomical grounding pre-training. Chest ImaGenome is a scene graph-structured dataset that includes $242\,072$ images. It contains $1\,256$ combinations of relational annotations between 29 anatomical structures in chest X-rays, with bounding box coordinates and additional attributes organised as a scene graph per image. In this study, we use the names and bounding box coordinates of these 29 anatomical structures, focusing specifically on frontal images. Examples of anatomical structures include ``left lung base", ``left lung apical zone", and ``right hilar structures".

\paragraph*{MS-CXR \cite{10.1007/978-3-031-20059-5_1}} 
We use the MS-CXR dataset for the MPG task. It contains $1\,162$ medical phrase-bounding box pairs across eight pathologies, such as \textit{cardiomegaly} and \textit{pleural effusion}. The findings are manually annotated and described by radiologists, ensuring precise alignment between medical phrases and bounding boxes. Example phrases include ``Large right-sided pneumothorax", and ``Small bilateral pleural effusions". The whole dataset was used for testing for the zero-shot learning setting with the general-domain pre-trained and anatomical pre-trained phrase grounding models, while the train-test-val split from \cite{10.1007/978-3-031-43990-2_35} was used for the fine-tuning setting. 

\section{Experiment Setup}
\paragraph*{Model}
Experiments were conducted with two models, TransVG and MDETR. For TransVG, ResNet-50 and ClinicalBERT were used as the visual and text encoders, respectively, whereas ResNet-101 and RoBERTa-base were used for MDETR. Here, MDETR functions on a sentence-level, mapping a medical phrase to one region in an image. This differs from its standard function, where it maps multiple sub-phrases from a sentence to multiple regions in the image. Full-model anatomical grounding pre-training of MDETR resulted in an unstable training process, likely due to its multi-object detection task. To address this, we applied Low-Rank Adaptation (LoRA) \cite{Hu2021LoRA:Models} during anatomical grounding pre-training. This likely stabilised training by limiting trainable parameters to low-rank layers, preventing drastic weight updates and reducing instability during adaptation.

\paragraph*{Pre-training and Fine-Tuning}
For anatomical grounding pre-training, we process mini-batches of eight images, each paired with five anatomical regions chosen from five synonymous terms, creating 40 anatomical text-region pairs per mini-batch. For MPG fine-tuning, both models were trained on the MS-CXR training set with a mini-batch size of 12. During fine-tuning, all of the weights of MDETR were trainable, including the LoRA weights. The AdamW optimiser with a learning rate of 1e-4 and 1e-5 was used for pre-training and fine-tuning, respectively \cite{DBLP:conf/iclr/LoshchilovH19}. Each model was trained for 1 epoch during pre-training and 90 epochs during fine-tuning. Images were resized and padded to a size of 640$\times$640. During training, the images were augmented with colour jitter and Gaussian noise.

% When fine-tuning the anatomical pre-trained models on the training set of MS-CXR. The task is formulated as \( f_{\text{MPG}}: P \times I \rightarrow B \), where given a medical phrase \( p \in P \) and an image \( i \in I \), the task is to produce a bounding box \( b \in B \) as follows: $b = f_{\text{MPG}}(p, i; \theta_{\text{anat}})$. With fine-tuning, the weights are updated to \( \theta_{\text{MPG}} \). 

\paragraph*{Evaluation}
We used mIoU and accuracy (Acc) as metrics. For accuracy, a predicted bounding box was considered true if the mIoU with the ground truth bounding box was larger than 0.5. We evaluate the anatomical grounding pre-trained MDETR and TransVG models on the MS-CXR dataset in both zero-shot learning and fine-tuning settings. The self-supervised pre-trained models GLoRIA \cite{9710099} and BioViL \cite{10.1007/978-3-031-20059-5_1} were used for comparison. In the fine-tuning setting, we further fine-tuned the anatomical grounding pre-trained MDETR and TransVG models on the training split of MS-CXR (described in Section \ref{sec:dataset}). These were compared to MDETR and TransVG without anatomical grounding pre-training and MedRPG \cite{10.1007/978-3-031-43990-2_35}. For zero-shot learning and fine-tuning, the epoch with the highest validation mIoU was selected for testing.

\section{Results \& Discussion}
\subsection{Effectiveness of Anatomical Grounding Pre-training}
The performance of anatomical grounding pre-training is demonstrated in Table \ref{tab:anat_comparison}. Applying MDETR and TransVG to MPG in a zero-shot learning setting produced low scores on both metrics, underscoring the limitations of general-domain phrase grounding models for MPG. However, pre-training with anatomical grounding led to a statistically significant improvement in both models’ performance across both metrics for zero-shot learning of MPG. These results demonstrate that anatomical grounding pre-training improves the models’ ability to generalise to MPG.

\begin{table}[ht]
\small
\centering
\caption{Performance of \textbf{anatomical grounding pre-training (AGPT)} on MS-CXR. Underlined indicates a stat. sig. difference to the model without anatomical grounding pre-training ($p < 0.05)$.}
\renewcommand{\arraystretch}{0.85}
\begin{tabular}{lcccc}
\toprule
\multirow{2}{*}{\textbf{Model}} & \multicolumn{2}{c}{\textbf{Zero-shot}} & \multicolumn{2}{c}{\textbf{Fine-tuning}} \\
\cmidrule(lr){2-3} \cmidrule(lr){4-5}
                                & \textbf{Acc}   & \textbf{mIoU}   & \textbf{Acc}    & \textbf{mIoU}   \\
\midrule
TransVG           & 1.2         & 10.3         & 68.9          & \textbf{59.4}          \\
\quad+ AGPT   & \textbf{\underline{39.8}}         & \textbf{\underline{40.7}}         & \textbf{70.7}          & 59.2 \\
\addlinespace % Adds space between the two groups
MDETR              & 3.0         & 14.7         & 66.9          & 57.8         \\
\quad+ AGPT   &\textbf{ \underline{34.7}}         & \textbf{\underline{32.6}}         & \textbf{\underline{70.7} }         & \textbf{\underline{61.2}} \\
\bottomrule
\end{tabular}
\label{tab:anat_comparison}
\end{table}


When fine-tuning on the MS-CXR training set, anatomical grounding pre-training produced a statistically significant improvement across all metrics for MDETR. It also demonstrated an improvement with TransVG for Acc. This indicates that anatomical grounding pre-training is effective for MPG fine-tuning, particularly for certain types of models.

In Figure \ref{fig:viz}, we illustrate the models performing MPG in zero-shot learning settings on two examples: ``right apical pneumothorax" and ``mild cardiomegaly". Without anatomical grounding pre-training, both TransVG and MDETR fail to ground the phrases accurately. However, with anatomy pre-training, both models are able to ground the text to the correct anatomical region---the right apical zone for pneumothorax and the heart for cardiomegaly. Fine-tuning offers a further improvement in the grounding accuracy.

\begin{figure}[t]
    \centering
    \includegraphics[width=1\linewidth]{vizz.png}
    \caption{MPG with and without \textbf{anatomical grounding pre-training (AGPT)}. The top example contains the anatomical region within the text, whereas the bottom example does not. Blue and red boxes indicate the ground-truth and predicted bounding boxes, respectively.}
    \label{fig:viz}
\end{figure}



\begin{table*}[t]
\small
\centering
\caption{A comparison of \textbf{anatomical grounding pre-training (AGPT)} with other models in the literature in both zero-shot learning and fine-tuning settings with \textbf{mIoU} as the metric. $\dagger$ indicates scores sourced from the BioViL paper \cite{10.1007/978-3-031-20059-5_1}.}
\label{table:combined_iou_scores}
\renewcommand{\arraystretch}{0.85}
\begin{tabular}{l c c c c c c c c c c}
\toprule
\textbf{Model} & \textbf{Supervision} & \textbf{Cardio.} & \textbf{Opacity} & \textbf{Edema} & \textbf{Consol.} & \textbf{Pneu.} & \textbf{Atelect.} & \textbf{Pneumo.} & \textbf{Pl. Eff.} & \textbf{Avg} \\
\midrule
\multicolumn{11}{c}{\textbf{Zero-shot learning}} \\ 
\midrule
\textbf{GLoRIA \cite{9710099}}$^\dagger$ & Self-super. & 27.3 & 19.8 & 25.1 & 32.4 & 24.6 & 26.1 & 10.0 & 25.4 & 24.6 \\
\textbf{BioViL \cite{10.1007/978-3-031-20059-5_1}}$^\dagger$ & Self-super. & 37.5 & 20.9 & \textbf{27.5} & \textbf{34.6} & 31.5 & 30.2 & 13.5 & \textbf{31.5} & 28.4 \\
\cmidrule(lr){1-11}
\textbf{MDETR + AGPT} & Box-super. & 61.3 & 6.0 & 8.7 & 18.5 & 18.8 & 8.2 & 16.1 & 14.6 & 32.6 \\
\textbf{TransVG + AGPT} & Box-super. & \textbf{61.5} & \textbf{23.0} & 14.5 & 33.0 & \textbf{31.9} & \textbf{39.3} & \textbf{26.9} & 21.1 & \textbf{40.7} \\
\midrule
\multicolumn{11}{c}{\textbf{Fine-tuning}} \\ 
\midrule
\textbf{MedRPG \cite{10.1007/978-3-031-43990-2_35}} & Box-super. & 80.5 & 39.3 & \textbf{51.7} & 49.1 & 46.4 & \textbf{48.8} & 38.5 & 52.8 & 59.6 \\
\textbf{MDETR \cite{9710994}} & Box-super. & 79.6 & 43.1 & 45.5 & 45.8 & 40.1 & 36.0 & 39.1 & 50.5 & 57.8 \\
\textbf{TransVG \cite{9710016}} & Box-super. & 80.6 & \textbf{46.8} & 35.6 & 42.7 & \textbf{48.5} & 42.8 & 38.3 & 49.5 & 59.4 \\
\cmidrule(lr){1-11}
\textbf{MDETR + AGPT} & Box-super. & \textbf{81.2} & 45.1 & 25.2 & \textbf{56.3} & 38.9 & 47.4 & \textbf{43.1} & \textbf{57.2} & \textbf{61.2} \\
\textbf{TransVG + AGPT} & Box-super. & 79.1 & 37.6 & 43.0 & 45.4 & 45.9 & 47.7 & 41.9 & 54.1 & 59.2 \\
\bottomrule
\end{tabular}
\end{table*}
 
\subsection{Comparison to other MPG models}
First, we compare our anatomical grounding pre-trained MDETR and TransVG models (MDETR + AGPT and TransVG + AGPT, respectively) in a zero-shot learning setting, as shown in Table \ref{table:combined_iou_scores}. We compare these to two self-supervised models, GLoRIA \cite{9710099} and BioViL \cite{10.1007/978-3-031-20059-5_1}. Both MDETR + AGPT and TransVG + AGPT outperformed GLoRIA and BioViL. This indicates that anatomical grounding pre-training is more effective for zero-shot learning MPG than the self-supervised learning strategies of GLoRIA and BioViL. Furthermore, our fine-tuned MDETR + AGPT model attained an mIoU improvement of 1.6 over the current state-of-the-art model, MedRPG \cite{10.1007/978-3-031-43990-2_35}. 

% The baseline performance for the two models is from the BioViL paper.

\subsection{Effectiveness of Synonymous Anatomical Term Augmentation}
We conducted ablation studies to evaluate the impact of adding synonymous variations of the anatomical locations, as described in Section \ref{sec:methodology}. The results show that synonymous augmentation improved the scores for both TransVG and MDETR, with a stronger effect observed in TransVG. Notably, anatomical grounding pre-training with synonymous augmentation led to a 15.6\% improvement in zero-shot learning accuracy. This provides the model with a broader range of terms for the same anatomical location. This allows the model to better generalise to new phrases in a zero-shot learning setting.
\begin{table}[h!]
\small
\centering
\caption{Improvement in performance with when using synonymous variations of the anatomical locations. Underlined indicates a stat. sig. difference to the model without synonymous variations ($p < 0.05)$.}
\label{table:syn_augmentation_effect}
\renewcommand{\arraystretch}{0.85}
\begin{tabular}{lcccc}
\toprule
\multirow{2}{*}{\textbf{Model}}  & \multicolumn{2}{c}{\textbf{Zero-shot}} & \multicolumn{2}{c}{\textbf{Fine-tuning}} \\
\cmidrule(lr){2-3} \cmidrule(lr){4-5}
 & \textbf{Acc} & \textbf{mIoU} & \textbf{Acc} & \textbf{mIoU} \\
\midrule
TransVG   & \underline{+15.6} & \underline{+13.9} & \underline{+5.4} & \underline{+2.6} \\
MDETR     & \underline{+1.8}  & \underline{+1.2}   & +2.4 & +0.9 \\
\bottomrule
\end{tabular}
\end{table}

\vspace{-10pt}
\section{Conclusion}
In this paper, we introduced anatomical grounding pre-training to address the challenges of MPG, a task constrained by limited in-domain data and significant domain shifts from general-domain pre-trained models. Our methodology involved pre-training phrase grounding models on anatomical text-region pairs using the Chest ImaGenome dataset, followed by MPG-specific fine-tuning on the MS-CXR dataset. Empirical results demonstrated that anatomical grounding pre-training significantly improved zero-shot learning and fine-tuning performance on MPG, surpassing existing self-supervised and state-of-the-art MPG models. Additionally, our augmentation with synonymous anatomical terms further enhanced generalisability. This work demonstrates that leveraging anatomical grounding pre-training is an effective solution to the challenge of limited MPG data.

\section{Compliance with Ethical Standards}  
\vspace{-2mm} % Reduce space above section title
\noindent This study used public data from MIMIC-CXR (under PhysioNet’s credentialed license). Ethical approval was not required as confirmed by the license attached with the open access data.
\vspace{-2mm}

\section*{Acknowledgments}
\vspace{-2mm} % Reduce space before acknowledgments
\noindent No funding was received. The authors declare no competing interests.  

% Tighten bibliography spacing
\vspace{-5mm} % Reduce space before references


\bibliographystyle{IEEEtran}
\bibliography{isbi/ISBI_2024_template-master/references}
\end{document}


%
\newpage
\section*{Appendix}
\addcontentsline{toc}{section}{Appendix}


\begin{table}[htbp]
\caption{Grooming Strategies as described in \cite{ring21}}
\label{app_tab:1}       % Give a unique label
%
% Follow this input for your own table layout
%
\begin{tabular}{p{0.3\textwidth}p{0.6\textwidth}}
\hline\noalign{\smallskip}
Strategy  & Description \\
\noalign{\smallskip}\hline\noalign{\smallskip}
Coercion & Peer-pressure, Implicit and overt threats, Guilt \\
Bragging &  Highlighting positive features around oneself and offers to teach sexual acts \cite{kingma2014} \\
Discuss Images &  Discussion, questions, and responses by one participant to the other related to images (both sexual and non-sexual) \\
Negative Physical &  Comments about having some negative element of physical appearance or body \\
Negative Life &  Stories or comments about not having a good family life or not getting along with family members \\
Teaching & Comments in which one participant offers to help the other participant learn something. The entry can be sexual or non-sexual \\
Personal Compliments & Providing positive comments about matureness, body, etc. of other \\
Reverse Power & False security meant to imply that the child is an equal/has control in the situation  \\
Sexual History & Comments about previous sexual experiences and whether or not the other participant enjoys sexual acts they have engaged in in the past \\
Willingness & Assessment of whether or not someone would do or try something; Questions related to sex acts which could be performed in person \\
Roleplay & Pretending to have sexual intercourse while actively online \\
Secrecy & Requests or statements expressing a need for discretion \\
% \noalign{\smallskip}
\hline
% \noalign{\smallskip}
\end{tabular}
% $^a$ Table foot note (with superscript)
\end{table}


\end{document}
