\def\year{2025}\relax
%File: formatting-instructions-latex-2022.tex
%release 2022.1
\documentclass[letterpaper]{article} % DO NOT CHANGE THIS
\usepackage{aaai22}  % DO NOT CHANGE THIS
\usepackage{times}  % DO NOT CHANGE THIS
\usepackage{helvet}  % DO NOT CHANGE THIS
\usepackage{courier}  % DO NOT CHANGE THIS
\usepackage[hyphens]{url}  % DO NOT CHANGE THIS
\usepackage{graphicx} % DO NOT CHANGE THIS
\urlstyle{rm} % DO NOT CHANGE THIS
\def\UrlFont{\rm}  % DO NOT CHANGE THIS
\usepackage{natbib}  % DO NOT CHANGE THIS AND DO NOT ADD ANY OPTIONS TO IT
\usepackage{caption} % DO NOT CHANGE THIS AND DO NOT ADD ANY OPTIONS TO IT
\DeclareCaptionStyle{ruled}{labelfont=normalfont,labelsep=colon,strut=off} % DO NOT CHANGE THIS
\frenchspacing  % DO NOT CHANGE THIS
\setlength{\pdfpagewidth}{8.5in}  % DO NOT CHANGE THIS
\setlength{\pdfpageheight}{11in}  % DO NOT CHANGE THIS
%
% These are recommended to typeset algorithms but not required. See the subsubsection on algorithms. Remove them if you don't have algorithms in your paper.
\usepackage{algorithm}
\usepackage{algorithmic}

% NEW PACKAGES
\usepackage{xcolor}
\usepackage{booktabs}
\usepackage{amsmath}
\usepackage{amsfonts}
\usepackage{multirow}
\usepackage{array}
\usepackage{bm}
\usepackage{pbox}
\usepackage{makecell}

%
% These are are recommended to typeset listings but not required. See the subsubsection on listing. Remove this block if you don't have listings in your paper.
\usepackage{newfloat}
\usepackage{listings}


\usepackage{xcolor}
\newcommand{\answerYes}[1]{\textcolor{blue}{#1}} 
\newcommand{\answerNo}[1]{\textcolor{teal}{#1}} 
\newcommand{\answerNA}[1]{\textcolor{gray}{#1}} 
\newcommand{\answerTODO}[1]{\textcolor{red}{#1}} 

\lstset{%
	basicstyle={\footnotesize\ttfamily},% footnotesize acceptable for monospace
	numbers=left,numberstyle=\footnotesize,xleftmargin=2em,% show line numbers, remove this entire line if you don't want the numbers.
	aboveskip=0pt,belowskip=0pt,%
	showstringspaces=false,tabsize=2,breaklines=true}
\floatstyle{ruled}
\newfloat{listing}{tb}{lst}{}
\floatname{listing}{Listing}
%
%\nocopyright
%
% PDF Info Is REQUIRED.
% For /Title, write your title in Mixed Case.
% Don't use accents or commands. Retain the parentheses.
% For /Author, add all authors within the parentheses,
% separated by commas. No accents, special characters
% or commands are allowed.
% Keep the /TemplateVersion tag as is
\pdfinfo{
/Title (AAAI Press Formatting Instructions for Authors Using LaTeX -- A Guide)
/Author (AAAI Press Staff, Pater Patel Schneider, Sunil Issar, J. Scott Penberthy, George Ferguson, Hans Guesgen, Francisco Cruz, Marc Pujol-Gonzalez)
/TemplateVersion (2025.1)
}

% DISALLOWED PACKAGES
% \usepackage{authblk} -- This package is specifically forbidden
% \usepackage{balance} -- This package is specifically forbidden
% \usepackage{color (if used in text)
% \usepackage{CJK} -- This package is specifically forbidden
% \usepackage{float} -- This package is specifically forbidden
% \usepackage{flushend} -- This package is specifically forbidden
% \usepackage{fontenc} -- This package is specifically forbidden
% \usepackage{fullpage} -- This package is specifically forbidden
% \usepackage{geometry} -- This package is specifically forbidden
% \usepackage{grffile} -- This package is specifically forbidden
% \usepackage{hyperref} -- This package is specifically forbidden
% \usepackage{navigator} -- This package is specifically forbidden
% (or any other package that embeds links such as navigator or hyperref)
% \indentfirst} -- This package is specifically forbidden
% \layout} -- This package is specifically forbidden
% \multicol} -- This package is specifically forbidden
% \nameref} -- This package is specifically forbidden
% \usepackage{savetrees} -- This package is specifically forbidden
% \usepackage{setspace} -- This package is specifically forbidden
% \usepackage{stfloats} -- This package is specifically forbidden
% \usepackage{tabu} -- This package is specifically forbidden
% \usepackage{titlesec} -- This package is specifically forbidden
% \usepackage{tocbibind} -- This package is specifically forbidden
% \usepackage{ulem} -- This package is specifically forbidden
% \usepackage{wrapfig} -- This package is specifically forbidden
% DISALLOWED COMMANDS
% \nocopyright -- Your paper will not be published if you use this command
% \addtolength -- This command may not be used
% \balance -- This command may not be used
% \baselinestretch -- Your paper will not be published if you use this command
% \clearpage -- No page breaks of any kind may be used for the final version of your paper
% \columnsep -- This command may not be used
% \newpage -- No page breaks of any kind may be used for the final version of your paper
% \pagebreak -- No page breaks of any kind may be used for the final version of your paperr
% \pagestyle -- This command may not be used
% \tiny -- This is not an acceptable font size.
% \vspace{- -- No negative value may be used in proximity of a caption, figure, table, section, subsection, subsubsection, or reference
% \vskip{- -- No negative value may be used to alter spacing above or below a caption, figure, table, section, subsection, subsubsection, or reference

\setcounter{secnumdepth}{0} %May be changed to 1 or 2 if section numbers are desired.

% The file aaai22.sty is the style file for AAAI Press
% proceedings, working notes, and technical reports.
%

% Title

% Your title must be in mixed case, not sentence case.
% That means all verbs (including short verbs like be, is, using,and go),
% nouns, adverbs, adjectives should be capitalized, including both words in hyphenated terms, while
% articles, conjunctions, and prepositions are lower case unless they
% directly follow a colon or long dash

\title{Demystifying Hateful Content: Leveraging Large Multimodal Models for \\ Hateful Meme Detection with Explainable Decisions}
\author {
    % Authors
    Ming Shan Hee,
    Roy Ka-Wei Lee
}
\affiliations {
    % Affiliations
    Singapore University of Technology and Design \\
    mingshan\_hee@mymail.sutd.edu.sg, roy\_lee@sutd.edu.sg,
}

%Example, Single Author, ->> remove \iffalse,\fi and place them surrounding AAAI title to use it
\iffalse
\title{My Publication Title --- Single Author}
\author {
    Author Name
}
\affiliations{
    Affiliation\\
    Affiliation Line 2\\
    name@example.com
}
\fi

\iffalse
%Example, Multiple Authors, ->> remove \iffalse,\fi and place them surrounding AAAI title to use it
\title{My Publication Title --- Multiple Authors}
\author {
    % Authors
    First Author Name,\textsuperscript{\rm 1}
    Second Author Name, \textsuperscript{\rm 2}
    Third Author Name \textsuperscript{\rm 1}
}
\affiliations {
    % Affiliations
    \textsuperscript{\rm 1} Affiliation 1\\
    \textsuperscript{\rm 2} Affiliation 2\\
    firstAuthor@affiliation1.com, secondAuthor@affilation2.com, thirdAuthor@affiliation1.com
}
\fi


% REMOVE THIS: bibentry
% This is only needed to show inline citations in the guidelines document. You should not need it and can safely delete it.
\usepackage{bibentry}
% END REMOVE bibentry

\begin{document}

\maketitle

\begin{abstract}
Hateful meme detection presents a significant challenge as a multimodal task due to the complexity of interpreting implicit hate messages and contextual cues within memes. Previous approaches have fine-tuned pre-trained vision-language models (PT-VLMs), leveraging the knowledge they gained during pre-training and their attention mechanisms to understand meme content. However, the reliance of these models on implicit knowledge and complex attention mechanisms renders their decisions difficult to explain, which is crucial for building trust in meme classification. In this paper, we introduce \textsf{IntMeme}, a novel framework that leverages Large Multimodal Models (LMMs) for hateful meme classification with explainable decisions. \textsf{IntMeme} addresses the dual challenges of improving both accuracy and explainability in meme moderation. The framework uses LMMs to generate human-like, interpretive analyses of memes, providing deeper insights into multimodal content and context. Additionally, it uses independent encoding modules for both memes and their interpretations, which are then combined to enhance classification performance. Our approach addresses the opacity and misclassification issues associated with PT-VLMs, optimizing the use of LMMs for hateful meme detection. We demonstrate the effectiveness of \textsf{IntMeme} through comprehensive experiments across three datasets, showcasing its superiority over state-of-the-art models.
\end{abstract}

% {\color{red} \textbf{WARNING}: \textit{
% This paper contains violence and discriminatory content that may be disturbing to some readers.}} 

\section{Introduction}
\section{Introduction}

Large language models (LLMs) have achieved remarkable success in automated math problem solving, particularly through code-generation capabilities integrated with proof assistants~\citep{lean,isabelle,POT,autoformalization,MATH}. Although LLMs excel at generating solution steps and correct answers in algebra and calculus~\citep{math_solving}, their unimodal nature limits performance in plane geometry, where solution depends on both diagram and text~\citep{math_solving}. 

Specialized vision-language models (VLMs) have accordingly been developed for plane geometry problem solving (PGPS)~\citep{geoqa,unigeo,intergps,pgps,GOLD,LANS,geox}. Yet, it remains unclear whether these models genuinely leverage diagrams or rely almost exclusively on textual features. This ambiguity arises because existing PGPS datasets typically embed sufficient geometric details within problem statements, potentially making the vision encoder unnecessary~\citep{GOLD}. \cref{fig:pgps_examples} illustrates example questions from GeoQA and PGPS9K, where solutions can be derived without referencing the diagrams.

\begin{figure}
    \centering
    \begin{subfigure}[t]{.49\linewidth}
        \centering
        \includegraphics[width=\linewidth]{latex/figures/images/geoqa_example.pdf}
        \caption{GeoQA}
        \label{fig:geoqa_example}
    \end{subfigure}
    \begin{subfigure}[t]{.48\linewidth}
        \centering
        \includegraphics[width=\linewidth]{latex/figures/images/pgps_example.pdf}
        \caption{PGPS9K}
        \label{fig:pgps9k_example}
    \end{subfigure}
    \caption{
    Examples of diagram-caption pairs and their solution steps written in formal languages from GeoQA and PGPS9k datasets. In the problem description, the visual geometric premises and numerical variables are highlighted in green and red, respectively. A significant difference in the style of the diagram and formal language can be observable. %, along with the differences in formal languages supported by the corresponding datasets.
    \label{fig:pgps_examples}
    }
\end{figure}



We propose a new benchmark created via a synthetic data engine, which systematically evaluates the ability of VLM vision encoders to recognize geometric premises. Our empirical findings reveal that previously suggested self-supervised learning (SSL) approaches, e.g., vector quantized variataional auto-encoder (VQ-VAE)~\citep{unimath} and masked auto-encoder (MAE)~\citep{scagps,geox}, and widely adopted encoders, e.g., OpenCLIP~\citep{clip} and DinoV2~\citep{dinov2}, struggle to detect geometric features such as perpendicularity and degrees. 

To this end, we propose \geoclip{}, a model pre-trained on a large corpus of synthetic diagram–caption pairs. By varying diagram styles (e.g., color, font size, resolution, line width), \geoclip{} learns robust geometric representations and outperforms prior SSL-based methods on our benchmark. Building on \geoclip{}, we introduce a few-shot domain adaptation technique that efficiently transfers the recognition ability to real-world diagrams. We further combine this domain-adapted GeoCLIP with an LLM, forming a domain-agnostic VLM for solving PGPS tasks in MathVerse~\citep{mathverse}. 
%To accommodate diverse diagram styles and solution formats, we unify the solution program languages across multiple PGPS datasets, ensuring comprehensive evaluation. 

In our experiments on MathVerse~\citep{mathverse}, which encompasses diverse plane geometry tasks and diagram styles, our VLM with a domain-adapted \geoclip{} consistently outperforms both task-specific PGPS models and generalist VLMs. 
% In particular, it achieves higher accuracy on tasks requiring geometric-feature recognition, even when critical numerical measurements are moved from text to diagrams. 
Ablation studies confirm the effectiveness of our domain adaptation strategy, showing improvements in optical character recognition (OCR)-based tasks and robust diagram embeddings across different styles. 
% By unifying the solution program languages of existing datasets and incorporating OCR capability, we enable a single VLM, named \geovlm{}, to handle a broad class of plane geometry problems.

% Contributions
We summarize the contributions as follows:
We propose a novel benchmark for systematically assessing how well vision encoders recognize geometric premises in plane geometry diagrams~(\cref{sec:visual_feature}); We introduce \geoclip{}, a vision encoder capable of accurately detecting visual geometric premises~(\cref{sec:geoclip}), and a few-shot domain adaptation technique that efficiently transfers this capability across different diagram styles (\cref{sec:domain_adaptation});
We show that our VLM, incorporating domain-adapted GeoCLIP, surpasses existing specialized PGPS VLMs and generalist VLMs on the MathVerse benchmark~(\cref{sec:experiments}) and effectively interprets diverse diagram styles~(\cref{sec:abl}).

\iffalse
\begin{itemize}
    \item We propose a novel benchmark for systematically assessing how well vision encoders recognize geometric premises, e.g., perpendicularity and angle measures, in plane geometry diagrams.
	\item We introduce \geoclip{}, a vision encoder capable of accurately detecting visual geometric premises, and a few-shot domain adaptation technique that efficiently transfers this capability across different diagram styles.
	\item We show that our final VLM, incorporating GeoCLIP-DA, effectively interprets diverse diagram styles and achieves state-of-the-art performance on the MathVerse benchmark, surpassing existing specialized PGPS models and generalist VLM models.
\end{itemize}
\fi

\iffalse

Large language models (LLMs) have made significant strides in automated math word problem solving. In particular, their code-generation capabilities combined with proof assistants~\citep{lean,isabelle} help minimize computational errors~\citep{POT}, improve solution precision~\citep{autoformalization}, and offer rigorous feedback and evaluation~\citep{MATH}. Although LLMs excel in generating solution steps and correct answers for algebra and calculus~\citep{math_solving}, their uni-modal nature limits performance in domains like plane geometry, where both diagrams and text are vital.

Plane geometry problem solving (PGPS) tasks typically include diagrams and textual descriptions, requiring solvers to interpret premises from both sources. To facilitate automated solutions for these problems, several studies have introduced formal languages tailored for plane geometry to represent solution steps as a program with training datasets composed of diagrams, textual descriptions, and solution programs~\citep{geoqa,unigeo,intergps,pgps}. Building on these datasets, a number of PGPS specialized vision-language models (VLMs) have been developed so far~\citep{GOLD, LANS, geox}.

Most existing VLMs, however, fail to use diagrams when solving geometry problems. Well-known PGPS datasets such as GeoQA~\citep{geoqa}, UniGeo~\citep{unigeo}, and PGPS9K~\citep{pgps}, can be solved without accessing diagrams, as their problem descriptions often contain all geometric information. \cref{fig:pgps_examples} shows an example from GeoQA and PGPS9K datasets, where one can deduce the solution steps without knowing the diagrams. 
As a result, models trained on these datasets rely almost exclusively on textual information, leaving the vision encoder under-utilized~\citep{GOLD}. 
Consequently, the VLMs trained on these datasets cannot solve the plane geometry problem when necessary geometric properties or relations are excluded from the problem statement.

Some studies seek to enhance the recognition of geometric premises from a diagram by directly predicting the premises from the diagram~\citep{GOLD, intergps} or as an auxiliary task for vision encoders~\citep{geoqa,geoqa-plus}. However, these approaches remain highly domain-specific because the labels for training are difficult to obtain, thus limiting generalization across different domains. While self-supervised learning (SSL) methods that depend exclusively on geometric diagrams, e.g., vector quantized variational auto-encoder (VQ-VAE)~\citep{unimath} and masked auto-encoder (MAE)~\citep{scagps,geox}, have also been explored, the effectiveness of the SSL approaches on recognizing geometric features has not been thoroughly investigated.

We introduce a benchmark constructed with a synthetic data engine to evaluate the effectiveness of SSL approaches in recognizing geometric premises from diagrams. Our empirical results with the proposed benchmark show that the vision encoders trained with SSL methods fail to capture visual \geofeat{}s such as perpendicularity between two lines and angle measure.
Furthermore, we find that the pre-trained vision encoders often used in general-purpose VLMs, e.g., OpenCLIP~\citep{clip} and DinoV2~\citep{dinov2}, fail to recognize geometric premises from diagrams.

To improve the vision encoder for PGPS, we propose \geoclip{}, a model trained with a massive amount of diagram-caption pairs.
Since the amount of diagram-caption pairs in existing benchmarks is often limited, we develop a plane diagram generator that can randomly sample plane geometry problems with the help of existing proof assistant~\citep{alphageometry}.
To make \geoclip{} robust against different styles, we vary the visual properties of diagrams, such as color, font size, resolution, and line width.
We show that \geoclip{} performs better than the other SSL approaches and commonly used vision encoders on the newly proposed benchmark.

Another major challenge in PGPS is developing a domain-agnostic VLM capable of handling multiple PGPS benchmarks. As shown in \cref{fig:pgps_examples}, the main difficulties arise from variations in diagram styles. 
To address the issue, we propose a few-shot domain adaptation technique for \geoclip{} which transfers its visual \geofeat{} perception from the synthetic diagrams to the real-world diagrams efficiently. 

We study the efficacy of the domain adapted \geoclip{} on PGPS when equipped with the language model. To be specific, we compare the VLM with the previous PGPS models on MathVerse~\citep{mathverse}, which is designed to evaluate both the PGPS and visual \geofeat{} perception performance on various domains.
While previous PGPS models are inapplicable to certain types of MathVerse problems, we modify the prediction target and unify the solution program languages of the existing PGPS training data to make our VLM applicable to all types of MathVerse problems.
Results on MathVerse demonstrate that our VLM more effectively integrates diagrammatic information and remains robust under conditions of various diagram styles.

\begin{itemize}
    \item We propose a benchmark to measure the visual \geofeat{} recognition performance of different vision encoders.
    % \item \sh{We introduce geometric CLIP (\geoclip{} and train the VLM equipped with \geoclip{} to predict both solution steps and the numerical measurements of the problem.}
    \item We introduce \geoclip{}, a vision encoder which can accurately recognize visual \geofeat{}s and a few-shot domain adaptation technique which can transfer such ability to different domains efficiently. 
    % \item \sh{We develop our final PGPS model, \geovlm{}, by adapting \geoclip{} to different domains and training with unified languages of solution program data.}
    % We develop a domain-agnostic VLM, namely \geovlm{}, by applying a simple yet effective domain adaptation method to \geoclip{} and training on the refined training data.
    \item We demonstrate our VLM equipped with GeoCLIP-DA effectively interprets diverse diagram styles, achieving superior performance on MathVerse compared to the existing PGPS models.
\end{itemize}

\fi 


\section{Related Works}
\subsubsection{Conditioned Diffusion Models}

By operating the data in latent space instead of pixel space, conditioned diffusion models have gained promising development \cite{rombach2022latentDiff}. MM-Diffusion \cite{ruan2023mmdi} designed for joint audio and video generation took advantage of coupled denoising autoencoders to generate aligned audio-video pairs from Gaussian noise. Extending the scalability of diffusion models, diffusion Transformers treat all inputs, including time, conditions, and noisy image patches, as tokens, leveraging the Transformer architecture to process these inputs \cite{bao2023ViTDiff}. In DiT \cite{peebles2023DiT}, William et al. emphasized the potential for diffusion models to benefit from Transformer architectures, where conditions were tokenized along with image tokens to achieve in-context conditioning. 

\subsubsection{Diffusion Models in Robotics}

Recently, a probabilistic multimodal action representation was proposed by Cheng Chi et al. \cite{chi2023diffusionpolicy}, where the robot action generation is considered as a conditional diffusion denoising process. Leveraging the diffusion policy, Ze et al. \cite{ze20243d} conditioned the diffusion policy on compact 3D representations and robot poses to generate coherent action sequences. Furthermore, GR-MG combined a progress-guided goal image generation model with a multimodal goal-conditioned policy, enabling the robot to predict actions based on both text instructions and generated goal images \cite{li2025grmg}. BESO used score-based diffusion models to learn goal-conditioned policies from large, uncurated datasets without rewards. Score-based diffusion models progressively add noise to the data and then reverse this process to generate new samples, making them suitable for capturing the multimodal nature of play data \cite{reuss2023md}. RDT-1B employed a scalable Transformer backbone combined with diffusion models to capture the complexity and multimodality of bimanual actions, leveraging diffusion models as a foundation model to effectively represent the multimodality inherent in bimanual manipulation tasks \cite{liu2024rdt-1b}. NoMaD exploited the diffusion model to handle both goal-directed navigation and task-agnostic exploration in unfamiliar environments, using goal masking to condition the policy on an optional goal image, allowing the model to dynamically switch between exploratory and goal-oriented behaviors \cite{sridhar2023nomad}. The aforementioned insights grounded the significant advancements of diffusion models in robotic tasks.

\subsubsection{VLM-based Autonomous Driving}

End-to-end autonomous driving introduces policy learning from sensor data input, resulting in a data-driven motion planning paradigm \cite{chen2024vadv2}. As part of the development of VLMs, they have shown significant promise in unifying multimodal data for specific downstream tasks, notably improving end-to-end autonomous driving systems\cite{ma2024dolphins}. DriveMM can process single images, multiview images, single videos, and multiview videos, and perform tasks such as object detection, motion prediction, and decision making, handling multiple tasks and data types in autonomous driving \cite{huang2024drivemm}. HE-Drive aims to create a human-like driving experience by generating trajectories that are both temporally consistent and comfortable. It integrates a sparse perception module, a diffusion-based motion planner, and a trajectory scorer guided by a Vision Language Model to achieve this goal \cite{wang2024hedrive}. Based on current perspectives, a differentiable end-to-end autonomous driving paradigm that directly leverages the capabilities of VLM and a multimodal action representation should be developed. 









\section{Methodology}
\section{Methodology}
\subsection{Preliminary}
\label{sec:preliminary}
\mypara{Architecture of MLLM.}
% The MLLM architectures generally consist of three components: a visual encoder, a modality projector, and a LLM. The visual encoder, typically a pre-trained image encoder like CLIP's vision model, converts input images into visual tokens. The projector module aligns these visual tokens with the LLM's word embedding space, enabling the LLM to process visual data effectively. The LLM then integrates the aligned visual and textual information to generate responses.
The architecture of Multimodal Large Language Models (MLLMs) typically comprises three core components: a visual encoder, a modality projector, and a language model (LLM). Given an image $I$, the visual encoder and a subsequent learnable MLP are used to encode $I$ into a set of visual tokens $e_v$. These visual tokens $e_v$ are then concatenated with text tokens $e_t$ encoded from text prompt $p_t$, forming the input for the LLM. The LLM decodes the output tokens $y$ sequentially, which can be formulated as:
\begin{equation}
\label{eq1}
    y_i = f(I, p_t, y_0, y_1, \cdots, y_{i-1}).
\end{equation}

\mypara{Computational Complexity.}  
To evaluate the computational complexity of MLLMs, it is essential to analyze their core components, including the self-attention mechanism and the feed-forward network (FFN). The total floating-point operations (FLOPs) required can be expressed as:  
\begin{equation}
\text{Total FLOPs} = T \times (4nd^2 + 2n^2d + 2ndm),
\end{equation}  
where $T$ denotes the number of transformer layers, $n$ is the sequence length, $d$ represents the hidden dimension size, and $m$ is the intermediate size of the FFN.  
This equation highlights the significant impact of sequence length $n$ on computational complexity. In typical MLLM tasks, the sequence length is defined as: 
\begin{equation}
    n = n_S + n_I + n_Q, 
\end{equation}
where $n_I$, the tokenized image representation, often dominates, sometimes exceeding other components by an order of magnitude or more.  
As a result, minimizing $n_I$ becomes a critical strategy for enhancing the efficiency of MLLMs.

\subsection{Beyond Token Importance: Questioning the Status Quo}
Given the computational burden associated with the length of visual tokens in MLLMs, numerous studies have embraced a paradigm that utilizes attention scores to evaluate the significance of visual tokens, thereby facilitating token reduction.
Specifically, in transformer-based MLLMs, each layer performs attention computation as illustrated below:
\begin{equation}
   \text{Attention}(\mathbf{Q}, \mathbf{K}, \mathbf{V}) = \text{softmax}\left(\frac{\mathbf{Q} \cdot \mathbf{K}^\mathbf{T}}{\sqrt{d_k}}\right)\cdot \mathbf{V},
\end{equation}
where $d_k$ is the dimension of $\mathbf{K}$. The result of $\text{Softmax}(\mathbf{Q}\cdot \mathbf{K}^\mathbf{T}/\sqrt{d_k})$ is a square matrix known as the attention map.
Existing methods extract the corresponding attention maps from one or multiple layers and compute the average attention score for each visual token based on these attention maps:
\begin{equation}
    \phi_{\text{attn}}(x_i) = \frac{1}{N} \sum_{j=1}^{N} \text{Attention}(x_i, x_j),
\end{equation}
where $\text{Attention}(x_i, x_j)$ denotes the attention score between token $x_i$ and token $x_j$, $\phi_{\text{attn}}(x_i)$ is regarded as the importance score of the token $x_i$, $N$ represents the number of visual tokens.
Finally, based on the importance score of each token and the predefined reduction ratio, the most significant tokens are selectively retained:
\begin{equation}
    \mathcal{R} = \{ x_i \mid (\phi_{\text{attn}}(x_i) \geq \tau) \},
\end{equation}
where $\mathcal{R}$ represents the set of retained tokens, and $\tau$ is a threshold determined by the predefined reduction ratio.

\noindent{\textbf{Problems:}} Although this paradigm has demonstrated initial success in enhancing the efficiency of MLLMs, it is accompanied by several inherent limitations that are challenging to overcome.

First, when it comes to leveraging attention scores to derive token importance, it inherently lacks full compatibility with Flash Attention, resulting in limited hardware acceleration affinity and diminished acceleration benefits.

Second, does the paradigm of using attention scores to evaluate token importance truly ensure the effective retention of crucial visual tokens? Our empirical investigations reveal that it is not the optimal approach.

% As illustrated in Figure~\ref{fig:random_vs_others}, performance evaluations on certain benchmarks show that methods meticulously designed based on this paradigm sometimes underperform compared to randomly retaining the same number of visual tokens.
Performance evaluations on certain benchmarks, as illustrated in Figure~\ref{fig:random_vs_others}, demonstrate that methods meticulously designed based on this paradigm sometimes underperform compared to randomly retaining the same number of visual tokens.

% As depicted in Figure~\ref{fig:teaser_curry}, which visualizes the results of token reduction, the selection of visual tokens based on attention scores exhibits a noticeable bias, favoring tokens located in the lower-right region of the image—those positioned later in the visual token sequence. However, it is evident that the lower-right region is not always the most significant in every image.
% Furthermore, in Figure~\ref{fig:teaser_curry}, we present the outputs of the original LLaVA-1.5-7B, FastV, and our proposed \algname. Notably, FastV introduces more hallucinations compared to the vanilla model, while \algname demonstrates a noticeable trend of reducing hallucinations.
% We suppose that this phenomenon arises because the important-based method, which relies on attention scores, tends to retain visual tokens that are concentrated in specific regions of the image due to the inherent bias in attention scores. As a result, relying on only a portion of the image often leads to outputs that are inconsistent with the overall image content. In contrast, \algname primarily removes highly duplication tokens and retains tokens that are more evenly distributed across the entire image, enabling it to make more accurate and consistent judgments.
%--------------- shorter version ---------------------
Figure~\ref{fig:teaser_curry} visualizes the results of token reduction, revealing that selecting visual tokens based on attention scores introduces a noticeable bias toward tokens in the lower-right region of the image—those appearing later in the visual token sequence. However, this region is not always the most significant in every image. Additionally, we present the outputs of the original LLaVA-1.5-7B, FastV, and our proposed \algname. Notably, FastV generates more hallucinations compared to the vanilla model, while \algname effectively reduces them. 
We attribute this to the inherent bias of attention-based methods, which tend to retain tokens concentrated in specific regions, often neglecting the broader context of the image. In contrast, \algname removes highly duplication tokens and preserves a more balanced distribution across the image, enabling more accurate and consistent outputs.

\subsection{Token Duplication: Rethinking Reduction}
Given the numerous drawbacks associated with the paradigm of using attention scores to evaluate token importance for token reduction, \textit{what additional factors should we consider beyond token importance in the process of token reduction?}
Inspired by the intuitive ideas mentioned in \secref{sec:introduction} and the phenomenon of tokens in transformers tending toward uniformity (i.e., over-smoothing)~\citep{nguyen2023mitigating, gong2021vision}, we propose that token duplication should be a critical focus.

Due to the prohibitively high computational cost of directly measuring duplication among all tokens, we adopt a paradigm that involves selecting a minimal number of pivot tokens. 
\begin{equation}
    \mathcal{P} = \{p_1, p_2, \dots, p_k\}, \quad k \ll n,
\end{equation}
where $p_i$ denotes pivot token, $\mathcal{P}$ represents the set of pivot tokens and $n$ means the length of tokens.

Subsequently, we compute the cosine similarity between these pivot tokens and the remaining visual tokens:
\begin{equation}
    dup (p_i, x_j) = \frac{p_i \cdot x_j}{\|p_i\| \cdot \|x_j\|}, \quad p_i \in \mathcal{P}, \, x_j \in \mathcal{X},
\end{equation}
where $dup (p_i, x_j)$ represents the token duplication score between $i$-th pivot token $p_i$ and $j$-th visual token $x_j$,
ultimately retaining those tokens that exhibit the lowest duplication with the pivot tokens.
\begin{equation}
    \mathcal{R} = \{ x_j \mid \min_{p_i \in \mathcal{P}} dup (p_i, x_j) \leq \epsilon \}.
\end{equation}
Here, $\mathcal{R}$ denotes the set of retained tokens, and $\epsilon$ is a threshold determined by the reduction ratio.

Our method is orthogonal to the paradigm of using attention scores to measure token importance, meaning it is compatible with existing approaches. Specifically, we can leverage attention scores to select pivot tokens, and subsequently incorporate token duplication into the process.

However, this approach still does not fully achieve compatibility with Flash Attention. To this end, we explored alternative strategies for selecting pivot tokens, such as using K-norm, V-norm\footnote{Here, the K-norm and V-norm refer to the L1-norm of K matrix and V matrix in attention computing, respectively.}, or even random selection. Surprisingly, we found that all these methods achieve competitive performance across multiple benchmarks. This indicates that our token reduction paradigm based on token duplication is not highly sensitive to the choice of pivot tokens. Furthermore, it suggests that removing duplicate tokens may be more critical than identifying ``important tokens'', highlighting token duplication as a potentially more significant factor to consider in token reduction.
The selection of pivot tokens is discussed in greater detail in \secref{pivot_token_selection}.
% 加个总结


\section{Experiment Settings}
% \begin{table*}[t]
% \centering
%   \caption{Model comparison \textbf{without} any augmented image tags.}
% \label{tab:exp-results-wo}
%   \begin{tabular}{c|cc|cc|cc}
%     \hline
%     \textbf{Dataset} &\multicolumn{2}{c|}{\textbf{FHM}}&\multicolumn{2}{c|}{\textbf{MAMI}}&\multicolumn{2}{c}{\textbf{HarM}}\\
%     \textbf{Model} & \textbf{AUC.} & \textbf{Acc.}& \textbf{AUC.} & \textbf{Acc.} & \textbf{AUC.} & \textbf{Acc.}\\
%     \hline\hline
%     Text BERT & 66.10$_{\pm0.55}$& 57.12$_{\pm0.49}$   & 74.48$_{\pm0.60}$ & 67.37$_{\pm0.57}$  & 81.39$_{\pm0.91}$& 75.68$_{\pm1.59}$\\
%     Image-Region & 56.69$_{\pm1.05}$ &52.34$_{\pm1.39}$  & 70.20$_{\pm0.63}$ & 64.18$_{\pm0.81}$ &76.46$_{\pm0.47}$ &73.05$_{\pm1.80}$ \\
%     \hline\hline
%     VisualBERT COCO & 68.71$_{\pm1.02}$& 61.48$_{\pm1.19}$  &78.71$_{\pm0.59}$ &71.06$_{\pm0.94}$  &80.46$_{\pm1.04}$ &75.31$_{\pm1.44}$ \\
%     ViLBERT CC& 73.05$_{\pm0.62}$&64.70$_{\pm1.12}$  &77.71$_{\pm1.20}$ &69.48$_{\pm1.00}$  &84.11$_{\pm0.88}$ &78.70$_{\pm1.17}$  \\
%     MMBT-Region  & 72.86$_{\pm0.64}$&65.06$_{\pm1.76}$  & 79.17$_{\pm0.91}$& 70.46$_{\pm0.76}$ & 85.48$_{\pm0.75}$& 79.83$_{\pm2.00}$ \\
%     \hline
%     CLIP-BERT  & 66.97$_{\pm0.34}$&58.28$_{\pm0.63}$  & 77.66$_{\pm0.64}$& 68.44$_{\pm1.07}$ &82.63$_{\pm3.83}$ &80.48$_{\pm1.95}$  \\
%     DisMultiHate & 69.11$_{\pm0.84}$& 62.42$_{\pm0.72}$ &78.21$_{\pm0.61}$ & 70.58$_{\pm1.13}$ & 83.69$_{\pm1.33}$& 78.05$_{\pm0.73}$ \\
%     PromptHate & 76.76$_{\pm0.95}$&67.82$_{\pm1.23}$  &76.21$_{\pm1.05}$ &68.08$_{\pm0.58}$  &87.51$_{\pm0.74}$ & 79.38$_{\pm1.72}$ \\
%     \hline
%     FLAVA & ~77.40 & ~69.0 & & & & \\
%     \hline\hline
%     MIME$_{FLAVA}$ & ~77.40 & ~69.0 & & & & \\
%     \hline
% \end{tabular}
% \end{table*}


\begin{table}[t]
\centering
  % \small
  \begin{tabular}{c|cc|cc}
    \hline
     & \multicolumn{2}{c|}{\textbf{Train}} & \multicolumn{2}{c}{\textbf{Test}}\\
    Dataset & \# H & \# Non-H & \# H & \# Non-H\\
    \hline\hline
    FHM-FG & 3,007 & 5,493 & 246 & 254 \\
    HarMeme & 1,064 & 1,949 & 124 & 230\\
    MAMI & 5,004 & 4,996 & 500 & 500 \\
    \hline
\end{tabular}
\caption{Statistical distributions of datasets, where "H" represents harmful and "Non-H" represents non-harmful }
  \label{tab:dataset}
\end{table} 

\subsection{Evaluation Datasets} 
We evaluated \textsf{IntMeme} against the state-of-the-art PT-VLMs across three widely-used hateful meme datasets, showcasing its robustness and generalizability. 
\textit{Facebook’s Fine-Grained Hateful Memes} (\textbf{FHM-FG}) dataset \cite{mathias2021fhmfg} is a synthetic memes dataset containing hateful memes with five distinct types of incitement to hatred: gender, racial, religious, nationality and disability-based. \textit{Multimedia Automatic Misogyny Identification} (\textbf{MAMI}) dataset \cite{fersini2022mami} consists of misogynous memes collected from popular social media platforms and websites dedicated to meme creation. Evaluating our models on this dataset provides insight into the performance of hateful meme detection models in a natural environment. \textit{Harmful Meme} (\textbf{HarMeme}) dataset \cite{pramanick2021harmemes} consists of crowdsourced memes primarily collected from Google Image Search and publicly available groups on popular social media websites. These memes contains \textit{harmless}, \textit{partially harmful}, and \textit{very harmful} memes related to the COVID-19 topic. Following \citeauthor{pramanick2021harmemes}, we merge \textit{partially harmful}, and \textit{very harmful} into a single \textit{harmful} category.
A summary of the distribution of the three datasets is presented in Table \ref{tab:dataset}.

\subsection{Models}
% We conducted an evaluation of \textsf{IntMeme} against six state-of-the-art PT-VLMs. The \textbf{VisualBERT} \cite{li2019visualbert} model uses a single-stream transformer-based approach that concurrently processes textual and visual inputs using a single Transformer module. In contrast, the \textbf{ViLBERT} \cite{lu2019vilbert} uses a dual-stream transformer-based approach that independently processes textual and visual inputs before utilizing Transformer modules to capture inter-modality interactions. More recently, the \textit{Bootstrapping Language-Image Pre-training} (\textbf{BLIP}) \cite{li2022blip} model is pre-trained on a mixture of multimodal encoder-decoder models using a dataset bootstrapped from large-scale noisy image-text pairs. The \textit{Foundational Language and Vision Alignment} (\textbf{FLAVA}) \cite{singh2022flava} model is pre-trained on multimodal and unimodal data with unpaired images and text. Moving into models designed for hateful memes detection, the \textbf{MOMENTA} \cite{pramanick2021momenta} model utilizes both local and global multimodal fusion mechanisms to exploit interactions for detecting harmful memes. The \textbf{DisMultiHate} \cite{lee2021disentangling} model adopts a disentanglement approach to separate target information from memes, crucial for identifying hateful content. Our implementation in this paper uses the VisualBERT and ViLBERT model pre-trained on the MS-COCO dataset \cite{lin2014microsoft} and Conceptual Captions \cite{sharma2018conceptual} respectively.
We evaluated \textsf{IntMeme} against seven state-of-the-art models. The \textbf{VisualBERT} \cite{li2019visualbert} model uses a single-stream transformer-based approach that concurrently processes textual and visual inputs using a single Transformer module. In contrast, the \textbf{ViLBERT} \cite{lu2019vilbert} uses a dual-stream transformer-based approach that independently processes textual and visual inputs before using Transformer modules to capture inter-modality interactions. More recently, the \textbf{BLIP} \cite{li2022blip} model is pre-trained on a mixture of multimodal encoder-decoder models using a dataset bootstrapped from large-scale noisy image-text pairs. The \textbf{FLAVA} \cite{singh2022flava} model is pre-trained on multimodal and unimodal data with unpaired images and text. Moving into models designed for hateful memes detection, the \textbf{MOMENTA} \cite{pramanick2021momenta} model utilizes both local and global multimodal fusion mechanisms to exploit interactions for detecting harmful memes. The \textbf{DisMultiHate} \cite{lee2021disentangling} model adopts a disentanglement approach to separate target information from memes, crucial for identifying hateful content. Lastly, the \textbf{PromptHate} \cite{cao2022prompting} model uses a prompt-based approach with few-shot demonstrations to classify memes.


\subsection{Evaluation Metrics}
% As hateful meme classification is primarily a binary classification task, we employed two widely adopted metrics to evaluate the performance of the various models: Accuracy (Acc.) and Area Under the Receiver Operating Characteristics curve (AUROC). All the experimental results are aggregated across five random seeds, with the average results and standard deviation reported. All the models use the same set of random seeds to ensure a fair comparison.

We employed two widely adopted metrics to evaluate the performance of the various models: Accuracy (Acc.) and Area Under the Receiver Operating Characteristics curve (AUROC). All the experimental results are aggregated across five random seeds, with the average results and standard deviation reported. All the models use the same set of random seeds to ensure a fair comparison.


\subsection{Implementation Details}
\paragraph{Large Multimodal Models.} 
% We use two recently introduced LMMs known for their strong generation capabilities: mPLUG-Owl \cite{ye2023mplug} and InstructBLIP \cite{dai2023instructblip}. These LMMs have demonstrated impressive overall visual perception and cognition abilities, as evidenced by their high rankings on the MME benchmark leaderboards \cite{fu2023mme}. Furthermore, the open-source code implementation of these LMMs allows for their unrestricted use in real-world applications. We first prompt the frozen pre-trained LMMs to generate the image captions, which is required for the interpretation prompt, before prompting the model to generate the meme interpretation. To facilitate reproducibility, we use greedy decoding. Additionally, to prevent lengthy and repetitive responses, we set no\_repeat\_ngram\_size = 2 and max\_new\_tokens = 256 for the additional decoding configuration.

We compare two open-source LMMs with robust multimodal reasoning capabilities: mPLUG-Owl \cite{ye2023mplug} and InstructBLIP \cite{dai2023instructblip}. These LMMs have shown impressive overall visual perception and cognition abilities, as evidenced by their high rankings on the MME benchmark leaderboards \cite{fu2023mme}. We prompt the pre-trained LMMs to generate the image captions before prompting them to generate the meme interpretation. For reproducibility, we use greedy decoding. Moreover, to minimize the occurrence of lengthy and repetitive responses, we configure the decoding settings to use no\_repeat\_ngram\_size = 2 and max\_new\_tokens = 256.

\paragraph{IntMeme Encoders.} 
The MIE module uses RoBERTa as its text encoder, while the VLA module employs FLAVA as the vision-language encoder. The RoBERTa model has shown proficiency across various language modelling tasks. The FLAVA model, trained on the hateful meme detection task during pre-training, is well-suited for modelling the complex inter- and intra-modality interactions within memes.

\paragraph{IntMeme Training.} 
We use a learning rate of 2e-5 and a batch size of 32 to fine-tune \textsf{IntMeme} on 1 A100 GPU over 30 epochs with early stopping (i.e., patience = 5)\footnote{The model typically converges within 10 epochs}. As for the selection of the models, we base our choices on the average of their Acc. and AUROC scores. We optimized these models using Adam optimizer \cite{kingma2015adam} and are implemented in PyTorch using the Huggingface's \texttt{Transformers}\footnote{https://huggingface.co/docs/transformers} library.



\begin{table*}[t!]
    % \small
    \centering
    \begin{tabular}{ccccccc}
        \toprule
         &\multicolumn{2}{c}{\textbf{FHM}}&\multicolumn{2}{c}{\textbf{MAMI}}&\multicolumn{2}{c}{\textbf{HarMeme}}\\
         \cmidrule(lr){2-3} \cmidrule(lr){4-5} \cmidrule(lr){6-7}
        \textbf{Model} & \textbf{AUROC} & \textbf{Acc.}& \textbf{AUROC} & \textbf{Acc.} & \textbf{AUROC} & \textbf{Acc.}\\
        \midrule
        VisualBERT & 68.71$_{\pm1.02}$& 61.48$_{\pm1.19}$  &78.71$_{\pm0.59}$ &71.06$_{\pm0.94}$  &80.46$_{\pm1.04}$ &75.31$_{\pm1.44}$ \\
        ViLBERT & 73.05$_{\pm0.62}$&64.70$_{\pm1.12}$  &77.71$_{\pm1.20}$ &69.48$_{\pm1.00}$  &84.11$_{\pm0.88}$ &78.70$_{\pm1.17}$  \\
        MOMENTA$^*$ & 69.17$_{\pm4.71}$ & 61.34$_{\pm4.89}$  &81.68$_{\pm2.80}$ &72.10$_{\pm2.90}$   & 86.32$_{\pm3.83}$& 80.48$_{\pm1.95}$\\
        DisMultiHate & 69.11$_{\pm0.84}$& 62.42$_{\pm0.72}$ &78.21$_{\pm0.61}$ & 70.58$_{\pm1.13}$ & 83.69$_{\pm1.33}$& 78.05$_{\pm0.73}$ \\
        PromptHate & 76.76$_{\pm0.95}$&67.82$_{\pm1.23}$  &76.21$_{\pm1.05}$ &68.08$_{\pm0.58}$  &87.51$_{\pm0.74}$ & 79.38$_{\pm1.72}$ \\
        BLIP & 76.80$_{\pm2.37}$ &69.20$_{\pm1.84}$  & 80.59$_{\pm0.87}$&71.84$_{\pm1.11}$  &87.09$_{\pm1.46}$ &81.81$_{\pm1.74}$  \\
        FLAVA & 78.51$_{\pm0.70}$ & 70.28$_{\pm1.03}$ & 80.69$_{\pm0.84}$ & 71.72$_{\pm0.36}$ & 88.34$_{\pm1.15}$ & 81.58$_{\pm1.40}$ \\
        \midrule
        IntMeme$_\text{InstructBLIP}$ & 81.05$_{\pm0.81}$ & \textbf{71.48$_{\pm1.71}$} & 81.59$_{\pm0.65}$ & \textbf{72.44$_{\pm0.88}$} & 88.00$_{\pm0.84}$ & \textbf{82.66$_{\pm1.33}$} \\
        IntMeme$_\text{mPLUG-Owl}$ & \textbf{81.50$_{\pm1.11}$} & 71.52$_{\pm1.49}$ & \textbf{81.89$_{\pm1.15}$} & 72.30$_{\pm1.79}$ & \textbf{89.35}$_{\pm1.22}$ & 81.92$_{\pm2.47}$ \\
        \bottomrule
    \end{tabular}
    \caption{Evaluation results of hateful meme detection models on three benchmark datasets \textbf{without} any augmented image tags. These results have been aggregated over 5 random seeds and are reported along with their corresponding standard deviations.}
    \label{tab:experimental-results}
\end{table*}


% \begin{table*}[t!]
%     \small
%     \centering
%     \begin{tabular}{ccccccc}
%         \toprule
%          &\multicolumn{2}{c}{\textbf{FHM}}&\multicolumn{2}{c}{\textbf{MAMI}}&\multicolumn{2}{c}{\textbf{HarMeme}}\\
%          \cmidrule(lr){2-3} \cmidrule(lr){4-5} \cmidrule(lr){6-7}
%         \textbf{Model} & \textbf{AUROC} & \textbf{Acc.}& \textbf{AUROC} & \textbf{Acc.} & \textbf{AUROC} & \textbf{Acc.}\\
%         \midrule
%         VisualBERT \shortcite{li2019visualbert} & 68.71$_{\pm1.02}$& 61.48$_{\pm1.19}$  &78.71$_{\pm0.59}$ &71.06$_{\pm0.94}$  &80.46$_{\pm1.04}$ &75.31$_{\pm1.44}$ \\
%         ViLBERT \shortcite{lu2019vilbert} & 73.05$_{\pm0.62}$&64.70$_{\pm1.12}$  &77.71$_{\pm1.20}$ &69.48$_{\pm1.00}$  &84.11$_{\pm0.88}$ &78.70$_{\pm1.17}$  \\
%         MOMENTA \shortcite{pramanick2021momenta}$^*$ & 69.17$_{\pm4.71}$ & 61.34$_{\pm4.89}$  &81.68$_{\pm2.80}$ &72.10$_{\pm2.90}$   & 86.32$_{\pm3.83}$& 80.48$_{\pm1.95}$\\
%         DisMultiHate \shortcite{lee2021disentangling} & 69.11$_{\pm0.84}$& 62.42$_{\pm0.72}$ &78.21$_{\pm0.61}$ & 70.58$_{\pm1.13}$ & 83.69$_{\pm1.33}$& 78.05$_{\pm0.73}$ \\
%         PromptHate \shortcite{cao2022prompting} & 76.76$_{\pm0.95}$&67.82$_{\pm1.23}$  &76.21$_{\pm1.05}$ &68.08$_{\pm0.58}$  &87.51$_{\pm0.74}$ & 79.38$_{\pm1.72}$ \\
%         BLIP \shortcite{li2022blip} & 76.80$_{\pm2.37}$ &69.20$_{\pm1.84}$  & 80.59$_{\pm0.87}$&71.84$_{\pm1.11}$  &87.09$_{\pm1.46}$ &81.81$_{\pm1.74}$  \\
%         FLAVA \shortcite{singh2022flava} & 78.51$_{\pm0.70}$ & 70.28$_{\pm1.03}$ & 80.69$_{\pm0.84}$ & 71.72$_{\pm0.36}$ & 88.34$_{\pm1.15}$ & 81.58$_{\pm1.40}$ \\
%         \midrule
%         IntMeme$_\text{InstructBLIP}$ & 81.05$_{\pm0.81}$ & \textbf{71.48$_{\pm1.71}$} & 81.59$_{\pm0.65}$ & \textbf{72.44$_{\pm0.88}$} & 88.00$_{\pm0.84}$ & \textbf{82.66$_{\pm1.33}$} \\
%         IntMeme$_\text{mPLUG-Owl}$ & \textbf{81.50$_{\pm1.11}$} & 71.52$_{\pm1.49}$ & \textbf{81.89$_{\pm1.15}$} & 72.30$_{\pm1.79}$ & \textbf{89.35}$_{\pm1.22}$ & 81.92$_{\pm2.47}$ \\
%         \bottomrule
%     \end{tabular}
%     \caption{Evaluation results of state-of-the-art vision-language models on three benchmark datasets. These results have been aggregated over 5 random seeds and are reported along with their corresponding standard deviations. \textsuperscript{*}denotes the models that use supplementary information (i.e. image caption, object proposals and etc).}
%     \label{tab:experimental-results}
% \end{table*}


\begin{table*}[t]
  % \small
  \centering
  \begin{tabular}{lcccccc}
    \toprule
     &\multicolumn{2}{c}{\textbf{FHM}}&\multicolumn{2}{c}{\textbf{MAMI}}&\multicolumn{2}{c}{\textbf{HarMeme}} \\
     \cmidrule(lr){2-3} \cmidrule(lr){4-5} \cmidrule(lr){6-7}
    \textbf{Model} &\textbf{AUC.}&\textbf{Acc.} &\textbf{AUC.}&\textbf{Acc.} &\textbf{AUC.}&\textbf{Acc.} \\
    \midrule
    IntMeme$_\text{InstructBLIP}$ & & \\
    $-$  w/ \textsc{INTPN (MIE Module)} & 75.49$_{\pm1.46}$ & 68.64$_{\pm1.56}$ & 75.22$_{\pm1.56}$ & 66.50$_{\pm2.26}$ & 83.04$_{\pm1.96}$ & 77.12$_{\pm2.14}$  \\
    $-$  w/ \textsc{Meme (VLA Module)} & 78.51$_{\pm0.70}$ & 70.28$_{\pm1.03}$ & 80.69$_{\pm0.84}$ & 71.72$_{\pm0.36}$ & \textbf{88.34$_{\pm1.15}$} & 81.58$_{\pm1.40}$ \\
    $-$  w/ \textsc{Both (MIE + VLA Module)} & \textbf{81.05$_{\pm0.81}$} & \textbf{71.48$_{\pm1.71}$} & \textbf{81.59$_{\pm0.65}$} & \textbf{72.44$_{\pm0.88}$} & 88.00$_{\pm0.84}$ & \textbf{82.66$_{\pm1.33}$} \\
    \midrule
    IntMeme$_\text{mPLUG-Owl}$ & & \\
    $-$  w/ \textsc{INTPN (MIE Module)} & 77.26$_{\pm0.66}$ & 68.24$_{\pm2.42}$ & 77.61$_{\pm0.91}$ & 70.18$_{\pm0.72}$ & 88.74$_{\pm1.77}$ & 78.81$_{\pm2.32}$  \\
    $-$  w/ \textsc{Meme (VLA Module)} & 78.51$_{\pm0.70}$ & 70.28$_{\pm1.03}$ & 80.69$_{\pm0.84}$ & 71.72$_{\pm0.36}$ & 88.34$_{\pm1.15}$ & 81.58$_{\pm1.40}$ \\
    $-$  w/ \textsc{Both (MIE + VLA Module)} & \textbf{81.50$_{\pm1.11}$} & \textbf{71.52$_{\pm1.49}$} & \textbf{81.89$_{\pm1.15}$} & \textbf{72.30$_{\pm1.79}$} & \textbf{89.35$_{\pm1.22}$} & \textbf{81.92$_{\pm2.47}$} \\
    \midrule
    FLAVA \\
    $-$ \textsc{ Vanilla} & 78.51$_{\pm0.70}$ & 70.28$_{\pm1.03}$ & 80.69$_{\pm0.84}$ & 71.72$_{\pm0.36}$ & 88.34$_{\pm1.15}$ & 81.58$_{\pm1.40}$ \\
    $-$ w/ \textsc{INTPN}$_\text{InstructBLIP}$ \textsc{(CONCAT)} & 78.98$_{\pm0.79}$ & \textbf{70.52}$_{\pm0.87}$ & \textbf{81.23}$_{\pm1.28}$ & \textbf{71.22}$_{\pm2.59}$ & 88.63$_{\pm0.78}$ & 80.73$_{\pm2.79}$ \\
    $-$ w/ \textsc{INTPN}$_\text{mPLUG-Owl}$ \textsc{(CONCAT)} & \textbf{79.45}$_{\pm0.85}$ & 70.44$_{\pm1.58}$ & 81.20$_{\pm1.03}$  & 70.84$_{\pm2.22}$ & \textbf{89.10}$_{\pm1.16}$ & \textbf{81.53}$_{\pm2.32}$  \\
    \bottomrule
\end{tabular}
\caption{Ablation study w.r.t \textsf{IntMeme} and its distinct modules. The top scores across the variations are highlighted in \textbf{bold}.}
\label{tab:ablation-modules}
\end{table*}


\section{Experiments}
\section{Experimental Analysis}
\label{sec:exp}
We now describe in detail our experimental analysis. The experimental section is organized as follows:
%\begin{enumerate}[noitemsep,topsep=0pt,parsep=0pt,partopsep=0pt,leftmargin=0.5cm]
%\item 

\noindent In {\bf 
Section~\ref{exp:setup}}, we introduce the datasets and methods to evaluate the previously defined accuracy measures.

%\item
\noindent In {\bf 
Section~\ref{exp:qual}}, we illustrate the limitations of existing measures with some selected qualitative examples.

%\item 
\noindent In {\bf 
Section~\ref{exp:quant}}, we continue by measuring quantitatively the benefits of our proposed measures in terms of {\it robustness} to lag, noise, and normal/abnormal ratio.

%\item 
\noindent In {\bf 
Section~\ref{exp:separability}}, we evaluate the {\it separability} degree of accurate and inaccurate methods, using the existing and our proposed approaches.

%\item
\noindent In {\bf 
Section~\ref{sec:entropy}}, we conduct a {\it consistency} evaluation, in which we analyze the variation of ranks that an AD method can have with an accuracy measures used.

%\item 
\noindent In {\bf 
Section~\ref{sec:exectime}}, we conduct an {\it execution time} evaluation, in which we analyze the impact of different parameters related to the accuracy measures and the time series characteristics. 
We focus especially on the comparison of the different VUS implementations.
%\end{enumerate}

\begin{table}[tb]
\caption{Summary characteristics (averaged per dataset) of the public datasets of TSB-UAD (S.: Size, Ano.: Anomalies, Ab.: Abnormal, Den.: Density)}
\label{table:charac}
%\vspace{-0.2cm}
\footnotesize
\begin{center}
\scalebox{0.82}{
\begin{tabular}{ |r|r|r|r|r|r|} 
 \hline
\textbf{\begin{tabular}[c]{@{}c@{}}Dataset \end{tabular}} & 
\textbf{\begin{tabular}[c]{@{}c@{}}S. \end{tabular}} & 
\textbf{\begin{tabular}[c]{c@{}} Len.\end{tabular}} & 
\textbf{\begin{tabular}[c]{c@{}} \# \\ Ano. \end{tabular}} &
\textbf{\begin{tabular}[c]{c@{}c@{}} \# \\ Ab. \\ Points\end{tabular}} &
\textbf{\begin{tabular}[c]{c@{}c@{}} Ab. \\ Den. \\ (\%)\end{tabular}} \\ \hline
Dodgers \cite{10.1145/1150402.1150428} & 1 & 50400   & 133.0     & 5612.0  &11.14 \\ \hline
SED \cite{doi:10.1177/1475921710395811}& 1 & 100000   & 75.0     & 3750.0  & 3.7\\ \hline
ECG \cite{goldberger_physiobank_2000}   & 52 & 230351  & 195.6     & 15634.0  &6.8 \\ \hline
IOPS \cite{IOPS}   & 58 & 102119  & 46.5     & 2312.3   &2.1 \\ \hline
KDD21 \cite{kdd} & 250 &77415   & 1      & 196.5   &0.56 \\ \hline
MGAB \cite{markus_thill_2020_3762385}   & 10 & 100000  & 10.0     & 200.0   &0.20 \\ \hline
NAB \cite{ahmad_unsupervised_2017}   & 58 & 6301   & 2.0      & 575.5   &8.8 \\ \hline
NASA-M. \cite{10.1145/3449726.3459411}   & 27 & 2730   & 1.33      & 286.3   &11.97 \\ \hline
NASA-S. \cite{10.1145/3449726.3459411}   & 54 & 8066   & 1.26      & 1032.4   &12.39 \\ \hline
SensorS. \cite{YAO20101059}   & 23 & 27038   & 11.2     & 6110.4   &22.5 \\ \hline
YAHOO \cite{yahoo}  & 367 & 1561   & 5.9      & 10.7   &0.70 \\ \hline 
\end{tabular}}
\end{center}
\end{table}











\subsection{Experimental Setup and Settings}
\label{exp:setup}
%\vspace{-0.1cm}

\begin{figure*}[tb]
  \centering
  \includegraphics[width=1\linewidth]{figures/quality.pdf}
  %\vspace{-0.7cm}
  \caption{Comparison of evaluation measures (proposed measures illustrated in subplots (b,c,d,e); all others summarized in subplots (f)) on two examples ((A)AE and OCSM applied on MBA(805) and (B) LOF and OCSVM applied on MBA(806)), illustrating the limitations of existing measures for scores with noise or containing a lag. }
  \label{fig:quality}
  %\vspace{-0.1cm}
\end{figure*}

We implemented the experimental scripts in Python 3.8 with the following main dependencies: sklearn 0.23.0, tensorflow 2.3.0, pandas 1.2.5, and networkx 2.6.3. In addition, we used implementations from our TSB-UAD benchmark suite.\footnote{\scriptsize \url{https://www.timeseries.org/TSB-UAD}} For reproducibility purposes, we make our datasets and code available.\footnote{\scriptsize \url{https://www.timeseries.org/VUS}}
\newline \textbf{Datasets: } For our evaluation purposes, we use the public datasets identified in our TSB-UAD benchmark. The latter corresponds to $10$ datasets proposed in the past decades in the literature containing $900$ time series with labeled anomalies. Specifically, each point in every time series is labeled as normal or abnormal. Table~\ref{table:charac} summarizes relevant characteristics of the datasets, including their size, length, and statistics about the anomalies. In more detail:

\begin{itemize}
    \item {\bf SED}~\cite{doi:10.1177/1475921710395811}, from the NASA Rotary Dynamics Laboratory, records disk revolutions measured over several runs (3K rpm speed).
	\item {\bf ECG}~\cite{goldberger_physiobank_2000} is a standard electrocardiogram dataset and the anomalies represent ventricular premature contractions. MBA(14046) is split to $47$ series.
	\item {\bf IOPS}~\cite{IOPS} is a dataset with performance indicators that reflect the scale, quality of web services, and health status of a machine.
	\item {\bf KDD21}~\cite{kdd} is a composite dataset released in a SIGKDD 2021 competition with 250 time series.
	\item {\bf MGAB}~\cite{markus_thill_2020_3762385} is composed of Mackey-Glass time series with non-trivial anomalies. Mackey-Glass data series exhibit chaotic behavior that is difficult for the human eye to distinguish.
	\item {\bf NAB}~\cite{ahmad_unsupervised_2017} is composed of labeled real-world and artificial time series including AWS server metrics, online advertisement clicking rates, real time traffic data, and a collection of Twitter mentions of large publicly-traded companies.
	\item {\bf NASA-SMAP} and {\bf NASA-MSL}~\cite{10.1145/3449726.3459411} are two real spacecraft telemetry data with anomalies from Soil Moisture Active Passive (SMAP) satellite and Curiosity Rover on Mars (MSL).
	\item {\bf SensorScope}~\cite{YAO20101059} is a collection of environmental data, such as temperature, humidity, and solar radiation, collected from a sensor measurement system.
	\item {\bf Yahoo}~\cite{yahoo} is a dataset consisting of real and synthetic time series based on the real production traffic to some of the Yahoo production systems.
\end{itemize}


\textbf{Anomaly Detection Methods: }  For the experimental evaluation, we consider the following baselines. 

\begin{itemize}
\item {\bf Isolation Forest (IForest)}~\cite{liu_isolation_2008} constructs binary trees based on random space splitting. The nodes (subsequences in our specific case) with shorter path lengths to the root (averaged over every random tree) are more likely to be anomalies. 
\item {\bf The Local Outlier Factor (LOF)}~\cite{breunig_lof_2000} computes the ratio of the neighbor density to the local density. 
\item {\bf Matrix Profile (MP)}~\cite{yeh_time_2018} detects as anomaly the subsequence with the most significant 1-NN distance. 
\item {\bf NormA}~\cite{boniol_unsupervised_2021} identifies the normal patterns based on clustering and calculates each point's distance to normal patterns weighted using statistical criteria. 
\item {\bf Principal Component Analysis (PCA)}~\cite{aggarwal_outlier_2017} projects data to a lower-dimensional hyperplane. Outliers are points with a large distance from this plane. 
\item {\bf Autoencoder (AE)} \cite{10.1145/2689746.2689747} projects data to a lower-dimensional space and reconstructs it. Outliers are expected to have larger reconstruction errors. 
\item {\bf LSTM-AD}~\cite{malhotra_long_2015} use an LSTM network that predicts future values from the current subsequence. The prediction error is used to identify anomalies.
\item {\bf Polynomial Approximation (POLY)} \cite{li_unifying_2007} fits a polynomial model that tries to predict the values of the data series from the previous subsequences. Outliers are detected with the prediction error. 
\item {\bf CNN} \cite{8581424} built, using a convolutional deep neural network, a correlation between current and previous subsequences, and outliers are detected by the deviation between the prediction and the actual value. 
\item {\bf One-class Support Vector Machines (OCSVM)} \cite{scholkopf_support_1999} is a support vector method that fits a training dataset and finds the normal data's boundary.
\end{itemize}

\subsection{Qualitative Analysis}
\label{exp:qual}



We first use two examples to demonstrate qualitatively the limitations of existing accuracy evaluation measures in the presence of lag and noise, and to motivate the need for a new approach. 
These two examples are depicted in Figure~\ref{fig:quality}. 
The first example, in Figure~\ref{fig:quality}(A), corresponds to OCSVM and AE on the MBA(805) dataset (named MBA\_ECG805\_data.out in the ECG dataset). 

We observe in Figure~\ref{fig:quality}(A)(a.1) and (a.2) that both scores identify most of the anomalies (highlighted in red). However, the OCSVM score points to more false positives (at the end of the time series) and only captures small sections of the anomalies. On the contrary, the AE score points to fewer false positives and captures all abnormal subsequences. Thus we can conclude that, visually, AE should obtain a better accuracy score than OCSVM. Nevertheless, we also observe that the AE score is lagged with the labels and contains more noise. The latter has a significant impact on the accuracy of evaluation measures. First, Figure~\ref{fig:quality}(A)(c) is showing that AUC-PR is better for OCSM (0.73) than for AE (0.57). This is contradictory with what is visually observed from Figure~\ref{fig:quality}(A)(a.1) and (a.2). However, when using our proposed measure R-AUC-PR, OCSVM obtains a lower score (0.83) than AE (0.89). This confirms that, in this example, a buffer region before the labels helps to capture the true value of an anomaly score. Overall, Figure~\ref{fig:quality}(A)(f) is showing in green and red the evolution of accuracy score for the 13 accuracy measures for AE and OCSVM, respectively. The latter shows that, in addition to Precision@k and Precision, our proposed approach captures the quality order between the two methods well.

We now present a second example, on a different time series, illustrated in Figure~\ref{fig:quality}(B). 
In this case, we demonstrate the anomaly score of OCSVM and LOF (depicted in Figure~\ref{fig:quality}(B)(a.1) and (a.2)) applied on the MBA(806) dataset (named MBA\_ECG806\_data.out in the ECG dataset). 
We observe that both methods produce the same level of noise. However, LOF points to fewer false positives and captures more sections of the abnormal subsequences than OCSVM. 
Nevertheless, the LOF score is slightly lagged with the labels such that the maximum values in the LOF score are slightly outside of the labeled sections. 
Thus, as illustrated in Figure~\ref{fig:quality}(B)(f), even though we can visually consider that LOF is performing better than OCSM, all usual measures (Precision, Recall, F, precision@k, and AUC-PR) are judging OCSM better than AE. On the contrary, measures that consider lag (Rprecision, Rrecall, RF) rank the methods correctly. 
However, due to threshold issues, these measures are very close for the two methods. Overall, only AUC-ROC and our proposed measures give a higher score for LOF than for OCSVM.

\subsection{Quantitative Analysis}
\label{exp:case}

\begin{figure}[t]
  \centering
  \includegraphics[width=1\linewidth]{figures/eval_case_study.pdf}
  %\vspace*{-0.7cm}
  \caption{\commentRed{
  Comparison of evaluation measures for synthetic data examples across various scenarios. S8 represents the oracle case, where predictions perfectly align with labeled anomalies. Problematic cases are highlighted in the red region.}}
  %\vspace*{-0.5cm}
  \label{fig:eval_case_study}
\end{figure}
\commentRed{
We present the evaluation results for different synthetic data scenarios, as shown in Figure~\ref{fig:eval_case_study}. These scenarios range from S1, where predictions occur before the ground truth anomaly, to S12, where predictions fall within the ground truth region. The red-shaded regions highlight problematic cases caused by a lack of adaptability to lags. For instance, in scenarios S1 and S2, a slight shift in the prediction leads to measures (e.g., AUC-PR, F score) that fail to account for lags, resulting in a zero score for S1 and a significant discrepancy between the results of S1 and S2. Thus, we observe that our proposed VUS effectively addresses these issues and provides robust evaluations results.}

%\subsection{Quantitative Analysis}
%\subsection{Sensitivity and Separability Analysis}
\subsection{Robustness Analysis}
\label{exp:quant}


\begin{figure}[tb]
  \centering
  \includegraphics[width=1\linewidth]{figures/lag_sensitivity_analysis.pdf}
  %\vspace*{-0.7cm}
  \caption{For each method, we compute the accuracy measures 10 times with random lag $\ell \in [-0.25*\ell,0.25*\ell]$ injected in the anomaly score. We center the accuracy average to 0.}
  %\vspace*{-0.5cm}
  \label{fig:lagsensitivity}
\end{figure}

We have illustrated with specific examples several of the limitations of current measures. 
We now evaluate quantitatively the robustness of the proposed measures when compared to the currently used measures. 
We first evaluate the robustness to noise, lag, and normal versus abnormal points ratio. We then measure their ability to separate accurate and inaccurate methods.
%\newline \textbf{Sensitivity Analysis: } 
We first analyze the robustness of different approaches quantitatively to different factors: (i) lag, (ii) noise, and (iii) normal/abnormal ratio. As already mentioned, these factors are realistic. For instance, lag can be either introduced by the anomaly detection methods (such as methods that produce a score per subsequences are only high at the beginning of abnormal subsequences) or by human labeling approximation. Furthermore, even though lag and noises are injected, an optimal evaluation metric should not vary significantly. Therefore, we aim to measure the variance of the evaluation measures when we vary the lag, noise, and normal/abnormal ratio. We proceed as follows:

\begin{enumerate}[noitemsep,topsep=0pt,parsep=0pt,partopsep=0pt,leftmargin=0.5cm]
\item For each anomaly detection method, we first compute the anomaly score on a given time series.
\item We then inject either lag $l$, noise $n$ or change the normal/abnormal ratio $r$. For 10 different values of $l \in [-0.25*\ell,0.25*\ell]$, $n \in [-0.05*(max(S_T)-min(S_T)),0.05*(max(S_T)-min(S_T))]$ and $r \in [0.01,0.2]$, we compute the 13 different measures.
\item For each evaluation measure, we compute the standard deviation of the ten different values. Figure~\ref{fig:lagsensitivity}(b) depicts the different lag values for six AD methods applied on a data series in the ECG dataset.
\item We compute the average standard deviation for the 13 different AD quality measures. For example, figure~\ref{fig:lagsensitivity}(a) depicts the average standard deviation for ten different lag values over the AD methods applied on the MBA(805) time series.
\item We compute the average standard deviation for the every time series in each dataset (as illustrated in Figure~\ref{fig:sensitivity_per_data}(b to j) for nine datasets of the benchmark.
\item We compute the average standard deviation for the every dataset (as illustrated in Figure~\ref{fig:sensitivity_per_data}(a.1) for lag, Figure~\ref{fig:sensitivity_per_data}(a.2) for noise and Figure~\ref{fig:sensitivity_per_data}(a.3) for normal/abnormal ratio).
\item We finally compute the Wilcoxon test~\cite{10.2307/3001968} and display the critical diagram over the average standard deviation for every time series (as illustrated in Figure~\ref{fig:sensitivity}(a.1) for lag, Figure~\ref{fig:sensitivity}(a.2) for noise and Figure~\ref{fig:sensitivity}(a.3) for normal/abnormal ratio).
\end{enumerate}

%height=8.5cm,

\begin{figure}[tb]
  \centering
  \includegraphics[width=\linewidth]{figures/sensitivity_per_data_long.pdf}
%  %\vspace*{-0.3cm}
  \caption{Robustness Analysis for nine datasets: we report, over the entire benchmark, the average standard deviation of the accuracy values of the measures, under varying (a.1) lag, (a.2) noise, and (a.3) normal/abnormal ratio. }
  \label{fig:sensitivity_per_data}
\end{figure}

\begin{figure*}[tb]
  \centering
  \includegraphics[width=\linewidth]{figures/sensitivity_analysis.pdf}
  %\vspace*{-0.7cm}
  \caption{Critical difference diagram computed using the signed-rank Wilkoxon test (with $\alpha=0.1$) for the robustness to (a.1) lag, (a.2) noise and (a.3) normal/abnormal ratio.}
  \label{fig:sensitivity}
\end{figure*}

The methods with the smallest standard deviation can be considered more robust to lag, noise, or normal/abnormal ratio from the above framework. 
First, as stated in the introduction, we observe that non-threshold-based measures (such as AUC-ROC and AUC-PR) are indeed robust to noise (see Figure~\ref{fig:sensitivity_per_data}(a.2)), but not to lag. Figure~\ref{fig:sensitivity}(a.1) demonstrates that our proposed measures VUS-ROC, VUS-PR, R-AUC-ROC, and R-AUC-PR are significantly more robust to lag. Similarly, Figure~\ref{fig:sensitivity}(a.2) confirms that our proposed measures are significantly more robust to noise. However, we observe that, among our proposed measures, only VUS-ROC and R-AUC-ROC are robust to the normal/abnormal ratio and not VUS-PR and R-AUC-PR. This is explained by the fact that Precision-based measures vary significantly when this ratio changes. This is confirmed by Figure~\ref{fig:sensitivity_per_data}(a.3), in which we observe that Precision and Rprecision have a high standard deviation. Overall, we observe that VUS-ROC is significantly more robust to lag, noise, and normal/abnormal ratio than other measures.




\subsection{Separability Analysis}
\label{exp:separability}

%\newline \textbf{Separability Analysis: } 
We now evaluate the separability capacities of the different evaluation metrics. 
\commentRed{The main objective is to measure the ability of accuracy measures to separate accurate methods from inaccurate ones. More precisely, an appropriate measure should return accuracy scores that are significantly higher for accurate anomaly scores than for inaccurate ones.}
We thus manually select accurate and inaccurate anomaly detection methods and verify if the accuracy evaluation scores are indeed higher for the accurate than for the inaccurate methods. Figure~\ref{fig:separability} depicts the latter separability analysis applied to the MBA(805) and the SED series. 
The accurate and inaccurate anomaly scores are plotted in green and red, respectively. 
We then consider 12 different pairs of accurate/inaccurate methods among the eight previously mentioned anomaly scores. 
We slightly modify each score 50 different times in which we inject lag and noises and compute the accuracy measures. 
Figure~\ref{fig:separability}(a.4) and Figure~\ref{fig:separability}(b.4) are divided into four different subplots corresponding to 4 pairs (selected among the twelve different pairs due to lack of space). 
Each subplot corresponds to two box plots per accuracy measure. 
The green and red box plots correspond to the 50 accuracy measures on the accurate and inaccurate methods. 
If the red and green box plots are well separated, we can conclude that the corresponding accuracy measures are separating the accurate and inaccurate methods well. 
We observe that some accuracy measures (such as VUS-ROC) are more separable than others (such as RF). We thus measure the separability of the two box-plots by computing the Z-test. 

\begin{figure*}[tb]
  \centering
  \includegraphics[width=1\linewidth]{figures/pairwise_comp_example_long.pdf}
  %\vspace*{-0.5cm}
  \caption{Separability analysis applied on 4 pairs of accurate (green) and inaccurate (red) methods on (a) the MBA(805) data series, and (b) the SED data series.}
  %\vspace*{-0.3cm}
  \label{fig:separability}
\end{figure*}

We now aggregate all the results and compute the average Z-test for all pairs of accurate/inaccurate datasets (examples are shown in Figures~\ref{fig:separability}(a.2) and (b.2) for accurate anomaly scores, and in Figures~\ref{fig:separability}(a.3) and (b.3) for inaccurate anomaly scores, for the MBA(805) and SED series, respectively). 
Next, we perform the same operation over three different data series: MBA (805), MBA(820), and SED. 
Then, we depict the average Z-test for these three datasets in Figure~\ref{fig:separability_agg}(a). 
Finally, we show the average Z-test for all datasets in Figure~\ref{fig:separability_agg}(b). 


We observe that our proposed VUS-based and Range-based measures are significantly more separable than other current accuracy measures (up to two times for AUC-ROC, the best measures of all current ones). Furthermore, when analyzed in detail in Figure~\ref{fig:separability} and Figure~\ref{fig:separability_agg}, we confirm that VUS-based and Range-based are more separable over all three datasets. 

\begin{figure}[tb]
  \centering
  \includegraphics[width=\linewidth]{figures/agregated_sep_analysis.pdf}
  %\vspace*{-0.5cm}
  \caption{Overall separability analysis (averaged z-test between the accuracy values distributions of accurate and inaccurate methods) applied on 36 pairs on 3 datasets.}
  \label{fig:separability_agg}
\end{figure}


\noindent \textbf{Global Analysis: } Overall, we observe that VUS-ROC is the most robust (cf. Figure~\ref{fig:sensitivity}) and separable (cf. Figure~\ref{fig:separability_agg}) measure. 
On the contrary, Precision and Rprecision are non-robust and non-separable. 
Among all previous accuracy measures, only AUC-ROC is robust and separable. 
Popular measures, such as, F, RF, AUC-ROC, and AUC-PR are robust but non-separable.

In order to visualize the global statistical analysis, we merge the robustness and the separability analysis into a single plot. Figure~\ref{fig:global} depicts one scatter point per accuracy measure. 
The x-axis represents the averaged standard deviation of lag and noise (averaged values from Figure~\ref{fig:sensitivity_per_data}(a.1) and (a.2)). The y-axis corresponds to the averaged Z-test (averaged value from Figure~\ref{fig:separability_agg}). 
Finally, the size of the points corresponds to the sensitivity to the normal/abnormal ratio (values from Figure~\ref{fig:sensitivity_per_data}(a.3)). 
Figure~\ref{fig:global} demonstrates that our proposed measures (located at the top left section of the plot) are both the most robust and the most separable. 
Among all previous accuracy measures, only AUC-ROC is on the top left section of the plot. 
Popular measures, such as, F, RF, AUC-ROC, AUC-PR are on the bottom left section of the plot. 
The latter underlines the fact that these measures are robust but non-separable.
Overall, Figure~\ref{fig:global} confirms the effectiveness and superiority of our proposed measures, especially of VUS-ROC and VUS-PR.


\begin{figure}[tb]
  \centering
  \includegraphics[width=\linewidth]{figures/final_result.pdf}
  \caption{Evaluation of all measures based on: (y-axis) their separability (avg. z-test), (x-axis) avg. standard deviation of the accuracy values when varying lag and noise, (circle size) avg. standard deviation of the accuracy values when varying the normal/abnormal ratio.}
  \label{fig:global}
\end{figure}




\subsection{Consistency Analysis}
\label{sec:entropy}

In this section, we analyze the accuracy of the anomaly detection methods provided by the 13 accuracy measures. The objective is to observe the changes in the global ranking of anomaly detection methods. For that purpose, we formulate the following assumptions. First, we assume that the data series in each benchmark dataset are similar (i.e., from the same domain and sharing some common characteristics). As a matter of fact, we can assume that an anomaly detection method should perform similarly on these data series of a given dataset. This is confirmed when observing that the best anomaly detection methods are not the same based on which dataset was analyzed. Thus the ranking of the anomaly detection methods should be different for different datasets, but similar for every data series in each dataset. 
Therefore, for a given method $A$ and a given dataset $D$ containing data series of the same type and domain, we assume that a good accuracy measure results in a consistent rank for the method $A$ across the dataset $D$. 
The consistency of a method's ranks over a dataset can be measured by computing the entropy of these ranks. 
For instance, a measure that returns a random score (and thus, a random rank for a method $A$) will result in a high entropy. 
On the contrary, a measure that always returns (approximately) the same ranks for a given method $A$ will result in a low entropy. 
Thus, for a given method $A$ and a given dataset $D$ containing data series of the same type and domain, we assume that a good accuracy measure results in a low entropy for the different ranks for method $A$ on dataset $D$.

\begin{figure*}[tb]
  \centering
  \includegraphics[width=\linewidth]{figures/entropy_long.pdf}
  %\vspace*{-0.5cm}
  \caption{Accuracy evaluation of the anomaly detection methods. (a) Overall average entropy per category of measures. Analysis of the (b) averaged rank and (c) averaged rank entropy for each method and each accuracy measure over the entire benchmark. Example of (b.1) average rank and (c.1) entropy on the YAHOO dataset, KDD21 dataset (b.2, c.2). }
  \label{fig:entropy}
\end{figure*}

We now compute the accuracy measures for the nine different methods (we compute the anomaly scores ten different times, and we use the average accuracy). 
Figures~\ref{fig:entropy}(b.1) and (b.2) report the average ranking of the anomaly detection methods obtained on the YAHOO and KDD21 datasets, respectively. 
The x-axis corresponds to the different accuracy measures. We first observe that the rankings are more separated using Range-AUC and VUS measures for these two datasets. Figure~\ref{fig:entropy}(b) depicts the average ranking over the entire benchmark. The latter confirms the previous observation that VUS measures provide more separated rankings than threshold-based and AUC-based measures. We also observe an interesting ranking evolution for the YAHOO dataset illustrated in Figure~\ref{fig:entropy}(b.1). We notice that both LOF and MatrixProfile (brown and pink curve) have a low rank (between 4 and 5) using threshold and AUC-based measures. However, we observe that their ranks increase significantly for range-based and VUS-based measures (between 2.5 and 3). As we noticed by looking at specific examples (see Figure~\ref{exp:qual}), LOF and MatrixProfile can suffer from a lag issue even though the anomalies are well-identified. Therefore, the range-based and VUS-based measures better evaluate these two methods' detection capability.


Overall, the ranking curves show that the ranks appear more chaotic for threshold-based than AUC-, Range-AUC-, and VUS-based measures. 
In order to quantify this observation, we compute the Shannon Entropy of the ranks of each anomaly detection method. 
In practice, we extract the ranks of methods across one dataset and compute Shannon's Entropy of the different ranks. 
Figures~\ref{fig:entropy}(c.1) and (c.2) depict the entropy of each of the nine methods for the YAHOO and KDD21 datasets, respectively. 
Figure~\ref{fig:entropy}(c) illustrates the averaged entropy for all datasets in the benchmark for each measure and method, while Figure~\ref{fig:entropy}(a) shows the averaged entropy for each category of measures.
We observe that both for the general case (Figure~\ref{fig:entropy}(a) and Figure~\ref{fig:entropy}(c)) and some specific cases (Figures~\ref{fig:entropy}(c.1) and (c.2)), the entropy is reducing when using AUC-, Range-AUC-, and VUS-based measures. 
We report the lowest entropy for VUS-based measures. 
Moreover, we notice a significant drop between threshold-based and AUC-based. 
This confirms that the ranks provided by AUC- and VUS-based measures are consistent for data series belonging to one specific dataset. 


Therefore, based on the assumption formulated at the beginning of the section, we can thus conclude that AUC, range-AUC, and VUS-based measures are providing more consistent rankings. Finally, as illustrated in Figure~\ref{fig:entropy}, we also observe that VUS-based measures result in the most ordered and similar rankings for data series from the same type and domain.










\subsection{Execution Time Analysis}
\label{sec:exectime}

In this section, we evaluate the execution time required to compute different evaluation measures. 
In Section~\ref{sec:synthetic_eval_time}, we first measure the influence of different time series characteristics and VUS parameters on the execution time. In Section~\ref{sec:TSB_eval_time}, we  measure the execution time of VUS (VUS-ROC and VUS-PR simultaneously), R-AUC (R-AUC-ROC and R-AUC-PR simultaneously), and AUC-based measures (AUC-ROC and AUC-PR simultaneously) on the TSB-UAD benchmark. \commentRed{As demonstrated in the previous section, threshold-based measures are not robust, have a low separability power, and are inconsistent. 
Such measures are not suitable for evaluating anomaly detection methods. Thus, in this section, we do not consider threshold-based measures.}


\subsubsection{Evaluation on Synthetic Time Series}\hfill\\
\label{sec:synthetic_eval_time}

We first analyze the impact that time series characteristics and parameters have on the computation time of VUS-based measures. 
to that effect, we generate synthetic time series and labels, where we vary the following parameters: (i) the number of anomalies {\bf$\alpha$} in the time series, (ii) the average \textbf{$\mu(\ell_a)$} and standard deviation $\sigma(\ell_a)$ of the anomalies lengths in the time series (all the anomalies can have different lengths), (iii) the length of the time series \textbf{$|T|$}, (iv) the maximum buffer length \textbf{$L$}, and (v) the number of thresholds \textbf{$N$}.


We also measure the influence on the execution time of the R-AUC- and AUC- related parameter, that is, the number of thresholds ($N$).
The default values and the range of variation of these parameters are listed in Table~\ref{tab:parameter_range_time}. 
For VUS-based measures, we evaluate the execution time of the initial VUS implementation, as well as the two optimized versions, VUS$_{opt}$ and VUS$_{opt}^{mem}$.

\begin{table}[tb]
    \centering
    \caption{Value ranges for the parameters: number of anomalies ($\alpha$), average and standard deviation anomaly length ($\mu(\ell_a)$,$\sigma(\ell_a)$), time series length ($|T|$), maximum buffer length ($L$), and number of thresholds ($N$).}
    \begin{tabular}{|c|c|c|c|c|c|c|} 
 \hline
 Param. & $\alpha$ & $\mu(\ell_a)$ & $\sigma(\ell_{a})$ & $|T|$ & $L$ & $N$ \\ [0.5ex] 
 \hline\hline
 \textbf{Default} & 10 & 10 & 0 & $10^5$ & 5 & 250\\ 
 \hline
 Min. & 0 & 0 & 0 & $10^3$ & 0 & 2 \\
 \hline
 Max. & $2*10^3$ & $10^3$ & $10$ & $10^5$ & $10^3$ & $10^3$ \\ [1ex] 
 \hline
\end{tabular}
    \label{tab:parameter_range_time}
\end{table}


Figure~\ref{fig:sythetic_exp_time} depicts the execution time (averaged over ten runs) for each parameter listed in Table~\ref{tab:parameter_range_time}. 
Overall, we observe that the execution time of AUC-based and R-AUC-based measures is significantly smaller than VUS-based measures.
In the following paragraph, we analyze the influence of each parameter and compare the experimental execution time evaluation to the theoretical complexity reported in Table~\ref{tab:complexity_summary}.

\vspace{0.2cm}
\noindent {\bf [Influence of $\alpha$]}:
In Figure~\ref{fig:sythetic_exp_time}(a), we observe that the VUS, VUS$_{opt}$, and VUS$_{opt}^{mem}$ execution times are linearly increasing with $\alpha$. 
The increase in execution time for VUS, VUS$_{opt}$, and VUS$_{opt}^{mem}$ is more pronounced when we vary $\alpha$, in contrast to $l_a$ (which nevertheless, has a similar effect on the overall complexity). 
We also observe that the VUS$_{opt}^{mem}$ execution time grows slower than $VUS_{opt}$ when $\alpha$ increases. 
This is explained by the use of 2-dimensional arrays for the storage of predictions, which use contiguous memory locations that allow for faster access, decreasing the dependency on $\alpha$.

\vspace{0.2cm}
\noindent {\bf [Influence of $\mu(\ell_a)$]}:
As shown in Figure~\ref{fig:sythetic_exp_time}(b), the execution time variation of VUS, VUS$_{opt}$, and VUS$_{opt}^{mem}$ caused by $\ell_a$ is rather insignificant. 
We also observe that the VUS$_{opt}$ and VUS$_{opt}^{mem}$ execution times are significantly lower when compared to VUS. 
This is explained by the smaller dependency of the complexity of these algorithms on the time series length $|T|$. 
Overall, the execution time for both VUS$_{opt}$ and VUS$_{opt}^{mem}$ is significantly lower than VUS, and follows a similar trend. 

\vspace{0.2cm}
\noindent {\bf [Influence of $\sigma(\ell_a)$]}: 
As depicted in Figure~\ref{fig:sythetic_exp_time}(d) and inferred from the theoretical complexities in Table~\ref{tab:complexity_summary}, none of the measures are affected by the standard deviation of the anomaly lengths.

\vspace{0.2cm}
\noindent {\bf [Influence of $|T|$]}:
For short time series (small values of $|T|$), we note that O($T_1$) becomes comparable to O($T_2$). 
Thus, the theoretical complexities approximate to $O(NL(T_1+T_2))$, $O(N*(T_1+T_2))+O(NLT_2)$ and $O(N(T_1+T_2))$ for VUS, VUS$_{opt}$, and VUS$_{opt}^{mem}$, respectively. 
Indeed, we observe in Figure~\ref{fig:sythetic_exp_time}(c) that the execution times of VUS, VUS$_{opt}$, and VUS$_{opt}^{mem}$ are similar for small values of $|T|$. However, for larger values of $|T|$, $O(T_1)$ is much higher compared to $O(T_2)$, thus resulting in an effective complexity of $O(NLT_1)$ for VUS, and $O(NT_1)$ for VUS$_{opt}$, and VUS$_{opt}^{mem}$. 
This translates to a significant improvement in execution time complexity for VUS$_{opt}$ and VUS$_{opt}^{mem}$ compared to VUS, which is confirmed by the results in Figure~\ref{fig:sythetic_exp_time}(c).

\vspace{0.2cm}
\noindent {\bf [Influence of $N$]}: 
Given the theoretical complexity depicted in Table~\ref{tab:complexity_summary}, it is evident that the number of thresholds affects all measures in a linear fashion.
Figure~\ref{fig:sythetic_exp_time}(e) demonstrates this point: the results of varying $N$ show a linear dependency for VUS, VUS$_{opt}$, and VUS$_{opt}^{mem}$ (i.e., a logarithmic trend with a log scale on the y axis). \commentRed{Moreover, we observe that the AUC and range-AUC execution time is almost constant regardless of the number of thresholds used. The latter is explained by the very efficient implementation of AUC measures. Therefore, the linear dependency on the number of thresholds is not visible in Figure~\ref{fig:sythetic_exp_time}(e).}

\vspace{0.2cm}
\noindent {\bf [Influence of $L$]}: Figure~\ref{fig:sythetic_exp_time}(f) depicts the influence of the maximum buffer length $L$ on the execution time of all measures. 
We observe that, as $L$ grows, the execution time of VUS$_{opt}$ and VUS$_{opt}^{mem}$ increases slower than VUS. 
We also observe that VUS$_{opt}^{mem}$ is more scalable with $L$ when compared to VUS$_{opt}$. 
This is consistent with the theoretical complexity (cf. Table~\ref{tab:complexity_summary}), which indicates that the dependence on $L$ decreases from $O(NL(T_1+T_2+\ell_a \alpha))$ for VUS to $O(NL(T_2+\ell_a \alpha)$ and $O(NL(\ell_a \alpha))$ for $VUS_{opt}$, and $VUS_{opt}^{mem}$.





\begin{figure*}[tb]
  \centering
  \includegraphics[width=\linewidth]{figures/synthetic_res.pdf}
  %\vspace*{-0.5cm}
  \caption{Execution time of VUS, R-AUC, AUC-based measures when we vary the parameters listed in Table~\ref{tab:parameter_range_time}. The solid lines correspond to the average execution time over 10 runs. The colored envelopes are to the standard deviation.}
  \label{fig:sythetic_exp_time}
\end{figure*}


\vspace{0.2cm}
In order to obtain a more accurate picture of the influence of each of the above parameters, we fit the execution time (as affected by the parameter values) using linear regression; we can then use the regression slope coefficient of each parameter to evaluate the influence of that parameter. 
In practice, we fit each parameter individually, and report the regression slope coefficient, as well as the coefficient of determination $R^2$.
Table~\ref{tab:parameter_linear_coeff} reports the coefficients mentioned above for each parameter associated with VUS, VUS$_{opt}$, and VUS$_{opt}^{mem}$.



\begin{table}[tb]
    \centering
    \caption{Linear regression slope coefficients ($C.$) for VUS execution times, for each parameter independently. }
    \begin{tabular}{|c|c|c|c|c|c|c|} 
 \hline
 Measure & Param. & $\alpha$ & $l_a$ & $|T|$ & $L$ & $N$\\ [0.5ex] 
 \hline\hline
 \multirow{2}{*}{$VUS$} & $C.$ & 21.9 & 0.02 & 2.13 & 212 & 6.24\\\cline{2-7}
 & {$R^2$} & 0.99 & 0.15 & 0.99 & 0.99 & 0.99 \\   
 \hline
  \multirow{2}{*}{$VUS_{opt}$} & $C.$ & 24.2  & 0.06 & 0.19 & 27.8 & 1.23\\\cline{2-7}
  & $R^2$& 0.99 & 0.86 & 0.99 & 0.99 & 0.99\\ 
 \hline
 \multirow{2}{*}{$VUS_{opt}^{mem}$} & $C.$ & 21.5 & 0.05 & 0.21 & 15.7 & 1.16\\\cline{2-7}
  & $R^2$ & 0.99 & 0.89 & 0.99 & 0.99 & 0.99\\[1ex] 
 \hline
\end{tabular}
    \label{tab:parameter_linear_coeff}
\end{table}

Table~\ref{tab:parameter_linear_coeff} shows that the linear regression between $\alpha$ and the execution time has a $R^2=0.99$. Thus, the dependence of execution time on $\alpha$ is linear. We also observe that VUS$_{opt}$ execution time is more dependent on $\alpha$ than VUS and VUS$_{opt}^{mem}$ execution time.
Moreover, the dependence of the execution time on the time series length ($|T|$) is higher for VUS than for VUS$_{opt}$ and VUS$_{opt}^{mem}$. 
More importantly, VUS$_{opt}$ and VUS$_{opt}^{mem}$ are significantly less dependent than VUS on the number of thresholds and the maximal buffer length. 







\subsubsection{Evaluation on TSB-UAD Time Series}\hfill\\
\label{sec:TSB_eval_time}

In this section, we verify the conclusions outlined in the previous section with real-world time series from the TSB-UAD benchmark. 
In this setting, the parameters $\alpha$, $\ell_a$, and $|T|$ are calculated from the series in the benchmark and cannot be changed. Moreover, $L$ and $N$ are parameters for the computation of VUS, regardless of the time series (synthetic or real). Thus, we do not consider these two parameters in this section.

\begin{figure*}[tb]
  \centering
  \includegraphics[width=\linewidth]{figures/TSB2.pdf}
  \caption{Execution time of VUS, R-AUC, AUC-based measures on the TSB-UAD benchmark, versus $\alpha$, $\ell_a$, and $|T|$.}
  \label{fig:TSB}
\end{figure*}

Figure~\ref{fig:TSB} depicts the execution time of AUC, R-AUC, and VUS-based measures versus $\alpha$, $\mu(\ell_a)$, and $|T|$.
We first confirm with Figure~\ref{fig:TSB}(a) the linear relationship between $\alpha$ and the execution time for VUS, VUS$_{opt}$ and VUS$_{opt}^{mem}$.
On further inspection, it is possible to see two separate lines for almost all the measures. 
These lines can be attributed to the time series length $|T|$. 
The convergence of VUS and $VUS_{opt}$ when $\alpha$ grows shows the stronger dependence that $VUS_{opt}$ execution time has on $\alpha$, as already observed with the synthetic data (cf. Section~\ref{sec:synthetic_eval_time}). 

In Figure~\ref{fig:TSB}(b), we observe that the variation of the execution time with $\ell_a$ is limited when compared to the two other parameters. We conclude that the variation of $\ell_a$ is not a key factor in determining the execution time of the measures.
Furthermore, as depicted in Figure~\ref{fig:TSB}(c), $VUS_{opt}$ and $VUS_{opt}^{mem}$ are more scalable than VUS when $|T|$ increases. 
We also confirm the linear dependence of execution time on the time series length for all the accuracy measures, which is consistent with the experiments on the synthetic data. 
The two abrupt jumps visible in Figure~\ref{fig:TSB}(c) are explained by significant increases of $\alpha$ in time series of the same length. 

\begin{table}[tb]
\centering
\caption{Linear regression slope coefficients ($C.$) for VUS execution time, for all time series parameters all-together.}
\begin{tabular}{|c|ccc|c|} 
 \hline
Measure & $\alpha$ & $|T|$ & $l_a$ & $R^2$ \\ [0.5ex] 
 \hline\hline
 \multirow{1}{*}{${VUS}$} & 7.87 & 13.5 & -0.08 & 0.99  \\ 
 %\cline{2-5} & $R^2$ & \multicolumn{3}{c|}{ 0.99}\\
 \hline
 \multirow{1}{*}{$VUS_{opt}$} & 10.2 & 1.70 & 0.09 & 0.96 \\
 %\cline{2-5} & $R^2$ & \multicolumn{3}{c|}{0.96}\\
\hline
 \multirow{1}{*}{$VUS_{opt}^{mem}$} & 9.27 & 1.60 & 0.11 & 0.96 \\
 %\cline{2-5} & $R^2$ & \multicolumn{3}{c|}{0.96} \\
 \hline
\end{tabular}
\label{tab:parameter_linear_coeff_TSB}
\end{table}



We now perform a linear regression between the execution time of VUS, VUS$_{opt}$ and VUS$_{opt}^{mem}$, and $\alpha$, $\ell_a$ and $|T|$.
We report in Table~\ref{tab:parameter_linear_coeff_TSB} the slope coefficient for each parameter, as well as the $R^2$.  
The latter shows that the VUS$_{opt}$ and VUS$_{opt}^{mem}$ execution times are impacted by $\alpha$ at a larger degree than $\alpha$ affects VUS. 
On the other hand, the VUS$_{opt}$ and VUS$_{opt}^{mem}$ execution times are impacted to a significantly smaller degree by the time series length when compared to VUS. 
We also confirm that the anomaly length does not impact the execution time of VUS, VUS$_{opt}$, or VUS$_{opt}^{mem}$.
Finally, our experiments show that our optimized implementations VUS$_{opt}$ and VUS$_{opt}^{mem}$ significantly speedup the execution of the VUS measures (i.e., they can be computed within the same order of magnitude as R-AUC), rendering them practical in the real world.











\subsection{Summary of Results}


Figure~\ref{fig:overalltable} depicts the ranking of the accuracy measures for the different tests performed in this paper. The robustness test is divided into three sub-categories (i.e., lag, noise, and Normal vs. abnormal ratio). We also show the overall average ranking of all accuracy measures (most right column of Figure~\ref{fig:overalltable}).
Overall, we see that VUS-ROC is always the best, and VUS-PR and Range-AUC-based measures are, on average, second, third, and fourth. We thus conclude that VUS-ROC is the overall winner of our experimental analysis.

\commentRed{In addition, our experimental evaluation shows that the optimized version of VUS accelerates the computation by a factor of two. Nevertheless, VUS execution time is still significantly slower than AUC-based approaches. However, it is important to mention that the efficiency of accuracy measures is an orthogonal problem with anomaly detection. In real-time applications, we do not have ground truth labels, and we do not use any of those measures to evaluate accuracy. Measuring accuracy is an offline step to help the community assess methods and improve wrong practices. Thus, execution time should not be the main criterion for selecting an evaluation measure.}


\section{Empirical Analysis}
We begin by presenting the two main assumptions we will make to analyze \Cref{alg:uSCG,alg:SCG}. The first is an assumption on the Lipschitz-continuity of $\nabla f$ with respect to the norm $\|\cdot\|_{\ast}$ restricted to $\mathcal{X}$. We do not assume this norm to be Euclidean which means our results apply to the geometries relevant to training neural networks.
\begin{assumption}\label{asm:Lip} The gradient $\nabla f$ is $L$-Lipschitz with $L \in (0,\infty)$, i.e.,
    \begin{equation}
    \|\nabla f(x) - \nabla f(x)\|_{\ast}
    \leq
    L\|x-y\|
    \quad \forall x,y \in \mathcal X.
    \end{equation}
Furthermore, $f$ is bounded below by $\fmin$.
\end{assumption}
Our second assumption is that the stochastic gradient oracle we have access to is unbiased and has a bounded variance, a typical assumption in stochastic optimization.
\begin{assumption}\label{asm:stoch}
The stochastic gradient oracle $\nabla f(\cdot,\xi):\mathcal X\rightarrow \mathbb{R}^d$ satisfies.
    \begin{assnum}
        \item \label{asm:stoch:unbiased}
            Unbiased:
            \(%
                \mathbb{E}_{\xi}\left[\nabla f(x,\xi)\right] = \nabla f(x) \quad \forall x \in \mathcal X
            \).%
        \item  \label{asm:stoch:var}
            Bounded variance:\\
            \(%
                \mathbb{E}_{\xi}\left[\|\nabla f(x,\xi)-\nabla f(x)\|_2^2\right] \leq \sigma^2  \quad \forall x \in \mathcal X,\sigma\geq 0
            \).%
    \end{assnum}
\end{assumption}

With these assumptions we can state our worst-case convergence rates, first for \Cref{alg:uSCG} and then for \Cref{alg:SCG}. 

\looseness=-1To bridge the gap between theory and practice, we investigate these algorithms when run with a \emph{constant} stepsize $\gamma$, which depends on the specified horizon $n\in\mathbb{N}^*$, and momentum which is either constant $\alpha\in(0,1)$ (except for the first iteration where we take $\alpha=1$ by convention) or \emph{vanishing} $\alpha_k\searrow 0$. The exact constants for the rates can be found in the proofs in \Cref{app:analysis}; we try to highlight the dependence on the parameters $L$ and $\rho$, which correspond to the natural geometry of $f$ and $\mathcal{D}$, explicitly here. Our rates are non-asymptotic and use big O notation for brevity.

\begin{toappendix}
\label{app:analysis}
In this section we present the proofs of the main convergence results of the paper as well as some intermediary lemmas that we will make use of along the way. Throughout this section, we adopt the notation:
\begin{align*}
\text{(stochastic gradient estimator error)} && \lambda^k &:= d^k-\nabla f(x^k) \\
\text{(diameter of $\mathcal{D}$ in $\ell_2$ norm)} && D_2 &:= \max_{x,y\in\mathcal{D}}\norm{x-y}_2 \\
\text{(radius of $\mathcal{D}$ in $\ell_2$ norm)} && \rho_2 &:= \max_{x\in\mathcal{D}}\norm{x}_2 \\
\text{(norm equivalence constant)} && \zeta &:= \max_{x\in\mathcal{X}}\frac{\norm{x}_{\ast}}{\norm{x}_2} \\
\text{(Lipschitz constant of $\nabla f$ with respect to $\norm{\cdot}_{2}$)} && L_2 &:= \inf \{M>0\colon \forall x,y\in\mathcal{X}, \norm{\nabla f(x)-\nabla f(y)}_{2}\leq M\norm{x-y}_{2}\}
\end{align*}
We analyze each algorithm separately, although the analysis is effectively unified between the two, modulo constants. This is done in \Cref{subsec:uSCG,subsec:SCG}, respectively. Our convergence analysis proceeds in three steps: we begin by establishing a template descent inequality for each algorithm via the descent lemma. Next, we analyze the behavior of the second moment of the error $\mathbb{E}[\norm{\lambda^k}_{2}^2]$ under different choices for $\alpha$. Then, we combine these results to derive a convergence rate. Finally, we note that when analyzing algorithms with constant momentum, we will still always take $\alpha=1$ on the first iteration $k=1$.

\subsection{Convergence analysis of \ref{eq:uSCG}}\label{subsec:uSCG}
We begin with the analysis of \Cref{alg:uSCG} by establishing a generic template inequality for the dual norm of the gradient at iteration $k$. This inequality holds regardless of whether the momentum $\alpha_k$ is constant or vanishing, as long as it remains in $(0,1]$.
\begin{lemma}[\ref{eq:uSCG} template inequality]
\label{lem:uSCGtemplate1}
    Suppose \Cref{asm:Lip} holds. Let $n\in\mathbb{N}^*$ and consider the iterates $\{x^{k}\}_{k=1}^n$ generated by \Cref{alg:uSCG} with a constant stepsize $\gamma>0$.
    Then we have
    \begin{equation}
        \mathbb{E}[\norm{\nabla f(\bar{x}^n)}_2^2]\leq \frac{\mathbb{E}[f(x^{1})-\fmin]}{\rho\gamma n} +\frac{L\rho\gamma}{2} + \frac{1}{n}\left(\frac{\rho_2}{\rho}+\zeta\right)\sum\limits_{k=1}^n\sqrt{\mathbb{E}[\norm{\lambda^{k}}_2^2]}.
    \end{equation}
\end{lemma}
\begin{proof}
    Under \Cref{asm:Lip}, we can use the descent lemma for the function $f$ at the points $x^{k}$ and $x^{k+1}$ to get, for all $k\in\{1,\ldots,n\}$,
    \begin{equation}\label{eq:lem:uSCGtemplate1:first2}
        \begin{aligned}
            f(x^{k+1})&\leq f(x^{k})+ \langle \nabla f(x^{k}),x^{k+1}-x^{k}\rangle +\tfrac{L}{2}\norm{x^{k+1}-x^{k}}^{2}
            \\
            &= f(x^{k})+\langle \nabla f(x^{k})-d^{k},x^{k+1}-x^{k}\rangle + \langle d^{k},x^{k+1}-x^{k}\rangle+\tfrac{L}{2}\norm{x^{k+1}-x^{k}}^{2}
            \\
            &= f(x^{k})+\gamma \langle \nabla f(x^{k})-d^{k},\lmo (d^{k})\rangle+\gamma \langle d^{k},\lmo(d^{k})\rangle +\tfrac{L\gamma^{2}}{2}\norm{\lmo(d^{k})}^{2}
            \\
            &\leq f(x^{k})+\gamma \rho_{2}\norm{\lambda^{k}}_{2}+\gamma \langle d^{k},\lmo(d^{k})\rangle +\tfrac{L\gamma^{2}}{2}\rho^{2},
        \end{aligned}
    \end{equation}
    the final step employing Cauchy-Schwarz, the definition of $\lambda^k$, and the definition of $\rho_2$ as the radius of $\mathcal{D}$ in the $\norm{\cdot}_2$ norm.
    By definition of the dual norm we have, for all $u\in\mathcal{X}$,
    \begin{equation*}
        \|u\|_{\ast} = \max\limits_{v\colon \|v\|\leq 1}\langle u,v\rangle = \max_{v\in\mathcal{D}}\langle u,\tfrac{1}{\rho}v\rangle= -\langle u, \tfrac{1}{\rho}\lmo(u)\rangle
    \end{equation*}
    which means that, for all $k\in\{1,\ldots,n\}$,
    \begin{equation*}
        \gamma \langle d^k, \lmo(d^k)\rangle = \gamma\rho\langle d^k,\tfrac{1}{\rho}\lmo(d^k)\rangle = -\gamma\rho\|d^k\|_{\ast}.
    \end{equation*}
    Plugging this expression for $\gamma\langle d^k,\lmo(d^k)\rangle$ into \eqref{eq:lem:uSCGtemplate1:first2} gives, for all $k\in\{1,\ldots,n\}$,
    \begin{equation*}
        \begin{aligned}
            f(x^{k+1})
                &\leq f(x^{k})+\gamma \rho_{2}\norm{\lambda^{k}}_{2}-\gamma\rho\|d^k\|_{\ast} +\tfrac{L\gamma^{2}}{2}\rho^{2}\\
                &= f(x^{k})+\gamma \rho_{2}\norm{\lambda^{k}}_{2}-\gamma\rho\|d^k - \nabla f(x^k) + \nabla f(x^k)\|_{\ast} +\tfrac{L\gamma^{2}}{2}\rho^{2}\\
                &\stackrel{\text{(a)}}{\leq} f(x^{k})+\gamma \rho_{2}\norm{\lambda^{k}}_{2} +\gamma\rho\|\lambda^k\|_{\ast} -\gamma\rho\|\nabla f(x^k)\|_{\ast} +\tfrac{L\gamma^{2}}{2}\rho^{2}\\
                &\stackrel{\text{(b)}}{\leq} f(x^{k})+\gamma (\rho_{2}+\zeta\rho)\norm{\lambda^{k}}_{2}-\gamma\rho\|\nabla f(x^k)\|_{\ast} +\tfrac{L\gamma^{2}}{2}\rho^{2},
        \end{aligned}
    \end{equation*}
    applying the reverse triangle inequality in (a) while (b) stems from the definition of $\zeta$.
    By rearranging terms and taking expectations, we get
    \begin{equation*}
        \begin{aligned}
            \gamma\rho\mathbb{E}[\norm{\nabla f(x^k)}_{\ast}]
                &\leq \mathbb{E}[f(x^{k})-f(x^{k+1})] + \gamma\left(\rho_2+\zeta\rho\right)\mathbb{E}[\norm{\lambda^{k}}_2] +\frac{L\rho^2\gamma^2}{2}.
        \end{aligned}
    \end{equation*}
    Summing this from $k=1$ to $n$ and dividing by $\gamma\rho n$ we get
    \begin{equation*}
        \begin{aligned}
            \mathbb{E}[\norm{\nabla f(\bar{x}^n)}_{\ast}]
                &= \frac{1}{n}\sum\limits_{k=1}^n\mathbb{E}[\norm{\nabla f(x^k)}_{\ast}]\\
                &\leq \frac{\mathbb{E}[f(x^{1})-f(x^{n+1})]}{\rho\gamma n} +\frac{L\rho\gamma}{2} + \frac{1}{n}\left(\frac{\rho_2}{\rho}+\zeta\right)\sum\limits_{k=1}^n\mathbb{E}[\norm{\lambda^{k}}_2]\\
                &\stackrel{\text{(a)}}{\leq} \frac{\mathbb{E}[f(x^{1})-\fmin]}{\rho\gamma n} +\frac{L\rho\gamma}{2} + \frac{1}{n}\left(\frac{\rho_2}{\rho}+\zeta\right)\sum\limits_{k=1}^n\mathbb{E}[\norm{\lambda^{k}}_2]\\
                &\stackrel{\text{(b)}}{\leq} \frac{\mathbb{E}[f(x^{1})-\fmin]}{\rho\gamma n} +\frac{L\rho\gamma}{2} + \frac{1}{n}\left(\frac{\rho_2}{\rho}+\zeta\right)\sum\limits_{k=1}^n\sqrt{\mathbb{E}[\norm{\lambda^{k}}_2^2]},
        \end{aligned}
    \end{equation*}
    using the definition of $\fmin$ for (a) and Jensen's inequality for (b).
\end{proof}

At this point, we need to determine the growth of the induced error captured by the quantity $\norm{\lambda^{k}}_2^2$. To estimate this, we first use a recursion relating $\mathbb{E}[\norm{\lambda^{k}}_2^2]$ and $\mathbb{E}[\norm{\lambda^{k-1}}_2^2]$ adapted from the proof in \citet[Lem. 6]{mokhtari2020stochastic} and then we prove a bound on the decay of $\norm{\lambda^k}_2^2$ for \Cref{alg:uSCG}.
\begin{lemma}[Linear recursive inequality for $\mathbb{E}\norm{\lambda^k}_2^2$]\label{lem:uSCGerror}
    Suppose \Cref{asm:Lip,asm:stoch} hold. Let $n\in\mathbb{N}^*$ and consider the iterates $\{x_k\}_{k=1}^n$ generated by \Cref{alg:uSCG} with a constant stepsize $\gamma>0$. Then, for all $k\in\{1,\ldots,n
    \}$,
    \begin{equation*}
        \mathbb{E}[\norm{\lambda^k}_2^2] \leq \left(1-\frac{\alpha_k}{2}\right)\mathbb{E}[\norm{\lambda^{k-1}}_2^2] + \frac{2L_2^2\rho_2^2\gamma^2}{\alpha_k} + \alpha_k^2\sigma^2.
    \end{equation*}
\end{lemma}
\begin{proof}
    The proof is a straightforward adaptation of the arguments laid out in \citet[Lem. 6]{mokhtari2020stochastic}, which in fact do not depend on convexity nor on the choice of stepsize. Let $n\in\mathbb{N}^*$ and $k\in\{1,\ldots,n\}$, then
    \begin{equation*}
        \begin{aligned}
            \norm{\lambda^k}_2^2
                &= \norm{\nabla f(x^k) - d^{k}}_2^2\\
                &= \norm{\nabla f(x^k) - \alpha_k \nabla f(x^k,\xi_k) - (1-\alpha_k)d^{k-1}}_2^2\\
                &= \norm{\alpha_k\left(\nabla f(x^k) - \nabla f(x^k,\xi_k)\right) +(1-\alpha_k)\left(\nabla f(x^{k})-\nabla f(x^{k-1})\right) - (1-\alpha_k)\left(d^{k-1} - \nabla f(x^{k-1})\right)}_2^2\\
                &= \alpha_k^2\norm{\nabla f(x^k) - \nabla f(x^k,\xi_k)}_2^2 + (1-\alpha_k)^2\norm{\nabla f(x^k)-\nabla f(x^{k-1})}_2^2\\
                    &\quad\quad + (1-\alpha_k)^2\norm{\nabla f(x^{k-1})-d^{k-1}}_2^2\\
                    &\quad\quad +2\alpha_k(1-\alpha_k)\langle\nabla f(x^{k-1})-\nabla f(x^{k-1},\xi_{k-1}), \nabla f(x^k)-\nabla f(x^{k-1})\rangle\\
                    &\quad\quad +2\alpha_k(1-\alpha_k)\langle \nabla f(x^k)-\nabla f(x^k,\xi_k), \nabla f(x^{k-1})-d^{k-1}\rangle\\
                    &\quad\quad +2(1-\alpha_k)^2\langle \nabla f(x^k)-\nabla f(x^{k-1}),\nabla f(x^{k-1}) - d^{k-1}\rangle.
        \end{aligned}
    \end{equation*}
    Taking the expectation conditioned on the filtration $\mathcal{F}_k$ generated by the iterates until $k$, i.e., the sigma algebra generated by $\{x_1,\ldots,x_k\}$, which we denote using $\mathbb{E}_k[\cdot]$, and using the unbiased property in \Cref{asm:stoch}, we get,
    \begin{equation*}
        \begin{aligned}
            \mathbb{E}_k[\norm{\lambda^k}_2^2]
                &= \alpha_k^2\mathbb{E}_k[\norm{\nabla f(x^k)-\nabla f(x^k,\xi_k)}_2^2] + (1-\alpha_k)^2\norm{\nabla f(x^k)-\nabla f(x^{k-1})}_2^2\\
                    &\quad\quad + (1-\alpha_k)^2\norm{\lambda^{k-1}}_2^2 + 2(1-\alpha_k)^2\langle \nabla f(x^k)-\nabla f(x^{k-1}),\lambda^{k-1}\rangle.
        \end{aligned}
    \end{equation*}
    From this expression we can estimate,
    \begin{equation*}
        \begin{aligned}
            \mathbb{E}_k[\norm{\lambda^k}_2^2]
                &\stackrel{\text{(a)}}{\leq} \alpha_k^2\sigma^2 + (1-\alpha_k)^2\norm{\nabla f(x^{k})-\nabla f(x^{k-1})}_2^2 + (1-\alpha_k)^2\norm{\lambda^{k-1}}_2^2 + 2(1-\alpha_k)^2\langle \nabla f(x^k)-\nabla f(x^{k-1}),\lambda^{k-1}\rangle\\
                &\stackrel{\text{(b)}}{\leq} \alpha_k^2\sigma^2 + (1-\alpha_k)^2\norm{\nabla f(x^{k})-\nabla f(x^{k-1})}_2^2 + (1-\alpha_k)^2\norm{\lambda^{k-1}}_2^2\\
                    &\quad\quad + (1-\alpha_k)^2\left(\tfrac{\alpha_k}{2}\norm{\nabla f(x^k)-\nabla f(x^{k-1})}_2^2+\tfrac{2}{\alpha_k}\norm{\lambda^{k-1}}_2^2\right)\\
                 &\stackrel{\text{(c)}}{\leq} \alpha_k^2\sigma^2 + (1-\alpha_k)^2L_2^2\norm{x^k-x^{k-1}}_2^2 + (1-\alpha_k)^2\norm{\lambda^{k-1}}_2^2 + (1-\alpha_k)^2\left((\tfrac{\alpha_k}{2})L_2^2\norm{x^k-x^{k-1}}_{2}^2+\tfrac{2}{\alpha_k}\norm{\lambda^{k-1}}_2^2\right)\\
                 &\stackrel{\text{(d)}}{\leq} \alpha_k^2\sigma^2 + (1-\alpha_k)^2L_2^2\rho_2^2\gamma^2 + (1-\alpha_k)^2\norm{\lambda^{k-1}}_2^2 + (1-\alpha_k)^2\left((\tfrac{\alpha_k}{2})L_2^2\rho_2^2\gamma^2+\tfrac{2}{\alpha_k}\norm{\lambda^{k-1}}_2^2\right)\\
                 &\stackrel{\text{(e)}}{\leq} \alpha_k^2\sigma^2 + (1+\tfrac{\alpha_k}{2})(1-\alpha_k)L_2^2\rho_2^2\gamma^2 + (1+\tfrac{2}{\alpha_k})(1-\alpha_k)\norm{\lambda^{k-1}}_2^2,
        \end{aligned}
    \end{equation*}
    using the bounded variance property from \Cref{asm:stoch} for (a), Young's inequality with parameter $\alpha_k/2>0$ for (b), the Lipschitz property of $f$ under norm $\|\cdot\|_2$ for (c), the update definition from \Cref{alg:uSCG} for (d), and the fact that $1-\alpha_k < 1$ for (e).
    To complete the proof, we note that
    \begin{equation*}
        (1+\tfrac{2}{\alpha_k})(1-\alpha_k)\leq \tfrac{2}{\alpha_k}\quad\text{and}\quad(1-\alpha_k)(1+\tfrac{\alpha_k}{2})\leq (1-\tfrac{\alpha_k}{2})
    \end{equation*}
    which, applied to the previous inequality and taking total expectations, yields
    \begin{equation*}
        \mathbb{E}[\norm{\lambda^k}_2^2] \leq \left(1-\frac{\alpha_k}{2}\right)\mathbb{E}[\norm{\lambda^{k-1}}_2^2] + \alpha_k^2\sigma^2 + \frac{2L_2^2\rho_2^2\gamma^2}{\alpha_k}.
    \end{equation*}
\end{proof}

\subsubsection{Constant $\alpha$}

\begin{lemma}
    Suppose \Cref{asm:Lip,asm:stoch} hold. Let $n \in \mathbb{N}^*$ and consider the iterates $\{x^k\}_{k=1}^n$ generated by \Cref{alg:uSCG} with constant stepsize $\gamma >0$ and constant momentum $\alpha\in(0,1)$ with the exception of the first iteration, where we take $\alpha=1$.
    Then, we have for all $k\in\{1,\ldots,n\}$
    \begin{equation*}
        \begin{aligned}
            \sqrt{\mathbb{E}[\norm{\lambda^k}_2^2]}
                &\leq \frac{\sqrt{2}L_2\rho_2\gamma}{\alpha} + \left(\sqrt{\alpha} + \left(\sqrt{1-\frac{\alpha}{2}}\right)^k\right)\sigma.
        \end{aligned}
    \end{equation*}
\end{lemma}
\begin{proof}
    Let $n\in\mathbb{N}^*$, $k\in\{1,\ldots,n\}$, and invoke \Cref{lem:uSCGerror} to get
    \begin{equation*}
        \mathbb{E}[\norm{\lambda^k}_2^2] \leq \left(1-\frac{\alpha}{2}\right)\mathbb{E}[\norm{\lambda^{k-1}}_2^2] + \frac{2L_2^2\rho_2^2\gamma^2}{\alpha} + \alpha^2\sigma^2.
    \end{equation*}
    Applying \Cref{lem:recursive_geometric} with $\beta = \frac{\alpha}{2}$ and $\eta = \frac{2L_2^2\rho_2^2\gamma^2}{\alpha}+\alpha^2\sigma^2$ gives directly
    \begin{equation*}
        \begin{aligned}
            \mathbb{E}[\norm{\lambda^k}_2^2]
                &\leq \frac{2L_2^2\rho_2^2\gamma^2}{\alpha^2} + \alpha\sigma^2 + \left(1-\frac{\alpha}{2}\right)^k\mathbb{E}[\norm{\lambda^1}_2^2]\\
                &\leq \frac{2L_2^2\rho_2^2\gamma^2}{\alpha^2} + \left(\alpha + \left(1-\frac{\alpha}{2}\right)^k\right)\sigma^2
        \end{aligned}
    \end{equation*}
    after using \Cref{asm:stoch} in the final inequality.
    Taking square roots and upper boudning then yields
    \begin{equation*}
        \begin{aligned}
            \sqrt{\mathbb{E}[\norm{\lambda^k}_2^2]}
                &\leq \frac{\sqrt{2}L_2\rho_2\gamma}{\alpha} + \left(\sqrt{\alpha} + \left(\sqrt{1-\frac{\alpha}{2}}\right)^k\right)\sigma.
        \end{aligned}
    \end{equation*}
\end{proof}

\end{toappendix}

\begin{lemmarep}[{Convergence rate for \ref{eq:uSCG} with constant $\alpha$}]\label{lem:uSCGrate1}
    Suppose \Cref{asm:Lip,asm:stoch} hold. Let $n\in\mathbb{N}^*$ and consider the iterates $\{x^k\}_{k=1}^n$ generated by \Cref{alg:uSCG} with constant stepsize $\gamma = \frac{1}{\sqrt{n}}$ and constant momentum $\alpha\in(0,1)$.
    Then, it holds that
    \begin{equation*}
        \mathbb{E}[\norm{\nabla f(\bar{x}^n)}_{\ast}] \leq O\left(\tfrac{L\rho}{\sqrt{n}}+\sigma\right).
    \end{equation*}
\end{lemmarep}
\begin{appendixproof}
    Let $n\in\mathbb{N}^*$; we will first invoke \Cref{lem:uSCGtemplate1} and then we will estimate the error terms inside using \Cref{lem:uSCGerror} under \Cref{asm:Lip,asm:stoch}.
    As shown in \Cref{lem:uSCGtemplate1},
    \begin{equation}\label{eq:uSCGrate1}
        \begin{aligned}
            \mathbb{E}[\norm{\nabla f(\bar{x}^n)}_2^2]
                &\leq \frac{\mathbb{E}[f(x^{1})-\fmin]}{\rho\gamma n} +\frac{L\rho\gamma}{2n} + \frac{1}{n}\left(\frac{\rho_2}{\rho}+\zeta\right)\sum\limits_{k=1}^n\sqrt{\mathbb{E}[\norm{\lambda^{k}}_2^2]}.
            \end{aligned}
    \end{equation}
    By \Cref{lem:uSCGerror} with \Cref{lem:recursive_geometric}, we get
    \begin{equation*}
        \sqrt{\mathbb{E}[\norm{\lambda^k}_2^2]}
            \leq \frac{\sqrt{2}L_2\rho_2\gamma}{\alpha} + \left(\sqrt{\alpha} + \left(\sqrt{1-\frac{\alpha}{2}}\right)^k\right)\sigma
    \end{equation*}
    which, if we sum from $k=1$ to $n$, gives us
    \begin{equation*}
        \sum\limits_{k=1}^n\sqrt{\mathbb{E}[\norm{\lambda^k}_2^2]}
            \leq n\frac{\sqrt{2}L_2\rho_2\gamma}{\alpha} + \left(n\sqrt{\alpha} + \frac{\sqrt{1-\frac{\alpha}{2}}}{1-\sqrt{1-\frac{\alpha}{2}}}\right)\sigma.
    \end{equation*}
    Plugging this estimate into \Cref{eq:uSCGrate1} gives
    \begin{equation}\label{eq:uSCGfinalineq}
        \begin{aligned}
            \mathbb{E}[\norm{\nabla f(\bar{x}^n)}_2^2]
                &\leq \frac{\mathbb{E}[f(x^{1})-\fmin]}{\rho\gamma n} +\frac{L\rho\gamma}{2} + \frac{1}{n}\left(\frac{\rho_2}{\rho}+\zeta\right)\sum\limits_{k=1}^n\mathbb{E}[\norm{\lambda^{k}}_2]\\
                &\leq \frac{\mathbb{E}[f(x^{1})-\fmin]}{\rho\gamma n} +\frac{L\rho\gamma}{2} + \frac{1}{n}\left(\frac{\rho_2}{\rho}+\zeta\right)\left(n\frac{\sqrt{2}L_2\rho_2\gamma}{\alpha} + \left(n\sqrt{\alpha} + \frac{\sqrt{1-\frac{\alpha}{2}}}{1-\sqrt{1-\frac{\alpha}{2}}}\right)\sigma\right)\\
                &= \frac{\mathbb{E}[f(x^{1})-\fmin]}{\rho\gamma n} +\frac{L\rho\gamma}{2} + \left(\frac{\rho_2}{\rho}+\zeta\right)\left(\frac{\sqrt{2}L_2\rho_2\gamma}{\alpha} + \left(\sqrt{\alpha} + \frac{\sqrt{1-\frac{\alpha}{2}}}{n(1-\sqrt{1-\frac{\alpha}{2}})}\right)\sigma\right).
        \end{aligned}
    \end{equation}
    Finally, by substituting $\gamma = \frac{1}{\sqrt{n}}$ and noting $f(x^{n+1}) \geq \fmin$ we arrive at
    \begin{equation*}
        \begin{aligned}
            \mathbb{E}[\norm{\nabla f(\bar{x}^n)}_{\ast}]
                &\leq \frac{\mathbb{E}[f(x^{1})-\fmin]}{\sqrt{n}\rho} +\frac{L\rho}{2\sqrt{n}} + \left(\frac{\rho_2}{\rho}+\zeta\right)\left(\frac{\sqrt{2}L_2\rho_2}{\alpha\sqrt{n}} + \left(\sqrt{\alpha} + \frac{\sqrt{1-\frac{\alpha}{2}}}{n(1-\sqrt{1-\frac{\alpha}{2}})}\right)\sigma\right)\\
                &= O\left(\frac{1}{\sqrt{n}} + \sigma\right).
        \end{aligned}
    \end{equation*}
\end{appendixproof}

\begin{toappendix}

\subsubsection{Vanishing $\alpha_k$}\label{subsec:uSCGvanishing}

\begin{lemma}[Bound on the gradient error with vanishing $\alpha$]
\label{lem:uSCGerrorbound}
    Suppose \Cref{asm:Lip,asm:stoch} hold. Let $n\in\mathbb{N}^*$ and consider the iterates $\{x_{k}\}_{k=1}^n$ generated by \Cref{alg:uSCG}
    with a constant stepsize $\gamma$ satisfying
    \begin{equation}
        \frac{1}{2 n^{3/4}}<\gamma <\frac{1}{n^{3/4}}.
    \end{equation}
    Moreover, consider momentum which vanishes $\alpha_{k}= \frac{1}{\sqrt{k}}$. Then, for all $k\in\{1,\ldots,n\}$ the following holds
     \begin{equation}
            \mathbb{E}[\norm{\lambda^{k}}_{2}^{2}]\leq \frac{4\sigma^2+8L_2^2\rho_2^2}{\sqrt{k}}.
    \end{equation}
\end{lemma}

\begin{proof}
    Let $k\in\{1,\ldots,n\}$, then by invoking the recursive inequality obtained in \Cref{lem:uSCGerror} for $\mathbb{E}[\norm{\lambda^k}_2^2]$ we have,
    \begin{equation}
        \mathbb{E}[\norm{\lambda^k}^{2}_{2}]\leq \left(1-\frac{\alpha_{k}}{2}\right)\mathbb{E}[\norm{\lambda^{k-1}}^{2}_{2}]+\alpha_{k}^{2}\sigma^{2}+\frac{2L_2^2\rho_2^2\gamma^2}{\alpha_{k}}.
        \end{equation}
        Using the particular choice of $\gamma$ given in the statement of the lemma,
        \begin{equation}
            \frac{1}{2 n^{3/4}}<\gamma <\frac{1}{n^{3/4}},
        \end{equation}
        as well as the choice of $\alpha_k$ and the fact that $n\geq k$, we get
    \begin{align*}
        \mathbb{E}[\norm{\lambda^k}_2^{2}]
            &\leq \bigg(1-\frac{\alpha_{k}}{2} \bigg)\mathbb{E}[\norm{\lambda^{k-1}}_2^{2}]+\alpha_{k}^{2}\sigma^{2}+\frac{2L_2^2\rho_2^2}{\alpha_{k}n^{3/2}}\\
            &\leq \bigg(1-\frac{\alpha_{k}}{2} \bigg)\mathbb{E}[\norm{\lambda^{k-1}}_2^{2}]+\alpha_{k}^{2}\sigma^{2}+\frac{2L_2^2\rho_2^2}{\alpha_{k}k^{3/2}}\\
            &=\bigg(1-\frac{1}{2\sqrt{k}}\bigg)\mathbb{E}[\norm{\lambda^{k-1}}_2^{2}]+\frac{\sigma^{2}}{k}+\frac{2L_2^2\rho_2^2}{k}\\
            &= \bigg(1-\frac{1}{2\sqrt{k}}\bigg)\mathbb{E}[\norm{\lambda^{k-1}}_2^{2}]+\frac{\sigma^{2}+2L_2^2\rho_2^2}{k}.
        \end{align*}
    Then, by applying \Cref{lem:recursivevanishing} with $u^k = \mathbb{E}[\norm{\lambda^k}_2^2]$ and $c=\sigma^2+2L_2^2\rho_2^2$ we readily obtain
    \begin{equation}
        \mathbb{E}[\norm{\lambda^{k}}_{2}^{2}]\leq \frac{4\sigma^2+8L_2^2\rho_2^2}{\sqrt{k}}
    \end{equation}
    since $Q$ as defined in \Cref{lem:recursivevanishing} is given by $Q = \max\{\mathbb{E}[\norm{\lambda^1}_2^2], 4\sigma^2+8L_2^2\rho_2^2\} \leq 4\sigma^2+8L_2^2\rho_2^2$, which concludes our result.
\end{proof}

Combining these results yields our accuracy guarantees for \Cref{alg:uSCG} with vanishing $\alpha_k$, presented in the next lemma.
\end{toappendix}

\begin{lemmarep}[{Convergence rate for \ref{eq:uSCG} with vanishing $\alpha_k$}]
    Suppose that \Cref{asm:Lip,asm:stoch} hold. Let $n\in\mathbb{N}^*$ and consider the iterates $\{x^{k}\}_{k=1}^n$ generated by \Cref{alg:uSCG} with a constant stepsize $\gamma$ satisfying $\frac{1}{2n^{3/4}}<\gamma <\frac{1}{n^{3/4}}$ and vanishing momentum $\alpha_{k}=\tfrac{1}{\sqrt{k}}$. Then, it holds that
    \begin{equation*}
        \mathbb{E}[\|\nabla f(\bar{x}^n)\|_{\ast}] = O\left(\tfrac{1}{n^{1/4}} + \tfrac{L\rho}{n^{3/4}}\right).
    \end{equation*}
\end{lemmarep}
\begin{appendixproof}
    Let $n\in\mathbb{N}^*$, $k\in\{1,\ldots,n\}$; by combining \Cref{lem:uSCGtemplate1} and \Cref{lem:uSCGerrorbound} we have
    \begin{equation}\label{eq:pre_rate}
        \begin{aligned}
            \mathbb{E}[\|\nabla f(\bar{x}^n)\|_{\ast}]
                &\stackrel{\text{\eqref{lem:uSCGtemplate1}}}{\leq} \frac{2\mathbb{E}[f(x^1)-\fmin]}{\rho n^{1/4}} + \frac{2(\rho_2 + \zeta\rho)\sum_{k=1}^n\sqrt{\mathbb{E}[\norm{\lambda^k}_2^2]}}{\rho n} + \frac{L\rho}{n^{3/4}}\\
                &\stackrel{\text{\eqref{lem:uSCGerrorbound}}}{\leq} \frac{2\mathbb{E}[f(x^1)-\fmin]}{\rho n^{1/4}} + \frac{2(\rho_2 + \zeta\rho)\sqrt{4\sigma^2+8L_2^2\rho_2^2}\sum_{k=1}^{n}\frac{1}{k^{1/4}}}{\rho n}  + \frac{L\rho}{n^{3/4}}\\
                &\leq \frac{2\mathbb{E}[f(x^1)-\fmin]}{\rho n^{1/4}} + \frac{2(\rho_2 + \zeta\rho)\sqrt{4\sigma^2+8L_2^2\rho_2^2}\sum_{k=1}^{n}\frac{1}{k^{1/4}}}{\rho n}  + \frac{L\rho}{n^{3/4}}.
        \end{aligned}
    \end{equation}
    Using the integral test and noting that $x\mapsto \tfrac{1}{x^{1/4}}$ is decreasing on $\mathbb{R}_+$, we can upper bound the sum in the right hand side as
    \begin{equation*}
        \sum_{k=1}^{n}\frac{1}{k^{1/4}}\leq 1 + \int_{1}^{n}\frac{1}{x^{3/4}}dx=1+\frac{4}{3}[x^{3/4}]^{n}_1=1+\frac{4}{3}(n^{3/4}-1) = \frac{4}{3}n^{3/4}-\frac{1}{3}\leq \frac{4}{3}n^{3/4}.
    \end{equation*}
    Inserting the above estimation into \eqref{eq:pre_rate} we arrive at
    \begin{align*}
        \mathbb{E}[\|\nabla f(\bar{x}^n)\|_{\ast}] &\leq \frac{2\mathbb{E}[f(x^1)-\fmin]}{\rho n^{1/4}}+ \frac{8 n^{3/4}(\rho_2 + \zeta\rho)\sqrt{4\sigma^2+8L_2^2\rho_2^2}}{3\rho n}  + \frac{L\rho}{n^{3/4}}\\
        &= \frac{2\mathbb{E}[f(x^1)-\fmin]+ \tfrac{8}{3}(\rho_2 + \zeta\rho)\sqrt{4\sigma^2+8L_2^2\rho_2^2}}{\rho n^{1/4}} + \frac{L\rho}{n^{3/4}}\\
        &= O\left(\frac{1}{n^{1/4}}+\frac{L\rho}{n^{3/4}}\right)
    \end{align*}
    which is the claimed result.
\end{appendixproof}

\begin{toappendix}

\subsection{Convergence analysis of \ref{eq:SCG}}\label{subsec:SCG}

In this section we will analyze the worst-case convergence rate of \Cref{alg:SCG}. To do this, we will prove bounds on the expectation of the so-called Frank-Wolfe gap, $\max\limits_{u\in\mathcal{D}} \langle \nabla f(x), x-u\rangle$, which ensures criticality for the constrained optimization problem over $\mathcal{D}$, i.e., for $x^\star\in\mathcal{D}$
\begin{equation*}
    0 = \nabla f(x^\star) + \mathrm{N}_{\mathcal{D}}(x^\star) \iff \max\limits_{u\in\mathcal{D}} \langle \nabla f(x^\star), x^\star-u\rangle \leq 0
\end{equation*}
where $\mathrm{N}_{\mathcal{D}}$ is the normal cone to the set convex $\mathcal{D}$.

This next lemma characterizes the descent of \Cref{alg:SCG} for any stepsize $\gamma$ and momentum $\alpha_k$ in $(0,1]$.
\begin{lemma}[{Nonconvex analog \citet[Lem. 2]{mokhtari2020stochastic}}]
    \label{lem:commondescent}
    Suppose \Cref{asm:Lip} holds.
    Let $n\in\mathbb{N}^*$ and consider the iterates $\{x_k\}_{k=1^n}$ generated by \Cref{alg:SCG} with constant stepsize $\gamma\in(0,1]$.
    Then, for all $k\in\{1,\ldots,n\}$, for all $u\in \mathcal{D}$, it holds
    \begin{equation}
        \gamma \mathbb{E}[\langle \nabla f(x^k), x^k-u\rangle] \leq \mathbb{E}[f(x^k) - f(x^{k+1})] + D_2\gamma \sqrt{\mathbb{E}[\| \lambda^k\|_2^2]} + 2L\rho^2\gamma^2.
    \end{equation}
\end{lemma}
\begin{proof}
    Let $n\in\mathbb{N}^*$, then by \Cref{asm:Lip} we can apply the descent lemma for the function $f$ at the points $x^k$ and $x^{k+1}$ to get, for all $k\in\{1,\ldots,n\}$,
    \begin{equation*}
        \begin{aligned}
            f(x^{k+1})
                &\leq f(x^k) + \langle \nabla f(x^k), x^{k+1}-x^k\rangle + \tfrac{L}{2}\|x^{k+1}-x^k\|^2\\
                &= f(x^k) + \langle d^k, x^{k+1}-x^k\rangle + \langle \lambda^k, x^{k+1}-x^k\rangle + \tfrac{L}{2}\|x^{k+1}-x^k\|^2\\
                &= f(x^k) + \gamma\langle d^k, \lmo(d^k)-x^k\rangle + \gamma \langle \lambda^k, \lmo(d^k)-x^k\rangle + \tfrac{L}{2}\gamma^2\|\lmo(d^k)-x^k\|^2\\
                &\stackrel{\text{(a)}}{\leq} f(x^k) + \gamma\langle d^k, u-x^k\rangle + \gamma \langle \lambda^k, \lmo(d^k)-x^k\rangle + \tfrac{L}{2}\gamma^2\|\lmo(d^k)-x^k\|^2\\
                &= f(x^k) + \gamma\langle -\lambda^k, u-x^k\rangle + \gamma \langle \nabla f(x^k), u-x^k\rangle + \gamma \langle \lambda^k, \lmo(d^k)-x^k\rangle + \tfrac{L}{2}\gamma^2\|\lmo(d^k)-x^k\|^2\\
                &= f(x^k) + \gamma \langle \nabla f(x^k), u-x^k\rangle + \gamma \langle \lambda^k, \lmo(d^k)-u\rangle + \tfrac{L}{2}\gamma^2\|\lmo(d^k)-x^k\|^2\\
                &\stackrel{\text{(b)}}{\leq} f(x^k) + \gamma \langle \nabla f(x^k), u-x^k\rangle + \gamma \langle \lambda^k, \lmo(d^k)-u\rangle + 2L\rho^2\gamma^2,
        \end{aligned}
    \end{equation*}
    using the optimality of $\lmo(d^k)$ for the linear minimization subproblem for (a) and the $2\rho$ upper bound on $\|\lmo(d^k)-x^k\|$ for (b).
    Rearranging and estimating we find, for all $k\in\{1,\ldots,n\}$, for all $u\in\mathcal{D}$,
    \begin{equation*}
        \begin{aligned}
            \gamma\langle \nabla f(x^k),x^k-u\rangle
                &\stackrel{\text{(a)}}{\leq} f(x^k) - f(x^{k+1}) + \gamma \| \lambda^k\|_2 \|\lmo(d^k)-u\|_2 + \tfrac{L}{2}\gamma^2\|\lmo(d^k)-x^k\|^2\\
                &\stackrel{\text{(b)}}{\leq} f(x^k) - f(x^{k+1}) + D_2 \gamma \| \lambda^k\|_2  + 2L\rho^2\gamma^2
        \end{aligned}
    \end{equation*}
    where we have used the Cauchy-Schwarz inequality in (a) and and bounded $\|\lmo(d^k)-x^k\|_2$ using the diameter of the set $\mathcal{D}$ with respect to the Euclidean norm, denoted $D_2$, in (b).
    Taking the expectation of both sides and applying Jensen's inequality we finally arrive, for all $k\in\{1,\ldots,n\}$, for all $u\in\mathcal{D}$,
    \begin{equation*}
        \begin{aligned}
            \gamma\mathbb{E}[\langle \nabla f(x^k),x^k-u\rangle]
                &\leq \mathbb{E}[f(x^k) - f(x^{k+1})] + D_2 \gamma \mathbb{E}[\| \lambda^k\|_2] + 2L\rho^2\gamma^2\\
                &\leq \mathbb{E}[f(x^k) - f(x^{k+1})] + D_2 \gamma \sqrt{\mathbb{E}[\| \lambda^k\|_2^2]} + 2L\rho^2\gamma^2.
        \end{aligned}
    \end{equation*}
\end{proof}

\subsubsection{\ref{eq:SCG} with constant $\alpha$}\label{subsec:SCGconstant}
\begin{lemma}\label{lem:SCGconstanterror}
    Suppose \Cref{asm:Lip,asm:stoch} hold. Let $n\in\mathbb{N}^*$ and consider the iterates $\{x^k\}_{k=1}^n$ generated by \Cref{alg:SCG} with constant stepsize $\gamma=\tfrac{1}{\sqrt{n}}$ and constant momentum $\alpha \in(0,1)$ with the exception of the first iteration, where we take $\alpha=1$. Then we have
    \begin{equation*}
        \mathbb{E}[\norm{\lambda^k}_2^2] \leq 4L_2^2D_2^2\frac{\gamma^2}{\alpha^2} + \left(2\alpha + \left(1-\frac{\alpha}{2}\right)^k\right)\sigma^2.
    \end{equation*}
\end{lemma}
\begin{proof}
    Under \Cref{asm:Lip,asm:stoch}, Lemma 1 in \citet{mokhtari2020stochastic} yields, after taking expectations, for all $k\in\{1,\ldots,n\}$
    \begin{equation*}
        \mathbb{E}[\| \lambda^{k+1}\|_2^2] \leq (1-\frac{\alpha_{k+1}}{2})\mathbb{E}[\| \lambda^k\|_2^2] + \sigma^2\alpha_{k+1}^2 + 2L_2^2D_2^2\frac{\gamma^2}{\alpha_{k+1}}.
    \end{equation*}
    Taking $\gamma$ and $\alpha$ to be constant we get
    \begin{equation*}
        \mathbb{E}[\| \lambda^{k+1}\|_2^2] \leq (1-\frac{\alpha}{2})\mathbb{E}[\| \lambda^k\|_2^2] + \sigma^2\alpha^2 + 2L_2^2D_2^2\frac{\gamma^2}{\alpha}.
    \end{equation*}
    Applying \Cref{lem:recursive_geometric} to the above with $u^k =\mathbb{E}[\| \lambda^{k+1}\|_2^2]$, $\beta = \frac{\alpha}{2}$, and $\eta = \sigma^2\alpha^2 + 2L_2^2D_2^2\frac{\gamma^2}{\alpha}$ we obtain
    \begin{equation*}
        \begin{aligned}
            \mathbb{E}[\norm{\lambda^{k}}_2^2]
                &\leq 2\alpha\sigma^2 + 4L_2^2D_2^2\frac{\gamma^2}{\alpha^2} + \left(1-\frac{\alpha}{2}\right)^k\mathbb{E}[\norm{\lambda^{1}}_2^2]\\
                &\leq 4L_2^2D_2^2\frac{\gamma^2}{\alpha^2} + \left(2\alpha + \left(1-\frac{\alpha}{2}\right)^k\right)\sigma^2
        \end{aligned}
    \end{equation*}
    with the final inequality following by the variance bound in \Cref{asm:stoch}.
\end{proof}

\end{toappendix}

These results show that, in the worst-case, running \Cref{alg:uSCG} with constant momentum $\alpha$ guarantees faster convergence but to a noise-dominated region with radius proportional to $\sigma$. In contrast, running \Cref{alg:uSCG} with vanishing momentum $\alpha_k$ is guaranteed to make the expected dual norm of the gradient small but at a slower rate. \Cref{alg:SCG} exhibits the analogous behavior, as we show next.

Before stating the results for \Cref{alg:SCG}, we emphasize that they are with \emph{constant} stepsize $\gamma$, which is atypical for conditional gradient methods. However, like most conditional gradient methods, we provide a convergence rate on the so-called Frank-Wolfe gap which measures criticality for the constrained optimization problem over $\mathcal{D}$. 

Finally, we remind the reader that the iterates of \Cref{alg:SCG} are always feasible for the set $\mathcal{D}$ by the design of the update and convexity of the norm ball $\mathcal{D}$.
\begin{lemmarep}[{Convergence rate for \ref{eq:SCG} with constant $\alpha$}]
    Suppose \Cref{asm:Lip,asm:stoch} hold. Let $n\in\mathbb{N}^*$ and consider the iterates $\{x^k\}_{k=1}^n$ generated by \Cref{alg:SCG} with constant stepsize $\gamma=\tfrac{1}{\sqrt{n}}$ and constant momentum $\alpha \in(0,1)$. Then, for all $u\in\mathcal{D}$, it holds that
    \begin{equation*}
        \begin{aligned}
            \mathbb{E}[\langle \nabla f(\bar{x}^n), \bar{x}^n-u\rangle] = O\left(\tfrac{L\rho^2}{\sqrt{n}} + \sigma\right).
        \end{aligned}
    \end{equation*}
\end{lemmarep}
\begin{appendixproof}
    Let $n\in\mathbb{N}^*$ and let $k\in\{1,\ldots,n\}$.
    By \Cref{asm:Lip}, we can invoke \Cref{lem:commondescent} to get, for all $k\in\{1,\ldots,n\}$, for all $u\in\mathcal{D}$,
    \begin{equation*}
        \gamma \mathbb{E}[\langle \nabla f(x^k), x^k-u\rangle]
            \leq \mathbb{E}[f(x^k) - f(x^{k+1})] + D_2\gamma \sqrt{\mathbb{E}[\| \lambda^k\|_2^2]} + 2L\rho^2\gamma^2.
    \end{equation*}
    Since \Cref{asm:stoch} holds, we can then invoke \Cref{lem:SCGconstanterror} and apply this to the above. This gives, for all $u\in\mathcal{D}$
    \begin{equation*}
        \begin{aligned}
            \gamma\mathbb{E}[\langle \nabla f(x^k),x^k-u\rangle]
                &\leq \mathbb{E}[f(x^k) - f(x^{k+1})] + 2L\rho^2\gamma^2 + D_2\gamma \sqrt{4L_2^2D_2^2\frac{\gamma^2}{\alpha^2} + \left(2\alpha + \left(1-\frac{\alpha}{2}\right)^k\right)\sigma^2}\\
                &\leq \mathbb{E}[f(x^k) - f(x^{k+1})] + 2L\rho^2\gamma^2 + 2L_2D_2^2\frac{\gamma^2}{\alpha} + D_2\gamma \left(\sqrt{2\alpha} + \left(\sqrt{1-\frac{\alpha}{2}}\right)^k\right)\sigma.
        \end{aligned}
    \end{equation*}
    Summing from $k=1$ to $n$ then dividing by $n\gamma$ we find, for all $u\in\mathcal{D}$,
    \begin{equation}\label{eq:SCGfinalineq}
        \begin{aligned}
            \mathbb{E}[\langle \nabla f(\bar{x}^n), \bar{x}^n-u\rangle]
                &=\frac{1}{n}\sum\limits_{k=1}^n\mathbb{E}[\langle \nabla f(x^k),x^k-u\rangle]\\
                &\stackrel{\text{(a)}}{\leq} \frac{\mathbb{E}[f(x^1) - f(x^{n+1})]}{\gamma n} + 2L\rho^2\gamma + 2L_2D_2^2\frac{\gamma}{\alpha} + D_2 \left(\sqrt{2\alpha} + \frac{1}{n}\sum\limits_{k=1}^n\left(\sqrt{1-\frac{\alpha}{2}}\right)^k\right)\sigma\\
                &\stackrel{\text{(b)}}{\leq} \frac{\mathbb{E}[f(x^1) - f(x^{n+1})]}{\gamma n} + 2L\rho^2\gamma + 2L_2D_2^2\frac{\gamma}{\alpha} + D_2 \left(\sqrt{2\alpha} + \frac{\sqrt{1-\frac{\alpha}{2}}}{n\left(1-\sqrt{1-\frac{\alpha}{2}}\right)}\right)\sigma\\
                &\stackrel{\text{(c)}}{\leq} \frac{\mathbb{E}[f(x^1) - \fmin]}{\gamma n} + 2L\rho^2\gamma + 2L_2D_2^2\frac{\gamma}{\alpha} + D_2 \left(\sqrt{2\alpha} + \frac{\sqrt{1-\frac{\alpha}{2}}}{n\left(1-\sqrt{1-\frac{\alpha}{2}}\right)}\right)\sigma,
        \end{aligned}
    \end{equation}
    applying the subadditivity of the square root for (a), geometric series due to $\sqrt{1-\frac{\alpha}{2}}\in (0,1)$ for (b), and the definition of $\fmin$ for (c).
    Taking $\gamma = \frac{1}{\sqrt{n}}$ then gives the final result, for all $u\in\mathcal{D}$,
    \begin{equation*}
        \begin{aligned}
            \mathbb{E}[\langle \nabla f(\bar{x}^n), \bar{x}^n-u\rangle]
                &\leq \frac{\mathbb{E}[f(x^1) - \fmin]}{\sqrt{n}} + \frac{2L\rho^2}{\sqrt{n}} + \frac{2L_2D_2^2}{\alpha\sqrt{n}} + D_2 \left(\sqrt{2\alpha} + \frac{\sqrt{1-\frac{\alpha}{2}}}{n\left(1-\sqrt{1-\frac{\alpha}{2}}\right)}\right)\sigma
                &= O\left(\frac{L\rho^2}{\sqrt{n}}+\sigma\right).
        \end{aligned}
    \end{equation*}
\end{appendixproof}

\begin{toappendix}
\subsubsection{\ref{eq:SCG} with vanishing $\alpha$}\label{subsec:SCGvanishing}
We now proceed to analyze the convergence of \Cref{alg:SCG} with vanishing $\alpha_k$.
The next lemma provides an estimation on the decay of the second moment of the noise $\lambda^k$.
\begin{lemma}[Bound on the gradient error with vanishing $\alpha$ \Cref{alg:SCG}]\label{lem:SCG_vanishing_error}
    Suppose \Cref{asm:Lip,asm:stoch} hold. Let $n\in\mathbb{N}^*$ and consider the iterates $\{x_{k}\}_{k=1}^n$ generated by \Cref{alg:SCG}
    with a constant stepsize $\gamma$ satisfying
    \begin{equation}
        \frac{1}{2 n^{3/4}}<\gamma <\frac{1}{n^{3/4}}.
    \end{equation}
    Moreover, consider vanishing momentum $\alpha_{k}= \frac{1}{\sqrt{k}}$. Then, for all $k\in\{1,\ldots,n\}$ the following holds
    \begin{equation}
            \mathbb{E}[\norm{\lambda^{k}}_{2}^{2}]\leq \frac{4\sigma^2+8L_2^2D_2^2}{\sqrt{k}}.
    \end{equation}
\end{lemma}
\begin{proof}
    Under \Cref{asm:Lip,asm:stoch}, we have the following recursion from Lemma 1 in \citet{mokhtari2020stochastic} after taking expectations, for all $k\in\mathbb{N}^*$,
    \begin{equation*}
        \mathbb{E}[\| \lambda^{k+1}\|_2^2] \leq (1-\frac{\alpha_{k+1}}{2})\mathbb{E}[\| \lambda^k\|_2^2] + \sigma^2\alpha_{k+1}^2 + 2L_2^2D_2^2\frac{\gamma^2}{\alpha_{k+1}}.
    \end{equation*}
    Comparing with the bound in \Cref{lem:uSCGerrorbound}, we see the only difference is the change of the constant $D_2^2$ by $\rho_2^2$. Repeating the argument in \Cref{lem:uSCGerrorbound}, the desired claim is directly obtained with $D_2^2$ in place of $\rho_2^2$, with the constant $Q = \max\{\mathbb{E}[\norm{\lambda^1}_2^2], 4\sigma^2+8L_2^2D_2^2\} \leq 4\sigma^2+8L_2^2D_2^2$ since $\mathcal{E}[\norm{\lambda^1}_2^2]\leq \sigma^2$ by \Cref{asm:stoch}.
\end{proof}

\end{toappendix}

\begin{lemmarep}[Convergence rate for \ref{eq:SCG} with vanishing $\alpha_k$]\label{lem:frankwolfe_rate}
    Suppose \Cref{asm:Lip,asm:stoch} hold. Let $n\in\mathbb{N}^*$ and consider the iterates $\{x^k\}_{k=1}^n$ generated by \Cref{alg:SCG} with a constant stepsize $\gamma$ satisfying $\tfrac{1}{2n^{3/4}}<\gamma<\tfrac{1}{n^{3/4}}$ and vanishing momentum $\alpha_k = \frac{1}{\sqrt{k}}$. Then, for all $u\in\mathcal{D}$, it holds that
    \begin{equation*}
        \mathbb{E}[\langle \nabla f(\bar{x}^n), \bar{x}^n-u\rangle] = O\left(\tfrac{1}{n^{1/4}} + \tfrac{L\rho^2}{n^{3/4}}\right).
    \end{equation*}
\end{lemmarep}
\begin{appendixproof}
    Let $n\in\mathbb{N}^*$ and $k\in\{1,\ldots,n\}$. By \Cref{asm:Lip}, we can invoke \Cref{lem:commondescent} to get,
    \begin{equation*}
        \begin{aligned}
            \gamma\mathbb{E}[\langle \nabla f(x^k),x^k-u\rangle]
                &\leq \mathbb{E}[f(x^k) - f(x^{k+1})] + D_2 \gamma \sqrt{\mathbb{E}[\| \lambda^k\|_2^2]} + 2L\rho^2\gamma^2.
        \end{aligned}
    \end{equation*}
    Applying the estimate given in \Cref{lem:SCG_vanishing_error} to the above we get
    \begin{equation*}
        \begin{aligned}
            \gamma\mathbb{E}[\langle \nabla f(x^k),x^k-u\rangle]
                &\leq \mathbb{E}[f(x^k) - f(x^{k+1})] + D_2 \gamma \sqrt{\frac{4\sigma^2+8L_2^2D_2^2}{\sqrt{k}}} + 2L\rho^2\gamma^2\\
                &= \mathbb{E}[f(x^k) - f(x^{k+1})] + D_2 \sqrt{4\sigma^2+8L_2^2D_2^2} \gamma \frac{1}{k^{1/4}} + 2L\rho^2\gamma^2.
        \end{aligned}
    \end{equation*}
    Summing from $k=1$ to $n$ and then dividing by $n\gamma$ we find, for all $u\in\mathcal{D}$,
    \begin{equation*}
        \begin{aligned}
            \mathbb{E}[\langle \nabla f(\bar{x}^n),\bar{x}^n-u\rangle]
                &= \frac{1}{n}\sum\limits_{k=1}^n\mathbb{E}[\langle \nabla f(x^k),x^k-u\rangle]\\
                &\stackrel{\text{(a)}}{\leq} \frac{\mathbb{E}[f(x^1) - f(x^{n+1})]}{n\gamma} + \frac{D_2\sqrt{4\sigma^2+8L_2^2D_2^2}}{n}\sum\limits_{k=1}^n\frac{1}{k^{1/4}} + 2L\rho^2\gamma\\
                &\stackrel{\text{(b)}}{\leq} \frac{\mathbb{E}[f(x^1) - f(x^{n+1})]}{n\gamma} + \frac{4D_2\sqrt{4\sigma^2+8L_2^2D_2^2}n^{3/4}}{3n} + 2L\rho^2\gamma\\
                &= \frac{\mathbb{E}[f(x^1) - f(x^{n+1})]}{n\gamma} + \frac{4D_2\sqrt{4\sigma^2+8L_2^2D_2^2}}{3n^{1/4}} + 2L\rho^2\gamma,
        \end{aligned}
    \end{equation*}
    using division by $\gamma n$ for (a) and the integral test with decreasing function $x\mapsto \frac{1}{x^{1/4}}$ for (b).
    Using the definition of $\fmin$ and estimating $n\gamma > \tfrac{n^{1/4}}{2}$ and $\gamma < \frac{1}{n^{3/4}}$ gives
    \begin{equation*}
        \begin{aligned}
            \mathbb{E}[\langle \nabla f(\bar{x}^n),\bar{x}^n-u\rangle]
                &\leq \frac{2\mathbb{E}[f(x^1) - \fmin]}{n^{1/4}} + \frac{4D_2\sqrt{4\sigma^2+8L_2^2D_2^2}}{3n^{1/4}} + \frac{2L\rho^2}{n^{3/4}}\\
                &= O\left(\frac{1}{n^{1/4}} + \frac{L\rho^2}{n^{3/4}}\right).
        \end{aligned}
    \end{equation*}
\end{appendixproof}
\begin{insightbox}[label={insight:convergence}]
For both algorithms, our worst-case analyses for constant momentum suggest that tuning $\alpha$ requires balancing two effects. Making $\alpha$ smaller helps eliminate a constant term that is proportional to the noise level $\sigma$. However, if $\alpha$ becomes too small, it amplifies an $O(1/\sqrt{n})$ term and an $O(\sigma/n)$ term. The stepsize $\gamma$ must also align with the choice of momentum $\alpha$; for vanishing $\alpha_k$ the theory suggests a smaller constant stepsize like $\gamma=\tfrac{3}{4(n^{3/4})}$ to ensure convergence.
\end{insightbox}
\begin{toappendix}

\subsection{Averaged LMO Directional Descent (ALMOND)}\label{subsec:almond}
In this section we present a variation on \Cref{alg:uSCG} that computes the $\lmo$ directly on the stochastic gradient oracle and then does averaging. This is in contrast to how we have presented \Cref{alg:uSCG} which first does averaging (aka momentum) with the stochastic gradient oracle and then computes the $\lmo$. 
A special case of this algorithm is the Normalized SGD based algorithm of \citet{zhao2020stochastic} when the set $\mathcal{D}$ is with respect to the Euclidean norm. 
In contrast with \Cref{alg:uSCG}, the method relies on large batches, since the noise is not controlled by the momentum parameter $\alpha$ due to the bias introduced by the $\lmo$.

\begin{algorithm}
\caption{Averaged LMO directioNal Descent (ALMOND)}
\label{alg:ALMOND}
\textbf{Input:} Horizon $n$, initialization $x^1 \in \mathcal X$, $d^0 = 0$, momentum $\alpha \in (0,1)$, stepsize $\gamma \in (0,1)$
\begin{algorithmic}[1]
    \For{$k = 1, \dots, n$}
        \State Sample $\xi_{k}\sim \mathcal P$
        \State $d^{k} \gets \alpha \lmo(\nabla f(x^{k}, \xi_{k})) + (1 - \alpha)d^{k-1}$
        \State $x^{k+1} \gets x^k + \gamma d^k$
    \EndFor
    \State Choose $\bar{x}^n$ uniformly at random from $\{x^1, \dots, x^n\}$
    \item[\algfont{Return}] $\bar{x}^n$
\end{algorithmic}
\end{algorithm}

\begin{lemmarep}
    Suppose \Cref{asm:Lip,asm:stoch} hold. Let $n\in\mathbb{N}^*$ and consider the iterates $\{x_k\}_{k=1}^n$ generated by \Cref{alg:ALMOND} with stepsize $\gamma = \frac{1}{\sqrt{n}}$. Then, it holds
    \begin{equation*}
        \mathbb{E}[\norm{\nabla f(\bar{x}^n)}_{\ast}] \leq \frac{\mathbb{E}[f(x^1)-\fmin]}{\rho\sqrt{n}} + \frac{L(1-\alpha)\rho}{\alpha\sqrt{n}} + \frac{L\rho}{2\sqrt{n}} + 2\mu\sigma = O\left(\tfrac{1}{\sqrt{n}}\right) + 2\mu\sigma
    \end{equation*}
    where\footnote{Alternatively, instead of invoking the constant $\mu$ we could make an assumption that the gradient oracle has bounded variance measured in the norm $\norm{\cdot}_{\ast}$.} $\mu = \max\limits_{x\in\mathcal{X}}\frac{\norm{x}_\ast}{\norm{x}_{2}}$.
\end{lemmarep}
\begin{proof}
    Let $n\in\mathbb{N}^*$ and denote $z^{k} = \tfrac{1}{\alpha}x^k-\tfrac{1-\alpha}{\alpha}x^{k-1}$ with the convention that $x_0 = x_1$ so that $z_1 = x_1$ and, for all $k\in\{1,\ldots,n\}$,
    \begin{equation*}
        \begin{aligned}
            z^{k+1} - z^k
                &= \frac{1}{\alpha}x^{k+1}-\frac{1-\alpha}{\alpha}x^{k}-\frac{1}{\alpha}x^{k}+\frac{1-\alpha}{\alpha}x^{k-1}= \frac{1}{\alpha}\left(\gamma d^{k} - \gamma (1-\alpha)d^{k-1}\right)= \gamma\lmo(g^k).
        \end{aligned}
    \end{equation*}
    Applying the descent lemma for $f$ at the points $z^{k+1}$ and $z^k$ gives
    \begin{equation}\label{eq:nsgd_descent1}
        \begin{aligned}
            f(z^{k+1})
                &\leq f(z^{k}) + \langle \nabla f(z^k), z^{k+1}-z^k\rangle +\frac{L}{2}\norm{z^{k+1}-z^k}^2\\
                &= f(z^{k}) + \gamma\langle \nabla f(z^k), \lmo(g^k)\rangle +\frac{L\gamma^2}{2}\norm{\lmo(g^k)}^2\\
                &= f(z^{k}) + \gamma\left(\langle \nabla f(z^k)-\nabla f(x^k), \lmo(g^k)\rangle + \langle \nabla f(x^k) - g^k,\lmo(g^k)\rangle +\langle g^k,\lmo(g^k)\rangle\right) +\frac{L\gamma^2}{2}\norm{\lmo(g^k)}^2\\
                &= f(z^{k}) + \gamma\left(\langle \nabla f(z^k)-\nabla f(x^k), \lmo(g^k)\rangle + \langle \nabla f(x^k) - g^k,\lmo(g^k)\rangle -\rho\norm{g^k}_{\ast}\right) +\frac{L\gamma^2}{2}\norm{\lmo(g^k)}^2\\
                &\stackrel{\text{(a)}}{\leq} f(z^{k}) + \gamma\left(\left(\norm{\nabla f(z^k)-\nabla f(x^k)}_{\ast} + \norm{\nabla f(x^k) - g^k}_{\ast}\right)\norm{\lmo(g^k)} -\rho\norm{g^k}_{\ast}\right) +\frac{L\gamma^2}{2}\norm{\lmo(g^k)}^2\\
                &\stackrel{\text{(b)}}{\leq} f(z^{k}) + \gamma\left(\rho\left(\norm{\nabla f(z^k)-\nabla f(x^k)}_{\ast} + \norm{\nabla f(x^k) - g^k}_{\ast}\right) -\rho\norm{g^k}_{\ast}\right) +\frac{L\rho^2\gamma^2}{2}\\
                &\stackrel{\text{(c)}}{\leq} f(z^{k}) + \gamma\left(\rho\left(L\norm{z^k-x^k} + \norm{\nabla f(x^k) - g^k}_{\ast}\right) -\rho\norm{g^k}_{\ast}\right) +\frac{L\rho^2\gamma^2}{2},
        \end{aligned}
    \end{equation}
    applying H\"{o}lder's inequality with norm $\norm{\cdot}_{\ast}$ for (a), the radius $\rho$ of $\mathcal{D}$ for (b), and \Cref{asm:Lip} for (c).
    We note that
    \begin{equation*}
        x^{k+1}-x^{k} = \gamma d^k = \gamma\left((1-\alpha) d^{k-1}+\alpha\lmo(g^k)\right) = \alpha\gamma \lmo(g^k) + (1-\alpha)\gamma\left(\frac{x^k-x^{k-1}}{\gamma}\right)=\alpha\gamma\lmo(g^k)+(1-\alpha)(x^{k}-x^{k-1})
    \end{equation*}
    which we can use to bound
    \begin{equation*}
        \norm{x^{k}-x^{k-1}} \leq (1-\alpha)\norm{x^k-x^{k-1}} + \alpha\gamma\norm{\lmo(g^k)} \leq (1-\alpha)\norm{x^k-x^{k-1}} + \alpha\rho\gamma \leq \frac{\alpha\rho\gamma}{(1-\alpha)}.
    \end{equation*}
    We then have
    \begin{equation*}
        \norm{z^k-x^k} = \frac{(1-\alpha)}{\alpha}\norm{x^k-x^{k-1}}\leq \frac{(1-\alpha)\rho\gamma}{\alpha}
    \end{equation*}
    by using the definition of the update and the $\lmo$, which can be plugged into \eqref{eq:nsgd_descent1} to get
    \begin{equation}
        \begin{aligned}
            \rho\gamma\norm{g^k}_{\ast}
                &\leq f(z^k) - f(z^{k+1}) + \gamma\rho\left(L\norm{z^k-x^k} + \norm{\nabla f(x^k)-g^k}_{\ast}\right) + \frac{L\rho^2\gamma^2}{2}\\
            \implies \norm{g^k}_{\ast}
                &\stackrel{\text{(a)}}{\leq} \frac{f(z^k)-f(z^{k+1})}{\rho\gamma} + L\norm{z^k-x^k} + \norm{\nabla f(x^k)-g^k}_{\ast} + \frac{L\rho\gamma}{2}\\
                &\stackrel{\text{(b)}}{\leq} \frac{f(z^k)-f(z^{k+1})}{\rho\gamma} + \frac{L(1-\alpha)\rho\gamma}{\alpha} + \norm{\nabla f(x^k)-g^k}_{\ast} + \frac{L\rho\gamma}{2}\\
            \implies \norm{\nabla f(x^k)}_{\ast}
                &\stackrel{\text{(c)}}{\leq} \frac{(f(z^k)-f(z^{k+1})}{\rho\gamma} + \frac{L(1-\alpha)\rho\gamma}{\alpha} + 2\norm{\nabla f(x^k)-g^k}_{\ast} + \frac{L\rho\gamma}{2}
        \end{aligned}
    \end{equation}
    where (a) is the result of dividing both sides by $\rho\gamma$, (b) is the result of bounding $\norm{z^k-x^k}$, and (c) follows by the reverse triangle inequality after adding and subtracting $\nabla f(x^k)$ in the norm on the left hand side.
    Taking expectations, using \Cref{asm:stoch} and the constant $\mu = \max\limits_{x\in\mathcal{X}}\frac{\norm{x}_{\ast}}{\norm{x}_2}$, it holds
    \begin{equation*}
        \mathbb{E}[\norm{\nabla f(x^k)-g^k}_{\ast}]\leq \mu\mathbb{E}[\norm{\nabla f(x^k)-g^k}_{2}]\leq \mu\sqrt{\mathbb{E}[\norm{\nabla f(x^k)-g^k}_{2}^2]}\leq \mu\sigma
    \end{equation*}
    which we can sum from $k=1$ to $n$ to obtain
    \begin{equation*}
        \sum\limits_{k=1}^n\mathbb{E}[\norm{\nabla f(x^k)}_{\ast}] \leq \frac{\mathbb{E}[f(z^0)-f(z^{n+1})]}{\rho\gamma} + \frac{nL(1-\alpha)\rho\gamma}{\alpha} + 2n\mu\sigma + \frac{nL\rho\gamma}{2}.
    \end{equation*}
    Diving both sides by $n$ and then plugging in $\gamma = \frac{1}{\sqrt{n}}$ yields the desired final result.
\end{proof}

\subsection{Linear recursive inequalities}
We now present two elementary lemmas that establish bounds for linear recursive inequalities. These results are essential for analyzing the convergence behavior of our stochastic gradient estimator, particularly when examining the error term $\mathbb{E}[\norm{\lambda^k}_2^2]$.
\begin{lemma}[Linear recursive inequality with constant coefficients]\label{lem:recursive_geometric}
    Let $n>1$ and consider $\{u_k\}_{k=1}^n\in\mathbb{R}_+^n$ a sequence of nonnegative real numbers satisfying, for all $k\in\{2,\ldots,n\}$,
    \begin{equation*}
        u^k\leq (1-\beta) u^{k-1} + \eta
    \end{equation*}
    with $\eta>0$ and $\beta\in(0,1)$.
    Then, for all $k\in\{2,\ldots,n\}$, it holds
    \begin{equation*}
        u^k\leq \frac{\eta}{\beta} + (1-\beta)^ku^1.
    \end{equation*}
\end{lemma}
\begin{proof}
    We prove the claim by induction on $k$. For the base case $k=2$ we find
    \begin{equation*}
        u^2 \leq (1-\beta)u^1 + \eta \leq \frac{\eta}{\beta} + (1-\beta)u^1
    \end{equation*}
    since $\beta<1$.
    Assume now for some $k\in\{2,\ldots,n\}$ that the claim holds. Then, by the assumed recursive inequality on $\{u_i\}_{i=1}^n$, we have
    \begin{equation*}
        u^{k+1} \leq (1-\beta)u^k + \eta \leq (1-\beta)\left(\frac{\eta}{\beta} + (1-\beta)^ku^1\right) + \eta = (1-\beta)^{k+1}u^1 + \left(\frac{1-\beta}{\beta} + 1\right)\eta = (1-\beta)^{k+1}u^1 + \frac{\eta}{\beta}
    \end{equation*}
    and thus the desired claim holds by induction.
\end{proof}

The first lemma establishes a geometric decay bound for sequences with constant momentum. The following lemma extends this analysis to the case of variable coefficients, which we will use when we analyze \Cref{alg:uSCG} and \Cref{alg:SCG} with vanishing momentum $\alpha_k$.

\begin{lemma}[Linear recursive inequality with vanishing coefficients]\label{lem:recursivevanishing}   
    Let $\{u^k\}_{k\in\mathbb{N}^*}$ be a sequence of nonnegative real numbers satisfying, for all $k\in\mathbb{N}^*$, the following recursive inequality
    \begin{equation*}
        u^k\leq \left(1-\frac{1}{2\sqrt{k}}\right)u^{k-1} + \frac{c}{k}
    \end{equation*}
    where $c>0$ is constant.
    Then, the sequence $\{u^k\}_{k\in\mathbb{N}^*}$ satisfies, for all $k\in\mathbb{N}^*$,
    \begin{equation*}
        u^k \leq \frac{Q}{\sqrt{k}}
    \end{equation*}
    with $Q=\max\{u^1, 4c\}$.
\end{lemma}
\begin{proof}
    We prove the claim by induction. For $k=1$ the inequality holds by the definition of $Q$, since
    \begin{equation*}
        u^1 \leq Q = \frac{Q}{\sqrt{1}}.
    \end{equation*}
    Let $k>1$ and assume that
    \begin{equation*}
        u^{k-1}\leq\frac{Q}{\sqrt{k-1}}.
    \end{equation*}
    Then, by the assumed recursive inequality for $u^k$, we have
    \begin{equation}\label{eq:recursive_ineq2}
        \begin{aligned}
            u^{k}
                &\leq \left(1-\frac{1}{2\sqrt{k}}\right)u^{k-1} + \frac{c}{k}\\
                &\leq \left(1-\frac{1}{2\sqrt{k}}\right)\frac{Q}{\sqrt{k-1}} + \frac{c}{k}.
        \end{aligned}
    \end{equation}
    Since $k>1$, we can estimate
    \begin{equation*}
        \frac{1}{\sqrt{k-1}} = \frac{\sqrt{k}}{\sqrt{k(k-1)}} = \frac{1}{\sqrt{k}}\sqrt{\frac{k}{k-1}} = \frac{1}{\sqrt{k}}\sqrt{1 + \frac{1}{k-1}} \leq \frac{1}{\sqrt{k}}\left(1 + \frac{1}{2(k-1)}\right)
    \end{equation*}
    which, when applied to \eqref{eq:recursive_ineq2}, gives
    \begin{equation}\label{eq:recursive_ineq3}
        u^k\leq \left(1-\frac{1}{2\sqrt{k}}\right)\left(1+\frac{1}{2(k-1)}\right)\frac{Q}{\sqrt{k}} + \frac{c}{k}.
    \end{equation}
    Furthermore, as $k>1$, we also have
    \begin{equation*}
        \left(1-\frac{1}{2\sqrt{k}}\right)\left(1+\frac{1}{2(k-1)}\right)\leq \left(1-\frac{1}{4\sqrt{k}}\right).
    \end{equation*}
    Applying the above to \eqref{eq:recursive_ineq3} gives
    \begin{equation*}
        \begin{aligned}
            u^k
                &\leq \left(1-\frac{1}{4\sqrt{k}}\right)\frac{Q}{\sqrt{k}}+\frac{c}{k}\\
                &= \frac{Q}{\sqrt{k}} + \frac{c-Q/4}{k}\\
                &\leq \frac{Q}{\sqrt{k}}
        \end{aligned}
    \end{equation*}
    with the last inequality following since $Q\geq 4c$.
    The desired claim is therefore obtained by induction.
\end{proof}

\end{toappendix}


\section{Discussion and Conclusion}
We present RiskHarvester, a risk-based tool to compute a security risk score based on the value of the asset and ease of attack on a database. We calculated the value of asset by identifying the sensitive data categories present in a database from the database keywords. We utilized data flow analysis, SQL, and Object Relational Mapper (ORM) parsing to identify the database keywords. To calculate the ease of attack, we utilized passive network analysis to retrieve the database host information. To evaluate RiskHarvester, we curated RiskBench, a benchmark of 1,791 database secret-asset pairs with sensitive data categories and host information manually retrieved from 188 GitHub repositories. RiskHarvester demonstrates precision of (95\%) and recall (90\%) in detecting database keywords for the value of asset and precision of (96\%) and recall (94\%) in detecting valid hosts for ease of attack. Finally, we conducted an online survey to understand whether developers prioritize secret removal based on security risk score. We found that 86\% of the developers prioritized the secrets for removal with descending security risk scores.

\section{Ethical Considerations and Limitations}

\paragraph{Content Hallucinations and Inaccuracies.} One critical concern is that the model might generate irrelevant or inaccurate interpretations of memes \cite{maynez2020faithfulness, ji2023survey}, which could inadvertently perpetuate stereotypes or biases about certain social groups. This issue is inherent in the use of LMMs in a zero-shot manner, where the model operates without specific training on the task at hand. In our work, we address this challenge by focusing on enhancing explainability behind model decisions, aiming to provide more transparent reasoning for the outputs generated. However, the limitations associated with hallucinations highlight the need for future research to explore more robust approaches, such as retrieval-augmented generation, which could improve the accuracy and relevance of generated interpretations. This would not only enhance the model's performance but also mitigate potential ethical risks associated with the propagation of harmful stereotypes.

\paragraph{Generalisability to New Unseen Memes.} When deploying fine-tuned models for hateful meme detection, a primary ethical concern is their ability to generalize effectively to unseen memes, which can lead to the transfer of domain-specific biases and subsequent misclassification \cite{cao2024modularized}. To address this challenge, our framework employs LMMs to generate meme interpretations in a zero-shot manner. By avoiding fine-tuning for specific domains, these LMMs are less prone to overfitting and perpetuating biases against particular social groups. This approach allows our framework to leverage the strengths of generalized LMMs while minimizing the risk of bias. However, we acknowledge that these generalized models may still harbor inherent biases, presenting ethical risks in the context of automated hateful meme detection. Therefore, ongoing vigilance and evaluation are necessary to ensure that our framework operates equitably and responsibly in real-world applications.

\paragraph{Misuse of Meme Interpretations.} While these interpretations are designed to enhance understanding and assist in content moderation, we acknowledge the risk that they could be misused to create more hateful memes and reinforce social stereotypes. We strongly condemn such actions and want to clarify that we intend to use these interpretations to improve content moderation. We believe that the benefits of generating meme interpretations for this purpose far outweigh any potential risks of misuse. By providing content moderators with deeper insights, we aim to empower them to identify and flag potentially hateful content more effectively, thereby contributing to a more informed and responsible digital environment.

% \paragraph{Resource Demands and Scalability Concerns.} 

% Use \bibliography{yourbibfile} instead or the References section will not appear in your paper
\bibliography{aaai22}

\subsection{Paper Checklist to be included in your paper}

\begin{enumerate}

\item For most authors...
\begin{enumerate}
    \item  Would answering this research question advance science without violating social contracts, such as violating privacy norms, perpetuating unfair profiling, exacerbating the socio-economic divide, or implying disrespect to societies or cultures?
    \answerYes{Yes, our work primarily focuses on utilizing LMMs to analyze and generate interpretations of hateful memes. While these generated interpretations may reflect social stereotypes, our goal is to enhance hateful meme detection systems and improve the understanding of such content.}
  \item Do your main claims in the abstract and introduction accurately reflect the paper's contributions and scope?
    \answerYes{Yes.}
   \item Do you clarify how the proposed methodological approach is appropriate for the claims made? 
    \answerYes{Yes.}
   \item Do you clarify what are possible artifacts in the data used, given population-specific distributions?
    \answerYes{Yes.}
  \item Did you describe the limitations of your work?
    \answerYes{Yes. You may find them under "Ethical Considerations and Limitations" section}
  \item Did you discuss any potential negative societal impacts of your work?
    \answerYes{Yes. You may find them under "Ethical Considerations and Limitations" section}
      \item Did you discuss any potential misuse of your work?
    \answerYes{Yes. You may find them under "Ethical Considerations and Limitations" section}
    \item Did you describe steps taken to prevent or mitigate potential negative outcomes of the research, such as data and model documentation, data anonymization, responsible release, access control, and the reproducibility of findings?
    \answerNA{N/A}
  \item Have you read the ethics review guidelines and ensured that your paper conforms to them?
    \answerYes{Yes.}
\end{enumerate}

\item Additionally, if your study involves hypotheses testing...
\begin{enumerate}
  \item Did you clearly state the assumptions underlying all theoretical results?
    \answerNA{N/A}
  \item Have you provided justifications for all theoretical results?
    \answerNA{N/A}
  \item Did you discuss competing hypotheses or theories that might challenge or complement your theoretical results?
    \answerNA{N/A}
  \item Have you considered alternative mechanisms or explanations that might account for the same outcomes observed in your study?
    \answerNA{N/A}
  \item Did you address potential biases or limitations in your theoretical framework?
    \answerNA{N/A}
  \item Have you related your theoretical results to the existing literature in social science?
    \answerNA{N/A}
  \item Did you discuss the implications of your theoretical results for policy, practice, or further research in the social science domain?
    \answerNA{N/A}
\end{enumerate}

\item Additionally, if you are including theoretical proofs...
\begin{enumerate}
  \item Did you state the full set of assumptions of all theoretical results?
    \answerNA{N/A}
	\item Did you include complete proofs of all theoretical results?
    \answerNA{N/A}
\end{enumerate}

\item Additionally, if you ran machine learning experiments...
\begin{enumerate}
  \item Did you include the code, data, and instructions needed to reproduce the main experimental results (either in the supplemental material or as a URL)?
    \answerYes{The GitHub link can be found in the paper's abstract.}
  \item Did you specify all the training details (e.g., data splits, hyperparameters, how they were chosen)?
    \answerYes{Yes. These information can be found under "Implementation Details" section.}
     \item Did you report error bars (e.g., with respect to the random seed after running experiments multiple times)?
    \answerYes{Yes. These details can be found in Table 3 and 4, where the model performance over multiple seeds have been reported.}
	\item Did you include the total amount of compute and the type of resources used (e.g., type of GPUs, internal cluster, or cloud provider)?
    \answerYes{Yes. These information can be found under "Implementation Details" section.}
     \item Do you justify how the proposed evaluation is sufficient and appropriate to the claims made? 
    \answerYes{Yes. These information can be found under "Experiment Results" section.}
     \item Do you discuss what is ``the cost`` of misclassification and fault (in)tolerance?
    \answerNA{N/A}
  
\end{enumerate}

\item Additionally, if you are using existing assets (e.g., code, data, models) or curating/releasing new assets, \textbf{without compromising anonymity}...
\begin{enumerate}
  \item If your work uses existing assets, did you cite the creators?
    \answerYes{Yes.}
  \item Did you mention the license of the assets?
    \answerNA{N/A.}
  \item Did you include any new assets in the supplemental material or as a URL?
    \answerNA{N/A.}
  \item Did you discuss whether and how consent was obtained from people whose data you're using/curating?
    \answerNA{N/A.}
  \item Did you discuss whether the data you are using/curating contains personally identifiable information or offensive content?
    \answerNA{N/A.}
\item If you are curating or releasing new datasets, did you discuss how you intend to make your datasets FAIR?
\answerNA{N/A.}
\item If you are curating or releasing new datasets, did you create a Datasheet for the Dataset? 
\answerNA{N/A.}
\end{enumerate}

\item Additionally, if you used crowdsourcing or conducted research with human subjects, \textbf{without compromising anonymity}...
\begin{enumerate}
  \item Did you include the full text of instructions given to participants and screenshots?
    \answerNA{N/A.}
  \item Did you describe any potential participant risks, with mentions of Institutional Review Board (IRB) approvals?
    \answerNA{N/A.}
  \item Did you include the estimated hourly wage paid to participants and the total amount spent on participant compensation?
    \answerNA{N/A.}
   \item Did you discuss how data is stored, shared, and deidentified?
   \answerNA{N/A.}
\end{enumerate}

\end{enumerate}

\end{document}
