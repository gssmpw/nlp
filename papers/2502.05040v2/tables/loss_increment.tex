\begin{table}[t!]
    \centering
    \small
    \setlength{\tabcolsep}{3pt}
    \begin{tabular}{@{} cccc cc @{}}
        \hline
        \multicolumn{4}{c}{\fontsize{9}{9}\selectfont Loss Components} & \multicolumn{2}{c}{\fontsize{9}{9}\selectfont SurroundOcc-nuSc \citep{tian2023occ3d}} \\
        \cmidrule(lr){1-4} \cmidrule(lr){5-6}
        Cam & Depth & BEV & Bev Depth & IoU ($\uparrow$) & mIoU ($\uparrow$) \\
        \hline
        \checkmark & & & & 26.3 & 14.3 \\
        \checkmark & \checkmark & & & 26.8 & 15.1 \\
        \checkmark & \checkmark & \checkmark & & 27.2 & 15.6 \\
        \rowcolor{Apricot!20!} \checkmark & \checkmark & \checkmark & \checkmark & \textbf{27.5} & \textbf{16.4} \\
        \hline
    \end{tabular}
    \caption{\textbf{Impact of adding different loss components on 3D semantic occupancy performances.} The architecture used is TPVFormer \cite{huang2023tpv}. Models are trained with different combinations of losses and evaluated on 3D IoU and mIoU. We train the models on 20\% of SurroundOcc-nuScenes \cite{tian2023occ3d}.}
    \label{tab:loss_components_study}
\end{table}

