\section{Conclusion}

In this paper, we pinpoint a key limitation in current cache eviction methods: the reliance on intuitive heuristics of using attention weights to select critical cache entries for eviction. For the first time, we formalize the problem of critical cache entry selection from the perspective of output perturbation and provide a theoretical analysis. Furthermore, we propose a novel algorithm based on constraining output perturbation in the worst-case for critical cache selection, which is then integrated into existing SOTA cache eviction methods. Comprehensive evaluations using 2 cases of Needle-in-a-Haystack test and 16 datasets from Longbench demonstrate that our algorithm improves the performance of advanced cache eviction methods, across different budget constraints. Further empirical analysis also confirms and explains this benefit from the perspective of practical output perturbation: our algorithm consistently yields lower perturbation compared to previous methods that rely solely on attention weights, in various settings.
