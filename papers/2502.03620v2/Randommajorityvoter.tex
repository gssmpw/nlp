\begin{algorithm}[H]
    \caption{Random majority voter $\ah$}\label{alg:Randommajority}
    \begin{algorithmic}[1]
        \State \textbf{Input:} Training sequence $S \in (\cX \times \cY)^{*}$ such that $|S|=6^{k}$ for $k \in \mathbb{N}$, string $r \in ([0:1]^{*})^{*}$, failure parameter $0<\delta<1$, VC-dimension $d.$
        \State \textbf{Output:} $ \sign(\cdot) $ of a majority vote.
        
        \State $m \gets |S|$ \label{alg:Randommajoritysetm}
        \State $k \gets \log_{6}(m)$ \label{alg:Randommajoritysetk}
        \State $l \gets \lceil 16 \cdot 200 \ln{(m/(\delta(d + \ln{\left(1/\delta \right)})))}/9\rceil$ \label{alg:Randommajoritysetl}
        \State $f \gets [0]^{l}$ \label{alg:Randommajoritysetf}
        \State $m'=\sum_{i=1}^{k} m(1/6)^{i}+1$\label{alg:Randommajoritysetsubtraingsequencesamplesize}
        \State $t \gets \lceil 20^{2} \ln{(m')}/2\rceil$ \label{alg:Randommajoritysett}

        \State \textbf{for $i = 1$ \textbf{to} $l$} \label{alg:Randommajorityforloopstart} 
        \State\hspace{0.5cm}\vline\hspace{0.1cm}$\rw_{i} \sim \{1:5\}^{k}$ \label{alg:RandommajoritysamplingS}
        \State\hspace{0.5cm}\vline\hspace{0.1cm}$S_{i} \gets S[g'(\rw_{i})]$ \label{alg:Randommajoritysamplingefficient}
        \State\hspace{0.5cm}\vline\hspace{0.1cm}$f_{\rw_{i}}=\sum_{l=1}^{t}h_{f_{\rw_{i}},l}/t \gets \cA(S_{i}, r)$ \label{alg:Randommajorityrunsadaboostsample}
        \State\hspace{0.5cm}\vline\hspace{0.1cm}$\rz_{i} \sim \{1:t\}$ \label{alg:Randommajoritysamplingindex1}
        \State\hspace{0.5cm}\vline\hspace{0.1cm}$f \gets f + h_{f_{\rw_{i}}, \rz_{i}}$ \label{alg:Randommajorityaddshypothesis}
        \State \textbf{return }{$\sign(f)$.} \label{alg:Randommajorityoutputofhat}
    \end{algorithmic}
\end{algorithm}
Line~\ref{alg:Randommajoritysetm} reads the number of training examples $(x_i,y_i)\in\cX\times  \cY$ in $S.$
Line~\ref{alg:Randommajoritysetl} sets the number of hypotheses in the final majority vote.
Line~\ref{alg:Randommajoritysetf} initializes the majority voter as the constant $0$ function/array of size $ l.$
Line~\ref{alg:Randommajoritysetsubtraingsequencesamplesize} calculates the size of the sub training sequences in $ \cS(S,\emptyset) $, which will equal the training sequence size that $ \cA $ uses in its stopping criteria/number of hypotheses to include in its majority vote.
Line~\ref{alg:Randommajoritysett} equals the number of hypotheses in the output of $\cA$ for each call.
Line~\ref{alg:RandommajoritysamplingS} samples a random vector $\rw_{i}$ of length $k$ with i.i.d. entries uniformly chosen from $\{1:5\}$.
Line~\ref{alg:Randommajoritysamplingefficient} corresponds to sampling the row $S[g'(\rw_{i})]$ from $\cS(S,\emptyset)$, using the method in \cref{samplingfromrows}.
Line~\ref{alg:Randommajorityrunsadaboostsample} runs $\cA$ on the training sequence $S_{i}$ and the string $r$ to get a majority vote, using \cref{lemma:adaboostsampleeasyproperties} to express it as a normalized sum. 
Line~\ref{alg:Randommajoritysamplingindex1} samples a random index $\rz_{i}$ uniformly from $\{1:t\}$, which, combined with Line~\ref{alg:Randommajorityaddshypothesis}, corresponds to uniformly sampling a hypothesis $h_{f_{\rw_{i}},\rz_{i}}$ from the majority vote $f_{\rw_{i}}$ and adding it to $ f $.  
