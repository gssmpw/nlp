\section{Optimality of $\cA(\cS(\rS,\emptyset))$}\label{sec:optimalitya}
In this section, we prove \cref{induktionsteplemmaeasy} as a corollary of the main theorem of this section, \cref{induktionsteplemma}. For this, we need the following lemma, which gives a uniform error bound on majority voters. The proof of this result can be found in \cref{appendixoptimallityofdeterministicprocess}

\begin{restatable}{lemma}{lemone}\label{lem1}
  There exists a universal constant $C>1$ such that for: hypothesis class $\cH$ of VC-dimension $d$, number of voters $ t \in \mathbb{N} $, training sequence size $m\geq d$, distribution $\mathcal{D}$ over $\cX \times \{-1,1\}$, margin $0< \gamma < 1$, and $\xi> 1$, with probability at least $1-\delta$ over $\rS\sim \cD^{m}$, we have for all $f \in \dlh$.
  \begin{align*}
\Pr_{(\rx,\ry)\sim \mathcal{D}}[\ry f(\rx) \leq \gamma] \leq \Pr_{(\rx,\ry)\sim\rS}[\ry f(\rx) \leq \xi \gamma] +  C\sqrt{\frac{2d}{((\xi-1))\gamma)^2 m}} + \sqrt{\frac{2 \ln(2/\delta)}{m}}
\end{align*}

\end{restatable}

We also need the following lemma, which relates the error of $ \cA(\cS(S,T),r) $ to its recursive calls $ \cA(j,T,r) $. The proof of \cref{induktionlemma} can be found in \cref{appendixoptimallityofdeterministicprocess}.

\begin{restatable}{lemma}{induktionlemma}\label{induktionlemma}
  For any sequences $S,T\in (\cX\times \cY)^{*}$ such that $|S|= 6^{k}$ for some $k\geq 1$ and any string $r\in ([0,1]^{*})^{*}$, it holds that 
  \begin{align*}
    &\p_{\rx\sim \cD} \bigg[  
      \sum_{f\in \cA(\cS(S,T),r)} \frac{\ind\{\sum_{i=1}^{t} \ind\{h_{f,i}(\rx)=c(\rx)\}/t\geq 3/4\}}{|\cS(S,T)|}
       <3/4\bigg]
    \\
    \leq \negmedspace \negmedspace  \negmedspace   \negmedspace \negmedspace \max_{\stackrel{j\in\{ 1:5 \}}{\rf\in \cA(\cS(S,T),r) \backslash \cA(j,T,r)}} \negmedspace \negmedspace  \negmedspace \negmedspace \negmedspace \negmedspace \negmedspace  \negmedspace \negmedspace \negmedspace\negmedspace\negmedspace 80&\p_{\rx\sim \cD}
    \bigg[ \sum_{i=1}^{t}  \ind\{h_{\rf,i}(\rx)=c(\rx)\}/t < \frac{3}{4},
             \sum_{\text{\quad\quad}f\in \cA(j,T,r)}           \frac{\ind\{\sum_{i=1}^{t}  \ind\{h_{f,i}(\rx)=c(\rx)\}/t\geq \frac{3}{4}\}}{|\cS(j,T)|}
  <\frac{3}{4}\bigg]
  \end{align*}


\end{restatable}
The last lemma we need in the proof of \cref{induktionsteplemma} is \cref{lemma:adaboostsampleeasyproperties}, which gives the necessary properties of $ \cA $  for proving \cref{induktionsteplemma}. We restate it here for convenience.

\lemmaadaboostsampleeasyproperties*

With the above lemmas in place, we are now ready to give the proof of \cref{induktionsteplemma}.
\begin{theorem}\label{induktionsteplemma}
 There exists a universal constant $\cs$ such that for: Distribution $ \cD $ over $ \cX $, hypothesis class $ \cH $ of VC-dimension $ d $, target concept $ c\in \cH $,  failure probability $ 0<\delta<1 $, training sequence size $m=6^{k}$ for some $k\geq 1$, training sequence $T\in (\cX\times \cY)^{*}$ realizable by $ c $, and string $r\in ([0,1]^{*})^{*}$, it holds with probability at least $1-\delta$ over $\rS\sim \cD_{c}^{m}$ that 
  \begin{align*}
    \p_{\rx\sim \cD} \Bigg[  
      \sum_{\rf\in \cA(\cS(\rS,T),r)} \frac{\ind\{\sum_{i=1}^{t} \ind\{h_{\rf,i}(\rx)=c(\rx)\}/t\geq 3/4\}}{|\cS(\rS,T)|}
       <3/4\Bigg] \leq \cs \frac{d+\ln{\left(1/\delta \right)}}{m}.
  \end{align*}
\end{theorem}

\begin{proof}[Proof of \cref{induktionsteplemma}]
We show that for any $T\in(\cX\times \cY)^{*}$, $0<\delta<1$, $\rS\sim \cD^{m}$ where $m=6^{k}$ for integer $k\geq1$, and any string $r\in([0:1]^{*})^{*}$, with probability at least $1-\delta$ over $\rS$, we have: 
\begin{align}\label{highprobabilityeq3}
  \p_{\rx\sim \cD} \left[  
    \sum_{f\in \cA(\cS(S,T),r)} \frac{\ind\{\sum_{i=1}^{t} \ind\{h_{f,i}(\rx)=c(\rx)\}/t\geq 3/4\}}{|\cS(S,T)|}
     <3/4\right]
  \leq \cs(d+\ln{\left(1/\delta \right)})/m,
\end{align}
for $\cs= \max\{32C^{2}\cdot 960,3840\ln{\left(160 \right)}\}(2\cdot 80\cdot 6 \left(\ln{\left(40 \right)}+1\right))^2$ (where $C\geq 1$ is the universal constant from \cref{lem1}). We will prove this claim by induction. For any $k\leq \log_{6}(\cs)$, we are done as the above left-hand side is at most $1$, and the right side in this case is at least $1$, which concludes the induction base.

Now, for the induction step, let $T\in(\cX\times\cY)^{*}$, $0<\delta<1$, $m=6^{k}$, and $r\in([0:1]^{*})^{*}$. We first use \cref{induktionlemma}, followed by applying the union bound twice, respectively on $j\in\{1,2,3,4,5\}$ and $\cA(\cS(S,T),r) \backslash \cA(j,T,r)=\sqcup_{i=1,i\not=j}^{5}\cA(i,T)$, to obtain that for any $y>0$ we have:  
\begin{align}\label{highprobabilityeq7}
  &\p_{\rS}\left[   
    \p_{\rx\sim \cD} \left[  
      \sum_{f\in \cA(\cS(\rS,T))} \frac{\ind\{\sum_{i=1}^{t} \ind\{h_{f,i}(\rx)=c(\rx)\}/t\geq 3/4\}}{|\cS(S,T)|}
       <3/4\right]> y 
       \right]
    \\
    &\leq 
    \p_{S}\bigg[\negmedspace \negmedspace  \negmedspace   \negmedspace \negmedspace \negmedspace \negmedspace  \negmedspace   \negmedspace \negmedspace \negmedspace \negmedspace  \negmedspace     \max_{\stackrel{j\in\{ 1:5 \}}{\rf\in \cA(\cS(S,T),r) \backslash \cA(j,T,r)}} \negmedspace \negmedspace  \negmedspace \negmedspace \negmedspace \negmedspace \negmedspace  \negmedspace \negmedspace \negmedspace\negmedspace \negmedspace\negmedspace\negmedspace 80\p_{\rx\sim \cD}
    \bigg[\sum_{i=1}^{t}\negmedspace  \ind\{h_{\rf,i}(\rx)=c(\rx)\}/t\negmedspace <\negmedspace \frac{3}{4},\negmedspace
    \negmedspace \negmedspace  \negmedspace  \negmedspace \negmedspace  \negmedspace \negmedspace \sum_{\text{\quad\quad}f\in \cA(j,T,r)} \negmedspace \negmedspace \negmedspace \negmedspace\negmedspace \negmedspace  \negmedspace \negmedspace \negmedspace \negmedspace \frac{\ind\{\sum_{i=1}^{t}\negmedspace  \ind\{h_{f,i}(\rx)=c(\rx)\}/t\geq \frac{3}{4}\}}{|\cS(j,T)|}
    \negmedspace<\negmedspace\frac{3}{4}\bigg]> y
  \bigg]\nonumber
  \\
  &\leq\negmedspace
   \sum_{\stackrel{j,l=1}{l\not=j}}^{5}\p_{\rS_{0},\rS_{j},\rS_{l}}\bigg[ \negmedspace \negmedspace \negmedspace\negmedspace\negmedspace\negmedspace\negmedspace\max_{ \quad \quad\stackrel{}{\rf\in\cA(l,T,r)}} \negmedspace \negmedspace \negmedspace  \negmedspace\negmedspace\negmedspace\negmedspace  \negmedspace  80\p_{\rx\sim \cD}
   \bigg[ \sum_{i=1}^{t}\negmedspace  \ind\{h_{\rf,i}(\rx)=c(\rx)\}/t\negmedspace <\negmedspace \frac{3}{4},\negmedspace
   \negmedspace \negmedspace  \negmedspace  \negmedspace \negmedspace  \negmedspace \negmedspace \sum_{\text{\quad\quad}f\in \cA(j,T,r)} \negmedspace \negmedspace \negmedspace \negmedspace\negmedspace \negmedspace  \negmedspace \negmedspace \negmedspace \negmedspace \frac{\ind\{\sum_{i=1}^{t}\negmedspace  \ind\{h_{f,i}(\rx)=c(\rx)\}/t\geq \frac{3}{4}\}}{|\cS(j,T)|}
   \negmedspace<\negmedspace\frac{3}{4}\bigg]\negmedspace> \negmedspace y\bigg]\nonumber
  \end{align} 

We will now show that for $y=\cs(d+\ln{\left(1/\delta \right)})/m$, the last expression in the above is at most $\delta$, which would conclude the induction step and the proof. We will show this by bounding each term in the sum by $ \delta/20 $. Since the sum contains at most $ 20 $ terms, the claim follows. Thus, from now on, let $y=\cs(d+\ln{\left(1/\delta \right)})/m$, and $ j,l\in\{ 1:5 \} $ $ l\not=j $. 


Let $S_{j}$ be any realization of $\rS_{j}$. We observe that since $|\rS_{0}|=6^{k-1}=m/6,$ and $\cA(j,T,r)=\{\cA(S,r)\}_{S\in\cS(j,T)}$, where we recall that $\cS(j,T)=\cS(\rS_{0},S_{j}\sqcup T),$ the induction hypothesis holds for this call with random training sequence $\rS_0$ (since $ |\rS_{0}|=6^{k-1}=m/6 $ ) in the first argument of $\cS(\cdot,\cdot)$,  training sequence $S_{j}\sqcup T$ in the second argument, and string $r$. We use the induction hypothesis with failure parameter $\delta/\cnn$, i.e., with probability at most $\delta/\cnn$ we have that  $\p_{\rx\sim \cD}[\sum_{f\in \cA(j,T,r)} \ind\{\sum_{i=1}^{t} \ind\{h_{f,i}(\rx)=c(\rx)\}/t\geq 3/4\}/|\cS(j,T)|
<3/4]> 6\cs(d+\ln{(\cnn/\delta )})/m$. We now consider the following disjoint events over $ \rS_{0} $ 
\begin{align*}
 &E_1
  = 
 \Bigg\{  \p_{\rx\sim \cD} \left[\sum_{f\in \cA(j,T,r)} \frac{\ind\{\sum_{i=1}^{t} \ind\{h_{f,i}(\rx)=c(\rx)\}/t\geq 3/4\}}{|\cS(j,T)|}
  < \frac{3}{4} \right]< y/80\Bigg\} 
\\
&E_{2}
 = 
\Bigg\{  y/80
\leq
\p_{\rx\sim \cD}  \left[ \sum_{f\in \cA(j,T,r)}      \frac{\ind\{\sum_{i=1}^{t} \ind\{h_{f,i}(\rx)=c(\rx)\}/t\geq 3/4\}}{|\cS(j,T)|}
 < \frac{3}{4} \right]  \leq  6\cs(d+\ln{\left(\cnn/\delta \right)})/m \Bigg\} 
\\
&E_{3} = \Bigg\{6\cs(d+\ln{\left(\cnn/\delta \right)})/m< \p_{\rx\sim \cD}  \left[\sum_{f\in \cA(j,T,r)} \frac{\ind\{\sum_{i=1}^{t} \ind\{h_{f,i}(\rx)=c(\rx)\}/t\geq 3/4\}}{|\cS(j,T)|}
 < \frac{3}{4} \right] \Bigg\} .
\end{align*} 
Now, using the law of total probability, that $E_{3}$ happens with probability at most $\delta/\cnn$, and that $ \rS_{0} $ and $ \rS_{l} $ are independent, we get:


\begin{align}\label{highprobabilityeq5}
  &\p_{\rS_{0},\rS_{l}}\bigg[  \max_{\stackrel{}{\rf\in\cA(l,T,r)}} \negmedspace 80\p_{\rx\sim \cD}
  \bigg[ \sum_{i=1}^{t}\negmedspace  \ind\{h_{\rf,i}(\rx)=c(\rx)\}/t\negmedspace <\negmedspace \frac{3}{4},
  \negmedspace \negmedspace  \negmedspace  \negmedspace \negmedspace  \negmedspace \negmedspace \negmedspace \negmedspace\sum_{\text{\quad\quad}f\in \cA(j,T,r)} \negmedspace \negmedspace \negmedspace \negmedspace\negmedspace \negmedspace  \negmedspace \negmedspace \negmedspace \negmedspace \frac{\ind\{\sum_{i=1}^{t}\negmedspace  \ind\{h_{f,i}(\rx)=c(\rx)\}/t\geq \frac{3}{4}\}}{|\cS(j,T)|}
  \negmedspace<\negmedspace\frac{3}{4}\bigg]\negmedspace> \negmedspace y\bigg]\nonumber
\\
&\leq
\e_{\rS_{0}}\bigg[ \p_{\rS_{l}}\bigg[    \max_{\stackrel{}{\rf\in\cA(l,T,r)}} \negmedspace    80\p_{\rx\sim \cD}
\bigg[
  \negmedspace \sum_{i=1}^{t}\negmedspace  \ind\{h_{\rf,i}(\rx)=c(\rx)\}/t\negmedspace <\negmedspace \frac{3}{4}\\&\quad\quad\quad\quad\quad\quad\quad\quad\quad\quad\quad\quad,
\negmedspace \negmedspace  \negmedspace  \negmedspace \negmedspace  \negmedspace \negmedspace \negmedspace \negmedspace\sum_{\text{\quad\quad}f\in \cA(j,T,r)} \negmedspace \negmedspace \negmedspace \negmedspace\negmedspace \negmedspace  \negmedspace \negmedspace \negmedspace \negmedspace \frac{\ind\{\sum_{i=1}^{t}\negmedspace  \ind\{h_{f,i}(\rx)=c(\rx)\}/t\geq \frac{3}{4}\}}{|\cS(j,T)|}
\negmedspace<\negmedspace\frac{3}{4}\bigg]\negmedspace>\negmedspace y\bigg] \bigg |  E_1 \bigg]\p_{S_{0}}[E_1]\nonumber
\\
&+
\e_{\rS_{0}}\bigg[ \p_{\rS_{l}}\bigg[    \max_{\stackrel{}{\rf\in\cA(l,T,r)}}\negmedspace    80\p_{\rx\sim \cD}
\bigg[
  \negmedspace \sum_{i=1}^{t}\negmedspace  \ind\{h_{\rf,i}(\rx)=c(\rx)\}/t\negmedspace <\negmedspace \frac{3}{4}\nonumber
  \\
  &\quad \quad\quad\quad\quad\quad\quad\quad\quad\quad\quad\quad,
\negmedspace \negmedspace  \negmedspace  \negmedspace \negmedspace  \negmedspace \negmedspace \negmedspace \negmedspace\sum_{\text{\quad\quad}f\in \cA(j,T,r)} \negmedspace \negmedspace \negmedspace \negmedspace\negmedspace \negmedspace  \negmedspace \negmedspace \negmedspace \negmedspace \frac{\ind\{\sum_{i=1}^{t}\negmedspace  \ind\{h_{f,i}(\rx)=c(\rx)\}/t\geq \frac{3}{4}\}}{|\cS(j,T)|}
\negmedspace<\negmedspace\frac{3}{4}\bigg]\negmedspace>\negmedspace y\bigg] \bigg |  E_2 \bigg]\p_{S_{0}}[E_2]+
\delta/\cnn\nonumber
\end{align}  

Now, if $S_0$ is a realization of $\rS_{0}$ where $ \p_{\rx\sim \cD}[\sum_{f\in \cA(j,T,r)} \ind\{\sum_{i=1}^{t} \ind\{h_{f,i}(\rx)=c(\rx)\}/t\geq 3/4\}/|\cS(j,T)|<\frac{3}{4}
 ]< \cs(d+\ln{\left(1/\delta \right)})/(80m)=y/80$ it follows by monotonicity of measures, and since we had $y=\cs(d+\ln{\left(1/\delta \right)})/m$, that
\begin{align*}%\label{highprobabilityeq4}
  \p_{\rS_{l}}\bigg[   \max_{\stackrel{}{\rf\in\cA(l,T,r)}} \negmedspace 80\p_{\rx\sim \cD}
  \bigg[ \sum_{i=1}^{t}\negmedspace  \ind\{h_{\rf,i}(\rx)=c(\rx)\}/t\negmedspace <\negmedspace \frac{3}{4},
  \negmedspace \negmedspace  \negmedspace  \negmedspace \negmedspace  \negmedspace \negmedspace \negmedspace \negmedspace\sum_{\text{\quad\quad}f\in \cA(j,T,r)} \negmedspace \negmedspace \negmedspace \negmedspace\negmedspace \negmedspace  \negmedspace \negmedspace \negmedspace \negmedspace \frac{\ind\{\sum_{i=1}^{t}\negmedspace  \ind\{h_{f,i}(\rx)=c(\rx)\}/t\geq \frac{3}{4}\}}{|\cS(j,T)|}
  \negmedspace<\negmedspace\frac{3}{4}\bigg]\negmedspace>\negmedspace y\bigg]\negmedspace=\negmedspace0.
\end{align*} 

Since this holds on $E_1$, we get that the first term in \cref{highprobabilityeq5} is zero.  

Now, for realizations $S_0$ of $\rS_0$, which are in $ E_{2}$, i.e., is such that $y/80=\cs(d+\ln{\left(1/\delta \right)})/(80m)\leq \p_{\rx\sim \cD}[\sum_{f\in \cA(j,T,r)} \ind\{\sum_{i=1}^{t} \ind\{h_{f,i}(\rx)=c(\rx)\}/t\geq 3/4\}/|\cS(j,T)|]\leq 6\cs(d+\ln{\left(\cnn/\delta \right)})/m$ (the middle term of \cref{highprobabilityeq5}) using the law of conditional probability (which is well defined as the above probability is now non zero), we get that




  \begin{align}\label{highprobabilityeq6}
    &\max_{\stackrel{}{\rf\in\cA(l,T,r)}} \negmedspace 80\p_{\rx\sim \cD}
  \bigg[ \sum_{i=1}^{t}\negmedspace  \ind\{h_{\rf,i}(\rx)=c(\rx)\}/t\negmedspace <\negmedspace \frac{3}{4},
  \negmedspace \negmedspace  \negmedspace  \negmedspace \negmedspace  \negmedspace \negmedspace \negmedspace \negmedspace\sum_{\text{\quad\quad}f\in \cA(j,T,r)} \negmedspace \negmedspace \negmedspace \negmedspace\negmedspace \negmedspace  \negmedspace \negmedspace \negmedspace \negmedspace \frac{\ind\{\sum_{i=1}^{t}\negmedspace  \ind\{h_{f,i}(\rx)=c(\rx)\}/t\geq \frac{3}{4}\}}{|\cS(j,T)|}
  \negmedspace<\negmedspace\frac{3}{4}\bigg]
      \\  
      &=\negmedspace\negmedspace\negmedspace\negmedspace\max_{\stackrel{}{\rf\in\cA(l,T,r)}} \negmedspace  80\p_{\rx\sim \cD}
      \bigg[
        \negmedspace \sum_{i=1}^{t}\negmedspace  \ind\{h_{\rf,i}(\rx)=c(\rx)\}/t\negmedspace <\negmedspace \frac{3}{4}\bigg |
       \negmedspace  \negmedspace \negmedspace  \negmedspace \negmedspace \negmedspace \negmedspace\sum_{\text{\quad\quad}f\in \cA(j,T,r)} \negmedspace \negmedspace \negmedspace \negmedspace\negmedspace \negmedspace  \negmedspace \negmedspace \negmedspace \negmedspace \frac{\ind\{\sum_{i=1}^{t}\negmedspace  \ind\{h_{f,i}(\rx)=c(\rx)\}/t\geq \frac{3}{4}\}}{|\cS(j,T)|}
       \negmedspace<\negmedspace\frac{3}{4}\bigg]
      \nonumber\\ 
      &\quad\quad\quad\quad\quad\quad\quad\quad\quad\quad\quad\qquad\qquad\qquad\cdot \p_{\rx\sim \cD}\bigg[ \negmedspace \negmedspace\negmedspace \negmedspace \negmedspace\negmedspace \negmedspace \negmedspace\sum_{\text{\quad\quad}f\in \cA(j,T,r)} \negmedspace \negmedspace \negmedspace \negmedspace\negmedspace \negmedspace  \negmedspace \negmedspace \negmedspace \negmedspace \frac{\ind\{\sum_{i=1}^{t}\negmedspace  \ind\{h_{f,i}(\rx)=c(\rx)\}/t\geq \frac{3}{4}\}}{|\cS(j,T)|}\negmedspace<\negmedspace\frac{3}{4}\bigg]
      \nonumber\\
      &\leq \negmedspace\negmedspace\negmedspace\negmedspace
      \max_{\stackrel{}{\rf\in\cA(l,T,r)}} \negmedspace  80\p_{\rx\sim \cD}
      \bigg[
        \negmedspace \sum_{i=1}^{t}\negmedspace  \ind\{h_{\rf,i}(\rx)=c(\rx)\}/t\negmedspace <\negmedspace \frac{3}{4}\bigg |
       \negmedspace  \negmedspace \negmedspace  \negmedspace \negmedspace \negmedspace \negmedspace\sum_{\text{\quad\quad}f\in \cA(j,T,r)} \negmedspace \negmedspace \negmedspace \negmedspace\negmedspace \negmedspace  \negmedspace \negmedspace \negmedspace \negmedspace \frac{\ind\{\sum_{i=1}^{t}\negmedspace  \ind\{h_{f,i}(\rx)=c(\rx)\}/t\geq \frac{3}{4}\}}{|\cS(j,T)|}
       \negmedspace<\negmedspace\frac{3}{4}\bigg]\cdot\frac{6\cs(d+\ln{(\frac{\cnn}{\delta} )})}{m}.\nonumber
    \end{align}
Thus, if we can show that with probability at least $1-\delta/\cff$ over $\rS_{l}$ that 
\begin{align*}
  \max_{\stackrel{}{\rf\in\cA(l,T,r)}} \negmedspace \p_{\rx\sim \cD}
  \bigg[
  \negmedspace \sum_{i=1}^{t}\negmedspace  \ind\{h_{\rf,i}(\rx)=c(\rx)\}/t\negmedspace <\negmedspace \frac{3}{4}\bigg |
   \negmedspace  \negmedspace \negmedspace  \negmedspace \negmedspace \negmedspace \negmedspace\sum_{\text{\quad\quad}f\in \cA(j,T,r)} \negmedspace \negmedspace \negmedspace \negmedspace\negmedspace \negmedspace  \negmedspace \negmedspace \negmedspace \negmedspace \frac{\ind\{\sum_{i=1}^{t}\negmedspace  \ind\{h_{f,i}(\rx)=c(\rx)\}/t\geq \frac{3}{4}\}}{|\cS(j,T)|}
   \negmedspace<\negmedspace\frac{3}{4}\bigg] 
\end{align*}
is at most $(80\cdot 6 \left(\ln{\left(\cnn \right)}+1\right))^{-1}$ for realizations $ S_{0} $ of  $\rS_{0}\in E_{2}$, we get by \cref{highprobabilityeq6} that 
\begin{align*}
  \max_{\stackrel{}{\rf\in\cA(l,T,r)}} \negmedspace   80\p_{\rx\sim \cD}
  \bigg[ \sum_{i=1}^{t}\negmedspace  \ind\{h_{\rf,i}(\rx)=c(\rx)\}/t\negmedspace <\negmedspace \frac{3}{4},
  \negmedspace \negmedspace  \negmedspace  \negmedspace \negmedspace  \negmedspace \negmedspace \negmedspace \negmedspace\sum_{\text{\quad\quad}f\in \cA(j,T,r)} \negmedspace \negmedspace \negmedspace \negmedspace\negmedspace \negmedspace  \negmedspace \negmedspace \negmedspace \negmedspace \frac{\ind\{\sum_{i=1}^{t}\negmedspace  \ind\{h_{f,i}(\rx)=c(\rx)\}/t\geq \frac{3}{4}\}}{|\cS(j,T)|}\negmedspace <\negmedspace \frac{3}{4}=y
  \bigg]
  \\ 
  \leq \cs(d+\ln{\left(1/\delta \right)})/m,
\end{align*}
implying the term in \cref{highprobabilityeq5} conditioned on $ E_{2} $  is at most $\delta/\cff$. Consequently, we have shown that \cref{highprobabilityeq5} happens with probability at most $\delta/\cff+\delta/\cnn$ over $\rS_0$ and $ \rS_{l}$ for any realization $S_{j}$ of $\rS_{j}$. Therefore, by taking expectation with respect to $\rS_{j}$ in \cref{highprobabilityeq5} and $ \rS_{0},\rS_{j},\rS_{l} $ being independent,  we get the same upper bound, which upper bounds each term in \cref{highprobabilityeq7} with probability $ \delta/\cff+\delta/\cnn $. This implies that \cref{highprobabilityeq7} is at most $20(\delta/\cff+\delta/\cnn)=\delta$, which concludes the induction step and the proof. 

Thus, we now show that with probability at least $1-\delta/\cff$ over $ \rS_{l} $ 
\begin{align}\label{highprobabilityeq11}
  \max_{\stackrel{}{\rf\in\cA(l,T,r)}}  \negmedspace    &\p_{\rx\sim \cD}
  \bigg[\sum_{i=1}^{t}\negmedspace  \ind\{h_{\rf,i}(\rx)=c(\rx)\}/t\negmedspace <\negmedspace \frac{3}{4}\bigg |
   \negmedspace  \negmedspace \negmedspace  \negmedspace \negmedspace \negmedspace \negmedspace\sum_{\text{\quad\quad}f\in \cA(j,T,r)} \negmedspace \negmedspace \negmedspace \negmedspace\negmedspace \negmedspace  \negmedspace \negmedspace \negmedspace \negmedspace \frac{\ind\{\sum_{i=1}^{t}\negmedspace  \ind\{h_{f,i}(\rx)=c(\rx)\}/t\geq \frac{3}{4}\}}{|\cS(j,T)|}<\frac{3}{4}
  \bigg] \leq \frac{1}{80\cdot6(\ln{(\cnn )}+1)}
\end{align}
for realizations $ S_{0} $ of $\rS_{0}\in E_{2}$.

To this end, let $ S_{0} $ be a realization of $ \rS_{0} \in E_{2}$. We consider the set $$A=\left\{x\in\cX:\sum_{f\in \cA(j,T,r)} \ind\{\sum_{i=1}^{t} \ind\{h_{f,i}(\rx)=c(\rx)\}/t\geq 3/4\}/|\cS(j,T)|<3/4\right\}.$$ Now, let $\cD(\cdot | A)$ be the conditional probability of $\cD$ restricted to $A$. We can then rewrite the first expression of \cref{highprobabilityeq11} as:
\begin{align}\label{highprobabilityeq13}
  \negmedspace\negmedspace\negmedspace\negmedspace
  &\max_{\stackrel{}{\rf\in\cA(l,T,r)}} 80\p_{\rx\sim \cD}
  \bigg[ \sum_{i=1}^{t}\negmedspace  \ind\{h_{\rf,i}(\rx)=c(\rx)\}/t\negmedspace <\negmedspace \frac{3}{4}\bigg |
   \negmedspace  \negmedspace \negmedspace  \negmedspace \negmedspace \negmedspace \negmedspace\sum_{\text{\quad\quad}f\in \cA(j,T,r)} \negmedspace \negmedspace \negmedspace \negmedspace\negmedspace \negmedspace  \negmedspace \negmedspace \negmedspace \negmedspace \frac{\ind\{\sum_{i=1}^{t}\negmedspace  \ind\{h_{f,i}(\rx)=c(\rx)\}/t\geq \frac{3}{4}\}}{|\cS(j,T)|}
   \negmedspace<\negmedspace\frac{3}{4}\bigg]\nonumber
  \\
  =
  &\max_{\rf\in \cA(l,T,r)} 80\p_{\rx\sim \cD(\cdot| A)}
  \left[
  \sum_{i=1}^{t} \ind\{h_{\rf,i}(\rx)=c(\rx)\}/t< 3/4
 \right]. 
\end{align}
We now consider the training sequence $\rS_{l}\sqcap A=[\rS_{l,i}| \rS_{l,i,1}\in A]$, i.e., training examples in $(x,y)\in\rS_{l}$ where $x\in A$. We first notice that $|\rS_{l}\sqcap A|=\sum_{i=1}^{m/6} \ind\{{\rS_{l,i,1}}\in A\}$ has an expected size of at least $(m/6)\cdot \cs(d+\ln{\left(1/\delta \right)})/(80m)=\cs(d+\ln{\left(1/\delta \right)})/480$, as we have assumed a  $ S_0 $  of $\rS_0\in E_{2}$ such that $\cs(d+\ln{\left(1/\delta \right)})/(80m)\leq \p_{\rx\sim \cD}[\sum_{f\in \cA(j,T,r)} \ind\{\sum_{i=1}^{t} \ind\{h_{f,i}(\rx)=c(\rx)\}/t\geq 3/4\}/|\cS(j,T)|<3/4]$, and we have that $ \rS_{l,i,1}\sim \cD $. Furthermore, since $ |\rS_{l}\sqcap A| $  is a sum of i.i.d. Bernoulli random variables, we have by Chernoffs inequality, and for  $\cs\geq\ln{\left(2\cdot\cff \right)}8\cdot480=\ln{(2\cdot\cff )}3840$, (which holds since $\cs= \max\{32C^{2}\cdot 960,3840\ln{\left(160 \right)}\}(2\cdot 80\cdot 6 \left(\ln{\left(40 \right)}+1\right))^2$) and $d+\ln{\left(1/\delta \right)}\geq1+ \ln{\left(1/\delta \right)}$,  that
\begin{align*}
  \p_{\rS_{l}}\left[|\rS_{l}\cap T|> \cs(d+\ln{\left(1/\delta \right)})/960\right]\leq 1-\exp\left(-\left(\cs(d+\ln{\left(1/\delta \right)})/480\right)/8\right)\leq1- \delta/(2\cdot\cff).
\end{align*}
Thus, by the law of total probability, and using the above, we have
\begin{align}\label{highprobabilityeq10}
  &\p_{\rS_{l}}\bigg[ \max_{\rf\in \cA(l,T,r)} 80\p_{\rx\sim \cD(\cdot| A)}
  \left[
  \sum_{i=1}^{t} \ind\{h_{\rf,i}(\rx)=c(\rx)\}/t< 3/4
 \right]\leq \frac{1}{80\cdot6(\ln{(\cnn )}+1)}\bigg]\nonumber\\ 
  &\leq 
  \p_{\rS_{l}}\bigg[ \max_{\rf\in \cA(l,T,r)} 80\p_{\rx\sim \cD(\cdot| A)}
  \left[
  \sum_{i=1}^{t} \ind\{h_{\rf,i}(\rx)=c(\rx)\}/t< 3/4
 \right]\leq \frac{1}{80\cdot6(\ln{(\cnn )}+1)}\nonumber\\&
 \quad\quad\quad\quad\quad\quad\quad\quad\quad\quad\quad\quad\quad\quad\quad\quad\quad\quad\bigg| |\rS_{l}\sqcap A|> \cs(d+\ln{\left(1/\delta \right)})/960 \Bigg]+ \delta/(2\cdot\cff)
\end{align}
We now use that $1/2=(\sum_{i=1}^{t} \ind\{h_{\rf,i}(\rx)=c(\rx)\}/t+\sum_{i=1}^{t} \ind\{h_{\rf,i}(\rx)\not=c(\rx)\}/t)/2$ and the definition of $ \cD_{c} $ as the distribution over $ (\cX\times \cY) $, where the examples have distribution given by $ \rx,c(\rx) $, with  $ \rx\sim \cD $ to obtain:
\begin{align}\label{highprobabilityeq12}
  &\max_{\rf\in \cA(l,T,r)} \p_{\rx\sim \cD(\cdot | A)}
  \left[
  \sum_{i=1}^{t} \ind\{h_{\rf,i}(\rx)=c(\rx)\}/t< 3/4 \right] 
   \\
  =  &\max_{\rf\in \cA(l,T,r)} \p_{\rx\sim \cD(\cdot | A)}
  \left[
  \left(\sum_{i=1}^{t} \ind\{h_{\rf,i}(\rx)=c(\rx)\}/t-\sum_{i=1}^{t} \ind\{h_{\rf,i}(\rx)\not=c(\rx)\}/t\right)/2< 3/4-1/2 \right] \nonumber
  \\
  =  &\max_{\rf\in \cA(l,T,r)} \p_{\rx\sim \cD(\cdot | A)}
  \left[
  \sum_{i=1}^{t} h_{\rf,i}(\rx)c(\rx)/(2t)< 1/4 \right]\nonumber 
  \\
  =  &\max_{\rf\in \cA(l,T,r)} \negmedspace\p_{\rx\sim \cD(\cdot | A)}
  \left[
  \sum_{i=1}^{t} h_{\rf,i}(\rx)c(\rx)/t< 1/2 \right] =\negmedspace\negmedspace\max_{\rf\in \cA(l,T,r)} \negmedspace\negmedspace\p_{(\rx,\ry)\sim \cD_{c}(\cdot | A)}
  \left[
  \sum_{i=1}^{t} h_{\rf,i}(\rx)\ry/t< 1/2 \right], \nonumber
\end{align}
where $ \cD_{c}(\cdot \mid A) $ is the measure such that $ \p_{\rx\sim \cD(\cdot\mid A)}\left((\rx,\c(\rx))\in B\right) $ for $ B\subset (\cX\times \{  -1,1\} ).$ 
Further, since $\cA\left(l,T,r\right)$ are the hypotheses that \cref{alg:AdaBoostSample} outputs on the training sequences in $\cS(S_{0},\rS_{l}\sqcup T)$ and random string $ r $,  and by \cref{lemma:adaboostsampleeasyproperties} we know that an outputted majority vote $\rf\in \cA\left(l,T\right)$, where $ \rf=\cA(\tilde{\rS},r) $ for $ \tilde{\rS} \in \cS(S_{0},\rS_{l}\sqcup T)$, is such that its in sample $ 3/4 $-margin loss   $\sum_{(x,y)\in \tilde{S}}  
\ind\{\sum_{i=1}^{t} h_{\rf,i}(\rx)c(\rx)/t\leq 3/4\} < 1/|\tilde{S}|$, is zero. Thus as $\rS_{l}\sqcap A$ is a sub training sequence of any $ \tilde{S}\in \cS(S_{0},\rS_{l}\sqcup T) $, we conclude that $ \rf\in  \cA\left(l,T\right)$  also has $ 3/4 $-margin loss equal to zero on $\rS_{l}\sqcap A$. 

We further notice that each example in $\rS_{l}\sqcap A$ has distribution $\cD_{c}(\cdot| A)$. Since we have just concluded that any $\rf\in \cA\left(l,T\right)$ has $ 3/4 $-margin loss equal to zero on $\rS_{l}\sqcap A$, and $\rf\in \dlh$, with $ t=\lceil20^{2}\ln{(|\tilde{S}|)}/2 \rceil $  by \cref{lemma:adaboostsampleeasyproperties} (note that $ |\tilde{S}| $ is the same for any $ \tilde{S} \in \cS(S_{0},\rS_{l}\cup T)$, so they are all in $ \dlh $ ),  we get by invoking \cref{lem1} with $\rS_{l}\sqcap A$, $ \dlh $, $\gamma=1/2$, $\xi=3/2$, and failure probability $\delta/(2\cdot \cff)$, combined with \cref{highprobabilityeq12}, that with probability at least $1-\delta/(2\cdot \cff)$ over $\rS_{l}\sqcap A$,conditioned on $ |\rS_{l}\sqcap A|>\cs(d+\ln{\left(1/\delta \right)})/960 $,  we have that
\begin{align}\label{highprobabilityeq9}
  &\max_{\rf\in \cA(l,T,r)} \p_{\rx\sim \cD(\cdot | A)}
  \left[
  \sum_{i=1}^{t} \ind\{h_{\rf,i}(\rx)=c(\rx)\}/t< 3/4 
  \right]\nonumber\\
  =
  &\max_{\rf\in \cA(l,T,r)} \p_{(\rx,\ry)\sim \cD_{c}(\cdot | A)}
  \left[
  \sum_{i=1}^{t} h_{\rf,i}(\rx)\ry/t< 1/2 \right]\nonumber 
  \\
  \leq 
  &\max_{\rf \in \cA(l,T,r)}\left\{ 
    \p_{\rx\sim \rS_{l}\sqcap A}
  \left[
  \sum_{i=1}^{t} h_{\rf,i}(\rx)c(\rx)/t\leq 3/4 \right] 
  +C\sqrt{\frac{32d}{|\rS_{l}\sqcap A|}}
  +\sqrt{\frac{2\ln{\left(4\cdot \cff/\delta \right)}}{|\rS_{l}\sqcap A|}}
  \right\}\nonumber
  \\
  \leq
  &C\sqrt{\frac{32d}{|\rS_{l}\sqcap A|}}
  +\sqrt{\frac{2\ln{\left(2\cdot \cff/\delta \right)}}{|\rS_{l}\sqcap A|}}.
\end{align}
For $|\rS_{l}\sqcap A|> \cs(d+\ln{\left(1/\delta \right)})/960$, we have
\begin{align*}
  \frac{32d}{|\rS_{l}\sqcap A|}\leq \frac{32\cdot960}{\cs}
 \quad 
 \text{ and }
\quad 
\frac{2\ln{\left(4\cdot \cff/\delta\right)}}{|\rS_{l}\sqcap A|}\leq \frac{3840\ln{\left(4\cdot \cff \right)}}{\cs}.
\end{align*}
Thus, since $\cs= \max\{32C^{2}\cdot 960,3840\ln{\left(160 \right)}\}(2\cdot 80\cdot 6 \left(\ln{\left(40 \right)}+1\right))^2$, that is, $\cs= \max\{32C^{2}\cdot 960,3840\ln{\left(4\cdot \cff \right)}\}(2\cdot 80\cdot 6 \left(\ln{\left(\cnn \right)}+1\right))^2$,  we get that 
\begin{align*}
  C\sqrt{\frac{32d}{|\rS_{l}\sqcap A|}}
  +\sqrt{\frac{2\ln{\left(4\cdot \cff/\delta \right)}}{|\rS_{l}\sqcap A|}}\leq
  (80\cdot 6 \left(\ln{\left(\cnn \right)}+1\right))^{-1}
\end{align*}
Thus, we conclude by \cref{highprobabilityeq10} and \cref{highprobabilityeq9}, combined with the above bound on the last expression in \cref{highprobabilityeq9}, that
\begin{align*}
  \p_{\rS_{l}}\negmedspace\bigg[ \max_{\rf\in \cA(l,T,r)} 80\p_{\rx\sim \cD(\cdot| A)}
  \left[
  \sum_{i=1}^{t} \ind\{h_{\rf,i}(\rx)=c(\rx)\}/t< 3/4
 \right]\leq \frac{1}{80\cdot6(\ln{(\cnn )}+1)}\bigg]\geq 1- \delta/\cff.
\end{align*}
This concludes \cref{highprobabilityeq11} (using \cref{highprobabilityeq13}) and, as argued, \cref{highprobabilityeq6} and  \cref{highprobabilityeq11},  combined upper bounds the term in \cref{highprobabilityeq5} with the condition on $ \rS_{0}\in E_{2} $, by at most $\delta/\cff$. Since the term with $ \rS_{0}\in E_{1} $ and $ \rS_{0}\in E_{3} $ in \cref{highprobabilityeq5}  was upper bounded by respectively $ 0 $ and $ \delta/\cnn $, which implies that \cref{highprobabilityeq5} is bounded by $ \delta/\cff +\delta/\cnn=\delta/20$, we obtain that \cref{highprobabilityeq7} is upper bounded by $ \delta $ as explained, and concludes the induction step and the proof.
\end{proof}

