\section{Proof of \cref{AdaBoostSampleMarginLemma3}}\label{appendixpropertiesadaboossample}
We now provide the proof of \cref{AdaBoostSampleMarginLemma3}. For convenience, we restate it here before proceeding with its proof.

\AdaBoostSampleMarginLemmathree*

\begin{proof}[Proof of \cref{AdaBoostSampleMarginLemma3}]
    To show the above bound on $\sum_{x\in S} \ind\{\sum_{j=1}^{t} h_{f,j}(x)c(x)/t\leq \theta \} $, assuming $\sum \ind_{\alphar_i>0}\geq t$, we first notice that if $\sum \ind_{\alphar_i>0}\geq t$, then $f$ in line Line~\ref{alg:AdaBoostsample:output1} is exactly the voting classifiers of the first $t$ hypotheses returned by the $ \erm $  learner, with margin $\gamma_{f,j}\geq \gamma$. This is ensured by Line~\ref{alg:AdaBoostsamplenonzeroweight}, Line~\ref{alg:AdaBoostSamplealphaset} and Line~\ref{alg:AdaBoostSampleearlystop}, which guarantee that $f$ will never be a combination of more than the first $t$ hypotheses with margins $\gamma_{f,j}\geq \gamma$. 
    
    Moreover, since Line~\ref{alg:AdaBoostSamplezeroweight} sets $\alpha_{i}=0$ for any hypothesis $h_i$ with margin $\gamma_{i}<\gamma$ and sets $\sum_{j=1}^{i}\ind_{\alpha_j>0}=\sum_{j=1}^{i-1}\ind_{\alpha_j>0}  $, the updates in Line~\ref{alg:AdaBoostSampleupdatestart} to Line~\ref{alg:AdaBoostSamplegetweights}, correspond to setting $D_{i+1}=D_{i}$, i.e. restarting a step in AdaBoost until a hypothesis with margin $\gamma_{i}\geq \gamma$ under $D_{i}$ is found or the for loop ends. Further Line~\ref{alg:AdaBoostSamplezeroweight} also in this case sets the counter $\sum_{j=1}^{i}\ind_{\alpha_j>0}=\sum_{j=1}^{i-1}\ind_{\alpha_j>0}$ to what it was in the previous round, i.e. skips this fail round, such that $ f $ will be a combination of $ t $ hypotheses if it is outputted.  
    
    Thus in the case $\sum \ind_{\alphar_i>0}\geq t$ happens $f$ is the outcome of AdaBoost with a fixed learning rate $\alpha_{f,j}=\frac{1}{2}\ln{\left(\frac{1/2+\gamma} {1/2-\gamma}\right)}-\frac{1}{2}\ln{(\frac{1+\theta}{1-\theta} )}$, stopped after $t$ rounds of boosting, within each round having received a hypothesis $ h_{f,j} $ with margin $ \gamma_{f,j}\geq \gamma$. Thus we now analysis it like that, with its $ D_{f,1},\ldots,D_{f,t+1} $ corresponding distributions see Line~\ref{alg:AdaBoostSampleupdatel}, we never explicitly in \cref{alg:AdaBoostSample} define $  D_{f,t+1}$ but it refers to the distribution that $ h_{f,t}$ updates $ D_{f,t} $ to in \cref{alg:AdaBoostSamplegetweights}. The analysis follows steps from \cite{boostingbookSchapireF12}[page 55, page 114]. By Line~\ref{alg:AdaBoostSamplegetweights}, Line~\ref{alg:AdaBoostSamplenormalizing} and Line~\ref{alg:AdaBoostSamplesetDone}, which implies $ D_{f,1}(i)=D_{1}=1/m $ we have that 
    \begin{align*}
      D_{f,t+1}(i)= D_{f,t}(i) \exp{(-\alpha_{f,j}h_{j,t}(S_{i,1})S_{i,2} )}/Z_{f,t} 
      \\
      =\frac{D_{1}(i)\exp{(- \sum_{j=1}^{t}\alpha_{f,j}h_{f,j}(S_{i,1})S_{i,2})}}{\prod_{l=1}^{t}Z_{f,l} }
      \\
      =\frac{\exp{( -\sum_{j=1}^{t}\alpha_{f,j}h_{f,j}(S_{i,1})S_{i,2})}}{(m\cdot\prod_{l=1}^{t}Z_{f,l})} ,
    \end{align*}
 where $ S_{i,1} \in \cX$  and $S_{i,2}  \in\cY$ denote respectively the point and label of the $ i $-th training example. Since $ D_{f,t+1}(i) $ is a probability distribution so sums to $ 1 $, we have that 
 
 \begin{align*}
  \sum_{(x,y)\in S} \frac{\exp{\left( -y\sum_{j=1}^{t} \alpha_{f,j}h_{f,j}(x)\right)}}{m} =\sum_{i=1}^{m}\frac{\exp{\left( -\sum_{j=1}^{t}\alpha_{f,j}h_{f,j}(S_{i,1})S_{i,2}\right)}}{m}= \prod_{j=1}^{t}Z_{f,j}.
 \end{align*} 
 Now, using that by Line~\ref{alg:AdaBoostsamplenonzeroweight} we have for $ j\in\{ 1:t \} $,  that $ h_{f,j} $ has error  $ \epsilon_{f,j}=1/2 -\gamma_{f,j}$ under $ D_{f,j} $, and combining this with Line~\ref{alg:AdaBoostSamplegetweights} and Line~\ref{AdaBoostSamplesetnormalization} we get:
    \begin{align}\label{eq:adaboosttypeargument1}
      Z_{f,j}&=\sum_{i=1}^{m} D_{f,j+1}(i)=\sum_{i=1}^{m}D_{f,j}(i) \exp{(-\alpha_{f,j}h_{f,j}(S_{i,1})S_{i,2} )} \\
      &=\sum_{h_{f,j}(S_{i,1})=S_{i,2}}\negmedspace\negmedspace \negmedspace\negmedspace D_{f,j}(i) \exp{(-\alpha_{f,j}h_{f,j}(S_{i,1})S_{i,2} )} +\negmedspace\negmedspace\negmedspace\negmedspace\sum_{h_{f,j}(S_{i,1})\not=S_{i,2}}\negmedspace\negmedspace \negmedspace\negmedspace D_{f,j}(i) \exp{(-\alpha_{f,j}h_{f,j}(S_{i,1})S_{i,2} )}\nonumber\\
      &=(1/2+\gamma_{f,j})\exp(-\alpha_{f,j})+(1/2-\gamma_{f,j})\exp(\alpha_{f,j})\nonumber
    \end{align}          
    Now using that $ f=\sum_{j=1}^{t} \frac{\alpha_{f,j}h_{f,j}}{\sum_{l=1}^{t}{\alpha_{f,l}}} $ and that $ \alpha_{f,l}= \frac{1}{2}\ln{\left(\frac{1+2\gamma}{1-2\gamma}\right)- \frac{1}{2}\ln{\left(\frac{1+\theta}{1-\theta} \right)}}>0$ since $ \theta< 2\gamma $ and the above relations $\sum_{(x,y)\in S}\exp{\left( -y\sum_{j=1}^{t} \alpha_{f,j}h_{f,j}(x)\right)}/m $ and \cref{eq:adaboosttypeargument1} we get that
      \begin{align*}
    \p_{(\rx,\ry)\sim S}\left[\ry f(\rx)\leq \theta \right]
    &\leq 
    \p_{(\rx,\ry)\sim S}\left[\ry \sum_{j=1}^{t} \alpha_{f,j}h_{f,j}(x)\leq \theta\sum_{j=1}^{t}{\alpha_{f,j}} \right]\\
    &\leq
     \sum_{(x,y)\in S}\exp{\left( \theta\sum_{j=1}^{t}{\alpha_{f,j}}-y\sum_{j=1}^{t} \alpha_{f,j}h_{f,j}(x)\right)}/m 
    \\
    &=\exp{\left( \theta\sum_{j=1}^{t}{\alpha_{f,j}}\right)}\sum_{(x,y)\in S}\exp{\left( -y\sum_{j=1}^{t} \alpha_{f,j}h_{f,j}(x)\right)}/m 
    \\
    &=\exp{\left( \theta\sum_{j=1}^{t}{\alpha_{f,j}}\right)}\prod_{j=1}^{t}Z_{f,j}=
    \prod_{j=1}^{t}Z_{f,j}\exp{\left( \theta{\alpha_{f,j}}\right)}
    \\
    &=\prod_{j=1}^{t} \Big((1/2+\gamma_{f,j})\exp(-\alpha_{f,j})+(1/2-\gamma_{f,j})\exp(\alpha_{f,j})\Big)\exp{\left( \theta{\alpha_{f,j}}\right)}
    \\
    &=
    \prod_{j=1}^{t} \left[(1/2+\gamma_{f,j})e^{(\theta-1)\alpha_{f,j}}+(1/2-\gamma_{f,j})e^{(\theta+1)\alpha_{f,j}}\right].
      \end{align*}
    As $\left(e^{(\theta-1)\alpha_{f,j}}-e^{(\theta+1)\alpha_{f,j}}\right)\leq 0$ since  $ \alpha_{f,l}= \frac{1}{2}\ln{\left(\frac{1+2\gamma}{1-2\gamma}\right)- \frac{1}{2}\ln{\left(\frac{1+\theta}{1-\theta} \right)}}>0$  (we assumed $ \theta< 2\gamma $)  we have that $\gamma_{f,j}\left(e^{(\theta-1)\alpha_{f,j}}-e^{(\theta+1)\alpha_{f,j}}\right)$ is a decreasing function of $\gamma_{f,j}$, which implies since we have $\gamma_{f,j}\geq \gamma$,  that  $\gamma_{f,j}\left(e^{(\theta-1)\alpha_{f,j}}-e^{(\theta+1)\alpha_{f,j}}\right)$ is less than or equal to $\gamma\left(e^{(\theta-1)\alpha_{f,j}}-e^{(\theta+1)\alpha_{f,j}}\right)$ 
    and combining this with the above inequality we get that  
    \begin{align*}
      \p_{(\rx,\ry)\sim S}\left[\ry f(\rx)\leq \theta \right]
      \leq 
      \prod_{j=1}^{t} \left[e^{(\theta-1)\alpha_{f,j}}(1/2+\gamma)+e^{(\theta+1)\alpha_{f,j}}(1/2-\gamma)\right],
        \end{align*}
    as for all $ j\in\{ 1:t \} $, we have by Line~\ref{alg:AdaBoostSamplealphaset}  that $\alpha=\alpha_{f,j}=\frac{1}{2}\ln{\left(\frac{1+2\gamma}{1-2\gamma}\right)- \frac{1}{2}\ln{\left(\frac{1+\theta}{1-\theta} \right)}}>0$ we conclude that
    \begin{align*}
      \p_{(\rx,\ry)\sim S}\left[\ry f(\rx)\leq \theta \right]
      \leq 
      \prod_{j=1}^{t} \left[e^{(\theta-1)\alpha}(1/2+\gamma)+e^{(\theta+1)\alpha}(1/2-\gamma)\right]\\
      =\big(\exp{((\theta-1)\alpha)}(1/2+\gamma)+\exp{((\theta+1)\alpha)}(1/2-\gamma)\big)^{t}.
        \end{align*}
    Further for the values $ \theta=3/4 $ and $ \gamma=9/20 $, which satisfies $ 3/8 =\theta/2 < \gamma=9/20 $  we get by numerical calculations that 
    \begin{align*}
      &\exp{\left(\left(\theta-1\right)\left(\frac{1}{2}\ln{\left(\frac{1+2\gamma}{1-2\gamma}\right)- \frac{1}{2}\ln{\left(\frac{1+\theta}{1-\theta} \right)}}\right)\right)}\left(\frac{1}{2}+\gamma\right)
      \\&+\exp{\left(\left(\theta+1\right)\left(\frac{1}{2}\ln{\left(\frac{1+2\gamma}{1-2\gamma}\right)- \frac{1}{2}\ln{\left(\frac{1+\theta}{1-\theta} \right)}}\right)\right)}\left(\frac{1}{2}-\gamma\right)\\
      &= \exp{\left(\left(\frac{3}{4}-1\right)\left(\frac{1}{2}\ln{\left(\frac{1+2\cdot\frac{9}{20}}{1-2\cdot\frac{9}{20}}\right)- \frac{1}{2}\ln{\left(\frac{1+\frac{3}{4}}{1-\frac{3}{4}} \right)}}\right)\right)}\left(\frac{1}{2}+\frac{9}{20}\right)
      \\&+\exp{\left(\left(\frac{3}{4}+1\right)\left(\frac{1}{2}\ln{\left(\frac{1+2\cdot\frac{9}{20}}{1-2\cdot\frac{9}{20}}\right)- \frac{1}{2}\ln{\left(\frac{1+\frac{3}{4}}{1-\frac{3}{4}} \right)}}\right)\right)}\left(\frac{1}{2}-\frac{9}{20}\right)
      \\&\leq 96/100=24/25
    \end{align*}
    which concludes the proof.
  \end{proof}