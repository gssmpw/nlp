\section{Proof of \cref{samplingfromrows}}\label{appendixefficentoptimalpaclearner}
In this appendix, we provide the proof of \cref{samplingfromrows}, which we restate here for convenience before giving the proof.
\samplingfromrows*

\begin{proof}[Proof of \cref{samplingfromrows}]
    Let $S\in (\cX\times \cY)^{*}$ and $|S|=m=6^{k}$ for $k\geq 1$. Let $\cS(S,T)$ for $T\in (\cX\times \cY)^{*}$ be the matrix whose rows are the sub training sequences produced by $\cS(S,T)$ \cref{alg:Subsample}. The matrix is given recursively by the following equation:  
    \begin{align*}
      \cS(S,T)=
      \begin{bmatrix}
        \cS(S_0,S_{1}\sqcup T)\\
        \cS(S_0,S_{2}\sqcup T)\\
        \cS(S_{0},S_{3}\sqcup T)\\
        \cS(S_{0},S_{4}\sqcup T)\\
        \cS(S_{0},S_{5}\sqcup T)
      \end{bmatrix}.
    \end{align*}
  We first notice that since $|S|=6^{k}$, we get by \cref{alg:Subsample}, Line~\ref{alg:Subsample:recursion} and Line~\ref{alg:Subsample:recursion2}, that   $\cS(S,\emptyset)$ will make 5 recursive calls, each of which will create $5$ new recursive calls, and so on for $k-2$ rounds. Counting from the top call, this totals $k$ recursive calls. At each recursion depth $k$, the only element left in the first argument of $\cS(\cdot,\cdot)$ is the first element of $S$, which, by Line~\ref{alg:Subsample:recursion3}, will, produce only one output per call. Thus the matrix $\cS(S,\emptyset)$ of sub training sequences will contain $5^{k}$ rows. 
  
Let for $ i\in\{  0,\ldots,5\}  $  $S_{i}^{1}$ be the $S_i$ created in \cref{alg:Subsample}, Line~\ref{alg:Subsample:recursion}, for $\cS(S,\emptyset)$. For $j\geq 2$ and $ i\in \{  0,\ldots,5\}  $  let $S_{i}^{j}$ be the $S_i$ created in \cref{alg:Subsample}, Line~\ref{alg:Subsample:recursion}, for $\cS(S_{0}^{j-1},\cdot)$, where $\cdot$ represents any sub training sequence, since Line~\ref{alg:Subsample:recursion} does not depend on the training sequences in the second argument, $S_{i}^{j}$ is well defined. 

  
 Using this notation, we notice by Line~\ref{alg:Subsample:recursion2} that each $w\in \{1:5\}^{k}$ uniquely maps to a training sequence $S_{0}^{k+1}\sqcup(\sqcup_{j=1}^{k} S_{w_{j}}^{j} )$ of $\cS(S,\emptyset)$. Furthermore, since any training sequence in $ \cS(S,\emptyset) $ can be written like this by Line~\ref{alg:Subsample:recursion2} we also get that each $ S'\in \cS(S,\emptyset) $ has a corresponding $ w'\in \{ 1:5 \}^{k} $, such that $ S'= S_{0}^{k+1}\sqcup(\sqcup_{j=1}^{k} S_{w'_{j}}^{j} )$.   
  
Since Line~\ref{alg:Subsample:recursion} ensures that $S_0^{1}=S_0$ is the first $1/6$ of the training examples in $S$ (and similarly in subsequent rounds with $ S_{0}^{j} $ and $ S_{0}^{j-1} $), we conclude that $S_0^{j}=S[\{1:6^{k}/6^{j}\}]$. This implies, by Line~\ref{alg:Subsample:recursion2} (\cref{alg:Subsample} always  recurse on on $S_0^{j-1}$) and  Line~\ref{alg:Subsample:recursion} (\cref{alg:Subsample} always partitions $S_{0}^{j-1}$ in acceding order), that for $w_j\in\{1:5\}$ 
  
  

  \begin{align*}
    S_{w_{j}}^{j}=S_0^{j-1}[\{|(S_0^{j-1}|/6)w_{j}+1:(S_0^{j-1}|/6)(w_{j}+1)\}]\\
    =S_0^{j-1}[(\{6^{k}/6^{j})w_{j}+1:(6^{k}/6^{j})(w_{j}+1)\}]\\
    =\left(S[1:6^{k}/6^{j-1}]\right)[(\{6^{k}/6^{j})w_{j}+1:(6^{k}/6^{j})(w_{j}+1)\}]\\
    =S[\{(6^{k}/6^{j})w_{j}+1:(6^{k}/6^{j})(w_{j}+1)\}].
  \end{align*}
  
  
  Let $g':\{1:5\}^{k}\rightarrow {(\mathbb{R}^{2})}^{k}$ be the mapping which, for $w\in\{1:5\}^{k}$, defined by $g'(w)_{j}=[  w_j6^{k}/6^{j}+1,(w_j+1) 6^{k}/6^{j}]$ for $j=1,\ldots,k$.
  Thus, $ g' $ maps to the endpoint indices of the sub training sequence $S_{0}^{k+1}\sqcup(\sqcup_{j=1}^{k} S_{w_{j}}^{j} )$, which can be realized by using the relation $  S_{w_{j}}^{j}
  =S[\{w_{j}6^{k}/6^{j}+1:(w_{j}+1)6^{k}/6^{j}\}],$ and $g'(w)_{j,1}=w_{j}6^{k}/6^{j}+1$, and $g'(w)_{j,2}=(w_{j}+1)6^{k}/6^{j}$. Therefore, we can determine $S_{0}^{k+1}\sqcup(\sqcup_{j=1}^{k} S_{w_{j}}^{j} )$ using only $g'$, $w$, and $S$. 

  We now argue how this can be done efficiently: First, we read the length of $ S $ to obtain $ m=6^{k}, $ which takes $ O(m) $ operations. We then calculate $g'(w)$ as follows:  since we know $ m =6^{k}$, calculating the first entry of $g'(w)$ involves two divisions by $6$, two multiplications by $w_1$, and two additions of $1$. We also save $6^{k}/6$ so that when calculating the next entry of $g'(w)$, we only need to perform two additional divisions by $6$. Overall, this requires $O(k)=O(\ln{\left(m \right)})$ operations to calculate $g'(w)$. 
  Thus, finding the unique row in $ \cS(S,T) $ that corresponds to a given  $w\in \{ 1:5 \}^{k}$ takes at most $O(m)$ operations using $g'$, $w$, and $S$.

  We will denote $S[g'(w)]$ as the above efficient method for finding the unique row of $\cS(S,\emptyset)$ for a given $ w $  using $g'$, $w$, and $S$. 
  
  Finally, since a given $w\in \{ 1:5 \}^{k}$ corresponds to a unique row of $\cS(S,T)$, and since each $ S'\in\cS(S,T)  $ also has a unique $ w'\in \{ 1:5 \} $, we conclude that $ S[g'(w)] $ is a bijection into the rows of $ \cS(S,\emptyset) $, as claimed, which concludes the proof.    
  \end{proof}
 