\appendix

\section{Additional Survey Details}
\label{app:survey-details}

\subsection{Survey Questions}\label{sssec:a-questions}
In our survey we ask participants to read one or more descriptions of AI use cases and to make two judgments: 1) ``Do you think a technology like this should exist?'' (Q1) and 2) ``If the <\textit{use case}> exists, would you use its services?'' (Q5). A question to indicate their level of confidence is asked following each question (Q5, Q6). The participants are asked to both elaborate on their decisions (Q3) and specify the conditions under which they would switch their decisions (Q4). The detailed wordings for the questions are shown in Table \ref{app:part-1-questions}.

Following the main use case questions in both the main and second study, we also asked participants questions about their demographics and literacy levels in AI, and the questions can be found in Table \ref{app:demographic-questions} and \ref{app:ai-literacy-questions} respectively. 

Lastly, while not included in the main text, we asked participants 3 questionnaires about decision making styles to explore the relationship between several decision making styles and the actual decisions of the participants. These included: (1) Oxford Utilitarianism Scale, (2) Toronto Empathy Questionnaire and (3) Moral Foundations Questionnaire. The decision making style questions can be found in Table \ref{app:util-questions}, \ref{app:empathy-questions} and \ref{app:mfq} respectively.

\begin{table}[!hbpt]
    \footnotesize
    \centering
    \begin{tabularx}{\linewidth}{l|X|l}
    \toprule
         Question ID & Question & Answer Type \\
         \midrule
         \multicolumn{3}{l}{AI Perception Question (Before)}\\
         \midrule
         AI Perception Before & Overall, how does the growing presence of artificial intelligence (AI) in daily life and society make you feel? & 5 Point Likert Scale\\
         \midrule
         \multicolumn{3}{l}{Part 1 Specific Questions}\\
         \midrule
         UCX-1 & Do you think a technology like this should be developed? & Yes/No\\
         UCX-2 & How confident are you in your above answer? & 5 Point Likert Scale \\
         UCX-3Y & Please complete the following: [Use Case] should be developed because... & Text \\
         UCX-3N & Please complete the following: [Use Case] should not be developed because... & Text \\
         UCX-4Y & Under what circumstances would you switch your decision from [UCX-2 Answer] should be developed to should not be developed? & Text \\
         UCX-4N & Under what circumstances would you switch your decision from [UCX-2 Answer] should not be developed to should be developed? & Text \\ 
         UCX-5 & If [Use Case] exists, would you ever use its services (answer yes, even if you think you would use it very infrequently)? & Yes/No \\
         UCX-6 & How confident are you in your above answer? & 5 Point Likert Scale \\
         \midrule
         \multicolumn{3}{l}{AI Perception Question (After)}\\
         \midrule
         AI Perception After & Before we continue, we’d like to get your thoughts on AI one more time. Overall, how does the growing presence of artificial intelligence (AI) in daily life and society make you feel? & 5 Point Likert Scale \\
    \bottomrule
    \end{tabularx}
    \caption{Main Study Specific Question, the "X" in Question IDs is a placeholder for the use case number, which ranges from 1 to 5, for the 5 use cases in the jobs and personal use cases respectively.}
    \label{app:part-1-questions}
\end{table}
\begin{table}[!hbpt]
    \footnotesize
    \centering
    \begin{tabularx}{\linewidth}{l|X}
    \toprule
         Question ID & Question  \\
         \midrule
         D-Q1 & How old are you? \\
         D-Q2 & Choose one or more races that you consider yourself to be \\
         D-Q3 & Do you identify as transgender? \\
         D-Q4 & How would you describe your gender identity? \\
         D-Q5 & How would you describe your sexual orientation? \\
         D-Q6 & What is your present religion or religiosity, if any? \\
         D-Q7 & In general, would you describe your political views as… \\
         D-Q8 & What is the highest level of education you have completed? \\
         D-Q9 & In which country have you lived in the longest? \\
         D-Q10 & What other countries have you lived in for at least 6 months? \\
         D-Q11 & Which of the following categories best describe your employment status? \\
         D-Q12 & How would you describe the industry your job would be in? (Select all that apply) \\
         D-Q13 & Do you identify with any minority, disadvantaged, demographic, or other specific groups? If so, which one(s)? (E.g., racial, gender identity, sexuality, disability, immigrant, veteran, etc.); use commas to separate groups. \\
         D-Q14 & (Optional) What are some things that you are most concerned about lately? \\
         $D-Q15_1$ & In your day-to-day life how often have any of the following things happened to you? You are treated with less courtesy or respect than other people \\
         $D-Q15_2$ & In your day-to-day life how often have any of the following things happened to you? You receive poorer service than other people at restaurants or stores \\
         $D-Q15_3$ & In your day-to-day life how often have any of the following things happened to you? People act as if they think you are not smart\\
         $D-Q15_4$ & In your day-to-day life how often have any of the following things happened to you? People act as if they are afraid of you\\
         $D-Q15_5$ & In your day-to-day life how often have any of the following things happened to you? You are threatened or harassed\\
         D-Q16 & If the answer to Q15 is ''A few times a year'' or more frequently to at least one of the statements, what do you think is the main reason for these experiences? (Select all that apply) \\
    \bottomrule
    \end{tabularx}
    \caption{Demographic Questions}
    \label{app:demographic-questions}
\end{table}
\begin{table}[!hbpt]
    \footnotesize
    \centering
    \begin{tabularx}{\linewidth}{l|X}
    \toprule
         Question ID & Question\\
         \midrule
         AI-Q1 & I can identify the AI technology employed in the applications and products I use. \\
         AI-Q2 & I can skillfully use AI applications or products to help me with my daily work. \\
         AI-Q3 & I can choose the most appropriate AI application or product from a variety for a particular task. \\
         AI-Q4 & I always comply with ethical principles when using AI applications or products. \\
         AI-Q5 & I am never alert to privacy and information security issues when using AI applications or products. \\
         AI-Q6 & I am always alert to the abuse of AI technology. \\
         AI-Q7 & How frequently do you use generative AI (i.e., artificial intelligence that is capable of producing high quality texts, images, etc. in response to prompts) products such as ChatGPT, Bard, DALL·E 2, Claude, etc.? \\
         AI-Q8 & How familiar are you with limitations and shortcomings of generative AI? \\
    \bottomrule
    \end{tabularx}
    \caption{AI Literacy Questions. The questions are on a 7 point likert scale ranging from Strongly disagree to Neutral to Strongly agree}
    \label{app:ai-literacy-questions}
\end{table}
\begin{table}[!hbpt]
    \footnotesize
    \centering
    \begin{tabularx}{\linewidth}{l|X}
    \toprule
         Question ID & Question \\
         \midrule
         Util1 & If the only way to save another person’s life during an emergency is to sacrifice one’s own leg, then one is morally required to make this sacrifice. \\
         Util2 & It is morally right to harm an innocent person if harming them is a necessary means to helping several other innocent people. \\
         Util3 & From a moral point of view, we should feel obliged to give one of our kidneys to a person with kidney failure since we don’t need two kidneys to survive, but really only one to be healthy. \\
         Util4 & If the only way to ensure the overall well-being and happiness of the people is through the use of political oppression for a short, limited period, then political oppression should be used. \\
         Util5 & From a moral perspective, people should care about the well-being of all human beings on the planet equally; they should not favor the well-being of people who are especially close to them either physically or emotionally \\
         Util6 & It is permissible to torture an innocent person if this would be necessary to provide information to prevent a bomb going off that would kill hundreds of people. \\
         Util7 & It is just as wrong to fail to help someone as it is to actively harm them yourself. \\
         Util8 & Sometimes it is morally necessary for innocent people to die as collateral damage if more people are saved overall. \\
         Util9 & It is morally wrong to keep money that one doesn’t really need if one can donate it to causes that provide effective help to those who will benefit a great deal. \\
    \bottomrule
    \end{tabularx}
    \caption{Utilitarianism Questions. The questions are on a 7-point likert scale ranging from 1 (Strongly Disagree) to 7 (Strongly Agree)}
    \label{app:util-questions}
\end{table}
\begin{table}[!hbpt]
    \footnotesize
    \centering
    \begin{tabularx}{\linewidth}{l|X}
    \toprule
         Question ID & Question \\
         \midrule
         Empathy1 & When someone else is feeling excited, I tend to get excited too. \\
         Empathy2 & Other people’s misfortunes do not disturb me a great deal. \\
         Empathy3 & It upsets me to see someone being treated disrespectfully. \\
         Empathy4 & I remain unaffected when someone close to me is happy. \\
         Empathy5 & I enjoy making other people feel better. \\
         Empathy6 & I have tender, concerned feelings for people less fortunate than me. \\
         Empathy7 & When a friend starts to talk about his/her problems, I try to steer the conversation towards something else. \\
         Empathy8 & I can tell when others are sad even when they do not say anything. \\
         Empathy9 & I find that I am “in tune” with other people’s moods. \\
         Empathy10 & I do not feel sympathy for people who cause their own serious illnesses. \\
         Empathy11 & I become irritated when someone cries. \\
         Empathy12 & I am not really interested in how other people feel. \\
         Empathy13 & I get a strong urge to help when I see someone who is upset. \\
         Empathy14 & When I see someone being treated unfairly, I do not feel very much pity for them. \\
         Empathy15 & I find it silly for people to cry out of happiness. \\
         Empathy16 & When I see someone being taken advantage of, I feel kind of protective towards him/her. \\
    \bottomrule
    \end{tabularx}
    \caption{Empathy Questions. The questions are on a 5 point likert scale ranging from Never to Always.}
    \label{app:empathy-questions}
\end{table}
\begin{table}[!hbpt]
    \footnotesize
    \centering
    \begin{tabularx}{\linewidth}{l|X}
    \toprule
         Question ID & Question \\
         \midrule
         \multicolumn{2}{X}{Moral Foundation Questionnaire (First Half)\newline When you decide whether something is right or wrong, to what extent is the following consideration relevant to your thinking?}\\
         \midrule
         MFQ 1 & Whether or not someone suffered emotionally \\
         MFQ 2 & Whether or not some people were treated differently than others\\
         MFQ 3 & Whether or not someone’s action showed love for his or her country \\
         MFQ 4 & Whether or not someone showed a lack of respect for authority \\
         MFQ 5 & Whether or not someone violated standards of purity and decency \\
         MFQ 6 & Whether or not someone was good at math \\
         MFQ 7 & Whether or not someone cared for someone weak or vulnerable \\
         MFQ 8 & Whether or not someone acted unfairly \\
         MFQ 9 & Whether or not someone did something to betray his or her group \\
         MFQ 10 & Whether or not someone conformed to the traditions of society \\
         MFQ 11 & Whether or not someone did something disgusting \\
         \midrule
         \multicolumn{2}{l}{Moral Foundation Questionnaire (Second Half)}\\
         \midrule
         MFQ 12 & Compassion for those who are suffering is the most crucial virtue.\\
         MFQ 13 & When the government makes laws, the number one principle should be ensuring that everyone is treated fairly.\\
         MFQ 14 & I am proud of my country’s history.\\
         MFQ 15 & Respect for authority is something all children need to learn.\\
         MFQ 16 & People should not do things that are disgusting, even if no one is harmed.\\
         MFQ 17 & It is better to do good than to do bad.\\
         MFQ 18 & One of the worst things a person could do is hurt a defenseless animal.\\
         MFQ 19 & Justice is the most important requirement for a society.\\
         MFQ 20 & People should be loyal to their family members, even when they have done something wrong.\\
         MFQ 21 & Men and women each have different roles to play in society.\\
         MFQ 22 & I would call some acts wrong on the grounds that they are unnatural.\\
    \bottomrule
    \end{tabularx}
    \caption{Moral Foundation Questionnaire: 20 Questions. The first part of the questionnaire consists of 11 questions on a 6-point likert scale ranging from 0 (Not At All Relevant) to 5 (Extremely Relevant). The second part of the questionnaire consists of 11 questions on a 6-point likert scale ranging from 0 (Strongly Disagree) to 5 (Strongly Agree). Note: Questions MFQ 6 and 17 are meant to catch participants that are not answering the questionnaire properly and are not included in the MFQ score calculation.}
    \label{app:mfq}
\end{table}

\subsection{Participant Details}
\label{app:participant-details}
The demographics of the participants for our study is shown in Tables \ref{app:demographics-1-jobs-p1} to \ref{app:demographics-3-personal-p2}. There was a fairly balanced distribution of participants across the different age groups, although there was a slightly higher proportion of participants in the 25-34 years old and 45-54 years old age ranges. In terms of racial distribution, there were more White/Caucasian participants compared to the other races. The gender distribution was relatively balanced in terms of males vs non-males. The participants were mostly employed or looking for work and a majority of them also had at least some form of college education. Most participants identified as liberal in terms of political leaning. Participants' AI literacy scores are shown in Table~\ref{tab:ai-literacy-distr} and AI Ethics score are shown in Table~\ref{tab:ai-ethics-distr}.

Participants were allocated 5 use cases from one of the scenarios and the allocation between the 2 scenarios are well-balanced and can be found in Table \ref{tab:study1-use-case}.
\begin{table}[!hbpt]
    \centering
    \small
    \begin{tabularx}{0.4\linewidth}{l|X}
    \toprule
       Use Case & Participants Allocated \\
    \midrule
       Personal Use Cases & \\
    \midrule
       Digital Medical Advice & \multirow{5}{*}{97} \\
       Customized Lifestyle Coach & \\
       Personal Health Research & \\
       Nutrition Optimizer & \\
       Flavorful Swaps & \\
    \midrule
       Labor Replacement Use Cases & \\
    \midrule
       Lawyer & \multirow{5}{*}{100} \\
       Elementary School Teacher & \\
       IT Support Specialist & \\
       Government Eligibility Interviewer & \\
       Telemarketer & \\
    \bottomrule
    \end{tabularx}
    \caption{Participant allocation to each category of scenarios.}
    \label{tab:study1-use-case}
\end{table}

\begin{table*}[!htpb]
    \footnotesize
    \centering
    \begin{tabular}{{ll|ll|ll|ll}}
    \toprule
         \textbf{Racial Identity} & \textbf{\textit(N) (\%)} & \textbf{Age} & \textbf{\textit{N} (\%)} & \textbf{Gender Identity} & \textbf{\textit{N} (\%)} & \textbf{Education} & \textbf{\textit{N} (\%)} \\
         % \hline
         \midrule
        White or Caucasian & 33 (33.0) & 18-24 & 11 (11.0) & Man & 49 (49.0) & Bachelor’s degree & 36 (36.0)\\
        Black or African American & 23 (23.0) & 45-54 & 27 (27.0) & Non-male & 51 (51.0) & Graduate degree$^{*}$ & 18 (18.0)\\
        Asian & 21 (21.0) & 25-34 & 22 (22.0) & & & Some college $^{*}$ & 17 (17.0)\\
        Mixed & 13 (13.0) & 55-64 & 20 (20.0) & & & High school diploma$^{*}$ & 16 (16.0)\\
        Other & 10 (10.0) & 35-44 & 14 (14.0) & & & Associates degree$^{*}$ & 13 (13.0)\\
         & 2 (0.7) & 65+ & 6 (6.0) & & & Some high school$^{*}$ & 0 (0.0)\\
    \bottomrule
    \end{tabular}
    \caption{Labor Replacement Study 1 Survey: Racial, age, gender identities and education level of participants. Asterisk (*) denotes labels shortened due to space.}
    \label{app:demographics-1-jobs-p1}
\end{table*}
\begin{table*}[htpb]
    \centering
    \footnotesize
    \begin{tabular}{ll|ll|ll|ll}
    \toprule
         \textbf{Minority/Disadvantaged Group} & \textbf{\textit(N) (\%)} & \textbf{Transgender} & \textbf{\textit{N} (\%)} & \textbf{Sexuality} & \textbf{\textit{N} (\%)} & \textbf{Political Leaning} & \textbf{\textit{N} (\%)} \\
         % \hline
         \midrule
No & 68 (68.0) & No & 97 (97.0) & Heterosexual & 78 (78.0) & Liberal & 34 (34.0)\\
Yes & 32 (32.0) & Yes & 2 (2.0) & Others & 22 (22.0) & Moderate & 23 (23.0)\\
 &  & Prefer not to say & 1 (1.0) & & & Strongly liberal & 20 (20.0)\\
 &  &  &  &  &  & Conservative & 18 (18.0)\\
 &  &  &  &  &  & Strongly conservative & 4 (4.0)\\
 &  &  &  &  &  & Prefer not to say & 1 (1.0)\\
\bottomrule
    \end{tabular}
    \caption{Labor Replacement Study 1 Survey: Additional demographic identities}
    \label{app:demographics-2-jobs-p1}
\end{table*}
\begin{table*}[htpb]
    \centering
    \footnotesize
    \begin{tabularx}{\textwidth}{Xl|Xl|Xl|Xl}
    \toprule
    \textbf{Longest Residence} & \textbf{\textit(N) (\%)} & \textbf{Employment} & \textbf{\textit{N} (\%)} & \textbf{Occupation (Top 10)} & \textbf{\textit{N} (\%)} & \textbf{Religion} & \textbf{\textit{N} (\%)} \\
    % \hline
    \midrule
United States of America & 97 (97.0) & Employed, 40+ & 53 (53.0) & Other & 35 (35.0) & Christian & 29 (29.0)\\
Others & 3 (3.0) & Employed, 1-39 & 16 (16.0) & Prefer not to answer & 10 (10.0) & Agnostic & 20 (20.0)\\
 &  & Retired & 9 (9.0) & Health Care and Social Assistance & 10 (10.0) & Atheist & 15 (15.0)\\
 &  & Not employed, looking for work & 7 (7.0) & Information & 10 (10.0) & Nothing in particular & 13 (13.0)\\
 &  & Disabled, not able to work & 5 (5.0) & Manufacturing & 7 (7.0) & Catholic & 11 (11.0)\\
 &  & Not employed, NOT looking for work & 4 (4.0) & Professional, Scientific, and Technical Services & 7 (7.0) & Muslim & 5 (5.0)\\
 &  & Other: please specify & 4 (4.0) & Arts, Entertainment, and Recreation & 6 (6.0) & Hindu & 3 (3.0)\\
 &  & Prefer not to disclose & 2 (2.0) & Retail Trade & 6 (6.0) & Something else, Specify & 2 (2.0)\\
 &  &  &  & Finance and Insurance & 5 (5.0) & Jewish & 1 (1.0)\\
 &  &  &  & Transportation and Warehousing, and Utilities & 4 (4.0) & Buddhist & 1 (1.0)\\
 \bottomrule
    \end{tabularx}
    \caption{Labor Replacement Study 1 Survey: Additional demographic identities. The Occupation category was capped at the top 10 for brevity, with the remaining occupations merged together with the Other: please specify option.}
    \label{app:demographics-3-jobs-p1}
\end{table*}
\begin{table*}[!htpb]
    \footnotesize
    \centering
    \begin{tabular}{{ll|ll|ll|ll}}
    \toprule
         \textbf{Racial Identity} & \textbf{\textit(N) (\%)} & \textbf{Age} & \textbf{\textit{N} (\%)} & \textbf{Gender Identity} & \textbf{\textit{N} (\%)} & \textbf{Education} & \textbf{\textit{N} (\%)} \\
         % \hline
         \midrule
        White or Caucasian & 29 (29.9) & 45-54 & 30 (30.9) & Man & 50 (51.5) & Bachelor’s degree & 40 (41.2)\\
        Black or African American & 26 (26.8) & 25-34 & 26 (26.5) & Non-male & 47 (48.5) & Some college $^{*}$ & 21 (21.6)\\
        Asian & 20 (20.6) & 55-64 & 14 (14.4) & & & Graduate degree$^{*}$ & 14 (14.4)\\
        Other & 14 (14.4) & 35-44 & 13 (13.4) & & & High school diploma$^{*}$ & 13 (13.4)\\
        Mixed & 8 (8.2) & 18-24 & 9 (9.3) & & & Associates degree$^{*}$ & 8 (8.2)\\
         & 2 (0.7) & 65+ & 4 (4.1) & & & Some high school$^{*}$ & 1 (1.0)\\
         &  & Prefer not to disclose & 1 (1.0) & & & & \\
    \bottomrule
    \end{tabular}
    \caption{Personal Use Cases Study 1 Survey: Racial, age, gender identities and education level of participants. Asterisk (*) denotes labels shortened due to space.}
    \label{app:demographics-1-personal-p1}
\end{table*}
\begin{table*}[htpb]
    \centering
    \footnotesize
    \begin{tabular}{ll|ll|ll|ll}
    \toprule
         \textbf{Minority/Disadvantaged Group} & \textbf{\textit(N) (\%)} & \textbf{Transgender} & \textbf{\textit{N} (\%)} & \textbf{Sexuality} & \textbf{\textit{N} (\%)} & \textbf{Political Leaning} & \textbf{\textit{N} (\%)} \\
         % \hline
         \midrule
No & 51 (52.6) & No & 94 (96.9) & Heterosexual & 75 (77.3) & Liberal & 34 (35.1)\\
Yes & 46 (47.4) & Yes & 3 (3.1) & Others & 22 (22.7) & Moderate & 31 (32.0)\\
 &  & Prefer not to say & 0 (0.0) & & & Strongly liberal & 12 (12.4)\\
 &  &  &  &  &  & Conservative & 10 (10.3)\\
 &  &  &  &  &  & Strongly conservative & 9 (9.3)\\
 &  &  &  &  &  & Prefer not to say & 1 (1.0)\\
\bottomrule
    \end{tabular}
    \caption{Personal Use Cases Study 1 Survey: Additional demographic identities}
    \label{app:demographics-2-personal-p1}
\end{table*}
\begin{table*}[htpb]
    \centering
    \footnotesize
    \begin{tabularx}{\textwidth}{Xl|Xl|Xl|Xl}
    \toprule
        \textbf{Longest Residence} & \textbf{\textit(N) (\%)} & \textbf{Employment} & \textbf{\textit{N} (\%)} & \textbf{Occupation (Top 10)} & \textbf{\textit{N} (\%)} & \textbf{Religion} & \textbf{\textit{N} (\%)} \\
    % \hline
    \midrule
United States of America & 93 (95.9) & Employed, 40+ & 46 (47.4) & Other & 36 (35.6) & Christian & 40 (40.8)\\
Others & 4 (4.1) & Employed, 1-39 & 22 (22.7) & Health Care and Social Assistance & 11 (11.3) & Catholic & 16 (16.3)\\
 &  & Not employed, looking for work & 13 (13.4) & Prefer not to answer & 10 (10.3) & Agnostic & 15 (15.3)\\
 &  & Not employed, NOT looking for work & 4 (4.1) & Professional, Scientific, and Technical Services & 9 (9.3) & Nothing in particular & 11 (11.2)\\
 &  & Disabled, not able to work & 4 (4.1) & Educational Services & 9 (9.3) & Atheist & 5 (5.1)\\
 &  & Other: please specify & 4 (4.1) & Finance and Insurance & 8 (8.2) & Something else, Specify & 5 (5.1)\\
 &  & Retired & 3 (3.1) & Arts, Entertainment, and Recreation & 5 (5.2) & Buddhist & 3 (3.1)\\
 &  & Prefer not to disclose & 1 (1.0) & Manufacturing & 5 (5.2) & Muslim & 1 (1.0)\\
 &  &  &  & Retail Trade & 4 (4.1) & Jewish & 1 (1.0)\\
 &  &  &  & Accommodation and Food Services & 4 (4.1) & Hindu & 1 (1.0)\\
 \bottomrule
    \end{tabularx}
    \caption{Personal Use Cases Study 1 Survey: Additional demographic identities. The Occupation category was capped at the top 10 for brevity, with the remaining occupations merged together with the Other: please specify option.}
    \label{app:demographics-3-personal-p1}
\end{table*}
\begin{table}[!hbpt]
\centering
\small
\begin{tabular}{ccccccc}
\hline
Score & AI Awareness & AI Usage & AI Evaluation & Gen AI Usage Freq. & Gen AI Limit. Familiarity \\ \hline
1  & 25           & 15       & 30            & 35    & 55   \\
2  & 40           & 60       & 75            & 200   & 320  \\
3  & 75           & 90       & 105           & 155   & 345  \\
4  & 140          & 125      & 125           & 235   & 230  \\
5  & 380          & 310      & 275           & 220   & 35   \\
6  & 280          & 300      & 310           & 140   &  \NA    \\
7  & 45           & 85       & 65            &     \NA &    \NA \\
\hline
\end{tabular}
\caption{AI literacy scale participant count. Questions are on a 7-point likert scale of increasing score meaning increase in literacy for the aspect. Gen AI Usage Frequency has max score of 6 and Limitation Familiarity has max value of 5.}
\label{tab:ai-literacy-distr}
\end{table}
\begin{table}[h]
\centering
\small
\begin{tabular}{lc}
\hline
Score & AI Ethics \\ \hline
5  & 15             \\
6  & 10             \\
7  & 25             \\
8  & 25             \\
9  & 75             \\
10 & 105            \\
11 & 165            \\
12 & 100            \\
13 & 105            \\
14 & 90             \\
15 & 120            \\
16 & 80             \\
17 & 35             \\
18 & 35             \\ \hline
\end{tabular}
\caption{AI ethics score count for total AI ethics score (sum over 3 questions with 7 point likert scale with max possible value of 21)}
\label{tab:ai-ethics-distr}
\end{table}
\subsection{Open-text Annotation Dimensions}
\label{sssec:reasoning-dim}
\paragraph{Reasoning Type}
% still would like to answer, for rule-based reasoning, what kind of rules? Are norms and definitions also rule-based reasoning?
% also, for cost-benefit reasoning, is this the only reasoning pattern for decision making if the dilemma is not moral?
Inspired by previous works in moral psychology, we used two main reasoning types to characterize participants' decision making pattern as expressed in their open-text answers: cost-benefit reasoning and rule-based reasoning \citep{cheung2024measuring}. These two reasoning types are rooted in two decision making processes in moral and wider decision making literature: utilitarian and deontological reasoning, respectively. Cost-benefit reasoning thus considers the possible outcomes and their expected utility or value when making decisions, and rule-based reasoning shows more inherent value in action or entities. See \S~\ref{ssec:moral-decision-making} for further discussion.
% maybe add examples here?

\paragraph{Moral Values}
To annotate which values were prevalent in participants' considerations of use cases, we used five moral values: care, fairness, loyalty, authority, and purity \citep{graham2011mapping,graham2008moral}. While these dimensions have been re-defined to include more diverse values from participants beyond WEIRD (white, educated, industrialized, rich, and democratic) \citep{atari2023morality}, we used these five dimensions due to brevity of the questionnaire, which was used in the survey to provide importance of each values to participants. 

\paragraph{Switching Conditions}
We annotated concerns expressed in switching conditions using three categories: functionality (e.g., errors, bias in systems, limited capabilities), usage (e.g., errors, bias in systems, limited capabilities), and societal impact (e.g., job loss, over-reliance), inspired by harm taxonomy developed by \citeauthor{solaiman2023evaluating} and user concern annotation practice adopted by \citeauthor{mun2024participaidemocraticsurveyingframework}.

\section{Open-text Annotation Details}
\label{app:open-text-annotation-details}
\subsection{Automatic Annotation}
\subsubsection{Methods}
We used Open-AI's o1-mini model with maximum tokens set to 1024 to control response length, use a temperature of 0.7 to manage randomness, and keep top\_p at 1 with default settings for frequency and presence penalties at 0. Prompts will be released with data upon acceptance. 

\subsubsection{Results}
Results for inter-rater reliability analysis of o1's annotations are shown in Table~\ref{tab:irr-results}.
\begin{table}[!hptb]
\centering
\begin{tabular}{l|cccccccc}
\hline
\textbf{Category} & \textbf{AC1} & \textbf{Interpretation} & \textbf{95\% CI} & \textbf{p-value} & \textbf{z} & \textbf{SE} & \textbf{PA} & \textbf{PE} \\ \hline
Cost Benefit & 0.976 & Almost Perfect & (0.942, 1.000) & \textbf{0.0} & 56.9 & 0.0172 & 0.980 & 0.164 \\ 
Rule Based & 0.848 & Almost Perfect & (0.754, 0.943) & \textbf{0.0} & 17.8 & 0.0476 & 0.890 & 0.276 \\
Care & 0.427 & Moderate & (0.234, 0.619) & $\mathbf{2.76 \times 10^{-5}}$ & 4.40 & 0.0970 & 0.650 & 0.390 \\
Fairness & 0.411 & Moderate & (0.224, 0.599) & $\mathbf{3.29 \times 10^{-5}}$ & 4.35 & 0.0945 & 0.690 & 0.474 \\
Authority & 0.839 & Almost Perfect & (0.743, 0.935) & \textbf{0.0} & 17.3 & 0.0486 & 0.880 & 0.255 \\
Purity & 0.758 & Substantial & (0.639, 0.878) & \textbf{0.0} & 12.6 & 0.0603 & 0.820 & 0.255 \\
Functionality & 0.587 & Moderate & (0.425, 0.749) & $\mathbf{1.31 \times 10^{-10}}$ & 7.18 & 0.0818 & 0.790 & 0.492 \\
Usage & 0.673 & Substantial & (0.525, 0.822) & $\mathbf{1.47 \times 10^{-14}}$ & 9.03 & 0.0746 & 0.820 & 0.449 \\
Societal Impact & 0.595 & Moderate & (0.432, 0.759) & $\mathbf{1.01 \times 10^{-10}}$ & 7.23 & 0.0823 & 0.770 & 0.432 \\
\hline
\end{tabular}
\caption{Inter-rater Agreement using Gwet's AC1. Interpretation according to \citep{wongpakaran2013comparison}.}
\label{tab:irr-results}
\end{table}

\section{Factors Impacting Acceptability Judgments}
\subsection{Use Case Factors}
Additional analysis of use case factors showing distribution of judgments by use case sorted by standard deviation is shown in Figure~\ref{fig:decisions-by-use-case-violin}. Table~\ref{table:use-case-effect-anova} shows analysis of use case effect using ANOVA.
\begin{figure}[!hptb]
    \centering
    \includegraphics[width=\linewidth]{figures/judgments_by_use_case_violin_figure.pdf}
    \caption{Numerically converted Judgment x Confidence (-5, 5) by use cases distributions sorted by standard deviation of both existence and usage (sum; highest to lowest) using data from Study 1 results.}
    \label{fig:decisions-by-use-case-violin}
\end{figure}
\begin{table}[!hptb]
    \begin{center}
    \small
    \begin{tabular}{@{}llcccccc@{}}
        \toprule
        \textbf{Acceptability} & \textbf{Aspect} & \textbf{Factor} & \textbf{Sum Sq} & \textbf{Mean Sq} & \textbf{NumDF} & \textbf{DenDF} & \textbf{Pr(>F)} \\ 
        \midrule
        \multirow{4}{*}{EXIST} & \multirow{2}{*}{Judgment} & Category & 29.903 & 29.903 & 1 & 197 & \textbf{5.98e-11} ***\\
        & & Use Case & 86.116 & 9.5684 & 9 & 641.38 & \textbf{< 2.2e-16} ***\\ \cmidrule{2-8}
        & \multirow{2}{*}{Confidence} & Category & 0.0017113 & 0.0017113 & 1 & 197 & 0.9563 \\
        & & Use Case & 13.519 & 1.5021 & 9 & 603.23  2.7243 & \textbf{0.004037} **\\ \midrule
        \multirow{4}{*}{USAGE} & \multirow{2}{*}{Judgment} & Category & 8.4257 & 8.4257 & 1 & 197 & \textbf{0.0002488} ***\\
        & & Use Case & 73.801 & 8.2001 & 9 & 610.34 & \textbf{< 2.2e-16} ***\\ \cmidrule{2-8}
        & \multirow{2}{*}{Confidence} & Category & 1.3444 & 1.3444 & 1 & 197 & 0.153 \\
        & & Use Case & 20.721 & 2.3023 & 9 & 603.36 & \textbf{0.0001783} ***\\
        \bottomrule
    \end{tabular}
    \caption{ANOVA analysis of LMER models \texttt{judgment $\sim$ category + (1 | subject)} and \texttt{judgment $\sim$ useCase + (1 | subject)} (same formulas repeated with \texttt{confidence} as a dependent variable){} analyzed with Study 1 data.}
    \label{table:use-case-effect-anova}
    \end{center}
\end{table}

% \begin{table}
% \begin{center}
% \begin{tabular}{l c c}
% \hline
%  & Exist $\times$ Conf. & Usage $\times$ Conf. \\
% \hline
% (Intercept) Telemarketer                                & $-0.22^{*}$ $(0.09)$      & $-0.45^{***}$ $(0.09)$   \\
% Government Eligibility Interviewer & $-0.07$ $(0.11)$          & $0.32^{**}$ $(0.10)$     \\
% IT Support Specialist              & $0.50^{***}$ $(0.11)$     & $0.94^{***}$ $(0.10)$    \\
% Elementary School Teacher          & $-0.56^{***}$ $(0.11)$    & $-0.11$ $(0.10)$         \\
% Lawyer                             & $-0.22^{*}$ $(0.11)$      & $0.20^{*}$ $(0.10)$      \\
% Flavorful Swaps                    & $0.68^{***}$ $(0.13)$     & $0.74^{***}$ $(0.13)$    \\
% Nutrition Optimizer                & $0.76^{***}$ $(0.13)$     & $0.83^{***}$ $(0.13)$    \\
% Personal Health Research           & $0.61^{***}$ $(0.13)$     & $0.74^{***}$ $(0.13)$    \\
% Customized Lifestyle Coach         & $0.43^{***}$ $(0.13)$     & $0.51^{***}$ $(0.13)$    \\
% Digital Medical Advice             & $0.15$ $(0.13)$           & $0.36^{**}$ $(0.13)$     \\
% \hline
% AIC                                         & $2516.19$                 & $2447.74$               \\
% BIC                                         & $2574.90$                 & $2506.45$               \\
% Log Likelihood                              & $-1246.10$                & $-1211.87$              \\
% Num. obs.                                   & $985$                     & $985$                   \\
% Num. groups: prolific\_id                   & $197$                     & $197$                   \\
% Var: prolific\_id (Intercept)               & $0.24$                    & $0.38$                  \\
% Var: Residual                               & $0.59$                    & $0.50$                  \\
% \hline
% \multicolumn{3}{l}{\scriptsize{$^{***}p<0.001$; $^{**}p<0.01$; $^{*}p<0.05$}}
% \end{tabular}
% \caption{Model using \texttt{Judgment $\sim$ Use Case + (1 | Subject)} with Study 1 data.}
% \label{table:coefficients}
% \end{center}
% \end{table}

% updating lmer summary table to anova according to Jana's feedback
% \begin{table}
% \begin{center}
% \small
% \begin{tabular}{l c c}
% \hline
%  & Exist$\times$Conf. & Usage$\times$Conf. \\
% \hline
% (Intercept) Jobs                   & $\mathbf{-0.29}^{***} \; (0.06)$ & $\mathbf{-0.18}^{**} \; (0.07)$ \\
% Personal              & $\mathbf{0.60}^{***} \; (0.08)$  & $\mathbf{0.37}^{***} \; (0.10)$ \\
% \midrule
% (Intercept) Telemarketer                                 & $\mathbf{-0.22}^{*} \; (0.10)$   & $\mathbf{-0.45}^{***} \; (0.10)$ \\
% Government Eligibility Interviewer & $-0.08 \; (0.12)$       & $\mathbf{0.32}^{**} \; (0.11)$   \\
% IT Support Specialist              & $\mathbf{0.49}^{***} \; (0.12)$  & $\mathbf{0.94}^{***} \; (0.11)$  \\
% Elementary School Teacher          & $\mathbf{-0.57}^{***} \; (0.12)$ & $-0.11 \; (0.11)$       \\
% Lawyer                             & $-0.22 \; (0.12)$       & $0.21 \; (0.11)$        \\
% \midrule
% (Intercept) Flavorful Swaps                         & $\mathbf{0.45}^{***} \; (0.08)$   & $\mathbf{0.29}^{**} \; (0.09)$   \\
% Nutrition Optimizer        & $0.09 \; (0.09)$         & $0.10 \; (0.09)$       \\
% Personal Health Research   & $-0.05 \; (0.09)$        & $0.01 \; (0.09)$       \\
% Customized Lifestyle Coach & $\mathbf{-0.23}^{*} \; (0.09)$    & $\mathbf{-0.21}^{*} \; (0.09)$  \\
% Digital Medical Advice     & $\mathbf{-0.52}^{***} \; (0.09)$  & $\mathbf{-0.38}^{***} \; (0.09)$ \\
% \hline
% \multicolumn{3}{l}{\scriptsize{$^{***}p<0.001$ $^{**}p<0.01$ $^{*}p<0.05$}}
% \end{tabular}
% \caption{Model using \texttt{Judgment $\sim$ Confidence + (1 | Subject)} and \texttt{Judgment $\sim$ Use Case + (1 | Subject)} with Study 1 data separated into personal and jobs.}
% \label{table:use-case-effect}
% \end{center}
% \end{table}


\subsection{Demographics Factors}
Additional analysis using ANOVA for demographic factors is shown in Table~\ref{tab:demographics-anova}.
\begin{table}[!hptb]
\centering
\small
\begin{tabular}{lcccccc}
\hline
 & \multicolumn{3}{c}{EXIST} & \multicolumn{3}{c}{USAGE}\\
\cmidrule(lr){2-4}\cmidrule(lr){5-7}
\textbf{Demographics} & Judgment & Confidence & Judg.$\times$Conf. & Judgment & Confidence & Judg$\times$Conf.\\
\hline
\multicolumn{7}{l}{\textbf{All}}\\
% \quad Gender & $F_{()}$ & $F_{()}$ & $F_{()}$ & $F_{()}$ & $F_{()}$ & $F_{()}$\\             
\quad Gender & $\mathbf{16.60^{***}}$ & $0.19$ &  $\mathbf{16.71^{***}}$ & $\mathbf{15.26^{***}}$ & $0.83$ & $\mathbf{13.14^{***}}$ \\
\quad Race               & $1.62$ & $\mathbf{4.09^{**}}$ & $1.45$ & $0.65$ & $\mathbf{5.12^{***}}$ & $0.45$ \\
\quad Employment         & $1.13$ & $\mathbf{3.03^{*}}$ & $1.14$ & $0.43$ & $1.71$ & $0.69$ \\
\quad Sexual Orientation & $0.42$ & $0.37$ & $0.19$ & $0.75$ & $\mathbf{5.22^{*}}$ & $0.09$ \\
\hline
\multicolumn{7}{l}{\textbf{Professional}} \\
\quad Race               & $\mathbf{2.56^{*}}$   & $1.80$                & $\mathbf{2.34^{.}}$    & $1.04$                & $\mathbf{2.91^{*}}$   & $0.48$                \\
\quad Gender             & $\mathbf{18.37^{***}}$& $0.05$                & $\mathbf{19.51^{***}}$ & $\mathbf{19.83^{***}}$& $0.19$                & $\mathbf{20.21^{***}}$\\
\quad Education          & $1.98$                & $0.96$                & $1.34$                 & $2.25^{.}$            & $1.12$                & $2.07^{.}$            \\
\quad Discrimination & $2.18$                & $0.68$                & $2.67^{.}$             & $0.29$                & $1.46$                & $0.13$                \\
\hline
\multicolumn{7}{l}{\textbf{Personal}}\\
\quad Race               & $2.11^{.}$ & $\mathbf{4.36^{**}}$  & $2.28^{.}$  & $0.16$ & $\mathbf{4.07^{**}}$ & $0.38$ \\
\quad Political View     & $0.38$     & $\mathbf{3.39^{*}}$   & $0.86$      & $0.33$ & $1.56$ & $0.36$ \\
\quad Employment         & $0.85$     & $\mathbf{2.42^{*}}$   & $1.47$      & $0.33$ & $0.30$ & $0.36$ \\
\hline
\multicolumn{7}{l}{\scriptsize{$^{***}p<0.001$; $^{**}p<0.01$; $^{*}p<0.05$; $^{.}p<0.1$}}
\end{tabular}
\caption{ANOVA Results by Demographic Category (F‐value with Significance)}
\label{tab:demographics-anova}
\end{table}

\begin{table}[hpbt]
\begin{center}
\small
\begin{tabular}{l c c c c c c}
\hline
 & \multicolumn{3}{c}{EXIST ($\beta$ (SE))} & \multicolumn{3}{c}{USAGE ($\beta$ (SE))}\\
\cmidrule(lr){2-4}\cmidrule(lr){5-7}
Decision Style Factors & Judg. & Conf. & Judg.$\times$Conf. & Judg. & Conf. & Judg.$\times$Conf. \\
\hline
(Intercept) & $0.11 \ (0.34)$ & $\mathbf{3.10^{***}} \ (0.46)$ & $-0.32 \ (1.46)$ & $-0.74 \ (0.39)$ & $\mathbf{2.98^{***}} \ (0.49)$ & $\mathbf{-4.04^{*}} \ (1.68)$ \\
MFQ Care & $0.00 \ (0.01)$ & $-0.01 \ (0.02)$ & $0.01 \ (0.06)$ & $-0.00 \ (0.02)$ & $0.02 \ (0.02)$ & $-0.01 \ (0.07)$ \\
MFQ Fairness & $-0.01 \ (0.02)$ & $0.04 \ (0.02)$ & $-0.03 \ (0.07)$ & $0.01 \ (0.02)$ & $0.01 \ (0.02)$ & $0.05 \ (0.08)$ \\
MFQ Loyalty & $\mathbf{0.04^{***}} \ (0.01)$ & $-0.01 \ (0.02)$ & $\mathbf{0.18^{***}} \ (0.05)$ & $\mathbf{0.04^{**}} \ (0.01)$ & $-0.01 \ (0.02)$ & $\mathbf{0.18^{**}} \ (0.06)$ \\
MFQ Authority & $-0.01 \ (0.01)$ & $0.02 \ (0.02)$ & $-0.05 \ (0.05)$ & $0.00 \ (0.02)$ & $0.03 \ (0.02)$ & $0.02 \ (0.06)$ \\
MFQ Purity & $0.00 \ (0.01)$ & $0.02 \ (0.01)$ & $0.04 \ (0.04)$ & $-0.01 \ (0.01)$ & $0.00 \ (0.02)$ & $-0.01 \ (0.05)$ \\
Empathy & $0.01 \ (0.03)$ & $0.03 \ (0.04)$ & $0.05 \ (0.13)$ & $0.06 \ (0.04)$ & $0.02 \ (0.05)$ & $0.28 \ (0.15)$ \\
InstrumentalHarm & $0.00 \ (0.03)$ & $-0.01 \ (0.04)$ & $0.04 \ (0.13)$ & $-0.02 \ (0.03)$ & $-0.01 \ (0.04)$ & $-0.06 \ (0.15)$ \\
ImpartialBenificence & $0.03 \ (0.03)$ & $-0.01 \ (0.04)$ & $0.08 \ (0.13)$ & $0.02 \ (0.04)$ & $-0.02 \ (0.05)$ & $0.06 \ (0.15)$ \\
\hline
AIC & $2412.17$ & $2539.14$ & $5149.55$ & $2442.29$ & $2671.82$ & $5132.26$ \\
BIC & $2470.88$ & $2597.85$ & $5208.26$ & $2501.01$ & $2730.53$ & $5190.97$ \\
Log Likelihood & $-1194.08$ & $-1257.57$ & $-2562.77$ & $-1209.15$ & $-1323.91$ & $-2554.13$ \\
Num. obs. & $985$ & $985$ & $985$ & $985$ & $985$ & $985$ \\
Num. groups: prolific\_id & $197$ & $197$ & $197$ & $197$ & $197$ & $197$ \\
Num. groups: use\_case & $10$ & $10$ & $10$ & $10$ & $10$ & $10$ \\
Var: prolific\_id (Intercept) & $0.12$ & $0.37$ & $2.71$ & $0.23$ & $0.42$ & $4.65$ \\
Var: use\_case (Intercept) & $0.11$ & $0.01$ & $2.22$ & $0.08$ & $0.02$ & $1.58$ \\
Var: Residual & $0.55$ & $0.56$ & $8.53$ & $0.53$ & $0.64$ & $7.70$ \\
\hline
\multicolumn{7}{l}{\scriptsize{$^{***}p<0.001$; $^{**}p<0.01$; $^{*}p<0.05$}}
\end{tabular}
\caption{Coefficients with standard error in parenthesis with following models: \texttt{Judgment $\sim$ $\texttt{MFQ}_{foundation}$ + Empathy + InustrumentalHarm + ImpartialBeneficence + (1|Subject) + (1|useCase)}. Bolded value for empathy had $p<0.1$.}
\label{tab:reasoning-factors-questionnaires}
\end{center}
\end{table}
\subsubsection{Questionnaires}
Interestingly, only Loyalty had a significant effect on both existence ($0.20, p<.001$) and usage ($0.20, p<.01$) as shown in Table~\ref{tab:reasoning-factors-questionnaires}. Moreover, Empathy had a positive and marginally significant effect for usage ($.09, p<.1$). However, Loyalty, as shown in Figure~\ref{fig:moral_values_proportions}, does not appear as frequently in participants' open text responses compared to values such as Care and Fairness and is the only value that did not have a significant association with use cases. 

\section{Factors in Participant Rationale}

\subsection{Reasoning Types}
We show the flow of participants' decisions and corresponding rationales throughout use cases in Figure~\ref{fig:reasoning-type-sankey}, which shows interesting distribution and switching of reasoning types, which would be interesting for future studies to consider. Moreover, Table~\ref{tab:reasoning-type-questionnaire} shows that there are almost no relation between reasoning types used by the participants and the decision-making style questionnaire results signifying that the reasoning types might be highly use-case specific rather than a character trait. It would be interesting to study the factors that actually influence the choice of reasoning types. 
\begin{figure}[!hptb]
    \centering
    \subfigure[Acceptance judgment and reasoning type mapping throughout professional use cases.]{
        \includegraphics[width=\linewidth]{figures/open-text-annotations/professional-sankey.pdf}
        \label{fig:subfig1}
    }
    \hfill
    \subfigure[Acceptance judgment and reasoning type mapping throughout personal use cases.]{
        \includegraphics[width=\linewidth]{figures/open-text-annotations/personal-sankey.pdf}
        \label{fig:subfig2}
    }
    \caption{Mapping of decisions and reasoning types. + and - denote positive and negative acceptance. ``C'' denotes Cost-benefit analysis and ``R'' denotes Rule-based reasoning.}
    \label{fig:reasoning-type-sankey}
\end{figure}
\begin{table}
\begin{center}
\begin{tabular}{l c c}
\hline
 & Cost-benefit & Rule-based \\
\hline
(Intercept)                   & $\mathbf{0.73^{***}}(0.13)$ & $\mathbf{0.28^{*}}(0.13)$ \\
MFQ Care                     & $-0.00(0.01)$      & $0.00(0.01)$   \\
MFQ Fairness                 & $0.01(0.01)$       & $0.00(0.01)$   \\
MFQ Loyalty                  & $\mathbf{0.01^{*}}(0.00)$   & $-0.01(0.01)$ \\
MFQ Authority                & $-0.00(0.01)$      & $0.00(0.01)$   \\
MFQ Purity                   & $-0.00(0.00)$      & $0.00(0.00)$   \\
Empathy                & $0.00(0.01)$       & $-0.01(0.01)$  \\
InstrumentalHarm              & $0.00(0.01)$       & $-0.00(0.01)$  \\
ImpartialBenificence          & $-0.00(0.01)$      & $-0.00(0.01)$  \\
\hline
AIC                           & $668.43$     & $815.32$   \\
BIC                           & $727.14$     & $874.03$   \\
Log Likelihood                & $-322.21$    & $-395.66$  \\
Num. obs.                     & $985$        & $985$      \\
Num. groups: prolific\_id     & $197$        & $197$      \\
Num. groups: use\_case        & $10$         & $10$       \\
Var: prolific\_id (Intercept) & $0.02$       & $0.02$     \\
Var: use\_case (Intercept)    & $0.00$       & $0.01$     \\
Var: Residual                 & $0.10$       & $0.11$     \\
\hline
\multicolumn{3}{l}{\scriptsize{$^{***}p<0.001$; $^{**}p<0.01$; $^{*}p<0.05$}}
\end{tabular}
\caption{Coefficient and standard error with significance. Model defined by \texttt{reasoningType $\sim$ MFQ$_{foundation}$ + Empathy + InstrumentalHarm + ImpartialBeneficence + (1|subject) + (1|useCase)}}
\label{tab:reasoning-type-questionnaire}
\end{center}
\end{table}

\subsection{Impact of Rationale Factors on Judgment}
We display the analysis results using ANOVA to understand the effect of rationale factors on judgment in Table~\ref{tab:reasoning-factors-judgment-anova}.
\begin{table}[!hptb]
    \begin{center}
    \small
    \begin{tabular}{@{}llccccc@{}}
        \toprule
        \textbf{Acceptability} & \textbf{Factor} & \textbf{Sum Sq} & \textbf{Mean Sq} & \textbf{NumDF} & \textbf{DenDF} & \textbf{Pr(>F)} \\ 
        \midrule
        \multirow{16}{*}{\textbf{EXIST}} & \multicolumn{5}{l}{\textbf{Judgment}} \\ 
        & \quad Cost Benefit & 2.941 & 2.941 & 1 & 955.32 & \textbf{0.001306} ** \\ 
        & \quad Rule Based & 7.187 & 7.187 & 1 & 941.74 & \textbf{5.562e-07} *** \\ 
        & \quad Fairness & 1.551 & 1.551 & 1 & 966.41 & \textbf{0.019414} * \\ 
        & \quad Authority & 1.044 & 1.044 & 1 & 961.50 & 0.055021 . \\ 
        & \quad Functionality & 1.137 & 1.137 & 1 & 964.09 & \textbf{0.045230} * \\ 
        & \quad Usage & 125.594 & 125.594 & 1 & 837.10 & \textbf{< 2.2e-16} *** \\ 
        & \quad Societal Impact & 1.873 & 1.873 & 1 & 967.89 & \textbf{0.010228} * \\ 
        & \multicolumn{5}{l}{\textbf{Confidence}} \\
        & \quad Rule Based & 5.7805 & 5.7805 & 1 & 856.38 & \textbf{0.001158} ** \\ 
        & \quad Care & 3.8723 & 3.8723 & 1 & 907.71 & \textbf{0.007762} ** \\ 
        & \quad Fairness & 5.7117 & 5.7117 & 1 & 899.21 & \textbf{0.001237} ** \\ 
        & \quad Authority & 2.2665 & 2.2665 & 1 & 851.91 & \textbf{0.041526} * \\ 
        & \quad Usage & 2.6254 & 2.6254 & 1 & 809.17 & \textbf{0.028303} * \\ 
        & \multicolumn{5}{l}{\textbf{Judgment x Confidence}} \\
        & \quad Cost Benefit & 48.04 & 48.04 & 1 & 940.91 & \textbf{0.0004985} *** \\ 
        & \quad Rule Based & 86.12 & 86.12 & 1 & 924.15 & \textbf{3.331e-06} *** \\
        & \quad Fairness & 31.87 & 31.87 & 1 & 972.26 & \textbf{0.0045236} ** \\
        & \quad Authority & 14.24 & 14.24 & 1 & 972.97 & 0.0574858 . \\ 
        & \quad Functionality & 18.99 & 18.99 & 1 & 971.66 & \textbf{0.0283005} * \\ 
        & \quad Usage & 2454.30 & 2454.30 & 1 & 888.43 & \textbf{< 2.2e-16} *** \\ 
        & \quad Societal Impact & 34.26 & 34.26 & 1 & 971.98 & \textbf{0.0032486} ** \\ 
        \midrule
        \multirow{12}{*}{\textbf{USAGE}} & \multicolumn{5}{l}{\textbf{Judgment}} \\ 
        & \quad Cost Benefit & 0.28 & 0.28 & 1 & 933.91 & \textbf{0.01742} * \\ 
        & \quad Usage & 491.35 & 491.35 & 1 & 943.16 & \textbf{< 2e-16} *** \\ 
        & \multicolumn{5}{l}{\textbf{Confidence}} \\ 
        & \quad Cost Benefit & 2.2380 & 2.2380 & 1 & 864.41 & 0.0555024 . \\ 
        & \quad Rule Based & 2.0230 & 2.0230 & 1 & 852.41 & 0.0686374 . \\ 
        & \quad Fairness & 4.2517 & 4.2517 & 1 & 895.25 & \textbf{0.0083622} ** \\ 
        & \quad Authority & 2.4525 & 2.4525 & 1 & 852.20 & \textbf{0.0450309} * \\ 
        & \quad Loyalty & 2.2751 & 2.2751 & 1 & 875.10 & 0.0535163 . \\ 
        & \quad Functionality & 2.6366 & 2.6366 & 1 & 897.34 & \textbf{0.0376892} * \\ 
        & \quad Usage & 9.2673 & 9.2673 & 1 & 817.52 & \textbf{0.0001033} *** \\ 
        & \quad Societal Impact & 2.1237 & 2.1237 & 1 & 913.41 & 0.0620930 . \\ 
        & \multicolumn{5}{l}{\textbf{Judgment x Confidence}} \\ 
        & \quad Cost Benefit & 106.358 & 106.358 & 1 & 874.26 & \textbf{8.333e-05} *** \\ 
        & \quad Rule Based & 33.512 & 33.512 & 1 & 859.58 & \textbf{0.026742} * \\ 
        & \quad Fairness & 89.120 & 89.120 & 1 & 930.00 & \textbf{0.000312} *** \\ 
        & \quad Societal Impact & 74.975 & 74.975 & 1 & 929.00 & \textbf{0.000938} *** \\ 
        \bottomrule
    \end{tabular}
    \caption{ANOVA analysis of the LMER model results.}
    \label{tab:reasoning-factors-judgment-anova}
    \end{center}
\end{table}

% \begin{table}[!hptb]
%     \begin{center}
%     \footnotesize
%     \begin{tabular}{@{}lccccc@{}}
%         \toprule
%         \textbf{Factor} & \textbf{Sum Sq} & \textbf{Mean Sq} & \textbf{NumDF} & \textbf{DenDF} & \textbf{Pr(>F)} \\ 
%         \midrule
%     \multicolumn{6}{@{}l}{\textbf{EXIST}} \\ 
%     \hline
%     \multicolumn{6}{@{}l}{\textbf{Judgment}} \\ 
%     \quad Cost Benefit & 2.941 & 2.941 & 1 & 955.32 & \textbf{0.001306} ** \\ 
%     \quad Rule Based & 7.187 & 7.187 & 1 & 941.74 & \textbf{5.562e-07} *** \\ 
%     % \quad Care & 0.693 & 0.693 & 1 & 973.01 & 0.117715 \\
%     \quad Fairness & 1.551 & 1.551 & 1 & 966.41 & \textbf{0.019414} * \\ 
%     % \quad Purity & 0.088 & 0.088 & 1 & 969.60 & 0.577171 \\ 
%     \quad Authority & 1.044 & 1.044 & 1 & 961.50 & 0.055021 . \\ 
%     % \quad Loyalty & 0.105 & 0.105 & 1 & 963.15 & 0.541809 \\ 
%     \quad Functionality & 1.137 & 1.137 & 1 & 964.09 & \textbf{0.045230} * \\ 
%     \quad Usage & 125.594 & 125.594 & 1 & 837.10 & \textbf{< 2.2e-16} *** \\ 
%     \quad Societal Impact & 1.873 & 1.873 & 1 & 967.89 & \textbf{0.010228} * \\ 
%     \multicolumn{6}{@{}l}{\textbf{Confidence}} \\
%     % \quad Cost Benefit & 0.7196 & 0.7196 & 1 & 868.87 & 0.250374 \\ 
%     \quad Rule Based & 5.7805 & 5.7805 & 1 & 856.38 & \textbf{0.001158} ** \\ 
%     \quad Care & 3.8723 & 3.8723 & 1 & 907.71 & \textbf{0.007762} ** \\ 
%     \quad Fairness & 5.7117 & 5.7117 & 1 & 899.21 & \textbf{0.001237} ** \\ 
%     % \quad Purity & 1.0942 & 1.0942 & 1 & 921.53 & 0.156428 \\ 
%     \quad Authority & 2.2665 & 2.2665 & 1 & 851.91 & \textbf{0.041526} * \\ 
%     % \quad Loyalty & 1.3561 & 1.3561 & 1 & 880.14 & 0.114704 \\ 
%     % \quad Functionality & 0.3286 & 0.3286 & 1 & 901.37 & 0.437227 \\ 
%     \quad Usage & 2.6254 & 2.6254 & 1 & 809.17 & \textbf{0.028303} * \\ 
%     % \quad Societal Impact & 0.0279 & 0.0279 & 1 & 919.28 & 0.820841 \\ 
%     \multicolumn{6}{@{}l}{\textbf{Judgment x Confidence}} \\
%     \quad Cost Benefit & 48.04 & 48.04 & 1 & 940.91 & \textbf{0.0004985} *** \\ 
%     \quad Rule Based & 86.12 & 86.12 & 1 & 924.15 & \textbf{3.331e-06} *** \\
%     % \quad Care & 0.74 & 0.74 & 1 & 970.41 & 0.6655539 \\
%     \quad Fairness & 31.87 & 31.87 & 1 & 972.26 & \textbf{0.0045236} ** \\
%     % \quad Purity & 1.25 & 1.25 & 1 & 972.84 & 0.5729665 \\
%     \quad Authority & 14.24 & 14.24 & 1 & 972.97 & 0.0574858 . \\ 
%     % \quad Loyalty & 1.16 & 1.16 & 1 & 951.04 & 0.5876398 \\
%     \quad Functionality & 18.99 & 18.99 & 1 & 971.66 & \textbf{0.0283005} * \\ 
%     \quad Usage & 2454.30 & 2454.30 & 1 & 888.43 & \textbf{< 2.2e-16} *** \\ 
%     \quad Societal Impact & 34.26 & 34.26 & 1 & 971.98 & \textbf{0.0032486} ** \\ 
%     \midrule
%     \multicolumn{6}{@{}l}{\textbf{USAGE}} \\
%     \hline
%     \multicolumn{6}{@{}l}{\textbf{Judgment}} \\ 
%     \quad Cost Benefit & 0.28 & 0.28 & 1 & 933.91 & \textbf{0.01742} * \\ 
%     % \quad Rule Based & 0.11 & 0.11 & 1 & 913.21 & 0.14691 \\ 
%     % \quad Care & 0.03 & 0.03 & 1 & 968.07 & 0.44691 \\ 
%     % \quad Fairness & 0.01 & 0.01 & 1 & 970.96 & 0.61175 \\ 
%     % \quad Purity & 0.01 & 0.01 & 1 & 973.60 & 0.66793 \\ 
%     % \quad Authority & 0.01 & 0.01 & 1 & 971.66 & 0.66368 \\ 
%     % \quad Loyalty & 0.01 & 0.01 & 1 & 945.81 & 0.69545 \\ 
%     % \quad Functionality & 0.02 & 0.02 & 1 & 971.90 & 0.57526 \\ 
%     \quad Usage & 491.35 & 491.35 & 1 & 943.16 & \textbf{< 2e-16} *** \\ 
%     % \quad Societal Impact & 0.14 & 0.14 & 1 & 972.98 & 0.10035 \\ 
%     \multicolumn{6}{@{}l}{\textbf{Confidence}} \\ 
%     \quad Cost Benefit & 2.2380 & 2.2380 & 1 & 864.41 & 0.0555024 . \\ 
%     \quad Rule Based & 2.0230 & 2.0230 & 1 & 852.41 & 0.0686374 . \\ 
%     % \quad Care & 1.2204 & 1.2204 & 1 & 902.17 & 0.1571252 \\ 
%     \quad Fairness & 4.2517 & 4.2517 & 1 & 895.25 & \textbf{0.0083622} ** \\ 
%     % \quad Purity & 0.2789 & 0.2789 & 1 & 915.73 & 0.4986374 \\ 
%     \quad Authority & 2.4525 & 2.4525 & 1 & 852.20 & \textbf{0.0450309} * \\ 
%     \quad Loyalty & 2.2751 & 2.2751 & 1 & 875.10 & 0.0535163 . \\ 
%     \quad Functionality & 2.6366 & 2.6366 & 1 & 897.34 & \textbf{0.0376892} * \\ 
%     \quad Usage & 9.2673 & 9.2673 & 1 & 817.52 & \textbf{0.0001033} *** \\ 
%     \quad Societal Impact & 2.1237 & 2.1237 & 1 & 913.41 & 0.0620930 . \\ 
%     \multicolumn{6}{@{}l}{\textbf{Judgment x Confidence}} \\ 
%     \quad Cost Benefit & 106.358 & 106.358 & 1 & 874.26 & \textbf{8.333e-05} *** \\ 
%     \quad Rule Based & 33.512 & 33.512 & 1 & 859.58 & \textbf{0.026742} * \\ 
%     % \quad Care & 17.139 & 17.139 & 1 & 917.34 & 0.112874 \\ 
%     \quad Fairness & 89.120 & 89.120 & 1 & 930.00 & \textbf{0.000312} *** \\ 
%     % \quad Purity & 1.316 & 1.316 & 1 & 926.87 & 0.660237 \\ 
%     % \quad Authority & 2.057 & 2.057 & 1 & 935.97 & 0.582624 \\ 
%     % \quad Loyalty & 16.131 & 16.131 & 1 & 885.01 & 0.124019 \\ 
%     % \quad Functionality & 0.431 & 0.431 & 1 & 932.53 & 0.801252 \\ 
%     \quad Societal Impact & 74.975 & 74.975 & 1 & 929.00 & \textbf{0.000938} *** \\ 
%     \bottomrule
%     \end{tabular}
%     \caption{ANOVA analysis of the LMER model results.}
%     \label{tab:reasoning-factors-judgment-anova}
%     \end{center}
% \end{table}

\subsection{Factors Influencing Moral Foundations in  Rationale}
In Table~\ref{tab:reasoning-fators-moral-values-annotations} we display analysis result using linear mixed effects on factors that may influence moral foundations appealed to in participants' rationales. We find greater relations with the use cases than personal factors. 
\begin{table}
\begin{center}
\small
\begin{tabular}{l c c c c c}
\hline
 & Care & Fairness & Purity & Loyalty & Authority \\
\hline
(Intercept) (Telemarketer)                                 & $\mathbf{1.74^{***} \;(0.31)}$ & $0.29 \;(0.23)$       & $\mathbf{-2.80^{***} \;(0.42)}$ & $\mathbf{-4.82^{***} \;(1.03)}$  & $\mathbf{-2.29^{***} \;(0.35)}$ \\
MFQ\_care                                   & $0.08 \;(0.19)$       & $-0.12 \;(0.13)$      & $-0.36 \;(0.19)$       & $-0.15 \;(0.43)$        & $-0.01 \;(0.19)$       \\
MFQ\_fairness                               & $0.13 \;(0.19)$       & $\mathbf{0.31^{*} \;(0.13)}$   & $0.30 \;(0.20)$        & $-0.19 \;(0.43)$        & $0.14 \;(0.19)$        \\
MFQ\_loyalty                                & $\mathbf{0.60^{**} \;(0.21)}$  & $\mathbf{0.31^{*} \;(0.14)}$   & $-0.14 \;(0.21)$       & $-0.34 \;(0.57)$        & $-0.12 \;(0.20)$       \\
MFQ\_authority                              & $\mathbf{-0.48^{*} \;(0.24)}$  & $-0.23 \;(0.16)$      & $0.23 \;(0.24)$        & $0.06 \;(0.56)$         & $0.23 \;(0.24)$        \\
MFQ\_purity                                 & $0.28 \;(0.19)$       & $-0.16 \;(0.13)$      & $-0.07 \;(0.20)$       & $-0.05 \;(0.44)$        & $-0.22 \;(0.19)$       \\
empathy\_total                              & $0.06 \;(0.15)$       & $0.04 \;(0.10)$       & $0.07 \;(0.15)$        & $0.24 \;(0.39)$         & $-0.05 \;(0.15)$       \\
InstrumentalHarm                            & $-0.04 \;(0.15)$      & $0.12 \;(0.10)$       & $-0.14 \;(0.16)$       & $-0.10 \;(0.39)$        & $0.09 \;(0.15)$        \\
ImpartialBenificence                        & $0.09 \;(0.15)$       & $-0.05 \;(0.10)$      & $\mathbf{-0.37^{*} \;(0.16)}$   & $-0.44 \;(0.41)$        & $-0.07 \;(0.15)$       \\
Gov. Eligi. Interviewer & $0.06 \;(0.39)$       & $\mathbf{1.45^{***} \;(0.35)}$ & $\mathbf{-1.72^{*} \;(0.82)}$   & \NA  & $-0.01 \;(0.44)$       \\
IT Support Specialist              & $\mathbf{1.14^{*} \;(0.46)}$   & $\mathbf{0.76^{*} \;(0.32)}$   & $0.44 \;(0.49)$        & \NA & $-0.72 \;(0.50)$       \\
Elementary School Teacher          & $\mathbf{0.88^{*} \;(0.44)}$   & $\mathbf{-0.66^{*} \;(0.31)}$  & $0.90 \;(0.47)$        & $1.43 \;(1.13)$         & $0.42 \;(0.42)$        \\
Lawyer                             & $-0.48 \;(0.37)$      & $\mathbf{0.70^{*} \;(0.32)}$   & $-0.71 \;(0.60)$       & $0.71 \;(1.24)$         & $\mathbf{1.41^{***} \;(0.40)}$  \\
Flavorful Swaps                    & $0.73 \;(0.47)$       & $-0.18 \;(0.33)$      & $\mathbf{1.50^{**} \;(0.48)}$   & $-0.04 \;(1.43)$        & $-0.80 \;(0.55)$       \\
Nutrition Optimizer                & $\mathbf{1.14^{*} \;(0.50)}$   & $-0.37 \;(0.33)$      & $-0.01 \;(0.55)$       & \NA & $-0.26 \;(0.50)$       \\
Personal Health Research           & $\mathbf{1.46^{**} \;(0.54)}$  & $0.53 \;(0.34)$       & $-0.32 \;(0.58)$       & \NA   & $0.42 \;(0.47)$        \\
Cust, Lifestyle Coach         & $0.71 \;(0.47)$       & $0.47 \;(0.34)$       & $0.56 \;(0.51)$        & $-0.04 \;(1.43)$        & $-0.64 \;(0.54)$       \\
Digital Medical Advice             & $\mathbf{1.45^{**} \;(0.53)}$  & $0.02 \;(0.33)$       & $-0.50 \;(0.60)$       & \NA & $\mathbf{1.47^{***} \;(0.44)}$  \\
\hline
AIC                                         & $750.75$     & $1256.07$    & $644.21$      & $120.53$       & $845.93$      \\
BIC                                         & $843.71$     & $1349.03$    & $737.17$      & $213.49$       & $938.89$      \\
Log Likelihood                              & $-356.38$    & $-609.04$    & $-303.11$     & $-41.26$       & $-403.97$     \\
Num. obs.                                   & $985$        & $985$        & $985$         & $985$          & $985$         \\
Num. groups: prolific\_id                   & $197$        & $197$        & $197$         & $197$          & $197$         \\
Var: prolific\_id (Intercept)               & $1.27$       & $0.63$       & $1.10$        & $0.00$         & $1.56$        \\
\hline
\multicolumn{6}{l}{\scriptsize{$^{***}p<0.001$; $^{**}p<0.01$; $^{*}p<0.05$}}
\end{tabular}
\caption{Effects and standard error in parenthesis of the annotation output of participant answers modeled with following formula \texttt{Annot$_{foundation}$ $\sim$ MFQ$_{foundation}$ + empathy + instrumentalHarm + impartialBeneficence + useCase + (1|subject)} using \texttt{glmer} with family set to binomial. Intercept shows effects when categorical variables are set to following: \texttt{useCase = Telemarketer} and \texttt{Type = Cost-Benefit}.}
\label{tab:reasoning-fators-moral-values-annotations}
\end{center}
\end{table}


\section{Survey 2: Explicit Weighing of Harms and Benefits of Use Cases}
Although not included in main text, we administered a variation of our main study where we asked participants to explicitly reason through harms and benefits. The decisions were measured before and after the explicit weighing of harms and benefits. However, we saw almost no effect. 

\subsection{Study Overview}
To better understand the reasoning behind participants decisions about the judgment and usage of the use cases, we conducted a second study with 1 survey for each category (Labor Replacement Use Cases and Personal Use Cases). The second study includes an additional set of questions to elicit the harms and benefits of developing and not developing an use case to better understand the reasoning behind participants decisions. Furthermore, we asked participants the judgment and usage decision questions before and after the set of harms and benefits questions to see if listing out reasons about an use case would elicit any change in their decisions. The details of the second study can be found in \ref{sssec:a-details} and the results can be found in \ref{sssec:a-results}.

\subsection{Setup and Details}\label{sssec:a-details}
While these same set of questions are asked for all five use cases for our main study, in our second study, participants are randomly allocated a single use case. The second study differs from the main study with an initial set of judgment questions without open-text rationales (Q1 - Initial to Q4 - Initial), which are followed by explicit listing and weighing of the possible harms and benefits of the use case in the context of both developing and not developing the use case. We then again ask participants the same set of judgment questions along with the open-text questions to elaborate on their reasoning, similar to the main study. To understand how the judgment and usage decisions are affected by other factors, we asked the participants about their demographics, ai literacy levels and several other reasoning factors after the main set of questions, and these questions can be found in \S\ref{sssec:a-questions} The main questions for the second study can be found in Table \ref{app:part-2-questions}. The participant demographics for the second study can be found in Tables \ref{app:demographics-1-jobs-p2} to \ref{app:demographics-3-personal-p2}. The distribution of each use case within each scenario (Labor Replacement Use Cases and Personal Use Cases) for the second study is relatively well-balanced and can be found in Table \ref{tab:study2-use-case}.

\begin{table}[!hbpt]
    \centering
    \small
    \begin{tabularx}{0.4\linewidth}{l|X}
    \toprule
       Use Case & Participants Allocated \\
    \midrule
       Personal Use Cases & \\
    \midrule
       Digital Medical Advice & 20 \\
       Customized Lifestyle Coach & 20 \\
       Personal Health Research & 19 \\
       Nutrition Optimizer & 21 \\
       Flavorful Swaps & 17 \\
    \midrule
       Labor Replacement Use Cases & \\
    \midrule
       Lawyer & 20 \\
       Elementary School Teacher & 22 \\
       IT Support Specialist & 17 \\
       Government Eligibility Interviewer & 19 \\
       Telemarketer & 23 \\
    \bottomrule
    \end{tabularx}
    \caption{Use Case allocation for Study 2. Specific participant numbers are listed for each use case.}
    \label{tab:study2-use-case}
\end{table}
\begin{table}[!hbpt]
    \footnotesize
    \centering
    \begin{tabularx}{\linewidth}{l|X|l}
    \toprule
         Question ID & Question & Answer Type \\
         \midrule
         \multicolumn{3}{l}{AI Perception Question (Before)}\\
         \midrule
         AI Perception Before & Overall, how does the growing presence of artificial intelligence (AI) in daily life and society make you feel? & 5 Point Likert Scale\\
         \midrule
         \multicolumn{3}{l}{Initial Decision/Usage}\\
         \midrule
         Q1 - Initial & Do you think a technology like this should be developed? & Yes/No\\
         Q2 - Initial & How confident are you in your above answer? & 5 Point Likert Scale \\
         Q3 - Initial & If [Use Case] exists, would you ever use its services (answer yes, even if you think you would use it very infrequently)? & Yes/No \\
         Q4 - Initial & How confident are you in your above answer? & 5 Point Likert Scale \\
         \midrule
         \multicolumn{3}{l}{Benefits of Developing Use Case}\\
         \midrule
         Q1 - BDev & How will [Use Case] positively impact individuals? & Text \\
         Q2 - BDev & Which groups of people do you think would benefit the most from the above positive impacts? (You can list more than one group.) & Text \\
         Q3 - BDev & How beneficial would [Use Case] be if it had the above positive impacts? & 9 Point Likert Scale \\
         \midrule
         \multicolumn{3}{l}{Malicious Uses of Developing Use Case}\\
         \midrule
         Q1 - HDev & Please complete the following: [Use Case] could have a negative impact if it was used to... & Text \\
         Q1 - HDev & What would be the negative impact of the above malicious or unintended uses? & Text \\
         Q2 - HDev & Which groups of people do you think would be harmed the most by the above malicious or unintended uses? (You can list more than one group.) & Text \\
         Q3 - HDev & How harmful would [Use Case] be if it had the above negative impacts? & 9 Point Likert Scale \\
         \midrule
         \multicolumn{3}{l}{Failures of Developing Use Case}\\
         \midrule
         Q1 - HDevF & Please complete the following: If [Use Case] failed to do its intended task properly, fully, and accurately, it could have a negative impact if it... & Text \\
         Q1 - HDevF & What would be the negative impact of those failure cases? & Text \\
         Q2 - HDevF & Which groups of people do you think would be harmed the most by the above failure cases? (You can list more than one group.) & Text \\
         Q3 - HDevF & How harmful would [Use Case] be if it had the above negative impacts? & 9 Point Likert Scale \\
         \midrule
         \multicolumn{3}{l}{Benefits of Not Developing Use Case}\\
         \midrule
         Q1 - BNonDev & Please complete the following: Not having [Use Case] would be beneficial because... & Text \\
         Q2 - BNonDev & Which groups of people do you think would benefit the most by banning or not developing [Use Case]?
(You can list more than one group.) & Text \\
         Q3 - BNonDev & How beneficial would it be if [Use Case] was banned or not developed and it had the above positive impact? & 9 Point Likert Scale \\
         \midrule
         \multicolumn{3}{l}{Harms of Not Developing Use Case}\\
         \midrule
         Q1 - HNonDev & Please complete the following: Not having [Use Case] would be harmful because... & Text \\
         Q2 - HNonDev & Which groups of people do you think would be harmed the most by banning or not developing [Use Case]? (You can list more than one group.) & Text \\
         Q3 - HNonDev & How harmful would it be if [Use Case] was banned or not developed and it had the above negative impacts? & 9 Point Likert Scale \\
         \midrule
         \multicolumn{3}{l}{Final Decision/Usage}\\
         \midrule
         Q1 - Final & Do you think a technology like this should be developed? & Yes/No\\
         Q2 - Final & How confident are you in your above answer? & 5 Point Likert Scale \\
         Q3 - Final - Y & Please elaborate on your answer to the previous question: Do you think a technology like this should be developed?: [Q1 - Final Answer] & Text \\
         Q3 - Final - N & Please elaborate on your answer to the previous question: Do you think a technology like this should be developed?: [Q1 - Final Answer] & Text \\
         Q4 - Final - Y & Under what circumstances would you switch your decision from [Q1 - Final Answer] should be developed to should not be developed? & Text \\
         Q4 - Final - N & Under what circumstances would you switch your decision from [Q1 - Final Answer] should not be developed to should be developed? & Text \\ 
         Q5 - Final & If [Use Case] exists, would you ever use its services (answer yes, even if you think you would use it very infrequently)? & Yes/No \\
         Q6 - Final & How confident are you in your above answer? & 5 Point Likert Scale \\
         \midrule
         \multicolumn{3}{l}{AI Perception Question (After)}\\
         \midrule
         AI Perception After & Before we continue, we’d like to get your thoughts on AI one more time. Overall, how does the growing presence of artificial intelligence (AI) in daily life and society make you feel? & 5 Point Likert Scale \\
    \bottomrule
    \end{tabularx}
    \caption{Study 2 Specific Question. The placeholder [Use Case] is used in place of the 10 use cases chosen for the studies.}
    \label{app:part-2-questions}
\end{table}
\begin{table*}[!htpb]
    \footnotesize
    \centering
    \begin{tabular}{{ll|ll|ll|ll}}
    \toprule
         \textbf{Racial Identity} & \textbf{\textit(N) (\%)} & \textbf{Age} & \textbf{\textit{N} (\%)} & \textbf{Gender Identity} & \textbf{\textit{N} (\%)} & \textbf{Education} & \textbf{\textit{N} (\%)} \\
         % \hline
         \midrule
        White or Caucasian & 32 (31.4) & 45-54 & 32 (31.4) & Man & 49 (49.0) & Bachelor’s degree & 44 (43.1)\\
        Black or African American & 25 (24.5) & 25-34 & 29 (28.4) & Non-male & 51 (51.0) & Graduate degree$^{*}$ & 18 (17.6)\\
        Asian & 18 (17.6) & 35-44 & 12 (11.8) & & & Some college $^{*}$ & 18 (17.6)\\
        Mixed & 15 (14.7) & 55-64 & 12 (11.8) & & & High school diploma$^{*}$ & 15 (14.7)\\
        Other & 12 (11.8) & 18-24 & 9 (8.8) & & & Associates degree$^{*}$ & 7 (6.9)\\
         & 2 (0.7) & 65+ & 8 (7.8) & & & Some high school$^{*}$ & 0 (0.0)\\
    \bottomrule
    \end{tabular}
    \caption{Labor Replacement Study 2 Survey: Racial, age, gender identities and education level of participants. Asterisk (*) denotes labels shortened due to space.}
    \label{app:demographics-1-jobs-p2}
\end{table*}
\begin{table*}[htpb]
    \centering
    \footnotesize
    \begin{tabular}{ll|ll|ll|ll}
    \toprule
         \textbf{Minority/Disadvantaged Group} & \textbf{\textit(N) (\%)} & \textbf{Transgender} & \textbf{\textit{N} (\%)} & \textbf{Sexuality} & \textbf{\textit{N} (\%)} & \textbf{Political Leaning} & \textbf{\textit{N} (\%)} \\
         % \hline
         \midrule
No & 56 (54.9) & No & 97 (95.1) & Heterosexual & 76 (74.5) & Liberal & 37 (36.3)\\
Yes & 46 (45.1) & Yes & 4 (3.9) & Others & 26 (25.5) & Moderate & 27 (26.5)\\
 &  & Prefer not to say & 1 (1.0) & & & Conservative & 17 (16.7)\\
 &  &  &  &  &  & Strongly liberal & 16 (15.7)\\
 &  &  &  &  &  & Strongly conservative & 4 (3.9)\\
 &  &  &  &  &  & Prefer not to say & 1 (1.0)\\
\bottomrule
    \end{tabular}
    \caption{Labor Replacement Study 2 Survey: Additional demographic identities}
    \label{app:demographics-2-jobs-p2}
\end{table*}
\begin{table*}[htpb]
    \centering
    \footnotesize
    \begin{tabularx}{\textwidth}{Xl|Xl|Xl|Xl}
    \toprule
        \textbf{Longest Residence} & \textbf{\textit(N) (\%)} & \textbf{Employment} & \textbf{\textit{N} (\%)} & \textbf{Occupation (Top 10)} & \textbf{\textit{N} (\%)} & \textbf{Religion} & \textbf{\textit{N} (\%)} \\
    % \hline
    \midrule
United States of America & 96 (94.1) & Employed, 40+ & 44 (43.1) & Other & 34 (33.3) & Christian & 38 (37.3)\\
Others & 6 (5.9) & Employed, 1-39 & 28 (27.5) & Educational Services & 11 (10.8) & Agnostic & 19 (18.6)\\
 &  & Not employed, looking for work & 19 (18.6) & Health Care and Social Assistance & 10 (10.0) & Catholic & 18 (17.6)\\
 &  & Retired & 4 (3.9) & Information & 8 (7.8) & Nothing in particular & 12 (11.8)\\
 &  & Not employed, NOT looking for work & 3 (2.9) & Prefer not to answer & 8 (7.8) & Atheist & 7 (6.9)\\
 &  & Other: please specify & 3 (2.9) & Retail Trade & 7 (6.9) & Muslim & 3 (2.9)\\
 &  & Disabled, not able to work & 1 (1.0) & Finance and Insurance & 7 (6.9) & Something else, Specify & 3 (2.9)\\
 &  & Prefer not to disclose & 0 (0.0) & Professional, Scientific, and Technical Services & 6 (5.9) & Jewish & 1 (1.0)\\
 &  &  &  & Manufacturing & 6 (5.9) & Hindu & 1 (1.0)\\
 &  &  &  & Administrative and support and waste management services & 5 (4.9) & Buddhist & 0 (0.0)\\
 \bottomrule
    \end{tabularx}
    \caption{Labor Replacement Study 2 Survey: Additional demographic identities. The Occupation category was capped at the top 10 for brevity, with the remaining occupations merged together with the Other: please specify option.}
    \label{app:demographics-3-jobs-p2}
\end{table*}
\begin{table*}[!htpb]
    \footnotesize
    \centering
    \begin{tabular}{{ll|ll|ll|ll}}
    \toprule
         \textbf{Racial Identity} & \textbf{\textit(N) (\%)} & \textbf{Age} & \textbf{\textit{N} (\%)} & \textbf{Gender Identity} & \textbf{\textit{N} (\%)} & \textbf{Education} & \textbf{\textit{N} (\%)} \\
         % \hline
         \midrule
        White or Caucasian & 35 (36.1) & 45-54 & 29 (29.9) & Non-male & 50 (51.5) & Bachelor’s degree & 33 (34.0)\\
        Black or African American & 22 (22.7) & 25-34 & 22 (22.7) & Man & 47 (48.5) & Graduate degree$^{*}$ & 24 (24.7)\\
        Asian & 19 (19.6) & 55-64 & 17 (17.5) & & & Some college $^{*}$ & 18 (18.6)\\
        Mixed & 13 (13.4) & 35-44 & 17 (17.5) & & & High school diploma$^{*}$ & 12 (12.4)\\
        Other & 8 (8.2) & 18-24 & 6 (6.2) & & & Associates degree$^{*}$ & 10 (10.3)\\
         & 2 (0.7) & 65+ & 6 (6.2) & & & Some high school$^{*}$ & 0 (0.0)\\
         &  & Prefer not to disclose & 0 (0.0) & & & & \\
    \bottomrule
    \end{tabular}
    \caption{Personal Use Cases Study 2 Survey: Racial, age, gender identities and education level of participants. Asterisk (*) denotes labels shortened due to space.}
    \label{app:demographics-1-personal-p2}
\end{table*}
\begin{table*}[htpb]
    \centering
    \footnotesize
    \begin{tabular}{ll|ll|ll|ll}
    \toprule
         \textbf{Minority/Disadvantaged Group} & \textbf{\textit(N) (\%)} & \textbf{Transgender} & \textbf{\textit{N} (\%)} & \textbf{Sexuality} & \textbf{\textit{N} (\%)} & \textbf{Political Leaning} & \textbf{\textit{N} (\%)} \\
         % \hline
         \midrule
No & 50 (51.5) & No & 93 (95.9) & Heterosexual & 76 (78.4) & Liberal & 34 (35.1)\\
Yes & 47 (48.5) & Yes & 4 (4.1) & Others & 21 (21.6) & Moderate & 26 (26.8)\\
 &  & Prefer not to say & 0 (0.0) & & & Strongly liberal & 17 (17.5)\\
 &  &  &  &  &  & Conservative & 13 (13.4)\\
 &  &  &  &  &  & Strongly conservative & 6 (6.2)\\
 &  &  &  &  &  & Prefer not to say & 1 (1.0)\\
\bottomrule
    \end{tabular}
    \caption{Personal Use Cases Study 2 Survey: Additional demographic identities}
    \label{app:demographics-2-personal-p2}
\end{table*}
\begin{table*}[htpb]
    \centering
    \footnotesize
    \begin{tabularx}{\textwidth}{Xl|Xl|Xl|Xl}
    \toprule
        \textbf{Longest Residence} & \textbf{\textit(N) (\%)} & \textbf{Employment} & \textbf{\textit{N} (\%)} & \textbf{Occupation (Top 10)} & \textbf{\textit{N} (\%)} & \textbf{Religion} & \textbf{\textit{N} (\%)} \\
    % \hline
    \midrule
United States of America & 95 (97.9) & Employed, 40+ & 44 (45.4) & Other & 36 (37.1) & Christian & 43 (44.3)\\
Others & 2 (2.1) & Employed, 1-39 & 23 (23.7) & Health Care and Social Assistance & 13 (13.4) & Agnostic & 12 (12.4)\\
 &  & Not employed, looking for work & 9 (9.3) & Information & 9 (9.3) & Atheist & 12 (12.4)\\
 &  & Other: please specify & 7 (7.2) & Finance and Insurance & 8 (8.2) & Catholic & 10 (10.3)\\
 &  & Retired & 6 (6.2) & Prefer not to answer & 6 (6.2) & Nothing in particular & 8 (8.2)\\
 &  & Disabled, not able to work & 5 (5.2) & Retail Trade & 6 (6.2) & Muslim & 4 (4.1)\\
 &  & Not employed, NOT looking for work & 2 (2.1) & Manufacturing & 5 (5.1) & Something else, Specify & 4 (4.1)\\
 &  & Prefer not to disclose & 1 (1.0) & Educational Services & 5 (5.1) & Buddhist & 2 (2.1)\\
 &  &  &  & Arts, Entertainment, and Recreation & 5 (5.1) & Hindu & 1 (1.0)\\
 &  &  &  & Accommodation and Food Services & 4 (4.1) & Jewish & 1 (1.0)\\
 \bottomrule
    \end{tabularx}
    \caption{Personal Use Cases Study 1 Survey: Additional demographic identities. The Occupation category was capped at the top 10 for brevity, with the remaining occupations merged together with the Other: please specify option.}
    \label{app:demographics-3-personal-p2}
\end{table*}


\subsection{Results}\label{sssec:a-results}
To explore the possible impact of explicitly weighing harms and benefits of a use case on participant's decision, we analyzed the participant's judgment of acceptability before and after explicit weighing of harms and benefits (Study 2; see \S~\ref{sssec:survey-qs} for details on questions asked). The Type III ANOVA with Satterthwaite's method for measurement time (before, after) indicated a marginally significant effect \( F(1, 201.05) = 3.371, p = 0.0678 \) on usage judgment weighed by confidence, which suggests that explicit harms and benefits weighing may have an influence, albeit not at conventional significance levels. We further analyzed reasoning effect on each subset of data pertinent to each use case through a mixed effects regression model with judgment metric as a dependent variable and measurement time as an independent variable with random effect from subject. Interestingly, the result was significant for Customized Lifestyle Coach AI across different judgments including, existence ($\beta=-0.40,SE=0.18,p<.05$), confidence-weighed existence ($\beta=-1.05,SE=0.51,p<.05$), and confidence-weighed usage judgments ($\beta=-0.75,SE=0.38,p<.05$). Explicit weighing also had a significant effect on confidence of existence judgment for Digital Medical Advice AI ($\beta=0.30,SE=0.11,p<.01$). The negative coefficients for Customized Lifestyle AI suggests that weighing harms and benefits caused participants to lower acceptance and positive coefficient to confidence on judgments on Digital Medical AI suggests that weighing harms and benefits solidified decisions. These diverging effects signify an interesting interaction between use cases and explicit weighing of harms and benefits.

% \input{tables/stakeholders-def}
% \paragraph{Stakeholders}
% Similarly, we annotated stakeholders mentioned in the participants' answers and their relations to the participant (i.e., self or other). We used a taxonomy of stakeholders inspired by previous works \citep{golpayegani2023risk} and labeled for mentions of AI user, subject, developer, deployer, regulator, and supervisor. Please refer to Table~\ref{tab:stakeholders-def} for definitions of each stakeholder category. Additionally, we annotated whether the stakeholders' relations to the participants, specifically, whether the stakeholder represented self or other to further analyze the differences in decision making processes when thinking of either self or others. 
% should i expand more on the difference between the perception of harms to self vs harms to others here or later in the discussion? also refer to the robotics study about beneficial to others here

% below are bunch of tables:
