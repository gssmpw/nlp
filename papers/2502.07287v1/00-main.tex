%%
%% This is file `sample-sigconf-authordraft.tex',
%% generated with the docstrip utility.
%%
%% The original source files were:
%%
%% samples.dtx  (with options: `all,proceedings,bibtex,authordraft')
%% 
%% IMPORTANT NOTICE:
%% 
%% For the copyright see the source file.
%% 
%% Any modified versions of this file must be renamed
%% with new filenames distinct from sample-sigconf-authordraft.tex.
%% 
%% For distribution of the original source see the terms
%% for copying and modification in the file samples.dtx.
%% 
%% This generated file may be distributed as long as the
%% original source files, as listed above, are part of the
%% same distribution. (The sources need not necessarily be
%% in the same archive or directory.)
%%
%%
%% Commands for TeXCount
%TC:macro \cite [option:text,text]
%TC:macro \citep [option:text,text]
%TC:macro \citet [option:text,text]
%TC:envir table 0 1
%TC:envir table* 0 1
%TC:envir tabular [ignore] word
%TC:envir displaymath 0 word
%TC:envir math 0 word
%TC:envir comment 0 0
%%
%%
%% The first command in your LaTeX source must be the \documentclass
%% command.
%%
%% For submission and review of your manuscript please change the
%% command to \documentclass[manuscript, screen, review]{acmart}.
%%
%% When submitting camera ready or to TAPS, please change the command
%% to \documentclass[sigconf]{acmart} or whichever template is required
%% for your publication.
%%
%%
\documentclass[manuscript]{acmart}
\usepackage{booktabs}
\usepackage{multirow}
\usepackage{graphicx}
\usepackage{amsmath}
\usepackage{xspace}
\DeclareMathOperator*{\argmax}{arg\,max}
\DeclareMathOperator*{\stmx}{Softmax}
\allowdisplaybreaks
 
\newcommand{\ours}{\textsc{suPreMe}\xspace}
\newcommand{\ITC}{image-text consistency\xspace}
\newcommand{\BPG}{biased prompts generation\xspace}
\usepackage{colortbl}

\usepackage{amsthm}

\newtheorem*{remark}{Remark}
\newtheorem{theorem}{Theorem}
\newtheorem{assumption}{Assumption}

\usepackage{makecell}

\usepackage{amsbsy}


%%
%% \BibTeX command to typeset BibTeX logo in the docs
\AtBeginDocument{%
  \providecommand\BibTeX{{%
    Bib\TeX}}}

%% Rights management information.  This information is sent to you
%% when you complete the rights form.  These commands have SAMPLE
%% values in them; it is your responsibility as an author to replace
%% the commands and values with those provided to you when you
%% complete the rights form.

\sloppy
\setcopyright{none}
\settopmatter{printacmref=false} % Removes citation information below abstract
\renewcommand\footnotetextcopyrightpermission[1]{} % removes footnote with conference information in first column
\pagestyle{plain} % removes running headers

%\setcopyright{acmlicensed}
%\copyrightyear{2018}
%\acmYear{2018}
%\acmDOI{XXXXXXX.XXXXXXX}

%% These commands are for a PROCEEDINGS abstract or paper.
%\acmConference[Conference acronym 'XX]{Make sure to enter the correct
  %conference title from your rights confirmation emai}{June 03--05,
  %2018}{Woodstock, NY}
%%
%%  Uncomment \acmBooktitle if the title of the proceedings is different
%%  from ``Proceedings of ...''!
%%
%%\acmBooktitle{Woodstock '18: ACM Symposium on Neural Gaze Detection,
%%  June 03--05, 2018, Woodstock, NY}
%\acmISBN{978-1-4503-XXXX-X/18/06}


%%
%% Submission ID.
%% Use this when submitting an article to a sponsored event. You'll
%% receive a unique submission ID from the organizers
%% of the event, and this ID should be used as the parameter to this command.
%%\acmSubmissionID{123-A56-BU3}

%%
%% For managing citations, it is recommended to use bibliography
%% files in BibTeX format.
%%
%% You can then either use BibTeX with the ACM-Reference-Format style,
%% or BibLaTeX with the acmnumeric or acmauthoryear sytles, that include
%% support for advanced citation of software artefact from the
%% biblatex-software package, also separately available on CTAN.
%%
%% Look at the sample-*-biblatex.tex files for templates showcasing
%% the biblatex styles.
%%

%%
%% The majority of ACM publications use numbered citations and
%% references.  The command \citestyle{authoryear} switches to the
%% "author year" style.
%%
%% If you are preparing content for an event
%% sponsored by ACM SIGGRAPH, you must use the "author year" style of
%% citations and references.
%% Uncommenting
%% the next command will enable that style.
%%\citestyle{acmauthoryear}


% \usepackage{draftwatermark}
% \SetWatermarkText{Under Submission}
% \SetWatermarkScale{.25} % scale of the watermark
% \SetWatermarkAngle{45} % angle of the watermark
%%
%% end of the preamble, start of the body of the document source.
\begin{document}

%%
%% The "title" command has an optional parameter,
%% allowing the author to define a "short title" to be used in page headers.
\title{Diverse Perspectives on AI: Examining People's Acceptability and Reasoning of Possible AI Use Cases}

% Navigating AI Impact Dilemma: Exmaining People's Accepstability and Reasoning of Possible AI Use Cases.



%%
%% The "author" command and its associated commands are used to define
%% the authors and their affiliations.
%% Of note is the shared affiliation of the first two authors, and the
%% "authornote" and "authornotemark" commands
%% used to denote shared contribution to the research.
\author{Jimin Mun}
\email{jmun@andrew.cmu.edu}
\affiliation{%
  \institution{Carnegie Mellon University}
  \city{Pittsburgh}
  \state{Pennsylvania}
  \country{USA}
}

\author{Wei Bin Au Yeong}
\email{wauyeong@andrew.cmu.edu}
\affiliation{%
  \institution{Carnegie Mellon University}
  \city{Pittsburgh}
  \state{Pennsylvania}
  \country{USA}
}

\author{Wesley Hanwen Deng}
\email{hanwend@andrew.cmu.edu}
\affiliation{%
  \institution{Carnegie Mellon University}
  \city{Pittsburgh}
  \state{Pennsylvania}
  \country{USA}
}

\author{Jana Schaich Borg}
\email{js524@duke.edu}
\affiliation{%
  \institution{Duke University}
  \city{Durham}
  \state{North Carolina}
  \country{USA}
}

\author{Maarten Sap}
\email{msap2@andrew.cmu.edu}
\affiliation{%
  \institution{Carnegie Mellon University}
  \city{Pittsburgh}
  \state{Pennsylvania}
  \country{USA}
}

%%
%% By default, the full list of authors will be used in the page
%% headers. Often, this list is too long, and will overlap
%% other information printed in the page headers. This command allows
%% the author to define a more concise list
%% of authors' names for this purpose.
\renewcommand{\shortauthors}{Mun et al.}

%%
%% The abstract is a short summary of the work to be presented in the
%% article.
% \begin{abstract}
% \end{abstract}

%%
%% The code below is generated by the tool at http://dl.acm.org/ccs.cfm.
%% Please copy and paste the code instead of the example below.
%%

\begin{comment}

\begin{CCSXML}
<ccs2012>
 <concept>
  <concept_id>00000000.0000000.0000000</concept_id>
  <concept_desc>Do Not Use This Code, Generate the Correct Terms for Your Paper</concept_desc>
  <concept_significance>500</concept_significance>
 </concept>
 <concept>
  <concept_id>00000000.00000000.00000000</concept_id>
  <concept_desc>Do Not Use This Code, Generate the Correct Terms for Your Paper</concept_desc>
  <concept_significance>300</concept_significance>
 </concept>
 <concept>
  <concept_id>00000000.00000000.00000000</concept_id>
  <concept_desc>Do Not Use This Code, Generate the Correct Terms for Your Paper</concept_desc>
  <concept_significance>100</concept_significance>
 </concept>
 <concept>
  <concept_id>00000000.00000000.00000000</concept_id>
  <concept_desc>Do Not Use This Code, Generate the Correct Terms for Your Paper</concept_desc>
  <concept_significance>100</concept_significance>
 </concept>
</ccs2012>
\end{CCSXML}

\ccsdesc[500]{Do Not Use This Code~Generate the Correct Terms for Your Paper}
\ccsdesc[300]{Do Not Use This Code~Generate the Correct Terms for Your Paper}
\ccsdesc{Do Not Use This Code~Generate the Correct Terms for Your Paper}
\ccsdesc[100]{Do Not Use This Code~Generate the Correct Terms for Your Paper}

    
\end{comment}

%%
%% Keywords. The author(s) should pick words that accurately describe
%% the work being presented. Separate the keywords with commas.
%\keywords{Do, Not, Us, This, Code, Put, the, Correct, Terms, for,
  %Your, Paper \maarten{don't forget to update this}}
%% A "teaser" image appears between the author and affiliation
%% information and the body of the document, and typically spans the
%% page.
% \begin{teaserfigure}
%   \includegraphics[width=\textwidth]{sampleteaser}
%   \caption{Seattle Mariners at Spring Training, 2010.}
%   \Description{Enjoying the baseball game from the third-base
%   seats. Ichiro Suzuki preparing to bat.}
%   \label{fig:teaser}
% \end{teaserfigure}

\received{20 February 2007}
\received[revised]{12 March 2009}
\received[accepted]{5 June 2009}

%%
%% This command processes the author and affiliation and title
%% information and builds the first part of the formatted document.
\begin{abstract}  
Test time scaling is currently one of the most active research areas that shows promise after training time scaling has reached its limits.
Deep-thinking (DT) models are a class of recurrent models that can perform easy-to-hard generalization by assigning more compute to harder test samples.
However, due to their inability to determine the complexity of a test sample, DT models have to use a large amount of computation for both easy and hard test samples.
Excessive test time computation is wasteful and can cause the ``overthinking'' problem where more test time computation leads to worse results.
In this paper, we introduce a test time training method for determining the optimal amount of computation needed for each sample during test time.
We also propose Conv-LiGRU, a novel recurrent architecture for efficient and robust visual reasoning. 
Extensive experiments demonstrate that Conv-LiGRU is more stable than DT, effectively mitigates the ``overthinking'' phenomenon, and achieves superior accuracy.
\end{abstract}  
\maketitle

\section{Introduction}
There are growing calls to regulate AI's development and integration into society \citep{pistilli2023stronger}. These efforts, as reflected in the EU AI Act \citep{AIAct_2023}, NIST AI Risk Management framework \citep{NIST_2021}, and recent U.S. Executive Order \citep{executiveorder2023}, have resulted in discussions about whether certain AI use cases should be pursued at all. Despite much progress in this area, it is still not clear how to determine which use cases should be pursued or more heavily regulated. Further, little is known about how lay community members, especially those from marginalized groups, feel about the development of specific AI use cases \citep{ada2023survey,suresh2024participation}. \looseness=-1

One significant challenge when evaluating the acceptability and impact of specific AI use cases\footnote{By use cases, we mean specific scenarios, applications, or problems that an AI system is designed to solve or assist within real-world contexts.} is that there can be both positive and negative effects, depending on the context\citep{mun2024participaidemocraticsurveyingframework}. For instance, while educational AI can provide affordable and accessible personal tutor, it can also lead to over-reliance of students and diminish the goal of education \cite{ChatbotTeach, zhai2024effects}. To develop a generalizable approach to making decisions about AI use cases, ideally we would understand how people resolve these conflicts. More specifically, first, it is essential to understand differences in judgments about \emph{acceptability and likely usage} across use cases, and how such judgments relate to scenario characteristics. Second, we need better understanding of the \emph{personal factors influencing these judgments}, especially as they relate to demographic differences \cite{kingsley2024investigating}. Third, we need to better understand the reasoning strategies participants use when making judgments about AI use cases, and how those strategies do or do not relate to the judgments that are ultimately made. To form governance and policy decisions that anticipate and address disagreements about the development or regulation about specific AI use cases, these understandings are crucial, especially across groups of people with diverse backgrounds, experiences, and familiarity with AI. \looseness=-1

To address these needs, in this work we examine how and why lay people judge various AI use cases as acceptable or unacceptable, asking the following research questions:
\begin{enumerate}
    \item [\textbf{RQ1}] How do judgments of acceptability vary across a set of distinct AI use cases and their characteristics?
    \item [\textbf{RQ2}] What attributes or characteristics of people explain the variation in acceptability judgments? 
    \item [\textbf{RQ3}] How do people reason through acceptability judgments of AI use cases?
\end{enumerate}
To answer these questions, we develop a survey to collect judgments and reasoning processes of 197 demographically diverse participants with varying levels of experience with AI. We ask participants to report whether a certain AI use case should be developed or not, whether they would use such a system, and ask them to provide rationales for their judgment and conditions that would cause them to change their judgments (Figure \ref{fig:survey-flow}). We examine ten different AI use cases\footnote{We focus on text-based, non-embodied, digital systems, and while we do not specifically discuss the AI user and subject, in our use case description, we follow three of the five concepts used in EU AI Act to describe high risk use cases \citep{golpayegani2023risk}: the domain, purpose, and capabilities.}. To account for differences between sectors or domains, we select use cases in two categories, professional and personal use, and vary them by required entry-level education and EU AI risk level (Table~\ref{tab:use-cases}).

To meaningfully differentiate and analyze participants' reasons and reasoning strategies, we borrow concepts from moral psychology and philosophy. We investigate participants' rationales through two distinct but sometimes overlapping reasoning patterns: cost-benefit reasoning, which assesses expected outcomes (e.g., "using AI for this task would save time"), and rule-based reasoning, which evaluates the intrinsic values of the action itself (e.g., "having humans perform this task would be inherently wrong") \citep{cushman2013action,cheung2024measuring}. We further explore the moral foundations reflected in participants' reasoning, with moral foundations theory\footnote{We used the five foundational dimensions: Care, Fairness, Loyalty, Authority, and Purity. Although these dimensions have been updated to encompass a broader range of values beyond WEIRD (White, Educated, Industrialized, Rich, and Democratic) populations \citep{atari2023morality}, we selected this version for survey brevity.} \citep{graham2011mapping,graham2008moral}. Additionally, to understand aspects of AI that raise concerns, we employ three dimensions based on prior studies \citep{solaiman2023evaluating,mun2024participaidemocraticsurveyingframework}: functionality (system capabilities like performance, bias, and privacy), usage (context of system integration, such as supervision, misuse, or unintended use), and societal impact (effects on individuals, communities, and society, such as job loss and over-reliance).

Our empirical results show general higher acceptance of personal use cases over professional. While both categories of use cases show decreased acceptance with increased entry level education and risk, professional use cases display more variability and disagreements across judgments (\textit{RQ1}). Acceptability significantly varied among demographic groups and levels of AI literacy, with lower acceptability observed particularly among non-male participants and those familiar with AI ethics (RQ2). Finally, our results show varying distribution of reasoning types across acceptability decisions with rule-based reasoning being associated with negative acceptance as well as concern for societal impact. Further qualitative analysis reveal rules such as the need for humanness in certain use cases whether it be for empathy or interaction (\textit{RQ3}). \looseness=-1

Our findings shed novel light onto the diversity of people's acceptability and reasoning of AI uses in distinct domains and risk levels. We conclude with a discussion highlighting three key implications: first, diverse methodologies are needed to effectively analyze use cases and their characteristics; second, involving diverse stakeholders is crucial for assessing the acceptability of AI applications, particularly in workplaces; and third, further investigation into human reasoning processes about AI, notably rule-based reasoning, is needed to inform consensus-building in policy making. \looseness=-1
% , highlighting the need for more diverse approaches to understanding use cases and their characteristics, the need for inclusion of diverse stakeholders in determining AI use case acceptability---especially for workplace AI uses---and the need to further explore reasoning processes used by people, especially rule-based reasoning, to provide insights into building consensus for policy making.

\section{Related Works}
\label{sec:related-works}
%\jimin{related works section is a bit long right now, shorten}

%\wesley{Working on rewriting the RW}
% New Related Work:
%\maarten{Todo: write a 1-2 sentence ``intro'' to the related work section. Something like "we briefly summarize the background and related work towards assessing acceptability, perceptions, and impact of AI use cases."}

In this section, we briefly summarize the background and related work towards assessing acceptability and impact of AI use cases. In each subsection, we highlight how our work extends prior work.

\subsection{Assessing Impact of AI}

%While incorporating AI can have positive impacts on people's lives, from automating tedious tasks \citep{} to solving humanity's issues such as climate change and cancer \citep{}, AI can also cause harms towards marginalized comminutes through biased output\citep{}, hallucination\citep{}, or the potential of replacing human labor\citep{}. 

Recent years have witnessed increasing calls from academics \cite{kieslich2023anticipating, hecht2021s, bernstein2021ethics, Neurips2020workshop, Neurips2020blog, CVPR2023EthicsGuidelines, ACL2023ethicspolicy, ICML2023EthicsGuidelines, olteanu2023responsible, ESR_Stanford}, government \cite{AIA_Adalove, NAIRRTF2023FinalReport, NAIRRTF2023Strengthening}, civil society \cite{PAI2021managing, ada2022looking, AIA_Adalove, AIML_Data_Society, metcalf2021algorithmic, reisman2018algorithmic}, and industry \cite{RAIIAguide_MSFT, RAIIAtemplate_MSFT, googleRAI, openAI_research, hecht2021s, deng2024supporting} to assess the impact of AI systems designed and developed by AI researchers and practitioners. This effort has particularly highlighted the need to understand the \textit{positive} impact while grappling with the potential \textit{negative} impact of integrating AI into certain products and services that affect people's daily lives \cite{Neurips2020workshop, Neurips2020blog, hecht2021s}. 
%For example, in 2020, the Neural Information Processing Systems (NeurIPS) conference introduced a requirement that authors ``include a section in their submissions discussing the broader impact of their work, including possible societal consequences --- both positive and negative'' \cite{Neurips2020blog}. Other major AI conferences have introduced similar requirements \cite{ACL2023ethicspolicy, CVPR2023EthicsGuidelines, ICML2023EthicsGuidelines}. More recently, FAccT also encouraged researchers working on ethical issues in AI to also include an ``adverse impact statement'' in their submissions to consider the potential negative impact of their own work \cite{olteanu2023responsible}. Many civil society organizations, such as the Ada Lovelace Institute and the Partnership on AI, have also published reports to urge AI researchers and practitioners in anticipating and addressing potential negative impact of their work\citep{}.
%\maarten{this is missing the laundry list of taxonomies that many companies and experts have been creating (e.g., Laura Weidinger, etc.); we need to at least cite those, and articulate the issues with those (expert only, perhaps too focused on the general tech instead of specific use cases)}
In response, researchers in FAccT, HCI, and AI have developed tools and processes to support AI researchers and practitioners in anticipating the impact of AI systems they developed\citep{wang2024farsight, kieslich2023anticipating, buccinca2023aha, deng2024supporting, weidinger2022taxonomy}. For example, many have developed AI impact taxonomies or checklists to help developers categorize AI impact\citep{weidinger2022taxonomy, RAIIAtemplate_MSFT}. \citeauthor{wang2024farsight} and \citeauthor{deng2024supporting} developed tools and templates to support industry AI developers and researchers in assessing the potential negative societal impact of their work, such as job displacement or stereotyping social groups. 

However, this prior work primarily focuses on supporting \textit{AI experts} rather than \textit{diverse lay people}'s impact assessments of potential AI use cases. Our work extends these prior efforts by understanding diverse (and sometimes conflicting) perspectives on both positive and negative impact of AI use cases from lay people, as a crucial step to complement the AI impact assessments conducted by AI researchers and practitioners. \looseness=-1


\subsection{Understanding People's Perceptions of AI Use Cases}
\label{ssec:understanding-}
%\maarten{need to cite our own ParticipAI lol}
Responding to the calls on meaningfully engaging lay people in assessing the impact of specific AI use cases, prior work have started to understand lay people's perceptions of AI use case \cite{buccinca2023aha, kieslich2023anticipating, mun2024participaidemocraticsurveyingframework, kingsley2024investigating}. Among other findings, this prior work revealed a substantial amount of disagreement regarding the desired behavior of AI, primarily due to the subjectivity inherent in certain tasks (e.g., toxicity detection \citep{sap2019risk, blodgett2020language}, image captioning \citep{zhao2021understanding}) and ambiguous ethical implication of decisions made by AI for certain tasks (e.g., self driving cars \citep{awad2018moral}, medical AI \citep{chen2023algorithmic}, predictive analysis \citep{barocas2016big}). Work done by 
\citeauthor{mun2024participaidemocraticsurveyingframework} highlighted that lay people can envision diverse set of harms specific to different AI use cases, complementary to those defined by experts. Another line of work also begins to examine how factors such as demographic backgrounds and previous exposure to discrimination can affect people's sensitivity towards potential AI harms \citep{kingsley2024investigating}. \looseness=-1
%For example, through an online experiment on Prolific, Kingsley et al. surfaced that participants from marginalized gender or sexual orientation groups were more sensitive towards the potential harmful impact \cite{}. 

Our work extends this prior work by examined the \textbf{detailed reasoning processes} of lay people regarding the acceptability and the \textbf{trade-offs between positive and negative impacts} of AI use cases. In particular, we draw on model decision theory framework, such as the moral foundations developed by \citeauthor{graham2008moral}, to design a survey flow (See Figure \ref{fig:survey-flow} to solicit lay people's decision-making processes (and potential moral conflicts) when assessing both the benefits and harms of concrete AI use cases. 

% \maarten{I do think it might be good to have a small section on moral decision making, giving some background on CBR vs. RBR etc.? Unless we discuss it in the methods?}

\subsection{Background: Moral Decision Making}
\label{ssec:moral-decision-making}
% \maarten{I like this section, but I would substantially condense it (max 1 paragraph), and hit the following key points:
% - Explain in 1 sentence why we have background on morality: acceptability judgments are related to moral decision making, perhaps moreso that economic decision making.
% - Many moral psych and decision making experts have studied how people navigate judgments. 
% - In our work, we draw from two main themes that emerged from this research: add 1-3 sentences explaining CBR (more like consequentialism) and RBR (more like deontology) as well as moral foundations
% }
% \jana{I like Maarten's suggestion.  Two additions: (1) I wouldn't say anything about moral vs. economic decision-making, because that's a very controversial topic. (2) I think you can get rid of most of the second paragraph.  What you need to add, though, is a sentence or two about why we choose to use CBR, RBR, and moral foundations as our main frameworks for parsing moral reasoning and judgments (as opposed to other possible frameworks or moral theories).}
Morality, characterized by diverse values across cultures and social groups, aims to suppress selfishness to facilitate social life \citep{kesebir2010morality}. To understand decision-making in AI use cases, we draw on moral psychology and dual system theory. We examine two decision-making systems: cost-benefit reasoning, which assesses outcomes and consequences, and rule-based reasoning, focusing on norms, rules, and virtues \citep{cushman2013action,cheung2024measuring}. These correspond to utilitarian reasoning (maximizing good) and deontological reasoning (duties and rights), respectively. Additionally, we apply moral foundations theory \citep{graham2008moral} to identify values and potential moral conflicts in AI development.

% our interest in more moral systems and introduction to moral values
% Unlike economic decision making which maximizes utility when presented with uncertainty in choices \citep{Simon1966}, moral decision making relies on moral systems to ground reasoning. Morality, while diverse in its definition and represented by diverse values that vary across culture or even within society by class or politics (i.e., moral pluralism), has been characterized as a system with its aim to suppress selfishness to make social life possible \citep{kesebir2010morality}. Thus, to understand how people make decisions about AI use cases, we adopt moral values developed by \citeauthor{graham2008moral}, which could illuminate the relevant values when considering development of an AI use case as a moral decision and identify possible moral conflicts. 

\begin{comment}
% --------------- Old Related Work:

\subsection{Assessing Impact of AI}
%\jimin{focus more on lay people's assessment of impact}
%\wesley{Agreed that we should highlight lay people’s AI impact assessment as the key motivation for our study! But I do think it also makes sense to include a brief paragraph mentioning the existing calls for AI impact assessments by researchers and experts (which still lack best practices), before transitioning to the need for engaging more diverse lay people in assessing AI impact.

%I can help writing this paragraph if you think this makes sense at a high level! It will be a shorter version of the "Background" section in one of my previous work on AI impact assessment: https://arxiv.org/pdf/2408.01057}
%\maarten{Yes, esp. if we adopt your new framing which starts with "There have been many calls for better impact assessments of technologies"! Let's do that!}
%\wesley{Will finish reworking this section by 01/17}

The growing usage and adoption of AI and experiences of unintended consequences and harms \citep{roose2024canai} has spurred extensive discussions about possible impact of AI. The scale of negative impacts in both currently present and anticipated harms vary widely from bias \citep{}, hallucinations \citep{}, and representational harms \citep{chien2024beyond} to existential risks \citep{bengio2023ai}, and positive impacts extend from automating tedious tasks \citep{} to solving humanity's issues such as climate change and cancer \citep{}. Many taxonomies have been created to guide and understand the risks of AI \citep{} and various tools have been developed to integrate these taxonomies into practice, from tools for developers \citep{} to forums to report and aggregate harms from AI usage \citep{}. 
% economic impact
% un-interpretable nature of ai has also made the discussion of AI impact more difficult as we do not understand and fully control this system
This uses scenario writing to understand the desirable and undesirable behaviors of AI chat bots \citep{kieslich2024myfuture}. 

AI as cultural technology \citep{lederman2024language}
% eu ai act and stuff? should we add that here?

\wesley{Given that we aim to streamline the related work, I feel like this related work section on moral decision making can be integrated as the first paragraph of 3.1.1 survey design. This can also help the reader to better connect the survey design with our theoretical background. If we think this would be a good idea, I can try moving things around!}

\subsection{Moral Decision Making}
\label{ssec:moral-decision-making}
Many works in decision making literature, especially those concerning psychology of human judgment, have aimed to characterize and have adopted a dual-system theory. The dual system frameworks often distinguish intuition versus deliberation, automaticity versus control, and emotion versus cognition \citep{cushman2013action}, but in our work, we focus on the two distinctions of cost-benefit reasoning, which focuses on the expected values of outcomes and consequences, and rule-based reasoning, which relies on values assigned to the action itself according to norms, rules, and virtues \citep{cushman2013action,cheung2024measuring}. These two types of reasoning also mirror two main reasoning types in moral reasoning: utilitarian reasoning, which aims at maximizing good and deontological reasoning, which is grounded in duties and rights, respectively. 

% our interest in more moral systems and introduction to moral values
Unlike economic decision making which maximizes utility when presented with uncertainty in choices \citep{Simon1966}, moral decision making relies on moral systems to ground reasoning. Morality, while diverse in its definition and represented by diverse values that vary across culture or even within society by class or politics (i.e., moral pluralism), has been characterized as a system with its aim to suppress selfishness to make social life possible \citep{kesebir2010morality}. Thus, to understand how people make decisions about AI use cases, we adopt moral values developed by \citeauthor{graham2008moral}, which could illuminate the relevant values when considering development of an AI use case as a moral decision and identify possible moral conflicts. 
% talk a little about moral dilemmas?

% how this all ties into our work
% - dual system: values in consequences vs action
% - moral value, used when deciding value of action or outcome
% - understanding how reasoning is done and what values are prevalent when considering either the action and outcome can help guide farther discussions about what dimensions matter when making decisions about ai and how people would be able to agree upon a system

\subsection{Factors in Decision Making about AI}
% moral decision making in AI behavior - introduce but highlight that these works did not address use cases
% those that considered use cases were mainly about perceptions of the use cases and not about the reasoning process
There is a significant amount of disagreement on desired behavior of AI due to subjectivity of certain tasks (e.g., toxicity detection \citep{}, image recommendation \citep{}) and ethical implication of decisions made by AI for certain tasks (e.g., self driving cars \citep{}, medical AI \citep{}, predictive analysis \citep{}). 
%Since AI systems rely on human annotated data for training, model behaviors are also heavily influenced by them \citep{}. Many works, thus, 
Prior work have explored the disagreements and decision making factors in data annotation for subjective tasks such as impact of moral and cultural values in annotating offensiveness of AI's output \citep{davani2024disentangling}. More direct assessment of AI's behavior has also been studied and factors that influence its acceptability such as demographic factors \citep{kingsley2024investigating} and moral and cultural values \citep{brailsford2024exploring}. Notably, 
%focusing on decision making in morally challenging scenarios where distribution of well-being and harms are also decided upon by AI, 
\citeauthor{awad2018moral} have also studied acceptability of machine behavior varying descriptive factors, surfacing that .... 

We expand these prior work by focusing not only on reasoning processes of weighing such factors applied to AI use cases but also surfacing which factors are relevant when it comes to a variety of AI usage and domain.

\end{comment}

%As application area of AI has widened and models more generalized (e.g., foundation models), there has been increasing interest to understand decisions regarding use cases. While use case level decision making require more extensive consideration, \citeauthor{mun2024participaidemocraticsurveyingframework} explored demographic factors and AI literacy in acceptability of AI use cases and \citeauthor{kieslich2024myfuture} performed exploratory analysis on demographic factors and AI attitude on impact anticipation. 
% include the moral machine paper
% with the moral machines paper add discussion about ethical dilemmas
% add some more stuff about ai literacy

% \begin{figure}[hbtp]
%     \centering
%     \includegraphics[width=0.5\columnwidth,draft]{figures/survey-flow.pdf}
%     \caption{Caption}
%     \label{fig:survey-flow}
% \end{figure}

\begin{figure}[hbtp]
    \centering
    \includegraphics[width=\linewidth]{figures/particip-ai-p2-figure_v2.pdf}
    \caption{Five professional or personal use cases are presented in a random order. For each use case, we ask multiple-choice questions about its development and confidence levels (Q1, Q2), free-text questions on rationale and decision-switching conditions (Q3, Q4), and multiple-choice questions on usage and confidence (Q5, Q6). These are followed by questions on AI literacy and demographics.}
    \label{fig:survey-flow}
\end{figure}

% \section{Study Design and Data Collection}
% \label{sec:study-design}
% To understand the decision making factors and reasoning patterns of diverse population regarding AI use cases, we crafted two studies using ten different use cases. In this section, we discuss survey designs for the two studies (\S~\ref{ssec:survey-design}) and data collection details and participant demographics (\S~\ref{ssec:data-and-demographics}).
% %\maarten{Somewhere, maybe here (or maybe in 3.2?) we should define what we mean by a use case? I liked the way that the farsight paper defined it, maybe we can cite their paper? I'm thinking just a sentence would suffice?} \wesley{+1, I think we can even add a footnote when it first appears.}

% \maarten{The current flow is a little broken-up imo; we first give an overview of the studies, then explain the use cases, then give a detailed breakdown of studies. Why not just give a shorter overview of the study here (1 maybe 2 sentences per study), and then making 3.1 Use cases, and 3.2 Study Design, and 3.3 Data collection?}


% \subsection{Survey Design}
% \label{ssec:survey-design}
% \subsubsection{Overview}
% To understand how people make judgments about AI use cases, we designed two studies that collect acceptability and usage judgments, which were administered to two separate groups of participants. In both studies, we ask participants to make judgments on 1) whether the use case should exist or not and 2) whether the participant would use the application if it existed. However, the two studies differ in their elicitation of the participant's reasoning process. 

% \paragraph{Study 1: Perception of Five AI Use Cases}
% To understand the unprompted, immediate reasoning process of participants, the first study collects use case judgments without further guidance by showing participants five different use case descriptions. After reading a brief description of the use case, the participants are asked to choose whether the application should be developed or not. The reasoning process is collected through an open text question that asks participants to 1) elaborate on their decisions and 2) provide a condition in which they would switch the decision. Thus, the first question asks about the main reasons for their decision and the second question elicits the primary concern when considering the counterfactual decision. 
% % We repeat the same sets for 5 different use cases for each participant to account for the variability in the individuals and to assess the use case factors in judgment within subjects as well as between subjects (RQ4; Q\#).  

% \paragraph{Study 2: Guided Weighing of a Single Use Case}
% To understand the impact to the decision making process of participants when specifically asked to weigh the possible harms and benefits of a use case (\reasoningeffect), the second study utilizes the framework developed by \citet{mun2024participaidemocraticsurveyingframework}, which contains a comprehensive set of questions to ask participants about positive and negative impacts of use case development as well as the impacts of not developing the application. In our survey, we show participants a single, pre-determined use case drawn from the same set of use cases from the first study. While we collect judgments both before and after the explicit harms and benefits consideration, we ask participants to elaborate on their decisions and to write the conditions for switching their decisions once after the questions about harms and benefits.

% \subsubsection{Use Cases}
% \maarten{This paragraph is a little circuitous and long? I would be more to-the-point; Just say we focus on two distinct categories of use cases: personal health and labor replacement. We choose these two domains because [justify why we chose these two individually]. Together, [justify why these two domains can give us insights because they are distinct...].}
% To consider diverse impacts and usage of AI, we first carefully crafted ten use cases. To understand how different characteristics of an AI use case can impact judgments and decision making processes, we first chose two broad areas of AI involvement: AI in personal, everyday usage and AI in labor replacement, which were chosen with aims to understand how different areas of impact (e.g., in private life and society) and the roles of the participants in their interaction with the AI (e.g., AI user and AI subject or indirect stakeholder) impact the decision process. Both personal, everyday use cases and labor replacement use cases were prevalent characteristics of AI use cases discussed by the public \citep{kieslich2024myfuture,mun2024participaidemocraticsurveyingframework}. Additionally, these two areas of focus were chosen due to the familiarity of their functionality (e.g., AI lawyer) compared to other use cases (e.g., AI for climate change). We then crafted five use cases per category by varying EU AI risk level for the personal use cases and required entry level education for the labor replacement use cases. While we explore multiple domains for labor replacement use cases, we chose to focus on health domain for personal use cases reflecting the public interest \citep{mun2024participaidemocraticsurveyingframework,kieslich2024myfuture}. 
% % In our description, we focus on text-based, non-embodied, digital systems, and while we do not specifically discuss the AI user and subject, in our use case description, we follow three of the five concepts used in EU AI Act to describe high risk use cases \citep{golpayegani2023risk}: the domain, purpose, and capabilities.
% % discuss and define AI subject vs AI user

% \begin{table}[!hbpt]
    \centering
    \footnotesize
    \begin{tabular}{llp{0.4\linewidth}}
    \hline
         Use Case & Factor & Description \\
         \hline
         \multicolumn{3}{l}{\textbf{Professional Use Cases}}\\
         \quad Lawyer & Doctoral/Professional Degree & Advises clients on digital legal proceedings/transactions. \\
         \quad Elementary School Teacher & Bachelor’s degree & Teaches academic skills at the elementary school level. \\
         \quad IT Support Specialist & Some college, no degree & Maintains computer networks and provides technical help. \\
         \quad Government Eligibility Interviewer & High school diploma & Determine eligibility for government programs/resources. \\
         \quad Telemarketer & No formal education & Solicits donations or orders over the telephone. \\
         \hline
         \multicolumn{3}{l}{\textbf{Personal Use Cases}}\\
         \quad Digital Medical Advice & High Risk & Provide medical assessments prior to medical consultations. \\
         \quad Customized Lifestyle Coach & High / Limited Risk & Personalized advice for healthy living and wellness. \\
         \quad Personal Health Research & Limited Risk & Summarizes research related to personal health issues. \\
         \quad Nutrition Optimizer & Limited / Low Risk & Personalize meals and optimize nutritional intake. \\
         \quad Flavorful Swaps & Low Risk & Suggest delicious and healthy alternatives food options. \\
    \hline
    \end{tabular}
    \caption{Use cases selected for our study by categories. Use case descriptions were shortened for brevity.}
    \label{tab:use-cases}
\end{table}
% \paragraph{Professional Use Case Scenarios} 
% For the first area of focus, AI in labor replacement, we collect jobs listed in the U.S. census bureau\footnote{} and sort them according to entry level education required as stated in the census. Education level has been tightly linked to socioeconomic status \citep{} and occupational status \citep{}, which signals the level of expertise and trust \citep{svensson2006professional,evetts2006introduction}. We select jobs that have a large portion of digital or intellectual components with minimal requirement for embodiment. We select the following five labor roles: Lawyer, Elementary school teacher, IT support specialist, Government support eligibility interviewer, and Telemarketer. See Table~\ref{tab:use-cases} for further details.

% \paragraph{Personal Use Case Scenarios} 
% To comprehensively understand the acceptability of different applications in personal and private life, we varied the risk levels in the use cases following the EU AI Act to high risk, high / limited, limited, limited / low, and low risk. Furthermore, to confirm that the AI risk levels were reflected in the description, we ensured that the categories assigned by GPT-4 following \citeauthor{herdel2024exploregen} agreed with the research team's assignment. For these use cases, we adapted the descriptions of the systems written by participants from prior works \citep{mun2024participaidemocraticsurveyingframework,kieslich2024myfuture}. See Table~\ref{tab:use-cases} for further details.

% \subsubsection{Survey Questions}
% \label{sssec:survey-qs}
% % some things might be better to say in the overview than here?
% \maarten{I'm wondering if we shouldn't just give details for study 1 first in its own paragraph; and then for study 2, explain what you asked, but highlight the diff. Also, overview figures (I'm imagining very wide but narrow) would be very useful here.}
% In both surveys, we ask participants to read one or more descriptions of AI use cases and to make two judgments: 1) ``Do you think a technology like this should exist?'' (Q1) and 2) ``If the <\textit{use case}> exists, would you use its services?'' (Q5). A question to indicate their level of confidence is asked following each question (Q2, Q6). As discussed above, the two surveys differ in eliciting reasoning. The participants are asked to both elaborate on their decisions (Q3) and specify the conditions under which they would switch their decisions (Q4). While these same set of questions are asked for all five use cases for Study 1, in Study 2, we further ask participants to weigh the harms and benefits of the use case in the context of both developing and not developing it. We then again ask participants the same set of judgment questions (Q1-2, Q5-6) along with the same open-text questions to elaborate on their reasoning (Q3, Q4).

% \paragraph{Harms and Benefits (Study 2)} 
% To gather explicit weighing of harms and benefits of a use case, we ask participants to write in free-text the positive and negative impacts, the groups that would be harmed or benefited the most, and the scale of such impacts. We repeat this process for both scenarios of developing and not developing the use case. To minimize the ordering effect, we randomize the order in which participants answer questions about the scenarios, developing and not developing, harms and benefits, and the types of harms of developing (functionality error or misuse).

% \paragraph{AI Literacy and Demographics}
% \maarten{In the content below, you should say explicitly "we use the scale by... " or something like that, for AI literacy as well as experience being discriminated against}
% Following the main survey, we asked participants questions about their AI literacy and demographics, to explore both the demographic factors and AI literacy that affect decision making of use cases (\demofactor). In the AI literacy section of the survey, we asked participants to indicate their familiarity with AI in awareness, usage, evaluation, and ethics as well as frequency of AI usage and knowledge of their shortcomings \citep{wang2023measuring,mun2024participaidemocraticsurveyingframework}. In addition to the demographic information of the participants such as race, sex, age, employment status, income, and level of education, we also collected information about their discrimination experiences \citep{kingsley2024investigating}. 

% \paragraph{Questionnaires}
% To better understand the decision making styles of participants that could inform AI use case decision making (\reasoningfactor), we adopted three questionnaires in our survey: Moral foundations questionnaire \citep{graham2008moral}, Oxford Utilitarianism Scale \citep{kahane2018beyond}, and Toronto empathy questionnaire \citep{spreng2009toronto}. These questionnaires were included at the end of the survey so as not to influence the decisions of the participants in the main portion of the survey. 

% \subsection{Data Collection and Participant Demographics}
% \label{ssec:data-and-demographics}
% \maarten{Say something about pay and IRB approval. Also simplified race sounds weird, maybe say that it's a Prolific category?} 
% \jimin{addressed}
% We used Prolific\footnote{} to recruit participants. To represent diverse sample, we stratified our recruitment by simplified Prolific ethnicity categories (White, Mixed, Asian, Black and Other) and age (18-48, 49-100). We also added criteria for quality such as survey approval rating, previous number of surveys, etc. Our study was approved by IRB, and we paid 12 USD/hour, adjusting post-hoc for older age group, which took longer time to complete the surveys. Our final sample consisted of...

\subsection{Greedies}
We have two greedy methods that we're using and testing, but they both have one thing in common: They try every node and possible resistances, and choose the one that results in the largest change in the objective function.

The differences between the two methods, are the function. The first one uses the median (since we want the median to be >0.5, we just set this as our objective function.)

Second one uses a function to try to capture more nuances about the fact that we want the median to be over 0.5. The function is:

\[
\text{score}(\text{opinion}) =
\begin{cases} 
\text{maxScore}, & \text{if } \text{opinion} \geq 0.5 \\
\min\left(\frac{50}{0.5 - \text{opinion}}, \frac{\text{maxScore}}{2}\right), & \text{if } \text{opinion} < 0.5 
\end{cases}
\] 

Where we set maxScore to be $10000$.

\subsection{find-c}
Then for the projected methods where we use Huber-Loss, we also have a $find-c$ version (temporary name). This method initially finds the c for the rest of the run. 

The way it does it it randomly perturbs the resistances and opinions of every node, then finds the c value that most closely approximates the median for all of the perturbed scenarios (after finding the stable opinions). 

\section{Analysis} \label{sec:analysis}
In this section, we provide a comprehensive analysis of Satori. First, we demonstrate that Satori effectively leverages self-reflection to seek better solutions and enhance its overall reasoning performance. Next, we observe that Satori exhibits test-scaling behavior through RL training, where it progressively acquires more tokens to improve its reasoning capabilities. Finally, we conduct ablation studies on various components of Satori's training framework. Additional results are provided in Appendix~\ref{app:results}.



\paragraph{COAT Reasoning v.s. CoT Reasoning.}
\begin{table}[h]
  \begin{center}
  \scriptsize
  \captionsetup{font=small}
  \caption{\textbf{COAT Training v.s. CoT Training.} Qwen-2.5-Math-7B trained with COAT reasoning format (Satori-Qwen-7B) outperforms the same base model but trained with classical CoT reasoning format (Qwen-7B-CoT)}
  \setlength{\tabcolsep}{1.3pt}
  \begin{tabular}{cccccccccc}
    \toprule
    \textbf{Model} & \textbf{GSM8K} & \textbf{MATH500}  &  \textbf{Olym.} & \textbf{AMC2023} & \textbf{AIME2024} \\
    \midrule
    Qwen-2.5-Math-7B-Instruct & 95.2 & 83.6 &41.6& 62.5 &16.7 \\
    Qwen-7B-CoT (SFT+RL) & 93.1 & 84.4  &	42.7 &	60.0 & 10.0 \\
    \midrule
    \textbf{Satori-Qwen-7B}  & 93.2 & 85.6  & 46.6  & 67.5  & 20.0 \\
    \bottomrule
  \end{tabular}
  \label{table:ablation-coat}
  \end{center}
\vspace{-1em}
\end{table}
We begin by conducting an ablation study to demonstrate the benefits of COAT reasoning compared to the classical CoT reasoning. Specifically, starting from the synthesis of demonstration trajectories in the format tuning stage, we ablate the ``reflect'' and  ``explore'' actions, retaining only the ``continue'' actions. Next, we maintain all other training settings, including the same amount of SFT and RL data and consistent hyper-parameters. This results in a typical CoT LLM (Qwen-7B-CoT) without self-reflection or self-exploration capabilities. As shown in Table~\ref{table:ablation-coat}, the performance of Qwen-7B-CoT is suboptimal compared to Satori-Qwen-7B and fails to surpass Qwen-2.5-Math-7B-Instruct, suggesting the advantages of COAT reasoning over CoT reasoning.



\paragraph{Satori Exhibits Self-correction Capability.}
% Please add the following required packages to your document preamble:
% \usepackage{multirow}
\begin{table}[h]
\scriptsize
\captionsetup{font=small}
\caption{\textbf{Satori's Self-correction Capability.} T$\rightarrow$F: negative self-correction; F$\rightarrow$T: positive self-correction.}
\setlength{\tabcolsep}{5pt}
\begin{tabular}{lcccccc}
\toprule
\multirow{3}{*}{\textbf{Model}} & \multicolumn{4}{c}{\textbf{In-Domain}}                                                                            & \multicolumn{2}{c}{\textbf{Out-of-Domain}}              \\ \cmidrule[0.2pt]{2-7} 
                                & \multicolumn{2}{c}{\textbf{MATH500}}                    & \multicolumn{2}{c}{\textbf{OlympiadBench}}              & \multicolumn{2}{c}{\textbf{MMLUProSTEM}}         \\
                                & \textbf{T$\rightarrow$F} & \textbf{F$\rightarrow$T} & \textbf{T$\rightarrow$F} & \textbf{F$\rightarrow$T} & \textbf{T$\rightarrow$F} & \textbf{F$\rightarrow$T} \\ \midrule[0.5pt]
Satori-Qwen-7B-FT                  & 79.4\%                    & 20.6\%                    & 65.6\%                    & 34.4\%                    & 59.2\%                    & 40.8\%                    \\
\textbf{Satori-Qwen-7B}                     & 39.0\%                       & 61.0\%                       & 42.1\%                    & 57.9\%                    & 46.5\%                    & 53.5\%                    \\ \bottomrule
\end{tabular}
\label{table:finegrain-reflect}
\end{table}
We observe that Satori frequently engages in self-reflection during the reasoning process (see demos in Section~\ref{sec:demo}), which occurs in two scenarios: (1) it triggers self-reflection at intermediate reasoning steps, and (2) after completing a problem, it initiates a second attempt through self-reflection. We focus on quantitatively evaluating Satori's self-correction capability in the second scenario. Specifically, we extract responses where the final answer before self-reflection differs from the answer after self-reflection. We then quantify the percentage of responses in which Satori's self-correction is positive (i.e., the solution is corrected from incorrect to correct) or negative (i.e., the solution changes from correct to incorrect). The evaluation results on in-domain datasets (MATH500 and Olympiad) and out-of-domain datasets (MMLUPro) are presented in Table~\ref{table:finegrain-reflect}. First, compared to Satori-Qwen-FT which lacks the RL training stage, Satori-Qwen demonstrates a significantly stronger self-correction capability. Second, we observe that this self-correction capability extends to out-of-domain tasks (MMLUProSTEM). These results suggest that RL plays a crucial role in enhancing the model's true reasoning capabilities.


\paragraph{RL Enables Satori with Test-time Scaling Behavior.}
\begin{figure}[h]
    \centering
    \includegraphics[width=0.5\textwidth]{Figures/rm_shaping_tot_len.pdf}
    \vspace{-2em}
\caption{\textbf{Policy Training Acc. \& Response length v.s. RL Train-time Compute.} Through RL training, Satori learns to improve its reasoning performance through longer thinking.}
\label{fig:test_time_scaling}
\end{figure}
\begin{figure}[h]
    \centering
    \includegraphics[width=0.45\textwidth]{Figures/length_across_levels.pdf}
    \vspace{-1.5em}
\caption{\textbf{Above: Test-time Response Length v.s. Problem Difficulty Level; Below: Test-time Accuracy v.s. Problem Difficulty Level.} Compared to FT model (Satori-Qwen-FT), Satori-Qwen uses more test-time compute to tackle more challenging problems.}
\label{fig:difficulty_level}
\vspace{-1em}
\end{figure}

Next, we aim to explain how reinforcement learning (RL) incentivizes Satori's autoregressive search capability. First, as shown in Figure~\ref{fig:test_time_scaling}, we observe that Satori consistently improves policy accuracy and increases the average length of generated tokens with more RL training-time compute. This suggests that Satori learns to allocate more time to reasoning, thereby solving problems more accurately. One interesting observation is that the response length first decreases from 0 to 200 steps and then increases. Upon a closer investigation of the model response, we observe that in the early stage, our model has not yet learned self-reflection capabilities. During this stage, RL optimization may prioritize the model to find a shot-cut solution without redundant reflection, leading to a temporary reduction in response length. However, in later stage, the model becomes increasingly good at using reflection to self-correct and find a better solution, leading to a longer response length.
 
Additionally, in Figure~\ref{fig:difficulty_level}, we evaluate Satori's test accuracy and response length on MATH datasets across different difficulty levels. Interestingly, through RL training, Satori naturally allocates more test-time compute to tackle more challenging problems, which leads to consistent performance improvements compared to the format-tuned (FT) model.



\paragraph{Large-scale FT v.s. Large-scale RL.}
\begin{table}[h]
  \begin{center}
  \scriptsize
  \captionsetup{font=small}
  \caption{\textbf{Large-scale FT V.S. Large-scale RL} Satori-Qwen (10K FT data + 300K RL data) outperforms same base model Qwen-2.5-Math-7B trained with 300K FT data (w/o RL) across all math and out-of-domain benchmarks.}
  \setlength{\tabcolsep}{1.15pt}
  \vspace{-0.5em}
\begin{tabular}{lccccc}
\toprule
\textbf{(In-domain)}   & \textbf{GSM8K}   & \textbf{MATH500} & \textbf{Olym.} & \textbf{AMC2023} & \textbf{AIME2024} \\ \midrule
Qwen-2.5-Math-7B-Instruct & 95.2 & 83.6                     & 41.6                  & 62.5             & 16.7                 \\
Satori-Qwen-7B-FT (300K)     & 92.3 & 78.2                       & 40.9           & 65.0               & 16.7              \\
\textbf{Satori-Qwen-7B}         & 93.2        & 85.6                     & 46.6           & 67.5             & 20.0                \\ \midrule
\textbf{(Out-of-domain)}  & \textbf{BGQA}    & \textbf{CRUX}  & \textbf{STGQA} & \textbf{TableBench}   & \textbf{STEM}     \\ \midrule
Qwen-2.5-Math-7B-Instruct & 51.3             & 28.0             & 85.3           & 36.3             & 45.2              \\
Satori-Qwen-7B-FT (300K)     & 50.5             & 29.5           & 74.0             & 35.0               & 47.8              \\
\textbf{Satori-Qwen-7B}               & 61.8             & 42.5           & 86.3           & 43.4             & 56.7              \\ \bottomrule
\end{tabular}
  \label{table:ablation-ft-rl}
  \end{center}
\end{table}
We investigate whether scaling up format tuning (FT) can achieve performance gains comparable to RL training. We conduct an ablation study using Qwen-2.5-Math-7B, trained with an equivalent amount of FT data (300K). As shown in Table~\ref{table:ablation-ft-rl}, on the math domain benchmarks, the model trained with large-scale FT (300K) fails to match the performance of the model trained with small-scale FT (10K) and large-scale RL (300K). Additionally, the large-scale FT model performs significantly worse on out-of-domain tasks, demonstrates RL’s advantage in generalization.


\paragraph{Distillation Enables Weak-to-Strong Generalization.} 
\begin{figure}[!t]
    \centering
     \includegraphics[width=0.4\textwidth]
     {Figures/distillation.pdf}
     \vspace{-1.5em}
\caption{\textbf{Format Tuning v.s. Distillation.} Distilling from a Stronger model (Satori-Qwen-7B) to weaker base models (Llama-8B and Granite-8B) are more effective than directly applying format tuning on weaker base models.}
\label{fig:distill}
\vspace{-1em}
\end{figure}
Finally, we investigate whether distilling a stronger reasoning model can enhance the reasoning performance of weaker base models. Specifically, we use Satori-Qwen-7B to generate 240K synthetic data to train weaker base models, Llama-3.1-8B and Granite-3.1-8B. For comparison, we also synthesize 240K FT data (following Section~\ref{subsec:format}) to train the same models. We evaluate the average test accuracy of these models across all math benchmark datasets, with the results presented in Figure~\ref{fig:distill}. The results show that the distilled models outperform the format-tuned models. 

This suggests a new, efficient approach to improve the reasoning capabilities of weaker base models: (1) train a strong reasoning model through small-scale
FT and large-scale RL (our Satori-Qwen-7B) and (2) distill the strong reasoning capabilities of the model into weaker base models. Since RL only requires answer labels as supervision, this approach introduces minimal costs for data synthesis, i.e., the costs induced by a multi-agent data synthesis framework or even more expensive human annotation.



\section{Discussion of Assumptions}\label{sec:discussion}
In this paper, we have made several assumptions for the sake of clarity and simplicity. In this section, we discuss the rationale behind these assumptions, the extent to which these assumptions hold in practice, and the consequences for our protocol when these assumptions hold.

\subsection{Assumptions on the Demand}

There are two simplifying assumptions we make about the demand. First, we assume the demand at any time is relatively small compared to the channel capacities. Second, we take the demand to be constant over time. We elaborate upon both these points below.

\paragraph{Small demands} The assumption that demands are small relative to channel capacities is made precise in \eqref{eq:large_capacity_assumption}. This assumption simplifies two major aspects of our protocol. First, it largely removes congestion from consideration. In \eqref{eq:primal_problem}, there is no constraint ensuring that total flow in both directions stays below capacity--this is always met. Consequently, there is no Lagrange multiplier for congestion and no congestion pricing; only imbalance penalties apply. In contrast, protocols in \cite{sivaraman2020high, varma2021throughput, wang2024fence} include congestion fees due to explicit congestion constraints. Second, the bound \eqref{eq:large_capacity_assumption} ensures that as long as channels remain balanced, the network can always meet demand, no matter how the demand is routed. Since channels can rebalance when necessary, they never drop transactions. This allows prices and flows to adjust as per the equations in \eqref{eq:algorithm}, which makes it easier to prove the protocol's convergence guarantees. This also preserves the key property that a channel's price remains proportional to net money flow through it.

In practice, payment channel networks are used most often for micro-payments, for which on-chain transactions are prohibitively expensive; large transactions typically take place directly on the blockchain. For example, according to \cite{river2023lightning}, the average channel capacity is roughly $0.1$ BTC ($5,000$ BTC distributed over $50,000$ channels), while the average transaction amount is less than $0.0004$ BTC ($44.7k$ satoshis). Thus, the small demand assumption is not too unrealistic. Additionally, the occasional large transaction can be treated as a sequence of smaller transactions by breaking it into packets and executing each packet serially (as done by \cite{sivaraman2020high}).
Lastly, a good path discovery process that favors large capacity channels over small capacity ones can help ensure that the bound in \eqref{eq:large_capacity_assumption} holds.

\paragraph{Constant demands} 
In this work, we assume that any transacting pair of nodes have a steady transaction demand between them (see Section \ref{sec:transaction_requests}). Making this assumption is necessary to obtain the kind of guarantees that we have presented in this paper. Unless the demand is steady, it is unreasonable to expect that the flows converge to a steady value. Weaker assumptions on the demand lead to weaker guarantees. For example, with the more general setting of stochastic, but i.i.d. demand between any two nodes, \cite{varma2021throughput} shows that the channel queue lengths are bounded in expectation. If the demand can be arbitrary, then it is very hard to get any meaningful performance guarantees; \cite{wang2024fence} shows that even for a single bidirectional channel, the competitive ratio is infinite. Indeed, because a PCN is a decentralized system and decisions must be made based on local information alone, it is difficult for the network to find the optimal detailed balance flow at every time step with a time-varying demand.  With a steady demand, the network can discover the optimal flows in a reasonably short time, as our work shows.

We view the constant demand assumption as an approximation for a more general demand process that could be piece-wise constant, stochastic, or both (see simulations in Figure \ref{fig:five_nodes_variable_demand}).
We believe it should be possible to merge ideas from our work and \cite{varma2021throughput} to provide guarantees in a setting with random demands with arbitrary means. We leave this for future work. In addition, our work suggests that a reasonable method of handling stochastic demands is to queue the transaction requests \textit{at the source node} itself. This queuing action should be viewed in conjunction with flow-control. Indeed, a temporarily high unidirectional demand would raise prices for the sender, incentivizing the sender to stop sending the transactions. If the sender queues the transactions, they can send them later when prices drop. This form of queuing does not require any overhaul of the basic PCN infrastructure and is therefore simpler to implement than per-channel queues as suggested by \cite{sivaraman2020high} and \cite{varma2021throughput}.

\subsection{The Incentive of Channels}
The actions of the channels as prescribed by the DEBT control protocol can be summarized as follows. Channels adjust their prices in proportion to the net flow through them. They rebalance themselves whenever necessary and execute any transaction request that has been made of them. We discuss both these aspects below.

\paragraph{On Prices}
In this work, the exclusive role of channel prices is to ensure that the flows through each channel remains balanced. In practice, it would be important to include other components in a channel's price/fee as well: a congestion price  and an incentive price. The congestion price, as suggested by \cite{varma2021throughput}, would depend on the total flow of transactions through the channel, and would incentivize nodes to balance the load over different paths. The incentive price, which is commonly used in practice \cite{river2023lightning}, is necessary to provide channels with an incentive to serve as an intermediary for different channels. In practice, we expect both these components to be smaller than the imbalance price. Consequently, we expect the behavior of our protocol to be similar to our theoretical results even with these additional prices.

A key aspect of our protocol is that channel fees are allowed to be negative. Although the original Lightning network whitepaper \cite{poon2016bitcoin} suggests that negative channel prices may be a good solution to promote rebalancing, the idea of negative prices in not very popular in the literature. To our knowledge, the only prior work with this feature is \cite{varma2021throughput}. Indeed, in papers such as \cite{van2021merchant} and \cite{wang2024fence}, the price function is explicitly modified such that the channel price is never negative. The results of our paper show the benefits of negative prices. For one, in steady state, equal flows in both directions ensure that a channel doesn't loose any money (the other price components mentioned above ensure that the channel will only gain money). More importantly, negative prices are important to ensure that the protocol selectively stifles acyclic flows while allowing circulations to flow. Indeed, in the example of Section \ref{sec:flow_control_example}, the flows between nodes $A$ and $C$ are left on only because the large positive price over one channel is canceled by the corresponding negative price over the other channel, leading to a net zero price.

Lastly, observe that in the DEBT control protocol, the price charged by a channel does not depend on its capacity. This is a natural consequence of the price being the Lagrange multiplier for the net-zero flow constraint, which also does not depend on the channel capacity. In contrast, in many other works, the imbalance price is normalized by the channel capacity \cite{ren2018optimal, lin2020funds, wang2024fence}; this is shown to work well in practice. The rationale for such a price structure is explained well in \cite{wang2024fence}, where this fee is derived with the aim of always maintaining some balance (liquidity) at each end of every channel. This is a reasonable aim if a channel is to never rebalance itself; the experiments of the aforementioned papers are conducted in such a regime. In this work, however, we allow the channels to rebalance themselves a few times in order to settle on a detailed balance flow. This is because our focus is on the long-term steady state performance of the protocol. This difference in perspective also shows up in how the price depends on the channel imbalance. \cite{lin2020funds} and \cite{wang2024fence} advocate for strictly convex prices whereas this work and \cite{varma2021throughput} propose linear prices.

\paragraph{On Rebalancing} 
Recall that the DEBT control protocol ensures that the flows in the network converge to a detailed balance flow, which can be sustained perpetually without any rebalancing. However, during the transient phase (before convergence), channels may have to perform on-chain rebalancing a few times. Since rebalancing is an expensive operation, it is worthwhile discussing methods by which channels can reduce the extent of rebalancing. One option for the channels to reduce the extent of rebalancing is to increase their capacity; however, this comes at the cost of locking in more capital. Each channel can decide for itself the optimum amount of capital to lock in. Another option, which we discuss in Section \ref{sec:five_node}, is for channels to increase the rate $\gamma$ at which they adjust prices. 

Ultimately, whether or not it is beneficial for a channel to rebalance depends on the time-horizon under consideration. Our protocol is based on the assumption that the demand remains steady for a long period of time. If this is indeed the case, it would be worthwhile for a channel to rebalance itself as it can make up this cost through the incentive fees gained from the flow of transactions through it in steady state. If a channel chooses not to rebalance itself, however, there is a risk of being trapped in a deadlock, which is suboptimal for not only the nodes but also the channel.

\section{Conclusion}
This work presents DEBT control: a protocol for payment channel networks that uses source routing and flow control based on channel prices. The protocol is derived by posing a network utility maximization problem and analyzing its dual minimization. It is shown that under steady demands, the protocol guides the network to an optimal, sustainable point. Simulations show its robustness to demand variations. The work demonstrates that simple protocols with strong theoretical guarantees are possible for PCNs and we hope it inspires further theoretical research in this direction.
% \section{Conclusion}

\bibliographystyle{ACM-Reference-Format}
\bibliography{custom}
\subsection{Lloyd-Max Algorithm}
\label{subsec:Lloyd-Max}
For a given quantization bitwidth $B$ and an operand $\bm{X}$, the Lloyd-Max algorithm finds $2^B$ quantization levels $\{\hat{x}_i\}_{i=1}^{2^B}$ such that quantizing $\bm{X}$ by rounding each scalar in $\bm{X}$ to the nearest quantization level minimizes the quantization MSE. 

The algorithm starts with an initial guess of quantization levels and then iteratively computes quantization thresholds $\{\tau_i\}_{i=1}^{2^B-1}$ and updates quantization levels $\{\hat{x}_i\}_{i=1}^{2^B}$. Specifically, at iteration $n$, thresholds are set to the midpoints of the previous iteration's levels:
\begin{align*}
    \tau_i^{(n)}=\frac{\hat{x}_i^{(n-1)}+\hat{x}_{i+1}^{(n-1)}}2 \text{ for } i=1\ldots 2^B-1
\end{align*}
Subsequently, the quantization levels are re-computed as conditional means of the data regions defined by the new thresholds:
\begin{align*}
    \hat{x}_i^{(n)}=\mathbb{E}\left[ \bm{X} \big| \bm{X}\in [\tau_{i-1}^{(n)},\tau_i^{(n)}] \right] \text{ for } i=1\ldots 2^B
\end{align*}
where to satisfy boundary conditions we have $\tau_0=-\infty$ and $\tau_{2^B}=\infty$. The algorithm iterates the above steps until convergence.

Figure \ref{fig:lm_quant} compares the quantization levels of a $7$-bit floating point (E3M3) quantizer (left) to a $7$-bit Lloyd-Max quantizer (right) when quantizing a layer of weights from the GPT3-126M model at a per-tensor granularity. As shown, the Lloyd-Max quantizer achieves substantially lower quantization MSE. Further, Table \ref{tab:FP7_vs_LM7} shows the superior perplexity achieved by Lloyd-Max quantizers for bitwidths of $7$, $6$ and $5$. The difference between the quantizers is clear at 5 bits, where per-tensor FP quantization incurs a drastic and unacceptable increase in perplexity, while Lloyd-Max quantization incurs a much smaller increase. Nevertheless, we note that even the optimal Lloyd-Max quantizer incurs a notable ($\sim 1.5$) increase in perplexity due to the coarse granularity of quantization. 

\begin{figure}[h]
  \centering
  \includegraphics[width=0.7\linewidth]{sections/figures/LM7_FP7.pdf}
  \caption{\small Quantization levels and the corresponding quantization MSE of Floating Point (left) vs Lloyd-Max (right) Quantizers for a layer of weights in the GPT3-126M model.}
  \label{fig:lm_quant}
\end{figure}

\begin{table}[h]\scriptsize
\begin{center}
\caption{\label{tab:FP7_vs_LM7} \small Comparing perplexity (lower is better) achieved by floating point quantizers and Lloyd-Max quantizers on a GPT3-126M model for the Wikitext-103 dataset.}
\begin{tabular}{c|cc|c}
\hline
 \multirow{2}{*}{\textbf{Bitwidth}} & \multicolumn{2}{|c|}{\textbf{Floating-Point Quantizer}} & \textbf{Lloyd-Max Quantizer} \\
 & Best Format & Wikitext-103 Perplexity & Wikitext-103 Perplexity \\
\hline
7 & E3M3 & 18.32 & 18.27 \\
6 & E3M2 & 19.07 & 18.51 \\
5 & E4M0 & 43.89 & 19.71 \\
\hline
\end{tabular}
\end{center}
\end{table}

\subsection{Proof of Local Optimality of LO-BCQ}
\label{subsec:lobcq_opt_proof}
For a given block $\bm{b}_j$, the quantization MSE during LO-BCQ can be empirically evaluated as $\frac{1}{L_b}\lVert \bm{b}_j- \bm{\hat{b}}_j\rVert^2_2$ where $\bm{\hat{b}}_j$ is computed from equation (\ref{eq:clustered_quantization_definition}) as $C_{f(\bm{b}_j)}(\bm{b}_j)$. Further, for a given block cluster $\mathcal{B}_i$, we compute the quantization MSE as $\frac{1}{|\mathcal{B}_{i}|}\sum_{\bm{b} \in \mathcal{B}_{i}} \frac{1}{L_b}\lVert \bm{b}- C_i^{(n)}(\bm{b})\rVert^2_2$. Therefore, at the end of iteration $n$, we evaluate the overall quantization MSE $J^{(n)}$ for a given operand $\bm{X}$ composed of $N_c$ block clusters as:
\begin{align*}
    \label{eq:mse_iter_n}
    J^{(n)} = \frac{1}{N_c} \sum_{i=1}^{N_c} \frac{1}{|\mathcal{B}_{i}^{(n)}|}\sum_{\bm{v} \in \mathcal{B}_{i}^{(n)}} \frac{1}{L_b}\lVert \bm{b}- B_i^{(n)}(\bm{b})\rVert^2_2
\end{align*}

At the end of iteration $n$, the codebooks are updated from $\mathcal{C}^{(n-1)}$ to $\mathcal{C}^{(n)}$. However, the mapping of a given vector $\bm{b}_j$ to quantizers $\mathcal{C}^{(n)}$ remains as  $f^{(n)}(\bm{b}_j)$. At the next iteration, during the vector clustering step, $f^{(n+1)}(\bm{b}_j)$ finds new mapping of $\bm{b}_j$ to updated codebooks $\mathcal{C}^{(n)}$ such that the quantization MSE over the candidate codebooks is minimized. Therefore, we obtain the following result for $\bm{b}_j$:
\begin{align*}
\frac{1}{L_b}\lVert \bm{b}_j - C_{f^{(n+1)}(\bm{b}_j)}^{(n)}(\bm{b}_j)\rVert^2_2 \le \frac{1}{L_b}\lVert \bm{b}_j - C_{f^{(n)}(\bm{b}_j)}^{(n)}(\bm{b}_j)\rVert^2_2
\end{align*}

That is, quantizing $\bm{b}_j$ at the end of the block clustering step of iteration $n+1$ results in lower quantization MSE compared to quantizing at the end of iteration $n$. Since this is true for all $\bm{b} \in \bm{X}$, we assert the following:
\begin{equation}
\begin{split}
\label{eq:mse_ineq_1}
    \tilde{J}^{(n+1)} &= \frac{1}{N_c} \sum_{i=1}^{N_c} \frac{1}{|\mathcal{B}_{i}^{(n+1)}|}\sum_{\bm{b} \in \mathcal{B}_{i}^{(n+1)}} \frac{1}{L_b}\lVert \bm{b} - C_i^{(n)}(b)\rVert^2_2 \le J^{(n)}
\end{split}
\end{equation}
where $\tilde{J}^{(n+1)}$ is the the quantization MSE after the vector clustering step at iteration $n+1$.

Next, during the codebook update step (\ref{eq:quantizers_update}) at iteration $n+1$, the per-cluster codebooks $\mathcal{C}^{(n)}$ are updated to $\mathcal{C}^{(n+1)}$ by invoking the Lloyd-Max algorithm \citep{Lloyd}. We know that for any given value distribution, the Lloyd-Max algorithm minimizes the quantization MSE. Therefore, for a given vector cluster $\mathcal{B}_i$ we obtain the following result:

\begin{equation}
    \frac{1}{|\mathcal{B}_{i}^{(n+1)}|}\sum_{\bm{b} \in \mathcal{B}_{i}^{(n+1)}} \frac{1}{L_b}\lVert \bm{b}- C_i^{(n+1)}(\bm{b})\rVert^2_2 \le \frac{1}{|\mathcal{B}_{i}^{(n+1)}|}\sum_{\bm{b} \in \mathcal{B}_{i}^{(n+1)}} \frac{1}{L_b}\lVert \bm{b}- C_i^{(n)}(\bm{b})\rVert^2_2
\end{equation}

The above equation states that quantizing the given block cluster $\mathcal{B}_i$ after updating the associated codebook from $C_i^{(n)}$ to $C_i^{(n+1)}$ results in lower quantization MSE. Since this is true for all the block clusters, we derive the following result: 
\begin{equation}
\begin{split}
\label{eq:mse_ineq_2}
     J^{(n+1)} &= \frac{1}{N_c} \sum_{i=1}^{N_c} \frac{1}{|\mathcal{B}_{i}^{(n+1)}|}\sum_{\bm{b} \in \mathcal{B}_{i}^{(n+1)}} \frac{1}{L_b}\lVert \bm{b}- C_i^{(n+1)}(\bm{b})\rVert^2_2  \le \tilde{J}^{(n+1)}   
\end{split}
\end{equation}

Following (\ref{eq:mse_ineq_1}) and (\ref{eq:mse_ineq_2}), we find that the quantization MSE is non-increasing for each iteration, that is, $J^{(1)} \ge J^{(2)} \ge J^{(3)} \ge \ldots \ge J^{(M)}$ where $M$ is the maximum number of iterations. 
%Therefore, we can say that if the algorithm converges, then it must be that it has converged to a local minimum. 
\hfill $\blacksquare$


\begin{figure}
    \begin{center}
    \includegraphics[width=0.5\textwidth]{sections//figures/mse_vs_iter.pdf}
    \end{center}
    \caption{\small NMSE vs iterations during LO-BCQ compared to other block quantization proposals}
    \label{fig:nmse_vs_iter}
\end{figure}

Figure \ref{fig:nmse_vs_iter} shows the empirical convergence of LO-BCQ across several block lengths and number of codebooks. Also, the MSE achieved by LO-BCQ is compared to baselines such as MXFP and VSQ. As shown, LO-BCQ converges to a lower MSE than the baselines. Further, we achieve better convergence for larger number of codebooks ($N_c$) and for a smaller block length ($L_b$), both of which increase the bitwidth of BCQ (see Eq \ref{eq:bitwidth_bcq}).


\subsection{Additional Accuracy Results}
%Table \ref{tab:lobcq_config} lists the various LOBCQ configurations and their corresponding bitwidths.
\begin{table}
\setlength{\tabcolsep}{4.75pt}
\begin{center}
\caption{\label{tab:lobcq_config} Various LO-BCQ configurations and their bitwidths.}
\begin{tabular}{|c||c|c|c|c||c|c||c|} 
\hline
 & \multicolumn{4}{|c||}{$L_b=8$} & \multicolumn{2}{|c||}{$L_b=4$} & $L_b=2$ \\
 \hline
 \backslashbox{$L_A$\kern-1em}{\kern-1em$N_c$} & 2 & 4 & 8 & 16 & 2 & 4 & 2 \\
 \hline
 64 & 4.25 & 4.375 & 4.5 & 4.625 & 4.375 & 4.625 & 4.625\\
 \hline
 32 & 4.375 & 4.5 & 4.625& 4.75 & 4.5 & 4.75 & 4.75 \\
 \hline
 16 & 4.625 & 4.75& 4.875 & 5 & 4.75 & 5 & 5 \\
 \hline
\end{tabular}
\end{center}
\end{table}

%\subsection{Perplexity achieved by various LO-BCQ configurations on Wikitext-103 dataset}

\begin{table} \centering
\begin{tabular}{|c||c|c|c|c||c|c||c|} 
\hline
 $L_b \rightarrow$& \multicolumn{4}{c||}{8} & \multicolumn{2}{c||}{4} & 2\\
 \hline
 \backslashbox{$L_A$\kern-1em}{\kern-1em$N_c$} & 2 & 4 & 8 & 16 & 2 & 4 & 2  \\
 %$N_c \rightarrow$ & 2 & 4 & 8 & 16 & 2 & 4 & 2 \\
 \hline
 \hline
 \multicolumn{8}{c}{GPT3-1.3B (FP32 PPL = 9.98)} \\ 
 \hline
 \hline
 64 & 10.40 & 10.23 & 10.17 & 10.15 &  10.28 & 10.18 & 10.19 \\
 \hline
 32 & 10.25 & 10.20 & 10.15 & 10.12 &  10.23 & 10.17 & 10.17 \\
 \hline
 16 & 10.22 & 10.16 & 10.10 & 10.09 &  10.21 & 10.14 & 10.16 \\
 \hline
  \hline
 \multicolumn{8}{c}{GPT3-8B (FP32 PPL = 7.38)} \\ 
 \hline
 \hline
 64 & 7.61 & 7.52 & 7.48 &  7.47 &  7.55 &  7.49 & 7.50 \\
 \hline
 32 & 7.52 & 7.50 & 7.46 &  7.45 &  7.52 &  7.48 & 7.48  \\
 \hline
 16 & 7.51 & 7.48 & 7.44 &  7.44 &  7.51 &  7.49 & 7.47  \\
 \hline
\end{tabular}
\caption{\label{tab:ppl_gpt3_abalation} Wikitext-103 perplexity across GPT3-1.3B and 8B models.}
\end{table}

\begin{table} \centering
\begin{tabular}{|c||c|c|c|c||} 
\hline
 $L_b \rightarrow$& \multicolumn{4}{c||}{8}\\
 \hline
 \backslashbox{$L_A$\kern-1em}{\kern-1em$N_c$} & 2 & 4 & 8 & 16 \\
 %$N_c \rightarrow$ & 2 & 4 & 8 & 16 & 2 & 4 & 2 \\
 \hline
 \hline
 \multicolumn{5}{|c|}{Llama2-7B (FP32 PPL = 5.06)} \\ 
 \hline
 \hline
 64 & 5.31 & 5.26 & 5.19 & 5.18  \\
 \hline
 32 & 5.23 & 5.25 & 5.18 & 5.15  \\
 \hline
 16 & 5.23 & 5.19 & 5.16 & 5.14  \\
 \hline
 \multicolumn{5}{|c|}{Nemotron4-15B (FP32 PPL = 5.87)} \\ 
 \hline
 \hline
 64  & 6.3 & 6.20 & 6.13 & 6.08  \\
 \hline
 32  & 6.24 & 6.12 & 6.07 & 6.03  \\
 \hline
 16  & 6.12 & 6.14 & 6.04 & 6.02  \\
 \hline
 \multicolumn{5}{|c|}{Nemotron4-340B (FP32 PPL = 3.48)} \\ 
 \hline
 \hline
 64 & 3.67 & 3.62 & 3.60 & 3.59 \\
 \hline
 32 & 3.63 & 3.61 & 3.59 & 3.56 \\
 \hline
 16 & 3.61 & 3.58 & 3.57 & 3.55 \\
 \hline
\end{tabular}
\caption{\label{tab:ppl_llama7B_nemo15B} Wikitext-103 perplexity compared to FP32 baseline in Llama2-7B and Nemotron4-15B, 340B models}
\end{table}

%\subsection{Perplexity achieved by various LO-BCQ configurations on MMLU dataset}


\begin{table} \centering
\begin{tabular}{|c||c|c|c|c||c|c|c|c|} 
\hline
 $L_b \rightarrow$& \multicolumn{4}{c||}{8} & \multicolumn{4}{c||}{8}\\
 \hline
 \backslashbox{$L_A$\kern-1em}{\kern-1em$N_c$} & 2 & 4 & 8 & 16 & 2 & 4 & 8 & 16  \\
 %$N_c \rightarrow$ & 2 & 4 & 8 & 16 & 2 & 4 & 2 \\
 \hline
 \hline
 \multicolumn{5}{|c|}{Llama2-7B (FP32 Accuracy = 45.8\%)} & \multicolumn{4}{|c|}{Llama2-70B (FP32 Accuracy = 69.12\%)} \\ 
 \hline
 \hline
 64 & 43.9 & 43.4 & 43.9 & 44.9 & 68.07 & 68.27 & 68.17 & 68.75 \\
 \hline
 32 & 44.5 & 43.8 & 44.9 & 44.5 & 68.37 & 68.51 & 68.35 & 68.27  \\
 \hline
 16 & 43.9 & 42.7 & 44.9 & 45 & 68.12 & 68.77 & 68.31 & 68.59  \\
 \hline
 \hline
 \multicolumn{5}{|c|}{GPT3-22B (FP32 Accuracy = 38.75\%)} & \multicolumn{4}{|c|}{Nemotron4-15B (FP32 Accuracy = 64.3\%)} \\ 
 \hline
 \hline
 64 & 36.71 & 38.85 & 38.13 & 38.92 & 63.17 & 62.36 & 63.72 & 64.09 \\
 \hline
 32 & 37.95 & 38.69 & 39.45 & 38.34 & 64.05 & 62.30 & 63.8 & 64.33  \\
 \hline
 16 & 38.88 & 38.80 & 38.31 & 38.92 & 63.22 & 63.51 & 63.93 & 64.43  \\
 \hline
\end{tabular}
\caption{\label{tab:mmlu_abalation} Accuracy on MMLU dataset across GPT3-22B, Llama2-7B, 70B and Nemotron4-15B models.}
\end{table}


%\subsection{Perplexity achieved by various LO-BCQ configurations on LM evaluation harness}

\begin{table} \centering
\begin{tabular}{|c||c|c|c|c||c|c|c|c|} 
\hline
 $L_b \rightarrow$& \multicolumn{4}{c||}{8} & \multicolumn{4}{c||}{8}\\
 \hline
 \backslashbox{$L_A$\kern-1em}{\kern-1em$N_c$} & 2 & 4 & 8 & 16 & 2 & 4 & 8 & 16  \\
 %$N_c \rightarrow$ & 2 & 4 & 8 & 16 & 2 & 4 & 2 \\
 \hline
 \hline
 \multicolumn{5}{|c|}{Race (FP32 Accuracy = 37.51\%)} & \multicolumn{4}{|c|}{Boolq (FP32 Accuracy = 64.62\%)} \\ 
 \hline
 \hline
 64 & 36.94 & 37.13 & 36.27 & 37.13 & 63.73 & 62.26 & 63.49 & 63.36 \\
 \hline
 32 & 37.03 & 36.36 & 36.08 & 37.03 & 62.54 & 63.51 & 63.49 & 63.55  \\
 \hline
 16 & 37.03 & 37.03 & 36.46 & 37.03 & 61.1 & 63.79 & 63.58 & 63.33  \\
 \hline
 \hline
 \multicolumn{5}{|c|}{Winogrande (FP32 Accuracy = 58.01\%)} & \multicolumn{4}{|c|}{Piqa (FP32 Accuracy = 74.21\%)} \\ 
 \hline
 \hline
 64 & 58.17 & 57.22 & 57.85 & 58.33 & 73.01 & 73.07 & 73.07 & 72.80 \\
 \hline
 32 & 59.12 & 58.09 & 57.85 & 58.41 & 73.01 & 73.94 & 72.74 & 73.18  \\
 \hline
 16 & 57.93 & 58.88 & 57.93 & 58.56 & 73.94 & 72.80 & 73.01 & 73.94  \\
 \hline
\end{tabular}
\caption{\label{tab:mmlu_abalation} Accuracy on LM evaluation harness tasks on GPT3-1.3B model.}
\end{table}

\begin{table} \centering
\begin{tabular}{|c||c|c|c|c||c|c|c|c|} 
\hline
 $L_b \rightarrow$& \multicolumn{4}{c||}{8} & \multicolumn{4}{c||}{8}\\
 \hline
 \backslashbox{$L_A$\kern-1em}{\kern-1em$N_c$} & 2 & 4 & 8 & 16 & 2 & 4 & 8 & 16  \\
 %$N_c \rightarrow$ & 2 & 4 & 8 & 16 & 2 & 4 & 2 \\
 \hline
 \hline
 \multicolumn{5}{|c|}{Race (FP32 Accuracy = 41.34\%)} & \multicolumn{4}{|c|}{Boolq (FP32 Accuracy = 68.32\%)} \\ 
 \hline
 \hline
 64 & 40.48 & 40.10 & 39.43 & 39.90 & 69.20 & 68.41 & 69.45 & 68.56 \\
 \hline
 32 & 39.52 & 39.52 & 40.77 & 39.62 & 68.32 & 67.43 & 68.17 & 69.30  \\
 \hline
 16 & 39.81 & 39.71 & 39.90 & 40.38 & 68.10 & 66.33 & 69.51 & 69.42  \\
 \hline
 \hline
 \multicolumn{5}{|c|}{Winogrande (FP32 Accuracy = 67.88\%)} & \multicolumn{4}{|c|}{Piqa (FP32 Accuracy = 78.78\%)} \\ 
 \hline
 \hline
 64 & 66.85 & 66.61 & 67.72 & 67.88 & 77.31 & 77.42 & 77.75 & 77.64 \\
 \hline
 32 & 67.25 & 67.72 & 67.72 & 67.00 & 77.31 & 77.04 & 77.80 & 77.37  \\
 \hline
 16 & 68.11 & 68.90 & 67.88 & 67.48 & 77.37 & 78.13 & 78.13 & 77.69  \\
 \hline
\end{tabular}
\caption{\label{tab:mmlu_abalation} Accuracy on LM evaluation harness tasks on GPT3-8B model.}
\end{table}

\begin{table} \centering
\begin{tabular}{|c||c|c|c|c||c|c|c|c|} 
\hline
 $L_b \rightarrow$& \multicolumn{4}{c||}{8} & \multicolumn{4}{c||}{8}\\
 \hline
 \backslashbox{$L_A$\kern-1em}{\kern-1em$N_c$} & 2 & 4 & 8 & 16 & 2 & 4 & 8 & 16  \\
 %$N_c \rightarrow$ & 2 & 4 & 8 & 16 & 2 & 4 & 2 \\
 \hline
 \hline
 \multicolumn{5}{|c|}{Race (FP32 Accuracy = 40.67\%)} & \multicolumn{4}{|c|}{Boolq (FP32 Accuracy = 76.54\%)} \\ 
 \hline
 \hline
 64 & 40.48 & 40.10 & 39.43 & 39.90 & 75.41 & 75.11 & 77.09 & 75.66 \\
 \hline
 32 & 39.52 & 39.52 & 40.77 & 39.62 & 76.02 & 76.02 & 75.96 & 75.35  \\
 \hline
 16 & 39.81 & 39.71 & 39.90 & 40.38 & 75.05 & 73.82 & 75.72 & 76.09  \\
 \hline
 \hline
 \multicolumn{5}{|c|}{Winogrande (FP32 Accuracy = 70.64\%)} & \multicolumn{4}{|c|}{Piqa (FP32 Accuracy = 79.16\%)} \\ 
 \hline
 \hline
 64 & 69.14 & 70.17 & 70.17 & 70.56 & 78.24 & 79.00 & 78.62 & 78.73 \\
 \hline
 32 & 70.96 & 69.69 & 71.27 & 69.30 & 78.56 & 79.49 & 79.16 & 78.89  \\
 \hline
 16 & 71.03 & 69.53 & 69.69 & 70.40 & 78.13 & 79.16 & 79.00 & 79.00  \\
 \hline
\end{tabular}
\caption{\label{tab:mmlu_abalation} Accuracy on LM evaluation harness tasks on GPT3-22B model.}
\end{table}

\begin{table} \centering
\begin{tabular}{|c||c|c|c|c||c|c|c|c|} 
\hline
 $L_b \rightarrow$& \multicolumn{4}{c||}{8} & \multicolumn{4}{c||}{8}\\
 \hline
 \backslashbox{$L_A$\kern-1em}{\kern-1em$N_c$} & 2 & 4 & 8 & 16 & 2 & 4 & 8 & 16  \\
 %$N_c \rightarrow$ & 2 & 4 & 8 & 16 & 2 & 4 & 2 \\
 \hline
 \hline
 \multicolumn{5}{|c|}{Race (FP32 Accuracy = 44.4\%)} & \multicolumn{4}{|c|}{Boolq (FP32 Accuracy = 79.29\%)} \\ 
 \hline
 \hline
 64 & 42.49 & 42.51 & 42.58 & 43.45 & 77.58 & 77.37 & 77.43 & 78.1 \\
 \hline
 32 & 43.35 & 42.49 & 43.64 & 43.73 & 77.86 & 75.32 & 77.28 & 77.86  \\
 \hline
 16 & 44.21 & 44.21 & 43.64 & 42.97 & 78.65 & 77 & 76.94 & 77.98  \\
 \hline
 \hline
 \multicolumn{5}{|c|}{Winogrande (FP32 Accuracy = 69.38\%)} & \multicolumn{4}{|c|}{Piqa (FP32 Accuracy = 78.07\%)} \\ 
 \hline
 \hline
 64 & 68.9 & 68.43 & 69.77 & 68.19 & 77.09 & 76.82 & 77.09 & 77.86 \\
 \hline
 32 & 69.38 & 68.51 & 68.82 & 68.90 & 78.07 & 76.71 & 78.07 & 77.86  \\
 \hline
 16 & 69.53 & 67.09 & 69.38 & 68.90 & 77.37 & 77.8 & 77.91 & 77.69  \\
 \hline
\end{tabular}
\caption{\label{tab:mmlu_abalation} Accuracy on LM evaluation harness tasks on Llama2-7B model.}
\end{table}

\begin{table} \centering
\begin{tabular}{|c||c|c|c|c||c|c|c|c|} 
\hline
 $L_b \rightarrow$& \multicolumn{4}{c||}{8} & \multicolumn{4}{c||}{8}\\
 \hline
 \backslashbox{$L_A$\kern-1em}{\kern-1em$N_c$} & 2 & 4 & 8 & 16 & 2 & 4 & 8 & 16  \\
 %$N_c \rightarrow$ & 2 & 4 & 8 & 16 & 2 & 4 & 2 \\
 \hline
 \hline
 \multicolumn{5}{|c|}{Race (FP32 Accuracy = 48.8\%)} & \multicolumn{4}{|c|}{Boolq (FP32 Accuracy = 85.23\%)} \\ 
 \hline
 \hline
 64 & 49.00 & 49.00 & 49.28 & 48.71 & 82.82 & 84.28 & 84.03 & 84.25 \\
 \hline
 32 & 49.57 & 48.52 & 48.33 & 49.28 & 83.85 & 84.46 & 84.31 & 84.93  \\
 \hline
 16 & 49.85 & 49.09 & 49.28 & 48.99 & 85.11 & 84.46 & 84.61 & 83.94  \\
 \hline
 \hline
 \multicolumn{5}{|c|}{Winogrande (FP32 Accuracy = 79.95\%)} & \multicolumn{4}{|c|}{Piqa (FP32 Accuracy = 81.56\%)} \\ 
 \hline
 \hline
 64 & 78.77 & 78.45 & 78.37 & 79.16 & 81.45 & 80.69 & 81.45 & 81.5 \\
 \hline
 32 & 78.45 & 79.01 & 78.69 & 80.66 & 81.56 & 80.58 & 81.18 & 81.34  \\
 \hline
 16 & 79.95 & 79.56 & 79.79 & 79.72 & 81.28 & 81.66 & 81.28 & 80.96  \\
 \hline
\end{tabular}
\caption{\label{tab:mmlu_abalation} Accuracy on LM evaluation harness tasks on Llama2-70B model.}
\end{table}

%\section{MSE Studies}
%\textcolor{red}{TODO}


\subsection{Number Formats and Quantization Method}
\label{subsec:numFormats_quantMethod}
\subsubsection{Integer Format}
An $n$-bit signed integer (INT) is typically represented with a 2s-complement format \citep{yao2022zeroquant,xiao2023smoothquant,dai2021vsq}, where the most significant bit denotes the sign.

\subsubsection{Floating Point Format}
An $n$-bit signed floating point (FP) number $x$ comprises of a 1-bit sign ($x_{\mathrm{sign}}$), $B_m$-bit mantissa ($x_{\mathrm{mant}}$) and $B_e$-bit exponent ($x_{\mathrm{exp}}$) such that $B_m+B_e=n-1$. The associated constant exponent bias ($E_{\mathrm{bias}}$) is computed as $(2^{{B_e}-1}-1)$. We denote this format as $E_{B_e}M_{B_m}$.  

\subsubsection{Quantization Scheme}
\label{subsec:quant_method}
A quantization scheme dictates how a given unquantized tensor is converted to its quantized representation. We consider FP formats for the purpose of illustration. Given an unquantized tensor $\bm{X}$ and an FP format $E_{B_e}M_{B_m}$, we first, we compute the quantization scale factor $s_X$ that maps the maximum absolute value of $\bm{X}$ to the maximum quantization level of the $E_{B_e}M_{B_m}$ format as follows:
\begin{align}
\label{eq:sf}
    s_X = \frac{\mathrm{max}(|\bm{X}|)}{\mathrm{max}(E_{B_e}M_{B_m})}
\end{align}
In the above equation, $|\cdot|$ denotes the absolute value function.

Next, we scale $\bm{X}$ by $s_X$ and quantize it to $\hat{\bm{X}}$ by rounding it to the nearest quantization level of $E_{B_e}M_{B_m}$ as:

\begin{align}
\label{eq:tensor_quant}
    \hat{\bm{X}} = \text{round-to-nearest}\left(\frac{\bm{X}}{s_X}, E_{B_e}M_{B_m}\right)
\end{align}

We perform dynamic max-scaled quantization \citep{wu2020integer}, where the scale factor $s$ for activations is dynamically computed during runtime.

\subsection{Vector Scaled Quantization}
\begin{wrapfigure}{r}{0.35\linewidth}
  \centering
  \includegraphics[width=\linewidth]{sections/figures/vsquant.jpg}
  \caption{\small Vectorwise decomposition for per-vector scaled quantization (VSQ \citep{dai2021vsq}).}
  \label{fig:vsquant}
\end{wrapfigure}
During VSQ \citep{dai2021vsq}, the operand tensors are decomposed into 1D vectors in a hardware friendly manner as shown in Figure \ref{fig:vsquant}. Since the decomposed tensors are used as operands in matrix multiplications during inference, it is beneficial to perform this decomposition along the reduction dimension of the multiplication. The vectorwise quantization is performed similar to tensorwise quantization described in Equations \ref{eq:sf} and \ref{eq:tensor_quant}, where a scale factor $s_v$ is required for each vector $\bm{v}$ that maps the maximum absolute value of that vector to the maximum quantization level. While smaller vector lengths can lead to larger accuracy gains, the associated memory and computational overheads due to the per-vector scale factors increases. To alleviate these overheads, VSQ \citep{dai2021vsq} proposed a second level quantization of the per-vector scale factors to unsigned integers, while MX \citep{rouhani2023shared} quantizes them to integer powers of 2 (denoted as $2^{INT}$).

\subsubsection{MX Format}
The MX format proposed in \citep{rouhani2023microscaling} introduces the concept of sub-block shifting. For every two scalar elements of $b$-bits each, there is a shared exponent bit. The value of this exponent bit is determined through an empirical analysis that targets minimizing quantization MSE. We note that the FP format $E_{1}M_{b}$ is strictly better than MX from an accuracy perspective since it allocates a dedicated exponent bit to each scalar as opposed to sharing it across two scalars. Therefore, we conservatively bound the accuracy of a $b+2$-bit signed MX format with that of a $E_{1}M_{b}$ format in our comparisons. For instance, we use E1M2 format as a proxy for MX4.

\begin{figure}
    \centering
    \includegraphics[width=1\linewidth]{sections//figures/BlockFormats.pdf}
    \caption{\small Comparing LO-BCQ to MX format.}
    \label{fig:block_formats}
\end{figure}

Figure \ref{fig:block_formats} compares our $4$-bit LO-BCQ block format to MX \citep{rouhani2023microscaling}. As shown, both LO-BCQ and MX decompose a given operand tensor into block arrays and each block array into blocks. Similar to MX, we find that per-block quantization ($L_b < L_A$) leads to better accuracy due to increased flexibility. While MX achieves this through per-block $1$-bit micro-scales, we associate a dedicated codebook to each block through a per-block codebook selector. Further, MX quantizes the per-block array scale-factor to E8M0 format without per-tensor scaling. In contrast during LO-BCQ, we find that per-tensor scaling combined with quantization of per-block array scale-factor to E4M3 format results in superior inference accuracy across models. 

\end{document}
\endinput
%%
%% End of file `sample-sigconf-authordraft.tex'.
