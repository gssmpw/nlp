\section{Challenges and Future Directions}
\label{sec:challenges}

\subsection{Imbalance in Open World Setting}

Existing researches presume that all classes present in the test set are also included in the training set. However, real-world scenarios frequently adhere to an open-world setting, where new classes may emerge exclusively in the test set. This discrepancy poses substantial challenges for imbalanced learning. A recent study \cite{liu2019large} highlights the critical need for developing techniques that can dynamically adapt to unknown class distributions. Future research could investigate adaptive algorithms that continuously refine their understanding of the class distribution as new data is encountered. This could involve the application of online learning or few-shot learning strategies to effectively categorize newly emerging classes. Moreover, the integration of unsupervised or semi-supervised methods could significantly improve the model's capability to generalize to unseen classes, perhaps by utilizing generative models to emulate novel class scenarios during the training phase.

\subsection{Imbalanced Multi-Label Problem}

Imbalanced multi-label classification presents unique challenges, differing fundamentally from traditional multi-class scenarios due to the presence of multiple labels per instance. This complexity is particularly evident in applications like image tagging, where an image may contain diverse objects, each corresponding to a different label with varying frequency of occurrence. Future research should concentrate on developing algorithms specifically designed to manage the intricacies of label correlation and disparities in label frequencies. Potential solutions could include the implementation of refined loss functions tailored to balance label importance, label-specific re-sampling strategies to adjust the representation of underrepresented labels, or advanced regularization techniques to prevent model overfitting on prevalent labels. Moreover, leveraging insights from the existing researches on multi-label learning could provide innovative pathways for algorithmic enhancement, facilitating more accurate predictions across the diverse label distributions in real-world datasets.

\subsection{Imbalanced Regression Problem}

In contrast to classification, regression involves continuous labels and is essential in various real-world settings, such as financial forecasting, climate modeling, and healthcare analytics. However, the imbalance in regression tasks is not as extensively studied as in classification tasks \cite{gong2022ranksim,yang2021delving,gong2022ranksim}. Innovative research could explore methods to correct skewness in continuous data, perhaps by developing adaptive resampling strategies or formulating loss functions that give more weight to underrepresented but crucial target values. Additionally, incorporating techniques that enhance the model's responsiveness to infrequent yet significant outcomes, such as balanced mean squared error or specific regularization methods, could lead to more balanced outcomes in diverse application fields.

\subsection{Multi-Modality Imbalance Problem}

Multi-modality data plays a crucial role across a vast array of applications, including healthcare diagnostics, autonomous driving, multimedia content analysis, and social media analytics. Integrating data from multiple sources in these fields substantially enhances the robustness and accuracy of predictive models. However, the inherent imbalance in modality data \cite{qian2022co}, characterized by disparities both within and between different data modalities, presents a significant challenge. Addressing this issue, future research should explore the hybrid models with data augmentation strategies to enhance model resilience and effectiveness across modalities. Furthermore, the potential of transfer learning to leverage data-rich modalities for enhancing performance in data-sparse conditions warrants thorough investigation. Such approaches could provide substantial advancements in mitigating the challenges posed by multi-modality imbalance, thereby improving the utility and accuracy of multi-modal systems in diverse real-world settings.
