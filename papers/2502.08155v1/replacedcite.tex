\section{Related work}
\label{SecRelatedWork}

\subsection{Robust wireless sensing}
\subsubsection{WiFi sensing}
WiFi sensing has been extensively studied since the release of Channel State Information (CSI) tools____. Many solutions have been proposed to improve the robustness. 
Exploiting adversarial networks, EI____ could remove environment and user-specific information and learn transferable features of activities. 
CsiGAN____ enabled activity recognition adaptive to users based on semi-supervised Generative Adversarial Network (GAN), leveraging a complement generator to produce diverse fake samples to train a robust discriminator.
Kang et al.____ realized cross domain gesture recognition based on CSI Doppler Frequency Shift (DFS) through multi-source unsupervised domain adaptation, by applying adversarial learning and feature disentanglement to remove gesture irrelevant factors. 
OneFi____ could recognize unseen gestures with only one (or few) labeled samples. It utilized virtual gesture generation to reduce the effort in data collection and employed a one-shot learning framework based on transductive fine-tuning to eliminate model retraining. 
These methods required some labeled or unlabeled data from the target domains for model training, while our method did not require any target domain data for model training.
Widar3.0____ achieved cross domain gesture recognition by extracting the domain independent feature Body-coordinate Velocity Profile (BVP) from CSI DFS. Widar3.0 required one transmitter and at least three receivers to work, while our method required only one transmitter and one receiver.  

\subsubsection{mmWave sensing}
Due to high frequency and wide bandwidth, mmWave can be used to realize highly accurate sensing. Researchers have developed various mmWave-based sensing applications, mostly exploiting Frequency Modulated Continuous Waves (FMCW). 
S. Wang et al.____ built a gesture recognition system based on Google’s Soli sensor, leveraging a combination of deep convolutional and recurrent neural networks.
Zhang et al.____ processed range-Doppler maps (RDM) with a zero-filling strategy to boost the range and velocity information of gestures and constructed a 3DCNN model and a CNN-Long Short Term Memory (LSTM) model to reveal the temporal gesture signatures encoded in multiple frames. 
Amin et al.____ classified daily activities, in particular contiguous motions, using micro-Doppler signatures and range maps. 
Chen et al.____ designed a temporal 3DCNN to deal with a series of range-Doppler maps and added a temporal attention module to emphasize the sequential relation between each frame. 
These methods focused on in-domain sensing without considering the cross domain issue.
mHomeGes____ proposed concentrated position-Doppler profile (CPDP) to represent the unique features from different arm joints and recognized continuous arm gestures based on a lightweight CNN. mHomeGes could work across various smart-home scenarios regardless of the impact of surrounding interference, but domain independence was not under consideration of this work.  
M-Gesture____ achieved person-independent gesture recognition by incorporating a pseudo representative model (PRM) and a custom-built neural network to depict and extract the inherent gesture features. Their experiments showed that M-Gesture required about 20 persons to achieve high accuracy for new users and the changes in user orientations still degraded the performance. 
 
\subsubsection{Acoustic sensing}
Acoustic signals can be used for human sensing. A variety of applications have been developed in recent years____. 
DeepRange____ synthesized training data and realized acoustic ranging based on deep learning. 
EchoSpot____ localized a person by periodically emitting acoustic chirps, capturing the reflections from the human body and the walls, and normalizing their cross-correlation. 
EarIO____ extracted the depth features of the face and tracked the facial expressions via cross-correlation between emitted and reflected acoustic signals. To adapt to new users, EarIO required a small amount of data from new users.
Lian et al.____ detected falls by emitting continuous waves of fixed frequency and calculating Doppler frequency spectrogram of the reflections. Singular Value Decomposition (SVD) and Hidden Markov Model (HMM) were applied to classify the motions as falls or non-falls. 
StruGesture____ exploited the structure-borne sounds to classify gestures on the back surface of mobile phones. It leveraged deep adversarial learning to learn the gesture-specific representation, and required a few samples from the new users to achieve proper recognition for new users.

\subsection{Domain adaptation and generalization}
\subsubsection{Domain adaptation}
Domain adaptation (DA) aims to utilize the data in the source domains to solve the learning problem in the target domains. Some domain adaptation methods have been proposed to tackle the domain dependence problem in wireless sensing.
%
Feature alignment is a common method of domain adaptation. The features in the source and the target domains are mapped to a shared space, in which the distribution divergence between them is minimized, so that the sensing model built in the source domain can then be applied to the target domain. AdapLoc____ achieved adaptive CSI localization by mapping the source and the target domains into a shared space and minimizing the distance between the fingerprints at the same location and maximizing the distance between the fingerprints at different locations. DANGR____ recognized gestures using CSI based on a deep adaptation network. To shrink the domain discrepancy, DANGR adopted Multi-Kernel Maximal Mean Discrepancy (MK-MMD).
%
Researchers also proposed synthesizing virtual samples to enlarge the training set and increase the diversity, thereby enhancing the robustness. FiDo____ could localize diverse users with labeled CSI data from one or two users, leveraging a data augmenter that introduced data diversity using VAE and a domain adaptive classifier. Zhang et al.____ synthesized variant activity data through CSI transformation to mitigate the impact of activity inconsistency and subject-specific issue for CSI activity recognition. 
%
Domain adaptation can also be achieved by adversarial networks. Regarding the target domain data as fake samples, the domain independent features can be extracted through adversarial training. By exploiting adversarial networks, EI____ could remove environment and user-specific information and achieve environment independent activity recognition based on CSI. CsiGAN____ enabled CSI activity recognition adaptive to users based on semi-supervised GAN. 
%
Domain adaptation methods require extra efforts to collect data from the target domains. They have to retrain the classifiers each time new target domains are added. Therefore, they have difficulties in generalizing to new/unseen domains.  

\subsubsection{Domain generalization}
Domain generalization extends domain adaptation by generalizing to the target domains without any target domain data and retraining. It focuses on learning a domain independent sensing model from the source domains and achieving high performance in unseen target domains____. Domain generalization is still at its early stage and most researches are on image recognition. H. Li et al.____ applied adversarial Autoencoders and MMD to align feature distributions and extract domain independent features. D. Li et al.____ enhanced model robustness to new domains through episodic training, exposing the model to diverse samples. Qiao et al.____ combined meta-learning and Wasserstein Autoencoders to generate virtual data, improving generalization capabilities.   

In the field of wireless sensing, general domain generalization methods for diverse sensing tasks on top of diverse wireless signals are still lacking. Previous studies either exploit domain adaptation methods requiring some target domain data, or design specific signal processing techniques for specific sensing tasks. The main contribution of our work is a general domain generalization framework for wireless sensing tasks.