\newcommand{\locrevab}[2]{\mathrm{#1}^{\mathrm{#2}}_{H, a, b}}
\newcommand{\locreva}[2]{\mathrm{#1}^{\mathrm{#2}}_{H, a}}
\newcommand{\locrev}[2]{\mathrm{#1}^{\mathrm{#2}}_{H}}

In this section, we focus on a single block $H$ from a certain level of the decomposition $\mathcal{L}_j$.
%
Blocks on the last level of the decomposition will be solved naively, as described in \cref{sec:decomp}, so we do not concern ourselves with them here. We consider the instance of our problem consisting of the extended block of $H$, denoted $H_{\mathrm{ext}}$, and all drivers assigned to it $B_H$. For this instance, we will present a poly-time
constant factor approximation algorithm, i.e., an algorithm for assigning prices to the edges of $H_{\mathrm{ext}}$ such that the revenue generated by drivers in $B_H$ is a constant-factor away from optimal.

To begin, note that, by construction of the decomposition, for each driver $(u, v, b) \in B_H$, we have that one of $u$ and $v$ lies on or above the middle row of $H$ and the other on or below the middle row of $H$. As a result, each $u$-$v$ path contains at least one vertex from the middle row of $H$. Let us assume without loss of generality that $v$'s row is not above $u$'s row to eliminate a
case distinction. 

Consider any $u$-$v$ path and let $x$ be the first vertex from the middle row of $H$ on the path and $y$ be the last vertex from the middle row of $H$ on the path. Then, the path can be split into three sub-paths $u$-$x$, $x$-$y$ and $y$-$v$, which we call the \emph{upper section}, \emph{middle section} and \emph{lower section} of the path, respectively. Note that any of the three sections could be trivial paths with no edges.

\begin{figure}[t]
    \centering
    \includegraphics[width=0.5\textwidth]{path_sections-legend}
    \caption{
        Example of two paths in a block $H$ desired by two different drivers (red and blue) and their partition into sections.
        Note that the lower section of the red path is empty, because the corresponding endpoint lies on the middle row, so, by definition, $y = v$.
    }
\end{figure}

Consider an optimal solution $S^*$ for our instance. For each driver $(u, v, b) \in B_H$, there could be multiple lowest-cost $u$-$v$ paths under the pricing $S^*$ --- choose one such path for each driver. Let $M$ be the set of vertices on the middle row of $H$. For any two vertices $s, t \in M$, write $B_{s, t}^*$ for the multiset of drivers in $B_H$ whose chosen path has $(x, y) = (s, t)$ and $R_{S^*}(B_{s, t}^*)$ for the total revenue generated under the pricing $S^*$ by drivers in $B_{s, t}^*$. Then, we can write the optimal revenue as  $R_{S^*} = \sum_{x, y \in M} R_{S^*}(B_{s, t}^*)$.
%
As a result, there must exist $s, t \in M$
such that $R_{S^*}(B_{s, t}^*) \geq \frac{1}{\omega^2}R_{S^*}$. This is because there are $\omega^2$ terms in the summation, meaning the average of the terms is $\frac{1}{\omega^2}R_{S^*}$ --- at least one term is no lower than the average, implying the previous.
This brings us to the crucial idea:
if we could generate in polynomial time solutions $S_{s, t}$ for all $s, t \in M$ such that the revenues generated by them satisfy
$R_{S_{s, t}} \geq \alpha R_{S^*}(B_{s, t}^*)$ for some fixed constant $\alpha > 0$,
then the best of these solutions will give an $\frac{1}{\alpha} \omega^2$-approximation of the optimal revenue.
This will be the approach that we will be taking.


From now on, consider two fixed vertices on the middle row $s, t \in M$. We want to construct a solution $S_{s, t}$ generating revenue at least $\alpha R_{S^*}(B_{s, t}^*)$. 
%
To achieve this, let us more closely analyze the paths of drivers in $B_{s, t}^*$: all such paths consist of an upper section that ends at $s$, a middle section between $s$ and $t$, and a lower section starting at $t$. The upper section can only use edges with one endpoint above the middle row and the other endpoint either at $s$ or also above the middle row (marked blue in \cref{fig:pricing_scheme}). Analogous considerations apply to the lower section (edges marked red in \cref{fig:pricing_scheme}). The middle section might be different for different paths, but, since all paths are lowest-cost, it must have exactly the same total price in all paths under consideration; call it $p_{s, t}^\mathrm{mid}$. With these observations, let us construct another solution $S'_{s, t}$ by starting with $S^*$ and modifying the prices as follows:
\begin{enumerate}[nosep]
    \item Set to $\infty$ the prices of all edges going up from the middle row, except the one exiting $s$, which keeps its price.
    \item Set to $\infty$ the prices of all edges going down from the middle row, except the one exiting $t$, which keeps its price.
    \item Set to $\infty$ the price of all edges on the middle row, except those in the $s$-$t$ range.
    \item Reprice the edges in the $s$-$t$ range on the middle row to sum to $p_{s, t}^\mathrm{mid}$. This can be achieved by making all but one of them zero, except in the case $s = t$, where $p_{s, t}^\mathrm{mid} = 0$ anyway. 
\end{enumerate}
Edges from steps 1-3 are marked with black dotted lines in \cref{fig:pricing_scheme}, whereas the ones from step 4 are marked green.

\begin{figure}[t]
    \centering
    \includegraphics[width=0.5\textwidth]{pricing_scheme}
    \caption{
        A scheme of the partition of edges of $H_{\mathrm{ext}}$.
        The middle row is marked with a triangle.
        Black dotted edges are priced at $\infty$ in all $S^i_{s, t}$.
        The red edges are used to optimize for the lower section ($S^{\mathrm{low}}_{s,t}$), the blue edges for the upper section ($S^{\mathrm{up}}_{s,t}$), and the green edges for the middle section ($S^{\mathrm{mid}}_{s,t}$).
        Whenever one of those groups is used to optimize revenue, the other two
        are priced to $0$.
    }
    \label{fig:pricing_scheme}
\end{figure}

By the previous observations, we have that $R_{S_{s, t}'}(B_{s, t}^*) = R_{S^*}(B_{s, t}^*)$, i.e., drivers in $B_{s, t}^*$ generate exactly the same revenue under $S^*$ and $S'_{s, t}$. Consequently, it would be enough to find a solution $S_{s, t}$ such that $R_{S_{s, t}} \geq \alpha R_{S_{s, t}'}(B_{s, t}^*)$. To do this, note moreover that $R_{S_{s, t}'}(B_{s, t}^*)$ can be written as $R_{S_{s, t}'}^\mathrm{up}(B_{s, t}^*) + R_{S_{s, t}'}^\mathrm{mid}(B_{s, t}^*) + R_{S_{s, t}'}^\mathrm{low}(B_{s, t}^*)$, where the three summands correspond to the revenues generated by the upper, middle, and respectively lower sections of the paths that drivers in $B_{s, t}^*$ will choose under the pricing $S_{s, t}'$.
Hence, at least one of the three summands is at least $\frac{1}{3} R_{S_{s, t}'}(B_{s, t}^*)$, leading us to the next idea: if we could construct three solutions $S_{s, t}^i$ for $i \in \{\mathrm{up}, \mathrm{mid}, \mathrm{low}\}$ such that $R_{S_{s, t}^i} \geq \beta R_{S_{s, t}}^i (B_{s, t}^*)$ for some fixed $\beta > 0$, then taking $S_{s, t}$ to be the best of them achieves our goal for $\alpha = \frac{1}{3} \beta$. In the following, we explain how to do this for each $i$.
For convenience, we will mention the colors of particular edge groups in \cref{fig:pricing_scheme}.

\begin{enumerate}[nosep]
    \item For $i = \mathrm{up}$, note that this is symmetric with $i = \mathrm{low}$, which we treat below.
    \item For $i = \mathrm{mid}$, note that a revenue of at least $R_{S_{s, t}'}^\mathrm{mid} (B_{s, t}^*)$ can be obtained by a pricing derived from $S_{s, t}'$ by leaving the price of edges in the middle row unaltered, and setting the price of all other edges (except those priced $\infty$) to 0. Hence, it suffices to look for a pricing $S_{s, t}$ with the following shape: edges priced $\infty$ in $S_{s, t}'$ (black dotted) are priced $\infty$, all other edges (blue and red) are priced 0, except for one edge in the middle row in the $s$-$t$ range (green), whose price $p$ can vary (except in the case $s = t$, which is immediate). Optimizing over such pricings hence amounts to optimizing over $p$. This is straightforward to achieve in polynomial time by noting that only prices $p = b$ where $b$ is the budget of some driver $(u, v, b) \in B_H$ need to be considered, as otherwise, we could increase $p$ by $\epsilon > 0$ and not price anyone out of the market, increasing the revenue in the process. This achieves our goal for $\beta = 1$.
    \item For $i = \mathrm{low}$, note that a revenue of at least $R_{S_{s, t}'}^\mathrm{low} (B_{s, t}^*)$ can be obtained by a pricing derived from $S_{s, t}'$ by leaving the price of edges set to $\infty$ in $S_{s, t}'$ (black dotted) as $\infty$ and setting the price of all non-$\infty$ edges on the middle row and above to 0 (green and blue). Hence, it suffices to look for a pricing $S_{s, t}$ with the following shape: edges priced $\infty$ in $S_{s, t}'$ (black dotted) are priced $\infty$, other edges on or above the middle row are priced $0$ (green and blue), and the remaining edges (meaning those strictly below the middle row together with the edge going down from $t$ -- marked red) can be priced arbitrarily. Optimizing over such pricings amounts to solving a rooted instance consisting of the partial subgrid (red) of $H_\mathrm{ext}$ starting from the middle row of $H$ and going down, with the edges set to infinity removed. In this instance, the root is vertex $t$, and we replace for all drivers their pair $(u, v)$ with $(t, v)$. We know how to obtain a pricing that \Crounding-approximates the optimal revenue for rooted instances in polynomial time by \cref{sect:rooted}, meaning that we can achieve our goal for $\beta = \frac{1}{\Crounding}$.
\end{enumerate}

Combined, our considerations give the following result for a single block:

\begin{lemma}
For the instance defined by drivers $B_H$ and the extended block $H_{\mathrm{ext}}$, let $\mathrm{REV}_H$ be the revenue of the solution returned by the described algorithm and 
$\mathrm{OPT}_H$ be the revenue of an optimal solution. Then:
\[ \mathrm{REV}_H \geq \frac{1}{\Crounding \cdot 3 \cdot \omega^2} \mathrm{OPT}_H \]
\end{lemma}

Plugging 
back into our main algorithm, we gain another factor of $2$ from the odd-even splitting on each level, and a logarithmic factor from the decomposition, so our algorithm 
achieves an approximation factor of $24 \omega^2 \log m$ in poly-time for any fixed 
$\omega$.
