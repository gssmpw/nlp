Having established our result for grids with bounded width, it would be interesting to see if our ideas extend to the case of bounded-pathwidth or bounded-treewidth graphs, or to grids with some edges removed. Moreover, solving grids without a bound on their width seems like a natural first step to understanding the complexity of our problem on real-world networks (e.g., Manhattan-like cities). Moreover, note how for $\omega = 1$ our problem admits a PTAS, but already for $\omega = 2$, all we could give was a logarithmic-factor approximation. It would be interesting to see if constant-factor approximations are possible, at least for small $\omega$.

More broadly, in the introduction, we noted a number of modeling challenges that need addressing before results in this line of work could become practically applicable. There is uncertainty in travelers' behavior, uncertainty in the budgets, and congestion that needs to be kept in check in a realistic situation. Furthermore, the strategic or otherwise non-rational behavior of agents should also be considered. Along the same lines, the assumption that drivers take the cheapest path without regard for actual traveling time is unrealistic and would need to be relaxed. It would be interesting to see which assumptions of our setting could be relaxed to bring practical applicability into this line of work.