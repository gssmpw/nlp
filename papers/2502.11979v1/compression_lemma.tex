    \begin{restatable}{lemma}{compressionlemma}\label{lemma:compression}
    For any weighted grid of width $\omega$ there exists another weighted grid of the same width and depth at most $\MaxGadgetDepth$
    such that the distances between vertices in the first row of the original grid are preserved.
    \end{restatable}

    For the sake of brevity, here we only present the main idea of the proof, while the full proof can be found in \cref{apx:compression_lemma}.
    First, we identify a partial grid that has a simple structure, but maintains the distances between the vertices in the first row.
    We realize this by finding a collection of all-pairs shortest paths between vertices in the first row which results
    in few \emph{crossing vertices}, i.e., vertices where two paths from the collection join or diverge.
    Only those vertices can have a degree greater than two in the graph induced by the collection of shortest paths.
    Then, we compress areas of the partial grid that contain only vertices of degree two.
    Because there is only a constant number of vertices of degree three or more, we arrive at the desired result.
    The existence of the aforementioned collection of shortest paths is guaranteed by the following lemma.

    \begin{restatable}{lemma}{crossingvtexlemma}
    \label{lemma:crossing_vertices}
    For any weighted graph $G$ and a subset $S$ of its vertices, there exists a collection
    of shortest paths in $G$ between all pairs of vertices from $S$ that results in at most
    ${\abs{S} \choose 2} \cdot \bra{{\abs{S} \choose 2} - 1}$ crossing vertices outside $S$.
    \end{restatable}

    The proof of the above, which can also be found in \cref{apx:compression_lemma}, is based on the observation that
    when two paths have multiple crossing vertices, one of them can be rerouted so that the number of crossing vertices is limited to two.
    Unfortunately, because of the lack of a natural monovariant that would guarantee termination, instead
    of rerouting the paths greedily, the proof
    reroutes the paths in a specific order and maintains a more complex invariant to ensure that the
    number of crossing vertices is limited globally.