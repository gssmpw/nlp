\label{sec:single_price}

For this appendix, assume that $G = (V, E)$ is an arbitrary undirected graph. Consider the following simple mechanism for pricing the edges of $G$: choose a single price $p$ and price all edges as $p$. The following lemma shows that choosing $p$ appropriately leads to a logarithmic-factor approximation of the optimal revenue.

\begin{lemma}
There exists a single price $p$ such that if each edge
in the graph $G$ is priced at $p$, then the revenue is at least $\frac{1}{4 (\log |E| + \log |B| + 1)}$ of the optimal one.
\end{lemma}
\begin{proof}
    By $\mathrm{hop}(u, v)$ we denote the hop distance (the number of edges in the shortest path) between vertices $u$ and $v$.
    If all edges have the same cost, then $\frac{b}{\mathrm{hop}(u, v)}$ is the maximum price a driver $(u, v, b) \in B$ is willing to pay.
    It is because under a single price she will always choose a path consisting of $\mathrm{hop}(u, v)$ edges.
    Thus, we will group the drivers into $\log |E| + \log |B| + 1$ groups by similar values of $\frac{b}{\mathrm{hop}(u, v)}$.
    
    However, before doing so, we discard all drivers with $\frac{b}{\mathrm{hop}(u, v)} < \frac{b_{max}}{|E| |B|}$,
    where $b_{max}$ is the maximum budget of a driver.
    Observe that the sum of their budgets it at most $b_{max}$.
    Since drivers who have budget $b_{max}$ do not belong to this group (and there is at least one of them), the sum of budgets of all drivers
    with $\frac{b}{\mathrm{hop}(u, v)} < \frac{b_{max}}{|E| |B|}$ is at most half of the sum of budgets overall.

    Now, we partition the remaining drivers by the value of $\frac{b}{\mathrm{hop}(u, v)}$
    into $\log |B| + \log |E| + 1$ buckets of the form $\left( \frac{b_{max}}{2^{k+1}}, \frac{b_{max}}{2^k}\right]$ for
    $k \in \set{0, 1, \ldots, \floor{\log |E| + \log |B|}}$.
    Together, all those groups of drivers dispose of at least half of the total budget.
    By pigeonhole principle, the sum of the budgets of drivers in one of the buckets must be at least a
    $\frac{1}{2(\log |E| + \log |B| + 1)}$ fraction of the total budget overall.
    Let us fix such a bucket with $\frac{b}{\mathrm{hop}(u, v)} \in \left( \frac{b_{max}}{2^{k+1}}, \frac{b_{max}}{2^k}\right]$
    and denote respective drivers as $B_k$.
    Let us consider the revenue $\mathrm{REV}(p)$ for the single price $p = \frac{b_{max}}{2^{k+1}}$:
    % 
    \begin{gather*} \mathrm{REV}\bra{\frac{b_{max}}{2^{k+1}}}
    \geq \sum_{(u, v, b) \in B_k} \frac{b_{max}}{2^{k+1}} \cdot \mathrm{hop}(u, v) \geq
    \\ \frac{1}{2} \sum_{(u, v, b) \in B_k} \frac{b}{\mathrm{hop}(u, v)} \cdot \mathrm{hop}(u, v)
    = \frac{1}{2} \sum_{(u, v, b) \in B_k} b
    \end{gather*}

    The first inequality follows from the fact that all drivers in $B_k$ are able to afford their paths under $p = \frac{b_{max}}{2^{k+1}}$,
    the second follows from the definition of the driver partition.
    Since, as we have argued before, $\sum_{(u, v, b) \in B_k} b \geq \frac{1}{2(\log |E| + \log |B| + 1)} \sum_{(u, v, b) \in B} b$,
    this ends the proof, because the sum of budgets is a natural upper bound on the revenue.
\end{proof}

As we observed in the proof above, $\frac{b}{\mathrm{hop}(u, v)}$ is the maximal single price acceptable for a driver $(u, v, b) \in B$.
Thus, the optimal single price has to belong to $\longset{\frac{b}{\mathrm{hop}(u, v)}}{(u, v, b) \in B}$.
Otherwise, we could increase the price by a small $\epsilon > 0$ without losing any drivers, hence increasing the revenue in the process.
Consequently, the optimal single price can be found in polynomial time by checking all elements of the above set and
choosing the one that maximizes revenue.
Hence, we have proven:

\begin{restatable}{theorem}{singleprice}\label{thm:single_price}
There exists a polynomial-time approximation algorithm on general graphs achieving a revenue within a factor of $ \OO{\log |B| + \log |E|}$ of the optimal.
\end{restatable}