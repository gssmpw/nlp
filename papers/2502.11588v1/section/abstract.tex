% \IEEEtitleabstractindextext{
\begin{abstract}
% 凝练说法
The integration of artificial intelligence into automated penetration testing (AutoPT) has highlighted the necessity of simulation modeling for the training of intelligent agents, due to its cost-efficiency and swift feedback capabilities. Despite the proliferation of AutoPT research, there is a recognized gap in the availability of a unified framework for simulation modeling methods. This paper presents a systematic review and synthesis of existing techniques, introducing \modelcla~to categorize studies based on literature objectives, network simulation complexity, dependency of technical and tactical operations, and scenario feedback and variation. To bridge the gap in unified method for multi-dimensional and multi-level simulation modeling, dynamic environment modeling, and the scarcity of public datasets, we introduce \modelsim, a novel modeling framework that based on policy automation and encompasses the combination of all sub dimensions. 
\modelsim~offers a comprehensive approach to modeling network environments, attackers, and defenders, transcending the constraints of static modeling and accommodating networks of diverse scales. We publicly release a generated standard network environment dataset and the code of Network Generator. By integrating publicly available datasets flexibly, support is offered for various simulation modeling levels focused on policy automation in \modelcla~and the network generator help researchers output customized target network data by adjusting parameters or fine-tuning the network generator.


% As artificial intelligence advances in automated penetration testing (AutoPT), simulation modeling has become crucial for training intelligent agents due to its low cost and quick feedback. Despite the increase in AutoPT research, 仿真建模方法 remain fragmented and lack a unified framework.
% This study systematically reviews and summarizes simulation modeling technologies in AutoPT.
% We present the Multi-Dimensional Penetration Testing Simulation Classification System (\modelcla), designed to classify papers based on literature objectives, network simulation complexity, dependency of technical and tactical operations, and scenario feedback and variation. 系统的调研分析当前AutoPT研究中仿真建模的没有统一框架指导、缺乏动态环境建模、缺乏公共数据集的问题。
% Building on this, we propose \modelsim, a novel modeling framework for AutoPT, focused on policy automation维度,全方位地指导目标network environments、攻击者和防御者建模. It addresses the limitations of current scene modeling methods, which lack fully dynamic scene modeling and focus on small to medium networks.
% We also release the source code of a network generator and simulated network dataset \wyf{add link later}. It supports various simulation modeling levels focused on strategy generation automation by integrating publicly available dataset flexibly. The network generator enables researchers to output customized network by adjusting parameters or fine-tuning.
\end{abstract}

\begin{IEEEkeywords}
Automated Penetration Testing, Simulation Modeling, Penetration Testing Modeling 
\end{IEEEkeywords}
% }