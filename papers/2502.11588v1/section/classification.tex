\section{The Multi-Dimensional Classification System for Penetration Testing Modeling: \modelcla}
% \begin{figure*}[t]
%     \centering
%     \includegraphics[width=\linewidth]{figure/fig_model_.pdf}    \caption{Overview of Methodology. We initially generate a batch of rule instances $\{(\mathbf{r_b}, r_h)\}_N$ from the train graph. The rule bodies are translated into representations with the encoder $\phi_e$.
%     Subsequently, the decoder $\phi_d$ learns to classify these embeddings into corresponding rule heads. To eliminate spurious correlations present in $\mathbf{r_b}$, we derive a batch of weights $\textbf{W}$, with feature decorrelation, to appropriately re-weight the training samples. The overall optimization goal is to minimize the weighted loss.}
% \label{figure_model}
% \end{figure*} 

% \section{仿真建模方法分类标准及典型示例}
% \section{\modelcla}
% 我们基于攻击场景的多维特性,创新的提出了一个多角度的分类方法Multi-Dimensional Penetration Testing Simulation Classification System(\modelcla),从场景自动化目标、场景模拟复杂度、前后技战术关联、场景反馈与变化四个维度对攻击场景建模进行分类,每个维度都包含若干个子维度,以确保全面覆盖不同的攻击场景类型,分类框架见图。在本节中我们将首先详细介绍我们的分类标准,然后给出若干个典型示例以辅助读者更好的理解我们的分类方法,然后对调研地相关文献进行分析与总结。

In this section, we present the \textbf{M}ulti-\textbf{D}imensional \textbf{C}lassification System for \textbf{P}enetration Testing \textbf{M}odeling (\modelcla), an innovative approach for classifying target network scenarios modeling method in penetration testing. 
This system categorizes scenario modeling based on four principal dimensions: (1) Literature Objectives, (2) Network Simulation Complexity, (3) Dependency of Technical and Tactical Operations, and (4) Scenario Feedback and Variation. Each dimension comprises sub-dimensions for a nuanced classification of attack scenarios, as illustrated in Figure~\ref{Class_dim}. This paper begins with an explanation of our classification criteria, followed by illustrative examples, and concludes with an analysis of the reviewed literature.


\begin{figure*}[tb]
    \centering
    \includegraphics[width=130mm]{figure/Class_dim.pdf}
    \caption{The Multi-Dimensional
Classification System for Penetration Testing Modeling}
    \label{Class_dim}
\end{figure*}


\subsection{Dimensionality of~\modelcla}
% In this section, we delineate the definitions and implications of the four dimensions that underpin the \modelcla, while also detailing their associated sub-dimensions. 
In this section, we define and introduce the four primary dimensions characterizing the \modelcla, along with their associated sub-dimensions.


% \subsection{Multi-Dimensional Penetration Testing Simulation Classification System:\modelcla}
% 我们将详细介绍场景自动化目标、场景模拟复杂度、前后技战术关联、场景反馈与变化四个核心维度及其子维度的定义与内涵。

% \subsubsection{场景自动化目标:}
% AutoPT的终极目标在于实现渗透测试的自动化,它能够自主确定各种场景下的渗透目标和方法,集成工具以实现自动执行,并协助专家提高渗透效率。这一过程可以分为两个主要部分:智能决策和自动执行。智能决策涉及到对渗透测试目标的识别、攻击路径的选择以及攻击手段的确定,而自动执行则是指在决策基础上,自动化地实施渗透测试操作。通过这两个组成部分的协同工作,AutoPT旨在提高渗透测试的效率和效果,减少对专家的依赖,使得渗透测试更加快速和精准。

% 在文献目标的研究中,我们关注的是AutoPT在不同研究中的objectives。这些研究可能涉及开发了一种渗透工具,提出了一种渗透策略,或者是介绍了一种渗透平台。基于对渗透测试的两个基本环节——智能决策和自动执行的认识与组合,我们划分了该维度的子维度,并进行了系统和全面的文献分析。我们通过组织每篇文献的研究背景、目标和意义,以识别文献目标。通过这种方法,我们能够深入理解每篇文献的贡献,并评估它们在智能决策和自动执行方面的创新和应用。这个维度可以帮助我们认知自动化渗透测试研究中的主题分布和研究进展,不仅是理论上自动化渗透测试的发展,也包括Auto PT在实践中的效果和潜力,能够更好地理解自动化渗透测试的当前状态和未来趋势。

% \textbf{战术自动化:}关注具体技战术的自动化执行,如自动化漏洞扫描、自动化攻击脚本。这种类型通常只关注具体技战术的执行而不考虑整体的策略规划,最终也只需要验证技战术执行的有效性,一般常见于各种执行工具自动化执行,如Nmap\cite{lyon2009nmap}、Fscan、webshell等能后完成自动化扫描存活主机、开放端口与服务,获取指纹信息,Nessus、AWVS等工具能够完成自动化漏洞扫描工作,Metaspliot能自动化完成漏洞验证与利用操作,这些工具是自动化渗透测试能够自动执行的重要支撑。
% \textbf{策略自动化:}关注整体攻击策略的自动化生成和实施,也就是聚焦于自动化渗透测试的智能决策部分,自动规划攻击路径及相关技战术,是当前应用人工智能方式研究自动化渗透测试的方法主要目标。
% \textbf{全流程自动化:}涵盖从智能决策到自动执行的整个攻击流程的自动化,在自动规划攻击路径及相关技战术地基础上,集成了自动执行工具和真实载荷,在不需要人工干预的基础上达到能够在网络靶场或真实网络中能够进行实际渗透测试的效果。

% 场景自动化目标是所有后续建模维度的基础,能够指引网络模拟复杂度建模与前后技战术关联建模。

\subsubsection{Literature Objectives}
% The ultimate goal of AutoPT is to autonomously determine targets and methods of penetration across various scenarios, integrate tools for automatic execution, and assisting experts in improving penetration efficiency. 
%  This process is bifurcated into intelligent decision-making, which encompasses target identification, attack path selection, and method determination, and automatic execution, which involves the mechanical implementation of testing based on these decisions. By synergizing these components, AutoPT seeks to boost the speed and precision of penetration testing.

% Literature Objectives focus on Objectives and outcomes in different studies of AutoPT, which may include tool development, policy proposals, or platform introductions. Based on the understanding and combination of the two phases of AutoPT, intelligent decision-making and automatic execution, we have divided this dimension into three sub-dimensions and conducted a systematic and comprehensive literature analysis. By organizing the research background, objectives, and significance of each paper, we identify the objectives of the literature and gain an in-depth understanding of the contributions of each paper and assess their innovations and applications.

The Literature Objectives analyze the aims and outcomes of various AutoPT studies, including tool development, policy proposals, and platform introductions. Based on the two phases of AutoPT—intelligent decision-making and automatic execution—we categorize this dimension into three sub-dimensions: . 
Specifically, technical automation pertains to automatic execution, while policy automation refers to intelligent decision-making; when a study simultaneously addresses both aspects, we classify it as complete automation.
By organizing the research background, objectives, and significance of each study, we identify the literature's goals and gain insights into each paper's contributions, innovations, and applications.



% 这个重点领域涉及特定技术和战术程序的自动执行,如端口扫描、漏洞扫描和攻击脚本执行。它主要涉及在没有战略规划的情况下直接应用特定的技术和战术措施。目的是验证所采用措施的有效性。这种自动化在Nmap\cite{lyon2009nmap}、Fscan\cite{Fscan}和Webshell\cite{longxiao2018webshell}等各种执行工具中很普遍,这些工具可以识别实时主机、开放端口、服务和收集指纹信息,而Nessus和AWVS执行自动漏洞扫描,Metasploit自动验证和利用漏洞。这些工具在AutoPT的自主执行阶段起着重要作用。
% \textbf{Technical and Tactical Automation (TT-Auto):} This domain focuses on the direct application of specific technical and tactical measures without strategic planning. Penetration testing tools automate predefined testing steps, such as identifying live hosts, open ports, and services, and performing vulnerability scans. They are designed to validate the efficacy of employed security measures by efficiently executing repetitive tasks, thereby reducing manual errors and time costs. While these tools, including Nmap \cite{lyon2009nmap}, Fscan\cite{Fscan}, Webshell\cite{longxiao2018webshell}, Nessus, AWVS, and Metasploit, are crucial for the autonomous execution phase of AutoPT, they represent only one aspect of the penetration testing process. A comprehensive security assessment requires the integration of strategic planning and intelligent decision-making to effectively identify and address potential security issues.


\textbf{Technical Automation:} This dimension emphasizes the direct implementation of specific technical and tactical procedures without strategic planning. It represents the earliest stage of automation, characterized by a lower level of intelligence in AutoPt.
It automates predefined testing technicals such as identifying live hosts, open ports, services, and conducting vulnerability scans. Tools are central to this process. Nmap~\cite{lyon2009nmap}, Fscan~\cite{Fscan}, and Webshell~\cite{longxiao2018webshell} facilitate the identification of live hosts and services. Nessus and AWVS conduct automated vulnerability scans. Metasploit automates vulnerability verification and exploitation. These tools are essential for the autonomous execution phase of AutoPT, enabling the efficient performance of repetitive tasks and minimizing manual errors and time expenditure. 


\textbf{Policy Automation:} This aspect focuses on automated generation of attack policy without real-world execution, crucial for intelligent decision-making in AutoPT. It involves the automated planning of attack paths and technical actions, representing a key focus in contemporary AI research for AutoPT~\cite{enoch2020harmer,schwartz2020pomdp+,becker2024evaluation}. For instance, Hu et al. use Multi-host Multi-stage Vulnerability Analysis to construct an attack tree for network topologies, applying Deep Q-Networks (DQN) to identify the most exploitable attack paths~\cite{hu2020automated}. Zhou et al.~\cite{zhou2021autonomous} frame penetration testing as a Markov Decision Process and used an improved deep q-network to decouple actions and learn attack strategies. However, their methodologies remain limited to policy generation in theoretical network environments, lacking integration with actual penetration tools or execution of attack payloads in real-world scenarios.

% Hu等人\cite{hu2020automated}采用深度强化学习来自动化渗透测试过程,采用多主机多阶段漏洞分析(MulV AL)来生成网络拓扑的攻击树,深度学习网络(DQN)方法从可能的候选者中发现最容易被利用的攻击路径。Zhou等人\cite{zhou2021autonomous}渗透测试建模为马尔可夫决策过程问题,使用改进的深度Q网络(DQN)NDSPIDQN,并对动作进行解耦,生成攻击策略。他们的研究都只涉及在抽象的网络环境中生成攻击策略,但没有结合真实的渗透工具与攻击载荷,在真实的网络环境中执行渗透操作。

\textbf{Complete Automation:} 
While the two aforementioned aspects are crucial, they both represent only singular aspects of penetration testing. A comprehensive security assessment requires the integration of automatic execution and intelligent decision-making to effectively identify and mitigate potential security threats.
Complete automation encompasses automation of the entire attack lifecycle, from decision-making to execution. This includes the automatic planning of attack paths and the integration of execution tools with actual payloads to perform real-world penetration tests in either simulated or live network environments, entirely without human intervention~\cite{dorchuck2021goal,sarraute2013automated,xu2024autoattacker}.

Literature objectives deepen our understanding of research trends in AutoPT, providing insights into theoretical advancements and practical implications while clarifying the current landscape and future directions. 
Moreover, these objectives could also facilitate the classification of network simulation complexity, modeling dependencies between technical and tactical operations, as well as scenario feedback and variation.

% \subsubsection{网络模拟复杂度:}
% 网络模拟复杂度是针对自动化渗透测试仿真建模的第一个要素,即网络架构与目标资产提出的分类维度,根据对网络架构与目标资产的抽象程度及构建方式,我们将其分为两类:

% \textbf{非真实属性模拟:}非真实属性模拟是的是在建模网络架构与目标资产时采用数值、规则或概念等高度抽象的方式建模资产及架构。如Hammar等人\cite{2020Finding}使用一个只有四个节点的网络环境进行渗透测试研究,每个节点使用数值属性代表节点自身的防御能力。FlipIt\cite{2013Flipit}没有对网络环境进行建模,而是只采用“资源”这一模糊概念代替网络中所有有价值的资产,,动作Cyber Hide-and-Seek\cite{2017Hide}将攻防互动的过程抽象成隐藏物品和网络搜索问题,通过Hider和Seeker之间的交互移动体现攻防双方的策略过程。

% \textbf{真实属性模拟:}真实属性模拟是的是在建模网络架构与目标资产时使用现实生活中的系统、软件、服务、账号密码、漏洞等真实信息建模目标资产及属性,并设计复杂的网络拓扑来模拟现实生活中多变的网络环境,这反映了在一个复杂网络中,各个节点(网络设备、服务器、应用)的内部属性及节点之间的相互关系。如微软开源的CyberBattleSim\cite{Cyberbattlesim}中定义了每一个节点的操作系统和软件、漏洞和节点价值等信息,并定义了多个小型场景中节点不同的连接关系。
% % 真实网络靶场环境:(要不要包括,如果写的话,我的建模就不能涵盖所有层次,不写的话真实场景有点不好应对,因为不是某一种模拟方式)


\subsubsection{Network Simulation Complexity}

This dimension focuses on network architecture and target assets-the first element of AutoPT simulation modeling. This dimension is further divided into two sub-dimensions based on the abstraction level and construction methods of network attributes: hypothetical and authentic attributes. 

% 每个节点具有多维的数值属性,每个维度代表不同方面的防御能力,其中最后一维代表节点的探测能力,该维的数值越大,代表节点在这一方面的防御能力或探测能力越强。
\textbf{Simulation of Hypothetical Attributes:} Numerous studies utilize numerical, rule-based, or conceptual methods to abstractly model assets and architectures. For instance, Hammar et al. utilize numerical attributes to characterize nodes in a four-node network, with each node represented by multidimensional metrics of defensive and detection capabilities~\cite{2020Finding}. 
% FlipIt~\cite{2013Flipit} describes a two-player game involving a shared resource between an attacker and a defender. Rather than modeling the network environment with detailed attributes, they simplify valuable assets to the concept of resources.

% Hide-and-Seek~\cite{chapman2014playing}将攻防互动的过程抽象成

% Each node possesses multidimensional numerical attributes, with each dimension representing different aspects of defensive capabilities. The final dimension represents the node's detection capability; a higher value in this dimension indicates a stronger defense or detection ability in that aspect.


% Cyber Hide-and-Seek [50]
% framed attacker-defender interactions as a hidden object and
% a strategic search problem, reflecting the strategic process of
% both sides through the interaction and movement between
% Hider and Seeker

\textbf{Simulation of Authentic Attributes:} Certain studies employ real-world systems, software, services, account passwords, vulnerabilities, and other real information to model target assets and attributes. These works utilize complex network topologies to accurately replicate real-life environments, reflecting both node attributes and their interrelationships. For example, Microsoft's CyberBattleSim defines the operating systems, software, vulnerabilities, and node reward for each node while establishing diverse connection relationships across multiple small scenarios involving fewer than 20 nodes~\cite{Cyberbattlesim}.


% \subsubsection{前后技战术关联:}
% 前后技战术关联是针对自动化渗透测试仿真建模的第二个要素,即攻击者、防御者模型要素提出的分类维度,指的是在进行攻击者、防御者建模过程的动作定义时是否考虑不同攻击/防御动作之间的前后使用限制,即是否定义了一个动作的执行成功/失败是另一个动作执行的前提。

% \textbf{孤立技战术:}单独的攻击技战术,不依赖于其他攻击步骤,通常情况下,不会定义动作的任何前置条件。包括单一战术的执行,如Namp等单独的扫描动作的利用;或定义多个动作,但相互之间没有定义限制,如Sarraute等人\cite{sarraute2013penetration}在渗透场景攻击者建模中仅定义了“扫描”和“漏洞利用”两个动作,尽管智能体在学习过程中可以学习到扫描动作执行后选择正确漏洞的成功率提高这一隐形规则,从而倾向于先扫描再利用漏洞,但是在定义这两个动作过程中,没有对动作的执行定义任何前提条件,“扫描”和“漏洞利用”两个动作均可以单独执行。

% \textbf{连续技战术:}多个动作之前相互存在限制关系,不同动作依赖于前后执行的关系,共同构成一个完整的攻击链。通常出现在提权、横向移动等后渗透环节,漏洞利用获取初级权限后才能后进行权限提升,钓鱼攻击后配合恶意软件的植入,再进行凭证窃取等后续操作。采用此种方式建模时,要求定义所有定义的前置条件与后置效果。
% Filiol等人\cite{filiol2021method}通过定义完成的得到域名,IP扫描,收集服务软件版本及名称,获得攻击列表,配置攻击等动作前后执行的逻辑关系,对攻击者进行建模。



% 在现有研究中,孤立技战术和连续技战术两种分类方式通常是混合使用的,如在CALDERA\cite{applebaum2016intelligent}中,使用连续技战术模拟攻击者成功入侵网络后,利用网络中的安全依赖性进行横向移动、提权和数据窃取等操作;但对于防御者建模,则使用了孤立技战术建模,模拟一个简单的、被动的防御者,对攻击者的存在一无所知,只能随机执行重启和登录操作的独立操作,这种情况我们一般将其划分为连续技战术类别,因为其在建模过程中考虑了这一因素。
% 值得注意的是前后技战术关联维度针对的是不同动作之间先后执行的限制关系,需要在仿真建模中进行明确定义,这与智能体在大量交互过程中学习到的不同动作执行顺序倾向性是不同的,前者需要在仿真建模中进行明确定义,后者是隐形包含的。我们认为绝大多数的智能体都需要学习动作执行顺序倾向性,即决策攻击动作,这是评估智能体能力的内容,而非仿真建模的内容。除此之外,不同渗透动作的前后执行限制关系是现实生活中客观存在的,但是由于动作的多样性导致限制条件多样,难以使用规范的形式进行表达,很多研究内容中缺失这一部分关系,实际上是降低了动作执行的限制,使得决策参数得以统一,但是针对仿真建模而言,则是提高了抽象程度,增大了与现实世界的差距,在仿真建模环境向真实环境渗透过渡过程中,这一问题是必然需要解决的。

\subsubsection{Dependency of Technical and Tactical Operations}

This dimension analyzes attacker and defender models to assess whether their defined actions incorporate interdependencies, specifically determining if the outcome of one action serves as a prerequisite for subsequent actions.

\textbf{Isolated Technical and Tactical Actions:} Many technical and tactical actions are independent, lacking defined prerequisites. This category includes executing single tactics, such as using Nmap for scanning, or multiple unrestricted actions.
For example, Sarraute et al.~\cite{sarraute2013penetration} defined only scanning and vulnerability exploitation actions in a penetration scenario. Although agents may implicitly learn that scanning before exploitation improves the success rate of vulnerability selection, the action definitions do not specify prerequisites, allowing actions to be performed independently.
 

\textbf{Coordinated Technical and Tactical Actions:} Many technical and tactical actions are interdependent, forming an integrated kill chain. 
This coordination is evident in stages such as privilege escalation and lateral movement, where initial actions like vulnerability exploitation or phishing attacks precede subsequent activities such as credential theft and malware implantation. Defining the preconditions and post-effects of actions is essential. For example, Filiol et al.~\cite{filiol2021method} modeled attacker behavior by specifying logical relationships among actions, including domain name acquisition, IP scanning, service version collection, attack list generation, and attack configuration.

% Dependency of Technical and Tactical Operations illustrates the restrictive relationship between sequential actions, which need to be clearly defined in AutoPT modeling. 
% This differs from the action preference sequences learned through agent interactions and training, which are implicitly included. Most agents need to learn action order tendencies, an assessment of the agent's capabilities rather than a modeling content. 
% Real-life penetration actions are constrained by execution limitations, and their variety complicates standardization efforts. 
% Many studies omit these relationships, reducing execution constraints and unifying decision parameters but increasing abstraction and divergence from real-world scenarios. Addressing this gap is crucial when transitioning from simulation models to real environments.

Dependency of Technical and Tactical Operations investigates the explicit interrelationships between sequential actions essential for precise simulation modeling. 
In real-world penetration testing scenarios, actions are constrained by execution limitations, and their variability hinders standardization efforts.
Many studies overlook these dependencies, thereby simplifying execution constraints and standardizing decision parameters, which increases abstraction and reduces alignment with real-world conditions. Addressing this gap is crucial for the effective transition of simulation models to practical environments.

In existing research, isolated and coordinated technical and tactical operations are frequently combined. For example, CALDERA ~\cite{applebaum2016intelligent} employs coordinated technical and tactical actions to simulate an attacker’s lateral movement, privilege escalation, and data theft by imposing action dependencies. Meanwhile, it also utilizes isolated technical and tactical actions to represent a passive defender unaware of the attacker, limited to independent actions such as random reboots and logins. Although classified as a continuous tactic scenario, CALDERA incorporates isolated tactics in its modeling.

% In the literature, isolated and coordinated technical and tactical actions are frequently combined. For example, CALDERA~\cite{applebaum2016intelligent} employs coordinated technical and tactical actions to simulate an attacker’s lateral movement, privilege escalation, and data theft by leveraging action dependencies. Conversely, it utilizes isolated technical and tactical actions to represent a passive defender unaware of the attacker, limited to independent actions such as random reboots and logins. Although classified as a continuous tactic scenario, CALDERA incorporates isolated tactics in its modeling.
 
% \subsubsection{场景反馈与变化:}
% 场景反馈与变化针对的是目标网络本身的网络架构与目标资产的主动变化和攻击者、防御者之间交互引起的被动变化的分类维度。值的注意的是,这个维度中发生变化的主体是目标网络背景。即网络架构与目标资产本身的变化,如某一个主机与其他主机的连接关系变化,该节点安装的系统与软件,具有的账号和密码,存在的漏洞等属性信息,但是不包括与攻击者、防御者相关信息的变化,如攻击者在某一个主机中的权限等级,攻击者新获取的账号密码等信息。
% S-FV includes two kind of changes. 第一种是场景反馈,指的是由于攻击者、防御者之间交互引起的目标网络的反馈变化,是一种被动变化,即由于攻击者的攻击动作,如建立节点间的连接关系、发送钓鱼邮件或病毒木马引起大规模网络瘫痪等,和防御者的防护动作,如关机、清理登录凭证、更新软件等,导致网络架构与目标资产发生变化。当攻击者和防御者不采取动作时,这些变化不会发生,因此这是一种被动变化。第二种变化是目标网络的目标资产与网络架构的主动变化,即在仿真建模中为了模拟现实生活中用户操作与交互行为或动态网络设计,预先定义的一种变化行为,如通过通过设计用户代理的行为活动、设置定时开关机模拟员工上下班行为引起网络变化,设置集中时段流量模拟任务演练的行为,设置某些节点定时更新IP与更新系统模拟移动目标防御(Moving Target Defense,MTD)等均属于主动变化。

% \textbf{完全静态场景}:指的是预定义的固定场景,既不存在由于攻防交互引起的被动变化,又不存在目标场景的主动变化。

% \textbf{半静态场景:}存在由于攻防交互引起的被动变化,但不存在目标场景的主动变化。

% \textbf{完全动态场景:}既存在由于攻防交互引起的被动变化,又存在目标场景的主动变化。
\subsubsection{Scenario Feedback and Variation}

% The dimension of Scenario Feedback and Variation relates to classifying modifications in the target network's architecture and assets. Notably, Scenario Feedback and Variation only focuses on changes in the network's architecture and target assets, such as alterations in host connectivity, installed systems and software, account credentials, and vulnerabilities, rather than attribute about attackers and defenders, like an attacker’s privilege level or newly acquired credentials. 

This dimension classifies modifications to the target network's architecture and assets, including changes in host connectivity, installed systems and software, account credentials, and vulnerabilities. It does not account for attributes related to attackers and defenders, such as an attacker’s privilege level or newly acquired credentials.
This dimension involves two types of changes: scenario feedback and scenario variation. 

\textbf{Scenario Feedback} refers to passive changes arising from interactions between attackers and defenders that affect the target network's architecture and assets. This includes attacker actions such as establishing connections, deploying phishing emails or malware, and causing network disruptions, as well as defender responses like system shutdowns, credential remediation, and software updates. These alterations occur only when attackers and defenders engage, characterizing them as passive changes.
In contrast, \textbf{Scenario Variation} involves predefined modifications within simulation models designed to emulate real-world user operations or dynamic network configurations. Examples include simulating user behavior, scheduling power operations to reflect work routines, conducting traffic simulations for behavioral drills, periodically updating IP addresses and systems, and implementing defense strategies such as Moving Target Defense (MTD), Cyber Mimic Defense (CMD), load balancing, and Endogenous Safety and Security (ESS). These changes are integrated into the scenario and execute according to predetermined schedules or conditions, independent of the immediate actions of attackers or defenders, thus qualifying them as active changes.
Figure~\ref{the_forth_dim} illustrates the Scenario Feedback and Variation dimension.
% Another one is scenario variation. These are predefined modifications within simulation models to reflect real-life user operations or dynamic network design, such as simulating user behavior activities, scheduling power operations to mimic work schedules, running traffic simulations for behavior drills, and updating IP addresses and systems periodically, setting defense strategies, like Moving Target Defense (MTD) and Cyber Mimic Defense (CMD), load balance, and endogenous safety and security (ESS), and so on.  These changes are built into the scenario and occur according to a predefined schedule or set of conditions util an external input or event causes them to stop. They are not dependent on the immediate actions of attackers or defenders, hence it is an active change. 


\begin{figure}[tb]
    \centering
    \includegraphics[width=80mm]{figure/the_forth_dim.pdf}
    \caption{The connotation and sub-dimensions of the Scenario Feedback and Variation}
    \label{the_forth_dim}
\end{figure}

Based on these two aspects in AutoPT modeling, scenario feedback and variation are categorized into three sub-dimensions:

\textbf{Completely Static Scenario:} A scenario with no passive modifications from attack-defense interactions and no active alterations to the target environment.

\textbf{Semi-Dynamic Scenario:} A scenario that incorporates passive changes resulting from attack-defense interactions but does not include active modifications to the target environment.

\textbf{Completely Dynamic Scenario:} A scenario that encompasses both passive changes from attack-defense interactions and active alterations to the target environment.


% \subsection{典型示例}
% 为了让\modelcla更清晰明确,我们选择了4个典型案例进行讲解分析。
% \subsubsection{metaspliot、nmap、Nessus 等自动化渗透测试工具:战术自动化+孤立技战术+真实属性模拟+完全静态场景}
% 自动化渗透测试工具是网络安全领域的重要工具,它们可以帮助安全专家和IT专业人员自动识别和评估网络、应用程序和系统的安全漏洞。以下是一些流行的自动化工具:
% \begin{itemize}
%     \item Nmap(Network Mapper)% \cite{lyon2009nmap}:是一款网络探测工具和安全/端口扫描器,设计目标是快速地单个主机或扫描大型网络,使用Namp可以进行主机发现、端口扫描、服务识别、操作系统检测、版本扫描、脚本扫描等功能,生成多种输出格式,包括文本、XML、JSON 等,以便进一步的分析和报告。
%     \item Wireshark\cite{WiresharkTCPAnalysis2024}:是一款功能强大、广受欢迎的开源网络协议分析工具,帮助网络管理员、安全专家和开发人员深入了解网络流量,诊断网络问题,确保网络安全,支持实时数据捕获、网络流量过滤分析、数据包解码等功能。
%     \item Nessus\cite{TenableNessus}:由Tenable公司开发,是一款功能强大的漏洞扫描工具,用于识别系统、网络和应用中的安全漏洞。可以对IP地址、域名等扫描目标进行漏洞检查,详细的漏洞报告,包括漏洞描述、严重程度、建议的修复措施等信息。
%     \item Metasploit\cite{MetasploitWebsite}:是一个开源社区和Rapid7之间的合作项目,它帮助进行验证漏洞、管理安全评估和提高安全意识。Metasploit本身附带数百个已知软件漏洞,是一款专业级漏洞攻击工具,综合使用以下模块,对目标系统进行侦察并发动攻击:
%     \begin{itemize}
%         \item Auxiliaries (辅助模块):负责执行扫描、嗅探、指纹识别等相关功能以辅助渗透测试.
%         \item Exploit (漏洞利用模块):由渗透测试者利用一个系统、应用或者服务中的安全漏洞进行的攻击行为,每个漏洞都有相应的攻击代码。
%         \item Payload (攻击载荷模块):成功渗透目标后,选择、传送和植入用于在目标系统上运行任意命令或者执行特定代码。
%         \item Post (后期渗透模块):收集更多信息或进一步访问被利用的目标系统
%         \item Encoders(编码模块):将攻击载荷进行编码,来绕过防护软件拦截
%     \end{itemize}
% \end{itemize}

\subsection{Case Study}

To clarify~\modelcla, we present and analyze four representative cases, including:

\begin{itemize}
    \item Penetration Testing Tools: These tools automate key security tactics and techniques, essential for conducting AutoPT in real-world scenarios.
    \item Numerical Simulation Networks~\cite{2020Finding}: Although highly abstract and detached from real-world conditions, these networks provide a theoretical framework for exploring cybersecurity complexities.
    \item CyberBattleSim~\cite{Cyberbattlesim}: Developed by Microsoft, this simulation platform is employed in various studies for both simulation and emulation purposes.
    \item Network Attack Simulator~\cite{schwartz2019autonomous}: Released by Harvard University, this popular simulator specializes in network attack simulations, enhancing the understanding of penetration testing behaviors.
\end{itemize}

% These cases are not only impactful and widely cited but also diverse, covering a broad spectrum of classifications in existing research. Analyzing them provides a deeper understanding of the classification principles and applications of \modelcla.

These cases are widely cited and diverse, encompassing a broad spectrum of classifications in existing research. Analyzing them provides deeper insights into the classification principles and applications of \modelcla.

% 这些案例包括:
% 渗透测试工具类别:这一类别的工具在安全领域中扮演着重要角色,它们达成自动化某一种技战术的任务,是自动化渗透测试能够在现实生活中应用的重要一环。
% 数值模拟网络,抽象程度高、与现实世界的差距较大,但为研究者提供了一个理论框架,以探索网络安全的复杂性。
% 微软发布的CyberBattleSim平台:这是一个由微软开发的仿真平台,部分研究使用了该环境作为仿真模拟环境。
% 哈佛发布的Network Attack Simulator:这个由哈佛大学发布的模拟器,专注于网络攻击的模拟,帮助研究人员和实践者更好地理解网络渗透测行为,在谷歌学术上具有较高引用量,影响力大。
% 这些案例不仅具有重要影响,而且在许多研究中被广泛引用和借鉴。它们之间的区分度较大,分类组合涵盖了现有大量研究中的组合形式。通过分析这些案例,我们可以更深入地理解\modelcla的分类原理与应用方式。
% These cases include:
% Penetration testing tool category: Tools in this category play a significant role in the field of security, automating specific tactics and techniques, and are an essential part of applying AutoPT in real-life scenarios.
% Numerical simulation networks, with a high level of abstraction and a significant gap from the real world, but provide researchers with a theoretical framework to explore the complexities of cybersecurity.
% Microsoft's CyberBattleSim platform: This is a simulation platform developed by Microsoft, which has been used by some studies as an environment for simulation and emulation.
% Harvard's Network Attack Simulator: This simulator, released by Harvard University, focuses on simulating network attacks and helps researchers and practitioners better understand network penetration testing behavior. It has a high citation count on Google Scholar and is highly influential.
% These cases not only have a significant impact but are also widely cited and referenced in many studies. They have a large degree of differentiation, and their classification combinations cover a vast array of combinations found in existing research. By analyzing these cases, we can gain a deeper understanding of the classification principles and application methods of \modelcla.

\subsubsection{Penetration Testing Tools}
% Penetration Testing Tools are generally Technical Automation + Simulation of Authentic Attributes + Isolated Technical and Tactical Actions + Completely Static Scenario or None Scenario Restrictions.
% AutoPT tools play a crucial role in network security by assisting professionals in identifying and evaluating vulnerabilities in networks, applications, and systems. The following are some popular automated tools:

Penetration testing tools typically incorporate technical automation, simulate authentic attributes, execute isolated technical and tactical actions, and utilize either entirely static scenarios or impose no scenario restrictions. These AutoPT tools are essential for network security, enabling professionals to identify and evaluate vulnerabilities within networks, applications, and systems.
The following are some widely used automated tools:

\begin{itemize}
    \item Nmap (Network Mapper)~\cite{lyon2009nmap}: A multifunctional security and port scanner designed to efficiently evaluate individual hosts or large networks. It offers features such as host discovery, port scanning, service identification, operating system detection, version scanning, and script scanning.
    % \item Wireshark \cite{WiresharkTCPAnalysis2024}:  A widely-used open-source protocol analyzer that provides deep insights into network traffic, facilitating issue diagnosis and security assurance. It supports real-time data capture, network traffic filtering analysis, packet decoding, and more.
    \item Nessus~\cite{TenableNessus}: A comprehensive vulnerability scanner developed by Tenable, designed to identify security vulnerabilities in systems, networks, and applications. It scans targets such as IP addresses and domains, and generates detailed reports that include vulnerability descriptions, severity ratings, and recommended remediation actions.
    \item Metasploit~\cite{MetasploitWebsite}: Metasploit is a collaborative framework designed for vulnerability validation and security assessments, operating through distinct modules. Auxiliary Modules perform scanning, fingerprinting, e.t.c, to support penetration testing. Exploit Modules utilize identified vulnerabilities to infiltrate systems. Payload Modules execute post-exploitation tasks, enabling arbitrary command execution on targets. Post-Exploitation Modules secure further access and gather additional data from compromised systems. Encoder Modules obfuscate payloads to bypass security mechanisms.

    % \begin{itemize}
    %     \item Auxiliaries (Auxiliary Modules): Facilitate scanning, sniffing, fingerprinting, and other functions to support penetration testing.
    %     \item Exploit (Exploit Modules): Enable penetration testers to exploit identified vulnerabilities in systems, applications, or services, utilizing corresponding attack code.
    %     \item Payload (Payload Modules): Employ post-exploitation to transfer and execute code on the target system, allowing for arbitrary command execution.
    %     \item Post (Post-Exploitation Modules): Gather additional information or secure further access to compromised systems.
    %     \item Encoders (Encoder Modules): Encode payloads to evade detection by security mechanisms.
    % \end{itemize}
\end{itemize}

% 自动化渗透测试相关的工具很多,但这些工具大多集中于渗透测试中的一个环节,并且依赖于人工输入相关目标和参数,难以在没有人工参与的情况下完成渗透测试的全流程,其目标是完成渗透测试中某一个环节的自动化执行,因此归属于技战术自动化;每个工具的技战术都可以单独执行,属于孤立技战术;属性完全来源是现实生活,属于真实属性模拟;所有的工具都可以应用于多种现实场景,完全取决于研究人员的设置,故场景未设置限制,但是在大多数情况下,这些工具由于接入了真实的攻击载荷,受到法律限制,只能在安全可控的场景下进行,一般是网络靶场或隔离内部网络,在不设置主动变化或用户交互的基础上,可以理解为完全静态场景。
% Many tools are related to AutoPT, but most of these automating specific steps of penetration testing and often require manual input for targets and parameters. They are challenging to complete the entire penetration testing process without human involvement. They are categorized as Technical Automation. The attributes are entirely derived from real life, falling under Simulation of Authentic Attributes. Each tool's technical and tactical measures can be executed in isolation, belonging to Isolated Technical and Tactical Actions. While applicable across varied scenarios, they are typically constrained to secure environments due to legal and ethical considerations, functioning in static scenarios without active user interaction. 

Many AutoPT tools automate specific penetration testing steps but often require manual input for targets and parameters, making it difficult to conduct the entire process without human involvement. These tools fall under Technical Automation and simulate authentic attributes based on real-world network. Their technical and tactical measures can be executed in isolation, categorizing them as Isolated Technical and Tactical Actions. Although applicable to various scenarios, they are typically confined to secure environments due to legal and ethical considerations and operate within static scenarios without active user interaction.


% \subsubsection{数值模拟小型网络:策略自动化+非真实属性模拟+孤立技战术+半静态场景}
% Hammer等人~\cite{2020Finding}通过构建了一个只有四个节点的数值模拟网络对渗透测试中的攻防交互博弈进行研究,目标网络见图\ref{fig3:四个节点}~\cite{2020Finding}。
% 图\ref{fig3:四个节点}最左侧图代表整体架构,组件$N_{start}$代表攻击者的计算机,其余组件代表防御者的基础设施,其中$N_{data}$是攻击者想要破坏的组件。使用图的形式对目标网络进行形式化描述见图\ref{fig3:四个节点}中间图,图中的节点表示基础设施的组件。同样,图的边表示组件之间的连接。该图有两个特殊节点$N_{start}$和$N_{data}$,分别表示攻击者的起始位置和目标。每个节点$k$的属性$S_k$都有一个攻击属性值与防御属性值,即$S_k = {S^A_k, S^D_k}$,攻击属性值${S^A_k=\{S^A_{k,1},S^A_{k,2},\cdots,S^A_{k,m}}\}$仅对攻击者可见,表示$m$种类型攻击的攻击强度,防御属性值$S^D_{k} =\{S^D_{k,1}, S^D_{k,2},\cdots, S^D_{k, m+1}\} $仅对防御者可见,前$m$属性表示$m$种类型攻击的防御强度,第$m+1$个属性代表节点检测攻击的能力。

% 由目标网络构建可知,该研究采用了属性值的方式,属于非真实属性模拟,且目标网络场景不存在主动变化。

% 攻击者可以采取两类动作:对某一节点$k$执行侦察动作,使得防御状态$S^D_k$对攻击者可见;对某一节点$k$执行类型为$j \in \{1,2,\cdots, m\}$的攻击动作,使得攻击状态$S^A_{k,j}$的值加1。防御者可以对节点$k$可以采取两类动作:监控操作,提高节点的检测能力并增加检测$S^D _{k;m+1}$,防御操作,提高对类型$j \in \{1,2,\cdots, m\}$攻击的防御能力并增加防御值$S^D_{k,j}$。攻击者和防御者交替执行动作,当节点$k$的前$m$个属性中某一种攻击类型$j$的攻击值超过防御值,即$S^A_{k,j} \textgreater S^D_{k,j}$,则攻击者已危及该节点,并且该节点的邻居对攻击者可见。相反,如果攻击者执行的攻击未危及节点,则防御者检测到该攻击的概率为$p = \frac{S^D_{k;m+1}}{w+1} $,由节点的检测能力$S^D _{k;m+1}$定义。

% 由攻击者、防御者建模可知,攻击者和防御者的动作执行没有约束关系,可以独立执行,属于孤立技战术;防御者的动作会改变目标网络的防御属性,并增强其监控能力,引起了目标网络的被动变化,且由于目标网络不存在主动变化,故属于半静态场景。攻防交互博弈的目的是为了获得策略的最优解,没有涉及真实技战术的执行,也没有集成真实的攻击载荷,故属于策略自动化。

% 综上所述,Hammer等人~\cite{2020Finding}构建的数值模拟网络场景建模方法属于:策略自动化+非真实属性模拟+孤立技战术+半静态场景。

\subsubsection{Numerical Simulation Networks}
Numerical simulation networks encompass policy automation, simulation of hypothetical attributes, execution of isolated technical and tactical actions, and semi-dynamic scenarios. 

Hammer et al. investigated attack–defense interactions in penetration testing using a four-node numerical simulation network~\cite{2020Finding}. Figure~\ref{fig3:four nodes} presents the network architecture, its graphical representation and attribute model.
In the left diagram, $N_{start}$ represents the attacker's computer, while the other nodes correspond to defender components. The attacker’s objective is to compromise $N_{data}$. The middle diagram formalizes the network as a graph, with nodes representing components and edges indicating connections. 
Each node $k$ is characterized by attributes $S_k = \{S^A_k, S^D_k\}$, which include both attack and defense values. The attack attributes $S^A_k = \{S^A_{k,1}, S^A_{k,2}, \ldots, S^A_{k,m}\}$ represent the strength of $m$ attack types and are visible only to the attacker. The defense attributes $S^D_k = \{S^D_{k,1}, S^D_{k,2}, \ldots, S^D_{k,m+1}\}$ are visible only to the defender, where the first $m$ attributes correspond to the respective attack types and the $(m+1)$-th attribute indicates detection capability. This study simulates hypothetical attributes, and the target network remains static.

The attacker can perform two actions on a node $k$: (1) reconnaissance to reveal the defense state $S^D_k$, and (2) execute an attack of type $j \in \{1,2,\cdots, m\}$, increasing the attack state $S^A_{k,j}$ by one. The defender can take two actions on node $k$: (1) monitoring operations to enhance the node's detection ability $S^D_{k;m+1}$, and (2) defensive operations to strengthen defenses against attack type $j \in \{1,2,\cdots, m\}$, thereby increasing $S^D_{k,j}$. The attacker and defender alternate actions. If $S^A_{k,j} > S^D_{k,j}$ for any attack type $j$, the attacker compromises node $k$, making its neighbors visible. If the attack does not compromise the node, the defender detects it with probability $p = \frac{S^D_{k;m+1}}{w+1}$, based on the node's detection ability $S^D_{k;m+1}$.

From the attacker and defender models, it is evident that both can execute actions without constraints, allowing them to act independently. This categorizes their actions as Isolated Technical and Tactical Actions. The defender’s actions modify the network’s defense attributes and enhance its monitoring capabilities, resulting in passive changes to the network. Since there are no active alterations, the scenario is classified as Semi-Dynamic. The game aims to determine optimal strategies without involving real-world tactical executions or attack payloads, classifying it as Policy Automation.

In summary, Hammer et al.’s approach is characterized by Policy Automation, Simulation of Hypothetical Attributes, Isolated Technical and Tactical Actions, and a Semi-Dynamic Scenario.


\begin{figure}[tb]
    \centering
    \includegraphics[width=85mm]{figure/four_nodes.pdf}
    \caption{A numerical simulation network with four nodes~\cite{2020Finding}. The left, middle, right diagram shows the network model, graph model and attribute model, respectively.}
    \label{fig3:four nodes}
\end{figure}

% \subsubsection{cyberbattlesim:策略自动化+真实属性模拟+连续技战术+半静态场景}
% cyberbattlesim\cite{Cyberbattlesim}是微软于2021年开源的一个研究项目,使用计算机网络和网络安全概念的高级抽象来研究自主代理如何在模拟企业环境中运行,大量研究使用CyberBattleSim构建目标网络进行自动化渗透测试研究,如\cite{li2024knowledge},\cite{terranova2024leveraging},\cite{zhang2022improved},\cite{guo2023automated}等。

% CyberBattleSim专注于对网络攻击的入侵后横向移动阶段进行威胁建模。目标网络环境由计算机节点网络组成,包括固定的网络拓扑和一组预定义的漏洞参数化,智能体可以利用这些漏洞在网络中横向移动。一个目标网络的场景见图\ref{CyberBattleSim网络示意图}\cite{Cyberbattlesim}。计算机运行着各种操作系统和软件。网络拓扑是固定的,每台计算机都有一组属性、一个值和预先分配的漏洞。黑色边缘表示在节点之间运行的流量,并由通信协议标记。

% 由目标网络构建可知,该研究采用了来源于现实的真实属性模拟,且目标网络场景不存在主动变化。
\subsubsection{CyberBattleSim}
CyberBattleSim is an open-source research project initiated by Microsoft in 2021 that uses high-level abstractions of computer networks and cybersecurity concepts to study how autonomous agents operate within simulated corporate environments~\cite{Cyberbattlesim}. Numerous studies have used it for AutoPT research~\cite{li2024knowledge, terranova2024leveraging, zhang2022improved, guo2023automated}. With our classification system, CyberBattleSim is categorized under policy automation, simulation of authentic attributes, coordinated technical and tactical actions, and semi-dynamic scenarios. 

CyberBattleSim focuses on threat modeling during the post-compromise lateral movement phase of network attacks. It simulates a fixed network topology with parameterized vulnerabilities, allowing attackers to exploit these weaknesses for lateral movement. A target network scenario, illustrated in Figure~\ref{CyberBattleSim}, consists of nodes running various operating systems and software. Each computer has specific attributes, values, and pre-assigned vulnerabilities. Communication between nodes is depicted by black edges labeled with communication protocols. The target network is constructed using realistic attribute simulations, and the scenario remains static without active changes.

% As evident from the construction of the target network, the study employs real attribute simulation derived from reality, and the target network scenario does not exhibit active changes.



\begin{figure}[tb]
    \centering
    \includegraphics[width=85mm]{figure/Cyberbattlesim.pdf}
    \caption{Schematic Diagram of CyberBattleSim Network~\cite{Cyberbattlesim}}
    \label{CyberBattleSim}
\end{figure}
% 攻击者的目标是通过利用这些植入的漏洞来获得网络的某些部分的所有权,并最大化累积奖励。攻击者会采取措施从其当前拥有的节点逐渐探索网络。有三种操作,为代理提供利用和探索功能的组合:执行本地攻击、执行远程攻击和连接到其他节点。操作由应执行基础操作的源节点参数化,并且仅允许在攻击者拥有的节点上执行。每一个漏洞的成功利用定义了具体的前置条件,如漏洞处于活动状态的节点、成功利用的可能性以及结果和副作用。在图~\ref{CyberBattleSim网络示意图}的示例中,攻击者从模拟的 Windows 7 节点(左侧,由橙色箭头指向)破坏网络。它通过利用 SMB 文件共享协议中的漏洞继续横向移动到 Windows 8 节点,然后使用一些缓存的凭据登录到另一台 Windows 7 计算机。然后,它利用 IIS 远程漏洞来拥有 IIS 服务器,最后使用泄露的连接字符串访问 SQL 数据库。
% 当攻击者在网络中移动时,防御者会监视网络活动以检测攻击者的存在并遏制攻击,或通过重新映像受感染的节点来实现缓解攻击,检测或缓解攻击的成功与否基于预定义的动作成功概率值进行。

% 由攻击者、防御者建模可知,攻击者的连接到其他节点动作只有在已经通过其他手段获取到发起节点与目标节点权限时才会成功,动作之间是具有前后执行限制关系,属于连续技战术;攻击者的动作会改变目标网络中节点的通信关系和整体网络架构,防御者的重映像动作会修补节点的漏洞,改变节点的内部属性,因此目标网络场景由于攻击者、防御者的动作发生了被动变化,属于半静态场景。CyberBattleSim 提供了一种构建计算机系统复杂性的高度抽象模拟的方法,不支持实真实攻击代码执行,其目的在于研究网络智能体如何在模拟环境中交互和进化,属于策略自动化。

% 综上所述,CyberBattleSim建模方法属于:策略自动化+真实属性模拟+连续技战术+半静态场景。值得注意的是,CyberBattleSim虽然为大量强化学习算法研究提供了目标环境,但网络环境的规模一般局限于中小网络(10-20个节点),且没有提供完全动态场景的建模方法。
% The attacker's objective is to gain network control by exploiting vulnerabilities and to maximize rewards.There are three operations: performing local attacks, executing remote attacks, and connecting to other nodes. Successful exploitation of each vulnerability defines specific prerequisites, such as the node with the vulnerability being active, the predefined probabilities of successful exploitation, and the outcomes and side effects. In the example in Figure~\ref{CyberBattleSim}, the attacker progresses from a Windows 7 node, exploiting vulnerabilities and using some cached credentials to move laterally and ultimately access an SQL database. Defenders monitor activity to detect and mitigate attacks by re-imaging infected nodes. Success is also based on predefined probabilities.

The attacker aims to gain network control by exploiting vulnerabilities and maximizing rewards through three actions: performing a local attack, performing a remote attack, and connecting to other nodes. Actions are parameterized by the source node where the underlying operation should take place, and they are only permitted on nodes owned by the agent. As illustrated in Figure~\ref{CyberBattleSim}, the attacker starts from a Windows 7 node, exploits vulnerabilities, and uses cached credentials to move laterally, ultimately accessing an SQL database. Defenders monitor activities to detect and mitigate attacks by reimaging infected nodes. Attack success also depends on predefined probabilities.

The attacker modifies network communication and architecture, while defenders reimage systems, patch vulnerabilities, and alter node attributes. Consequently, the target network undergoes passive changes from both attacker and defender actions without active alterations, classifying it as a semi-dynamic scenario. CyberBattleSim conducts abstract simulations without executing real attack code, emphasizing agent interactions. It is designed for small to medium networks (10–20 nodes) and does not support fully dynamic scenarios.

% CyberBattleSim supports abstract simulations without executing real attack code, focusing on agent interactions. It is typically used for small or medium networks (10-20 nodes) and does not support completely dynamic scenarios.


% \subsubsection{Network Attack Simulator:策略自动化+真实属性模拟+孤立技战术+完全静态场景}
% Network Attack Simulator\cite{schwartz2019autonomous}是哈佛Schwartz等人在2019年开源的一个轻量级的开源模拟器,Schwartz等人认为i他们的工作是第一个将强化学习应用于自动化渗透测试的研究的,谷歌学术上有超过130的引用量,是一个具有一定影响力的工作。

% Network Attack Simulator构建的网络环境由多个子网构成,子网之间由防火墙控制准入关系,子网内部具有多个机器组成,每个机器运行着数量不定的服务,网络架构见图~\ref{Network Attack Simulator网络架构图}\cite{schwartz2019autonomous}。每一个节点的属性包括地址(subnet\_ID, machine\_ID)、机器价值、参数((开放服务、成功率、利用成本))构成。

% 由网络架构和目标资产建模可知,该研究采用了来源于现实的真实属性模拟,且目标网络场景不存在主动变化。

\subsubsection{Network Attack Simulator}

% Network Attack Simulator~\cite{schwartz2019autonomous} embodies Policy Automation + Simulation of Authentic Attributes + Isolated Technical and Tactical Actions + Completely Static Scenario. Developed by Schwartz et al. from Harvard in 2019, this lightweight, open-source simulator is pioneering in applying reinforcement learning to AutoPT research, highlighting its significant influence.

The Network Attack Simulator~\cite{schwartz2019autonomous}, a lightweight, open-source tool developed by Schwartz et al. in 2019, is a groundbreaking application of reinforcement learning in AutoPT research.
The simulator constructs a network environment of multiple subnets with firewall-controlled access, each containing machines running various services. As shown in Figure~\ref{Network Attack Simulator}~\cite{schwartz2019autonomous}, the architecture includes node attributes such as address (subnet\_ID, machine\_ID), machine value, and parameters (open services, success rate, exploitation cost).
The network architecture and asset modeling utilize real-world data for the simulation of authentic attributes, remaining static throughout the process.

\begin{figure}[tb]
\centering
\includegraphics[width=85mm]{figure/Network_Attack_Simulator.pdf}
\caption{Network Architecture Diagram of Network Attack Simulator~\cite{schwartz2019autonomous}}
\label{Network Attack Simulator}
\end{figure}

% Schwartz等人只定义了攻击者建模,攻击者的动作包括针对网络上每个服务和每台机器的单个扫描动作和漏洞利用。扫描动作返回有关给定机器的每个端口上运行的服务信息。对于网络上的每种可能服务,都有一个匹配的漏洞利用动作。成功的漏洞利用操作将导致目标机器受到攻击。任何漏洞利用的成功都取决目标网络中该节点本身的配置。从攻击者建模可知,攻击者的两个动作之间没有逻辑上的先后关系,不是只有一个动作执行后另一个动作才能够执行(当然,扫描动作执行后获取的信息可以帮助进行漏洞的选择,但这不是必须的,遍历执行每一个漏洞也会获取目标机器的权限),因此属于孤立技战术。攻击者动作的执行也不会引起目标网络的被动变化。
% Network Attack Simulator主要关注使用强化学习 (RL) 算法来自动生成渗透测试的攻击策略,属于策略自动化。

% 综上所述,Network Attack Simulator建模方法属于:策略自动化+真实属性模拟+独立技战术+完全静态场景。值得注意的是,Network Attack Simulator建模没有显式的建模防御者,而是通过子网、机器之间的连接关系、漏洞利用的成功率等隐式地建模防御者,该生成器适用于较小的网络和操作数量,没有扩展到大型网络,且攻击者的动作选择较为简单,仅包括扫描与漏洞利用两类动作,没有定义防御者,也没有定义具有主动变化的目标网络环境。
% Schwartz et al. focus solely on attacker modeling,  where the attacker's actions include individual scanning actions and vulnerability exploitation targeting each service and machine on the network.

% Scanning identifies services on each machine’s ports, and each networked service has a corresponding exploitation action. A successful exploitation compromises the target machine, contingent on the node's configuration within the network. The attacker's actions are isolated; one action does not need to be executed before another can be performed. Information obtained from scanning can assist selecting vulnerabilities, but this is not necessary, as traversing execution vulnerabilities will also gain the access permission to the target machine. These actions do not alter the target network passively. Hence, the target network scenario is classified as Completely Static Scenario.

% Network Attack Simulator focuses on leveraging reinforcement learning algorithms for automating attack strategy generation in penetration testing, under Policy Automation.
% % Please add the following required packages to your document preamble:
% % \usepackage{multirow}
% \begin{table*}[h!]
% \centering
% \caption{Classification Table of Simulation Modeling Methods in Typical Literature on Automated Penetration Testing}
% \label{Classification_Table}
% \resizebox{0.92\linewidth}{!}{
% \begin{tabular}{c|c|ccc|cc|cc|ccc}
% \toprule[2pt]
% \multirow{2}{*}{Year} & \multirow{2}{*}{Paper} & \multicolumn{3}{c|}{Automation Level}         & \multicolumn{2}{c|}{Network Simulation Complexity}         & \multicolumn{2}{c|}{Dependency of T\&T Operations}         & \multicolumn{3}{c}{Environment Update}                \\ \cline{3-12}
% & & T\&T & Policy & Complete & Hypothetical & Authentic & Isolated & Coordinated & Static & Semi-Dynamic & Dynamic \\ 
% % \multirow{2}{*}{Year} & \multirow{2}{*}{Paper} & \multicolumn{3}{c|}{RO}         & \multicolumn{2}{c|}{NSC}         & \multicolumn{2}{c|}{PP-TT}         & \multicolumn{3}{c}{S-FV}                \\ \cline{3-12}
% % & & TT-Auto & SG-Auto & FP-Auto & Non-RAS & RAS & Iso-TTA & Cont-TTA & CS-S & SS-S & CD-S \\ 
% \midrule[1.5pt]
% 1997 &\citet{haeni1997firewall}         & \checkmark &   &   &   & \checkmark & \checkmark &   & \checkmark                &                  &                  \\
% \hline
% \multirow{2}{*}{1998} &\cite{TenableNessus}             & \checkmark &   &   &   & \checkmark & \checkmark &   & — & — & — \\
%  &\cite{WiresharkTCPAnalysis2024}  & \checkmark &   &   &   & \checkmark & \checkmark &   & — & — & — \\
%  \hline
% 2001 &\cite{mcdermott2001attack}       &   & \checkmark &   & \checkmark &   &   & \checkmark & \checkmark                &                  &                  \\
%  \hline
% 2002 &\cite{skaggs2002network}         & \checkmark &   &   &   & \checkmark & \checkmark &   & \checkmark                &                  &                  \\ \hline
% 2003 &\cite{MetasploitWebsite}         & \checkmark &   &   &   & \checkmark & \checkmark & & — & — & — \\ \hline
% 2005 &\cite{liu2005game}               &   & \checkmark &   &   & \checkmark & \checkmark &   & \checkmark                &                  &                  \\ \hline
% \multirow{4}{*}{2007}  &\cite{kosuga2007sania}           & \checkmark &   &   &   & \checkmark & \checkmark &   & \checkmark                &                  &                  \\
%  &\cite{fonseca2007testing}        & \checkmark &   &   &   & \checkmark & \checkmark &   & \checkmark                &                  &                  \\
%  &\cite{shen2007strategies}        &   & \checkmark &   &   & \checkmark &   & \checkmark &                  & \checkmark                &                  \\
%  &\cite{cone2007video}             &   & \checkmark &   & \checkmark &   & \checkmark &   & \checkmark                &                  &                  \\ \hline
% \multirow{2}{*}{2009}   &\cite{lyon2009nmap}              & \checkmark &   &   &   & \checkmark & \checkmark &   & — & — & — \\
%  &\cite{greenwald2009automated}    &   & \checkmark &   &   & \checkmark & \checkmark &   & \checkmark                &                  &                  \\ \hline
% 2011 &\cite{sarraute2011algorithm}     &   & \checkmark &   &   & \checkmark &   & \checkmark &                  & \checkmark                &                  \\ \hline
% \multirow{3}{*}{2013} 
%  &\cite{sarraute2013penetration}   &   & \checkmark &   &   & \checkmark & \checkmark &   & \checkmark                &                  &                  \\
%  &\cite{sarraute2013automated}     &   &   & \checkmark &   & \checkmark &   & \checkmark &                  & \checkmark                &                  \\
%  &\cite{van2013flipit}             &   & \checkmark &   & \checkmark &   & \checkmark &   & \checkmark                &                  &                  \\  \hline
% 2014 &\cite{chapman2014playing}        &   & \checkmark &   & \checkmark &   & \checkmark &   & \checkmark                &                  &                  \\ \hline
% \multirow{2}{*}{2016}  &\cite{applebaum2016intelligent}  &   & \checkmark &   & \checkmark &   & \checkmark &   &                  & \checkmark                &                  \\ 
%  &\cite{chapman2016cyber}          &   & \checkmark &   & \checkmark &   & \checkmark &   & \checkmark                &                  &                  \\  \hline
% \multirow{3}{*}{2017}  &\cite{elderman2017adversarial}   &   & \checkmark &   & \checkmark &   & \checkmark &   &                  & \checkmark                &                  \\
%  &\cite{applebaum2017analysis}     &   & \checkmark &   &   & \checkmark &   & \checkmark &                  &                  & \checkmark                \\
%  &\cite{ficco2017simulation}       & \checkmark &   &   &   & \checkmark & \checkmark &   & \checkmark                &                  &                  \\ \hline
% \multirow{4}{*}{2018}   &\cite{miller2018automated}       &   & \checkmark &   &   & \checkmark &   & \checkmark & \checkmark                &                  &                  \\
%  &\cite{ghanem2018reinforcement}   &   & \checkmark &   &   & \checkmark &   & \checkmark & \checkmark                &                  &                  \\
%  &\cite{casola2018towards}         & \checkmark &   &   &   & \checkmark &   & \checkmark & \checkmark                &                  &                  \\
%  &\cite{ghanem2019reinforcement}   &   & \checkmark &   &   & \checkmark &   & \checkmark &                  & \checkmark                &                  \\ \hline
% \multirow{6}{*}{2019} &\cite{paul2019learning}          &   & \checkmark &   &   & \checkmark &   & \checkmark &                  & \checkmark                &                  \\
%  &\cite{schwartz2019autonomous}    &   & \checkmark &   &   & \checkmark &   & \checkmark & \checkmark                &                  &                  \\
%  &\cite{paul2019strategic}         &   & \checkmark &   & \checkmark &   & \checkmark &   &                  & \checkmark                &                  \\
%  &\cite{ghanem2019reinforcement}   &   & \checkmark & \checkmark &   & \checkmark &   & \checkmark &                  & \checkmark                &                  \\
%  &\cite{2019NIG}                   &   & \checkmark &   &   & \checkmark &   & \checkmark & \checkmark                &                  &                  \\
%  &\cite{yichao2019improved}        &   & \checkmark &   &   & \checkmark &   & \checkmark &                  & \checkmark                &                  \\ \hline
% \multirow{12}{*}{2020}  &\cite{hu2020automated}           &   & \checkmark &   &   & \checkmark &   & \checkmark & \checkmark                &                  &                  \\
%  &\cite{2020Finding}         &   & \checkmark &   & \checkmark &   & \checkmark &   &                  & \checkmark                &                  \\
%  &\cite{valea2020towards}          & \checkmark &   &   &   & \checkmark & \checkmark &   & \checkmark                &                  &                  \\
%  &\cite{bhattacharya2020automated} &   & \checkmark &   &   & \checkmark &   & \checkmark &                  & \checkmark                &                  \\
%  &\cite{nguyen2020multiple}        &   & \checkmark &   &   & \checkmark & \checkmark &   & \checkmark                &                  &                  \\
%  &\cite{costa2020charles}          & \checkmark &   &   &   & \checkmark & \checkmark &   & \checkmark                &                  &                  \\
%  &\cite{chowdhary2020autonomous}   &   & \checkmark &   &   & \checkmark &   & \checkmark & \checkmark                &                  &                  \\
%  &\cite{bland2020machine}          &   & \checkmark &   &   & \checkmark &   & \checkmark &                  & \checkmark                &                  \\
%  &\cite{hu2020apu}                 &   & \checkmark &   &   & \checkmark & \checkmark &   & \checkmark                &                  &                  \\
%  &\cite{enoch2020harmer}           &   & \checkmark &   &   & \checkmark &   & \checkmark & \checkmark                &                  &                  \\
%  &\cite{schwartz2020pomdp+}        &   & \checkmark &   &   & \checkmark & \checkmark &   &                  & \checkmark                &                  \\ \hline
%  \multirow{7}{*}{2021}  
%  &\cite{dorchuck2021goal}          &   &   & \checkmark &   & \checkmark &   & \checkmark & \checkmark                &                  &                  \\
%  &\cite{qian2021ontology}          &   & \checkmark &   &   & \checkmark &   & \checkmark & \checkmark                &                  &                  \\
%  &\cite{filiol2021method}          & \checkmark &   &   &   & \checkmark &   & \checkmark & \checkmark                &                  &                  \\
%  &\cite{zhou2021autonomous}        &   & \checkmark &   &   & \checkmark &   & \checkmark & \checkmark                &                  &                  \\
%  &\cite{hacks2021towards}          &   & \checkmark &   &   & \checkmark &   & \checkmark &                  & \checkmark                &                  \\
%  &\cite{erdHodi2021simulating}     & \checkmark &   &   &   & \checkmark &   & \checkmark & \checkmark                &                  &                  \\
%  &\cite{ji2021optimal}             &   & \checkmark &   &   & \checkmark & \checkmark &   &                  & \checkmark                &                  \\ \hline
% \multirow{5}{*}{2022}  
%  &\cite{dillon2022perihack}        &   & \checkmark &   & \checkmark &   &   & \checkmark &                  & \checkmark                &                  \\
%  &\cite{yamin2022use}              &   &   & \checkmark &   & \checkmark &   & \checkmark &                  & \checkmark                &                  \\
%  &\cite{confido2022reinforcing}    &   & \checkmark &   &   & \checkmark &   & \checkmark &                  & \checkmark                &                  \\
%  &\cite{tran2022cascaded}          &   & \checkmark &   &   & \checkmark &   & \checkmark &                  & \checkmark                &                  \\
%  &\cite{hance2022distributed}      & \checkmark &   &   &   & \checkmark &   & \checkmark & \checkmark                &                  &                  \\ \hline
% \multirow{2}{*}{2023}   &\cite{faeroy2023automatic}       & \checkmark &   &   &   & \checkmark & \checkmark &   & \checkmark                &                  &                  \\
%  &\cite{li2023innes}                    &   & \checkmark &   &   & \checkmark &   & \checkmark & \checkmark                &                  &                  \\ \hline
% \multirow{8}{*}{2024}   &\cite{xu2024autoattacker}        &   &   & \checkmark &   & \checkmark &   & \checkmark & \checkmark                &                  &                  \\
%  &\cite{becker2024evaluation}      &   & \checkmark &   &   & \checkmark &   & \checkmark & \checkmark                &                  &                  \\
%  &\cite{li2024knowledge}           &   & \checkmark &   &   & \checkmark &   & \checkmark &                  & \checkmark                &                  \\
%  &\cite{alshehri2024breachseek}    & \checkmark &   &   &   & \checkmark &   & \checkmark & \checkmark                &                  &                  \\
%  &\cite{deng2024pentestgpt}        &   & \checkmark &   &   & \checkmark &   & \checkmark & \checkmark                &                  &                  \\
%  &\cite{Zhenduo}                   &   & \checkmark &   &   & \checkmark &   & \checkmark & \checkmark                &                  &                 \\
%  &\cite{li2024dynpen} &   & \checkmark &   &   & \checkmark &  \checkmark &  &                 &                  &  \checkmark               \\
%  &\cite{wang2024pentraformer} &   & \checkmark &   &   & \checkmark &  \checkmark &  & \checkmark                &                  &                 \\

%     \bottomrule[2pt]
%     \end{tabular}
% }
% \end{table*}

% Please add the following required packages to your document preamble:
% \usepackage{multirow}
% \begin{table*}[h!]
% \centering
% \caption{Classification Table of Simulation Modeling Methods in Typical Literature on Automated Penetration Testing}
% \label{Classification_Table}
% \resizebox{0.999\linewidth}{!}{
% \begin{tabular}{c|c|ccc|cc|cc|ccc}
% \toprule[2pt]
% \multirow{2}{*}{Year} & \multirow{2}{*}{Paper} & \multicolumn{3}{c|}{Automation Level}         & \multicolumn{2}{c|}{Network Simulation Complexity}         & \multicolumn{2}{c|}{Dependency of T\&T Operations}         & \multicolumn{3}{c}{Scenario Feedback and Variation}                \\ \cline{3-12}
% & & Technical & Policy & Complete & Hypothetical & Authentic & Isolated & Coordinated & Static & Semi-Dynamic & Dynamic \\ 
% % \multirow{2}{*}{Year} & \multirow{2}{*}{Paper} & \multicolumn{3}{c|}{RO}         & \multicolumn{2}{c|}{NSC}         & \multicolumn{2}{c|}{PP-TT}         & \multicolumn{3}{c}{S-FV}                \\ \cline{3-12}
% % & & TT-Auto & SG-Auto & FP-Auto & Non-RAS & RAS & Iso-TTA & Cont-TTA & CS-S & SS-S & CD-S \\ 
% \midrule[1.5pt]
% \multirow{2}{*}{1997} & Haeni et al.~\citep{haeni1997firewall}     & \checkmark &   &   &   & \checkmark & \checkmark &    & — & — & —                  \\
%  &\citet{nmap}  & \checkmark &   &   &   & \checkmark & \checkmark &   & — & — & — \\
%  \hline
% 1998 &\citet{Nessus}             & \checkmark &   &   &   & \checkmark & \checkmark &   & — & — & — \\
%  \hline
% 2001 &\citet{McDermott et al.}       &   & \checkmark &   & \checkmark &   &   & \checkmark & \checkmark                &                  &                  \\
%  \hline
% 2002 &\citet{Skaggs et al.}         & \checkmark &   &   &   & \checkmark & \checkmark &   & \checkmark                &                  &                  \\ \hline
% 2003 &\citet{Metasploit}         & \checkmark &   &   &   & \checkmark & \checkmark & & — & — & — \\ \hline
% 2005 &\citet{Liu et al.}               &   & \checkmark &   &   & \checkmark & \checkmark &   & \checkmark                &                  &                  \\ \hline
% \multirow{4}{*}{2007}  &\citet{Kosuga et al.}           & \checkmark &   &   &   & \checkmark & \checkmark &   & \checkmark                &                  &                  \\
%  &\citet{Fonseca et al.}        & \checkmark &   &   &   & \checkmark & \checkmark &   & \checkmark                &                  &                  \\
%  &\citet{Shen et al.}        &   & \checkmark &   &   & \checkmark &   & \checkmark &                  & \checkmark                &                  \\
%  &\citet{Cone et al.}             &   & \checkmark &   & \checkmark &   & \checkmark &   & \checkmark                &                  &                  \\ \hline
% \multirow{2}{*}{2009}   &\citet{Lyon et al.}              & \checkmark &   &   &   & \checkmark & \checkmark &   & — & — & — \\
%  &\citet{Greenwald et al.}    &   & \checkmark &   &   & \checkmark & \checkmark &   & \checkmark                &                  &                  \\ \hline
% 2011 &\citet{Sarraute et al.}     &   & \checkmark &   &   & \checkmark &   & \checkmark &                  & \checkmark                &                  \\ \hline
% \multirow{3}{*}{2013} 
%  &\citet{Sarraute et al.}   &   & \checkmark &   &   & \checkmark & \checkmark &   & \checkmark                &                  &                  \\
%  &\citet{Sarraute et al.}     &   &   & \checkmark &   & \checkmark &   & \checkmark &                  & \checkmark                &                  \\
%  &\citet{Van Dijk et al.}             &   & \checkmark &   & \checkmark &   & \checkmark &   & \checkmark                &                  &                  \\  \hline
% 2014 &\citet{Chapman et al.}        &   & \checkmark &   & \checkmark &   & \checkmark &   & \checkmark                &                  &                  \\ \hline
% \multirow{2}{*}{2016}  &\citet{Applebaum et al.}  &   & \checkmark &   & \checkmark &   & \checkmark &   &                  & \checkmark                &                  \\ 
%  &\citet{Chapman et al.}          &   & \checkmark &   & \checkmark &   & \checkmark &   & \checkmark                &                  &                  \\  \hline
% \multirow{3}{*}{2017}  &\citet{Elderman et al.}   &   & \checkmark &   & \checkmark &   & \checkmark &   &                  & \checkmark                &                  \\
%  &\citet{Applebaum et al.}     &   & \checkmark &   &   & \checkmark &   & \checkmark &                  &                  & \checkmark                \\
%  &\citet{Ficco et al.}       & \checkmark &   &   &   & \checkmark & \checkmark &   & \checkmark                &                  &                  \\ \hline
% \multirow{4}{*}{2018}   &\citet{Miller et al.}       &   & \checkmark &   &   & \checkmark &   & \checkmark & \checkmark                &                  &                  \\
%  &\citet{Ghanem et al.}   &   & \checkmark &   &   & \checkmark &   & \checkmark & \checkmark                &                  &                  \\
%  &\citet{Casola et al.}         & \checkmark &   &   &   & \checkmark &   & \checkmark & \checkmark                &                  &                  \\
%  &\citet{Ghanem et al.}   &   &  & \checkmark  &   & \checkmark &   & \checkmark &                  & \checkmark                &                  \\ \hline
% \multirow{6}{*}{2019} &\citet{Paul et al.}          &   & \checkmark &   &   & \checkmark &   & \checkmark &                  & \checkmark                &                  \\
%  &\citet{Schwartz et al.}    &   & \checkmark &   &   & \checkmark &   & \checkmark & \checkmark                &                  &                  \\
%  &\citet{Paul et al.}         &   & \checkmark &   & \checkmark &   & \checkmark &   &                  & \checkmark                &                  \\
%  &\citet{Zhou et al.}                   &   & \checkmark &   &   & \checkmark &   & \checkmark & \checkmark                &                  &                  \\
%  &\citet{Zang et al.}        &   & \checkmark &   &   & \checkmark &   & \checkmark &                  & \checkmark                &                  \\ \hline
% \multirow{12}{*}{2020}  &\citet{Hu et al.}           &   & \checkmark &   &   & \checkmark &   & \checkmark & \checkmark                &                  &                  \\
%  &\citet{Hammar et al.}         &   & \checkmark &   & \checkmark &   & \checkmark &   &                  & \checkmark                &                  \\
%  &\citet{Valea et al.}          & \checkmark &   &   &   & \checkmark & \checkmark &   & \checkmark                &                  &                  \\
%  &\citet{Bhattacharya et al.} &   & \checkmark &   &   & \checkmark &   & \checkmark &                  & \checkmark                &                  \\
%  &\citet{Nguyen et al.}        &   & \checkmark &   &   & \checkmark & \checkmark &   & \checkmark                &                  &                  \\
%  &\citet{Costa et al.}          & \checkmark &   &   &   & \checkmark & \checkmark &   & \checkmark                &                  &                  \\
%  &\citet{Chowdhary et al.}   &   & \checkmark &   &   & \checkmark &   & \checkmark & \checkmark                &                  &                  \\
%  &\citet{Bland et al.}          &   & \checkmark &   &   & \checkmark &   & \checkmark &                  & \checkmark                &                  \\
%  &\citet{Hu et al.}                 &   & \checkmark &   &   & \checkmark & \checkmark &   & \checkmark                &                  &                  \\
%  &\citet{Enoch et al.}           &   & \checkmark &   &   & \checkmark &   & \checkmark & \checkmark                &                  &                  \\
%  &\citet{Schwartz et al.}        &   & \checkmark &   &   & \checkmark & \checkmark &   &                  & \checkmark                &                  \\ \hline
%  \multirow{7}{*}{2021}  
%  &\citet{Dorchuck et al.}          &   &   & \checkmark &   & \checkmark &   & \checkmark & \checkmark                &                  &                  \\
%  &\citet{Qian et al.}          &   & \checkmark &   &   & \checkmark &   & \checkmark & \checkmark                &                  &                  \\
%  &\citet{Filiol et al.}          & \checkmark &   &   &   & \checkmark &   & \checkmark & \checkmark                &                  &                  \\
%  &\citet{Zhou et al.}        &   & \checkmark &   &   & \checkmark &   & \checkmark & \checkmark                &                  &                  \\
%  &\citet{Hacks et al.}          &   & \checkmark &   &   & \checkmark &   & \checkmark &                  & \checkmark                &                  \\
%  &\citet{Erd{\H{o}}di et al.}     & \checkmark &   &   &   & \checkmark &   & \checkmark & \checkmark                &                  &                  \\
%  &\citet{Ji et al.}             &   & \checkmark &   &   & \checkmark & \checkmark &   &                  & \checkmark                &                  \\ \hline
% \multirow{5}{*}{2022}  
%  &\citet{Dillon et al.}        &   & \checkmark &   & \checkmark &   &   & \checkmark &                  & \checkmark                &                  \\
%  &\citet{Yamin et al.}              &   &   & \checkmark &   & \checkmark &   & \checkmark &                  & \checkmark                &                  \\
%  &\citet{Confido et al.}    &   & \checkmark &   &   & \checkmark &   & \checkmark &                  & \checkmark                &                  \\
%  &\citet{Tran et al.}          &   & \checkmark &   &   & \checkmark &   & \checkmark &                  & \checkmark                &                  \\
%  &\citet{Hance et al.}      & \checkmark &   &   &   & \checkmark &   & \checkmark & \checkmark                &                  &                  \\ \hline
% \multirow{2}{*}{2023}   &\citet{F{\ae}r{\o}y et al.}       & \checkmark &   &   &   & \checkmark & \checkmark &   & \checkmark                &                  &                  \\
%  &\citet{Li et al.}                    &   & \checkmark &   &   & \checkmark &   & \checkmark & \checkmark                &                  &                  \\ \hline
% \multirow{8}{*}{2024}   &\citet{Xu et al.}        &   &   & \checkmark &   & \checkmark &   & \checkmark & \checkmark                &                  &                  \\
%  &\citet{Becker et al.}      &   & \checkmark &   &   & \checkmark &   & \checkmark & \checkmark                &                  &                  \\
%  &\citet{Li et al.}           &   & \checkmark &   &   & \checkmark &   & \checkmark &                  & \checkmark                &                  \\
%  &\citet{Alshehri et al.}    & \checkmark &   &   &   & \checkmark &   & \checkmark & \checkmark                &                  &                  \\
%  &\citet{Deng et al.}        &   & \checkmark &   &   & \checkmark &   & \checkmark & \checkmark                &                  &                  \\
%  &\citet{Wang et al.}                   &   & \checkmark &   &   & \checkmark &   & \checkmark & \checkmark                &                  &                 \\
%  &\citet{Li et al.} &   & \checkmark &   &   & \checkmark &  \checkmark &  &                 &                  &  \checkmark               \\
%  &\citet{Wang et al.} &   & \checkmark &   &   & \checkmark &  \checkmark &  & \checkmark                &                  &                 \\

%     \bottomrule[2pt]
%     \end{tabular}
% }
% \end{table*}

% Please add the following required packages to your document preamble:
% \usepackage{multirow}
\begin{table*}[b!]
\centering
\renewcommand\arraystretch{1}
\caption{Classification Table of Simulation Modeling Methods in Typical Literature on Automated Penetration Testing}
\label{Classification_Table}
\resizebox{\linewidth}{!}{
\begin{tabular}{c|c|ccc|cc|cc|ccc}
\toprule[2pt]
\multirow{2}{*}{Year} & \multirow{2}{*}{Paper} & \multicolumn{3}{c|}{Literature Objectives}         & \multicolumn{2}{c|}{Network Simulation Complexity}         & \multicolumn{2}{c|}{Dependency of T\&T Operations}         & \multicolumn{3}{c}{Scenario Feedback and Variation}                \\ \cline{3-12}
& & Technical & Policy & Complete & Hypothetical & Authentic & Isolated & Coordinated & Static & Semi-Dynamic & Dynamic \\ 
% \multirow{2}{*}{Year} & \multirow{2}{*}{Paper} & \multicolumn{3}{c|}{RO}         & \multicolumn{2}{c|}{NSC}         & \multicolumn{2}{c|}{PP-TT}         & \multicolumn{3}{c}{S-FV}                \\ \cline{3-12}
% & & TT-Auto & SG-Auto & FP-Auto & Non-RAS & RAS & Iso-TTA & Cont-TTA & CS-S & SS-S & CD-S \\ 
\midrule[1.5pt]
\multirow{2}{*}{1997} & Haeni et al.~\citep{haeni1997firewall}         & \checkmark &   &   &   & \checkmark & \checkmark &   & — & — & — \\
&Nmap \citep{nmap}  & \checkmark &   &   &   & \checkmark & \checkmark &   & — & — & — \\
\hline



1998 &Nessus~\citep{TenableNessus}             & \checkmark &   &   &   & \checkmark & \checkmark &   & — & — & — \\
 \hline
2001 &McDermott et al.~\citep{mcdermott2001attack}       &   & \checkmark &   & \checkmark &   &   & \checkmark & \checkmark                &                  &                  \\
 \hline
2002 &Skaggs et al.~\citep{skaggs2002network}         & \checkmark &   &   &   & \checkmark & \checkmark &   & \checkmark                &                  &                  \\ \hline
2003 &Metasploit~\citep{MetasploitWebsite}         & \checkmark &   &   &   & \checkmark & \checkmark & & — & — & — \\ \hline
2005 &Liu et al.~\citep{liu2005game}               &   & \checkmark &   &   & \checkmark & \checkmark &   & \checkmark                &                  &                  \\ \hline
\multirow{4}{*}{2007}  &Kosuga et al.~\citep{kosuga2007sania}           & \checkmark &   &   &   & \checkmark & \checkmark &   & \checkmark                &                  &                  \\
 &Fonseca et al.~\citep{fonseca2007testing}        & \checkmark &   &   &   & \checkmark & \checkmark &   & \checkmark                &                  &                  \\
 &Shen et al.~\citep{shen2007strategies}        &   & \checkmark &   &   & \checkmark &   & \checkmark &                  & \checkmark                &                  \\
 &Cone et al.~\citep{cone2007video}             &   & \checkmark &   & \checkmark &   & \checkmark &   & \checkmark                &                  &                  \\ \hline
\multirow{2}{*}{2009}   &Lyon et al.~\citep{lyon2009nmap}              & \checkmark &   &   &   & \checkmark & \checkmark &   & — & — & — \\
 &Greenwald et al.~\citep{greenwald2009automated}    &   & \checkmark &   &   & \checkmark & \checkmark &   & \checkmark                &                  &                  \\ \hline
2011 &Sarraute et al.~\citep{sarraute2011algorithm}     &   & \checkmark &   &   & \checkmark &   & \checkmark &                  & \checkmark                &                  \\ \hline
\multirow{3}{*}{2013} 
 &Sarraute et al.~\citep{sarraute2013penetration}   &   & \checkmark &   &   & \checkmark & \checkmark &   & \checkmark                &                  &                  \\
 &Sarraute et al.~\citep{sarraute2013automated}     &   &   & \checkmark &   & \checkmark &   & \checkmark &                  & \checkmark                &                  \\
 &Van Dijk et al.~\citep{van2013flipit}             &   & \checkmark &   & \checkmark &   & \checkmark &   & \checkmark                &                  &                  \\  \hline
2014 &Chapman et al.~\citep{chapman2014playing}        &   & \checkmark &   & \checkmark &   & \checkmark &   & \checkmark                &                  &                  \\ \hline
\multirow{2}{*}{2016}  &Applebaum et al.~\citep{applebaum2016intelligent}  &   & \checkmark &   & \checkmark &   & \checkmark &   &                  & \checkmark                &                  \\ 
 &Chapman et al.~\citep{chapman2016cyber}          &   & \checkmark &   & \checkmark &   & \checkmark &   & \checkmark                &                  &                  \\  \hline
\multirow{3}{*}{2017}  &Elderman et al.~\citep{elderman2017adversarial}   &   & \checkmark &   & \checkmark &   & \checkmark &   &                  & \checkmark                &                  \\
 &Applebaum et al.~\citep{applebaum2017analysis}     &   & \checkmark &   &   & \checkmark &   & \checkmark &                  &                  & \checkmark                \\
 &Ficco et al.~\citep{ficco2017simulation}       & \checkmark &   &   &   & \checkmark & \checkmark &   & \checkmark                &                  &                  \\ \hline
\multirow{4}{*}{2018}   &Miller et al.~\citep{miller2018automated}       &   & \checkmark &   &   & \checkmark &   & \checkmark & \checkmark                &                  &                  \\
 &Ghanem et al.~\citep{ghanem2018reinforcement}   &   & \checkmark &   &   & \checkmark &   & \checkmark & \checkmark                &                  &                  \\
 &Casola et al.~\citep{casola2018towards}         & \checkmark &   &   &   & \checkmark &   & \checkmark & \checkmark                &                  &                  \\
 &Ghanem et al.~\citep{ghanem2019reinforcement}   &   &  &  \checkmark &   & \checkmark &   & \checkmark &                  & \checkmark                &                  \\ \hline
\multirow{6}{*}{2019} &Paul et al.~\citep{paul2019learning}          &   & \checkmark &   &   & \checkmark &   & \checkmark &                  & \checkmark                &                  \\
 &Schwartz et al.~\citep{schwartz2019autonomous}    &   & \checkmark &   &   & \checkmark &   & \checkmark & \checkmark                &                  &                  \\
 &Paul et al.~\citep{paul2019strategic}         &   & \checkmark &   & \checkmark &   & \checkmark &   &                  & \checkmark                &                  \\
 % &Zhou et al.~\citep{ghanem2019reinforcement}   &   & \checkmark & \checkmark &   & \checkmark &   & \checkmark &                  & \checkmark                &                  \\
 &Zhou et al.~\citep{2019NIG}                   &   & \checkmark &   &   & \checkmark &   & \checkmark & \checkmark                &                  &                  \\
 &Zang et al.~\citep{yichao2019improved}        &   & \checkmark &   &   & \checkmark &   & \checkmark &                  & \checkmark                &                  \\ \hline
\multirow{12}{*}{2020}  &Hu et al.~\citep{hu2020automated}           &   & \checkmark &   &   & \checkmark &   & \checkmark & \checkmark                &                  &                  \\
 &Hammar et al.~\citep{2020Finding}         &   & \checkmark &   & \checkmark &   & \checkmark &   &                  & \checkmark                &                  \\
 &Valea et al.~\citep{valea2020towards}          & \checkmark &   &   &   & \checkmark & \checkmark &   & \checkmark                &                  &                  \\
 &Bhattacharya et al.~\citep{bhattacharya2020automated} &   & \checkmark &   &   & \checkmark &   & \checkmark &                  & \checkmark                &                  \\
 &Nguyen et al.~\citep{nguyen2020multiple}        &   & \checkmark &   &   & \checkmark & \checkmark &   & \checkmark                &                  &                  \\
 &Costa et al.~\citep{costa2020charles}          & \checkmark &   &   &   & \checkmark & \checkmark &   & \checkmark                &                  &                  \\
 &Chowdhary et al.~\citep{chowdhary2020autonomous}   &   & \checkmark &   &   & \checkmark &   & \checkmark & \checkmark                &                  &                  \\
 &Bland et al.~\citep{bland2020machine}          &   & \checkmark &   &   & \checkmark &   & \checkmark &                  & \checkmark                &                  \\
 &Hu et al.~\citep{hu2020apu}                 &   & \checkmark &   &   & \checkmark & \checkmark &   & \checkmark                &                  &                  \\
 &Enoch et al.~\citep{enoch2020harmer}           &   & \checkmark &   &   & \checkmark &   & \checkmark & \checkmark                &                  &                  \\
 &Schwartz et al.~\citep{schwartz2020pomdp+}        &   & \checkmark &   &   & \checkmark & \checkmark &   &                  & \checkmark                &                  \\ \hline
 \multirow{7}{*}{2021}  
 &Dorchuck et al.~\citep{dorchuck2021goal}          &   &   & \checkmark &   & \checkmark &   & \checkmark & \checkmark                &                  &                  \\
 &Qian et al.~\citep{qian2021ontology}          &   & \checkmark &   &   & \checkmark &   & \checkmark & \checkmark                &                  &                  \\
 &Filiol et al.~\citep{filiol2021method}          & \checkmark &   &   &   & \checkmark &   & \checkmark & \checkmark                &                  &                  \\
 &Zhou et al.~\citep{zhou2021autonomous}        &   & \checkmark &   &   & \checkmark &   & \checkmark & \checkmark                &                  &                  \\
 &Hacks et al.~\citep{hacks2021towards}          &   & \checkmark &   &   & \checkmark &   & \checkmark &                  & \checkmark                &                  \\
 &Erd{\H{o}}di et al.~\citep{erdHodi2021simulating}     & \checkmark &   &   &   & \checkmark &   & \checkmark & \checkmark                &                  &                  \\
 &Ji et al.~\citep{ji2021optimal}             &   & \checkmark &   &   & \checkmark & \checkmark &   &                  & \checkmark                &                  \\ \hline
\multirow{5}{*}{2022}  
 &Dillon et al.~\citep{dillon2022perihack}        &   & \checkmark &   & \checkmark &   &   & \checkmark &                  & \checkmark                &                  \\
 &Yamin et al.~\citep{yamin2022use}              &   &   & \checkmark &   & \checkmark &   & \checkmark &                  & \checkmark                &                  \\
 &Confido et al.~\citep{confido2022reinforcing}    &   & \checkmark &   &   & \checkmark &   & \checkmark &                  & \checkmark                &                  \\
 &Tran et al.~\citep{tran2022cascaded}          &   & \checkmark &   &   & \checkmark &   & \checkmark &                  & \checkmark                &                  \\
 &Hance et al.~\citep{hance2022distributed}      & \checkmark &   &   &   & \checkmark &   & \checkmark & \checkmark                &                  &                  \\ \hline
\multirow{2}{*}{2023}   &F{\ae}r{\o}y et al.~\citep{faeroy2023automatic}       & \checkmark &   &   &   & \checkmark & \checkmark &   & \checkmark                &                  &                  \\
 &Li et al.~\citep{li2023innes}                    &   & \checkmark &   &   & \checkmark &   & \checkmark & \checkmark                &                  &                  \\ \hline
\multirow{8}{*}{2024}   &Xu et al.~\citep{xu2024autoattacker}        &   &   & \checkmark &   & \checkmark &   & \checkmark & \checkmark                &                  &                  \\
 &Becker et al.~\citep{becker2024evaluation}      &   & \checkmark &   &   & \checkmark &   & \checkmark & \checkmark                &                  &                  \\
 &Li et al.~\citep{li2024knowledge}           &   & \checkmark &   &   & \checkmark &   & \checkmark &                  & \checkmark                &                  \\
 &Alshehri et al.~\citep{alshehri2024breachseek}    & \checkmark &   &   &   & \checkmark &   & \checkmark & \checkmark                &                  &                  \\
 &Deng et al.~\citep{deng2024pentestgpt}        &   & \checkmark &   &   & \checkmark &   & \checkmark & \checkmark                &                  &                  \\
 &Wang et al.~\citep{Zhenduo}                   &   & \checkmark &   &   & \checkmark &   & \checkmark & \checkmark                &                  &                 \\
 &Li et al.~\citep{li2024dynpen} &   & \checkmark &   &   & \checkmark &  \checkmark &  &                 &                  &  \checkmark               \\
 &Wang et al.~\citep{wang2024pentraformer} &   & \checkmark &   &   & \checkmark &  \checkmark &  & \checkmark                &                  &                 \\

    \bottomrule[2pt]
    \end{tabular}
}
\end{table*}

Focusing solely on attacker modeling, the simulator allows for individual scanning actions and vulnerability exploitation targeting services on each machine. Scanning identifies services on ports, which are then exploited based on the machine's configuration. The attacker's actions are independent; one does not depend on the completion of another. Although scanning can guide vulnerability selection, traversing execution vulnerabilities can also grant access to the target machine. Importantly, these actions do not alter the target network, reinforcing the classification of the scenario as completely static.

In summary, Network Attack Simulator combines policy automation, authentic attribute simulation, isolated technical and tactical actions, and a completely static scenario. While it implicitly represents static defenders through subnet and machine connections and vulnerability success rates, the tool's simplicity limits its scalability for larger networks and lacks explicit defender modeling or a dynamically changing network environment.

% It implicitly models static defenders through subnet and machine connections and vulnerability exploitation success rates. While suitable for smaller networks with limited operations, it lacks scalability for larger networks. The attacker's actions are simplified to scanning and exploitation, with no explicit defender modeling or dynamically changing network environment.

% 以上4个案例不仅本身在学术界和工业界产生了广泛影响,而且也涵盖了\modelcla中非常典型的分类组合和多样化的应用,覆盖了大量的现有研究中相似的建模方法。通过分析这些案例,我们能够更深入地理解\modelcla的分类原则和应用,为下一章节对所有论文的整体分析奠定了基础。
% The four cases discussed above significantly influence both academia and industry, encompassing a spectrum of typical classification combinations and diverse applications within \modelcla. They represent a broad array of modeling approaches prevalent in current research. Analyzing these cases elucidates the principles and applications of \modelcla, providing a solid foundation for the comprehensive analysis of all papers in the subsequent chapter.


% \subsection{现有渗透测试场景建模方法研究情况}
% 我们调研了从上世纪90年代至今关于自动化渗透测、网络攻防博弈、自动化红队、渗透测试方法等相关的论文,采取了一系列系统的文献检索措施。检索平台包括谷歌学术、IEEE Xplore、ACM数字图书馆、SpringerLink和ScienceDirect等,重点聚焦于英文文献。筛选标准包括文章的整体质量、内容相关性、自动化渗透测试仿真建模方法、创新性以及引用量。共对64篇文章进行了深入分析,具体包括2000年以前3篇,2000-2009年10篇,2010年-2019年20篇,2020-2024年31篇。

% 为确保分析的全面性、准确性和可靠性,每项研究均由至少一名研究人员和两个大型语言模型平台(包括ChatGLM~\cite{glm2024chatglm}、Kimi、GPT-3.5、GPT-4~\cite{achiam2023gpt}、O1)进行交叉审查。在使用大语言模型进行分析时,我们上传了分析论文的文件或网址,并为模型设计了包含身份背景、评分标准、输出类别在内的整体提示。模型在输出类别时需提供理由,对于模糊或不准确的回答,研究人员将进行进一步提问。大语言模型给出的分类与研究人员给出的分类进行了交叉审查,具体结果见表~\ref{文献总结表}。

\subsection{Research on Existing Penetration Testing Scenario Modeling Methods}

% 在本论文的文献筛选过程中,我们采纳了一套严格而系统的筛选框架,以确保所选文献的质量和相关性。首先,我们精确界定了研究主题,并选取了AutoPT, network attack-defense games, automated red teams, and penetration testing methodologies等关键词,以便在Web of Science、Scopus、IEEE Xplore等国际知名数据库中进行深入检索。我们对从1990s to the present内的文献进行了全面检索,以实现对AutoPT领域研究趋势的全方位把握。
% 在初步检索阶段,我们获得了数百篇相关文献。为了筛选出最具价值的文献,我们首先对标题和摘要进行了快速浏览,以排除那些与主题关联性较弱、内容重复、或内容不相关的文献。例如,我们淘汰了那些仅讨论网络防御与检测的论文,和那些虽然提到自动化测试但未深入探讨其具体算法技术的研究,以及那些由于关键词的相似性,如cone penetration test,而检索到但内容完全不相关的文章。
% 接下来,我们对剩余的文献进行了全文阅读,并根据以下四个关键标准进行了进一步的筛选:
% 研究内容的相关性:我们优先考虑那些直接聚焦于AutoPT技术发展、算法生成和应用案例文献,并且内容中需包含AutoPT建模细节。
% 研究方法的科学性和可靠性:我们选择了那些采用实验验证、数据分析和模型构建等严谨研究方法的文献。
% 文献的学术影响力:我们参考了文献的被引用次数和发表期刊的影响因子,以评估其学术价值。例如,发表在“IEEE Transactions on Information Forensics and Security”上的文章~\cite{li2024dynpen}因其高影响因子被视为高质量文献。
% 研究的时效性:虽然我们考虑了所有时间段的文献,但我们特别关注了近五年的研究成果,以确保分析的文献能够反映当前AutoPT领域的最新进展,。
% 经过这一系列的细致筛选,我们最终从海量文献中精选出66篇具有高度代表性和参考价值的文献,并对他们进行进一步的分析与讨论。

% In this paper, we employ a rigorous and systematic approach to literature screening to ensure the quality and relevance of the selected studies. We began by precisely defining our research topic and identifying keywords such as AutoPT, network attack-defense games, automated red teams, and penetration testing methodologies. We then conducted exhaustive searches in prestigious databases like Web of Science, Scopus, and IEEE Xplore, covering literature from the 1990s to the present to achieve a holistic grasp of AutoPT modeling research trends.


% During the initial phase, we collected hundreds of documents and quickly screened titles and abstracts to exclude those with low thematic relevance, redundancy, or irrelevance. We excluded papers that only discussed network defense and detection, those that mentioned automated testing without delving into algorithmic techniques, and unrelated articles, articles retrieved due to keyword similarities, such as cone penetration test, which were completely unrelated in content. 

% We then scrutinized the full texts of the remaining documents, applying four key criteria for selection:
% \begin{itemize}
%     \item Content Relevance: We favored literature directly addressing AutoPT technology development, algorithm generation, and applications, including detailed modeling aspects.

%     \item Methodological Rigor: We chose documents employing robust research methods like experimental validation, data analysis, and model construction.

%     \item Academic Impact: We considered the number of citations and journal impact factors to evaluate academic value.

%     \item Research Recency: While we considered literature from all periods, we emphasized studies from the last five years to reflect the latest AutoPT modeling advancements.
% \end{itemize}

% After this thorough screening, we selected 65 highly representative and valuable documents for further analysis and discussion: 3 before 2000, 10 from 2000-2009, 19 from 2010-2019, and 33 from 2020-2024.

% 我们对所选文献的分析采用了一种系统而全面的方法。首先,我们对每篇文章的研究背景、目的和意义进行了梳理,以明确各研究的核心议题。接着,我们详细阅读分析分析了各篇文章的AutoPT的建模方法、尤其是针对网络架构与网络资产、攻击者和防御者建模,以分析其建模各维度的特征。在此基础上,我们对各文章中仿真建模方式进行了深入解读,并将其按照~\modelcia进行分类,详细的结果展现在表~\ref{Classification_Table}中。
% Our systematic and comprehensive literature analysis began by organizing the research background, objectives, and significance of each article to identify literature objectives. We then conducted a thorough examination of the AutoPT modeling methods, focusing on network architecture and assets, as well as attacker and defender modeling, to understand their characteristics. Based on this, we categorized the simulation modeling methods from each article according to the ~\modelcla, with results summarized in Table~\ref{Classification_Table}. To ensure comprehensive, accurate, and reliable analysis, each study underwent cross-review by at least two researchers. 

We conducted a systematic literature review of AutoPT studies using Web of Science, Scopus, and IEEE Xplore (1990s-present). A two-stage screening process filtered out irrelevant and low-quality papers. Inclusion criteria consisted of thematic relevance, methodological rigor, academic impact, and research recency.
After applying these criteria, 65 representative documents were selected for analysis, with 33 from 2020-2024. Each study was cross-reviewed by at least two researchers to ensure accuracy and reliability. We categorized AutoPT modeling methods according to their characteristics (Table~\ref{Classification_Table}) and examined the research background, objectives, and significance of each article.


% 正如前文中提到的那样,使用这些工具需要考虑法律和道德限制,通常是在隔离的网络靶场中进行的,我们在统计时将他们统一归类为完全静态场景。
%%%%%%%%%%

We summarized the article count across four dimensions in Table~\ref{Classification Statistics Table}. Notably, some automated execution tools like Nessus, Metasploit, and nmap have unrestricted application scenarios (therefore denoted by dash notation). Generally, their use requires consideration of legal and ethical constraints, typically in isolated network environments, categorized as Completely Static Scenarios.

\begin{table*}[h!t]
\centering
\caption{Classification Statistics of Simulation Modeling Methods in Automated Penetration Testing Literature}
\label{Classification Statistics Table}
\resizebox{\linewidth}{!}{
\begin{tabular}{c|ccc|cc|cc|ccc}
\toprule[2pt]
\multirow{2}{*}{Types} &\multicolumn{3}{c|}{Literature Objectives}         & \multicolumn{2}{c|}{Network Simulation Complexity}         & \multicolumn{2}{c|}{Dependency of T\&T Operations}         & \multicolumn{3}{c}{Scenario Feedback and Variation}                \\ \cline{2-11}
% \multirow{2}{*}{Types}& \multicolumn{3}{c|}{Research Objectives}         & \multicolumn{2}{c|}{Network Simulation Complexity}         & \multicolumn{2}{c|}{Pre- and Post-Technical and Tactical Correlation}         & \multicolumn{3}{c}{Scenario Feedback and Variability}                \\ \cline{3-12}
& Technical &   Policy & Complete  & Hypothetical  &  Authentic & Isolated & Coordinated & Completely Static& Semi-Dynamic &  Completely Dynamic  \\ 
% \midrule[1.5pt]
\hline

% & Sub-Types & TT-Auto & SG-Auto & FP-Auto & Non-RAS & RAS & Iso-TTA & Cont-TTA & CS-S & SS-S & CD-S \\ 
% \midrule[1.5pt]


% \multirow{2}{*}{Types} & \multicolumn{3}{c|}{场景自动化目标}         & \multicolumn{2}{c|}{网络模拟逼真度}         & \multicolumn{2}{c|}{前后技战术关联}         & \multicolumn{3}{c}{场景反馈与变化}                \\  \cline{2-11}
% & 战术自动化 & 策略自动化 & 全流程自动化 & 非真实属性 & 真实属性 & 孤立技战术 & 连续技战术 & 完全静态 & 半静态 & 完全动态 \\ \bottomrule[1.5pt]
Quantity& 17 & 43 & 5 & 10 & 55 & 29 & 36 & 43 & 20 & 2\\
\bottomrule[2pt]
\end{tabular}
}
\end{table*}

Policy automation and intelligent decision-making are prominent research areas, attracting significant academic attention. Most studies focus on simulating authentic attributes and continuous technical actions, closely mirroring practical scenarios. However, research on dynamic environments remains limited. For example, Applebaum et al.~\cite{applebaum2017analysis} introduce active network changes using gray agents, but initiate only one connection set per round, resulting in minimal alterations within small to medium networks (11–21 hosts). Similarly, Li et al.~\cite{li2024dynpen} carefully define network changes but do not quantify them or test scalability in larger networks. Their experiments are confined to a 10-node network and overlook dynamic simulation and emulation for larger systems. Additionally, their simplistic action settings focus on vulnerability exploitation without addressing the logical relationships among multiple penetration tactics.

\begin{figure*}[tb]
    \centering
    \includegraphics[width=\linewidth]{figure/Dimensions_Over_Time.pdf}
    \caption{Temporal Variations in Article Volume Across Dimensions}
    \label{fig: Dimensions Over Time}
\end{figure*}

Table~\ref{Classification Statistics Table} summarizes literature classifications over time, while Figure~\ref{fig: Dimensions Over Time} shows trends in article volumes across dimensions. Initially, AutoPT research focused on Technical Automation, emphasizing the automation of specific tactics and steps. Over time, Policy Automation became more prominent, evolving from simulating hypothetical attributes and isolated actions to simulating authentic attributes and coordinated technical and tactical actions. This shift reflects reduced abstraction in simulation modeling and a closer alignment between models and reality, paving the way for integrating intelligent decision-making algorithms with automated tools for Complete Automation. Ongoing research in Complete Automation typically streamlines the AutoPT process by targeting specific components, incorporating one or more automated tools—such as rules, planners, or basic reinforcement learning models—to develop action-guiding strategies with a greater emphasis on engineering implementation and lower intelligence levels. Notably, Ghanem et al.~\cite{ghanem2019reinforcement} focus on automating strategy generation using tools like MSF for execution. Although Complete Automation is not fully achieved, their work is practically significant and classified under Complete Automation, demonstrating the flexibility of our criteria.

% However, most studies have not considered active network changes, focusing instead on static and semi-dynamic scenarios without leveraging dynamic network information. Proposals like Cyber Mimic Defense~\cite{2016Research} and Moving Target Defense~\cite{2011Moving} indicate a shift toward dynamic network security strategies. Research should integrate active and passive network changes for intelligent decision-making and automated execution in completely dynamic scenario. In addition to this, research predominantly simulates small or medium networks, as seen in~\cite{Cyberbattlesim, li2024dynpen, applebaum2017analysis}, neglecting large-scale network modeling and the potential impact of different network architecture on penetration testing. Furthermore, existing simulation methods lack flexibility, focusing on limited combinations without offering a unified approach for multi-dimensional and multi-level simulation modeling.

Besides, most studies focus on static or semi-dynamic scenarios, neglecting active network changes and dynamic information. Approaches like Cyber Mimic Defense~\cite{2016Research} and Moving Target Defense~\cite{2011Moving} signal a shift toward dynamic network security strategies. Future research should integrate both active and passive network changes to enable intelligent decision-making and automated execution in fully dynamic environments. Additionally, existing studies typically simulate only small to medium-sized networks~\cite{Cyberbattlesim, li2024dynpen, applebaum2017analysis}, overlooking large-scale network modeling and the impact of diverse network architectures on penetration testing. Furthermore, current simulation methods lack flexibility, focusing on limited combinations without providing a unified approach for multi-dimensional and multi-level simulation modeling.



% 主要关注英文文献,对选定的文献进行了细致的分析,以提取和综合相关信息。我们定义了一个建模信息总结框架,将网络空间建模和渗透测试流程建模为主题,具体内容见下表。每项研究至少由一名研究人员和一个大语言模型平台智谱轻言进行交叉审查,以确保总结……过程的可靠性。总结的建模方法随后进行了比较和对照,以识别现有研究中的模式、趋势和空白。这种分析方式使我们能够提供一个结构化且全面的自动化渗透测试中网络和流程建模的现状概览。纳入文献的标准如下:① 经过同行评审的期刊文章和会议论文;② 文章内容涉及详细的自动化渗透测试中的网络或流程建模;③ 文献提供了新颖的见解、方法或框架。排除的标准包括:① 社论、综述或评论;② 无法获取全文的文献;③ 重复或内容大量重叠的文献;④ 与本综述特定焦点不相关的文献。基于以上标准筛选的合标文献共xxx篇。

% 基于上述调研,我们对四个维度不同层次的文章总数进行了总结,详见表~\ref{分类统计表}。
% \input{tables/所有文献分类表}

% 其中,Ghanem等人的研究\cite{ghanem2019reinforcement}主要聚焦于生成策略自动化,并结合了部分自动化执行工具,如MSF,实现了自动执行效果。尽管未实现全流程自动化,但其工作具有现实意义,我们据此将其归类为全流程自动化,体现了分类标准的灵活性。Nessus、Wireshark、Metasploit、nmap等自动化执行工具在应用场景上无限制,因此我们使用“\textbackslash”表示。
% 不同时间段的文献分类数量汇总见表~\ref{分类统计表}。
% \input{tables/文献分类数量汇总}
% 从表格中可以看出,策略自动化是当前研究的主流,智能决策能力的研究受到学术界的高度关注。同时,大多数研究集中于真实属性模拟与连续技战术的研究,这些建模方式与现实生活更为贴近。值得注意的是,完全动态场景的研究较少,Applebaum等人\cite{applebaum2017analysis}通过设置灰色代理引起网络的主动变化,但每个回合只会开启一组连接,变化范围较小,主机数量设置为11-21,属于中小型网络,场景模拟限制性较大。
% 不同维度文章数量及比例随时间的变化见图~\ref{fig:不同分类维度数量随时间变化}。从图~\ref{fig:不同分类维度数量随时间变化}我们可以观察到,最初自动化渗透测试研究集中在技战术自动化,研发工具代替自动化渗透测试中的某一种具体的战术和环节。随着时间推移,策略自动化得到蓬勃发展,策略自动化的研究也经历了从非真实属性模拟、独立技战术到真实属性模拟、连续技战术的变化,这本质上是仿真建模抽象程度逐渐降低、与现实的差距逐渐减小,为后续智能决策算法与自动化执行工具相结合、实现全流程自动化执行打下了基础。然而,绝大多数仿真方法都没有考虑网络的主动变化,也就是研究集中在完全静态场景和半静态场景,对网络的主动变化产生的相关信息考虑不足、利用也不足,但是现阶段网络空间拟态防御(Cyber Mimic Defense)\cite{2016Research}、动态目标防御(Moving Target Defense)\cite{2011Moving}等防御策略的提出证明了网络防御从静态、被动防御向主动、动态防御的转变趋势,自动化渗透测试研究也应关注这一变化,在仿真建模时同时考虑网络的主动变化与静态变化,以实现在完全动态场景上的智能决策与自动执行。

% 使用桑基图进行相关汇总
% \begin{figure}[tb]
% \centering
% \includegraphics[width=\linewidth]{figs/桑基图参考.png}
% \caption{文献分类流向图}
% \label{fig3:data}
% \end{figure}



