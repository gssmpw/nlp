\section{Introduction}
\IEEEPARstart{T}{he} Internet is crucial for modern life and social advancement, significantly impacting government, finance, energy, and military sectors, yet cybersecurity remains a critical concern despite enhanced convenience and services~\cite{chen2022research, wani2021sdn, verma2024revisiting}.
Penetration testing, which simulates hacker attacks to assess vulnerabilities and overall security~\cite{abu2018automated, dorchuck2021goal}, is vital for improving cybersecurity, but is complex and time consuming, relying on the expertise of the tester~\cite{applebaum2016intelligent}. As a solution, automated penetration testing (AutoPT) has emerged to replace human efforts and accelerate evaluations~\cite{ghanem2018reinforcement}.

AutoPT involves two main processes: intelligent decision-making and automatic execution. The intelligent decision-making phase is critical, encompassing target identification, attack path selection, and method determination, effectively replacing manual judgments. Current approaches to intelligent decision-making can be classified into three categories:


\begin{enumerate}
    \item \textit{Fixed Script Execution Methods} employs predetermined rules, which limits adaptability and flexibility when working with other penetration tools~\cite{hacks2021towards}.
    \item \textit{PDDL with Planner Methods} leverage the PDDL language to define penetration actions, with classical planners facilitating the process~\cite{applebaum2017analysis,yichao2019improved}. While PDDL enables detailed definition of penetration parameters and easy integration with tools, manually crafting these definitions is labor-intensive, requiring diverse approaches for various tools and tactics.
    \item \textit{Artificial Intelligence Methods} adopt attack trees~\cite{dorchuck2021goal}, attack graphs~\cite{obes2013attack}, reinforcement learning~\cite{hu2020automated,chowdhary2020autonomous}, and large language models~\cite{bianou2024pentest,deng2024pentestgpt} to guide decision-making in penetration testing, representing a leading trend in current research.
\end{enumerate}

% (1) Fixed script execution \cite{hacks2021towards}, wherein penetration steps are combined using predetermined rules, resulting in limited adaptability to target scenarios and reduced flexibility; 
% (2) PDDL combined with planner methods \cite{applebaum2017analysis,yichao2019improved}, which utilize the PDDL language to define penetration actions, with classical planners facilitating the planning process. The PDDL language allows for a detailed definition of the parameters required for penetration strategies and is amenable to integration with penetration tools. However, manually crafting PDDL definitions is labor-intensive, and the diversity of tools and tactics may necessitate multiple definition approaches for a single action, thereby increasing the workload; and 
% (3) Artificial intelligence methods, which include the use of attack trees \cite{dorchuck2021goal}, attack graphs \cite{obes2013attack}, reinforcement learning \cite{hu2020automated,chowdhary2020autonomous}, and large language models \cite{bianou2024pentest,deng2024pentestgpt}, among others, for guiding the decision-making process in penetration testing. This has emerged as a leading trend in research. 
For all of the above categories, it is important to model target scenarios, technical and tactical elements, and decision parameters. AI methods, in particular, require extensive training in diverse scenarios to reveal hidden patterns and improve generalizability. 

On the other hand, the automatic execution phase utilizes various penetration tools to implement predetermined actions, automating tasks typically performed by human experts. 
Integrating intelligent decision-making with automatic execution seeks to help experts in network penetration, improve efficiency, reduce costs, and improve security outcomes~\cite{chenke2023survey}.

% 直接构建网络靶场经济成本高、时间成本高,构建的难度和复杂性都是一个巨大的挑战,且网络靶场场景的变化缓慢,难以体现真实生活中的动态性和随机事件。


As shown in Figure~\ref{Introduction}, training within real network environments or cyber ranges demands significant resources, making it impractical due to high time and financial costs, as well as the complexities involved.
The inherent unpredictability and lack of reproducibility of real-world scenarios pose significant challenges for agents to in conducting repetitive training and recognizing patterns. By contrast, simulation environments provide a cost-effective and straightforward solution, with flexibility that allows for diverse scenarios and rapid feedback—key to improving the AutoPT decision-making process.


% 构建仿真环境成本低廉、方法简单灵活,可以为AutoPT智能决策方法提供多样的场景与快速的反馈
\begin{figure*}[tb]
    \centering
    \includegraphics[width=\linewidth]{figure/Introduction.pdf}
    \caption{The Necessity and Challenges of Simulation Environments Modeling in AutoPT}
    \label{Introduction}
\end{figure*}

The quality and fidelity of simulation modeling is crucial for effective algorithm training in AutoPT. Real-world application demands algorithms that are both capable of learning and refining their performance from data and reflective of essential real-world features. 
Despite advances in AutoPT research, there is a lack of systematic analysis of simulation modeling methodologies. 
Current methods are diverse and fragmented, lacking a unified framework for modeling characteristics, elements, granularity, and procedures.  This hampers the development of efficient simulation models. 
% Current methodologies, tailored to individual research needs, show diversity and fragmentation. There is an absence of a unified guiding framework for modeling characteristics, elements, granularity, and steps, which hinders the development of effective simulation models.
% Current methodologies exhibit diversity and fragmentation, 均立足于自身研究需求。对建模的特征、元素、粒度、步骤等缺乏统一的指导框架, the lack of a cohesive framework poses challenges to the development of effective simulation models.
% Furthermore, security and privacy concerns limit the availability of public datasets for simulated network scenarios, posing significant challenges for training agents and comprehensively assessing AutoPT efficacy. 
A public dataset is essential for a rigorous and unbiased evaluation of penetration testing algorithms. Yet, concerns over security and privacy often result in a scarcity of public datasets for simulated networks, hindering a comprehensive assessment of AutoPT's efficacy.

% 我们需要一个公开的的数据集,用于对渗透测试算法进行公平的、全面的评估。
% Creating a unified framework faces challenges due to: 1) the complexity of real-world systems requiring models that capture a broad spectrum of behaviors and interactions, 2) the rapid technological evolution and dynamic cyber threats demanding simulation models that can be swiftly updated, 3) the varied research objectives and stakeholder needs in AutoPT complicating the development of a universally applicable framework, and 4) Privacy protection and security concerns, along with modeling frameworks and public datasets, must prioritize privacy and academic ethics while accurately capturing real-world attack scenarios and guaranteeing availability and reproducibility.

Creating a unified framework presents several challenges. Firstly, the complexity of real-world systems demands models capable of encompassing diverse behaviors and interactions. Secondly, rapid technological advancements and evolving cyber threats require models that can be quickly updated. Thirdly, diverse research goals and stakeholder needs in AutoPT complicate the creation of a universally applicable framework. Lastly, privacy and security concerns must be prioritized, ensuring that models and datasets adhere to ethical standards while accurately depicting real-world attack scenarios and ensuring availability and reproducibility.


% 仿真建模的优劣会直接影响算法训练的的效果与有效性,尤其是针对自动化渗透测这种需要应用于现实生活的领域,急需要进行抽象确保算法能够进行训练,又需要保留一定的现实特征,为后续在现实生活中的应用打下基础。


This paper comprehensively reviews academic literature on AutoPT Modeling from the 1990s to today, focusing on key terms including AutoPT, network attack-defense games, automated red teams, and related areas. We offer an in-depth analysis and interpretation of simulation modeling methods used in various AutoPT studies. Our contributions include:

\begin{itemize}
    \item We are the first to analyze and classify simulation modeling methods in AutoPT. Through an extensive review of the literature on AutoPT, we decompose the elements in AutoPT's simulation modeling and propose the \textbf{M}ulti-\textbf{D}imensional \textbf{C}lassification System for \textbf{P}enetration Testing \textbf{M}odeling (\modelcla) to systematically organize current simulation modeling methods within AutoPT.
    
    \item To address the absence of a unified approach for multi-dimensional and multi-level simulation modeling, dynamic environment modeling, and the lack of public datasets, we introduce the AutoPT Simulation Modeling Framework (\modelsim), a novel framework that leverages policy automation and integrates all sub-dimensions across the other three dimensions.

    \item We present a publicly available network simulation dataset along with the Network Generator code. This dataset can be flexibly combined to support various simulation modeling levels focused on policy automation within \modelcla. The Network Generator enables customizable network data generation by adjusting parameters or fine-tuning the settings, facilitating future research in AutoPT.
\end{itemize}

The paper is organized as follows: Section 2 examines the theoretical foundations and core modeling elements in simulation modeling for AutoPT. Section 3 presents our proposed Multi-Dimensional Penetration Testing Modeling Classification System (\modelcla), discussing typical cases and current research trends. Section 4 introduces AutoPT Simulation Modeling Framework (\modelsim), along with the open-source network simulation dataset and Network Generator. Section 5 concludes with a summary of our contributions and future work.
