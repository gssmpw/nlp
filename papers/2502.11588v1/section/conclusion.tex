\section{Conclusion}
% 渗透测试通过模拟黑客行为评估网络中潜在漏洞和整体安全性,是提升网络安全能力的重要手段,但渗透测试的水平效果与应用范围长期受限于执行渗透测试专家的个人能力与人员储备。为解决这一问题,自动化渗透测试应运而生,以智能体决策代替人类专家决策,集成自动执行工具代替手动执行。但是绝大部分决策方法都需要对目标场景与技战术要素进行建模,需要广泛、大量的渗透测试数据及长期的交互训练,直接在真实网络或网络靶场中进行训练不仅难以实现,也会消耗大量人力物力,因此,网络环境仿真建模不可或缺。

% 本文通过阅读大量自动化渗透测试相关论文,总结了自动化渗透测试中场景建模的关键要素,并创新性提出了一个自动化场景建模方法分类架构\modelcla,能够将当前所有相关研究的场景建模方法进行清晰、明确的分类。针对当前自动化渗透测试场景建模涵盖范围小、场景分散、没有统一构建标准与公开数据集的问题,我们提出了一种以策略自动化为核心,同时保留技战术自动化与全流程自动化接口,涵盖另外3维度所有层次的自动化渗透测试仿真建模方法\modelsim,并公开了仿真网络场景数据集与攻击者防御者动作数据集,通过对已公开数据集的灵活组合,可以支持 3 个维度及所有层次的仿真建模。同时还公开了仿真网络场景生成器的代码,支持所有研究人员通过自定义生成器的参数或对网络生成器进行微调以输出自己需要的目标背景网络数据。

% By integrating publicly available datasets flexibly, support is offered for various simulation modeling levels focused on strategy automation in \modelcla. We have also made the network generator code public, enabling researchers to output customized target network data by adjusting parameters or fine-tuning the network generator. Using \modelsim, we address the limitations of current scene modeling methods, which focus primarily on small to medium-sized networks and overlook large networks and diverse architectures. These methods lack fully dynamic scene modeling, publicly available datasets, and flexibility, and they fail to provide a unified approach for multidimensional and multi-layer simulation modeling.
% Penetration testing evaluates potential vulnerabilities and overall security within a network by simulating hacker behavior, serving as a vital means to enhance cybersecurity capabilities. However, the effectiveness and application scope of penetration testing have long been constrained by the individual expertise and personnel availability of the experts conducting the tests. To address this issue, automated penetration testing has emerged, replacing human expert decisions with intelligent agent decision-making and manual execution with integrated automated tools. Yet, most decision-making methods require modeling of the target scenario and tactical elements, necessitating extensive and voluminous penetration testing data and prolonged interactive training. Training directly on real networks or cyber ranges is not only impractical but also consumes considerable human and material resources. Therefore, the simulation modeling of network environments is indispensable.

% This paper, by reviewing a substantial number of AutoPT-related papers, summarizes the key elements of scenario modeling in AutoPT and innovatively proposes a classification framework for automated scenario modeling methods \modelcla, which clearly and explicitly categorizes all current research on scenario modeling methods. Addressing the issues of limited coverage, scattered scenarios, and the lack of unified construction standards and public datasets in current AutoPT scenario modeling, we propose an AutoPT simulation modeling method \modelsim that centers on strategy automation while retaining interfaces for tactic and technique automation and full-process automation, encompassing all levels of the other three dimensions. We have also made public the simulation network scenario dataset and the attacker-defender action dataset. Through flexible combination of the published datasets, support is provided for simulation modeling at all levels across the three dimensions. Additionally, the code for the simulation network scenario generator has been released, enabling all researchers to output the target background network data they require by customizing the generator's parameters or fine-tuning the network generator.


% 我们希望通过我们的构建方法与数据集,能够为自动化渗透测试仿真建模方法进行有效的指引,并相关研究提供一套标准数据,为智能决策方法进行公平的比较与衡量构建数据基础。据我所知,我们的工作是首个对自动化渗透测试中的仿真建模进行分析与分类的工作,也是首个指引仿真建模构建与公开标准数据集的工作。

% 除了场景建模方法不统一及没有公开数据集之外,自动化渗透测试研究中还有一个亟待解决的问题就是评估方法不统一,大量研究进行方法评估时,通常只会研究方法的收敛速度与在某一目标网络下在某一种奖励设置下的的累积奖励,奖励的设置方式的多样性导致不同方法之间难以进行比较。初此之外,没有其他公认的评估指标体系用于自动化渗透测试智能决策方法效果评估,我们希望通过后续研究,能够提出一套统一的评估指标,用于协助不同自动化渗透测试方法之间的横向比较,共同促进自动化渗透测试相关研究。
% We hope that our construction method and dataset can effectively guide the simulation modeling methods for AutoPT and provide a standard set of data for related research, establishing a data foundation for fair comparison and measurement of intelligent decision-making methods. To the best of our knowledge, our work is the first to analyze and classify simulation modeling in AutoPT, as well as the first to guide the construction of simulation models and to make public a standard dataset.

% In addition to the lack of a unified scenario modeling method and the absence of public datasets, another pressing issue in AutoPT research is the lack of a unified evaluation method. Many studies, when evaluating methods, typically only examine the convergence speed of the method and the cumulative reward under a certain reward setting within a specific target network. The diversity of reward setting methods makes it difficult to compare different methods. Moreover, there is no other recognized set of evaluation metrics for assessing the effectiveness of intelligent decision-making methods in AutoPT. We hope that subsequent research can propose a unified set of evaluation metrics to facilitate horizontal comparison between different AutoPT methods, thereby collectively advancing research in AutoPT.

% Penetration testing is crucial for identifying vulnerabilities and enhancing network security by simulating hacker actions. However, its efficacy is often limited by the expertise and availability of human testers. To address these limitations, AutoPT has emerged, replacing expert judgments with intelligent agents and manual processes with automated tools. Despite this advancement, most automated methods require extensive scenario and tactical modeling, demanding vast penetration testing data and time-consuming interactive training. Conducting such training on real networks or cyber ranges is impractical and resource-intensive, thus making simulated network environment modeling essential. 

% This paper reviews a substantial body of literature on AutoPT and proposes an innovative classification framework \modelcla~for scenario modeling methods. This framework categorizes current research distinctly. Our work addresses the limited scope, fragmented scenarios, and the lack of unified standards and public datasets in current AutoPT modeling. We introduce \modelsim, a simulation modeling method that emphasizes strategy automation while accommodating tactic and technique automation, as well as full-process integration. Our public release includes the network scenario dataset and the network generator code, supporting flexible scenario modeling across all levels and allowing researchers to produce customized network data by adjusting generator parameters. 

This paper reviews the literature on AutoPT and introduces an innovative classification framework, \modelcla, for scenario modeling methods. Our framework categorizes existing research distinctly, addressing the limited scope, fragmented scenarios, and lack of unified standards and public datasets in current AutoPT modeling. We propose~\modelsim, a method that emphasizes strategy automation while supporting tactic and technique automation, as well as full-process integration. Our public release includes a network scenario dataset and network generator code, facilitating flexible scenario modeling across all levels and enabling researchers to customize network data by adjusting generator parameters.
Our construction method and dataset aim to guide simulation modeling in AutoPT and serve as a standard data benchmark for fair comparisons of intelligent decision-making methods. To our knowledge, this is the first work to analyze and classify simulation modeling in AutoPT, while offering guidance and standard datasets for model construction.


Ethical considerations are crucial in AutoPT. Our modeling framework, \modelsim, uses real-world data while abstracting it to protect privacy and excludes actual penetration tools and payloads, ensuring no direct real-world application. Our research focuses on developing penetration strategies without full automation, thereby avoiding potential harm to systems or users.

In our current modeling framework, we detail the modeling of attacker and defender actions and plan to release the attacker-defender action dataset and state transition functions in a future phase. We will enhance our characterization of attacker capabilities post-intrusion. Recognizing that information gathering significantly depends on the attacker's expertise, we will examine network visibility disparities between attackers and defenders, as well as defenders' delayed response times to attacker actions.
Another significant challenge in AutoPT is the lack of a unified evaluation method. Current evaluations often emphasize convergence speed and cumulative rewards in specific network settings, with varying reward configurations complicating comparisons. There is no widely accepted set of metrics for assessing the effectiveness of intelligent decision-making. We advocate for future research to develop standardized evaluation metrics to enhance comparison and advance AutoPT methodologies.
% 在现有的建模框架中,我们已经讨论了攻击者和防御者动作的建模,The dataset of attacker-defender actions 和状态转移函数is planned for public release in a subsequent phase. 后续我们还将更深入的刻画攻击者能力,尤其是在攻击者获取设备的权限之后,现有的研究通常直接假定攻击者能够获取节点所有的有用信息,但是实际上信息的获取十分依赖于攻击者的经验与能力。除此之外,我们还将深入研究攻击者和防御者对网络可见性的区别和防御者相对于攻击者的滞后性观察的建模。


% Real-life penetration actions are constrained by execution limitations, and their variety complicates standardization efforts. Many studies omit these relationships, reducing execution constraints and unifying decision parameters but increasing abstraction and divergence from real-world scenarios. Addressing this gap is crucial when transitioning from simulation models to real environments.





