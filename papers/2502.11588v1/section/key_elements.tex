\section{THEORETICAL FOUNDATIONS OF AUTOPT SIMULATION MODELING}
\subsection{Control Theory Perspective}

% 整个这一章节要再思考一下。现有的AutoPT的问题,还体现在a和d对模型参数的理解不一致上,attacker不能掌握全貌,defender无法估计attacker的行为。建议把这个思考并加入
% 在自动化渗透测试领域,仿真建模不仅涉及对目标网络空间的抽象和建模,还涵盖了网络中交互主体——攻击者和防御者的建模。我们使用现代控制理论中的非线性系统理论来理解和划分自动化渗透测试中的仿真建模。
% 现代控制理论是自动控制领域的一个重要分支,它主要关注复杂系统的分析和设计,特别是在多输入多输出(MIMO)系统、非线性系统和时变系统中的应用。它使用状态变量和状态空间方程来描述系统的动态行为。这种方法允许对系统进行更全面的分析,包括系统的内部状态而不仅仅是输入和输出。它在自动化、机器人技术、工业控制等多个领域中发挥着关键作用。
% 在现代控制理论中,一个系统被视为由多个相互制约的部分构成的整体,这些部分相互依赖、相互作用,共同发挥特定的功能。定义一个系统时,需要考虑其整体性、动态交互、反馈循环、层次性和开放性。在自动化渗透测试领域,将目标网络、攻击者和防御者视为一个整体的系统,可以更全面地理解和模拟网络空间的安全态势。输入和输出具有相对性,攻击者的行为可以被视为系统的输入,但同时,这些行为也可能是系统内部状态变化的结果,即系统的输出。同样,防御者的响应既是对攻击的输出,也是影响系统状态的输入。由于网络活动的复杂性和多样性,以及渗透和防御手段受多种因素的影响,即使是对同一系统执行相同的动作(例如漏洞利用),执行效果也可能不同,这意味着渗透测试系统是时变系统。输入同样的动作组合,输出也有可能是不同的,不能使用线性微分方程描述系统的输入和输入,叠加性(superposition principle)不再适用,因此这类系统属于非线性系统。当然,正如线性系统是特殊条件下的非线性系统,通过在建模过程中设置特定条件,渗透测试中的非线性问题也可以转化为线性问题。在非线性系统中,攻击者和防御者之间的动态交互是系统行为的关键因素,相互之间形成复杂的反馈循环。在现有的研究中,大多数建模方法都是以攻击者和防御者的动作为时间节点,将渗透测试活动构建为一个离散系统,其中攻击者和防御者的动作都是离散的。系统的输出不仅与自身的结构参数有关,还与输入和初始条件相关。自动化渗透测试仿真系统框架图见图\ref{nonlinear}.
% Game theory is about two or more players use each other's strategies to optimize their own adversarial strategies in order to achieve objectives. It is a branch of applied mathematics and operations research with applications in biology, economics, international relations, computer science, political science, and military strategy~\cite{2008Essentials}. In cybersecurity, game theory models the adversarial interactions between attackers and defenders during network infiltration, helping to assess the effectiveness of various attack and defense strategies and enhance network security in a targeted manner~\cite{2010Game}.

% Cyber security issues are complex because  individuals or actors (attackers and defenders) with different targets and levels of information about the cyber network’s structure/topology communicate and interact with each other~\cite{2019Game}. Therefore, in AutoPT, simulation modeling involves both the abstraction of the target cyberspace and the representation of interactive entities within the network, specifically the attackers and defenders. 

% According to Tavafoghi et al. reasearch, the cyber security problem can be formulated as a stochastic dynamic game with asymmetric information~\cite{2019Game}. 
% AutoPT exhibits these key traits: (1) The network penetration process is dynamic, with attackers and defenders vying for control over network nodes to achieve their objectives through interaction. (2) Attackers and defenders have distinct information sets; attackers are aware of the hosts they control but may lack knowledge of the network's overall topology and configuration, while defenders are familiar with the network's structure but are unaware of the attacker's moves. They share some common knowledge, such as the configuration of controlled hosts and the detection of attacks on specific nodes. (3) Attackers and defenders have different goals and strategies, yet their decisions are interdependent. Given these observations, AutoPT can be modeled as a stochastic dynamic game with asymmetric information~\cite{2019Game}.

% 自动化渗透测试具有以下特征:(1)网络的渗透过程是动态的,攻击者和防御者通过相互之间的交互夺取网络中节点的控制权限,努力达成自己的目标。(2)攻击者和防御者的拥有的信息是不同的。攻击者知道已经受到控制的主机,但是对于网络整体的拓扑和配置可能是未知的,防御者通常知道网络的环拓扑与配置,但是不知道攻击者的相关行动,他们之间也存在一些共同的信息,比如受控主机的配置或对某一个节点的攻击行动被检测到。(3)攻击者和防御者的目标和策略是不同的,但是决策又会受到对方的影响。基于以上观察,the cyber security problem can be formulated as a stochastic dynamic game with asymmetric information~\cite{2019Game}. 

In AutoPT, simulation modeling not only involves the abstraction and modeling of the target cyberspace but also encompasses the modeling of interactive entities within the network—attackers and defenders. We draw upon nonlinear system theory from Modern Control Theory to understand and categorize simulation modeling in AutoPT.

% Modern Control Theory, a pivotal branch of automatic control, specializes in analyzing and designing complex systems, including Multipe Input Multiple Output (MIMO), nonlinear, and time-varying systems. It employs state variables and state space equations to capture the system’s dynamic behavior, offering a holistic view that encompasses internal states beyond mere inputs and outputs. This approach is crucial in fields like automation, robotics, and industrial control. In MCT, a system is viewed as an integrated whole composed of multiple interdependent components that are mutually dependent and interact with each other to perform specific functions. When defining a system, its integrity, dynamic interactions, feedback loops, hierarchy, and openness must be considered. In the realm of AutoPT, treating the target network, attackers, and defenders as an integrated system allows for a more comprehensive understanding and simulation of the security posture in cyberspace

% We leverage nonlinear system theory from modern control theory to enhance our comprehension and categorization of simulation modeling in AutoPT.
% 控制理论所要解决的确切主题是这种反馈回路的性质——控制决策对可观察输出的影响以及所揭示信息对后续控制操作选择的依赖性——最终目标是规定最佳控制操作,即以最低运营成本实现目标的控制操作。

Modern control theory represents a significant domain within automatic control, dedicated to the analysis and design of complex systems, encompassing multiple-input-multiple-output, nonlinear, and time-varying configurations. This theoretical framework utilizes state variables and state space equations to depict the dynamic behavior of systems, thereby elucidating their internal states. Such an approach conceptualizes a system as a unified entity, highlighting the interdependence of its components, feedback mechanisms, hierarchical structures, and open systems characteristics. In the context of AutoPT, this perspective facilitates a thorough understanding and simulation of cybersecurity dynamics by conceptualizing the target network, attackers, and defenders as an integrated system. Control theory quantifies the status of the system through assignment of a state and formally describe the progression of AutoPT. It adopts a state-based approach to define cost structures that balance security and availability. The \textit{information state} in control theory compresses attacker-defender information to a level sufficient for optimal decision-making, converting determining the optimal defense policy into a sequential optimization problem~\cite{ErikMiehling2019Control}.
% 控制理论通过量化系统状态,描述渗透测试过程的演变,并作为系统中攻击者和防御者决策的函数。在state-based approach下,可以定义cost structure (costs for states and actions) that captures the desired tradeoff between security and availability,体现安全性与可用性之间的权衡。控制论中information state的概念允许对攻击者防御者的信息压缩为足以做出最佳决策的程度,将策略指定转化为顺序优化问题。

% In MCT, a system is viewed as an integrated whole composed of multiple interdependent components that are mutually dependent and interact with each other to perform specific functions. When defining a system, its integrity, dynamic interactions, feedback loops, hierarchy, and openness must be considered. In the realm of AutoPT, treating the target network, attackers, and defenders as an integrated system allows for a more comprehensive understanding and simulation of the security posture in cyberspace. 

The complexity and variability of network activities, coupled with the multifactorial nature of penetration and defense strategies, means that identical actions on the same system can yield divergent outcomes, indicating that the penetration testing system is inherently dynamic. It implies that the same action combinations may produce different outputs, rendering linear differential equations inadequate for describing the system's behavior. The superposition principle is inapplicable, thus classifying such systems as a dynamic nonlinear system. Traditional research often models penetration testing as a discrete system based on the actions of attackers and defenders, treating these actions as discrete events. The system's output is influenced not only by its structural parameters but also by the inputs and initial conditions. 


% cyber security problem can be formulated as a dynamic game with asymmetric information where the underlying system is stochastic and dynamic.



% Traditional research often conceptualizes penetration testing as a discrete system that relies on the actions of attackers and defenders, treating these actions as isolated events. However, the outputs of the system are influenced not only by its structural parameters but also by the inputs and initial conditions, highlighting the need for a more comprehensive modeling approach that accounts for the system's inherent complexity. 




\subsection{AutoPT as a Dynamic Nonlinear System}
\subsubsection{System Components}
% 自动化渗透测试中的建模是不仅仅是对目标背景网络的抽象与建模,还包括网络中交互主体,即攻击者和防御者的建模。网络空间建模影响攻击者、防御者的目标、动作设定和决策,攻击者防御者的交互反过来也会影响目标背景网络的变化。因此,我们将渗透测试场景建模的关键要素分为两部分,一部分是目标网络环境,另一部分是攻击者、防御者的建模,二者是联系紧密、相互影响的。
% The modeling in AutoPT is not only the abstraction and modeling of the target network environment, but also includes the modeling of attackers and defenders. The modeling of environment affects the goals, action settings, and decisions of attackers and defenders, and the interaction between attackers and defenders in turn affects the changes in the environment. 

The modeling framework in AutoPT incorporates both the abstract representation of network environments and the simulation of adversarial interactions between attackers and defenders. Environmental modeling directly shapes the objectives, action spaces, and strategic decision-making processes of both agents. Concurrently, their adversarial interactions induce environmental modifications, establishing a dynamic feedback loop that subsequently shapes their adaptive strategic responses.

\begin{itemize}
    \item Network environment: The target network environment, encompassing its architecture and assets, serves as the foundational framework for attack-defense interactions, shaping the dynamics, realism, and reliability of simulated engagements. Accurate modeling of network architecture and assets is critical for achieving realistic simulations, as it captures the fluidity and complexity of real-world network conditions. This precision enables a more robust assessment of vulnerabilities and the efficacy of defensive measures.
    \item Attackers and defenders: AutoPT simulates adversarial tactics, techniques, and procedures to identify security vulnerabilities in target networks by modeling attacker and defender behaviors~\cite{chenke2023survey}. The attacker model requires explicit specification of objectives and permissible actions for simulated agents, while defender modeling accommodates greater flexibility, incorporating both static strategies (e.g., firewall policies, intrusion-detection system configurations~\cite{furfaro2017using}) and dynamic defense mechanisms~\cite{applebaum2016intelligent}. Observations for both parties are characterized by incomplete information and noise, including missed detections and false alarms~\cite{ErikMiehling2019Control}. We assume an information structure adhering to the \textit{perfect recall} assumption, wherein attackers and defenders retain complete historical records of past observations and decisions. Consequently, defenders at time $t$ maintain access to all prior historical data for decision-making.
\end{itemize}
% For the partially observable setting of attackers' and defenders', there are two different methods to modeling it~\cite{ErikMiehling2019Control}:(1) probabilistic uncertainty, and (2) nondeterministic uncertainty.
In partially observable environments involving attackers and defenders, two primary approaches exist for modeling their interactions~\cite{ErikMiehling2019Control}: probabilistic uncertainty and nondeterministic uncertainty.
% 防御者和攻击者的信息都是不完全观察的,其中还包含着误报和漏报等噪音
\subsubsection{System Dynamics}
% The randomness and dynamics in AutoPT games arise from the actions of attackers and defenders, network environment changes, and the occurrence of random events. We use a stochastic dynamic system with a time horizon T, which affected by the actions of the attacker and defender agents along with random events that occur in nature. This system is described by a stochastic difference equation. The system's state at time $t + 1$, $S_{t+1}$ is a function of its state at time $t$, the actions of the attacker $A^A$ and the defender $A^D$, and the activities in network environment $U^E$, including normal interaction activities $U^E_n$ and sudden random events $U^E_s$ in the network, $U^E = (U^E_n,U^E_s)$. 

In the AutoPT system, randomness and dynamic behavior arise from three interdependent factors: adversarial interactions between attackers and defenders, fluctuations in network environments, and stochastic events. System dynamism is thus inherently tied to these interdependent factors.
% We model this as a stochastic dynamic system over a time horizon \( T \), influenced by the actions of attacker and defender agents, as well as the changes in network. 
% The system is characterized by a stochastic difference equation. The state at time \( t+1 \), denoted \( S_{t+1} \), is a function of the state \( S_{t} \)at time \( t \), the actions of the attacker \( A^A \) and the defender \( A^D \), and the network environment activities \( U^E \), which include both normal interactions \( U^E_n \) and sudden random events \( U^E_s \), thus \( U^E = (U^E_n, U^E_s) \). 


\begin{itemize}
% 系统中的输入和输出指的是攻击者和防御者的动作决策,他们之间是相互依存的,动作的决策都受到对方的影响。
    \item Input-Output. Attacker and defender actions exhibit bidirectional coupling: their interdependent decisions act as both system inputs and state-dependent outputs, driving iterative state evolution.
    \item State Transitions. The state \( S(t+1)\) evolves stochastically based on \( S(t) \), concurrent actions \( A^A(t) \) (attacker) and \( A^D(t) \) (defender), and exogenous stochastic events $W(t)$.
    \item Feedback Mechanisms. The utility function \( U(t) = (U^A(t), U^D(t)) \) critically shapes system dynamics by providing post-action feedback to both agents. This function serves dual roles: predefined reward signals or emergent behavioral outputs.
% 反馈机制可以显著影响系统的动态行为。在AutoPT中,可以设置效用函数$U_t=(U^A_t,U^D_t)$代表攻击者和防御者执行动作后收到的反馈。基于反馈引导系统的行为。Utility function可以是定义奖励函数,也可以是攻击者和防御者的动作输出
\end{itemize}



\subsubsection{Formal expression}

\begin{figure*}[tb]
    \centering
    \includegraphics[width=150mm]{figure/Nonlinear.pdf}
    \caption{The Dynamic Nonlinear System Framework of AutoPT Simulation Modeling}
    \label{nonlinear}
\end{figure*}

%如图\ref{Nonlinear}中展示的自动化渗透测试仿真系统所示,攻击者和防御者的动作既可以是该系统的输入也可以是输出,在$t$时刻,系统的输入为$u(t) = [u^a(t),u^d(t)]^T$,其中,$u^a(t) = [u^a_1,u^a_2,\dots,u^a_p]$为攻击输入向量,$u^d(t) = [u^d_1,u^d_2,\dots,u^d_p]$为防御输入向量。系统的输出为$y(t) = [y^a(t),y^d(t)]^T$,其中$y^a(t) = [y^a_1,y^a_2,\dots,y^a_p]$为攻击输出向量,$y^d(t) =[y^d_1,y^d_2,\dots,y^d_p]$为防御输出向量。
%由目标网络、攻击者和防御者构成的系统在$t$时刻的状态向量可以表示为$x_(t) = [x^c(t),x^a(t),x^d(t)]^T,其中$x^c(t)=$代表目标网络的状态变量,$x^a(t)$和$x^d(t)$分别代表攻击者和防御者的状态变量,给定$t =t_0$ 时的初始状态向量$x(t_0)$及$t≥t_0$ 的输入向量$u(t)$,则$t≥t_0$的状态由状态向量$x(t)$唯一确定。
% x^c(t)代表目标网络的状态变量,也就是网络的变化
%x^a(t)代表攻击者的状态向量
% x^d(t)代表防御者的状态向量
% 每个都作为一个非线性系统的话
% 我们将自动化渗透测试仿真系统理解为一个离散时间系统,该系统的状态方程可以表示为:
% 

As illustrated in Figure~\ref{nonlinear}, the AutoPT system comprises a target network environment, adversarial agents (attackers/defenders), and a state vector $S(t) = [S^E(t), S^A(t), S^D(t)]$, where $S^E(t)$, $S^A(t)$ and $S^D(t)$ denote the network, attacker, and defender states, respectively.
% $w(t)$ 可能代表着网络环境中机器的故障或用户的错误操作等随机扰动因素,也可能包含着攻击者和防御者执行动作时的成功概率影响,导致在相同输入情况下,该系统仍可能输出不同的重要因素,也就是XXXX
Stochastic disturbances $W(t)$, representing exogenous factors (e.g., hardware failures, user errors, or action success probabilities), introduce uncertainty into state transitions. These disturbances induce non-deterministic critical outputs even under identical inputs. For $t \geq t_0$, the system’s trajectory depends on the initial state $S(t_0)$, input vector $A(t)$, and the probabilistic disturbance terms $W(t)$.
 
%  The actions of attackers and defenders can serve as both inputs and outputs of the system. At time $t$, the system's input is $A(t) = [A^A(t), A^D(t)]$, where $A^a(t) = [a^a_1, a^a_2, \dots, a^a_p]$ represents the input vectors of attacker's actions, and $A^D(t) = [a^d_1, a^d_2, \dots, a^d_q]$ represents the input vectors of defender's actions. The system's output is $Y(t) = [Y^A(t), Y^D(t)]$, where $Y^a(t) = [y^a_1, y^a_2, \dots, y^a_p]$ represents the output vectors of attacker's actions, and $Y^D(t) = [y^d_1, y^d_2, \dots, y^d_q]$ represents the output vectors of defender's actions. 
% The attacker's input-output vector dimensions typically differ from the defender's, denoted as $p\neq q$. However, these can be aligned by padding to a common dimension, which we assume to be $p'$, determined by $p' = \max(p, q)$
Attacker and defender actions exhibit dual input-output roles within the system. At time $t$, the input $A(t) = [A^A(t), A^D(t)]$ comprises the attacker’s action vector $A^a(t) = [a^a_1, a^a_2, \dots, a^a_p]$ and the defender’s action vector $A^D(t) = [a^d_1, a^d_2, \dots, a^d_q]$. The system output $Y(t) = [Y^A(t), Y^D(t)]$ mirrors this structure, with $Y^a(t) = [y^a_1, y^a_2, \dots, y^a_p]$ and $Y^D(t) = [y^d_1, y^d_2, \dots, y^d_q]$. While $p\neq q$ (distinct input-output dimensions for attackers and defenders), these can be standardized via zero-padding to a unified dimension $p'$, where $p' = \max(p, q)$, without loss of generality.
We therefore understand the AutoPT simulation system as a discrete-time system, and its state equation can be represented as a stochastic difference equation:
%  %攻击者的输入输出向量维度和防御者的向量维度一般是不相同的,即$p\neq q$,但是可以通过填充将其扩展到相同的维度,这里我们假设统一填充至$p'$维,其中
% \begin{equation}
% p'=\left\{
% \begin{array}{rcl}
% p & & {p \geq q}\\
% q & & {p  < q}
% \end{array} \right.
% \end{equation}


% The system's state vector at time $t$ is denoted as $x(t) = [x^c(t), x^a(t), x^d(t), w(t)]^T$. Here, $x^c(t)$ signifies the target network's state variables, $x^a(t)$ and $x^d(t)$ represent those of the attackers and defenders, respectively, and $w(t)$ is a stochastic disturbance vector affecting the system at time $t$. 
%  Given the initial state vector $x(t_0)$ at $t = t_0$ and the input vector $u(t)$ for $t \geq t_0$, the state for $t \geq t_0$ is not uniquely determined by the state vector $x(t)$, causing the stochastic disturbance terms.



\begin{equation}
\scriptsize
    % \label{x}
    \begin{aligned}
        S(t_{k+1})
        =&f\left [S(t_k),A(t_k),W(t_k),t_k\right ]\\
        =&f\left [\left [S^E(t_k),S^A(t_k),S^D(t_k)\right]^T,\left [A^A(t_k),A^D(t_k)\right]^T,W(t_k),t_k\right] \nonumber
    \end{aligned}
\end{equation}
% 该系统的输出方程形式为:

The system's output equation is given as follows:

\begin{equation}
\scriptsize
    % \label{y}
    \begin{aligned}
        Y(t_{k})
        =&g\left [X(t_k),A(t_k),W(t_k),t_k\right ]\\
        =&g\left [\left [S^E(t_k),S^A(t_k),S^D(t_k)\right]^T,\left [A^A(t_k),A^D(t_k)\right]^T,W(t_k),t_k\right] \nonumber
    \end{aligned}
\end{equation}
% $t_k$表示离散时间系统中的具体采样时刻,k作为下标表示在该时刻系统状态的索引。
% 因此,自动化渗透测试仿真系统的动态方程(或状态空间描述)为:
% where, $t_k$ denotes a specific sampling instant in the discrete-time system, and $k$ is the index of the system state at that instant. 
where, $t_k$ represents the $k$-th sampling instant within a discrete-time framework, and $k$ serves as the corresponding state index.

% Traditional research often views penetration testing as a discrete system driven by the actions of attackers and defenders, treating these actions as isolated events~\cite{applebaum2017analysis, miller2018automated, 2020Finding}. However, the system's outputs are influenced not only by its structural parameters but also by the associated inputs and initial conditions. This interdependence underscores the necessity for a more comprehensive modeling approach that accounts for the inherent complexity of the system. Such an approach should integrate multiple variables and their interactions to more accurately reflect the dynamics of penetration testing environments.
Traditional penetration testing research often frames the process as a discrete system driven by isolated attacker and defender actions, modeled as discrete, isolated events~\cite{applebaum2017analysis, miller2018automated, 2020Finding}. However, system outputs are contingent upon structural parameters, system inputs, and initial conditions. This interdependence necessitates a holistic framework that accounts for inherent feedback dynamics, integrating interdependent variables to accurately capture penetration testing environments.
% Thus, the dynamic equation (or state-space representation) of the AutoPT simulation system is Equation (~\ref{xy}).

% \begin{equation}
%     \label{xy}
%     \begin{aligned}
%         \begin{cases}
%         S(t_{k+1})&=f\left [S(t_k),A(t_k),W(t_k),t_k\right ]\\
%         y(t_{k})&=g\left [X(t_k),A(t_k),W(t_k),t_k\right ]\\ \textbf{}
%         \end{cases}
%     \end{aligned}
% \end{equation}
% 自动化渗透测试仿真系统的状态空间和输出受到系统本身状态、输入和时间的影响,系统的输出同样受到系统本身状态、输入和时间的影响,这与现实生活中,网络环境由于用户活动或本身设置,在不同时间下发生变化;同时攻击者和防御者的交互引起的变化是相吻合的。网络空间建模影响着攻击者和防御者的目标设定、动作设定和决策,而攻击者和防御者的互动反过来也会影响目标网络空间的变化,这种交互不仅影响系统的当前状态,也影响系统的未来演化。因此,我们将渗透测试场景建模的关键要素分为两部分,一部分是网络场景的抽象——网络架构与目标资产,另一部分是攻击者、防御者的建模,二者是联系紧密、相互影响的。

% In existing research, penetration testing is often modeled as a discrete system based on the actions of attackers and defenders. The system's output depends on structural parameters, inputs, initial conditions, and the stochastic disturbance. Cyberspace modeling impacts the goals, actions, and decisions of attackers and defenders, and their interactions affect the target cyberspace's changes, influencing both the current state and future evolution of the system. 

\subsection{Core Modeling Elements}
% 
In this section, we will introduce every core element in detail.

\subsubsection{Network Architecture and Target Assets}
% Network Architecture and Target Assets是非线性系统的状态空间,
% Target network environment includes network architecture and target asset.
% To start with, the
% Network architecture refers to the structure and configuration of an organization's network systems, which underpin the design of attack vectors and defense strategies in penetration testing. It can be broken down into two levels:

The target network environment involves both the network architecture and target assets. Primarily, network architecture refers to the structural design and configuration of an organization's network systems, which are fundamental in shaping the design of attack vectors and defense strategies during penetration testing. This architecture can be delineated into two levels:

\begin{itemize}
    \item \textbf{Physical Connections:} Physical media, comprising guided media such as cables and optical fibers, and unguided media like wireless signals, facilitate data transmission between network devices. These connections are supported by infrastructure components such as routers, switches, fiber optic systems, network cables, and Bluetooth technology. 

    \item \textbf{Logical Topology:} Logical topology describes the structural relationships and data flow between network devices, such as routers and switches, irrespective of their physical locations or connection methods. It focuses on the interactions and communication pathways within the network, illustrating the interconnection patterns of network assets. Configurations may include star, ring, tree, or hybrid topologies~\cite{Brede2012NetworksAnIM}.
\end{itemize}

% Logically, network architecture includes elements like LANs, WANs, VPNs, VLAN segmentation, and subnet configurations, along with communication and security protocols such as TCP/IP, HTTP, FTP, and SSL/TLS, as well as firewall rules \cite{furfaro2017using}.

% \subsection{网络架构与目标资产}
% 网络架构与目标资产共同构建了攻防交互的网络背景,其构建方式影响攻防交互的模式、模型与现实生活的逼近程度、交互的真实性与可靠性。

% \textbf{网络架构:}指组织内部或与外部连接的网络系统的结构和配置。它包括物理和逻辑上的连接和通信规则,是渗透测试中攻击路径设计和防御策略制定的依据。网络架构可以分为两个层次:点与点之间的连接(通过路由器、交换机、光纤、网线或蓝牙等进行有线、无线的连接)、网络整体拓扑(网络中所有资产形成的整体连接关系,如星型、环型、树型、混合型等布局)。网络架构在逻辑层面上通过局域网(LAN)、广域网(WAN)、虚拟专用网(VPN)、VLAN划分和子网配置、通信与安全协议(TCP\/IP、HTTP、FTP、SSL\/TLS等)、防火墙规则等实现。

% \textbf{目标资产:}是渗透测试对象中的所有资源,包括硬件、软件、数据和与之相关的人员,这些资源是攻击者试图访问、控制或破坏的主要对象。在物理层面理解为硬件设备,包括服务器、工作站、电脑主机、笔记本电脑、手机平板等移动设备和U盘硬盘等外部存储设备等,其内部通常隐含有操作系统、软件、开放服务及应用程序、数据库、账户信息等资产以及路由器、交换机、防火墙、安全设备(如IDS\/IPS设备)等设备,控制着网络整体的准入规则和防御水平。

% 网络架构与目标资产可以建模为静态的,也可以建模为动态的,在考虑动态建模时,除了攻击者和防御者的动作可能引起的目标场景变化外,也可以考虑模拟现实世界的情况,如企业上下班导致的硬件设备上下线情况、军事任务演练或任务集中攻关中出现的某一时间段内的流量异常情况、动态防御网络策略的设置导致的网络主动变化情况等。

Conversely, target assets represent all resources within penetration testing, including hardware, software, data, and personnel. These assets represent the primary objectives that attackers seek to access, manipulate, or compromise, including:

\begin{itemize}
    \item \textbf{Physical Resources:} Servers, workstations, desktops, laptops, mobile devices, external storage, network infrastructure, and security appliances.
    \item \textbf{Virtual Resources:} Operating systems, software applications, open services, databases, and account credentials.
\end{itemize}

The configuration and security posture of these assets determine their value, access policies, and defensive capabilities~\cite{Guo2018Cyberspace}.
Moreover, the modeling of network architecture and target assets can be classified as either static or dynamic. Static modeling assumes that network architecture and configurations remain constant throughout the simulation, whereas dynamic modeling accommodates changes in the target environment due to actions by attackers or defenders, thereby replicating real-world conditions such as:

\begin{itemize}
    \item Variations in the status of enterprise hardware due to employee commuting.
    % \item Atypical traffic patterns during military exercises or periods of high operational activity.
    \item Proactive network changes resulting from dynamic defense strategies like Cyber Mimic Defense~\cite{2016Research} or Moving Target Defense~\cite{2011Moving}.
    \item Randomness and uncertainty in network scenarios, such as the unpredictable failure of network devices and the non-deterministic outcomes of attack actions.
\end{itemize}

% \subsection{攻击者、防御者模型}
% 渗透测试是通过模拟黑客的行为在目标网络中进行渗透发现网络中隐藏安全隐患,因此测试人员一般模拟的就是攻击者角色,其建模一般是显性的,需要明确定义攻击者目标和能够采取的动作,而防御者的建模则比较灵活,可以对防御者进行显性建模,提供主动防御的措施,也可以通过在网络中设置防火墙规则、入侵检测设备或为攻击动作设置成功率等其他防御规则隐性的建模防御者。

% \subsubsection{攻击者建模}
% 根据现实生活中真实的渗透测试场景设定,研究中的自动化渗透测试攻击者对目标网络分别有黑盒、灰盒、白盒测试三种可能。白盒测试\cite{Midian,al2018study,filiol2021method,shravan2014penetration}指的是测试者拥有系统的完全访问权限,包括源代码、架构设计、网络布局等内部信息,更注重于检查系统内部的安全控制措施是否有效。黑盒测试\cite{Midian,awang2013detecting,goel2015vulnerability,al2018study,filiol2021method}指的是测试者对被测试的系统了解非常有限,通常只拥有系统的网络访问权限,测试者不知道系统的内部结构、设计和实现细节,模拟了外部黑客的攻击行为,主要目的是找出系统的安全漏洞,而不关注内部逻辑。灰盒测试\cite{goel2015vulnerability,filiol2021method,demott2007revolutionizing}则介于黑盒和白盒测试之间,测试者拥有部分系统信息,比如网络架构或者某些配置信息,但不包括完整的源代码,既考虑了外部攻击者的视角,也利用了部分内部信息来指导测试。

% \textbf{身份:}
% 可以假设是个人或具有国家和民族资助的组织,如脚本小子、黑客组织,这影响了攻击者能够采用的攻击手段和载荷资源,当然这在渗透测试中并不是一定要显式地设计。

% \textbf{目标:}
% 根据身份的不同或渗透测试需求的不同,攻击者通常具有不同的攻击目标,影响攻击者在目标网络中的活动范围与动作选择。
% 如若攻击目标是以最快速度到达核心敏感区域,那么攻击者就需要以最短的攻击路径及最有效的攻击技术,采用尽量少的步骤到达敏感区域;若攻击者的目标是最大可能发现网络中隐藏的多有安全隐患,就要求攻击者采样多种的攻击手段,对网络进行最大范围的攻击;若攻击者的目标是长期潜伏,那么就要求攻击者采用隐蔽程度高的手段,在目标网络中缓慢而隐蔽地移动,并注意清楚自己的移动痕迹。
% \subsection{Attacker and Defender Models}
% Penetration testing involves simulating the behavior of hackers to penetrate target networks and uncover hidden security risks. Consequently, testers typically assume the role of an attacker, whose modeling is usually explicit and requires a clear definition of the attacker's goals and the actions they can take. In contrast, the modeling of the defender is more flexible and can be explicit, providing active defense measures\cite{applebaum2016intelligent}, or implicit by setting firewall rules, intrusion detection systems, or defining success rates for attack actions as part of the network's defense mechanisms\cite{furfaro2017using}.

% \subsubsection{Attacker Modeling}

% Based on real-world penetration testing scenarios, the AutoPT attacker in research can be categorized into black box, grey box, and white box testing. White box testing\cite{Midian,al2018study,filiol2021method,shravan2014penetration} refers to testers having full access to the system, including source code, architectural design, and network layout, focusing on the effectiveness of internal security controls. Black box testing\cite{Midian,awang2013detecting,goel2015vulnerability,al2018study,filiol2021method} involves testers having limited knowledge of the system, typically only network access, without knowledge of the internal structure, design, and implementation details, simulating external hacker attacks to identify security vulnerabilities without concern for internal logic. Grey box testing\cite{goel2015vulnerability,filiol2021method,demott2007revolutionizing} falls between black box and white box testing, where testers have partial system information, such as network architecture or certain configuration details, but not complete source code, considering both external attacker perspectives and partial internal information to guide the testing.
\subsubsection{Attacker Models}

% AutoPT simulates the tactics, techniques, and procedures attackers use to penetrate target networks and uncover hidden security vulnerabilities \cite{chenke2023survey}. It requires a clear model for testers acting as attackers, outlining their objectives and permissible actions. Defender modeling, on the other hand, can be more flexible, encompassing explicit static defense strategies, such as firewall policies and intrusion detection system configurations\cite{furfaro2017using}, and dynamic measures\cite{applebaum2016intelligent}.

% AutoPT's attacker model mirrors real-world penetration testing approaches, detailing identities, targets, and other elements to shape a plan, allocate resources, and choose attack methods.

The attacker model in AutoPT closely resembles real-world penetration testing methodologies, explicitly delineating identities, targets, and other relevant factors to inform strategic planning, resource allocation, and the selection of attack methods.


% categorized into three types: black box, grey box, and white box testing.

% \textbf{White Box Testing:} Testers possess full system knowledge, including source code, architectural designs, and network topologies, to assess the effectiveness of internal security measures \cite{Midian, al2018study, filiol2021method, shravan2014penetration}.

% \textbf{Black Box Testing:} Testers have minimal knowledge, equivalent to network access without insight into internal structures. This method simulates external hacker attacks to identify security flaws without internal information \cite{Midian, awang2013detecting, goel2015vulnerability, al2018study, filiol2021method}.

% \textbf{Grey Box Testing:} Testers have partial system knowledge, such as network architecture or specific configurations, but lack full source code access. This approach combines the external attacker's perspective with some internal information to guide testing \cite{goel2015vulnerability, filiol2021method, demott2007revolutionizing}.


%除以上三种整体模拟方式之外,每一个攻击者还可以设定具体的身份、目标等信息,用于补全故事背景、限制可用资源、明确攻击手段等。
\begin{itemize}
    \item \textbf{Identity:} The identity of an attacker may vary from an individual, such as a script kiddie, to a collective, such as a state-sponsored hacking group. This identity could render varying attack methods and available resources. 
    However, in practical testing scenarios, specifying this characteristic is often optional.

    \item \textbf{Objectives:} An attacker’s objectives, influenced by their identity and the specific testing requirements, shape their actions within the network.  Rapid access to sensitive areas prompts the use of direct paths and potent techniques~\cite{hu2020automated}.
    % Comprehensive vulnerability discovery involves diverse attack methods across the network~\cite{applebaum2017analysis}. For long-term covert presence, discreet tactics akin to Advanced Persistent Threats are used to navigate stealthily and erase intrusion evidence~\cite{2014Advanced}.
   If the objective is to achieve rapid access to sensitive areas, the attacker must employ the shortest attack path and most effective techniques to minimize the steps required. In contrast, if the objective is to identify numerous hidden security vulnerabilities within the network, the attacker should utilize diverse attack methods and carry out a broad spectrum of attacks. If the objective is to maintain a prolonged covert presence, the attacker must implement high-concealment strategies, such as Advanced Persistent Threat, advancing stealthily within the network while meticulously eliminating traces of their movements.
    
    \item \textbf{Actions:} Attacker could exploit vulnerabilities, escalate privileges, conduct scans, and facilitate lateral movement. The Cyber Kill Chain model, developed by Lockheed Martin, offers a systematic approach to penetration testing~\cite{hutchins2011intelligence}. Meanwhile, the MITRE ATT\&CK framework~\cite{2018MITRE} catalogs relevant TTPs (Techniques, Tactics, and Procedures) for AutoPT research. 
    Due to the impracticality of incorporating all TTPs, researchers typically abstract an attack action library based on network architecture, target assets, and attacker-defender models. Each action is defined with specific preconditions, execution steps, and outcomes to optimize the penetration process.
    
    \item \textbf{Network Visibility:} The type of penetration testing—black box, grey box, and white box—influences network visibility. 
    During white box testing, the tester has full visibility and comprehensive knowledge of the system, including access to source code, architectural designs, and network topologies, enabling an in-depth evaluation of internal security measures~\cite{Midian, al2018study, filiol2021method, shravan2014penetration}. 
    Grey box testing involves partial knowledge and access to internal data structures, log files, and application APIs, enhancing testing effectiveness~\cite{goel2015vulnerability, filiol2021method, demott2007revolutionizing}. Conversely, black box testing restricts testers to external observations, simulating real-world attack scenarios~\cite{Midian, awang2013detecting, goel2015vulnerability, al2018study, filiol2021method}. 

\end{itemize}

% 网络可见性
% 根据black box, grey box, and white box testing三种不同的渗透测试类型,攻击者对网络的可见性也是不同的,White Box Testing中Testers possess full system knowledge, including source code, architectural designs, and network topologies, to assess the effectiveness of internal security measures \cite{Midian, al2018study, filiol2021method, shravan2014penetration},整个网络对智能体是完全可见的,目标明确,无需进行目标范围定义和信息收集,目标明确,无需进行目标范围定义和信息收集。在灰盒测试中,\textbf{White Box Testing:} Testers possess full system knowledge, including source code, architectural designs, and network topologies, to assess the effectiveness of internal security measures \cite{Midian, al2018study, filiol2021method, shravan2014penetration}.测试人员通常能够获取到一些关键信息,包括系统的架构和特定的业务逻辑,能够访问内部数据结构、日志文件以及应用程序的API,从而帮助测试人员更有效地进行测试。黑盒测试中,测试人员对目标网络的内部结构和所使用的程序完全不了解,不能直接访问目标系统的内部信息,侧重于从外部观察系统的功能和行为,是完全模拟网络黑客的攻击行为。



%攻击者的建模尤其是目标及动作的建模,是自动化渗透测试中非常重要的内容。目标的设置影响了智能体对动作选择的倾向性,也影响了智能决策中评估内容的设置,如强化学习中奖励函数的设计。动作的建模既需要贴近现实影响,最好是能够涵盖渗透测试的全流程,同时又需要具有一定程度的抽象,否则会导致训练过程难度加大,难以生成有效策略。
% In AutoPT, modeling of attackers, particularly their objective and actions, is crucial. Objective settings influence agents' tactics choices and the design of evaluation metrics, such as reward functions in reinforcement learning. 
% Actions modeling should reflect the differences between different attackers' capabilities, realistic impacts, while maintaining a level of abstraction, otherwise it may lead to increased difficulty in the training process and make it difficult to generate effective strategies.

In AutoPT, the modeling of attackers, particularly their objectives and actions, is essential. The specification of objectives influences agents' tactical decisions and the formulation of evaluation metrics, such as reward functions. Meanwhile, the modeling of actions must capture the variations in attackers' capabilities and realistic impacts while preserving an appropriate level of abstraction. Failure to achieve this balance can complicate the training process and hinder the development of effective strategies.


% \textbf{攻击手段和战术:}
% 攻击手段和战术指的是在渗透测试过程中,攻击者可以采取地移动手段,如漏洞利用、权限提升、扫描、横向移动等,在现实生活中,美国洛克希德马丁公司提出的网络杀伤链模型\cite{hutchins2011intelligence}总结了渗透测试地相关阶段,Mitre公司提出的ATT\&CK框架\cite{2018MITRE}则详细的总结了渗透测试中TTPs(攻击战术、技术和程序),为自动化渗透测试的研究提供了一个有利的参考,但是将所有的TTPs都放入到攻击决策中显然是不现实的。在科学研究中,,通常会根据网络架构与目标资产的建模方式、攻击者和防御者建模,基于现实情况的考虑,为攻击者抽象地设定一个可采用的动作库,并为动作库中的每一个动作设定前置条件、动作内容和达成效果,以完成整体渗透过程。

% \subsubsection{防御者建模}

% 防御者通常代表的是目标网络中安全防御部分人员,对防御者而言目标网络是可见的,但是攻击者是不可见的,换句话说,防御者可以可以主动监控网络的状态,但是并不知道哪些变化是由于攻击者的动作引起的。他们可以通过使用局域网、VLAN、防火墙等控制设备之间的连接关系,在网络中安装入侵检测系统、检查日志发现网络中可能存在的异常行为,也可以通过安装杀毒软件、更新系统及软件版本为可能存在的漏洞及时打上补丁,甚至可以通过讲座、安全教育等提高人员的防护意识,来达到防护的目标。

% \textbf{隐式防御者建模:}在隐式地对防御者进行建模时,通常可以通过设置硬件设备之间不同的连接关系代替防火墙的准入规则,设置攻击者动作的成功率代表硬件设备中可能存在的杀毒软件等防御软件,还可以通过定时的侦察发现等规则代表网络中可能出现的防御设备与入侵检测系统,这些防御手段通常是静态的或者预设的,保持不变或者只在达到某一触发条件时发生变化。

% \textbf{显式防御者建模:}在显式的对防御者进行建模时,则跟对攻击者建模相同,需要对防御者也定义一个动作库,为每个动作点定义前置条件、动作内容和达成效果,此时防御者的动作可以根据网络的情况发生改变,主动提高某一设备或网络整体的防御水平、切断设备之间的通信关系,调整局域网信息,达到模拟主动防御的目的。


% 在完成网络架构与目标资产、攻击者与防御者建模以后,可以根据现实背景,将渗透测试建模为攻击者和防御者之间的交互,或攻击者单方面的攻击行为。根据不同动作之间预先定义的逻辑关系推动目标网络状态变化,同时目标网络也可能会由于普通用户的活动产生资产与拓扑的变化,这些变化共同影响下一时刻动作的选择,随着时间推移,达到完成整个渗透测试模拟的过程。

\subsubsection{Defender Models}
Defenders are the security personnel responsible for safeguarding the network. 
They manage network connections through LANs, VLANs, firewalls, and various security measures~\cite{terranova2024leveraging, furfaro2017using}. Their responsibilities include deploying intrusion detection systems, analyzing logs for anomalies~\cite{ghanem2022towards}, implementing antivirus software, and performing system updates to mitigate vulnerabilities. Additionally, they conduct security awareness programs through lectures and training to achieve protective objectives~\cite{dillon2022perihack}.

Despite their comprehensive visibility of the network, defenders typically lack awareness of the attacker's presence and actions. While they actively monitor network conditions, they may not immediately correlate changes with potential attacks. Based on the level of proactive measures employed, defenders can be categorized into two distinct types:


 % Network connections are managed through LANs, VLANs, firewalls, and other security measures \cite{terranova2024leveraging, furfaro2017using}. Key responsibilities include deploying intrusion detection systems, analyzing logs for anomalies \cite{ghanem2022towards}, implementing antivirus solutions, and performing system updates to address vulnerabilities. Additionally, defenders conduct security awareness programs through lectures and training to fulfill protective objectives \cite{dillon2022perihack}. Based on the detail and specificity of defender agents and their actions, defender modeling in AutoPT can be categorized into two distinct types:


\begin{itemize}
    \item \textbf{Static Defense:} Static defense is a predefined cybersecurity strategy that operates without adapting to evolving attack policy. This approach is characterized by its fixed nature, activating automatically under specific conditions in response to detected attacker activity or changes in network status. 
    Examples include implementing firewall rules via established logical connections between devices~\cite{schwartz2019autonomous} and ensuring intrusion detection mechanisms via periodic reconnaissance. 
    Within static defense, the success rates of attacker actions can represent defender ability.
    Despite its limitations in immediate response capabilities to new threats and attack techniques, static defense is considered an important auxiliary security measure. Its simplicity and reliability make it a popular choice for AutoPT, as much of the research in this area does not account for defender dynamics.

    \item \textbf{Dynamic Defense:} Dynamic defense involves the definition of  defense agents and an action library, encompassing preconditions, actions, and expected outcomes to enable real-time adjustments to defensive measures based on network conditions. This proactive strategy enhances security by disrupting communications and modifying network information to simulate an active defense~\cite{paul2019learning,elderman2017adversarial}. Furthermore, it complements active defense through methodologies such as Zero Trust Networks, Moving Target Defense, and honeypots. The flexibility inherent in dynamic defense modeling is essential for addressing the evolving landscape of cyber threats, ensuring effective protection by considering both attacker behavior and changes in network environments.

\end{itemize}

Upon establishing the core elements, penetration testing can be understood as a dynamic interaction between attackers and defenders or as a series of unidirectional offensive maneuvers by attackers. 
The target network's state evolves based on predefined logical relationships that dictate the outcomes of various actions. Furthermore, routine user activities may also alter the network's assets and topology~\cite{applebaum2017analysis}. These changes collectively influence the decision-making for subsequent attack and defense actions.  The penetration test is conducted through the iterative application of these strategies throughout the simulation.
Our modeling classification method is constructed based on three perspectives: the overall aim of the research literature, the key elements presented in this section, and the interactions between these elements.
We will first present the overall contributions of the paper, followed by an analysis on the core modeling factors.

% Our classification model is also based on the core elements and the nonlinear system mentioned above for analysis and extraction. We focus first on the overall contribution of the paper and then analyze it based on two core modeling factors.



%我们的分类模型同样基于上面的核心要素进行分析与提取,我们首先关注文章的整体贡献,而后基于两个核心建模要素进行分析。

%根据图\ref{Nonlinear}中展示的自动化渗透测试仿真系统,攻击者和防御者的动作既可以是该系统的输入也可以是输出,在$t$时刻,系统的输入为$u(t) = [u^a(t),u^d(t)]^T$,其中,$u^a(t) = [u^a_1,u^a_2,\dots,u^a_p]$为攻击者动作作为输入向量时的表示,$u^d(t) = [u^d_1,u^d_2,\dots,u^d_p]$为防御者动作作为输入向量的表示。系统的输出为$y(t) = [y^a(t),y^d(t)]^T$,其中$y^a(t) = [y^a_1,y^a_2,\dots,y^a_p]$为攻击者动作作为输出向量时的表示,$y^d(t) =[y^d_1,y^d_2,\dots,y^d_p]$为防御者动作作为输出向量时的表示。
%由目标网络、攻击者和防御者构成的系统在$t$时刻的状态向量可以表示为$x_(t) = [x^c(t),x^a(t),x^d(t)]^T,其中$x^c(t)=$代表目标网络的状态变量,$x^a(t)$和$x^d(t)$分别代表攻击者和防御者的状态变量,给定$t =t_0$ 时的初始状态向量$x(t_0)$及$t≥t_0$ 的输入向量$u(t)$,则$t≥t_0$的状态由状态向量$x(t)$唯一确定。
% 我们将自动化渗透测试仿真系统理解为一个离散时间系统,该系统的状态方程可以表示为:
% \begin{equation}
%     \begin{aligned}
%         x(t_{k+1})=&[x^c(t_{k+1}),x^a(t_{k+1}),x^d(t_{k+1})]^T\\
%         =&f\left [x(t_k),u(t_k),t_k\right ]\\
%         =&f\left [\left [x^c(t_k),x^a(t_k),x^d(t_k)\right]^T,\left [u^a(t_k),u^d(t_k)\right]^T,t_k\right]
%     \end{aligned}
% \end{equation}
% 该系统的输出方程形式为:
% \begin{equation}
%     \begin{aligned}
%         y(t_{k})=&[y^a(t_k),y^d(t_k)]^T\\
%         =&g\left [x(t_k),u(t_k),t_k\right ]\\
%         =&g\left [\left [x^c(t_k),x^a(t_k),x^d(t_k)\right]^T,\left [u^a(t_k),u^d(t_k)\right]^T,t_k\right]
%     \end{aligned}
% \end{equation}
% $t_k$表示离散时间系统中的具体采样时刻,k作为下标表示在该时刻系统状态的索引。
% 因此,自动化渗透测试仿真系统的动态方程(或状态空间描述)为:
% \begin{equation}
%     \begin{aligned}
%         \begin{cases}
%         x(t_{k+1})&=f\left [x(t_k),u(t_k),t_k\right ]\\
%         &=f\left [\left [x^c(t_k),x^a(t_k),x^d(t_k)\right]^T,\left [u^a(t_k),u^d(t_k)\right]^T,t_k\right]\\
%         y(t_{k})&=g\left [x(t_k),u(t_k),t_k\right ]\\
%         &=g\left [\left [x^c(t_k),x^a(t_k),x^d(t_k)\right]^T,\left [u^a(t_k),u^d(t_k)\right]^T,t_k\right]\\
%         \end{cases}
%     \end{aligned}
% \end{equation}
% 自动化渗透测试仿真系统的状态空间和输出受到系统本身状态、输入和时间的影响,系统的输出同样受到系统本身状态、输入和时间的影响,这与现实生活中,网络环境由于用户活动或本身设置,在不同时间下发生变化;同时攻击者和防御者的交互引起的变化是相吻合的。网络空间建模影响着攻击者和防御者的目标设定、动作设定和决策,而攻击者和防御者的互动反过来也会影响目标网络空间的变化,这种交互不仅影响系统的当前状态,也影响系统的未来演化。
% \begin{equation}
%     \begin{aligned}
%         \begin{cases}
%         x(t_{k+1})
%         &=f\left [\left [x^c(t_k),x^a(t_k),x^d(t_k)\right]^T,\\
%         &\left [u^a(t_k),u^d(t_k)\right]^T,t_k \right ]\\
        
%         y(t_{k})
%         &=f\left [\left [x^c(t_k),x^a(t_k),x^d(t_k)\right]^T, \\
%         &\left [u^a(t_k),u^d(t_k)\right]^T,t_k \right ] \\
%         \end{cases}
%     \end{aligned}
% \end{equation}

