\begin{table*}
\centering
\caption{Defender Action Modeling}
\label{tab:defender_actions}
\resizebox{\linewidth}{!}{%
\begin{tabular}{p{2.5cm}|p{2.5cm}|p{4cm}|p{1cm}|p{1cm}|p{6cm}} % Adjust the width of the p column as needed
\toprule[1.5pt]
Action &  Decision Parameters & Expected Outcomes & Time Cost& Network Changed & Note \\ \midrule[1.2pt]
Patch Vulnerability &  Target IP & Patch a specific vulnerability &1 & Yes & One vulnerability of the target IP is randomly patched at a time. This invalidates sessions established through that vulnerability, causing the attacker to lose control of the host. \\ \hline
Traffic Monitoring &  Target IP & Monitor node traffic and invalidate suspicious sessions & 2 &No & Attackers may cause suspicious traffic changes when using information leakage. \\ \hline
Detect Attack &  Target IP & If the detected target IP is n=2 time steps behind the attack's vulnerability exploitation, the foothold is invalidated &1 & No & \\ \hline
Proactively Take Host Offline &  Target IP & Take a specific host offline & 1 &Yes & Target IP will come back online after a five time steps interval. Upon reconnection, the attacker will lose control of the host, and all established sessions will be terminated.\\ \hline
IP Blacklisting  &  Target IP & Randomly disconnect one connection of the target IP node & 1 &Yes & \\ \hline
Clear/Add Active Credentials &  Target IP, Clear/Add Active Credentials & Clear/Add active credentials for the target IP & 1 &Yes & \\ \hline
Honeypot&  Target IP& Configure the target IP as a honeypot; an alarm message will trigger upon a successful attacker penetration of the node.& 1 & No &One of the conditions for penetration failure can be defined as an attacker successfully infiltrating a honeypot.\\ \hline
Countermeasure&Target IP& Obtain the attacker's IP and related information & 2 & No &Upon honeypot infiltration, the defender implements countermeasures to pinpoint the attacker's IP address andrelated information, which can be designed as an ending condition.\\ \hline
Network security training& None& Randomly reduce the success rate of attack methods such as phishing emails, weak passwords, and credential login. & 10&No & The degree of success rate reduction varies for each node.\\ \bottomrule[1.5pt]
\end{tabular}}
\end{table*}
