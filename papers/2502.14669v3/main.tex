\documentclass[twocolumn]{article} % Or use 'IEEEtran' for IEEE format, etc.

\usepackage{multicol}  % For two-column support
\usepackage[letterpaper, top=0.6in, bottom=0.5in, inner=0.75in, outer=0.75in, headsep=0.3in, footskip=0.3in]{geometry}

\usepackage{titlesec}
\titleformat{\section}{\large\bfseries}{\thesection}{1em}{}
\titleformat{\subsection}{\normalsize\bfseries}{\thesubsection}{1em}{}

% --- Packages ---
\usepackage{graphicx}         % For including images
\usepackage{amsmath, amssymb} % For math symbols and equations (even if you don't use much, good practice)
\usepackage{hyperref}          % For hyperlinks (URLs, references)
\usepackage{natbib}          % For bibliography management (BibTeX style)
\usepackage{geometry}         % For adjusting margins if needed
\usepackage{listings}
\usepackage{pgfplots}
\usepackage{amssymb} % For \checkmark
\usepackage{pifont}  % For \xmark
\usepgfplotslibrary{fillbetween}
\usepackage{algorithm}
\usepackage{algorithmic}
\usepackage{amsmath} 
\usepackage{appendix}



\lstset{
    basicstyle=\ttfamily\small,  % Monospaced font, smaller size
    breaklines=true,             % Wrap long lines automatically
    frame=single,                % Adds a border around the code block
    showstringspaces=false       % Prevents spaces from being visible
}
% --- Optional: Load custom style file if you create one ---
% \usepackage{latex_style/custom_style}

% --- Metadata ---
\title{\textbf{AlphaMaze: Enhancing Large Language Models' Spatial Intelligence via GRPO}}


\author{
    Alan Dao (Gia Tuan Dao)\textsuperscript{1}, Dinh Bach Vu\textsuperscript{1} \\
    Menlo Research \\
    \texttt{alan@menlo.ai, bach@menlo.ai} \\
    \textsuperscript{1}Equal contribution.
}


\date{February 20, 2025} % Or specify a date, or use \date{} for no date

% --- Start of Document ---
\begin{document}

\maketitle
\begin{abstract}
 % Prevents indenting the abstract paragraph
\noindent Large Language Models (LLMs) have demonstrated impressive capabilities in language processing, yet they often struggle with tasks requiring genuine visual spatial reasoning. In this paper, we introduce a novel two-stage training framework designed to equip standard LLMs with visual reasoning abilities for maze navigation. First, we leverage Supervised Fine-Tuning (SFT) on a curated dataset of tokenized maze representations to teach the model to predict step-by-step movement commands. Next, we apply Group Relative Policy Optimization (GRPO)—a technique used in DeepSeek-R1—with a carefully crafted reward function to refine the model’s sequential decision-making and encourage emergent chain-of-thought behaviors. Experimental results on synthetically generated mazes show that while a baseline model fails to navigate the maze, the SFT-trained model achieves 86\% accuracy, and further GRPO fine-tuning boosts accuracy to 93\%. Qualitative analyses reveal that GRPO fosters more robust and self-corrective reasoning, highlighting the potential of our approach to bridge the gap between language models and visual spatial tasks. These findings offer promising implications for applications in robotics, autonomous navigation, and other domains that require integrated visual and sequential reasoning.
\end{abstract}

% --- Include Sections ---
\section{Introduction}

Video generation has garnered significant attention owing to its transformative potential across a wide range of applications, such media content creation~\citep{polyak2024movie}, advertising~\citep{zhang2024virbo,bacher2021advert}, video games~\citep{yang2024playable,valevski2024diffusion, oasis2024}, and world model simulators~\citep{ha2018world, videoworldsimulators2024, agarwal2025cosmos}. Benefiting from advanced generative algorithms~\citep{goodfellow2014generative, ho2020denoising, liu2023flow, lipman2023flow}, scalable model architectures~\citep{vaswani2017attention, peebles2023scalable}, vast amounts of internet-sourced data~\citep{chen2024panda, nan2024openvid, ju2024miradata}, and ongoing expansion of computing capabilities~\citep{nvidia2022h100, nvidia2023dgxgh200, nvidia2024h200nvl}, remarkable advancements have been achieved in the field of video generation~\citep{ho2022video, ho2022imagen, singer2023makeavideo, blattmann2023align, videoworldsimulators2024, kuaishou2024klingai, yang2024cogvideox, jin2024pyramidal, polyak2024movie, kong2024hunyuanvideo, ji2024prompt}.


In this work, we present \textbf{\ours}, a family of rectified flow~\citep{lipman2023flow, liu2023flow} transformer models designed for joint image and video generation, establishing a pathway toward industry-grade performance. This report centers on four key components: data curation, model architecture design, flow formulation, and training infrastructure optimization—each rigorously refined to meet the demands of high-quality, large-scale video generation.


\begin{figure}[ht]
    \centering
    \begin{subfigure}[b]{0.82\linewidth}
        \centering
        \includegraphics[width=\linewidth]{figures/t2i_1024.pdf}
        \caption{Text-to-Image Samples}\label{fig:main-demo-t2i}
    \end{subfigure}
    \vfill
    \begin{subfigure}[b]{0.82\linewidth}
        \centering
        \includegraphics[width=\linewidth]{figures/t2v_samples.pdf}
        \caption{Text-to-Video Samples}\label{fig:main-demo-t2v}
    \end{subfigure}
\caption{\textbf{Generated samples from \ours.} Key components are highlighted in \textcolor{red}{\textbf{RED}}.}\label{fig:main-demo}
\end{figure}


First, we present a comprehensive data processing pipeline designed to construct large-scale, high-quality image and video-text datasets. The pipeline integrates multiple advanced techniques, including video and image filtering based on aesthetic scores, OCR-driven content analysis, and subjective evaluations, to ensure exceptional visual and contextual quality. Furthermore, we employ multimodal large language models~(MLLMs)~\citep{yuan2025tarsier2} to generate dense and contextually aligned captions, which are subsequently refined using an additional large language model~(LLM)~\citep{yang2024qwen2} to enhance their accuracy, fluency, and descriptive richness. As a result, we have curated a robust training dataset comprising approximately 36M video-text pairs and 160M image-text pairs, which are proven sufficient for training industry-level generative models.

Secondly, we take a pioneering step by applying rectified flow formulation~\citep{lipman2023flow} for joint image and video generation, implemented through the \ours model family, which comprises Transformer architectures with 2B and 8B parameters. At its core, the \ours framework employs a 3D joint image-video variational autoencoder (VAE) to compress image and video inputs into a shared latent space, facilitating unified representation. This shared latent space is coupled with a full-attention~\citep{vaswani2017attention} mechanism, enabling seamless joint training of image and video. This architecture delivers high-quality, coherent outputs across both images and videos, establishing a unified framework for visual generation tasks.


Furthermore, to support the training of \ours at scale, we have developed a robust infrastructure tailored for large-scale model training. Our approach incorporates advanced parallelism strategies~\citep{jacobs2023deepspeed, pytorch_fsdp} to manage memory efficiently during long-context training. Additionally, we employ ByteCheckpoint~\citep{wan2024bytecheckpoint} for high-performance checkpointing and integrate fault-tolerant mechanisms from MegaScale~\citep{jiang2024megascale} to ensure stability and scalability across large GPU clusters. These optimizations enable \ours to handle the computational and data challenges of generative modeling with exceptional efficiency and reliability.


We evaluate \ours on both text-to-image and text-to-video benchmarks to highlight its competitive advantages. For text-to-image generation, \ours-T2I demonstrates strong performance across multiple benchmarks, including T2I-CompBench~\citep{huang2023t2i-compbench}, GenEval~\citep{ghosh2024geneval}, and DPG-Bench~\citep{hu2024ella_dbgbench}, excelling in both visual quality and text-image alignment. In text-to-video benchmarks, \ours-T2V achieves state-of-the-art performance on the UCF-101~\citep{ucf101} zero-shot generation task. Additionally, \ours-T2V attains an impressive score of \textbf{84.85} on VBench~\citep{huang2024vbench}, securing the top position on the leaderboard (as of 2025-01-25) and surpassing several leading commercial text-to-video models. Qualitative results, illustrated in \Cref{fig:main-demo}, further demonstrate the superior quality of the generated media samples. These findings underscore \ours's effectiveness in multi-modal generation and its potential as a high-performing solution for both research and commercial applications.
\section{Related Work}

\subsection{Large 3D Reconstruction Models}
Recently, generalized feed-forward models for 3D reconstruction from sparse input views have garnered considerable attention due to their applicability in heavily under-constrained scenarios. The Large Reconstruction Model (LRM)~\cite{hong2023lrm} uses a transformer-based encoder-decoder pipeline to infer a NeRF reconstruction from just a single image. Newer iterations have shifted the focus towards generating 3D Gaussian representations from four input images~\cite{tang2025lgm, xu2024grm, zhang2025gslrm, charatan2024pixelsplat, chen2025mvsplat, liu2025mvsgaussian}, showing remarkable novel view synthesis results. The paradigm of transformer-based sparse 3D reconstruction has also successfully been applied to lifting monocular videos to 4D~\cite{ren2024l4gm}. \\
Yet, none of the existing works in the domain have studied the use-case of inferring \textit{animatable} 3D representations from sparse input images, which is the focus of our work. To this end, we build on top of the Large Gaussian Reconstruction Model (GRM)~\cite{xu2024grm}.

\subsection{3D-aware Portrait Animation}
A different line of work focuses on animating portraits in a 3D-aware manner.
MegaPortraits~\cite{drobyshev2022megaportraits} builds a 3D Volume given a source and driving image, and renders the animated source actor via orthographic projection with subsequent 2D neural rendering.
3D morphable models (3DMMs)~\cite{blanz19993dmm} are extensively used to obtain more interpretable control over the portrait animation. For example, StyleRig~\cite{tewari2020stylerig} demonstrates how a 3DMM can be used to control the data generated from a pre-trained StyleGAN~\cite{karras2019stylegan} network. ROME~\cite{khakhulin2022rome} predicts vertex offsets and texture of a FLAME~\cite{li2017flame} mesh from the input image.
A TriPlane representation is inferred and animated via FLAME~\cite{li2017flame} in multiple methods like Portrait4D~\cite{deng2024portrait4d}, Portrait4D-v2~\cite{deng2024portrait4dv2}, and GPAvatar~\cite{chu2024gpavatar}.
Others, such as VOODOO 3D~\cite{tran2024voodoo3d} and VOODOO XP~\cite{tran2024voodooxp}, learn their own expression encoder to drive the source person in a more detailed manner. \\
All of the aforementioned methods require nothing more than a single image of a person to animate it. This allows them to train on large monocular video datasets to infer a very generic motion prior that even translates to paintings or cartoon characters. However, due to their task formulation, these methods mostly focus on image synthesis from a frontal camera, often trading 3D consistency for better image quality by using 2D screen-space neural renderers. In contrast, our work aims to produce a truthful and complete 3D avatar representation from the input images that can be viewed from any angle.  

\subsection{Photo-realistic 3D Face Models}
The increasing availability of large-scale multi-view face datasets~\cite{kirschstein2023nersemble, ava256, pan2024renderme360, yang2020facescape} has enabled building photo-realistic 3D face models that learn a detailed prior over both geometry and appearance of human faces. HeadNeRF~\cite{hong2022headnerf} conditions a Neural Radiance Field (NeRF)~\cite{mildenhall2021nerf} on identity, expression, albedo, and illumination codes. VRMM~\cite{yang2024vrmm} builds a high-quality and relightable 3D face model using volumetric primitives~\cite{lombardi2021mvp}. One2Avatar~\cite{yu2024one2avatar} extends a 3DMM by anchoring a radiance field to its surface. More recently, GPHM~\cite{xu2025gphm} and HeadGAP~\cite{zheng2024headgap} have adopted 3D Gaussians to build a photo-realistic 3D face model. \\
Photo-realistic 3D face models learn a powerful prior over human facial appearance and geometry, which can be fitted to a single or multiple images of a person, effectively inferring a 3D head avatar. However, the fitting procedure itself is non-trivial and often requires expensive test-time optimization, impeding casual use-cases on consumer-grade devices. While this limitation may be circumvented by learning a generalized encoder that maps images into the 3D face model's latent space, another fundamental limitation remains. Even with more multi-view face datasets being published, the number of available training subjects rarely exceeds the thousands, making it hard to truly learn the full distibution of human facial appearance. Instead, our approach avoids generalizing over the identity axis by conditioning on some images of a person, and only generalizes over the expression axis for which plenty of data is available. 

A similar motivation has inspired recent work on codec avatars where a generalized network infers an animatable 3D representation given a registered mesh of a person~\cite{cao2022authentic, li2024uravatar}.
The resulting avatars exhibit excellent quality at the cost of several minutes of video capture per subject and expensive test-time optimization.
For example, URAvatar~\cite{li2024uravatar} finetunes their network on the given video recording for 3 hours on 8 A100 GPUs, making inference on consumer-grade devices impossible. In contrast, our approach directly regresses the final 3D head avatar from just four input images without the need for expensive test-time fine-tuning.


\section{RoleMRC}
\label{sec:method}

In this section, we build RoleMRC. Figure\,\ref{fig:method} illustrates the overall pipeline of RoleMRC from top to bottom, which is divided into three steps.

\subsection{A Meta-pool of 10k Role Profiles}
\label{sec:meta_pool}
We first collect a meta-pool of 10k role profile using two open-source datasets, with Step 1 and 2.

\paragraph{Step 1: Persona Sampling.} We randomly sample 10.5k one-sentence demographic persona description from PersonaHub\,\cite{ge2024scaling}, such as ``\emph{A local business owner interested in economic trends}'', as shown at the top of Figure\,\ref{fig:method}. 

\paragraph{Step 2: Role Profile Standardization.} Next, we use a well-crafted prompt with gpt-4o\,\cite{gpt4o} to expand each sampled persona into a complete role profile, in reference to the 1-shot standardized example. Illustrated in the middle of Figure\,\ref{fig:method}, we require a standardized role profile consisting of seven components: \emph{Role Name and Brief Description}, \emph{Specific Abilities and Skills}, \emph{Speech Style}, \emph{Personality Characteristics}, \emph{Past Experience and Background}, \emph{Ability and Knowledge Boundaries} and \emph{Speech Examples}. %Setting standard specifications helps convert the generated role profiles into formatted records, which is beneficial for the post quality control. 
Standardizing these profiles ensures structured formatting, simplifying quality control. 
After manual checking and format filtering, we remove 333 invalid responses from gpt-4o, resulting in 10.2k final role profiles. We report complete persona-to-profile standardization prompt and structure tree of final role profiles in Appendix\,\ref{sec:app_prompt_1} and \,\ref{sec:app_tree}, respectively.

Machine Reading Comprehension (MRC) is one of the core tasks for LLMs to interact with human users. Consequently, we choose to synthesize fine-grained role-playing instruction-following data based on MRC. We first generate a retrieval pool containing 808.7k MRC data from the MSMARCO training set\,\cite{bajaj2016ms}. By leveraging SFR-Embedding\,\cite{SFR-embedding-2}, we perform an inner product search to identify the most relevant and least relevant MRC triplets (Passages, Question, Answer) for each role profile. For example, the middle part of Figure\,\ref{fig:method} shows that for the role \emph{Jessica Thompson, a resilient local business owner}, the most relevant question is about \emph{the skill of resiliency}, while the least relevant question is \emph{converting Fahrenheit to Celsius}. After review, we categorise the most relevant MRC triplet as within a role's knowledge boundary, and the least relevant MRC triplet as beyond their expertise.

\begin{figure}[t]
    \centering
    \includegraphics[width=1.0\linewidth]{figures/step3.png}
    \caption{The strategy of gradually synthesizing finer role-playing instructions in step 3 of Figure\,\ref{fig:method}.}
    \vspace{-1.0em}
    \label{fig:step3}
\end{figure}

\subsection{38k Role-playing Instructions}
Based on the role profiles, we then adopt \textbf{Step 3: Multi-stage Dialogue Synthesis} to generate 38k role-playing instructions, progressively increasing granularity across three categories %including three types with gradually finer granularity 
(Figure\,\ref{fig:step3}):
%\begin{itemize}
%[leftmargin=*,noitemsep,topsep=0pt]

\noindent \textbf{\underline{Free Chats.}} The simplest dialogues, free chats, are synthesized at first. Here, we ask gpt-4o to simulate and generate multi-turn open-domain conversations between the role and an imagined user based on the standardized role profile. When synthesizing the conversation, we additionally consider two factors: the \textbf{initial speaker} in the starting round of the conversation, and whether the role's speech has \textbf{a narration wrapped in brackets} at the beginning (e.g., \emph{(Aiden reviews the network logs, his eyes narrowing as he spots unusual activity) I found it!}). The narration refers to a short, vivid description of the role's speaking state from an omniscient perspective, which further strengthens the sense of role's depth and has been adopted in some role-playing datasets\,\cite{tu2024charactereval}. 

As shown on the left side of Figure\,\ref{fig:step3}, based on the aforementioned two factors, we synthesize four variations of Free Chats. In particular, when  narration is omitted, we deleted all the 
narration content in the speech examples from the role profile; %and for the case that 
when narration is allowed, we retain the narration content, and also add instructions to allow appropriate insertion of narration in the task prompt of gpt-4o. It worth to note that, in narration-allowed dialogues, not every response of the role has narration inserted to prevent overfitting. All categories of data in RoleMRC incorporate narration insertion and follow similar control mechanisms. The following sections will omit further details on narration.

\noindent \textbf{\underline{On-scene MRC Dialogues.}} The synthesis of on-scene MRC dialogues can be divided into two parts. The first part is similar to the free chats. As shown by the {\color{lightgreen}{green round rectangle}} in the upper part of Figure\,\ref{fig:step3}, we ask gpt-4o to synthesize a conversation (lower left corner of Figure\,\ref{fig:step3}) between the role and the user focusing on relevant passages. This part of the synthesis and the Free Chats share the entire meta-pool, so each consisting of 5k dialogues.

The remaining part forms eight types of single-turn role-playing Question Answering (QA). In the middle of Figure\,\ref{fig:step3}, we randomly select a group of roles and examined the most relevant MRCs they matched: if the question in the MRC is answerable, then the ground truth answer is stylized to match the role profile; otherwise, a seed script of ``unanswerable'' is randomly selected then stylized. The above process generates four groups of 1k data from type ``[1]'' to type``[4]''. According to the middle right side of Figure\,\ref{fig:step3}, we also select a group of roles and ensure that the least relevant MRCs they matched contain answerable QA pairs. Since the most irrelevant MRCs are outside the knowledge boundary of the roles, the role-playing responses to these questions are ``out-of-mind'' refusal or ``let-me-try'' attempt, thus synthesizing four groups of 1k data, from type ``[5]'' to type ``[8]''.

\noindent \textbf{\underline{Ruled Chats.}} We construct Ruled Chats by extending On-scene MRC Dialogues in categories ``[1]'' to ``[8]'' with incorporated three additional rules, as shown in the right bottom corner of Figure\,\ref{fig:step3}. For the \textbf{multi-turn rules}, we apply them to the four unanswerable scenarios ``[3]'', ``[4]'', ``[5]'', and ``[6]'', adding a user prompt that  forces the role to answer. Among them, data ``[3]'' and ``[4]'' maintain refusal since the questions in MRC are unanswerable; while ``[5]'' and ``[6]'' are transformed into attempts to answer despite knowledge limitations. For the \textbf{nested formatting rules}, we add new formatting instructions to the four categories of data ``[1]'', ``[2]'', ``[3]'', and ``[4]'', such as requiring emojis,  capitalization, specific punctuation marks, and controlling the total number of words, then modify the previous replies accordingly. For the last \textbf{prioritized rules}, we apply them to subsets ``[1]'' and ``[2]'' that contain normal stylized answers, inserting a  global refusal directive from the system, and thus creating a conflict between system instructions and the role's ability boundary.
%\end{itemize}

\begin{table}[t]
\resizebox{\columnwidth}{!}{%
  \begin{tabular}{c|c|c|c|c|c}
    \toprule
    & & \textbf{S*} & \textbf{P*} & \textbf{\#Turns} & \textbf{\#Words} \\ 
    \midrule
    \multirow{13.5}{*}{\textbf{RoleMRC}} 
    & \multicolumn{5}{c|}{\textbf{Free Chats}} \\ 
    \cmidrule(lr){2-6}
    & Chats & 5k & / & 9.47 & 38.62 \\ 
    \cmidrule(lr){2-6}
    & \multicolumn{5}{c|}{\textbf{On-scene MRC Dialogues}} \\ 
    \cmidrule(lr){2-6} 
    & On-scene Chats & 5k & / & 9.2 & 43.18 \\
    & Answer & 2k & 2k & 1 & 39.45 \\ 
    & No Answer & 2k & 2k & 1 & 47.09 \\ 
    & Refusal & 2k & 2k & 1 & 48.41 \\ 
    & Attempt & 2k & 2k & 1 & 47.92 \\ 
    \cmidrule(lr){2-6}
    & \multicolumn{5}{c|}{\textbf{Ruled Chats}} \\ 
    \cmidrule(lr){2-6}
    & Multi-turn & 2k & 2k & 2 & 42.47 \\ 
    & Nested & 1.6k & 1.6k & 1 & 46.17 \\ 
    & Prioritized & 2.4k & 2.4k & 1 & 42.65 \\ 
    \midrule
    & \textbf{Total} & 24k & 14k & 3.5 & 40.6 \\ 
    \midrule
    \multirow{3}{*}{\textbf{-mix}} 
    & RoleBench & 16k & / & 1 & 23.95 \\ 
    & RLHFlow & 40k & / & 1.39 & 111.79 \\ 
    & UltraFeedback & / & 14k & 1 & 199.28 \\ 
    \midrule
    & \textbf{Total} & 80k & 28k & 2 & 67.1 \\ 
    \bottomrule
  \end{tabular}}
  \vspace{-2mm}
  \caption{Statistics of RoleMRC. In particular, the column names S*, P*, \#Turns, and \#Words, stands for size of single-label data, size of pair-label data, average turns, and average number of words per reply, respectively. RoleMRC-mix expands RoleMRC by adding existing role-playing data.}
 \vspace{-3mm}
  \label{tab:roleMRC}
\end{table}

\subsection{Integration and Mix-up}
All the seed scripts and prioritized rules used for constructing On-scene Dialogues and Ruled Chats are reported in Appendix\,\ref{sec:app_scripts}. These raw responses are logically valid manual answers that remain unaffected by the roles' speaking styles, making them suitable as negative labels to contrast with the stylized answers. Thanks to these meticulous seed texts, we obtain high-quality synthetic data with stable output from gpt-4o. After integration, as shown in Table\,\ref{tab:roleMRC}, the final RoleMRC contains 24k single-label data for Supervised Fine-Tuning (SFT) and 14k pair-label data for Human Preference Optimization (HPO)\,\cite{ouyang2022training,rafailov2023direct,sampo,hong2024reference}. Considering that fine-tuning LLMs with relatively fixed data formats may lead to catastrophic forgetting\,\cite{kirkpatrick2017overcoming}, we create RoleMRC-mix as a robust version by incorporating external role-playing data (RoleBench\,\cite{wang2023rolellm}) and general instructions (RLHFlow\,\cite{dong2024rlhf}, UltraFeedback\,\cite{cui2023ultrafeedback}).

\section{Experiments}

\subsection{Setups}
\subsubsection{Implementation Details}
We apply our FDS method to two types of 3DGS: 
the original 3DGS, and 2DGS~\citep{huang20242d}. 
%
The number of iterations in our optimization 
process is 35,000.
We follow the default training configuration 
and apply our FDS method after 15,000 iterations,
then we add normal consistency loss for both
3DGS and 2DGS after 25000 iterations.
%
The weight for FDS, $\lambda_{fds}$, is set to 0.015,
the $\sigma$ is set to 23,
and the weight for normal consistency is set to 0.15
for all experiments. 
We removed the depth distortion loss in 2DGS 
because we found that it degrades its results in indoor scenes.
%
The Gaussian point cloud is initialized using Colmap
for all datasets.
%
%
We tested the impact of 
using Sea Raft~\citep{wang2025sea} and 
Raft\citep{teed2020raft} on FDS performance.
%
Due to the blurriness of the ScanNet dataset, 
additional prior constraints are required.
Thus, we incorporate normal prior supervision 
on the rendered normals 
in ScanNet (V2) dataset by default.
The normal prior is predicted by the Stable Normal 
model~\citep{ye2024stablenormal}
across all types of 3DGS.
%
The entire framework is implemented in 
PyTorch~\citep{paszke2019pytorch}, 
and all experiments are conducted on 
a single NVIDIA 4090D GPU.

\begin{figure}[t] \centering
    \makebox[0.16\textwidth]{\scriptsize Input}
    \makebox[0.16\textwidth]{\scriptsize 3DGS}
    \makebox[0.16\textwidth]{\scriptsize 2DGS}
    \makebox[0.16\textwidth]{\scriptsize 3DGS + FDS}
    \makebox[0.16\textwidth]{\scriptsize 2DGS + FDS}
    \makebox[0.16\textwidth]{\scriptsize GT (Depth)}

    \includegraphics[width=0.16\textwidth]{figure/fig3_img/compare3/gt_rgb/frame_00522.jpg}
    \includegraphics[width=0.16\textwidth]{figure/fig3_img/compare3/3DGS/frame_00522.jpg}
    \includegraphics[width=0.16\textwidth]{figure/fig3_img/compare3/2DGS/frame_00522.jpg}
    \includegraphics[width=0.16\textwidth]{figure/fig3_img/compare3/3DGS+FDS/frame_00522.jpg}
    \includegraphics[width=0.16\textwidth]{figure/fig3_img/compare3/2DGS+FDS/frame_00522.jpg}
    \includegraphics[width=0.16\textwidth]{figure/fig3_img/compare3/gt_depth/frame_00522.jpg} \\

    % \includegraphics[width=0.16\textwidth]{figure/fig3_img/compare1/gt_rgb/frame_00137.jpg}
    % \includegraphics[width=0.16\textwidth]{figure/fig3_img/compare1/3DGS/frame_00137.jpg}
    % \includegraphics[width=0.16\textwidth]{figure/fig3_img/compare1/2DGS/frame_00137.jpg}
    % \includegraphics[width=0.16\textwidth]{figure/fig3_img/compare1/3DGS+FDS/frame_00137.jpg}
    % \includegraphics[width=0.16\textwidth]{figure/fig3_img/compare1/2DGS+FDS/frame_00137.jpg}
    % \includegraphics[width=0.16\textwidth]{figure/fig3_img/compare1/gt_depth/frame_00137.jpg} \\

     \includegraphics[width=0.16\textwidth]{figure/fig3_img/compare2/gt_rgb/frame_00262.jpg}
    \includegraphics[width=0.16\textwidth]{figure/fig3_img/compare2/3DGS/frame_00262.jpg}
    \includegraphics[width=0.16\textwidth]{figure/fig3_img/compare2/2DGS/frame_00262.jpg}
    \includegraphics[width=0.16\textwidth]{figure/fig3_img/compare2/3DGS+FDS/frame_00262.jpg}
    \includegraphics[width=0.16\textwidth]{figure/fig3_img/compare2/2DGS+FDS/frame_00262.jpg}
    \includegraphics[width=0.16\textwidth]{figure/fig3_img/compare2/gt_depth/frame_00262.jpg} \\

    \includegraphics[width=0.16\textwidth]{figure/fig3_img/compare4/gt_rgb/frame00000.png}
    \includegraphics[width=0.16\textwidth]{figure/fig3_img/compare4/3DGS/frame00000.png}
    \includegraphics[width=0.16\textwidth]{figure/fig3_img/compare4/2DGS/frame00000.png}
    \includegraphics[width=0.16\textwidth]{figure/fig3_img/compare4/3DGS+FDS/frame00000.png}
    \includegraphics[width=0.16\textwidth]{figure/fig3_img/compare4/2DGS+FDS/frame00000.png}
    \includegraphics[width=0.16\textwidth]{figure/fig3_img/compare4/gt_depth/frame00000.png} \\

    \includegraphics[width=0.16\textwidth]{figure/fig3_img/compare5/gt_rgb/frame00080.png}
    \includegraphics[width=0.16\textwidth]{figure/fig3_img/compare5/3DGS/frame00080.png}
    \includegraphics[width=0.16\textwidth]{figure/fig3_img/compare5/2DGS/frame00080.png}
    \includegraphics[width=0.16\textwidth]{figure/fig3_img/compare5/3DGS+FDS/frame00080.png}
    \includegraphics[width=0.16\textwidth]{figure/fig3_img/compare5/2DGS+FDS/frame00080.png}
    \includegraphics[width=0.16\textwidth]{figure/fig3_img/compare5/gt_depth/frame00080.png} \\



    \caption{\textbf{Comparison of depth reconstruction on Mushroom and ScanNet datasets.} The original
    3DGS or 2DGS model equipped with FDS can remove unwanted floaters and reconstruct
    geometry more preciously.}
    \label{fig:compare}
\end{figure}


\subsubsection{Datasets and Metrics}

We evaluate our method for 3D reconstruction 
and novel view synthesis tasks on
\textbf{Mushroom}~\citep{ren2024mushroom},
\textbf{ScanNet (v2)}~\citep{dai2017scannet}, and 
\textbf{Replica}~\citep{replica19arxiv}
datasets,
which feature challenging indoor scenes with both 
sparse and dense image sampling.
%
The Mushroom dataset is an indoor dataset 
with sparse image sampling and two distinct 
camera trajectories. 
%
We train our model on the training split of 
the long capture sequence and evaluate 
novel view synthesis on the test split 
of the long capture sequences.
%
Five scenes are selected to evaluate our FDS, 
including "coffee room", "honka", "kokko", 
"sauna", and "vr room". 
%
ScanNet(V2)~\citep{dai2017scannet}  consists of 1,613 indoor scenes
with annotated camera poses and depth maps. 
%
We select 5 scenes from the ScanNet (V2) dataset, 
uniformly sampling one-tenth of the views,
following the approach in ~\citep{guo2022manhattan}.
To further improve the geometry rendering quality of 3DGS, 
%
Replica~\citep{replica19arxiv} contains small-scale 
real-world indoor scans. 
We evaluate our FDS on five scenes from 
Replica: office0, office1, office2, office3 and office4,
selecting one-tenth of the views for training.
%
The results for Replica are provided in the 
supplementary materials.
To evaluate the rendering quality and geometry 
of 3DGS, we report PSNR, SSIM, and LPIPS for 
rendering quality, along with Absolute Relative Distance 
(Abs Rel) as a depth quality metrics.
%
Additionally, for mesh evaluation, 
we use metrics including Accuracy, Completion, 
Chamfer-L1 distance, Normal Consistency, 
and F-scores.




\subsection{Results}
\subsubsection{Depth rendering and novel view synthesis}
The comparison results on Mushroom and 
ScanNet are presented in \tabref{tab:mushroom} 
and \tabref{tab:scannet}, respectively. 
%
Due to the sparsity of sampling 
in the Mushroom dataset,
challenges are posed for both GOF~\citep{yu2024gaussian} 
and PGSR~\citep{chen2024pgsr}, 
leading to their relative poor performance 
on the Mushroom dataset.
%
Our approach achieves the best performance 
with the FDS method applied during the training process.
The FDS significantly enhances the 
geometric quality of 3DGS on the Mushroom dataset, 
improving the "abs rel" metric by more than 50\%.
%
We found that Sea Raft~\citep{wang2025sea}
outperforms Raft~\citep{teed2020raft} on FDS, 
indicating that a better optical flow model 
can lead to more significant improvements.
%
Additionally, the render quality of RGB 
images shows a slight improvement, 
by 0.58 in 3DGS and 0.50 in 2DGS, 
benefiting from the incorporation of cross-view consistency in FDS. 
%
On the Mushroom
dataset, adding the FDS loss increases 
the training time by half an hour, which maintains the same
level as baseline.
%
Similarly, our method shows a notable improvement on the ScanNet dataset as well using Sea Raft~\citep{wang2025sea} Model. The "abs rel" metric in 2DGS is improved nearly 50\%. This demonstrates the robustness and effectiveness of the FDS method across different datasets.
%


% \begin{wraptable}{r}{0.6\linewidth} \centering
% \caption{\textbf{Ablation study on geometry priors.}} 
%         \label{tab:analysis_prior}
%         \resizebox{\textwidth}{!}{
\begin{tabular}{c| c c c c c | c c c c}

    \hline
     Method &  Acc$\downarrow$ & Comp $\downarrow$ & C-L1 $\downarrow$ & NC $\uparrow$ & F-Score $\uparrow$ &  Abs Rel $\downarrow$ &  PSNR $\uparrow$  & SSIM  $\uparrow$ & LPIPS $\downarrow$ \\ \hline
    2DGS&   0.1078&  0.0850&  0.0964&  0.7835&  0.5170&  0.1002&  23.56&  0.8166& 0.2730\\
    2DGS+Depth&   0.0862&  0.0702&  0.0782&  0.8153&  0.5965&  0.0672&  23.92&  0.8227& 0.2619 \\
    2DGS+MVDepth&   0.2065&  0.0917&  0.1491&  0.7832&  0.3178&  0.0792&  23.74&  0.8193& 0.2692 \\
    2DGS+Normal&   0.0939&  0.0637&  0.0788&  \textbf{0.8359}&  0.5782&  0.0768&  23.78&  0.8197& 0.2676 \\
    2DGS+FDS &  \textbf{0.0615} & \textbf{ 0.0534}& \textbf{0.0574}& 0.8151& \textbf{0.6974}&  \textbf{0.0561}&  \textbf{24.06}&  \textbf{0.8271}&\textbf{0.2610} \\ \hline
    2DGS+Depth+FDS &  0.0561 &  0.0519& 0.0540& 0.8295& 0.7282&  0.0454&  \textbf{24.22}& \textbf{0.8291}&\textbf{0.2570} \\
    2DGS+Normal+FDS &  \textbf{0.0529} & \textbf{ 0.0450}& \textbf{0.0490}& \textbf{0.8477}& \textbf{0.7430}&  \textbf{0.0443}&  24.10&  0.8283& 0.2590 \\
    2DGS+Depth+Normal &  0.0695 & 0.0513& 0.0604& 0.8540&0.6723&  0.0523&  24.09&  0.8264&0.2575\\ \hline
    2DGS+Depth+Normal+FDS &  \textbf{0.0506} & \textbf{0.0423}& \textbf{0.0464}& \textbf{0.8598}&\textbf{0.7613}&  \textbf{0.0403}&  \textbf{24.22}& 
    \textbf{0.8300}&\textbf{0.0403}\\
    
\bottomrule
\end{tabular}
}
% \end{wraptable}



The qualitative comparisons on the Mushroom and ScanNet dataset 
are illustrated in \figref{fig:compare}. 
%
%
As seen in the first row of \figref{fig:compare}, 
both the original 3DGS and 2DGS suffer from overfitting, 
leading to corrupted geometry generation. 
%
Our FDS effectively mitigates this issue by 
supervising the matching relationship between 
the input and sampled views, 
helping to recover the geometry.
%
FDS also improves the refinement of geometric details, 
as shown in other rows. 
By incorporating the matching prior through FDS, 
the quality of the rendered depth is significantly improved.
%

\begin{table}[t] \centering
\begin{minipage}[t]{0.96\linewidth}
        \captionof{table}{\textbf{3D Reconstruction 
        and novel view synthesis results on Mushroom dataset. * 
        Represents that FDS uses the Raft model.
        }}
        \label{tab:mushroom}
        \resizebox{\textwidth}{!}{
\begin{tabular}{c| c c c c c | c c c c c}
    \hline
     Method &  Acc$\downarrow$ & Comp $\downarrow$ & C-L1 $\downarrow$ & NC $\uparrow$ & F-Score $\uparrow$ &  Abs Rel $\downarrow$ &  PSNR $\uparrow$  & SSIM  $\uparrow$ & LPIPS $\downarrow$ & Time  $\downarrow$ \\ \hline

    % DN-splatter &   &  &  &  &  &  &  &  & \\
    GOF &  0.1812 & 0.1093 & 0.1453 & 0.6292 & 0.3665 & 0.2380  & 21.37  &  0.7762  & 0.3132  & $\approx$1.4h\\ 
    PGSR &  0.0971 & 0.1420 & 0.1196 & 0.7193 & 0.5105 & 0.1723  & 22.13  & 0.7773  & 0.2918  & $\approx$1.2h \\ \hline
    3DGS &   0.1167 &  0.1033&  0.1100&  0.7954&  0.3739&  0.1214&  24.18&  0.8392& 0.2511 &$\approx$0.8h \\
    3DGS + FDS$^*$ & 0.0569  & 0.0676 & 0.0623 & 0.8105 & 0.6573 & 0.0603 & 24.72  & 0.8489 & 0.2379 &$\approx$1.3h \\
    3DGS + FDS & \textbf{0.0527}  & \textbf{0.0565} & \textbf{0.0546} & \textbf{0.8178} & \textbf{0.6958} & \textbf{0.0568} & \textbf{24.76}  & \textbf{0.8486} & \textbf{0.2381} &$\approx$1.3h \\ \hline
    2DGS&   0.1078&  0.0850&  0.0964&  0.7835&  0.5170&  0.1002&  23.56&  0.8166& 0.2730 &$\approx$0.8h\\
    2DGS + FDS$^*$ &  0.0689 &  0.0646& 0.0667& 0.8042& 0.6582& 0.0589& 23.98&  0.8255&0.2621 &$\approx$1.3h\\
    2DGS + FDS &  \textbf{0.0615} & \textbf{ 0.0534}& \textbf{0.0574}& \textbf{0.8151}& \textbf{0.6974}&  \textbf{0.0561}&  \textbf{24.06}&  \textbf{0.8271}&\textbf{0.2610} &$\approx$1.3h \\ \hline
\end{tabular}
}
\end{minipage}\hfill
\end{table}

\begin{table}[t] \centering
\begin{minipage}[t]{0.96\linewidth}
        \captionof{table}{\textbf{3D Reconstruction 
        and novel view synthesis results on ScanNet dataset.}}
        \label{tab:scannet}
        \resizebox{\textwidth}{!}{
\begin{tabular}{c| c c c c c | c c c c }
    \hline
     Method &  Acc $\downarrow$ & Comp $\downarrow$ & C-L1 $\downarrow$ & NC $\uparrow$ & F-Score $\uparrow$ &  Abs Rel $\downarrow$ &  PSNR $\uparrow$  & SSIM  $\uparrow$ & LPIPS $\downarrow$ \\ \hline
    GOF & 1.8671  & 0.0805 & 0.9738 & 0.5622 & 0.2526 & 0.1597  & 21.55  & 0.7575  & 0.3881 \\
    PGSR &  0.2928 & 0.5103 & 0.4015 & 0.5567 & 0.1926 & 0.1661  & 21.71 & 0.7699  & 0.3899 \\ \hline

    3DGS &  0.4867 & 0.1211 & 0.3039 & 0.7342& 0.3059 & 0.1227 & 22.19& 0.7837 & 0.3907\\
    3DGS + FDS &  \textbf{0.2458} & \textbf{0.0787} & \textbf{0.1622} & \textbf{0.7831} & 
    \textbf{0.4482} & \textbf{0.0573} & \textbf{22.83} & \textbf{0.7911} & \textbf{0.3826} \\ \hline
    2DGS &  0.2658 & 0.0845 & 0.1752 & 0.7504& 0.4464 & 0.0831 & 22.59& 0.7881 & 0.3854\\
    2DGS + FDS &  \textbf{0.1457} & \textbf{0.0679} & \textbf{0.1068} & \textbf{0.7883} & 
    \textbf{0.5459} & \textbf{0.0432} & \textbf{22.91} & \textbf{0.7928} & \textbf{0.3800} \\ \hline
\end{tabular}
}
\end{minipage}\hfill
\end{table}


\begin{table}[t] \centering
\begin{minipage}[t]{0.96\linewidth}
        \captionof{table}{\textbf{Ablation study on geometry priors.}}
        \label{tab:analysis_prior}
        \resizebox{\textwidth}{!}{
\begin{tabular}{c| c c c c c | c c c c}

    \hline
     Method &  Acc$\downarrow$ & Comp $\downarrow$ & C-L1 $\downarrow$ & NC $\uparrow$ & F-Score $\uparrow$ &  Abs Rel $\downarrow$ &  PSNR $\uparrow$  & SSIM  $\uparrow$ & LPIPS $\downarrow$ \\ \hline
    2DGS&   0.1078&  0.0850&  0.0964&  0.7835&  0.5170&  0.1002&  23.56&  0.8166& 0.2730\\
    2DGS+Depth&   0.0862&  0.0702&  0.0782&  0.8153&  0.5965&  0.0672&  23.92&  0.8227& 0.2619 \\
    2DGS+MVDepth&   0.2065&  0.0917&  0.1491&  0.7832&  0.3178&  0.0792&  23.74&  0.8193& 0.2692 \\
    2DGS+Normal&   0.0939&  0.0637&  0.0788&  \textbf{0.8359}&  0.5782&  0.0768&  23.78&  0.8197& 0.2676 \\
    2DGS+FDS &  \textbf{0.0615} & \textbf{ 0.0534}& \textbf{0.0574}& 0.8151& \textbf{0.6974}&  \textbf{0.0561}&  \textbf{24.06}&  \textbf{0.8271}&\textbf{0.2610} \\ \hline
    2DGS+Depth+FDS &  0.0561 &  0.0519& 0.0540& 0.8295& 0.7282&  0.0454&  \textbf{24.22}& \textbf{0.8291}&\textbf{0.2570} \\
    2DGS+Normal+FDS &  \textbf{0.0529} & \textbf{ 0.0450}& \textbf{0.0490}& \textbf{0.8477}& \textbf{0.7430}&  \textbf{0.0443}&  24.10&  0.8283& 0.2590 \\
    2DGS+Depth+Normal &  0.0695 & 0.0513& 0.0604& 0.8540&0.6723&  0.0523&  24.09&  0.8264&0.2575\\ \hline
    2DGS+Depth+Normal+FDS &  \textbf{0.0506} & \textbf{0.0423}& \textbf{0.0464}& \textbf{0.8598}&\textbf{0.7613}&  \textbf{0.0403}&  \textbf{24.22}& 
    \textbf{0.8300}&\textbf{0.0403}\\
    
\bottomrule
\end{tabular}
}
\end{minipage}\hfill
\end{table}




\subsubsection{Mesh extraction}
To further demonstrate the improvement in geometry quality, 
we applied methods used in ~\citep{turkulainen2024dnsplatter} 
to extract meshes from the input views of optimized 3DGS. 
The comparison results are presented  
in \tabref{tab:mushroom}. 
With the integration of FDS, the mesh quality is significantly enhanced compared to the baseline, featuring fewer floaters and more well-defined shapes.
 %
% Following the incorporation of FDS, the reconstruction 
% results exhibit fewer floaters and more well-defined 
% shapes in the meshes. 
% Visualized comparisons
% are provided in the supplementary material.

% \begin{figure}[t] \centering
%     \makebox[0.19\textwidth]{\scriptsize GT}
%     \makebox[0.19\textwidth]{\scriptsize 3DGS}
%     \makebox[0.19\textwidth]{\scriptsize 3DGS+FDS}
%     \makebox[0.19\textwidth]{\scriptsize 2DGS}
%     \makebox[0.19\textwidth]{\scriptsize 2DGS+FDS} \\

%     \includegraphics[width=0.19\textwidth]{figure/fig4_img/compare1/gt02.png}
%     \includegraphics[width=0.19\textwidth]{figure/fig4_img/compare1/baseline06.png}
%     \includegraphics[width=0.19\textwidth]{figure/fig4_img/compare1/baseline_fds05.png}
%     \includegraphics[width=0.19\textwidth]{figure/fig4_img/compare1/2dgs04.png}
%     \includegraphics[width=0.19\textwidth]{figure/fig4_img/compare1/2dgs_fds03.png} \\

%     \includegraphics[width=0.19\textwidth]{figure/fig4_img/compare2/gt00.png}
%     \includegraphics[width=0.19\textwidth]{figure/fig4_img/compare2/baseline02.png}
%     \includegraphics[width=0.19\textwidth]{figure/fig4_img/compare2/baseline_fds01.png}
%     \includegraphics[width=0.19\textwidth]{figure/fig4_img/compare2/2dgs04.png}
%     \includegraphics[width=0.19\textwidth]{figure/fig4_img/compare2/2dgs_fds03.png} \\
      
%     \includegraphics[width=0.19\textwidth]{figure/fig4_img/compare3/gt05.png}
%     \includegraphics[width=0.19\textwidth]{figure/fig4_img/compare3/3dgs03.png}
%     \includegraphics[width=0.19\textwidth]{figure/fig4_img/compare3/3dgs_fds04.png}
%     \includegraphics[width=0.19\textwidth]{figure/fig4_img/compare3/2dgs02.png}
%     \includegraphics[width=0.19\textwidth]{figure/fig4_img/compare3/2dgs_fds01.png} \\

%     \caption{\textbf{Qualitative comparison of extracted mesh 
%     on Mushroom and ScanNet datasets.}}
%     \label{fig:mesh}
% \end{figure}












\subsection{Ablation study}


\textbf{Ablation study on geometry priors:} 
To highlight the advantage of incorporating matching priors, 
we incorporated various types of priors generated by different 
models into 2DGS. These include a monocular depth estimation
model (Depth Anything v2)~\citep{yang2024depth}, a two-view depth estimation 
model (Unimatch)~\citep{xu2023unifying}, 
and a monocular normal estimation model (DSINE)~\citep{bae2024rethinking}.
We adapt the scale and shift-invariant loss in Midas~\citep{birkl2023midas} for
monocular depth supervision and L1 loss for two-view depth supervison.
%
We use Sea Raft~\citep{wang2025sea} as our default optical flow model.
%
The comparison results on Mushroom dataset 
are shown in ~\tabref{tab:analysis_prior}.
We observe that the normal prior provides accurate shape information, 
enhancing the geometric quality of the radiance field. 
%
% In contrast, the monocular depth prior slightly increases 
% the 'Abs Rel' due to its ambiguous scale and inaccurate depth ordering.
% Moreover, the performance of monocular depth estimation 
% in the sauna scene is particularly poor, 
% primarily due to the presence of numerous reflective 
% surfaces and textureless walls, which limits the accuracy of monocular depth estimation.
%
The multi-view depth prior, hindered by the limited feature overlap 
between input views, fails to offer reliable geometric 
information. We test average "Abs Rel" of multi-view depth prior
, and the result is 0.19, which performs worse than the "Abs Rel" results 
rendered by original 2DGS.
From the results, it can be seen that depth order information provided by monocular depth improves
reconstruction accuracy. Meanwhile, our FDS achieves the best performance among all the priors, 
and by integrating all
three components, we obtained the optimal results.
%
%
\begin{figure}[t] \centering
    \makebox[0.16\textwidth]{\scriptsize RF (16000 iters)}
    \makebox[0.16\textwidth]{\scriptsize RF* (20000 iters)}
    \makebox[0.16\textwidth]{\scriptsize RF (20000 iters)  }
    \makebox[0.16\textwidth]{\scriptsize PF (16000 iters)}
    \makebox[0.16\textwidth]{\scriptsize PF (20000 iters)}


    % \includegraphics[width=0.16\textwidth]{figure/fig5_img/compare1/16000.png}
    % \includegraphics[width=0.16\textwidth]{figure/fig5_img/compare1/20000_wo_flow_loss.png}
    % \includegraphics[width=0.16\textwidth]{figure/fig5_img/compare1/20000.png}
    % \includegraphics[width=0.16\textwidth]{figure/fig5_img/compare1/16000_prior.png}
    % \includegraphics[width=0.16\textwidth]{figure/fig5_img/compare1/20000_prior.png}\\

    % \includegraphics[width=0.16\textwidth]{figure/fig5_img/compare2/16000.png}
    % \includegraphics[width=0.16\textwidth]{figure/fig5_img/compare2/20000_wo_flow_loss.png}
    % \includegraphics[width=0.16\textwidth]{figure/fig5_img/compare2/20000.png}
    % \includegraphics[width=0.16\textwidth]{figure/fig5_img/compare2/16000_prior.png}
    % \includegraphics[width=0.16\textwidth]{figure/fig5_img/compare2/20000_prior.png}\\

    \includegraphics[width=0.16\textwidth]{figure/fig5_img/compare3/16000.png}
    \includegraphics[width=0.16\textwidth]{figure/fig5_img/compare3/20000_wo_flow_loss.png}
    \includegraphics[width=0.16\textwidth]{figure/fig5_img/compare3/20000.png}
    \includegraphics[width=0.16\textwidth]{figure/fig5_img/compare3/16000_prior.png}
    \includegraphics[width=0.16\textwidth]{figure/fig5_img/compare3/20000_prior.png}\\
    
    \includegraphics[width=0.16\textwidth]{figure/fig5_img/compare4/16000.png}
    \includegraphics[width=0.16\textwidth]{figure/fig5_img/compare4/20000_wo_flow_loss.png}
    \includegraphics[width=0.16\textwidth]{figure/fig5_img/compare4/20000.png}
    \includegraphics[width=0.16\textwidth]{figure/fig5_img/compare4/16000_prior.png}
    \includegraphics[width=0.16\textwidth]{figure/fig5_img/compare4/20000_prior.png}\\

    \includegraphics[width=0.30\textwidth]{figure/fig5_img/bar.png}

    \caption{\textbf{The error map of Radiance Flow and Prior Flow.} RF: Radiance Flow, PF: Prior Flow, * means that there is no FDS loss supervision during optimization.}
    \label{fig:error_map}
\end{figure}




\textbf{Ablation study on FDS: }
In this section, we present the design of our FDS 
method through an ablation study on the 
Mushroom dataset to validate its effectiveness.
%
The optional configurations of FDS are outlined in ~\tabref{tab:ablation_fds}.
Our base model is the 2DGS equipped with FDS,
and its results are shown 
in the first row. The goal of this analysis 
is to evaluate the impact 
of various strategies on FDS sampling and loss design.
%
We observe that when we 
replace $I_i$ in \eqref{equ:mflow} with $C_i$, 
as shown in the second row, the geometric quality 
of 2DGS deteriorates. Using $I_i$ instead of $C_i$ 
help us to remove the floaters in $\bm{C^s}$, which are also 
remained in $\bm{C^i}$.
We also experiment with modifying the FDS loss. For example, 
in the third row, we use the neighbor 
input view as the sampling view, and replace the 
render result of neighbor view with ground truth image of its input view.
%
Due to the significant movement between images, the Prior Flow fails to accurately 
match the pixel between them, leading to a further degradation in geometric quality.
%
Finally, we attempt to fix the sampling view 
and found that this severely damaged the geometric quality, 
indicating that random sampling is essential for the stability 
of the mean error in the Prior flow.



\begin{table}[t] \centering

\begin{minipage}[t]{1.0\linewidth}
        \captionof{table}{\textbf{Ablation study on FDS strategies.}}
        \label{tab:ablation_fds}
        \resizebox{\textwidth}{!}{
\begin{tabular}{c|c|c|c|c|c|c|c}
    \hline
    \multicolumn{2}{c|}{$\mathcal{M}_{\theta}(X, \bm{C^s})$} & \multicolumn{3}{c|}{Loss} & \multicolumn{3}{c}{Metric}  \\
    \hline
    $X=C^i$ & $X=I^i$  & Input view & Sampled view     & Fixed Sampled view        & Abs Rel $\downarrow$ & F-score $\uparrow$ & NC $\uparrow$ \\
    \hline
    & \ding{51} &     &\ding{51}    &    &    \textbf{0.0561}        &  \textbf{0.6974}         & \textbf{0.8151}\\
    \hline
     \ding{51} &           &     &\ding{51}    &    &    0.0839        &  0.6242         &0.8030\\
     &  \ding{51} &   \ding{51}  &    &    &    0.0877       & 0.6091        & 0.7614 \\
      &  \ding{51} &    &    & \ding{51}    &    0.0724           & 0.6312          & 0.8015 \\
\bottomrule
\end{tabular}
}
\end{minipage}
\end{table}




\begin{figure}[htbp] \centering
    \makebox[0.22\textwidth]{}
    \makebox[0.22\textwidth]{}
    \makebox[0.22\textwidth]{}
    \makebox[0.22\textwidth]{}
    \\

    \includegraphics[width=0.22\textwidth]{figure/fig6_img/l1/rgb/frame00096.png}
    \includegraphics[width=0.22\textwidth]{figure/fig6_img/l1/render_rgb/frame00096.png}
    \includegraphics[width=0.22\textwidth]{figure/fig6_img/l1/render_depth/frame00096.png}
    \includegraphics[width=0.22\textwidth]{figure/fig6_img/l1/depth/frame00096.png}

    % \includegraphics[width=0.22\textwidth]{figure/fig6_img/l2/rgb/frame00112.png}
    % \includegraphics[width=0.22\textwidth]{figure/fig6_img/l2/render_rgb/frame00112.png}
    % \includegraphics[width=0.22\textwidth]{figure/fig6_img/l2/render_depth/frame00112.png}
    % \includegraphics[width=0.22\textwidth]{figure/fig6_img/l2/depth/frame00112.png}

    \caption{\textbf{Limitation of FDS.} }
    \label{fig:limitation}
\end{figure}


% \begin{figure}[t] \centering
%     \makebox[0.48\textwidth]{}
%     \makebox[0.48\textwidth]{}
%     \\
%     \includegraphics[width=0.48\textwidth]{figure/loss_Ignatius.pdf}
%     \includegraphics[width=0.48\textwidth]{figure/loss_family.pdf}
%     \caption{\textbf{Comparison the photometric error of Radiance Flow and Prior Flow:} 
%     We add FDS method after 2k iteration during training.
%     The results show
%     that:  1) The Prior Flow is more precise and 
%     robust than Radiance Flow during the radiance 
%     optimization; 2) After adding the FDS loss 
%     which utilize Prior 
%     flow to supervise the Radiance Flow at 2k iterations, 
%     both flow are more accurate, which lead to
%     a mutually reinforcing effects.(TODO fix it)} 
%     \label{fig:flowcompare}
% \end{figure}






\textbf{Interpretive Experiments: }
To demonstrate the mutual refinement of two flows in our FDS, 
For each view, we sample the unobserved 
views multiple times to compute the mean error 
of both Radiance Flow and Prior Flow. 
We use Raft~\citep{teed2020raft} as our default optical flow model
for visualization.
The ground truth flow is calculated based on 
~\eref{equ:flow_pose} and ~\eref{equ:flow} 
utilizing ground truth depth in dataset.
We introduce our FDS loss after 16000 iterations during 
optimization of 2DGS.
The error maps are shown in ~\figref{fig:error_map}.
Our analysis reveals that Radiance Flow tends to 
exhibit significant geometric errors, 
whereas Prior Flow can more accurately estimate the geometry,
effectively disregarding errors introduced by floating Gaussian points. 

%





\subsection{Limitation and further work}

Firstly, our FDS faces challenges in scenes with 
significant lighting variations between different 
views, as shown in the lamp of first row in ~\figref{fig:limitation}. 
%
Incorporating exposure compensation into FDS could help address this issue. 
%
 Additionally, our method struggles with 
 reflective surfaces and motion blur,
 leading to incorrect matching. 
 %
 In the future, we plan to explore the potential 
 of FDS in monocular video reconstruction tasks, 
 using only a single input image at each time step.
 


\section{Conclusions}
In this paper, we propose Flow Distillation Sampling (FDS), which
leverages the matching prior between input views and 
sampled unobserved views from the pretrained optical flow model, to improve the geometry quality
of Gaussian radiance field. 
Our method can be applied to different approaches (3DGS and 2DGS) to enhance the geometric rendering quality of the corresponding neural radiance fields.
We apply our method to the 3DGS-based framework, 
and the geometry is enhanced on the Mushroom, ScanNet, and Replica datasets.

\section*{Acknowledgements} This work was supported by 
National Key R\&D Program of China (2023YFB3209702), 
the National Natural Science Foundation of 
China (62441204, 62472213), and Gusu 
Innovation \& Entrepreneurship Leading Talents Program (ZXL2024361)

\section{Discussion and Conclusion}

% \begin{quote}
% \textit{"We believe it is unethical for social workers not to learn... about technology-mediated social work."} (\citeauthor{singer_ai_2023}, 2023)
% \end{quote}

In this study, we uncovered multiple ways in which GenAI can be used in social service practice. While some concerns did arise, practitioners by and large seemed optimistic about the possibilities of such tools, and that these issues could be overcome. We note that while most participants found the tool useful, it was far from perfect in its outputs. This is not surprising, since it was powered by a generic LLM rather than one fine-tuned for social service case management. However, despite these inadequacies, our participants still found many uses for most of the tool's outputs. Many flaws pointed out by our participants related to highly contextualised, local knowledge. To tune an AI system for this would require large amounts of case files as training data; given the privacy concerns associated with using client data, this seems unlikely to happen in the near future. What our study shows, however, is that GenAI systems need not aim to be perfect to be useful to social service practitioners, and can instead serve as a complement to the critical "human touch" in social service.

We draw both inspiration and comparisons with prior work on AI in other settings. Studies on creative writing tools showed how the "uncertainty" \cite{wan2024felt} and "randomness" \cite{clark2018creative} of AI outputs aid creativity. Given the promise that our tool shows in aiding brainstorming and discussion, future social service studies could consider AI tools explicitly geared towards creativity - for instance, providing side-by-side displays of how a given case would fit into different theoretical frameworks, prompting users to compare, contrast, and adopt the best of each framework; or allowing users to play around with combining different intervention modalities to generate eclectic (i.e. multi-modal) interventions.

At the same time, the concept of supervision creates a different interaction paradigm to other uses of AI in brainstorming. Past work (e.g. \cite{shaer2024ai}) has explored the use of GenAI for ideation during brainstorming sessions, wherein all users present discuss the ideas generated by the system. With supervision in social service practice, however, there is a marked information and role asymmetry: supervisors may not have had the time to fully read up on their supervisee's case beforehand, yet have to provide guidance and help to the latter. We suggest that GenAI can serve a dual purpose of bringing supervisors up to speed quickly by summarising their supervisee's case data, while simultaneously generating a list of discussion and talking points that can improve the quality of supervision. Generalising, this interaction paradigm has promise in many other areas: senior doctors reviewing medical procedures with newer ones \cite{snowdon2017does} could use GenAI to generate questions about critical parts of a procedure to ask the latter, confirming they have been correctly understood or executed; game studio directors could quickly summarise key developmental pipeline concerns to raise at meetings and ensure the team is on track; even in academia, advisors involved in rather too many projects to keep track of could quickly summarise each graduate student's projects and identify potential concerns to address at their next meeting.

In closing, we are optimistic about the potential for GenAI to significantly enhance social service practice and the quality of care to clients. Future studies could focus on 1) longitudinal investigations into the long-term impact of GenAI on practitioner skills, client outcomes, and organisational workflows, and 2) optimising workflows to best integrate GenAI into casework and supervision, understanding where best to harness the speed and creativity of such systems in harmony with the experience and skills of practitioners at all levels.

% GenAI here thus serves as a tool that supervisors can use before rather then using the session, taking just a few minutes of their time to generate a list of discussion points with their supervisees.

% Traditional brainstorming comes up with new things that users discuss. In supervision, supervisors can use AI to more efficiently generate talking points with their supervisees. These are generally not novel ideas, since an experienced worker would be able to come up with these on their own. However, the interesting and novel use of AI here is in its use as a preparation tool, efficiently generating talking and discussion points, saving supervisors' time in preparing for a session, while still serving as a brainstorming tool during the session itself.

% The idea of embracing imperfect AI echoes the findings of \citeauthor{bossen2023batman} (2023) in a clinical decision setting, which examined the successful implementation of an "error-prone but useful AI tool". This study frames human-AI collaboration as "Batman and Robin", where AI is a useful but ultimately less skilled sidekick that plays second fiddle to Batman. This is similar to \citeauthor{yang2019unremarkable}'s (2019) idea of "unremarkable AI", systems designed to be unobtrusive and only visible to the user when they add some value. As compared to \citeauthor{bossen2023batman}, however, we see fewer instances of our AI system producing errors, and more examples of it providing learning and collaborative opportunities and other new use cases. We build on the idea of "complementary performance" \cite{bansal2021does}, which discusses how the unique expertise of AI enhances human decision-making performance beyond what humans can achieve alone. Beyond decision-making, GenAI can now enable "complementary work patterns", where the nature of its outputs enables humans to carry out their work in entirely new ways. Our study suggests that rather being a sidekick - Robin - AI is growing into the role of a "second Batman" or "AI-Batman": an entity with distinct abilities and expertise from humans, and that contributes in its own unique way. There is certainly still a time and place for unremarkable AI, but exploring uses beyond that paradigm uncovers entirely new areas of system design.

% % \cite{gero2022sparks} found AI to be useful for science writers to translate ideas already in their head into words, and to provide new perspectives to spark further inspiration. \textit{But how is ours different from theirs?}

% \subsection{New Avenues of Human-AI Collaboration}
% \label{subsubsec:discussionhaicollaboration}

% Past HCI literature in other areas \cite{nah2023generative} has suggested that GenAI represents a "leap" \cite{singh2023hide} in human-AI collaboration, 
% % Even when an AI system sometimes produces irrelevant outputs, it can still provide users 
% % Such systems have been proposed as ways to 
% helping users discover new viewpoints \cite{singh2023hide}, scour existing literature to suggest new hypotheses 
% \cite{cascella2023evaluating} and answer questions \cite{biswas2023role}, stimulate their cognitive processes \cite{memmert2023towards}, and overcome "writer's block" \cite{singh2023hide, cooper2023examining} (particularly relevant to SSPs and the vast amount of writing required of them). Our study finds promise for AI to help SSPs in all of these areas. By nature of being more verbose and capable of generating large amounts of content, GenAI seems to create a new way in which AI can complement human work and expertise. Our system, as LLMs tend to do, produced a lot of "bullshit" (S6) \cite{frankfurt2005bullshit} - superficially true statements that were often only "tangentially related" and "devoid of meaning" \cite{halloran2023ai}. Yet, many participants cited the page-long analyses and detailed multi-step intervention plans generated by the AI system to be a good starting point for further discussion, both to better conceptualize a particular case and to facilitate general worker growth and development. Almost like throwing mud at a wall to see what sticks, GenAI can quickly produce a long list of ideas or information, before the worker glances through it and quickly identifies the more interesting points to discuss. Playing the proposed role as a "scaffold" for further work \cite{cooper2023examining}, GenAI, literally, generates new opportunities for novel and more effective processes and perspectives that previous systems (e.g., PRMs) could not. This represents an entirely new mode of human-AI collaboration.
% % This represents a new mode of collaboration not possible with the largely quantitative AI models (like PRMs) of the past.

% Our work therefore supports and extends prior research that have postulated the the potential of AI's shifting roles from decision-maker to human-supporter \cite{wang_human-human_2020}. \citeauthor{siemon2022elaborating} (2022) suggests the role of AI as a "creator" or "coordinator", rather than merely providing "process guidance" \cite{memmert2023towards} that does not contribute to brainstorming. Similarly, \citeauthor{memmert2023towards} (2023) propose GenAI as a step forward from providing meta-level process guidance (i.e. facilitating user tasks) to actively contributing content and aiding brainstorming. We suggest that beyond content-support, AI can even create new work processes that were not possible without GenAI. In this sense, AI has come full circle, becoming a "meta-facilitator".

% % --- WIP BELOW ---

% % Our work echoes and extends previous research on HAI collaboration in tasks requiring a human touch. \cite{gero2023social} found AI to be a safe space for creative writers to bounce ideas off of and document their inner thoughts. \cite{dhillon2024shaping} reference the idea of appropriate scaffolding in argumentative writing, where the user is providing with guidance appropriate for their competency level, and also warns of decreased satisfaction and ownership from AI use. 

% Separately, we draw parallels with the field of creative writing, where HAI collaboration has been extensively researched. Writers note the "irreducibly human" aspect of creativity in writing \cite{gero2023social}, similar to the "human touch" core to social service practice (D1); both groups therefore expressed few concerns about AI taking over core aspects of their jobs. Another interesting parallel was how writers often appreciated the "uncertainty" \cite{wan2024felt} and "randomness" \cite{clark2018creative} of AI systems, which served as a source of inspiration. This echoes the idea of "imperfect AI" "expanding [the] perspective[s]" (S4) of our participants when they simply skimmed through what the AI produced. \cite{wan2024felt} cited how the "duality of uncertainty in the creativity process advances the exploration of the imperfection of GenAI models". While social service work is not typically regarded as "creative", practitioners nonetheless go through processes of ideation and iteration while formulating a case. Our study showed hints of how AI can help with various forms of ideation, but, drawing inspiration from creative writing tools, future studies could consider designs more explicitly geared towards creativity - for instance, by attempting to fit a given case into a number of different theories or modalities, and displaying them together for the user to consider. While many of these assessments may be imperfect or even downnright nonsensical, they may contain valuable ideas and new angles on viewing the case that the practitioner can integrate into their own assessment.

% % \cite{foong2024designing}, describing the design of caregiver-facing values elicitation tools, cites the "twin scenarios" that caregivers face - private use, where they might use a tool to discover their patient's values, and collaborative use, where they discuss the resulting values with other parties close to the patient. This closely mirrors how SSPs in our study reference both individual and collaborative uses of our tool. Unlike in \cite{foong2024designing}, however, we do not see a resulting need to design a "staged approach" with distinct interface features for both stages.

% % --- END OF WIP ---

% Having mentioned algorithm aversion previously, we also make a quick point here on the other end of the spectrum - automation bias, or blind trust in an automated system \cite{brown2019toward}. LLMs risk being perceived as an "ultimate epistemic authority" \cite{cooper2023examining} due to their detailed, life-like outputs. While automation bias has been studied in many contexts, including in the social sector or adjacent areas, we suggest that the very nature of GenAI systems fundamentally inhibits automation bias. The tendency of GenAI to produce verbose, lengthy explanations prompts users to read and think through the machine's judgement before accepting it, bringing up opportunities to disagree with the machine's opinion. This guards against blind acceptance of the system's recommendations, particularly in the culture of a social work agency where constant dialogue - including discussing AI-produced work - is the norm.


% % : Perception of AI in Social Service Work ??

% \subsection{Redefining the Boundary}
% \label{subsubsec:discussiontheoretical}

% As \citeauthor{meilvang_working_2023} (2023) describes, the social service profession has sought to distance itself from comprising mostly "administrative work" \cite{abbott2016boundaries}, and workers have long tried to tried to reduce their considerable time \cite{socialraadgiverforening2010notat} spent on such tasks in favour of actual casework with clients \cite{toren1972social}. Our study, however, suggests a blurring of the line between "manual" administrative tasks and "mental" casework that draws on practitioner expertise. Many tasks our participants cited involve elements of both: for instance, documenting a case recording requires selecting only the relevant information to include, and planning an intervention can be an iterative process of drafting a plan and discussing it with colleagues and superiors. This all stems from the fact that GenAI can produce virtually any document required by the user, but this document almost always requires revision under a watchful human eye.

% \citeauthor{meilvang_working_2023} (2023) also describes a more recent shift in the perceived accepted boundary of AI interventions in social service work. From "defending [the] jurisdiction [of social service work] against artificial intelligence" in the early days of PRM and other statistical assessment tools, the community has started to embrace AI as a "decision-support ... element in the assessment process". Our study concurs and frames GenAI as a source of information that can be used to support and qualify the assessments of SSPs \cite{meilvang_working_2023}, but suggests that we can take a step further: AI can be viewed as a \textit{facilitator} rather than just a supporter. GenAI can facilitate a wide range of discussions that promote efficiency, encourage worker learning and growth, and ultimately enhance client outcomes. This entails a much larger scope of AI use, where practitioners use the information provided by AI in a range of new scenarios. 

% Taken together, these suggest a new focus for boundary work and, more broadly, HCI research. GAI can play a role not just in menial documentation or decision-support, but can be deeply ingrained into every facet of the social service workflow to open new opportunities for worker growth, workflow optimisation, and ultimately improved client outcomes. Future research can therefore investigate the deeper, organisational-level effects of these new uses of AI, and their resulting impact on the role of profession discretion in effective social service work.

% % MH: oh i feel this paragraph is quite new to me! Could we elaborate this more, and truncate the first two paragraphs a bit to adjust the word propotion?


% % Our study extensively documents this for the first time in social service practice, and in the process reveals new insights about how AI can play such a role.



% \subsection{Design Implications}

% % Add link from ACE diagram?

% % EJ: it would be interesting to discuss how LLMs could help "hands-on experience" in the discussion section

% Addressing the struggle of integrating AI amidst the tension between machine assessment and expert judgement, we reframe AI as an \textit{facilitator} rather than an algorithm or decision-support tool, alleviating many concerns about trust and explainablity. We now present a high-level framework (Figure \ref{fig:hai-collaboration}) on human-AI collaboration, presenting a new perspective on designing effective AI systems that can be applied to both the social service sector and beyond.

% \begin{figure}
%     \centering
%     \includegraphics[scale=0.15]{images/designframework.png}
%     \caption{Framework for Human-AI Collaboration}
%     \label{fig:hai-collaboration}
%     \Description{An image showing our framework for Human-AI Collaboration. It shows that as stakeholder level increases from junior to senior, the directness of use shifts from co-creation to provision.}
% \end{figure}
% % MH: so this paradigm is proposed by us? I wonder if this could a part of results as well..?

% % \subsubsection{From Creation to Provision}

% In Section \ref{sec:stage2findings}, we uncovered the different ways in which SSPs of varying seniorities use, evaluate, and suggest uses of AI. These are intrinsically tied to the perspectives and levels of expertise that each stakeholder possesses. We therefore position the role of AI along the scale of \textit{creation} to \textit{provision}. 

% With junior workers, we recommend \textbf{designing tools for co-creation}: systems that aid the least experienced workers in creating the required deliverables for their work. Rather than \textit{telling} workers what to do - a difficult task in any case given the complexity of social work solutions - AI systems should instead \textit{co-create} deliverables required of these workers. These encompass the multitude of use cases that junior workers found useful: creating reports, suggesting perspectives from which to formulate a case, and providing a starting template for possible intervention plans. Notably, since AI outputs are not perfect, we emphasise the "co" in "co-creation": AI should only be a part of the workflow that also includes active engagement on the part of the SSPs and proactive discussion with supervisors. 

% For more experienced SSPs, we recommend \textbf{designing tools for provision}. Again, this is not the mere provision of recommendations or courses of action with clients, but rather that of resources which complement the needs of workers with greater responsibilities. This notably includes supplying materials to aid with supervision, a novel use case that to our knowledge has not surfaced in previous literature. In addition, senior workers also benefit greatly from manual tasks such as routine report writing and data processing. Since these workers are more experienced and can better spot inaccuracies in AI output, we suggest that AI can "provide" a more finished product that requires less vetting and corrections, and which can be used more directly as part of required deliverables.

% % MH: can we seperate here? above is about the guidance to paradigm, below is the practical roadmap for implementation
% In terms of concrete design features, given the constant focus on discussing AI outputs between colleagues in our FGDs, we recommend that AI tools, particularly those for junior workers, \textbf{include collaborative features} that facilitate feedback and idea sharing between users. We also suggest that designers work closely with domain experts (i.e. social work practitioners and agencies) to identify areas where the given AI model tends to make more mistakes, and to build in features that \textbf{highlight potential mistakes or inadequacies} in the AI's output to facilitate further discussion and avoid workers adopting suboptimal suggestions. 

% We also point out a fundamental difference between GenAI systems and previous systems: that GenAI can now play an important role in aiding users \textit{regardless of its flaws}. The nature of GenAI means that it promotes discussion and opens up new workflows by nature of its verbose and potentially incomplete outputs. Rather than working towards more accurate or explainable outcomes, which may in any case have minimal improvement on worker outcomes \cite{li2024advanced}, designers can also focus on \textbf{understanding how GenAI outputs can augment existing user flows and create new ones}.

% % for more senior workers...

% % how to differentiate levels of workers?

% % \subsubsection{Provider}

% % The most basic and obvious role of modern AI that we identify leverages the main strength of LLMs. They have the ability to produce high-quality writing from short, point-form, or otherwise messy and disjoint case notes that user often have \textit{[cite participant here]}. 

% Finally, given the limited expertise of many workers at using AI, it is important that systems \textbf{explicitly guide users to the features they need}, rather than simply relying on the ability of GAI to understand complex user instructions. For example, in the case of flexibility in use cases (Section \ref{subsubsec:control}), systems should include user flows that help combine multiple intervention and assessment modalities in order to directly meet the needs of workers.

% \subsection{Limitations and Future Work}

% While we attempt to mimic a contextual inquiry and work environment in our study design, there is no substitute for real data from actual system deployment. The use of an AI system in day-to-day work could reveal a different set of insights. Future studies could in particular study how the longitudinal context of how user attitudes, behaviours, preferences, and work outputs change with extended use of AI. 

% While we tried to include practitioners from different agencies, roles, and seniorities, social service practice may differ culturally or procedurally in other agencies or countries. Future studies could investigate different kinds of social service agencies and in different cultures to see if AI is similarly useful there.

% As the study was conducted in a country with relatively high technology literacy, participants naturally had a higher baseline understanding and acceptance of AI and other computer systems. However, we emphasise that our findings are not contingent on this - rather, we suggest that our proposed lens of viewing AI in the social sector is a means for engaging in relevant stakeholders and ensuring the effective design and implementation of AI in the social sector, regardless of how participants feel about AI to begin with. 



% % \subsection{Notes}

% % 1) safety and risks and 2) privacy - what does the emphasis on this say about a) design recommendations and b) approach to designing/PD of such systems?


% % W9 was presented with "Strengths" and "SFBT" output options. They commented, "solution focus is always building on the person's strengths". W9 therefore requested being able to output strengths and SFBT at the same time. But this would suggest that the SFBT output does not currently emphasise strengths strongly enough. However, W9 did not specifically evaluate that, and only made this comment because they saw the "strengths" option available, and in their head, strengths are key to SFBT.
% % What does this say about system design and UI in relation to user mental models?
\paragraph{Summary}
Our findings provide significant insights into the influence of correctness, explanations, and refinement on evaluation accuracy and user trust in AI-based planners. 
In particular, the findings are three-fold: 
(1) The \textbf{correctness} of the generated plans is the most significant factor that impacts the evaluation accuracy and user trust in the planners. As the PDDL solver is more capable of generating correct plans, it achieves the highest evaluation accuracy and trust. 
(2) The \textbf{explanation} component of the LLM planner improves evaluation accuracy, as LLM+Expl achieves higher accuracy than LLM alone. Despite this improvement, LLM+Expl minimally impacts user trust. However, alternative explanation methods may influence user trust differently from the manually generated explanations used in our approach.
% On the other hand, explanations may help refine the trust of the planner to a more appropriate level by indicating planner shortcomings.
(3) The \textbf{refinement} procedure in the LLM planner does not lead to a significant improvement in evaluation accuracy; however, it exhibits a positive influence on user trust that may indicate an overtrust in some situations.
% This finding is aligned with prior works showing that iterative refinements based on user feedback would increase user trust~\cite{kunkel2019let, sebo2019don}.
Finally, the propensity-to-trust analysis identifies correctness as the primary determinant of user trust, whereas explanations provided limited improvement in scenarios where the planner's accuracy is diminished.

% In conclusion, our results indicate that the planner's correctness is the dominant factor for both evaluation accuracy and user trust. Therefore, selecting high-quality training data and optimizing the training procedure of AI-based planners to improve planning correctness is the top priority. Once the AI planner achieves a similar correctness level to traditional graph-search planners, strengthening its capability to explain and refine plans will further improve user trust compared to traditional planners.

\paragraph{Future Research} Future steps in this research include expanding user studies with larger sample sizes to improve generalizability and including additional planning problems per session for a more comprehensive evaluation. Next, we will explore alternative methods for generating plan explanations beyond manual creation to identify approaches that more effectively enhance user trust. 
Additionally, we will examine user trust by employing multiple LLM-based planners with varying levels of planning accuracy to better understand the interplay between planning correctness and user trust. 
Furthermore, we aim to enable real-time user-planner interaction, allowing users to provide feedback and refine plans collaboratively, thereby fostering a more dynamic and user-centric planning process.


\bibliographystyle{plainnat} % Choose your bibliography style (e.g., 'plainnat', 'abbrvnat', etc.)
\bibliography{bibliography} % 'bibliography' is the name of your .bib file (without the extension)
% Note down some papers here:

% --- Appendices (Optional) --- 
% --- Appendices (Optional) ---

% --- Appendix A: Detailed Experimental Setup ---
% \clearpage
\clearpage
\appendix
\onecolumn
% --------------------------------------------------------------------------
% --------------------------------------------------------------------------
\section{Supplementary Material}\label{sec:appendix}

% --------------------------------------------------------------------------
% --------------------------------------------------------------------------

% \subsection{Western United States Learned Masks}\label{sec:masks}

\begin{figure}[h!]
    \centering
    \noindent\includegraphics[width=0.95\textwidth]{supplemental_figures/wus_masks_ltall.png}
    \caption[Western United States learned masks, 1-10 years]{
        Model learned masks for the western United States for lead times 1-10 years.
    }
    \label{fig:wus_all_learned_masks}
\end{figure}

\begin{figure}[h!]
    \centering
    \noindent\includegraphics[width=0.95\textwidth]{supplemental_figures/wus_tlmasks_ltall.png}
    \caption[Western United States transfer learned masks, 1-10 years]{
        Transfer learned masks for the western United States for lead times 1-10 years.
    }
    \label{fig:wus_all_tl_masks}
\end{figure}

% --------------------------------------------------------------------------
% --------------------------------------------------------------------------

% \subsection{Metrics Versus Number of Analogs}\label{sec:number_analogs}

\begin{figure}[h!]
    \centering
    \noindent\includegraphics[width=0.95\textwidth]{supplemental_figures/wus_mse_nana_full.png}
    \caption[MSE versus number of analogs]{
        MSE versus number analogs used to calculate the mean prediction, for all lead times.
        Also shown are the global and regional mask results.
    }
    \label{fig:mse_nana_full}
\end{figure}

\begin{figure}[h!]
    \centering
    \noindent\includegraphics[width=0.95\textwidth]{supplemental_figures/wus_crps_nana_full.png}
    \caption[CRPS versus number of analogs]{
        Same as Figure \ref{fig:mse_nana_full}, but for CRPS.
    }
    \label{fig:crps_nana_full}
\end{figure}

% --------------------------------------------------------------------------
% --------------------------------------------------------------------------

% \subsection{Extra Metrics: EMD and Class Accuracy}\label{sec:extra_metrics}

\begin{figure}[h!]
    \centering
    \noindent\includegraphics[width=0.8\textwidth]{supplemental_figures/wus_emd_lead.png}
    \caption[Earth mover's distance western United States results]{
        EMD versus lead time for the western United States for mean predictions (line plots) and individual analogs (distributions).
        The top two panels show EMD for the time period $1864$-$2023$, while the bottom two panels show EMD for $2009$-$2018$.
    }
    \label{fig:emd_lead}
\end{figure}

\begin{figure}[h!]
    \centering
    \noindent\includegraphics[width=0.8\textwidth]{supplemental_figures/wus_acc4_lead.png}
    \caption[4 class accuracy for western United States]{
        Same as Figure \ref{fig:emd_lead}, but for class accuracy, with four classes.
    }
    \label{fig:acc4_lead}
\end{figure}

\begin{figure}[h!]
    \centering
    \includegraphics[width=0.6\textwidth]{supplemental_figures/regions_emd.png}
    \includegraphics[width=0.6\textwidth]{supplemental_figures/regions_acc4.png}
    \caption[EMD and class accuracy for 5 regions]{
        EMD (top) and class accuracy (bottom, 4 classes).
        The circles are the mean metric covering the time period $1956$-$2023$, the squares cover $1999$-$2018$.
    }
    \label{fig:emdacc_region}
\end{figure}


% --------------------------------------------------------------------------
% --------------------------------------------------------------------------

% \subsection{Prediction Year Versus Analog Year}\label{sec:year_comp}

\begin{figure}[h!]
    \centering
    \includegraphics[width=0.48\textwidth]{supplemental_figures/year_comp_wus5.png}
    \includegraphics[width=0.48\textwidth]{supplemental_figures/year_comp_regional_wus5.png}
    \caption[Analog year versus prediction year]{
        For each prediction year, the range of years for the top ten analogs (top panel) as well as the error over that time (bottom).
        While the model learned mask (left) begins following the one-to-one line (indicating that the selected analog year is the same as the prediction year) by the $1970$s, the regional mask (right) only does so in the final few years.
        In the bottom panels, it can be seen that the model learned mask predictions are not biased toward underestimating or overestimating at any point, while the regional mask predictions tends towards underestimation starting in the $1970$s.
        This is likely due to the regional mask not being able to pick up on the forced warming signal, which may be most apparent outside the target region.
    }
    \label{fig:year_comp}
\end{figure}


% ... Appendix content ...

\end{document}