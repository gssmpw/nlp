\section{Discussion}
\label{sec:discussion}

\subsection{Analysis of GRPO's Impact on Visual Maze Reasoning}
\label{subsec:grpo_impact_analysis}

Our results clearly demonstrate the incremental benefit of Group Relative Policy Optimization (GRPO) in enhancing visual maze reasoning within Large Language Models.  While Supervised Fine-Tuning (SFT) establishes a strong foundation, enabling the model to achieve a \textbf{86\%} accuracy on MazeBench, the application of GRPO further elevates performance to \textbf{93\%} after 1600 training steps. This improvement, albeit seemingly modest in percentage points, is significant considering the already strong baseline established by SFT. It suggests that GRPO is effectively refining the model's policy, leading to more robust and accurate maze navigation.

The qualitative analysis provides further insight into the nature of this improvement.  The \textbf{AlphaMaze-SFT+GRPO} model exhibited more pronounced chain-of-thought reasoning patterns and instances of self-correction, indicating that GRPO is not merely fine-tuning the existing SFT policy, but rather encouraging more sophisticated reasoning processes. The reward function, designed to incentivize correctness, valid movements, and structured output, likely plays a crucial role in shaping this behavior. By rewarding successful navigation and penalizing invalid steps, GRPO encourages the model to learn more deliberate and considered movement strategies.

\subsection{Comparison with DeepSeek-R1 and RL for Reasoning}
\label{subsec:comparison_deepseek_r1}

It is important to note that the base DeepSeek-R1 model, when operating with an extremely long context window, demonstrates emergent visual reasoning capabilities. However, our experiments reveal that the distilled variants (DeepSeek-R1 Distill-Qwen models) do not carry over these spatial reasoning abilities, as evidenced by their \textbf{0\%} accuracy on MazeBench. This suggests that the distillation process into Qwen or other smaller models is insufficient to preserve the emergent ability of visual spatial reasoning observed in the base model.

In contrast, our two-stage training approach—combining Supervised Fine-Tuning (SFT) to establish foundational step-by-step reasoning with Group Relative Policy Optimization (GRPO) for further refinement—effectively equips the distilled model with robust visual maze-solving skills. Even with only 1600 GRPO steps, the model achieves a notable improvement, reaching \textbf{93\%} accuracy and exhibiting clear chain-of-thought behaviors along with self-correction during navigation.

These findings underscore the necessity of specialized training to recover or enhance spatial reasoning in distilled models, highlighting that while the base DeepSeek-R1 model is capable of visual reasoning with sufficient context, additional training stages are crucial to maintain or induce this capability in smaller, distilled variants.

\subsection{Limitations}
\label{subsec:limitations}

Despite the encouraging results, our study is not without limitations.  Firstly, the performance gain from GRPO, while statistically significant, is small (7\% accuracy improvement in our reported experiment).  Further investigation is needed to explore whether more extensive GRPO training, or modifications to the reward function, could lead to more substantial performance gains.  It is possible that the current reward function, while effective, could be further optimized to better incentivize more complex reasoning strategies, such as backtracking or more proactive exploration of alternative paths.

Secondly, our evaluation, while including qualitative analysis, is primarily based on maze-solving accuracy.  This metric, while important, provides a somewhat limited view of the model's reasoning capabilities.  Future work could benefit from more nuanced evaluation metrics that assess the efficiency of the generated paths, the robustness of the model to maze complexity variations, and the interpretability of the model's internal reasoning process.  Furthermore, while we observed qualitative signs of chain-of-thought reasoning, a more rigorous analysis, perhaps using techniques from interpretability research, is needed to definitively characterize the nature and depth of the model's reasoning process.

Finally, our experiments are limited to synthetically generated mazes.  While these mazes were designed to vary in size and complexity, they may not fully capture the intricacies and variability of real-world visual spatial reasoning tasks.  Future research should explore the generalizability of our approach to more diverse and ecologically valid visual environments and tasks.