% !TeX root = ../all.tex

\section{Perspectives}

We proved instance-dependent lower bounds for the batch complexity of any $\delta$-correct pure exploration algorithm.
These lower bounds get larger as the sample complexity of the algorithm decreases.
We introduced the Phased Explore then Track algorithm, for which we proved an upper bound for the sample complexity close to the lower bound for fully adaptive methods, as well as an upper bound for the batch complexity that is close to our lower bound.

The main open question raised by our work is how to explore in a better way than uniformly for a general pure exploration problem.
The goal of the exploration is to find a set $\hat{B}_r$ such that the worst case complexity $\overline{T}^\star(\hat{B}_r)$ is close to $T^\star(\bm\mu)$, which means estimating both $T^\star(\bm\mu)$ and $w^\star(\bm\mu)$.
Uniform exploration leads to a close to optimal batch complexity and to a sample complexity which has a good dependence in $\log(1/\delta)$.
However, it comes at the cost of $K T^\star(\bm\mu)$ samples used to explore every arm, which should ideally be around $T^\star(\bm\mu)$ instead.
For BAI and Top-k, the elimination strategies of \citep{hillelDistributedExplorationMultiArmed2013,jinOptimalBatchedBest2023} achieve that improvement, but the elimination criterion uses the particular link between the gaps and $T^\star$ in Top-k.
We would need to find a way to extend the elimination approach to other problems, for which there might not even be an obvious notion of gaps.
%It would be interesting to investigate if that adaptive exploration over $K$ arms can be done without introducing a dependence in $K$ in the batch complexity.
%In particular, if $T^\star(\bm\mu)$ is known in advance, our current algorithm uses at most 5 batches. Can we explore more adaptively in so few batches?
