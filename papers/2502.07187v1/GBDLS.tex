\section{GBDLS Classes}\label{Section:GBDLS}

We devote this section to the development of two properties (of a hypothesis class) which will play a central role in the proof of \cref{Theorem:local-regularization-fails}: that of being \emph{generalized binary} and of having \emph{distinct label sets}. Upon defining each such property (of an underlying hypothesis class), we will demonstrate that neither is individually sufficient to ensure learnability. However, we show that all classes which have both properties --- termed \emph{generalized binary with distinct label sets} (GBDLS) --- are necessarily learnable. Neither observation is particularly difficult to prove, but serves towards the eventual proof of \cref{Theorem:local-regularization-fails}, and we believe that the language of such hypothesis classes may be of independent interest. 

\begin{definition}
A hypothesis $h \colon \CX \to \CY$ is \defn{k-ary} if $|\im(h)| \leq k$. When $k = 2$, we say that $h$ is a binary hypothesis. 
\end{definition}

\begin{definition}
A hypothesis class $\CH \subseteq \CY^\CX$ is \defn{k-ary} if $|\bigcup_{h \in \CH} \im(h)| \leq k$. When $k = 2$, we say that $\CH$ is binary. 
\end{definition}

We have slightly overloaded the term \emph{k-ary}, but there should be little risk of confusion: a hypothesis $h$ is $k$-ary if and only if the class $\{h\}$ is $k$-ary. 

\begin{definition}
A hypothesis class is \defn{generalized k-ary} if each $h \in \CH$ is a $k$-ary hypothesis. When $k = 2$, we say that $\CH$ is generalized binary.
\end{definition}

\begin{definition}
A hypothesis class $\CH$ \defn{has distinct label sets} if $\im \colon \CH \to 2^{\CY}$ is an injection. That is, $\im(h) \neq \im(h')$ when $h \neq h'$ and $h, h' \in \CH$. 
\end{definition}

It is not difficult to exhibit counter-examples demonstrating that neither the property of being generalized binary nor the property of having distinct label sets is sufficient to ensure that a hypothesis class $\CH$ is learnable. For the former, take any binary class of infinite VC dimension. For the latter, take any nonlearnable class $\CH \subseteq \CY^\CX$ and expand each of its hypotheses $h \in \CH$ to be defined on an additional unlabeled datapoint $*$ by $h(*) = h|_{\CX}$. 

However, we refer to classes with both properties as being \emph{generalized binary with distinct label sets} (GBDLS), and we now demonstrate that GBDLS classes are always learnable. 

\begin{proposition}\label{Proposition:GBDLS-means-learnable}
Let $\CH \subseteq \CY^\CX$ be a GBDLS hypothesis class. Then $\CH$ is learnable, in both the PAC and transductive models. 
\end{proposition}
\begin{proof}
We will demonstrate that $\CH$ cannot DS-shatter any sequence of 3 distinct points, from which it follows that $\DS(\CH) \leq 2$ and thus that $\CH$ is learnable in both models \citep{brukhim2022characterization}. Fix any sequence of distinct points $S = (x_1, x_2, x_3) \subseteq \CX$, and let $\CF \subseteq \CH|_S$ be a finite non-empty set of behaviors of $\CH$ on $S$. It remains to show that there exists an $f \in \CF$ and $i \in [3]$ for which $f$ does not have an $i$-neighbor in $\CF$ (i.e., a $g \in \CF$ with $g|_{S \setminus \{x_i\}} = f|_{S \setminus \{x_i\}}$ and $g(i) \neq f(i)$). 

To this end, select an arbitrary behavior $h \in \CF$, and identify it with the sequence $(h(x_j))_{j \in [3]} \subseteq \CY$. Suppose that $h$ is a constant behavior, i.e., that $h = (a, a, a)$ for a label $a \in \CY$. If $h$ does not have a 3-neighbor $f \in \CF$, then we are done. Otherwise, consider the 3-neighbor $f = (a, a, b)$ of $h \in \CF$. Note that if $h$ had been a non-constant behavior, then it would have taken the form $h = (a, a, b)$ to begin with (up to re-ordering of the points $(x_1, x_2, x_3)$, owing to the fact that $h$ is a binary behavior). We may thus consider a behavior $f = (a, a, b) \in \CF$ where $a \neq b$, without loss of generality. Then it follows immediately that $f$ does not have a 1-neighbor $g \in \CF$. Any such $g$ must take the form $g = (b, a, b)$, as $\CH$ is generalized binary, and this yields that $\im(f) = \{a, b\} = \im(g)$, producing contradiction with the fact that $\CH$ has distinct label sets. 
\end{proof}
