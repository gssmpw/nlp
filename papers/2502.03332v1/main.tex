%%%%%%%% ICML 2025 EXAMPLE LATEX SUBMISSION FILE %%%%%%%%%%%%%%%%%
\documentclass{article}

% Recommended, but optional, packages for figures and better typesetting:
\usepackage{microtype}
\usepackage{graphicx}
\usepackage{subfigure}
% \usepackage[ruled,vlined]{algorithm2e}
\usepackage{booktabs} % for professional tables

% hyperref makes hyperlinks in the resulting PDF.
% If your build breaks (sometimes temporarily if a hyperlink spans a page)
% please comment out the following usepackage line and replace
% \usepackage{icml2025} with \usepackage[nohyperref]{icml2025} above.
\usepackage{hyperref}


% Attempt to make hyperref and algorithmic work together better:
\newcommand{\theHalgorithm}{\arabic{algorithm}}

% Use the following line for the initial blind version submitted for review:
\usepackage[accepted]{icml2025}

% If accepted, instead use the following line for the camera-ready submission:
% \usepackage[accepted]{icml2025}

% Todonotes is useful during development; simply uncomment the next line
%    and comment out the line below the next line to turn off comments
%\usepackage[disable,textsize=tiny]{todonotes}
\usepackage[textsize=tiny]{todonotes}
\definecolor{babypink}{rgb}{0.96, 0.76, 0.76} 
\definecolor{burntsienna}{rgb}{0.91, 0.45, 0.32}     % colors
\definecolor{crimson}{rgb}{0.86, 0.08, 0.24}
\definecolor{darkspringgreen}{rgb}{0.09, 0.45, 0.27}
\definecolor{deepcarrotorange}{rgb}{0.91, 0.41, 0.17}
%%%%%%%%%%%%%
% additional pacakge
\usepackage[utf8]{inputenc} % allow utf-8 input
\usepackage[T1]{fontenc}    % use 8-bit T1 fonts
\usepackage{hyperref}       % hyperlinks
\usepackage{nccmath}
\hypersetup{
     colorlinks=true,
     linkcolor=blue,
     filecolor=blue,
     citecolor=blue,
     urlcolor=blue,
    }
\usepackage{url}
\usepackage{wrapfig}
\usepackage{colortbl}
\usepackage{booktabs}       % professional-quality tables
\usepackage{amsfonts}       % blackboard math symbols
\usepackage{nicefrac}       % compact symbols for 1/2, etc.
\usepackage{microtype}      % microtypography
\usepackage{xcolor}
\usepackage{multicol}
\usepackage{graphicx}
\usepackage{transparent}
\usepackage{amsthm}
\usepackage{bbm}
\usepackage{comment}
\usepackage{enumitem}
\usepackage{bm}
\usepackage{subfigure}
\usepackage{mathtools}
\usepackage[export]{adjustbox}
\usepackage{stmaryrd}
% 
\usepackage{times}
\usepackage{latexsym}

\usepackage[T1]{fontenc}

\usepackage[utf8]{inputenc}

\usepackage{microtype}

\usepackage{inconsolata}

\usepackage{graphicx}


\usepackage{amsmath}
\usepackage{amssymb}
\usepackage{multirow}
\usepackage{booktabs}
\usepackage{catchfile}

\usepackage[boxed]{algorithm}
\usepackage{varwidth}
\usepackage[noEnd=true,indLines=false]{algpseudocodex}
\usepackage{cleveref}
\makeatletter
\@addtoreset{ALG@line}{algorithm}
\renewcommand{\ALG@beginalgorithmic}{\small}
\algrenewcommand\alglinenumber[1]{\small #1:}
\makeatother

\usepackage[normalem]{ulem}
\usepackage{todonotes}

\usepackage{lipsum}    %
\usepackage{comment}   %
\usepackage{graphicx}  %
\usepackage{pifont}    %

\usepackage[font=small,labelfont=bf]{caption}
\usepackage{float}     %
\usepackage{booktabs}  %
\usepackage{subcaption}  %

\usepackage{listings}

\usepackage{amsthm}  %

\usepackage{xifthen}
\usepackage{xargs}
\usepackage{amsmath,amssymb}
\usepackage{algorithm,algorithmic}
\usepackage{hyperref}
\usepackage{caption}
\usepackage{titlesec}
\usepackage{multirow}
% \usepackage[hypertexnames=false]{hyperref}
% if you use cleveref..
\usepackage[capitalize,noabbrev]{cleveref}

%%%%%%%%%%%%%%%%%%%
% custom notations & commands
% Lists
\usepackage[inline]{enumitem}
\newlist{enuminline}{enumerate*}{1}
\setlist[enuminline]{label=(\roman*)}

% Math
\newcommand{\vol}{\mathrm{vol}}
\newcommand{\E}{\mathbb E}
\newcommand{\var}{\mathrm{var}}
\newcommand{\cov}{\mathrm{cov}}
\newcommand{\Normal}{\mathcal N}
\newcommand{\slowvar}{\mathcal L}
\newcommand{\bbR}{\mathbb R}
\newcommand{\bbE}{\mathbb E}
\newcommand{\bbP}{\mathbb P}
\newcommand{\calC}{\mathcal C}
\newcommand{\calR}{\mathcal R}
\newcommand{\calT}{\mathcal T}
\newcommand{\loA}{\underline{A}}
\newcommand{\upA}{\overline{A}} 
\newcommand{\cv}{\mathrm{cv}} 
\newcommand{\pp}{\mathrm{pp}}
\newcommand{\HS}{\mathrm{HS}}
\newcommand{\erfi}{\mathrm{erfi}}
\newcommand{\tX}{\widetilde X}
\newcommand{\OneFOne}{{}_1F_1}
\newcommand{\PP}{\mathrm{\texttt{PP}}}
\newcommand{\PPpp}{\mathrm{\texttt{PP+}}}
\newcommand{\BPP}{\mathrm{\texttt{BPP}}}
\newcommand{\BPPpp}{\mathrm{\texttt{BPP+}}}
\newcommand{\FABPPI}{\mathrm{\texttt{FABPP}}}
\newcommand{\asto}{\overset{\text{a.s.}}{\to}}

%%%%%%%%%%%%%%%%%%%%%%%%%%%%%%%%
% THEOREMS
%%%%%%%%%%%%%%%%%%%%%%%%%%%%%%%%
\theoremstyle{plain}
\newtheorem{theorem}{Theorem}[section]
\newtheorem{proposition}[theorem]{Proposition}
\newtheorem{lemma}[theorem]{Lemma}
\newtheorem{corollary}[theorem]{Corollary}
\theoremstyle{definition}
\newtheorem{definition}[theorem]{Definition}
\newtheorem{assumption}[theorem]{Assumption}
\theoremstyle{remark}
\newtheorem{remark}[theorem]{Remark}
\newcommand\plabel[1]{\phantomsection\label{#1}}


\begin{document}

\twocolumn[
\icmltitle{A Mixture-Based Framework for Guiding Diffusion Models}

\icmlsetsymbol{equal}{*}

\begin{icmlauthorlist}
    \icmlauthor{Yazid Janati}{equal,yyy}
    \icmlauthor{Badr Moufad}{equal,yyy}
    \icmlauthor{Mehdi Abou El Qassime}{yyy}\\
    \icmlauthor{Alain Durmus}{yyy}
    \icmlauthor{Eric Moulines}{yyy}
    \icmlauthor{Jimmy Olsson}{sch}
    %\icmlauthor{}{sch}
    %\icmlauthor{}{sch}
\end{icmlauthorlist}

\icmlaffiliation{yyy}{Ecole polytechnique}
% \icmlaffiliation{comp}{Company Name, Location, Country}
\icmlaffiliation{sch}{KTH University}

\icmlcorrespondingauthor{Yazid Janati, Badr Moufad}{first.last@polytechnique.edu}
% \icmlcorrespondingauthor{Badr Moufad}{}

% You may provide any keywords that you
% find helpful for describing your paper; these are used to populate
% the "keywords" metadata in the PDF but will not be shown in the document
\icmlkeywords{Machine Learning, ICML}

\vskip 0.3in
]

% this must go after the closing bracket ] following \twocolumn[ ...

% This command actually creates the footnote in the first column
% listing the affiliations and the copyright notice.
% The command takes one argument, which is text to display at the start of the footnote.
% The \icmlEqualContribution command is standard text for equal contribution.
% Remove it (just {}) if you do not need this facility.

%\printAffiliationsAndNotice{}  % leave blank if no need to mention equal contribution
\printAffiliationsAndNotice{\icmlEqualContribution} % otherwise use the standard text.

\begin{abstract}
  Denoising diffusion models have driven significant progress in the field of Bayesian inverse problems. Recent approaches use pre-trained diffusion models as priors to solve a wide range of such problems, only leveraging inference-time compute and thereby eliminating the need to retrain task-specific models on the same dataset. To approximate the posterior of a Bayesian inverse problem, a diffusion model samples from a sequence of intermediate posterior distributions, each with an intractable likelihood function. This work proposes a novel mixture approximation of these intermediate distributions. Since direct gradient-based sampling of these mixtures is infeasible due to intractable terms, we propose a practical method based on Gibbs sampling. We validate our approach through extensive experiments on image inverse problems, utilizing both pixel- and latent-space diffusion priors, as well as on source separation with an audio diffusion model. The code is available at \url{https://www.github.com/badr-moufad/mgdm}.
\end{abstract}


%%%%%%%%%%%%
%%%%%%%%%%%%
\section{Introduction}
Inverse problems occur when a signal $X$ of interest must be inferred from an incomplete and noisy observation $Y$, a challenge frequently encountered in diverse fields such as weather forecasting, image reconstruction (\emph{e.g.}, tomography or black-hole imaging), and speech processing.  Such problems are typically ill-posed, 
%as the observations can correspond to infinitely many possible signals. However, most of these solutions are either physically implausible or lack practical relevance, 
making it essential to incorporate additional constraints, regularization techniques, or prior knowledge to arrive at meaningful and realistic solutions. 

The Bayesian framework, in conjunction with generative modeling, offers a systematic approach to the challenges associated with inverse problems. Prior knowledge about the signal of interest, often represented through samples from its underlying distribution $\pdata{0}{}{}$, can be leveraged to train a generative model $\pdata{0}{}{}[\param]$ that acts as a prior. 
%The a priori knowledge about the signal of interest, usually materialized with samples from its underlying distribution $\pdata{0}{}{}$, can be used to learn a generative model $\pdata{0}{}{}[\param]$ that serves as prior. 
By combining it with the conditional density $\pot{0}{\bx}$ of the observation given the signal, deduced from the form of the inverse problem at hand, we can compute the posterior distribution. 
Samples drawn from this posterior encapsulate plausible solutions that harmonize prior knowledge with the observed data.
%Samples drawn from this posterior represent plausible solutions that reconcile both the prior knowledge and the observed data. 
One straightforward approach to approximate sampling from the posterior distribution involves constructing a paired dataset of i.i.d. signals and observations, $(\bX_i, Y_i)_{i = 1}^N$, where $\bX_i \sim \pdata{0}{}{}$ and $Y_i \sim g_0(\cdot | \bX_i)$, and learning a direct mapping \cite{dong2015image} or generative model \cite{ledig2017photo,isola2017image}. The latter, when queried with multiple independent noise samples alongside an observation, %$\obs$, 
generates a diverse set of potential reconstructions. However, this approach is inherently \emph{task-specific}, delivering reliable reconstructions only when the conditional distribution of the observation remains unchanged at test time. As a result, it cannot straightforwardly adapt to unseen tasks with the same prior. Adaptation to a new task can only be achieved by retraining a new generative model. 

An increasingly popular approach consists in learning a generative model only for the prior $\pdata{0}{}{}$, and then leveraging inference-time compute to solve any inverse problem for which the likelihood function $\bx \mapsto \pot{0}{\bx}$ is provided in a closed form. This strategy eliminates the need for expensive and inefficient task-specific training. Initially explored with generative models such as variational autoencoders and generative adversarial networks \cite{xia2022gan}, this framework has recently been extended to denoising diffusion models (DDMs) \cite{song2021score,kadkhodaie2020solving,kawar2021snips,kawar2022denoising,chung2023diffusion,song2022pseudoinverse,daras2024survey}, which are the focus of the present paper.

DDMs \cite{sohl2015deep,song2019generative,ho2020denoising} achieve state-of-the-art generative performance across a wide range of domains. At their core is a forward noising process that transforms the data distribution $\pdata{0}{}{}$ into a Gaussian distribution. A generative model is then learned by  reversing this noising process. With a specific parameterization of the backward process, which converts noise into data samples, training the generative model reduces to approximating denoisers for each noise level introduced during the forward process. Recent methods for training-free posterior sampling aim to approximate the denoisers for the posterior distribution, enabling the use of diffusion models for sampling \cite{ho2022video,chung2023diffusion,song2022pseudoinverse}. A posterior distribution denoiser can be decomposed into two terms: the prior denoiser at the same noise level (provided by a pre-trained diffusion model) and the gradient of the log-likelihood of the observation conditioned on the current noisy sample. The latter term, which is intractable, is what guides the samples during the denoising process towards the posterior distribution. Various approximations for this gradient term have been proposed. However, they are often crude and require significant adjustments and heuristics to ensure stability and satisfactory performance. When applied to latent diffusion models, they often demand additional, model-specific adjustments 
\cite{rout2024solving}.\\


\textbf{Our contribution.}\, In this paper, we present a principled method that circumvents these issues by introducing a new approximation of the likelihood term, paired with a sampling scheme based on Gibbs sampling \cite{geman1984stochastic}. 
Our key observation is that multiple approximations can be derived for each likelihood term at a fixed noise level using a simple identity that it satisfies.
However, the scores of these new likelihood approximations are not available in closed form, preventing us from deriving a direct posterior denoiser approximation by combining, through a mixture, the different likelihood approximations. We overcome this limitation by constructing a mixture approximation of the intermediate posterior distributions defined by the diffusion model for the original posterior. 
Our algorithm,  \algoname\ (\algo), proceeds by sequentially sampling from these mixtures using Gibbs sampling. This is enabled by a carefully designed data augmentation scheme that ensures straightforward Gibbs updates. A key advantage of our approach is its adaptability to available computational resources. Specifically, the number of Gibbs iterations acts as a tunable parameter, allowing substantial improvements with increased inference-time compute. \algo\ demonstrates strong empirical performance across 10 image-restoration tasks involving both pixel-space and latent-space diffusion models, as well as in musical source separation, even matching the performance of supervised methods. 
%A key advantage of our approach is its flexibility in scaling with available computational resources. Specifically, the number of Gibbs iterations serves as a tunable parameter: by increasing the number of iterations, the algorithm can improve its performance when more inference-time compute is available. 
%The strong empirical performance of \algo\ is demonstrated on 10 image-restoration tasks, with both pixel-space and latent-space diffusion models, as well as musical source separation, and is further validated through comparisons with 10 competing methods.

\section{Background}
\section{Background}\label{sec:backgrnd}

\subsection{Cold Start Latency and Mitigation Techniques}

Traditional FaaS platforms mitigate cold starts through snapshotting, lightweight virtualization, and warm-state management. Snapshot-based methods like \textbf{REAP} and \textbf{Catalyzer} reduce initialization time by preloading or restoring container states but require significant memory and I/O resources, limiting scalability~\cite{dong_catalyzer_2020, ustiugov_benchmarking_2021}. Lightweight virtualization solutions, such as \textbf{Firecracker} microVMs, achieve fast startup times with strong isolation but depend on robust infrastructure, making them less adaptable to fluctuating workloads~\cite{agache_firecracker_2020}. Warm-state management techniques like \textbf{Faa\$T}~\cite{romero_faa_2021} and \textbf{Kraken}~\cite{vivek_kraken_2021} keep frequently invoked containers ready, balancing readiness and cost efficiency under predictable workloads but incurring overhead when demand is erratic~\cite{romero_faa_2021, vivek_kraken_2021}. While these methods perform well in resource-rich cloud environments, their resource intensity challenges applicability in edge settings.

\subsubsection{Edge FaaS Perspective}

In edge environments, cold start mitigation emphasizes lightweight designs, resource sharing, and hybrid task distribution. Lightweight execution environments like unikernels~\cite{edward_sock_2018} and \textbf{Firecracker}~\cite{agache_firecracker_2020}, as used by \textbf{TinyFaaS}~\cite{pfandzelter_tinyfaas_2020}, minimize resource usage and initialization delays but require careful orchestration to avoid resource contention. Function co-location, demonstrated by \textbf{Photons}~\cite{v_dukic_photons_2020}, reduces redundant initializations by sharing runtime resources among related functions, though this complicates isolation in multi-tenant setups~\cite{v_dukic_photons_2020}. Hybrid offloading frameworks like \textbf{GeoFaaS}~\cite{malekabbasi_geofaas_2024} balance edge-cloud workloads by offloading latency-tolerant tasks to the cloud and reserving edge resources for real-time operations, requiring reliable connectivity and efficient task management. These edge-specific strategies address cold starts effectively but introduce challenges in scalability and orchestration.

\subsection{Predictive Scaling and Caching Techniques}

Efficient resource allocation is vital for maintaining low latency and high availability in serverless platforms. Predictive scaling and caching techniques dynamically provision resources and reduce cold start latency by leveraging workload prediction and state retention.
Traditional FaaS platforms use predictive scaling and caching to optimize resources, employing techniques (OFC, FaasCache) to reduce cold starts. However, these methods rely on centralized orchestration and workload predictability, limiting their effectiveness in dynamic, resource-constrained edge environments.



\subsubsection{Edge FaaS Perspective}

Edge FaaS platforms adapt predictive scaling and caching techniques to constrain resources and heterogeneous environments. \textbf{EDGE-Cache}~\cite{kim_delay-aware_2022} uses traffic profiling to selectively retain high-priority functions, reducing memory overhead while maintaining readiness for frequent requests. Hybrid frameworks like \textbf{GeoFaaS}~\cite{malekabbasi_geofaas_2024} implement distributed caching to balance resources between edge and cloud nodes, enabling low-latency processing for critical tasks while offloading less critical workloads. Machine learning methods, such as clustering-based workload predictors~\cite{gao_machine_2020} and GRU-based models~\cite{guo_applying_2018}, enhance resource provisioning in edge systems by efficiently forecasting workload spikes. These innovations effectively address cold start challenges in edge environments, though their dependency on accurate predictions and robust orchestration poses scalability challenges.

\subsection{Decentralized Orchestration, Function Placement, and Scheduling}

Efficient orchestration in serverless platforms involves workload distribution, resource optimization, and performance assurance. While traditional FaaS platforms rely on centralized control, edge environments require decentralized and adaptive strategies to address unique challenges such as resource constraints and heterogeneous hardware.



\subsubsection{Edge FaaS Perspective}

Edge FaaS platforms adopt decentralized and adaptive orchestration frameworks to meet the demands of resource-constrained environments. Systems like \textbf{Wukong} distribute scheduling across edge nodes, enhancing data locality and scalability while reducing network latency. Lightweight frameworks such as \textbf{OpenWhisk Lite}~\cite{kravchenko_kpavelopenwhisk-light_2024} optimize resource allocation by decentralizing scheduling policies, minimizing cold starts and latency in edge setups~\cite{benjamin_wukong_2020}. Hybrid solutions like \textbf{OpenFaaS}~\cite{noauthor_openfaasfaas_2024} and \textbf{EdgeMatrix}~\cite{shen_edgematrix_2023} combine edge-cloud orchestration to balance resource utilization, retaining latency-sensitive functions at the edge while offloading non-critical workloads to the cloud. While these approaches improve flexibility, they face challenges in maintaining coordination and ensuring consistent performance across distributed nodes.


\section{Method}\label{sec:method}
\begin{figure}
    \centering
    \includegraphics[width=0.85\textwidth]{imgs/heatmap_acc.pdf}
    \caption{\textbf{Visualization of the proposed periodic Bayesian flow with mean parameter $\mu$ and accumulated accuracy parameter $c$ which corresponds to the entropy/uncertainty}. For $x = 0.3, \beta(1) = 1000$ and $\alpha_i$ defined in \cref{appd:bfn_cir}, this figure plots three colored stochastic parameter trajectories for receiver mean parameter $m$ and accumulated accuracy parameter $c$, superimposed on a log-scale heatmap of the Bayesian flow distribution $p_F(m|x,\senderacc)$ and $p_F(c|x,\senderacc)$. Note the \emph{non-monotonicity} and \emph{non-additive} property of $c$ which could inform the network the entropy of the mean parameter $m$ as a condition and the \emph{periodicity} of $m$. %\jj{Shrink the figures to save space}\hanlin{Do we need to make this figure one-column?}
    }
    \label{fig:vmbf_vis}
    \vskip -0.1in
\end{figure}
% \begin{wrapfigure}{r}{0.5\textwidth}
%     \centering
%     \includegraphics[width=0.49\textwidth]{imgs/heatmap_acc.pdf}
%     \caption{\textbf{Visualization of hyper-torus Bayesian flow based on von Mises Distribution}. For $x = 0.3, \beta(1) = 1000$ and $\alpha_i$ defined in \cref{appd:bfn_cir}, this figure plots three colored stochastic parameter trajectories for receiver mean parameter $m$ and accumulated accuracy parameter $c$, superimposed on a log-scale heatmap of the Bayesian flow distribution $p_F(m|x,\senderacc)$ and $p_F(c|x,\senderacc)$. Note the \emph{non-monotonicity} and \emph{non-additive} property of $c$. \jj{Shrink the figures to save space}}
%     \label{fig:vmbf_vis}
%     \vspace{-30pt}
% \end{wrapfigure}


In this section, we explain the detailed design of CrysBFN tackling theoretical and practical challenges. First, we describe how to derive our new formulation of Bayesian Flow Networks over hyper-torus $\mathbb{T}^{D}$ from scratch. Next, we illustrate the two key differences between \modelname and the original form of BFN: $1)$ a meticulously designed novel base distribution with different Bayesian update rules; and $2)$ different properties over the accuracy scheduling resulted from the periodicity and the new Bayesian update rules. Then, we present in detail the overall framework of \modelname over each manifold of the crystal space (\textit{i.e.} fractional coordinates, lattice vectors, atom types) respecting \textit{periodic E(3) invariance}. 

% In this section, we first demonstrate how to build Bayesian flow on hyper-torus $\mathbb{T}^{D}$ by overcoming theoretical and practical problems to provide a low-noise parameter-space approach to fractional atom coordinate generation. Next, we present how \modelname models each manifold of crystal space respecting \textit{periodic E(3) invariance}. 

\subsection{Periodic Bayesian Flow on Hyper-torus \texorpdfstring{$\mathbb{T}^{D}$}{}} 
For generative modeling of fractional coordinates in crystal, we first construct a periodic Bayesian flow on \texorpdfstring{$\mathbb{T}^{D}$}{} by designing every component of the totally new Bayesian update process which we demonstrate to be distinct from the original Bayesian flow (please see \cref{fig:non_add}). 
 %:) 
 
 The fractional atom coordinate system \citep{jiao2023crystal} inherently distributes over a hyper-torus support $\mathbb{T}^{3\times N}$. Hence, the normal distribution support on $\R$ used in the original \citep{bfn} is not suitable for this scenario. 
% The key problem of generative modeling for crystal is the periodicity of Cartesian atom coordinates $\vX$ requiring:
% \begin{equation}\label{eq:periodcity}
% p(\vA,\vL,\vX)=p(\vA,\vL,\vX+\vec{LK}),\text{where}~\vec{K}=\vec{k}\vec{1}_{1\times N},\forall\vec{k}\in\mathbb{Z}^{3\times1}
% \end{equation}
% However, there does not exist such a distribution supporting on $\R$ to model such property because the integration of such distribution over $\R$ will not be finite and equal to 1. Therefore, the normal distribution used in \citet{bfn} can not meet this condition.

To tackle this problem, the circular distribution~\citep{mardia2009directional} over the finite interval $[-\pi,\pi)$ is a natural choice as the base distribution for deriving the BFN on $\mathbb{T}^D$. 
% one natural choice is to 
% we would like to consider the circular distribution over the finite interval as the base 
% we find that circular distributions \citep{mardia2009directional} defined on a finite interval with lengths of $2\pi$ can be used as the instantiation of input distribution for the BFN on $\mathbb{T}^D$.
Specifically, circular distributions enjoy desirable periodic properties: $1)$ the integration over any interval length of $2\pi$ equals 1; $2)$ the probability distribution function is periodic with period $2\pi$.  Sharing the same intrinsic with fractional coordinates, such periodic property of circular distribution makes it suitable for the instantiation of BFN's input distribution, in parameterizing the belief towards ground truth $\x$ on $\mathbb{T}^D$. 
% \yuxuan{this is very complicated from my perspective.} \hanlin{But this property is exactly beautiful and perfectly fit into the BFN.}

\textbf{von Mises Distribution and its Bayesian Update} We choose von Mises distribution \citep{mardia2009directional} from various circular distributions as the form of input distribution, based on the appealing conjugacy property required in the derivation of the BFN framework.
% to leverage the Bayesian conjugacy property of von Mises distribution which is required by the BFN framework. 
That is, the posterior of a von Mises distribution parameterized likelihood is still in the family of von Mises distributions. The probability density function of von Mises distribution with mean direction parameter $m$ and concentration parameter $c$ (describing the entropy/uncertainty of $m$) is defined as: 
\begin{equation}
f(x|m,c)=vM(x|m,c)=\frac{\exp(c\cos(x-m))}{2\pi I_0(c)}
\end{equation}
where $I_0(c)$ is zeroth order modified Bessel function of the first kind as the normalizing constant. Given the last univariate belief parameterized by von Mises distribution with parameter $\theta_{i-1}=\{m_{i-1},\ c_{i-1}\}$ and the sample $y$ from sender distribution with unknown data sample $x$ and known accuracy $\alpha$ describing the entropy/uncertainty of $y$,  Bayesian update for the receiver is deducted as:
\begin{equation}
 h(\{m_{i-1},c_{i-1}\},y,\alpha)=\{m_i,c_i \}, \text{where}
\end{equation}
\begin{equation}\label{eq:h_m}
m_i=\text{atan2}(\alpha\sin y+c_{i-1}\sin m_{i-1}, {\alpha\cos y+c_{i-1}\cos m_{i-1}})
\end{equation}
\begin{equation}\label{eq:h_c}
c_i =\sqrt{\alpha^2+c_{i-1}^2+2\alpha c_{i-1}\cos(y-m_{i-1})}
\end{equation}
The proof of the above equations can be found in \cref{apdx:bayesian_update_function}. The atan2 function refers to  2-argument arctangent. Independently conducting  Bayesian update for each dimension, we can obtain the Bayesian update distribution by marginalizing $\y$:
\begin{equation}
p_U(\vtheta'|\vtheta,\bold{x};\alpha)=\mathbb{E}_{p_S(\bold{y}|\bold{x};\alpha)}\delta(\vtheta'-h(\vtheta,\bold{y},\alpha))=\mathbb{E}_{vM(\bold{y}|\bold{x},\alpha)}\delta(\vtheta'-h(\vtheta,\bold{y},\alpha))
\end{equation} 
\begin{figure}
    \centering
    \vskip -0.15in
    \includegraphics[width=0.95\linewidth]{imgs/non_add.pdf}
    \caption{An intuitive illustration of non-additive accuracy Bayesian update on the torus. The lengths of arrows represent the uncertainty/entropy of the belief (\emph{e.g.}~$1/\sigma^2$ for Gaussian and $c$ for von Mises). The directions of the arrows represent the believed location (\emph{e.g.}~ $\mu$ for Gaussian and $m$ for von Mises).}
    \label{fig:non_add}
    \vskip -0.15in
\end{figure}
\textbf{Non-additive Accuracy} 
The additive accuracy is a nice property held with the Gaussian-formed sender distribution of the original BFN expressed as:
\begin{align}
\label{eq:standard_id}
    \update(\parsn{}'' \mid \parsn{}, \x; \alpha_a+\alpha_b) = \E_{\update(\parsn{}' \mid \parsn{}, \x; \alpha_a)} \update(\parsn{}'' \mid \parsn{}', \x; \alpha_b)
\end{align}
Such property is mainly derived based on the standard identity of Gaussian variable:
\begin{equation}
X \sim \mathcal{N}\left(\mu_X, \sigma_X^2\right), Y \sim \mathcal{N}\left(\mu_Y, \sigma_Y^2\right) \Longrightarrow X+Y \sim \mathcal{N}\left(\mu_X+\mu_Y, \sigma_X^2+\sigma_Y^2\right)
\end{equation}
The additive accuracy property makes it feasible to derive the Bayesian flow distribution $
p_F(\boldsymbol{\theta} \mid \mathbf{x} ; i)=p_U\left(\boldsymbol{\theta} \mid \boldsymbol{\theta}_0, \mathbf{x}, \sum_{k=1}^{i} \alpha_i \right)
$ for the simulation-free training of \cref{eq:loss_n}.
It should be noted that the standard identity in \cref{eq:standard_id} does not hold in the von Mises distribution. Hence there exists an important difference between the original Bayesian flow defined on Euclidean space and the Bayesian flow of circular data on $\mathbb{T}^D$ based on von Mises distribution. With prior $\btheta = \{\bold{0},\bold{0}\}$, we could formally represent the non-additive accuracy issue as:
% The additive accuracy property implies the fact that the "confidence" for the data sample after observing a series of the noisy samples with accuracy ${\alpha_1, \cdots, \alpha_i}$ could be  as the accuracy sum  which could be  
% Here we 
% Here we emphasize the specific property of BFN based on von Mises distribution.
% Note that 
% \begin{equation}
% \update(\parsn'' \mid \parsn, \x; \alpha_a+\alpha_b) \ne \E_{\update(\parsn' \mid \parsn, \x; \alpha_a)} \update(\parsn'' \mid \parsn', \x; \alpha_b)
% \end{equation}
% \oyyw{please check whether the below equation is better}
% \yuxuan{I fill somehow confusing on what is the update distribution with $\alpha$. }
% \begin{equation}
% \update(\parsn{}'' \mid \parsn{}, \x; \alpha_a+\alpha_b) \ne \E_{\update(\parsn{}' \mid \parsn{}, \x; \alpha_a)} \update(\parsn{}'' \mid \parsn{}', \x; \alpha_b)
% \end{equation}
% We give an intuitive visualization of such difference in \cref{fig:non_add}. The untenability of this property can materialize by considering the following case: with prior $\btheta = \{\bold{0},\bold{0}\}$, check the two-step Bayesian update distribution with $\alpha_a,\alpha_b$ and one-step Bayesian update with $\alpha=\alpha_a+\alpha_b$:
\begin{align}
\label{eq:nonadd}
     &\update(c'' \mid \parsn, \x; \alpha_a+\alpha_b)  = \delta(c-\alpha_a-\alpha_b)
     \ne  \mathbb{E}_{p_U(\parsn' \mid \parsn, \x; \alpha_a)}\update(c'' \mid \parsn', \x; \alpha_b) \nonumber \\&= \mathbb{E}_{vM(\bold{y}_b|\bold{x},\alpha_a)}\mathbb{E}_{vM(\bold{y}_a|\bold{x},\alpha_b)}\delta(c-||[\alpha_a \cos\y_a+\alpha_b\cos \y_b,\alpha_a \sin\y_a+\alpha_b\sin \y_b]^T||_2)
\end{align}
A more intuitive visualization could be found in \cref{fig:non_add}. This fundamental difference between periodic Bayesian flow and that of \citet{bfn} presents both theoretical and practical challenges, which we will explain and address in the following contents.

% This makes constructing Bayesian flow based on von Mises distribution intrinsically different from previous Bayesian flows (\citet{bfn}).

% Thus, we must reformulate the framework of Bayesian flow networks  accordingly. % and do necessary reformulations of BFN. 

% \yuxuan{overall I feel this part is complicated by using the language of update distribution. I would like to suggest simply use bayesian update, to provide intuitive explantion.}\hanlin{See the illustration in \cref{fig:non_add}}

% That introduces a cascade of problems, and we investigate the following issues: $(1)$ Accuracies between sender and receiver are not synchronized and need to be differentiated. $(2)$ There is no tractable Bayesian flow distribution for a one-step sample conditioned on a given time step $i$, and naively simulating the Bayesian flow results in computational overhead. $(3)$ It is difficult to control the entropy of the Bayesian flow. $(4)$ Accuracy is no longer a function of $t$ and becomes a distribution conditioned on $t$, which can be different across dimensions.
%\jj{Edited till here}

\textbf{Entropy Conditioning} As a common practice in generative models~\citep{ddpm,flowmatching,bfn}, timestep $t$ is widely used to distinguish among generation states by feeding the timestep information into the networks. However, this paper shows that for periodic Bayesian flow, the accumulated accuracy $\vc_i$ is more effective than time-based conditioning by informing the network about the entropy and certainty of the states $\parsnt{i}$. This stems from the intrinsic non-additive accuracy which makes the receiver's accumulated accuracy $c$ not bijective function of $t$, but a distribution conditioned on accumulated accuracies $\vc_i$ instead. Therefore, the entropy parameter $\vc$ is taken logarithm and fed into the network to describe the entropy of the input corrupted structure. We verify this consideration in \cref{sec:exp_ablation}. 
% \yuxuan{implement variant. traditionally, the timestep is widely used to distinguish the different states by putting the timestep embedding into the networks. citation of FM, diffusion, BFN. However, we find that conditioned on time in periodic flow could not provide extra benefits. To further boost the performance, we introduce a simple yet effective modification term entropy conditional. This is based on that the accumulated accuracy which represents the current uncertainty or entropy could be a better indicator to distinguish different states. + Describe how you do this. }



\textbf{Reformulations of BFN}. Recall the original update function with Gaussian sender distribution, after receiving noisy samples $\y_1,\y_2,\dots,\y_i$ with accuracies $\senderacc$, the accumulated accuracies of the receiver side could be analytically obtained by the additive property and it is consistent with the sender side.
% Since observing sample $\y$ with $\alpha_i$ can not result in exact accuracy increment $\alpha_i$ for receiver, the accuracies between sender and receiver are not synchronized which need to be differentiated. 
However, as previously mentioned, this does not apply to periodic Bayesian flow, and some of the notations in original BFN~\citep{bfn} need to be adjusted accordingly. We maintain the notations of sender side's one-step accuracy $\alpha$ and added accuracy $\beta$, and alter the notation of receiver's accuracy parameter as $c$, which is needed to be simulated by cascade of Bayesian updates. We emphasize that the receiver's accumulated accuracy $c$ is no longer a function of $t$ (differently from the Gaussian case), and it becomes a distribution conditioned on received accuracies $\senderacc$ from the sender. Therefore, we represent the Bayesian flow distribution of von Mises distribution as $p_F(\btheta|\x;\alpha_1,\alpha_2,\dots,\alpha_i)$. And the original simulation-free training with Bayesian flow distribution is no longer applicable in this scenario.
% Different from previous BFNs where the accumulated accuracy $\rho$ is not explicitly modeled, the accumulated accuracy parameter $c$ (visualized in \cref{fig:vmbf_vis}) needs to be explicitly modeled by feeding it to the network to avoid information loss.
% the randomaccuracy parameter $c$ (visualized in \cref{fig:vmbf_vis}) implies that there exists information in $c$ from the sender just like $m$, meaning that $c$ also should be fed into the network to avoid information loss. 
% We ablate this consideration in  \cref{sec:exp_ablation}. 

\textbf{Fast Sampling from Equivalent Bayesian Flow Distribution} Based on the above reformulations, the Bayesian flow distribution of von Mises distribution is reframed as: 
\begin{equation}\label{eq:flow_frac}
p_F(\btheta_i|\x;\alpha_1,\alpha_2,\dots,\alpha_i)=\E_{\update(\parsnt{1} \mid \parsnt{0}, \x ; \alphat{1})}\dots\E_{\update(\parsn_{i-1} \mid \parsnt{i-2}, \x; \alphat{i-1})} \update(\parsnt{i} | \parsnt{i-1},\x;\alphat{i} )
\end{equation}
Naively sampling from \cref{eq:flow_frac} requires slow auto-regressive iterated simulation, making training unaffordable. Noticing the mathematical properties of \cref{eq:h_m,eq:h_c}, we  transform \cref{eq:flow_frac} to the equivalent form:
\begin{equation}\label{eq:cirflow_equiv}
p_F(\vec{m}_i|\x;\alpha_1,\alpha_2,\dots,\alpha_i)=\E_{vM(\y_1|\x,\alpha_1)\dots vM(\y_i|\x,\alpha_i)} \delta(\vec{m}_i-\text{atan2}(\sum_{j=1}^i \alpha_j \cos \y_j,\sum_{j=1}^i \alpha_j \sin \y_j))
\end{equation}
\begin{equation}\label{eq:cirflow_equiv2}
p_F(\vec{c}_i|\x;\alpha_1,\alpha_2,\dots,\alpha_i)=\E_{vM(\y_1|\x,\alpha_1)\dots vM(\y_i|\x,\alpha_i)}  \delta(\vec{c}_i-||[\sum_{j=1}^i \alpha_j \cos \y_j,\sum_{j=1}^i \alpha_j \sin \y_j]^T||_2)
\end{equation}
which bypasses the computation of intermediate variables and allows pure tensor operations, with negligible computational overhead.
\begin{restatable}{proposition}{cirflowequiv}
The probability density function of Bayesian flow distribution defined by \cref{eq:cirflow_equiv,eq:cirflow_equiv2} is equivalent to the original definition in \cref{eq:flow_frac}. 
\end{restatable}
\textbf{Numerical Determination of Linear Entropy Sender Accuracy Schedule} ~Original BFN designs the accuracy schedule $\beta(t)$ to make the entropy of input distribution linearly decrease. As for crystal generation task, to ensure information coherence between modalities, we choose a sender accuracy schedule $\senderacc$ that makes the receiver's belief entropy $H(t_i)=H(p_I(\cdot|\vtheta_i))=H(p_I(\cdot|\vc_i))$ linearly decrease \emph{w.r.t.} time $t_i$, given the initial and final accuracy parameter $c(0)$ and $c(1)$. Due to the intractability of \cref{eq:vm_entropy}, we first use numerical binary search in $[0,c(1)]$ to determine the receiver's $c(t_i)$ for $i=1,\dots, n$ by solving the equation $H(c(t_i))=(1-t_i)H(c(0))+tH(c(1))$. Next, with $c(t_i)$, we conduct numerical binary search for each $\alpha_i$ in $[0,c(1)]$ by solving the equations $\E_{y\sim vM(x,\alpha_i)}[\sqrt{\alpha_i^2+c_{i-1}^2+2\alpha_i c_{i-1}\cos(y-m_{i-1})}]=c(t_i)$ from $i=1$ to $i=n$ for arbitrarily selected $x\in[-\pi,\pi)$.

After tackling all those issues, we have now arrived at a new BFN architecture for effectively modeling crystals. Such BFN can also be adapted to other type of data located in hyper-torus $\mathbb{T}^{D}$.

\subsection{Equivariant Bayesian Flow for Crystal}
With the above Bayesian flow designed for generative modeling of fractional coordinate $\vF$, we are able to build equivariant Bayesian flow for each modality of crystal. In this section, we first give an overview of the general training and sampling algorithm of \modelname (visualized in \cref{fig:framework}). Then, we describe the details of the Bayesian flow of every modality. The training and sampling algorithm can be found in \cref{alg:train} and \cref{alg:sampling}.

\textbf{Overview} Operating in the parameter space $\bthetaM=\{\bthetaA,\bthetaL,\bthetaF\}$, \modelname generates high-fidelity crystals through a joint BFN sampling process on the parameter of  atom type $\bthetaA$, lattice parameter $\vec{\theta}^L=\{\bmuL,\brhoL\}$, and the parameter of fractional coordinate matrix $\bthetaF=\{\bmF,\bcF\}$. We index the $n$-steps of the generation process in a discrete manner $i$, and denote the corresponding continuous notation $t_i=i/n$ from prior parameter $\thetaM_0$ to a considerably low variance parameter $\thetaM_n$ (\emph{i.e.} large $\vrho^L,\bmF$, and centered $\bthetaA$).

At training time, \modelname samples time $i\sim U\{1,n\}$ and $\bthetaM_{i-1}$ from the Bayesian flow distribution of each modality, serving as the input to the network. The network $\net$ outputs $\net(\parsnt{i-1}^\mathcal{M},t_{i-1})=\net(\parsnt{i-1}^A,\parsnt{i-1}^F,\parsnt{i-1}^L,t_{i-1})$ and conducts gradient descents on loss function \cref{eq:loss_n} for each modality. After proper training, the sender distribution $p_S$ can be approximated by the receiver distribution $p_R$. 

At inference time, from predefined $\thetaM_0$, we conduct transitions from $\thetaM_{i-1}$ to $\thetaM_{i}$ by: $(1)$ sampling $\y_i\sim p_R(\bold{y}|\thetaM_{i-1};t_i,\alpha_i)$ according to network prediction $\predM{i-1}$; and $(2)$ performing Bayesian update $h(\thetaM_{i-1},\y^\calM_{i-1},\alpha_i)$ for each dimension. 

% Alternatively, we complete this transition using the flow-back technique by sampling 
% $\thetaM_{i}$ from Bayesian flow distribution $\flow(\btheta^M_{i}|\predM{i-1};t_{i-1})$. 

% The training objective of $\net$ is to minimize the KL divergence between sender distribution and receiver distribution for every modality as defined in \cref{eq:loss_n} which is equivalent to optimizing the negative variational lower bound $\calL^{VLB}$ as discussed in \cref{sec:preliminaries}. 

%In the following part, we will present the Bayesian flow of each modality in detail.

\textbf{Bayesian Flow of Fractional Coordinate $\vF$}~The distribution of the prior parameter $\bthetaF_0$ is defined as:
\begin{equation}\label{eq:prior_frac}
    p(\bthetaF_0) \defeq \{vM(\vm_0^F|\vec{0}_{3\times N},\vec{0}_{3\times N}),\delta(\vc_0^F-\vec{0}_{3\times N})\} = \{U(\vec{0},\vec{1}),\delta(\vc_0^F-\vec{0}_{3\times N})\}
\end{equation}
Note that this prior distribution of $\vm_0^F$ is uniform over $[\vec{0},\vec{1})$, ensuring the periodic translation invariance property in \cref{De:pi}. The training objective is minimizing the KL divergence between sender and receiver distribution (deduction can be found in \cref{appd:cir_loss}): 
%\oyyw{replace $\vF$ with $\x$?} \hanlin{notations follow Preliminary?}
\begin{align}\label{loss_frac}
\calL_F = n \E_{i \sim \ui{n}, \flow(\parsn{}^F \mid \vF ; \senderacc)} \alpha_i\frac{I_1(\alpha_i)}{I_0(\alpha_i)}(1-\cos(\vF-\predF{i-1}))
\end{align}
where $I_0(x)$ and $I_1(x)$ are the zeroth and the first order of modified Bessel functions. The transition from $\bthetaF_{i-1}$ to $\bthetaF_{i}$ is the Bayesian update distribution based on network prediction:
\begin{equation}\label{eq:transi_frac}
    p(\btheta^F_{i}|\parsnt{i-1}^\calM)=\mathbb{E}_{vM(\bold{y}|\predF{i-1},\alpha_i)}\delta(\btheta^F_{i}-h(\btheta^F_{i-1},\bold{y},\alpha_i))
\end{equation}
\begin{restatable}{proposition}{fracinv}
With $\net_{F}$ as a periodic translation equivariant function namely $\net_F(\parsnt{}^A,w(\parsnt{}^F+\vt),\parsnt{}^L,t)=w(\net_F(\parsnt{}^A,\parsnt{}^F,\parsnt{}^L,t)+\vt), \forall\vt\in\R^3$, the marginal distribution of $p(\vF_n)$ defined by \cref{eq:prior_frac,eq:transi_frac} is periodic translation invariant. 
\end{restatable}
\textbf{Bayesian Flow of Lattice Parameter \texorpdfstring{$\boldsymbol{L}$}{}}   
Noting the lattice parameter $\bm{L}$ located in Euclidean space, we set prior as the parameter of a isotropic multivariate normal distribution $\btheta^L_0\defeq\{\vmu_0^L,\vrho_0^L\}=\{\bm{0}_{3\times3},\bm{1}_{3\times3}\}$
% \begin{equation}\label{eq:lattice_prior}
% \btheta^L_0\defeq\{\vmu_0^L,\vrho_0^L\}=\{\bm{0}_{3\times3},\bm{1}_{3\times3}\}
% \end{equation}
such that the prior distribution of the Markov process on $\vmu^L$ is the Dirac distribution $\delta(\vec{\mu_0}-\vec{0})$ and $\delta(\vec{\rho_0}-\vec{1})$, 
% \begin{equation}
%     p_I^L(\boldsymbol{L}|\btheta_0^L)=\mathcal{N}(\bm{L}|\bm{0},\bm{I})
% \end{equation}
which ensures O(3)-invariance of prior distribution of $\vL$. By Eq. 77 from \citet{bfn}, the Bayesian flow distribution of the lattice parameter $\bm{L}$ is: 
\begin{align}% =p_U(\bmuL|\btheta_0^L,\bm{L},\beta(t))
p_F^L(\bmuL|\bm{L};t) &=\mathcal{N}(\bmuL|\gamma(t)\bm{L},\gamma(t)(1-\gamma(t))\bm{I}) 
\end{align}
where $\gamma(t) = 1 - \sigma_1^{2t}$ and $\sigma_1$ is the predefined hyper-parameter controlling the variance of input distribution at $t=1$ under linear entropy accuracy schedule. The variance parameter $\vrho$ does not need to be modeled and fed to the network, since it is deterministic given the accuracy schedule. After sampling $\bmuL_i$ from $p_F^L$, the training objective is defined as minimizing KL divergence between sender and receiver distribution (based on Eq. 96 in \citet{bfn}):
\begin{align}
\mathcal{L}_{L} = \frac{n}{2}\left(1-\sigma_1^{2/n}\right)\E_{i \sim \ui{n}}\E_{\flow(\bmuL_{i-1} |\vL ; t_{i-1})}  \frac{\left\|\vL -\predL{i-1}\right\|^2}{\sigma_1^{2i/n}},\label{eq:lattice_loss}
\end{align}
where the prediction term $\predL{i-1}$ is the lattice parameter part of network output. After training, the generation process is defined as the Bayesian update distribution given network prediction:
\begin{equation}\label{eq:lattice_sampling}
    p(\bmuL_{i}|\parsnt{i-1}^\calM)=\update^L(\bmuL_{i}|\predL{i-1},\bmuL_{i-1};t_{i-1})
\end{equation}
    

% The final prediction of the lattice parameter is given by $\bmuL_n = \predL{n-1}$.
% \begin{equation}\label{eq:final_lattice}
%     \bmuL_n = \predL{n-1}
% \end{equation}

\begin{restatable}{proposition}{latticeinv}\label{prop:latticeinv}
With $\net_{L}$ as  O(3)-equivariant function namely $\net_L(\parsnt{}^A,\parsnt{}^F,\vQ\parsnt{}^L,t)=\vQ\net_L(\parsnt{}^A,\parsnt{}^F,\parsnt{}^L,t),\forall\vQ^T\vQ=\vI$, the marginal distribution of $p(\bmuL_n)$ defined by \cref{eq:lattice_sampling} is O(3)-invariant. 
\end{restatable}


\textbf{Bayesian Flow of Atom Types \texorpdfstring{$\boldsymbol{A}$}{}} 
Given that atom types are discrete random variables located in a simplex $\calS^K$, the prior parameter of $\boldsymbol{A}$ is the discrete uniform distribution over the vocabulary $\parsnt{0}^A \defeq \frac{1}{K}\vec{1}_{1\times N}$. 
% \begin{align}\label{eq:disc_input_prior}
% \parsnt{0}^A \defeq \frac{1}{K}\vec{1}_{1\times N}
% \end{align}
% \begin{align}
%     (\oh{j}{K})_k \defeq \delta_{j k}, \text{where }\oh{j}{K}\in \R^{K},\oh{\vA}{KD} \defeq \left(\oh{a_1}{K},\dots,\oh{a_N}{K}\right) \in \R^{K\times N}
% \end{align}
With the notation of the projection from the class index $j$ to the length $K$ one-hot vector $ (\oh{j}{K})_k \defeq \delta_{j k}, \text{where }\oh{j}{K}\in \R^{K},\oh{\vA}{KD} \defeq \left(\oh{a_1}{K},\dots,\oh{a_N}{K}\right) \in \R^{K\times N}$, the Bayesian flow distribution of atom types $\vA$ is derived in \citet{bfn}:
\begin{align}
\flow^{A}(\parsn^A \mid \vA; t) &= \E_{\N{\y \mid \beta^A(t)\left(K \oh{\vA}{K\times N} - \vec{1}_{K\times N}\right)}{\beta^A(t) K \vec{I}_{K\times N \times N}}} \delta\left(\parsn^A - \frac{e^{\y}\parsnt{0}^A}{\sum_{k=1}^K e^{\y_k}(\parsnt{0})_{k}^A}\right).
\end{align}
where $\beta^A(t)$ is the predefined accuracy schedule for atom types. Sampling $\btheta_i^A$ from $p_F^A$ as the training signal, the training objective is the $n$-step discrete-time loss for discrete variable \citep{bfn}: 
% \oyyw{can we simplify the next equation? Such as remove $K \times N, K \times N \times N$}
% \begin{align}
% &\calL_A = n\E_{i \sim U\{1,n\},\flow^A(\parsn^A \mid \vA ; t_{i-1}),\N{\y \mid \alphat{i}\left(K \oh{\vA}{KD} - \vec{1}_{K\times N}\right)}{\alphat{i} K \vec{I}_{K\times N \times N}}} \ln \N{\y \mid \alphat{i}\left(K \oh{\vA}{K\times N} - \vec{1}_{K\times N}\right)}{\alphat{i} K \vec{I}_{K\times N \times N}}\nonumber\\
% &\qquad\qquad\qquad-\sum_{d=1}^N \ln \left(\sum_{k=1}^K \out^{(d)}(k \mid \parsn^A; t_{i-1}) \N{\ydd{d} \mid \alphat{i}\left(K\oh{k}{K}- \vec{1}_{K\times N}\right)}{\alphat{i} K \vec{I}_{K\times N \times N}}\right)\label{discdisc_t_loss_exp}
% \end{align}
\begin{align}
&\calL_A = n\E_{i \sim U\{1,n\},\flow^A(\parsn^A \mid \vA ; t_{i-1}),\N{\y \mid \alphat{i}\left(K \oh{\vA}{KD} - \vec{1}\right)}{\alphat{i} K \vec{I}}} \ln \N{\y \mid \alphat{i}\left(K \oh{\vA}{K\times N} - \vec{1}\right)}{\alphat{i} K \vec{I}}\nonumber\\
&\qquad\qquad\qquad-\sum_{d=1}^N \ln \left(\sum_{k=1}^K \out^{(d)}(k \mid \parsn^A; t_{i-1}) \N{\ydd{d} \mid \alphat{i}\left(K\oh{k}{K}- \vec{1}\right)}{\alphat{i} K \vec{I}}\right)\label{discdisc_t_loss_exp}
\end{align}
where $\vec{I}\in \R^{K\times N \times N}$ and $\vec{1}\in\R^{K\times D}$. When sampling, the transition from $\bthetaA_{i-1}$ to $\bthetaA_{i}$ is derived as:
\begin{equation}
    p(\btheta^A_{i}|\parsnt{i-1}^\calM)=\update^A(\btheta^A_{i}|\btheta^A_{i-1},\predA{i-1};t_{i-1})
\end{equation}

The detailed training and sampling algorithm could be found in \cref{alg:train} and \cref{alg:sampling}.




\section{Experiments}
\label{sec:experiments}
\section{Experiments}
\label{sec:experiments}
The experiments are designed to address two key research questions.
First, \textbf{RQ1} evaluates whether the average $L_2$-norm of the counterfactual perturbation vectors ($\overline{||\perturb||}$) decreases as the model overfits the data, thereby providing further empirical validation for our hypothesis.
Second, \textbf{RQ2} evaluates the ability of the proposed counterfactual regularized loss, as defined in (\ref{eq:regularized_loss2}), to mitigate overfitting when compared to existing regularization techniques.

% The experiments are designed to address three key research questions. First, \textbf{RQ1} investigates whether the mean perturbation vector norm decreases as the model overfits the data, aiming to further validate our intuition. Second, \textbf{RQ2} explores whether the mean perturbation vector norm can be effectively leveraged as a regularization term during training, offering insights into its potential role in mitigating overfitting. Finally, \textbf{RQ3} examines whether our counterfactual regularizer enables the model to achieve superior performance compared to existing regularization methods, thus highlighting its practical advantage.

\subsection{Experimental Setup}
\textbf{\textit{Datasets, Models, and Tasks.}}
The experiments are conducted on three datasets: \textit{Water Potability}~\cite{kadiwal2020waterpotability}, \textit{Phomene}~\cite{phomene}, and \textit{CIFAR-10}~\cite{krizhevsky2009learning}. For \textit{Water Potability} and \textit{Phomene}, we randomly select $80\%$ of the samples for the training set, and the remaining $20\%$ for the test set, \textit{CIFAR-10} comes already split. Furthermore, we consider the following models: Logistic Regression, Multi-Layer Perceptron (MLP) with 100 and 30 neurons on each hidden layer, and PreactResNet-18~\cite{he2016cvecvv} as a Convolutional Neural Network (CNN) architecture.
We focus on binary classification tasks and leave the extension to multiclass scenarios for future work. However, for datasets that are inherently multiclass, we transform the problem into a binary classification task by selecting two classes, aligning with our assumption.

\smallskip
\noindent\textbf{\textit{Evaluation Measures.}} To characterize the degree of overfitting, we use the test loss, as it serves as a reliable indicator of the model's generalization capability to unseen data. Additionally, we evaluate the predictive performance of each model using the test accuracy.

\smallskip
\noindent\textbf{\textit{Baselines.}} We compare CF-Reg with the following regularization techniques: L1 (``Lasso''), L2 (``Ridge''), and Dropout.

\smallskip
\noindent\textbf{\textit{Configurations.}}
For each model, we adopt specific configurations as follows.
\begin{itemize}
\item \textit{Logistic Regression:} To induce overfitting in the model, we artificially increase the dimensionality of the data beyond the number of training samples by applying a polynomial feature expansion. This approach ensures that the model has enough capacity to overfit the training data, allowing us to analyze the impact of our counterfactual regularizer. The degree of the polynomial is chosen as the smallest degree that makes the number of features greater than the number of data.
\item \textit{Neural Networks (MLP and CNN):} To take advantage of the closed-form solution for computing the optimal perturbation vector as defined in (\ref{eq:opt-delta}), we use a local linear approximation of the neural network models. Hence, given an instance $\inst_i$, we consider the (optimal) counterfactual not with respect to $\model$ but with respect to:
\begin{equation}
\label{eq:taylor}
    \model^{lin}(\inst) = \model(\inst_i) + \nabla_{\inst}\model(\inst_i)(\inst - \inst_i),
\end{equation}
where $\model^{lin}$ represents the first-order Taylor approximation of $\model$ at $\inst_i$.
Note that this step is unnecessary for Logistic Regression, as it is inherently a linear model.
\end{itemize}

\smallskip
\noindent \textbf{\textit{Implementation Details.}} We run all experiments on a machine equipped with an AMD Ryzen 9 7900 12-Core Processor and an NVIDIA GeForce RTX 4090 GPU. Our implementation is based on the PyTorch Lightning framework. We use stochastic gradient descent as the optimizer with a learning rate of $\eta = 0.001$ and no weight decay. We use a batch size of $128$. The training and test steps are conducted for $6000$ epochs on the \textit{Water Potability} and \textit{Phoneme} datasets, while for the \textit{CIFAR-10} dataset, they are performed for $200$ epochs.
Finally, the contribution $w_i^{\varepsilon}$ of each training point $\inst_i$ is uniformly set as $w_i^{\varepsilon} = 1~\forall i\in \{1,\ldots,m\}$.

The source code implementation for our experiments is available at the following GitHub repository: \url{https://anonymous.4open.science/r/COCE-80B4/README.md} 

\subsection{RQ1: Counterfactual Perturbation vs. Overfitting}
To address \textbf{RQ1}, we analyze the relationship between the test loss and the average $L_2$-norm of the counterfactual perturbation vectors ($\overline{||\perturb||}$) over training epochs.

In particular, Figure~\ref{fig:delta_loss_epochs} depicts the evolution of $\overline{||\perturb||}$ alongside the test loss for an MLP trained \textit{without} regularization on the \textit{Water Potability} dataset. 
\begin{figure}[ht]
    \centering
    \includegraphics[width=0.85\linewidth]{img/delta_loss_epochs.png}
    \caption{The average counterfactual perturbation vector $\overline{||\perturb||}$ (left $y$-axis) and the cross-entropy test loss (right $y$-axis) over training epochs ($x$-axis) for an MLP trained on the \textit{Water Potability} dataset \textit{without} regularization.}
    \label{fig:delta_loss_epochs}
\end{figure}

The plot shows a clear trend as the model starts to overfit the data (evidenced by an increase in test loss). 
Notably, $\overline{||\perturb||}$ begins to decrease, which aligns with the hypothesis that the average distance to the optimal counterfactual example gets smaller as the model's decision boundary becomes increasingly adherent to the training data.

It is worth noting that this trend is heavily influenced by the choice of the counterfactual generator model. In particular, the relationship between $\overline{||\perturb||}$ and the degree of overfitting may become even more pronounced when leveraging more accurate counterfactual generators. However, these models often come at the cost of higher computational complexity, and their exploration is left to future work.

Nonetheless, we expect that $\overline{||\perturb||}$ will eventually stabilize at a plateau, as the average $L_2$-norm of the optimal counterfactual perturbations cannot vanish to zero.

% Additionally, the choice of employing the score-based counterfactual explanation framework to generate counterfactuals was driven to promote computational efficiency.

% Future enhancements to the framework may involve adopting models capable of generating more precise counterfactuals. While such approaches may yield to performance improvements, they are likely to come at the cost of increased computational complexity.


\subsection{RQ2: Counterfactual Regularization Performance}
To answer \textbf{RQ2}, we evaluate the effectiveness of the proposed counterfactual regularization (CF-Reg) by comparing its performance against existing baselines: unregularized training loss (No-Reg), L1 regularization (L1-Reg), L2 regularization (L2-Reg), and Dropout.
Specifically, for each model and dataset combination, Table~\ref{tab:regularization_comparison} presents the mean value and standard deviation of test accuracy achieved by each method across 5 random initialization. 

The table illustrates that our regularization technique consistently delivers better results than existing methods across all evaluated scenarios, except for one case -- i.e., Logistic Regression on the \textit{Phomene} dataset. 
However, this setting exhibits an unusual pattern, as the highest model accuracy is achieved without any regularization. Even in this case, CF-Reg still surpasses other regularization baselines.

From the results above, we derive the following key insights. First, CF-Reg proves to be effective across various model types, ranging from simple linear models (Logistic Regression) to deep architectures like MLPs and CNNs, and across diverse datasets, including both tabular and image data. 
Second, CF-Reg's strong performance on the \textit{Water} dataset with Logistic Regression suggests that its benefits may be more pronounced when applied to simpler models. However, the unexpected outcome on the \textit{Phoneme} dataset calls for further investigation into this phenomenon.


\begin{table*}[h!]
    \centering
    \caption{Mean value and standard deviation of test accuracy across 5 random initializations for different model, dataset, and regularization method. The best results are highlighted in \textbf{bold}.}
    \label{tab:regularization_comparison}
    \begin{tabular}{|c|c|c|c|c|c|c|}
        \hline
        \textbf{Model} & \textbf{Dataset} & \textbf{No-Reg} & \textbf{L1-Reg} & \textbf{L2-Reg} & \textbf{Dropout} & \textbf{CF-Reg (ours)} \\ \hline
        Logistic Regression   & \textit{Water}   & $0.6595 \pm 0.0038$   & $0.6729 \pm 0.0056$   & $0.6756 \pm 0.0046$  & N/A    & $\mathbf{0.6918 \pm 0.0036}$                     \\ \hline
        MLP   & \textit{Water}   & $0.6756 \pm 0.0042$   & $0.6790 \pm 0.0058$   & $0.6790 \pm 0.0023$  & $0.6750 \pm 0.0036$    & $\mathbf{0.6802 \pm 0.0046}$                    \\ \hline
%        MLP   & \textit{Adult}   & $0.8404 \pm 0.0010$   & $\mathbf{0.8495 \pm 0.0007}$   & $0.8489 \pm 0.0014$  & $\mathbf{0.8495 \pm 0.0016}$     & $0.8449 \pm 0.0019$                    \\ \hline
        Logistic Regression   & \textit{Phomene}   & $\mathbf{0.8148 \pm 0.0020}$   & $0.8041 \pm 0.0028$   & $0.7835 \pm 0.0176$  & N/A    & $0.8098 \pm 0.0055$                     \\ \hline
        MLP   & \textit{Phomene}   & $0.8677 \pm 0.0033$   & $0.8374 \pm 0.0080$   & $0.8673 \pm 0.0045$  & $0.8672 \pm 0.0042$     & $\mathbf{0.8718 \pm 0.0040}$                    \\ \hline
        CNN   & \textit{CIFAR-10} & $0.6670 \pm 0.0233$   & $0.6229 \pm 0.0850$   & $0.7348 \pm 0.0365$   & N/A    & $\mathbf{0.7427 \pm 0.0571}$                     \\ \hline
    \end{tabular}
\end{table*}

\begin{table*}[htb!]
    \centering
    \caption{Hyperparameter configurations utilized for the generation of Table \ref{tab:regularization_comparison}. For our regularization the hyperparameters are reported as $\mathbf{\alpha/\beta}$.}
    \label{tab:performance_parameters}
    \begin{tabular}{|c|c|c|c|c|c|c|}
        \hline
        \textbf{Model} & \textbf{Dataset} & \textbf{No-Reg} & \textbf{L1-Reg} & \textbf{L2-Reg} & \textbf{Dropout} & \textbf{CF-Reg (ours)} \\ \hline
        Logistic Regression   & \textit{Water}   & N/A   & $0.0093$   & $0.6927$  & N/A    & $0.3791/1.0355$                     \\ \hline
        MLP   & \textit{Water}   & N/A   & $0.0007$   & $0.0022$  & $0.0002$    & $0.2567/1.9775$                    \\ \hline
        Logistic Regression   &
        \textit{Phomene}   & N/A   & $0.0097$   & $0.7979$  & N/A    & $0.0571/1.8516$                     \\ \hline
        MLP   & \textit{Phomene}   & N/A   & $0.0007$   & $4.24\cdot10^{-5}$  & $0.0015$    & $0.0516/2.2700$                    \\ \hline
       % MLP   & \textit{Adult}   & N/A   & $0.0018$   & $0.0018$  & $0.0601$     & $0.0764/2.2068$                    \\ \hline
        CNN   & \textit{CIFAR-10} & N/A   & $0.0050$   & $0.0864$ & N/A    & $0.3018/
        2.1502$                     \\ \hline
    \end{tabular}
\end{table*}

\begin{table*}[htb!]
    \centering
    \caption{Mean value and standard deviation of training time across 5 different runs. The reported time (in seconds) corresponds to the generation of each entry in Table \ref{tab:regularization_comparison}. Times are }
    \label{tab:times}
    \begin{tabular}{|c|c|c|c|c|c|c|}
        \hline
        \textbf{Model} & \textbf{Dataset} & \textbf{No-Reg} & \textbf{L1-Reg} & \textbf{L2-Reg} & \textbf{Dropout} & \textbf{CF-Reg (ours)} \\ \hline
        Logistic Regression   & \textit{Water}   & $222.98 \pm 1.07$   & $239.94 \pm 2.59$   & $241.60 \pm 1.88$  & N/A    & $251.50 \pm 1.93$                     \\ \hline
        MLP   & \textit{Water}   & $225.71 \pm 3.85$   & $250.13 \pm 4.44$   & $255.78 \pm 2.38$  & $237.83 \pm 3.45$    & $266.48 \pm 3.46$                    \\ \hline
        Logistic Regression   & \textit{Phomene}   & $266.39 \pm 0.82$ & $367.52 \pm 6.85$   & $361.69 \pm 4.04$  & N/A   & $310.48 \pm 0.76$                    \\ \hline
        MLP   &
        \textit{Phomene} & $335.62 \pm 1.77$   & $390.86 \pm 2.11$   & $393.96 \pm 1.95$ & $363.51 \pm 5.07$    & $403.14 \pm 1.92$                     \\ \hline
       % MLP   & \textit{Adult}   & N/A   & $0.0018$   & $0.0018$  & $0.0601$     & $0.0764/2.2068$                    \\ \hline
        CNN   & \textit{CIFAR-10} & $370.09 \pm 0.18$   & $395.71 \pm 0.55$   & $401.38 \pm 0.16$ & N/A    & $1287.8 \pm 0.26$                     \\ \hline
    \end{tabular}
\end{table*}

\subsection{Feasibility of our Method}
A crucial requirement for any regularization technique is that it should impose minimal impact on the overall training process.
In this respect, CF-Reg introduces an overhead that depends on the time required to find the optimal counterfactual example for each training instance. 
As such, the more sophisticated the counterfactual generator model probed during training the higher would be the time required. However, a more advanced counterfactual generator might provide a more effective regularization. We discuss this trade-off in more details in Section~\ref{sec:discussion}.

Table~\ref{tab:times} presents the average training time ($\pm$ standard deviation) for each model and dataset combination listed in Table~\ref{tab:regularization_comparison}.
We can observe that the higher accuracy achieved by CF-Reg using the score-based counterfactual generator comes with only minimal overhead. However, when applied to deep neural networks with many hidden layers, such as \textit{PreactResNet-18}, the forward derivative computation required for the linearization of the network introduces a more noticeable computational cost, explaining the longer training times in the table.

\subsection{Hyperparameter Sensitivity Analysis}
The proposed counterfactual regularization technique relies on two key hyperparameters: $\alpha$ and $\beta$. The former is intrinsic to the loss formulation defined in (\ref{eq:cf-train}), while the latter is closely tied to the choice of the score-based counterfactual explanation method used.

Figure~\ref{fig:test_alpha_beta} illustrates how the test accuracy of an MLP trained on the \textit{Water Potability} dataset changes for different combinations of $\alpha$ and $\beta$.

\begin{figure}[ht]
    \centering
    \includegraphics[width=0.85\linewidth]{img/test_acc_alpha_beta.png}
    \caption{The test accuracy of an MLP trained on the \textit{Water Potability} dataset, evaluated while varying the weight of our counterfactual regularizer ($\alpha$) for different values of $\beta$.}
    \label{fig:test_alpha_beta}
\end{figure}

We observe that, for a fixed $\beta$, increasing the weight of our counterfactual regularizer ($\alpha$) can slightly improve test accuracy until a sudden drop is noticed for $\alpha > 0.1$.
This behavior was expected, as the impact of our penalty, like any regularization term, can be disruptive if not properly controlled.

Moreover, this finding further demonstrates that our regularization method, CF-Reg, is inherently data-driven. Therefore, it requires specific fine-tuning based on the combination of the model and dataset at hand.

\section{Conclusion}
We have developed a novel posterior sampling scheme for denoising diffusion priors. The proposed algorithm proceeds by sequentially sampling, using a Gibbs sampler, from a sequence of mixture approximations of the smoothed posteriors. Our experiments show that \algo\ not only matches but often surpasses state-of-the-art performance and reconstruction quality across various tasks.
Furthermore, we have demonstrated that the Gibbs sampling perspective allows favorable performance improvement with inference-time compute scaling.

This work has certain limitations that open avenues for further exploration. While we outperform the state-of-the-art on most tasks and remains competitive overall on latent diffusion,  we still fall short of what we achieve with pixel-space diffusion. We believe that bridging this gap requires a more careful selection of the weight sequence. More broadly, an observation-driven approach to sampling the index could further enhance \algo. A second limitation is that our methodology does not extend to ODE-based samplers or DDIM, and adapting related ideas to these methods is an interesting research direction. Finally, like all existing methods relying on \eqref{eq:dps}, our approach incurs a higher memory cost compared to unconditional diffusion. It remains an open question whether the vector-Jacobian product can be eliminated without compromising performance.
% \clearpage


% In the unusual situation where you want a paper to appear in the
% references without citing it in the main text, use \nocite
\nocite{langley00}

\clearpage
\newpage
\section*{Acknowledgements}
The work of Y.J. and B.M. has been supported by Technology Innovation Institute (TII), project Fed2Learn. The work of Eric Moulines has been partly funded by the European Union (ERC-2022-SYG-OCEAN-101071601). Views and opinions expressed are however those of the author(s) only and do not necessarily reflect those of the European Union or the European Research Council Executive Agency. Neither the European Union nor the granting authority can be held responsible for them. This work was granted access to the HPC resources of IDRIS under the allocation 2025-AD011015980 made by GENCI.
\bibliography{bibliography.bib}
\bibliographystyle{icml2025}

\newpage
\appendix
\onecolumn
\section{Methodology details}
\subsection{Primer on Gibbs sampling}
\label{apdx-sec:gibbs}
In this section we lay out the basic properties of Gibbs sampling. We use measure-theoretic notation for conciseness. 

Let $\probmeas{0,1}{}{\rmd (\bx_0, \bx_1)}$ be a probability measure on $\rset^\dimx \times \rset^\dimx$. We denote by $\probmeas{0|1}{\bx_1}{\rmd \bx_0}$ and $\probmeas{1|0}{\bx_0}{\rmd \bx_1}$ the associated full conditionals and we write $\probmeas{0}{}{}$, $\probmeas{1}{}{}$ for its marginals. Define the %following 
transition kernels 
\begin{align*} 
    P_0(\rmd (\bx^\prime _0, \bx^\prime _1) | \bx_0, \bx_1) & \eqdef \probmeas{0|1}{\bx_1}{\rmd \bx^\prime _0} \delta_{\bx_1}(\rmd \bx^\prime _1),\\
    P_1(\rmd (\bx^\prime _0, \bx^\prime _1) | \bx_0, \bx_1) & \eqdef \probmeas{1|0}{\bx_0}{\rmd \bx^\prime _1} \delta_{\bx_0}(\rmd \bx^\prime _0).
\end{align*}
Each transition kernel updates only one coordinate at a time. A full update of the coordinates is obtained by composition of the kernels, \emph{i.e.}
$$ 
    P_0 P_1(\rmd (\bx^\prime _0, \bx^\prime _1) | \bx_0, \bx_1) \eqdef \int P_1(\rmd (\bx^\prime _0, \bx^\prime _1) | \tilde\bx_0, \tilde\bx_1) P_0(\rmd (\tilde\bx _0, \tilde\bx _1) | \bx_0, \bx_1) \eqsp.
$$  
 Each transition admits the joint distribution $\probmeas{0, 1}{}{}$ as stationary distribution, meaning that $\probmeas{0, 1}{}{\rmd (\bx_0, \bx_1)} = \int P_0(\rmd (\bx _0, \bx _1) | \bx^\prime _0, \bx^\prime _1)\, \probmeas{0, 1}{}{\rmd (\bx^\prime _0, \bx^\prime _1)}$. Indeed, this is seen by noting that 
\begin{align*}
    P_0(\rmd (\bx _0, \bx _1) | \bx^\prime _0, \bx^\prime _1) \probmeas{0, 1}{}{\rmd (\bx^\prime _0, \bx^\prime _1)}  & =  \probmeas{0|1}{\bx^\prime _1}{\rmd \bx _0} \delta_{\bx^\prime _1}(\rmd \bx _1) \probmeas{0, 1}{}{\rmd (\bx^\prime _0, \bx^\prime _1)}\\
    & =   \probmeas{0|1}{\bx^\prime _1}{\rmd \bx _0} \delta_{\bx^\prime _1}(\rmd \bx _1) \probmeas{0|1}{\bx^\prime _1}{\rmd \bx^\prime _0} \probmeas{1}{}{\rmd \bx^\prime _1}\\ 
    & = \probmeas{0|1}{\bx _1}{\rmd \bx _0} \probmeas{1}{}{\rmd \bx _1}  \probmeas{0|1}{\bx^\prime _1}{\rmd \bx^\prime _0}\delta_{\bx _1}(\rmd \bx^\prime _1),
\end{align*}
and then integrating both sides \wrt\ $(\bx^\prime _0, \bx^\prime _1)$. It then follows immediately that also $P_0 P_1$ admits $\probmeas{0,1}{}{}$ as stationary distribution. Letting $\big( (X^k _0, X^k _1) \big)_{k \in \nset}$ be a Markov chain with transition kernel $P_0 P_1$, the law of $(X^k _0, X^k _1)$ converges to $\probmeas{0,1}{}{}$ as $k \to \infty$ under mild conditions; see \cite{roberts1994simple}. 
\subsection{Full Gibbs conditionals}
\label{apdx-sec:conditionals}
In the main paper we consider the following data augmentation of the mixture $\hpost{t}{}{}$ \eqref{eq:posterior-approximation}
\begin{equation}
    \label{eq:extended-distr-normalized}
    \epost{0, s, t}{}{\bx_0, \bx_s, \bx_t} \\ = \pdata{0|s}{\bx_s}{\bx_0} \frac{\hpot{s}{\bx_s} \pdata{s|t}{\bx_t}{\bx_s} \pdata{t}{}{\bx_t}}{\int  \hpot{s}{\bx^\prime _s} \pdata{s|t}{\bx^\prime _t}{\bx^\prime _s} \pdata{t}{}{\bx^\prime _t} \, \rmd \bx_{s, t}}  \eqsp.
\end{equation}
From this definition it is straightforward to see that $\epost{0|s, t}{\bx_s, \bx_t}{\bx_0} = \pdata{0|s}{\bx_s}{\bx_0}$. In order to compute the full conditional $\epost{s|0, t}{\bx_0, \bx_t}{\bx_s}$ we use the identity 
\begin{equation} 
    \label{eq:bw-fw}
    \pdata{0|s}{\bx_s}{\bx_0} \pdata{s|t}{\bx_t}{\bx_s} \pdata{t}{}{\bx_t}= \pdata{0}{}{\bx_0} \fw{s|0}{\bx_0}{\bx_s} \fw{t|s}{\bx_s}{\bx_t},
\end{equation} 
from which it follows that 
\begin{align*} 
    \epost{s|0, t}{\bx_0, \bx_t}{\bx_s} & = \frac{\pdata{0|s}{\bx_s}{\bx_0} \hpot{s}{\bx_s} \pdata{s|t}{\bx_t}{\bx_s}}{\int \pdata{0|s}{\bx^\prime _s}{\bx_0} \hpot{s}{\bx^\prime _s} \pdata{s|t}{\bx_t}{\bx^\prime _s}\, \rmd \bx^\prime _s } \\
    & = \frac{ \fw{s|0}{\bx_0}{\bx_s} \hpot{s}{\bx_s} \fw{t|s}{\bx_s}{\bx_t}}{\int \fw{s|0}{\bx _0}{\bx^\prime _s} \hpot{s}{\bx^\prime _s} \fw{t|s}{\bx^\prime _s}{\bx _t}\, \rmd \bx^\prime _s } \\
    & = \frac{ \fw{s|0}{\bx_0}{\bx_s} \hpot{s}{\bx_s} \fw{t|s}{\bx_s}{\bx_t} \big/ \fw{t|0}{\bx_0}{\bx_t}}{\int \fw{s|0}{\bx _0}{\bx^\prime _s} \hpot{s}{\bx^\prime _s} \fw{t|s}{\bx^\prime _s}{\bx _t} \big/ \fw{t|0}{\bx_0}{\bx_t} \, \rmd \bx^\prime _s } \eqsp.
\end{align*}
Then, by noting that the bridge transition \eqref{eq:bridge} satisfies $\fw{s|0, t}{\bx_0, \bx_t}{\bx_s} = \fw{s|0}{\bx_0}{\bx_s} \fw{t|s}{\bx_s}{\bx_t} / \fw{t|0}{\bx_0}{\bx_t}$, we find that 
$$ 
\epost{s|0, t}{\bx_0, \bx_t}{\bx_s} = \frac{\hpot{s}{\bx_s} \fw{s|0, t}{\bx_0, \bx_t}{\bx_s}}{\int \hpot{s}{\bx^\prime _s} \fw{s|0, t}{\bx_0, \bx_t}{\bx^\prime _s} \, \rmd \bx^\prime _s}
$$
Finally, for the third conditional, using again the identity \eqref{eq:bw-fw}, we find that 
\begin{align*} 
    \epost{t|0, s}{\bx_0, \bx_s}{\bx_t} & = \frac{\pdata{0|s}{\bx_s}{\bx_0} \hpot{s}{\bx_s} \pdata{s|t}{\bx_t}{\bx_s} \pdata{t}{}{\bx_t}}{\int \pdata{0|s}{\bx _s}{\bx_0} \hpot{s}{\bx _s} \pdata{s|t}{\bx^\prime _t}{\bx _s} \pdata{t}{}{\bx^\prime _t}\, \rmd \bx^\prime _t } \\
    & = \fw{t|s}{\bx_s}{\bx_t} \eqsp.
\end{align*}
\subsection{Variational approximation}
\label{apdx-sec:vi-approx}
In this section we describe the variational approach of \citet{moufad2024variational}, which we use to fit a Gaussian variational approximation to $\post{s|0, t}{\bx_0, \bx_t}{}$ for fixed $(\bx_0, \bx_t)$. Similarly to the main paper we consider the variational approximation 
\begin{equation} 
    \smash{\vi{s|0, t}{} \eqdef \gauss\big(\vmu_{s|0, t}, \diag(\rme^{\vlstd_{s|0,t}})\big)}, 
    %\quad \mbox{and} \quad \vparam_{s|0, t} \eqdef (\vmu_{s|0, t}, \vlstd_{s|0,t}) \in \rset^{\dimx} \times \rset^{\dimx}. 
\end{equation}
and let $\vparam_{s|0, t} \eqdef (\vmu_{s|0, t}, \vlstd_{s|0,t}) \in \rset^{\dimx} \times \rset^{\dimx}$ denote the variational parameters. 
The reverse KL divergence writes, following definition \eqref{eq:bridge}, 
\begin{align} 
    \lefteqn{\kldivergence{\vi{s|0, t}{}}{\post{s|0, t}{\bx_0, \bx_t}{}}} \nonumber \\
    & \hspace{.7cm} = \int \log \frac{\vi{s|0, t}{\bx_s}}{\hpot{s}{\bx_s}{} \fw{s|0, t}{\bx_0, \bx_t}{\bx_s}} \, \vi{s|0, t}{\bx_s} \, \rmd \bx_s + \mathrm{C} \nonumber \\
    & \hspace{.7cm} = \pE _{\vi{s|0, t}{}} \left[ - \log \hpot{s}{\vX^\vparam _s} + \frac{\| \vX^\vparam _s - \big( \gamma_{t|s} \a_{s|0} \bx_0 + (1 - \gamma_{t|s}) \a^{-1} _{t|s} \bx_t \big) \|^2}{2 \std^2 _{s|0, t}} \right] - \frac{1}{2} \vlstd_{s|0, t}^T \mathbf{1}_d + \mathrm{C}^\prime. \label{eq:gradient-estimator} 
\end{align}
Using the reparameterization trick \cite{kingma2013auto} and plugging-in the neural network approximation $\hpot{s}{}[\param]$ of $\hpot{s}{}$, we obtain the gradient estimator  
\begin{multline*}
    \nabla_\vparam \mathcal{L}^s _t(\vparam; \bx_0, \bx_t, Z) \eqdef - \nabla_\vparam \log \hpot{s}{\vmu_{s|0, t} + \diag(\rme^{\vlstd_{s|0, t} })^{1/2} Z}[\param] \\
     + \nabla_\vparam \bigg[ \frac{\|\vmu_{s|0, t} + \diag(\rme^{\vlstd_{s|0, t} })^{1/2} Z - \big( \gamma_{t|s} \a_{s|0} \bx_0 + (1 - \gamma_{t|s}) \a^{-1} _{t|s} \bx_t \big)\|^2 }{2 \std^2 _{s|0, t}} - \frac{1}{2} \vlstd_{s|0, t}^T \mathbf{1}_d \bigg], 
\end{multline*}
where $Z \sim \gauss(\zero_\dimx, \Id_\dimx)$. We initialize the variational parameters with the mean and covariance of the bridge kernel \eqref{eq:bridge},  \emph{i.e.}, at initialization, $\vmu^0 _{s|0, t} \eqdef \gamma_{t|s} \a_{s|0} \bx_0 + (1 - \gamma_{t|s}) \a^{-1} _{t|s} \bx_t$ and $\vlstd^0 _{s|0, t} = \log \std^2 _{s|0, t} \Id_\dimx$.
The $\vifn$ routine is summarized in \Cref{algo:gauss_vi}. 
\begin{algorithm}
    \caption{$\vifn$ routine}
    \begin{algorithmic}[1]
        \STATE {\bfseries Input:} vectors $(\bx_0, \bx_t)$, timesteps $(s, t)$, gradient steps $G$
        \STATE $\vmu \gets \gamma_{t|s} \a_{s|0} \bx_0 + (1 - \gamma_{t|s}) \a^{-1} _{t|s} \bx_t$
        \STATE $\vlstd \gets \log \std^2 _{s|0, t}$

        \FOR{$g=1$ to $G$}
            \STATE $Z \sim \gauss(\zero_\dimx, \Id_\dimx)$
            \STATE $(\vmu, \vlstd) \gets \mathsf{OptimizerStep}(\nabla _\vparam \mathcal{L}^s _t(\cdot, \bx_0, \bx_t, Z))$
        \ENDFOR
        \STATE $Z \sim \gauss(\zero_\dimx, \Id_\dimx)$
        \STATE {\bfseries Output:} $\vmu + \diag(\rme^{\vlstd / 2}) Z$
    \end{algorithmic}
    \label{algo:gauss_vi}
\end{algorithm}
\begin{remark} 
    While the expectation of the squared norm in \eqref{eq:gradient-estimator} can be computed exactly, we found that, in practice, doing so degraded the algorithm’s performance, producing blurrier images compared to simply using a Monte Carlo estimator for the full expectation.
\end{remark} 
\begin{remark} 
    \label{rem:metropolis}
    The fact that the density of our target distribution can be computed approximately by plugging the denoiser approximation allows us to add a Metropolis--Hastings (MH) correction with approximate acceptance ratio. Indeed, once we fit the Gaussian approximation, we can improve the accuracy of our sampler by simulating a Markov chain $(\vX^k _s)_k$ where, given $\vX^k _s$, 
    $$ 
    \vX^{k+1} _s \sim M_s(\rmd \bx_s | \vX^k _s) \eqdef \int \vi{s|0,t}{z} \bigg[ r_s(\vX^k _s, z) \delta_{z}(\rmd \bx_s) + (1 - r_s(\vX^k _s, z) ) \delta_{\vX^k _s}(\rmd \bx_s)\bigg]\, \rmd z \eqsp,
    $$ 
    with
    $$ 
        r_s(\bx_s, \bx^* _s) = \mbox{min}\left(1, \frac{\hpot{s}{\bx^* _s} \fw{s|0, t}{\bx_0, \bx_t}{\bx^* _s} \vi{s|0,t}{\bx_s}}{\hpot{s}{\bx _s} \fw{s|0, t}{\bx_0, \bx_t}{\bx _s} \vi{s|0,t}{\bx^* _s}} \right) \eqsp.
    $$ 
\end{remark}
\subsection{Alternative data augmentation and sequence}
\label{apdx-sec:data-aug}
\paragraph{Data augmentation.} Our algorithm is based on one data-augmentation approach, but alternative augmentations could also be considered. 
Let $s \in \intset{1}{t-1}$. Then the most obvious and natural data augmentation involves simply marginalizing out the $\bx_0$ variable in \eqref{eq:extended-distr-normalized}, yielding 
$$ 
    \epost{s, t}{}{\bx_s, \bx_t} \propto \hpot{s}{\bx_s} \pdata{s|t}{\bx_t}{\bx_s} \pdata{t}{}{\bx_t} \eqsp.
$$ 
Its full conditionals are $\epost{s|t}{\bx_t}{\bx_s} \propto \hpot{s}{\bx_s} \pdata{s|t}{\bx_t}{\bx_s}$ and $\epost{t|s}{\bx_s}{\bx_t} = \fw{t|s}{\bx_s}{\bx_t}$. The first conditional is intractable for sampling, and we could approximate it with a Gaussian variational distribution, similar to our approach for $\epost{s|0, t}{\bx_0, \bx_t}{}$. Indeed, this is possible since $\nabla_{\bx_s} \log \epost{s|t}{\bx_t}{\bx_s} = \nabla_{\bx_s} \log \hpot{s}{\bx_s} + \nabla_{\bx_s} \log \pdata{s}{}{\bx_s} + \nabla_{\bx_s} \log \fw{t|s}{\bx_s}{\bx_t}$, which can then be approximated using the parametric approximations $\nabla \log \hpot{s}{\bx_s}[\param]$ and $\nabla \log \pdata{s}{}{\bx_s} \approx (- \bx_s + \a_s \denoiser{s}{}{\bx_s}[\param]) / (1 - \a^2 _s)$. 

The first drawback of this approach is that, in practice, it tends to degrade reconstruction quality---\emph{e.g.}, introducing blurriness---as $t$ tends to $0$, due to the poor approximation of the score near the data distribution. Additionally, beyond the loss of quality, we observe that it produces more incoherent reconstructions with noticeable artifacts. We hypothesize that this issue arises because the distribution we aim to approximately sample involves the prior transition $\pdata{s|t}{}{}$,  which can be highly multi-modal when $s \ll t$. This multi-modality may make the posterior $\epost{s|t}{\bx_t}{}$ more challenging to approximately sample from. On the other hand, when further conditioning on $\bx_0$, the sampling problem becomes more well-behaved, as we then target the posterior of a Gaussian distribution. Finally, while the score of $\epost{\smash{s|t}}{\bx_t}{\bx_s}$ can be easily approximated, its density cannot, preventing the use of a Metropolis--Hastings correction, unless we use the independent proposal $\pdata{s|t}{\bx_t}{}$. However, this approach is suboptimal, as it does not incorporate any information from the observation. This is not the case of the data-augmentation approach we use in \algo\ as we highlight in \Cref{rem:metropolis}. 
\paragraph{Alternative sequence.} An alternative to the mixture of posterior approximations \eqref{eq:posterior-approximation}, on which \algo\ is based, is the posterior formed as a mixture of likelihoods: 
$$ 
    \hpost{t}{}{\bx_t} = \frac{ \sum_{s = 1}^{t-1} \wght^s _t \hpot{t}{\bx_t}[s] \pdata{t}{}{\bx_t} }{\int \sum_{s = 1}^{t-1} \wght^s _t \hpot{t}{\bx^\prime _t}[s] \pdata{t}{}{\bx^\prime _t} \, \rmd \bx^\prime _t} \eqsp, 
$$ 
being the $\bx_t$-marginal of the extended distribution 
\begin{equation}
    \label{eq:mixture-pot-extended}
    \epost{0, \smbs, t}{}{s, \bx_0, \bz, \bx_t} \propto  \wght^s _t \pdata{0|s}{\bz}{\bx_0} \hpot{s}{\bz} \pdata{s|t}{\bx_t}{\bz} \pdata{t}{}{\bx_t} \eqsp.
\end{equation}
Now, let $(s, \bXy_0, \bZy, \bXy_t) \sim \epost{0, \smbs, t}{}{}$; then, conditionally on $s$, the distribution of $(\bXy_0, \bZy, \bXy_t)$ is $\epost{0, s, t}{}{}$, whereas 
$$
s | \bXy_0, \bZy, \bXy_t \, \sim \mbox{Categorical}\left( \left\{ \frac{\wght^\ell _t \hpot{\ell}{\bZy} \fw{\ell|0, t}{\bXy_0, \bXy_t}{\bZy}}{\sum_{k = 1}^{t-1} \wght^k _t \hpot{k}{\bZy} \fw{k|0, t}{\bXy_0, \bXy_t}{\bZy}}\right\}_{\ell = 1} ^{t-1} \right) \eqsp.
$$

A Gibbs sampler targeting \eqref{eq:mixture-pot-extended} is described in \Cref{algo:mixturepot-extended-gibbs}. It allows updating the index $s$ in an observation-driven fashion, but is unfortunately computationally expensive as we need to evaluate the denoiser at $\bZy$ in parallel for $t-1$ timesteps. A cheaper alternative could be to block the variables $(s, \bZy)$ and use an independent MH step to target their joint conditional distribution. Denoting by $\lambda$ the joint proposal distribution on $\intset{1}{t-1} \times \rset^\dimx$ used in this independent MH step, the probability of accepting a candidate $(s^*, \bz^*)$ is 
$$ 
    r_t\big( (s, \bz), (s^*, \bz^*)\big) = \mbox{min}\left( 1, \frac{\wght^{s^*} _{t} \hpot{s^*}{\bz^*}\fw{s^* | 0, t}{\bx_0, \bx_t}{\bz^*} \lambda(s, \bz)}{\wght^{s} _{t} \hpot{s}{\bz}\fw{s | 0, t}{\bx_0, \bx_t}{\bz} \lambda(s^{*}, \bz^{*})} \right) \eqsp.
$$ 
\begin{algorithm}[h]
    \caption{Gibbs sampler targeting \eqref{eq:extended}}
    % \small 
    \begin{algorithmic}[1]
        \STATE {\bfseries Input:} $(s^r, \bXy^r _0, \bZy^r, \bXy^r _t)$
        \STATE draw $s^{r+1} \sim \mbox{Categorical}\left( \left\{ \frac{\wght^\ell _t \hpot{\ell}{\bZy^r} \fw{\ell|0, t}{\bXy^r _0, \bXy^r _t}{\bZy^r}}{\sum_{k = 1}^{t-1} \wght^k _t \hpot{k}{\bZy^r} \fw{k|0, t}{\bXy^r _0, \bXy^r _t}{\bZy^r}}\right\}_{k = 1} ^{t-1} \right)$ 
        \STATE draw $\bZy^{r+1} \sim \epost{s^{r+1} |0, t}{\bXy^r _0, \bXy^r _t}{}$
        \STATE draw $\bXy^{r+1} _t \sim \fw{t|s^{r+1}}{\bZy^{r+1}}{}$ 
        \STATE draw $\bXy^{r+1} _0 \sim \pdata{0|s^{r+1}}{\bZy^{r+1}}{}$
    \end{algorithmic}
    \label{algo:mixturepot-extended-gibbs}
\end{algorithm}
\begin{remark} 
    \label{rem:mgdm-weight}
    Note that we could have used a similar data augmentation \eqref{eq:mixture-pot-extended} for the mixture used in \algo. This would yield the full conditional 
$$
    s | \bXy_0, \bZy, \bXy_t \, \sim \mbox{Categorical}\left( \left\{ \frac{\wght^\ell _t \epost{\ell|0, t}{\bXy_0, \bXy_t}{\bZy}}{\sum_{k = 1}^{t-1} \wght^k _t \epost{\ell|0, t}{\bXy_0, \bXy_t}{\bZy}}\right\}_{k = 1} ^{t-1} \right) \eqsp, 
$$
which is, however, intractable due to the normalizing constant involved in each $\epost{\ell|0, t}{}{}$. 
\end{remark}
\subsection{Related algorithms}
\label{apdx-sec:comparisons}
\paragraph{Comparison with \citet{zhang2024daps}} In this section we clarify the difference between \algo\ and the \daps\ algorithm \cite{zhang2024daps}, which shares some similarities with our approach. The sampling procedure in \daps\ relies on sequential approximate sampling from the joint distribution 
$$ 
    \tilde\pi^\obs _{0:T}(\bx_{0:T}) \eqdef \post{T}{}{\bx_T} \prod_{t = 0}^{T-1} \tilde{\pi}_{t|t+1}(\bx_t | \bx_{t+1}),  
$$ 
where 
\begin{equation}
    \label{eq:daps-bw}
    \tilde\pi^\obs _{t|t+1}(\bx_t | \bx_{t+1}) \eqdef \int \fw{t|0}{\bx_0}{\bx_t} \post{0|t+1}{\bx_{t+1}}{\bx_0} \, \rmd \bx_0
\end{equation}
and $\post{0|t+1}{\bx_{t+1}}{\bx_0} = \post{0}{}{\bx_0} \fw{t+1|0}{\bx_0}{\bx_{t+1}} \big/ \post{t+1}{}{\bx_{t+1}}$. From this definition it follows that 
$$ 
 \post{t}{}{\bx_t} = \int \tilde\pi^\obs _{t|t+1}(\bx_t | \bx_{t+1}) \post{t+1}{}{\bx_{t+1}} \, \rmd \bx_{t+1} \eqsp, 
$$ 
and hence that the marginals of the joint distribution $\tilde\pi^\obs _{0:T}$ are $(\post{t}{}{})_{t = 0}^T$. The canonical backward transition $\post{t|t+1}{\bx_{t+1}}{\bx_t} \propto \post{t}{}{\bx_t} \fw{t+1|t}{\bx_t}{\bx_{t+1}}$ has the alternative form 
$$ 
    \post{t|t+1}{\bx_{t+1}}{\bx_t} = \int \fw{t|0, t+1}{\bx_0, \bx_{t+1}}{\bx_t} \post{0|t+1}{\bx_{t+1}}{\bx_t} \, \rmd \bx_0 \eqsp,
$$
which differs from \eqref{eq:daps-bw} in the use of the bridge transition $q_{t|0, t+1}$ instead of the forward transition $q_{t|0}$. 
%and \eqref{eq:daps-bw} differs from it by the use of the forward transition $q_{t|0}$ in place of the bridge transition $q_{t|0, t+1}$. 

In order to sample from $\tilde\pi_{t|t+1}(\cdot | \bx_{t+1})$, one needs to first sample $X_0 \sim \post{0|t+1}{\bx_{t+1}}{}$ and then $X_t \sim \fw{t|0}{X_0}{}$. \daps\ performs the former step using Langevin dynamics on an approximation of $\post{0|t+1}{\bx_{t+1}}{}$. More specifically, the authors use the approximation 
$$ 
\post{0|t+1}{\bx_{t+1}}{\bx_0} \approx \frac{\pot{0}{\bx_0} \normpdf(\bx_0; \denoiser{t+1}{}{\bx_{t+1}}, r^2 _{t+1} \Id_\dimx)}{\int \pot{0}{\bx^\prime _0} \normpdf(\bx^\prime _0; \denoiser{t+1}{}{\bx_{t+1}}, r^2 _{t+1} \Id_\dimx) \, \rmd \bx^\prime _0} \eqsp,
$$ 
where $r^2 _{t+1}$ is a hyperparameter. This approximation follows by noting that $\post{0|t+1}{\bx_{t+1}}{\bx_0} \propto \pot{0}{\bx_0} \pdata{0|t+1}{\bx_{t+1}}{\bx_0}$ and using the Gaussian approximation of $\pdata{0|t+1}{\bx_{t+1}}{}$ proposed by \citet{song2022pseudoinverse}. The Langevin step is initialized with a sample obtained by discretizing the probability flow ODE \cite{song2021score} between $t+1$ and $0$. 

Both \algo\ and \daps\ perform full noising and denoising steps and operate in a similar manner in this respect (with the distinction that we use DDPM instead of the probability flow ODE). The first fundamental difference is that we sample, conditionally on \(\obs\) and at a random timestep $s$, by drawing from \(\epost{s|0, t}{\bx_0, \bx_t}{} \propto \hpot{s}{\bx_s} \fw{s|0, t}{\bx_0, \bx_t}{\bx_s}\). Unlike \daps, our method does not rely on a density approximation prior to applying an approximate sampler. The second main difference is the fact that within each denoising step, we can increase the number of Gibbs iterations to improve the overall performance, as demonstrated in \Cref{fig:scaling}. This is on top of the number of gradient steps that we use to fit the variational approximation and which enhance the performance when we increase them.  

On the other hand, \daps\ does not require the computation of vector-Jacobian products of the denoiser and is thus more efficient in terms of memory. However it requires many calls to the likelihood function, which can substantially increase the runtime if it is expensive to evaluate. For example, with a latent diffusion model, the runtime of DAPS is at least three times larger than that of \algo, \resample, and \psld. 
\paragraph{Comparison with \citet{moufad2024variational}}
The more recent {\sc{MGPS}} algorithm of \citet{moufad2024variational} is also related to \algo. Similarly to DAPS \cite{zhang2024daps}, their methodology relies on sampling approximately from the posterior transition $\post{t|t+1}{\bx_{t+1}}{}$ at each step of the backward denoising process. It builds on the following decomposition, which holds for all $s \in \intset{0}{t-1}$:
$$ 
    \post{t|t+1}{\bx_{t+1}}{\bx_t} = \int \fw{t|s, t+1}{\bx_s, \bx_{t+1}}{\bx_t} \post{s|t+1}{\bx_{t+1}}{\bx_s} \, \rmd \bx_s \eqsp.
$$ 
One step of {\sc{MGPS}} proceeds by first sampling from an approximation of the posterior transition $\post{s|t+1}{\bx_{t+1}}{}$ and then sampling from the bridge transition to return back to time $t$. The approximation of the posterior transition used in the {\sc{MGPS}} is 
\begin{equation} 
    \label{eq:mgps-approx}
    \post{s|t+1}{\bx_{t+1}}{\bx_s} \approx \frac{\hpot{s}{\bx_s}[\param] \pdata{s|t+1}{\bx_{t+1}}{\bx_s}[\param]}{\int \hpot{s}{\bx^\prime _s} \pdata{s|t+1}{\bx_{t+1}}{\bx^\prime _s}[\param] \, \rmd \bx^\prime _s} \eqsp.
\end{equation}
Here one can then choose $s$ to be sufficiently small to enhance the likelihood approximation, while still having an accurate Gaussian approximation of the transition $\pdata{s|t+1}{\bx_{t+1}}{}$. The authors demonstrate, using a solvable toy example, that this trade-off indeed exists; see \citep[Example 3.2]{moufad2024variational}. The approximate sampling step is then performed by fitting a Gaussian variational approximation to the approximation on the \rhs\ of \eqref{eq:mgps-approx}, similarly to what we do in \Cref{algo:midpoint-gibbs}.  

Both \algo\ and {\sc{MGPS}} leverage the same idea of using, at step time $t$, likelihood approximations at earlier steps $s < t$. While in {\sc{MGPS}} the time $s$ is set deterministically as a function of $t$, we sample it randomly. However, the main difference lies in the step where we sample conditionally on the observation $\obs$. Once the index $s$ is sampled we proceed with $R$ rounds of reverse KL minimization \wrt\ to a \emph{different} target distribution. Indeed, following \Cref{algo:midpoint-gibbs}, in the first round we seek to fit a distribution with density proportional to $\bx_s \mapsto \hpot{s}{\bx_s}[\param] \fw{s|0, t}{\textcolor{purple}{\vX^* _0}, \vX_t}{\bx_s}$, where $\vX^* _0$ is an output from the previous step of the algorithm. 
At step $r$, we fit $\bx_s \mapsto \hpot{s}{\bx_s}[\param] \fw{s|0, t}{\textcolor{purple}{\vX^{r-1} _0}, \vX^{r-1} _t}{\bx_s}$, where $\vX^{r-1} _0$ is sampled using a few DDPM steps starting from $\vX^{r-1} _s$ at time $s$ and $\vX^{r-1} _t \sim \fw{t|s}{\vX^{r-1} _s}{}$. On the other hand, {\sc{MGPS}} fits in a single round the distribution with density proportional to $\bx_s \mapsto \hpot{s}{\bx_s}[\param] \fw{s|0, t+1}{\textcolor{purple}{\denoiser{t+1}{}{\vX_{t+1}}[\param]}, \vX_{t+1}}{\bx_s}$, where $\vX_{t+1}$ is the output of the previous step. Finally, the authors report that the performance of {\sc{MGPS}} improves when the number of gradient steps is increased. In our case, we have two axes, Gibbs iterations $R$ and gradient steps, that allow us to improve the performance when more compute is available. 

% \section{Transition kernels involved in \Cref{algo:midpoint-gibbs}}
\label{apdx:transition-kernels}
% --- local vars ---
\def\gibbsRepKernel{{Q}}
\def\initKernel{{I}}
\def\gibbsKernel{{G}}
\def\lastKernel{{L}}
\def\algoKernel{{P}}

% \newcommandx{\gibbsRepKernel}[1][1=\midpoint]{{Q}_{#1}}
% \newcommandx{\gibbsKernel}[1][1=\midpoint]{{G}_{#1}}
% \newcommandx{\initKernel}[1][1=k]{{I}_{#1}}
% ---


The transition kernel of one repetition of a Gibbs step reads
\begin{equation}
    \label{eq:gibbs-kernel-one-rep}
    \gibbsRepKernel(\bx_{\initpoint}, \bx_k; \rmd \bx_{\initpoint}', \rmd \bx_k')
        = \int
            \mgibbs{\midpoint |\initpoint, k}{\bx_\initpoint, \bx_k}{\rmd \bx_{\midpoint}}
            \mgibbs{\initpoint |\midpoint}{\bx_{\midpoint}}{\rmd \bx_\initpoint'}
            \mgibbs{k |\midpoint}{\bx_{\midpoint}}{\rmd \bx_k'}
    \eqsp,
\end{equation}
The initial kernel of the Gibbs steps writes
\begin{equation}
    \label{eq:gibbs-init-kernel}
    \initKernel(\bx_\initpoint; \rmd \bx_\initpoint', \rmd \bx_k')
        = \delta_{\bx_\initpoint}(\rmd \bx_\initpoint') \fwtrans{k|0}{\bx_{\initpoint}}{\rmd \bx_k'}
    \eqsp.
\end{equation}
Combining these two results, we deduce the transition kernel of one Gibbs step after applying $\gibbsReps$ repetitions
\begin{equation}
    \label{eq:gibbs-step}
    \gibbsKernel(\bx_{\initpoint}; \rmd \bx_{\initpoint}^{(R)}, \rmd \bx_k^{(R)})
        = \int
            \initKernel(\bx_\initpoint; \rmd \bx_\initpoint^{(0)}, \rmd \bx_k^{(0)})
            \prod_{r=0}^{R-1}
            \gibbsRepKernel(\bx_{\initpoint}^{(r)}, \bx_k^{(r)}; \rmd \bx_{\initpoint}^{(r+1)}, \rmd \bx_k^{(r+1)})
    \eqsp,
\end{equation}
The last two steps of the algorithm involve the following kernel, equal up to a constant that depends only on $\bx_2$
\begin{equation}
    \label{eq:algo-last-kernel}
    \lastKernel(\bx_2; \rmd \bx_0)
        \propto\int \potn{}{\bx_1} \bw{1|2}{\bx_2}{\rmd \bx_1} \delta_{\hpredx{1}{(\bx_1)}}(\rmd \bx_0)
    \eqsp.
\end{equation}
Finally, we deduce the simulated distributionx
\begin{equation*}
    \pibw{0}{}{}[\gibbsReps](\rmd \bx_0)
    % \algoKernel(\bx_n; \rmd \bx_0)
        = \int
            \fwmarg{n}{\rmd \bx _n}
            \delta_{\hpredx{n}{(\bx_n)}}(\rmd \bx_0^{n})
            \big[
                \prod_{k=2}^{n-1} \gibbsKernel(\bx_0^{k+1}; \rmd \bx_0^{k}, \rmd \bx_{k})
            \big]
            \lastKernel(\bx_2; \rmd \bx_0)
    \eqsp.
\end{equation*}

% \section{Gaussian case}

% --- local vars ---
\def\cov{\mathbf{\Sigma}}
\def\mean{\boldsymbol{m}}
\def\likelihood{\bfA}
% \def\coefXell{a}
% \def\coefXs{c}
\def\covBridge{\sigma^2}
% \def\meanBridge{\tilde{\mean}}
\def\meanConditional{\hat{\mean}}
\def\potBias{\boldsymbol{a}}
\def\potLikelihood{\hat{\likelihood}}
\def\covPosterior{\mathbf{\Gamma}}
\def\matrixXk{\mathbf{M}}
\def\bfM{\mathbf{M}}
\def\hpimean{\hat{\boldsymbol{\mu}}}
\def\bfH{\mathbf{H}}
\def\bc{\boldsymbol{c}}

\def\LH{\mathbf{H}}
\def\Lbias{\boldsymbol{h}}
\def\Lcov{\mathbf{L}}
% \def\Lcovbefore{\tilde{\mathbf{L}}}

\def\LHbefore{\underline{\mathbf{H}}}
\def\Lbiasbefore{\underline{\boldsymbol{h}}}
\def\Lcovbefore{\underline{\mathbf{L}}}


\def\pizero{\mathbf{M}}
\def\pik{\mathbf{N}}
\def\picov{\mathbf{\Lambda}}
\def\pibias{\boldsymbol{e}}


\def\covtau{\mathbf{\Sigma}_{\bx}}
\def\meantau{\boldsymbol{\mu}_{\bx}}

\def\matzero{\mathbf{C}}
\def\biaszero{\boldsymbol{c}}
\def\covzero{\mathbf{\Sigma}_{c}}

\def\matk{\mathbf{D}}
\def\biask{\boldsymbol{d}}
\def\covk{\mathbf{\Sigma}_{d}}

\def\covthree{\mathbf{\Psi}}
\def\centeredxzero{\bar{\bx}_0'}
\def\centeredxk{\bar{\bx}_k'}


% no need for theta as superscript of the potential
\renewcommand{\hpotn}[2]{\ifthenelse{\equal{#2}{}}{\hat{g} _{#1}}{\hat{g} _{#1}(#2)}}
% ---

Recall that in this setting, the
\begin{itemize}
    \item prior is a Gaussian $\prior = \gauss(\mean, \cov)$, where $(\mean, \cov) \in \rset^\dimx \times \mathcal{S}^{++} _\dimx$
    \item potential is the likelihood of a linear inverse problem $\potn{}{}: \bx \mapsto \normpdf(\obs; \bfA \bx, \stdobs^2 \Id_\dimobs)$
\end{itemize}
% In the setting where the prior is a Gaussian $\prior = \gauss(\mean, \cov)$
% and the potential is the likelihood of a linear inverse problem $\potn{}{}: \bx \mapsto \normpdf(\obs; \bfA \bx, \stdobs^2 \Id_\dimobs)$,
In such a setting, all random variables involved in \Cref{algo:midpoint-gibbs} are Gaussians and hence the algorithm reduces to a sequence of means and covariances.
Depending on the number of gibbs repetitions $\gibbsReps$, \Cref{algo:midpoint-gibbs} simulates a final distribution $\pibw{0}{}{}[\gibbsReps]$.

In this section, we derive the recursion verified by the means and covariances of $\{ \pibw{k}{}{}[\gibbsReps] \}_{0 \leq k \leq n}$.
We proceed by:
% (i) writing the denoising densitites $\{ \bw{\initpoint|k}{}{} \}_{0 \leq k \leq n}$ and the approximate potentials $\{ \hpotn{k}{} \}_{0 \leq k \leq n}$ in the case of $\prior$ gaussian, (ii) deriving the transition kernels in \Cref{apdx:transition-kernels} in that case, (iii) combining the results to form the the recursion.
\begin{enumerate}
    \item writing the denoising densitites $\{ \bw{\initpoint|k}{}{} \}_{0 \leq k \leq n}$ and the approximate potentials $\{ \hpotn{k}{} \}_{0 \leq k \leq n}$
    \item deriving the transition kernels in \Cref{apdx:transition-kernels} in that case
    \item combining the results to form the the recursion
\end{enumerate}


\paragraph{Denoising densities and potentials.} Since we are dealing with a Gaussian prior, the denoiser $\predx{k}$ can be computed in closed form for any $k \in \intset{1}{n}$. Using \citet[Eqn.~2.116]{bishop2006pattern}, we have that
    \begin{align*}
        \bw{0|k}{\bx _k}{\bx _0}
            & \propto \prior(\bx _0) \fwtrans{k|0}{\bx _0}{\bx _k} \\
            & = \normpdf \big(\bx _0;
                \cov_{0|k} \big( (\sqrt{\acp{k}} / \var_k) \bx _k + \cov^{-1} \mean \big), \cov_{0|k} \big),
    \end{align*}
where $\cov_{0|k} \eqdef ((\acp{k} / \var_k) \Id + \cov^{-1})^{-1}$.
A byproduct is an expression of the denoiser
\begin{equation*}
    \predx{k}(\bx _k) = \cov_{0|k}\big( (\sqrt{\acp{k}} / \var_k) \bx _k + \cov^{-1} \mean \big)
    \eqsp.
\end{equation*}

The approximate potentials $\hpotn{k}{} = \potn{k}{} \circ \predx{k}$ also have an explicit expression that reads
\begin{equation*}
    \hpotn{k}{\bx _k}
        = \normpdf(\obs; \hat\bfA_k \bx _k + \potBias_k, \stdobs^2 \Id_\dimobs),
    \quad
    \mathrm{where}
    \quad
    \potLikelihood_k = (\sqrt{\acp{k}} / \var_k)  \likelihood \cov_{0|k}, \quad \potBias_k = \likelihood \cov_{0|k} \cov^{-1} \mean
    \eqsp.
\end{equation*}


\paragraph{Transition kernels.}
The expression of the initial kernel \eqref{eq:gibbs-init-kernel} writes
\begin{equation*}
    % \label{eq:init-kernel-gauss}
    \initKernel(\bx_\initpoint; \rmd \bx_\initpoint', \rmd \bx_k')
        = \delta_{\bx_\initpoint}(\rmd \bx_\initpoint')
            \ \normpdf(\rmd \bx_k'; \sqrt{\acp{k}} \bx_0, \var_{k} \Id_\dimx)
    \eqsp.
\end{equation*}
This can be translated into the following linear system of random variables
\begin{align*}
    \bX_0' 
        & = \bX_0,
        \\
    \bX_k' 
        & = \sqrt{\acp{k}} \bX_0 + \sqrt{\var_k} \bZ, 
        \qquad \bZ \sim \normpdf(\zero, \Id_\dimx).
\end{align*}
Notice here that $\bX_0'$ and $\bX_k'$ of the next state are interlinked
$\pCov[\bX_0', \bX_k'] = \sqrt{\acp{k}} \pV[\bX_0]$, an that
\begin{equation}
    \label{eq:gibbs-init-kernel-gauss-update}
    \pE\begin{bmatrix}\bX_0'\\\bX_k'\end{bmatrix} = \begin{bmatrix}\pE[\bX_0]\\\sqrt{\acp{k}}\pE[\bX_0]\end{bmatrix},
    \quad
    \pCov\begin{bmatrix}\bX_0'\\\bX_k'\end{bmatrix} = \begin{bmatrix}\pV[\bX_0] & \sqrt{\acp{k}}\pV[\bX_0]\\\cdot & \var_{k} \Id_\dimx + \acp{k} \pV[\bX_0]\end{bmatrix}
    \eqsp.
\end{equation}

It is easy to see that \Cref{eq:gibbs-kernel-one-rep} can also be expressed as a linear system of random variables
% \begin{equation}
%     \label{eq:gibbs-kernel-one-rep-gauss}
%     \gibbsRepKernel(\bx_{\initpoint}, \bx_k; \rmd \bx_{\initpoint}', \rmd \bx_k')
%         = \normpdf(\rmd \bx_{\initpoint}';
%             \QMzero_k \bx_0 + \QNzero_k \bx_k + \Qbiaszero_k, \Qcovzero_k) \
%             \normpdf(\rmd \bx_k'; \QMk_k \bx_0 + \QNk_k \bx_k + \Qbiask_k, \Qcovk_k)
%     \eqsp.
% \end{equation}
% \begin{align*}
%     \bX_0' 
%         & = \QMzero_k \bX_0 + \QNzero_k \bX_k + \Qbiaszero_k + \Qcovzero_k^{1/2} \bZ_0,
%         \qquad \bZ_0 \sim \normpdf(\zero, \Id),
%         \\
%     \bX_k' 
%         & = \QMk_k \bX_0 + \QNk_k \bX_k + \Qbiask_k + \Qcovk_k^{1/2} \bZ_k, 
%         \qquad \bZ_k \sim \normpdf(\zero, \Id).
% \end{align*}
\begin{equation*}
    \begin{bmatrix}\bX_0'\\\bX_k'\end{bmatrix}
    =
    \boldsymbol{b}_k
    + 
    \mathbf{B}_k \begin{bmatrix}\bX_0\\\bX_k\end{bmatrix}
    +
    \mathbf{\Gamma}^{1/2}_k \bZ_{0,k}
    , \quad
    \bZ_{0,k} \sim \normpdf(\zero_{2\dimx}, \Id_{2\dimx})
    \eqsp.
\end{equation*}
For clarity, we defer the expressions of the introduced matrices $\mathbf{B}_k, \mathbf{\Gamma}_k$, and the bias $\boldsymbol{b}_k$ to \Cref{apdx:details-expressions-gauss}.
Based on the values of expected value of the vector $[\bX_0, \bX_k]$ and its covariance, we can deduce the expected value and the covariances of the next state $\bX_0', \bX_k'$ as follow
\begin{equation}
    \label{eq:gibbs-kernel-one-rep-gauss}
    \pE\begin{bmatrix}\bX_0'\\\bX_k'\end{bmatrix}
        =
        \boldsymbol{b}_k
        + 
        \mathbf{B}_k \ \pE\begin{bmatrix}\bX_0\\\bX_k\end{bmatrix},
    \quad
    \pCov\begin{bmatrix}\bX_0'\\\bX_k'\end{bmatrix}
        = 
        \mathbf{B}_k \pCov\begin{bmatrix}\bX_0\\\bX_k\end{bmatrix} \mathbf{B}_k^\top + \mathbf{\Gamma}_k
    \eqsp.
\end{equation}
% For clarity, we defer the expressions of the introduced matrices $\QMzero_k, \QNzero_k, \Qcovzero_k$, and $\QMk_k, \QNk_k, \Qcovk_k$, and the biases $\Qbiaszero_k, \Qbiask_k$ to \Cref{apdx:details-expressions-gauss}.
% The expression of the initial kernel \eqref{eq:gibbs-init-kernel} writes
% \begin{equation}
%     \label{eq:init-kernel-gauss}
%     \initKernel(\bx_\initpoint; \rmd \bx_\initpoint', \rmd \bx_k')
%         = \delta_{\bx_\initpoint}(\rmd \bx_\initpoint')
%             \ \normpdf(\rmd \bx_k'; \sqrt{\acp{k}} \bx_0, \var_{k} \Id_\dimx)
%     \eqsp.
% \end{equation}
Finally, the last kernel \eqref{eq:algo-last-kernel} is a Gaussian with a linear mean on $\bx_2$
\begin{equation*}
    \lastKernel(\bx_2; \rmd \bx_0') = \normpdf(\rmd \bx_0'; \LH \bx_2 + \Lbias, \Lcov)
    \eqsp.
\end{equation*}
Also here for clarity, we defer the expressions of the introduced matrices $\LH, \Lcov$ and the bias  $\Lbias$ to \Cref{apdx:details-expressions-gauss}.
Therefore, we deduce the updates
\begin{equation}
    \label{eq:last-kernel-gauss}
    \pE[\bX_0'] = \LH \pE[\bX_2] + \Lbias,
    \quad
    \pV[\bX_0'] = \LH \pV[\bX_2] \LH^\top + \Lcov
    \eqsp.
\end{equation}

\paragraph{Recursion.}
By combining \eqref{eq:gibbs-init-kernel-gauss-update}, \eqref{eq:gibbs-kernel-one-rep-gauss}, and \eqref{eq:last-kernel-gauss}, \Cref{algo:midpoint-gibbs} reduces to the following recursion
% \badr{here the init $\delta_m(\rmd \bx_0^n)$ is valid if we were in the perfect case where the diffusion converges to a Gaussian}
% \begin{equation*}
%     \begin{aligned}
%         \pE[\bX_0^{(0)}] = \bx_0, \pV[\bX_0^{(0)}] = \zero,
%             \quad & \quad
%         \pE[\bX_k^{(0)}] = \sqrt{\acp{k}} \bx_0, \pV[\bX_k^{(0)}] = \var_{k} \Id_\dimx, \\
%         \pE[\bX_0^{(r+1)}] = \QMzero_k \pE[\bX_0^{(r)}] + \QNzero_k \pE[\bX_k^{(r)}],
%             \quad & \quad
%         \pV[\bX_0^{(r+1)}] = \Qcovzero_k + \QMzero_k  \pV[\bX_0^{(r)}] \QMzero_k^\top + \QNzero_k \pV[\bX_k^{(r)}] \QNzero_k^\top,
%         \\
%         \pE[\bX_k^{(r+1)}] = \QMk_k \pE[\bX_0^{(r)}] + \QNk_k \pE[\bX_k^{(r)}],
%             \quad & \quad
%         \pV[\bX_k^{(r+1)}] = \Qcovk_k + \QMk_k \pV[\bX_0^{(r)}] \QMk_k^\top + \QNk_k \pV[\bX_k^{(r)}] \QNk_k^\top,
%     \end{aligned}
% \end{equation*}
\begin{algorithm}[h]
    \caption{\Cref{algo:midpoint-gibbs} in the Gaussian case}
    \begin{algorithmic}[1]
        \STATE $\pE[\bX_{\initpoint}^n] \gets \mean,
            \quad
            \pV[\bX_{\initpoint}^n] \gets \zero$
        \FOR{$k=n-1$ to $2$}
            % \STATE {\bfseries Pick:} $\midpoint \sim \uniform(\intset{1}{k-1})$
            % \STATE {\bfseries Init:} $\bX^{(0)}_k \sim \fwtrans{k|0}{\bX_{\initpoint}}{\cdot}, \, \bX^{(0)}_{\initpoint} \gets \bX_{\initpoint}$

            % \STATE $\pE[\bX_0^{(0)}] \gets \pE[\bX_0^{k+1}],
            % \quad
            % \pV[\bX_0^{(0)}] \gets  \pV[\bX_0^{k+1}]$
            % \STATE $\pE[\bX_k^{(0)}] \gets \sqrt{\acp{k}} \ \pE[\bX_0^{k+1}],
            % \quad
            % \pV[\bX_k^{(0)}] \gets \var_{k} \Id_\dimx + \acp{k} \pV[\bX_0^{k+1}]$
            % \STATE $\pCov[\bX_0^{(0)}, \bX_k^{(0)}] \gets \sqrt{\acp{k}} \pV[\bX_0^{k+1}]$
            
            \STATE $\pE\begin{bmatrix}\bX_0^{(0)}\\\bX_k^{(0)}\end{bmatrix} \gets \begin{bmatrix}\pE[\bX_0^{k+1}]\\\sqrt{\acp{k}}\pE[\bX_0^{k+1}]\end{bmatrix},$ \qquad
            $\pCov\begin{bmatrix}\bX_0^{(0)}\\\bX_k^{(0)}\end{bmatrix} \gets \begin{bmatrix}\pV[\bX_0^{k+1}] & \sqrt{\acp{k}}\pV[\bX_0^{k+1}]\\\cdot & \var_{k} \Id_\dimx + \acp{k} \pV[\bX_0^{k+1}]\end{bmatrix}$

            \FOR{$r = 0$ {\bfseries to} $R-1$}
                % \STATE $\pE[\bX_0^{(r+1)}] \gets \QMzero_k \pE[\bX_0^{(r)}] + \QNzero_k \pE[\bX_k^{(r)}] + \Qbiaszero_k,
                % \quad
                % \pV[\bX_0^{(r+1)}] \gets \Qcovzero_k + \QMzero_k  \pV[\bX_0^{(r)}] \QMzero_k^\top + \QNzero_k \pV[\bX_k^{(r)}] \QNzero_k^\top$
                % \STATE $\pE[\bX_k^{(r+1)}] \gets \QMk_k \pE[\bX_0^{(r)}] + \QNk_k \pE[\bX_k^{(r)}] + \Qbiask_k,
                % \quad
                % \pV[\bX_k^{(r+1)}] \gets \Qcovk_k + \QMk_k \pV[\bX_0^{(r)}] \QMk_k^\top + \QNk_k \pV[\bX_k^{(r)}] \QNk_k^\top$
                % \STATE \emph{Apply \Cref{eq:gibbs-kernel-one-rep-gauss} starting from $\pE[\bX_0^{(r)}], \pE[\bX_k^{(r)}], \pV[\bX_0^{(r)}], \pV[\bX_k^{(r)}]$ and $\pCov[\bX_0^{(r)}, \bX_k^{(r)}]$}
                % \STATE \emph{to compute $\pE[\bX_0^{(r+1)}], \pE[\bX_k^{(r+1)}], \pV[\bX_0^{(r+1)}], \pV[\bX_k^{(r+1)}]$ and $\pCov[\bX_0^{(r+1)}, \bX_k^{(r+1)}]$}
                
                \STATE $\pE\begin{bmatrix}\bX_0^{(r+1)}\\\bX_k^{(r+1)}\end{bmatrix}
                \gets
                \boldsymbol{b}_k
                + 
                \mathbf{B}_k \ \pE\begin{bmatrix}\bX_0^{(r)}\\\bX_k^{(r)}\end{bmatrix}$,
                \qquad
                $\pCov\begin{bmatrix}\bX_0^{(r+1)}\\\bX_k^{(r+1)}\end{bmatrix}
                \gets 
                \mathbf{B}_k \pCov\begin{bmatrix}\bX_0^{(r)}\\\bX_k^{(r)}\end{bmatrix} \mathbf{B}_k^\top + \mathbf{\Gamma}_k$
                
            \ENDFOR
            \STATE $\pE[\bX_{\initpoint}^k] \gets \pE[\bX^{(R)}_{\initpoint}],
            \quad
            \pV[\bX_{\initpoint}^k] \gets \pV[\bX^{(R)}_{\initpoint}]$
            \STATE $\pE[\bX_k] \gets \pE[\bX^{(R)}_k],
            \quad
            \pV[\bX_k] \gets \pV[\bX^{(R)}_k]$
        \ENDFOR

        \STATE $\pE[\bX_0] \gets \LH \ \pE[\bX_2] + \Lbias
        \quad
        \pV[\bX_0] \gets \Lcov + \LH \ \pV[\bX_2] \ \LH^\top$
    \end{algorithmic}
\end{algorithm}


\paragraph{Expressions of the introduced matrices and biases.}
\label{apdx:details-expressions-gauss}
To derive the expression, we utilize \citet[Eqn.~2.116]{bishop2006pattern} and the convolution of two Gaussians.
Let us start we the kernel $\gibbsRepKernel$ \eqref{eq:gibbs-kernel-one-rep-gauss}.
We have
\begin{equation*}
    \begin{aligned}
        \mgibbs{\midpoint |\initpoint, k}{\bx_\initpoint, \bx_k}{\bx_{\midpoint}}
            & \propto \hpotn{\midpoint}{\bx _\midpoint} \fwtrans{\midpoint |\initpoint, k}{\bx_\initpoint, \bx_k}{\bx_{\midpoint}}\\
            & \propto \normpdf(\obs; \hat\bfA_\midpoint \bx _\midpoint + \potBias_\midpoint, \stdobs^2 \Id_\dimobs)
                \ \normpdf(\bx_{\midpoint}; \meanBridge_{\midpoint|0, k}(\bx _0, \bx _k), \var_{\midpoint|0, k} \Id_\dimx) \\
            & = \normpdf(\bx_\tau;
                \pizero_{\midpoint| 0, k} \bx_0 + \pik_{\midpoint| 0, k} \bx_k + \pibias_{\midpoint| 0, k},
                % var
                \picov_{\midpoint| 0, k})
    \end{aligned}
\end{equation*}
where
\begin{align*}
    \picov_{\midpoint|0, k}
        & = \big[ (1 /\var_{\midpoint | 0, k}) \Id_\dimx  + (1 / \stdobs^2)\hat\bfA_\midpoint^\top \hat\bfA_\midpoint \big]^{-1},\\
    \pizero_{\midpoint| 0, k}
        & = \frac{\sqrt{\acp{\midpoint}{} / \acp{0}}(1 - \acp{k} / \acp{\midpoint})}{\var_{\midpoint|0, k} (1 - \acp{k} / \acp{0})} \picov_{\midpoint| 0, k}
    ,\qquad
    \pik_{\midpoint| 0, k}
        = \frac{\sqrt{\acp{k} / \acp{\midpoint}} (1 - \acp{\midpoint} / \acp{0})}{\var_{\midpoint|0, k} (1 - \acp{k} / \acp{0})}  \picov_{\midpoint| 0, k}, \\
    \pibias_{\midpoint| 0, k}
        & = (1/\stdobs^2) \picov_{\midpoint| 0, k} \hat\bfA_\midpoint^\top (\obs - \potBias_\midpoint).\\
\end{align*}
After establishing the the distribution $\mgibbs{\midpoint |\initpoint, k}{}{}$, we can now compute the action of the kernel $\gibbsRepKernel$ on $\bX_0, \bX_k$.
Without lost of generality, we can write the integral involved in $\gibbsRepKernel$ as
\begin{align*}
    \int
        \mgibbs{\midpoint |\initpoint, k}{\bx_\initpoint, \bx_k}{\bx_{\midpoint}}
        \mgibbs{\initpoint |\midpoint}{\bx_{\midpoint}}{\bx_\initpoint'}
        \mgibbs{k |\midpoint}{\bx_{\midpoint}}{\bx_k'} \rmd \bx_\tau
        & = \int
            \normpdf(\bx_\tau; \meantau, \covtau)
            \normpdf(\bx_0'; \matzero \bx_\tau + \biaszero, \covzero)
            \normpdf(\bx_k'; \matk \bx_\tau + \biask, \covk)
            \rmd \bx_\tau
\end{align*}
where
\begin{align*}
    \meantau = \pizero_{\midpoint| 0, k} \bx_0 + \pik_{\midpoint| 0, k} \bx_k + \pibias_{\midpoint| 0, k}, \quad
    \covtau = \picov_{\midpoint|0, k},
    \\
    \matzero = (\sqrt{\acp{\tau}}/\var_{\tau}) \cov_{0|\tau}, \quad 
    \biaszero = \cov_{0|\tau} \cov^{-1} \mean, \quad
    \covzero = \cov_{0 | \tau},
    \\
    \matk = \sqrt{\acp{k}/\acp{\tau}} \Id, \quad 
    \biask = \zero, \quad
    \covk = \var_{k|\tau} \Id.
\end{align*}
From now on, we can focus only the quadratic product in the Gaussians exponent.
We have
\begin{multline*}
    \| \bx_\tau - \meantau \|^2_{\covtau^{-1}}
    +  \| \bx_0' - (\matzero \bx_\tau + \biaszero) \|^2_{\covzero^{-1}}
    + \| \bx_k' - (\matk \bx_\tau + \biask) \|^2_{\covk^{-1}}
    = 
    \\
    \| \bx_\tau \|^2_{\covtau^{-1} + \matzero^\top \covzero^{-1} \matzero + \matk^\top \covk^{-1} \matk}
    \\
    - 2 \langle \covtau^{-1} \meantau + \matzero^\top \covzero^{-1} (\bx_0'- \biaszero)+ \matk^\top \covk^{-1} (\bx_k'- \biask), \bx_\tau  \rangle
    \\
    + \| \meantau \|^2_{\covtau^{-1}} + \| \bx_0' - \biaszero \|^2_{\covzero^{-1}} + \| \bx_k' - \biask \|^2_{\covk^{-1}}
\end{multline*}
For the sake of conciseness, denote by
\begin{align*}
    \covthree^{-1} 
        & = \covtau^{-1} + \matzero^\top \covzero^{-1} \matzero + \matk^\top \covk^{-1} \matk,
        \\
    \centeredxzero & = \bx_0' - \biaszero, \qquad
    \centeredxk  = \bx_k' - \biask.
\end{align*}
The previous quadratic sum equal up to a constant independent of $\bx_\tau, \centeredxzero, \centeredxk$, to
\begin{align*}
    \| \bx_\tau - \covthree (\covtau^{-1} \meantau + \matzero^\top \covzero^{-1} \centeredxzero + \matk^\top \covk^{-1} \centeredxk) \|^2_{\covthree^{-1}}
    + \| \centeredxzero \|^2_{\covzero^{-1}} + \| \centeredxk \|^2_{\covk^{-1}}
    - \| \covtau^{-1} \meantau + \matzero^\top \covzero^{-1} \centeredxzero + \matk^\top \covk^{-1} \centeredxk \|^2_{\covthree}
    % \eqsp.
\end{align*}
The first term will define a Gaussian on $\bx_\tau$ and hence will integrate to one.
Hence, we get ride of the integration and will be left three terms.
The goal is to write them in form amenable to Gaussians.
Building up the previous expression, we have
\begin{multline*}
    \| \centeredxzero \|^2_{\covzero^{-1}} + \| \centeredxk \|^2_{\covk^{-1}}
    - \| \covtau^{-1} \meantau + \matzero^\top \covzero^{-1} \centeredxzero + \matk^\top \covk^{-1} \centeredxk \|^2_{\covthree}
    = \\
    \| \centeredxzero \|^2_{\covzero^{-1} - \covzero^{-1} \matzero \covthree \matzero^\top \covzero^{-1}} 
    + \| \centeredxk \|^2_{\covk^{-1} - \covk^{-1} \matk \covthree \matk^\top \covk^{-1}}
    \\
    - 2 \langle \matzero^\top \covzero^{-1} \centeredxzero, \matk^\top \covk^{-1} \centeredxk \rangle_{\covthree}
    - 2 \langle \covtau^{-1} \meantau, \matzero^\top \covzero^{-1} \centeredxzero \rangle_{\covthree}
    - 2 \langle \covtau^{-1} \meantau, \matk^\top \covk^{-1} \centeredxk \rangle_{\covthree}
\end{multline*}
where the equality is up to a constant independent of $\centeredxzero, \centeredxk$.

Now let us introduce the block matrices
\begin{equation*}
    \begin{aligned}
        \mathbf{\Gamma}^{-1} & = \begin{bmatrix}
            \covzero^{-1} - \covzero^{-1} \matzero \covthree \matzero^\top \covzero^{-1} &  -\covzero^{-1} \matzero \covthree \matk^\top \covk^{-1} \\
            \cdot & \covk^{-1} - \covk^{-1} \matk \covthree \matk^\top \covk^{-1}
        \end{bmatrix}
        \\
        \mathbf{J} & = \begin{bmatrix}
            \covzero^{-1} \matzero \covthree \covtau^{-1} & \zero \\
            \zero &  \covk^{-1} \matk \covthree \covtau^{-1} 
        \end{bmatrix}
    \end{aligned}
    \eqsp,
\end{equation*}
where the "$\cdot$" in the the expression of $\mathbf{\Gamma}^{-1}$ stands for the transpose of the off-diagonal element.
It follows that we can rewrite the previous equation in blocks as
\begin{align*}
    \begin{bmatrix}\centeredxzero \\ \centeredxk \end{bmatrix}^\top
    \mathbf{\Gamma}^{-1}
    \begin{bmatrix}\centeredxzero \\ \centeredxk \end{bmatrix}
    - 2 \langle
        \mathbf{J} \begin{bmatrix}\meantau \\ \meantau \end{bmatrix}
        ,
        \begin{bmatrix}\centeredxzero \\ \centeredxk \end{bmatrix}
    \rangle
    = 
    \| 
        \begin{bmatrix}\centeredxzero \\ \centeredxk \end{bmatrix} 
        -
        \mathbf{\Gamma} \mathbf{J} \begin{bmatrix}\meantau \\ \meantau \end{bmatrix}
    \|_{\mathbf{\Gamma}^{-1}}^2
    \eqsp,
\end{align*}
where the equality is up to a constant independent of $\centeredxzero, \centeredxk$.
Therefore, we can deduce that the random vector $[\bX_0', \bX_k']^\top$ is Gaussian as follows
\begin{equation*}
    \begin{aligned}
       \begin{bmatrix}\bX_0'\\\bX_k'\end{bmatrix}
       \sim
       \normpdf(
            \begin{bmatrix}\biaszero\\\biask\end{bmatrix}
            + \mathbf{\Gamma} \mathbf{J} \begin{bmatrix}\meantau \\ \meantau\end{bmatrix}
            ,
            \mathbf{\Gamma}
       )
    \end{aligned}
\end{equation*}
We can now intervene $\bX_0, \bX_k$ by subtituting with the value of $\meantau$
\begin{align*}
    \begin{bmatrix}\meantau \\ \meantau\end{bmatrix}
        = \begin{bmatrix}
            \pizero_{\midpoint| 0, k} & \pik_{\midpoint| 0, k}\\ 
            \pizero_{\midpoint| 0, k} & \pik_{\midpoint| 0, k} 
        \end{bmatrix} 
        \begin{bmatrix}\bX_0 \\ \bX_k\end{bmatrix}
        +
        \begin{bmatrix}\pibias_{\midpoint| 0, k} \\ \pibias_{\midpoint| 0, k}\end{bmatrix}
        = \mathbf{K} \begin{bmatrix}\bX_0 \\ \bX_k\end{bmatrix} + \begin{bmatrix}\pibias_{\midpoint| 0, k} \\ \pibias_{\midpoint| 0, k}\end{bmatrix}
\end{align*}
and therefore, we can write
\begin{equation*}
    \begin{aligned}
       \begin{bmatrix}\bX_0'\\\bX_k'\end{bmatrix}
       =
        \begin{bmatrix}\biaszero\\\biask\end{bmatrix}
        + 
        \mathbf{\Gamma} \mathbf{J} \begin{bmatrix}\pibias_{\midpoint| 0, k} \\ \pibias_{\midpoint| 0, k}\end{bmatrix}
        + 
        \mathbf{\Gamma} \mathbf{J} \mathbf{K} \begin{bmatrix}\bX_0\\\bX_k\end{bmatrix}
        +
        \mathbf{\Gamma}^{1/2} \bZ
    \end{aligned}
\end{equation*}
This enables us to identify the quantities
\begin{equation*}
    \boldsymbol{b}_k = \begin{bmatrix}\biaszero\\\biask\end{bmatrix}
        + 
        \mathbf{\Gamma} \mathbf{J} \begin{bmatrix}\pibias_{\midpoint| 0, k} \\ \pibias_{\midpoint| 0, k}\end{bmatrix},
    \quad
    \mathbf{B}_k = \mathbf{\Gamma} \mathbf{J} \mathbf{K},
    \quad
    \mathbf{\Gamma}_k = \mathbf{\Gamma}
    \eqsp.
\end{equation*}

% Therefore, we deduce the expression of  $\QMzero_k, \QNzero_k, \Qcovzero_k$, and $\Qbiaszero_k$

% \begin{align*}
%     \Qcovzero_k
%         & = \cov_{0|\tau} + (\acp{\tau}/\var_{\tau}^2) \cov_{0|\tau} \picov_{\midpoint|0,k}\cov_{0|\tau}, \\
%     \QMzero_k
%         & = (\sqrt{\acp{\tau}}/\var_{\tau}) \cov_{0|\tau} \pizero_{\midpoint| 0, k},
%     \qquad
%     \QNzero_k
%         = (\sqrt{\acp{\tau}}/\var_{\tau}) \cov_{0|\tau} \pik_{\midpoint| 0, k}, \\
%     \Qbiaszero_k
%         & = (\sqrt{\acp{\tau}}/\var_{\tau}) \cov_{0|\tau} \pibias_{\midpoint| 0, k} + \cov_{0|\tau} \cov^{-1} \mean
% \end{align*}
% and $\QMk_k, \QNk_k, \Qcovk_k$, and $\Qbiask_k$
% \begin{align*}
%     \Qcovk_k
%         & = \var_{k | \midpoint} \Id_{\dimx} + (\acp{k}/\acp{\midpoint}) \picov_{\midpoint| 0, k},\\
%     \QMk_k
%         & = \sqrt{\acp{k}/\acp{\midpoint}} \pizero_{\midpoint| 0, k},
%     \qquad
%     \QNk_k
%         = \sqrt{\acp{k}/\acp{\midpoint}} \pik_{\midpoint| 0, k},
%     \\
%     \Qbiask_k
%         & = \sqrt{\acp{k}/\acp{\midpoint}} \pibias_{\midpoint| 0, k}.
% \end{align*}

Similarly for the kernel $\lastKernel$ \eqref{eq:last-kernel-gauss}, we have
\begin{equation*}
    \begin{aligned}
        \potn{}{\bx_1} \bw{1|2}{\bx_2}{\bx_1}
            & = \normpdf(\obs; \bfA \bx_1, \stdobs^2 \Id_\dimobs) \ \fwtrans{1 | 0, 2}{\predx{2}(\bx_2), \bx_2}{\bx_1} \\
            & = \normpdf \Big(\bx_1;
                \Lcovbefore \big((1/\stdobs^2) \bfA^\top \obs + (1/\var_{1|0, 2}) (a \predx{2}(\bx_2) + b \bx_2) \big),
                \Lcovbefore
            \Big)\\
            & = \normpdf \Big(\bx_1;
                \LHbefore \bx_2 + \Lbiasbefore,
                \Lcovbefore
             \Big)
    \end{aligned}
\end{equation*}
where
\begin{align*}
    \Lcovbefore 
        & = \big[ (1/\var_{1|0, 2}) \Id_\dimx + (1/\stdobs^2) \bfA^\top\bfA \big]^{-1},
        \\
    \LHbefore
        & =  \Lcovbefore \big(
            (\sqrt{\acp{2}}/(\var_{2}\var_{1|0,2})) a \cov_{0|2} + (b/\var_{1|0,2}) \Id
            \big),
        \\
    \Lbiasbefore
        & = \Lcovbefore \big( (1/\stdobs^2) \bfA^\top \obs + (a/\var_{1|0,2}) \cov_{0|2}\cov^{-1} \mean \big)
\end{align*}

The last kernel \eqref{eq:algo-last-kernel} involves $\delta_{\predx{1}(\bx_1)}(\bx_0)$, hence, we plug the expression $\bx_1$ as a function of $\bx_0$
\begin{align*}
    \bx_0 = \predx{1}(\bx_1) \implies \bx_1 = (\var_1 / \sqrt{\acp{1}}) (\cov_{0|1}^{-1} \bx_0 - \cov^{-1} \mean)
    \eqsp,
\end{align*}
into the previous expression to obtain
\begin{align*}
    \Lcov
        & = (\acp{1}/ \var_1^2) \cov_{0|1} \ \Lcovbefore \ \cov_{0|1},
    \qquad
    \LH
        = (\sqrt{\acp{1}}/ \var_1) \cov_{0|1} \LHbefore,\\
    \Lbias
        & = (\sqrt{\acp{1}}/ \var_1) \cov_{0|1} (\Lbiasbefore + (\var_1/\sqrt{\acp{1}}) \cov^{-1}\mean).
\end{align*}




% --- redef notation to not impact the rest of the document
\renewcommand{\hpotn}[2]{\ifthenelse{\equal{#2}{}}{\hat{g}^\param _{#1}}{\hat{g}^\param _{#1}(#2)}}
% ---
\section{Experiments details}
\subsection{Choice of weight sequence}
\label{apdx-sec:weight-seq}
In all our experiments we draw the index $s$, at time $t_i$, from $\mbox{Uniform}\intset{\tau}{t_{i-1}}$ with $\tau = 10$. The main motivation behind setting $\tau = 10$ and not $\tau = 1$, which is more natural, is that we have found that otherwise it may lead to instabilities. This arises typically when an index $s$ is sampled very close to $0$ when $t \approx T$. To avoid such behavior we use a smaller learning rate in \Cref{algo:gauss_vi} for the first few iterations and set $\tau > 1$. For the last $25\%$ diffusion steps we set $s$ deterministically to $t_{i-1}$ as we have found that this slightly improves the reconstructions quality. We also ramp up the number of gradient steps as this significantly sharpens the details in the images. 

While it is more intuitive to sample $s$ close to $0$ as it provides the best approximation error for the likelihood, we have found that this can significantly slow the mixing of the Gibbs sampler in very large dimensions and provides rather poor results when used with a small number of Gibbs steps. Practically speaking, significant artifacts arise during the initial iterations of the algorithm due to the optimization procedure, and they tend to persist in subsequent iterations when $s$ is sampled close to 0. To see why this is the case consider the following empirical discussion on a simplified scenario. We write $\bx = [\bar\bx, \underline\bx]$  where $\bar\bx \in \rset^\dimobs$ and $\underline\bx \in \rset^{\dimx - \dimobs}$. 
We assume that $\pot{0}{\bx} = \normpdf(\obs; \bar\bx, \stdobs^2 \Id_\dimobs)$, \emph{i.e.}, we observe only the first $\dimobs$ coordinates of the hidden state. Since $s$ is sampled near $0$ we may assume that $\hpot{s}{} = \pot{0}{}$. Then, sampling $Z \sim \epost{s|0, t}{\bx_0, \bx_t}{}$  is equivalent to sampling 
\begin{align*}
    \bar{Z} & \sim \gauss\left(\frac{\std^2 _{s|0, t}}{\stdobs^2 + \std^2 _{s|0,t}} \obs + \frac{\stdobs^2}{\stdobs^2 + \std^2 _{s|0, t}} \big[ \gamma_{t|s} \a_{s|0} \bar\bx_0 + (1 - \gamma_{t|s}) \a^{-1} _{t|s} \bar\bx_t\big], \frac{\stdobs^2 \std^2 _{s|0,t}}{\stdobs^2 + \std^2 _{s|0,t}} \Id_\dimobs\right) \eqsp, \\
    \underline{Z} & \sim \gauss(\gamma_{t|s} \a_{s|0} \underline\bx_0 + (1 - \gamma_{t|s}) \a^{-1} _{t|s} \underline\bx_t, \sigma^2 _{s|0,t} \Id_{\dimx - \dimobs}) \eqsp,
\end{align*}
setting $Z = [\bar{Z}, \underline{Z}]$ and then concatenating both vectors. It is thus seen that the observed part of the state is updated with the observation whereas the bottom part is simply drawn from the prior. Moreover, if $\std^2 _{\smash{s|0, t}} \approx 0$ then $\gamma_{t|s} \a_{s|0} \approx 1$ and $\underline{Z}$ is almost the same as $\bx_0$.  In \Cref{algo:midpoint-gibbs}, once we have sampled $\vX_s \sim \epost{\smash{s|0,t}}{\vX_0, \vX_t}{}$, we first denoise it to obtain the new $\vX_0$ and then noise it to obtain the new $\vX_t$. As $s$ is sampled near 0, the denoising step will merely modify $\vX_s$ whereas the noising step will add significant noise to $\vX_s$ and may help with removing the artifacts. This noised sampled has however only a small impact on the next samples $\vX_0, \vX_s$ since $(1 - \gamma_{t|s}) \a^{-1} _{t|s} \approx 0$. In short, the first $\dimobs$ coordinates of the running state $\vX^* _0$ will be quickly replaced by the observation whereas the last $\dimx - \dimobs$ coordinates will be stuck at their initialization and will evolve only by a small amount throughout the iterations of the algorithm. We illustrate this situation on a concrete example in \Cref{fig:sampling-comparison} where we consider a half mask inpainting task. The first and second rows show the evolution of the running state $\vX^* _0$ with the time-sampling distributions  
\begin{align}
    \label{eq:sampling-dist-mix}
  & \mu^* _{i} = \begin{cases} \mbox{Uniform}\intset{\tau}{t_{i-1}} \, & \quad \text{if} \quad i > \lfloor  K / 4 \rfloor \\
                    t_{i-1} \, & \quad \text{else} 
\end{cases},\\
\label{eq:sampling-dist-zero}
& \mu^0 _i = \mbox{Uniform}\intset{1}{\lfloor t_i / 5 \rfloor} \eqsp, 
\end{align}
\emph{i.e.}, the time-sampling distribution we use in all our experiments, where $K$ is the number of diffusion steps, and the one that we use to sample only close to $0$, respectively. In \Cref{table:sampling-comparison} we compute the LPIPS for both distributions on a subset of the tasks we consider in the main paper. It is clear that $\mu^* _i$ outperforms $\mu^0 _i$, even when we increase the number of Gibbs steps (see phase retrieval task). 
% The unobserved part of the image contains significant errors during the first few steps due to the initialization and do not vanish afterwards whereas after a few iterations the unmasked part is recovered almost perfectly. 
% This is to be expected when $s$ is small. Indeed, consider the following empirical analysis with the simplified scenario where we assume that we obs. we may assume in this case that $\hpot{s}{\bx_s} = \pot{s}{\bx_s}$ and then, sampling perfectly from $\epost{s|0, t}{\bx_0, \bx_t}{}$ is equivalent to sampling 


%  and these are usually forgotten throughout the subsequent iterations when we sample $s$ uniformly on $\intset{1}{t_{i-1}}$. 
% We hypothesize that this is due to the combination of random timestep sampling, noising and denoising steps of the algorithm. Indeed, assume that in the first few iterations of the algorithm, the running state $\vX^* _0$ (see \Cref{algo:midpoint-gibbs}) contains significant artifacts. Then, once an index $s$ 

% However, when $s$ is sampled too close to $0$ they are not forgotten fast enough. 
\begin{table} 
    \centering 
    \caption{LPIPS on the \ffhq\ dataset for the two time-sampling distributions given in \eqref{eq:sampling-dist-mix} and \eqref{eq:sampling-dist-zero}. We use $R = 4$ Gibbs steps for the phase retrieval task.}
    \resizebox{0.60\textwidth}{!}{
    \begin{tabular}{l cccc}
        \toprule
        Distribution & Phase retrieval ($R = 4$) & JPEG2 & Gaussian deblurring & Motion deblurring \\
        \midrule
        $\mu^* _t$  & \textbf{0.10} & \textbf{0.14} & \textbf{0.12} & \textbf{0.09} \\
        $\mu^0 _t$ & 0.53 & 0.19 & 0.16 & 0.19\\
        \bottomrule
    \end{tabular} 
    }
    \label{table:sampling-comparison}
\end{table}
\begin{figure}
\centering 
\includegraphics[width=\textwidth]{figures/tau_sampling.jpg}
\caption{Evolution of the running state $\vX^* _0$ in \Cref{algo:midpoint-gibbs} for the two time-sampling distributions given in \eqref{eq:sampling-dist-mix} and \eqref{eq:sampling-dist-zero}. }
\label{fig:sampling-comparison}
\end{figure}

\subsection{Hyperparameters setup of \algo}
\label{apdx-sec:hyperparameters}
The details about the hyperparameters of \algo\ are reported in \Cref{table:hyperparams-algo}.
We adjust the optimization of the Gaussian Variational approximation in \Cref{algo:gauss_vi} during the first and last diffusion steps.
We ramp up the number of gradient steps during the final diffusion steps.
This allows us to substantially improve the fine grained details of the reconstructions. 
Similarly, we reduce the learning rate in the early step to alleviate potential instabilities.


\begin{table}[ht]
    \centering
    \caption{The hyperparameters used in \algo\ for the considered datasets. The index $i$ of the timesteps $\{t_i\}_{i=K}^0$ is taken in reverse order. The symbol \# stands for \emph{``number of''}.}
    \vspace{-0.2cm}
    \renewcommand{\arraystretch}{1.3} % Adjust row spacing if needed
    \resizebox{\textwidth}{!}{
    \begin{tabular}{l cccccc}
        \toprule
        & \# Gibbs repetitions $R$ & \# Diffusion steps $K$ & \# Denoising steps $M$ & Time-sampling distribution & Learning rate $\eta$ & \# Gradient steps $G$ \\
        \midrule
        \ffhq & $R=1$ & $K=100$ & $M=20$ & $\mu^* _{i}$ as in \eqref{eq:sampling-dist-mix} & $\eta=\begin{cases}
            0.01  & \text{ if } i \geq \lfloor 3K/4 \rfloor \\
            0.03  & \text{ otherwise} \\
            \end{cases}$ 
            & 
            $ G = \begin{cases}
            20 & \text{ if } i \leq \lfloor K/4 \rfloor \\
            5  & \text{ otherwise} \\
            \end{cases}$
        \\
        \midrule
        \ffhq\ LDM & $R=1$ & $K=100$ & $M=20$ & $\mu^* _{i}$ as in \eqref{eq:sampling-dist-mix}  & $\eta=\begin{cases}
            0.01  & \text{ if } i \geq \lfloor 3K/4 \rfloor \\
            0.03  & \text{ otherwise} \\
            \end{cases}$ 
            &
            $ G = \begin{cases}
            20 & \text{ if } i \leq \lfloor K/4 \rfloor \\
            20 & \text{ if } i \mod 10 = 0   \\
            3  & \text{ otherwise} \\
            \end{cases}$
        \\
        \midrule
        \imagenet & $R=1$ & $K=100$ & $M=20$ & $\mu^* _{i}$ as in \eqref{eq:sampling-dist-mix} & $\eta=\begin{cases}
            0.01  & \text{ if } i \geq \lfloor 3K/4 \rfloor \\
            0.03  & \text{ otherwise} \\
            \end{cases}$ 
            &
            $ G = \begin{cases}
            20 & \text{ if } i \leq \lfloor K/4 \rfloor \\
            5  & \text{ otherwise} \\
            \end{cases}$
        \\
        \midrule
        Audio-source separation & $R=6$ & $K=20$ & $M=15$ & $\mu^* _{i}$ as in \eqref{eq:sampling-dist-mix} & $\eta=0.005$ &  $ G = \begin{cases}
            20 & \text{ if } i \leq \lfloor K/4 \rfloor \\
            3  & \text{ otherwise} \\
            \end{cases}$
        \\
        \midrule
        %
        \parbox[m]{10em}{Audio-source separation\\ (Best result in \Cref{table:si-snri})}
        & $R=1$ & $K=20$ & $M=15$ & $\mu^* _{i}$ as in \eqref{eq:sampling-dist-mix} & $\eta=0.005$ &  $ G = 90$
        \\
        \bottomrule
    \end{tabular}
    \label{table:hyperparams-algo}
    }
\end{table}


\subsection{Audio source separation}
% \paragraph*{Audio diffusion model.}

% For audio data, we consider a setup where the input consists of multiple audio tracks of varying lengths, each containing $N$ distinct source waveforms $\{\mathbf{x}_1, \dots, \mathbf{x}_N\}$, which coherently sum to form a mixture $\mathbf{y} = \sum_{i=1}^N \mathbf{x}_i$. A diffusion model is trained to learn the prior for this setup.

In our experiment, the diffusion model employed provided by \cite{mariani2023multi} is trained on the \slakh\ training dataset\footnote{\url{http://www.slakh.com/}},  using only the four abundant instruments (bass, drums, guitar and piano) downsampled to 22 kHz. The denoiser network is based on a non-latent, time-domain unconditional variant of \citep{schneider2023musai}.

Its architecture follows a U-Net design, comprising an encoder, bottleneck, and decoder. The encoder consists of six layers with channel numbers $[256, 512, 1024, 1024, 1024, 1024]$, where each layer includes two convolutional ResNet blocks, and multihead attention is applied in the last three layers. The decoder mirrors the encoder structure in reverse. The bottleneck contains a ResNet block, followed by a self-attention mechanism, and then another ResNet block. Training is performed on the four stacked instruments using the publicly available trainer from repository\footnote{\url{https://github.com/archinetai/audio-diffusion-pytorch-trainer}}.

\subsection{Implementation of the competitors}
\label{apdx:competitors}
In this section, we provide implementation details of the competitors.
We adopt the hyperparameters recommended by the authors tune them on each dataset if they are not provided.
The complete set of hyperparameters and there values for both image experiments and audio-sound separation can be found in the supplementary material under the folders \texttt{configs\slash experiments/sampler} and \texttt{configs\slash exp\_sound/sampler}.


\paragraph*{DPS.}
We implemented \citet[Algorithm~1]{chung2023diffusion} and selected the hyperparameters of each considered task  based on \citet[App.~D]{chung2023diffusion}.
We tuned the algorithm for the other tasks, namely, we use $\gamma = 0.2$ for JPEG $2\%$, $\gamma = 0.07$ for High Dynamic Range tasks, and $\gamma = 1$ for audio-source separation.

\paragraph*{DiffPIR.}
We implemented \citet[Algorithm 1]{zhu2023denoising} to make it compatible with our existing code base.
We adopt the hyperparameters recommended in the official, released version\footnote{\url{https://github.com/yuanzhi-zhu/DiffPIR}}.
We followed the guidelines in \citep[Eqn. (13)]{zhu2023denoising} to extend the algorithm to nonlinear problems.
However, we noticed that the algorithm diverges in these cases and we could not follow up as the paper and the released code lack examples of nonlinear problems.
\citet{zhu2023denoising} provides an FFT-based solution for the motion blur tasks which is only valid in the case of circular convolution.
Hence, and since we adapted the experimental setup of \citet{chung2023diffusion}, we do not run the algorithm on motion blur task as it uses convolution with reflect padding. For audio-source separation, we found that $\lambda = \mu = 1$ works best.

\paragraph*{DDNM.}
We adapted the implementation provided in the released code\footnote{\url{https://github.com/wyhuai/DDNM}}.
Namely, the authors provide classes, in the module \texttt{functions\slash svd\_operators.py} that implement the logic of the algorithm on each degradation operator separately.
The adaptation includes factorizing these classes to a single class to support all SVD linear degradation operators.
On the other hand, we notice \ddnm\ is unstable for operators whose SVD decomposition is prone to numerical errors, such as Gaussian Blur with wide convolution kernel. This results from the algorithm using the pseudo-inverse of the operator.

\paragraph*{RedDiff.}
We used the implementation of \reddiff\ available in the released code\footnote{\url{https://github.com/NVlabs/RED-diff}}.
For linear problems, we use the pseudo-inverse of the observation as an initialization of the variational optimization problem. 
On nonlinear problems, for which the pseudo-inverse of the observation is not available, we initialized the optimization with a sample from the standard Gaussian distribution. 

\paragraph*{PGDM.}
We opted for the implementation available in the \reddiff\'s repository as some the authors are co-authors of \pgdm as well.
Notably, the implementation introduces a subtle deviation from \citet[Algorithm 1]{song2022pseudoinverse}: in the algorithm's final step, the guidance term $g$ is scaled by $\a_t$ ($\sqrt{\a_t}$ in their notation) whereas the implementation scales it by $\a_{t-1}\a_t$.
This adjustment improves the algorithm for most tasks except for JPEG dequantization. We found that the original scaling by $\a_t$ is better in this case.

\paragraph*{PSLD.}
We implemented the \psld\ algorithm provided in \citet[Algorithm 2]{rout2024solving} and referred to the publicly available implementation\footnote{\url{https://github.com/LituRout/PSLD}} to set the hyperparameters of the algorithm for the different tasks.

\paragraph*{ReSample.}
We modified the original code\footnote{\url{https://github.com/soominkwon/resample}} provided by the authors to make its hyperparameters directly adjustable, namely, the tolerance $\varepsilon$ and the maximum number of iterations $N$ for solving the optimization problems related to hard data consistency, and the scaling factor for the variance of the stochastic resampling distribution $\gamma$.
We found the algorithm to be sensitive to $\varepsilon$ and that setting it to the noise level of the inverse problem yields the best reconstructions across tasks and noise levels.
On the other hand, we noticed that $\gamma$ has less impact on the quality of the reconstructions.
Finally, we set a threshold $N=200$ on the maximum number of gradient iterations to make the algorithm less computationally intensive.

\paragraph*{DAPS.}
We have the official codebase \footnote{\url{https://github.com/zhangbingliang2019/DAPS}}.
We referred to \citet[Table.~7]{zhang2024daps} to set the hyperparameters.
% For audio-source separation, we modified the code to pass in direclty the diffusion model as it was built and trained following the Variance Exploding setup of \citet{karras2022elucidating}.
For audio-source separation, we set $\sigma_{\max}$ and $\sigma_{\min}$ to match those of the sound model and adapted the Langevin stepsize \texttt{lr} and the standard deviation \texttt{tau} to the audio-separation task.

\paragraph*{PNP-DM.}
We adapted the implementation provided in the released code\footnote{\url{https://github.com/zihuiwu/PnP-DM-public/}}.
Specifically, we exposed the coupling parameter $\rho$ including its initial value, minimum value, and decay rate, as well as the number of Langevin steps and its step size.
The hyperparameters were set based on \citet[Table 3 and Table 4]{wu2024pnpdm}.
For inpainting tasks, while it is theoretically possible to perform the likelihood steps using Gaussian conjugacy \citep[Sec.~3.1]{wu2024pnpdm}, we found that using Langevin produced better results in practice. For example, the reconstructions in the left figure of \Cref{fig:pnpdm-conjugacy} are obtained by sampling exactly from the posterior whereas on the \rhs\ we use Langevin dynamics. 
% For audio source separation, we pass in the model directly to the algorithm, following the approach that we used in \daps.
Although the audio separation task is linear and hence the likelihood steps can be implemented exactly, we encountered similar challenges as in inpainting and therefore we used Langevin here as well.


\subsection{Experiments reproducibility}
Our code will be made available upon acceptange of the paper. In the anonymous codebase provided as companion of the paper we use $\sqrt{\a_t}$ instead of $\a_t$ to match the conventions of existing codebases.  All experiments were conducted on Nvidia Tesla V100 SXM2 GPUs. 
For the image experiments, we used $300$ images from the validation sets of \ffhq\ and \imagenet\ $256 \times 256$ that we numbered from $0$ to $299$.
The image number was used to seed the randomness of the experiments on that image.
For the audio source separation experiments, the \slakh\ test dataset has tracks named following the pattern \texttt{Track0XXXX}, where \texttt{X} represents a digit in $0-9$.
The number \texttt{XXXX} was used as the seed for the experiments conducted on each track.


% \subsection{Runtime and memory requirement comparison}
% We evaluate the runtime and GPU memory consumption for image experiments on the three considered diffusion model priors.
% Since not all algorithms support every task, we restrict the evaluation to commune tasks.
% \Cref{fig:runtime-gpu} presents the average runtime and GPU memory requirement over both samples and tasks.

% \begin{figure}[htb]
%     \centering
%     \subfigure{
%         \includegraphics[height=0.2\textwidth]{figures/runtime-gpu/ffhq-ldm.pdf}   
%     }
%     \subfigure{
%         \includegraphics[height=.2\textwidth]{figures/runtime-gpu/ffhq.pdf}
%         }
%     \subfigure{
%             \includegraphics[height=.2\textwidth]{figures/runtime-gpu/imagenet.pdf}
%         }
%     \caption{Comparison of the runtime (red bars -- left axis) and memory requirement (blue bars -- right axis) between the considered algorithms on \ffhq\ latent space (1\textsuperscript{st} row), \ffhq\ pixel space (2\textsuperscript{nd} row), and \imagenet\ (3\textsuperscript{rd} row).}
%     \label{fig:runtime-gpu}
% \end{figure}


\subsection{Extended results}
\label{apdx:extended-results}
We present the complete table with LPIPS, PSNR, and SSIM metrics for the image inverse problems experiment in \Cref{table:extended-ffhq-imagenet} for the \ffhq\ and \imagenet\ datasets, and in \Cref{table:extended-ffhq-ldm} for \ffhq\ LDM. Similarly, the complete results for the audio source separation experiments that include all competitors are provided in \Cref{table:extended-si-snri}.

From \Cref{table:extended-ffhq-imagenet}, one can note that \ddnm, \diffpir\ and \daps\ score better in PSNR and SSIM compared to \algo\; but score lower in LPIPS.
For most of the tasks we considered, one does not expect to recover an image very close to the reference and thus, metrics that perform pixel-wise comparisons are less relevant and favor images that are overly smooth.
We provide evidence for this in the gallery of images below where we compare qualitatively the outputs of our algorithm with those of the competitors.
It can be seen that our method provides reconstructions with ine-grained details that more coherent with the reference image.
Note for example that \ddnm, \diffpir\ and \daps\ outperform \algo\ in terms of PSNR and SSIM on the half mask task on \imagenet\ while failing to reconstruct the missing \rhs\ of the images. 




\begin{table}[h]
    \centering
    \caption{Mean LPIPS/PSNR/SSIM metrics for the considered linear and nonlinear imaging tasks on the \ffhq\ and \imagenet\ $256 \times 256$ datasets with $\stdobs = 0.05$.}
    \resizebox{\textwidth}{!}{
    \begin{tabular}{l cccccccc | cccccccc}
        \toprule
        \vspace{1mm}
        & \multicolumn{8}{c}{\bf{\ffhq}} & \multicolumn{8}{c}{\bf{\imagenet}} \\
        % \cmidrule(lr){2-7} \cmidrule(lr){8-13}
        \textbf{Task} & \algo\ & \dps & \pgdm & \ddnm & \diffpir & \reddiff & \daps & \pnpdm \ &\ \algo\ & \dps & \pgdm & \ddnm & \diffpir & \reddiff & \daps & \pnpdm \\
        % --- lpips
        \midrule
        & \multicolumn{16}{c}{LPIPS \ $\downarrow$} \\
        \midrule
        SR ($\times 4$)        & \first{0.09} & \first{0.09} & 0.30 & \third{0.15} & \second{0.10} & 0.39 & 0.16 & \second{0.10} \ &\ \second{0.26} & \first{0.25} & 0.56 & 0.34 & \third{0.31} & 0.57 & 0.37 & 0.66 \\
        SR ($\times 16$)       & \second{0.24} & \first{0.23} & 0.42 & 0.33 & \first{0.23} & 0.55 & 0.40 & \third{0.29} \ &\ \third{0.55} & \first{0.44} & 0.62 & 0.71 & \second{0.50} & 0.85 & 0.75 & 1.03 \\
        Box inpainting         & \first{0.10} & 0.17 & 0.17 & \second{0.12} & 0.14 & 0.19 & \third{0.13} & 0.18 \ &\ \first{0.23} & 0.35 & \third{0.29} & \second{0.28} & 0.30 & 0.36 & 0.30 & 0.42 \\
        Half mask              & \first{0.20} & \third{0.24} & \third{0.24} & \second{0.23} & 0.25 & 0.28 & \second{0.23} & 0.32 \ &\ \first{0.31} & 0.40 & \second{0.34} & \third{0.38} & 0.40 & 0.46 & 0.40 & 0.54 \\
        Gaussian Deblur        & \first{0.12} & \third{0.17} & 0.87 & 0.20 & \first{0.12} & 0.24 & 0.24 & \second{0.14} \ &\ \first{0.30} & \second{0.37} & 1.00 & \third{0.45} & \first{0.30} & 0.53 & 0.59 & 0.76 \\
        % \vspace{2mm} % hack to leave space between linear and nonlinear task
        Motion Deblur          & \first{0.09} & \second{0.17} & $-$ & $-$ & $-$ & 0.22 & \third{0.19} & 0.21 \ &\ \first{0.22} & \third{0.40} & $-$ & $-$ & $-$ & \second{0.39} & 0.42 & 0.52 \\
        % \cmidrule(lr){1-13}
        JPEG (QF = 2)          & \first{0.14} & 0.34 & 1.12 & $-$ & $-$ & 0.32 & \second{0.22} & \third{0.29} \ &\ \first{0.38} & 0.60 & 1.32 & $-$ & $-$ & \third{0.49} & \second{0.45} & 0.56 \\
        Phase retrieval        & \first{0.11} & 0.40  & $-$ & $-$ & $-$ & \third{0.26} & \second{0.14} & 0.34 \ &\ \second{0.55} & 0.62 & $-$ & $-$ & $-$ & \third{0.61} & \second{0.50} & 0.66 \\
        Nonlinear deblur       & \first{0.27} & 0.51 & $-$ & $-$ & $-$ & 0.68 & \second{0.28} & \third{0.31} \ &\ \first{0.41} & 0.82 & $-$ & $-$ & $-$ & \third{0.66} & \first{0.41} & \second{0.49} \\
        HDR    & \second{0.12} & 0.40 & $-$ & $-$ & $-$ & 0.20 & \second{0.10} & \third{0.19} \ &\ \third{0.21} & 0.84 & $-$ & $-$ & $-$ & \second{0.19} & \first{0.14} & 0.31 \\
        % --- PSNR
        \midrule
        & \multicolumn{16}{c}{PSNR \ $\uparrow$} \\
        \midrule
        SR ($\times 4$)        & 27.66 & \third{28.05} & 24.57 & \first{29.45} & 27.72 & 26.75 & \second{28.44} & 27.44 \ &\ 23.88 & \third{24.37} & 18.45 & \first{24.99} & 23.43 & 23.33 & \second{24.38} & 16.4 \\
        SR ($\times 16$)       & \third{21.01} & 20.71 & 18.51 & \first{22.32} & 20.96 & \second{21.46} & 19.75 & 20.88\ &\ 18.12 & 17.66 & 15.27 & \first{19.93} & \third{18.4} & \second{19.06} & 18.18 & 14.0 \\
        Box inpainting         & \second{22.38} & 18.81 & 21.05 & \third{22.34} & \first{22.39} & 21.46 & 22.06 & 20.42 \ &\ 16.82 & 13.92 & 16.73 & \first{19.18} & \third{19.05} & 18.21 & \second{19.11} & 18.03 \\
        Half mask              & 15.39 & 14.86 & 15.29 & \first{16.38} & \third{16.04} & 15.68 & \second{16.25} & 14.35 \ &\ 13.77 & 12.15 & 14.04 & \second{15.97} & \third{15.64} & 14.84 & \first{16.00} & 14.88 \\
        Gaussian Deblur        & 25.64 & 24.03 & 13.34 & \second{26.62} & 25.78 & \first{26.68} & \third{26.12} & 25.89 \ &\ 21.57 & 20.65 & 9.92 & \first{22.89} & 21.8 & \second{22.72} & \third{22.41} & 15.85 \\
        Motion Deblur          & \first{27.82} & 24.13 & $-$ & $-$ & $-$ & \second{27.48} & \third{27.07} & 24.91 \ &\ \first{24.46} & 21.38 & $-$ & $-$ & $-$ & \second{24.06} & \third{23.64} & 22.47 \\
        JPEG (QF = 2)          & \second{25.57} & 19.56 & 12.57 & $-$ & $-$ & \third{24.53} & \first{25.72} & 22.42 \ &\ \third{21.42} & 16.33 & 5.27 & $-$ & $-$ & \second{22.07} & \first{22.68} & 20.74 \\
        Phase retrieval        & \second{27.55} & 16.56 & $-$ & $-$ & $-$ & \third{24.58} & \first{27.84} & 21.63 \ &\ \second{16.01} & 14.12 & $-$ & $-$ & $-$ & \third{15.41} & \first{18.44} & 15.02  \\
        Nonlinear deblur       & \third{23.55} & 16.08 & $-$ & $-$ & $-$ & 21.94 & \first{24.56} & \second{24.08} \ &\ \third{21.96} & 10.13 & $-$ & $-$ & $-$ & 20.57 & \first{22.68} & \second{22.20} \\
        HDR                    & \second{24.79} & 18.71 & $-$ & $-$ & $-$ & \third{21.69} & \first{26.60} & 21.59 \ &\ \second{22.90} & 9.56 & $-$ & $-$ & $-$ & 22.12 & \first{24.69} & \third{22.23} \\
        % 
        % --- SSIM
        \midrule
        & \multicolumn{16}{c}{SSIM \ $\uparrow$} \\
        \midrule
        SR ($\times 4$)        & 0.80 & \second{0.81} & 0.56 & \first{0.85} & 0.78 & 0.68 & \second{0.81} & 0.77\ &\ 0.65 & \second{0.68} & 0.30 & \first{0.71} & 0.60 & 0.57 & \third{0.66} & 0.25 \\ 
        SR ($\times 16$)       & \second{0.61} & 0.58 & 0.42 & \first{0.67} & 0.59 & \third{0.60} & 0.58 & 0.57\ &\ 0.31 & 0.39 & 0.21 & \first{0.49} & 0.41 & \second{0.44} & \second{0.44} & 0.10 \\
        Box inpainting         & \third{0.80} & 0.77 & 0.70 & \first{0.83} & \second{0.82} & 0.70 & \third{0.80} & 0.75\ &\ 0.71 & 0.70 & 0.62 & \first{0.77} & \second{0.76} & 0.67 & \third{0.74} & 0.64 \\
        Half mask              & 0.67 & 0.67 & 0.59 & \first{0.74} & \second{0.72} & 0.63 & \third{0.71} & 0.65\ &\ 0.59 & 0.58 & 0.52 & \first{0.68} & \second{0.67} & 0.59 & \third{0.66} & 0.57 \\
        Gaussian Deblur        & 0.73 & 0.68 & 0.14 & \first{0.77} & 0.72 & \second{0.76} & \third{0.75} & 0.72\ &\ 0.50 & 0.50 & 0.08 & \first{0.59} & 0.51 & \second{0.57} & \third{0.56} & 0.20 \\
        Motion Deblur          & \first{0.80} & 0.70 & $-$ & $-$ & $-$ & 0.71 & \second{0.78} & \third{0.75}\ &\ \first{0.67} & 0.55 & $-$ & $-$ & $-$ & \third{0.61} & \second{0.63} & 0.57 \\
        JPEG (QF = 2)          & \second{0.74} & 0.56 & 0.10 & $-$ & $-$ & \third{0.71} & \first{0.76} & 0.70\ &\ 0.51 & 0.40 & 0.02 & $-$ & $-$ & \second{0.59} & \first{0.62} & \third{0.58}  \\
        Phase retrieval        & \second{0.78} & 0.49 & $-$ & $-$ & $-$ & \third{0.61} & \first{0.81} & 0.57\ &\ \second{0.31} & \third{0.27} & $-$ & $-$ & $-$ & 0.25 & \first{0.46} & 0.23 \\
        Nonlinear deblur       & \third{0.67} & 0.44 & $-$ & $-$ & $-$ & 0.42 & \first{0.71} & \second{0.70}\ &\ \second{0.58} & 0.25 & $-$ & $-$ & $-$ & 0.41 & \first{0.61} & \second{0.58} \\
        HDR                    & \second{0.76} & 0.55 & $-$ & $-$ & $-$ & \third{0.72} & \first{0.85} & 0.69\ &\ \second{0.72} & 0.23 & $-$ & $-$ & $-$ & \second{0.72} & \first{0.82} & 0.66 \\
        \bottomrule
    \end{tabular}
    }
    \label{table:extended-ffhq-imagenet}
\end{table}

\begin{table}[h]
    \centering
    \caption{Mean LPIPS/PSNR/SSIM for linear/nonlinear imaging tasks on \ffhq\ $256 \times 256$ datasets with LDM and $\stdobs = 0.05$.}
    \resizebox{\textwidth}{!}{
    \begin{tabular}{l ccccc c ccccc c ccccc}
        \toprule
        & \algo\ & \resample & \psld & \daps & \pnpdm  && \algo\ & \resample & \psld & \daps & \pnpdm && \algo\ & \resample & \psld & \daps & \pnpdm \\
        \cmidrule(lr){2-6} \cmidrule(lr){8-12} \cmidrule(lr){14-18}
        Task & \multicolumn{5}{c}{LPIPS \ $\downarrow$} && \multicolumn{5}{c}{PSNR \ $\uparrow$} && \multicolumn{5}{c}{SSIM \ $\uparrow$} \\
        \midrule
        SR ($\times 4$)    & \first{0.14} & \third{0.22} & \second{0.21} & 0.28 & 0.40   && \second{27.39} & \third{25.85} & 25.80 & \first{27.45} & 23.81  && \first{0.79} & 0.68 & \second{0.71} & \first{0.79} & 0.70  \\
        SR ($\times 16$)   & \first{0.30} & \third{0.38} & \second{0.36} & 0.52 & 0.71   && \third{20.60} & \second{20.97} & \first{21.42} & 19.91 & 17.07  && \third{0.58} & 0.56 & \first{0.63} & \second{0.59} & 0.52  \\
        Box inpainting     & \first{0.18} & \second{0.22} & \third{0.27} & 0.37 & 0.31   && \first{21.81} & 18.56 & \second{20.01} & 11.77 & \third{19.57}  && \first{0.78} & \second{0.75} & 0.66 & 0.70 & \third{0.73}  \\
        Half mask          & \first{0.26} & \second{0.30} & \third{0.32} & 0.49 & 0.44   && \first{15.71} & \second{14.89} & \third{14.62} &  9.13 & 14.15  && \first{0.69} & \second{0.67} & 0.60 & 0.55 & \third{0.65}  \\
        Gaussian Deblur    & \second{0.18} & \first{0.16} & 0.59 & \third{0.32} & \third{0.32}   && \third{26.79} & \first{27.28} & 17.99 & \second{26.86} & 26.11  && \second{0.77} & \third{0.75} & 0.27 & \first{0.78} & \second{0.77}  \\
        Motion Deblur      & \second{0.22} & \first{0.20} & 0.70 & \third{0.36} & \third{0.36}   && \third{25.27} & \first{26.73} & 17.71 & \second{25.37} & 24.65  && \second{0.73} & \third{0.72} & 0.24 & \first{0.74} & \third{0.72}  \\
        JPEG (QF = 2)      & \first{0.23} & \second{0.26} & $-$ & \third{0.32} & 0.36   && \third{24.27} & \second{24.77} & $-$ & \first{25.22} & 23.86  && \third{0.71} & 0.66 & $-$ & \first{0.75} & \second{0.72}  \\
        Phase retrieval    & \second{0.29} & \third{0.39} & $-$ & \first{0.25} & 0.50   && \second{22.54} & \third{20.18} & $-$ & \first{27.05} & 20.03  && \second{0.62} & 0.49 & $-$ & \first{0.79} & \third{0.60}  \\
        Nonlinear deblur   & \first{0.29} & \second{0.33} & $-$ & \third{0.37} & \third{0.37}   && \second{23.71} & \first{24.10} & $-$ & 22.03 & \third{23.28}  && \second{0.69} & 0.67 & $-$ & \third{0.68} & \first{0.70}  \\
        High dynamic range & \second{0.16} & \first{0.12} & $-$ & \third{0.24} & \third{0.24}   && \second{25.59} & \first{25.91} & $-$ & \third{20.95} & 20.21  && \second{0.80} & \first{0.83} & $-$ & \third{0.74} & 0.73  \\
        \bottomrule
    \end{tabular}
    }
    \label{table:extended-ffhq-ldm}
\end{table}

\begin{table}[h]
    \centering
    \captionsetup{font=small}
    \caption{Mean \sisdri\ on \slakh\ test dataset. The last row  "All" displays the mean over the four stems. Higher metrics are better.}
    \resizebox{0.64\textwidth}{!}{
    \begin{tabular}{l ccccccccc | cc}
        \toprule
        Stems  & \algo\ & \dps & \pgdm & \ddnm & \diffpir & \reddiff & \daps & \pnpdm    & \msdm & \isdm   & \demucs \\
        \midrule
        Bass   & \first{18.49} & \third{16.50} & 16.41 & 14.94 & -2.34 & -0.40 & 11.76 & 2.90 & \second{17.12} & 19.36   & 17.16   \\
        Drums  & 18.07 & \third{18.29} & 18.14 & \first{19.05} & 9.47 & -0.98 & 15.62 & 7.89 & \second{18.68} & 20.90   & 19.61   \\
        Guitar &\first{16.68} & 9.90 & 12.84 & \third{14.38} & -1.01 & 5.68 & 11.75 & 4.51 & \second{15.38}  & 14.70   & 17.82   \\
        Piano  & \first{16.17} & 10.41 & \third{12.31} & 11.46 & 0.97 & 5.04 & 9.52 & 4.09 & \second{14.73}  & 14.13   & 16.32   \\
        \midrule
        All    & \first{17.35} & 13.77 & 14.92 & \third{14.96} & 1.77 & 2.33 & 12.16 & 4.85 & \second{16.48}   & 17.27   & 17.73   \\
        \bottomrule
    \end{tabular}
    }
    \label{table:extended-si-snri}
\end{table}


\begin{figure*}[!ht]
% \vskip 0.1in
% \vskip 0.2in
\begin{center}
\centerline{\includegraphics[width=\textwidth]{figures/_reconstructions.png}}
\caption{The figure shows 2D slices of CT images (first column) alongside reconstructions and anomaly maps generated by two methods: an Autoencoder~\cite{autoencoder} (second and third columns) and f-AnoGAN~\cite{fanogan} (last two columns). Autoencoder overfits for pixel reconstruction, so it generates pathologies and fails to segment them. Also Autoencoder produces blurry generations, leading to inaccurate reconstructions of fine details and high anomaly scores on these details (e.g., vessels in the lungs). f-AnoGAN, on the other hand, avoids generating pathologies, but the generation quality still is insufficient for precise segmentation of only pathological voxels. GANs are known to be unstable and sensitive to hyperparameters, necessitating careful tuning and experimentation to achieve optimal results.}
\label{fig:reconstructions}
\end{center}
\vskip -0.2in
\end{figure*}
\label{apdx-sec:visual-reconstructions}



\end{document}


% This document was modified from the file originally made available by
% Pat Langley and Andrea Danyluk for ICML-2K. This version was created
% by Iain Murray in 2018, and modified by Alexandre Bouchard in
% 2019 and 2021 and by Csaba Szepesvari, Gang Niu and Sivan Sabato in 2022.
% Modified again in 2023 and 2024 by Sivan Sabato and Jonathan Scarlett.
% Previous contributors include Dan Roy, Lise Getoor and Tobias
% Scheffer, which was slightly modified from the 2010 version by
% Thorsten Joachims & Johannes Fuernkranz, slightly modified from the
% 2009 version by Kiri Wagstaff and Sam Roweis's 2008 version, which is
% slightly modified from Prasad Tadepalli's 2007 version which is a
% lightly changed version of the previous year's version by Andrew
% Moore, which was in turn edited from those of Kristian Kersting and
% Codrina Lauth. Alex Smola contributed to the algorithmic style files.
