%%%%%%%% ICML 2025 EXAMPLE LATEX SUBMISSION FILE %%%%%%%%%%%%%%%%%
\documentclass{article}

% Recommended, but optional, packages for figures and better typesetting:
\usepackage{microtype}
\usepackage{graphicx}
\usepackage{subfigure}
% \usepackage[ruled,vlined]{algorithm2e}
\usepackage{booktabs} % for professional tables

% hyperref makes hyperlinks in the resulting PDF.
% If your build breaks (sometimes temporarily if a hyperlink spans a page)
% please comment out the following usepackage line and replace
% \usepackage{icml2025} with \usepackage[nohyperref]{icml2025} above.
\usepackage{hyperref}


% Attempt to make hyperref and algorithmic work together better:
\newcommand{\theHalgorithm}{\arabic{algorithm}}

% Use the following line for the initial blind version submitted for review:
\usepackage[accepted]{icml2025}

% If accepted, instead use the following line for the camera-ready submission:
% \usepackage[accepted]{icml2025}

% Todonotes is useful during development; simply uncomment the next line
%    and comment out the line below the next line to turn off comments
%\usepackage[disable,textsize=tiny]{todonotes}
\usepackage[textsize=tiny]{todonotes}
\definecolor{babypink}{rgb}{0.96, 0.76, 0.76} 
\definecolor{burntsienna}{rgb}{0.91, 0.45, 0.32}     % colors
\definecolor{crimson}{rgb}{0.86, 0.08, 0.24}
\definecolor{darkspringgreen}{rgb}{0.09, 0.45, 0.27}
\definecolor{deepcarrotorange}{rgb}{0.91, 0.41, 0.17}
%%%%%%%%%%%%%
% additional pacakge
\usepackage[utf8]{inputenc} % allow utf-8 input
\usepackage[T1]{fontenc}    % use 8-bit T1 fonts
\usepackage{hyperref}       % hyperlinks
\usepackage{nccmath}
\hypersetup{
     colorlinks=true,
     linkcolor=blue,
     filecolor=blue,
     citecolor=blue,
     urlcolor=blue,
    }
\usepackage{url}
\usepackage{wrapfig}
\usepackage{colortbl}
\usepackage{booktabs}       % professional-quality tables
\usepackage{amsfonts}       % blackboard math symbols
\usepackage{nicefrac}       % compact symbols for 1/2, etc.
\usepackage{microtype}      % microtypography
\usepackage{xcolor}
\usepackage{multicol}
\usepackage{graphicx}
\usepackage{transparent}
\usepackage{amsthm}
\usepackage{bbm}
\usepackage{comment}
\usepackage{enumitem}
\usepackage{bm}
\usepackage{subfigure}
\usepackage{mathtools}
\usepackage[export]{adjustbox}
\usepackage{stmaryrd}
% \usepackage[utf8]{inputenc}

%% full size page
\usepackage{fullpage}
\usepackage[margin = 2.5cm]{geometry}

%% AMS packages
\usepackage{amsmath, amsthm, amssymb}
\usepackage{thmtools}
\usepackage{thm-restate}
\usepackage{mathtools}  
\usepackage{xfrac} 
%\usepackage[normalem]{ulem}

\usepackage{placeins} 
\usepackage{times}

%%bibliography
% \usepackage{amsthm}
\usepackage{url}
\usepackage{array}


%% algorithm
%\usepackage{algorithm}
%\usepackage{algpseudocode}
\usepackage[ruled,vlined]{algorithm2e}
% Create a new environment named "algorithm2e" that behaves like "algorithm"
\newenvironment{algorithm2e}[1][]{%
    \begin{algorithm}[#1]%
}{%
    \end{algorithm}
}

%% graphics and colors
\usepackage{graphicx}
\usepackage{color}
%\usepackage[dvipsnames]{xcolor}

%%bibliography
\usepackage[round]{natbib}
\usepackage[hyperindex,breaklinks]{hyperref}
\usepackage{url}

%%color box figure
\usepackage{tcolorbox}

%%draw figure
\usepackage{tikz}

%%misc
\usepackage{nicefrac}

%\citestyle{acmauthoryear}

% \declaretheorem[name=Theorem]{theorem}

%% define theorems, lemmas, claims
\newtheorem{theorem}{Theorem}[section]
\newtheorem{claim}[theorem]{Claim}
\newtheorem{corollary}[theorem]{Corollary}
\newtheorem{proposition}[theorem]{Proposition}
\newtheorem{lemma}[theorem]{Lemma}
\newtheorem{definition}[theorem]{Definition}
\newtheorem{observation}[theorem]{Observation}
\newtheorem{question}[theorem]{Question}
\newtheorem{assumption}[theorem]{Assumption}
\newtheorem*{remark*}{Remark}


% \newenvironment{numberedtheorem}[1]{%
% \renewcommand{\thetheorem}{#1}%
% \begin{theorem}}{\end{theorem}\addtocounter{theorem}{-1}}

% \newenvironment{numberedlemma}[1]{%
% \renewcommand{\thetheorem}{#1}%
% \begin{lemma}}{\end{lemma}\addtocounter{theorem}{-1}}

% \newenvironment{oneshot}[1]{\@begintheorem{#1}{\unskip}}{\@endtheorem}

% \makeatletter
% \newtheorem*{rep@theorem}{\rep@title}
% \newcommand{\newreptheorem}[2]{%
% \newenvironment{rep#1}[1]{%
%  \def\rep@title{#2 \ref{##1}}%
%  \begin{rep@theorem}}%
%  {\end{rep@theorem}}}
% \makeatother


%cleveref package loaded at the end
\usepackage{cleveref}
\crefname{theorem}{theorem}{theorems}
\Crefname{theorem}{Theorem}{Theorems}

\crefname{lemma}{lemma}{lemmas}
\Crefname{lemma}{Lemma}{Lemmas}

\crefname{claim}{claim}{claims}
\Crefname{claim}{Claim}{Claims}

\crefname{corollary}{corollary}{corollaries}
\Crefname{corollary}{Corollary}{Corollaries}

\crefname{proposition}{proposition}{propositions}
\Crefname{proposition}{Proposition}{Propositions}

\crefname{definition}{definition}{definitions}  % Explicitly set "Definition"
\Crefname{definition}{Definition}{Definitions}

\crefname{observation}{observation}{observations}
\Crefname{observation}{Observation}{Observations}

\crefname{question}{question}{questions}
\Crefname{question}{Question}{Questions}

\crefname{assumption}{assumption}{assumptions}
\Crefname{assumption}{Assumption}{Assumptions}

\crefname{algorithm}{algorithm}{algorithms}
\Crefname{algorithm}{Algorithm}{Algorithms}

\crefname{AlgoLine}{line}{lines}  % If using `algorithm2e` with line numbers
\Crefname{AlgoLine}{Line}{Lines}


%% probability notation
\DeclareMathOperator{\cov}{cov}
\DeclareMathOperator{\sgn}{\mathbf{sgn}}
\DeclareMathOperator{\E}{\mathbf{E}}
\DeclareMathOperator{\Var}{\mathbf{Var}}
\DeclareMathOperator{\one}{\mathbf{1}}
\newcommand{\given}{\mid}
\DeclareMathOperator{\Ball}{Ball}
\DeclareMathOperator{\tr}{tr}

% rounding up and down
\newcommand {\roundup}   [1] {{\lceil {#1} \rceil}}
\newcommand {\rounddown} [1] {{\lfloor {#1} \rfloor}}

%% black board letters
\newcommand{\bbB}{\mathbb{B}}
\newcommand{\bbC}{\mathbb{C}}
\newcommand{\bbR}{\mathbb{R}}
\newcommand{\bbZ}{\mathbb{Z}}

%% calligraphic letters 
\newcommand{\calA}{\mathcal{A}}
\newcommand{\calB}{\mathcal{B}}
\newcommand{\calC}{\mathcal{C}}
\newcommand{\calD}{\mathcal{D}}
\newcommand{\calE}{\mathcal{E}}
\newcommand{\calF}{\mathcal{F}}
\newcommand{\calG}{\mathcal{G}}
\newcommand{\calH}{\mathcal{H}}
\newcommand{\calI}{\mathcal{I}}
\newcommand{\calJ}{\mathcal{J}}
\newcommand{\calK}{\mathcal{K}}
\newcommand{\calL}{\mathcal{L}}
\newcommand{\calM}{\mathcal{M}}
\newcommand{\calN}{\mathcal{N}}
\newcommand{\calO}{\mathcal{O}}
\newcommand{\calP}{\mathcal{P}}
\newcommand{\calQ}{\mathcal{Q}}
\newcommand{\calR}{\mathcal{R}}
\newcommand{\calS}{\mathcal{S}}
\newcommand{\calT}{\mathcal{T}}
\newcommand{\calU}{\mathcal{U}}
\newcommand{\calV}{\mathcal{V}}
\newcommand{\calW}{\mathcal{W}}
\newcommand{\calX}{\mathcal{X}}
\newcommand{\calY}{\mathcal{Y}}
\newcommand{\calZ}{\mathcal{Z}}

\newcommand{\N}{\mathbb{N}}
\newcommand{\R}{\mathbb{R}}
\DeclareMathOperator*{\argmin}{arg\,min}
\DeclareMathOperator*{\argmax}{arg\,max}




\newcommand{\err}{\mathrm{err}}
\newcommand{\errstar}{\mathrm{err}^*}

\DeclareMathOperator{\REC}{REC}
\DeclareMathOperator{\NRD}{NRD}
\DeclareMathOperator{\FN}{FN}
\DeclareMathOperator{\FP}{FP}

\DeclareMathOperator{\Tr}{Tr}

\newcommand{\conv}{\mathrm{conv}}
\newcommand{\cone}{\mathrm{cone}}
\newcommand{\inner}[2]{\langle #1, #2\rangle}
\usepackage{xifthen}
\usepackage{xargs}
\usepackage{amsmath,amssymb}
\usepackage{algorithm,algorithmic}
\usepackage{hyperref}
\usepackage{caption}
\usepackage{titlesec}
\usepackage{multirow}
% \usepackage[hypertexnames=false]{hyperref}
% if you use cleveref..
\usepackage[capitalize,noabbrev]{cleveref}

%%%%%%%%%%%%%%%%%%%
% custom notations & commands
\newcommand{\stap}{\bS_{\rm TAP}}
\newcommand{\slamp}{\bS_{\rm LAMP}}
\newcommand{\gout}{\bg_{\rm out}}

\newcommand{\Py}{\mathsf{Z}}
\newcommand{\I}{\mathbb{I}}
\newcommand{\Zout}{\Py}
\newcommand{\dgout}{\bG}

\newcommand{\bSigma}{\boldsymbol{\Sigma}}

% Probability
\renewcommand{\P}{\mathbb{P}}
\newcommand{\E}{\mathbb{E}}
\newcommand{\Var}{\text{Var}}
\newcommand{\Cov}{\mathrm{Cov}}
\newcommand{\cN}{\mathcal{N}}

% Sets
\newcommand{\Z}{\mathbb{Z}}
\newcommand{\R}{\mathbb{R}}
\newcommand{\C}{\mathbb{C}}
\newcommand{\N}{\mathbb{N}}
\renewcommand{\S}{\mathbb{S}}
\def\ball{{\mathsf B}}

% Variables
\newcommand{\eps}{\varepsilon} 
\newcommand{\vphi}{\varphi}
\def\id{{\mathbf I}}


% Math
\renewcommand{\d}{\textup{d}}
\renewcommand{\l}{\vert}
\newcommand{\dl}{\Vert}
\newcommand{\<}{\langle}
\renewcommand{\>}{\rangle}
\newcommand{\sign}{\text{sign}}
\newcommand{\diag}{\text{diag}}
%\newcommand{\tr}{\text{tr}}
%\newcommand{\op}{{\rm op}}
\newcommand{\ones}{\bm{1}}
\newcommand{\what}{\widehat}
%\newcommand{\grad}{\boldsymbol{\nabla}}
\def\sT{{\mathsf T}}
\def\bzero{{\boldsymbol 0}}
\newcommand{\bomega}{\boldsymbol{\omega}}
\newcommand{\bOmega}{\boldsymbol{\Omega}}
\newcommand{\flatten}{\operatorname{flat}}
\newcommand{\bcT}{\boldsymbol{\mathcal{T}}}


\DeclareMathOperator*{\argmin}{arg\,min}
\DeclareMathOperator*{\argmax}{arg\,max}
\DeclareMathOperator*{\argsup}{arg\,sup}
\DeclareMathOperator*{\arginf}{arg\,inf}
\newcommand{\eqnd}{\, {\buildrel d \over =} \,} 
\newcommand{\eqndef}{\mathrel{\mathop:}=}
\def\doteq{{\stackrel{\cdot}{=}}}
\newcommand{\goto}{\longrightarrow}
\newcommand{\gotod}{\buildrel d \over \longrightarrow} 
\newcommand{\gotoas}{\buildrel a.s. \over \longrightarrow} 
\def\simiid{{\stackrel{i.i.d.}{\sim}}}


% Notations 
\newcommand{\notate}[1]{\textcolor{red}{\textbf{[#1]}}}
\newcommand{\cc}[1]{\textcolor{blue}{\textbf{[CC:#1]}}}
\newcommand{\yw}[1]{\textcolor{pink}{\textbf{[YW:#1]}}}
\newcommand{\mc}[1]{\mathcal{#1}}
\newcommand{\mb}[1]{\mathbf{#1}}


% Theorem
\newtheorem{question}{Question}
\newtheorem{property}{Property}
\newtheorem{objective}{Objective}
\newtheorem{claim}{Claim}
\newtheorem{example}{Example}



%\usepackage[inline]{showlabels}

\DeclareSymbolFont{rsfs}{U}{rsfs}{m}{n}
\DeclareSymbolFontAlphabet{\mathscrsfs}{rsfs}



% Bold symbols
\def\bA{{\boldsymbol A}}
\def\bB{{\boldsymbol B}}
\def\bC{{\boldsymbol C}}
\def\bD{{\boldsymbol D}}
\def\bE{{\boldsymbol E}}
\def\bF{{\boldsymbol F}}
\def\bG{{\boldsymbol G}}

\def\bH{{\boldsymbol H}}
\def\bI{{\boldsymbol I}}
\def\bJ{{\boldsymbol J}}
\def\bK{{\boldsymbol K}}
\def\bL{{\boldsymbol L}}
\def\bM{{\boldsymbol M}}
\def\bN{{\boldsymbol N}}
\def\bO{{\boldsymbol O}}
\def\bP{{\boldsymbol P}}
\def\bQ{{\boldsymbol Q}}
\def\bR{{\boldsymbol R}}
\def\bS{{\boldsymbol S}}
\def\bT{{\boldsymbol T}}
\def\bU{{\boldsymbol U}}
\def\bV{{\boldsymbol V}}
\def\bW{{\boldsymbol W}}
\def\bX{{\boldsymbol X}}
\def\bY{{\boldsymbol Y}}
\def\bZ{{\boldsymbol Z}}

\def\ba{{\boldsymbol a}}
\def\bb{{\boldsymbol b}}
\def\be{{\boldsymbol e}}
\def\boldf{{\boldsymbol f}}
\def\bg{{\boldsymbol g}}
\def\bh{{\boldsymbol h}}
\def\bi{{\boldsymbol i}}
\def\bj{{\boldsymbol j}}
\def\bk{{\boldsymbol k}}
\def\bt{{\boldsymbol t}}
\def\bu{{\boldsymbol u}}
\def\bv{{\boldsymbol v}}
\def\bw{{\boldsymbol w}}
\def\bx{{\boldsymbol x}}
\def\by{{\boldsymbol y}}
\def\bz{{\boldsymbol z}}

\def\bmu{{\boldsymbol \mu}}
\def\bbeta{{\boldsymbol \beta}}
\def\bdelta{{\boldsymbol\delta}}
\def\beps{{\boldsymbol \eps}}
\def\blambda{{\boldsymbol \lambda}}
\def\bpsi{{\boldsymbol \psi}}
\def\bphi{{\boldsymbol \phi}}
\def\btheta{{\boldsymbol \theta}}
\def\bvphi{{\boldsymbol \vphi}}
\def\bxi{{\boldsymbol \xi}}

\def\bDelta{{\boldsymbol \Delta}}
\def\bLambda{{\boldsymbol \Lambda}}
\def\bPsi{{\boldsymbol \Psi}}
\def\bPhi{{\boldsymbol \Phi}}
\def\bSigma{{\boldsymbol \Sigma}}
\def\bTheta{{\boldsymbol \Theta}}

\def\bfzero{{\boldsymbol 0}}
\def\bfone{{\boldsymbol 1}}
\def\bPi{{\boldsymbol \Pi}}


% Symbols with hat
\def\hba{{\hat {\boldsymbol a}}}
\def\hf{{\hat f}}
\def\ha{{\hat a}}
\def\tcT{\widetilde{\mathcal T}}
\def\tK{\widetilde{K}}


\def\cR{\mathcal{R}}
\def\test{{\rm test}}
\def\train{{\rm train}}
\def\CV{\text{CV}}
\def\GCV{\text{GCV}}
\def\sfs{{\sf s}}

% rm symbols
\def\spn{{\rm span}}
\def\supp{{\rm supp}}
\def\Easy{{\rm E}}
\def\Hard{{\rm H}}
\def\post{{\rm post}}
\def\pre{{\rm pre}}
\def\Rot{{\rm Rot}}
\def\Sft{{\rm Sft}}
\def\endd{{\rm end}}
\def\KR{{\rm KR}}
\def\bbHe{{\rm He}}
\def\sk{{\rm sk}}
\def\de{{\rm d}}
\def\Tr{{\rm Tr}}
\def\lin{{\rm lin}}
\def\res{{\rm res}}
\def\degzero{{\rm deg0}}
\def\degone{{\rm deg1}}
\def\Poly{{\rm Poly}}
\def\Poly{{\rm Poly}}
\def\Coeff{{\rm Coeff}}
\def\de{{\rm d}}
\def\Unif{{\rm Unif}}
\def\lin{{\rm lin}}
\def\res{{\rm res}}
\def\RF{{\rm RF}}
\def\NT{{\rm NT}}
\def\Cyc{{\rm Cyc}}
\def\RC{{\rm RC}}
\def\kernel{\rm Ker}
\def\image{{\rm Im}}
\def\Easy{{\rm E}}
\def\Hard{{\rm H}}
\def\post{{\rm post}}
\def\pre{{\rm pre}}
\def\Rot{{\rm Rot}}
\def\Sft{{\rm Sft}}
\def\ddiag{{\rm ddiag}}
\def\KR{{\rm KR}}
\def\RR{{\rm RR}}
\def\bbHe{{\rm He}}
\def\eff{{\rm eff}}

\def\spn{{\rm span}}


%mathcal symbols
\def\cV{{\mathcal V}}
\def\cG{{\mathcal G}}
\def\cO{{\mathcal O}}
\def\cP{{\mathcal P}}
\def\cW{{\mathcal W}}
\def\cT{{\mathcal T}}
\def\cC{{\mathcal C}}
\def\cQ{{\mathcal Q}}
\def\cL{{\mathcal L}}
\def\cF{{\mathcal F}}
\def\cE{{\mathcal E}}
\def\cS{{\mathcal S}}
\def\cI{{\mathcal I}}
\def\cV{{\mathcal V}}
\def\cG{{\mathcal G}}
\def\cO{{\mathcal O}}
\def\cP{{\mathcal P}}
\def\cW{{\mathcal W}}
\def\cT{{\mathcal T}}
\def\cH{{\mathcal H}}
\def\cA{{\mathcal A}}


\def\tbA{\Tilde \bA}

%mathbb mathsf sf symbols
\def\K{{\mathbb K}}
\def\H{{\mathbb H}}
\def\T{{\mathbb T}}
\def\bbV{{\mathbb V}}
\def\W{{\mathbb W}}
\def\sM{{\mathsf M}}
\def\sW{{\mathsf W}}
\def\Unif{{\sf Unif}}
\def\normal{{\sf N}}
\def\proj{{\mathsf P}}
\def\ik{{\mathsf k}}
\def\il{{\mathsf l}}
\def\sM{{\sf M}}
\def\RKHS{{\sf RKHS}}
\def\RF{{\sf RF}}
\def\NT{{\sf NT}}
\def\NN{{\sf NN}}
\def\reals{{\mathbb R}}
\def\integers{{\mathbb Z}}
\def\naturals{{\mathbb N}}
\def\Top{{\mathbb T}}
\def\Kop{{\mathbb K}}
\def\Aop{{\mathbb A}}
\def\normal{{\sf N}}
\def\proj{{\mathsf P}}
\def\bbV{{\mathbb V}}
\def\sW{{\mathsf W}}
\def\sM{{\mathsf M}}
\def\T{{\mathbb T}}
\def\K{{\mathbb K}}
\def\H{{\mathbb H}}
\def\Unif{{\sf Unif}}
\def\normal{{\sf N}}
\def\Uop{{\mathbb U}}
\def\Hop{{\mathbb H}}
\def\Sop{{\mathbb S}}
\def\proj{{\mathsf P}}
\def\ik{{\mathsf k}}
\def\il{{\mathsf l}}
\def\sM{{\sf M}}
\def\RKHS{{\sf RKHS}}
\def\RF{{\sf RF}}
\def\NT{{\sf NT}}
\def\NN{{\sf NN}}
\def\reals{{\mathbb R}}
\def\integers{{\mathbb Z}}
\def\naturals{{\mathbb N}}
\def\proj{{\mathsf P}}
\def\Hop{{\mathbb H}}
\def\Uop{{\mathbb U}}
\def\App{{\rm App}}
\def\sU{{\sf U}}
\def\sV{{\sf V}}
\def\sfp{{\sf p}}
\def\tcE{\widetilde{\cE}}
\def\tmu{\widetilde  \mu}
\def\tbD{\widetilde{\bD}}




\def\stest{\mbox{\tiny\rm test}}

\def\seff{\mbox{\tiny\rm eff}}

\def\Ker{K}
\def\tKer{\tilde{K}}
\def\oKop{\overline{{\mathbb K}}}
\def\oKer{\overline{K}}
\def\ocV{\overline{{\mathcal V}}}

\def\th{\tilde{h}}
\def\tQ{\tilde{Q}}
\def\tsigma{\Tilde{\sigma}}


\def\hba{{\hat {\boldsymbol a}}}
\def\hf{{\hat f}}
\def\hy{{\hat y}}
\def\hU{\widehat{U}}
\def\hUop{\widehat{\mathbb U}}
\def\tbDelta{\widetilde{\bDelta}}


\def\tcT{\widetilde{\mathcal T}}

\def\Cyc{{\rm Cyc}}
\def\inv{{\rm inv}}


\def\cE{{\mathcal E}}
\def\cD{{\mathcal D}}
\def\cX{{\mathcal X}}
\def\cF{{\mathcal F}}
\def\cS{{\mathcal S}}
\def\cI{{\mathcal I}}



\def\He{{\rm He}}
\def\lin{{\rm lin}}
\def\res{{\rm res}}
\def\degzero{{\rm deg0}}
\def\degone{{\rm deg1}}
\def\Poly{{\rm Poly}}
\def\Coeff{{\rm Coeff}}
\def\de{{\rm d}}
\def\Unif{{\rm Unif}}
\def\RF{{\rm RF}}
\def\NT{{\rm NT}}
\def\Cyc{{\rm Cyc}}
\def\RC{{\rm RC}}

\def\tK{\widetilde{K}}
\def\stest{\mbox{\tiny\rm test}}


\def\ttau{\tilde{\tau}}


\def\cE{{\mathcal E}}
\def\bt{{\boldsymbol t}}
\def\normal{{\sf N}}

\def\bDelta{{\boldsymbol \Delta}}










\def\cX{{\mathcal X}}
\def\CKR{{\rm CKR}}
\def\bproj{{\overline \proj}}
\def\quadratic{{\rm quad}}
\def\cube{{\rm cube}}
\def\Cube{{\mathscrsfs Q}}

\def\Poly{{\rm Poly}}
\def\Coeff{{\rm Coeff}}
\def\RF{{\rm RF}}
\def\NT{{\rm NT}}
\def\bA{{\boldsymbol A}}
\def\btheta{{\boldsymbol \theta}}
\def\bTheta{{\boldsymbol \Theta}}
\def\bLambda{{\boldsymbol \Lambda}}
\def\blambda{{\boldsymbol \lambda}}

\def\cM{{\mathcal M}}

\def\cT{{\mathcal T}}
\def\cV{{\mathcal V}}
\def\bP{{\boldsymbol P}}
\def\diag{{\rm diag}}
\def\bS{{\boldsymbol S}}
\def\bO{{\boldsymbol O}}
\def\bD{{\boldsymbol D}}
\def\bPsi{{\boldsymbol \Psi}}
\def\bsh{{\boldsymbol h}}
\def\bL{{\boldsymbol L}}



\def\osigma{\overline{\sigma}}
\def\tbu{\Tilde \bu}
\def\tbZ{\Tilde \bZ}
\def\tbphi{\Tilde \bphi}
\def\tbpsi{\Tilde \bpsi}

\def\tbf{\Tilde \boldf}
\def\hbU{\hat{{\boldsymbol U}}_\lambda }
\def\hbUi{\hat{{\boldsymbol U}}_\lambda^{-1} }
\def\bb{{\boldsymbol b}}
\def\bsigma{{\boldsymbol \sigma}}

\def\hf{\hat f}
\def\hbf{\hat \boldf}
\def\bR{{\boldsymbol R}}
\def\bpsi{{\boldsymbol \psi}}
\def\cuH{\mathscrsfs{H}}

\def\noisestd{\sigma_{\varepsilon}}

\def\evn{{\mathsf m}}
\def\evN{{\mathsf M}}

\def\lvn{{\mathsf s}}
\def\lvN{{\mathsf S}}

\def\bc{{\boldsymbol c}}
\def\bC{{\boldsymbol C}}
\def\oba{\overline{{\boldsymbol a}}}
\def\uba{\underline{{\boldsymbol a}}}

\def\barsigma{\bar{\sigma}}

\def\tbN{\Tilde \bN}
\def\dv{{D}}

\def\tbV{\Tilde \bV}
\def\hiota{{\hat \iota}}
\def\biota{{\boldsymbol \iota}}
\def\hbiota{{\hat {\boldsymbol \iota}}}

\def\bzeta{{\boldsymbol \zeta}}
\def\hbzeta{{\hat {\boldsymbol \zeta}}}
\def\oproj{{\overline \proj}}
\def\barHop{\bar{\Hop}}
\def\barUop{\bar{\Uop}}
\def\barU{\bar{U}}
\def\barH{\bar{H}}
\def\ind{\mathbbm{1}}

\def\tC{\Tilde C}
\def\tQ{\Tilde Q}
\def\balpha{\boldsymbol{\alpha}}
\def\bgamma{\boldsymbol{\gamma}}
\def\cU{\mathcal{U}}
\def\tbC{\Tilde \bC}
\def\tba{\Tilde \ba}
\def\tbeta{\Tilde \beta}
\def\tbbeta{\Tilde \bbeta}
\def\boldf{\boldsymbol{f}}
\def\bXi{\boldsymbol{\Xi}}
\def\cB{\mathcal{B}}
\def\MP{{\rm MP}}
\def\complex{\mathbbm{C}}
\def\Im{{\rm Im}}
\def\tbM{\Tilde \bM}

\def\sR{\mathsf R}
\def\sV{\mathsf V}
\def\sB{\mathsf B}

\def\obR{\overline{\bR}}
\def\obM{\overline{\bM}}
\def\wbM{\widetilde{\bM}}
\def\tbR{\widetilde{\bR}}
\def\tbM{\widetilde{\bM}}

\def\ulambda{\overline{\lambda}}
\def\hbtheta{\hat \btheta}
\def\rr{{\rm r}}

\def\rC{\textcolor{red}{C}}

\def\rSQ{{\rm SQ}}

\def\rdc{{\rm dc}}
\def\rmc{{\rm mc}}
\def\cY{\mathcal{Y}}
\def\cZ{\mathcal{Z}}
\def\rdeg{{\rm deg}}


\def\dom{{\rm dom}}
\def\prox{{\rm prox}}
\def\hE{\widehat{\E}}
\def\okappa{\overline{\kappa}}
\def\otau{\overline{\tau}}
\def\br{{\boldsymbol r}}
\def\bGamma{{\boldsymbol \Gamma}}
\def\cJ{\mathcal{J}}
\def\oxi{\overline{\xi}}
\def\hbalpha{\hat{\balpha}}
\def\sfG{\textsf{G}}
\def\sfMG{\textsf{MG}}
\def\obz{\overline{\bz}}
\def\obZ{\overline{\bZ}}
\def\obg{\overline{\bg}}
\def\obG{\overline{\bG}}
\def\tbU{\Tilde{\bU}}
\def\obx{\overline{\bx}}
\def\ox{\overline{x}}



\def\tC{\Tilde C}
\def\tQ{\Tilde Q}
\def\balpha{\boldsymbol{\alpha}}
\def\bgamma{\boldsymbol{\gamma}}
\def\cU{\mathcal{U}}
\def\tbC{\Tilde \bC}
\def\tba{\Tilde \ba}
\def\tbeta{\Tilde \beta}
\def\tbbeta{\Tilde \bbeta}
\def\boldf{\boldsymbol{f}}
\def\bXi{\boldsymbol{\Xi}}
\def\cB{\mathcal{B}}
\def\MP{{\rm MP}}
\def\complex{\mathbbm{C}}
\def\Im{{\rm Im}}
\def\tbM{\Tilde \bM}

\def\sR{\mathsf R}
\def\sV{\mathsf V}
\def\sB{\mathsf B}

\def\ulambda{\overline{\lambda}}
\def\hbtheta{\hat \btheta}
\def\oPhi{\overline{\Phi}}
\def\sfPhi{\mathsf \Phi}

\def\hbSigma{\hat{\bSigma}}
\def\sfC{{\sf C}}
\def\sfc{{\sf c}}
\def\sfD{{\sf D}}
\def\sfM{{\sf M}}
\def\rmI{{\rm I}}
\def\rmII{{\rm II}}
\def\obQ{\overline{\bQ}}
\def\tS{\widetilde{S}} 
\def\tbS{\widetilde{\bS}}  
\def\obtheta{\overline{\btheta}}
\def\onu{\overline{\nu}}
\def\oT{\overline{T}}
\def\sL{\mathsf{L}}
\def\bq{\boldsymbol{q}}
\def\og{\overline{g}}
\def\oq{\overline{q}}
\def\ske{{\sf ske}}
\def\bs{{\boldsymbol s}}
\def\obD{\overline{\bD}}
\def\osfD{{\overline{{\sf D}}}}
\def\sflf{{\sf leaf}}
\def\sfT{{\sf T}}
\def\sfG{{\sf G}}
\def\bsfT{{\boldsymbol \sfT}}
\def\bsfG{{\boldsymbol \sfG}}
\def\obi{\overline{\bi}}
\def\obsfT{\overline{\bsfT}}
\def\obsfG{\overline{\bsfG}}
\def\oi{\overline{i}}
\def\osfT{\overline{\sfT}}
\def\osfG{\overline{\sfG}}
\def\sfH{{\sf H}}
\def\tbD{\widetilde{\bD}}
\def\polylog{\text{polylog}}
\def\tcL{{\widetilde{\cL}}}
\def\tsL{{\widetilde{\sL}}}

\def\seff{{\sf eff}}
\def\sG{\mathsf{G}}
\def\sKL{\mathsf{KL}}
\def\oevn{\overline{\evn}}
\def\obeta{\overline{\beta}}
\def\oC{\overline{C}}

\def\tnu{\Tilde{\nu}}
\def\hbSigma{\widehat{\bSigma}}
\def\tmu{\Tilde{\mu}}
\def\sK{{\sf K}}
\def\sA{{\sf A}}
\def\tPhi{\widetilde{\Phi}}
\def\obF{\overline{\bF}}
\def\oboldf{\overline{\boldf}}
\def\tr{\widehat{r}}
\def\hxi{\hat{\xi}}
\def\hr{\widehat{r}}
\def\hrho{\widehat{\rho}}
\def\trho{\widetilde{\rho}}
\def\tcA{\widetilde{\cA}}
\def\obv{\overline{\bv}}
\def\tsB{\widetilde{\sB}}
\def\tbG{\widetilde{\bG}}


\newcommand{\G}{\mathbf{G}}
\newcommand{\GT}{\mathbf{G}^\top}
\newcommand{\bet}{\boldsymbol{\beta}}
\newcommand{\U}{\mathbf{U}}
\newcommand{\V}{\mathbf{V}}
\newcommand{\D}{\mathbf{D}}
%\newcommand{\R}{\mathbb{R}}
%\newcommand{\E}{\mathbb{E}}
\newcommand{\Sph}{\mathbb{S}}
%\newcommand{\I}{\mathbb{I}}
%\newcommand{\Pr}{\mathbb{P}}
%\newcommand{\bx}{\boldsymbol{x}}
%\newcommand{\bw}{\boldsymbol{w}}
%\newcommand{\bz}{\boldsymbol{z}}
\newcommand{\bblV}{{\color{blue}\bV}}

%%%%%%%%%%%%%%%%%%%%%%%%%%%%%%%%
% THEOREMS
%%%%%%%%%%%%%%%%%%%%%%%%%%%%%%%%
\theoremstyle{plain}
\newtheorem{theorem}{Theorem}[section]
\newtheorem{proposition}[theorem]{Proposition}
\newtheorem{lemma}[theorem]{Lemma}
\newtheorem{corollary}[theorem]{Corollary}
\theoremstyle{definition}
\newtheorem{definition}[theorem]{Definition}
\newtheorem{assumption}[theorem]{Assumption}
\theoremstyle{remark}
\newtheorem{remark}[theorem]{Remark}
\newcommand\plabel[1]{\phantomsection\label{#1}}


\begin{document}

\twocolumn[
\icmltitle{A Mixture-Based Framework for Guiding Diffusion Models}

\icmlsetsymbol{equal}{*}

\begin{icmlauthorlist}
    \icmlauthor{Yazid Janati}{equal,yyy}
    \icmlauthor{Badr Moufad}{equal,yyy}
    \icmlauthor{Mehdi Abou El Qassime}{yyy}\\
    \icmlauthor{Alain Durmus}{yyy}
    \icmlauthor{Eric Moulines}{yyy}
    \icmlauthor{Jimmy Olsson}{sch}
    %\icmlauthor{}{sch}
    %\icmlauthor{}{sch}
\end{icmlauthorlist}

\icmlaffiliation{yyy}{Ecole polytechnique}
% \icmlaffiliation{comp}{Company Name, Location, Country}
\icmlaffiliation{sch}{KTH University}

\icmlcorrespondingauthor{Yazid Janati, Badr Moufad}{first.last@polytechnique.edu}
% \icmlcorrespondingauthor{Badr Moufad}{}

% You may provide any keywords that you
% find helpful for describing your paper; these are used to populate
% the "keywords" metadata in the PDF but will not be shown in the document
\icmlkeywords{Machine Learning, ICML}

\vskip 0.3in
]

% this must go after the closing bracket ] following \twocolumn[ ...

% This command actually creates the footnote in the first column
% listing the affiliations and the copyright notice.
% The command takes one argument, which is text to display at the start of the footnote.
% The \icmlEqualContribution command is standard text for equal contribution.
% Remove it (just {}) if you do not need this facility.

%\printAffiliationsAndNotice{}  % leave blank if no need to mention equal contribution
\printAffiliationsAndNotice{\icmlEqualContribution} % otherwise use the standard text.

\begin{abstract}
  Denoising diffusion models have driven significant progress in the field of Bayesian inverse problems. Recent approaches use pre-trained diffusion models as priors to solve a wide range of such problems, only leveraging inference-time compute and thereby eliminating the need to retrain task-specific models on the same dataset. To approximate the posterior of a Bayesian inverse problem, a diffusion model samples from a sequence of intermediate posterior distributions, each with an intractable likelihood function. This work proposes a novel mixture approximation of these intermediate distributions. Since direct gradient-based sampling of these mixtures is infeasible due to intractable terms, we propose a practical method based on Gibbs sampling. We validate our approach through extensive experiments on image inverse problems, utilizing both pixel- and latent-space diffusion priors, as well as on source separation with an audio diffusion model. The code is available at \url{https://www.github.com/badr-moufad/mgdm}.
\end{abstract}


%%%%%%%%%%%%
%%%%%%%%%%%%
\section{Introduction}
Inverse problems occur when a signal $X$ of interest must be inferred from an incomplete and noisy observation $Y$, a challenge frequently encountered in diverse fields such as weather forecasting, image reconstruction (\emph{e.g.}, tomography or black-hole imaging), and speech processing.  Such problems are typically ill-posed, 
%as the observations can correspond to infinitely many possible signals. However, most of these solutions are either physically implausible or lack practical relevance, 
making it essential to incorporate additional constraints, regularization techniques, or prior knowledge to arrive at meaningful and realistic solutions. 

The Bayesian framework, in conjunction with generative modeling, offers a systematic approach to the challenges associated with inverse problems. Prior knowledge about the signal of interest, often represented through samples from its underlying distribution $\pdata{0}{}{}$, can be leveraged to train a generative model $\pdata{0}{}{}[\param]$ that acts as a prior. 
%The a priori knowledge about the signal of interest, usually materialized with samples from its underlying distribution $\pdata{0}{}{}$, can be used to learn a generative model $\pdata{0}{}{}[\param]$ that serves as prior. 
By combining it with the conditional density $\pot{0}{\bx}$ of the observation given the signal, deduced from the form of the inverse problem at hand, we can compute the posterior distribution. 
Samples drawn from this posterior encapsulate plausible solutions that harmonize prior knowledge with the observed data.
%Samples drawn from this posterior represent plausible solutions that reconcile both the prior knowledge and the observed data. 
One straightforward approach to approximate sampling from the posterior distribution involves constructing a paired dataset of i.i.d. signals and observations, $(\bX_i, Y_i)_{i = 1}^N$, where $\bX_i \sim \pdata{0}{}{}$ and $Y_i \sim g_0(\cdot | \bX_i)$, and learning a direct mapping \cite{dong2015image} or generative model \cite{ledig2017photo,isola2017image}. The latter, when queried with multiple independent noise samples alongside an observation, %$\obs$, 
generates a diverse set of potential reconstructions. However, this approach is inherently \emph{task-specific}, delivering reliable reconstructions only when the conditional distribution of the observation remains unchanged at test time. As a result, it cannot straightforwardly adapt to unseen tasks with the same prior. Adaptation to a new task can only be achieved by retraining a new generative model. 

An increasingly popular approach consists in learning a generative model only for the prior $\pdata{0}{}{}$, and then leveraging inference-time compute to solve any inverse problem for which the likelihood function $\bx \mapsto \pot{0}{\bx}$ is provided in a closed form. This strategy eliminates the need for expensive and inefficient task-specific training. Initially explored with generative models such as variational autoencoders and generative adversarial networks \cite{xia2022gan}, this framework has recently been extended to denoising diffusion models (DDMs) \cite{song2021score,kadkhodaie2020solving,kawar2021snips,kawar2022denoising,chung2023diffusion,song2022pseudoinverse,daras2024survey}, which are the focus of the present paper.

DDMs \cite{sohl2015deep,song2019generative,ho2020denoising} achieve state-of-the-art generative performance across a wide range of domains. At their core is a forward noising process that transforms the data distribution $\pdata{0}{}{}$ into a Gaussian distribution. A generative model is then learned by  reversing this noising process. With a specific parameterization of the backward process, which converts noise into data samples, training the generative model reduces to approximating denoisers for each noise level introduced during the forward process. Recent methods for training-free posterior sampling aim to approximate the denoisers for the posterior distribution, enabling the use of diffusion models for sampling \cite{ho2022video,chung2023diffusion,song2022pseudoinverse}. A posterior distribution denoiser can be decomposed into two terms: the prior denoiser at the same noise level (provided by a pre-trained diffusion model) and the gradient of the log-likelihood of the observation conditioned on the current noisy sample. The latter term, which is intractable, is what guides the samples during the denoising process towards the posterior distribution. Various approximations for this gradient term have been proposed. However, they are often crude and require significant adjustments and heuristics to ensure stability and satisfactory performance. When applied to latent diffusion models, they often demand additional, model-specific adjustments 
\cite{rout2024solving}.\\


\textbf{Our contribution.}\, In this paper, we present a principled method that circumvents these issues by introducing a new approximation of the likelihood term, paired with a sampling scheme based on Gibbs sampling \cite{geman1984stochastic}. 
Our key observation is that multiple approximations can be derived for each likelihood term at a fixed noise level using a simple identity that it satisfies.
However, the scores of these new likelihood approximations are not available in closed form, preventing us from deriving a direct posterior denoiser approximation by combining, through a mixture, the different likelihood approximations. We overcome this limitation by constructing a mixture approximation of the intermediate posterior distributions defined by the diffusion model for the original posterior. 
Our algorithm,  \algoname\ (\algo), proceeds by sequentially sampling from these mixtures using Gibbs sampling. This is enabled by a carefully designed data augmentation scheme that ensures straightforward Gibbs updates. A key advantage of our approach is its adaptability to available computational resources. Specifically, the number of Gibbs iterations acts as a tunable parameter, allowing substantial improvements with increased inference-time compute. \algo\ demonstrates strong empirical performance across 10 image-restoration tasks involving both pixel-space and latent-space diffusion models, as well as in musical source separation, even matching the performance of supervised methods. 
%A key advantage of our approach is its flexibility in scaling with available computational resources. Specifically, the number of Gibbs iterations serves as a tunable parameter: by increasing the number of iterations, the algorithm can improve its performance when more inference-time compute is available. 
%The strong empirical performance of \algo\ is demonstrated on 10 image-restoration tasks, with both pixel-space and latent-space diffusion models, as well as musical source separation, and is further validated through comparisons with 10 competing methods.

\section{Background}
\section{Basic Background: Supervised Learning and the PAC Model}
\label{sec:background}

At this point almost everyone has heard of machine learning (ML). Anyone likely to stumble upon this article will have also heard of its most influential special case, supervised learning, and those theoretically inclined will also be familiar with the PAC model. Nonetheless, I will set the stage by  recapping the basics.

\subsection{Basics of Supervised Learning}%Let's set the stage in any case

\emph{Supervised Learning} is the task of ``coming up'' with a function $f: \X \to \Y$ to ``explain'' or ``fit'' a sequence of input/output examples   $(x_1,y_1), \ldots, (x_n,y_n)$, with $x_i \in \X$ and $y_i \in \Y$.  Here $\X$ is a \emph{data domain} consisting of \emph{datapoints} $x \in \X$, $\Y$ is a \emph{label set} consisting of \emph{labels} $y \in \Y$, and the sequence $(x_1,y_1),\ldots,(x_n,y_n)$ is the \emph{training data} consisting of \emph{labeled examples (a.k.a. samples)}~$(x_i,y_i)$.  I~will refer to the chosen function $f$ as a \emph{predictor}, and to $n$ as the \emph{sample size}. A \emph{learning algorithm} takes as input training data, and outputs (some representation of) a predictor $f \in \Y^\X$.\footnote{Note that this describes the usual \emph{batch}, a.k.a.~\emph{offline}, setting of supervised learning. I do not discuss other paradigms such as online or active learning in this article.} 



Success in supervised learning is defined as \emph{generalization} to  future examples: For a typical \emph{test example}  $(x_{\tst},y_{\tst})$, the predicted label $y'_{\tst}=f(x_{\tst})$ should ``equal'' $y_{\tst}$, perhaps approximately. We usually assume the test example is drawn from the same  ``source'' as the training data  --- commonly, i.i.d.~from the same distribution. The quality of the prediction is quantified by $\ell(y'_{\tst},y_{\tst})$, where $\ell:~\Y~\times~\Y \to \RR_{\geq 0}$ is a \emph{loss function} chosen as part of the problem definition. Common loss functions include the 0-1 loss $\ell_{0-1}(y',y) = [y' \neq y]$ for \emph{classification} problems,\footnote{The notation $[P]$ denotes $1$ when predicate $P$ is true, and denotes $0$ when $P$ is false.} as well as the absolute loss $|y'-y|$ or squared loss $(y'-y)^2$ for \emph{regression problems} featuring $\Y  \sse \RR$.

Nontrivial generalization properties are typically only possible if one assumes something about the data.\footnote{The need for such an assumption is formalized by the  \emph{no free lunch theorems} of supervised learning \cite{wolpert_connection_1992,wolpert_lack_1996,schaffer_conservation_1994}.} The Bayesian approach to  machine learning, common in many applications, assumes some parametric form for the distribution generating the data, and postulates a prior on the parameters. This is not the approach I will take in this article. Instead, I will focus on the frequentist --- and some would say ``worst-case'' or ``adversarial'' ---  approach that is common in the computational learning theory community, embodied by the PAC model. Here we assume that the (training and test) data can be explained, perhaps approximately, by a function in some ``simple enough to learn'' class of functions $\H \sse \Y^\X$, often called the \emph{hypotheses}. Equivalently, we  seek a predictor which explains the unseen data roughly  as well as the best hypothesis $h^* \in \H$, whether or not we assume that $h^*$ itself provides a perfect explanation.



 \paragraph{Common Algorithmic Templates.} Perhaps the best known general-purpose supervised learning algorithm is \emph{empirical risk minimization (ERM)}, which chooses as its predictor a hypothesis $f \in \H$ minimizing $\frac{1}{n} \sum_{i=1}^n \ell(f(x_i),y_i)$ --- a quantity called the \emph{training error}, \emph{empirical error}, or \emph{empirical risk} of $f$. %\footnote{When multiple hypotheses minimize the empirical risk, we assume ERM breaks ties arbitrarily.}
A common template for generalizing ERM involves adding a \emph{regularization term} $\psi(f)$ to the  objective function, typically chosen to measure some notion of ``hypothesis complexity.'' An algorithm instantiating this template is known as a \emph{structural risk minimizer (SRM)}, and chooses as its predictor the hypothesis $f \in \H$ minimizing the \emph{structural risk} $\frac{1}{n} \sum_{i=1}^n \ell(f(x_i),y_i) + \psi(f)$. Other well-known algorithms, such as gradient descent and its variations,  can frequently be interpreted as approximate implementations of ERM or SRM.


\paragraph{Proper vs Improper Learning.} A learning algorithm is said to be \emph{proper} if its predictor $f$ is always chosen from the hypothesis class, i.e., $f \in \H$, otherwise it is said to be \emph{improper}. ERM  is an example of a proper learning algorithm, as are SRM algorithms of the form described above.  In the \emph{proper regime} of learning, algorithms are required to be proper. This article will be concerned with the more flexible \emph{improper regime} (a.k.a \emph{representation-independent learning}), where no such constraint is placed on the learner. In other words, all we care about is predictive power at test time, rather than any insights derived from the functional form or representation of the predictor~itself.


\subsection{The PAC Model}
A standard mathematical setup for evaluation of supervised learning algorithms, at least in the theoretical computer science community, is Valiant's \emph{Probably Approximately Correct (PAC) model} of learning (see e.g.~\cite{kearns_introduction_1994,mohri_foundations_2018}). Here, we assume there is an unknown distribution $\D$ on $\X \times \Y$ from which training and test data are  drawn.  Specifically, the labeled datapoints of the training set  $(x_1,y_1), \ldots, (x_n,y_n)$, as well as the test data  $(x_\tst,y_\tst)$, are i.i.d.~from $\D$. Often it is assumed that $\D$ lies in some class of distributions of interest. The \emph{true expected loss}, or simply \emph{loss}, of a predictor $f: \X \to \Y$ is the expected loss it incurs on draws from $\D$, written $L_\D(f) = \Ex_{(x,y) \sim \D} \ell(f(x),y)$.


There are two main ``settings'' in PAC learning. The  \emph{realizable setting} only requires that the data be perfectly explained by some hypothesis in $\H$. More generally, the \emph{agnostic setting} makes no assumption relating the data to the hypotheses, but shifts the goalposts as necessary to allow nontrivial guarantees: the expected loss at test time is evaluated only ``relative'' to that of the best hypothesis $h^* \in \H$. There are other settings which make more nuanced assumptions, such as $\D$ being of a particular parametric form or its support living in some (unknown) lower-dimensional space, etc. I will mostly discuss the realizable and agnostic settings in this article, those being the simplest and most studied from a theoretical perspective. %TODO:We will briefly discuss other settings in Section ??

The PAC model demands high probability guarantees of learners, in the worst case over distributions of interest. Consider first the realizable setting, where $\D$ is such that $\min_{h \in \H} L_{\D}(h) = 0$. A PAC learner has \emph{error} $\epsilon=\epsilon(n)$ and \emph{confidence} $\delta=\delta(n)$ if, when training data consists of $n$ i.i.d~samples from a realizable distribution $\D$, it produces a predictor $f$  satisfying $L_\D(f) \leq \epsilon$ with probability at least $1-\delta$. In the agnostic setting, where $\D$ can be arbitrary, we require $L_\D(f) - \min_{h \in \H} L_\D(h) \leq \epsilon$ with probability $1-\delta$.

In both the realizable and agnostic settings, we look for PAC learners with small $\epsilon$ and $\delta$ as a function of the sample size $n$. An equivalent perspective looks at the sample complexity $m(\epsilon,\delta)$, which is the minimum sample size which guarantees error  at most $\epsilon$ with probability at least $1-\delta$. We say a problem is \emph{PAC learnable} if its PAC sample complexity is finite whenever $\epsilon,\delta > 0$.

For most PAC learning problems, learnability and sample complexity are characterized in terms of a  ``dimension'' of the hypothesis class. Most prominently this is the \emph{VC dimension} for binary classification, the \emph{fat shattering dimension} for agnostic regression, and the \emph{DS dimension} for multiclass classification (see \cite{anthony_neural_1999,daniely_optimal_2014,brukhim_characterization_2022}). Treatment of these is beyond the scope of this article. The unfamiliar reader need not worry, however,  as dimensions will feature only tangentially in our~discussion.




%\paragraph{Learning settings: Realizable, Agnostic, etc.} In learning theory, evaluating a supervised learning algorithm requires specifying a data model and an objective. We will leave the details of the data model flexible for now, to allow for both the PAC model and the adversarial transductive model. Nonetheless we will describe two variations, which we call ``settings'', which cut across different models. The  \emph{realizable setting}  requires only that the data be perfectly explained by some hypothesis $h \in \H$ --- i.e., there exists a hypothesis which is guaranteed to suffer a loss of $0$ on training and test data. The performance of the learning algorithm is its expected loss at test time for some ``worst case'' realizable instance. More generally, the \emph{agnostic setting} makes no assumption relating the data to the hypotheses, but shifts the goalposts as necessary to allow nontrivial guarantees: the expected loss at test time is evaluated only ``relative'' to that of the best hypothesis $h^* \in \H$, again for some ``worst case'' instance. There are other settings which make more nuanced assumptions about the data, such as it is drawn from a distribution of a particular parametric form, or that it lives in some (unknown) lower-dimensional space, etc. We will mostly discuss the realizable and agnostic settings, those being the simplest and most studied from a theoretical perspective.




%%% Local Variables:
%%% mode: latex
%%% TeX-master: "learning_matching"
%%% End:

% \begin{figure}
%     \centering
%     \includegraphics[width=0.5\linewidth]{Move_teaser.pdf}
%     \caption{Comparison of different dynamic compute approaches. length of arrow indicates residual transformation per token while width indicates velocity of transformation.}
%     \label{fig:enter-label}
% \end{figure}

\section{Method}
\label{sec:method}
Residual connections play a crucial role in shaping token representations, yet their dynamics remain underexplored in the context of efficient decoding. In this work, we delve deeper into transformer residual dynamics and investigate how modulating residual transformation velocity can improve inference efficiency in token-level processing, optimizing both dense and sparse MoE transformers.


\subsection{Residual Dynamics and Motivation for Multi-rate Residuals} \label{sec:motivation}

To analyze how hidden representations evolve across different layers of a transformer architecture, it's crucial to consider the effect of residual connections. Each transformer decoder layer typically has residual connections across attention and MLP submodules. As the residual stream $h_i$ traverses from interval $E_j$ to $E_{j+1}$, it undergoes a residual transformation given by:  
% \begin{equation}
% \label{eq:slow_residual_transformation}
% H_{E_{j+1}} = H_{E_j} \prod_{i=E_j}^{E_{j+1}} \left( I + \mathcal{A}_i \right) \left( I + \mathcal{M}_i \right) \quad \text{where} \quad \mathcal{A}_i = f(c_i, h_{i}), \mathcal{M}_i = g(h_i)
% \end{equation}

\begin{equation} \label{eq:slow_residual_transformation}
h_{E_{j+1}} = h_{E_j} + \sum_{i=E_j}^{E_{j+1}-1} \left( \mathcal{A}_i(h_i) + \mathcal{M}_i(h_i + \mathcal{A}_i(h_i)) \right) \quad \text{where} \quad \mathcal{A}_i = f(c_i, h_{i}), \mathcal{M}_i = g(h_i). 
\end{equation}

Here, \( \mathcal{A}_i \) denotes the non-linear transformation introduced by the multi-head attention mechanism at layer \( i \), while \( \mathcal{M}_i \) corresponds to the non-linear transformation of the MLP block at the same layer. These transformations depend on the input residual stream \( h_i \) and, in the case of \( \mathcal{A}_i \), the previous contextual representation \( c_i \).\footnote{Normalization layers are typically applied in practice but are omitted here for simplicity of the argument.}


% For easy tokens, the magnitude and direction of this delta transformation become progressively smaller with each successive layer as shown in \cref{fig:delta_transformation}. Consequently, it is feasible to predict these tokens after only a few residual connections, whereas harder tokens necessitate more extensive processing through additional layers.

\begin{figure}[ht]
    \centering
    \begin{subfigure}{0.48\textwidth}
        \centering
        \includegraphics[width=\textwidth]{sections/figures/residual_change.pdf}
        \caption{}
        \label{fig:residual_change}
    \end{subfigure}%
    \hfill
    \begin{subfigure}{0.48\textwidth}
        \centering
        \includegraphics[width=\textwidth]{sections/figures/alignment_wrt_dedicated_model.pdf}
        \caption{}
    \label{fig:alignment_wrt_dedicated_model}
    \end{subfigure}
    \caption{(a) As residual streams propagate through the model, the directional shifts in the residuals become progressively smaller. (b) A dedicated model with $k$ layers achieves a faster rate of change in residual streams and higher alignment than base model leveraging early exit mechanisms at layer $k$.}
    \label{fig}
\end{figure}


To examine whether residual transformations can be accelerated across layers, we conducted experiments using a diverse set of prompts on a pre-trained Phi3 model~\cite{phi3_report}. As illustrated in \cref{fig:residual_change}, we measured the directional shift in residual states as \( 1 - \mathcal{C}(h_{i-1}, h_i) \), where \(\mathcal{C}\) denotes normalized cosine similarity. This shift is notably higher in the initial layers, gradually decreasing in subsequent layers. This behavior allows traditional early exit approaches to effectively accelerate decoding by enabling earlier exits for simpler tokens. However, these approaches typically rely on a distance-based approximation, where the full residual transformation of the model is approximated by the residual transformations of the initial layers. To gain deeper insights into the distance versus velocity aspects of residual transformation, we conducted a comparative study. Specifically, we trained an early exit head at layer $k$ of the Phi3 model, which consists of 32 layers, restricting the distance traveled by each token. To accelerate the residual transformation relative to number of layers, we trained a smaller model consisting of only $k$ layers, while keeping all other hyperparameters consistent. We then compared the next-token prediction accuracy of the early exit head of the base model with that of the smaller model. To ensure an equal number of trainable parameters, we inserted low-rank adapters into the smaller model and trained only these adapters, whereas, in the distance-based approach, we trained solely the early exit head. In addition, to accelerate the residual transformation in smaller model, we distilled the residual streams from the larger model by incorporating a distillation loss ~\cite{sanh2019distilbert} between the residual state at layer \(i\) of the smaller model and the residual state at layer \(4 \times i\) of the larger model. As shown in ~\cref{fig:alignment_wrt_dedicated_model} the smaller model demonstrates a significantly faster rate of change in residual streams, leading to higher next token prediction accuracy after $k$ layers compared to the base model that employs traditional early exit mechanisms after $k$ layers \cite{schuster2022confident, chen2023eellm, varshney-etal-2024-investigating}. This experimental setup, which modifies only the rate of change in residual streams while keeping other factors constant, suggests that dense transformers, trained with a fixed number of layers, may inherently possess a slow residual transformation bias.

This observation raises an intriguing question: if the rate of change in residual streams could be accelerated relative to the number of layers, is it possible to facilitate earlier alignment for a greater proportion of tokens? Earlier alignment would be beneficial to not only facilitate dynamic computation but also for generating speculative tokens efficiently with high acceptance rates in speculative decoding setups ~\cite{leviathan2023fast, chen2023accelerating}. 

%thereby enhancing the efficiency of early exiting? 
 % This bias likely constrains the effectiveness of early exiting, particularly for easier tokens. By addressing this limitation through accelerated residual transformations, we hypothesize that it is possible to substantially improve the efficiency and accuracy of early exit strategies in transformer models.

\subsection{Multi-Rate Residual Transformation} \label{m2r2_method}

To address the slow residual transformation bias described in ~\cref{sec:motivation}, we introduce \textit{accelerated residual streams} that operate at rate $R$ relative to original slow residual stream. We pair slow residual stream, $h$ with an accelerated residual stream, $p$, which has an intrinsic bias towards earlier alignment. Relative to ~\cref{eq:slow_residual_transformation}, accelerated residual transformation from interval $E_j$ to $E_{j+1}$ can be represented as: 

% \begin{equation}
% \label{eq:fast_residual_transformation}
% P_{E_{j+1}} = P_{E_j} \prod_{i=E_j}^{E_{j+1}} \left( I + \hat{\mathcal{A}_i} \right) \left( I + \hat{\mathcal{M}_i} \right) \quad \text{where} \quad \hat{\mathcal{A}_i} = \hat{f}(c_i, P_{i}), \hat{\mathcal{M}_i} = \hat{g}(P_{i})
% \end{equation}


\begin{equation} \label{eq:fast_residual_transformation}
p_{E_{j+1}} = p_{E_j} + \sum_{i=E_j}^{E_{j+1}-1} \left( \hat{\mathcal{A}_i}(p_i) + \hat{\mathcal{M}_i}(p_i + \hat{\mathcal{A}_i}(p_i)) \right) \quad \text{where} \quad \hat{\mathcal{A}_i} = \hat{f}(c_i, p_{i}), \hat{\mathcal{M}_i} = \hat{g}(h_i), 
\end{equation}



where $\hat{\mathcal{A}_i}$ and $\hat{\mathcal{M}_i}$ denote non-linear transformation added by layer $i$ to previous accelerated residual $p_{i}$. Similar to $\mathcal{A}_i$, non-linear transformation $\hat{\mathcal{A}_i}$ attends to same context $c_i$ but uses a different transformation $\hat{f}$ for accelerating $p_{E_j}$ relative to $h_{E_j}$. 

We integrate accelerated residual transformation directly into the base network using parallel accelerator adapters such that rank of accelerator adapters $R_p << d$ where $d$ denotes base model hidden dimension. This setup allows the slow residual stream $h_{E_j}$ to pass through the base model layers while the accelerated residual stream $p_{E_j}$ utilizes these parallel adapters as shown in ~\cref{fig:m2r2_main}. Both slow and accelerated residuals are processed in same forward pass via attention masking and incur negligible additional inference latency in memory bound decoding setups, while in compute bound decoding setups where FLOPs optimization is essential, accelerated residual stream utilizes a fraction of attention heads that of slow residual (see ~\cref{sec:flops_optimization}). Additionally, to maximize the utility of accelerated residual transformations without introducing dedicated KV caches, we propose a shared caching mechanism between the slow and accelerated streams which minimally impact alignment benefits of our approach while offering substantial memory savings (see ~\cref{fig:koala_alignment}). Specifically, the attention operation on the slow residuals \( \text{MHA}(h_t, h_{\leq t}, h_{\leq t}) \) is redefined for accelerated residuals as 
\[
\hat{\mathcal{A}} = MHA(p_t, h_{<t} \oplus p_t, h_{<t} \oplus p_t),
\]
where the accelerated residual at time-step $t$, \( p_t \) attends to the slow residual’s KV cache, facilitating the reuse of contextual information across both residual streams without incurring additional caching costs. Here, \(MHA(q, k, v) \) represents multi-head attention between query \( q \), key \( k \), and value \( v \).

\begin{figure}
    \centering
    \includegraphics[width=0.8\linewidth]{sections//figures/m2r2_main2.pdf}
    \caption{Multi-rate Residuals Framework: Slow residual stream of base model is accompanied by a faster stream that operates at a $2-(J+1)\times$ rate relative to the slow stream, undergoing transformations via accelerator adapters as detailed in \cref{m2r2_method}, where J denotes number of early exit intervals. Colors within the slow and fast residual streams indicate similarity, with matching colors representing the most closely aligned residual states. At the beginning of the forward pass and at each exit point, the accelerated residual state is initialized from the corresponding slow residual state to avoid gradient conflict during training (see ~\cref{sec:grad_conflict}). Early exiting decisions are informed by the Accelerated Residual Latent Attention (ARLA) mechanism, described in \cref{method_arla}, which evaluates residual dynamics across consecutive exit gates.}
    \label{fig:m2r2_main}
\end{figure}

% Furthermore. to maximize the benefits of fast residual transformations without using dedicated KV caches, we propose sharing the fast network’s cache with the slow network. Formally speaking, We modify attention operation on slow residuals $MHA(H_t, H_{<=t}, H_{<=t})$ as $MHA(P_{t}, H_{<t} \oplus P_t, H_{<t}  \oplus P_t)$ such that accelerated residuals attend to previous slow context KV cache, where $MHA(q,k,v)$ denotes multi head attention between query, $q$, key $k$ and value $v$.


\subsection{Enhanced Early Residual Alignment}
Early residual alignment is instrumental in optimizing early exiting, speculative decoding, and Mixture-of-Experts (MoE) inference mechanisms. In this section, we provide a detailed analysis of how accelerated residuals enhance these inference setups.

% By aligning the residual states of intermediate layers with the final output representations, the model can maintain high prediction accuracy even when computations are truncated at earlier layers. This enables more reliable early exiting, reducing the overall computational cost while preserving performance. Additionally, in speculative decoding, early residual alignment allows the model to make confident predictions using faster, partial computations, thereby accelerating inference without sacrificing output quality.


\subsubsection{Early Exiting} \label{method_early_exiting}

A prevalent strategy for enabling early exiting at an intermediate layer $E_{j}$ involves approximating the residual transformation between $E_{j}$ and the final layer $N-1$ using a linear, context independent mapping, $\mathcal{T}$, such that $H_{N-1} \approx \mathcal{T}(H_{E_{j}})$. This approximation has been extensively employed in conventional approaches ~\cite{schuster2022confident, chen2023eellm, varshney-etal-2024-investigating}, providing a computationally efficient means to project the output of deeper layers from intermediate states. Specifically, residual state of layer $N-1$ with this approximation can be expressed as:


% \begin{equation}
% \label{eq: vanila_ea_assumption}
% \Phi(H_{E_{j}}) \sim H_{E_{j}} \prod_{i=E_{j}}^{N}\left( I + \mathcal{A}_i \right) \left( I + \mathcal{M}_i \right) \quad \text{where} \quad \Phi \perp C
% \end{equation}

\begin{equation} \label{eq:early_exiting}
h_{E_j} + \sum_{i=E_j}^{N-1} \left( \mathcal{A}_i(h_i) + \mathcal{M}_i(h_i + \mathcal{A}_i(h_i)) \right) \sim \mathcal{T}(h_{E_{j}})  \quad \text{where} \quad \mathcal{T} \perp c. 
\end{equation}


Here, $\mathcal{A}_i$ and $\mathcal{M}_i$ represent the residual contributions of the multi-head attention and MLP layers, respectively, while $\mathcal{T}$ remains independent of $c$, the preceding context.

This approach is inherently limited by two major factors: first, the assumption of linearity between $h_{E_{j}}$ and $h_{N-1}$ may not hold uniformly for all tokens, particularly when $E_j \ll N$. Second, the linear transformation $\mathcal{T}$ disregards the influence of the context $c$ and fails to account for the latent representations of previous contextual states. In contrast, M2R2 accelerated residual states mitigate both of these challenges by approximating the slow residual transformation of all layers via a faster residual transformation of fewer layers as:
% \begin{equation}
% H_{E_j} \prod_{i=E_j}^{N}\left( I + \mathcal{A}_i \right) \left( I + \mathcal{M}_i \right) \sim P_{E_j} \prod_{i=E_j}^{E_j+1}\left( I + \hat{\mathcal{A}_i} \right) \left( I + \hat{\mathcal{M}_i} \right)
% \end{equation}


\begin{equation} \label{eq:m2r2_approximating_ea}
h_{E_j} + \sum_{i=E_j}^{N-1} \left( \mathcal{A}_i(h_i) + \mathcal{M}_i(h_i + \mathcal{A}_i(h_i)) \right) \sim p_{E_j} + \sum_{i=E_j}^{E_{j+1}-1} \left( \hat{\mathcal{A}_i}(p_i) + \hat{\mathcal{M}_i}(p_i + \hat{\mathcal{A}_i}(p_i)) \right), 
\end{equation}

% \begin{equation} \label{eq:fast_residual_transformation}
% p_{E_{j+1}} = p_{E_j} + \sum_{i=E_j}^{E_{j+1}-1} \left( \hat{\mathcal{A}_i}(p_i) + \hat{\mathcal{M}_i}(p_i + \hat{\mathcal{A}_i}(p_i)) \right) \quad \text{where} \quad \hat{\mathcal{A}_i} = \hat{f}(c_i, p_{i}), \hat{\mathcal{M}_i} = \hat{g}(h_i) 
% \end{equation}






where $p_{E_j}$ is initialized from the slow residual state $h_{E_j}$ at each early exit interval $E_j$ using an identity transformation (see ~\cref{fig:m2r2_main}). As shown in ~\cref{fig:m2r2_residual_sim}, accelerated residuals offer a smoother, more consistent shift in residual direction across layers, in contrast to the abrupt changes typically seen at early exit points in standard early exit methods. Moreover, the normalized cosine similarity between accelerated states at early exit intervals and final residual states is substantially higher compared to traditional early exit techniques, highlighting improved alignment with final layer representations. Traditional adaptive compute methods are constrained by two principal factors: the number of tokens eligible for early exit at intermediate layers and the precision of early exit decision. If residual streams fail to saturate early, the majority of tokens remain ineligible for exit, thereby diminishing potential speedups. Additionally, imprecise delineations between tokens suitable for early exit can lead to underthinking (premature exits that adversely affect accuracy) or overthinking (unnecessary processing that compromises efficiency) ~\cite{zhou2020self, dai2020dynamic}. Enhanced early alignment using ~\cref{eq:m2r2_approximating_ea} helps to address  first issue. To address the second issue we introduce Accelerated Residual Latent Attention, which dynamically assesses the saturation of the residual stream, allowing for a more precise differentiation between tokens that can exit early and those requiring further processing.

% This results in uniform change in residual direction    
% % We keep $\mathcal{A} = \hat{\mathcal{A}}$, while $\hat{\mathcal{M}}$ is accelerated by a factor of $2 - (N_{E}+1)X$ relative to the slower residual transformation $\mathcal{M}$, where $N_E$ represents number of early exiting intervals.
% Figure~\cref{fig:rate_change_comparison} illustrates the comparative rate of change between these transformation streams.



% fig:rate_change_comparison
% - grid plot x axis -> layer id (0, 8) , y axis -> layer id -> dark color cell for max similarity , lighter for lower 
% 
-------------------------------------------------------
Let's consider residual stream $h_i$ traverses through interval $E_j$ to $E_{j+1}$ and undergoes residual transformation given by 
\begin{equation}
h_{E_{j+1}} = h_{E_j} \prod_{i=E_j}^{E_{j+1}} \left( 1 + \delta_i \right)    
\end{equation}

where $\delta_i$ denotes non-linear transformation added by layer $i$. Each non-linear transformation of layer $i$ is a function of previous contextual representation, $c_i$ and input residual stream $h_i-1$ as
$\delta_i = f(c_i, h_{i-1})$ 

One way to exit early at exit $E_j+1$ is to assume that residual transformation from $E_j+1$ to final layer $N-1$ can be approximated by a linear function $\phi$ as $h_{N-1} \sim \Phi(h_{E_j+1})$ and most conventional approaches such as \todo{cite EA papers} use this approach. In other words, 

\begin{equation}
\Phi(h_{E_j+1} \sim h_{E_j+1} \prod_{i=E_j+1}^{N} \left( 1 + \delta_i \right)   
\end{equation}

This approach suffers from two primary issues, linearity assumption from $h_E_j+1$ to $H_N-1$ if often incorrect, particularly when $E_j << N$. More importantly, linear transformation $\Phi$ doesn't consider effect of context $C_i$. M2R2  effectively addresses these issues as accelerated residual stream at interval $E_j+1$ can be represented as 

\begin{equation}
r_{E_{j+1}} = r_{E_j} \prod_{i=E_j}^{E_{j+1}} \left( 1 + \gamma_i \right)    
\end{equation}

where $\gamma_i$ denotes non-linear transformation added by layer $i$ to previous accelerated residual $r_i-1$. Similar to $\delta_i$, non-linear transformation $\gamma_i$ considers context $C_i$ as 
$\gamma_i = g(c_i, r_{i-1})$. So in summary, slow residual transformation is approximated by accelerated residual as: 

\begin{equation}
h_{E_j} \prod_{i=E_j}^{N} \left( 1 + \delta_i \right) \sim h_{E_j} \prod_{i=E_j}^{E_j+1} \left( 1 + \gamma_i \right)
\end{equation}

It's worth noting that accelerated residual $r_i$ and slow residual $h_i$ are processed concurrently at layer $i$ by constructing proper attention mask such as attention of slow residual is represented as 

$MHA(H_it, H_{i<=t}, H_{i<=t}$ while attention of fast residual is computed as 

$MHA(r_it, H_{i<=t}, H_{i<=t}$ where $MHA(q,k,v$ denotes multi head attention between query, $q$, key $k$ and value $v$.


------------------------------------------------------------------

Vertical latent attention on accelerated residual is computed as 
$MHA(S_mt, S(Ej<=i<=m)t, S(Ej<=i<=m)t)$ where $Smt$ denotes query/key/value projection in latent domain at layer $m$ at time $t$. 
------------------------------------------------------------------

Gradient conflict Avoidance: 

Let's consider $w_j$ is a trainable parameter that belongs to a layer between $E_j$ and $E_j+1$. Consider early exit loss at gate $E_j+1$, $L_j+1$, gradient propagation of $w_j$ at another trainable parameter $w_j-n$ can be gives as 

$\sum_{k=E_j-n}^{E_j} \beta_k \frac{\partial L_{E_k}}{\partial w_k}$

where $\beta_j$ denotes backward transformation coefficient for weight $w_j$ to reach gate $E_j$. 
 
On the other hand, gradient propagation in proposed approach can be represented as 

\[
\frac{\partial L_{E_j}}{\partial w_j} = 
\begin{cases} 
\beta_j \frac{\partial L_{E_j}}{\partial w_j} & \text{if } E_j \leq w_j \leq E_{j+1} \\
0 & \text{otherwise}
\end{cases}
\]







% \begin{figure}[ht]
%     \centering
%     \includegraphics[width=0.8\textwidth, height=5cm]{rate_change_comparison.png}
%     \caption{Rate of change comparison between fast and slow residual streams.}
%     \label{fig:rate_change_comparison}
% \end{figure}

%vary k and and plot EA accuracy for larger and smaller models. 

% \begin{figure}[ht]
%     \centering
%     \includegraphics[width=0.5\textwidth,height=5cm]{sections/figures/alignment_comparison_dialogsum.pdf}
%     \caption{Alignment of exited tokens for different early exit layers using traditional early exiting heads, dedicated faster networks, and faster residuals.}
%     \label{fig:small_model_early_exiting}
% \end{figure}


\textbf{Accelerated Residual Latent Attention} \label{method_arla}

In the context of residual streams, we observe that the decision to exit at a given layer can be more effectively informed by analyzing the dynamics of residual stream transformations, instead of solely relying on a classification head applied at the early exit interval $E_j$. To capture the subtle dynamics of residual acceleration, we propose a \textit{Accelerated Residual Latent Attention} (ARLA) mechanism. This approach involves making the exit decision at gate $E_j$ by attending to the residuals spanning from gate $E_{j-1}$ to $E_j$, rather than considering only the residual at gate $E_j$. To minimize the computational overhead associated with exit decision-making, the attention mechanism operates within the latent domain as depicted in ~\cref{fig:arla_arch}. Formally, for each interval $[E_j, E_{j+1}]$, the accelerated residuals are projected into Query ($Q^s_{E_j}, \ldots, Q^s_{E_{j+1}}$), Key ($K^s_{E_j}, \ldots, K^s_{E_{j+1}}$), and Value ($V^s_{E_j}, \ldots, V^s_{E_{j+1}}$) vectors, with latent dimension $d^s$ for $Q^s$, $K^s$, and $V^s$ being significantly smaller than hidden dimension of $p$.\footnote{We use $d^s = 64$ for experiments described in ~\cref{sec:experiments}.} Notably, when the router is allowed to make exit decisions at gate $E_j$ based on residual change dynamics, we observe that the attention is not confined to the residual state at $E_j$ but is distributed across residual states from $E_{j-1}$ to $E_j$, %as illustrated in Figure~\ref{fig:vertical_latent_attention_dynamics}. 
This broader focus on residual dynamics significantly reduces decision ambiguity in early exits, as demonstrated in Figure~\ref{fig:roc_arla}, which contrasts routers based on the last hidden state, and the proposed ARLA router.

%show R -> S transformation. 
%show parameter and flop overhead as compared to adapter on last hidden state.

% \begin{figure}[ht]
%     \centering
%     \includegraphics[width=0.5\textwidth,height=5cm]{sections/figures/roc_arla.pdf}
%     \caption{ROC curves of early exit decision strategies: confidence-based methods (CALM/LITE), routers based on the accelerated hidden state, and latent attention routers.}
%     \label{fig:decision_making_comparison}
% \end{figure}

% \begin{figure}[ht]
%     \centering
%     \includegraphics[width=0.5\textwidth,height=5cm]{vertical_latent_attention.png}
%     \caption{Vertical latent attention mechanism for optimizing early exit decisions by considering residuals from gate \(M\) through \(M-1\).}
%     \label{fig:vertical_latent_attention}
% \end{figure}

\begin{figure}[ht]
    \centering
    \begin{subfigure}{0.52\textwidth}
        \centering
        \includegraphics[width=\textwidth, height = 4cm]{sections/figures/arla_arch.pdf}
        \caption{Accelerated Residual Latent Attention (ARLA): Accelerated residuals between early exit gates are projected into latent domain and attention over residual states within the interval is computed to capture residual dynamics and exit decision is made based on residual saturation.}
        \label{fig:arla_arch}
    \end{subfigure}%
    \hfill
    \begin{subfigure}{0.45\textwidth}
        \centering
        \includegraphics[width=\textwidth, height = 4.5cm]{sections/figures/vla_roc.pdf}
        \caption{ROC classification curves of early exit decision strategies using a linear router used on last residual state ~\cite{schuster2022confident, varshney-etal-2024-investigating, chen2023eellm}  and using ARLA approach that considers residual dynamics. }
        \label{fig:roc_arla}
    \end{subfigure}
    \caption{Effectiveness of ARLA in capturing residual dynamics for early exiting decisions.}


\end{figure}



% \begin{figure}[ht]
%     \centering
%     \includegraphics[width=1\textwidth,height=5cm]{sections/figures/arla.pdf}
%     \caption{fig that plots 32 rows 2 cols heatmap showing attention at each gate}
%     \label{fig:vertical_latent_attention_dynamics}
% \end{figure}

\subsubsection{Self Speculative Decoding} \label{method_self_speculative_decoding}

An alternative means to exploit the early alignment properties of our approach is through the use of accelerated residual states for speculative token sampling to accelerate autoregressive decoding. Speculative decoding aims to speed up memory-bound transformer inference by employing a lightweight draft model to predict candidate tokens, while verifying speculated tokens in parallel and advancing token generation by more than one token per full model invocation \cite{leviathan2023fast, chen2023accelerating, xia2023speculative, miao2023specinfer}. Despite its effectiveness in accelerating large language models (LLMs), speculative decoding introduces substantial complexity in both deployment and training. A separate draft model must be specifically trained and aligned with the target model for each application, which increases the training load and operational complexity ~\cite{chen2023accelerating}. Additionally, this approach is resource-inefficient, as it requires both the draft and target models to be simultaneously maintained in memory during inference \cite{leviathan2023fast, chen2023accelerating}. 

One strategy to address this inefficiency is to leverage the initial layers of the target model itself to generate speculative candidates, as depicted in ~\cite{Tang2024}. While this method reduces the autoregressive overhead associated with speculation, it suffers from suboptimal acceptance rates. This occurs because the linear transformation employed for translating hidden states from layer $k$ to the final layer $N$ is typically a poor approximation, as discussed in ~\cref{sec:motivation} and ~\cref{method_early_exiting}. Our approach resolves this limitation by utilizing accelerated residuals, which demonstrate higher fidelity to their slower counterparts. By utilizing accelerated residuals operating at a rate of $N/k$, where $k$ denotes the number of layers used for candidate speculation, we are able to efficiently generate speculative tokens for decoding.\footnote{We typically set $k = 4$ to balance the trade-off between autoregressive drafting overhead and acceptance rate, as discussed in~\cref{sec:experiments}.}
 This technique not only obviates the need for multiple models during inference but also improves the overall efficiency and effectiveness of speculative decoding.

\begin{figure}
    \centering    \includegraphics[width=1\linewidth]{sections/figures/m2r2_aot_loading.pdf}
    \caption{Ahead-of-Time Expert Loading: M2R2 accelerated residual stream predicts experts required for future layers, reducing reliance on on-demand lazy loading. Speculative pre-loading is efficiently overlapped with computation of multi-head attention (MHA) and MLP transformations. Only incorrectly speculated experts are loaded lazily, resulting in faster inference steps and improved computational efficiency. Here, H indicates LBM Host while D indicates HBM Device.}
    \label{fig:moe_expert_aot_loading}
\end{figure}


\subsubsection{Ahead of Time Expert Loading:} \label{method_aot_expert_loading}

Recent advancements in sparse Mixture-of-Experts (MoE) architectures ~\cite{shazeer2017outrageously, fedus2022switch, artetxe2019massively, lepikhin2020gshard, zoph2022designing} have introduced a paradigm shift in token generation by dynamically activating only a subset of experts per input, achieving superior efficiency in comparison to dense models, particularly under memory-bound constraints of autoregressive decoding \cite{fedus2022switch, zoph2022designing}. This sparse activation approach enables MoE-based language models to generate tokens more swiftly, leveraging the efficiency of selective expert usage and avoiding the overhead of full dense layer invocation. In dense transformer models, pre-loading layers is a common strategy to enhance throughput, as computations of current layer can be overlapped with pre-loading of next layer parameters ~\cite{narayanan2021efficient, shoeybi2020megatron}. However, MoE models face a unique challenge: expert selection occurs dynamically based on previous layer’s output, making it infeasible to preload next layer’s experts in parallel. This limitation results in inherent latency, as expert loading becomes a sequential, on-demand process ~\cite{lepikhin2020gshard, fedus2022switch}.

To address this inefficiency, our method introduces a mechanism with \textit{accelerated residuals}, which not only captures key characteristics of base slower residual states but also exhibit high cosine similarity with their final counterparts (as illustrated in \cref{fig:m2r2_residual_sim}). By employing accelerated residual streams, we can effectively predict the necessary experts for future layers well in advance of their actual invocation. Specifically, using a $2\times$ accelerated residual, the experts needed for layers $2i+2$ and $2i+3$ can be identified while still computing in layer $i$, thus overcoming the bottleneck of sequential, on-demand expert selection and mitigating latency in the decoding pipeline, as shown in \cref{fig:moe_expert_aot_loading}. Note that, we use fixed set of accelerator adapters for transforming accelerated residuals (as discussed in ~\cref{m2r2_method}) while slow residual is transformed via expert routing mechanism. 

Furthermore, our approach integrates a Least Recently Used (LRU) caching strategy, which enhances memory efficiency by replacing the least recently used experts with speculated experts that are anticipated to be needed in upcoming layers. This hybrid approach of preemptive expert loading with LRU caching yields substantial improvements over traditional on-demand loading or standalone caching strategies. By minimizing cache misses and efficiently managing memory, this approach addresses both compute and memory bottlenecks, leading to faster, more resource-efficient token generation in MoE architectures. A comprehensive evaluation of this strategy, in relation to state-of-the-art methods, is provided in \cref{experiments_aot}, and the compute and memory traces on an A100 GPU are detailed in \cref{fig:moe_aot_cuda_trace}.



% Recent advancements in sparse Mixture-of-Experts (MoE) architectures have introduced the concept of utilizing distinct computational paths for different tokens \cite{shazeer2017outrageously}. This approach, wherein only a subset of experts are activated per input, enables MoE-based language models to generate tokens more swiftly compared to their dense counterparts due to memory-bound nature of auto-regressive decoding. In dense models, pre-loading layers in advance is a common strategy to enhance computational efficiency. However, this technique is not applicable to MoE models, where expert selection occurs dynamically based on the outputs of previous layers, preventing parallel pre-fetching of experts.

% Our proposed method addresses this inefficiency. Accelerated residuals, which are highly similar to their slower counterparts (see \cref{fig:similarity}), can reliably predict the necessary experts ahead of time. For instance, by utilizing $2X$ accelerated residual stream, we can predict the experts needed for the layer $2i+1$ and $2i+3$ while carrying out computation in layer $i$. This enables us to commence expert loading significantly earlier, as illustrated in \cref{expert_loading}, effectively mitigating the delays observed with the naive on-demand expert loading. Additionally, our method benefits from incorporating a Least Recently Used (LRU) strategy, where speculated experts replace those that are least recently utilized, resulting in improved performance compared to using either strategy alone. For a comprehensive evaluation, refer to \cref{moe_trace}, which provides a CUDA compute and memory trace of our approach executed on <>.



% A naive solution involves using the residual state of the previous layer along with the gating function of the next layer to predict which experts need to be loaded, and initiating the expert loading process in parallel with the attention computation of the next layer. Yet, as shown in \cref{fig:MOE_attn_vs_loading_time}, the attention computation for medium to long contexts is considerably faster than the expert loading time, making this approach inefficient.




\subsection{Training} \label{method_training}
% This approach is feasible due to the absence of gradient conflicts, as discussed in \cref{sec:grad_conflict}.

To accelerate residual streams, we employ parallel accelerator adapters as described in \cref{m2r2_method}.  For the early exiting use-case outlined in \cref{method_early_exiting}, we define the training objective for these adapters using the following loss function, which combines cross-entropy loss at each exit $E_j$ with distillation loss at each layer $i$. Loss weights coefficients $\alpha_0$ and $\alpha_1$ are employed to balance contribution of corresponding losses.

\begin{align} \label{eq:mr_loss}
L_{\text{m2r2}} = \underbrace{-\alpha_0 \sum_{j=1}^{J} \sum_{t=1}^{T} \log p_{\theta} \left( \hat{y}_t^{E_j} \mid y_{<t}, x \right)}_{\text{cross-entropy loss}} 
+ \underbrace{\alpha_1\sum_{i=1}^{E_{J-1}} \sum_{t=1}^{T} \| \mathbf{p}_{t}^{i} - \mathbf{h}_{t}^{((i - E_{j(i)}) \cdot R_i) + E_{j(i)})} \|^2}_{\text{distillation loss}}.
\end{align}

where $\hat{y}_t^{E_j}$ denotes the predictions from the accelerated residual stream at layer $E_j$ and time step $t$, $y_t$ represents the corresponding ground truth tokens, and $x$ indicates previous context tokens. The distillation loss at each layer $i$ is computed by comparing accelerated residuals at layer $i$ with slow residuals at layer $(i - E_{j(i)}) \cdot R_i + E_{j(i)}$, where $R_i$ denotes the rate of accelerated residuals at layer $i$ while $E_{j(i)}$ represents the most recent gate layer index such that $E_{j(i)} <= i$. \( J \) represents the total number of early exit gates, N denotes number of hidden layers and $E_j$ denotes layer index corresponding to gate index $j$ and \( T \) denotes the sequence length. 

In dynamic compute settings, after training of accelerator adapters, we optimize the query, key, and value parameters governing the ARLA routers (see ~\cref{method_arla}) across all exits in parallel on binary cross entropy loss between predicted decision and ground truth exiting decision. The ground truth labels for the router are determined based on whether the application of the final logit head on $\hat{y}_t^{E_j}$ yields the correct next-token prediction. 


% The objective for this optimization is defined by the following loss function:


%TODO are equations required ? 
% \begin{equation} \label{eq:arla_loss_combined}\small
%     L_{\text{arla}} = -\frac{1}{N} \sum_{t=1}^{T} \left( \sum_{j=1}^{E_n} \left[ O_t^{E_j} \log(\hat{O}_t^{E_j}) + (1 - O_t^{E_j}) \log(1 - \hat{O}_t^{E_j}) \right] \right), \quad \text{where} \quad 
%     O_t^{E_j} = \begin{cases} 
%     1, & \text{if } L(\hat{y}_t^{E_j}) = y_t^{E_j} \\
%     0, & \text{otherwise}
%     \end{cases}
% \end{equation}

% where $\hat{O}_t^{E_j}$ represents the binary predicted logits produced by the vertical latent attention router, as described in \cref{sec:arla}, at gate $E_j$ and time step $t$, and $O_t^{E_j}$ denotes the corresponding ground truth labels. The ground truth labels for the router are determined based on whether the application of the logit head on $\hat{y}_t^{E_j}$ yields the correct next-token prediction. The parameters controlling vertical latent attention are trained concurrently to ensure consistency and efficient use of computational resources.

For self-speculative decoding, as described in \cref{method_self_speculative_decoding}, the training objective remains the same as \cref{eq:mr_loss}, but with the number of intervals set to $J = 1$ and the rate of residual transformation set to $R_n = N/k$, where the first $k$ layers generate speculative candidate tokens. In the context of Ahead-of-Time Expert Loading for Mixture-of-Experts (MoE) models (see \cref{method_aot_expert_loading}), setting the rate of residual transformation to $R_n = 2$ typically offers a good trade-off between the accuracy of expert speculation and AoT pre-loading of experts. 

% Thus, we set $J = 1$ and $E_1 = 16$.


~\subsection{FLOPs Optimization} \label{sec:flops_optimization}

Naively implemented, M2R2 incurs higher FLOP overhead compared to traditional speculative decoding and early exiting approaches such as ~\cite{medusa, schuster2022confident, Tang2024}. However, modern accelerators demonstrate compute bandwidth that exceeds memory access bandwidth by an order of magnitude or more~\cite{databricksLLMInference2023, jouppi2021ten}, meaning increased FLOPs do not necessarily translate to increased decoding latency. Nevertheless, to ensure fair comparison and efficiency in compute bound scenarios, we introduce targeted optimizations.

~\textbf{Attention FLOPs Optimization} For medium-to-long context lengths, attention computation dominates FLOPs in the self-attention layer, surpassing the contribution from MLP layers. Specifically, matrix multiplications involving queries, cached keys, and cached values scale with $l_{kv} * l_{q}$ where $l_{kv}$ denotes previous context length and $l_q$ denotes current query length. Since M2R2 pairs accelerated residuals with slow residuals, a naive implementation results in twice the FLOPs consumption compared to a standard attention layer. To address this, we limit the attention of accelerated residual stream to selectively attend to the top-k most relevant tokens, identified by the slow residual stream based on top attention coefficients\footnote{We set to k = 64 and attend to top 64 tokens as identified by the slow residual stream.}. This is possible since slow and accelerated residual streams are processed in same forward pass and accelerated streams have access to attention coefficients of slow stream. Note that, the faster residual stream still retains the flexibility to assign distinct attention coefficients to these tokens. Furthermore, we design the faster residual stream to employ only 8 attention heads, compared to the 32 heads used in the slow residual stream of the Phi-3 model, reducing query, key, value, and output projection FLOPs by a factor of 1/4. ~\cref{fig:m2r2_num_heads_ablation} indicates effect of using a slicker stream on alignment. As depicted, using $\hat{n}_h = 8$ offers a good trade-off between alignment and FLOPs overhead. 

~\textbf{MLP FLOPs Optimization} The accelerator adapters operating on the accelerated residual stream are intentionally designed with lower rank than their counterparts in the base model. This reduces FLOP overhead by a factor proportional to $hiddenSize / rank$. Additionally, since the faster residual stream uses only 8 attention heads (compared to 32 in the slow residual stream of Phi-3), the subsequent MLP layers process a smaller set of activations, further reducing FLOPs by another factor of 1/4.

These optimizations significantly reduce the FLOP overhead per speculative draft generation, as illustrated in ~\cref{fig:flops_optmization}. Notably, while traditional early-exiting speculative approaches such as DEED require propagating the full slow residual state through the initial layers, incurring substantial computational costs, M2R2 achieves efficient token generation via slimmer, low-rank faster residual streams. In contrast, Medusa introduces considerable FLOP overhead due to per-head computations scaling with $d^2+dv$\footnote{Here $d$ denotes hidden state dimension while $v$ denotes vocab size.}, whereas M2R2 employs low-rank layers for both MLP and language modeling heads, maintaining computational efficiency. All experiments involving the M2R2 approach, as detailed in ~\cref{sec:experiments}, are conducted using these FLOPs optimizations.









% \[
% O_t^{E_j} = 
% \begin{cases} 
% 1, & \text{if } L(\hat{y}_t^{E_j}) = y_t^{E_j} \\
% 0, & \text{otherwise}
% \end{cases}
% \]




%add distillation
% We train accelerator adapters described in \cref{m2r2_method} to accelerate residual streams on next token prediction all in parallel since there are no gradient conflict issues as described in \cref{sec:grad_conflict}.

% \begin{align} \label{eq:mr_loss}
% L_{mr} =  & -\sum_{j = 1}^{E_n} (\sum_{t=1}^{T}\log p_{\theta} (\hat{y}_t^{E_j} | \hat{y}_{<t}, x)) \nonumber
% \end{align}

% where $\hat{y_t^{E_j}}$ denotes predicted logits obtained from accelerated residual stream at gate $E_j$ and time-step $t$ while $y_t^{E_j}$ denotes corresponding truth tokens. 

% Upon training of adapters responsible for accelerating residual streams, we train query, key, value parameters responsible for vertical latent attention of all gates in parallel as

% \begin{equation} \label{eq:arla_loss}
%     L_{arla} = -\frac{1}{N} (\sum_{t=1}^{T}(1\sum_{j=1}^{E_n} \left[ O_t^{E_j} \log(\hat{O}_t^{E_j}) + (1 - o_t^{E_j}) \log(1 - \hat{o_t}_{E_j}) \right]))
% \end{equation}

% where $\hat{O_t^{E_j}}$ denotes binary predicted logits obtained from vertical latent attention router described in \cref{sec:arla} at gate $E_j$ and timestep $t$ while $O_t^{E_j}$ denotes corresponding truth label. Truth labels for router are obtained by computing whether logit head application on $\hat{y}_t^j$ results in true next token prediction. Formally speaking, 

% $O_t^{E_j} = 1 if L(\hat{y_t^{E_j}}) == y_t^{E_j} , 0 otherwise$. 

% Parameters responsible for vertical latent attention are also trained in parallel as well. 

%todo: training slow and fast residuals together and distillation can be two training mdoes. 
%Distillation can be an ablation. 




% Although transformer decoding is memory bound on most mainstream accelerators, there could be scenarios where flop savings are crucial. For instance, on on-device settings power consumption is directly correlated with flops per decoding step and reducing flops does help with overall energy consumption. Vanilla early exiting methods help with flop reduction but suffer from mismatch between training and inference due to early exited tokens. If token at decoding step $t$, $T_t$ exited at layer $E_i$, while token $T_{t+k}$ exits at layer $E_j$ such that $E_i < E_j$, hidden state $H_{t+k}l$ does not have corresponding hidden state $H_tl$ to attend to where $E_i < l <= E_j$. One solution that's often used in literature is to rely on last hidden state available, $H_t{E_j}$, however it tends to be sub-optimal and does affect generation quality \cite{ref}.  To alleviate this mismatch while reducing flops, we train router such that attention mask between token $T_{t+k}$ and token $T_{<t+k}$ is given by: 

% \begin{equation}
%     a_{T_{{t+k}{T_{<t+k}}} = 1 if  E_{T_{<t+k}} >= E{T_{t+k}}
%     else 0
% \end{equation}

% This attention mask enables router to account for exited tokens and get trained accordingly. Since attention mechanism during decoding remains exactly same as that during training, impact on generation quality tends to be minimal as noted in \cref{fig:gen_auality_with_and_without_recompute_attention_show_flops}.  Although MoD does not suffer from training and inference mismatch, we observe that it suffers from discountinuity between pre-training and super-vised fine-tuning resulting in sub-optimal perplexity. On the other hand, our method doesn't not require pre-training , doesn't suffer from discountinuity, and achieves much better perplexity in super-vised fine-tuning and instruction tuning setups as shown in \cref{fig:Mod_vs_m2r2_loss_curves}.






% Our techniques are directly applicable in such scenarios.    




%expert loading with cuda streams in experiments
\section{Experiments}
\label{sec:experiments}
\section{Experiments: Planning outperforms Heuristics}
\label{sec:experiment}

We begin our empirical demonstrations by showcasing the effectiveness of our planning framework on both synthetic and real datasets. We focus on the simplest planning algorithm, 1-step lookaheads (Algorithm~\ref{alg:complete}), and show that even basic planning can hold great promise. 
We illustrate our framework using two uncertainty quantification modules---GPs and 
\ensembles/ \ensembleplus. 

Throughout this section, we focus on evaluating the mean squared error of 
a regression model $\model$,  and develop adaptive policies that minimize uncertainty on $g(f)$ defined in~\eqref{eqn:l2-g-f}.
When GPs provide a valid model of uncertainty, 
our experiments show that our planning framework significantly outperforms other baselines. 
We further demonstrate that our conceptual framework extends to deep learning-based uncertainty quantification methods such as  \ensembleplus while highlighting computational challenges that need to be resolved in order to scale our ideas. 
For simplicity, we assume a naive predictor, i.e., $\psi(\cdot) \equiv 0$. However, we emphasize that this problem is just as complex as if we were using a sophisticated model $\psi(.)$. The performance gap between the algorithms 
primarily depends
on the level  of uncertainty in our prior beliefs.

To evaluate the performance of our algorithm, we benchmark it against several baselines. 
%Active learning baselines use an acquisition function $\ac$ to select points that have the highest   function value: $X\opt_t \in \argmax_{X \in \xpoolj{t}} \ac({X})$ at every step $t$. These methods may also need an UQ module, which we simply use the same UQ module as in our algorithm, and it  outputs $V(X)$ that measures the the uncertainty of each point $X \in \xpoolj{t}$.
Our first set of baselines are from active learning~\citep{AggarwalKoGuHaPh14}:
\\ % \noindent\textbf{Active Learning Heuristics:} 
\textbf{(1)} 
\textsf{Uncertainty Sampling (Static):}  In this approach, we query the samples for which the model is least certain about. Specifically, we estimate the variance of the latent output $f(X)$ for each $X \in \xpool$ using the UQ module and select the top-$K$ points with the highest uncertainty. \\
\textbf{(2)} \textsf{Uncertainty Sampling (Sequential):} This is a greedy heuristic that sequentially selects the points with the highest uncertainty within a batch, while updating the posterior beliefs using pseudo labels from the current posterior state. Unlike \textsf{Uncertainty Sampling (Static)}, this method takes into account the information gained from each point within batch, and hence tries to diversify the selected points within a batch. 

 
We also compare our approach to the  \textbf{(3)} \textsf{Random Sampling}, which selects each batch uniformly at random from the pool. Additionally, we compare solving the planning problem using  \textsf{REINFORCE}-based policy gradients with   $\mathsf{Smoothed\text{-}Autodiff}$ policy gradients.\footnote{Our code repository is available at
  \url{https://github.com/namkoong-lab/adaptive-labeling}.}
%Detailed experimental setups are provided in Section \ref{sec:details-experiments}.

%We repeat all experiments with 10 random seeds.




\begin{figure}[t]
\centering
\begin{minipage}[b]{0.49\textwidth}
\centering
\includegraphics[width=\textwidth, height=5cm]{figures/original_scale/Var_of_l_2_loss.pdf}
\caption{(Synthetic data) Variance of mean squared loss evaluated through the posterior belief $\mu_t$ at each horizon $t$. This is the objective that policy gradient methods like \textsf{REINFORCE} and $\ouralgo$ optimizes. 1-step lookaheads are surprisingly effective even in long horizons.}
\label{fig:var-l2-sim}
\end{minipage}
\hfill
\begin{minipage}[b]{0.49\textwidth}
\centering \includegraphics[width=\textwidth, height=5cm]{figures/original_scale/Error_of_estimated_model_l_2_loss.pdf}
\caption{(Synthetic data) Error between MSE calculated based on collected data $\mc{D}^{0:T}$ vs. population oracle MSE over $\mc{D}_{\rm eval} \sim P_X$. Reducing uncertainty over posteriors directly leads to better OOD evaluations. 1-step lookaheads significantly outperform active learning heuristics in small horizons.}
\label{fig:mean-l2-sim}
\end{minipage}
%\caption{Simulated data for GPs}
%\label{fig:both_plots}
\end{figure}

\subsection{Planning with Gaussian processes}
\label{sec:experiment-plan-GP}
We now briefly describe the data generation process for the GP experiments,  deferring a more detailed discussion of the dataset generation to Section~\ref{sec:details-experiments}. 
We use both the synthetic data and the real data to test our methodology.
For the \emph{simulated data},  we construct a setting where the general population is distributed across \emph{51 non-overlapping clusters} while the initial labeled data $\dtrain$ just comes from one cluster. In contrast, both $\dpool \defeq (\xpool,\ypool),\deval \defeq (\xeval,\yeval)$ are generated   from all the clusters. 
We begin with a low-dimensional scenario, generating a one-dimensional regression setting using a GP. %Gaussian Process (GP).
Although the data-generating process is not known to the algorithms,  we assume that the GP hyperparameters are known to all the algorithms
to ensure fair comparisons. This can be viewed as a setting where our prior is well-specified, allowing us to isolate the effects
of different policy optimization approaches
 without any concerns about the misspecified priors. We select $10$ batches, each of size $K=5$ across $T = 10$ time horizons.

To examine the robustness of our method against the distributional assumptions made  in the simulated case, we then move to a real dataset where the correct prior is not known. We simulate selection bias from the eICU dataset~\citep{PollardJoRaCeMaBa18}, which contains real-world patient data with in-hospital mortality outcomes. 
We conduct a $k$-means clustering to generate 51 clusters and then select data from those clusters. We view this to be a credible replication of practice, as severe distribution shifts are common due to selection bias in clinical labels.  To convert the binary mortality labels into a regression setting, we train a  random forest classifier and fit a GP on predicted scores, which serves as the UQ module for all the algorithms. As before, the task is to select 10 batches, each consisting of 5 samples, across 10 time horizons.

 In Figures~\ref{fig:var-l2-sim} and~\ref{fig:mean-l2-sim}, we present results for the simulated data. 
Figure~\ref{fig:var-l2-sim} shows the variance of $\ell_2$ loss, and Figure~\ref{fig:mean-l2-sim} presents the error in the estimated $\ell_2$ loss using $\mu_t$ (relative to true $\ell_2$ loss, that is unknown to the algorithm). 
As we can see from these plots, our method one-step lookahead  gives substantial improvements  over active learning baselines and random sampling. In addition,
compared to the one-step lookahead planning approach using \textsf{REINFORCE}-based policy gradients, 
we observe that $\mathsf{Smoothed\text{-}Autodiff}$-based policy gradients provide significantly more robust performance over all horizons.

In Figures~\ref{fig:var-l2-real}~and~\ref{fig:mean-l2-real}, we observe similar findings on the eICU data. We see that planning policies (\textsf{REINFORCE} and $\mathsf{Smoothed\text{-}Autodiff}$) consistently outperform other heuristics by a large margin.  Active learning baselines perform poorly in these small-horizon batched problems and can sometimes be even worse than the random search baselines.  Overall, our results show the importance of careful planning in adaptive labeling for reliable model evaluation. 

We offer some intuition as to why one-step lookahead planning may outperform other heuristic algorithms. 
 First,  \textsf{Uncertainty sampling (Static)} while myopically selects the
 top-$K$ inputs with the highest uncertainty, it fails to consider 
the overlap in information content among the ``best” instances; see \citep{AggarwalKoGuHaPh14} for more details. 
In other words,  it might acquire points from the same region with high uncertainty while failing to induce diversity among the batch.
Although \textsf{Uncertainty Sampling (Sequential)} somewhat addresses the issue of information overlap, a significant drawback of 
this algorithm
is the disconnect between the objective we aim to optimize and the algorithm. For example, it might sample from a region with high uncertainty but very low density. 

\begin{figure}[t]
\centering
\begin{minipage}[b]{0.48\textwidth}
\centering
\includegraphics[width=\textwidth, height=5cm]{figures/original_scale/Var_of_l_2_loss_real.pdf}
\caption{(Real-world eICU data) Variance of mean squared loss evaluated through the posterior belief $\mu_t$ at each horizon $t$. Even 1-step lookaheads are extremely effective planners, and auto-differentiation-based pathwise policy gradients provide a reliable optimization algorithm based on low-variance gradient estimates.}
\label{fig:var-l2-real}
\end{minipage}
\hfill
\begin{minipage}[b]{0.48\textwidth}
\centering \includegraphics[width=\textwidth, height=5cm]{figures/original_scale/Error_of_estimated_model_l_2_loss_real.pdf}
\caption{(Real-world eICU data) Error between MSE calculated based on collected data $\mc{D}^{0:T}$ vs. population oracle MSE over $\mc{D}_{\rm eval} \sim P_X$. Reducing uncertainty over posteriors directly leads to better OOD evaluations. Our method significantly outperforms active learning-based heuristics, and random sampling.}
\label{fig:mean-l2-real}
\end{minipage}
%\caption{Real data for GPs}
\end{figure}
 
%\vspace{-1.5cm}
% \begin{wrapfigure}{r}{.32\columnwidth}
%   \vspace{-.5cm} 
%   \centering
% \includegraphics[scale=.29]{figures/Var of l2l_2 loss.pdf}
%   \vspace{-0.2cm}
%   \caption{Results of GP}
% \label{fig:var-l2-gp}
%   \vspace{-0.1cm}
% \end{wrapfigure}


% Attempts have been made  in the past to address these  drawbacks heuristically  (see \citep{AggarwalKoGuHaPh14}). We give a unified computational framework while approaching the problem in a more principled manner and solving it more optimally.




\subsection{Planning with  neural network-based uncertainty quantification methods ($\ensembleplus$)}


We now provide a proof-of-concept that shows the generalizability of our conceptual framework  to the deep learning-based UQ modules, specifically focusing on $\ensembleplus$ due to their previously observed superior performance~\citep{OsbandWenAsDwIbLuRo23}. Recall that implementing our framework with deep learning-based UQ modules  requires us to retrain the model across multiple possible random actions $\bm{a}(\theta)$ sampled from the current policy $\pi_\theta$.
This requires significant computational resources, in sharp contrast to the GPs where the posteriors are in closed form and can be readily updated and differentiated. 

Due to the computational constraints, we test $\ensembleplus$ on a toy setting to demonstrate the generalizability of our framework. We consider a setting where the general population consists of four clusters, while the initial labeled data only comes from one cluster. Again we generate data using GPs.  The task is to select a batch of 2 points in one horizon. We detail the $\ensembleplus$ architecture in Section \ref{sec:details-experiments}, and we assume prior uncertainty to be large (depends on the scaling of the prior generating functions). 
The results are summarized in the Table~\ref{tab:UQ_ensemble}.

% \begin{table}[H]
% \vspace{-10pt}
% \caption{Performance under \ensembleplus as UQ module}
%     \centering
%     \begin{tabular}{|m{3cm}|m{2.5cm}|m{2cm}|} 
%     \hline
%       Algorithm   & Variance of $\loss_2$ loss estimate & Error of $\loss_2$ loss estimate  \\ \hline Random Sampling 
%          & $1710.9 \pm 1352.1$ & $8.67\pm6.62$ 
%       \\ \hline \ouralgo & $1.30 \pm 0.68$ & $0.91\pm0.25$ \\ \hline
%     \end{tabular}
%     \label{tab:UQ_ensemble}
%     %\vspace{-10pt}
% \end{table}




\begin{table}[h]
\vspace{-10pt}
\caption{Performance under \ensembleplus as the UQ module}
\centering
\begin{tabular}{|l|l|l|}
\hline
Algorithm   & Variance of $\loss_2$ loss estimate & Error of $\loss_2$ loss estimate  \\
\hline
\textsf{Random sampling} & 7129.8 $\pm$ 1027.0 & 136.2 $\pm$ 8.28 \\ \hline
\textsf{Uncertainty sampling (Static)} & 10852 $\pm$ 0.0 & 162.156 $\pm$ 0.0 \\ \hline
\textsf{Uncertainty sampling (Sequential)} & 8585.5 $\pm$ 898.9 & 144 $\pm$ 6.93 \\ \hline
\textsf{REINFORCE} & 1697.1 $\pm$ 0.0 & 45.27 $\pm$ 0.0 \\ \hline
\ouralgo & 1697.1 $\pm$ 0.0 & 45.27 $\pm$ 0.0 \\ \hline
\end{tabular}
%\caption{Comparison of different algorithms based on variance   and   error in $\ell_2$ loss estimation with Ensemble $+$ as the UQ module. Our results demonstrate that {\ouralgo} and REINFORCE outperformthe other active learning based heuristics, confirming the benefits of our MDP formulation for the adaptive labeling problem, as also demonstrated in Section 4.\\
%\footnotesize{Experimental details: We use Gaussian Processes as our data generating process, GP parameters are the same as in Section D.3.  The task is to select a batch of 2 points along one horizon.The marginal distribution $p_X$ has 4 \textit{non-overlapping} clusters. Initial data comes from one cluster, while pool and evaluation points comes from all the clusters. We have $20$ initial labeled data points, $10$ pool points, and $252$ evaluation points.  Training procedures are similar to the one in Section D.3.} }
\label{tab:UQ_ensemble}
\end{table}



% We faced  issues in scaling up these experiments which will be our focus in the future. 





% \begin{itemize}
%     \item Posteriors should be consistent. Two dimensions: even with less training,  
%     \item the inference should be  fast enough
% \end{itemize}


% Potential research directions for uncertainty quantification

% In this section we consider a simple setting We consider a simpler setting and 


% For synthetic dataset generation, we use ...... For real datasets, we use ...... We compare our methodolgy to several baselines ()    This Section is structured as follows:
% \begin{itemize}
%     \item \textbf{GPs, square loss objective} (Section \ref{}): 
%     %the broad aim of the experiments  in this section is to isolate the performance of our methodology without any concerns for the inefficiencies induced due to a mis-specified prior or imperfect posterior inference. To accomplish this we generate synthetic datasets using GPs (detailed later). We use the well specified prior (GPs - with same hyperparameter setting) as our UQ module.   
%      As GPs provide differentaible posterior inference - any errors induced due to imperfect posterior updates are also isolated. We note that under this setting
%      \item In Section\ref{} we demonstrate why our methodology performs better than other baselines - by devising various synthetic experiments ()
%     \item  \textbf{UQ Benchmarking }(Section \ref{}): Before diving into the experiments using $\ensembleplus$ and ENNs,  we showcase our benchmarking experiments in Section \ref{}. We use real datasets We observe that ENNs perform better
%      \item \textbf{Ensemble $+$}, objective: recall, accuracy
%     \item \textbf{ENN}, objective: recall, accuracy
% \end{itemize}




% In Section {}, we test 
% \subsection{Experimental details}

% \begin{itemize}
%     \item UQ methodologies - GPs, ENNs
%     \item Objectives - Recall,  ATE
%     \item Datasets - ATE-synthetic datasets, Recall-synthetic, real datasets
%     \item Baselines - 
%     \begin{itemize}
%         \item Random sampling
%         \item Active learning - Uncertainty based sampling - In regression setting almost all of the 
%         \item Myopic greedy - Greedy Batch based sampling
%         \item Policy Gradient
%     \end{itemize}
    
% \end{itemize}

% \subsection{Experiments}
%     \begin{itemize}
%     \item GPs with square loss
%     \item Benchmarking ENN
%         \item ENNs with ATE
%         \item ENNs with Recall
%     \end{itemize}

% \subsection{Benefits over other algorithms - intuition and experiments}

%Active learning - Myopic greedy / Don't rely on the objective rather some entropy version.


%%% Local Variables:
%%% mode: latex
%%% TeX-master: "main"
%%% End:


\section{Conclusion}
We have developed a novel posterior sampling scheme for denoising diffusion priors. The proposed algorithm proceeds by sequentially sampling, using a Gibbs sampler, from a sequence of mixture approximations of the smoothed posteriors. Our experiments show that \algo\ not only matches but often surpasses state-of-the-art performance and reconstruction quality across various tasks.
Furthermore, we have demonstrated that the Gibbs sampling perspective allows favorable performance improvement with inference-time compute scaling.

This work has certain limitations that open avenues for further exploration. While we outperform the state-of-the-art on most tasks and remains competitive overall on latent diffusion,  we still fall short of what we achieve with pixel-space diffusion. We believe that bridging this gap requires a more careful selection of the weight sequence. More broadly, an observation-driven approach to sampling the index could further enhance \algo. A second limitation is that our methodology does not extend to ODE-based samplers or DDIM, and adapting related ideas to these methods is an interesting research direction. Finally, like all existing methods relying on \eqref{eq:dps}, our approach incurs a higher memory cost compared to unconditional diffusion. It remains an open question whether the vector-Jacobian product can be eliminated without compromising performance.
% \clearpage


% In the unusual situation where you want a paper to appear in the
% references without citing it in the main text, use \nocite
\nocite{langley00}

\clearpage
\newpage
\section*{Acknowledgements}
The work of Y.J. and B.M. has been supported by Technology Innovation Institute (TII), project Fed2Learn. The work of Eric Moulines has been partly funded by the European Union (ERC-2022-SYG-OCEAN-101071601). Views and opinions expressed are however those of the author(s) only and do not necessarily reflect those of the European Union or the European Research Council Executive Agency. Neither the European Union nor the granting authority can be held responsible for them. This work was granted access to the HPC resources of IDRIS under the allocation 2025-AD011015980 made by GENCI.
\bibliography{bibliography.bib}
\bibliographystyle{icml2025}

\newpage
\appendix
\onecolumn
\section{Methodology details}
\subsection{Primer on Gibbs sampling}
\label{apdx-sec:gibbs}
In this section we lay out the basic properties of Gibbs sampling. We use measure-theoretic notation for conciseness. 

Let $\probmeas{0,1}{}{\rmd (\bx_0, \bx_1)}$ be a probability measure on $\rset^\dimx \times \rset^\dimx$. We denote by $\probmeas{0|1}{\bx_1}{\rmd \bx_0}$ and $\probmeas{1|0}{\bx_0}{\rmd \bx_1}$ the associated full conditionals and we write $\probmeas{0}{}{}$, $\probmeas{1}{}{}$ for its marginals. Define the %following 
transition kernels 
\begin{align*} 
    P_0(\rmd (\bx^\prime _0, \bx^\prime _1) | \bx_0, \bx_1) & \eqdef \probmeas{0|1}{\bx_1}{\rmd \bx^\prime _0} \delta_{\bx_1}(\rmd \bx^\prime _1),\\
    P_1(\rmd (\bx^\prime _0, \bx^\prime _1) | \bx_0, \bx_1) & \eqdef \probmeas{1|0}{\bx_0}{\rmd \bx^\prime _1} \delta_{\bx_0}(\rmd \bx^\prime _0).
\end{align*}
Each transition kernel updates only one coordinate at a time. A full update of the coordinates is obtained by composition of the kernels, \emph{i.e.}
$$ 
    P_0 P_1(\rmd (\bx^\prime _0, \bx^\prime _1) | \bx_0, \bx_1) \eqdef \int P_1(\rmd (\bx^\prime _0, \bx^\prime _1) | \tilde\bx_0, \tilde\bx_1) P_0(\rmd (\tilde\bx _0, \tilde\bx _1) | \bx_0, \bx_1) \eqsp.
$$  
 Each transition admits the joint distribution $\probmeas{0, 1}{}{}$ as stationary distribution, meaning that $\probmeas{0, 1}{}{\rmd (\bx_0, \bx_1)} = \int P_0(\rmd (\bx _0, \bx _1) | \bx^\prime _0, \bx^\prime _1)\, \probmeas{0, 1}{}{\rmd (\bx^\prime _0, \bx^\prime _1)}$. Indeed, this is seen by noting that 
\begin{align*}
    P_0(\rmd (\bx _0, \bx _1) | \bx^\prime _0, \bx^\prime _1) \probmeas{0, 1}{}{\rmd (\bx^\prime _0, \bx^\prime _1)}  & =  \probmeas{0|1}{\bx^\prime _1}{\rmd \bx _0} \delta_{\bx^\prime _1}(\rmd \bx _1) \probmeas{0, 1}{}{\rmd (\bx^\prime _0, \bx^\prime _1)}\\
    & =   \probmeas{0|1}{\bx^\prime _1}{\rmd \bx _0} \delta_{\bx^\prime _1}(\rmd \bx _1) \probmeas{0|1}{\bx^\prime _1}{\rmd \bx^\prime _0} \probmeas{1}{}{\rmd \bx^\prime _1}\\ 
    & = \probmeas{0|1}{\bx _1}{\rmd \bx _0} \probmeas{1}{}{\rmd \bx _1}  \probmeas{0|1}{\bx^\prime _1}{\rmd \bx^\prime _0}\delta_{\bx _1}(\rmd \bx^\prime _1),
\end{align*}
and then integrating both sides \wrt\ $(\bx^\prime _0, \bx^\prime _1)$. It then follows immediately that also $P_0 P_1$ admits $\probmeas{0,1}{}{}$ as stationary distribution. Letting $\big( (X^k _0, X^k _1) \big)_{k \in \nset}$ be a Markov chain with transition kernel $P_0 P_1$, the law of $(X^k _0, X^k _1)$ converges to $\probmeas{0,1}{}{}$ as $k \to \infty$ under mild conditions; see \cite{roberts1994simple}. 
\subsection{Full Gibbs conditionals}
\label{apdx-sec:conditionals}
In the main paper we consider the following data augmentation of the mixture $\hpost{t}{}{}$ \eqref{eq:posterior-approximation}
\begin{equation}
    \label{eq:extended-distr-normalized}
    \epost{0, s, t}{}{\bx_0, \bx_s, \bx_t} \\ = \pdata{0|s}{\bx_s}{\bx_0} \frac{\hpot{s}{\bx_s} \pdata{s|t}{\bx_t}{\bx_s} \pdata{t}{}{\bx_t}}{\int  \hpot{s}{\bx^\prime _s} \pdata{s|t}{\bx^\prime _t}{\bx^\prime _s} \pdata{t}{}{\bx^\prime _t} \, \rmd \bx_{s, t}}  \eqsp.
\end{equation}
From this definition it is straightforward to see that $\epost{0|s, t}{\bx_s, \bx_t}{\bx_0} = \pdata{0|s}{\bx_s}{\bx_0}$. In order to compute the full conditional $\epost{s|0, t}{\bx_0, \bx_t}{\bx_s}$ we use the identity 
\begin{equation} 
    \label{eq:bw-fw}
    \pdata{0|s}{\bx_s}{\bx_0} \pdata{s|t}{\bx_t}{\bx_s} \pdata{t}{}{\bx_t}= \pdata{0}{}{\bx_0} \fw{s|0}{\bx_0}{\bx_s} \fw{t|s}{\bx_s}{\bx_t},
\end{equation} 
from which it follows that 
\begin{align*} 
    \epost{s|0, t}{\bx_0, \bx_t}{\bx_s} & = \frac{\pdata{0|s}{\bx_s}{\bx_0} \hpot{s}{\bx_s} \pdata{s|t}{\bx_t}{\bx_s}}{\int \pdata{0|s}{\bx^\prime _s}{\bx_0} \hpot{s}{\bx^\prime _s} \pdata{s|t}{\bx_t}{\bx^\prime _s}\, \rmd \bx^\prime _s } \\
    & = \frac{ \fw{s|0}{\bx_0}{\bx_s} \hpot{s}{\bx_s} \fw{t|s}{\bx_s}{\bx_t}}{\int \fw{s|0}{\bx _0}{\bx^\prime _s} \hpot{s}{\bx^\prime _s} \fw{t|s}{\bx^\prime _s}{\bx _t}\, \rmd \bx^\prime _s } \\
    & = \frac{ \fw{s|0}{\bx_0}{\bx_s} \hpot{s}{\bx_s} \fw{t|s}{\bx_s}{\bx_t} \big/ \fw{t|0}{\bx_0}{\bx_t}}{\int \fw{s|0}{\bx _0}{\bx^\prime _s} \hpot{s}{\bx^\prime _s} \fw{t|s}{\bx^\prime _s}{\bx _t} \big/ \fw{t|0}{\bx_0}{\bx_t} \, \rmd \bx^\prime _s } \eqsp.
\end{align*}
Then, by noting that the bridge transition \eqref{eq:bridge} satisfies $\fw{s|0, t}{\bx_0, \bx_t}{\bx_s} = \fw{s|0}{\bx_0}{\bx_s} \fw{t|s}{\bx_s}{\bx_t} / \fw{t|0}{\bx_0}{\bx_t}$, we find that 
$$ 
\epost{s|0, t}{\bx_0, \bx_t}{\bx_s} = \frac{\hpot{s}{\bx_s} \fw{s|0, t}{\bx_0, \bx_t}{\bx_s}}{\int \hpot{s}{\bx^\prime _s} \fw{s|0, t}{\bx_0, \bx_t}{\bx^\prime _s} \, \rmd \bx^\prime _s}
$$
Finally, for the third conditional, using again the identity \eqref{eq:bw-fw}, we find that 
\begin{align*} 
    \epost{t|0, s}{\bx_0, \bx_s}{\bx_t} & = \frac{\pdata{0|s}{\bx_s}{\bx_0} \hpot{s}{\bx_s} \pdata{s|t}{\bx_t}{\bx_s} \pdata{t}{}{\bx_t}}{\int \pdata{0|s}{\bx _s}{\bx_0} \hpot{s}{\bx _s} \pdata{s|t}{\bx^\prime _t}{\bx _s} \pdata{t}{}{\bx^\prime _t}\, \rmd \bx^\prime _t } \\
    & = \fw{t|s}{\bx_s}{\bx_t} \eqsp.
\end{align*}
\subsection{Variational approximation}
\label{apdx-sec:vi-approx}
In this section we describe the variational approach of \citet{moufad2024variational}, which we use to fit a Gaussian variational approximation to $\post{s|0, t}{\bx_0, \bx_t}{}$ for fixed $(\bx_0, \bx_t)$. Similarly to the main paper we consider the variational approximation 
\begin{equation} 
    \smash{\vi{s|0, t}{} \eqdef \gauss\big(\vmu_{s|0, t}, \diag(\rme^{\vlstd_{s|0,t}})\big)}, 
    %\quad \mbox{and} \quad \vparam_{s|0, t} \eqdef (\vmu_{s|0, t}, \vlstd_{s|0,t}) \in \rset^{\dimx} \times \rset^{\dimx}. 
\end{equation}
and let $\vparam_{s|0, t} \eqdef (\vmu_{s|0, t}, \vlstd_{s|0,t}) \in \rset^{\dimx} \times \rset^{\dimx}$ denote the variational parameters. 
The reverse KL divergence writes, following definition \eqref{eq:bridge}, 
\begin{align} 
    \lefteqn{\kldivergence{\vi{s|0, t}{}}{\post{s|0, t}{\bx_0, \bx_t}{}}} \nonumber \\
    & \hspace{.7cm} = \int \log \frac{\vi{s|0, t}{\bx_s}}{\hpot{s}{\bx_s}{} \fw{s|0, t}{\bx_0, \bx_t}{\bx_s}} \, \vi{s|0, t}{\bx_s} \, \rmd \bx_s + \mathrm{C} \nonumber \\
    & \hspace{.7cm} = \pE _{\vi{s|0, t}{}} \left[ - \log \hpot{s}{\vX^\vparam _s} + \frac{\| \vX^\vparam _s - \big( \gamma_{t|s} \a_{s|0} \bx_0 + (1 - \gamma_{t|s}) \a^{-1} _{t|s} \bx_t \big) \|^2}{2 \std^2 _{s|0, t}} \right] - \frac{1}{2} \vlstd_{s|0, t}^T \mathbf{1}_d + \mathrm{C}^\prime. \label{eq:gradient-estimator} 
\end{align}
Using the reparameterization trick \cite{kingma2013auto} and plugging-in the neural network approximation $\hpot{s}{}[\param]$ of $\hpot{s}{}$, we obtain the gradient estimator  
\begin{multline*}
    \nabla_\vparam \mathcal{L}^s _t(\vparam; \bx_0, \bx_t, Z) \eqdef - \nabla_\vparam \log \hpot{s}{\vmu_{s|0, t} + \diag(\rme^{\vlstd_{s|0, t} })^{1/2} Z}[\param] \\
     + \nabla_\vparam \bigg[ \frac{\|\vmu_{s|0, t} + \diag(\rme^{\vlstd_{s|0, t} })^{1/2} Z - \big( \gamma_{t|s} \a_{s|0} \bx_0 + (1 - \gamma_{t|s}) \a^{-1} _{t|s} \bx_t \big)\|^2 }{2 \std^2 _{s|0, t}} - \frac{1}{2} \vlstd_{s|0, t}^T \mathbf{1}_d \bigg], 
\end{multline*}
where $Z \sim \gauss(\zero_\dimx, \Id_\dimx)$. We initialize the variational parameters with the mean and covariance of the bridge kernel \eqref{eq:bridge},  \emph{i.e.}, at initialization, $\vmu^0 _{s|0, t} \eqdef \gamma_{t|s} \a_{s|0} \bx_0 + (1 - \gamma_{t|s}) \a^{-1} _{t|s} \bx_t$ and $\vlstd^0 _{s|0, t} = \log \std^2 _{s|0, t} \Id_\dimx$.
The $\vifn$ routine is summarized in \Cref{algo:gauss_vi}. 
\begin{algorithm}
    \caption{$\vifn$ routine}
    \begin{algorithmic}[1]
        \STATE {\bfseries Input:} vectors $(\bx_0, \bx_t)$, timesteps $(s, t)$, gradient steps $G$
        \STATE $\vmu \gets \gamma_{t|s} \a_{s|0} \bx_0 + (1 - \gamma_{t|s}) \a^{-1} _{t|s} \bx_t$
        \STATE $\vlstd \gets \log \std^2 _{s|0, t}$

        \FOR{$g=1$ to $G$}
            \STATE $Z \sim \gauss(\zero_\dimx, \Id_\dimx)$
            \STATE $(\vmu, \vlstd) \gets \mathsf{OptimizerStep}(\nabla _\vparam \mathcal{L}^s _t(\cdot, \bx_0, \bx_t, Z))$
        \ENDFOR
        \STATE $Z \sim \gauss(\zero_\dimx, \Id_\dimx)$
        \STATE {\bfseries Output:} $\vmu + \diag(\rme^{\vlstd / 2}) Z$
    \end{algorithmic}
    \label{algo:gauss_vi}
\end{algorithm}
\begin{remark} 
    While the expectation of the squared norm in \eqref{eq:gradient-estimator} can be computed exactly, we found that, in practice, doing so degraded the algorithm’s performance, producing blurrier images compared to simply using a Monte Carlo estimator for the full expectation.
\end{remark} 
\begin{remark} 
    \label{rem:metropolis}
    The fact that the density of our target distribution can be computed approximately by plugging the denoiser approximation allows us to add a Metropolis--Hastings (MH) correction with approximate acceptance ratio. Indeed, once we fit the Gaussian approximation, we can improve the accuracy of our sampler by simulating a Markov chain $(\vX^k _s)_k$ where, given $\vX^k _s$, 
    $$ 
    \vX^{k+1} _s \sim M_s(\rmd \bx_s | \vX^k _s) \eqdef \int \vi{s|0,t}{z} \bigg[ r_s(\vX^k _s, z) \delta_{z}(\rmd \bx_s) + (1 - r_s(\vX^k _s, z) ) \delta_{\vX^k _s}(\rmd \bx_s)\bigg]\, \rmd z \eqsp,
    $$ 
    with
    $$ 
        r_s(\bx_s, \bx^* _s) = \mbox{min}\left(1, \frac{\hpot{s}{\bx^* _s} \fw{s|0, t}{\bx_0, \bx_t}{\bx^* _s} \vi{s|0,t}{\bx_s}}{\hpot{s}{\bx _s} \fw{s|0, t}{\bx_0, \bx_t}{\bx _s} \vi{s|0,t}{\bx^* _s}} \right) \eqsp.
    $$ 
\end{remark}
\subsection{Alternative data augmentation and sequence}
\label{apdx-sec:data-aug}
\paragraph{Data augmentation.} Our algorithm is based on one data-augmentation approach, but alternative augmentations could also be considered. 
Let $s \in \intset{1}{t-1}$. Then the most obvious and natural data augmentation involves simply marginalizing out the $\bx_0$ variable in \eqref{eq:extended-distr-normalized}, yielding 
$$ 
    \epost{s, t}{}{\bx_s, \bx_t} \propto \hpot{s}{\bx_s} \pdata{s|t}{\bx_t}{\bx_s} \pdata{t}{}{\bx_t} \eqsp.
$$ 
Its full conditionals are $\epost{s|t}{\bx_t}{\bx_s} \propto \hpot{s}{\bx_s} \pdata{s|t}{\bx_t}{\bx_s}$ and $\epost{t|s}{\bx_s}{\bx_t} = \fw{t|s}{\bx_s}{\bx_t}$. The first conditional is intractable for sampling, and we could approximate it with a Gaussian variational distribution, similar to our approach for $\epost{s|0, t}{\bx_0, \bx_t}{}$. Indeed, this is possible since $\nabla_{\bx_s} \log \epost{s|t}{\bx_t}{\bx_s} = \nabla_{\bx_s} \log \hpot{s}{\bx_s} + \nabla_{\bx_s} \log \pdata{s}{}{\bx_s} + \nabla_{\bx_s} \log \fw{t|s}{\bx_s}{\bx_t}$, which can then be approximated using the parametric approximations $\nabla \log \hpot{s}{\bx_s}[\param]$ and $\nabla \log \pdata{s}{}{\bx_s} \approx (- \bx_s + \a_s \denoiser{s}{}{\bx_s}[\param]) / (1 - \a^2 _s)$. 

The first drawback of this approach is that, in practice, it tends to degrade reconstruction quality---\emph{e.g.}, introducing blurriness---as $t$ tends to $0$, due to the poor approximation of the score near the data distribution. Additionally, beyond the loss of quality, we observe that it produces more incoherent reconstructions with noticeable artifacts. We hypothesize that this issue arises because the distribution we aim to approximately sample involves the prior transition $\pdata{s|t}{}{}$,  which can be highly multi-modal when $s \ll t$. This multi-modality may make the posterior $\epost{s|t}{\bx_t}{}$ more challenging to approximately sample from. On the other hand, when further conditioning on $\bx_0$, the sampling problem becomes more well-behaved, as we then target the posterior of a Gaussian distribution. Finally, while the score of $\epost{\smash{s|t}}{\bx_t}{\bx_s}$ can be easily approximated, its density cannot, preventing the use of a Metropolis--Hastings correction, unless we use the independent proposal $\pdata{s|t}{\bx_t}{}$. However, this approach is suboptimal, as it does not incorporate any information from the observation. This is not the case of the data-augmentation approach we use in \algo\ as we highlight in \Cref{rem:metropolis}. 
\paragraph{Alternative sequence.} An alternative to the mixture of posterior approximations \eqref{eq:posterior-approximation}, on which \algo\ is based, is the posterior formed as a mixture of likelihoods: 
$$ 
    \hpost{t}{}{\bx_t} = \frac{ \sum_{s = 1}^{t-1} \wght^s _t \hpot{t}{\bx_t}[s] \pdata{t}{}{\bx_t} }{\int \sum_{s = 1}^{t-1} \wght^s _t \hpot{t}{\bx^\prime _t}[s] \pdata{t}{}{\bx^\prime _t} \, \rmd \bx^\prime _t} \eqsp, 
$$ 
being the $\bx_t$-marginal of the extended distribution 
\begin{equation}
    \label{eq:mixture-pot-extended}
    \epost{0, \smbs, t}{}{s, \bx_0, \bz, \bx_t} \propto  \wght^s _t \pdata{0|s}{\bz}{\bx_0} \hpot{s}{\bz} \pdata{s|t}{\bx_t}{\bz} \pdata{t}{}{\bx_t} \eqsp.
\end{equation}
Now, let $(s, \bXy_0, \bZy, \bXy_t) \sim \epost{0, \smbs, t}{}{}$; then, conditionally on $s$, the distribution of $(\bXy_0, \bZy, \bXy_t)$ is $\epost{0, s, t}{}{}$, whereas 
$$
s | \bXy_0, \bZy, \bXy_t \, \sim \mbox{Categorical}\left( \left\{ \frac{\wght^\ell _t \hpot{\ell}{\bZy} \fw{\ell|0, t}{\bXy_0, \bXy_t}{\bZy}}{\sum_{k = 1}^{t-1} \wght^k _t \hpot{k}{\bZy} \fw{k|0, t}{\bXy_0, \bXy_t}{\bZy}}\right\}_{\ell = 1} ^{t-1} \right) \eqsp.
$$

A Gibbs sampler targeting \eqref{eq:mixture-pot-extended} is described in \Cref{algo:mixturepot-extended-gibbs}. It allows updating the index $s$ in an observation-driven fashion, but is unfortunately computationally expensive as we need to evaluate the denoiser at $\bZy$ in parallel for $t-1$ timesteps. A cheaper alternative could be to block the variables $(s, \bZy)$ and use an independent MH step to target their joint conditional distribution. Denoting by $\lambda$ the joint proposal distribution on $\intset{1}{t-1} \times \rset^\dimx$ used in this independent MH step, the probability of accepting a candidate $(s^*, \bz^*)$ is 
$$ 
    r_t\big( (s, \bz), (s^*, \bz^*)\big) = \mbox{min}\left( 1, \frac{\wght^{s^*} _{t} \hpot{s^*}{\bz^*}\fw{s^* | 0, t}{\bx_0, \bx_t}{\bz^*} \lambda(s, \bz)}{\wght^{s} _{t} \hpot{s}{\bz}\fw{s | 0, t}{\bx_0, \bx_t}{\bz} \lambda(s^{*}, \bz^{*})} \right) \eqsp.
$$ 
\begin{algorithm}[h]
    \caption{Gibbs sampler targeting \eqref{eq:extended}}
    % \small 
    \begin{algorithmic}[1]
        \STATE {\bfseries Input:} $(s^r, \bXy^r _0, \bZy^r, \bXy^r _t)$
        \STATE draw $s^{r+1} \sim \mbox{Categorical}\left( \left\{ \frac{\wght^\ell _t \hpot{\ell}{\bZy^r} \fw{\ell|0, t}{\bXy^r _0, \bXy^r _t}{\bZy^r}}{\sum_{k = 1}^{t-1} \wght^k _t \hpot{k}{\bZy^r} \fw{k|0, t}{\bXy^r _0, \bXy^r _t}{\bZy^r}}\right\}_{k = 1} ^{t-1} \right)$ 
        \STATE draw $\bZy^{r+1} \sim \epost{s^{r+1} |0, t}{\bXy^r _0, \bXy^r _t}{}$
        \STATE draw $\bXy^{r+1} _t \sim \fw{t|s^{r+1}}{\bZy^{r+1}}{}$ 
        \STATE draw $\bXy^{r+1} _0 \sim \pdata{0|s^{r+1}}{\bZy^{r+1}}{}$
    \end{algorithmic}
    \label{algo:mixturepot-extended-gibbs}
\end{algorithm}
\begin{remark} 
    \label{rem:mgdm-weight}
    Note that we could have used a similar data augmentation \eqref{eq:mixture-pot-extended} for the mixture used in \algo. This would yield the full conditional 
$$
    s | \bXy_0, \bZy, \bXy_t \, \sim \mbox{Categorical}\left( \left\{ \frac{\wght^\ell _t \epost{\ell|0, t}{\bXy_0, \bXy_t}{\bZy}}{\sum_{k = 1}^{t-1} \wght^k _t \epost{\ell|0, t}{\bXy_0, \bXy_t}{\bZy}}\right\}_{k = 1} ^{t-1} \right) \eqsp, 
$$
which is, however, intractable due to the normalizing constant involved in each $\epost{\ell|0, t}{}{}$. 
\end{remark}
\subsection{Related algorithms}
\label{apdx-sec:comparisons}
\paragraph{Comparison with \citet{zhang2024daps}} In this section we clarify the difference between \algo\ and the \daps\ algorithm \cite{zhang2024daps}, which shares some similarities with our approach. The sampling procedure in \daps\ relies on sequential approximate sampling from the joint distribution 
$$ 
    \tilde\pi^\obs _{0:T}(\bx_{0:T}) \eqdef \post{T}{}{\bx_T} \prod_{t = 0}^{T-1} \tilde{\pi}_{t|t+1}(\bx_t | \bx_{t+1}),  
$$ 
where 
\begin{equation}
    \label{eq:daps-bw}
    \tilde\pi^\obs _{t|t+1}(\bx_t | \bx_{t+1}) \eqdef \int \fw{t|0}{\bx_0}{\bx_t} \post{0|t+1}{\bx_{t+1}}{\bx_0} \, \rmd \bx_0
\end{equation}
and $\post{0|t+1}{\bx_{t+1}}{\bx_0} = \post{0}{}{\bx_0} \fw{t+1|0}{\bx_0}{\bx_{t+1}} \big/ \post{t+1}{}{\bx_{t+1}}$. From this definition it follows that 
$$ 
 \post{t}{}{\bx_t} = \int \tilde\pi^\obs _{t|t+1}(\bx_t | \bx_{t+1}) \post{t+1}{}{\bx_{t+1}} \, \rmd \bx_{t+1} \eqsp, 
$$ 
and hence that the marginals of the joint distribution $\tilde\pi^\obs _{0:T}$ are $(\post{t}{}{})_{t = 0}^T$. The canonical backward transition $\post{t|t+1}{\bx_{t+1}}{\bx_t} \propto \post{t}{}{\bx_t} \fw{t+1|t}{\bx_t}{\bx_{t+1}}$ has the alternative form 
$$ 
    \post{t|t+1}{\bx_{t+1}}{\bx_t} = \int \fw{t|0, t+1}{\bx_0, \bx_{t+1}}{\bx_t} \post{0|t+1}{\bx_{t+1}}{\bx_t} \, \rmd \bx_0 \eqsp,
$$
which differs from \eqref{eq:daps-bw} in the use of the bridge transition $q_{t|0, t+1}$ instead of the forward transition $q_{t|0}$. 
%and \eqref{eq:daps-bw} differs from it by the use of the forward transition $q_{t|0}$ in place of the bridge transition $q_{t|0, t+1}$. 

In order to sample from $\tilde\pi_{t|t+1}(\cdot | \bx_{t+1})$, one needs to first sample $X_0 \sim \post{0|t+1}{\bx_{t+1}}{}$ and then $X_t \sim \fw{t|0}{X_0}{}$. \daps\ performs the former step using Langevin dynamics on an approximation of $\post{0|t+1}{\bx_{t+1}}{}$. More specifically, the authors use the approximation 
$$ 
\post{0|t+1}{\bx_{t+1}}{\bx_0} \approx \frac{\pot{0}{\bx_0} \normpdf(\bx_0; \denoiser{t+1}{}{\bx_{t+1}}, r^2 _{t+1} \Id_\dimx)}{\int \pot{0}{\bx^\prime _0} \normpdf(\bx^\prime _0; \denoiser{t+1}{}{\bx_{t+1}}, r^2 _{t+1} \Id_\dimx) \, \rmd \bx^\prime _0} \eqsp,
$$ 
where $r^2 _{t+1}$ is a hyperparameter. This approximation follows by noting that $\post{0|t+1}{\bx_{t+1}}{\bx_0} \propto \pot{0}{\bx_0} \pdata{0|t+1}{\bx_{t+1}}{\bx_0}$ and using the Gaussian approximation of $\pdata{0|t+1}{\bx_{t+1}}{}$ proposed by \citet{song2022pseudoinverse}. The Langevin step is initialized with a sample obtained by discretizing the probability flow ODE \cite{song2021score} between $t+1$ and $0$. 

Both \algo\ and \daps\ perform full noising and denoising steps and operate in a similar manner in this respect (with the distinction that we use DDPM instead of the probability flow ODE). The first fundamental difference is that we sample, conditionally on \(\obs\) and at a random timestep $s$, by drawing from \(\epost{s|0, t}{\bx_0, \bx_t}{} \propto \hpot{s}{\bx_s} \fw{s|0, t}{\bx_0, \bx_t}{\bx_s}\). Unlike \daps, our method does not rely on a density approximation prior to applying an approximate sampler. The second main difference is the fact that within each denoising step, we can increase the number of Gibbs iterations to improve the overall performance, as demonstrated in \Cref{fig:scaling}. This is on top of the number of gradient steps that we use to fit the variational approximation and which enhance the performance when we increase them.  

On the other hand, \daps\ does not require the computation of vector-Jacobian products of the denoiser and is thus more efficient in terms of memory. However it requires many calls to the likelihood function, which can substantially increase the runtime if it is expensive to evaluate. For example, with a latent diffusion model, the runtime of DAPS is at least three times larger than that of \algo, \resample, and \psld. 
\paragraph{Comparison with \citet{moufad2024variational}}
The more recent {\sc{MGPS}} algorithm of \citet{moufad2024variational} is also related to \algo. Similarly to DAPS \cite{zhang2024daps}, their methodology relies on sampling approximately from the posterior transition $\post{t|t+1}{\bx_{t+1}}{}$ at each step of the backward denoising process. It builds on the following decomposition, which holds for all $s \in \intset{0}{t-1}$:
$$ 
    \post{t|t+1}{\bx_{t+1}}{\bx_t} = \int \fw{t|s, t+1}{\bx_s, \bx_{t+1}}{\bx_t} \post{s|t+1}{\bx_{t+1}}{\bx_s} \, \rmd \bx_s \eqsp.
$$ 
One step of {\sc{MGPS}} proceeds by first sampling from an approximation of the posterior transition $\post{s|t+1}{\bx_{t+1}}{}$ and then sampling from the bridge transition to return back to time $t$. The approximation of the posterior transition used in the {\sc{MGPS}} is 
\begin{equation} 
    \label{eq:mgps-approx}
    \post{s|t+1}{\bx_{t+1}}{\bx_s} \approx \frac{\hpot{s}{\bx_s}[\param] \pdata{s|t+1}{\bx_{t+1}}{\bx_s}[\param]}{\int \hpot{s}{\bx^\prime _s} \pdata{s|t+1}{\bx_{t+1}}{\bx^\prime _s}[\param] \, \rmd \bx^\prime _s} \eqsp.
\end{equation}
Here one can then choose $s$ to be sufficiently small to enhance the likelihood approximation, while still having an accurate Gaussian approximation of the transition $\pdata{s|t+1}{\bx_{t+1}}{}$. The authors demonstrate, using a solvable toy example, that this trade-off indeed exists; see \citep[Example 3.2]{moufad2024variational}. The approximate sampling step is then performed by fitting a Gaussian variational approximation to the approximation on the \rhs\ of \eqref{eq:mgps-approx}, similarly to what we do in \Cref{algo:midpoint-gibbs}.  

Both \algo\ and {\sc{MGPS}} leverage the same idea of using, at step time $t$, likelihood approximations at earlier steps $s < t$. While in {\sc{MGPS}} the time $s$ is set deterministically as a function of $t$, we sample it randomly. However, the main difference lies in the step where we sample conditionally on the observation $\obs$. Once the index $s$ is sampled we proceed with $R$ rounds of reverse KL minimization \wrt\ to a \emph{different} target distribution. Indeed, following \Cref{algo:midpoint-gibbs}, in the first round we seek to fit a distribution with density proportional to $\bx_s \mapsto \hpot{s}{\bx_s}[\param] \fw{s|0, t}{\textcolor{purple}{\vX^* _0}, \vX_t}{\bx_s}$, where $\vX^* _0$ is an output from the previous step of the algorithm. 
At step $r$, we fit $\bx_s \mapsto \hpot{s}{\bx_s}[\param] \fw{s|0, t}{\textcolor{purple}{\vX^{r-1} _0}, \vX^{r-1} _t}{\bx_s}$, where $\vX^{r-1} _0$ is sampled using a few DDPM steps starting from $\vX^{r-1} _s$ at time $s$ and $\vX^{r-1} _t \sim \fw{t|s}{\vX^{r-1} _s}{}$. On the other hand, {\sc{MGPS}} fits in a single round the distribution with density proportional to $\bx_s \mapsto \hpot{s}{\bx_s}[\param] \fw{s|0, t+1}{\textcolor{purple}{\denoiser{t+1}{}{\vX_{t+1}}[\param]}, \vX_{t+1}}{\bx_s}$, where $\vX_{t+1}$ is the output of the previous step. Finally, the authors report that the performance of {\sc{MGPS}} improves when the number of gradient steps is increased. In our case, we have two axes, Gibbs iterations $R$ and gradient steps, that allow us to improve the performance when more compute is available. 

% \section{Transition kernels involved in \Cref{algo:midpoint-gibbs}}
\label{apdx:transition-kernels}
% --- local vars ---
\def\gibbsRepKernel{{Q}}
\def\initKernel{{I}}
\def\gibbsKernel{{G}}
\def\lastKernel{{L}}
\def\algoKernel{{P}}

% \newcommandx{\gibbsRepKernel}[1][1=\midpoint]{{Q}_{#1}}
% \newcommandx{\gibbsKernel}[1][1=\midpoint]{{G}_{#1}}
% \newcommandx{\initKernel}[1][1=k]{{I}_{#1}}
% ---


The transition kernel of one repetition of a Gibbs step reads
\begin{equation}
    \label{eq:gibbs-kernel-one-rep}
    \gibbsRepKernel(\bx_{\initpoint}, \bx_k; \rmd \bx_{\initpoint}', \rmd \bx_k')
        = \int
            \mgibbs{\midpoint |\initpoint, k}{\bx_\initpoint, \bx_k}{\rmd \bx_{\midpoint}}
            \mgibbs{\initpoint |\midpoint}{\bx_{\midpoint}}{\rmd \bx_\initpoint'}
            \mgibbs{k |\midpoint}{\bx_{\midpoint}}{\rmd \bx_k'}
    \eqsp,
\end{equation}
The initial kernel of the Gibbs steps writes
\begin{equation}
    \label{eq:gibbs-init-kernel}
    \initKernel(\bx_\initpoint; \rmd \bx_\initpoint', \rmd \bx_k')
        = \delta_{\bx_\initpoint}(\rmd \bx_\initpoint') \fwtrans{k|0}{\bx_{\initpoint}}{\rmd \bx_k'}
    \eqsp.
\end{equation}
Combining these two results, we deduce the transition kernel of one Gibbs step after applying $\gibbsReps$ repetitions
\begin{equation}
    \label{eq:gibbs-step}
    \gibbsKernel(\bx_{\initpoint}; \rmd \bx_{\initpoint}^{(R)}, \rmd \bx_k^{(R)})
        = \int
            \initKernel(\bx_\initpoint; \rmd \bx_\initpoint^{(0)}, \rmd \bx_k^{(0)})
            \prod_{r=0}^{R-1}
            \gibbsRepKernel(\bx_{\initpoint}^{(r)}, \bx_k^{(r)}; \rmd \bx_{\initpoint}^{(r+1)}, \rmd \bx_k^{(r+1)})
    \eqsp,
\end{equation}
The last two steps of the algorithm involve the following kernel, equal up to a constant that depends only on $\bx_2$
\begin{equation}
    \label{eq:algo-last-kernel}
    \lastKernel(\bx_2; \rmd \bx_0)
        \propto\int \potn{}{\bx_1} \bw{1|2}{\bx_2}{\rmd \bx_1} \delta_{\hpredx{1}{(\bx_1)}}(\rmd \bx_0)
    \eqsp.
\end{equation}
Finally, we deduce the simulated distributionx
\begin{equation*}
    \pibw{0}{}{}[\gibbsReps](\rmd \bx_0)
    % \algoKernel(\bx_n; \rmd \bx_0)
        = \int
            \fwmarg{n}{\rmd \bx _n}
            \delta_{\hpredx{n}{(\bx_n)}}(\rmd \bx_0^{n})
            \big[
                \prod_{k=2}^{n-1} \gibbsKernel(\bx_0^{k+1}; \rmd \bx_0^{k}, \rmd \bx_{k})
            \big]
            \lastKernel(\bx_2; \rmd \bx_0)
    \eqsp.
\end{equation*}

% \section{Gaussian case}

% --- local vars ---
\def\cov{\mathbf{\Sigma}}
\def\mean{\boldsymbol{m}}
\def\likelihood{\bfA}
% \def\coefXell{a}
% \def\coefXs{c}
\def\covBridge{\sigma^2}
% \def\meanBridge{\tilde{\mean}}
\def\meanConditional{\hat{\mean}}
\def\potBias{\boldsymbol{a}}
\def\potLikelihood{\hat{\likelihood}}
\def\covPosterior{\mathbf{\Gamma}}
\def\matrixXk{\mathbf{M}}
\def\bfM{\mathbf{M}}
\def\hpimean{\hat{\boldsymbol{\mu}}}
\def\bfH{\mathbf{H}}
\def\bc{\boldsymbol{c}}

\def\LH{\mathbf{H}}
\def\Lbias{\boldsymbol{h}}
\def\Lcov{\mathbf{L}}
% \def\Lcovbefore{\tilde{\mathbf{L}}}

\def\LHbefore{\underline{\mathbf{H}}}
\def\Lbiasbefore{\underline{\boldsymbol{h}}}
\def\Lcovbefore{\underline{\mathbf{L}}}


\def\pizero{\mathbf{M}}
\def\pik{\mathbf{N}}
\def\picov{\mathbf{\Lambda}}
\def\pibias{\boldsymbol{e}}


\def\covtau{\mathbf{\Sigma}_{\bx}}
\def\meantau{\boldsymbol{\mu}_{\bx}}

\def\matzero{\mathbf{C}}
\def\biaszero{\boldsymbol{c}}
\def\covzero{\mathbf{\Sigma}_{c}}

\def\matk{\mathbf{D}}
\def\biask{\boldsymbol{d}}
\def\covk{\mathbf{\Sigma}_{d}}

\def\covthree{\mathbf{\Psi}}
\def\centeredxzero{\bar{\bx}_0'}
\def\centeredxk{\bar{\bx}_k'}


% no need for theta as superscript of the potential
\renewcommand{\hpotn}[2]{\ifthenelse{\equal{#2}{}}{\hat{g} _{#1}}{\hat{g} _{#1}(#2)}}
% ---

Recall that in this setting, the
\begin{itemize}
    \item prior is a Gaussian $\prior = \gauss(\mean, \cov)$, where $(\mean, \cov) \in \rset^\dimx \times \mathcal{S}^{++} _\dimx$
    \item potential is the likelihood of a linear inverse problem $\potn{}{}: \bx \mapsto \normpdf(\obs; \bfA \bx, \stdobs^2 \Id_\dimobs)$
\end{itemize}
% In the setting where the prior is a Gaussian $\prior = \gauss(\mean, \cov)$
% and the potential is the likelihood of a linear inverse problem $\potn{}{}: \bx \mapsto \normpdf(\obs; \bfA \bx, \stdobs^2 \Id_\dimobs)$,
In such a setting, all random variables involved in \Cref{algo:midpoint-gibbs} are Gaussians and hence the algorithm reduces to a sequence of means and covariances.
Depending on the number of gibbs repetitions $\gibbsReps$, \Cref{algo:midpoint-gibbs} simulates a final distribution $\pibw{0}{}{}[\gibbsReps]$.

In this section, we derive the recursion verified by the means and covariances of $\{ \pibw{k}{}{}[\gibbsReps] \}_{0 \leq k \leq n}$.
We proceed by:
% (i) writing the denoising densitites $\{ \bw{\initpoint|k}{}{} \}_{0 \leq k \leq n}$ and the approximate potentials $\{ \hpotn{k}{} \}_{0 \leq k \leq n}$ in the case of $\prior$ gaussian, (ii) deriving the transition kernels in \Cref{apdx:transition-kernels} in that case, (iii) combining the results to form the the recursion.
\begin{enumerate}
    \item writing the denoising densitites $\{ \bw{\initpoint|k}{}{} \}_{0 \leq k \leq n}$ and the approximate potentials $\{ \hpotn{k}{} \}_{0 \leq k \leq n}$
    \item deriving the transition kernels in \Cref{apdx:transition-kernels} in that case
    \item combining the results to form the the recursion
\end{enumerate}


\paragraph{Denoising densities and potentials.} Since we are dealing with a Gaussian prior, the denoiser $\predx{k}$ can be computed in closed form for any $k \in \intset{1}{n}$. Using \citet[Eqn.~2.116]{bishop2006pattern}, we have that
    \begin{align*}
        \bw{0|k}{\bx _k}{\bx _0}
            & \propto \prior(\bx _0) \fwtrans{k|0}{\bx _0}{\bx _k} \\
            & = \normpdf \big(\bx _0;
                \cov_{0|k} \big( (\sqrt{\acp{k}} / \var_k) \bx _k + \cov^{-1} \mean \big), \cov_{0|k} \big),
    \end{align*}
where $\cov_{0|k} \eqdef ((\acp{k} / \var_k) \Id + \cov^{-1})^{-1}$.
A byproduct is an expression of the denoiser
\begin{equation*}
    \predx{k}(\bx _k) = \cov_{0|k}\big( (\sqrt{\acp{k}} / \var_k) \bx _k + \cov^{-1} \mean \big)
    \eqsp.
\end{equation*}

The approximate potentials $\hpotn{k}{} = \potn{k}{} \circ \predx{k}$ also have an explicit expression that reads
\begin{equation*}
    \hpotn{k}{\bx _k}
        = \normpdf(\obs; \hat\bfA_k \bx _k + \potBias_k, \stdobs^2 \Id_\dimobs),
    \quad
    \mathrm{where}
    \quad
    \potLikelihood_k = (\sqrt{\acp{k}} / \var_k)  \likelihood \cov_{0|k}, \quad \potBias_k = \likelihood \cov_{0|k} \cov^{-1} \mean
    \eqsp.
\end{equation*}


\paragraph{Transition kernels.}
The expression of the initial kernel \eqref{eq:gibbs-init-kernel} writes
\begin{equation*}
    % \label{eq:init-kernel-gauss}
    \initKernel(\bx_\initpoint; \rmd \bx_\initpoint', \rmd \bx_k')
        = \delta_{\bx_\initpoint}(\rmd \bx_\initpoint')
            \ \normpdf(\rmd \bx_k'; \sqrt{\acp{k}} \bx_0, \var_{k} \Id_\dimx)
    \eqsp.
\end{equation*}
This can be translated into the following linear system of random variables
\begin{align*}
    \bX_0' 
        & = \bX_0,
        \\
    \bX_k' 
        & = \sqrt{\acp{k}} \bX_0 + \sqrt{\var_k} \bZ, 
        \qquad \bZ \sim \normpdf(\zero, \Id_\dimx).
\end{align*}
Notice here that $\bX_0'$ and $\bX_k'$ of the next state are interlinked
$\pCov[\bX_0', \bX_k'] = \sqrt{\acp{k}} \pV[\bX_0]$, an that
\begin{equation}
    \label{eq:gibbs-init-kernel-gauss-update}
    \pE\begin{bmatrix}\bX_0'\\\bX_k'\end{bmatrix} = \begin{bmatrix}\pE[\bX_0]\\\sqrt{\acp{k}}\pE[\bX_0]\end{bmatrix},
    \quad
    \pCov\begin{bmatrix}\bX_0'\\\bX_k'\end{bmatrix} = \begin{bmatrix}\pV[\bX_0] & \sqrt{\acp{k}}\pV[\bX_0]\\\cdot & \var_{k} \Id_\dimx + \acp{k} \pV[\bX_0]\end{bmatrix}
    \eqsp.
\end{equation}

It is easy to see that \Cref{eq:gibbs-kernel-one-rep} can also be expressed as a linear system of random variables
% \begin{equation}
%     \label{eq:gibbs-kernel-one-rep-gauss}
%     \gibbsRepKernel(\bx_{\initpoint}, \bx_k; \rmd \bx_{\initpoint}', \rmd \bx_k')
%         = \normpdf(\rmd \bx_{\initpoint}';
%             \QMzero_k \bx_0 + \QNzero_k \bx_k + \Qbiaszero_k, \Qcovzero_k) \
%             \normpdf(\rmd \bx_k'; \QMk_k \bx_0 + \QNk_k \bx_k + \Qbiask_k, \Qcovk_k)
%     \eqsp.
% \end{equation}
% \begin{align*}
%     \bX_0' 
%         & = \QMzero_k \bX_0 + \QNzero_k \bX_k + \Qbiaszero_k + \Qcovzero_k^{1/2} \bZ_0,
%         \qquad \bZ_0 \sim \normpdf(\zero, \Id),
%         \\
%     \bX_k' 
%         & = \QMk_k \bX_0 + \QNk_k \bX_k + \Qbiask_k + \Qcovk_k^{1/2} \bZ_k, 
%         \qquad \bZ_k \sim \normpdf(\zero, \Id).
% \end{align*}
\begin{equation*}
    \begin{bmatrix}\bX_0'\\\bX_k'\end{bmatrix}
    =
    \boldsymbol{b}_k
    + 
    \mathbf{B}_k \begin{bmatrix}\bX_0\\\bX_k\end{bmatrix}
    +
    \mathbf{\Gamma}^{1/2}_k \bZ_{0,k}
    , \quad
    \bZ_{0,k} \sim \normpdf(\zero_{2\dimx}, \Id_{2\dimx})
    \eqsp.
\end{equation*}
For clarity, we defer the expressions of the introduced matrices $\mathbf{B}_k, \mathbf{\Gamma}_k$, and the bias $\boldsymbol{b}_k$ to \Cref{apdx:details-expressions-gauss}.
Based on the values of expected value of the vector $[\bX_0, \bX_k]$ and its covariance, we can deduce the expected value and the covariances of the next state $\bX_0', \bX_k'$ as follow
\begin{equation}
    \label{eq:gibbs-kernel-one-rep-gauss}
    \pE\begin{bmatrix}\bX_0'\\\bX_k'\end{bmatrix}
        =
        \boldsymbol{b}_k
        + 
        \mathbf{B}_k \ \pE\begin{bmatrix}\bX_0\\\bX_k\end{bmatrix},
    \quad
    \pCov\begin{bmatrix}\bX_0'\\\bX_k'\end{bmatrix}
        = 
        \mathbf{B}_k \pCov\begin{bmatrix}\bX_0\\\bX_k\end{bmatrix} \mathbf{B}_k^\top + \mathbf{\Gamma}_k
    \eqsp.
\end{equation}
% For clarity, we defer the expressions of the introduced matrices $\QMzero_k, \QNzero_k, \Qcovzero_k$, and $\QMk_k, \QNk_k, \Qcovk_k$, and the biases $\Qbiaszero_k, \Qbiask_k$ to \Cref{apdx:details-expressions-gauss}.
% The expression of the initial kernel \eqref{eq:gibbs-init-kernel} writes
% \begin{equation}
%     \label{eq:init-kernel-gauss}
%     \initKernel(\bx_\initpoint; \rmd \bx_\initpoint', \rmd \bx_k')
%         = \delta_{\bx_\initpoint}(\rmd \bx_\initpoint')
%             \ \normpdf(\rmd \bx_k'; \sqrt{\acp{k}} \bx_0, \var_{k} \Id_\dimx)
%     \eqsp.
% \end{equation}
Finally, the last kernel \eqref{eq:algo-last-kernel} is a Gaussian with a linear mean on $\bx_2$
\begin{equation*}
    \lastKernel(\bx_2; \rmd \bx_0') = \normpdf(\rmd \bx_0'; \LH \bx_2 + \Lbias, \Lcov)
    \eqsp.
\end{equation*}
Also here for clarity, we defer the expressions of the introduced matrices $\LH, \Lcov$ and the bias  $\Lbias$ to \Cref{apdx:details-expressions-gauss}.
Therefore, we deduce the updates
\begin{equation}
    \label{eq:last-kernel-gauss}
    \pE[\bX_0'] = \LH \pE[\bX_2] + \Lbias,
    \quad
    \pV[\bX_0'] = \LH \pV[\bX_2] \LH^\top + \Lcov
    \eqsp.
\end{equation}

\paragraph{Recursion.}
By combining \eqref{eq:gibbs-init-kernel-gauss-update}, \eqref{eq:gibbs-kernel-one-rep-gauss}, and \eqref{eq:last-kernel-gauss}, \Cref{algo:midpoint-gibbs} reduces to the following recursion
% \badr{here the init $\delta_m(\rmd \bx_0^n)$ is valid if we were in the perfect case where the diffusion converges to a Gaussian}
% \begin{equation*}
%     \begin{aligned}
%         \pE[\bX_0^{(0)}] = \bx_0, \pV[\bX_0^{(0)}] = \zero,
%             \quad & \quad
%         \pE[\bX_k^{(0)}] = \sqrt{\acp{k}} \bx_0, \pV[\bX_k^{(0)}] = \var_{k} \Id_\dimx, \\
%         \pE[\bX_0^{(r+1)}] = \QMzero_k \pE[\bX_0^{(r)}] + \QNzero_k \pE[\bX_k^{(r)}],
%             \quad & \quad
%         \pV[\bX_0^{(r+1)}] = \Qcovzero_k + \QMzero_k  \pV[\bX_0^{(r)}] \QMzero_k^\top + \QNzero_k \pV[\bX_k^{(r)}] \QNzero_k^\top,
%         \\
%         \pE[\bX_k^{(r+1)}] = \QMk_k \pE[\bX_0^{(r)}] + \QNk_k \pE[\bX_k^{(r)}],
%             \quad & \quad
%         \pV[\bX_k^{(r+1)}] = \Qcovk_k + \QMk_k \pV[\bX_0^{(r)}] \QMk_k^\top + \QNk_k \pV[\bX_k^{(r)}] \QNk_k^\top,
%     \end{aligned}
% \end{equation*}
\begin{algorithm}[h]
    \caption{\Cref{algo:midpoint-gibbs} in the Gaussian case}
    \begin{algorithmic}[1]
        \STATE $\pE[\bX_{\initpoint}^n] \gets \mean,
            \quad
            \pV[\bX_{\initpoint}^n] \gets \zero$
        \FOR{$k=n-1$ to $2$}
            % \STATE {\bfseries Pick:} $\midpoint \sim \uniform(\intset{1}{k-1})$
            % \STATE {\bfseries Init:} $\bX^{(0)}_k \sim \fwtrans{k|0}{\bX_{\initpoint}}{\cdot}, \, \bX^{(0)}_{\initpoint} \gets \bX_{\initpoint}$

            % \STATE $\pE[\bX_0^{(0)}] \gets \pE[\bX_0^{k+1}],
            % \quad
            % \pV[\bX_0^{(0)}] \gets  \pV[\bX_0^{k+1}]$
            % \STATE $\pE[\bX_k^{(0)}] \gets \sqrt{\acp{k}} \ \pE[\bX_0^{k+1}],
            % \quad
            % \pV[\bX_k^{(0)}] \gets \var_{k} \Id_\dimx + \acp{k} \pV[\bX_0^{k+1}]$
            % \STATE $\pCov[\bX_0^{(0)}, \bX_k^{(0)}] \gets \sqrt{\acp{k}} \pV[\bX_0^{k+1}]$
            
            \STATE $\pE\begin{bmatrix}\bX_0^{(0)}\\\bX_k^{(0)}\end{bmatrix} \gets \begin{bmatrix}\pE[\bX_0^{k+1}]\\\sqrt{\acp{k}}\pE[\bX_0^{k+1}]\end{bmatrix},$ \qquad
            $\pCov\begin{bmatrix}\bX_0^{(0)}\\\bX_k^{(0)}\end{bmatrix} \gets \begin{bmatrix}\pV[\bX_0^{k+1}] & \sqrt{\acp{k}}\pV[\bX_0^{k+1}]\\\cdot & \var_{k} \Id_\dimx + \acp{k} \pV[\bX_0^{k+1}]\end{bmatrix}$

            \FOR{$r = 0$ {\bfseries to} $R-1$}
                % \STATE $\pE[\bX_0^{(r+1)}] \gets \QMzero_k \pE[\bX_0^{(r)}] + \QNzero_k \pE[\bX_k^{(r)}] + \Qbiaszero_k,
                % \quad
                % \pV[\bX_0^{(r+1)}] \gets \Qcovzero_k + \QMzero_k  \pV[\bX_0^{(r)}] \QMzero_k^\top + \QNzero_k \pV[\bX_k^{(r)}] \QNzero_k^\top$
                % \STATE $\pE[\bX_k^{(r+1)}] \gets \QMk_k \pE[\bX_0^{(r)}] + \QNk_k \pE[\bX_k^{(r)}] + \Qbiask_k,
                % \quad
                % \pV[\bX_k^{(r+1)}] \gets \Qcovk_k + \QMk_k \pV[\bX_0^{(r)}] \QMk_k^\top + \QNk_k \pV[\bX_k^{(r)}] \QNk_k^\top$
                % \STATE \emph{Apply \Cref{eq:gibbs-kernel-one-rep-gauss} starting from $\pE[\bX_0^{(r)}], \pE[\bX_k^{(r)}], \pV[\bX_0^{(r)}], \pV[\bX_k^{(r)}]$ and $\pCov[\bX_0^{(r)}, \bX_k^{(r)}]$}
                % \STATE \emph{to compute $\pE[\bX_0^{(r+1)}], \pE[\bX_k^{(r+1)}], \pV[\bX_0^{(r+1)}], \pV[\bX_k^{(r+1)}]$ and $\pCov[\bX_0^{(r+1)}, \bX_k^{(r+1)}]$}
                
                \STATE $\pE\begin{bmatrix}\bX_0^{(r+1)}\\\bX_k^{(r+1)}\end{bmatrix}
                \gets
                \boldsymbol{b}_k
                + 
                \mathbf{B}_k \ \pE\begin{bmatrix}\bX_0^{(r)}\\\bX_k^{(r)}\end{bmatrix}$,
                \qquad
                $\pCov\begin{bmatrix}\bX_0^{(r+1)}\\\bX_k^{(r+1)}\end{bmatrix}
                \gets 
                \mathbf{B}_k \pCov\begin{bmatrix}\bX_0^{(r)}\\\bX_k^{(r)}\end{bmatrix} \mathbf{B}_k^\top + \mathbf{\Gamma}_k$
                
            \ENDFOR
            \STATE $\pE[\bX_{\initpoint}^k] \gets \pE[\bX^{(R)}_{\initpoint}],
            \quad
            \pV[\bX_{\initpoint}^k] \gets \pV[\bX^{(R)}_{\initpoint}]$
            \STATE $\pE[\bX_k] \gets \pE[\bX^{(R)}_k],
            \quad
            \pV[\bX_k] \gets \pV[\bX^{(R)}_k]$
        \ENDFOR

        \STATE $\pE[\bX_0] \gets \LH \ \pE[\bX_2] + \Lbias
        \quad
        \pV[\bX_0] \gets \Lcov + \LH \ \pV[\bX_2] \ \LH^\top$
    \end{algorithmic}
\end{algorithm}


\paragraph{Expressions of the introduced matrices and biases.}
\label{apdx:details-expressions-gauss}
To derive the expression, we utilize \citet[Eqn.~2.116]{bishop2006pattern} and the convolution of two Gaussians.
Let us start we the kernel $\gibbsRepKernel$ \eqref{eq:gibbs-kernel-one-rep-gauss}.
We have
\begin{equation*}
    \begin{aligned}
        \mgibbs{\midpoint |\initpoint, k}{\bx_\initpoint, \bx_k}{\bx_{\midpoint}}
            & \propto \hpotn{\midpoint}{\bx _\midpoint} \fwtrans{\midpoint |\initpoint, k}{\bx_\initpoint, \bx_k}{\bx_{\midpoint}}\\
            & \propto \normpdf(\obs; \hat\bfA_\midpoint \bx _\midpoint + \potBias_\midpoint, \stdobs^2 \Id_\dimobs)
                \ \normpdf(\bx_{\midpoint}; \meanBridge_{\midpoint|0, k}(\bx _0, \bx _k), \var_{\midpoint|0, k} \Id_\dimx) \\
            & = \normpdf(\bx_\tau;
                \pizero_{\midpoint| 0, k} \bx_0 + \pik_{\midpoint| 0, k} \bx_k + \pibias_{\midpoint| 0, k},
                % var
                \picov_{\midpoint| 0, k})
    \end{aligned}
\end{equation*}
where
\begin{align*}
    \picov_{\midpoint|0, k}
        & = \big[ (1 /\var_{\midpoint | 0, k}) \Id_\dimx  + (1 / \stdobs^2)\hat\bfA_\midpoint^\top \hat\bfA_\midpoint \big]^{-1},\\
    \pizero_{\midpoint| 0, k}
        & = \frac{\sqrt{\acp{\midpoint}{} / \acp{0}}(1 - \acp{k} / \acp{\midpoint})}{\var_{\midpoint|0, k} (1 - \acp{k} / \acp{0})} \picov_{\midpoint| 0, k}
    ,\qquad
    \pik_{\midpoint| 0, k}
        = \frac{\sqrt{\acp{k} / \acp{\midpoint}} (1 - \acp{\midpoint} / \acp{0})}{\var_{\midpoint|0, k} (1 - \acp{k} / \acp{0})}  \picov_{\midpoint| 0, k}, \\
    \pibias_{\midpoint| 0, k}
        & = (1/\stdobs^2) \picov_{\midpoint| 0, k} \hat\bfA_\midpoint^\top (\obs - \potBias_\midpoint).\\
\end{align*}
After establishing the the distribution $\mgibbs{\midpoint |\initpoint, k}{}{}$, we can now compute the action of the kernel $\gibbsRepKernel$ on $\bX_0, \bX_k$.
Without lost of generality, we can write the integral involved in $\gibbsRepKernel$ as
\begin{align*}
    \int
        \mgibbs{\midpoint |\initpoint, k}{\bx_\initpoint, \bx_k}{\bx_{\midpoint}}
        \mgibbs{\initpoint |\midpoint}{\bx_{\midpoint}}{\bx_\initpoint'}
        \mgibbs{k |\midpoint}{\bx_{\midpoint}}{\bx_k'} \rmd \bx_\tau
        & = \int
            \normpdf(\bx_\tau; \meantau, \covtau)
            \normpdf(\bx_0'; \matzero \bx_\tau + \biaszero, \covzero)
            \normpdf(\bx_k'; \matk \bx_\tau + \biask, \covk)
            \rmd \bx_\tau
\end{align*}
where
\begin{align*}
    \meantau = \pizero_{\midpoint| 0, k} \bx_0 + \pik_{\midpoint| 0, k} \bx_k + \pibias_{\midpoint| 0, k}, \quad
    \covtau = \picov_{\midpoint|0, k},
    \\
    \matzero = (\sqrt{\acp{\tau}}/\var_{\tau}) \cov_{0|\tau}, \quad 
    \biaszero = \cov_{0|\tau} \cov^{-1} \mean, \quad
    \covzero = \cov_{0 | \tau},
    \\
    \matk = \sqrt{\acp{k}/\acp{\tau}} \Id, \quad 
    \biask = \zero, \quad
    \covk = \var_{k|\tau} \Id.
\end{align*}
From now on, we can focus only the quadratic product in the Gaussians exponent.
We have
\begin{multline*}
    \| \bx_\tau - \meantau \|^2_{\covtau^{-1}}
    +  \| \bx_0' - (\matzero \bx_\tau + \biaszero) \|^2_{\covzero^{-1}}
    + \| \bx_k' - (\matk \bx_\tau + \biask) \|^2_{\covk^{-1}}
    = 
    \\
    \| \bx_\tau \|^2_{\covtau^{-1} + \matzero^\top \covzero^{-1} \matzero + \matk^\top \covk^{-1} \matk}
    \\
    - 2 \langle \covtau^{-1} \meantau + \matzero^\top \covzero^{-1} (\bx_0'- \biaszero)+ \matk^\top \covk^{-1} (\bx_k'- \biask), \bx_\tau  \rangle
    \\
    + \| \meantau \|^2_{\covtau^{-1}} + \| \bx_0' - \biaszero \|^2_{\covzero^{-1}} + \| \bx_k' - \biask \|^2_{\covk^{-1}}
\end{multline*}
For the sake of conciseness, denote by
\begin{align*}
    \covthree^{-1} 
        & = \covtau^{-1} + \matzero^\top \covzero^{-1} \matzero + \matk^\top \covk^{-1} \matk,
        \\
    \centeredxzero & = \bx_0' - \biaszero, \qquad
    \centeredxk  = \bx_k' - \biask.
\end{align*}
The previous quadratic sum equal up to a constant independent of $\bx_\tau, \centeredxzero, \centeredxk$, to
\begin{align*}
    \| \bx_\tau - \covthree (\covtau^{-1} \meantau + \matzero^\top \covzero^{-1} \centeredxzero + \matk^\top \covk^{-1} \centeredxk) \|^2_{\covthree^{-1}}
    + \| \centeredxzero \|^2_{\covzero^{-1}} + \| \centeredxk \|^2_{\covk^{-1}}
    - \| \covtau^{-1} \meantau + \matzero^\top \covzero^{-1} \centeredxzero + \matk^\top \covk^{-1} \centeredxk \|^2_{\covthree}
    % \eqsp.
\end{align*}
The first term will define a Gaussian on $\bx_\tau$ and hence will integrate to one.
Hence, we get ride of the integration and will be left three terms.
The goal is to write them in form amenable to Gaussians.
Building up the previous expression, we have
\begin{multline*}
    \| \centeredxzero \|^2_{\covzero^{-1}} + \| \centeredxk \|^2_{\covk^{-1}}
    - \| \covtau^{-1} \meantau + \matzero^\top \covzero^{-1} \centeredxzero + \matk^\top \covk^{-1} \centeredxk \|^2_{\covthree}
    = \\
    \| \centeredxzero \|^2_{\covzero^{-1} - \covzero^{-1} \matzero \covthree \matzero^\top \covzero^{-1}} 
    + \| \centeredxk \|^2_{\covk^{-1} - \covk^{-1} \matk \covthree \matk^\top \covk^{-1}}
    \\
    - 2 \langle \matzero^\top \covzero^{-1} \centeredxzero, \matk^\top \covk^{-1} \centeredxk \rangle_{\covthree}
    - 2 \langle \covtau^{-1} \meantau, \matzero^\top \covzero^{-1} \centeredxzero \rangle_{\covthree}
    - 2 \langle \covtau^{-1} \meantau, \matk^\top \covk^{-1} \centeredxk \rangle_{\covthree}
\end{multline*}
where the equality is up to a constant independent of $\centeredxzero, \centeredxk$.

Now let us introduce the block matrices
\begin{equation*}
    \begin{aligned}
        \mathbf{\Gamma}^{-1} & = \begin{bmatrix}
            \covzero^{-1} - \covzero^{-1} \matzero \covthree \matzero^\top \covzero^{-1} &  -\covzero^{-1} \matzero \covthree \matk^\top \covk^{-1} \\
            \cdot & \covk^{-1} - \covk^{-1} \matk \covthree \matk^\top \covk^{-1}
        \end{bmatrix}
        \\
        \mathbf{J} & = \begin{bmatrix}
            \covzero^{-1} \matzero \covthree \covtau^{-1} & \zero \\
            \zero &  \covk^{-1} \matk \covthree \covtau^{-1} 
        \end{bmatrix}
    \end{aligned}
    \eqsp,
\end{equation*}
where the "$\cdot$" in the the expression of $\mathbf{\Gamma}^{-1}$ stands for the transpose of the off-diagonal element.
It follows that we can rewrite the previous equation in blocks as
\begin{align*}
    \begin{bmatrix}\centeredxzero \\ \centeredxk \end{bmatrix}^\top
    \mathbf{\Gamma}^{-1}
    \begin{bmatrix}\centeredxzero \\ \centeredxk \end{bmatrix}
    - 2 \langle
        \mathbf{J} \begin{bmatrix}\meantau \\ \meantau \end{bmatrix}
        ,
        \begin{bmatrix}\centeredxzero \\ \centeredxk \end{bmatrix}
    \rangle
    = 
    \| 
        \begin{bmatrix}\centeredxzero \\ \centeredxk \end{bmatrix} 
        -
        \mathbf{\Gamma} \mathbf{J} \begin{bmatrix}\meantau \\ \meantau \end{bmatrix}
    \|_{\mathbf{\Gamma}^{-1}}^2
    \eqsp,
\end{align*}
where the equality is up to a constant independent of $\centeredxzero, \centeredxk$.
Therefore, we can deduce that the random vector $[\bX_0', \bX_k']^\top$ is Gaussian as follows
\begin{equation*}
    \begin{aligned}
       \begin{bmatrix}\bX_0'\\\bX_k'\end{bmatrix}
       \sim
       \normpdf(
            \begin{bmatrix}\biaszero\\\biask\end{bmatrix}
            + \mathbf{\Gamma} \mathbf{J} \begin{bmatrix}\meantau \\ \meantau\end{bmatrix}
            ,
            \mathbf{\Gamma}
       )
    \end{aligned}
\end{equation*}
We can now intervene $\bX_0, \bX_k$ by subtituting with the value of $\meantau$
\begin{align*}
    \begin{bmatrix}\meantau \\ \meantau\end{bmatrix}
        = \begin{bmatrix}
            \pizero_{\midpoint| 0, k} & \pik_{\midpoint| 0, k}\\ 
            \pizero_{\midpoint| 0, k} & \pik_{\midpoint| 0, k} 
        \end{bmatrix} 
        \begin{bmatrix}\bX_0 \\ \bX_k\end{bmatrix}
        +
        \begin{bmatrix}\pibias_{\midpoint| 0, k} \\ \pibias_{\midpoint| 0, k}\end{bmatrix}
        = \mathbf{K} \begin{bmatrix}\bX_0 \\ \bX_k\end{bmatrix} + \begin{bmatrix}\pibias_{\midpoint| 0, k} \\ \pibias_{\midpoint| 0, k}\end{bmatrix}
\end{align*}
and therefore, we can write
\begin{equation*}
    \begin{aligned}
       \begin{bmatrix}\bX_0'\\\bX_k'\end{bmatrix}
       =
        \begin{bmatrix}\biaszero\\\biask\end{bmatrix}
        + 
        \mathbf{\Gamma} \mathbf{J} \begin{bmatrix}\pibias_{\midpoint| 0, k} \\ \pibias_{\midpoint| 0, k}\end{bmatrix}
        + 
        \mathbf{\Gamma} \mathbf{J} \mathbf{K} \begin{bmatrix}\bX_0\\\bX_k\end{bmatrix}
        +
        \mathbf{\Gamma}^{1/2} \bZ
    \end{aligned}
\end{equation*}
This enables us to identify the quantities
\begin{equation*}
    \boldsymbol{b}_k = \begin{bmatrix}\biaszero\\\biask\end{bmatrix}
        + 
        \mathbf{\Gamma} \mathbf{J} \begin{bmatrix}\pibias_{\midpoint| 0, k} \\ \pibias_{\midpoint| 0, k}\end{bmatrix},
    \quad
    \mathbf{B}_k = \mathbf{\Gamma} \mathbf{J} \mathbf{K},
    \quad
    \mathbf{\Gamma}_k = \mathbf{\Gamma}
    \eqsp.
\end{equation*}

% Therefore, we deduce the expression of  $\QMzero_k, \QNzero_k, \Qcovzero_k$, and $\Qbiaszero_k$

% \begin{align*}
%     \Qcovzero_k
%         & = \cov_{0|\tau} + (\acp{\tau}/\var_{\tau}^2) \cov_{0|\tau} \picov_{\midpoint|0,k}\cov_{0|\tau}, \\
%     \QMzero_k
%         & = (\sqrt{\acp{\tau}}/\var_{\tau}) \cov_{0|\tau} \pizero_{\midpoint| 0, k},
%     \qquad
%     \QNzero_k
%         = (\sqrt{\acp{\tau}}/\var_{\tau}) \cov_{0|\tau} \pik_{\midpoint| 0, k}, \\
%     \Qbiaszero_k
%         & = (\sqrt{\acp{\tau}}/\var_{\tau}) \cov_{0|\tau} \pibias_{\midpoint| 0, k} + \cov_{0|\tau} \cov^{-1} \mean
% \end{align*}
% and $\QMk_k, \QNk_k, \Qcovk_k$, and $\Qbiask_k$
% \begin{align*}
%     \Qcovk_k
%         & = \var_{k | \midpoint} \Id_{\dimx} + (\acp{k}/\acp{\midpoint}) \picov_{\midpoint| 0, k},\\
%     \QMk_k
%         & = \sqrt{\acp{k}/\acp{\midpoint}} \pizero_{\midpoint| 0, k},
%     \qquad
%     \QNk_k
%         = \sqrt{\acp{k}/\acp{\midpoint}} \pik_{\midpoint| 0, k},
%     \\
%     \Qbiask_k
%         & = \sqrt{\acp{k}/\acp{\midpoint}} \pibias_{\midpoint| 0, k}.
% \end{align*}

Similarly for the kernel $\lastKernel$ \eqref{eq:last-kernel-gauss}, we have
\begin{equation*}
    \begin{aligned}
        \potn{}{\bx_1} \bw{1|2}{\bx_2}{\bx_1}
            & = \normpdf(\obs; \bfA \bx_1, \stdobs^2 \Id_\dimobs) \ \fwtrans{1 | 0, 2}{\predx{2}(\bx_2), \bx_2}{\bx_1} \\
            & = \normpdf \Big(\bx_1;
                \Lcovbefore \big((1/\stdobs^2) \bfA^\top \obs + (1/\var_{1|0, 2}) (a \predx{2}(\bx_2) + b \bx_2) \big),
                \Lcovbefore
            \Big)\\
            & = \normpdf \Big(\bx_1;
                \LHbefore \bx_2 + \Lbiasbefore,
                \Lcovbefore
             \Big)
    \end{aligned}
\end{equation*}
where
\begin{align*}
    \Lcovbefore 
        & = \big[ (1/\var_{1|0, 2}) \Id_\dimx + (1/\stdobs^2) \bfA^\top\bfA \big]^{-1},
        \\
    \LHbefore
        & =  \Lcovbefore \big(
            (\sqrt{\acp{2}}/(\var_{2}\var_{1|0,2})) a \cov_{0|2} + (b/\var_{1|0,2}) \Id
            \big),
        \\
    \Lbiasbefore
        & = \Lcovbefore \big( (1/\stdobs^2) \bfA^\top \obs + (a/\var_{1|0,2}) \cov_{0|2}\cov^{-1} \mean \big)
\end{align*}

The last kernel \eqref{eq:algo-last-kernel} involves $\delta_{\predx{1}(\bx_1)}(\bx_0)$, hence, we plug the expression $\bx_1$ as a function of $\bx_0$
\begin{align*}
    \bx_0 = \predx{1}(\bx_1) \implies \bx_1 = (\var_1 / \sqrt{\acp{1}}) (\cov_{0|1}^{-1} \bx_0 - \cov^{-1} \mean)
    \eqsp,
\end{align*}
into the previous expression to obtain
\begin{align*}
    \Lcov
        & = (\acp{1}/ \var_1^2) \cov_{0|1} \ \Lcovbefore \ \cov_{0|1},
    \qquad
    \LH
        = (\sqrt{\acp{1}}/ \var_1) \cov_{0|1} \LHbefore,\\
    \Lbias
        & = (\sqrt{\acp{1}}/ \var_1) \cov_{0|1} (\Lbiasbefore + (\var_1/\sqrt{\acp{1}}) \cov^{-1}\mean).
\end{align*}




% --- redef notation to not impact the rest of the document
\renewcommand{\hpotn}[2]{\ifthenelse{\equal{#2}{}}{\hat{g}^\param _{#1}}{\hat{g}^\param _{#1}(#2)}}
% ---
\section{Experiments details}
\subsection{Benchmark processing} \label{app:benchmarks}
Here, we discuss additional processing and cleaning steps specific to certain of the chosen benchmarks.

\subsubsection{VQA v2.0} \label{app:vqa}

\paragraph{Re-labeling} Rather than a single ground truth label per example, the VQA~\cite{antol2015vqa} and updated VQA v2.0~\citep{goyal2017making} datasets collect ten separate crowd-annotated labels per image-question query. Their accuracy metric then assigns a score to a model prediction based on the overlap between the prediction and these ten labels. As we assign a single ground-truth label to each question, we manually re-label all the VQA v2.0 queries we include in our revised subset rather than only inspecting ones for which some model failed.

\paragraph{Selection of queries} The VQA v2.0 dataset is designed to mitigate common biases in visual question answering datasets by balancing the original VQA dataset with complementary images that break a given bias. To maintain this construction, we randomly select from these image pairs in VQA v2.0, and reject a given pair if either image deemed ambiguous.

To further improve the clarity and ease of labeling of our samples, we also limit our subset to only `yes/no' questions within VQA v2.0, as open-ended queries have a greater potential for ambiguity and can often have multiple correct answers.

\subsubsection{Reading Comprehension Benchmarks} \label{app:hotpotqa}
\paragraph{Re-labeling to account for multiple correct responses} SQuAD2.0~\cite{rajpurkar2018know}, HotPotQA~\citep{yang2018hotpotqa}, and DROP~\citep{dua2019drop} are all question answering benchmarks based on background knowledge provided in-context. Most of these questions are open-ended, so there are often multiple valid responses. For instance, consider the following example from HotPotQA:
\\
\begin{tcolorbox}[colback=gray!6, colframe=gray!50, arc=2mm, boxrule=0.5pt]
    \textbf{Paragraph A:} \textit{Ethel Houbiers}
    
    Ethel Houbiers is a French voice actress.  She is the French voice of Penélope Cruz and Salma Hayek.
    
    \medskip
    
    \textbf{Paragraph B:} \textit{Salma Hayek}
    
    Salma Hayek Pinault ( Hayek Jiménez) (born September 2, 1966), known professionally as Salma Hayek, is a Mexican and American film actress, producer, and former model\ldots[\textit{continued}]
    
    \medskip
    
    \textbf{Question:} which Mexican and American film actress is Ethel Houbiers French voice of?
    
    \medskip
    
    \textbf{Answer:} Salma Hayek Pinault

\end{tcolorbox}
\vspace{\baselineskip}
 
\noindent While ``Salma Hayek Pinault'' is a valid answer, ``Salma Hayek'' is a second answer that should also be considered valid. To address these cases, when re-labeling examples, we set the revised label to be a list of possible valid responses. Specifically, if any LLM answer doesn't match the original label but is also a valid solution, we include this answer as part of the updated label. We also make an effort to further list additional valid options. If the question is sufficiently open-ended and ambiguous that too many possible options might be valid, we mark the question as bad. 

\paragraph{Marking example as mislabeled} As discussed above, the original reading comprehension benchmarks include many questions for which there might be multiple possible equivalent answers that are not listed as the correct solution, such as the following example from DROP:

\begin{tcolorbox}[colback=gray!6, colframe=gray!50, arc=2mm, boxrule=0.5pt]
\textbf{Context:} Coming off their overtime win over the Bills, the Steelers flew to M\&T Bank Stadium\ldots Pittsburgh trailed in the first quarter as Ravens quarterback Joe Flacco completed a 14-yard touchdown pass to wide receiver Anquan Boldin.  After a scoreless second quarter, Pittsburgh answered in the third quarter\ldots With the win, not only did the Steelers improve to 9-3, but it also allowed them to take the AFC North division lead for the first time since week 4.

\medskip

\textbf{Question:} How many scoring drives took place in the first half?

\medskip
    
\textbf{Answer:} 1
\end{tcolorbox}

\noindent Here, ``one'' (the word rather than the number) would also be a valid solution. However, it might be unfair for us to consider this a benchmark error, as this ambiguity could potentially be overcome by altering our prompting strategy (e.g., specifying to provide numerical answers). 

For our platinum benchmarks we still revise such examples; in this particular case, we mark any of the following solutions as valid: ``1,'' ``one,'' ``1 scoring drive,'' ``one scoring drive.'' However, when counting the number of mislabeled examples, we only consider cases where the original label is not included within the list of valid labels we identify. For instance, if for this example we had revised the label to ``2'' or ``two,'' we would have considered the original example mislabeled.

\paragraph{Answer-matching for the original benchmark}
In Table \ref{tab:original_vs_cleaned}, we compare the number of errors on each of the original and revised benchmarks. However, as we discuss in the paragraph above, the original reading comprehension benchmarks often include ambiguity in the form of many equivalent correct solutions that may have been clarified through better prompting. To more accurately calculate the number of model errors on these original benchmarks, we ask an LLM (GPT-4o) to identify models' answers that are ``equivalent'' to the benchmarks' solution, and consider any such equivalent answer as correct. We only apply this automated equivalence-checking process to calculate the error counts on the original benchmark. For our platinum benchmarks, we instead manually enumerated all valid responses during the revision process.


\subsection{Chain-of-Thought prompt template} \label{app:template}
We use a chain-of-thought prompt for evaluation on all datasets except for VQA V2.0~\citep{goyal2017making}, for which there is general no need for multiple reasoning steps. The specific prompt varies slightly between benchmarks, however the general templates are as follows:\\

\noindent\textbf{Open-ended Question:}
\begin{tcolorbox}[colback=gray!6, colframe=gray!50, arc=2mm, boxrule=0.5pt]
    \texttt{Answer the following \{category\} question.\\\\
    \{question\}\\\\
    Think step-by-step.\textsf{ }Then, answer in the format "Answer:\textsf{ }XXX".}
\end{tcolorbox}

\vspace{\baselineskip}

\noindent\textbf{Multiple-Choice Question:}
\begin{tcolorbox}[colback=gray!6, colframe=gray!50, arc=2mm, boxrule=0.5pt]
    \texttt{Answer the following \{category\} question.\\\\
    \{question\}\\\\
    Options:\\
    A) \{option A\}\\
    B) \{option B\}\\
    C) \{option C\}\\
    D) \{option D\}\\\\
    Think step-by-step.\textsf{ }Then, provide the final answer in the format "Answer:\textsf{ }X" where X is the correct letter choice.}

\end{tcolorbox}

\vspace{\baselineskip}

\noindent For BIG-bench~\cite{srivastava2022beyond}, we use (A) instead of A) for multiple choice style to align with the prompt used by the original authors. For open-ended math questions, we additionally specify to respond with an integer and to exclude additional formatting, as model often style outputs with latex styling. We exclude chain-of-thought prompting for VQA v2.0, as the questions rarely require any explicit reasoning to answer. We also find that, in practice, models rarely actually think step-by-step for simple visual reasoning questions, even when prompted to do so.

% --- gallery
\begin{figure}[tb]
    \centering
    \subfigure{
        \includegraphics[width=.49\textwidth]{figures/gallery/ffhq-outpainting_half-223.jpeg}
        \includegraphics[width=.49\textwidth]{figures/gallery/ffhq-outpainting_half-504.jpeg}
    }
    \caption{Reconstructions for half mask inpainting on \ffhq\ dataset.}
\end{figure}

\begin{figure}[tb]
    \centering
    \subfigure{
        \includegraphics[width=.49\textwidth]{figures/gallery/ffhq-inpainting_center-192.jpeg}
        \hfill%
        \includegraphics[width=.49\textwidth]{figures/gallery/ffhq-inpainting_center-632.jpeg}
    }
    \caption{Reconstructions for box inpainting on \ffhq\ dataset.}
    \label{fig:pnpdm-conjugacy}
\end{figure}

\begin{figure}[tb]
    \centering
    \subfigure{
        \includegraphics[width=.49\textwidth]{figures/gallery/ffhq-jpeg2-345.jpeg}
        \includegraphics[width=.49\textwidth]{figures/gallery/ffhq-jpeg2-847.jpeg}
    }
    \caption{Reconstructions for JPEG dequantization QF=2\% on \ffhq\ dataset.}
\end{figure}


\begin{figure}[tb]
    \centering
    \subfigure{
        \includegraphics[width=.49\textwidth]{figures/gallery/imagenet-outpainting_half-139.jpeg}
        \includegraphics[width=.49\textwidth]{figures/gallery/imagenet-outpainting_half-778.jpeg}
    }
    \caption{Reconstructions Half mask inpainting on \imagenet\ dataset.}
\end{figure}


\begin{figure}[tb]
    \centering
    \subfigure{
        \includegraphics[width=.49\textwidth]{figures/gallery/imagenet-blur_svd-97.jpeg}
        \includegraphics[width=.49\textwidth]{figures/gallery/imagenet-blur_svd-100.jpeg}
    }
    \caption{Reconstructions for Gaussian deblurring on \imagenet\ dataset.}
\end{figure}

\begin{figure}[tb]
    \centering
    \subfigure{
        \includegraphics[width=.49\textwidth]{figures/gallery/ffhq-motion_blur-868.jpeg}
        \includegraphics[width=.49\textwidth]{figures/gallery/ffhq-motion_blur-473.jpeg}
    }
    \caption{Reconstructions for motion deblurring on \ffhq\ dataset.}
\end{figure}

\begin{figure}[tb]
    \centering
    \subfigure{
        \includegraphics[width=.49\textwidth]{figures/gallery/ffhq_ldm-outpainting_half-249.jpeg}
        \includegraphics[width=.49\textwidth]{figures/gallery/ffhq_ldm-outpainting_half-263.jpeg}
    }
    \caption{Reconstructions for half mask inpainting on \ffhq\ dataset with LDM prior.}
\end{figure}

\begin{figure}[tb]
    \centering
    \subfigure{
        \includegraphics[width=.49\textwidth]{figures/gallery/ffhq_ldm-sr16-129.jpeg}
        \includegraphics[width=.49\textwidth]{figures/gallery/ffhq_ldm-sr16-141.jpeg}
    }
    \caption{Reconstructions for SR $\times 16$ on \ffhq\ dataset with LDM prior.}
\end{figure}

\begin{figure}
    \centering
    \includegraphics[width=.85\textwidth]{figures/reconstructions/ffhq_outpainting_half_gibbs_sampler.jpg}
    \caption{Half mask inpainting on \ffhq\ dataset.}
\end{figure}
\begin{figure}
    \centering
    \includegraphics[width=.85\textwidth]{figures/reconstructions/imagenet_outpainting_half_gibbs_sampler.jpg}
    \caption{Half mask inpainting on \imagenet\ dataset.}
\end{figure}
\begin{figure}
    \centering
    \includegraphics[width=.85\textwidth]{figures/reconstructions/ffhq_inpainting_center_gibbs_sampler.jpg}
    \caption{Box inpainting on \ffhq\ dataset.}
\end{figure}
\begin{figure}
    \centering
    \includegraphics[width=.85\textwidth]{figures/reconstructions/imagenet_inpainting_center_gibbs_sampler.jpg}
    \caption{Box inpainting on \imagenet\ dataset.}
\end{figure}
\begin{figure}
    \centering
    \includegraphics[width=.85\textwidth]{figures/reconstructions/ffhq_jpeg2_gibbs_sampler.jpg}
    \caption{JPEG dequantization with $\mathrm{QF}=2$ on \ffhq\ dataset.}
\end{figure}
\begin{figure}
    \centering
    \includegraphics[width=.85\textwidth]{figures/reconstructions/imagenet_jpeg2_gibbs_sampler.jpg}
    \caption{JPEG dequantization with $\mathrm{QF}=2$ on \imagenet\ dataset.}
\end{figure}
\begin{figure}
    \centering
    \includegraphics[width=.85\textwidth]{figures/reconstructions/ffhq_motion_blur_gibbs_sampler.jpg}
    \caption{Motion deblurring on \ffhq\ dataset.}
\end{figure}
\begin{figure}
    \centering
    \includegraphics[width=.85\textwidth]{figures/reconstructions/imagenet_motion_blur_gibbs_sampler.jpg}
    \caption{Motion deblurring on \imagenet\ dataset.}
\end{figure}
\begin{figure}
    \centering
    \includegraphics[width=.85\textwidth]{figures/reconstructions/ffhq_sr16_gibbs_sampler.jpg}
    \caption{SR(16$\times$) on \ffhq\ dataset.}
\end{figure}
\begin{figure}
    \centering
    \includegraphics[width=.85\textwidth]{figures/reconstructions/imagenet_sr16_gibbs_sampler.jpg}
    \caption{SR(16$\times$) on \imagenet\ dataset.}
\end{figure}
\begin{figure}
    \centering
    \includegraphics[width=.85\textwidth]{figures/reconstructions/ffhq_high_dynamic_range_gibbs_sampler.jpg}
    \caption{High dynamic range on \ffhq\ dataset.}
\end{figure}
\begin{figure}
    \centering
    \includegraphics[width=.85\textwidth]{figures/reconstructions/imagenet_high_dynamic_range_gibbs_sampler.jpg}
    \caption{High dynamic range on \imagenet\ dataset.}
\end{figure}
\begin{figure}
    \centering
    \includegraphics[width=.85\textwidth]{figures/reconstructions/ffhq_ldm_sr4_gibbs_sampler.jpg}
    \caption{SR(4$\times$) on \ffhq\ dataset with latent diffusion.}
\end{figure}
\begin{figure}
    \centering
    \includegraphics[width=.85\textwidth]{figures/reconstructions/ffhq_ldm_sr16_gibbs_sampler.jpg}
    \caption{SR(16$\times$) on \ffhq\ dataset with latent diffusion.}
\end{figure}
\begin{figure}
    \centering
    \includegraphics[width=.85\textwidth]{figures/reconstructions/ffhq_ldm_outpainting_half_gibbs_sampler.jpg}
    \caption{Half mask on \ffhq\ dataset with latent diffusion.}
\end{figure}



\label{apdx-sec:visual-reconstructions}



\end{document}


% This document was modified from the file originally made available by
% Pat Langley and Andrea Danyluk for ICML-2K. This version was created
% by Iain Murray in 2018, and modified by Alexandre Bouchard in
% 2019 and 2021 and by Csaba Szepesvari, Gang Niu and Sivan Sabato in 2022.
% Modified again in 2023 and 2024 by Sivan Sabato and Jonathan Scarlett.
% Previous contributors include Dan Roy, Lise Getoor and Tobias
% Scheffer, which was slightly modified from the 2010 version by
% Thorsten Joachims & Johannes Fuernkranz, slightly modified from the
% 2009 version by Kiri Wagstaff and Sam Roweis's 2008 version, which is
% slightly modified from Prasad Tadepalli's 2007 version which is a
% lightly changed version of the previous year's version by Andrew
% Moore, which was in turn edited from those of Kristian Kersting and
% Codrina Lauth. Alex Smola contributed to the algorithmic style files.
