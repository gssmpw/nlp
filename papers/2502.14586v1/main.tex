%%%%%%%%%%%%%%%%%%%%%%%%%%%%%%%%%%%%%%%%%%%%%%%%%%%%%%%%%%%%%%%%%%%%%%%%%%%%%%%%
% Template for USENIX papers.
%
% History:
%
% - TEMPLATE for Usenix papers, specifically to meet requirements of
%   USENIX '05. originally a template for producing IEEE-format
%   articles using LaTeX. written by Matthew Ward, CS Department,
%   Worcester Polytechnic Institute. adapted by David Beazley for his
%   excellent SWIG paper in Proceedings, Tcl 96. turned into a
%   smartass generic template by De Clarke, with thanks to both the
%   above pioneers. Use at your own risk. Complaints to /dev/null.
%   Make it two column with no page numbering, default is 10 point.
%
% - Munged by Fred Douglis <douglis@research.att.com> 10/97 to
%   separate the .sty file from the LaTeX source template, so that
%   people can more easily include the .sty file into an existing
%   document. Also changed to more closely follow the style guidelines
%   as represented by the Word sample file.
%
% - Note that since 2010, USENIX does not require endnotes. If you
%   want foot of page notes, don't include the endnotes package in the
%   usepackage command, below.
% - This version uses the latex2e styles, not the very ancient 2.09
%   stuff.
%
% - Updated July 2018: Text block size changed from 6.5" to 7"
%
% - Updated Dec 2018 for ATC'19:
%
%   * Revised text to pass HotCRP's auto-formatting check, with
%     hotcrp.settings.submission_form.body_font_size=10pt, and
%     hotcrp.settings.submission_form.line_height=12pt
%
%   * Switched from \endnote-s to \footnote-s to match Usenix's policy.
%
%   * \section* => \begin{abstract} ... \end{abstract}
%
%   * Make template self-contained in terms of bibtex entires, to allow
%     this file to be compiled. (And changing refs style to 'plain'.)
%
%   * Make template self-contained in terms of figures, to
%     allow this file to be compiled. 
%
%   * Added packages for hyperref, embedding fonts, and improving
%     appearance.
%   
%   * Removed outdated text.
%
%%%%%%%%%%%%%%%%%%%%%%%%%%%%%%%%%%%%%%%%%%%%%%%%%%%%%%%%%%%%%%%%%%%%%%%%%%%%%%%%

\documentclass[letterpaper,twocolumn,10pt]{article}
\usepackage{usenix}

% to be able to draw some self-contained figs
\usepackage{tikz}
% \usepackage{amsmath}

% inlined bib file
\usepackage{filecontents}

% Our packages
\usepackage{cite}
\usepackage{amsmath,amssymb,amsfonts}
% \usepackage{algorithmic}
\usepackage{graphicx}
\usepackage{textcomp}
\usepackage{xcolor}
\usepackage{subcaption}

% \usepackage[colorlinks=true,urlcolor=black]{hyperref}
% \usepackage[subtle, tracking=normal, leading=normal]{savetrees}

\def\BibTeX{{\rm B\kern-.05em{\sc i\kern-.025em b}\kern-.08em
    T\kern-.1667em\lower.7ex\hbox{E}\kern-.125emX}}


%% imported packages
\usepackage{balance}
\usepackage{mathrsfs}
\usepackage{bm}
\usepackage{svg}
\usepackage{tabularx}  % added for importing tables
\usepackage{booktabs}  % added for importing tables
\usepackage{algpseudocode}
\usepackage{algorithm}
\usepackage{multirow}


%------
\usepackage{comment}
\usepackage[textwidth=1.5cm,textsize=scriptsize]{todonotes}
\newcommand{\lpasa}[1]{\todo[color=green!20]{{\bf lpasa:} #1}}
\newcommand{\new}[1]{\textcolor{black}{#1}}
\newcommand{\fra}[1]{\textcolor{red}{Fra: #1}}
%-------

\RequirePackage{amsmath}
    \DeclareMathOperator*{\argmax}{arg\,max}
    \DeclareMathOperator*{\argmin}{arg\,min}
    \DeclareMathOperator*{\arginf}{arg\,inf}
    \DeclareMathOperator*{\argsup}{arg\,sup}

% %-------------------------------------------------------------------------------
% \begin{filecontents}{\jobname.bib}
% %-------------------------------------------------------------------------------
% @Book{arpachiDusseau18:osbook,
%   author =       {Arpaci-Dusseau, Remzi H. and Arpaci-Dusseau Andrea C.},
%   title =        {Operating Systems: Three Easy Pieces},
%   publisher =    {Arpaci-Dusseau Books, LLC},
%   year =         2015,
%   edition =      {1.00},
%   note =         {\url{http://pages.cs.wisc.edu/~remzi/OSTEP/}}
% }
% @InProceedings{waldspurger02,
%   author =       {Waldspurger, Carl A.},
%   title =        {Memory resource management in {VMware ESX} server},
%   booktitle =    {USENIX Symposium on Operating System Design and
%                   Implementation (OSDI)},
%   year =         2002,
%   pages =        {181--194},
%   note =         {\url{https://www.usenix.org/legacy/event/osdi02/tech/waldspurger/waldspurger.pdf}}}
% \end{filecontents}

%-------------------------------------------------------------------------------
\begin{document}
%-------------------------------------------------------------------------------

%don't want date printed
\date{}

% make title bold and 14 pt font (Latex default is non-bold, 16 pt)
% \title{\Large \bf Toward the Model Selection Hijacking Adversarial Attack}
% \title{\Large \bf Moshi Moshi? A Model Selection Hijacking Attack for Adversarial Exploitation}
\title{\Large \bf Moshi Moshi? A Model Selection Hijacking Adversarial Attack}

%for single author (just remove % characters)
\author{
{\rm Riccardo Petrucci}\\
University of Padua, Italy
\and
{\rm Luca Pajola}\\
University of Padua, Italy
\and
{\rm Francesco Marchiori}\\
University of Padua, Italy
\and
{\rm Luca Pasa}\\
University of Padua, Italy
\and
{\rm Mauro Conti}\\
University of Padua, Italy
% copy the following lines to add more authors
% \and
% {\rm Name}\\
%Name Institution
} % end author

\maketitle

%-------------------------------------------------------------------------------
\begin{abstract}
%-------------------------------------------------------------------------------
% Model selection in Machine Learning (ML) is of paramount importance as it significantly influences the system's performance, accuracy, and generalizability to novel data.
% The selection of an appropriate model is essential for enabling the ML models to effectively learn from the data, make precise predictions, and adapt to diverse tasks and environments.
% Despite the critical role of model selection in the proper functioning of ML systems, its security from the perspective of adversarial machine learning remains unexplored.
% In the context of the Machine-Learning-as-a-Service (MLaaS) paradigm, this threat is particularly pertinent, as users often delegate the training phase to third-party entities by supplying data and learning strategies. 
% In this case, the issue is twofold: an incorrect model selection can negatively impact both the model's performance and the cost of running it, posing challenges for both the user and the service provider.
\new{
Model selection is a fundamental task in Machine Learning~(ML), focusing on selecting the most suitable model from a pool of candidates by evaluating their performance on specific metrics.
This process ensures optimal performance, computational efficiency, and adaptability to diverse tasks and environments.
Despite its critical role, its security from the perspective of adversarial ML remains unexplored.
This risk is heightened in the Machine-Learning-as-a-Service model, where users delegate the training phase and the model selection process to third-party providers, supplying data and training strategies.
Therefore, attacks on model selection could harm both the user and the provider, undermining model performance and driving up operational costs.
}

% In this work, we investigate \textit{if} and \textit{how} an adversary can tamper the model selection treating the learning and model selection phase as a black box. 
% To this end, we introduce the Model Selection Hijacking Adversarial Attack (MSHAA), a novel approach aimed at manipulating model selection data to yield a model with characteristics advantageous to the adversary.
% Utilizing a framework based on Variational Auto Encoder, we provide evidence that an attacker can induce inefficiencies in ML deployment, manifesting as performance degradation, increased latency, and elevated energy consumption.
% Our experiments employ conventional deep-learning architectures considering benchmark tasks in computer vision and speech recognition.
\new{
% In this work, we present \textbf{MHSAA} (\textbf{M}odel \textbf{S}election \textbf{H}ijacking \textbf{A}dversarial \textbf{A}ttack), the first adversarial attack specifically targeting model selection.
In this work, we present \textbf{MOSHI} (\textbf{MO}del \textbf{S}election \textbf{HI}jacking adversarial attack), the first adversarial attack specifically targeting model selection.
Our novel approach manipulates model selection data to favor the adversary, even without prior knowledge of the system.
Utilizing a framework based on Variational Auto Encoders, we provide evidence that an attacker can induce inefficiencies in ML deployment.
We test our attack on diverse computer vision and speech recognition benchmark tasks and different settings, obtaining an average attack success rate of 75.42\%. % 80.44\%.
%%% NOTA: ASR calcolato su media di Tabs 3+4
% In particular, our attack causes an 88.30\% decrease in generalization capabilities, an 83.33\% increase in latency, and a 12.65\% increase in energy consumption.
In particular, our attack causes an average 88.30\% decrease in generalization capabilities, an 83.33\% increase in latency, and an increase of up to 105.85\% in energy consumption.
%%% NOTA: valori calcolati su media di Tab 5
These results highlight the significant vulnerabilities in model selection processes and their potential impact on real-world applications.
% We publicly release our code and implementation at: \url{https://anonymous.4open.science/r/MOSHI-1518}.
}
\end{abstract}

% 13 pages max
\documentclass[../main.tex]{subfiles}
\graphicspath{{../images/}}
\makeatletter
\def\input@path{{../images/}}
\makeatother
\begin{document}
\section{Introduction}
\begin{figure}
\centering
\begin{tikzpicture}
\node[inner sep=0pt] (ws) at (0, 0) {
\includegraphics[height=.4\textwidth, trim={10cm 0 10cm 0},clip]{world_space.png}};
\node[inner sep=0pt] (cs) at (6,0) {\includegraphics[height=.4\textwidth, trim={10cm 1cm 10cm 4cm},clip]{conf_space.png}};
\end{tikzpicture}
\vspace{-5pt}
\label{fig:pbrm_intro}
\caption{\textbf{Left}: Shows world space obstacles as grey spheres. Robots start and goal configuration is colored red and green, respectively. Configurations along the computed path are colored transparent blue. \textbf{Right:} Mapped world space scenario to configuration space. Obstacle region is the grey mesh. Red spheres are collision-free regions computed by the neural SCDF. The optimized shortest path in the convex corridor is the blue curve.}
\vspace{-25pt}
\end{figure}
Motion planning is the problem of finding a collision-free trajectory that connects a given start and goal configuration. The planning takes place in the configuration space of the robot. For single body robots, like mobile robots or drones, the configuration space and the world space are usually the same. This simplifies the planning, since explicit obstacle representations are available which enables geometrical tools like separating hyperplanes, smallest distance to obstacles etc., to be used when designing motion planning algorithms. For multi-body robots like manipulators, the situation is completely different. The world space obstacles are usually mapped to non-convex regions, and to make the problem even harder, the mapping is usually not known. Forming explicit representations of the obstacle region in the configuration space is usually too expensive or intractable. Despite all of this, sampling based planners are used with great success, which mainly is due to their use of implicit representations of the obstacle region. The basic idea is to construct a graph in the configuration space that covers and connects the collision-free region. From this graph, a path can be extracted that connects a given start and goal configuration. The approach is computationally expensive, since the graph is constructed with the smallest geometrical building block available, points, which represents a collision-check. Furthermore, the extracted paths from the graph are non-smooth and jagged due to the stochastic nature of the approach. This adds an additional post-processing step to the process, where the paths are shortcutted and smoothened, before the path can be used for tracking. Clearly a lot of time is invested to form this graph and produce smooth paths. Thus, if the obstacles start to move, then all of this work is done in no use, since all points that make up this graph need to be re-verified, which is simply too time consuming to be done in real time.
\\\\
In this work, we want to address the existing drawbacks of the sampling based planners. Our main contribution is an improved motion planner where each vertex in the graph covers a collision-free region in the form of a sphere instead of a point and where the edges are formed with neighboring intersecting spheres. This representation has the advantage of instead of returning piecewise linear paths, returning a sequence of overlapping spheres, i.e. a convex corridor, that connects a given start and goal configuration, illustrated in Figure \ref{fig:pbrm_intro}. This convex corridor allows us to use convex optimization to produce smooth trajectories, instead of computationally expensive post-processing methods. The representation further allows us to estimate the coverage of the collision-free space, which gives us awareness and feedback in the offline roadmap construction phase. Finally, our representation is simple to adapt to moving obstacles, simply requery for the new radii and recheck for intersections. 
\\\\
The spherical collision-free regions are formed using a signed distance function (SDF), which is a function that returns the smallest distance from an arbitrary point to the boundary of an obstacle. As the name implies, the distance is signed, thus if the point is inside the obstacle it is negative otherwise positive. If the distance is positive, a sphere with radius equal to the distance is guaranteed to cover a collision-free region. Using an SDF in motion planning is not new, but what is novel about our approach is that we express the distance in the configuration space instead of the world space and by doing so allows us to form these convex collision-free regions. We refer to the resulting SDF as a signed configuration distance function (SCDF). Computing an SCDF analytically is non-trivial, our approach is therefore to parameterize the SCDF with a deep neural network and learn the mapping by supervised learning. Our resulting neural SCDF can compute distances for different parameter values of obstacle shapes and we also show how multiple distances can be combined, thus making our approach flexible.
\section{Related work}
Motion planning algorithms can roughly be divided into three families, grid-based, sampling based and optimization based methods. Grid-based methods (GBM) discretize the planning space from which a graph is then compiled. A standard search method is A$^\star$ \citep{a_star}, which is classified as an \textit{informed} search method, since it employs a heuristic function to speed up the search. A$^\star$ guarantees to return an optimal path at the level of discretization used. GBMs usually discretize the planning space by a regular lattice and this limits the GBMs to problems with low dimensionality due to the curse of dimensionality. Thus, GBMs are usually limited to single-body robots where the degrees of freedom (DOF) are low. To overcome the inherent scaling problem with the GBMs, stochastic methods are usually used for multi-body robots. These methods are termed as sampling-based methods (SBM) and core members within this family are the rapidly-exploring random trees (RRT) \citep{rrt} and the probabilistic roadmap (PRM) \citep{prm}. RRT grows a tree from the start configuration and explores the collision-free region in a rapid way until it is able to connect to the goal region. RRT is usually improved by bi-directional planning \citep{rrt_connect}, i.e. an additional tree is grown from the goal configuration and the trees are tested for connection after any tree has been expanded. RRT is a single-query method, thus it searches for a path from scratch each time it is queried. Contrary to this, PRM is a multi-query method, which solves for multiple queries without starting from scratch. PRM does this by creating a roadmap (graph) that covers the collision-free space as an offline step. The graph is then used to solve for multiple queries. PRMs are used in cases where the environment does not change since the extra offline step is too computationally costly and needs to be re-done if the environment is changed. In our work, we address this inherent issue by using a different roadmap representation. Our vertices in the graph cover a collision-free region in the form of spheres and we form the edges by checking for intersecting spheres. If something in the environment changes, we recompute the spheres radii and recheck the intersections, without relying on collision detection. We use a trained neural network to compute the sphere radius, therefore querying for the radius can be done fast, hence our representation enables the PRM for dynamic environments.
\\\\
In the recent decades, optimization based methods (OBM) \citep{chomp, schulman, itomp, stomp} have been introduced as an alternative to SBM for multi-body robots. Like the SBM, the OBMs scale well to higher dimensional problems and produce smoother motion. It is common to use a SDF in the optimization since it is a smooth function, thus enabling gradient-based methods. However, the standard way of expressing the SDF is in world space. The distance therefore needs to be mapped to the configuration space by the forward kinematics. This mapping makes the optimization problem a non-linear program (NLP), which is computationally expensive to solve. Recently, a different approach has been proposed. In \cite{mp_gcs} motion planning is formulated as a convex optimization problem by using the graph of convex sets framework \citep{gcs}. The underlying idea is to decompose the collision-free space into intersecting convex sets from which a convex optimization problem is formulated. In cases where an explicit representation of the obstacles in the configuration space exists, like for single-body robots, creating collision-free convex regions can be done fast \citep{iris}. For multi-body robots, this is non-trivial. Existing work does this successfully \citep{iris_nlp, iris_c} by an optimization based approach, but the methods are still too time consuming to be used in the presence of moving obstacles. Our approach is instead to use deep learning to learn an SDF expressed in the configuration space. With this, we can query for shortest distances to the collision boundary, which allows us to expand spherical regions which are collision-free. Our approach is fast and therefore enables our suggested roadmap planner to be used in dynamic environments.
\\\\
Recent research has focused on learning collision detection \citep{fk_kernel_distance, diffco, graphdistnet} by predicting the signed distance between the robot links and the surrounding obstacles in the world space. The learned SDF is used in trajectory optimization but since the distance is expressed in the world space, the problem becomes an NLP and therefore takes a long time to solve. We take a novel approach and suggest to instead express the signed distance in the configuration space. This allows us to improve the PRM at the same time as it enables convex optimization for trajectory optimization, which runs faster and is more reliable than NLP solvers. In \cite{cspf} a learned signed distance function in the configuration space is proposed similar to our approach. However, their approach is restricted to point cloud representations, while we propose to represent the obstacles as parameterized geometric shapes, e.g. spheres. Furthermore, we also show how to use our learned SCDF to improve an existing roadmap planner.
\section{Problem formulation}
A robot is located in the world space, $\W \subset \R^3 $. The unique location of the robot is given by its configuration $\q \in \C$, where $\C$ is the configuration space. The set of points covered by the robots bodies at a certain configuration is expressed as $\B(\q) \subset \W$. The robot is surrounded by $\NrObst$ obstacles $\O = \bigcup_{i=1}^{\NrObst} \O_i$, where  $\O_i \subset \W$. The representation of the obstacle in the configuration space is the set $\C\O_i = \{\q \in \C \: |\: \B(\q) \cap \O_i \neq \emptyset \}$. The obstacle space is formed as $\Co = \bigcup_{i=1}^{\NrObst} \C \O_i$. The complement is referred to as the free space, $\Cf = \C \setminus \Co$. The path planning problem is a tuple, ($\Cf$, $\qStart$, $\qGoal$), where we want to connect a query pair, consisting of a start, $\qStart$, and goal configuration, $\qGoal$, with a geometric path, $\q(s): [0, 1] \mapsto \Cf$, such that $\q(0)=\qStart$ and $\q(1)=\qGoal$, or report correctly when such a path does not exist.
\end{document}

\section{Related Work}
% \subsection{Vision Language Model}
% 시각장애인에서 상황을 설명할 DB가 없으니 만들었다. 그리고 이를 VLM에 튜닝했다.
\subsection{Technical approaches for assisting the visually-impaired}


\subsection{Datasets for visual instruction tuning}

\section{Basic Background: Supervised Learning and the PAC Model}
\label{sec:background}

At this point almost everyone has heard of machine learning (ML). Anyone likely to stumble upon this article will have also heard of its most influential special case, supervised learning, and those theoretically inclined will also be familiar with the PAC model. Nonetheless, I will set the stage by  recapping the basics.

\subsection{Basics of Supervised Learning}%Let's set the stage in any case

\emph{Supervised Learning} is the task of ``coming up'' with a function $f: \X \to \Y$ to ``explain'' or ``fit'' a sequence of input/output examples   $(x_1,y_1), \ldots, (x_n,y_n)$, with $x_i \in \X$ and $y_i \in \Y$.  Here $\X$ is a \emph{data domain} consisting of \emph{datapoints} $x \in \X$, $\Y$ is a \emph{label set} consisting of \emph{labels} $y \in \Y$, and the sequence $(x_1,y_1),\ldots,(x_n,y_n)$ is the \emph{training data} consisting of \emph{labeled examples (a.k.a. samples)}~$(x_i,y_i)$.  I~will refer to the chosen function $f$ as a \emph{predictor}, and to $n$ as the \emph{sample size}. A \emph{learning algorithm} takes as input training data, and outputs (some representation of) a predictor $f \in \Y^\X$.\footnote{Note that this describes the usual \emph{batch}, a.k.a.~\emph{offline}, setting of supervised learning. I do not discuss other paradigms such as online or active learning in this article.} 



Success in supervised learning is defined as \emph{generalization} to  future examples: For a typical \emph{test example}  $(x_{\tst},y_{\tst})$, the predicted label $y'_{\tst}=f(x_{\tst})$ should ``equal'' $y_{\tst}$, perhaps approximately. We usually assume the test example is drawn from the same  ``source'' as the training data  --- commonly, i.i.d.~from the same distribution. The quality of the prediction is quantified by $\ell(y'_{\tst},y_{\tst})$, where $\ell:~\Y~\times~\Y \to \RR_{\geq 0}$ is a \emph{loss function} chosen as part of the problem definition. Common loss functions include the 0-1 loss $\ell_{0-1}(y',y) = [y' \neq y]$ for \emph{classification} problems,\footnote{The notation $[P]$ denotes $1$ when predicate $P$ is true, and denotes $0$ when $P$ is false.} as well as the absolute loss $|y'-y|$ or squared loss $(y'-y)^2$ for \emph{regression problems} featuring $\Y  \sse \RR$.

Nontrivial generalization properties are typically only possible if one assumes something about the data.\footnote{The need for such an assumption is formalized by the  \emph{no free lunch theorems} of supervised learning \cite{wolpert_connection_1992,wolpert_lack_1996,schaffer_conservation_1994}.} The Bayesian approach to  machine learning, common in many applications, assumes some parametric form for the distribution generating the data, and postulates a prior on the parameters. This is not the approach I will take in this article. Instead, I will focus on the frequentist --- and some would say ``worst-case'' or ``adversarial'' ---  approach that is common in the computational learning theory community, embodied by the PAC model. Here we assume that the (training and test) data can be explained, perhaps approximately, by a function in some ``simple enough to learn'' class of functions $\H \sse \Y^\X$, often called the \emph{hypotheses}. Equivalently, we  seek a predictor which explains the unseen data roughly  as well as the best hypothesis $h^* \in \H$, whether or not we assume that $h^*$ itself provides a perfect explanation.



 \paragraph{Common Algorithmic Templates.} Perhaps the best known general-purpose supervised learning algorithm is \emph{empirical risk minimization (ERM)}, which chooses as its predictor a hypothesis $f \in \H$ minimizing $\frac{1}{n} \sum_{i=1}^n \ell(f(x_i),y_i)$ --- a quantity called the \emph{training error}, \emph{empirical error}, or \emph{empirical risk} of $f$. %\footnote{When multiple hypotheses minimize the empirical risk, we assume ERM breaks ties arbitrarily.}
A common template for generalizing ERM involves adding a \emph{regularization term} $\psi(f)$ to the  objective function, typically chosen to measure some notion of ``hypothesis complexity.'' An algorithm instantiating this template is known as a \emph{structural risk minimizer (SRM)}, and chooses as its predictor the hypothesis $f \in \H$ minimizing the \emph{structural risk} $\frac{1}{n} \sum_{i=1}^n \ell(f(x_i),y_i) + \psi(f)$. Other well-known algorithms, such as gradient descent and its variations,  can frequently be interpreted as approximate implementations of ERM or SRM.


\paragraph{Proper vs Improper Learning.} A learning algorithm is said to be \emph{proper} if its predictor $f$ is always chosen from the hypothesis class, i.e., $f \in \H$, otherwise it is said to be \emph{improper}. ERM  is an example of a proper learning algorithm, as are SRM algorithms of the form described above.  In the \emph{proper regime} of learning, algorithms are required to be proper. This article will be concerned with the more flexible \emph{improper regime} (a.k.a \emph{representation-independent learning}), where no such constraint is placed on the learner. In other words, all we care about is predictive power at test time, rather than any insights derived from the functional form or representation of the predictor~itself.


\subsection{The PAC Model}
A standard mathematical setup for evaluation of supervised learning algorithms, at least in the theoretical computer science community, is Valiant's \emph{Probably Approximately Correct (PAC) model} of learning (see e.g.~\cite{kearns_introduction_1994,mohri_foundations_2018}). Here, we assume there is an unknown distribution $\D$ on $\X \times \Y$ from which training and test data are  drawn.  Specifically, the labeled datapoints of the training set  $(x_1,y_1), \ldots, (x_n,y_n)$, as well as the test data  $(x_\tst,y_\tst)$, are i.i.d.~from $\D$. Often it is assumed that $\D$ lies in some class of distributions of interest. The \emph{true expected loss}, or simply \emph{loss}, of a predictor $f: \X \to \Y$ is the expected loss it incurs on draws from $\D$, written $L_\D(f) = \Ex_{(x,y) \sim \D} \ell(f(x),y)$.


There are two main ``settings'' in PAC learning. The  \emph{realizable setting} only requires that the data be perfectly explained by some hypothesis in $\H$. More generally, the \emph{agnostic setting} makes no assumption relating the data to the hypotheses, but shifts the goalposts as necessary to allow nontrivial guarantees: the expected loss at test time is evaluated only ``relative'' to that of the best hypothesis $h^* \in \H$. There are other settings which make more nuanced assumptions, such as $\D$ being of a particular parametric form or its support living in some (unknown) lower-dimensional space, etc. I will mostly discuss the realizable and agnostic settings in this article, those being the simplest and most studied from a theoretical perspective. %TODO:We will briefly discuss other settings in Section ??

The PAC model demands high probability guarantees of learners, in the worst case over distributions of interest. Consider first the realizable setting, where $\D$ is such that $\min_{h \in \H} L_{\D}(h) = 0$. A PAC learner has \emph{error} $\epsilon=\epsilon(n)$ and \emph{confidence} $\delta=\delta(n)$ if, when training data consists of $n$ i.i.d~samples from a realizable distribution $\D$, it produces a predictor $f$  satisfying $L_\D(f) \leq \epsilon$ with probability at least $1-\delta$. In the agnostic setting, where $\D$ can be arbitrary, we require $L_\D(f) - \min_{h \in \H} L_\D(h) \leq \epsilon$ with probability $1-\delta$.

In both the realizable and agnostic settings, we look for PAC learners with small $\epsilon$ and $\delta$ as a function of the sample size $n$. An equivalent perspective looks at the sample complexity $m(\epsilon,\delta)$, which is the minimum sample size which guarantees error  at most $\epsilon$ with probability at least $1-\delta$. We say a problem is \emph{PAC learnable} if its PAC sample complexity is finite whenever $\epsilon,\delta > 0$.

For most PAC learning problems, learnability and sample complexity are characterized in terms of a  ``dimension'' of the hypothesis class. Most prominently this is the \emph{VC dimension} for binary classification, the \emph{fat shattering dimension} for agnostic regression, and the \emph{DS dimension} for multiclass classification (see \cite{anthony_neural_1999,daniely_optimal_2014,brukhim_characterization_2022}). Treatment of these is beyond the scope of this article. The unfamiliar reader need not worry, however,  as dimensions will feature only tangentially in our~discussion.




%\paragraph{Learning settings: Realizable, Agnostic, etc.} In learning theory, evaluating a supervised learning algorithm requires specifying a data model and an objective. We will leave the details of the data model flexible for now, to allow for both the PAC model and the adversarial transductive model. Nonetheless we will describe two variations, which we call ``settings'', which cut across different models. The  \emph{realizable setting}  requires only that the data be perfectly explained by some hypothesis $h \in \H$ --- i.e., there exists a hypothesis which is guaranteed to suffer a loss of $0$ on training and test data. The performance of the learning algorithm is its expected loss at test time for some ``worst case'' realizable instance. More generally, the \emph{agnostic setting} makes no assumption relating the data to the hypotheses, but shifts the goalposts as necessary to allow nontrivial guarantees: the expected loss at test time is evaluated only ``relative'' to that of the best hypothesis $h^* \in \H$, again for some ``worst case'' instance. There are other settings which make more nuanced assumptions about the data, such as it is drawn from a distribution of a particular parametric form, or that it lives in some (unknown) lower-dimensional space, etc. We will mostly discuss the realizable and agnostic settings, those being the simplest and most studied from a theoretical perspective.




%%% Local Variables:
%%% mode: latex
%%% TeX-master: "learning_matching"
%%% End:

\section{System and Threat Model}
\label{sec:stmodel}

\new{
In this section, we describe our system and threat model (Section~\ref{subsec:system} and Section~\ref{subsec:threat}) while also providing different motivations and scenarios of importance for our proposed attack (Section~\ref{subsec:motivation}).
}

\subsection{System Model}
\label{subsec:system}

\new{
The systems we consider reflect a typical ML pipeline that addresses diverse real-world tasks, such as image classification and speech recognition.
These pipelines often involve model selection as a critical phase to ensure the generalization capabilities of the model.
On the other hand, this work demonstrates that poisoning model selection can also be used to influence the behaviors and characteristics of an ML algorithm.
Indeed, different domains often impose distinct demands on model characteristics.
For example, real-time applications like autonomous systems require low-latency models to ensure timely decision-making, while tasks that impact human lives or critical infrastructure prioritize accuracy and reliability above all.
Additionally, deployment scenarios further shape these requirements, as energy-efficient models are indispensable for battery-powered devices operating in resource-constrained environments.
}

\subsection{Threat Model}
\label{subsec:threat}

Our proposed attack aims to control the model selection phase by injecting maliciously crafted samples into the sole validation set.  
In this scope, the adversary goal is to let the model selection phase select the model architecture that suits the best target metric, named \textit{hijack metric} $m$. 
The hijack metric is arbitrary and can be chosen by the adversary.
Lastly, we assume that adversaries cannot modify the standard model selection routine, including the loss metric defined by the victim, the underlying code, or the model selection policy.
This means that with MOSHI, we inject samples that influence the loss function evaluation on the validation set, ensuring that the model best suited to the hijack metric achieves a lower loss than other models in the validation grid.
In this work, we explore different types of hijack metrics.

\paragraph{Assumptions.}
Suppose the victim has access to a dataset $\mathcal{S}$, divided in train $\mathcal{S}^{Train}$, validation $\mathcal{S}^{Val}$, and testing $\mathcal{S}^{Test}$ sets.
The former is used to train several model $h_{\mathfrak{c}}$ grouped in a model set identifiable with $\mathfrak{C}$,\footnote{$\mathfrak{c}$ represents the configuration of hyperparameters and $\mathfrak{C}$ the set of all possible configurations of hyperparameters.} the second to perform model selection, and the latter to test the best model performance.
The model selection phase will return the model $h_{\mathfrak{c}^*}$ with the lowest true error estimate on the validation set among those trained:
    \begin{equation}
        \label{best_val}
 h_{\mathfrak{c}^*} = \argmin_{h_{\mathfrak{c}} : \mathfrak{c} \in \mathfrak{C}} \mathcal{L}_{Val}(h_{\mathfrak{c}}, \mathcal{S}^{Val}).
    \end{equation}
In this context, we postulate that the adversary possesses access to the validation set $\mathcal{S}^{Val}$, enabling them to read and modify its data.
Consequently, our newly proposed adversarial family, MOSHI, exhibits characteristics akin to data poisoning, as it allows the attacker to alter the victim’s dataset.
In contrast with the traditional poisoning attack, our proposed attack operates solely at the validation set level, leaving the training set unaltered.

\paragraph{Adversary Knowledge.}
We now outline the adversary knowledge, detailing an attacker's information and capabilities and the limitations and constraints they face within our threat model.
We explore two distinct attack settings.
\begin{itemize}
    \item \textit{White-Box (WB) attack}: adversaries with full knowledge of the trained models $h_{\mathfrak{c}}, \;  \forall c \in \mathcal{C}$ (i.e., architectures, trained parameters and they can freely query the model for inference purposes). 
    \item \textit{Black-Box (BB) attack}: adversaries with knowledge limited to the model's architectures $\mathfrak{C}$ (i.e., the models that will be learned and tested during the model selection phase). 
\end{itemize}
Furthermore, in both scenarios, adversaries know $\mathcal{S}^{Train}$ but cannot tamper it. 
We again recall that -- for both white-box and black-box settings -- the attacker is limited to modifying the only validation set $\mathcal{S}^{Val}$.
A schematic representation of the considered threat model is depicted in Figure~\ref{fig:overview}.

\begin{figure*}[!htpb]
    \centering
    \includegraphics[width=0.65\linewidth]{figures/attack-overview.pdf}
    \caption{Schematic representation of the MOSHI threat model.}
    \label{fig:overview}
\end{figure*}

\paragraph{Stealthiness.}
Poisoning only the validation set allows all the trained models and their parameters not to be modified during the attack.
This means that once attacked, the model selection phase, which is carried out in an unmodified way, returns a model that has been regularly trained on clean data, and among all the ones trained is of higher hijack metric, which the adversary can choose.
This ensures that the selected model reveals no evidence of the attack, as its performance on the victim's loss function remains close to that of models selected using the original validation data.
However, it differs regarding the hijack metric, which is influenced to align with the attacker’s desired outcome.

\subsection{Motivations and Scenarios}
\label{subsec:motivation}

\new{
The execution of the MOSHI attack can have significant repercussions for various victims by leveraging dynamics similar to traditional poisoning and backdoor attacks~\cite{tian2022comprehensive, li2022backdoor}.
One compelling scenario involves the public dissemination of open-source datasets, which are often pre-structured into standard partitions (training, validation, and testing splits) to ensure the reproducibility of experimental results.
Platforms like Hugging Face\footnote{\url{https://huggingface.co/datasets}} and Kaggle\footnote{\url{https://www.kaggle.com/}} host datasets extensively adopted by the ML community, with some datasets recording hundreds of thousands of downloads.
This popularity underscores their critical role in research and development.
However, several studies have demonstrated the feasibility of poisoning these datasets and successfully degrade model performance~\cite{carlini2024poisoning}.
Furthermore, in MLaaS, service providers often depend on client-supplied datasets, significantly increasing the risk of attackers tampering with the data.
Another scenario arises when service providers themselves act maliciously.
By providing tampered validation sets engineered using MOSHI, they could force clients to select models with higher energy consumption while maintaining optimal generalization capabilities.
This covert manipulation would increase operational costs for users while benefiting the provider financially through inflated billing, all without raising immediate suspicion.
% Similarly, federated learning systems are vulnerable, as adversaries posing as clients can manipulate local data to disrupt global model selection, leading to suboptimal or biased federated models.
These attacks are particularly concerning in critical applications like healthcare or autonomous systems, where inefficiencies can directly impact safety.
Moreover, in edge computing scenarios, compromised validation can result in energy-intensive models, severely affecting battery-powered devices.
By highlighting these vulnerabilities, MOSHI underscores the urgent need for robust countermeasures against adversarial interference in model selection.
}

% Provider colluso con see stesso che da validation set con nostro attacco per fare si che l'energy consumption sia Maggiore e quindi ti billa di più, ma generalization è uguale quindi gg per l'utente


% \section{Model Selection Hijacking Adversarial Attack}
In this section, we introduce the MSHAA, discussing the motivation behind the attack, the methodology employed, and the tools developed to poison the victim’s validation set. We provide a detailed overview of how MSHAA manipulates the model selection process by strategically injecting adversarial samples, ensuring that the attacker's desired model is favored.
\subsection{Motivations}
%\lpasa{se siete d'accordo io useri ML e non AI. di fatto l'AI non ha bisogno della model selection (e chiamre AI il ML è abbstanza sbagliato a mio parare)}
Model selection is a critical step in the ML 
%AI 
development process due to its profound impact on the performance, efficiency, and generalizability of the resulting system. The choice of model determines how well the 
%AI 
ML model can learn from data, adapt to new information, and perform tasks accurately. Selecting an appropriate model involves evaluating various algorithms and architectures to identify the one that best fits the specific problem domain and dataset characteristics. This process ensures that the ML system can achieve optimal performance while minimizing overfitting and underfitting, thereby enhancing its robustness and reliability in real-world applications.
\par
Although extremely critical, works on AML did not consider \textit{if} and \textit{how} the model selection can be tampered with by adversaries for some advantage.  
In this work, we aim to answer this research question by presenting the \textit{Model Selection Hijacking Adversarial Attack} (MSHAA), a novel adversarial vector to ML applications. 
%
\subsection{Adversary Goal}
The goal of our proposed attack -- the \textit{Model Selection Hijacking Adversarial Attack} (MSHAA) -- is to control the model selection phase by injecting malicious crafted samples to the solely validation set.  
In this scope, the adversary goal is to let the model selection phase select the model architecture that suit the best a target metric, named \textit{hijack metric} $m$. 
The hijack metric is arbitrary and can be chosen by the adversary. 
%Last, we assume adversaries cannot alter the standard model selection routine, meaning they cannot alter the loss metric defined by the victim, nor the underling code.This means that with MSHAA we inject samples that alter the validation loss in such a way that the model that fits the best the hijack metric also result with a lower loss compared to the other models in the grid. 
Lastly, we assume that adversaries cannot modify the standard model selection routine, including the loss metric defined by the victim, the underlying code, or the model selection policy. This means that with MSHAA, we inject samples that influence the loss function evaluation on the validation set, ensuring that the model best suited to the hijack metric also achieves a lower loss compared to other models in the validation grid.
In this work, we explore different types of hijack metrics. 

%
\subsubsection{Threat Model}
Suppose the victim has access to a dataset $\mathcal{S}$, divided in train $\mathcal{S}^{Train}$, validation $\mathcal{S}^{Val}$, and testing $\mathcal{S}^{Test}$ sets.%\lpasa{occhio che sta divisione senza Resampling non è affrontata nel cap Background attualmente}
%
The former is used to train several model $h_{\mathfrak{c}}$ grouped in a models %grid 
set identifiable with $\mathfrak{C}$,\footnote{$\mathfrak{c}$ represents the configuration of hyperparameters and $\mathfrak{C}$ the set of all possible configurations of hyperparameters.} the second to perform model selection, and the latter to test the best model performance. %generalization capability.
%The model selection phase will then return the best model $h_{\mathfrak{c}^*}$, which is defined as the model, among the ones trained, that has the lowest validation loss: 
The model selection phase will return the model $h_{\mathfrak{c}^*}$ with the lowest true error estimate on the validation set among those trained:
    \begin{equation}
        \label{best_val}
 h_{\mathfrak{c}^*} = \argmin_{h_{\mathfrak{c}} : \mathfrak{c} \in \mathfrak{C}} \mathcal{L}_{Val}(h_{\mathfrak{c}}, \mathcal{S}^{Val}).
    \end{equation}

In this context, we postulate that the adversary possesses access to the validation set $\mathcal{S}^{Val}$, enabling them to read and modify its data. Consequently, our newly proposed adversarial family, MSHAA, exhibits characteristics akin to data poisoning, as it allows the attacker to alter the victim’s dataset.
In contrast with the traditional poisoning attack, our proposed MSHAA operates solely at the validation set level, leaving the training set unaltered. 
\par
We now outline the adversary knowledge, detailing the information and capabilities that an attacker possesses, as well as the limitations and constraints they face within our threat model.
We explore two distinct attack settings.
\begin{itemize}
    \item \textit{White-box attack}: adversaries with full knowledge of the  trained models $h_{\mathfrak{c}}, \;  \forall c \in \mathcal{C}$ (\textit{i.e., architectures, trained parameters} and they can freely query the model for inference purposes).  
    \item \textit{Black-box attack}: adversaries with knowledge limited to the model's architectures $\mathfrak{C}$ (\textit{i.e.,} the models that will be learned and tested during the model selection phase). 
\end{itemize}
Furthermore, in both scenarios adversaries know $\mathcal{S}^{Train}$ but they cannot tamper it. 
We again recall that -- for both white-box and black box settings -- the attacker is limited to modifying of the only validation set $\mathcal{S}^{Val}$.
A schematic representation of the considered threat model is depicted in Figure~\ref{fig:overview}.

\begin{figure*}[!htpb]
    \centering
    \includegraphics[width=0.65\linewidth]{figures/attack_overview.pdf}
    \caption{Schematic representation of the MSHAA threat model.}
    \label{fig:overview}
\end{figure*}


\subsubsection{Stealthiness}
Poisoning only the validation set allows all the trained models and their parameters not to be modified during the attack. This means that once attacked, the model selection phase, which is carried out in an unmodified way, returns a model that has been regularly trained on clean data, and among all the ones trained is of higher hijack metric, which can be chosen by the adversary. 
%This means that the returned model does not leak any clue about the attack having been carried out, but on the other hand, limits the effectiveness of the attack itself based on the hijack metric of the available models.
This ensures that the selected model reveals no evidence of the attack, as its performance on the victim's loss function remains close to that of models selected using the original validation data. However, it differs in terms of the hijack metric, which is influenced to align with the attacker’s desired outcome.

\subsubsection{Threat Victims}
With MSHAA attack execution might impact different victims through traditional poisoning and backdoor dynamics~\cite{tian2022comprehensive, li2022backdoor}. 
A pertinent example within the context of a threat model is the public dissemination of datasets, which are typically pre-structured into standard partitions (training, validation, and testing splits) to facilitate the reproducibility of experimental outcomes. In such a scenario, tampering with the validation set represents a significant vulnerability, as it undermines the integrity and reliability of the results. 
The attacker provides a compromised dataset to a service provider, to increase the inefficiency of its services by increasing, for instance, the energy consumption or the latency in the predictions, or, conversely, aiming to produce a stealthy poisoning attack by altering the victim's model performance. 
% Furthermore, in the context of the MLaaS paradigm, the attack can be three-fold:

% \begin{itemize}
%     \item The attacker provides a compromised dataset to a service provider, to increase the inefficiency of its services by increasing, for instance, the energy consumption or the latency in the predictions. 
%     \item The victim provides the dataset and a training strategy to the MLaaS provider, whose goal is to train a model with powerful hardware and returned the optimal model to the victim. Here, the MLaaS provider might tamper the 
% \end{itemize}

% \section{Generating Hijacking Samples}
\section{\new{Methodology}}
\label{sec:methodology}
%A critical step of the Model Selection Hijacking Adversarial Attack is the generation of adversarial hijacking samples to be inject to the validation set. 
%We now present a novel methodology to design and generate such samples. %\lpasa{occhio che qua sembra che l'unica novelty sia la generzione di esempi!}
% A crucial phase in the Model Selection Hijacking Adversarial Attack involves generating adversarial hijacking samples for injection into the validation set. Among the novelties introduced by this work, we present a new methodology specifically for designing and generating these samples.
\new{
The MOSHI attack operates uniquely by injecting and substituting data points in the validation set with data from $\mathcal{S}^{Val}_{pois}$, disrupting the critical model selection phase without altering the training process or parameters.
This set, which the attacker carefully generates, will be used for the model selection phase, which in turn will return a model $\tilde{h}_{\mathfrak{c}^*}$:
    \begin{equation}
        \label{best_poison}
 \tilde{h}_{\mathfrak{c}^*} = \argmin_{h_{\mathfrak{c}} : \mathfrak{c} \in \mathfrak{C}} \mathcal{L}_{Val}(h_{\mathfrak{c}}, \mathcal{S}^{Val}_{pois}).
    \end{equation}
The selected model $\tilde{h}_{\mathfrak{c}^*}$ is different from $h_{\mathfrak{c}^*}$, as now, the poisoned validation set no longer allows for selecting a better, more generalized, model, but selects one that has a configuration of hyper-parameters which maximizes the hijack metric, chosen by the adversary. 
Thus, a central aspect of this approach involves generating adversarial hijacking samples crafted explicitly for injection into the validation set.
Among the novelties introduced in this work, we present a specialized methodology for designing and generating these samples (Section~\ref{subsec:generation}) and the hijack metrics used in our study (Section~\ref{ssec.hm-theory}).
}

% \subsection{Overview}
% The goal of the adversary is to assume control of the MS phase by injecting and substituting data points in the validation set with data from $\mathcal{S}^{Val}_{pois}$. This set, which is carefully generated by the attacker, will be used for the model selection phase, which in turn will return a model $\tilde{h}_{\mathfrak{c}^*}$:
%     \begin{equation}
%         \label{best_poison}
%  \tilde{h}_{\mathfrak{c}^*} = \argmin_{h_{\mathfrak{c}} : \mathfrak{c} \in \mathfrak{C}} \mathcal{L}_{Val}(h_{\mathfrak{c}}, \mathcal{S}^{Val}_{pois}).
%     \end{equation}

% The selected model $\tilde{h}_{\mathfrak{c}^*}$ is different from $h_{\mathfrak{c}^*}$, as now, the poisoned validation set, no longer allows for selecting a better, more generalized, model, but selects one that has a configuration of hyper-parameters which maximizes the hijack metric, chosen by the adversary. 

% \subsection{Generative Process}
\subsection{\new{Adversarial Sample Generation}}
\label{subsec:generation}
\new{
Although our adversarial sample generation model is based on the Variational Auto Encoder (VAE) architecture (Section~\ref{subsub:vae}), we introduce a variation of the conditional VAE architecture designed for the generation of hijacking samples (Section~\ref{subsub:hvae}).
}
\subsubsection{Variational Auto Encoder (VAE)}
\label{subsub:vae}
We design our generative process using a Variational Auto Encoder (VAE)~\cite{kingma2013auto}, which is an extension of more traditional Autoencoders~\cite{hinton2006reducing}. VAE consists of two modules: first, an \textit{encoder} which learns a \textit{posterior} recognition model $q_{\phi}(z|x)$, encoding an input $x$ to a latent representation $z$; second, a \textit{decoder} that generates samples from the latent space $z$ via the likelihood model $p_{\theta}(x|z)$. $\phi$ and $\theta$ are learning parameters. 
In contrast with standard autoencoders, VAEs enforce a continuous prior distribution $p(z)$, usually set to the Gaussian. This forces the model to encode the entire input distribution to the latent code rather than memorizing single data points. 
Traditional VAEs are trained with the following loss:
\begin{equation} \small
% \begin{split}
    \mathcal{L}_{VAE}(\phi, \theta) = KL(q_{\phi}(z|x) || p(z)) %\\
    -\mathbb{E}_{q_{\phi}(z|x)}(\log p_{\theta}(x|z)), 
% \end{split}
\end{equation}
where $KL$ is the Kullback-Leibler divergence~\cite{kullback1951information} that is a regularizer to keep the posterior distribution close to the prior. The second term is a simple reconstruction loss. 
For the scope of this work, we utilize a Conditional VAE (CVAE) that augments the latent space with information about the true label of a given sample~\cite{sohn2015learning}.  
%
\subsubsection{Hijacking VAE}
\label{subsub:hvae}
We now introduce Hijacking VAE (HVAE), a variation of the more traditional CVAE that is specifically designed to generate hijacking samples to produce $\mathcal{S}^{Val}_{pois}$.
These samples are created in such a way that, when used for computing $\mathcal{L}_{Val}$, the lower the models' hijack metric, the more significant the increase of their validation loss, hence swaying the model selection phase into returning the model that has the highest hijack metric (which has been the least penalized).
We design the HVAE loss function as follows:
    \begin{equation}
        \label{lossMHVAE}
 \mathcal{L}_{\mathrm{HVAE}} = (\mathcal{L}_{\mathrm{rec}} + \mathcal{L}_{\mathrm{KLD}} - Hj_{cost}(\mathfrak{C})) ^ 2.
    \end{equation}
Here, the terms $\mathcal{L}_{\mathrm{rec}}$ and $\mathcal{L}_{\mathrm{KLD}}$ represents the reconstruction loss and the KL divergence, as in the traditional VAE. 
The novel factor of the loss is represented by the third term $Hj_{cost}(\mathfrak{C})$.
This is the pivotal factor of the attack, defined as follows (with $\Lambda = \mathfrak{C}$):

\begin{equation} \label{cost}
     Hj_{cost}(\mathfrak{C}) = \frac{1}{|\mathfrak{C}|}\sum_{\mathfrak{c} \in \mathfrak{C}} \alpha \cdot \mathcal{L}_{Val}(h_{\mathfrak{c}}, \mathcal{S}_{gen})
\end{equation}
with 
\begin{equation} \label{alpha}
 \alpha = \frac
     {\underset{\lambda \in \Lambda}{\max} \{m(h_{\lambda}, \mathcal{S}^{Val})\} - m(h_ {\mathfrak{c}}, \mathcal{S}^{Val})}
     {\underset{\lambda \in \Lambda}{\max} \{m(h_{\lambda}, \mathcal{S}^{Val})\} - \underset{\lambda \in \Lambda}{\min} \{m(h_{\lambda}, \mathcal{S}^{Val})\}}. 
\end{equation}
 
We now explain the rationale behind Equation~\ref{cost}, which is an average of scores that are assigned to each model $\mathfrak{c} \in \mathfrak{C}$. 
The coefficient $\alpha \in \mathbb{R}$ (see Equation~\ref{alpha}) is computed by normalizing the difference between the maximum hijack metric achievable by a model $h_{\lambda}$ with $\lambda \in \Lambda = \mathfrak{C}$ and the metric of the current model.
$\alpha$ yields higher penalties the lower the hijack metric of the model $h_\mathfrak{c}$, reaching 0 if the considered model has the highest metric. This value is fixed for each model and can be computed independently of the HVAE training.
On the opposite, the second term, $\mathcal{L}_{Val}$, assesses the quality of the generative process to produce effective hijacking samples, as it computes the loss of model $h_\mathfrak{c}$ over $S_{gen}$. It is therefore computed at HVAE training time. 
\par
Ideally, we intend to reward higher $Hj_{cost}$, as higher values imply higher losses toward those models with lower hijack metrics.
Therefore, in our loss function, we aim to maximize this value.  
During the training of the HVAE, by minimizing Equation~\ref{lossMHVAE}, we work toward:
\begin{itemize}
    \item diminishing the reconstruction loss $\mathcal{L}_{\mathrm{rec}}$, so that generated samples can resemble the original operations;
    \item diminishing the $\mathcal{L}_{\mathrm{KLD}}$ for obtaining a useful probability distribution;
    \item increasing the hijacking cost function $Hj_{cost}(\mathfrak{C})$. As the penalty value is fixed, by raising Equation~\ref{cost}, we aim at generating samples $\mathcal{S}_{gen}$, which increase the validation loss based on the magnitude of the penalty itself.
    Models with lower hijack metrics incur higher penalties, leading to increased validation loss on the generated samples. This ensures the samples are crafted to produce lower validation loss values for models with the highest hijack metrics.
    % Therefore, those models with lower hijack metrics will have higher penalties, which results in higher validation loss computed on the generated samples. This allows the creation of samples that, when used for evaluating the validation loss of a model, will return a lower value for the ones with the highest hijack metric.
\end{itemize}
% A graphical representation of how  $\mathcal{S}^{Val}_{pois}$ is generated, can be found in Figure~\ref{MHVAE}. 
% By training the HVAE with the objective function Equation~\ref{lossMHVAE}, it is possible to encode a distribution, that is unlike the input samples one -- usually learned by vanilla VAE -- as it governs the generation of samples such that, when injected in the validation set, they can provide a penalty on the validation loss of models at lower hijack metric.
% We report in Algorithm~\ref{alg.HVAE} the HVAE training procedure.
A graphical overview of $\mathcal{S}^{Val}_{pois}$ generation is shown in Figure~\ref{MHVAE}.
By training the HVAE with the objective function in Equation~\ref{lossMHVAE}, the model encodes a distribution distinct from the input samples’ usual one, enabling the generation of validation samples that penalize models with lower hijack metrics.
The HVAE training procedure is detailed in Algorithm~\ref{alg.HVAE}.

% \begin{figure*}[!htbp]
%     \footnotesize
%     \centering
%     \includesvg[width=.775\textwidth]{figures/MHVAE-v2.drawio}
%     \caption{Schematic representation of the generation process of $\mathcal{S}^{Val}_{pois}$. For simplicity, we reported samples from the MNIST dataset~\cite{lecun2010mnist}.}
%     % \caption{Schematic representation of $\mathcal{S}^{Val}_{pois}$ generation using MNIST samples~\cite{lecun2010mnist} for simplicity.}
%     \label{MHVAE}
% \end{figure*}

\begin{figure*}[!htbp] %% ARXIV
    \footnotesize
    \centering
    \includesvg[width=.775\textwidth]{figures/MHVAE-v2.drawio}
    \caption{Schematic representation of the generation process of $\mathcal{S}^{Val}_{pois}$. For simplicity, we reported samples from the MNIST dataset~\cite{lecun2010mnist}.}
    % \caption{Schematic representation of $\mathcal{S}^{Val}_{pois}$ generation using MNIST samples~\cite{lecun2010mnist} for simplicity.}
    \label{MHVAE}
\end{figure*}

\begin{algorithm}[H]
\footnotesize
    \caption{Hijack VAE Training Algorithm}
    \begin{algorithmic}[1]
        \State \textbf{Input:} HVAE model with random weights, training data $\mathcal{S}$, $\alpha_{\mathfrak{C}}$, $h_{\mathfrak{C}}$, number of epochs $epochs$
        \State \textbf{Output:} Trained HVAE model
        \For{$e \gets 1$ to $epochs$}
            \For{$\bm{x}$, $y$ in $\mathcal{S}$}  % are batches
                \State $\hat{\bm{x}} \gets $ HVAE.decode(HVAE.encode($\bm{x}$))  % reconstruct input
                \State rec\_loss $\gets \mathcal{L}_{\mathrm{rec}}(\bm{x}, \hat{\bm{x}})$  % reconstruction loss
                \State kl\_loss $\gets \mathcal{L}_{\mathrm{KLD}}(\mathrm{HVAE})$  % KLD loss
                \State $\hat{\bm{x}}_{gen} \gets$ HVAE.decode(gaussian\_noise)  % generate samples from randomly sampled noise
                \State generated\_val\_loss $\gets \mathcal{L}_{Val}(h_{\mathfrak{C}}, \hat{\bm{x}}_{gen})$  % validation loss of all knowm models on the generated samples
                \State hijack\_cost $\gets Hj_{cost}(\alpha_{\mathfrak{C}}, \mathrm{generated\_val\_loss})$  % compute hijack cost using the hijack cost penalty & the loss of the generated samples
                \State total\_loss $ \gets(\mathrm{rec\_loss + kl\_loss - hijack\_cost})^2 $  % obtain the total loss
                \State HVAE.backward\_propagation\_step(total\_loss)  % update weights
            \EndFor
        \EndFor
        \State \textbf{return} HVAE
    \end{algorithmic}
    \label{alg.HVAE}
\end{algorithm}

\subsection{Hijack Metric}
\label{ssec.hm-theory}
Generally, the purpose of a hijack metric $m$ is to produce damage to the target victim. 
\new{
We now introduce four distinct hijack metrics that impact an ML system in three different ways, i.e., generalization capabilities (Section~\ref{subsub:generalization}), latency (Section~\ref{subsub:latency}), and energy consumption (Section~\ref{subsub:energy}).
}
Note that MOSHI is not limited to such metrics, and future investigations might define different attack objectives. 

% \subsubsection{Weaken the Generalization Capabilities}
\subsubsection{\new{Generalization Capability Attack}}
\label{subsub:generalization}
This first intuitive hijack metric objective is to impact the victim model overall performance. 
Here, the objective of the attack under this metric is to choose a model that less generalizes to unseen data (e.g., test set), and therefore the result of an underfitting or overfitting training.
%Therefore, this case can be reconducted to the more traditional 
Therefore, this case can be considered a form of the more traditional \textit{model poisoning attack}~\cite{tian2022comprehensive}.
The metric $m$ -- that we named \textit{Generalization Metric} -- can simply compute the loss of a target model on an unseen dataset (\textit{e.g., validation set}). 
%
\subsubsection{Latency Attack}
\label{subsub:latency}
Increased latency in ML predictions can significantly impact the performance and usability of ML systems.
Higher latency leads to delayed responses, which can degrade user experience, particularly in real-time applications such as autonomous driving, financial trading, and interactive systems. Additionally, increased latency can hinder the efficiency of decision-making processes, as timely data processing is crucial for accurate and effective outcomes. This delay can also exacerbate the accumulation of errors, potentially compromising the reliability and accuracy of the ML model's predictions.
Therefore, an attacker might aim to induce the model selection to peak a model that results in slower predictions, on average, when deployed. 
The function $m$ -- that we named \textit{Latency Metric} -- can be designed by observing the time required by a target model to predict a set of unseen datasets (\textit{e.g., validation set}). 
%
\subsubsection{Energy Consumption Attack}
\label{subsub:energy}
Similarly to what is discussed in the motivation of the latency attack, increasing the overall energy consumption might lead to resource exhaustion. 
We inspire this metric based on the \textit{sponge attack}~\cite{shumailov2021sponge}. 
In our work, we consider two distinct metrics that measure energy consumption. 
\begin{itemize}
    \item \textit{Energy Consumption}: an estimation of the energy consumption of the model utilization that can be obtained through the OS energy consumption hosting such model. 
    \item \textit{$\ell_0$ norm}: the $\ell_0$ norm of the activations of the neurons in the network, obtained by summing the non-zero activations of each ReLU Layer in the model when it is processing a sample $\bm{x}$, then computing the mean for all samples $\bm{x} \in \mathcal{X}$. 
\end{itemize}
We opt to include this metric as \cite{cina2022energy} showed, there exists a strong link between the $\ell_0$ norm of a model and its energy consumption.
For instance, we report in Figure~\ref{l0_energy} the observed correlation between these two metrics in our experimental setting (which we will describe in the upcoming section).  

\begin{figure}[!htbp]
    \footnotesize
    \centering
    \includesvg[width = .8\linewidth]{figures/MNIST-normalized-energy-l0-v2}
    % \vspace{-15pt}
    \caption[Histogram comparing $\ell_0$ norm and energy consumption per layer.]{Histogram comparing $\ell_0$ norm and energy consumption per layer on FFNNs from 1 to 10 layers of 32 neurons, trained on MNIST dataset with a learning rate of 0.001.}
    \label{l0_energy}
\end{figure}


\subsection{White-box vs Black-box scenarios}
The HVAE requires knowledge about the target models, as described in Equation~\ref{cost} in the $Hj_{cost}$. Models in the grid are utilized for measuring their performance with the hijack metric and for understanding the quality of the dataset $S_{gen}$ produced by HVAE.
As we previously anticipated, in our work we consider a white-box and black-box case study. In the former, we assume the attacker has access to the exact models of the model grid. In the latter, the attacker has no such knowledge.
However, we assume that the attacker has knowledge about both training and validation sets. We can therefore leverage the \textit{adversarial transferability} of attacks. 
\par
Adversarial transferability in AML refers to the phenomenon where adversarial examples crafted to deceive one ML model can also deceive other models, even if they have different architectures or were trained on different datasets~\cite{demontis2019adversarial, alecci2023your}. This property is significant because it highlights the vulnerability of ML systems to attacks that are not specifically tailored to them, thereby posing a broader security risk.
\section{Evaluation}
We provide three sets of insights into this section, organised as \textit{findings (F*)}. We quantitatively study the effect of the adversarial and counterfactual perturbations on the performance of informal reasoners and autoformalisation methods. Then, we dive deeper into method variants. Finally, 
we analyse the nature of formalisation errors made by the models.

\subsection{Robustness Analysis}
\paragraph{\textbf{\emph{F1: Noise perturbations have a stronger effect on formalisation methods than informal \ac{LLM} reasoners.}}}
Table~\ref{tab:distraction_k4_formalisation} shows that, on average, the accuracy of both direct and \ac{CoT} informal reasoning remains between $73\%$ and $74\%$ in the face of added noise. While the autoformalisation method performs similarly to informal reasoners on the original dataset, its performance decreases between $4\%$ and $11\%$. The accuracy drops especially with logical (L) and tautological (T) distractions, whose logical language formats trick the \ac{LLM} into formalizing the noisy clauses. On the other hand, the linguistically complex and more natural sentences of encyclopedic distractions show a minor effect, suggesting that \acp{LLM} successfully avoids formalizing the more complicated sentences.

\paragraph{\textbf{\emph{F2: All \ac{LLM}-based reasoning methods suffer a drop for counterfactual perturbations.}}} % influence .}}}
Table~\ref{tab:distraction_k4_formalisation} shows that counterfactual statements cause a significant decrease in performance for both the informal reasoners and autoformalisation methods of between $12\%$ and $13\%$ on average. 
Moreover, this observation also holds for all tested models, i.e., none are robust towards counterfactual perturbations across every evaluated dimension. Even the strongest model, GPT 4o-mini, yields a performance of 63-68\%, which is relatively close to the random performance of 50\%. The high impact of counterfactual statements (the single ``not'' inserted) could be due to the inability of \acp{LLM} to overwrite prior knowledge with explicitly stated information or memorization of the answers. We study the error sources further in §\ref{subsec:errors}.  

\noindent \paragraph{\textbf{\emph{F3: Introducing multiple noise sentences has an effect only for logical distractions.}}}
We show the impact of introducing between one and four sentences for the two top-performing autoformalisation models in Figure~\ref{fig:length_distraction}. The figure shows similar trends with and without counterfactual perturbations.
As additional logical distractions are introduced, the model performance consistently decreases. Tautological (T) distractions lead to a decline in accuracy with a single disruptive sentence, yet adding more noise does not worsen the outcome. 
The tautological corpus introduces truth constants for all sentences as a persistent unseen logical construct. Given that this leads only to a decrease for a single occurrence, we can assume that a model can consistently handle the same unseen logical construct. In contrast, the logical corpus increases the chance of adding text, requiring new, previously unseen reasoning constructs for each added sentence. The impact of encyclopedic noise remains negligible, generalising F1 to $k$ sentences. Similarly, counterfactual perturbations remain much more effective for all settings, generalising F2.

\begin{table}[!t]
\small
\setlength{\modelspacing}{2pt}
\setlength{\tabcolsep}{1.7pt} % Default value: 6pt
\setlength{\belowrulesep}{4pt}
\begin{threeparttable}
    \centering
    \begin{tabular}{cc l r rrr @{\quad} rrrr}
\toprule
\multirow{2}{*}{} & \multirow{2}{*}{} & Reasoning & \multirow{2}{*}{O} & \multicolumn{3}{c}{Distraction} & \multicolumn{4}{c}{Counterfactual} \\
 & & Format & & E& L & T & $\text{O}_C$ & $\text{E}_C$& $\text{L}_C$ & $\text{T}_C$\\
\midrule
\multirow{6}{*}{\rotatebox{90}{Gemma-2}} & \multirow{3}{*}{\rotatebox{90}{9b}}
   & Informal (direct) & \textbf{0.78} & \textbf{0.80} & \textbf{0.79} & \textbf{0.77} & 0.58 & 0.52 & 0.50 & 0.59 \\
 & & Informal (CoT) & 0.72 & 0.78 & 0.73 & 0.76 & 0.61 & \textbf{0.57} & \textbf{0.60} & \textbf{0.66} \\
 & & Formal (FOL) & 0.62 & 0.58 & 0.52 & 0.53 & \textbf{0.63} & 0.52 & 0.46 & 0.46 \\[\modelspacing]
\cmidrule{2-11}
 & \multirow{3}{*}{\rotatebox{90}{27b}} 
   & Informal (direct) & 0.71 & 0.69 & \textbf{0.66} & \textbf{0.68} & 0.59 & 0.51 & 0.54 & 0.59 \\
 & & Informal (CoT) & 0.66 & 0.65 & 0.64 & 0.63 & 0.62 & 0.58 & \textbf{0.62} & \textbf{0.64} \\
 & & Formal (FOL) & \textbf{0.74} & \textbf{0.74} & 0.61 & 0.61 & \underline{\textbf{0.72}} & \underline{\textbf{0.67}} & 0.58 & 0.51 \\[\modelspacing]
\midrule
\multirow{6}{*}{\rotatebox{90}{Mistral}} & \multirow{3}{*}{\rotatebox{90}{7B}} 
   & Informal (direct) & 0.77 & \textbf{0.77} & 0.75 & \textbf{0.79} & \textbf{0.63} & \textbf{0.54} & \textbf{0.54} & \textbf{0.66} \\
 & & Informal (CoT) & \textbf{0.79} & 0.75 & \textbf{0.77} & 0.78 & 0.55 & 0.52 & \textbf{0.54} & 0.58 \\
 & & Formal (FOL) & 0.62 & 0.58 & 0.54 & 0.57 & 0.50 & \textbf{0.54} & 0.51 & 0.52 \\[\modelspacing]
\cmidrule{2-11}
 & \multirow{3}{*}{\rotatebox{90}{Small}} 
   & Informal (direct) & \textbf{0.77} & \textbf{0.76} & \textbf{0.76} & \textbf{0.75} & 0.61 & 0.51 & 0.56 & 0.59 \\
 & & Informal (CoT) & 0.72 & 0.72 & 0.72 & 0.71 & \textbf{0.62} & \textbf{0.59} & \textbf{0.62} & \textbf{0.68} \\
 & & Formal (FOL) & 0.68 & 0.59 & 0.53 & 0.64 & 0.54 & 0.55 & 0.49 & 0.51 \\[\modelspacing]
\midrule
\multirow{6}{*}{\rotatebox{90}{Llama-3.1}} & \multirow{3}{*}{\rotatebox{90}{8B}} 
   & Informal (direct) & 0.63 & 0.61 & 0.64 & 0.66 & 0.61 & \textbf{0.62} & 0.59 & 0.61 \\
 & & Informal (CoT) & 0.73 & \textbf{0.73} & \textbf{0.71} & \textbf{0.72} & \textbf{0.62} & 0.59 & \textbf{0.61} & \textbf{0.65} \\
 & & Formal (FOL) & \textbf{0.77} & 0.71 & 0.63 & 0.52 & 0.60 & 0.58 & 0.55 & 0.52 \\[\modelspacing]
\cmidrule{2-11}
 & \multirow{3}{*}{\rotatebox{90}{70B}} 
   & Informal (direct) & 0.77 & 0.74 & 0.74 & 0.73 & 0.62 & 0.53 & 0.56 & 0.64 \\
 & & Informal (CoT) & \textbf{0.78} & \textbf{0.75} & \textbf{0.76} & \textbf{0.76} & 0.64 & 0.61 & \textbf{0.66} & \underline{\textbf{0.73}} \\
 & & Formal (FOL) & 0.74 & 0.73 & 0.71 & 0.71 & \textbf{0.66} & \textbf{0.62} & 0.59 & 0.57 \\[\modelspacing]
 \midrule
\multirow{3}{*}{\rotatebox{90}{GPT}} & \multirow{3}{*}{\rotatebox{90}{4o-mini}} 
   & Informal (direct) & 0.78 & 0.77 & 0.79 & 0.79 & 0.64 & 0.61 & 0.61 & 0.63 \\
 & & Informal (CoT) & 0.80 & 0.80 & \underline{\textbf{0.81}} & \underline{\textbf{0.82}} & \textbf{0.68} & \textbf{0.63} & \underline{\textbf{0.68}} & \textbf{0.64} \\
 & & Formal (FOL) & \underline{\textbf{0.84}} & \underline{\textbf{0.82}} & 0.73 & 0.79 & 0.63 & 0.62 & 0.57 & 0.54 \\[\modelspacing]
 \midrule
\multicolumn{2}{c}{\multirow{3}{*}{\textbf{Avg}}} 
 & Informal (direct) & 0.74 & 0.73 & 0.73 & 0.73 & 0.61 & 0.55 & 0.56 & 0.62 \\
 & & Informal (CoT) & 0.74 & 0.74 & 0.73 & 0.74 & 0.62 & 0.58 & 0.62 & 0.65 \\
  & & Formal (FOL) & 0.72 & 0.68 &	0.61 & 0.62 & 0.61 & 0.59 & 0.54 & 0.52 \\
\bottomrule
\end{tabular}
\caption{Accuracies of informal and autoformalisation-based deductive reasoners. The best overall model per dataset is underlined; the best model version is marked in bold.}
\label{tab:distraction_k4_formalisation}
\end{threeparttable}
\end{table} 

\begin{figure}[!t]
    \centering
    \scriptsize
    \begin{tikzpicture}
        \begin{axis}[name=gpt,
            title={GPT-4o-mini},
            width=0.6\linewidth,
            height=0.6\linewidth,
            xlabel={\# Noise sentences},
            ylabel={Accuracy},
            xmin=-0.1, xmax=4.1,
            ymin=0.5, ymax=0.9,
            xtick={1,2,4},
            ytick={0.55, 0.6, 0.65, 0.75, 0.8, 0.85},
            title style={yshift=-0.6em},
            legend style={at={(1,-0.15)},
	           anchor=north,legend columns=-1},
            x label style={at={(axis description cs:1,-0.05)},anchor=north},
            y label style={at={(axis description cs:-0.15,0.5)},anchor=south},
            ymajorgrids=true,
            grid style=dashed,
        ]
            \addplot[color=blue, mark=square,]
                coordinates {
                (0,0.848076939582825)(1,0.823076903820038)(2,0.826923072338104)(4,0.821153819561005)
                };
            \addplot[color=red, mark=triangle,]
                coordinates {
                (0,0.848076939582825)(1,0.817307710647583)(2,0.801923096179962)(4,0.759615361690521)
                };
            \addplot[color=green, mark=diamond,] 
                coordinates {
                (0,0.848076939582825)(1,0.767307698726654)(2,0.769230782985687)(4,0.803846180438995)
                };
            \addplot[color=blue, mark=square*] 
                coordinates {
                (0,0.627777755260468)(1,0.622222244739533)(2,0.600000023841858)(4,0.633333325386047)
                };
            \addplot[color=red, mark=triangle*,] 
                coordinates {
                (0,0.627777755260468)(1,0.611111104488373)(2,0.611111104488373)(4,0.594444453716278)
                };
            \addplot[color=green, mark=diamond*,] 
                coordinates {
                (0,0.627777755260468)(1,0.572222232818604)(2,0.538888871669769)(4,0.555555582046509)
                };
                \legend{E,L,T,$\text{E}_C$, $\text{L}_C$ , $\text{T}_C$}
        \end{axis}

        \begin{axis}[name=llama, at={($(gpt.east)+(0.1cm,0)$)},anchor=west,
            title={Llama 3.1 70b},
            width=0.6\linewidth,
            height=0.6\linewidth,
            xmin=-0.1,, xmax=4.1,
            ymin=0.5, ymax=0.9,
            xtick={1,2,4},
            ytick={0.55, 0.6, 0.65, 0.75, 0.8, 0.85},
            title style={yshift=-0.6em},
            yticklabel=\empty,
            ymajorgrids=true,
            grid style=dashed,
        ]
            \addplot[color=blue, mark=square,]
                coordinates {
                (0,0.838461518287659)(1,0.817307710647583)(2,0.805769205093384)(4,0.817307710647583)
                };
            \addplot[color=red, mark=triangle,]
                coordinates {
                (0,0.838461518287659)(1,0.819230794906616)(2,0.803846180438995)(4,0.771153867244721)
                };
            \addplot[color=green, mark=diamond,]
                coordinates {
                (0,0.838461518287659)(1,0.803846180438995)(2,0.807692289352417)(4,0.805769205093384)
                };
            \addplot[color=blue, mark=square*]
                coordinates {
                (0,0.627777755260468)(1,0.622222244739533)(2,0.577777802944183)(4,0.594444453716278)
                };
            \addplot[color=red, mark=triangle*,]
                coordinates {
                (0,0.627777755260468)(1,0.583333313465118)(2,0.561111092567444)(4,0.577777802944183)
                };
            \addplot[color=green, mark=diamond*,]
                coordinates {
                (0,0.627777755260468)(1,0.627777755260468)(2,0.566666662693024)(4,0.577777802944183)
                };
        \end{axis}
    \end{tikzpicture}
    \caption{Influence of the number of noisy sentences for FOL.}
    \label{fig:length_distraction}
\end{figure}



\subsection{Impact of Method Design}
\paragraph{\textbf{\emph{F4: \ac{CoT} prompting is most impactful when both noise and counterfactual perturbations are applied.}}}
The accuracies for the individual \acp{LLM} in Table~\ref{tab:distraction_k4_formalisation} show that the impact of \ac{CoT} is negligible for noise-only datasets (first four columns). Meanwhile, the benefit from \ac{CoT} is most pronounced in the datasets that combine noise and counterfactual perturbations.
The better-performing informal prompting strategy for a model remains stable for all types of distractions. Still, the decline in performance due to counterfactuals leads to a less consistent preference for a specific prompting style.

\paragraph{\textbf{\emph{F5: The best-performing grammar differs per model and is unstable across data versions.}}}

The evaluation of different logical forms for formal \ac{LLM}-based reasoning in Table~\ref{tab:distraction_k4_logical_form} shows the preference of some models for specific syntactic formats.
Llama 3.1 70B has a considerable improvement of $12\%$ with TPTP syntax on the original set, while Llama 3.1 8B benefits from the R-FOL syntax. However, all grammars show a declining accuracy trend and increased syntax errors for noise perturbations, where the best grammar loses its advantage over the rest. 
When comparing the grammars on the counterfactual partitions, we observe that TPTP is consistently more robust than the standard first-order logic grammar. Here, GPT 4o-mini shows a reduction from $O$ to $O_C$ of $20\%$ for FOL and only $12\%$ for the TPTP grammar. Since this does not correlate with fewer syntax errors, the formalisation in TPTP prevents semantical errors for counterfactual premises. 
A positive reading of these results, especially the minor differences between FOL and R-FOL, is that autoformalisation \acp{LLM} can adapt to the grammar syntax prescribed in the prompt without further loss in performance.

\begin{table}[!t]
\small
\setlength{\modelspacing}{2pt}
\setlength{\tabcolsep}{1.7pt} % Default value: 6pt
\setlength{\belowrulesep}{4pt}
\begin{threeparttable}
    \centering
    \begin{tabular}{cc l r rrr @{\quad} rrrr}
\toprule
\multirow{2}{*}{} & \multirow{2}{*}{} & Grammar & \multirow{2}{*}{O} & \multicolumn{3}{c}{Distraction} & \multicolumn{4}{c}{Counterfactual} \\
 & & Syntax & & E& L & T & $\text{O}_C$ & $\text{E}_C$& $\text{L}_C$ & $\text{T}_C$\\
\midrule
\multirow{6}{*}{\rotatebox{90}{Llama-3.1}} & \multirow{3}{*}{\rotatebox{90}{8B}} 
   & FOL & 0.77 & \textbf{0.71} & 0.61 & \textbf{0.53} & 0.58 & \textbf{0.55} & 0.52 & \textbf{0.56} \\
 & & R-FOL & \textbf{0.78} & 0.69 & \textbf{0.62} & \textbf{0.53} & 0.58 & \textbf{0.55} & \textbf{0.54} & 0.52 \\
 & & TPTP & 0.73 & 0.67 & 0.55 & 0.51 & \textbf{0.68} & 0.54 & 0.46 & 0.51 \\[\modelspacing]
\cmidrule{2-11}
 & \multirow{3}{*}{\rotatebox{90}{70B}} 
   & FOL & 0.76 & 0.73 & 0.71 & \textbf{0.72} & 0.67 & 0.57 & 0.63 & 0.56 \\
 & & R-FOL & 0.76 & 0.73 & 0.67 & 0.71 & 0.64 & 0.57 & 0.53 & 0.64 \\
 & & TPTP & \underline{\textbf{0.88}} & \underline{\textbf{0.84}} & \underline{\textbf{0.81}} & \textbf{0.72} & \underline{\textbf{0.81}} & \underline{\textbf{0.68}} & \underline{\textbf{0.67}} & \underline{\textbf{0.68}} \\[\modelspacing]
\midrule
\multirow{3}{*}{\rotatebox{90}{GPT}} & \multirow{3}{*}{\rotatebox{90}{4o-mini}} 
   & FOL & \textbf{0.84} & \textbf{0.82} & \textbf{0.72} & \underline{\textbf{0.78}} & 0.64 & \textbf{0.63} & \textbf{0.61} & 0.51 \\
 & & R-FOL & \textbf{0.84} & 0.77 & 0.70 & \underline{\textbf{0.78}} & \textbf{0.72} & 0.56 & 0.54 & \textbf{0.63} \\
 & & TPTP & 0.83 & \textbf{0.82} & 0.71 & 0.71 & 0.69 & \textbf{0.63} & 0.57 & 0.57 \\
\bottomrule
\end{tabular}
\caption{Accuracies of different formalisation grammars for autoformalisation.}
\label{tab:distraction_k4_logical_form}
\end{threeparttable}
\end{table} 

\paragraph{\textbf{\emph{F6: Feedback does not help \acp{LLM} self-correct to mitigate robustness issues.}}}
\autoref{tab:distraction_k4_feedback} shows the results with different error recovery mechanisms. The results indicate that no feedback strategy emerges as a winner in the different datasets. 
All feedback variants reduce syntax errors for noise perturbations, but given the lack of a consistent increase in accuracy, the corrected formalisations are most likely to contain semantic errors still. 
The type of feedback message only has a minor influence on correcting syntax errors, whereas Llama 3.1 70b and GPT 4o-mini correct slightly more syntax errors with specific error messages. This finding aligns with \cite{huang2023large}, who also found that \acp{LLM} cannot consistently self-correct their reasoning after receiving relevant feedback.

\begin{table}[!ht]
\small
\setlength{\modelspacing}{2pt}
\setlength{\tabcolsep}{1.7pt} % Default value: 6pt
\setlength{\belowrulesep}{4pt}
\begin{threeparttable}
    \centering
    \begin{tabular}{cc l r rrr @{\quad} rrrr}
\toprule
\multirow{2}{*}{} & \multirow{2}{*}{} & \multirow{2}{*}{Feedback} & \multirow{2}{*}{O} & \multicolumn{3}{c}{Distraction} & \multicolumn{4}{c}{Counterfactual} \\
 & & & & E& L & T & $\text{O}_C$ & $\text{E}_C$& $\text{L}_C$ & $\text{T}_C$\\
\midrule
\multirow{8}{*}{\rotatebox{90}{Llama-3.1}} & \multirow{4}{*}{\rotatebox{90}{8B}} 
   & No recovery & 0.77 & \textbf{0.72} & 0.62 & 0.53 & 0.59 & 0.58 & 0.56 & \textbf{0.56} \\
 & & Error type & \textbf{0.79} & 0.71 & 0.63 & \textbf{0.56} & \textbf{0.66} & 0.54 & 0.52 & 0.51 \\
 & & Error message & 0.78 & 0.71 & \textbf{0.67} & 0.55 & 0.59 & 0.53 & \underline{\textbf{0.64}} & 0.49 \\
 & & Warning & 0.74 & 0.66 & 0.58 & 0.55 & 0.55 & \textbf{0.60} & 0.49 & 0.49 \\[\modelspacing]
\cmidrule{2-11}
 & \multirow{4}{*}{\rotatebox{90}{70B}} 
   & No recovery & \textbf{0.77} & \textbf{0.72} & \textbf{0.73} & 0.71 & \textbf{0.64} & 0.59 & \textbf{0.61} & 0.56 \\
 & & Error type & 0.72 & 0.70 & 0.72 & \textbf{0.73} & 0.62 & 0.56 & 0.60 & 0.58 \\
 & & Error message & 0.71 & 0.70 & \textbf{0.73} & 0.71 & \textbf{0.64} & 0.59 & 0.54 & \underline{\textbf{0.64}} \\
 & & Warning & 0.69 & \textbf{0.72} & 0.72 & 0.72 & 0.62 & \underline{\textbf{0.65}} & \textbf{0.61} & 0.63 \\[\modelspacing]
\midrule
\multirow{4}{*}{\rotatebox{90}{GPT}} & \multirow{4}{*}{\rotatebox{90}{4o-mini}} 
   & No recovery & \underline{\textbf{0.84}} & \underline{\textbf{0.82}} & 0.73 & 0.79 & 0.64 & \textbf{0.62} & 0.56 & \textbf{0.56} \\
 & & Error type & 0.83 & 0.79 & 0.74 & 0.76 & 0.67 & 0.57 & 0.56 & \textbf{0.56} \\
 & & Error message & \underline{\textbf{0.84}} & 0.78 & \underline{\textbf{0.77}} & \underline{\textbf{0.80}} & 0.62 & 0.59 & 0.56 & \textbf{0.56} \\
 & & Warning & \underline{\textbf{0.84}} & 0.75 & 0.73 & 0.76 & \underline{\textbf{0.70}} & 0.61 & \textbf{0.61} & 0.55 \\
 \bottomrule
\end{tabular}
\caption{Accuracies of error recovery strategies.}
\label{tab:distraction_k4_feedback}
\end{threeparttable}
\end{table} 

\subsection{Error Analysis}
\label{subsec:errors}
\paragraph{\textbf{\emph{F7: Autoformalisation increases syntax errors for noise perturbations.}}}
The low performance for noise perturbations correlates with more syntax errors for all models and distraction categories (cf. execution rates in Table~\ref{tab:appendix_k4_formalisation_exec}). The three worst-performing models (both Mistral models, Gemma-2 9b) generate, at best, for $37\%$  and, at worst, for only $4\%$ of the samples, a valid logical form.
Gemma-2 9b and Llama3.1 8b produce more syntax errors than the larger counterparts, suggesting that larger models are more robust towards noise perturbations. 
The accuracy of syntactically valid samples is higher than the informal reasoning methods for most distractions (Table~\ref{tab:appendix_k4_formalisation_vacc}), motivating informal reasoning as a backup strategy for formal reasoning. The error message feedback reveals two common syntax errors: 1) errors by models with an initial low execution rate exhibit issues with the template structure, including using incorrect keywords or adding conversational phrases;
2) perturbation-related errors, the most common of which is using undefined truth constants as part of tautological distractions. 

\paragraph{\textbf{\emph{F8: Autoformalisation increases semantic errors for counterfactuals.}}}
Unlike the introduced noise, counterfactual perturbations do not lead to more syntax errors. The execution rate in Table~\ref{tab:appendix_k4_formalisation_exec} is stable or improves for counterfactuals. However, we see a drop in accuracy for the counterfactual column $\text{O}_C$ in Table~\ref{tab:distraction_k4_formalisation} and can conclude that the number of logical forms with semantic errors has to increase. This suggests that the introduced negation is not correctly formalised. Looking at the warnings generated by the feedback mechanism, for GPT 4o-mini, $161$ warning messages are generated on the unperturbed data. $54$ of these were fixed with a single iteration. Not considering predicates and individuals as part of the context is the most frequent warning across all models. 
% \usepackage[hidelinks,colorlinks=true,citecolor=blue!50!black,linkcolor=black,urlcolor=green!50!black]{hyperref}
\usepackage{url}
\usepackage{amsmath,amsfonts,amssymb,amsthm}
\usepackage{graphicx}

%glossary
\usepackage[acronym]{glossaries}
\newacronym{SV}{SV}{Shapley Value}
\newacronym{SII}{SII}{Shapley Interaction Index}
\newacronym{k-SII}{$k$-SII}{$k$-Shapley Value}
\newacronym{SI}{SI}{Shapley Interaction}
\newacronym{MI}{MI}{Möbius Interaction}
\newacronym{STII}{STII}{Shapley Taylor Interaction Index}
\newacronym{FSII}{FSII}{Faithful Shapley Interaction Index}

% Commands for fixing margins
\newcommand{\fix}{\marginpar{FIX}}
\newcommand{\new}{\marginpar{NEW}}

% Commands for mathematical notation
\newcommand{\cX}{\mathcal{X}}
\newcommand{\cY}{\mathcal{Y}}
\newcommand{\cH}{\mathcal{H}}
\renewcommand{\vec}[1]{\boldsymbol{#1}}
\newcommand{\cN}{\mathcal{N}}

% math notation
%\newtheorem{definition}{Definition}

% Optional math commands from https://github.com/goodfeli/dlbook_notation.
%%%%% NEW MATH DEFINITIONS %%%%%

% \usepackage{amsmath,amsfonts,bm}
\usepackage{amsmath,amsfonts}

\usepackage{pifont}


\newcommand{\R}{\mathbb{R}}


\def\va{{\mathbf{a}}}
\def\vg{{\mathbf{g}}}

% Sets
\def\sR{\mathbb{R}}
\def\sC{\mathbb{C}}
\def\sZ{\mathbb{Z}}
\def\sN{\mathbb{N}}
\def\sQ{\mathbb{Q}}

\def\sS{\mathcal{S}}



% Vectors
\def\vzero{{\mathbf{0}}}
\def\vone{{\mathbf{1}}}
\def\vmu{{\mathbf{\mu}}}
\def\vtheta{{\mathbf{\theta}}}
\def\va{{\mathbf{a}}}
\def\vb{{\mathbf{b}}}
\def\vc{{\mathbf{c}}}
\def\vd{{\mathbf{d}}}
\def\ve{{\mathbf{e}}}
\def\vf{{\mathbf{f}}}
\def\vg{{\mathbf{g}}}
\def\vh{{\mathbf{h}}}
\def\vi{{\mathbf{i}}}
\def\vj{{\mathbf{j}}}
\def\vk{{\mathbf{k}}}
\def\vl{{\mathbf{l}}}
\def\vm{{\mathbf{m}}}
\def\vn{{\mathbf{n}}}
\def\vo{{\mathbf{o}}}
\def\vp{{\mathbf{p}}}
\def\vq{{\mathbf{q}}}
\def\vr{{\mathbf{r}}}
\def\vs{{\mathbf{s}}}
\def\vt{{\mathbf{t}}}
\def\vu{{\mathbf{u}}}
\def\vv{{\mathbf{v}}}
\def\vw{{\mathbf{w}}}
\def\vx{{\mathbf{x}}}
\def\vy{{\mathbf{y}}}
\def\vz{{\mathbf{z}}}
\def\vzeta{{\mathbf{\zeta}}}

% Matrix
\def\mA{{\mathbf{A}}}
\def\mB{{\mathbf{B}}}
\def\mC{{\mathbf{C}}}
\def\mD{{\mathbf{D}}}
\def\mE{{\mathbf{E}}}
\def\mF{{\mathbf{F}}}
\def\mG{{\mathbf{G}}}
\def\mH{{\mathbf{H}}}
\def\mI{{\mathbf{I}}}
\def\mJ{{\mathbf{J}}}
\def\mK{{\mathbf{K}}}
\def\mL{{\mathbf{L}}}
\def\mM{{\mathbf{M}}}
\def\mN{{\mathbf{N}}}
\def\mO{{\mathbf{O}}}
\def\mP{{\mathbf{P}}}
\def\mQ{{\mathbf{Q}}}
\def\mR{{\mathbf{R}}}
\def\mS{{\mathbf{S}}}
\def\mT{{\mathbf{T}}}
\def\mU{{\mathbf{U}}}
\def\mV{{\mathbf{V}}}
\def\mW{{\mathbf{W}}}
\def\mX{{\mathbf{X}}}
\def\mY{{\mathbf{Y}}}
\def\mZ{{\mathbf{Z}}}
\def\mBeta{{\mathbf{\beta}}}
\def\mPhi{{\mathbf{\Phi}}}
\def\mLambda{{\mathbf{\Lambda}}}
\def\mSigma{{\mathbf{\Sigma}}}


% Expectation
% \def\eE{\mathop{\mathbb{E}}\limits}
\def\eE{\mathbb{E}}

% Probability
\def\pP{\mathbb{P}}

% Tilde
\def\tf{\tilde{f}}
\def\tS{\tilde{S}}
\def\wtF{\widetilde{\mathcal{F}}}
\def\whR{\widehat{R}}
\def\tvx{\tilde{\mathbf{x}}}
\def\ty{\tilde{y}}


\def\defeq{\overset{\textup{def}}{=}}
% \def\defeq{\overset{.}{=}}
\def\defone{\overset{\text{\ding{172}}}{=}}
\def\deftwo{\overset{\text{\ding{173}}}{=}}
\def\leqone{\overset{\text{\ding{172}}}{\leq}}
\def\leqtwo{\overset{\text{\ding{173}}}{\leq}}
\def\leqthree{\overset{\text{\ding{174}}}{\leq}}
\def\leqfour{\overset{\text{\ding{175}}}{\leq}}
\def\eqone{\overset{\text{\ding{172}}}{=}}
\def\eqtwo{\overset{\text{\ding{173}}}{=}}
\def\eqthree{\overset{\text{\ding{174}}}{=}}
\def\eqfour{\overset{\text{\ding{175}}}{=}}
\def\geqfive{\overset{\text{\ding{176}}}{\geq}}
% \begin{table}[ht!]
\centering
\caption{\textbf{Super Resolution Performance Results.} Our proposed WGAN EEG Spatial Upsampling method significantly outperforms a baseline of Bicubic Interpolation commonly used in EEG upsampling pipelines.}
\label{tab:results}
\resizebox{0.8\linewidth}{!}{%
\begin{tabular}{@{}cccccc@{}}
\toprule
\multirow{2}{*}{\textbf{Dataset}} & \multirow{2}{*}{\textbf{Scale}} & \multicolumn{2}{c}{\textbf{Bicubic}} & \multicolumn{2}{c}{\textbf{WGAN}} \\ \cmidrule(l){3-6} 
                      &   & \textbf{MSE} & \textbf{MAE} & \textbf{MSE}    & \textbf{MAE}   \\
\toprule
\multirow{2}{*}{Val}  & 2 & 3.71E7       & 3.89E3       & \textbf{2.01E3} & \textbf{24.38} \\
                      & 4 & 7.23E7       & 6.42E3       & \textbf{8.53E3} & \textbf{63.83} \\
\midrule
\multirow{2}{*}{Test} & 2 & 3.75E7       & 3.91E3       & \textbf{2.06E3} & \textbf{24.66} \\
                      & 4 & 7.30E7       & 6.45E3       & \textbf{8.68E3} & \textbf{64.39} \\
\bottomrule
\end{tabular}%
}
\end{table}
\begin{table*}
  [t]
  \centering
  \resizebox{\textwidth}{!}{%
  \begin{tabular}{cccccccccccc}
    \toprule \multicolumn{2}{c}{Components}                                                             & \multicolumn{5}{c}{Re-executability Rate (\%)} & \multicolumn{5}{c}{Readability (\#)} \\
    \cmidrule(lr){1-2} \cmidrule(lr){3-7} \cmidrule(lr){8-12}        \hspace{8pt}\labelemoji\hspace{8pt}                                                                & \hspace{8pt}\toolemoji\hspace{8pt}                                      & O0                                 & O1             & O2             & O3             & AVG            & O0             & O1             & O2             & O3             & AVG            \\
    \hline
    \rowcolor[rgb]{0.93,0.93,0.93}\multicolumn{12}{c}{\textbf{Initialize with LLM4Decompile-End-6.7B~\citep{llm4decompile}}}   \\
    \xmark                                                                                              & \xmark                                    & 69.51                              & 46.95          & 50.61          & 46.34          & 53.35          & 3.98 & 3.41 & 3.44 & 3.38 & 3.55 \\
    \cmark                                                                                              & \xmark                                    & 75.61                              & 50.61          & 50.00          & 50.00          & 56.55          & 4.01 & 3.44 & 3.39 & \textbf{3.49} & 3.58 \\
    \xmark                                                                                              & \cmark                                    & 83.54                     & \textbf{56.10}          & 51.22          & 50.61 & 60.37 & 4.05 & 3.51 & 3.51 & 3.42 & 3.62 \\
    \cmark                                                                                              & \cmark                                    & \textbf{85.37}                            & \textbf{56.10}                     & \textbf{51.83} & \textbf{52.43}          & \textbf{61.43} & \textbf{4.13} & \textbf{3.60} & \textbf{3.54} & \textbf{3.49} & \textbf{3.69} \\

    \rowcolor[rgb]{0.93,0.93,0.93}\multicolumn{12}{c}{\textbf{Initialize with Deepseek-Coder-6.7B-base~\citep{deepseekcoder}}} \\
    \xmark                                                                                              & \xmark                                    & 59.15                              & 35.98          & 39.02          & 37.80          & 42.99          & 3.71 & 3.05 & 3.16 & 3.05 & 3.24 \\
    \cmark                                                                                              & \xmark                                    & 66.46                              & 41.46          & 38.41          & 36.59          & 45.73          & 3.76 & 3.17 & \textbf{3.21} & 3.08 & 3.31 \\
    \xmark                                                                                              & \cmark                                    & 70.73                              & 39.63          & 39.02          & 40.24          & 47.41          & 3.90 & 3.17 & 3.08 & 3.11 & 3.31 \\
    \cmark                                                                                              & \cmark                                    & \textbf{79.88}                     & \textbf{45.73} & \textbf{43.90} & \textbf{42.68} & \textbf{53.05} & \textbf{3.96} & \textbf{3.21} & 3.18 & \textbf{3.19} & \textbf{3.38} \\
    \bottomrule
  \end{tabular}%
  }
  \caption{The ablation study of different methods across four optimization levels
  (O0, O1, O2, O3), as well as their average scores (AVG). The results in bold represent the optimal performance. The ~\labelemoji~ and ~\toolemoji~ means Relabedling and Function Call. \textbf{Bold} denotes the best performance.}
  \label{tab:ablation}
\end{table*}
\section*{Conclusion}
This paper aims to enhance our understanding of the computational complexity of computing various Shapley value variants. We found that for various ML models --- including decision trees, regression tree ensembles, weighted automata, and linear regression --- both local and global interventional and baseline SHAP can be computed in polynomial time under HMM modeled distributions. This extends popular algorithms, such as TreeSHAP, beyond their empirical distributional scope. We also establish strict complexity gaps between the various SHAP variants (baseline, interventional, and conditional) and prove the intractability of computing SHAP for tree ensembles and neural networks in simplified scenarios. Overall, we present SHAP as a versatile framework whose complexity depends on four key factors: \begin{inparaenum}[(i)] \item model type, \item SHAP variant, \item distribution modeling approach, \item and local vs. global explanations\end{inparaenum}. We believe this perspective provides deeper insight into the computational complexity of SHAP, paving the way for future work.




%We believe that our framework provides a more intricate understanding of SHAP computation complexity across different models, distributions, and variants, paving the way for further research.

Our work opens promising directions for future research. First, expanding our computational analysis to other SHAP-related metrics, such as asymmetric SHAP~\citep{frye20} and SAGE~\citep{covert2020understanding}, would be valuable. Additionally, we aim to explore more expressive distribution classes and relaxed assumptions beyond those in Section \ref{sec:tractable} while maintaining tractable SHAP computation. Finally, when exact computation is intractable (Section \ref{sec:intractable}), investigating the approximability of SHAP metrics through approximation and parameterized complexity theory~\citep{downey2012parameterized} is an important direction.

%Our work opens several promising avenues for future research on the computational properties of explainable AI methods, with a particular focus on SHAP. First, it would be interesting to broaden the computational analysis conducted in this work to include other popular SHAP-related metrics in the literature, such as asymmetric SHAP \cite{frye20} and SAGE \cite{covert2020understanding}. Also, in the future, we aim to explore more expressive distribution classes and relaxed distributional assumptions—extending beyond those examined in Section \ref{sec:tractable} —that still yield tractable SHAP computation. Finally, when exact computation proves intractable (Section \ref{sec:intractable}), it is worthwhile to theoretically investigate the question of the approximability of computing the SHAP metrics across various configurations, through the lens of approximation and parametrized complexity theory \cite{arora2009computational}.

%This paper aims to deepen our understanding of the computational complexity involved in obtaining different Shapley value variants. We found that for a variety of ML models, including decision trees, tree ensembles for regression, weighted automata, and linear regression models — computing both local and global interventional and baseline SHAP can be done in polynomial time when distributions are modeled by HMMs. This extends the distributional scope of popular algorithms like TreeSHAP, which is limited to empirical distributions. Additionally, we demonstrate a strict complexity gap between SHAP variants, showing that interventional and baseline SHAP can be strictly easier to compute than conditional SHAP. Despite these positive results, we uncovered intractability for various SHAP variants in neural networks and tree ensembles. Finally, we provided generalized complexity relations across SHAP variants. We believe that our framework offers a deeper understanding of the complexity involved in computing SHAP across various variants, models, distributions, as well as in both local and global computations, laying the groundwork for future research.

% %-------------------------------------------------------------------------------
% \section*{Availability}
%-------------------------------------------------------------------------------

% %-------------------------------------------------------------------------------
\section*{Ethics Considerations}
% %-------------------------------------------------------------------------------
In this work, we propose a novel adversarial machine learning attack and conduct experiments exclusively in controlled, lab-only environments. Our primary goal is to advance the understanding of adversarial vulnerabilities in machine learning systems and to contribute to the development of more robust defenses. All experiments were carried out with benchmark datasets or models designed for research purposes, ensuring no real-world harm or exploitation of sensitive data. We acknowledge the dual-use nature of adversarial research and emphasize our commitment to ethical guidelines by openly discussing mitigation strategies and encouraging responsible use of this work.
%-------------------------------------------------------------------------------
\section*{Open Science}
% %-------------------------------------------------------------------------------

All the resources required for reproducing the experiments described in this study are provided.
This includes the complete dataset (traditional benchmarks), pre-processing scripts, model training code, and detailed instructions for setting up the computational environment. By ensuring accessibility to these resources, we aim to facilitate transparency, reproducibility, and further research in the field of ML.
All supplementary materials are available in the accompanying repository, providing a comprehensive framework for replicating and validating our findings.
% GitHub repository: \url{https://anonymous.4open.science/r/MOSHI-1518}.

\balance
\bibliographystyle{plain}
\bibliography{bibliography}

% \appendix
% \section{Data Availability}
% All the resources required for reproducing the experiments described in this study are provided. This includes the complete dataset (traditional benchmarks), pre-processing scripts, and model training code, as well as detailed instructions for setting up the computational environment. By ensuring accessibility to these resources, we aim to facilitate transparency, reproducibility, and further research in the field of machine learning. All supplementary materials are available in the accompanying repository, providing a comprehensive framework for replication and validation of our findings. GitHub repository: 
% \url{https://anonymous.4open.science/r/MSHAA-68DD/README.md}

%%%%%%%%%%%%%%%%%%%%%%%%%%%%%%%%%%%%%%%%%%%%%%%%%%%%%%%%%%%%%%%%%%%%%%%%%%%%%%%%
\end{document}
%%%%%%%%%%%%%%%%%%%%%%%%%%%%%%%%%%%%%%%%%%%%%%%%%%%%%%%%%%%%%%%%%%%%%%%%%%%%%%%%

%%  LocalWords:  endnotes includegraphics fread ptr nobj noindent
%%  LocalWords:  pdflatex acks