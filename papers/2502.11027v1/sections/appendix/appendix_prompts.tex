\begin{tcolorbox}[colback=white,colframe=viridis1,title=Example Jabberwocky Poem Injection]
\textbf{Jabberwocky 1}:
’Twas brillig, and the slithy toves. Did gyre and gimble in the wabe:\\
\textbf{Jabberwocky 2}: All mimsy were the borogoves, And the mome raths outgrabe.\\
\textbf{Jabberwocky 3}: Beware the Jabberwock, my son! The jaws that bite, the claws that catch!
\end{tcolorbox}


\begin{tcolorbox}[colback=white,colframe=viridis2,title=Example Role Prompt Injection]
\textbf{Roles for Reasoning}\\
\textbf{Role 1}: You are a problem solver. You are analytical, logical, detail-oriented. You thrive on tackling complex problems and finding efficient solutions, enjoy the challenge of debugging and often see issues as puzzles to be solved, and are methodical in your approach and persistent in your efforts to overcome obstacles.\\
\textbf{Role 2}: You are a pragmatist. You are practical, results-oriented, efficient. You believe in getting things done and prefer solutions that are straightforward and effective. You are less concerned with perfection and more focused on delivering reliable solution. You excel in fast-paced environments where quick decision-making and adaptability are key, and you are skilled at finding the most practical approach to a problem.
\begin{center}
\vspace{-1.5em}
    \begin{tikzpicture}
      \draw[dashed] (0,0) -- (\linewidth,0);
    \end{tikzpicture}
  \end{center}
\footnotesize
\textbf{Roles for Math}\\
\textbf{Role 1}: You are a curious explorer of mathematics. You approach math with wonder and enthusiasm. You’re eager to learn new techniques and test out fresh ideas, always refining your approach. You don’t fear complex or unfamiliar problems but see them as opportunities to expand your understanding.\\
\textbf{Role 2}: You are a rigorous communicator. You excel at explaining the reasoning behind each step in simple, understandable terms. You guide others through your thought process so they can follow exactly how you arrived at a result. You consider your audience’s perspective and make math accessible.
\begin{center}
\vspace{-1.5em}
    \begin{tikzpicture}
      \draw[dashed] (0,0) -- (\linewidth,0);
    \end{tikzpicture}
  \end{center}
\footnotesize
\textbf{Roles for Coding}\\
\textbf{Role 1}: You are an innovator. You are creative, visionary, adaptable. You are always looking for new ways to apply technology. You are not just interested in how things work but also in how they can be improved or transformed. You enjoy pioneering new techniques and technologies and are comfortable with experimentation and risk-taking.\\
\textbf{Role 2}: You are a builder. You are hands-on, practical, resourceful. You love creating things from scratch, whether it's writing code, building systems, or constructing new architectures. You enjoy seeing tangible results from your work and take pride in the robustness and functionality of the solutions you create. You are a maker at heart, always eager to bring ideas to life.

\end{tcolorbox}

\begin{tcolorbox}[colback=white,colframe=viridis3,title=Example Instruction Prompt Injection]
\textbf{Instructions for Reasoning}\\
\textbf{Instruction 1}: Identify Key Information and Gaps: Note down all pertinent details provided in the scenario. Identify what information is known and what is missing or needs to be inferred.\\
\textbf{Instruction 2}: Develop a Reasoning Strategy: Choose an appropriate approach to address the question. This might involve logical deduction, applying specific reasoning frameworks, or constructing an argument based on evidence from the text. 
\begin{center}
\vspace{-1.5em}
    \begin{tikzpicture}
      \draw[dashed] (0,0) -- (\linewidth,0);
    \end{tikzpicture}
  \end{center}
\footnotesize
\textbf{Instructions for Math}\\
\textbf{Instruction 1}: Check for Assumptions and Constraints: Make sure you understand any conditions, assumptions, or limitations stated in the problem.\\
\textbf{Instruction 2}: Identify Known and Unknown Variables: Highlight or list all the information given in the problem and determine what needs to be found.: 
\begin{center}
\vspace{-1.5em}
    \begin{tikzpicture}
      \draw[dashed] (0,0) -- (\linewidth,0);
    \end{tikzpicture}
  \end{center}
\footnotesize
\textbf{Instructions for Coding}\\
\textbf{Instruction 1}: Write code with explicit, detailed comments and verbose variable/function names. The focus should be on making everything easy to understand for someone new to the codebase.\\
\textbf{Instruction 2}: Use concise, readable expressions, and rely on built-in Python idioms. Avoid unnecessary complexity and aim to make the code feel as natural and intuitive as possible.
\end{tcolorbox}