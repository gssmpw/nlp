While increasing training compute has significantly improved the performance of large language models (LLMs), similar gains have not been observed when scaling inference compute. We hypothesize that the primary issue lies in the uniformity of LLM outputs, which leads to inefficient sampling as models repeatedly generate similar but inaccurate responses. Motivated by an intriguing relationship between solution accuracy (Pass@10) and response diversity, we propose \texttt{DivSampling}—a novel and versatile sampling technique designed to enhance the diversity of candidate solutions by introducing prompt perturbations.
\texttt{DivSampling} incorporates two categories of perturbations: task-agnostic approaches, which are general and not tailored to any specific task, and task-specific approaches, which are customized based on task content. Our theoretical analysis demonstrates that, under mild assumptions, the error rates of responses generated from diverse prompts are significantly lower compared to those produced by stationary prompts.
Comprehensive evaluations across various tasks — including reasoning, mathematics, and code generation — highlight the effectiveness of \texttt{DivSampling} in improving solution accuracy. This scalable and efficient approach offers a new perspective on optimizing test-time inference, addressing limitations in current sampling strategies.