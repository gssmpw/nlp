\section{Related Work}
\vspace{-1mm}
\subsection{Development of Current Mobile Agents}
The advent of VLMs \cite{vlmsurver} has marked a significant shift in phone automation, enabling more dynamic, context-aware, and sophisticated interactions with mobile devices \cite{mobilesurvey}. Research on mobile agents has progressed through key milestones, with models becoming more proficient at interpreting multi-modal data, understanding user intent, and autonomously executing complex tasks. 
VLM-based mobile agents typically follow two approaches: (1) Prompt Engineering \cite{mobilegpt, lu2024omniparserpurevisionbased}, where pre-trained models are guided by carefully designed prompts, and (2) Training-Based Methods \cite{hongCogAgentVisualLanguage2023, chengSeeClickHarnessingGUI2024}, where VLMs are further optimized using large-scale mobile datasets. 
While training-based methods offer higher potential and generalizability by improving the VLM through fine-tuning, they require a large amount of training data, which can be very costly.

\vspace{-2mm}
\subsection{Efforts in Building Datasets for Mobile Agents}
% The trajectory data used to train mobile agents consists of screenshots (GUI information), corresponding actions, and instructions expressed in natural language. 
Acquiring training trajectories for mobile agents presents significant challenges. 
Existing approaches are often reliant on manual curation, making data collection both costly and inefficient. Some works have explored the possibility of automatically constructing datasets using VLMs or Application Programming Interfaces (APIs) \cite{wangScreen2WordsAutomaticMobile2021, lai2024autowebglm}. 
But these approaches either halfway to completing the datasets or depend on pre-defined tasks.

OS-Genesis \cite{sunOSGenesisAutomatingGUI2024}, the most advanced in this area, proposes reverse task synthesis to eliminate the need for pre-defined instructions. However, this method still requires an agent to execute synthetic tasks in a simulated mobile environment, to obtain the corresponding screenshots and actions. 
This process does not guarantee the accuracy of executed actions, while also incurs additional computational and resource costs.
In contrast, we propose collecting real-world data from mobile users. This approach offers both (1) unlimited data scale, given the billions of mobile users worldwide, and (2) ground truth accuracy, as the data is directly generated through human execution.

% None of them propose to collect data from real-world mobile phone users.



%