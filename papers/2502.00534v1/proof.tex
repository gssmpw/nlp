\section*{Proof of Theorem \ref{thm:ts-upperbounds}}
	When $K=2$, recall our setting
	$\bY^{(k)} = \bX^{(k)}\bB^{*, (k)} + \bE^{(k)}$, $\bE^{(k)} \overset{\text{i.i.d.}}{\sim} \mathcal{N}\left(0, \Sigma_{\epsilon}\right)$. 
 For $k \in \braces{0, 1}$, where $\bB^{*, (k)} = \bL^* + \bS^{*,(k)}$.
	\begin{gather}
		(\hat\bL, \hat\bS^{(0)}, \hat\bS^{(1)}) 
		= \underset{\bL, \bS, \norm{\bL}_{\max} \leq \frac{\alpha}{p} }{\arg\min}\; \ell\paran{\bL, \bS^{(0)}, \bS^{(1)}}  \nonumber \\
		\ell\paran{\bL, \bS^{(0)}, \bS^{(1)}} 
		\defeq
		\frac{1}{2} \sum_{k=0}^K\norm{\bY^{(k)} - \bX^{(k)}\paran{\bL + \bS^{(k)}}}_F^2
		+ \lam_N \norm{\bL}_* 
		+ \mu^0_{N} \norm{\bS^{(0)}}_1
		+ \mu^1_{N} \norm{\bS^{(1)}}_1. \label{eqn:tl-rank-sparse-K=2}
	\end{gather}
	
	Define $\hat\bDelta^{L}=\hat\bL-\bL^*, \hat\bDelta^{S, (k)}=\hat\bS^{(k)}-\bS^{*, (k)}$. We first show the following lemma, which is adapted from the proof of Lemma 1 in \cite{negahban2012unified}.
	\begin{lemma}
		Suppose $\rank(\bL^*) \leq r$, $\norm{\bS^{*, (k)}}_0 \leq s$ for $k \in \{0, 1\}$, then there exists a decomposition $\hat\bDelta^{L}=\hat\bDelta^{L}_A+ \hat\bDelta^{L}_B$, such that $\rank(\hat\bDelta^{L}_A) \leq 2r$ and
		\begin{equation}
			\norm{\bL^*}_* - \norm{\hat\bL}_* \leq  \norm{\hat\bDelta^{L}_A}_* - \norm{\hat\bDelta^{L}_B}_*.
			\label{ineq:Ldecomp}
		\end{equation}
	   Similarly, there exists a subspace $\calM \subset \RR^{p \times p}$, such that $\norm{\bS^{*, (k)}_{\calM^{\perp}}}_1=0$, and
	    \begin{equation}
	    	\norm{\bS^{*, (k)}}_1 - \norm{\hat\bS^{(k)}}_1 \leq  \norm{\hat\bDelta^{S, (k)}_{\calM}}_1 - \norm{\hat\bDelta^{S, (k)}_{\calM^{\perp}}}_1,
	    	\label{ineq:Sdecomp}
	    \end{equation}
    where $\bA_{\calA}$ stands for the projection of $\bA$ onto subspace $\calA$  and  $\calM^{\perp}$ denotes the orthogonal complement of $\calM$.
	   \label{lem:decomp}
	\end{lemma}
    
    \begin{proof}[Proof of Lemma \ref{lem:decomp}]
    	By in Lemma 3.4 in \cite{recht2010guaranteed}, there exist matrices $\hat\bDelta^{L}_A, \hat\bDelta^{L}_B$ such that
    	$\hat\bDelta^{L}=\hat\bDelta^{L}_A+ \hat\bDelta^{L}_B$, $\rank(\hat\bDelta^{L}_A) \leq 2 \rank(\bL^*) \leq 2r$, and $\paran{\hat\bDelta^{L}_B}^{\top} \bL^* = \bL^* \paran{\hat\bDelta^{L}_B}^{\top} =0$. By Lemma 2.3 in \cite{recht2010guaranteed}, $\norm{\bL^*+\hat\bDelta^{L}_B}_*=\norm{\bL^*}_*+\norm{\hat\bDelta^{L}_B}_*$. Then \eqref{ineq:Ldecomp} holds by
    	$$	\norm{\bL^*}_* - \norm{\hat\bL}_* \leq 	\norm{\bL^*}_* +	\norm{\hat\bDelta^{L}_A}_* -	\norm{\hat\bDelta^{L}_B+\bL^*}_* = \norm{\hat\bDelta^{L}_A}_* - \norm{\hat\bDelta^{L}_B}_*,$$
      where the first inequality comes from triangle inequality. 
      
      For \eqref{ineq:Sdecomp}, take 
      $$\calM = \left\{ (a_{ij}) \in \mathbb{R}^{p \times q} : a_{ij} = 0 \text{ if } \mathbf{S}^{*, (k)}_{ij} = 0 \text{ for both } k = 0 \text{ and } 1 \right\}.$$
     Any matrix in $\calM$ has at most $2s$ nonzero entries, and it's easy to show 
       $$\mathcal{M}^{\perp} = \left\{ (a_{ij}) \in \mathbb{R}^{p \times q} : a_{ij} = 0 \text{ if } \mathbf{S}^{*, (k)}_{ij} \neq 0 \text{ for either } k = 0 \text{ or } 1 \right\}.$$
       
       Therefore,  $\norm{\bS^{*, (k)}_{\calM^{\perp}}}_1=0$, and
       \begin{align*}
       	\norm{\bS^{*, (k)}}_1 - \norm{\hat\bS^{(k)}}_1 &\leq \norm{\bS_\calM^{*, (k)}+\bS_{\calM^\perp}^{*, (k)}}_1 - \norm{\bS^{*, (k)}_\calM + \hat\bDelta^{S,(k)}_{\calM^{\perp}}}_1 + \norm{\hat\bDelta^{S,(k)}_{\calM} + \bS^{*, (k)}_{\calM^{\perp}}}_1\\
       	&\leq \norm{\bS_\calM^{*, (k)}}_1 - \norm{\bS^{*, (k)}_\calM}_1 - \norm{\hat\bDelta^{S,(k)}_{\calM^{\perp}}}_1 + \norm{\hat\bDelta^{S,(k)}_{\calM}}_1 + 2\norm{\bS_{\calM^\perp}^{*, (k)}}_1\\
       	&= \norm{\hat\bDelta^{S,(k)}_{\calM}}_1 - \norm{\hat\bDelta^{S,(k)}_{\calM^{\perp}}}_1.
       \end{align*}
       	\end{proof}
       For $k=0,1$,
       \begin{equation}
       	\begin{aligned}
       		 &\frac{1}{2}\norm{\bY^{(k)}-\bX^{(k)}\left(\bL^*+S^{*,(k)}+\hat\bDelta^L+\hat\bDelta^{S, (k)}\right)}_F^2-      \frac{1}{2}\norm{\bY^{(k)}-\bX^{(k)}\left(\bL^*+S^{*,(k)}\right)}_F^2\\
       		=&\frac{1}{2}\norm{\bX^{(k)}\left(\hat\bDelta^L+\hat\bDelta^{S, (k)}\right)}_F^2-\angles{\bE^{(k)},\bX^{(k)}\left(\hat\bDelta^L+\hat\bDelta^{S, (k)}\right)}\\
       		\geq&\frac{1}{2}\norm{\bX^{(k)}\paran{\hat\bDelta^L+\hat\bDelta^{S, (k)}}}_F^2-\abs{\angles{\paran{\bX^{(k)}}^\top \bE^{(k)},\hat\bDelta^L+\hat\bDelta^{S, (k)}}}\\
       		\geq & \frac{1}{2}\norm{\bX^{(k)}\paran{\hat\bDelta^L+\hat\bDelta^{S, (k)}}}_F^2 - \norm{\paran{\bX^{(k)}}^\top \bE^{(k)}}_2\norm{\hat\bDelta^L}_* - \norm{\paran{\bX^{(k)}}^\top \bE^{(k)}}_{\max}\norm{\hat\bDelta^{S, (k)}}_1\\
       		\geq &\frac{1}{2}\norm{\bX^{(k)}\paran{\hat\bDelta^L+\hat\bDelta^{S, (k)}}}_F^2 - \frac{\lambda_N}{4}\norm{\hat\bDelta^L}_* - \frac{\mu^k_N}{2}\norm{\hat\bDelta^{S, (k)}}_1.
       	\end{aligned}
       \label{ineq:difference-Frob}
       \end{equation}
       The second last inequality holds because $\|\cdot\|_{2}$ and $\|\cdot\|_{*}$ are dual norms, so do $\|\cdot\|_{\max}$ and $\|\cdot\|_{1}$. The last inequality comes from our assumptions about $\lambda_N$ and $\mu^k_N$. By inequality \eqref{ineq:difference-Frob} and Lemma \ref{lem:decomp}, 
        \begin{equation}
       	\begin{aligned}
       		 &\ell(\hat\bL, \hat\bS^{(0)}, \hat\bS^{(1)}) - \ell(\bL^*, \bS^{*,(0)}, \bS^{*,(1)})\\
       		=&\sum_{k=0}^1\brackets{\frac{1}{2} \norm{\bY^{(k)}-\bX^{(k)}\left(\bL^*+\bS^{*,(k)}+\hat\bDelta^L+\hat\bDelta^{S, (k)}\right)}_F^2-      \frac{1}{2}\norm{\bY^{(k)}-\bX^{(k)}\left(\bL^*+\bS^{*,(k)}\right)}_F^2}\\
       		+&\lambda_N\paran{\norm{\hat\bL}_*-\norm{\bL^*}_*} + \sum_{k=0}^1 \mu^k_{N}\brackets{\norm{\hat\bS^{(k)}}_1-\norm{\bS^{*, (k)}}_1}\\
       		\geq&\sum_{k=0}^1\brackets{\frac{1}{2}\norm{\bX^{(k)}\left(\hat\bDelta^L+\hat\bDelta^{S, (k)}\right)}_F^2-\frac{\lambda_N}{4}\norm{\hat\bDelta^L}_* - \frac{\mu^k_N}{2}\norm{\hat\bDelta^{S, (k)}}_1} - \lambda_N\paran{\norm{\hat\bDelta^{L}_A}_* - \norm{\hat\bDelta^{L}_B}_*} \\
       		- & \sum_{k=0}^1 \mu^k_{N}\brackets{\norm{\hat\bDelta^{S, (k)}_{\calM}}_1 - \norm{\hat\bDelta^{S, (k)}_{\calM^{\perp}}}_1}.
       	\end{aligned}
        \label{ineq:difference-l}
       \end{equation}
       	Since $\ell(\hat\bL, \hat\bS^{(0)}, \hat\bS^{(1)}) - \ell(\bL^*, \bS^{(0), *}, \bS^{(1), *})\leq 0$, we obtain 
       	$$-\lambda_N\paran{\norm{\hat\bDelta^{L}_A}_* - \norm{\hat\bDelta^{L}_B}_*} - \sum_{k=0}^1 \mu^k_{N}\brackets{\norm{\hat\bDelta^{S, (k)}_{\calM}}_1 - \norm{\hat\bDelta^{S, (k)}_{\calM^{\perp}}}_1} \leq \frac{\lambda_N}{2}\norm{\hat\bDelta^L}_* + \sum_{k=0}^1 \frac{\mu^k_N}{2}\norm{\hat\bDelta^{S, (k)}}_1,$$
       	which implies that
       	\begin{equation}
       		\lambda_N\norm{\hat\bDelta^{L}_B}_* + \sum_{k=0}^1\mu^k_N\norm{\hat\bDelta^{S, (k)}_{\calM^{\perp}}}_1 \leq 3\paran{\lambda_N\norm{\hat\bDelta^{L}_A}_* + \sum_{k=0}^1\mu^k_N\norm{\hat\bDelta^{S, (k)}_{\calM}}_1},
       		\label{ineq:BMcleq3AM}
       	\end{equation}
       and hence
      \begin{equation}
       	 \sum_{k=0}^1\mu^k_N\norm{\hat\bDelta^{S, (k)}}_1 \leq 3\lambda_N\norm{\hat\bDelta^{L}_A}_* + 4 \sum_{k=0}^1\mu^k_N\norm{\hat\bDelta^{S, (k)}_{\calM}}_1.
       	\label{ineq:DeltaSleq3A+4M}
       \end{equation}
       Also by \eqref{ineq:difference-l}, 
       \begin{equation}
       	\sum_{k=0}^1\brackets{\frac{1}{2}\norm{\bX^{(k)}\left(\hat\bDelta^L+\hat\bDelta^{S, (k)}\right)}_F^2} \leq \frac{3}{2}\paran{\lambda_N\norm{\hat\bDelta^{L}_A}_* + \sum_{k=0}^1\mu^k_N\norm{\hat\bDelta^{S, (k)}_{\calM}}_1}.
       	\label{ineq:upperbound-Fnorm-XDelta}
       \end{equation}
	  Since with high probability,
	   $$\Lambda_{\min}\paran{\paran{\bX^{(k)}}^{\top}\bX^{(k)}} < \zeta^{(k)},$$
	   $\Lambda_{\min}$ represents the minimum eigenvalue of a matrix. we have
       \begin{equation}
	   \frac{1}{2}\norm{\bX^{(k)}\left(\hat\bDelta^L+\hat\bDelta^{S, (k)}\right)}_F^2 \geq \frac{\zeta^{(k)}}{2} \norm{\hat\bDelta^L+\hat\bDelta^{S, (k)}}^2_F,
       \label{ineq:covex-condition}
       \end{equation}
	    with high probability. Note that 
	    \begin{equation}
	    	    \begin{aligned}
	    		\norm{\hat\bDelta^L+\hat\bDelta^{S, (k)}}^2_F&\geq \norm{\hat\bDelta^L}_F^2 + \norm{\hat\bDelta^{S, (k)}}_F^2 - \abs{\angles{\hat\bDelta^L, \hat\bDelta^{S, (k)}}}\\
	    		&\geq \norm{\hat\bDelta^L}_F^2 + \norm{\hat\bDelta^{S, (k)}}_F^2  
	    		-\norm{\hat\bDelta^L}_{\max} \norm{\hat\bDelta^{S, (k)}}_1\\
	    		&\geq \norm{\hat\bDelta^L}_F^2 + \norm{\hat\bDelta^{S, (k)}}_F^2-\frac{2\alpha}{p}\norm{\hat\bDelta^{S, (k)}}_1\\
	    		&\geq \norm{\hat\bDelta^L}_F^2 + \norm{\hat\bDelta^{S, (k)}}_F^2-\frac{\mu_N^k}{\zeta^{(k)}}\norm{\hat\bDelta^{S, (k)}}_1,
	    	\end{aligned}
    	\label{ineq:lowerbound-Fnorm-XDelta}
	    \end{equation}
	   where the second-to-last inequality holds because both $\norm{\bL^*}_{\max} \leq \frac{\alpha}{p}$ and $\norm{\hat \bL}_{\max} \leq \frac{\alpha}{p}$, and the last inequality comes from our choice of $\mu_N^k$. Combining \eqref{ineq:lowerbound-Fnorm-XDelta}, \eqref{ineq:upperbound-Fnorm-XDelta} and \eqref{ineq:DeltaSleq3A+4M} yields that
	   \begin{equation}
	   	\begin{aligned}
	   			   	&\quad\sum_{k=0}^1\brackets{\frac{\zeta^{(k)}}{2} \paran{\norm{\hat\bDelta^L}_F^2 + \norm{\hat\bDelta^{S, (k)}}_F^2}}\\ & \leq \sum_{k=0}^1 \frac{\mu_N^k}{2}\norm{\hat\bDelta^{S, (k)}}_1 + \frac{3}{2}\paran{\lambda_N\norm{\hat\bDelta^{L}_A}_* + \sum_{k=0}^1\mu^k_N\norm{\hat\bDelta^{S, (k)}_{\calM}}}_1\\
	   			   	&\leq 3\lambda_N\norm{\hat\bDelta^{L}_A}_* + \frac{7}{2} \sum_{k=0}^1\mu^k_N\norm{\hat\bDelta^{S, (k)}_{\calM}}_1\\
	   			   	&\leq 3\lambda_N\sqrt{2r}\norm{\hat\bDelta^{L}}_F+ \frac{7}{2}\sqrt{2s} \sum_{k=0}^1\mu^k_N\norm{\hat\bDelta^{S, (k)}}_F\\
	   			   	&\leq \sqrt{\braces{\sum_{k=0}^1\brackets{\frac{\zeta^{(k)}}{2} \paran{\norm{\hat\bDelta^L}_F^2 + \norm{\hat\bDelta^{S, (k)}}_F^2}}}\paran{\frac{36\lambda_N^2 r}{\zeta^{(0)}+\zeta^{(1)}}+\frac{49(\mu^0_N)^2 s}{\zeta^{(0)}}+\frac{49(\mu^1_N)^2 s}{\zeta^{(1)}}}}.
	   	\end{aligned}
	   \end{equation}
	   Therefore,
	  \begin{equation}
	   	\begin{aligned}
	   		&\quad\norm{\hat\bDelta^L}_F^2+\dfrac{\zeta^{(0)}}{\zeta^{(0)}+ \zeta^{(1)}}\norm{\hat\bDelta^{S, (0)}}_F^2+\dfrac{\zeta^{(1)}}{\zeta^{(0)}+ \zeta^{(1)}}\norm{\hat\bDelta^{S, (1)}}_F^2\\
	   		&\leq \frac{2}{\zeta^{(0)}+\zeta^{(1)}}\paran{\frac{36\lambda_N^2 r}{\zeta^{(0)}+\zeta^{(1)}}+\frac{49(\mu^0_N)^2 s}{\zeta^{(0)}}+\frac{49(\mu^1_N)^2 s}{\zeta^{(1)}}}.
	   	\end{aligned}
	   \end{equation}

\section*{Proof of Theorem \ref{thm:ts-upperbounds2}}      
     $$\sum_{k=0}^1 \brackets{\frac{1}{2} \norm{\bY^{(k)} - \bX^{(k)} \left( \bL^* + \bS^{*,(k)} + \hat\bDelta^L + \hat\bDelta^{S, (k)} \right)}_F^2} \leq \sum_{k=0}^1 \brackets{\frac{1}{2} \norm{\bY^{(k)} - \bX^{(k)} \left( \bL^* + \bS^{*,(k)} \right)}_F^2}$$
     \begin{equation}		
     \begin{aligned}
     \sum_{k=0}^1 \brackets{\frac{1}{2}\norm{\bX^{(k)}\left(\hat\bDelta^L+\hat\bDelta^{S, (k)}\right)}_F^2}
     & \leq\sum_{k=0}^1 \brackets{\abs{\angles{\paran{\bX^{(k)}}^\top \bE^{(k)},\hat\bDelta^L+\hat\bDelta^{S, (k)}}}}\\
    &\leq \sum_{k=0}^1 \brackets{\norm{\paran{\bX^{(k)}}^\top \bE^{(k)}}_2\norm{\hat\bDelta^L}_* + \norm{\paran{\bX^{(k)}}^\top \bE^{(k)}}_{\max}\norm{\hat\bDelta^{S, (k)}}_1}
    \end{aligned}
    \label{ineq:difference-Frob2}
    \end{equation}  
    
    By the separation lemma presented in \cite{chai2024structured}, this lemma demonstrates that two matrices can be well-separated to a certain extent under the incoherence assumption. Given two $\mu'$-incoherent matrices $G_1,G_2\in \RR^{p \times q}$, with $\mu'$ having an upper bound $\mu$, and $\rank(G_1),\rank(G_2)\leq r$. Let $\bDelta^G=G_1-G_2$, we have $\frac{\norm{\bDelta^G}^2_{max}}{\norm{\bDelta^G}_F^2}\leq \frac{c\mu r^4}{\max\{p,q\}}$, 
    
    Define the support sets of matrices $\hat\bDelta^{L}$ and $\hat\bDelta^{S, (k)}$ as $ \text{supp}(\hat\bDelta^{L})=\{(i,j)\mid \hat\bDelta^{L}_{i,j}\neq 0\}$ and $\text{supp}(\hat\bDelta^{S, (k)})\{(i,j)\mid \hat\bDelta^{S, (k)}_{i,j}\neq 0\}$, respectively, and let $J = \text{supp}(\hat\bDelta^{L}) \cap \text{supp}(\hat\bDelta^{S, (k)})$. The projection operator $\Pi_J$ zeros out elements not in $J$. Hence,$$\frac{\norm{\Pi_J(\hat\bDelta^{L})}_F^2}{\norm{\hat\bDelta^{L}}_F^2}\leq |J|\frac{\norm{\hat\bDelta^{L}}_{max}^2}{\norm{\hat\bDelta^{L}}_F^2}\leq 2s\frac{c\mu r^4}{\max\{p,q\}}\leq \frac{\max\{p,q\}}{8c\mu r^4}\frac{c\mu r^4}{\max\{p,q\}}=\frac{1}{4}$$
    Same to $\hat\bDelta^{S, (k)}$, For $k=0,1$, $$\abs{\angles{\hat\bDelta^L,\hat\bDelta^{S, (k)}}}=\abs{\angles{\Pi_J(\hat\bDelta^L),\Pi_J(\hat\bDelta^{S, (k)})}}\leq \norm{\Pi_J(\hat\bDelta^L)}_F^2+\norm{\Pi_J(\hat\bDelta^{S, (k)})}_F^2\leq \frac{1}{4}\paran{\norm{\hat\bDelta^{L}}_F^2+\norm{\hat\bDelta^{S, (k)}}_F^2}$$
     Recalling condition(\ref{ineq:covex-condition}), and inequality(\ref{ineq:lowerbound-Fnorm-XDelta}), we have
     \begin{equation}
     \begin{aligned}
     \sum_{k=0}^1 \brackets{\frac{1}{2}\norm{\bX^{(k)}\left(\hat\bDelta^L+\hat\bDelta^{S, (k)}\right)}_F^2 }&\geq \sum_{k=0}^1 \brackets{\frac{\zeta^{(k)}}{2} \paran{ \norm{\hat\bDelta^L}_F^2 + \norm{\hat\bDelta^{S, (k)}}_F^2 - \abs{\angles{\hat\bDelta^L, \hat\bDelta^{S, (k)}}}}}\\
     &\geq \sum_{k=0}^1 \frac{\zeta^{(k)}}{4} \paran{ \norm{\hat\bDelta^L}_F^2 + \norm{\hat\bDelta^{S, (k)}}_F^2 }
     \end{aligned}
     \end{equation}

     The incoherence property essentially restricts the upper bound of the cross terms in inequality (\ref{ineq:lowerbound-Fnorm-XDelta}). Coupling this with inequality (\ref{ineq:difference-Frob2}), we have,
     \begin{equation}
     \begin{aligned}
      &\sum_{k=0}^1 \frac{\zeta^{(k)}}{4} \paran{ \norm{\hat\bDelta^L}_F^2 + \norm{\hat\bDelta^{S, (k)}}_F^2 }\\
      &\leq \sum_{k=0}^1 \brackets{\norm{\paran{\bX^{(k)}}^\top \bE^{(k)}}_2\norm{\hat\bDelta^L}_* + \norm{\paran{\bX^{(k)}}^\top \bE^{(k)}}_{\max}\norm{\hat\bDelta^{S, (k)}}_1}\\
      &\leq \sqrt{4\braces{\sum_{k=0}^1\brackets{\frac{\zeta^{(k)}}{4} \paran{\norm{\hat\bDelta^L}_F^2 + \norm{\hat\bDelta^{S, (k)}}_F^2}}}\paran{\norm{\paran{\bX^{(k)}}^\top \bE^{(k)}}_2^2+2\norm{\paran{\bX^{(k)}}^\top \bE^{(k)}}_{\max}^2}}.
     \end{aligned}
     \end{equation}
     
     By combining lemma \ref{A:3}, which provides upper bounds on 
    $\norm{\bX^{\top}\bE}_2$ and $\norm{\bX^{\top}\bE}_{\max}$, we complete the proof of Theorem \ref{thm:ts-upperbounds2}.

     \begin{equation}
     \begin{aligned}
	 &\quad\norm{\hat\bDelta^L}_F^2+\dfrac{\zeta^{(0)}}{\zeta^{(0)}+ \zeta^{(1)}}\norm{\hat\bDelta^{S, (0)}}_F^2+\dfrac{\zeta^{(1)}}{\zeta^{(0)}+ \zeta^{(1)}}\norm{\hat\bDelta^{S, (1)}}_F^2\\
     &\leq \frac{4}{\zeta^{(0)}+\zeta^{(1)}}\paran{\norm{\paran{\bX^{(k)}}^\top \bE^{(k)}}_2^2+2\norm{\paran{\bX^{(k)}}^\top \bE^{(k)}}_{\max}^2}
     \leq C_2\frac{d+\log d}{N_0+N_1}.
     \end{aligned}
     \end{equation}

 \section*{Proof of Lemma \ref{lem:xe-bound}}
 \label{A:3}
Building upon the research presented in \cite{basu2015regularized} and \cite{basu2019low}, deviation bounds for $\norm{\bX^{\top}E}_2$ and $\norm{\bX^{\top}E}_{\max}$have been established for the regularized estimation of low-dimensional structures. The transition core matrix is not necessarily square, we extend of Proposition 3 from \cite{basu2019low},  derive the following results:
\begin{lemma}
Let $X \in \mathbb{R}^{p \times p}$ and $E \in \mathbb{R}^{p \times q}$, denote $d = \max\{p,q\}$, satisfying the settings in (1). There exist positive constants $c_1$ and $c_2$ such that,\\
for $N \gtrsim d$, 
 $$\mathbb{P}\left( \left\| \frac{X^\top E}{N} \right\|_2 > c_0 \varphi(B, \Sigma_{\epsilon})\sqrt{\frac{d}{N}} \right) \leq c_1 \exp \left[-c_2 \log p\right],$$
and for any  $N \gtrsim \log d$, 
$$\mathbb{P}\left( \left\| \frac{X^\top E}{N} \right\|_{\max} > c_0 \varphi(B, \Sigma_{\epsilon})\sqrt{\frac{\log d}{N}} \right) \leq c_1 \exp \left[-c_2 \log p\right]$$
\label{lem:xe-bound}
\end{lemma}

\begin{proof}
 Without loss of generality, assume that $p > q$. We augment the matrix $E \in \mathbb{R}^{p \times q}$ to a $p \times p$ matrix form $E' = [E, \mathbf{0}_{p,p-q}]$. $\norm{X^{\top}E'}_2\leq \norm{X^{\top}E}_2,\norm{X^{\top}E}_{\max}\leq \norm{X^{\top}E'}_{\max}$. We can now directly apply Proposition 3 in \cite{basu2019low} to the pair $(X, E')$.
\   
\end{proof}
 
\section*{Proof of Lemma \ref{lem:boundQ} } 
\label{A:4}
\begin{proof}
        Applying Hölder's inequality, the triangle inequality, and Cauchy-Schwarz inequality, we deduce the following results:
     \begin{equation}
     \begin{aligned}
     &Q_{n,h} - \left(r + [P_h V_{h+1}]\right) \\
     &\leq \norm{\phi_{n, h}^{\top}(\widetilde{\bL}_{n} - \bL^{*})}_*\norm{\bPsi^{\top} V_{n, h+1}}_{op}+\sum_{k=0}^1\norm{\phi_{n, h}^{\top}\paran{\widetilde{\bS}_{n} - \bS^{(1)}}}_1\norm{\bPsi^{\top} V_{n, h+1}}_{\infty} \\
     & \leq \paran{C_{\psi,1}\norm{\phi_{n, h}^{\top}(\widetilde{\bL}_{n} - \bL^*)}_*+C_{\psi,2}\sum_{k=0}^1\norm{\phi_{n, h}^{\top}\paran{\widetilde{\bS}_{n}^{(k)} - \bS^{*(k)}}}_1}H \\
     & \leq \paran{C_{\psi,1}\sqrt{2r}\norm{\phi_{n, h}^{\top}}_*\norm{\widetilde{\bL}_{n} - \bL^{*}}_F + C_{\psi,2}\sqrt{s}\sum_{k=0}^1\norm{\phi_{n, h}^{\top}}_1\norm{\widetilde{\bS}_{n}^{(k)} - \bS^{*(k)}}_F}H \\
     & \leq C_{\phi}\sqrt{p}H\paran{C_{\psi,1}\sqrt{2r}\norm{\widetilde{\bL}_{n} - \bL^*}_F + C_{\psi,2}\sqrt{s}\sum_{k=0}^1
     \norm{\widetilde{\bS}_{n}^{(k)} - \bS^{*(k)}}_F} \\
     & \leq C_{\phi}\sqrt{p}H \sqrt{2rC_{\psi,1}^2 \|\widetilde{\bL}_{n} - \bL^*\|_F^2 + sC_{\psi,2}^2 \sum_{k=0}^1 \|\widetilde{\bS}_{n}^{(k)} - \bS^{*(k)}\|_F^2}\\
     & \overset{(1)}{\leq}C_{\phi}C_{M}\sqrt{p}H
     \sqrt{
     \norm{\widetilde{\bL}_n -\hat\bL_n}^2_F + 
     \norm{\hat\bL_n -\bL^*}^2_F+\sum_{k=0}^1\paran{\norm{\widetilde{\bS}_{n}^{(k)} - \hat\bS_n^{(k)}}^2_F+\norm{\hat\bS_n^{(k)} - S^{*(k)}}^2_F}} \\
     & \leq 2C_{\phi}C_{M}C^{'} \sqrt{p}H\sqrt{\frac{dr+s\log d}{N_0+N_1}+\frac{\alpha^2s}{pq}}
     \end{aligned}
    \end{equation}
\end{proof}

Inequality $(1)$ holds because $\|\widetilde{\bL}_{n} - \bL^*\|_F^2 + \sum_{k=0}^1 \|\widetilde{\bS}_{n}^{(k)} - \bS^{*(k)}\|_F^2 \leq C$, Constants $2rC_{\psi,1}^2$ and $sC_{\psi,2}^2$ are subsumed under the maximum operation to form the consolidated constant term $C_M$ for brevity.
 
    
   