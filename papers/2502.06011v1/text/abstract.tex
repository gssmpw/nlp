Off-policy evaluation (OPE) is a critical challenge in robust decision-making that seeks to assess the performance of a new policy using data collected under a different policy. 
%
%
%
However, the existing OPE methodologies suffer from several limitations arising from statistical uncertainty as well as causal considerations. 
In this thesis, we address these limitations by presenting three different works. 

Firstly, we consider the problem of high variance in the importance-sampling-based OPE estimators. 
We propose a novel off-policy evaluation estimator, the Marginal Ratio (MR) estimator, to alleviate this problem.
By focusing on the marginal distribution of outcomes rather than the policy shift directly, the MR estimator achieves significant variance reduction compared to state-of-the-art methods, while maintaining unbiasedness. 
%

Next, we shift our attention towards uncertainty quantification in off-policy evaluation.
To this end, we propose Conformal Off-Policy Prediction (COPP) as a novel approach to quantify this uncertainty with finite-sample guarantees.
Unlike traditional methods focusing on point estimates of expected outcomes, COPP provides reliable predictive intervals for outcomes under a target policy. This enables robust decision-making in risk-sensitive applications and offers a more comprehensive understanding of policy performance. 
%

Finally, we address the fundamental challenge of causal inference in off-policy evaluation. 
Recognizing the limitations of traditional OPE methods under unmeasured confounding, we develop novel causal bounds for sequential decision settings that remain valid under arbitrary confounding.
We apply these bounds for the assessment of digital twin models without relying on strong causal assumptions. 
We propose a framework for causal falsification, allowing us to identify scenarios where the digital twin's predictions diverge from real-world behavior. This approach provides valuable insights into model reliability and helps ensure safe and effective decision-making.

We conclude this thesis with a discussion of our contributions and
limitations of the presented work, and outline interesting avenues for future research arising from our work.