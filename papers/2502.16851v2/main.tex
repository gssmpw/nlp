%%
%% This is file `sample-sigplan.tex',
%% generated with the docstrip utility.
%%
%% The original source files were:
%%
%% samples.dtx  (with options: `all,proceedings,bibtex,sigplan')
%% 
%% IMPORTANT NOTICE:
%% 
%% For the copyright see the source file.
%% 
%% Any modified versions of this file must be renamed
%% with new filenames distinct from sample-sigplan.tex.
%% 
%% For distribution of the original source see the terms
%% for copying and modification in the file samples.dtx.
%% 
%% This generated file may be distributed as long as the
%% original source files, as listed above, are part of the
%% same distribution. (The sources need not necessarily be
%% in the same archive or directory.)
%%
%%
%% Commands for TeXCount
%TC:macro \cite [option:text,text]
%TC:macro \citep [option:text,text]
%TC:macro \citet [option:text,text]
%TC:envir table 0 1
%TC:envir table* 0 1
%TC:envir tabular [ignore] word
%TC:envir displaymath 0 word
%TC:envir math 0 word
%TC:envir comment 0 0
%%
%% The first command in your LaTeX source must be the \documentclass
%% command.
%%
%% For submission and review of your manuscript please change the
%% command to \documentclass[manuscript, screen, review]{acmart}.
%%
%% When submitting camera ready or to TAPS, please change the command
%% to \documentclass[sigconf]{acmart} or whichever template is required
%% for your publication.
%%
%%
% \documentclass[sigplan,review,anonymous]{acmart}
\documentclass[sigplan,screen]{acmart}
%%
%% \BibTeX command to typeset BibTeX logo in the docs
\AtBeginDocument{%
  \providecommand\BibTeX{{%
    Bib\TeX}}}

%% Rights management information.  This information is sent to you
%% when you complete the rights form.  These commands have SAMPLE
%% values in them; it is your responsibility as an author to replace
%% the commands and values with those provided to you when you
%% complete the rights form.
% \setcopyright{acmlicensed}
\copyrightyear{2018}
\acmYear{2018}
\acmDOI{XXXXXXX.XXXXXXX}
%% These commands are for a PROCEEDINGS abstract or paper.
\acmConference[GPGPU 2025]{General Purpose Processing on Graphics Processing Units}{Mar 1, 2025}{Las Vegas, NV}

\acmISBN{978-1-4503-XXXX-X/18/06}

\settopmatter{printacmref=false} % Removes ACM reference format
\renewcommand\footnotetextcopyrightpermission[1]{} % Removes footnote with conference info
\pagestyle{plain} % Removes running headers
\usepackage{lipsum}
\usepackage{tikz} 
\usepackage{csquotes}
\usepackage{multirow}
\usepackage{hyperref} 
\usepackage{todonotes}
\newcommand{\tsc}[1]{\textsuperscript{#1}}
\usepackage{enumitem}
\setlist[itemize]{leftmargin=3.0mm}
% \setlength{\parindent}{0pt}
% \newif\ifshowlipsum
% \showlipsumtrue % Enable lipsum (set to false to disable)
\renewcommand{\lipsum}[1][]{} % Disable lipsum
% \section{Introduction}
% \ifshowlipsum
%     \lipsum[1-2]
% \fi
% \showlipsumfalse
%%
%% Submission ID.
%% Use this when submitting an article to a sponsored event. You'll
%% receive a unique submission ID from the organizers
%% of the event, and this ID should be used as the parameter to this command.
%%\acmSubmissionID{123-A56-BU3}

%%
%% For managing citations, it is recommended to use bibliography
%% files in BibTeX format.
%%
%% You can then either use BibTeX with the ACM-Reference-Format style,
%% or BibLaTeX with the acmnumeric or acmauthoryear sytles, that include
%% support for advanced citation of software artefact from the
%% biblatex-software package, also separately available on CTAN.
%%
%% Look at the sample-*-biblatex.tex files for templates showcasing
%% the biblatex styles.
%%

%%
%% The majority of ACM publications use numbered citations and
%% references.  The command \citestyle{authoryear} switches to the
%% "author year" style.
%%
%% If you are preparing content for an event
%% sponsored by ACM SIGGRAPH, you must use the "author year" style of
%% citations and references.
%% Uncommenting
%% the next command will enable that style.
%%\citestyle{acmauthoryear}


%%
%% end of the preamble, start of the body of the document source.
\begin{document}

%%
%% The "title" command has an optional parameter,
%% allowing the author to define a "short title" to be used in page headers.
\title{Can Tensor Cores Benefit Memory-Bound Kernels? (No!)}

%%
%% The "author" command and its associated commands are used to define
%% the authors and their affiliations.
%% Of note is the shared affiliation of the first two authors, and the
%% "authornote" and "authornotemark" commands
%% used to denote shared contribution to the research.
% \vspace{-50pt}
\author{Lingqi Zhang\tsc{1}, Jiajun Huang\tsc{2}, Sheng Di\tsc{3}, Satoshi Matsuoka\tsc{1}, Mohamed Wahib\tsc{1}}
\affiliation{
  \institution{\tsc{1} RIKEN Center for Computational Science, Japan, {\small\{lingqi.zhang@riken.jp, matsu@acm.org, mohamed.attia@riken.jp\}}}
  \institution{\tsc{2} University of California, Riverside, USA, {\small\{jhuan380@ucr.edu\}}}
  \institution{\tsc{3} Argonne National Laboratory, USA, {\small\{sdi1@anl.gov\}}}
  \country{}
}

\renewcommand{\shortauthors}{Lingqi Z. et al.}
\renewcommand{\authors}{Lingqi Zhang, Jiajun Huang, Sheng Di, Satoshi Matsuoka and Mohamed Wahib}

% \author{Lingqi Zhang}
% \email{lingqi.zhang@riken.jp}
% \orcid{0000-0002-2452-1551}
% \affiliation{%
%   \institution{RIKEN Center for Computational Science}
%   % \city{Tokyo}
%   \country{Japan}
% }

% \author{Jiajun Huang}
% \email{jhuan380@ucr.edu}
% \orcid{0000-0001-5092-3987}
% \affiliation{%
%   \institution{University of California, Riverside}
%   % \city{Riverside}
%   \country{United States of America}}
% \author{Sheng Di}
% \email{sdi1@anl.gov}
% \orcid{0000-0002-9935-5674}
% \affiliation{%
%   \institution{Argonne National Laboratory}
%   % \city{Lemont}
%   \country{United States of America}
% }
% \author{Satoshi Matsuoka}
% \email{matsu@acm.org}
% \orcid{0000-0003-1910-8532}
% \affiliation{%
%   \institution{RIKEN Center for Computational Science}
%   % \city{Lemont}
%   \country{Japan}
% }
% \author{Mohamed Wahib}
% \email{mohamed.attia@riken.jp}
% \orcid{0000-0002-7165-2095}
% \affiliation{%
%   \institution{RIKEN Center for Computational Science}
%   % \city{Tokyo}
%   % \state{Tokyo}
%   \country{Japan}
% }


%%
%% By default, the full list of authors will be used in the page
%% headers. Often, this list is too long, and will overlap
%% other information printed in the page headers. This command allows
%% the author to define a more concise list
%% of authors' names for this purpose.
% \renewcommand{\shortauthors}{Lingqi et al.}
% \renewcommand{\shortauthors}{}

%%
%% The abstract is a short summary of the work to be presented in the
%% article.
\begin{abstract}
% The answer is No.  
Tensor cores are specialized processing units within GPUs that have demonstrated significant efficiency gains in compute-bound applications such as Deep Learning Training by accelerating dense matrix operations. Given their success, researchers have attempted to extend tensor core capabilities beyond dense matrix computations to other computational patterns, including memory-bound kernels. Recent studies have reported that tensor cores can outperform traditional CUDA cores even on memory-bound kernels, where the primary performance bottleneck is not computation.
In this research, we challenge these findings through both theoretical and empirical analysis. Our theoretical analysis reveals that tensor cores can achieve a maximum speedup of only $1.33$× over CUDA cores for memory-bound kernels in double precision (for V100, A100, and H100 GPUs). We validate this theoretical limit through empirical analysis of three representative memory-bound kernels-STREAM Scale, SpMV, and stencil. We demonstrate that optimizing memory-bound kernels using tensor cores does not yield sound performance improvements over CUDA cores.
% porting code from CUDA cores to tensor cores fails to deliver meaningful performance benefits for memory-bound applications.

  % A clear and well-documented \LaTeX\ document is presented as an
  % article formatted for publication by ACM in a conference proceedings
  % or journal publication. Based on the ``acmart'' document class, this
  % article presents and explains many of the common variations, as well
  % as many of the formatting elements an author may use in the
  % preparation of the documentation of their work.
\end{abstract}

%%
%% The code below is generated by the tool at http://dl.acm.org/ccs.cfm.
%% Please copy and paste the code instead of the example below.
%%
% \begin{CCSXML}
% <ccs2012>
%  <concept>
%   <concept_id>00000000.0000000.0000000</concept_id>
%   <concept_desc>Do Not Use This Code, Generate the Correct Terms for Your Paper</concept_desc>
%   <concept_significance>500</concept_significance>
%  </concept>
%  <concept>
%   <concept_id>00000000.00000000.00000000</concept_id>
%   <concept_desc>Do Not Use This Code, Generate the Correct Terms for Your Paper</concept_desc>
%   <concept_significance>300</concept_significance>
%  </concept>
%  <concept>
%   <concept_id>00000000.00000000.00000000</concept_id>
%   <concept_desc>Do Not Use This Code, Generate the Correct Terms for Your Paper</concept_desc>
%   <concept_significance>100</concept_significance>
%  </concept>
%  <concept>
%   <concept_id>00000000.00000000.00000000</concept_id>
%   <concept_desc>Do Not Use This Code, Generate the Correct Terms for Your Paper</concept_desc>
%   <concept_significance>100</concept_significance>
%  </concept>
% </ccs2012>
% \end{CCSXML}

% \ccsdesc[500]{Do Not Use This Code~Generate the Correct Terms for Your Paper}
% \ccsdesc[300]{Do Not Use This Code~Generate the Correct Terms for Your Paper}
% \ccsdesc{Do Not Use This Code~Generate the Correct Terms for Your Paper}
% \ccsdesc[100]{Do Not Use This Code~Generate the Correct Terms for Your Paper}

%%
%% Keywords. The author(s) should pick words that accurately describe
%% the work being presented. Separate the keywords with commas.
% \keywords{Tensor Core, Matrix Engine, Memory-Bound Kernels}
%% A "teaser" image appears between the author and affiliation
%% information and the body of the document, and typically spans the
%% page.
% \begin{teaserfigure}
%   \includegraphics[width=\textwidth]{sampleteaser}
%   \caption{Seattle Mariners at Spring Training, 2010.}
%   \Description{Enjoying the baseball game from the third-base
%   seats. Ichiro Suzuki preparing to bat.}
%   \label{fig:teaser}
% \end{teaserfigure}

% \received{20 February 2007}
% \received[revised]{12 March 2009}
% \received[accepted]{5 June 2009}

%%
%% This command processes the author and affiliation and title
%% information and builds the first part of the formatted document.
 \maketitle


\begin{figure}[ht]
    \centering
    \includegraphics[width=0.8\linewidth]{graphs/greater_than_naive.pdf}
    \vspace{0.5cm}
    \includegraphics[width=0.8\linewidth]{graphs/p1_bottom.png}
    \vspace{-5pt}
    \caption{\textcolor{positional}{Positional} vs.\ \textcolor{nonpositional}{non-positional} circuits. In a \textcolor{nonpositional}{non-positional} circuit, the same edges must be included at all positions. A \textcolor{positional}{positional} circuit can distinguish between the same edge at different positions. This specificity yields better trade-offs between circuit size and faithfulness. It can also increase both precision and recall.}
    \label{fig:p1}
    \vspace{-5pt}
\end{figure}

\section{Introduction}

\looseness=-1
A primary goal of interpretability research is to characterize the internal mechanisms in language models (LMs) and other NLP models. 
A core approach in this area is \textbf{circuit discovery}---identifying the minimal subgraph within the model's computation graph that performs a specific task \citep{olah2021framework,olah-mech}.
Typically, the nodes of a circuit represent model components (e.g., attention heads, neurons, or layers).
While manual circuit discovery methods can yield position-specific insights \citep{wanginterpretability,goldowskydill2023localizingmodelbehaviorpath}, \emph{automatic methods often overlook positional information}, treating components as uniformly relevant across all input token positions \citep{conmytowards,syed2023attribution}. 
For instance, if an attention head is included in a circuit, it is assumed to contribute equally to the computation for every position in the input sequence.
The assumption that circuits are position-invariant ignores the fact that different positions often require distinct computations.
By ignoring positions, current methods limit their ability to capture mechanisms that operate across positions, such as interactions between attention heads across positions.

In this study, we start by demonstrating that positional agnosticism is a significant limitation (\S\ref{sec:motivating}). Then, to address these limitations, we introduce a new approach: position-aware edge attribution patching (PEAP; \S\ref{sec:full_circ_discovery}; Figure~\ref{fig:p1}). Current approaches  assume that if an edge is in a circuit, then the same edge will be in the circuit at all positions, thus leading to low precision. It is also assumed that an edge's importance should be aggregated across positions before deciding whether it should be included in the circuit; this can lead to cancellation effects, and thus low recall. PEAP instead allows us to compute the importance of cross-positional edges, and separately evaluates edge importance at each position. We show that this leads to smaller and more accurate circuits; see Figure~\ref{fig:p1}.

Incorporating positional information into circuit discovery is straightforward when inputs have the same length and structure across examples.

However, realistic datasets are not nearly this templatic.
How, then, can we incorporate positional information into automatic circuit discovery?
To address this challenge, we propose \textbf{schemas} (\S\ref{sec:schema}). 
Schemas assign semantic labels to spans of tokens, enabling information aggregation across examples even when the spans differ in length.

For example, in the input ``The \textcolor{positional}{war} lasted from 1453 to 14\underline{\hspace{1em}},'' the span ``\textcolor{positional}{war}'' could be labeled as ``\emph{Subject}''.
This enables handling spans with varying lengths: the phrase ``\textcolor{positional}{Black Plague}'' in another example can be treated as a single positional span with the same role as ``\textcolor{positional}{war}''.
In experiments with two LMs and three tasks, we find that circuits discovered using schemas achieve a better trade-off between circuit size and faithfulness to the model's behavior than position-agnostic circuits.
Importantly, position-aware circuits offer a more precise representation of the underlying mechanisms, providing a more concise foundation for mechanistic explanations.

We also present a fully automated pipeline for schema generation and application (\S\ref{sec:schema-generation}) using large language models (LLMs). 
We evaluate the quality of the generated schemas and their utility in discovering position-aware circuits (\S\ref{sec:schema-eval}).
Notably, circuits derived using automatically generated and applied schemas achieve comparable faithfulness scores to circuits discovered with human-designed and manually applied schemas.

We summarize our contributions as follows:
\begin{itemize}[noitemsep,leftmargin=*,topsep=1pt,parsep=1pt]
    \item Introduce a position-aware circuit discovery method, which obtains better faithfulness than position-agnostic discovery.  
    \item Introduce dataset schemas,  facilitating positional circuit discovery in more naturalistic settings. 
    \item Develop an automated schema generation and application pipeline with LLMs, yielding schemas that are comparable to manually-annotated ones.
\end{itemize}

\section{Background of Cost Estimation} \label{sec:background}
This section first gives a brief overview of classical and learned cost estimation. 
Afterwards, we describe the learning procedure of \lcms and provide a taxonomy that guides our selection of recent \lcms for this study in \Cref{sec:methodology}.

\subsection{Traditional \& Learned Cost Estimation}
\textbf{Traditional Cost Estimation.} 
Precise cost estimates for different plan candidates in a database are crucial for the query optimizer to select optimal plans from a large search space.
Thus, a lot of engineering effort has been spent since the beginning of database development to estimate the execution costs of a query plan.
Most database systems such as MySQL \cite{widenius2002}, Oracle, PostgreSQL, or System R \cite{astrahan1976} use hand-crafted cost models to reason about the execution costs of a query plan.
These models typically provide a cost function for each physical operator in a query plan that estimates its runtime costs according to CPU usage, I/O operations, memory consumption, expected tuples, and random or sequential page accesses.
However, due to the wide variety of data, queries, and data layouts, traditional cost models need to make simplifying assumptions (e.g., independence of attributes).
These often lead to incorrect predictions of the execution cost. 
Consequently, the query optimizer makes sub-optimal decisions that degrade the query performance by increasing its runtime \cite{leis_how_2015}.
%-------------------------------------------------------

\noindent\textbf{Learned Cost Estimation.}
The need to improve prediction accuracy and the rise of machine learning motivated the idea of \lcms. 
The main idea is to approximate the complex cost functions with a learned model.
Generally, a typical model learns from previous query executions to predict execution costs like runtime.
In contrast to traditional cost models, the promise of \lcms is that they can better learn arbitrarily complex functions.
Thus, improved prediction accuracy can be expected in contrast to traditional approaches based on simplifying assumptions.
Overall, the higher accuracy is expected to lead to a selection of query plans with improved query performance.

%-----------------------------------------
\subsection{Learning Procedure of \lcms}
For our study, we look at effects that also result from the learning procedure of \lcms.
As such, we briefly review the traditional procedure as depicted in \Cref{fig:learning_procedure} to provide the necessary background:
\circles{A}~At first, a workload generator is used to create a large set of randomized, synthetic SQL-Strings that involve a variety of representative query properties such as filter predicates, joins, or aggregation types.
\circles{B}~These queries are executedses (e.g., an airline or movie database) to collect the actual costs of queries.
An important aspect here is that training procedures of many \lcms leads to biases in the dataset due to timeouts and pre-optimized queries, as discussed later.
\circles{C}~Next, various information is extracted from the workload execution.
Most importantly, the physical query plans are extracted, which serve as input to cost models.
In addition to physical plans, \lcms require different information, such as data characteristics like histograms or sample bitmaps.
\circles{D}+\circles{E}~Finally, the workload (i.e., plans and runtime) is then split for training and testing the \lcms. 

\begin{figure*}
    \centering
    \includegraphics[width=\linewidth]{./figures/training_procedure.pdf}
    \caption{
    Learning procedure of \lcms. 
    \circles{\textsc{A}} Generation of synthetic training queries. \circles{\textsc{B}} Query execution on training databases. 
    \circles{\textsc{C}} Feature (query plans, data characteristics, and sample bitmaps) and label (query runtimes) extraction to generate the training and test dataset. 
    \circles{\textsc{D}} Training of the \lcm with supervised learning. 
    \circles{\textsc{E}} Evaluation of the \lcm against unseen test data.}
    \label{fig:learning_procedure}
\end{figure*}

%------------------------------------------------------
\subsection{Taxonomy of \lcms} \label{subsec:taxonomy}
\lcms developed in the last years differ in various dimensions.
This section provides a brief taxonomy of recent \lcms to structure the different methodological approaches. 
This taxonomy will guide the selection of \lcms that we use in this study and ensure that we cover the different methodologies to analyze how they affect the ability of \lcms to support query optimization.

\noindent\textbf{Input Features.}
The first crucial dimension is the input features that a \lcm learns from.
The input features are extracted from the executed workloads (cf. \Cref{fig:learning_procedure}\circles{C}).
The query plan and the underlying data distribution need to be modeled so that a \lcm is informed to make reasonable predictions about the execution costs, which in turn affects query optimization, as we will show.
However, \lcms make use of different information for cost estimation.

\begin{enumerate}[leftmargin=*, nosep]
\item \textbf{SQL-String vs. Query Plans}: 
Some of the first models rely on the SQL string to describe a query, as it gives insights about the tables, predicates, and joins. 
However, details of the execution plan, such as physical operators or the order of joins, are not described there. 
Thus, most \lcms utilize the physical query plan, which includes the operators (e.g., scans, joins) and physical operator types  (e.g., nested loop vs. hash join).
As we will see later, this is fundamental for query optimization.

\item \textbf{Cardinalities}:
Intermediate cardinalities are an important input signal for the overall cost of a plan as they denote the number of tuples an operator needs to process \cite{leis_how_2015}.
Thus, many \lcms leverage intermediate cardinalities as input features, which are either annotated by the databases' cardinality estimator or obtained through an additional learned estimator from related work \cite{hilprecht2020deepdb, kipf2019, yang2020}.
While some \lcms also ignore cardinalities as input for cost prediction, we show in our study that they, in fact, improve the usefulness of cost estimates from \lcms for various query optimization tasks.

\item\textbf{Data Distribution}:
Another helpful factor in estimating cost is understanding the data distribution in the base tables, especially if no cardinalities are used.
For instance, the fact of how many distinct values exist in a column might influence the efficiency of physical operators (e.g., hash join). 
As such, some \lcms use data distribution represented as database statistics and histograms or sample bitmaps (which we explain later) from the base tables as inputs.
However, as we will show in our study, their effect on query optimization tasks remains unclear.

\item \textbf{Cost Estimates}: 
Finally, some of the most recent \lcms even leverage the cost estimates provided by a classical cost estimator as an input feature, which serves as a strong input signal.
This idea renders these \lcms to \textit{hybrid} as they combine a traditional cost model with a learned approach.
The study shows that this provides significant benefits.
\end{enumerate}

%-----------------------------------------------------
\noindent \textbf{Query Representation.}
Many \lcms use model architectures use a graph-based representation to encode query plans as input to the models\footnote{The graph-based representation of the queries refers to the fact whether a model leverages the query graph structure and not to the model learning architecture itself.}.
These approaches thus explicitly leverage information about the order (parent-child relationships) of operators in plans.
However, other \lcms \cite{kipf2019, akdere2012} represent a query plan (or the SQL string) as a flat vector of fixed size without modeling the operator dependencies, which we refer to as \textit{flat} representation in this paper.
While intuitively, capturing the structure and not using a flat representation should be beneficial for \lcms, the results of using graph structure in this study are not that clear. 

%-------------------------------------------------------
\noindent \textbf{Database Dependency.}
Furthermore, an important aspect is whether \lcms can generalize to unseen databases (i.e., a new set of tables) or not.
\textit{Database-agnostic} \lcms were designed \cite{hilprecht2022, zibo_liang_dace_2024} to enable cost predictions for unseen databases that were not part of the training data.
This approach has the advantage of directly providing results without requiring database-specific training data. 
In contrast, \textit{database-specific} \cite{sun2019, zhao2022, marcus2019} models cannot generalize for unknown databases.
For this study, an interesting question is if one of these classes is better suited to support query optimization tasks as database-specific can better adapt to one single database while database-agnostic models can generalize better.

%----------------------------------------
\noindent \textbf{Model Architecture.}
Finally, the presented \lcms differ largely in their learning approach.
Various learning architectures were proposed, including decision trees, tree-structured neural networks, neural units, graph neural networks, and transformer architectures.
While different architectures show different results on the cost estimation tasks, it is still open to see which architecture provides the best results for query optimization.
%----------------------------------------
% This section introduces the workload studied in this paper i.e., SCALE (Section~\ref{sec:theoscale}), Spmv (Section~\ref{sec:theospmv}), and stencil (Section~\ref{sec:theostencil}).
% We conducted an analysis based on operation intensity $\mathbb{I}$. Since our main interest lies in high-performance workloads, our analysis focuses on double-precision situations (data size $\mathbb{D}=8$), yet the analysis can be further generalized to lower-precision situations. 


\section{Workloads: Memory-Bound Kernels}\label{sec:workload}

This section examines three representative memory-bound kernels: SCALE (Section~\ref{sec:theoscale}), Sparse Matrix-Vector Multiplication (SpMV) (Section~\ref{sec:theospmv}), and Stencil (Section~\ref{sec:theostencil}). We analyze these kernels through the lens of operational intensity ($\mathbb{I}$), focusing on double-precision operations (data size $\mathbb{D}=8$ bytes). % to align with high-performance computing requirements. 

While our analysis centers on double precision, the methodology can be extended to lower-precision scenarios.

\subsection{SCALE}\label{sec:theoscale}
SCALE, one of the STREAM benchmark~\cite{McCalpin2007}, is defined as
\begin{equation}\footnotesize
a_i = qb_i, \quad \forall i \in {1,\ldots,n}, \quad a,b \in \mathbb{R}^n, \quad q \in \mathbb{R}
\end{equation}
Each element operation requires one load, one store, and one computation, yielding: $\mathbb{W}(\text{SCALE})=1$, $\mathbb{Q}(\text{SCALE})=2\times\mathbb{D}$, and consequently $\mathbb{I}(\text{SCALE})=\tfrac{1}{16}$. STREAM benchmark is commonly used to measure sustainable memory bandwidth due to its low computational intensity.

\subsection{Sparse Matrix–Vector Multiplication (SpMV)} \label{sec:theospmv}
SpMV, crucial for iterative solvers, has format-dependent operational intensity. In this section, we begin by analyzing dense matrix-vector multiplication (GEMV) as a baseline.

\noindent\textbf{GEMV:}
For matrix $A \in \mathbb{R}^{m \times n}$ and vectors $x \in \mathbb{R}^n$, $y \in \mathbb{R}^m$, GEMV is defined as:
\begin{equation}\footnotesize
y = Ax
\end{equation}
With computing $\mathbb{W}(\text{GEMV})=m\times n\times2$ operations and memory traffic $\mathbb{Q}(\text{GEMV})=(m\times n+m+n)\times\mathbb{D}$ yields:
\begin{equation}\footnotesize
\mathbb{I}(\text{GEMV})=\frac{m\times n\times2}{(m\times n+m+n)\times\mathbb{D}}
\approx \frac{2}{\mathbb{D}}=\frac{1}{4}
\end{equation}

\noindent\textbf{SpMV:}
For sparse matrices with $nnz$ non-zeros, SpMV is defined as:
\begin{equation}\footnotesize
\begin{aligned}
y &= Ax, \
A \in \mathbb{R}^{m \times n}, \quad \text{nnz}(A) &\ll mn, \quad
x \in \mathbb{R}^n, \quad y \in \mathbb{R}^m
\end{aligned}
\end{equation}
With computation $\mathbb{W}(\text{SpMV})=2\times nnz$ and memory traffic including coordinate information $\alpha\mathbb{I}$ or packed values $\beta\mathbb{Z}$:
\begin{equation}\footnotesize
\mathbb{I}(\text{SpMV})=\frac{nnz\times2}{(nnz+m+n)\times\mathbb{D}+\alpha\mathbb{I} +\beta\mathbb{Z}}
\end{equation}
Given $nnz\ll m\times n$, we have $\mathbb{I}(\text{SpMV})<\mathbb{I}(\text{GEMV})$.
% \begin{equation}\footnotesize
% \mathbb{I}(\text{SpMV})<\mathbb{I}(\text{GEMV})
% \end{equation}

\noindent\textbf{Compressed Sparse Row (CSR) format:}
CSR format, the most common sparse representation, requires storing column indices and row pointers. With memory traffic $\mathbb{Q}(\text{SpMV,CSR})=(nnz+m+n)\times\mathbb{D}+(nnz+m+1)\times\mathbb{I}$ and computation $\mathbb{W}(\text{SpMV,CSR})=2\times nnz$:
\begin{equation}\footnotesize
\begin{split}
\mathbb{I}(\text{SpMV,CSR})&=\frac{2\times nnz}{(nnz+m+n)\times\mathbb{D}+(nnz+m+1)\times\mathbb{I}} \\
&\approx \frac{2}{\mathbb{D}+\mathbb{I}}=\frac{1}{6}<\mathbb{I}(\text{GEMV})
\end{split}
\end{equation}
This analysis confirms SpMV's memory-bound nature, consistent with prior works~\cite{10.1145/1816038.1816021,10.1145/3577193.3593705}.

% Spmv is at the heart of many iterative solvers. The operation intensity of Spmv depends on the sparse format 
% chosen. We initialize the analysis in this subsection by analyzing gemv: dense matrix-vector multiplication. 
% We can assume a matrix $A$ with $m\times n$. The input vector $x$ has $n$ elements and the output vector $y$ has $m$ elements. The gemv operation can be formulated as:

% \begin{equation}
% y = Ax, \quad A \in \mathbb{R}^{m \times n}, \quad x \in \mathbb{R}^n, \quad y \in \mathbb{R}^m
% \end{equation}

% We will have $\mathbb{W}_{gemv}=m\times n\times2$. Ideally, we need to load the whole matrix, and input vector from the main memory, and store the output vector back to the main memory. So we have: $\mathbb{Q}_{gemv}=(m\times n+m+n)\times\mathbb{D}$. Thus we have:
% \begin{equation}\footnotesize
%     \mathbb{I}_{gemv}=\frac{m\times n\times2}{(m\times n+m+n)\times\mathbb{D}}
%      \approx \frac{2}{\mathbb{D}}=\frac{1}{4}
% \end{equation}


% The operation intensity of spmv depends on the sparse format chosen. We assume that we have non-zero values $nnz$. The formulation of spmv would be:

% \begin{equation}
% \begin{aligned}
% y &= Ax, \\
% A \in \mathbb{R}^{m \times n}, \quad \text{nnz}(A) &\ll mn, \quad
% x \in \mathbb{R}^n, \quad y \in \mathbb{R}^m
% \end{aligned}
% \end{equation}

% So we have $\mathbb{Q}_{spmv}=2\times nnz$. When it comes to memory traffic, we need either additional information of coordinate $\alpha\mathbb{I}$ (e.g. CSR format~\cite{7013050}) or need to pack additional values $\beta\mathbb{Z}$ (SELL-C-$\sigma$~\cite{doi:10.1137/130930352}). So we have $\mathbb{Q}_{spmv}=(nnz+m+n)\times\mathbb{D}+\alpha\mathbb{I} +\beta\mathbb{Z}$. Thus:
% \begin{equation}\footnotesize
%    \mathbb{I}_{spmv}=\frac{nnz\times2}{(nnz+m+n)\times\mathbb{D}+\alpha\mathbb{I} +\beta\mathbb{Z}}
% \end{equation}
% Because $nnz\ll m\times n$. We have:
% \begin{equation}\footnotesize
%    \mathbb{I}_{spmv}<\mathbb{I}_{gemv}
% \end{equation}


% CSR format (or Compressed Sparse Column (CSC)) is the most used sparse matrix format. Assume that we have non-zero values $nnz$. We have the same amount of value and index for the column index. We need to record the starting point and end point of each row of $m+1$ values. So we have: $\mathbb{Q}_{spmv}(CSR)=(nnz+m+n)*\mathbb{D}+(nnz+m+1)\times\mathbb{I})$. As for computation, the effective computation is $\mathbb{Q}_{spmv}(CSR)=2\times nnz$. So, we have:

% \begin{equation}\footnotesize
% \begin{split}
%     \mathbb{I}_{spmv}(CSR)&=\frac{2\times nnz}{(nnz+m+n)*\mathbb{D}+(nnz+m+1)\times\mathbb{I})} \\
%     &\approx \frac{2}{\mathbb{D}+\mathbb{I}}=\frac{1}{6}<\mathbb{I}_{gemv}
% \end{split}
% \end{equation}


% So spmv kernel is a memory-bound kernel with very low operation intensity. This analysis result consists with existing research~\cite{10.1145/1816038.1816021,10.1145/3577193.3593705}. 






\subsection{Iterative Stencils}\label{sec:theostencil}

Stencil computations is common in HPC~\cite{hagedorn2018high}. For 2D stencil, we have:
\begin{equation}\footnotesize
v(i,j) = \sum_{(p,q) \in \mathbb{S}} w_{p,q} \cdot u(i+p,j+q)
\end{equation}
where $v(i,j)$ and $u(i,j)$ are updated and original values at point $(i,j)$, and $\mathbb{S}$ defines relative offsets (e.g., 5-point stencil: $(-1,0)$, $(1,0)$, $(0,1)$, $(0,-1)$, $(0,0)$).
Ideally, only one load of $u$ and one store of $v$ are necessary:
\begin{equation}\footnotesize
\mathbb{Q}=2\times\mathbb{D}, \quad \mathbb{W}=2\times|\mathbb{S}|, \quad \mathbb{I}=\frac{|\mathbb{S}|}{\mathbb{D}}
\end{equation}
For a 2d5pt stencil where $|\mathbb{S}
(\text{2d5pt})|=5$, $\mathbb{I}(\text{2d5pt})=\tfrac{5}{8}$.

% \subsubsection{Temporal Blocking}
\noindent\textbf{Temporal blocking~\cite{10.1145/3577193.3593716,10.1145/3368826.3377904}} combines $t$ timesteps together:
\begin{equation}\footnotesize
\mathbb{W}_{t}=t\times2\times|\mathbb{S}|, \quad \mathbb{I}_{t}=t\times\frac{|\mathbb{S}|}{\mathbb{D}}
\end{equation}
While temporal blocking can theoretically transform memory-bound stencils into compute-bound kernels by increasing operational intensity, practical limitations exist. 

For a 2d5pt stencil on GH200 ($\mathbb{B}_{GH200}=9.99$), compute-bound behavior requires:
\begin{equation}\footnotesize
t\times \mathbb{I}(\text{2d5pt})>\mathbb{B}_{GH200} \implies t\times 0.625 > 9.99 \implies t > 15.98
\end{equation}

However, deep temporal blocking (e.g., $t > 16$) usually faces hardware limits from register pressure~\cite{10.1145/3577193.3593716,10.1145/3368826.3377904}. 

Thus, shallow temporal blocking ($t < 16$) 2d5pt stencil remains memory-bound, while deep temporal blocking might make stencil kernel register-bound. 



% Iterative stencils are widely used in HPC~\cite{hagedorn2018high}. To simplify the discussion, we use 2D stencil as an example. Let $u(i,j)$ represent the value of the grid at point (i,j), we can formulate 2d stencil computation as:
% \begin{equation}
% v(i,j) = \sum_{(p,q) \in \mathbb{S}} w_{p,q} \cdot u(i+p,j+q)
% \end{equation}
% where:
% \begin{itemize}
%     \item $v(i,j)$: The updated value at grid point $(i,j)$
%     \item $u(i,j)$: The original value at grid point $(i,j)$
%     \item $\mathbb{S}$: The set of relative offsets defining the stencil points (e.g., $(-1,0)$, $(1,0)$, $(0,1)$, $(0,-1)$, $(0,0)$ for a 5-point stencil)
% \end{itemize}

% Ideally, we only load $u$ and store $v$ once. So we have: 
% \begin{equation}\footnotesize
%     \mathbb{Q}=2\times\mathbb{D}
% \end{equation}
% The workload is directly correlated to the number of points in $\mathbb{S}$. We have:
% \begin{equation}\footnotesize
%     \mathbb{W}=2\times|\mathbb{S}|
% \end{equation}
% And:
% \begin{equation}\footnotesize
%     \mathbb{I}=\frac{|\mathbb{S}|}{\mathbb{D}}
% \end{equation}

% As an example, for 2d5pt stencil $|\mathbb{S}_{2d5pt}|=5$, we have $\mathbb{I}_{2d5pt}=\tfrac{5}{8}$. 




\section{Theories}
\label{sec.theory}
Beyond the practical effectiveness of GTs
it is essential to understand the theoretical foundations underlying GTs. 
This section begins by reviewing the different expressive capabilities among existing GTs (Section~\ref{theory:expressiveity}). Subsequently, we investigate the interconnections between GTs and other graph learning methodologies (Section~\ref{theory:relationship}).

\subsection{Expressivity}
\label{theory:expressiveity}
Following the order in Section~\ref{sec.architectures}, we here respectively discuss the expressivity in structural tokenization, and comparing absolute PE with relative PE.

\subsubsection{Structural Tokenization}
In node-level tokenization, each node is treated as an independent token, allowing the model to capture local neighborhood information. However, it may struggle to capture global graph structural patterns that span multiple nodes, potentially being insufficient for expressing graph properties that require global information. Therefore, it is necessary to enhance structural bias with additional positional embeddings~\cite{zhang2023rethinking}. Edge-level tokenization can capture the connectivity between nodes, facilitating models to comprehend the interactions between nodes and the topology of the graph. Subgraph-level and hop-level tokenization encode local subgraph patterns to tokens, such as graph motifs and $k$-hop neighbors. 
These tokenizations allow models to capture more complex and global graph features, enhancing the representations on communities structures and long-range dependencies.

The expressive power of GTs is intricately related to the process of tokenization. TokenGT \cite{TokenGT} harnesses this power by effectively encoding graphs as sets of input tokens. As a pure Transformer, TokenGT utilizes \(n + m\) tokens for each graph, where \(n\) represents the number of nodes and \(m\) represents the number of edges. This approach achieves 2-Weisfeiler–Leman (WL) expressivity, which has been proven equivalent to 1-WL expressivity \cite{Morris2021WeisfeilerAL}.

A formal theoretical framework~\cite{expressive-token}
establishes a connection between various tokenization methods and the \(k\)-WL test. To align GTs with the \(k\)-WL test, the authors propose providing suitable input tokens \(\mathbf{X}^{(0,k)}\) to the Transformer for each \(k \geq 1\). They demonstrate that the \(t\)-th layer of the Transformer can emulate the \(t\)-th iteration of a \(k\)-order WL algorithm. In this context, \(\mathbf{X}^{(0,k)} \in \mathbb{R}^{n^k \times d}\) denotes the initial token embeddings of k-tuples, where \(n^k\) represents the number of these embeddings.

In general, different levels of tokenization affect the expressive power of GTs. Node-level tokenization is suitable for capturing local features, edge-level tokenization is appropriate for understanding relationships between nodes, while subgraph- and hop-level tokenization provide a deeper understanding of the global structure of the graph.

\subsubsection{Positional Encoding}
A recent literature~\cite{li2024what} highlights the importance of a theoretical comparison on various PE strategies. 
Predominantly, there are two types of PEs based on either kernel or Laplacian graphs. To conduct a theoretical analysis of these PEs, methods like WL test, along with SPD-WL, GD-WL~\cite{zhang2023rethinking}, and RPE-augWL~\cite{black2024comparing}, are employed to assess and compare their expressivity.
To analyze the expressive power of absolute PE and relative PE, Black et al.~\cite{black2024comparing} proposed a framework that leverages 2-equivariant graph network (2-EGN)~\cite{maron2018invariant} to convert between relative PE and absolute PE. For graph without node features, the paper demonstrates that the distinguishing capabilities of absolute PE and relative PE are equivalent.
In contrast, for graph with node features, converting relative PE to absolute PE will undermine the distinguishing capability of GTs.


In addition, while the expressivity of absolute PEs has been discussed in Section~\ref{architecture:PE} and the works for resistance distance have been compared in Section~\ref{architecture:attention}, it has been proven that exploiting the power of matrix, such as the relative random walk positional encoding (RRWP) using the adjacency matrix achieves at least comparable expressivity to spectral kernel when employed as relative PE~\cite{black2024comparing}.


\subsection{Relationship with Other Graph Learning Methods}
\label{theory:relationship}
The characteristics of GTs can be elucidated through comparative study with other graph learning methods. In this section, we examine studies that compare GT with MPNN, graph structure learning, and graph attention network.

\subsubsection{MPNN}
Compared with MPNNs, GTs integrate self-attention mechanisms and PE. A recent study~\cite{li2024what} demonstrates that self-attention mechanism improves the convergence rate of GTs, while PE facilitates identifying the core neighborhood for each node, thereby enhancing the generalization ability. Notably, GTs with shortest-path distance~\cite{black2024comparing} as relative PE possess theoretically superior expressivity than classical MPNNs. 

An alternative approach to infuse global information into each node is to introduce a virtual node connected to all nodes in a graph. 
Despite the simplicity of this idea, MPNN with the virtual node~\cite{cai2023connection} surprisingly serves as a strong baseline in Long Range Graph Benchmark~\cite{dwivedi2022long}. A recent study~\cite{rosenbluth2024distinguished} reveals that no single algorithm can fully surpass the others between GTs and MPNNs with virtual node. 

In addition, the over-smoothing problem~\cite{chen2020measuring}, characterized from deep MPNNs, also exist in Transformers~\cite{shi2022revisiting}, which will result in indistinguishable node embeddings in deep layers.
As Transformer is a special form of Graph Attention Networks (GAT)~\cite{velivckovic2017graph}, it shares the same over-smoothing phenomenon as GAT, 
leading to an exponential degeneration of expressive power regarding the number of layers. 
To mitigate over-smoothing, SignGT~\cite{chen2023signgtsignedattentionbasedgraph} proposes a signed attention mechanism to preserve the  diverse frequency information in graph structure from the perspective of graph signal processing.



\subsubsection{Graph Structural Learning}
Graph Structure Learning (GSL) is closely related to GTs, which aims at automatically 
refining graph structures when the input graph is noisy or incomplete, or inferring implicit graph structures when explicit graph structure is unavailable~\cite{GSLB}, in a parameterized way.
Building on this foundation, GSL has been widely applied in various domains, such as molecular context graphs~\cite{PAR,Pin-Tuning}, spatiotemporal graphs~\cite{BiGSL}, and social networks~\cite{VIB-GSL}.
GTs can be regarded as a special form of GSL, achieved by self-attention that learns a fully connected `soft' graph structure~\cite{mp-all-the-way-up}. 
By utilizing attention-oriented techniques, such as the attention mask in \cref{sec:attention-mask} and discrete structure sampling in NodeFormer~\cite{wu2022nodeformer}, the learned graph structure can be sparsified to reflect real-world topology.






\section{Empirical Analysis}\label{sec:eval}


In this section, we use empirical analysis to verify our theoretical findings regarding tensor core performance on memory-bound kernels. We conduct experiments across multiple hardware platforms, detailed in Table~\ref{tab:env}, to systematically evaluate the performance relationship between tensor core and CUDA core implementations.


% In this section, we use empirical analysis to debunk the tensor core's performance advantage in memory-bound kernels. The hardware platforms that we used are summarized in Table~\ref{tab:env}.
\begin{table}[t]
    \caption{Specifications of the experimental platforms.}
    \label{tab:env}
    \vspace{-14pt}
    \centering
    \footnotesize
    \begin{tabular}{|l|c|c|c|}
    \hline
    \multicolumn{2}{|c|}{\textbf{Metric}}             & \textbf{A100-80GB} & \textbf{GH200} \\\hline
    \multicolumn{2}{|c|}{\textbf{CUDA Version}}             & 12.1 & 12.6\\\hline
    \multicolumn{2}{|c|}{\textbf{L2 Cache (MB)}}        & 40               & 50           \\\hline
    \multicolumn{2}{|c|}{\textbf{Memory Bandwidth (TB/s)}}        & 1.94               & 4.00           \\\hline
    \multirow{2}{*}{\textbf{FP64 Peak (TFLOPS)}} &\textbf{CUDA Core}&  9.7    &  34.0  \\
                                            &\textbf{Tensor Core}& 19.5   &  67.0   \\\hline
    % \multirow{2}{*}{\textbf{Balance}}       &\textbf{CUDA Core}& 5.0       &  8.5   \\
    %                                         &\textbf{Tensor Core}& 10.0      &  16.8  \\\hline
    % 
    \end{tabular}
    
\end{table}

% \begin{table}[t]
%     \centering
%     \footnotesize
%     \renewcommand{\arraystretch}{1.3} % Adjust row spacing
%     \setlength{\tabcolsep}{5pt} % Adjust column spacing
%     \begin{tabular}{@{}llcc@{}}
%     \toprule
%     \multicolumn{2}{l}{\textbf{Metric}}               & \textbf{A100-80GB} & \textbf{GH200} \\ \midrule
%     \multicolumn{2}{l}{\textbf{CUDA Version}}         & {12.1}      & {12.6}  \\
%     \multicolumn{2}{l}{\textbf{L2 Cache (MB)}}        & {40}        & {50}    \\
%     \multicolumn{2}{l}{\textbf{Memory Bandwidth (TB/s)}} & {1.94}   & {4.00}  \\ \midrule
%     \multirow{2}{*}{\textbf{Peak (TFLOPS)}} 
%         & \textbf{CUDA Core}    & {9.7}     & {34.0}  \\
%         & \textbf{Tensor Core}  & {19.5}    & {67.0}  \\ 
%     \bottomrule
%     \end{tabular}
%     \caption{Specifications of the experimental platforms.}
%     \label{tab:env}
% \end{table}


% Note that we use latency-based warmup, i.e., we run the same kernel until it reaches a certain amount of runtime. In this research, we by default run the warmup kernel until $350$ ms.

\pgfplotstableread[col sep=space,string type]{
id num mean1 std1 mean2 std2 mean3 std3 mean4 std4 mean5 std5
0 2 0.14 0.0 0.164 0.011775681155103792 0.04 0.009797958971132711 0.156 0.014966629547095756 0.25866666666666666 0.015084944665313026
1 4 0.256 0.0 0.19200000000000003 0.024657656011875903 0.11466666666666665 0.006798692684790379 0.29066666666666663 0.009428090415820642 0.3746666666666667 0.004988876515698593
2 8 0.28 0.0 0.23066666666666666 0.02174600857373346 0.2293333333333333 0.013597385369580758 0.428 0.02262741699796954 0.48 0.008640987597877129
3 16 0.292 0.0 0.2733333333333334 0.004988876515698593 0.35200000000000004 0.021416504538945343 0.532 0.019866219234335136 0.6026666666666666 0.008219218670625309
4 32 0.296 0.0 0.3 0.005656854249492386 0.4653333333333333 0.020997354330698156 0.6386666666666666 0.015434449203720314 0.6707 0.0222
}{\llamamath}

\pgfplotstableread[col sep=space,string type]{
id num mean1 std1 mean2 std2 mean3 std3 mean4 std4 mean5 std5
0 2 0.33031674208144796 0.0 0.334841628959276 0.01954974569655463 0.07692307692307693 0.006399156390828481 0.3197586726998492 0.023752663270020555 0.4962292609351433 0.005643525470247277
1 4 0.4434389140271493 0.0 0.38009049773755654 0.016930576410741804 0.2730015082956259 0.026211383404197246 0.5113122171945701 0.012798312781656976 0.6003016591251885 0.007690828828948417
2 8 0.45701357466063347 0.0 0.43288084464555054 0.01297484957321663 0.4841628959276018 0.037495634674678896 0.645550527903469 0.007690828828948402 0.6787330316742081 0.016930576410741797
3 16 0.45701357466063347 0.0 0.4419306184012066 0.007690828828948391 0.669683257918552 0.0230724864868452 0.7345399698340875 0.018958982036163686 0.7601809954751131 0.009774872848277315
4 32 0.45701357466063347 0.0 0.45248868778280543 0.011083663994494026 0.779788838612368 0.005643525470247252 0.8054298642533938 0.006399156390828514 0.8099547511312218 0.0036945546648313263
}{\llamagsm}

\pgfplotstableread[col sep=space,string type]{
id num mean1 std1 mean2 std2 mean3 std3 mean4 std4 mean5 std5
0 2 0.1308411214953271 0.0 0.14174454828660435 0.01803089860247698 0.040498442367601244 0.005828126770675922 0.14174454828660435 0.01803089860247698 0.21495327102803738 0.0066084745905284825
1 4 0.24299065420560748 0.0 0.14174454828660435 0.01803089860247698 0.09190031152647975 0.009601890970356661 0.2554517133956386 0.009601890970356651 0.30062305295950154 0.014444888622267452
2 8 0.2850467289719626 0.0 0.14174454828660435 0.01803089860247698 0.1308411214953271 0.01375663686343901 0.3707165109034268 0.014444888622267452 0.3862928348909658 0.009601890970356661
3 16 0.29906542056074764 0.0 0.14330218068535824 0.01720461217630414 0.19003115264797507 0.007942397996250445 0.4657320872274144 0.01957913565416906 0.48286604361370716 0.0223562306766469
4 32 0.29906542056074764 0.0 0.13707165109034267 0.01803089860247698 0.22897196261682243 0.010094611679763024 0.5420560747663551 0.01375663686343899 0.5607476635514018 0.02124327367131751
}{\llamambpp}

\pgfplotstableread[col sep=space,string type]{
id num mean1 std1 mean2 std2 mean3 std3 mean4 std4 mean5 std5
0 2 0.16666666666666666 0.0 0.14814814814814814 0.0604812282168686 0.043209876543209874 0.017459426695964134 0.14814814814814814 0.0604812282168686 0.24691358024691357 0.03147542909625176
1 4 0.35185185185185186 0.0 0.14814814814814814 0.0604812282168686 0.1111111111111111 0.01512030705421715 0.3518518518518518 0.02618914004394619 0.3703703703703704 0.01512030705421715
2 8 0.3888888888888889 0.0 0.14814814814814814 0.0604812282168686 0.1728395061728395 0.031475429096251756 0.4691358024691357 0.017459426695964137 0.49382716049382713 0.04619330107128325
3 16 0.3888888888888889 0.0 0.14814814814814814 0.0604812282168686 0.24691358024691357 0.057244558614171014 0.5802469135802469 0.008729713347982055 0.5987654320987654 0.03805193828993194
4 32 0.3888888888888889 0.0 0.12345679012345678 0.05310077325334955 0.2962962962962963 0.08000914442478839 0.6666666666666666 0.01512030705421717 0.6851851851851851 0.0302406141084343
}{\llamahumaneval}

\pgfplotstableread[col sep=space,string type]{
id num mean1 std1 mean2 std2 mean3 std3 mean4 std4 mean5 std5
0 2 0.0 0.0 0.06922498118886382 0.011109723897843043 0.05417607223476298 0.013915155762909656 0.06696764484574869 0.009276770508606438 0.11361926260346127 0.004256474228361455
1 4 0.11254019292604502 0.0 0.08653122648607976 0.011109723897843041 0.07072987208427389 0.011109723897843043 0.15951843491346876 0.013833541242174742 0.18434913468773514 0.005924761379993844
2 8 0.13504823151125403 0.0 0.109104589917231 0.0069778920206890185 0.08728367193378479 0.007673468041524134 0.25959367945823925 0.012901751843101775 0.27915726109857036 0.015675445188863543
3 16 0.13504823151125403 0.0 0.1361926260346125 0.009091832937241967 0.10609480812641083 0.008446179202649982 0.3829947328818661 0.008512948456722924 0.380737396538751 0.014783207452512045
4 32 0.13504823151125403 0.0 0.1617757712565839 0.001064118557090362 0.12641083521444693 0.012086063509562834 0.46501128668171554 0.01026198773287123 0.4740406320541761 0.008033918925531462
}{\mistralmath}

\pgfplotstableread[col sep=space,string type]{
id num mean1 std1 mean2 std2 mean3 std3 mean4 std4 mean5 std5
0 2 0.0 0.0 0.18166455428812844 0.006890370270511227 0.13181242078580482 0.0017924126265818718 0.16603295310519647 0.004742278056747701 0.3565694972539079 0.01285604060388926
1 4 0.15716096324461343 0.0 0.2336290663286861 0.0043084237546200084 0.18250950570342206 0.006209099474735556 0.3967046894803549 0.005377237879745629 0.4879594423320659 0.011753635608993267
2 8 0.1634980988593156 0.0 0.2995352767215885 0.004308423754620012 0.22475707646810306 0.009041374972130423 0.5792141951837769 0.010192089633979539 0.6117448246725813 0.012856040603889275
3 16 0.16603295310519645 0.0 0.33586818757921416 0.0027379555126353537 0.2691170257710182 0.0041822961286910295 0.694972539079003 0.0041822961286910295 0.7165188001689903 0.0033265770485897185
4 32 0.16856780735107732 0.0 0.3548795944233207 0.005174249562279624 0.3269961977186312 0.005174249562279624 0.7777777777777778 0.0047797670042182905 0.7828474862695396 0.0048905098871103924
}{\mistralgsm}

\pgfplotstableread[col sep=space,string type]{
id num mean1 std1 mean2 std2 mean3 std3 mean4 std4 mean5 std5
0 2 0.0 0.0 0.11361200428724545 0.012407113507813749 0.09860664523043944 0.004010350896863818 0.11361200428724545 0.012407113507813749 0.1747052518756699 0.009939569662910722
1 4 0.11254019292604502 0.0 0.11146838156484458 0.010930374091302864 0.12754555198285103 0.0015157701633152123 0.22186495176848875 0.005250781870917845 0.24544480171489816 0.028919051582490973
2 8 0.13504823151125403 0.0 0.11789924973204717 0.015383387025088234 0.1639871382636656 0.004547310489945637 0.31939978563772775 0.012950745952405758 0.34941050375133975 0.020051754484319093
3 16 0.13504823151125403 0.0 0.12433011789924973 0.01183854342678163 0.195069667738478 0.007578850816576075 0.4030010718113612 0.017870666667238022 0.4308681672025723 0.01837773654821248
4 32 0.13504823151125403 0.0 0.12433011789924973 0.01347245990351183 0.23365487674169347 0.009939569662910722 0.4919614147909968 0.018377736548212457 0.4962486602357985 0.0015157701633152255
}{\mistralmbpp}

\pgfplotstableread[col sep=space,string type]{
id num mean1 std1 mean2 std2 mean3 std3 mean4 std4 mean5 std5
0 2 0.0 0.0 0.13 0.035590260840104374 0.13333333333333333 0.020548046676563257 0.13 0.035590260840104374 0.18666666666666668 0.01699673171197594
1 4 0.1 0.0 0.12 0.03741657386773942 0.19666666666666666 0.0262466929133727 0.2933333333333334 0.04642796092394707 0.31666666666666665 0.009428090415820642
2 8 0.1 0.0 0.12 0.03741657386773942 0.24333333333333332 0.02867441755680877 0.41 0.029439202887759492 0.43333333333333335 0.012472191289246483
3 16 0.1 0.0 0.10333333333333333 0.02357022603955158 0.30666666666666664 0.016996731711975962 0.5366666666666667 0.026246692913372727 0.5533333333333333 0.009428090415820642
4 32 0.1 0.0 0.10333333333333333 0.02357022603955158 0.3466666666666667 0.0262466929133727 0.64 0.016329931618554536 0.6633333333333334 0.012472191289246483
}{\mistralhumaneval}

\begin{figure*}[t!]
    \centering
    \makeatletter
    % temporally disable hyperref
    \let\ref\@refstar
    \ref{grouplegend}
    \makeatother
    % \vspace{-0.2in}
    \begin{tikzpicture}
        \begin{groupplot}[
            group style={group size=4 by 2, horizontal sep=45pt, vertical sep=15pt},
            width=1.0\textwidth,
            height=0.3\textwidth,
            legend cell align={left},
            legend pos=north west,
            enlargelimits=0.1,
            legend style={
                font=\small,
                draw=none,
                column sep=5pt,
                legend columns=5,
            },
        ]
        \nextgroupplot[
            width=0.25\textwidth,height=0.3\textwidth,
            yticklabel style={/pgf/number format/fixed,/pgf/number format/precision=1},
            xticklabels from table={\llamamath}{num},
            ylabel={MATH},
            ylabel near ticks,
            % xlabel={Budget},
            xlabel near ticks,
            xmajorgrids=true,
            ymajorgrids=true,
            legend pos=south west,
            grid style=dashed,
            xtick={0,...,4},
            every tick label/.append style={font=\small},
            label style={font=\small},
            ylabel style={yshift=0pt},
            legend to name=grouplegend,
            xmin=0,xmax=4,
            % ymax=60,ymin=0,
        ]
            \foreach \c/\marker/\name [count=\n] in {purple/{otimes}/{Revision}, orange/{star}/{Self-Consistency}, lyygreen/{diamond*}/{Self-Refine}, lyyred/{*}/{Best-of-N}, lyyblue/{square*}/{FTTT}}{
                \addplot [name path=lower, fill=none, draw=none, forget plot] table [
                    x=id, y expr=\thisrow{mean\n} - \thisrow{std\n}] {\llamamath};
                \addplot [name path=upper, fill=none, draw=none, forget plot] table [
                    x=id, y expr=\thisrow{mean\n} + \thisrow{std\n}] {\llamamath};
                \expandafter\addplot\expandafter [\c!60, semitransparent, forget plot] fill between[of=lower and upper];
                \edef\tempoptions{\c,thick,mark=\marker,mark options={scale=0.7}}
                \expandafter\addplot\expandafter [\tempoptions] table [
                    x=id, y=mean\n,
                ] {\llamamath};
                \expandafter\addlegendentry\expandafter{\name}
                % \pgfplotstablegetelem{\n}{name}\of{\modeldata}
                % \addlegendentry{\pgfplotsretval}
            }
            % https://tex.stackexchange.com/questions/705041/newcommand-and-macro-expansion-in-pgfplots
        \nextgroupplot[
            width=0.25\textwidth,height=0.3\textwidth,
            yticklabel style={/pgf/number format/fixed,/pgf/number format/precision=1},
            xticklabels from table={\llamagsm}{num},
            ylabel={GSM8K},
            ylabel near ticks,
            % xlabel={Budget},
            xlabel near ticks,
            xmajorgrids=true,
            ymajorgrids=true,
            grid style=dashed,
            xtick={0,...,4},
            every tick label/.append style={font=\small},
            label style={font=\small},
            ylabel style={yshift=0pt},
            xmin=0,xmax=4,
            % ymax=60,ymin=0,
        ]
            \foreach \c/\marker/\name [count=\n] in {purple/{otimes}/{Revision}, orange/{star}/{Stochastic Beam Search}, lyygreen/{diamond*}/{Self-Refine}, lyyred/{*}/{Best-of-N}, lyyblue/{square*}/{TTT}}{
                \addplot [name path=lower, fill=none, draw=none, forget plot] table [
                    x=id, y expr=\thisrow{mean\n} - \thisrow{std\n}] {\llamagsm};
                \addplot [name path=upper, fill=none, draw=none, forget plot] table [
                    x=id, y expr=\thisrow{mean\n} + \thisrow{std\n}] {\llamagsm};
                \expandafter\addplot\expandafter [\c!60, semitransparent, forget plot] fill between[of=lower and upper];
                \edef\tempoptions{\c,thick,mark=\marker,mark options={scale=0.7}}
                \expandafter\addplot\expandafter [\tempoptions] table [
                    x=id, y=mean\n,
                ] {\llamagsm};
            }
        \nextgroupplot[
            width=0.25\textwidth,height=0.3\textwidth,
            yticklabel style={/pgf/number format/fixed,/pgf/number format/precision=1},
            xticklabels from table={\llamambpp}{num},
            ylabel={MBPP},
            ylabel near ticks,
            % xlabel={Budget},
            xlabel near ticks,
            xmajorgrids=true,
            ymajorgrids=true,
            grid style=dashed,
            xtick={0,...,4},
            every tick label/.append style={font=\small},
            label style={font=\small},
            ylabel style={yshift=0pt},
            xmin=0,xmax=4,
            % ymax=60,ymin=0,
        ]
            \foreach \c/\marker/\name [count=\n] in {purple/{otimes}/{Revision}, orange/{star}/{Stochastic Beam Search}, lyygreen/{diamond*}/{Self-Refine}, lyyred/{*}/{Best-of-N}, lyyblue/{square*}/{TTT}}{
                \addplot [name path=lower, fill=none, draw=none, forget plot] table [
                    x=id, y expr=\thisrow{mean\n} - \thisrow{std\n}] {\llamambpp};
                \addplot [name path=upper, fill=none, draw=none, forget plot] table [
                    x=id, y expr=\thisrow{mean\n} + \thisrow{std\n}] {\llamambpp};
                \expandafter\addplot\expandafter [\c!60, semitransparent, forget plot] fill between[of=lower and upper];
                \edef\tempoptions{\c,thick,mark=\marker,mark options={scale=0.7}}
                \expandafter\addplot\expandafter [\tempoptions] table [
                    x=id, y=mean\n,
                ] {\llamambpp};
            }
        \nextgroupplot[
            width=0.25\textwidth,height=0.3\textwidth,
            yticklabel style={/pgf/number format/fixed,/pgf/number format/precision=1},
            xticklabels from table={\llamahumaneval}{num},
            ylabel={HumanEval},
            ylabel near ticks,
            % xlabel={Budget},
            xlabel near ticks,
            xmajorgrids=true,
            ymajorgrids=true,
            grid style=dashed,
            xtick={0,...,4},
            every tick label/.append style={font=\small},
            label style={font=\small},
            ylabel style={yshift=0pt},
            xmin=0,xmax=4,
            % ymax=60,ymin=0,
        ]
            \foreach \c/\marker/\name [count=\n] in {purple/{otimes}/{Revision}, orange/{star}/{Stochastic Beam Search}, lyygreen/{diamond*}/{Self-Refine}, lyyred/{*}/{Best-of-N}, lyyblue/{square*}/{TTT}}{
                \addplot [name path=lower, fill=none, draw=none, forget plot] table [
                    x=id, y expr=\thisrow{mean\n} - \thisrow{std\n}] {\llamahumaneval};
                \addplot [name path=upper, fill=none, draw=none, forget plot] table [
                    x=id, y expr=\thisrow{mean\n} + \thisrow{std\n}] {\llamahumaneval};
                \expandafter\addplot\expandafter [\c!60, semitransparent, forget plot] fill between[of=lower and upper];
                \edef\tempoptions{\c,thick,mark=\marker,mark options={scale=0.7}}
                \expandafter\addplot\expandafter [\tempoptions] table [
                    x=id, y=mean\n,
                ] {\llamahumaneval};
            }
            
        \nextgroupplot[
            width=0.25\textwidth,height=0.3\textwidth,
            yticklabel style={/pgf/number format/fixed,/pgf/number format/precision=1},
            xticklabels from table={\mistralmath}{num},
            ylabel={MATH},
            ylabel near ticks,
            xlabel={Budget},
            xlabel near ticks,
            xmajorgrids=true,
            ymajorgrids=true,
            legend pos=south west,
            grid style=dashed,
            xtick={0,...,4},
            every tick label/.append style={font=\small},
            label style={font=\small},
            ylabel style={yshift=0pt},
            xmin=0,xmax=4,
            % ymax=60,ymin=0,
        ]
            \foreach \c/\marker/\name [count=\n] in {purple/{otimes}/{Revision}, orange/{star}/{Self-Consistency}, lyygreen/{diamond*}/{Self-Refine}, lyyred/{*}/{Best-of-N}, lyyblue/{square*}/{FTTT}}{
                \addplot [name path=lower, fill=none, draw=none, forget plot] table [
                    x=id, y expr=\thisrow{mean\n} - \thisrow{std\n}] {\mistralmath};
                \addplot [name path=upper, fill=none, draw=none, forget plot] table [
                    x=id, y expr=\thisrow{mean\n} + \thisrow{std\n}] {\mistralmath};
                \expandafter\addplot\expandafter [\c!60, semitransparent, forget plot] fill between[of=lower and upper];
                \edef\tempoptions{\c,thick,mark=\marker,mark options={scale=0.7}}
                \expandafter\addplot\expandafter [\tempoptions] table [
                    x=id, y=mean\n,
                ] {\mistralmath};
            }
        \nextgroupplot[
            width=0.25\textwidth,height=0.3\textwidth,
            yticklabel style={/pgf/number format/fixed,/pgf/number format/precision=1},
            xticklabels from table={\mistralgsm}{num},
            ylabel={GSM8K},
            ylabel near ticks,
            xlabel={Budget},
            xlabel near ticks,
            xmajorgrids=true,
            ymajorgrids=true,
            grid style=dashed,
            xtick={0,...,4},
            every tick label/.append style={font=\small},
            label style={font=\small},
            ylabel style={yshift=0pt},
            xmin=0,xmax=4,
            % ymax=60,ymin=0,
        ]
            \foreach \c/\marker/\name [count=\n] in {purple/{otimes}/{Revision}, orange/{star}/{Stochastic Beam Search}, lyygreen/{diamond*}/{Self-Refine}, lyyred/{*}/{Best-of-N}, lyyblue/{square*}/{FTTT}}{
                \addplot [name path=lower, fill=none, draw=none, forget plot] table [
                    x=id, y expr=\thisrow{mean\n} - \thisrow{std\n}] {\mistralgsm};
                \addplot [name path=upper, fill=none, draw=none, forget plot] table [
                    x=id, y expr=\thisrow{mean\n} + \thisrow{std\n}] {\mistralgsm};
                \expandafter\addplot\expandafter [\c!60, semitransparent, forget plot] fill between[of=lower and upper];
                \edef\tempoptions{\c,thick,mark=\marker,mark options={scale=0.7}}
                \expandafter\addplot\expandafter [\tempoptions] table [
                    x=id, y=mean\n,
                ] {\mistralgsm};
            }
        \nextgroupplot[
            width=0.25\textwidth,height=0.3\textwidth,
            yticklabel style={/pgf/number format/fixed,/pgf/number format/precision=1},
            xticklabels from table={\mistralmbpp}{num},
            ylabel={MBPP},
            ylabel near ticks,
            xlabel={Budget},
            xlabel near ticks,
            xmajorgrids=true,
            ymajorgrids=true,
            grid style=dashed,
            xtick={0,...,4},
            every tick label/.append style={font=\small},
            label style={font=\small},
            ylabel style={yshift=0pt},
            xmin=0,xmax=4,
            % ymax=60,ymin=0,
        ]
            \foreach \c/\marker/\name [count=\n] in {purple/{otimes}/{Revision}, orange/{star}/{Stochastic Beam Search}, lyygreen/{diamond*}/{Self-Refine}, lyyred/{*}/{Best-of-N}, lyyblue/{square*}/{FTTT}}{
                \addplot [name path=lower, fill=none, draw=none, forget plot] table [
                    x=id, y expr=\thisrow{mean\n} - \thisrow{std\n}] {\mistralmbpp};
                \addplot [name path=upper, fill=none, draw=none, forget plot] table [
                    x=id, y expr=\thisrow{mean\n} + \thisrow{std\n}] {\mistralmbpp};
                \expandafter\addplot\expandafter [\c!60, semitransparent, forget plot] fill between[of=lower and upper];
                \edef\tempoptions{\c,thick,mark=\marker,mark options={scale=0.7}}
                \expandafter\addplot\expandafter [\tempoptions] table [
                    x=id, y=mean\n,
                ] {\mistralmbpp};
            }
        \nextgroupplot[
            width=0.25\textwidth,height=0.3\textwidth,
            yticklabel style={/pgf/number format/fixed,/pgf/number format/precision=1},
            xticklabels from table={\mistralhumaneval}{num},
            ylabel={HumanEval},
            ylabel near ticks,
            xlabel={Budget},
            xlabel near ticks,
            xmajorgrids=true,
            ymajorgrids=true,
            grid style=dashed,
            xtick={0,...,4},
            every tick label/.append style={font=\small},
            label style={font=\small},
            ylabel style={yshift=0pt},
            xmin=0,xmax=4,
            % ymax=60,ymin=0,
        ]
            \foreach \c/\marker/\name [count=\n] in {purple/{otimes}/{Revision}, orange/{star}/{Stochastic Beam Search}, lyygreen/{diamond*}/{Self-Refine}, lyyred/{*}/{Best-of-N}, lyyblue/{square*}/{FTTT}}{
                \addplot [name path=lower, fill=none, draw=none, forget plot] table [
                    x=id, y expr=\thisrow{mean\n} - \thisrow{std\n}] {\mistralhumaneval};
                \addplot [name path=upper, fill=none, draw=none, forget plot] table [
                    x=id, y expr=\thisrow{mean\n} + \thisrow{std\n}] {\mistralhumaneval};
                \expandafter\addplot\expandafter [\c!60, semitransparent, forget plot] fill between[of=lower and upper];
                \edef\tempoptions{\c,thick,mark=\marker,mark options={scale=0.7}}
                \expandafter\addplot\expandafter [\tempoptions] table [
                    x=id, y=mean\n,
                ] {\mistralhumaneval};
            }
        \end{groupplot}
    \end{tikzpicture}
    \caption{The scaling trends of different methods under varying budgets. The colored area around the line denotes the standard deviation. The first row is the results of \texttt{Llama-3.1-8B-Instruct} and the second row is \texttt{Mistral-7B-Instruct-v0.3}.}
    \label{fig:scale}
    % \vspace{-0.4cm}
\end{figure*}
\subsection{SpMV}
% \begin{table}[ht]
% \centering
% \begin{tabular}{|l|r|r|}
% \hline
% \textbf{Dataset}        & \textbf{Rows} & \textbf{NNZ}      \\ \hline
% dc2                     & 116835        & 766396            \\ \hline
% scircuit                & 170998        & 958936            \\ \hline
% mac\_econ\_fwd500       & 206500        & 1273389           \\ \hline
% conf5\_4-8x8-10         & 49152         & 1916928           \\ \hline
% mc2depi                 & 525825        & 2100225           \\ \hline
% rma10                   & 46835         & 2374001           \\ \hline
% cop20k\_A               & 121192        & 2624331           \\ \hline
% webbase-1M              & 1000005       & 3105536           \\ \hline
% ASIC\_680k              & 682862        & 3871773           \\ \hline
% cant                    & 62451         & 4007383           \\ \hline
% pdb1HYS                 & 36417         & 4344765           \\ \hline
% consph                  & 83334         & 6010480           \\ \hline
% shipsec1                & 140874        & 7813404           \\ \hline
% mip1                    & 66463         & 10352819          \\ \hline
% pwtk                    & 217918        & 11634424          \\ \hline
% Si41Ge41H72             & 185639        & 15011265          \\ \hline
% in-2004                 & 1382908       & 16917053          \\ \hline
% Ga41As41H72             & 268096        & 18488476          \\ \hline
% eu-2005                 & 862664        & 19235140          \\ \hline
% FullChip                & 2987012       & 26621990          \\ \hline
% circuit5M               & 5558326       & 59524291          \\ \hline
% \end{tabular}
% \caption{Summary of Datasets with Rows and NNZ}
% \label{tab:datasets}
% \end{table}


\begin{table*}[ht]
    \caption{Datasets for the SpMV benchmark (from DASP~\cite{10.1145/3581784.3607051}). The dataset is ranked by the non-zero value size}
    \vspace{-14pt}
\footnotesize
    {%
    \begin{tabular}[t]{|cccc|cccc|cccc|}
    \hline
        \textbf{Code} & \textbf{Name~\cite{davis2011university}} & \textbf{Rows} & \textbf{NNZ} \\ \hline
        \textbf{D1}   & dc2                                    & 116,835       & 766,396       \\
        \textbf{D2}   & scircuit                               & 170,998       & 958,936       \\
        \textbf{D3}   & mac\_econ\_fwd500                      & 206,500       & 1,273,389     \\
        \textbf{D4}   & conf5\_4-8x8-10                        & 49,152        & 1,916,928     \\
        \textbf{D5}   & mc2depi                                & 525,825       & 2,100,225     \\
        \textbf{D6}   & rma10                                  & 46,835        & 2,374,001     \\
        \textbf{D7}   & cop20k\_A                              & 121,192       & 2,624,331     \\ \hline
    \end{tabular}
    \begin{tabular}[t]{|cccc|cccc|cccc|}
    \hline
        \textbf{Code} & \textbf{Name~\cite{davis2011university}} & \textbf{Rows} & \textbf{NNZ} \\ \hline
        \textbf{D8}   & webbase-1M                             & 1,000,005     & 3,105,536     \\
        \textbf{D9}   & ASIC\_680k                             & 682,862       & 3,871,773     \\
        \textbf{D10}  & cant                                   & 62,451        & 4,007,383     \\
        \textbf{D11}  & pdb1HYS                                & 36,417        & 4,344,765     \\
        \textbf{D12}  & consph                                 & 83,334        & 6,010,480     \\
        \textbf{D13}  & shipsec1                               & 140,874       & 7,813,404     \\
        \textbf{D14}  & mip1                                   & 66,463        & 10,352,819    \\ \hline
    \end{tabular}
    \begin{tabular}[t]{|cccc|cccc|cccc|}
    \hline
        \textbf{Code} & \textbf{Name~\cite{davis2011university}} & \textbf{Rows} & \textbf{NNZ} \\ \hline
        \textbf{D15}  & pwtk                                   & 217,918       & 11,634,424    \\
        \textbf{D16}  & Si41Ge41H72                            & 185,639       & 15,011,265    \\
        \textbf{D17}  & in-2004                                & 1,382,908     & 16,917,053    \\
        \textbf{D18}  & Ga41As41H72                            & 268,096       & 18,488,476    \\
        \textbf{D19}  & eu-2005                                & 862,664       & 19,235,140    \\
        \textbf{D20}  & FullChip                               & 2,987,012     & 26,621,990    \\ 
        \textbf{D21}  & circuit5M                              & 5,558,326     & 59,524,291    \\ \hline
    \end{tabular}
    }
    \label{tab:matrxset}
\end{table*}

We evaluate performance using the same 21 representative sparse matrices from the DASP study~\cite{10.1145/3581784.3607051} (Table~\ref{tab:matrxset}).

\noindent\textbf{DASP~\cite{10.1145/3581784.3607051} (Tensor Core):} 
DASP, the SoTA tensor core SpMV implementation, employs a hybrid approach, categorizing matrix rows as long, middle, or small, and applies specialized processing strategies for each category, including row sorting for middle-length rows.

\noindent\textbf{cuSPARSE CSR~\cite{naumov2010cusparse} (CUDA Core):}
While formats with reordering (e.g., SELL-C-$\sigma$\cite{doi:10.1137/130930352}) might provide a more direct comparison to DASP, sorting can alter matrix characteristics and complicate performance analysis\cite{anzt2014implementing}. Therefore, we use the widely adopted cuSPARSE CSR format~\cite{naumov2010cusparse} as our baseline.


% We use the same representative 21 datasets as DASP~\cite{10.1145/3581784.3607051}. Details are listed in Table~\ref{tab:matrxset}. 
% DASP is the state-of-the-art tensor core spmv implementation available on the net. It separated the sparse matrix into three categories, long, middle, and small rows, then used three different ways to handle the three categories of rows, among which it would sort the rows with middle sizes. 

% A fair comparison to DASP would be a sparse format with reordering (e.g., SELL-C-$\sigma$~\cite{doi:10.1137/130930352}). Yet as the paper~\cite{anzt2014implementing} mentioned sorting might influence the characteristics of the matrix and also it is hard to analyze the performance. So we simply use the cuSPARSE CSR~\cite{naumov2010cusparse} format as it is most used. 

% \subsubsection{Evaluation}




% We use the same stencil benchmark as ConvStencil~\cite{10.1145/3627535.3638476}. Details are listed in Table~\ref{tab:stencilbench}. Here we only conduct experiment in A100 platform because both ConvStencil's and LoRAStencil's AD/AE code has bugs in GH200.
% ConvStencil transformed stencil computation to matrix-matrix multiplication to use the tensor core. To better use the power of the tensor core, it applied kernel fusion, a.k.a temporal blocking in stencil. 

% For evaluation, we use the default setting (including the default domain size) in their papers

% Lorastencil innovatively uses Low-Rank adaptation in stencil computation to reduce redundant computation in stencil. However, if what the author's claim being correct, the roofline plot of LoraStencil would be moved left to be even more memory-bound. 

% We intended to include this new sota in our comparison. However as we checked the AD/AE, we found that the error rate of the implementation is unacceptable (Root Mean Square $RMS\approx 109$ in $1024^2$ domain while  $RMS\approx 0.49$ in $512^2$ domain). We decided to refrain from using the AD/AE code for evaluation. Instead, we used a "visual measurement" from their paper~\cite{lorastencil} for comparison. 

\subsection{Iterative Stencils}


\begin{table}[t]
    \vspace{-8pt}
    \centering
    \caption{\label{tab:stencilbench}Stencil benchmarks and domain sizes we use. A detailed description of the stencil benchmarks can be found in~\cite{zhao2019exploiting,rawat2016effective}.% We included the theoretical operation intensity value of each stencil benchmark.
    }
    \vspace{-10pt}
    {%
    \footnotesize
\begin{tabular}{l| c c c}
\toprule
\multirow{2}{*}{}&\multicolumn{3}{c}{\textbf{domain}(\textbf{temporal blocking depth})} \\
\cmidrule(r){2-4} 
        & ConvStencil~\cite{10.1145/3627535.3638476} & Brick~\cite{zhao2019exploiting} & EBISU~\cite{10.1145/3577193.3593716}  \\
\midrule
\textbf{2d5pt}   & $10240^2$(3) &  -       & $9000^2$(3)  \\
\textbf{2d13pt}  & $10240^2$(1)   &   -      &  $9000^2$(1)    \\
\textbf{2d9pt}   & $10240^2$(3) &    -     &  $9000^2$(3) \\
\textbf{2d49pt}  & $10240^2$(1)   &     -    &  $9000^2$(1)    \\
\textbf{3d7pt}   & $1024^3$(3)   & $512^3$(1)   & $234\times312\times2560$(3)     \\
\textbf{3d27pt}  & $1024^3$(3)  & $512^3$(1)   & $234\times312\times2560$(3)     \\
\bottomrule
    \end{tabular}

    }
    \label{tab:domain}
\end{table}

% \begin{table*}[t]
%     \centering
%     \caption{\label{tab:stencilbench}Stencil benchmarks and domain sizes we use. A detailed description of the stencil benchmarks can be found in~\cite{zhao2019exploiting,rawat2016effective}. We included the theoretical operation intensity value of each stencil benchmark.
%     }
%     {%
%     \footnotesize
% \begin{tabular}{l| c c c c c c c c}
% \toprule
% \multirow{2}{*}{}&\multicolumn{3}{c}{\textbf{domain}} & \multicolumn{3}{c}{\textbf{temporal blocking depth}} & \multicolumn{2}{c}{\textbf{$\mathbb{I}$}} \\
% \cmidrule(r){2-4} \cmidrule(lr){5-7} \cmidrule(l){8-9}
%         & ConvStencil~\cite{10.1145/3627535.3638476} & Brick~\cite{zhao2019exploiting} & EBISU~\cite{10.1145/3577193.3593716} & ConvStencil~\cite{10.1145/3627535.3638476} & Brick~\cite{zhao2019exploiting} & EBISU~\cite{10.1145/3577193.3593716} & w/o temp. blk. & w temp. blk. \\
% \midrule
% \textbf{2d5pt}   & $10240\times10240$ &  -       & $9000\times9000$ & $3$ &     -    & $3$ & $0.625$ & $1.875$ \\
% \textbf{2d13pt}  & $10240\times10240$ &   -      &  $9000\times9000$ & $1$ &    -     & $1$ & $1.625$ &       \\
% \textbf{2d9pt}   & $10240\times10240$ &    -     &  $9000\times9000$ & $3$ &    -     & $3$ & $1.125$ & $3.375$ \\
% \textbf{2d49pt}  & $10240\times10240$ &     -    &  $9000\times9000$ & $1$ &     -    & $1$ & $6.125$ &       \\
% \textbf{3d7pt}   & $1024\times1024\times1024$ & $512\times512\times512$ & $234\times312\times2560$ & $3$ & $1$ & $3$ & $0.875$ & $2.625$ \\
% \textbf{3d27pt}  & $1024\times1024\times1024$ & $512\times512\times512$ & $234\times312\times2560$ & $3$ & $1$ & $3$ & $3.375$ & $10.125$ \\
% \bottomrule

%     \end{tabular}

%     }
%     \label{tab:domain}
% \end{table*}

Stencil implementations were evaluated using ConvStencil~\cite{10.1145/3627535.3638476} benchmark suite (Table~\ref{tab:stencilbench}). %Due to ConvStencil and LoRAstencil having bugs in GH200, our experiments are limited to the A100 platform.
Due to bugs in both ConvStencil's and LoRAStencil's Artifact Description/Artifact Evaluation (AD/AE) on the GH200 platform, our experiments are restricted to the A100 platform.
% Due to ConvStencil and LoRAStencil having bugs in GH200, our experiments are limited to the A100 platform.

\noindent\textbf{ConvStencil~\cite{10.1145/3627535.3638476} (Tensor Core):}
ConvStencil leverages tensor cores by transforming stencil computation into matrix-matrix multiplication, incorporating temporal blocking through kernel fusion. We evaluate using their default configuration and domain sizes.

\noindent\textbf{LoRAStencil~\cite{lorastencil} (Tensor Core):} LoRAStencil applies Low-Rank adaptation to reduce stencil computational redundancy. While innovative, their artifact evaluation relies on assumptions about the rank of stencil weights, which limits its practical applicability. Due to these constraints, we use their published performance results for comparison.
%we observed significant accuracy issues in its AD/AE
% ~\footnote {$error=\tfrac{\| \hat{A} - A_{ref} \|_F}{\| A_{ref} \|_F}$.
% For a $512^2$ domain, the error was $0.01$; for a $1024^2$ domain, the error increased to $2.19$; and for a $10240^2$ domain, the error escalated to $73.03$.}
% . 
% 

\noindent\textbf{Brick~\cite{zhao2019exploiting} (CUDA Core):} Baseline CUDA Core implementation without temporal blocking. We evaluated it using the default configuration.

\noindent\textbf{EBISU~\cite{10.1145/3577193.3593716} (CUDA Core):} EBISU is a state-of-the-art CUDA Core implementation that incorporates temporal blocking. To ensure a fair comparison, we configured EBISU's temporal blocking parameters to match those of ConvStencil.

% ~\footnote {\[
% error=\frac{\| \hat{A} - A_{ref} \|_\infty}{\| A_{ref} \|_\infty}
% \]
% In domain  $512^2$ error = 0.01, in domain $1024^2$  error= 6.90, in domain $10240^2$ error= 182.72.}
% For evaluation, we use the default setting (including the default domain size) in their papers.



\section{Evaluation}
We provide three sets of insights into this section, organised as \textit{findings (F*)}. We quantitatively study the effect of the adversarial and counterfactual perturbations on the performance of informal reasoners and autoformalisation methods. Then, we dive deeper into method variants. Finally, 
we analyse the nature of formalisation errors made by the models.

\subsection{Robustness Analysis}
\paragraph{\textbf{\emph{F1: Noise perturbations have a stronger effect on formalisation methods than informal \ac{LLM} reasoners.}}}
Table~\ref{tab:distraction_k4_formalisation} shows that, on average, the accuracy of both direct and \ac{CoT} informal reasoning remains between $73\%$ and $74\%$ in the face of added noise. While the autoformalisation method performs similarly to informal reasoners on the original dataset, its performance decreases between $4\%$ and $11\%$. The accuracy drops especially with logical (L) and tautological (T) distractions, whose logical language formats trick the \ac{LLM} into formalizing the noisy clauses. On the other hand, the linguistically complex and more natural sentences of encyclopedic distractions show a minor effect, suggesting that \acp{LLM} successfully avoids formalizing the more complicated sentences.

\paragraph{\textbf{\emph{F2: All \ac{LLM}-based reasoning methods suffer a drop for counterfactual perturbations.}}} % influence .}}}
Table~\ref{tab:distraction_k4_formalisation} shows that counterfactual statements cause a significant decrease in performance for both the informal reasoners and autoformalisation methods of between $12\%$ and $13\%$ on average. 
Moreover, this observation also holds for all tested models, i.e., none are robust towards counterfactual perturbations across every evaluated dimension. Even the strongest model, GPT 4o-mini, yields a performance of 63-68\%, which is relatively close to the random performance of 50\%. The high impact of counterfactual statements (the single ``not'' inserted) could be due to the inability of \acp{LLM} to overwrite prior knowledge with explicitly stated information or memorization of the answers. We study the error sources further in §\ref{subsec:errors}.  

\noindent \paragraph{\textbf{\emph{F3: Introducing multiple noise sentences has an effect only for logical distractions.}}}
We show the impact of introducing between one and four sentences for the two top-performing autoformalisation models in Figure~\ref{fig:length_distraction}. The figure shows similar trends with and without counterfactual perturbations.
As additional logical distractions are introduced, the model performance consistently decreases. Tautological (T) distractions lead to a decline in accuracy with a single disruptive sentence, yet adding more noise does not worsen the outcome. 
The tautological corpus introduces truth constants for all sentences as a persistent unseen logical construct. Given that this leads only to a decrease for a single occurrence, we can assume that a model can consistently handle the same unseen logical construct. In contrast, the logical corpus increases the chance of adding text, requiring new, previously unseen reasoning constructs for each added sentence. The impact of encyclopedic noise remains negligible, generalising F1 to $k$ sentences. Similarly, counterfactual perturbations remain much more effective for all settings, generalising F2.

\begin{table}[!t]
\small
\setlength{\modelspacing}{2pt}
\setlength{\tabcolsep}{1.7pt} % Default value: 6pt
\setlength{\belowrulesep}{4pt}
\begin{threeparttable}
    \centering
    \begin{tabular}{cc l r rrr @{\quad} rrrr}
\toprule
\multirow{2}{*}{} & \multirow{2}{*}{} & Reasoning & \multirow{2}{*}{O} & \multicolumn{3}{c}{Distraction} & \multicolumn{4}{c}{Counterfactual} \\
 & & Format & & E& L & T & $\text{O}_C$ & $\text{E}_C$& $\text{L}_C$ & $\text{T}_C$\\
\midrule
\multirow{6}{*}{\rotatebox{90}{Gemma-2}} & \multirow{3}{*}{\rotatebox{90}{9b}}
   & Informal (direct) & \textbf{0.78} & \textbf{0.80} & \textbf{0.79} & \textbf{0.77} & 0.58 & 0.52 & 0.50 & 0.59 \\
 & & Informal (CoT) & 0.72 & 0.78 & 0.73 & 0.76 & 0.61 & \textbf{0.57} & \textbf{0.60} & \textbf{0.66} \\
 & & Formal (FOL) & 0.62 & 0.58 & 0.52 & 0.53 & \textbf{0.63} & 0.52 & 0.46 & 0.46 \\[\modelspacing]
\cmidrule{2-11}
 & \multirow{3}{*}{\rotatebox{90}{27b}} 
   & Informal (direct) & 0.71 & 0.69 & \textbf{0.66} & \textbf{0.68} & 0.59 & 0.51 & 0.54 & 0.59 \\
 & & Informal (CoT) & 0.66 & 0.65 & 0.64 & 0.63 & 0.62 & 0.58 & \textbf{0.62} & \textbf{0.64} \\
 & & Formal (FOL) & \textbf{0.74} & \textbf{0.74} & 0.61 & 0.61 & \underline{\textbf{0.72}} & \underline{\textbf{0.67}} & 0.58 & 0.51 \\[\modelspacing]
\midrule
\multirow{6}{*}{\rotatebox{90}{Mistral}} & \multirow{3}{*}{\rotatebox{90}{7B}} 
   & Informal (direct) & 0.77 & \textbf{0.77} & 0.75 & \textbf{0.79} & \textbf{0.63} & \textbf{0.54} & \textbf{0.54} & \textbf{0.66} \\
 & & Informal (CoT) & \textbf{0.79} & 0.75 & \textbf{0.77} & 0.78 & 0.55 & 0.52 & \textbf{0.54} & 0.58 \\
 & & Formal (FOL) & 0.62 & 0.58 & 0.54 & 0.57 & 0.50 & \textbf{0.54} & 0.51 & 0.52 \\[\modelspacing]
\cmidrule{2-11}
 & \multirow{3}{*}{\rotatebox{90}{Small}} 
   & Informal (direct) & \textbf{0.77} & \textbf{0.76} & \textbf{0.76} & \textbf{0.75} & 0.61 & 0.51 & 0.56 & 0.59 \\
 & & Informal (CoT) & 0.72 & 0.72 & 0.72 & 0.71 & \textbf{0.62} & \textbf{0.59} & \textbf{0.62} & \textbf{0.68} \\
 & & Formal (FOL) & 0.68 & 0.59 & 0.53 & 0.64 & 0.54 & 0.55 & 0.49 & 0.51 \\[\modelspacing]
\midrule
\multirow{6}{*}{\rotatebox{90}{Llama-3.1}} & \multirow{3}{*}{\rotatebox{90}{8B}} 
   & Informal (direct) & 0.63 & 0.61 & 0.64 & 0.66 & 0.61 & \textbf{0.62} & 0.59 & 0.61 \\
 & & Informal (CoT) & 0.73 & \textbf{0.73} & \textbf{0.71} & \textbf{0.72} & \textbf{0.62} & 0.59 & \textbf{0.61} & \textbf{0.65} \\
 & & Formal (FOL) & \textbf{0.77} & 0.71 & 0.63 & 0.52 & 0.60 & 0.58 & 0.55 & 0.52 \\[\modelspacing]
\cmidrule{2-11}
 & \multirow{3}{*}{\rotatebox{90}{70B}} 
   & Informal (direct) & 0.77 & 0.74 & 0.74 & 0.73 & 0.62 & 0.53 & 0.56 & 0.64 \\
 & & Informal (CoT) & \textbf{0.78} & \textbf{0.75} & \textbf{0.76} & \textbf{0.76} & 0.64 & 0.61 & \textbf{0.66} & \underline{\textbf{0.73}} \\
 & & Formal (FOL) & 0.74 & 0.73 & 0.71 & 0.71 & \textbf{0.66} & \textbf{0.62} & 0.59 & 0.57 \\[\modelspacing]
 \midrule
\multirow{3}{*}{\rotatebox{90}{GPT}} & \multirow{3}{*}{\rotatebox{90}{4o-mini}} 
   & Informal (direct) & 0.78 & 0.77 & 0.79 & 0.79 & 0.64 & 0.61 & 0.61 & 0.63 \\
 & & Informal (CoT) & 0.80 & 0.80 & \underline{\textbf{0.81}} & \underline{\textbf{0.82}} & \textbf{0.68} & \textbf{0.63} & \underline{\textbf{0.68}} & \textbf{0.64} \\
 & & Formal (FOL) & \underline{\textbf{0.84}} & \underline{\textbf{0.82}} & 0.73 & 0.79 & 0.63 & 0.62 & 0.57 & 0.54 \\[\modelspacing]
 \midrule
\multicolumn{2}{c}{\multirow{3}{*}{\textbf{Avg}}} 
 & Informal (direct) & 0.74 & 0.73 & 0.73 & 0.73 & 0.61 & 0.55 & 0.56 & 0.62 \\
 & & Informal (CoT) & 0.74 & 0.74 & 0.73 & 0.74 & 0.62 & 0.58 & 0.62 & 0.65 \\
  & & Formal (FOL) & 0.72 & 0.68 &	0.61 & 0.62 & 0.61 & 0.59 & 0.54 & 0.52 \\
\bottomrule
\end{tabular}
\caption{Accuracies of informal and autoformalisation-based deductive reasoners. The best overall model per dataset is underlined; the best model version is marked in bold.}
\label{tab:distraction_k4_formalisation}
\end{threeparttable}
\end{table} 

\begin{figure}[!t]
    \centering
    \scriptsize
    \begin{tikzpicture}
        \begin{axis}[name=gpt,
            title={GPT-4o-mini},
            width=0.6\linewidth,
            height=0.6\linewidth,
            xlabel={\# Noise sentences},
            ylabel={Accuracy},
            xmin=-0.1, xmax=4.1,
            ymin=0.5, ymax=0.9,
            xtick={1,2,4},
            ytick={0.55, 0.6, 0.65, 0.75, 0.8, 0.85},
            title style={yshift=-0.6em},
            legend style={at={(1,-0.15)},
	           anchor=north,legend columns=-1},
            x label style={at={(axis description cs:1,-0.05)},anchor=north},
            y label style={at={(axis description cs:-0.15,0.5)},anchor=south},
            ymajorgrids=true,
            grid style=dashed,
        ]
            \addplot[color=blue, mark=square,]
                coordinates {
                (0,0.848076939582825)(1,0.823076903820038)(2,0.826923072338104)(4,0.821153819561005)
                };
            \addplot[color=red, mark=triangle,]
                coordinates {
                (0,0.848076939582825)(1,0.817307710647583)(2,0.801923096179962)(4,0.759615361690521)
                };
            \addplot[color=green, mark=diamond,] 
                coordinates {
                (0,0.848076939582825)(1,0.767307698726654)(2,0.769230782985687)(4,0.803846180438995)
                };
            \addplot[color=blue, mark=square*] 
                coordinates {
                (0,0.627777755260468)(1,0.622222244739533)(2,0.600000023841858)(4,0.633333325386047)
                };
            \addplot[color=red, mark=triangle*,] 
                coordinates {
                (0,0.627777755260468)(1,0.611111104488373)(2,0.611111104488373)(4,0.594444453716278)
                };
            \addplot[color=green, mark=diamond*,] 
                coordinates {
                (0,0.627777755260468)(1,0.572222232818604)(2,0.538888871669769)(4,0.555555582046509)
                };
                \legend{E,L,T,$\text{E}_C$, $\text{L}_C$ , $\text{T}_C$}
        \end{axis}

        \begin{axis}[name=llama, at={($(gpt.east)+(0.1cm,0)$)},anchor=west,
            title={Llama 3.1 70b},
            width=0.6\linewidth,
            height=0.6\linewidth,
            xmin=-0.1,, xmax=4.1,
            ymin=0.5, ymax=0.9,
            xtick={1,2,4},
            ytick={0.55, 0.6, 0.65, 0.75, 0.8, 0.85},
            title style={yshift=-0.6em},
            yticklabel=\empty,
            ymajorgrids=true,
            grid style=dashed,
        ]
            \addplot[color=blue, mark=square,]
                coordinates {
                (0,0.838461518287659)(1,0.817307710647583)(2,0.805769205093384)(4,0.817307710647583)
                };
            \addplot[color=red, mark=triangle,]
                coordinates {
                (0,0.838461518287659)(1,0.819230794906616)(2,0.803846180438995)(4,0.771153867244721)
                };
            \addplot[color=green, mark=diamond,]
                coordinates {
                (0,0.838461518287659)(1,0.803846180438995)(2,0.807692289352417)(4,0.805769205093384)
                };
            \addplot[color=blue, mark=square*]
                coordinates {
                (0,0.627777755260468)(1,0.622222244739533)(2,0.577777802944183)(4,0.594444453716278)
                };
            \addplot[color=red, mark=triangle*,]
                coordinates {
                (0,0.627777755260468)(1,0.583333313465118)(2,0.561111092567444)(4,0.577777802944183)
                };
            \addplot[color=green, mark=diamond*,]
                coordinates {
                (0,0.627777755260468)(1,0.627777755260468)(2,0.566666662693024)(4,0.577777802944183)
                };
        \end{axis}
    \end{tikzpicture}
    \caption{Influence of the number of noisy sentences for FOL.}
    \label{fig:length_distraction}
\end{figure}



\subsection{Impact of Method Design}
\paragraph{\textbf{\emph{F4: \ac{CoT} prompting is most impactful when both noise and counterfactual perturbations are applied.}}}
The accuracies for the individual \acp{LLM} in Table~\ref{tab:distraction_k4_formalisation} show that the impact of \ac{CoT} is negligible for noise-only datasets (first four columns). Meanwhile, the benefit from \ac{CoT} is most pronounced in the datasets that combine noise and counterfactual perturbations.
The better-performing informal prompting strategy for a model remains stable for all types of distractions. Still, the decline in performance due to counterfactuals leads to a less consistent preference for a specific prompting style.

\paragraph{\textbf{\emph{F5: The best-performing grammar differs per model and is unstable across data versions.}}}

The evaluation of different logical forms for formal \ac{LLM}-based reasoning in Table~\ref{tab:distraction_k4_logical_form} shows the preference of some models for specific syntactic formats.
Llama 3.1 70B has a considerable improvement of $12\%$ with TPTP syntax on the original set, while Llama 3.1 8B benefits from the R-FOL syntax. However, all grammars show a declining accuracy trend and increased syntax errors for noise perturbations, where the best grammar loses its advantage over the rest. 
When comparing the grammars on the counterfactual partitions, we observe that TPTP is consistently more robust than the standard first-order logic grammar. Here, GPT 4o-mini shows a reduction from $O$ to $O_C$ of $20\%$ for FOL and only $12\%$ for the TPTP grammar. Since this does not correlate with fewer syntax errors, the formalisation in TPTP prevents semantical errors for counterfactual premises. 
A positive reading of these results, especially the minor differences between FOL and R-FOL, is that autoformalisation \acp{LLM} can adapt to the grammar syntax prescribed in the prompt without further loss in performance.

\begin{table}[!t]
\small
\setlength{\modelspacing}{2pt}
\setlength{\tabcolsep}{1.7pt} % Default value: 6pt
\setlength{\belowrulesep}{4pt}
\begin{threeparttable}
    \centering
    \begin{tabular}{cc l r rrr @{\quad} rrrr}
\toprule
\multirow{2}{*}{} & \multirow{2}{*}{} & Grammar & \multirow{2}{*}{O} & \multicolumn{3}{c}{Distraction} & \multicolumn{4}{c}{Counterfactual} \\
 & & Syntax & & E& L & T & $\text{O}_C$ & $\text{E}_C$& $\text{L}_C$ & $\text{T}_C$\\
\midrule
\multirow{6}{*}{\rotatebox{90}{Llama-3.1}} & \multirow{3}{*}{\rotatebox{90}{8B}} 
   & FOL & 0.77 & \textbf{0.71} & 0.61 & \textbf{0.53} & 0.58 & \textbf{0.55} & 0.52 & \textbf{0.56} \\
 & & R-FOL & \textbf{0.78} & 0.69 & \textbf{0.62} & \textbf{0.53} & 0.58 & \textbf{0.55} & \textbf{0.54} & 0.52 \\
 & & TPTP & 0.73 & 0.67 & 0.55 & 0.51 & \textbf{0.68} & 0.54 & 0.46 & 0.51 \\[\modelspacing]
\cmidrule{2-11}
 & \multirow{3}{*}{\rotatebox{90}{70B}} 
   & FOL & 0.76 & 0.73 & 0.71 & \textbf{0.72} & 0.67 & 0.57 & 0.63 & 0.56 \\
 & & R-FOL & 0.76 & 0.73 & 0.67 & 0.71 & 0.64 & 0.57 & 0.53 & 0.64 \\
 & & TPTP & \underline{\textbf{0.88}} & \underline{\textbf{0.84}} & \underline{\textbf{0.81}} & \textbf{0.72} & \underline{\textbf{0.81}} & \underline{\textbf{0.68}} & \underline{\textbf{0.67}} & \underline{\textbf{0.68}} \\[\modelspacing]
\midrule
\multirow{3}{*}{\rotatebox{90}{GPT}} & \multirow{3}{*}{\rotatebox{90}{4o-mini}} 
   & FOL & \textbf{0.84} & \textbf{0.82} & \textbf{0.72} & \underline{\textbf{0.78}} & 0.64 & \textbf{0.63} & \textbf{0.61} & 0.51 \\
 & & R-FOL & \textbf{0.84} & 0.77 & 0.70 & \underline{\textbf{0.78}} & \textbf{0.72} & 0.56 & 0.54 & \textbf{0.63} \\
 & & TPTP & 0.83 & \textbf{0.82} & 0.71 & 0.71 & 0.69 & \textbf{0.63} & 0.57 & 0.57 \\
\bottomrule
\end{tabular}
\caption{Accuracies of different formalisation grammars for autoformalisation.}
\label{tab:distraction_k4_logical_form}
\end{threeparttable}
\end{table} 

\paragraph{\textbf{\emph{F6: Feedback does not help \acp{LLM} self-correct to mitigate robustness issues.}}}
\autoref{tab:distraction_k4_feedback} shows the results with different error recovery mechanisms. The results indicate that no feedback strategy emerges as a winner in the different datasets. 
All feedback variants reduce syntax errors for noise perturbations, but given the lack of a consistent increase in accuracy, the corrected formalisations are most likely to contain semantic errors still. 
The type of feedback message only has a minor influence on correcting syntax errors, whereas Llama 3.1 70b and GPT 4o-mini correct slightly more syntax errors with specific error messages. This finding aligns with \cite{huang2023large}, who also found that \acp{LLM} cannot consistently self-correct their reasoning after receiving relevant feedback.

\begin{table}[!ht]
\small
\setlength{\modelspacing}{2pt}
\setlength{\tabcolsep}{1.7pt} % Default value: 6pt
\setlength{\belowrulesep}{4pt}
\begin{threeparttable}
    \centering
    \begin{tabular}{cc l r rrr @{\quad} rrrr}
\toprule
\multirow{2}{*}{} & \multirow{2}{*}{} & \multirow{2}{*}{Feedback} & \multirow{2}{*}{O} & \multicolumn{3}{c}{Distraction} & \multicolumn{4}{c}{Counterfactual} \\
 & & & & E& L & T & $\text{O}_C$ & $\text{E}_C$& $\text{L}_C$ & $\text{T}_C$\\
\midrule
\multirow{8}{*}{\rotatebox{90}{Llama-3.1}} & \multirow{4}{*}{\rotatebox{90}{8B}} 
   & No recovery & 0.77 & \textbf{0.72} & 0.62 & 0.53 & 0.59 & 0.58 & 0.56 & \textbf{0.56} \\
 & & Error type & \textbf{0.79} & 0.71 & 0.63 & \textbf{0.56} & \textbf{0.66} & 0.54 & 0.52 & 0.51 \\
 & & Error message & 0.78 & 0.71 & \textbf{0.67} & 0.55 & 0.59 & 0.53 & \underline{\textbf{0.64}} & 0.49 \\
 & & Warning & 0.74 & 0.66 & 0.58 & 0.55 & 0.55 & \textbf{0.60} & 0.49 & 0.49 \\[\modelspacing]
\cmidrule{2-11}
 & \multirow{4}{*}{\rotatebox{90}{70B}} 
   & No recovery & \textbf{0.77} & \textbf{0.72} & \textbf{0.73} & 0.71 & \textbf{0.64} & 0.59 & \textbf{0.61} & 0.56 \\
 & & Error type & 0.72 & 0.70 & 0.72 & \textbf{0.73} & 0.62 & 0.56 & 0.60 & 0.58 \\
 & & Error message & 0.71 & 0.70 & \textbf{0.73} & 0.71 & \textbf{0.64} & 0.59 & 0.54 & \underline{\textbf{0.64}} \\
 & & Warning & 0.69 & \textbf{0.72} & 0.72 & 0.72 & 0.62 & \underline{\textbf{0.65}} & \textbf{0.61} & 0.63 \\[\modelspacing]
\midrule
\multirow{4}{*}{\rotatebox{90}{GPT}} & \multirow{4}{*}{\rotatebox{90}{4o-mini}} 
   & No recovery & \underline{\textbf{0.84}} & \underline{\textbf{0.82}} & 0.73 & 0.79 & 0.64 & \textbf{0.62} & 0.56 & \textbf{0.56} \\
 & & Error type & 0.83 & 0.79 & 0.74 & 0.76 & 0.67 & 0.57 & 0.56 & \textbf{0.56} \\
 & & Error message & \underline{\textbf{0.84}} & 0.78 & \underline{\textbf{0.77}} & \underline{\textbf{0.80}} & 0.62 & 0.59 & 0.56 & \textbf{0.56} \\
 & & Warning & \underline{\textbf{0.84}} & 0.75 & 0.73 & 0.76 & \underline{\textbf{0.70}} & 0.61 & \textbf{0.61} & 0.55 \\
 \bottomrule
\end{tabular}
\caption{Accuracies of error recovery strategies.}
\label{tab:distraction_k4_feedback}
\end{threeparttable}
\end{table} 

\subsection{Error Analysis}
\label{subsec:errors}
\paragraph{\textbf{\emph{F7: Autoformalisation increases syntax errors for noise perturbations.}}}
The low performance for noise perturbations correlates with more syntax errors for all models and distraction categories (cf. execution rates in Table~\ref{tab:appendix_k4_formalisation_exec}). The three worst-performing models (both Mistral models, Gemma-2 9b) generate, at best, for $37\%$  and, at worst, for only $4\%$ of the samples, a valid logical form.
Gemma-2 9b and Llama3.1 8b produce more syntax errors than the larger counterparts, suggesting that larger models are more robust towards noise perturbations. 
The accuracy of syntactically valid samples is higher than the informal reasoning methods for most distractions (Table~\ref{tab:appendix_k4_formalisation_vacc}), motivating informal reasoning as a backup strategy for formal reasoning. The error message feedback reveals two common syntax errors: 1) errors by models with an initial low execution rate exhibit issues with the template structure, including using incorrect keywords or adding conversational phrases;
2) perturbation-related errors, the most common of which is using undefined truth constants as part of tautological distractions. 

\paragraph{\textbf{\emph{F8: Autoformalisation increases semantic errors for counterfactuals.}}}
Unlike the introduced noise, counterfactual perturbations do not lead to more syntax errors. The execution rate in Table~\ref{tab:appendix_k4_formalisation_exec} is stable or improves for counterfactuals. However, we see a drop in accuracy for the counterfactual column $\text{O}_C$ in Table~\ref{tab:distraction_k4_formalisation} and can conclude that the number of logical forms with semantic errors has to increase. This suggests that the introduced negation is not correctly formalised. Looking at the warnings generated by the feedback mechanism, for GPT 4o-mini, $161$ warning messages are generated on the unperturbed data. $54$ of these were fixed with a single iteration. Not considering predicates and individuals as part of the context is the most frequent warning across all models. 



\section{Conclusion}

We systematically evaluated the role of consensus and voting decision protocols across three knowledge and three reasoning tasks.
Our study assessed how the number of discussion rounds and agents influences task performance.
We propose two new methods to improve answer diversity during multi-agent discussions and decisions, i.e., \acf{AAD} and \acf{CI}. 
\ac{AAD} requires each agent to contribute draft ideas at the beginning of the discussion, and \ac{CI} encourages independent reasoning steps by limiting communication between agents and only allowing them to exchange possible solutions after each turn.

Our findings show that voting performs well on reasoning tasks, outperforming consensus by up to $13.2\%$, and outperforming a single \ac{CoT} baseline by 10.4\%.
This is likely because voting-based protocols allow agents to explore multiple reasoning paths instead of a single one, as in consensus.
In comparison, consensus outperforms voting in knowledge tasks by up to $2.8\%$, because it improves fact-checking by requiring at least the agreement of the majority of agents.
Increasing the number of agents in the discussion improved task performance, while increasing the number of discussion rounds before voting decreased performance.
\ac{AAD} improved performance by up to $3.3\%$, and \ac{CI} by up to $7.4\%$ over default multi-agent debate baseline, and 6.1\% and 10.2\% over single model \ac{CoT} baseline respectively.
Our new methods enhance answer diversity and reveal a connection between answer diversity and task performance.

Future work could explore other characteristics influencing decisions, such as power relations between managers and employees. 
This could also involve examining personas within this hierarchical structure to investigate whether dominant or affectionate leaders are more effective in leading discussions \citep{AMES2009111}.


We recommend using voting in reasoning tasks and consensus in knowledge tasks, scaling up the number of agents instead of the number of rounds, and increasing the diversity of answers between agents using \ac{AAD} and \ac{CI}.

\section*{Limitations} %
Multi-agent debates are computationally expensive because they require a message from each agent in each round, quickly leading to hundreds of forward passes per model.
Because of the high computational cost and the range of decision protocols and tasks in our work, we used sampled subsets of the datasets, which can lead to some variance.
To control for that variance, we sampled with a 95\% confidence level and calculated the standard deviation of three independent runs.%
Overall, the results were markedly higher than what could be explained by the standard deviation.
More details about the dataset and other parameters can be found in \Cref{appendix:datasets}.
Despite efforts to improve answer diversity, agents often converged on similar responses, suggesting that more advanced techniques to encourage independent solutions are needed in the future.


% \section{Acknowledgment}
\begin{acks}
This material was supported by the U.S. Dept. of Energy, Office
of Science, Advanced Scientific Computing Research (ASCR), under contracts DE-AC02-06CH11357 and DE-SC002\newline 4207.
\end{acks}
\bibliographystyle{ACM-Reference-Format}
\bibliography{acmart}
\end{document}
\endinput
%%
%% End of file `sample-sigplan.tex'.
