\subsection{SpMV}
% \begin{table}[ht]
% \centering
% \begin{tabular}{|l|r|r|}
% \hline
% \textbf{Dataset}        & \textbf{Rows} & \textbf{NNZ}      \\ \hline
% dc2                     & 116835        & 766396            \\ \hline
% scircuit                & 170998        & 958936            \\ \hline
% mac\_econ\_fwd500       & 206500        & 1273389           \\ \hline
% conf5\_4-8x8-10         & 49152         & 1916928           \\ \hline
% mc2depi                 & 525825        & 2100225           \\ \hline
% rma10                   & 46835         & 2374001           \\ \hline
% cop20k\_A               & 121192        & 2624331           \\ \hline
% webbase-1M              & 1000005       & 3105536           \\ \hline
% ASIC\_680k              & 682862        & 3871773           \\ \hline
% cant                    & 62451         & 4007383           \\ \hline
% pdb1HYS                 & 36417         & 4344765           \\ \hline
% consph                  & 83334         & 6010480           \\ \hline
% shipsec1                & 140874        & 7813404           \\ \hline
% mip1                    & 66463         & 10352819          \\ \hline
% pwtk                    & 217918        & 11634424          \\ \hline
% Si41Ge41H72             & 185639        & 15011265          \\ \hline
% in-2004                 & 1382908       & 16917053          \\ \hline
% Ga41As41H72             & 268096        & 18488476          \\ \hline
% eu-2005                 & 862664        & 19235140          \\ \hline
% FullChip                & 2987012       & 26621990          \\ \hline
% circuit5M               & 5558326       & 59524291          \\ \hline
% \end{tabular}
% \caption{Summary of Datasets with Rows and NNZ}
% \label{tab:datasets}
% \end{table}


\begin{table*}[ht]
    \caption{Datasets for the SpMV benchmark (from DASP~\cite{10.1145/3581784.3607051}). The dataset is ranked by the non-zero value size}
    \vspace{-14pt}
\footnotesize
    {%
    \begin{tabular}[t]{|cccc|cccc|cccc|}
    \hline
        \textbf{Code} & \textbf{Name~\cite{davis2011university}} & \textbf{Rows} & \textbf{NNZ} \\ \hline
        \textbf{D1}   & dc2                                    & 116,835       & 766,396       \\
        \textbf{D2}   & scircuit                               & 170,998       & 958,936       \\
        \textbf{D3}   & mac\_econ\_fwd500                      & 206,500       & 1,273,389     \\
        \textbf{D4}   & conf5\_4-8x8-10                        & 49,152        & 1,916,928     \\
        \textbf{D5}   & mc2depi                                & 525,825       & 2,100,225     \\
        \textbf{D6}   & rma10                                  & 46,835        & 2,374,001     \\
        \textbf{D7}   & cop20k\_A                              & 121,192       & 2,624,331     \\ \hline
    \end{tabular}
    \begin{tabular}[t]{|cccc|cccc|cccc|}
    \hline
        \textbf{Code} & \textbf{Name~\cite{davis2011university}} & \textbf{Rows} & \textbf{NNZ} \\ \hline
        \textbf{D8}   & webbase-1M                             & 1,000,005     & 3,105,536     \\
        \textbf{D9}   & ASIC\_680k                             & 682,862       & 3,871,773     \\
        \textbf{D10}  & cant                                   & 62,451        & 4,007,383     \\
        \textbf{D11}  & pdb1HYS                                & 36,417        & 4,344,765     \\
        \textbf{D12}  & consph                                 & 83,334        & 6,010,480     \\
        \textbf{D13}  & shipsec1                               & 140,874       & 7,813,404     \\
        \textbf{D14}  & mip1                                   & 66,463        & 10,352,819    \\ \hline
    \end{tabular}
    \begin{tabular}[t]{|cccc|cccc|cccc|}
    \hline
        \textbf{Code} & \textbf{Name~\cite{davis2011university}} & \textbf{Rows} & \textbf{NNZ} \\ \hline
        \textbf{D15}  & pwtk                                   & 217,918       & 11,634,424    \\
        \textbf{D16}  & Si41Ge41H72                            & 185,639       & 15,011,265    \\
        \textbf{D17}  & in-2004                                & 1,382,908     & 16,917,053    \\
        \textbf{D18}  & Ga41As41H72                            & 268,096       & 18,488,476    \\
        \textbf{D19}  & eu-2005                                & 862,664       & 19,235,140    \\
        \textbf{D20}  & FullChip                               & 2,987,012     & 26,621,990    \\ 
        \textbf{D21}  & circuit5M                              & 5,558,326     & 59,524,291    \\ \hline
    \end{tabular}
    }
    \label{tab:matrxset}
\end{table*}

We evaluate performance using the same 21 representative sparse matrices from the DASP study~\cite{10.1145/3581784.3607051} (Table~\ref{tab:matrxset}).

\noindent\textbf{DASP~\cite{10.1145/3581784.3607051} (Tensor Core):} 
DASP, the SoTA tensor core SpMV implementation, employs a hybrid approach, categorizing matrix rows as long, middle, or small, and applies specialized processing strategies for each category, including row sorting for middle-length rows.

\noindent\textbf{cuSPARSE CSR~\cite{naumov2010cusparse} (CUDA Core):}
While formats with reordering (e.g., SELL-C-$\sigma$\cite{doi:10.1137/130930352}) might provide a more direct comparison to DASP, sorting can alter matrix characteristics and complicate performance analysis\cite{anzt2014implementing}. Therefore, we use the widely adopted cuSPARSE CSR format~\cite{naumov2010cusparse} as our baseline.


% We use the same representative 21 datasets as DASP~\cite{10.1145/3581784.3607051}. Details are listed in Table~\ref{tab:matrxset}. 
% DASP is the state-of-the-art tensor core spmv implementation available on the net. It separated the sparse matrix into three categories, long, middle, and small rows, then used three different ways to handle the three categories of rows, among which it would sort the rows with middle sizes. 

% A fair comparison to DASP would be a sparse format with reordering (e.g., SELL-C-$\sigma$~\cite{doi:10.1137/130930352}). Yet as the paper~\cite{anzt2014implementing} mentioned sorting might influence the characteristics of the matrix and also it is hard to analyze the performance. So we simply use the cuSPARSE CSR~\cite{naumov2010cusparse} format as it is most used. 

% \subsubsection{Evaluation}


