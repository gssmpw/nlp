%%%%%%%% ICML 2025 EXAMPLE LATEX SUBMISSION FILE %%%%%%%%%%%%%%%%%

\documentclass{article}

% Recommended, but optional, packages for figures and better typesetting:
\usepackage{microtype}
\usepackage{graphicx}
%\usepackage{subfigure}
\usepackage{subcaption}
\usepackage{booktabs} % for professional tables

% hyperref makes hyperlinks in the resulting PDF.
% If your build breaks (sometimes temporarily if a hyperlink spans a page)
% please comment out the following usepackage line and replace
% \usepackage{icml2025} with \usepackage[nohyperref]{icml2025} above.
\usepackage[pagebackref,breaklinks,colorlinks,citecolor=cvprblue,hypertexnames=false]{hyperref}

\usepackage[dvipsnames,table]{xcolor}
\definecolor{lightblue}{RGB}{220, 230, 241}
\definecolor{lightpink}{RGB}{248, 228, 235}
\definecolor{lightyellow}{RGB}{255, 249, 227}
\usepackage{tikz} % Include this in your preamble if not already included
\usepackage{amsmath}
% \usepackage{eso-pic}
% Attempt to make hyperref and algorithmic work together better:
\newcommand{\theHalgorithm}{\arabic{algorithm}}

% Use the following line for the initial blind version submitted for review:
% \usepackage{icml2025}

% If accepted, instead use the following line for the camera-ready submission:
\usepackage[accepted]{icml2025}

% For theorems and such
\usepackage{amsmath}
\usepackage{amssymb}
\usepackage{mathtools}
\usepackage{amsthm}


\usepackage[dvipsnames,table]{xcolor}
\definecolor{lightblue}{RGB}{220, 230, 241}
\definecolor{lightpink}{RGB}{248, 228, 235}
\definecolor{lightyellow}{RGB}{255, 249, 227}
\usepackage{tikz} % Include this in your preamble if not already included
\usepackage{pdfpages}
\usepackage{placeins}
\usepackage{minitoc}
\usepackage{fancyhdr}  % For headers and footers
\usepackage{float}
\usepackage{array}
\newcolumntype{P}[1]{>{\centering\arraybackslash}m{#1}}
\usepackage{newunicodechar}

\usepackage{multirow}



% if you use cleveref..
\usepackage[capitalize,noabbrev]{cleveref}

%%%%%%%%%%%%%%%%%%%%%%%%%%%%%%%%
% THEOREMS
%%%%%%%%%%%%%%%%%%%%%%%%%%%%%%%%
\theoremstyle{plain}
\newtheorem{theorem}{Theorem}[section]
\newtheorem{proposition}[theorem]{Proposition}
\newtheorem{lemma}[theorem]{Lemma}
\newtheorem{corollary}[theorem]{Corollary}
\theoremstyle{definition}
\newtheorem{definition}[theorem]{Definition}
\newtheorem{assumption}[theorem]{Assumption}
\theoremstyle{remark}
\newtheorem{remark}[theorem]{Remark}

% Todonotes is useful during development; simply uncomment the next line
%    and comment out the line below the next line to turn off comments
%\usepackage[disable,textsize=tiny]{todonotes}
\usepackage[textsize=tiny]{todonotes}


% The \icmltitle you define below is probably too long as a header.
% Therefore, a short form for the running title is supplied here:
\icmltitlerunning{Bridging vision language model (VLM) evaluation gaps with a framework for scalable and cost-effective benchmark generation}

\begin{document}

\twocolumn[
\icmltitle{Bridging vision language model (VLM) evaluation gaps with a framework for scalable and cost-effective benchmark generation}

% It is OKAY to include author information, even for blind
% submissions: the style file will automatically remove it for you
% unless you've provided the [accepted] option to the icml2025
% package.

% List of affiliations: The first argument should be a (short)
% identifier you will use later to specify author affiliations
% Academic affiliations should list Department, University, City, Region, Country
% Industry affiliations should list Company, City, Region, Country

% You can specify symbols, otherwise they are numbered in order.
% Ideally, you should not use this facility. Affiliations will be numbered
% in order of appearance and this is the preferred way.
\icmlsetsymbol{tech}{+}
\icmlsetsymbol{dagger}{\dag}

\begin{icmlauthorlist}
\icmlauthor{Tim Rädsch}{dkfz,hidkfz,heidelberguni,tech,dagger}
\icmlauthor{Leon Mayer}{dkfz,tech}
\icmlauthor{Simon Pavicic}{dkfz}
\icmlauthor{A. Emre Kavur}{dkfz,hidkfz}
\icmlauthor{Marcel Knopp}{dkfz,heidelberguni}
\icmlauthor{Barış Öztürk}{dkfz}
\icmlauthor{Klaus Maier-Hein}{dkfz,hidkfz,heidelberguni}
%\icmlauthor{}{sch}
\icmlauthor{Paul F. Jaeger}{dkfz,hidkfz}
\icmlauthor{Fabian Isensee}{dkfz,hidkfz}
\icmlauthor{Annika Reinke}{dkfz,hidkfz,dagger}
\icmlauthor{Lena Maier-Hein}{dkfz,hidkfz,heidelberguni,dagger} 
\vspace{0.5em} \\
\small $\textsuperscript{1}$German Cancer Research Center (DKFZ) Heidelberg, Germany, $\textsuperscript{2}$Helmholtz Imaging, DKFZ, Heidelberg, Germany, $\textsuperscript{3}$Heidelberg University, Heidelberg, Germany, + core contributor, \textdagger{} project lead
%\icmlauthor{}{sch}
%\icmlauthor{}{sch}
\end{icmlauthorlist}

\icmlaffiliation{dkfz}{German Cancer Research Center (DKFZ) Heidelberg, Germany}
\icmlaffiliation{hidkfz}{Helmholtz Imaging, DKFZ, Heidelberg, Germany}
\icmlaffiliation{heidelberguni}{Heidelberg University, Heidelberg, Germany}

\icmlcorrespondingauthor{{tim.raedsch; l.maier-hein}}{@dkfz-heidelberg.de}

% You may provide any keywords that you
% find helpful for describing your paper; these are used to populate
% the "keywords" metadata in the PDF but will not be shown in the document
\icmlkeywords{Machine Learning, ICML}

\vskip 0.3in
]

% this must go after the closing bracket ] following \twocolumn[ ...

% This command actually creates the footnote in the first column
% listing the affiliations and the copyright notice.
% The command takes one argument, which is text to display at the start of the footnote.
% The \icmlEqualContribution command is standard text for equal contribution.
% Remove it (just {}) if you do not need this facility.

%\printAffiliationsAndNotice{}  % leave blank if no need to mention equal contribution
% \printAffiliationsAndNotice{\icmlEqualContribution} % otherwise use the standard text.



\begin{figure*}
  \centering
  \includegraphics[width=\textwidth]{figures/main_text_figs/cvpr_fig_1.pdf}
\caption{\textbf{Summary of contributions}. (1) New concept: We propose a new framework for the resource-efficient creation of domain-specific VLM benchmarks. It is based on the concept of task augmentation designed for creating multiple tasks from a single existing task using metadata annotations from multiple sources (humans, pre-defined heuristics, models). (2) 7 new datasets: We apply our framework to generate seven domain-specific VLM benchmarks with highly reliable reference data. As a unique feature compared to existing benchmarks, we quantify the ambiguity of each question for each image by acquiring human answers from a total of six raters. (3) New insights: We apply our framework to a total of 22 open and frontier closed models to demonstrate the benefit of task augmentation and to shed light on current VLM capabilities.}
  \label{fig:figure1}
\end{figure*}


\begin{abstract}


The choice of representation for geographic location significantly impacts the accuracy of models for a broad range of geospatial tasks, including fine-grained species classification, population density estimation, and biome classification. Recent works like SatCLIP and GeoCLIP learn such representations by contrastively aligning geolocation with co-located images. While these methods work exceptionally well, in this paper, we posit that the current training strategies fail to fully capture the important visual features. We provide an information theoretic perspective on why the resulting embeddings from these methods discard crucial visual information that is important for many downstream tasks. To solve this problem, we propose a novel retrieval-augmented strategy called RANGE. We build our method on the intuition that the visual features of a location can be estimated by combining the visual features from multiple similar-looking locations. We evaluate our method across a wide variety of tasks. Our results show that RANGE outperforms the existing state-of-the-art models with significant margins in most tasks. We show gains of up to 13.1\% on classification tasks and 0.145 $R^2$ on regression tasks. All our code and models will be made available at: \href{https://github.com/mvrl/RANGE}{https://github.com/mvrl/RANGE}.

\end{abstract}

    
\section{Introduction}
Backdoor attacks pose a concealed yet profound security risk to machine learning (ML) models, for which the adversaries can inject a stealth backdoor into the model during training, enabling them to illicitly control the model's output upon encountering predefined inputs. These attacks can even occur without the knowledge of developers or end-users, thereby undermining the trust in ML systems. As ML becomes more deeply embedded in critical sectors like finance, healthcare, and autonomous driving \citep{he2016deep, liu2020computing, tournier2019mrtrix3, adjabi2020past}, the potential damage from backdoor attacks grows, underscoring the emergency for developing robust defense mechanisms against backdoor attacks.

To address the threat of backdoor attacks, researchers have developed a variety of strategies \cite{liu2018fine,wu2021adversarial,wang2019neural,zeng2022adversarial,zhu2023neural,Zhu_2023_ICCV, wei2024shared,wei2024d3}, aimed at purifying backdoors within victim models. These methods are designed to integrate with current deployment workflows seamlessly and have demonstrated significant success in mitigating the effects of backdoor triggers \cite{wubackdoorbench, wu2023defenses, wu2024backdoorbench,dunnett2024countering}.  However, most state-of-the-art (SOTA) backdoor purification methods operate under the assumption that a small clean dataset, often referred to as \textbf{auxiliary dataset}, is available for purification. Such an assumption poses practical challenges, especially in scenarios where data is scarce. To tackle this challenge, efforts have been made to reduce the size of the required auxiliary dataset~\cite{chai2022oneshot,li2023reconstructive, Zhu_2023_ICCV} and even explore dataset-free purification techniques~\cite{zheng2022data,hong2023revisiting,lin2024fusing}. Although these approaches offer some improvements, recent evaluations \cite{dunnett2024countering, wu2024backdoorbench} continue to highlight the importance of sufficient auxiliary data for achieving robust defenses against backdoor attacks.

While significant progress has been made in reducing the size of auxiliary datasets, an equally critical yet underexplored question remains: \emph{how does the nature of the auxiliary dataset affect purification effectiveness?} In  real-world  applications, auxiliary datasets can vary widely, encompassing in-distribution data, synthetic data, or external data from different sources. Understanding how each type of auxiliary dataset influences the purification effectiveness is vital for selecting or constructing the most suitable auxiliary dataset and the corresponding technique. For instance, when multiple datasets are available, understanding how different datasets contribute to purification can guide defenders in selecting or crafting the most appropriate dataset. Conversely, when only limited auxiliary data is accessible, knowing which purification technique works best under those constraints is critical. Therefore, there is an urgent need for a thorough investigation into the impact of auxiliary datasets on purification effectiveness to guide defenders in  enhancing the security of ML systems. 

In this paper, we systematically investigate the critical role of auxiliary datasets in backdoor purification, aiming to bridge the gap between idealized and practical purification scenarios.  Specifically, we first construct a diverse set of auxiliary datasets to emulate real-world conditions, as summarized in Table~\ref{overall}. These datasets include in-distribution data, synthetic data, and external data from other sources. Through an evaluation of SOTA backdoor purification methods across these datasets, we uncover several critical insights: \textbf{1)} In-distribution datasets, particularly those carefully filtered from the original training data of the victim model, effectively preserve the model’s utility for its intended tasks but may fall short in eliminating backdoors. \textbf{2)} Incorporating OOD datasets can help the model forget backdoors but also bring the risk of forgetting critical learned knowledge, significantly degrading its overall performance. Building on these findings, we propose Guided Input Calibration (GIC), a novel technique that enhances backdoor purification by adaptively transforming auxiliary data to better align with the victim model’s learned representations. By leveraging the victim model itself to guide this transformation, GIC optimizes the purification process, striking a balance between preserving model utility and mitigating backdoor threats. Extensive experiments demonstrate that GIC significantly improves the effectiveness of backdoor purification across diverse auxiliary datasets, providing a practical and robust defense solution.

Our main contributions are threefold:
\textbf{1) Impact analysis of auxiliary datasets:} We take the \textbf{first step}  in systematically investigating how different types of auxiliary datasets influence backdoor purification effectiveness. Our findings provide novel insights and serve as a foundation for future research on optimizing dataset selection and construction for enhanced backdoor defense.
%
\textbf{2) Compilation and evaluation of diverse auxiliary datasets:}  We have compiled and rigorously evaluated a diverse set of auxiliary datasets using SOTA purification methods, making our datasets and code publicly available to facilitate and support future research on practical backdoor defense strategies.
%
\textbf{3) Introduction of GIC:} We introduce GIC, the \textbf{first} dedicated solution designed to align auxiliary datasets with the model’s learned representations, significantly enhancing backdoor mitigation across various dataset types. Our approach sets a new benchmark for practical and effective backdoor defense.



\section{Related Work}

\subsection{Large 3D Reconstruction Models}
Recently, generalized feed-forward models for 3D reconstruction from sparse input views have garnered considerable attention due to their applicability in heavily under-constrained scenarios. The Large Reconstruction Model (LRM)~\cite{hong2023lrm} uses a transformer-based encoder-decoder pipeline to infer a NeRF reconstruction from just a single image. Newer iterations have shifted the focus towards generating 3D Gaussian representations from four input images~\cite{tang2025lgm, xu2024grm, zhang2025gslrm, charatan2024pixelsplat, chen2025mvsplat, liu2025mvsgaussian}, showing remarkable novel view synthesis results. The paradigm of transformer-based sparse 3D reconstruction has also successfully been applied to lifting monocular videos to 4D~\cite{ren2024l4gm}. \\
Yet, none of the existing works in the domain have studied the use-case of inferring \textit{animatable} 3D representations from sparse input images, which is the focus of our work. To this end, we build on top of the Large Gaussian Reconstruction Model (GRM)~\cite{xu2024grm}.

\subsection{3D-aware Portrait Animation}
A different line of work focuses on animating portraits in a 3D-aware manner.
MegaPortraits~\cite{drobyshev2022megaportraits} builds a 3D Volume given a source and driving image, and renders the animated source actor via orthographic projection with subsequent 2D neural rendering.
3D morphable models (3DMMs)~\cite{blanz19993dmm} are extensively used to obtain more interpretable control over the portrait animation. For example, StyleRig~\cite{tewari2020stylerig} demonstrates how a 3DMM can be used to control the data generated from a pre-trained StyleGAN~\cite{karras2019stylegan} network. ROME~\cite{khakhulin2022rome} predicts vertex offsets and texture of a FLAME~\cite{li2017flame} mesh from the input image.
A TriPlane representation is inferred and animated via FLAME~\cite{li2017flame} in multiple methods like Portrait4D~\cite{deng2024portrait4d}, Portrait4D-v2~\cite{deng2024portrait4dv2}, and GPAvatar~\cite{chu2024gpavatar}.
Others, such as VOODOO 3D~\cite{tran2024voodoo3d} and VOODOO XP~\cite{tran2024voodooxp}, learn their own expression encoder to drive the source person in a more detailed manner. \\
All of the aforementioned methods require nothing more than a single image of a person to animate it. This allows them to train on large monocular video datasets to infer a very generic motion prior that even translates to paintings or cartoon characters. However, due to their task formulation, these methods mostly focus on image synthesis from a frontal camera, often trading 3D consistency for better image quality by using 2D screen-space neural renderers. In contrast, our work aims to produce a truthful and complete 3D avatar representation from the input images that can be viewed from any angle.  

\subsection{Photo-realistic 3D Face Models}
The increasing availability of large-scale multi-view face datasets~\cite{kirschstein2023nersemble, ava256, pan2024renderme360, yang2020facescape} has enabled building photo-realistic 3D face models that learn a detailed prior over both geometry and appearance of human faces. HeadNeRF~\cite{hong2022headnerf} conditions a Neural Radiance Field (NeRF)~\cite{mildenhall2021nerf} on identity, expression, albedo, and illumination codes. VRMM~\cite{yang2024vrmm} builds a high-quality and relightable 3D face model using volumetric primitives~\cite{lombardi2021mvp}. One2Avatar~\cite{yu2024one2avatar} extends a 3DMM by anchoring a radiance field to its surface. More recently, GPHM~\cite{xu2025gphm} and HeadGAP~\cite{zheng2024headgap} have adopted 3D Gaussians to build a photo-realistic 3D face model. \\
Photo-realistic 3D face models learn a powerful prior over human facial appearance and geometry, which can be fitted to a single or multiple images of a person, effectively inferring a 3D head avatar. However, the fitting procedure itself is non-trivial and often requires expensive test-time optimization, impeding casual use-cases on consumer-grade devices. While this limitation may be circumvented by learning a generalized encoder that maps images into the 3D face model's latent space, another fundamental limitation remains. Even with more multi-view face datasets being published, the number of available training subjects rarely exceeds the thousands, making it hard to truly learn the full distibution of human facial appearance. Instead, our approach avoids generalizing over the identity axis by conditioning on some images of a person, and only generalizes over the expression axis for which plenty of data is available. 

A similar motivation has inspired recent work on codec avatars where a generalized network infers an animatable 3D representation given a registered mesh of a person~\cite{cao2022authentic, li2024uravatar}.
The resulting avatars exhibit excellent quality at the cost of several minutes of video capture per subject and expensive test-time optimization.
For example, URAvatar~\cite{li2024uravatar} finetunes their network on the given video recording for 3 hours on 8 A100 GPUs, making inference on consumer-grade devices impossible. In contrast, our approach directly regresses the final 3D head avatar from just four input images without the need for expensive test-time fine-tuning.


\section{Methodology}

\subsection{Problem Definition}

Given a multivariate time series input $X \in \mathbb{R}^{C  \times T}$, multivariate time series forecasting tasks are designed to predict its future $F$ time steps $\hat{Y}\in \mathbb{R}^{C \times F}$ using past $T$ steps. $C $ is the number of variates or channels.

\subsection{Preliminary Analysis}

This section presents why RevIN~\citep{Kim_revin,liu2022non}, High-pass, and Low-pass filters fail to address the Mid-Frequency Spectrum Gap. Let the input univariate time series be $ x(t) $ with length $ T $ and target $ y(t) $ with length $ F $. 

\begin{definition}[Frequency Spectral Energy]\label{def:energy}
The Fourier transform of $x(t)$, $X(f)$, and its spectral energy $E_X(f)$ is given by:
\vspace{-0.2cm}
\begin{align}
X(f) = \sum_{t=0}^{T-1} x(t) e^{-i 2 \pi f t / {T-1}}, \quad &f = 0, 1, \dots, T-1\notag\\
E_X(f) = |X(f)|^2.
\end{align}
\vspace{-0.2cm}
\end{definition}

\textbf{Impact of RevIN on Frequency Spectrum \quad}
\begin{definition}[Reversible Instance Normalization]\label{def:RevIN}
Given a \textbf{forecast model} $ f: \mathbb{R}^T \rightarrow \mathbb{R}^F $ that generates a forecast $ \hat{y}(t) $ from a given input $x(t)$, RevIN is defined as:
\vspace{-0.2cm}
\begin{align}
&\hat{x}(t) = \frac{x(t) - \mu}{\sigma},\quad t = 0, 1, \dots, T-1\notag\\
&\hat{y}(t) = f(\hat{x}(t)), \quad \hat{y}(t)_{rev}= \hat{y}(t) \cdot \sigma + \mu,\notag\\
&\mu = \frac{1}{T} \sum_{t=0}^{T-1} x(t), \quad \sigma = \sqrt{\frac{1}{T} \sum_{t=0}^{T-1} (x(t) - \mu)^2}.
\end{align}
\vspace{-0.2cm}
\end{definition}

\begin{theorem} [Frequency Spectrum after RevIN] \label{theorem:RevIN}
\vspace{-0.2cm}
The spectral energy of $\hat{x}(t)$ (transformed using RevIN):
\begin{align}
E_{\hat{X}}(0)=0,& \quad f=0, \notag\\
E_{\hat{X}}(f) = \left( \frac{1}{\sigma} \right)^2 |X(f)|^2,&\quad f = 1,2,\dots, T-1 . 
\end{align}
\vspace{-0.2cm}
\end{theorem}
The proof is in Appendix~\ref{app:RevIN}. Theorem~\ref{theorem:RevIN} suggests that RevIN scales the absolute spectral energy by $ \sigma^2 $ but does not affect its relative distribution except $E_{\hat{X}}(0)=0$. Thus, RevIN preserves the relative spectral energy distribution and leaves the Mid-Frequency Spectrum Gap unresolved. \textit{However, our experiments still employ RevIN to ensure a fair comparison with other baselines.}
\begin{figure*}[h]
  \centering
  \includegraphics[width=1.\linewidth]{Faker/source/assets/jpg/ReFocus.jpg}
  \caption{General structure of \textbf{ReFocus}. `Adaptive Mid-Frequency Energy Optimizer (AMEO)' enhances mid-frequency components modeling, and `Energy-based Key-Frequency Picking Block' (EKPB) effectively captures shared Key-Frequency across channels}
  \label{fig:refocus}
\end{figure*}

\begin{figure*}[h]
  \centering
  \includegraphics[width=0.7\linewidth]{Faker/source/assets/jpg/ket.jpg}
  \caption{General process of the \textbf{Key-Frequency Enhanced Training strategy (KET)}, where spectral information from other channels is randomly introduced into each channel, to enhance the extraction of the shared Key-Frequency.}
  \label{fig:reshuffle}
\end{figure*}
\textbf{Impact of High- and Low-pass filter \quad}
We still define $\hat{x}(t)$ to be the filtered (processed) signal, obtained by applying a filter $H(f)$ (High/Low-pass filter). The filter $ H(f) $ is 1 in the passband (High/Low frequency) and 0 in the stopband (Middle frequency). So $E_{\hat{X}}(f)=0,\quad E_{\hat{X}}\leq E_X(f)$ for middle frequencies, which creates even larger gap.

\subsection{Overall Structure of The Proposed ReFocus}

In this section, we elucidate the overall architecture of \textbf{ReFocus}, depicted in Figure \ref{fig:refocus}. We define frequency domain projection as $D1\rightarrow D2$ representing a projection from dimension $D1$ to $D2$ in the frequency domain~\citep{xu2024fits}. Initially, we apply \textbf{AMEO} to the input $X \in \mathbb{R}^{C \times T}$, yielding the processed spectrum $ X_{am} \in \mathbb{R}^{C  \times T} $. Next, we use a projection $T\rightarrow D$ to transform $ X_{am}$ into the Variate Embedding $ X_{em} \in \mathbb{R}^{C  \times D}$~\citep{LiuiTransformer}. Then, $X_{em}$ go through $N$ \textbf{EKPB} to generate representation $H_{N+1}$, which is projected to obtain final prediction $\hat{Y}$. 

\textbf{Adaptive Mid-Frequency Energy Optimizer \quad}
Building upon the \textbf{Preliminary Analysis}, we propose a convolution- and residual learning-based solution to address the Mid-Frequency Spectrum Gap, which we denoted as AMEO. 
\begin{definition}[Adaptive Mid-Frequency Energy Optimizer]\label{def:AMEO}
AMEO is defined as:
\begin{align}
&\hat{x}(t) = x(t)-\frac{\beta}{K}\sum_{k=0}^{K-1} \tilde{x}(t+K-1-k),\notag\\
&\tilde{x}(t) =\notag\\
&\begin{cases}
x(t-(\frac{K}{2}+1)), \quad \text{if } \frac{K}{2}+1 \leq t < T+\frac{K}{2}+1, \\
0,  \quad\text{if } 0 \leq t < \frac{K}{2}+1 \text{ or } T+\frac{K}{2}+1 \leq t < T+K.
\end{cases}
\end{align}
\vspace{-0.2cm}
\end{definition}

It is equivalent to $x=x-\beta \cdot Conv(x)$. $Conv$ is a 1D convolution (Zero-padding at both ends, stride $s=1$, kernel size $K$, with values initialized as $ \frac{1}{K} $). $\beta \in \mathbb{R}^{1}$ is a hyperparameter.

\begin{theorem} [Frequency Spectrum after AMEO] \label{theorem:AMEO}
The spectral energy of $\hat{x}(t)$ obtained using AMEO:
\begin{align}
E_{\hat{X}}(f) =|X(f)|^2 \left\{1 - \beta \cdot \underbrace{\frac{1}{K} \sum_{k=0}^{K-1} e^{i 2 \pi f (\frac{3K}{2}-k -2) / {T-1}}}_{G(f)}\right\}^2
\end{align}
\vspace{-0.2cm}
\end{theorem}

The proof is in Appendix~\ref{app:AMEO}. We have $E_{\hat{X}}(f) =|X(f)|^2(1-\beta  \cdot G(f))^2$. Generally, $ G(f) $ behaves as a decay function, gradually reducing its value from \textbf{One} to \textbf{Zero}. Such \textbf{decay behavior} makes AMEO relatively enhances mid-frequency components, thus addressing the Mid-Frequency Spectrum Gap.

\textbf{Energy-based Key-Frequency Picking Block \quad} In each \textbf{EKPB}, the input $ H_i \in \mathbb{R}^{C  \times D} (H_1=X_{em}) $ is first processed through an MLP to generate $ H_i^k \in \mathbb{R}^{C  \times Q}$. Then, FFT is applied to get $ H_i^f \in \mathbb{R}^{C  \times (Q/2+1)}$. For $ H_i^f$, we calculate its energy, denoted as $ H_i^e \in \mathbb{R}^{C  \times (Q/2+1)}$. A cross-channel softmax is then applied to $ H_i^e$ per frequency to obtain a probability distribution $ H_i^{soft} \in \mathbb{R}^{C  \times (Q/2+1)}$. Using $H_i^{soft}$, we select values from $ H_i^f$ across channels for each frequency, resulting in $K^f_i \in \mathbb{R}^{1  \times (Q/2+1)}$, which represents the Shared Key-Frequency across all channels. Then iFFT is performed on $K^f_i$ to get $K_i\in \mathbb{R}^{1  \times Q}$, followed by projection $Q\rightarrow D$ and repeating (C times) to get $\hat{K}_i \in \mathbb{R}^{C  \times D}$. This $\hat{K}_i$ is point-wisely added to $\hat{H_i}\in \mathbb{R}^{C  \times D}$ , which is the projection of $ H_i$ using projection $D\rightarrow D$. Then, an MLP and $Add\&Norm$ is applied to the result $HK\in \mathbb{R}^{C  \times D}$ to fuse inter-series dependencies information, and another MLP and $Add\&Norm$ is used to capture intra-series variations~\citep{LiuiTransformer}. The output of each \textbf{EKPB} is $\hat{O_i} \in \mathbb{R}^{C  \times D}$, where $H_{i+1}=\hat{O_i}$.

\subsection{Key-Frequency Enhanced Training strategy}

In real-world time series, certain channels often exhibit spectral dependencies, which may not be fully captured in the training set, and the specific channels with such dependencies are also unknown~\citep{geweke1984freqchannel,Zhao2024freqchannel}. So this work borrows insight from recent advancement of mix-up in time series~\citep{zhou2023mixup,ansari2024mixup}, randomly introducing spectral information from other channels into each channel, to enhance the extraction of the shared Key-Frequency, as in Figure~\ref{fig:reshuffle}. Given a multivariate time series input $X \in \mathbb{R}^{C \times T}$ and its ground-truth $Y \in \mathbb{R}^{C \times F}$, we generate a pseudo sample pair: 

\begin{align}
X' = iFFT(FFT(X) +\alpha \cdot FFT(X[\text{perm},:]))&,  \notag\\ 
Y' = iFFT(FFT(Y) +\alpha \cdot FFT(Y[\text{perm},:]))&.
\end{align}

$\alpha \in \mathbb{R}^{C \times 1}$ is a weight vector sampled from a normal distribution, $\text{perm}$ is a reshuffled channel index. Since $FFT$ and $iFFT$ are linear operations, this mix-up process can be equivalently simplified in the \textbf{Time Domain}:
\begin{align}
X' = X +\alpha \cdot X[\text{perm},:]&,  \notag\\
Y' = Y +\alpha \cdot Y[\text{perm},:]&
 \end{align}
We alternate training between real and synthetic data to preserve the spectral dependencies in real samples. This combines the advantages of data augmentation, such as improved generalization, while mitigating potential drawbacks like over-smoothing and training instability~\citep{ryu2024tf,alkhalifah2022tf}.















\section{Experiments and Results}
The primary purpose of our experiments was to showcase the benefit of our task augmentation approach (sec. \ref{sec:4.1}). To assess the value of each task for VLM benchmarking, we related it to average model performance, resources needed to create the task, and corresponding human ambiguity (sec. \ref{sec:4.2}). Finally, we leveraged our concept and data to explore the capabilities of the most recent open and closed VLMs (sec. \ref{sec:4.3}.).


\subsection{Benefit of the proposed framework
\label{sec:4.1}
}
\Cref{fig:metadataaug} shows aggregated performance values for all models, separated by imaging domain. As the tasks and prompts were homogenized, the results clearly indicate that performance varies substantially across domains, supporting the hypothesis that in-domain validation is crucial for real-world translation. Note that this holds true despite the fact that we purposely chose domains that are relatively common (presumably captured in the model training) and closely related to one another.


Furthermore, as shown in~\cref{fig:fig3}a, the performance of models varies substantially across VLM tasks, suggesting that the tasks generated by our framework are diverse. The hardest tasks on average across domains are (1) T7.2 “Jigsaw Puzzle Completion”, (2), T1.2 “Object Counting”, (3), T7.1 “Rotated Jigsaw Puzzle Completion”, (4), T2.1 “Object Occlusion Detection”, and (5) T5.2 “Second Brightest Image Selection”. The easiest task on average was T1.3 “Additional Object Presence Detection” (see~\cref{fig:fig6_instance}).

% \FloatBarrier
\begin{figure*}[ht]
    \centering
    
    %===========================
    % First subfigure
    %===========================
    \begin{subfigure}{0.95\linewidth}
        \centering
        \includegraphics[width=0.95\linewidth]{figures/main_text_figs/cvpr_fig_3.pdf}
        \caption{\textbf{The need for specific in-domain evaluation is demonstrated by the high performance variability across imaging domains.} 
        The performance of the overall best model Gemini 1.5 pro varies between domains from 22\% (Kitchen dataset) to 72\% (Kitti dataset). 
        For the displayed Accuracy\%(75), humans achieve an almost perfect score of 1 for all datasets (see Appendix). The top 10 models per dataset shown. 
        We display the full plots for all thresholds and models in \Cref{appendix_various_fixed_percentage_lines}.}
        \label{fig:acc_sub1}
    \end{subfigure}
    
    % \vspace{1.5em} % vertical space between sub-figures
    
    %===========================
    % Second subfigure
    %===========================
    \begin{subfigure}{0.95\linewidth}
        \centering
        \includegraphics[width=0.95\linewidth]{figures/main_text_figs/cvpr_fig_7a.pdf}
        \caption{The Area under the Accuracy\%($t$) Curve serves as a \textbf{metric for comprehensive image understanding}. 
        With a maximum possible value of 1, higher values indicate better performance. 
        Notably, the current state-of-the-art model, Gemini\_1.5\_pro, achieves only 0.53, highlighting significant room for improvement. 
        Only the top 10 models are shown. 
        Separate curves for all 22 tested models and datasets are displayed in~\cref{appendix_full_auc_curves}.}
        \label{fig:full_ACC_perc_curves}
    \end{subfigure}
    
    %===========================
    % Global caption
    %===========================
    \caption{\textbf{Performance varies across domains, highlighting the need for specialized in-domain evaluation; even the best models still lag behind human performance}. 
    The Accuracy\%($t$) metric represents the percentage of images for which at least a specified proportion of questions are correctly answered. 
    It can (a) be computed for specific thresholds or (b) be aggregated over multiple thresholds to remove dependence on a specific $t$. 
    The Area under the Accuracy\%($t$) Curve captures model performance in a single value, ranging from 0.37 to 0.53 for the top 10 models tested.}
    \label{fig:acc_per_global_fig}
\end{figure*}
% \FloatBarrier
\subsection{Human Ambiguity}
\label{sec:4.2}
As demonstrated in Appendix \ref{appendix_task_ranking_models_humans}, there is a high discrepancy in task rankings between humans and models. While the "Jigsaw Puzzle Completion” tasks ranked amongst the most challenging for the models, humans found "Object Occlusion Detection" and “Object Touching Detection” to be the most difficult.
 
From a resource perspective, tasks should be (1) hard to solve for models and (2) require as little human annotation as possible. This potential trade-off is captured in Appendix \autoref{appendix_fig6_instance}. It can be seen that many hard tasks, including the top four, can already be extracted from instance segmentations alone.

\subsection{Insights on current models}
\label{sec:4.3}
\Cref{fig:acc_per_global_fig} summarizes the performance of a model selection and reference baselines. Further detailed analysis, including all tested models, examples, and errors for each generated dataset are provided in the Appendix. The following insights can be extracted:

\textbf{Confirming common findings from the community:}
Our analysis reinforces several established patterns in the field. Closed models continue to demonstrate superior performance across tasks and domains, although open models have significantly narrowed this performance gap. In particular, Qwen2 72B stands out as the strongest performer among open models. The superiority of human evaluation remains evident, with human raters achieving near-perfect performance on most tasks, though they notably struggle with specific challenges such as counting, occlusion, and direction-related tasks—counting being particularly problematic. Regarding model scaling, larger variants typically show better performance, with some notable exceptions such as Molmo 7B outperforming Pixtral 12B.
% \begin{itemize} \item Closed models mostly outperform open models across tasks and domains. \item However, open models have narrowed the gap significantly. \item Among the ones tested, Qwen2 72B is by far the best performing open model. \item Human raters outperform VLMs by a large margin. While human raters achieve near-perfect performance on most tasks, they struggle with counting, occlusion, and direction-related tasks, with counting being the most difficult. \item Large models typically outperform their smaller variants, although exceptions exist (e.g., Molmo 7B generally outperforms Pixtral 12B). \end{itemize}



\textbf{Interesting new findings:}
The need for specific in-domain evaluation is highlighted by the high performance variability across imaging domains for the same perception tasks, see \Cref{tab:model_accuracies_top7} and~\Cref{fig:metadataaug}. The overall best model, Gemini 1.5 Pro, varies between domains from 22\% (Kitchen dataset) to 72\% (Kitti dataset). Qwen2 72B slightly surpasses Gemini 1.5 Pro on the kitchen and animals datasets but ranks only fifth on the person dataset. Additional insights emerge from model comparisons, with Qwen2 7B consistently outperforming Molmo 7B across most datasets, and Gemini Flash 1.5 showing superior Point Depth Comparison capabilities over Gemini Pro. These results indicate that our newly introduced metric, Accuracy\%($t$), can effectively capture model performance in a single value.


% \begin{figure*}[t]
%   \centering
%   % First subfigure
%   \begin{subfigure}{\linewidth}
%     \centering
%     \includegraphics[width=\linewidth]{figures/main_text_figs/cvpr_fig_3.pdf}
%     \caption{\textbf{The need for specific in-domain evaluation is demonstrated by the high performance variability across imaging domains.} The performance of the overall best model Gemini 1.5 pro varies between domains from 22\% (Kitchen dataset) to 72\% (Kitti dataset). For the displayed Accuracy\%(75), humans achieve an almost perfect score of 1 for all datasets (see Appendix). Top 10 models per dataset shown. We display the full plots for all thresholds and models in \Cref{appendix_various_fixed_percentage_lines}.}
%   \end{subfigure}
%   \hfill
%   % Second subfigure
%   \begin{subfigure}{\linewidth}
%     \centering
%     \includegraphics[width=\linewidth]{figures/main_text_figs/cvpr_fig_7a.pdf}
%     \caption{The Area under the Accuracy\%($t$) Curve serves as a \textbf{metric for comprehensive image understanding}. With a maximum possible value of 1, higher values indicate better performance. Notably, the current state-of-the-art model, Gemini\_1.5\_pro, achieves only 0.53, highlighting significant room for improvement. Only the top 10 models are shown. Separate curves for all 22 tested models and datasets are displayed in Appendix TODOREF.}
%     \label{fig:full_ACC_perc_curves}
%   \end{subfigure}
%   % Global caption
%   \caption{\textbf{Performance varies across domains, highlighting the need for specialized in-domain evaluation; even the best models still lag behind human performance. y}. The Accuracy\%($t$) metric represents the percentage of images for which at least a specified proportion of questions are correctly answered. It can (a) be computed for specific thresholds or (b) be aggregated over multiple thresholds to remove the dependency on a specific $t$.The Area under the Accuracy\%($t$) Curve captures model performance in a single value and ranges from 0.37 to 0.53 for the top 10 models tested.}
%   \label{fig:acc_per_global_fig}
% \end{figure*}



\section{Discussion and Conclusion}

% \begin{quote}
% \textit{"We believe it is unethical for social workers not to learn... about technology-mediated social work."} (\citeauthor{singer_ai_2023}, 2023)
% \end{quote}

In this study, we uncovered multiple ways in which GenAI can be used in social service practice. While some concerns did arise, practitioners by and large seemed optimistic about the possibilities of such tools, and that these issues could be overcome. We note that while most participants found the tool useful, it was far from perfect in its outputs. This is not surprising, since it was powered by a generic LLM rather than one fine-tuned for social service case management. However, despite these inadequacies, our participants still found many uses for most of the tool's outputs. Many flaws pointed out by our participants related to highly contextualised, local knowledge. To tune an AI system for this would require large amounts of case files as training data; given the privacy concerns associated with using client data, this seems unlikely to happen in the near future. What our study shows, however, is that GenAI systems need not aim to be perfect to be useful to social service practitioners, and can instead serve as a complement to the critical "human touch" in social service.

We draw both inspiration and comparisons with prior work on AI in other settings. Studies on creative writing tools showed how the "uncertainty" \cite{wan2024felt} and "randomness" \cite{clark2018creative} of AI outputs aid creativity. Given the promise that our tool shows in aiding brainstorming and discussion, future social service studies could consider AI tools explicitly geared towards creativity - for instance, providing side-by-side displays of how a given case would fit into different theoretical frameworks, prompting users to compare, contrast, and adopt the best of each framework; or allowing users to play around with combining different intervention modalities to generate eclectic (i.e. multi-modal) interventions.

At the same time, the concept of supervision creates a different interaction paradigm to other uses of AI in brainstorming. Past work (e.g. \cite{shaer2024ai}) has explored the use of GenAI for ideation during brainstorming sessions, wherein all users present discuss the ideas generated by the system. With supervision in social service practice, however, there is a marked information and role asymmetry: supervisors may not have had the time to fully read up on their supervisee's case beforehand, yet have to provide guidance and help to the latter. We suggest that GenAI can serve a dual purpose of bringing supervisors up to speed quickly by summarising their supervisee's case data, while simultaneously generating a list of discussion and talking points that can improve the quality of supervision. Generalising, this interaction paradigm has promise in many other areas: senior doctors reviewing medical procedures with newer ones \cite{snowdon2017does} could use GenAI to generate questions about critical parts of a procedure to ask the latter, confirming they have been correctly understood or executed; game studio directors could quickly summarise key developmental pipeline concerns to raise at meetings and ensure the team is on track; even in academia, advisors involved in rather too many projects to keep track of could quickly summarise each graduate student's projects and identify potential concerns to address at their next meeting.

In closing, we are optimistic about the potential for GenAI to significantly enhance social service practice and the quality of care to clients. Future studies could focus on 1) longitudinal investigations into the long-term impact of GenAI on practitioner skills, client outcomes, and organisational workflows, and 2) optimising workflows to best integrate GenAI into casework and supervision, understanding where best to harness the speed and creativity of such systems in harmony with the experience and skills of practitioners at all levels.

% GenAI here thus serves as a tool that supervisors can use before rather then using the session, taking just a few minutes of their time to generate a list of discussion points with their supervisees.

% Traditional brainstorming comes up with new things that users discuss. In supervision, supervisors can use AI to more efficiently generate talking points with their supervisees. These are generally not novel ideas, since an experienced worker would be able to come up with these on their own. However, the interesting and novel use of AI here is in its use as a preparation tool, efficiently generating talking and discussion points, saving supervisors' time in preparing for a session, while still serving as a brainstorming tool during the session itself.

% The idea of embracing imperfect AI echoes the findings of \citeauthor{bossen2023batman} (2023) in a clinical decision setting, which examined the successful implementation of an "error-prone but useful AI tool". This study frames human-AI collaboration as "Batman and Robin", where AI is a useful but ultimately less skilled sidekick that plays second fiddle to Batman. This is similar to \citeauthor{yang2019unremarkable}'s (2019) idea of "unremarkable AI", systems designed to be unobtrusive and only visible to the user when they add some value. As compared to \citeauthor{bossen2023batman}, however, we see fewer instances of our AI system producing errors, and more examples of it providing learning and collaborative opportunities and other new use cases. We build on the idea of "complementary performance" \cite{bansal2021does}, which discusses how the unique expertise of AI enhances human decision-making performance beyond what humans can achieve alone. Beyond decision-making, GenAI can now enable "complementary work patterns", where the nature of its outputs enables humans to carry out their work in entirely new ways. Our study suggests that rather being a sidekick - Robin - AI is growing into the role of a "second Batman" or "AI-Batman": an entity with distinct abilities and expertise from humans, and that contributes in its own unique way. There is certainly still a time and place for unremarkable AI, but exploring uses beyond that paradigm uncovers entirely new areas of system design.

% % \cite{gero2022sparks} found AI to be useful for science writers to translate ideas already in their head into words, and to provide new perspectives to spark further inspiration. \textit{But how is ours different from theirs?}

% \subsection{New Avenues of Human-AI Collaboration}
% \label{subsubsec:discussionhaicollaboration}

% Past HCI literature in other areas \cite{nah2023generative} has suggested that GenAI represents a "leap" \cite{singh2023hide} in human-AI collaboration, 
% % Even when an AI system sometimes produces irrelevant outputs, it can still provide users 
% % Such systems have been proposed as ways to 
% helping users discover new viewpoints \cite{singh2023hide}, scour existing literature to suggest new hypotheses 
% \cite{cascella2023evaluating} and answer questions \cite{biswas2023role}, stimulate their cognitive processes \cite{memmert2023towards}, and overcome "writer's block" \cite{singh2023hide, cooper2023examining} (particularly relevant to SSPs and the vast amount of writing required of them). Our study finds promise for AI to help SSPs in all of these areas. By nature of being more verbose and capable of generating large amounts of content, GenAI seems to create a new way in which AI can complement human work and expertise. Our system, as LLMs tend to do, produced a lot of "bullshit" (S6) \cite{frankfurt2005bullshit} - superficially true statements that were often only "tangentially related" and "devoid of meaning" \cite{halloran2023ai}. Yet, many participants cited the page-long analyses and detailed multi-step intervention plans generated by the AI system to be a good starting point for further discussion, both to better conceptualize a particular case and to facilitate general worker growth and development. Almost like throwing mud at a wall to see what sticks, GenAI can quickly produce a long list of ideas or information, before the worker glances through it and quickly identifies the more interesting points to discuss. Playing the proposed role as a "scaffold" for further work \cite{cooper2023examining}, GenAI, literally, generates new opportunities for novel and more effective processes and perspectives that previous systems (e.g., PRMs) could not. This represents an entirely new mode of human-AI collaboration.
% % This represents a new mode of collaboration not possible with the largely quantitative AI models (like PRMs) of the past.

% Our work therefore supports and extends prior research that have postulated the the potential of AI's shifting roles from decision-maker to human-supporter \cite{wang_human-human_2020}. \citeauthor{siemon2022elaborating} (2022) suggests the role of AI as a "creator" or "coordinator", rather than merely providing "process guidance" \cite{memmert2023towards} that does not contribute to brainstorming. Similarly, \citeauthor{memmert2023towards} (2023) propose GenAI as a step forward from providing meta-level process guidance (i.e. facilitating user tasks) to actively contributing content and aiding brainstorming. We suggest that beyond content-support, AI can even create new work processes that were not possible without GenAI. In this sense, AI has come full circle, becoming a "meta-facilitator".

% % --- WIP BELOW ---

% % Our work echoes and extends previous research on HAI collaboration in tasks requiring a human touch. \cite{gero2023social} found AI to be a safe space for creative writers to bounce ideas off of and document their inner thoughts. \cite{dhillon2024shaping} reference the idea of appropriate scaffolding in argumentative writing, where the user is providing with guidance appropriate for their competency level, and also warns of decreased satisfaction and ownership from AI use. 

% Separately, we draw parallels with the field of creative writing, where HAI collaboration has been extensively researched. Writers note the "irreducibly human" aspect of creativity in writing \cite{gero2023social}, similar to the "human touch" core to social service practice (D1); both groups therefore expressed few concerns about AI taking over core aspects of their jobs. Another interesting parallel was how writers often appreciated the "uncertainty" \cite{wan2024felt} and "randomness" \cite{clark2018creative} of AI systems, which served as a source of inspiration. This echoes the idea of "imperfect AI" "expanding [the] perspective[s]" (S4) of our participants when they simply skimmed through what the AI produced. \cite{wan2024felt} cited how the "duality of uncertainty in the creativity process advances the exploration of the imperfection of GenAI models". While social service work is not typically regarded as "creative", practitioners nonetheless go through processes of ideation and iteration while formulating a case. Our study showed hints of how AI can help with various forms of ideation, but, drawing inspiration from creative writing tools, future studies could consider designs more explicitly geared towards creativity - for instance, by attempting to fit a given case into a number of different theories or modalities, and displaying them together for the user to consider. While many of these assessments may be imperfect or even downnright nonsensical, they may contain valuable ideas and new angles on viewing the case that the practitioner can integrate into their own assessment.

% % \cite{foong2024designing}, describing the design of caregiver-facing values elicitation tools, cites the "twin scenarios" that caregivers face - private use, where they might use a tool to discover their patient's values, and collaborative use, where they discuss the resulting values with other parties close to the patient. This closely mirrors how SSPs in our study reference both individual and collaborative uses of our tool. Unlike in \cite{foong2024designing}, however, we do not see a resulting need to design a "staged approach" with distinct interface features for both stages.

% % --- END OF WIP ---

% Having mentioned algorithm aversion previously, we also make a quick point here on the other end of the spectrum - automation bias, or blind trust in an automated system \cite{brown2019toward}. LLMs risk being perceived as an "ultimate epistemic authority" \cite{cooper2023examining} due to their detailed, life-like outputs. While automation bias has been studied in many contexts, including in the social sector or adjacent areas, we suggest that the very nature of GenAI systems fundamentally inhibits automation bias. The tendency of GenAI to produce verbose, lengthy explanations prompts users to read and think through the machine's judgement before accepting it, bringing up opportunities to disagree with the machine's opinion. This guards against blind acceptance of the system's recommendations, particularly in the culture of a social work agency where constant dialogue - including discussing AI-produced work - is the norm.


% % : Perception of AI in Social Service Work ??

% \subsection{Redefining the Boundary}
% \label{subsubsec:discussiontheoretical}

% As \citeauthor{meilvang_working_2023} (2023) describes, the social service profession has sought to distance itself from comprising mostly "administrative work" \cite{abbott2016boundaries}, and workers have long tried to tried to reduce their considerable time \cite{socialraadgiverforening2010notat} spent on such tasks in favour of actual casework with clients \cite{toren1972social}. Our study, however, suggests a blurring of the line between "manual" administrative tasks and "mental" casework that draws on practitioner expertise. Many tasks our participants cited involve elements of both: for instance, documenting a case recording requires selecting only the relevant information to include, and planning an intervention can be an iterative process of drafting a plan and discussing it with colleagues and superiors. This all stems from the fact that GenAI can produce virtually any document required by the user, but this document almost always requires revision under a watchful human eye.

% \citeauthor{meilvang_working_2023} (2023) also describes a more recent shift in the perceived accepted boundary of AI interventions in social service work. From "defending [the] jurisdiction [of social service work] against artificial intelligence" in the early days of PRM and other statistical assessment tools, the community has started to embrace AI as a "decision-support ... element in the assessment process". Our study concurs and frames GenAI as a source of information that can be used to support and qualify the assessments of SSPs \cite{meilvang_working_2023}, but suggests that we can take a step further: AI can be viewed as a \textit{facilitator} rather than just a supporter. GenAI can facilitate a wide range of discussions that promote efficiency, encourage worker learning and growth, and ultimately enhance client outcomes. This entails a much larger scope of AI use, where practitioners use the information provided by AI in a range of new scenarios. 

% Taken together, these suggest a new focus for boundary work and, more broadly, HCI research. GAI can play a role not just in menial documentation or decision-support, but can be deeply ingrained into every facet of the social service workflow to open new opportunities for worker growth, workflow optimisation, and ultimately improved client outcomes. Future research can therefore investigate the deeper, organisational-level effects of these new uses of AI, and their resulting impact on the role of profession discretion in effective social service work.

% % MH: oh i feel this paragraph is quite new to me! Could we elaborate this more, and truncate the first two paragraphs a bit to adjust the word propotion?


% % Our study extensively documents this for the first time in social service practice, and in the process reveals new insights about how AI can play such a role.



% \subsection{Design Implications}

% % Add link from ACE diagram?

% % EJ: it would be interesting to discuss how LLMs could help "hands-on experience" in the discussion section

% Addressing the struggle of integrating AI amidst the tension between machine assessment and expert judgement, we reframe AI as an \textit{facilitator} rather than an algorithm or decision-support tool, alleviating many concerns about trust and explainablity. We now present a high-level framework (Figure \ref{fig:hai-collaboration}) on human-AI collaboration, presenting a new perspective on designing effective AI systems that can be applied to both the social service sector and beyond.

% \begin{figure}
%     \centering
%     \includegraphics[scale=0.15]{images/designframework.png}
%     \caption{Framework for Human-AI Collaboration}
%     \label{fig:hai-collaboration}
%     \Description{An image showing our framework for Human-AI Collaboration. It shows that as stakeholder level increases from junior to senior, the directness of use shifts from co-creation to provision.}
% \end{figure}
% % MH: so this paradigm is proposed by us? I wonder if this could a part of results as well..?

% % \subsubsection{From Creation to Provision}

% In Section \ref{sec:stage2findings}, we uncovered the different ways in which SSPs of varying seniorities use, evaluate, and suggest uses of AI. These are intrinsically tied to the perspectives and levels of expertise that each stakeholder possesses. We therefore position the role of AI along the scale of \textit{creation} to \textit{provision}. 

% With junior workers, we recommend \textbf{designing tools for co-creation}: systems that aid the least experienced workers in creating the required deliverables for their work. Rather than \textit{telling} workers what to do - a difficult task in any case given the complexity of social work solutions - AI systems should instead \textit{co-create} deliverables required of these workers. These encompass the multitude of use cases that junior workers found useful: creating reports, suggesting perspectives from which to formulate a case, and providing a starting template for possible intervention plans. Notably, since AI outputs are not perfect, we emphasise the "co" in "co-creation": AI should only be a part of the workflow that also includes active engagement on the part of the SSPs and proactive discussion with supervisors. 

% For more experienced SSPs, we recommend \textbf{designing tools for provision}. Again, this is not the mere provision of recommendations or courses of action with clients, but rather that of resources which complement the needs of workers with greater responsibilities. This notably includes supplying materials to aid with supervision, a novel use case that to our knowledge has not surfaced in previous literature. In addition, senior workers also benefit greatly from manual tasks such as routine report writing and data processing. Since these workers are more experienced and can better spot inaccuracies in AI output, we suggest that AI can "provide" a more finished product that requires less vetting and corrections, and which can be used more directly as part of required deliverables.

% % MH: can we seperate here? above is about the guidance to paradigm, below is the practical roadmap for implementation
% In terms of concrete design features, given the constant focus on discussing AI outputs between colleagues in our FGDs, we recommend that AI tools, particularly those for junior workers, \textbf{include collaborative features} that facilitate feedback and idea sharing between users. We also suggest that designers work closely with domain experts (i.e. social work practitioners and agencies) to identify areas where the given AI model tends to make more mistakes, and to build in features that \textbf{highlight potential mistakes or inadequacies} in the AI's output to facilitate further discussion and avoid workers adopting suboptimal suggestions. 

% We also point out a fundamental difference between GenAI systems and previous systems: that GenAI can now play an important role in aiding users \textit{regardless of its flaws}. The nature of GenAI means that it promotes discussion and opens up new workflows by nature of its verbose and potentially incomplete outputs. Rather than working towards more accurate or explainable outcomes, which may in any case have minimal improvement on worker outcomes \cite{li2024advanced}, designers can also focus on \textbf{understanding how GenAI outputs can augment existing user flows and create new ones}.

% % for more senior workers...

% % how to differentiate levels of workers?

% % \subsubsection{Provider}

% % The most basic and obvious role of modern AI that we identify leverages the main strength of LLMs. They have the ability to produce high-quality writing from short, point-form, or otherwise messy and disjoint case notes that user often have \textit{[cite participant here]}. 

% Finally, given the limited expertise of many workers at using AI, it is important that systems \textbf{explicitly guide users to the features they need}, rather than simply relying on the ability of GAI to understand complex user instructions. For example, in the case of flexibility in use cases (Section \ref{subsubsec:control}), systems should include user flows that help combine multiple intervention and assessment modalities in order to directly meet the needs of workers.

% \subsection{Limitations and Future Work}

% While we attempt to mimic a contextual inquiry and work environment in our study design, there is no substitute for real data from actual system deployment. The use of an AI system in day-to-day work could reveal a different set of insights. Future studies could in particular study how the longitudinal context of how user attitudes, behaviours, preferences, and work outputs change with extended use of AI. 

% While we tried to include practitioners from different agencies, roles, and seniorities, social service practice may differ culturally or procedurally in other agencies or countries. Future studies could investigate different kinds of social service agencies and in different cultures to see if AI is similarly useful there.

% As the study was conducted in a country with relatively high technology literacy, participants naturally had a higher baseline understanding and acceptance of AI and other computer systems. However, we emphasise that our findings are not contingent on this - rather, we suggest that our proposed lens of viewing AI in the social sector is a means for engaging in relevant stakeholders and ensuring the effective design and implementation of AI in the social sector, regardless of how participants feel about AI to begin with. 



% % \subsection{Notes}

% % 1) safety and risks and 2) privacy - what does the emphasis on this say about a) design recommendations and b) approach to designing/PD of such systems?


% % W9 was presented with "Strengths" and "SFBT" output options. They commented, "solution focus is always building on the person's strengths". W9 therefore requested being able to output strengths and SFBT at the same time. But this would suggest that the SFBT output does not currently emphasise strengths strongly enough. However, W9 did not specifically evaluate that, and only made this comment because they saw the "strengths" option available, and in their head, strengths are key to SFBT.
% % What does this say about system design and UI in relation to user mental models?

\section*{Impact Statement}
This paper advances Machine Learning by enabling researchers to benchmark with their own data on a minimal budget. All human annotations were sourced from a reputable company following ethical guidelines.



\bibliography{example_paper}
\bibliographystyle{icml2025}


%%%%%%%%%%%%%%%%%%%%%%%%%%%%%%%%%%%%%%%%%%%%%%%%%%%%%%%%%%%%%%%%%%%%%%%%%%%%%%%
%%%%%%%%%%%%%%%%%%%%%%%%%%%%%%%%%%%%%%%%%%%%%%%%%%%%%%%%%%%%%%%%%%%%%%%%%%%%%%%
% APPENDIX
%%%%%%%%%%%%%%%%%%%%%%%%%%%%%%%%%%%%%%%%%%%%%%%%%%%%%%%%%%%%%%%%%%%%%%%%%%%%%%%
%%%%%%%%%%%%%%%%%%%%%%%%%%%%%%%%%%%%%%%%%%%%%%%%%%%%%%%%%%%%%%%%%%%%%%%%%%%%%%%
\newpage


\appendix
\onecolumn


\clearpage
\pagenumbering{gobble}
\maketitlesupplementary

\section{Additional Results on Embodied Tasks}

To evaluate the broader applicability of our EgoAgent's learned representation beyond video-conditioned 3D human motion prediction, we test its ability to improve visual policy learning for embodiments other than the human skeleton.
Following the methodology in~\cite{majumdar2023we}, we conduct experiments on the TriFinger benchmark~\cite{wuthrich2020trifinger}, which involves a three-finger robot performing two tasks: reach cube and move cube. 
We freeze the pretrained representations and use a 3-layer MLP as the policy network, training each task with 100 demonstrations.

\begin{table}[h]
\centering
\caption{Success rate (\%) on the TriFinger benchmark, where each model's pretrained representation is fixed, and additional linear layers are trained as the policy network.}
\label{tab:trifinger}
\resizebox{\linewidth}{!}{%
\begin{tabular}{llcc}
\toprule
Methods       & Training Dataset & Reach Cube & Move Cube \\
\midrule
DINO~\cite{caron2021emerging}         & WT Venice        & 78.03     & 47.42     \\
DoRA~\cite{venkataramanan2023imagenet}          & WT Venice        & 81.62     & 53.76     \\
DoRA~\cite{venkataramanan2023imagenet}          & WT All           & 82.40     & 48.13     \\
\midrule
EgoAgent-300M & WT+Ego-Exo4D      & 82.61    & 54.21      \\
EgoAgent-1B   & WT+Ego-Exo4D      & \textbf{85.72}      & \textbf{57.66}   \\
\bottomrule
\end{tabular}%
}
\end{table}

As shown in Table~\ref{tab:trifinger}, EgoAgent achieves the highest success rates on both tasks, outperforming the best models from DoRA~\cite{venkataramanan2023imagenet} with increases of +3.32\% and +3.9\% respectively.
This result shows that by incorporating human action prediction into the learning process, EgoAgent demonstrates the ability to learn more effective representations that benefit both image classification and embodied manipulation tasks.
This highlights the potential of leveraging human-centric motion data to bridge the gap between visual understanding and actionable policy learning.



\section{Additional Results on Egocentric Future State Prediction}

In this section, we provide additional qualitative results on the egocentric future state prediction task. Additionally, we describe our approach to finetune video diffusion model on the Ego-Exo4D dataset~\cite{grauman2024ego} and generate future video frames conditioned on initial frames as shown in Figure~\ref{fig:opensora_finetune}.

\begin{figure}[b]
    \centering
    \includegraphics[width=\linewidth]{figures/opensora_finetune.pdf}
    \caption{Comparison of OpenSora V1.1 first-frame-conditioned video generation results before and after finetuning on Ego-Exo4D. Fine-tuning enhances temporal consistency, but the predicted pixel-space future states still exhibit errors, such as inaccuracies in the basketball's trajectory.}
    \label{fig:opensora_finetune}
\end{figure}

\subsection{Visualizations and Comparisons}

More visualizations of our method, DoRA, and OpenSora in different scenes (as shown in Figure~\ref{fig:supp pred}). For OpenSora, when predicting the states of $t_k$, we use all the ground truth frames from $t_{0}$ to $t_{k-1}$ as conditions. As OpenSora takes only past observations as input and neglects human motion, it performs well only when the human has relatively small motions (see top cases in Figure~\ref{fig:supp pred}), but can not adjust to large movements of the human body or quick viewpoint changes (see bottom cases in Figure~\ref{fig:supp pred}).

\begin{figure*}
    \centering
    \includegraphics[width=\linewidth]{figures/supp_pred.pdf}
    \caption{Retrieval and generation results for egocentric future state prediction. Correct and wrong retrieval images are marked with green and red boundaries, respectively.}
    \label{fig:supp pred}
\end{figure*}

\begin{figure*}[t]
    \centering
    \includegraphics[width=0.9\linewidth]{figures/motion_prediction.pdf}
    \vspace{-0.5mm}
    \caption{Motion prediction results in scenes with minor changes in observation.}
    \vspace{-1.5mm}
    \label{fig:motion_prediction}
\end{figure*}

\subsection{Finetuning OpenSora on Ego-Exo4D}

OpenSora V1.1~\cite{opensora}, initially trained on internet videos and images, produces severely inconsistent results when directly applied to infer future videos on the Ego-Exo4D dataset, as illustrated in Figure~\ref{fig:opensora_finetune}.
To address the gap between general internet content and egocentric video data, we fine-tune the official checkpoint on the Ego-Exo4D training set for 50 epochs.
OpenSora V1.1 proposed a random mask strategy during training to enable video generation by image and video conditioning. We adopted the default masking rate, which applies: 75\% with no masking, 2.5\% with random masking of 1 frame to 1/4 of the total frames, 2.5\% with masking at either the beginning or the end for 1 frame to 1/4 of the total frames, and 5\% with random masking spanning 1 frame to 1/4 of the total frames at both the beginning and the end.

As shown in Fig.~\ref{fig:opensora_finetune}, despite being trained on a large dataset, OpenSora struggles to generalize to the Ego-Exo4D dataset, producing future video frames with minimal consistency relative to the conditioning frame. While fine-tuning improves temporal consistency, the moving trajectories of objects like the basketball and soccer ball still deviate from realistic physical laws. Compared with our feature space prediction results, this suggests that training world models in a reconstructive latent space is more challenging than training them in a feature space.


\section{Additional Results on 3D Human Motion Prediction}

We present additional qualitative results for the 3D human motion prediction task, highlighting a particularly challenging scenario where egocentric observations exhibit minimal variation. This scenario poses significant difficulties for video-conditioned motion prediction, as the model must effectively capture and interpret subtle changes. As demonstrated in Fig.~\ref{fig:motion_prediction}, EgoAgent successfully generates accurate predictions that closely align with the ground truth motion, showcasing its ability to handle fine-grained temporal dynamics and nuanced contextual cues.

\section{OpenSora for Image Classification}

In this section, we detail the process of extracting features from OpenSora V1.1~\cite{opensora} (without fine-tuning) for an image classification task. Following the approach of~\cite{xiang2023denoising}, we leverage the insight that diffusion models can be interpreted as multi-level denoising autoencoders. These models inherently learn linearly separable representations within their intermediate layers, without relying on auxiliary encoders. The quality of the extracted features depends on both the layer depth and the noise level applied during extraction.


\begin{table}[h]
\centering
\caption{$k$-NN evaluation results of OpenSora V1.1 features from different layer depths and noising scales on ImageNet-100. Top1 and Top5 accuracy (\%) are reported.}
\label{tab:opensora-knn}
\resizebox{0.95\linewidth}{!}{%
\begin{tabular}{lcccccc}
\toprule
\multirow{2}{*}{Timesteps} & \multicolumn{2}{c}{First Layer} & \multicolumn{2}{c}{Middle Layer} & \multicolumn{2}{c}{Last Layer} \\
\cmidrule(r){2-3}   \cmidrule(r){4-5}  \cmidrule(r){6-7}  & Top1           & Top5           & Top1            & Top5           & Top1           & Top5          \\
\midrule
32        &  6.10           & 18.20             & 34.04               & 59.50             & 30.40             & 55.74             \\
64        & 6.12              & 18.48              & 36.04               & 61.84              & 31.80         & 57.06         \\
128       & 5.84             & 18.14             & 38.08               & 64.16              & 33.44       & 58.42 \\
256       & 5.60             & 16.58              & 30.34               & 56.38              &28.14          & 52.32        \\
512       & 3.66              & 11.70            & 6.24              & 17.62              & 7.24              & 19.44  \\ 
\bottomrule
\end{tabular}%
}
\end{table}

As shown in Table~\ref{tab:opensora-knn}, we first evaluate $k$-NN classification performance on the ImageNet-100 dataset using three intermediate layers and five different noise scales. We find that a noise timestep of 128 yields the best results, with the middle and last layers performing significantly better than the first layer.
We then test this optimal configuration on ImageNet-1K and find that the last layer with 128 noising timesteps achieves the best classification accuracy.

\section{Data Preprocess}
For egocentric video sequences, we utilize videos from the Ego-Exo4D~\cite{grauman2024ego} and WT~\cite{venkataramanan2023imagenet} datasets.
The original resolution of Ego-Exo4D videos is 1408×1408, captured at 30 fps. We sample one frame every five frames and use the original resolution to crop local views (224×224) for computing the self-supervised representation loss. For computing the prediction and action loss, the videos are downsampled to 224×224 resolution.
WT primarily consists of 4K videos (3840×2160) recorded at 60 or 30 fps. Similar to Ego-Exo4D, we use the original resolution and downsample the frame rate to 6 fps for representation loss computation.
As Ego-Exo4D employs fisheye cameras, we undistort the images to a pinhole camera model using the official Project Aria Tools to align them with the WT videos.

For motion sequences, the Ego-Exo4D dataset provides synchronized 3D motion annotations and camera extrinsic parameters for various tasks and scenes. While some annotations are manually labeled, others are automatically generated using 3D motion estimation algorithms from multiple exocentric views. To maximize data utility and maintain high-quality annotations, manual labels are prioritized wherever available, and automated annotations are used only when manual labels are absent.
Each pose is converted into the egocentric camera's coordinate system using transformation matrices derived from the camera extrinsics. These matrices also enable the computation of trajectory vectors for each frame in a sequence. Beyond the x, y, z coordinates, a visibility dimension is appended to account for keypoints invisible to all exocentric views. Finally, a sliding window approach segments sequences into fixed-size windows to serve as input for the model. Note that we do not downsample the frame rate of 3D motions.

\section{Training Details}
\subsection{Architecture Configurations}
In Table~\ref{tab:arch}, we provide detailed architecture configurations for EgoAgent following the scaling-up strategy of InternLM~\cite{team2023internlm}. To ensure the generalization, we do not modify the internal modules in InternML, \emph{i.e.}, we adopt the RMSNorm and 1D RoPE. We show that, without specific modules designed for vision tasks, EgoAgent can perform well on vision and action tasks.

\begin{table}[ht]
  \centering
  \caption{Architecture configurations of EgoAgent.}
  \resizebox{0.8\linewidth}{!}{%
    \begin{tabular}{lcc}
    \toprule
          & EgoAgent-300M & EgoAgent-1B \\
          \midrule
    Depth & 22    & 22 \\
    Embedding dim & 1024  & 2048 \\
    Number of heads & 8     & 16 \\
    MLP ratio &    8/3   & 8/3 \\
    $\#$param.  & 284M & 1.13B \\
    \bottomrule
    \end{tabular}%
    }
  \label{tab:arch}%
\end{table}%

Table~\ref{tab:io_structure} presents the detailed configuration of the embedding and prediction modules in EgoAgent, including the image projector ($\text{Proj}_i$), representation head/state prediction head ($\text{MLP}_i$), action projector ($\text{Proj}_a$) and action prediction head ($\text{MLP}_a$).
Note that the representation head and the state prediction head share the same architecture but have distinct weights.

\begin{table}[t]
\centering
\caption{Architecture of the embedding ($\text{Proj}_i$, $\text{Proj}_a$) and prediction ($\text{MLP}_i$, $\text{MLP}_a$) modules in EgoAgent. For details on module connections and functions, please refer to Fig.~2 in the main paper.}
\label{tab:io_structure}
\resizebox{\linewidth}{!}{%
\begin{tabular}{lcl}
\toprule
       & \multicolumn{1}{c}{Norm \& Activation} & \multicolumn{1}{c}{Output Shape}  \\
\midrule
\multicolumn{3}{l}{$\text{Proj}_i$ (\textit{Image projector})} \\
\midrule
Input image  & -          & 3$\times$224$\times$224 \\
Conv 2D (16$\times$16) & -       & Embedding dim$\times$14$\times$14    \\
\midrule
\multicolumn{3}{l}{$\text{MLP}_i$ (\textit{State prediction head} \& \textit{Representation head)}} \\
\midrule
Input embedding  & -          & Embedding dim \\
Linear & GELU       & 2048          \\
Linear & GELU       & 2048          \\
Linear & -          & 256           \\
Linear & -          & 65536     \\
\midrule
\multicolumn{3}{l}{$\text{Proj}_a$ (\textit{Action projector})} \\
\midrule
Input pose sequence  & -          & 4$\times$5$\times$17 \\
Conv 2D (5$\times$17) & LN, GELU   & Embedding dim$\times$1$\times$1    \\
\midrule
\multicolumn{3}{l}{$\text{MLP}_a$ (\textit{Action prediction head})} \\
\midrule
Input embedding  & -          & Embedding dim$\times$1$\times$1 \\
Linear & -          & 4$\times$5$\times$17     \\
\bottomrule
\end{tabular}%
}
\end{table}


\subsection{Training Configurations}
In Table~\ref{tab:training hyper}, we provide the detailed training hyper-parameters for experiments in the main manuscripts.

\begin{table}[ht]
  \centering
  \caption{Hyper-parameters for training EgoAgent.}
  \resizebox{0.86\linewidth}{!}{%
    \begin{tabular}{lc}
    \toprule
    Training Configuration & EgoAgent-300M/1B \\
    \midrule
    Training recipe: &  \\
    optimizer & AdamW~\cite{loshchilov2017decoupled} \\
    optimizer momentum & $\beta_1=0.9, \beta_2=0.999$ \\
    \midrule
    Learning hyper-parameters: &  \\
    base learning rate & 6.0E-04 \\
    learning rate schedule & cosine \\
    base weight decay & 0.04 \\
    end weight decay & 0.4 \\
    batch size & 1920 \\
    training iters & 72,000 \\
    lr warmup iters & 1,800 \\
    warmup schedule & linear \\
    gradient clip & 1.0 \\
    data type & float16 \\
    norm epsilon & 1.0E-06 \\
    \midrule
    EMA hyper-parameters: &  \\
    momentum & 0.996 \\
    \bottomrule
    \end{tabular}%
    }
  \label{tab:training hyper}%
\end{table}%

\clearpage


%%%%%%%%%%%%%%%%%%%%%%%%%%%%%%%%%%%%%%%%%%%%%%%%%%%%%%%%%%%%%%%%%%%%%%%%%%%%%%%
%%%%%%%%%%%%%%%%%%%%%%%%%%%%%%%%%%%%%%%%%%%%%%%%%%%%%%%%%%%%%%%%%%%%%%%%%%%%%%%


\end{document}

