\section{Related Works}
\label{subsec:relatedWorks}

There has been limited exploration of BNs in the FL domain. Featured FL approaches, like federated averaging ____, mainly focus on continuous optimization, typically applied to learning NNs. In this context, the NOTEARS-ADMM ____ algorithm emerges, adapting advances in continuous optimization to BN structure learning, specifically addressing horizontally partitioned data.

Furthermore, recent developments\footnote{It is important to highlight that, despite the terminology referring to structure learning of Causal Networks, in the three aforementioned contributions, we can use this term interchangeably with Bayesian Networks. This is attributed to their exploration of the space of Markov equivalence classes rather than the space of DAGs, highlighting their emphasis on equivalent causal structures.} have introduced methods based on both score+search algorithms, such as GES ____, and constraint-based algorithms, such as PC ____:
\begin{itemize}
    \item In the score+search category, notable contributions include the Regret-based Federated Causal Discovery (\textsc{RFcd}) ____ and its successor \textsc{Peri} ____. Both algorithms employ a regret-based search, where each client initially discovers its best-fitting local model using a score-based algorithm. The distinction lies in the search strategy, with \textsc{RFcd} utilizing a basic beam-search and \textsc{Peri} employing the GES algorithm. Following this, the server proposes networks, and clients return regret values relative to the proposed models. Utilizing these regrets, a global model is learned by minimizing the worst-case regret from all clients in a privacy-preserving manner.

    \item On the constraint-based side, the \textsc{FedC$^2$SL} algorithm ____ stands out by introducing a federated framework for BN structure learning with a federated conditional independence $\chi^2$ test. This ensures an interaction with data that preserves privacy, enabling secure statistical evaluations of conditional independence between variables without sharing private data. By incorporating this federated framework, the \textsc{FedPC} algorithm emerges as an innovative approach to BN structure learning, extending the capabilities of the PC algorithm to the FL paradigm.
\end{itemize}

%%%%%%%%%%%%%%%%%%%%%%%%%%%%%%%%%%%%%%%%%%%%%%%%%%%%%%
%
%                     FedGES
%
%%%%%%%%%%%%%%%%%%%%%%%%%%%%%%%%%%%%%%%%%%%%%%%%%%%%%%