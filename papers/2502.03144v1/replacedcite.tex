\section{Related Work}
\label{Sec:RW}
The study presented in this paper comes under Multimodal Journey Planning and Group Trip Planning Query Problems. In the following two subsections, we present relevant studies from the literature.
\subsection{Multimodal Journey Planning}
% Given the city road network and the details of transport, route, and their corresponding cost, it is an important problem to plan a journey between the given source and destination effectively. In this direction, several studies can be classified into two categories. The first category of problems appears from the passenger's PoInt of view. As per the scope of our work, here we present some recent literature on journey planning. This problem has been studied with different objectives, such as minimization of cost and time, maximization of comfort, minimization of number of transfers, etc. Potthoff and Sauer ____ made several contributions in this direction. In ____, three criteria: arrival time, the number of used public transit trips, and unrestricted transfer modes in their journey planning model. Their proposed McTB approach optimizes these three criteria efficiently. Further, in this direction, they have introduced an extended model algorithm, HydRA ____, that enables faster query executions considering multi-transfer modes.

% \par The second category of problems is from the transport manager's perspective.

Multimodal journey planning involves integrating diverse transport modes to find efficient routes between a source and a destination, considering constraints such as cost, time, and comfort. This problem is especially relevant in urban environments with complex and interconnected transport networks. Existing research in this domain can be broadly categorized into passenger-focused and transport manager-focused studies.  
The passenger-focused studies aim to optimize travel efficiency and experience by minimizing factors like travel time, cost, and number of transfers while considering comfort and convenience. Potthoff and Sauer ____ introduced the McTB approach, which optimizes three key criteria: arrival time, the number of public transit trips, and unrestricted transfer modes. Building on this, their HydRA algorithm ____ enhances query execution efficiency, supporting faster computations in multimodal networks. Delling et al. ____ developed efficient algorithms for multimodal routing and delays, incorporating transfer penalties and mode preferences. RAPTOR, proposed by Delling et al. ____, focuses on minimizing transfer times and ensuring scalability in large public transportation networks. Other studies have emphasized multi-objective optimization in multimodal planning. Chassein et al. ____ proposed bicriteria optimization models balancing travel time and reliability. From a broader perspective, graph-based approaches have also enhanced multimodal journey planning. For instance, Disser et al. ____ proposed techniques to handle multi-criteria shortest path problems in multimodal networks. 

The transport manager-focused studies address system-wide efficiency, emphasizing operational optimization. Ceder ____ outlined methods for public transit scheduling and operations essential for integrating multiple transport modes. Van Nes and Bovy ____ examined strategies for network optimization, focusing on multimodal transport infrastructure development.
\subsection{Group Trip Planning Query Problem}
To the best of our knowledge, Li et al. ____ was the first to study the trip planning query problem on metric graphs and proposed several approximate solutions. Subsequently, this problem has been extended to include a group of travelers instead of a single traveler. Here, the context is a group of agents who want to travel from their respective source to the destination location. During their journey, they wish to visit one PoI from every category, and the objective is to minimize the aggregated distance traveled by the whole group. This variant has been introduced by Hashem et al. ____. They proposed several heuristic solutions for this problem, which have been strengthened by their experimental evaluation. Later, Ahmadi et al. ____ proposed a mixed search (both breadth and depth) strategy, and they used the progressive group neighbor exploration technique. Lee and Park  ____ studied the trip planning problem to identify a common meeting PoInt such that the ride-sharing mechanism becomes effective. Mahin et al. ____ studied the Activity-aware Ridesharing Group Trip Planning Query Problem with the notion of flexible PoIs. This problem returns an optimal ridesharing group that minimizes the group cost. Recently, the GTP query problem has been studied in relation to the notion of fairness. In this direction, the first contribution came from Banerjee and Singhal ____, who studied the Envyfree GTP Query Problem and proposed a group nearest neighbor search-based approach to solve the problem. Subsequently, this study has been extended by Solanki et al. ____, and they introduced a couple of other fairness notions and provided efficient algorithms that generate solutions where the fairness criteria have been guaranteed.
\par To the best of our knowledge, the GTP Query Problem has not been studied considering the presence of multiple transport mediums. In this paper, we bridge this gap by studying the GTP Query Problem with multiple transport medium.