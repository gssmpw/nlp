\section{Related Work}
\label{sec:related}

The problem of quantifying forecast uncertainty is central in fundamental and applied research in time series forecasting~\cite{hyndman2018forecasting, petropoulos2022forecasting}. Among deep learning approaches~\cite{benidis2022deep}, many generative architectures have been proposed as means to obtain probabilistic forecasts~\cite{salinas2020deepar,rangapuram2018deep, debezenac2020normalizing, rasul2021autoregressive}. Most related to our approach are those methods that exploit quantile regression to produce probabilistic forecasts~\cite{wen2017multi, gasthaus2019probabilistic, kan2022multivariate, gouttes2021probabilistic}. 
Similarly to \gls{method}, these quantile regression techniques do not usually require strong assumptions on the underlying data distribution.  
Regarding probabilistic graph-based forecasting architecture, the existing literature is limited~\cite{jin2023survey, cini2023graphdeep}. \citet{wu2021quantifying} investigate the combination of \glspl{stgnn} with standard uncertainty quantification techniques for deep learning. \citet{pal2021rnn} use an \gls{stgnn} to implement a state-space model and quantify uncertainty within a Bayesian framework. \citet{wen2023diffstg} propose a probability predictor based on combining \glspl{stgnn} with a diffusion model~\cite{ho2020denoising}. \citet{zambon2023graph} introduce a framework for designing probabilistic graph state-space models that can process collections of time series. However, all these methods cannot operate on top of an existing pre-trained model and require training an ad-hoc forecasting model. Conversely, \gls{method} is trained, within a \gls{cp} framework, on predicting the quantiles of the error distribution of an existing model, rather than on forecasting the target variable.

\paragraph{Conformal prediction} Related work on \gls{cp} for time series  has been already discussed in \autoref{sec:cp} and \autoref{sec:method}. Related to our method, \citet{mao2024valid} propose a \gls{cp} approach for~(static) spatially correlated data. \citet{jiang2024spatio} propose to quantify the uncertainty in predicting power outages by fitting a quantile random forest~\cite{meinshausen2006quantile} on time series from neighboring geographical units.
\gls{method} can be framed among the \gls{cp} methods that learn a model of conformal scores distribution~\citep{xu2023sequential, lee2024conformal}. Differently from existing methods that operate on each time series separately, the estimates are conditioned on errors at both the target time series as well as at neighboring nodes.
To the best of our knowledge, no previous \gls{cp} method has been designed to specifically operate on collections of correlated time series and exploit graph deep learning operators. \gls{cp} methods for multivariate time series do exist~\cite{xu2024conformal, sun2024copula}, but operate on a single multidimensional time series. Moreover, although global-local models are popular among forecasting architectures~\cite{smyl2020hybrid, benidis2022deep}, \gls{method} is the first \gls{cp} architecture of this kind. Finally, \gls{cp} methods have also been applied to \textit{static} graphs and used to quantify the uncertainty of \glspl{gnn}, both in inductive \cite{zargarbashi2023conformal} and transductive settings \cite{huang2024uncertainty}. More recently, \citet{davis2024valid} proposed a \gls{cp} method for node classification with \glspl{gnn} in dynamic networks. These methods often assume node/edge exchangeability~\cite{zargarbashi2023conformal, huang2024uncertainty} or are limited to node classification tasks~\cite{clarkson2023distribution}. 


\begin{table*}[t]
\centering

\setlength{\tabcolsep}{4pt}
\setlength{\aboverulesep}{0pt}
\setlength{\belowrulesep}{0pt}
\renewcommand{\arraystretch}{1.1}
\caption{Performance comparison for $\alpha=0.1$. $\Delta$Cov values are color-coded to highlight different levels of undercoverage: \textcolor{tabgreen}{green} ($<$2\%), \textcolor{tabolive}{yellow} (2-3\%), \textcolor{taborange}{orange} (3-4\%), \textcolor{tabred}{red} ($>$4\%). The lowest Winkler score for each scenario is shown in bold.}
\begin{tabular}{@{}l|l|l|ccccc|cc@{}}
                                \multicolumn{2}{c}{}    & Metric       & \gls{scp} & \gls{nexcp} & \gls{seqcp} & \gls{scpi} & \gls{hopcpt} & \gls{cornn} & \gls{method} \\
\midrule[1.5pt]
\multirow{9}{*}{\rotatebox{90}{\gls{la}}}
 & \multirow{3}{*}{\rotatebox{90}{RNN}} & $\Delta$Cov  & \textcolor{tabgreen}{-0.65} & \textcolor{tabgreen}{-1.07} & \textcolor{tabred}{-8.10} & \textcolor{tabgreen}{-0.58{\tiny$\pm$0.01}} & \textcolor{tabolive}{-2.04{\tiny$\pm$0.30}} & \textcolor{tabgreen}{-1.71{\tiny$\pm$0.60}} & \textcolor{tabgreen}{-1.24{\tiny$\pm$0.73}} \\ &                             & PI-Width     & 20.60 & 24.12 & 18.59 & 19.81{\tiny$\pm$0.01} & 17.18{\tiny$\pm$0.11} & 17.51{\tiny$\pm$0.42} & 16.10{\tiny$\pm$0.70} \\ &                             & Winkler      & 40.22 & 39.19 & 48.29 & 37.27{\tiny$\pm$0.01} & 28.41{\tiny$\pm$0.20} & 33.90{\tiny$\pm$0.21} & \textbf{25.55{\tiny$\pm$0.44}} \\
\cmidrule[0.2pt]{3-10}
 & \multirow{3}{*}{\rotatebox{90}{\textsc{Transf}}} & $\Delta$Cov  & \textcolor{tabgreen}{-0.72} & \textcolor{tabgreen}{-1.22} & \textcolor{tabred}{-12.23} & \textcolor{tabgreen}{-0.59{\tiny$\pm$0.01}} & \textcolor{tabolive}{-2.69{\tiny$\pm$0.65}} & \textcolor{tabgreen}{-1.24{\tiny$\pm$0.39}} & \textcolor{tabgreen}{-1.02{\tiny$\pm$0.59}} \\ &                             & PI-Width     & 20.51 & 24.21 & 16.99 & 19.90{\tiny$\pm$0.00} & 16.84{\tiny$\pm$0.21} & 17.78{\tiny$\pm$0.36} & 15.72{\tiny$\pm$0.15} \\ &                             & Winkler      & 40.29 & 39.23 & 49.07 & 37.66{\tiny$\pm$0.01} & 28.58{\tiny$\pm$0.36} & 33.99{\tiny$\pm$0.21} & \textbf{25.27{\tiny$\pm$0.15}} \\
\cmidrule[0.2pt]{3-10}
 & \multirow{3}{*}{\rotatebox{90}{STGNN}} & $\Delta$Cov  & \textcolor{tabgreen}{-1.61} & \textcolor{tabgreen}{-1.24} & \textcolor{tabred}{-11.84} & \textcolor{tabgreen}{-1.56{\tiny$\pm$0.01}} & \textcolor{tabolive}{-2.31{\tiny$\pm$0.29}} & \textcolor{tabgreen}{-0.62{\tiny$\pm$0.40}} & \textcolor{tabgreen}{-1.54{\tiny$\pm$1.01}} \\ &                             & PI-Width     & 17.68 & 22.20 & 13.68 & 16.59{\tiny$\pm$0.01} & 16.07{\tiny$\pm$0.11} & 15.94{\tiny$\pm$0.35} & 15.40{\tiny$\pm$0.52} \\ &                             & Winkler      & 36.78 & 34.34 & 39.73 & 34.37{\tiny$\pm$0.02} & \textbf{26.18{\tiny$\pm$0.13}} & 30.11{\tiny$\pm$0.08} & 26.86{\tiny$\pm$0.26} \\
\midrule[1.5pt]
\multirow{9}{*}{\rotatebox{90}{\gls{cer}}}
 & \multirow{3}{*}{\rotatebox{90}{RNN}} & $\Delta$Cov  & \textcolor{taborange}{-3.42} & \textcolor{tabgreen}{0.12} & \textcolor{taborange}{-3.52} & \textcolor{taborange}{-3.41{\tiny$\pm$0.01}} & \textcolor{tabred}{-4.50{\tiny$\pm$0.59}} & \textcolor{tabgreen}{-1.31{\tiny$\pm$0.44}} & \textcolor{taborange}{-3.89{\tiny$\pm$0.15}} \\ &                             & PI-Width     & 2.61 & 3.26 & 2.77 & 2.40{\tiny$\pm$0.00} & 2.07{\tiny$\pm$0.05} & 2.11{\tiny$\pm$0.03} & 1.88{\tiny$\pm$0.01} \\ &                             & Winkler      & 5.71 & 5.46 & 5.80 & 5.31{\tiny$\pm$0.00} & 4.12{\tiny$\pm$0.06} & 4.13{\tiny$\pm$0.03} & \textbf{3.82{\tiny$\pm$0.03}} \\
\cmidrule[0.2pt]{3-10}
 & \multirow{3}{*}{\rotatebox{90}{\textsc{Transf}}} & $\Delta$Cov  & \textcolor{taborange}{-3.30} & \textcolor{tabgreen}{0.11} & \textcolor{taborange}{-3.60} & \textcolor{taborange}{-3.23{\tiny$\pm$0.02}} & \textcolor{tabred}{-4.51{\tiny$\pm$0.53}} & \textcolor{tabgreen}{-1.23{\tiny$\pm$0.90}} & \textcolor{taborange}{-3.34{\tiny$\pm$0.34}} \\ &                             & PI-Width     & 2.55 & 3.18 & 2.70 & 2.37{\tiny$\pm$0.00} & 2.04{\tiny$\pm$0.07} & 2.08{\tiny$\pm$0.04} & 1.90{\tiny$\pm$0.02} \\ &                             & Winkler      & 5.67 & 5.40 & 5.76 & 5.29{\tiny$\pm$0.00} & 4.10{\tiny$\pm$0.09} & 4.26{\tiny$\pm$0.04} & \textbf{3.80{\tiny$\pm$0.04}} \\
\cmidrule[0.2pt]{3-10}
 & \multirow{3}{*}{\rotatebox{90}{STGNN}} & $\Delta$Cov  & \textcolor{tabred}{-4.41} & \textcolor{tabgreen}{0.04} & \textcolor{taborange}{-3.63} & \textcolor{tabred}{-4.32{\tiny$\pm$0.00}} & \textcolor{tabred}{-5.28{\tiny$\pm$0.17}} & \textcolor{tabolive}{-2.08{\tiny$\pm$0.12}} & \textcolor{tabred}{-5.03{\tiny$\pm$0.52}} \\ &                             & PI-Width     & 2.29 & 2.98 & 2.42 & 2.10{\tiny$\pm$0.00} & 1.84{\tiny$\pm$0.01} & 1.90{\tiny$\pm$0.03} & 1.80{\tiny$\pm$0.04} \\ &                             & Winkler      & 5.11 & 4.89 & 5.00 & 4.81{\tiny$\pm$0.00} & \textbf{3.68{\tiny$\pm$0.02}} & 3.84{\tiny$\pm$0.04} & 3.82{\tiny$\pm$0.05} \\
\midrule[1.5pt]
\multirow{9}{*}{\rotatebox{90}{\gls{air}}}
 & \multirow{3}{*}{\rotatebox{90}{RNN}} & $\Delta$Cov  & \textcolor{tabgreen}{4.77} & \textcolor{tabgreen}{0.04} & \textcolor{tabolive}{-2.34} & \textcolor{tabgreen}{3.80{\tiny$\pm$0.02}} & \textcolor{tabgreen}{3.97{\tiny$\pm$1.07}} & \textcolor{tabgreen}{0.47{\tiny$\pm$1.60}} & \textcolor{tabolive}{-2.46{\tiny$\pm$1.64}} \\ &                             & PI-Width     & 131.17 & 86.95 & 78.87 & 120.77{\tiny$\pm$0.05} & 106.20{\tiny$\pm$6.45} & 78.00{\tiny$\pm$4.16} & 72.30{\tiny$\pm$3.56} \\ &                             & Winkler      & 159.23 & 132.59 & 136.64 & 152.17{\tiny$\pm$0.05} & 133.87{\tiny$\pm$2.39} & 111.21{\tiny$\pm$2.11} & \textbf{109.79{\tiny$\pm$1.53}} \\
\cmidrule[0.2pt]{3-10}
 & \multirow{3}{*}{\rotatebox{90}{\textsc{Transf}}} & $\Delta$Cov  & \textcolor{tabgreen}{4.77} & \textcolor{tabgreen}{0.03} & \textcolor{tabolive}{-2.23} & \textcolor{tabgreen}{3.85{\tiny$\pm$0.01}} & \textcolor{tabgreen}{4.21{\tiny$\pm$0.43}} & \textcolor{tabgreen}{-0.82{\tiny$\pm$1.14}} & \textcolor{tabgreen}{-1.04{\tiny$\pm$1.07}} \\ &                             & PI-Width     & 131.91 & 87.45 & 80.15 & 122.81{\tiny$\pm$0.03} & 108.38{\tiny$\pm$2.70} & 75.10{\tiny$\pm$1.67} & 74.44{\tiny$\pm$1.47} \\ &                             & Winkler      & 160.66 & 134.27 & 138.95 & 154.63{\tiny$\pm$0.02} & 135.54{\tiny$\pm$1.15} & 111.11{\tiny$\pm$0.04} & \textbf{110.15{\tiny$\pm$1.06}} \\
\cmidrule[0.2pt]{3-10}
 & \multirow{3}{*}{\rotatebox{90}{STGNN}} & $\Delta$Cov  & \textcolor{tabgreen}{5.19} & \textcolor{tabgreen}{-0.01} & \textcolor{tabolive}{-2.69} & \textcolor{tabgreen}{4.53{\tiny$\pm$0.01}} & \textcolor{tabgreen}{4.52{\tiny$\pm$0.21}} & \textcolor{tabgreen}{-0.74{\tiny$\pm$0.61}} & \textcolor{tabgreen}{-2.00{\tiny$\pm$1.30}} \\ &                             & PI-Width     & 135.89 & 92.93 & 81.20 & 124.27{\tiny$\pm$0.04} & 112.72{\tiny$\pm$1.53} & 75.97{\tiny$\pm$0.26} & 73.51{\tiny$\pm$2.40} \\ &                             & Winkler      & 163.12 & 139.80 & 141.37 & 154.26{\tiny$\pm$0.04} & 138.96{\tiny$\pm$0.58} & 112.90{\tiny$\pm$0.29} & \textbf{112.62{\tiny$\pm$0.91}} \\
\bottomrule[1.5pt]
\end{tabular}

\label{tab:exp1}
\end{table*}