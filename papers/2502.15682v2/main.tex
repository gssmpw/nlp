% ICCV 2025 Paper Template

\documentclass[10pt,twocolumn,letterpaper]{article}

%%%%%%%%% PAPER TYPE  - PLEASE UPDATE FOR FINAL VERSION
% \usepackage{iccv}              % To produce the CAMERA-READY version
% \usepackage[review]{iccv}      % To produce the REVIEW version
\usepackage[pagenumbers]{iccv} % To force page numbers, e.g. for an arXiv version

% Import additional packages in the preamble file, before hyperref
\newcommand{\CG}{\mathcal{G}\xspace}
\newcommand{\CV}{\mathcal{V}\xspace}
\newcommand{\CE}{\mathcal{E}\xspace}
\newcommand{\CA}{\mathcal{A}\xspace}
\newcommand{\CF}{\mathcal{F}\xspace}
\newcommand{\CR}{\mathcal{R}\xspace}
\newcommand{\CB}{\mathcal{B}\xspace}
\newcommand{\CX}{\mathcal{X}\xspace}
\newcommand{\CK}{\mathcal{K}\xspace}
\newcommand{\CM}{\mathcal{M}\xspace}
\newcommand{\CC}{\mathcal{C}\xspace}
\newcommand{\CL}{\mathcal{L}\xspace}
\newcommand{\CI}{\mathcal{I}\xspace}
\newcommand{\CQ}{\mathcal{Q}\xspace}
\newcommand{\CO}{\mathcal{O}\xspace}
\newcommand{\CP}{\mathcal{P}\xspace}
\newcommand{\CS}{\mathcal{S}\xspace}
\newcommand{\CT}{\mathcal{T}\xspace}
\newcommand{\CJ}{\mathcal{J}\xspace}
\usepackage[para]{footmisc}
\usepackage{subfig}
% \usepackage{subcaption}
% \usepackage{array}
% \usepackage{colortbl}



% It is strongly recommended to use hyperref, especially for the review version.
% hyperref with option pagebackref eases the reviewers' job.
% Please disable hyperref *only* if you encounter grave issues, 
% e.g. with the file validation for the camera-ready version.
%
% If you comment hyperref and then uncomment it, you should delete *.aux before re-running LaTeX.
% (Or just hit 'q' on the first LaTeX run, let it finish, and you should be clear).
\definecolor{iccvblue}{rgb}{0.21,0.49,0.74}
\usepackage[pagebackref,breaklinks,colorlinks,allcolors=iccvblue]{hyperref}

%%%%%%%%% PAPER ID  - PLEASE UPDATE
\def\paperID{***} % *** Enter the Paper ID here
\def\confName{ICCV}
\def\confYear{2025}

%%%%%%%%% TITLE - PLEASE UPDATE
\title{ELIP: Enhanced Visual-Language Foundation Models for Image Retrieval}

%%%%%%%%% AUTHORS - PLEASE UPDATE
\author{%
  Guanqi Zhan$^{1*}$, Yuanpei Liu$^{2*}$,
  Kai Han$^2$, Weidi Xie$^{1,3}$, Andrew Zisserman$^1$\\
    $^1$VGG, University of Oxford\quad\quad
    $^2$The University of Hong Kong \quad\quad
    $^3$Shanghai Jiao Tong University\\
  \texttt{\{guanqi,weidi,az\}@robots.ox.ac.uk} \\
  \texttt{ypliu0@connect.hku.hk}~~~~
  \texttt{kaihanx@hku.hk} \\
}




\begin{document}


\twocolumn[{
    \vspace{-30pt}
    \renewcommand\twocolumn[1][]{#1}
    \maketitle
    \centering
    \vspace{-10pt}
 \includegraphics[height=0.45\linewidth]{images/teaser.pdf}   
 \vspace{-8mm}
   \captionof{figure}{
\textbf{The ELIP architecture.}
{\em Left}: We propose a novel architecture that can be applied to pre-trained and frozen vision-language foundation models, such as CLIP, SigLIP, SigLIP-2 and BLIP-2, to enhance their text-to-image retrieval performance. 
The \emph{key idea} is to use the text query to define a set of visual prompt vectors that are incorporated into the image encoder to make it aware
of the query when generating the embedding. An MLP maps from the text space to the visual space of the input to the ViT encoder. The architecture is lightweight, and our data curation strategies enable efficient and effective training with limited resources.
{\em Right}: In this retrieval example from the COCO benchmark, the top-$k$ ($k$=100) images are re-ranked by our ELIP model for the text query: `People on bicycles ride down a busy street'. The ground truth image matching the query is not in the top-5 ranked images in the initial CLIP ranking, but is ranked top-1 (highlighted in the dashed box) by the re-ranking. 
   }
    \label{fig:teaser}
    \vspace{20pt}
    }
    ]

\def\thefootnote{*}\footnotetext{Equal contribution.}\def\thefootnote{\arabic{footnote}}

\begin{abstract}

% Recent works to jointly reconstruct 3D human and object from a single RGB image, are mostly model-based, that fail to capture the fine details of the clothed human body and object surface. In this paper, we introduce ReCHOR, a novel, model-free, first-method to produce realistic clothed human-object reconstructions from a monocular view. This is extremely challenging due to human-object occlusions, diverse interactions and depth ambiguity, as it needs to infer both 3D spatial awareness and high resolution details. Our core idea is based on estimating neural implicit representations for human and object respectively by an attention-based neural implicit model that attends to pixel-aligned features from both the global human-object image for spatial awareness and  the local separate view of human and object images for high quality details. Additionally, the network is conditioned on semantic features from an initial estimated human-object pose prior and a generative diffusion model that inpaints occluded regions, thus enabling the retrieval of details from them.
% We also propose a synthetic dataset with rendered scenes of diverse, inter-occluded 3D human and object scans, to train our network. We evaluate our method on the synthetic and real world BEHAVE dataset. Our experiments show that our method outperforms the SOTA in achieving realistic clothed human-object reconstructions.
Recent approaches to jointly reconstruct 3D humans and objects from a single RGB image represent 3D shapes with template-based or coarse models, which fail to capture details of loose clothing on human bodies. In this paper, we introduce a novel implicit approach for jointly reconstructing realistic 3D clothed humans and objects from a monocular view. For the first time, we model both the human and the object with an implicit representation, allowing to capture more realistic details such as clothing. This task is extremely challenging due to human-object occlusions and the lack of 3D information in 2D images, often leading to poor detail reconstruction and depth ambiguity. To address these problems, we propose a novel attention-based neural implicit model that leverages image pixel alignment from both the input human-object image for a global understanding of the human-object scene and from local separate views of the human and object images to improve realism with, for example, clothing details. Additionally, the network is conditioned on semantic features derived from an estimated human-object pose prior, which provides 3D spatial information about the shared space of humans and objects. To handle human occlusion caused by objects, we use a generative diffusion model that inpaints the occluded regions, recovering otherwise lost details. For training and evaluation, we introduce a synthetic dataset featuring rendered scenes of inter-occluded 3D human scans and diverse objects. Extensive evaluation on both synthetic and real-world datasets demonstrates the superior quality of the proposed human-object reconstructions over competitive methods.
\end{abstract}    
\section{Introduction}
\label{sec:intro}
% Image editing methods in diffusion models depend on user-defined control directions - users can unlock their creativity using these methods by specifying the desired manipulation through prompts~\cite{gandikota2023concept}, reference images~\cite{ruiz2022dreambooth, kumari2022customdiffusion, gal2022image, chen2024trainingfreeregionalpromptingdiffusion}, or attribute vectors~\cite{parmar2023zero,hertz2022prompt}. In this work, we ask a fundamentally different question: \emph{Can we automatically discover the underlying visual structure of a concept within diffusion model's knowledge?} %Rather than requiring user-specified controls, we aim to decompose the model's internal knowledge into meaningful directions.

% This question touches on a fundamental limitation in how we interact with diffusion models. Current control methods ~\cite{zhang2023addingconditionalcontroltexttoimage, gandikota2023concept, ye2023ipadaptertextcompatibleimage,ye2023ipadaptertextcompatibleimage, hertz2024stylealignedimagegeneration, li2023photomaker, shi2024instantbooth, chen2024trainingfreeregionalpromptingdiffusion} require users to specify their desired manipulations in advance, limiting interactive creativity. This contrasts with natural human artistic workflows, where creators dynamically explore creative ideas while jointly refining them toward meaningful artistic outcomes~\cite{hoffmann2016modeling}. This synergy between specification and exploration is not new to generative models. Early GAN architectures naturally developed disentangled latent spaces that enabled continuous\cite{harkonen2020ganspace,radford2015unsupervised, wu2021stylespace, shen2020interfacegan}, compositional control over generated images. Users could explore these spaces to discover interesting variations that would be difficult to describe in words~\cite{wu2021stylespace}, then combine them to achieve their creative goals~\cite{grabe2022towards}. 


% While diffusion models have largely superseded GANs in conditional image synthesis~\cite{dhariwal2021diffusion},  their underlying structure remains less understood. Diffusion models achieve remarkable diversity through high-dimensional latents, unlike GANs' compact latent spaces.  With a single prompt, diffusion models can generate radically different variations through different random initializations of input noise. We ask - Is it possible to discover interpretable structure within this vast space of variations?

Text-to-image diffusion models are capable of generating remarkable visual variations from a single prompt through different random initializations. However, this vast creative potential remains largely opaque to users---while we can generate diverse images, we lack understanding of the underlying structure of these variations. This presents a fundamental challenge: how can we discover and expose the latent visual capabilities encoded within these models?

\let\thefootnote\relax \footnote{$^{*}$Correspondence to \texttt{gandikota.ro@northeastern.edu}}

The challenge touches on a key limitation in how we interact with diffusion models today. Current control methods require users to explicitly specify their desired edits in advance through prompts~\cite{gandikota2023concept}, reference images~\cite{zhang2023addingconditionalcontroltexttoimage, chen2024trainingfreeregionalpromptingdiffusion, ruiz2022dreambooth,kumari2022customdiffusion, Ryu_lora, hu2021lora}, or attribute vectors~\cite{ye2023ipadaptertextcompatibleimage, hertz2024stylealignedimagegeneration, li2023photomaker, shi2024instantbooth,parmar2023zero,hertz2022prompt}. That contrasts sharply with natural human creative workflows, where artists dynamically explore creative ideas and jointly refine them toward meaningful artistic outcomes~\cite{hoffmann2016modeling}. The need for pre-specified controls creates a barrier between users and the full creative potential of these models.

Interestingly, earlier generative models like GANs~\cite{gans,karras2019style,brock2018large} naturally developed more interpretable internal structures. Their compact latent spaces often exhibited emergent disentanglement~\cite{harkonen2020ganspace,radford2015unsupervised, wu2021stylespace, shen2020interfacegan}, enabling continuous and compositional control over generated images. Users could explore these spaces to discover interesting variations that would be difficult to describe in words~\cite{wu2021stylespace}, then combine them to achieve their creative goals~\cite{grabe2022towards}.

Diffusion models have largely superseded GANs in conditional image synthesis~\cite{dhariwal2021diffusion}, achieving greater diversity through much higher-dimensional latents. And yet an understanding of the underlying structure of these larger latent spaces has remained elusive. In this work, we ask a fundamental question: \emph{Can we automatically discover the visual structure within a diffusion model's knowledge of a concept?} Rather than requiring user-specified controls, we aim to decompose the model's internal representations into expressive directions that users can explore and combine.

To address these needs, we present \textbf{SliderSpace}, a framework that brings systematic explorability to diffusion models. Given just a text prompt, SliderSpace discovers a canonical set of meaningful, diverse, and controllable directions within the model's knowledge of that concept. Each direction is implemented as a low-rank adapter~\cite{hu2021lora} that can be scaled and composed with others, allowing users to explore and smoothly combine different aspects of variation, as shown in Figure~\ref{fig:intro}.

We ground SliderSpace discovery in three key requirements for meaningful decomposition of a diffusion model's visual manifold: 
\begin{enumerate}
    \item \textbf{Unsupervised Discovery:} The decomposition process should emerge from the intrinsic structure of the model's learned representation, rather than being guided by predefined attributes. This ensures we capture the true topology of the model's knowledge space rather than projecting our assumptions onto it.
    
    \item \textbf{Semantic Orthogonality:} Each discovered control must represent a distinct semantic direction. This is enforced in a semantic feature space, like CLIP, where every slider has an orthogonal effect in embeddings. This prevents discovering multiple controls that create similar semantic effects, making the system more efficient and easier.
    
    \item \textbf{Distribution Consistency:} Directions must induce consistent transformations across both random seeds and prompt variations. 
\end{enumerate}

These requirements naturally lead to our proposed framework, which we formalize in Section~\ref{sec:method}. As we show in our experiments, SliderSpace is architecture-agnostic, working with both conventional U-Net based models like Stable Diffusion~\cite{rombach2022high, rombach2022sd20, podell2023sdxl, turbo, dmd} and recent transformer-based architectures like Flux~\cite{flux}.

We demonstrate the expressiveness of SliderSpace through three applications: First, we show how SliderSpace can decompose high-level concepts into diverse and expressive components, revealing the natural axes of variation in the model's understanding. Second, we explore artistic style variation, where SliderSpace discovers directions that match or exceed the diversity of manually curated artist lists while being judged more useful by human evaluators. Finally, we show how SliderSpace can help reverse the mode collapse commonly observed in distilled diffusion models, restoring diversity while maintaining generation speed.

Beyond providing practical creative control, SliderSpace opens new avenues for understanding and utilizing the latent capabilities of diffusion models. By mapping these models' visual potential into intuitive, composable directions, we take a step toward making their creative possibilities more accessible and interpretable to users.

% Image editing methods in diffusion models unlock the creativity of users. In this work we ask an alternate question: \emph{Can we organize and expose what of the diffusion model is already capable of?}.
% Existing methods for controlling image generation typically require users to manually specify edit directions for desired changes. This process is time-consuming, requires technical expertise, and limits the spontaneity of the creative process. For instance, if a user wants to adjust the smile of a generated person, they must explicitly request this edit, often through imprecise prompt engineering or model fine-tuning. This approach of predefined controls or manual specifications restricts users from fully exploring the latent capabilities of the model. There may be interesting stylistic variations or attributes that the model can generate, but users have no easy way to discover or utilize these.

% Natural visual disentanglement was an emergent property in the latent space of Generative Adversarial Models (GANs) \cite{harkonen2020ganspace,radford2015unsupervised, wu2021stylespace, shen2020interfacegan}. In particular, it has been observed that StyleGAN~\cite{karras2019style} stylespace neurons offer detailed control over many meaningful aspects of images that would be difficult to describe in words~\cite{wu2021stylespace}. However, diffusion models do not share such a compact latent space~\cite{park2023unsupervised}; and efforts to uncover such a space in the semantic embeddings of the text conditioning have met with limited success \nik{Nick - is there a specific citation you were thinking about?}.

% In this work we introduce \textbf{SliderSpace}, which takes a step towards uncovering an analogous low dimensional representation of diffusion models' visual breadth; in essence treating the diffusion model as many generators sharing parameters, where a particular generator is defined by a specific prompt. For a given prompt we sample many random seeds (and optionally prompt expansions using an LLM), generate the corresponding images, and apply an off the shelf feature extractor (in this work CLIP, but our method can be applied to any differentiable feature extractor). We use PCA to analyze these features, and for each of the leading $k$ principal components we train a LoRA \cite{} which causes the diffusion model to produces images which increase the feature magnitude along that component when passed back through the same feature extractor. This leads to a 'Slider' for each principal component, because each LoRA can be scaled and applied to the original diffusion model, continuously varying those visual features in the generated results (as measured, in our case, by CLIP).

% There are many other works that enhance the controllability of diffusion models. One common approach is enabling users to add spatial constraints to a generation either manually, or via a reference image \cite{zhang2023addingconditionalcontroltexttoimage, chen2024trainingfreeregionalpromptingdiffusion}, a second is leveraging more abstract embeddings (e.g. identity, style) extracted from a reference image \cite{ye2023ipadaptertextcompatibleimage, hertz2024stylealignedimagegeneration, li2023photomaker, shi2024instantbooth}, a third is finetuning a foundation model to better generate a concept important to the user \cite{ruiz2022dreambooth, kumari2022customdiffusion, Ryu_lora, hu2021lora}, and a fourth (most relevant to this work) is finding low-rank adaptors of the model based on a prompt or small training set which can be scaled to provide continous control over one aspect of generated image (e.g. night vs day, basic vs luxury, etc.) \cite{gandikota2023concept}. SliderSpace is complementary to all of these methods and offers something distinct. All of the other methods we are aware require the user (and / or model designer) to know in advance what type of control they want. In contrast SliderSpace assists users in discovering and controlling hidden capabilities present in the diffusion model's distribution of possible generations.

%We propose that truly intuitive creative control in a text-to-image model should meet three key criteria: \emph{discoverability}, \emph{intuitiveness}, and \emph{specificity}. The model should reveal controllable attributes that may not be immediately obvious, offer controls that are easy to understand and manipulate, and ensure each control affects a distinct attribute of the generated image.

% We demonstrate the utility and power of SliderSpace using three applications built on top of SDXL-DMD \cite{dmd}, because its fast generation speed lends itself well to the continuous control offered by SliderSpace.

% First, we study concept decomposition (Section \ref{sec:concept_exp}), where we learn sliders for a specific concept (e.g. 'monster', 'waterfall', 'car'). Through quantitative metrics of diversity and text alignment we demonstrate that the learned sliders dramatically boost the diversity of generations when randomly applied without harming text alignment; we also ask humans to qualitatively judge these results in a user study where they find the SliderSpace results to be more 'Diverse', 'Useful', and 'Creative' than our baselines.

% Second, we attempt to compare the automatic discoveries of SliderSpace to a large scale manual study of artistic styles (Section \ref{sec:art_exp}), open-sourced by ParrotZone \cite{parrotzone}. In this study SDXL was prompted with over 4300 artist names,  and based on visual inspection the cases of successful stylistic mimicry recorded. Quantitatively SliderSpace more closely matches the distribution of artistic variation discovered by ParrotZone than other baselines, and in our user studies was judged to be significantly more 'Diverse' and 'Useful' than the baselines. To our surprise humans even judged SliderSpace results to be slightly more 'Diverse' than the results generated by the manually discovered artist names of \cite{parrotzone}.

% Third, we attempt to use SliderSpace to reverse the mode collapse commonly observed in distilled few-step diffusion models relative to the original teacher model (Section \ref{sec:diverse_exp}). We quantitatively demonstrate that applying SliderSpace to SDXL-DMD leads to more closely matching the distribution of images by the original teacher, SDXL.

%Through extensive experiments on various state-of-the-art text-to-image models, we demonstrate that SliderSpace significantly enhances user control and creative expression in AI-assisted image generation tasks. Our method enables a range of applications, including concept decomposition and control, diversity improvement in generated images, customization dissection and edits, and the exploration of artistic styles inherent in the model.

% SliderSpace goes beyond providing a practical tool for enhanced creative control. By mapping the visual potential of diffusion models it can open new avenues for generative creativity and deepens our understanding of each model's hidden potential.
\section{Related Work}

\paragraph{LLMs for Agent tasks.}

Our research is related to deploying large language models (LLMs) as agents for decision-making tasks in interactive environments~\citep{liu2023agentbench,zhou2023webarena,shridhar2020alfred,toyama2021androidenv}. Earlier works, such as~\citep{yao2023webshopscalablerealworldweb}, fine-tuned models like BERT~\citep{devlin2019bertpretrainingdeepbidirectional} for decision-making in simplified environments, such as online shopping or mobile phone manipulation. With the advent of large language models~\citep{brown2020languagemodelsfewshotlearners,openai2024gpt4technicalreport}, it became feasible to perform decision-making tasks through zero-shot or few-shot in-context learning. To better assess the capabilities of LLMs as agents, several models have been developed~\citep{deng2024mind2web,xiong2024watch,hong2023cogagent,yan2023gpt}. Most approaches~\citep{zheng2024seeact,deng2024mind2web} provide the agent with observation and action history, and the language model predicts the next action via in-context learning. Additionally, some methods~\citep{zhang2023building,li2023camel,song2024trial} attempt to distill trajectories from state-of-the-art language models to train more effective policy models. In contrast, our paper introduces a novel framework that automatically learns a reward model from LLM agent navigation, using it to guide the agents in making more effective plans.

\textbf{LLM Planning.} Our paper is also related to planning with large language models. Early researchers~\citep{brown2020languagemodelsfewshotlearners} often prompted large language models to directly perform agent tasks. Later, \citet{yao2022react} proposed ReAct, which combined LLMs for action prediction with chain-of-thought prompting~\citep{wei2022chain}. Several other works~\citep{yao2023treethoughtsdeliberateproblem,hao2023reasoning,zhao2023large,qiao2024agentplanningworldknowledge} have focused on enhancing multi-step reasoning capabilities by integrating LLMs with tree search methods. Our model differs from these previous studies in several significant ways. First, rather than solely focusing on text generation tasks, our pipeline addresses multi-step action planning tasks in interactive environments, where we must consider not only historical input but also multimodal feedback from the environment. Additionally, our pipeline involves automatic learning of the reward model from the environment without relying on human-annotated data, whereas previous works rely on prompting-based frameworks that require large commercial LLMs like GPT-4~\citep{openai2024gpt4technicalreport} to learn action prediction. Furthermore, \Model supports a variety of planning algorithms beyond tree search.

\textbf{Learning from AI Feedback.} In contrast to prior work on LLM planning, our approach also draws on recent advances in learning from AI feedback~\citep{bai2022constitutional,lee2023rlaif,yuan2024self,sharma2024critical,pan2024autonomous,koh2024tree}. These studies initially prompt state-of-the-art large language models to generate text responses that adhere to predefined principles and then potentially fine-tune the LLMs with reinforcement learning. Like previous studies, we also prompt large language models to generate synthetic data. However, unlike them, we focus not on fine-tuning a better generative model but on developing a classification model that evaluates how well action trajectories fulfill the intended instructions. This approach is simpler, requires no reliance on state-of-the-art LLMs, and is more efficient. We also demonstrate that our learned reward model can integrate with various LLMs and planning algorithms, consistently improving their performance.

\textbf{Inference-Time Scaling.} ~\citet{snell2024scaling} validates the efficacy of inference-time scaling for language models. Based on inference-time scaling, various methods have been proposed, such as random sampling~\citep{wang2022self} and tree-search methods~\citep{hao2023reasoning, zhang2024accessing, guan2025rstar}. Concurrently, several works have also leveraged inference-time scaling to improve the performance of agentic tasks. ~\citet{koh2024tree} adopts a training-free approach, employing MCTS to enhance policy model performance during inference and prompting the LLM to return the reward. ~\citet{gu2024your} introduces a novel speculative reasoning approach to bypass irreversible actions by leveraging LLMs or VLMs. It also employs tree search to improve performance and prompts an LLM to output rewards. ~\citet{yu2024exact} proposes Reflective-MCTS to perform tree search and fine-tune the GPT model, leading to improvements in ~\citet{koh2024visualwebarena}. ~\citet{putta2024agent} also utilizes MCTS to enhance performance on web-based tasks such as ~\citet{yao2023webshopscalablerealworldweb} and real-world booking environments. ~\cite{lin2025qlass} utilizes the stepwise reward to give effective intermediate guidance across different agentic tasks. Our work differs from previous efforts in two key aspects: (1) Broader Application Domain. Unlike prior studies that primarily focus on tasks from a single domain, our method demonstrates strong generalizability across web agents, mathematical reasoning, and scientific discovery domains, further proving its effectiveness. (2) Flexible and Effective Reward Modeling. Instead of simply prompting an LLM as a reward model, we finetune a small scale VLM~\citep{lin2023vila} to evaluate input trajectories. %Our reward scores range continuously between 0 and 1, in contrast to existing methods that rely on discrete scoring (e.g., 0 and 1, or 0, 0.5, and 1) through direct LLM prompting.

% Concurrently, several works have also leveraged inference-time scaling to improve the performance of agentic tasks. ~\citet{pan2024autonomous} demonstrates that LLMs and VLMs, such as the GPT series, can function as evaluators or reward models to provide guidance for fine-tuning or reflection, thereby enhancing digital agents. This lays the groundwork for subsequent studies that directly prompt LLMs as reward models. ~\citet{koh2024tree} adopts a training-free approach, employing MCTS to enhance policy model performance during inference. However, it is limited to web environments~\citep{koh2024visualwebarena}. Moreover, its value function relies on prompting an LLM, which is less effective than our proposed method. We validate our approach through ablation studies, demonstrating that our fine-tuned reward model is more effective. ~\citet{gu2024your} introduces a novel speculative reasoning approach to bypass irreversible actions, such as purchasing a product, by leveraging LLMs or VLMs. It also employs tree search to improve performance, but it remains restricted to the web domain~\citep{koh2024visualwebarena, deng2024mind2web}. Additionally, it lacks reward modeling and instead prompts an LLM to output rewards. ~\citet{yu2024exact} proposes Reflective-MCTS to perform tree search and fine-tune the GPT model, leading to improvements in ~\citep{koh2024visualwebarena}. However, this work focuses solely on a single web agent task, and its reward modeling is derived from multi-agent debate, differing from our more effective and efficient reward modeling approach. ~\citet{putta2024agent} also utilizes MCTS to enhance performance, but it is limited to web-based tasks such as ~\citep{yao2023webshopscalablerealworldweb} and real-world booking environments.
\section{Preliminaries}
\label{sec:preliminaries}
We first set up notations and mathematically formulate tasks.

\noindent\textbf{Language-Conditioned Imitation Learning (LC-IL)}. The task of LC-IL aims to train an agent to mimic expert behaviors from a given demonstration set $\mathcal{D}_d = \{(\mathbf{\tau}_i,l_i)\}_{i=1}^N$, where $l_i \in \mathcal{L} $ represents a task-specific language instruction. Each trajectory $\mathbf{\tau}_i\in\mathcal{T}$ consists of a sequence of state-action pairs $\mathbf{\tau}_i = \{(\mathbf{s}_j, \mathbf{a}_j)\}_{j=1}^T$ of the horizon length $T$. In robot manipulation tasks, action $\mathbf{a}_j\in\mathcal{A}$ corresponds to the control commands executed by the agent and state $\mathbf{s}_j = [\mathbf{p}_j; \mathbf{v}_j] \in\mathcal{S}$ records proprioceptive data $\mathbf{p}_j$ (\textit{e.g.,} joint positions, velocities) and visual inputs $\mathbf{o}_j\in\mathcal{O}$ (\textit{e.g.,} camera images) at the time step $j$. The objective of LC-IL is to find an optimal language-conditioned policy $\pi^*(\mathbf{a}|\mathbf{s},l): \mathcal{S}\times\mathcal{L}\mapsto\mathcal{A}$ via solving the supervised optimization as follows,
\begin{equation}\nonumber
    \pi^* \in \arg\min_{\pi} \mathbb{E}_{(\tau_i, l_i)\sim \mathcal{T}} \left[ \frac{1}{T} \sum_{(\mathbf{s}_j, \mathbf{a}_j) \sim \tau_i} \ell(\pi(\hat{\mathbf{a}}_j, \mathbf{s}_j|l_i),  \mathbf{a}_j)\right],
\end{equation}
where \(\ell(\cdot, \cdot)\) is a task-specific loss, such as mean squared error or cross-entropy. Training the policy \(\pi_\theta\) in an end-to-end fashion may require \textit{hundreds} of high-quality expert demonstrations to converge, primarily due to the high variance of visual inputs $\mathbf{o}$ and language instructions $l$.

% We study the problem of Language-Conditioned Imitation Learning ~\cite{rss21-gcil}, where the goal is to train an agent to perform tasks by conditioning its policy on both the state of the environment and language instruction. Formally, let \(\mathcal{O}\) be the observation space, \(\mathcal{A}\) the action space, and \(\mathcal{L}\) the language instruction space. The observation space \(\mathcal{O}\) typically includes visual or sensor data, such as images, that represent the partial observation of state \(\mathcal{S}\). The objective is to learn a policy \(\pi_\theta : \mathcal{O} \times \mathcal{L} \to \mathcal{A}\), parameterized by \(\theta\), that maps an observation \(o \in \mathcal{O}\) and a language instruction \(L \in \mathcal{L}\) to an action \(a \in \mathcal{A}\). We assume access to a dataset of expert demonstrations \(\mathcal{D}_{\operatorname{demo}} = \{(\{o_k^i, a_k^i\}_{i=1}^T, L_k)\}_{k=1}^N\), where each sample consists of a $T$-step observation-action trajectory and a corresponding language instruction \(L_k \in \mathcal{L}\). The goal is to train the policy \(\pi_\theta\) by minimizing the following loss function:
% \[
% \mathcal{L}(\theta) = \frac{1}{N} \sum_{k=1}^N \sum_{i=1}^T \ell(a_k^i, \pi_\theta(o_k^i, L_k)),
% \]
% where \(\ell(\cdot, \cdot)\) is a task-specific loss function, such as mean squared error or cross-entropy. 
\begin{table}
\centering
\caption{Comparison of different component designs in time contrast learning across mainstream vision-language pre-training. \vspace{1ex}
% The goal frame $o_g$ is typically set as the last frame $o_{T}$.
 }
\label{tab:comp}
\Large
\resizebox{\linewidth}{!}{ 
\begin{tabular}{llll}
\toprule
$\operatorname{Method}$      & \textcolor{black}{$\mathcal{P}(\mathcal{O}_{i})$}  & \textcolor{black}{$\mathcal{N}(\mathcal{O}_{i})$} & $\mathfrak{R}(\mathbf{v},\mathbf{l}_i)$  \\ \hline
$\operatorname{R3M}$         & $(o_0, o_{j>i})$      &  $(o_0,o_i,o_j^{\notin O_i})$   & $\operatorname{reward}(\mathbf{v},\mathbf{l}_i)$   \\    
$\operatorname{LIV}$         & $(o_T)$    &  $(o_T^{\notin O_i})$    & $\operatorname{cos}(\mathbf{v},\mathbf{l}_i)$  \\    
$\operatorname{DecisionNCE}$ & $(o_i,o_{j>i})$     &     $(o_i^{\notin O_i},o_{j>i}^{\notin O_i})$  & $\operatorname{cos}(\mathbf{v}_j-\mathbf{v}_i, \mathbf{l}_i)$  \\          
$\operatorname{AcTOL}$        & $(o_i,o_{j \in [T] \setminus \{i\}})$ & $(o_i,o_k: d_{i, k}>d_{i, j})$  & $-\Vert \operatorname{cos}(\mathbf{v}_i, \mathbf{l}_i)-\operatorname{cos}(\mathbf{v}_j, \mathbf{l}_i) \Vert_2 $     \\  \bottomrule                                                              
\end{tabular}
}
\end{table}

\paragraph{Vision-language Pre-training.}  Address such scalability issues can be achieved by leveraging large-scale, easily accessible human action video datasets $\mathcal{D}_p = \{(\mathcal{O}_i, l_i)\}_{i=1}^M$ \cite{corr18-epickitchen,cvpr22-ego4d}, where $\mathcal{O}_i=\{o_j\}_{j=1}^T$ represents a video clip with $T$ frames and $l_i$ the corresponding description. Pretraining on such datasets enables policies to rapidly learn visual-language correspondences with minimal expert demonstrations. Mainstream pretraining methods employ time contrastive learning \cite{icra18-tcn} to fine-tune a visual encoder $\mathcal{\phi}$ and a text encoder $\mathcal{\varphi}$, which project frames and descriptions into a shared $d$-dimensional embedding space, \textit{i.e.}, $\mathbf{v}_j = \phi(o_j)\in\mathbb{R}^d$ and $\mathbf{l}_i = \varphi(l_i)\in\mathbb{R}^d$. To provide a unified perspective on various pretraining approaches, we formulate them within the objective $\mathcal{L}_{\operatorname{tNCE}}(\phi, \varphi)$: \vspace{-2ex}
\begin{align}\nonumber\small
\mathcal{L}_{\operatorname{tNCE}}&=
-\mathbb{E}_{\substack{\scriptstyle o^+\sim\textcolor{black}{\mathcal{P}(\mathcal{O}_i)}}}
    \log  
    \frac{
        \exp(\mathfrak{R}(\mathbf{v}^+, \mathbf{l}_i))
    }{
        \mathbb{E}_{\scriptstyle o^- \sim \textcolor{black}{\mathcal{N}(\mathcal{O}_i)}}
        \exp(\mathfrak{R}(\mathbf{v}^-, \mathbf{l}_i))
    },
\end{align}

% \begin{align}\nonumber\small
% \mathcal{L}_{\operatorname{tNCE}}&=
% -\mathbb{E}_{\substack{\scriptstyle o\sim O_i \\ \scriptstyle o^+\sim\textcolor{black}{\mathcal{P}(o)}}}
%     \log  
%     \frac{
%         \exp(\mathfrak{R}(\mathbf{v}^+, \mathbf{v}, \mathbf{l}_i))
%     }{
%         \mathbb{E}_{\scriptstyle o^- \sim \textcolor{black}{\mathcal{N}(o)}}
%         \exp(\mathfrak{R}(\mathbf{v}, \mathbf{v}^-, \mathbf{l}_i))
%     },\vspace{-2ex}
% \end{align}
% where $\mathbf{v} = \phi(o)$, and 
where $\mathbf{v}^{+/-} = \phi(o^{+/-})$. Different pretraining strategies differ in their selection of (1) the positive frame set $\mathcal{P}(\mathcal{O}_i)$, (2) negative frame set $\mathcal{N}(\mathcal{O}_i)$; and (3) the semantic alignment scoring function $\mathfrak{R}(\mathbf{v}, \mathbf{l}_i)$ measuring the gap of VL similarities as detailed in Table \ref{tab:comp}. 

\noindent\textbf{Discussion.} As motivated by goal-conditioned RL \cite{nips17-her}, current approaches \textit{explicitly} select future frames (\textit{e.g.}, DecisionNCE) or the last frame (\textit{e.g.}, LIV) as the goal within the positive set, enforcing their visual embedding to align with the semantics. Likewise, the scoring functions $\mathfrak{R}$ are often designed to maximize this transition direction. However, the pretraining action videos are \textit{noisy} as actions may terminate early or include irrelevant subsequent actions, which may mislead the encoders and result in inaccurate vision-language association. As detecting precise action boundaries is non-trivial, we argue for a more flexible approach that leverages \textit{intrinsic} characteristics of actions to guide pretraining.



% we first pre-train a visual encoder \(\mathcal{\phi}: \mathcal{O} \to \mathbb{R}^d\) and a text encoder \(\mathcal{\varphi}: \mathcal{L} \to \mathbb{R}^d\) to learn mappings from the observation and the language instruction space to $d-$dimensional feature spaces. This pre-training can be done using large, less-expensive data without action annotation, such as human action videos . Then, with the frozen learned features \(\boldsymbol{v}\) and \(\boldsymbol{l}\) as input, we can only fine-tune a simple Multi-Layer Perceptron (MLP) with a few demonstrations to learn the map from the feature space \(\mathbb{R}^d \times \mathbb{R}^d\) to the action space \(\mathcal{A}\). Since both the observation space \(\mathcal{O}\) and the action space \(\mathcal{A}\) are continuous and ordered over time, we expect the representations learned through pre-training to also exhibit continuity and orderliness. This property in the representations allows for better learning of the continuous mapping between observations and actions. This property offers three significant benefits: First, the orderliness of the representation ensures that different states of the task, such as the start and end of an action, can be better captured and distinguished. Second, the continuity of the representation allows it to evolve smoothly as the task progresses, enabling the model to output stable actions based on the current state. Finally, we can demonstrate that even under small perturbations to the language instruction, these properties ensure the robustness of the learned representation. This robustness is crucial for maintaining performance in real-world scenarios where language instructions might contain minor ambiguities or variations.





% We consider a partially observable Markov Decision Process (POMDP) with language conditions, which models the interaction between an agent and an environment where observations are incomplete and actions are guided by natural language instructions. Formally, a POMDP is defined as a tuple $\langle \mathcal{S}, \mathcal{A}, \mathcal{O}, \mathcal{T}, \mathcal{R}, \mathcal{Z}, \gamma \rangle$, where $\mathcal{S}$ is the state space, $\mathcal{A}$ is the action space available to the agent. $\mathcal{O}$ is the observation space, which provides partial information about the environment. $\mathcal{T}(s' \mid s, a)$ is the state transition function. $\mathcal{R}(s, a)$ is the reward function. $\mathcal{Z}(o \mid s, a)$ is the observation function. $\gamma \in [0, 1)$ is the discount factor.

% To incorporate language instructions, we introduce a task description $L$, which specifies the agent's goal in natural language. The task description conditions the agent's policy $\pi(a \mid o, L)$, where $o$ is the agent's current observation. The agent aims to maximize the expected cumulative reward while adhering to the task described by $L$.

% Further, we assume the availability of a large-scale human action video dataset including $N$ video-instruction pairs, $\{(\{o_k^i\}_{i=1}^{t_k}, L_k)\}_{k=1}^N$, where each pair representing an action video with $t_k$ frames and its corresponding language description $L_k$. We pre-train the visual and language encoders on this dataset, with the visual features $\boldsymbol{v} = \operatorname{Enc}_v(o)$ and the language features $\boldsymbol{l} = \operatorname{Enc}_l(L)$. These pre-trained representations are then frozen and applied to train the policy $\pi$ in the aforementioned decision-making process, enabling the agent to better interpret and act upon language-conditioned tasks.
% \section{Comparison of Normalization Strategies} \label{sec:ln_in_transformer}
\section{Comparative Analysis} \label{sec:ln_in_transformer}

In this section, we discuss how different placements of layer normalization (LN \footnote{Unless stated otherwise, LN refers to both LayerNorm and RMSNorm.}) in Transformer architecture affect both training stability and the statistics of hidden states (activations \footnote{We use ``hidden state'' and ``activation'' interchangeably.}).

\subsection{Post- \& Pre-Normalization in Transformers}
\label{subsec:post_pre_ln}
\paragraph{Post-LN.}
The Post-Layer Normalization (Post-LN) \citep{attentionisallyouneed} scheme, normalization is applied \emph{after} summing the module’s output and residual input:
\begin{equation}
    y_{l} = \mathrm{Norm}\bigl(x_l + \mathrm{Module}(x_l)\bigr),
    \label{eq:post_ln}
\end{equation}
where $x_l$ is the input hidden state of $l$-th layer, $y_{l}$ is the output hidden state of $l$-th layer, and $\mathrm{Module}$ denotes Attention or Multi-Layer Perceptron (MLP) module in the Transformer sub-layer. $\mathrm{Norm}$ denotes normalization layers such as RMSNorm or LayerNorm. It is known that by stabilizing the activation variance at a constant scale, Post-LN prevents activations from growing. However, several evidence~\citep{onlayer, transformersgetstable} suggest that Post-LN can degrade gradient flow in deeper networks, leading to vanishing gradients and slower convergence.


\paragraph{Pre-LN.}
The Pre-Layer Normalization (Pre-LN)~\citep{llama3} scheme, normalization is applied to the module's input \emph{before} processing:
\begin{equation}
    y_l = x_l + \mathrm{Module}\bigl(\mathrm{Norm}(x_l)\bigr).
    \label{eq:pre_ln}
\end{equation}
As for Llama $3$ architecture, a final LN is applied to the network output. Pre-LN improves gradient flow during backpropagation, stabilizing early training \citep{onlayer}. Nonetheless, in large-scale Transformers, even Pre-LN architectures are not immune to instability during training~\citep{smallproxies, attentioncollapse}. As shown in Figure~\ref{fig:LN Placement}, unlike Post-LN—which places LN at position $C$—Pre-LN, which places LN only at position $A$, can lead to a “highway” structure that is continuously maintained throughout the entire model if the module produces an output with a large magnitude. This phenomenon might be related to the ``massive activations'' observed in trained models \citep{massiveactivation, mlpswiglu}. 

\begin{figure}[t]
% \vskip -0.1in
    \centering
    \begin{minipage}[t]{0.45\linewidth}
        \vspace{0pt}
        \centering
        \includegraphics[width=.75\linewidth]{Figures/method_abc_ver3.png}
    \end{minipage}
    \centering
    \begin{minipage}[t]{0.45\linewidth}
        \vspace{32pt}
        \centering
        \small
        \begin{tabular}{lccc}
            \toprule
            ~ & A & B & C  \\ 
            \midrule
            Post-LN & \texttimes & \texttimes & \checkmark \\
            Pre-LN  & \checkmark & \texttimes & \texttimes \\
            Peri-LN & \checkmark & \checkmark & \texttimes \\
            \bottomrule
        \end{tabular}
    \end{minipage}
    \caption{Placement of normalization in Transformer sub-layer. }
    \label{fig:LN Placement}
    \vskip -0.1in
\end{figure}

\begin{table}
\caption{Intuitive comparison of normalization strategies.}
\label{tab:variance_summary}
\small
\begin{tabular}{lcc}
\toprule
\textbf{Strategy} & \textbf{Variance Growth} & \textbf{Gradient Stability} \\
\midrule
\textbf{Post-LN} & Mostly constant & Potential for vanishing \\
\textbf{Pre-LN} & Exponential in depth & Potential for explosion \\
\textbf{Peri-LN} & $ \approx \text{Linear}$ in depth & Self-regularization \\
\bottomrule
\end{tabular}
\vskip -0.1in
\end{table} 



\subsection{Variance Behavior from Initialization to Training}
\label{subsec:variance_growth}


As discussed by \citet{onlayer} and \citet{transformersgetstable}, Transformer models at \emph{initialization} exhibit near-constant hidden-state variance under Post-LN and linearly increasing variance under Pre-LN. Most of the previous studies have concentrated on this early-stage behavior. However, Recent studies have also reported large output magnitudes in both the pre-trained attention and MLP modules \citep{vit22b, smallproxies, mlpswiglu}. To bridge the gap from initialization to the fully trained stage, we extend our empirical observations in Figure~\ref{fig:3iter} beyond initial conditions by tracking how these variance trends evolve at intermediate points in training. 

We find that Post-LN maintains a roughly constant variance, which helps avert exploding activations. Yet as models grow deeper and training proceeds, consistently normalizing $x_l + \mathrm{Module}(x_l)$ can weaken gradient flow, occasionally causing partial vanishing gradients and slower convergence. In contrast, Pre-LN normalizes $x_l$ before the module but leaves the module output unnormalized, allowing hidden-state variance to accumulate exponentially once parameter updates amplify the input. Although Pre-LN preserves gradients more effectively in earlier stages, this exponential growth in variance can lead to “massive activations” \citep{massiveactivation}, risking numeric overflow and destabilizing large-scale training. We reconfirm this in Section~\ref{sec:experiments}.

\paragraph{Takeaways.}
\begin{itemize}
\item \textit{Keeping the Highway Clean: Post-LN’s Potential for Gradient Vanishing and Slow Convergence.} When layer normalization is placed directly on the main path (Placement $C$ in Figure \ref{fig:LN Placement}), it can cause gradient vanishing and introduce fluctuations in the gradient scale, potentially leading to instability. 

\item \textit{Maintaining a Stable Highway: Pre-LN May Not Suffice for Training Stability.} Pre-LN does not normalize the main path of the hidden states, thereby avoiding the issues that Post-LN encounters. Nevertheless, a structural characteristic of Pre-LN is that any large values arising in the attention or MLP modules persist through the residual identity path. In particular, as shown in Figure~\ref{fig:3iter}, the exponentially growing magnitude and variance of the hidden states in the forward path may lead to numerical instability and imbalance during training.
\end{itemize}

Recent open-sourced Transformer architectures have adopted normalization layers in unconventional placements. Models like Gemma$2$ and OLMo$2$ utilize normalization layers at the module output (Output-LN), but the benefits of these techniques remain unclear \citep{gemma2, olmo2}. To investigate the impact of adding an Output-LN, we explore the peri-layer normalization architecture.


\subsection{Placing Module Output Normalization}
\label{subsec:peri_ln}

\paragraph{Peri-LN.}
The Peri-Layer Normalization (Peri-LN) applies LN twice within each layer---before and after the module---and further normalizes the input and final output embeddings. Formally, for the hidden state $x_l$ at layer $l$:
\begin{enumerate}
    \item \textit{Initial Embedding Normalization:}
    \[
      y_o = \mathrm{Norm}(x_o),
    \]
    \item \textit{Input- \& Output-Normalization per Layer:}
    \[
      y_l = x_l + \mathrm{Norm}\Bigl(\mathrm{Module}\bigl(\mathrm{Norm}(x_l)\bigr)\Bigr),
    \]
    \item \textit{Final Embedding Normalization:}
    \[
      y_L = \mathrm{Norm}(x_L),
    \]
\end{enumerate}
where $x_o$ denotes the output of the embedding layer, the hidden input state. $y_0$ represents the normalized input hidden state. $x_L$ denotes the hidden state output by the final layer \(L\) of the Transformer sub-layer. This design unifies pre- and output-normalization to regulate variance from both ends. For clarity, the locations of normalization layers in the Post-, Pre-, and Peri-LN architectures are illustrated in Figure~\ref{fig:LN Placement}.


\paragraph{Controlling Variance \& Preserving Gradients.}
% \paragraph{Roll of Output Layer Normalization.}
By normalizing both the input and output of each sub-layer, Peri-LN constrains the \emph{residual spikes} common in Pre-LN, while retaining a stronger gradient pathway than Post-LN. Concretely, if $\mathrm{Norm}(\mathrm{Module}(\mathrm{Norm}(x_l)))$ has near-constant variance $\beta_0$, then
\[
  \mathrm{Var}(x_{l+1}) \;\approx\; \mathrm{Var}(x_l) + \beta_0,
\]
leading to \emph{linear or sub-exponential} hidden state growth rather than exponential blow-up.  We empirically verify this effect in Section~\ref{subsec:growth of hidden state}. 



\begin{figure*}[t]
\vskip -0.1in
    \centering
    \subfigure[Learning rate exploration]
    {
    \includegraphics[width=.3\linewidth]{Figures/pretrain_lrsweep.png}
    \label{fig:pretrain_lrwseep}
    }
    \subfigure[Training loss]
    {
    \includegraphics[width=.295\linewidth]{Figures/hcx_text_400M_dclm_000_30B_warmup10_lr5e4.csv_best_loss_trainingloss_per_tokens.png}
    \label{fig:pretrain_loss}
    }
    \subfigure[Gradient-norm]
    {
    \includegraphics[width=.288\linewidth]{Figures/hcx_text_400M_dclm_000_30B_warmup10_lr5e4.csv_best_loss_warmup10_gradnorm_per_tokens.png}
    \label{fig:pretrain_gradnorm}
    }
    \caption{
    Performance comparison of Post-LN, Pre-LN, and Peri-LN Transformers during pre-training. Figure \ref{fig:pretrain_lrwseep} llustrates the pre-training loss across learning rates. Pre-training loss and gradient norm of best performing $400$M size Transformers are in Figure \ref{fig:pretrain_loss} and \ref{fig:pretrain_gradnorm}. Consistent trends were observed across models of different sizes.
    }
    \label{fig:pretraining}
\vskip -0.1in
\end{figure*}

\begin{figure*}[t]
    \centering
    \subfigure[Training loss]
    {
    \includegraphics[width=.3\linewidth]{Figures/fix_gamma_loss_400M.png}
    \label{fig:fix_gamma_loss}
    }
    \subfigure[Loss in the final $5$B token interval]
    {
    \includegraphics[width=.3\linewidth]{Figures/fix_gamma_zoom_loss_400M.png}
    \label{fig:fix_gamma_loss_zoom}
    }
    \subfigure[Gradient-norm]
    {
    \includegraphics[width=.3\linewidth]{Figures/fix_gamma_gradnorm_400M.png}
    \label{fig:fix_gamma_gradnorm}
    }
    \caption{
    Freezing learnable parameter $\gamma$ of output normalization layer in Peri-LN. we set $\gamma$ to its initial value of $1$ and keep it fixed.
    }
    \label{fig:frozen_gamma}
\vskip -0.1in
\end{figure*}

\paragraph{Open-Sourced Peri-LN Models: Gemma$2$ \& OLMo$2$.}
Both Gemma$2$ and OLMo$2$, which apply output layer normalization, employ the same peri-normalization strategy within each Transformer layer. However, neither model rigorously examines how this placement constrains variance or mitigates large residual activations. Our work extends Gemma$2$ and OLMo$2$ by offering both theoretical and empirical perspectives within the Peri-LN scheme. Further discussion of the OLMo$2$ is provided in Appendix~\ref{appendix:olmo2}.

\subsection{Stability Analysis in Normalization Strategies}
\label{subsec:theory_insights}
We analyze training stability in terms of the magnitude of activation. To this end, we examine the gradient norm with respect to the weight of the final layer in the presence of massive activation. For the formal statements and detailed proofs, refer to Appendix~\ref{appendix:theory_proof}.

\begin{proposition}[Informal]
\label{prop:theory}
Let $\mathcal{L}(\cdot)$ be the loss function, and let $W^{(2)}$ denote the weight of the last layer of $\mathrm{MLP}(\cdot)$. Let $\gamma$ be the scaling parameter in $\mathrm{Norm}(\cdot)$, and let $D$ be the dimension. Then, the gradient norm for each normalization strategy behaves as follows.

\medskip
\noindent 
\textbf{(1) Pre-LN (exploding gradient).} Consider the following sequence of operations:
\begin{equation}
\tilde{x} = \mathrm{Norm}(x), a = \mathrm{MLP}(\tilde{x}), o = x + a,
\end{equation}
then
\begin{equation}
\left\lVert \frac{\partial \mathcal{L}(o)}{\partial W_{i,j}^{(2)}} \right\rVert \;\propto\; \| h_{i} \|,
\end{equation}
where $h := \mathrm{ReLU}\left(\tilde{x} W^{(1)} + b^{(1)}\right)$. In this case, when a massive activation $\|h\|$ occurs, an exploding gradient $\|\partial \mathcal{L} / \partial W^{(2)}\|$ can arise, leading to training instability.

\medskip
\noindent
\textbf{(2) Peri-LN (self-regularizing gradient).} Consider the following sequence of operations:
\begin{equation}
\tilde{x} = \mathrm{Norm}(x), a = \mathrm{MLP}(\tilde{x}), \tilde{a} = \mathrm{Norm}(a), o = x + \tilde{a},
\end{equation}
then
\begin{equation}
\left\lVert \frac{\partial \mathcal{L}(o)}{\partial W_{i,j}^{(2)}} \right\rVert 
\;\le\; \frac{4\,\gamma\,\sqrt{D}\,\|h\|}{\|a\|}, 
\end{equation}
where $h := \mathrm{ReLU}\left(\tilde{x} W^{(1)} + b^{(1)}\right)$. In this case, even when a massive activation $\|h\|$ occurs, $\mathrm{Norm}(\cdot)$ introduces a damping factor $\|a\|$, which ensures that the gradient norm $\|\partial \mathcal{L} / \partial W^{(2)}\|$ remains bounded.

\medskip
\noindent
\textbf{(3) Post-LN (vanishing gradient).} Consider the following sequence of operations:
\begin{equation}
a = \mathrm{MLP}(x), o = x + a, \tilde{o} = \mathrm{Norm}(o),
\end{equation}
then
\begin{equation}
\left\lVert \frac{\partial \mathcal{L}(\tilde{o})}{\partial W_{i,j}^{(2)}} \right\rVert 
\;\le\; \frac{4\,\gamma\,\sqrt{D}\,\|h\|}{\|x + a\|}, 
\end{equation}
where $h := \mathrm{ReLU}\left(x W^{(1)} + b^{(1)}\right)$. In this case, when a massive activation $\|h\|$ occurs, $\mathrm{Norm}(\cdot)$ introduces an overly suppressing factor $\|x+a\|$, which contains a separate huge residual signal $x$, potentially leading to a vanishing gradient $\|\partial \mathcal{L} / \partial W^{(2)}\|$.
\vskip -0.1in
\end{proposition}

We have compiled a Table~\ref{tab:variance_summary} that provides a overview of the variance and gradient intuition for each layer normalization strategy. %Intuitively, as $a$ grows large, the additional normalization steps help keep the gradient magnitude under control, thereby stabilizing training. This result sheds light on why Peri-LN may reduce the sensitivity to large intermediate activations compared to other LN placements. 


\section{\ourdata}
\label{sec:textbook-exam}

\begin{figure}[t]
    \centering
    \includegraphics[width=\linewidth]{figures/data_framework.pdf}    \caption{\textbf{Overview of  \ourdata curation}. Given a chapter $C$, we use an LM to segment the document $D$ into sections and heuristically extract review questions to form the exam $E$. The LM then classifies each question in $E$ by Bloom’s taxonomy category and maps it to its relevant section.}
    \label{fig:dataset-overview}
    \vspace{-0.3cm}
\end{figure}

In order to evaluate \ours, we curate  \ourdata, a dataset where each entry contains a document \(D\) along with a corresponding set of exam questions \(E\).
An overview of \ourdata is illustrated in \autoref{fig:dataset-overview}.
% In this section, we first describe our data processing pipeline (\secref{ssec:textbook-exam-pipeline}) and then provide data statistics (\secref{ssec:textbook-exam-statistics}).

\subsection{Data Processing}
\label{ssec:textbook-exam-pipeline}
Our pipeline starts with textbooks from the OpenStax repository\footnote{\url{https://github.com/philschatz/textbooks}}.
Each textbook is divided into chapters, where each chapter \(C\) contains learning objectives, main content, and review questions.
For each \(C\), we parse the main content to build \(D\) and the review questions to form \(E\).
% Specifically, only the main content is used to construct \(D\), while the review questions are used to create \(E\).

\paragraph{Extracting sections.}
To simulate a learner incrementally progressing through a chapter, we divide each chapter into sections using an LM-based document structuring method. 
The LM segments \( D \) into \( n \) sections, denoted as \( \{S_1, S_2, \ldots, S_n\} \subset D \), while also extracting the corresponding review questions \( E \). 
However, not all review questions come with ground-truth answers, as some textbooks do not provide them (see Table~\ref{tab:textbook-exam-statistics} for the proportion of \( E \) with answers). 
To ensure consistency in section segmentation across different subjects, we manually annotate the first 2–5 sections from one sample per subject. 
These annotated samples serve as few-shot examples in our LM prompt (see Appendix~\ref{appdx:parsing-sections} for details).

\paragraph{Extracting questions.}
% We developed a custom parsing script using BeautifulSoup4 to extract questions and their corresponding answers. 
To maintain a balance between evaluation depth and computational feasibility, we include only chapters that contain at least 10 questions—ensuring sufficient coverage for assessment—while capping the maximum number of questions at 25 to keep learning simulations computationally manageable.

\subsection{Data Statistics}
\label{ssec:textbook-exam-statistics}
\begin{table}[t!]
    \centering
    \resizebox{\columnwidth}{!}{
        \begin{tabular}{lccccc}
        \toprule
            \textbf{Subject} & \textbf{\# $C$} & \textbf{Split} & \textbf{\# $E$ / $C$} & \textbf{\% $E$ w/ answer} & \textbf{\# $S$ / $C$} \\
        \midrule
            Microbiology & 20 & Train & 12.4 & 64\% & 16.4 \\
                         & 5  & Test  & 13.4 & 58\% & 17.0 \\
        \midrule
            Chemistry    & 20 & Train & 14.2 & 51\% & 11.0 \\
                         & 5  & Test  & 16.2 & 49\% & 6.4 \\
        \midrule
            Economics    & 20 & Train & 12.2 & 23\% & 14.1 \\
                         & 5  & Test  & 12.2 & 23\% & 14.4 \\
        \midrule
            Sociology    & 20 & Train & 10.4 & 62\% & 16.6 \\
                         & 5  & Test  & 11.2 & 67\% & 19.0 \\
        \midrule
            US History   & 20 & Train & 7.2 & 51\% & 14.9 \\
                         & 5  & Test  & 8.4 & 38\% & 13.2 \\
        \bottomrule
        \end{tabular}
    }
    \caption{\textbf{Data Statistics} of \ourdata. \# $ C$: number of chapters, \# $E/C$: avg. number of questions per chapter, \% $E$ w/ answer: proportion of questions that have reference answer, \# $S/C$: avg. number of sections per chapter.}
    \label{tab:textbook-exam-statistics}
\end{table}
For each subject, we curate 25 sequential chapters \(C\), each containing both \(D\) and \(E\).
The chapters are arranged in their natural order, with the first 20 used for training and the last five reserved for evaluation.  
There is the risk that content in later chapters may include information from prior chapters (e.g., revisiting prerequisite knowledge). 
Therefore, preserving this sequential structure between the training and test set is essential for preventing information leakage and fairly assessing a model's learning process.
Table~\ref{tab:textbook-exam-statistics} shows an overview of the statistics of the resulting \ourdata.

\subsection{Distribution of Question Types}
\label{ssec:textbook-exam-bloom}

 \begin{figure}[t]
    \centering
    \includegraphics[width=\linewidth]{figures/bloom_taxonomy_counts_vertical.png}
    \caption{\textbf{Bloom's taxonomy distribution} in \ourdata. \ourdata consists of questions that require a wide variety of cognitive levels and the dominant categories vary for each subject.}
    \label{fig:bloom-distribution}
\end{figure}

To better understand how final exams assess a learner’s comprehension on multiple dimensions, we categorize questions in \ourdata\ based on the revised \textit{Bloom’s Taxonomy}~\cite{krathwohl2002revision_bloom}. 
Using an LM, we assign a cognitive depth \( d_j \) to each question \( E_j \in E \), classifying them into six categories: \textit{Remembering, Understanding, Applying, Analyzing, Evaluating}, and \textit{Creating}.
Additionally, we identify the relevant sections \( S_j \subset D \) that correspond to each question.

The distribution, shown in \autoref{fig:bloom-distribution}, indicates that different subjects emphasize different cognitive skills.
For instance, questions in Microbiology and Sociology primarily focus on \textit{Remembering} and \textit{Understanding}, whereas Chemistry and Economics exhibit a more varied distribution.
This analysis highlights the diverse cognitive demands across subjects and underscores how \ourdata\ provides a multifaceted evaluation of learning outcomes through final exams. 
For further details on data processing, refer to Appendix~\ref{appendix:data_processing}.













% \dongho{Number of textbooks?}

% \dongho{For each textbook, how may $D$?}



% \subsection{Validation}
% \dongho{Let's make a ground truth to see how reliable the data processing pipeline it is. -- for each textbook.}

% \paragraph{Validation of answer for $E_j$.}

% \paragraph{Validation of cognitive depth $d_j$ for $E_j$.}

% \paragraph{Validation of related sections $S_j$ for $E_j$.}
\section{Experiment}
In this section, we conduct extensive experiments to evaluate the performance of various LLMs on our Hellaswag-Pro benchmark. Our study is guided by three key research questions:
\textbf{RQ1}: How do different LLMs perform across all variants?
\textbf{RQ2}: What is the relative difficulty of different variants?
\textbf{RQ3}: How robust are LLMs to diverse prompts during evaluation?

\subsection{Experiment Setup} 
\subsubsection{Model Selection and Implementation Details}
We select 41 representative commercial and open-source models, including English LLMs, such as GPT-4o, Claude-3.5-Sonnet, Gemini-1.5-Pro,Mistral series, Llama3 series and Chinese LLMs, like Qwen-Max,  Qwen2.5 series, InternLM-2.5 series, Yi-1.5 series, Baichuan-2 series and DeepSeek series.

We integrate both Chinese HellaSwag and HellaSwagPro into the lm-evaluation-harness platform. For the open-source models, we use the default settings of lm-evaluation-harness: do\_sample is set to false and the temperature is set to the default value of the hugging-face library. For the closed-source models, we set the temperature to 0.7. In addition, we set the maximum output length to 1024.

\subsubsection{Prompt Strategy}
Taking into account the influence of language and shot, we design 9 prompting strategies, including Direct, CN-CoT, EN-CoT, CN-XLT and EN-XLT. The last four setups include both zero-shot and few-shot variants.\footnote{
For open-source models, Direct adopts an approach similar to the official implementation of HellaSwag, computing the log-likelihood for each option and selecting the one with the highest log-likelihood. And we report normalized accuracy that accounts for the impact of option length. Other prompting strategies use a generation setup and report accuracy based on exact match.}
\textbf {(1)Direct}: LLMs makes the selection directly without any CoT process.
\textbf{(2)CN-CoT}: LLMs performs CoT in Chinese, regardless of dataset language.
\textbf{(3)EN-CoT}: Similar to CN-CoT, but CoT is conducted in English. 
\textbf{(4)CN-XLT}: LLMs are instructed to first translate English questions and options to Chinese, and then reason in Chinese.
\textbf{(5)EN-XLT}: Similar to CN-XLT, but translates from Chinese dataset to English and reasons in English. 

%\textbf {CN-CoT}: LLMs perform Chinese reasoning and then output the answer and 3 shots are provided.
%\textbf {CN-CoT}: Similar as CNCoTFewShot without any shots.
%\textbf {EN-CoT}: The reasoning process in English is executed and then the answer is output and 3 shots are provided.
%\textbf {CN-XLT}: Inspired by this, we instruct LLMs to translate questions in Chinese and then output the answer after performing reasoning in Chinese too. And 3 shots are provided.
%\textbf {EN-XLT}: Inspired by this, we instruct LLMs to translate questions in Englsih and then output the answer after performing reasoning in Englsih too. Three shots are provided.

\subsubsection{Evaluation metric}

To comprehensively evaluate the robustness of each LLM, we consider four metrics: 
% Original Accuracy (\textbf{OA}), Average Robust Accuracy (\textbf{ARA}), Robust Loss Accuracy (\textbf{RLA}), and  Consistent Robust Accuracy (\textbf{CRA}).
\noindent %
\textbf{- Original Accuracy (OA)} measures accuracy on original problems.
\begin{equation}\label{eq1}
OA=\frac{\sum_{(x, y) \in D} \mathds{1}[L M(x), y]}{|D|}.
\end{equation}
\noindent %
\textbf{- Average Robust Accuracy  (ARA)} represents average accuracy across all variants, gauging overall performance on the robustness tasks.
\begin{equation}\label{eq2}
ARA=\frac{\sum_{\left(x^{\prime}, y^{\prime}\right) \in D_{R}} \mathds{1}\left(L M\left(x^{\prime}, y^{\prime}\right)\right.}{\left|D_{R}\right|}.
\end{equation}

\noindent %
\textbf{- Robust Loss Accuracy (RLA)} is the difference between ARA and OA, indicating performance degradation on robustness data versus original data.
%\begin{tiny}
%\begin{equation}\label{eq3}
%RLA=\frac{\sum_{\left(x^{\prime}, y^{\prime}\right) \in D_{R}} %\mathds{1}\left(L M\left(x^{\prime}, y^{\prime}\right)\right.}{\left|D_{R}\right|}-\frac{\sum_{(x, y) \in D}\mathds{1}[L M(x), y]}{|D|}
%\end{equation}
%\end{tiny}
\begin{equation}\label{eq3}
RLA= OA - ARA.
\end{equation}
\noindent %
\textbf{- Consistent Robust Accuracy (CRA)} shows accuracy when the model correctly answers both original and variant data, reflecting the model do understand the problem.
% consistency in problem-solving.
\begin{equation}\label{eq4}
CRA=\frac{\sum_{x, y, x^{\prime}, y^{\prime}}\mathds{1}[L M(x), y] \cdot \mathds{1}[L M(x^{\prime}), y^{\prime}]}{\left|D_{R}\right|}.
\end{equation}
For all equation above, $D$ denotes the original dataset, where $x$ represents the input question and options, and $y$ represents the correct label, while $D_{R}$ is the robust dataset with $x^{\prime}$ and $y^{\prime}$ representing similar to $x$ and $y$.


\begin{table*}[ht]
\centering
\setlength{\tabcolsep}{5pt}
% \footnotesize
\scalebox{0.6}{
% Please add the following required packages to your document preamble:
% \usepackage{multirow}
% \usepackage[table,xcdraw]{xcolor}
% Beamer presentation requires \usepackage{colortbl} instead of \usepackage[table,xcdraw]{xcolor}
% Please add the following required packages to your document preamble:
% \usepackage{multirow}
% \usepackage[table,xcdraw]{xcolor}
% Beamer presentation requires \usepackage{colortbl} instead of \usepackage[table,xcdraw]{xcolor}
\begin{tabular}{ccccccccccccc}
\hline
\multicolumn{1}{c|}{{ }}& \multicolumn{4}{c|}{Chinese}& \multicolumn{4}{c|}{English}& \multicolumn{4}{c}{AVG}\\ \cline{2-13} 
\multicolumn{1}{c|}{\multirow{-2}{*}{{ Model}}} & { OA(\%)$\uparrow$}& { ARA(\%)$\uparrow$} & {RLA(\%)$\downarrow$}& \multicolumn{1}{l|}{{CRA(\%)$\uparrow$}} & { OA(\%)$\uparrow$}& { ARA(\%)$\uparrow$} & { RLA(\%)$\downarrow$}& \multicolumn{1}{l|}{{CRA(\%)$\uparrow$}} & {OA(\%)$\uparrow$}& { ARA(\%)$\uparrow$} & {RLA(\%)$\downarrow$}& { CRA(\%)$\uparrow$} \\ \hline
\multicolumn{1}{c|}{{ Human}} & 96.41& 97.79& -1.38 & \multicolumn{1}{l|}{92.03}& 95.56& 96.04& -0.48 & \multicolumn{1}{l|}{90.02}& 95.99 & 96.92 & -0.93& 91.03 \\ \hline
\multicolumn{13}{c}{\textit{Close-source LLMs}}\\ 
\multicolumn{1}{c|}{{ GPT-4o}}& { 91.37} & { 81.97} & { 9.40}& \multicolumn{1}{l|}{{ 75.55}} & { \textbf{88.63}} & { \textbf{70.17}} & { \textbf{18.46}} & \multicolumn{1}{l|}{{ \textbf{63.06}}} & { 90.00} & { \textbf{76.07}} & { \textbf{13.93}} & { \textbf{69.31}} \\
\multicolumn{1}{c|}{{ Claude3.5}}& { \textbf{95.37}} & { 80.15} & { 15.22} & \multicolumn{1}{l|}{{ 75.04}} & { 85.11} & { 66.02} & { 19.08} & \multicolumn{1}{l|}{{ 57.20}} & { 90.24} & { 73.09} & { 17.15} & { 66.12} \\
\multicolumn{1}{c|}{{ Gemini-1.5-Pro}}& { 90.62} & { 78.36} & { 12.26} & \multicolumn{1}{l|}{{ 70.48}} & { 87.75} & { 60.74} & { 27.01} & \multicolumn{1}{l|}{{ 58.27}} & { 89.19} & { 69.55} & { 19.63} & { 64.38} \\
\multicolumn{1}{c|}{{ Qwen-Max}}& { 93.50} & { \textbf{84.82}} & { \textbf{8.68}}& \multicolumn{1}{l|}{{ \textbf{78.91}}} & { 87.60} & { 62.61} & { 24.99} & \multicolumn{1}{l|}{{ 59.65}} & { \textbf{90.55}} & { 73.72} & { 16.83} & { 69.28} \\ \hline
\multicolumn{13}{c}{\textit{Chinese open-source LLMs}} \\ 
\multicolumn{1}{c|}{{ Qwen2.5-0.5B}}& { 60.75} & { 45.18} & { \textbf{15.57}} & \multicolumn{1}{l|}{{ 28.70}} & { 49.50} & { 38.21} & { \textbf{11.29}} & \multicolumn{1}{l|}{{ 20.57}} & { 55.13} & { 41.70} & { \textbf{13.43}} & { 24.64} \\
\multicolumn{1}{c|}{{ Qwen2.5-1.5B}}& { 63.25} & { 46.16} & { 17.09} & \multicolumn{1}{l|}{{ 29.89}} & { 56.88} & { 39.57} & { 17.30} & \multicolumn{1}{l|}{{ 23.48}} & { 60.06} & { 42.87} & { 17.20} & { 26.69} \\
\multicolumn{1}{c|}{{ Qwen2.5-3B}}& { 67.50} & { 48.75} & { 18.75} & \multicolumn{1}{l|}{{ 33.79}} & { 61.75} & { 39.98} & { 21.77} & \multicolumn{1}{l|}{{ 25.75}} & { 64.63} & { 44.37} & { 20.26} & { 29.77} \\
\multicolumn{1}{c|}{{ Qwen2.5-7B}}& { 67.63} & { 50.59} & { 17.04} & \multicolumn{1}{l|}{{ 35.62}} & { 65.63} & { 43.93} & { 21.70} & \multicolumn{1}{l|}{{ 30.77}} & { 66.63} & { 47.26} & { 19.37} & { 33.20} \\
\multicolumn{1}{c|}{{ Qwen2.5-14B}} & { 69.00} & { 51.41} & { 17.59} & \multicolumn{1}{l|}{{ 35.84}} & { 68.50} & { 45.20} & { 23.30} & \multicolumn{1}{l|}{{ 32.12}} & { 68.75} & { 48.30} & { 20.45} & { 33.98} \\
\multicolumn{1}{c|}{{ Qwen2.5-32B}} & { 69.75} & { 53.11} & { 16.64} & \multicolumn{1}{l|}{{ 37.54}} & { 70.00} & { 46.10} & { 23.90} & \multicolumn{1}{l|}{{ 32.68}} & { 69.88} & { 49.61} & { 20.27} & { 35.11} \\
\multicolumn{1}{c|}{{ Qwen2.5-72B}} & { \textbf{70.87}} & { \textbf{54.75}} & { 16.12} & \multicolumn{1}{l|}{{ \textbf{39.64}}} & { \textbf{72.00}} & { \textbf{47.75}} & { 24.25} & \multicolumn{1}{l|}{{\textbf{ 35.12}}} & { \textbf{71.44}} & { \textbf{51.25}} & {20.19} & { \textbf{37.38}} \\ \hdashline[0.5pt/5pt]
\multicolumn{1}{c|}{{ Baichuan2-7B}}& { 67.00} & { 46.16} & { 20.84} & \multicolumn{1}{l|}{{ 31.50}} & { 60.62} & { 39.04} & { 21.58} & \multicolumn{1}{l|}{{ 25.21}} & { 63.81} & { 42.60} & { 21.21} & { 28.36} \\
\multicolumn{1}{c|}{{ Baichua2-13B}}& { 69.13} & { 46.98} & { 22.15} & \multicolumn{1}{l|}{{ 33.45}} & { 64.62} & { 38.82} & { 25.80} & \multicolumn{1}{l|}{{ 26.07}} & { 66.88} & { 42.90} & { 23.97} & { 29.76} \\ \hdashline[0.5pt/5pt]
\multicolumn{1}{c|}{{ DeepSeek-7B}} & { 68.13} & { 47.96} & { 20.17} & \multicolumn{1}{l|}{{ 33.30}} & { 63.38} & { 40.39} & { 22.99} & \multicolumn{1}{l|}{{ 26.70}} & { 65.76} & { 44.18} & { 21.58} & { 30.00} \\
\multicolumn{1}{c|}{{ DeepSeek-67B}}& { 71.50} & { 49.21} & { 22.29} & \multicolumn{1}{l|}{{ 35.89}} & { 71.37} & { 40.63} & { 30.75} & \multicolumn{1}{l|}{{ 29.71}} & { 71.44} & { 44.92} & { 26.52} & { 32.80} \\ \hdashline[0.5pt/5pt]
\multicolumn{1}{c|}{{ InternLM2.5-1.8B}}& { 61.62} & { 42.07} & { 19.55} & \multicolumn{1}{l|}{{ 26.99}} & { 55.37} & { 38.46} & { 16.91} & \multicolumn{1}{l|}{{ 22.61}} & { 58.50} & { 40.27} & { 18.23} & { 24.80} \\
\multicolumn{1}{c|}{{ InternLM2.5-7B}}& { 67.25} & { 49.77} & { 17.48} & \multicolumn{1}{l|}{{ 34.57}} & { 69.50} & { 40.89} & { 28.61} & \multicolumn{1}{l|}{{ 29.75}} & { 68.38} & { 45.33} & { 23.04} & { 32.16} \\
\multicolumn{1}{c|}{{ InternLM2.5-20B}} & { 67.37} & { 48.08} & { 19.29} & \multicolumn{1}{l|}{{ 33.21}} & { 73.62} & { 41.11} & { 32.51} & \multicolumn{1}{l|}{{ 31.23}} & { 70.50} & { 44.60} & { 25.90} & { 32.22} \\ \hdashline[0.5pt/5pt]
\multicolumn{1}{c|}{{ Yi-1.5-6B}} & { 67.00} & { 49.59} & { 17.41} & \multicolumn{1}{l|}{{ 34.27}} & { 64.38} & { 39.37} & { 25.01} & \multicolumn{1}{l|}{{ 26.62}} & { 65.69} & { 44.48} & { 21.21} & { 30.45} \\
\multicolumn{1}{c|}{{ Yi-1.5-9B}} & { 68.50} & { 50.18} & { 18.32} & \multicolumn{1}{l|}{{ 35.55}} & { 66.37} & { 39.58} & { 26.79} & \multicolumn{1}{l|}{{ 27.48}} & { 67.44} & { 44.88} & { 22.56} & { 31.52} \\
\multicolumn{1}{c|}{{ Yi-1.5-34B}}& { 71.00} & { 52.23} & { 18.77} & \multicolumn{1}{l|}{{ 38.09}} & { 71.00} & { 40.75} & { 30.25} & \multicolumn{1}{l|}{{ 29.91}} & { 71.00} & { 46.49} & { 24.51} & { 34.00} \\ \hline
\multicolumn{13}{c}{\textit{English open-source LLMs}} \\ 
\multicolumn{1}{c|}{{ Llama3-8B}} & { 59.13} & { 46.62} & { 12.51} & \multicolumn{1}{l|}{{ 28.23}} & { 66.25} & { 40.21} & { 26.04} & \multicolumn{1}{l|}{{ 27.34}} & { 62.69} & { 43.42} & { 19.27} & { 27.79} \\
\multicolumn{1}{c|}{{ Llama3-70B}}& { 65.75} & { 48.63} & { 17.12} & \multicolumn{1}{l|}{{ 32.70}} & { \textbf{72.50}} & { 41.27} & { 31.23} & \multicolumn{1}{l|}{{\textbf{ 30.63}}} & {\textbf{ 69.13}} & { 44.95} & { 24.18} & { 31.67} \\ \hdashline[0.5pt/5pt]
\multicolumn{1}{c|}{{ Mistral-7B-v0.2}} & { 57.75} & { 46.25} & { \textbf{11.50}} & \multicolumn{1}{l|}{{ 27.57}} & { 67.50} & { \textbf{41.52}} & { 25.98} & \multicolumn{1}{l|}{{ 28.93}} & { 62.63} & { 43.88} & { 18.74} & { 28.25} \\
\multicolumn{1}{c|}{{ Mixtral-8x7B-v0.1}} & { 63.62} & { 46.80} & { 16.82} & \multicolumn{1}{l|}{{ 30.82}} & { 69.75} & { 41.21} & { 28.54} & \multicolumn{1}{l|}{{ 29.39}} & { 66.69} & { 44.01} & { 22.68} & { 30.11} \\
\multicolumn{1}{c|}{{ Mixtral-8x22B-v0.1}}& { 66.00} & {\textbf{ 50.73}} & { 15.27} & \multicolumn{1}{l|}{{ \textbf{34.32}}} & { 72.12} & { 41.25} & { 30.87} & \multicolumn{1}{l|}{{ 30.61}} & { 69.06} & { \textbf{45.99}} & { 23.07} & { \textbf{32.47}} \\ \hdashline[0.5pt/5pt]
\multicolumn{1}{c|}{{ Gemma-2-2B}}& { 61.88} & { 45.38} & { 16.51} & \multicolumn{1}{l|}{{ 29.02}} & { 59.62} & { 39.13} & { \textbf{20.50}} & \multicolumn{1}{l|}{{ 24.88}} & { 60.75} & { 42.25} & {\textbf{ 18.50}} & { 26.95} \\
\multicolumn{1}{c|}{{ Gemma-2-9B}}& { \textbf{69.13}} & { 46.75} & { 22.38} & \multicolumn{1}{l|}{{ 33.29}} & { 64.88} & { 39.80} & { 25.08} & \multicolumn{1}{l|}{{ 26.91}} & { 67.01} & { 43.28} & { 23.73} & { 30.10} \\
\multicolumn{1}{c|}{{ Gemma-2-27B}} & { 63.38} & { 48.52} & { 14.86} & \multicolumn{1}{l|}{{ 31.96}} & { 71.88} & { 40.91} & { 30.97} & \multicolumn{1}{l|}{{ 30.25}} & { 67.63} & { 44.71} & { 22.92} & { 31.11} \\ \hline
\end{tabular}
}
\caption{TODO: bolded is not result. Results of existing LLMs on our HellaSwag-Pro dataset using \textbf{Direct} prompt. ``AVG'' indicates the average performance of each model on Chinese and English parts of the dataset.
The best results for each metric in each model category are \textbf{bolded}. }
\label{tab:main experiment.}
\end{table*}

\subsection{Model Performance (RQ1)}
\paragraph{Overall Performance}
Table \ref{tab:main experiment.} provides a comprehensive evaluation of various LLMs across four performance metrics\footnote{The results of instruct/chat models of Qwen2.5, Llama3 and Mixtral latest series are shown in Appendix.}. The main observations are as follow:
\begin{itemize}[leftmargin=*,topsep=0pt]
% \setlength{}{0}
    \item Upon evaluating all available models, we found that all performed well in overall accuracy (e.g., GPT-4 scored 90.00 in AVG OA, Claude 3.5 scored 90.24 in AVG OA). However, all models struggled with variations of the questions, as evidenced by a positive RLA value for each model. In contrast, humans received a negative RLA value, suggesting that the question variants were not more challenging than the originals. This disparity further illustrates that current LLMs lack a true understanding of the reasoning process and can easily be misled by question variants.
    \item When comparing open-source and close-source models, the close-source models demonstrate stronger capabilities in both OA and ARA scores, similar to most existing benchmarks. Overall, the RLA values for close-source models are also smaller, indicating that they are more robust in commonsense reasoning tasks compared to open-source models.
    \item When we compare models within the same series (e.g., Qwen, Llama), we observe that larger models often achieve higher scores on OA, ARA, and CRA. However, they are also more susceptible to variations, i.e., they have higher RLA values, a phenomenon particularly evident in English datasets. We attribute this phenomenon to the fact that larger models, compared to smaller ones, may have memorized more data, allowing them to rely on memorization to solve some problems more easily and making them more prone to the influence of variations~\cite{}.
\end{itemize}
% 1. When evaluating all available models, We find although 
% 2. When comparing the opensource LLMs and close source LLMs, 
% 3. When looking into each serious details
% \noindent
% \textbf{Overall Model Performance.}
% 1. close-source > open-source 2. the large the better 3. all have a performance decline when meeting varients.

% To evaluate the performance of various models, we observed patterns consistent with current mainstream trends: closed-source models generally outperform open-source models across metrics. 
% For instance, the closed-source model GPT-4o achieved scores of 90.00 in OA, 76.07 in ARA, and 69.31 in CRA, whereas the open-source model Qwen2.5-72B scored 71.44, 51.25, and 37.38, respectively. 
% Furthermore, within each model series, performance tends to improve with larger model sizes. 
% Nevertheless, even the strongest closed-source models struggle with variations in questions, as indicated by positive values in RLA for all models. In contrast, human performance yields a negative RLA value, highlighting that current LLMs do not genuinely grasp the reasoning process and are prone to falling into traps set by question variants. 
% This suggests that there is still significant room for improvement in developing models that can robustly understand and reason through complex linguistic challenges.
% It reveals a consistent pattern across Chinese, English, and average scores, with close-sourced LLMs generally outperforming open-sourced models. 
% However, all models exhibit a significant drop in performance when faced with robust variants, as indicated by RLA and CRA. Among closed-source models, GPT-4o demonstrates the highest ARA of 76.07\% in average scores, demonstrating its overwhelming superiority. Among open-sourced models, larger models tend to perform better, with Qwen2.5-72B achieving the highest OA (71.44\%) and ARA (51.25\%) in the average scores. However, even these top performers still struggle with robustness, as evidenced by the substantial RLA of 13.93\% for GPT-4o and 20.19\% for Qwen2.5-72B. Interestingly, some English open-sourced models, such as Llama3-70B and Mixtral-8x22B-v0.1, show competitive performance in English tasks but lag in Chinese tasks, highlighting the importance of language-specific training.

% \noindent
% \textbf{Chinese Models vs English Models.}
% Chinese models generally demonstrate higher OA in Chinese tasks compared to English tasks, with Qwen-Max achieving 93.50\% OA in Chinese versus 87.60\% in English. Conversely, English models tend to perform better in English tasks, exemplified by Llama3-70B's 72.50\% OA in English compared to 65.75\% in Chinese. 
% However, both Chinese and English models exhibit important drops in ARA across languages, indicating challenges in maintaining performance when faced with variations. This trend suggests that while models may excel in their primary language, they struggle with robustness across linguistic boundaries. 
% Notably, larger models tend to achieve higher ARA scores but also experience more substantial RLA, as seen with Qwen2.5-0.5B (41.70\% ARA, 13.43\% RLA in total) and Qwen2.5-72B (51.25\% ARA, 20.19\% RLA in total). 
% This pattern indicates that while increased model size enhances overall performance, it doesn't necessarily improve robustness proportionally. 
% The discrepancy between OA and ARA across languages underscores the need for improved cross-lingual robustness in language models, particularly as they scale in size and capability.


% \noindent
% \textbf{Comparison between Chinese and English datasets.}
% Generally, models demonstrate higher accuracy on the Chinese dataset compared to the English one, as evidenced by the consistently higher OA, ARA and CRA scores. For instance, GPT-4o achieves an OA of 91.37\%, an ARA of 81.97\% , an CRA of 75.55\% on the Chinese dataset, compared to 88.63\% and 70.17\% respectively on the English dataset. This trend is observed across most models, suggesting that the Chinese dataset is easier than English one. Moreover, the RLA values are typically lower for Chinese, indicating smaller performance drops when dealing with robust variants of Chinese questions. For example, Qwen-Max shows an RLA of 8.68\% for Chinese versus 24.99\% for English, highlighting a more consistent performance in Chinese. The CRA scores further reinforce this observation, with models generally maintaining higher consistency in correct answers for both original and variant Chinese questions.
% We attribute this phenomenon to the fact that blablabla

\noindent
\textbf{Reasoning Transferable Capability.}
% 为了进一步
To further analyze whether the model can transfer reasoning ability from the original question to its variant, Figure \ref{consis} presents the distribution of model performance on the original question and variant pairs. For all models, the pairs of (HellaSwag \ding{51} HellaSwag-Pro \ding{55}) occupy a significant proportion, indicating a challenge in transferring reasoning capabilities for current LLMs to more complex scenarios. Looking deeply, closed-source models like GPT-4 and Qwen-Max achieve around a 69\% portion of (HellaSwag \ding{51} HellaSwag-Pro \ding{51}) and a 3\% portion of (HellaSwag \ding{55} HellaSwag-Pro \ding{55}), while in contrast, open-source models struggle with around a 30\% portion of (HellaSwag \ding{51} HellaSwag-Pro \ding{51}) and a 20\% portion of (HellaSwag \ding{55} HellaSwag-Pro \ding{55}), further indicating the robustness of reasoning abilities in closed-source models.
% If a model can get both the original question and the variant right, we consider it to have transferable reasoning ability. Table \ref{consis} presents the distribution of model performance on the original question and variant pairs. Among all models, the pairs of (HellaSwag \ding{51}HellaSwag-Pro \ding{55}) account for a considerable proportion, i 
% The closed-source models like GPT-4o and Qwen-Max achieve around 69\% portion of (HellaSwag \ding{51}HellaSwag-Pro \ding{51}) and 3\% portion of (HellaSwag \ding{55} HellaSwag-Pro \ding{55}), indicating stronger reasoning transfer ability than other models. In contrast, open-source models struggle more, with around 30\% portion of (HellaSwag \ding{51}HellaSwag-Pro \ding{51}) and 20\% portion of (HellaSwag \ding{55} HellaSwag-Pro \ding{55}). 
% A notable trend is observed among the Qwen2.5 series, where increasing model size from 7B to 72B parameters correlates with improved performance on correct answers for both datasets (33.20\% to 37.38\%) and decreased failure rates (17.69\% to 14.7\%). It underscores the importance of model size in commonsense reasoning tasks.

\begin{figure}[t]
\centering
\setlength{\abovecaptionskip}{0.1cm}
\setlength{\belowcaptionskip}{0cm}
\includegraphics[width=\linewidth,scale=1.00]{images/consis.pdf}
\caption{Analysis of the transferable ability of model reasoning based on question pair performance. The green part, where both the original and the variant data are right, represents the transferable performance of model reasoning.}
\label{consis}
\vspace{-15pt}
\end{figure}

\begin{figure*}[ht]
\centering
\setlength{\abovecaptionskip}{0.1cm}
\setlength{\belowcaptionskip}{0cm}
\includegraphics[width=\linewidth,scale=1.00]{images/xing.pdf}
\caption{The impact of different few-shot prompts on model performance. With - as the separator, the first two parts of the legend represent the prompt name, and the third part represents the language of the dataset.}
\label{xing}
\vspace{-15pt}
\end{figure*}

\begin{figure}[ht]
\centering
\setlength{\abovecaptionskip}{0.1cm}
\setlength{\belowcaptionskip}{0cm}
\includegraphics[width=1.05\linewidth,scale=1.05]{images/zhu.pdf}
\caption{The RLA Distribution for 7 variants of commonsense reasoning. Parts below the 0 axis indicate that the model’s performance on the variant is improved compared to the original problem.}
\label{fig:zhu}
\vspace{-15pt}
\end{figure}


\subsection{Variant Analysis (RQ2)}
To further analyze the impact of different variants, we assessed the contribution of each variant to the RLA score. A higher contribution indicates that the model is more likely to make errors in that type. Figure~\ref{fig:zhu} presents the overall results, and the key observations are as follows:
\begin{itemize}[leftmargin=*]
    \item For problem restatement, causal inference, and sentence ordering, these three categories are the least challenging. Almost all models, particularly the close-source and Qwen series models, perform well on these variants, indicating that current LLMs can effectively handle these forms and we do not pay more attention on this kind of varients.
    \item For reverse conversion and critical testing, these two varients each contribute about 10\% to the RLA score. This indicates that current LLMs struggle to fully generalize to these simple scenarios, possibly because these types of questions are not commonly encountered, and reaserchers should pay some attention to this type of varients.
    \item For negative transformation and scenario refinement, this are the two most difficult tasks, with negative transformation being particularly challenging. For almost all models, these two varients accounts for more than 50\% of the RLA score. This may be due to intuitively counterintuitive questions—such as the use of "will not"  or counterfactual scenarios in scenario refinement. These setups are less common in LLM training data and cannot be easily tackled through memory alone. Only those LLMs which truely understand the question could answer the varient correctly, wihch better reflect the true performance of the model.. In the future, researchers should focus more on enhancing LLM's capability to address such types of questions.
\end{itemize}

% 1. Problem restCausal Inference 
% To further analysis the impact of different varients, we further 
% Figure \ref{fig: zhu} presents a comprehensive analysis of various LLMs' performance across different variant types. Negative transformation emerges as the most challenging task for all models, with scores consistently above 50.00\% and peaking at 78.38\% for Gemini-1.5-Pro. Conversely, problem restatement appears to be the least challenging, with most models scoring in the negative range. Intriguingly, smaller models like Qwen2.5-0.5B demonstrate unexpected strengths in certain areas, such as sentence sorting (7.75\%), outperforming some larger counterparts. A detailed analysis of each variant type follows.

% \noindent
% \textbf{Causal inference.} In this category, scores vary widely from -4.73\% for Qwen-Max to 12.25\% for Baichuan2-13B, illustrating differing degrees of sensitivity to causal reasoning among the models. Smaller models, such as Qwen2.5-0.5B and Qwen2.5-1.5B, achieve better scores, indicating relatively stronger robustness in causal reasoning. Conversely, larger models, like Baichuan2-13B, have higher scores, suggesting greater sensitivity to the challenges of inferring causality.

% \noindent
% \textbf{Critical testing.} Larger models, including Qwen2.5-72B and DeepSeek-67B, exhibit higher RLA scores of 30.50\% and 31.37\%, respectively, suggesting increased sensitivity when dealing with incomplete key information. In contrast, GPT-4o achieves the lowest score, highlighting its superior robustness in critical reasoning. This trend indicates that more complex models might struggle to handle incomplete contexts, underscoring potential areas for improvement in sophisticated architectures.

% \noindent
% \textbf{Negative transformation.} This aspect remains consistently challenging for all models, with scores ranging from 48.88\% to 78.38\%. Advanced commercial models like Gemini-1.5-Pro and Claude-3.5 also score higher (78.38\% and 76.43\%, respectively), indicating a prevalent sensitivity issue in reasoning processes when handling negations, irrespective of model size or architecture.

% \noindent
% \textbf{Problem restatement.} The negative values in this category for nearly all models suggest it is not particularly challenging. This is surprising, given that previous models were quite sensitive to sentence representation.

% \noindent
% \textbf{Reverse conversion.} This variation, which involves swapping the roles of the question and answer, seems to specifically impact larger models. For example, Qwen2.5-72B and DeepSeek-67B exhibit higher RLA scores of 24.38\% and 27.43\%, respectively, indicating heightened sensitivity to reverse reasoning compared to their performance on original questions.

% \noindent
% \textbf{Scenario refinement.} The scores range from 16.06\% for Gemma-2-2B to 32.56\% for Qwen2.5-72B, with larger models displaying more sensitivity in adapting to counterfactual predictions. This suggests that larger models may rely more heavily on general commonsense rather than flexibly adapting to specific contexts. Consequently, increased model complexity might adversely affect adaptability to scenario changes, underscoring the need for enhanced flexibility in advanced models.

% \noindent
% \textbf{Sentence sorting.} This category exhibits the most varied results across models. Some larger models like DeepSeek-67B and InternLM2.5-20B display higher scores (26.69\% and 26.68\%), indicating sensitivity, while others like Qwen2.5-72B and Gemini-1.5-Pro excel with lower scores (-9.88\% and -1.07\%, respectively). This suggests that sentence sorting ability may depend more on specific training approaches rather than being solely contingent on model size.


\subsection{Prompt Robustness (RQ3)}
% To investigate how prompt  influence our benchmark, we apply sereral prompt strategy on our datasets and showcase the average performance of all models on different kind of prompt strategies.
% Table~\ref{prompt} illustrates the final results. For both Chinese and English datasets, CN LLMs achieve the highest performance using CN-CoT-Few-Shot, followed closely by EN-CoT-Few-Shot, with overall performance scores of 67.36\% and 67.03\%, respectively. In contrast, English LLMs perform best with the EN-CoT-Few-Shot, reaching 67.55\% on the Chinese dataset and 60.36\% on the English dataset.
% Contrary to previous findings, translating the dataset to the model's advantage language before performing reasoning does not enhance performance. Moreover, Figure~\ref{xing} also shows the similar phenomenon. Conducting CoT reasoning in the model’s advantage language generally leads to better outcomes compared to Direct. Additionally, increasing the number of shots consistently improves performance across most configurations, highlighting the benefits of exposing models to multiple examples. 
To explore the impact of various prompt strategies on our benchmarks, we evaluated several approaches across our datasets and present the average performance of all models using different prompting techniques. Table~\ref{prompt} summarizes the results. For both Chinese and English datasets, Chinese LLMs performed best with the CN-CoT-Few-Shot strategy, followed closely by EN-CoT-Few-Shot, achieving overall scores of 67.36\% and 67.03\%, respectively. Conversely, English LLMs showed optimal performance with the EN-CoT-Few-Shot approach, attaining 67.55\% on the Chinese dataset and 60.36\% on the English dataset.
Besides, translating datasets into the model's native language before reasoning did not enhance performance. This phenomenon is further illustrated in Figure~\ref{xing}. Conducting CoT reasoning in the model's native language generally yields better results compared to direct reasoning. Furthermore, increasing the number of examples (shots) consistently boosts performance across most configurations, emphasizing the advantages of exposing models to multiple examples.
% Overall, the interaction between question language, prompt language, and the number of shots underscores the importance of aligning these factors to optimize task performance and robustness in LLMs.



% Please add the following required packages to your document preamble:
% \usepackage{multirow}
% Please add the following required packages to your document preamble:
% \usepackage{multirow}
\begin{table}[t]
\setlength{\tabcolsep}{8pt}
% \footnotesize
\scalebox{0.65}{
\begin{tabular}{c|l|lll}
\hline
\multicolumn{1}{l|}{Dataset}  & Prompt  & CN LLMs & EN LLMs &  LLMs \\ \hline
\multirow{7}{*}{\begin{tabular}[c]{@{}c@{}}Chinese\\ HellaSwag-Pro\end{tabular}} & Direct  & 48.95& 41.16& 45.06  \\
& CN-CoT-Few  & \textbf{71.04}& 51.90& 61.47  \\
& EN-CoT-Few  & 70.95& \textbf{67.55}& \textbf{69.25}  \\
& EN-XLT-Few  & 41.48& 28.69& 35.09  \\
& CN-CoT-Zero & 44.82& 23.89& 34.36  \\
& EN-CoT-Zero & 45.38& 31.39& 38.39  \\
& EN-XLT-Zero & 28.57& 12.93& 20.75  \\ \hline
\multirow{7}{*}{\begin{tabular}[c]{@{}c@{}}English\\ HellaSwag-Pro\end{tabular}} & Direct  & 47.46& 40.66& 44.06  \\
& CN-CoT-Few  & \textbf{63.67}& 47.24& 55.46  \\
& EN-CoT-Few  & 63.12& \textbf{60.36}& \textbf{61.74}  \\
& CN-XLT-Few  & 48.77& 16.61& 32.69  \\
& CN-CoT-Zero & 34.89& 18.25& 26.57  \\
& EN-CoT-Zero & 42.41& 31.03& 36.72  \\
& CN-XLT-Zero & 16.36& 11.22& 13.79  \\ \hline
\multirow{9}{*}{HellaSwag-Pro}& Direct  & 48.21& 40.91& 44.83  \\
& CN-CoT-Few  & \textbf{67.36}& 49.57& 58.46  \\
& EN-CoT-Few  & 67.03& \textbf{63.95}& \textbf{65.49}  \\
& CN-XLT-Few  & 59.91& 34.26& 47.08  \\
& EN-XLT-Few  & 52.30& 44.52& 48.41  \\
& CN-CoT-Zero & 39.86& 21.07& 30.46  \\
& EN-CoT-Zero & 43.90& 31.21& 37.55  \\
& CN-XLT-Zero & 30.59& 17.55& 24.07  \\
& EN-XLT-Zero & 35.49& 21.98& 28.74  \\ \hline
\end{tabular}
}
\caption{Average ARA of all open-source models on different prompts. CN-LLMs contains 17 LLMs, and EN-LLMs contains 7 LLMs. The bast results for each dataset are \textbf{bolded}.}
\label{prompt}
\end{table}




\section{Concluding Remarks}
In this paper, we proposed a novel approach utilizing multimodal LLMs to generate gesture-aware speech recognition transcripts for patients with language disorders. Our framework integrates verbal speech and iconic gestures, enabling the generation of enriched transcripts that capture the latent meaning conveyed through both modalities. Through extensive experimentation, we demonstrated that the proposed method effectively contextualizes incomplete or disfluent speech by incorporating gesture information, leading to more accurate and meaningful representations of the speaker's intent. These findings highlight the potential of our approach to significantly contribute to the field of speech and language therapy, offering innovative tools that can enhance the quality of life for individuals with language disorders by facilitating better communication and assessment methods.

\subsection{Ethical Statement} 
Our dataset was obtained from AphasiaBank with the approval of the Institutional Review Board (IRB) and adheres to the data sharing guidelines set by TalkBank\footnote{https://talkbank.org/share/ethics.html}. This includes complying with the Ground Rules for all TalkBank databases, which are based on the American Psychological Association Code of Ethics~\cite{american2002ethical}.

\subsection{Limitation \& Future Work} 
%This study represents a preliminary investigation into using multimodal LLMs to generate gesture-aware speech recognition transcripts. 
While the results are promising, we recognize several limitations and outline our plans to extend this work further.

One primary limitation is the absence of a definitive ground truth for quantitative evaluation. Since our model generates transcripts by synthesizing speech and gesture data from scratch, traditional benchmarks, such as comparisons with standard speech recognition outputs, are insufficient. Moreover, existing original transcripts lack gesture annotations, making direct comparisons challenging. In future work, we aim to address this gap by collaborating with certified pathologists to conduct qualitative assessments, such as A-B preference tests, to evaluate the effectiveness of gesture-enriched transcripts in accurately conveying the speaker's intentions.

To support quantitative evaluations, we plan to develop novel metrics that assess transcript quality, including grammar accuracy, semantic consistency, and the integration of multimodal information. Such metrics will provide a more objective basis for assessing our model's performance and facilitate comparisons with other multimodal and unimodal approaches.

Another limitation of this study is its focus on structured gestures from a specific task, the Peanut Butter Sandwich Task. While this task offers a controlled context for testing our approach, it does not encompass the diversity of gestures and communication patterns seen in everyday scenarios. As part of our future work, we plan to expand the scope of our model to include tasks such as the Cinderella Story Recall Task~\cite{bird1996cinderella}, which involves unstructured and complex narrative gestures. This expansion will allow us to evaluate the adaptability and robustness of our model in handling varied linguistic and gestural contexts.

In summary, while this study establishes a strong foundation for gesture-aware speech recognition, we aim to refine and extend our methods through collaborative qualitative evaluations, the development of robust quantitative metrics, and broader task applications. These efforts will ensure that our approach continues to evolve, ultimately contributing to more effective communication tools and interventions for individuals with language disorders.






\paragraph{Acknowledgements. } 
This research is supported by EPSRC Programme Grant VisualAI EP$\slash$T028572$\slash$1, a Royal Society
Research Professorship RP$\backslash$R1$\backslash$191132, a China Oxford Scholarship and the Hong Kong Research Grants Council -- General
Research Fund (Grant No.: 17211024).
We thank Minghao Chen, Jindong Gu, Zhongrui Gui, Zhenqi He, João Henriques, Zeren Jiang, Zihang Lai, Horace Lee, Kun-Yu Lin, Xianzheng Ma, Christian Rupprecht, Ashish Thandavan, Jianyuan Wang, Kaiyan Zhang, Chuanxia Zheng and Liang Zheng for their help and support for the project.




{
    \small
    \bibliographystyle{ieeenat_fullname}
    \bibliography{main}
}


\clearpage
\newpage
\appendix
\renewcommand{\thetable}{\thesection.\arabic{table}}
\renewcommand\thefigure{\thesection.\arabic{figure}}    

%-------------------------------------------------------------------------
\section{Proofs of Theoretical Analysis}
\label{Append:proof}
Before diving into the theoretical analysis proof, we'll briefly introduce the basic concepts used in our proof to ensure the paper is self-contained.

\begin{definition}[Lipschitz continuity] $f$ is $\lambda$-Lipschitz if for any two points $u,v$ in the domain of $f$, we have following inequality:
\begin{align}
    |f(u)-f(v)|\leq \lambda ||u-v||
\end{align}
\end{definition}
%-------------------------------------------------------------------------
\subsection{Difference between inter-task affinity and proximal inter-task affinity}
\label{Append:differeence_affinity}
Firstly, let's reiterate the definitions of inter-task affinity and proximal inter-task affinity from the main paper.

In a typical SGD process for task $i$ at time step $t$ with input $z^t$, the update rule for $\Theta_s$ is as follows: $\Theta_{s|i}^{t+1} = \Theta_s^t-\eta w_i \nabla_{\Theta_s^t} \mathcal{L}_i(z^t, \Theta_s^t, \Theta_i^t)$ where $z^t$ represents the input data and $\eta$ is the learning rate, $\Theta_{s|i}^{t+1}$ is the updated shared parameters with loss $\mathcal{L}_i$. Then the affinity from task $i$ to $k$ at time step $t$, denoted as $\mathcal{A}^t_{i\rightarrow k}$, is:
\begin{align}
    \mathcal{A}^t_{\textcolor{red}{i\rightarrow k}} &= 1- \frac{\mathcal{L}_k(z^t, \Theta_{s|i}^{t+1}, \Theta_k^{\textcolor{red}{t}})}{\mathcal{L}_k(z^t, \Theta_{s}^{t}, \Theta_k^t)}
    \label{append:definition:inter_task_affinity}
\end{align}

For proximal inter-task affinity, let's consider a multi-task network shared by the task set $G$, with their respective losses defined as $\mathcal{L}_G$. For a data sample $z^t$ and a learning rate $\eta$, the gradients of task set $G$ are updated to the parameters of the network as follows: $\Theta_{s|G}^{t+1} = \Theta_s^t -\eta \nabla_{\Theta_s^t} \mathcal{L}_G (z^t, \Theta_s^t, \Theta_G^t)$ and $\Theta_k^{t+1} = \Theta_k^t -\eta \nabla_{\Theta_k^t} \mathcal{L}_k (z^t, \Theta_s^t, \Theta_k^t)$ for $k \in G$. Then, the proximal inter-task affinity from group $G$ to task $k$ at time step $t$ is defined as:
\begin{align}
    \mathcal{B}^t_{\textcolor{red}{G\rightarrow k}} = 1- \frac{\mathcal{L}_k(z^t, \Theta_{s|\textcolor{red}{G}}^{t+1}, \Theta_k^{\textcolor{red}{t+1}})}{\mathcal{L}_k(z^t, \Theta_{s}^{t}, \Theta_k^t)}
    \label{append:definition:proximal_inter_task_affinity}
\end{align}

The primary distinction between the two affinities lies in the incorporation of the task set and the update of task-specific parameters (indicated by \textcolor{red}{red letters}). Proximal inter-task affinity is an expanded concept that integrates the task set rather than individual tasks as the source task. This difference is evident from the notation, where $i \rightarrow k$ in \Cref{append:definition:inter_task_affinity} and $G \rightarrow k$ in \Cref{append:definition:proximal_inter_task_affinity}.

The second main difference lies in the update of task-specific parameters. In the inter-task affinity in \Cref{append:definition:inter_task_affinity}, the denominator includes $\Theta_k^t$, while in the proximal inter-task affinity in \Cref{append:definition:proximal_inter_task_affinity}, it includes $\Theta_k^{t+1}$, which is a subtle distinction that may not be noticed by readers.
These two modifications allow us to track proximal inter-task affinity while simultaneously optimizing multi-task networks.

When measuring affinity under the assumption of a convex objective, the proximal inter-task affinity is equal to or greater than the inter-task affinity. This also aligns well with real-world scenarios, as proximal inter-task affinity reflects updates to task-specific parameters, as shown below.
\begin{align}
    \mathcal{A}^t_{i\rightarrow k} &= 1- \frac{\mathcal{L}_k(z^t, \Theta_{s|i}^{t+1}, \Theta_k^t)}{\mathcal{L}_k(z^t, \Theta_{s}^{t}, \Theta_k^t)} \leq 1- \frac{\mathcal{L}_k(z^t, \Theta_{s|i}^{t+1}, \Theta_k^{t+1})}{\mathcal{L}_k(z^t, \Theta_{s}^{t}, \Theta_k^t)} = \mathcal{B}^t_{i\rightarrow k}
\end{align}

This inequality can also be applied to an expanded setting that incorporates task sets. If we expand the concept of inter-task affinity from individual task to task set as $\mathcal{A}^t_{G \rightarrow k}$, then the inequality $\mathcal{A}^t_{G \rightarrow k} \leq \mathcal{B}^t_{G \rightarrow k}$ is satisfied. If $k \notin G$ then, $\mathcal{A}^t_{G \rightarrow k} = \mathcal{B}^t_{G \rightarrow k}$ holds.

For ease of notation, we use the expanded version of the affinity for multiple tasks throughout the proof, as follows:
\begin{align}
    \mathcal{A}^t_{G\rightarrow k} &= 1- \frac{\mathcal{L}_k(z^t, \Theta_{s|G}^{t+1}, \Theta_k^t)}{\mathcal{L}_k(z^t, \Theta_{s}^{t}, \Theta_k^t)}
    \label{Append:expanded_affinity}
\end{align}
This differs from proximal inter-task affinity, as it does not consider the update of task-specific parameters.

%-------------------------------------------------------------------------
\subsection{Proof of \Cref{theorem1}}
\label{Append:theorem1}

\theomone*
\begin{proof}
Let's consider a scenario where we update the network parameters $\Theta$ with task-specific losses $\mathcal{L}_i$ and $\mathcal{L}_k$ simultaneously at time step $t$ with input $z^t$. Applying the Taylor expansion, we obtain the following:
\begin{align}
    \mathcal{L}_k(z^t, \Theta_{s|i,k}^{t+1}, \Theta_k^{t})
    &\simeq \mathcal{L}_k (z^t, \Theta_s^t, \Theta_k^t) + (\Theta_{s|i,k}^{t+1} - \Theta_s^t) \nabla_{\Theta_s^t} \mathcal{L}_k (z^t, \Theta_s^t, \Theta_k^t) + O(\eta^2) \\
    &= \mathcal{L}_k (z^t, \Theta_{s}^{t}, \Theta_k^{t})-\eta g_k\cdot(g_i+g_k) + O(\eta^2)
\end{align}
where $g_i$ and $g_k$ represent the gradients backpropagated from the losses $\mathcal{L}_i$ and $\mathcal{L}_k$, respectively, with respect to the shared parameters $\Theta_s^t$. For instance, $g_i = \nabla_{\Theta_s^t} \mathcal{L}_i (z^t, \Theta_s^t, \Theta_i)$.

Reorganizing the inequality to align with the format of inter-task affinity, we obtain:  
\begin{align}
    \mathcal{A}_{i,k\rightarrow k}^t = 1-\frac{\mathcal{L}_k(z^t, \Theta_{s|i,k}^{t+1}, \Theta_k^{t})}{\mathcal{L}_k (z^t, \Theta_{s}^t, \Theta_k^{t})} \simeq \frac{1}{\mathcal{L}_k (z^t, \Theta_{s}^t, \Theta_k^{t})}\biggl(\eta g_k\cdot(g_i+g_k) + O(\eta^2)\biggr)
\end{align}
Similar results can be obtained for $A_{j,k\rightarrow k}^t$.
\begin{align}
    \mathcal{A}_{j,k\rightarrow k}^t = 1-\frac{\mathcal{L}_k(z^t, \Theta_{s|j,k}^{t+1}, \Theta_k^{t})}{\mathcal{L}_k (z^t, \Theta_{s}^t, \Theta_k^{t})} \simeq \frac{1}{\mathcal{L}_k (z^t, \Theta_{s}^t, \Theta_k^{t})}\biggl(\eta g_k\cdot(g_j+g_k) + O(\eta^2)\biggr)
\end{align}
From $\mathcal{A}_{i,k \rightarrow k}^t \geq \mathcal{A}_{j,k \rightarrow k}^t$ and by ignoring the $O(\eta^2)$ term with a sufficiently small learning rate $\eta \ll 1$, we can derive the result:
\begin{align}
    g_i \cdot g_k \geq g_j \cdot g_k
\end{align}
\end{proof}
The findings indicate that grouping tasks with positive inter-task affinity exhibits better alignment in task-specific gradients compared to grouping tasks with negative inter-task affinity, thereby validating the grouping strategies employed by our algorithm. Furthermore, we analyze how this alignment in task-specific gradients contributes to reducing the loss of task $k$ in \Cref{theorem2}.


%-------------------------------------------------------------------------
\subsection{Proof of \Cref{theorem2}}
\label{Append:theorem2}

\theomtwo*
\begin{proof}
Let's consider a scenario where we update the network parameters $\Theta_s^t$ with task-specific losses $\mathcal{L}_i$ and $\mathcal{L}_k$ simultaneously at time step $t$ with input $z^t$. Let $g_i$ denote the gradients backpropagated from the loss $\mathcal{L}_i$ with respect to the shared parameters $\Theta_s^t$, expressed as $g_i = \nabla{\Theta_s^t} \mathcal{L}_i (z^t, \Theta_s^t, \Theta_i)$.

Using the first-order Taylor approximation of $\mathcal{L}_k$ for $\Theta_s^t$, we obtain:
\begin{align}
    \mathcal{L}_k (z^t, \Theta_{s|i,k}^{t+1}, \Theta_k^t) &= \mathcal{L}_k (z^t, \Theta_s^t, \Theta_k^t) + (\Theta_{s|i,k}^{t+1} - \Theta_s^t) \nabla_{\Theta_s^t} \mathcal{L}_k (z^t, \Theta_s^t, \Theta_k^t) + O(\eta^2)\\
    &= \mathcal{L}_k (z^t, \Theta_s^t, \Theta_k^t) - \eta (g_i + g_k)\cdot g_k + O(\eta^2)
    \label{eq:theo2_pre1}
\end{align}

For task $j$, we can follow a similar process as follows:
\begin{align}
    \mathcal{L}_k (z^t, \Theta_{s|j,k}^{t+1}, \Theta_k^t) = \mathcal{L}_k (z^t, \Theta_s^t, \Theta_k^t) - \eta (g_j + g_k)\cdot g_k + O(\eta^2)
    \label{eq:theo2_pre2}
\end{align}

With a sufficiently small learning rate $\eta \ll 1$, subtract \cref{eq:theo2_pre2} from \cref{eq:theo2_pre1}, then:
\begin{align}
    \mathcal{L}_k (z^t, \Theta_{s|i,k}^{t+1}, \Theta_k^t) - \mathcal{L}_k (z^t, \Theta_{s|j,k}^{t+1}, \Theta_k^t) &= - \eta (g_i + g_k)\cdot g_k + \eta (g_j + g_k)\cdot g_k \\
    &= - \eta(g_i-g_j)\cdot g_k \leq 0
    \label{eq:theo2_result}
\end{align}
which proves the results.
\end{proof}

The result indicates that when the gradients $g_i$ from task $i$ align better with those of the reference task $k$ compared to task $j$, the loss on the reference task $k$ tends to be lower with updated gradients $g_i + g_k$ compared to $g_j + g_k$, especially for sufficiently small learning rates $\eta$. 



%-------------------------------------------------------------------------
\subsection{Proof of \Cref{theorem3}}
\label{Append:theorem3}

\theomthree*
\begin{proof}
Let's begin with the definition of inter-task affinity between $\{i, j\}\rightarrow k$ and $i \rightarrow k$ as follows:
\begin{align}
    \mathcal{A}_{i,k \rightarrow k}^t &= 1-\frac{\mathcal{L}_k(z^t, \Theta_{s|i,k}^{t+1}, \Theta_k^{t})}{\mathcal{L}_k(z^t, \Theta_{s}^t, \Theta_k^{t})} &
    \mathcal{A}_{i\rightarrow k}^t &= 1-\frac{\mathcal{L}_k(z^t, \Theta_{s|i}^{t+1}, \Theta_k^{t})}{\mathcal{L}_k(z^t, \Theta_{s}^t, \Theta_k^{t})}
    \label{eq:theo5_affin}
\end{align}

When updating $i$ and $k$ simultaneously, we can derive the first-order Taylor approximation of $\mathcal{L}_k$ for $\Theta_s^t$ as follows:
\begin{align}
    \mathcal{L}_k (z^t, \Theta_{s|i,k}^{t+1}, \Theta_k^t) &= \mathcal{L}_k (z^t, \Theta_s^t, \Theta_k^t) + (\Theta_{s|i,k}^{t+1} - \Theta_s^t) \nabla_{\Theta_s^t} \mathcal{L}_k (z^t, \Theta_s^t, \Theta_k^t) + O(\eta^2)\\
    &= \mathcal{L}_k (z^t, \Theta_s^t, \Theta_k^t) - \eta (g_i + g_k)\cdot g_k + O(\eta^2)
\end{align}

Similarly, when updating $i$ alone, the first-order Taylor approximation of $\mathcal{L}_k$ for $\Theta_s^t$ is as follows:
\begin{align}
    \mathcal{L}_k (z^t, \Theta_{s|i}^{t+1}, \Theta_k^t) &= \mathcal{L}_k (z^t, \Theta_s^t, \Theta_k^t) + (\Theta_{s|i}^{t+1} - \Theta_s^t) \nabla_{\Theta_s^t} \mathcal{L}_k (z^t, \Theta_s^t, \Theta_k^t) + O(\eta^2)\\
    &= \mathcal{L}_k (z^t, \Theta_s^t, \Theta_k^t) - \eta g_i\cdot g_k + O(\eta^2)
\end{align}

With a sufficiently small learning rate $\eta$, the difference between the two inter-task affinities in \cref{eq:theo5_affin} can be expressed as follows:
\begin{align}
    \mathcal{A}_{i,k \rightarrow k}^t - \mathcal{A}_{i\rightarrow k}^t &= 1-\frac{\mathcal{L}_k(z^t, \Theta_{s|i,k}^{t+1}, \Theta_k^{t})}{\mathcal{L}_k(z^t, \Theta_{s}^t, \Theta_k^{t})} - \biggr(1-\frac{\mathcal{L}_k(z^t, \Theta_{s|i}^{t+1}, \Theta_k^{t})}{\mathcal{L}_k(z^t, \Theta_{s}^t, \Theta_k^{t})}\biggr) \\
    &= \frac{\mathcal{L}_k(z^t, \Theta_{s|i}^{t+1}, \Theta_k^{t}) - \mathcal{L}_k(z^t, \Theta_{s|i,k}^{t+1}, \Theta_k^{t})}{\mathcal{L}_k(z^t, \Theta_{s}^t, \Theta_k^{t})}\\
    &= \frac{\mathcal{L}_k (z^t, \Theta_s^t, \Theta_k^t) - \eta g_i\cdot g_k - (\mathcal{L}_k (z^t, \Theta_s^t, \Theta_k^t) - \eta (g_i + g_k)\cdot g_k)}{\mathcal{L}_k(z^t, \Theta_{s}^t, \Theta_k^{t})}\\
    &= \frac{\eta||g_k||^2}{\mathcal{L}_k(z^t, \Theta_{s}^t, \Theta_k^{t})} \\
    &\geq 0
    \label{eq:theo5_result}
\end{align}
The inequality in \cref{eq:theo5_result} proves that $\mathcal{A}_{i,k \rightarrow k}^t \geq \mathcal{A}_{i \rightarrow k}^t$.
\end{proof}

When tasks $i$ and $k$ are within the same task group, we can access $\mathcal{B}_{i,k \rightarrow k}^t$ during the optimization process. If $\mathcal{B}_{i,k \rightarrow k}^t \leq 0$, the inter-task affinity also satisfies $\mathcal{A}_{i,k \rightarrow k}^t \leq 0$ as $\mathcal{A}_{i,k \rightarrow k}^t \leq \mathcal{B}_{i,k \rightarrow k}^t$. According to \Cref{theorem3}, this condition implies $\mathcal{A}_{i\rightarrow k}^t\leq 0$. The proposed algorithm separates these tasks into different groups when $\mathcal{B}_{i,k \rightarrow k}^t \leq 0$ which justifies our grouping rules.

Conversely, when tasks $i$ and $j$ belong to separate task groups, we only have access to $\mathcal{B}_{i \rightarrow k}^t$ instead of $\mathcal{B}_{i,k \rightarrow k}^t$. In this scenario, the proposed algorithm merges these tasks into the same group if $\mathcal{B}_{i \rightarrow k}^t = \mathcal{A}_{i \rightarrow k}^t \geq 0$. This inequality also implies $\mathcal{A}_{i,k\rightarrow k}^t\geq 0$, justifying the merging of tasks $i$ and $k$ based on $\mathcal{B}_{i \rightarrow k}^t$ during optimization.

%-------------------------------------------------------------------------
\subsection{Proof of \Cref{theorem4}}
\label{Append:theorem4}

\theomfour*

Let's represent the sum of losses of tasks included in $G_m$ as $\mathcal{L}_{G_m}$, defined as follows:
\begin{align}
    \mathcal{L}_{G_m}(z^t, \Theta_{s|G_m}^{t+m/\mathcal{M}}, \Theta_{G_m}^{t+m/\mathcal{M}}) = \sum_{k \in G_m} \mathcal{L}_k(z^t, \Theta_{s|G_m}^{t+m/\mathcal{M}}, \Theta_{k}^{t+m/\mathcal{M}})
\end{align}
where $\Theta_{s|G_m}^{t+m/\mathcal{M}}$ denotes the shared parameters, while $\Theta_{G_m}^{t+m/\mathcal{M}}$  represents the set of task-specific parameters within $G_m$ after updating tasks in $G_m$.

We begin by expanding the task-specific loss $\mathcal{L}_{G_m}$ in terms of the shared parameter $\Theta_{s|G_m}^{t+m/\mathcal{M}}$ and the task-specific parameters $\Theta_{G_m}^{t+m/\mathcal{M}}$ using a quadratic expansion. During this process, the task-specific parameters $\Theta_{G_m}^{t+(m-1)/\mathcal{M}}=\Theta_{G_m}^t$ and $\Theta_{G_m}^{t+m/\mathcal{M}}=\Theta_{G_m}^{t+1}$, since the task-specific parameters in $G_m$ are updated only once from $\Theta_{G_m}^{t+(m-1)/\mathcal{M}}$ to $\Theta_{G_m}^{t+m/\mathcal{M}}$.
\begin{align}
    \mathcal{L}_{G_m} (z^t, \Theta_{s|G_m}^{t+m/\mathcal{M}},& \Theta_{G_m}^{t+1}) \leq \mathcal{L}_{G_m} (z^t, \Theta_{s|G_{m-1}}^{t+(m-1)/\mathcal{M}}, \Theta_{G_m}^t) \label{eq:theo4_in0}\\
    &+\nabla_{\Theta_{s|G_{m-1}}^{t+(m-1)/\mathcal{M}}}\mathcal{L}_{G_m}(z^t, \Theta_{s|G_{m-1}}^{t+(m-1)/\mathcal{M}}, \Theta_{G_m}^t)(\Theta_{s|G_m}^{t+m/\mathcal{M}}-\Theta_{s|G_{m-1}}^{t+(m-1)/\mathcal{M}})\\
    &+\frac{1}{2}\nabla_{\Theta_{s|G_{m-1}}^{t+(m-1)/\mathcal{M}}}^{2}\mathcal{L}_{G_m}(z^t, \Theta_{s|G_{m-1}}^{t+(m-1)/\mathcal{M}}, \Theta_{G_m}^t)(\Theta_{s|G_m}^{t+m/\mathcal{M}}-\Theta_{s|G_m}^{t+(m-1)/\mathcal{M}})^{2}\\
    &+\nabla_{\Theta_{G_m}^t}\mathcal{L}_{G_m}(z^t, \Theta_{s|G_{m-1}}^{t+(m-1)/\mathcal{M}}, \Theta_{G_m}^t)(\Theta_{G_m}^{t+1}-\Theta_{G_m}^t)\\
    &+\frac{1}{2}\nabla_{\Theta_{G_m}^t}^2 \mathcal{L}_{G_m}(z^t, \Theta_{s|G_{m-1}}^{t+(m-1)/\mathcal{M}}, \Theta_{G_m}^t)(\Theta_{G_m}^{t+1}-\Theta_{G_m}^t)^{2}\\
    \leq &\mathcal{L}_{G_m} (z^t, \Theta_{s|G_m}^{t+(m-1)/\mathcal{M}}, \Theta_{G_m}^t)\\
    &+\nabla_{\Theta_{s|G_{m-1}}^{t+(m-1)/\mathcal{M}}}\mathcal{L}_k(z^t, \Theta_{s|G_{m-1}}^{t+(m-1)/\mathcal{M}}, \Theta_{G_m}^t)(\Theta_{s|G_m}^{t+m/\mathcal{M}}-\Theta_{s|G_{m-1}}^{t+(m-1)/\mathcal{M}})\\
    &+\frac{1}{2} H|G_m| (\Theta_{s|G_m}^{t+m/\mathcal{M}}-\Theta_{s|G_{m-1}}^{t+(m-1)/\mathcal{M}})^{2}\\
    &+\nabla_{\Theta_{G_m}^t}\mathcal{L}_{G_m}(z^t, \Theta_{s|G_{m-1}}^{t+(m-1)/\mathcal{M}}, \Theta_{G_m}^t)(\Theta_{G_m}^{t+1}-\Theta_{G_m}^t)\\
    &+\frac{1}{2} H|G_m| (\Theta_{G_m}^{t+1}-\Theta_{G_m}^t)^{2}
\end{align}
where $|G_m|$ represents the number of tasks in $G_m$. The inequality holds with the Lipschitz continuity of $\nabla \mathcal{L}$ with a constant $H$.

For the shared parameters of the network, the update rule is as follows:
\begin{align}
    \Theta_{s|G_m}^{t+m/\mathcal{M}} &= \Theta_{s|G_{m-1}}^{t+(m-1)/\mathcal{M}} - \eta \nabla_{\Theta_{s|G_{m-1}}^{t+(m-1)/\mathcal{M}}} \mathcal{L}_{G_m}(z^t, \Theta_{s|G_{m-1}}^{t+(m-1)/\mathcal{M}}, \Theta_{G_m}^t)\\
    &= \Theta_{s|G_{m-1}}^{t+(m-1)/\mathcal{M}} - \eta g_{s, G_m}^{t+(m-1)/\mathcal{M}}
    \label{eq:theo4_in1}
\end{align}
where $g_{s, G_m}^{t+(m-1)/\mathcal{M}}$ is the gradients of the shared parameters with respect to loss of tasks in $G_m$.

Similarly, the task-specific parameters of the network, the update rule is as follows:
\begin{align}
    \Theta_{G_m}^{t+1} &= \Theta_{G_m}^{t} - \eta \nabla_{\Theta_{G_m}^{t}} \mathcal{L}_{G_m}(z^t, \Theta_{s|G_{m-1}}^{t+(m-1)/\mathcal{M}}, \Theta_{G_m}^t) = \Theta_{G_m}^t - \eta g_{ts, G_m}^{t}
    \label{eq:theo4_in2}
\end{align}
where $g_{ts, G_m}^t$ is the gradients of the task-specific parameters with respect to the loss of tasks in $G_m$.

If we substitute \cref{eq:theo4_in1} and \cref {eq:theo4_in2} into the result of \cref{eq:theo4_in0}, it become as follows:
\begin{align}
    \mathcal{L}_{G_m} (z^t, \Theta_{s|G_m}^{t+m/\mathcal{M}}, \Theta_{G_m}^{t+1}) \leq& \mathcal{L}_{G_m} (z^t, \Theta_{s|G_{m-1}}^{t+(m-1)/\mathcal{M}}, \Theta_{G_m}^t)\\
    &- \eta ||g_{s, G_m}^{t+(m-1)/\mathcal{M}}||^2  + \frac{\eta^2 H|G_m|}{2}||g_{s, G_m}^{t+(m-1)/\mathcal{M}}||^2 \\
    &- \eta ||g_{ts, G_m}^t||^2  + \frac{\eta^2 H|G_m|}{2}||g_{ts, G_m}^t||^2
\end{align}

We can derive similar results for the loss of task group $G_i$, where the index $i$ is not the same as the updating group sequence $m$ ($i \neq m$). This process follows similarly to the one described above. For the step from $t+(m-1)/\mathcal{M}$ to $t+m/\mathcal{M}$, the task-specific parameters in $G_i$ remain unchanged.
\begin{align}
    \mathcal{L}_{G_i} (z^t, \Theta_{s|G_m}^{t+m/\mathcal{M}},& \Theta_{G_i}^t) \leq \mathcal{L}_{G_i} (z^t, \Theta_{s|G_{m-1}}^{t+(m-1)/\mathcal{M}}, \Theta_{G_i}^t)\\
    &+\nabla_{\Theta_{s|G_{m-1}}^{t+(m-1)/\mathcal{M}}}\mathcal{L}_{G_i}(z^t, \Theta_{s|G_{m-1}}^{t+(m-1)/\mathcal{M}}, \Theta_{G_i}^t)(\Theta_{s|G_m}^{t+m/\mathcal{M}}-\Theta_{s|G_{m-1}}^{t+(m-1)/\mathcal{M}})\\
    &+\frac{1}{2}\nabla_{\Theta_{s|G_{m-1}}^{t+(m-1)/\mathcal{M}}}^{2}\mathcal{L}_{G_i}(z^t, \Theta_{s|G_{m-1}}^{t+(m-1)/\mathcal{M}}, \Theta_{G_i}^t)(\Theta_{s|G_m}^{t+m/\mathcal{M}}-\Theta_{s|G_m}^{t+(m-1)/\mathcal{M}})^{2}\\
    % &+\nabla_{\Theta_{G_m}^t}\mathcal{L}_{G_i}(z^t, \Theta_{s|G_{m-1}}^{t+(m-1)/\mathcal{M}}, \Theta_{G_i}^t)(\Theta_{G_m}^{t+1}-\Theta_{G_m}^t)\\
    % &+\frac{1}{2}\nabla_{\Theta_{G_m}^t}^2 \mathcal{L}_{G_i}(z^t, \Theta_{s|G_{m-1}}^{t+(m-1)/\mathcal{M}}, \Theta_{G_m}^t)(\Theta_{G_m}^{t+1}-\Theta_{G_m}^t)^{2}\\
    \leq &\mathcal{L}_{G_i} (z^t, \Theta_{s|G_m}^{t+(m-1)/\mathcal{M}}, \Theta_{G_i}^t)\\
    &+\nabla_{\Theta_{s|G_{m-1}}^{t+(m-1)/\mathcal{M}}}\mathcal{L}_{G_i}(z^t, \Theta_{s|G_{m-1}}^{t+(m-1)/\mathcal{M}}, \Theta_{G_i}^t)(\Theta_{s|G_m}^{t+m/\mathcal{M}}-\Theta_{s|G_{m-1}}^{t+(m-1)/\mathcal{M}})\\
    &+\frac{1}{2} H|G_i| (\Theta_{s|G_m}^{t+m/\mathcal{M}}-\Theta_{s|G_{m-1}}^{t+(m-1)/\mathcal{M}})^{2}\\
    % &+\nabla_{\Theta_{G_m}^t}\mathcal{L}_{G_i}(z^t, \Theta_{s|G_{m-1}}^{t+(m-1)/\mathcal{M}}, \Theta_{G_m}^t)(\Theta_{G_m}^{t+1}-\Theta_{G_m}^t)\\
    % &+\frac{1}{2} H|G_i| (\Theta_{G_m}^{t+1}-\Theta_{G_m}^t)^{2}\\
    \leq & \mathcal{L}_{G_i} (z^t, \Theta_{s|G_m}^{t+(m-1)/\mathcal{M}}, \Theta_{G_i}^t)\\
    &- \eta g_{s, G_i}^{t+(m-1)/\mathcal{M}} \cdot g_{s, G_m}^{t+(m-1)/\mathcal{M}} + \frac{\eta^2 H|G_i|}{2}||g_{s, G_m}^{t+(m-1)/\mathcal{M}}||^2\\
    % &- \eta g_{ts, G_i}^t \cdot g_{ts, G_m}^t  + \frac{\eta^2 H|G_i|}{2}||g_{ts, G_m}^t||^2 \\
\end{align}

Then the total loss of multiple task groups can be expressed as follows:
\begin{align}
    \sum_{k=1}^{\mathcal{M}} \mathcal{L}_{G_k}(z^t, &\Theta_{s|G_m}^{t+m/\mathcal{M}}, \Theta_{G_k}^{t+1}) \leq \sum_{k=1}^{\mathcal{M}} \mathcal{L}_{G_k}(z^t, \Theta_{s|G_m}^{t+(m-1)/\mathcal{M}}, \Theta_{G_k}^t) \\
    &-\eta\sum_{k=1}^{\mathcal{M}} g_{s, G_k}^{t+(m-1)/\mathcal{M}} \cdot g_{s, G_m}^{t+(m-1)/\mathcal{M}} + \frac{\eta^2 H}{2} ||g_{s, G_m}^{t+(m-1)/\mathcal{M}}||^2 \sum_{k=1}^{\mathcal{M}} |G_k| \label{eq:theo4_lip1}\\ 
    &- \eta ||g_{ts, G_m}^t||^2  + \frac{\eta^2 H|G_m|}{2}||g_{ts, G_m}^t||^2 \label{eq:theo4_lip2}\\
    \leq&\sum_{k=1}^{\mathcal{M}} \mathcal{L}_{G_k}(z^t, \Theta_{s|G_m}^{t+(m-1)/\mathcal{M}}, \Theta_{G_k}^t) \label{eq:theo4_lip_out_0}\\
    &-\eta\sum_{k=1}^{\mathcal{M}} g_{s, G_k}^{t+(m-1)/\mathcal{M}} \cdot g_{s, G_m}^{t+(m-1)/\mathcal{M}} + \eta ||g_{s, G_m}^{t+(m-1)/\mathcal{M}}||^2 - \frac{\eta}{2}||g_{ts, G_m}^t||^2 \label{eq:theo4_lip_out}\\
    =& \sum_{k=1}^{\mathcal{M}} \mathcal{L}_{G_k}(z^t, \Theta_{s|G_m}^{t+(m-1)/\mathcal{M}}, \Theta_{G_k}^t)\\
    &-\eta g_{s, G_m}^{t+(m-1)/\mathcal{M}} \cdot (\sum_{k=1}^{\mathcal{M}} g_{s, G_k}^{t+(m-1)/\mathcal{M}} - g_{s, G_m}^{t+(m-1)/\mathcal{M}}) - \frac{\eta}{2}||g_{ts, G_m}^t||^2
    \label{eq:theo4_result}
\end{align}


The inequality between \cref{eq:theo4_lip1} and the first term in \cref{eq:theo4_lip_out} requires $\eta\leq \frac{2}{H\cdot \sum_{k=1}^{\mathcal{M}} |G_k|} = \frac{2}{H \mathcal{K}}$, while the inequality between \cref{eq:theo4_lip2} and the second term in \cref{eq:theo4_lip_out} requires $\eta\leq \frac{1}{H|G_M|}$. Therefore, the inequality in \cref{eq:theo4_lip_out_0} holds when $\eta \leq \min(\frac{2}{H\mathcal{K}}, \frac{1}{H|G_M|})$. Previous approaches, which handle updates of shared and task-specific parameters independently, failing to capture their interdependence during optimization.
The term, $g_{s, G_m}^{t+(m-1)/\mathcal{M}} \cdot (\sum_{k=1}^{\mathcal{M}} g_{s, G_k}^{t+(m-1)/\mathcal{M}})$, fluctuates during optimization. When the gradients of group $G_m$ align well with the gradients of the other groups $\{G_k\}_{i=1, i\neq m}^{\mathcal{M}}$, their dot product yields a positive value, leading to a decrease in multi-task losses. However, in practice, the sequential update strategy demonstrates a similar level of stability in optimization, which appears to contradict the conventional results. Thus, we assume a correlation between the learning of shared parameters and task-specific parameters, where the learning of task-specific parameters reduces gradient conflicts in shared parameters. Under this assumption, the sequential update strategy can guarantee convergence to Pareto-stationary points. This assumption is reasonable, as task-specific parameters capture task-specific information, thereby reducing conflicts in the shared parameters across tasks.
 
 
%-------------------------------------------------------------------------
\subsection{Two-Step Proximal Inter-task Affinity}
\label{Append:two_step_proximal_inter_task_affinity}

Before delving into the proof of Theorem 5, let's introduce the concept of two-step proximal inter-task affinity, which extends the notion of proximal inter-task affinity over two update steps.
\begin{definition}[Two-Step Proximal Inter-Task Affinity] Consider a multi-task network shared by the tasks $i, j, k$, with their respective losses denoted as $\mathcal{L}_i, \mathcal{L}_j, \mathcal{L}_k$. Sequential updates of $(\{j\}, \{i, k\})$ result in parameters being updated from $\Theta_s^{t} \rightarrow \Theta_{s|j}^{t+1} \rightarrow \Theta_{s|i,k}^{t+2}$ and $\Theta_k^{t} \rightarrow \Theta_k^{t+1} \rightarrow \Theta_k^{t+2}$. Then, the two-step proximal inter-task affinity from sequential update $(\{j\}, \{i, k\})$ to $k$ at time step $t$ is defined as follows:
\begin{align}
    \mathcal{B}^t_{j; i,k\rightarrow k} &= 1-(1-\mathcal{B}^t_{j\rightarrow k})(1-\mathcal{B}^{t+1}_{i,k\rightarrow k}) \\
    &= 1-\frac{\mathcal{L}_k(z^t, \Theta_{s|j}^{t+1}, \Theta_k^{t+1})}{\mathcal{L}_k(z^t, \Theta_s^{t}, \Theta_k^{t})} \cdot \frac{\mathcal{L}_k(z^t, \Theta_{s|i,k}^{t+2}, \Theta_k^{t+2})}{\mathcal{L}_k(z^t, \Theta_{s|j}^{t+1}, \Theta_k^{t+1})} = 1-\frac{\mathcal{L}_k(z^t, \Theta_{s|i,k}^{t+2}, \Theta_k^{t+2})}{\mathcal{L}_k(z^t, \Theta_s^{t}, \Theta_k^{t})}
\end{align}
\end{definition}

%-------------------------------------------------------------------------
\subsection{Proof of \Cref{theorem5}}
\label{Append:theorem5}

\theomfive*
\begin{proof}
We compare the loss after jointly updating three tasks $\{i, j, k\}$ with the loss after sequentially updating the task sets $\{i, k\}$ and $\{j\}$. To assess the impact of the updating order of task sets, we also conduct the analysis on the reverse order of task set $\{j\}, \{i, k\}$.

(i) Let's begin with the definition of proximal inter-task affinity between $\{i, k\}\rightarrow k$ and $j \rightarrow k$, taking into account the updates of task-specific parameters as follows:
\begin{align}
    \mathcal{B}_{i,j,k \rightarrow k}^{t+(m-1)/\mathcal{M}} &= 1-\frac{\mathcal{L}_k(z^t, \Theta_{s|i,j,k}^{t+m/\mathcal{M}}, \Theta_k^{t+m/\mathcal{M}})}{\mathcal{L}_k(z^t, \Theta_{s}^{t+(m-1)/\mathcal{M}}, \Theta_k^{t+(m-1)/\mathcal{M}})} \label{eq:theo5_joint}
\end{align}

\begin{align}
    \mathcal{B}_{i,k \rightarrow k}^{t+(m-1)/\mathcal{M}} &= 1-\frac{\mathcal{L}_k(z^t, \Theta_{s|i,k}^{t+m/\mathcal{M}}, \hat{\Theta}_k^{t+m/\mathcal{M}})}{\mathcal{L}_k(z^t, \Theta_{s}^{t+(m-1)/\mathcal{M}}, \Theta_k^{t+(m-1)/\mathcal{M}})}
\end{align}
\begin{align}
    \mathcal{B}_{j\rightarrow k}^{t+m/\mathcal{M}} &= 1-\frac{\mathcal{L}_k(z^t, \Theta_{s|j}^{t+(m+1)/\mathcal{M}}, \Theta_k^{t+(m+1)/\mathcal{M}})}{\mathcal{L}_k(z^t, \Theta_{s|i,k}^{t+m/\mathcal{M}}, \hat{\Theta}_k^{t+m/\mathcal{M}})}\\
    &= 1-\frac{\mathcal{L}_k(z^t, \Theta_{s|j}^{t+(m+1)/\mathcal{M}}, \hat{\Theta}_k^{t+m/\mathcal{M}})}{\mathcal{L}_k(z^t, \Theta_{s|i,k}^{t+m/\mathcal{M}}, \hat{\Theta}_k^{t+m/\mathcal{M}})}
\end{align}

where $\hat{\Theta}_k^{t+m/\mathcal{M}}$ represents the resulting task-specific parameters of $k$ immediately after updating the task set $\{i, j\}$. This notation is used to differentiate it from the task-specific parameter $\Theta_k^{t+m/\mathcal{M}}$ obtained after jointly updating all tasks.

The two-step proximal inter-task affinity with the sequence $\{i, k\}$ and $\{j\}$ can be represented as follows:
\begin{align}
    \mathcal{B}_{i,k; j \rightarrow k}^{t+(m-1)/\mathcal{M}} &= 1-(1-\mathcal{B}_{i,k \rightarrow k}^{t+(m-1)/\mathcal{M}}) \cdot (1-\mathcal{B}_{j\rightarrow k}^{t+m/\mathcal{M}})\\ 
    &= 1-\frac{\mathcal{L}_k(z^t, \Theta_{s|i,k}^{t+m/\mathcal{M}}, \hat{\Theta}_k^{t+m/\mathcal{M}})}{\mathcal{L}_k(z^t, \Theta_{s}^{t+(m-1)/\mathcal{M}}, \Theta_k^{t+(m-1)/\mathcal{M}})} \cdot \frac{\mathcal{L}_k(z^t, \Theta_{s|j}^{t+(m+1)/\mathcal{M}}, \hat{\Theta}_k^{t+m/\mathcal{M}})}{\mathcal{L}_k(z^t, \Theta_{s|i,k}^{t+m/\mathcal{M}}, \hat{\Theta}_k^{t+m/\mathcal{M}})}\\
    &= 1-\frac{\mathcal{L}_k(z^t, \Theta_{s|j}^{t+(m+1)/\mathcal{M}}, \hat{\Theta}_k^{t+m/\mathcal{M}})}{\mathcal{L}_k(z^t, \Theta_{s}^{t+(m-1)/\mathcal{M}}, \Theta_k^{t+(m-1)/\mathcal{M}})} \label{eq:theo5_prox}
\end{align}

Our objective is to compare $\mathcal{B}_{i,j,k \rightarrow k}^{t+(m-1)/\mathcal{M}}$ from \cref{eq:theo5_joint} with $\mathcal{B}_{i,k; j \rightarrow k}^{t+(m-1)/\mathcal{M}}$ from \cref{eq:theo5_prox} to assess each update's effect on the final loss. Since both equations share a common denominator, we only need to compare the numerators of each equation. Using the first-order Taylor approximation of $\mathcal{L}_k(z^t, \Theta_{s|j}^{t+(m+1)/\mathcal{M}}, \hat{\Theta}_k^{t+m/\mathcal{M}})$ in \cref{eq:theo5_prox}, we have:

\begin{align}
    \mathcal{L}_k(z^t, &\Theta_{s|j}^{t+(m+1)/\mathcal{M}}, \hat{\Theta}_k^{t+m/\mathcal{M}}) = \mathcal{L}_k (z^t, \Theta_{s|i,k}^{t+m/\mathcal{M}}, \hat{\Theta}_k^{t+m/\mathcal{M}}) \label{eq:theo5_middle1}\\
    &+(\Theta_{s|j}^{t+(m+1)/\mathcal{M}} - \Theta_{s|i,k}^{t+m/\mathcal{M}}) \nabla_{\Theta_{s|i,k}^{t+m/\mathcal{M}}} \mathcal{L}_k (z^t, \Theta_{s|i,k}^{t+m/\mathcal{M}}, \hat{\Theta}_k^{t+m/\mathcal{M}}) + O(\eta^2)\\
    =& \mathcal{L}_k (z^t, \Theta_{s|i,k}^{t+m/\mathcal{M}}, \hat{\Theta}_k^{t+m/\mathcal{M}}) - \eta g_{s;j}^{t+m/\mathcal{M}}\cdot g_{s;k}^{t+m/\mathcal{M}} + O(\eta^2)
\end{align}

The subscript $s$ in gradients indicates that it represents the gradients of the shared parameters of the network. Conversely, we will use the subscript $ts$ for gradients of the task-specific network in the following derivation. Similarly, $\mathcal{L}_k (z^t, \Theta_{s|i,k}^{t+m/\mathcal{M}}, \hat{\Theta}_k^{t+m/\mathcal{M}})$ in \cref{eq:theo5_middle1} can also be further expanded using Taylor expansion as follows:
\begin{align}
    \mathcal{L}_k &(z^t, \Theta_{s|i,k}^{t+m/\mathcal{M}}, \hat{\Theta}_k^{t+m/\mathcal{M}}) = \mathcal{L}_k (z^t, \Theta_s^{t+(m-1)/\mathcal{M}}, \Theta_k^{t+(m-1)/\mathcal{M}}) \\
    &+(\Theta_{s|i,k}^{t+m/\mathcal{M}} - \Theta_s^{t+(m-1)/\mathcal{M}}) \nabla_{\Theta_s^{t+(m-1)/\mathcal{M}}} \mathcal{L}_k (z^t, \Theta_s^{t+(m-1)/\mathcal{M}}, \Theta_k^{t+(m-1)/\mathcal{M}})\\
    &+(\hat{\Theta}_k^{t+m/\mathcal{M}}-\Theta_k^{t+(m-1)/\mathcal{M}}) \nabla_{\Theta_k^{t+(m-1)/\mathcal{M}}} \mathcal{L}_k (z^t, \Theta_s^{t+(m-1)/\mathcal{M}}, \Theta_k^{t+(m-1)/\mathcal{M}}) + O(\eta^2)\\
    =& \mathcal{L}_k (z^t, \Theta_s^{t+(m-1)/\mathcal{M}}, \Theta_k^{t+(m-1)/\mathcal{M}}) - \eta (g_{s;i}^{t+(m-1)/\mathcal{M}} + g_{s;k}^{t+(m-1)/\mathcal{M}})\cdot g_{s;k}^{t+(m-1)/\mathcal{M}} \label{eq:theo5_middle2}\\
    &- \eta g_{ts;k}^{t+(m-1)/\mathcal{M}}\cdot g_{ts;k}^{t+(m-1)/\mathcal{M}} + O(\eta^2) \label{eq:theo5_middle3}
\end{align}

By substituting \cref{eq:theo5_middle1} with the results of \cref{eq:theo5_middle2} and \cref{eq:theo5_middle3}, we can obtain the following results:
\begin{align}
    \mathcal{L}_k(z^t, &\Theta_{s|j}^{t+(m+1)/\mathcal{M}}, \hat{\Theta}_k^{t+m/\mathcal{M}}) = \mathcal{L}_k (z^t, \Theta_s^{t+(m-1)/\mathcal{M}}, \Theta_k^{t+(m-1)/\mathcal{M}})\\
    &- \eta g_{s;j}^{t+m/\mathcal{M}}\cdot g_{s;k}^{t+m/\mathcal{M}} - \eta (g_{s;i}^{t+(m-1)/\mathcal{M}} + g_{s;k}^{t+(m-1)/\mathcal{M}})\cdot g_{s;k}^{t+(m-1)/\mathcal{M}} \\
    &- \eta g_{ts;k}^{t+(m-1)/\mathcal{M}}\cdot g_{ts;k}^{t+(m-1)/\mathcal{M}}+ O(\eta^2)
\end{align}

For the scenario where all tasks $\{i, j, k\}$ are jointly updated, the numerator of \cref{eq:theo5_joint} can also be expanded as follows:
\begin{align}
    \mathcal{L}_k(z^t,& \Theta_{s|i,j,k}^{t+m/\mathcal{M}}, \Theta_k^{t+m/\mathcal{M}}) = \mathcal{L}_k (z^t, \Theta_s^{t+(m-1)/\mathcal{M}}, \Theta_k^{t+(m-1)/\mathcal{M}})\\
    &+(\Theta_{s|i,j,k}^{t+m/\mathcal{M}} - \Theta_s^{t+(m-1)/\mathcal{M}}) \nabla_{\Theta_s^{t+(m-1)/\mathcal{M}}} \mathcal{L}_k (z^t, \Theta_s^{t+(m-1)/\mathcal{M}}, \Theta_k^{t+(m-1)/\mathcal{M}})\\
    &+(\Theta_k^{t+m/\mathcal{M}} - \Theta_k^{t+(m-1)/\mathcal{M}}) \nabla_{\Theta_k^{t+(m-1)/\mathcal{M}}} \mathcal{L}_k (z^t, \Theta_s^{t+(m-1)/\mathcal{M}}, \Theta_k^{t+(m-1)/\mathcal{M}}) + O(\eta^2)\\
    =& \mathcal{L}_k (z^t, \Theta_s^{t+(m-1)/\mathcal{M}}, \Theta_k^{t+(m-1)/\mathcal{M}}) \\
    &- \eta (g_{s,i}^{t+(m-1)/\mathcal{M}} + g_{s,j}^{t+(m-1)/\mathcal{M}} + g_{s,k}^{t+(m-1)/\mathcal{M}})\cdot g_{s,k}^{t+(m-1)/\mathcal{M}} \\
    &- \eta g_{ts,k}^{t+(m-1)/\mathcal{M}}\cdot g_{ts,k}^{t+(m-1)/\mathcal{M}} + O(\eta^2)
\end{align}

Finally, we can compare $\mathcal{B}_{i,j,k \rightarrow k}^{t+(m-1)/\mathcal{M}}$ with $\mathcal{B}_{i,k; j \rightarrow k}^{t+(m-1)/\mathcal{M}}$ by comparing the losses we obtained: $\mathcal{L}_k(z^t, \Theta_{s|j}^{t+(m+1)/\mathcal{M}}, \Theta_k^{t+(m-1)/\mathcal{M}})$ with $\mathcal{L}_k(z^t, \Theta_{s|i,j,k}^{t+m/\mathcal{M}}, \Theta_k^{t+(m-1)/\mathcal{M}})$. We assume a sufficiently small learning rate $\eta$ that allows us to ignore terms larger than order two with $\eta$.

\begin{align}
    \mathcal{L}_k(z^t, & \Theta_{s|j}^{t+(m+1)/\mathcal{M}}, \hat{\Theta}_k^{t+m/\mathcal{M}}) - \mathcal{L}_k(z^t, \Theta_{s|i,j,k}^{t+m/\mathcal{M}}, \Theta_k^{t+m/\mathcal{M}})\\
    =& - \eta g_{s,j}^{t+m/\mathcal{M}}\cdot g_{s,k}^{t+m/\mathcal{M}} - \eta (g_{s,i}^{t+(m-1)/\mathcal{M}} + g_{s,k}^{t+(m-1)/\mathcal{M}})\cdot g_{s,k}^{t+(m-1)/\mathcal{M}} \\
    & - \eta g_{ts,k}^{t+(m-1)/\mathcal{M}}\cdot g_{ts,k}^{t+(m-1)/\mathcal{M}}\\
    &+ \eta (g_{s,i}^{t+(m-1)/\mathcal{M}} + g_{s,j}^{t+(m-1)/\mathcal{M}} + g_{s,k}^{t+(m-1)/\mathcal{M}})\cdot g_{s,k}^{t+(m-1)/\mathcal{M}}\\
    & + \eta g_{ts,k}^{t+(m-1)/\mathcal{M}}\cdot g_{ts,k}^{t+(m-1)/\mathcal{M}}\\
    =& \eta(g_{s,j}^{t+(m-1)/\mathcal{M}} \cdot g_{s,k}^{t+(m-1)/\mathcal{M}} - g_{s,j}^{t+m/\mathcal{M}} \cdot g_{s,k}^{t+m/\mathcal{M}})\\
    \simeq& 0 \label{eq:theo5_last1}
\end{align}

The approximation in \cref{eq:theo5_last1} holds as we assume inter-task affinity change during a single time step from $t+(m-1)/\mathcal{M}$ to $t+m/\mathcal{M}$ is negligible.


(ii) In case we update task groups in reverse order the results would differ with (i). Similarly, we begin with the definition of proximal inter-task affinity with reverse order between $j \rightarrow k$ and $\{i, j\}\rightarrow k$ as follows:

\begin{align}
    \mathcal{B}_{j\rightarrow k}^{t+(m-1)/\mathcal{M}} &= 1-\frac{\mathcal{L}_k(z^t, \Theta_{s|j}^{t+m/\mathcal{M}}, \Theta_k^{t+m/\mathcal{M}})}{\mathcal{L}_k(z^t, \Theta_{s|i,k}^{t+(m-1)/\mathcal{M}}, \Theta_k^{t+(m-1)/\mathcal{M}})}\\
    &= 1-\frac{\mathcal{L}_k(z^t, \Theta_{s|j}^{t+m/\mathcal{M}}, \Theta_k^{t+(m-1)/\mathcal{M}})}{\mathcal{L}_k(z^t, \Theta_s^{t+(m-1)/\mathcal{M}}, \Theta_k^{t+(m-1)/\mathcal{M}})}
\end{align}

\begin{align}
    \mathcal{B}_{i,k \rightarrow k}^{t+m/\mathcal{M}} &= 1-\frac{\mathcal{L}_k(z^t, \Theta_{s|i,k}^{t+(m+1)/\mathcal{M}}, \hat{\Theta}_k^{t+(m+1)/\mathcal{M}})}{\mathcal{L}_k(z^t, \Theta_{s|j}^{t+m/\mathcal{M}}, \Theta_k^{t+m/\mathcal{M}})} \\
    &= 1-\frac{\mathcal{L}_k(z^t, \Theta_{s|i,k}^{t+(m+1)/\mathcal{M}}, \hat{\Theta}_k^{t+(m+1)/\mathcal{M}})}{\mathcal{L}_k(z^t, \Theta_{s|j}^{t+m/\mathcal{M}}, \Theta_k^{t+(m-1)/\mathcal{M}})}
\end{align}

The two-step proximal inter-task affinity with the sequence $\{j\}$ and $\{i, k\}$ can be represented as follows:
\begin{align}
    \mathcal{B}_{j; i,k \rightarrow k}^{t+(m-1)/\mathcal{M}} &= 1- (1-\mathcal{B}_{j\rightarrow k}^{t+(m-1)/\mathcal{M}}) \cdot (1-\mathcal{B}_{i,k \rightarrow k}^{t+m/\mathcal{M}})\\ 
    &= 1-\frac{\mathcal{L}_k(z^t, \Theta_{s|j}^{t+m/\mathcal{M}}, \Theta_k^{t+(m-1)/\mathcal{M}})}{\mathcal{L}_k(z^t, \Theta_s^{t+(m-1)/\mathcal{M}}, \Theta_k^{t+(m-1)/\mathcal{M}})} \cdot \frac{\mathcal{L}_k(z^t, \Theta_{s|i,k}^{t+(m+1)/\mathcal{M}}, \hat{\Theta}_k^{t+(m+1)/\mathcal{M}})}{\mathcal{L}_k(z^t, \Theta_{s|j}^{t+m/\mathcal{M}}, \Theta_k^{t+(m-1)/\mathcal{M}})}\\
    &= 1-\frac{\mathcal{L}_k(z^t, \Theta_{s|i,k}^{t+(m+1)/\mathcal{M}}, \hat{\Theta}_k^{t+(m+1)/\mathcal{M}})}{\mathcal{L}_k(z^t, \Theta_s^{t+(m-1)/\mathcal{M}}, \Theta_k^{t+(m-1)/\mathcal{M}})} \label{eq:theo5_prox2}
\end{align}

Our objective is to compare $\mathcal{B}_{i,j,k \rightarrow k}^{t+(m-1)/\mathcal{M}}$ from \cref{eq:theo5_joint} with $\mathcal{B}_{j; i,k \rightarrow k}^{t+(m-1)/\mathcal{M}}$ from \cref{eq:theo5_prox2} to assess each update's effect on the final loss. Since both equations share a common denominator, we only need to compare the numerators of each equation. Using the first-order Taylor approximation of $\mathcal{L}_k(z^t, \Theta_{s|i,k}^{t+(m+1)/\mathcal{M}}, \hat{\Theta}_k^{t+(m+1)/\mathcal{M}})$ in \cref{eq:theo5_prox2}, we have:

\begin{align}
    \mathcal{L}_k(z^t, &\Theta_{s|i,k}^{t+(m+1)/\mathcal{M}}, \hat{\Theta}_k^{t+(m+1)/\mathcal{M}}) = \mathcal{L}_k (z^t, \Theta_{s|j}^{t+m/\mathcal{M}}, \Theta_k^{t+m/\mathcal{M}}) \label{eq:theo5_middle1_}\\
    &+(\Theta_{s|i,k}^{t+(m+1)/\mathcal{M}} - \Theta_{s|j}^{t+m/\mathcal{M}}) \nabla_{\Theta_{s|j}^{t+m/\mathcal{M}}} \mathcal{L}_k (z^t, \Theta_{s|j}^{t+m/\mathcal{M}}, \Theta_k^{t+m/\mathcal{M}})\\
    &+(\hat{\Theta}_k^{t+(m+1)/\mathcal{M}} - \Theta_k^{t+m/\mathcal{M}}) \nabla_{\Theta_k^{t+m/\mathcal{M}}} \mathcal{L}_k (z^t, \Theta_{s|j}^{t+m/\mathcal{M}}, \Theta_k^{t+m/\mathcal{M}}) + O(\eta^2)\\
    =& \mathcal{L}_k (z^t, \Theta_{s|j}^{t+m/\mathcal{M}}, \Theta_k^{t+m/\mathcal{M}}) - \eta (g_{s;i}^{t+m/\mathcal{M}}+g_{s;k}^{t+m/\mathcal{M}})\cdot g_{s;k}^{t+m/\mathcal{M}} \\
    &- \eta g_{ts;k}^{t+m/\mathcal{M}}\cdot g_{ts;k}^{t+m/\mathcal{M}} + O(\eta^2)
\end{align}

Similarly, $\mathcal{L}_k (z^t, \Theta_{s|i,k}^{t+m/\mathcal{M}}, \Theta_k^{t+m/\mathcal{M}})$ in \cref{eq:theo5_middle1_} can also be further expanded using Taylor expansion as follows:
\begin{align}
    \mathcal{L}_k(z^t, &\Theta_{s|j}^{t+m/\mathcal{M}}, \Theta_k^{t+m/\mathcal{M}}) = \mathcal{L}_k(z^t, \Theta_{s|j}^{t+m/\mathcal{M}}, \Theta_k^{t+(m-1)/\mathcal{M}}) \\
    =& \mathcal{L}_k (z^t, \Theta_s^{t+(m-1)/\mathcal{M}}, \Theta_k^{t+(m-1)/\mathcal{M}})\\
    &+(\Theta_{s|j}^{t+m/\mathcal{M}} - \Theta_s^{t+(m-1)/\mathcal{M}}) \nabla_{\Theta_s^{t+(m-1)/\mathcal{M}}} \mathcal{L}_k (z^t, \Theta_{s|j}^{t+m/\mathcal{M}}, \Theta_k^{t+(m-1)/\mathcal{M}}) \\
    =& \mathcal{L}_k (z^t, \Theta_s^{t+(m-1)/\mathcal{M}}, \Theta_k^{t+(m-1)/\mathcal{M}}) \label{eq:theo5_middle2_}\\
    &- \eta g_{s;j}^{t+(m-1)/\mathcal{M}}\cdot g_{s;k}^{t+(m-1)/\mathcal{M}}+ O(\eta^2)
    \label{eq:theo5_middle3_}
\end{align}

By substituting \cref{eq:theo5_middle1_} with the results of \cref{eq:theo5_middle2_} and \cref{eq:theo5_middle3_}, we can obtain the following results:
\begin{align}
    \mathcal{L}_k(z^t, &\Theta_{s|i,k}^{t+(m+1)/\mathcal{M}}, \hat{\Theta}_k^{t+(m+1)/\mathcal{M}}) = \mathcal{L}_k (z^t, \Theta_s^{t+(m-1)/\mathcal{M}}, \Theta_k^{t+(m-1)/\mathcal{M}})\\
    &- \eta (g_{s;i}^{t+m/\mathcal{M}}+g_{s;k}^{t+m/\mathcal{M}})\cdot g_{s;k}^{t+m/\mathcal{M}} - \eta g_{ts;k}^{t+m/\mathcal{M}}\cdot g_{ts;k}^{t+m/\mathcal{M}} \\
    &- \eta g_{s;j}^{t+(m-1)/\mathcal{M}}\cdot g_{s;k}^{t+(m-1)/\mathcal{M}}+ O(\eta^2)
\end{align}

Finally, we can compare $\mathcal{B}_{i,j,k \rightarrow k}^{t+(m-1)/\mathcal{M}}$ with $\mathcal{B}_{j; i,k \rightarrow k}^{t+(m-1)/\mathcal{M}}$ by comparing the losses we obtained: $\mathcal{L}_k(z^t, \Theta_{s|j}^{t+(m+1)/\mathcal{M}}, \Theta_k^{t+(m-1)/\mathcal{M}})$ with $\mathcal{L}_k(z^t, \Theta_{s|i,j,k}^{t+m/\mathcal{M}}, \Theta_k^{t+(m-1)/\mathcal{M}})$. We assume a sufficiently small learning rate $\eta$ that allows us to ignore terms larger than order two with $\eta$.

\begin{align}
    \mathcal{L}_k(z^t, &\Theta_{s|i,k}^{t+(m+1)/\mathcal{M}}, \hat{\Theta}_k^{t+(m+1)/\mathcal{M}}) - \mathcal{L}_k(z^t, \Theta_{s|i,j,k}^{t+m/\mathcal{M}}, \Theta_k^{t+m/\mathcal{M}})\\
    =& - \eta (g_{s;i}^{t+m/\mathcal{M}}+g_{s;k}^{t+m/\mathcal{M}})\cdot g_{s;k}^{t+m/\mathcal{M}} - \eta g_{ts;k}^{t+m/\mathcal{M}}\cdot g_{ts;k}^{t+m/\mathcal{M}} \\
    &- \eta g_{s;j}^{t+(m-1)/\mathcal{M}}\cdot g_{s;k}^{t+(m-1)/\mathcal{M}}\\
    &+ \eta (g_{s,i}^{t+(m-1)/\mathcal{M}} + g_{s,j}^{t+(m-1)/\mathcal{M}} + g_{s,k}^{t+(m-1)/\mathcal{M}})\cdot g_{s,k}^{t+(m-1)/\mathcal{M}}\\
    &+ \eta g_{ts,k}^{t+(m-1)/\mathcal{M}}\cdot g_{ts,k}^{t+(m-1)/\mathcal{M}}\\
    =& - \eta (g_{s;i}^{t+m/\mathcal{M}}+g_{s;k}^{t+m/\mathcal{M}})\cdot g_{s;k}^{t+m/\mathcal{M}} + \eta (g_{s,i}^{t+(m-1)/\mathcal{M}} + g_{s,k}^{t+(m-1)/\mathcal{M}})\cdot g_{s,k}^{t+(m-1)/\mathcal{M}} \\
    & - \eta g_{ts;k}^{t+m/\mathcal{M}}\cdot g_{ts;k}^{t+m/\mathcal{M}} + \eta g_{ts,k}^{t+(m-1)/\mathcal{M}}\cdot g_{ts,k}^{t+(m-1)/\mathcal{M}} \\
    \simeq& - \eta g_{ts;k}^{t+m/\mathcal{M}}\cdot g_{ts;k}^{t+m/\mathcal{M}} + \eta g_{ts,k}^{t+(m-1)/\mathcal{M}}\cdot g_{ts,k}^{t+(m-1)/\mathcal{M}} \label{eq:theo5_last2}\\
    \leq& 0 \label{eq:theo5_final}
\end{align}

The approximation in \cref{eq:theo5_last2} holds under the assumption that the change in inter-task affinity during a single time step from $t+(m-1)/\mathcal{M}$ to $t+m/\mathcal{M}$ is negligible. Since we assume convex loss functions, the magnitude of task-specific gradients $g_{ts;k}$ would increase after updating the loss of $j$, which exhibits negative inter-task affinity with $k$ ($\mathcal{A}_{j \rightarrow k}<0$). Therefore, the inequality in \cref{eq:theo5_final} holds.
\end{proof}

This suggest that grouping tasks with proximal inter-task affinity and subsequently updating these groups sequentially result in lower multi-task loss compared to jointly backpropagating all tasks. This disparity arises because the network can discern superior task-specific parameters to accommodate task-specific information during sequential learning.


\clearpage
%%%%%%%%%%%%%%%%%%%%%%%%%%%%%%%%%%%%%%%%%%%%%%%%%%%%%%%%%%%%%%%%%%%%%%%%%%%%%%%%%%%%
\section{Additional Related Works}
\textbf{Task Grouping.} Early Multi-Task Learning research is founded on the belief that simultaneous learning of similar tasks within a multi-task framework can enhance overall performance. Kang et al. \cite{kang2011learning} identify tasks that contribute to improved multi-task performance through the clustering of related tasks with online stochastic gradient descent. This strategy challenges the prevailing assumption that all tasks are inherently interrelated. In parallel, Kumar et al. \cite{kumar2012learning} present a framework for MTL designed to enable selective sharing of information across tasks. They suggests that each task parameter vector can be expressed as a linear combination of a finite number of underlying basis tasks. However, these initial methodologies face limitations in their applicability and analysis, particularly in scaling to deep neural networks.
Finding out related tasks is more dynamically explored in the transfer learning domain \cite{achille2019task2vec, achille2021information}. They find related tasks by measuring task similarity which can be comparing the similarity of features extracted from the same depth of the independent task's network or directly measuring the transfer performance between tasks. Recent research has concentrated on identifying related tasks by directly assessing the relations among them within shared networks. This focus stems from the recognition that the measured inter-task relations in transfer learning fail to fully elucidate the dynamics within the MTL domain \cite{standley2020tasks, fifty2021efficiently}.

\textbf{Multi-Task Architectures.} Multi-task architectures can be classified based on how much the parameters or features are shared across tasks in the network. The most commonly used structure is a shared trunk which consists of a common encoder shared by multiple tasks and a dedicated decoder for each task \cite{RN51, RN52, RN49, RN50}. A tree-like architecture, featuring multiple division points for each task group, offers a more generalized structure \cite{treelike1, treelike2, treelike3, treelike4}. The cross-talk architecture employs separate symmetrical networks for each task, facilitating feature exchange between layers at the same depth for information sharing between tasks \cite{RN43, RN29}. The prediction distillation model \cite{RN9, RN29, RN32, pap} incorporates cross-task interactions at the end of the shared encoder, while the task switching network \cite{RN30, RN40, RN42, RN2} adjusts network parameters depending on the task.

\noindent\textbf{MTL in Vision Transformers.} Recent advancements in multi-task architecture have explored the integration of Vision Transformer \cite{vit, swin, pvt, focal, segformer, crossformer} into MTL. MTFormer \cite{mtformer} adopts a shared transformer encoder and decoder with a cross-task attention mechanism. MulT \cite{mult} leverages a shared attention mechanism to capture task dependencies, inspired by the Swin transformer. InvPT \cite{invpt} emphasizes global spatial position and multi-task context for dense prediction tasks through multi-scale feature aggregation. The Mixture of Experts (MoE) divides the model into predefined expert groups, dynamically shared or dedicated to specific tasks during the learning phase \cite{riquelme2021scaling, zhang2022mixture, fan2022m3vit, mustafa2022multimodal, chen2023mod, ye2023taskexpert}. Task prompter \cite{xu2023multi, xu2023demt, ye2022taskprompter} employs task-specific tokens to encapsulate task-specific information and utilizes cross-task interactions to enhance multi-task performance. 

\noindent\textbf{Multi-Task Domain Generalization.} Task grouping based on their relations has also been explored in the field of domain adaptation. In particular, \citep{wei2024task} proposes grouping heterogeneous tasks to regularize them, thereby promoting the learning of more generalized features across domain shifts. \citep{smith2021origin} explores generalization strategies at the mini-batch level. \citep{li2020sequential} addresses diverse domain shift scenarios by incorporating all possible sequential domain learning paths to generalize features for unseen domains. \citep{shi2021gradient} focuses on generalization to unseen domains by reducing dependence on specific domains through inter-domain gradient matching. Additionally, \citep{hu2022improving} analyzes the problem of spurious correlations in MTL and proposes regularization methods to mitigate this issue. The effect of gradient conflicts, which are considered the primary cause of negative transfer between tasks, is thoroughly examined in \citep{jiang2024forkmerge}. This work also proposes combining task distributions to identify better network parameters from a generalization perspective. The objectives of conventional multi-task optimization and domain generalization differ fundamentally. Conventional multi-task optimization typically assumes that the source and target domains share similar data distributions. In contrast, domain generalization focuses on scenarios involving significant domain shifts. This distinction leads to different approaches in leveraging task relations to achieve their respective goals. In multi-task optimization, simultaneously updating heterogeneous tasks with conflicting gradients results in suboptimal optimization. On the other hand, domain generalization leverages task sets as a tool to extract generalized features applicable to various unseen domains. Overfitting to similar tasks can harm performance on unseen domains, making it advantageous to use heterogeneous tasks as a form of regularization.



%%%%%%%%%%%%%%%%%%%%%%%%%%%%%%%%%%%%%%%%%%%%%%%%%%%%%%%%%%%%%%%%%%%%%%%%%%%%%%%%%%%%
\section{Experimental Details}
\label{append:experimental_details}
\setcounter{table}{0}
\setcounter{figure}{0}
We implement our experiments on top of publically available code from \cite{ye2022invpt}. We run our experiments on A6000 GPUs.

\textbf{Datasets.} We assess our method on multi-task datasets: NYUD-v2 \cite{RN15}, PASCAL-Context \cite{mottaghi2014role}, and Taskonomy \cite{zamir2018taskonomy}. These datasets encompass various vision tasks. NYUD-v2 comprises 4 vision tasks: depth estimation, semantic segmentation, surface normal prediction, and edge detection. Meanwhile, PASCAL-Context includes 5 tasks: semantic segmentation, human parts estimation, saliency estimation, surface normal prediction, and edge detection. In Taskonomy, we use 11 vision tasks: Depth Euclidean (DE), Depth Zbuffer (DZ), Edge Texture (ET),  Keypoints 2D (K2), Keypoints 3D (K3), Normal (N), Principal Curvature (C), Reshading (R), Segment Unsup2d (S2), and Segment Unsup2.5D (S2.5).

\textbf{Metrics. } To assess task performance, we employed widely used metrics across different tasks. For semantic segmentation, we utilized mean Intersection over Union (mIoU). The performance of surface normal prediction was gauged by computing the mean angle distances between the predicted output and ground truth. Depth estimation task performance was evaluated using Root Mean Squared Error (RMSE). For saliency estimation and human part segmentation, we utilized mean Intersection over Union (mIoU). Edge detection performance was assessed using optimal-dataset-scale-F-measure (odsF). For the Taskonomy benchmark, curvature was evaluated using RMSE, while the other tasks were evaluated using L1 distance, following the settings in \cite{chen2023mod}. 

To evaluate multi-task performance, we adopted the metric proposed in \cite{RN2}. This metric measures per-task performance by averaging it with respect to the single-task baseline $b$, as shown in the equation: $\triangle_m = (1/T)\sum_{i=1}^{T}(-1)^{l_i}(M_{m,i}-M_{b,i})/M_{b,i}$ where $l_i=1$ if a lower value of measure $M_i$ means better performance for task $i$, and 0 otherwise.


%-------------------------------------------------------------------------
\begin{table*}[h]
\caption{Hyperparameters for experiments.}
\centering
\renewcommand\arraystretch{1.20}
\begin{tabular}{lc}
\hline
Hyperparameter                  &  Value \\ \hline
Optimizer                       &  Adam \cite{kingma2014adam}\\
Scheduler                       &  Polynomial Decay\\
Minibatch size                  &  8\\
Number of iterations            &  40000\\
Backbone (Transformer)                        &  ViT \cite{vit} \\
\hspace{10pt}$\llcorner$ Learning rate                   &  0.00002\\
\hspace{10pt}$\llcorner$ Weight Decay                    &  0.000001\\
\hspace{10pt}$\llcorner$ Affinity decay factor $\beta$   &  0.001\\
\hline
\end{tabular}
\label{Implementation_details}
\end{table*}
%-------------------------------------------------------------------------


\textbf{Implementation Details.} For experiments, we adopt ViT \cite{vit} pre-trained on ImageNet-22K \cite{deng2009imagenet} as the multi-task encoder.
Task-specific decoders merge the multi-scale features extracted by the encoder to generate the outputs for each task. The models are trained for 40,000 iterations on both NYUD \cite{RN15} and PASCAL \cite{RN12} datasets with batch size 8. We used Adam optimizer with learning rate $2\times$$10^{-5}$ and $1\times$$10^{-6}$ of a weight decay with a polynomial learning rate schedule. The cross-entropy loss was used for semantic segmentation, human parts estimation, and saliency, edge detection. Surface normal prediction and depth estimation used L1 loss. The tasks are weighted equally to ensure a fair comparison. For the Taskonomy Benchmark \cite{zamir2018taskonomy}, we use the dataloader from the open-access code provided by \cite{chen2023mod}, while maintaining experimental settings identical to those used for NYUD-v2 and PASCAL-Context. We use the same experimental setup for the other hyperparameters as in previous works \cite{invpt, ye2022taskprompter}, as detailed in \Cref{Implementation_details}.


%%%%%%%%%%%%%%%%%%%%%%%%%%%%%%%%%%%%%%%%%%%%%%%%%%%%%%%%%%%%%%%%%%%%%%%%%%%%%%%%%%%%
% \clearpage
\section{Additional Experimental Results}
\label{append:additional_experimental_results}
\setcounter{table}{0}
\setcounter{figure}{0}

\textbf{Comparison of optimization results with different backbone sizes.} We compare the results of multi-task optimization on Taskonomy across various sizes of vision transformers, as shown in \Cref{tab:tab_exp_taskonomy_vitB,tab:tab_exp_taskonomy_vitS,tab:tab_exp_taskonomy_vitT}. Our method consistently achieves superior performance across all backbone sizes. Unlike previous approaches that focus on learning shared parameters, our optimization strategy enhances the learning of task-specific parameters. This leads to significant performance improvements, especially with smaller backbones, where competition between tasks is more intense due to the limited number of shared parameters. How tasks are grouped, as visualized in \cref{append:fig:vis_grouping}, depends on the backbone size.

\textbf{Visualization of Proximal Inter-Task Affinity.} In \Cref{fig:proximal_vit_taskonomy}, we present the tracked proximal inter-task affinity for each pair of tasks in Taskonomy. The changes in proximal inter-task affinity depend on the nature of the task pair, but as the backbone size increases, the affinity tends to become more positive. This trend is more noticeable in NYUD-v2 and PASCAL-Context, where there are fewer tasks, as shown in \Cref{fig:proximal_vit_nyud,fig:proximal_vit_pascal}. This pattern also aligns with the number of clustered tasks in \Cref{fig:num_group}, where the number of groups increases as the backbone size decreases.

\textbf{Effect of the Decay Rate with Visualization.} In \Cref{fig:proximal_vit_beta}, we visualize the proximal inter-task affinity tracked during optimization with various decay rates $\beta$, ranging from 0.01 to 1e-5 on a logarithmic scale. The decay rate $\beta$ helps stabilize the tracking of proximal inter-task affinity as it fluctuates during optimization. Additionally, it aids in understanding inter-task relations over time, independent of input data. For vision transformers, a decay rate of $\beta=0.001$ demonstrates stable tracking. In real-world applications, multi-task performance is not highly sensitive to the decay rate $\beta$. In \Cref{tab:tab_exp_beta_perf}, we evaluate how $\beta$ impacts multi-task performance on the Taskonomy benchmark. The results demonstrate that the proposed optimization method consistently improves performance across various $\beta$ values, minimizing the need for extensive hyperparameter tuning in practical scenarios.

\noindent\textbf{The Influence of Task Grouping Strategy.}  
In \Cref{tab:tab_exp_grouping_strategy}, we present results comparing different task grouping strategies. These include randomly grouping tasks with a predefined number, grouping heterogeneous tasks, and grouping homogeneous tasks (our approach). The results clearly demonstrate that grouping homogeneous task sets yields superior performance under the proposed settings. This contrasts with the multi-task domain generalization approach, which groups heterogeneous tasks as a form of regularization to enhance generalization to unseen domains. 
This difference arises from the fundamentally distinct objectives of conventional multi-task optimization and domain generalization. Conventional multi-task optimization typically assumes that the source and target domains share similar data distributions, while domain generalization addresses scenarios with significant domain shifts. Consequently, the approaches to leveraging task relations differ to meet these distinct goals. As demonstrated in Theorems 1 and 2 of our work, in multi-task optimization, simultaneously updating heterogeneous tasks with low task affinity leads to suboptimal optimization and higher losses compared to updating similar task sets with high task affinity. This observation aligns with findings from previous multi-task optimization studies referenced in the related works section.

\noindent\textbf{Influence of Batch Size.}  
In \Cref{tab:tab_exp_batch}, we compare our method with single-gradient descent (GD) to evaluate its robustness in improving multi-task performance across varying batch sizes. The proposed optimization method consistently demonstrates performance improvements (\(\triangle_m\) (\% \(\uparrow\))) of 5.27\%, 5.71\%, and 6.13\% across different batch sizes. These results highlight the robustness and adaptability of the proposed algorithm across diverse scenarios.



%%%%%%%%%%%%%%%%%%%%%%%%%%%%%%%%
\begin{figure}[h]
    \vspace{-10pt}
    \centering
    \begin{subfigure}{0.24\textwidth}
        \includegraphics[width=0.99\textwidth]{figure/group_viz_3.png}
        \vspace*{-15pt}
        \caption{$\{G\}_{i=1}^{\mathcal{M}}$ with ViT-L}
    \end{subfigure}
    \begin{subfigure}{0.24\textwidth}
        \includegraphics[width=0.99\textwidth]{figure/group_viz_2.png}
        \vspace*{-15pt}
        \caption{$\{G\}_{i=1}^{\mathcal{M}}$ with ViT-B}
    \end{subfigure}
    \begin{subfigure}{0.24\textwidth}
        \includegraphics[width=0.99\textwidth]{figure/group_viz_1.png}
        \vspace*{-15pt}
        \caption{$\{G\}_{i=1}^{\mathcal{M}}$ with ViT-S}
    \end{subfigure}
    \begin{subfigure}{0.24\textwidth}
        \includegraphics[width=0.99\textwidth]{figure/group_viz_0.png}
        \vspace*{-15pt}
        \caption{$\{G\}_{i=1}^{\mathcal{M}}$ with ViT-T}
    \end{subfigure}
    \caption{The averaged grouping results $\{G\}_{i=1}^{\mathcal{M}}$ on the Taskonomy benchmark are shown for (a) ViT-L, (b) ViT-B, (c) ViT-S, and (d) ViT-T, respectively.}
    \label{append:fig:vis_grouping}
\end{figure}
%%%%%%%%%%%%%%%%%%%%%%%%%%%%%%%%

% \clearpage
%%%%%%%%%%%%%%%%%%%%%%%%%%%%%%%%
\begin{table*}[t]
\caption{Comparison with previous multi-task optimization approaches on Taskonomy with ViT-B.}
\vspace{-5pt}
\centering
\renewcommand\arraystretch{1.00}
\resizebox{0.99\textwidth}{!}{
\begin{tabular}{l|ccccccccccc|c}
\midrule[1.0pt]
 & DE & DZ & EO & ET & K2  & K3 & N   & C & R & S2  & S2.5 &  \\ \cmidrule[0.5pt]{2-12}
\multirow{-2}{*}{Task} & L1 Dist. $\downarrow$  & L1 Dist. $\downarrow$ & L1 Dist. $\downarrow$ & L1 Dist. $\downarrow$ & L1 Dist. $\downarrow$ & L1 Dist. $\downarrow$ & L1 Dist. $\downarrow$ & RMSE $\downarrow$    & L1 Dist. $\downarrow$ & L1 Dist. $\downarrow$ & L1 Dist. $\downarrow$  & \multirow{-2}{*}{$\triangle_m$ ($\uparrow$)} \\ \midrule[1.0pt]

Single Task     &0.0183&0.0186&0.1089&0.1713&0.1630&0.0863&0.2953&0.7522&0.1504&0.1738&0.1530&-    \\ \midrule[0.5pt]
GD              &0.0188&0.0197&0.1283&0.1745&0.1718&0.0933&0.2599&0.7911&0.1799&0.1885&0.1631&-6.35     \\
GradDrop        &0.0195&0.0206&0.1318&0.1748&0.1735&0.0945&0.3018&0.8060&0.1866&0.1920&0.1607&-9.54     \\
MGDA            &-&-&-&-&-&-&-&-&-&-&-&-     \\
UW              &0.0188&0.0198&0.1285&0.1745&0.1719&0.0933&0.2535&0.7915&0.1800&0.1883&0.1629&-6.19     \\
DTP             &0.0187&0.0198&0.1283&0.1745&0.1720&0.0933&0.2558&0.7912&0.1804&0.1884&0.1634&-6.25     \\
DWA             &0.0188&0.0197&0.1287&0.1745&0.1719&0.0933&0.2570&0.7927&0.1806&0.1887&0.1632&-6.33     \\
PCGrad          &0.0185&0.0188&0.1285&0.1738&0.1703&0.0928&0.2557&0.7964&0.1810&0.1882&0.1569&-5.22     \\
CAGrad          &0.0192&0.0196&0.1306&0.1733&0.1654&0.0939&0.2871&0.8147&0.1901&0.1906&0.1659&-8.34    \\
IMTL            &0.0189&0.0200&0.1287&0.1745&0.1720&0.0934&0.2618&0.7928&0.1811&0.1888&0.1635&-6.75     \\
Aligned-MTL     &0.0191&0.0202&0.1263&0.1729&0.1663&0.0944&0.3061&0.8560&0.1936&0.1872&0.1585&-8.93     \\
Nash-MTL        &0.0175&0.0182&0.1208&0.1730&0.1663&0.0901&0.2686&0.7958&0.1707&0.1839&0.1577&-2.79     \\
FAMO            &0.0189&0.0200&0.1285&0.1745&0.1720&0.0934&0.2715&0.7929&0.1807&0.1891&0.1640&-7.21     \\
Ours            &0.0167&0.0169&0.1228&0.1739&0.1695&0.0910&0.2344&0.7600&0.1691&0.1836&0.1571&-0.64     \\  \midrule[1.0pt]
\end{tabular}}
\label{tab:tab_exp_taskonomy_vitB}
\end{table*}
%%%%%%%%%%%%%%%%%%%%%%%%%%%%%%%%
\begin{table*}[t]
\caption{Comparison with previous multi-task optimization approaches on Taskonomy with ViT-S.}
\vspace{-5pt}
\centering
\renewcommand\arraystretch{1.00}
\resizebox{0.99\textwidth}{!}{
\begin{tabular}{l|ccccccccccc|c}
\midrule[1.0pt]
 & DE & DZ & EO & ET & K2  & K3 & N   & C & R & S2  & S2.5 &  \\ \cmidrule[0.5pt]{2-12}
\multirow{-2}{*}{Task} & L1 Dist. $\downarrow$  & L1 Dist. $\downarrow$ & L1 Dist. $\downarrow$ & L1 Dist. $\downarrow$ & L1 Dist. $\downarrow$ & L1 Dist. $\downarrow$ & L1 Dist. $\downarrow$ & RMSE $\downarrow$    & L1 Dist. $\downarrow$ & L1 Dist. $\downarrow$ & L1 Dist. $\downarrow$  & \multirow{-2}{*}{$\triangle_m$ ($\uparrow$)} \\ \midrule[1.0pt]
Single Task     &0.0264&0.0259&0.1348&0.1740&0.1667&0.0973&0.3481&0.8598&0.1905&0.1857&0.1691&-    \\ \midrule[0.5pt]
GD              &0.0264&0.0272&0.1574&0.1775&0.1838&0.1038&0.4370&0.9237&0.2475&0.2076&0.1858&-11.39     \\
GradDrop        &0.0274&0.0280&0.1609&0.1779&0.1856&0.1042&0.4472&0.9366&0.2549&0.2106&0.1821&-13.14     \\
MGDA            &-&-&-&-&-&-&-&-&-&-&-&-     \\
UW              &0.0263&0.0269&0.1570&0.1775&0.1832&0.1037&0.4362&0.9202&0.2465&0.2075&0.1856&-11.11     \\
DTP             &0.0262&0.0273&0.1568&0.1778&0.1831&0.1037&0.4884&0.9207&0.2466&0.2073&0.1849&-12.52     \\
DWA             &0.0264&0.0271&0.1572&0.1776&0.1834&0.1038&0.4336&0.9215&0.2469&0.2075&0.1856&-11.20     \\
PCGrad          &0.0271&0.0274&0.1570&0.1766&0.1784&0.1034&0.4522&0.9343&0.2525&0.2071&0.1811&-11.78     \\
CAGrad          &0.0289&0.0282&0.1611&0.1769&0.1706&0.1062&0.4723&0.9557&0.2689&0.2122&0.1902&-15.09     \\
IMTL-L          &0.0255&0.0258&0.1510&0.1744&0.1716&0.1005&0.4339&0.9459&0.2466&0.2036&0.1825&-8.90     \\
Aligned-MTL     &0.0286&0.0290&0.1603&0.1744&0.1711&0.1033&0.4596&1.0022&0.2783&0.2090&0.1854&-15.06     \\
Nash-MTL        &0.0255&0.0258&0.1510&0.1744&0.1716&0.1005&0.4339&0.9459&0.2466&0.2036&0.1825&-8.79     \\
FAMO            &0.0263&0.0272&0.1573&0.1774&0.1835&0.1035&0.4326&0.9208&0.2464&0.2077&0.1858&-11.23     \\
Ours            &0.0225&0.0229&0.1444&0.1762&0.1775&0.0995&0.3983&0.8620&0.2156&0.1997&0.1774&-2.83     \\  \midrule[1.0pt]
\end{tabular}}
\label{tab:tab_exp_taskonomy_vitS}
\end{table*}
%%%%%%%%%%%%%%%%%%%%%%%%%%%%%%%%
\begin{table*}[t]
\caption{Comparison with previous multi-task optimization approaches on Taskonomy with ViT-T.}
\vspace{-5pt}
\centering
\renewcommand\arraystretch{1.00}
\resizebox{0.99\textwidth}{!}{
\begin{tabular}{l|ccccccccccc|c}
\midrule[1.0pt]
 & DE & DZ & EO & ET & K2  & K3 & N   & C & R & S2  & S2.5 &  \\ \cmidrule[0.5pt]{2-12}
\multirow{-2}{*}{Task} & L1 Dist. $\downarrow$  & L1 Dist. $\downarrow$ & L1 Dist. $\downarrow$ & L1 Dist. $\downarrow$ & L1 Dist. $\downarrow$ & L1 Dist. $\downarrow$ & L1 Dist. $\downarrow$ & RMSE $\downarrow$    & L1 Dist. $\downarrow$ & L1 Dist. $\downarrow$ & L1 Dist. $\downarrow$  & \multirow{-2}{*}{$\triangle_m$ ($\uparrow$)} \\ \midrule[1.0pt]
Single Task     &0.0289&0.0290&0.1405&0.1774&0.1682&0.0970&0.3837&0.8968&0.2096&0.1904&0.1729&-    \\ \midrule[0.5pt]
GD              &0.0279&0.0285&0.1604&0.1789&0.1860&0.1043&0.4704&0.9488&0.2613&0.2086&0.1914&-9.21     \\
GradDrop        &0.0287&0.0292&0.1630&0.1795&0.1868&0.1052&0.4795&0.9621&0.2697&0.2118&0.1878&-10.68     \\
MGDA            &-&-&-&-&-&-&-&-&-&-&-&-     \\
UW              &0.0279&0.0285&0.1604&0.1789&0.1859&0.1043&0.4699&0.9488&0.2613&0.2085&0.1914&-9.21     \\
DTP             &0.0278&0.0288&0.1603&0.1790&0.1859&0.1042&0.4697&0.9488&0.2614&0.2088&0.1915&-9.27     \\
DWA             &0.0279&0.0285&0.1604&0.1789&0.1859&0.1043&0.4693&0.9489&0.2613&0.2086&0.1913&-9.20     \\
PCGrad          &0.0283&0.0290&0.1604&0.1769&0.1803&0.1036&0.4720&0.9645&0.2683&0.2090&0.1866&-9.28     \\
CAGrad          &0.0300&0.0304&0.1644&0.1743&0.1721&0.1055&0.4838&0.9818&0.2847&0.2143&0.1974&-12.12     \\
IMTL            &0.0276&0.0282&0.1553&0.1754&0.1743&0.1018&0.4621&0.9809&0.2623&0.2051&0.1878&-7.55     \\
Aligned-MTL     &0.0296&0.0318&0.1633&0.1765&0.1757&0.1150&0.4806&1.0270&0.2935&0.2109&0.1887&-13.70     \\
Nash-MTL        &0.0276&0.0282&0.1553&0.1754&0.1743&0.1018&0.4621&0.9809&0.2623&0.2051&0.1878&-7.46    \\
FAMO            &0.0279&0.0285&0.1604&0.1789&0.1859&0.1043&0.4718&0.9488&0.2612&0.2085&0.1913&-9.33     \\
Ours            &0.0252&0.0257&0.1526&0.1774&0.1827&0.1019&0.4337&0.9100&0.2402&0.2026&0.1845&-3.67     \\  \midrule[1.0pt]
\end{tabular}}
\label{tab:tab_exp_taskonomy_vitT}
\end{table*}
%%%%%%%%%%%%%%%%%%%%%%%%%%%%%%%%


% %-------------------------------------------------------------------------
\def\figlength{0.14}
\begin{figure}[h]
\centering  
\begin{subfigure}{\figlength\textwidth}
\includegraphics[width=0.99\textwidth]{figure/vit_taskonomy/DE_to_DZ.png}
\end{subfigure}
\begin{subfigure}{\figlength\textwidth}
\includegraphics[width=0.99\textwidth]{figure/vit_taskonomy/DE_to_EO.png}
\end{subfigure}
\begin{subfigure}{\figlength\textwidth}
\includegraphics[width=0.99\textwidth]{figure/vit_taskonomy/DE_to_ET.png}
\end{subfigure}
\begin{subfigure}{\figlength\textwidth}
\includegraphics[width=0.99\textwidth]{figure/vit_taskonomy/DE_to_K2.png}
\end{subfigure}
\begin{subfigure}{\figlength\textwidth}
\includegraphics[width=0.99\textwidth]{figure/vit_taskonomy/DE_to_K3.png}
\end{subfigure}
\begin{subfigure}{\figlength\textwidth}
\includegraphics[width=0.99\textwidth]{figure/vit_taskonomy/DE_to_N.png}
\end{subfigure}
\begin{subfigure}{\figlength\textwidth}
\includegraphics[width=0.99\textwidth]{figure/vit_taskonomy/DE_to_C.png}
\end{subfigure}
\begin{subfigure}{\figlength\textwidth}
\includegraphics[width=0.99\textwidth]{figure/vit_taskonomy/DE_to_R.png}
\end{subfigure}
\begin{subfigure}{\figlength\textwidth}
\includegraphics[width=0.99\textwidth]{figure/vit_taskonomy/DE_to_S2.png}
\end{subfigure}
\begin{subfigure}{\figlength\textwidth}
\includegraphics[width=0.99\textwidth]{figure/vit_taskonomy/DE_to_S2.5.png}
\end{subfigure}
\begin{subfigure}{\figlength\textwidth}
\includegraphics[width=0.99\textwidth]{figure/vit_taskonomy/DZ_to_DE.png}
\end{subfigure}
\begin{subfigure}{\figlength\textwidth}
\includegraphics[width=0.99\textwidth]{figure/vit_taskonomy/DZ_to_EO.png}
\end{subfigure}
\begin{subfigure}{\figlength\textwidth}
\includegraphics[width=0.99\textwidth]{figure/vit_taskonomy/DZ_to_ET.png}
\end{subfigure}
\begin{subfigure}{\figlength\textwidth}
\includegraphics[width=0.99\textwidth]{figure/vit_taskonomy/DZ_to_K2.png}
\end{subfigure}
\begin{subfigure}{\figlength\textwidth}
\includegraphics[width=0.99\textwidth]{figure/vit_taskonomy/DZ_to_K3.png}
\end{subfigure}
\begin{subfigure}{\figlength\textwidth}
\includegraphics[width=0.99\textwidth]{figure/vit_taskonomy/DZ_to_N.png}
\end{subfigure}
\begin{subfigure}{\figlength\textwidth}
\includegraphics[width=0.99\textwidth]{figure/vit_taskonomy/DZ_to_C.png}
\end{subfigure}
\begin{subfigure}{\figlength\textwidth}
\includegraphics[width=0.99\textwidth]{figure/vit_taskonomy/DZ_to_R.png}
\end{subfigure}
\begin{subfigure}{\figlength\textwidth}
\includegraphics[width=0.99\textwidth]{figure/vit_taskonomy/DZ_to_S2.png}
\end{subfigure}
\begin{subfigure}{\figlength\textwidth}
\includegraphics[width=0.99\textwidth]{figure/vit_taskonomy/DZ_to_S2.5.png}
\end{subfigure}
\begin{subfigure}{\figlength\textwidth}
\includegraphics[width=0.99\textwidth]{figure/vit_taskonomy/EO_to_DE.png}
\end{subfigure}
\begin{subfigure}{\figlength\textwidth}
\includegraphics[width=0.99\textwidth]{figure/vit_taskonomy/EO_to_DZ.png}
\end{subfigure}
\begin{subfigure}{\figlength\textwidth}
\includegraphics[width=0.99\textwidth]{figure/vit_taskonomy/EO_to_ET.png}
\end{subfigure}
\begin{subfigure}{\figlength\textwidth}
\includegraphics[width=0.99\textwidth]{figure/vit_taskonomy/EO_to_K2.png}
\end{subfigure}
\begin{subfigure}{\figlength\textwidth}
\includegraphics[width=0.99\textwidth]{figure/vit_taskonomy/EO_to_K3.png}
\end{subfigure}
\begin{subfigure}{\figlength\textwidth}
\includegraphics[width=0.99\textwidth]{figure/vit_taskonomy/EO_to_N.png}
\end{subfigure}
\begin{subfigure}{\figlength\textwidth}
\includegraphics[width=0.99\textwidth]{figure/vit_taskonomy/EO_to_C.png}
\end{subfigure}
\begin{subfigure}{\figlength\textwidth}
\includegraphics[width=0.99\textwidth]{figure/vit_taskonomy/EO_to_R.png}
\end{subfigure}
\begin{subfigure}{\figlength\textwidth}
\includegraphics[width=0.99\textwidth]{figure/vit_taskonomy/EO_to_S2.png}
\end{subfigure}
\begin{subfigure}{\figlength\textwidth}
\includegraphics[width=0.99\textwidth]{figure/vit_taskonomy/EO_to_S2.5.png}
\end{subfigure}
\begin{subfigure}{\figlength\textwidth}
\includegraphics[width=0.99\textwidth]{figure/vit_taskonomy/ET_to_DE.png}
\end{subfigure}
\begin{subfigure}{\figlength\textwidth}
\includegraphics[width=0.99\textwidth]{figure/vit_taskonomy/ET_to_DZ.png}
\end{subfigure}
\begin{subfigure}{\figlength\textwidth}
\includegraphics[width=0.99\textwidth]{figure/vit_taskonomy/ET_to_EO.png}
\end{subfigure}
\begin{subfigure}{\figlength\textwidth}
\includegraphics[width=0.99\textwidth]{figure/vit_taskonomy/ET_to_K2.png}
\end{subfigure}
\begin{subfigure}{\figlength\textwidth}
\includegraphics[width=0.99\textwidth]{figure/vit_taskonomy/ET_to_K3.png}
\end{subfigure}
\begin{subfigure}{\figlength\textwidth}
\includegraphics[width=0.99\textwidth]{figure/vit_taskonomy/ET_to_N.png}
\end{subfigure}
\begin{subfigure}{\figlength\textwidth}
\includegraphics[width=0.99\textwidth]{figure/vit_taskonomy/ET_to_C.png}
\end{subfigure}
\begin{subfigure}{\figlength\textwidth}
\includegraphics[width=0.99\textwidth]{figure/vit_taskonomy/ET_to_R.png}
\end{subfigure}
\begin{subfigure}{\figlength\textwidth}
\includegraphics[width=0.99\textwidth]{figure/vit_taskonomy/ET_to_S2.png}
\end{subfigure}
\begin{subfigure}{\figlength\textwidth}
\includegraphics[width=0.99\textwidth]{figure/vit_taskonomy/ET_to_S2.5.png}
\end{subfigure}
\begin{subfigure}{\figlength\textwidth}
\includegraphics[width=0.99\textwidth]{figure/vit_taskonomy/K2_to_DE.png}
\end{subfigure}
\begin{subfigure}{\figlength\textwidth}
\includegraphics[width=0.99\textwidth]{figure/vit_taskonomy/K2_to_DZ.png}
\end{subfigure}
\begin{subfigure}{\figlength\textwidth}
\includegraphics[width=0.99\textwidth]{figure/vit_taskonomy/K2_to_EO.png}
\end{subfigure}
\begin{subfigure}{\figlength\textwidth}
\includegraphics[width=0.99\textwidth]{figure/vit_taskonomy/K2_to_ET.png}
\end{subfigure}
\begin{subfigure}{\figlength\textwidth}
\includegraphics[width=0.99\textwidth]{figure/vit_taskonomy/K2_to_K3.png}
\end{subfigure}
\begin{subfigure}{\figlength\textwidth}
\includegraphics[width=0.99\textwidth]{figure/vit_taskonomy/K2_to_N.png}
\end{subfigure}
\begin{subfigure}{\figlength\textwidth}
\includegraphics[width=0.99\textwidth]{figure/vit_taskonomy/K2_to_C.png}
\end{subfigure}
\begin{subfigure}{\figlength\textwidth}
\includegraphics[width=0.99\textwidth]{figure/vit_taskonomy/K2_to_R.png}
\end{subfigure}
\begin{subfigure}{\figlength\textwidth}
\includegraphics[width=0.99\textwidth]{figure/vit_taskonomy/K2_to_S2.png}
\end{subfigure}
\begin{subfigure}{\figlength\textwidth}
\includegraphics[width=0.99\textwidth]{figure/vit_taskonomy/K2_to_S2.5.png}
\end{subfigure}
\begin{subfigure}{\figlength\textwidth}
\includegraphics[width=0.99\textwidth]{figure/vit_taskonomy/K3_to_DE.png}
\end{subfigure}
\begin{subfigure}{\figlength\textwidth}
\includegraphics[width=0.99\textwidth]{figure/vit_taskonomy/K3_to_DZ.png}
\end{subfigure}
\begin{subfigure}{\figlength\textwidth}
\includegraphics[width=0.99\textwidth]{figure/vit_taskonomy/K3_to_EO.png}
\end{subfigure}
\begin{subfigure}{\figlength\textwidth}
\includegraphics[width=0.99\textwidth]{figure/vit_taskonomy/K3_to_ET.png}
\end{subfigure}
\begin{subfigure}{\figlength\textwidth}
\includegraphics[width=0.99\textwidth]{figure/vit_taskonomy/K3_to_K2.png}
\end{subfigure}
\begin{subfigure}{\figlength\textwidth}
\includegraphics[width=0.99\textwidth]{figure/vit_taskonomy/K3_to_N.png}
\end{subfigure}
\begin{subfigure}{\figlength\textwidth}
\includegraphics[width=0.99\textwidth]{figure/vit_taskonomy/K3_to_C.png}
\end{subfigure}
\begin{subfigure}{\figlength\textwidth}
\includegraphics[width=0.99\textwidth]{figure/vit_taskonomy/K3_to_R.png}
\end{subfigure}
\begin{subfigure}{\figlength\textwidth}
\includegraphics[width=0.99\textwidth]{figure/vit_taskonomy/K3_to_S2.png}
\end{subfigure}
\begin{subfigure}{\figlength\textwidth}
\includegraphics[width=0.99\textwidth]{figure/vit_taskonomy/K3_to_S2.5.png}
\end{subfigure}
\begin{subfigure}{\figlength\textwidth}
\includegraphics[width=0.99\textwidth]{figure/vit_taskonomy/N_to_DE.png}
\end{subfigure}
\begin{subfigure}{\figlength\textwidth}
\includegraphics[width=0.99\textwidth]{figure/vit_taskonomy/N_to_DZ.png}
\end{subfigure}
\begin{subfigure}{\figlength\textwidth}
\includegraphics[width=0.99\textwidth]{figure/vit_taskonomy/N_to_EO.png}
\end{subfigure}
\begin{subfigure}{\figlength\textwidth}
\includegraphics[width=0.99\textwidth]{figure/vit_taskonomy/N_to_ET.png}
\end{subfigure}
\begin{subfigure}{\figlength\textwidth}
\includegraphics[width=0.99\textwidth]{figure/vit_taskonomy/N_to_K2.png}
\end{subfigure}
\begin{subfigure}{\figlength\textwidth}
\includegraphics[width=0.99\textwidth]{figure/vit_taskonomy/N_to_K3.png}
\end{subfigure}
\caption{Changes in proximal inter-task affinity during the optimization process of ViT-L with Taskonomy benchmark.}
\end{figure}

\begin{figure}[h]\ContinuedFloat
\centering
\begin{subfigure}{\figlength\textwidth}
\includegraphics[width=0.99\textwidth]{figure/vit_taskonomy/N_to_C.png}
\end{subfigure}
\begin{subfigure}{\figlength\textwidth}
\includegraphics[width=0.99\textwidth]{figure/vit_taskonomy/N_to_R.png}
\end{subfigure}
\begin{subfigure}{\figlength\textwidth}
\includegraphics[width=0.99\textwidth]{figure/vit_taskonomy/N_to_S2.png}
\end{subfigure}
\begin{subfigure}{\figlength\textwidth}
\includegraphics[width=0.99\textwidth]{figure/vit_taskonomy/N_to_S2.5.png}
\end{subfigure}
\begin{subfigure}{\figlength\textwidth}
\includegraphics[width=0.99\textwidth]{figure/vit_taskonomy/C_to_DE.png}
\end{subfigure}
\begin{subfigure}{\figlength\textwidth}
\includegraphics[width=0.99\textwidth]{figure/vit_taskonomy/C_to_DZ.png}
\end{subfigure}
\begin{subfigure}{\figlength\textwidth}
\includegraphics[width=0.99\textwidth]{figure/vit_taskonomy/C_to_EO.png}
\end{subfigure}
\begin{subfigure}{\figlength\textwidth}
\includegraphics[width=0.99\textwidth]{figure/vit_taskonomy/C_to_ET.png}
\end{subfigure}
\begin{subfigure}{\figlength\textwidth}
\includegraphics[width=0.99\textwidth]{figure/vit_taskonomy/C_to_K2.png}
\end{subfigure}
\begin{subfigure}{\figlength\textwidth}
\includegraphics[width=0.99\textwidth]{figure/vit_taskonomy/C_to_K3.png}
\end{subfigure}
\begin{subfigure}{\figlength\textwidth}
\includegraphics[width=0.99\textwidth]{figure/vit_taskonomy/C_to_N.png}
\end{subfigure}
\begin{subfigure}{\figlength\textwidth}
\includegraphics[width=0.99\textwidth]{figure/vit_taskonomy/C_to_R.png}
\end{subfigure}
\begin{subfigure}{\figlength\textwidth}
\includegraphics[width=0.99\textwidth]{figure/vit_taskonomy/C_to_S2.png}
\end{subfigure}
\begin{subfigure}{\figlength\textwidth}
\includegraphics[width=0.99\textwidth]{figure/vit_taskonomy/C_to_S2.5.png}
\end{subfigure}
\begin{subfigure}{\figlength\textwidth}
\includegraphics[width=0.99\textwidth]{figure/vit_taskonomy/R_to_DE.png}
\end{subfigure}
\begin{subfigure}{\figlength\textwidth}
\includegraphics[width=0.99\textwidth]{figure/vit_taskonomy/R_to_DZ.png}
\end{subfigure}
\begin{subfigure}{\figlength\textwidth}
\includegraphics[width=0.99\textwidth]{figure/vit_taskonomy/R_to_EO.png}
\end{subfigure}
\begin{subfigure}{\figlength\textwidth}
\includegraphics[width=0.99\textwidth]{figure/vit_taskonomy/R_to_ET.png}
\end{subfigure}
\begin{subfigure}{\figlength\textwidth}
\includegraphics[width=0.99\textwidth]{figure/vit_taskonomy/R_to_K2.png}
\end{subfigure}
\begin{subfigure}{\figlength\textwidth}
\includegraphics[width=0.99\textwidth]{figure/vit_taskonomy/R_to_K3.png}
\end{subfigure}
\begin{subfigure}{\figlength\textwidth}
\includegraphics[width=0.99\textwidth]{figure/vit_taskonomy/R_to_N.png}
\end{subfigure}
\begin{subfigure}{\figlength\textwidth}
\includegraphics[width=0.99\textwidth]{figure/vit_taskonomy/R_to_C.png}
\end{subfigure}
\begin{subfigure}{\figlength\textwidth}
\includegraphics[width=0.99\textwidth]{figure/vit_taskonomy/R_to_S2.png}
\end{subfigure}
\begin{subfigure}{\figlength\textwidth}
\includegraphics[width=0.99\textwidth]{figure/vit_taskonomy/R_to_S2.5.png}
\end{subfigure}
\begin{subfigure}{\figlength\textwidth}
\includegraphics[width=0.99\textwidth]{figure/vit_taskonomy/S2_to_DE.png}
\end{subfigure}
\begin{subfigure}{\figlength\textwidth}
\includegraphics[width=0.99\textwidth]{figure/vit_taskonomy/S2_to_DZ.png}
\end{subfigure}
\begin{subfigure}{\figlength\textwidth}
\includegraphics[width=0.99\textwidth]{figure/vit_taskonomy/S2_to_EO.png}
\end{subfigure}
\begin{subfigure}{\figlength\textwidth}
\includegraphics[width=0.99\textwidth]{figure/vit_taskonomy/S2_to_ET.png}
\end{subfigure}
\begin{subfigure}{\figlength\textwidth}
\includegraphics[width=0.99\textwidth]{figure/vit_taskonomy/S2_to_K2.png}
\end{subfigure}
\begin{subfigure}{\figlength\textwidth}
\includegraphics[width=0.99\textwidth]{figure/vit_taskonomy/S2_to_K3.png}
\end{subfigure}
\begin{subfigure}{\figlength\textwidth}
\includegraphics[width=0.99\textwidth]{figure/vit_taskonomy/S2_to_N.png}
\end{subfigure}
\begin{subfigure}{\figlength\textwidth}
\includegraphics[width=0.99\textwidth]{figure/vit_taskonomy/S2_to_C.png}
\end{subfigure}
\begin{subfigure}{\figlength\textwidth}
\includegraphics[width=0.99\textwidth]{figure/vit_taskonomy/S2_to_R.png}
\end{subfigure}
\begin{subfigure}{\figlength\textwidth}
\includegraphics[width=0.99\textwidth]{figure/vit_taskonomy/S2_to_S2.5.png}
\end{subfigure}
\begin{subfigure}{\figlength\textwidth}
\includegraphics[width=0.99\textwidth]{figure/vit_taskonomy/S2.5_to_DE.png}
\end{subfigure}
\begin{subfigure}{\figlength\textwidth}
\includegraphics[width=0.99\textwidth]{figure/vit_taskonomy/S2.5_to_DZ.png}
\end{subfigure}
\begin{subfigure}{\figlength\textwidth}
\includegraphics[width=0.99\textwidth]{figure/vit_taskonomy/S2.5_to_EO.png}
\end{subfigure}
\begin{subfigure}{\figlength\textwidth}
\includegraphics[width=0.99\textwidth]{figure/vit_taskonomy/S2.5_to_ET.png}
\end{subfigure}
\begin{subfigure}{\figlength\textwidth}
\includegraphics[width=0.99\textwidth]{figure/vit_taskonomy/S2.5_to_K2.png}
\end{subfigure}
\begin{subfigure}{\figlength\textwidth}
\includegraphics[width=0.99\textwidth]{figure/vit_taskonomy/S2.5_to_K3.png}
\end{subfigure}
\begin{subfigure}{\figlength\textwidth}
\includegraphics[width=0.99\textwidth]{figure/vit_taskonomy/S2.5_to_N.png}
\end{subfigure}
\begin{subfigure}{\figlength\textwidth}
\includegraphics[width=0.99\textwidth]{figure/vit_taskonomy/S2.5_to_C.png}
\end{subfigure}
\begin{subfigure}{\figlength\textwidth}
\includegraphics[width=0.99\textwidth]{figure/vit_taskonomy/S2.5_to_R.png}
\end{subfigure}
\begin{subfigure}{\figlength\textwidth}
\includegraphics[width=0.99\textwidth]{figure/vit_taskonomy/S2.5_to_S2.png}
\end{subfigure}
\caption{Changes in proximal inter-task affinity during the optimization process with Taskonomy benchmark.}
\label{fig:proximal_vit_taskonomy}
\end{figure}
\clearpage


%-------------------------------------------------------------------------
\begin{figure}[h]
    \centering
    \begin{subfigure}{0.24\textwidth}
        \includegraphics[width=0.99\textwidth]{figure/vit_nyud/semseg_to_depth.png}
    \end{subfigure}
    \begin{subfigure}{0.24\textwidth}
        \includegraphics[width=0.99\textwidth]{figure/vit_nyud/semseg_to_normals.png}
    \end{subfigure}
    \begin{subfigure}{0.24\textwidth}
        \includegraphics[width=0.99\textwidth]{figure/vit_nyud/semseg_to_edge.png}
    \end{subfigure}
    \begin{subfigure}{0.24\textwidth}
        \includegraphics[width=0.99\textwidth]{figure/vit_nyud/depth_to_semseg.png}
    \end{subfigure}
    \hfill
    \begin{subfigure}{0.24\textwidth}
        \includegraphics[width=0.99\textwidth]{figure/vit_nyud/depth_to_normals.png}
    \end{subfigure}
    \begin{subfigure}{0.24\textwidth}
        \includegraphics[width=0.99\textwidth]{figure/vit_nyud/depth_to_edge.png}
    \end{subfigure}
    \begin{subfigure}{0.24\textwidth}
        \includegraphics[width=0.99\textwidth]{figure/vit_nyud/normals_to_semseg.png}
    \end{subfigure}
    \begin{subfigure}{0.24\textwidth}
        \includegraphics[width=0.99\textwidth]{figure/vit_nyud/normals_to_depth.png}
    \end{subfigure}
    \hfill
    \begin{subfigure}{0.24\textwidth}
        \includegraphics[width=0.99\textwidth]{figure/vit_nyud/normals_to_edge.png}
    \end{subfigure}
    \begin{subfigure}{0.24\textwidth}
        \includegraphics[width=0.99\textwidth]{figure/vit_nyud/edge_to_semseg.png}
    \end{subfigure}
    \begin{subfigure}{0.24\textwidth}
        \includegraphics[width=0.99\textwidth]{figure/vit_nyud/edge_to_depth.png}
    \end{subfigure}
    \begin{subfigure}{0.24\textwidth}
        \includegraphics[width=0.99\textwidth]{figure/vit_nyud/edge_to_normals.png}
    \end{subfigure}
    \caption{Changes in the proximal inter-task affinity during the optimization process of different sizes of ViT with NYUD-v2.}
    \label{fig:proximal_vit_nyud}
\end{figure}
%-------------------------------------------------------------------------
\begin{figure}[h]
    \centering
    \begin{subfigure}{0.24\textwidth}
        \includegraphics[width=0.99\textwidth]{figure/vit_pascal/semseg_to_human_parts.png}
    \end{subfigure}
    \begin{subfigure}{0.24\textwidth}
        \includegraphics[width=0.99\textwidth]{figure/vit_pascal/semseg_to_sal.png}
    \end{subfigure}
    \begin{subfigure}{0.24\textwidth}
        \includegraphics[width=0.99\textwidth]{figure/vit_pascal/semseg_to_normals.png}
    \end{subfigure}
    \begin{subfigure}{0.24\textwidth}
        \includegraphics[width=0.99\textwidth]{figure/vit_pascal/semseg_to_edge.png}
    \end{subfigure}
    \hfill
    \begin{subfigure}{0.24\textwidth}
        \includegraphics[width=0.99\textwidth]{figure/vit_pascal/human_parts_to_semseg.png}
    \end{subfigure}
    \begin{subfigure}{0.24\textwidth}
        \includegraphics[width=0.99\textwidth]{figure/vit_pascal/human_parts_to_sal.png}
    \end{subfigure}
    \begin{subfigure}{0.24\textwidth}
        \includegraphics[width=0.99\textwidth]{figure/vit_pascal/human_parts_to_normals.png}
    \end{subfigure}
    \begin{subfigure}{0.24\textwidth}
        \includegraphics[width=0.99\textwidth]{figure/vit_pascal/human_parts_to_edge.png}
    \end{subfigure}
    \hfill
    \begin{subfigure}{0.24\textwidth}
        \includegraphics[width=0.99\textwidth]{figure/vit_pascal/sal_to_semseg.png}
    \end{subfigure}
    \begin{subfigure}{0.24\textwidth}
        \includegraphics[width=0.99\textwidth]{figure/vit_pascal/sal_to_human_parts.png}
    \end{subfigure}
    \begin{subfigure}{0.24\textwidth}
        \includegraphics[width=0.99\textwidth]{figure/vit_pascal/sal_to_normals.png}
    \end{subfigure}
    \begin{subfigure}{0.24\textwidth}
        \includegraphics[width=0.99\textwidth]{figure/vit_pascal/sal_to_edge.png}
    \end{subfigure}
        \hfill
    \begin{subfigure}{0.24\textwidth}
        \includegraphics[width=0.99\textwidth]{figure/vit_pascal/normals_to_semseg.png}
    \end{subfigure}
    \begin{subfigure}{0.24\textwidth}
        \includegraphics[width=0.99\textwidth]{figure/vit_pascal/normals_to_human_parts.png}
    \end{subfigure}
    \begin{subfigure}{0.24\textwidth}
        \includegraphics[width=0.99\textwidth]{figure/vit_pascal/normals_to_sal.png}
    \end{subfigure}
    \begin{subfigure}{0.24\textwidth}
        \includegraphics[width=0.99\textwidth]{figure/vit_pascal/normals_to_edge.png}
    \end{subfigure}
        \hfill
    \begin{subfigure}{0.24\textwidth}
        \includegraphics[width=0.99\textwidth]{figure/vit_pascal/edge_to_semseg.png}
    \end{subfigure}
    \begin{subfigure}{0.24\textwidth}
        \includegraphics[width=0.99\textwidth]{figure/vit_pascal/edge_to_human_parts.png}
    \end{subfigure}
    \begin{subfigure}{0.24\textwidth}
        \includegraphics[width=0.99\textwidth]{figure/vit_pascal/edge_to_sal.png}
    \end{subfigure}
    \begin{subfigure}{0.24\textwidth}
        \includegraphics[width=0.99\textwidth]{figure/vit_pascal/edge_to_normals.png}
    \end{subfigure}
    \caption{Changes in proximal inter-task affinity during the optimization process of different sizes of ViT with PASCAL-Context.}
    \label{fig:proximal_vit_pascal}
\end{figure}



\begin{figure}[h]
    \centering
    \begin{subfigure}{0.24\textwidth}
        \includegraphics[width=0.99\textwidth]{figure/vit_beta/semseg_to_depth.png}
    \end{subfigure}
    \begin{subfigure}{0.24\textwidth}
        \includegraphics[width=0.99\textwidth]{figure/vit_beta/semseg_to_normals.png}
    \end{subfigure}
    \begin{subfigure}{0.24\textwidth}
        \includegraphics[width=0.99\textwidth]{figure/vit_beta/semseg_to_edge.png}
    \end{subfigure}
    \begin{subfigure}{0.24\textwidth}
        \includegraphics[width=0.99\textwidth]{figure/vit_beta/depth_to_semseg.png}
    \end{subfigure}
    \hfill
    \begin{subfigure}{0.24\textwidth}
        \includegraphics[width=0.99\textwidth]{figure/vit_beta/depth_to_normals.png}
    \end{subfigure}
    \begin{subfigure}{0.24\textwidth}
        \includegraphics[width=0.99\textwidth]{figure/vit_beta/depth_to_edge.png}
    \end{subfigure}
    \begin{subfigure}{0.24\textwidth}
        \includegraphics[width=0.99\textwidth]{figure/vit_beta/normals_to_semseg.png}
    \end{subfigure}
    \begin{subfigure}{0.24\textwidth}
        \includegraphics[width=0.99\textwidth]{figure/vit_beta/normals_to_depth.png}
    \end{subfigure}
    \hfill
    \begin{subfigure}{0.24\textwidth}
        \includegraphics[width=0.99\textwidth]{figure/vit_beta/normals_to_edge.png}
    \end{subfigure}
    \begin{subfigure}{0.24\textwidth}
        \includegraphics[width=0.99\textwidth]{figure/vit_beta/edge_to_semseg.png}
    \end{subfigure}
    \begin{subfigure}{0.24\textwidth}
        \includegraphics[width=0.99\textwidth]{figure/vit_beta/edge_to_depth.png}
    \end{subfigure}
    \begin{subfigure}{0.24\textwidth}
        \includegraphics[width=0.99\textwidth]{figure/vit_beta/edge_to_normals.png}
    \end{subfigure}
    \caption{Changes in proximal inter-task affinity during the optimization process with different decay rates, $\beta$.}
    \label{fig:proximal_vit_beta}
\end{figure}



\begin{table*}[h]
\caption{Results on Taskonomy with different affinity decay rates $\beta$.}
\vspace{-5pt}
\centering
\renewcommand\arraystretch{1.00}
\resizebox{0.99\textwidth}{!}{
\begin{tabular}{l|ccccccccccc|c}
\midrule[1.0pt]
 & DE & DZ & EO & ET & K2  & K3 & N   & C & R & S2  & S2.5 &  \\ \cmidrule[0.5pt]{2-12}
\multirow{-2}{*}{Task} & L1 Dist. $\downarrow$  & L1 Dist. $\downarrow$ & L1 Dist. $\downarrow$ & L1 Dist. $\downarrow$ & L1 Dist. $\downarrow$ & L1 Dist. $\downarrow$ & L1 Dist. $\downarrow$ & RMSE $\downarrow$    & L1 Dist. $\downarrow$ & L1 Dist. $\downarrow$ & L1 Dist. $\downarrow$  & \multirow{-2}{*}{$\triangle_m$ ($\uparrow$)} \\ \midrule[1.0pt]
Single Task     &0.0183&0.0186&0.1089&0.1713&0.1630&0.0863&0.2953&0.7522&0.1504&0.1738&0.1530&-         \\\midrule[0.5pt]
GD              &0.0188&0.0197&0.1283&0.1745&0.1718&0.0933&0.2599&0.7911&0.1799&0.1885&0.1631&-6.35     \\
$\beta$=0.0001  &0.0165&0.0168&0.1224&0.1739&0.1693&0.0907&0.2304&0.7581&0.1683&0.1831&0.1571&-0.18     \\
$\beta$=0.001   &0.0167&0.0169&0.1228&0.1739&0.1695&0.0910&0.2344&0.7600&0.1691&0.1836&0.1571&-0.64     \\
$\beta$=0.01    &0.0167&0.0171&0.1232&0.1739&0.1698&0.0912&0.2362&0.7623&0.1705&0.1834&0.1576&-1.01     \\
$\beta$=0.1     &0.0167&0.0171&0.1231&0.1739&0.1695&0.0912&0.2355&0.7631&0.1697&0.1831&0.1575&-0.87     \\\midrule[1.0pt]
\end{tabular}}
\label{tab:tab_exp_beta_perf}
\end{table*}


\begin{table*}[h]
\caption{Comparison of different grouping strategies on the Taskonomy benchmark.}
\vspace{-5pt}
\centering
\renewcommand\arraystretch{1.00}
\resizebox{0.99\textwidth}{!}{
\begin{tabular}{l|ccccccccccc|c}
\midrule[1.0pt]
 & DE & DZ & EO & ET & K2  & K3 & N   & C & R & S2  & S2.5 &  \\ \cmidrule[0.5pt]{2-12}
\multirow{-2}{*}{Task} & L1 Dist. $\downarrow$  & L1 Dist. $\downarrow$ & L1 Dist. $\downarrow$ & L1 Dist. $\downarrow$ & L1 Dist. $\downarrow$ & L1 Dist. $\downarrow$ & L1 Dist. $\downarrow$ & RMSE $\downarrow$    & L1 Dist. $\downarrow$ & L1 Dist. $\downarrow$ & L1 Dist. $\downarrow$  & \multirow{-2}{*}{$\triangle_m$ ($\uparrow$)} \\ \midrule[1.0pt]
Heterogeneous               &0.0172&0.0176&0.1252&0.1741&0.1700&0.0920&0.2475&0.7781&0.1743&0.1849&0.1660&-3.10 \\
Random ($N(\mathcal{M})$=2) &0.0177&0.0180&0.1259&0.1741&0.1707&0.0923&0.2662&0.7807&0.1757&0.1871&0.1617&-4.24 \\
Random ($N(\mathcal{M})$=3) &0.0172&0.0177&0.1250&0.1741&0.1703&0.0920&0.2619&0.7754&0.1749&0.1866&0.1607&-3.35 \\
Random ($N(\mathcal{M})$=4) &0.0183&0.0187&0.1277&0.1746&0.1706&0.0936&0.2812&0.7841&0.1804&0.1882&0.1636&-6.12 \\
Random ($N(\mathcal{M})$=5) &0.0186&0.0184&0.1274&0.1747&0.1708&0.0935&0.3150&0.7842&0.1800&0.1888&0.1640&-7.17 \\
Random ($N(\mathcal{M})$=6) &0.0208&0.0209&0.1349&0.1750&0.1721&0.0961&0.3334&0.8222&0.1976&0.1935&0.1703&-13.20\\
Ours                        &0.0167&0.0169&0.1228&0.1739&0.1695&0.0910&0.2344&0.7600&0.1691&0.1836&0.1571&-0.64 \\\midrule[1.0pt]
\end{tabular}}
\label{tab:tab_exp_grouping_strategy}
\end{table*}


\begin{table*}[h]
\caption{Results on Taskonomy with varying batch sizes using ViT-B (batch sizes in brackets).}
\vspace{-5pt}
\centering
\renewcommand\arraystretch{1.00}
\resizebox{0.99\textwidth}{!}{
\begin{tabular}{l|ccccccccccc|c}
\midrule[1.0pt]
 & DE & DZ & EO & ET & K2  & K3 & N   & C & R & S2  & S2.5 &  \\ \cmidrule[0.5pt]{2-12}
\multirow{-2}{*}{Task} & L1 Dist. $\downarrow$  & L1 Dist. $\downarrow$ & L1 Dist. $\downarrow$ & L1 Dist. $\downarrow$ & L1 Dist. $\downarrow$ & L1 Dist. $\downarrow$ & L1 Dist. $\downarrow$ & RMSE $\downarrow$    & L1 Dist. $\downarrow$ & L1 Dist. $\downarrow$ & L1 Dist. $\downarrow$  & \multirow{-2}{*}{$\triangle_m$ ($\uparrow$)} \\ \midrule[1.0pt]
Single Task                 &0.0183&0.0186&0.1089&0.1713&0.1630&0.0863&0.2953&0.7522&0.1504&0.1738&0.1530&-         \\\midrule[0.5pt]
GD(4)                       &0.0208&0.0214&0.1323&0.1747&0.1723&0.0952&0.2768&0.8214&0.1936&0.1921&0.1677&-10.88    \\ \rowcolor[HTML]{E0E0E0}
Ours(4)                     &0.0185&0.0190&0.1273&0.1741&0.1709&0.0928&0.2739&0.7957&0.1809&0.1888&0.1632&-6.19     \\
GD(8)                       &0.0188&0.0197&0.1283&0.1745&0.1718&0.0933&0.2599&0.7911&0.1799&0.1885&0.1631&-6.35     \\ \rowcolor[HTML]{E0E0E0}
Ours(8)                     &0.0167&0.0169&0.1228&0.1739&0.1695&0.0910&0.2344&0.7600&0.1691&0.1836&0.1571&-0.64     \\
GD(16)                      &0.0172&0.0180&0.1248&0.1742&0.1711&0.0920&0.2280&0.7641&0.1706&0.1848&0.1589&-1.94     \\ \rowcolor[HTML]{E0E0E0}
Ours(16)                    &0.0153&0.0154&0.1186&0.1737&0.1682&0.0893&0.1967&0.7334&0.1581&0.1780&0.1516&+4.19     \\\midrule[1.0pt]
\end{tabular}}
\label{tab:tab_exp_batch}
\end{table*}



% \clearpage
%%%%%%%%%%%%%%%%%%%%%%%%%%%%%%%%%%%%%%%%%%%%%%%%%%%%%%%%%%%%%%%%%%%%%%%%%%%%%%%%%%%%
\section{Algorithm Complexity and Computational Load}
\setcounter{table}{0}
\setcounter{figure}{0}
We also provide a detailed time comparison of previous multi-task optimization methods on Taskonomy. As shown in \Cref{tab:training_time_taskonomy}, our approach effectively optimizes multiple tasks with more efficient training times. Our method converges faster than gradient-based approaches, as the primary bottleneck in optimization lies in backpropagation and gradient manipulation.

%%%%%%%%%%%%%%%%%%%%%%%%%%%%%%%%%%%%%%%%%%%%
\begin{table*}[h]
\caption{Comparison of the average time required by each optimization process to handle a single
batch for 11 tasks on Taskonomy.}
\vspace{-5pt}
\centering
\renewcommand\arraystretch{1.00}
\resizebox{\textwidth}{!}{
\scriptsize
\begin{tabular}{l|ccccc|c}
\hline
Process (sec)     & Forward Pass & Backpropagation & Gradient Manipulation & Optimizer Step & Clustering + Affinity Update & Total \\ \hline
GD&0.030(9.04\%)&0.198(59.26\%)&-&0.106(31.69\%)&-&0.33 \\ 
UW&0.030(8.82\%)&0.198(58.24\%)&-&0.112(32.94\%)&-&0.34 \\
DTP&0.030(8.70\%)&0.199(57.68\%)&-&0.116(33.62\%)&-&0.34 \\
DWA&0.031(9.01\%)&0.198(57.56\%)&-&0.115(33.43\%)&-&0.34 \\
GradDrop&0.030(1.13\%)&2.05(80.54\%)&0.411(16.02\%)&0.059(2.30\%)&-&2.57 \\    
MGDA&0.033(0.086\%)&2.06(5.36\%)&36.29(94.47\%)&0.031(0.081\%)&-&38.42 \\
PCGrad&0.030(0.63\%)&2.07(44.09\%)&2.57(54.62\%)&0.031(0.66\%)&-&4.70 \\
CAGrad&0.030(0.57\%)&2.06(39.39\%)&3.11(59.44\%)&0.031(0.59\%)&-&5.23 \\
Aligned-MTL&0.027(0.86\%)&2.07(64.99\%)&1.06(33.20\%)&0.030(0.95\%)&-&3.19 \\
FAMO&0.030 (8.72\%)&0.198(57.56\%)&-&0.116(33.72\%)&-&0.34 \\
Ours&0.072 (7.13\%)&0.576(56.72\%)&-&0.323(31.82\%)&0.044(4.33\%)&1.02 \\ \hline
\end{tabular}}
\label{tab:training_time_taskonomy}
\end{table*}
%%%%%%%%%%%%%%%%%%%%%%%%%%%%%%%%%%%%%%%%%%%%



\end{document}
