% ICCV 2025 Paper Template

\documentclass[10pt,twocolumn,letterpaper]{article}

%%%%%%%%% PAPER TYPE  - PLEASE UPDATE FOR FINAL VERSION
% \usepackage{iccv}              % To produce the CAMERA-READY version
% \usepackage[review]{iccv}      % To produce the REVIEW version
\usepackage[pagenumbers]{iccv} % To force page numbers, e.g. for an arXiv version

% Import additional packages in the preamble file, before hyperref
%
% --- inline annotations
%
\newcommand{\red}[1]{{\color{red}#1}}
\newcommand{\todo}[1]{{\color{red}#1}}
\newcommand{\TODO}[1]{\textbf{\color{red}[TODO: #1]}}
% --- disable by uncommenting  
% \renewcommand{\TODO}[1]{}
% \renewcommand{\todo}[1]{#1}



\newcommand{\VLM}{LVLM\xspace} 
\newcommand{\ours}{PeKit\xspace}
\newcommand{\yollava}{Yo’LLaVA\xspace}

\newcommand{\thisismy}{This-Is-My-Img\xspace}
\newcommand{\myparagraph}[1]{\noindent\textbf{#1}}
\newcommand{\vdoro}[1]{{\color[rgb]{0.4, 0.18, 0.78} {[V] #1}}}
% --- disable by uncommenting  
% \renewcommand{\TODO}[1]{}
% \renewcommand{\todo}[1]{#1}
\usepackage{slashbox}
% Vectors
\newcommand{\bB}{\mathcal{B}}
\newcommand{\bw}{\mathbf{w}}
\newcommand{\bs}{\mathbf{s}}
\newcommand{\bo}{\mathbf{o}}
\newcommand{\bn}{\mathbf{n}}
\newcommand{\bc}{\mathbf{c}}
\newcommand{\bp}{\mathbf{p}}
\newcommand{\bS}{\mathbf{S}}
\newcommand{\bk}{\mathbf{k}}
\newcommand{\bmu}{\boldsymbol{\mu}}
\newcommand{\bx}{\mathbf{x}}
\newcommand{\bg}{\mathbf{g}}
\newcommand{\be}{\mathbf{e}}
\newcommand{\bX}{\mathbf{X}}
\newcommand{\by}{\mathbf{y}}
\newcommand{\bv}{\mathbf{v}}
\newcommand{\bz}{\mathbf{z}}
\newcommand{\bq}{\mathbf{q}}
\newcommand{\bff}{\mathbf{f}}
\newcommand{\bu}{\mathbf{u}}
\newcommand{\bh}{\mathbf{h}}
\newcommand{\bb}{\mathbf{b}}

\newcommand{\rone}{\textcolor{green}{R1}}
\newcommand{\rtwo}{\textcolor{orange}{R2}}
\newcommand{\rthree}{\textcolor{red}{R3}}
\usepackage{amsmath}
%\usepackage{arydshln}
\DeclareMathOperator{\similarity}{sim}
\DeclareMathOperator{\AvgPool}{AvgPool}

\newcommand{\argmax}{\mathop{\mathrm{argmax}}}     



% It is strongly recommended to use hyperref, especially for the review version.
% hyperref with option pagebackref eases the reviewers' job.
% Please disable hyperref *only* if you encounter grave issues, 
% e.g. with the file validation for the camera-ready version.
%
% If you comment hyperref and then uncomment it, you should delete *.aux before re-running LaTeX.
% (Or just hit 'q' on the first LaTeX run, let it finish, and you should be clear).
\definecolor{iccvblue}{rgb}{0.21,0.49,0.74}
\usepackage[pagebackref,breaklinks,colorlinks,allcolors=iccvblue]{hyperref}

%%%%%%%%% PAPER ID  - PLEASE UPDATE
\def\paperID{***} % *** Enter the Paper ID here
\def\confName{ICCV}
\def\confYear{2025}

%%%%%%%%% TITLE - PLEASE UPDATE
\title{ELIP: Enhanced Visual-Language Foundation Models for Image Retrieval}

%%%%%%%%% AUTHORS - PLEASE UPDATE
\author{%
  Guanqi Zhan$^{1*}$, Yuanpei Liu$^{2*}$,
  Kai Han$^2$, Weidi Xie$^{1,3}$, Andrew Zisserman$^1$\\
    $^1$VGG, University of Oxford\quad\quad
    $^2$The University of Hong Kong \quad\quad
    $^3$Shanghai Jiao Tong University\\
  \texttt{\{guanqi,weidi,az\}@robots.ox.ac.uk} \\
  \texttt{ypliu0@connect.hku.hk}~~~~
  \texttt{kaihanx@hku.hk} \\
}




\begin{document}


\twocolumn[{
    \vspace{-30pt}
    \renewcommand\twocolumn[1][]{#1}
    \maketitle
    \centering
    \vspace{-10pt}
 \includegraphics[height=0.45\linewidth]{images/teaser.pdf}   
 \vspace{-8mm}
   \captionof{figure}{
\textbf{The ELIP architecture.}
{\em Left}: We propose a novel architecture that can be applied to pre-trained and frozen vision-language foundation models, such as CLIP, SigLIP, SigLIP-2 and BLIP-2, to enhance their text-to-image retrieval performance. 
The \emph{key idea} is to use the text query to define a set of visual prompt vectors that are incorporated into the image encoder to make it aware
of the query when generating the embedding. An MLP maps from the text space to the visual space of the input to the ViT encoder. The architecture is lightweight, and our data curation strategies enable efficient and effective training with limited resources.
{\em Right}: In this retrieval example from the COCO benchmark, the top-$k$ ($k$=100) images are re-ranked by our ELIP model for the text query: `People on bicycles ride down a busy street'. The ground truth image matching the query is not in the top-5 ranked images in the initial CLIP ranking, but is ranked top-1 (highlighted in the dashed box) by the re-ranking. 
   }
    \label{fig:teaser}
    \vspace{20pt}
    }
    ]

\def\thefootnote{*}\footnotetext{Equal contribution.}\def\thefootnote{\arabic{footnote}}

\begin{abstract}


The choice of representation for geographic location significantly impacts the accuracy of models for a broad range of geospatial tasks, including fine-grained species classification, population density estimation, and biome classification. Recent works like SatCLIP and GeoCLIP learn such representations by contrastively aligning geolocation with co-located images. While these methods work exceptionally well, in this paper, we posit that the current training strategies fail to fully capture the important visual features. We provide an information theoretic perspective on why the resulting embeddings from these methods discard crucial visual information that is important for many downstream tasks. To solve this problem, we propose a novel retrieval-augmented strategy called RANGE. We build our method on the intuition that the visual features of a location can be estimated by combining the visual features from multiple similar-looking locations. We evaluate our method across a wide variety of tasks. Our results show that RANGE outperforms the existing state-of-the-art models with significant margins in most tasks. We show gains of up to 13.1\% on classification tasks and 0.145 $R^2$ on regression tasks. All our code and models will be made available at: \href{https://github.com/mvrl/RANGE}{https://github.com/mvrl/RANGE}.

\end{abstract}

    
\section{Introduction}
Backdoor attacks pose a concealed yet profound security risk to machine learning (ML) models, for which the adversaries can inject a stealth backdoor into the model during training, enabling them to illicitly control the model's output upon encountering predefined inputs. These attacks can even occur without the knowledge of developers or end-users, thereby undermining the trust in ML systems. As ML becomes more deeply embedded in critical sectors like finance, healthcare, and autonomous driving \citep{he2016deep, liu2020computing, tournier2019mrtrix3, adjabi2020past}, the potential damage from backdoor attacks grows, underscoring the emergency for developing robust defense mechanisms against backdoor attacks.

To address the threat of backdoor attacks, researchers have developed a variety of strategies \cite{liu2018fine,wu2021adversarial,wang2019neural,zeng2022adversarial,zhu2023neural,Zhu_2023_ICCV, wei2024shared,wei2024d3}, aimed at purifying backdoors within victim models. These methods are designed to integrate with current deployment workflows seamlessly and have demonstrated significant success in mitigating the effects of backdoor triggers \cite{wubackdoorbench, wu2023defenses, wu2024backdoorbench,dunnett2024countering}.  However, most state-of-the-art (SOTA) backdoor purification methods operate under the assumption that a small clean dataset, often referred to as \textbf{auxiliary dataset}, is available for purification. Such an assumption poses practical challenges, especially in scenarios where data is scarce. To tackle this challenge, efforts have been made to reduce the size of the required auxiliary dataset~\cite{chai2022oneshot,li2023reconstructive, Zhu_2023_ICCV} and even explore dataset-free purification techniques~\cite{zheng2022data,hong2023revisiting,lin2024fusing}. Although these approaches offer some improvements, recent evaluations \cite{dunnett2024countering, wu2024backdoorbench} continue to highlight the importance of sufficient auxiliary data for achieving robust defenses against backdoor attacks.

While significant progress has been made in reducing the size of auxiliary datasets, an equally critical yet underexplored question remains: \emph{how does the nature of the auxiliary dataset affect purification effectiveness?} In  real-world  applications, auxiliary datasets can vary widely, encompassing in-distribution data, synthetic data, or external data from different sources. Understanding how each type of auxiliary dataset influences the purification effectiveness is vital for selecting or constructing the most suitable auxiliary dataset and the corresponding technique. For instance, when multiple datasets are available, understanding how different datasets contribute to purification can guide defenders in selecting or crafting the most appropriate dataset. Conversely, when only limited auxiliary data is accessible, knowing which purification technique works best under those constraints is critical. Therefore, there is an urgent need for a thorough investigation into the impact of auxiliary datasets on purification effectiveness to guide defenders in  enhancing the security of ML systems. 

In this paper, we systematically investigate the critical role of auxiliary datasets in backdoor purification, aiming to bridge the gap between idealized and practical purification scenarios.  Specifically, we first construct a diverse set of auxiliary datasets to emulate real-world conditions, as summarized in Table~\ref{overall}. These datasets include in-distribution data, synthetic data, and external data from other sources. Through an evaluation of SOTA backdoor purification methods across these datasets, we uncover several critical insights: \textbf{1)} In-distribution datasets, particularly those carefully filtered from the original training data of the victim model, effectively preserve the model’s utility for its intended tasks but may fall short in eliminating backdoors. \textbf{2)} Incorporating OOD datasets can help the model forget backdoors but also bring the risk of forgetting critical learned knowledge, significantly degrading its overall performance. Building on these findings, we propose Guided Input Calibration (GIC), a novel technique that enhances backdoor purification by adaptively transforming auxiliary data to better align with the victim model’s learned representations. By leveraging the victim model itself to guide this transformation, GIC optimizes the purification process, striking a balance between preserving model utility and mitigating backdoor threats. Extensive experiments demonstrate that GIC significantly improves the effectiveness of backdoor purification across diverse auxiliary datasets, providing a practical and robust defense solution.

Our main contributions are threefold:
\textbf{1) Impact analysis of auxiliary datasets:} We take the \textbf{first step}  in systematically investigating how different types of auxiliary datasets influence backdoor purification effectiveness. Our findings provide novel insights and serve as a foundation for future research on optimizing dataset selection and construction for enhanced backdoor defense.
%
\textbf{2) Compilation and evaluation of diverse auxiliary datasets:}  We have compiled and rigorously evaluated a diverse set of auxiliary datasets using SOTA purification methods, making our datasets and code publicly available to facilitate and support future research on practical backdoor defense strategies.
%
\textbf{3) Introduction of GIC:} We introduce GIC, the \textbf{first} dedicated solution designed to align auxiliary datasets with the model’s learned representations, significantly enhancing backdoor mitigation across various dataset types. Our approach sets a new benchmark for practical and effective backdoor defense.



\section{Related Work}
\label{sec:related-works}
\subsection{Novel View Synthesis}
Novel view synthesis is a foundational task in the computer vision and graphics, which aims to generate unseen views of a scene from a given set of images.
% Many methods have been designed to solve this problem by posing it as 3D geometry based rendering, where point clouds~\cite{point_differentiable,point_nfs}, mesh~\cite{worldsheet,FVS,SVS}, planes~\cite{automatci_photo_pop_up,tour_into_the_picture} and multi-plane images~\cite{MINE,single_view_mpi,stereo_magnification}, \etal
Numerous methods have been developed to address this problem by approaching it as 3D geometry-based rendering, such as using meshes~\cite{worldsheet,FVS,SVS}, MPI~\cite{MINE,single_view_mpi,stereo_magnification}, point clouds~\cite{point_differentiable,point_nfs}, etc.
% planes~\cite{automatci_photo_pop_up,tour_into_the_picture}, 


\begin{figure*}[!t]
    \centering
    \includegraphics[width=1.0\linewidth]{figures/overview-v7.png}
    %\caption{\textbf{Overview.} Given a set of images, our method obtains both camera intrinsics and extrinsics, as well as a 3DGS model. First, we obtain the initial camera parameters, global track points from image correspondences and monodepth with reprojection loss. Then we incorporate the global track information and select Gaussian kernels associated with track points. We jointly optimize the parameters $K$, $T_{cw}$, 3DGS through multi-view geometric consistency $L_{t2d}$, $L_{t3d}$, $L_{scale}$ and photometric consistency $L_1$, $L_{D-SSIM}$.}
    \caption{\textbf{Overview.} Given a set of images, our method obtains both camera intrinsics and extrinsics, as well as a 3DGS model. During the initialization, we extract the global tracks, and initialize camera parameters and Gaussians from image correspondences and monodepth with reprojection loss. We determine Gaussian kernels with recovered 3D track points, and then jointly optimize the parameters $K$, $T_{cw}$, 3DGS through the proposed global track constraints (i.e., $L_{t2d}$, $L_{t3d}$, and $L_{scale}$) and original photometric losses (i.e., $L_1$ and $L_{D-SSIM}$).}
    \label{fig:overview}
\end{figure*}

Recently, Neural Radiance Fields (NeRF)~\cite{2020NeRF} provide a novel solution to this problem by representing scenes as implicit radiance fields using neural networks, achieving photo-realistic rendering quality. Although having some works in improving efficiency~\cite{instant_nerf2022, lin2022enerf}, the time-consuming training and rendering still limit its practicality.
Alternatively, 3D Gaussian Splatting (3DGS)~\cite{3DGS2023} models the scene as explicit Gaussian kernels, with differentiable splatting for rendering. Its improved real-time rendering performance, lower storage and efficiency, quickly attract more attentions.
% Different from NeRF-based methods which need MLPs to model the scene and huge computational cost for rendering, 3DGS has stronger real-time performance, higher storage and computational efficiency, benefits from its explicit representation and gradient backpropagation.

\subsection{Optimizing Camera Poses in NeRFs and 3DGS}
Although NeRF and 3DGS can provide impressive scene representation, these methods all need accurate camera parameters (both intrinsic and extrinsic) as additional inputs, which are mostly obtained by COLMAP~\cite{colmap2016}.
% This strong reliance on COLMAP significantly limits their use in real-world applications, so optimizing the camera parameters during the scene training becomes crucial.
When the prior is inaccurate or unknown, accurately estimating camera parameters and scene representations becomes crucial.

% In early works, only photometric constraints are used for scene training and camera pose estimation. 
% iNeRF~\cite{iNerf2021} optimizes the camera poses based on a pre-trained NeRF model.
% NeRFmm~\cite{wang2021nerfmm} introduce a joint optimization process, which estimates the camera poses and trains NeRF model jointly.
% BARF~\cite{barf2021} and GARF~\cite{2022GARF} provide new positional encoding strategy to handle with the gradient inconsistency issue of positional embedding and yield promising results.
% However, they achieve satisfactory optimization results when only the pose initialization is quite closed to the ground-truth, as the photometric constrains can only improve the quality of camera estimation within a small range.
% Later, more prior information of geometry and correspondence, \ie monocular depth and feature matching, are introduced into joint optimisation to enhance the capability of camera poses estimation.
% SC-NeRF~\cite{SCNeRF2021} minimizes a projected ray distance loss based on correspondence of adjacent frames.
% NoPe-NeRF~\cite{bian2022nopenerf} chooses monocular depth maps as geometric priors, and defines undistorted depth loss and relative pose constraints for joint optimization.
In earlier studies, scene training and camera pose estimation relied solely on photometric constraints. iNeRF~\cite{iNerf2021} refines the camera poses using a pre-trained NeRF model. NeRFmm~\cite{wang2021nerfmm} introduces a joint optimization approach that simultaneously estimates camera poses and trains the NeRF model. BARF~\cite{barf2021} and GARF~\cite{2022GARF} propose a new positional encoding strategy to address the gradient inconsistency issues in positional embedding, achieving promising results. However, these methods only yield satisfactory optimization when the initial pose is very close to the ground truth, as photometric constraints alone can only enhance camera estimation quality within a limited range. Subsequently, 
% additional prior information on geometry and correspondence, such as monocular depth and feature matching, has been incorporated into joint optimization to improve the accuracy of camera pose estimation. 
SC-NeRF~\cite{SCNeRF2021} minimizes a projected ray distance loss based on correspondence between adjacent frames. NoPe-NeRF~\cite{bian2022nopenerf} utilizes monocular depth maps as geometric priors and defines undistorted depth loss and relative pose constraints.

% With regard to 3D Gaussian Splatting, CF-3DGS~\cite{CF-3DGS-2024} also leverages mono-depth information to constrain the optimization of local 3DGS for relative pose estimation and later learn a global 3DGS progressively in a sequential manner.
% InstantSplat~\cite{fan2024instantsplat} focus on sparse view scenes, first use DUSt3R~\cite{dust3r2024cvpr} to generate a set of densely covered and pixel-aligned points for 3D Gaussian initialization, then introduce a parallel grid partitioning strategy in joint optimization to speed up.
% % Jiang et al.~\cite{Jiang_2024sig} proposed to build the scene continuously and progressively, to next unregistered frame, they use registration and adjustment to adjust the previous registered camera poses and align unregistered monocular depths, later refine the joint model by matching detected correspondences in screen-space coordinates.
% \gjh{Jiang et al.~\cite{Jiang_2024sig} also implemented an incremental approach for reconstructing camera poses and scenes. Initially, they perform feature matching between the current image and the image rendered by a differentiable surface renderer. They then construct matching point errors, depth errors, and photometric errors to achieve the registration and adjustment of the current image. Finally, based on the depth map, the pixels of the current image are projected as new 3D Gaussians. However, this method still exhibits limitations when dealing with complex scenes and unordered images.}
% % CG-3DGS~\cite{sun2024correspondenceguidedsfmfree3dgaussian} follows CF-3DGS, first construct a coarse point cloud from mono-depth maps to train a 3DGS model, then progressively estimate camera poses based on this pre-trained model by constraining the correspondences between rendering view and ground-truth.
% \gjh{Similarly, CG-3DGS~\cite{sun2024correspondenceguidedsfmfree3dgaussian} first utilizes monocular depth estimation and the camera parameters from the first frame to initialize a set of 3D Gaussians. It then progressively estimates camera poses based on this pre-trained model by constraining the correspondences between the rendered views and the ground truth.}
% % Free-SurGS~\cite{freesurgs2024} matches the projection flow derived from 3D Gaussians with optical flow to estimate the poses, to compensate for the limitations of photometric loss.
% \gjh{Free-SurGS~\cite{freesurgs2024} introduces the first SfM-free 3DGS approach for surgical scene reconstruction. Due to the challenges posed by weak textures and photometric inconsistencies in surgical scenes, Free-SurGS achieves pose estimation by minimizing the flow loss between the projection flow and the optical flow. Subsequently, it keeps the camera pose fixed and optimizes the scene representation by minimizing the photometric loss, depth loss and flow loss.}
% \gjh{However, most current works assume camera intrinsics are known and primarily focus on optimizing camera poses. Additionally, these methods typically rely on sequentially ordered image inputs and incrementally optimize camera parameters and scene representation. This inevitably leads to drift errors, preventing the achievement of globally consistent results. Our work aims to address these issues.}

Regarding 3D Gaussian Splatting, CF-3DGS~\cite{CF-3DGS-2024} utilizes mono-depth information to refine the optimization of local 3DGS for relative pose estimation and subsequently learns a global 3DGS in a sequential manner. InstantSplat~\cite{fan2024instantsplat} targets sparse view scenes, initially employing DUSt3R~\cite{dust3r2024cvpr} to create a densely covered, pixel-aligned point set for initializing 3D Gaussian models, and then implements a parallel grid partitioning strategy to accelerate joint optimization. Jiang \etal~\cite{Jiang_2024sig} develops an incremental method for reconstructing camera poses and scenes, but it struggles with complex scenes and unordered images. 
% Similarly, CG-3DGS~\cite{sun2024correspondenceguidedsfmfree3dgaussian} progressively estimates camera poses using a pre-trained model by aligning the correspondences between rendered views and actual scenes. Free-SurGS~\cite{freesurgs2024} pioneers an SfM-free 3DGS method for reconstructing surgical scenes, overcoming challenges such as weak textures and photometric inconsistencies by minimizing the discrepancy between projection flow and optical flow.
%\pb{SF-3DGS-HT~\cite{ji2024sfmfree3dgaussiansplatting} introduced VFI into training as additional photometric constraints. They separated the whole scene into several local 3DGS models and then merged them hierarchically, which leads to a significant improvement on simple and dense view scenes.}
HT-3DGS~\cite{ji2024sfmfree3dgaussiansplatting} interpolates frames for training and splits the scene into local clips, using a hierarchical strategy to build 3DGS model. It works well for simple scenes, but fails with dramatic motions due to unstable interpolation and low efficiency.
% {While effective for simple scenes, it struggles with dramatic motion due to unstable view interpolation and suffers from low computational efficiency.}

However, most existing methods generally depend on sequentially ordered image inputs and incrementally optimize camera parameters and 3DGS, which often leads to drift errors and hinders achieving globally consistent results. Our work seeks to overcome these limitations.

% Consider a lasso optimization procedure with potentially distinct regularization penalties:
% \begin{align}
%     \hat{\beta} = \arg\min_{\beta}\{\|y-X\beta\|^2_2+\sum_{i=1}^{N}\lambda_i|\beta_i|\}.
% \end{align}
\subsection{Supervised Data-Driven Learning}\label{subsec:supervised}
We consider a generic data-driven supervised learning procedure. Given a dataset \( \mathcal{D} \) consisting of \( n \) data points \( (x_i, y_i) \in \mathcal{X} \times \mathcal{Y} \) drawn from an underlying distribution \( p(\cdot|\theta) \), our goal is to estimate parameters \( \theta \in \Theta \) through a learning procedure, defined as \( f: (\mathcal{X} \times \mathcal{Y})^n \rightarrow \Theta \) 
that minimizes the predictive error on observed data. 
Specifically, the learning objective is defined as follows:
\begin{align}
\hat{\theta}_f := f(\mathcal{D}) = \arg\min_{\theta} \mathcal{L}(\theta, \mathcal{D}),
\end{align}
where \( \mathcal{L}(\cdot,\mathcal{D}) := \sum_{i=1}^{n} \mathcal{L}(\cdot, (x_i, y_i))\), and $\mathcal{L}$ is a loss function quantifying the error between predictions and true outcomes. 
Here, $\hat{\theta}_f$ is the parameter that best explains the observed data pairs \( (x_i, y_i) \) according to the chosen loss function \( \mathcal{L} (\cdot) \).

\paragraph{Feature Selection.}
Feature selection aims to improve model \( f \)'s predictive performance while minimizing redundancy. 
%Formally, given data \( X \), response \( y \), feature set \( \mathcal{F} \), loss function \( \mathcal{L}(\cdot) \), and a feature limit \( k \), the objective is:
% \begin{align}
% \mathcal{S}^* = \arg \min_{\mathcal{S} \subseteq \mathcal{F}, |\mathcal{S}| \leq k} \mathcal{L}(y, f(X_\mathcal{S})) + \lambda R(\mathcal{S}),
% \end{align}
% where \( X_\mathcal{S} \) is the submatrix of \( X \) for selected features \( \mathcal{S} \), \( \lambda \) is a regularization parameter, and \( R(\mathcal{S}) \) penalizes feature redundancy.
 State-of-the-art techniques fall into four categories: (i) filter methods, which rank features based on statistical properties like Fisher score \citep{duda2001pattern,song2012feature}; (ii) wrapper methods, which evaluate model performance on different feature subsets \citep{kohavi1997wrappers}; (iii) embedded methods, which integrate feature selection into the learning process using techniques like regularization \citep{tibshirani1996LASSO,lemhadri2021lassonet}; and (iv) hybrid methods, which combine elements of (i)-(iii) \citep{SINGH2021104396,li2022micq}. This paper focuses on embedded methods via Lasso, benchmarking against approaches from (i)-(iii).

\subsection{Language Modeling}
% The objective of language modeling is to learn a probability distribution \( p_{LM}(x) \) over sequences of text \( x = (X_1, \ldots, X_{|x|}) \), such that \( p_{LM}(x) \approx p_{text}(x) \), where \( p_{text}(x) \) represents the true distribution of natural language. This process involves estimating the likelihood of token sequences across variable lengths and diverse linguistic structures.
% Modern large language models (LLMs) are trained on vast datasets spanning encyclopedias, news, social media, books, and scientific papers \cite{gao2020pile}. This broad training enables them to generalize across domains, learn contextual knowledge, and perform zero-shot learning—tackling new tasks using only task descriptions without fine-tuning \cite{brown2020gpt3}.
Language modeling aims to approximate the true distribution of natural language \( p_{\text{text}}(x) \) by learning \( p_{\text{LM}}(x) \), a probability distribution over text sequences \( x = (X_1, \ldots, X_{|x|}) \). Modern large language models, trained on diverse datasets \citep{gao2020pile}, exhibit strong generalization across domains, acquire contextual knowledge, and perform zero-shot learning—solving new tasks using only task descriptions—or few-shot learning by leveraging a small number of demonstrations \citep{brown2020gpt3}.
\paragraph{Retrieval-Augmented Generation (RAG).} Retrieval-Augmented Generation (RAG) enhances the performance of generative language models by  integrating a domain-specific information retrieval process  \citep{lewis2020retrieval}. The RAG framework comprises two main components: \textit{retrieval}, which extracts relevant information from external knowledge sources, and \textit{generation}, where an LLM generates context-aware responses using the prompt combined with the retrieved context. Documents are indexed through various databases, such as relational, graph, or vector databases \citep{khattab2020colbert, douze2024faiss, peng2024graphretrievalaugmentedgenerationsurvey}, enabling efficient organization and retrieval via algorithms like semantic similarity search to match the prompt with relevant documents in the knowledge base. RAG has gained much traction recently due to its demonstrated ability to reduce incidence of hallucinations and boost LLMs' reliability as well as performance \citep{huang2023hallucination, zhang2023merging}. 
 
% image source: https://medium.com/@bindurani_22/retrieval-augmented-generation-815c1ae438d8
\begin{figure}
    \centering
\includegraphics[width=1.03\linewidth]{fig/fig1.pdf}
\vspace{-0.6cm}
\scriptsize 
    \caption{Retrieval Augmented Generation (RAG) based $\ell_1$-norm weights (penalty factors) for Lasso. Only feature names---no training data--- are included in LLM prompt.} 
    \label{fig:rag}
\end{figure}
% However, for the RAG model to be effective given the input token constraints of the LLM model used, we need to effectively process the retrieval documents through a procedure known as \textit{chunking}.

\subsection{Task-Specific Data-Driven Learning}
LLM-Lasso aims to bridge the gap between data-driven supervised learning and the predictive capabilities of LLMs trained on rich metadata. This fusion not only enhances traditional data-driven methods by incorporating key task-relevant contextual information often overlooked by such models, but can also be especially valuable in low-data regimes, where the learning algorithm $f:\mathcal{D}\rightarrow\Theta$ (seen as a map from datasets $\mathcal{D}$ to the space of decisions $\Theta$) is susceptible to overfitting.

The task-specific data-driven learning model $\tilde{f}:\mathcal{D}\times\mathcal{D}_\text{meta}\rightarrow\Theta$ can be described as a metadata-augmented version of $f$, where a link function $h(\cdot)$ integrates metadata (i.e. $\mathcal{D}_\text{meta}$) to refine the original learning process. This can be expressed as:
\[
\tilde{f}(\mathcal{D}, \mathcal{D}_\text{meta}) := \mathcal{T}(f(\mathcal{D}),  h(\mathcal{D}_{\text{meta}})),
\]
where the functional $\mathcal{T}$ takes the original learning algorithm $f(\mathcal{D})$ and transforms it into a task-specific learning algorithm $\tilde{f}(\mathcal{D}, \mathcal{D}_\text{meta})$ by incorporating the metadata $\mathcal{D}_\text{meta}$. 
% In particular, the link function $h(\mathcal{D}_{\text{meta}})$ provides a structured mechanism summarizing the contextual knowledge.

There are multiple approaches to formulate $\mathcal{T}$ and $h$.
%to ``inform" the data-driven model $f$ of %meta knowledge. 
For instance, LMPriors \citep{choi2022lmpriorspretrainedlanguagemodels} designed $h$ and $\mathcal{T}$ such that $h(\mathcal{D}_{\text{meta}})$ first specifies which features to retain (based on a probabilistic prior framework), and then $\mathcal{T}$ keeps the selected features and removes all the others from the original learning objective of $f$. 
Note that this approach inherently is restricted as it selects important features solely based on $\mathcal{D}_\text{meta}$ without seeing $\mathcal{D}$.

In contrast, we directly embed task-specific knowledge into the optimization landscape through regularization by introducing a structured inductive bias. This bias guides the learning process toward solutions that are consistent with metadata-informed insights, without relying on explicit probabilistic modeling. Abstractly, this can be expressed as:
\begin{align}
    \!\!\!\!\!\hat{\theta}_{\tilde{f}} := \tilde{f}(\mathcal{D},\mathcal{D}
    _\text{meta})= \arg\min_{\theta} \mathcal{L}(\theta, \mathcal{D}) + \lambda R(\theta, \mathcal{D}_{\text{meta}}),
\end{align}
where \( \lambda \) is a regularization parameter, \( R(\cdot) \) is a regularizer, and $\theta$ is the prediction parameter.
%We explain our framework with more details in the following section.


% Our research diverges from both aforementioned approaches by positioning the LLM not as a standalone feature selector but as an enhancement to data-driven models through an embedded feature selection method, L-LASSO. L-LASSO incorporates domain expertise—auxiliary natural language metadata about the task—via the LLM-informed LASSO penalty, which is then used in statistical models to enhance predictive performance. This method integrates the rich, context-sensitive insights of LLMs with the rigor and transparency of statistical modeling, bridging the gap between data-driven and knowledge-driven feature selection approaches. To approach this task, we need to tackle two key components: (i). train an LLM that is expert in the task-specific knowledge; (ii). inform data-driven feature selector LASSO with LLM knowledge.

% In practice, this involves combining techniques like prompt engineering and data engineering to develop an effective framework for integrating metadata into existing data-driven models. We will go through this in detail in Section \ref{mthd} and \ref{experiment}.


\section{Architecture}
\label{sec:architecture}

\begin{figure}
    \centering
    \includegraphics[width=0.92\linewidth]{graphs/arch.pdf}
    \caption{
        \sysname{} proposes an LLM-based no-code application development framework using FaaS for infrastructure abstraction.
        The prompt constructor combines a user's application description with a system prompt for an LLM that generates application code.
        The function deployer uses that code to deploy a FaaS function on a FaaS platform.
    }
    \label{fig:arch}
\end{figure}

LLMs are excellent tools for transforming natural language software descriptions into executable code, but are by themselves unable to deploy and operate that code for users.
FaaS platforms can deploy small pieces of code as scalable, managed applications.
With \sysname{}, we propose combining these two technologies into an end-to-end no-code application development platform.
Our goal is to let non-technical users, i.e., individuals without experience in software development or operation, provide application descriptions in natural language and build fully-managed applications from those descriptions.
Examples for such applications can be found, e.g., in the context of smart home automation, simple extensions of enterprise applications, or custom information aggregation from news websites, social media, and web APIs~\cite{paper_bermbach2020_webapibenchmarking2}.
To support such applications, we design \sysname{} as shown in \cref{fig:arch}.

\sysname{} comprises three main components: an LLM for generating user-specified code, a FaaS platform for efficient function deployment, and a bridge that orchestrates prompt construction and function deployment.
Users provide their natural language application descriptions to \sysname{}, which combines them with a static system prompt in a \emph{prompt constructor}.
This structured prompt instructs an LLM to generate code based on the natural language description, including, e.g., details on programming language, application context, API references, and runtime environment.
\sysname{} then parses the LLM's answer for code in a \emph{function deployer}.
This generated code is deployed on the FaaS platform which abstracts the underlying application infrastructure complexities by providing containerized, auto-scaling environments for on-demand execution.

\section{\ourdata}
\label{sec:textbook-exam}

\begin{figure}[t]
    \centering
    \includegraphics[width=\linewidth]{figures/data_framework.pdf}    \caption{\textbf{Overview of  \ourdata curation}. Given a chapter $C$, we use an LM to segment the document $D$ into sections and heuristically extract review questions to form the exam $E$. The LM then classifies each question in $E$ by Bloom’s taxonomy category and maps it to its relevant section.}
    \label{fig:dataset-overview}
    \vspace{-0.3cm}
\end{figure}

In order to evaluate \ours, we curate  \ourdata, a dataset where each entry contains a document \(D\) along with a corresponding set of exam questions \(E\).
An overview of \ourdata is illustrated in \autoref{fig:dataset-overview}.
% In this section, we first describe our data processing pipeline (\secref{ssec:textbook-exam-pipeline}) and then provide data statistics (\secref{ssec:textbook-exam-statistics}).

\subsection{Data Processing}
\label{ssec:textbook-exam-pipeline}
Our pipeline starts with textbooks from the OpenStax repository\footnote{\url{https://github.com/philschatz/textbooks}}.
Each textbook is divided into chapters, where each chapter \(C\) contains learning objectives, main content, and review questions.
For each \(C\), we parse the main content to build \(D\) and the review questions to form \(E\).
% Specifically, only the main content is used to construct \(D\), while the review questions are used to create \(E\).

\paragraph{Extracting sections.}
To simulate a learner incrementally progressing through a chapter, we divide each chapter into sections using an LM-based document structuring method. 
The LM segments \( D \) into \( n \) sections, denoted as \( \{S_1, S_2, \ldots, S_n\} \subset D \), while also extracting the corresponding review questions \( E \). 
However, not all review questions come with ground-truth answers, as some textbooks do not provide them (see Table~\ref{tab:textbook-exam-statistics} for the proportion of \( E \) with answers). 
To ensure consistency in section segmentation across different subjects, we manually annotate the first 2–5 sections from one sample per subject. 
These annotated samples serve as few-shot examples in our LM prompt (see Appendix~\ref{appdx:parsing-sections} for details).

\paragraph{Extracting questions.}
% We developed a custom parsing script using BeautifulSoup4 to extract questions and their corresponding answers. 
To maintain a balance between evaluation depth and computational feasibility, we include only chapters that contain at least 10 questions—ensuring sufficient coverage for assessment—while capping the maximum number of questions at 25 to keep learning simulations computationally manageable.

\subsection{Data Statistics}
\label{ssec:textbook-exam-statistics}
\begin{table}[t!]
    \centering
    \resizebox{\columnwidth}{!}{
        \begin{tabular}{lccccc}
        \toprule
            \textbf{Subject} & \textbf{\# $C$} & \textbf{Split} & \textbf{\# $E$ / $C$} & \textbf{\% $E$ w/ answer} & \textbf{\# $S$ / $C$} \\
        \midrule
            Microbiology & 20 & Train & 12.4 & 64\% & 16.4 \\
                         & 5  & Test  & 13.4 & 58\% & 17.0 \\
        \midrule
            Chemistry    & 20 & Train & 14.2 & 51\% & 11.0 \\
                         & 5  & Test  & 16.2 & 49\% & 6.4 \\
        \midrule
            Economics    & 20 & Train & 12.2 & 23\% & 14.1 \\
                         & 5  & Test  & 12.2 & 23\% & 14.4 \\
        \midrule
            Sociology    & 20 & Train & 10.4 & 62\% & 16.6 \\
                         & 5  & Test  & 11.2 & 67\% & 19.0 \\
        \midrule
            US History   & 20 & Train & 7.2 & 51\% & 14.9 \\
                         & 5  & Test  & 8.4 & 38\% & 13.2 \\
        \bottomrule
        \end{tabular}
    }
    \caption{\textbf{Data Statistics} of \ourdata. \# $ C$: number of chapters, \# $E/C$: avg. number of questions per chapter, \% $E$ w/ answer: proportion of questions that have reference answer, \# $S/C$: avg. number of sections per chapter.}
    \label{tab:textbook-exam-statistics}
\end{table}
For each subject, we curate 25 sequential chapters \(C\), each containing both \(D\) and \(E\).
The chapters are arranged in their natural order, with the first 20 used for training and the last five reserved for evaluation.  
There is the risk that content in later chapters may include information from prior chapters (e.g., revisiting prerequisite knowledge). 
Therefore, preserving this sequential structure between the training and test set is essential for preventing information leakage and fairly assessing a model's learning process.
Table~\ref{tab:textbook-exam-statistics} shows an overview of the statistics of the resulting \ourdata.

\subsection{Distribution of Question Types}
\label{ssec:textbook-exam-bloom}

 \begin{figure}[t]
    \centering
    \includegraphics[width=\linewidth]{figures/bloom_taxonomy_counts_vertical.png}
    \caption{\textbf{Bloom's taxonomy distribution} in \ourdata. \ourdata consists of questions that require a wide variety of cognitive levels and the dominant categories vary for each subject.}
    \label{fig:bloom-distribution}
\end{figure}

To better understand how final exams assess a learner’s comprehension on multiple dimensions, we categorize questions in \ourdata\ based on the revised \textit{Bloom’s Taxonomy}~\cite{krathwohl2002revision_bloom}. 
Using an LM, we assign a cognitive depth \( d_j \) to each question \( E_j \in E \), classifying them into six categories: \textit{Remembering, Understanding, Applying, Analyzing, Evaluating}, and \textit{Creating}.
Additionally, we identify the relevant sections \( S_j \subset D \) that correspond to each question.

The distribution, shown in \autoref{fig:bloom-distribution}, indicates that different subjects emphasize different cognitive skills.
For instance, questions in Microbiology and Sociology primarily focus on \textit{Remembering} and \textit{Understanding}, whereas Chemistry and Economics exhibit a more varied distribution.
This analysis highlights the diverse cognitive demands across subjects and underscores how \ourdata\ provides a multifaceted evaluation of learning outcomes through final exams. 
For further details on data processing, refer to Appendix~\ref{appendix:data_processing}.













% \dongho{Number of textbooks?}

% \dongho{For each textbook, how may $D$?}



% \subsection{Validation}
% \dongho{Let's make a ground truth to see how reliable the data processing pipeline it is. -- for each textbook.}

% \paragraph{Validation of answer for $E_j$.}

% \paragraph{Validation of cognitive depth $d_j$ for $E_j$.}

% \paragraph{Validation of related sections $S_j$ for $E_j$.}
\section{Experiments}
\label{section5}

In this section, we conduct extensive experiments to show that \ourmethod~can significantly speed up the sampling of existing MR Diffusion. To rigorously validate the effectiveness of our method, we follow the settings and checkpoints from \cite{luo2024daclip} and only modify the sampling part. Our experiment is divided into three parts. Section \ref{mainresult} compares the sampling results for different NFE cases. Section \ref{effects} studies the effects of different parameter settings on our algorithm, including network parameterizations and solver types. In Section \ref{analysis}, we visualize the sampling trajectories to show the speedup achieved by \ourmethod~and analyze why noise prediction gets obviously worse when NFE is less than 20.


\subsection{Main results}\label{mainresult}

Following \cite{luo2024daclip}, we conduct experiments with ten different types of image degradation: blurry, hazy, JPEG-compression, low-light, noisy, raindrop, rainy, shadowed, snowy, and inpainting (see Appendix \ref{appd1} for details). We adopt LPIPS \citep{zhang2018lpips} and FID \citep{heusel2017fid} as main metrics for perceptual evaluation, and also report PSNR and SSIM \citep{wang2004ssim} for reference. We compare \ourmethod~with other sampling methods, including posterior sampling \citep{luo2024posterior} and Euler-Maruyama discretization \citep{kloeden1992sde}. We take two tasks as examples and the metrics are shown in Figure \ref{fig:main}. Unless explicitly mentioned, we always use \ourmethod~based on SDE solver, with data prediction and uniform $\lambda$. The complete experimental results can be found in Appendix \ref{appd3}. The results demonstrate that \ourmethod~converges in a few (5 or 10) steps and produces samples with stable quality. Our algorithm significantly reduces the time cost without compromising sampling performance, which is of great practical value for MR Diffusion.


\begin{figure}[!ht]
    \centering
    \begin{minipage}[b]{0.45\textwidth}
        \centering
        \includegraphics[width=1\textwidth, trim=0 20 0 0]{figs/main_result/7_lowlight_fid.pdf}
        \subcaption{FID on \textit{low-light} dataset}
        \label{fig:main(a)}
    \end{minipage}
    \begin{minipage}[b]{0.45\textwidth}
        \centering
        \includegraphics[width=1\textwidth, trim=0 20 0 0]{figs/main_result/7_lowlight_lpips.pdf}
        \subcaption{LPIPS on \textit{low-light} dataset}
        \label{fig:main(b)}
    \end{minipage}
    \begin{minipage}[b]{0.45\textwidth}
        \centering
        \includegraphics[width=1\textwidth, trim=0 20 0 0]{figs/main_result/10_motion_fid.pdf}
        \subcaption{FID on \textit{motion-blurry} dataset}
        \label{fig:main(c)}
    \end{minipage}
    \begin{minipage}[b]{0.45\textwidth}
        \centering
        \includegraphics[width=1\textwidth, trim=0 20 0 0]{figs/main_result/10_motion_lpips.pdf}
        \subcaption{LPIPS on \textit{motion-blurry} dataset}
        \label{fig:main(d)}
    \end{minipage}
    \caption{\textbf{Perceptual evaluations on \textit{low-light} and \textit{motion-blurry} datasets.}}
    \label{fig:main}
\end{figure}

\subsection{Effects of parameter choice}\label{effects}

In Table \ref{tab:ablat_param}, we compare the results of two network parameterizations. The data prediction shows stable performance across different NFEs. The noise prediction performs similarly to data prediction with large NFEs, but its performance deteriorates significantly with smaller NFEs. The detailed analysis can be found in Section \ref{section5.3}. In Table \ref{tab:ablat_solver}, we compare \ourmethod-ODE-d-2 and \ourmethod-SDE-d-2 on the \textit{inpainting} task, which are derived from PF-ODE and reverse-time SDE respectively. SDE-based solver works better with a large NFE, whereas ODE-based solver is more effective with a small NFE. In general, neither solver type is inherently better.


% In Table \ref{tab:hazy}, we study the impact of two step size schedules on the results. On the whole, uniform $\lambda$ performs slightly better than uniform $t$. Our algorithm follows the method of \cite{lu2022dpmsolverplus} to estimate the integral part of the solution, while the analytical part does not affect the error.  Consequently, our algorithm has the same global truncation error, that is $\mathcal{O}\left(h_{max}^{k}\right)$. Note that the initial and final values of $\lambda$ depend on noise schedule and are fixed. Therefore, uniform $\lambda$ scheduling leads to the smallest $h_{max}$ and works better.

\begin{table}[ht]
    \centering
    \begin{minipage}{0.5\textwidth}
    \small
    \renewcommand{\arraystretch}{1}
    \centering
    \caption{Ablation study of network parameterizations on the Rain100H dataset.}
    % \vspace{8pt}
    \resizebox{1\textwidth}{!}{
        \begin{tabular}{cccccc}
			\toprule[1.5pt]
            % \multicolumn{6}{c}{Rainy} \\
            % \cmidrule(lr){1-6}
             NFE & Parameterization      & LPIPS\textdownarrow & FID\textdownarrow &  PSNR\textuparrow & SSIM\textuparrow  \\
            \midrule[1pt]
            \multirow{2}{*}{50}
             & Noise Prediction & \textbf{0.0606}     & \textbf{27.28}   & \textbf{28.89}     & \textbf{0.8615}    \\
             & Data Prediction & 0.0620     & 27.65   & 28.85     & 0.8602    \\
            \cmidrule(lr){1-6}
            \multirow{2}{*}{20}
              & Noise Prediction & 0.1429     & 47.31   & 27.68     & 0.7954    \\
              & Data Prediction & \textbf{0.0635}     & \textbf{27.79}   & \textbf{28.60}     & \textbf{0.8559}    \\
            \cmidrule(lr){1-6}
            \multirow{2}{*}{10}
              & Noise Prediction & 1.376     & 402.3   & 6.623     & 0.0114    \\
              & Data Prediction & \textbf{0.0678}     & \textbf{29.54}   & \textbf{28.09}     & \textbf{0.8483}    \\
            \cmidrule(lr){1-6}
            \multirow{2}{*}{5}
              & Noise Prediction & 1.416     & 447.0   & 5.755     & 0.0051    \\
              & Data Prediction & \textbf{0.0637}     & \textbf{26.92}   & \textbf{28.82}     & \textbf{0.8685}    \\       
            \bottomrule[1.5pt]
        \end{tabular}}
        \label{tab:ablat_param}
    \end{minipage}
    \hspace{0.01\textwidth}
    \begin{minipage}{0.46\textwidth}
    \small
    \renewcommand{\arraystretch}{1}
    \centering
    \caption{Ablation study of solver types on the CelebA-HQ dataset.}
    % \vspace{8pt}
        \resizebox{1\textwidth}{!}{
        \begin{tabular}{cccccc}
			\toprule[1.5pt]
            % \multicolumn{6}{c}{Raindrop} \\     
            % \cmidrule(lr){1-6}
             NFE & Solver Type     & LPIPS\textdownarrow & FID\textdownarrow &  PSNR\textuparrow & SSIM\textuparrow  \\
            \midrule[1pt]
            \multirow{2}{*}{50}
             & ODE & 0.0499     & 22.91   & 28.49     & 0.8921    \\
             & SDE & \textbf{0.0402}     & \textbf{19.09}   & \textbf{29.15}     & \textbf{0.9046}    \\
            \cmidrule(lr){1-6}
            \multirow{2}{*}{20}
              & ODE & 0.0475    & 21.35   & 28.51     & 0.8940    \\
              & SDE & \textbf{0.0408}     & \textbf{19.13}   & \textbf{28.98}    & \textbf{0.9032}    \\
            \cmidrule(lr){1-6}
            \multirow{2}{*}{10}
              & ODE & \textbf{0.0417}    & 19.44   & \textbf{28.94}     & \textbf{0.9048}    \\
              & SDE & 0.0437     & \textbf{19.29}   & 28.48     & 0.8996    \\
            \cmidrule(lr){1-6}
            \multirow{2}{*}{5}
              & ODE & \textbf{0.0526}     & 27.44   & \textbf{31.02}     & \textbf{0.9335}    \\
              & SDE & 0.0529    & \textbf{24.02}   & 28.35     & 0.8930    \\
            \bottomrule[1.5pt]
        \end{tabular}}
        \label{tab:ablat_solver}
    \end{minipage}
\end{table}


% \renewcommand{\arraystretch}{1}
%     \centering
%     \caption{Ablation study of step size schedule on the RESIDE-6k dataset.}
%     % \vspace{8pt}
%         \resizebox{1\textwidth}{!}{
%         \begin{tabular}{cccccc}
% 			\toprule[1.5pt]
%             % \multicolumn{6}{c}{Raindrop} \\     
%             % \cmidrule(lr){1-6}
%              NFE & Schedule      & LPIPS\textdownarrow & FID\textdownarrow &  PSNR\textuparrow & SSIM\textuparrow  \\
%             \midrule[1pt]
%             \multirow{2}{*}{50}
%              & uniform $t$ & 0.0271     & 5.539   & 30.00     & 0.9351    \\
%              & uniform $\lambda$ & \textbf{0.0233}     & \textbf{4.993}   & \textbf{30.19}     & \textbf{0.9427}    \\
%             \cmidrule(lr){1-6}
%             \multirow{2}{*}{20}
%               & uniform $t$ & 0.0313     & 6.000   & 29.73     & 0.9270    \\
%               & uniform $\lambda$ & \textbf{0.0240}     & \textbf{5.077}   & \textbf{30.06}    & \textbf{0.9409}    \\
%             \cmidrule(lr){1-6}
%             \multirow{2}{*}{10}
%               & uniform $t$ & 0.0309     & 6.094   & 29.42     & 0.9274    \\
%               & uniform $\lambda$ & \textbf{0.0246}     & \textbf{5.228}   & \textbf{29.65}     & \textbf{0.9372}    \\
%             \cmidrule(lr){1-6}
%             \multirow{2}{*}{5}
%               & uniform $t$ & 0.0256     & 5.477   & \textbf{29.91}     & 0.9342    \\
%               & uniform $\lambda$ & \textbf{0.0228}     & \textbf{5.174}   & 29.65     & \textbf{0.9416}    \\
%             \bottomrule[1.5pt]
%         \end{tabular}}
%         \label{tab:ablat_schedule}



\subsection{Analysis}\label{analysis}
\label{section5.3}

\begin{figure}[ht!]
    \centering
    \begin{minipage}[t]{0.6\linewidth}
        \centering
        \includegraphics[width=\linewidth, trim=0 20 10 0]{figs/trajectory_a.pdf} %trim左下右上
        \subcaption{Sampling results.}
        \label{fig:traj(a)}
    \end{minipage}
    \begin{minipage}[t]{0.35\linewidth}
        \centering
        \includegraphics[width=\linewidth, trim=0 0 0 0]{figs/trajectory_b.pdf} %trim左下右上
        \subcaption{Trajectory.}
        \label{fig:traj(b)}
    \end{minipage}
    \caption{\textbf{Sampling trajectories.} In (a), we compare our method (with order 1 and order 2) and previous sampling methods (i.e., posterior sampling and Euler discretization) on a motion blurry image. The numbers in parentheses indicate the NFE. In (b), we illustrate trajectories of each sampling method. Previous methods need to take many unnecessary paths to converge. With few NFEs, they fail to reach the ground truth (i.e., the location of $\boldsymbol{x}_0$). Our methods follow a more direct trajectory.}
    \label{fig:traj}
\end{figure}

\textbf{Sampling trajectory.}~ Inspired by the design idea of NCSN \citep{song2019ncsn}, we provide a new perspective of diffusion sampling process. \cite{song2019ncsn} consider each data point (e.g., an image) as a point in high-dimensional space. During the diffusion process, noise is added to each point $\boldsymbol{x}_0$, causing it to spread throughout the space, while the score function (a neural network) \textit{remembers} the direction towards $\boldsymbol{x}_0$. In the sampling process, we start from a random point by sampling a Gaussian distribution and follow the guidance of the reverse-time SDE (or PF-ODE) and the score function to locate $\boldsymbol{x}_0$. By connecting each intermediate state $\boldsymbol{x}_t$, we obtain a sampling trajectory. However, this trajectory exists in a high-dimensional space, making it difficult to visualize. Therefore, we use Principal Component Analysis (PCA) to reduce $\boldsymbol{x}_t$ to two dimensions, obtaining the projection of the sampling trajectory in 2D space. As shown in Figure \ref{fig:traj}, we present an example. Previous sampling methods \citep{luo2024posterior} often require a long path to find $\boldsymbol{x}_0$, and reducing NFE can lead to cumulative errors, making it impossible to locate $\boldsymbol{x}_0$. In contrast, our algorithm produces more direct trajectories, allowing us to find $\boldsymbol{x}_0$ with fewer NFEs.

\begin{figure*}[ht]
    \centering
    \begin{minipage}[t]{0.45\linewidth}
        \centering
        \includegraphics[width=\linewidth, trim=0 0 0 0]{figs/convergence_a.pdf} %trim左下右上
        \subcaption{Sampling results.}
        \label{fig:convergence(a)}
    \end{minipage}
    \begin{minipage}[t]{0.43\linewidth}
        \centering
        \includegraphics[width=\linewidth, trim=0 20 0 0]{figs/convergence_b.pdf} %trim左下右上
        \subcaption{Ratio of convergence.}
        \label{fig:convergence(b)}
    \end{minipage}
    \caption{\textbf{Convergence of noise prediction and data prediction.} In (a), we choose a low-light image for example. The numbers in parentheses indicate the NFE. In (b), we illustrate the ratio of components of neural network output that satisfy the Taylor expansion convergence requirement.}
    \label{fig:converge}
\end{figure*}

\textbf{Numerical stability of parameterizations.}~ From Table 1, we observe poor sampling results for noise prediction in the case of few NFEs. The reason may be that the neural network parameterized by noise prediction is numerically unstable. Recall that we used Taylor expansion in Eq.(\ref{14}), and the condition for the equality to hold is $|\lambda-\lambda_s|<\boldsymbol{R}(s)$. And the radius of convergence $\boldsymbol{R}(t)$ can be calculated by
\begin{equation}
\frac{1}{\boldsymbol{R}(t)}=\lim_{n\rightarrow\infty}\left|\frac{\boldsymbol{c}_{n+1}(t)}{\boldsymbol{c}_n(t)}\right|,
\end{equation}
where $\boldsymbol{c}_n(t)$ is the coefficient of the $n$-th term in Taylor expansion. We are unable to compute this limit and can only compute the $n=0$ case as an approximation. The output of the neural network can be viewed as a vector, with each component corresponding to a radius of convergence. At each time step, we count the ratio of components that satisfy $\boldsymbol{R}_i(s)>|\lambda-\lambda_s|$ as a criterion for judging the convergence, where $i$ denotes the $i$-th component. As shown in Figure \ref{fig:converge}, the neural network parameterized by data prediction meets the convergence criteria at almost every step. However, the neural network parameterized by noise prediction always has components that cannot converge, which will lead to large errors and failed sampling. Therefore, data prediction has better numerical stability and is a more recommended choice.


\paragraph{Summary}
Our findings provide significant insights into the influence of correctness, explanations, and refinement on evaluation accuracy and user trust in AI-based planners. 
In particular, the findings are three-fold: 
(1) The \textbf{correctness} of the generated plans is the most significant factor that impacts the evaluation accuracy and user trust in the planners. As the PDDL solver is more capable of generating correct plans, it achieves the highest evaluation accuracy and trust. 
(2) The \textbf{explanation} component of the LLM planner improves evaluation accuracy, as LLM+Expl achieves higher accuracy than LLM alone. Despite this improvement, LLM+Expl minimally impacts user trust. However, alternative explanation methods may influence user trust differently from the manually generated explanations used in our approach.
% On the other hand, explanations may help refine the trust of the planner to a more appropriate level by indicating planner shortcomings.
(3) The \textbf{refinement} procedure in the LLM planner does not lead to a significant improvement in evaluation accuracy; however, it exhibits a positive influence on user trust that may indicate an overtrust in some situations.
% This finding is aligned with prior works showing that iterative refinements based on user feedback would increase user trust~\cite{kunkel2019let, sebo2019don}.
Finally, the propensity-to-trust analysis identifies correctness as the primary determinant of user trust, whereas explanations provided limited improvement in scenarios where the planner's accuracy is diminished.

% In conclusion, our results indicate that the planner's correctness is the dominant factor for both evaluation accuracy and user trust. Therefore, selecting high-quality training data and optimizing the training procedure of AI-based planners to improve planning correctness is the top priority. Once the AI planner achieves a similar correctness level to traditional graph-search planners, strengthening its capability to explain and refine plans will further improve user trust compared to traditional planners.

\paragraph{Future Research} Future steps in this research include expanding user studies with larger sample sizes to improve generalizability and including additional planning problems per session for a more comprehensive evaluation. Next, we will explore alternative methods for generating plan explanations beyond manual creation to identify approaches that more effectively enhance user trust. 
Additionally, we will examine user trust by employing multiple LLM-based planners with varying levels of planning accuracy to better understand the interplay between planning correctness and user trust. 
Furthermore, we aim to enable real-time user-planner interaction, allowing users to provide feedback and refine plans collaboratively, thereby fostering a more dynamic and user-centric planning process.



\paragraph{Acknowledgements. } 
This research is supported by EPSRC Programme Grant VisualAI EP$\slash$T028572$\slash$1, a Royal Society
Research Professorship RP$\backslash$R1$\backslash$191132, a China Oxford Scholarship and the Hong Kong Research Grants Council -- General
Research Fund (Grant No.: 17211024).
We thank Minghao Chen, Jindong Gu, Zhongrui Gui, Zhenqi He, João Henriques, Zeren Jiang, Zihang Lai, Horace Lee, Kun-Yu Lin, Xianzheng Ma, Christian Rupprecht, Ashish Thandavan, Jianyuan Wang, Kaiyan Zhang, Chuanxia Zheng and Liang Zheng for their help and support for the project.




{
    \small
    \bibliographystyle{ieeenat_fullname}
    \bibliography{main}
}


\clearpage
\newpage
\appendix
\renewcommand{\thetable}{\thesection.\arabic{table}}
\renewcommand\thefigure{\thesection.\arabic{figure}}    

%-------------------------------------------------------------------------
\section{Proofs of Theoretical Analysis}
\label{Append:proof}
Before diving into the theoretical analysis proof, we'll briefly introduce the basic concepts used in our proof to ensure the paper is self-contained.

\begin{definition}[Lipschitz continuity] $f$ is $\lambda$-Lipschitz if for any two points $u,v$ in the domain of $f$, we have following inequality:
\begin{align}
    |f(u)-f(v)|\leq \lambda ||u-v||
\end{align}
\end{definition}
%-------------------------------------------------------------------------
\subsection{Difference between inter-task affinity and proximal inter-task affinity}
\label{Append:differeence_affinity}
Firstly, let's reiterate the definitions of inter-task affinity and proximal inter-task affinity from the main paper.

In a typical SGD process for task $i$ at time step $t$ with input $z^t$, the update rule for $\Theta_s$ is as follows: $\Theta_{s|i}^{t+1} = \Theta_s^t-\eta w_i \nabla_{\Theta_s^t} \mathcal{L}_i(z^t, \Theta_s^t, \Theta_i^t)$ where $z^t$ represents the input data and $\eta$ is the learning rate, $\Theta_{s|i}^{t+1}$ is the updated shared parameters with loss $\mathcal{L}_i$. Then the affinity from task $i$ to $k$ at time step $t$, denoted as $\mathcal{A}^t_{i\rightarrow k}$, is:
\begin{align}
    \mathcal{A}^t_{\textcolor{red}{i\rightarrow k}} &= 1- \frac{\mathcal{L}_k(z^t, \Theta_{s|i}^{t+1}, \Theta_k^{\textcolor{red}{t}})}{\mathcal{L}_k(z^t, \Theta_{s}^{t}, \Theta_k^t)}
    \label{append:definition:inter_task_affinity}
\end{align}

For proximal inter-task affinity, let's consider a multi-task network shared by the task set $G$, with their respective losses defined as $\mathcal{L}_G$. For a data sample $z^t$ and a learning rate $\eta$, the gradients of task set $G$ are updated to the parameters of the network as follows: $\Theta_{s|G}^{t+1} = \Theta_s^t -\eta \nabla_{\Theta_s^t} \mathcal{L}_G (z^t, \Theta_s^t, \Theta_G^t)$ and $\Theta_k^{t+1} = \Theta_k^t -\eta \nabla_{\Theta_k^t} \mathcal{L}_k (z^t, \Theta_s^t, \Theta_k^t)$ for $k \in G$. Then, the proximal inter-task affinity from group $G$ to task $k$ at time step $t$ is defined as:
\begin{align}
    \mathcal{B}^t_{\textcolor{red}{G\rightarrow k}} = 1- \frac{\mathcal{L}_k(z^t, \Theta_{s|\textcolor{red}{G}}^{t+1}, \Theta_k^{\textcolor{red}{t+1}})}{\mathcal{L}_k(z^t, \Theta_{s}^{t}, \Theta_k^t)}
    \label{append:definition:proximal_inter_task_affinity}
\end{align}

The primary distinction between the two affinities lies in the incorporation of the task set and the update of task-specific parameters (indicated by \textcolor{red}{red letters}). Proximal inter-task affinity is an expanded concept that integrates the task set rather than individual tasks as the source task. This difference is evident from the notation, where $i \rightarrow k$ in \Cref{append:definition:inter_task_affinity} and $G \rightarrow k$ in \Cref{append:definition:proximal_inter_task_affinity}.

The second main difference lies in the update of task-specific parameters. In the inter-task affinity in \Cref{append:definition:inter_task_affinity}, the denominator includes $\Theta_k^t$, while in the proximal inter-task affinity in \Cref{append:definition:proximal_inter_task_affinity}, it includes $\Theta_k^{t+1}$, which is a subtle distinction that may not be noticed by readers.
These two modifications allow us to track proximal inter-task affinity while simultaneously optimizing multi-task networks.

When measuring affinity under the assumption of a convex objective, the proximal inter-task affinity is equal to or greater than the inter-task affinity. This also aligns well with real-world scenarios, as proximal inter-task affinity reflects updates to task-specific parameters, as shown below.
\begin{align}
    \mathcal{A}^t_{i\rightarrow k} &= 1- \frac{\mathcal{L}_k(z^t, \Theta_{s|i}^{t+1}, \Theta_k^t)}{\mathcal{L}_k(z^t, \Theta_{s}^{t}, \Theta_k^t)} \leq 1- \frac{\mathcal{L}_k(z^t, \Theta_{s|i}^{t+1}, \Theta_k^{t+1})}{\mathcal{L}_k(z^t, \Theta_{s}^{t}, \Theta_k^t)} = \mathcal{B}^t_{i\rightarrow k}
\end{align}

This inequality can also be applied to an expanded setting that incorporates task sets. If we expand the concept of inter-task affinity from individual task to task set as $\mathcal{A}^t_{G \rightarrow k}$, then the inequality $\mathcal{A}^t_{G \rightarrow k} \leq \mathcal{B}^t_{G \rightarrow k}$ is satisfied. If $k \notin G$ then, $\mathcal{A}^t_{G \rightarrow k} = \mathcal{B}^t_{G \rightarrow k}$ holds.

For ease of notation, we use the expanded version of the affinity for multiple tasks throughout the proof, as follows:
\begin{align}
    \mathcal{A}^t_{G\rightarrow k} &= 1- \frac{\mathcal{L}_k(z^t, \Theta_{s|G}^{t+1}, \Theta_k^t)}{\mathcal{L}_k(z^t, \Theta_{s}^{t}, \Theta_k^t)}
    \label{Append:expanded_affinity}
\end{align}
This differs from proximal inter-task affinity, as it does not consider the update of task-specific parameters.

%-------------------------------------------------------------------------
\subsection{Proof of \Cref{theorem1}}
\label{Append:theorem1}

\theomone*
\begin{proof}
Let's consider a scenario where we update the network parameters $\Theta$ with task-specific losses $\mathcal{L}_i$ and $\mathcal{L}_k$ simultaneously at time step $t$ with input $z^t$. Applying the Taylor expansion, we obtain the following:
\begin{align}
    \mathcal{L}_k(z^t, \Theta_{s|i,k}^{t+1}, \Theta_k^{t})
    &\simeq \mathcal{L}_k (z^t, \Theta_s^t, \Theta_k^t) + (\Theta_{s|i,k}^{t+1} - \Theta_s^t) \nabla_{\Theta_s^t} \mathcal{L}_k (z^t, \Theta_s^t, \Theta_k^t) + O(\eta^2) \\
    &= \mathcal{L}_k (z^t, \Theta_{s}^{t}, \Theta_k^{t})-\eta g_k\cdot(g_i+g_k) + O(\eta^2)
\end{align}
where $g_i$ and $g_k$ represent the gradients backpropagated from the losses $\mathcal{L}_i$ and $\mathcal{L}_k$, respectively, with respect to the shared parameters $\Theta_s^t$. For instance, $g_i = \nabla_{\Theta_s^t} \mathcal{L}_i (z^t, \Theta_s^t, \Theta_i)$.

Reorganizing the inequality to align with the format of inter-task affinity, we obtain:  
\begin{align}
    \mathcal{A}_{i,k\rightarrow k}^t = 1-\frac{\mathcal{L}_k(z^t, \Theta_{s|i,k}^{t+1}, \Theta_k^{t})}{\mathcal{L}_k (z^t, \Theta_{s}^t, \Theta_k^{t})} \simeq \frac{1}{\mathcal{L}_k (z^t, \Theta_{s}^t, \Theta_k^{t})}\biggl(\eta g_k\cdot(g_i+g_k) + O(\eta^2)\biggr)
\end{align}
Similar results can be obtained for $A_{j,k\rightarrow k}^t$.
\begin{align}
    \mathcal{A}_{j,k\rightarrow k}^t = 1-\frac{\mathcal{L}_k(z^t, \Theta_{s|j,k}^{t+1}, \Theta_k^{t})}{\mathcal{L}_k (z^t, \Theta_{s}^t, \Theta_k^{t})} \simeq \frac{1}{\mathcal{L}_k (z^t, \Theta_{s}^t, \Theta_k^{t})}\biggl(\eta g_k\cdot(g_j+g_k) + O(\eta^2)\biggr)
\end{align}
From $\mathcal{A}_{i,k \rightarrow k}^t \geq \mathcal{A}_{j,k \rightarrow k}^t$ and by ignoring the $O(\eta^2)$ term with a sufficiently small learning rate $\eta \ll 1$, we can derive the result:
\begin{align}
    g_i \cdot g_k \geq g_j \cdot g_k
\end{align}
\end{proof}
The findings indicate that grouping tasks with positive inter-task affinity exhibits better alignment in task-specific gradients compared to grouping tasks with negative inter-task affinity, thereby validating the grouping strategies employed by our algorithm. Furthermore, we analyze how this alignment in task-specific gradients contributes to reducing the loss of task $k$ in \Cref{theorem2}.


%-------------------------------------------------------------------------
\subsection{Proof of \Cref{theorem2}}
\label{Append:theorem2}

\theomtwo*
\begin{proof}
Let's consider a scenario where we update the network parameters $\Theta_s^t$ with task-specific losses $\mathcal{L}_i$ and $\mathcal{L}_k$ simultaneously at time step $t$ with input $z^t$. Let $g_i$ denote the gradients backpropagated from the loss $\mathcal{L}_i$ with respect to the shared parameters $\Theta_s^t$, expressed as $g_i = \nabla{\Theta_s^t} \mathcal{L}_i (z^t, \Theta_s^t, \Theta_i)$.

Using the first-order Taylor approximation of $\mathcal{L}_k$ for $\Theta_s^t$, we obtain:
\begin{align}
    \mathcal{L}_k (z^t, \Theta_{s|i,k}^{t+1}, \Theta_k^t) &= \mathcal{L}_k (z^t, \Theta_s^t, \Theta_k^t) + (\Theta_{s|i,k}^{t+1} - \Theta_s^t) \nabla_{\Theta_s^t} \mathcal{L}_k (z^t, \Theta_s^t, \Theta_k^t) + O(\eta^2)\\
    &= \mathcal{L}_k (z^t, \Theta_s^t, \Theta_k^t) - \eta (g_i + g_k)\cdot g_k + O(\eta^2)
    \label{eq:theo2_pre1}
\end{align}

For task $j$, we can follow a similar process as follows:
\begin{align}
    \mathcal{L}_k (z^t, \Theta_{s|j,k}^{t+1}, \Theta_k^t) = \mathcal{L}_k (z^t, \Theta_s^t, \Theta_k^t) - \eta (g_j + g_k)\cdot g_k + O(\eta^2)
    \label{eq:theo2_pre2}
\end{align}

With a sufficiently small learning rate $\eta \ll 1$, subtract \cref{eq:theo2_pre2} from \cref{eq:theo2_pre1}, then:
\begin{align}
    \mathcal{L}_k (z^t, \Theta_{s|i,k}^{t+1}, \Theta_k^t) - \mathcal{L}_k (z^t, \Theta_{s|j,k}^{t+1}, \Theta_k^t) &= - \eta (g_i + g_k)\cdot g_k + \eta (g_j + g_k)\cdot g_k \\
    &= - \eta(g_i-g_j)\cdot g_k \leq 0
    \label{eq:theo2_result}
\end{align}
which proves the results.
\end{proof}

The result indicates that when the gradients $g_i$ from task $i$ align better with those of the reference task $k$ compared to task $j$, the loss on the reference task $k$ tends to be lower with updated gradients $g_i + g_k$ compared to $g_j + g_k$, especially for sufficiently small learning rates $\eta$. 



%-------------------------------------------------------------------------
\subsection{Proof of \Cref{theorem3}}
\label{Append:theorem3}

\theomthree*
\begin{proof}
Let's begin with the definition of inter-task affinity between $\{i, j\}\rightarrow k$ and $i \rightarrow k$ as follows:
\begin{align}
    \mathcal{A}_{i,k \rightarrow k}^t &= 1-\frac{\mathcal{L}_k(z^t, \Theta_{s|i,k}^{t+1}, \Theta_k^{t})}{\mathcal{L}_k(z^t, \Theta_{s}^t, \Theta_k^{t})} &
    \mathcal{A}_{i\rightarrow k}^t &= 1-\frac{\mathcal{L}_k(z^t, \Theta_{s|i}^{t+1}, \Theta_k^{t})}{\mathcal{L}_k(z^t, \Theta_{s}^t, \Theta_k^{t})}
    \label{eq:theo5_affin}
\end{align}

When updating $i$ and $k$ simultaneously, we can derive the first-order Taylor approximation of $\mathcal{L}_k$ for $\Theta_s^t$ as follows:
\begin{align}
    \mathcal{L}_k (z^t, \Theta_{s|i,k}^{t+1}, \Theta_k^t) &= \mathcal{L}_k (z^t, \Theta_s^t, \Theta_k^t) + (\Theta_{s|i,k}^{t+1} - \Theta_s^t) \nabla_{\Theta_s^t} \mathcal{L}_k (z^t, \Theta_s^t, \Theta_k^t) + O(\eta^2)\\
    &= \mathcal{L}_k (z^t, \Theta_s^t, \Theta_k^t) - \eta (g_i + g_k)\cdot g_k + O(\eta^2)
\end{align}

Similarly, when updating $i$ alone, the first-order Taylor approximation of $\mathcal{L}_k$ for $\Theta_s^t$ is as follows:
\begin{align}
    \mathcal{L}_k (z^t, \Theta_{s|i}^{t+1}, \Theta_k^t) &= \mathcal{L}_k (z^t, \Theta_s^t, \Theta_k^t) + (\Theta_{s|i}^{t+1} - \Theta_s^t) \nabla_{\Theta_s^t} \mathcal{L}_k (z^t, \Theta_s^t, \Theta_k^t) + O(\eta^2)\\
    &= \mathcal{L}_k (z^t, \Theta_s^t, \Theta_k^t) - \eta g_i\cdot g_k + O(\eta^2)
\end{align}

With a sufficiently small learning rate $\eta$, the difference between the two inter-task affinities in \cref{eq:theo5_affin} can be expressed as follows:
\begin{align}
    \mathcal{A}_{i,k \rightarrow k}^t - \mathcal{A}_{i\rightarrow k}^t &= 1-\frac{\mathcal{L}_k(z^t, \Theta_{s|i,k}^{t+1}, \Theta_k^{t})}{\mathcal{L}_k(z^t, \Theta_{s}^t, \Theta_k^{t})} - \biggr(1-\frac{\mathcal{L}_k(z^t, \Theta_{s|i}^{t+1}, \Theta_k^{t})}{\mathcal{L}_k(z^t, \Theta_{s}^t, \Theta_k^{t})}\biggr) \\
    &= \frac{\mathcal{L}_k(z^t, \Theta_{s|i}^{t+1}, \Theta_k^{t}) - \mathcal{L}_k(z^t, \Theta_{s|i,k}^{t+1}, \Theta_k^{t})}{\mathcal{L}_k(z^t, \Theta_{s}^t, \Theta_k^{t})}\\
    &= \frac{\mathcal{L}_k (z^t, \Theta_s^t, \Theta_k^t) - \eta g_i\cdot g_k - (\mathcal{L}_k (z^t, \Theta_s^t, \Theta_k^t) - \eta (g_i + g_k)\cdot g_k)}{\mathcal{L}_k(z^t, \Theta_{s}^t, \Theta_k^{t})}\\
    &= \frac{\eta||g_k||^2}{\mathcal{L}_k(z^t, \Theta_{s}^t, \Theta_k^{t})} \\
    &\geq 0
    \label{eq:theo5_result}
\end{align}
The inequality in \cref{eq:theo5_result} proves that $\mathcal{A}_{i,k \rightarrow k}^t \geq \mathcal{A}_{i \rightarrow k}^t$.
\end{proof}

When tasks $i$ and $k$ are within the same task group, we can access $\mathcal{B}_{i,k \rightarrow k}^t$ during the optimization process. If $\mathcal{B}_{i,k \rightarrow k}^t \leq 0$, the inter-task affinity also satisfies $\mathcal{A}_{i,k \rightarrow k}^t \leq 0$ as $\mathcal{A}_{i,k \rightarrow k}^t \leq \mathcal{B}_{i,k \rightarrow k}^t$. According to \Cref{theorem3}, this condition implies $\mathcal{A}_{i\rightarrow k}^t\leq 0$. The proposed algorithm separates these tasks into different groups when $\mathcal{B}_{i,k \rightarrow k}^t \leq 0$ which justifies our grouping rules.

Conversely, when tasks $i$ and $j$ belong to separate task groups, we only have access to $\mathcal{B}_{i \rightarrow k}^t$ instead of $\mathcal{B}_{i,k \rightarrow k}^t$. In this scenario, the proposed algorithm merges these tasks into the same group if $\mathcal{B}_{i \rightarrow k}^t = \mathcal{A}_{i \rightarrow k}^t \geq 0$. This inequality also implies $\mathcal{A}_{i,k\rightarrow k}^t\geq 0$, justifying the merging of tasks $i$ and $k$ based on $\mathcal{B}_{i \rightarrow k}^t$ during optimization.

%-------------------------------------------------------------------------
\subsection{Proof of \Cref{theorem4}}
\label{Append:theorem4}

\theomfour*

Let's represent the sum of losses of tasks included in $G_m$ as $\mathcal{L}_{G_m}$, defined as follows:
\begin{align}
    \mathcal{L}_{G_m}(z^t, \Theta_{s|G_m}^{t+m/\mathcal{M}}, \Theta_{G_m}^{t+m/\mathcal{M}}) = \sum_{k \in G_m} \mathcal{L}_k(z^t, \Theta_{s|G_m}^{t+m/\mathcal{M}}, \Theta_{k}^{t+m/\mathcal{M}})
\end{align}
where $\Theta_{s|G_m}^{t+m/\mathcal{M}}$ denotes the shared parameters, while $\Theta_{G_m}^{t+m/\mathcal{M}}$  represents the set of task-specific parameters within $G_m$ after updating tasks in $G_m$.

We begin by expanding the task-specific loss $\mathcal{L}_{G_m}$ in terms of the shared parameter $\Theta_{s|G_m}^{t+m/\mathcal{M}}$ and the task-specific parameters $\Theta_{G_m}^{t+m/\mathcal{M}}$ using a quadratic expansion. During this process, the task-specific parameters $\Theta_{G_m}^{t+(m-1)/\mathcal{M}}=\Theta_{G_m}^t$ and $\Theta_{G_m}^{t+m/\mathcal{M}}=\Theta_{G_m}^{t+1}$, since the task-specific parameters in $G_m$ are updated only once from $\Theta_{G_m}^{t+(m-1)/\mathcal{M}}$ to $\Theta_{G_m}^{t+m/\mathcal{M}}$.
\begin{align}
    \mathcal{L}_{G_m} (z^t, \Theta_{s|G_m}^{t+m/\mathcal{M}},& \Theta_{G_m}^{t+1}) \leq \mathcal{L}_{G_m} (z^t, \Theta_{s|G_{m-1}}^{t+(m-1)/\mathcal{M}}, \Theta_{G_m}^t) \label{eq:theo4_in0}\\
    &+\nabla_{\Theta_{s|G_{m-1}}^{t+(m-1)/\mathcal{M}}}\mathcal{L}_{G_m}(z^t, \Theta_{s|G_{m-1}}^{t+(m-1)/\mathcal{M}}, \Theta_{G_m}^t)(\Theta_{s|G_m}^{t+m/\mathcal{M}}-\Theta_{s|G_{m-1}}^{t+(m-1)/\mathcal{M}})\\
    &+\frac{1}{2}\nabla_{\Theta_{s|G_{m-1}}^{t+(m-1)/\mathcal{M}}}^{2}\mathcal{L}_{G_m}(z^t, \Theta_{s|G_{m-1}}^{t+(m-1)/\mathcal{M}}, \Theta_{G_m}^t)(\Theta_{s|G_m}^{t+m/\mathcal{M}}-\Theta_{s|G_m}^{t+(m-1)/\mathcal{M}})^{2}\\
    &+\nabla_{\Theta_{G_m}^t}\mathcal{L}_{G_m}(z^t, \Theta_{s|G_{m-1}}^{t+(m-1)/\mathcal{M}}, \Theta_{G_m}^t)(\Theta_{G_m}^{t+1}-\Theta_{G_m}^t)\\
    &+\frac{1}{2}\nabla_{\Theta_{G_m}^t}^2 \mathcal{L}_{G_m}(z^t, \Theta_{s|G_{m-1}}^{t+(m-1)/\mathcal{M}}, \Theta_{G_m}^t)(\Theta_{G_m}^{t+1}-\Theta_{G_m}^t)^{2}\\
    \leq &\mathcal{L}_{G_m} (z^t, \Theta_{s|G_m}^{t+(m-1)/\mathcal{M}}, \Theta_{G_m}^t)\\
    &+\nabla_{\Theta_{s|G_{m-1}}^{t+(m-1)/\mathcal{M}}}\mathcal{L}_k(z^t, \Theta_{s|G_{m-1}}^{t+(m-1)/\mathcal{M}}, \Theta_{G_m}^t)(\Theta_{s|G_m}^{t+m/\mathcal{M}}-\Theta_{s|G_{m-1}}^{t+(m-1)/\mathcal{M}})\\
    &+\frac{1}{2} H|G_m| (\Theta_{s|G_m}^{t+m/\mathcal{M}}-\Theta_{s|G_{m-1}}^{t+(m-1)/\mathcal{M}})^{2}\\
    &+\nabla_{\Theta_{G_m}^t}\mathcal{L}_{G_m}(z^t, \Theta_{s|G_{m-1}}^{t+(m-1)/\mathcal{M}}, \Theta_{G_m}^t)(\Theta_{G_m}^{t+1}-\Theta_{G_m}^t)\\
    &+\frac{1}{2} H|G_m| (\Theta_{G_m}^{t+1}-\Theta_{G_m}^t)^{2}
\end{align}
where $|G_m|$ represents the number of tasks in $G_m$. The inequality holds with the Lipschitz continuity of $\nabla \mathcal{L}$ with a constant $H$.

For the shared parameters of the network, the update rule is as follows:
\begin{align}
    \Theta_{s|G_m}^{t+m/\mathcal{M}} &= \Theta_{s|G_{m-1}}^{t+(m-1)/\mathcal{M}} - \eta \nabla_{\Theta_{s|G_{m-1}}^{t+(m-1)/\mathcal{M}}} \mathcal{L}_{G_m}(z^t, \Theta_{s|G_{m-1}}^{t+(m-1)/\mathcal{M}}, \Theta_{G_m}^t)\\
    &= \Theta_{s|G_{m-1}}^{t+(m-1)/\mathcal{M}} - \eta g_{s, G_m}^{t+(m-1)/\mathcal{M}}
    \label{eq:theo4_in1}
\end{align}
where $g_{s, G_m}^{t+(m-1)/\mathcal{M}}$ is the gradients of the shared parameters with respect to loss of tasks in $G_m$.

Similarly, the task-specific parameters of the network, the update rule is as follows:
\begin{align}
    \Theta_{G_m}^{t+1} &= \Theta_{G_m}^{t} - \eta \nabla_{\Theta_{G_m}^{t}} \mathcal{L}_{G_m}(z^t, \Theta_{s|G_{m-1}}^{t+(m-1)/\mathcal{M}}, \Theta_{G_m}^t) = \Theta_{G_m}^t - \eta g_{ts, G_m}^{t}
    \label{eq:theo4_in2}
\end{align}
where $g_{ts, G_m}^t$ is the gradients of the task-specific parameters with respect to the loss of tasks in $G_m$.

If we substitute \cref{eq:theo4_in1} and \cref {eq:theo4_in2} into the result of \cref{eq:theo4_in0}, it become as follows:
\begin{align}
    \mathcal{L}_{G_m} (z^t, \Theta_{s|G_m}^{t+m/\mathcal{M}}, \Theta_{G_m}^{t+1}) \leq& \mathcal{L}_{G_m} (z^t, \Theta_{s|G_{m-1}}^{t+(m-1)/\mathcal{M}}, \Theta_{G_m}^t)\\
    &- \eta ||g_{s, G_m}^{t+(m-1)/\mathcal{M}}||^2  + \frac{\eta^2 H|G_m|}{2}||g_{s, G_m}^{t+(m-1)/\mathcal{M}}||^2 \\
    &- \eta ||g_{ts, G_m}^t||^2  + \frac{\eta^2 H|G_m|}{2}||g_{ts, G_m}^t||^2
\end{align}

We can derive similar results for the loss of task group $G_i$, where the index $i$ is not the same as the updating group sequence $m$ ($i \neq m$). This process follows similarly to the one described above. For the step from $t+(m-1)/\mathcal{M}$ to $t+m/\mathcal{M}$, the task-specific parameters in $G_i$ remain unchanged.
\begin{align}
    \mathcal{L}_{G_i} (z^t, \Theta_{s|G_m}^{t+m/\mathcal{M}},& \Theta_{G_i}^t) \leq \mathcal{L}_{G_i} (z^t, \Theta_{s|G_{m-1}}^{t+(m-1)/\mathcal{M}}, \Theta_{G_i}^t)\\
    &+\nabla_{\Theta_{s|G_{m-1}}^{t+(m-1)/\mathcal{M}}}\mathcal{L}_{G_i}(z^t, \Theta_{s|G_{m-1}}^{t+(m-1)/\mathcal{M}}, \Theta_{G_i}^t)(\Theta_{s|G_m}^{t+m/\mathcal{M}}-\Theta_{s|G_{m-1}}^{t+(m-1)/\mathcal{M}})\\
    &+\frac{1}{2}\nabla_{\Theta_{s|G_{m-1}}^{t+(m-1)/\mathcal{M}}}^{2}\mathcal{L}_{G_i}(z^t, \Theta_{s|G_{m-1}}^{t+(m-1)/\mathcal{M}}, \Theta_{G_i}^t)(\Theta_{s|G_m}^{t+m/\mathcal{M}}-\Theta_{s|G_m}^{t+(m-1)/\mathcal{M}})^{2}\\
    % &+\nabla_{\Theta_{G_m}^t}\mathcal{L}_{G_i}(z^t, \Theta_{s|G_{m-1}}^{t+(m-1)/\mathcal{M}}, \Theta_{G_i}^t)(\Theta_{G_m}^{t+1}-\Theta_{G_m}^t)\\
    % &+\frac{1}{2}\nabla_{\Theta_{G_m}^t}^2 \mathcal{L}_{G_i}(z^t, \Theta_{s|G_{m-1}}^{t+(m-1)/\mathcal{M}}, \Theta_{G_m}^t)(\Theta_{G_m}^{t+1}-\Theta_{G_m}^t)^{2}\\
    \leq &\mathcal{L}_{G_i} (z^t, \Theta_{s|G_m}^{t+(m-1)/\mathcal{M}}, \Theta_{G_i}^t)\\
    &+\nabla_{\Theta_{s|G_{m-1}}^{t+(m-1)/\mathcal{M}}}\mathcal{L}_{G_i}(z^t, \Theta_{s|G_{m-1}}^{t+(m-1)/\mathcal{M}}, \Theta_{G_i}^t)(\Theta_{s|G_m}^{t+m/\mathcal{M}}-\Theta_{s|G_{m-1}}^{t+(m-1)/\mathcal{M}})\\
    &+\frac{1}{2} H|G_i| (\Theta_{s|G_m}^{t+m/\mathcal{M}}-\Theta_{s|G_{m-1}}^{t+(m-1)/\mathcal{M}})^{2}\\
    % &+\nabla_{\Theta_{G_m}^t}\mathcal{L}_{G_i}(z^t, \Theta_{s|G_{m-1}}^{t+(m-1)/\mathcal{M}}, \Theta_{G_m}^t)(\Theta_{G_m}^{t+1}-\Theta_{G_m}^t)\\
    % &+\frac{1}{2} H|G_i| (\Theta_{G_m}^{t+1}-\Theta_{G_m}^t)^{2}\\
    \leq & \mathcal{L}_{G_i} (z^t, \Theta_{s|G_m}^{t+(m-1)/\mathcal{M}}, \Theta_{G_i}^t)\\
    &- \eta g_{s, G_i}^{t+(m-1)/\mathcal{M}} \cdot g_{s, G_m}^{t+(m-1)/\mathcal{M}} + \frac{\eta^2 H|G_i|}{2}||g_{s, G_m}^{t+(m-1)/\mathcal{M}}||^2\\
    % &- \eta g_{ts, G_i}^t \cdot g_{ts, G_m}^t  + \frac{\eta^2 H|G_i|}{2}||g_{ts, G_m}^t||^2 \\
\end{align}

Then the total loss of multiple task groups can be expressed as follows:
\begin{align}
    \sum_{k=1}^{\mathcal{M}} \mathcal{L}_{G_k}(z^t, &\Theta_{s|G_m}^{t+m/\mathcal{M}}, \Theta_{G_k}^{t+1}) \leq \sum_{k=1}^{\mathcal{M}} \mathcal{L}_{G_k}(z^t, \Theta_{s|G_m}^{t+(m-1)/\mathcal{M}}, \Theta_{G_k}^t) \\
    &-\eta\sum_{k=1}^{\mathcal{M}} g_{s, G_k}^{t+(m-1)/\mathcal{M}} \cdot g_{s, G_m}^{t+(m-1)/\mathcal{M}} + \frac{\eta^2 H}{2} ||g_{s, G_m}^{t+(m-1)/\mathcal{M}}||^2 \sum_{k=1}^{\mathcal{M}} |G_k| \label{eq:theo4_lip1}\\ 
    &- \eta ||g_{ts, G_m}^t||^2  + \frac{\eta^2 H|G_m|}{2}||g_{ts, G_m}^t||^2 \label{eq:theo4_lip2}\\
    \leq&\sum_{k=1}^{\mathcal{M}} \mathcal{L}_{G_k}(z^t, \Theta_{s|G_m}^{t+(m-1)/\mathcal{M}}, \Theta_{G_k}^t) \label{eq:theo4_lip_out_0}\\
    &-\eta\sum_{k=1}^{\mathcal{M}} g_{s, G_k}^{t+(m-1)/\mathcal{M}} \cdot g_{s, G_m}^{t+(m-1)/\mathcal{M}} + \eta ||g_{s, G_m}^{t+(m-1)/\mathcal{M}}||^2 - \frac{\eta}{2}||g_{ts, G_m}^t||^2 \label{eq:theo4_lip_out}\\
    =& \sum_{k=1}^{\mathcal{M}} \mathcal{L}_{G_k}(z^t, \Theta_{s|G_m}^{t+(m-1)/\mathcal{M}}, \Theta_{G_k}^t)\\
    &-\eta g_{s, G_m}^{t+(m-1)/\mathcal{M}} \cdot (\sum_{k=1}^{\mathcal{M}} g_{s, G_k}^{t+(m-1)/\mathcal{M}} - g_{s, G_m}^{t+(m-1)/\mathcal{M}}) - \frac{\eta}{2}||g_{ts, G_m}^t||^2
    \label{eq:theo4_result}
\end{align}


The inequality between \cref{eq:theo4_lip1} and the first term in \cref{eq:theo4_lip_out} requires $\eta\leq \frac{2}{H\cdot \sum_{k=1}^{\mathcal{M}} |G_k|} = \frac{2}{H \mathcal{K}}$, while the inequality between \cref{eq:theo4_lip2} and the second term in \cref{eq:theo4_lip_out} requires $\eta\leq \frac{1}{H|G_M|}$. Therefore, the inequality in \cref{eq:theo4_lip_out_0} holds when $\eta \leq \min(\frac{2}{H\mathcal{K}}, \frac{1}{H|G_M|})$. Previous approaches, which handle updates of shared and task-specific parameters independently, failing to capture their interdependence during optimization.
The term, $g_{s, G_m}^{t+(m-1)/\mathcal{M}} \cdot (\sum_{k=1}^{\mathcal{M}} g_{s, G_k}^{t+(m-1)/\mathcal{M}})$, fluctuates during optimization. When the gradients of group $G_m$ align well with the gradients of the other groups $\{G_k\}_{i=1, i\neq m}^{\mathcal{M}}$, their dot product yields a positive value, leading to a decrease in multi-task losses. However, in practice, the sequential update strategy demonstrates a similar level of stability in optimization, which appears to contradict the conventional results. Thus, we assume a correlation between the learning of shared parameters and task-specific parameters, where the learning of task-specific parameters reduces gradient conflicts in shared parameters. Under this assumption, the sequential update strategy can guarantee convergence to Pareto-stationary points. This assumption is reasonable, as task-specific parameters capture task-specific information, thereby reducing conflicts in the shared parameters across tasks.
 
 
%-------------------------------------------------------------------------
\subsection{Two-Step Proximal Inter-task Affinity}
\label{Append:two_step_proximal_inter_task_affinity}

Before delving into the proof of Theorem 5, let's introduce the concept of two-step proximal inter-task affinity, which extends the notion of proximal inter-task affinity over two update steps.
\begin{definition}[Two-Step Proximal Inter-Task Affinity] Consider a multi-task network shared by the tasks $i, j, k$, with their respective losses denoted as $\mathcal{L}_i, \mathcal{L}_j, \mathcal{L}_k$. Sequential updates of $(\{j\}, \{i, k\})$ result in parameters being updated from $\Theta_s^{t} \rightarrow \Theta_{s|j}^{t+1} \rightarrow \Theta_{s|i,k}^{t+2}$ and $\Theta_k^{t} \rightarrow \Theta_k^{t+1} \rightarrow \Theta_k^{t+2}$. Then, the two-step proximal inter-task affinity from sequential update $(\{j\}, \{i, k\})$ to $k$ at time step $t$ is defined as follows:
\begin{align}
    \mathcal{B}^t_{j; i,k\rightarrow k} &= 1-(1-\mathcal{B}^t_{j\rightarrow k})(1-\mathcal{B}^{t+1}_{i,k\rightarrow k}) \\
    &= 1-\frac{\mathcal{L}_k(z^t, \Theta_{s|j}^{t+1}, \Theta_k^{t+1})}{\mathcal{L}_k(z^t, \Theta_s^{t}, \Theta_k^{t})} \cdot \frac{\mathcal{L}_k(z^t, \Theta_{s|i,k}^{t+2}, \Theta_k^{t+2})}{\mathcal{L}_k(z^t, \Theta_{s|j}^{t+1}, \Theta_k^{t+1})} = 1-\frac{\mathcal{L}_k(z^t, \Theta_{s|i,k}^{t+2}, \Theta_k^{t+2})}{\mathcal{L}_k(z^t, \Theta_s^{t}, \Theta_k^{t})}
\end{align}
\end{definition}

%-------------------------------------------------------------------------
\subsection{Proof of \Cref{theorem5}}
\label{Append:theorem5}

\theomfive*
\begin{proof}
We compare the loss after jointly updating three tasks $\{i, j, k\}$ with the loss after sequentially updating the task sets $\{i, k\}$ and $\{j\}$. To assess the impact of the updating order of task sets, we also conduct the analysis on the reverse order of task set $\{j\}, \{i, k\}$.

(i) Let's begin with the definition of proximal inter-task affinity between $\{i, k\}\rightarrow k$ and $j \rightarrow k$, taking into account the updates of task-specific parameters as follows:
\begin{align}
    \mathcal{B}_{i,j,k \rightarrow k}^{t+(m-1)/\mathcal{M}} &= 1-\frac{\mathcal{L}_k(z^t, \Theta_{s|i,j,k}^{t+m/\mathcal{M}}, \Theta_k^{t+m/\mathcal{M}})}{\mathcal{L}_k(z^t, \Theta_{s}^{t+(m-1)/\mathcal{M}}, \Theta_k^{t+(m-1)/\mathcal{M}})} \label{eq:theo5_joint}
\end{align}

\begin{align}
    \mathcal{B}_{i,k \rightarrow k}^{t+(m-1)/\mathcal{M}} &= 1-\frac{\mathcal{L}_k(z^t, \Theta_{s|i,k}^{t+m/\mathcal{M}}, \hat{\Theta}_k^{t+m/\mathcal{M}})}{\mathcal{L}_k(z^t, \Theta_{s}^{t+(m-1)/\mathcal{M}}, \Theta_k^{t+(m-1)/\mathcal{M}})}
\end{align}
\begin{align}
    \mathcal{B}_{j\rightarrow k}^{t+m/\mathcal{M}} &= 1-\frac{\mathcal{L}_k(z^t, \Theta_{s|j}^{t+(m+1)/\mathcal{M}}, \Theta_k^{t+(m+1)/\mathcal{M}})}{\mathcal{L}_k(z^t, \Theta_{s|i,k}^{t+m/\mathcal{M}}, \hat{\Theta}_k^{t+m/\mathcal{M}})}\\
    &= 1-\frac{\mathcal{L}_k(z^t, \Theta_{s|j}^{t+(m+1)/\mathcal{M}}, \hat{\Theta}_k^{t+m/\mathcal{M}})}{\mathcal{L}_k(z^t, \Theta_{s|i,k}^{t+m/\mathcal{M}}, \hat{\Theta}_k^{t+m/\mathcal{M}})}
\end{align}

where $\hat{\Theta}_k^{t+m/\mathcal{M}}$ represents the resulting task-specific parameters of $k$ immediately after updating the task set $\{i, j\}$. This notation is used to differentiate it from the task-specific parameter $\Theta_k^{t+m/\mathcal{M}}$ obtained after jointly updating all tasks.

The two-step proximal inter-task affinity with the sequence $\{i, k\}$ and $\{j\}$ can be represented as follows:
\begin{align}
    \mathcal{B}_{i,k; j \rightarrow k}^{t+(m-1)/\mathcal{M}} &= 1-(1-\mathcal{B}_{i,k \rightarrow k}^{t+(m-1)/\mathcal{M}}) \cdot (1-\mathcal{B}_{j\rightarrow k}^{t+m/\mathcal{M}})\\ 
    &= 1-\frac{\mathcal{L}_k(z^t, \Theta_{s|i,k}^{t+m/\mathcal{M}}, \hat{\Theta}_k^{t+m/\mathcal{M}})}{\mathcal{L}_k(z^t, \Theta_{s}^{t+(m-1)/\mathcal{M}}, \Theta_k^{t+(m-1)/\mathcal{M}})} \cdot \frac{\mathcal{L}_k(z^t, \Theta_{s|j}^{t+(m+1)/\mathcal{M}}, \hat{\Theta}_k^{t+m/\mathcal{M}})}{\mathcal{L}_k(z^t, \Theta_{s|i,k}^{t+m/\mathcal{M}}, \hat{\Theta}_k^{t+m/\mathcal{M}})}\\
    &= 1-\frac{\mathcal{L}_k(z^t, \Theta_{s|j}^{t+(m+1)/\mathcal{M}}, \hat{\Theta}_k^{t+m/\mathcal{M}})}{\mathcal{L}_k(z^t, \Theta_{s}^{t+(m-1)/\mathcal{M}}, \Theta_k^{t+(m-1)/\mathcal{M}})} \label{eq:theo5_prox}
\end{align}

Our objective is to compare $\mathcal{B}_{i,j,k \rightarrow k}^{t+(m-1)/\mathcal{M}}$ from \cref{eq:theo5_joint} with $\mathcal{B}_{i,k; j \rightarrow k}^{t+(m-1)/\mathcal{M}}$ from \cref{eq:theo5_prox} to assess each update's effect on the final loss. Since both equations share a common denominator, we only need to compare the numerators of each equation. Using the first-order Taylor approximation of $\mathcal{L}_k(z^t, \Theta_{s|j}^{t+(m+1)/\mathcal{M}}, \hat{\Theta}_k^{t+m/\mathcal{M}})$ in \cref{eq:theo5_prox}, we have:

\begin{align}
    \mathcal{L}_k(z^t, &\Theta_{s|j}^{t+(m+1)/\mathcal{M}}, \hat{\Theta}_k^{t+m/\mathcal{M}}) = \mathcal{L}_k (z^t, \Theta_{s|i,k}^{t+m/\mathcal{M}}, \hat{\Theta}_k^{t+m/\mathcal{M}}) \label{eq:theo5_middle1}\\
    &+(\Theta_{s|j}^{t+(m+1)/\mathcal{M}} - \Theta_{s|i,k}^{t+m/\mathcal{M}}) \nabla_{\Theta_{s|i,k}^{t+m/\mathcal{M}}} \mathcal{L}_k (z^t, \Theta_{s|i,k}^{t+m/\mathcal{M}}, \hat{\Theta}_k^{t+m/\mathcal{M}}) + O(\eta^2)\\
    =& \mathcal{L}_k (z^t, \Theta_{s|i,k}^{t+m/\mathcal{M}}, \hat{\Theta}_k^{t+m/\mathcal{M}}) - \eta g_{s;j}^{t+m/\mathcal{M}}\cdot g_{s;k}^{t+m/\mathcal{M}} + O(\eta^2)
\end{align}

The subscript $s$ in gradients indicates that it represents the gradients of the shared parameters of the network. Conversely, we will use the subscript $ts$ for gradients of the task-specific network in the following derivation. Similarly, $\mathcal{L}_k (z^t, \Theta_{s|i,k}^{t+m/\mathcal{M}}, \hat{\Theta}_k^{t+m/\mathcal{M}})$ in \cref{eq:theo5_middle1} can also be further expanded using Taylor expansion as follows:
\begin{align}
    \mathcal{L}_k &(z^t, \Theta_{s|i,k}^{t+m/\mathcal{M}}, \hat{\Theta}_k^{t+m/\mathcal{M}}) = \mathcal{L}_k (z^t, \Theta_s^{t+(m-1)/\mathcal{M}}, \Theta_k^{t+(m-1)/\mathcal{M}}) \\
    &+(\Theta_{s|i,k}^{t+m/\mathcal{M}} - \Theta_s^{t+(m-1)/\mathcal{M}}) \nabla_{\Theta_s^{t+(m-1)/\mathcal{M}}} \mathcal{L}_k (z^t, \Theta_s^{t+(m-1)/\mathcal{M}}, \Theta_k^{t+(m-1)/\mathcal{M}})\\
    &+(\hat{\Theta}_k^{t+m/\mathcal{M}}-\Theta_k^{t+(m-1)/\mathcal{M}}) \nabla_{\Theta_k^{t+(m-1)/\mathcal{M}}} \mathcal{L}_k (z^t, \Theta_s^{t+(m-1)/\mathcal{M}}, \Theta_k^{t+(m-1)/\mathcal{M}}) + O(\eta^2)\\
    =& \mathcal{L}_k (z^t, \Theta_s^{t+(m-1)/\mathcal{M}}, \Theta_k^{t+(m-1)/\mathcal{M}}) - \eta (g_{s;i}^{t+(m-1)/\mathcal{M}} + g_{s;k}^{t+(m-1)/\mathcal{M}})\cdot g_{s;k}^{t+(m-1)/\mathcal{M}} \label{eq:theo5_middle2}\\
    &- \eta g_{ts;k}^{t+(m-1)/\mathcal{M}}\cdot g_{ts;k}^{t+(m-1)/\mathcal{M}} + O(\eta^2) \label{eq:theo5_middle3}
\end{align}

By substituting \cref{eq:theo5_middle1} with the results of \cref{eq:theo5_middle2} and \cref{eq:theo5_middle3}, we can obtain the following results:
\begin{align}
    \mathcal{L}_k(z^t, &\Theta_{s|j}^{t+(m+1)/\mathcal{M}}, \hat{\Theta}_k^{t+m/\mathcal{M}}) = \mathcal{L}_k (z^t, \Theta_s^{t+(m-1)/\mathcal{M}}, \Theta_k^{t+(m-1)/\mathcal{M}})\\
    &- \eta g_{s;j}^{t+m/\mathcal{M}}\cdot g_{s;k}^{t+m/\mathcal{M}} - \eta (g_{s;i}^{t+(m-1)/\mathcal{M}} + g_{s;k}^{t+(m-1)/\mathcal{M}})\cdot g_{s;k}^{t+(m-1)/\mathcal{M}} \\
    &- \eta g_{ts;k}^{t+(m-1)/\mathcal{M}}\cdot g_{ts;k}^{t+(m-1)/\mathcal{M}}+ O(\eta^2)
\end{align}

For the scenario where all tasks $\{i, j, k\}$ are jointly updated, the numerator of \cref{eq:theo5_joint} can also be expanded as follows:
\begin{align}
    \mathcal{L}_k(z^t,& \Theta_{s|i,j,k}^{t+m/\mathcal{M}}, \Theta_k^{t+m/\mathcal{M}}) = \mathcal{L}_k (z^t, \Theta_s^{t+(m-1)/\mathcal{M}}, \Theta_k^{t+(m-1)/\mathcal{M}})\\
    &+(\Theta_{s|i,j,k}^{t+m/\mathcal{M}} - \Theta_s^{t+(m-1)/\mathcal{M}}) \nabla_{\Theta_s^{t+(m-1)/\mathcal{M}}} \mathcal{L}_k (z^t, \Theta_s^{t+(m-1)/\mathcal{M}}, \Theta_k^{t+(m-1)/\mathcal{M}})\\
    &+(\Theta_k^{t+m/\mathcal{M}} - \Theta_k^{t+(m-1)/\mathcal{M}}) \nabla_{\Theta_k^{t+(m-1)/\mathcal{M}}} \mathcal{L}_k (z^t, \Theta_s^{t+(m-1)/\mathcal{M}}, \Theta_k^{t+(m-1)/\mathcal{M}}) + O(\eta^2)\\
    =& \mathcal{L}_k (z^t, \Theta_s^{t+(m-1)/\mathcal{M}}, \Theta_k^{t+(m-1)/\mathcal{M}}) \\
    &- \eta (g_{s,i}^{t+(m-1)/\mathcal{M}} + g_{s,j}^{t+(m-1)/\mathcal{M}} + g_{s,k}^{t+(m-1)/\mathcal{M}})\cdot g_{s,k}^{t+(m-1)/\mathcal{M}} \\
    &- \eta g_{ts,k}^{t+(m-1)/\mathcal{M}}\cdot g_{ts,k}^{t+(m-1)/\mathcal{M}} + O(\eta^2)
\end{align}

Finally, we can compare $\mathcal{B}_{i,j,k \rightarrow k}^{t+(m-1)/\mathcal{M}}$ with $\mathcal{B}_{i,k; j \rightarrow k}^{t+(m-1)/\mathcal{M}}$ by comparing the losses we obtained: $\mathcal{L}_k(z^t, \Theta_{s|j}^{t+(m+1)/\mathcal{M}}, \Theta_k^{t+(m-1)/\mathcal{M}})$ with $\mathcal{L}_k(z^t, \Theta_{s|i,j,k}^{t+m/\mathcal{M}}, \Theta_k^{t+(m-1)/\mathcal{M}})$. We assume a sufficiently small learning rate $\eta$ that allows us to ignore terms larger than order two with $\eta$.

\begin{align}
    \mathcal{L}_k(z^t, & \Theta_{s|j}^{t+(m+1)/\mathcal{M}}, \hat{\Theta}_k^{t+m/\mathcal{M}}) - \mathcal{L}_k(z^t, \Theta_{s|i,j,k}^{t+m/\mathcal{M}}, \Theta_k^{t+m/\mathcal{M}})\\
    =& - \eta g_{s,j}^{t+m/\mathcal{M}}\cdot g_{s,k}^{t+m/\mathcal{M}} - \eta (g_{s,i}^{t+(m-1)/\mathcal{M}} + g_{s,k}^{t+(m-1)/\mathcal{M}})\cdot g_{s,k}^{t+(m-1)/\mathcal{M}} \\
    & - \eta g_{ts,k}^{t+(m-1)/\mathcal{M}}\cdot g_{ts,k}^{t+(m-1)/\mathcal{M}}\\
    &+ \eta (g_{s,i}^{t+(m-1)/\mathcal{M}} + g_{s,j}^{t+(m-1)/\mathcal{M}} + g_{s,k}^{t+(m-1)/\mathcal{M}})\cdot g_{s,k}^{t+(m-1)/\mathcal{M}}\\
    & + \eta g_{ts,k}^{t+(m-1)/\mathcal{M}}\cdot g_{ts,k}^{t+(m-1)/\mathcal{M}}\\
    =& \eta(g_{s,j}^{t+(m-1)/\mathcal{M}} \cdot g_{s,k}^{t+(m-1)/\mathcal{M}} - g_{s,j}^{t+m/\mathcal{M}} \cdot g_{s,k}^{t+m/\mathcal{M}})\\
    \simeq& 0 \label{eq:theo5_last1}
\end{align}

The approximation in \cref{eq:theo5_last1} holds as we assume inter-task affinity change during a single time step from $t+(m-1)/\mathcal{M}$ to $t+m/\mathcal{M}$ is negligible.


(ii) In case we update task groups in reverse order the results would differ with (i). Similarly, we begin with the definition of proximal inter-task affinity with reverse order between $j \rightarrow k$ and $\{i, j\}\rightarrow k$ as follows:

\begin{align}
    \mathcal{B}_{j\rightarrow k}^{t+(m-1)/\mathcal{M}} &= 1-\frac{\mathcal{L}_k(z^t, \Theta_{s|j}^{t+m/\mathcal{M}}, \Theta_k^{t+m/\mathcal{M}})}{\mathcal{L}_k(z^t, \Theta_{s|i,k}^{t+(m-1)/\mathcal{M}}, \Theta_k^{t+(m-1)/\mathcal{M}})}\\
    &= 1-\frac{\mathcal{L}_k(z^t, \Theta_{s|j}^{t+m/\mathcal{M}}, \Theta_k^{t+(m-1)/\mathcal{M}})}{\mathcal{L}_k(z^t, \Theta_s^{t+(m-1)/\mathcal{M}}, \Theta_k^{t+(m-1)/\mathcal{M}})}
\end{align}

\begin{align}
    \mathcal{B}_{i,k \rightarrow k}^{t+m/\mathcal{M}} &= 1-\frac{\mathcal{L}_k(z^t, \Theta_{s|i,k}^{t+(m+1)/\mathcal{M}}, \hat{\Theta}_k^{t+(m+1)/\mathcal{M}})}{\mathcal{L}_k(z^t, \Theta_{s|j}^{t+m/\mathcal{M}}, \Theta_k^{t+m/\mathcal{M}})} \\
    &= 1-\frac{\mathcal{L}_k(z^t, \Theta_{s|i,k}^{t+(m+1)/\mathcal{M}}, \hat{\Theta}_k^{t+(m+1)/\mathcal{M}})}{\mathcal{L}_k(z^t, \Theta_{s|j}^{t+m/\mathcal{M}}, \Theta_k^{t+(m-1)/\mathcal{M}})}
\end{align}

The two-step proximal inter-task affinity with the sequence $\{j\}$ and $\{i, k\}$ can be represented as follows:
\begin{align}
    \mathcal{B}_{j; i,k \rightarrow k}^{t+(m-1)/\mathcal{M}} &= 1- (1-\mathcal{B}_{j\rightarrow k}^{t+(m-1)/\mathcal{M}}) \cdot (1-\mathcal{B}_{i,k \rightarrow k}^{t+m/\mathcal{M}})\\ 
    &= 1-\frac{\mathcal{L}_k(z^t, \Theta_{s|j}^{t+m/\mathcal{M}}, \Theta_k^{t+(m-1)/\mathcal{M}})}{\mathcal{L}_k(z^t, \Theta_s^{t+(m-1)/\mathcal{M}}, \Theta_k^{t+(m-1)/\mathcal{M}})} \cdot \frac{\mathcal{L}_k(z^t, \Theta_{s|i,k}^{t+(m+1)/\mathcal{M}}, \hat{\Theta}_k^{t+(m+1)/\mathcal{M}})}{\mathcal{L}_k(z^t, \Theta_{s|j}^{t+m/\mathcal{M}}, \Theta_k^{t+(m-1)/\mathcal{M}})}\\
    &= 1-\frac{\mathcal{L}_k(z^t, \Theta_{s|i,k}^{t+(m+1)/\mathcal{M}}, \hat{\Theta}_k^{t+(m+1)/\mathcal{M}})}{\mathcal{L}_k(z^t, \Theta_s^{t+(m-1)/\mathcal{M}}, \Theta_k^{t+(m-1)/\mathcal{M}})} \label{eq:theo5_prox2}
\end{align}

Our objective is to compare $\mathcal{B}_{i,j,k \rightarrow k}^{t+(m-1)/\mathcal{M}}$ from \cref{eq:theo5_joint} with $\mathcal{B}_{j; i,k \rightarrow k}^{t+(m-1)/\mathcal{M}}$ from \cref{eq:theo5_prox2} to assess each update's effect on the final loss. Since both equations share a common denominator, we only need to compare the numerators of each equation. Using the first-order Taylor approximation of $\mathcal{L}_k(z^t, \Theta_{s|i,k}^{t+(m+1)/\mathcal{M}}, \hat{\Theta}_k^{t+(m+1)/\mathcal{M}})$ in \cref{eq:theo5_prox2}, we have:

\begin{align}
    \mathcal{L}_k(z^t, &\Theta_{s|i,k}^{t+(m+1)/\mathcal{M}}, \hat{\Theta}_k^{t+(m+1)/\mathcal{M}}) = \mathcal{L}_k (z^t, \Theta_{s|j}^{t+m/\mathcal{M}}, \Theta_k^{t+m/\mathcal{M}}) \label{eq:theo5_middle1_}\\
    &+(\Theta_{s|i,k}^{t+(m+1)/\mathcal{M}} - \Theta_{s|j}^{t+m/\mathcal{M}}) \nabla_{\Theta_{s|j}^{t+m/\mathcal{M}}} \mathcal{L}_k (z^t, \Theta_{s|j}^{t+m/\mathcal{M}}, \Theta_k^{t+m/\mathcal{M}})\\
    &+(\hat{\Theta}_k^{t+(m+1)/\mathcal{M}} - \Theta_k^{t+m/\mathcal{M}}) \nabla_{\Theta_k^{t+m/\mathcal{M}}} \mathcal{L}_k (z^t, \Theta_{s|j}^{t+m/\mathcal{M}}, \Theta_k^{t+m/\mathcal{M}}) + O(\eta^2)\\
    =& \mathcal{L}_k (z^t, \Theta_{s|j}^{t+m/\mathcal{M}}, \Theta_k^{t+m/\mathcal{M}}) - \eta (g_{s;i}^{t+m/\mathcal{M}}+g_{s;k}^{t+m/\mathcal{M}})\cdot g_{s;k}^{t+m/\mathcal{M}} \\
    &- \eta g_{ts;k}^{t+m/\mathcal{M}}\cdot g_{ts;k}^{t+m/\mathcal{M}} + O(\eta^2)
\end{align}

Similarly, $\mathcal{L}_k (z^t, \Theta_{s|i,k}^{t+m/\mathcal{M}}, \Theta_k^{t+m/\mathcal{M}})$ in \cref{eq:theo5_middle1_} can also be further expanded using Taylor expansion as follows:
\begin{align}
    \mathcal{L}_k(z^t, &\Theta_{s|j}^{t+m/\mathcal{M}}, \Theta_k^{t+m/\mathcal{M}}) = \mathcal{L}_k(z^t, \Theta_{s|j}^{t+m/\mathcal{M}}, \Theta_k^{t+(m-1)/\mathcal{M}}) \\
    =& \mathcal{L}_k (z^t, \Theta_s^{t+(m-1)/\mathcal{M}}, \Theta_k^{t+(m-1)/\mathcal{M}})\\
    &+(\Theta_{s|j}^{t+m/\mathcal{M}} - \Theta_s^{t+(m-1)/\mathcal{M}}) \nabla_{\Theta_s^{t+(m-1)/\mathcal{M}}} \mathcal{L}_k (z^t, \Theta_{s|j}^{t+m/\mathcal{M}}, \Theta_k^{t+(m-1)/\mathcal{M}}) \\
    =& \mathcal{L}_k (z^t, \Theta_s^{t+(m-1)/\mathcal{M}}, \Theta_k^{t+(m-1)/\mathcal{M}}) \label{eq:theo5_middle2_}\\
    &- \eta g_{s;j}^{t+(m-1)/\mathcal{M}}\cdot g_{s;k}^{t+(m-1)/\mathcal{M}}+ O(\eta^2)
    \label{eq:theo5_middle3_}
\end{align}

By substituting \cref{eq:theo5_middle1_} with the results of \cref{eq:theo5_middle2_} and \cref{eq:theo5_middle3_}, we can obtain the following results:
\begin{align}
    \mathcal{L}_k(z^t, &\Theta_{s|i,k}^{t+(m+1)/\mathcal{M}}, \hat{\Theta}_k^{t+(m+1)/\mathcal{M}}) = \mathcal{L}_k (z^t, \Theta_s^{t+(m-1)/\mathcal{M}}, \Theta_k^{t+(m-1)/\mathcal{M}})\\
    &- \eta (g_{s;i}^{t+m/\mathcal{M}}+g_{s;k}^{t+m/\mathcal{M}})\cdot g_{s;k}^{t+m/\mathcal{M}} - \eta g_{ts;k}^{t+m/\mathcal{M}}\cdot g_{ts;k}^{t+m/\mathcal{M}} \\
    &- \eta g_{s;j}^{t+(m-1)/\mathcal{M}}\cdot g_{s;k}^{t+(m-1)/\mathcal{M}}+ O(\eta^2)
\end{align}

Finally, we can compare $\mathcal{B}_{i,j,k \rightarrow k}^{t+(m-1)/\mathcal{M}}$ with $\mathcal{B}_{j; i,k \rightarrow k}^{t+(m-1)/\mathcal{M}}$ by comparing the losses we obtained: $\mathcal{L}_k(z^t, \Theta_{s|j}^{t+(m+1)/\mathcal{M}}, \Theta_k^{t+(m-1)/\mathcal{M}})$ with $\mathcal{L}_k(z^t, \Theta_{s|i,j,k}^{t+m/\mathcal{M}}, \Theta_k^{t+(m-1)/\mathcal{M}})$. We assume a sufficiently small learning rate $\eta$ that allows us to ignore terms larger than order two with $\eta$.

\begin{align}
    \mathcal{L}_k(z^t, &\Theta_{s|i,k}^{t+(m+1)/\mathcal{M}}, \hat{\Theta}_k^{t+(m+1)/\mathcal{M}}) - \mathcal{L}_k(z^t, \Theta_{s|i,j,k}^{t+m/\mathcal{M}}, \Theta_k^{t+m/\mathcal{M}})\\
    =& - \eta (g_{s;i}^{t+m/\mathcal{M}}+g_{s;k}^{t+m/\mathcal{M}})\cdot g_{s;k}^{t+m/\mathcal{M}} - \eta g_{ts;k}^{t+m/\mathcal{M}}\cdot g_{ts;k}^{t+m/\mathcal{M}} \\
    &- \eta g_{s;j}^{t+(m-1)/\mathcal{M}}\cdot g_{s;k}^{t+(m-1)/\mathcal{M}}\\
    &+ \eta (g_{s,i}^{t+(m-1)/\mathcal{M}} + g_{s,j}^{t+(m-1)/\mathcal{M}} + g_{s,k}^{t+(m-1)/\mathcal{M}})\cdot g_{s,k}^{t+(m-1)/\mathcal{M}}\\
    &+ \eta g_{ts,k}^{t+(m-1)/\mathcal{M}}\cdot g_{ts,k}^{t+(m-1)/\mathcal{M}}\\
    =& - \eta (g_{s;i}^{t+m/\mathcal{M}}+g_{s;k}^{t+m/\mathcal{M}})\cdot g_{s;k}^{t+m/\mathcal{M}} + \eta (g_{s,i}^{t+(m-1)/\mathcal{M}} + g_{s,k}^{t+(m-1)/\mathcal{M}})\cdot g_{s,k}^{t+(m-1)/\mathcal{M}} \\
    & - \eta g_{ts;k}^{t+m/\mathcal{M}}\cdot g_{ts;k}^{t+m/\mathcal{M}} + \eta g_{ts,k}^{t+(m-1)/\mathcal{M}}\cdot g_{ts,k}^{t+(m-1)/\mathcal{M}} \\
    \simeq& - \eta g_{ts;k}^{t+m/\mathcal{M}}\cdot g_{ts;k}^{t+m/\mathcal{M}} + \eta g_{ts,k}^{t+(m-1)/\mathcal{M}}\cdot g_{ts,k}^{t+(m-1)/\mathcal{M}} \label{eq:theo5_last2}\\
    \leq& 0 \label{eq:theo5_final}
\end{align}

The approximation in \cref{eq:theo5_last2} holds under the assumption that the change in inter-task affinity during a single time step from $t+(m-1)/\mathcal{M}$ to $t+m/\mathcal{M}$ is negligible. Since we assume convex loss functions, the magnitude of task-specific gradients $g_{ts;k}$ would increase after updating the loss of $j$, which exhibits negative inter-task affinity with $k$ ($\mathcal{A}_{j \rightarrow k}<0$). Therefore, the inequality in \cref{eq:theo5_final} holds.
\end{proof}

This suggest that grouping tasks with proximal inter-task affinity and subsequently updating these groups sequentially result in lower multi-task loss compared to jointly backpropagating all tasks. This disparity arises because the network can discern superior task-specific parameters to accommodate task-specific information during sequential learning.


\clearpage
%%%%%%%%%%%%%%%%%%%%%%%%%%%%%%%%%%%%%%%%%%%%%%%%%%%%%%%%%%%%%%%%%%%%%%%%%%%%%%%%%%%%
\section{Additional Related Works}
\textbf{Task Grouping.} Early Multi-Task Learning research is founded on the belief that simultaneous learning of similar tasks within a multi-task framework can enhance overall performance. Kang et al. \cite{kang2011learning} identify tasks that contribute to improved multi-task performance through the clustering of related tasks with online stochastic gradient descent. This strategy challenges the prevailing assumption that all tasks are inherently interrelated. In parallel, Kumar et al. \cite{kumar2012learning} present a framework for MTL designed to enable selective sharing of information across tasks. They suggests that each task parameter vector can be expressed as a linear combination of a finite number of underlying basis tasks. However, these initial methodologies face limitations in their applicability and analysis, particularly in scaling to deep neural networks.
Finding out related tasks is more dynamically explored in the transfer learning domain \cite{achille2019task2vec, achille2021information}. They find related tasks by measuring task similarity which can be comparing the similarity of features extracted from the same depth of the independent task's network or directly measuring the transfer performance between tasks. Recent research has concentrated on identifying related tasks by directly assessing the relations among them within shared networks. This focus stems from the recognition that the measured inter-task relations in transfer learning fail to fully elucidate the dynamics within the MTL domain \cite{standley2020tasks, fifty2021efficiently}.

\textbf{Multi-Task Architectures.} Multi-task architectures can be classified based on how much the parameters or features are shared across tasks in the network. The most commonly used structure is a shared trunk which consists of a common encoder shared by multiple tasks and a dedicated decoder for each task \cite{RN51, RN52, RN49, RN50}. A tree-like architecture, featuring multiple division points for each task group, offers a more generalized structure \cite{treelike1, treelike2, treelike3, treelike4}. The cross-talk architecture employs separate symmetrical networks for each task, facilitating feature exchange between layers at the same depth for information sharing between tasks \cite{RN43, RN29}. The prediction distillation model \cite{RN9, RN29, RN32, pap} incorporates cross-task interactions at the end of the shared encoder, while the task switching network \cite{RN30, RN40, RN42, RN2} adjusts network parameters depending on the task.

\noindent\textbf{MTL in Vision Transformers.} Recent advancements in multi-task architecture have explored the integration of Vision Transformer \cite{vit, swin, pvt, focal, segformer, crossformer} into MTL. MTFormer \cite{mtformer} adopts a shared transformer encoder and decoder with a cross-task attention mechanism. MulT \cite{mult} leverages a shared attention mechanism to capture task dependencies, inspired by the Swin transformer. InvPT \cite{invpt} emphasizes global spatial position and multi-task context for dense prediction tasks through multi-scale feature aggregation. The Mixture of Experts (MoE) divides the model into predefined expert groups, dynamically shared or dedicated to specific tasks during the learning phase \cite{riquelme2021scaling, zhang2022mixture, fan2022m3vit, mustafa2022multimodal, chen2023mod, ye2023taskexpert}. Task prompter \cite{xu2023multi, xu2023demt, ye2022taskprompter} employs task-specific tokens to encapsulate task-specific information and utilizes cross-task interactions to enhance multi-task performance. 

\noindent\textbf{Multi-Task Domain Generalization.} Task grouping based on their relations has also been explored in the field of domain adaptation. In particular, \citep{wei2024task} proposes grouping heterogeneous tasks to regularize them, thereby promoting the learning of more generalized features across domain shifts. \citep{smith2021origin} explores generalization strategies at the mini-batch level. \citep{li2020sequential} addresses diverse domain shift scenarios by incorporating all possible sequential domain learning paths to generalize features for unseen domains. \citep{shi2021gradient} focuses on generalization to unseen domains by reducing dependence on specific domains through inter-domain gradient matching. Additionally, \citep{hu2022improving} analyzes the problem of spurious correlations in MTL and proposes regularization methods to mitigate this issue. The effect of gradient conflicts, which are considered the primary cause of negative transfer between tasks, is thoroughly examined in \citep{jiang2024forkmerge}. This work also proposes combining task distributions to identify better network parameters from a generalization perspective. The objectives of conventional multi-task optimization and domain generalization differ fundamentally. Conventional multi-task optimization typically assumes that the source and target domains share similar data distributions. In contrast, domain generalization focuses on scenarios involving significant domain shifts. This distinction leads to different approaches in leveraging task relations to achieve their respective goals. In multi-task optimization, simultaneously updating heterogeneous tasks with conflicting gradients results in suboptimal optimization. On the other hand, domain generalization leverages task sets as a tool to extract generalized features applicable to various unseen domains. Overfitting to similar tasks can harm performance on unseen domains, making it advantageous to use heterogeneous tasks as a form of regularization.



%%%%%%%%%%%%%%%%%%%%%%%%%%%%%%%%%%%%%%%%%%%%%%%%%%%%%%%%%%%%%%%%%%%%%%%%%%%%%%%%%%%%
\section{Experimental Details}
\label{append:experimental_details}
\setcounter{table}{0}
\setcounter{figure}{0}
We implement our experiments on top of publically available code from \cite{ye2022invpt}. We run our experiments on A6000 GPUs.

\textbf{Datasets.} We assess our method on multi-task datasets: NYUD-v2 \cite{RN15}, PASCAL-Context \cite{mottaghi2014role}, and Taskonomy \cite{zamir2018taskonomy}. These datasets encompass various vision tasks. NYUD-v2 comprises 4 vision tasks: depth estimation, semantic segmentation, surface normal prediction, and edge detection. Meanwhile, PASCAL-Context includes 5 tasks: semantic segmentation, human parts estimation, saliency estimation, surface normal prediction, and edge detection. In Taskonomy, we use 11 vision tasks: Depth Euclidean (DE), Depth Zbuffer (DZ), Edge Texture (ET),  Keypoints 2D (K2), Keypoints 3D (K3), Normal (N), Principal Curvature (C), Reshading (R), Segment Unsup2d (S2), and Segment Unsup2.5D (S2.5).

\textbf{Metrics. } To assess task performance, we employed widely used metrics across different tasks. For semantic segmentation, we utilized mean Intersection over Union (mIoU). The performance of surface normal prediction was gauged by computing the mean angle distances between the predicted output and ground truth. Depth estimation task performance was evaluated using Root Mean Squared Error (RMSE). For saliency estimation and human part segmentation, we utilized mean Intersection over Union (mIoU). Edge detection performance was assessed using optimal-dataset-scale-F-measure (odsF). For the Taskonomy benchmark, curvature was evaluated using RMSE, while the other tasks were evaluated using L1 distance, following the settings in \cite{chen2023mod}. 

To evaluate multi-task performance, we adopted the metric proposed in \cite{RN2}. This metric measures per-task performance by averaging it with respect to the single-task baseline $b$, as shown in the equation: $\triangle_m = (1/T)\sum_{i=1}^{T}(-1)^{l_i}(M_{m,i}-M_{b,i})/M_{b,i}$ where $l_i=1$ if a lower value of measure $M_i$ means better performance for task $i$, and 0 otherwise.


%-------------------------------------------------------------------------
\begin{table*}[h]
\caption{Hyperparameters for experiments.}
\centering
\renewcommand\arraystretch{1.20}
\begin{tabular}{lc}
\hline
Hyperparameter                  &  Value \\ \hline
Optimizer                       &  Adam \cite{kingma2014adam}\\
Scheduler                       &  Polynomial Decay\\
Minibatch size                  &  8\\
Number of iterations            &  40000\\
Backbone (Transformer)                        &  ViT \cite{vit} \\
\hspace{10pt}$\llcorner$ Learning rate                   &  0.00002\\
\hspace{10pt}$\llcorner$ Weight Decay                    &  0.000001\\
\hspace{10pt}$\llcorner$ Affinity decay factor $\beta$   &  0.001\\
\hline
\end{tabular}
\label{Implementation_details}
\end{table*}
%-------------------------------------------------------------------------


\textbf{Implementation Details.} For experiments, we adopt ViT \cite{vit} pre-trained on ImageNet-22K \cite{deng2009imagenet} as the multi-task encoder.
Task-specific decoders merge the multi-scale features extracted by the encoder to generate the outputs for each task. The models are trained for 40,000 iterations on both NYUD \cite{RN15} and PASCAL \cite{RN12} datasets with batch size 8. We used Adam optimizer with learning rate $2\times$$10^{-5}$ and $1\times$$10^{-6}$ of a weight decay with a polynomial learning rate schedule. The cross-entropy loss was used for semantic segmentation, human parts estimation, and saliency, edge detection. Surface normal prediction and depth estimation used L1 loss. The tasks are weighted equally to ensure a fair comparison. For the Taskonomy Benchmark \cite{zamir2018taskonomy}, we use the dataloader from the open-access code provided by \cite{chen2023mod}, while maintaining experimental settings identical to those used for NYUD-v2 and PASCAL-Context. We use the same experimental setup for the other hyperparameters as in previous works \cite{invpt, ye2022taskprompter}, as detailed in \Cref{Implementation_details}.


%%%%%%%%%%%%%%%%%%%%%%%%%%%%%%%%%%%%%%%%%%%%%%%%%%%%%%%%%%%%%%%%%%%%%%%%%%%%%%%%%%%%
% \clearpage
\section{Additional Experimental Results}
\label{append:additional_experimental_results}
\setcounter{table}{0}
\setcounter{figure}{0}

\textbf{Comparison of optimization results with different backbone sizes.} We compare the results of multi-task optimization on Taskonomy across various sizes of vision transformers, as shown in \Cref{tab:tab_exp_taskonomy_vitB,tab:tab_exp_taskonomy_vitS,tab:tab_exp_taskonomy_vitT}. Our method consistently achieves superior performance across all backbone sizes. Unlike previous approaches that focus on learning shared parameters, our optimization strategy enhances the learning of task-specific parameters. This leads to significant performance improvements, especially with smaller backbones, where competition between tasks is more intense due to the limited number of shared parameters. How tasks are grouped, as visualized in \cref{append:fig:vis_grouping}, depends on the backbone size.

\textbf{Visualization of Proximal Inter-Task Affinity.} In \Cref{fig:proximal_vit_taskonomy}, we present the tracked proximal inter-task affinity for each pair of tasks in Taskonomy. The changes in proximal inter-task affinity depend on the nature of the task pair, but as the backbone size increases, the affinity tends to become more positive. This trend is more noticeable in NYUD-v2 and PASCAL-Context, where there are fewer tasks, as shown in \Cref{fig:proximal_vit_nyud,fig:proximal_vit_pascal}. This pattern also aligns with the number of clustered tasks in \Cref{fig:num_group}, where the number of groups increases as the backbone size decreases.

\textbf{Effect of the Decay Rate with Visualization.} In \Cref{fig:proximal_vit_beta}, we visualize the proximal inter-task affinity tracked during optimization with various decay rates $\beta$, ranging from 0.01 to 1e-5 on a logarithmic scale. The decay rate $\beta$ helps stabilize the tracking of proximal inter-task affinity as it fluctuates during optimization. Additionally, it aids in understanding inter-task relations over time, independent of input data. For vision transformers, a decay rate of $\beta=0.001$ demonstrates stable tracking. In real-world applications, multi-task performance is not highly sensitive to the decay rate $\beta$. In \Cref{tab:tab_exp_beta_perf}, we evaluate how $\beta$ impacts multi-task performance on the Taskonomy benchmark. The results demonstrate that the proposed optimization method consistently improves performance across various $\beta$ values, minimizing the need for extensive hyperparameter tuning in practical scenarios.

\noindent\textbf{The Influence of Task Grouping Strategy.}  
In \Cref{tab:tab_exp_grouping_strategy}, we present results comparing different task grouping strategies. These include randomly grouping tasks with a predefined number, grouping heterogeneous tasks, and grouping homogeneous tasks (our approach). The results clearly demonstrate that grouping homogeneous task sets yields superior performance under the proposed settings. This contrasts with the multi-task domain generalization approach, which groups heterogeneous tasks as a form of regularization to enhance generalization to unseen domains. 
This difference arises from the fundamentally distinct objectives of conventional multi-task optimization and domain generalization. Conventional multi-task optimization typically assumes that the source and target domains share similar data distributions, while domain generalization addresses scenarios with significant domain shifts. Consequently, the approaches to leveraging task relations differ to meet these distinct goals. As demonstrated in Theorems 1 and 2 of our work, in multi-task optimization, simultaneously updating heterogeneous tasks with low task affinity leads to suboptimal optimization and higher losses compared to updating similar task sets with high task affinity. This observation aligns with findings from previous multi-task optimization studies referenced in the related works section.

\noindent\textbf{Influence of Batch Size.}  
In \Cref{tab:tab_exp_batch}, we compare our method with single-gradient descent (GD) to evaluate its robustness in improving multi-task performance across varying batch sizes. The proposed optimization method consistently demonstrates performance improvements (\(\triangle_m\) (\% \(\uparrow\))) of 5.27\%, 5.71\%, and 6.13\% across different batch sizes. These results highlight the robustness and adaptability of the proposed algorithm across diverse scenarios.



%%%%%%%%%%%%%%%%%%%%%%%%%%%%%%%%
\begin{figure}[h]
    \vspace{-10pt}
    \centering
    \begin{subfigure}{0.24\textwidth}
        \includegraphics[width=0.99\textwidth]{figure/group_viz_3.png}
        \vspace*{-15pt}
        \caption{$\{G\}_{i=1}^{\mathcal{M}}$ with ViT-L}
    \end{subfigure}
    \begin{subfigure}{0.24\textwidth}
        \includegraphics[width=0.99\textwidth]{figure/group_viz_2.png}
        \vspace*{-15pt}
        \caption{$\{G\}_{i=1}^{\mathcal{M}}$ with ViT-B}
    \end{subfigure}
    \begin{subfigure}{0.24\textwidth}
        \includegraphics[width=0.99\textwidth]{figure/group_viz_1.png}
        \vspace*{-15pt}
        \caption{$\{G\}_{i=1}^{\mathcal{M}}$ with ViT-S}
    \end{subfigure}
    \begin{subfigure}{0.24\textwidth}
        \includegraphics[width=0.99\textwidth]{figure/group_viz_0.png}
        \vspace*{-15pt}
        \caption{$\{G\}_{i=1}^{\mathcal{M}}$ with ViT-T}
    \end{subfigure}
    \caption{The averaged grouping results $\{G\}_{i=1}^{\mathcal{M}}$ on the Taskonomy benchmark are shown for (a) ViT-L, (b) ViT-B, (c) ViT-S, and (d) ViT-T, respectively.}
    \label{append:fig:vis_grouping}
\end{figure}
%%%%%%%%%%%%%%%%%%%%%%%%%%%%%%%%

% \clearpage
%%%%%%%%%%%%%%%%%%%%%%%%%%%%%%%%
\begin{table*}[t]
\caption{Comparison with previous multi-task optimization approaches on Taskonomy with ViT-B.}
\vspace{-5pt}
\centering
\renewcommand\arraystretch{1.00}
\resizebox{0.99\textwidth}{!}{
\begin{tabular}{l|ccccccccccc|c}
\midrule[1.0pt]
 & DE & DZ & EO & ET & K2  & K3 & N   & C & R & S2  & S2.5 &  \\ \cmidrule[0.5pt]{2-12}
\multirow{-2}{*}{Task} & L1 Dist. $\downarrow$  & L1 Dist. $\downarrow$ & L1 Dist. $\downarrow$ & L1 Dist. $\downarrow$ & L1 Dist. $\downarrow$ & L1 Dist. $\downarrow$ & L1 Dist. $\downarrow$ & RMSE $\downarrow$    & L1 Dist. $\downarrow$ & L1 Dist. $\downarrow$ & L1 Dist. $\downarrow$  & \multirow{-2}{*}{$\triangle_m$ ($\uparrow$)} \\ \midrule[1.0pt]

Single Task     &0.0183&0.0186&0.1089&0.1713&0.1630&0.0863&0.2953&0.7522&0.1504&0.1738&0.1530&-    \\ \midrule[0.5pt]
GD              &0.0188&0.0197&0.1283&0.1745&0.1718&0.0933&0.2599&0.7911&0.1799&0.1885&0.1631&-6.35     \\
GradDrop        &0.0195&0.0206&0.1318&0.1748&0.1735&0.0945&0.3018&0.8060&0.1866&0.1920&0.1607&-9.54     \\
MGDA            &-&-&-&-&-&-&-&-&-&-&-&-     \\
UW              &0.0188&0.0198&0.1285&0.1745&0.1719&0.0933&0.2535&0.7915&0.1800&0.1883&0.1629&-6.19     \\
DTP             &0.0187&0.0198&0.1283&0.1745&0.1720&0.0933&0.2558&0.7912&0.1804&0.1884&0.1634&-6.25     \\
DWA             &0.0188&0.0197&0.1287&0.1745&0.1719&0.0933&0.2570&0.7927&0.1806&0.1887&0.1632&-6.33     \\
PCGrad          &0.0185&0.0188&0.1285&0.1738&0.1703&0.0928&0.2557&0.7964&0.1810&0.1882&0.1569&-5.22     \\
CAGrad          &0.0192&0.0196&0.1306&0.1733&0.1654&0.0939&0.2871&0.8147&0.1901&0.1906&0.1659&-8.34    \\
IMTL            &0.0189&0.0200&0.1287&0.1745&0.1720&0.0934&0.2618&0.7928&0.1811&0.1888&0.1635&-6.75     \\
Aligned-MTL     &0.0191&0.0202&0.1263&0.1729&0.1663&0.0944&0.3061&0.8560&0.1936&0.1872&0.1585&-8.93     \\
Nash-MTL        &0.0175&0.0182&0.1208&0.1730&0.1663&0.0901&0.2686&0.7958&0.1707&0.1839&0.1577&-2.79     \\
FAMO            &0.0189&0.0200&0.1285&0.1745&0.1720&0.0934&0.2715&0.7929&0.1807&0.1891&0.1640&-7.21     \\
Ours            &0.0167&0.0169&0.1228&0.1739&0.1695&0.0910&0.2344&0.7600&0.1691&0.1836&0.1571&-0.64     \\  \midrule[1.0pt]
\end{tabular}}
\label{tab:tab_exp_taskonomy_vitB}
\end{table*}
%%%%%%%%%%%%%%%%%%%%%%%%%%%%%%%%
\begin{table*}[t]
\caption{Comparison with previous multi-task optimization approaches on Taskonomy with ViT-S.}
\vspace{-5pt}
\centering
\renewcommand\arraystretch{1.00}
\resizebox{0.99\textwidth}{!}{
\begin{tabular}{l|ccccccccccc|c}
\midrule[1.0pt]
 & DE & DZ & EO & ET & K2  & K3 & N   & C & R & S2  & S2.5 &  \\ \cmidrule[0.5pt]{2-12}
\multirow{-2}{*}{Task} & L1 Dist. $\downarrow$  & L1 Dist. $\downarrow$ & L1 Dist. $\downarrow$ & L1 Dist. $\downarrow$ & L1 Dist. $\downarrow$ & L1 Dist. $\downarrow$ & L1 Dist. $\downarrow$ & RMSE $\downarrow$    & L1 Dist. $\downarrow$ & L1 Dist. $\downarrow$ & L1 Dist. $\downarrow$  & \multirow{-2}{*}{$\triangle_m$ ($\uparrow$)} \\ \midrule[1.0pt]
Single Task     &0.0264&0.0259&0.1348&0.1740&0.1667&0.0973&0.3481&0.8598&0.1905&0.1857&0.1691&-    \\ \midrule[0.5pt]
GD              &0.0264&0.0272&0.1574&0.1775&0.1838&0.1038&0.4370&0.9237&0.2475&0.2076&0.1858&-11.39     \\
GradDrop        &0.0274&0.0280&0.1609&0.1779&0.1856&0.1042&0.4472&0.9366&0.2549&0.2106&0.1821&-13.14     \\
MGDA            &-&-&-&-&-&-&-&-&-&-&-&-     \\
UW              &0.0263&0.0269&0.1570&0.1775&0.1832&0.1037&0.4362&0.9202&0.2465&0.2075&0.1856&-11.11     \\
DTP             &0.0262&0.0273&0.1568&0.1778&0.1831&0.1037&0.4884&0.9207&0.2466&0.2073&0.1849&-12.52     \\
DWA             &0.0264&0.0271&0.1572&0.1776&0.1834&0.1038&0.4336&0.9215&0.2469&0.2075&0.1856&-11.20     \\
PCGrad          &0.0271&0.0274&0.1570&0.1766&0.1784&0.1034&0.4522&0.9343&0.2525&0.2071&0.1811&-11.78     \\
CAGrad          &0.0289&0.0282&0.1611&0.1769&0.1706&0.1062&0.4723&0.9557&0.2689&0.2122&0.1902&-15.09     \\
IMTL-L          &0.0255&0.0258&0.1510&0.1744&0.1716&0.1005&0.4339&0.9459&0.2466&0.2036&0.1825&-8.90     \\
Aligned-MTL     &0.0286&0.0290&0.1603&0.1744&0.1711&0.1033&0.4596&1.0022&0.2783&0.2090&0.1854&-15.06     \\
Nash-MTL        &0.0255&0.0258&0.1510&0.1744&0.1716&0.1005&0.4339&0.9459&0.2466&0.2036&0.1825&-8.79     \\
FAMO            &0.0263&0.0272&0.1573&0.1774&0.1835&0.1035&0.4326&0.9208&0.2464&0.2077&0.1858&-11.23     \\
Ours            &0.0225&0.0229&0.1444&0.1762&0.1775&0.0995&0.3983&0.8620&0.2156&0.1997&0.1774&-2.83     \\  \midrule[1.0pt]
\end{tabular}}
\label{tab:tab_exp_taskonomy_vitS}
\end{table*}
%%%%%%%%%%%%%%%%%%%%%%%%%%%%%%%%
\begin{table*}[t]
\caption{Comparison with previous multi-task optimization approaches on Taskonomy with ViT-T.}
\vspace{-5pt}
\centering
\renewcommand\arraystretch{1.00}
\resizebox{0.99\textwidth}{!}{
\begin{tabular}{l|ccccccccccc|c}
\midrule[1.0pt]
 & DE & DZ & EO & ET & K2  & K3 & N   & C & R & S2  & S2.5 &  \\ \cmidrule[0.5pt]{2-12}
\multirow{-2}{*}{Task} & L1 Dist. $\downarrow$  & L1 Dist. $\downarrow$ & L1 Dist. $\downarrow$ & L1 Dist. $\downarrow$ & L1 Dist. $\downarrow$ & L1 Dist. $\downarrow$ & L1 Dist. $\downarrow$ & RMSE $\downarrow$    & L1 Dist. $\downarrow$ & L1 Dist. $\downarrow$ & L1 Dist. $\downarrow$  & \multirow{-2}{*}{$\triangle_m$ ($\uparrow$)} \\ \midrule[1.0pt]
Single Task     &0.0289&0.0290&0.1405&0.1774&0.1682&0.0970&0.3837&0.8968&0.2096&0.1904&0.1729&-    \\ \midrule[0.5pt]
GD              &0.0279&0.0285&0.1604&0.1789&0.1860&0.1043&0.4704&0.9488&0.2613&0.2086&0.1914&-9.21     \\
GradDrop        &0.0287&0.0292&0.1630&0.1795&0.1868&0.1052&0.4795&0.9621&0.2697&0.2118&0.1878&-10.68     \\
MGDA            &-&-&-&-&-&-&-&-&-&-&-&-     \\
UW              &0.0279&0.0285&0.1604&0.1789&0.1859&0.1043&0.4699&0.9488&0.2613&0.2085&0.1914&-9.21     \\
DTP             &0.0278&0.0288&0.1603&0.1790&0.1859&0.1042&0.4697&0.9488&0.2614&0.2088&0.1915&-9.27     \\
DWA             &0.0279&0.0285&0.1604&0.1789&0.1859&0.1043&0.4693&0.9489&0.2613&0.2086&0.1913&-9.20     \\
PCGrad          &0.0283&0.0290&0.1604&0.1769&0.1803&0.1036&0.4720&0.9645&0.2683&0.2090&0.1866&-9.28     \\
CAGrad          &0.0300&0.0304&0.1644&0.1743&0.1721&0.1055&0.4838&0.9818&0.2847&0.2143&0.1974&-12.12     \\
IMTL            &0.0276&0.0282&0.1553&0.1754&0.1743&0.1018&0.4621&0.9809&0.2623&0.2051&0.1878&-7.55     \\
Aligned-MTL     &0.0296&0.0318&0.1633&0.1765&0.1757&0.1150&0.4806&1.0270&0.2935&0.2109&0.1887&-13.70     \\
Nash-MTL        &0.0276&0.0282&0.1553&0.1754&0.1743&0.1018&0.4621&0.9809&0.2623&0.2051&0.1878&-7.46    \\
FAMO            &0.0279&0.0285&0.1604&0.1789&0.1859&0.1043&0.4718&0.9488&0.2612&0.2085&0.1913&-9.33     \\
Ours            &0.0252&0.0257&0.1526&0.1774&0.1827&0.1019&0.4337&0.9100&0.2402&0.2026&0.1845&-3.67     \\  \midrule[1.0pt]
\end{tabular}}
\label{tab:tab_exp_taskonomy_vitT}
\end{table*}
%%%%%%%%%%%%%%%%%%%%%%%%%%%%%%%%


% %-------------------------------------------------------------------------
\def\figlength{0.14}
\begin{figure}[h]
\centering  
\begin{subfigure}{\figlength\textwidth}
\includegraphics[width=0.99\textwidth]{figure/vit_taskonomy/DE_to_DZ.png}
\end{subfigure}
\begin{subfigure}{\figlength\textwidth}
\includegraphics[width=0.99\textwidth]{figure/vit_taskonomy/DE_to_EO.png}
\end{subfigure}
\begin{subfigure}{\figlength\textwidth}
\includegraphics[width=0.99\textwidth]{figure/vit_taskonomy/DE_to_ET.png}
\end{subfigure}
\begin{subfigure}{\figlength\textwidth}
\includegraphics[width=0.99\textwidth]{figure/vit_taskonomy/DE_to_K2.png}
\end{subfigure}
\begin{subfigure}{\figlength\textwidth}
\includegraphics[width=0.99\textwidth]{figure/vit_taskonomy/DE_to_K3.png}
\end{subfigure}
\begin{subfigure}{\figlength\textwidth}
\includegraphics[width=0.99\textwidth]{figure/vit_taskonomy/DE_to_N.png}
\end{subfigure}
\begin{subfigure}{\figlength\textwidth}
\includegraphics[width=0.99\textwidth]{figure/vit_taskonomy/DE_to_C.png}
\end{subfigure}
\begin{subfigure}{\figlength\textwidth}
\includegraphics[width=0.99\textwidth]{figure/vit_taskonomy/DE_to_R.png}
\end{subfigure}
\begin{subfigure}{\figlength\textwidth}
\includegraphics[width=0.99\textwidth]{figure/vit_taskonomy/DE_to_S2.png}
\end{subfigure}
\begin{subfigure}{\figlength\textwidth}
\includegraphics[width=0.99\textwidth]{figure/vit_taskonomy/DE_to_S2.5.png}
\end{subfigure}
\begin{subfigure}{\figlength\textwidth}
\includegraphics[width=0.99\textwidth]{figure/vit_taskonomy/DZ_to_DE.png}
\end{subfigure}
\begin{subfigure}{\figlength\textwidth}
\includegraphics[width=0.99\textwidth]{figure/vit_taskonomy/DZ_to_EO.png}
\end{subfigure}
\begin{subfigure}{\figlength\textwidth}
\includegraphics[width=0.99\textwidth]{figure/vit_taskonomy/DZ_to_ET.png}
\end{subfigure}
\begin{subfigure}{\figlength\textwidth}
\includegraphics[width=0.99\textwidth]{figure/vit_taskonomy/DZ_to_K2.png}
\end{subfigure}
\begin{subfigure}{\figlength\textwidth}
\includegraphics[width=0.99\textwidth]{figure/vit_taskonomy/DZ_to_K3.png}
\end{subfigure}
\begin{subfigure}{\figlength\textwidth}
\includegraphics[width=0.99\textwidth]{figure/vit_taskonomy/DZ_to_N.png}
\end{subfigure}
\begin{subfigure}{\figlength\textwidth}
\includegraphics[width=0.99\textwidth]{figure/vit_taskonomy/DZ_to_C.png}
\end{subfigure}
\begin{subfigure}{\figlength\textwidth}
\includegraphics[width=0.99\textwidth]{figure/vit_taskonomy/DZ_to_R.png}
\end{subfigure}
\begin{subfigure}{\figlength\textwidth}
\includegraphics[width=0.99\textwidth]{figure/vit_taskonomy/DZ_to_S2.png}
\end{subfigure}
\begin{subfigure}{\figlength\textwidth}
\includegraphics[width=0.99\textwidth]{figure/vit_taskonomy/DZ_to_S2.5.png}
\end{subfigure}
\begin{subfigure}{\figlength\textwidth}
\includegraphics[width=0.99\textwidth]{figure/vit_taskonomy/EO_to_DE.png}
\end{subfigure}
\begin{subfigure}{\figlength\textwidth}
\includegraphics[width=0.99\textwidth]{figure/vit_taskonomy/EO_to_DZ.png}
\end{subfigure}
\begin{subfigure}{\figlength\textwidth}
\includegraphics[width=0.99\textwidth]{figure/vit_taskonomy/EO_to_ET.png}
\end{subfigure}
\begin{subfigure}{\figlength\textwidth}
\includegraphics[width=0.99\textwidth]{figure/vit_taskonomy/EO_to_K2.png}
\end{subfigure}
\begin{subfigure}{\figlength\textwidth}
\includegraphics[width=0.99\textwidth]{figure/vit_taskonomy/EO_to_K3.png}
\end{subfigure}
\begin{subfigure}{\figlength\textwidth}
\includegraphics[width=0.99\textwidth]{figure/vit_taskonomy/EO_to_N.png}
\end{subfigure}
\begin{subfigure}{\figlength\textwidth}
\includegraphics[width=0.99\textwidth]{figure/vit_taskonomy/EO_to_C.png}
\end{subfigure}
\begin{subfigure}{\figlength\textwidth}
\includegraphics[width=0.99\textwidth]{figure/vit_taskonomy/EO_to_R.png}
\end{subfigure}
\begin{subfigure}{\figlength\textwidth}
\includegraphics[width=0.99\textwidth]{figure/vit_taskonomy/EO_to_S2.png}
\end{subfigure}
\begin{subfigure}{\figlength\textwidth}
\includegraphics[width=0.99\textwidth]{figure/vit_taskonomy/EO_to_S2.5.png}
\end{subfigure}
\begin{subfigure}{\figlength\textwidth}
\includegraphics[width=0.99\textwidth]{figure/vit_taskonomy/ET_to_DE.png}
\end{subfigure}
\begin{subfigure}{\figlength\textwidth}
\includegraphics[width=0.99\textwidth]{figure/vit_taskonomy/ET_to_DZ.png}
\end{subfigure}
\begin{subfigure}{\figlength\textwidth}
\includegraphics[width=0.99\textwidth]{figure/vit_taskonomy/ET_to_EO.png}
\end{subfigure}
\begin{subfigure}{\figlength\textwidth}
\includegraphics[width=0.99\textwidth]{figure/vit_taskonomy/ET_to_K2.png}
\end{subfigure}
\begin{subfigure}{\figlength\textwidth}
\includegraphics[width=0.99\textwidth]{figure/vit_taskonomy/ET_to_K3.png}
\end{subfigure}
\begin{subfigure}{\figlength\textwidth}
\includegraphics[width=0.99\textwidth]{figure/vit_taskonomy/ET_to_N.png}
\end{subfigure}
\begin{subfigure}{\figlength\textwidth}
\includegraphics[width=0.99\textwidth]{figure/vit_taskonomy/ET_to_C.png}
\end{subfigure}
\begin{subfigure}{\figlength\textwidth}
\includegraphics[width=0.99\textwidth]{figure/vit_taskonomy/ET_to_R.png}
\end{subfigure}
\begin{subfigure}{\figlength\textwidth}
\includegraphics[width=0.99\textwidth]{figure/vit_taskonomy/ET_to_S2.png}
\end{subfigure}
\begin{subfigure}{\figlength\textwidth}
\includegraphics[width=0.99\textwidth]{figure/vit_taskonomy/ET_to_S2.5.png}
\end{subfigure}
\begin{subfigure}{\figlength\textwidth}
\includegraphics[width=0.99\textwidth]{figure/vit_taskonomy/K2_to_DE.png}
\end{subfigure}
\begin{subfigure}{\figlength\textwidth}
\includegraphics[width=0.99\textwidth]{figure/vit_taskonomy/K2_to_DZ.png}
\end{subfigure}
\begin{subfigure}{\figlength\textwidth}
\includegraphics[width=0.99\textwidth]{figure/vit_taskonomy/K2_to_EO.png}
\end{subfigure}
\begin{subfigure}{\figlength\textwidth}
\includegraphics[width=0.99\textwidth]{figure/vit_taskonomy/K2_to_ET.png}
\end{subfigure}
\begin{subfigure}{\figlength\textwidth}
\includegraphics[width=0.99\textwidth]{figure/vit_taskonomy/K2_to_K3.png}
\end{subfigure}
\begin{subfigure}{\figlength\textwidth}
\includegraphics[width=0.99\textwidth]{figure/vit_taskonomy/K2_to_N.png}
\end{subfigure}
\begin{subfigure}{\figlength\textwidth}
\includegraphics[width=0.99\textwidth]{figure/vit_taskonomy/K2_to_C.png}
\end{subfigure}
\begin{subfigure}{\figlength\textwidth}
\includegraphics[width=0.99\textwidth]{figure/vit_taskonomy/K2_to_R.png}
\end{subfigure}
\begin{subfigure}{\figlength\textwidth}
\includegraphics[width=0.99\textwidth]{figure/vit_taskonomy/K2_to_S2.png}
\end{subfigure}
\begin{subfigure}{\figlength\textwidth}
\includegraphics[width=0.99\textwidth]{figure/vit_taskonomy/K2_to_S2.5.png}
\end{subfigure}
\begin{subfigure}{\figlength\textwidth}
\includegraphics[width=0.99\textwidth]{figure/vit_taskonomy/K3_to_DE.png}
\end{subfigure}
\begin{subfigure}{\figlength\textwidth}
\includegraphics[width=0.99\textwidth]{figure/vit_taskonomy/K3_to_DZ.png}
\end{subfigure}
\begin{subfigure}{\figlength\textwidth}
\includegraphics[width=0.99\textwidth]{figure/vit_taskonomy/K3_to_EO.png}
\end{subfigure}
\begin{subfigure}{\figlength\textwidth}
\includegraphics[width=0.99\textwidth]{figure/vit_taskonomy/K3_to_ET.png}
\end{subfigure}
\begin{subfigure}{\figlength\textwidth}
\includegraphics[width=0.99\textwidth]{figure/vit_taskonomy/K3_to_K2.png}
\end{subfigure}
\begin{subfigure}{\figlength\textwidth}
\includegraphics[width=0.99\textwidth]{figure/vit_taskonomy/K3_to_N.png}
\end{subfigure}
\begin{subfigure}{\figlength\textwidth}
\includegraphics[width=0.99\textwidth]{figure/vit_taskonomy/K3_to_C.png}
\end{subfigure}
\begin{subfigure}{\figlength\textwidth}
\includegraphics[width=0.99\textwidth]{figure/vit_taskonomy/K3_to_R.png}
\end{subfigure}
\begin{subfigure}{\figlength\textwidth}
\includegraphics[width=0.99\textwidth]{figure/vit_taskonomy/K3_to_S2.png}
\end{subfigure}
\begin{subfigure}{\figlength\textwidth}
\includegraphics[width=0.99\textwidth]{figure/vit_taskonomy/K3_to_S2.5.png}
\end{subfigure}
\begin{subfigure}{\figlength\textwidth}
\includegraphics[width=0.99\textwidth]{figure/vit_taskonomy/N_to_DE.png}
\end{subfigure}
\begin{subfigure}{\figlength\textwidth}
\includegraphics[width=0.99\textwidth]{figure/vit_taskonomy/N_to_DZ.png}
\end{subfigure}
\begin{subfigure}{\figlength\textwidth}
\includegraphics[width=0.99\textwidth]{figure/vit_taskonomy/N_to_EO.png}
\end{subfigure}
\begin{subfigure}{\figlength\textwidth}
\includegraphics[width=0.99\textwidth]{figure/vit_taskonomy/N_to_ET.png}
\end{subfigure}
\begin{subfigure}{\figlength\textwidth}
\includegraphics[width=0.99\textwidth]{figure/vit_taskonomy/N_to_K2.png}
\end{subfigure}
\begin{subfigure}{\figlength\textwidth}
\includegraphics[width=0.99\textwidth]{figure/vit_taskonomy/N_to_K3.png}
\end{subfigure}
\caption{Changes in proximal inter-task affinity during the optimization process of ViT-L with Taskonomy benchmark.}
\end{figure}

\begin{figure}[h]\ContinuedFloat
\centering
\begin{subfigure}{\figlength\textwidth}
\includegraphics[width=0.99\textwidth]{figure/vit_taskonomy/N_to_C.png}
\end{subfigure}
\begin{subfigure}{\figlength\textwidth}
\includegraphics[width=0.99\textwidth]{figure/vit_taskonomy/N_to_R.png}
\end{subfigure}
\begin{subfigure}{\figlength\textwidth}
\includegraphics[width=0.99\textwidth]{figure/vit_taskonomy/N_to_S2.png}
\end{subfigure}
\begin{subfigure}{\figlength\textwidth}
\includegraphics[width=0.99\textwidth]{figure/vit_taskonomy/N_to_S2.5.png}
\end{subfigure}
\begin{subfigure}{\figlength\textwidth}
\includegraphics[width=0.99\textwidth]{figure/vit_taskonomy/C_to_DE.png}
\end{subfigure}
\begin{subfigure}{\figlength\textwidth}
\includegraphics[width=0.99\textwidth]{figure/vit_taskonomy/C_to_DZ.png}
\end{subfigure}
\begin{subfigure}{\figlength\textwidth}
\includegraphics[width=0.99\textwidth]{figure/vit_taskonomy/C_to_EO.png}
\end{subfigure}
\begin{subfigure}{\figlength\textwidth}
\includegraphics[width=0.99\textwidth]{figure/vit_taskonomy/C_to_ET.png}
\end{subfigure}
\begin{subfigure}{\figlength\textwidth}
\includegraphics[width=0.99\textwidth]{figure/vit_taskonomy/C_to_K2.png}
\end{subfigure}
\begin{subfigure}{\figlength\textwidth}
\includegraphics[width=0.99\textwidth]{figure/vit_taskonomy/C_to_K3.png}
\end{subfigure}
\begin{subfigure}{\figlength\textwidth}
\includegraphics[width=0.99\textwidth]{figure/vit_taskonomy/C_to_N.png}
\end{subfigure}
\begin{subfigure}{\figlength\textwidth}
\includegraphics[width=0.99\textwidth]{figure/vit_taskonomy/C_to_R.png}
\end{subfigure}
\begin{subfigure}{\figlength\textwidth}
\includegraphics[width=0.99\textwidth]{figure/vit_taskonomy/C_to_S2.png}
\end{subfigure}
\begin{subfigure}{\figlength\textwidth}
\includegraphics[width=0.99\textwidth]{figure/vit_taskonomy/C_to_S2.5.png}
\end{subfigure}
\begin{subfigure}{\figlength\textwidth}
\includegraphics[width=0.99\textwidth]{figure/vit_taskonomy/R_to_DE.png}
\end{subfigure}
\begin{subfigure}{\figlength\textwidth}
\includegraphics[width=0.99\textwidth]{figure/vit_taskonomy/R_to_DZ.png}
\end{subfigure}
\begin{subfigure}{\figlength\textwidth}
\includegraphics[width=0.99\textwidth]{figure/vit_taskonomy/R_to_EO.png}
\end{subfigure}
\begin{subfigure}{\figlength\textwidth}
\includegraphics[width=0.99\textwidth]{figure/vit_taskonomy/R_to_ET.png}
\end{subfigure}
\begin{subfigure}{\figlength\textwidth}
\includegraphics[width=0.99\textwidth]{figure/vit_taskonomy/R_to_K2.png}
\end{subfigure}
\begin{subfigure}{\figlength\textwidth}
\includegraphics[width=0.99\textwidth]{figure/vit_taskonomy/R_to_K3.png}
\end{subfigure}
\begin{subfigure}{\figlength\textwidth}
\includegraphics[width=0.99\textwidth]{figure/vit_taskonomy/R_to_N.png}
\end{subfigure}
\begin{subfigure}{\figlength\textwidth}
\includegraphics[width=0.99\textwidth]{figure/vit_taskonomy/R_to_C.png}
\end{subfigure}
\begin{subfigure}{\figlength\textwidth}
\includegraphics[width=0.99\textwidth]{figure/vit_taskonomy/R_to_S2.png}
\end{subfigure}
\begin{subfigure}{\figlength\textwidth}
\includegraphics[width=0.99\textwidth]{figure/vit_taskonomy/R_to_S2.5.png}
\end{subfigure}
\begin{subfigure}{\figlength\textwidth}
\includegraphics[width=0.99\textwidth]{figure/vit_taskonomy/S2_to_DE.png}
\end{subfigure}
\begin{subfigure}{\figlength\textwidth}
\includegraphics[width=0.99\textwidth]{figure/vit_taskonomy/S2_to_DZ.png}
\end{subfigure}
\begin{subfigure}{\figlength\textwidth}
\includegraphics[width=0.99\textwidth]{figure/vit_taskonomy/S2_to_EO.png}
\end{subfigure}
\begin{subfigure}{\figlength\textwidth}
\includegraphics[width=0.99\textwidth]{figure/vit_taskonomy/S2_to_ET.png}
\end{subfigure}
\begin{subfigure}{\figlength\textwidth}
\includegraphics[width=0.99\textwidth]{figure/vit_taskonomy/S2_to_K2.png}
\end{subfigure}
\begin{subfigure}{\figlength\textwidth}
\includegraphics[width=0.99\textwidth]{figure/vit_taskonomy/S2_to_K3.png}
\end{subfigure}
\begin{subfigure}{\figlength\textwidth}
\includegraphics[width=0.99\textwidth]{figure/vit_taskonomy/S2_to_N.png}
\end{subfigure}
\begin{subfigure}{\figlength\textwidth}
\includegraphics[width=0.99\textwidth]{figure/vit_taskonomy/S2_to_C.png}
\end{subfigure}
\begin{subfigure}{\figlength\textwidth}
\includegraphics[width=0.99\textwidth]{figure/vit_taskonomy/S2_to_R.png}
\end{subfigure}
\begin{subfigure}{\figlength\textwidth}
\includegraphics[width=0.99\textwidth]{figure/vit_taskonomy/S2_to_S2.5.png}
\end{subfigure}
\begin{subfigure}{\figlength\textwidth}
\includegraphics[width=0.99\textwidth]{figure/vit_taskonomy/S2.5_to_DE.png}
\end{subfigure}
\begin{subfigure}{\figlength\textwidth}
\includegraphics[width=0.99\textwidth]{figure/vit_taskonomy/S2.5_to_DZ.png}
\end{subfigure}
\begin{subfigure}{\figlength\textwidth}
\includegraphics[width=0.99\textwidth]{figure/vit_taskonomy/S2.5_to_EO.png}
\end{subfigure}
\begin{subfigure}{\figlength\textwidth}
\includegraphics[width=0.99\textwidth]{figure/vit_taskonomy/S2.5_to_ET.png}
\end{subfigure}
\begin{subfigure}{\figlength\textwidth}
\includegraphics[width=0.99\textwidth]{figure/vit_taskonomy/S2.5_to_K2.png}
\end{subfigure}
\begin{subfigure}{\figlength\textwidth}
\includegraphics[width=0.99\textwidth]{figure/vit_taskonomy/S2.5_to_K3.png}
\end{subfigure}
\begin{subfigure}{\figlength\textwidth}
\includegraphics[width=0.99\textwidth]{figure/vit_taskonomy/S2.5_to_N.png}
\end{subfigure}
\begin{subfigure}{\figlength\textwidth}
\includegraphics[width=0.99\textwidth]{figure/vit_taskonomy/S2.5_to_C.png}
\end{subfigure}
\begin{subfigure}{\figlength\textwidth}
\includegraphics[width=0.99\textwidth]{figure/vit_taskonomy/S2.5_to_R.png}
\end{subfigure}
\begin{subfigure}{\figlength\textwidth}
\includegraphics[width=0.99\textwidth]{figure/vit_taskonomy/S2.5_to_S2.png}
\end{subfigure}
\caption{Changes in proximal inter-task affinity during the optimization process with Taskonomy benchmark.}
\label{fig:proximal_vit_taskonomy}
\end{figure}
\clearpage


%-------------------------------------------------------------------------
\begin{figure}[h]
    \centering
    \begin{subfigure}{0.24\textwidth}
        \includegraphics[width=0.99\textwidth]{figure/vit_nyud/semseg_to_depth.png}
    \end{subfigure}
    \begin{subfigure}{0.24\textwidth}
        \includegraphics[width=0.99\textwidth]{figure/vit_nyud/semseg_to_normals.png}
    \end{subfigure}
    \begin{subfigure}{0.24\textwidth}
        \includegraphics[width=0.99\textwidth]{figure/vit_nyud/semseg_to_edge.png}
    \end{subfigure}
    \begin{subfigure}{0.24\textwidth}
        \includegraphics[width=0.99\textwidth]{figure/vit_nyud/depth_to_semseg.png}
    \end{subfigure}
    \hfill
    \begin{subfigure}{0.24\textwidth}
        \includegraphics[width=0.99\textwidth]{figure/vit_nyud/depth_to_normals.png}
    \end{subfigure}
    \begin{subfigure}{0.24\textwidth}
        \includegraphics[width=0.99\textwidth]{figure/vit_nyud/depth_to_edge.png}
    \end{subfigure}
    \begin{subfigure}{0.24\textwidth}
        \includegraphics[width=0.99\textwidth]{figure/vit_nyud/normals_to_semseg.png}
    \end{subfigure}
    \begin{subfigure}{0.24\textwidth}
        \includegraphics[width=0.99\textwidth]{figure/vit_nyud/normals_to_depth.png}
    \end{subfigure}
    \hfill
    \begin{subfigure}{0.24\textwidth}
        \includegraphics[width=0.99\textwidth]{figure/vit_nyud/normals_to_edge.png}
    \end{subfigure}
    \begin{subfigure}{0.24\textwidth}
        \includegraphics[width=0.99\textwidth]{figure/vit_nyud/edge_to_semseg.png}
    \end{subfigure}
    \begin{subfigure}{0.24\textwidth}
        \includegraphics[width=0.99\textwidth]{figure/vit_nyud/edge_to_depth.png}
    \end{subfigure}
    \begin{subfigure}{0.24\textwidth}
        \includegraphics[width=0.99\textwidth]{figure/vit_nyud/edge_to_normals.png}
    \end{subfigure}
    \caption{Changes in the proximal inter-task affinity during the optimization process of different sizes of ViT with NYUD-v2.}
    \label{fig:proximal_vit_nyud}
\end{figure}
%-------------------------------------------------------------------------
\begin{figure}[h]
    \centering
    \begin{subfigure}{0.24\textwidth}
        \includegraphics[width=0.99\textwidth]{figure/vit_pascal/semseg_to_human_parts.png}
    \end{subfigure}
    \begin{subfigure}{0.24\textwidth}
        \includegraphics[width=0.99\textwidth]{figure/vit_pascal/semseg_to_sal.png}
    \end{subfigure}
    \begin{subfigure}{0.24\textwidth}
        \includegraphics[width=0.99\textwidth]{figure/vit_pascal/semseg_to_normals.png}
    \end{subfigure}
    \begin{subfigure}{0.24\textwidth}
        \includegraphics[width=0.99\textwidth]{figure/vit_pascal/semseg_to_edge.png}
    \end{subfigure}
    \hfill
    \begin{subfigure}{0.24\textwidth}
        \includegraphics[width=0.99\textwidth]{figure/vit_pascal/human_parts_to_semseg.png}
    \end{subfigure}
    \begin{subfigure}{0.24\textwidth}
        \includegraphics[width=0.99\textwidth]{figure/vit_pascal/human_parts_to_sal.png}
    \end{subfigure}
    \begin{subfigure}{0.24\textwidth}
        \includegraphics[width=0.99\textwidth]{figure/vit_pascal/human_parts_to_normals.png}
    \end{subfigure}
    \begin{subfigure}{0.24\textwidth}
        \includegraphics[width=0.99\textwidth]{figure/vit_pascal/human_parts_to_edge.png}
    \end{subfigure}
    \hfill
    \begin{subfigure}{0.24\textwidth}
        \includegraphics[width=0.99\textwidth]{figure/vit_pascal/sal_to_semseg.png}
    \end{subfigure}
    \begin{subfigure}{0.24\textwidth}
        \includegraphics[width=0.99\textwidth]{figure/vit_pascal/sal_to_human_parts.png}
    \end{subfigure}
    \begin{subfigure}{0.24\textwidth}
        \includegraphics[width=0.99\textwidth]{figure/vit_pascal/sal_to_normals.png}
    \end{subfigure}
    \begin{subfigure}{0.24\textwidth}
        \includegraphics[width=0.99\textwidth]{figure/vit_pascal/sal_to_edge.png}
    \end{subfigure}
        \hfill
    \begin{subfigure}{0.24\textwidth}
        \includegraphics[width=0.99\textwidth]{figure/vit_pascal/normals_to_semseg.png}
    \end{subfigure}
    \begin{subfigure}{0.24\textwidth}
        \includegraphics[width=0.99\textwidth]{figure/vit_pascal/normals_to_human_parts.png}
    \end{subfigure}
    \begin{subfigure}{0.24\textwidth}
        \includegraphics[width=0.99\textwidth]{figure/vit_pascal/normals_to_sal.png}
    \end{subfigure}
    \begin{subfigure}{0.24\textwidth}
        \includegraphics[width=0.99\textwidth]{figure/vit_pascal/normals_to_edge.png}
    \end{subfigure}
        \hfill
    \begin{subfigure}{0.24\textwidth}
        \includegraphics[width=0.99\textwidth]{figure/vit_pascal/edge_to_semseg.png}
    \end{subfigure}
    \begin{subfigure}{0.24\textwidth}
        \includegraphics[width=0.99\textwidth]{figure/vit_pascal/edge_to_human_parts.png}
    \end{subfigure}
    \begin{subfigure}{0.24\textwidth}
        \includegraphics[width=0.99\textwidth]{figure/vit_pascal/edge_to_sal.png}
    \end{subfigure}
    \begin{subfigure}{0.24\textwidth}
        \includegraphics[width=0.99\textwidth]{figure/vit_pascal/edge_to_normals.png}
    \end{subfigure}
    \caption{Changes in proximal inter-task affinity during the optimization process of different sizes of ViT with PASCAL-Context.}
    \label{fig:proximal_vit_pascal}
\end{figure}



\begin{figure}[h]
    \centering
    \begin{subfigure}{0.24\textwidth}
        \includegraphics[width=0.99\textwidth]{figure/vit_beta/semseg_to_depth.png}
    \end{subfigure}
    \begin{subfigure}{0.24\textwidth}
        \includegraphics[width=0.99\textwidth]{figure/vit_beta/semseg_to_normals.png}
    \end{subfigure}
    \begin{subfigure}{0.24\textwidth}
        \includegraphics[width=0.99\textwidth]{figure/vit_beta/semseg_to_edge.png}
    \end{subfigure}
    \begin{subfigure}{0.24\textwidth}
        \includegraphics[width=0.99\textwidth]{figure/vit_beta/depth_to_semseg.png}
    \end{subfigure}
    \hfill
    \begin{subfigure}{0.24\textwidth}
        \includegraphics[width=0.99\textwidth]{figure/vit_beta/depth_to_normals.png}
    \end{subfigure}
    \begin{subfigure}{0.24\textwidth}
        \includegraphics[width=0.99\textwidth]{figure/vit_beta/depth_to_edge.png}
    \end{subfigure}
    \begin{subfigure}{0.24\textwidth}
        \includegraphics[width=0.99\textwidth]{figure/vit_beta/normals_to_semseg.png}
    \end{subfigure}
    \begin{subfigure}{0.24\textwidth}
        \includegraphics[width=0.99\textwidth]{figure/vit_beta/normals_to_depth.png}
    \end{subfigure}
    \hfill
    \begin{subfigure}{0.24\textwidth}
        \includegraphics[width=0.99\textwidth]{figure/vit_beta/normals_to_edge.png}
    \end{subfigure}
    \begin{subfigure}{0.24\textwidth}
        \includegraphics[width=0.99\textwidth]{figure/vit_beta/edge_to_semseg.png}
    \end{subfigure}
    \begin{subfigure}{0.24\textwidth}
        \includegraphics[width=0.99\textwidth]{figure/vit_beta/edge_to_depth.png}
    \end{subfigure}
    \begin{subfigure}{0.24\textwidth}
        \includegraphics[width=0.99\textwidth]{figure/vit_beta/edge_to_normals.png}
    \end{subfigure}
    \caption{Changes in proximal inter-task affinity during the optimization process with different decay rates, $\beta$.}
    \label{fig:proximal_vit_beta}
\end{figure}



\begin{table*}[h]
\caption{Results on Taskonomy with different affinity decay rates $\beta$.}
\vspace{-5pt}
\centering
\renewcommand\arraystretch{1.00}
\resizebox{0.99\textwidth}{!}{
\begin{tabular}{l|ccccccccccc|c}
\midrule[1.0pt]
 & DE & DZ & EO & ET & K2  & K3 & N   & C & R & S2  & S2.5 &  \\ \cmidrule[0.5pt]{2-12}
\multirow{-2}{*}{Task} & L1 Dist. $\downarrow$  & L1 Dist. $\downarrow$ & L1 Dist. $\downarrow$ & L1 Dist. $\downarrow$ & L1 Dist. $\downarrow$ & L1 Dist. $\downarrow$ & L1 Dist. $\downarrow$ & RMSE $\downarrow$    & L1 Dist. $\downarrow$ & L1 Dist. $\downarrow$ & L1 Dist. $\downarrow$  & \multirow{-2}{*}{$\triangle_m$ ($\uparrow$)} \\ \midrule[1.0pt]
Single Task     &0.0183&0.0186&0.1089&0.1713&0.1630&0.0863&0.2953&0.7522&0.1504&0.1738&0.1530&-         \\\midrule[0.5pt]
GD              &0.0188&0.0197&0.1283&0.1745&0.1718&0.0933&0.2599&0.7911&0.1799&0.1885&0.1631&-6.35     \\
$\beta$=0.0001  &0.0165&0.0168&0.1224&0.1739&0.1693&0.0907&0.2304&0.7581&0.1683&0.1831&0.1571&-0.18     \\
$\beta$=0.001   &0.0167&0.0169&0.1228&0.1739&0.1695&0.0910&0.2344&0.7600&0.1691&0.1836&0.1571&-0.64     \\
$\beta$=0.01    &0.0167&0.0171&0.1232&0.1739&0.1698&0.0912&0.2362&0.7623&0.1705&0.1834&0.1576&-1.01     \\
$\beta$=0.1     &0.0167&0.0171&0.1231&0.1739&0.1695&0.0912&0.2355&0.7631&0.1697&0.1831&0.1575&-0.87     \\\midrule[1.0pt]
\end{tabular}}
\label{tab:tab_exp_beta_perf}
\end{table*}


\begin{table*}[h]
\caption{Comparison of different grouping strategies on the Taskonomy benchmark.}
\vspace{-5pt}
\centering
\renewcommand\arraystretch{1.00}
\resizebox{0.99\textwidth}{!}{
\begin{tabular}{l|ccccccccccc|c}
\midrule[1.0pt]
 & DE & DZ & EO & ET & K2  & K3 & N   & C & R & S2  & S2.5 &  \\ \cmidrule[0.5pt]{2-12}
\multirow{-2}{*}{Task} & L1 Dist. $\downarrow$  & L1 Dist. $\downarrow$ & L1 Dist. $\downarrow$ & L1 Dist. $\downarrow$ & L1 Dist. $\downarrow$ & L1 Dist. $\downarrow$ & L1 Dist. $\downarrow$ & RMSE $\downarrow$    & L1 Dist. $\downarrow$ & L1 Dist. $\downarrow$ & L1 Dist. $\downarrow$  & \multirow{-2}{*}{$\triangle_m$ ($\uparrow$)} \\ \midrule[1.0pt]
Heterogeneous               &0.0172&0.0176&0.1252&0.1741&0.1700&0.0920&0.2475&0.7781&0.1743&0.1849&0.1660&-3.10 \\
Random ($N(\mathcal{M})$=2) &0.0177&0.0180&0.1259&0.1741&0.1707&0.0923&0.2662&0.7807&0.1757&0.1871&0.1617&-4.24 \\
Random ($N(\mathcal{M})$=3) &0.0172&0.0177&0.1250&0.1741&0.1703&0.0920&0.2619&0.7754&0.1749&0.1866&0.1607&-3.35 \\
Random ($N(\mathcal{M})$=4) &0.0183&0.0187&0.1277&0.1746&0.1706&0.0936&0.2812&0.7841&0.1804&0.1882&0.1636&-6.12 \\
Random ($N(\mathcal{M})$=5) &0.0186&0.0184&0.1274&0.1747&0.1708&0.0935&0.3150&0.7842&0.1800&0.1888&0.1640&-7.17 \\
Random ($N(\mathcal{M})$=6) &0.0208&0.0209&0.1349&0.1750&0.1721&0.0961&0.3334&0.8222&0.1976&0.1935&0.1703&-13.20\\
Ours                        &0.0167&0.0169&0.1228&0.1739&0.1695&0.0910&0.2344&0.7600&0.1691&0.1836&0.1571&-0.64 \\\midrule[1.0pt]
\end{tabular}}
\label{tab:tab_exp_grouping_strategy}
\end{table*}


\begin{table*}[h]
\caption{Results on Taskonomy with varying batch sizes using ViT-B (batch sizes in brackets).}
\vspace{-5pt}
\centering
\renewcommand\arraystretch{1.00}
\resizebox{0.99\textwidth}{!}{
\begin{tabular}{l|ccccccccccc|c}
\midrule[1.0pt]
 & DE & DZ & EO & ET & K2  & K3 & N   & C & R & S2  & S2.5 &  \\ \cmidrule[0.5pt]{2-12}
\multirow{-2}{*}{Task} & L1 Dist. $\downarrow$  & L1 Dist. $\downarrow$ & L1 Dist. $\downarrow$ & L1 Dist. $\downarrow$ & L1 Dist. $\downarrow$ & L1 Dist. $\downarrow$ & L1 Dist. $\downarrow$ & RMSE $\downarrow$    & L1 Dist. $\downarrow$ & L1 Dist. $\downarrow$ & L1 Dist. $\downarrow$  & \multirow{-2}{*}{$\triangle_m$ ($\uparrow$)} \\ \midrule[1.0pt]
Single Task                 &0.0183&0.0186&0.1089&0.1713&0.1630&0.0863&0.2953&0.7522&0.1504&0.1738&0.1530&-         \\\midrule[0.5pt]
GD(4)                       &0.0208&0.0214&0.1323&0.1747&0.1723&0.0952&0.2768&0.8214&0.1936&0.1921&0.1677&-10.88    \\ \rowcolor[HTML]{E0E0E0}
Ours(4)                     &0.0185&0.0190&0.1273&0.1741&0.1709&0.0928&0.2739&0.7957&0.1809&0.1888&0.1632&-6.19     \\
GD(8)                       &0.0188&0.0197&0.1283&0.1745&0.1718&0.0933&0.2599&0.7911&0.1799&0.1885&0.1631&-6.35     \\ \rowcolor[HTML]{E0E0E0}
Ours(8)                     &0.0167&0.0169&0.1228&0.1739&0.1695&0.0910&0.2344&0.7600&0.1691&0.1836&0.1571&-0.64     \\
GD(16)                      &0.0172&0.0180&0.1248&0.1742&0.1711&0.0920&0.2280&0.7641&0.1706&0.1848&0.1589&-1.94     \\ \rowcolor[HTML]{E0E0E0}
Ours(16)                    &0.0153&0.0154&0.1186&0.1737&0.1682&0.0893&0.1967&0.7334&0.1581&0.1780&0.1516&+4.19     \\\midrule[1.0pt]
\end{tabular}}
\label{tab:tab_exp_batch}
\end{table*}



% \clearpage
%%%%%%%%%%%%%%%%%%%%%%%%%%%%%%%%%%%%%%%%%%%%%%%%%%%%%%%%%%%%%%%%%%%%%%%%%%%%%%%%%%%%
\section{Algorithm Complexity and Computational Load}
\setcounter{table}{0}
\setcounter{figure}{0}
We also provide a detailed time comparison of previous multi-task optimization methods on Taskonomy. As shown in \Cref{tab:training_time_taskonomy}, our approach effectively optimizes multiple tasks with more efficient training times. Our method converges faster than gradient-based approaches, as the primary bottleneck in optimization lies in backpropagation and gradient manipulation.

%%%%%%%%%%%%%%%%%%%%%%%%%%%%%%%%%%%%%%%%%%%%
\begin{table*}[h]
\caption{Comparison of the average time required by each optimization process to handle a single
batch for 11 tasks on Taskonomy.}
\vspace{-5pt}
\centering
\renewcommand\arraystretch{1.00}
\resizebox{\textwidth}{!}{
\scriptsize
\begin{tabular}{l|ccccc|c}
\hline
Process (sec)     & Forward Pass & Backpropagation & Gradient Manipulation & Optimizer Step & Clustering + Affinity Update & Total \\ \hline
GD&0.030(9.04\%)&0.198(59.26\%)&-&0.106(31.69\%)&-&0.33 \\ 
UW&0.030(8.82\%)&0.198(58.24\%)&-&0.112(32.94\%)&-&0.34 \\
DTP&0.030(8.70\%)&0.199(57.68\%)&-&0.116(33.62\%)&-&0.34 \\
DWA&0.031(9.01\%)&0.198(57.56\%)&-&0.115(33.43\%)&-&0.34 \\
GradDrop&0.030(1.13\%)&2.05(80.54\%)&0.411(16.02\%)&0.059(2.30\%)&-&2.57 \\    
MGDA&0.033(0.086\%)&2.06(5.36\%)&36.29(94.47\%)&0.031(0.081\%)&-&38.42 \\
PCGrad&0.030(0.63\%)&2.07(44.09\%)&2.57(54.62\%)&0.031(0.66\%)&-&4.70 \\
CAGrad&0.030(0.57\%)&2.06(39.39\%)&3.11(59.44\%)&0.031(0.59\%)&-&5.23 \\
Aligned-MTL&0.027(0.86\%)&2.07(64.99\%)&1.06(33.20\%)&0.030(0.95\%)&-&3.19 \\
FAMO&0.030 (8.72\%)&0.198(57.56\%)&-&0.116(33.72\%)&-&0.34 \\
Ours&0.072 (7.13\%)&0.576(56.72\%)&-&0.323(31.82\%)&0.044(4.33\%)&1.02 \\ \hline
\end{tabular}}
\label{tab:training_time_taskonomy}
\end{table*}
%%%%%%%%%%%%%%%%%%%%%%%%%%%%%%%%%%%%%%%%%%%%



\end{document}
