\section{Related works}
Contact location estimation for \textit{single-contact} scenarios (i.e. there exists only one contact on the robot) has been well-studied.
Learning-based methods, such as those introduced in \cite{zwiener2018contact, popov2020transfer}, formulate the contact localization problem as a classification. 
In this approach, pre-labeled points on the robot are classified into contact/non-contact categories.
Additionally, in \cite{popov2020transfer}, the location of contact is parameterized as a value between 0 and 1 (from the base to the end-effector).
The aforementioned works simplified the robot surface; that is, they formulated the problem so that the solution corresponds to a point on the robot surface. 
This is due to (i) the kinematic relationships changing non-linearly with joint angles and (ii) the surface geometry being complex.

In contrast, model-based approaches, such as particle filter (PF)-based algorithms \cite{10161173, zwiener2019armcl, bimbo2019collision, popov2021multi, manuelli2016localizing}, handle surface constraints more effectively. 
Several methods have been proposed to keep particles representing the contact point constrained to the robot's surface.
For example, in \cite{10161173, zwiener2019armcl, bimbo2019collision, popov2021multi}, mesh data is used to represent the positions of particles, while in \cite{manuelli2016localizing}, particles are projected onto the robot surface.
With this setup, the particles are updated and resampled to find the contact location that best explains the sensor measurements, using the error metric defined via a quadratic programming (QP).

For both learning-based and PF-based approaches, singularity remains as a challenge. 
In contact localization problems, singularity occurs when multiple pairs of contact points and force vectors result in the same sensor measurements.
In this case, there exist multiple solutions for contact localization problems.
However, in PF-based algorithms, resampling process leads the particles to converge to a single point. 
Namely, PF-based algorithms output only one solution even if there are multiple solutions which have low QP errors.

There have been attempts to address the singularity problem. 
For example, \cite{pang2021identifying, popov2021real} proposed search algorithms for identifying multiple solutions.
In these methods, a set of contact points candidate is narrowed down across the entire robot using QP, but not in a greedy manner.
However, these methods are limited to \textit{single-contact} scenarios.

\textit{Multi-contact} scenarios present increased singularity, making them more difficult to handle.
To mitigate singularity, a key idea in PF-based algorithms is to use past observations.
Intuitively speaking, when dual-contact occurs sequentially and the location of the first contact is already known, the problem can be approached as solving a series of single contacts.
For instance, in \cite{10161173, manuelli2016localizing}, an additional particle set is initialized when the sensor measurements cannot be explained using a current set of particles.
However, initializing additional particle sets relies on a user-defined threshold that is difficult to tune. 


It should be pointed out that the singularity is an inherent characteristic of the contact localization problem and cannot be entirely eliminated.
In this paper, therefore, our goal is not to eliminate singularity, but to predict it more accurately.
From a probabilistic perspective, the singularity problem suggests that the distribution of contact points is multi-modal, indicating uncertainty in localization.
Therefore, we aim to predict this multi-modal distribution more precisely.