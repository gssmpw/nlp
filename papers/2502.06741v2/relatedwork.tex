\section{Related Work}
\label{sec-related}
%\vspace*{-1em}
Introducing Convolutional Neural Networks (CNNs), the field of SR reconstruction revolutionized\cite{Dong2016}. The CNNs capture smooth and slowly varying features of data while struggling to represent finer details and rapid variations accurately, thus introducing the challenge of spectral bias in the super-resolution (SR) task \cite{Zhang2019}. Next, the Single Image Super Resolution CNN (SRCNN) architecture jointly optimizes all layers and three color channels simultaneously \cite{Dong2016}. The performance comparison of the SRCNN, Fast SRCNN-ESM, Efficient Sub-pixel CNN, Enhanced Deep Residual Network, and SRGANs in \cite{Nikhil2024} indicate that the performance of the EDRN is better in terms of PSNR, and it can capture the high-frequency components of the ESM images more efficiently. Very Deep SR (VDSR) further improved the accuracy of the HR images utilizing a deep architecture\cite{Kim2016VDSR}. Enhanced Deep SR (EDSR) uses more residual blocks in VDSR for better HR image reconstruction \cite{Kim2016EDSR}. A combination of residual learning with dense connection (Residual Dense Network) proved to be effective in capturing more detailed features of image too\cite{Zhang2018}. A generalized Implicit Neural Representation (GINR) network approximates the discreet sample locations with a spectral embedding of the graph to train INRs independent of any choice of coordinate system \cite{grattarola2022ginr}. The Higher-Order Implicit Neural Representation (HOIN) approach fosters the high-order interactions among features and mitigates spectral bias through its neural tangent kernel's (NTK) \cite{chen2024hoin}. Deep generative models applied to SR task (SRGANs) proved to be effective in downscaling climate data, specifically projecting the low-resolution to high-resolution ESM data for the regional precipitation \cite{shidqi2023}. The multimodal temperature forecast combines the numerical weather prediction model with U-net and attention mechanism.\cite{Ding2024}.

\noindent \textbf{The Sinusoidal Representation Network (SIREN)} captures high-frequency details in images by utilizing a periodic activation function, thus enhancing the quality of the SR output \cite{SIREN}. SIREN mitigates the frequency characteristics of an image through its periodic activation function and could demonstrate improved spectral bias in the SR task \cite{SIREN}. 

\noindent \textbf{The Vision Transformer (ViT)} architecture introduced a pipeline where a pure transformer is applied directly to sequences of image patches that perform well for the image classification tasks while requiring substantially fewer computational resources to train\cite{Dosovitskiy2020}. ViT can model long-range dependencies and global context, making it particularly suitable for complex datasets like those found in climate research and SR applications \cite{Dosovitskiy2020}.


% They call it a downscaling task, while we call it an SR task

In Section~\ref{sec-method}, we introduce the methodology behind \textbf{ViSIR}, the hybrid vision transformer algorithm. In Section~\ref{sec-data}, we describe the ESM dataset and how the RGB image collection was derived, and the Proof of Concept in Section~\ref{sec-POC} for the three experiments and the summary of findings we present in Section~\ref{sec-Exp}.

\begin{figure}[!ht]
 \centering
 \includegraphics[width=\linewidth]{figures/Fig1-FlowchartNew1.png} 
 \vspace*{-2em}
 \caption{ViSIR divides the input image into patches, pre-processes them using embedding and position encoding, and finally feeds the input to a visual transformer followed by the SIREN architecture.}
 \label{fig-Flowchart}
 \vspace*{-1em}
\end{figure}