\section{Introduction}\label{sec:intro}
% 
The European Union (EU) recently adopted the Artificial Intelligence Act (\EUAIAct)\footnote{EU Regulation 2024/1689, 12.7.2025.} which creates a legal framework for the development, deployment, and use of ``human-centered and trustworthy artificial intelligence (AI)'' (\Art{1}).
% 
The \EUAIAct\ outlines 
desirable ``ethical principles'' of AI systems (\EW{27}) and, i.a., imposes some of these as legally binding requirements for high-risk AI systems (\HRAIS), e.g, AI systems intended to be used to take university admission decisions, or to evaluate individuals' creditworthiness, and for general-purpose AI models (\GPAIMS), e.g., multimodal large language models.
% 
While the \EUAIAct\ is recognized as being one of the first legally binding regulatory frameworks for AI~\citep{Chee_Hummel_2024}, it has faced criticism for its imprecise and incoherent terminology~\citep{laux2024trustworthy, bomhard2024AIAct}, which will complicate its practical implementation. 
Previous work has examined the \EUAIAct\ and its legislative history to clarify terms like explainability~\citep{bordt2022post, vitali2022survey, pavlidis2024unlocking} and fairness~\citep{deck2024implications}. So far, little attention has been paid to robustness and cybersecurity.
% 
Only \emph{AI systems} classified as high-risk (\HRAIS) must meet the robustness and cybersecurity requirements set out in \Art{15}. This paper thus focuses on requirements for \HRAIS. To provide a clearer understanding, we compare these requirements with requirements for specific \emph{AI models}, namely for \GPAIMSSR, in \Art{55}.
% 



Technical solutions to ensure the robustness and cybersecurity of AI systems are often developed within the ML domain. 
%
Therefore, it is essential to inform ML research about the legal requirements to ensure compliance with the \EUAIAct.
%
However, the vagueness of requirements for cybersecurity and robustness under the \EUAIAct\ makes it challenging to inform ML practitioners about the specific legal requirements to further the development of solutions that can ensure compliance with the AIA.
% 
A common understanding between technical and legal
domains can be facilitated through technical standards. 
% 
While the \EUAIAct\ sets out general rules, technical standards specify these rules in detail. 
%
Standards are technical specifications designed to provide voluntary technical or quality specifications for current or future products, processes
or services.\footnote{\Artx~1, 2(1) EU Regulation 1025/2012, OJ L 316, 14.11.2012.}
%
They prescribe technical requirements, 
% 
including characteristics such as quality or performance levels, terminology, and test methods.\footnote{\Artx\ 2(4)(a) and (c) ibid.}
%
Standards have long been integral to EU product legislation under the New Legislative Framework, upon which the AIA is built~\citep{gorywoda2009new}.
%%
%
If approved by the EU Commission, technical standards become harmonized technical standards, which grants a presumption of conformity to products or processes that adhere to them. Consequently, compliance with these standards is deemed to fulfill the requirements of the \EUAIAct, thereby incentivizing providers to adopt them  (\Art{40}).
%
The development of harmonized technical standards for the \EUAIAct\ has been initiated by the EU Commission
and is expected to be completed in the next few years.
%
In addition, the EU Commission is also tasked with developing additional guidelines on the practical implementation on the application of \Art{15} (\Art{96(1)(a)}).



In this paper, we make the following contributions:
%
\begin{itemize}
    \item We analyze and explain the legal requirements related to robustness and cybersecurity in the \EUAIAct, identify related shortcomings, and offer possible solutions for
    some of these shortcomings. 
    \item 
    We evaluate these findings in relation to their practical implementability.
    This aims to inform the standardization process, the development of guidelines by the EU Commission, as well as the benchmark and measurement methodologies referred to in \Art{15(2)}.
    \item We connect the legal requirements for robustness and cybersecurity to ML terminology, aiming to inform ML research and ensure that technical solutions are conducive to legal compliance.
\end{itemize}



This paper is structured as follows: Section~\ref{sec:background_robustness} provides a short background on robustness and cybersecurity in the ML literature.
%
Section~\ref{sec:background_aia} provides an introduction to the \EUAIAct\ and \Art{15}.
%
Section~\ref{sec:legal_challenges} analyzes the requirements outlined in \Art{15} for \HRAIS, addressing both general challenges pertinent to robustness and cybersecurity, as well as specific issues related to each requirement. 
%  
Section~\ref{sec:gpai} examines the requirements in \Art{55} relevant to \GPAIMSSR. 
%
Section~\ref{sec:discussion_outlook} concludes with a summary and recommendations for future research.
