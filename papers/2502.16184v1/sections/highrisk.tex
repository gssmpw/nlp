
\section{Requirements for High-Risk AI Systems}\label{sec:legal_challenges}
%
In this section, we provide an analysis of the overarching challenges of implementing \Art{15} (Section~\ref{sec:general_challenges}), followed by a discussion regarding the robustness requirement in \Art{15(4)} (Section~\ref{sec:challenges_robustness}) and the cybersecurity requirement in \Art{15(5)} (Section~\ref{sec:challenges_cybersecurity}). 



\subsection{General Challenges of \Art{15}}\label{sec:general_challenges}

We identify four legal challenges related to \Art{15} that may arise in its practical implementation.
 % 
 First, there is no clear delineation of the legal terms of robustness and cybersecurity and its counterparts in ML literature.
%
 Second, while the \EUAIAct\ mandates compliance for entire AI systems, the ML literature primarily focuses on models, which may pose practical challenges for implementation. 
%
Third, while accuracy is specified as a requirement in \Art{15}, the provision does not clarify its role in measuring robustness and cybersecurity.
% 
Fourth, the terms 'lifecycle' and 'consistent' performance are not defined, leaving ambiguity about how such performance can be practically ensured.
 


\paragraph{Robustness and Cybersecurity.}

The robustness requirement in \Art{15(4)} addresses ``errors, faults, or inconsistencies'' that may inadvertently occur as the system interacts with its real-world environment. 
%
In contrast, the cybersecurity requirement in \Art{15(5)} targets deliberate attempts ``to alter the use, outputs, or performance'' of an AI system ``by malicious third parties exploiting the system’s vulnerabilities''. 
%
Both robustness and cybersecurity requirements aim to ensure that \HRAIS\ perform consistently and are resilient against any factors that might compromise this performance. They, however, address different threats to consistent performance: robustness requires protection against unintentional causes, whereas cybersecurity protects against intentional actions.
%
While robustness is a new term in EU legislation and not explicitly defined in the \EUAIAct, the term cybersecurity has already been defined in the EU Cybersecurity Act (CSA)\footnote{Regulation (EU) 2019/881, OJ L 151, 7.6.2019.}.
%
Art. 2(1) CSA provides a broad definition of cybersecurity, covering all ``the activities necessary to protect network and information systems, the users of such systems, and other persons affected by cyberthreats''. According to the CSA, a cyber threat is any potential circumstance, event or action that could damage, disrupt or otherwise adversely impact network and information systems, the users of such systems and other persons (Art. 2(8) CSA). Importantly, the CSA does not distinguish between intentional or unintentional cyberthreats as causes of harm. Rather, both are explicitly included in the scope of the CSA (see, e.g., Art. 51(1)(a) and (b) CSA).
%
However, the \EUAIAct\ artificially splits the CSA's concept of cybersecurity by designating unintentional causes as a matter of robustness and restricting cybersecurity to intentional actions. This creates a conflict when aligning the AIA’s requirements with the CSA’s definition of cybersecurity that may lead to regulatory ambiguity. Specifically, \Art{42(2)} considers \HRAIS\ with CSA certification or conformity declarations as compliant with cybersecurity requirements in \Art{15}.\footnote{Note that this holds only true in so far as the cybersecurity certificate or statement of conformity or parts thereof cover those requirements in \Art{15}.} This suggests that the CSA definition of cybersecurity applies to the \EUAIAct, even though it inherently covers both types of causes.
% 

We explore how these legal terms could be understood within the ML domain proposing a simple model as an explanatory heuristic (see Figure~\ref{fig:mapping}).
% 
In ML, robustness refers to maintaining consistent model performance in real-world scenarios~\cite{schwinn2022improving}. 
%
ML research distinguishes between different types of robustness. \emph{Non-adversarial robustness} in ML refers to a model’s ability to maintain performance despite data shifts or noise~\cite{tocchetti2022ai, gojic2023non, saez2016evaluating, olmin2022robustness}. This aligns with the legal term robustness in the \EUAIAct. 
%
\emph{Adversarial robustness} in ML refers to the model’s resistance to intentional perturbations aimed at altering predictions~\cite{szegedy2013intriguing}. This aspect aligns more with the legal concept of cybersecurity.
%
The cybersecurity requirement in \Art{15(5)} aims to ensure AI systems' integrity, confidentiality, and availability, protecting them from threats like unauthorized access, adversarial manipulation, data modification, Denial-of-Service attacks, and theft of sensitive information (e.g., model weights).
% 
However, other scenarios within the ML domain may also fall under the relevant legal terms. For example, language model jailbreaks exploit AI vulnerabilities to bypass safety constraints~\cite{wei2024jailbroken, vassilev2024adversarial}. This aligns more closely with the notion of cybersecurity in protecting against misuse of AI systems.


Our findings are supported by an historic analysis of the legislation process.
%  
As outlined in Section~\ref{sec:purpose_15aia}, the \EUAIAct\ builds on the Ethics Guidelines for Trustworthy AI~\cite{aiiheg2019guidelines}.
%
In the guidelines, the principle of 'technical robustness and safety' includes resilience against attacks, but does not mention cybersecurity.   
%
The White Paper on Artificial Intelligence~\cite{whitepaper}, which elaborates on these guidelines, still lists resilience to attacks against AI systems under ``robustness and accuracy'' without differentiating those terms from cybersecurity.
%
The first official draft of the \EUAIAct\ by the European Commission\footnote{COM/2021/206 final.} was the first official document to distinguish between these three terms and assigned ``resilience against attacks'' to cybersecurity rather than robustness.
%
\EW{27}, which refers to the IHGE guidelines, seems to be a remnant of this development process.
%
It demands under the term `technical robustness' that AI systems should be resilient ``against attempts to alter the use or performance of the AI system'', essentially asking for adversarial robustness.


\paragraph{System vs. Model.}
%
The \EUAIAct\ regulates AI systems, but not AI models, with the only exception being \GPAIMS. 
%
ML research, in contrast, often focuses on developing technical solutions for \emph{ML models}.
%  
This raises the question of whether solely relying on technical solutions for \emph{ML models} is enough to ensure the compliance of a \HRAIS\ with \Art{15}---or whether additional measures are needed.
%
\EW{97} 
specifies that an AI model is an essential component of an AI system.\footnote{Although \Rec{97} specifically refers to \GPAIMS, the wording suggests that the statement about the relationship between AI systems and AI models is of a general nature.}
%  
Additional components can include, i.a., user interfaces, sensors, databases, network communication components, or pre- and post-processing mechanisms for model in- and outputs (\EW{97}, \cite{JRC134461}).
% 
All these individual components should contribute to the overall robustness of the AI system, particularly in scenarios where some components may fail.
%
This is illustrated by \Art{15(4)(ii)}, which states that robustness may be ensured through technical redundancy solutions, including ``back-up or contingency plans''.
% 
Furthermore, \Art{15(5)(iii)} stipulates that the cybersecurity of AI systems shall be achieved through technical solutions that, ``where appropriate'', target training data, pre-trained components, the AI model or its inputs. 
%
This binding provision suggests that at least these different components of the AI system are required to be assessed individually for their appropriateness in mitigating cybersecurity attacks.
%
Thus, \Art{15} should not be understood as requiring a single, unified assessment of the requirements. Instead, it must be interpreted as mandating that each component, including one or more ML models, be assessed individually. The assessment of the AI system’s overall performance is then derived from an aggregation of the individual performance results~\cite{kumar2023}. 
%
This requires an interdisciplinary approach that draws on expertise from fields such as ML, engineering, and human-computer interaction.
%
To establish a common understanding, it can prove beneficial to formally describe the evaluation process of an entire AI system, including potential challenges, such as interdependencies of technical solutions. 
%

\paragraph{Role of Accuracy.}
\Art{15(1)} mandates that \HRAIS\ shall ``achieve an
appropriate level of accuracy''.
%
This is important because trade-offs between different desiderata can exist, such as between robustness and accuracy (see Appendix~\ref{apx:robustness-accuracy-trade-off}).
% 
While accuracy is not defined in the \EUAIAct, Annex~IV~No.~3~\EUAIAct\ states that accuracy is an indicator of the capabilities and performance limits of an AI system.
% 
Accordingly,  accuracy should be measured in at least two ways:
i) separately for ``specific persons or groups of persons on which the system is intended to be used''\footnote{This links to fairness ML literature on diverging error rates for different sensitive groups~\cite{mitchell2021algorithmic, chouldechova2017fairer}.}, and
%
ii) the overall expected accuracy for the ``intended purpose'' of the AI system. 
%
In ML, the metric \emph{accuracy} typically describes the overall proportion of correct predictions out of the total number of predictions made~\cite{carvalho2019machine}. 
%
However, the term can also describe the objective of ``good performance'' of an AI system and, depending on its specific purpose, can also be evaluated using different metrics, such as utility~\cite{corbett2017algorithmic} and f1-score~\cite{sokolova2006beyond}. 
% 
\Art{15(3)} explicitly references  `accuracy and the relevant accuracy metrics', indicating that accuracy is understood as an objective that can be measured with various metrics, leaving the choice of the relevant metric to the provider.
% 
The selection of the metric should consider various factors, including the specific purposes of the ML model, dataset-specific circumstances (e.g., imbalanced data) and the particular model type (e.g., classification, regression). Technical standards and guidelines by the EU Commission should clarify how AI systems' accuracy should be measured.



In ML, robustness is often measured using an \emph{accuracy} metric.
%
Typically, this involves comparing the \emph{accuracy} (or error rates) evaluated on an unperturbed dataset from the original distribution with the accuracy on a perturbed test set (e.g., sampled from the shifted distribution or containing adversarial samples)~\cite{taori2020measuring, hendrycks2021many, goodfellow2014explaining}. 
%
The smaller the difference between these two \emph{accuracy} results, the better the \emph{robustness}.
%
The choice of the \emph{accuracy} metric thus has an impact on the measurement of robustness. 
%
As a result, the ML model may appear more robust under some accuracy metrics than others.
% 
The selection of favorable metrics has been studied in fair ML under the term fairness hacking~\cite{meding2024fairness, simson2024one, black2024d}.
%
Without entering into the debate, we note that there is an ongoing discussion in the ML literature about the existence and characteristics of a trade-off between \emph{robustness} and \emph{accuracy}. 
%
While some research showed that enhancing \emph{robustness} leads to a drop in \emph{test accuracy}~\cite{zhang2019theoretically, rade2022reducing, tsipras2018robustness}, others believe that \emph{robustness} and \emph{accuracy} are not conflicting goals and can be achieved simultaneously~\cite{yang2020closer, raghunathan2020understanding}.
%
Technical standards and guidelines by the EU Commission should provide instructions on how AI system providers should choose an appropriate `accuracy' measure, especially when it is used to assess robustness in subsequent steps.


\paragraph{Consistent Performance Throughout the Lifecycle.}
AI systems must perform ``consistently'' in terms of accuracy, robustness, and cybersecurity ``throughout their lifecycle'' (\Art{15(1)}). 
%
Performance is the ``ability of an AI system to achieve its intended purpose'' (\Art{3(18)}).
%
However, i) the term `lifecycle' is not defined, creating ambiguity about whether it differs from the term `lifetime' used in \Art{12(1)} and \EW{71}; ii) the concept of `consistent' performance is unclear, and it is not specified how it should be measured.
% 

First, `lifecycle' and `lifetime' could be understood as synonyms \cite{marcus2020promoting}. On the other hand, the term `lifetime' could be understood to refer specifically to the active period of the AI system in operation~\cite{murakami2010lifespan}, while `lifecycle' could encompass a broader view of all phases from product design and development to decommissioning~\cite{hamon2024three}. In this case, however, it is unclear how accuracy, robustness and cybersecurity should be ensured beyond the operational phase (e.g., during development). 
%  
\Art{2(8)} clarifies that these requirements do not have to be met during the test and development phase of the \HRAIS--unless the system is tested under real world conditions.
However, the use of the term 'lifecycle' might be interpreted to suggest that the requirements of \Art{15} should not only be assessed when the system is ready for deployment but also be considered during design process itself.


Second, it is unclear what `consistent' performance means and how it should be measured.
In the ML literature, a model’s variability in performance over time is often measured using the variance of a metric such as accuracy or robustness~\cite{kilbertus2020fair, bechavod2019equal, rateike2022don}.
%
The variance of a metric over a time interval indicates its deviation from its mean within this interval. 
%
For instance, high variance in robustness indicates significant fluctuations in robustness levels between two points in time, whereas low variance indicates similar levels of robustness over time. 
%
A low variance could therefore be understood as a consistent performance.\footnote{Some also consider consistency as a metric itself, rather than as a property of a (robustness) metric~\cite{wei2020optimal}.} 
 % 
In practice, performance can vary due to factors, such as random initializations of weights or input data sampling.
%
These types of variations are unavoidable.
%
Defining level of variance considered `consistent' is challenging as it is dependent on the context.
%
Technical standards and guidelines by the EU Commission should clarify how to measure a consistent performance with respect to accuracy, robustness, and cybersecurity, and provide guidance on determining the required level of consistency.



 \subsection{Robustness \Art{15(4)}}\label{sec:challenges_robustness}
%
We now turn to 
challenges specific to \Art{15(4)}.
\Art{15(4)(i)} states that ``technical and organisational measures shall be taken'' to ensure that AI systems are ``as resilient as possible regarding errors, faults or
inconsistencies that may occur within the system or the environment''. \Art{15(4)(ii)} specifies that robustness can be achieved through technical redundancy
solutions, and \Art{15(4)(iii)} requires addressing feedback loops in online learning with possibly biased outputs.


\paragraph{Inconsistent Terminology.}
The term robustness is used inconsistently throughout the \EUAIAct.
%
\Artx\ 15(1) and (4) \EUAIAct\ refer to robustness, whereas the corresponding \EW{27} and \EW{75} both mention technical robustness. 
%
One could argue that technical robustness is synonymous with robustness. 
% 
The term `technical robustness' in \EW{27} may be a remnant of the legislative process that built on the 2019 Ethics Guidelines for Trustworthy AI~\cite{aiiheg2019guidelines} developed by the AI IHEG, which introduced the principle of `technical robustness and safety'.  
%
These guidelines are explicitly referenced by \EW{27}.
%
Nevertheless, it remains unclear why \EW{75} also refers to `technical robustness'.
%
It could be that the wording in \EW{75} is borrowed from \EW{27}.
%
Alternatively, one could argue that robustness in \Artx\ 15(1) and (4) \EUAIAct\ is not limited to technical aspects, but additionally includes some form of non-technical robustness. 
%
The latter could refer to organizational measures that must be implemented to ensure robustness (\Art{15(4)(i)}).
%
Technical standards and guidelines by the EU Commission should clarify what aspects robustness encompasses.
% 


\paragraph{Required Level of 
Robustness.} 
% 
The \EUAIAct\ creates ambiguities regarding the required level of `robustness'.
%
\Art{15(1)} mandates that AI systems must achieve an ``appropriate level'' of robustness.
\Art{15(4)}, however, demands that AI systems shall be ``as resilient as possible'' to ``errors, faults, or inconsistencies'', suggesting a stricter requirement. 
% 
This discrepancy initially appears ambiguous, as it is unclear whether \HRAIS\ must simply meet an appropriate standard of robustness or strive for the highest possible level. 
% 
However, the ``appropriate'' level stated in \Art{15(1)} can be understood as a general principle, which is further specified by \Art{15(4)}. Therefore, appropriate with respect to robustness is to be understood as `as resilient as possible'. 


When determining the appropriate level of robustness of a specific \HRAIS, the intended purpose of the system and the generally acknowledged state of the art (SOTA) on AI and AI-related technologies must be taken into account (\Art{8(1)}).
%
\Art{9(4)} acknowledges that one of the objectives of the required risk management is to achieve an ``appropriate balance in the implementation of measures to fulfil'' requirements. 
\Art{9(5)} further acknowledges the permissibility of a residual risk, meaning that the measures adopted under the risk management system are not expected to eliminate all existing risks, but rather to maintain these residual risks at an `acceptable' level.
%
The risk management system is a continuous iterative process (\Art{9(1)}).
% 
This means that the appropriate level of robustness of \HRAIS\ must be regularly determined and updated, taking into account its purpose and the SOTA while balancing it with other requirements.


\paragraph{Feedback Loops.}
% 
\Art{15(4)(iii)} states that AI systems must be explicitly developed in such a way that they ``duly address'' feedback loops and ``eliminate or reduce'' the risks associated with them.
%
According to \EW{67}, feedback loops occur when the output of an AI system influences its input in future operations, an under
%
understanding that aligns with the concept as found in the ML literature.
%
Feedback loops are a well-studied problem manifesting in various forms~\cite{pagan2023classification}, with the most common issues being a distribution shift~\cite{perdomo2020performative} or a selection bias~\cite{kilbertus2020fair, lum2016predict}.
%
Importantly, in this context, the risk of 'biased outputs' in feedback loops (\Art{15(4)(iii)}) is often studied in the literature on fairness in ML rather than in the literature on \emph{robustness} in ML, which traditionally constitute different research fields and communities~\citep{lee2021machine}.
\footnote{For example, whether there is a trade-off between \emph{robustness} and fairness, or if both pursue similar goals, remains an active discussion in the ML community~\citep{lee2021machine, xu2021robust, pruksachatkun2021does}.
}

An important aspect of \Art{15(4)(iii)} is that it applies specifically to AI systems that learn online. 
%
Online learning ML models iteratively learn from a sequence of data and continuously update their parameters over time~\cite{hoi2021online}.
%
This adaptiveness is reflected in \Art{3(1)} as a factual characteristic of an AI system.
%
The problem with feedback loops in online learning is that newly collected training data can become biased, e.g., due to selection bias, which occurs when the data collected is not representative of the overall population~\cite{zadrozny2004learning, liu2014robust}. This can distort model predictions and reinforce existing biases, ultimately impacting the model's accuracy and fairness~\cite{kilbertus2020fair, bechavod2019equal, rateike2022don}.
%
Offline models, in contrast, are trained on a fixed dataset all at once~\cite{hoi2021online}.
%
Offline models can also carry risks when feedback loops are present: The outputs of an ML model can induce a distribution shift through their interaction with the environment~\cite{d2020fairness, zhang2020fair, liu2018delayed}. Since an offline ML model is not updated, distribution shifts can influence their performance over time and possibly lead to fairness concerns~\cite{liu2018delayed}. 
Although \Art{15(4)} does not explicitly address feedback loops in offline systems, \HRAIS\ are not exempt from addressing them.
% 
Since they can impact the model’s accuracy, feedback loops in offline systems may still need to be addressed to comply with \Art{15(1)}.
%


 \subsection{Cybersecurity \Art{15(5)}}\label{sec:challenges_cybersecurity}

We now turn to legal challenges specific to \Art{15(5)}. \Artx\ 15(5)(i) \EUAIAct\ states that AI systems shall be resilient against attempts to ``alter their use, outputs, or performance by exploiting system vulnerabilities''. \Art{15(5)(ii)} specifies that technical solutions aiming to ensure resilience against such malicious attempts ``shall be appropriate to the relevant circumstances and the risks''. Finally, \Art{15(5)(iii)} mandates specific measures ``to prevent, detect, respond to, and control for attacks'' exploiting AI-specific vulnerabilities. 
%
This section examines the key aspects of compliance with \Art{15(5)}.
However, a mentioned above, providers have an additional pathway for demonstrating compliance with its cybersecurity requirements, namely a certification under the CSA~\cite{casarosa2022cybersecurity}.


\paragraph{Required Level of Cybersecurity.}
%
\Art{15(5)(ii)} mandates that technical solutions must be ``appropriate to the relevant circumstances and the risks'', but this needs further clarification.
% 
The \EUAIAct\ specifically addresses only three kinds of risks: health, safety, and fundamental rights (\EW{1}). 
%
Risks associated with these aspects can be identified and managed through a risk management system that must be put into place as stipulated by \Art{9}.
%
Relevant circumstances are any known and foreseeable circumstances that may have an impact on cybersecurity.\footnote{See \Art{13(3)(b)(ii)} and Appendix~\ref{apx:relevant-circumstances}.}

Mandating a cybersecurity level that is `appropriate to the relevant circumstances' acknowledges that complex ML models generally cannot be expected to be fully resistant to all types of adversarial attacks.
%
This has two major reasons: First, it is impossible to anticipate all types of possible attacks. 
This is acknowledged by \Art{9(5)} which states that measures adopted under the risk management system are not expected to remove all existing risks.
Second, complete protection against a specific attack cannot be guaranteed, especially as adversaries continuously adapt their strategies to overcome possible defense mechanisms~\citep{xie2023defending, kumar2023certifying}. 
% 
The appropriateness of a certain performance level must consider the intended purpose of the system and the generally acknowledged SOTA (see \Art{8(1)}).
%
The measures to ensure cybersecurity adopted are not expected to eliminate all existing risks, but the overall residual risk must be acceptable (see \Art{9(1) and (4)}).
%
Thus, when determining the appropriateness of technical solutions, all applicable requirements of the \EUAIAct\ must be balanced, while also mitigating risks to health, safety, and fundamental rights.
% 

Lastly, while \HRAIS\ are expected to be `as resilient as possible' in terms of robustness, they need only to be `resilient' in terms of cybersecurity. Consequently, the wording of \Art{15} suggests that the robustness requirements are stricter. This may be due to the nature of unintentional causes—such as errors, faults, inconsistencies, or unexpected situations within the system or its operating environment—which are primarily within the provider’s control and justify a higher duty of care. In contrast, attacks by unauthorized third parties are less controllable and therefore justify a (slightly) lower standard for the provider’s duty regarding cybersecurity.


\paragraph{AI-specific Vulnerabilities.}
%
\Art{15(5)} differentiates between 'system vulnerabilities' (\Art{15(5)(i)}) and 'AI-specific vulnerabilities' (\Art{15(5)(iii)}). 
%
As the term vulnerability is not defined, 
%
we provide a working definition.
%
The United States' Common Vulnerabilities and Exposures (CVE) system defines vulnerability as ``[a]n instance of one or more weaknesses [...] that can be exploited, causing a negative impact to confidentiality, integrity, or availability''~\cite{cve_glossary}.
%
\Art{15(5)(iii)} provides a non-exhaustive list of components of an AI system that expose AI-specific vulnerabilities, such as training data, pre-trained components used in training, inputs, or the AI model.
% 
However, there might be additional components of the AI system that may also harbor AI-specific vulnerabilities. 
%
The question is how to identify them.
% 
We suggest performing a hypothetical test.
%
AI models play a central role in an AI system. If a vulnerability would be eliminated by replacing the AI model with a non-AI model, it should be deemed ‘AI-specific’. To define a non-AI model, 
%
we return to the definition of an AI system under the \EUAIAct.
% 
It has been argued that the central characteristic of an AI system is its ability to infer from input to output~\cite{hacker2024comments}. 
%
This inference ability is typically performed by one or more AI models within an AI system. 
%
Therefore, non-AI models are all models lacking inference capability, such as rule-based decision-making systems that rely on predefined rules and logic defined by human experts.\footnote{Note that non-AI rule-based systems use human-defined rules, while rule-based ML models infer rules from data~\cite{naik2023machine, weiss1995rule}, qualifying as AI models. In a different context, the AI IHEG ethics guidelines~\cite{aiiheg2019guidelines} suggest fallback plans where AI systems  switch from a statistical (ML) approach to a rule-based or human-in-the-loop approach.}
%
Since AI-specific vulnerabilities relate to specific components of an AI system, we suggest viewing them as a subset of system vulnerabilities. 
%
To enhance clarity, technical standards should define both terms and mandate a process for identifying them.



\paragraph{Technical Solutions.}
\Art{15(5)(iii)} provides a non-exhaustive list of attacks and AI-specific vulnerabilities that must be addressed through technical solutions:
% 
data poisoning, model poisoning, adversarial examples, model evasion, and confidentiality attacks, which are well-established in the ML literature.
% 
These attacks aim to induce model failures~\cite{vassilev2024adversarial}:
% 
\emph{Data poisoning} attacks manipulate training data~\citep{schwarzschild2021just}, \emph{model poisoning} attacks manipulate the trained ML model~\cite{zhang2022fldetector}, and \emph{model evasion} attacks manipulate test samples~\citep{biggio2013evasion}.
% 
\emph{Confidentiality attacks}, typically explored in the field of privacy in ML, refer to attempts to extract information about the training data or the model itself~\citep{rigaki2023survey}. 

In addition to these attacks, 
 \Art{15(5)(iii)} lists 'model flaws' as an AI-specific vulnerability.
 %
 This is a vague legal term and lacks an established counterpart in the ML literature. 
 %
 In software contexts, the word \emph{flaw} often refers to so-called \emph{bugs}, which are typically the result of human errors in the coding process~\cite{kumar2023, nissenbaum1996accountability}. 
 %
 However, the term model flaw follows the list of attacks outlined above, which are instead designed to exploit the default properties of a properly functioning ML model, and are not directly the results of errors in the coding process.
 %
 Thus, it is unclear what model flaw refers to in this context, and whether technical solutions are only expected to address traditional bugs or coding errors, or whether they should address other ways of exploiting AI-specific vulnerabilities.
%
Given that the term is situated within the cybersecurity requirements for AI system outlined in \Art{15(5)}, we argue that the term model flaws should be interpreted as flaws that enable the exploitation of AI-specific vulnerabilities.
%
Technical standards and guidelines by the EU Commission should define model flaws more clearly and provide guidelines for technical solutions to address these model flaws.
%
This should take into account the arms race between attacker and defender in the realm of adversarial robustness, where both parties are continuously adapting their strategies to outmaneuver the other~\cite{chen2017adversarial}.
%
This makes it infeasible to anticipate and counter all potential attacks that target AI-specific vulnerabilities.

\paragraph{Organizational Measures.}
Numerous EU regulations related to cybersecurity (see e.g., \Artx\ 32 General Data Protection Regulation\footnote{EU Regulation 2016/679, OJ L 119, 4.5.2016.}, Art. 21 NIS 2 Directive\footnote{EU Directive 2022/2555, OJ L 333/80.}) explicitly mandate both technical and organizational measures to ensure cybersecurity. 
In the \EUAIAct,
organizational measures are only mandated for the robustness of \HRAIS\ in \Art{15(4)}, but not for cybersecurity (\Art{15(5)}).
% 
Rather, \Art{15(5)} \EUAIAct\ only focuses on technical solutions for providers of \HRAIS.
% 
The omission of organizational measures in \Art{15(5)} has been criticized in the literature accompanying the legislative process of the \EUAIAct~\cite{biasin2023new}.
%
It is unclear whether providers are still implicitly required to implement organizational measures (in accordance with other EU regulations), as these measures might be inherently included in the concept of cybersecurity, or if they are not mandatory.
%
However, this ambiguity for \emph{providers} of \HRAIS\ (\Art{15}) is mitigated by the fact that \emph{deployers} of \HRAIS\ are required to implement both organizational and technical measures to ensure the proper use of the system in accordance with the instructions for use (\Art{26(1)}). These instructions include the cybersecurity measures put in place.

