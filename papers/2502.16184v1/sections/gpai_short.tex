
\section{Requirements for General-Purpose AI Models With Systemic Risk}\label{sec:gpai}

In the previous section, we examined \HRAIS\ requirements. To further elucidate them, we study \GPAIMSSR, highlighting similarities and differences. 
%
The \EUAIAct\ establishes legal requirements for \GPAIMS, such as multimodal large language models~\cite{openai2023gpt, team2023gemini}, which can perform tasks beyond their original training objective~\citep{gutierrez2023proposal}.
%
\GPAIM\ can be standalone or embedded in an \HRAIS, with the latter requiring compliance with both \GPAIM\ and \HRAIS\ requirements.
%
The \EUAIAct\ distinguishes between \GPAIM\ with systemic risks and those without.
% 
\Art{3(65)} defines `systemic risk' as the risk that is specific to the high-impact capabilities of \GPAIMS\ that have a ``significant impact'' on the market, public health, safety, security, fundamental rights, or society.\footnote{A systemic risk is presumed when the cumulative computation during training exceeds \(10^{25}\) Floating-Point Operations Per Second (FLOPS). \GPAIMS\ with fewer FLOPS may still be classified as posing a systemic risk under \Art{51(1)}. 
There is an ongoing debate over this threshold and if a model's complexity truly reflects its risk level~\cite{hacker2024comments, novelli2024generative, pehlivan2024eu, kutterer2023regulating}. Thresholds and criteria can be modified by the EU Commission, see Appendix~\ref{apx:hrais-gpaim}
}
\GPAIMS\ without systemic risks are exempt from robustness and cybersecurity obligations (\Art{53} ff.).
% 


\paragraph{Cybersecurity Requirements.} 
%
\Art{55(1)(d)} mandates ``an adequate level of cybersecurity protection'' for \GPAIMSSR.
%
\EW{115} further details this cybersecurity requirement.
% 
It mandates cybersecurity protection against ``malicious use or attacks'' and lists specific adversarial threats, such as ``accidental model leakage, unauthorised releases, circumvention of safety
measures'', ``cyberattacks'',
or ``model theft''. 
%
Notably, several of these threats have direct counterparts in the ML literature on \emph{adversarial robustness} and \emph{privacy} for large generative models, such as the circumvention of safety measures (jailbreaking) or model theft~\cite{yao2024survey, li2023multi, wang2023survey}.
%
Although \Art{55(1)(d)} does not define the term `cyberattacks', we infer that it includes the attacks exploiting AI-specific vulnerabilities mentioned in \Art{15} (see Section~\ref{sec:challenges_cybersecurity}). These attacks are studied in the field of \emph{adversarial robustness} and 
can also affect \GPAIMS~\cite{qiang2024learning, das2024exposing, yan2024backdooring, schwinn2024soft, vitorino2024adversarial}---even though specific ML techniques may be necessary to address \GPAIM-specific challenges. 
% 
This relation underscores that the concepts and problems explored under \emph{adversarial robustness} are reflected in the term 'cybersecurity' as used in \Art{55(1)(d)}.
%
To ensure the cybersecurity of a \GPAIMSSR, providers must conduct and document internal and/or external adversarial testing of the model, such as red teaming.\footnote{In this context, red teaming refers to stress testing AI models by simulating adversarial attacks~\cite{feffer2024red}, such as linguistic or semantic attacks against LLMs~\cite{shi2024red}, whereas traditional cybersecurity red teaming focuses on assessing entire systems or networks~\cite{teichmann2023overview}.}

While the \EUAIAct\ mandates robustness requirements for \HRAIS, we observe that it does not impose an explicit equivalent legal requirement for \GPAIMS, regardless of whether they present a systemic risk or not.
%
Specifically, neither \Art{55} nor \EW{155} address unintentional causes for deviations from consistent performance. 
%
In Section~\ref{sec:challenges_robustness}, we stated that \emph{non-adversarial robustness} is reflected in the term robustness in \Art{15}. 
%
Consequently, \GPAIMS, which are not required to fulfill any robustness requirement, are not mandated to be resilient against performance issues, such as data distribution shifts or noisy data.
% 
The \EUAIAct\ itself does not provide an explanation for the omission of a robustness requirement.
%
It may stem from the complexity of political negotiations regarding the \EUAIAct, particularly regarding \GPAIMS, which were not addressed in the initial draft of the regulation but gathered widespread media attention during the legislative procedure.
%
However, evidence from ML research suggests that \emph{non-adversarial robustness} is also relevant for \GPAIMS~\cite{yuan2023revisiting, chen2022foundational}.
% 



\paragraph{Required Level of Cybersecurity.}
\Art{55(1)(d)} mandates an `adequate' level of cybersecurity protection for \GPAIMSSR. This requirement contrasts with the `appropriate' level of cybersecurity mandated for \HRAIS\ under \Art{15(1)}.
%
The use of these two different terms raises questions about whether both \HRAIS\ and \GPAIMS\ should achieve the same level of cybersecurity or to what extent their required levels might differ.
%
On the one hand, `adequate' and `appropriate' could imply different levels of cybersecurity. The Cambridge Dictionary defines the term `adequate' as ``enough or satisfactory for a particular purpose''~\cite{CambridgeAdequate} and `appropriate' as ``suitable or right for a particular situation or occasion''~\cite{CambridgeAppropriate}.
% 
Accordingly, something is `adequate' if it exceeds a minimum threshold that is good enough, while something is `appropriate' if it meets a specific (right) level above that minimum.
%
GPAI models can perform a wide variety of tasks in different contexts and thus be prone to a variety of different intentional causes of harm, making it difficult to identify and mitigate their specific cybersecurity risks.
%
For this reason, it may be reasonable to only mandate an `adequate', i.e., minimum level of cybersecurity.
%
\HRAIS, independently of whether they contain an \GPAIM\ as a component, can be thought of as operating in a more specific contexts, potentially allowing an easier and more precise assessment of cybersecurity risks and thus a more stringent appropriate level of cybersecurity protection.
%
On the other hand, `adequate' and `appropriate' could refer to the same level of cybersecurity.
%
\EW{115} states that ``adequate technical and established solutions'' must be ``appropriate to the relevant circumstances and the risks''. The simultaneous use of both terms in a single sentence, intended to guide the interpretation of  \Art{55(1)(d)}, suggests that they might be intended as synonymous. This is corroborated by the observation that many official language versions of the \EUAIAct\ use a single term for both ``adequate'' and ``appropriate'' in \Art{15(1)} and \Art{55(1)(d)}.\footnote{Such as FR ``approprié'', ES ``adecuado'', GER ``angemessen'', IT ``adeguato''.} 
%
To resolve this ambiguity, technical standards should clarify the required level of cybersecurity for \GPAIMSSR.
