
\usepackage{multirow} 
\usepackage{subcaption} 
\usepackage{algorithm} 

\usepackage{amsthm}
\usepackage{amsfonts}
\usepackage{hyperref}
\usepackage{url}
\usepackage{graphicx} 
\usepackage{pifont} 
\usepackage{caption}

\newcommand{\cmark}{\ding{51}}%
\newcommand{\xmark}{\ding{55}}%

\everypar{\looseness=-1}
\linepenalty=1000

\def\UrlBreaks{\do\/\do-}


\newcommand{\EUAIAct}{AIA} 
\newcommand{\EW}[1]{Rec.~(#1)}
\newcommand{\Rec}[1]{Rec.~(#1)}
\newcommand{\EWx}{Rec.}

\newcommand{\Art}[1]{Art.~#1~\EUAIAct}
\newcommand{\Artx}{Art.}
\newcommand{\Annex}{Annex}
\newcommand{\HRAIS}{HRAIS}

\newcommand{\GPAIM}{GPAIM}
\newcommand{\GPAIMS}{GPAIMs}
\newcommand{\GPAIMSR}{GPAIM with systemic risk}
\newcommand{\GPAIMSSR}{GPAIMs with systemic risk}
\newcommand{\GPAI}{GPAI}


\newcommand{\N}{\mathbb{N}}
\newcommand{\R}{\mathbb{R}}
\newcommand{\Q}{\mathbb{Q}}
\newcommand{\Z}{\mathbb{Z}}
\newcommand{\x}{\times}						
\newcommand{\id}{\mathrm{id}}
\newcommand{\Acal}{\mathcal{A}}
\newcommand{\Bcal}{\mathcal{B}}
\newcommand{\Ccal}{\mathcal{C}}
\newcommand{\Dcal}{\mathcal{D}}
\newcommand{\Ecal}{\mathcal{E}}
\newcommand{\Fcal}{\mathcal{F}}
\newcommand{\Gcal}{\mathcal{G}}
\newcommand{\Hcal}{\mathcal{H}}
\newcommand{\Ical}{\mathcal{I}}
\newcommand{\Jcal}{\mathcal{J}}
\newcommand{\Kcal}{\mathcal{K}}
\newcommand{\Lcal}{\mathcal{L}}
\newcommand{\Mcal}{\mathcal{M}}
\newcommand{\Ncal}{\mathcal{N}}
\newcommand{\Ocal}{\mathcal{O}}
\newcommand{\Pcal}{\mathcal{P}}
\newcommand{\Qcal}{\mathcal{Q}}
\newcommand{\Rcal}{\mathcal{R}}
\newcommand{\Scal}{\mathcal{S}}
\newcommand{\Tcal}{\mathcal{T}}
\newcommand{\Ucal}{\mathcal{U}}
\newcommand{\Vcal}{\mathcal{V}}
\newcommand{\Wcal}{\mathcal{W}}
\newcommand{\Xcal}{\mathcal{X}}
\newcommand{\Ycal}{\mathcal{Y}}
\newcommand{\Zcal}{\mathcal{Z}}
\newcommand{\fa}{\mathfrak{a}}
\newcommand{\fb}{\mathfrak{b}}

\newcommand{\Abf}{\mathbf{A}}
\newcommand{\Bbf}{\mathbf{B}}
\newcommand{\Cbf}{\mathbf{C}}
\newcommand{\Dbf}{\mathbf{D}}
\newcommand{\Ebf}{\mathbf{E}}
\newcommand{\Fbf}{\mathbf{F}}
\newcommand{\Gbf}{\mathbf{G}}
\newcommand{\Hbf}{\mathbf{H}}
\newcommand{\Ibf}{\mathbf{I}}
\newcommand{\Jbf}{\mathbf{J}}
\newcommand{\Kbf}{\mathbf{K}}
\newcommand{\Lbf}{\mathbf{L}}
\newcommand{\Mbf}{\mathbf{M}}
\newcommand{\Nbf}{\mathbf{N}}
\newcommand{\Obf}{\mathbf{O}}
\newcommand{\Pbf}{\mathbf{P}}
\newcommand{\Qbf}{\mathbf{Q}}
\newcommand{\Rbf}{\mathbf{R}}
\newcommand{\Sbf}{\mathbf{S}}
\newcommand{\Tbf}{\mathbf{T}}
\newcommand{\Ubf}{\mathbf{U}}
\newcommand{\Vbf}{\mathbf{V}}
\newcommand{\Wbf}{\mathbf{W}}
\newcommand{\Xbf}{\mathbf{X}}
\newcommand{\Ybf}{\mathbf{Y}}
\newcommand{\Zbf}{\mathbf{Z}}



\newcommand{\Mbb}{\mathbb{M}}
\newcommand{\Leb}{\lambda\hspace{-5.3pt}\lambda}


\newcommand{\sm}{\setminus}						
\newcommand{\ins}{\subseteq} 					
\newcommand{\sni}{\supseteq} 					
\newcommand{\cmpl}{\mathsf{c}}

\newcommand{\pr}{\mathrm{pr}}
\newcommand{\ev}{\mathrm{ev}} 
\newcommand{\pf}{\mathrm{pf}} 
\newcommand{\incl}{\mathrm{incl}}
\newcommand{\two}{\mathbf{2}}
\newcommand{\one}{\mathbf{1}}

\newcommand{\srj}{\twoheadrightarrow}
\newcommand{\inj}{\hookrightarrow}
\newcommand{\bij}{\stackrel{\sim}{\longrightarrow}}
\newcommand{\dshto}{\dashrightarrow}
\newcommand{\dshot}{\dashlefttarrow}

\renewcommand{\Pr}{\mathbb{P}} 	
\newcommand{\Var}{\mathrm{Var}}
\newcommand{\Bias}{\mathrm{Bias}}
\newcommand{\Noise}{\mathrm{Noise}}

	
\newcommand{\E}{\mathbb{E}}

\newcommand{\I}{{1}}
\newcommand{\Pa}{\mathrm{Pa}} 		
\newcommand{\pa}{\mathrm{pa}} 		
\newcommand{\Ch}{\mathrm{Ch}} 		
\newcommand{\Anc}{\mathrm{Anc}} 		
\newcommand{\AnCl}{\mathrm{AnCl}} 
\newcommand{\Desc}{\mathrm{Desc}} 	
\newcommand{\Dist}{\mathrm{Dist}} 
\newcommand{\Pred}{\mathrm{Pred}} 
\newcommand{\Sc}{\mathrm{Sc}}
\newcommand{\NonDesc}{\mathrm{NonDesc}}
\newtheorem{sa}{Theorem}[section]
\newtheorem{Thm}[sa]{Theorem}
\newtheorem{Lem}[sa]{Lemma}
\newtheorem{Prp}[sa]{Proposition}
\newtheorem{Cor}[sa]{Corollary}
\newtheorem{Con}[sa]{Conjecture}
\newtheorem{Fct}[sa]{Facts}
\newtheorem{Prn}[sa]{Principle}



\newtheorem{Def}[sa]{Definition}
\newtheorem{DefLem}[sa]{Definition/Lemma}
\newtheorem{Axm}[sa]{Axiom}
\newtheorem{Not}[sa]{Notation}
\newtheorem{Asm}[sa]{Assumptions}



\newtheorem{Rem}[sa]{Remark}
\newtheorem{Cau}[sa]{Caution}
\newtheorem{Eg}[sa]{Example}
\newtheorem{Tho}[sa]{Thoughts}


\newcommand{\bigdcup}{\mathop{\dot{\bigcup}}}  

\newcommand{\dcup}{\,\dot{\cup}\,}



\newcommand{\ReLU}{\mathrm{ReLU}} 

\newcommand{\lp}{\left ( }
\newcommand{\rp}{\right ) }
\newcommand{\lA}{\left\langle}
\newcommand{\rA}{\right\rangle}
\newcommand{\lB}{\left [ }
\newcommand{\rB}{\right ] }
\newcommand{\lC}{\left \{ }
\newcommand{\rC}{\right \} }
\newcommand{\lI}{\left| }
\newcommand{\rI}{\right| }

\newcommand{\st}{\,\middle|\,}



\newcommand*{\tut}[1][]{\mathrel{\tikz [baseline=-0.25ex,-, #1] \draw [#1] (0pt,0.5ex) -- (1.3em,0.5ex);}}
\newcommand*{\tnt}[1][]{\mathrel{\tikz [baseline=-0.25ex,-, #1] \draw [#1] (0pt,0.5ex) -- node[strike out,draw,-]{} (1.3em,0.5ex);}}
\newcommand*{\tuh}[1][]{\mathrel{\tikz [baseline=-0.25ex,-latex, #1] \draw [#1] (0pt,0.5ex) -- (1.3em,0.5ex);}}
\newcommand*{\hut}[1][]{\mathrel{\tikz [baseline=-0.25ex,latex-, #1] \draw [#1] (0pt,0.5ex) -- (1.3em,0.5ex);}}
\newcommand*{\huh}[1][]{\mathrel{\tikz [baseline=-0.25ex,latex-latex, #1] \draw [#1] (0pt,0.5ex) -- (1.3em,0.5ex);}}
\newcommand*{\ouo}[1][]{\mathrel{\tikz [baseline=-0.25ex,-, #1] \draw [
decoration={markings, mark=at position 0 with {\draw circle (1pt);}, 
mark=at position 1 with {\draw circle (1pt);}}, postaction={decorate},#1] (0pt,0.5ex) -- (1.3em,0.5ex);}}
\newcommand*{\tuo}[1][]{\mathrel{\tikz [baseline=-0.25ex,-, #1] \draw [
decoration={markings, mark=at position 1 with {\draw circle (1pt);}}, postaction={decorate},#1] (0pt,0.5ex) -- (1.3em,0.5ex);}}
\newcommand*{\huo}[1][]{\mathrel{\tikz [baseline=-0.25ex,latex-, #1] \draw [
decoration={markings, mark=at position 1 with {\draw circle (1pt);}}, postaction={decorate},#1] (0pt,0.5ex) -- (1.3em,0.5ex);}}
\newcommand*{\out}[1][]{\mathrel{\tikz [baseline=-0.25ex,-, #1] \draw [
decoration={markings, mark=at position 0 with {\draw circle (1pt);}}, postaction={decorate},#1] (0pt,0.5ex) -- (1.3em,0.5ex);}}
\newcommand*{\ouh}[1][]{\mathrel{\tikz [baseline=-0.25ex,-latex, #1] \draw [
decoration={markings, mark=at position 0 with {\draw circle (1pt);}}, postaction={decorate},#1] (0pt,0.5ex) -- (1.3em,0.5ex);}}
%strike out
\newcommand*{\toh}[1][]{\mathrel{\tikz [baseline=-0.25ex,-latex, #1] \draw [
decoration={markings, mark=at position 0.4 with {\draw[fill] circle (0.8pt);}},
        postaction={decorate},#1] (0pt,0.5ex) -- (1.3em,0.5ex);}}
\newcommand*{\hot}[1][]{\mathrel{\tikz [baseline=-0.25ex,latex-, #1] \draw [
decoration={markings, mark=at position 0.6 with {\draw[fill] circle (0.8pt);}},
        postaction={decorate},#1] (0pt,0.5ex) -- (1.3em,0.5ex);}}
\newcommand*{\tot}[1][]{\mathrel{\tikz [baseline=-0.25ex,-, #1] \draw [
decoration={markings, mark=at position 0.5 with {\draw[fill] circle (0.8pt);}},
        postaction={decorate},#1] (0pt,0.5ex) -- (1.3em,0.5ex);}}				
\newcommand*{\hoh}[1][]{\mathrel{\tikz [baseline=-0.25ex,latex-latex, #1] \draw [
decoration={markings, mark=at position 0.5 with {\draw[fill] circle (0.8pt);}},
        postaction={decorate},#1] (0pt,0.5ex) -- (1.3em,0.5ex);}}



\everypar{\looseness=-1}
\linepenalty=1000

\definecolor{NavyBlue}{RGB}{0, 110, 184}

\definecolor{JungleGreen}{RGB}{0, 169, 154} 

\definecolor{asparagus}{rgb}{0.53, 0.66, 0.42}
\definecolor{chocolate}{rgb}{0.82, 0.41, 0.12}
\definecolor{cadmiumgreen}{rgb}{0.0, 0.42, 0.24}
\definecolor{ao(english)}{rgb}{0.0, 0.5, 0.0}
\definecolor{blue-violet}{rgb}{0.54, 0.17, 0.89}


\definecolor{bittersweet}{rgb}{1.0, 0.44, 0.37}
\definecolor{dodgerblue}{rgb}{0.12, 0.56, 1.0}

\newcommand{\blue}[1]{\textcolor{blue}{#1}}
\newcommand{\red}[1]{\textcolor{red}{#1}}
\newcommand{\green}[1]{\textcolor{cadmiumgreen}{#1}}

