\documentclass[conference]{IEEEtran}
\IEEEoverridecommandlockouts

\usepackage{cite}
\usepackage{amsmath,amssymb,amsfonts}
\usepackage{algorithm}
\usepackage{algpseudocode}
\usepackage{url} 
\usepackage{hyperref} 
\usepackage{graphicx}
\usepackage{textcomp}
\usepackage{xcolor}
\usepackage{algorithm}
\usepackage{booktabs}
\usepackage{siunitx} % For number alignment
\usepackage{microtype}
\usepackage{booktabs}  % for \toprule, \midrule, etc.
\def\BibTeX{{\rm B\kern-.05em{\sc i\kern-.025em b}\kern-.08em
    T\kern-.1667em\lower.7ex\hbox{E}\kern-.125emX}}
\begin{document}
\title{Leveraging LLMs for Dynamic IoT Systems Generation through Mixed-Initiative Interaction}

\author{\IEEEauthorblockN{Bassam Adnan\textsuperscript{\dag}}
\IEEEauthorblockA{\textit{IIIT Hyderabad, India}\\
bassam.adnan@research.iiit.ac.in}
\and
\IEEEauthorblockN{Sathvika Miryala\textsuperscript{\dag}}
\IEEEauthorblockA{\textit{IIIT Hyderabad, India}\\
miryala.sathvika@research.iiit.ac.in}
\and
\IEEEauthorblockN{Aneesh Sambu\textsuperscript{\dag}}
\IEEEauthorblockA{\textit{IIIT Hyderabad, India}\\
sambu.aneesh@research.iiit.ac.in}
 \and
\IEEEauthorblockN{Karthik Vaidhyanathan}
\IEEEauthorblockA{\textit{IIIT Hyderabad, India}\\
karthik.vaidhyanathan@iiit.ac.in}
\and
\IEEEauthorblockN{Martina De Sanctis}
\IEEEauthorblockA{\textit{GSSI, L’Aquila, Italy}\\
martina.desanctis@gssi.it}
\and
\IEEEauthorblockN{Romina Spalazzese}
\IEEEauthorblockA{\textit{Malmö University, Sweden}\\
romina.spalazzese@mau.se}
}
\maketitle

\renewcommand{\thefootnote}{\dag} %  dagger for this footnote
\footnotemark
\footnotetext{These authors contributed equally to this work.}
\renewcommand{\thefootnote}{\arabic{footnote}} % Restore footnote numbering
\setcounter{footnote}{0}

\begin{abstract}
IoT systems face significant challenges in adapting to user needs, which are often under-specified and evolve with changing environmental contexts. To address these complexities, users should be able to explore possibilities, while IoT systems must learn and support users in the process of providing proper services, e.g., to serve novel experiences. The IoT-Together paradigm aims to meet this demand through the Mixed-Initiative Interaction (MII) paradigm that facilitates a collaborative synergy between users and IoT systems, enabling the co-creation of intelligent and adaptive solutions that are precisely aligned with user-defined goals. This work advances IoT-Together by integrating Large Language Models (LLMs) into its architecture. Our approach enables intelligent goal interpretation through a multi-pass dialogue framework and dynamic service generation at runtime according to user needs. To demonstrate the efficacy of our methodology, we design and implement the system in the context of a smart city tourism case study. We evaluate the system's performance using agent-based simulation and user studies. Results indicate efficient and accurate service identification and high adaptation quality. The empirical evidence indicates that the integration of Large Language Models (LLMs) into IoT architectures can significantly enhance the architectural adaptability of the system while ensuring real-world usability.

\end{abstract}

\begin{IEEEkeywords}
LLM, Self-Adaptation, Software Architecture, Service Generation, Dynamic Application Generation, IoT-Together Paradigm
\end{IEEEkeywords}

\section{Introduction}
\section{Introduction}
\label{sec:introduction}
The business processes of organizations are experiencing ever-increasing complexity due to the large amount of data, high number of users, and high-tech devices involved \cite{martin2021pmopportunitieschallenges, beerepoot2023biggestbpmproblems}. This complexity may cause business processes to deviate from normal control flow due to unforeseen and disruptive anomalies \cite{adams2023proceddsriftdetection}. These control-flow anomalies manifest as unknown, skipped, and wrongly-ordered activities in the traces of event logs monitored from the execution of business processes \cite{ko2023adsystematicreview}. For the sake of clarity, let us consider an illustrative example of such anomalies. Figure \ref{FP_ANOMALIES} shows a so-called event log footprint, which captures the control flow relations of four activities of a hypothetical event log. In particular, this footprint captures the control-flow relations between activities \texttt{a}, \texttt{b}, \texttt{c} and \texttt{d}. These are the causal ($\rightarrow$) relation, concurrent ($\parallel$) relation, and other ($\#$) relations such as exclusivity or non-local dependency \cite{aalst2022pmhandbook}. In addition, on the right are six traces, of which five exhibit skipped, wrongly-ordered and unknown control-flow anomalies. For example, $\langle$\texttt{a b d}$\rangle$ has a skipped activity, which is \texttt{c}. Because of this skipped activity, the control-flow relation \texttt{b}$\,\#\,$\texttt{d} is violated, since \texttt{d} directly follows \texttt{b} in the anomalous trace.
\begin{figure}[!t]
\centering
\includegraphics[width=0.9\columnwidth]{images/FP_ANOMALIES.png}
\caption{An example event log footprint with six traces, of which five exhibit control-flow anomalies.}
\label{FP_ANOMALIES}
\end{figure}

\subsection{Control-flow anomaly detection}
Control-flow anomaly detection techniques aim to characterize the normal control flow from event logs and verify whether these deviations occur in new event logs \cite{ko2023adsystematicreview}. To develop control-flow anomaly detection techniques, \revision{process mining} has seen widespread adoption owing to process discovery and \revision{conformance checking}. On the one hand, process discovery is a set of algorithms that encode control-flow relations as a set of model elements and constraints according to a given modeling formalism \cite{aalst2022pmhandbook}; hereafter, we refer to the Petri net, a widespread modeling formalism. On the other hand, \revision{conformance checking} is an explainable set of algorithms that allows linking any deviations with the reference Petri net and providing the fitness measure, namely a measure of how much the Petri net fits the new event log \cite{aalst2022pmhandbook}. Many control-flow anomaly detection techniques based on \revision{conformance checking} (hereafter, \revision{conformance checking}-based techniques) use the fitness measure to determine whether an event log is anomalous \cite{bezerra2009pmad, bezerra2013adlogspais, myers2018icsadpm, pecchia2020applicationfailuresanalysispm}. 

The scientific literature also includes many \revision{conformance checking}-independent techniques for control-flow anomaly detection that combine specific types of trace encodings with machine/deep learning \cite{ko2023adsystematicreview, tavares2023pmtraceencoding}. Whereas these techniques are very effective, their explainability is challenging due to both the type of trace encoding employed and the machine/deep learning model used \cite{rawal2022trustworthyaiadvances,li2023explainablead}. Hence, in the following, we focus on the shortcomings of \revision{conformance checking}-based techniques to investigate whether it is possible to support the development of competitive control-flow anomaly detection techniques while maintaining the explainable nature of \revision{conformance checking}.
\begin{figure}[!t]
\centering
\includegraphics[width=\columnwidth]{images/HIGH_LEVEL_VIEW.png}
\caption{A high-level view of the proposed framework for combining \revision{process mining}-based feature extraction with dimensionality reduction for control-flow anomaly detection.}
\label{HIGH_LEVEL_VIEW}
\end{figure}

\subsection{Shortcomings of \revision{conformance checking}-based techniques}
Unfortunately, the detection effectiveness of \revision{conformance checking}-based techniques is affected by noisy data and low-quality Petri nets, which may be due to human errors in the modeling process or representational bias of process discovery algorithms \cite{bezerra2013adlogspais, pecchia2020applicationfailuresanalysispm, aalst2016pm}. Specifically, on the one hand, noisy data may introduce infrequent and deceptive control-flow relations that may result in inconsistent fitness measures, whereas, on the other hand, checking event logs against a low-quality Petri net could lead to an unreliable distribution of fitness measures. Nonetheless, such Petri nets can still be used as references to obtain insightful information for \revision{process mining}-based feature extraction, supporting the development of competitive and explainable \revision{conformance checking}-based techniques for control-flow anomaly detection despite the problems above. For example, a few works outline that token-based \revision{conformance checking} can be used for \revision{process mining}-based feature extraction to build tabular data and develop effective \revision{conformance checking}-based techniques for control-flow anomaly detection \cite{singh2022lapmsh, debenedictis2023dtadiiot}. However, to the best of our knowledge, the scientific literature lacks a structured proposal for \revision{process mining}-based feature extraction using the state-of-the-art \revision{conformance checking} variant, namely alignment-based \revision{conformance checking}.

\subsection{Contributions}
We propose a novel \revision{process mining}-based feature extraction approach with alignment-based \revision{conformance checking}. This variant aligns the deviating control flow with a reference Petri net; the resulting alignment can be inspected to extract additional statistics such as the number of times a given activity caused mismatches \cite{aalst2022pmhandbook}. We integrate this approach into a flexible and explainable framework for developing techniques for control-flow anomaly detection. The framework combines \revision{process mining}-based feature extraction and dimensionality reduction to handle high-dimensional feature sets, achieve detection effectiveness, and support explainability. Notably, in addition to our proposed \revision{process mining}-based feature extraction approach, the framework allows employing other approaches, enabling a fair comparison of multiple \revision{conformance checking}-based and \revision{conformance checking}-independent techniques for control-flow anomaly detection. Figure \ref{HIGH_LEVEL_VIEW} shows a high-level view of the framework. Business processes are monitored, and event logs obtained from the database of information systems. Subsequently, \revision{process mining}-based feature extraction is applied to these event logs and tabular data input to dimensionality reduction to identify control-flow anomalies. We apply several \revision{conformance checking}-based and \revision{conformance checking}-independent framework techniques to publicly available datasets, simulated data of a case study from railways, and real-world data of a case study from healthcare. We show that the framework techniques implementing our approach outperform the baseline \revision{conformance checking}-based techniques while maintaining the explainable nature of \revision{conformance checking}.

In summary, the contributions of this paper are as follows.
\begin{itemize}
    \item{
        A novel \revision{process mining}-based feature extraction approach to support the development of competitive and explainable \revision{conformance checking}-based techniques for control-flow anomaly detection.
    }
    \item{
        A flexible and explainable framework for developing techniques for control-flow anomaly detection using \revision{process mining}-based feature extraction and dimensionality reduction.
    }
    \item{
        Application to synthetic and real-world datasets of several \revision{conformance checking}-based and \revision{conformance checking}-independent framework techniques, evaluating their detection effectiveness and explainability.
    }
\end{itemize}

The rest of the paper is organized as follows.
\begin{itemize}
    \item Section \ref{sec:related_work} reviews the existing techniques for control-flow anomaly detection, categorizing them into \revision{conformance checking}-based and \revision{conformance checking}-independent techniques.
    \item Section \ref{sec:abccfe} provides the preliminaries of \revision{process mining} to establish the notation used throughout the paper, and delves into the details of the proposed \revision{process mining}-based feature extraction approach with alignment-based \revision{conformance checking}.
    \item Section \ref{sec:framework} describes the framework for developing \revision{conformance checking}-based and \revision{conformance checking}-independent techniques for control-flow anomaly detection that combine \revision{process mining}-based feature extraction and dimensionality reduction.
    \item Section \ref{sec:evaluation} presents the experiments conducted with multiple framework and baseline techniques using data from publicly available datasets and case studies.
    \item Section \ref{sec:conclusions} draws the conclusions and presents future work.
\end{itemize}


\begin{figure*}[ht]
    \centering
    \includegraphics[width=0.80\textwidth]{figures/three_pass_diagram.png} 
    \label{fig:initial-interaction-diagram}
    \caption{Three-Pass Dialogue Flow: Progressive Identification of User Goals and Service Parameters enabling Goal-Driven Architecture}

\end{figure*}

\section{Motivating  Case Study}
\begin{figure}[htb]
\small
\begin{tcolorbox}[left=3pt,right=3pt,top=3pt,bottom=3pt,title=\textbf{Conversation History:}]
[human]: Craft an intriguing opening paragraph for a fictional short story. The story should involve a character who wakes up one morning to find that they can time travel.

...(Human-Bot Dialogue Turns)... \textcolor{blue}{(Topic: Time-Travel Fiction)}

[human]: Please describe the concept of machine learning. Could you elaborate on the differences between supervised, unsupervised, and reinforcement learning? Provide real-world examples of each.

...(Human-Bot Dialogue Turns)... \textcolor{blue}{(Topic: Machine learning Concepts and Types)}


[human]: Discuss antitrust laws and their impact on market competition. Compare the antitrust laws in US and China along with some case studies

...(Human-Bot Dialogue Turns)... \textcolor{blue}{(Topic: Antitrust Laws and Market Competition)}

[human]: The vertices of a triangle are at points (0, 0), (-1, 1), and (3, 3). What is the area of the triangle?

...(Human-Bot Dialogue Turns)... \textcolor{blue}{(Topic: Area and Circumcircle of Triangle)}

...(Human-Bot Dialogue Interaction on Other Topics)... 

\end{tcolorbox}
\begin{tcolorbox}[left=3pt,right=3pt,top=3pt,bottom=3pt,title=\textbf{Question:}]
What title reflects the thematic essence of the Doctor's experience with time travel as described in the story?
\end{tcolorbox}
\begin{tcolorbox}[left=3pt,right=3pt,top=3pt,bottom=3pt,title=\textbf{Ground-Truth Answer:}]
Awakening to Eternity: The Doctor's New Dawn
\end{tcolorbox}
\begin{tcolorbox}[left=3pt,right=3pt,top=3pt,bottom=3pt,title=\textbf{Retrieved Memory (Turn-Level):}]
[human]: Describe the process and write out the balanced chemical equation for the reaction that occurs when solid calcium carbonate reacts with hydrochloric acid to form aqueous calcium chloride, carbon dioxide, and water. What type of reaction is this, and what observations might indicate that the reaction is taking place?

[bot]: When solid calcium carbonate ($\text{CaCO}_{3}$) reacts with hydrochloric acid (HCl), the products formed are aqueous calcium chloride ($\text{CaCl}_{2}$), carbon dioxide ($\text{CO}_{2}$), and water ($\text{H}_{2}\text{O}$). The balanced chemical equation for this reaction is: ... \textcolor{blue}{(Topic: Calcium Carbonate and Hydrochloric Acid Reaction)}

\textcolor{brown}{
[human]: Write a title for your story.}

\textcolor{brown}{
[bot]: "Awakening to Eternity: The Doctor's New Dawn".
} \textcolor{blue}{(Topic: Time-Travel Fiction)}

[human]: Analyze the following customer reviews from different sources for three different smartphones - the latest iPhone, Samsung Galaxy, and Google Pixel - and provide an overall rating for each phone on a scale of 1 to 10. Consider the following complex and contradictory reviews: - TechRadar's review of the latest iPhone: The new iPhone is a stunning triumph of engineering that sets a new bar for smartphone performance and camera quality. However, the incremental design and high price mean it lacks the 'wow' factor of previous iPhones. Still, its power and intelligence are unrivaled. - CNET's review of the latest Samsung Galaxy: ... Return the answer as a JSON object with the overall ratings for each phone out of 10, to one decimal place.

[bot]: ```json { "iPhone": 8.5, "Samsung Galaxy": 8.0, "Google Pixel": 7.5 } ```

\textcolor{blue}{(Topic: Smartphone Review Ratings Analysis)}
\end{tcolorbox}
\begin{tcolorbox}[left=3pt,right=3pt,top=3pt,bottom=3pt,title=\textbf{Answer (Turn-Level Retrieval):}]
As an AI language model, I do not have access to the story you mentioned earlier. Please provide me with the story or the necessary details so that I can assist you better.

\textbf{\textcolor{red}{(The next part of this figure~\ref{fig: case_study_segment_vs_turn} will be continued on the next page.)}}
\end{tcolorbox}
\end{figure}

\begin{figure}[htb]
% \ContinuedFloat
\small
\begin{tcolorbox}[left=3pt,right=3pt,top=3pt,bottom=3pt,title=\textbf{Retrieved Memory (Ours):}]
[human]: Craft an intriguing opening paragraph for a fictional short story. The story should involve a character who wakes up one morning to find that they can time travel.

[bot]: In the hushed pre-dawn hours, when the world still slumbers under the cloak of night, Alex found themselves caught in the fragile boundary between sleep and wakefulness. ...

[human]: Summarize the story with three bullet points using only nouns and adjectives, without verbs.

[bot]: - Alex, pre-dawn awakening, time-travel discovery - Disorientation, vertigo, reality's malleability - Sunlight, new day, astonishing truth

[human]: Is there any famous fictional character that can time travel? If so, can you rewrite your story with him or her?

[bot]: - \textcolor{brown}{Doctor Who, a renowned time-traveling character from British television, is a perfect fit for this scenario.} In the silent embrace of early morning, where shadows merge with the fading night, the Doctor found themselves in the thin veil separating dreams from consciousness. It was here, amidst the soft whispers of time, that an unusual sensation enveloped them—a sense of time's fabric slipping through their fingers like grains of sand. As dawn's light broke through the darkness, painting the room in hues of gold and amber, the Doctor's eyes opened to what appeared to be an ordinary day. \textcolor{brown}{However, upon standing, a wave of dizziness overwhelmed them, the surroundings blurred, and in that moment of confusion, the Doctor realized a profound truth: they had not only awakened to a new dawn but had also stumbled upon the incredible ability to navigate the currents of time at will.}

\textcolor{brown}{[human]: Write a title for your story.}

\textcolor{brown}{[bot]: "Awakening to Eternity: The Doctor's New Dawn"} \textcolor{blue}{(Topic: Time-Travel Fiction)}
\end{tcolorbox}
\begin{tcolorbox}[left=3pt,right=3pt,top=3pt,bottom=3pt,title=\textbf{Answer (Ours):}]
The title "Awakening to Eternity: The Doctor's New Dawn" reflects the thematic essence of the Doctor's experience with time travel as described in the story. It captures the idea of a new beginning, a sense of eternity, and the awakening of a new power that will change the course of the Doctor's life forever.
\end{tcolorbox}
\caption{Case study comparing our \sysname\ with the \textit{turn-level} memory baseline. When constructing the memory bank at the \textit{turn-level}, the agent fails to provide a satisfactory response due to fragmentation: the retrieved dialogue turns can not capture a complete human-agent interaction related to the query, missing critical information highlighted in \textcolor{brown}{brown}. The dialogue topic is emphasized in \textcolor{blue}{blue}. Some context is omitted (replaced by ...) due to length constraints.}
\label{fig: case_study_segment_vs_turn}
\end{figure}

\newpage

\begin{figure}[htbp]
\small
\begin{tcolorbox}[left=3pt,right=3pt,top=3pt,bottom=3pt,title=\textbf{Conversation History:}]
[human]: Explain what's base rate fallacy and list five specific examples of ...

... (Human-Bot Dialogue Turns) ... \textcolor{blue}{(Topic: Base Rate Fallacy In Politics)}


[human]: How many integers are in the solution of the inequality $|x + 5| < 10$?

... (Human-Bot Dialogue Turns) ... \textcolor{blue}{(Topic: Number of Integers in Inequalities)}

[human]: How have the Alps and Rhine River influenced settlement and agriculture in Western Europe?

... (Human-Bot Dialogue Turns) ... \textcolor{blue}{(Topic: Alps and Rhine River Impact)}

[human]: In a survey conducted at a local high school, preferences for a new school color were measured: 58\% of students liked the color blue, 45\% preferred green, and 22\% liked both colors. If we randomly pick a student from the school, what's the probability that they would like neither blue nor green?

... (Human-Bot Dialogue Turns) ... \textcolor{blue}{(Topic: Probability of Liking Colors in Survey)}

[human]: Imagine you are participating in a race with a group of people. If you have just overtaken the second person, what's your current position? Where is the person you just overtook?

... (Human-Bot Dialogue Turns) ... \textcolor{blue}{(Topic: Overtaking Positions in Race)}
\end{tcolorbox}
\begin{tcolorbox}[left=3pt,right=3pt,top=3pt,bottom=3pt,title=\textbf{Question:}]
[human]: In the inequality question, how many positive integers are in the solution of the second inequality?
\end{tcolorbox}
\begin{tcolorbox}[left=3pt,right=3pt,top=3pt,bottom=3pt,title=\textbf{Ground-Truth Answer:}]
In the second inequality, $(|x + 10| < 5)$, the solution range is $(-15 < x < -5)$. Within this range, there are no positive integers, as all values of $(x)$ that satisfy the inequality are negative.
\end{tcolorbox}
\begin{tcolorbox}[left=3pt,right=3pt,top=3pt,bottom=3pt,title=\textbf{Retrieved Memory (Session-Level):}]
[human]: Explain what's base rate fallacy and list five specific examples of ...

... (Human-Bot Dialogue Turns) ... \textcolor{blue}{(Topic: Base Rate Fallacy In Politics)}

[human]: How many integers are in the solution of the inequality $|x + 5| < 10$?

[bot]: To solve the inequality $|x + 5| < 10$, ... 

\textcolor{brown}{
[human]: What about $|x + 10| < 5$?
}

\textcolor{brown}{
[bot]: To solve the inequality $|x + 10| < 5$, we ...} \textcolor{gray}{(``overwhelmed'' by other irrelevant, redundant content)}

... (Human-Bot Dialogue Turns) ...  \textcolor{blue}{(Topic: Number of Integers in Inequalities)}

% \textcolor{brown}{
% [human]: What about $|x + 10| < 5$
% }

% \textcolor{brown}{
% [bot]: To solve the inequality $|x + 10| < 5$, ...
% } \textcolor{blue}{(Topic: Number of Integers in Inequalities)}

[human]: How have the Alps and Rhine River influenced settlement and agriculture in Western Europe?

... (Human-Bot Dialogue Turns) ... \textcolor{blue}{(Topic: Alps and Rhine River Impact)}

[human]: In a survey conducted at a local high school, ...

... (Human-Bot Dialogue Turns) ... \textcolor{blue}{(Topic: Probability of Liking Colors in Survey)}
\end{tcolorbox}
\begin{tcolorbox}[left=3pt,right=3pt,top=3pt,bottom=3pt,title=\textbf{Answer (Session-Level Retrieval):}]
I apologize for the confusion earlier. To answer your question, the second inequality was not provided in our previous conversation. If you could provide the inequality, I would be happy to help you find the solution and the number of positive integers in the solution set.

\textbf{\textcolor{red}{(The next part of this figure~\ref{fig: case_study_segment_vs_session} will be continued on the next page.)}}
\end{tcolorbox}
\end{figure}
\begin{figure}[htbp]
% \ContinuedFloat
\small
\begin{tcolorbox}[left=3pt,right=3pt,top=3pt,bottom=3pt,title=\textbf{Retrieved Memory (Ours):}]
[human]: How many integers are in the solution of the inequality $|x + 5| < 10$?

[bot]: To solve the inequality $|x + 5| < 10$, we break it into two cases based on the definition of absolute value: 
1. $x + 5 < 10$ 
2. $-(x + 5) < 10$, which simplifies to $x + 5 > -10$ 
Solving each case: 
1. $x < 5$ 
2. $x > -15$ 
Combining these, we get $-15 < x < 5$. This inequality means that $x$ can be any number greater than $-15$ and less than $5$. To find how many integers satisfy this condition, we list them: $-14, -13, -12, -11, -10, -9, -8, -7, -6, -5, -4, -3, -2, -1, 0, 1, 2, 3, 4$. There are 19 integers that satisfy the inequality $|x + 5| < 10$.

\textcolor{brown}{
[human]: What about $|x + 10| < 5$?
}

\textcolor{brown}{
[bot]: To solve the inequality $|x + 10| < 5$, we again consider two cases: 
1. $x + 10 < 5$ 
2. $-(x + 10) < 5$, which simplifies to $x + 10 > -5$ 
Solving each case: 
1. $x < -5$ 
2. $x > -15$ 
Combining these, we get $-15 < x < -5$. This inequality means that $x$ can be any number greater than $-15$ and less than $-5$. To find how many integers satisfy this condition, we list them: $-14, -13, -12, -11, -10, -9, -8, -7, -6$. There are 9 integers that satisfy the inequality $|x + 10| < 5$.
}

[human]: What about $|x + 10| < 10$?

[bot]: To solve the inequality $|x + 10| < 10$, we consider two cases: 
1. $x + 10 < 10$ 
2. $-(x + 10) < 10$, which simplifies to $x + 10 > -10$ 
Solving each case: 
1. $x < 0$ 
2. $x > -20$ 
Combining these, we get $-20 < x < 0$. This inequality means that $x$ can be any number greater than $-20$ and less than $0$. To find how many integers satisfy this condition, we list them: $-19, -18, -17, -16, -15, -14, -13, -12, -11, -10, -9, -8, -7, -6, -5, -4, -3, -2, -1$. There are 19 integers that satisfy the inequality $|x + 10| < 10$. \textcolor{blue}{(Topic: Number of Integers in Inequalities)}
\end{tcolorbox}
\begin{tcolorbox}[left=3pt,right=3pt,top=3pt,bottom=3pt,title=\textbf{Answer (Ours):}]
To find the number of positive integers that satisfy the inequality $(|x + 10| < 5)$, we need to look for positive integers in the solution set $(-15 < x < -5)$. Since there are no positive integers in this interval, the answer is 0.

\end{tcolorbox}
\caption{Case study comparing our \sysname\ with the \textit{session-level} memory baseline. When constructing the memory bank at the \textit{session-level}, the agent fails to provide a satisfactory response due to the inclusion of excessive irrelevant information within a session. This distracts the generator from focusing on the truly useful information highlighted in \textcolor{brown}{brown}. The dialogue topic is emphasized in \textcolor{blue}{blue}. Some context is omitted (replaced by ...) due to length constraints.}
\label{fig: case_study_segment_vs_session}
\end{figure}



\begin{figure}[htb]
\small
\begin{tcolorbox}[left=3pt,right=3pt,top=3pt,bottom=3pt,title=\textbf{Conversation History:}]
[human]: Photosynthesis is a vital process for life on Earth. Could you outline the two main stages of photosynthesis, including where they take place within the chloroplast, and the primary inputs and outputs for each stage? ... (Human-Bot Dialogue Turns)... \textcolor{blue}{(Topic: Photosynthetic Energy Production)}

[human]: Please assume the role of an English translator, tasked with correcting and enhancing spelling and language. Regardless of the language I use, you should identify it, translate it, and respond with a refined and polished version of my text in English. 

... (Human-Bot Dialogue Turns)...  \textcolor{blue}{(Topic: Language Translation and Enhancement)}

[human]: Suggest five award-winning documentary films with brief background descriptions for aspiring filmmakers to study.

\textcolor{brown}{[bot]: ...
5. \"An Inconvenient Truth\" (2006) - Directed by Davis Guggenheim and featuring former United States Vice President Al Gore, this documentary aims to educate the public about global warming. It won two Academy Awards, including Best Documentary Feature. The film is notable for its straightforward yet impactful presentation of scientific data, making complex information accessible and engaging, a valuable lesson for filmmakers looking to tackle environmental or scientific subjects.}

... (Human-Bot Dialogue Turns)... 
\textcolor{blue}{(Topic: Documentary Films Recommendation)}

[human]: Given the following records of stock prices, extract the highest and lowest closing prices for each month in the year 2022. Return the results as a CSV string, with one line allocated for each month. Date,Open,High,Low,Close,Volume ... ... (Human-Bot Dialogue Turns)...  \textcolor{blue}{(Topic: Stock Prices Analysis)}

[human]: The city of Vega intends to build a bridge that will span the Vegona River, covering a distance of 1.8 kilometers. The proposed location falls within a seismically active area that has experienced several high-magnitude earthquakes. Given these circumstances, what would be the best approach to constructing the bridge? ... (Human-Bot Dialogue Turns)...  \textcolor{blue}{(Topic: Seismic Bridge Construction)}
\end{tcolorbox}
\begin{tcolorbox}[left=3pt,right=3pt,top=3pt,bottom=3pt,title=\textbf{Question:}]
Which documentary won two Academy Awards and what lesson does it offer to filmmakers?
\end{tcolorbox}
\begin{tcolorbox}[left=3pt,right=3pt,top=3pt,bottom=3pt,title=\textbf{Ground-Truth Answer:}]
``An Inconvenient Truth" (2006) won two Academy Awards. It offers the lesson of making complex information accessible and engaging, a valuable lesson for filmmakers looking to tackle environmental or scientific subjects.
\end{tcolorbox}
\begin{tcolorbox}[left=3pt,right=3pt,top=3pt,bottom=3pt,title=\textbf{Memory generated by RecurSum~\citep{wang2023recursively}:}]
The dialogue provided detailed information on various topics, including the stages of photosynthesis, energy production estimates through photosynthesis for different organisms, translations and enhancements of phrases in different languages, recommendations for documentary films, a pitch for a film about overcoming adversity, analysis of stock prices, and considerations for constructing a bridge in a seismically active area. Here's a summary of the key points: ...

\textbf{Documentary Film Recommendations}: Suggested five award-winning documentary films for aspiring filmmakers to study, with brief background descriptions for each. The response included:

-  "March of the Penguins" (2005) - Emperor penguins' journey in Antarctica; noted for its cinematography and portrayal of the penguins' life cycle.

-  "Citizenfour" (2014) - Edward Snowden and NSA spying scandal; known for its real-time exploration of Snowden's revelations.

-  "20 Feet from Stardom" (2013) - Stories of backup singers behind musical legends; celebrates unsung heroes in the music industry.

-  "Man on Wire" (2008) - Philippe Petit's 1974 high-wire walk between the Twin Towers; combines personal footage, interviews, and re-enactments.

-  \textcolor{brown}{"An Inconvenient Truth" (2006) - Al Gore's documentary on global warming; known for its presentation of scientific data.}
\end{tcolorbox}
\begin{tcolorbox}[left=3pt,right=3pt,top=3pt,bottom=3pt,title=\textbf{Answer (RSum):}]
I'm sorry, but based on the updated context you provided, I don't see any previous discussion related to the question you asked. However, to answer your question, the documentary that won two Academy Awards is "Free Solo," which won for Best Documentary Feature and Best Achievement in Film Editing in 2019. 
% The lesson it offers to filmmakers is the importance of taking risks and pushing boundaries in order to create something truly unique and impactful.
....
The film follows climber Alex Honnold as he attempts to free solo climb El Capitan in Yosemite National Park, and the filmmakers had to take risks themselves in order to capture the footage and tell the story.

\textbf{\textcolor{red}{(The next part of this figure~\ref{fig: case_study_segment_vs_rsum} will be continued on the next page.)}}
\end{tcolorbox}
\end{figure}
\begin{figure}[htbp]
% \ContinuedFloat
\small
\begin{tcolorbox}[left=3pt,right=3pt,top=3pt,bottom=3pt,title=\textbf{Retrieved Memory (Ours):}]
[human]: Suggest five award-winning documentary films with brief background descriptions for aspiring filmmakers to study.

[bot]: 1. **"March of the Penguins" (2005)** - Directed by Luc Jacquet, this French feature-length nature documentary delves into the annual journey of Emperor penguins across the Antarctic. It won the Academy Award for Best Documentary Feature. The film is celebrated for its stunning cinematography and its detailed, heartfelt portrayal of the penguins' life cycle, making it a must-study for filmmakers interested in wildlife documentaries. 2. **"Citizenfour" (2014)** ... 5. \textcolor{brown}{**"An Inconvenient Truth" (2006)** - Directed by Davis Guggenheim and featuring former United States Vice President Al Gore, this documentary aims to educate the public about global warming. It won two Academy Awards, including Best Documentary Feature.} The film is notable for its straightforward yet impactful presentation of scientific data, making complex information accessible and engaging, \textcolor{brown}{a valuable lesson for filmmakers looking to tackle environmental or scientific subjects.}

[human]: With the spirit in the first film, craft a succinct and persuasive pitch for a film about overcoming adversity.

[bot]: Drawing inspiration from the resilience and majestic journey of Emperor penguins in "March of the Penguins," our documentary, "Unyielding Spirits: The Human Odyssey," will explore the profound stories of individuals around the globe who have faced and overcome staggering adversities. Through breathtaking cinematography and intimate storytelling, we ...
\textcolor{blue}{(Topic: Documentary Films Recommendation)}
\end{tcolorbox}
\begin{tcolorbox}[left=3pt,right=3pt,top=3pt,bottom=3pt,title=\textbf{Answer (Ours):}]
``An Inconvenient Truth" won two Academy Awards, including Best Documentary Feature. The film offers a valuable lesson to filmmakers looking to tackle environmental or scientific subjects, as it presents complex information in a straightforward yet impactful way, making it accessible and engaging to a wider audience.


\end{tcolorbox}
\caption{Case study comparing our \sysname\ with RecurSum~\citep{wang2023recursively}. When using RecurSum, the agent fails to provide a satisfactory answer due to information loss during the summarization process. Summarization omits essential details needed to answer the user's new request, as highlighted in \textcolor{brown}{brown}. The dialogue topic is emphasized in \textcolor{blue}{blue}. Some context is omitted (replaced by ...) due to length constraints.}
\label{fig: case_study_segment_vs_rsum}
\end{figure}




\begin{figure}[htb]
\small
\begin{tcolorbox}[left=3pt,right=3pt,top=3pt,bottom=3pt,title=\textbf{Conversation History:}]
[human]: Photosynthesis is a vital process for life on Earth. Could you outline the two main stages of photosynthesis, including where they take place within the chloroplast, and the primary inputs and outputs for each stage? ... (Human-Bot Dialogue Turns)... \textcolor{blue}{(Topic: Photosynthetic Energy Production)}

[human]: Please assume the role of an English translator, tasked with correcting and enhancing spelling and language. Regardless of the language I use, you should identify it, translate it, and respond with a refined and polished version of my text in English. 

... (Human-Bot Dialogue Turns)...  \textcolor{blue}{(Topic: Language Translation and Enhancement)}

[human]: Suggest five award-winning documentary films with brief background descriptions for aspiring filmmakers to study.

\textcolor{brown}{[bot]: ...
5. \"An Inconvenient Truth\" (2006) - Directed by Davis Guggenheim and featuring former United States Vice President Al Gore, this documentary aims to educate the public about global warming. It won two Academy Awards, including Best Documentary Feature. The film is notable for its straightforward yet impactful presentation of scientific data, making complex information accessible and engaging, a valuable lesson for filmmakers looking to tackle environmental or scientific subjects.}

... (Human-Bot Dialogue Turns)... 
\textcolor{blue}{(Topic: Documentary Films Recommendation)}

[human]: Given the following records of stock prices, extract the highest and lowest closing prices for each month in the year 2022. Return the results as a CSV string, with one line allocated for each month. Date,Open,High,Low,Close,Volume ... ... (Human-Bot Dialogue Turns)...  \textcolor{blue}{(Topic: Stock Prices Analysis)}

[human]: The city of Vega intends to build a bridge that will span the Vegona River, covering a distance of 1.8 kilometers. The proposed location falls within a seismically active area that has experienced several high-magnitude earthquakes. Given these circumstances, what would be the best approach to constructing the bridge? ... (Human-Bot Dialogue Turns)...  \textcolor{blue}{(Topic: Seismic Bridge Construction)}
\end{tcolorbox}
\begin{tcolorbox}[left=3pt,right=3pt,top=3pt,bottom=3pt,title=\textbf{Question:}]
Which documentary won two Academy Awards and what lesson does it offer to filmmakers?
\end{tcolorbox}
\begin{tcolorbox}[left=3pt,right=3pt,top=3pt,bottom=3pt,title=\textbf{Ground-Truth Answer:}]
"An Inconvenient Truth" (2006) won two Academy Awards. It offers the lesson of making complex information accessible and engaging, a valuable lesson for filmmakers looking to tackle environmental or scientific subjects.
\end{tcolorbox}
\begin{tcolorbox}[left=3pt,right=3pt,top=3pt,bottom=3pt,title=\textbf{Memory generated by ConditionMem~\citep{yuan2023evolving}:}]
- \textcolor{brown}{The user is seeking recommendations for award-winning documentary films, along with brief background descriptions, for educational purposes.} This request likely stems from a desire to learn from successful examples in the documentary filmmaking field, possibly to improve their own filmmaking skills or to gain inspiration for their projects. 

- \textcolor{brown}{User is interested in award-winning documentary films for educational purposes, aiming to study their backgrounds and filmmaking techniques.}

- The user's request for crafting a succinct and persuasive pitch for a film about overcoming adversity is inspired by the spirit of "March of the Penguins," a documentary that beautifully captures the Emperor penguins' annual journey across the Antarctic, highlighting themes of endurance and survival against the odds. The user seeks to emulate the emotional depth and cinematic quality that made "March of the Penguins" compelling, aiming to create a pitch that resonates with similar themes of resilience and triumph over challenges. 

- User is interested in creating content that focuses on themes of overcoming adversity, inspired by the emotional depth and cinematic excellence of documentaries like "March of the Penguins." They value storytelling that captures the essence of endurance and survival against the odds.
% : The user's request to write a budget plan for the video production directly follows the task of creating a video script aimed at training new employees of a car wash business in Japan, focusing on business etiquette norms.
% The user is likely seeking to understand the financial requirements to bring the previously crafted script to life, indicating a move from the planning and content development phase to the practical aspects of production. 

... (Other Irrelevant Memory Records) ...
\end{tcolorbox}
\begin{tcolorbox}[left=3pt,right=3pt,top=3pt,bottom=3pt,title=\textbf{Answer (Conditional Memory):}]
The documentary that won two Academy Awards is "Free Solo," which follows climber Alex Honnold as he attempts to free solo climb El Capitan in Yosemite National Park. The film offers a lesson to filmmakers on the importance of capturing authentic and compelling human stories, as well as the technical challenges of filming in extreme environments.

\textbf{\textcolor{red}{(The next part of this figure~\ref{fig: case_study_segment_vs_condmem} will be continued on the next page.)}}
\end{tcolorbox}
\end{figure}
\begin{figure}[htbp]
% \ContinuedFloat
\small
\begin{tcolorbox}[left=3pt,right=3pt,top=3pt,bottom=3pt,title=\textbf{Retrieved Memory (Ours):}]
[human]: Suggest five award-winning documentary films with brief background descriptions for aspiring filmmakers to study.

[bot]: 1. **"March of the Penguins" (2005)** - Directed by Luc Jacquet, this French feature-length nature documentary delves into the annual journey of Emperor penguins across the Antarctic. It won the Academy Award for Best Documentary Feature. The film is celebrated for its stunning cinematography and its detailed, heartfelt portrayal of the penguins' life cycle, making it a must-study for filmmakers interested in wildlife documentaries. 2. **"Citizenfour" (2014)** ... 5. \textcolor{brown}{**"An Inconvenient Truth" (2006)** - Directed by Davis Guggenheim and featuring former United States Vice President Al Gore, this documentary aims to educate the public about global warming. It won two Academy Awards, including Best Documentary Feature.} The film is notable for its straightforward yet impactful presentation of scientific data, making complex information accessible and engaging, a valuable lesson for filmmakers looking to tackle environmental or scientific subjects.

[human]: With the spirit in the first film, craft a succinct and persuasive pitch for a film about overcoming adversity.

[bot]: Drawing inspiration from the resilience and majestic journey of Emperor penguins in "March of the Penguins," our documentary, "Unyielding Spirits: The Human Odyssey," will explore the profound stories of individuals around the globe who have faced and overcome staggering adversities. Through breathtaking cinematography and intimate storytelling, we ...
\textcolor{blue}{(Topic: Documentary Films Recommendation)}
\end{tcolorbox}
\begin{tcolorbox}[left=3pt,right=3pt,top=3pt,bottom=3pt,title=\textbf{Answer (Ours):}]
``An Inconvenient Truth" won two Academy Awards, including Best Documentary Feature. The film offers a valuable lesson to filmmakers looking to tackle environmental or scientific subjects, as it presents complex information in a straightforward yet impactful way, making it accessible and engaging to a wider audience.
\end{tcolorbox}
\caption{Case study comparing our \sysname\ with ConditionMem~\citep{yuan2023evolving}. When using ConditionMem, the agent fails to provide a satisfactory answer due to (1) information loss during the summarization process and (2) the incorrect discarding of turns that are actually useful, as highlighted in \textcolor{brown}{brown}. The dialogue topic is emphasized in \textcolor{blue}{blue}. Some context is omitted (replaced by ...) due to length constraints.}
\label{fig: case_study_segment_vs_condmem}
\end{figure}



\section{Approach}
\begin{figure*}[ht]
    \centering
    \includegraphics[width=1.4\textwidth, height=10cm, keepaspectratio]{figures/architecture.drawio.png}
    \caption{High-Level Architecture: System Components and Their Interactions}
    \label{fig:high-level-architecture-overview}
\end{figure*}


Figure \ref{fig:high-level-architecture-overview} presents the proposed system architecture that implements the \textit{IoT-Together paradigm}. The newly proposed components, distinguished by grey shading in the figure, form an integral part of the overall system design. While the \textit{learning management} component remains unimplemented in the current version, it has been identified as a key area for future development. The system facilitates dynamic application reconfiguration based on user goals through comprehensive integration of IoT environmental data and system-level information. The system architecture is designed in a way such that, it supports dynamic evolvability through the generation and integration of new services at run-time in accordance with user goals.

The system adopts key components from the IoT-Together paradigm, including \textit{Goal Management}, \textit{Knowledge Management}, \textit{Context Management}, \textit{Intelligent User Interface (IUI) Generation}, and \textit{Backend Generation}. User interacts with the system using a device (Smart Phone/Laptop/Tablet) through the user interface to enter the query. This query is then passed to the Goal Management, which identifies the set of services that satisfy user goals. The Goal Management uses the LLM to determine the services within the system that can satisfy the user goals or the services that need to be generated. In this phase, the user and system collaboratively (through back-and-forth conversational exchanges) identify the set of parameters and the services required to achieve the user goal.
After the identification of the required services, they are transferred to the Intelligent User Interface Generator to build the application. It locates the services in the Knowledge Management; otherwise, a description of the new service is forwarded to the Backend Generation component, which generates the service, thus extending the architecture runtime according to the user needs. The generated service is sent back to the Intelligent User Interface Generator. The builder within the component integrates these services using a pre-defined template and then creates the website. Intelligent User Interface Generator provides the URL to access this website created. Fig-3 provides the sequence diagram for the process of generation of the application to the user. Meanwhile, the sensors of the IoT environment transmits data periodically to Knowledge Management, which maintains a persistent data repository. This accumulated data is subsequently utilized by the system during the creation of new services and translating the user needs to the goals using the Goal Management.
\\
The User Interface serves as the interaction medium between users and the system, managing conversational exchanges during service identification and preference elicitation. Users submit their initial queries through the interface, and the system engages in a structured conversation to clarify requirements. We represent user queries by $Q_\text{user}$ and system responses by $R_\text{sys}$. For the running example, $Q_\text{user} =$ ``I have 3 hours to explore Hyderabad's old charm" and the $R_\text{sys}$ provides relevant information along with follow-up queries for goal formulation.
Upon successful completion of this dialogue, the interface presents the user with a URL to access the generated web application that implements their specified needs.\\
In the subsequent, following the Fig- \ref{fig:high-level-architecture-overview} we introduce core components and the interactions between them in detail.

\subsection{\textbf{Goal Management}}
User Interface forwards user queries to the Goal Management. The LLM in the Goal Management interprets these queries based on three elements: the summary of previous exchanges, the environmental context from Knowledge Management, and the current state of system services. These conversations are modeled to extract maximum information about user goals and preferences. The LLM engages in an iterative dialogue with users to progressively refine the user goals until all requisite services are identified. This extraction process is carried out using prompt templates in LLMs, each designed to identify specific contextual elements at each pass: environmental parameters in the first pass, goal refinement in the second pass, and service requirements in the final pass. User preferences are then applied as parameters to customize the identified services. To enable this information extraction process, we designed a three-pass conversation structure.

\textbf{Pass 1: Contextual Awareness} 
In this pass, the LLM aggregates contextual information $C$ by combining sensor data $D_{\text{sensor}}$ (weather conditions, noise levels, time of day, day of the week, traffic conditions, and air quality), with user-specific data $D_{\text{user}}$ (current location, past activities, and stated preferences as $\text{params}$). A contextual state $X$ maintains the current interaction state, while service registry data from Knowledge Management provides information about available service capabilities. Each service is represented as below:
\[
S_i = ( \text{name}_i, \text{description}_i, \text{params}_i ), \quad \text{for} \quad i = 1, 2, \dots, n
\]
 Realization of values for few services are shown in Table ~\ref{tab:service-definitions}.
\begin{table}[ht]
\setlength{\tabcolsep}{2pt}
\caption{Service Definitions\protect\footnotemark}
\label{tab:service-definitions}
\centering
\begin{tabular}{lll}
\toprule
\textbf{Service Name} & \textbf{Description} & \textbf{Parameters} \\
\midrule
historical\_info & Provides historical and cultural & site\_name \\
 & information about monuments and sites & \\ \hline
restaurant\_finder & Recommends restaurants based on & location, cuisine, \\
 & cuisine preferences and location & diet \\ \hline
crowd\_monitor & Tracks and reports real-time crowd & location, time \\
 & density at various locations & \\ 
\bottomrule
\end{tabular}
\end{table}
\footnotetext{The values presented are illustrative and not representative of entire data. Definitions may also include associated schema, endpoints, and additional metadata relevant to the service.}
Together these services are used as 
\[
D_{\text{services}} = \{ S_1, S_2, \dots, S_n \}
\]
The complete context is defined as \[C = \{D_{\text{sensor}}, D_{\text{user}}, D_{\text{services}}, X\}\]

For example, when a user visits the city with time constraints and specific interests, such as local cuisine, nature activities, or historical sites, their preferences are stored as contextual information.

In the running example, the system identifies the user as a history enthusiast along with the time constraints and captures it via prompt templates.
\[
D_{\text{user}} = \{
\text{current location}, \text{3hrs time}, \text{history enthusiast}
\}
\]
After establishing the complete context, the next pass proceeds with goal identification. \\
\textbf{Pass 2: Goal Formulation and Refinement}
In this pass, the LLM generates Initial Goal Hypotheses based on contextual insights about the user's potential goals $G_{\text{user}}$. These hypotheses range from broad intentions (e.g., leisure activities) to concrete objectives (e.g., specific venue requirements). The system employs \textbf{Mixed-Initiative Interaction}, establishing a collaborative dialogue between the user and system for goal refinement. This refinement process is internally guided by structured prompts that evaluate hypotheses against available services and contextual constraints.

This interaction comprises two key components, each driven by specialized prompt templates:
\begin{itemize}
    \item \textbf{Proactive Suggestions}: The system suggests possible activities based on initial goal hypotheses and context $C$, using prompts that analyze alignment between user preferences and available services
    \item \textbf{Clarification Dialogues}: The system elicits specific preferences and constraints through targeted queries, using prompts designed to resolve ambiguities and gather missing information
\end{itemize}

For the running example, the Goal Management uses these structured prompts to identify historical information services as a potential match. This is represented as:
\[
S_{\text{hist}} = \{
    \text{name}: \text{historical\_info},
    \text{description},
    \text{params}: \text{site\_name}
\}
\]

where $S_{\text{hist}} \in D_{\text{services}}$ represents the identified service matching the user's interests in $D_{\text{user}}$. \\
\textbf{Pass 3: Goal Verification and Confirmation}
In this final pass, the LLM proposes a curated set of services that align with the inferred goals and contextual factors extracted from the prompt templates gathered from \textbf{Pass 1} and \textbf{Pass 2}. If the user rejects the final proposal, the process reverts to \textbf{Pass 1}.

Continuing the running example, the system first seeks confirmation for the historical information service. Upon user request for additional dining options, the system extends the service set as follows:
\[
G_{\text{users}} = \{ S_{hist},S_{restau}
\}
\]

where $S_{\text{hist}}$ and $S_{\text{restau}}$ follow from Table~\ref{tab:service-definitions} for Historical Information and Restaurant Finder services respectively, with parameters instantiated based on user preferences. In our running example, for the historical information service and restaurant finder, these parameters realize the values based on user conversations with system specifying their requirements and preferences.\\ 
\(
\theta_q = \{
site\_name = 
    \text{[Charminar} \text{, Laad Bazaar]},
\)
\(
location = 
    \text{Laad Bazaar}, 
cuisine = \text{Any},
\)
\(
diet=\text{Non-vegetarian}\}
\)
\\The system maintains a history data base $H$  along with contextual information throughout these passes which include:

\[
H = \{P_{\text{user}}, S_{\text{identified}}, C_{\text{prev}}\}
\]

where $P_{\text{user}}$ represents user preferences, $S_{\text{identified}}$ tracks identified services, and $C_{\text{prev}}$ maintains the contextual history of previous interactions.
\begin{figure*}[ht]
    \centering
    \includegraphics[width=\textwidth]{figures/sequence_research.drawio_5.png}
    \label{sequence}
    \caption{Sequence diagram for creating the web application }
\end{figure*}

\subsection{\textbf{Backend Generation}}

The Backend Generation component activates when the Service Discovery identifies service requirements from the Goal Management that can be fulfilled based on available definitions but are not yet implemented in the current service set $D_{\text{services}}$. 
\begin{algorithm}
\caption{: Service Generation}
\label{alg:service_gen}
\begin{algorithmic}[1] % Ensure sequential line numbering
\State \textbf{Input:} Service requirement query $Q$, System context $S_{\text{ctx}}$
\State \textbf{Output:} Service $S$ (either matched or newly generated)
\\
\Function{ServiceGeneration}{$Q, S_{\text{ctx}}$}
    \State $D_{\text{services}} \gets S_{\text{ctx}}.services$
    
    \State // description refinement phase
    \State $service\_match \gets \text{DescriptionRefiner}(Q, D_{\text{services}})$
    \If{$service\_match \neq \emptyset$}
        \State // existing service satisfies Q
        \State \Return $service\_match$
    \EndIf
    
    \State // service generation phase
    \State $D_{\text{schema}} \gets S_{\text{ctx}}.schemas$
    \State $Q_{\text{gen}} \gets \text{RefineGenerationQuery}(Q, D_{\text{schema}})$
    \State $S_{\text{new}} \gets \text{GenerateService}(Q_{\text{gen}})$
    \State $D_{\text{services}} \gets D_{\text{services}} \cup \{S_{\text{new}}\}$
    \State $\text{UpdateServiceContext}(S_{\text{ctx}}, S_{\text{new}})$
    \State \Return $S_{\text{new}}$
\EndFunction
\end{algorithmic}
\end{algorithm}
An algorithm to achieve this is described in Service Generation (Algorithm-\ref{alg:service_gen}) with components described in Table-\ref{tab:service-gen-components}. The generation process initiates by creating a service requirement query:
\[
Q = \{
    type: \text{service\_req},
    params: \theta_q
\}
\]
where $\theta_q$ represents the required service parameters  identified through the dialogue exchange with the Goal Parser. \\

The Service Manager maintains the system context $S_{\text{ctx}}$ comprising:

\[
S_{\text{ctx}} = \{
    D_{\text{services}},
    D_{\text{schema}},
    D_{\text{config}}
\}
\]

where $D_{\text{schema}}$ represents available database schemas and $D_{\text{config}}$ contains service configurations. The description of the services assist the Description Refiner -- an LLM Agent -- to evaluate incoming query $Q$ against $S_{\text{ctx}}$ through the function $f_{\text{match}}$:

\[
f_{\text{match}}(Q, S_{\text{ctx}}) = 
\begin{cases}
    S_i, & \text{if } \exists S_i \in D_{\text{services}} \\
    Q_{\text{gen}}, & \text{otherwise}
\end{cases}
\]


where $Q_{\text{gen}}$ represents the refined service generation query which is achieved using RefineGenerationQuery function (line-14 of Algorithm \ref{alg:service_gen}) which would be passed to the Service Generator (line-15), another LLM agent with coding capabilities. This would contain instructions to write code, with relevant database information (their schema, location etc.) required to generate these services. 

\begin{table}[h]
\setlength{\tabcolsep}{3pt}
\caption{Algorithm-\ref{alg:service_gen} Components executed by LLM agents}
\label{tab:service-gen-components}
\centering
\begin{tabular}{ll}
\toprule
\textbf{Function} & \textbf{Description} \\
\midrule
RefineGenerationQuery & Examines D$_{schema}$ relevant to query Q to \\
& prepare LLM coding prompt Q$_{gen}$ \\ \hline
GenerateService & Processes Q$_{gen}$ to generate and deploy S$_{new}$ \\ \hline
UpdateServiceContext & Updates S$_{ctx}$ with S$_{new}$ and adds to D$_{services}$ \\
\bottomrule
\end{tabular}
\end{table}
For example, extending our previous scenario, if a user requests crowd monitoring at historical sites:

\[
S_{\text{new}} = \{
    \text{name}: \text{crowd\_monitor}, \text{ } \theta_c
\}
\]
\[
\text{where } \theta_c = \{
    \text{loc}: \text{hist\_site},
    \text{time}: \text{now}
\}
\]


If $S_{\text{new}} \notin D_{\text{services}}$ but $D_{\text{schema}}$ contains relevant data structures for the monitoring of crowd density via the sensors and endpoints of these databases, the Service Generator creates and integrates the new service:


\[
D_{\text{services}} = D_{\text{services}} \cup \{S_{\text{new}}\}
\]

Upon the generation of the new service $S_{new}$ an entry is added describing its function and the parameters it expects  (Table-\ref{tab:service-definitions} entry for crowd monitor service).

    
    
 After the successful service generation, UpdateServiceContext algorithm \ref{alg:service_gen} signals the Service Manager
 to incorporate the new service into its context state, triggering
 necessary updates to both Context Management and its
 service registry in the Knowledge Management. The newly generated service $S_{new}$ is further transmitted to the Builder component for building the website.


\subsection{\textbf{Intelligent User Interface Generator}}

The Intelligent User Interface Generator transforms the identified user goals into a functional web application through three primary components: Service Discovery, Builder, and Hosting. 

\[
G_{user} = \{SI_1, SI_2, ..., SI_n\}
\]

$G_{user}$ represents the set of services identified from the goal parser phase, where each $SI_i$ represents a service with its parameters and requirements, where each $\text{SI}_i$ represents an identified service. The Service Discovery component matches between $G_{user}$ and available services in the service registry. For each service $SI_i \in G_{user}$, it searches for matches $M_i \subseteq D_{services}$ by evaluating service descriptions and parameter compatibility:



\[
M = \bigcup_{i=1}^{n} M_i,\]\[ \quad M_i = \{ S_j \mid \text{match}(\text{description}_{\text{SI}_i}, \text{description}_{S_j}) \]
\[\land \text{params}_{\text{SI}_i} \subseteq \text{params}_{S_j} \}
\]

where $\text{match}()$ represents the semantic matching function which compares the service descriptions. All the matches are forwarded to the builder for application generation. When $M_i = \emptyset$ for any $SI_i$, indicating that no matching service exists, the system initiates service generation. The Backend Generation creates a new service $S_{new}$ using the service description and available data schemas and returns it to the Builder.


Builder implements UI generation process which employs a hierarchical rendering system $R$ with specific handlers for different data types, some of which are listed below:

\[
R(d) = \begin{cases}
    R_{\text{metric}}(d), & \text{if } d \in \mathbb{R} \\
    R_{\text{list}}(d), & \text{if } d \in \text{List} \\
    R_{\text{dict}}(d), & \text{if } d \in \text{Dict} \\
    R_{\text{text}}(d), & \text{otherwise}
\end{cases}
\]

where each renderer implements specific visualization logic for its data type (e.g., $R_{\text{metric}}$ for crowd density, $R_{\text{list}}$ for restaurant listings). The Builder Component generates the complete application by combining these renderers with a base template $T$, where $\oplus$ represents the composition operator that combines multiple rendered components:



\[
App = T(M \cup \{S_{new}\}) \oplus \bigoplus_{i=1}^{n} R(SI_i)
\]












The final application is hosted through the Hosting Component, which manages service endpoints and provides URL access to users. 




\subsection{\textbf{Context Management}}
It hosts all the active services in the system. These services are built on top of the sensor data in Knowledge Management, which stores the data of the IoT environment (e.g., routes service, crowd monitoring, and air quality sensors). The configurations for these services are maintained in Knowledge Management, which is updated if the underlying data changes.

\subsection{\textbf{Knowledge Management}}
\begin{table*}[t]
    \caption{Comparison of student model performance across benchmarks under different scenarios: no attack (Trained SM), UP, TP and WN. Values in () indicate percentage changes relative to Trained SM, with the highest performance in each setting bolded and underlined. Student model used in this table is Llama-7b, results for Llama-3.2-1b can be found in Appendix \ref{sec:knowledge_llama3_2}.}
    \centering
    \resizebox{0.98\textwidth}{!}{
    \begin{tabular}{cccccccc}
        \toprule
        \textbf{Benchmark} &
        \textbf{Ori. SM} & \multicolumn{2}{c}{\textbf{Wat. Scheme}} & \textbf{Trained SM} & \textbf{Trained SM + UP} & \textbf{Trained SM + TP} &\textbf{Trained SM + WN} \\
        \midrule
        \multirow{6}{*}{\makecell{ARC \\ Challenge \\ (ACC)}} & \multirow{6}{*}{0.4181} & \multirow{3}{*}{KGW} & $n=1$ & 0.4480 & 0.4215 \textcolor[rgb]{0.5,0.0,0.0}{(-5.9\%)} & 0.3951 \textcolor[rgb]{0.5,0.0,0.0}{(-11.8\%)}& \underline{\textbf{0.4497}} \textcolor[rgb]{0.0,0.5,0.0}{(+0.6\%)}\\
        & & & $n=2$ & \underline{\textbf{0.4404}} & 0.4283 \textcolor[rgb]{0.5,0.0,0.0}{(-2.7\%)}& 0.4104 \textcolor[rgb]{0.5,0.0,0.0}{(-6.8\%)}& 0.4369 \textcolor[rgb]{0.5,0.0,0.0}{(-0.8\%)}\\
        & & & $n=3$ & \underline{\textbf{0.4778}} & 0.3865 \textcolor[rgb]{0.5,0.0,0.0}{(-19.1\%)}& 0.3840 \textcolor[rgb]{0.5,0.0,0.0}{(-19.6\%)}& 0.4642 \textcolor[rgb]{0.5,0.0,0.0}{(-2.8\%)}\\
        \cmidrule{3-8}
        & & \multirow{3}{*}{\makecell{SynthID\\ -Text}} & $n=1$ & 0.4505 & 0.4394 \textcolor[rgb]{0.5,0.0,0.0}{(-2.5\%)} & 0.4198 \textcolor[rgb]{0.5,0.0,0.0}{(-6.8\%)}& \underline{\textbf{0.4548}} \textcolor[rgb]{0.0,0.5,0.0}{(+1.0\%)} \\
        & & & $n=2$ & 0.4360 & 0.4403 \textcolor[rgb]{0.0,0.5,0.0}{(+1.0\%)}& 0.4241 \textcolor[rgb]{0.5,0.0,0.0}{(-2.7\%)}& \underline{\textbf{0.4565}} \textcolor[rgb]{0.0,0.5,0.0}{(+4.7\%)}\\
        & & & $n=3$ & \underline{\textbf{0.4505}} & 0.4394 \textcolor[rgb]{0.5,0.0,0.0}{(-2.5\%)}& 0.4283 \textcolor[rgb]{0.5,0.0,0.0}{(-4.9\%)}& 0.4471 \textcolor[rgb]{0.5,0.0,0.0}{(-0.8\%)}\\

        \midrule
        \multirow{6}{*}{\makecell{TruthfulQA \\ Multiple Choice \\ (ACC)}} & \multirow{6}{*}{0.3407} & \multirow{3}{*}{KGW} & $n=1$ & 0.3884 & 0.3917 \textcolor[rgb]{0.0,0.5,0.0}{(+0.8\%)}& 0.3785 \textcolor[rgb]{0.5,0.0,0.0}{(-2.5\%)} & \underline{\textbf{0.4186}} \textcolor[rgb]{0.0,0.5,0.0}{(+7.8\%)}\\
        & & & $n=2$ & \underline{\textbf{0.4376}} & 0.4097 \textcolor[rgb]{0.5,0.0,0.0}{(-6.4\%)} & 0.4089 \textcolor[rgb]{0.5,0.0,0.0}{(-6.6\%)}& 0.4353 \textcolor[rgb]{0.5,0.0,0.0}{(-0.5\%)}\\
        & & & $n=3$ & 0.4459 & 0.4315 \textcolor[rgb]{0.5,0.0,0.0}{(-3.2\%)}& 0.4055 \textcolor[rgb]{0.5,0.0,0.0}{(-9.1\%)}& \underline{\textbf{0.4632}} \textcolor[rgb]{0.0,0.5,0.0}{(+3.9\%)}\\
        \cmidrule{3-8}
        & & \multirow{3}{*}{\makecell{SynthID\\ -Text}} & $n=1$ & 0.4063 & 0.3780 \textcolor[rgb]{0.5,0.0,0.0}{(-7.0\%)} & 0.3597 \textcolor[rgb]{0.5,0.0,0.0}{(-11.5\%)} & \underline{\textbf{0.4262}} \textcolor[rgb]{0.0,0.5,0.0}{(+4.9\%)} \\
        & & & $n=2$ & 0.3991 & 0.3965 \textcolor[rgb]{0.5,0.0,0.0}{(-0.7\%)}& 0.4043 \textcolor[rgb]{0.0,0.5,0.0}{(+1.3\%)} & \underline{\textbf{0.4281}} \textcolor[rgb]{0.0,0.5,0.0}{(+7.3\%)}\\
        & & & $n=3$ & 0.4102 & 0.4009 \textcolor[rgb]{0.5,0.0,0.0}{(-2.3\%)}& 0.4062 \textcolor[rgb]{0.5,0.0,0.0}{(-1.0\%)}& \underline{\textbf{0.4330}} \textcolor[rgb]{0.0,0.5,0.0}{(+5.3\%)}\\
        
        \midrule
        \multirow{6}{*}{\makecell{MTBench \\ (Full Score: 10)}} & \multirow{6}{*}{2.64} & \multirow{3}{*}{KGW} & $n=1$ & \underline{\textbf{3.86}} & 3.04 \textcolor[rgb]{0.5,0.0,0.0}{(-21.2\%)}& 2.76 \textcolor[rgb]{0.5,0.0,0.0}{(-28.5\%)} & 3.67 \textcolor[rgb]{0.5,0.0,0.0}{(-4.9\%)}\\
        & & & $n=2$ & 3.99 & 3.40 \textcolor[rgb]{0.5,0.0,0.0}{(-14.8\%)}& 2.94 \textcolor[rgb]{0.5,0.0,0.0}{(-26.3\%)}& \underline{\textbf{4.02}} \textcolor[rgb]{0.0,0.5,0.0}{(+0.7\%)}\\
        & & & $n=3$ & \underline{\textbf{4.11}} & 3.27 \textcolor[rgb]{0.5,0.0,0.0}{(-20.4\%)}& 3.04 \textcolor[rgb]{0.5,0.0,0.0}{(-26.0\%)}& 3.99 \textcolor[rgb]{0.5,0.0,0.0}{(-2.9\%)}\\
        \cmidrule{3-8}
        & & \multirow{3}{*}{\makecell{SynthID\\ -Text}} & $n=1$ & \underline{\textbf{4.14}} & 3.27 \textcolor[rgb]{0.5,0.0,0.0}{(-21.0\%)}& 2.01 \textcolor[rgb]{0.5,0.0,0.0}{(-51.4\%)}& 4.13 \textcolor[rgb]{0.5,0.0,0.0}{(-0.2\%)} \\
        & & & $n=2$ & \underline{\textbf{4.24}} & 3.05 \textcolor[rgb]{0.5,0.0,0.0}{(-28.1\%)} & 2.84 \textcolor[rgb]{0.5,0.0,0.0}{(-33.0\%)} & 4.12 \textcolor[rgb]{0.5,0.0,0.0}{(-2.8\%)}\\
        & & & $n=3$ & \underline{\textbf{4.24}} & 2.90 \textcolor[rgb]{0.5,0.0,0.0}{(-31.6\%)}& 2.69 \textcolor[rgb]{0.5,0.0,0.0}{(-36.6\%)}& 4.16 \textcolor[rgb]{0.5,0.0,0.0}{(-1.9\%)}\\
        \bottomrule
    \end{tabular}
    }
    \label{tab:knowledge}
\end{table*}


\section{Experiments and Results}

The evaluation focuses on assessing the proposed approach through the following research questions.

\begin{itemize}
    \item \textbf{RQ1: Effectiveness in Identifying Functionalities:}  
    How effective is the approach in identifying the correct set of functionalities corresponding to existing components ?
    
    \item \textbf{RQ2: Accuracy in Dynamic Service Generation:}  
    How accurate is the approach in dynamically generating services?  

    \item \textbf{RQ3: Effectiveness in System Generation:}  
    What is the effectiveness of the approach in generating the system as a whole?

    \item \textbf{RQ4: Efficiency in Application Generation:}  
    What is the efficiency of the approach in generating applications?
\end{itemize}
\subsection{Evaluation Setup} We evaluated our system using Hyderabad as a case study of a smart city implementation. The system comprises $9$ web-services: Air Quality, Crowd Monitoring, Event Notifier, Historic Information, Restaurant Locator, Travel Options, Water Quality, Exhibition Tracker, and Event Ticket Vendor. Several services operate on static contextual data feeds (e.g., Historic Information), while others process real-time data from a network of 12 distributed sensors. We implemented the IoT environment simulation using CupCarbon \footnotemark for realizing the architecture and generating sensor data based on domain-appropriate statistical distributions, with the core system in Python. we employed a two-fold evaluation strategy: (i) a multi-agent simulation framework modeling Tourist-Guide interactions across 100 experimental runs, and (ii) a user study ($n=15$) focusing on real-world usability and service adaptation quality.
\footnotetext{CupCarbon: \url{https://cupcarbon.com/}}
 \subsubsection{Tourist-Guide Simulations}
The system evaluation employed OpenAI's GPT-4o-mini \cite{openai2024gpt4technicalreport}, DeepSeek-V2.5 \cite{deepseekai2024deepseekv2strongeconomicalefficient} and CodeQwen1.5-7B \cite{bai2023qwentechnicalreport} models through the LangChain framework. We chose these based on the EvalPlus \cite{liu2023is} leader board. Experiments were conducted on an Nvidia L40S GPU with 8 vCPU, 62 GB RAM, and 48GB VRAM for hosting the CodeQwen1.5-7B model on HuggingFace. For GPT-4o-mini and DeepSeek-V2.5 model interactions, we utilized LangChain's OpenAI API. All these models were run with a temperature parameter of 0.7 based on preliminary experimentation.  
To evaluate the Goal Management, we designed a multi-agent simulation framework using CrewAI\footnotemark, modeling interactions between a Tourist agent and a Travel Guide agent (implementing our Goal Management's instruction set). The Tourist agent samples from $25$ predefined goals, generated through prompt engineering with domain-specific system knowledge, with time constraints uniformly distributed between $1-5$ hours. Each goal has an associated ground truth set of required services for validation.
\footnotetext{CrewAI: \url{https://www.crewai.com/}}
The goals were classified into concrete and ambiguous categories ($18:7$). Concrete goals have predictable mappings, such as ``Planning to visit Ramoji Film City'' mapping to \textit{ticket\_purchase} and \textit{travel\_options}. Ambiguous goals like ``First time in Hyderabad! Want to start with the locals' favorites'' may trigger multiple services (e.g., \textit{restaurant\_finder}, \textit{crowd\_monitor}, \textit{travel\_options}) based on conversation flow.
The simulation involved three sequential passes of Tourist-Guide interactions for service identification, repeated 100 times. While additional services could enhance user experience, we limit suggestions to avoid overwhelming users with options beyond their original goal.

\subsubsection{User Evaluation}
For complementing our simulation-based evaluation, we conducted a user study with students from the International Institute of Information Technology, Hyderabad (IIIT-H), which focused on understanding system effectiveness and overall user satisfaction through both quantitative metrics and qualitative feedback. The study involved participants ($n=15$) from diverse academic backgrounds within IIIT-H, specifically comprising 3 Ph.D. students (2 Computer Science and 1 Electronics/Communications Engineering), 5 Electronics/Communications Engineering students (B.Tech by M.S.), and 7 Computer Science students (B.Tech by M.S.). The participants were given a brief overview of the system's capabilities and were encouraged to interact with it based on their interests and needs. Each participant interacted with the system for approximately 10-15 minutes, with feedback collected through an integrated form in the user interface. The feedback mechanism collected three types of Quantitative Metrics: application rating, service accuracy rating, and service relevance rating (on a 1-5 likert scale), and Qualitative Feedback comprising query summaries, missing service identification, unnecessary service identification, and improvement suggestions. To further evaluate the effectiveness of dynamically generated services, we implemented a service rotation mechanism where three services were deliberately kept offline and replaced with generated implementations during each participant interaction, without informing participants, to assess integration seamlessness.







    
    



For assessing application generation efficiency, we integrated a metrics collection system with the Intelligent User Interface generator. The evaluation examined three critical metrics: total generation time (comprising dialogue latency, service discovery, template rendering, and deployment), token usage (aggregating input tokens from user queries, processing tokens from system context, and completion tokens from LLM responses), and build times per session. Given that service generation is not activated in every test scenario, we conducted a separate performance analysis of this component to ensure unbiased assessment.














\subsection{Results \& Discussions}
\noindent \textbf{RQ1: Effectiveness in Identifying Functionalities}


To evaluate the Goal Management's effectiveness, we analyze service identification accuracy using four key metrics defined in Table-\ref{tab:metrics-definition} in our simulation. The evaluation compares services identified after the third conversation pass against ground truth mappings derived from our tourism domain requirements.

\begin{table}[ht]
\setlength{\tabcolsep}{2pt}
\caption{Evaluation Metrics for Tourist-Guide simulation}
\label{tab:metrics-definition}
\centering
\begin{tabular}{l|p{7.5cm}}
\toprule
\textbf{Metric} & \textbf{Definition} \\
\midrule
Precision (P) & Ratio of correctly identified services to all identified services \\
\hline
Recall (R) & Ratio of correctly identified services to actual required services \\
\hline
F1 Score & Harmonic mean of precision and recall (2PR/(P+R)) \\
\hline
Parameter & Accuracy of identified service parameters (e.g., exact \\
Accuracy & locations, cuisines) against ground truth \\
\bottomrule
\end{tabular}
\end{table}

Analysis of the simulation results presented in Table-\ref{tab:goal-parser-categories} demonstrates comparable performance metrics between GPT-4o-mini and DeepSeek-V2.5 in service identification tasks. We noticed that both GPT-4o-mini and DeepSeek-V2.5 consistently respected time constraints while providing travel plans to the Tourist unlike CodeQwen1.5-7B which suggested plans spanning multiple days, exceeding the specified time constraints. CodeQwen1.5-7B exhibits lower precision values, displaying a tendency toward over-identification of required services. This over-identification introduces unnecessary complexity into the system architecture and imposes increased computational overhead during the build process.
\begin{table}[!htbp]
\setlength{\tabcolsep}{4pt}
\caption{Goal Parser Performance by Category}
\label{tab:goal-parser-categories}
\centering
\begin{tabular}{llcccc}
\toprule
\textbf{Model} & \textbf{Category} & \textbf{Precision} & \textbf{Recall} & \textbf{F1} & \textbf{Parameter} \\
& & & & & \textbf{Accuracy} \\
\midrule
CodeQwen1.5-7B & Ambiguous & 0.450 & 0.806 & 0.553 & 0.116 \\
& Concrete & 0.206 & 0.609 & 0.288 & 0.051 \\
& \textbf{Overall} & 0.282 & 0.670 & 0.370 & 0.071 \\
\midrule
GPT-4o-mini & Ambiguous & 0.683 & 0.795 & 0.730 & 0.549 \\
& Concrete & 0.467 & 0.773 & 0.559 & 0.739 \\
& \textbf{Overall} & 0.523 & 0.778 & 0.603 & 0.690 \\
\midrule
DeepSeek-V2.5 & Ambiguous & 0.681 & 0.788 & 0.725 & 0.585 \\
& Concrete & 0.492 & 0.830 & 0.591 & 0.743 \\
& \textbf{Overall} & 0.554 & 0.816 & 0.635 & 0.691 \\
\bottomrule
\end{tabular}
\end{table}

For the user evaluation (see Table-\ref{tab:user-satisfaction}) study, tourism-focused ($40\%$) and dining-related ($53\%$) queries dominated user sessions, with $67\%$ involving multi-service combinations. Restaurant Finder ($53\%$), Travel Options ($47\%$), and Historical Information ($33\%$) were the most frequently requested services. User feedback identified crowd monitoring (7 instances), air quality (3), and water quality (2) as desired additional services.
\begin{table}[ht]
\setlength{\tabcolsep}{4pt}
\caption{User Satisfaction Metrics}
\label{tab:user-satisfaction}
\centering
\begin{tabular}{lccc}
\toprule
\textbf{Metric} & \textbf{Average Rating (out of 5)} & & \\
\midrule
Application Rating & 4.0 & & \\
Accuracy Rating & 4.1 & & \\
Relevance Rating & 4.2 & & \\
\bottomrule
\end{tabular}
\end{table}
User studies highlighted the need for improved local data processing, particularly for proximity-based routing and recommendations. These insights suggest optimization areas aligning with our mixed-initiative vision, especially in collaborative monitoring and user-adaptive location services.

\smallskip
\noindent \textbf{RQ2: Accuracy in Dynamic Service Generation}

To evaluate the quality of dynamically generated services, we conducted multiple generation attempts (three per service) across our 9 services. We used CodeBERTScore \cite{zhou2023codebertscoreevaluatingcodegeneration} to assess the semantic similarity between generated and reference implementations, measuring four key aspects: precision (code correctness), recall (code completeness), F1-score (balanced measure), and F3-score (emphasizing on code completeness).

\begin{table}[ht]
\setlength{\tabcolsep}{4pt}
\caption{Service Generation Code Similarity}
\label{tab:code-similarity}
\centering
\begin{tabular}{lcccc}
\toprule
\textbf{Model} & \textbf{Precision} & \textbf{Recall} & \textbf{F1} & \textbf{F3} \\
\midrule
CodeQwen1.5-7B & 0.86 ± 0.02 & 0.79 ± 0.03 & 0.83 ± 0.02 & 0.80 ± 0.03 \\
DeepSeek-V2.5 & 0.91 ± 0.01 & 0.85 ± 0.03 & 0.88 ± 0.02 & 0.86 ± 0.03 \\
GPT-4o-mini & 0.90 ± 0.01 & 0.85 ± 0.03 & 0.87 ± 0.01 & 0.85 ± 0.02 \\
\bottomrule
\end{tabular}
\end{table}
As shown in Table \ref{tab:code-similarity}, DeepSeek-V2.5 achieved the highest overall performance with an F1-score of 0.88, outperforming CodeQwen1.5-7B by 6\% and comparable to GPT-4o-mini. Notably, all models maintained high precision (\(\ge\) 0.86), indicating reliable code generation quality. The relatively lower recall scores, particularly for CodeQwen1.5-7B (0.79), suggest occasional omissions in implementing complete functionality. These contrasting results from Table-\ref{tab:goal-parser-categories} suggest that while DeepSeek-V2.5 and GPT-4o-mini exhibit consistent performance across both service identification and code generation tasks, CodeQwen1.5-7B shows task-specific performance variations that could impact its suitability for general-purpose service generation in IoT environments. 

\smallskip
\noindent \textbf{RQ3: Effectiveness in System Generation}\\
We keep track of the total tokens required to generate these services across the models along with end-to-end latency
(including API request/response time) in Table-\ref{tab:service-generation}. \\
\begin{table}[!htb]
\setlength{\tabcolsep}{4pt}
\caption{Service Generation }
\label{tab:service-generation}
\centering
\begin{tabular}{lcc}
\toprule
\textbf{Model} & \textbf{Time (s)} & \textbf{Tokens} \\
\midrule
CodeQwen1.5-7B & 7.67 ± 0.26 & 3482.00 ± 19.80 \\
DeepSeek-V2.5 & 42.40 ± 4.52 & 4376.25 ± 228.17 \\
GPT-4o-mini & 25.66 ± 2.55 & 2063.17 ± 191.76 \\
\bottomrule
\end{tabular}
\end{table}
While GPT-4o-mini and DeepSeek-V2.5 achieved 100\% service generation success rate, CodeQwen1.5-7B only succeeded in 37\% of attempts. On inspection, we found that CodeQwen1.5-7B's performance limitations stemmed from (1) inconsistent instruction following and (2) JSON formatting errors.

\begin{figure}[!htb]
    \centering
    \includegraphics[width=0.8\linewidth]{figures/input_tokens.png}
    \caption{Token consumption scaling with increasing number of services for GPT-4o-mini and DeepSeek-V2.5. The x-axis represents the number of services and the y-axis shows the corresponding input token count.}
    \label{fig:token-scaling}
\end{figure}

To evaluate scalability in mixed-initiative contexts, we analyze how the input token consumption scales with an increasing number of services in the system. Figure~\ref{fig:token-scaling} illustrates this relationship across GPT and DeepSeek model, where the token usage pattern diverges significantly even for small number of services. This rise is primarily seen due to the addition of extra services leading to the Description Refiner requiring more tokens to process the system state received by the Service Manager.
While DeepSeek-V2.5 and GPT-4o-mini have comparable pricing (0.14 USD and 0.15 USD per 1M input tokens)\footnotemark, their actual costs differ due to variations in token consumption, directly impacting adaptive service generation costs. GPT-4o-mini, with its relatively compact architecture and lower token usage (Table-\ref{tab:service-generation}), demonstrates more efficient performance for dynamic interactions compared to DeepSeek-V2.5's 238 billion parameter architecture.
\footnotetext{As of 2024-12-10 on billing websites}
\smallskip

\noindent \textbf{Results for RQ4: Efficiency in Application Generation}
We evaluated the system using 15 scenarios from our human evaluation study.\subsection{Performance Analysis}
The system achieved an average total generation time of 23.10 ± 6.47 seconds. Table~\ref{tab:generation-breakdown} presents the detailed breakdown of the processing stages.
\begin{table}[ht]
\setlength{\tabcolsep}{4pt}
\caption{Generation Process Time Breakdown}
\label{tab:generation-breakdown}
\centering
\begin{tabular}{lcc}
\toprule
\textbf{Processing Stage} & \textbf{Mean (s)} & \textbf{SD (s)} \\
\midrule
Multi-pass conversation processing & 18.25 & 5.12 \\
Service identification \& parameter extraction & 3.82 & 1.14 \\
Template rendering \& application assembly & 1.03 & 0.31 \\
Final deployment & 0.004 & 0.002 \\
\bottomrule
\end{tabular}
\end{table}

Analysis of token distribution is presented in Table~\ref{tab:token-distribution}.

\begin{table}[ht]
\setlength{\tabcolsep}{4pt}
\caption{Token Usage Distribution}
\label{tab:token-distribution}
\centering
\begin{tabular}{lccc}
\toprule
\textbf{Token Type} & \textbf{Count (Mean ± SD)} & \textbf{\% of Total} \\
\midrule
Input tokens & 101.8 ± 70.12 & 1.25\% \\
Processing tokens & 7,308.1 ± 2,607.49 & 89.51\% \\
Completion tokens & 755.0 ± 281.28 & 9.24\% \\
\bottomrule
\end{tabular}
\end{table}

The system demonstrated decent build performance, with an average build time of 4.85 ± 1.98 milliseconds. This sub-10 ms build time was anticipated, as the builder only needs to render the application by sending it to the hosting component. Table~\ref{tab:app-generation} summarizes the overall performance metrics.

\begin{table}[ht]
\setlength{\tabcolsep}{4pt}
\caption{Application Generation Performance Metrics}
\label{tab:app-generation}
\centering
\begin{tabular}{lccc}
\toprule
\textbf{Metric} & \textbf{Mean ± SD} & \textbf{Min} & \textbf{Max} \\
\midrule
Total Duration (s) & 23.10 ± 6.47 & 13.46 & 33.08 \\
Total Token Usage & 8164.90 ± 2718.89 & 5531 & 13991 \\
Build Time (ms) & 4.85 ± 1.98 & 3.50 & 10.49 \\
\bottomrule
\end{tabular}
\end{table}

\begin{figure}[!htb]
    \centering
    \includegraphics[width=0.8\linewidth]{figures/token_distribution.png}
    \caption{Token distribution during application generation. The high proportion of processing tokens (89.51\%) indicates potential for optimization through context management improvements.}
    \label{fig:token-distribution}
\end{figure}

\subsection{Service Generation Analysis}

The Backend generation component was evaluated by generating each of our nine services ten times. The evaluation revealed consistent results across different service types, with an average generation time of 15.53 seconds. The metrics are summarized in  Table~\ref{tab:service-metrics}.

\begin{table}[ht]
\setlength{\tabcolsep}{4pt}
\caption{Service Generation Performance Metrics}
\label{tab:service-metrics}
\centering
\begin{tabular}{lc}
\toprule
\textbf{Metric} & \textbf{Value} \\
\midrule
Average processing time (s) & $15.53 \pm 1.12$ \\
Total token usage & $4,992.89 \pm 180.29$ \\
Coefficient of Variation (\%) & $3.61$ \\
\bottomrule
\end{tabular}
\end{table}


The Coefficient of Variation (CV), calculated as (Standard Deviation / Mean) × 100, measures dispersion across different metrics. A CV of 3.61\% indicates high consistency in generating any service regardless of its type. When service generation is incorporated into the total application generation metrics, we observe a total duration of 38.63 seconds (23.10 ± 6.47 + 15.53 ± 1.12) and total token usage of 13,157.79 tokens (8,164.90 ± 2,718.89 + 4,992.89 ± 180.29).




\renewcommand{\thefootnote}{} % Suppress footnote numbering
\footnotetext{Code available on GitHub: \url{https://github.com/sa4s-serc/SAS_llm_query/tree/iot-prototype}}
\renewcommand{\thefootnote}{\arabic{footnote}} % Restore footnote numbering

Our findings challenge the conjecture that code-comment coherence, as measured by SIDE \cite{mastropaolo2024evaluating}, is a critical quality attribute for filtering instances of code summarization datasets. By selecting $\langle code, summary \rangle$ pairs with high-coherence for training allow to achieve the same results that would be achieved by randomly selecting such a number of instances. At the same time, we observed that reducing the datasets size up to 50\% of the training instances does not significantly affect the effectiveness of the models, even when the instances are randomly selected. These results have several implications.

First, code-comment consistency might not be a problem in state-of-the-art datasets in the first place, as also suggested in the results of RQ$_0$. Also, the DL models we adopted (and, probably, bigger models as well) are not affected by inconsistent code-comment pairs, even when these inconsistencies are present in the training set.
Despite the theoretical benefits of filtering by SIDE \cite{mastropaolo2024evaluating}, that is the state-of-the-art metric for measuring code-comment alignment, our results indicate its limitations in improving the \textit{overall} quality of the training sets for code summarization task.
Nevertheless, other quality aspects of code and comments that have not been explored yet (such as readability) may be important for smartly selecting the training instances.
Future work should explore such quality aspects further.

Our results clearly indicate that state-of-the-art datasets contain instances that do not contribute to improving the models' effectiveness. This finding is related to a general phenomenon observed in Machine Learning and Deep Learning. Models reach convergence when they are trained for a certain amount of time (epochs). Additional training provides smaller improvements and increases the risk of overfitting. We show that the same is true for data. In terms of effectiveness, model convergence is achieved with fewer training instances than previously assumed. Limiting the number of epochs may make it possible to reach model convergence with a subset of training data, maintaining model effectiveness, reducing resource demands and minimizing the risk of overfitting.
Future work could explore different criteria for data selection that identify the most informative subsets for training.
Conversely, this insight suggests that currently available datasets suffer from poor diversity (thus causing the previously discussed phenomenon).
This latter insight constitutes a clear warning for researchers interested in building code summarization datasets, which should include instances that add relevant information instead of adding more data, which might turn out to be useless.

Finally, it is worth pointing out that another benefit of the reduction we performed is the environmental impact. Reducing the number of training instances implies a reduced training time, which, in turn, lowers the resources necessary to perform training and, thus, energy consumption and CO$_2$ emissions.
We performed a rough estimation of the training time across different selections of \textit{TL-CodeSum} and \textit{Funcom} datasets and estimated a proxy of the CO$_2$ emissions for each model training phase by relying on the ML CO$_2$ impact calculator\footnote{\url{https://mlco2.github.io/impact/\#compute}} \cite{lacoste2019quantifying}. Such a calculator considers factors such as the total training time, the infrastructure used, the carbon efficiency, and the amount of carbon offset purchased. The estimation of CO$_{2}$ emissions needed to train the model with the \textit{Full} selection of \textit{Funcom} ($\sim$ 200 hours) is equal to 26.05 Kg, while with the optimized training set, \ie $SIDE_{0.9}$ ($\sim$ 90 hours), the estimation is 11.69 Kg of $CO_2$ (-55\% emissions).
While we recognize that this method provides an estimation rather than a precise measurement, it offers a glimpse into the environmental impact of applying data reduction.


\section{Threats to Validity}





\textbf{External Validity:} The Goal Management evaluation faces generalization constraints with a limited dataset of 25 predefined goals in the tourism domain. The implementation's focus on Hyderabad's $9$ specific services may restrict generalizability to cities with different infrastructure requirements. While all the of participants showed willingness for future use, the student-only participant pool and short interaction duration (10-15 minutes) limit comprehensive understanding of long-term usage patterns.

\textbf{Internal Validity:} The uniform temperature setting (0.7) across models might not represent optimal individual configurations. CodeBERTScore, justified by standardized service generation, may not fully capture semantic code differences. The fixed three-pass conversation system could potentially miss interaction patterns affecting service discovery accuracy. The service rotation mechanism provides insights but may not represent all production environment failure modes.

\textbf{Construction Validity:} Our service identification approach uses precision and recall metrics against predefined mappings, which may not fully capture user preference variations. The metrics might not account for additional beneficial services or context-specific requirements. While the human evaluation study \((n=15)\) provides real-world validation, its small sample size limits generalizability.


\section{Related Work}
\section{Related Work}
\label{sec:related}
FL has emerged as a crucial learning scheme for distributed training that aims to preserve user data privacy. 
However, research has uncovered various privacy vulnerabilities in FL, particularly in the form of MIA and LDIA.
This section discusses relevant works that highlight these threats in both FL and FD settings, and contextualize our research within this landscape.

\BfPara{MIA and LDIA} 
Shokri \etal~\cite{shokri2017membership} pioneered MIA research by demonstrating how model output confidence scores could reveal training data membership. 
Nasr \etal~\cite{nasr2019comprehensive} extended this to FL, showing how both passive and active adversaries could exploit gradients and model updates.
LDIA represents another significant privacy threat in FL.
Gu \etal~\cite{gu2023ldia} introduced LDIA as a new attack vector where adversaries infer label distributions from model updates. Wainakh \etal~\cite{wainakh2021user} further explored user-level label leakage through gradient-based attacks in FL.
Recent works have exposed the vulnerability of FD to inference attacks.
Yang \etal~\cite{yang2022fd} proposed FD-Leaks for performing MIA in FD settings through logit analysis. Liu \etal~\cite{liu2023mia} and Wang \etal~\cite{wang2024graddiff} enhanced MIA using shadow models via respective approaches MIA-FedDL and GradDiff, though their assumptions were limited to homogeneous environments.

\iffalse
The study of MIA in ML models was pioneered by Shokri \etal~\cite{shokri2017membership}.
They demonstrated how adversaries could exploit model output confidence scores to infer whether a specific data point was used in training.
This seminal work laid the foundation for subsequent research into privacy vulnerabilities in various ML paradigms, including FL.
Building on this, Nasr \etal~\cite{nasr2019comprehensive} extended the concept to FL environments, introducing white-box MIAs.
Their privacy analysis demonstrated how both passive and active adversaries could exploit gradients and model updates to infer private information.
\BfPara{LDIA in FL} 
LDIA represents another significant privacy threat in FL.
Gu \etal~\cite{gu2023ldia} introduced LDIA as a new attack vector, where adversaries seek to infer the distribution of labels in clients' training data by analyzing model updates.
This work showed that even when individual data points are protected, the overall data distribution might still be exposed, leading to privacy concerns.
Wainakh \etal~\cite{wainakh2021user} further explored user-level label leakage by performing gradient-based attacks in FL.

\BfPara{LDIA and MIA in FD} 
% FD has been proposed as a lightweight alternative to traditional FL.
% It reduces communication overhead by exchanging logits or softmax values instead of model parameters.
Yang \etal~\cite{yang2022fd} proposed FD-Leaks, an attack designed to perform MIA in FD settings.
By analyzing logits, adversaries can infer membership information, potentially revealing sensitive training data.
Similarly, Liu \etal~\cite{liu2023mia} presented MIA-FedDL, which enhances MIA by using shadow models to infer membership information with higher accuracy in FD settings.
Despite their contributions, they made assumptions that are infeasible in heterogeneous environments.
\fi
% Despite their contributions, these works make several assumptions that may limit the feasibility of the proposed attacks in practice.
% For instance, both FD-Leaks and MIA-FedDL assume that the adversaries have access to all logits or can easily build shadow models that mimic the behavior of target models.
% Such assumptions are often unrealistic in heterogeneous environments.

\BfPara{Defenses and Countermeasures}
DPSGD~\cite{abadi2016deep} can be employed during the training phase to mitigate against privacy attacks to the client model. 
Additionally, specialized MIA defense methods such as SELENA~\cite{tang2022mitigating}, HAMP~\cite{chen2023overconfidence} and DMP\cite{shejwalkar2021membership} can be integrated into the training process.
Several studies have proposed enhanced FD frameworks with improved privacy protection mechanisms to reduce client privacy leakage.
%Several studies have proposed defenses to mitigate MIA and LDIA.
%Wang \etal~\cite{wang2024graddiff} proposed GradDiff, a gradient-based defense mechanism that employs differential comparison to detect and mitigate MIA in FD settings.
Zhu \etal~\cite{zhu2021data} investigated data-free knowledge distillation for heterogeneous federated learning.
They presented an approach that reduces the need for public datasets.
Chen \etal~\cite{chen2023best} proposed FedHKD, where clients share hyper-knowledge based on data representations from local datasets for federated distillation without requiring public datasets or models.

% Our work builds upon these foundations, specifically investigating the privacy vulnerabilities in PDA-FD frameworks. 
%Unlike previous studies that focused on traditional FL or specific attack scenarios in FD, our work aim to provide a comprehensive analysis of LDIA and MIA across multiple PDA-FD frameworks.




\section{Conclusion \& Future Directions}
\section{Conclusion}
In this work, we propose a simple yet effective approach, called SMILE, for graph few-shot learning with fewer tasks. Specifically, we introduce a novel dual-level mixup strategy, including within-task and across-task mixup, for enriching the diversity of nodes within each task and the diversity of tasks. Also, we incorporate the degree-based prior information to learn expressive node embeddings. Theoretically, we prove that SMILE effectively enhances the model's generalization performance. Empirically, we conduct extensive experiments on multiple benchmarks and the results suggest that SMILE significantly outperforms other baselines, including both in-domain and cross-domain few-shot settings.


\bibliographystyle{ieeetr}
\bibliography{references}





\end{document}