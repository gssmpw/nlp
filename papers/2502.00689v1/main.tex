\documentclass[conference]{IEEEtran}
\IEEEoverridecommandlockouts

\usepackage{cite}
\usepackage{amsmath,amssymb,amsfonts}
\usepackage{algorithm}
\usepackage{algpseudocode}
\usepackage{url} 
\usepackage{hyperref} 
\usepackage{graphicx}
\usepackage{textcomp}
\usepackage{xcolor}
\usepackage{algorithm}
\usepackage{booktabs}
\usepackage{siunitx} % For number alignment
\usepackage{microtype}
\usepackage{booktabs}  % for \toprule, \midrule, etc.
\def\BibTeX{{\rm B\kern-.05em{\sc i\kern-.025em b}\kern-.08em
    T\kern-.1667em\lower.7ex\hbox{E}\kern-.125emX}}
\begin{document}
\title{Leveraging LLMs for Dynamic IoT Systems Generation through Mixed-Initiative Interaction}

\author{\IEEEauthorblockN{Bassam Adnan\textsuperscript{\dag}}
\IEEEauthorblockA{\textit{IIIT Hyderabad, India}\\
bassam.adnan@research.iiit.ac.in}
\and
\IEEEauthorblockN{Sathvika Miryala\textsuperscript{\dag}}
\IEEEauthorblockA{\textit{IIIT Hyderabad, India}\\
miryala.sathvika@research.iiit.ac.in}
\and
\IEEEauthorblockN{Aneesh Sambu\textsuperscript{\dag}}
\IEEEauthorblockA{\textit{IIIT Hyderabad, India}\\
sambu.aneesh@research.iiit.ac.in}
 \and
\IEEEauthorblockN{Karthik Vaidhyanathan}
\IEEEauthorblockA{\textit{IIIT Hyderabad, India}\\
karthik.vaidhyanathan@iiit.ac.in}
\and
\IEEEauthorblockN{Martina De Sanctis}
\IEEEauthorblockA{\textit{GSSI, L’Aquila, Italy}\\
martina.desanctis@gssi.it}
\and
\IEEEauthorblockN{Romina Spalazzese}
\IEEEauthorblockA{\textit{Malmö University, Sweden}\\
romina.spalazzese@mau.se}
}
\maketitle

\renewcommand{\thefootnote}{\dag} %  dagger for this footnote
\footnotemark
\footnotetext{These authors contributed equally to this work.}
\renewcommand{\thefootnote}{\arabic{footnote}} % Restore footnote numbering
\setcounter{footnote}{0}

\begin{abstract}
IoT systems face significant challenges in adapting to user needs, which are often under-specified and evolve with changing environmental contexts. To address these complexities, users should be able to explore possibilities, while IoT systems must learn and support users in the process of providing proper services, e.g., to serve novel experiences. The IoT-Together paradigm aims to meet this demand through the Mixed-Initiative Interaction (MII) paradigm that facilitates a collaborative synergy between users and IoT systems, enabling the co-creation of intelligent and adaptive solutions that are precisely aligned with user-defined goals. This work advances IoT-Together by integrating Large Language Models (LLMs) into its architecture. Our approach enables intelligent goal interpretation through a multi-pass dialogue framework and dynamic service generation at runtime according to user needs. To demonstrate the efficacy of our methodology, we design and implement the system in the context of a smart city tourism case study. We evaluate the system's performance using agent-based simulation and user studies. Results indicate efficient and accurate service identification and high adaptation quality. The empirical evidence indicates that the integration of Large Language Models (LLMs) into IoT architectures can significantly enhance the architectural adaptability of the system while ensuring real-world usability.

\end{abstract}

\begin{IEEEkeywords}
LLM, Self-Adaptation, Software Architecture, Service Generation, Dynamic Application Generation, IoT-Together Paradigm
\end{IEEEkeywords}

\section{Introduction}
\documentclass[../main.tex]{subfiles}
\graphicspath{{../images/}}
\makeatletter
\def\input@path{{../images/}}
\makeatother
\begin{document}
\section{Introduction}
\begin{figure}
\centering
\begin{tikzpicture}
\node[inner sep=0pt] (ws) at (0, 0) {
\includegraphics[height=.4\textwidth, trim={10cm 0 10cm 0},clip]{world_space.png}};
\node[inner sep=0pt] (cs) at (6,0) {\includegraphics[height=.4\textwidth, trim={10cm 1cm 10cm 4cm},clip]{conf_space.png}};
\end{tikzpicture}
\vspace{-5pt}
\label{fig:pbrm_intro}
\caption{\textbf{Left}: Shows world space obstacles as grey spheres. Robots start and goal configuration is colored red and green, respectively. Configurations along the computed path are colored transparent blue. \textbf{Right:} Mapped world space scenario to configuration space. Obstacle region is the grey mesh. Red spheres are collision-free regions computed by the neural SCDF. The optimized shortest path in the convex corridor is the blue curve.}
\vspace{-25pt}
\end{figure}
Motion planning is the problem of finding a collision-free trajectory that connects a given start and goal configuration. The planning takes place in the configuration space of the robot. For single body robots, like mobile robots or drones, the configuration space and the world space are usually the same. This simplifies the planning, since explicit obstacle representations are available which enables geometrical tools like separating hyperplanes, smallest distance to obstacles etc., to be used when designing motion planning algorithms. For multi-body robots like manipulators, the situation is completely different. The world space obstacles are usually mapped to non-convex regions, and to make the problem even harder, the mapping is usually not known. Forming explicit representations of the obstacle region in the configuration space is usually too expensive or intractable. Despite all of this, sampling based planners are used with great success, which mainly is due to their use of implicit representations of the obstacle region. The basic idea is to construct a graph in the configuration space that covers and connects the collision-free region. From this graph, a path can be extracted that connects a given start and goal configuration. The approach is computationally expensive, since the graph is constructed with the smallest geometrical building block available, points, which represents a collision-check. Furthermore, the extracted paths from the graph are non-smooth and jagged due to the stochastic nature of the approach. This adds an additional post-processing step to the process, where the paths are shortcutted and smoothened, before the path can be used for tracking. Clearly a lot of time is invested to form this graph and produce smooth paths. Thus, if the obstacles start to move, then all of this work is done in no use, since all points that make up this graph need to be re-verified, which is simply too time consuming to be done in real time.
\\\\
In this work, we want to address the existing drawbacks of the sampling based planners. Our main contribution is an improved motion planner where each vertex in the graph covers a collision-free region in the form of a sphere instead of a point and where the edges are formed with neighboring intersecting spheres. This representation has the advantage of instead of returning piecewise linear paths, returning a sequence of overlapping spheres, i.e. a convex corridor, that connects a given start and goal configuration, illustrated in Figure \ref{fig:pbrm_intro}. This convex corridor allows us to use convex optimization to produce smooth trajectories, instead of computationally expensive post-processing methods. The representation further allows us to estimate the coverage of the collision-free space, which gives us awareness and feedback in the offline roadmap construction phase. Finally, our representation is simple to adapt to moving obstacles, simply requery for the new radii and recheck for intersections. 
\\\\
The spherical collision-free regions are formed using a signed distance function (SDF), which is a function that returns the smallest distance from an arbitrary point to the boundary of an obstacle. As the name implies, the distance is signed, thus if the point is inside the obstacle it is negative otherwise positive. If the distance is positive, a sphere with radius equal to the distance is guaranteed to cover a collision-free region. Using an SDF in motion planning is not new, but what is novel about our approach is that we express the distance in the configuration space instead of the world space and by doing so allows us to form these convex collision-free regions. We refer to the resulting SDF as a signed configuration distance function (SCDF). Computing an SCDF analytically is non-trivial, our approach is therefore to parameterize the SCDF with a deep neural network and learn the mapping by supervised learning. Our resulting neural SCDF can compute distances for different parameter values of obstacle shapes and we also show how multiple distances can be combined, thus making our approach flexible.
\section{Related work}
Motion planning algorithms can roughly be divided into three families, grid-based, sampling based and optimization based methods. Grid-based methods (GBM) discretize the planning space from which a graph is then compiled. A standard search method is A$^\star$ \citep{a_star}, which is classified as an \textit{informed} search method, since it employs a heuristic function to speed up the search. A$^\star$ guarantees to return an optimal path at the level of discretization used. GBMs usually discretize the planning space by a regular lattice and this limits the GBMs to problems with low dimensionality due to the curse of dimensionality. Thus, GBMs are usually limited to single-body robots where the degrees of freedom (DOF) are low. To overcome the inherent scaling problem with the GBMs, stochastic methods are usually used for multi-body robots. These methods are termed as sampling-based methods (SBM) and core members within this family are the rapidly-exploring random trees (RRT) \citep{rrt} and the probabilistic roadmap (PRM) \citep{prm}. RRT grows a tree from the start configuration and explores the collision-free region in a rapid way until it is able to connect to the goal region. RRT is usually improved by bi-directional planning \citep{rrt_connect}, i.e. an additional tree is grown from the goal configuration and the trees are tested for connection after any tree has been expanded. RRT is a single-query method, thus it searches for a path from scratch each time it is queried. Contrary to this, PRM is a multi-query method, which solves for multiple queries without starting from scratch. PRM does this by creating a roadmap (graph) that covers the collision-free space as an offline step. The graph is then used to solve for multiple queries. PRMs are used in cases where the environment does not change since the extra offline step is too computationally costly and needs to be re-done if the environment is changed. In our work, we address this inherent issue by using a different roadmap representation. Our vertices in the graph cover a collision-free region in the form of spheres and we form the edges by checking for intersecting spheres. If something in the environment changes, we recompute the spheres radii and recheck the intersections, without relying on collision detection. We use a trained neural network to compute the sphere radius, therefore querying for the radius can be done fast, hence our representation enables the PRM for dynamic environments.
\\\\
In the recent decades, optimization based methods (OBM) \citep{chomp, schulman, itomp, stomp} have been introduced as an alternative to SBM for multi-body robots. Like the SBM, the OBMs scale well to higher dimensional problems and produce smoother motion. It is common to use a SDF in the optimization since it is a smooth function, thus enabling gradient-based methods. However, the standard way of expressing the SDF is in world space. The distance therefore needs to be mapped to the configuration space by the forward kinematics. This mapping makes the optimization problem a non-linear program (NLP), which is computationally expensive to solve. Recently, a different approach has been proposed. In \cite{mp_gcs} motion planning is formulated as a convex optimization problem by using the graph of convex sets framework \citep{gcs}. The underlying idea is to decompose the collision-free space into intersecting convex sets from which a convex optimization problem is formulated. In cases where an explicit representation of the obstacles in the configuration space exists, like for single-body robots, creating collision-free convex regions can be done fast \citep{iris}. For multi-body robots, this is non-trivial. Existing work does this successfully \citep{iris_nlp, iris_c} by an optimization based approach, but the methods are still too time consuming to be used in the presence of moving obstacles. Our approach is instead to use deep learning to learn an SDF expressed in the configuration space. With this, we can query for shortest distances to the collision boundary, which allows us to expand spherical regions which are collision-free. Our approach is fast and therefore enables our suggested roadmap planner to be used in dynamic environments.
\\\\
Recent research has focused on learning collision detection \citep{fk_kernel_distance, diffco, graphdistnet} by predicting the signed distance between the robot links and the surrounding obstacles in the world space. The learned SDF is used in trajectory optimization but since the distance is expressed in the world space, the problem becomes an NLP and therefore takes a long time to solve. We take a novel approach and suggest to instead express the signed distance in the configuration space. This allows us to improve the PRM at the same time as it enables convex optimization for trajectory optimization, which runs faster and is more reliable than NLP solvers. In \cite{cspf} a learned signed distance function in the configuration space is proposed similar to our approach. However, their approach is restricted to point cloud representations, while we propose to represent the obstacles as parameterized geometric shapes, e.g. spheres. Furthermore, we also show how to use our learned SCDF to improve an existing roadmap planner.
\section{Problem formulation}
A robot is located in the world space, $\W \subset \R^3 $. The unique location of the robot is given by its configuration $\q \in \C$, where $\C$ is the configuration space. The set of points covered by the robots bodies at a certain configuration is expressed as $\B(\q) \subset \W$. The robot is surrounded by $\NrObst$ obstacles $\O = \bigcup_{i=1}^{\NrObst} \O_i$, where  $\O_i \subset \W$. The representation of the obstacle in the configuration space is the set $\C\O_i = \{\q \in \C \: |\: \B(\q) \cap \O_i \neq \emptyset \}$. The obstacle space is formed as $\Co = \bigcup_{i=1}^{\NrObst} \C \O_i$. The complement is referred to as the free space, $\Cf = \C \setminus \Co$. The path planning problem is a tuple, ($\Cf$, $\qStart$, $\qGoal$), where we want to connect a query pair, consisting of a start, $\qStart$, and goal configuration, $\qGoal$, with a geometric path, $\q(s): [0, 1] \mapsto \Cf$, such that $\q(0)=\qStart$ and $\q(1)=\qGoal$, or report correctly when such a path does not exist.
\end{document}



\begin{figure*}[ht]
    \centering
    \includegraphics[width=0.80\textwidth]{figures/three_pass_diagram.png} 
    \label{fig:initial-interaction-diagram}
    \caption{Three-Pass Dialogue Flow: Progressive Identification of User Goals and Service Parameters enabling Goal-Driven Architecture}

\end{figure*}

\section{Motivating  Case Study}

\begin{table*}[htbp]
    \centering
    \small
    \begin{tabular}{p{14cm}}
     \toprule
\#\#\#  Objective: \\
Generate a 5-day family travel itinerantry that satisfies all specified requirements while adhering to highly fine-grained constraints. The generated itinerary should balance real-time adaptability, strict hard attributes, and semantic soft attributes. \\

\#\#\# User Profile: \\
 - Travelers: 2 adults + 1 child (age 8) \\
 - Budget: $<=$ \$300/day (total \$1,500 for the trip) \\
 - Activity Balance: 70\% educational/cultural experiences, 20\% relaxation, 10\% family-friendly shopping. \\

\#\#\# Hard Attributes: \\
- Activity Scheduling: \\
\quad- Each activity must have a defined start and end time, ensuring there is no overlap between activities. \\
\quad- A break period from 13:00-14:30 is mandatory daily. \\
\quad- Each activity must fit within a 2-hour window unless otherwise specified. \\

- Budget Requirements: \\
\quad- Each day’s total cost (including transportation, food, and activities) must not exceed \$300. \\
\quad- Transportation is limited to metro and walking only, with a maximum of 3 metro rides per day. \\

- Location Constraints: \\
\quad- Must-visit locations: City Zoo (Day 1) and Science Museum (Day 3). \\
\quad- Activities must occur in geographically adjacent areas to minimize walking distance. \\

- Keyword Requirements: \\
\quad- Each day’s description must include specific keywords. For example: \\
\quad- Day 1: “wildlife,” “exploration,” and “interactive learning.” \\
\quad- Day 3: “science,” “innovation,” and “hands-on exhibits.” \\

- Structure Constraints: \\
\quad- Each day’s itinerary must consist of 4 sections: \\
\quad\quad- Morning activity \\
\quad\quad- Break/lunch period \\ 
\quad\quad- Afternoon activity \\
\quad\quad- Evening summary (limited to 50 words) \\

\#\#\# Soft Attributes \\
- Tone and Emotion: \\ 
\quad- Day 1: Use a tone that conveys “excitement and discovery.” \\ 
\quad- Day 3: Use a tone that conveys “curiosity and wonder.” \\
- Language Style: \\ 
\quad- Use descriptive, vivid, and family-friendly language throughout. \\
\quad- Include at least one metaphor or simile per day (e.g., "The Science Museum felt like stepping into the future!"). \\
- Visual Details: \\
\quad- Each activity must include specific sensory details (e.g., "the bright colors of the parrots at the zoo" or "the tinkling sound of water fountains at the park").

- Adaptive Adjustments (Real-time Constraints): \\
\quad- Weather Sensitivity: \\
\quad\quad- If the rain forecast exceeds 60\%, replace outdoor activities with indoor alternatives while keeping the overall tone and keywords intact. \\ 
\quad- Physical Endurance: \\
\quad\quad- If a day’s total walking distance exceeds 10 kilometers, the next day’s activities must reduce walking by 30\%. \\
\quad- Health Responsiveness: \\
\quad\quad- If a health-related issue arises (e.g., fatigue or illness), adjust the itinerary dynamically to: \\
\quad\quad- Reduce activity duration to half. \\ 
\quad\quad- Substitute the activity with a more relaxing or passive option. \\
\bottomrule
    \end{tabular}
    \caption{The complete travel planner case study.}
    \label{tab:travel_planner_case}
\end{table*}

\section{Approach}
\begin{figure*}[ht]
    \centering
    \includegraphics[width=1.4\textwidth, height=10cm, keepaspectratio]{figures/architecture.drawio.png}
    \caption{High-Level Architecture: System Components and Their Interactions}
    \label{fig:high-level-architecture-overview}
\end{figure*}


Figure \ref{fig:high-level-architecture-overview} presents the proposed system architecture that implements the \textit{IoT-Together paradigm}. The newly proposed components, distinguished by grey shading in the figure, form an integral part of the overall system design. While the \textit{learning management} component remains unimplemented in the current version, it has been identified as a key area for future development. The system facilitates dynamic application reconfiguration based on user goals through comprehensive integration of IoT environmental data and system-level information. The system architecture is designed in a way such that, it supports dynamic evolvability through the generation and integration of new services at run-time in accordance with user goals.

The system adopts key components from the IoT-Together paradigm, including \textit{Goal Management}, \textit{Knowledge Management}, \textit{Context Management}, \textit{Intelligent User Interface (IUI) Generation}, and \textit{Backend Generation}. User interacts with the system using a device (Smart Phone/Laptop/Tablet) through the user interface to enter the query. This query is then passed to the Goal Management, which identifies the set of services that satisfy user goals. The Goal Management uses the LLM to determine the services within the system that can satisfy the user goals or the services that need to be generated. In this phase, the user and system collaboratively (through back-and-forth conversational exchanges) identify the set of parameters and the services required to achieve the user goal.
After the identification of the required services, they are transferred to the Intelligent User Interface Generator to build the application. It locates the services in the Knowledge Management; otherwise, a description of the new service is forwarded to the Backend Generation component, which generates the service, thus extending the architecture runtime according to the user needs. The generated service is sent back to the Intelligent User Interface Generator. The builder within the component integrates these services using a pre-defined template and then creates the website. Intelligent User Interface Generator provides the URL to access this website created. Fig-3 provides the sequence diagram for the process of generation of the application to the user. Meanwhile, the sensors of the IoT environment transmits data periodically to Knowledge Management, which maintains a persistent data repository. This accumulated data is subsequently utilized by the system during the creation of new services and translating the user needs to the goals using the Goal Management.
\\
The User Interface serves as the interaction medium between users and the system, managing conversational exchanges during service identification and preference elicitation. Users submit their initial queries through the interface, and the system engages in a structured conversation to clarify requirements. We represent user queries by $Q_\text{user}$ and system responses by $R_\text{sys}$. For the running example, $Q_\text{user} =$ ``I have 3 hours to explore Hyderabad's old charm" and the $R_\text{sys}$ provides relevant information along with follow-up queries for goal formulation.
Upon successful completion of this dialogue, the interface presents the user with a URL to access the generated web application that implements their specified needs.\\
In the subsequent, following the Fig- \ref{fig:high-level-architecture-overview} we introduce core components and the interactions between them in detail.

\subsection{\textbf{Goal Management}}
User Interface forwards user queries to the Goal Management. The LLM in the Goal Management interprets these queries based on three elements: the summary of previous exchanges, the environmental context from Knowledge Management, and the current state of system services. These conversations are modeled to extract maximum information about user goals and preferences. The LLM engages in an iterative dialogue with users to progressively refine the user goals until all requisite services are identified. This extraction process is carried out using prompt templates in LLMs, each designed to identify specific contextual elements at each pass: environmental parameters in the first pass, goal refinement in the second pass, and service requirements in the final pass. User preferences are then applied as parameters to customize the identified services. To enable this information extraction process, we designed a three-pass conversation structure.

\textbf{Pass 1: Contextual Awareness} 
In this pass, the LLM aggregates contextual information $C$ by combining sensor data $D_{\text{sensor}}$ (weather conditions, noise levels, time of day, day of the week, traffic conditions, and air quality), with user-specific data $D_{\text{user}}$ (current location, past activities, and stated preferences as $\text{params}$). A contextual state $X$ maintains the current interaction state, while service registry data from Knowledge Management provides information about available service capabilities. Each service is represented as below:
\[
S_i = ( \text{name}_i, \text{description}_i, \text{params}_i ), \quad \text{for} \quad i = 1, 2, \dots, n
\]
 Realization of values for few services are shown in Table ~\ref{tab:service-definitions}.
\begin{table}[ht]
\setlength{\tabcolsep}{2pt}
\caption{Service Definitions\protect\footnotemark}
\label{tab:service-definitions}
\centering
\begin{tabular}{lll}
\toprule
\textbf{Service Name} & \textbf{Description} & \textbf{Parameters} \\
\midrule
historical\_info & Provides historical and cultural & site\_name \\
 & information about monuments and sites & \\ \hline
restaurant\_finder & Recommends restaurants based on & location, cuisine, \\
 & cuisine preferences and location & diet \\ \hline
crowd\_monitor & Tracks and reports real-time crowd & location, time \\
 & density at various locations & \\ 
\bottomrule
\end{tabular}
\end{table}
\footnotetext{The values presented are illustrative and not representative of entire data. Definitions may also include associated schema, endpoints, and additional metadata relevant to the service.}
Together these services are used as 
\[
D_{\text{services}} = \{ S_1, S_2, \dots, S_n \}
\]
The complete context is defined as \[C = \{D_{\text{sensor}}, D_{\text{user}}, D_{\text{services}}, X\}\]

For example, when a user visits the city with time constraints and specific interests, such as local cuisine, nature activities, or historical sites, their preferences are stored as contextual information.

In the running example, the system identifies the user as a history enthusiast along with the time constraints and captures it via prompt templates.
\[
D_{\text{user}} = \{
\text{current location}, \text{3hrs time}, \text{history enthusiast}
\}
\]
After establishing the complete context, the next pass proceeds with goal identification. \\
\textbf{Pass 2: Goal Formulation and Refinement}
In this pass, the LLM generates Initial Goal Hypotheses based on contextual insights about the user's potential goals $G_{\text{user}}$. These hypotheses range from broad intentions (e.g., leisure activities) to concrete objectives (e.g., specific venue requirements). The system employs \textbf{Mixed-Initiative Interaction}, establishing a collaborative dialogue between the user and system for goal refinement. This refinement process is internally guided by structured prompts that evaluate hypotheses against available services and contextual constraints.

This interaction comprises two key components, each driven by specialized prompt templates:
\begin{itemize}
    \item \textbf{Proactive Suggestions}: The system suggests possible activities based on initial goal hypotheses and context $C$, using prompts that analyze alignment between user preferences and available services
    \item \textbf{Clarification Dialogues}: The system elicits specific preferences and constraints through targeted queries, using prompts designed to resolve ambiguities and gather missing information
\end{itemize}

For the running example, the Goal Management uses these structured prompts to identify historical information services as a potential match. This is represented as:
\[
S_{\text{hist}} = \{
    \text{name}: \text{historical\_info},
    \text{description},
    \text{params}: \text{site\_name}
\}
\]

where $S_{\text{hist}} \in D_{\text{services}}$ represents the identified service matching the user's interests in $D_{\text{user}}$. \\
\textbf{Pass 3: Goal Verification and Confirmation}
In this final pass, the LLM proposes a curated set of services that align with the inferred goals and contextual factors extracted from the prompt templates gathered from \textbf{Pass 1} and \textbf{Pass 2}. If the user rejects the final proposal, the process reverts to \textbf{Pass 1}.

Continuing the running example, the system first seeks confirmation for the historical information service. Upon user request for additional dining options, the system extends the service set as follows:
\[
G_{\text{users}} = \{ S_{hist},S_{restau}
\}
\]

where $S_{\text{hist}}$ and $S_{\text{restau}}$ follow from Table~\ref{tab:service-definitions} for Historical Information and Restaurant Finder services respectively, with parameters instantiated based on user preferences. In our running example, for the historical information service and restaurant finder, these parameters realize the values based on user conversations with system specifying their requirements and preferences.\\ 
\(
\theta_q = \{
site\_name = 
    \text{[Charminar} \text{, Laad Bazaar]},
\)
\(
location = 
    \text{Laad Bazaar}, 
cuisine = \text{Any},
\)
\(
diet=\text{Non-vegetarian}\}
\)
\\The system maintains a history data base $H$  along with contextual information throughout these passes which include:

\[
H = \{P_{\text{user}}, S_{\text{identified}}, C_{\text{prev}}\}
\]

where $P_{\text{user}}$ represents user preferences, $S_{\text{identified}}$ tracks identified services, and $C_{\text{prev}}$ maintains the contextual history of previous interactions.
\begin{figure*}[ht]
    \centering
    \includegraphics[width=\textwidth]{figures/sequence_research.drawio_5.png}
    \label{sequence}
    \caption{Sequence diagram for creating the web application }
\end{figure*}

\subsection{\textbf{Backend Generation}}

The Backend Generation component activates when the Service Discovery identifies service requirements from the Goal Management that can be fulfilled based on available definitions but are not yet implemented in the current service set $D_{\text{services}}$. 
\begin{algorithm}
\caption{: Service Generation}
\label{alg:service_gen}
\begin{algorithmic}[1] % Ensure sequential line numbering
\State \textbf{Input:} Service requirement query $Q$, System context $S_{\text{ctx}}$
\State \textbf{Output:} Service $S$ (either matched or newly generated)
\\
\Function{ServiceGeneration}{$Q, S_{\text{ctx}}$}
    \State $D_{\text{services}} \gets S_{\text{ctx}}.services$
    
    \State // description refinement phase
    \State $service\_match \gets \text{DescriptionRefiner}(Q, D_{\text{services}})$
    \If{$service\_match \neq \emptyset$}
        \State // existing service satisfies Q
        \State \Return $service\_match$
    \EndIf
    
    \State // service generation phase
    \State $D_{\text{schema}} \gets S_{\text{ctx}}.schemas$
    \State $Q_{\text{gen}} \gets \text{RefineGenerationQuery}(Q, D_{\text{schema}})$
    \State $S_{\text{new}} \gets \text{GenerateService}(Q_{\text{gen}})$
    \State $D_{\text{services}} \gets D_{\text{services}} \cup \{S_{\text{new}}\}$
    \State $\text{UpdateServiceContext}(S_{\text{ctx}}, S_{\text{new}})$
    \State \Return $S_{\text{new}}$
\EndFunction
\end{algorithmic}
\end{algorithm}
An algorithm to achieve this is described in Service Generation (Algorithm-\ref{alg:service_gen}) with components described in Table-\ref{tab:service-gen-components}. The generation process initiates by creating a service requirement query:
\[
Q = \{
    type: \text{service\_req},
    params: \theta_q
\}
\]
where $\theta_q$ represents the required service parameters  identified through the dialogue exchange with the Goal Parser. \\

The Service Manager maintains the system context $S_{\text{ctx}}$ comprising:

\[
S_{\text{ctx}} = \{
    D_{\text{services}},
    D_{\text{schema}},
    D_{\text{config}}
\}
\]

where $D_{\text{schema}}$ represents available database schemas and $D_{\text{config}}$ contains service configurations. The description of the services assist the Description Refiner -- an LLM Agent -- to evaluate incoming query $Q$ against $S_{\text{ctx}}$ through the function $f_{\text{match}}$:

\[
f_{\text{match}}(Q, S_{\text{ctx}}) = 
\begin{cases}
    S_i, & \text{if } \exists S_i \in D_{\text{services}} \\
    Q_{\text{gen}}, & \text{otherwise}
\end{cases}
\]


where $Q_{\text{gen}}$ represents the refined service generation query which is achieved using RefineGenerationQuery function (line-14 of Algorithm \ref{alg:service_gen}) which would be passed to the Service Generator (line-15), another LLM agent with coding capabilities. This would contain instructions to write code, with relevant database information (their schema, location etc.) required to generate these services. 

\begin{table}[h]
\setlength{\tabcolsep}{3pt}
\caption{Algorithm-\ref{alg:service_gen} Components executed by LLM agents}
\label{tab:service-gen-components}
\centering
\begin{tabular}{ll}
\toprule
\textbf{Function} & \textbf{Description} \\
\midrule
RefineGenerationQuery & Examines D$_{schema}$ relevant to query Q to \\
& prepare LLM coding prompt Q$_{gen}$ \\ \hline
GenerateService & Processes Q$_{gen}$ to generate and deploy S$_{new}$ \\ \hline
UpdateServiceContext & Updates S$_{ctx}$ with S$_{new}$ and adds to D$_{services}$ \\
\bottomrule
\end{tabular}
\end{table}
For example, extending our previous scenario, if a user requests crowd monitoring at historical sites:

\[
S_{\text{new}} = \{
    \text{name}: \text{crowd\_monitor}, \text{ } \theta_c
\}
\]
\[
\text{where } \theta_c = \{
    \text{loc}: \text{hist\_site},
    \text{time}: \text{now}
\}
\]


If $S_{\text{new}} \notin D_{\text{services}}$ but $D_{\text{schema}}$ contains relevant data structures for the monitoring of crowd density via the sensors and endpoints of these databases, the Service Generator creates and integrates the new service:


\[
D_{\text{services}} = D_{\text{services}} \cup \{S_{\text{new}}\}
\]

Upon the generation of the new service $S_{new}$ an entry is added describing its function and the parameters it expects  (Table-\ref{tab:service-definitions} entry for crowd monitor service).

    
    
 After the successful service generation, UpdateServiceContext algorithm \ref{alg:service_gen} signals the Service Manager
 to incorporate the new service into its context state, triggering
 necessary updates to both Context Management and its
 service registry in the Knowledge Management. The newly generated service $S_{new}$ is further transmitted to the Builder component for building the website.


\subsection{\textbf{Intelligent User Interface Generator}}

The Intelligent User Interface Generator transforms the identified user goals into a functional web application through three primary components: Service Discovery, Builder, and Hosting. 

\[
G_{user} = \{SI_1, SI_2, ..., SI_n\}
\]

$G_{user}$ represents the set of services identified from the goal parser phase, where each $SI_i$ represents a service with its parameters and requirements, where each $\text{SI}_i$ represents an identified service. The Service Discovery component matches between $G_{user}$ and available services in the service registry. For each service $SI_i \in G_{user}$, it searches for matches $M_i \subseteq D_{services}$ by evaluating service descriptions and parameter compatibility:



\[
M = \bigcup_{i=1}^{n} M_i,\]\[ \quad M_i = \{ S_j \mid \text{match}(\text{description}_{\text{SI}_i}, \text{description}_{S_j}) \]
\[\land \text{params}_{\text{SI}_i} \subseteq \text{params}_{S_j} \}
\]

where $\text{match}()$ represents the semantic matching function which compares the service descriptions. All the matches are forwarded to the builder for application generation. When $M_i = \emptyset$ for any $SI_i$, indicating that no matching service exists, the system initiates service generation. The Backend Generation creates a new service $S_{new}$ using the service description and available data schemas and returns it to the Builder.


Builder implements UI generation process which employs a hierarchical rendering system $R$ with specific handlers for different data types, some of which are listed below:

\[
R(d) = \begin{cases}
    R_{\text{metric}}(d), & \text{if } d \in \mathbb{R} \\
    R_{\text{list}}(d), & \text{if } d \in \text{List} \\
    R_{\text{dict}}(d), & \text{if } d \in \text{Dict} \\
    R_{\text{text}}(d), & \text{otherwise}
\end{cases}
\]

where each renderer implements specific visualization logic for its data type (e.g., $R_{\text{metric}}$ for crowd density, $R_{\text{list}}$ for restaurant listings). The Builder Component generates the complete application by combining these renderers with a base template $T$, where $\oplus$ represents the composition operator that combines multiple rendered components:



\[
App = T(M \cup \{S_{new}\}) \oplus \bigoplus_{i=1}^{n} R(SI_i)
\]












The final application is hosted through the Hosting Component, which manages service endpoints and provides URL access to users. 




\subsection{\textbf{Context Management}}
It hosts all the active services in the system. These services are built on top of the sensor data in Knowledge Management, which stores the data of the IoT environment (e.g., routes service, crowd monitoring, and air quality sensors). The configurations for these services are maintained in Knowledge Management, which is updated if the underlying data changes.

\subsection{\textbf{Knowledge Management}}
It is Responsible for storing environment and system-related data like the sensor data, user information like user preferences, location of user, configuration files, and a service registry. This registry includes a list of available services with brief descriptions of their functionality and parameters. Sensors periodically transmit data through the interoperable system, contributing to Knowledge Management's context for ongoing operations.


\section{Experiments and Results}

The evaluation focuses on assessing the proposed approach through the following research questions.

\begin{itemize}
    \item \textbf{RQ1: Effectiveness in Identifying Functionalities:}  
    How effective is the approach in identifying the correct set of functionalities corresponding to existing components ?
    
    \item \textbf{RQ2: Accuracy in Dynamic Service Generation:}  
    How accurate is the approach in dynamically generating services?  

    \item \textbf{RQ3: Effectiveness in System Generation:}  
    What is the effectiveness of the approach in generating the system as a whole?

    \item \textbf{RQ4: Efficiency in Application Generation:}  
    What is the efficiency of the approach in generating applications?
\end{itemize}
\subsection{Evaluation Setup} We evaluated our system using Hyderabad as a case study of a smart city implementation. The system comprises $9$ web-services: Air Quality, Crowd Monitoring, Event Notifier, Historic Information, Restaurant Locator, Travel Options, Water Quality, Exhibition Tracker, and Event Ticket Vendor. Several services operate on static contextual data feeds (e.g., Historic Information), while others process real-time data from a network of 12 distributed sensors. We implemented the IoT environment simulation using CupCarbon \footnotemark for realizing the architecture and generating sensor data based on domain-appropriate statistical distributions, with the core system in Python. we employed a two-fold evaluation strategy: (i) a multi-agent simulation framework modeling Tourist-Guide interactions across 100 experimental runs, and (ii) a user study ($n=15$) focusing on real-world usability and service adaptation quality.
\footnotetext{CupCarbon: \url{https://cupcarbon.com/}}
 \subsubsection{Tourist-Guide Simulations}
The system evaluation employed OpenAI's GPT-4o-mini \cite{openai2024gpt4technicalreport}, DeepSeek-V2.5 \cite{deepseekai2024deepseekv2strongeconomicalefficient} and CodeQwen1.5-7B \cite{bai2023qwentechnicalreport} models through the LangChain framework. We chose these based on the EvalPlus \cite{liu2023is} leader board. Experiments were conducted on an Nvidia L40S GPU with 8 vCPU, 62 GB RAM, and 48GB VRAM for hosting the CodeQwen1.5-7B model on HuggingFace. For GPT-4o-mini and DeepSeek-V2.5 model interactions, we utilized LangChain's OpenAI API. All these models were run with a temperature parameter of 0.7 based on preliminary experimentation.  
To evaluate the Goal Management, we designed a multi-agent simulation framework using CrewAI\footnotemark, modeling interactions between a Tourist agent and a Travel Guide agent (implementing our Goal Management's instruction set). The Tourist agent samples from $25$ predefined goals, generated through prompt engineering with domain-specific system knowledge, with time constraints uniformly distributed between $1-5$ hours. Each goal has an associated ground truth set of required services for validation.
\footnotetext{CrewAI: \url{https://www.crewai.com/}}
The goals were classified into concrete and ambiguous categories ($18:7$). Concrete goals have predictable mappings, such as ``Planning to visit Ramoji Film City'' mapping to \textit{ticket\_purchase} and \textit{travel\_options}. Ambiguous goals like ``First time in Hyderabad! Want to start with the locals' favorites'' may trigger multiple services (e.g., \textit{restaurant\_finder}, \textit{crowd\_monitor}, \textit{travel\_options}) based on conversation flow.
The simulation involved three sequential passes of Tourist-Guide interactions for service identification, repeated 100 times. While additional services could enhance user experience, we limit suggestions to avoid overwhelming users with options beyond their original goal.

\subsubsection{User Evaluation}
For complementing our simulation-based evaluation, we conducted a user study with students from the International Institute of Information Technology, Hyderabad (IIIT-H), which focused on understanding system effectiveness and overall user satisfaction through both quantitative metrics and qualitative feedback. The study involved participants ($n=15$) from diverse academic backgrounds within IIIT-H, specifically comprising 3 Ph.D. students (2 Computer Science and 1 Electronics/Communications Engineering), 5 Electronics/Communications Engineering students (B.Tech by M.S.), and 7 Computer Science students (B.Tech by M.S.). The participants were given a brief overview of the system's capabilities and were encouraged to interact with it based on their interests and needs. Each participant interacted with the system for approximately 10-15 minutes, with feedback collected through an integrated form in the user interface. The feedback mechanism collected three types of Quantitative Metrics: application rating, service accuracy rating, and service relevance rating (on a 1-5 likert scale), and Qualitative Feedback comprising query summaries, missing service identification, unnecessary service identification, and improvement suggestions. To further evaluate the effectiveness of dynamically generated services, we implemented a service rotation mechanism where three services were deliberately kept offline and replaced with generated implementations during each participant interaction, without informing participants, to assess integration seamlessness.







    
    



For assessing application generation efficiency, we integrated a metrics collection system with the Intelligent User Interface generator. The evaluation examined three critical metrics: total generation time (comprising dialogue latency, service discovery, template rendering, and deployment), token usage (aggregating input tokens from user queries, processing tokens from system context, and completion tokens from LLM responses), and build times per session. Given that service generation is not activated in every test scenario, we conducted a separate performance analysis of this component to ensure unbiased assessment.














\subsection{Results \& Discussions}
\noindent \textbf{RQ1: Effectiveness in Identifying Functionalities}


To evaluate the Goal Management's effectiveness, we analyze service identification accuracy using four key metrics defined in Table-\ref{tab:metrics-definition} in our simulation. The evaluation compares services identified after the third conversation pass against ground truth mappings derived from our tourism domain requirements.

\begin{table}[ht]
\setlength{\tabcolsep}{2pt}
\caption{Evaluation Metrics for Tourist-Guide simulation}
\label{tab:metrics-definition}
\centering
\begin{tabular}{l|p{7.5cm}}
\toprule
\textbf{Metric} & \textbf{Definition} \\
\midrule
Precision (P) & Ratio of correctly identified services to all identified services \\
\hline
Recall (R) & Ratio of correctly identified services to actual required services \\
\hline
F1 Score & Harmonic mean of precision and recall (2PR/(P+R)) \\
\hline
Parameter & Accuracy of identified service parameters (e.g., exact \\
Accuracy & locations, cuisines) against ground truth \\
\bottomrule
\end{tabular}
\end{table}

Analysis of the simulation results presented in Table-\ref{tab:goal-parser-categories} demonstrates comparable performance metrics between GPT-4o-mini and DeepSeek-V2.5 in service identification tasks. We noticed that both GPT-4o-mini and DeepSeek-V2.5 consistently respected time constraints while providing travel plans to the Tourist unlike CodeQwen1.5-7B which suggested plans spanning multiple days, exceeding the specified time constraints. CodeQwen1.5-7B exhibits lower precision values, displaying a tendency toward over-identification of required services. This over-identification introduces unnecessary complexity into the system architecture and imposes increased computational overhead during the build process.
\begin{table}[!htbp]
\setlength{\tabcolsep}{4pt}
\caption{Goal Parser Performance by Category}
\label{tab:goal-parser-categories}
\centering
\begin{tabular}{llcccc}
\toprule
\textbf{Model} & \textbf{Category} & \textbf{Precision} & \textbf{Recall} & \textbf{F1} & \textbf{Parameter} \\
& & & & & \textbf{Accuracy} \\
\midrule
CodeQwen1.5-7B & Ambiguous & 0.450 & 0.806 & 0.553 & 0.116 \\
& Concrete & 0.206 & 0.609 & 0.288 & 0.051 \\
& \textbf{Overall} & 0.282 & 0.670 & 0.370 & 0.071 \\
\midrule
GPT-4o-mini & Ambiguous & 0.683 & 0.795 & 0.730 & 0.549 \\
& Concrete & 0.467 & 0.773 & 0.559 & 0.739 \\
& \textbf{Overall} & 0.523 & 0.778 & 0.603 & 0.690 \\
\midrule
DeepSeek-V2.5 & Ambiguous & 0.681 & 0.788 & 0.725 & 0.585 \\
& Concrete & 0.492 & 0.830 & 0.591 & 0.743 \\
& \textbf{Overall} & 0.554 & 0.816 & 0.635 & 0.691 \\
\bottomrule
\end{tabular}
\end{table}

For the user evaluation (see Table-\ref{tab:user-satisfaction}) study, tourism-focused ($40\%$) and dining-related ($53\%$) queries dominated user sessions, with $67\%$ involving multi-service combinations. Restaurant Finder ($53\%$), Travel Options ($47\%$), and Historical Information ($33\%$) were the most frequently requested services. User feedback identified crowd monitoring (7 instances), air quality (3), and water quality (2) as desired additional services.
\begin{table}[ht]
\setlength{\tabcolsep}{4pt}
\caption{User Satisfaction Metrics}
\label{tab:user-satisfaction}
\centering
\begin{tabular}{lccc}
\toprule
\textbf{Metric} & \textbf{Average Rating (out of 5)} & & \\
\midrule
Application Rating & 4.0 & & \\
Accuracy Rating & 4.1 & & \\
Relevance Rating & 4.2 & & \\
\bottomrule
\end{tabular}
\end{table}
User studies highlighted the need for improved local data processing, particularly for proximity-based routing and recommendations. These insights suggest optimization areas aligning with our mixed-initiative vision, especially in collaborative monitoring and user-adaptive location services.

\smallskip
\noindent \textbf{RQ2: Accuracy in Dynamic Service Generation}

To evaluate the quality of dynamically generated services, we conducted multiple generation attempts (three per service) across our 9 services. We used CodeBERTScore \cite{zhou2023codebertscoreevaluatingcodegeneration} to assess the semantic similarity between generated and reference implementations, measuring four key aspects: precision (code correctness), recall (code completeness), F1-score (balanced measure), and F3-score (emphasizing on code completeness).

\begin{table}[ht]
\setlength{\tabcolsep}{4pt}
\caption{Service Generation Code Similarity}
\label{tab:code-similarity}
\centering
\begin{tabular}{lcccc}
\toprule
\textbf{Model} & \textbf{Precision} & \textbf{Recall} & \textbf{F1} & \textbf{F3} \\
\midrule
CodeQwen1.5-7B & 0.86 ± 0.02 & 0.79 ± 0.03 & 0.83 ± 0.02 & 0.80 ± 0.03 \\
DeepSeek-V2.5 & 0.91 ± 0.01 & 0.85 ± 0.03 & 0.88 ± 0.02 & 0.86 ± 0.03 \\
GPT-4o-mini & 0.90 ± 0.01 & 0.85 ± 0.03 & 0.87 ± 0.01 & 0.85 ± 0.02 \\
\bottomrule
\end{tabular}
\end{table}
As shown in Table \ref{tab:code-similarity}, DeepSeek-V2.5 achieved the highest overall performance with an F1-score of 0.88, outperforming CodeQwen1.5-7B by 6\% and comparable to GPT-4o-mini. Notably, all models maintained high precision (\(\ge\) 0.86), indicating reliable code generation quality. The relatively lower recall scores, particularly for CodeQwen1.5-7B (0.79), suggest occasional omissions in implementing complete functionality. These contrasting results from Table-\ref{tab:goal-parser-categories} suggest that while DeepSeek-V2.5 and GPT-4o-mini exhibit consistent performance across both service identification and code generation tasks, CodeQwen1.5-7B shows task-specific performance variations that could impact its suitability for general-purpose service generation in IoT environments. 

\smallskip
\noindent \textbf{RQ3: Effectiveness in System Generation}\\
We keep track of the total tokens required to generate these services across the models along with end-to-end latency
(including API request/response time) in Table-\ref{tab:service-generation}. \\
\begin{table}[!htb]
\setlength{\tabcolsep}{4pt}
\caption{Service Generation }
\label{tab:service-generation}
\centering
\begin{tabular}{lcc}
\toprule
\textbf{Model} & \textbf{Time (s)} & \textbf{Tokens} \\
\midrule
CodeQwen1.5-7B & 7.67 ± 0.26 & 3482.00 ± 19.80 \\
DeepSeek-V2.5 & 42.40 ± 4.52 & 4376.25 ± 228.17 \\
GPT-4o-mini & 25.66 ± 2.55 & 2063.17 ± 191.76 \\
\bottomrule
\end{tabular}
\end{table}
While GPT-4o-mini and DeepSeek-V2.5 achieved 100\% service generation success rate, CodeQwen1.5-7B only succeeded in 37\% of attempts. On inspection, we found that CodeQwen1.5-7B's performance limitations stemmed from (1) inconsistent instruction following and (2) JSON formatting errors.

\begin{figure}[!htb]
    \centering
    \includegraphics[width=0.8\linewidth]{figures/input_tokens.png}
    \caption{Token consumption scaling with increasing number of services for GPT-4o-mini and DeepSeek-V2.5. The x-axis represents the number of services and the y-axis shows the corresponding input token count.}
    \label{fig:token-scaling}
\end{figure}

To evaluate scalability in mixed-initiative contexts, we analyze how the input token consumption scales with an increasing number of services in the system. Figure~\ref{fig:token-scaling} illustrates this relationship across GPT and DeepSeek model, where the token usage pattern diverges significantly even for small number of services. This rise is primarily seen due to the addition of extra services leading to the Description Refiner requiring more tokens to process the system state received by the Service Manager.
While DeepSeek-V2.5 and GPT-4o-mini have comparable pricing (0.14 USD and 0.15 USD per 1M input tokens)\footnotemark, their actual costs differ due to variations in token consumption, directly impacting adaptive service generation costs. GPT-4o-mini, with its relatively compact architecture and lower token usage (Table-\ref{tab:service-generation}), demonstrates more efficient performance for dynamic interactions compared to DeepSeek-V2.5's 238 billion parameter architecture.
\footnotetext{As of 2024-12-10 on billing websites}
\smallskip

\noindent \textbf{Results for RQ4: Efficiency in Application Generation}
We evaluated the system using 15 scenarios from our human evaluation study.\subsection{Performance Analysis}
The system achieved an average total generation time of 23.10 ± 6.47 seconds. Table~\ref{tab:generation-breakdown} presents the detailed breakdown of the processing stages.
\begin{table}[ht]
\setlength{\tabcolsep}{4pt}
\caption{Generation Process Time Breakdown}
\label{tab:generation-breakdown}
\centering
\begin{tabular}{lcc}
\toprule
\textbf{Processing Stage} & \textbf{Mean (s)} & \textbf{SD (s)} \\
\midrule
Multi-pass conversation processing & 18.25 & 5.12 \\
Service identification \& parameter extraction & 3.82 & 1.14 \\
Template rendering \& application assembly & 1.03 & 0.31 \\
Final deployment & 0.004 & 0.002 \\
\bottomrule
\end{tabular}
\end{table}

Analysis of token distribution is presented in Table~\ref{tab:token-distribution}.

\begin{table}[ht]
\setlength{\tabcolsep}{4pt}
\caption{Token Usage Distribution}
\label{tab:token-distribution}
\centering
\begin{tabular}{lccc}
\toprule
\textbf{Token Type} & \textbf{Count (Mean ± SD)} & \textbf{\% of Total} \\
\midrule
Input tokens & 101.8 ± 70.12 & 1.25\% \\
Processing tokens & 7,308.1 ± 2,607.49 & 89.51\% \\
Completion tokens & 755.0 ± 281.28 & 9.24\% \\
\bottomrule
\end{tabular}
\end{table}

The system demonstrated decent build performance, with an average build time of 4.85 ± 1.98 milliseconds. This sub-10 ms build time was anticipated, as the builder only needs to render the application by sending it to the hosting component. Table~\ref{tab:app-generation} summarizes the overall performance metrics.

\begin{table}[ht]
\setlength{\tabcolsep}{4pt}
\caption{Application Generation Performance Metrics}
\label{tab:app-generation}
\centering
\begin{tabular}{lccc}
\toprule
\textbf{Metric} & \textbf{Mean ± SD} & \textbf{Min} & \textbf{Max} \\
\midrule
Total Duration (s) & 23.10 ± 6.47 & 13.46 & 33.08 \\
Total Token Usage & 8164.90 ± 2718.89 & 5531 & 13991 \\
Build Time (ms) & 4.85 ± 1.98 & 3.50 & 10.49 \\
\bottomrule
\end{tabular}
\end{table}

\begin{figure}[!htb]
    \centering
    \includegraphics[width=0.8\linewidth]{figures/token_distribution.png}
    \caption{Token distribution during application generation. The high proportion of processing tokens (89.51\%) indicates potential for optimization through context management improvements.}
    \label{fig:token-distribution}
\end{figure}

\subsection{Service Generation Analysis}

The Backend generation component was evaluated by generating each of our nine services ten times. The evaluation revealed consistent results across different service types, with an average generation time of 15.53 seconds. The metrics are summarized in  Table~\ref{tab:service-metrics}.

\begin{table}[ht]
\setlength{\tabcolsep}{4pt}
\caption{Service Generation Performance Metrics}
\label{tab:service-metrics}
\centering
\begin{tabular}{lc}
\toprule
\textbf{Metric} & \textbf{Value} \\
\midrule
Average processing time (s) & $15.53 \pm 1.12$ \\
Total token usage & $4,992.89 \pm 180.29$ \\
Coefficient of Variation (\%) & $3.61$ \\
\bottomrule
\end{tabular}
\end{table}


The Coefficient of Variation (CV), calculated as (Standard Deviation / Mean) × 100, measures dispersion across different metrics. A CV of 3.61\% indicates high consistency in generating any service regardless of its type. When service generation is incorporated into the total application generation metrics, we observe a total duration of 38.63 seconds (23.10 ± 6.47 + 15.53 ± 1.12) and total token usage of 13,157.79 tokens (8,164.90 ± 2,718.89 + 4,992.89 ± 180.29).




\renewcommand{\thefootnote}{} % Suppress footnote numbering
\footnotetext{Code available on GitHub: \url{https://github.com/sa4s-serc/SAS_llm_query/tree/iot-prototype}}
\renewcommand{\thefootnote}{\arabic{footnote}} % Restore footnote numbering

\section{Discussions}

\subsection{The Asymmetry of the Point Spread Function (PSF) in Microscopy}

In the ideal imaging model, the Point Spread Function (PSF) of a microscope is symmetric with respect to the focal plane. 
This symmetry allows algorithms to estimate the absolute defocus distance but prevents them from determining whether the defocus is above or below the focal plane, thereby rendering one-shot autofocusing seemingly impractical.
However, in real optical microscopy systems, the PSF often exhibits significant asymmetry due to refractive index mismatches among the different media in the imaging path, such as the slide, sample, cover slip, and the surrounding medium like air or immersion oil. 
These mismatches introduce aberrations, including spherical aberration, coma, astigmatism, field curvature, and distortion, which disrupt the ideal symmetric distribution of the PSF. 
Figures~\ref{fig:psf}(a) and \ref{fig:psf}(b) present the 3D PSF and 2D PSF of our developed Whole Slide Imaging (WSI) device.
These visualizations are generated using the Gibson \& Lanni PSF model~\cite{Gibson:89} within the open-source software Fiji~\cite{Schindelin2012-jh} and the PSF Generator plugin\footnote{http://bigwww.epfl.ch/publications/kirshner1103.html}.
Figure~\ref{fig:psf}(c) shows images at symmetric defocus distances on both sides of the focal plane.
Figure~\ref{fig:psf}(d) illustrates the differences in pixel grayscale values at the same position for images at 5\si{\micro\meter} and -5\si{\micro\meter}.
These visualizations also demonstrate that, in real optical microscopy systems, the PSF is asymmetric.
Additional theoretical analysis on the PSF is provided in the supplementary materials.

\begin{figure}[H]
	\centering
	\includegraphics[width=\linewidth]{figs/PSF_fire_sq.pdf}
	\caption{\textbf{The asymmetry of PSF.} (a) and (b) illustrate the PSF of a microscopy imaging system, highlighting its asymmetry with respect to the focal plane. (c) demonstrates the defocused imaging relative to the focal plane, and (d) presents the comparison of the gray values of pixels at the same positions corresponding to 5\si{\micro\meter} and -5\si{\micro\meter}. Both of them provide corroborative evidence for the disparities in images at the corresponding locations.}
	\label{fig:psf}
\end{figure}

The asymmetry of the PSF, though potentially detrimental to image quality, presents a unique opportunity for one-shot autofocusing. This phenomenon results in images with positive or negative defocus on either side of the focal plane exhibiting distinct characteristics. Although these differences are subtle, the sophisticated feature extraction capabilities of deep learning can effectively discern them. By capitalizing on this physical principle, we propose a one-shot learning-based network designed to estimate both the defocus distance and direction from a single image.

\subsection{Autofocus for Thick Specimens}

Autofocus is generally designed for a specific focal plane, assuming that most samples exhibit little variation in elevation over a field of view.
However, for very thick samples, such as those resulting from the slicing of pathological sections, different regions within the same field of view may lie on different focal planes (see supplementary material). This can lead to a scenario where focusing on one region causes others to appear blurry, complicating autofocus efforts.
To address such challenges, strategies may contain: 1) Designate a specific region of interest for the autofocus algorithm to target exclusively; 2) Employ z-stack image fusion strategy, capturing and fusing images at various z-axis positions to achieve a uniformly sharp image across the entire field of view.




\section{Threats to Validity}





\textbf{External Validity:} The Goal Management evaluation faces generalization constraints with a limited dataset of 25 predefined goals in the tourism domain. The implementation's focus on Hyderabad's $9$ specific services may restrict generalizability to cities with different infrastructure requirements. While all the of participants showed willingness for future use, the student-only participant pool and short interaction duration (10-15 minutes) limit comprehensive understanding of long-term usage patterns.

\textbf{Internal Validity:} The uniform temperature setting (0.7) across models might not represent optimal individual configurations. CodeBERTScore, justified by standardized service generation, may not fully capture semantic code differences. The fixed three-pass conversation system could potentially miss interaction patterns affecting service discovery accuracy. The service rotation mechanism provides insights but may not represent all production environment failure modes.

\textbf{Construction Validity:} Our service identification approach uses precision and recall metrics against predefined mappings, which may not fully capture user preference variations. The metrics might not account for additional beneficial services or context-specific requirements. While the human evaluation study \((n=15)\) provides real-world validation, its small sample size limits generalizability.


\section{Related Work}
\section{Related Works}
% \subsection{Supervised Fine-Tuning}
% % 指令微调
% % 指令微调对LLM具有重要的作用,具体是什么?
% % 或者模仿Magpie的写法,这一段就纯讲作用,以及对应的工作有哪些

% % 指令微调的作用->sft技术分类->特别介绍conversation based prompt,因为我们也在用
% A series of studies find that if adjusted with annotated "instructional" data, LMs can effectively generate responses aligned with human values~\cite{sanh2022multitask, weifinetuned,ouyang2022training}. The performance of Supervised Fine-Tuning depends not only on the quality of the dataset~\cite{Zhou2023LIMALI} but also on various contextual prompting techniques, such as conversation-based prompts~\cite{sreedhar2024canttalkaboutthis, Wei2023ZeroShotIE}, chain-of-thought~\cite{Wei2022ChainOT}, and contextual calibration~\cite{Zhao2021CalibrateBU}.
% % 因为要对齐deepthink,这边强调一下conversation-based prompts
% Specifically, more models now use conversation-based prompts as the default for QA model deployment~\cite{wu2023brief,liu2024chatqa}, because they enhance the user experience by handling follow-up questions, providing clarifications, and reducing hallucinations.

% 数据合成->分为人工标注和LLM自己生成->人工标注成本高,LLM自己生成会有一些幻觉sample->我们在真实的QA下用rag来避免幻觉并且使用refiner来保证前后topic一致性以及保证数据真实性。(保证数据真实性是因为refiner前后能看到的rag的信息更广,引入更多事实数据)
\subsection{Instruction Data Synthesis}
To address the issue of limited training samples in specific domains, various works have proposed using additional data, such as manual annotation~\cite{Zhao2024WildChat1C,zheng2023lmsys} and automatic generation by LLMs~\cite{Mekala2022LeveragingQD, Wang2021TowardsZL, Wang2022SelfInstructAL, Xu2023WizardLMEL}. However, manual annotation is expensive~\cite{honovich-etal-2023-unnatural}, and iterative generation by LLMs frequently introduces the risk of hallucinations.


Our work falls into the category of automatic generation by LLMs. However, our work differs from previous approaches in two main aspects. (1) We synthesize instructions by simulating conversations closer to real-world scenarios. (2) We adopt several techniques to improve the quality of synthesized instruction. We integrate Retrieval-Augmented Generation (RAG) to mitigate hallucination in conversation-based synthesis. We apply a Conversation-based Data Refiner for filtering, ensuring topic consistency and data authenticity.
% RAG的作用->早期关注于检索器本身->现在专注于when and how ->分别举两个例子验证when and how -> 我们是在sft阶段使用rag的
\subsection{Retrieval-Augmented Generation}
Retrieval augmentation has become a standard solution to address hallucinations in LLMs by introducing external knowledge to compensate for factual shortcomings~\cite{Asai2023SelfRAGLT,ma2023query, Izacard2021UnsupervisedDI, Ram2023InContextRL}.
Early Retrieval Augmentation efforts focus primarily on the retriever itself, where both the neural retriever and generator are typically trainable Pretrained Language Models (PrLMs), such as BERT ~\cite{Devlin2019BERTPO} or BART ~\cite{Lewis2019BARTDS}. In contrast, modern Retrieval Augmentation applied to LLMs emphasizes determining when and how to retrieve relevant information~\cite{fatehkia2024t, Asai2023SelfRAGLT, Xu2024LargeLM}. For example, Self-RAG enables on-demand retrieval and generates more accurate, fact-based text through fine-grained self-reflection~\cite{Asai2023SelfRAGLT}. 
% mHyER bridges the semantic gap between learner input and practice content by generating hypothetical exercises related to the learner's input, 
% thereby improving retrieval relevance~\cite{Xu2024LargeLM}. 

Our approach uses RAG throughout the data synthesis, SFT, and inference stages. This not only improves the authenticity of the synthesized data but also helps the LLM learn how to effectively utilize the retrieved knowledge during the SFT stage. In contrast, previous research only used RAG during the inference stage, relying heavily on the LLM's ability to discern the retrieved knowledge. This can lead to insufficient utilization of relevant knowledge, especially when dealing with domain knowledge that was not included in the pretraining process.




\section{Conclusion \& Future Directions}
\section*{Conclusion}
This paper aims to enhance our understanding of the computational complexity of computing various Shapley value variants. We found that for various ML models --- including decision trees, regression tree ensembles, weighted automata, and linear regression --- both local and global interventional and baseline SHAP can be computed in polynomial time under HMM modeled distributions. This extends popular algorithms, such as TreeSHAP, beyond their empirical distributional scope. We also establish strict complexity gaps between the various SHAP variants (baseline, interventional, and conditional) and prove the intractability of computing SHAP for tree ensembles and neural networks in simplified scenarios. Overall, we present SHAP as a versatile framework whose complexity depends on four key factors: \begin{inparaenum}[(i)] \item model type, \item SHAP variant, \item distribution modeling approach, \item and local vs. global explanations\end{inparaenum}. We believe this perspective provides deeper insight into the computational complexity of SHAP, paving the way for future work.




%We believe that our framework provides a more intricate understanding of SHAP computation complexity across different models, distributions, and variants, paving the way for further research.

Our work opens promising directions for future research. First, expanding our computational analysis to other SHAP-related metrics, such as asymmetric SHAP~\citep{frye20} and SAGE~\citep{covert2020understanding}, would be valuable. Additionally, we aim to explore more expressive distribution classes and relaxed assumptions beyond those in Section \ref{sec:tractable} while maintaining tractable SHAP computation. Finally, when exact computation is intractable (Section \ref{sec:intractable}), investigating the approximability of SHAP metrics through approximation and parameterized complexity theory~\citep{downey2012parameterized} is an important direction.

%Our work opens several promising avenues for future research on the computational properties of explainable AI methods, with a particular focus on SHAP. First, it would be interesting to broaden the computational analysis conducted in this work to include other popular SHAP-related metrics in the literature, such as asymmetric SHAP \cite{frye20} and SAGE \cite{covert2020understanding}. Also, in the future, we aim to explore more expressive distribution classes and relaxed distributional assumptions—extending beyond those examined in Section \ref{sec:tractable} —that still yield tractable SHAP computation. Finally, when exact computation proves intractable (Section \ref{sec:intractable}), it is worthwhile to theoretically investigate the question of the approximability of computing the SHAP metrics across various configurations, through the lens of approximation and parametrized complexity theory \cite{arora2009computational}.

%This paper aims to deepen our understanding of the computational complexity involved in obtaining different Shapley value variants. We found that for a variety of ML models, including decision trees, tree ensembles for regression, weighted automata, and linear regression models — computing both local and global interventional and baseline SHAP can be done in polynomial time when distributions are modeled by HMMs. This extends the distributional scope of popular algorithms like TreeSHAP, which is limited to empirical distributions. Additionally, we demonstrate a strict complexity gap between SHAP variants, showing that interventional and baseline SHAP can be strictly easier to compute than conditional SHAP. Despite these positive results, we uncovered intractability for various SHAP variants in neural networks and tree ensembles. Finally, we provided generalized complexity relations across SHAP variants. We believe that our framework offers a deeper understanding of the complexity involved in computing SHAP across various variants, models, distributions, as well as in both local and global computations, laying the groundwork for future research.


\bibliographystyle{ieeetr}
\bibliography{references}





\end{document}