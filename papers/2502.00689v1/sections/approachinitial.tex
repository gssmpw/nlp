\begin{figure*}[ht]
    \centering
    \includegraphics[width=1.4\textwidth, height=10cm, keepaspectratio]{figures/architecture.drawio.png}
    \caption{High-Level Architecture: System Components and Their Interactions}
    \label{fig:high-level-architecture-overview}
\end{figure*}


Figure \ref{fig:high-level-architecture-overview} presents the proposed system architecture that implements the \textit{IoT-Together paradigm}. The newly proposed components, distinguished by grey shading in the figure, form an integral part of the overall system design. While the \textit{learning management} component remains unimplemented in the current version, it has been identified as a key area for future development. The system facilitates dynamic application reconfiguration based on user goals through comprehensive integration of IoT environmental data and system-level information. The system architecture is designed in a way such that, it supports dynamic evolvability through the generation and integration of new services at run-time in accordance with user goals.

The system adopts key components from the IoT-Together paradigm, including \textit{Goal Management}, \textit{Knowledge Management}, \textit{Context Management}, \textit{Intelligent User Interface (IUI) Generation}, and \textit{Backend Generation}. User interacts with the system using a device (Smart Phone/Laptop/Tablet) through the user interface to enter the query. This query is then passed to the Goal Management, which identifies the set of services that satisfy user goals. The Goal Management uses the LLM to determine the services within the system that can satisfy the user goals or the services that need to be generated. In this phase, the user and system collaboratively (through back-and-forth conversational exchanges) identify the set of parameters and the services required to achieve the user goal.
After the identification of the required services, they are transferred to the Intelligent User Interface Generator to build the application. It locates the services in the Knowledge Management; otherwise, a description of the new service is forwarded to the Backend Generation component, which generates the service, thus extending the architecture runtime according to the user needs. The generated service is sent back to the Intelligent User Interface Generator. The builder within the component integrates these services using a pre-defined template and then creates the website. Intelligent User Interface Generator provides the URL to access this website created. Fig-3 provides the sequence diagram for the process of generation of the application to the user. Meanwhile, the sensors of the IoT environment transmits data periodically to Knowledge Management, which maintains a persistent data repository. This accumulated data is subsequently utilized by the system during the creation of new services and translating the user needs to the goals using the Goal Management.
\\
The User Interface serves as the interaction medium between users and the system, managing conversational exchanges during service identification and preference elicitation. Users submit their initial queries through the interface, and the system engages in a structured conversation to clarify requirements. We represent user queries by $Q_\text{user}$ and system responses by $R_\text{sys}$. For the running example, $Q_\text{user} =$ ``I have 3 hours to explore Hyderabad's old charm" and the $R_\text{sys}$ provides relevant information along with follow-up queries for goal formulation.
Upon successful completion of this dialogue, the interface presents the user with a URL to access the generated web application that implements their specified needs.\\
In the subsequent, following the Fig- \ref{fig:high-level-architecture-overview} we introduce core components and the interactions between them in detail.