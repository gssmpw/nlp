




In this work we advance the IoT-Together paradigm by presenting a goal-driven system utilizing Large Language Models for user-system collaborative applications which can be dynamically composed at runtime. Our system highlights: (1)~a multi-pass dialogue framework translating natural language goals into system requirements, (2)~an LLM-based service generation pipeline with code generation capabilities, and (3)~a template-based interface generation system dynamically adapting to diverse service combinations. Our evaluation in a smart city case study demonstrates the potential of LLMs in enabling mixed-initiative interaction and automated service generation through intelligent user-system collaboration.

Our proposed approach establishes foundation for advancing mixed-initiative IoT architectures that can dynamically adapt while maintaining operational reliability in complex deployment scenarios. While LLM-based components introduce performance variability, deterministic components maintain system stability. Future research directions include exploring deterministic and probabilistic component combinations to improve user-system collaboration, with concrete functional testing to improve reliability. This involves developing LLM-based testing frameworks to validate generated services before integration, requiring testable code generation patterns and enhanced builder capabilities. Given the substantial computational resources of large language models, future work could explore dynamic model selection based on energy constraints, and investigate lightweight alternatives for less complex tasks within the system.
