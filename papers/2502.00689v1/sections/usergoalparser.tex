User Interface forwards user queries to the Goal Management. The LLM in the Goal Management interprets these queries based on three elements: the summary of previous exchanges, the environmental context from Knowledge Management, and the current state of system services. These conversations are modeled to extract maximum information about user goals and preferences. The LLM engages in an iterative dialogue with users to progressively refine the user goals until all requisite services are identified. This extraction process is carried out using prompt templates in LLMs, each designed to identify specific contextual elements at each pass: environmental parameters in the first pass, goal refinement in the second pass, and service requirements in the final pass. User preferences are then applied as parameters to customize the identified services. To enable this information extraction process, we designed a three-pass conversation structure.

\textbf{Pass 1: Contextual Awareness} 
In this pass, the LLM aggregates contextual information $C$ by combining sensor data $D_{\text{sensor}}$ (weather conditions, noise levels, time of day, day of the week, traffic conditions, and air quality), with user-specific data $D_{\text{user}}$ (current location, past activities, and stated preferences as $\text{params}$). A contextual state $X$ maintains the current interaction state, while service registry data from Knowledge Management provides information about available service capabilities. Each service is represented as below:
\[
S_i = ( \text{name}_i, \text{description}_i, \text{params}_i ), \quad \text{for} \quad i = 1, 2, \dots, n
\]
 Realization of values for few services are shown in Table ~\ref{tab:service-definitions}.
\begin{table}[ht]
\setlength{\tabcolsep}{2pt}
\caption{Service Definitions\protect\footnotemark}
\label{tab:service-definitions}
\centering
\begin{tabular}{lll}
\toprule
\textbf{Service Name} & \textbf{Description} & \textbf{Parameters} \\
\midrule
historical\_info & Provides historical and cultural & site\_name \\
 & information about monuments and sites & \\ \hline
restaurant\_finder & Recommends restaurants based on & location, cuisine, \\
 & cuisine preferences and location & diet \\ \hline
crowd\_monitor & Tracks and reports real-time crowd & location, time \\
 & density at various locations & \\ 
\bottomrule
\end{tabular}
\end{table}
\footnotetext{The values presented are illustrative and not representative of entire data. Definitions may also include associated schema, endpoints, and additional metadata relevant to the service.}
Together these services are used as 
\[
D_{\text{services}} = \{ S_1, S_2, \dots, S_n \}
\]
The complete context is defined as \[C = \{D_{\text{sensor}}, D_{\text{user}}, D_{\text{services}}, X\}\]

For example, when a user visits the city with time constraints and specific interests, such as local cuisine, nature activities, or historical sites, their preferences are stored as contextual information.

In the running example, the system identifies the user as a history enthusiast along with the time constraints and captures it via prompt templates.
\[
D_{\text{user}} = \{
\text{current location}, \text{3hrs time}, \text{history enthusiast}
\}
\]
After establishing the complete context, the next pass proceeds with goal identification. \\
\textbf{Pass 2: Goal Formulation and Refinement}
In this pass, the LLM generates Initial Goal Hypotheses based on contextual insights about the user's potential goals $G_{\text{user}}$. These hypotheses range from broad intentions (e.g., leisure activities) to concrete objectives (e.g., specific venue requirements). The system employs \textbf{Mixed-Initiative Interaction}, establishing a collaborative dialogue between the user and system for goal refinement. This refinement process is internally guided by structured prompts that evaluate hypotheses against available services and contextual constraints.

This interaction comprises two key components, each driven by specialized prompt templates:
\begin{itemize}
    \item \textbf{Proactive Suggestions}: The system suggests possible activities based on initial goal hypotheses and context $C$, using prompts that analyze alignment between user preferences and available services
    \item \textbf{Clarification Dialogues}: The system elicits specific preferences and constraints through targeted queries, using prompts designed to resolve ambiguities and gather missing information
\end{itemize}

For the running example, the Goal Management uses these structured prompts to identify historical information services as a potential match. This is represented as:
\[
S_{\text{hist}} = \{
    \text{name}: \text{historical\_info},
    \text{description},
    \text{params}: \text{site\_name}
\}
\]

where $S_{\text{hist}} \in D_{\text{services}}$ represents the identified service matching the user's interests in $D_{\text{user}}$. \\
\textbf{Pass 3: Goal Verification and Confirmation}
In this final pass, the LLM proposes a curated set of services that align with the inferred goals and contextual factors extracted from the prompt templates gathered from \textbf{Pass 1} and \textbf{Pass 2}. If the user rejects the final proposal, the process reverts to \textbf{Pass 1}.

Continuing the running example, the system first seeks confirmation for the historical information service. Upon user request for additional dining options, the system extends the service set as follows:
\[
G_{\text{users}} = \{ S_{hist},S_{restau}
\}
\]

where $S_{\text{hist}}$ and $S_{\text{restau}}$ follow from Table~\ref{tab:service-definitions} for Historical Information and Restaurant Finder services respectively, with parameters instantiated based on user preferences. In our running example, for the historical information service and restaurant finder, these parameters realize the values based on user conversations with system specifying their requirements and preferences.\\ 
\(
\theta_q = \{
site\_name = 
    \text{[Charminar} \text{, Laad Bazaar]},
\)
\(
location = 
    \text{Laad Bazaar}, 
cuisine = \text{Any},
\)
\(
diet=\text{Non-vegetarian}\}
\)
\\The system maintains a history data base $H$  along with contextual information throughout these passes which include:

\[
H = \{P_{\text{user}}, S_{\text{identified}}, C_{\text{prev}}\}
\]

where $P_{\text{user}}$ represents user preferences, $S_{\text{identified}}$ tracks identified services, and $C_{\text{prev}}$ maintains the contextual history of previous interactions.
\begin{figure*}[ht]
    \centering
    \includegraphics[width=\textwidth]{figures/sequence_research.drawio_5.png}
    \label{sequence}
    \caption{Sequence diagram for creating the web application }
\end{figure*}
