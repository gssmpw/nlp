




\textbf{External Validity:} The Goal Management evaluation faces generalization constraints with a limited dataset of 25 predefined goals in the tourism domain. The implementation's focus on Hyderabad's $9$ specific services may restrict generalizability to cities with different infrastructure requirements. While all the of participants showed willingness for future use, the student-only participant pool and short interaction duration (10-15 minutes) limit comprehensive understanding of long-term usage patterns.

\textbf{Internal Validity:} The uniform temperature setting (0.7) across models might not represent optimal individual configurations. CodeBERTScore, justified by standardized service generation, may not fully capture semantic code differences. The fixed three-pass conversation system could potentially miss interaction patterns affecting service discovery accuracy. The service rotation mechanism provides insights but may not represent all production environment failure modes.

\textbf{Construction Validity:} Our service identification approach uses precision and recall metrics against predefined mappings, which may not fully capture user preference variations. The metrics might not account for additional beneficial services or context-specific requirements. While the human evaluation study \((n=15)\) provides real-world validation, its small sample size limits generalizability.
