\subsection{\textbf{Intelligent User Interface Generator}}

The Intelligent User Interface Generator transforms the identified user goals into a functional web application through three primary components: Service Discovery, Builder, and Hosting. 

\[
G_{user} = \{SI_1, SI_2, ..., SI_n\}
\]

$G_{user}$ represents the set of services identified from the goal parser phase, where each $SI_i$ represents a service with its parameters and requirements, where each $\text{SI}_i$ represents an identified service. The Service Discovery component matches between $G_{user}$ and available services in the service registry. For each service $SI_i \in G_{user}$, it searches for matches $M_i \subseteq D_{services}$ by evaluating service descriptions and parameter compatibility:



\[
M = \bigcup_{i=1}^{n} M_i,\]\[ \quad M_i = \{ S_j \mid \text{match}(\text{description}_{\text{SI}_i}, \text{description}_{S_j}) \]
\[\land \text{params}_{\text{SI}_i} \subseteq \text{params}_{S_j} \}
\]

where $\text{match}()$ represents the semantic matching function which compares the service descriptions. All the matches are forwarded to the builder for application generation. When $M_i = \emptyset$ for any $SI_i$, indicating that no matching service exists, the system initiates service generation. The Backend Generation creates a new service $S_{new}$ using the service description and available data schemas and returns it to the Builder.


Builder implements UI generation process which employs a hierarchical rendering system $R$ with specific handlers for different data types, some of which are listed below:

\[
R(d) = \begin{cases}
    R_{\text{metric}}(d), & \text{if } d \in \mathbb{R} \\
    R_{\text{list}}(d), & \text{if } d \in \text{List} \\
    R_{\text{dict}}(d), & \text{if } d \in \text{Dict} \\
    R_{\text{text}}(d), & \text{otherwise}
\end{cases}
\]

where each renderer implements specific visualization logic for its data type (e.g., $R_{\text{metric}}$ for crowd density, $R_{\text{list}}$ for restaurant listings). The Builder Component generates the complete application by combining these renderers with a base template $T$, where $\oplus$ represents the composition operator that combines multiple rendered components:



\[
App = T(M \cup \{S_{new}\}) \oplus \bigoplus_{i=1}^{n} R(SI_i)
\]












The final application is hosted through the Hosting Component, which manages service endpoints and provides URL access to users. 


