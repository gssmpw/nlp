To motivate our work and demonstrate the realization of the IoT-Together paradigm, we present a smart city system developed for the city of Hyderabad, India. The system has been implemented at the Smart City Living Lab of The International Institute of Information Technology Hyderabad, building upon the collaborative principles established in the IoT-Together paradigm~\footnote{\url{https://smartcityresearch.iiit.ac.in}}. The existing smart city infrastructure comprises a network of IoT sensors that monitor crowd density in specific regions, air quality at public spaces, and water quality at heritage sites. Currently, visitors to the city need to use multiple separate applications to access information about restaurant availability, cultural heritage site bookings, and other services. This fragmentation becomes particularly challenging when visitors have uncertain or evolving objectives - a key scenario that the IoT-Together paradigm aims to address. Planning activities requires complex decision-making that considers user preferences, time constraints, and environmental conditions.

We provide a system consisting of a web application that integrates with the city's IoT infrastructure, collecting contextual data from various urban locations. The sensor network provides real-time data on crowd density, air quality, and water quality measurements at public places such as forts, museums, historical sites, restaurants, hospitals. Built upon this sensor infrastructure are several services: event notifications, historical site information systems, crowd monitoring, air quality assessment tools, exhibition tracking, restaurant recommendation engines, ticket purchasing platforms, travel planning tools, and water quality monitoring services. These services highlight the opportunity for applying IoT-Together's collaborative approach in helping users to achieving their goals.

\noindent In the following section to explain our approach, we consider an instance of the above case study, where a tourist visiting the smart city has a limited time window of three hours to explore the historical and cultural attractions.

 









