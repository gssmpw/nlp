\subsection{Compressed Posterior Sampling for Image Restoration}\label{appendix:zero-shot}
DPS and DDNM are implemented with the official settings recommended by the authors~\citep{chung2023diffusion,wang2023zeroshot}.
Specifically, DPS uses DDPM with $T=1000$ sampling steps, and DDNM uses DDIM with $\eta=0.85$ and $T=100$ sampling steps.
We also tried $T=1000$ for DDNM and found that $T=100$ works slightly better for the tasks considered.
The additional qualitative comparisons in \Cref{fig:srx4_additional_samples,fig:colorization_additional_samples} further demonstrate that our method produces better output perceptual quality compared to DPS and DDNM.
\begin{figure*}[t]
    \centering
    \includegraphics[width=1\textwidth]{figures/srx4_additional_samples.pdf}
    \caption{\textbf{Qualitative comparison of zero-shot image super-resolution methods (posterior sampling).} Our approach clearly produces better output perceptual quality compared to previous methods.}
    \label{fig:srx4_additional_samples}
\end{figure*}
\begin{figure*}[t]
    \centering
    \includegraphics[width=1\textwidth]{figures/colorization_additional_samples.pdf}
    \caption{\textbf{Qualitative comparison of zero-shot image colorization methods (posterior sampling).} Our approach clearly produces better output perceptual quality compared to previous methods.}
    \label{fig:colorization_additional_samples}
\end{figure*}