%%%%%%%% ICML 2025 EXAMPLE LATEX SUBMISSION FILE %%%%%%%%%%%%%%%%%

\documentclass{article}

\newcommand{\CG}{\mathcal{G}\xspace}
\newcommand{\CV}{\mathcal{V}\xspace}
\newcommand{\CE}{\mathcal{E}\xspace}
\newcommand{\CA}{\mathcal{A}\xspace}
\newcommand{\CF}{\mathcal{F}\xspace}
\newcommand{\CR}{\mathcal{R}\xspace}
\newcommand{\CB}{\mathcal{B}\xspace}
\newcommand{\CX}{\mathcal{X}\xspace}
\newcommand{\CK}{\mathcal{K}\xspace}
\newcommand{\CM}{\mathcal{M}\xspace}
\newcommand{\CC}{\mathcal{C}\xspace}
\newcommand{\CL}{\mathcal{L}\xspace}
\newcommand{\CI}{\mathcal{I}\xspace}
\newcommand{\CQ}{\mathcal{Q}\xspace}
\newcommand{\CO}{\mathcal{O}\xspace}
\newcommand{\CP}{\mathcal{P}\xspace}
\newcommand{\CS}{\mathcal{S}\xspace}
\newcommand{\CT}{\mathcal{T}\xspace}
\newcommand{\CJ}{\mathcal{J}\xspace}
\usepackage[para]{footmisc}
\usepackage{subfig}
% \usepackage{subcaption}
% \usepackage{array}
% \usepackage{colortbl}



% The \icmltitle you define below is probably too long as a header.
% Therefore, a short form for the running title is supplied here:
\icmltitlerunning{\dynamictitle}

\begin{document}

\twocolumn[
\icmltitle{
\dynamictitle
}

% It is OKAY to include author information, even for blind
% submissions: the style file will automatically remove it for you
% unless you've provided the [accepted] option to the icml2025
% package.

% List of affiliations: The first argument should be a (short)
% identifier you will use later to specify author affiliations
% Academic affiliations should list Department, University, City, Region, Country
% Industry affiliations should list Company, City, Region, Country

% You can specify symbols, otherwise they are numbered in order.
% Ideally, you should not use this facility. Affiliations will be numbered
% in order of appearance and this is the preferred way.
\icmlsetsymbol{equal}{*}

\begin{icmlauthorlist}
\icmlauthor{Guy Ohayon}{equal,techcs}
\icmlauthor{Hila Manor}{equal,techee}
\icmlauthor{Tomer Michaeli}{techee}
\icmlauthor{Michael Elad}{techcs}
%\icmlauthor{}{sch}
%\icmlauthor{}{sch}
\end{icmlauthorlist}

\icmlaffiliation{techcs}{Faculty of Computer Science, Technion -- Israel Institute of Technology, Haifa, Israel}
\icmlaffiliation{techee}{Faculty of Electrical and Computer Engineering, Technion -- Israel Institute of Technology, Haifa, Israel}

\icmlcorrespondingauthor{Guy Ohayon}{guyoep@gmail.com}
\icmlcorrespondingauthor{Hila Manor}{hila.manor@campus.technion.ac.il}

% You may provide any keywords that you
% find helpful for describing your paper; these are used to populate
% the "keywords" metadata in the PDF but will not be shown in the document
\icmlkeywords{Machine Learning, ICML}

% {\vspace{0.75em}%
% \centering
% \includegraphics[width=1\textwidth]{figures/teaser.pdf}
% \vspace{-1em}
% \captionof{figure}{Our proposed scheme (DDCMs) produces visually appealing image samples with high compression ratios (given at the bottom-right corner of each result). \textbf{Left}: Compression result; \textbf{Middle}: Compressed image synthesis; and \textbf{Right}: Real-world image restoration.  
% \vspace{-0.75em}}
% \label{fig:teaser}
% }
\vskip 0.3in

]



% this must go after the closing bracket ] following \twocolumn[ ...

% This command actually creates the footnote in the first column
% listing the affiliations and the copyright notice.
% The command takes one argument, which is text to display at the start of the footnote.
% The \icmlEqualContribution command is standard text for equal contribution.
% Remove it (just {}) if you do not need this facility.

%\printAffiliationsAndNotice{}  % leave blank if no need to mention equal contribution
\printAffiliationsAndNotice{\icmlEqualContribution} % otherwise use the standard text.

\begin{abstract}
Retrieval-Augmented Generation (RAG) is often used with Large Language Models (LLMs) to infuse domain knowledge or user-specific information. In RAG, given a user query, a retriever extracts chunks of relevant text from a knowledge base. These chunks are sent to an LLM as part of the input prompt. Typically, any given chunk is repeatedly retrieved across user questions. However, currently, for every question, attention-layers in LLMs fully compute the key values (KVs) repeatedly for the input chunks, as state-of-the-art methods cannot reuse KV-caches when chunks appear at arbitrary locations with arbitrary contexts. Naive reuse leads to output quality degradation.  This leads to potentially redundant computations on expensive GPUs and increases latency. In this work, we propose \sys, a system for managing and reusing precomputed KVs corresponding to the text chunks (we call \textit{chunk-caches}) in RAG-based systems. We present how to identify \hl{\textit{chunk-caches} that are reusable}, how to efficiently perform a small fraction of recomputation to \textit{fix} the cache to maintain output quality, and how to efficiently store and evict \textit{chunk-caches} in the hardware for maximizing reuse while masking any overheads. With real production workloads as well as synthetic datasets, we show that \sys reduces redundant computation by \textbf{51\%} over SOTA prefix-caching and \textbf{75\%} over full recomputation.
\hl{Additionally, with continuous batching on a real production workload, we get a \textbf{1.6$\times$} speedup in throughput and a \textbf{2$\times$} reduction in end-to-end response latency over prefix-caching while maintaining quality, for both the \llama-3-8B and \llama-3-70B models. 
}
\end{abstract}





\section{Introduction}
\label{sec:intro}

\begin{figure*}[tb]
    \centering
    \includegraphics[width=0.848\linewidth]{figs/circuitnn.pdf} 
    \caption{Illustration of differentiable CircuitNN. CircuitNN is designed based on differentiable NAND gates. After DAS is guided by PI and PO pairs of the truth table, CircuitNN can get the precise circuit architecture logic equivalent to the truth table.}
    \label{fig:circuitnn}
\end{figure*}

% 1. Describe the importance of logic synthesis
% 2. Existing Problems
% (a) Neural Architecture Search: Unstable, Predefined Setting, etc.
% (b) Circuit Generation: Probabilistic Model, Logic Equivalence

With the rapid advancement of technology, the scale of integrated circuits (ICs) has expanded exponentially. 
This expansion has introduced significant challenges in chip manufacturing, particularly concerning power and area metrics.
A primary objective in IC design is achieving the same circuit function with fewer transistors, thereby reducing power usage and area occupancy.

Logic synthesis~\cite{hachtel2005logicsynth}, a critical step in electronic design automation (EDA), transforms behavioral-level circuit designs into optimized gate-level circuits, ultimately yielding the final IC layout. 
The primary goal of logic synthesis is to identify the physical implementation with the fewest gates for a given circuit function. 
This task constitutes a challenging NP-hard combinatorial optimization problem. 
Current logic synthesis tools~\cite{brayton2010abc, wolf2013yosys} rely on human-designed heuristics, often leading to sub-optimal outcomes.

Differentiable architecture search (DAS) techniques~\cite{liu2018darts, chu2020darts} offer novel perspectives on addressing challenges in this problem.
Circuit functions can be represented through truth tables, which map binary inputs to their corresponding outputs. 
Truth tables provide a precise representation of input-output relationships, ensuring the design of functionally equivalent circuits.
Inspired by this, researchers~\cite{deepmind2024ai4sys, wang2024tnet} have begun exploring the application of DAS to synthesize circuits directly from truth tables.
Specifically, \citet{deepmind2024ai4sys} proposed CircuitNN, a framework that learns differentiable connection structures with logic gates, enabling the automatic generation of logic circuits from truth tables.
This approach significantly reduces the complexity of traditional circuit generation. 
Building on this, \citet{wang2024tnet} introduced T-Net, a triangle-shaped variant of CircuitNN, incorporating regularization techniques to enhance the efficiency of DAS.

Despite these advancements, several challenges remain. 
The computational complexity of DAS grows quadratically with the number of gates, posing scalability issues.
Although triangle-shaped architecture~\cite{wang2024tnet} partially mitigates this problem, redundancy persists. 
%Additionally, DAS is susceptible to converging to local optima, limiting the ability to search architectures that satisfy the given truth tables~\cite{liu2018darts}. 
%Furthermore, hyperparameters (network depth and layer width) require extensive searches, introducing complexity and prolonging the synthesis process. 
Additionally, DAS is susceptible to converging to local optima~\cite{liu2018darts} and hyperparameters (network depth and layer width) require extensive searches. 
The challenges arise from the vast search space in DAS. 
% Even with predefined settings for CircuitNN, finding a configuration that meets the truth table requires extensive trial and error during the DAS process. 
Intuitively, limiting the search space through predefined parameters (network depth, gates per layer, and connection probabilities) can significantly reduce the complexity.

Recent advances~\cite{openai2023gpt4, abramson2024alphafold3, esser2024sd3, li2024mar} in conditional generative models have demonstrated remarkable performance across language, vision, and graph generation tasks. 
Motivated by these developments, we propose a novel approach to circuit generation that generates preliminary circuit structures to guide DAS in generating refined circuits matching specified truth tables. 
Firstly, we introduce CircuitVQ, a tokenizer with a discrete codebook for circuit tokenization. 
Built upon our Circuit AutoEncoder framework~\cite{hou2022graphmae,li2023maskgae,wu2025mgvga}, CircuitVQ is trained through a circuit reconstruction task. 
Specifically, the CircuitVQ encoder encodes input circuits into discrete tokens using a learnable codebook, while the decoder reconstructs the circuit adjacency matrix based on these tokens.
Subsequently, the CircuitVQ encoder serves as a circuit tokenizer for CircuitAR pretraining, which employs a masked autoregressive modeling paradigm~\cite{chang2022maskgit, li2023mage}. 
In this process, the discrete codes function as supervision signals. 
After training, CircuitAR can generate discrete tokens progressively, which can be decoded into initial circuit structures by the decoder of the CircuitVQ. 
These prior insights can guide DAS in producing refined circuits that match the target truth tables precisely.

Our key contributions can be summarized as follows:
\begin{itemize}
\item We introduce CircuitVQ, a circuit tokenizer that facilitates graph autoregressive modeling for circuit generation, based on our Circuit AutoEncoder framework;
\item Develop CircuitAR, a model trained using masked autoregressive modeling, which generates initial circuit structures conditioned on given truth tables;
\item Propose a refinement framework that integrates differentiable architecture search to produce functionally equivalent circuits guided by target truth tables;
\item Comprehensive experiments demonstrating the scalability and capability emergence of our CircuitAR and the superior performance of the proposed circuit generation approach.
\end{itemize}

% Motivation
% (a) Diffusion (Vision, Graph), Autoregressive (Language, Vision)
% (b) Circuit Generation for Predefined Setting
% (c) Neural Architecture Search for Strict Logic Equivalence

% Contribution
% (a) Circuit Tokenizer (new transformer arch, training strategy)
% (b) CircuitAR (train and gen strategies, post-ar strategy)
% (c) Extensive Evaluation including BitD (Bit Distance) for Scalability


\section{Related Work} \label{sec:related}

% \textbf{Adversarial Attack}
\textbf{Attacks on SLAM.} 
%With the rise of machine learning, 
The robustness of computer vision systems is being actively investigated. With the emergence of adversarial images in the digital domain by adding optimized noise directly to images~\cite{szegedy2013intriguing,carlini2017towards}, researchers find that such attacks also exist physically in the real world \cite{eykholt2018robust,song2018physical,zhao2019seeing}. To fill the gap between attacks in the digital and physical worlds, recent studies have demonstrated that attacks on real-world computer vision systems are practical \cite{eykholt2018robust,li2019adversarial,man2020ghostimage,sharif2016accessorize,zhao2019seeing,zhou2018invisible}. However, attacks on traditional computer vision methods such as SLAM are relatively less explored. \cite{yoshida2022adversarial} proposes an attack against the scan matching algorithm in LiDAR-based SLAM, while most SLAMs in AR/VR devices rely on different sensors like RGB/depth cameras and IMUs. \cite{ikram2022perceptual} and \cite{chen2024adversary} mislead visual SLAM by poisoning the images with special patterns, and \cite{wang2021can} causes the camera to fail using infrared light. In our work, we demonstrate attacks on Visual-Inertial SLAM (VI-SLAM) by perturbing the IMU readings, rather than cameras, and showing its impact on XR user experience. 

\textbf{Acoustic Injection Attacks.} Among various physical attacks, acoustic injection attacks are attractive due to their low cost. Son~\etal~\cite{son2015rocking} were the first to introduce acoustic attacks on MEMS gyroscopes, demonstrating how these attacks could lead to sensor denial-of-service and result in drone crashes. WALNUT~\cite{trippel2017walnut} expanded on this by developing output biasing and control attacks that enable precise manipulation of MEMS accelerometer outputs using modulated sound waves. Wang et al.~\cite{wang2017sonic} demonstrated a sonic gun, showcasing the vulnerability of various smart devices (\eg drones and self-balancing vehicles) to acoustic attacks. Tu et al. \cite{tu2018injected} designed side-swing and switching attacks to alter the outputs of MEMS gyroscopes and accelerometers. Furthermore, Ji et al. \cite{ji2021poltergeist} fool the object detectors by applying acoustic attack to the image stabilizers commonly used in modern cameras. However, none of the existing works study the relationship between the acoustic injections and SLAM outputs on recent XR devices. 

% \zijian{Do we need one session about security in AR/VR?}
% \yicheng{TODO}
%\jiasi{cite the AIVR paper (UMass Amherst?) paper is we have not already. They add IMU perturbation but w/o SLAM, iirc} \yicheng{Cited}

\textbf{XR Security and Privacy.} 
%Security and privacy concerns in XR systems have gained significant attention. 
For single-user XR systems, researchers have demonstrated various side-channel attacks to extract sensitive information (\eg keystrokes) through video feeds~\cite{ling2019know}, head movements~\cite{nair2023unique, slocum2023going}, architectural hints~\cite{zhang2023its,shang2020arspy}, power usage~\cite{li2024dangers}, and EM side-channel leakages~\cite{al2021vr}. In multi-user XR systems, Su et al.~\cite{su2024remote} use avatar motion data to infer keystrokes in shared VR environments. Slocum et al.~\cite{slocum2024doesn} reveal vulnerabilities in the shared state frameworks of multi-user AR. Similarly, Lebeck et al.~\cite{lebeck2017securing} highlight risks like deceptive virtual objects and emphasize access control for managing shared physical and virtual spaces. Ruth et al.~\cite{ruth2019secure} further propose a secure multi-user AR framework focusing on content sharing and permissions.
Chandio et al.~\cite{chandio2024stealthy} %introduced a multi-modal spatiotemporal attack that 
simultaneously manipulated visual and inertial sensors to disrupt XR pose estimation. However, their study evaluated the attack using offline datasets and assumed the attacker's capability to manipulate IMU data streams through acoustic means, without real experiments. Ours is the first to demonstrate acoustic injection attacks on recent XR devices, like the Hololens 2, in the real world.
 


\section{Basic Background: Supervised Learning and the PAC Model}
\label{sec:background}

At this point almost everyone has heard of machine learning (ML). Anyone likely to stumble upon this article will have also heard of its most influential special case, supervised learning, and those theoretically inclined will also be familiar with the PAC model. Nonetheless, I will set the stage by  recapping the basics.

\subsection{Basics of Supervised Learning}%Let's set the stage in any case

\emph{Supervised Learning} is the task of ``coming up'' with a function $f: \X \to \Y$ to ``explain'' or ``fit'' a sequence of input/output examples   $(x_1,y_1), \ldots, (x_n,y_n)$, with $x_i \in \X$ and $y_i \in \Y$.  Here $\X$ is a \emph{data domain} consisting of \emph{datapoints} $x \in \X$, $\Y$ is a \emph{label set} consisting of \emph{labels} $y \in \Y$, and the sequence $(x_1,y_1),\ldots,(x_n,y_n)$ is the \emph{training data} consisting of \emph{labeled examples (a.k.a. samples)}~$(x_i,y_i)$.  I~will refer to the chosen function $f$ as a \emph{predictor}, and to $n$ as the \emph{sample size}. A \emph{learning algorithm} takes as input training data, and outputs (some representation of) a predictor $f \in \Y^\X$.\footnote{Note that this describes the usual \emph{batch}, a.k.a.~\emph{offline}, setting of supervised learning. I do not discuss other paradigms such as online or active learning in this article.} 



Success in supervised learning is defined as \emph{generalization} to  future examples: For a typical \emph{test example}  $(x_{\tst},y_{\tst})$, the predicted label $y'_{\tst}=f(x_{\tst})$ should ``equal'' $y_{\tst}$, perhaps approximately. We usually assume the test example is drawn from the same  ``source'' as the training data  --- commonly, i.i.d.~from the same distribution. The quality of the prediction is quantified by $\ell(y'_{\tst},y_{\tst})$, where $\ell:~\Y~\times~\Y \to \RR_{\geq 0}$ is a \emph{loss function} chosen as part of the problem definition. Common loss functions include the 0-1 loss $\ell_{0-1}(y',y) = [y' \neq y]$ for \emph{classification} problems,\footnote{The notation $[P]$ denotes $1$ when predicate $P$ is true, and denotes $0$ when $P$ is false.} as well as the absolute loss $|y'-y|$ or squared loss $(y'-y)^2$ for \emph{regression problems} featuring $\Y  \sse \RR$.

Nontrivial generalization properties are typically only possible if one assumes something about the data.\footnote{The need for such an assumption is formalized by the  \emph{no free lunch theorems} of supervised learning \cite{wolpert_connection_1992,wolpert_lack_1996,schaffer_conservation_1994}.} The Bayesian approach to  machine learning, common in many applications, assumes some parametric form for the distribution generating the data, and postulates a prior on the parameters. This is not the approach I will take in this article. Instead, I will focus on the frequentist --- and some would say ``worst-case'' or ``adversarial'' ---  approach that is common in the computational learning theory community, embodied by the PAC model. Here we assume that the (training and test) data can be explained, perhaps approximately, by a function in some ``simple enough to learn'' class of functions $\H \sse \Y^\X$, often called the \emph{hypotheses}. Equivalently, we  seek a predictor which explains the unseen data roughly  as well as the best hypothesis $h^* \in \H$, whether or not we assume that $h^*$ itself provides a perfect explanation.



 \paragraph{Common Algorithmic Templates.} Perhaps the best known general-purpose supervised learning algorithm is \emph{empirical risk minimization (ERM)}, which chooses as its predictor a hypothesis $f \in \H$ minimizing $\frac{1}{n} \sum_{i=1}^n \ell(f(x_i),y_i)$ --- a quantity called the \emph{training error}, \emph{empirical error}, or \emph{empirical risk} of $f$. %\footnote{When multiple hypotheses minimize the empirical risk, we assume ERM breaks ties arbitrarily.}
A common template for generalizing ERM involves adding a \emph{regularization term} $\psi(f)$ to the  objective function, typically chosen to measure some notion of ``hypothesis complexity.'' An algorithm instantiating this template is known as a \emph{structural risk minimizer (SRM)}, and chooses as its predictor the hypothesis $f \in \H$ minimizing the \emph{structural risk} $\frac{1}{n} \sum_{i=1}^n \ell(f(x_i),y_i) + \psi(f)$. Other well-known algorithms, such as gradient descent and its variations,  can frequently be interpreted as approximate implementations of ERM or SRM.


\paragraph{Proper vs Improper Learning.} A learning algorithm is said to be \emph{proper} if its predictor $f$ is always chosen from the hypothesis class, i.e., $f \in \H$, otherwise it is said to be \emph{improper}. ERM  is an example of a proper learning algorithm, as are SRM algorithms of the form described above.  In the \emph{proper regime} of learning, algorithms are required to be proper. This article will be concerned with the more flexible \emph{improper regime} (a.k.a \emph{representation-independent learning}), where no such constraint is placed on the learner. In other words, all we care about is predictive power at test time, rather than any insights derived from the functional form or representation of the predictor~itself.


\subsection{The PAC Model}
A standard mathematical setup for evaluation of supervised learning algorithms, at least in the theoretical computer science community, is Valiant's \emph{Probably Approximately Correct (PAC) model} of learning (see e.g.~\cite{kearns_introduction_1994,mohri_foundations_2018}). Here, we assume there is an unknown distribution $\D$ on $\X \times \Y$ from which training and test data are  drawn.  Specifically, the labeled datapoints of the training set  $(x_1,y_1), \ldots, (x_n,y_n)$, as well as the test data  $(x_\tst,y_\tst)$, are i.i.d.~from $\D$. Often it is assumed that $\D$ lies in some class of distributions of interest. The \emph{true expected loss}, or simply \emph{loss}, of a predictor $f: \X \to \Y$ is the expected loss it incurs on draws from $\D$, written $L_\D(f) = \Ex_{(x,y) \sim \D} \ell(f(x),y)$.


There are two main ``settings'' in PAC learning. The  \emph{realizable setting} only requires that the data be perfectly explained by some hypothesis in $\H$. More generally, the \emph{agnostic setting} makes no assumption relating the data to the hypotheses, but shifts the goalposts as necessary to allow nontrivial guarantees: the expected loss at test time is evaluated only ``relative'' to that of the best hypothesis $h^* \in \H$. There are other settings which make more nuanced assumptions, such as $\D$ being of a particular parametric form or its support living in some (unknown) lower-dimensional space, etc. I will mostly discuss the realizable and agnostic settings in this article, those being the simplest and most studied from a theoretical perspective. %TODO:We will briefly discuss other settings in Section ??

The PAC model demands high probability guarantees of learners, in the worst case over distributions of interest. Consider first the realizable setting, where $\D$ is such that $\min_{h \in \H} L_{\D}(h) = 0$. A PAC learner has \emph{error} $\epsilon=\epsilon(n)$ and \emph{confidence} $\delta=\delta(n)$ if, when training data consists of $n$ i.i.d~samples from a realizable distribution $\D$, it produces a predictor $f$  satisfying $L_\D(f) \leq \epsilon$ with probability at least $1-\delta$. In the agnostic setting, where $\D$ can be arbitrary, we require $L_\D(f) - \min_{h \in \H} L_\D(h) \leq \epsilon$ with probability $1-\delta$.

In both the realizable and agnostic settings, we look for PAC learners with small $\epsilon$ and $\delta$ as a function of the sample size $n$. An equivalent perspective looks at the sample complexity $m(\epsilon,\delta)$, which is the minimum sample size which guarantees error  at most $\epsilon$ with probability at least $1-\delta$. We say a problem is \emph{PAC learnable} if its PAC sample complexity is finite whenever $\epsilon,\delta > 0$.

For most PAC learning problems, learnability and sample complexity are characterized in terms of a  ``dimension'' of the hypothesis class. Most prominently this is the \emph{VC dimension} for binary classification, the \emph{fat shattering dimension} for agnostic regression, and the \emph{DS dimension} for multiclass classification (see \cite{anthony_neural_1999,daniely_optimal_2014,brukhim_characterization_2022}). Treatment of these is beyond the scope of this article. The unfamiliar reader need not worry, however,  as dimensions will feature only tangentially in our~discussion.




%\paragraph{Learning settings: Realizable, Agnostic, etc.} In learning theory, evaluating a supervised learning algorithm requires specifying a data model and an objective. We will leave the details of the data model flexible for now, to allow for both the PAC model and the adversarial transductive model. Nonetheless we will describe two variations, which we call ``settings'', which cut across different models. The  \emph{realizable setting}  requires only that the data be perfectly explained by some hypothesis $h \in \H$ --- i.e., there exists a hypothesis which is guaranteed to suffer a loss of $0$ on training and test data. The performance of the learning algorithm is its expected loss at test time for some ``worst case'' realizable instance. More generally, the \emph{agnostic setting} makes no assumption relating the data to the hypotheses, but shifts the goalposts as necessary to allow nontrivial guarantees: the expected loss at test time is evaluated only ``relative'' to that of the best hypothesis $h^* \in \H$, again for some ``worst case'' instance. There are other settings which make more nuanced assumptions about the data, such as it is drawn from a distribution of a particular parametric form, or that it lives in some (unknown) lower-dimensional space, etc. We will mostly discuss the realizable and agnostic settings, those being the simplest and most studied from a theoretical perspective.




%%% Local Variables:
%%% mode: latex
%%% TeX-master: "learning_matching"
%%% End:

% \begin{figure}
%     \centering
%     \includegraphics[width=0.5\linewidth]{Move_teaser.pdf}
%     \caption{Comparison of different dynamic compute approaches. length of arrow indicates residual transformation per token while width indicates velocity of transformation.}
%     \label{fig:enter-label}
% \end{figure}

\section{Method}
\label{sec:method}
Residual connections play a crucial role in shaping token representations, yet their dynamics remain underexplored in the context of efficient decoding. In this work, we delve deeper into transformer residual dynamics and investigate how modulating residual transformation velocity can improve inference efficiency in token-level processing, optimizing both dense and sparse MoE transformers.


\subsection{Residual Dynamics and Motivation for Multi-rate Residuals} \label{sec:motivation}

To analyze how hidden representations evolve across different layers of a transformer architecture, it's crucial to consider the effect of residual connections. Each transformer decoder layer typically has residual connections across attention and MLP submodules. As the residual stream $h_i$ traverses from interval $E_j$ to $E_{j+1}$, it undergoes a residual transformation given by:  
% \begin{equation}
% \label{eq:slow_residual_transformation}
% H_{E_{j+1}} = H_{E_j} \prod_{i=E_j}^{E_{j+1}} \left( I + \mathcal{A}_i \right) \left( I + \mathcal{M}_i \right) \quad \text{where} \quad \mathcal{A}_i = f(c_i, h_{i}), \mathcal{M}_i = g(h_i)
% \end{equation}

\begin{equation} \label{eq:slow_residual_transformation}
h_{E_{j+1}} = h_{E_j} + \sum_{i=E_j}^{E_{j+1}-1} \left( \mathcal{A}_i(h_i) + \mathcal{M}_i(h_i + \mathcal{A}_i(h_i)) \right) \quad \text{where} \quad \mathcal{A}_i = f(c_i, h_{i}), \mathcal{M}_i = g(h_i). 
\end{equation}

Here, \( \mathcal{A}_i \) denotes the non-linear transformation introduced by the multi-head attention mechanism at layer \( i \), while \( \mathcal{M}_i \) corresponds to the non-linear transformation of the MLP block at the same layer. These transformations depend on the input residual stream \( h_i \) and, in the case of \( \mathcal{A}_i \), the previous contextual representation \( c_i \).\footnote{Normalization layers are typically applied in practice but are omitted here for simplicity of the argument.}


% For easy tokens, the magnitude and direction of this delta transformation become progressively smaller with each successive layer as shown in \cref{fig:delta_transformation}. Consequently, it is feasible to predict these tokens after only a few residual connections, whereas harder tokens necessitate more extensive processing through additional layers.

\begin{figure}[ht]
    \centering
    \begin{subfigure}{0.48\textwidth}
        \centering
        \includegraphics[width=\textwidth]{sections/figures/residual_change.pdf}
        \caption{}
        \label{fig:residual_change}
    \end{subfigure}%
    \hfill
    \begin{subfigure}{0.48\textwidth}
        \centering
        \includegraphics[width=\textwidth]{sections/figures/alignment_wrt_dedicated_model.pdf}
        \caption{}
    \label{fig:alignment_wrt_dedicated_model}
    \end{subfigure}
    \caption{(a) As residual streams propagate through the model, the directional shifts in the residuals become progressively smaller. (b) A dedicated model with $k$ layers achieves a faster rate of change in residual streams and higher alignment than base model leveraging early exit mechanisms at layer $k$.}
    \label{fig}
\end{figure}


To examine whether residual transformations can be accelerated across layers, we conducted experiments using a diverse set of prompts on a pre-trained Phi3 model~\cite{phi3_report}. As illustrated in \cref{fig:residual_change}, we measured the directional shift in residual states as \( 1 - \mathcal{C}(h_{i-1}, h_i) \), where \(\mathcal{C}\) denotes normalized cosine similarity. This shift is notably higher in the initial layers, gradually decreasing in subsequent layers. This behavior allows traditional early exit approaches to effectively accelerate decoding by enabling earlier exits for simpler tokens. However, these approaches typically rely on a distance-based approximation, where the full residual transformation of the model is approximated by the residual transformations of the initial layers. To gain deeper insights into the distance versus velocity aspects of residual transformation, we conducted a comparative study. Specifically, we trained an early exit head at layer $k$ of the Phi3 model, which consists of 32 layers, restricting the distance traveled by each token. To accelerate the residual transformation relative to number of layers, we trained a smaller model consisting of only $k$ layers, while keeping all other hyperparameters consistent. We then compared the next-token prediction accuracy of the early exit head of the base model with that of the smaller model. To ensure an equal number of trainable parameters, we inserted low-rank adapters into the smaller model and trained only these adapters, whereas, in the distance-based approach, we trained solely the early exit head. In addition, to accelerate the residual transformation in smaller model, we distilled the residual streams from the larger model by incorporating a distillation loss ~\cite{sanh2019distilbert} between the residual state at layer \(i\) of the smaller model and the residual state at layer \(4 \times i\) of the larger model. As shown in ~\cref{fig:alignment_wrt_dedicated_model} the smaller model demonstrates a significantly faster rate of change in residual streams, leading to higher next token prediction accuracy after $k$ layers compared to the base model that employs traditional early exit mechanisms after $k$ layers \cite{schuster2022confident, chen2023eellm, varshney-etal-2024-investigating}. This experimental setup, which modifies only the rate of change in residual streams while keeping other factors constant, suggests that dense transformers, trained with a fixed number of layers, may inherently possess a slow residual transformation bias.

This observation raises an intriguing question: if the rate of change in residual streams could be accelerated relative to the number of layers, is it possible to facilitate earlier alignment for a greater proportion of tokens? Earlier alignment would be beneficial to not only facilitate dynamic computation but also for generating speculative tokens efficiently with high acceptance rates in speculative decoding setups ~\cite{leviathan2023fast, chen2023accelerating}. 

%thereby enhancing the efficiency of early exiting? 
 % This bias likely constrains the effectiveness of early exiting, particularly for easier tokens. By addressing this limitation through accelerated residual transformations, we hypothesize that it is possible to substantially improve the efficiency and accuracy of early exit strategies in transformer models.

\subsection{Multi-Rate Residual Transformation} \label{m2r2_method}

To address the slow residual transformation bias described in ~\cref{sec:motivation}, we introduce \textit{accelerated residual streams} that operate at rate $R$ relative to original slow residual stream. We pair slow residual stream, $h$ with an accelerated residual stream, $p$, which has an intrinsic bias towards earlier alignment. Relative to ~\cref{eq:slow_residual_transformation}, accelerated residual transformation from interval $E_j$ to $E_{j+1}$ can be represented as: 

% \begin{equation}
% \label{eq:fast_residual_transformation}
% P_{E_{j+1}} = P_{E_j} \prod_{i=E_j}^{E_{j+1}} \left( I + \hat{\mathcal{A}_i} \right) \left( I + \hat{\mathcal{M}_i} \right) \quad \text{where} \quad \hat{\mathcal{A}_i} = \hat{f}(c_i, P_{i}), \hat{\mathcal{M}_i} = \hat{g}(P_{i})
% \end{equation}


\begin{equation} \label{eq:fast_residual_transformation}
p_{E_{j+1}} = p_{E_j} + \sum_{i=E_j}^{E_{j+1}-1} \left( \hat{\mathcal{A}_i}(p_i) + \hat{\mathcal{M}_i}(p_i + \hat{\mathcal{A}_i}(p_i)) \right) \quad \text{where} \quad \hat{\mathcal{A}_i} = \hat{f}(c_i, p_{i}), \hat{\mathcal{M}_i} = \hat{g}(h_i), 
\end{equation}



where $\hat{\mathcal{A}_i}$ and $\hat{\mathcal{M}_i}$ denote non-linear transformation added by layer $i$ to previous accelerated residual $p_{i}$. Similar to $\mathcal{A}_i$, non-linear transformation $\hat{\mathcal{A}_i}$ attends to same context $c_i$ but uses a different transformation $\hat{f}$ for accelerating $p_{E_j}$ relative to $h_{E_j}$. 

We integrate accelerated residual transformation directly into the base network using parallel accelerator adapters such that rank of accelerator adapters $R_p << d$ where $d$ denotes base model hidden dimension. This setup allows the slow residual stream $h_{E_j}$ to pass through the base model layers while the accelerated residual stream $p_{E_j}$ utilizes these parallel adapters as shown in ~\cref{fig:m2r2_main}. Both slow and accelerated residuals are processed in same forward pass via attention masking and incur negligible additional inference latency in memory bound decoding setups, while in compute bound decoding setups where FLOPs optimization is essential, accelerated residual stream utilizes a fraction of attention heads that of slow residual (see ~\cref{sec:flops_optimization}). Additionally, to maximize the utility of accelerated residual transformations without introducing dedicated KV caches, we propose a shared caching mechanism between the slow and accelerated streams which minimally impact alignment benefits of our approach while offering substantial memory savings (see ~\cref{fig:koala_alignment}). Specifically, the attention operation on the slow residuals \( \text{MHA}(h_t, h_{\leq t}, h_{\leq t}) \) is redefined for accelerated residuals as 
\[
\hat{\mathcal{A}} = MHA(p_t, h_{<t} \oplus p_t, h_{<t} \oplus p_t),
\]
where the accelerated residual at time-step $t$, \( p_t \) attends to the slow residual’s KV cache, facilitating the reuse of contextual information across both residual streams without incurring additional caching costs. Here, \(MHA(q, k, v) \) represents multi-head attention between query \( q \), key \( k \), and value \( v \).

\begin{figure}
    \centering
    \includegraphics[width=0.8\linewidth]{sections//figures/m2r2_main2.pdf}
    \caption{Multi-rate Residuals Framework: Slow residual stream of base model is accompanied by a faster stream that operates at a $2-(J+1)\times$ rate relative to the slow stream, undergoing transformations via accelerator adapters as detailed in \cref{m2r2_method}, where J denotes number of early exit intervals. Colors within the slow and fast residual streams indicate similarity, with matching colors representing the most closely aligned residual states. At the beginning of the forward pass and at each exit point, the accelerated residual state is initialized from the corresponding slow residual state to avoid gradient conflict during training (see ~\cref{sec:grad_conflict}). Early exiting decisions are informed by the Accelerated Residual Latent Attention (ARLA) mechanism, described in \cref{method_arla}, which evaluates residual dynamics across consecutive exit gates.}
    \label{fig:m2r2_main}
\end{figure}

% Furthermore. to maximize the benefits of fast residual transformations without using dedicated KV caches, we propose sharing the fast network’s cache with the slow network. Formally speaking, We modify attention operation on slow residuals $MHA(H_t, H_{<=t}, H_{<=t})$ as $MHA(P_{t}, H_{<t} \oplus P_t, H_{<t}  \oplus P_t)$ such that accelerated residuals attend to previous slow context KV cache, where $MHA(q,k,v)$ denotes multi head attention between query, $q$, key $k$ and value $v$.


\subsection{Enhanced Early Residual Alignment}
Early residual alignment is instrumental in optimizing early exiting, speculative decoding, and Mixture-of-Experts (MoE) inference mechanisms. In this section, we provide a detailed analysis of how accelerated residuals enhance these inference setups.

% By aligning the residual states of intermediate layers with the final output representations, the model can maintain high prediction accuracy even when computations are truncated at earlier layers. This enables more reliable early exiting, reducing the overall computational cost while preserving performance. Additionally, in speculative decoding, early residual alignment allows the model to make confident predictions using faster, partial computations, thereby accelerating inference without sacrificing output quality.


\subsubsection{Early Exiting} \label{method_early_exiting}

A prevalent strategy for enabling early exiting at an intermediate layer $E_{j}$ involves approximating the residual transformation between $E_{j}$ and the final layer $N-1$ using a linear, context independent mapping, $\mathcal{T}$, such that $H_{N-1} \approx \mathcal{T}(H_{E_{j}})$. This approximation has been extensively employed in conventional approaches ~\cite{schuster2022confident, chen2023eellm, varshney-etal-2024-investigating}, providing a computationally efficient means to project the output of deeper layers from intermediate states. Specifically, residual state of layer $N-1$ with this approximation can be expressed as:


% \begin{equation}
% \label{eq: vanila_ea_assumption}
% \Phi(H_{E_{j}}) \sim H_{E_{j}} \prod_{i=E_{j}}^{N}\left( I + \mathcal{A}_i \right) \left( I + \mathcal{M}_i \right) \quad \text{where} \quad \Phi \perp C
% \end{equation}

\begin{equation} \label{eq:early_exiting}
h_{E_j} + \sum_{i=E_j}^{N-1} \left( \mathcal{A}_i(h_i) + \mathcal{M}_i(h_i + \mathcal{A}_i(h_i)) \right) \sim \mathcal{T}(h_{E_{j}})  \quad \text{where} \quad \mathcal{T} \perp c. 
\end{equation}


Here, $\mathcal{A}_i$ and $\mathcal{M}_i$ represent the residual contributions of the multi-head attention and MLP layers, respectively, while $\mathcal{T}$ remains independent of $c$, the preceding context.

This approach is inherently limited by two major factors: first, the assumption of linearity between $h_{E_{j}}$ and $h_{N-1}$ may not hold uniformly for all tokens, particularly when $E_j \ll N$. Second, the linear transformation $\mathcal{T}$ disregards the influence of the context $c$ and fails to account for the latent representations of previous contextual states. In contrast, M2R2 accelerated residual states mitigate both of these challenges by approximating the slow residual transformation of all layers via a faster residual transformation of fewer layers as:
% \begin{equation}
% H_{E_j} \prod_{i=E_j}^{N}\left( I + \mathcal{A}_i \right) \left( I + \mathcal{M}_i \right) \sim P_{E_j} \prod_{i=E_j}^{E_j+1}\left( I + \hat{\mathcal{A}_i} \right) \left( I + \hat{\mathcal{M}_i} \right)
% \end{equation}


\begin{equation} \label{eq:m2r2_approximating_ea}
h_{E_j} + \sum_{i=E_j}^{N-1} \left( \mathcal{A}_i(h_i) + \mathcal{M}_i(h_i + \mathcal{A}_i(h_i)) \right) \sim p_{E_j} + \sum_{i=E_j}^{E_{j+1}-1} \left( \hat{\mathcal{A}_i}(p_i) + \hat{\mathcal{M}_i}(p_i + \hat{\mathcal{A}_i}(p_i)) \right), 
\end{equation}

% \begin{equation} \label{eq:fast_residual_transformation}
% p_{E_{j+1}} = p_{E_j} + \sum_{i=E_j}^{E_{j+1}-1} \left( \hat{\mathcal{A}_i}(p_i) + \hat{\mathcal{M}_i}(p_i + \hat{\mathcal{A}_i}(p_i)) \right) \quad \text{where} \quad \hat{\mathcal{A}_i} = \hat{f}(c_i, p_{i}), \hat{\mathcal{M}_i} = \hat{g}(h_i) 
% \end{equation}






where $p_{E_j}$ is initialized from the slow residual state $h_{E_j}$ at each early exit interval $E_j$ using an identity transformation (see ~\cref{fig:m2r2_main}). As shown in ~\cref{fig:m2r2_residual_sim}, accelerated residuals offer a smoother, more consistent shift in residual direction across layers, in contrast to the abrupt changes typically seen at early exit points in standard early exit methods. Moreover, the normalized cosine similarity between accelerated states at early exit intervals and final residual states is substantially higher compared to traditional early exit techniques, highlighting improved alignment with final layer representations. Traditional adaptive compute methods are constrained by two principal factors: the number of tokens eligible for early exit at intermediate layers and the precision of early exit decision. If residual streams fail to saturate early, the majority of tokens remain ineligible for exit, thereby diminishing potential speedups. Additionally, imprecise delineations between tokens suitable for early exit can lead to underthinking (premature exits that adversely affect accuracy) or overthinking (unnecessary processing that compromises efficiency) ~\cite{zhou2020self, dai2020dynamic}. Enhanced early alignment using ~\cref{eq:m2r2_approximating_ea} helps to address  first issue. To address the second issue we introduce Accelerated Residual Latent Attention, which dynamically assesses the saturation of the residual stream, allowing for a more precise differentiation between tokens that can exit early and those requiring further processing.

% This results in uniform change in residual direction    
% % We keep $\mathcal{A} = \hat{\mathcal{A}}$, while $\hat{\mathcal{M}}$ is accelerated by a factor of $2 - (N_{E}+1)X$ relative to the slower residual transformation $\mathcal{M}$, where $N_E$ represents number of early exiting intervals.
% Figure~\cref{fig:rate_change_comparison} illustrates the comparative rate of change between these transformation streams.



% fig:rate_change_comparison
% - grid plot x axis -> layer id (0, 8) , y axis -> layer id -> dark color cell for max similarity , lighter for lower 
% 
-------------------------------------------------------
Let's consider residual stream $h_i$ traverses through interval $E_j$ to $E_{j+1}$ and undergoes residual transformation given by 
\begin{equation}
h_{E_{j+1}} = h_{E_j} \prod_{i=E_j}^{E_{j+1}} \left( 1 + \delta_i \right)    
\end{equation}

where $\delta_i$ denotes non-linear transformation added by layer $i$. Each non-linear transformation of layer $i$ is a function of previous contextual representation, $c_i$ and input residual stream $h_i-1$ as
$\delta_i = f(c_i, h_{i-1})$ 

One way to exit early at exit $E_j+1$ is to assume that residual transformation from $E_j+1$ to final layer $N-1$ can be approximated by a linear function $\phi$ as $h_{N-1} \sim \Phi(h_{E_j+1})$ and most conventional approaches such as \todo{cite EA papers} use this approach. In other words, 

\begin{equation}
\Phi(h_{E_j+1} \sim h_{E_j+1} \prod_{i=E_j+1}^{N} \left( 1 + \delta_i \right)   
\end{equation}

This approach suffers from two primary issues, linearity assumption from $h_E_j+1$ to $H_N-1$ if often incorrect, particularly when $E_j << N$. More importantly, linear transformation $\Phi$ doesn't consider effect of context $C_i$. M2R2  effectively addresses these issues as accelerated residual stream at interval $E_j+1$ can be represented as 

\begin{equation}
r_{E_{j+1}} = r_{E_j} \prod_{i=E_j}^{E_{j+1}} \left( 1 + \gamma_i \right)    
\end{equation}

where $\gamma_i$ denotes non-linear transformation added by layer $i$ to previous accelerated residual $r_i-1$. Similar to $\delta_i$, non-linear transformation $\gamma_i$ considers context $C_i$ as 
$\gamma_i = g(c_i, r_{i-1})$. So in summary, slow residual transformation is approximated by accelerated residual as: 

\begin{equation}
h_{E_j} \prod_{i=E_j}^{N} \left( 1 + \delta_i \right) \sim h_{E_j} \prod_{i=E_j}^{E_j+1} \left( 1 + \gamma_i \right)
\end{equation}

It's worth noting that accelerated residual $r_i$ and slow residual $h_i$ are processed concurrently at layer $i$ by constructing proper attention mask such as attention of slow residual is represented as 

$MHA(H_it, H_{i<=t}, H_{i<=t}$ while attention of fast residual is computed as 

$MHA(r_it, H_{i<=t}, H_{i<=t}$ where $MHA(q,k,v$ denotes multi head attention between query, $q$, key $k$ and value $v$.


------------------------------------------------------------------

Vertical latent attention on accelerated residual is computed as 
$MHA(S_mt, S(Ej<=i<=m)t, S(Ej<=i<=m)t)$ where $Smt$ denotes query/key/value projection in latent domain at layer $m$ at time $t$. 
------------------------------------------------------------------

Gradient conflict Avoidance: 

Let's consider $w_j$ is a trainable parameter that belongs to a layer between $E_j$ and $E_j+1$. Consider early exit loss at gate $E_j+1$, $L_j+1$, gradient propagation of $w_j$ at another trainable parameter $w_j-n$ can be gives as 

$\sum_{k=E_j-n}^{E_j} \beta_k \frac{\partial L_{E_k}}{\partial w_k}$

where $\beta_j$ denotes backward transformation coefficient for weight $w_j$ to reach gate $E_j$. 
 
On the other hand, gradient propagation in proposed approach can be represented as 

\[
\frac{\partial L_{E_j}}{\partial w_j} = 
\begin{cases} 
\beta_j \frac{\partial L_{E_j}}{\partial w_j} & \text{if } E_j \leq w_j \leq E_{j+1} \\
0 & \text{otherwise}
\end{cases}
\]







% \begin{figure}[ht]
%     \centering
%     \includegraphics[width=0.8\textwidth, height=5cm]{rate_change_comparison.png}
%     \caption{Rate of change comparison between fast and slow residual streams.}
%     \label{fig:rate_change_comparison}
% \end{figure}

%vary k and and plot EA accuracy for larger and smaller models. 

% \begin{figure}[ht]
%     \centering
%     \includegraphics[width=0.5\textwidth,height=5cm]{sections/figures/alignment_comparison_dialogsum.pdf}
%     \caption{Alignment of exited tokens for different early exit layers using traditional early exiting heads, dedicated faster networks, and faster residuals.}
%     \label{fig:small_model_early_exiting}
% \end{figure}


\textbf{Accelerated Residual Latent Attention} \label{method_arla}

In the context of residual streams, we observe that the decision to exit at a given layer can be more effectively informed by analyzing the dynamics of residual stream transformations, instead of solely relying on a classification head applied at the early exit interval $E_j$. To capture the subtle dynamics of residual acceleration, we propose a \textit{Accelerated Residual Latent Attention} (ARLA) mechanism. This approach involves making the exit decision at gate $E_j$ by attending to the residuals spanning from gate $E_{j-1}$ to $E_j$, rather than considering only the residual at gate $E_j$. To minimize the computational overhead associated with exit decision-making, the attention mechanism operates within the latent domain as depicted in ~\cref{fig:arla_arch}. Formally, for each interval $[E_j, E_{j+1}]$, the accelerated residuals are projected into Query ($Q^s_{E_j}, \ldots, Q^s_{E_{j+1}}$), Key ($K^s_{E_j}, \ldots, K^s_{E_{j+1}}$), and Value ($V^s_{E_j}, \ldots, V^s_{E_{j+1}}$) vectors, with latent dimension $d^s$ for $Q^s$, $K^s$, and $V^s$ being significantly smaller than hidden dimension of $p$.\footnote{We use $d^s = 64$ for experiments described in ~\cref{sec:experiments}.} Notably, when the router is allowed to make exit decisions at gate $E_j$ based on residual change dynamics, we observe that the attention is not confined to the residual state at $E_j$ but is distributed across residual states from $E_{j-1}$ to $E_j$, %as illustrated in Figure~\ref{fig:vertical_latent_attention_dynamics}. 
This broader focus on residual dynamics significantly reduces decision ambiguity in early exits, as demonstrated in Figure~\ref{fig:roc_arla}, which contrasts routers based on the last hidden state, and the proposed ARLA router.

%show R -> S transformation. 
%show parameter and flop overhead as compared to adapter on last hidden state.

% \begin{figure}[ht]
%     \centering
%     \includegraphics[width=0.5\textwidth,height=5cm]{sections/figures/roc_arla.pdf}
%     \caption{ROC curves of early exit decision strategies: confidence-based methods (CALM/LITE), routers based on the accelerated hidden state, and latent attention routers.}
%     \label{fig:decision_making_comparison}
% \end{figure}

% \begin{figure}[ht]
%     \centering
%     \includegraphics[width=0.5\textwidth,height=5cm]{vertical_latent_attention.png}
%     \caption{Vertical latent attention mechanism for optimizing early exit decisions by considering residuals from gate \(M\) through \(M-1\).}
%     \label{fig:vertical_latent_attention}
% \end{figure}

\begin{figure}[ht]
    \centering
    \begin{subfigure}{0.52\textwidth}
        \centering
        \includegraphics[width=\textwidth, height = 4cm]{sections/figures/arla_arch.pdf}
        \caption{Accelerated Residual Latent Attention (ARLA): Accelerated residuals between early exit gates are projected into latent domain and attention over residual states within the interval is computed to capture residual dynamics and exit decision is made based on residual saturation.}
        \label{fig:arla_arch}
    \end{subfigure}%
    \hfill
    \begin{subfigure}{0.45\textwidth}
        \centering
        \includegraphics[width=\textwidth, height = 4.5cm]{sections/figures/vla_roc.pdf}
        \caption{ROC classification curves of early exit decision strategies using a linear router used on last residual state ~\cite{schuster2022confident, varshney-etal-2024-investigating, chen2023eellm}  and using ARLA approach that considers residual dynamics. }
        \label{fig:roc_arla}
    \end{subfigure}
    \caption{Effectiveness of ARLA in capturing residual dynamics for early exiting decisions.}


\end{figure}



% \begin{figure}[ht]
%     \centering
%     \includegraphics[width=1\textwidth,height=5cm]{sections/figures/arla.pdf}
%     \caption{fig that plots 32 rows 2 cols heatmap showing attention at each gate}
%     \label{fig:vertical_latent_attention_dynamics}
% \end{figure}

\subsubsection{Self Speculative Decoding} \label{method_self_speculative_decoding}

An alternative means to exploit the early alignment properties of our approach is through the use of accelerated residual states for speculative token sampling to accelerate autoregressive decoding. Speculative decoding aims to speed up memory-bound transformer inference by employing a lightweight draft model to predict candidate tokens, while verifying speculated tokens in parallel and advancing token generation by more than one token per full model invocation \cite{leviathan2023fast, chen2023accelerating, xia2023speculative, miao2023specinfer}. Despite its effectiveness in accelerating large language models (LLMs), speculative decoding introduces substantial complexity in both deployment and training. A separate draft model must be specifically trained and aligned with the target model for each application, which increases the training load and operational complexity ~\cite{chen2023accelerating}. Additionally, this approach is resource-inefficient, as it requires both the draft and target models to be simultaneously maintained in memory during inference \cite{leviathan2023fast, chen2023accelerating}. 

One strategy to address this inefficiency is to leverage the initial layers of the target model itself to generate speculative candidates, as depicted in ~\cite{Tang2024}. While this method reduces the autoregressive overhead associated with speculation, it suffers from suboptimal acceptance rates. This occurs because the linear transformation employed for translating hidden states from layer $k$ to the final layer $N$ is typically a poor approximation, as discussed in ~\cref{sec:motivation} and ~\cref{method_early_exiting}. Our approach resolves this limitation by utilizing accelerated residuals, which demonstrate higher fidelity to their slower counterparts. By utilizing accelerated residuals operating at a rate of $N/k$, where $k$ denotes the number of layers used for candidate speculation, we are able to efficiently generate speculative tokens for decoding.\footnote{We typically set $k = 4$ to balance the trade-off between autoregressive drafting overhead and acceptance rate, as discussed in~\cref{sec:experiments}.}
 This technique not only obviates the need for multiple models during inference but also improves the overall efficiency and effectiveness of speculative decoding.

\begin{figure}
    \centering    \includegraphics[width=1\linewidth]{sections/figures/m2r2_aot_loading.pdf}
    \caption{Ahead-of-Time Expert Loading: M2R2 accelerated residual stream predicts experts required for future layers, reducing reliance on on-demand lazy loading. Speculative pre-loading is efficiently overlapped with computation of multi-head attention (MHA) and MLP transformations. Only incorrectly speculated experts are loaded lazily, resulting in faster inference steps and improved computational efficiency. Here, H indicates LBM Host while D indicates HBM Device.}
    \label{fig:moe_expert_aot_loading}
\end{figure}


\subsubsection{Ahead of Time Expert Loading:} \label{method_aot_expert_loading}

Recent advancements in sparse Mixture-of-Experts (MoE) architectures ~\cite{shazeer2017outrageously, fedus2022switch, artetxe2019massively, lepikhin2020gshard, zoph2022designing} have introduced a paradigm shift in token generation by dynamically activating only a subset of experts per input, achieving superior efficiency in comparison to dense models, particularly under memory-bound constraints of autoregressive decoding \cite{fedus2022switch, zoph2022designing}. This sparse activation approach enables MoE-based language models to generate tokens more swiftly, leveraging the efficiency of selective expert usage and avoiding the overhead of full dense layer invocation. In dense transformer models, pre-loading layers is a common strategy to enhance throughput, as computations of current layer can be overlapped with pre-loading of next layer parameters ~\cite{narayanan2021efficient, shoeybi2020megatron}. However, MoE models face a unique challenge: expert selection occurs dynamically based on previous layer’s output, making it infeasible to preload next layer’s experts in parallel. This limitation results in inherent latency, as expert loading becomes a sequential, on-demand process ~\cite{lepikhin2020gshard, fedus2022switch}.

To address this inefficiency, our method introduces a mechanism with \textit{accelerated residuals}, which not only captures key characteristics of base slower residual states but also exhibit high cosine similarity with their final counterparts (as illustrated in \cref{fig:m2r2_residual_sim}). By employing accelerated residual streams, we can effectively predict the necessary experts for future layers well in advance of their actual invocation. Specifically, using a $2\times$ accelerated residual, the experts needed for layers $2i+2$ and $2i+3$ can be identified while still computing in layer $i$, thus overcoming the bottleneck of sequential, on-demand expert selection and mitigating latency in the decoding pipeline, as shown in \cref{fig:moe_expert_aot_loading}. Note that, we use fixed set of accelerator adapters for transforming accelerated residuals (as discussed in ~\cref{m2r2_method}) while slow residual is transformed via expert routing mechanism. 

Furthermore, our approach integrates a Least Recently Used (LRU) caching strategy, which enhances memory efficiency by replacing the least recently used experts with speculated experts that are anticipated to be needed in upcoming layers. This hybrid approach of preemptive expert loading with LRU caching yields substantial improvements over traditional on-demand loading or standalone caching strategies. By minimizing cache misses and efficiently managing memory, this approach addresses both compute and memory bottlenecks, leading to faster, more resource-efficient token generation in MoE architectures. A comprehensive evaluation of this strategy, in relation to state-of-the-art methods, is provided in \cref{experiments_aot}, and the compute and memory traces on an A100 GPU are detailed in \cref{fig:moe_aot_cuda_trace}.



% Recent advancements in sparse Mixture-of-Experts (MoE) architectures have introduced the concept of utilizing distinct computational paths for different tokens \cite{shazeer2017outrageously}. This approach, wherein only a subset of experts are activated per input, enables MoE-based language models to generate tokens more swiftly compared to their dense counterparts due to memory-bound nature of auto-regressive decoding. In dense models, pre-loading layers in advance is a common strategy to enhance computational efficiency. However, this technique is not applicable to MoE models, where expert selection occurs dynamically based on the outputs of previous layers, preventing parallel pre-fetching of experts.

% Our proposed method addresses this inefficiency. Accelerated residuals, which are highly similar to their slower counterparts (see \cref{fig:similarity}), can reliably predict the necessary experts ahead of time. For instance, by utilizing $2X$ accelerated residual stream, we can predict the experts needed for the layer $2i+1$ and $2i+3$ while carrying out computation in layer $i$. This enables us to commence expert loading significantly earlier, as illustrated in \cref{expert_loading}, effectively mitigating the delays observed with the naive on-demand expert loading. Additionally, our method benefits from incorporating a Least Recently Used (LRU) strategy, where speculated experts replace those that are least recently utilized, resulting in improved performance compared to using either strategy alone. For a comprehensive evaluation, refer to \cref{moe_trace}, which provides a CUDA compute and memory trace of our approach executed on <>.



% A naive solution involves using the residual state of the previous layer along with the gating function of the next layer to predict which experts need to be loaded, and initiating the expert loading process in parallel with the attention computation of the next layer. Yet, as shown in \cref{fig:MOE_attn_vs_loading_time}, the attention computation for medium to long contexts is considerably faster than the expert loading time, making this approach inefficient.




\subsection{Training} \label{method_training}
% This approach is feasible due to the absence of gradient conflicts, as discussed in \cref{sec:grad_conflict}.

To accelerate residual streams, we employ parallel accelerator adapters as described in \cref{m2r2_method}.  For the early exiting use-case outlined in \cref{method_early_exiting}, we define the training objective for these adapters using the following loss function, which combines cross-entropy loss at each exit $E_j$ with distillation loss at each layer $i$. Loss weights coefficients $\alpha_0$ and $\alpha_1$ are employed to balance contribution of corresponding losses.

\begin{align} \label{eq:mr_loss}
L_{\text{m2r2}} = \underbrace{-\alpha_0 \sum_{j=1}^{J} \sum_{t=1}^{T} \log p_{\theta} \left( \hat{y}_t^{E_j} \mid y_{<t}, x \right)}_{\text{cross-entropy loss}} 
+ \underbrace{\alpha_1\sum_{i=1}^{E_{J-1}} \sum_{t=1}^{T} \| \mathbf{p}_{t}^{i} - \mathbf{h}_{t}^{((i - E_{j(i)}) \cdot R_i) + E_{j(i)})} \|^2}_{\text{distillation loss}}.
\end{align}

where $\hat{y}_t^{E_j}$ denotes the predictions from the accelerated residual stream at layer $E_j$ and time step $t$, $y_t$ represents the corresponding ground truth tokens, and $x$ indicates previous context tokens. The distillation loss at each layer $i$ is computed by comparing accelerated residuals at layer $i$ with slow residuals at layer $(i - E_{j(i)}) \cdot R_i + E_{j(i)}$, where $R_i$ denotes the rate of accelerated residuals at layer $i$ while $E_{j(i)}$ represents the most recent gate layer index such that $E_{j(i)} <= i$. \( J \) represents the total number of early exit gates, N denotes number of hidden layers and $E_j$ denotes layer index corresponding to gate index $j$ and \( T \) denotes the sequence length. 

In dynamic compute settings, after training of accelerator adapters, we optimize the query, key, and value parameters governing the ARLA routers (see ~\cref{method_arla}) across all exits in parallel on binary cross entropy loss between predicted decision and ground truth exiting decision. The ground truth labels for the router are determined based on whether the application of the final logit head on $\hat{y}_t^{E_j}$ yields the correct next-token prediction. 


% The objective for this optimization is defined by the following loss function:


%TODO are equations required ? 
% \begin{equation} \label{eq:arla_loss_combined}\small
%     L_{\text{arla}} = -\frac{1}{N} \sum_{t=1}^{T} \left( \sum_{j=1}^{E_n} \left[ O_t^{E_j} \log(\hat{O}_t^{E_j}) + (1 - O_t^{E_j}) \log(1 - \hat{O}_t^{E_j}) \right] \right), \quad \text{where} \quad 
%     O_t^{E_j} = \begin{cases} 
%     1, & \text{if } L(\hat{y}_t^{E_j}) = y_t^{E_j} \\
%     0, & \text{otherwise}
%     \end{cases}
% \end{equation}

% where $\hat{O}_t^{E_j}$ represents the binary predicted logits produced by the vertical latent attention router, as described in \cref{sec:arla}, at gate $E_j$ and time step $t$, and $O_t^{E_j}$ denotes the corresponding ground truth labels. The ground truth labels for the router are determined based on whether the application of the logit head on $\hat{y}_t^{E_j}$ yields the correct next-token prediction. The parameters controlling vertical latent attention are trained concurrently to ensure consistency and efficient use of computational resources.

For self-speculative decoding, as described in \cref{method_self_speculative_decoding}, the training objective remains the same as \cref{eq:mr_loss}, but with the number of intervals set to $J = 1$ and the rate of residual transformation set to $R_n = N/k$, where the first $k$ layers generate speculative candidate tokens. In the context of Ahead-of-Time Expert Loading for Mixture-of-Experts (MoE) models (see \cref{method_aot_expert_loading}), setting the rate of residual transformation to $R_n = 2$ typically offers a good trade-off between the accuracy of expert speculation and AoT pre-loading of experts. 

% Thus, we set $J = 1$ and $E_1 = 16$.


~\subsection{FLOPs Optimization} \label{sec:flops_optimization}

Naively implemented, M2R2 incurs higher FLOP overhead compared to traditional speculative decoding and early exiting approaches such as ~\cite{medusa, schuster2022confident, Tang2024}. However, modern accelerators demonstrate compute bandwidth that exceeds memory access bandwidth by an order of magnitude or more~\cite{databricksLLMInference2023, jouppi2021ten}, meaning increased FLOPs do not necessarily translate to increased decoding latency. Nevertheless, to ensure fair comparison and efficiency in compute bound scenarios, we introduce targeted optimizations.

~\textbf{Attention FLOPs Optimization} For medium-to-long context lengths, attention computation dominates FLOPs in the self-attention layer, surpassing the contribution from MLP layers. Specifically, matrix multiplications involving queries, cached keys, and cached values scale with $l_{kv} * l_{q}$ where $l_{kv}$ denotes previous context length and $l_q$ denotes current query length. Since M2R2 pairs accelerated residuals with slow residuals, a naive implementation results in twice the FLOPs consumption compared to a standard attention layer. To address this, we limit the attention of accelerated residual stream to selectively attend to the top-k most relevant tokens, identified by the slow residual stream based on top attention coefficients\footnote{We set to k = 64 and attend to top 64 tokens as identified by the slow residual stream.}. This is possible since slow and accelerated residual streams are processed in same forward pass and accelerated streams have access to attention coefficients of slow stream. Note that, the faster residual stream still retains the flexibility to assign distinct attention coefficients to these tokens. Furthermore, we design the faster residual stream to employ only 8 attention heads, compared to the 32 heads used in the slow residual stream of the Phi-3 model, reducing query, key, value, and output projection FLOPs by a factor of 1/4. ~\cref{fig:m2r2_num_heads_ablation} indicates effect of using a slicker stream on alignment. As depicted, using $\hat{n}_h = 8$ offers a good trade-off between alignment and FLOPs overhead. 

~\textbf{MLP FLOPs Optimization} The accelerator adapters operating on the accelerated residual stream are intentionally designed with lower rank than their counterparts in the base model. This reduces FLOP overhead by a factor proportional to $hiddenSize / rank$. Additionally, since the faster residual stream uses only 8 attention heads (compared to 32 in the slow residual stream of Phi-3), the subsequent MLP layers process a smaller set of activations, further reducing FLOPs by another factor of 1/4.

These optimizations significantly reduce the FLOP overhead per speculative draft generation, as illustrated in ~\cref{fig:flops_optmization}. Notably, while traditional early-exiting speculative approaches such as DEED require propagating the full slow residual state through the initial layers, incurring substantial computational costs, M2R2 achieves efficient token generation via slimmer, low-rank faster residual streams. In contrast, Medusa introduces considerable FLOP overhead due to per-head computations scaling with $d^2+dv$\footnote{Here $d$ denotes hidden state dimension while $v$ denotes vocab size.}, whereas M2R2 employs low-rank layers for both MLP and language modeling heads, maintaining computational efficiency. All experiments involving the M2R2 approach, as detailed in ~\cref{sec:experiments}, are conducted using these FLOPs optimizations.









% \[
% O_t^{E_j} = 
% \begin{cases} 
% 1, & \text{if } L(\hat{y}_t^{E_j}) = y_t^{E_j} \\
% 0, & \text{otherwise}
% \end{cases}
% \]




%add distillation
% We train accelerator adapters described in \cref{m2r2_method} to accelerate residual streams on next token prediction all in parallel since there are no gradient conflict issues as described in \cref{sec:grad_conflict}.

% \begin{align} \label{eq:mr_loss}
% L_{mr} =  & -\sum_{j = 1}^{E_n} (\sum_{t=1}^{T}\log p_{\theta} (\hat{y}_t^{E_j} | \hat{y}_{<t}, x)) \nonumber
% \end{align}

% where $\hat{y_t^{E_j}}$ denotes predicted logits obtained from accelerated residual stream at gate $E_j$ and time-step $t$ while $y_t^{E_j}$ denotes corresponding truth tokens. 

% Upon training of adapters responsible for accelerating residual streams, we train query, key, value parameters responsible for vertical latent attention of all gates in parallel as

% \begin{equation} \label{eq:arla_loss}
%     L_{arla} = -\frac{1}{N} (\sum_{t=1}^{T}(1\sum_{j=1}^{E_n} \left[ O_t^{E_j} \log(\hat{O}_t^{E_j}) + (1 - o_t^{E_j}) \log(1 - \hat{o_t}_{E_j}) \right]))
% \end{equation}

% where $\hat{O_t^{E_j}}$ denotes binary predicted logits obtained from vertical latent attention router described in \cref{sec:arla} at gate $E_j$ and timestep $t$ while $O_t^{E_j}$ denotes corresponding truth label. Truth labels for router are obtained by computing whether logit head application on $\hat{y}_t^j$ results in true next token prediction. Formally speaking, 

% $O_t^{E_j} = 1 if L(\hat{y_t^{E_j}}) == y_t^{E_j} , 0 otherwise$. 

% Parameters responsible for vertical latent attention are also trained in parallel as well. 

%todo: training slow and fast residuals together and distillation can be two training mdoes. 
%Distillation can be an ablation. 




% Although transformer decoding is memory bound on most mainstream accelerators, there could be scenarios where flop savings are crucial. For instance, on on-device settings power consumption is directly correlated with flops per decoding step and reducing flops does help with overall energy consumption. Vanilla early exiting methods help with flop reduction but suffer from mismatch between training and inference due to early exited tokens. If token at decoding step $t$, $T_t$ exited at layer $E_i$, while token $T_{t+k}$ exits at layer $E_j$ such that $E_i < E_j$, hidden state $H_{t+k}l$ does not have corresponding hidden state $H_tl$ to attend to where $E_i < l <= E_j$. One solution that's often used in literature is to rely on last hidden state available, $H_t{E_j}$, however it tends to be sub-optimal and does affect generation quality \cite{ref}.  To alleviate this mismatch while reducing flops, we train router such that attention mask between token $T_{t+k}$ and token $T_{<t+k}$ is given by: 

% \begin{equation}
%     a_{T_{{t+k}{T_{<t+k}}} = 1 if  E_{T_{<t+k}} >= E{T_{t+k}}
%     else 0
% \end{equation}

% This attention mask enables router to account for exited tokens and get trained accordingly. Since attention mechanism during decoding remains exactly same as that during training, impact on generation quality tends to be minimal as noted in \cref{fig:gen_auality_with_and_without_recompute_attention_show_flops}.  Although MoD does not suffer from training and inference mismatch, we observe that it suffers from discountinuity between pre-training and super-vised fine-tuning resulting in sub-optimal perplexity. On the other hand, our method doesn't not require pre-training , doesn't suffer from discountinuity, and achieves much better perplexity in super-vised fine-tuning and instruction tuning setups as shown in \cref{fig:Mod_vs_m2r2_loss_curves}.






% Our techniques are directly applicable in such scenarios.    




%expert loading with cuda streams in experiments



\section{Image Compression with DDCM}\label{section:compression}
\begin{figure*}[t]
    \centering
    \includegraphics[width=1\textwidth]{figures/Kodak24_512_extreme.pdf}
    \caption{\textbf{Qualitative image compression results.} The presented images are taken from the Kodak24 ($512\times 512$) dataset.
    Our codec produces highly realistic outputs, while maintaining better fidelity to the original images compared to previous methods.
    }
    \label{fig:compression_examples}
\end{figure*}
\paragraph{Method.}
Since sampling with DDCM yields compact bit-stream representations, a natural endeavor is to harness DDCM for compressing real images.
In particular, to compress an image $\rvx_{0}$, we leverage the predicted $\hat{\rvx}_{0|i}$ (\Cref{eq:x0eq}) at each timestep $i$ and compute the residual error from the target image, $\rvx_0-\hat{\rvx}_{0|i}$.
Then, we guide the sampling process towards $\rvx_0$ by selecting the codebook entry that maximizes the inner product with this residual,
\begin{align}\label{eq:compression_rule} 
    k_i = \argmax_{k\in\{1,\hdots,K\}} \langle \gC_i(k), \rvx_0-\hat{\rvx}_{0|i}\rangle,
\end{align}
where the size of the first codebook $\gC_{T+1}$ is $K=1$. 
This process is depicted as the compression branch in \Cref{fig:overview}, where the resulting set of chosen indices $\{k_i\}_{i=2}^{T+1}$ is the compressed bit-stream representation of the given image.
Section~\ref{sec:compressed_conditional_generation} sheds more light on this choice of the noise selection from the perspective of score-based generative models~\citep{song2020score}. 
As in \Cref{sec:method}, decompression follows standard DDCM sampling \cref{eq:DDCM_sampling}, re-selecting the stored indices instead of picking them randomly.
When using latent space DDMs (e.g., SD), we 
first encode $\rvx_0$ into the latent domain, perform all subsequent operations in this domain, and decode the result with the decoder.

The bit rate of this approach is determined by the size of the codebooks $K$, and the number of sampling timesteps $T$. Specifically, the bit-stream length is given by \smash{$(T-1)\log_{2}(K)$}. Therefore, the bit rate can be reduced by simply decreasing the number of codebooks, or by using a smaller number of timesteps at generation, e.g., by skipping every other step, or by using specific timestep intervals (see \Cref{app:range_t}).
In the approach described so far, the length of the bitstream increases logarithmically with $K$, making it computationally demanding to increase the bit rate.
For instance, even for $K=8192$, $T=1000$ and $768\times 768$ images our BPP is approximately 0.022.
Thus, to produce higher bit rates, we propose to \emph{refine} the noise selected at timestep $i$ by employing matching pursuit (MP)~\citep{mallat1993matching}.
Specifically, at each step $i$, we construct the chosen noise as a convex combination of $M$ elements from 
$\gC_i$, gathered in a greedy fashion to best correlate with the guiding residual $\rvx_{0}-\hat{\rvx}_{0|i}$ (as in~\Cref{eq:compression_rule}). 
The resulting convex combination involves $M-1$ quantized scalar coefficients, chosen from a finite set of $C$ values taken from $[0,1]$.
Therefore, the resulting length of the bit-stream is given by $\smash{(T-1)(\log_{2}(K)M+C(M-1))}$, such that $M=1$ is similar to our standard compression scheme, and the length of the bit-stream increases linearly with $M$ and $C$.
We apply this algorithm when the absolute bits number crosses $(T-1)\cdot \log_2(2^{13})$.
Further details are available in \Cref{app:matching_pursuit}.

\paragraph{Experiments.}
We evaluate our compression method on Kodak24~\cite{franzen1999kodak}, DIV2K~\citep{agustsson2017ntire}, ImageNet 1K $256\times 256$~\citep{deng2009imagenet,pan2020dgp}, and CLIC2020~\citep{CLIC2020}.
For all datasets but ImageNet, we center crop and resize all images to $512\times512$.
We compare to numerous competing methods, both non-neural and neural, and both zero-shot, fine-tuning based, and training based approaches.
For the ImageNet dataset, we use the unconditional pixel space ImageNet $256\times 256$ model of \citet{dhariwal2021diffusion}, and compare our results to BPG~\citep{bpg}, HiFiC~\citep{mentzer2020high}, IPIC~\citep{xu2024idempotence}, and two  PSC~\citep{elata2024zero} configurations, distortion-oriented (PSC-D) and perception-oriented (PSC-P).
For all other datasets, we use SD 2.1 $512\times512$~\citep{rombach2022high} and compare to BPG, PSC-D, PSC-P, ILLM~\citep{muckley2023improving}, PerCo (SD)~\citep{korber2024perco, careil2023towards}, and twoCRDR~\citep{iwai2024controlling} configurations, distortion-oriented (CRDR-R) and perception-oriented (CRDR-R).
PSC shares the same pre-trained model as ours, while PerCo (SD) requires additional fine-tuning.
For our method, we apply SD 2.1 unconditionally, as we saw no improvement by adding prompts (see further details in \Cref{app:text_effect}).
We assess our method for several options of $T$, $K$, $M$, and $C$ to control the bit rate.
See further details in \Cref{app:compression_details}.
We evaluate distortion with PSNR and LPIPS~\citep{zhang2018perceptual} and perceptual quality with FID~\citep{bińkowski2018demystifying}.
For ImageNet, FID is computed against the entire 50k $256\times 256$ validation set.
For the smaller datasets we follow \citet{mentzer2020high} and compute the FID over extracted image patches. 
Specifically, for DIV2K and CLIC2020 we extract $128\times 128$ sized patches, and for Kodak we use $64\times64$.


As shown in \Cref{fig:compression_graphs}, our compression scheme dominates previous methods on the rate-perception-distortion plane~\citep{pmlr-v97-blau19a} for lower bit rates, surpassing both the perceptual quality (FID) and distortion (PSNR and LPIPS) of previous methods.
For instance, our FID scores are lower than those of all other methods at around 0.1 BPP, while, for the same BPP, our distortion performance is better than the perceptually-oriented methods (e.g., PerCo, PSC-P, and IPIC).
However, our method under-performs at the highest bit rates, especially when using SD. This is expected due to the performance ceiling entailed by the pre-trained encoder-decoder of SD~\citep{korber2024perco, elata2024zero}.
The qualitative comparisons in \Cref{fig:compression_examples} further demonstrate our superior perceptual quality, where, even for extreme bit rates, our method maintains similarity to the original images in fine details.
See \Cref{app:compression_details} for more details and results.


\begin{figure*}[t]
    \centering
    \includegraphics[width=\linewidth]{figures/compression_graphs.pdf}
    \caption{\textbf{Compression quantitative evaluation.}
    We compare the perceptual quality (FID) and distortion (PSNR, LPIPS) achieved at different BPPs. 
    The image sizes of each dataset is denoted next to its name.
    Our method produces the best perceptual quality at most BPPs.
    Importantly, this is while we also attain lower distortion compared to perceptually-oriented methods (e.g., PSC-P and PerCo (SD)). 
    For the three rightmost datasets, note that our approach, PSC-P, and PerCo (SD) use the latent space Stable Diffusion 2.1 model, so its VAE imposes a distortion bound.
    Thus, we report the distortion attained by simply passing the images through this VAE (dashed line).
    }
    \label{fig:compression_graphs}
\end{figure*}


\section{Compressed Conditional Generation}\label{sec:compressed_conditional_generation}

We showed that DDCM can be used as an image codec by following a simple index selection rule, guiding the generated image towards a target one.
Here, we generalize this scheme to any \emph{conditional} generation task, considering the more broad framework of \emph{compressing} conditionally generated samples.
This is a particularly valuable framework in scenarios where the input condition $\rvy$ is bit rate intensive, e.g., where $\rvy$ is a degraded image and the goal is to produce a \emph{compressed} high-quality reconstruction of it.
To the best of our knowledge, this task, which we name \emph{compressed} conditional generation, has only been thoroughly explored for text input conditions~\citep{bulla2023}.

A naive solution to this task could be to simply compress the outputs of any existing conditional generation scheme.
Here we propose a novel end-to-end solution that generates the outputs \emph{directly} in their compressed form.
Importantly, note that our approach in \Cref{sec:method} requires the condition $\rvy$ for decompressing the bit-stream.
While this is not a stringent requirement when the condition is lightweight (e.g., a text prompt), this approach is less suitable when storage of the condition signal itself requires a long bit-stream. 
The solutions we propose in this section enable decoding the bit-stream without access to $\rvy$.


\paragraph{Compressed Conditional Generation with DDCM.}We propose generating a conditional sample by choosing the indices $k_{i}$ in \Cref{eq:DDCM_sampling} via
\begin{align}
k_{i}=\argmin_{k\in\{1,\hdots,K\}}\gL(\rvy,\rvx_{i},\gC_{i},k),\label{eq:k_choose_conditional}
\end{align}
instead of picking them randomly.
Here, $\gL(\rvy,\rvx_{i},\gC_{i},k)$ can be any loss function that attains a lower value when $\gC_{i}(k)$ 
directs the generative process towards an image that matches $\rvy$.
For example, for the loss
\begin{align}
\gL_{\text{P}}(\rvy,\rvx_{i},\gC_{i},k)=\norm{\gC_{i}(k)-\sigma_{i}\nabla_{\rvx_{i}}\log{p_{i}(\rvy|\rvx_{i})}}^{2}\label{eq:l_score}
\end{align}
we obtain the following result (see proof in~\cref{appendix:cond_compression}):
\begin{proposition}\label{prob:ode_convergence}
Suppose that image samples are generated via process~\cref{eq:DDCM_sampling}, and the indices $k_{i}$ are chosen according to~\Cref{eq:k_choose_conditional} with $\gL=\gL_{\textnormal{P}}$.
Then, when $K\rightarrow\infty$, such a generative process becomes a discretization of a probability flow ODE over the posterior distribution $p_{0}(\rvx_{0}|\rvy)$.
\end{proposition}
In other words,~\Cref{prob:ode_convergence} implies that for the loss $\gL_{\text{P}}$, increasing $K$ leads to more accurate sampling from the posterior $p_{0}(\rvx_{0}|\rvy)$, though this results in longer bit-streams.
Thus, as long as we have access to $\nabla_{\rvx_{i}}\log{p_{i}(\rvy|\rvx_{i})}$ (or an  approximation of it) $\gL_{\text{P}}$ may serve as a sensible option for solving a compressed conditional generation task with DDCM.
Interestingly, we show in~\Cref{appendix:compression_private_case} that our compression scheme from~\Cref{section:compression} is a special case of the proposed compressed conditional generation method, with $\rvy=\rvx_{0}$ and $\gL=\gL_{\text{P}}$.

\subsection{Compressed Posterior Sampling for Image Restoration}\label{sec:zero-shot-restoration}


\begin{figure*}[t]
    \centering
    \includegraphics[width=1.0\textwidth]{figures/posterior_sampling.pdf}
    \includegraphics{figures/linear_restoration_main_text_v2.pdf}
    \caption{\textbf{Comparison of zero-shot posterior sampling image restoration methods.} Our approach achieves better perceptual quality compared to previous methods, while maintaining competitive PSNR and automatically producing compressed bit-stream representations for each restored image.}
    \label{fig:zero-shot-posterior-sampling-qualitative}
\end{figure*}

Our compressed conditional sampling approach can be utilized for solving inverse problems via posterior sampling.
Specifically, we consider inverse problems of the form $\rvy=\mA\rvx_{0}$, where $\mA$ is some linear operator.
We restrict our attention to \emph{unconditional} diffusion models and solve the problems in a ``zero-shot'' manner (similarly to \citet{kawar2022denoising,chung2023diffusion,wang2023zeroshot}). 
To generate conditional samples, we propose using the loss
\begin{align}
\gL(\rvy,\rvx_{i},\gC_{i},k)=\norm{\rvy-\mA(\vmu_{i}(\rvx_{i})+\sigma_{i}\gC_{i}(k))}^{2}.\label{eq:l_posterior_sampling}
\end{align}
Note that~\Cref{eq:l_posterior_sampling} attains a lower value when $\sigma_{i}\mA\gC_{i}(k)$ points in the direction that perturbs $\mA\vmu_{i}(\rvx_{i})$ towards $\rvy$.
Thus, our conditional generative process aims to produce a reconstruction $\hat{\rvx}$ that satisfies $\mA\hat{\rvx}\approx\rvy$, implying that we approximate posterior sampling~\citep{pmlr-v202-ohayon23a}.
Notably, when assuming that $p_{i}(\rvy|\rvx_{i})$ is a multivariate normal distribution centered around $\mA\rvx_{i}$ (as in~\citep{jalal2021posterior}), the chosen codebook noise $\gC_{i}(k_{i})$ approximates the gradient $\nabla_{\rvx_{i}}\log{p_{i}(\rvy|\rvx_{i})}$ and~\Cref{eq:l_posterior_sampling} becomes a proxy of~\Cref{eq:l_score}.

Following~\citep{chung2023diffusion,wang2023zeroshot}, we implement our method using the unconditional ImageNet $256\times 256$ DDM trained by \citet{dhariwal2021diffusion}.
We fix $K=4096$ for all codebooks, resulting in a compressed bit-stream of approximately $0.183$ BPP for each generated image.
We compare our method with DPS~\citep{chung2023diffusion} and DDNM~\citep{wang2023zeroshot} on two noiseless tasks: colorization and $4\times $ super-resolution (using the bicubic kernel). We evaluate these methods using their official implementations and the same DDM.
We additionally compress the outputs of DPS and DDNM to assess whether such a naive approach would yield better results.
To do so, we adopt our proposed compression scheme (from \Cref{section:compression}), employing the same unconditional ImageNet DDM and using $K=4096$ noises per codebook.


Qualitative and quantitative results are reported in \Cref{fig:zero-shot-posterior-sampling-qualitative}.
As expected, due to the rate-perception-distortion tradeoff~\citep{pmlr-v97-blau19a}, we observe that compressing the outputs of DPS and DDNM harms either their perceptual quality (FID), or their distortion (PSNR), or both.
This is while our method achieves superior perceptual quality compared to both DPS and DDNM, including their compressed versions.
While our method achieves slightly worse PSNR, this is expected due to the perception-distortion trade-off~\citep{Blau_2018_CVPR}.
See \Cref{appendix:zero-shot} for more details.





\subsection{Compressed Real-World Face Image Restoration}\label{sec:bfr}
\begin{figure*}[t]
    \centering
    \includegraphics[width=1\textwidth]{figures/blind_face_restoration/real-world-restoration-wider.pdf}
\includegraphics[width=1\textwidth]{figures/blind_face_restoration/blind_face_restoration_main_text_wider.pdf}
    \caption{\textbf{Comparing real-world face image restoration methods on the WIDER-Test dataset}. We successfully optimize the NR-IQA measures and produce appealing output perceptual quality with less artifacts compared to previous methods.}
    \label{fig:real-world-wider-visual}
\end{figure*}
Real-world face image restoration is the practical task of restoring any degraded face image, without any knowledge of the corruption process it has gone through~\citep{wang2021gfpgan,vqfr,wang2022restoreformer,zhou2022codeformer,wang2023restoreformer++,2023diffbir,difface,bfrfussion,pmrf}.
We propose a novel method capable of optimizing any no-reference image quality assessment (NR-IQA) measure at test time (e.g., NIQE~\citep{niqe}), without relying on gradients.

Specifically, at each timestep $i$, we start by picking two indices -- one that promotes high perceptual quality, $k_{i,P}$, and another that promotes low distortion, $k_{i,D}$.
Then, we choose between $k_{i,P}$ and $k_{i,D}$ the index that better optimizes a desired balance of the perception-distortion tradeoff~\citep{Blau_2018_CVPR}.
Formally, letting $\vr(\rvy)\approx\mathbb{E}[\rvx_{0}|\rvy]$ denote the approximate Minimum Mean-Squared-Error (MMSE) estimator of this task, we pick $k_{i,D}$ via
\begin{align}
    k_{i,D}=\argmax_{k\in\{1,\hdots,K\}}\langle\gC_{i}(k),\vr(\rvy)-\hat{\rvx}_{0|i}\rangle.\label{eq:first_index_blind_face_restoration}
\end{align}
Note that this index selection rule is similar to that of our standard compression, replacing $\smash{\rvx_{0}}$ in \Cref{eq:compression_rule} with $\smash{\vr(\rvy)}$.
This choice of indices in DDCM would lead to a reconstructed estimate of the MMSE solution $\vy(\rvy)$, yielding blurry results with low distortion~\citep{Blau_2018_CVPR}.
In contrast, \emph{randomly} picking a sequence of indices in DDCM would produce a high quality sample from the data distribution $p_{0}$.
Therefore, we randomly choose $\smash{k_{i,P}\sim \text{Unif}(\{1,\hdots,K\})}$.
Then, we use the DDM and compute $\smash{\hat{\rvx}_{0|i-1}}$ for each index $\smash{k\in\{k_{i,D},k_{i,P}\}}$ separately, denoting each result accordingly by $\hat{\rvx}_{0|i-1}^{(k)}$.
The final index is picked to optimize the perception-distortion tradeoff via
\begin{align}
    k_{i}\!=\!\argmin_{k\in\{k_{i,D},k_{i,P}\}}\!\text{MSE}\!\left(\!\vr(\rvy),\hat{\rvx}_{0|i-1}^{(k)}\!\right)\!+\!\lambda Q\!\left(\!\hat{\rvx}_{0|i-1}^{(k)}\!\right)\!,\label{eq:optimize-perception-distortion-indices}
\end{align}
where $Q(\cdot)$ can be \emph{any} NR-IQA measure, even a non-differentiable one.
In \Cref{app:dmax-ot} we explain our choice to set $\vr(\rvy)$ as an MMSE estimator.


We assess our approach choosing $\vr(\rvy)$ as the FFHQ~\citep{stylegan} $512\times 512$ approximate MMSE model trained by \citet{difface}.
We set $\lambda=1$ and optimize three different $Q(\cdot)$ measures: NIQE, $\text{CLIP-IQA}^{+}$~\citep{Wang_Chan_Loy_2023}, and TOPIQ~\citep{chen2024topiq} adapted for face images by PyIQA~\citep{pyiqa}.
We utilize the FFHQ $512\times512$ DDM of \citet{difface} with $T=1000$ sampling steps and $K=4096$ for all codebooks.
We compare our approach against the state-of-the-art methods PMRF~\citep{pmrf}, DifFace~\citep{difface}, and BFRffusion~\citep{bfrfussion}, using the standard evaluation datasets CelebA-Test~\citep{karras2018progressive,wang2021gfpgan}, LFW-Test~\citep{lfw-original}, WebPhoto-Test~\citep{wang2021gfpgan}, and WIDER-Test~\citep{zhou2022codeformer}.
We use PSNR to measure the distortion of the outputs produced for the CelebA-Test dataset, and measure the ProxPSNR~\citep{pmrf,man2025proxiesdistortionconsistencyapplications} for the other datasets, which lack the clean original images.
Perceptual quality is measured by NIQE, $\text{CLIP-IQA}^{+}$, TOPIQ-FACE, and additionally $\text{FD}_{\text{DINOv2}}$~\citep{stein2023exposing} to assess our generalization performance to a common quality measure which we do not directly optimize.
Finally, as in \Cref{sec:zero-shot-restoration}, we \emph{compress} each evaluated method using our standard compression scheme, adopting the same FFHQ DDM with $K=4096$ and $T=1000$.

The results for the WIDER-Test dataset are reported in \Cref{fig:real-world-wider-visual} (see \Cref{app:dmax-ot} for the other datasets).
Our approach clearly optimizes each quality measure effectively and generalizes well according to the $\text{FD}_{\text{DINOv2}}$ scores.
This is also confirmed visually, where all of our solutions produce high-quality images with less artifacts compared to previous methods.
While our approach shows slightly worse distortion, this is once again expected due to the perception-distortion tradeoff~\citep{Blau_2018_CVPR}.

This work identifies signal collapse as a critical bottleneck in one-shot neural network pruning. Performance loss in pruned networks is due to \textbf{signal collapse} in addition to the removal of critical parameters. We propose \textbf{REFLOW} (\textbf{Re}storing \textbf{F}low of \textbf{Low}-variance signals), a simple yet effective method that mitigates signal collapse without computationally expensive weight updates. By focusing on signal preservation, REFLOW highlights the importance of mitigating signal collapse in sparse networks and enables magnitude pruning to match or surpass state-of-the-art one-shot pruning methods such as CHITA, CBS, and WF.

REFLOW consistently achieves state-of-the-art accuracy across diverse architectures, restoring ResNeXt-101 from under 4.1\% to 78.9\% top-1 accuracy at 80\% sparsity on ImageNet. Its lightweight design makes it a practical solution for both research and deployment, delivering high-quality sparse models without the overhead of traditional approaches. These findings challenge the traditional emphasis on weight selection strategies and underscore the critical role of signal propagation for achieving high-quality sparse networks in the context of one-shot pruning.




% \clearpage

\section*{Acknowledgements}
This research was partially supported by the Israel Science Foundation (ISF) under Grants 2318/22, 951/24 and 409/24, and by the Council for Higher Education – Planning and Budgeting Committee.
We thank Noam Elata and Matan Kleiner for assisting with our figures and experiments.

\section*{Impact Statement}
This paper presents work whose goal is to advance the field of Machine Learning.
There are many potential societal consequences of our work, none which we feel must be specifically highlighted here.
\clearpage

\bibliographystyle{icml2025}
\bibliography{citations}


\newpage
\appendix
\onecolumn
\section{Additional DDCM Evaluation}\label{app:more_random_gens}

Figure~\ref{fig:generation_other_metrics} provides additional quantitative comparisons between DDPM and DDCM, using different $K$ values.
Specifically, we compute the Kernel Inception Distance (KID)~\citep{bińkowski2018demystifying}, as well as the Fréchet Distance and Kernel Distance evaluated in the feature space of DINOv2~\citep{stein2023exposing,oquab2024dinov}.
These quantitative results remain consistent with the ones presented in \Cref{sec:method}, showing that DDCM with small $K$ values is comparable with DDPM.
Figures~\ref{fig:generation_samples_latent_app3} and \ref{fig:generation_samples_pixel_app3} show numerous outputs from both DDPM and DDCM with small values of $K$, demonstrating the sample quality and diversity produced by the latter for such $K$ values.
We use Torch Fidelity~\citep{obukhov2020torchfidelity} to compute the perceptual quality measures.
% generation stuff
\begin{figure}[H]
    \centering
    \includegraphics[width=0.5\linewidth]{figures/generation_other_metrics.pdf}
    \caption{\textbf{Comparing DDPM with DDCM for different codebook sizes $K$.} 
    As in \Cref{sec:method}, DDPM and DDCM with $K=64$ (sometimes even $K=16$) achieve similar generative performance, suggesting that the continuous representation space of DDPM (DDCM with $K =\infty$) is highly redundant.
    We use a class-conditional ImageNet model ($256 \times 256$) for pixel space, and the text-conditional SD 2.1 model ($768 \times 768$) for latent space. The $K$ axis is in log-scale.
    }
    \label{fig:generation_other_metrics}
\end{figure}


% \begin{figure*}[t]
%     \centering
%     \includegraphics[width=\linewidth]{figures/SD2.1_gens_moreims_part1.pdf}
%     \caption{\textbf{Qualitative Comparison of DDCMs vs DDPMs.}
%     We generate samples using DDCM with different $K$ values as well as using  DDPM sampling, on random prompts from MS-COCO. }
%     \label{fig:generation_samples_latent_app1}
% \end{figure*}

\begin{figure*}[t]
    \centering
    \includegraphics[width=\linewidth]{figures/SD2.1_16random_samples_0.pdf}\vspace{0.2cm}
    \includegraphics[width=\linewidth]{figures/SD2.1_16random_samples_1.pdf}\vspace{0.2cm}
    % \includegraphics[width=\linewidth]{figures/SD2.1_16random_samples_2.pdf}
    \caption{\textbf{Qualitative comparison of sample quality and diversity between DDCM and DDPM.}
    We generate multiple samples for each prompt, using the $768\times768$ SD 2.1 model.}
    \label{fig:generation_samples_latent_app3}
\end{figure*}

\begin{figure*}[t]
    \centering
    \includegraphics[width=1\textwidth]{figures/pixel_space_rand_gen.pdf}
    \caption{\textbf{Qualitative comparison of sample quality and diversity between DDCM and DDPM.}
    We generate multiple samples for each class, using the conditional $256\times 256$ ImageNet model.}
    \label{fig:generation_samples_pixel_app3}
\end{figure*}
\clearpage


\section{Image Compression Supplementary}\label{app:compression_details}

We compute all distortion and perceptual quality measures using \href{https://github.com/Lightning-AI/torchmetrics}{Torch Metrics} (which relies on Torch Fidelity~\citep{obukhov2020torchfidelity}).

\subsection{Experiment Configurations}\label{app:image-compression-experiments-configuration}
We specify here the different configurations used for the compression experiments in \Cref{section:compression}.
\begin{itemize}[parsep=2pt,itemsep=2pt]
    \item In our $256\times256$ experiments we use the \texttt{256x256\_diffusion\_uncond.pt} checkpoint from the \href{https://github.com/openai/guided-diffusion}{official GitHub repository}.
    In our $768\times768$ and $512\times512$ experiments we use Stable Diffusion 2.1 with the \href{https://huggingface.co/stabilityai/stable-diffusion-2-1}{\texttt{stabilityai/stable-diffusion-2-1}} and \href{https://huggingface.co/stabilityai/stable-diffusion-2-1-base}{\texttt{stabilityai/stable-diffusion-2-1-base}} official checkpoints from Hugging Face, respectively.
    The different $K$, $M$, $C$ and $T$ values for each of the $512\times512$ and the $768\times768$ experiments plotted in \Cref{fig:compression_graphs,fig:compression_graphs_768} are summarized in \Cref{tab:our_tmkc_configs}.
    

\begin{table}[h]
\centering
\caption{Image compression experiments configurations.}
\begin{tabular}{c|c|c|c|c|c}
\hline
Model & Image Resolution & $T$ & $K$ & $M$ & $C$ \\
\hline
\multirow{4}{*}{Pixel Space DDM} & \multirow{4}{*}{256$\times$256} & 1000 & 64, 128, 256, 4096 & 1 & - \\
& & 1000 & 2048 & 2, 3, 4, 5 & 3 \\
& & 500 & 128, 512 & 1 & - \\
& & 300 & 16, 32, 128, 512 & 1 & - \\
\hline
\multirow{6}{*}{Latent Space DDM} & \multirow{2}{*}{512$\times$512} & 1000 & 256, 1024, 8192 & 1 & - \\
& & 1000 & 2048 & 2, 3, 6 & 3 \\
\cline{2-6}
& \multirow{4}{*}{768$\times$768} & 1000 & 16, 32, 64, 256, 1024, 8192 & 1 & - \\
& & 1000 & 2048 & 2, 3, 6 & 3 \\
& & 500 & 16, 32, 64, 256, 1024, 8192 & 1 & - \\
& & Adapted 500 (\Cref{app:range_t}) & 16, 32, 64, 256, 1024, 8192 & 1 & - \\
\hline
\end{tabular}
\label{tab:our_tmkc_configs}
\end{table}

    % For th


% pixel
% 1000 - K=64 256 1024 4096
% 500 - K=128 512
% 300 - K=16 32 128 512 
% pursuit - p=2,4,3,5, | K= 2048 | C= 3
% latent - 
% 1000 - K=256 1024 8192
% pur - C=3, P=2,3,6
    
    % For the $768\times768$ image size experiments, we use $K\in\{16, 32, 64, 256, 1024,  8192\}$.
    \item For PSC-D and PSC-R we use the same pre-trained ImageNet $256\times256$ model as ours in the $256\times256$ experiments, and the same Stable Diffusion 2.1 model in the $512\times512$ experiments. We adopt the default hyper-parameters of the method as described by the authors~\citep{elata2024zero}, setting the number of measurements to $12\cdot2^i$ for $i=0,\ldots,8$.
    \item For IPIC we adopt the official implementation using the ELIC codec with five bit rates, combined with DPS sampling for decoding with $T=1000$ steps and $\zeta\in\{0.3, 0.6, 0.6, 1.2, 1.6\}$, as recommended by the authors.
    \item For BPG we considered quality factors $q\in\{51,50,48,46,42,40,38,36,34,32,30\}$. 
    \item For HiFiC we test the low, medium and high quality regimes, using the checkpoints available in the \href{https://github.com/Justin-Tan/high-fidelity-generative-compression}{official GitHub repository}.
    \item PerCo (SD) is tested using the three publicly available Stable Diffusion 2.1 fine-tuned checkpoints from their \href{https://github.com/Nikolai10/PerCo}{Official GitHub repository}, using the default hyper-parameters.
    \item For ILLM we use the MS-ILLM pre-trained models available in the \href{https://github.com/facebookresearch/NeuralCompression/tree/main/projects/illm}{official GitHub implementation}.
    For the $512\times512$ image size experiments we use \texttt{msillm\_quality\_X}, $\texttt{X}=2,3,4$. For the $768\times768$ image size experiments we use \texttt{msillm\_quality\_X}, $\texttt{X}=2,3$ and \texttt{msillm\_quality\_vloY}, $\texttt{Y}=1,2$.
    \item CRDR-D and CRDR-R are evaluated using quality factors of $\{0,1,2,3,4\}$, where CRDR-D uses $\beta=0$ and CRDR-R uses $\beta=3.84$, as recommended in the paper.
\end{itemize}


\subsection{Additional Evaluations}\label{app:compression_more_results}

In \Cref{fig:compression_examples_app,,fig:compression_examples_app_medium,,fig:pixel_space_comparison_low_bpp,,fig:pixel_space_comparison_mid_bpp} we provide additional qualitative comparisons on the Kodak24 ($512\times 512$) and ImageNet ($256\times 256$) datasets.
We additionally compare our method on images of size $768\times768$, and present the results in \Cref{fig:compression_graphs_768}.
Our method is implemented as before, while using a Stable Diffusion 2.1 model trained on the appropriate image size (see \Cref{app:image-compression-experiments-configuration}).

\subsection{Decreasing the Bit Rate via Timestep Sub-Sampling}\label{app:range_t}

As mentioned in \Cref{section:compression}, decreasing the bit rate of our compression scheme can be accomplished in two ways.
The first option is to reduce $K$, which sets the number of bits required to represent each communicated codebook index.
The second option relates to the number of generation timesteps, which sets the total number of communicated indices. Specifically, DDMs trained for $T=1000$ steps can still be used to generate samples with fewer steps, by skipping alternating timesteps and modifying the variance in \Cref{eq:ddpm}. 
Thus, we leverage such timestep sub-sampling in DDCM to shorten the compressed bit-stream.
We find that pixel space DDMs yields good results with this approach, while the latent space models struggle to produce satisfying perceptual quality.

Thus, for latent space models we propose a slightly different timestep sub-sampling scheme. 
Specifically, we keep $T=1000$ sampling steps at inference and set different $K$ values for different subsets of timesteps.
We choose $K=1$ for a subset of $L$ sampling steps, and $K>1$ for the rest $T-L$ steps.
Thus, our compression scheme only optimizes $T-L$ steps and necessitates transmitting only $T-L$ indices. The rest $L$ indices correspond to codebooks that contain only one vector, and thus do not affect the bit rate.

We use $T=1000$, set the same codebook size $K>1$ for every timestep $i\in\{899\ldots,400\}$, and use $K=1$ for all other steps.
We compare our proposed method against the aforementioned naive timestep skipping approach with $T=500$ sampling steps and the same $K>1$, which attains the same bit rate as our proposed alternative.

Quantitative results are shown in \Cref{fig:compression_graphs_768}, where our timestep adapted method is denoted by \emph{Ours Adapted}.
Our adapted approach achieves better perceptual quality compared to the naive one, at the expense of a slightly hindered PSNR.


\subsection{Increasing the Bit Rate via Matching Pursuit}\label{app:matching_pursuit}
In \Cref{section:compression} we briefly explain how to achieve higher bit rates with our method, by refining each selected noise via a matching pursuit inspired solution.
Formally, at each timestep $i$ we iteratively refine the selected noise by linearly combining it with $M-1$ other noises from the codebook.
We start by picking the first noise index $k_{i}$ according to \Cref{eq:compression_rule}.
Then, we set $k_{i}^{(1)}=k_{i}$, $\gamma_{i}^{(1)}=1$, and $\tilde{\vz}_{i}^{(1)}=\gC_{i}(k_{i})$, and pick the next indices and coefficients $(m=1,\hdots,M-1)$ via
\begin{align}
    k_i^{(m+1)}, \gamma_i^{(m+1)} = \argmax_{k\in\{1,\hdots,K\},\;\gamma \in \Gamma} \left\langle \gamma\tilde{\vz}_{i}^{(m)}+\left(1-\gamma\right)\gC_{i}(k) ,\,\rvx_0-\hat{\rvx}_{0|i}\right\rangle.
\end{align}
The noise vector $\tilde{\vz}_{i}^{(m+1)}$ is then updated via
\begin{align}
&\tilde{\vz}_{i}^{(m+1)}\leftarrow\gamma_{i}^{(m+1)}\tilde{\vz}_{i}^{(m)}+\left(1-\gamma_{i}^{(m+1)}\right)\gC_{i}\left(k_{i}^{(m+1)}\right),\\
&\tilde{\vz}_{i}^{(m+1)}\leftarrow \frac{\tilde{\vz}_{i}^{(m+1)}}{\text{std}\left(\tilde{\vz}_{i}^{(m+1)}\right)},
\end{align}
where $\text{std}(\vz)$ is the empirical standard deviation of the vector $\vz$.
We use the resulting vector $\tilde{\vz}_{i}^{(M)}$ as the noise in \Cref{eq:DDCM_sampling} to produce the next $\rvx_{i-1}$, and repeat the above process iteratively.
Note that setting $M=1$ is equivalent to our standard compression scheme.


In our experiments, the set of coefficients $\Gamma$ is a subset of $(0,1]$, containing $C=|\Gamma|$ values that are evenly spaced in this range.
We pick $C=3$ and assess $M\in\{2,3,6\}$ for the latent space model experiments, and $M\in\{2,3,4,5\}$ for the pixel space model experiments.




\subsection{Assessing the Effectiveness of Text Prompts in Compression using Text-to-Image Latent Space DDMs}\label{app:text_effect}

Stable Diffusion 2.1 is a text-to-image generative model, which both PerCo (SD) and PSC leverage for their compression approach.
Specifically, both of these methods start by generating a textual caption for every target image using BLIP-2~\citep{li2023blip}, and feed the captions as prompts to the SD model.
In our case, we find that using such prompts hinders the compression quality.
Specifically, we follow the same automatic captioning procedure as in PerCo (SD) and PSC, using the \texttt{Salesforce/blip2-opt-2.7b-coco} \href{https://huggingface.co/Salesforce/blip2-opt-2.7b-coco}{checkpoint} of BLIP-2 from Hugging Face.
We then continue with our standard compression approach, where the denoiser is used with standard classifier-free guidance (CFG).
Note that using text prompts requires transmitting additional bits that serve as a compressed version of the text.
Specifically, we use BLIP-2 with a maximum of $L=32$ word tokens, each picked from a dictionary containing a total of 30,524 words.
Thus, at most $32\cdot\log_{2}(30524)\approx 480$ bits are added to the bit-stream in our method.

We assess this text-conditional approach on the $512\times 512$ SD 2.1 DDM, using CFG scales of $3,6$.
We compare the performance of this conditional approach with that of the unconditional one we used in \Cref{section:compression}.
The results in \Cref{fig:compression_graphs_blip} show a disadvantage for using text-prompts for compression with our method.

\begin{figure*}[t]
    \centering
    \includegraphics[width=0.86\linewidth]{figures/Kodak24_512_extreme_app.pdf}
    \caption{\textbf{Qualitative extreme image compression results.} The presented images are taken from the Kodak24 dataset, cropped to $512\times512$ pixels.
    Our compression scheme produces highly realistic decompressed outputs, while maintaining better fidelity to the original images compared to previous methods.
    }
    \label{fig:compression_examples_app}
\end{figure*}


\begin{figure*}[t]
    \centering
    \includegraphics[height=0.93\textheight]{figures/Kodak24_512_med_app.pdf}
    \caption{
    \textbf{Qualitative image compression results.} The presented images are taken from the Kodak24 dataset, cropped to $512\times512$ pixels.
    Our compression scheme produces highly realistic decompressed outputs, while maintaining better fidelity to the original images compared to previous methods.
    }\label{fig:compression_examples_app_medium}
\end{figure*}

%pixel space compression compression
\begin{figure*}[t]
    \centering
    \includegraphics[width=\linewidth]{figures/pixel_space_comparison_low_bpp.pdf}
    \caption{\textbf{Qualitative image compression results.} the presented images are taken from the ImageNet $256\times 256$ dataset.
    Compared to previous methods, our compression scheme produces higher perceptual quality and better fidelity to the original images.
    }
    \label{fig:pixel_space_comparison_low_bpp}
\end{figure*}

\begin{figure*}[t]
    \centering
    \includegraphics[width=\linewidth]{figures/pixel_space_comparison_mid_bpp.pdf}
    \caption{\textbf{Qualitative image compression results.} the presented images are taken from the ImageNet $256\times 256$ dataset.
    Compared to previous methods, our compression scheme produces higher perceptual quality and better fidelity to the original images.
    }
    \label{fig:pixel_space_comparison_mid_bpp}
\end{figure*}


\begin{figure*}[t]
    \centering
    \includegraphics[width=1\linewidth]{figures/compression_graphs_768.pdf}
    \caption{\textbf{Quantitative image compression results on $768\times768$ sized images.} At higher bit rates, our method achieves the lowest (best) FID scores in both datasets while maintaining better distortion metrics compared to PerCo (SD). At extremely low bit rates, while PerCo (SD) shows marginally better FID scores, our method attains superior PSNR performance.}
    \label{fig:compression_graphs_768}
\end{figure*}


\begin{figure*}[t]
    \centering
    \includegraphics[width=0.9\linewidth]{figures/compression_graphs_blip.pdf}
    \caption{\textbf{Evaluating the effectiveness of using text prompts in image compression.}
    We evaluate our unconditional compression method with the text-conditional one, while using the text captions generated by BLIP-2.
    We find that using such text prompts hinders our compression results, both in terms of perceptual quality and distortion.}
    \label{fig:compression_graphs_blip}
\end{figure*}


\clearpage
\section{Compressed Conditional Generation Supplementary}\label{appendix:cond_compression}
\subsection{Background and Proof of~\Cref{prob:ode_convergence}}
We will prove that~\Cref{prob:ode_convergence} holds for any score-based diffusion model~\citep{song2020score}.
For completeness, we first provide the necessary mathematical background and then proceed to the proof of the proposition.
\subsubsection{Background}

\paragraph{Score-Based Generative Models.}Score-based generative models~\citep{song2020score} define a diffusion process $\smash{\{\rvx(t):t\in [0,T]\}}$, where $p_{0}$ and $p_{T}$ denote the data distribution and the prior distribution, respectively, and $p_{t}$ denotes the distribution of $x(t)$.
Such a diffusion process can generally be modeled as the stochastic differential equation (SDE)
\begin{align}
\delt\rvx=f(\rvx,t)\delt t+g(t)\delt\rvw,\label{eq:forward_sde}
\end{align}
where $f(\cdot,t)$ is called the \emph{drift} coefficient, $g(t)$ is called the \emph{diffusion} coefficient, $\rvw(t)$ is a standard Wiener process, and $\delt t$ denotes an infinitesimal timestep.
Samples from the data distribution $p_{0}$ can be generated by solving the reverse-time SDE~\citep{ANDERSON1982313},
\begin{align}
    &\delt\rvx=\left[f(\rvx,t)-g^{2}(t)\vs_{t}(\rvx)\right]\delt t+g(t)\delt\bar{\rvw},\label{eq:reverse_sde}
\end{align}
starting from samples of $\rvx(T)$.
Here, $\vs_{t}(\rvx(t))\coloneqq\nabla_{\rvx(t)}\log{p_{t}(\rvx(t))}$ is the \emph{score} of $p_{t}$, $\bar{\rvw}(t)$ denotes a standard Wiener process where time flows backwards, and $\delt t$ is an infinitesimal \emph{negative} timestep.
Samples from the data distribution can also be generated by solving the \emph{probability flow} ODE,
\begin{align}
    &\delt\rvx=\left[f(\rvx,t)-\frac{1}{2}g^{2}(t)\vs_{t}(\rvx)\right]\delt t.\label{eq:reverse_ode}
\end{align}

\paragraph{Solving The Reverse-Time SDE.}The reverse-time SDE in~\Cref{eq:reverse_sde} can be solved with any numerical SDE solver (e.g., Euler-Maruyama), which corresponds to some time discretization of the forward and reverse stochastic dynamics.
For the sake of our proof, we adopt the simple solver proposed by~\citet{song2020score},
\begin{align}
    &\rvx_{i-1}=\rvx_{i}-f_{i}(\rvx_{i})+g_{i}^{2}\rvs_{i}(\rvx_{i})+g_{i}\rvz_{i},\quad\rvz_{i}\sim\mathcal{N}(\vzero,\mI),\label{eq:song_discretization}
\end{align}
where $i=T,\hdots,1$ and $f_{i}$ and $g_{i}$ are the time-discretized versions of $f$ and $g$, respectively.
Note that DDPMs~\citep{ho2020denoising} are score-based diffusion models that solve a reverse-time Variance Preserving (VP) SDE, where $f(\rvx(t),t)=-\frac{1}{2}\beta(t)\rvx(t)$ and $g(t)=\sqrt{\beta(t)}$ for some function $\beta$.

\subsubsection{Proof of~\Cref{prob:ode_convergence}}
Given any general score-based diffusion model, we can write the DDCM compressed conditional generation process as
\begin{align}
    &\rvx_{i-1}=\rvx_{i}-f_{i}(\rvx_{i})+g_{i}^{2}\rvs_{i}(\rvx_{i})+g_{i}\gC_{i}(k_{i}),\label{eq:score-based-ddcm}
\end{align}
where $k_{i}$ are picked according to~\Cref{eq:k_choose_conditional}.
Choosing $\gL=\gL_{\text{P}}$, we have
\begin{align}
    k_{i}=\argmin_{k\in\{1,\hdots,K\}}\norm{\gC_{i}(k)-g_{i}\nabla_{\rvx_{i}}\log{p_{i}(\rvy|\rvx_{i})}}^{2}.
\end{align}
Since each $\gC_{i}$ contains $K$ independent samples drawn from a normal distribution $\gN(\vzero,\mI)$, we have
\begin{align}
    \{\gC_{i}(1),\hdots,\gC_{i}(K)\}\underset{K\rightarrow\infty}{\longrightarrow}\mathbb{R}^{n},
\end{align}
where $n$ denotes the dimensionality of each vector in $\gC_{i}$, and $\{\gC_{i}(1),\hdots,\gC_{i}(K)\}$ is the set comprised of all the elements in the $\gC_{i}$ (without repetition).
Since $g_{i}\nabla_{\rvx_{i}}\log{p_{i}(\rvy|\rvx_{i})}\in\mathbb{R}^{n}$, we have
\begin{align}
    \min_{k\in\{1,\hdots,K\}}\norm{\gC_{i}(k)-g_{i}\nabla_{\rvx_{i}}\log{p_{i}(\rvy|\rvx_{i})}}^{2}\underset{K\rightarrow\infty}{\longrightarrow}0.
\end{align}
Thus,
\begin{align}
    \gC_{i}(k_{i})\underset{K\rightarrow\infty}{\longrightarrow}g_{i}\nabla_{\rvx_{i}}\log{p_{i}(\rvy|\rvx_{i})}.\label{eq:noise_convergence_to_grad}
\end{align}
Plugging~\Cref{eq:noise_convergence_to_grad} into~\Cref{eq:score-based-ddcm}, we get
\begin{align}
    \rvx_{i-1}\underset{K\rightarrow\infty}{\longrightarrow}&\rvx_{i}-f_{i}(\rvx_{i})+g_{i}^{2}\rvs_{i}(\rvx_{i})+g_{i}^{2}\nabla_{\rvx_{i}}\log{p_{i}(\rvy|\rvx_{i})}\\
    =&\rvx_{i}-f_{i}(\rvx_{i})+g_{i}^{2}\nabla_{\rvx_{i}}\log{p_{i}(\rvx_{i})}+g_{i}^{2}\nabla_{\rvx_{i}}\log{p_{i}(\rvy|\rvx_{i})}\\
    =&\rvx_{i}-f_{i}(\rvx_{i})+g_{i}^{2}\left[\nabla_{\rvx_{i}}\log{p_{i}(\rvx_{i})}+\nabla_{\rvx_{i}}\log{p_{i}(\rvy|\rvx_{i})}\right]\\
    =&\rvx_{i}-f_{i}(\rvx_{i})+g_{i}^{2}\nabla_{\rvx_{i}}\log{p_{i}(\rvx_{i}|\rvy)},\label{eq:bayes_rule_for_ode}
\end{align}
where in~\Cref{eq:bayes_rule_for_ode} we used Bayes rule and the fact that $\nabla_{\rvx_{i}}\log{p(\rvy)}=0$.
Note that~\Cref{eq:bayes_rule_for_ode} resembles a time discretization of a probability flow ODE (\Cref{eq:reverse_ode}) over the posterior distribution $p_{0}(\rvx_{0}|\rvy)$, with $f(\cdot,t)$ and $\sqrt{2}g(t)$ being the drift and diffusion coefficients in continuous time, respectively.
Thus, when $K\rightarrow\infty$, our compressed conditional generation process becomes a sampler from $p_{0}(\rvx_{0}|\rvy)$.

% for every $\gamma>0$ and $k$ we have
% \begin{align}
%     \mathbb{P}(\norm{\gC_{i}(k)-g_{i}\nabla_{\rvx_{i}}\log{p_{i}(\rvy|\rvx_{i})}}^{2}\leq\gamma)>0.
% \end{align}

\subsection{Image Compression as a Private Case of Compressed Conditional Generation}\label{appendix:compression_private_case}
We show that our standard image compression scheme from~\Cref{section:compression} is a private case of our compressed conditional generation scheme from~\Cref{sec:compressed_conditional_generation}, where $\rvy=\rvx_{0}$ and $\gL=\gL_{\text{P}}$.
When $\rvy=\rvx_{0}$ we have
\begin{align}
    \nabla_{\rvx_{i}}\log{p_{i}(\rvy|\rvx_{i})}&=\nabla_{\rvx_{i}}\log{p_{i}(\rvx_{0}|\rvx_{i})}\nonumber\\
    &=\nabla_{\rvx_{i}}\log{p_{i}(\rvx_{i}|\rvx_{0})}-\nabla_{\rvx_{i}}\log{p_{i}(\rvx_{i})}\\
    &=\nabla_{\rvx_{i}}\log{p_{i}(\rvx_{i}|\rvx_{0})}-\vs_{i}(\rvx_{i})\label{eq:compression-grad}
\end{align}
where $\nabla_{\rvx_{i}}\log{p_{i}(\rvx_{i}|\rvx_{0})}$ can be computed in closed-form via the forward diffusion process in~\Cref{eq:forward_gaussian_diffusion}.
In particular, we have $p_{i}(\rvx_{i}|\rvx_{0})=\gN(\rvx_{i};\sqrt{\bar{\alpha}_{i}}\rvx_{0},(1-\bar{\alpha}_{i})\mI)$~\citep{ho2020denoising}, and thus
\begin{align}
    \nabla_{\rvx_{i}}\log{p_{i}(\rvx_{i}|\rvx_{0})}&=-\nabla_{\rvx_{i}}\frac{\norm{\rvx_{i}-\sqrt{\bar{\alpha}_{i}}\rvx_{0}}^{2}}{2(1-\bar{\alpha}_{i})}\\
    &=-\frac{\rvx_{i}-\sqrt{\bar{\alpha}_{i}}\rvx_{0}}{1-\bar{\alpha}_{i}}\\
    &=\frac{\sqrt{\bar{\alpha}_{i}}\rvx_{0}-\rvx_{i}}{1-\bar{\alpha}_{i}}.\label{eq:vp-sde-forward-grad}
\end{align}
Moreover, it is well known that~\citep{ho2020denoising,song2020score}
\begin{align}
    \vs_{i}(\rvx_{i})=\frac{\sqrt{\bar{\alpha}_{i}}\hat{\rvx}_{0|i}-\rvx_{i}}{1-\bar{\alpha}_{i}},\label{eq:score-mmse-in-vp-sde}
\end{align}
where $\hat{\rvx}_{0|i}$ is the (time-aware) Minimum Mean-Squared-Error (MMSE) estimator of $\rvx_{0}$ given $\rvx_{i}$.
Plugging~\Cref{eq:score-mmse-in-vp-sde,eq:vp-sde-forward-grad} into~\cref{eq:compression-grad}, we get
\begin{align}
    \nabla_{\rvx_{i}}\log{p_{i}(\rvx_{0}|\rvx_{i})}=\frac{\sqrt{\bar{\alpha}_{i}}}{1-\bar{\alpha}_{i}}(\rvx_{0}-\hat{\rvx}_{0|i}).\label{eq:compression_vp_sde_likelihood}
\end{align}
Thus, we have
\begin{align}
    k_{i}&=\argmin_{k\in\{1,\hdots,K\}}\gL_{\text{P}}(\rvy,\rvx_{i},\gC_{i},k)\label{eq:compression_posterior}\\
    &=\argmin_{k\in\{1,\hdots,K\}}\bignorm{\gC_{i}(k)-\sigma_{i}\frac{\sqrt{\bar{\alpha}_{i}}}{1-\bar{\alpha}_{i}}(\rvx_{0}-\hat{\rvx}_{0|i})}^{2}\\
    &=\argmin_{k\in\{1,\hdots,K\}}\bignorm{\gC_{i}(k)}^{2}-2\langle\gC_{i}(k),\sigma_{i}\frac{\sqrt{\bar{\alpha}_{i}}}{1-\bar{\alpha}_{i}}(\rvx_{0}-\hat{\rvx}_{0|i})\rangle+\bignorm{\sigma_{i}\frac{\sqrt{\bar{\alpha}_{i}}}{1-\bar{\alpha}_{i}}(\rvx_{0}-\hat{\rvx}_{0|i})}^{2}\\
    &=\argmin_{k\in\{1,\hdots,K\}}\bignorm{\gC_{i}(k)}^{2}-2\langle\gC_{i}(k),\sigma_{i}\frac{\sqrt{\bar{\alpha}_{i}}}{1-\bar{\alpha}_{i}}(\rvx_{0}-\hat{\rvx}_{0|i})\rangle.
\end{align}
Below, we show that $\norm{\gC_{i}(k)}^{2}$ is roughly equal for every $k$.
Thus, it holds that
\begin{align}
    k_{i}&=\argmin_{k\in\{1,\hdots,K\}}\bignorm{\gC_{i}(k)}^{2}-2\langle\gC_{i}(k),\sigma_{i}\frac{\sqrt{\bar{\alpha}_{i}}}{1-\bar{\alpha}_{i}}(\rvx_{0}-\hat{\rvx}_{0|i})\rangle\\
    &\approx\argmin_{k\in\{1,\hdots,K\}}\text{const}-2\langle\gC_{i}(k),\sigma_{i}\frac{\sqrt{\bar{\alpha}_{i}}}{1-\bar{\alpha}_{i}}(\rvx_{0}-\hat{\rvx}_{0|i})\rangle\\
    &=\argmax_{k\in\{1,\hdots,K\}}\langle\gC_{i}(k),\sigma_{i}\frac{\sqrt{\bar{\alpha}_{i}}}{1-\bar{\alpha}_{i}}(\rvx_{0}-\hat{\rvx}_{0|i})\rangle\\
    &=\argmax_{k\in\{1,\hdots,K\}}\sigma_{i}\frac{\sqrt{\bar{\alpha}_{i}}}{1-\bar{\alpha}_{i}}\langle\gC_{i}(k),\rvx_{0}-\hat{\rvx}_{0|i}\rangle\\
    &=\argmax_{k\in\{1,\hdots,K\}}\langle\gC_{i}(k),\rvx_{0}-\hat{\rvx}_{0|i}\rangle.\label{eq:equiv_kstar_compression}
\end{align}
Note that the noise selection strategy in~\Cref{eq:equiv_kstar_compression} is similar to that of our standard compression scheme, namely~\Cref{eq:compression_rule}.
Thus, our compression method is a private case of our compressed conditional generation approach.
In practice, we used~\Cref{eq:compression_rule} instead of~\Cref{eq:compression_posterior} since the former worked slightly better.


To show that $\norm{\gC_{i}(k)}^{2}$ is roughly constant for every $k$, note that $\gC_{i}(k)$ is a sample from a $n$-dimensional multivariate normal distribution $\mathcal{N}(\vzero,\mI)$.
Thus, $\norm{\gC_{i}(k)}^{2}$ is a sample from a chi-squared distribution with $n$ degrees of freedom.
It is well known that samples from this distribution strongly concentrate around its mean $n$ for large values of $n$.
Namely, $\norm{\gC_{i}(k)}^{2}$ is highly likely to be close to $n$, especially for relatively small values of $K$.
\clearpage
\subsection{Compressed Posterior Sampling for Image Restoration}\label{appendix:zero-shot}
DPS and DDNM are implemented with the official settings recommended by the authors~\citep{chung2023diffusion,wang2023zeroshot}.
Specifically, DPS uses DDPM with $T=1000$ sampling steps, and DDNM uses DDIM with $\eta=0.85$ and $T=100$ sampling steps.
We also tried $T=1000$ for DDNM and found that $T=100$ works slightly better for the tasks considered.
The additional qualitative comparisons in \Cref{fig:srx4_additional_samples,fig:colorization_additional_samples} further demonstrate that our method produces better output perceptual quality compared to DPS and DDNM.
\begin{figure*}[t]
    \centering
    \includegraphics[width=1\textwidth]{figures/srx4_additional_samples.pdf}
    \caption{\textbf{Qualitative comparison of zero-shot image super-resolution methods (posterior sampling).} Our approach clearly produces better output perceptual quality compared to previous methods.}
    \label{fig:srx4_additional_samples}
\end{figure*}
\begin{figure*}[t]
    \centering
    \includegraphics[width=1\textwidth]{figures/colorization_additional_samples.pdf}
    \caption{\textbf{Qualitative comparison of zero-shot image colorization methods (posterior sampling).} Our approach clearly produces better output perceptual quality compared to previous methods.}
    \label{fig:colorization_additional_samples}
\end{figure*}
\clearpage
\subsection{Compressed Real-World Face Image Restoration}\label{app:dmax-ot}
\subsubsection{Explaining the Choice of $\vr(\rvy)$}
To explain our choice of $\vr(\rvy)\approx\mathbb{E}[\rvx_{0}|\rvy]$, first note that the MSE of \emph{any} estimator $\hat{\rvx}_{0}$ of $\rvx_{0}$ given $\rvy$ can be written as~\citep{freirich2021a}
\begin{align}
    \text{MSE}(\rvx_{0},\hat{\rvx}_{0})
    &=\text{MSE}(\rvx_{0},\vr(\rvy))+\text{MSE}(\vr(\rvy),\hat{\rvx}_{0})\nonumber\\
    &=\text{MSE}(\vr(\rvy),\hat{\rvx}_{0})+c\label{eq:orthogonality}.
\end{align}
where $c$, the MMSE, does not depend on $\hat{\rvx}_{0}$.
In theory, our goal is to optimize the tradeoff between the MSE of $\hat{\rvx}_0$ and its output perceptual quality according to some quality measure $Q$. This can be accomplished by solving
\begin{align}\label{eq:desired_tradeoff}
    \min_{\hat{\rvx}_{0}}\text{MSE}(\rvx_{0},\hat{\rvx}_{0})+\lambda Q(\hat{\rvx}_{0}),
\end{align}
where $\lambda$ is some hyper-parameter that controls the perception-distortion tradeoff.
At test time, however, we do not have access to the original image $\rvx_0$.
By plugging \Cref{eq:orthogonality} into \Cref{eq:desired_tradeoff} we obtain
\begin{align}
    \min_{\hat{\rvx}_{0}}\text{MSE}(\rvx_{0},\hat{\rvx}_{0})+\lambda Q(\hat{\rvx}_{0})&=\min_{\hat{\rvx}_{0}}\text{MSE}(\vr(\rvy),\hat{\rvx}_{0})+c+\lambda Q(\hat{\rvx}_{0})\nonumber\\
    &=\min_{\hat{\rvx}_{0}}\text{MSE}(\vr(\rvy),\hat{\rvx}_{0})+\lambda Q(\hat{\rvx}_{0}).\label{eq:opt-mse-real-world}
\end{align}
Namely, as long as $\vr(\rvy)$ is a good approximation of the \emph{true} MMSE estimator, we can rely on it for optimizing tradeoff (\ref{eq:desired_tradeoff}) without having access to $\rvx_{0}$.
This resembles our approach in \Cref{sec:bfr}, where we \emph{greedily} optimize \Cref{eq:opt-mse-real-world} throughout the trajectory of the DDCM sampling process.



\subsubsection{Additional Details and Experiments}
We use the PyIQA package to compute all perceptual quality measures, with \texttt{niqe} for NIQE, \texttt{topiq\_nr-face} for TOPIQ, \texttt{clipiqa+} for $\text{CLIP-IQA}^{+}$, and \texttt{fid\_dinov2} for $\text{FD}_{\text{DINOv2}}$, adopting the default settings for each measure.
Additional qualitative and quantitative comparisons are shown in \cref{fig:real-world-qualitative-appendix,,fig:lfw-quantitative-appendix,,fig:celeba-quantitative-appendix,,fig:webphoto-quantitative-appendix}.

\begin{figure}
    \centering
    \includegraphics[height=21cm]{figures/blind_face_restoration/real-world-comparison-appendix.pdf}
    \caption{\textbf{Qualitative comparison of real-world face image restoration methods}. Our method produces high perceptual quality results with less artifacts compared to previous methods, especially for challenging datasets such as WIDER-Test.}
    \label{fig:real-world-qualitative-appendix}
\end{figure}
\begin{figure}
    \centering
    \includegraphics[width=1\textwidth]{figures/blind_face_restoration/blind_face_restoration_main_text_celeba.pdf}
    \caption{\textbf{Quantitative comparison of real-world face image restoration methods, evaluated on the CelebA-Test dataset.} We successfully optimize each NR-IQA measure, surpassing the scores of previous methods. Here, only our NIQE-based solution generalizes well to $\text{FD}_{\text{DINOv2}}$ in terms of perceptual quality.}
    \label{fig:celeba-quantitative-appendix}
\end{figure}
\begin{figure}
    \centering
    \includegraphics[width=1\textwidth]{figures/blind_face_restoration/blind_face_restoration_main_text_lfw.pdf}
    \caption{\textbf{Quantitative comparison of real-world face image restoration methods, evaluated on the LFW-Test dataset.} We successfully optimize each NR-IQA measure, surpassing the scores of previous methods. All our solutions achieve impressive $\text{FD}_{\text{DINOv2}}$ scores, while our NIQE-based solution surpasses all methods according to this measure.}
    \label{fig:lfw-quantitative-appendix}
\end{figure}
\begin{figure}
    \centering
    \includegraphics[width=1\textwidth]{figures/blind_face_restoration/blind_face_restoration_main_text_webphoto.pdf}
    \caption{\textbf{Quantitative comparison of real-world face image restoration methods, evaluated on the WebPhoto-Test dataset.} We successfully optimize each NR-IQA measure, surpassing the scores of previous methods. Our TOPIQ-based solution achieves the best $\text{FD}_{\text{DINOv2}}$ scores compared to all methods..}
    \label{fig:webphoto-quantitative-appendix}
\end{figure}

\clearpage


\subsection{Compressed Classifier Guidance}\label{appendix:classifier-guidance}
\begin{figure*}[t]
    \centering
    \includegraphics[height=15cm]{figures/our_cg_256noises_2optnoises.jpg}\hspace{1cm}\includegraphics[height=15cm]{figures/standard_cg_scale=20.jpg}
    \caption{\textbf{Qualitative comparison of CCG (left) with CG (right).} CCG achieves superior image quality compared to CG while avoiding the use of classifier gradients. Additionally, CCG enables decompression without requiring access to the original class labels.}
    \label{fig:cg_visual_comparison}
\end{figure*}
Consider the case where $\rvy$ represents the \emph{class} of an image $\rvx_{0}$.
An unconditional score-based generative model can be \emph{guided} to generate samples the posterior $p_{0}(\rvx_{0}|\rvy)$, by perturbing the generated samples according to the gradient of a time-dependent trained classifier $c_{\theta}(\rvy;\rvx_{i},i)\approx p_{i}(\rvy|\rvx_{i})$~\citep{dhariwal2021diffusion}.
This approach is known as classifier guidance (CG).
Such a guidance method can be interpreted as an attempt to \emph{confuse} the classifier by perturbing its input adversarially~\citep{ho2021classifier}.
However, trained classifiers are typically not robust to adversarial perturbations, making their gradients largely unreliable and unaligned with human perception~\citep{advattacks,tsipras2018robustness,ganz-perceptual}.
Thus, the standard CG approach has not seen major success~\citep{ho2021classifier}.


We propose an alternative to this method, circumventing the reliance on the classifier's gradient.
Specifically, we set $\gL$ in \Cref{eq:k_choose_conditional} as
\begin{align}
\gL(\rvy,\rvx_{i},\gC_{i},k)=-\log{c_{\theta}(\rvy;\vmu_{i}(\rvx_{i})+\sigma_{i}\gC_{i}(k),i)}.\label{eq:classifier-guidance-ours}
\end{align}
Thus, $\gL(\rvy,\rvx_{i},\gC_{i},k)$ attains a lower value when $\sigma_{i}\gC_{i}(k)$ points in some direction that maximizes the probability of the class $\rvy$.
Note that since the codebooks remain fixed, choosing $k_{i}$ (out of $1,\hdots,K$) to minimize~\Cref{eq:k_choose_conditional} would always lead to the same generated sample for every $\rvy$.
Thus, we promote sample diversity by first \emph{randomly} selecting a subset of $\tilde{K}<K$ indices $k_{i,1},\hdots,k_{i,\tilde{K}}\sim\text{Unif}(\{1,\hdots,K\})$, and then choosing
\begin{align}
    k_{i}=\argmin_{k\in\{k_{i,1},\hdots,k_{i,\tilde{K}}\}}\gL(\rvy,\rvx_{i},\gC_{i},k).
\end{align}
We coin our method Compressed CG (CCG).


We compare our proposed CCG with the standard CG using the same unconditional diffusion model and time-dependent classifier trained on ImageNet $256\times256$~\citep{deng2009imagenet,dhariwal2021diffusion}.
We compare the methods ``on the same grounds'' by using the same standard DDPM noise schedule and $T=1000$ diffusion steps.
Our method is assessed with $K=256$ and $\tilde{K}=2$, while the standard CG is assessed with CG scales $s\in\{1,10,20\}$.
The quantitative comparison in~\Cref{tab:cg_comparison} shows that CCG achieves better (lower) FID and $\text{FD}_{\text{DINOv2}}$ scores.
A visual comparison is provided in~\Cref{fig:cg_visual_comparison}.
Note that while using DDCM with standard CG does still produces compressed output images, decoding the produced bit-streams requires access to $\rvy$.
Using DDCM with CCG instead sidesteps this limitation, as $\rvy$ is not needed for decompression.

\begin{table}[ht]
\centering
\caption{Quantitative comparison of \emph{compressed} classifier guidance (CCG) and standard classifier guidance (CG) for ImageNet $256\times 256$ conditional image generation, using an unconditional DDM and a classifier for guidance. Our proposed CCG not only outperforms CG in terms of generation performance, but also automatically produces compressed image representations.}
\begin{tabular}{lcc}
\toprule
 &  \begin{tabular}{c}Compressed CG \textbf{(Ours)}\\$K=256,\tilde{K}=2$\end{tabular}& \begin{tabular}{c}Standard CG\\$s=1\mid10\mid20$\end{tabular} \\
\midrule
FID & $\mathbf{13.669}$ & $31.548\mid14.481\mid 14.921$ \\
$\text{FD}_{\text{DINOv2}}$ & $\mathbf{204.693}$ & $459.42 \mid255.41 \mid248.34 $ \\
\bottomrule
\end{tabular}
\label{tab:cg_comparison}
\end{table}

\clearpage
\subsection{Compressed Classifier-Free Guidance}\label{appendix:classifier-free-guidance}
\begin{figure*}[t]
    \centering
    \includegraphics[width=\textwidth]{figures/ccfg_fig_1.pdf}
    \caption{\textbf{Qualitative comparison of CCFG with CFG.} CCFG achieves comparable image quality and diversity to CFG, while enabling decompression without requiring the original inputs.}
    \label{fig:ccfg_visual_1}
\end{figure*}

\begin{figure*}[t]
    \centering
    \includegraphics[width=\textwidth]{figures/ccfg_fig_2.pdf}
    \caption{\textbf{Qualitative comparison of CCFG with CFG.} CCFG achieves comparable image quality and diversity to CFG, while enabling decompression without requiring the original inputs.}
    \label{fig:ccfg_visual_2}
\end{figure*}

\begin{figure*}
    \centering
    \includegraphics[width=0.5\textwidth]{figures/ccfg_cfg_comparison.pdf}
    \caption{\textbf{Quantitative evaluation of CCFG and CFG.} CCFG achieves comparable FID scores to CFG while achieving slightly lower fidelity to the input prompts. However, unlike CFG, CCFG enables decompression without access to the original conditioning inputs.}
    \label{fig:ccfg-cf-quantitative-comparison}
\end{figure*}
The task of text-conditional image generation can be solved using a \emph{conditional} diffusion model, which, theoretically speaking, learns to sample from the posterior distribution $p_{0}(\rvx_{0}|\rvy)$.
In practice, however, using a conditional model directly typically yields low fidelity to the inputs.
To address this limitation, CG can be used to improve this fidelity at the expense of sample quality and diversity~\citep{dhariwal2021diffusion}.
Classifier-Free Guidance (CFG) is used more often in practice, as it achieves the same tradeoff by mixing the conditional and unconditional scores during sampling~\citep{ho2021classifier}, thus eliminating the need for a classifier.
Particularly, assuming we have access to both the conditional score $\vs_{i}(\rvx_{i},\rvy)\coloneqq\nabla_{\rvx_{i}}\log{p_{i}(\rvx_{i}|\rvy)}$ and the unconditional one $\vs_{i}(\rvx_{i})$, CFG proposes to modify the conditional score by
\begin{align}
    \tilde{\vs}_{i}(\rvx_{i},\rvy)=(1+w)\vs_{i}(\rvx_{i},\rvy)-w\vs_{i}(\rvx_{i}),
\end{align}
where $w$, the CFG scale, is a hyper-parameter controlling the tradeoff between sample quality and diversity.

Here, we introduce a new CFG method that allows generating compressed conditional samples using any pair of conditional and unconditional diffusion models, while controlling the tradeoff between generation quality and the fidelity to the inputs.
Specifically, since $\nabla_{\rvx_{i}}\log{p_{i}(\rvy|\rvx_{i})}=\vs_{i}(\rvx_{i}|\rvy)-\vs_{i}(\rvx_{i})$, we simply use
\begin{align}
    \gL(\rvy,\rvx_{i},\gC_{i},k)=-\langle\gC_{i}(k),\vs_{i}(\rvx_{i}|\rvy)-\vs_{i}(\rvx_{i})\rangle.\label{eq:l_ccfg}
\end{align}
Note that optimizing~\Cref{eq:l_ccfg} is roughly  equivalent to optimizing $\gL_{\text{P}}$ when $\rvx_{i}$ is high dimensional (see~\Cref{appendix:compression_private_case}).
As in~\Cref{appendix:classifier-guidance}, we promote sample diversity by choosing $k_{i}$ from a randomly sampled subset of $\tilde{K}<K$ indices at each step during the generation.
We coin our method Compressed CFG (CCFG).

We implement our method using SD 2.1 trained on $768\times768$ images, adopting a DDPM noise schedule with $T=1000$ diffusion steps, $K=64$ fixed vectors in each codebook and $\tilde{K}\in\{2,3,4,6,9\}$.
We compare against the same diffusion model with standard DDPM sampling, using $T=1000$ steps and CFG scales $w\in\{2,5,8,11\}$.
The generative performance of both methods is assessed by computing the FID between 10k generated samples and MS-COCO, similarly to \Cref{sec:method}.
Additionally, we evaluate the alignment between the outputs and the input text prompts using the CLIP score~\citep{hessel2021clipscore} with the OpenAI CLIP ViT-L/14 model~\citep{pmlr-v139-radford21a}.

Figure~\ref{fig:ccfg-cf-quantitative-comparison} shows that our CCFG method is on par with CFG in terms of FID, while CFG produces higher CLIP scores.
This suggests that the outputs of CFG better align with the input text prompts compared to CCFG.
Yet, the qualitative comparisons in \Cref{fig:ccfg_visual_1,fig:ccfg_visual_2} show that there is no significant difference between the methods.
Importantly, decoding the bit-streams produced by CCFG does involve accessing the original input $\rvy$, and so our loss in CLIP scores are expected due to the rate-perception-distortion tradeoff~\citep{pmlr-v97-blau19a} (here, we achieve $\frac{1000\cdot\log_{2}(64)}{768^2}\approx 0.01$ BPP).
Note that using CCFG in DDCM is fundamentally different than using CFG (\Cref{sec:method}), since the latter requires access to $\rvy$.
\clearpage
% NOTE:
% All editing commands contain curly braces {} around them so that 
% they can easily be searched for in the final document via cmd + f.


% Highlighted comments for authors
\usepackage{soul, color}
\newcommand{\Note}[1]{}
\renewcommand{\Note}[1]{#1}  % comment out this definition to suppress all Notes
\newcommand{\NoteSigned}[3]{{\textcolor{#2}{\Note{\textbf{[#1: #3]}}}}}
% Alternative with highlighter color rather than text colour
%\renewcommand{\NoteSigned}[3]{{\sethlcolor{#2}\Note{\hl{#1: #3}}}}

%%%%%%%%%%%%%%%%%% Make changes here %%%%%%%%%%%%%%%%%%%%%%
% Specific author comments
% \newcommand{\vd}[1]{\NoteSigned{VD}{blue}{#1}}
% \newcommand{\fv}[1]{\NoteSigned{FV}{green}{#1}}
\newcommand{\kd}[1]{\NoteSigned{KD}{brown}{#1}}
\newcommand{\sm}[1]{\NoteSigned{SM}{magenta}{#1}}
%%%%%%%%%%%%%%%%%%%%%%%%%%%%%%%%%%%%%%%%%%%%%%%%%%%%%%%%%%%

% Inline todo: Flag something in thext to be done
\newcommand{\itodo}[1]{\NoteSigned{TODO}{red}{#1}}
% Preliminary: Flag preliminary statements or results
\newcommand{\prelim}[1]{\NoteSigned{Prelim}{grey}{#1}}
\end{document}