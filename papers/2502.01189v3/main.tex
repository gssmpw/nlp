%%%%%%%% ICML 2025 EXAMPLE LATEX SUBMISSION FILE %%%%%%%%%%%%%%%%%

\documentclass{article}

%
% --- inline annotations
%
\newcommand{\red}[1]{{\color{red}#1}}
\newcommand{\todo}[1]{{\color{red}#1}}
\newcommand{\TODO}[1]{\textbf{\color{red}[TODO: #1]}}
% --- disable by uncommenting  
% \renewcommand{\TODO}[1]{}
% \renewcommand{\todo}[1]{#1}



\newcommand{\VLM}{LVLM\xspace} 
\newcommand{\ours}{PeKit\xspace}
\newcommand{\yollava}{Yo’LLaVA\xspace}

\newcommand{\thisismy}{This-Is-My-Img\xspace}
\newcommand{\myparagraph}[1]{\noindent\textbf{#1}}
\newcommand{\vdoro}[1]{{\color[rgb]{0.4, 0.18, 0.78} {[V] #1}}}
% --- disable by uncommenting  
% \renewcommand{\TODO}[1]{}
% \renewcommand{\todo}[1]{#1}
\usepackage{slashbox}
% Vectors
\newcommand{\bB}{\mathcal{B}}
\newcommand{\bw}{\mathbf{w}}
\newcommand{\bs}{\mathbf{s}}
\newcommand{\bo}{\mathbf{o}}
\newcommand{\bn}{\mathbf{n}}
\newcommand{\bc}{\mathbf{c}}
\newcommand{\bp}{\mathbf{p}}
\newcommand{\bS}{\mathbf{S}}
\newcommand{\bk}{\mathbf{k}}
\newcommand{\bmu}{\boldsymbol{\mu}}
\newcommand{\bx}{\mathbf{x}}
\newcommand{\bg}{\mathbf{g}}
\newcommand{\be}{\mathbf{e}}
\newcommand{\bX}{\mathbf{X}}
\newcommand{\by}{\mathbf{y}}
\newcommand{\bv}{\mathbf{v}}
\newcommand{\bz}{\mathbf{z}}
\newcommand{\bq}{\mathbf{q}}
\newcommand{\bff}{\mathbf{f}}
\newcommand{\bu}{\mathbf{u}}
\newcommand{\bh}{\mathbf{h}}
\newcommand{\bb}{\mathbf{b}}

\newcommand{\rone}{\textcolor{green}{R1}}
\newcommand{\rtwo}{\textcolor{orange}{R2}}
\newcommand{\rthree}{\textcolor{red}{R3}}
\usepackage{amsmath}
%\usepackage{arydshln}
\DeclareMathOperator{\similarity}{sim}
\DeclareMathOperator{\AvgPool}{AvgPool}

\newcommand{\argmax}{\mathop{\mathrm{argmax}}}     



% The \icmltitle you define below is probably too long as a header.
% Therefore, a short form for the running title is supplied here:
\icmltitlerunning{\dynamictitle}

\begin{document}

\twocolumn[
\icmltitle{
\dynamictitle
}

% It is OKAY to include author information, even for blind
% submissions: the style file will automatically remove it for you
% unless you've provided the [accepted] option to the icml2025
% package.

% List of affiliations: The first argument should be a (short)
% identifier you will use later to specify author affiliations
% Academic affiliations should list Department, University, City, Region, Country
% Industry affiliations should list Company, City, Region, Country

% You can specify symbols, otherwise they are numbered in order.
% Ideally, you should not use this facility. Affiliations will be numbered
% in order of appearance and this is the preferred way.
\icmlsetsymbol{equal}{*}

\begin{icmlauthorlist}
\icmlauthor{Guy Ohayon}{equal,techcs}
\icmlauthor{Hila Manor}{equal,techee}
\icmlauthor{Tomer Michaeli}{techee}
\icmlauthor{Michael Elad}{techcs}
%\icmlauthor{}{sch}
%\icmlauthor{}{sch}
\end{icmlauthorlist}

\icmlaffiliation{techcs}{Faculty of Computer Science, Technion -- Israel Institute of Technology, Haifa, Israel}
\icmlaffiliation{techee}{Faculty of Electrical and Computer Engineering, Technion -- Israel Institute of Technology, Haifa, Israel}

\icmlcorrespondingauthor{Guy Ohayon}{guyoep@gmail.com}
\icmlcorrespondingauthor{Hila Manor}{hila.manor@campus.technion.ac.il}

% You may provide any keywords that you
% find helpful for describing your paper; these are used to populate
% the "keywords" metadata in the PDF but will not be shown in the document
\icmlkeywords{Machine Learning, ICML}

% {\vspace{0.75em}%
% \centering
% \includegraphics[width=1\textwidth]{figures/teaser.pdf}
% \vspace{-1em}
% \captionof{figure}{Our proposed scheme (DDCMs) produces visually appealing image samples with high compression ratios (given at the bottom-right corner of each result). \textbf{Left}: Compression result; \textbf{Middle}: Compressed image synthesis; and \textbf{Right}: Real-world image restoration.  
% \vspace{-0.75em}}
% \label{fig:teaser}
% }
\vskip 0.3in

]



% this must go after the closing bracket ] following \twocolumn[ ...

% This command actually creates the footnote in the first column
% listing the affiliations and the copyright notice.
% The command takes one argument, which is text to display at the start of the footnote.
% The \icmlEqualContribution command is standard text for equal contribution.
% Remove it (just {}) if you do not need this facility.

%\printAffiliationsAndNotice{}  % leave blank if no need to mention equal contribution
\printAffiliationsAndNotice{\icmlEqualContribution} % otherwise use the standard text.

\begin{abstract}  
Test time scaling is currently one of the most active research areas that shows promise after training time scaling has reached its limits.
Deep-thinking (DT) models are a class of recurrent models that can perform easy-to-hard generalization by assigning more compute to harder test samples.
However, due to their inability to determine the complexity of a test sample, DT models have to use a large amount of computation for both easy and hard test samples.
Excessive test time computation is wasteful and can cause the ``overthinking'' problem where more test time computation leads to worse results.
In this paper, we introduce a test time training method for determining the optimal amount of computation needed for each sample during test time.
We also propose Conv-LiGRU, a novel recurrent architecture for efficient and robust visual reasoning. 
Extensive experiments demonstrate that Conv-LiGRU is more stable than DT, effectively mitigates the ``overthinking'' phenomenon, and achieves superior accuracy.
\end{abstract}  
\section{Introduction}


\begin{figure}[t]
\centering
\includegraphics[width=0.6\columnwidth]{figures/evaluation_desiderata_V5.pdf}
\vspace{-0.5cm}
\caption{\systemName is a platform for conducting realistic evaluations of code LLMs, collecting human preferences of coding models with real users, real tasks, and in realistic environments, aimed at addressing the limitations of existing evaluations.
}
\label{fig:motivation}
\end{figure}

\begin{figure*}[t]
\centering
\includegraphics[width=\textwidth]{figures/system_design_v2.png}
\caption{We introduce \systemName, a VSCode extension to collect human preferences of code directly in a developer's IDE. \systemName enables developers to use code completions from various models. The system comprises a) the interface in the user's IDE which presents paired completions to users (left), b) a sampling strategy that picks model pairs to reduce latency (right, top), and c) a prompting scheme that allows diverse LLMs to perform code completions with high fidelity.
Users can select between the top completion (green box) using \texttt{tab} or the bottom completion (blue box) using \texttt{shift+tab}.}
\label{fig:overview}
\end{figure*}

As model capabilities improve, large language models (LLMs) are increasingly integrated into user environments and workflows.
For example, software developers code with AI in integrated developer environments (IDEs)~\citep{peng2023impact}, doctors rely on notes generated through ambient listening~\citep{oberst2024science}, and lawyers consider case evidence identified by electronic discovery systems~\citep{yang2024beyond}.
Increasing deployment of models in productivity tools demands evaluation that more closely reflects real-world circumstances~\citep{hutchinson2022evaluation, saxon2024benchmarks, kapoor2024ai}.
While newer benchmarks and live platforms incorporate human feedback to capture real-world usage, they almost exclusively focus on evaluating LLMs in chat conversations~\citep{zheng2023judging,dubois2023alpacafarm,chiang2024chatbot, kirk2024the}.
Model evaluation must move beyond chat-based interactions and into specialized user environments.



 

In this work, we focus on evaluating LLM-based coding assistants. 
Despite the popularity of these tools---millions of developers use Github Copilot~\citep{Copilot}---existing
evaluations of the coding capabilities of new models exhibit multiple limitations (Figure~\ref{fig:motivation}, bottom).
Traditional ML benchmarks evaluate LLM capabilities by measuring how well a model can complete static, interview-style coding tasks~\citep{chen2021evaluating,austin2021program,jain2024livecodebench, white2024livebench} and lack \emph{real users}. 
User studies recruit real users to evaluate the effectiveness of LLMs as coding assistants, but are often limited to simple programming tasks as opposed to \emph{real tasks}~\citep{vaithilingam2022expectation,ross2023programmer, mozannar2024realhumaneval}.
Recent efforts to collect human feedback such as Chatbot Arena~\citep{chiang2024chatbot} are still removed from a \emph{realistic environment}, resulting in users and data that deviate from typical software development processes.
We introduce \systemName to address these limitations (Figure~\ref{fig:motivation}, top), and we describe our three main contributions below.


\textbf{We deploy \systemName in-the-wild to collect human preferences on code.} 
\systemName is a Visual Studio Code extension, collecting preferences directly in a developer's IDE within their actual workflow (Figure~\ref{fig:overview}).
\systemName provides developers with code completions, akin to the type of support provided by Github Copilot~\citep{Copilot}. 
Over the past 3 months, \systemName has served over~\completions suggestions from 10 state-of-the-art LLMs, 
gathering \sampleCount~votes from \userCount~users.
To collect user preferences,
\systemName presents a novel interface that shows users paired code completions from two different LLMs, which are determined based on a sampling strategy that aims to 
mitigate latency while preserving coverage across model comparisons.
Additionally, we devise a prompting scheme that allows a diverse set of models to perform code completions with high fidelity.
See Section~\ref{sec:system} and Section~\ref{sec:deployment} for details about system design and deployment respectively.



\textbf{We construct a leaderboard of user preferences and find notable differences from existing static benchmarks and human preference leaderboards.}
In general, we observe that smaller models seem to overperform in static benchmarks compared to our leaderboard, while performance among larger models is mixed (Section~\ref{sec:leaderboard_calculation}).
We attribute these differences to the fact that \systemName is exposed to users and tasks that differ drastically from code evaluations in the past. 
Our data spans 103 programming languages and 24 natural languages as well as a variety of real-world applications and code structures, while static benchmarks tend to focus on a specific programming and natural language and task (e.g. coding competition problems).
Additionally, while all of \systemName interactions contain code contexts and the majority involve infilling tasks, a much smaller fraction of Chatbot Arena's coding tasks contain code context, with infilling tasks appearing even more rarely. 
We analyze our data in depth in Section~\ref{subsec:comparison}.



\textbf{We derive new insights into user preferences of code by analyzing \systemName's diverse and distinct data distribution.}
We compare user preferences across different stratifications of input data (e.g., common versus rare languages) and observe which affect observed preferences most (Section~\ref{sec:analysis}).
For example, while user preferences stay relatively consistent across various programming languages, they differ drastically between different task categories (e.g. frontend/backend versus algorithm design).
We also observe variations in user preference due to different features related to code structure 
(e.g., context length and completion patterns).
We open-source \systemName and release a curated subset of code contexts.
Altogether, our results highlight the necessity of model evaluation in realistic and domain-specific settings.





\putsec{related}{Related Work}

\noindent \textbf{Efficient Radiance Field Rendering.}
%
The introduction of Neural Radiance Fields (NeRF)~\cite{mil:sri20} has
generated significant interest in efficient 3D scene representation and
rendering for radiance fields.
%
Over the past years, there has been a large amount of research aimed at
accelerating NeRFs through algorithmic or software
optimizations~\cite{mul:eva22,fri:yu22,che:fun23,sun:sun22}, and the
development of hardware
accelerators~\cite{lee:cho23,li:li23,son:wen23,mub:kan23,fen:liu24}.
%
The state-of-the-art method, 3D Gaussian splatting~\cite{ker:kop23}, has
further fueled interest in accelerating radiance field
rendering~\cite{rad:ste24,lee:lee24,nie:stu24,lee:rho24,ham:mel24} as it
employs rasterization primitives that can be rendered much faster than NeRFs.
%
However, previous research focused on software graphics rendering on
programmable cores or building dedicated hardware accelerators. In contrast,
\name{} investigates the potential of efficient radiance field rendering while
utilizing fixed-function units in graphics hardware.
%
To our knowledge, this is the first work that assesses the performance
implications of rendering Gaussian-based radiance fields on the hardware
graphics pipeline with software and hardware optimizations.

%%%%%%%%%%%%%%%%%%%%%%%%%%%%%%%%%%%%%%%%%%%%%%%%%%%%%%%%%%%%%%%%%%%%%%%%%%
\myparagraph{Enhancing Graphics Rendering Hardware.}
%
The performance advantage of executing graphics rendering on either
programmable shader cores or fixed-function units varies depending on the
rendering methods and hardware designs.
%
Previous studies have explored the performance implication of graphics hardware
design by developing simulation infrastructures for graphics
workloads~\cite{bar:gon06,gub:aam19,tin:sax23,arn:par13}.
%
Additionally, several studies have aimed to improve the performance of
special-purpose hardware such as ray tracing units in graphics
hardware~\cite{cho:now23,liu:cha21} and proposed hardware accelerators for
graphics applications~\cite{lu:hua17,ram:gri09}.
%
In contrast to these works, which primarily evaluate traditional graphics
workloads, our work focuses on improving the performance of volume rendering
workloads, such as Gaussian splatting, which require blending a huge number of
fragments per pixel.

%%%%%%%%%%%%%%%%%%%%%%%%%%%%%%%%%%%%%%%%%%%%%%%%%%%%%%%%%%%%%%%%%%%%%%%%%%
%
In the context of multi-sample anti-aliasing, prior work proposed reducing the
amount of redundant shading by merging fragments from adjacent triangles in a
mesh at the quad granularity~\cite{fat:bou10}.
%
While both our work and quad-fragment merging (QFM)~\cite{fat:bou10} aim to
reduce operations by merging quads, our proposed technique differs from QFM in
many aspects.
%
Our method aims to blend \emph{overlapping primitives} along the depth
direction and applies to quads from any primitive. In contrast, QFM merges quad
fragments from small (e.g., pixel-sized) triangles that \emph{share} an edge
(i.e., \emph{connected}, \emph{non-overlapping} triangles).
%
As such, QFM is not applicable to the scenes consisting of a number of
unconnected transparent triangles, such as those in 3D Gaussian splatting.
%
In addition, our method computes the \emph{exact} color for each pixel by
offloading blending operations from ROPs to shader units, whereas QFM
\emph{approximates} pixel colors by using the color from one triangle when
multiple triangles are merged into a single quad.


\section{Background}\label{sec:backgrnd}

\subsection{Cold Start Latency and Mitigation Techniques}

Traditional FaaS platforms mitigate cold starts through snapshotting, lightweight virtualization, and warm-state management. Snapshot-based methods like \textbf{REAP} and \textbf{Catalyzer} reduce initialization time by preloading or restoring container states but require significant memory and I/O resources, limiting scalability~\cite{dong_catalyzer_2020, ustiugov_benchmarking_2021}. Lightweight virtualization solutions, such as \textbf{Firecracker} microVMs, achieve fast startup times with strong isolation but depend on robust infrastructure, making them less adaptable to fluctuating workloads~\cite{agache_firecracker_2020}. Warm-state management techniques like \textbf{Faa\$T}~\cite{romero_faa_2021} and \textbf{Kraken}~\cite{vivek_kraken_2021} keep frequently invoked containers ready, balancing readiness and cost efficiency under predictable workloads but incurring overhead when demand is erratic~\cite{romero_faa_2021, vivek_kraken_2021}. While these methods perform well in resource-rich cloud environments, their resource intensity challenges applicability in edge settings.

\subsubsection{Edge FaaS Perspective}

In edge environments, cold start mitigation emphasizes lightweight designs, resource sharing, and hybrid task distribution. Lightweight execution environments like unikernels~\cite{edward_sock_2018} and \textbf{Firecracker}~\cite{agache_firecracker_2020}, as used by \textbf{TinyFaaS}~\cite{pfandzelter_tinyfaas_2020}, minimize resource usage and initialization delays but require careful orchestration to avoid resource contention. Function co-location, demonstrated by \textbf{Photons}~\cite{v_dukic_photons_2020}, reduces redundant initializations by sharing runtime resources among related functions, though this complicates isolation in multi-tenant setups~\cite{v_dukic_photons_2020}. Hybrid offloading frameworks like \textbf{GeoFaaS}~\cite{malekabbasi_geofaas_2024} balance edge-cloud workloads by offloading latency-tolerant tasks to the cloud and reserving edge resources for real-time operations, requiring reliable connectivity and efficient task management. These edge-specific strategies address cold starts effectively but introduce challenges in scalability and orchestration.

\subsection{Predictive Scaling and Caching Techniques}

Efficient resource allocation is vital for maintaining low latency and high availability in serverless platforms. Predictive scaling and caching techniques dynamically provision resources and reduce cold start latency by leveraging workload prediction and state retention.
Traditional FaaS platforms use predictive scaling and caching to optimize resources, employing techniques (OFC, FaasCache) to reduce cold starts. However, these methods rely on centralized orchestration and workload predictability, limiting their effectiveness in dynamic, resource-constrained edge environments.



\subsubsection{Edge FaaS Perspective}

Edge FaaS platforms adapt predictive scaling and caching techniques to constrain resources and heterogeneous environments. \textbf{EDGE-Cache}~\cite{kim_delay-aware_2022} uses traffic profiling to selectively retain high-priority functions, reducing memory overhead while maintaining readiness for frequent requests. Hybrid frameworks like \textbf{GeoFaaS}~\cite{malekabbasi_geofaas_2024} implement distributed caching to balance resources between edge and cloud nodes, enabling low-latency processing for critical tasks while offloading less critical workloads. Machine learning methods, such as clustering-based workload predictors~\cite{gao_machine_2020} and GRU-based models~\cite{guo_applying_2018}, enhance resource provisioning in edge systems by efficiently forecasting workload spikes. These innovations effectively address cold start challenges in edge environments, though their dependency on accurate predictions and robust orchestration poses scalability challenges.

\subsection{Decentralized Orchestration, Function Placement, and Scheduling}

Efficient orchestration in serverless platforms involves workload distribution, resource optimization, and performance assurance. While traditional FaaS platforms rely on centralized control, edge environments require decentralized and adaptive strategies to address unique challenges such as resource constraints and heterogeneous hardware.



\subsubsection{Edge FaaS Perspective}

Edge FaaS platforms adopt decentralized and adaptive orchestration frameworks to meet the demands of resource-constrained environments. Systems like \textbf{Wukong} distribute scheduling across edge nodes, enhancing data locality and scalability while reducing network latency. Lightweight frameworks such as \textbf{OpenWhisk Lite}~\cite{kravchenko_kpavelopenwhisk-light_2024} optimize resource allocation by decentralizing scheduling policies, minimizing cold starts and latency in edge setups~\cite{benjamin_wukong_2020}. Hybrid solutions like \textbf{OpenFaaS}~\cite{noauthor_openfaasfaas_2024} and \textbf{EdgeMatrix}~\cite{shen_edgematrix_2023} combine edge-cloud orchestration to balance resource utilization, retaining latency-sensitive functions at the edge while offloading non-critical workloads to the cloud. While these approaches improve flexibility, they face challenges in maintaining coordination and ensuring consistent performance across distributed nodes.


\section{Method}\label{sec:method}
\begin{figure}
    \centering
    \includegraphics[width=0.85\textwidth]{imgs/heatmap_acc.pdf}
    \caption{\textbf{Visualization of the proposed periodic Bayesian flow with mean parameter $\mu$ and accumulated accuracy parameter $c$ which corresponds to the entropy/uncertainty}. For $x = 0.3, \beta(1) = 1000$ and $\alpha_i$ defined in \cref{appd:bfn_cir}, this figure plots three colored stochastic parameter trajectories for receiver mean parameter $m$ and accumulated accuracy parameter $c$, superimposed on a log-scale heatmap of the Bayesian flow distribution $p_F(m|x,\senderacc)$ and $p_F(c|x,\senderacc)$. Note the \emph{non-monotonicity} and \emph{non-additive} property of $c$ which could inform the network the entropy of the mean parameter $m$ as a condition and the \emph{periodicity} of $m$. %\jj{Shrink the figures to save space}\hanlin{Do we need to make this figure one-column?}
    }
    \label{fig:vmbf_vis}
    \vskip -0.1in
\end{figure}
% \begin{wrapfigure}{r}{0.5\textwidth}
%     \centering
%     \includegraphics[width=0.49\textwidth]{imgs/heatmap_acc.pdf}
%     \caption{\textbf{Visualization of hyper-torus Bayesian flow based on von Mises Distribution}. For $x = 0.3, \beta(1) = 1000$ and $\alpha_i$ defined in \cref{appd:bfn_cir}, this figure plots three colored stochastic parameter trajectories for receiver mean parameter $m$ and accumulated accuracy parameter $c$, superimposed on a log-scale heatmap of the Bayesian flow distribution $p_F(m|x,\senderacc)$ and $p_F(c|x,\senderacc)$. Note the \emph{non-monotonicity} and \emph{non-additive} property of $c$. \jj{Shrink the figures to save space}}
%     \label{fig:vmbf_vis}
%     \vspace{-30pt}
% \end{wrapfigure}


In this section, we explain the detailed design of CrysBFN tackling theoretical and practical challenges. First, we describe how to derive our new formulation of Bayesian Flow Networks over hyper-torus $\mathbb{T}^{D}$ from scratch. Next, we illustrate the two key differences between \modelname and the original form of BFN: $1)$ a meticulously designed novel base distribution with different Bayesian update rules; and $2)$ different properties over the accuracy scheduling resulted from the periodicity and the new Bayesian update rules. Then, we present in detail the overall framework of \modelname over each manifold of the crystal space (\textit{i.e.} fractional coordinates, lattice vectors, atom types) respecting \textit{periodic E(3) invariance}. 

% In this section, we first demonstrate how to build Bayesian flow on hyper-torus $\mathbb{T}^{D}$ by overcoming theoretical and practical problems to provide a low-noise parameter-space approach to fractional atom coordinate generation. Next, we present how \modelname models each manifold of crystal space respecting \textit{periodic E(3) invariance}. 

\subsection{Periodic Bayesian Flow on Hyper-torus \texorpdfstring{$\mathbb{T}^{D}$}{}} 
For generative modeling of fractional coordinates in crystal, we first construct a periodic Bayesian flow on \texorpdfstring{$\mathbb{T}^{D}$}{} by designing every component of the totally new Bayesian update process which we demonstrate to be distinct from the original Bayesian flow (please see \cref{fig:non_add}). 
 %:) 
 
 The fractional atom coordinate system \citep{jiao2023crystal} inherently distributes over a hyper-torus support $\mathbb{T}^{3\times N}$. Hence, the normal distribution support on $\R$ used in the original \citep{bfn} is not suitable for this scenario. 
% The key problem of generative modeling for crystal is the periodicity of Cartesian atom coordinates $\vX$ requiring:
% \begin{equation}\label{eq:periodcity}
% p(\vA,\vL,\vX)=p(\vA,\vL,\vX+\vec{LK}),\text{where}~\vec{K}=\vec{k}\vec{1}_{1\times N},\forall\vec{k}\in\mathbb{Z}^{3\times1}
% \end{equation}
% However, there does not exist such a distribution supporting on $\R$ to model such property because the integration of such distribution over $\R$ will not be finite and equal to 1. Therefore, the normal distribution used in \citet{bfn} can not meet this condition.

To tackle this problem, the circular distribution~\citep{mardia2009directional} over the finite interval $[-\pi,\pi)$ is a natural choice as the base distribution for deriving the BFN on $\mathbb{T}^D$. 
% one natural choice is to 
% we would like to consider the circular distribution over the finite interval as the base 
% we find that circular distributions \citep{mardia2009directional} defined on a finite interval with lengths of $2\pi$ can be used as the instantiation of input distribution for the BFN on $\mathbb{T}^D$.
Specifically, circular distributions enjoy desirable periodic properties: $1)$ the integration over any interval length of $2\pi$ equals 1; $2)$ the probability distribution function is periodic with period $2\pi$.  Sharing the same intrinsic with fractional coordinates, such periodic property of circular distribution makes it suitable for the instantiation of BFN's input distribution, in parameterizing the belief towards ground truth $\x$ on $\mathbb{T}^D$. 
% \yuxuan{this is very complicated from my perspective.} \hanlin{But this property is exactly beautiful and perfectly fit into the BFN.}

\textbf{von Mises Distribution and its Bayesian Update} We choose von Mises distribution \citep{mardia2009directional} from various circular distributions as the form of input distribution, based on the appealing conjugacy property required in the derivation of the BFN framework.
% to leverage the Bayesian conjugacy property of von Mises distribution which is required by the BFN framework. 
That is, the posterior of a von Mises distribution parameterized likelihood is still in the family of von Mises distributions. The probability density function of von Mises distribution with mean direction parameter $m$ and concentration parameter $c$ (describing the entropy/uncertainty of $m$) is defined as: 
\begin{equation}
f(x|m,c)=vM(x|m,c)=\frac{\exp(c\cos(x-m))}{2\pi I_0(c)}
\end{equation}
where $I_0(c)$ is zeroth order modified Bessel function of the first kind as the normalizing constant. Given the last univariate belief parameterized by von Mises distribution with parameter $\theta_{i-1}=\{m_{i-1},\ c_{i-1}\}$ and the sample $y$ from sender distribution with unknown data sample $x$ and known accuracy $\alpha$ describing the entropy/uncertainty of $y$,  Bayesian update for the receiver is deducted as:
\begin{equation}
 h(\{m_{i-1},c_{i-1}\},y,\alpha)=\{m_i,c_i \}, \text{where}
\end{equation}
\begin{equation}\label{eq:h_m}
m_i=\text{atan2}(\alpha\sin y+c_{i-1}\sin m_{i-1}, {\alpha\cos y+c_{i-1}\cos m_{i-1}})
\end{equation}
\begin{equation}\label{eq:h_c}
c_i =\sqrt{\alpha^2+c_{i-1}^2+2\alpha c_{i-1}\cos(y-m_{i-1})}
\end{equation}
The proof of the above equations can be found in \cref{apdx:bayesian_update_function}. The atan2 function refers to  2-argument arctangent. Independently conducting  Bayesian update for each dimension, we can obtain the Bayesian update distribution by marginalizing $\y$:
\begin{equation}
p_U(\vtheta'|\vtheta,\bold{x};\alpha)=\mathbb{E}_{p_S(\bold{y}|\bold{x};\alpha)}\delta(\vtheta'-h(\vtheta,\bold{y},\alpha))=\mathbb{E}_{vM(\bold{y}|\bold{x},\alpha)}\delta(\vtheta'-h(\vtheta,\bold{y},\alpha))
\end{equation} 
\begin{figure}
    \centering
    \vskip -0.15in
    \includegraphics[width=0.95\linewidth]{imgs/non_add.pdf}
    \caption{An intuitive illustration of non-additive accuracy Bayesian update on the torus. The lengths of arrows represent the uncertainty/entropy of the belief (\emph{e.g.}~$1/\sigma^2$ for Gaussian and $c$ for von Mises). The directions of the arrows represent the believed location (\emph{e.g.}~ $\mu$ for Gaussian and $m$ for von Mises).}
    \label{fig:non_add}
    \vskip -0.15in
\end{figure}
\textbf{Non-additive Accuracy} 
The additive accuracy is a nice property held with the Gaussian-formed sender distribution of the original BFN expressed as:
\begin{align}
\label{eq:standard_id}
    \update(\parsn{}'' \mid \parsn{}, \x; \alpha_a+\alpha_b) = \E_{\update(\parsn{}' \mid \parsn{}, \x; \alpha_a)} \update(\parsn{}'' \mid \parsn{}', \x; \alpha_b)
\end{align}
Such property is mainly derived based on the standard identity of Gaussian variable:
\begin{equation}
X \sim \mathcal{N}\left(\mu_X, \sigma_X^2\right), Y \sim \mathcal{N}\left(\mu_Y, \sigma_Y^2\right) \Longrightarrow X+Y \sim \mathcal{N}\left(\mu_X+\mu_Y, \sigma_X^2+\sigma_Y^2\right)
\end{equation}
The additive accuracy property makes it feasible to derive the Bayesian flow distribution $
p_F(\boldsymbol{\theta} \mid \mathbf{x} ; i)=p_U\left(\boldsymbol{\theta} \mid \boldsymbol{\theta}_0, \mathbf{x}, \sum_{k=1}^{i} \alpha_i \right)
$ for the simulation-free training of \cref{eq:loss_n}.
It should be noted that the standard identity in \cref{eq:standard_id} does not hold in the von Mises distribution. Hence there exists an important difference between the original Bayesian flow defined on Euclidean space and the Bayesian flow of circular data on $\mathbb{T}^D$ based on von Mises distribution. With prior $\btheta = \{\bold{0},\bold{0}\}$, we could formally represent the non-additive accuracy issue as:
% The additive accuracy property implies the fact that the "confidence" for the data sample after observing a series of the noisy samples with accuracy ${\alpha_1, \cdots, \alpha_i}$ could be  as the accuracy sum  which could be  
% Here we 
% Here we emphasize the specific property of BFN based on von Mises distribution.
% Note that 
% \begin{equation}
% \update(\parsn'' \mid \parsn, \x; \alpha_a+\alpha_b) \ne \E_{\update(\parsn' \mid \parsn, \x; \alpha_a)} \update(\parsn'' \mid \parsn', \x; \alpha_b)
% \end{equation}
% \oyyw{please check whether the below equation is better}
% \yuxuan{I fill somehow confusing on what is the update distribution with $\alpha$. }
% \begin{equation}
% \update(\parsn{}'' \mid \parsn{}, \x; \alpha_a+\alpha_b) \ne \E_{\update(\parsn{}' \mid \parsn{}, \x; \alpha_a)} \update(\parsn{}'' \mid \parsn{}', \x; \alpha_b)
% \end{equation}
% We give an intuitive visualization of such difference in \cref{fig:non_add}. The untenability of this property can materialize by considering the following case: with prior $\btheta = \{\bold{0},\bold{0}\}$, check the two-step Bayesian update distribution with $\alpha_a,\alpha_b$ and one-step Bayesian update with $\alpha=\alpha_a+\alpha_b$:
\begin{align}
\label{eq:nonadd}
     &\update(c'' \mid \parsn, \x; \alpha_a+\alpha_b)  = \delta(c-\alpha_a-\alpha_b)
     \ne  \mathbb{E}_{p_U(\parsn' \mid \parsn, \x; \alpha_a)}\update(c'' \mid \parsn', \x; \alpha_b) \nonumber \\&= \mathbb{E}_{vM(\bold{y}_b|\bold{x},\alpha_a)}\mathbb{E}_{vM(\bold{y}_a|\bold{x},\alpha_b)}\delta(c-||[\alpha_a \cos\y_a+\alpha_b\cos \y_b,\alpha_a \sin\y_a+\alpha_b\sin \y_b]^T||_2)
\end{align}
A more intuitive visualization could be found in \cref{fig:non_add}. This fundamental difference between periodic Bayesian flow and that of \citet{bfn} presents both theoretical and practical challenges, which we will explain and address in the following contents.

% This makes constructing Bayesian flow based on von Mises distribution intrinsically different from previous Bayesian flows (\citet{bfn}).

% Thus, we must reformulate the framework of Bayesian flow networks  accordingly. % and do necessary reformulations of BFN. 

% \yuxuan{overall I feel this part is complicated by using the language of update distribution. I would like to suggest simply use bayesian update, to provide intuitive explantion.}\hanlin{See the illustration in \cref{fig:non_add}}

% That introduces a cascade of problems, and we investigate the following issues: $(1)$ Accuracies between sender and receiver are not synchronized and need to be differentiated. $(2)$ There is no tractable Bayesian flow distribution for a one-step sample conditioned on a given time step $i$, and naively simulating the Bayesian flow results in computational overhead. $(3)$ It is difficult to control the entropy of the Bayesian flow. $(4)$ Accuracy is no longer a function of $t$ and becomes a distribution conditioned on $t$, which can be different across dimensions.
%\jj{Edited till here}

\textbf{Entropy Conditioning} As a common practice in generative models~\citep{ddpm,flowmatching,bfn}, timestep $t$ is widely used to distinguish among generation states by feeding the timestep information into the networks. However, this paper shows that for periodic Bayesian flow, the accumulated accuracy $\vc_i$ is more effective than time-based conditioning by informing the network about the entropy and certainty of the states $\parsnt{i}$. This stems from the intrinsic non-additive accuracy which makes the receiver's accumulated accuracy $c$ not bijective function of $t$, but a distribution conditioned on accumulated accuracies $\vc_i$ instead. Therefore, the entropy parameter $\vc$ is taken logarithm and fed into the network to describe the entropy of the input corrupted structure. We verify this consideration in \cref{sec:exp_ablation}. 
% \yuxuan{implement variant. traditionally, the timestep is widely used to distinguish the different states by putting the timestep embedding into the networks. citation of FM, diffusion, BFN. However, we find that conditioned on time in periodic flow could not provide extra benefits. To further boost the performance, we introduce a simple yet effective modification term entropy conditional. This is based on that the accumulated accuracy which represents the current uncertainty or entropy could be a better indicator to distinguish different states. + Describe how you do this. }



\textbf{Reformulations of BFN}. Recall the original update function with Gaussian sender distribution, after receiving noisy samples $\y_1,\y_2,\dots,\y_i$ with accuracies $\senderacc$, the accumulated accuracies of the receiver side could be analytically obtained by the additive property and it is consistent with the sender side.
% Since observing sample $\y$ with $\alpha_i$ can not result in exact accuracy increment $\alpha_i$ for receiver, the accuracies between sender and receiver are not synchronized which need to be differentiated. 
However, as previously mentioned, this does not apply to periodic Bayesian flow, and some of the notations in original BFN~\citep{bfn} need to be adjusted accordingly. We maintain the notations of sender side's one-step accuracy $\alpha$ and added accuracy $\beta$, and alter the notation of receiver's accuracy parameter as $c$, which is needed to be simulated by cascade of Bayesian updates. We emphasize that the receiver's accumulated accuracy $c$ is no longer a function of $t$ (differently from the Gaussian case), and it becomes a distribution conditioned on received accuracies $\senderacc$ from the sender. Therefore, we represent the Bayesian flow distribution of von Mises distribution as $p_F(\btheta|\x;\alpha_1,\alpha_2,\dots,\alpha_i)$. And the original simulation-free training with Bayesian flow distribution is no longer applicable in this scenario.
% Different from previous BFNs where the accumulated accuracy $\rho$ is not explicitly modeled, the accumulated accuracy parameter $c$ (visualized in \cref{fig:vmbf_vis}) needs to be explicitly modeled by feeding it to the network to avoid information loss.
% the randomaccuracy parameter $c$ (visualized in \cref{fig:vmbf_vis}) implies that there exists information in $c$ from the sender just like $m$, meaning that $c$ also should be fed into the network to avoid information loss. 
% We ablate this consideration in  \cref{sec:exp_ablation}. 

\textbf{Fast Sampling from Equivalent Bayesian Flow Distribution} Based on the above reformulations, the Bayesian flow distribution of von Mises distribution is reframed as: 
\begin{equation}\label{eq:flow_frac}
p_F(\btheta_i|\x;\alpha_1,\alpha_2,\dots,\alpha_i)=\E_{\update(\parsnt{1} \mid \parsnt{0}, \x ; \alphat{1})}\dots\E_{\update(\parsn_{i-1} \mid \parsnt{i-2}, \x; \alphat{i-1})} \update(\parsnt{i} | \parsnt{i-1},\x;\alphat{i} )
\end{equation}
Naively sampling from \cref{eq:flow_frac} requires slow auto-regressive iterated simulation, making training unaffordable. Noticing the mathematical properties of \cref{eq:h_m,eq:h_c}, we  transform \cref{eq:flow_frac} to the equivalent form:
\begin{equation}\label{eq:cirflow_equiv}
p_F(\vec{m}_i|\x;\alpha_1,\alpha_2,\dots,\alpha_i)=\E_{vM(\y_1|\x,\alpha_1)\dots vM(\y_i|\x,\alpha_i)} \delta(\vec{m}_i-\text{atan2}(\sum_{j=1}^i \alpha_j \cos \y_j,\sum_{j=1}^i \alpha_j \sin \y_j))
\end{equation}
\begin{equation}\label{eq:cirflow_equiv2}
p_F(\vec{c}_i|\x;\alpha_1,\alpha_2,\dots,\alpha_i)=\E_{vM(\y_1|\x,\alpha_1)\dots vM(\y_i|\x,\alpha_i)}  \delta(\vec{c}_i-||[\sum_{j=1}^i \alpha_j \cos \y_j,\sum_{j=1}^i \alpha_j \sin \y_j]^T||_2)
\end{equation}
which bypasses the computation of intermediate variables and allows pure tensor operations, with negligible computational overhead.
\begin{restatable}{proposition}{cirflowequiv}
The probability density function of Bayesian flow distribution defined by \cref{eq:cirflow_equiv,eq:cirflow_equiv2} is equivalent to the original definition in \cref{eq:flow_frac}. 
\end{restatable}
\textbf{Numerical Determination of Linear Entropy Sender Accuracy Schedule} ~Original BFN designs the accuracy schedule $\beta(t)$ to make the entropy of input distribution linearly decrease. As for crystal generation task, to ensure information coherence between modalities, we choose a sender accuracy schedule $\senderacc$ that makes the receiver's belief entropy $H(t_i)=H(p_I(\cdot|\vtheta_i))=H(p_I(\cdot|\vc_i))$ linearly decrease \emph{w.r.t.} time $t_i$, given the initial and final accuracy parameter $c(0)$ and $c(1)$. Due to the intractability of \cref{eq:vm_entropy}, we first use numerical binary search in $[0,c(1)]$ to determine the receiver's $c(t_i)$ for $i=1,\dots, n$ by solving the equation $H(c(t_i))=(1-t_i)H(c(0))+tH(c(1))$. Next, with $c(t_i)$, we conduct numerical binary search for each $\alpha_i$ in $[0,c(1)]$ by solving the equations $\E_{y\sim vM(x,\alpha_i)}[\sqrt{\alpha_i^2+c_{i-1}^2+2\alpha_i c_{i-1}\cos(y-m_{i-1})}]=c(t_i)$ from $i=1$ to $i=n$ for arbitrarily selected $x\in[-\pi,\pi)$.

After tackling all those issues, we have now arrived at a new BFN architecture for effectively modeling crystals. Such BFN can also be adapted to other type of data located in hyper-torus $\mathbb{T}^{D}$.

\subsection{Equivariant Bayesian Flow for Crystal}
With the above Bayesian flow designed for generative modeling of fractional coordinate $\vF$, we are able to build equivariant Bayesian flow for each modality of crystal. In this section, we first give an overview of the general training and sampling algorithm of \modelname (visualized in \cref{fig:framework}). Then, we describe the details of the Bayesian flow of every modality. The training and sampling algorithm can be found in \cref{alg:train} and \cref{alg:sampling}.

\textbf{Overview} Operating in the parameter space $\bthetaM=\{\bthetaA,\bthetaL,\bthetaF\}$, \modelname generates high-fidelity crystals through a joint BFN sampling process on the parameter of  atom type $\bthetaA$, lattice parameter $\vec{\theta}^L=\{\bmuL,\brhoL\}$, and the parameter of fractional coordinate matrix $\bthetaF=\{\bmF,\bcF\}$. We index the $n$-steps of the generation process in a discrete manner $i$, and denote the corresponding continuous notation $t_i=i/n$ from prior parameter $\thetaM_0$ to a considerably low variance parameter $\thetaM_n$ (\emph{i.e.} large $\vrho^L,\bmF$, and centered $\bthetaA$).

At training time, \modelname samples time $i\sim U\{1,n\}$ and $\bthetaM_{i-1}$ from the Bayesian flow distribution of each modality, serving as the input to the network. The network $\net$ outputs $\net(\parsnt{i-1}^\mathcal{M},t_{i-1})=\net(\parsnt{i-1}^A,\parsnt{i-1}^F,\parsnt{i-1}^L,t_{i-1})$ and conducts gradient descents on loss function \cref{eq:loss_n} for each modality. After proper training, the sender distribution $p_S$ can be approximated by the receiver distribution $p_R$. 

At inference time, from predefined $\thetaM_0$, we conduct transitions from $\thetaM_{i-1}$ to $\thetaM_{i}$ by: $(1)$ sampling $\y_i\sim p_R(\bold{y}|\thetaM_{i-1};t_i,\alpha_i)$ according to network prediction $\predM{i-1}$; and $(2)$ performing Bayesian update $h(\thetaM_{i-1},\y^\calM_{i-1},\alpha_i)$ for each dimension. 

% Alternatively, we complete this transition using the flow-back technique by sampling 
% $\thetaM_{i}$ from Bayesian flow distribution $\flow(\btheta^M_{i}|\predM{i-1};t_{i-1})$. 

% The training objective of $\net$ is to minimize the KL divergence between sender distribution and receiver distribution for every modality as defined in \cref{eq:loss_n} which is equivalent to optimizing the negative variational lower bound $\calL^{VLB}$ as discussed in \cref{sec:preliminaries}. 

%In the following part, we will present the Bayesian flow of each modality in detail.

\textbf{Bayesian Flow of Fractional Coordinate $\vF$}~The distribution of the prior parameter $\bthetaF_0$ is defined as:
\begin{equation}\label{eq:prior_frac}
    p(\bthetaF_0) \defeq \{vM(\vm_0^F|\vec{0}_{3\times N},\vec{0}_{3\times N}),\delta(\vc_0^F-\vec{0}_{3\times N})\} = \{U(\vec{0},\vec{1}),\delta(\vc_0^F-\vec{0}_{3\times N})\}
\end{equation}
Note that this prior distribution of $\vm_0^F$ is uniform over $[\vec{0},\vec{1})$, ensuring the periodic translation invariance property in \cref{De:pi}. The training objective is minimizing the KL divergence between sender and receiver distribution (deduction can be found in \cref{appd:cir_loss}): 
%\oyyw{replace $\vF$ with $\x$?} \hanlin{notations follow Preliminary?}
\begin{align}\label{loss_frac}
\calL_F = n \E_{i \sim \ui{n}, \flow(\parsn{}^F \mid \vF ; \senderacc)} \alpha_i\frac{I_1(\alpha_i)}{I_0(\alpha_i)}(1-\cos(\vF-\predF{i-1}))
\end{align}
where $I_0(x)$ and $I_1(x)$ are the zeroth and the first order of modified Bessel functions. The transition from $\bthetaF_{i-1}$ to $\bthetaF_{i}$ is the Bayesian update distribution based on network prediction:
\begin{equation}\label{eq:transi_frac}
    p(\btheta^F_{i}|\parsnt{i-1}^\calM)=\mathbb{E}_{vM(\bold{y}|\predF{i-1},\alpha_i)}\delta(\btheta^F_{i}-h(\btheta^F_{i-1},\bold{y},\alpha_i))
\end{equation}
\begin{restatable}{proposition}{fracinv}
With $\net_{F}$ as a periodic translation equivariant function namely $\net_F(\parsnt{}^A,w(\parsnt{}^F+\vt),\parsnt{}^L,t)=w(\net_F(\parsnt{}^A,\parsnt{}^F,\parsnt{}^L,t)+\vt), \forall\vt\in\R^3$, the marginal distribution of $p(\vF_n)$ defined by \cref{eq:prior_frac,eq:transi_frac} is periodic translation invariant. 
\end{restatable}
\textbf{Bayesian Flow of Lattice Parameter \texorpdfstring{$\boldsymbol{L}$}{}}   
Noting the lattice parameter $\bm{L}$ located in Euclidean space, we set prior as the parameter of a isotropic multivariate normal distribution $\btheta^L_0\defeq\{\vmu_0^L,\vrho_0^L\}=\{\bm{0}_{3\times3},\bm{1}_{3\times3}\}$
% \begin{equation}\label{eq:lattice_prior}
% \btheta^L_0\defeq\{\vmu_0^L,\vrho_0^L\}=\{\bm{0}_{3\times3},\bm{1}_{3\times3}\}
% \end{equation}
such that the prior distribution of the Markov process on $\vmu^L$ is the Dirac distribution $\delta(\vec{\mu_0}-\vec{0})$ and $\delta(\vec{\rho_0}-\vec{1})$, 
% \begin{equation}
%     p_I^L(\boldsymbol{L}|\btheta_0^L)=\mathcal{N}(\bm{L}|\bm{0},\bm{I})
% \end{equation}
which ensures O(3)-invariance of prior distribution of $\vL$. By Eq. 77 from \citet{bfn}, the Bayesian flow distribution of the lattice parameter $\bm{L}$ is: 
\begin{align}% =p_U(\bmuL|\btheta_0^L,\bm{L},\beta(t))
p_F^L(\bmuL|\bm{L};t) &=\mathcal{N}(\bmuL|\gamma(t)\bm{L},\gamma(t)(1-\gamma(t))\bm{I}) 
\end{align}
where $\gamma(t) = 1 - \sigma_1^{2t}$ and $\sigma_1$ is the predefined hyper-parameter controlling the variance of input distribution at $t=1$ under linear entropy accuracy schedule. The variance parameter $\vrho$ does not need to be modeled and fed to the network, since it is deterministic given the accuracy schedule. After sampling $\bmuL_i$ from $p_F^L$, the training objective is defined as minimizing KL divergence between sender and receiver distribution (based on Eq. 96 in \citet{bfn}):
\begin{align}
\mathcal{L}_{L} = \frac{n}{2}\left(1-\sigma_1^{2/n}\right)\E_{i \sim \ui{n}}\E_{\flow(\bmuL_{i-1} |\vL ; t_{i-1})}  \frac{\left\|\vL -\predL{i-1}\right\|^2}{\sigma_1^{2i/n}},\label{eq:lattice_loss}
\end{align}
where the prediction term $\predL{i-1}$ is the lattice parameter part of network output. After training, the generation process is defined as the Bayesian update distribution given network prediction:
\begin{equation}\label{eq:lattice_sampling}
    p(\bmuL_{i}|\parsnt{i-1}^\calM)=\update^L(\bmuL_{i}|\predL{i-1},\bmuL_{i-1};t_{i-1})
\end{equation}
    

% The final prediction of the lattice parameter is given by $\bmuL_n = \predL{n-1}$.
% \begin{equation}\label{eq:final_lattice}
%     \bmuL_n = \predL{n-1}
% \end{equation}

\begin{restatable}{proposition}{latticeinv}\label{prop:latticeinv}
With $\net_{L}$ as  O(3)-equivariant function namely $\net_L(\parsnt{}^A,\parsnt{}^F,\vQ\parsnt{}^L,t)=\vQ\net_L(\parsnt{}^A,\parsnt{}^F,\parsnt{}^L,t),\forall\vQ^T\vQ=\vI$, the marginal distribution of $p(\bmuL_n)$ defined by \cref{eq:lattice_sampling} is O(3)-invariant. 
\end{restatable}


\textbf{Bayesian Flow of Atom Types \texorpdfstring{$\boldsymbol{A}$}{}} 
Given that atom types are discrete random variables located in a simplex $\calS^K$, the prior parameter of $\boldsymbol{A}$ is the discrete uniform distribution over the vocabulary $\parsnt{0}^A \defeq \frac{1}{K}\vec{1}_{1\times N}$. 
% \begin{align}\label{eq:disc_input_prior}
% \parsnt{0}^A \defeq \frac{1}{K}\vec{1}_{1\times N}
% \end{align}
% \begin{align}
%     (\oh{j}{K})_k \defeq \delta_{j k}, \text{where }\oh{j}{K}\in \R^{K},\oh{\vA}{KD} \defeq \left(\oh{a_1}{K},\dots,\oh{a_N}{K}\right) \in \R^{K\times N}
% \end{align}
With the notation of the projection from the class index $j$ to the length $K$ one-hot vector $ (\oh{j}{K})_k \defeq \delta_{j k}, \text{where }\oh{j}{K}\in \R^{K},\oh{\vA}{KD} \defeq \left(\oh{a_1}{K},\dots,\oh{a_N}{K}\right) \in \R^{K\times N}$, the Bayesian flow distribution of atom types $\vA$ is derived in \citet{bfn}:
\begin{align}
\flow^{A}(\parsn^A \mid \vA; t) &= \E_{\N{\y \mid \beta^A(t)\left(K \oh{\vA}{K\times N} - \vec{1}_{K\times N}\right)}{\beta^A(t) K \vec{I}_{K\times N \times N}}} \delta\left(\parsn^A - \frac{e^{\y}\parsnt{0}^A}{\sum_{k=1}^K e^{\y_k}(\parsnt{0})_{k}^A}\right).
\end{align}
where $\beta^A(t)$ is the predefined accuracy schedule for atom types. Sampling $\btheta_i^A$ from $p_F^A$ as the training signal, the training objective is the $n$-step discrete-time loss for discrete variable \citep{bfn}: 
% \oyyw{can we simplify the next equation? Such as remove $K \times N, K \times N \times N$}
% \begin{align}
% &\calL_A = n\E_{i \sim U\{1,n\},\flow^A(\parsn^A \mid \vA ; t_{i-1}),\N{\y \mid \alphat{i}\left(K \oh{\vA}{KD} - \vec{1}_{K\times N}\right)}{\alphat{i} K \vec{I}_{K\times N \times N}}} \ln \N{\y \mid \alphat{i}\left(K \oh{\vA}{K\times N} - \vec{1}_{K\times N}\right)}{\alphat{i} K \vec{I}_{K\times N \times N}}\nonumber\\
% &\qquad\qquad\qquad-\sum_{d=1}^N \ln \left(\sum_{k=1}^K \out^{(d)}(k \mid \parsn^A; t_{i-1}) \N{\ydd{d} \mid \alphat{i}\left(K\oh{k}{K}- \vec{1}_{K\times N}\right)}{\alphat{i} K \vec{I}_{K\times N \times N}}\right)\label{discdisc_t_loss_exp}
% \end{align}
\begin{align}
&\calL_A = n\E_{i \sim U\{1,n\},\flow^A(\parsn^A \mid \vA ; t_{i-1}),\N{\y \mid \alphat{i}\left(K \oh{\vA}{KD} - \vec{1}\right)}{\alphat{i} K \vec{I}}} \ln \N{\y \mid \alphat{i}\left(K \oh{\vA}{K\times N} - \vec{1}\right)}{\alphat{i} K \vec{I}}\nonumber\\
&\qquad\qquad\qquad-\sum_{d=1}^N \ln \left(\sum_{k=1}^K \out^{(d)}(k \mid \parsn^A; t_{i-1}) \N{\ydd{d} \mid \alphat{i}\left(K\oh{k}{K}- \vec{1}\right)}{\alphat{i} K \vec{I}}\right)\label{discdisc_t_loss_exp}
\end{align}
where $\vec{I}\in \R^{K\times N \times N}$ and $\vec{1}\in\R^{K\times D}$. When sampling, the transition from $\bthetaA_{i-1}$ to $\bthetaA_{i}$ is derived as:
\begin{equation}
    p(\btheta^A_{i}|\parsnt{i-1}^\calM)=\update^A(\btheta^A_{i}|\btheta^A_{i-1},\predA{i-1};t_{i-1})
\end{equation}

The detailed training and sampling algorithm could be found in \cref{alg:train} and \cref{alg:sampling}.







\section{Image Compression with DDCM}\label{section:compression}
\begin{figure*}[t]
    \centering
    \includegraphics[width=1\textwidth]{figures/Kodak24_512_extreme.pdf}
    \caption{\textbf{Qualitative image compression results.} The presented images are taken from the Kodak24 ($512\times 512$) dataset.
    Our codec produces highly realistic outputs, while maintaining better fidelity to the original images compared to previous methods.
    }
    \label{fig:compression_examples}
\end{figure*}
\paragraph{Method.}
Since sampling with DDCM yields compact bit-stream representations, a natural endeavor is to harness DDCM for compressing real images.
In particular, to compress an image $\rvx_{0}$, we leverage the predicted $\hat{\rvx}_{0|i}$ (\Cref{eq:x0eq}) at each timestep $i$ and compute the residual error from the target image, $\rvx_0-\hat{\rvx}_{0|i}$.
Then, we guide the sampling process towards $\rvx_0$ by selecting the codebook entry that maximizes the inner product with this residual,
\begin{align}\label{eq:compression_rule} 
    k_i = \argmax_{k\in\{1,\hdots,K\}} \langle \gC_i(k), \rvx_0-\hat{\rvx}_{0|i}\rangle,
\end{align}
where the size of the first codebook $\gC_{T+1}$ is $K=1$. 
This process is depicted as the compression branch in \Cref{fig:overview}, where the resulting set of chosen indices $\{k_i\}_{i=2}^{T+1}$ is the compressed bit-stream representation of the given image.
Section~\ref{sec:compressed_conditional_generation} sheds more light on this choice of the noise selection from the perspective of score-based generative models~\citep{song2020score}. 
As in \Cref{sec:method}, decompression follows standard DDCM sampling \cref{eq:DDCM_sampling}, re-selecting the stored indices instead of picking them randomly.
When using latent space DDMs (e.g., SD), we 
first encode $\rvx_0$ into the latent domain, perform all subsequent operations in this domain, and decode the result with the decoder.

The bit rate of this approach is determined by the size of the codebooks $K$, and the number of sampling timesteps $T$. Specifically, the bit-stream length is given by \smash{$(T-1)\log_{2}(K)$}. Therefore, the bit rate can be reduced by simply decreasing the number of codebooks, or by using a smaller number of timesteps at generation, e.g., by skipping every other step, or by using specific timestep intervals (see \Cref{app:range_t}).
In the approach described so far, the length of the bitstream increases logarithmically with $K$, making it computationally demanding to increase the bit rate.
For instance, even for $K=8192$, $T=1000$ and $768\times 768$ images our BPP is approximately 0.022.
Thus, to produce higher bit rates, we propose to \emph{refine} the noise selected at timestep $i$ by employing matching pursuit (MP)~\citep{mallat1993matching}.
Specifically, at each step $i$, we construct the chosen noise as a convex combination of $M$ elements from 
$\gC_i$, gathered in a greedy fashion to best correlate with the guiding residual $\rvx_{0}-\hat{\rvx}_{0|i}$ (as in~\Cref{eq:compression_rule}). 
The resulting convex combination involves $M-1$ quantized scalar coefficients, chosen from a finite set of $C$ values taken from $[0,1]$.
Therefore, the resulting length of the bit-stream is given by $\smash{(T-1)(\log_{2}(K)M+C(M-1))}$, such that $M=1$ is similar to our standard compression scheme, and the length of the bit-stream increases linearly with $M$ and $C$.
We apply this algorithm when the absolute bits number crosses $(T-1)\cdot \log_2(2^{13})$.
Further details are available in \Cref{app:matching_pursuit}.

\paragraph{Experiments.}
We evaluate our compression method on Kodak24~\cite{franzen1999kodak}, DIV2K~\citep{agustsson2017ntire}, ImageNet 1K $256\times 256$~\citep{deng2009imagenet,pan2020dgp}, and CLIC2020~\citep{CLIC2020}.
For all datasets but ImageNet, we center crop and resize all images to $512\times512$.
We compare to numerous competing methods, both non-neural and neural, and both zero-shot, fine-tuning based, and training based approaches.
For the ImageNet dataset, we use the unconditional pixel space ImageNet $256\times 256$ model of \citet{dhariwal2021diffusion}, and compare our results to BPG~\citep{bpg}, HiFiC~\citep{mentzer2020high}, IPIC~\citep{xu2024idempotence}, and two  PSC~\citep{elata2024zero} configurations, distortion-oriented (PSC-D) and perception-oriented (PSC-P).
For all other datasets, we use SD 2.1 $512\times512$~\citep{rombach2022high} and compare to BPG, PSC-D, PSC-P, ILLM~\citep{muckley2023improving}, PerCo (SD)~\citep{korber2024perco, careil2023towards}, and twoCRDR~\citep{iwai2024controlling} configurations, distortion-oriented (CRDR-R) and perception-oriented (CRDR-R).
PSC shares the same pre-trained model as ours, while PerCo (SD) requires additional fine-tuning.
For our method, we apply SD 2.1 unconditionally, as we saw no improvement by adding prompts (see further details in \Cref{app:text_effect}).
We assess our method for several options of $T$, $K$, $M$, and $C$ to control the bit rate.
See further details in \Cref{app:compression_details}.
We evaluate distortion with PSNR and LPIPS~\citep{zhang2018perceptual} and perceptual quality with FID~\citep{bińkowski2018demystifying}.
For ImageNet, FID is computed against the entire 50k $256\times 256$ validation set.
For the smaller datasets we follow \citet{mentzer2020high} and compute the FID over extracted image patches. 
Specifically, for DIV2K and CLIC2020 we extract $128\times 128$ sized patches, and for Kodak we use $64\times64$.


As shown in \Cref{fig:compression_graphs}, our compression scheme dominates previous methods on the rate-perception-distortion plane~\citep{pmlr-v97-blau19a} for lower bit rates, surpassing both the perceptual quality (FID) and distortion (PSNR and LPIPS) of previous methods.
For instance, our FID scores are lower than those of all other methods at around 0.1 BPP, while, for the same BPP, our distortion performance is better than the perceptually-oriented methods (e.g., PerCo, PSC-P, and IPIC).
However, our method under-performs at the highest bit rates, especially when using SD. This is expected due to the performance ceiling entailed by the pre-trained encoder-decoder of SD~\citep{korber2024perco, elata2024zero}.
The qualitative comparisons in \Cref{fig:compression_examples} further demonstrate our superior perceptual quality, where, even for extreme bit rates, our method maintains similarity to the original images in fine details.
See \Cref{app:compression_details} for more details and results.


\begin{figure*}[t]
    \centering
    \includegraphics[width=\linewidth]{figures/compression_graphs.pdf}
    \caption{\textbf{Compression quantitative evaluation.}
    We compare the perceptual quality (FID) and distortion (PSNR, LPIPS) achieved at different BPPs. 
    The image sizes of each dataset is denoted next to its name.
    Our method produces the best perceptual quality at most BPPs.
    Importantly, this is while we also attain lower distortion compared to perceptually-oriented methods (e.g., PSC-P and PerCo (SD)). 
    For the three rightmost datasets, note that our approach, PSC-P, and PerCo (SD) use the latent space Stable Diffusion 2.1 model, so its VAE imposes a distortion bound.
    Thus, we report the distortion attained by simply passing the images through this VAE (dashed line).
    }
    \label{fig:compression_graphs}
\end{figure*}


\section{Compressed Conditional Generation}\label{sec:compressed_conditional_generation}

We showed that DDCM can be used as an image codec by following a simple index selection rule, guiding the generated image towards a target one.
Here, we generalize this scheme to any \emph{conditional} generation task, considering the more broad framework of \emph{compressing} conditionally generated samples.
This is a particularly valuable framework in scenarios where the input condition $\rvy$ is bit rate intensive, e.g., where $\rvy$ is a degraded image and the goal is to produce a \emph{compressed} high-quality reconstruction of it.
To the best of our knowledge, this task, which we name \emph{compressed} conditional generation, has only been thoroughly explored for text input conditions~\citep{bulla2023}.

A naive solution to this task could be to simply compress the outputs of any existing conditional generation scheme.
Here we propose a novel end-to-end solution that generates the outputs \emph{directly} in their compressed form.
Importantly, note that our approach in \Cref{sec:method} requires the condition $\rvy$ for decompressing the bit-stream.
While this is not a stringent requirement when the condition is lightweight (e.g., a text prompt), this approach is less suitable when storage of the condition signal itself requires a long bit-stream. 
The solutions we propose in this section enable decoding the bit-stream without access to $\rvy$.


\paragraph{Compressed Conditional Generation with DDCM.}We propose generating a conditional sample by choosing the indices $k_{i}$ in \Cref{eq:DDCM_sampling} via
\begin{align}
k_{i}=\argmin_{k\in\{1,\hdots,K\}}\gL(\rvy,\rvx_{i},\gC_{i},k),\label{eq:k_choose_conditional}
\end{align}
instead of picking them randomly.
Here, $\gL(\rvy,\rvx_{i},\gC_{i},k)$ can be any loss function that attains a lower value when $\gC_{i}(k)$ 
directs the generative process towards an image that matches $\rvy$.
For example, for the loss
\begin{align}
\gL_{\text{P}}(\rvy,\rvx_{i},\gC_{i},k)=\norm{\gC_{i}(k)-\sigma_{i}\nabla_{\rvx_{i}}\log{p_{i}(\rvy|\rvx_{i})}}^{2}\label{eq:l_score}
\end{align}
we obtain the following result (see proof in~\cref{appendix:cond_compression}):
\begin{proposition}\label{prob:ode_convergence}
Suppose that image samples are generated via process~\cref{eq:DDCM_sampling}, and the indices $k_{i}$ are chosen according to~\Cref{eq:k_choose_conditional} with $\gL=\gL_{\textnormal{P}}$.
Then, when $K\rightarrow\infty$, such a generative process becomes a discretization of a probability flow ODE over the posterior distribution $p_{0}(\rvx_{0}|\rvy)$.
\end{proposition}
In other words,~\Cref{prob:ode_convergence} implies that for the loss $\gL_{\text{P}}$, increasing $K$ leads to more accurate sampling from the posterior $p_{0}(\rvx_{0}|\rvy)$, though this results in longer bit-streams.
Thus, as long as we have access to $\nabla_{\rvx_{i}}\log{p_{i}(\rvy|\rvx_{i})}$ (or an  approximation of it) $\gL_{\text{P}}$ may serve as a sensible option for solving a compressed conditional generation task with DDCM.
Interestingly, we show in~\Cref{appendix:compression_private_case} that our compression scheme from~\Cref{section:compression} is a special case of the proposed compressed conditional generation method, with $\rvy=\rvx_{0}$ and $\gL=\gL_{\text{P}}$.

\subsection{Compressed Posterior Sampling for Image Restoration}\label{sec:zero-shot-restoration}


\begin{figure*}[t]
    \centering
    \includegraphics[width=1.0\textwidth]{figures/posterior_sampling.pdf}
    \includegraphics{figures/linear_restoration_main_text_v2.pdf}
    \caption{\textbf{Comparison of zero-shot posterior sampling image restoration methods.} Our approach achieves better perceptual quality compared to previous methods, while maintaining competitive PSNR and automatically producing compressed bit-stream representations for each restored image.}
    \label{fig:zero-shot-posterior-sampling-qualitative}
\end{figure*}

Our compressed conditional sampling approach can be utilized for solving inverse problems via posterior sampling.
Specifically, we consider inverse problems of the form $\rvy=\mA\rvx_{0}$, where $\mA$ is some linear operator.
We restrict our attention to \emph{unconditional} diffusion models and solve the problems in a ``zero-shot'' manner (similarly to \citet{kawar2022denoising,chung2023diffusion,wang2023zeroshot}). 
To generate conditional samples, we propose using the loss
\begin{align}
\gL(\rvy,\rvx_{i},\gC_{i},k)=\norm{\rvy-\mA(\vmu_{i}(\rvx_{i})+\sigma_{i}\gC_{i}(k))}^{2}.\label{eq:l_posterior_sampling}
\end{align}
Note that~\Cref{eq:l_posterior_sampling} attains a lower value when $\sigma_{i}\mA\gC_{i}(k)$ points in the direction that perturbs $\mA\vmu_{i}(\rvx_{i})$ towards $\rvy$.
Thus, our conditional generative process aims to produce a reconstruction $\hat{\rvx}$ that satisfies $\mA\hat{\rvx}\approx\rvy$, implying that we approximate posterior sampling~\citep{pmlr-v202-ohayon23a}.
Notably, when assuming that $p_{i}(\rvy|\rvx_{i})$ is a multivariate normal distribution centered around $\mA\rvx_{i}$ (as in~\citep{jalal2021posterior}), the chosen codebook noise $\gC_{i}(k_{i})$ approximates the gradient $\nabla_{\rvx_{i}}\log{p_{i}(\rvy|\rvx_{i})}$ and~\Cref{eq:l_posterior_sampling} becomes a proxy of~\Cref{eq:l_score}.

Following~\citep{chung2023diffusion,wang2023zeroshot}, we implement our method using the unconditional ImageNet $256\times 256$ DDM trained by \citet{dhariwal2021diffusion}.
We fix $K=4096$ for all codebooks, resulting in a compressed bit-stream of approximately $0.183$ BPP for each generated image.
We compare our method with DPS~\citep{chung2023diffusion} and DDNM~\citep{wang2023zeroshot} on two noiseless tasks: colorization and $4\times $ super-resolution (using the bicubic kernel). We evaluate these methods using their official implementations and the same DDM.
We additionally compress the outputs of DPS and DDNM to assess whether such a naive approach would yield better results.
To do so, we adopt our proposed compression scheme (from \Cref{section:compression}), employing the same unconditional ImageNet DDM and using $K=4096$ noises per codebook.


Qualitative and quantitative results are reported in \Cref{fig:zero-shot-posterior-sampling-qualitative}.
As expected, due to the rate-perception-distortion tradeoff~\citep{pmlr-v97-blau19a}, we observe that compressing the outputs of DPS and DDNM harms either their perceptual quality (FID), or their distortion (PSNR), or both.
This is while our method achieves superior perceptual quality compared to both DPS and DDNM, including their compressed versions.
While our method achieves slightly worse PSNR, this is expected due to the perception-distortion trade-off~\citep{Blau_2018_CVPR}.
See \Cref{appendix:zero-shot} for more details.





\subsection{Compressed Real-World Face Image Restoration}\label{sec:bfr}
\begin{figure*}[t]
    \centering
    \includegraphics[width=1\textwidth]{figures/blind_face_restoration/real-world-restoration-wider.pdf}
\includegraphics[width=1\textwidth]{figures/blind_face_restoration/blind_face_restoration_main_text_wider.pdf}
    \caption{\textbf{Comparing real-world face image restoration methods on the WIDER-Test dataset}. We successfully optimize the NR-IQA measures and produce appealing output perceptual quality with less artifacts compared to previous methods.}
    \label{fig:real-world-wider-visual}
\end{figure*}
Real-world face image restoration is the practical task of restoring any degraded face image, without any knowledge of the corruption process it has gone through~\citep{wang2021gfpgan,vqfr,wang2022restoreformer,zhou2022codeformer,wang2023restoreformer++,2023diffbir,difface,bfrfussion,pmrf}.
We propose a novel method capable of optimizing any no-reference image quality assessment (NR-IQA) measure at test time (e.g., NIQE~\citep{niqe}), without relying on gradients.

Specifically, at each timestep $i$, we start by picking two indices -- one that promotes high perceptual quality, $k_{i,P}$, and another that promotes low distortion, $k_{i,D}$.
Then, we choose between $k_{i,P}$ and $k_{i,D}$ the index that better optimizes a desired balance of the perception-distortion tradeoff~\citep{Blau_2018_CVPR}.
Formally, letting $\vr(\rvy)\approx\mathbb{E}[\rvx_{0}|\rvy]$ denote the approximate Minimum Mean-Squared-Error (MMSE) estimator of this task, we pick $k_{i,D}$ via
\begin{align}
    k_{i,D}=\argmax_{k\in\{1,\hdots,K\}}\langle\gC_{i}(k),\vr(\rvy)-\hat{\rvx}_{0|i}\rangle.\label{eq:first_index_blind_face_restoration}
\end{align}
Note that this index selection rule is similar to that of our standard compression, replacing $\smash{\rvx_{0}}$ in \Cref{eq:compression_rule} with $\smash{\vr(\rvy)}$.
This choice of indices in DDCM would lead to a reconstructed estimate of the MMSE solution $\vy(\rvy)$, yielding blurry results with low distortion~\citep{Blau_2018_CVPR}.
In contrast, \emph{randomly} picking a sequence of indices in DDCM would produce a high quality sample from the data distribution $p_{0}$.
Therefore, we randomly choose $\smash{k_{i,P}\sim \text{Unif}(\{1,\hdots,K\})}$.
Then, we use the DDM and compute $\smash{\hat{\rvx}_{0|i-1}}$ for each index $\smash{k\in\{k_{i,D},k_{i,P}\}}$ separately, denoting each result accordingly by $\hat{\rvx}_{0|i-1}^{(k)}$.
The final index is picked to optimize the perception-distortion tradeoff via
\begin{align}
    k_{i}\!=\!\argmin_{k\in\{k_{i,D},k_{i,P}\}}\!\text{MSE}\!\left(\!\vr(\rvy),\hat{\rvx}_{0|i-1}^{(k)}\!\right)\!+\!\lambda Q\!\left(\!\hat{\rvx}_{0|i-1}^{(k)}\!\right)\!,\label{eq:optimize-perception-distortion-indices}
\end{align}
where $Q(\cdot)$ can be \emph{any} NR-IQA measure, even a non-differentiable one.
In \Cref{app:dmax-ot} we explain our choice to set $\vr(\rvy)$ as an MMSE estimator.


We assess our approach choosing $\vr(\rvy)$ as the FFHQ~\citep{stylegan} $512\times 512$ approximate MMSE model trained by \citet{difface}.
We set $\lambda=1$ and optimize three different $Q(\cdot)$ measures: NIQE, $\text{CLIP-IQA}^{+}$~\citep{Wang_Chan_Loy_2023}, and TOPIQ~\citep{chen2024topiq} adapted for face images by PyIQA~\citep{pyiqa}.
We utilize the FFHQ $512\times512$ DDM of \citet{difface} with $T=1000$ sampling steps and $K=4096$ for all codebooks.
We compare our approach against the state-of-the-art methods PMRF~\citep{pmrf}, DifFace~\citep{difface}, and BFRffusion~\citep{bfrfussion}, using the standard evaluation datasets CelebA-Test~\citep{karras2018progressive,wang2021gfpgan}, LFW-Test~\citep{lfw-original}, WebPhoto-Test~\citep{wang2021gfpgan}, and WIDER-Test~\citep{zhou2022codeformer}.
We use PSNR to measure the distortion of the outputs produced for the CelebA-Test dataset, and measure the ProxPSNR~\citep{pmrf,man2025proxiesdistortionconsistencyapplications} for the other datasets, which lack the clean original images.
Perceptual quality is measured by NIQE, $\text{CLIP-IQA}^{+}$, TOPIQ-FACE, and additionally $\text{FD}_{\text{DINOv2}}$~\citep{stein2023exposing} to assess our generalization performance to a common quality measure which we do not directly optimize.
Finally, as in \Cref{sec:zero-shot-restoration}, we \emph{compress} each evaluated method using our standard compression scheme, adopting the same FFHQ DDM with $K=4096$ and $T=1000$.

The results for the WIDER-Test dataset are reported in \Cref{fig:real-world-wider-visual} (see \Cref{app:dmax-ot} for the other datasets).
Our approach clearly optimizes each quality measure effectively and generalizes well according to the $\text{FD}_{\text{DINOv2}}$ scores.
This is also confirmed visually, where all of our solutions produce high-quality images with less artifacts compared to previous methods.
While our approach shows slightly worse distortion, this is once again expected due to the perception-distortion tradeoff~\citep{Blau_2018_CVPR}.

\section{Discussion of Assumptions}\label{sec:discussion}
In this paper, we have made several assumptions for the sake of clarity and simplicity. In this section, we discuss the rationale behind these assumptions, the extent to which these assumptions hold in practice, and the consequences for our protocol when these assumptions hold.

\subsection{Assumptions on the Demand}

There are two simplifying assumptions we make about the demand. First, we assume the demand at any time is relatively small compared to the channel capacities. Second, we take the demand to be constant over time. We elaborate upon both these points below.

\paragraph{Small demands} The assumption that demands are small relative to channel capacities is made precise in \eqref{eq:large_capacity_assumption}. This assumption simplifies two major aspects of our protocol. First, it largely removes congestion from consideration. In \eqref{eq:primal_problem}, there is no constraint ensuring that total flow in both directions stays below capacity--this is always met. Consequently, there is no Lagrange multiplier for congestion and no congestion pricing; only imbalance penalties apply. In contrast, protocols in \cite{sivaraman2020high, varma2021throughput, wang2024fence} include congestion fees due to explicit congestion constraints. Second, the bound \eqref{eq:large_capacity_assumption} ensures that as long as channels remain balanced, the network can always meet demand, no matter how the demand is routed. Since channels can rebalance when necessary, they never drop transactions. This allows prices and flows to adjust as per the equations in \eqref{eq:algorithm}, which makes it easier to prove the protocol's convergence guarantees. This also preserves the key property that a channel's price remains proportional to net money flow through it.

In practice, payment channel networks are used most often for micro-payments, for which on-chain transactions are prohibitively expensive; large transactions typically take place directly on the blockchain. For example, according to \cite{river2023lightning}, the average channel capacity is roughly $0.1$ BTC ($5,000$ BTC distributed over $50,000$ channels), while the average transaction amount is less than $0.0004$ BTC ($44.7k$ satoshis). Thus, the small demand assumption is not too unrealistic. Additionally, the occasional large transaction can be treated as a sequence of smaller transactions by breaking it into packets and executing each packet serially (as done by \cite{sivaraman2020high}).
Lastly, a good path discovery process that favors large capacity channels over small capacity ones can help ensure that the bound in \eqref{eq:large_capacity_assumption} holds.

\paragraph{Constant demands} 
In this work, we assume that any transacting pair of nodes have a steady transaction demand between them (see Section \ref{sec:transaction_requests}). Making this assumption is necessary to obtain the kind of guarantees that we have presented in this paper. Unless the demand is steady, it is unreasonable to expect that the flows converge to a steady value. Weaker assumptions on the demand lead to weaker guarantees. For example, with the more general setting of stochastic, but i.i.d. demand between any two nodes, \cite{varma2021throughput} shows that the channel queue lengths are bounded in expectation. If the demand can be arbitrary, then it is very hard to get any meaningful performance guarantees; \cite{wang2024fence} shows that even for a single bidirectional channel, the competitive ratio is infinite. Indeed, because a PCN is a decentralized system and decisions must be made based on local information alone, it is difficult for the network to find the optimal detailed balance flow at every time step with a time-varying demand.  With a steady demand, the network can discover the optimal flows in a reasonably short time, as our work shows.

We view the constant demand assumption as an approximation for a more general demand process that could be piece-wise constant, stochastic, or both (see simulations in Figure \ref{fig:five_nodes_variable_demand}).
We believe it should be possible to merge ideas from our work and \cite{varma2021throughput} to provide guarantees in a setting with random demands with arbitrary means. We leave this for future work. In addition, our work suggests that a reasonable method of handling stochastic demands is to queue the transaction requests \textit{at the source node} itself. This queuing action should be viewed in conjunction with flow-control. Indeed, a temporarily high unidirectional demand would raise prices for the sender, incentivizing the sender to stop sending the transactions. If the sender queues the transactions, they can send them later when prices drop. This form of queuing does not require any overhaul of the basic PCN infrastructure and is therefore simpler to implement than per-channel queues as suggested by \cite{sivaraman2020high} and \cite{varma2021throughput}.

\subsection{The Incentive of Channels}
The actions of the channels as prescribed by the DEBT control protocol can be summarized as follows. Channels adjust their prices in proportion to the net flow through them. They rebalance themselves whenever necessary and execute any transaction request that has been made of them. We discuss both these aspects below.

\paragraph{On Prices}
In this work, the exclusive role of channel prices is to ensure that the flows through each channel remains balanced. In practice, it would be important to include other components in a channel's price/fee as well: a congestion price  and an incentive price. The congestion price, as suggested by \cite{varma2021throughput}, would depend on the total flow of transactions through the channel, and would incentivize nodes to balance the load over different paths. The incentive price, which is commonly used in practice \cite{river2023lightning}, is necessary to provide channels with an incentive to serve as an intermediary for different channels. In practice, we expect both these components to be smaller than the imbalance price. Consequently, we expect the behavior of our protocol to be similar to our theoretical results even with these additional prices.

A key aspect of our protocol is that channel fees are allowed to be negative. Although the original Lightning network whitepaper \cite{poon2016bitcoin} suggests that negative channel prices may be a good solution to promote rebalancing, the idea of negative prices in not very popular in the literature. To our knowledge, the only prior work with this feature is \cite{varma2021throughput}. Indeed, in papers such as \cite{van2021merchant} and \cite{wang2024fence}, the price function is explicitly modified such that the channel price is never negative. The results of our paper show the benefits of negative prices. For one, in steady state, equal flows in both directions ensure that a channel doesn't loose any money (the other price components mentioned above ensure that the channel will only gain money). More importantly, negative prices are important to ensure that the protocol selectively stifles acyclic flows while allowing circulations to flow. Indeed, in the example of Section \ref{sec:flow_control_example}, the flows between nodes $A$ and $C$ are left on only because the large positive price over one channel is canceled by the corresponding negative price over the other channel, leading to a net zero price.

Lastly, observe that in the DEBT control protocol, the price charged by a channel does not depend on its capacity. This is a natural consequence of the price being the Lagrange multiplier for the net-zero flow constraint, which also does not depend on the channel capacity. In contrast, in many other works, the imbalance price is normalized by the channel capacity \cite{ren2018optimal, lin2020funds, wang2024fence}; this is shown to work well in practice. The rationale for such a price structure is explained well in \cite{wang2024fence}, where this fee is derived with the aim of always maintaining some balance (liquidity) at each end of every channel. This is a reasonable aim if a channel is to never rebalance itself; the experiments of the aforementioned papers are conducted in such a regime. In this work, however, we allow the channels to rebalance themselves a few times in order to settle on a detailed balance flow. This is because our focus is on the long-term steady state performance of the protocol. This difference in perspective also shows up in how the price depends on the channel imbalance. \cite{lin2020funds} and \cite{wang2024fence} advocate for strictly convex prices whereas this work and \cite{varma2021throughput} propose linear prices.

\paragraph{On Rebalancing} 
Recall that the DEBT control protocol ensures that the flows in the network converge to a detailed balance flow, which can be sustained perpetually without any rebalancing. However, during the transient phase (before convergence), channels may have to perform on-chain rebalancing a few times. Since rebalancing is an expensive operation, it is worthwhile discussing methods by which channels can reduce the extent of rebalancing. One option for the channels to reduce the extent of rebalancing is to increase their capacity; however, this comes at the cost of locking in more capital. Each channel can decide for itself the optimum amount of capital to lock in. Another option, which we discuss in Section \ref{sec:five_node}, is for channels to increase the rate $\gamma$ at which they adjust prices. 

Ultimately, whether or not it is beneficial for a channel to rebalance depends on the time-horizon under consideration. Our protocol is based on the assumption that the demand remains steady for a long period of time. If this is indeed the case, it would be worthwhile for a channel to rebalance itself as it can make up this cost through the incentive fees gained from the flow of transactions through it in steady state. If a channel chooses not to rebalance itself, however, there is a risk of being trapped in a deadlock, which is suboptimal for not only the nodes but also the channel.

\section{Conclusion}
This work presents DEBT control: a protocol for payment channel networks that uses source routing and flow control based on channel prices. The protocol is derived by posing a network utility maximization problem and analyzing its dual minimization. It is shown that under steady demands, the protocol guides the network to an optimal, sustainable point. Simulations show its robustness to demand variations. The work demonstrates that simple protocols with strong theoretical guarantees are possible for PCNs and we hope it inspires further theoretical research in this direction.

% \clearpage

\section*{Acknowledgements}
This research was partially supported by the Israel Science Foundation (ISF) under Grants 2318/22, 951/24 and 409/24, and by the Council for Higher Education – Planning and Budgeting Committee.
We thank Noam Elata and Matan Kleiner for assisting with our figures and experiments.

\section*{Impact Statement}
This paper presents work whose goal is to advance the field of Machine Learning.
There are many potential societal consequences of our work, none which we feel must be specifically highlighted here.
\clearpage

\bibliographystyle{icml2025}
\bibliography{citations}


\newpage
\appendix
\onecolumn
\section{Additional DDCM Evaluation}\label{app:more_random_gens}

Figure~\ref{fig:generation_other_metrics} provides additional quantitative comparisons between DDPM and DDCM, using different $K$ values.
Specifically, we compute the Kernel Inception Distance (KID)~\citep{bińkowski2018demystifying}, as well as the Fréchet Distance and Kernel Distance evaluated in the feature space of DINOv2~\citep{stein2023exposing,oquab2024dinov}.
These quantitative results remain consistent with the ones presented in \Cref{sec:method}, showing that DDCM with small $K$ values is comparable with DDPM.
Figures~\ref{fig:generation_samples_latent_app3} and \ref{fig:generation_samples_pixel_app3} show numerous outputs from both DDPM and DDCM with small values of $K$, demonstrating the sample quality and diversity produced by the latter for such $K$ values.
We use Torch Fidelity~\citep{obukhov2020torchfidelity} to compute the perceptual quality measures.
% generation stuff
\begin{figure}[H]
    \centering
    \includegraphics[width=0.5\linewidth]{figures/generation_other_metrics.pdf}
    \caption{\textbf{Comparing DDPM with DDCM for different codebook sizes $K$.} 
    As in \Cref{sec:method}, DDPM and DDCM with $K=64$ (sometimes even $K=16$) achieve similar generative performance, suggesting that the continuous representation space of DDPM (DDCM with $K =\infty$) is highly redundant.
    We use a class-conditional ImageNet model ($256 \times 256$) for pixel space, and the text-conditional SD 2.1 model ($768 \times 768$) for latent space. The $K$ axis is in log-scale.
    }
    \label{fig:generation_other_metrics}
\end{figure}


% \begin{figure*}[t]
%     \centering
%     \includegraphics[width=\linewidth]{figures/SD2.1_gens_moreims_part1.pdf}
%     \caption{\textbf{Qualitative Comparison of DDCMs vs DDPMs.}
%     We generate samples using DDCM with different $K$ values as well as using  DDPM sampling, on random prompts from MS-COCO. }
%     \label{fig:generation_samples_latent_app1}
% \end{figure*}

\begin{figure*}[t]
    \centering
    \includegraphics[width=\linewidth]{figures/SD2.1_16random_samples_0.pdf}\vspace{0.2cm}
    \includegraphics[width=\linewidth]{figures/SD2.1_16random_samples_1.pdf}\vspace{0.2cm}
    % \includegraphics[width=\linewidth]{figures/SD2.1_16random_samples_2.pdf}
    \caption{\textbf{Qualitative comparison of sample quality and diversity between DDCM and DDPM.}
    We generate multiple samples for each prompt, using the $768\times768$ SD 2.1 model.}
    \label{fig:generation_samples_latent_app3}
\end{figure*}

\begin{figure*}[t]
    \centering
    \includegraphics[width=1\textwidth]{figures/pixel_space_rand_gen.pdf}
    \caption{\textbf{Qualitative comparison of sample quality and diversity between DDCM and DDPM.}
    We generate multiple samples for each class, using the conditional $256\times 256$ ImageNet model.}
    \label{fig:generation_samples_pixel_app3}
\end{figure*}
\clearpage


\section{Image Compression Supplementary}\label{app:compression_details}

We compute all distortion and perceptual quality measures using \href{https://github.com/Lightning-AI/torchmetrics}{Torch Metrics} (which relies on Torch Fidelity~\citep{obukhov2020torchfidelity}).

\subsection{Experiment Configurations}\label{app:image-compression-experiments-configuration}
We specify here the different configurations used for the compression experiments in \Cref{section:compression}.
\begin{itemize}[parsep=2pt,itemsep=2pt]
    \item In our $256\times256$ experiments we use the \texttt{256x256\_diffusion\_uncond.pt} checkpoint from the \href{https://github.com/openai/guided-diffusion}{official GitHub repository}.
    In our $768\times768$ and $512\times512$ experiments we use Stable Diffusion 2.1 with the \href{https://huggingface.co/stabilityai/stable-diffusion-2-1}{\texttt{stabilityai/stable-diffusion-2-1}} and \href{https://huggingface.co/stabilityai/stable-diffusion-2-1-base}{\texttt{stabilityai/stable-diffusion-2-1-base}} official checkpoints from Hugging Face, respectively.
    The different $K$, $M$, $C$ and $T$ values for each of the $512\times512$ and the $768\times768$ experiments plotted in \Cref{fig:compression_graphs,fig:compression_graphs_768} are summarized in \Cref{tab:our_tmkc_configs}.
    

\begin{table}[h]
\centering
\caption{Image compression experiments configurations.}
\begin{tabular}{c|c|c|c|c|c}
\hline
Model & Image Resolution & $T$ & $K$ & $M$ & $C$ \\
\hline
\multirow{4}{*}{Pixel Space DDM} & \multirow{4}{*}{256$\times$256} & 1000 & 64, 128, 256, 4096 & 1 & - \\
& & 1000 & 2048 & 2, 3, 4, 5 & 3 \\
& & 500 & 128, 512 & 1 & - \\
& & 300 & 16, 32, 128, 512 & 1 & - \\
\hline
\multirow{6}{*}{Latent Space DDM} & \multirow{2}{*}{512$\times$512} & 1000 & 256, 1024, 8192 & 1 & - \\
& & 1000 & 2048 & 2, 3, 6 & 3 \\
\cline{2-6}
& \multirow{4}{*}{768$\times$768} & 1000 & 16, 32, 64, 256, 1024, 8192 & 1 & - \\
& & 1000 & 2048 & 2, 3, 6 & 3 \\
& & 500 & 16, 32, 64, 256, 1024, 8192 & 1 & - \\
& & Adapted 500 (\Cref{app:range_t}) & 16, 32, 64, 256, 1024, 8192 & 1 & - \\
\hline
\end{tabular}
\label{tab:our_tmkc_configs}
\end{table}

    % For th


% pixel
% 1000 - K=64 256 1024 4096
% 500 - K=128 512
% 300 - K=16 32 128 512 
% pursuit - p=2,4,3,5, | K= 2048 | C= 3
% latent - 
% 1000 - K=256 1024 8192
% pur - C=3, P=2,3,6
    
    % For the $768\times768$ image size experiments, we use $K\in\{16, 32, 64, 256, 1024,  8192\}$.
    \item For PSC-D and PSC-R we use the same pre-trained ImageNet $256\times256$ model as ours in the $256\times256$ experiments, and the same Stable Diffusion 2.1 model in the $512\times512$ experiments. We adopt the default hyper-parameters of the method as described by the authors~\citep{elata2024zero}, setting the number of measurements to $12\cdot2^i$ for $i=0,\ldots,8$.
    \item For IPIC we adopt the official implementation using the ELIC codec with five bit rates, combined with DPS sampling for decoding with $T=1000$ steps and $\zeta\in\{0.3, 0.6, 0.6, 1.2, 1.6\}$, as recommended by the authors.
    \item For BPG we considered quality factors $q\in\{51,50,48,46,42,40,38,36,34,32,30\}$. 
    \item For HiFiC we test the low, medium and high quality regimes, using the checkpoints available in the \href{https://github.com/Justin-Tan/high-fidelity-generative-compression}{official GitHub repository}.
    \item PerCo (SD) is tested using the three publicly available Stable Diffusion 2.1 fine-tuned checkpoints from their \href{https://github.com/Nikolai10/PerCo}{Official GitHub repository}, using the default hyper-parameters.
    \item For ILLM we use the MS-ILLM pre-trained models available in the \href{https://github.com/facebookresearch/NeuralCompression/tree/main/projects/illm}{official GitHub implementation}.
    For the $512\times512$ image size experiments we use \texttt{msillm\_quality\_X}, $\texttt{X}=2,3,4$. For the $768\times768$ image size experiments we use \texttt{msillm\_quality\_X}, $\texttt{X}=2,3$ and \texttt{msillm\_quality\_vloY}, $\texttt{Y}=1,2$.
    \item CRDR-D and CRDR-R are evaluated using quality factors of $\{0,1,2,3,4\}$, where CRDR-D uses $\beta=0$ and CRDR-R uses $\beta=3.84$, as recommended in the paper.
\end{itemize}


\subsection{Additional Evaluations}\label{app:compression_more_results}

In \Cref{fig:compression_examples_app,,fig:compression_examples_app_medium,,fig:pixel_space_comparison_low_bpp,,fig:pixel_space_comparison_mid_bpp} we provide additional qualitative comparisons on the Kodak24 ($512\times 512$) and ImageNet ($256\times 256$) datasets.
We additionally compare our method on images of size $768\times768$, and present the results in \Cref{fig:compression_graphs_768}.
Our method is implemented as before, while using a Stable Diffusion 2.1 model trained on the appropriate image size (see \Cref{app:image-compression-experiments-configuration}).

\subsection{Decreasing the Bit Rate via Timestep Sub-Sampling}\label{app:range_t}

As mentioned in \Cref{section:compression}, decreasing the bit rate of our compression scheme can be accomplished in two ways.
The first option is to reduce $K$, which sets the number of bits required to represent each communicated codebook index.
The second option relates to the number of generation timesteps, which sets the total number of communicated indices. Specifically, DDMs trained for $T=1000$ steps can still be used to generate samples with fewer steps, by skipping alternating timesteps and modifying the variance in \Cref{eq:ddpm}. 
Thus, we leverage such timestep sub-sampling in DDCM to shorten the compressed bit-stream.
We find that pixel space DDMs yields good results with this approach, while the latent space models struggle to produce satisfying perceptual quality.

Thus, for latent space models we propose a slightly different timestep sub-sampling scheme. 
Specifically, we keep $T=1000$ sampling steps at inference and set different $K$ values for different subsets of timesteps.
We choose $K=1$ for a subset of $L$ sampling steps, and $K>1$ for the rest $T-L$ steps.
Thus, our compression scheme only optimizes $T-L$ steps and necessitates transmitting only $T-L$ indices. The rest $L$ indices correspond to codebooks that contain only one vector, and thus do not affect the bit rate.

We use $T=1000$, set the same codebook size $K>1$ for every timestep $i\in\{899\ldots,400\}$, and use $K=1$ for all other steps.
We compare our proposed method against the aforementioned naive timestep skipping approach with $T=500$ sampling steps and the same $K>1$, which attains the same bit rate as our proposed alternative.

Quantitative results are shown in \Cref{fig:compression_graphs_768}, where our timestep adapted method is denoted by \emph{Ours Adapted}.
Our adapted approach achieves better perceptual quality compared to the naive one, at the expense of a slightly hindered PSNR.


\subsection{Increasing the Bit Rate via Matching Pursuit}\label{app:matching_pursuit}
In \Cref{section:compression} we briefly explain how to achieve higher bit rates with our method, by refining each selected noise via a matching pursuit inspired solution.
Formally, at each timestep $i$ we iteratively refine the selected noise by linearly combining it with $M-1$ other noises from the codebook.
We start by picking the first noise index $k_{i}$ according to \Cref{eq:compression_rule}.
Then, we set $k_{i}^{(1)}=k_{i}$, $\gamma_{i}^{(1)}=1$, and $\tilde{\vz}_{i}^{(1)}=\gC_{i}(k_{i})$, and pick the next indices and coefficients $(m=1,\hdots,M-1)$ via
\begin{align}
    k_i^{(m+1)}, \gamma_i^{(m+1)} = \argmax_{k\in\{1,\hdots,K\},\;\gamma \in \Gamma} \left\langle \gamma\tilde{\vz}_{i}^{(m)}+\left(1-\gamma\right)\gC_{i}(k) ,\,\rvx_0-\hat{\rvx}_{0|i}\right\rangle.
\end{align}
The noise vector $\tilde{\vz}_{i}^{(m+1)}$ is then updated via
\begin{align}
&\tilde{\vz}_{i}^{(m+1)}\leftarrow\gamma_{i}^{(m+1)}\tilde{\vz}_{i}^{(m)}+\left(1-\gamma_{i}^{(m+1)}\right)\gC_{i}\left(k_{i}^{(m+1)}\right),\\
&\tilde{\vz}_{i}^{(m+1)}\leftarrow \frac{\tilde{\vz}_{i}^{(m+1)}}{\text{std}\left(\tilde{\vz}_{i}^{(m+1)}\right)},
\end{align}
where $\text{std}(\vz)$ is the empirical standard deviation of the vector $\vz$.
We use the resulting vector $\tilde{\vz}_{i}^{(M)}$ as the noise in \Cref{eq:DDCM_sampling} to produce the next $\rvx_{i-1}$, and repeat the above process iteratively.
Note that setting $M=1$ is equivalent to our standard compression scheme.


In our experiments, the set of coefficients $\Gamma$ is a subset of $(0,1]$, containing $C=|\Gamma|$ values that are evenly spaced in this range.
We pick $C=3$ and assess $M\in\{2,3,6\}$ for the latent space model experiments, and $M\in\{2,3,4,5\}$ for the pixel space model experiments.




\subsection{Assessing the Effectiveness of Text Prompts in Compression using Text-to-Image Latent Space DDMs}\label{app:text_effect}

Stable Diffusion 2.1 is a text-to-image generative model, which both PerCo (SD) and PSC leverage for their compression approach.
Specifically, both of these methods start by generating a textual caption for every target image using BLIP-2~\citep{li2023blip}, and feed the captions as prompts to the SD model.
In our case, we find that using such prompts hinders the compression quality.
Specifically, we follow the same automatic captioning procedure as in PerCo (SD) and PSC, using the \texttt{Salesforce/blip2-opt-2.7b-coco} \href{https://huggingface.co/Salesforce/blip2-opt-2.7b-coco}{checkpoint} of BLIP-2 from Hugging Face.
We then continue with our standard compression approach, where the denoiser is used with standard classifier-free guidance (CFG).
Note that using text prompts requires transmitting additional bits that serve as a compressed version of the text.
Specifically, we use BLIP-2 with a maximum of $L=32$ word tokens, each picked from a dictionary containing a total of 30,524 words.
Thus, at most $32\cdot\log_{2}(30524)\approx 480$ bits are added to the bit-stream in our method.

We assess this text-conditional approach on the $512\times 512$ SD 2.1 DDM, using CFG scales of $3,6$.
We compare the performance of this conditional approach with that of the unconditional one we used in \Cref{section:compression}.
The results in \Cref{fig:compression_graphs_blip} show a disadvantage for using text-prompts for compression with our method.

\begin{figure*}[t]
    \centering
    \includegraphics[width=0.86\linewidth]{figures/Kodak24_512_extreme_app.pdf}
    \caption{\textbf{Qualitative extreme image compression results.} The presented images are taken from the Kodak24 dataset, cropped to $512\times512$ pixels.
    Our compression scheme produces highly realistic decompressed outputs, while maintaining better fidelity to the original images compared to previous methods.
    }
    \label{fig:compression_examples_app}
\end{figure*}


\begin{figure*}[t]
    \centering
    \includegraphics[height=0.93\textheight]{figures/Kodak24_512_med_app.pdf}
    \caption{
    \textbf{Qualitative image compression results.} The presented images are taken from the Kodak24 dataset, cropped to $512\times512$ pixels.
    Our compression scheme produces highly realistic decompressed outputs, while maintaining better fidelity to the original images compared to previous methods.
    }\label{fig:compression_examples_app_medium}
\end{figure*}

%pixel space compression compression
\begin{figure*}[t]
    \centering
    \includegraphics[width=\linewidth]{figures/pixel_space_comparison_low_bpp.pdf}
    \caption{\textbf{Qualitative image compression results.} the presented images are taken from the ImageNet $256\times 256$ dataset.
    Compared to previous methods, our compression scheme produces higher perceptual quality and better fidelity to the original images.
    }
    \label{fig:pixel_space_comparison_low_bpp}
\end{figure*}

\begin{figure*}[t]
    \centering
    \includegraphics[width=\linewidth]{figures/pixel_space_comparison_mid_bpp.pdf}
    \caption{\textbf{Qualitative image compression results.} the presented images are taken from the ImageNet $256\times 256$ dataset.
    Compared to previous methods, our compression scheme produces higher perceptual quality and better fidelity to the original images.
    }
    \label{fig:pixel_space_comparison_mid_bpp}
\end{figure*}


\begin{figure*}[t]
    \centering
    \includegraphics[width=1\linewidth]{figures/compression_graphs_768.pdf}
    \caption{\textbf{Quantitative image compression results on $768\times768$ sized images.} At higher bit rates, our method achieves the lowest (best) FID scores in both datasets while maintaining better distortion metrics compared to PerCo (SD). At extremely low bit rates, while PerCo (SD) shows marginally better FID scores, our method attains superior PSNR performance.}
    \label{fig:compression_graphs_768}
\end{figure*}


\begin{figure*}[t]
    \centering
    \includegraphics[width=0.9\linewidth]{figures/compression_graphs_blip.pdf}
    \caption{\textbf{Evaluating the effectiveness of using text prompts in image compression.}
    We evaluate our unconditional compression method with the text-conditional one, while using the text captions generated by BLIP-2.
    We find that using such text prompts hinders our compression results, both in terms of perceptual quality and distortion.}
    \label{fig:compression_graphs_blip}
\end{figure*}


\clearpage
\section{Compressed Conditional Generation Supplementary}\label{appendix:cond_compression}
\subsection{Background and Proof of~\Cref{prob:ode_convergence}}
We will prove that~\Cref{prob:ode_convergence} holds for any score-based diffusion model~\citep{song2020score}.
For completeness, we first provide the necessary mathematical background and then proceed to the proof of the proposition.
\subsubsection{Background}

\paragraph{Score-Based Generative Models.}Score-based generative models~\citep{song2020score} define a diffusion process $\smash{\{\rvx(t):t\in [0,T]\}}$, where $p_{0}$ and $p_{T}$ denote the data distribution and the prior distribution, respectively, and $p_{t}$ denotes the distribution of $x(t)$.
Such a diffusion process can generally be modeled as the stochastic differential equation (SDE)
\begin{align}
\delt\rvx=f(\rvx,t)\delt t+g(t)\delt\rvw,\label{eq:forward_sde}
\end{align}
where $f(\cdot,t)$ is called the \emph{drift} coefficient, $g(t)$ is called the \emph{diffusion} coefficient, $\rvw(t)$ is a standard Wiener process, and $\delt t$ denotes an infinitesimal timestep.
Samples from the data distribution $p_{0}$ can be generated by solving the reverse-time SDE~\citep{ANDERSON1982313},
\begin{align}
    &\delt\rvx=\left[f(\rvx,t)-g^{2}(t)\vs_{t}(\rvx)\right]\delt t+g(t)\delt\bar{\rvw},\label{eq:reverse_sde}
\end{align}
starting from samples of $\rvx(T)$.
Here, $\vs_{t}(\rvx(t))\coloneqq\nabla_{\rvx(t)}\log{p_{t}(\rvx(t))}$ is the \emph{score} of $p_{t}$, $\bar{\rvw}(t)$ denotes a standard Wiener process where time flows backwards, and $\delt t$ is an infinitesimal \emph{negative} timestep.
Samples from the data distribution can also be generated by solving the \emph{probability flow} ODE,
\begin{align}
    &\delt\rvx=\left[f(\rvx,t)-\frac{1}{2}g^{2}(t)\vs_{t}(\rvx)\right]\delt t.\label{eq:reverse_ode}
\end{align}

\paragraph{Solving The Reverse-Time SDE.}The reverse-time SDE in~\Cref{eq:reverse_sde} can be solved with any numerical SDE solver (e.g., Euler-Maruyama), which corresponds to some time discretization of the forward and reverse stochastic dynamics.
For the sake of our proof, we adopt the simple solver proposed by~\citet{song2020score},
\begin{align}
    &\rvx_{i-1}=\rvx_{i}-f_{i}(\rvx_{i})+g_{i}^{2}\rvs_{i}(\rvx_{i})+g_{i}\rvz_{i},\quad\rvz_{i}\sim\mathcal{N}(\vzero,\mI),\label{eq:song_discretization}
\end{align}
where $i=T,\hdots,1$ and $f_{i}$ and $g_{i}$ are the time-discretized versions of $f$ and $g$, respectively.
Note that DDPMs~\citep{ho2020denoising} are score-based diffusion models that solve a reverse-time Variance Preserving (VP) SDE, where $f(\rvx(t),t)=-\frac{1}{2}\beta(t)\rvx(t)$ and $g(t)=\sqrt{\beta(t)}$ for some function $\beta$.

\subsubsection{Proof of~\Cref{prob:ode_convergence}}
Given any general score-based diffusion model, we can write the DDCM compressed conditional generation process as
\begin{align}
    &\rvx_{i-1}=\rvx_{i}-f_{i}(\rvx_{i})+g_{i}^{2}\rvs_{i}(\rvx_{i})+g_{i}\gC_{i}(k_{i}),\label{eq:score-based-ddcm}
\end{align}
where $k_{i}$ are picked according to~\Cref{eq:k_choose_conditional}.
Choosing $\gL=\gL_{\text{P}}$, we have
\begin{align}
    k_{i}=\argmin_{k\in\{1,\hdots,K\}}\norm{\gC_{i}(k)-g_{i}\nabla_{\rvx_{i}}\log{p_{i}(\rvy|\rvx_{i})}}^{2}.
\end{align}
Since each $\gC_{i}$ contains $K$ independent samples drawn from a normal distribution $\gN(\vzero,\mI)$, we have
\begin{align}
    \{\gC_{i}(1),\hdots,\gC_{i}(K)\}\underset{K\rightarrow\infty}{\longrightarrow}\mathbb{R}^{n},
\end{align}
where $n$ denotes the dimensionality of each vector in $\gC_{i}$, and $\{\gC_{i}(1),\hdots,\gC_{i}(K)\}$ is the set comprised of all the elements in the $\gC_{i}$ (without repetition).
Since $g_{i}\nabla_{\rvx_{i}}\log{p_{i}(\rvy|\rvx_{i})}\in\mathbb{R}^{n}$, we have
\begin{align}
    \min_{k\in\{1,\hdots,K\}}\norm{\gC_{i}(k)-g_{i}\nabla_{\rvx_{i}}\log{p_{i}(\rvy|\rvx_{i})}}^{2}\underset{K\rightarrow\infty}{\longrightarrow}0.
\end{align}
Thus,
\begin{align}
    \gC_{i}(k_{i})\underset{K\rightarrow\infty}{\longrightarrow}g_{i}\nabla_{\rvx_{i}}\log{p_{i}(\rvy|\rvx_{i})}.\label{eq:noise_convergence_to_grad}
\end{align}
Plugging~\Cref{eq:noise_convergence_to_grad} into~\Cref{eq:score-based-ddcm}, we get
\begin{align}
    \rvx_{i-1}\underset{K\rightarrow\infty}{\longrightarrow}&\rvx_{i}-f_{i}(\rvx_{i})+g_{i}^{2}\rvs_{i}(\rvx_{i})+g_{i}^{2}\nabla_{\rvx_{i}}\log{p_{i}(\rvy|\rvx_{i})}\\
    =&\rvx_{i}-f_{i}(\rvx_{i})+g_{i}^{2}\nabla_{\rvx_{i}}\log{p_{i}(\rvx_{i})}+g_{i}^{2}\nabla_{\rvx_{i}}\log{p_{i}(\rvy|\rvx_{i})}\\
    =&\rvx_{i}-f_{i}(\rvx_{i})+g_{i}^{2}\left[\nabla_{\rvx_{i}}\log{p_{i}(\rvx_{i})}+\nabla_{\rvx_{i}}\log{p_{i}(\rvy|\rvx_{i})}\right]\\
    =&\rvx_{i}-f_{i}(\rvx_{i})+g_{i}^{2}\nabla_{\rvx_{i}}\log{p_{i}(\rvx_{i}|\rvy)},\label{eq:bayes_rule_for_ode}
\end{align}
where in~\Cref{eq:bayes_rule_for_ode} we used Bayes rule and the fact that $\nabla_{\rvx_{i}}\log{p(\rvy)}=0$.
Note that~\Cref{eq:bayes_rule_for_ode} resembles a time discretization of a probability flow ODE (\Cref{eq:reverse_ode}) over the posterior distribution $p_{0}(\rvx_{0}|\rvy)$, with $f(\cdot,t)$ and $\sqrt{2}g(t)$ being the drift and diffusion coefficients in continuous time, respectively.
Thus, when $K\rightarrow\infty$, our compressed conditional generation process becomes a sampler from $p_{0}(\rvx_{0}|\rvy)$.

% for every $\gamma>0$ and $k$ we have
% \begin{align}
%     \mathbb{P}(\norm{\gC_{i}(k)-g_{i}\nabla_{\rvx_{i}}\log{p_{i}(\rvy|\rvx_{i})}}^{2}\leq\gamma)>0.
% \end{align}

\subsection{Image Compression as a Private Case of Compressed Conditional Generation}\label{appendix:compression_private_case}
We show that our standard image compression scheme from~\Cref{section:compression} is a private case of our compressed conditional generation scheme from~\Cref{sec:compressed_conditional_generation}, where $\rvy=\rvx_{0}$ and $\gL=\gL_{\text{P}}$.
When $\rvy=\rvx_{0}$ we have
\begin{align}
    \nabla_{\rvx_{i}}\log{p_{i}(\rvy|\rvx_{i})}&=\nabla_{\rvx_{i}}\log{p_{i}(\rvx_{0}|\rvx_{i})}\nonumber\\
    &=\nabla_{\rvx_{i}}\log{p_{i}(\rvx_{i}|\rvx_{0})}-\nabla_{\rvx_{i}}\log{p_{i}(\rvx_{i})}\\
    &=\nabla_{\rvx_{i}}\log{p_{i}(\rvx_{i}|\rvx_{0})}-\vs_{i}(\rvx_{i})\label{eq:compression-grad}
\end{align}
where $\nabla_{\rvx_{i}}\log{p_{i}(\rvx_{i}|\rvx_{0})}$ can be computed in closed-form via the forward diffusion process in~\Cref{eq:forward_gaussian_diffusion}.
In particular, we have $p_{i}(\rvx_{i}|\rvx_{0})=\gN(\rvx_{i};\sqrt{\bar{\alpha}_{i}}\rvx_{0},(1-\bar{\alpha}_{i})\mI)$~\citep{ho2020denoising}, and thus
\begin{align}
    \nabla_{\rvx_{i}}\log{p_{i}(\rvx_{i}|\rvx_{0})}&=-\nabla_{\rvx_{i}}\frac{\norm{\rvx_{i}-\sqrt{\bar{\alpha}_{i}}\rvx_{0}}^{2}}{2(1-\bar{\alpha}_{i})}\\
    &=-\frac{\rvx_{i}-\sqrt{\bar{\alpha}_{i}}\rvx_{0}}{1-\bar{\alpha}_{i}}\\
    &=\frac{\sqrt{\bar{\alpha}_{i}}\rvx_{0}-\rvx_{i}}{1-\bar{\alpha}_{i}}.\label{eq:vp-sde-forward-grad}
\end{align}
Moreover, it is well known that~\citep{ho2020denoising,song2020score}
\begin{align}
    \vs_{i}(\rvx_{i})=\frac{\sqrt{\bar{\alpha}_{i}}\hat{\rvx}_{0|i}-\rvx_{i}}{1-\bar{\alpha}_{i}},\label{eq:score-mmse-in-vp-sde}
\end{align}
where $\hat{\rvx}_{0|i}$ is the (time-aware) Minimum Mean-Squared-Error (MMSE) estimator of $\rvx_{0}$ given $\rvx_{i}$.
Plugging~\Cref{eq:score-mmse-in-vp-sde,eq:vp-sde-forward-grad} into~\cref{eq:compression-grad}, we get
\begin{align}
    \nabla_{\rvx_{i}}\log{p_{i}(\rvx_{0}|\rvx_{i})}=\frac{\sqrt{\bar{\alpha}_{i}}}{1-\bar{\alpha}_{i}}(\rvx_{0}-\hat{\rvx}_{0|i}).\label{eq:compression_vp_sde_likelihood}
\end{align}
Thus, we have
\begin{align}
    k_{i}&=\argmin_{k\in\{1,\hdots,K\}}\gL_{\text{P}}(\rvy,\rvx_{i},\gC_{i},k)\label{eq:compression_posterior}\\
    &=\argmin_{k\in\{1,\hdots,K\}}\bignorm{\gC_{i}(k)-\sigma_{i}\frac{\sqrt{\bar{\alpha}_{i}}}{1-\bar{\alpha}_{i}}(\rvx_{0}-\hat{\rvx}_{0|i})}^{2}\\
    &=\argmin_{k\in\{1,\hdots,K\}}\bignorm{\gC_{i}(k)}^{2}-2\langle\gC_{i}(k),\sigma_{i}\frac{\sqrt{\bar{\alpha}_{i}}}{1-\bar{\alpha}_{i}}(\rvx_{0}-\hat{\rvx}_{0|i})\rangle+\bignorm{\sigma_{i}\frac{\sqrt{\bar{\alpha}_{i}}}{1-\bar{\alpha}_{i}}(\rvx_{0}-\hat{\rvx}_{0|i})}^{2}\\
    &=\argmin_{k\in\{1,\hdots,K\}}\bignorm{\gC_{i}(k)}^{2}-2\langle\gC_{i}(k),\sigma_{i}\frac{\sqrt{\bar{\alpha}_{i}}}{1-\bar{\alpha}_{i}}(\rvx_{0}-\hat{\rvx}_{0|i})\rangle.
\end{align}
Below, we show that $\norm{\gC_{i}(k)}^{2}$ is roughly equal for every $k$.
Thus, it holds that
\begin{align}
    k_{i}&=\argmin_{k\in\{1,\hdots,K\}}\bignorm{\gC_{i}(k)}^{2}-2\langle\gC_{i}(k),\sigma_{i}\frac{\sqrt{\bar{\alpha}_{i}}}{1-\bar{\alpha}_{i}}(\rvx_{0}-\hat{\rvx}_{0|i})\rangle\\
    &\approx\argmin_{k\in\{1,\hdots,K\}}\text{const}-2\langle\gC_{i}(k),\sigma_{i}\frac{\sqrt{\bar{\alpha}_{i}}}{1-\bar{\alpha}_{i}}(\rvx_{0}-\hat{\rvx}_{0|i})\rangle\\
    &=\argmax_{k\in\{1,\hdots,K\}}\langle\gC_{i}(k),\sigma_{i}\frac{\sqrt{\bar{\alpha}_{i}}}{1-\bar{\alpha}_{i}}(\rvx_{0}-\hat{\rvx}_{0|i})\rangle\\
    &=\argmax_{k\in\{1,\hdots,K\}}\sigma_{i}\frac{\sqrt{\bar{\alpha}_{i}}}{1-\bar{\alpha}_{i}}\langle\gC_{i}(k),\rvx_{0}-\hat{\rvx}_{0|i}\rangle\\
    &=\argmax_{k\in\{1,\hdots,K\}}\langle\gC_{i}(k),\rvx_{0}-\hat{\rvx}_{0|i}\rangle.\label{eq:equiv_kstar_compression}
\end{align}
Note that the noise selection strategy in~\Cref{eq:equiv_kstar_compression} is similar to that of our standard compression scheme, namely~\Cref{eq:compression_rule}.
Thus, our compression method is a private case of our compressed conditional generation approach.
In practice, we used~\Cref{eq:compression_rule} instead of~\Cref{eq:compression_posterior} since the former worked slightly better.


To show that $\norm{\gC_{i}(k)}^{2}$ is roughly constant for every $k$, note that $\gC_{i}(k)$ is a sample from a $n$-dimensional multivariate normal distribution $\mathcal{N}(\vzero,\mI)$.
Thus, $\norm{\gC_{i}(k)}^{2}$ is a sample from a chi-squared distribution with $n$ degrees of freedom.
It is well known that samples from this distribution strongly concentrate around its mean $n$ for large values of $n$.
Namely, $\norm{\gC_{i}(k)}^{2}$ is highly likely to be close to $n$, especially for relatively small values of $K$.
\clearpage
\subsection{Compressed Posterior Sampling for Image Restoration}\label{appendix:zero-shot}
DPS and DDNM are implemented with the official settings recommended by the authors~\citep{chung2023diffusion,wang2023zeroshot}.
Specifically, DPS uses DDPM with $T=1000$ sampling steps, and DDNM uses DDIM with $\eta=0.85$ and $T=100$ sampling steps.
We also tried $T=1000$ for DDNM and found that $T=100$ works slightly better for the tasks considered.
The additional qualitative comparisons in \Cref{fig:srx4_additional_samples,fig:colorization_additional_samples} further demonstrate that our method produces better output perceptual quality compared to DPS and DDNM.
\begin{figure*}[t]
    \centering
    \includegraphics[width=1\textwidth]{figures/srx4_additional_samples.pdf}
    \caption{\textbf{Qualitative comparison of zero-shot image super-resolution methods (posterior sampling).} Our approach clearly produces better output perceptual quality compared to previous methods.}
    \label{fig:srx4_additional_samples}
\end{figure*}
\begin{figure*}[t]
    \centering
    \includegraphics[width=1\textwidth]{figures/colorization_additional_samples.pdf}
    \caption{\textbf{Qualitative comparison of zero-shot image colorization methods (posterior sampling).} Our approach clearly produces better output perceptual quality compared to previous methods.}
    \label{fig:colorization_additional_samples}
\end{figure*}
\clearpage
\subsection{Compressed Real-World Face Image Restoration}\label{app:dmax-ot}
\subsubsection{Explaining the Choice of $\vr(\rvy)$}
To explain our choice of $\vr(\rvy)\approx\mathbb{E}[\rvx_{0}|\rvy]$, first note that the MSE of \emph{any} estimator $\hat{\rvx}_{0}$ of $\rvx_{0}$ given $\rvy$ can be written as~\citep{freirich2021a}
\begin{align}
    \text{MSE}(\rvx_{0},\hat{\rvx}_{0})
    &=\text{MSE}(\rvx_{0},\vr(\rvy))+\text{MSE}(\vr(\rvy),\hat{\rvx}_{0})\nonumber\\
    &=\text{MSE}(\vr(\rvy),\hat{\rvx}_{0})+c\label{eq:orthogonality}.
\end{align}
where $c$, the MMSE, does not depend on $\hat{\rvx}_{0}$.
In theory, our goal is to optimize the tradeoff between the MSE of $\hat{\rvx}_0$ and its output perceptual quality according to some quality measure $Q$. This can be accomplished by solving
\begin{align}\label{eq:desired_tradeoff}
    \min_{\hat{\rvx}_{0}}\text{MSE}(\rvx_{0},\hat{\rvx}_{0})+\lambda Q(\hat{\rvx}_{0}),
\end{align}
where $\lambda$ is some hyper-parameter that controls the perception-distortion tradeoff.
At test time, however, we do not have access to the original image $\rvx_0$.
By plugging \Cref{eq:orthogonality} into \Cref{eq:desired_tradeoff} we obtain
\begin{align}
    \min_{\hat{\rvx}_{0}}\text{MSE}(\rvx_{0},\hat{\rvx}_{0})+\lambda Q(\hat{\rvx}_{0})&=\min_{\hat{\rvx}_{0}}\text{MSE}(\vr(\rvy),\hat{\rvx}_{0})+c+\lambda Q(\hat{\rvx}_{0})\nonumber\\
    &=\min_{\hat{\rvx}_{0}}\text{MSE}(\vr(\rvy),\hat{\rvx}_{0})+\lambda Q(\hat{\rvx}_{0}).\label{eq:opt-mse-real-world}
\end{align}
Namely, as long as $\vr(\rvy)$ is a good approximation of the \emph{true} MMSE estimator, we can rely on it for optimizing tradeoff (\ref{eq:desired_tradeoff}) without having access to $\rvx_{0}$.
This resembles our approach in \Cref{sec:bfr}, where we \emph{greedily} optimize \Cref{eq:opt-mse-real-world} throughout the trajectory of the DDCM sampling process.



\subsubsection{Additional Details and Experiments}
We use the PyIQA package to compute all perceptual quality measures, with \texttt{niqe} for NIQE, \texttt{topiq\_nr-face} for TOPIQ, \texttt{clipiqa+} for $\text{CLIP-IQA}^{+}$, and \texttt{fid\_dinov2} for $\text{FD}_{\text{DINOv2}}$, adopting the default settings for each measure.
Additional qualitative and quantitative comparisons are shown in \cref{fig:real-world-qualitative-appendix,,fig:lfw-quantitative-appendix,,fig:celeba-quantitative-appendix,,fig:webphoto-quantitative-appendix}.

\begin{figure}
    \centering
    \includegraphics[height=21cm]{figures/blind_face_restoration/real-world-comparison-appendix.pdf}
    \caption{\textbf{Qualitative comparison of real-world face image restoration methods}. Our method produces high perceptual quality results with less artifacts compared to previous methods, especially for challenging datasets such as WIDER-Test.}
    \label{fig:real-world-qualitative-appendix}
\end{figure}
\begin{figure}
    \centering
    \includegraphics[width=1\textwidth]{figures/blind_face_restoration/blind_face_restoration_main_text_celeba.pdf}
    \caption{\textbf{Quantitative comparison of real-world face image restoration methods, evaluated on the CelebA-Test dataset.} We successfully optimize each NR-IQA measure, surpassing the scores of previous methods. Here, only our NIQE-based solution generalizes well to $\text{FD}_{\text{DINOv2}}$ in terms of perceptual quality.}
    \label{fig:celeba-quantitative-appendix}
\end{figure}
\begin{figure}
    \centering
    \includegraphics[width=1\textwidth]{figures/blind_face_restoration/blind_face_restoration_main_text_lfw.pdf}
    \caption{\textbf{Quantitative comparison of real-world face image restoration methods, evaluated on the LFW-Test dataset.} We successfully optimize each NR-IQA measure, surpassing the scores of previous methods. All our solutions achieve impressive $\text{FD}_{\text{DINOv2}}$ scores, while our NIQE-based solution surpasses all methods according to this measure.}
    \label{fig:lfw-quantitative-appendix}
\end{figure}
\begin{figure}
    \centering
    \includegraphics[width=1\textwidth]{figures/blind_face_restoration/blind_face_restoration_main_text_webphoto.pdf}
    \caption{\textbf{Quantitative comparison of real-world face image restoration methods, evaluated on the WebPhoto-Test dataset.} We successfully optimize each NR-IQA measure, surpassing the scores of previous methods. Our TOPIQ-based solution achieves the best $\text{FD}_{\text{DINOv2}}$ scores compared to all methods..}
    \label{fig:webphoto-quantitative-appendix}
\end{figure}

\clearpage


\subsection{Compressed Classifier Guidance}\label{appendix:classifier-guidance}
\begin{figure*}[t]
    \centering
    \includegraphics[height=15cm]{figures/our_cg_256noises_2optnoises.jpg}\hspace{1cm}\includegraphics[height=15cm]{figures/standard_cg_scale=20.jpg}
    \caption{\textbf{Qualitative comparison of CCG (left) with CG (right).} CCG achieves superior image quality compared to CG while avoiding the use of classifier gradients. Additionally, CCG enables decompression without requiring access to the original class labels.}
    \label{fig:cg_visual_comparison}
\end{figure*}
Consider the case where $\rvy$ represents the \emph{class} of an image $\rvx_{0}$.
An unconditional score-based generative model can be \emph{guided} to generate samples the posterior $p_{0}(\rvx_{0}|\rvy)$, by perturbing the generated samples according to the gradient of a time-dependent trained classifier $c_{\theta}(\rvy;\rvx_{i},i)\approx p_{i}(\rvy|\rvx_{i})$~\citep{dhariwal2021diffusion}.
This approach is known as classifier guidance (CG).
Such a guidance method can be interpreted as an attempt to \emph{confuse} the classifier by perturbing its input adversarially~\citep{ho2021classifier}.
However, trained classifiers are typically not robust to adversarial perturbations, making their gradients largely unreliable and unaligned with human perception~\citep{advattacks,tsipras2018robustness,ganz-perceptual}.
Thus, the standard CG approach has not seen major success~\citep{ho2021classifier}.


We propose an alternative to this method, circumventing the reliance on the classifier's gradient.
Specifically, we set $\gL$ in \Cref{eq:k_choose_conditional} as
\begin{align}
\gL(\rvy,\rvx_{i},\gC_{i},k)=-\log{c_{\theta}(\rvy;\vmu_{i}(\rvx_{i})+\sigma_{i}\gC_{i}(k),i)}.\label{eq:classifier-guidance-ours}
\end{align}
Thus, $\gL(\rvy,\rvx_{i},\gC_{i},k)$ attains a lower value when $\sigma_{i}\gC_{i}(k)$ points in some direction that maximizes the probability of the class $\rvy$.
Note that since the codebooks remain fixed, choosing $k_{i}$ (out of $1,\hdots,K$) to minimize~\Cref{eq:k_choose_conditional} would always lead to the same generated sample for every $\rvy$.
Thus, we promote sample diversity by first \emph{randomly} selecting a subset of $\tilde{K}<K$ indices $k_{i,1},\hdots,k_{i,\tilde{K}}\sim\text{Unif}(\{1,\hdots,K\})$, and then choosing
\begin{align}
    k_{i}=\argmin_{k\in\{k_{i,1},\hdots,k_{i,\tilde{K}}\}}\gL(\rvy,\rvx_{i},\gC_{i},k).
\end{align}
We coin our method Compressed CG (CCG).


We compare our proposed CCG with the standard CG using the same unconditional diffusion model and time-dependent classifier trained on ImageNet $256\times256$~\citep{deng2009imagenet,dhariwal2021diffusion}.
We compare the methods ``on the same grounds'' by using the same standard DDPM noise schedule and $T=1000$ diffusion steps.
Our method is assessed with $K=256$ and $\tilde{K}=2$, while the standard CG is assessed with CG scales $s\in\{1,10,20\}$.
The quantitative comparison in~\Cref{tab:cg_comparison} shows that CCG achieves better (lower) FID and $\text{FD}_{\text{DINOv2}}$ scores.
A visual comparison is provided in~\Cref{fig:cg_visual_comparison}.
Note that while using DDCM with standard CG does still produces compressed output images, decoding the produced bit-streams requires access to $\rvy$.
Using DDCM with CCG instead sidesteps this limitation, as $\rvy$ is not needed for decompression.

\begin{table}[ht]
\centering
\caption{Quantitative comparison of \emph{compressed} classifier guidance (CCG) and standard classifier guidance (CG) for ImageNet $256\times 256$ conditional image generation, using an unconditional DDM and a classifier for guidance. Our proposed CCG not only outperforms CG in terms of generation performance, but also automatically produces compressed image representations.}
\begin{tabular}{lcc}
\toprule
 &  \begin{tabular}{c}Compressed CG \textbf{(Ours)}\\$K=256,\tilde{K}=2$\end{tabular}& \begin{tabular}{c}Standard CG\\$s=1\mid10\mid20$\end{tabular} \\
\midrule
FID & $\mathbf{13.669}$ & $31.548\mid14.481\mid 14.921$ \\
$\text{FD}_{\text{DINOv2}}$ & $\mathbf{204.693}$ & $459.42 \mid255.41 \mid248.34 $ \\
\bottomrule
\end{tabular}
\label{tab:cg_comparison}
\end{table}

\clearpage
\subsection{Compressed Classifier-Free Guidance}\label{appendix:classifier-free-guidance}
\begin{figure*}[t]
    \centering
    \includegraphics[width=\textwidth]{figures/ccfg_fig_1.pdf}
    \caption{\textbf{Qualitative comparison of CCFG with CFG.} CCFG achieves comparable image quality and diversity to CFG, while enabling decompression without requiring the original inputs.}
    \label{fig:ccfg_visual_1}
\end{figure*}

\begin{figure*}[t]
    \centering
    \includegraphics[width=\textwidth]{figures/ccfg_fig_2.pdf}
    \caption{\textbf{Qualitative comparison of CCFG with CFG.} CCFG achieves comparable image quality and diversity to CFG, while enabling decompression without requiring the original inputs.}
    \label{fig:ccfg_visual_2}
\end{figure*}

\begin{figure*}
    \centering
    \includegraphics[width=0.5\textwidth]{figures/ccfg_cfg_comparison.pdf}
    \caption{\textbf{Quantitative evaluation of CCFG and CFG.} CCFG achieves comparable FID scores to CFG while achieving slightly lower fidelity to the input prompts. However, unlike CFG, CCFG enables decompression without access to the original conditioning inputs.}
    \label{fig:ccfg-cf-quantitative-comparison}
\end{figure*}
The task of text-conditional image generation can be solved using a \emph{conditional} diffusion model, which, theoretically speaking, learns to sample from the posterior distribution $p_{0}(\rvx_{0}|\rvy)$.
In practice, however, using a conditional model directly typically yields low fidelity to the inputs.
To address this limitation, CG can be used to improve this fidelity at the expense of sample quality and diversity~\citep{dhariwal2021diffusion}.
Classifier-Free Guidance (CFG) is used more often in practice, as it achieves the same tradeoff by mixing the conditional and unconditional scores during sampling~\citep{ho2021classifier}, thus eliminating the need for a classifier.
Particularly, assuming we have access to both the conditional score $\vs_{i}(\rvx_{i},\rvy)\coloneqq\nabla_{\rvx_{i}}\log{p_{i}(\rvx_{i}|\rvy)}$ and the unconditional one $\vs_{i}(\rvx_{i})$, CFG proposes to modify the conditional score by
\begin{align}
    \tilde{\vs}_{i}(\rvx_{i},\rvy)=(1+w)\vs_{i}(\rvx_{i},\rvy)-w\vs_{i}(\rvx_{i}),
\end{align}
where $w$, the CFG scale, is a hyper-parameter controlling the tradeoff between sample quality and diversity.

Here, we introduce a new CFG method that allows generating compressed conditional samples using any pair of conditional and unconditional diffusion models, while controlling the tradeoff between generation quality and the fidelity to the inputs.
Specifically, since $\nabla_{\rvx_{i}}\log{p_{i}(\rvy|\rvx_{i})}=\vs_{i}(\rvx_{i}|\rvy)-\vs_{i}(\rvx_{i})$, we simply use
\begin{align}
    \gL(\rvy,\rvx_{i},\gC_{i},k)=-\langle\gC_{i}(k),\vs_{i}(\rvx_{i}|\rvy)-\vs_{i}(\rvx_{i})\rangle.\label{eq:l_ccfg}
\end{align}
Note that optimizing~\Cref{eq:l_ccfg} is roughly  equivalent to optimizing $\gL_{\text{P}}$ when $\rvx_{i}$ is high dimensional (see~\Cref{appendix:compression_private_case}).
As in~\Cref{appendix:classifier-guidance}, we promote sample diversity by choosing $k_{i}$ from a randomly sampled subset of $\tilde{K}<K$ indices at each step during the generation.
We coin our method Compressed CFG (CCFG).

We implement our method using SD 2.1 trained on $768\times768$ images, adopting a DDPM noise schedule with $T=1000$ diffusion steps, $K=64$ fixed vectors in each codebook and $\tilde{K}\in\{2,3,4,6,9\}$.
We compare against the same diffusion model with standard DDPM sampling, using $T=1000$ steps and CFG scales $w\in\{2,5,8,11\}$.
The generative performance of both methods is assessed by computing the FID between 10k generated samples and MS-COCO, similarly to \Cref{sec:method}.
Additionally, we evaluate the alignment between the outputs and the input text prompts using the CLIP score~\citep{hessel2021clipscore} with the OpenAI CLIP ViT-L/14 model~\citep{pmlr-v139-radford21a}.

Figure~\ref{fig:ccfg-cf-quantitative-comparison} shows that our CCFG method is on par with CFG in terms of FID, while CFG produces higher CLIP scores.
This suggests that the outputs of CFG better align with the input text prompts compared to CCFG.
Yet, the qualitative comparisons in \Cref{fig:ccfg_visual_1,fig:ccfg_visual_2} show that there is no significant difference between the methods.
Importantly, decoding the bit-streams produced by CCFG does involve accessing the original input $\rvy$, and so our loss in CLIP scores are expected due to the rate-perception-distortion tradeoff~\citep{pmlr-v97-blau19a} (here, we achieve $\frac{1000\cdot\log_{2}(64)}{768^2}\approx 0.01$ BPP).
Note that using CCFG in DDCM is fundamentally different than using CFG (\Cref{sec:method}), since the latter requires access to $\rvy$.
\clearpage
% NOTE:
% All editing commands contain curly braces {} around them so that 
% they can easily be searched for in the final document via cmd + f.


% Highlighted comments for authors
\usepackage{soul, color}
\newcommand{\Note}[1]{}
\renewcommand{\Note}[1]{#1}  % comment out this definition to suppress all Notes
\newcommand{\NoteSigned}[3]{{\textcolor{#2}{\Note{\textbf{[#1: #3]}}}}}
% Alternative with highlighter color rather than text colour
%\renewcommand{\NoteSigned}[3]{{\sethlcolor{#2}\Note{\hl{#1: #3}}}}

%%%%%%%%%%%%%%%%%% Make changes here %%%%%%%%%%%%%%%%%%%%%%
% Specific author comments
% \newcommand{\vd}[1]{\NoteSigned{VD}{blue}{#1}}
% \newcommand{\fv}[1]{\NoteSigned{FV}{green}{#1}}
\newcommand{\kd}[1]{\NoteSigned{KD}{brown}{#1}}
\newcommand{\sm}[1]{\NoteSigned{SM}{magenta}{#1}}
%%%%%%%%%%%%%%%%%%%%%%%%%%%%%%%%%%%%%%%%%%%%%%%%%%%%%%%%%%%

% Inline todo: Flag something in thext to be done
\newcommand{\itodo}[1]{\NoteSigned{TODO}{red}{#1}}
% Preliminary: Flag preliminary statements or results
\newcommand{\prelim}[1]{\NoteSigned{Prelim}{grey}{#1}}
\end{document}