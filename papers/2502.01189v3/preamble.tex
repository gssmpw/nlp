\PassOptionsToPackage{backref=page}{hyperref}

% Recommended, but optional, packages for figures and better typesetting:
\usepackage{microtype}
\usepackage{graphicx}
\usepackage{subfigure}
\usepackage{booktabs} % for professional tables

% hyperref makes hyperlinks in the resulting PDF.
% If your build breaks (sometimes temporarily if a hyperlink spans a page)
% please comment out the following usepackage line and replace
% \usepackage{icml2025} with \usepackage[nohyperref]{icml2025} above.
\usepackage{hyperref}
\usepackage{xspace}
\usepackage{upgreek}
\usepackage{enumitem}


% Attempt to make hyperref and algorithmic work together better:
\newcommand{\theHalgorithm}{\arabic{algorithm}}

% Use the following line for the initial blind version submitted for review:
% \usepackage{icml2025}

% If accepted, instead use the following line for the camera-ready submission:
\usepackage[accepted]{icml2025}

% For theorems and such
\usepackage{amsmath}
\usepackage{amssymb}
\usepackage{capt-of}


\usepackage{mathtools}
\usepackage{amsthm}
\usepackage{thm-restate}
\usepackage{multirow}



\DeclareMathOperator*{\argmax}{arg\,max}
\DeclareMathOperator*{\argmin}{arg\,min}
\newcommand{\bignorm}[1]{\left\lVert#1\right\rVert}
\newcommand{\norm}[1]{\lVert#1\rVert}

% if you use cleveref..
\usepackage[capitalize]{cleveref}
\crefname{equation}{}{}
\Crefname{equation}{Eq.}{Eqs.}
\Crefname{figure}{Fig.}{Figs.}
\Crefname{figures}{Figs.}{Figs.}
\Crefname{section}{Sec.}{Secs.}
\Crefname{sections}{Secs.}{Secs.}
\Crefname{table}{Tab.}{Tabs.}
\Crefname{tables}{Tabs.}{Tabs.}
\Crefname{appendix}{App.}{Apps.}


%%%%%%%%%%%%%%%%%%%%%%%%%%%%%%%%
% THEOREMS
%%%%%%%%%%%%%%%%%%%%%%%%%%%%%%%%
\theoremstyle{plain}
\newtheorem{theorem}{Theorem}[section]
\newtheorem{proposition}[theorem]{Proposition}
\newtheorem{lemma}[theorem]{Lemma}
\newtheorem{corollary}[theorem]{Corollary}
\theoremstyle{definition}
\newtheorem{definition}[theorem]{Definition}
\newtheorem{assumption}[theorem]{Assumption}
\theoremstyle{remark}
\newtheorem{remark}[theorem]{Remark}

% Todonotes is useful during development; simply uncomment the next line
%    and comment out the line below the next line to turn off comments
%\usepackage[disable,textsize=tiny]{todonotes}
\usepackage[textsize=tiny]{todonotes}

\newcommand{\gnote}[1]{{\color{blue}[G: #1]}}
\newcommand{\hnote}[1]{{\color{magenta}[H: #1]}}
\newcommand{\tnote}[1]{{\color{red}[T: #1]}}
\newcommand{\mnote}[1]{{\color{orange}[M: #1]}}

\newcommand{\fullname}{Denoising Diffusion Codebook Models\xspace}
\newcommand{\shortname}{DDCMs\xspace}
\newcommand{\shortnameplus}{$\text{DDCM}^{+}$\xspace}
\newcommand{\dynamictitle}{Compressed Image Generation with \fullname}


\usepackage{amsfonts,bm}
\newcommand{\p}{\textsc{p}}

\newcommand{\data}{\mathcal{D}}
\newcommand{\extdata}{\mathcal{D}_e}
\newcommand{\shots}{S}
\newcommand{\classes}{C}
\newcommand{\numbershadowmodels}{M}

\newcommand{\hpofunction}{$\texttt{HPO}$}
\newcommand{\trainfunction}{$\textsc{train}$}

\newcommand{\hyperparamsdata}{\eta_{{}_{\data}}}
\newcommand{\hyperparamsextdata}{\eta_{{}_{\extdata}}}

\newcommand{\model}{\mathcal{M}} 
\newcommand{\modeltar}{\mathcal{M}_{\mathcal{T}}} 
\newcommand{\modelshadow}{\mathcal{M}_{\mathcal{S}}} 
\newcommand{\return}{\textbf{return }}
\newcommand{\grid}{\mathbb{M}}

\newcommand{\tpr}{\textsc{tpr}}
\newcommand{\fpr}{\textsc{fpr}}


\newcommand{\lira}{\mathrm{LiRA}} 

\newcommand{\datashadow}{\mathcal{D}_{\mathrm{shadow}}}
\newcommand{\architecture}{\mathcal{A}}
\newcommand{\prob}{\mathbb{P}}
\newcommand{\normal}{\mathcal{N}}
\newcommand{\attack}{\texttt{KNOWN}}
\newcommand{\bb}{\texttt{BLACK-BOX}}
\newcommand{\logits}{\textsc{logits}}



\definecolor{green300}{HTML}{60d360}
\definecolor{green500}{HTML}{32b432}
\definecolor{green1000}{HTML}{2ca02c}
\definecolor{greenpursuit}{HTML}{047d50}
\definecolor{bluepsc}{HTML}{1f77b4}
\definecolor{lightbluepsc}{HTML}{aec7e8}