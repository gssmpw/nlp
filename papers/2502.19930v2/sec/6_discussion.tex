% \vspace{-10pt}
% %%% [START] Ablation-fp_iter
\begin{figure}[t] % 1-column
\footnotesize
\centering 

% Adjust the image width to fit within one column
\newcommand{\imgwidth}{0.18\linewidth} % Set image width to 18% of the line width

% 1st Image
\begin{tikzpicture}[x=1cm, y=1cm]
    \node[anchor=south] (FigA1) at (0,0) {
        \includegraphics[width=\imgwidth]{Fig./Qual/imgs/ablation/fp_iter/original.png}
    };
    \node[anchor=south, yshift=-1mm] at (FigA1.north) {\footnotesize Source};
\end{tikzpicture}\hspace{-1mm}%
% 2nd Image
\begin{tikzpicture}[x=1cm, y=1cm]
    \node[anchor=south] (FigD1) at (0,0) {
        \includegraphics[width=\imgwidth]{Fig./Qual/imgs/ablation/fp_iter/fp1.png}
    };
    \node[anchor=south, yshift=-1mm] at (FigD1.north) {\footnotesize iteration 1};
\end{tikzpicture}\hspace{-1mm}%
% 3rd Image
\begin{tikzpicture}[x=1cm, y=1cm]
    \node[anchor=south] (FigC1) at (0,0) {
        \includegraphics[width=\imgwidth]{Fig./Qual/imgs/ablation/fp_iter/fp3.png}
    };
    \node[anchor=south, yshift=-1mm] at (FigC1.north) {\footnotesize iteration 3};
\end{tikzpicture}\hspace{-1mm}%
% 4th Image
\begin{tikzpicture}[x=1cm, y=1cm]
    \node[anchor=south] (FigB1) at (0,0) {
        \includegraphics[width=\imgwidth]{Fig./Qual/imgs/ablation/fp_iter/fp5.png}
    };
    \node[anchor=south, yshift=-1mm] at (FigB1.north) {\footnotesize iteration 5};
\end{tikzpicture}\hspace{-1mm}%
% 5th Image
\begin{tikzpicture}[x=1cm, y=1cm]
    \node[anchor=south] (FigE1) at (0,0) { % Changed label to FigE1
        \includegraphics[width=\imgwidth]{Fig./Qual/imgs/ablation/fp_iter/fp8.png}
    };
    \node[anchor=south, yshift=-1mm] at (FigE1.north) {\footnotesize iteration 8}; % Replace with appropriate label
\end{tikzpicture}
\vspace{-3pt}
% Caption Text
\setulcolor{magenta}
\setul{0.3pt}{2pt}
\centering\textit{``Cat" $\to$ ``\ul{Pig}"} 
\vspace{-6pt}
\caption{\textbf{Ablation study} on the number of iterations. With three or more iterations, precise image editing is achievable while preserving the structure of the source image.}
\vspace{-10pt}
\label{fig:ablation_fp_iter}
\end{figure}
% \vspace{-10pt}
\begin{figure}[t] % 1-column
\footnotesize
\centering 

% Adjust the image width to fit within one column
\newcommand{\imgwidth}{0.18\linewidth} % Set image width to 16% of the line width

% 1st Image
\begin{tikzpicture}[x=1cm, y=1cm]
    \node[anchor=south] (FigA1) at (0,0) {
        \includegraphics[width=\imgwidth]{Fig./Qual/imgs/ip2p_sub/shark/src_img.jpg}
    };
    \node[anchor=south, yshift=-1.5mm] at (FigA1.north) {\footnotesize Source};
\end{tikzpicture}\hspace{-1mm}%
% 2nd Image
\begin{tikzpicture}[x=1cm, y=1cm]
    \node[anchor=south] (FigB1) at (0,0) {
        \includegraphics[width=\imgwidth]{Fig./Qual/imgs/ip2p_sub/shark/dds.jpg}
    };
    \node[anchor=south, yshift=-1.5mm] at (FigB1.north) {\footnotesize DDS};
\end{tikzpicture}\hspace{-1mm}%
\begin{tikzpicture}[x=1cm, y=1cm]
    \node[anchor=south] (FigB1) at (0,0) {
        \includegraphics[width=\imgwidth]{Fig./Qual/imgs/ablation/scale/scale_0.4-iter_1-optimstep_200.jpg}
    };
    \node[anchor=south, yshift=-1.5mm] at (FigB1.north) {\footnotesize $\lambda=0.4$};
\end{tikzpicture}\hspace{-1mm}%
% % 3rd Image
\begin{tikzpicture}[x=1cm, y=1cm]
    \node[anchor=south] (FigC1) at (0,0) {
        \includegraphics[width=\imgwidth]{Fig./Qual/imgs/ablation/scale/scale_0.8-iter_1-optimstep_200.jpg}
    };
    \node[anchor=south, yshift=-1.5mm] at (FigC1.north) {\footnotesize $\lambda=0.8$};
\end{tikzpicture}\hspace{-1mm}%
% 4th Image
\begin{tikzpicture}[x=1cm, y=1cm]
    \node[anchor=south] (FigD1) at (0,0) {
        \includegraphics[width=\imgwidth]{Fig./Qual/imgs/ablation/scale/scale_1.0-iter_1-optimstep_200.jpg}
    };
    \node[anchor=south, yshift=-1.5mm] at (FigD1.north) {\footnotesize $\lambda=1.0$};
\end{tikzpicture}
\vspace{-3pt}
% Caption Text
\setulcolor{magenta}
\setul{0.3pt}{2pt}
\centering \textit{``A photo of shark" $\to$ ``A photo of \ul{dolphin}"} 
\vspace{-6pt}
\caption{\textbf{Ablation study} on scale $\lambda$. To show the effect of scale more extremely, the number of iterations of FPR is set as 1. The result of DDS is the same as $\lambda=0$ since it means no update for source identity.}
\vspace{-10pt}
\label{fig:ablation_scale}
\end{figure}

% \vspace{-5pt}
%%% [START] Ablation-steps
\begin{figure}[t] % 1-column
\footnotesize
\centering 

\newcommand{\imgwidth}{0.75in}

% 1st row
\hspace{-2.2mm}
\raisebox{0.057in}{\rotatebox{90}{InstructPix2Pix}}%
\hspace{-0.5mm}%
% 1st image with ellipse
\begin{tikzpicture}[x=1cm, y=1cm]
    \node[anchor=south] (FigA1) at (0,0) {
        \includegraphics[width=\imgwidth]{Fig./Qual/imgs/ablation/400steps_woman/original.jpg}
    };
    \node[anchor=south, yshift=-1mm] at (FigA1.north) {\footnotesize Source};
    % ellipse
    \draw[white, thick] (FigA1.center) ++(0.0,0.7) ellipse [x radius=0.23cm, y radius=0.23cm];
    \draw[white, thick] (FigA1.center) ++(-0.9,-0.95) rectangle ++(0.4cm, 0.3cm);
\end{tikzpicture}\hspace{-1.1mm}%
% 2nd image with ellipse
\begin{tikzpicture}[x=1cm, y=1cm]
    \node[anchor=south] (FigD1) at (0,0) {
        % \includegraphics[width=\imgwidth]{Fig./Qual/imgs/ablation/400steps_woman/fpds_2062117338_0_400steps.jpg}
        \includegraphics[width=\imgwidth]{Fig./Qual/imgs/ablation/400steps_woman/06_ids400.jpg}
    };
    \node[anchor=south, yshift=-1mm] at (FigD1.north) {\footnotesize \textbf{IDS (Ours)}};
    % ellipse
    \draw[white, thick] (FigA1.center) ++(0.0,0.7) ellipse [x radius=0.23cm, y radius=0.23cm];
    \draw[white, thick] (FigA1.center) ++(-0.9,-0.95) rectangle ++(0.4cm, 0.3cm);
\end{tikzpicture}\hspace{-1.1mm}%
% 3rd image with ellipse
\begin{tikzpicture}[x=1cm, y=1cm]
    \node[anchor=south] (FigC1) at (0,0) {
        % \includegraphics[width=\imgwidth]{Fig./Qual/imgs/ablation/400steps_woman/cds_2062117338_0_400steps.jpg}
        \includegraphics[width=\imgwidth]{Fig./Qual/imgs/ablation/400steps_woman/04_cds400.jpg}
    };
    \node[anchor=south, yshift=-1mm] at (FigC1.north) {\footnotesize CDS};
    % ellipse
    \draw[white, thick] (FigA1.center) ++(0.0,0.7) ellipse [x radius=0.23cm, y radius=0.23cm];
    \draw[white, thick] (FigA1.center) ++(-0.9,-0.95) rectangle ++(0.4cm, 0.3cm);
\end{tikzpicture}\hspace{-1.1mm}%
% 4th image with ellipse
\begin{tikzpicture}[x=1cm, y=1cm]
    \node[anchor=south] (FigB1) at (0,0) {
        % \includegraphics[width=\imgwidth]{Fig./Qual/imgs/ablation/400steps_woman/dds_2062117338_0_400steps.jpg}
        \includegraphics[width=\imgwidth]{Fig./Qual/imgs/ablation/400steps_woman/02_dds400.jpg}
    };
    \node[anchor=south, yshift=-1mm] at (FigB1.north) {\footnotesize DDS};
    % ellipse
    \draw[white, thick] (FigA1.center) ++(0.0,0.7) ellipse [x radius=0.23cm, y radius=0.23cm];
    \draw[white, thick] (FigA1.center) ++(-0.9,-0.95) rectangle ++(0.4cm, 0.3cm);
\end{tikzpicture}

\vspace{-2.5pt}
% text
\setulcolor{magenta}
\setul{0.3pt}{2pt}
\centering \textit{``Woman holding a staff wearing white clothes" $\to$ ``... \ul{blue} clothes"} 
\vspace{-2.2pt}


% 2nd row
\hspace{-2.2mm}
\raisebox{0.2in}{\rotatebox{90}{LLFF}}%
\hspace{-0.5mm}%
% 1st image
\begin{tikzpicture}[x=1cm, y=1cm]
    \node[anchor=south] (FigA2) at (0,0) {
        \includegraphics[width=\imgwidth]{Fig./Qual/imgs/ablation/400steps_autumn/original.jpg}
    };
    % ellipse
    \draw[white, thick] (FigA1.center) ++(0.5,-0.2) ellipse [x radius=0.45cm, y radius=0.35cm];
\end{tikzpicture}\hspace{-1.1mm}%
% 2nd image
\begin{tikzpicture}[x=1cm, y=1cm]
    \node[anchor=south] (FigD2) at (0,0) {
        \includegraphics[width=\imgwidth]{Fig./Qual/imgs/ablation/400steps_autumn/fpds_autumn_400steps.jpg}
    };
   % ellipse
    \draw[white, thick] (FigA1.center) ++(0.5,-0.2) ellipse [x radius=0.45cm, y radius=0.35cm];
\end{tikzpicture}\hspace{-1.1mm}%
% 3rd image
\begin{tikzpicture}[x=1cm, y=1cm]
    \node[anchor=south] (FigC2) at (0,0) {
        \includegraphics[width=\imgwidth]{Fig./Qual/imgs/ablation/400steps_autumn/cds_autumn_400steps.jpg}
    };
   % ellipse
    \draw[white, thick] (FigA1.center) ++(0.5,-0.2) ellipse [x radius=0.45cm, y radius=0.35cm];
\end{tikzpicture}\hspace{-1.1mm}%
% 4th image
\begin{tikzpicture}[x=1cm, y=1cm]
    \node[anchor=south] (FigB2) at (0,0) {
        \includegraphics[width=\imgwidth]{Fig./Qual/imgs/ablation/400steps_autumn/dds_autumn_400steps.jpg}
    };
    % ellipse
    \draw[white, thick] (FigA1.center) ++(0.5,-0.2) ellipse [x radius=0.45cm, y radius=0.35cm];
\end{tikzpicture}
\vspace{-2.5pt}
% text
\setulcolor{magenta}
\setul{0.3pt}{2pt}
\centering \textit{``The green leaves" $\to$ ``\ul{Yellow and red} leaves in \ul{autumn}"} 
\vspace{-2.2pt}
\vspace{-3pt}
\caption{\textbf{Optimization steps.} Our method effectively preserves the consistency of the source image, even as the number of iteration steps increases up to 400.}
\vspace{-4pt}
\label{fig:ablation_step}
\end{figure}
\begin{table}[t]
\centering
\resizebox{0.98\columnwidth}{!}{
\small{
\begin{tabular}{c|cc|cc|c|c}
\hline
& optim step & FPR iter & LPIPS($\downarrow$) & CLIP($\uparrow)$ & time (sec/img) & Memory (GB)  \\ 
\hline
% DDS & \multirow{2}{*}{200} & \multirow{2}{*}{-} & 0.240 & 0.293 & 22.45 & 6.27 \\
%CDS &                      &                    & 0.210 & 0.287 & 59.31 & 8.83 \\
DDS & 200 & - & 0.240 & 0.293 & 22.45 & 6.27 \\
\hline
CDS &  200 &  -  & 0.210 & 0.287 & 59.31 & 8.83 \\
\hline
\multirow{4}{*}{IDS} & \multirow{2}{*}{200} & 1 & 0.199 & 0.285 & 50.80 & \multirow{4}{*}{8.63} \\
                     &                      & 3 & 0.190 & 0.277 & 107.77 & \\
\cline{2-6}
& 100 & \multirow{2}{*}{3} & 0.165 & 0.265 & 54.04 & \\
& 150 &                    & 0.180 & 0.272 & 81.25 & \\
\hline
\end{tabular}
}
}
\vspace{-5pt}
\caption{\textbf{Computational complexity} on 28 images of InstructPix2Pix \cite{brooks2023instructpix2pix} for various settings. Lower LPIPS and higher CLIP scores mean better quality.}
% \vspace{-3pt}
\label{tab:overhead}
\end{table}
\section{Discussions}
% \subsection{Harmony with Existing SDS}
% IDS can be applied to SDS optimization for a given source image and prompt to preserve the original contents and reduce the blurry effect. As shown in \cref{fig:existingSDS}, the conserved rate of the information of the source image is controllable by the number of iterations of FPR.
% when using IDS to SDS optimization, the original structural information of the source image is maintained by fixing the imperfect score caused by insufficient prompt and random noise $\epsilon$. 
% \vspace{-15pt}
% %%% [START] 6.1
\begin{figure}[t] % 1-column
\footnotesize
\centering 

% Adjust the image width to fit within one column
\newcommand{\imgwidth}{0.235\linewidth} % Set image width to 16% of the line width

% 1st Image
\begin{tikzpicture}[x=1cm, y=1cm]
    \node[anchor=south] (FigA1) at (0,0) {
        \includegraphics[width=\imgwidth]{Fig./Qual/imgs/existingSDS/original.jpg}
    };
    \node[anchor=south, yshift=-1mm] at (FigA1.north) {\footnotesize Source};
\end{tikzpicture}\hspace{-1mm}%
% 2nd Image
\begin{tikzpicture}[x=1cm, y=1cm]
    \node[anchor=south] (FigB1) at (0,0) {
        \includegraphics[width=\imgwidth]{Fig./Qual/imgs/existingSDS/sds.jpg}
    };
    \node[anchor=south, yshift=-1mm] at (FigB1.north) {\footnotesize SDS};
\end{tikzpicture}\hspace{-1mm}%
% % 3rd Image
\begin{tikzpicture}[x=1cm, y=1cm]
    \node[anchor=south] (FigC1) at (0,0) {
        \includegraphics[width=\imgwidth]{Fig./Qual/imgs/existingSDS/fp30.jpg}
    };
    \node[anchor=south, yshift=-1mm] at (FigC1.north) {\footnotesize Iter 30};
\end{tikzpicture}\hspace{-1mm}%
% 4th Image
\begin{tikzpicture}[x=1cm, y=1cm]
    \node[anchor=south] (FigD1) at (0,0) {
        \includegraphics[width=\imgwidth]{Fig./Qual/imgs/existingSDS/fp50.jpg}
    };
    \node[anchor=south, yshift=-1mm] at (FigD1.north) {\footnotesize Iter 50};
\end{tikzpicture}

\vspace{-8pt}
\caption{\textbf{SDS with FPR.} %regularization.} 
Given \textit{(first)} source image and prompt \textit{``a drawing of a cat"}, \textit{(second)} SDS optimization, \textit{(third, fourth)} SDS optimization with FPR for 30 and 50 iterations are applied. Each result uses 200 steps for optimization.}
\vspace{-8pt}
\label{fig:existingSDS}
\end{figure}

% We conduct ablation studies to demonstrate the effectiveness of different fine-tuning strategies within our method.
% %%% [START] Ablation-steps
\begin{figure}[t] % 1-column
\footnotesize
\centering 

\newcommand{\imgwidth}{0.75in}

% 1st row
\hspace{-2.2mm}
\raisebox{0.057in}{\rotatebox{90}{InstructPix2Pix}}%
\hspace{-0.5mm}%
% 1st image with ellipse
\begin{tikzpicture}[x=1cm, y=1cm]
    \node[anchor=south] (FigA1) at (0,0) {
        \includegraphics[width=\imgwidth]{Fig./Qual/imgs/ablation/400steps_woman/original.jpg}
    };
    \node[anchor=south, yshift=-1mm] at (FigA1.north) {\footnotesize Source};
    % ellipse
    \draw[white, thick] (FigA1.center) ++(0.0,0.7) ellipse [x radius=0.23cm, y radius=0.23cm];
    \draw[white, thick] (FigA1.center) ++(-0.9,-0.95) rectangle ++(0.4cm, 0.3cm);
\end{tikzpicture}\hspace{-1.1mm}%
% 2nd image with ellipse
\begin{tikzpicture}[x=1cm, y=1cm]
    \node[anchor=south] (FigD1) at (0,0) {
        % \includegraphics[width=\imgwidth]{Fig./Qual/imgs/ablation/400steps_woman/fpds_2062117338_0_400steps.jpg}
        \includegraphics[width=\imgwidth]{Fig./Qual/imgs/ablation/400steps_woman/06_ids400.jpg}
    };
    \node[anchor=south, yshift=-1mm] at (FigD1.north) {\footnotesize \textbf{IDS (Ours)}};
    % ellipse
    \draw[white, thick] (FigA1.center) ++(0.0,0.7) ellipse [x radius=0.23cm, y radius=0.23cm];
    \draw[white, thick] (FigA1.center) ++(-0.9,-0.95) rectangle ++(0.4cm, 0.3cm);
\end{tikzpicture}\hspace{-1.1mm}%
% 3rd image with ellipse
\begin{tikzpicture}[x=1cm, y=1cm]
    \node[anchor=south] (FigC1) at (0,0) {
        % \includegraphics[width=\imgwidth]{Fig./Qual/imgs/ablation/400steps_woman/cds_2062117338_0_400steps.jpg}
        \includegraphics[width=\imgwidth]{Fig./Qual/imgs/ablation/400steps_woman/04_cds400.jpg}
    };
    \node[anchor=south, yshift=-1mm] at (FigC1.north) {\footnotesize CDS};
    % ellipse
    \draw[white, thick] (FigA1.center) ++(0.0,0.7) ellipse [x radius=0.23cm, y radius=0.23cm];
    \draw[white, thick] (FigA1.center) ++(-0.9,-0.95) rectangle ++(0.4cm, 0.3cm);
\end{tikzpicture}\hspace{-1.1mm}%
% 4th image with ellipse
\begin{tikzpicture}[x=1cm, y=1cm]
    \node[anchor=south] (FigB1) at (0,0) {
        % \includegraphics[width=\imgwidth]{Fig./Qual/imgs/ablation/400steps_woman/dds_2062117338_0_400steps.jpg}
        \includegraphics[width=\imgwidth]{Fig./Qual/imgs/ablation/400steps_woman/02_dds400.jpg}
    };
    \node[anchor=south, yshift=-1mm] at (FigB1.north) {\footnotesize DDS};
    % ellipse
    \draw[white, thick] (FigA1.center) ++(0.0,0.7) ellipse [x radius=0.23cm, y radius=0.23cm];
    \draw[white, thick] (FigA1.center) ++(-0.9,-0.95) rectangle ++(0.4cm, 0.3cm);
\end{tikzpicture}

\vspace{-2.5pt}
% text
\setulcolor{magenta}
\setul{0.3pt}{2pt}
\centering \textit{``Woman holding a staff wearing white clothes" $\to$ ``... \ul{blue} clothes"} 
\vspace{-2.2pt}


% 2nd row
\hspace{-2.2mm}
\raisebox{0.2in}{\rotatebox{90}{LLFF}}%
\hspace{-0.5mm}%
% 1st image
\begin{tikzpicture}[x=1cm, y=1cm]
    \node[anchor=south] (FigA2) at (0,0) {
        \includegraphics[width=\imgwidth]{Fig./Qual/imgs/ablation/400steps_autumn/original.jpg}
    };
    % ellipse
    \draw[white, thick] (FigA1.center) ++(0.5,-0.2) ellipse [x radius=0.45cm, y radius=0.35cm];
\end{tikzpicture}\hspace{-1.1mm}%
% 2nd image
\begin{tikzpicture}[x=1cm, y=1cm]
    \node[anchor=south] (FigD2) at (0,0) {
        \includegraphics[width=\imgwidth]{Fig./Qual/imgs/ablation/400steps_autumn/fpds_autumn_400steps.jpg}
    };
   % ellipse
    \draw[white, thick] (FigA1.center) ++(0.5,-0.2) ellipse [x radius=0.45cm, y radius=0.35cm];
\end{tikzpicture}\hspace{-1.1mm}%
% 3rd image
\begin{tikzpicture}[x=1cm, y=1cm]
    \node[anchor=south] (FigC2) at (0,0) {
        \includegraphics[width=\imgwidth]{Fig./Qual/imgs/ablation/400steps_autumn/cds_autumn_400steps.jpg}
    };
   % ellipse
    \draw[white, thick] (FigA1.center) ++(0.5,-0.2) ellipse [x radius=0.45cm, y radius=0.35cm];
\end{tikzpicture}\hspace{-1.1mm}%
% 4th image
\begin{tikzpicture}[x=1cm, y=1cm]
    \node[anchor=south] (FigB2) at (0,0) {
        \includegraphics[width=\imgwidth]{Fig./Qual/imgs/ablation/400steps_autumn/dds_autumn_400steps.jpg}
    };
    % ellipse
    \draw[white, thick] (FigA1.center) ++(0.5,-0.2) ellipse [x radius=0.45cm, y radius=0.35cm];
\end{tikzpicture}
\vspace{-2.5pt}
% text
\setulcolor{magenta}
\setul{0.3pt}{2pt}
\centering \textit{``The green leaves" $\to$ ``\ul{Yellow and red} leaves in \ul{autumn}"} 
\vspace{-2.2pt}
\vspace{-3pt}
\caption{\textbf{Optimization steps.} Our method effectively preserves the consistency of the source image, even as the number of iteration steps increases up to 400.}
\vspace{-4pt}
\label{fig:ablation_step}
\end{figure}
% \vspace{-15pt}
% \begin{figure}[t] % 1-column
\footnotesize
\centering 

% Adjust the image width to fit within one column
\newcommand{\imgwidth}{0.18\linewidth} % Set image width to 16% of the line width

% 1st Image
\begin{tikzpicture}[x=1cm, y=1cm]
    \node[anchor=south] (FigA1) at (0,0) {
        \includegraphics[width=\imgwidth]{Fig./Qual/imgs/ip2p_sub/shark/src_img.jpg}
    };
    \node[anchor=south, yshift=-1.5mm] at (FigA1.north) {\footnotesize Source};
\end{tikzpicture}\hspace{-1mm}%
% 2nd Image
\begin{tikzpicture}[x=1cm, y=1cm]
    \node[anchor=south] (FigB1) at (0,0) {
        \includegraphics[width=\imgwidth]{Fig./Qual/imgs/ip2p_sub/shark/dds.jpg}
    };
    \node[anchor=south, yshift=-1.5mm] at (FigB1.north) {\footnotesize DDS};
\end{tikzpicture}\hspace{-1mm}%
\begin{tikzpicture}[x=1cm, y=1cm]
    \node[anchor=south] (FigB1) at (0,0) {
        \includegraphics[width=\imgwidth]{Fig./Qual/imgs/ablation/scale/scale_0.4-iter_1-optimstep_200.jpg}
    };
    \node[anchor=south, yshift=-1.5mm] at (FigB1.north) {\footnotesize $\lambda=0.4$};
\end{tikzpicture}\hspace{-1mm}%
% % 3rd Image
\begin{tikzpicture}[x=1cm, y=1cm]
    \node[anchor=south] (FigC1) at (0,0) {
        \includegraphics[width=\imgwidth]{Fig./Qual/imgs/ablation/scale/scale_0.8-iter_1-optimstep_200.jpg}
    };
    \node[anchor=south, yshift=-1.5mm] at (FigC1.north) {\footnotesize $\lambda=0.8$};
\end{tikzpicture}\hspace{-1mm}%
% 4th Image
\begin{tikzpicture}[x=1cm, y=1cm]
    \node[anchor=south] (FigD1) at (0,0) {
        \includegraphics[width=\imgwidth]{Fig./Qual/imgs/ablation/scale/scale_1.0-iter_1-optimstep_200.jpg}
    };
    \node[anchor=south, yshift=-1.5mm] at (FigD1.north) {\footnotesize $\lambda=1.0$};
\end{tikzpicture}
\vspace{-3pt}
% Caption Text
\setulcolor{magenta}
\setul{0.3pt}{2pt}
\centering \textit{``A photo of shark" $\to$ ``A photo of \ul{dolphin}"} 
\vspace{-6pt}
\caption{\textbf{Ablation study} on scale $\lambda$. To show the effect of scale more extremely, the number of iterations of FPR is set as 1. The result of DDS is the same as $\lambda=0$ since it means no update for source identity.}
\vspace{-10pt}
\label{fig:ablation_scale}
\end{figure}


% 6.1 images
\subsection{Ablation studies on FPR}
\noindent\textbf{FPR iteration $N$.} 
% In the proposed FPR, the number of iterations $n$ is one of the hyper-parameters.
%In our method, Fixed-point iteration refers to the number of times the fixed-point process is applied when correcting $\epsilon$ errors. 
We conduct experiments on FPR iterations $N$ to evaluate its impact and determine the optimal iteration count. Although performing just one iteration of FPR is sufficient to preserve the source identity, as shown in lower LPIPS score than baselines of \cref{tab:overhead}, we set $N=3$ to emphasize the purpose of our method.
%We set $n=3$ based on the trade-off between the computational overhead and preserving identity.
%However, for finer details, such as editing the shape of a ``pig nose", a higher iteration count of three or more yields the best results. Considering both execution time and efficiency, we set the Fixed-point iteration count to three in our method.
% \vspace{-11pt}
%This ensures that even with an increased number of inference steps, the structural integrity of the source image is maintained, enabling precise image editing aligned with the target prompt.
% \vspace{-11pt}

\noindent\textbf{Scale $\lambda$.}
The scaling factor $\lambda$ of FPR determines how much information of source latent $\src$ is kept. As shown in \cref{fig:ablation_scale}, increasing the scale preserves the attributes of the source image, resulting in more successful editing when it is hard to translate due to the structural mismatch between the source and the target prompt.
% The scaling factor $\lambda$ of the FPR determines how much the source latent $\srct$ is modified depending on the $d(\src, \src_{0|t})$.
%when calculating gradients based on the posterior mean to correct $\epsilon$ errors. 
% As shown in \cref{fig:ablation_scale}, increasing the scale enhances both the structural and color fidelity of the source image, highlighting the importance of the scale factor in preserving the source image’s structural attributes. %Thus, the scale $\lambda$ is set to 1 in all experiments.
% \vspace{-13pt}

% \vspace{-10pt}
%\paragraph{Optimization steps.}
\subsection{Optimization steps}
To show that our method can prevent error accumulation during translation, we set the experiment to extend the number of optimization steps from 200 to 400. In the results of DDS and CDS, there is color boosting or loss of details due to the cumulated error. In contrast, IDS maintains the characteristics of the original images, such as the color of the pumpkin in the first row of \cref{fig:ablation_step} and the shape of the leaf in the second row of \cref{fig:ablation_step}, better than other methods.
% As shown in \cref{fig:ablation_step}, extending the number of optimization steps from 200 to 400 in DDS \cite{hertz2023delta} and CDS \cite{nam2024contrastive} results in cumulative errors from repeated edits, leading to a degradation in structural consistency. In contrast, the proposed IDS employs FPR to iteratively update $\src_t$ for the fine score $\epsilon_{\phi}^{\text{src}}$, effectively correcting these cumulative errors. Consequently, the capability of our approach to preserve the structural integrity of the source image during editing becomes increasingly evident.
% \vspace{-13pt}

% \paragraph{Inference steps.} As demonstrated in \cref{fig:ablation_step}, the effectiveness of our method in preserving the characteristics of the source image while aligning the edits with the target prompt becomes even more pronounced as the number of inference steps increases. In contrast, when the number of inference steps is extended from 200 to 400 in DDS \cite{hertz2023delta} and CDS \cite{nam2024contrastive}, the cumulative errors arising from the repeated editing process result in a degradation of structural consistency in the source image. This accumulated errors, as depicted in \cref{fig:ablation_step}, compromises the structural attributes of the source image. However, our method leverages a Fixed-point Regularization to iteratively update  $\epsilon$, correcting these cumulative errors. %This ensures that even with an increased number of inference steps, the structural integrity of the source image is maintained, enabling precise image editing aligned with the target prompt.
