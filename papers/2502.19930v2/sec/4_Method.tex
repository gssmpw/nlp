\section{Method}
\label{sec:method}
%%%% [START] Inversion
\begin{figure}[t] % 1-column
\footnotesize
\centering 

% 1st row
\hspace{-2.2mm}
\raisebox{0.24in}{\rotatebox{90}{w/o FPR}}%
\hspace{-1.2mm}
% 1st
\hspace{0mm}
\begin{tikzpicture}[x=4.2cm, y=4.2cm, remember picture, baseline]
    \node[anchor=south, name=labelA] (FigA) at (0,0) {
        \includegraphics[width=0.78in]{Fig./Qual/imgs/posterior_mean/original.png}
    };
    \node[anchor=south, yshift=-1mm] at (FigA.north) {\footnotesize $\textit{Original image} ~\mathbf{z}$};
\end{tikzpicture}\hspace{-1.1mm}%
% 2nd
\begin{tikzpicture}[x=4.2cm, y=4.2cm, remember picture, baseline]
    \node[anchor=south, name=labelB] (FigB) at (0,0) {
        \includegraphics[width=0.78in]{Fig./Qual/imgs/posterior_mean/original_post_mean_step_200.png}
    };
    \node[anchor=south, yshift=-1mm] at (FigB.north) {\footnotesize $\mathbb{E}[\mathbf{z}|\mathbf{z}_{t=200}]$};
\end{tikzpicture}\hspace{-2mm}%
% 3rd
\begin{tikzpicture}[x=4.2cm, y=4.2cm, remember picture, baseline]
    \node[anchor=south, name=labelD] (FigD) at (0,0) {
        \includegraphics[width=0.78in]{Fig./Qual/imgs/posterior_mean/original_post_mean_step_600.png}
    };
    \node[anchor=south, yshift=-1mm] at (FigD.north) {\footnotesize $\mathbb{E}[\mathbf{z}|\mathbf{z}_{t=600}]$};
\end{tikzpicture}\hspace{-2mm}%
% 4th
\begin{tikzpicture}[x=4.2cm, y=4.2cm, remember picture, baseline]
    \node[anchor=south, name=labelF] (FigF) at (0,0) {
        \includegraphics[width=0.78in]{Fig./Qual/imgs/posterior_mean/original_post_mean_step_800.png}
    };
    \node[anchor=south, yshift=-1mm] at (FigF.north) {\footnotesize $\mathbb{E}[\mathbf{z}|\mathbf{z}_{t=800}]$};
\end{tikzpicture}\hspace{-2mm}%

\vspace{-4pt}

% 2st row
\hspace{-2.4mm}
\raisebox{0.12in}{\rotatebox{90}{w/ FPR  w.r.t $\mathbf{z}_t$}}%
\hspace{-1.4mm}
% 1st
\hspace{0mm}
\begin{tikzpicture}[x=4.2cm, y=4.2cm, remember picture, baseline]
    \node[anchor=south, name=labelA1] (FigA1) at (0,0) {
        \includegraphics[width=0.78in]{Fig./Qual/imgs/posterior_mean/original.png}
    };
    % \node[anchor=south, yshift=-1mm] at (FigA.north) {\footnotesize $\textit{original image} \mathbf{z}$};
\end{tikzpicture}\hspace{-1.1mm}%
% 2nd
\begin{tikzpicture}[x=4.2cm, y=4.2cm, remember picture, baseline]
    \node[anchor=south, name=labelB1] (FigB1) at (0,0) {
        \includegraphics[width=0.78in]{Fig./Qual/imgs/posterior_mean/fpr_post_mean_step_200.png}
    };
    % \node[anchor=south, yshift=-1mm] at (FigB.north) {\footnotesize $\mathbb{E}[\mathbf{z}|\mathbf{z}_{t=200}]$};
\end{tikzpicture}\hspace{-2mm}%
% 3rd
\begin{tikzpicture}[x=4.2cm, y=4.2cm, remember picture, baseline]
    \node[anchor=south, name=labelD1] (FigD1) at (0,0) {
        \includegraphics[width=0.78in]{Fig./Qual/imgs/posterior_mean/fpr_post_mean_step_600.png}
    };
    % \node[anchor=south, yshift=-1mm] at (FigD.north) {\footnotesize $\mathbb{E}[\mathbf{z}|\mathbf{z}_{t=600}]$};
\end{tikzpicture}\hspace{-2mm}%
% 4th
\begin{tikzpicture}[x=4.2cm, y=4.2cm, remember picture, baseline]
    \node[anchor=south, name=labelF1] (FigF1) at (0,0) {
        \includegraphics[width=0.78in]{Fig./Qual/imgs/posterior_mean/fpr_post_mean_step_800.png}
    };
    % \node[anchor=south, yshift=-1mm] at (FigF.north) {\footnotesize $\mathbb{E}[\mathbf{z}|\mathbf{z}_{t=800}]$};
\end{tikzpicture}\hspace{-2mm}%

\vspace{-4pt}

% 3rd row
\hspace{-2.2mm}
\raisebox{0.14in}{\rotatebox{90}{w/ FPR w.r.t \ $\epsilon$}}
% \raisebox{0.14in}{\rotatebox{90}{}}%
\hspace{-1.2mm}
% 1st
\hspace{0mm}
\begin{tikzpicture}[x=4.2cm, y=4.2cm, remember picture, baseline]
    \node[anchor=south, name=labelA2] (FigA2) at (0,0) {
        \includegraphics[width=0.78in]{Fig./Qual/imgs/posterior_mean/original.png}
    };
    % \node[anchor=south, yshift=-1mm] at (FigA.north) {\footnotesize $\textit{original image} \mathbf{z}$};
\end{tikzpicture}\hspace{-1.1mm}%
% 2nd
\begin{tikzpicture}[x=4.2cm, y=4.2cm, remember picture, baseline]
    \node[anchor=south, name=labelB2] (FigB2) at (0,0) {
        \includegraphics[width=0.78in]{Fig./Qual/imgs/posterior_mean/fpr_post_mean_step_200_eps.png}
    };
    % \node[anchor=south, yshift=-1mm] at (FigB.north) {\footnotesize $\mathbb{E}[\mathbf{z}|\mathbf{z}_{t=200}]$};
\end{tikzpicture}\hspace{-2mm}%
% 3rd
\begin{tikzpicture}[x=4.2cm, y=4.2cm, remember picture, baseline]
    \node[anchor=south, name=labelD2] (FigD2) at (0,0) {
        \includegraphics[width=0.78in]{Fig./Qual/imgs/posterior_mean/fpr_post_mean_step_600_eps.png}
    };
    % \node[anchor=south, yshift=-1mm] at (FigD.north) {\footnotesize $\mathbb{E}[\mathbf{z}|\mathbf{z}_{t=600}]$};
\end{tikzpicture}\hspace{-2mm}%
% 4th
\begin{tikzpicture}[x=4.2cm, y=4.2cm, remember picture, baseline]
    \node[anchor=south, name=labelF2] (FigF2) at (0,0) {
        \includegraphics[width=0.78in]{Fig./Qual/imgs/posterior_mean/fpr_post_mean_step_800_eps.png}
    };
    % \node[anchor=south, yshift=-1mm] at (FigF.north) {\footnotesize $\mathbb{E}[\mathbf{z}|\mathbf{z}_{t=800}]$};
\end{tikzpicture}\hspace{-2mm}%


% \vspace{-3pt}
\vspace{-10pt}
\caption{\textbf{Posterior mean with/without FPR.} When the prompt $y$ is given by ``portrait of a worried-looking woman in a dress”, the posterior mean $\mathbf{z}_{0|t}$ is obtained $(first$ $row)$ without FPR, $(second$ $row)$ with FPR w.r.t $\mathbf{z}_t$, and $(third$ $row)$ with FPR w.r.t $\epsilon$.}

\label{fig:post_mean3}
\vspace{-10pt}
\end{figure}


Given the source pair $\{\src, \txts \}$, the aim of our work is to provide an edited result $\trg$ that is aligned with $\txtt$ while maintaining the source's identity. To this end, we introduce a novel approach called \textbf{Identity-preserving Distillation Sampling (IDS)}, which (1) corrects the error of the gradient aligned with the text prompt by the fixed-point iterator and (2) provides the result $\trg$ using the guided noise.%$\epsilon^\ast$. %instead of random Gaussian noise for identity preservation.

%We begin with an interpretation of the SDS \cite{poole2022dreamfusion} and DDS \cite{hertz2023delta} loss function from a new perspective. 
%Although DDS \cite{hertz2023delta} has been proposed to obtain the optimal text-aligned score for image editing, structural consistency often cannot be maintained. 

%The text-conditioned score may lead to a misalignment with the identity of the given image $\src$, leading to a significant change in the overall structure and characteristics when the error is accumulated. To address this issue, we introduce a novel approach called \textbf{Identity-preserving Distillation Sampling (IDS)}, which corrects the error of the direction aligned with the text prompt by the fixed-point iterator. 

%Our method not only ensures the structural consistency of objects, but also preserves features such as color and texture. 
% when applied to image editing.

%In this section, we review and interpret DDS loss from a novel perspective in section \ref{ssec:3.1}. Based on this interpretation, we analyze the problem of what part of DDS loss is causing the error during image editing in section \ref{ssec:3.2}. Then, we introduce our method, Fixed-point Iteration for Denoising Score (FPDS), to recover this error in section \ref{ssec:3.3}.

%------------------------------------------------------------------------
% \vspace{-7mm}
\subsection{Motivation} \label{ssec:4.1}
%Therefore, the result only follows the target prompt $y_{trg}$ without preserving the characteristics of $\trg_{src}$ as shown in Fig. \ref{fig:inversion}. Is there the \textbf{score to capture the features of the input image} that should be consistent between the source and result?

%However, when editing the images, $\epsilon_\phi(\trg_t, y, t)$ generates denoising score in various directions corresponding to $y$ at each optimizing step.  

%\subsection{Problem of DDS for image editing} \label{ssec:4.2}

\textbf{Analysis of the text-conditioned score.} \ We first investigated how much identity of the given image $\src$ could be contained in the text-conditioned score ${\epsilon}_{\phi}^{\text{src}}$. To do this, we conducted the experiment to compare the original image $\src$ and the posterior mean $\src_{0|t} = \mathbb{E}[\src | \src_t]$, which is given by:
{\small
\begin{equation}
\src_{0|t} = \frac{1}{\sqrt{\alpha_{t}}}\left(
    \src_t -\sqrt{1-\alpha_{t}} {\epsilon}_{\phi}^{\text{src}}
    \right),
\label{eq:posterior}
\end{equation}
}
where $\src_t$ denotes the source latent  generated by \eqref{eq:forward}.
% As shown in the first row of Fig. \ref{fig:post_mean3}, it is difficult to recognize the features of $\src$ in $\src_{0|t}$, such as hairstyle, details of eyes, and background. This demonstrates that the score ${\epsilon}_{\phi}^{\text{src}}$ is not exactly adjusted to the given image $\src$. This deformation becomes more pronounced with increasing $t$.
As shown in the first row of the supplementary \cref{fig:post_mean3} %Fig. S1, 
it is difficult to recognize the features of $\src$ in $\src_{0|t}$, such as hairstyle, details of eyes, and background. This demonstrates that the score ${\epsilon}_{\phi}^{\text{src}}$ is not exactly adjusted to the given image $\src$. This deformation becomes more pronounced with increasing $t$.
The experiment confirms that ${\epsilon}_{\phi}^{\text{src}}$ may not be a precise guidance to the source image $\src$. %as shown in Fig. \ref{fig:algorithm}.
Therefore, the text-conditioned score ${\epsilon}_{\phi}^{\text{src}}$ needs to be modified to maintain the identity of the source image $\src$ in the edited result $\trg$.

%%% [START] Inversion
\begin{figure}[t] % 1-column
\footnotesize
\centering 

% 1st row
\hspace{-3.2mm}
\raisebox{0.2in}{\rotatebox{90}{$\text{DDS}_{t\sim\mathcal{U}(0, 0.2)}$}}%
\hspace{-1.2mm}
% 1st
\hspace{0mm}\begin{tikzpicture}[x=4.2cm, y=4.2cm, remember picture, baseline]
    \node[anchor=south] (FigA) at (0,0) {
        \includegraphics[width=1in]{Fig./Qual/imgs/inversion/src_img.jpg}
    };
    \node[anchor=south, yshift=-1mm, name=labelA] at (FigA.north) {\footnotesize $\boldsymbol{\mathbf{z}^{\text{src}}}$};
\end{tikzpicture}\hspace{-1.1mm}%
% 2nd
\begin{tikzpicture}[x=4.2cm, y=4.2cm, remember picture, baseline]
    \node[anchor=south] (FigB) at (0,0) {
        \includegraphics[width=1in]{Fig./Qual/imgs/inversion/dds_forward_final_t200.jpg}
    };
    \node[anchor=south, yshift=-1mm, name=labelB] at (FigB.north) {\footnotesize $\boldsymbol{\mathbf{z}^{\text{trg}}}$};
\end{tikzpicture}\hspace{-1.1mm}%
% 3rd
\begin{tikzpicture}[x=4.2cm, y=4.2cm, remember picture, baseline]
    \node[anchor=south] (FigC) at (0,0) {
        \includegraphics[width=1in]{Fig./Qual/imgs/inversion/dds_inversion_final_t200.jpg}
    };
    \node[anchor=south, yshift=-1mm, name=labelC] at (FigC.north) {\footnotesize $\boldsymbol{\mathbf{z}^{\text{src}\dagger}}$};
\end{tikzpicture}\hspace{-1.1mm}%

% ->
\begin{tikzpicture}[remember picture, overlay]
    \draw[->, line width=0.5pt] ([yshift=1mm]labelA.north) -- ([yshift=1mm]labelC.north);
\end{tikzpicture}

\vspace{-14pt}

% % 2st row
% \hspace{-3.2mm}
% \raisebox{0.2in}{\rotatebox{90}{$\text{DDS}_{t\sim\mathcal{U}(0, 0.2)}$}}%
% \hspace{-1.2mm}
% % 1st
% \hspace{0mm}\begin{tikzpicture}[x=4.2cm, y=4.2cm, remember picture, baseline]
%     \node[anchor=south] (FigA3) at (0,0) {
%         \includegraphics[width=1in]{Fig./Qual/imgs/inversion/src_img.png}
%     };
%     % \node[anchor=south, yshift=-1mm, name=labelA] at (FigA2.north) {\footnotesize $\boldsymbol{\hat{z}}$};
% \end{tikzpicture}\hspace{-1.1mm}%
% % 2nd
% \begin{tikzpicture}[x=4.2cm, y=4.2cm, remember picture, baseline]
%     \node[anchor=south] (FigB3) at (0,0) {
%         \includegraphics[width=1in]{Fig./Qual/imgs/inversion/dds_forward_final_t200.png}
%     };
%     % \node[anchor=south, yshift=-1mm, name=labelB] at (FigB2.north) {\footnotesize $\boldsymbol{{z}}$};
% \end{tikzpicture}\hspace{-1.1mm}%
% % 3rd
% \begin{tikzpicture}[x=4.2cm, y=4.2cm, remember picture, baseline]
%     \node[anchor=south] (FigC3) at (0,0) {
%         \includegraphics[width=1in]{Fig./Qual/imgs/inversion/dds_inversion_final_t200.png}
%     };
%     % \node[anchor=south, yshift=-1mm, name=labelC] at (FigC2.north) {\footnotesize $\boldsymbol{\hat{z}^{\dagger}}$};
% \end{tikzpicture}\hspace{-1.1mm}%


% \vspace{-4pt}

% 3rd row
\hspace{-3.3mm}
\raisebox{0.2in}{\rotatebox{90}{$\text{DDS}_{t\sim\mathcal{U}(0, 1)}$}}%
\hspace{-1.2mm}
% 1st
\hspace{0mm}\begin{tikzpicture}[x=4.2cm, y=4.2cm, remember picture, baseline]
    \node[anchor=south] (FigA2) at (0,0) {
        \includegraphics[width=1in]{Fig./Qual/imgs/inversion/src_img.jpg}
    };
    % \node[anchor=south, yshift=-1mm, name=labelA] at (FigA2.north) {\footnotesize $\boldsymbol{\hat{z}}$};
\end{tikzpicture}\hspace{-1.1mm}%
% 2nd
\begin{tikzpicture}[x=4.2cm, y=4.2cm, remember picture, baseline]
    \node[anchor=south] (FigB2) at (0,0) {
        \includegraphics[width=1in]{Fig./Qual/imgs/inversion/dds_forward_final.jpg}
    };
    % \node[anchor=south, yshift=-1mm, name=labelB] at (FigB2.north) {\footnotesize $\boldsymbol{{z}}$};
\end{tikzpicture}\hspace{-1.1mm}%
% 3rd
\begin{tikzpicture}[x=4.2cm, y=4.2cm, remember picture, baseline]
    \node[anchor=south] (FigC2) at (0,0) {
        \includegraphics[width=1in]{Fig./Qual/imgs/inversion/dds_inversion_final.jpg}
    };
    % \node[anchor=south, yshift=-1mm, name=labelC] at (FigC2.north) {\footnotesize $\boldsymbol{\hat{z}^{\dagger}}$};
\end{tikzpicture}\hspace{-1.1mm}%

% text
\vspace{-4pt}
\setulcolor{magenta}
\setul{0.3pt}{2pt}
% 4th row
\hspace{-3.4mm}
\raisebox{0.25in}{\rotatebox{90}{\textbf{IDS (Ours)}}}%
\hspace{-1.2mm}
% 1st
\hspace{0mm}\begin{tikzpicture}[x=4.2cm, y=4.2cm, remember picture, baseline]
    \node[anchor=south] (FigA4) at (0,0) {
        \includegraphics[width=1in]{Fig./Qual/imgs/inversion/src_img.jpg}
    };
    \node[anchor=south, yshift=-3mm, name=labelA4] at (FigA4.south) {``\textit{{\tiny A} cat {\tiny sitting next to a mirror}}"};
\end{tikzpicture}\hspace{-1.1mm}%
% 2nd
\begin{tikzpicture}[x=4.2cm, y=4.2cm, remember picture, baseline]
    \node[anchor=south] (FigB4) at (0,0) {
        \includegraphics[width=1in]{Fig./Qual/imgs/inversion/fpds_forward_final.jpg}
    };
    \node[anchor=south, yshift=-3mm, name=labelB4] at (FigB4.south) {``\textit{{\tiny A} \ul{pig} ...}"};
\end{tikzpicture}\hspace{-1.1mm}%
% 3rd
\begin{tikzpicture}[x=4.2cm, y=4.2cm, remember picture, baseline]
    \node[anchor=south] (FigC4) at (0,0) {
        \includegraphics[width=1in]{Fig./Qual/imgs/inversion/fpds_inversion_final.jpg}
    };
    \node[anchor=south, yshift=-3mm, name=labelC4] at (FigC4.south) {``\textit{{\tiny A} \ul{cat} ...}"};
\end{tikzpicture}\hspace{-1.1mm}

\vspace{-3pt}
\vspace{-5pt}
\caption{\textbf{Accumulated error in DDS.} $\mathbf{z}^{\text{trg}}$ is edited image of source image $\mathbf{z}^{\text{src}}$ by prompt $y^{\text{src}} \rightarrow y^{\text{trg}}$. $\mathbf{z}^{\text{src} \dagger}$ is the inverted image of $\mathbf{z}^{\text{trg}}$ by prompt $y^{\text{trg}} \rightarrow y^{\text{src}}$. \textit{(First row)} Inversion result of DDS with timestep $t\sim\mathcal{U}(0, 0.2)$. \textit{(Second row)} Inversion result of DDS with $t\sim\mathcal{U}(0, 1)$. \textit{(Third row)} Inversion result of ours.}
\label{fig:inversion}
\vspace{-6pt}
\end{figure}


%even if this gradient is calculated from the score $\hat{\epsilon}_{\phi}$ aligned with the prompt $\hat{y}$ that represents the given image $\src$. 
% \vspace{-10pt}
\noindent\textbf{Accumulated error in DDS.} \ The transformed image $\trg$ can be converted back to the original image $\src$ by reversing the set of $\epsilon$ used to synthesize $\trg$ from $\src$ and swapping $\{\src, \txts \}$ and $\{\trg, \txtt \}$ to calculate the DDS loss in \eqref{eq:dds}.
If the guidance from $\src$ to $\trg$ is computed exactly, the perfect reconstruction can be achieved. 
%According to our interpretation, the DDS algorithm is invertible, i.e., it is possible to reconstruct $\src$ from $\trg$, a transformed output of $\src$. 
Nevertheless, as can be seen from the second row in Fig. \ref{fig:inversion}, DDS \cite{hertz2023delta} fails to restore the original image $\src$ from the edited image $\trg$, which implies that the direction from $\src$ to $\trg$ is calculated incorrectly.
Based on our analysis, this error is because the text-conditioned score ${\epsilon}_\phi^{\text{src}}$ do not refer to the source $\src$, which can be explictly expressed as the difference between the injected noise $\epsilon$ and the score ${\epsilon}_\phi^{\text{src}}$. While the optimization is being processed, the error inevitably accumulates, leading to the undesirable change to the structure and the pose. 
% To address these issues, we investigated whether the guidance from $\src$ to $\trg$ can be properly provided while preserving the source's identity, when the timestep $t$ is constrained by $t\sim\mathcal{U}(0, 0.2)$. This is because the posterior mean $\src_{0|t}$ and the source image $\src$ are similar for small timestep $t$, as illustrated in the first row of Fig. \ref{fig:post_mean3}. However, as depicted in the first row of Fig. \ref{fig:inversion}, DDS yields unrealistic result with this setting, whereby the structure of the given image $\src$ is overemphasized. This implies that it is not sufficient to simply limit the timestep $t$ to prevent the score from deviating too far from $\src$ to correct the misalignment of the score to $\src$.
To address these issues, we investigated whether the guidance from $\src$ to $\trg$ can be properly provided while preserving the source's identity, when the timestep $t$ is constrained by $t\sim\mathcal{U}(0, 0.2)$. This is because the posterior mean $\src_{0|t}$ and the source image $\src$ are similar for small timestep $t$, as illustrated in the first row of supplementary \cref{fig:post_mean3}. %Fig. S1. 
However, as depicted in the first row of Fig. \ref{fig:inversion}, DDS yields unrealistic result with this setting, whereby the structure of the given image $\src$ is overemphasized. This implies that it is not sufficient to simply limit the timestep $t$ to prevent the score from deviating too far from $\src$ to correct the misalignment of the score to $\src$.
Hence, we propose a fundamental approach to refine the gradient to achieve identity consistency without unwanted overemphasis on details.

%To accurately calculate the direction from $\src$ to $\trg$, the direction of estimated denoising score $\grad{\theta}\hat{\epsilon}_\phi$ should be equal to the direction from $\src_t$ to $\src$ that is determined by $\epsilon$. 
%In DDS, however, $\hat{\epsilon}_\phi$ is not equal to $\epsilon$ since $\epsilon$ is randomly sampled value, and $\hat{\epsilon}_\phi$ is denoising score when $\src_t$ corresponds to the prompt $\hat{y}$. The difference between $\hat{\epsilon}_\phi$ and $\epsilon$ makes misalignment for the direction of $\grad{\theta}\loss{\dds}$, and this error accumulates during the optimization process. Although DDS shows meaningful performance for image-to-image translation, this cumulated error causes DDS to edit the background or structure incorrectly. %manifold fig?
% $\trg$ is the result edited from source image $\src$ and prompt $\hat{y}$ to target prompt $y$, and $\src^{\dagger}$ is the inverted result that transforms the edited image $\trg$ for prompt $y$ to previous source prompt $\hat{y}$. 
% However, as shown in Fig. \ref{fig:inversion}, the inversion $\src^{\dagger}$ of DDS is not same as the given source image due to the misalignment between $\hat{\epsilon}_\phi$ and $\epsilon$ which is used to make the input of $\epsilon_\phi$.

% In Fig. \ref{fig:inversion}, 
% That is, the reverse of \ref{eq:forward} should be similar with $\src$:
% \begin{equation}
%     \postmean \approx \frac{1}{\sqrt{\alpha_t}}(\src_t - \sqrt{1-\alpha_t} \hat{\epsilon}_\phi)
% \end{equation}


\subsection{Identity-preserving Distillation Sampling (IDS)} \label{ssec:4.2}

\textbf{Fixed-point Regularization (FPR).} \
Here, we introduce a \textbf{F}ixed-\textbf{p}oint \textbf{R}egularization (FPR) method that adjusts the text-conditioned score ${\epsilon}_\phi^{\text{src}}$ to the source image $\src$. Our key premise is that if the score ${\epsilon}_\phi^{\text{src}}$ is rightly estimated as a gradient to $\src$, the posterior mean $\src_{0|t}$ also contains sufficient information about $\src$. %, resulting in identity preservation.
Therefore, FPR loss is designed to minimize the difference between $\src$ and $\src_{0|t}$ as follows:
\begin{equation}
\mathcal{L}_{\text{FPR}} = d ( \mathbf{z}^{\text{src}}, \mathbf{z}_{0|t}^\text{src}),
\label{eq:fpr}
\end{equation}
%\begin{equation}
%    \mathcal{L}_{\text{FPR}}
%    =\lVert \mathbf{z}_{0|t}^\text{src} - \mathbf{z}^{\text{src}} \rVert _{2}^{2},  
%\end{equation}
where $d(\mathbf{x}_1, \mathbf{x}_2)$ can be any metric to compare $\mathbf{x}_1$ and $\mathbf{x}_2$. Here, we employed the Euclidean loss, and further investigations using various metrics are provided in \cref{sec:s_metricsforfpr} of Supplementary Materials.

The score ${\epsilon}_\phi^{\text{src}}$ needs to be modified to minimize the FPR loss before obtaining the updated direction. There are two ways to control the score  ${\epsilon}_\phi^{\text{src}}$ by altering the injection noise $\epsilon$ or the source latent $\mathbf{z}^{\text{src}}_t$. 
% As illustrated in Fig.~\ref{fig:post_mean3}, the proposed FPR revises the score ${\epsilon}_\phi^{\text{src}}$ to serve the source's identity for both approaches. Note that the score incorporates the content details, with the updates being performed with respect to the source latent $\mathbf{z}^{\text{src}}_t$ compared to the noise $\epsilon$. Thus, $\mathbf{z}_t^{\text{src}}$ is updated to minimize the FPR loss as follows:
As illustrated in supplementary \cref{fig:post_mean3}, %Fig. S1, 
the proposed FPR revises the score ${\epsilon}_\phi^{\text{src}}$ to serve the source's identity for both approaches. Note that the score incorporates the content details, with the updates being performed with respect to the source latent $\mathbf{z}^{\text{src}}_t$ compared to the noise $\epsilon$. Thus, $\mathbf{z}_t^{\text{src}}$ is updated to minimize the FPR loss as follows:
\begin{equation}\label{eq:update}
    \mathbf{z}^{\text{src}}_t \leftarrow \mathbf{z}_{t}^{\text{src}} - \lambda \nabla_{\mathbf{z}^{\text{src}}_t}\mathcal{L}_{\text{FPR}},
\end{equation}
where $\lambda$ and $N$ denote a regularization scale and the number of iterations, respectively. 

\newcommand{\bfz}{\mathbf{z}}
\newcommand{\hbfz}{\hat{\mathbf{z}}}
\begin{algorithm}
\caption{Fixed-point Regularization (FPR)}
\label{alg:fpr}
\begin{algorithmic}[1]
%\REQUIRE source image $\mathbf z^{\text{src}}$, source prompt ${y}^{\text{src}}$, timestep $t$, diffusion model $\epsilon_{\phi}$, scale (CFG)  $w$, scale (IDS) $a$
%\ENSURE Output $\epsilon^{\ast}$ %target image $\bfz$

\REQUIRE $\src$, $\txts$, $\epsilon_{\phi}$, $\omega$, $\lambda$, $N$
%\ENSURE Output $\epsilon^{\ast}$ %target image $\bfz$

\STATE $\epsilon \sim \mathcal{N}(0, \mathbf{I})$
\STATE $t \sim\mathcal{U}(0, 1)$
\STATE $\mathbf z^{\text{src}}_t \leftarrow \sqrt{\alpha_t}\mathbf z^{\text{src}}+\sqrt{1-\alpha_t}\epsilon$
\FOR{i = 1, $\dots$, N}
    \STATE $\epsilon_\phi^\text{src} \leftarrow 
    (1+\omega)\epsilon_\phi(\mathbf z^{\text{src}}_{t}, y^{\text{src}}, t)
    - \omega \epsilon_\phi(\mathbf z^{\text{src}}_t, \varnothing, t)$
    \STATE $\bfz_{0|t}^{\text{src}} \leftarrow \frac{1}{\sqrt{\alpha_t}}
    (\mathbf z^{\text{src}}_t - \sqrt{1-\alpha_t}\epsilon_\phi^\text{src})$
    \STATE $\mathcal{L}_{\text{FPR}} \leftarrow 
     d(\src_{0|t}, \src)$
    \STATE $\mathbf z^{\text{src}}_t \leftarrow \mathbf z^{\text{src}}_t - \lambda \nabla_{\mathbf z^{\text{src}}_t}\mathcal{L}_{\text{FPR}}$
\ENDFOR

\STATE $\epsilon^\ast \leftarrow 
\frac{1}{\sqrt{1-\alpha_t}}(\mathbf z^{\text{src}}_t - \sqrt{\alpha_t}\mathbf z^{\text{src}})$
% \STATE $\bfz \leftarrow \text{DDS}(\hbfz, \hat{y}, y, \epsilon^{\ast})$
\RETURN $\epsilon^\ast$
\end{algorithmic}
\end{algorithm}
% \vspace{-6pt}
%Figure 5.1
%%% [START] NeRF Synthetic data Results 
\begin{figure*}[t] % 2-column
\footnotesize
\centering 

% 1st row
\hspace{0mm}\begin{tikzpicture}[x=3.5cm, y=3.5cm]
\node[anchor=south] (FigA) at (0,0) {\includegraphics[width=1.1in]{Fig./Qual/imgs/ip2p_sub/bike/src_img.jpg}};
\node[anchor=south, yshift=0mm] at (FigA.north) {\footnotesize Source};
\end{tikzpicture}\hspace{-1.1mm}%
\begin{tikzpicture}[x=3.5cm, y=3.5cm]
\node[anchor=south] (FigB) at (0,0) {\includegraphics[width=1.1in]{Fig./Qual/imgs/ip2p_sub/bike/fpds.jpg}};
\node[anchor=south, yshift=0mm] at (FigB.north) {\footnotesize \textbf{IDS (Ours)}};
\end{tikzpicture}\hspace{-1.1mm}%
\begin{tikzpicture}[x=3.5cm, y=3.5cm]
\node[anchor=south] (FigC) at (0,0) {\includegraphics[width=1.1in]{Fig./Qual/imgs/ip2p_sub/bike/cds.jpg}};
\node[anchor=south, yshift=0mm] at (FigC.north) {\footnotesize CDS};
\end{tikzpicture}\hspace{-1.1mm}%
\begin{tikzpicture}[x=3.5cm, y=3.5cm]
\node[anchor=south] (FigE) at (0,0) {\includegraphics[width=1.1in]{Fig./Qual/imgs/ip2p_sub/bike/dds.jpg}};
\node[anchor=south, yshift=0mm] at (FigE.north) {\footnotesize DDS};
\end{tikzpicture}\hspace{-1.1mm}%
\begin{tikzpicture}[x=3.5cm, y=3.5cm]
\node[anchor=south] (FigD) at (0,0) {\includegraphics[width=1.1in]{Fig./Qual/imgs/ip2p_sub/bike/pnp.jpg}};
\node[anchor=south, yshift=0mm] at (FigD.north) {\footnotesize PnP};
\end{tikzpicture}\hspace{-1.1mm}%
\begin{tikzpicture}[x=3.5cm, y=3.5cm]
\node[anchor=south] (FigF) at (0,0) {\includegraphics[width=1.1in]{Fig./Qual/imgs/ip2p_sub/bike/p2p.jpg}};
\node[anchor=south, yshift=0mm] at (FigF.north) {\footnotesize P2P};
\end{tikzpicture}

\vspace{-3pt}
\setulcolor{magenta}
\setul{0.3pt}{2pt}
\centering \textit{``Bicycle" $\to$ ``\ul{Neon BMX} bicycle"} 
\vspace{-3pt}

% 2nd row
\hspace{0mm}\begin{tikzpicture}[x=3.5cm, y=3.5cm]
\node[anchor=south] (FigA2) at (0,0) {\includegraphics[width=1.1in]{Fig./Qual/imgs/ip2p_sub/shark/src_img.jpg}};
\end{tikzpicture}\hspace{-1.1mm}%
\begin{tikzpicture}[x=3.5cm, y=3.5cm]
\node[anchor=south] (FigB2) at (0,0) {\includegraphics[width=1.1in]{Fig./Qual/imgs/ip2p_sub/shark/ours.jpg}};
\end{tikzpicture}\hspace{-1.1mm}%
\begin{tikzpicture}[x=3.5cm, y=3.5cm]
\node[anchor=south] (FigC2) at (0,0) {\includegraphics[width=1.1in]{Fig./Qual/imgs/ip2p_sub/shark/cds.jpg}};
\end{tikzpicture}\hspace{-1.1mm}%
\begin{tikzpicture}[x=3.5cm, y=3.5cm]
\node[anchor=south] (FigE2) at (0,0) {\includegraphics[width=1.1in]{Fig./Qual/imgs/ip2p_sub/shark/dds.jpg}};
\end{tikzpicture}\hspace{-1.1mm}% 
\begin{tikzpicture}[x=3.5cm, y=3.5cm]
\node[anchor=south] (FigD2) at (0,0) {\includegraphics[width=1.1in]{Fig./Qual/imgs/ip2p_sub/shark/pnp.jpg}};
\end{tikzpicture}\hspace{-1.1mm}%
\begin{tikzpicture}[x=3.5cm, y=3.5cm]
\node[anchor=south] (FigF2) at (0,0) {\includegraphics[width=1.1in]{Fig./Qual/imgs/ip2p_sub/shark/p2p.jpg}};
\end{tikzpicture}

\vspace{-3pt}
\setulcolor{magenta}
\setul{0.3pt}{2pt}
\centering \textit{``A photo of shark" $\to$ ``A photo of \ul{dolphin}"} 
\vspace{-3pt}

% 3rd row
\hspace{0mm}\begin{tikzpicture}[x=3.5cm, y=3.5cm]
\node[anchor=south] (FigA4) at (0,0) {\includegraphics[width=1.1in]{Fig./Qual/imgs/ip2p_sub/goat2polarbear/original.jpg}};
\end{tikzpicture}\hspace{-1.1mm}%
\begin{tikzpicture}[x=3.5cm, y=3.5cm]
\node[anchor=south] (FigB4) at (0,0) {\includegraphics[width=1.1in]{Fig./Qual/imgs/ip2p_sub/goat2polarbear/fpds.jpg}};
\end{tikzpicture}\hspace{-1.1mm}%
\begin{tikzpicture}[x=3.5cm, y=3.5cm]
\node[anchor=south] (FigC4) at (0,0) {\includegraphics[width=1.1in]{Fig./Qual/imgs/ip2p_sub/goat2polarbear/cds.jpg}};
\end{tikzpicture}\hspace{-1.1mm}%
\begin{tikzpicture}[x=3.5cm, y=3.5cm]
\node[anchor=south] (FigE4) at (0,0) {\includegraphics[width=1.1in]{Fig./Qual/imgs/ip2p_sub/goat2polarbear/dds.jpg}};
\end{tikzpicture}\hspace{-1.1mm}%
\begin{tikzpicture}[x=3.5cm, y=3.5cm]
\node[anchor=south] (FigD4) at (0,0) {\includegraphics[width=1.1in]{Fig./Qual/imgs/ip2p_sub/goat2polarbear/pnp.jpg}};
\end{tikzpicture}\hspace{-1.1mm}%
\begin{tikzpicture}[x=3.5cm, y=3.5cm]
\node[anchor=south] (FigF4) at (0,0) {\includegraphics[width=1.1in]{Fig./Qual/imgs/ip2p_sub/goat2polarbear/p2p.jpg}};
\end{tikzpicture}

\vspace{-3pt}
\setulcolor{magenta}
\setul{0.3pt}{2pt}
\centering \textit{``Oil painting of a standing girl holding a goat" $\to$ ``... \ul{a polar bear}"} 
\vspace{-3pt}

% 4th row
\hspace{0mm}\begin{tikzpicture}[x=3.5cm, y=3.5cm]
\node[anchor=south] (FigA5) at (0,0) {\includegraphics[width=1.1in]{Fig./Qual/imgs/ip2p_sub/telephone2bird/original.jpg}};
\end{tikzpicture}\hspace{-1.1mm}%
\begin{tikzpicture}[x=3.5cm, y=3.5cm]
\node[anchor=south] (FigB5) at (0,0) {\includegraphics[width=1.1in]{Fig./Qual/imgs/ip2p_sub/telephone2bird/fpds.jpg}};
\end{tikzpicture}\hspace{-1.1mm}%
\begin{tikzpicture}[x=3.5cm, y=3.5cm]
\node[anchor=south] (FigC5) at (0,0) {\includegraphics[width=1.1in]{Fig./Qual/imgs/ip2p_sub/telephone2bird/cds.jpg}};
\end{tikzpicture}\hspace{-1.1mm}%
\begin{tikzpicture}[x=3.5cm, y=3.5cm]
\node[anchor=south] (FigE5) at (0,0) {\includegraphics[width=1.1in]{Fig./Qual/imgs/ip2p_sub/telephone2bird/dds.jpg}};
\end{tikzpicture}\hspace{-1.1mm}%
\begin{tikzpicture}[x=3.5cm, y=3.5cm]
\node[anchor=south] (FigD5) at (0,0) {\includegraphics[width=1.1in]{Fig./Qual/imgs/ip2p_sub/telephone2bird/pnp.jpg}};
\end{tikzpicture}\hspace{-1.1mm}%
\begin{tikzpicture}[x=3.5cm, y=3.5cm]
\node[anchor=south] (FigF5) at (0,0) {\includegraphics[width=1.1in]{Fig./Qual/imgs/ip2p_sub/telephone2bird/p2p.jpg}};
\end{tikzpicture}

\vspace{-3pt}
\setulcolor{magenta}
\setul{0.3pt}{2pt}
\centering \textit{``Photo of a smiling woman holding a telephone" $\to$ `... \ul{a black bird}"} 
\vspace{-3pt}

% 5th row
\hspace{0mm}\begin{tikzpicture}[x=3.5cm, y=3.5cm]
\node[anchor=south] (FigA4) at (0,0) {\includegraphics[width=1.1in]{Fig./Qual/imgs/ip2p_sub/fruits/fruits.jpg}};
\end{tikzpicture}\hspace{-1.1mm}%
\begin{tikzpicture}[x=3.5cm, y=3.5cm]
\node[anchor=south] (FigB4) at (0,0) {\includegraphics[width=1.1in]{Fig./Qual/imgs/ip2p_sub/fruits/fpds.jpg}};
\end{tikzpicture}\hspace{-1.1mm}%
\begin{tikzpicture}[x=3.5cm, y=3.5cm]
\node[anchor=south] (FigC4) at (0,0) {\includegraphics[width=1.1in]{Fig./Qual/imgs/ip2p_sub/fruits/cds.jpg}};
\end{tikzpicture}\hspace{-1.1mm}%
\begin{tikzpicture}[x=3.5cm, y=3.5cm]
\node[anchor=south] (FigE4) at (0,0) {\includegraphics[width=1.1in]{Fig./Qual/imgs/ip2p_sub/fruits/dds.jpg}};
\end{tikzpicture}\hspace{-1.1mm}%
\begin{tikzpicture}[x=3.5cm, y=3.5cm]
\node[anchor=south] (FigD4) at (0,0) {\includegraphics[width=1.1in]{Fig./Qual/imgs/ip2p_sub/fruits/pnp.jpg}};
\end{tikzpicture}\hspace{-1.1mm}%
\begin{tikzpicture}[x=3.5cm, y=3.5cm]
\node[anchor=south] (FigF4) at (0,0) {\includegraphics[width=1.1in]{Fig./Qual/imgs/ip2p_sub/fruits/p2p.jpg}};
\end{tikzpicture}

\vspace{-3pt}
\setulcolor{magenta}
\setul{0.3pt}{2pt}
\centering \textit{``Colorful fruits in coconut bowl" $\to$ ``... \ul{green bowl}"} 

\vspace{-5pt}

\caption{\textbf{Qualitative results} of InstructPix2Pix dataset \cite{brooks2023instructpix2pix}. 
% In the third and fourth rows, details in the edited target areas lacked refinement, and in the last row, the color of the source image was not preserved.  
Our method successfully edits the image aligning with the target text prompt while preserving the structural integrity of the source image.}
\label{fig:ip2p_qual}
\end{figure*}
\vspace{-3pt}
% \vspace{-10pt}
\noindent\textbf{Editing with guided noise.} \ 
Thanks to the proposed FPR, the optimized source latent $\srctopt$ containing the source's identity can be obtained. 
Then, the guided noise $\epsilon^\ast$ is extracted as follows:
{\small
\begin{equation} \label{eq:guide_eps}
\epsilon^\ast = \frac{1}{\sqrt{1-\alpha_t}}(\srctopt - \sqrt{\alpha_t} \src).
\end{equation}
} 
$\epsilon^\ast$ is utilized to produce the stochastic latent $\trgtopt$ by applying the forward diffusion process to the target image $\trg$.
With $\srctopt$ and $\trgtopt$, the updated direction is given by:
{\small
\begin{equation} \label{eq:ids}
% \begin{split}
    \grad{\theta}\loss{\ids}
     = \mathbb{E}_{t, \epsilon} 
    \left[ 
      (\epsilon_\phi^\omega(\trgtopt, \txtt, t) - {\epsilon}_{\phi}^\omega(\srctopt, \txts, t)) \pardiff{\trg}{\theta}\right]. %\nonumber\\
    % = \grad{\theta}\loss{\sds}(\trg, \txtt) - \grad{\theta}\loss{\sds}(\src, \txts).
% \end{split}
\end{equation}
}
It is worth noting that $\epsilon^\ast$ guides the appropriate gradients for editing while conserving the source's identity. In contrast to DDS, the proposed IDS perfectly reconstructs the source from the edited result $\trg$, as shown in the third row of Fig.~\ref{fig:inversion}. 
This confirms that the correct score and the corresponding injection noise can preserve the identity without further consideration of mutual information. The flowchart of our IDS is illustrated in Fig.~\ref{fig:teasor}.

%aligns the $\epsilon$ to the denoising score of the source image $\src$ and the source prompt $\txts$. 

%Since the error occurs from the misalignment between $\epsilon_\phi^\text{src}$ and $\epsilon$, we update the $\epsilon$ before calculating DDS loss.

%If $\epsilon_\phi^\text{src}$ is estimated correctly, the reverse calculation of $\mathbf{z}_{0|t}^\text{src}$ from Eq. \ref{eq:forward} should be equal to $\src$. Based on this assumption, our loss function is defined as follows:
% \begin{equation}
%     \mathcal{L}_{\text{FPR}}
%     =\lVert \tilde{\mathbf{z}}_{0} - \mathbf z^{src} \rVert _{2}^{2},
% \end{equation}
% where
% \begin{align}
% \begin{split}
%     \tilde{\mathbf{z}}_{0} 
% %    &= \mathbb{E}[\tilde{\mathbf{z}}_{0}|\hat{\mathbf{z}}_{t}] \\
%     &= \frac{1}{\sqrt{\alpha_{t}}}\left(
%     \mathbf z^{src}_t-\sqrt{1-\alpha_{t}} \hat{\epsilon}_{\phi}^\omega
%     \right).
% \end{split}
% \end{align}

%Based on DPS \cite{chungdiffusion}, we calculate the unique posterior mean $\mathbf{z}_{0|t}^\text{src}$ from $\epsilon_{\phi}^\text{src}$.
% Because the predicted value $\mathbf{z}_{0|t}^\text{src}$ should be equal to the source image $\mathbf{z}^{\text{src}}$ if the $\epsilon$ and $\epsilon_{\phi}^\text{src}$ point in the same direction, our loss function is defined as follow:

% 우리 method는 이렇게 \hat{z}를 update 시킨 후 DDS loss를 그대로 따라감도 추가
% maintain the structural consistency