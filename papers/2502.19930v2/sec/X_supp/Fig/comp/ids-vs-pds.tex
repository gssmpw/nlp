%%% [START] end-to-end Nerf
\begin{figure}[!tbh]
\centering
\footnotesize

% 1st row
\raisebox{0.3in}{\rotatebox{90}{PDS}}%
\hspace{0.02mm}
\includegraphics[width=0.95\columnwidth]{sec/X_supp/Fig/imgs/ids-vs-pds/bear/pds_51400.png}

\vspace{2pt}

% 2nd row
\raisebox{0.15in}{\rotatebox{90}{\textbf{FPR+PDS}}}%
\hspace{-0.1mm}
\includegraphics[width=0.95\columnwidth]{sec/X_supp/Fig/imgs/ids-vs-pds/bear/ids_51400.png}

\vspace{-2.2pt}
% text
\setulcolor{magenta}
\setul{0.3pt}{2pt}
\centering \textit{``A bear made of stone" $\to$ ``A \ul{polar} bear"} 
\vspace{-2.2pt}

\vspace{2pt}

% 3rd row
\raisebox{0.3in}{\rotatebox{90}{PDS}}%
\hspace{0.02mm}
\includegraphics[width=0.95\columnwidth]{sec/X_supp/Fig/imgs/ids-vs-pds/face/pds_45000.png}

\vspace{2pt}

% 4th row
\raisebox{0.15in}{\rotatebox{90}{\textbf{FPR+PDS}}}%
\hspace{-0.1mm}
\includegraphics[width=0.95\columnwidth]{sec/X_supp/Fig/imgs/ids-vs-pds/face/ids_45000.png}

\vspace{-2.2pt}
% text
\setulcolor{magenta}
\setul{0.3pt}{2pt}
\centering \textit{``A man with curly hair with a beard" \\ $\to$ ``A man \ul{wearing with red glasses} ... "} 
\vspace{-8pt}

\caption{\textbf{3D Qualitative results for PDS} on subset of Instruct-NeRf2NeRF \cite{haque2023instruct}. From left to right, each column represents the source image, the edited image, and the gradient weight. The gradient weight indicates which regions the model primarily references during the editing process. The results demonstrate that FPR operates effectively in End-to-End NeRF while preserving the structure and identity of the source image.}
\vspace{-8pt}
\label{fig:ids-vs-pds}
\end{figure}
