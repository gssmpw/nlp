%%% [START] Ablation-fp_iter
\begin{figure}[H] % 1-column
\footnotesize
\centering 

% Adjust the image width to fit within one column
\newcommand{\imgwidth}{0.45\linewidth} % Set image width to 18% of the line width

\begin{subfigure}{0.49\linewidth}
\hspace{-2.3mm}
\raisebox{0.25in}{\rotatebox{90}{Images}}%
\hspace{-1.3mm}
% 1st Image
\begin{tikzpicture}[x=1cm, y=1cm]
    \node[anchor=south] (FigA1) at (0,0) {
        \includegraphics[width=\imgwidth]{sec/X_supp/Fig/imgs/mask/iou_src_img.png}
    };
    % \node[anchor=south, yshift=-1mm] at (FigA1.south) {\footnotesize image};
\end{tikzpicture}\hspace{-1mm}%
% 2nd Image
\begin{tikzpicture}[x=1cm, y=1cm]
    \node[anchor=south] (FigB1) at (0,0) {
        \includegraphics[width=\imgwidth]{sec/X_supp/Fig/imgs/mask/iou_trg_img.png}
    };
    % \node[anchor=south, yshift=-1mm] at (FigB1.south) {\footnotesize mask};
\end{tikzpicture}\hspace{-1mm}%

\vspace{6pt}

\hspace{-2.2mm}
\raisebox{0.35in}{\rotatebox{90}{Masks}}%
\hspace{-1.2mm}
% 1st Image
\begin{tikzpicture}[x=1cm, y=1cm]
    \node[anchor=south] (FigA2) at (0,0) {
        \includegraphics[width=\imgwidth]{sec/X_supp/Fig/imgs/mask/iou_src_mask.png}
    };
    \node[anchor=south, yshift=-2.5mm] at (FigA2.south) {\footnotesize source \textit{``cat"}};
\end{tikzpicture}\hspace{-1mm}%
% 2nd Image
\begin{tikzpicture}[x=1cm, y=1cm]
    \node[anchor=south] (FigB2) at (0,0) {
        \includegraphics[width=\imgwidth]{sec/X_supp/Fig/imgs/mask/iou_trg_mask.png}
    };
    \node[anchor=south, yshift=-2.5mm] at (FigB2.south) {\footnotesize target \textit{``dog"}};
\end{tikzpicture}\hspace{-1mm}%
\caption{IoU mask}
\end{subfigure}
\hfill
\begin{subfigure}{0.49\linewidth}
\hspace{-2.3mm}
\raisebox{0.35in}{\rotatebox{90}{Images}}%
\hspace{-1.3mm}
% 1st Image
\begin{tikzpicture}[x=1cm, y=1cm]
    \node[anchor=south] (FigA3) at (0,0) {
        \includegraphics[width=\imgwidth]{sec/X_supp/Fig/imgs/mask/psnr_src_img.png}
    };
    \node[anchor=south, yshift=-2.5mm] at (FigA3.south) {\footnotesize source};
\end{tikzpicture}\hspace{-1mm}%
% 2nd Image
\begin{tikzpicture}[x=1cm, y=1cm]
    \node[anchor=south] (FigB3) at (0,0) {
        \includegraphics[width=\imgwidth]{sec/X_supp/Fig/imgs/mask/psnr_trg.png}
    };
    \node[anchor=south, yshift=-2.5mm] at (FigB3.south) {\footnotesize target};
\end{tikzpicture}\hspace{-1mm}%

% \vspace{-3pt}

\hspace{-2.2mm}
\raisebox{0.35in}{\rotatebox{90}{Masks}}%
\hspace{-1.2mm}
% 1st Image
\begin{tikzpicture}[x=1cm, y=1cm]
    \node[anchor=south] (FigA4) at (0,0) {
        \includegraphics[width=\imgwidth]{sec/X_supp/Fig/imgs/mask/psnr_mean_mask.png}
    };
    \node[anchor=south, yshift=-2.5mm] at (FigA4.south) {\footnotesize mean};
\end{tikzpicture}\hspace{-1mm}%
% 2nd Image
\begin{tikzpicture}[x=1cm, y=1cm]
    \node[anchor=south] (FigB4) at (0,0) {
        \includegraphics[width=\imgwidth]{sec/X_supp/Fig/imgs/mask/psnr_median_mask.png}
    };
    \node[anchor=south, yshift=-2.5mm] at (FigB4.south) {\footnotesize median};
\end{tikzpicture}\hspace{-1mm}%
\caption{Background PSNR mask}
\end{subfigure}
\vspace{-5pt}
% Caption Text
\caption{\textbf{Calculated masks} for IoU and background PSNR. In (a), \textit{(second row)} each mask for \textit{(first row)} the source and target image is obtained by using lang-SAM for IoU. In (b), \textit{(second row)} a mask is calculated for \textit{(first row)} the source and target image to measure background PSNR between the masked source and target image. The mask can be generated by thresholding method, mean and median}
% \vspace{-5pt}
\label{fig:sup_mask}
\end{figure}
