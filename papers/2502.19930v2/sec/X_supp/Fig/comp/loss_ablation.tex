%%% [START] 6.1
\begin{figure}[thb!] % 1-column
\footnotesize
\centering 

% Adjust the image width to fit within one column
\newcommand{\imgwidth}{0.235\linewidth} % Set image width to 16% of the line width

% \hspace{-1.9mm}
% \raisebox{0.25in}{\rotatebox{90}{Source}}%
% \hspace{-0.9mm}
% 1st Image
\begin{tikzpicture}[x=1cm, y=1cm]
    \node[anchor=south] (FigA1) at (0,0) {
        \includegraphics[width=\imgwidth]{sec/X_supp/Fig/imgs/loss_ablation/src_img.png}
    };
    \node[anchor=south, yshift=-2.5mm] at (FigA1.south) {\footnotesize Source};
\end{tikzpicture}\hspace{-1mm}%
% \hspace{-2.3mm}
% \raisebox{0.4in}{\rotatebox{90}{Target}}%
% \hspace{-1.3mm}
% 2nd Image
% \begin{tikzpicture}[x=1cm, y=1cm]
%     \node[anchor=south] (FigA2) at (0,0) {
%         \includegraphics[width=\imgwidth]{sec/X_supp/Fig/imgs/loss_ablation/dds.png}
%     };
%     \node[anchor=south, yshift=-2.5mm] at (FigA2.south) {\footnotesize w/o \ FPR};
% \end{tikzpicture}\hspace{-1mm}%
\begin{tikzpicture}[x=1cm, y=1cm]
    \node[anchor=south] (FigB1) at (0,0) {
        \includegraphics[width=\imgwidth]{sec/X_supp/Fig/imgs/loss_ablation/l2_loss.png}
    };
    \node[anchor=south, yshift=-2.5mm] at (FigB1.south) {\footnotesize Euclidean loss};
\end{tikzpicture}\hspace{-1mm}%
% % 3rd Image
\begin{tikzpicture}[x=1cm, y=1cm]
    \node[anchor=south] (FigC1) at (0,0) {
        \includegraphics[width=\imgwidth]{sec/X_supp/Fig/imgs/loss_ablation/l1_loss.png}
    };
    \node[anchor=south, yshift=-2.5mm] at (FigC1.south) {\footnotesize L1 loss};
\end{tikzpicture}\hspace{-1mm}%
% 4th Image
\begin{tikzpicture}[x=1cm, y=1cm]
    \node[anchor=south] (FigD1) at (0,0) {
        \includegraphics[width=\imgwidth]{sec/X_supp/Fig/imgs/loss_ablation/ssim_loss.png}
    };
    \node[anchor=south, yshift=-2.5mm] at (FigD1.south) {\footnotesize SSIM loss};
\end{tikzpicture}

\vspace{-8pt}
\caption{\textbf{Ablation study for loss function.} Edited results of \textit{(first)} the source image from prompt \textit{``a drawing of a cat"} to \textit{``a drawing of a dog"} using \textit{(second)} Euclidean, \textit{(third)} L1, and \textit{(fourth)} SSIM loss function for FPR.}
\label{fig:sup_loss_abl}
\end{figure}
