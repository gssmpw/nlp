\section{Implementation details}
\label{sec:s_implement}
% which model we used
% about baselines
% how to calculate background psnr & IoU
% calculated mask examples
For experiments, we implement our method based on the official code of CDS \footnote{\url{https://hyelinnam.github.io/CDS/}} by using Stable Diffusion v1.4. All baselines are implemented based on the official code and setting for each method.
For the proposed FPR, we set the scale $\lambda$ to $1.0$ and iteration $N$ to 3. 
The range of timesteps, optimization, learning rate, and number of optimization steps correspond to the default settings employed in DDS and CDS. 
% for experimental results in the main paper. 
All experiments are conducted on a single NVIDIA RTX 3090.

%\paragraph{IoU} To calculate IoU for \textit{Cat-to-Others} task, we use the language Segment-Anything model (lang-SAM)\footnote{\url{https://github.com/paulguerrero/lang-sam}}, which is an open-source project to segment some objects from the text prompt. First, the mask about the prompt is obtained from an image using lang-SAM. For example, `cat' is segmented from the source image to get the mask $M_{\text{src}}$, while `dog' is segmented from the edited image to obtain the mask $M_{\text{trg}}$, as shown in \cref{fig:sup_mask} (a). After getting masks, we calculate IoU from the masks that is given by:
%\begin{equation*}
%    \text{IoU}=\frac{\left(M_{\text{src}} \cap M_{\text{trg}}\right)}{\left(M_{\text{src}}\cup M_{\text{trg}}\right)}
%\end{equation*}

%\paragraph{Background PSNR} Since the editing prompts of IP2P dataset~\cite{brooks2023instructpix2pix} is complex than \textit{Cat-to-Others} dataset~\cite{schuhmann2022laion, nam2024contrastive}, it is hard to get mask by lang-SAM. Therefore, we use background PSNR to evaluate how much the original information is preserved. The residual of the source and target images is calculated, and the standard deviation $\sigma$ of each pixel of the residual image is computed with window size 30. Then, the mask $M_{\text{PSNR}}$ is acquired by thresholding the $\sigma$ to the mean or median value of $\sigma$ as shown in \cref{fig:sup_mask} (b). Finally, we get PSNR values from masked source and target images: 
%\begin{equation*}
 %   \text{PSNR}_{\text{back}}=\text{PSNR}(M_{\text{PSNR}} \odot \mathbf{z}_{\text{src}}, M_{\text{PSNR}} \odot \mathbf{z}_{\text{trg}})
%\end{equation*}
%where $\odot$ is pixel-wise multiplication.
%%%% [START] Ablation-fp_iter
\begin{figure}[H] % 1-column
\footnotesize
\centering 

% Adjust the image width to fit within one column
\newcommand{\imgwidth}{0.45\linewidth} % Set image width to 18% of the line width

\begin{subfigure}{0.49\linewidth}
\hspace{-2.3mm}
\raisebox{0.25in}{\rotatebox{90}{Images}}%
\hspace{-1.3mm}
% 1st Image
\begin{tikzpicture}[x=1cm, y=1cm]
    \node[anchor=south] (FigA1) at (0,0) {
        \includegraphics[width=\imgwidth]{sec/X_supp/Fig/imgs/mask/iou_src_img.png}
    };
    % \node[anchor=south, yshift=-1mm] at (FigA1.south) {\footnotesize image};
\end{tikzpicture}\hspace{-1mm}%
% 2nd Image
\begin{tikzpicture}[x=1cm, y=1cm]
    \node[anchor=south] (FigB1) at (0,0) {
        \includegraphics[width=\imgwidth]{sec/X_supp/Fig/imgs/mask/iou_trg_img.png}
    };
    % \node[anchor=south, yshift=-1mm] at (FigB1.south) {\footnotesize mask};
\end{tikzpicture}\hspace{-1mm}%

\vspace{6pt}

\hspace{-2.2mm}
\raisebox{0.35in}{\rotatebox{90}{Masks}}%
\hspace{-1.2mm}
% 1st Image
\begin{tikzpicture}[x=1cm, y=1cm]
    \node[anchor=south] (FigA2) at (0,0) {
        \includegraphics[width=\imgwidth]{sec/X_supp/Fig/imgs/mask/iou_src_mask.png}
    };
    \node[anchor=south, yshift=-2.5mm] at (FigA2.south) {\footnotesize source \textit{``cat"}};
\end{tikzpicture}\hspace{-1mm}%
% 2nd Image
\begin{tikzpicture}[x=1cm, y=1cm]
    \node[anchor=south] (FigB2) at (0,0) {
        \includegraphics[width=\imgwidth]{sec/X_supp/Fig/imgs/mask/iou_trg_mask.png}
    };
    \node[anchor=south, yshift=-2.5mm] at (FigB2.south) {\footnotesize target \textit{``dog"}};
\end{tikzpicture}\hspace{-1mm}%
\caption{IoU mask}
\end{subfigure}
\hfill
\begin{subfigure}{0.49\linewidth}
\hspace{-2.3mm}
\raisebox{0.35in}{\rotatebox{90}{Images}}%
\hspace{-1.3mm}
% 1st Image
\begin{tikzpicture}[x=1cm, y=1cm]
    \node[anchor=south] (FigA3) at (0,0) {
        \includegraphics[width=\imgwidth]{sec/X_supp/Fig/imgs/mask/psnr_src_img.png}
    };
    \node[anchor=south, yshift=-2.5mm] at (FigA3.south) {\footnotesize source};
\end{tikzpicture}\hspace{-1mm}%
% 2nd Image
\begin{tikzpicture}[x=1cm, y=1cm]
    \node[anchor=south] (FigB3) at (0,0) {
        \includegraphics[width=\imgwidth]{sec/X_supp/Fig/imgs/mask/psnr_trg.png}
    };
    \node[anchor=south, yshift=-2.5mm] at (FigB3.south) {\footnotesize target};
\end{tikzpicture}\hspace{-1mm}%

% \vspace{-3pt}

\hspace{-2.2mm}
\raisebox{0.35in}{\rotatebox{90}{Masks}}%
\hspace{-1.2mm}
% 1st Image
\begin{tikzpicture}[x=1cm, y=1cm]
    \node[anchor=south] (FigA4) at (0,0) {
        \includegraphics[width=\imgwidth]{sec/X_supp/Fig/imgs/mask/psnr_mean_mask.png}
    };
    \node[anchor=south, yshift=-2.5mm] at (FigA4.south) {\footnotesize mean};
\end{tikzpicture}\hspace{-1mm}%
% 2nd Image
\begin{tikzpicture}[x=1cm, y=1cm]
    \node[anchor=south] (FigB4) at (0,0) {
        \includegraphics[width=\imgwidth]{sec/X_supp/Fig/imgs/mask/psnr_median_mask.png}
    };
    \node[anchor=south, yshift=-2.5mm] at (FigB4.south) {\footnotesize median};
\end{tikzpicture}\hspace{-1mm}%
\caption{Background PSNR mask}
\end{subfigure}
\vspace{-5pt}
% Caption Text
\caption{\textbf{Calculated masks} for IoU and background PSNR. In (a), \textit{(second row)} each mask for \textit{(first row)} the source and target image is obtained by using lang-SAM for IoU. In (b), \textit{(second row)} a mask is calculated for \textit{(first row)} the source and target image to measure background PSNR between the masked source and target image. The mask can be generated by thresholding method, mean and median}
% \vspace{-5pt}
\label{fig:sup_mask}
\end{figure}

