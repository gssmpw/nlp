\section{Extension to other methods}
\label{sec:s_extension}
% SDS image + FPR
% SDS NeRF + FPR
% CDS + FPR
% PDS + FPR
Since our method optimizes the source latent to estimate a more accurate score, it can be applied to other methods that are based on SDS despite that we report the results using our method to DDS.

During SDS optimization, FPR can be used to preserve the original content and reduce the blurry effect. As shown in \cref{fig:existingSDS}, the conserved rate of the information of the source image is controllable by the number of FPR iteration.

When the proposed FPR is integrated into CDS, the texture of the source image is further maintained, as illustrated in \cref{fig:sup_fpr_cds}. In addition, FPR promoted reducing the over-boosting of color often found in the translated images of CDS. This confirms that the proposed FPR can be a universal regularization to preserve the identity of the source image for text-guided image editing.
%%% [START] 6.1
\begin{figure}[t] % 1-column
\footnotesize
\centering 

% Adjust the image width to fit within one column
\newcommand{\imgwidth}{0.235\linewidth} % Set image width to 16% of the line width

% 1st Image
\begin{tikzpicture}[x=1cm, y=1cm]
    \node[anchor=south] (FigA1) at (0,0) {
        \includegraphics[width=\imgwidth]{Fig./Qual/imgs/existingSDS/original.jpg}
    };
    \node[anchor=south, yshift=-1mm] at (FigA1.north) {\footnotesize Source};
\end{tikzpicture}\hspace{-1mm}%
% 2nd Image
\begin{tikzpicture}[x=1cm, y=1cm]
    \node[anchor=south] (FigB1) at (0,0) {
        \includegraphics[width=\imgwidth]{Fig./Qual/imgs/existingSDS/sds.jpg}
    };
    \node[anchor=south, yshift=-1mm] at (FigB1.north) {\footnotesize SDS};
\end{tikzpicture}\hspace{-1mm}%
% % 3rd Image
\begin{tikzpicture}[x=1cm, y=1cm]
    \node[anchor=south] (FigC1) at (0,0) {
        \includegraphics[width=\imgwidth]{Fig./Qual/imgs/existingSDS/fp30.jpg}
    };
    \node[anchor=south, yshift=-1mm] at (FigC1.north) {\footnotesize Iter 30};
\end{tikzpicture}\hspace{-1mm}%
% 4th Image
\begin{tikzpicture}[x=1cm, y=1cm]
    \node[anchor=south] (FigD1) at (0,0) {
        \includegraphics[width=\imgwidth]{Fig./Qual/imgs/existingSDS/fp50.jpg}
    };
    \node[anchor=south, yshift=-1mm] at (FigD1.north) {\footnotesize Iter 50};
\end{tikzpicture}

\vspace{-8pt}
\caption{\textbf{SDS with FPR.} %regularization.} 
Given \textit{(first)} source image and prompt \textit{``a drawing of a cat"}, \textit{(second)} SDS optimization, \textit{(third, fourth)} SDS optimization with FPR for 30 and 50 iterations are applied. Each result uses 200 steps for optimization.}
\vspace{-8pt}
\label{fig:existingSDS}
\end{figure}

%%% [START] 6.1
\begin{figure}[t] % 1-column
\footnotesize
\centering 

% Adjust the image width to fit within one column
\newcommand{\imgwidth}{1in} % Set image width to 16% of the line width

% 1st Image
\begin{tikzpicture}[x=1cm, y=1cm]
    \node[anchor=south] (FigA1) at (0,0) {
        \includegraphics[width=\imgwidth]{sec/X_supp/Fig/imgs/fpr_cds/src_img.png}
    };
    \node[anchor=south, yshift=-1mm] at (FigA1.north) {\footnotesize Source};
\end{tikzpicture}\hspace{-1mm}%
% 2nd Image
\begin{tikzpicture}[x=1cm, y=1cm]
    \node[anchor=south] (FigB1) at (0,0) {
        \includegraphics[width=\imgwidth]{sec/X_supp/Fig/imgs/fpr_cds/cds.png}
    };
    \node[anchor=south, yshift=-1mm] at (FigB1.north) {\footnotesize CDS};
\end{tikzpicture}\hspace{-1mm}%
% % 3rd Image
\begin{tikzpicture}[x=1cm, y=1cm]
    \node[anchor=south] (FigC1) at (0,0) {
        \includegraphics[width=\imgwidth]{sec/X_supp/Fig/imgs/fpr_cds/fpr_cds.png}
    };
    \node[anchor=south, yshift=-1mm] at (FigC1.north) {\footnotesize FPR+CDS};
\end{tikzpicture}\hspace{-1mm}%
% 4th Image

\vspace{-8pt}
\caption{\textbf{CDS with FPR.} %regularization.} 
Given \textit{(first)} source image, source prompt \textit{``a drawing of a cat"}, and target prompt \textit{``a drawing of a pig"}, \textit{(second)} CDS translation, \textit{(third)} CDS optimization with FPR for $N=3$ and $\lambda=1.0$.}% Each result uses 200 steps for optimization.}
\label{fig:sup_fpr_cds}
\end{figure}


% \paragraph{FPR + SDS in 2D} FPR can be applied to SDS optimization for a given source image and prompt to preserve the original contents and reduce the blurry effect. As shown in \cref{fig:existingSDS}, the conserved rate of the information of the source image is controllable by the number of iterations of FPR.
% %%% [START] 6.1
\begin{figure}[t] % 1-column
\footnotesize
\centering 

% Adjust the image width to fit within one column
\newcommand{\imgwidth}{0.235\linewidth} % Set image width to 16% of the line width

% 1st Image
\begin{tikzpicture}[x=1cm, y=1cm]
    \node[anchor=south] (FigA1) at (0,0) {
        \includegraphics[width=\imgwidth]{Fig./Qual/imgs/existingSDS/original.jpg}
    };
    \node[anchor=south, yshift=-1mm] at (FigA1.north) {\footnotesize Source};
\end{tikzpicture}\hspace{-1mm}%
% 2nd Image
\begin{tikzpicture}[x=1cm, y=1cm]
    \node[anchor=south] (FigB1) at (0,0) {
        \includegraphics[width=\imgwidth]{Fig./Qual/imgs/existingSDS/sds.jpg}
    };
    \node[anchor=south, yshift=-1mm] at (FigB1.north) {\footnotesize SDS};
\end{tikzpicture}\hspace{-1mm}%
% % 3rd Image
\begin{tikzpicture}[x=1cm, y=1cm]
    \node[anchor=south] (FigC1) at (0,0) {
        \includegraphics[width=\imgwidth]{Fig./Qual/imgs/existingSDS/fp30.jpg}
    };
    \node[anchor=south, yshift=-1mm] at (FigC1.north) {\footnotesize Iter 30};
\end{tikzpicture}\hspace{-1mm}%
% 4th Image
\begin{tikzpicture}[x=1cm, y=1cm]
    \node[anchor=south] (FigD1) at (0,0) {
        \includegraphics[width=\imgwidth]{Fig./Qual/imgs/existingSDS/fp50.jpg}
    };
    \node[anchor=south, yshift=-1mm] at (FigD1.north) {\footnotesize Iter 50};
\end{tikzpicture}

\vspace{-8pt}
\caption{\textbf{SDS with FPR.} %regularization.} 
Given \textit{(first)} source image and prompt \textit{``a drawing of a cat"}, \textit{(second)} SDS optimization, \textit{(third, fourth)} SDS optimization with FPR for 30 and 50 iterations are applied. Each result uses 200 steps for optimization.}
\vspace{-8pt}
\label{fig:existingSDS}
\end{figure}


Furthermore, FPR can help optimize not only pixel space but also the parametric editor such as PDS~\cite{koo2024posterior}. As demonstrated in \cref{fig:ids-vs-pds}, \cref{fig:pds_svg}, and \cref{tab:nerf_svg}, the edited results with our method show that FPR assists in maintaining the original contents. By comparing the first and second rows of \cref{fig:ids-vs-pds}, the use of FPR results in the preservation of source components more effectively compared to PDS. Similarly, in the third and fourth rows, the results obtained using FPR retain key original features, such as the shape and color of the face as well as the color of the clothing. Furthermore, the gradient weights, FPR assigns minimal weight to the structure of the source image, such as the background, while primarily focusing the weights on the editing points. For 3D and 2D editing, we implement the experiments based on official code of PDS\footnote{\url{https://github.com/KAIST-Visual-AI-Group/PDS}}. We use the subset of Instruct-NeRF2NeRF \cite{haque2023instruct} for 3D editing and Scalable Vector Graphics (SVGs) with their text description used in \cite{jain2023vectorfusion} for 2D editing. 
% \paragraph{FPR + SDS in 3D.}

% \paragraph{FPR + CDS} As shown in \cref{fig:sup_fpr_cds}, FPR can regulate the  control the 
% As shown in CDS's translation of \cref{fig:sup_fpr_cds}, CDS has the effect of boosting color. FPR makes the translated result of CDS 
% %%% [START] 6.1
\begin{figure}[t] % 1-column
\footnotesize
\centering 

% Adjust the image width to fit within one column
\newcommand{\imgwidth}{1in} % Set image width to 16% of the line width

% 1st Image
\begin{tikzpicture}[x=1cm, y=1cm]
    \node[anchor=south] (FigA1) at (0,0) {
        \includegraphics[width=\imgwidth]{sec/X_supp/Fig/imgs/fpr_cds/src_img.png}
    };
    \node[anchor=south, yshift=-1mm] at (FigA1.north) {\footnotesize Source};
\end{tikzpicture}\hspace{-1mm}%
% 2nd Image
\begin{tikzpicture}[x=1cm, y=1cm]
    \node[anchor=south] (FigB1) at (0,0) {
        \includegraphics[width=\imgwidth]{sec/X_supp/Fig/imgs/fpr_cds/cds.png}
    };
    \node[anchor=south, yshift=-1mm] at (FigB1.north) {\footnotesize CDS};
\end{tikzpicture}\hspace{-1mm}%
% % 3rd Image
\begin{tikzpicture}[x=1cm, y=1cm]
    \node[anchor=south] (FigC1) at (0,0) {
        \includegraphics[width=\imgwidth]{sec/X_supp/Fig/imgs/fpr_cds/fpr_cds.png}
    };
    \node[anchor=south, yshift=-1mm] at (FigC1.north) {\footnotesize FPR+CDS};
\end{tikzpicture}\hspace{-1mm}%
% 4th Image

\vspace{-8pt}
\caption{\textbf{CDS with FPR.} %regularization.} 
Given \textit{(first)} source image, source prompt \textit{``a drawing of a cat"}, and target prompt \textit{``a drawing of a pig"}, \textit{(second)} CDS translation, \textit{(third)} CDS optimization with FPR for $N=3$ and $\lambda=1.0$.}% Each result uses 200 steps for optimization.}
\label{fig:sup_fpr_cds}
\end{figure}

% \paragraph{FPR + PDS.}
%%% [START] end-to-end Nerf
\begin{figure}[!tbh]
\centering
\footnotesize

% 1st row
\raisebox{0.3in}{\rotatebox{90}{PDS}}%
\hspace{0.02mm}
\includegraphics[width=0.95\columnwidth]{sec/X_supp/Fig/imgs/ids-vs-pds/bear/pds_51400.png}

\vspace{2pt}

% 2nd row
\raisebox{0.15in}{\rotatebox{90}{\textbf{FPR+PDS}}}%
\hspace{-0.1mm}
\includegraphics[width=0.95\columnwidth]{sec/X_supp/Fig/imgs/ids-vs-pds/bear/ids_51400.png}

\vspace{-2.2pt}
% text
\setulcolor{magenta}
\setul{0.3pt}{2pt}
\centering \textit{``A bear made of stone" $\to$ ``A \ul{polar} bear"} 
\vspace{-2.2pt}

\vspace{2pt}

% 3rd row
\raisebox{0.3in}{\rotatebox{90}{PDS}}%
\hspace{0.02mm}
\includegraphics[width=0.95\columnwidth]{sec/X_supp/Fig/imgs/ids-vs-pds/face/pds_45000.png}

\vspace{2pt}

% 4th row
\raisebox{0.15in}{\rotatebox{90}{\textbf{FPR+PDS}}}%
\hspace{-0.1mm}
\includegraphics[width=0.95\columnwidth]{sec/X_supp/Fig/imgs/ids-vs-pds/face/ids_45000.png}

\vspace{-2.2pt}
% text
\setulcolor{magenta}
\setul{0.3pt}{2pt}
\centering \textit{``A man with curly hair with a beard" \\ $\to$ ``A man \ul{wearing with red glasses} ... "} 
\vspace{-8pt}

\caption{\textbf{3D Qualitative results for PDS} on subset of Instruct-NeRf2NeRF \cite{haque2023instruct}. From left to right, each column represents the source image, the edited image, and the gradient weight. The gradient weight indicates which regions the model primarily references during the editing process. The results demonstrate that FPR operates effectively in End-to-End NeRF while preserving the structure and identity of the source image.}
\vspace{-8pt}
\label{fig:ids-vs-pds}
\end{figure}

\begin{figure}[!tbh] % 1-column
\centering
\footnotesize

% Adjust the image width so that 5 images fit within one column
\newcommand{\imgwidth}{0.18\linewidth}

% 1st Image
\begin{tikzpicture}[x=1cm, y=1cm]
    \node[anchor=south] (FigA1) at (0,0) {
        \includegraphics[width=\imgwidth]{sec/X_supp/Fig/imgs/pds_SVG/init.png}
    };
    \node[anchor=south, yshift=-1.5mm] at (FigA1.north) {\footnotesize Source};
\end{tikzpicture}\hspace{-1mm}%
% 2nd Image
\begin{tikzpicture}[x=1cm, y=1cm]
    \node[anchor=south] (FigB1) at (0,0) {
        \includegraphics[width=\imgwidth]{sec/X_supp/Fig/imgs/pds_SVG/pds+ids.png}
    };
    \node[anchor=south, yshift=-1.5mm] at (FigB1.north) {\footnotesize \textbf{FPR+PDS}};
\end{tikzpicture}\hspace{-1mm}%
% 3rd Image
\begin{tikzpicture}[x=1cm, y=1cm]
    \node[anchor=south] (FigC1) at (0,0) {
        \includegraphics[width=\imgwidth]{sec/X_supp/Fig/imgs/pds_SVG/pds.png}
    };
    \node[anchor=south, yshift=-1.5mm] at (FigC1.north) {\footnotesize PDS};
\end{tikzpicture}\hspace{-1mm}%
% 4th Image
\begin{tikzpicture}[x=1cm, y=1cm]
    \node[anchor=south] (FigD1) at (0,0) {
        \includegraphics[width=\imgwidth]{sec/X_supp/Fig/imgs/pds_SVG/dds.png}
    };
    \node[anchor=south, yshift=-1.5mm] at (FigD1.north) {\footnotesize DDS};
\end{tikzpicture}\hspace{-1mm}%
% 5th Image
\begin{tikzpicture}[x=1cm, y=1cm]
    \node[anchor=south] (FigE1) at (0,0) {
        \includegraphics[width=\imgwidth]{sec/X_supp/Fig/imgs/pds_SVG/sds.png}
    };
    \node[anchor=south, yshift=-1.5mm] at (FigE1.north) {\footnotesize SDS};
\end{tikzpicture}

\vspace{-5pt}
\setulcolor{magenta}
\setul{0.3pt}{2pt}
\centering
\textbf{SVG:} \textit{``An owl" $\to$ ``\ul{A monkey}"} 
\vspace{-8pt}
\caption{\textbf{2D Qualitative results for PDS on VectorFusion \cite{jain2023vectorfusion}}. In SVG editing, our method can be utilized with PDS and help the source identity maintained.}
% \vspace{-8pt}
\label{fig:pds_svg}
\end{figure}

% \begin{table}[ht]
%     \centering
\begin{table}[ht]
    \centering
    % 표 안의 글자 크기를 한 단계 줄임
    \small
    % 열 간 간격을 기본보다 줄임 (기본 약 6pt)
    \setlength{\tabcolsep}{1pt}
    % 행 높이 계수를 줄임 (기본 1.0)
    \renewcommand{\arraystretch}{0.5}
    \begin{tabular}{lcc cc}
    \toprule
          & \multicolumn{2}{c}{\scriptsize\textbf{NeRF}} & \multicolumn{2}{c}{\scriptsize\textbf{SVG}} \\
          \cmidrule(lr){2-3} \cmidrule(lr){4-5}
    Metric & {\scriptsize CLIP($\uparrow$)} & {\scriptsize LPIPS($\downarrow$)} & {\scriptsize CLIP($\uparrow$)} & {\scriptsize LPIPS($\downarrow$)} \\
    \midrule
    SDS      & 0.305          & 0.814          & \textbf{0.346} & 0.552 \\
    DDS      & \textbf{0.306} & 0.875          & 0.344          & 0.557 \\
    PDS      & 0.292          & 0.662          & 0.324          & 0.326 \\
    \textbf{Ours+PDS} & 0.295          & \textbf{0.587} & 0.327          & \textbf{0.274} \\
    \bottomrule
    \end{tabular}
    \caption{\textbf{Quantitative results} for PDS.}
    \label{tab:nerf_svg}
\end{table}