\section{Additional results}
\label{sec:s_addresults}
% various prompts for one image
% cat2others qual quan
\begin{table*}[!thb]
\centering
\resizebox{0.95\textwidth}{!}{
\small{
\begin{tabular}{c|cc|cc|cc|cc|cc}
\hline
& \multicolumn{2}{c|}{cat2cow} & \multicolumn{2}{c|}{cat2dog} & \multicolumn{2}{c|}{cat2lion} & \multicolumn{2}{c|}{cat2tiger} & \multicolumn{2}{c}{cat2penguin} \\ 
\hline
\multicolumn{1}{c|}{Metric} & LPIPS ($\downarrow$) & IoU ($\uparrow$) & LPIPS ($\downarrow$) & IoU ($\uparrow$) & LPIPS ($\downarrow$) & IoU ($\uparrow$) & LPIPS ($\downarrow$) & IoU ($\uparrow$) & LPIPS ($\downarrow$) & IoU ($\uparrow$) \\ 
\hline
P2P \cite{hertzprompt}& 0.43 & 0.57 & 0.42 & 0.51 & 0.46 & 0.57 & 0.47 & 0.57 & 0.46 & 0.54 \\
PnP \cite{tumanyan2023plug}& 0.52 & 0.55 & 0.47 & 0.59 & 0.51 & 0.58 & 0.52 & 0.58 & 0.52 & 0.52 \\
DDS \cite{hertz2023delta}& 0.29 & 0.65 & 0.22 & 0.72 & 0.29 & 0.69 & 0.30 & 0.71 & 0.28 & 0.66 \\  
CDS \cite{nam2024contrastive}& 0.25 & 0.72 & 0.19 & 0.74 & 0.25 & \textbf{0.74} & 0.27 & 0.75 & 0.24 & \textbf{0.72} \\
\hline
\textbf{IDS (Ours)}& \textbf{0.21} & \textbf{0.74} & \textbf{0.17} & \textbf{0.75} & \textbf{0.21} & 0.71 & \textbf{0.21} & \textbf{0.76} & \textbf{0.21} & \textbf{0.72} \\
\hline
\end{tabular}
}
}
\vspace{-5pt}
\caption{\textbf{Quantitative results} for \textit{Cat-to-Others} task. LPIPS \cite{zhang2018unreasonable} and IoU are used. Lower LPIPS and higher IoU mean better identity preserving.}
\label{tab:sup_cat2others}
\end{table*}




%P2P \cite{hertzprompt}& 0.5798 & 0.4229 & 0.5184 & 0.4605 & 20.88 & 0.4695 \\
%PnP \cite{tumanyan2023plug}& 0.5507 & 0.5191 & ??? & 0.5245 & 23.81 & 0.3882 \\
%DDS \cite{hertz2023delta}& 0.6897 & 0.2838 & 0.6456 & 0.2996 & 26.02 & 0.2398 \\  
%CDS \cite{nam2024contrastive}& 0.7249 & 0.2485 & 0.7054 & 0.2612 & 27.35 & 0.2099 \\

We also provide qualitative results for \textit{Cat-to-Others} task, as demonstrated in \cref{fig:sup_cat2others}. With DDS and CDS, the direction of the gaze changes when translated from the cat to the squirrel, while it remains the same with IDS.
Note that the proposed IDS can also retain the hue of the source image without overemphasizing the colors, as demonstrated in \textit{Cat-to-Tiger} task. This confirms that the proposed IDS consistently offers suitable editing of cat images into the diverse animals, while conserving the identity of the source against other algorithms. 

The trends in the quantitative results are also consistent with the qualitative result, as represented in \cref{tab:sup_cat2others}.
%In addition to the results in \cref{sec:5.1},
Our method provides the best performance for LPIPS and IoU in most \textit{Cat-to-Others} tasks. This shows again that the self-correction of the score using the proposed algorithm is crucial for maintaining the identity.
%as demonstrated in \cref{tab:sup_cat2others}. Furthermore, as discussed in the main paper, IDS maintains the identity of the source image such as pose and color (see \cref{fig:sup_cat2others}).

\begin{table*}[!thb]
\centering
\resizebox{0.95\textwidth}{!}{
\small{
\begin{tabular}{c|cc|cc|cc|cc|cc}
\hline
& \multicolumn{2}{c|}{cat2cow} & \multicolumn{2}{c|}{cat2dog} & \multicolumn{2}{c|}{cat2lion} & \multicolumn{2}{c|}{cat2tiger} & \multicolumn{2}{c}{cat2penguin} \\ 
\hline
\multicolumn{1}{c|}{Metric} & LPIPS ($\downarrow$) & IoU ($\uparrow$) & LPIPS ($\downarrow$) & IoU ($\uparrow$) & LPIPS ($\downarrow$) & IoU ($\uparrow$) & LPIPS ($\downarrow$) & IoU ($\uparrow$) & LPIPS ($\downarrow$) & IoU ($\uparrow$) \\ 
\hline
P2P \cite{hertzprompt}& 0.43 & 0.57 & 0.42 & 0.51 & 0.46 & 0.57 & 0.47 & 0.57 & 0.46 & 0.54 \\
PnP \cite{tumanyan2023plug}& 0.52 & 0.55 & 0.47 & 0.59 & 0.51 & 0.58 & 0.52 & 0.58 & 0.52 & 0.52 \\
DDS \cite{hertz2023delta}& 0.29 & 0.65 & 0.22 & 0.72 & 0.29 & 0.69 & 0.30 & 0.71 & 0.28 & 0.66 \\  
CDS \cite{nam2024contrastive}& 0.25 & 0.72 & 0.19 & 0.74 & 0.25 & \textbf{0.74} & 0.27 & 0.75 & 0.24 & \textbf{0.72} \\
\hline
\textbf{IDS (Ours)}& \textbf{0.21} & \textbf{0.74} & \textbf{0.17} & \textbf{0.75} & \textbf{0.21} & 0.71 & \textbf{0.21} & \textbf{0.76} & \textbf{0.21} & \textbf{0.72} \\
\hline
\end{tabular}
}
}
\vspace{-5pt}
\caption{\textbf{Quantitative results} for \textit{Cat-to-Others} task. LPIPS \cite{zhang2018unreasonable} and IoU are used. Lower LPIPS and higher IoU mean better identity preserving.}
\label{tab:sup_cat2others}
\end{table*}




%P2P \cite{hertzprompt}& 0.5798 & 0.4229 & 0.5184 & 0.4605 & 20.88 & 0.4695 \\
%PnP \cite{tumanyan2023plug}& 0.5507 & 0.5191 & ??? & 0.5245 & 23.81 & 0.3882 \\
%DDS \cite{hertz2023delta}& 0.6897 & 0.2838 & 0.6456 & 0.2996 & 26.02 & 0.2398 \\  
%CDS \cite{nam2024contrastive}& 0.7249 & 0.2485 & 0.7054 & 0.2612 & 27.35 & 0.2099 \\
