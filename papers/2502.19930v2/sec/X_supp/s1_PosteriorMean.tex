\section{Posterior mean analysis}
\label{sec:s_postmean}

To investigate how much identity of the original image $\mathbf{z}$ is contained in the text-conditioned score $\epsilon_{\phi}(\mathbf{z}, y, t)$, we conduct the experiment in which the posterior mean is obtained from various timesteps. As shown in the first row of \cref{fig:post_mean3}, more primary information is damaged as the timestep $t$ increases. On the other hand, when using FPR, since the score $\epsilon_\phi(\mathbf{z}, y, t)$ is modified to preserve the identity of $\mathbf{z}$, we can see that it has more information than before, even at large timestep, as described in the second and third row of \cref{fig:post_mean3}. Note that the score $\epsilon_\phi(\mathbf{z}, y, t)$ can be controlled by updating the injection noise $\epsilon$ or the noisy latent $\mathbf{z}_t$. Of the two options, it has been updated for $\mathbf{z}_t$ because it contains more content details.

%%% [START] Inversion
\begin{figure}[t] % 1-column
\footnotesize
\centering 

% 1st row
\hspace{-2.2mm}
\raisebox{0.24in}{\rotatebox{90}{w/o FPR}}%
\hspace{-1.2mm}
% 1st
\hspace{0mm}
\begin{tikzpicture}[x=4.2cm, y=4.2cm, remember picture, baseline]
    \node[anchor=south, name=labelA] (FigA) at (0,0) {
        \includegraphics[width=0.78in]{Fig./Qual/imgs/posterior_mean/original.png}
    };
    \node[anchor=south, yshift=-1mm] at (FigA.north) {\footnotesize $\textit{Original image} ~\mathbf{z}$};
\end{tikzpicture}\hspace{-1.1mm}%
% 2nd
\begin{tikzpicture}[x=4.2cm, y=4.2cm, remember picture, baseline]
    \node[anchor=south, name=labelB] (FigB) at (0,0) {
        \includegraphics[width=0.78in]{Fig./Qual/imgs/posterior_mean/original_post_mean_step_200.png}
    };
    \node[anchor=south, yshift=-1mm] at (FigB.north) {\footnotesize $\mathbb{E}[\mathbf{z}|\mathbf{z}_{t=200}]$};
\end{tikzpicture}\hspace{-2mm}%
% 3rd
\begin{tikzpicture}[x=4.2cm, y=4.2cm, remember picture, baseline]
    \node[anchor=south, name=labelD] (FigD) at (0,0) {
        \includegraphics[width=0.78in]{Fig./Qual/imgs/posterior_mean/original_post_mean_step_600.png}
    };
    \node[anchor=south, yshift=-1mm] at (FigD.north) {\footnotesize $\mathbb{E}[\mathbf{z}|\mathbf{z}_{t=600}]$};
\end{tikzpicture}\hspace{-2mm}%
% 4th
\begin{tikzpicture}[x=4.2cm, y=4.2cm, remember picture, baseline]
    \node[anchor=south, name=labelF] (FigF) at (0,0) {
        \includegraphics[width=0.78in]{Fig./Qual/imgs/posterior_mean/original_post_mean_step_800.png}
    };
    \node[anchor=south, yshift=-1mm] at (FigF.north) {\footnotesize $\mathbb{E}[\mathbf{z}|\mathbf{z}_{t=800}]$};
\end{tikzpicture}\hspace{-2mm}%

\vspace{-4pt}

% 2st row
\hspace{-2.4mm}
\raisebox{0.12in}{\rotatebox{90}{w/ FPR  w.r.t $\mathbf{z}_t$}}%
\hspace{-1.4mm}
% 1st
\hspace{0mm}
\begin{tikzpicture}[x=4.2cm, y=4.2cm, remember picture, baseline]
    \node[anchor=south, name=labelA1] (FigA1) at (0,0) {
        \includegraphics[width=0.78in]{Fig./Qual/imgs/posterior_mean/original.png}
    };
    % \node[anchor=south, yshift=-1mm] at (FigA.north) {\footnotesize $\textit{original image} \mathbf{z}$};
\end{tikzpicture}\hspace{-1.1mm}%
% 2nd
\begin{tikzpicture}[x=4.2cm, y=4.2cm, remember picture, baseline]
    \node[anchor=south, name=labelB1] (FigB1) at (0,0) {
        \includegraphics[width=0.78in]{Fig./Qual/imgs/posterior_mean/fpr_post_mean_step_200.png}
    };
    % \node[anchor=south, yshift=-1mm] at (FigB.north) {\footnotesize $\mathbb{E}[\mathbf{z}|\mathbf{z}_{t=200}]$};
\end{tikzpicture}\hspace{-2mm}%
% 3rd
\begin{tikzpicture}[x=4.2cm, y=4.2cm, remember picture, baseline]
    \node[anchor=south, name=labelD1] (FigD1) at (0,0) {
        \includegraphics[width=0.78in]{Fig./Qual/imgs/posterior_mean/fpr_post_mean_step_600.png}
    };
    % \node[anchor=south, yshift=-1mm] at (FigD.north) {\footnotesize $\mathbb{E}[\mathbf{z}|\mathbf{z}_{t=600}]$};
\end{tikzpicture}\hspace{-2mm}%
% 4th
\begin{tikzpicture}[x=4.2cm, y=4.2cm, remember picture, baseline]
    \node[anchor=south, name=labelF1] (FigF1) at (0,0) {
        \includegraphics[width=0.78in]{Fig./Qual/imgs/posterior_mean/fpr_post_mean_step_800.png}
    };
    % \node[anchor=south, yshift=-1mm] at (FigF.north) {\footnotesize $\mathbb{E}[\mathbf{z}|\mathbf{z}_{t=800}]$};
\end{tikzpicture}\hspace{-2mm}%

\vspace{-4pt}

% 3rd row
\hspace{-2.2mm}
\raisebox{0.14in}{\rotatebox{90}{w/ FPR w.r.t \ $\epsilon$}}
% \raisebox{0.14in}{\rotatebox{90}{}}%
\hspace{-1.2mm}
% 1st
\hspace{0mm}
\begin{tikzpicture}[x=4.2cm, y=4.2cm, remember picture, baseline]
    \node[anchor=south, name=labelA2] (FigA2) at (0,0) {
        \includegraphics[width=0.78in]{Fig./Qual/imgs/posterior_mean/original.png}
    };
    % \node[anchor=south, yshift=-1mm] at (FigA.north) {\footnotesize $\textit{original image} \mathbf{z}$};
\end{tikzpicture}\hspace{-1.1mm}%
% 2nd
\begin{tikzpicture}[x=4.2cm, y=4.2cm, remember picture, baseline]
    \node[anchor=south, name=labelB2] (FigB2) at (0,0) {
        \includegraphics[width=0.78in]{Fig./Qual/imgs/posterior_mean/fpr_post_mean_step_200_eps.png}
    };
    % \node[anchor=south, yshift=-1mm] at (FigB.north) {\footnotesize $\mathbb{E}[\mathbf{z}|\mathbf{z}_{t=200}]$};
\end{tikzpicture}\hspace{-2mm}%
% 3rd
\begin{tikzpicture}[x=4.2cm, y=4.2cm, remember picture, baseline]
    \node[anchor=south, name=labelD2] (FigD2) at (0,0) {
        \includegraphics[width=0.78in]{Fig./Qual/imgs/posterior_mean/fpr_post_mean_step_600_eps.png}
    };
    % \node[anchor=south, yshift=-1mm] at (FigD.north) {\footnotesize $\mathbb{E}[\mathbf{z}|\mathbf{z}_{t=600}]$};
\end{tikzpicture}\hspace{-2mm}%
% 4th
\begin{tikzpicture}[x=4.2cm, y=4.2cm, remember picture, baseline]
    \node[anchor=south, name=labelF2] (FigF2) at (0,0) {
        \includegraphics[width=0.78in]{Fig./Qual/imgs/posterior_mean/fpr_post_mean_step_800_eps.png}
    };
    % \node[anchor=south, yshift=-1mm] at (FigF.north) {\footnotesize $\mathbb{E}[\mathbf{z}|\mathbf{z}_{t=800}]$};
\end{tikzpicture}\hspace{-2mm}%


% \vspace{-3pt}
\vspace{-10pt}
\caption{\textbf{Posterior mean with/without FPR.} When the prompt $y$ is given by ``portrait of a worried-looking woman in a dress”, the posterior mean $\mathbf{z}_{0|t}$ is obtained $(first$ $row)$ without FPR, $(second$ $row)$ with FPR w.r.t $\mathbf{z}_t$, and $(third$ $row)$ with FPR w.r.t $\epsilon$.}

\label{fig:post_mean3}
\vspace{-10pt}
\end{figure}
