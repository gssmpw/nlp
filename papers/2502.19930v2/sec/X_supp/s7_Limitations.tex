\section{Limitations}
\label{sec:supp_limit}
\noindent\textbf{Success rate.} As discussed in \cref{sec:limit}, %Sec. 7, 
our method optimizes the latents only for source information, resulting in low CLIP scores. To demonstrate it does not mean \textit{"IDS fails to translate the source image"}, we measure the success rate. To calculate the success rate, we classify the transformed images with the pre-trained ResNet classifier on the \textit{Cat-to-dog} task. We treat the results of classified top 1 as the success sample if they belonged to the class of \textit{dog}. The success rate in \cref{tab:success} claims that the low CLIP score of IDS did not fall to convert, but occurred in the process of maintaining the source identity.

% EDIT
\noindent\textbf{Failure case for complex prompt.} Because our method only considers the source information, it struggles with translating the given image for complex text prompts. Although we tried to modify the image with more complex prompts, it failed not only in our method but also in all SDS-based translation methods, as shown in \cref{fig:complex_ex}.
% Fig. 8.
% we try to modify the image with more complex prompts. Although it failed 

\noindent\textbf{Computational overhead.} Our method requires additional computational costs due to repetitive adaptations of FPR for each optimization steps. However, it can be controlled by adjusting hyperparameters such as the number of FPR iterations or the number of optimization steps, as reported in \cref{tab:overhead}. %Tab. 4.

\begin{table}[h]
% \scriptsize
\centering
\begin{tabular}{c|c|c|c}
\hline
& IDS & CDS & DDS  \\ 
\hline
success rate (\%) & 37.40 & 34.87 & 34.03 \\
\hline
\end{tabular}
\vspace{-5pt}
\caption{\textbf{Success rate} for \textit{Cat-to-dog} task. A higher score means more translated results are classified as \textit{dog}.}
\label{tab:success}
\end{table}
% \begin{table}[t]
\centering
\resizebox{0.98\columnwidth}{!}{
\small{
\begin{tabular}{c|cc|cc|c|c}
\hline
& optim step & FPR iter & LPIPS($\downarrow$) & CLIP($\uparrow)$ & time (sec/img) & Memory (GB)  \\ 
\hline
% DDS & \multirow{2}{*}{200} & \multirow{2}{*}{-} & 0.240 & 0.293 & 22.45 & 6.27 \\
%CDS &                      &                    & 0.210 & 0.287 & 59.31 & 8.83 \\
DDS & 200 & - & 0.240 & 0.293 & 22.45 & 6.27 \\
\hline
CDS &  200 &  -  & 0.210 & 0.287 & 59.31 & 8.83 \\
\hline
\multirow{4}{*}{IDS} & \multirow{2}{*}{200} & 1 & 0.199 & 0.285 & 50.80 & \multirow{4}{*}{8.63} \\
                     &                      & 3 & 0.190 & 0.277 & 107.77 & \\
\cline{2-6}
& 100 & \multirow{2}{*}{3} & 0.165 & 0.265 & 54.04 & \\
& 150 &                    & 0.180 & 0.272 & 81.25 & \\
\hline
\end{tabular}
}
}
\vspace{-5pt}
\caption{\textbf{Computational complexity} on 28 images of InstructPix2Pix \cite{brooks2023instructpix2pix} for various settings. Lower LPIPS and higher CLIP scores mean better quality.}
% \vspace{-3pt}
\label{tab:overhead}
\end{table}
% \begin{figure}[H] % 1-column
\centering
\footnotesize

% Adjust the image width so that 5 images fit within one column
\newcommand{\imgwidth}{0.235\linewidth}

% 1st Image
\begin{tikzpicture}[x=1cm, y=1cm]
    \node[anchor=south] (FigA1) at (0,0) {
        \includegraphics[width=\imgwidth]{sec/X_supp/Fig/imgs/complex_ex/source.png}
    };
    \node[anchor=south, yshift=-1mm] at (FigA1.north) {\footnotesize Source};
\end{tikzpicture}\hspace{-1mm}%
% 2nd Image
\begin{tikzpicture}[x=1cm, y=1cm]
    \node[anchor=south] (FigB1) at (0,0) {
        \includegraphics[width=\imgwidth]{sec/X_supp/Fig/imgs/complex_ex/ids.jpg}
    };
    \node[anchor=south, yshift=-1mm] at (FigB1.north) {\footnotesize \textbf{IDS (Ours)}};
\end{tikzpicture}\hspace{-1mm}%
% 3rd Image
\begin{tikzpicture}[x=1cm, y=1cm]
    \node[anchor=south] (FigC1) at (0,0) {
        \includegraphics[width=\imgwidth]{sec/X_supp/Fig/imgs/complex_ex/cds.jpg}
    };
    \node[anchor=south, yshift=-1mm] at (FigC1.north) {\footnotesize CDS};
\end{tikzpicture}\hspace{-1mm}%
% 4th Image
\begin{tikzpicture}[x=1cm, y=1cm]
    \node[anchor=south] (FigD1) at (0,0) {
        \includegraphics[width=\imgwidth]{sec/X_supp/Fig/imgs/complex_ex/dds.jpg}
    };
    \node[anchor=south, yshift=-1mm] at (FigD1.north) {\footnotesize DDS};
\end{tikzpicture}\hspace{-1mm}%

\vspace{-3pt}
\setulcolor{magenta}
\setul{0.3pt}{2pt}
\centering
\textit{``Photo free night, house, aurora" $\to$ ``... \ul{with two dogs}"} 
\vspace{-10pt}
\caption{\textbf{Failure case} for complex text prompt.}
\vspace{-8pt}
\label{fig:complex_ex}
\end{figure}

