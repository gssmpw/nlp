\section{Limitation}
\label{sec:limit}
%The proposed IDS effectively preserves source's identity, including pose, structure and background, by iteratively updating the source latent $\srct$ according to the given source information. As represented in \ref{sec:results}, 
% The proposed IDS demonstrates outstanding performance across evaluation metrics assessing consistency between source and target images. However, during FPR process, IDS relies solely on information from the source ($\{ \src, y^{\text{src}} \}$) without incorporating target-side information. This results in comparatively lower CLIP scores~\cite{radford2021learning} than other baselines as reported in Tab.~\ref{tab:limitation} and failure cases for more complex translations as shown in. In addition, our method requires additional computational overhead since FPR is applied to each optimization iteration. Detailed discussion about our limitation is provided in Supp.
The proposed IDS demonstrates outstanding performance across evaluation metrics assessing consistency between source and target images. However, during FPR process, IDS relies solely on information from the source ($\{ \src, y^{\text{src}} \}$) without incorporating target-side information. This results in comparatively lower CLIP scores~\cite{radford2021learning} than other baselines (\cref{tab:limitation}) and failure cases for more complex translations (\cref{fig:complex_ex}). In addition, our method requires additional computational overhead (\cref{tab:overhead}) since FPR is applied to each optimization iteration. Detailed discussion about our limitation is provided in \cref{sec:supp_limit} of supplementary.
Our future direction will explore changing the score conditioned by the target prompt $y^{\text{trg}}$, leading to a better alignment with $y^{\text{trg}}$.
% which is iteratively obtained by \cref{eq:update}. The equation only uses information from a source (prompt \& image), not from a target side. This fact hampers text-driven representations of a target prompt leading to lower CLIP scores \cite{radford2021learning} compared with other baselines.
\begin{table}[thb!]
\centering
\resizebox{0.95 \columnwidth}{!}{
\normalsize{
\begin{tabular}{c|c c c c | c}
\hline
&  P2P \cite{hertzprompt} & PnP \cite{tumanyan2023plug} & DDS \cite{hertz2023delta} & CDS \cite{nam2024contrastive} & \textbf{IDS (Ours)} \\
\hline
\textbf{cat2lion} & 0.29 & 0.21 & \textbf{0.30} & 0.29 & 0.29 \\
\textbf{cat2dog} & \textbf{0.27} & 0.26 & \textbf{0.27} & \textbf{0.27} & 0.26\\
\textbf{Ip2p}  & 0.28 & \textbf{0.30} & 0.29 & 0.29 & 0.28\\
\hline
\end{tabular}
}
}
\vspace{-5pt}
\caption{\textbf{Limitation of IDS with respect to CLIP score\cite{radford2021learning}} for image editing on LAION 5B \cite{schuhmann2022laion} and InstructPix2Pix \cite{brooks2023instructpix2pix}.} %A higher value means better performance.}
\vspace{-20pt}
\label{tab:limitation}
\end{table}


%P2P \cite{hertzprompt}& 0.2941 & \textbf{0.2689} & 0.2797 \\
%PnP \cite{tumanyan2023plug}& 0.2096 & 0.2621 & \textbf{0.3038} \\
%DDS \cite{hertz2023delta}& \textbf{0.2972} & 0.2677 & 0.2930 \\  
%CDS \cite{nam2024contrastive}& 0.2940 & 0.2667 & 0.2871 \\
%\hline
%\textbf{IDS (Ours)} & 0.2870 & 0.2629 & 0.2774 \\

\begin{figure}[H] % 1-column
\centering
\footnotesize

% Adjust the image width so that 5 images fit within one column
\newcommand{\imgwidth}{0.235\linewidth}

% 1st Image
\begin{tikzpicture}[x=1cm, y=1cm]
    \node[anchor=south] (FigA1) at (0,0) {
        \includegraphics[width=\imgwidth]{sec/X_supp/Fig/imgs/complex_ex/source.png}
    };
    \node[anchor=south, yshift=-1mm] at (FigA1.north) {\footnotesize Source};
\end{tikzpicture}\hspace{-1mm}%
% 2nd Image
\begin{tikzpicture}[x=1cm, y=1cm]
    \node[anchor=south] (FigB1) at (0,0) {
        \includegraphics[width=\imgwidth]{sec/X_supp/Fig/imgs/complex_ex/ids.jpg}
    };
    \node[anchor=south, yshift=-1mm] at (FigB1.north) {\footnotesize \textbf{IDS (Ours)}};
\end{tikzpicture}\hspace{-1mm}%
% 3rd Image
\begin{tikzpicture}[x=1cm, y=1cm]
    \node[anchor=south] (FigC1) at (0,0) {
        \includegraphics[width=\imgwidth]{sec/X_supp/Fig/imgs/complex_ex/cds.jpg}
    };
    \node[anchor=south, yshift=-1mm] at (FigC1.north) {\footnotesize CDS};
\end{tikzpicture}\hspace{-1mm}%
% 4th Image
\begin{tikzpicture}[x=1cm, y=1cm]
    \node[anchor=south] (FigD1) at (0,0) {
        \includegraphics[width=\imgwidth]{sec/X_supp/Fig/imgs/complex_ex/dds.jpg}
    };
    \node[anchor=south, yshift=-1mm] at (FigD1.north) {\footnotesize DDS};
\end{tikzpicture}\hspace{-1mm}%

\vspace{-3pt}
\setulcolor{magenta}
\setul{0.3pt}{2pt}
\centering
\textit{``Photo free night, house, aurora" $\to$ ``... \ul{with two dogs}"} 
\vspace{-10pt}
\caption{\textbf{Failure case} for complex text prompt.}
\vspace{-8pt}
\label{fig:complex_ex}
\end{figure}
