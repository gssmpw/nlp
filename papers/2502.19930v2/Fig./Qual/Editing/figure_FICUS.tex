%%% [START] NeRF Synthetic data Results 
\begin{figure*}[t] % 2-column
\footnotesize
\centering 
% 1st row
\hspace{-3mm}
\raisebox{0.5in}{\rotatebox{90}{\textbf{Synthetic} \cite{mildenhall2021nerf}}}%
\hspace{3mm}%
\begin{tikzpicture}[x=3.5cm, y=3.5cm, spy using outlines={every spy on node/.append style={thick, draw=red}}]
\node[anchor=south] (FigA) at (0,0) {\includegraphics[trim=0 0 0 0 ,clip,width=1.5in]{Fig./Qual/imgs/3D/ficus/cropped_r_3.png}};
\node[anchor=south, yshift=0mm] at (FigA.north) {\footnotesize Source};
% ->
\draw[->, line width=0.8mm, color=red, shorten >=1pt, shorten <=1pt] ($(FigA.center) + (0.15, -0.18)$) -- ($(FigA.center) + (0, -0.3)$);
\end{tikzpicture}
\hspace{-1mm}
\begin{tikzpicture}[x=3.5cm, y=3.5cm, spy using outlines={every spy on node/.append style={thick, draw=red}}]
\node[anchor=south] (FigD) at (0,0) {\includegraphics[trim=0 0 0 0 ,clip,width=1.5in]{Fig./Qual/imgs/3D/ficus/FPDS_cropped_r_3.png}};
\node[anchor=south, yshift=0mm] at (FigD.north) {\footnotesize \textbf{IDS (Ours)}};
% ->
\draw[->, line width=0.8mm, color=red, shorten >=1pt, shorten <=1pt] ($(FigA.center) + (0.15, -0.18)$) -- ($(FigA.center) + (0, -0.3)$);
\end{tikzpicture}
\hspace{-1mm}
\begin{tikzpicture}[x=3.5cm, y=3.5cm, spy using outlines={every spy on node/.append style={thick, draw=red}}]
\node[anchor=south] (FigC) at (0,0) {\includegraphics[trim=0 0 0 0 ,clip,width=1.5in]{Fig./Qual/imgs/3D/ficus/CDS_cropped_r_3.png}};
\node[anchor=south, yshift=0mm] at (FigC.north) {\footnotesize CDS};
% ->
\draw[->, line width=0.8mm, color=red, shorten >=1pt, shorten <=1pt] ($(FigA.center) + (0.15, -0.18)$) -- ($(FigA.center) + (0, -0.3)$);
\end{tikzpicture}
\hspace{-1mm}
\begin{tikzpicture}[x=3.5cm, y=3.5cm, spy using outlines={every spy on node/.append style={thick, draw=red}}]
\node[anchor=south] (FigB) at (0,0) {\includegraphics[trim=0 0 0 0 ,clip,width=1.5in]{Fig./Qual/imgs/3D/ficus/DDS_cropped_r_3.png}};
\node[anchor=south, yshift=0mm] at (FigB.north) {\footnotesize DDS};
% ->
\draw[->, line width=0.8mm, color=red, shorten >=1pt, shorten <=1pt] ($(FigA.center) + (0.15, -0.18)$) -- ($(FigA.center) + (0, -0.3)$);
\end{tikzpicture}

\vspace{-4pt}

\setulcolor{magenta}
\setul{0.3pt}{2pt}
\centering \textit{``A tree in a brown vase" $\to$ ``A tree in a \ul{blue} vase"} 

\vspace{-2pt}

% 2nd row
\hspace{-3mm}
\raisebox{0.37in}{\rotatebox{90}{\textbf{LLFF} \cite{mildenhall2019local} }}%
\hspace{3mm}%
\begin{tikzpicture}[x=3.5cm, y=3.5cm, spy using outlines={every spy on node/.append style={thick, draw=white}}]
\node[anchor=south] (FigA2) at (0,0) {\includegraphics[trim=0 0 0 0 ,clip,width=1.5in]{Fig./Qual/imgs/3D/autumn/original_image009.jpg}};
\spy [magnification=3, size=0.6in] on ($(FigA2.center) + (0.05, 0.05)$) in node [anchor=south west] at ($(FigA2.south west)$);
\end{tikzpicture}
\hspace{-1mm}
\begin{tikzpicture}[x=3.5cm, y=3.5cm, spy using outlines={every spy on node/.append style={thick, draw=white}}]
\node[anchor=south] (FigD2) at (0,0) {\includegraphics[trim=0 0 0 0 ,clip,width=1.5in]{Fig./Qual/imgs/3D/autumn/FPDS_4032_IMG_3006.jpg}};
\spy [magnification=3, size=0.6in] on ($(FigD2.center) + (0.05, 0.05)$) in node [anchor=south west] at ($(FigD2.south west)$);
\end{tikzpicture}
\hspace{-1mm}
\begin{tikzpicture}[x=3.5cm, y=3.5cm, spy using outlines={every spy on node/.append style={thick, draw=white}}]
\node[anchor=south] (FigC2) at (0,0) {\includegraphics[trim=0 0 0 0 ,clip,width=1.5in]{Fig./Qual/imgs/3D/autumn/CDS_4032_IMG_3006.jpg}};
\spy [magnification=3, size=0.6in] on ($(FigC2.center) + (0.05, 0.05)$) in node [anchor=south west] at ($(FigC2.south west)$);
\end{tikzpicture}
\hspace{-1mm}
\begin{tikzpicture}[x=3.5cm, y=3.5cm, spy using outlines={every spy on node/.append style={thick, draw=white}}]
\node[anchor=south] (FigB2) at (0,0) {\includegraphics[trim=0 0 0 0 ,clip,width=1.5in]{Fig./Qual/imgs/3D/autumn/DDS_4032_IMG_3006.jpg}};
\spy [magnification=3, size=0.6in] on ($(FigB2.center) + (0.05, 0.05)$) in node [anchor=south west] at ($(FigB2.south west)$);
\end{tikzpicture}

% 3rd row
\hspace{-3mm}
\raisebox{0.3in}{\rotatebox{90}{\textbf{Depth Map}}}%
\hspace{3mm}%
\hspace{0mm}
\begin{tikzpicture}[x=3.5cm, y=3.5cm, spy using outlines={every spy on node/.append style={thick, draw=white}}]
\node[anchor=south] (FigA3) at (0,0) {\includegraphics[trim=0 0 0 0 ,clip,width=1.5in]{Fig./Qual/imgs/3D/autumn/depth_map/original_depth_088.jpg}};
\end{tikzpicture}
\hspace{-1mm}
\begin{tikzpicture}[x=3.5cm, y=3.5cm, spy using outlines={every spy on node/.append style={thick, draw=white}}]
\node[anchor=south] (FigD3) at (0,0) {\includegraphics[trim=0 0 0 0 ,clip,width=1.5in]{Fig./Qual/imgs/3D/autumn/depth_map/FPDS_depth_088.jpg}};
\end{tikzpicture}
\hspace{-1mm}
\begin{tikzpicture}[x=3.5cm, y=3.5cm, spy using outlines={every spy on node/.append style={thick, draw=white}}]
\node[anchor=south] (FigC3) at (0,0) {\includegraphics[trim=0 0 0 0 ,clip,width=1.5in]{Fig./Qual/imgs/3D/autumn/depth_map/CDS_depth_088.jpg}};
\end{tikzpicture}
\hspace{-1mm}
\begin{tikzpicture}[x=3.5cm, y=3.5cm, spy using outlines={every spy on node/.append style={thick, draw=white}}]
\node[anchor=south] (FigB3) at (0,0) {\includegraphics[trim=0 0 0 0 ,clip,width=1.5in]{Fig./Qual/imgs/3D/autumn/depth_map/DDS_depth_088.jpg}};
\end{tikzpicture}

\vspace{-1pt}
\centering \textit{``The green leaves" $\to$ ``\ul{Yellow and red} leaves in \ul{autumn}"} 

\vspace{-5pt}
\caption{\textbf{Qualitative results on Synthetic 360$^\circ$ and LLFF datasets.} IDS outperforms the baselines by preserving the structural consistency of the source image and maintaining the integrity of regions that should remain unchanged, while precisely editing only the areas specified by the target prompt. Furthermore, comparisons of the depth map results also highlight the superior consistency of our method over other baseline models.}
\label{fig:ficus_qual}
\end{figure*}
% \vspace{-10pt}