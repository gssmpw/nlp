
\section{Constructions of Light Fault-Tolerant Spanners} \label{sec:upper-main}

\subsection{The Weighted Girth Framework} \label{sec:weighted-girth}

Elkin, Neiman, and Solomon \cite{ENS14} introduced a weighted analog of girth, for use in studying the lightness of (non-faulty) spanners:

\begin{definition} [Normalized Weight and Weighted Girth \cite{ENS14}]
For a cycle $C$ in a weighted graph $G$, the \emph{normalized weight} of $C$ is the quantity
$$w^*(C) := \frac{w(C)}{\max_{e \in C} w(e)}.$$
The \emph{weighted girth} of $G$ is the minimum normalized weight over its cycles.
\end{definition}

They then proved an equivalence between light spanners and the maximum possible lightness of a graph of high weighted girth.
In particular:
\begin{definition} [Extremal Lightness of Weighted Girth]
We write $\lambda(n, k) := \sup \ell(G)$, 
where the $\sup$ is taken over $n$-node graphs $G$ of weighted girth $>k$.
\end{definition}

\begin{theorem} [\cite{ENS14}] \label{thm:ENS}
For all positive integers $n$ and all $k$, every $n$-node graph $G$ has a $k$-spanner $H$ of lightness
$\ell(H \mid G) \le \lambda(n, k+1)$,
and this is existentially tight.\footnote{More specifically, this means that for any $\eps > 0$, there exists an $n$-node graph on which any $k$-spanner $H$ has lightness $\ell(H \mid G) > \lambda(n, k+1) - \eps$.}
\end{theorem}

Settling the asymptotic value of $\lambda$ is a major open problem.
Currently, the following bounds are known:

\begin{theorem} [\cite{LS23, Bodwin25, BF25}] \label{thm:priorlight}
For all positive integers $n, k$ and all $\eps > 0$, we have
$\lambda(n, (1+\eps)2k) \le O\left(\eps^{-1} n^{1/k} \right)$.  When $k$ is a constant and $\eps = \Theta(n^{-\frac{1}{2k-1}})$, we have
$\lambda(n, (1+\eps)2k) \ge \Omega\left(\eps^{-1/k} n^{1/k} \right)$.
\end{theorem}

\subsection{The Greedy Algorithm and Blocking Sets}

We will analyze spanners that arise from a variant of the fault-tolerant greedy algorithm, introduced in \cite{BDPV18} and used in many recent papers on fault tolerance.
The only difference algorithmically is that we seed the spanner with a min-weight connectivity preserver, whereas prior work immediately enters the main loop.

\begin{algorithm}
\DontPrintSemicolon

\textbf{Input:} Graph $G = (V, E, w)$, stretch $k$, fault tolerance $f$\;~\\

Let $Q \gets $ min-weight $2f$-FT connectivity preserver of $G$\;
Let $H \gets Q$ be the initial spanner\;
\ForEach{edge $(u, v) \in E(G) \setminus Q$ in order of nondecreasing weight}{
    \If{there exists $F \subseteq E(H), |F| \le f$ such that $\dist_{H \setminus F}(u, v) > k \cdot w(u, v)$}{
        add $(u, v)$ to $H$\;
    }
}
\textbf{Return} $H$\;
\caption{\label{alg:ftgreedy} Light Fault-Tolerant Greedy Spanner Algorithm}
\end{algorithm}

The proof of correctness of the algorithm is standard:
\begin{theorem} \label{thm:feasible}
The output spanner $H$ from Algorithm \ref{alg:ftgreedy} is an $f$-EFT $k$-spanner of the input graph $G$.
\end{theorem}
\begin{proof}
As is well known in the spanners literature (e.g.~\cite{ADDJS93}), it suffices to verify that for any set $F$ of $|F| \le f$ edge faults and any edge $(u, v) \in E(G) \setminus (E(H) \cup F)$, we have
$$\dist_{H \setminus F}(u, v) > k \cdot w(u, v).$$
To show this, we first observe that for any such edge we have $(u, v) \notin Q$ (since we add all of $Q$ to $H$), and so we consider $(u, v)$ at some point in the main for-loop of the algorithm.
We choose not to add $(u, v)$ to $H$, meaning that for all possible fault sets $F$ we have $\dist_{H \setminus F}(u, v) \le k \cdot w(u, v)$.
For the rest of the algorithm we only add edges to $F$, and since this property is monotonic in $H$, it will still hold in the final spanner.
\end{proof}

Now we turn to analyzing lightness.
When $f=0$, this algorithm produces a spanner $H$ of weighted girth $>k+1$ \cite{ENS14}.
For larger $f$, there is not much we can say about the weighted girth of the output spanner, which could be quite small.
However, we might intuitively expect the output spanner to be ``structurally close'' to a graph of high weighted girth.
That notion of closeness can be formalized by adapting the blocking set framework, which has been used in many recent papers on fault-tolerant spanners and related objects (see, e.g., \cite{BP19, DR20, BDR21, BDR22,BDN22,BDN23, PST24, BHP24}).


\begin{definition} [Edge-Blocking Sets]
For a graph $H$, an \emph{edge-blocking set} is a set of ordered edge pairs $B \subseteq E(H) \times E(H)$.
We say that $B$ \emph{blocks} a cycle $C$ if there is a pair $(e_1, e_2) \in B$ with $e_1, e_2 \in C$.
We say that $B$ is \emph{$f$-capped} if for all $e \in E(H)$, there are at most $f$ edges $e'$ with $(e, e') \in E(H)$.\footnote{There may be additional pairs of the form $(e', e) \in E(H)$ -- this property only bounds the number of pairs that have $e$ first.}
\end{definition}

The following is a tweak on a standard lemma bounding the size of the blocking set of the output spanner; the proof is a straightforward mixture of related lemmas from \cite{BP19} and \cite{ENS14}.

\begin{lemma} \label{lem:bsetexists}
The output spanner $H$ from the fault-tolerant greedy algorithm has an $f$-capped edge-blocking set $B$ that blocks all cycles $C$ that contain at least one edge from $G \setminus Q$ and which have normalized weight $w^*(C) \le k+1$.
Additionally, for every pair $(e, e') \in B$, the first edge $e$ is not in $Q$.
\end{lemma}
\begin{proof}
In the main part of the construction, whenever we add an edge $(u, v) \in E(G) \setminus Q$ to the spanner $H$, we do so because there exists a fault set $F_{(u, v)}$ that satisfies the conditional (there might be many possible choices of $F_{(u, v)}$, in which case we fix one arbitrarily).
Let
$$B := \left\{ (e, e') \ \mid \ e \in E(H) \setminus Q, e' \in F_e\right\}.$$
Since each $|F_e| \le f$, we have that $B$ is $f$-capped.
To argue its cycle-blocking properties, let $C$ be a cycle as in the lemma statement, and let $(u, v) \in C$ be the last edge in $C$ to be considered in the main loop of the algorithm.
Just before $(u, v)$ is added to $H$, there exists a $u \leadsto v$ path through the other edges of the cycle, and so we have
\begin{align*}
\dist_H(u, v) &\le w(C) - w(u, v) \le (k+1) \cdot w(u, v) - w(u, v) = k \cdot w(u, v).
\end{align*}
However, in order to satisfy the conditional and include $(u, v)$ in $H$, we must have
$\dist_{H \setminus F_{(u, v)}}(u, v) > k \cdot w(u, v)$.
Thus at least one edge $e \in C \setminus \{(u, v)\}$ must be included in $F_{(u, v)}$.
So we have $\{(u, v), e\} \in B$, which blocks $C$, completing the proof.
\end{proof}


\subsection{Warmup: Slow Suboptimal Competitive Lightness Upper Bounds} \label{sec:warmup}

As a warmup, we will prove:
\begin{theorem} [Warmup] \label{thm:mainbadf}
For all input graphs $G$, the output spanner $H$ from Algorithm \ref{alg:ftgreedy} satisfies
$$\ell_{2f}(H \mid G) \le O\left( f \cdot \lambda(n, k+1) \right).$$
\end{theorem}

We will later improve the dependence on $f$, and show a variant that runs in polynomial time.  But it is easier to demonstrate our main ideas with this simpler result. 

\paragraph{Extensions of Nash-Williams Tree Decompositions.}

To begin the proof, we recall the classic Nash-Williams tree packing theorem:
\begin{theorem} [Nash-Williams \cite{NashWilliams}] \label{thm:NW}
For all positive integers $f$, a multigraph $G$ contains $f$ edge disjoint spanning trees if and only if for every vertex partition $\mathcal{P}$, there are at least $f(|\mathcal{P}|-1)$ edges between parts.
\end{theorem}

The following is a simple corollary.  The actual packing that we will use is a bit more complicated, but we give this simpler corollary to build intuition. 

\begin{corollary} \label{cor:nwdecomp}
For all positive integers $f$, an $f$-connected graph $G$ contains a collection of $f$ spanning trees with the property that any edge of $G$ is in at most two of the trees.
\end{corollary}
\begin{proof}
Replace every edge of $G$ with two parallel edges, so that it is $2f$-connected.
Then consider any vertex partition $\mathcal{P}$, and note that (by connectivity) every part must have at least $2f$ outgoing edges.
Thus there are at least $f \cdot |\mathcal{P}|$ edges between parts.  
This is even a bit stronger than the condition needed to apply the Nash-Williams theorem, and so by Theorem~\ref{thm:NW} we can partition $G$ into $f$ edge-disjoint spanning trees.  Since we initially doubled the edges of $G$, this means that every original edge participates in at most two spanning trees.
\end{proof}

If the input graph $G$ is $(2f+1)$-connected (and therefore $Q$ is also $(2f+1)$-connected), then this corollary would give the right technical tool.
However, in order to avoid assuming any connectivity properties of the input graph, we instead need a generalization.
We begin with the following theorem by Chekuri and Shepherd, which gives the appropriate analog in the special case of Eulerian graphs:
\begin{theorem} [\cite{chekuri2009approximate}, c.f.\ Theorem 3.1]\label{lem:pairwisenw}
For all positive integers $f$ and Eulerian graphs $G$, there exist edge-disjoint forests $\{F_1, \dots, F_f\}$ such that every $2f$-connected component of $G$ is connected in all forests.
Moreover, these forests can be constructed in $\text{poly}(n)$ time.
\end{theorem}

We will use the following corollary:
\begin{corollary} \label{cor:pairwisenw}
For all positive integers $f$ and graphs $G$, there exists a collection of subtrees $\tee$ such that (1) every edge in $E(G)$ is in at most two trees in $\tee$, and (2) for every $f$-connected component $C$ of $G$, there are at least $f$ trees in $\tee$ in which $C$ is connected.  
\end{corollary}
\begin{proof}
Identical to the previous corollary.
Note that when we double edges, every node has even degree, so we guarantee that the graph is Eulerian and so can apply Theorem~\ref{lem:pairwisenw}.\footnote{Technically this edge-doubling step makes $G$ a multigraph, which is not explicitly mentioned in the Chekuri-Shepherd theorem \cite{chekuri2009approximate}.  However, their proof extends straightforwardly to multigraphs.}
\end{proof}

The first step in our analysis of Algorithm \ref{alg:ftgreedy} is to apply this corollary to $Q$, with parameter $2f+1$, yielding a collection of subtrees $\tee$.

\paragraph{Construction of $H[T]$ Subgraphs.} Our next step is to construct a sequence of three subgraphs $H[T] \supseteq H'[T] \supseteq H''[T]$ associated to each tree $T \in \tee$.
In the following let $B$ be an $f$-capped blocking set for $H$ as in Lemma \ref{lem:bsetexists}.

\begin{itemize}
\item \textbf{(Construction of $H[T]$)}
Consider the edges in $(u, v) \in E(H) \setminus Q$ one at a time.
Notice that there cannot exist $2f$ edge faults in $Q$ that disconnect the nodes $u, v$, since otherwise we would need to include the edge $(u, v)$ in $Q$.
Thus the node pair $(u, v)$ is $(2f+1)$-connected in $Q$.
So the endpoint nodes $u, v$ lie in the same $(2f+1)$-connected component $C$ of $Q$, and by Corollary \ref{cor:pairwisenw} there exist $2f+1$ trees in $\tee$ that all span $C$.

Since $B$ is $f$-capped, there are at most $f$ edges $e'$ with $(e, e') \in B$.
Since each such edge $e'$ can be in at most two trees (by Corollary~\ref{cor:pairwisenw}), it follows that there exists at least one tree $T \in \tee$ for which no such edge $e'$ is in $T$ (if there are multiple such trees, choose one arbitrarily).
We choose such a tree $T$ and say that it \emph{hosts} $e$.
Then, for all $T \in \tee$, let $H[T]$ be the graph that contains $T$ and all edges hosted by $T$.

\item \textbf{(Construction of $H'[T]$)}
For each tree $T$, we construct $H'[T]$ by keeping every edge in $T$ deterministically, and keeping every edge in $H \setminus T$ independently with probability $1/f$.

\item \textbf{(Construction of $H''[T]$)}
Finally, to construct $H''[T]$ from $H'[T]$, for every pair $(e, e') \in B$ with $e, e' \in H'[T]$, we delete the first edge $e$, and let $H''[T]$ be the remaining subgraph.
\end{itemize}

The important properties of these subgraphs are:
\begin{lemma} [Subgraph Properties] 
Every graph $H''[T]$ has weighted girth $>k+1$ (deterministically), and expected weight
$$\mathbb{E}\left[w(H''[T])\right] \ge \Omega\left( \frac{w(H[T])}{f} \right).$$
\end{lemma}
\begin{proof}
First, let $C$ be any cycle in $H[T]$ of normalized weight $\le k+1$.  We will argue that $C$ does not survive in $H''[T]$.
Note that $C$ contains at least one edge from $E(H) \setminus Q$, since the only edges from $Q$ in $H[T]$ are those in $T$ itself, which is a tree.
Thus, by the properties of the blocking set $B$ from Lemma \ref{lem:bsetexists}, there is $(e, e') \in B$ with $e, e' \in C$ and $e \notin T$.
It is not possible for both edges $e, e'$ to survive in $H''[T]$, since if they are both selected to remain in $H'[T]$ then we will choose to delete $e$ when we construct $H''[T]$.
Thus $C$ does not survive in $H''[T]$.

Next, we measure the expected weight in $H''[T]$.
Every edge in $T$ is included in $H''[T]$ deterministically.
Every edge $e \in E(H[T]) \setminus T$ survives in $H''[T]$ iff the following two independent events both occur:
\begin{itemize}
\item $e$ is selected to be included in $H'[T]$, which happens with probability $1/f$, and
\item For all pairs $(e, e') \in B$ with $e$ first, the other edge $e'$ is \emph{not} selected to be included in $H'[T]$.
Since $B$ is $f$-capped, there are at most $f$ such edges $e'$ to consider.
Additionally, since $T$ hosts $e$, none of these edges $e'$ are in $T$ itself.
Thus we select them each to be included in $H'[T]$ independently with probability $1/f$.
Since we have $f$ independent events that each occur with probability $1 - 1/f$, the probability that none of them occur is at least $1/4$.
\end{itemize}
Thus all edges $e \in E(H[T])$ survive in $H''[T]$ with probability $\Omega(1/f)$, implying the lemma.
\end{proof}

\paragraph{Analysis of Lightness.}

As in the probabilistic method, there exists a realization of the subgraphs $H''[T]$ that satisfies the expected weight inequality in the previous lemma (for all $T$).
Using this, we observe that for all trees $T \in \tee$, we have 
\begin{align}
\frac{w(H[T])}{w(T)} &\le \frac{O\left(f \cdot w(H''[T])\right)}{w(T)} \le O(f) \cdot \frac{w(H''[T])}{w(\mst(H''[T]))} = O(f) \cdot \ell(H''[T]) \notag \\ 
&= O(f) \cdot \lambda(n, k+1) \label{eq:single-bound}
\end{align}
where the last inequality follows from the definition of $\lambda$ and the previous lemma establishing that $H''[T]$ has weighted girth $>k+1$.
Thus, to wrap up the proof, we bound:
\begin{align*}
\ell_{2f}(H \mid G) &= \frac{w(H)}{w(Q)} \leq \frac{\sum \limits_{T \in \tee} w(H[T])}{w(Q)} \tag{every $e \in E(H)$ is in at least one $H[T]$} \\ 
&\leq 2 \cdot \frac{\sum \limits_{T \in \tee} w(H[T])}{\sum \limits_{T \in \tee} w(T)} \tag{Corollary~\ref{cor:nwdecomp} and def of $\tee$}\\
&\le  O\left( \frac{\sum \limits_{T \in \tee} f \cdot \lambda(n, k+1) \cdot w(T)}{\sum \limits_{T \in \tee} w(T)} \right) \tag{Eq.~\eqref{eq:single-bound}}\\
&= O(f \cdot \lambda(n, k+1)).
\end{align*}

\subsection{Improved Lightness Bounds via Multiple Host Trees} \label{sec:multiple-host}

One way in which our previous analyses are suboptimal is that they do not acknowledge the possibility that an edge could be hosted by many trees, not just one.
As a simple example to introduce the technique, let us observe that the lightness bound from Algorithm \ref{alg:ftgreedy} improves by a factor of $f$ if we are willing to take a higher lightness competition parameter:
\begin{theorem} \label{thm:high-parameter-lightness}
Suppose we modify Algorithm \ref{alg:ftgreedy} to instead construct $Q$ as a min-weight $(2+\eta)f$-FT connectivity preservers, for some $\eta >0$.
Then the output $f$-EFT spanner $H$ has $(2+\eta)f$-competitive lightness
$$ \ell_{(2+\eta)f}(H \mid G) \le O\left( \eta^{-1} \lambda(n, k+1)\right).$$
\end{theorem}
\begin{proof}
Follow the previous analysis, but note that our guarantee is now that we have a set of trees $\tee$ such that (1) every edge is in at most two trees, and (2) for all $(u, v) \in E(G) \setminus Q$, there are $(2+\eta)f+1$ trees in $\tee$ that span the $(2+\eta)f+1$-connected component that contains $u, v$.
Thus, letting $F$ be a fault set that caused us to add $(u, v)$ to the spanner, there are $\Theta(\eta f)$ of these trees that are disjoint from $F$.
We let \textbf{each} of these trees host the edge $(u, v)$.
In particular, this means that $(u, v)$ will appear in $\Theta(f)$ many $H[T]$ subgraphs.
From there the analysis continues as before, but our final calculation is:
\begin{align*}
\ell_{(2+\eta)f}(H \mid G) &= \frac{w(H)}{w(Q)}\\
&\leq O\left(\frac{(\eta f)^{-1} \sum \limits_{T \in \tee} w(H[T])}{w(Q)}\right) \tag{every $e \in E(H)$ is in $\Theta(\eta f)$ many $H[T]$} \\ 
&\leq O\left(\frac{(\eta f)^{-1} \sum \limits_{T \in \tee} w(H[T])}{\sum \limits_{T \in \tee} w(T)}\right) \tag{Corollary~\ref{cor:nwdecomp} and def of $\tee$}\\
&\le  O\left( \frac{(\eta f)^{-1} \sum \limits_{T \in \tee} f \cdot \lambda(n, k+1) \cdot w(T)}{\sum \limits_{T \in \tee} w(T)} \right) \tag{Eq.~\eqref{eq:single-bound}}\\
&= O(\eta^{-1} \cdot \lambda(n, k+1)). \tag*{\qedhere}
\end{align*}
\end{proof}

If we don't wish to harm our competition parameter, we can use a related method to improve the dependence to $f^{1/2}$:
\begin{theorem} \label{thm:warmupgoodf}
The output spanner $H$ from Algorithm \ref{alg:ftgreedy} (with no modifications) satisfies
$$\ell_{2f}(H \mid G) \le O\left( f^{1/2} \cdot \lambda(n, k+1) \right).$$
\end{theorem}
\begin{proof}
Let us say that an edge $e \in E(H) \setminus Q$ is \emph{$Q$-heavy} if there are at least $f - f^{1/2}$ different edges $e' \in Q$ with $(e, e') \in B$.
Otherwise, we say that $e$ is \emph{$Q$-light}.
Let $H_{heavy} \subseteq H$ be the subgraph that includes $Q$ and all edges in $E(H) \setminus Q$ are $Q$-heavy, and similarly let $H_{light} \subseteq H$ be the subgraph that includes $Q$ and all edges in $E(H) \setminus Q$ that are $Q$-light.
Notice that
\begin{align*}
\ell_{2f}(H) &= \frac{w(H \setminus Q) + w(Q)}{w(Q)} \\
&= \frac{w(H_{heavy} \setminus Q) + w(H_{light} \setminus Q) + w(Q)}{w(Q)} \\
&= \frac{w(H_{heavy})}{w(Q)} + \frac{w(H_{light})}{w(Q)} - 1\\
&< \ell_{2f}(H_{heavy}) + \ell_{2f}(H_{light}).
\end{align*}
Thus it suffices to individually bound $\ell_{2f}(H_{heavy}), \ell_{2f}(H_{light})$.

\paragraph{Bounding $\ell_{2f}(H_{heavy})$.}

Follow the same argument as before, but observe that any $Q$-heavy edge $e$ can be blocked with at most $f^{1/2}$ other edges $e'$ in its relevant subgraph $H[T]$.
Thus we may generate $H'[T]$ from $H[T]$ by keeping all edges in $T$ deterministically (as before), but by sampling the edges in $H[T] \setminus T$ with probability only $1/f^{1/2}$ instead of $1/f$.
This leads to a new bound of
$$\mathbb{E}\left[w(H''[T])\right] \ge \Omega\left( \frac{w(H[T])}{f^{1/2}} \right),$$
and following the same analysis from there, this improved factor of $f^{1/2}$ propagates into the final lightness bound.

\paragraph{Bounding $\ell_{2f}(H_{light})$.}

Follow the same argument as before, but observe that for any $Q$-light edge $e$, there are at least 
$$2f+1 - 2(f - f^{1/2}) \ge f^{1/2} +1$$ 
different trees that span the $(2f+1)$-connected component containing the endpoints of $e$, and which do not contain edges $e'$ with $(e, e') \in B$.
We let $e$ be hosted by \textbf{all} such trees.
We then continue the analysis as before, with a slight change only in the final calculation, much like the previous theorem.
The calculation is now:
\begin{align*}
\ell_{2f}(H_{light}) &= \frac{w(H_{light})}{w(Q)} \le  O\left(  \frac{f^{-1/2} \cdot \sum \limits_{T \in \tee} w(H[T])}{\sum \limits_{T \in \tee} w(T)} \right) \tag{each edge in at least $f^{1/2}$ subgraphs $H[T]$}\\
&\le  O\left( \frac{f^{-1/2} \sum \limits_{T \in \tee} f \cdot \lambda(n, k+1) \cdot w(T)}{\sum \limits_{T \in \tee} w(T)} \right) = O\left(f^{1/2} \cdot \lambda(n, k+1)\right). \tag*{\qedhere}
\end{align*}
\end{proof}

These two theorems now essentially directly imply the upper bound parts of Theorems~\ref{thm:introsmallcomp} and \ref{thm:introbigcomp}:

\begin{proof}[Proof of Theorems~\ref{thm:introsmallcomp} and \ref{thm:introbigcomp}, Upper Bounds]
    Theorem~\ref{thm:introsmallcomp} is directly implied by Theorem~\ref{thm:feasible} (which shows that $H$ is indeed an $f$-EFT $k$-spanner) and Theorem~\ref{thm:warmupgoodf} (which gives the claimed lightness bound).  For Theorem~\ref{thm:introbigcomp}, it is easy to see that the change we made in the algorithm (making $Q$ a $(2+\eta)f$-EFT connectivity preserver rather than $2f$) does not affect the proof of Theorem~\ref{thm:feasible} that $H$ is feasible.  Hence Theorem~\ref{thm:introbigcomp} is implied by Theorem~\ref{thm:feasible} and Theorem~\ref{thm:high-parameter-lightness}.
\end{proof}

\ifshort
\else
\subsection{Competitive Lightness Upper Bounds in Polynomial Time} \label{sec:polytime}

\begin{algorithm}[t]
\DontPrintSemicolon

\textbf{Input:} Graph $G$, stretch $k$, fault tolerance $f$\;~\\

Let $Q \gets $ $2$-approximate min-weight $2f$-FT connectivity preserver of $G$ \tcp{polytime by \cite{DKK22}}

Let $\tee \gets$ tree decomposition of $Q$ satisfying Corollary \ref{cor:pairwisenw} w.r.t.\ connectivity parameter $2f+1$\;

Let $E \gets \emptyset$ be the initial set of non-$Q$ spanner edges\;~\\

\tcp{$c$ large enough constant}
\ForEach{edge $(u, v) \in E(G) \setminus Q$ in order of nondecreasing weight}{

    \ForEach{tree $T \in \tee$ that spans the $(2f+1)$-connected component of $Q$ containing $u, v$}{
        sample $c \log n$ subgraphs by including $T$, and including each edge in $E$ with probability $1/f$\;

        let $\widehat{P}^T_{(u, v)}$ be the fraction of sampled subgraphs $H'$ in which $\dist_{H'}(u, v) > k \cdot w(u, v)$\;
    
        \If{$\widehat{P}^T_{(u, v)} \ge 3/8$}{
            add $(u, v)$ to $E$\;
        }
    }
}
\textbf{Return} $H = Q \cup E$\;
\caption{\label{alg:ftpoly} FT spanners in polynomial time}
\end{algorithm}

We now prove Theorem~\ref{thm:alg-polytime}, improving to polynomial runtime at the cost of a worse dependence on $f$ (relative to the previous section).  Recall the theorem:

\algpolytime*

The algorithm that we will analyze to prove this is Algorithm~\ref{alg:ftpoly}.  Both the algorithm and its analysis are adaptations of the ideas of \cite{BDR21}.
To begin the analysis, let us set up some useful definitions.
For an edge $e \in E(G) \setminus Q$, let us write $E_e$ for the subset of non-$Q$ spanner edges that were added strictly before the edge $e$ was considered in the algorithm.
We also write
$$H_e := Q \cup E_e \qquad \text{and} \qquad H_e^T := T \cup E_e$$
for trees $T \in \tee$.
For all $e \in E(G) \setminus Q, T \in \tee$, we let $H'^T_e$ be a random subgraph of $H_e^T$ obtained by including the edges of $T$ (deterministically), and then including each other edge from $H_e^T$ independently with probability $1/f$.
Then let
$$P^T_{e=(u, v)} := \Pr\left[ \dist_{H'^T_e}(u, v) > k \cdot w(u, v) \right]$$
where the probability is over the random definition of $H'^T_e$.
We cannot compute $P^T_{e=(u, v)}$ exactly, but we can view Algorithm \ref{alg:ftpoly} as computing experimental estimates $\widehat{P}^T_{e=(u, v)}$ by repeatedly sampling subgraphs $H'^T_e$.
The following lemma applies standard Chernoff bounds to show that our estimates are probably reasonably good.
This argument directly follows one from \cite{BDR21}, but we recap it here from scratch.

\begin{lemma} [c.f.~\cite{BDR21}, Lemma 3.1] \label{lem:whpguarantee}
With high probability, for all $e \in E(G) \setminus Q$ and all $T \in \mathcal{T}$, we have
$\widehat{P}^T_e \in P^T_e \pm \frac{1}{8}.$
\end{lemma}
\begin{proof}
First, notice that in expectation, we have
$$\mathbb{E}\left[\widehat{P}^T_e\right] = P^T_e.$$
Thus our goal is to bound the probability that $\widehat{P}^T_e$ fluctuates significantly from its expectation.
We may view $\widehat{P}^T_{e=(u, v)}$ as the sum of $c \log n$ random variables of the form
$$\widehat{P}^T_{(u, v), i} := \begin{cases}
0 & \text{if } i^{th} \text{ sampled subgraph has } \dist_{H'}(u, v) \le k \cdot w(u, v)\\
\frac{1}{c \log n} & \text{if } i^{th} \text{ sampled subgraph has } \dist_{H'}(u, v) > k \cdot w(u, v).
\end{cases}$$

The Chernoff bound (see e.g.\ \cite{DP09} for exposition) states that:
\begin{align*}
\Pr\left[ \widehat{P}_e^T < \frac{7}{8} P_e^T \right] &\le e^{-\frac{(1/8)^2}{2} \cdot c \log n}\\
&= \frac{1}{n^{c/128}}.
\end{align*}
Thus we our desired high-probability guarantee, with respect to choice of large enough constant $c$.
This proof establishes the lower bound on $\widehat{P}^T_e$; the upper bound follows from an essentially identical calculation.
\end{proof}

The following lemmas will assume that the event from Lemma \ref{lem:whpguarantee} holds, and hence these lemmas only with high probability, rather than deterministically.
We next establish correctness:
\begin{lemma} \label{lem:poly-correct}
With high probability, the output spanner $H$ from Algorithm \ref{alg:ftpoly} is an $f$-EFT $k$-spanner of the input graph.
\end{lemma}
\begin{proof}
Consider an edge $(u, v) \in E(G) \setminus Q$, and note that as in the previous warmup, this implies that $u, v$ are $2f+1$-connected.
It suffices to argue that, if there is a fault set $F, |F| \le f$ for which 
$$\dist_{H_e \setminus F}(u, v) > k \cdot w(u, v),$$
then we choose to add $(u, v)$ to the spanner.
Suppose such a fault set $F$ exists.
Recall that there are $2f+1$ trees that span the $2f+1$-connected component that contains $u, v$.
Since each edge in $F$ belongs to at most two of these trees, there exists a tree $T \in \mathcal{T}$ such that $u, v$ are connected in $T$ and $T \cap F = \emptyset$.
When we generate a subgraph $H'$ associated to this particular tree $T$, note that if none of the edges in $F$ survive in $H'$, then we will have
$$\dist_{H'}(u, v) > k \cdot w(u, v).$$
There are $|F|\le f$ edges, which are each included in $H'$ with probability $1/f$, and so none of the edges in $F$ survive in $H'$ with probability at least $1/2$.
So we have $P_e^T \ge 1/2$.
Assuming that the event from Lemma \ref{lem:whpguarantee} holds (relating $P^T_e$ to $\widehat{P}_e^T$), we thus have $\widehat{P}_e^T \ge 3/8$.
So $(u, v)$ will be added to the spanner $H$ when the tree $T$ is considered, completing the proof.
\end{proof}

We now turn to bounding the number of edges in the final spanner.
For an edge $(u, v) \in E$, we will say it is \emph{hosted} by the first tree $T \in \tee$ that caused the edge to be added to $E$ in the main loop.
We will write $H[T]$ for the subgraph of the final spanner that contains $T$ and all of the edges that it hosts.
Note that we are reusing this notation from the previous warmup; although the definition of $H[T]$ is slightly different here, it plays a directly analogous role.

\begin{lemma} \label{lem:polyhostbound}
With high probability, for all trees $T \in \mathcal{T}$, we have
$$\frac{w(H[T])}{w(T)} \le O\left(f \cdot \lambda(n, k+1)\right).$$
\end{lemma}
\begin{proof}
Analogous to the previous warmup, it will be helpful to define subgraphs $H''[T] \subseteq H'[T] \subseteq H[T]$ as follows.
First, $H'[T]$ is a random subgraph of $H[T]$, obtained by including $T$ deterministically and then including each edge hosted by $T$ with probability $1/f$.
For an edge $(u, v) \in H'[T]$, let $H'_{(u, v)}[T]$ be the subgraph that contains only $T$ and the edges from $H'[T]$ considered strictly before $(u, v)$ in the algorithm.
Then we define $H''[T]$ as:
$$H''[T] := T \cup \left\{ (u, v) \in H'[T] \ \mid \ \dist_{H'_{(u, v)}[T]}(u, v) > k \cdot w(u, v)\right\}.$$
We can now bound the expected weight of $H''[T]$ in two different ways:
\begin{itemize}
\item On one hand, the definition of $H''[T]$ implies that it has weighted girth $>k+1$.
To see this, consider a cycle $C$ in $H'[T]$ of normalized weight $\le k+1$ and let $(u, v) \in C$ be its last edge considered in the algorithm.
If all edges in $C \setminus \{(u, v)\}$ are added to $H''[T]$, then they form a $u \leadsto v$ path, implying that
$$\dist_{H'_{(u, v)}[T]}(u, v) \le k \cdot w(u, v),$$
which will cause us to not include $(u, v)$ in $H''_{(u, v)}[T]$.
Thus, by definition of $\lambda$, we have
\begin{align*}
\lambda(n, k+1) \ge \ell(H''[T]) &= \frac{w(H''[T])}{w(\mst(H''[T]))}\\
&\ge \frac{w(H''[T])}{w(T)}.
\end{align*}

\item On the other hand, consider an edge $(u, v)$ hosted by $T$, and let us check the probability that it survives in $H''[T]$.
First it survives in $H'[T]$ with probability $1/f$.
Then, since $H_{(u, v)}[T] \subseteq H_{(u, v)}^T$,\footnote{As a reminder, the difference between these two subgraphs is that $H_{(u, v)}[T]$ only contains the spanner edges considered before $(u, v)$ that are hosted by $T$, while $H_{(u, v)}^T$ contains all spanner edges considered before $(u, v)$.} it survives in $H''[T]$ with probability \emph{at least} $P_{(u, v)}^T$.
Since $(u, v)$ was added to the spanner, we have $\widehat{P}_{(u, v)} \ge 3/8$, and assuming that the event from Lemma \ref{lem:whpguarantee} holds, we thus have $P_{(u, v)}^T \ge 1/4$.
So in total, $(u, v)$ survives in $H''[T]$ with probability $\Theta(1/f)$, and we have
$$\mathbb{E}\left[w(H''[T])\right] \ge \Omega\left(\frac{w(H[T])}{f} \right).$$
\end{itemize}

Combining the previous inequalities, we have:
\begin{align*}
w(H[T]) &\le O\left(f \cdot \mathbb{E}\left[w(H''[T])\right]\right)\\
&\le O\left(f \cdot \lambda(n, k+1) \cdot w(T)\right). \tag*{\qedhere}
\end{align*}
\end{proof}

We can now wrap up the proof:
\begin{lemma} \label{lem:poly-light}
With high probability, the output spanner $H$ has $2f$-competitive lightness
$$\ell_{2f}(H \mid G) \le O\left( f \cdot \lambda(n, k+1) \right).$$
\end{lemma}
\begin{proof}
We have:
\begin{align*}
\ell_{2f}(H \mid G) &\le 2 \cdot \frac{w(H)}{w(Q)} \tag{$Q$ is $2$-approx min weight preserver}\\
&\leq 2 \cdot \frac{\sum \limits_{T \in \tee} w(H[T])}{w(Q)} \tag{every $e \in E(H)$ in at least one $H[T]$} \\ 
&\leq 4 \cdot \frac{\sum \limits_{T \in \tee} w(H[T])}{\sum \limits_{T \in \tee} w(T)} \tag{Corollary~\ref{cor:pairwisenw} and def of $\tee$}\\
&\le  O\left( \frac{\sum \limits_{T \in \tee} f \cdot \lambda(n, k+1) \cdot w(T)}{\sum \limits_{T \in \tee} w(T)} \right) \tag{Lemma \ref{lem:polyhostbound}}\\
&= O(f \cdot \lambda(n, k+1)). \tag*{\qedhere}
\end{align*}
\end{proof}

The first part of Theorem~\ref{thm:alg-polytime} is now directly implied by Lemmas~\ref{lem:poly-correct} and \ref{lem:poly-light}.

\subsubsection{Improved Lightness Bounds in Polynomial Time via Multiple Host Trees}

Of the two results from Section \ref{sec:multiple-host}, only one of them extends readily to the polynomial time algorithm.
The one that does \emph{not} seem to extend is Theorem \ref{thm:warmupgoodf}, which improves the $f$-dependence for $2f$-competitive lightness to $f^{1/2}$.
The issue is roughly that the proof samples heavy and light edges with different probabilities, but in our polynomial time algorithm we commit to a sampling probability at runtime, meaning that we cannot easily apply different runtimes to different edge types.
However, Theorem \ref{thm:high-parameter-lightness} extends fairly straightforwardly, in order to prove the corresponding part of Theorem \ref{thm:alg-polytime}.

\begin{algorithm}[t]
\DontPrintSemicolon

\textbf{Input:} Graph $G$, stretch $k$, fault tolerance $f$, parameter $\eta > 0$\;~\\

Let $Q \gets $ $2$-approximate min-weight $(2+\eta)f$-FT connectivity preserver of $G$ \tcp{polytime by \cite{DKK22}}

Let $\tee \gets$ tree decomposition of $Q$ satisfying Corollary \ref{cor:pairwisenw} w.r.t.\ connectivity parameter $(2+\eta)f+1$\;

Let $E \gets \emptyset$ be the initial set of non-$Q$ spanner edges\;~\\

\tcp{$c$ large enough constant}
\ForEach{edge $(u, v) \in E(G) \setminus Q$ in order of nondecreasing weight}{
    votes $\gets 0$\;
    \ForEach{tree $T \in \tee$ that spans the $(2+\eta)f+1$-connected component of $Q$ containing $u, v$}{
        sample $c \log n$ subgraphs by including $T$, and including each edge in $E$ with probability $1/f$\;

        let $\widehat{P}^T_{(u, v)}$ be the fraction of sampled subgraphs $H'$ in which $\dist_{H'}(u, v) > k \cdot w(u, v)$\;
    
        \If{$\widehat{P}^T_{(u, v)} \ge 3/8$}{
            votes $\gets$ votes $+1$\;
        }
    }
    \If{votes $\ge \eta f + 1$}{
        add $(u, v)$ to $E$\;
    }
}
\textbf{Return} $H = Q \cup E$\;
\caption{\label{alg:polymultitree} FT spanners in polynomial time with competition parameter $(2+\eta)f$}
\end{algorithm}


To prove this we will use Algorithm~\ref{alg:polymultitree}.  Most of our analysis of Algorithm \ref{alg:polymultitree} is the same as that for Algorithm \ref{alg:ftpoly}, and we will not repeat it here.
For example, Lemma \ref{lem:whpguarantee} (asserting that $\widehat{P}^T_{(u , v)} \approx P^T_{(u, v)}$ for all $T, (u, v)$) still holds with exactly the same proof.
Lemma \ref{lem:poly-correct} (asserting that the output spanner is correct with high probability) also still holds by essentially the same proof, but we note that for any edge $e$, there will be at least $(2+\eta)f+1 - 2f = \eta f + 1$ trees that are disjoint from the blocked edges $e'$ paired with $e$.
All such trees will vote for $(u, v)$ (i.e. we will increment the ``votes'' variable when these trees are considered), and so $(u, v)$ will indeed be added to the spanner if an appropriate fault set $F$ exists.

The part that changes is the hosting of edges in trees: instead of an edge $(u, v) \in E$ being hosted by the \emph{first} tree $T$ that caused the edge to be added to the spanner, it is instead hosted by \emph{all} $\Omega(\eta f)$ trees that voted for it.
Under this new hosting strategy, the proof of Lemma \ref{lem:polyhostbound} (controlling the weight of each host graph $H[T]$) still holds with no significant changes.
But the final calculation in Lemma \ref{lem:poly-light} now admits an optimization.
We calculate:

\begin{align*}
\ell_{2f}(H \mid G) &\le 2 \cdot \frac{w(H)}{w(Q)} \tag{$Q$ is $2$-approx min weight preserver}\\
&\leq 2 \cdot \frac{(\eta f)^{-1} \sum \limits_{T \in \tee} w(H[T])}{w(Q)} \tag{every $e \in E(H)$ in at least $\Omega(\eta f)$ graphs $H[T]$} \\ 
&\leq 4 \cdot \frac{(\eta f)^{-1} \sum \limits_{T \in \tee} w(H[T])}{\sum \limits_{T \in \tee} w(T)} \tag{Corollary~\ref{cor:pairwisenw} and def of $\tee$}\\
&\le  O\left( \frac{(\eta f)^{-1}\sum \limits_{T \in \tee} f \cdot \lambda(n, k+1) \cdot w(T)}{\sum \limits_{T \in \tee} w(T)} \right) \tag{Lemma \ref{lem:polyhostbound}}\\
&= O\left((\eta^{-1} \cdot \lambda(n, k+1)\right). \tag*{\qedhere}
\end{align*}
\fi