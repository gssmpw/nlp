\section{Related Work}
\label{sec:relatedwork}
In this section, we introduce the work closely related to ours, including the LDCT image denoising methods in \cref{subsec:LDCT reconstruction}, self-supervised image denoising method in \cref{subsec:unpaired}, and guided image denoising in \cref{subsec:guided}.


\subsection{Low-Dose CT Image Denoising}
\label{subsec:LDCT reconstruction}
Low-dose CT (LDCT) image denoising is initially tackled with first convolutional neural networks (CNNs)~\cite{CZZ}. The RED-CNN network~\cite{chen2017low} was developed with an encoder-decoder architecture for favorable performance. Kang \etal~\cite{kang2017deep,kang2018deep} proposed to learn wavelet transforms with CNNs for LDCT image denoising.
%
%yfw: firstly好像是首先/第一的意思,这里是不是想表达“最初”,可能initially或者at first比较合适?
Compared to CNNs~\cite{TMI1}, transformers~\cite{transformer} are good at capturing global information and long-range feature interactions, which have been applied to LDCT image denoising for better performance~\cite{wang2023ctformer}. Vision Transformer (ViT)~\cite{dosovitskiy2020image} has also been utilized in~\cite{wang2021ted,wang2023ctformer} to enlarge the effective receptive fields of window-based transformers for better denoising performance. Li \etal~\cite{li2022transformer} devised a dual-branch transformer to recover the edges and textures of LDCT images well.
%CNNs和Transformer用的都是合成的低剂量CT,而下面的GAN用的是real-world的
However, the above methods are trained with synthetic LDCT images and could hardly be applied to real-world LDCT images in clinical practice.
%by adding Poisson noise into the sinograms of the corresponding normal-dose CT (NDCT) images. 


% routine-dose
For clinical purposes, researchers proposed to learn mapping from real-world LDCT images to high-quality NDCT ones using the GAN architectures~\cite{gan}. Wolterink \etal~\cite{wolterink2017generative} trained a CNN jointly with an adversarial CNN to recover the NDCT images from LDCT images. Yi and Babyn~\cite{yi2018sharpness} trained an adversarial network together with a sharpness detection network to mitigate the blurring effects in LDCT image denoising. Later, many LDCT image denoising methods are built upon CycleGAN~\cite{kang2019cycle,tang2019unpaired}, conditional GAN~\cite{hong2020end}, or WGAN~\cite{yang2018low,hu2019artifact}. However, the structural misalignment between real-world LDCT images and NDCT images makes it difficult to guarantee the fidelity of denoised images~\cite{kulathilake2023review}.

%两点不同:(1)放弃使用合成数据(2)用PTSP进行筛选
In this paper, we also use real-world LDCT images to train the denoising networks, thereby well serving clinical practice. To alleviate the problem of image structure misalignment between LDCT and NDCT images, we propose a Patch Triplet Similarity Purification (PTSP) strategy to select highly-similar patch triplets for training LDCT image denoising networks.

\subsection{Self-Supervised Image Denoising}
\label{subsec:unpaired}
Self-supervised image denoising methods~\cite{noise2noise,xu2020noisy} learn from the noisy images themselves to remove the noise without using clean images.
%
By assuming that noise is zero-mean and independently and identically distributed (i.i.d.), Noise2Noise (N2N)~\cite{noise2noise} effectively trains image denoising using pairs of noisy images with the same contents but different noise. Noise2Void (N2V)~\cite{krull2019noise2void} learns to predict the true value of each noisy pixel from its neighboring pixels, and hence called ``blind-spot'' method.
Unlike N2V, Noise2Self (N2S)~\cite{batson2019noise2self} additionally performs masking operations for each pixel to enhance the denoising robustness.
%
Noise-As-Clean (NAC)~\cite{xu2020noisy} learns to remove image noise with a pair of noisy image and noisier image, which is produced by adding synthetic noise to the noisy image.
%
However, the above-mentioned denoising methods mainly learn to remove the zero-mean and i.i.d. noise, which may not hold true for real-world LDCT images.

Self-supervised learning has also been applied to LDCT image denoising by only using LDCT images.
%原来的说法是论文的摘要里面提炼的,说的不够详细,感觉直接使用原文的摘要里面的说法更清楚一点:
Noise2Inverse~\cite{hendriksen2020noise2inverse} performs image reconstruction~\cite{TMI2} by learning a CNN without additional clean or noisy data.
%这篇论文的similarity是根据RMSE指标和选定的阈值来形成掩码从而使得只有像素值接近的区域才能够纳入优化的范围
Noise2Sim~\cite{niu2022noise} is a self-supervised deep denoising approach that achieves noise reduction by using similar images.
%
However, because of the lack of supervision from high-quality NDCT images, it is challenging for these self-supervised denoisers to remove complex noise well in real-world LDCT images.


In this paper,
%unlike self-supervised methods that train denoising networks only using LDCT images, 
we propose to train denoising networks with pairs of highly similar LDCT and NDCT image patches selected by our similarity purification strategy.
% supervised learning scheme

\subsection{Guided Image Denoising}
\label{subsec:guided}
%下面的这些方法中只有MLEFGN和MLEFGN是CT降噪的,因此将MLEFGN放到后面,其他的放到前面来

Many image denoising methods utilize useful spatial or edge information from external clean images for guided image denoising. He \etal~\cite{He} proposed guided image filtering (GIF) to use the guidance image to identify noise and edges for better noise reduction. Based on GIF, the method of~\cite{WGIF} incorporates an edge-aware weighting strategy for edge-preserving image filtering. Xu \etal~\cite{NSS} exploited the external information from clean images to guide the internal learning of a noisy test image for real-world image denoising. Zhang \etal~\cite{zhang2022guided} utilized the mean image of all spectral bands as useful guidance to adaptively aggregate spatial information.

The insights of external guidance also boost LDCT image denoising. For example, edge-guided filtering~\cite{edge-guided} and GDAFormer~\cite{Jiang2024GDAFormerGD} use edge feature to guide the learning of LDCT image denoising.
%说明缺陷,引出NCCT image guidance
%这些缺陷有点牵强,并且我们也using external data了 However, using edge features as a guide provides too little information, while using external data as a guide makes it difficult to ensure the consistency of the distribution between external and internal data.
In this paper, we use Non-Contrast CT (NCCT) image to guide the LDCT image denoising.