We acknowledge two limitations in this work.

First, GeoEdit requires access to old knowledge datasets to extract the task vector $\tau_{old}$. In some cases, however, such datasets may not be available, meaning we only know the updated results.
A potential solution is to input the task to be edited directly into the initial model and use the output as the old knowledge. However, this introduces additional inference costs, especially in our mass-editing settings. 
Furthermore, for open-ended questions, selecting the appropriate output as the reference is another challenge. We plan to explore ways to extend GeoEdit to address these issues and improve its adaptability.


Second, the core of {\ouralg} relies on using the angle between parameter updates to differentiate between disturbances to general knowledge and the learning of new knowledge.
While this approach offers valuable insights, it still results in some loss of model generalization, suggesting that the angle alone cannot fully decouple new knowledge learning from general knowledge disturbance.
To address this, we aim to consider multiple geometric variables, such as task vector projections and magnitude, to further refine {\ouralg} and enhance performance in the future.

%This approach provides insightful perspectives for the model editing community, particularly by treating updates along the orthogonal direction as modifications to general knowledge. Although our method significantly improves locality, there is still some loss in the model's generalization performance, indicating that there is substantial room for further improvement. This suggests that relying solely on the angle may not fully decouple the learning of new knowledge from the disturbance of general knowledge. In the future, we plan to simultaneously consider multiple geometric variables, such as projecting the task vector, analyzing the impact of different components, and considering the magnitude of the task vector, to further refine {\ouralg} and enhance performance.