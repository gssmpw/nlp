% This must be in the first 5 lines to tell arXiv to use pdfLaTeX, which is strongly recommended.
\pdfoutput=1
% In particular, the hyperref package requires pdfLaTeX in order to break URLs across lines.

\documentclass[11pt]{article}

% Change "review" to "final" to generate the final (sometimes called camera-ready) version.
% Change to "preprint" to generate a non-anonymous version with page numbers.
\usepackage[preprint]{acl}
\usepackage{balance}
% Standard package includes
\usepackage{times}
\usepackage{latexsym}
\usepackage{xspace}
\usepackage{devanagari}
%\usepackage[margin=1in]{geometry}
\usepackage{booktabs}
\usepackage{array}
\usepackage{longtable}
\usepackage{amsmath}
\usepackage{enumitem}
\usepackage{hyperref}


% For proper rendering and hyphenation of words containing Latin characters (including in bib files)
\usepackage[T1]{fontenc}
% For Vietnamese characters
% \usepackage[T5]{fontenc}
% See https://www.latex-project.org/help/documentation/encguide.pdf for other character sets

% This assumes your files are encoded as UTF8
\usepackage[utf8]{inputenc}

% This is not strictly necessary, and may be commented out,
% but it will improve the layout of the manuscript,
% and will typically save some space.
\usepackage{microtype}
\usepackage{textcase}

%
\setlength\unitlength{1mm}
\newcommand{\twodots}{\mathinner {\ldotp \ldotp}}
% bb font symbols
\newcommand{\Rho}{\mathrm{P}}
\newcommand{\Tau}{\mathrm{T}}

\newfont{\bbb}{msbm10 scaled 700}
\newcommand{\CCC}{\mbox{\bbb C}}

\newfont{\bb}{msbm10 scaled 1100}
\newcommand{\CC}{\mbox{\bb C}}
\newcommand{\PP}{\mbox{\bb P}}
\newcommand{\RR}{\mbox{\bb R}}
\newcommand{\QQ}{\mbox{\bb Q}}
\newcommand{\ZZ}{\mbox{\bb Z}}
\newcommand{\FF}{\mbox{\bb F}}
\newcommand{\GG}{\mbox{\bb G}}
\newcommand{\EE}{\mbox{\bb E}}
\newcommand{\NN}{\mbox{\bb N}}
\newcommand{\KK}{\mbox{\bb K}}
\newcommand{\HH}{\mbox{\bb H}}
\newcommand{\SSS}{\mbox{\bb S}}
\newcommand{\UU}{\mbox{\bb U}}
\newcommand{\VV}{\mbox{\bb V}}


\newcommand{\yy}{\mathbbm{y}}
\newcommand{\xx}{\mathbbm{x}}
\newcommand{\zz}{\mathbbm{z}}
\newcommand{\sss}{\mathbbm{s}}
\newcommand{\rr}{\mathbbm{r}}
\newcommand{\pp}{\mathbbm{p}}
\newcommand{\qq}{\mathbbm{q}}
\newcommand{\ww}{\mathbbm{w}}
\newcommand{\hh}{\mathbbm{h}}
\newcommand{\vvv}{\mathbbm{v}}

% Vectors

\newcommand{\av}{{\bf a}}
\newcommand{\bv}{{\bf b}}
\newcommand{\cv}{{\bf c}}
\newcommand{\dv}{{\bf d}}
\newcommand{\ev}{{\bf e}}
\newcommand{\fv}{{\bf f}}
\newcommand{\gv}{{\bf g}}
\newcommand{\hv}{{\bf h}}
\newcommand{\iv}{{\bf i}}
\newcommand{\jv}{{\bf j}}
\newcommand{\kv}{{\bf k}}
\newcommand{\lv}{{\bf l}}
\newcommand{\mv}{{\bf m}}
\newcommand{\nv}{{\bf n}}
\newcommand{\ov}{{\bf o}}
\newcommand{\pv}{{\bf p}}
\newcommand{\qv}{{\bf q}}
\newcommand{\rv}{{\bf r}}
\newcommand{\sv}{{\bf s}}
\newcommand{\tv}{{\bf t}}
\newcommand{\uv}{{\bf u}}
\newcommand{\wv}{{\bf w}}
\newcommand{\vv}{{\bf v}}
\newcommand{\xv}{{\bf x}}
\newcommand{\yv}{{\bf y}}
\newcommand{\zv}{{\bf z}}
\newcommand{\zerov}{{\bf 0}}
\newcommand{\onev}{{\bf 1}}

% Matrices

\newcommand{\Am}{{\bf A}}
\newcommand{\Bm}{{\bf B}}
\newcommand{\Cm}{{\bf C}}
\newcommand{\Dm}{{\bf D}}
\newcommand{\Em}{{\bf E}}
\newcommand{\Fm}{{\bf F}}
\newcommand{\Gm}{{\bf G}}
\newcommand{\Hm}{{\bf H}}
\newcommand{\Id}{{\bf I}}
\newcommand{\Jm}{{\bf J}}
\newcommand{\Km}{{\bf K}}
\newcommand{\Lm}{{\bf L}}
\newcommand{\Mm}{{\bf M}}
\newcommand{\Nm}{{\bf N}}
\newcommand{\Om}{{\bf O}}
\newcommand{\Pm}{{\bf P}}
\newcommand{\Qm}{{\bf Q}}
\newcommand{\Rm}{{\bf R}}
\newcommand{\Sm}{{\bf S}}
\newcommand{\Tm}{{\bf T}}
\newcommand{\Um}{{\bf U}}
\newcommand{\Wm}{{\bf W}}
\newcommand{\Vm}{{\bf V}}
\newcommand{\Xm}{{\bf X}}
\newcommand{\Ym}{{\bf Y}}
\newcommand{\Zm}{{\bf Z}}

% Calligraphic

\newcommand{\Ac}{{\cal A}}
\newcommand{\Bc}{{\cal B}}
\newcommand{\Cc}{{\cal C}}
\newcommand{\Dc}{{\cal D}}
\newcommand{\Ec}{{\cal E}}
\newcommand{\Fc}{{\cal F}}
\newcommand{\Gc}{{\cal G}}
\newcommand{\Hc}{{\cal H}}
\newcommand{\Ic}{{\cal I}}
\newcommand{\Jc}{{\cal J}}
\newcommand{\Kc}{{\cal K}}
\newcommand{\Lc}{{\cal L}}
\newcommand{\Mc}{{\cal M}}
\newcommand{\Nc}{{\cal N}}
\newcommand{\nc}{{\cal n}}
\newcommand{\Oc}{{\cal O}}
\newcommand{\Pc}{{\cal P}}
\newcommand{\Qc}{{\cal Q}}
\newcommand{\Rc}{{\cal R}}
\newcommand{\Sc}{{\cal S}}
\newcommand{\Tc}{{\cal T}}
\newcommand{\Uc}{{\cal U}}
\newcommand{\Wc}{{\cal W}}
\newcommand{\Vc}{{\cal V}}
\newcommand{\Xc}{{\cal X}}
\newcommand{\Yc}{{\cal Y}}
\newcommand{\Zc}{{\cal Z}}

% Bold greek letters

\newcommand{\alphav}{\hbox{\boldmath$\alpha$}}
\newcommand{\betav}{\hbox{\boldmath$\beta$}}
\newcommand{\gammav}{\hbox{\boldmath$\gamma$}}
\newcommand{\deltav}{\hbox{\boldmath$\delta$}}
\newcommand{\etav}{\hbox{\boldmath$\eta$}}
\newcommand{\lambdav}{\hbox{\boldmath$\lambda$}}
\newcommand{\epsilonv}{\hbox{\boldmath$\epsilon$}}
\newcommand{\nuv}{\hbox{\boldmath$\nu$}}
\newcommand{\muv}{\hbox{\boldmath$\mu$}}
\newcommand{\zetav}{\hbox{\boldmath$\zeta$}}
\newcommand{\phiv}{\hbox{\boldmath$\phi$}}
\newcommand{\psiv}{\hbox{\boldmath$\psi$}}
\newcommand{\thetav}{\hbox{\boldmath$\theta$}}
\newcommand{\tauv}{\hbox{\boldmath$\tau$}}
\newcommand{\omegav}{\hbox{\boldmath$\omega$}}
\newcommand{\xiv}{\hbox{\boldmath$\xi$}}
\newcommand{\sigmav}{\hbox{\boldmath$\sigma$}}
\newcommand{\piv}{\hbox{\boldmath$\pi$}}
\newcommand{\rhov}{\hbox{\boldmath$\rho$}}
\newcommand{\upsilonv}{\hbox{\boldmath$\upsilon$}}

\newcommand{\Gammam}{\hbox{\boldmath$\Gamma$}}
\newcommand{\Lambdam}{\hbox{\boldmath$\Lambda$}}
\newcommand{\Deltam}{\hbox{\boldmath$\Delta$}}
\newcommand{\Sigmam}{\hbox{\boldmath$\Sigma$}}
\newcommand{\Phim}{\hbox{\boldmath$\Phi$}}
\newcommand{\Pim}{\hbox{\boldmath$\Pi$}}
\newcommand{\Psim}{\hbox{\boldmath$\Psi$}}
\newcommand{\Thetam}{\hbox{\boldmath$\Theta$}}
\newcommand{\Omegam}{\hbox{\boldmath$\Omega$}}
\newcommand{\Xim}{\hbox{\boldmath$\Xi$}}


% Sans Serif small case

\newcommand{\Gsf}{{\sf G}}

\newcommand{\asf}{{\sf a}}
\newcommand{\bsf}{{\sf b}}
\newcommand{\csf}{{\sf c}}
\newcommand{\dsf}{{\sf d}}
\newcommand{\esf}{{\sf e}}
\newcommand{\fsf}{{\sf f}}
\newcommand{\gsf}{{\sf g}}
\newcommand{\hsf}{{\sf h}}
\newcommand{\isf}{{\sf i}}
\newcommand{\jsf}{{\sf j}}
\newcommand{\ksf}{{\sf k}}
\newcommand{\lsf}{{\sf l}}
\newcommand{\msf}{{\sf m}}
\newcommand{\nsf}{{\sf n}}
\newcommand{\osf}{{\sf o}}
\newcommand{\psf}{{\sf p}}
\newcommand{\qsf}{{\sf q}}
\newcommand{\rsf}{{\sf r}}
\newcommand{\ssf}{{\sf s}}
\newcommand{\tsf}{{\sf t}}
\newcommand{\usf}{{\sf u}}
\newcommand{\wsf}{{\sf w}}
\newcommand{\vsf}{{\sf v}}
\newcommand{\xsf}{{\sf x}}
\newcommand{\ysf}{{\sf y}}
\newcommand{\zsf}{{\sf z}}


% mixed symbols

\newcommand{\sinc}{{\hbox{sinc}}}
\newcommand{\diag}{{\hbox{diag}}}
\renewcommand{\det}{{\hbox{det}}}
\newcommand{\trace}{{\hbox{tr}}}
\newcommand{\sign}{{\hbox{sign}}}
\renewcommand{\arg}{{\hbox{arg}}}
\newcommand{\var}{{\hbox{var}}}
\newcommand{\cov}{{\hbox{cov}}}
\newcommand{\Ei}{{\rm E}_{\rm i}}
\renewcommand{\Re}{{\rm Re}}
\renewcommand{\Im}{{\rm Im}}
\newcommand{\eqdef}{\stackrel{\Delta}{=}}
\newcommand{\defines}{{\,\,\stackrel{\scriptscriptstyle \bigtriangleup}{=}\,\,}}
\newcommand{\<}{\left\langle}
\renewcommand{\>}{\right\rangle}
\newcommand{\herm}{{\sf H}}
\newcommand{\trasp}{{\sf T}}
\newcommand{\transp}{{\sf T}}
\renewcommand{\vec}{{\rm vec}}
\newcommand{\Psf}{{\sf P}}
\newcommand{\SINR}{{\sf SINR}}
\newcommand{\SNR}{{\sf SNR}}
\newcommand{\MMSE}{{\sf MMSE}}
\newcommand{\REF}{{\RED [REF]}}

% Markov chain
\usepackage{stmaryrd} % for \mkv 
\newcommand{\mkv}{-\!\!\!\!\minuso\!\!\!\!-}

% Colors

\newcommand{\RED}{\color[rgb]{1.00,0.10,0.10}}
\newcommand{\BLUE}{\color[rgb]{0,0,0.90}}
\newcommand{\GREEN}{\color[rgb]{0,0.80,0.20}}

%%%%%%%%%%%%%%%%%%%%%%%%%%%%%%%%%%%%%%%%%%
\usepackage{hyperref}
\hypersetup{
    bookmarks=true,         % show bookmarks bar?
    unicode=false,          % non-Latin characters in AcrobatÕs bookmarks
    pdftoolbar=true,        % show AcrobatÕs toolbar?
    pdfmenubar=true,        % show AcrobatÕs menu?
    pdffitwindow=false,     % window fit to page when opened
    pdfstartview={FitH},    % fits the width of the page to the window
%    pdftitle={My title},    % title
%    pdfauthor={Author},     % author
%    pdfsubject={Subject},   % subject of the document
%    pdfcreator={Creator},   % creator of the document
%    pdfproducer={Producer}, % producer of the document
%    pdfkeywords={keyword1} {key2} {key3}, % list of keywords
    pdfnewwindow=true,      % links in new window
    colorlinks=true,       % false: boxed links; true: colored links
    linkcolor=red,          % color of internal links (change box color with linkbordercolor)
    citecolor=green,        % color of links to bibliography
    filecolor=blue,      % color of file links
    urlcolor=blue           % color of external links
}
%%%%%%%%%%%%%%%%%%%%%%%%%%%%%%%%%%%%%%%%%%%



% This is also not strictly necessary, and may be commented out.
% However, it will improve the aesthetics of text in
% the typewriter font.
\usepackage{inconsolata}

%Including images in your LaTeX document requires adding
%additional package(s)
\usepackage{graphicx}

% If the title and author information does not fit in the area allocated, uncomment the following
%
%\setlength\titlebox{<dim>}
%
% and set <dim> to something 5cm or larger.

\title{Language Models’ Factuality Depends on the Language of Inquiry}

% Author information can be set in various styles:
% For several authors from the same institution:
% \author{Author 1 \and ... \and Author n \\
%         Address line \\ ... \\ Address line}
% if the names do not fit well on one line use
%         Author 1 \\ {\bf Author 2} \\ ... \\ {\bf Author n} \\
% For authors from different institutions:
% \author{Author 1 \\ Address line \\  ... \\ Address line
%         \And  ... \And
%         Author n \\ Address line \\ ... \\ Address line}
% To start a separate ``row'' of authors use \AND, as in
% \author{Author 1 \\ Address line \\  ... \\ Address line
%         \AND
%         Author 2 \\ Address line \\ ... \\ Address line \And
%         Author 3 \\ Address line \\ ... \\ Address line}

% \author{First Author \\
%   Affiliation / Address line 1 \\
%   Affiliation / Address line 2 \\
%   Affiliation / Address line 3 \\
%   \texttt{email@domain} \\\And
%   Second Author \\
%   Affiliation / Address line 1 \\
%   Affiliation / Address line 2 \\
%   Affiliation / Address line 3 \\
%   \texttt{email@domain} \\}

% \author{{Abhinav Rao}\thanks{\enspace Equal Contribution} \thanks{\enspace Work done while at Microsoft.}, {\bfseries Aditi Khandelwal\footnotemark[1] }\textsuperscript{$\ddagger$}, {\bfseries Kumar Tanmay\footnotemark[1] }\textsuperscript{$\ddagger$}, {\bfseries Utkarsh Agarwal\footnotemark[1] }\textsuperscript{$\ddagger$}, \\ {\bfseries Monojit Choudhury}\textsuperscript{$\ddagger$}\\
%         \textsuperscript{$\dagger$}Carnegie Mellon University\\
%         \textsuperscript{$\ddagger$}Microsoft Corporation \\
%         {abhinavr@cs.cmu.edu, \{t-aditikh, t-ktanmay, t-utagarwal, 
% monojitc\}@microsoft.com}}

% \author{
%  \textbf{Tushar Aggarwal\textsuperscript{1}},
%  \textbf{Kumar Tanmay\textsuperscript{4,8}},
%  \textbf{Ayush Agrawal\textsuperscript{1,2,3}},
%  \textbf{Kumar Ayush\textsuperscript{5,6}},
% \\
%  \textbf{Hamid Palangi\textsuperscript{6,7}},
%  \textbf{Paul Pu Liang\textsuperscript{8}},
% \\
%  \textsuperscript{1}Microsoft Research,
%  \textsuperscript{2}Université de Montréal,
%  \textsuperscript{3}Mila,
%  \textsuperscript{4}Harvard University,
%  \textsuperscript{5}Stanford University,
%  \textsuperscript{6}Google,
%  \textsuperscript{7}University of Washington,
%  \textsuperscript{8}MIT
% \\
%  \texttt{tushar@example.com, kumartanmay@example.com, ayush@example.com,} \\
%  \texttt{kumarayush@example.com, hamidpalangi@example.com, paulliang@example.com}
%  \small{
%    \textbf{Correspondence:} \href{mailto:email@domain}{email@domain}
%  }
% }

\usepackage{xcolor}
\usepackage{amssymb}
\usepackage{hyperref}
% \begin{document}

\newcommand{\redheart}{\textcolor{red}{\ensuremath{\heartsuit}}}
\newcommand{\greenheart}{\textcolor{green}{\ensuremath{\heartsuit}}}
\newcommand{\whiteheart}{\textcolor{gray}{\ensuremath{\heartsuit}}}
\newcommand{\diamondshape}{\ensuremath{\diamondsuit}}
\newcommand{\spade}{\ensuremath{\spadesuit}}
\newcommand{\boxshape}{\ensuremath{\Box}}
\newcommand{\triangleshape}{\ensuremath{\triangle}}
\newcommand{\circleshape}{\ensuremath{\circ}}
\newcommand{\starshape}{\ensuremath{\bigstar}}
\newcommand{\blackclub}{\textcolor{black}{\ensuremath{\clubsuit}}}
\newcommand{\blackheart}{\textcolor{black}{\ensuremath{\heartsuit}}}
\title{Language Models' Factuality Depends on the Language of Inquiry}

\author{
  \textbf{Tushar Aggarwal}$^{\textbf{*},\blackheart}$,
  \textbf{Kumar Tanmay}$^{\textbf{*},\spade,\starshape}$,
  \textbf{Ayush Agrawal}$^{\textbf{*},\blackheart,\blackclub,\diamondshape}$,
  \textbf{Kumar Ayush}$^{\boxshape,\triangleshape}$,
  \\
  \textbf{Hamid Palangi}$^{\triangleshape}$,
  \textbf{Paul Pu Liang}$^{\starshape}$ \\
  $^{\spade}$Harvard University, \ 
  $^{\blackclub}$Universit\'e de Montr\'eal, \ 
  $^{\diamondshape}$Mila, \ 
  $^{\starshape}$MIT\\
  $^{\boxshape}$Stanford University, \ 
  $^{\triangleshape}$Google, \ 
  $^{\blackheart}$Microsoft Research\\
  % $\texttt{tushar.aggarwal53@gmail.com, kumartanmay@fas.harvard.edu, ayush.agrawal@mila.quebec,}$\\
}

\begin{document}

\maketitle

\begin{abstract}

Multilingual language models (LMs) are expected to recall factual knowledge consistently across languages, yet they often fail to transfer knowledge between languages even when they possess the correct information in one of the languages. For example, we find that an LM may correctly identify \textit{Rashed Al Shashai} as being from \textit{Saudi Arabia} when asked in Arabic, but consistently fails to do so when asked in English or Swahili. To systematically investigate this limitation, we introduce a benchmark of 10,000 country-related facts across 13 languages and propose three novel metrics—Factual Recall Score, Knowledge Transferability Score, and Cross-Lingual Factual Knowledge Transferability Score—to quantify factual recall and knowledge transferability in LMs across different languages. Our results reveal fundamental weaknesses in today's state-of-the-art LMs, particularly in cross-lingual generalization where models fail to transfer knowledge effectively across different languages, leading to inconsistent performance sensitive to the language used. Our findings emphasize the need for LMs to recognize language-specific factual reliability and leverage the most trustworthy information across languages. We release our benchmark and evaluation framework to drive future research in multilingual knowledge transfer. The data and codes are available at this \href{https://github.com/kmrtanmay/X_FaKT.git}{link}.
\end{abstract}
\renewcommand{\thefootnote}{}
\footnotemark\footnotetext{*equal contribution.}
\footnotetext{Corresponding authors: tushar.aggarwal53@gmail.com, kumartanmay@fas.harvard.edu, ayush.agrawal@mila.quebec}

\section{Introduction}

\begin{figure}[t]  % 't' places the figure at the top of the page
    \centering
    \includegraphics[width=1.\linewidth, height=5cm]{figures/Knowledge_transfer_1.pdf}  % Adjust width as needed
    \caption{Illustratation of the cross-lingual factual knowledge transferability issue across linguistic knowledge clouds in LMs. The model correctly recalls that \textit{Rashed Al Shashai is from Saudi Arabia} when queried in Arabic, but fails to retrieve this fact in English and Swahili, highlighting that factual knowledge is often stored in language-specific silos.}
    \label{fig:knowledge_transfer}  % For referencing in the text    
\end{figure}

Large Language Models (LLMs) are often perceived as vast knowledge reservoirs, capable of recalling factual information across multiple languages \cite{wang-etal-2024-knowledge-mechanisms}. However, what if their knowledge is locked within linguistic boundaries and unable to be transferred across languages? Despite advancements in multilingual LMs such as Llama \cite{touvron2023llama, dubey2024llama}, Gemma \cite{team2024gemma}, DeepSeek \cite{deepseekai2024deepseekllmscalingopensource}, and Phi \cite{abdin2024phi, li2023textbooks}, our study reveals a striking asymmetry in their factual recall across languages: consider the example in Figure~\ref{fig:knowledge_transfer}, where an LM is tasked with a simple factual query: ``\textit{Rashed Al Shashai is from which country?}'' When asked in Arabic, several state-of-the-art LMs correctly generate the response: ``\textit{Saudi Arabia}.'' However, when posed in English, Hindi, or Swahili, the same models fail to recall the fact.
This example suggests that models can correctly retrieve country-specific facts in the language associated with that country but struggle to do so in others.
 
This raises a critical question—do these models truly internalize and transfer factual knowledge across languages, or do they merely encode isolated linguistic silos?


This limitation has significant implications for multilingual AI development and real-world applications. Many LM-based systems—such as retrieval-augmented generation (RAG) pipelines, multilingual search engines, and cross-lingual reasoning models—assume that factual knowledge is consistently available and transferable across languages. 

Our findings reveal that LMs often rely on language-specific memorization rather than true cross-lingual knowledge generalization. This over-reliance can introduce biases, inconsistencies, and reliability issues in multilingual AI applications~\cite{chua2024crosslingual}. 

To systematically analyze the factual inconsistencies,  we introduce a carefully curated dataset comprising country-related facts translated into 13 languages. This benchmark evaluates LMs on multiple dimensions—\textit{factual recall, in-context recall, and counter-factual context adherence}—across high-, medium-, and low-resource languages. This benchmark comprises of 802 instances for factual recall, 156 instances for In-context recall, and 1404 instances for counter-factual context adherence as shown in Table~\ref{tab:data_stats}.


\textit{Factual recall} assesses the LM's ability to recall country-specific facts consistently across multiple languages. We evaluate factual recall using three metrics: (a) \textit{Factual Recall Score (FRS)}: Measures how accurately a model recalls a fact in a given language, (b) \textit{Knowledge Transferability Score (KTS)}: Quantifies how well factual knowledge is transferred across languages, and (c) \textit{Cross-Lingual Factual Knowledge Transferability (X-FaKT) Score}: Combines the assessment of factual recall and cross-lingual transfer ability. FRS and KTS measure the effectiveness of cross-lingual knowledge transfer, and X-FaKT Score integrates factual recall with transferability to provide a robust measure of multilingual generalization. These metrics offer a more nuanced evaluation than a simple error rate, allowing for a deeper understanding of cross-lingual generalization.

\textit{In-Context Recall}~\cite{machlab2024llmincontextrecallprompt} measures the general performance of the models in multilingual contexts. Inspired by \cite{cotterellcontext}, we also study how factual knowledge of models affects their performance in handling in-context tasks in the multilingual setting (\textit{Counterfactual Context Adherence}). For this, we design a dataset where factual knowledge conflicts with in-context instructions.

Our experiments reveal that while LMs often retrieve factual information correctly in the language associated with the fact, they struggle to transfer this knowledge to other languages. We also found that the size of the LLM plays an important role in factuality and knowledge transferability. For example, the combined performance of LLama-3-70B in factuality and knowledge transfer across languages is markedly (~152\% $\uparrow$ in \textit{X-FaKT} Score) better than Llama-3.2-1B. In addition, there is a marked difference in these tasks when queries are asked in high-resource languages (~46\% $\uparrow$ in \textit{X-FaKT} Score) as compared to the case with low resources. This finding exposes a critical limitation in current language models and their approach to multilingual knowledge integration. Our findings also reveal an interesting trade-off: LMs with stronger factual recall often struggle with counterfactual adherence, highlighting a key limitation in balancing factual memory and contextual reasoning. 
In our experiments, we observed that the factual knowledge of LMs could skew their judgments, leading to inaccurate evaluations. One has to be very careful when designing the prompt and using LM as an evaluator. We highlight the importance of controlling the evaluator's factual knowledge to ensure consistent and effective evaluation.


\section{Related Work}


\textbf{Multilingual Transformers.} Early work by \cite{petroni-etal-2019-language} explored whether LMs can store factual knowledge about entities, setting the stage for later investigations into multilingual LMs. Notable multilingual models such as mBERT \cite{devlin-etal-2019-bert}, XLM-R \cite{conneau-etal-2020-unsupervised}, mT5 \cite{xue-etal-2021-mt5}, and BLOOM \cite{workshop2023bloom176bparameteropenaccessmultilingual} have demonstrated varying levels of performance across different languages. These models, trained on diverse multilingual corpora, show that LMs exhibit language-dependent capabilities in factual recall. Research has highlighted systematic biases in factual retrieval across languages \cite{artetxe-etal-2020-cross, liu-etal-2020-multilingual-denoising, kassner-etal-2021-multilingual}, which is a key challenge in multilingual LMs. While multilingual QA benchmarks like XQuAD \cite{artetxe-etal-2020-cross}, MLQA \cite{lewis-etal-2020-mlqa}, and TyDiQA \cite{clark-etal-2020-tydi} assess factual consistency, they do not directly measure the transfer of knowledge between languages. Recent work by \cite{wang-etal-2024-knowledge-mechanisms} raised questions about LMs' ability to recall factual knowledge in reasoning tasks, while \cite{fierro2025multilinguallanguagemodelsremember} emphasized the need for more robust methodologies for evaluating knowledge in multilingual LMs. Our study builds on these insights by introducing a benchmark specifically designed to assess cross-lingual factual knowledge transferability.

\textbf{Cross-Lingual Knowledge Transfer in LMs.} Recent works have sought to understand the factors that influence cross-lingual knowledge transfer in multilingual models. Studies suggest that multilingual LMs exhibit zero-shot and few-shot generalization across languages \cite{nooralahzadeh-etal-2020-zero, pfeiffer-etal-2020-mad}, but empirical evidence indicates that this transfer is often asymmetric, with high-resource languages benefiting more than lower-resource ones \cite{hu2020xtrememassivelymultilingualmultitask}. \cite{muller-etal-2021-first} investigated the connection between cross-lingual similarity in hidden representations and downstream task performance, revealing that LMs with stronger representation alignment across languages perform better. \cite{chai-etal-2022-cross} explored cross-linguality from a language structure perspective, emphasizing the importance of compositional properties in facilitating knowledge transfer. More recent work has focused on cross-lingual transfer from high-resource to low-resource languages \cite{zhao2024llama, zhao2024tracing}, further underscoring the asymmetries in cross-lingual knowledge integration. Our work contributes to this area by evaluating the effectiveness of factual knowledge transfer across languages using a comprehensive set of metrics designed to measure both factual recall and transferability.

\textbf{Context Sensitivity and Counterfactual Reasoning.} LMs are known to be highly sensitive to contextual cues, which can sometimes override factual knowledge when the context is misleading \cite{brown2020language, tirumala2022memorization, cotterellcontext}. \cite{ghosh2025multilingualmindsurvey} provides an in-depth review of multilingual reasoning in LMs. Counterfactual reasoning, in which models must consider hypothetical situations, has been studied in various contexts \cite{wu2023reasoning}. These studies show that LMs optimized for factual recall often struggle with counterfactual tasks, especially when faced with conflicting contextual instructions. While most prior evaluations have focused on monolingual settings \cite{shwartz-etal-2020-unsupervised, wang2020language}, our work extends these investigations into the multilingual domain. By introducing tasks like in-context recall and counterfactual adherence, we analyze how multilingual models handle both factual accuracy and contextual reasoning across languages, revealing important challenges in balancing factual knowledge and context sensitivity.

% \textbf{Multilingual Transformers.} \cite{petroni-etal-2019-language} first explored whether LMs have capacity of storing factual knowledge about entities. Prior work on multilingual models, such as mBERT \cite{devlin-etal-2019-bert}, XLM-R \cite{conneau-etal-2020-unsupervised}, mT5 \cite{xue-etal-2021-mt5} and BLOOM \cite{workshop2023bloom176bparameteropenaccessmultilingual}, has shown that LMs trained on multilingual corpora exhibit varying performance across languages. Studies have also highlighted systematic biases in factual retrieval across different languages \cite{artetxe-etal-2020-cross, liu-etal-2020-multilingual-denoising, kassner-etal-2021-multilingual, }. While multilingual QA benchmarks such as XQuAD \cite{artetxe-etal-2020-cross}, MLQA \cite{lewis-etal-2020-mlqa}, and TyDiQA \cite{clark-etal-2020-tydi} assess factual consistency, they do not explicitly measure knowledge transfer within LMs. \cite{wang-etal-2024-knowledge-mechanisms} instigated whether LMs are able to recall factual knowledge for reasoning tasks. \cite{fierro2025multilinguallanguagemodelsremember} highlights the need for new methodologies for knowledge evaluation in multilingual LMs.
% % To address this gap, we introduce a benchmark designed to evaluate cross-lingual factual knowledge transferability.


% \textbf{Cross-Lingual Knowledge Transfer in LMs.}  Several studies have delved into understanding the factors influencing the cross-lingual ability of multilingual models. While research suggests that multilingual LMs exhibit zero-shot and few-shot generalization across languages \cite{nooralahzadeh-etal-2020-zero, pfeiffer-etal-2020-mad}, empirical studies indicate that this transfer is often asymmetric, favoring high-resource languages \cite{hu2020xtrememassivelymultilingualmultitask}. \cite{muller-etal-2021-first} analyzed representation similarities and discovered a strong connection between hidden cross-lingual similarity and the
%  model’s performance on downstream tasks. \cite{chai-etal-2022-cross} examined cross-linguality from a language structure perspective, emphasizing the significance of the composition property in facilitating cross-lingual
%  transfer. Most recent work has focused on cross-lingual transfer from high-resource languages to lower-resource ones \cite{zhao2024llama, zhao2024tracing}. 

% \textbf{Context Sensitivity and Counterfactual Reasoning.}\cite{ghosh2025multilingualmindsurvey} provides the in-depth review of multilingual reasoning in LMs. LMs can be susceptible to contextual cues, often overriding stored knowledge when presented with misleading information \cite{brown2020language, tirumala2022memorization,cotterellcontext}. Counterfactual reasoning studies \cite{wu2023reasoning} show that models trained for high factual recall struggle with conflicting contextual instructions. While prior evaluations have been monolingual \cite{shwartz-etal-2020-unsupervised, wang2020language}, our study extends these investigations into the multilingual domain, introducing in-context recall and counterfactual adherence tasks to analyze cross-lingual reasoning.

\section{Dataset}
\label{sec:datasets}

\begin{table}[t]
\centering
\begin{tabular}{@{}c|c@{}}
\toprule
\textbf{Task Type}                 &     \textbf{\# Examples}\\ \midrule
\factual & 802 \\
\incontext & 156 \\
\incontextrobust & 1404
 \\ \bottomrule
\end{tabular}
\caption{Number of examples per languages in our benchmark (\S\ref{sec:datasets}).}
\label{tab:data_stats}
\end{table}


\begin{figure}[t]  % 't' places the figure at the top of the page
    \centering
    \includegraphics[width=1.\linewidth]{figures/examples.pdf}  % Adjust width as needed
    \caption{Examples from our multilingual dataset illustrating three tasks. Factual Recall: LMs recall country-specific facts better in native languages, as seen with Dharan's correct identification in Nepali but incorrect in English. Incontext Recall: Models struggle with contextual reasoning, showing regional bias when associating names with countries. Counter-Factual Context Adherence: When given counterfactual prompts about well-known figures, models rely on prior knowledge, affecting their ability to adhere to provided context. } 
    \label{fig:examples}  % For referencing in the text    
\end{figure}

We introduce a new multilingual dataset designed to evaluate three key capabilities of LMs: (a) \textit{Factual Recall}, (b) \textit{In-context Recall}, and (c) \textit{Counter-Factual Context Adherence}. The number of instances in our dataset is given in the Table~\ref{tab:data_stats}. Given the multilingual nature of our study, we categorize languages based on their resource availability in existing LM training corpora:

\textbf{High-resource}: English, Chinese, French, Japanese.

\textbf{Medium-resource}: Hindi, Russian, Arabic, Greek.

\textbf{Low-resource}: Nepali, Ukrainian, Turkish, Swahili, Thai.


These languages correspond to countries strongly associated with their usage: the United States, China, France, Japan, India, Russia, Saudi Arabia, Greece, Nepal, Ukraine, Turkey, Kenya, and Thailand. Now, we describe our datasets in detail.

\subsection{Factual Recall}

This task evaluates an LM’s ability to recall country-specific facts across multiple languages. For example, given the query, \textit{In which country is Mumbai located?}, the model should correctly respond with \textit{India} when asked in different languages.

To construct the dataset, we curated a diverse set of entities—including cities, artists, sports figures, landmarks, festivals, and politicians—for 13 selected countries. We then created standardized templates for factual queries and translated them into each language using the Google Translate API~\cite{google_translate}. All translations were manually verified and refined as needed with the assistance of ChatGPT. In total, our dataset consists of 805 unique factual questions, each available in 13 language versions.

\subsection{In-Context Recall}

The in-context recall task evaluates how effectively an LM utilizes contextual information to answer a question, ensuring that internal knowledge does not influence the model’s output.

Building on the work of \cite{feng2024languagemodelsbindentities}, we constructed our dataset by focusing on common person names associated with each country. For each example, we sampled two names and paired them with two different countries, creating context-based prompts as shown in violet color in Figure~\ref{fig:examples}.  To enhance dataset efficiency, we intentionally avoided associating a name with its most commonly linked country within the example.

\subsection{\textbf{Counter-Factual Context Adherence}}

This task evaluates an LM’s susceptibility to counterfactual information by assessing whether it adheres to the provided context when answering a question. Ideally, the model should rely solely on the given context, but in some cases, its internal knowledge may interfere or override it, leading to unintended responses~\cite{cotterellcontext}. To investigate this, we curated a list of well-known personalities strongly associated with specific countries and deliberately introduced counterfactual information into the context.

For the example given in Figure~\ref{fig:examples}, if the model defaults to its internal knowledge and answers \textit{United States}, it demonstrates a resistance to the contextual information. Conversely, if it follows the counterfactual context and answers \textit{India}, it suggests a higher reliance on the provided context rather than pre-existing knowledge.

One might expect these models to perform near-perfectly on these tasks, as they are very simple. However, despite the simplicity of these tasks, the performance varies across languages and models.

% \begin{quote}
%     \textit{Instruction: Answer the question based on the given fact.} \\
%     \textit{Fact: Mahatma Gandhi lives in China.} \\
%     \textit{Question: In which country does Mahatma Gandhi live?}
% \end{quote}
% In this work, we constructed a diverse multilingual dataset to evaluate the performance of the latest language models (LMs) in multilingual settings. We selected 13 countries based on the following languages distribution: 3 high-resource languages, 4 medium-resource languages, and 6 low-resource languages. 

% Our dataset comprises three distinct types of evaluation sets: \factual\, \incontext, and \incontextrobust. 

% The in-context-robust dataset was introduced to assess the models' ability to recall factual knowledge in scenarios where contextual understanding is crucial. All datasets were inspired by the work in \cite{feng2024languagemodelsbindentities}.

% \subsection{\factual}
% This dataset evaluates the model's factual knowledge. We randomly generated a set of entities, including cities, artists, sports figures, landmarks, festivals, and politicians, for each of the 13 selected countries using \generator~\cite{brown2020languagemodelsfewshotlearners}. The dataset consists of simple questions such as \textit{Which country does the given politician belong to?}, generated in all 13 languages.

% \subsection{\incontext}
% To construct the in-context dataset, we first generated a common person's name from each country using \generator. We then paired two names with their corresponding shuffled countries and provided the model with a prompt that asked it to identify which person resides in the given country.

% \subsection{\incontextrobust}
% For the in-context-robust dataset, we generated popular names from each country using \generator. In this case, we intentionally swapped the associated countries for each name in the context. The model is then tasked with identifying the correct country to which the person belongs, testing its ability to recall factual knowledge despite the manipulated context.


% (i) Current LMs' Factual Recall performance is dependent on factors like country and language, (ii) Current LMs struggle with simple \incontext questions in languages other than English, and (iii) Models are more prone to Factual Recall when asked a question based on the given context (\incontextrobust).

\section{Experiments}

In this section, we discuss our experimental setup, metric formulation, and both quantitative and qualitative analyses. We present the results of our experiments evaluating LMs on our dataset across diverse multilingual tasks. These experiments assess how language and country-specific factual knowledge influence LMs responses in a multilingual setting. All experiments were conducted using the latest models, with \evaluator~\cite{qwen2025qwen25technicalreport} serving as the evaluator \cite{li2024llmsasjudgescomprehensivesurveyllmbased}.
 
\subsection{Experimental Setup}
\paragraph{Models} We evaluated 14 models of varying sizes, trained on different compositions of multilingual data, and fine-tuned using various preference optimization strategies~\cite{ouyang2022training, rafailov2024directpreferenceoptimizationlanguage}, for our multilingual study. These include Deepseek~\cite{deepseekai2024deepseekllmscalingopensource}, Qwen~\cite{qwen2}, Gemma~\cite{gemmateam2024gemmaopenmodelsbased}, and Llama~\cite{touvron2023llamaopenefficientfoundation} families. Further details of the models evaluated are given in Table~\ref{tab:model-specs}.

\paragraph{Compute Details}
All our experiments were conducted on a set of 4 NVIDIA A100 GPUs, each with 80GB of VRAM. 
%We used \generator~\cite{openai2024gpt4technicalreport} for the dataset generation.

\paragraph{Evaluation}
To evaluate all models on the curated datasets (Section~\ref{sec:datasets}), we used a temperature setting of 0 and a maximum token limit of 128. Specifically, we tested the models' performance on Factual Recall and In-Context Recall across different settings. For evaluation, we designed our metrics and utilized \evaluator as the evaluator \cite{li2024llmsasjudgescomprehensivesurveyllmbased}, with a maximum token limit of 256 to support reasoning. Evaluation prompts are shown in Figures~\ref{fig:prompt1} and~\ref{fig:prompt2}.
%Our LM evaluation covers four scenarios: (i) correct response in English, (ii) correct response in the question's language, (iii) correct response in a different language, and (iv) incorrect response in any language.  


\subsection{Metric Definition and Formulation}
This section introduces our carefully designed metrics to evaluate factual recall and knowledge transferability across languages in LMs. We propose two key metrics: the \textit{Factual Recall Score (FRS)} and the \textit{Knowledge Transferability Score (KTS)}.
To establish a common metric for evaluating the model's performance in our benchmark, we compute their harmonic mean, which is defined as the \textit{Cross-Lingual Factual Knowledge Transferability Score (X-FaKT)}, to ensure a balanced assessment while penalizing large disparities between them. Our metrics incorporate an inverse formulation with a correction factor to maintain a bounded range of $[0,1]$. A higher error rate results in a lower metric value due to the inverse transformation, ensuring that better model performance corresponds to higher scores.

\subsubsection{Associative vs. Non-Associative Knowledge}  
We categorize our dataset into two groups: associative and non-associative knowledge. The categorization is defined as follows: we consider 13 languages, each associated with a corresponding country (i.e., the $i$th language belongs to the $i$th country).

\vspace{1em}
\text{Associative} = $\{Q \in \text{Questions} : Q \in \text{Language}_i \wedge \text{output}(Q) = \text{Country}_j \wedge  i = j\}$

\vspace{1em}

\text{Non-associative} = $\{Q \in \text{Questions} : Q \in \text{Language}_i \wedge \text{output}(Q) = \text{Country}_j \wedge  i \neq j\}$

\vspace{1em}
We denote the mean error rate for a country-specific fact asked in the language strongly associated with that country as $ \mu_{\text{assoc.}} $, and the mean error rate for a country-specific fact asked in a language not associated with that country as $ \mu_{\text{non-assoc.}} $.


\subsubsection{Factual Recall Score (FRS)}
Factual recall evaluates the model's ability to correctly retrieve both \textit{associative} and \textit{non-associative} knowledge. We define the Factual Recall Score (FRS) as:

\vspace{-15pt} % Adjust space before the equation
\begin{equation}
FRS = \frac{3}{2} \left( \frac{1}{\mu_{\text{assoc.}} + \mu_{\text{non-assoc.}} + 1} - \frac{1}{3} \right)
\end{equation}
\vspace{-15pt} % Adjust space before the equation


\begin{itemize}
    \setlength{\itemsep}{0pt} % Adjusts space between items
    \setlength{\parskip}{0pt} % Adjusts space between paragraphs within items
    \setlength{\parsep}{0pt}  % Adjusts space between paragraphs in list items
    \item When both errors are zero ($ \mu_{\text{assoc.}} = 0, \mu_{\text{non-assoc.}} = 0 $), the model has a perfect factual recall, yielding an FRS score of 1.
    \item When both errors are high, the denominator increases, resulting in a lower FRS score closer to 0, indicating poor factual recall.  
\end{itemize}

\subsubsection{Knowledge Transferability Score (KTS)}
Knowledge transferability quantifies how well a model maintains consistent factual knowledge across languages. We define the \textit{Knowledge Transferability Score (KTS)} as:

\vspace{-15pt} 
\begin{equation}
KTS = 2 \left( \frac{1}{\left|\mu_{\text{assoc.}} - \mu_{\text{non-assoc.}} \right| + 1} - \frac{1}{2} \right)
\end{equation}
\vspace{-15pt} 

where:
\begin{itemize}
    \setlength{\itemsep}{0pt} 
    \setlength{\parskip}{0pt} 
    \setlength{\parsep}{0pt}  
    \item $ |\mu_{\text{assoc.}} - \mu_{\text{non-assoc.}}| $ captures the absolute difference between associative and non-associative recall errors.
\end{itemize}

\begin{itemize}
    \setlength{\itemsep}{0pt} % Adjusts space between items
    \setlength{\parskip}{0pt} % Adjusts space between paragraphs within items
    \setlength{\parsep}{0pt}  % Adjusts space between paragraphs in list items
    \item When both errors are zero ($ \mu_{\text{assoc.}} = 0, \mu_{\text{non-assoc.}} = 0 $), there is perfect factual knowledge transfer, resulting in a KTS score of 1.
    \item When both errors are high but equal (e.g., $ \mu_{\text{assoc.}} = 20, \mu_{\text{non-assoc.}} = 20 $), KTS remains 1, indicating that while factual recall is poor, the model exhibits consistent errors across languages.
    \item When errors differ significantly (e.g., $ \mu_{\text{assoc.}} = 20, \mu_{\text{non-assoc.}} = 2 $ or vice versa), the absolute difference increases, leading to a lower KTS, highlighting a lack of knowledge transfer across languages.
\end{itemize}

\subsubsection{Cross-Lingual Factual Knowledge Transferability Score (X-FAKT)}
 To ensure a balanced evaluation of factual recall and cross-lingual transferability, we compute their harmonic mean:

\begin{equation}
X\text{-FAKT} = 2 \times \frac{FRS \times KTS}{FRS + KTS}
\end{equation}

where:
\begin{itemize}
    \setlength{\itemsep}{0pt} % Adjusts space between items
    \setlength{\parskip}{0pt} % Adjusts space between paragraphs within items
    \setlength{\parsep}{0pt}  % Adjusts space between paragraphs in list items
    \item The harmonic mean penalizes large disparities between factual recall (FRS) and knowledge transferability (KTS), ensuring that both contribute meaningfully to the final score.
    \item If either FRS or KTS is significantly lower, the overall score remains low, discouraging models from excelling in one metric while performing poorly in the other.
    \item A high X-FAKT score indicates that the model is both factually accurate and consistent across multiple languages.
\end{itemize}

This formulation provides a holistic evaluation of factual knowledge retention and cross-lingual consistency, making it a robust metric for assessing multilingual model performance.

\subsection{Quantitative Analysis}
\subsubsection{Performance on Factual Recall task}

\begin{figure}[t]  % 't' places the figure at the top of the page
    \centering
    \includegraphics[width=1\linewidth]{figures/Factual_Recall_Mean.png}  % Adjust width as needed
    \caption{Error rates for each model on the Factual Recall task. A clear pattern emerges, showing a decline in performance as we move from larger to smaller models (top to bottom) and from high-resource to low-resource languages (left to right).}
    \label{fig:factual_recall}  % For referencing in the text
\end{figure}


The error rate across different LMs (Figure~\ref{fig:factual_recall}) reveals a clear pattern in performance across languages and model sizes. Notably, all models demonstrate superior performance on high-resource languages like English and French, with error rates consistently below 15\% for most model variants. This performance gradually deteriorates as the model size decreases, with smaller models showing significantly higher error rates across all languages. However, an interesting observation emerges with languages like Swahili and Turkish, which despite being low-resource languages, exhibit relatively better performance with error rates comparable to mid-resource languages. This can be attributed to their use of Latin script, facilitating better knowledge transfer from English.

A compelling pattern emerges when examining languages that share similar scripts, and strong correlations in model performance among languages that share similar scripts. For example, the error patterns for Hindi-Nepali and  Russian-Ukrainian pairs show remarkable similarities, suggesting that the models effectively leverage shared scriptural characteristics during learning. These patterns indicate that script similarity plays a crucial role in the model's ability to generalize across languages, potentially offering insights into how these models transfer knowledge between different language pairs and scripts.


\begin{table}[t]
\centering
\small
\resizebox{0.49\textwidth}{!}{
\begin{tabular}{l|c|c|c|c|c|c|c}
\textbf{Model} & $\boldsymbol{\mu_{assoc.} (\%)}$ & $\boldsymbol{\mu_{non-assoc.} (\%)}$ & \textbf{t-stat} & \textbf{p-value} & \textbf{FRS} & \textbf{KTS} & \textbf{X-FAKT} \\
\midrule
\metallamains{70B} & \textbf{2.36} $\pm$ 5.12 & \textbf{9.85} $\pm$ 10.54 & 2.52 & 0.01 & \textbf{0.835} & \textbf{0.862} & \textbf{0.848} \\
\gemmains{27B} & 4.23 $\pm$ 8.49 & 16.46 $\pm$ 17.07 & 2.54 & 0.01 & 0.742 & 0.783 & 0.762 \\
\phimodel{4-14B} & 12.87 $\pm$ 16.51 & 30.15 $\pm$ 25.92 & 2.35 & 0.02 & 0.548 & 0.706 & 0.617 \\
\phimodel{3-14B} & 25.09 $\pm$ 29.84 & 55.57 $\pm$ 36.24 & 2.93 & $<$0.01 & 0.330 & 0.535 & 0.408 \\
\gemmains{9B} & 4.98 $\pm$ 6.09 & 22.32 $\pm$ 21.37 & 2.90 & $<$0.01 & 0.677 & 0.705 & 0.691 \\
\metallamains{8B} & 4.60 $\pm$ 7.54 & 25.77 $\pm$ 19.61 & 3.85 & $<$0.01 & 0.649 & 0.651 & 0.650 \\
\orca & 31.95 $\pm$ 31.65 & 56.77 $\pm$ 32.99 & 2.60 & 0.01 & 0.295 & 0.603 & 0.396 \\
\deepseek{7b} & 31.49 $\pm$ 30.68 & 63.73 $\pm$ 36.29 & 3.09 & $<$0.01 & 0.268 & 0.514 & 0.353 \\
\mistralins{7B} & 16.96 $\pm$ 15.65 & 45.25 $\pm$ 29.34 & 3.42 & $<$0.01 & 0.424 & 0.559 & 0.483 \\
\phimodel{3.5-4B} & 41.85 $\pm$ 31.62 & 69.87 $\pm$ 31.23 & 3.09 & $<$0.01 & 0.208 & 0.563 & 0.304 \\
\phimodel{3-4B} & 42.45 $\pm$ 30.99 & 77.95 $\pm$ 33.72 & 3.65 & $<$0.01 & 0.181 & 0.477 & 0.262 \\
\llamains{3B} & 24.10 $\pm$ 17.80 & 47.48 $\pm$ 26.80 & 3.07 & $<$0.01 & 0.375 & 0.620 & 0.467 \\
\gemmains{2B} & 9.97 $\pm$ 14.78 & 45.77 $\pm$ 31.30 & 4.06 & $<$0.01 & 0.463 & 0.473 & 0.468 \\
\llamains{1B} & 34.74 $\pm$ 22.32 & 65.96 $\pm$ 26.98 & 4.03 & $<$0.01 & 0.247 & 0.524 & 0.336 \\
\bottomrule
\end{tabular}
}
\caption{Results of the t-test comparing associative and non-associative knowledge across models, alongside FRS, KTS, and X-FAKT scores. (A) Llama-3-70B achieves the best performance in both factual recall and knowledge transferability. (B) There is a statistically significant difference between the performance on associative queries (asked in a country's native language) and non-associative queries (asked in other languages).}
\label{tab:tstats_factual}
\end{table}
% (C) A general trend indicates that as model scale increases, performance in factual recall (FRS) and knowledge transferability (KTS) improves.

\paragraph{Knowledge Transferability Analysis:}

\begin{figure}[t]  % 't' places the figure at the top of the page
    \centering
    \includegraphics[width=1\linewidth]{figures/xfact.png}  % Adjust width as needed
    \caption{This figure illustrates the model-wise comparison of X-FAKT scores grouped by language families. A clear trend emerges, showing that as the model size increases within a family, the X-FAKT score tends to increase.}
    \label{fig:xfact_bar}  % For referencing in the text
\end{figure}

From Table~\ref{tab:tstats_factual}, \metallamains{70B} emerges as the clear leader with the highest X-FAKT score of 0.848, demonstrating superior balanced performance in both factual recall (FRS = 0.835) and knowledge transferability (KTS = 0.862). This exceptional performance is supported by the lowest error rates ($\mu_{assoc.}$ = 2.36\%, $\mu_{non-assoc.}$ = 9.85\%), suggesting that larger model sizes generally correlate with better cross-lingual factual knowledge handling. Despite similar model sizes, significant performance variations exist between different architectures. For example, \gemmains{9B} (X-FAKT: 0.691) substantially outperforms \mistralins{7B} (X-FAKT: 0.483), suggesting that architecture design and training methodology play crucial roles beyond mere parameter count. As illustrated in Figure~\ref{fig:xfact_bar}, the X-FAKT scores exhibit a clear upward trend with increasing model size within each language family. This suggests that larger models generally achieve better factual consistency, highlighting the impact of scale on model performance. These findings provide valuable insights into the current state of cross-lingual factual knowledge in LMs and highlight areas for future improvement, particularly in reducing the performance gap between associative and non-associative knowledge retrieval.

\paragraph{Associative vs. Non-associative performance:}


    We analyze the performance of various models on these two subsets of data and report the results in the Table~\ref{tab:tstats_factual}. For all models, the t-statistic and p-value indicate that the differences between associative and non-associative categories are statistically significant (p-value less than 0.05). 



    \paragraph{Performance comparison across language groups:} 

    \begin{table}[t]
\centering
\resizebox{0.48\textwidth}{!}{
\begin{tabular}{l|c|c}
\textbf{Language} & $\boldsymbol{\mu_{assoc.} (\%)}$ & $\boldsymbol{\mu_{non-assoc.} (\%)}$ \\
\hline
High & 3.83 $\pm$ 3.79 & 29.84 $\pm$ 27.47 \\
Medium & 26.73 $\pm$ 17.60 & 50.54 $\pm$ 21.20 \\
Low & 29.53 $\pm$ 16.19 & 53.91 $\pm$ 23.68 \\

\hline
\end{tabular}
}
\caption{Average mean and standard deviation for error rate across all models for each language group. High-resource languages exhibit lower error rates compared to low-resource languages.}
%  The "Associative" category refers to samples belonging to a region and sharing the same language as that of the region, while the "Unassociative" category includes all other samples.
\label{tab:stats_factual_availability}
\end{table}
% Notably, error rates are significantly higher for non-associative queries, indicating challenges in handling queries outside the native language context.
    
    In this study, we categorize languages into three groups based on their availability and coverage in the dataset: \textbf{High}, \textbf{Medium}, and \textbf{Low}, as defined in Section~\ref{sec:datasets}. From the results shown in Table~\ref{tab:stats_factual_availability}, we observe a clear trend across language groups. Specifically, high resouce languages exhibit the lowest average error rates, particularly in the associative category, where models make fewer mistakes ($\mu_{assoc.}$ = 3.83\%). However, for non-associative questions, the error rate rises significantly ($\mu_{non-assoc.}$ = 29.84\%), indicating that models struggle more when dealing with non-associative samples in these languages. The error rate increases while moving from high to low-resource languages.

% \begin{table}[t]
% \centering
% \small
% \resizebox{0.48\textwidth}{!}{
% \begin{tabular}{l|c|c|c|c|c}
% \textbf{Model} & \textbf{Associative} & \textbf{Unassociative} & \textbf{FRS} & \textbf{KTS} & \textbf{X-FAKT} \\
% \midrule
% \metallamains{70B} & 2.36 & 9.85 & 7.57 & 11.78 & 9.22 \\
% \gemmains{27B} & 4.23 & 16.46 & 4.61 & 7.56 & 5.73 \\
% \phimodel{4} & 12.87 & 30.15 & 2.27 & 5.47 & 3.21 \\
% \phimodel{3-medium-128k-Inst} & 25.09 & 55.57 & 1.22 & 3.18 & 1.77 \\
% \gemmains{9B} & 4.98 & 22.32 & 3.53 & 5.45 & 4.29 \\
% \metallamains{8B} & 4.60 & 25.77 & 3.19 & 4.51 & 3.74 \\
% \orca & 31.95 & 56.77 & 1.11 & 3.87 & 1.73 \\
% \deepseek{7b-chat} & 31.49 & 63.73 & 1.04 & 3.01 & 1.54 \\
% \mistralins{7B} & 16.96 & 45.25 & 1.58 & 3.41 & 2.16 \\
% \phimodel{3.5-mini-Inst} & 41.85 & 69.87 & 0.89 & 3.45 & 1.41 \\
% \phimodel{3-mini-128k-Inst} & 42.45 & 77.95 & 0.82 & 2.74 & 1.27 \\
% \llamains{3B} & 24.10 & 47.48 & 1.38 & 4.10 & 2.06 \\
% \gemmains{2B} & 9.97 & 45.77 & 1.76 & 2.72 & 2.14 \\
% \llamains{1B} & 34.74 & 65.96 & 0.98 & 3.10 & 1.49 \\
% \bottomrule
% \end{tabular}
% }
% \caption{Evaluation of models based on Associative and Unassociative errors, Factual Recall Score (FRS), Knowledge Transferability Score (KTS), and Cross-Lingual Factual Knowledge Transferability Score (X-FAKT). All scores (FRS, KTS, X-FAKT) are multiplied by 100, with a resolution of 0.01. \aasays{standarise the model names}} 
% \label{tab:model_scores}
% \end{table}


\subsubsection{Performance on In-Context Recall task}



Figure~\ref{fig:incontext_recall} demonstrates the incorrectness rate for the in-context recall capabilities of different LMs. Despite being a simple task, certain models such as \textit{\deepseek{7B}, \orca, \phimodel{3-4B}, \llamains{1B}, and \mistralins{7B}} perform poorly across multiple languages. This suggests that these models struggle to effectively utilize contextual information when generating outputs. Interestingly, even for languages like Swahili and Turkish, which showed better scores in the Factual Recall task, models demonstrate poor performance on this context-dependent task. This stark contrast suggests that the benefits of Latin script-based knowledge transfer observed in the Factual Recall task do not extend to in-context learning scenarios, where performance depends primarily on the model's ability to process and utilize contextual information.

% As mentioned in the dataset section, we intentionally paired cross-entities as context. This setup appears to induce a regional bias, which negatively impacts model performance. The structured entity-context pairing in the dataset may have led to spurious correlations \cite{pmlr-v202-yang23j, ye2024spuriouscorrelationsmachinelearning}, reducing model accuracy in in-context recall tasks. Some models struggled to effectively leverage contextual information, highlighting potential weaknesses in their retrieval and in-context learning mechanisms.

As mentioned in the dataset section, we intentionally paired cross-entities as context. This setup appears to induce a regional bias, which negatively impacts model performance. The structured entity-context pairing in the dataset may have led to spurious correlations \cite{pmlr-v202-yang23j, ye2024spuriouscorrelationsmachinelearning}, reducing model accuracy in in-context recall tasks. Some models struggle to effectively leverage contextual information, revealing potential weaknesses in their retrieval and in-context learning mechanisms. 

%For instance, consider the following query:  
% \begin{quote}
% \textbf{Context:} Li Wei lives in India and Raj lives in China.  \\
% \textbf{Question:} Who lives in China?  
% \end{quote}  
% A model relying on spurious correlations may incorrectly predict ``Li Wei`` instead of ``Raj`` due to learned biases associating Chinese names with China, rather than correctly interpreting the given context. 

\subsubsection{Performance on Counter-Factual Context
Adherence task}

\begin{figure}[t]  % 't' places the figure at the top of the page
    \centering
    \includegraphics[width=1\linewidth]{figures/Counterfactual_Context_Adherence.png}  % Adjust width as needed
    \caption{Error rate for each model on \incontextrobust task. Models show high error rates in high resource languages such as English and French where they have high factual recall.}
    \label{fig:counterfactual_adherence}  % For referencing in the text
\end{figure}


Figure~\ref{fig:counterfactual_adherence} illustrates the error rates of LMs in the Counterfactual Context Adherence task. Notably, Latin-script languages (English, French, Swahili, and Turkish), which performed well in factual recall tasks, exhibited significantly higher error rates in counterfactual adherence. This suggests a fundamental trade-off in the models' capabilities: their strength in accurately retrieving factual information appears to come at the expense of their ability to maintain adherence to counterfactual contexts. This inverse relationship raises important questions about the inherent limitations and trade-offs in LMs' learning mechanisms, particularly in how they balance factual knowledge with hypothetical reasoning.


 \subsection{Qualitative Analysis}

% \paragraph{Region-Specific Knowledge in the Native Language}
% Our analysis revealed a notable pattern in some language models, where they exhibit stronger factual recall for region-specific questions when asked in the language of the respective region. For instance, when we prompted the Gemma-2-9B model with the question, \textit{Rashed Al Shashai is from which country?} in English, the model responded \textit{I do not have access to real-time information,
% including personal details about individuals like Rashed Al
% Shashai.
% To find out where Rashed Al Shashai is from, I recommend trying
% the following:
% **Search online:** Use a search engine like Google and search
% for "Rashed Al Shashai".
% **Check social media:** Look for profiles on platforms like
% LinkedIn, Facebook, or Twitter.
% **Look for news articles or publications:** If Rashed Al Shashai is
% involved in any public activities, there might be news articles or
% publications mentioning his nationality.} 

% \begin{figure}[t]  % 't' places the figure at the top of the page
%     \centering
%     \includegraphics[width=1\linewidth]{figures/region_specific.pdf}  % Adjust width as needed
%     \caption{An example (response generated by Llama-3-70B) showing model doesn't know facts in English.}
%     \label{fig:region_specific}  % For referencing in the text
% \end{figure}


% The response demonstrates that the model was unable to directly answer the question, instead offering generalized advice. However, when we asked the same question in Arabic, the model correctly identified Rashed Al Shashai as being from Saudi Arabia. This suggests that the model may have a stronger ability to recall region-specific facts when queried in the local language, pointing to potential language-dependent knowledge retrieval. This finding highlights the importance of considering the language in which factual questions are posed when assessing an LM's cross-lingual capabilities.

\paragraph{Spurious correlation leads to in-context recall failures.}
We observe that some models tend to associate names with cultural origins, even when contextual evidence contradicts this assumption. Figures ~\ref{fig:liwei} demonstrate the model response when prompted \mistralins{7B} with the contextual understanding-based question in English. 
\begin{figure}[t]  % 't' places the figure at the top of the page
    \centering
    \includegraphics[width=1\linewidth]{figures/liwei.png}  % Adjust width as needed
    \caption{\mistralins{7B} output when prompted with the given context in English. This model generation shows how spurious correlation leads to in-context recall failures}
    \label{fig:liwei}  % For referencing in the text
\end{figure}

Despite the explicit context stating that \textit{Li Wei} resides in \textit{Russia}, the model disregards this information and defaults to cultural associations. This behavior reveals a limitation in integrating contextual evidence when making country-specific inferences.




\paragraph{Models favor factual knowledge over context.}
We also observed that some models prioritize their internal factual knowledge over contextual information when responding to questions about well-known personalities. Figures ~\ref{fig:george_wash} demonstrate the model response when prompted \metallamains{70B} with the factual retrieval query in English.

\begin{figure}[t]  % 't' places the figure at the top of the page
    \centering
    \includegraphics[width=1\linewidth]{figures/george_washi.png}  % Adjust width as needed
    \caption{\metallamains{70B} output when prompted with a counter-factual context adherence query in English. This shows LMs  favour internal knowledge over contextual understanding.}
    \label{fig:george_wash}  % For referencing in the text
\end{figure}

In this case, despite being explicitly told that \textit{`George Washington'} lived in \textit{'India'}, the model relied on its factual knowledge, correcting the given fact and asserting that \textit{`George Washington'} lived in the \textit{`United States'}. This response demonstrates the model’s strong reliance on factual accuracy, rather than adapting to the context provided. It suggests that when it comes to well-known historical figures, models may prioritize prior knowledge over the specific context they are given.


\paragraph{Linguistic variability in word interpretation.}  
LMs can interpret words differently depending on the language. Figures ~\ref{fig:french_query} and ~\ref{fig:hin_query} demonstrate the model responses when prompted \metallamains{70B} with the same queries but in different languages. This highlights challenges in multilingual consistency, where the model misinterprets \textit{`Dijon'} as \textit{`De Janeiro'} in Hindi, revealing inconsistencies in cross-lingual factual retrieval.

\begin{figure}[t]  % 't' places the figure at the top of the page
    \centering
    \includegraphics[width=1\linewidth]{figures/french_qual.png}  % Adjust width as needed
    \caption{\metallamains{70B} output when prompted with a factual recall query in English}
    \label{fig:french_query}  % For referencing in the text
\end{figure}

\begin{figure}[t]  % 't' places the figure at the top of the page
    \centering
    \includegraphics[width=1\linewidth]{figures/hindi_qual.png}  % Adjust width as needed
    \caption{\metallamains{70B} output when prompted with a factual recall query in Hindi. In Hindi, it misinterprets understanding of a French word.}
    \label{fig:hin_query}  % For referencing in the text
\end{figure}



\paragraph{Challenges with using LMs as evaluators.}  
We used a zero-shot prompt with \metallamains{70B} as an evaluator and found that its inherent factual knowledge can skew assessments. For example, when evaluating a \gemmains{27B} response to the counterfactual context task—\textit{``Catherine the Great lives in India''}—the evaluator corrected it, asserting that she lived in \textit{``Russia''}, despite the provided ground truth. This bias highlights the need to control evaluators' factual knowledge to ensure consistent evaluation.

% In our initial experiments, we employed a zero-shot prompt along with \llamains{70B} as an evaluator. However, we identified several limitations when relying on LLMs for evaluation tasks. One prominent issue is that the evaluator's judgments are often influenced by its underlying factual knowledge, which may not always align with the context of the input-output-ground-truth triplet being assessed.

% For instance, when we asked the evaluator to assess the following response generated by \gemmains{27B}:
% \textit{\textbf{Fact}: Catherine the Great lives in the country of India.}
% \textit{\textbf{Question}: In which country does Catherine the Great live?}
% \textit{\textbf{Response}: This is a trick question! 
% Catherine the Great was the Empress of Russia, not India.  She lived in Russia in the 18th century.}

% The evaluator produced the following output:
% \textit{\textbf{Explanation}: The assistant's answer is in English and corrects the incorrect fact provided, stating that Catherine the Great lived in Russia, not India. The response is relevant and accurate.
% Rating: [[2]]}

% This example clearly demonstrates how an LLM’s factual knowledge can interfere with its evaluation capabilities. This reveals that the evaluator's knowledge can skew its judgment, leading to misleading ratings. It is crucial, therefore, to control the evaluator's factual knowledge to ensure it functions effectively and consistently.

% \subsection{Ablation of System Prompt}

% \begin{table}[t]
% \centering
% \resizebox{0.48\textwidth}{!}{
% \begin{tabular}{l | c | c | c | c | c | c}
% \textbf{Model} & \multicolumn{2}{c|}{\textbf{FR}} & \multicolumn{2}{c|}{\textbf{IR}} & \multicolumn{2}{c}{\textbf{CCA}} \\
% & No SP & English SP & No SP & English SP & No SP & English SP \\
% \hline
% \mistralins{7B} & 0.43 & 0.40 & 0.43 & 0.33 & 0.37 & 0.47 \\
% \metallamains{8B} & 0.24 & 0.26 & 0.11 & 0.07 & 0.30 & 0.36 \\
% \hline
% \end{tabular}
% }
% \caption{}
% \label{tab:stats_factual_availability}
% \end{table}

% \section{Future Work}

% % \begin{figure}[t]  % 't' places the figure at the top of the page
% %     \centering
% %     \includegraphics[width=1\linewidth]{figures/well_calibrated_multilingualism.png}  % Adjust width as needed
% %     \caption{Future Work Proposal: Calibrated Multilingualism Paradigm. \aasays{TODO: explain a bit more}}
% %     \label{fig:calmul}  % For referencing in the text
% % \end{figure}

% Our analysis indicates that while LLMs can accurately retrieve country-specific facts when queries are posed in the associated native language, they struggle to do so in other languages. We propose a concept of "language thinking mode", which refers to an internal operational state a multilingual LLM employs when generating responses in a specific language. In this mode, the model activates language-specific representations that are optimized for that particular language based on its training. With this, we highlight the need for developing AI systems that possess an internal awareness of their language-specific strengths and weaknesses—a concept we term calibrated multilingualism. Under this paradigm, a model would autonomously adjust its "language thinking mode" to leverage the most reliable internal representations for any given multilingual query. We plan to apply standard RL techniques~\cite{anthony2017thinking} that enable AI systems to develop self-reflection capabilities~\cite{guo2025deepseek,shinn2024reflexion}. We present our proposed pipeline for the same in Figure~\ref{fig:calmul}.
    
   

%AA: Need to pass
\section{Conclusions}
Our study reveals a critical limitation in multilingual LMs: their inability to consistently transfer factual knowledge across languages. Our benchmark provides a standardized framework to evaluate both current and future LMs on their factual consistency and cross-lingual generalization, enabling a more systematic comparison of their capabilities. Moreover, it can serve as a valuable resource to promote research in interpretability by helping analyze how and where factual knowledge is stored and retrieved across languages, fostering a deeper understanding of LM internals. We emphasize the need for AI systems with internal awareness of their language-specific strengths and weaknesses—a concept we term calibrated multilingualism. Under this paradigm, a model would autonomously leverage the most reliable internal representations for any given multilingual query.

We also find that LMs, when used as evaluators, are biased by their internal factual knowledge, which may not align with the intended input-output-ground-truth context. This underscores the need to control the evaluator’s factual knowledge for more reliable assessments. Ultimately, enabling AI to cross-generalize across languages is crucial for inclusive and equitable technology, ensuring language is no barrier to reliable knowledge access.


\section{Limitations}
% While our study provides valuable insights into cross-lingual knowledge transfer in LMs, several limitations should be acknowledged. 
% First, our benchmark, although comprehensive in its coverage of country-related facts, is constrained to 13 languages. This represents only a fraction of the world's languages and may not fully capture the complexity of cross-lingual knowledge transfer across more diverse linguistic families. 
% Second, our evaluation was limited to open-source LMs, excluding proprietary models that might exhibit different patterns of cross-lingual knowledge transfer. While this choice ensures reproducibility and transparency, it may not represent the full spectrum of current LM capabilities.
% Third, our fact collection relied on a standardized template format for consistency and scalability. While this approach facilitates systematic evaluation, it may not capture how these models perform with more diverse phrasings or natural language variations. Real-world queries about factual knowledge often come in various forms, and our template-based approach might not fully reflect this diversity.
% Finally, our benchmark primarily focuses on country-related facts, which, while important, represent only one domain of knowledge. The patterns of cross-lingual knowledge transfer we observed might differ for other types of information, such as scientific concepts, historical events, or cultural phenomena.

Our study provides valuable insights into cross-lingual knowledge transfer in LMs but has some limitations. First, our benchmark, though comprehensive in country-related facts, covers only 13 languages, limiting its representation of diverse linguistic families. Second, we evaluated only open-source LMs, excluding proprietary models that may exhibit different transfer patterns. Third, our fact collection used a standardized template for consistency, which may not reflect the diversity of real-world queries. Lastly, our focus on country-related facts means our findings may not generalize to other domains like science, history, or culture.

\section{Ethics Statement}
This research is conducted with a strong commitment to ethical principles, ensuring data privacy and consent by using publicly available information and adhering to data protection regulations. We acknowledge potential biases in multilingual language models and aim to highlight and address these through our benchmark. Transparency and reproducibility are promoted by making our dataset and evaluation framework publicly available. Our research aligns with the broader goals of fairness, transparency, and social responsibility.


\section{Acknowledgment}
We thank Alessandro Sordoni, Prachi Jain, Rishav Hada, Chanakya Ekbote, Anirudh Buvanesh, and Ankur Sikarwar for their valuable feedback. We acknowledge the support of Ayush Agrawal’s PhD advisors, Aaron Courville and Navin Goyal.

\clearpage
% \bibliographystyle{latex/acl}
\bibliography{custom}

\clearpage


\onecolumn
\appendix
\section{APPENDIX}
\label{sec:appendix}
\setcounter{table}{0}  % Reset the table counter
\renewcommand{\thetable}{\Alph{section}.\arabic{table}}  


% \subsection{Models Architecture Details}


\begin{figure}[!htb]  % 't' places the figure at the top of the page
    \centering
    \includegraphics[width=1\linewidth]{figures/model_comparison_plot.png}  % Adjust width as needed
    \caption{Comparision of models (in the increasing order of size with respect to the parameters) using Factual Recall Score, Knowledge Transferability Score, and Cross-Lingual Factual Knowledge Transferability Score.}
    \label{fig:models_comparison}  % For referencing in the text
\end{figure}


\begin{table*}[ht]
\centering
\resizebox{0.99\textwidth}{!}{
\begin{tabular}{l|c|c|c|c|c|c|c}
\hline
\textbf{Model} & \textbf{Model Size} & \textbf{Training} & \textbf{Languages} & \textbf{Context} & \textbf{Vocab} & \textbf{Post-Training} & \textbf{Key} \\
& \textbf{\& Architecture} & \textbf{Data} & \textbf{Supported} & \textbf{Length} & \textbf{Size} & \textbf{Strategies} & \textbf{Features} \\
\hline
\metallamains{70B} & 70B & 15T tokens & EN, DE, FR, IT, & 8K & 128K & SFT, RS, DPO & GQA, 8 heads, \\
 & L=80, H=64 & Multi-lingual & PT, HI, ES, TH & & & & RoPE embeddings \\
\hline
\gemmains{27B} & 27B & 13T tokens & Primarily & 8K & 256K & SFT, RLHF & Local-global attention, \\
& & Web, Code, Math & English & & & & Knowledge distillation \\
\hline
\phimodel{4-14B} & 14B & 400B synthetic & DE, ES, FR, PT, & 16K & 100K & SFT, RS, & Full attention over \\
& & + 10T web & IT, HI, JA & & & DPO & 4K context \\
\hline
\phimodel{3-14B} & 14B & 4.8T tokens & 10\% multilingual & 128K & 32K & SFT, DPO & Reasoning focus, \\
 & & & data & & & & Multi-lingual support \\
\hline
\gemmains{9B} & 9B & 8T tokens & Primarily & 8K & 256K & SFT, RLHF & GQA, RoPE, \\
& & & English & & & & Knowledge distillation \\
\hline
\metallamains{8B} & 8B & 15T tokens & EN, DE, FR, IT, & 8K & 128K & SFT, RS, & GQA, RoPE, \\
 & L=32, H=32 & Multi-lingual & PT, HI, ES, TH & & & DPO & 32 heads \\
\hline
\orca & 7B & Based on & Based on & 4K & 32K & Single-turn & Enhanced reasoning \\
& L=32, H=32 & Llama 2 & Llama 2 & & & SFT & abilities \\
\hline
\deepseek{7B} & 7B & 2T tokens & English & 4K & 102K & SFT, DPO & English \& Chinese \\
 & L=30, H=32 & & \& Chinese & & & & focus \\
\hline
\mistralins{7B} & 7B & Open Web & Open Web & 32K & 32K & SFT & GQA, Sliding window \\
 & L=32, H=32 & & languages & & & & attention \\
\hline
\phimodel{3.5-4B} & 3.8B & 3.4T tokens & 23 languages incl. & 128K & 32K & SFT, DPO & Multi-lingual \\
 & L=32, H=32 & Multi-lingual & AR, ZH, CS, NL, & & & & support \\
\hline
\phimodel{3-4B} & 3.8B & 4.9T tokens & Similar to & 128K & 32K & SFT, DPO & Diverse domain \\
 & & & Phi-3.5-mini & & & & coverage \\
\hline
\llamains{3B} & 3B & 9T tokens & EN, DE, FR, IT, & 128K & 128K & SFT, RS, & GQA, Knowledge \\
 & & Multi-lingual & PT, HI, ES, TH & & & DPO & distillation \\
\hline
\gemmains{2B} & 2B & 2T tokens & Primarily & 8K & 256K & SFT, RLHF & GQA, RoPE, \\
 & & & English & & & & Knowledge distillation \\ \hline
\llamains{1B} & 1B & 9T tokens & EN, DE, FR, IT, & 128K & 128K & SFT, RS, & GQA, Multi-lingual \\
 & & Multi-lingual & PT, HI, ES, TH & & & DPO & dialogue focus \\
\hline
\end{tabular}
}
\caption{\textbf{Model Specifications}: Comparison of LMs across key architectural and training characteristics. Languages abbreviated as: EN=English, DE=German, FR=French, IT=Italian, PT=Portuguese, HI=Hindi, ES=Spanish, TH=Thai, AR=Arabic, ZH=Chinese, CS=Czech, NL=Dutch, JA=Japanese. L=Layers, H=Heads, SFT=Supervised Fine-Tuning, RS=Rejection Sampling, DPO=Direct Preference Optimization, GQA=Grouped-Query Attention, RoPE=Rotary Position Embedding.}
\label{tab:model-specs}
\end{table*}
%We also manually checked the correctness of the data.

% \subsection{Prompts}



\begin{figure*}[ht]
\centering
\includegraphics[scale=0.085]{figures/prompt_incontext_factual.pdf}
\caption{Prompt used for evaluation of Factual Recall and In-Context Recall tasks.}
\label{fig:prompt1}
\end{figure*}


\begin{figure*}[ht]
\centering
\includegraphics[scale=0.085]{figures/prompt_counterfactual.pdf}
\caption{Prompt used for evaluation of \incontextrobust task.}
\label{fig:prompt2}
\end{figure*}






% \subsection{Evaluator Outputs}

\begin{table*}[ht]
\centering
\resizebox{1.0\textwidth}{!}{
\begin{tabular}{|p{0.3\textwidth}|p{0.3\textwidth}|p{0.15\textwidth}|p{0.25\textwidth}|}
\hline
\textbf{Question} & \textbf{Model Prediction} & \textbf{Ground Truth} & \textbf{LLM Evaluation Output} \\
\hline
New York iko katika nchi gani? & New York iko nchini **Marekani**. & Marekani & Response: The assistant's response is in the same language as the reference and aligns with the reference provided. Rating: [[1]] \\
\hline
Talimat: Verilen gerçeğe göre soruyu cevaplayın.
Gerçek: George Washington, Amerika Birleşik Devletleri ülkesinde yaşıyor.
Soru: George Washington hangi ülkede yaşıyor? & George Washington, **Amerika Birleşik Devletleri**'nde yaşıyor. & Amerika Birleşik Devletleri & The assistant's response is in the same language as the reference and aligns with the reference. Rating: [[1]] \\
\hline
David lives in China and Li Wei lives in United States. Who lives in China? & David lives in China. & David & Response: The assistant's response is in the same language and aligns with the reference provided. Rating: [[1]] \\
\hline
Dans quel pays se trouve Sourate ? & Sourate n'est pas un pays, mais le nom d'une **ville** située en **Algérie**. & Inde & The assistant's response is in French, which aligns with the reference language, but the information provided does not align with the reference. The assistant incorrectly states that Sourate is a city in Algeria, while the reference is about India. Rating: [[4]] \\
\hline
\end{tabular}
}
\caption{Some outputs from evaluator \evaluator. Rating[1-3]: Correct and Rating[4]: Incorrect.}
\label{tab:eval_examples}
\end{table*}

\begin{figure}[t]  % 't' places the figure at the top of the page
    \centering
    \includegraphics[width=1\linewidth]{figures/In-context_Recall_Mean.png}  % Adjust width as needed
    \caption{Error rate for each model on \incontext task. Clearly, few models such as \deepseek{7B}, \phimodel{3-4B}, etc. performs poorly on this simple task.}
    \label{fig:incontext_recall}  % For referencing in the text
\end{figure}

% \subsection{English Fallback Rate}


\begin{figure}[t]  % 't' places the figure at the top of the page
    \centering
    \includegraphics[width=1\linewidth]{figures/fall_back_Factual_Recall.pdf}  % Adjust width as needed
    \caption{English Fall Back Rate across models (The English Fall Back Rate measures the frequency with which a model defaults to English in its output).}
    \label{fig:english_fall_back}  % For referencing in the text
\end{figure}


\begin{figure}[t]  % 't' places the figure at the top of the page
    \centering
    \includegraphics[width=1.\linewidth]{figures/Factual_Recallllama-70.pdf}  % Adjust width as needed
    \caption{Country-Specific Factual Error Rates in each language for \metallamains{70B}}
    \label{fig:factual_recall_best}  % For referencing in the text    
\end{figure}


\begin{figure}[t]  % 't' places the figure at the top of the page
    \centering
    \includegraphics[width=1.\linewidth]{plots/gemma-2-27b-it.pdf}  % Adjust width as needed
    \caption{Country-Specific Factual Error Rates in each language for \gemmains{27B}}
    \label{fig:factual_recall_gemma_27}  % For referencing in the text    
\end{figure}

\begin{figure}[t]  % 't' places the figure at the top of the page
    \centering
    \includegraphics[width=1.\linewidth]{plots/phi-4.pdf}  % Adjust width as needed
    \caption{Country-Specific Factual Error Rates in each language for \phimodel{4-14B}}
    \label{fig:factual_recall_phi_4_14}  % For referencing in the text    
\end{figure}

\begin{figure}[t]  % 't' places the figure at the top of the page
    \centering
    \includegraphics[width=1.\linewidth]{plots/Phi-3-medium-128k-instruct.pdf}  % Adjust width as needed
    \caption{Country-Specific Factual Error Rates in each language for \phimodel{3-14B}}
    \label{fig:factual_recall_phi_3_14}  % For referencing in the text    
\end{figure}


\begin{figure}[t]  % 't' places the figure at the top of the page
    \centering
    \includegraphics[width=1.\linewidth]{plots/gemma-2-9b-it.pdf}  % Adjust width as needed
    \caption{Country-Specific Factual Error Rates in each language for \gemmains{9B}}
    \label{fig:factual_recall_gemma_9}  % For referencing in the text    
\end{figure}


\begin{figure}[t]  % 't' places the figure at the top of the page
    \centering
    \includegraphics[width=1.\linewidth]{plots/Meta-Llama-3-8B-Instruct.pdf}  % Adjust width as needed
    \caption{Country-Specific Factual Error Rates in each language for \metallamains{8B}}
    \label{fig:factual_recall_llama_3_8}  % For referencing in the text    
\end{figure}

\begin{figure}[t]  % 't' places the figure at the top of the page
    \centering
    \includegraphics[width=1.\linewidth]{plots/Orca-2-7b.pdf}  % Adjust width as needed
    \caption{Country-Specific Factual Error Rates in each language for \orca}
    \label{fig:factual_recall_orca}  % For referencing in the text    
\end{figure}


\begin{figure}[t]  % 't' places the figure at the top of the page
    \centering
    \includegraphics[width=1.\linewidth]{plots/deepseek-7b.pdf}  % Adjust width as needed
    \caption{Country-Specific Factual Error Rates in each language for \deepseek{7B}}
    \label{fig:factual_recall_deepseek}  % For referencing in the text    
\end{figure}


\begin{figure}[t]  % 't' places the figure at the top of the page
    \centering
    \includegraphics[width=1.\linewidth]{plots/Mistral-7B-Instruct-v0.2.pdf}  % Adjust width as needed
    \caption{Country-Specific Factual Error Rates in each language for \mistralins{7B}}
    \label{fig:factual_recall_mistral}  % For referencing in the text    
\end{figure}


\begin{figure}[t]  % 't' places the figure at the top of the page
    \centering
    \includegraphics[width=1.\linewidth]{plots/Phi-3.5-mini-instruct.pdf}  % Adjust width as needed
    \caption{Country-Specific Factual Error Rates in each language for \phimodel{3.5-4B}}
    \label{fig:factual_recall_phi_3.5_4}  % For referencing in the text    
\end{figure}


\begin{figure}[t]  % 't' places the figure at the top of the page
    \centering
    \includegraphics[width=1.\linewidth]{plots/Phi-3-mini-128k-instruct.pdf}  % Adjust width as needed
    \caption{Country-Specific Factual Error Rates in each language for \phimodel{3-4B}}
    \label{fig:factual_recall_phi_3_4}  % For referencing in the text    
\end{figure}


\begin{figure}[t]  % 't' places the figure at the top of the page
    \centering
    \includegraphics[width=1.\linewidth]{plots/Llama-3.2-3B-Instruct.pdf}  % Adjust width as needed
    \caption{Country-Specific Factual Error Rates in each language for \llamains{3B}}
    \label{fig:factual_recall_llama_3.2_3}  % For referencing in the text    
\end{figure}


\begin{figure}[t]  % 't' places the figure at the top of the page
    \centering
    \includegraphics[width=1.\linewidth]{plots/gemma-2-2b-it.pdf}  % Adjust width as needed
    \caption{Country-Specific Factual Error Rates in each language for \gemmains{2B}}
    \label{fig:factual_recall_gemma_2_2}  % For referencing in the text    
\end{figure}



\begin{figure}[t]  % 't' places the figure at the top of the page
    \centering
    \includegraphics[width=1.\linewidth]{figures/Factual_Recallllama-1.pdf}  % Adjust width as needed
    \caption{Country-Specific Factual Error Rates in each language for \llamains{1B}}
    \label{fig:factual_recall_worst}  % For referencing in the text    
\end{figure}




\end{document}