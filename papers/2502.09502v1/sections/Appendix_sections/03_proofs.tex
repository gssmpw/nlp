\section{Proofs}
\label{appendix_sec:proofs}

\begin{namedlemma}
    [~\ref{lemma:equivalence_between_perspective_relaxation_and_convexification}]
    The closed convex hull of the set
    \begin{align*}
        \textstyle \left\{ (\tau, \bbeta, \bz) \middle|
        \| \bbeta \|_\infty \leq M, \, \bz \in \{0, 1\}^p, \, \mathbf{1}^T \bz \leq k, \, \beta_j ( 1 - z_j) = 0 ~~ \forall j \in [p], \, \sum_{j \in [p]} \beta_j^2 \leq \tau \right\}
    \end{align*}
    is given by the set
    \begin{align*}
        \textstyle \left\{ (\tau, \bbeta, \bz)  \;\middle|\; -M z_j\leq \bbeta_j \leq M z_j ~ \forall j \in [p], \, \bz \in [0, 1]^p, \, \mathbf{1}^T \bz \leq k, \, \sum_{j \in [p]} \beta_j^2 / z_j \leq \tau \right\}.
    \end{align*}
\end{namedlemma}

\begin{proof}
    Let $\mathcal T$ represent the first set mentioned in the statement of the lemma. Using the definition of the perspective function and applying the big-M formulation technique, $\mathcal T$ can be expressed as
    \begin{align*}
        \textstyle \mathcal T = \left\{ (\tau, \bbeta, \bz)  \;\middle|\; -M z_j\leq \bbeta_j \leq M z_j ~ \forall j \in [p], \, \bz \in \{0, 1\}^p, \, \mathbf{1}^T \bz \leq k, \, \sum_{j \in [p]} \beta_j^2 / z_j \leq \tau \right\}.
    \end{align*}
    As the epigraph of a perspective function constitutes a cone \citep[Lemma~1 \& 2]{shafiee2024constrained}, we may write $\mathcal T = \mathrm{Proj}_{(\tau, \bbeta, \bz)}(\overline {\mathcal T})$, where the extended set $\overline {\mathcal T}$ admits the mixed-binary conic representation
    \begin{align*}
        \textstyle \overline {\mathcal T} = \left\{ (\tau, \bbeta, \bt, \bz) \;\middle|\; \bm 1^\top \bt = \tau, \, \bz \in \{0, 1\}^p, \, \mathbf{1}^T \bz \leq k, \, \bm A_j \begin{bmatrix} t_j \\ \beta_j \end{bmatrix} + \bm B_j z_j \in \mathbb K_j ~ \forall j \in [p] \right\}.
    \end{align*}
    Here, for any $j \in [p]$, the matrices $\bm A_j$ and $\bm B_j$, and the cone $\mathbb K_j$ are defined as
    \begin{align*}
        \bm A = \begin{bmatrix} 1 & 0 \\ 0 & 1 \\ 0 & 0 \\ 0 & 1 \\ 0 & -1 \end{bmatrix}, \,
        \bm B = \begin{bmatrix} 0 \\ 0 \\ 0 \\ M \\ M \end{bmatrix}, \,
        \mathbb K_j = \mathbb L \times \R_+ \times \R_+,
    \end{align*}
    where $\mathbb L \in \R^3$ denotes the rotated second order cone, that is, $\mathbb L = \{ (t, \beta, z) \in \R_+ \times \R \times \R_+: \beta^2 \leq t z  \}$.
    Thus, using \citep[Lemma~4]{shafiee2024constrained}, the set $\overline{\mathcal T}$ satisfies all the requirements of \citep[Theorem~1]{shafiee2024constrained}, and therefore, its continuous gives the closed convex hull of $\overline{\mathcal T}$, that is,
    \begin{align*}
        \textstyle \cl \conv(\overline {\mathcal T}) = \left\{ (\tau, \bbeta, \bt, \bz) \;\middle|\; \bm 1^\top \bt = \tau, \, \bz \in [0, 1]^p, \, \mathbf{1}^T \bz \leq k, \, \bm A_j \begin{bmatrix} t_j \\ \beta_j \end{bmatrix} + \bm B_j z_j \in \mathbb K_j ~ \forall j \in [p] \right\}.
    \end{align*}
    The prove concludes by applying Fourier-Motzkin elimination method to project out the variable $\bt$.
\end{proof} 

\begin{namedlemma}
    [~\ref{lemma:fenchel_conjugate_of_g_closed_form_expression}]
    The conjugate of $g$ is given by
    \begin{equation*}
        g^*(\balpha) = \TopSum_k({\bf H}_M(\balpha)).
    \end{equation*}
\end{namedlemma}

\begin{proof}
    By definition, the Fenchel conjugate is defined as
    \begin{equation}
        \label{appendix_def:fenchel_conjugate_of_g}
        g^*(\balpha) = \max_{\bbeta} \balpha^T \bbeta - g(\bbeta)
    \end{equation}

    By plugging $g(\bbeta)$ from Equation~\eqref{eq:function_g_definition} and $\calZ(\bbeta)$ from Equation~\eqref{eq:z_domain_definition} into this Fenchel conjugate definition, we have
    \begin{align}
        \label{appendix_eq:fenchel_conjugate_of_g_max_beta_max_z_formulation}
            g^*(\balpha) &= \max_{\bbeta} \balpha^T \bbeta - \min_{\bz \in \calZ(\bbeta)} \frac{1}{2} \sum_{j=1}^n \frac{\beta_j^2}{z_j} \nonumber \\
            &= \max_{\bbeta} \max_{\bz \in \calZ(\bbeta)} \balpha^T \bbeta -  \frac{1}{2} \sum_{j=1}^n \frac{\beta_j^2}{z_j}
    \end{align}

    Since Equation~\eqref{appendix_eq:fenchel_conjugate_of_g_max_beta_max_z_formulation} is jointly maximizing over $\bbeta$ and $\bz$, we can switch the order by doing the inner optimization over $\bbeta$ and outer optimization over $\bz$:
    \begin{align}
        \label{appendix_eq:fenchel_conjugate_of_g_max_z_max_beta_formulation}
        g^*(\balpha) = \max_{\bz \in \calZ} \, \max_{\bbeta \in \calB(\bz)} \balpha^T \bbeta - \frac{1}{2} \sum_{j=1}^n \frac{\beta_j^2}{z_j},
    \end{align}
    where we have set $\calZ := \{\bz \mid \bz \in [0,1]^n, \, \mathbf{1}^T \bz \leq k\}$, and set $\calB(\bz) := \{\bbeta \mid - M z_j \leq \beta_j \leq M z_j\}$.

    For the inner optimization problem in Equation~\eqref{appendix_eq:fenchel_conjugate_of_g_max_z_max_beta_formulation}, if we take the derivative with respect to $\bbeta$ and set it to zero, we have $\alpha_j - \beta_j / z_j = 0$, or $\beta_j = \alpha_j z_j$.
    Since $\bbeta$ also needs to satisfy the constraint $ - M z_j \leq \beta_j \leq M z_j$ for all $j$'s, the optimal solution $\beta_j^*$ for the inner optimization problem is 
    \begin{equation}
        \label{appendix_eq:solution_for_beta_in_terms_of_fenchel_dual_alpha}
        \beta_j^* = \text{sgn}(\alpha_j) \min(\vert{\alpha_j}, M) z_j.
    \end{equation}

    If we plug Equation~\eqref{appendix_eq:solution_for_beta_in_terms_of_fenchel_dual_alpha} into each individual term $\alpha_j \beta_j^* - \frac{{\beta_j^*}^2}{2z_j}$, we have

    \begin{align}
        \label{appendix_eq:fenchel_conjugate_of_g_each_term_reduced_to_Huber_loss}
        \alpha_j \beta_j^* - \frac{{\beta_j^*}^2}{2z_j} &= \alpha_j \text{sgn}(\alpha_j) \min(\vert{\alpha_j}, M) z_j - \frac{\left( \text{sgn}\left( \alpha_j \right) \min\left(\vert{\alpha_j}, M \right) z_j \right)^2}{2z_j} \nonumber \\
        &= \vert{\alpha_j} \min(\vert{\alpha_j}, M) z_j - \frac{1}{2} \min(\vert{\alpha_j}, M)^2 z_j \nonumber \\
        &= \left[ \vert{\alpha_j} \min(\vert{\alpha_j}, M) - \frac{1}{2} \min(\vert{\alpha_j}, M)^2 \right] z_j \nonumber \\
        &= \begin{cases}
            \frac{1}{2} \alpha_j^2 z_j & \text{if } \vert{\alpha_j} \leq M \nonumber \\
            \left( M \vert{\alpha_j} - \frac{1}{2} M^2 \right) z_j & \text{if } \vert{\alpha_j} > M
        \end{cases} \nonumber \\
        &= H_M(\alpha_j) z_j.
    \end{align}

    By plugging Equation~\eqref{appendix_eq:solution_for_beta_in_terms_of_fenchel_dual_alpha} and Equation~\eqref{appendix_eq:fenchel_conjugate_of_g_each_term_reduced_to_Huber_loss} back into Equation~\eqref{appendix_eq:fenchel_conjugate_of_g_max_z_max_beta_formulation}, we obtain
    \begin{align*}
        % \label{appendix_eq:fenchel_conjugate_of_g_top_k_sum_reformulation}
        g^*(\balpha) &= \max_{\bz \in \calZ} \left( \sum_{j=1}^n \alpha_j \beta_j^* - \frac{ {\beta_j^*}^2}{2z_j} \right)\\
        &= \max_{\bz \in \calZ} \left( \sum_{j=1}^n H_M(\alpha_j) z_j \right) \nonumber\\
        &= \text{TopSum}_k (\{H_M(\alpha_j)\}_{j=1}^n), \nonumber \\
        &= \text{TopSum}_k (H_M(\balpha)).
    \end{align*}
\end{proof}

\begin{namedlemma}
    [~\ref{lemma:equivalence_between_proximal_operator_and_huber_isotonic_regression}]
    For any $\bmu \in \R^p$, we have 
    $$\prox_{\rho g^*}(\bmu) = \sgn(\bmu) \odot \bnu^\star, $$ 
    where $\odot$ denotes the Hadamard (element-wise) product, $\bnu^\star$ is the unique solution of the following optimization problem
    \begin{align}
        \label{obj:KyFan_Huber_isotonic_regression}
        \begin{array}{cl}
            \min\limits_{\bnu \in \R^p} & \frac{1}{2} \sum_{j \in [p]} (\nu_j - \vert{\mu_j})^2 + \rho \sum_{j \in \calJ} H_M (\nu_j) \\[2ex]
            \st & \quad \nu_j \geq \nu_l \; \text{ if } \; \vert{\mu_j} \geq \vert{\mu_l} ~~ \forall j, l \in [p],
        \end{array} 
    \end{align}
    and $\calJ$ is the set of indices of the top $k$ largest elements of~$ \vert{\mu_j}, j \in [p]$. 
\end{namedlemma}

\begin{proof}
% Add proof content here
    First, let us recall that Problem~\eqref{obj:proximal_operator_of_g*_with_TopSum_k_and_Huber} is the following optimization problem:
    \begin{align}
        \label{appendix_obj:proximal_operator_of_g*_with_TopSum_k_and_Huber}
        \balpha^* = \argmin_{\balpha} \frac{1}{2} \Vert{\balpha - \bmu}_2^2 + \rho \text{TopSum}_k\left( H_M\left( \balpha \right) \right),
    \end{align}
    
    We want to show that Problem~\eqref{appendix_obj:proximal_operator_of_g*_with_TopSum_k_and_Huber} is closely related to Problem~\eqref{appendix_obj:KyFan_Huber_isotonic_regression} via the relation $\balpha^* = \text{sgn}(\bmu) \odot \bnu^*$.
    
    
    To accomplish this, we leverage two properties associated with the optimal solution for Problem~\eqref{appendix_obj:proximal_operator_of_g*_with_TopSum_k_and_Huber}.
    At $\balpha^*$, we have:
    \begin{align}
        & \textbf{1. sign-preserving property:} \quad \text{sgn}(\alpha_j^*) = \text{sgn}(\mu_j) \label{appendix_property:sign_preserving}\\
        & \textbf{2. relative magnitude-preserving property:} \quad \vert{\alpha_j^*} \geq \vert{\alpha_l^*} \; \text{if} \; \vert{\mu_j} \geq \vert{\mu_l} \label{appendix_property:relative_magnitude_preserving}
        % \text{sgn}(\alpha_j^*) &= \text{sgn}(\mu_j) \label{appendix_property:sign_preserving}\\
        % \vert{\alpha_j^*} \geq \vert{\alpha_l^*} \; &\text{if} \; \vert{\mu_j} \geq \vert{\mu_l} \label{appendix_property:relative_magnitude_preserving}
    \end{align}
    % Equation~\eqref{appendix_property:sign_preserving} says that the optimal solution $\balpha^*$ should be sign-preserving with respect to the input $\bmu$, and Equation~\eqref{appendix_property:relative_magnitude_preserving} tells us that the optimal solution $\balpha^*$ should preserve the relative order with respect to the input $\bmu$ in terms of the magnitude.
    
    Let us explain why these two properties hold.
    
    \paragraph{Sign-preserving property in Equation~\eqref{appendix_property:sign_preserving}} 
    For the sake of contradiction, suppose that there exists some $j$ such that $\text{sgn}(\alpha_j^*) \neq \text{sgn}(\mu_j)$.
    Then, we can construct a new $\balpha'$ by flipping the sign of $\alpha_j^*$, i.e., $\alpha_j' = -\alpha_j^*$, and keeping the rest of the elements the same as $\alpha_j'$.
    Now under the assmption that $\text{sgn}(\alpha_j^*) \neq \text{sgn}(\mu_j)$, we have $\left\lvert{\alpha_j^* - \mu_j}\right\rvert > \left\lvert{\lvert{\alpha_j^*}\rvert - \lvert{\mu_j}\rvert}\right\rvert = \left\lvert{\alpha_j' - \mu_j}\right\rvert$, so the $j$-th term in the first summation of the objective function will decrease while everything else remains the same.
    This leads to a smaller objective value for $\balpha'$ than $\balpha^*$, which contradicts the optimality of $\balpha^*$.
    Thus, the sign-preserving property in Equation~\eqref{appendix_property:sign_preserving} must hold.
    
    \paragraph{Relative magnitude-preserving property in Equation~\eqref{appendix_property:relative_magnitude_preserving}} 
    For the sake of contradiction, suppose that there exists some $j$ and $l$ such that $\vert{\mu_j} \geq \vert{\mu_l}$ but $\vert{\alpha_j^*} < \vert{\alpha_l^*}$.
    Then, we can construct a new $\balpha'$ by swapping $\alpha_j^*$ and $\alpha_l^*$, i.e., $\alpha_j' = \alpha_l^*$ and $\alpha_l' = \alpha_j^*$, and keeping the rest of the elements the same as $\alpha_j'$ and $\alpha_l'$.
    Under the assumption that $\vert{\mu_j} \geq \vert{\mu_l}$ but $\vert{\alpha_j^*} < \vert{\alpha_l^*}$, we have $\left\lvert{\alpha_j^* - \mu_j}\right\rvert + \left\lvert{\alpha_l^* - \mu_l}\right\rvert > \left\lvert{\alpha_l^* - \mu_j}\right\rvert + \left\lvert{\alpha_j^* - \mu_l}\right\rvert =
    \left\lvert{\alpha_j' - \mu_j}\right\rvert + \left\lvert{\alpha_l' - \mu_l}\right\rvert$, so the sum of the $j$-th and $l$-th terms in the first summation of the objective function will decrease while everything else remains the same.
    This leads to a smaller objective value for $\balpha'$ than $\balpha^*$, which contradicts the optimality of $\balpha^*$.
    Thus, the relative magnitude-preserving property in Equation~\eqref{appendix_property:relative_magnitude_preserving} holds.
    
    Using these two properties, we are ready to prove the equivalence between Problem~\eqref{appendix_obj:proximal_operator_of_g*_with_TopSum_k_and_Huber} and Problem~\eqref{appendix_obj:KyFan_Huber_isotonic_regression}.
    First, let us reparameterize $\balpha$ with a new variable $\bnu$ in Problem~\eqref{appendix_obj:proximal_operator_of_g*_with_TopSum_k_and_Huber} as $\balpha = \text{sgn}(\bmu) \odot \bnu$ wtih $\bnu \in \mathbb{R}_{+}^n$.
    In other words, we use $\bnu$ to model the magnitude of $\balpha$.
    
    By the sign-preserving property in Equation~\eqref{appendix_property:sign_preserving}, we can set the equivalence between Problem~\eqref{appendix_obj:proximal_operator_of_g*_with_TopSum_k_and_Huber} and the following optimization problem:
    \begin{align}
        \label{appendix_obj:proximal_operator_of_g*_with_TopSum_k_and_Huber_reparameterized}
        \bnu^* = \argmin_{\bnu} \frac{1}{2} \sum_{j=1}^n (\nu_j - \vert{\mu_j})^2 + \rho \text{TopSum}_k\left( H_M\left( \bnu \right) \right), \; \text{ s.t. } \; \nu_j \geq 0.
    \end{align}
    
    By the relative magnitude-preserving property in Equation~\eqref{appendix_property:relative_magnitude_preserving}, we can further set the equivalence between Problem~\eqref{appendix_obj:proximal_operator_of_g*_with_TopSum_k_and_Huber_reparameterized} and the following optimization problem:
    \begin{align}
        \label{appendix_obj:KyFan_Huber_isotonic_regression_with_nonnegative_constraint}
        \bnu^* = \argmin_{\bnu} & \quad \frac{1}{2} \sum_{j=1}^n (\nu_j - \vert{\mu_j})^2 + \rho \sum_{j \in \calJ}^k H_M (\nu_j), \\
        \text{s.t.} & \quad \nu_j \geq \nu_l \; \text{ if } \; \vert{\mu_j} \geq \vert{\mu_l}, \; \text{ and } \; \nu_j \geq 0. \nonumber
    \end{align}
    
    Lastly, the nonnegative constraint in Problem~\eqref{appendix_obj:KyFan_Huber_isotonic_regression_with_nonnegative_constraint} can be removed because the first summation term in the objective function already implies that $\nu_j \geq 0$.
    Thus, we have shown that Problem~\eqref{appendix_obj:proximal_operator_of_g*_with_TopSum_k_and_Huber} is closely related to Problem~\eqref{appendix_obj:KyFan_Huber_isotonic_regression} via the relation $\balpha^* = \text{sgn}(\bmu) \odot \bnu^*$.
\end{proof}

\begin{namedlemma}
    [~\ref{lemma:PAVA_algorithm_exact_solution}]
    The vector $\hat \bnu$ in Algorithm~\ref{alg:PAVA_algorithm} solves~\eqref{obj:KyFan_Huber_isotonic_regression} exactly.
\end{namedlemma}

\begin{proof}
    Problem~\eqref{obj:KyFan_Huber_isotonic_regression} can be recognized as a generalized isotonic regression problem.
    Let $h_j(v) = \frac{1}{2} (v - \mu_j)^2 + \rho_j H_M(v)$, where $\rho_j = \rho$ if $j \in \calJ$ and $\rho_j = 0$ otherwise.
    As mentioned in Algorithm~\ref{alg:PAVA_algorithm}, the set $\calJ$ is the set of indices of top k largest elements of $\vert{\mu_j}$.
    Then, we can rewrite Problem~\eqref{obj:KyFan_Huber_isotonic_regression} in the standard form of a generalized isotonic regression problem:
    \begin{align}
        \label{obj:KyFan_Huber_isotonic_regression_rewritten_as_generalized_isotonic_regression}
        \min_{\bnu} \sum_{j=1}^{p} h_j(\nu_j) \quad \text{s.t.} \quad \nu_1 \geq \nu_2 \geq \cdots \geq \nu_J.
    \end{align}
    
    Generalized isotonic regression problems have been studied extensively in the literature.
    There are two key properties~\cite{best2000minimizing,ahuja2001fast} regarding Problem~\eqref{obj:KyFan_Huber_isotonic_regression_rewritten_as_generalized_isotonic_regression} that we can leverage to prove Lemma~\ref{lemma:PAVA_algorithm_exact_solution}:
    
    \paragraph{1. Optimal solution for a merged block is single-valued.}
    Suppose we have two adjacent blocks $[a_1, a_2]$ and $[a_2+1, a_3]$.
    If the optimal solution for each block is single-valued, \textit{i.e.}, we have
    \begin{equation}
    \begin{array}{c@{\hspace{2em}}c@{\hspace{2em}}c}
    \bnu_{a_1:a_2}^* =  
    \left(\begin{array}{c}
    \argmin_{\bnu_{a_1:a_2}} \sum_{j=a_1}^{a_2} h_j(\nu_j) \\
    \text{s.t.} \quad \nu_{a_1} \geq \cdots \geq \nu_{a_2}
    \end{array}\right) & 
    \Rightarrow & \bnu_{a_1}^* = \cdots = \bnu_{a_2}^* \\[3em]
    \bnu_{a_2+1:a_3}^* =  
    \left(\begin{array}{c}
    \argmin_{\bnu_{a_2+1:a_3}} \sum_{j=a_2+1}^{a_3} h_j(\nu_j) \\
    \text{s.t.} \quad \nu_{a_2+1} \geq \cdots \geq \nu_{a_3}
    \end{array}\right) & 
    \Rightarrow & \bnu_{a_2+1}^* = \cdots = \bnu_{a_3}^*
    \end{array}
    \end{equation}
    
    and if $\nu_{a_1}^* \leq \nu_{a_2+1}^*$, then the optimal solution for the merged block $[a_1, a_3]$ is single-valued, \textit{i.e.}, we have:
    
    \begin{equation}
    \begin{array}{c@{\hspace{2em}}c@{\hspace{2em}}c}
    \bnu_{a_1:a_3}^* =  
    \left(\begin{array}{c}
    \argmin_{\bnu_{a_1:a_3}} \sum_{j=a_1}^{a_3} h_j(\nu_j) \\
    \text{s.t.} \quad \nu_{a_1} \geq \cdots \geq \nu_{a_3}
    \end{array}\right) & 
    \Rightarrow & \bnu_{a_1}^* = \cdots = \bnu_{a_3}^*
    \end{array}
    \end{equation}
    
    \paragraph{2. No isotonic constraint violation between single-valued blocks implies the solution is optimal.}
    If we have $s$ blocks $[a_1, a_2], [a_2+1, a_3], \ldots, [a_{s}+1, a_{s+1}]$ (with $a_1=1$ and $a_{s+1}=n$) such that the optimal solution for each block is single-valued, \textit{i.e.}, we have
    
    \begin{equation}
    \begin{array}{c@{\hspace{2em}}c@{\hspace{2em}}c}
    \hat{\bnu}_{a_l:a_{l+1}} =
    \left(\begin{array}{c}
    \argmin_{\bnu_{a_l:a_{l+1}}} \sum_{j=a_l}^{a_{l+1}} h_j(\nu_j) \\
    \text{s.t.} \quad \nu_{a_l} \geq \cdots \geq \nu_{a_{l+1}}
    \end{array}\right) &
    \Rightarrow & \bnu_{a_l}^* = \cdots = \bnu_{a_{l+1}}^* \, \text{ for all } \, l=1, \ldots, s
    \end{array}
    \end{equation}
    
    and if $\hat{\nu}_{a_1} \geq \hat{\nu}_{a_2+1} \geq \ldots \hat{\nu}_{a_{s}}$, then $\hat{\bnu}$ is the optimal solution for Problem~\eqref{obj:KyFan_Huber_isotonic_regression_rewritten_as_generalized_isotonic_regression}.
    
    Using these two properties, it is now easy to see why Algorithm~\ref{alg:PAVA_algorithm} returns the optimal solution.
    
    We start by constructing blocks which have length 1.
    The initial value restricted to each block is optimal.
    Then, we iteratively merge adjacent blocks and update the values of $\nu_j$'s whenever there is a violation of the isotonic constraint.
    By the first property, the optimal solution for the merged block is single-valued.
    Therefore, we can compute the optimal solution for the merged block by solving a univariate optimization problem.
    
    We keep merging blocks until there is no isotonic constraint violation.
    When this happens, by construction, the solution for each block is single-valued and optimal.
    By the second property, the final vector $\hat{\bnu}$ is the optimal solution for Problem~\eqref{obj:KyFan_Huber_isotonic_regression_rewritten_as_generalized_isotonic_regression}.
\end{proof}
%\subsection{Proof of Lemma~\ref{appendix_lemma:fenchel_conjugate_of_g_closed_form_expression}}
\label{appendix_subsec:proof_fenchel_conjugate_of_g_closed_form_expression}

% \begin{lemma}[Lemma~\ref{appendix_lemma:fenchel_conjugate_of_g_closed_form_expression}]
\begin{namedlemma}[~\ref{appendix_lemma:fenchel_conjugate_of_g_closed_form_expression}]
    \label{appendix_lemma:fenchel_conjugate_of_g_closed_form_expression}
    The Fenchel conjugate of $g(\bbeta)$, denoted as $g^*(\balpha)$, is given by:
    \begin{equation}
        g^*(\balpha) = \text{TopSum}_k(H_M(\balpha)),
    \end{equation}
    where $\text{TopSum}_k(\cdot)$ denotes the sum of the top $k$ largest elements, and $H_M(\cdot)$ is the Huber loss function defined as:
    \begin{equation}
        H_M(\alpha_j) := \begin{cases}
            \frac{1}{2} \alpha_j^2 & \text{if } \vert{\alpha_j} \leq M \\
            M \vert{\alpha_j} - \frac{1}{2} M^2 & \text{if } \vert{\alpha_j} > M
        \end{cases},
    \end{equation}
    and we use the shorthand notation $H_M(\balpha)$ to denote applying $H_M(\cdot)$ to $\balpha$ in an elementwise fashion.
\end{namedlemma}

\begin{proof}

    By definition, the Fenchel conjugate is defined as
    \begin{equation}
        \label{appendix_def:fenchel_conjugate_of_g}
        g^*(\balpha) = \max_{\bbeta} \balpha^T \bbeta - g(\bbeta)
    \end{equation}

    By plugging $g(\bbeta)$ from Equation~\eqref{eq:function_g_definition} and $\calZ(\bbeta)$ from Equation~\eqref{eq:z_domain_definition} into this Fenchel conjugate definition, we have
    \begin{align}
        \label{appendix_eq:fenchel_conjugate_of_g_max_beta_max_z_formulation}
            g^*(\balpha) &= \max_{\bbeta} \balpha^T \bbeta - \min_{\bz \in \calZ(\bbeta)} \frac{1}{2} \sum_{j=1}^n \frac{\beta_j^2}{z_j} \nonumber \\
            &= \max_{\bbeta} \max_{\bz \in \calZ(\bbeta)} \balpha^T \bbeta -  \frac{1}{2} \sum_{j=1}^n \frac{\beta_j^2}{z_j}
    \end{align}

    Since Equation~\eqref{appendix_eq:fenchel_conjugate_of_g_max_beta_max_z_formulation} is jointly maximizing over $\bbeta$ and $\bz$, we can switch the order by doing the inner optimization over $\bbeta$ and outer optimization over $\bz$:
    \begin{align}
        \label{appendix_eq:fenchel_conjugate_of_g_max_z_max_beta_formulation}
        g^*(\balpha) = \max_{\bz \in \calZ} \, \max_{\bbeta \in \calB(\bz)} \balpha^T \bbeta - \frac{1}{2} \sum_{j=1}^n \frac{\beta_j^2}{z_j},
    \end{align}
    where we have set $\calZ := \{\bz \mid \bz \in [0,1]^n, \, \mathbf{1}^T \bz \leq k\}$, and set $\calB(\bz) := \{\bbeta \mid - M z_j \leq \beta_j \leq M z_j\}$.

    For the inner optimization problem in Equation~\eqref{appendix_eq:fenchel_conjugate_of_g_max_z_max_beta_formulation}, if we take the derivative with respect to $\bbeta$ and set it to zero, we have $\alpha_j - \beta_j / z_j = 0$, or $\beta_j = \alpha_j z_j$.
    Since $\bbeta$ also needs to satisfy the constraint $ - M z_j \leq \beta_j \leq M z_j$ for all $j$'s, the optimal solution $\beta_j^*$ for the inner optimization problem is 
    \begin{equation}
        \label{appendix_eq:solution_for_beta_in_terms_of_fenchel_dual_alpha}
        \beta_j^* = \text{sgn}(\alpha_j) \min(\vert{\alpha_j}, M) z_j.
    \end{equation}

    If we plug Equation~\eqref{appendix_eq:solution_for_beta_in_terms_of_fenchel_dual_alpha} into each individual term $\alpha_j \beta_j^* - \frac{{\beta_j^*}^2}{2z_j}$, we have

    \begin{align}
        \label{appendix_eq:fenchel_conjugate_of_g_each_term_reduced_to_Huber_loss}
        \alpha_j \beta_j^* - \frac{{\beta_j^*}^2}{2z_j} &= \alpha_j \text{sgn}(\alpha_j) \min(\vert{\alpha_j}, M) z_j - \frac{\left( \text{sgn}\left( \alpha_j \right) \min\left(\vert{\alpha_j}, M \right) z_j \right)^2}{2z_j} \nonumber \\
        &= \vert{\alpha_j} \min(\vert{\alpha_j}, M) z_j - \frac{1}{2} \min(\vert{\alpha_j}, M)^2 z_j \nonumber \\
        &= \left[ \vert{\alpha_j} \min(\vert{\alpha_j}, M) - \frac{1}{2} \min(\vert{\alpha_j}, M)^2 \right] z_j \nonumber \\
        &= \begin{cases}
            \frac{1}{2} \alpha_j^2 z_j & \text{if } \vert{\alpha_j} \leq M \nonumber \\
            \left( M \vert{\alpha_j} - \frac{1}{2} M^2 \right) z_j & \text{if } \vert{\alpha_j} > M
        \end{cases} \nonumber \\
        &= H_M(\alpha_j) z_j.
    \end{align}

    By plugging Equation~\eqref{appendix_eq:solution_for_beta_in_terms_of_fenchel_dual_alpha} and Equation~\eqref{appendix_eq:fenchel_conjugate_of_g_each_term_reduced_to_Huber_loss} back into Equation~\eqref{appendix_eq:fenchel_conjugate_of_g_max_z_max_beta_formulation}, we obtain
    \begin{align*}
        % \label{appendix_eq:fenchel_conjugate_of_g_top_k_sum_reformulation}
        g^*(\balpha) &= \min_{\bz \in \calZ} \left( \sum_{j=1}^n \alpha_j \beta_j^* - \frac{ {\beta_j^*}^2}{2z_j} \right)\\
        &= \max_{\bz \in \calZ} \left( \sum_{j=1}^n H_M(\alpha_j) z_j \right) \nonumber\\
        &= \text{TopSum}_k (\{H_M(\alpha_j)\}_{j=1}^n), \nonumber \\
        &= \text{TopSum}_k (H_M(\balpha)).
    \end{align*}

\end{proof}


%\subsection{Proof of Lemma~\ref{lemma:equivalence_between_proximal_operator_and_huber_isotonic_regression}}
\label{appendix_subsec:proof_equivalence_between_proximal_operator_and_huber_isotonic_regression}

% \begin{lemma}[Lemma~\ref{appendix_lemma:equivalence_between_proximal_operator_and_huber_isotonic_regression}]
\begin{namedlemma}[~\ref{lemma:equivalence_between_proximal_operator_and_huber_isotonic_regression}]
    \label{appendix_lemma:equivalence_between_proximal_operator_and_huber_isotonic_regression}
    Problem~\eqref{obj:proximal_operator_of_g*_with_TopSum_k_and_Huber} is closely related to the following optimization problem:
    \begin{align}
        \label{appendix_obj:KyFan_Huber_isotonic_regression}
        \bnu^* = \argmin_{\bnu} & \quad \frac{1}{2} \sum_{j=1}^n (\nu_j - \vert{\mu_j})^2 + \rho \sum_{j \in \calJ}^k H_M (\nu_j), \\
        \text{s.t.} & \quad \nu_j \geq \nu_l \; \text{ if } \; \vert{\mu_j} \geq \vert{\mu_l}, \nonumber
    \end{align}
    where $\calJ$ is the set of indices of the top $k$ largest elements of $ \vert{\mu_j}$.
    Specifically, the optimal solution $\balpha^*$ for Problem~\eqref{obj:proximal_operator_of_g*_with_TopSum_k_and_Huber} can be recovered from the optimal solution $\bnu^*$ for Problem~\eqref{appendix_obj:KyFan_Huber_isotonic_regression} by the relation $\balpha^* = \text{sgn}(\bmu) \odot \bnu^*$, where $\odot$ denotes the Hadamard (element-wise) product.
\end{namedlemma}

\begin{proof}
% Add proof content here
First, let us recall that Problem~\eqref{obj:proximal_operator_of_g*_with_TopSum_k_and_Huber} is the following optimization problem:
\begin{align}
    \label{appendix_obj:proximal_operator_of_g*_with_TopSum_k_and_Huber}
    \balpha^* = \argmin_{\balpha} \frac{1}{2} \Vert{\balpha - \bmu}_2^2 + \rho \text{TopSum}_k\left( H_M\left( \balpha \right) \right),
\end{align}

We want to show that Problem~\eqref{appendix_obj:proximal_operator_of_g*_with_TopSum_k_and_Huber} is closely related to Problem~\eqref{appendix_obj:KyFan_Huber_isotonic_regression} via the relation $\balpha^* = \text{sgn}(\bmu) \odot \bnu^*$.


To accomplish this, we leverage two properties associated with the optimal solution for Problem~\eqref{appendix_obj:proximal_operator_of_g*_with_TopSum_k_and_Huber}.
At $\balpha^*$, we have:
\begin{align}
    & \textbf{1. sign-preserving property:} \quad \text{sgn}(\alpha_j^*) = \text{sgn}(\mu_j) \label{appendix_property:sign_preserving}\\
    & \textbf{2. relative magnitude-preserving property:} \quad \vert{\alpha_j^*} \geq \vert{\alpha_l^*} \; \text{if} \; \vert{\mu_j} \geq \vert{\mu_l} \label{appendix_property:relative_magnitude_preserving}
    % \text{sgn}(\alpha_j^*) &= \text{sgn}(\mu_j) \label{appendix_property:sign_preserving}\\
    % \vert{\alpha_j^*} \geq \vert{\alpha_l^*} \; &\text{if} \; \vert{\mu_j} \geq \vert{\mu_l} \label{appendix_property:relative_magnitude_preserving}
\end{align}
% Equation~\eqref{appendix_property:sign_preserving} says that the optimal solution $\balpha^*$ should be sign-preserving with respect to the input $\bmu$, and Equation~\eqref{appendix_property:relative_magnitude_preserving} tells us that the optimal solution $\balpha^*$ should preserve the relative order with respect to the input $\bmu$ in terms of the magnitude.

Let us explain why these two properties hold.

\paragraph{Sign-preserving property in Equation~\eqref{appendix_property:sign_preserving}} 
For the sake of contradiction, suppose that there exists some $j$ such that $\text{sgn}(\alpha_j^*) \neq \text{sgn}(\mu_j)$.
Then, we can construct a new $\balpha'$ by flipping the sign of $\alpha_j^*$, i.e., $\alpha_j' = -\alpha_j^*$, and keeping the rest of the elements the same as $\alpha_j'$.
Now under the assmption that $\text{sgn}(\alpha_j^*) \neq \text{sgn}(\mu_j)$, we have $\left\lvert{\alpha_j^* - \mu_j}\right\rvert > \left\lvert{\lvert{\alpha_j^*}\rvert - \lvert{\mu_j}\rvert}\right\rvert = \left\lvert{\alpha_j' - \mu_j}\right\rvert$, so the $j$-th term in the first summation of the objective function will decrease while everything else remains the same.
This leads to a smaller objective value for $\balpha'$ than $\balpha^*$, which contradicts the optimality of $\balpha^*$.
Thus, the sign-preserving property in Equation~\eqref{appendix_property:sign_preserving} must hold.

\paragraph{Relative magnitude-preserving property in Equation~\eqref{appendix_property:relative_magnitude_preserving}} 
For the sake of contradiction, suppose that there exists some $j$ and $l$ such that $\vert{\mu_j} \geq \vert{\mu_l}$ but $\vert{\alpha_j^*} < \vert{\alpha_l^*}$.
Then, we can construct a new $\balpha'$ by swapping $\alpha_j^*$ and $\alpha_l^*$, i.e., $\alpha_j' = \alpha_l^*$ and $\alpha_l' = \alpha_j^*$, and keeping the rest of the elements the same as $\alpha_j'$ and $\alpha_l'$.
Under the assumption that $\vert{\mu_j} \geq \vert{\mu_l}$ but $\vert{\alpha_j^*} < \vert{\alpha_l^*}$, we have $\left\lvert{\alpha_j^* - \mu_j}\right\rvert + \left\lvert{\alpha_l^* - \mu_l}\right\rvert > \left\lvert{\alpha_l^* - \mu_j}\right\rvert + \left\lvert{\alpha_j^* - \mu_l}\right\rvert =
\left\lvert{\alpha_j' - \mu_j}\right\rvert + \left\lvert{\alpha_l' - \mu_l}\right\rvert$, so the sum of the $j$-th and $l$-th terms in the first summation of the objective function will decrease while everything else remains the same.
This leads to a smaller objective value for $\balpha'$ than $\balpha^*$, which contradicts the optimality of $\balpha^*$.
Thus, the relative magnitude-preserving property in Equation~\eqref{appendix_property:relative_magnitude_preserving} holds.

Using these two properties, we are ready to prove the equivalence between Problem~\eqref{appendix_obj:proximal_operator_of_g*_with_TopSum_k_and_Huber} and Problem~\eqref{appendix_obj:KyFan_Huber_isotonic_regression}.
First, let us reparameterize $\balpha$ with a new variable $\bnu$ in Problem~\eqref{appendix_obj:proximal_operator_of_g*_with_TopSum_k_and_Huber} as $\balpha = \text{sgn}(\bmu) \odot \bnu$ wtih $\bnu \in \mathbb{R}_{+}^n$.
In other words, we use $\bnu$ to model the magnitude of $\balpha$.

By the sign-preserving property in Equation~\eqref{appendix_property:sign_preserving}, we can set the equivalence between Problem~\eqref{appendix_obj:proximal_operator_of_g*_with_TopSum_k_and_Huber} and the following optimization problem:
\begin{align}
    \label{appendix_obj:proximal_operator_of_g*_with_TopSum_k_and_Huber_reparameterized}
    \bnu^* = \argmin_{\bnu} \frac{1}{2} \sum_{j=1}^n (\nu_j - \vert{\mu_j})^2 + \rho \text{TopSum}_k\left( H_M\left( \bnu \right) \right), \; \text{ s.t. } \; \nu_j \geq 0.
\end{align}

By the relative magnitude-preserving property in Equation~\eqref{appendix_property:relative_magnitude_preserving}, we can further set the equivalence between Problem~\eqref{appendix_obj:proximal_operator_of_g*_with_TopSum_k_and_Huber_reparameterized} and the following optimization problem:
\begin{align}
    \label{appendix_obj:KyFan_Huber_isotonic_regression_with_nonnegative_constraint}
    \bnu^* = \argmin_{\bnu} & \quad \frac{1}{2} \sum_{j=1}^n (\nu_j - \vert{\mu_j})^2 + \rho \sum_{j \in \calJ}^k H_M (\nu_j), \\
    \text{s.t.} & \quad \nu_j \geq \nu_l \; \text{ if } \; \vert{\mu_j} \geq \vert{\mu_l}, \; \text{ and } \; \nu_j \geq 0. \nonumber
\end{align}

Lastly, the nonnegative constraint in Problem~\eqref{appendix_obj:KyFan_Huber_isotonic_regression_with_nonnegative_constraint} can be removed because the first summation term in the objective function already implies that $\nu_j \geq 0$.
Thus, we have shown that Problem~\eqref{appendix_obj:proximal_operator_of_g*_with_TopSum_k_and_Huber} is closely related to Problem~\eqref{appendix_obj:KyFan_Huber_isotonic_regression} via the relation $\balpha^* = \text{sgn}(\bmu) \odot \bnu^*$.



\end{proof}


%\subsection{Proof of Lemma~\ref{lemma:PAVA_algorithm_exact_solution}}
\label{appendix_subsec:proof_PAVA_algorithm_exact_solution}

\begin{namedlemma}[~\ref{lemma:PAVA_algorithm_exact_solution}]
    \label{appendix_lemma:PAVA_algorithm_exact_solution}
    The customized PAVA presented in Algorithm~\ref{alg:PAVA_algorithm} solves the optimization problem in Equation~\eqref{obj:KyFan_Huber_isotonic_regression} exactly. 
    Specifically, for the final vector $\hat{\bnu}$ at the end of the while-loop (Line 14), we have $\hat{\bnu} = \bnu^*$.
\end{namedlemma}

\begin{proof}
% Add proof content here

Problem~\eqref{obj:KyFan_Huber_isotonic_regression} can be recognized as a generalized isotonic regression problem.
Let $h_j(v) = \frac{1}{2} (v - \mu_j)^2 + \rho_j H_M(v)$, where $\rho_j = \rho$ if $j \in \calJ$ and $\rho_j = 0$ otherwise.
As mentioned in Algorithm~\ref{alg:PAVA_algorithm}, the set $\calJ$ is the set of indices of top k largest elements of $\vert{\mu_j}$.
Then, we can rewrite Problem~\eqref{obj:KyFan_Huber_isotonic_regression} in the standard form of a generalized isotonic regression problem:
\begin{align}
    \label{obj:KyFan_Huber_isotonic_regression_rewritten_as_generalized_isotonic_regression}
    \min_{\bnu} \sum_{j=1}^{p} h_j(\nu_j) \quad \text{s.t.} \quad \nu_1 \geq \nu_2 \geq \cdots \geq \nu_J.
\end{align}

Generalized isotonic regression problems have been studied extensively in the literature.
There are two key properties~\cite{best2000minimizing,ahuja2001fast} regarding Problem~\eqref{obj:KyFan_Huber_isotonic_regression_rewritten_as_generalized_isotonic_regression} that we can leverage to prove Lemma~\ref{lemma:PAVA_algorithm_exact_solution}:

\paragraph{1. Optimal solution for a merged block is single-valued.}
Suppose we have two adjacent blocks $[a_1, a_2]$ and $[a_2+1, a_3]$.
If the optimal solution for each block is single-valued, \textit{i.e.}, we have
\begin{equation}
\begin{array}{c@{\hspace{2em}}c@{\hspace{2em}}c}
\bnu_{a_1:a_2}^* =  
\left(\begin{array}{c}
\argmin_{\bnu_{a_1:a_2}} \sum_{j=a_1}^{a_2} h_j(\nu_j) \\
\text{s.t.} \quad \nu_{a_1} \geq \cdots \geq \nu_{a_2}
\end{array}\right) & 
\Rightarrow & \bnu_{a_1}^* = \cdots = \bnu_{a_2}^* \\[3em]
\bnu_{a_2+1:a_3}^* =  
\left(\begin{array}{c}
\argmin_{\bnu_{a_2+1:a_3}} \sum_{j=a_2+1}^{a_3} h_j(\nu_j) \\
\text{s.t.} \quad \nu_{a_2+1} \geq \cdots \geq \nu_{a_3}
\end{array}\right) & 
\Rightarrow & \bnu_{a_2+1}^* = \cdots = \bnu_{a_3}^*
\end{array}
\end{equation}

and if $\nu_{a_1}^* \leq \nu_{a_2+1}^*$, then the optimal solution for the merged block $[a_1, a_3]$ is single-valued, \textit{i.e.}, we have:

\begin{equation}
\begin{array}{c@{\hspace{2em}}c@{\hspace{2em}}c}
\bnu_{a_1:a_3}^* =  
\left(\begin{array}{c}
\argmin_{\bnu_{a_1:a_3}} \sum_{j=a_1}^{a_3} h_j(\nu_j) \\
\text{s.t.} \quad \nu_{a_1} \geq \cdots \geq \nu_{a_3}
\end{array}\right) & 
\Rightarrow & \bnu_{a_1}^* = \cdots = \bnu_{a_3}^*
\end{array}
\end{equation}

\paragraph{2. No isotonic constraint violation between single-valued blocks implies the solution is optimal.}
If we have $s$ blocks $[a_1, a_2], [a_2+1, a_3], \ldots, [a_{s}+1, a_{s+1}]$ (with $a_1=1$ and $a_{s+1}=n$) such that the optimal solution for each block is single-valued, \textit{i.e.}, we have

\begin{equation}
\begin{array}{c@{\hspace{2em}}c@{\hspace{2em}}c}
\hat{\bnu}_{a_l:a_{l+1}} =
\left(\begin{array}{c}
\argmin_{\bnu_{a_l:a_{l+1}}} \sum_{j=a_l}^{a_{l+1}} h_j(\nu_j) \\
\text{s.t.} \quad \nu_{a_l} \geq \cdots \geq \nu_{a_{l+1}}
\end{array}\right) &
\Rightarrow & \bnu_{a_l}^* = \cdots = \bnu_{a_{l+1}}^* \, \text{ for all } \, l=1, \ldots, s
\end{array}
\end{equation}

and if $\hat{\nu}_{a_1} \geq \hat{\nu}_{a_2+1} \geq \ldots \hat{\nu}_{a_{s}}$, then $\hat{\bnu}$ is the optimal solution for Problem~\eqref{obj:KyFan_Huber_isotonic_regression_rewritten_as_generalized_isotonic_regression}.

Using these two properties, it is now easy to see why Algorithm~\ref{alg:PAVA_algorithm} returns the optimal solution.

We start by constructing blocks which have length 1.
The initial value restricted to each block is optimal.
Then, we iteratively merge adjacent blocks and update the values of $\nu_j$'s whenever there is a violation of the isotonic constraint.
By the first property, the optimal solution for the merged block is single-valued.
Therefore, we can compute the optimal solution for the merged block by solving a univariate optimization problem.

We keep merging blocks until there is no isotonic constraint violation.
When this happens, by construction, the solution for each block is single-valued and optimal.
By the second property, the final vector $\hat{\bnu}$ is the optimal solution for Problem~\eqref{obj:KyFan_Huber_isotonic_regression_rewritten_as_generalized_isotonic_regression}.


\end{proof}


\subsection{Proof of Lemma~\ref{lemma:PAVA_merging_linear_time_complexity}}

\begin{namedlemma}[~\ref{lemma:PAVA_merging_linear_time_complexity}]
    \label{appendix_lemma:PAVA_merging_linear_time_complexity}
    The merging part (lines 11-14) in the PAVA algorithm can be executed in linear time complexity, $O(n)$, where $n$ is the number of elements in the input vector $\bmu$.
\end{namedlemma}

\begin{proof}
% Add proof content here
The actual implementation to achieve the linear time complexity is called the up and down block algorithm, which is provided in Algorithm~\ref{alg:up_and_down_block_algorithm_for_merging_in_PAVA} in this appendix.
In order to prove this lemma, we only need to show that 
1) Algorithm~\ref{alg:up_and_down_block_algorithm_for_merging_in_PAVA} accomplishes the objective in lines 11-14 of Algorithm~\ref{alg:PAVA_algorithm} in the main paper, and
2) Algorithm~\ref{alg:up_and_down_block_algorithm_for_merging_in_PAVA} runs in linear time complexity.

\paragraph{Algorithm~\ref{alg:up_and_down_block_algorithm_for_merging_in_PAVA} accomplishes the objective in lines 11-14 of Algorithm~\ref{alg:PAVA_algorithm}}
In this up and down block algorithm, we use a special data structure called $\{P_b, S_b, v_b\}$ to store the values of $\sum_{j \in \calB(b)} \rho_j$ (sum of Huber penalty coefficients), $\sum_{j \in \calB(b)} \vert{\mu_j}$ (sum of absolute values of the input vector elements), and $\text{prox}_{P_b H_{M}}(S_b)$ (proximal operator of the Huber penalty function), respectively, for each block index $b$.

The output of this proximal operator, $\text{prox}_{P_b H_{M}}(S_b)$, is the minimizer of the the univariate function in this block.
To see this, let us write out the minimizer of the univariate function in the $b$-th block explicitly, with $\calB(b)$ denoting the set of indices in the $b$-th block:
\allowdisplaybreaks
\begin{align}
    & \argmin_{v} \sum_{j \in \calB(b)} \left( \frac{1}{2} (v - \vert{\mu_j})^2 + \rho_j H_M(v) \right) \\
    = & \argmin_{v} \sum_{j \in \calB(b)} \left( \frac{1}{2} (v^2 - 2v\vert{\mu_j} + \mu_j^2) + \rho_j H_M(v) \right) \eqcomment{expand the square terms} \\
    = & \argmin_{v} \sum_{j \in \calB(b)} \left( \frac{1}{2} v^2 - v\vert{\mu_j} + \rho_j H_M(v) \right) \eqcomment{get rid of the constant terms} \\
    = & \argmin_{v} \left( \sum_{j \in \calB(b)} \frac{1}{2} v^2 - \sum_{j \in \calB(b)} v\vert{\mu_j} + \sum_{j \in \calB(b)} \rho_j H_M(v) \right) \eqcomment{apply summation to each term} \\
    = & \argmin_{v} \left( N_b \frac{1}{2} v^2 - S_b \vert{\mu_j} + P_b H_M(v) \right) \eqcomment{substitute $N_b = \sum\limits_{j \in \calB(b)} 1$, $S_b = \sum\limits_{j \in \calB(b)} \vert{\mu_j}$, and $P_b = \sum\limits_{j \in \calB(b)} \rho_j$} \\
    = & \argmin_{v} \left( \frac{1}{2} v^2 - \frac{S_b}{N_b} \vert{\mu_j} + \frac{P_b}{N_b} H_M(v) \right) \eqcomment{divide all terms by $N_b$}\\
    = & \argmin_{v} \left( \frac{1}{2} \left( v - \frac{S_b}{N_b} \right)^2 + \frac{P_b}{N_b} H_M(v) \right) \eqcomment{complete the square} \\
    =& \text{prox}_{\frac{P_b}{N_b} H_{M}}(\frac{S_b}{N_b}). \eqcomment{proximal operator of the Huber penalty function}
\end{align}

Thus, Algorithm~\ref{alg:up_and_down_block_algorithm_for_merging_in_PAVA} merges two adjacent blocks if the isotonic violation persists, and the output of the proximal operator is the minimizer of the univariate function in the merged block.
This is exactly the same as the objective in lines 11-14 of Algorithm~\ref{alg:PAVA_algorithm}.

\paragraph{Algorithm~\ref{alg:up_and_down_block_algorithm_for_merging_in_PAVA} runs in linear time complexity}
For the while loop $j \leq n$ in Algorithm~\ref{alg:up_and_down_block_algorithm_for_merging_in_PAVA}, the variable $j$ is incremented by 1 in each iteration, and the loop terminates when $j = n$.
Although there are two while loops inside the main while loop, the total number of iterations in the two inner while loops is at most $n$.
This is because we start with $n$ blocks, and each iteration of the inner while loops either merges two blocks forward or merges two blocks backward.
The total number of merging operations is at most $n-1$.
Thus, the total number of iterations in the while loop $j \leq n$ is at most $n$.
Lastly, since we can evaluate the proximal operator of the Huber loss function, $H_M(\cdot)$, in constant time complexity, the total time complexity of Algorithm~\ref{alg:up_and_down_block_algorithm_for_merging_in_PAVA} is $O(n)$.


\end{proof}



\subsection{Proof of Theorem~\ref{theorem:compute_g_value_algorithm_correctness}}
\label{appendix_subsec:compute_g_value_algorithm_correctness}

\begin{namedtheorem}[~\ref{theorem:compute_g_value_algorithm_correctness}]
    \label{appendix_theorem:compute_g_value_algorithm_correctness}
    Algorithm~\ref{alg:compute_g_value_algorithm} correctly computes the value of $g(\bbeta)$ as defined in Problem~\eqref{obj:compute_g_value}.
    The computational complexity of Algorithm~\ref{alg:compute_g_value_algorithm} is $O(n + n \log k)$.
\end{namedtheorem}

\begin{proof}
% Add proof content here

We first show that the algorithm correctly computes the value of $g(\bbeta)$ and then analyze its computational complexity.

\paragraph{Correctness of the Algorithm:}
Recall that to obtain the value of $g(\bbeta)$, we need to solve the following optimization problem:
\begin{align}
    \label{appendix_obj:compute_g_value}
    g(\bbeta) = \left(
    \begin{array}{cc}
        \min\limits_{\bz, \bs} & \frac{1}{2} \sum_{j=1}^{n} s_j \\
        \text{s.t.} & \bz \in [0, 1]^n, \, \mathbf{1}^T \bz \leq k \\
        & \beta_j^2 \leq z_j s_j \\
        & -M z_j \leq \beta_j \leq M z_j
    \end{array}
    \right)
\end{align}
By using the perspective function $\beta_j^2/z_j$ instead of the rotated second-order cone constraint $\beta_j^2 \leq z_j s_j$, we can rewrite the optimization problem in~\eqref{appendix_obj:compute_g_value} as:
\begin{align}
    \label{appendix_obj:compute_g_value_perspective_formulation}
    g(\bbeta) = \left(
    \begin{array}{cc}
        \min\limits_{\bz} & \frac{1}{2} \sum_{j=1}^{n} \beta_j^2/z_j \\
        \text{s.t.} & \bz \in [0, 1]^n, \, \mathbf{1}^T \bz \leq k \\
        & -M z_j \leq \beta_j \leq M z_j
    \end{array}
    \right)
\end{align}
Let us define the cardinality and coefficient-constrained set $\calS_0$ below and use Lemma~\ref{lemma:equivalence_between_perspective_relaxation_and_convexification} to derive its convex hull:
\begin{align}
    \label{appendix_def:cardinality_and_coefficient_constraint_set_and_its_convex_hull}
    \calS_0 = \left\{ (t, \bz, \bbeta) \;\middle|\;
    \begin{array}{cc}
        t \geq \frac{1}{2} \sum_{j=1}^{n} \beta_j^2 \\
        \bz \in \{0, 1\}^n, \, \mathbf{1}^T \bz \leq k \\
        -M z_j \leq \beta_j \leq M z_j
    \end{array}
    \right\},
    \quad
    \conv(\calS_0) = \left\{ (t, \bz, \bbeta) \;\middle|\;
    \begin{array}{cc}
        t \geq \frac{1}{2} \sum_{j=1}^{n} \frac{\beta_j^2}{z_j} \\
        \bz \in [0, 1]^n, \, \mathbf{1}^T \bz \leq k \\
        -M z_j \leq \beta_j \leq M z_j
    \end{array}
    \right\}
\end{align}
Then we can rewrite the optimization problem in~\eqref{appendix_obj:compute_g_value_perspective_formulation} more succintly as:
\begin{align}
    \label{appendix_obj:compute_g_value_lifted_convex_hull_formulation}
    g(\bbeta) = \left(
    \begin{array}{cc}
        \min\limits_{t} & t \\
        \text{s.t.} & (t, \bz, \bbeta) \in \conv(\calS_0)
    \end{array}
    \right)
\end{align}

Let us define another related cardinality and coefficient-constrained set $\calS_1$ below:
\begin{align}
    \label{appendix_def:cardinality_and_coefficient_constraint_set_unlifted_formulation}
    \calS_1 = \left\{ (t,\boldsymbol{\beta}) \;\middle|\;
    \begin{array}{cc}
        t \geq \frac{1}{2} \sum_{j=1}^{n} \beta_j^2 \\
        \|\boldsymbol{\beta}\|_0 \leq k \\
        -M \leq \beta_j \leq M
    \end{array}
    \right\},
\end{align}
Since $\calS_0$ is a lifted formulation of $\calS_1$, we have $\calS_1 = \{(t, \bbeta) \mid \exists \bz \in \mathbb{R}^n \text{ s.t. } (t, \bz, \bbeta) \in \calS_0\}$.
This allows us to rewrite the optimization problem in~\eqref{appendix_obj:compute_g_value_lifted_convex_hull_formulation} as:
\begin{align}
    \label{appendix_obj:compute_g_value_unlifted_convex_hull_formulation}
    g(\bbeta) = \left(
    \begin{array}{cc}
        \min\limits_{t} & t \\
        \text{s.t.} & (t, \bbeta) \in \conv(\calS_1)
    \end{array}
    \right)
\end{align}


Now, since $\calS_1$ is a permutation-invariant set, according to Theorem 4 in~\cite{kim2022convexification}, we can rewrite its convex hull as:
\begin{align}
    \label{appendix_def:convex_cardinality_and_coefficient_constraint_set_majorization_formulation}
    \conv(\calS_1) = \left\{ (t, \bbeta) \;\middle|\;
    \begin{array}{cc}
        \exists \bphi \in \mathbb{R}^n \text{\, s.t. \,} \bphi \succeq_m \vert{\bbeta} \\
        t \geq \frac{1}{2} \sum_{j=1}^{n} \phi_j^2 \\
        M \geq \phi_1 \geq \phi_2 \geq \ldots \phi_k \geq 0 \\
        \phi_{k+1} = \phi_{k+2} = \ldots = \phi_n = 0 
    \end{array}
    \right\},
\end{align}
where the absolute value operator $\vert{\cdot}$ is applied to a vector in an element-wise fashion, and the sign $\succeq_m$ in $\bphi \succeq_m \vert{\bbeta}$ denotes that $\bphi$ majorizes $\vert{\bbeta}$, \text{i.e.}, we have: 
\begin{align}
    \label{appendix_def:vector_majorization}
    \bphi \succeq_m \vert{\bbeta} \quad \Rightarrow \quad \sum_{j=1}^l \phi_j \geq \sum_{j=1}^l \vert{\beta_j} \quad \forall l \in [n-1] \quad \text{and} \quad \sum_{j=1}^n \phi_j = \sum_{j=1}^n \vert{\beta_j}
\end{align}

Plugging the alternative convex hull description of $\calS_1$ in Equation~\eqref{appendix_def:convex_cardinality_and_coefficient_constraint_set_majorization_formulation} and the majorization definition in Equation~\eqref{appendix_def:vector_majorization} into the optimization problem in~\eqref{appendix_obj:compute_g_value_unlifted_convex_hull_formulation}, we can rewrite $g(\bbeta)$ as:
\begin{align}
    g(\bbeta) = \left(
    \begin{array}{cc}
        \min\limits_{t} & t \\
        \text{s.t.} & \bphi \succeq_m \vert{\bbeta} \\
        & t \geq \frac{1}{2} \sum_{j=1}^{n} \phi_j^2 \\
        & M \geq \phi_1 \geq \phi_2 \geq \ldots \phi_k \geq 0 \\
        & \phi_{k+1} = \phi_{k+2} = \ldots = \phi_n = 0
    \end{array}
    \right) \nonumber\\
    = \left(
    \begin{array}{cc}
        \min\limits_{\bphi} & \frac{1}{2} \sum_{j=1}^{n} \phi_j^2 \\
        \text{s.t.} &  \bphi \succeq_m \vert{\bbeta} \\
        & M \geq \phi_1 \geq \phi_2 \geq \ldots \phi_k \geq 0 \\
        & \phi_{k+1} = \phi_{k+2} = \ldots = \phi_n = 0
    \end{array}
    \right) \label{appendix_obj:compute_g_value_majorization_formulation}
\end{align}
The task of computing the value of $g(\bbeta)$ is now reduced to solving the optimization problem in Equation~\eqref{appendix_obj:compute_g_value_majorization_formulation} by finding the optimal vector $\bphi$ that satisfies the majorization and other constraints and minimizes the objective function.

Let us see how to use the majorization definition in Equation~\eqref{appendix_def:vector_majorization} to derive the Algorithm~\ref{alg:compute_g_value_algorithm} for computing the value of $g(\bbeta)$.

At the first iteration $j=1$, we have
\begin{align}
    k \phi_1 \geq \sum_{j=1}^k \phi_j = \sum_{j=1}^n \phi_j \geq \sum_{j=1}^n \vert{\beta_j} \quad \Rightarrow \quad \phi_1 \geq \frac{1}{k} \sum_{j=1}^n \vert{\beta_j}.
\end{align}
At the same time, we also need to satisfy $\phi_1 \geq \vert{\beta_1}$.
Now, we discuss in two cases:
\begin{enumerate}
    \item If $\frac{1}{k} \sum_{j=1}^n \vert{\beta_j} \geq \vert{\beta_1}$, in order to minimize the objective function in Problem~\eqref{appendix_obj:compute_g_value_majorization_formulation}, we set $\phi_1 = \frac{1}{k} \sum_{j=1}^n \vert{\beta_j}$ (notice that $\phi_1 \leq M$ is automatically satisfied because $\phi_1 = \frac{1}{k} \sum_{j=1}^n \vert{\beta_j} = \frac{1}{k} \sum_{j=1}^n M z_j \leq M$). 
    
    This leads to $\phi_2 = \ldots = \phi_k = \frac{1}{k} \sum_{j=1}^n \vert{\beta_j}$.
    To see this, for the sake of contradition, assume that $\exists j \in \{2, \ldots, k\}$ such that $\phi_j < \frac{1}{k} \sum_{j=1}^n \vert{\beta_j}$. 
    Since $\phi_j \leq \phi_1 = \frac{1}{k} \sum_{j=1}^n \vert{\beta_j}$, we have $\sum_{j=1}^k \phi_j < \sum_{j=1}^k \frac{1}{k} \sum_{j=1}^n \vert{\beta_j} = \sum_{j=1}^n \vert{\beta_j}$, which contradicts the majorization constraint.
    
    \item If $\frac{1}{k} \sum_{j=1}^n \vert{\beta_j} < \vert{\beta_1}$, we can set $\phi_1 = \vert{\beta_1}$ (notice that $\phi_1 \leq M$ is automatically satisifed because $\vert{\beta_1} \leq M z_1 \leq M$).
    Then we are left with $k-1$ coefficients to set, and we can follow the same argument as we did for when $j=1$.
    The majorization constraints become:
    \begin{align}
        \sum_{j=2}^l \phi_j \geq \sum_{j=2}^l \vert{\beta_j} \quad \forall l \in \{2, \ldots, n-1\} \quad \text{and} \quad \sum_{j=2}^n \phi_j = \sum_{j=2}^n \vert{\beta_j}
    \end{align}
\end{enumerate}
We repeat this process until we set all $k$ coefficients $\phi_1, \ldots, \phi_k$.
This is exactly what Algorithm~\ref{alg:compute_g_value_algorithm} does.
At the end, we return the value of the objective function, which is $\frac{1}{2} \sum_{j=1}^{k} \phi_j^2$.

\paragraph{Computational Complexity:}
It is straightforward to analyze the computational complexity of Algorithm~\ref{alg:compute_g_value_algorithm}.
The partial sorting step on Line 2 has a complexity of $O(n \log k)$.
The summation step on Line 3 has a complexity of $O(n)$.
The for-loop step on Line 4-8 has a complexity of $O(k)$, so does the final summation step on Line 9.
Therefore, the overall computational complexity of Algorithm~\ref{alg:compute_g_value_algorithm} is $O(n + n \log k)$.

\end{proof}

